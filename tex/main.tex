\documentclass[oneside,english,a4paper]{amsart}
\usepackage{bbold}
\usepackage[T1]{fontenc}
\usepackage{inputenc}
\setlength{\parskip}{\smallskipamount}
\setlength{\parindent}{0pt}
\usepackage{amstext}
\usepackage{amsthm}
\usepackage{amsmath}
\usepackage{pgfplots}
\usepackage{amssymb}
\usepackage{graphicx}
\usepackage{multirow}
\usepackage{subcaption}
\usepackage{wasysym}
\usepackage{fullpage}
\usepackage{adjustbox}

\makeatletter
\numberwithin{equation}{section}
\numberwithin{figure}{section}
\theoremstyle{plain}
\newtheorem{thm}{\protect\theoremname}
\theoremstyle{definition}
\newtheorem{defn}[thm]{\protect\definitionname}
\theoremstyle{plain}
\newtheorem{prop}[thm]{\protect\propositionname}
\theoremstyle{remark}
\newtheorem{rem}[thm]{\protect\remarkname}
\theoremstyle{plain}
\newtheorem{lem}[thm]{\protect\lemmaname}
\theoremstyle{definition}
\newtheorem{example}[thm]{\protect\examplename}
\newtheorem{cor}[thm]{\protect\corollaryname}
\theoremstyle{definition}

\makeatother

\usepackage{babel}
\providecommand{\definitionname}{Definition}
\providecommand{\lemmaname}{Lemma}
\providecommand{\propositionname}{Proposition}
\providecommand{\remarkname}{Remark}
\providecommand{\theoremname}{Theorem}
\providecommand{\examplename}{Example}
\providecommand{\corollaryname}{Corollary}

\begin{document}
\title{Tropical Support Vector Machines and mean payoff games}
\author{Xavier Allamigeon, Stéphane Gaubert, Samuel Boïté, Théo Molfessis}
\maketitle

\begin{abstract}
    We develop new algorithms around tropical support vector machines (SVMs) for binary and multi-classification. We explore the use of tropical hyperplanes to partition tropical space, laying the groundwork for our classification approach. We address the challenge of separating overlapping data in tropical space, proposing a method to measure and handle data overlap using non-expansive Shapley operators and a Krasnoselskii-Mann iteration scheme. For separable data, we extend our method to determine hard-margin tropical hyperplanes, ensuring maximal separation. This is further applied to multi-classification scenarios, providing conditions for tropical separability and demonstrating margin optimality. Additionally, the tropical kernel trick is explored as a means to embed data into a higher-dimensional tropical space, transforming decision boundaries into tropical polynomials. This approach is empirically tested on several classic datasets. Finally, we show that tropical hyperplanes emerge as limiting cases of classical hyperplanes on logarithmic paper, making our approach more stable numerically.
\end{abstract}

\section{Introduction}

% TODO: add a "motivation & context" subsection to explain the relevance of the topic

\subsection*{The tropical linear classification problem}

The tropical \emph{max-plus} semifield $\mathbb{R}_{\max}$ is the
set of real numbers, completed by $-\infty$ and equipped with the
addition $a\oplus b=\max(a,b)$ and the multiplication $a\odot b=a+b$.
A \emph{tropical hyperplane} in the $d$-dimensional tropical vector
space $\mathbb{R}_{\max}^{d}$ is a set of vectors of the form
\[
\mathcal{H}_{a}=\left\{x\in\mathbb{R}_{\max}^{d},\quad\max_{1\le i\le d}(a_{i}+x_{i})\,\text{is achieved at least twice}\right\}.
\]
Such a hyperplane is parametrized by the vector $a=(a_{1},\ldots,a_{d})\in\mathbb{R}_{\max}^{d}$,
which is required to be non-identically $-\infty$. It hence partitions
the space depending on which coordinate reaches its maximum, laying the
ground for $d$-class classification. By definition, they are invariant by translation along the constant vector $(1,\ldots, 1)$.

In this paper, we address the following tropical analogue of support
vector machines. Given $n\in\{1,\ldots, d\}$ classes of $d$-dimensional data
points $X^{1},\ldots,X^{p}$, we look for a maximum-margin separating
tropical hyperplane to build a classifier. The notion of margin depends
on the metric: the canonical choice in tropical geometry is the (additive
version of) Hilbert's projective metric. Its restriction to $\mathbb{R}^{d}$
is induced by the so-called \emph{Hilbert's seminorm} or \emph{Hopf
oscillation}
\[
\lVert x\rVert_{H}:=\max_{1\le i\le d}x_{i}-\min_{1\le i\le d}x_{i}.
\]
It is a projective metric, in the sense that the distance between
two points is zero if and only if these two points differ by an additive
constant. When dealing with hard-margin support vector machines (SVMs),
we also have to ensure that every point lands in the right sector. Each class $k$ belongs to some sectors $I^k\subset \{1,\ldots, d\}$, with each sector assigned to only one class. We then define the \emph{tropical parametrized hyperplane} of configuration
$\sigma=\left\{I^{1},\ldots,I^{n}\right\}$, where $I^{k}$ and $I^{\ell}$
are disjoint for $k\ne\ell$, as:
\[
\mathcal{H}=\left\{x\in\mathbb{R}_{\max}^{d},\quad\exists k\ne\ell,\quad (x-a)\,\text{reaches its max coordinate in \ensuremath{I^{k}} and \ensuremath{I^{\ell}}}\right\}.
\]
We note $\mathcal{S}^k\subset \mathbb{R}_{\max}^d$ the \emph{sector} classified as class $k$, where the maximum coordinate is reached in $I^k$. $\mathcal{H}$ is said to \emph{separate} point
clouds $(X^{k})_{1\le k\le p}$ with a margin of $\nu$ when for every point
$x^{k}$ of class $k$:
\begin{enumerate}
\item $x^{k}$ is on the right sector of the hyperplane, i.e its maximum
coordinate is reached in $I^{k}$
\item $x^{k}$ is at distance at least $\nu$ from any other sector $\mathcal{S}^\ell$, $l \ne k$: %$\mathcal{H}_{a}^{\sigma}$:
\begin{equation*}
d_H(\mathcal{S}^\ell,x^{k})=\max(x^{k}-a)-\max_{i\in I^\ell}(x^{k}-a)_i\ge\nu.
\end{equation*}
\end{enumerate}

In Figure \ref{fig:MaxMargin}, we separate two classes
of $3$-dimensional points from a toy tropical dataset using a tropical
hyperplane. The two lower sectors are assigned to one class, the upper to the other. Its margin is maximal and is represented by the Hilbert
ball in gray. The figure is represented in the projective space $\mathbb{P}\left(\mathbb{R}_{\text{max}}^{3}\right)$,
i.e. the quotient of the set of non-identically $-\infty$ vectors
of $\mathbb{R}_{\text{max}}^{3}$ by the equivalence relation which
identifies tropically proportional values.

Since our separator is translation-invariant by the constant vector, in order to separate any $d$-dimensional dataset, we plunge it into the $x_1+\cdots+x_{d+1}=0$ subspace of $\mathbb{R}_{\max}^{d+1}$. This amounts to artificially adding a new coordinate to each point, equal to the opposite of the sum of the initial coordinates. In the rare case where we are dealing with an intrinsically tropical dataset, for which this translation invariance makes sense, it is better to apply the tropical classifier on the initial features.

\begin{figure}[!h]
    \centering
    \begin{subfigure}{0.55\textwidth}
        \centering
        \resizebox{0.5\textwidth}{!}{%
        \centering
    \clipbox{0.35\width{} 0.25\height{} 0.25\width{} 0.25\height{}}{%% Creator: Matplotlib, PGF backend
%%
%% To include the figure in your LaTeX document, write
%%   \input{<filename>.pgf}
%%
%% Make sure the required packages are loaded in your preamble
%%   \usepackage{pgf}
%%
%% Also ensure that all the required font packages are loaded; for instance,
%% the lmodern package is sometimes necessary when using math font.
%%   \usepackage{lmodern}
%%
%% Figures using additional raster images can only be included by \input if
%% they are in the same directory as the main LaTeX file. For loading figures
%% from other directories you can use the `import` package
%%   \usepackage{import}
%%
%% and then include the figures with
%%   \import{<path to file>}{<filename>.pgf}
%%
%% Matplotlib used the following preamble
%%   
%%   \usepackage{fontspec}
%%   \setmainfont{DejaVuSerif.ttf}[Path=\detokenize{/Users/sam/Library/Python/3.9/lib/python/site-packages/matplotlib/mpl-data/fonts/ttf/}]
%%   \setsansfont{DejaVuSans.ttf}[Path=\detokenize{/Users/sam/Library/Python/3.9/lib/python/site-packages/matplotlib/mpl-data/fonts/ttf/}]
%%   \setmonofont{DejaVuSansMono.ttf}[Path=\detokenize{/Users/sam/Library/Python/3.9/lib/python/site-packages/matplotlib/mpl-data/fonts/ttf/}]
%%   \makeatletter\@ifpackageloaded{underscore}{}{\usepackage[strings]{underscore}}\makeatother
%%
\begingroup%
\makeatletter%
\begin{pgfpicture}%
\pgfpathrectangle{\pgfpointorigin}{\pgfqpoint{6.000000in}{6.000000in}}%
\pgfusepath{use as bounding box, clip}%
\begin{pgfscope}%
\pgfsetbuttcap%
\pgfsetmiterjoin%
\definecolor{currentfill}{rgb}{1.000000,1.000000,1.000000}%
\pgfsetfillcolor{currentfill}%
\pgfsetlinewidth{0.000000pt}%
\definecolor{currentstroke}{rgb}{1.000000,1.000000,1.000000}%
\pgfsetstrokecolor{currentstroke}%
\pgfsetdash{}{0pt}%
\pgfpathmoveto{\pgfqpoint{0.000000in}{0.000000in}}%
\pgfpathlineto{\pgfqpoint{6.000000in}{0.000000in}}%
\pgfpathlineto{\pgfqpoint{6.000000in}{6.000000in}}%
\pgfpathlineto{\pgfqpoint{0.000000in}{6.000000in}}%
\pgfpathlineto{\pgfqpoint{0.000000in}{0.000000in}}%
\pgfpathclose%
\pgfusepath{fill}%
\end{pgfscope}%
\begin{pgfscope}%
\pgfsetbuttcap%
\pgfsetmiterjoin%
\definecolor{currentfill}{rgb}{1.000000,1.000000,1.000000}%
\pgfsetfillcolor{currentfill}%
\pgfsetlinewidth{0.000000pt}%
\definecolor{currentstroke}{rgb}{0.000000,0.000000,0.000000}%
\pgfsetstrokecolor{currentstroke}%
\pgfsetstrokeopacity{0.000000}%
\pgfsetdash{}{0pt}%
\pgfpathmoveto{\pgfqpoint{0.765000in}{0.660000in}}%
\pgfpathlineto{\pgfqpoint{5.385000in}{0.660000in}}%
\pgfpathlineto{\pgfqpoint{5.385000in}{5.280000in}}%
\pgfpathlineto{\pgfqpoint{0.765000in}{5.280000in}}%
\pgfpathlineto{\pgfqpoint{0.765000in}{0.660000in}}%
\pgfpathclose%
\pgfusepath{fill}%
\end{pgfscope}%
\begin{pgfscope}%
\pgfpathrectangle{\pgfqpoint{0.765000in}{0.660000in}}{\pgfqpoint{4.620000in}{4.620000in}}%
\pgfusepath{clip}%
\pgfsetbuttcap%
\pgfsetroundjoin%
\definecolor{currentfill}{rgb}{1.000000,0.894118,0.788235}%
\pgfsetfillcolor{currentfill}%
\pgfsetlinewidth{0.000000pt}%
\definecolor{currentstroke}{rgb}{1.000000,0.894118,0.788235}%
\pgfsetstrokecolor{currentstroke}%
\pgfsetdash{}{0pt}%
\pgfpathmoveto{\pgfqpoint{-3.891113in}{6.386809in}}%
\pgfpathlineto{\pgfqpoint{-4.344914in}{6.173762in}}%
\pgfpathlineto{\pgfqpoint{-4.344914in}{6.598751in}}%
\pgfpathlineto{\pgfqpoint{-3.891113in}{6.811798in}}%
\pgfpathlineto{\pgfqpoint{-3.891113in}{6.386809in}}%
\pgfpathclose%
\pgfusepath{fill}%
\end{pgfscope}%
\begin{pgfscope}%
\pgfpathrectangle{\pgfqpoint{0.765000in}{0.660000in}}{\pgfqpoint{4.620000in}{4.620000in}}%
\pgfusepath{clip}%
\pgfsetbuttcap%
\pgfsetroundjoin%
\definecolor{currentfill}{rgb}{1.000000,0.894118,0.788235}%
\pgfsetfillcolor{currentfill}%
\pgfsetlinewidth{0.000000pt}%
\definecolor{currentstroke}{rgb}{1.000000,0.894118,0.788235}%
\pgfsetstrokecolor{currentstroke}%
\pgfsetdash{}{0pt}%
\pgfpathmoveto{\pgfqpoint{-3.891113in}{6.386809in}}%
\pgfpathlineto{\pgfqpoint{-3.437311in}{6.173762in}}%
\pgfpathlineto{\pgfqpoint{-3.437311in}{6.598751in}}%
\pgfpathlineto{\pgfqpoint{-3.891113in}{6.811798in}}%
\pgfpathlineto{\pgfqpoint{-3.891113in}{6.386809in}}%
\pgfpathclose%
\pgfusepath{fill}%
\end{pgfscope}%
\begin{pgfscope}%
\pgfpathrectangle{\pgfqpoint{0.765000in}{0.660000in}}{\pgfqpoint{4.620000in}{4.620000in}}%
\pgfusepath{clip}%
\pgfsetbuttcap%
\pgfsetroundjoin%
\definecolor{currentfill}{rgb}{1.000000,0.894118,0.788235}%
\pgfsetfillcolor{currentfill}%
\pgfsetlinewidth{0.000000pt}%
\definecolor{currentstroke}{rgb}{1.000000,0.894118,0.788235}%
\pgfsetstrokecolor{currentstroke}%
\pgfsetdash{}{0pt}%
\pgfpathmoveto{\pgfqpoint{-3.891113in}{6.386809in}}%
\pgfpathlineto{\pgfqpoint{-4.344914in}{6.173762in}}%
\pgfpathlineto{\pgfqpoint{-3.891113in}{5.960715in}}%
\pgfpathlineto{\pgfqpoint{-3.437311in}{6.173762in}}%
\pgfpathlineto{\pgfqpoint{-3.891113in}{6.386809in}}%
\pgfpathclose%
\pgfusepath{fill}%
\end{pgfscope}%
\begin{pgfscope}%
\pgfpathrectangle{\pgfqpoint{0.765000in}{0.660000in}}{\pgfqpoint{4.620000in}{4.620000in}}%
\pgfusepath{clip}%
\pgfsetbuttcap%
\pgfsetroundjoin%
\definecolor{currentfill}{rgb}{1.000000,0.894118,0.788235}%
\pgfsetfillcolor{currentfill}%
\pgfsetlinewidth{0.000000pt}%
\definecolor{currentstroke}{rgb}{1.000000,0.894118,0.788235}%
\pgfsetstrokecolor{currentstroke}%
\pgfsetdash{}{0pt}%
\pgfpathmoveto{\pgfqpoint{-3.891113in}{6.811798in}}%
\pgfpathlineto{\pgfqpoint{-4.344914in}{6.598751in}}%
\pgfpathlineto{\pgfqpoint{-3.891113in}{6.385704in}}%
\pgfpathlineto{\pgfqpoint{-3.437311in}{6.598751in}}%
\pgfpathlineto{\pgfqpoint{-3.891113in}{6.811798in}}%
\pgfpathclose%
\pgfusepath{fill}%
\end{pgfscope}%
\begin{pgfscope}%
\pgfpathrectangle{\pgfqpoint{0.765000in}{0.660000in}}{\pgfqpoint{4.620000in}{4.620000in}}%
\pgfusepath{clip}%
\pgfsetbuttcap%
\pgfsetroundjoin%
\definecolor{currentfill}{rgb}{1.000000,0.894118,0.788235}%
\pgfsetfillcolor{currentfill}%
\pgfsetlinewidth{0.000000pt}%
\definecolor{currentstroke}{rgb}{1.000000,0.894118,0.788235}%
\pgfsetstrokecolor{currentstroke}%
\pgfsetdash{}{0pt}%
\pgfpathmoveto{\pgfqpoint{-3.891113in}{5.960715in}}%
\pgfpathlineto{\pgfqpoint{-3.437311in}{6.173762in}}%
\pgfpathlineto{\pgfqpoint{-3.437311in}{6.598751in}}%
\pgfpathlineto{\pgfqpoint{-3.891113in}{6.385704in}}%
\pgfpathlineto{\pgfqpoint{-3.891113in}{5.960715in}}%
\pgfpathclose%
\pgfusepath{fill}%
\end{pgfscope}%
\begin{pgfscope}%
\pgfpathrectangle{\pgfqpoint{0.765000in}{0.660000in}}{\pgfqpoint{4.620000in}{4.620000in}}%
\pgfusepath{clip}%
\pgfsetbuttcap%
\pgfsetroundjoin%
\definecolor{currentfill}{rgb}{1.000000,0.894118,0.788235}%
\pgfsetfillcolor{currentfill}%
\pgfsetlinewidth{0.000000pt}%
\definecolor{currentstroke}{rgb}{1.000000,0.894118,0.788235}%
\pgfsetstrokecolor{currentstroke}%
\pgfsetdash{}{0pt}%
\pgfpathmoveto{\pgfqpoint{-4.344914in}{6.173762in}}%
\pgfpathlineto{\pgfqpoint{-3.891113in}{5.960715in}}%
\pgfpathlineto{\pgfqpoint{-3.891113in}{6.385704in}}%
\pgfpathlineto{\pgfqpoint{-4.344914in}{6.598751in}}%
\pgfpathlineto{\pgfqpoint{-4.344914in}{6.173762in}}%
\pgfpathclose%
\pgfusepath{fill}%
\end{pgfscope}%
\begin{pgfscope}%
\pgfpathrectangle{\pgfqpoint{0.765000in}{0.660000in}}{\pgfqpoint{4.620000in}{4.620000in}}%
\pgfusepath{clip}%
\pgfsetbuttcap%
\pgfsetroundjoin%
\definecolor{currentfill}{rgb}{1.000000,0.894118,0.788235}%
\pgfsetfillcolor{currentfill}%
\pgfsetlinewidth{0.000000pt}%
\definecolor{currentstroke}{rgb}{1.000000,0.894118,0.788235}%
\pgfsetstrokecolor{currentstroke}%
\pgfsetdash{}{0pt}%
\pgfpathmoveto{\pgfqpoint{3.071405in}{3.123758in}}%
\pgfpathlineto{\pgfqpoint{2.617603in}{2.910711in}}%
\pgfpathlineto{\pgfqpoint{2.617603in}{3.335700in}}%
\pgfpathlineto{\pgfqpoint{3.071405in}{3.548747in}}%
\pgfpathlineto{\pgfqpoint{3.071405in}{3.123758in}}%
\pgfpathclose%
\pgfusepath{fill}%
\end{pgfscope}%
\begin{pgfscope}%
\pgfpathrectangle{\pgfqpoint{0.765000in}{0.660000in}}{\pgfqpoint{4.620000in}{4.620000in}}%
\pgfusepath{clip}%
\pgfsetbuttcap%
\pgfsetroundjoin%
\definecolor{currentfill}{rgb}{1.000000,0.894118,0.788235}%
\pgfsetfillcolor{currentfill}%
\pgfsetlinewidth{0.000000pt}%
\definecolor{currentstroke}{rgb}{1.000000,0.894118,0.788235}%
\pgfsetstrokecolor{currentstroke}%
\pgfsetdash{}{0pt}%
\pgfpathmoveto{\pgfqpoint{3.071405in}{3.123758in}}%
\pgfpathlineto{\pgfqpoint{3.525206in}{2.910711in}}%
\pgfpathlineto{\pgfqpoint{3.525206in}{3.335700in}}%
\pgfpathlineto{\pgfqpoint{3.071405in}{3.548747in}}%
\pgfpathlineto{\pgfqpoint{3.071405in}{3.123758in}}%
\pgfpathclose%
\pgfusepath{fill}%
\end{pgfscope}%
\begin{pgfscope}%
\pgfpathrectangle{\pgfqpoint{0.765000in}{0.660000in}}{\pgfqpoint{4.620000in}{4.620000in}}%
\pgfusepath{clip}%
\pgfsetbuttcap%
\pgfsetroundjoin%
\definecolor{currentfill}{rgb}{1.000000,0.894118,0.788235}%
\pgfsetfillcolor{currentfill}%
\pgfsetlinewidth{0.000000pt}%
\definecolor{currentstroke}{rgb}{1.000000,0.894118,0.788235}%
\pgfsetstrokecolor{currentstroke}%
\pgfsetdash{}{0pt}%
\pgfpathmoveto{\pgfqpoint{3.071405in}{3.123758in}}%
\pgfpathlineto{\pgfqpoint{2.617603in}{2.910711in}}%
\pgfpathlineto{\pgfqpoint{3.071405in}{2.697664in}}%
\pgfpathlineto{\pgfqpoint{3.525206in}{2.910711in}}%
\pgfpathlineto{\pgfqpoint{3.071405in}{3.123758in}}%
\pgfpathclose%
\pgfusepath{fill}%
\end{pgfscope}%
\begin{pgfscope}%
\pgfpathrectangle{\pgfqpoint{0.765000in}{0.660000in}}{\pgfqpoint{4.620000in}{4.620000in}}%
\pgfusepath{clip}%
\pgfsetbuttcap%
\pgfsetroundjoin%
\definecolor{currentfill}{rgb}{1.000000,0.894118,0.788235}%
\pgfsetfillcolor{currentfill}%
\pgfsetlinewidth{0.000000pt}%
\definecolor{currentstroke}{rgb}{1.000000,0.894118,0.788235}%
\pgfsetstrokecolor{currentstroke}%
\pgfsetdash{}{0pt}%
\pgfpathmoveto{\pgfqpoint{3.071405in}{3.548747in}}%
\pgfpathlineto{\pgfqpoint{2.617603in}{3.335700in}}%
\pgfpathlineto{\pgfqpoint{3.071405in}{3.122653in}}%
\pgfpathlineto{\pgfqpoint{3.525206in}{3.335700in}}%
\pgfpathlineto{\pgfqpoint{3.071405in}{3.548747in}}%
\pgfpathclose%
\pgfusepath{fill}%
\end{pgfscope}%
\begin{pgfscope}%
\pgfpathrectangle{\pgfqpoint{0.765000in}{0.660000in}}{\pgfqpoint{4.620000in}{4.620000in}}%
\pgfusepath{clip}%
\pgfsetbuttcap%
\pgfsetroundjoin%
\definecolor{currentfill}{rgb}{1.000000,0.894118,0.788235}%
\pgfsetfillcolor{currentfill}%
\pgfsetlinewidth{0.000000pt}%
\definecolor{currentstroke}{rgb}{1.000000,0.894118,0.788235}%
\pgfsetstrokecolor{currentstroke}%
\pgfsetdash{}{0pt}%
\pgfpathmoveto{\pgfqpoint{2.617603in}{2.910711in}}%
\pgfpathlineto{\pgfqpoint{3.071405in}{2.697664in}}%
\pgfpathlineto{\pgfqpoint{3.071405in}{3.122653in}}%
\pgfpathlineto{\pgfqpoint{2.617603in}{3.335700in}}%
\pgfpathlineto{\pgfqpoint{2.617603in}{2.910711in}}%
\pgfpathclose%
\pgfusepath{fill}%
\end{pgfscope}%
\begin{pgfscope}%
\pgfpathrectangle{\pgfqpoint{0.765000in}{0.660000in}}{\pgfqpoint{4.620000in}{4.620000in}}%
\pgfusepath{clip}%
\pgfsetbuttcap%
\pgfsetroundjoin%
\definecolor{currentfill}{rgb}{1.000000,0.894118,0.788235}%
\pgfsetfillcolor{currentfill}%
\pgfsetlinewidth{0.000000pt}%
\definecolor{currentstroke}{rgb}{1.000000,0.894118,0.788235}%
\pgfsetstrokecolor{currentstroke}%
\pgfsetdash{}{0pt}%
\pgfpathmoveto{\pgfqpoint{3.071405in}{2.697664in}}%
\pgfpathlineto{\pgfqpoint{3.525206in}{2.910711in}}%
\pgfpathlineto{\pgfqpoint{3.525206in}{3.335700in}}%
\pgfpathlineto{\pgfqpoint{3.071405in}{3.122653in}}%
\pgfpathlineto{\pgfqpoint{3.071405in}{2.697664in}}%
\pgfpathclose%
\pgfusepath{fill}%
\end{pgfscope}%
\begin{pgfscope}%
\pgfpathrectangle{\pgfqpoint{0.765000in}{0.660000in}}{\pgfqpoint{4.620000in}{4.620000in}}%
\pgfusepath{clip}%
\pgfsetbuttcap%
\pgfsetroundjoin%
\definecolor{currentfill}{rgb}{1.000000,0.894118,0.788235}%
\pgfsetfillcolor{currentfill}%
\pgfsetlinewidth{0.000000pt}%
\definecolor{currentstroke}{rgb}{1.000000,0.894118,0.788235}%
\pgfsetstrokecolor{currentstroke}%
\pgfsetdash{}{0pt}%
\pgfpathmoveto{\pgfqpoint{3.071405in}{3.123758in}}%
\pgfpathlineto{\pgfqpoint{2.617603in}{2.910711in}}%
\pgfpathlineto{\pgfqpoint{-4.344914in}{6.173762in}}%
\pgfpathlineto{\pgfqpoint{-3.891113in}{6.386809in}}%
\pgfpathlineto{\pgfqpoint{3.071405in}{3.123758in}}%
\pgfpathclose%
\pgfusepath{fill}%
\end{pgfscope}%
\begin{pgfscope}%
\pgfpathrectangle{\pgfqpoint{0.765000in}{0.660000in}}{\pgfqpoint{4.620000in}{4.620000in}}%
\pgfusepath{clip}%
\pgfsetbuttcap%
\pgfsetroundjoin%
\definecolor{currentfill}{rgb}{1.000000,0.894118,0.788235}%
\pgfsetfillcolor{currentfill}%
\pgfsetlinewidth{0.000000pt}%
\definecolor{currentstroke}{rgb}{1.000000,0.894118,0.788235}%
\pgfsetstrokecolor{currentstroke}%
\pgfsetdash{}{0pt}%
\pgfpathmoveto{\pgfqpoint{3.525206in}{2.910711in}}%
\pgfpathlineto{\pgfqpoint{3.071405in}{3.123758in}}%
\pgfpathlineto{\pgfqpoint{-3.891113in}{6.386809in}}%
\pgfpathlineto{\pgfqpoint{-3.437311in}{6.173762in}}%
\pgfpathlineto{\pgfqpoint{3.525206in}{2.910711in}}%
\pgfpathclose%
\pgfusepath{fill}%
\end{pgfscope}%
\begin{pgfscope}%
\pgfpathrectangle{\pgfqpoint{0.765000in}{0.660000in}}{\pgfqpoint{4.620000in}{4.620000in}}%
\pgfusepath{clip}%
\pgfsetbuttcap%
\pgfsetroundjoin%
\definecolor{currentfill}{rgb}{1.000000,0.894118,0.788235}%
\pgfsetfillcolor{currentfill}%
\pgfsetlinewidth{0.000000pt}%
\definecolor{currentstroke}{rgb}{1.000000,0.894118,0.788235}%
\pgfsetstrokecolor{currentstroke}%
\pgfsetdash{}{0pt}%
\pgfpathmoveto{\pgfqpoint{3.071405in}{3.123758in}}%
\pgfpathlineto{\pgfqpoint{3.071405in}{3.548747in}}%
\pgfpathlineto{\pgfqpoint{-3.891113in}{6.811798in}}%
\pgfpathlineto{\pgfqpoint{-3.437311in}{6.173762in}}%
\pgfpathlineto{\pgfqpoint{3.071405in}{3.123758in}}%
\pgfpathclose%
\pgfusepath{fill}%
\end{pgfscope}%
\begin{pgfscope}%
\pgfpathrectangle{\pgfqpoint{0.765000in}{0.660000in}}{\pgfqpoint{4.620000in}{4.620000in}}%
\pgfusepath{clip}%
\pgfsetbuttcap%
\pgfsetroundjoin%
\definecolor{currentfill}{rgb}{1.000000,0.894118,0.788235}%
\pgfsetfillcolor{currentfill}%
\pgfsetlinewidth{0.000000pt}%
\definecolor{currentstroke}{rgb}{1.000000,0.894118,0.788235}%
\pgfsetstrokecolor{currentstroke}%
\pgfsetdash{}{0pt}%
\pgfpathmoveto{\pgfqpoint{3.525206in}{2.910711in}}%
\pgfpathlineto{\pgfqpoint{3.525206in}{3.335700in}}%
\pgfpathlineto{\pgfqpoint{-3.437311in}{6.598751in}}%
\pgfpathlineto{\pgfqpoint{-3.891113in}{6.386809in}}%
\pgfpathlineto{\pgfqpoint{3.525206in}{2.910711in}}%
\pgfpathclose%
\pgfusepath{fill}%
\end{pgfscope}%
\begin{pgfscope}%
\pgfpathrectangle{\pgfqpoint{0.765000in}{0.660000in}}{\pgfqpoint{4.620000in}{4.620000in}}%
\pgfusepath{clip}%
\pgfsetbuttcap%
\pgfsetroundjoin%
\definecolor{currentfill}{rgb}{1.000000,0.894118,0.788235}%
\pgfsetfillcolor{currentfill}%
\pgfsetlinewidth{0.000000pt}%
\definecolor{currentstroke}{rgb}{1.000000,0.894118,0.788235}%
\pgfsetstrokecolor{currentstroke}%
\pgfsetdash{}{0pt}%
\pgfpathmoveto{\pgfqpoint{3.071405in}{3.548747in}}%
\pgfpathlineto{\pgfqpoint{2.617603in}{3.335700in}}%
\pgfpathlineto{\pgfqpoint{-4.344914in}{6.598751in}}%
\pgfpathlineto{\pgfqpoint{-3.891113in}{6.811798in}}%
\pgfpathlineto{\pgfqpoint{3.071405in}{3.548747in}}%
\pgfpathclose%
\pgfusepath{fill}%
\end{pgfscope}%
\begin{pgfscope}%
\pgfpathrectangle{\pgfqpoint{0.765000in}{0.660000in}}{\pgfqpoint{4.620000in}{4.620000in}}%
\pgfusepath{clip}%
\pgfsetbuttcap%
\pgfsetroundjoin%
\definecolor{currentfill}{rgb}{1.000000,0.894118,0.788235}%
\pgfsetfillcolor{currentfill}%
\pgfsetlinewidth{0.000000pt}%
\definecolor{currentstroke}{rgb}{1.000000,0.894118,0.788235}%
\pgfsetstrokecolor{currentstroke}%
\pgfsetdash{}{0pt}%
\pgfpathmoveto{\pgfqpoint{3.525206in}{3.335700in}}%
\pgfpathlineto{\pgfqpoint{3.071405in}{3.548747in}}%
\pgfpathlineto{\pgfqpoint{-3.891113in}{6.811798in}}%
\pgfpathlineto{\pgfqpoint{-3.437311in}{6.598751in}}%
\pgfpathlineto{\pgfqpoint{3.525206in}{3.335700in}}%
\pgfpathclose%
\pgfusepath{fill}%
\end{pgfscope}%
\begin{pgfscope}%
\pgfpathrectangle{\pgfqpoint{0.765000in}{0.660000in}}{\pgfqpoint{4.620000in}{4.620000in}}%
\pgfusepath{clip}%
\pgfsetbuttcap%
\pgfsetroundjoin%
\definecolor{currentfill}{rgb}{1.000000,0.894118,0.788235}%
\pgfsetfillcolor{currentfill}%
\pgfsetlinewidth{0.000000pt}%
\definecolor{currentstroke}{rgb}{1.000000,0.894118,0.788235}%
\pgfsetstrokecolor{currentstroke}%
\pgfsetdash{}{0pt}%
\pgfpathmoveto{\pgfqpoint{3.071405in}{2.697664in}}%
\pgfpathlineto{\pgfqpoint{3.525206in}{2.910711in}}%
\pgfpathlineto{\pgfqpoint{-3.437311in}{6.173762in}}%
\pgfpathlineto{\pgfqpoint{-3.891113in}{5.960715in}}%
\pgfpathlineto{\pgfqpoint{3.071405in}{2.697664in}}%
\pgfpathclose%
\pgfusepath{fill}%
\end{pgfscope}%
\begin{pgfscope}%
\pgfpathrectangle{\pgfqpoint{0.765000in}{0.660000in}}{\pgfqpoint{4.620000in}{4.620000in}}%
\pgfusepath{clip}%
\pgfsetbuttcap%
\pgfsetroundjoin%
\definecolor{currentfill}{rgb}{1.000000,0.894118,0.788235}%
\pgfsetfillcolor{currentfill}%
\pgfsetlinewidth{0.000000pt}%
\definecolor{currentstroke}{rgb}{1.000000,0.894118,0.788235}%
\pgfsetstrokecolor{currentstroke}%
\pgfsetdash{}{0pt}%
\pgfpathmoveto{\pgfqpoint{2.617603in}{2.910711in}}%
\pgfpathlineto{\pgfqpoint{3.071405in}{2.697664in}}%
\pgfpathlineto{\pgfqpoint{-3.891113in}{5.960715in}}%
\pgfpathlineto{\pgfqpoint{-4.344914in}{6.173762in}}%
\pgfpathlineto{\pgfqpoint{2.617603in}{2.910711in}}%
\pgfpathclose%
\pgfusepath{fill}%
\end{pgfscope}%
\begin{pgfscope}%
\pgfpathrectangle{\pgfqpoint{0.765000in}{0.660000in}}{\pgfqpoint{4.620000in}{4.620000in}}%
\pgfusepath{clip}%
\pgfsetbuttcap%
\pgfsetroundjoin%
\definecolor{currentfill}{rgb}{1.000000,0.894118,0.788235}%
\pgfsetfillcolor{currentfill}%
\pgfsetlinewidth{0.000000pt}%
\definecolor{currentstroke}{rgb}{1.000000,0.894118,0.788235}%
\pgfsetstrokecolor{currentstroke}%
\pgfsetdash{}{0pt}%
\pgfpathmoveto{\pgfqpoint{2.617603in}{2.910711in}}%
\pgfpathlineto{\pgfqpoint{2.617603in}{3.335700in}}%
\pgfpathlineto{\pgfqpoint{-4.344914in}{6.598751in}}%
\pgfpathlineto{\pgfqpoint{-3.891113in}{5.960715in}}%
\pgfpathlineto{\pgfqpoint{2.617603in}{2.910711in}}%
\pgfpathclose%
\pgfusepath{fill}%
\end{pgfscope}%
\begin{pgfscope}%
\pgfpathrectangle{\pgfqpoint{0.765000in}{0.660000in}}{\pgfqpoint{4.620000in}{4.620000in}}%
\pgfusepath{clip}%
\pgfsetbuttcap%
\pgfsetroundjoin%
\definecolor{currentfill}{rgb}{1.000000,0.894118,0.788235}%
\pgfsetfillcolor{currentfill}%
\pgfsetlinewidth{0.000000pt}%
\definecolor{currentstroke}{rgb}{1.000000,0.894118,0.788235}%
\pgfsetstrokecolor{currentstroke}%
\pgfsetdash{}{0pt}%
\pgfpathmoveto{\pgfqpoint{3.071405in}{2.697664in}}%
\pgfpathlineto{\pgfqpoint{3.071405in}{3.122653in}}%
\pgfpathlineto{\pgfqpoint{-3.891113in}{6.385704in}}%
\pgfpathlineto{\pgfqpoint{-4.344914in}{6.173762in}}%
\pgfpathlineto{\pgfqpoint{3.071405in}{2.697664in}}%
\pgfpathclose%
\pgfusepath{fill}%
\end{pgfscope}%
\begin{pgfscope}%
\pgfpathrectangle{\pgfqpoint{0.765000in}{0.660000in}}{\pgfqpoint{4.620000in}{4.620000in}}%
\pgfusepath{clip}%
\pgfsetbuttcap%
\pgfsetroundjoin%
\definecolor{currentfill}{rgb}{1.000000,0.894118,0.788235}%
\pgfsetfillcolor{currentfill}%
\pgfsetlinewidth{0.000000pt}%
\definecolor{currentstroke}{rgb}{1.000000,0.894118,0.788235}%
\pgfsetstrokecolor{currentstroke}%
\pgfsetdash{}{0pt}%
\pgfpathmoveto{\pgfqpoint{2.617603in}{3.335700in}}%
\pgfpathlineto{\pgfqpoint{3.071405in}{3.122653in}}%
\pgfpathlineto{\pgfqpoint{-3.891113in}{6.385704in}}%
\pgfpathlineto{\pgfqpoint{-4.344914in}{6.598751in}}%
\pgfpathlineto{\pgfqpoint{2.617603in}{3.335700in}}%
\pgfpathclose%
\pgfusepath{fill}%
\end{pgfscope}%
\begin{pgfscope}%
\pgfpathrectangle{\pgfqpoint{0.765000in}{0.660000in}}{\pgfqpoint{4.620000in}{4.620000in}}%
\pgfusepath{clip}%
\pgfsetbuttcap%
\pgfsetroundjoin%
\definecolor{currentfill}{rgb}{1.000000,0.894118,0.788235}%
\pgfsetfillcolor{currentfill}%
\pgfsetlinewidth{0.000000pt}%
\definecolor{currentstroke}{rgb}{1.000000,0.894118,0.788235}%
\pgfsetstrokecolor{currentstroke}%
\pgfsetdash{}{0pt}%
\pgfpathmoveto{\pgfqpoint{3.071405in}{3.122653in}}%
\pgfpathlineto{\pgfqpoint{3.525206in}{3.335700in}}%
\pgfpathlineto{\pgfqpoint{-3.437311in}{6.598751in}}%
\pgfpathlineto{\pgfqpoint{-3.891113in}{6.385704in}}%
\pgfpathlineto{\pgfqpoint{3.071405in}{3.122653in}}%
\pgfpathclose%
\pgfusepath{fill}%
\end{pgfscope}%
\begin{pgfscope}%
\pgfpathrectangle{\pgfqpoint{0.765000in}{0.660000in}}{\pgfqpoint{4.620000in}{4.620000in}}%
\pgfusepath{clip}%
\pgfsetbuttcap%
\pgfsetroundjoin%
\definecolor{currentfill}{rgb}{1.000000,0.894118,0.788235}%
\pgfsetfillcolor{currentfill}%
\pgfsetlinewidth{0.000000pt}%
\definecolor{currentstroke}{rgb}{1.000000,0.894118,0.788235}%
\pgfsetstrokecolor{currentstroke}%
\pgfsetdash{}{0pt}%
\pgfpathmoveto{\pgfqpoint{10.033922in}{6.386809in}}%
\pgfpathlineto{\pgfqpoint{9.580120in}{6.173762in}}%
\pgfpathlineto{\pgfqpoint{9.580120in}{6.598751in}}%
\pgfpathlineto{\pgfqpoint{10.033922in}{6.811798in}}%
\pgfpathlineto{\pgfqpoint{10.033922in}{6.386809in}}%
\pgfpathclose%
\pgfusepath{fill}%
\end{pgfscope}%
\begin{pgfscope}%
\pgfpathrectangle{\pgfqpoint{0.765000in}{0.660000in}}{\pgfqpoint{4.620000in}{4.620000in}}%
\pgfusepath{clip}%
\pgfsetbuttcap%
\pgfsetroundjoin%
\definecolor{currentfill}{rgb}{1.000000,0.894118,0.788235}%
\pgfsetfillcolor{currentfill}%
\pgfsetlinewidth{0.000000pt}%
\definecolor{currentstroke}{rgb}{1.000000,0.894118,0.788235}%
\pgfsetstrokecolor{currentstroke}%
\pgfsetdash{}{0pt}%
\pgfpathmoveto{\pgfqpoint{10.033922in}{6.386809in}}%
\pgfpathlineto{\pgfqpoint{10.487724in}{6.173762in}}%
\pgfpathlineto{\pgfqpoint{10.487724in}{6.598751in}}%
\pgfpathlineto{\pgfqpoint{10.033922in}{6.811798in}}%
\pgfpathlineto{\pgfqpoint{10.033922in}{6.386809in}}%
\pgfpathclose%
\pgfusepath{fill}%
\end{pgfscope}%
\begin{pgfscope}%
\pgfpathrectangle{\pgfqpoint{0.765000in}{0.660000in}}{\pgfqpoint{4.620000in}{4.620000in}}%
\pgfusepath{clip}%
\pgfsetbuttcap%
\pgfsetroundjoin%
\definecolor{currentfill}{rgb}{1.000000,0.894118,0.788235}%
\pgfsetfillcolor{currentfill}%
\pgfsetlinewidth{0.000000pt}%
\definecolor{currentstroke}{rgb}{1.000000,0.894118,0.788235}%
\pgfsetstrokecolor{currentstroke}%
\pgfsetdash{}{0pt}%
\pgfpathmoveto{\pgfqpoint{10.033922in}{6.386809in}}%
\pgfpathlineto{\pgfqpoint{9.580120in}{6.173762in}}%
\pgfpathlineto{\pgfqpoint{10.033922in}{5.960715in}}%
\pgfpathlineto{\pgfqpoint{10.487724in}{6.173762in}}%
\pgfpathlineto{\pgfqpoint{10.033922in}{6.386809in}}%
\pgfpathclose%
\pgfusepath{fill}%
\end{pgfscope}%
\begin{pgfscope}%
\pgfpathrectangle{\pgfqpoint{0.765000in}{0.660000in}}{\pgfqpoint{4.620000in}{4.620000in}}%
\pgfusepath{clip}%
\pgfsetbuttcap%
\pgfsetroundjoin%
\definecolor{currentfill}{rgb}{1.000000,0.894118,0.788235}%
\pgfsetfillcolor{currentfill}%
\pgfsetlinewidth{0.000000pt}%
\definecolor{currentstroke}{rgb}{1.000000,0.894118,0.788235}%
\pgfsetstrokecolor{currentstroke}%
\pgfsetdash{}{0pt}%
\pgfpathmoveto{\pgfqpoint{10.033922in}{6.811798in}}%
\pgfpathlineto{\pgfqpoint{9.580120in}{6.598751in}}%
\pgfpathlineto{\pgfqpoint{10.033922in}{6.385704in}}%
\pgfpathlineto{\pgfqpoint{10.487724in}{6.598751in}}%
\pgfpathlineto{\pgfqpoint{10.033922in}{6.811798in}}%
\pgfpathclose%
\pgfusepath{fill}%
\end{pgfscope}%
\begin{pgfscope}%
\pgfpathrectangle{\pgfqpoint{0.765000in}{0.660000in}}{\pgfqpoint{4.620000in}{4.620000in}}%
\pgfusepath{clip}%
\pgfsetbuttcap%
\pgfsetroundjoin%
\definecolor{currentfill}{rgb}{1.000000,0.894118,0.788235}%
\pgfsetfillcolor{currentfill}%
\pgfsetlinewidth{0.000000pt}%
\definecolor{currentstroke}{rgb}{1.000000,0.894118,0.788235}%
\pgfsetstrokecolor{currentstroke}%
\pgfsetdash{}{0pt}%
\pgfpathmoveto{\pgfqpoint{10.033922in}{5.960715in}}%
\pgfpathlineto{\pgfqpoint{10.487724in}{6.173762in}}%
\pgfpathlineto{\pgfqpoint{10.487724in}{6.598751in}}%
\pgfpathlineto{\pgfqpoint{10.033922in}{6.385704in}}%
\pgfpathlineto{\pgfqpoint{10.033922in}{5.960715in}}%
\pgfpathclose%
\pgfusepath{fill}%
\end{pgfscope}%
\begin{pgfscope}%
\pgfpathrectangle{\pgfqpoint{0.765000in}{0.660000in}}{\pgfqpoint{4.620000in}{4.620000in}}%
\pgfusepath{clip}%
\pgfsetbuttcap%
\pgfsetroundjoin%
\definecolor{currentfill}{rgb}{1.000000,0.894118,0.788235}%
\pgfsetfillcolor{currentfill}%
\pgfsetlinewidth{0.000000pt}%
\definecolor{currentstroke}{rgb}{1.000000,0.894118,0.788235}%
\pgfsetstrokecolor{currentstroke}%
\pgfsetdash{}{0pt}%
\pgfpathmoveto{\pgfqpoint{9.580120in}{6.173762in}}%
\pgfpathlineto{\pgfqpoint{10.033922in}{5.960715in}}%
\pgfpathlineto{\pgfqpoint{10.033922in}{6.385704in}}%
\pgfpathlineto{\pgfqpoint{9.580120in}{6.598751in}}%
\pgfpathlineto{\pgfqpoint{9.580120in}{6.173762in}}%
\pgfpathclose%
\pgfusepath{fill}%
\end{pgfscope}%
\begin{pgfscope}%
\pgfpathrectangle{\pgfqpoint{0.765000in}{0.660000in}}{\pgfqpoint{4.620000in}{4.620000in}}%
\pgfusepath{clip}%
\pgfsetbuttcap%
\pgfsetroundjoin%
\definecolor{currentfill}{rgb}{1.000000,0.894118,0.788235}%
\pgfsetfillcolor{currentfill}%
\pgfsetlinewidth{0.000000pt}%
\definecolor{currentstroke}{rgb}{1.000000,0.894118,0.788235}%
\pgfsetstrokecolor{currentstroke}%
\pgfsetdash{}{0pt}%
\pgfpathmoveto{\pgfqpoint{3.071405in}{3.123758in}}%
\pgfpathlineto{\pgfqpoint{2.617603in}{2.910711in}}%
\pgfpathlineto{\pgfqpoint{2.617603in}{3.335700in}}%
\pgfpathlineto{\pgfqpoint{3.071405in}{3.548747in}}%
\pgfpathlineto{\pgfqpoint{3.071405in}{3.123758in}}%
\pgfpathclose%
\pgfusepath{fill}%
\end{pgfscope}%
\begin{pgfscope}%
\pgfpathrectangle{\pgfqpoint{0.765000in}{0.660000in}}{\pgfqpoint{4.620000in}{4.620000in}}%
\pgfusepath{clip}%
\pgfsetbuttcap%
\pgfsetroundjoin%
\definecolor{currentfill}{rgb}{1.000000,0.894118,0.788235}%
\pgfsetfillcolor{currentfill}%
\pgfsetlinewidth{0.000000pt}%
\definecolor{currentstroke}{rgb}{1.000000,0.894118,0.788235}%
\pgfsetstrokecolor{currentstroke}%
\pgfsetdash{}{0pt}%
\pgfpathmoveto{\pgfqpoint{3.071405in}{3.123758in}}%
\pgfpathlineto{\pgfqpoint{3.525206in}{2.910711in}}%
\pgfpathlineto{\pgfqpoint{3.525206in}{3.335700in}}%
\pgfpathlineto{\pgfqpoint{3.071405in}{3.548747in}}%
\pgfpathlineto{\pgfqpoint{3.071405in}{3.123758in}}%
\pgfpathclose%
\pgfusepath{fill}%
\end{pgfscope}%
\begin{pgfscope}%
\pgfpathrectangle{\pgfqpoint{0.765000in}{0.660000in}}{\pgfqpoint{4.620000in}{4.620000in}}%
\pgfusepath{clip}%
\pgfsetbuttcap%
\pgfsetroundjoin%
\definecolor{currentfill}{rgb}{1.000000,0.894118,0.788235}%
\pgfsetfillcolor{currentfill}%
\pgfsetlinewidth{0.000000pt}%
\definecolor{currentstroke}{rgb}{1.000000,0.894118,0.788235}%
\pgfsetstrokecolor{currentstroke}%
\pgfsetdash{}{0pt}%
\pgfpathmoveto{\pgfqpoint{3.071405in}{3.123758in}}%
\pgfpathlineto{\pgfqpoint{2.617603in}{2.910711in}}%
\pgfpathlineto{\pgfqpoint{3.071405in}{2.697664in}}%
\pgfpathlineto{\pgfqpoint{3.525206in}{2.910711in}}%
\pgfpathlineto{\pgfqpoint{3.071405in}{3.123758in}}%
\pgfpathclose%
\pgfusepath{fill}%
\end{pgfscope}%
\begin{pgfscope}%
\pgfpathrectangle{\pgfqpoint{0.765000in}{0.660000in}}{\pgfqpoint{4.620000in}{4.620000in}}%
\pgfusepath{clip}%
\pgfsetbuttcap%
\pgfsetroundjoin%
\definecolor{currentfill}{rgb}{1.000000,0.894118,0.788235}%
\pgfsetfillcolor{currentfill}%
\pgfsetlinewidth{0.000000pt}%
\definecolor{currentstroke}{rgb}{1.000000,0.894118,0.788235}%
\pgfsetstrokecolor{currentstroke}%
\pgfsetdash{}{0pt}%
\pgfpathmoveto{\pgfqpoint{3.071405in}{3.548747in}}%
\pgfpathlineto{\pgfqpoint{2.617603in}{3.335700in}}%
\pgfpathlineto{\pgfqpoint{3.071405in}{3.122653in}}%
\pgfpathlineto{\pgfqpoint{3.525206in}{3.335700in}}%
\pgfpathlineto{\pgfqpoint{3.071405in}{3.548747in}}%
\pgfpathclose%
\pgfusepath{fill}%
\end{pgfscope}%
\begin{pgfscope}%
\pgfpathrectangle{\pgfqpoint{0.765000in}{0.660000in}}{\pgfqpoint{4.620000in}{4.620000in}}%
\pgfusepath{clip}%
\pgfsetbuttcap%
\pgfsetroundjoin%
\definecolor{currentfill}{rgb}{1.000000,0.894118,0.788235}%
\pgfsetfillcolor{currentfill}%
\pgfsetlinewidth{0.000000pt}%
\definecolor{currentstroke}{rgb}{1.000000,0.894118,0.788235}%
\pgfsetstrokecolor{currentstroke}%
\pgfsetdash{}{0pt}%
\pgfpathmoveto{\pgfqpoint{2.617603in}{2.910711in}}%
\pgfpathlineto{\pgfqpoint{3.071405in}{2.697664in}}%
\pgfpathlineto{\pgfqpoint{3.071405in}{3.122653in}}%
\pgfpathlineto{\pgfqpoint{2.617603in}{3.335700in}}%
\pgfpathlineto{\pgfqpoint{2.617603in}{2.910711in}}%
\pgfpathclose%
\pgfusepath{fill}%
\end{pgfscope}%
\begin{pgfscope}%
\pgfpathrectangle{\pgfqpoint{0.765000in}{0.660000in}}{\pgfqpoint{4.620000in}{4.620000in}}%
\pgfusepath{clip}%
\pgfsetbuttcap%
\pgfsetroundjoin%
\definecolor{currentfill}{rgb}{1.000000,0.894118,0.788235}%
\pgfsetfillcolor{currentfill}%
\pgfsetlinewidth{0.000000pt}%
\definecolor{currentstroke}{rgb}{1.000000,0.894118,0.788235}%
\pgfsetstrokecolor{currentstroke}%
\pgfsetdash{}{0pt}%
\pgfpathmoveto{\pgfqpoint{3.071405in}{2.697664in}}%
\pgfpathlineto{\pgfqpoint{3.525206in}{2.910711in}}%
\pgfpathlineto{\pgfqpoint{3.525206in}{3.335700in}}%
\pgfpathlineto{\pgfqpoint{3.071405in}{3.122653in}}%
\pgfpathlineto{\pgfqpoint{3.071405in}{2.697664in}}%
\pgfpathclose%
\pgfusepath{fill}%
\end{pgfscope}%
\begin{pgfscope}%
\pgfpathrectangle{\pgfqpoint{0.765000in}{0.660000in}}{\pgfqpoint{4.620000in}{4.620000in}}%
\pgfusepath{clip}%
\pgfsetbuttcap%
\pgfsetroundjoin%
\definecolor{currentfill}{rgb}{1.000000,0.894118,0.788235}%
\pgfsetfillcolor{currentfill}%
\pgfsetlinewidth{0.000000pt}%
\definecolor{currentstroke}{rgb}{1.000000,0.894118,0.788235}%
\pgfsetstrokecolor{currentstroke}%
\pgfsetdash{}{0pt}%
\pgfpathmoveto{\pgfqpoint{3.071405in}{3.123758in}}%
\pgfpathlineto{\pgfqpoint{2.617603in}{2.910711in}}%
\pgfpathlineto{\pgfqpoint{9.580120in}{6.173762in}}%
\pgfpathlineto{\pgfqpoint{10.033922in}{6.386809in}}%
\pgfpathlineto{\pgfqpoint{3.071405in}{3.123758in}}%
\pgfpathclose%
\pgfusepath{fill}%
\end{pgfscope}%
\begin{pgfscope}%
\pgfpathrectangle{\pgfqpoint{0.765000in}{0.660000in}}{\pgfqpoint{4.620000in}{4.620000in}}%
\pgfusepath{clip}%
\pgfsetbuttcap%
\pgfsetroundjoin%
\definecolor{currentfill}{rgb}{1.000000,0.894118,0.788235}%
\pgfsetfillcolor{currentfill}%
\pgfsetlinewidth{0.000000pt}%
\definecolor{currentstroke}{rgb}{1.000000,0.894118,0.788235}%
\pgfsetstrokecolor{currentstroke}%
\pgfsetdash{}{0pt}%
\pgfpathmoveto{\pgfqpoint{3.525206in}{2.910711in}}%
\pgfpathlineto{\pgfqpoint{3.071405in}{3.123758in}}%
\pgfpathlineto{\pgfqpoint{10.033922in}{6.386809in}}%
\pgfpathlineto{\pgfqpoint{10.487724in}{6.173762in}}%
\pgfpathlineto{\pgfqpoint{3.525206in}{2.910711in}}%
\pgfpathclose%
\pgfusepath{fill}%
\end{pgfscope}%
\begin{pgfscope}%
\pgfpathrectangle{\pgfqpoint{0.765000in}{0.660000in}}{\pgfqpoint{4.620000in}{4.620000in}}%
\pgfusepath{clip}%
\pgfsetbuttcap%
\pgfsetroundjoin%
\definecolor{currentfill}{rgb}{1.000000,0.894118,0.788235}%
\pgfsetfillcolor{currentfill}%
\pgfsetlinewidth{0.000000pt}%
\definecolor{currentstroke}{rgb}{1.000000,0.894118,0.788235}%
\pgfsetstrokecolor{currentstroke}%
\pgfsetdash{}{0pt}%
\pgfpathmoveto{\pgfqpoint{3.071405in}{3.123758in}}%
\pgfpathlineto{\pgfqpoint{3.071405in}{3.548747in}}%
\pgfpathlineto{\pgfqpoint{10.033922in}{6.811798in}}%
\pgfpathlineto{\pgfqpoint{10.487724in}{6.173762in}}%
\pgfpathlineto{\pgfqpoint{3.071405in}{3.123758in}}%
\pgfpathclose%
\pgfusepath{fill}%
\end{pgfscope}%
\begin{pgfscope}%
\pgfpathrectangle{\pgfqpoint{0.765000in}{0.660000in}}{\pgfqpoint{4.620000in}{4.620000in}}%
\pgfusepath{clip}%
\pgfsetbuttcap%
\pgfsetroundjoin%
\definecolor{currentfill}{rgb}{1.000000,0.894118,0.788235}%
\pgfsetfillcolor{currentfill}%
\pgfsetlinewidth{0.000000pt}%
\definecolor{currentstroke}{rgb}{1.000000,0.894118,0.788235}%
\pgfsetstrokecolor{currentstroke}%
\pgfsetdash{}{0pt}%
\pgfpathmoveto{\pgfqpoint{3.525206in}{2.910711in}}%
\pgfpathlineto{\pgfqpoint{3.525206in}{3.335700in}}%
\pgfpathlineto{\pgfqpoint{10.487724in}{6.598751in}}%
\pgfpathlineto{\pgfqpoint{10.033922in}{6.386809in}}%
\pgfpathlineto{\pgfqpoint{3.525206in}{2.910711in}}%
\pgfpathclose%
\pgfusepath{fill}%
\end{pgfscope}%
\begin{pgfscope}%
\pgfpathrectangle{\pgfqpoint{0.765000in}{0.660000in}}{\pgfqpoint{4.620000in}{4.620000in}}%
\pgfusepath{clip}%
\pgfsetbuttcap%
\pgfsetroundjoin%
\definecolor{currentfill}{rgb}{1.000000,0.894118,0.788235}%
\pgfsetfillcolor{currentfill}%
\pgfsetlinewidth{0.000000pt}%
\definecolor{currentstroke}{rgb}{1.000000,0.894118,0.788235}%
\pgfsetstrokecolor{currentstroke}%
\pgfsetdash{}{0pt}%
\pgfpathmoveto{\pgfqpoint{3.071405in}{3.548747in}}%
\pgfpathlineto{\pgfqpoint{2.617603in}{3.335700in}}%
\pgfpathlineto{\pgfqpoint{9.580120in}{6.598751in}}%
\pgfpathlineto{\pgfqpoint{10.033922in}{6.811798in}}%
\pgfpathlineto{\pgfqpoint{3.071405in}{3.548747in}}%
\pgfpathclose%
\pgfusepath{fill}%
\end{pgfscope}%
\begin{pgfscope}%
\pgfpathrectangle{\pgfqpoint{0.765000in}{0.660000in}}{\pgfqpoint{4.620000in}{4.620000in}}%
\pgfusepath{clip}%
\pgfsetbuttcap%
\pgfsetroundjoin%
\definecolor{currentfill}{rgb}{1.000000,0.894118,0.788235}%
\pgfsetfillcolor{currentfill}%
\pgfsetlinewidth{0.000000pt}%
\definecolor{currentstroke}{rgb}{1.000000,0.894118,0.788235}%
\pgfsetstrokecolor{currentstroke}%
\pgfsetdash{}{0pt}%
\pgfpathmoveto{\pgfqpoint{3.525206in}{3.335700in}}%
\pgfpathlineto{\pgfqpoint{3.071405in}{3.548747in}}%
\pgfpathlineto{\pgfqpoint{10.033922in}{6.811798in}}%
\pgfpathlineto{\pgfqpoint{10.487724in}{6.598751in}}%
\pgfpathlineto{\pgfqpoint{3.525206in}{3.335700in}}%
\pgfpathclose%
\pgfusepath{fill}%
\end{pgfscope}%
\begin{pgfscope}%
\pgfpathrectangle{\pgfqpoint{0.765000in}{0.660000in}}{\pgfqpoint{4.620000in}{4.620000in}}%
\pgfusepath{clip}%
\pgfsetbuttcap%
\pgfsetroundjoin%
\definecolor{currentfill}{rgb}{1.000000,0.894118,0.788235}%
\pgfsetfillcolor{currentfill}%
\pgfsetlinewidth{0.000000pt}%
\definecolor{currentstroke}{rgb}{1.000000,0.894118,0.788235}%
\pgfsetstrokecolor{currentstroke}%
\pgfsetdash{}{0pt}%
\pgfpathmoveto{\pgfqpoint{3.071405in}{2.697664in}}%
\pgfpathlineto{\pgfqpoint{3.525206in}{2.910711in}}%
\pgfpathlineto{\pgfqpoint{10.487724in}{6.173762in}}%
\pgfpathlineto{\pgfqpoint{10.033922in}{5.960715in}}%
\pgfpathlineto{\pgfqpoint{3.071405in}{2.697664in}}%
\pgfpathclose%
\pgfusepath{fill}%
\end{pgfscope}%
\begin{pgfscope}%
\pgfpathrectangle{\pgfqpoint{0.765000in}{0.660000in}}{\pgfqpoint{4.620000in}{4.620000in}}%
\pgfusepath{clip}%
\pgfsetbuttcap%
\pgfsetroundjoin%
\definecolor{currentfill}{rgb}{1.000000,0.894118,0.788235}%
\pgfsetfillcolor{currentfill}%
\pgfsetlinewidth{0.000000pt}%
\definecolor{currentstroke}{rgb}{1.000000,0.894118,0.788235}%
\pgfsetstrokecolor{currentstroke}%
\pgfsetdash{}{0pt}%
\pgfpathmoveto{\pgfqpoint{2.617603in}{2.910711in}}%
\pgfpathlineto{\pgfqpoint{3.071405in}{2.697664in}}%
\pgfpathlineto{\pgfqpoint{10.033922in}{5.960715in}}%
\pgfpathlineto{\pgfqpoint{9.580120in}{6.173762in}}%
\pgfpathlineto{\pgfqpoint{2.617603in}{2.910711in}}%
\pgfpathclose%
\pgfusepath{fill}%
\end{pgfscope}%
\begin{pgfscope}%
\pgfpathrectangle{\pgfqpoint{0.765000in}{0.660000in}}{\pgfqpoint{4.620000in}{4.620000in}}%
\pgfusepath{clip}%
\pgfsetbuttcap%
\pgfsetroundjoin%
\definecolor{currentfill}{rgb}{1.000000,0.894118,0.788235}%
\pgfsetfillcolor{currentfill}%
\pgfsetlinewidth{0.000000pt}%
\definecolor{currentstroke}{rgb}{1.000000,0.894118,0.788235}%
\pgfsetstrokecolor{currentstroke}%
\pgfsetdash{}{0pt}%
\pgfpathmoveto{\pgfqpoint{2.617603in}{2.910711in}}%
\pgfpathlineto{\pgfqpoint{2.617603in}{3.335700in}}%
\pgfpathlineto{\pgfqpoint{9.580120in}{6.598751in}}%
\pgfpathlineto{\pgfqpoint{10.033922in}{5.960715in}}%
\pgfpathlineto{\pgfqpoint{2.617603in}{2.910711in}}%
\pgfpathclose%
\pgfusepath{fill}%
\end{pgfscope}%
\begin{pgfscope}%
\pgfpathrectangle{\pgfqpoint{0.765000in}{0.660000in}}{\pgfqpoint{4.620000in}{4.620000in}}%
\pgfusepath{clip}%
\pgfsetbuttcap%
\pgfsetroundjoin%
\definecolor{currentfill}{rgb}{1.000000,0.894118,0.788235}%
\pgfsetfillcolor{currentfill}%
\pgfsetlinewidth{0.000000pt}%
\definecolor{currentstroke}{rgb}{1.000000,0.894118,0.788235}%
\pgfsetstrokecolor{currentstroke}%
\pgfsetdash{}{0pt}%
\pgfpathmoveto{\pgfqpoint{3.071405in}{2.697664in}}%
\pgfpathlineto{\pgfqpoint{3.071405in}{3.122653in}}%
\pgfpathlineto{\pgfqpoint{10.033922in}{6.385704in}}%
\pgfpathlineto{\pgfqpoint{9.580120in}{6.173762in}}%
\pgfpathlineto{\pgfqpoint{3.071405in}{2.697664in}}%
\pgfpathclose%
\pgfusepath{fill}%
\end{pgfscope}%
\begin{pgfscope}%
\pgfpathrectangle{\pgfqpoint{0.765000in}{0.660000in}}{\pgfqpoint{4.620000in}{4.620000in}}%
\pgfusepath{clip}%
\pgfsetbuttcap%
\pgfsetroundjoin%
\definecolor{currentfill}{rgb}{1.000000,0.894118,0.788235}%
\pgfsetfillcolor{currentfill}%
\pgfsetlinewidth{0.000000pt}%
\definecolor{currentstroke}{rgb}{1.000000,0.894118,0.788235}%
\pgfsetstrokecolor{currentstroke}%
\pgfsetdash{}{0pt}%
\pgfpathmoveto{\pgfqpoint{2.617603in}{3.335700in}}%
\pgfpathlineto{\pgfqpoint{3.071405in}{3.122653in}}%
\pgfpathlineto{\pgfqpoint{10.033922in}{6.385704in}}%
\pgfpathlineto{\pgfqpoint{9.580120in}{6.598751in}}%
\pgfpathlineto{\pgfqpoint{2.617603in}{3.335700in}}%
\pgfpathclose%
\pgfusepath{fill}%
\end{pgfscope}%
\begin{pgfscope}%
\pgfpathrectangle{\pgfqpoint{0.765000in}{0.660000in}}{\pgfqpoint{4.620000in}{4.620000in}}%
\pgfusepath{clip}%
\pgfsetbuttcap%
\pgfsetroundjoin%
\definecolor{currentfill}{rgb}{1.000000,0.894118,0.788235}%
\pgfsetfillcolor{currentfill}%
\pgfsetlinewidth{0.000000pt}%
\definecolor{currentstroke}{rgb}{1.000000,0.894118,0.788235}%
\pgfsetstrokecolor{currentstroke}%
\pgfsetdash{}{0pt}%
\pgfpathmoveto{\pgfqpoint{3.071405in}{3.122653in}}%
\pgfpathlineto{\pgfqpoint{3.525206in}{3.335700in}}%
\pgfpathlineto{\pgfqpoint{10.487724in}{6.598751in}}%
\pgfpathlineto{\pgfqpoint{10.033922in}{6.385704in}}%
\pgfpathlineto{\pgfqpoint{3.071405in}{3.122653in}}%
\pgfpathclose%
\pgfusepath{fill}%
\end{pgfscope}%
\begin{pgfscope}%
\pgfpathrectangle{\pgfqpoint{0.765000in}{0.660000in}}{\pgfqpoint{4.620000in}{4.620000in}}%
\pgfusepath{clip}%
\pgfsetrectcap%
\pgfsetroundjoin%
\pgfsetlinewidth{1.204500pt}%
\definecolor{currentstroke}{rgb}{1.000000,0.576471,0.309804}%
\pgfsetstrokecolor{currentstroke}%
\pgfsetdash{}{0pt}%
\pgfpathmoveto{\pgfqpoint{2.824478in}{2.470663in}}%
\pgfusepath{stroke}%
\end{pgfscope}%
\begin{pgfscope}%
\pgfpathrectangle{\pgfqpoint{0.765000in}{0.660000in}}{\pgfqpoint{4.620000in}{4.620000in}}%
\pgfusepath{clip}%
\pgfsetbuttcap%
\pgfsetroundjoin%
\definecolor{currentfill}{rgb}{1.000000,0.576471,0.309804}%
\pgfsetfillcolor{currentfill}%
\pgfsetlinewidth{1.003750pt}%
\definecolor{currentstroke}{rgb}{1.000000,0.576471,0.309804}%
\pgfsetstrokecolor{currentstroke}%
\pgfsetdash{}{0pt}%
\pgfsys@defobject{currentmarker}{\pgfqpoint{-0.033333in}{-0.033333in}}{\pgfqpoint{0.033333in}{0.033333in}}{%
\pgfpathmoveto{\pgfqpoint{0.000000in}{-0.033333in}}%
\pgfpathcurveto{\pgfqpoint{0.008840in}{-0.033333in}}{\pgfqpoint{0.017319in}{-0.029821in}}{\pgfqpoint{0.023570in}{-0.023570in}}%
\pgfpathcurveto{\pgfqpoint{0.029821in}{-0.017319in}}{\pgfqpoint{0.033333in}{-0.008840in}}{\pgfqpoint{0.033333in}{0.000000in}}%
\pgfpathcurveto{\pgfqpoint{0.033333in}{0.008840in}}{\pgfqpoint{0.029821in}{0.017319in}}{\pgfqpoint{0.023570in}{0.023570in}}%
\pgfpathcurveto{\pgfqpoint{0.017319in}{0.029821in}}{\pgfqpoint{0.008840in}{0.033333in}}{\pgfqpoint{0.000000in}{0.033333in}}%
\pgfpathcurveto{\pgfqpoint{-0.008840in}{0.033333in}}{\pgfqpoint{-0.017319in}{0.029821in}}{\pgfqpoint{-0.023570in}{0.023570in}}%
\pgfpathcurveto{\pgfqpoint{-0.029821in}{0.017319in}}{\pgfqpoint{-0.033333in}{0.008840in}}{\pgfqpoint{-0.033333in}{0.000000in}}%
\pgfpathcurveto{\pgfqpoint{-0.033333in}{-0.008840in}}{\pgfqpoint{-0.029821in}{-0.017319in}}{\pgfqpoint{-0.023570in}{-0.023570in}}%
\pgfpathcurveto{\pgfqpoint{-0.017319in}{-0.029821in}}{\pgfqpoint{-0.008840in}{-0.033333in}}{\pgfqpoint{0.000000in}{-0.033333in}}%
\pgfpathlineto{\pgfqpoint{0.000000in}{-0.033333in}}%
\pgfpathclose%
\pgfusepath{stroke,fill}%
}%
\begin{pgfscope}%
\pgfsys@transformshift{2.824478in}{2.470663in}%
\pgfsys@useobject{currentmarker}{}%
\end{pgfscope}%
\end{pgfscope}%
\begin{pgfscope}%
\pgfpathrectangle{\pgfqpoint{0.765000in}{0.660000in}}{\pgfqpoint{4.620000in}{4.620000in}}%
\pgfusepath{clip}%
\pgfsetrectcap%
\pgfsetroundjoin%
\pgfsetlinewidth{1.204500pt}%
\definecolor{currentstroke}{rgb}{1.000000,0.576471,0.309804}%
\pgfsetstrokecolor{currentstroke}%
\pgfsetdash{}{0pt}%
\pgfpathmoveto{\pgfqpoint{4.030793in}{1.931130in}}%
\pgfusepath{stroke}%
\end{pgfscope}%
\begin{pgfscope}%
\pgfpathrectangle{\pgfqpoint{0.765000in}{0.660000in}}{\pgfqpoint{4.620000in}{4.620000in}}%
\pgfusepath{clip}%
\pgfsetbuttcap%
\pgfsetroundjoin%
\definecolor{currentfill}{rgb}{1.000000,0.576471,0.309804}%
\pgfsetfillcolor{currentfill}%
\pgfsetlinewidth{1.003750pt}%
\definecolor{currentstroke}{rgb}{1.000000,0.576471,0.309804}%
\pgfsetstrokecolor{currentstroke}%
\pgfsetdash{}{0pt}%
\pgfsys@defobject{currentmarker}{\pgfqpoint{-0.033333in}{-0.033333in}}{\pgfqpoint{0.033333in}{0.033333in}}{%
\pgfpathmoveto{\pgfqpoint{0.000000in}{-0.033333in}}%
\pgfpathcurveto{\pgfqpoint{0.008840in}{-0.033333in}}{\pgfqpoint{0.017319in}{-0.029821in}}{\pgfqpoint{0.023570in}{-0.023570in}}%
\pgfpathcurveto{\pgfqpoint{0.029821in}{-0.017319in}}{\pgfqpoint{0.033333in}{-0.008840in}}{\pgfqpoint{0.033333in}{0.000000in}}%
\pgfpathcurveto{\pgfqpoint{0.033333in}{0.008840in}}{\pgfqpoint{0.029821in}{0.017319in}}{\pgfqpoint{0.023570in}{0.023570in}}%
\pgfpathcurveto{\pgfqpoint{0.017319in}{0.029821in}}{\pgfqpoint{0.008840in}{0.033333in}}{\pgfqpoint{0.000000in}{0.033333in}}%
\pgfpathcurveto{\pgfqpoint{-0.008840in}{0.033333in}}{\pgfqpoint{-0.017319in}{0.029821in}}{\pgfqpoint{-0.023570in}{0.023570in}}%
\pgfpathcurveto{\pgfqpoint{-0.029821in}{0.017319in}}{\pgfqpoint{-0.033333in}{0.008840in}}{\pgfqpoint{-0.033333in}{0.000000in}}%
\pgfpathcurveto{\pgfqpoint{-0.033333in}{-0.008840in}}{\pgfqpoint{-0.029821in}{-0.017319in}}{\pgfqpoint{-0.023570in}{-0.023570in}}%
\pgfpathcurveto{\pgfqpoint{-0.017319in}{-0.029821in}}{\pgfqpoint{-0.008840in}{-0.033333in}}{\pgfqpoint{0.000000in}{-0.033333in}}%
\pgfpathlineto{\pgfqpoint{0.000000in}{-0.033333in}}%
\pgfpathclose%
\pgfusepath{stroke,fill}%
}%
\begin{pgfscope}%
\pgfsys@transformshift{4.030793in}{1.931130in}%
\pgfsys@useobject{currentmarker}{}%
\end{pgfscope}%
\end{pgfscope}%
\begin{pgfscope}%
\pgfpathrectangle{\pgfqpoint{0.765000in}{0.660000in}}{\pgfqpoint{4.620000in}{4.620000in}}%
\pgfusepath{clip}%
\pgfsetrectcap%
\pgfsetroundjoin%
\pgfsetlinewidth{1.204500pt}%
\definecolor{currentstroke}{rgb}{1.000000,0.576471,0.309804}%
\pgfsetstrokecolor{currentstroke}%
\pgfsetdash{}{0pt}%
\pgfpathmoveto{\pgfqpoint{1.903257in}{2.979147in}}%
\pgfusepath{stroke}%
\end{pgfscope}%
\begin{pgfscope}%
\pgfpathrectangle{\pgfqpoint{0.765000in}{0.660000in}}{\pgfqpoint{4.620000in}{4.620000in}}%
\pgfusepath{clip}%
\pgfsetbuttcap%
\pgfsetroundjoin%
\definecolor{currentfill}{rgb}{1.000000,0.576471,0.309804}%
\pgfsetfillcolor{currentfill}%
\pgfsetlinewidth{1.003750pt}%
\definecolor{currentstroke}{rgb}{1.000000,0.576471,0.309804}%
\pgfsetstrokecolor{currentstroke}%
\pgfsetdash{}{0pt}%
\pgfsys@defobject{currentmarker}{\pgfqpoint{-0.033333in}{-0.033333in}}{\pgfqpoint{0.033333in}{0.033333in}}{%
\pgfpathmoveto{\pgfqpoint{0.000000in}{-0.033333in}}%
\pgfpathcurveto{\pgfqpoint{0.008840in}{-0.033333in}}{\pgfqpoint{0.017319in}{-0.029821in}}{\pgfqpoint{0.023570in}{-0.023570in}}%
\pgfpathcurveto{\pgfqpoint{0.029821in}{-0.017319in}}{\pgfqpoint{0.033333in}{-0.008840in}}{\pgfqpoint{0.033333in}{0.000000in}}%
\pgfpathcurveto{\pgfqpoint{0.033333in}{0.008840in}}{\pgfqpoint{0.029821in}{0.017319in}}{\pgfqpoint{0.023570in}{0.023570in}}%
\pgfpathcurveto{\pgfqpoint{0.017319in}{0.029821in}}{\pgfqpoint{0.008840in}{0.033333in}}{\pgfqpoint{0.000000in}{0.033333in}}%
\pgfpathcurveto{\pgfqpoint{-0.008840in}{0.033333in}}{\pgfqpoint{-0.017319in}{0.029821in}}{\pgfqpoint{-0.023570in}{0.023570in}}%
\pgfpathcurveto{\pgfqpoint{-0.029821in}{0.017319in}}{\pgfqpoint{-0.033333in}{0.008840in}}{\pgfqpoint{-0.033333in}{0.000000in}}%
\pgfpathcurveto{\pgfqpoint{-0.033333in}{-0.008840in}}{\pgfqpoint{-0.029821in}{-0.017319in}}{\pgfqpoint{-0.023570in}{-0.023570in}}%
\pgfpathcurveto{\pgfqpoint{-0.017319in}{-0.029821in}}{\pgfqpoint{-0.008840in}{-0.033333in}}{\pgfqpoint{0.000000in}{-0.033333in}}%
\pgfpathlineto{\pgfqpoint{0.000000in}{-0.033333in}}%
\pgfpathclose%
\pgfusepath{stroke,fill}%
}%
\begin{pgfscope}%
\pgfsys@transformshift{1.903257in}{2.979147in}%
\pgfsys@useobject{currentmarker}{}%
\end{pgfscope}%
\end{pgfscope}%
\begin{pgfscope}%
\pgfpathrectangle{\pgfqpoint{0.765000in}{0.660000in}}{\pgfqpoint{4.620000in}{4.620000in}}%
\pgfusepath{clip}%
\pgfsetrectcap%
\pgfsetroundjoin%
\pgfsetlinewidth{1.204500pt}%
\definecolor{currentstroke}{rgb}{1.000000,0.576471,0.309804}%
\pgfsetstrokecolor{currentstroke}%
\pgfsetdash{}{0pt}%
\pgfpathmoveto{\pgfqpoint{3.470591in}{2.706305in}}%
\pgfusepath{stroke}%
\end{pgfscope}%
\begin{pgfscope}%
\pgfpathrectangle{\pgfqpoint{0.765000in}{0.660000in}}{\pgfqpoint{4.620000in}{4.620000in}}%
\pgfusepath{clip}%
\pgfsetbuttcap%
\pgfsetroundjoin%
\definecolor{currentfill}{rgb}{1.000000,0.576471,0.309804}%
\pgfsetfillcolor{currentfill}%
\pgfsetlinewidth{1.003750pt}%
\definecolor{currentstroke}{rgb}{1.000000,0.576471,0.309804}%
\pgfsetstrokecolor{currentstroke}%
\pgfsetdash{}{0pt}%
\pgfsys@defobject{currentmarker}{\pgfqpoint{-0.033333in}{-0.033333in}}{\pgfqpoint{0.033333in}{0.033333in}}{%
\pgfpathmoveto{\pgfqpoint{0.000000in}{-0.033333in}}%
\pgfpathcurveto{\pgfqpoint{0.008840in}{-0.033333in}}{\pgfqpoint{0.017319in}{-0.029821in}}{\pgfqpoint{0.023570in}{-0.023570in}}%
\pgfpathcurveto{\pgfqpoint{0.029821in}{-0.017319in}}{\pgfqpoint{0.033333in}{-0.008840in}}{\pgfqpoint{0.033333in}{0.000000in}}%
\pgfpathcurveto{\pgfqpoint{0.033333in}{0.008840in}}{\pgfqpoint{0.029821in}{0.017319in}}{\pgfqpoint{0.023570in}{0.023570in}}%
\pgfpathcurveto{\pgfqpoint{0.017319in}{0.029821in}}{\pgfqpoint{0.008840in}{0.033333in}}{\pgfqpoint{0.000000in}{0.033333in}}%
\pgfpathcurveto{\pgfqpoint{-0.008840in}{0.033333in}}{\pgfqpoint{-0.017319in}{0.029821in}}{\pgfqpoint{-0.023570in}{0.023570in}}%
\pgfpathcurveto{\pgfqpoint{-0.029821in}{0.017319in}}{\pgfqpoint{-0.033333in}{0.008840in}}{\pgfqpoint{-0.033333in}{0.000000in}}%
\pgfpathcurveto{\pgfqpoint{-0.033333in}{-0.008840in}}{\pgfqpoint{-0.029821in}{-0.017319in}}{\pgfqpoint{-0.023570in}{-0.023570in}}%
\pgfpathcurveto{\pgfqpoint{-0.017319in}{-0.029821in}}{\pgfqpoint{-0.008840in}{-0.033333in}}{\pgfqpoint{0.000000in}{-0.033333in}}%
\pgfpathlineto{\pgfqpoint{0.000000in}{-0.033333in}}%
\pgfpathclose%
\pgfusepath{stroke,fill}%
}%
\begin{pgfscope}%
\pgfsys@transformshift{3.470591in}{2.706305in}%
\pgfsys@useobject{currentmarker}{}%
\end{pgfscope}%
\end{pgfscope}%
\begin{pgfscope}%
\pgfpathrectangle{\pgfqpoint{0.765000in}{0.660000in}}{\pgfqpoint{4.620000in}{4.620000in}}%
\pgfusepath{clip}%
\pgfsetrectcap%
\pgfsetroundjoin%
\pgfsetlinewidth{1.204500pt}%
\definecolor{currentstroke}{rgb}{1.000000,0.576471,0.309804}%
\pgfsetstrokecolor{currentstroke}%
\pgfsetdash{}{0pt}%
\pgfpathmoveto{\pgfqpoint{3.624155in}{2.904750in}}%
\pgfusepath{stroke}%
\end{pgfscope}%
\begin{pgfscope}%
\pgfpathrectangle{\pgfqpoint{0.765000in}{0.660000in}}{\pgfqpoint{4.620000in}{4.620000in}}%
\pgfusepath{clip}%
\pgfsetbuttcap%
\pgfsetroundjoin%
\definecolor{currentfill}{rgb}{1.000000,0.576471,0.309804}%
\pgfsetfillcolor{currentfill}%
\pgfsetlinewidth{1.003750pt}%
\definecolor{currentstroke}{rgb}{1.000000,0.576471,0.309804}%
\pgfsetstrokecolor{currentstroke}%
\pgfsetdash{}{0pt}%
\pgfsys@defobject{currentmarker}{\pgfqpoint{-0.033333in}{-0.033333in}}{\pgfqpoint{0.033333in}{0.033333in}}{%
\pgfpathmoveto{\pgfqpoint{0.000000in}{-0.033333in}}%
\pgfpathcurveto{\pgfqpoint{0.008840in}{-0.033333in}}{\pgfqpoint{0.017319in}{-0.029821in}}{\pgfqpoint{0.023570in}{-0.023570in}}%
\pgfpathcurveto{\pgfqpoint{0.029821in}{-0.017319in}}{\pgfqpoint{0.033333in}{-0.008840in}}{\pgfqpoint{0.033333in}{0.000000in}}%
\pgfpathcurveto{\pgfqpoint{0.033333in}{0.008840in}}{\pgfqpoint{0.029821in}{0.017319in}}{\pgfqpoint{0.023570in}{0.023570in}}%
\pgfpathcurveto{\pgfqpoint{0.017319in}{0.029821in}}{\pgfqpoint{0.008840in}{0.033333in}}{\pgfqpoint{0.000000in}{0.033333in}}%
\pgfpathcurveto{\pgfqpoint{-0.008840in}{0.033333in}}{\pgfqpoint{-0.017319in}{0.029821in}}{\pgfqpoint{-0.023570in}{0.023570in}}%
\pgfpathcurveto{\pgfqpoint{-0.029821in}{0.017319in}}{\pgfqpoint{-0.033333in}{0.008840in}}{\pgfqpoint{-0.033333in}{0.000000in}}%
\pgfpathcurveto{\pgfqpoint{-0.033333in}{-0.008840in}}{\pgfqpoint{-0.029821in}{-0.017319in}}{\pgfqpoint{-0.023570in}{-0.023570in}}%
\pgfpathcurveto{\pgfqpoint{-0.017319in}{-0.029821in}}{\pgfqpoint{-0.008840in}{-0.033333in}}{\pgfqpoint{0.000000in}{-0.033333in}}%
\pgfpathlineto{\pgfqpoint{0.000000in}{-0.033333in}}%
\pgfpathclose%
\pgfusepath{stroke,fill}%
}%
\begin{pgfscope}%
\pgfsys@transformshift{3.624155in}{2.904750in}%
\pgfsys@useobject{currentmarker}{}%
\end{pgfscope}%
\end{pgfscope}%
\begin{pgfscope}%
\pgfpathrectangle{\pgfqpoint{0.765000in}{0.660000in}}{\pgfqpoint{4.620000in}{4.620000in}}%
\pgfusepath{clip}%
\pgfsetrectcap%
\pgfsetroundjoin%
\pgfsetlinewidth{1.204500pt}%
\definecolor{currentstroke}{rgb}{1.000000,0.576471,0.309804}%
\pgfsetstrokecolor{currentstroke}%
\pgfsetdash{}{0pt}%
\pgfpathmoveto{\pgfqpoint{2.806474in}{1.807308in}}%
\pgfusepath{stroke}%
\end{pgfscope}%
\begin{pgfscope}%
\pgfpathrectangle{\pgfqpoint{0.765000in}{0.660000in}}{\pgfqpoint{4.620000in}{4.620000in}}%
\pgfusepath{clip}%
\pgfsetbuttcap%
\pgfsetroundjoin%
\definecolor{currentfill}{rgb}{1.000000,0.576471,0.309804}%
\pgfsetfillcolor{currentfill}%
\pgfsetlinewidth{1.003750pt}%
\definecolor{currentstroke}{rgb}{1.000000,0.576471,0.309804}%
\pgfsetstrokecolor{currentstroke}%
\pgfsetdash{}{0pt}%
\pgfsys@defobject{currentmarker}{\pgfqpoint{-0.033333in}{-0.033333in}}{\pgfqpoint{0.033333in}{0.033333in}}{%
\pgfpathmoveto{\pgfqpoint{0.000000in}{-0.033333in}}%
\pgfpathcurveto{\pgfqpoint{0.008840in}{-0.033333in}}{\pgfqpoint{0.017319in}{-0.029821in}}{\pgfqpoint{0.023570in}{-0.023570in}}%
\pgfpathcurveto{\pgfqpoint{0.029821in}{-0.017319in}}{\pgfqpoint{0.033333in}{-0.008840in}}{\pgfqpoint{0.033333in}{0.000000in}}%
\pgfpathcurveto{\pgfqpoint{0.033333in}{0.008840in}}{\pgfqpoint{0.029821in}{0.017319in}}{\pgfqpoint{0.023570in}{0.023570in}}%
\pgfpathcurveto{\pgfqpoint{0.017319in}{0.029821in}}{\pgfqpoint{0.008840in}{0.033333in}}{\pgfqpoint{0.000000in}{0.033333in}}%
\pgfpathcurveto{\pgfqpoint{-0.008840in}{0.033333in}}{\pgfqpoint{-0.017319in}{0.029821in}}{\pgfqpoint{-0.023570in}{0.023570in}}%
\pgfpathcurveto{\pgfqpoint{-0.029821in}{0.017319in}}{\pgfqpoint{-0.033333in}{0.008840in}}{\pgfqpoint{-0.033333in}{0.000000in}}%
\pgfpathcurveto{\pgfqpoint{-0.033333in}{-0.008840in}}{\pgfqpoint{-0.029821in}{-0.017319in}}{\pgfqpoint{-0.023570in}{-0.023570in}}%
\pgfpathcurveto{\pgfqpoint{-0.017319in}{-0.029821in}}{\pgfqpoint{-0.008840in}{-0.033333in}}{\pgfqpoint{0.000000in}{-0.033333in}}%
\pgfpathlineto{\pgfqpoint{0.000000in}{-0.033333in}}%
\pgfpathclose%
\pgfusepath{stroke,fill}%
}%
\begin{pgfscope}%
\pgfsys@transformshift{2.806474in}{1.807308in}%
\pgfsys@useobject{currentmarker}{}%
\end{pgfscope}%
\end{pgfscope}%
\begin{pgfscope}%
\pgfpathrectangle{\pgfqpoint{0.765000in}{0.660000in}}{\pgfqpoint{4.620000in}{4.620000in}}%
\pgfusepath{clip}%
\pgfsetrectcap%
\pgfsetroundjoin%
\pgfsetlinewidth{1.204500pt}%
\definecolor{currentstroke}{rgb}{1.000000,0.576471,0.309804}%
\pgfsetstrokecolor{currentstroke}%
\pgfsetdash{}{0pt}%
\pgfpathmoveto{\pgfqpoint{3.469826in}{2.766789in}}%
\pgfusepath{stroke}%
\end{pgfscope}%
\begin{pgfscope}%
\pgfpathrectangle{\pgfqpoint{0.765000in}{0.660000in}}{\pgfqpoint{4.620000in}{4.620000in}}%
\pgfusepath{clip}%
\pgfsetbuttcap%
\pgfsetroundjoin%
\definecolor{currentfill}{rgb}{1.000000,0.576471,0.309804}%
\pgfsetfillcolor{currentfill}%
\pgfsetlinewidth{1.003750pt}%
\definecolor{currentstroke}{rgb}{1.000000,0.576471,0.309804}%
\pgfsetstrokecolor{currentstroke}%
\pgfsetdash{}{0pt}%
\pgfsys@defobject{currentmarker}{\pgfqpoint{-0.033333in}{-0.033333in}}{\pgfqpoint{0.033333in}{0.033333in}}{%
\pgfpathmoveto{\pgfqpoint{0.000000in}{-0.033333in}}%
\pgfpathcurveto{\pgfqpoint{0.008840in}{-0.033333in}}{\pgfqpoint{0.017319in}{-0.029821in}}{\pgfqpoint{0.023570in}{-0.023570in}}%
\pgfpathcurveto{\pgfqpoint{0.029821in}{-0.017319in}}{\pgfqpoint{0.033333in}{-0.008840in}}{\pgfqpoint{0.033333in}{0.000000in}}%
\pgfpathcurveto{\pgfqpoint{0.033333in}{0.008840in}}{\pgfqpoint{0.029821in}{0.017319in}}{\pgfqpoint{0.023570in}{0.023570in}}%
\pgfpathcurveto{\pgfqpoint{0.017319in}{0.029821in}}{\pgfqpoint{0.008840in}{0.033333in}}{\pgfqpoint{0.000000in}{0.033333in}}%
\pgfpathcurveto{\pgfqpoint{-0.008840in}{0.033333in}}{\pgfqpoint{-0.017319in}{0.029821in}}{\pgfqpoint{-0.023570in}{0.023570in}}%
\pgfpathcurveto{\pgfqpoint{-0.029821in}{0.017319in}}{\pgfqpoint{-0.033333in}{0.008840in}}{\pgfqpoint{-0.033333in}{0.000000in}}%
\pgfpathcurveto{\pgfqpoint{-0.033333in}{-0.008840in}}{\pgfqpoint{-0.029821in}{-0.017319in}}{\pgfqpoint{-0.023570in}{-0.023570in}}%
\pgfpathcurveto{\pgfqpoint{-0.017319in}{-0.029821in}}{\pgfqpoint{-0.008840in}{-0.033333in}}{\pgfqpoint{0.000000in}{-0.033333in}}%
\pgfpathlineto{\pgfqpoint{0.000000in}{-0.033333in}}%
\pgfpathclose%
\pgfusepath{stroke,fill}%
}%
\begin{pgfscope}%
\pgfsys@transformshift{3.469826in}{2.766789in}%
\pgfsys@useobject{currentmarker}{}%
\end{pgfscope}%
\end{pgfscope}%
\begin{pgfscope}%
\pgfpathrectangle{\pgfqpoint{0.765000in}{0.660000in}}{\pgfqpoint{4.620000in}{4.620000in}}%
\pgfusepath{clip}%
\pgfsetrectcap%
\pgfsetroundjoin%
\pgfsetlinewidth{1.204500pt}%
\definecolor{currentstroke}{rgb}{1.000000,0.576471,0.309804}%
\pgfsetstrokecolor{currentstroke}%
\pgfsetdash{}{0pt}%
\pgfpathmoveto{\pgfqpoint{1.931154in}{3.039864in}}%
\pgfusepath{stroke}%
\end{pgfscope}%
\begin{pgfscope}%
\pgfpathrectangle{\pgfqpoint{0.765000in}{0.660000in}}{\pgfqpoint{4.620000in}{4.620000in}}%
\pgfusepath{clip}%
\pgfsetbuttcap%
\pgfsetroundjoin%
\definecolor{currentfill}{rgb}{1.000000,0.576471,0.309804}%
\pgfsetfillcolor{currentfill}%
\pgfsetlinewidth{1.003750pt}%
\definecolor{currentstroke}{rgb}{1.000000,0.576471,0.309804}%
\pgfsetstrokecolor{currentstroke}%
\pgfsetdash{}{0pt}%
\pgfsys@defobject{currentmarker}{\pgfqpoint{-0.033333in}{-0.033333in}}{\pgfqpoint{0.033333in}{0.033333in}}{%
\pgfpathmoveto{\pgfqpoint{0.000000in}{-0.033333in}}%
\pgfpathcurveto{\pgfqpoint{0.008840in}{-0.033333in}}{\pgfqpoint{0.017319in}{-0.029821in}}{\pgfqpoint{0.023570in}{-0.023570in}}%
\pgfpathcurveto{\pgfqpoint{0.029821in}{-0.017319in}}{\pgfqpoint{0.033333in}{-0.008840in}}{\pgfqpoint{0.033333in}{0.000000in}}%
\pgfpathcurveto{\pgfqpoint{0.033333in}{0.008840in}}{\pgfqpoint{0.029821in}{0.017319in}}{\pgfqpoint{0.023570in}{0.023570in}}%
\pgfpathcurveto{\pgfqpoint{0.017319in}{0.029821in}}{\pgfqpoint{0.008840in}{0.033333in}}{\pgfqpoint{0.000000in}{0.033333in}}%
\pgfpathcurveto{\pgfqpoint{-0.008840in}{0.033333in}}{\pgfqpoint{-0.017319in}{0.029821in}}{\pgfqpoint{-0.023570in}{0.023570in}}%
\pgfpathcurveto{\pgfqpoint{-0.029821in}{0.017319in}}{\pgfqpoint{-0.033333in}{0.008840in}}{\pgfqpoint{-0.033333in}{0.000000in}}%
\pgfpathcurveto{\pgfqpoint{-0.033333in}{-0.008840in}}{\pgfqpoint{-0.029821in}{-0.017319in}}{\pgfqpoint{-0.023570in}{-0.023570in}}%
\pgfpathcurveto{\pgfqpoint{-0.017319in}{-0.029821in}}{\pgfqpoint{-0.008840in}{-0.033333in}}{\pgfqpoint{0.000000in}{-0.033333in}}%
\pgfpathlineto{\pgfqpoint{0.000000in}{-0.033333in}}%
\pgfpathclose%
\pgfusepath{stroke,fill}%
}%
\begin{pgfscope}%
\pgfsys@transformshift{1.931154in}{3.039864in}%
\pgfsys@useobject{currentmarker}{}%
\end{pgfscope}%
\end{pgfscope}%
\begin{pgfscope}%
\pgfpathrectangle{\pgfqpoint{0.765000in}{0.660000in}}{\pgfqpoint{4.620000in}{4.620000in}}%
\pgfusepath{clip}%
\pgfsetrectcap%
\pgfsetroundjoin%
\pgfsetlinewidth{1.204500pt}%
\definecolor{currentstroke}{rgb}{1.000000,0.576471,0.309804}%
\pgfsetstrokecolor{currentstroke}%
\pgfsetdash{}{0pt}%
\pgfpathmoveto{\pgfqpoint{2.153310in}{2.776091in}}%
\pgfusepath{stroke}%
\end{pgfscope}%
\begin{pgfscope}%
\pgfpathrectangle{\pgfqpoint{0.765000in}{0.660000in}}{\pgfqpoint{4.620000in}{4.620000in}}%
\pgfusepath{clip}%
\pgfsetbuttcap%
\pgfsetroundjoin%
\definecolor{currentfill}{rgb}{1.000000,0.576471,0.309804}%
\pgfsetfillcolor{currentfill}%
\pgfsetlinewidth{1.003750pt}%
\definecolor{currentstroke}{rgb}{1.000000,0.576471,0.309804}%
\pgfsetstrokecolor{currentstroke}%
\pgfsetdash{}{0pt}%
\pgfsys@defobject{currentmarker}{\pgfqpoint{-0.033333in}{-0.033333in}}{\pgfqpoint{0.033333in}{0.033333in}}{%
\pgfpathmoveto{\pgfqpoint{0.000000in}{-0.033333in}}%
\pgfpathcurveto{\pgfqpoint{0.008840in}{-0.033333in}}{\pgfqpoint{0.017319in}{-0.029821in}}{\pgfqpoint{0.023570in}{-0.023570in}}%
\pgfpathcurveto{\pgfqpoint{0.029821in}{-0.017319in}}{\pgfqpoint{0.033333in}{-0.008840in}}{\pgfqpoint{0.033333in}{0.000000in}}%
\pgfpathcurveto{\pgfqpoint{0.033333in}{0.008840in}}{\pgfqpoint{0.029821in}{0.017319in}}{\pgfqpoint{0.023570in}{0.023570in}}%
\pgfpathcurveto{\pgfqpoint{0.017319in}{0.029821in}}{\pgfqpoint{0.008840in}{0.033333in}}{\pgfqpoint{0.000000in}{0.033333in}}%
\pgfpathcurveto{\pgfqpoint{-0.008840in}{0.033333in}}{\pgfqpoint{-0.017319in}{0.029821in}}{\pgfqpoint{-0.023570in}{0.023570in}}%
\pgfpathcurveto{\pgfqpoint{-0.029821in}{0.017319in}}{\pgfqpoint{-0.033333in}{0.008840in}}{\pgfqpoint{-0.033333in}{0.000000in}}%
\pgfpathcurveto{\pgfqpoint{-0.033333in}{-0.008840in}}{\pgfqpoint{-0.029821in}{-0.017319in}}{\pgfqpoint{-0.023570in}{-0.023570in}}%
\pgfpathcurveto{\pgfqpoint{-0.017319in}{-0.029821in}}{\pgfqpoint{-0.008840in}{-0.033333in}}{\pgfqpoint{0.000000in}{-0.033333in}}%
\pgfpathlineto{\pgfqpoint{0.000000in}{-0.033333in}}%
\pgfpathclose%
\pgfusepath{stroke,fill}%
}%
\begin{pgfscope}%
\pgfsys@transformshift{2.153310in}{2.776091in}%
\pgfsys@useobject{currentmarker}{}%
\end{pgfscope}%
\end{pgfscope}%
\begin{pgfscope}%
\pgfpathrectangle{\pgfqpoint{0.765000in}{0.660000in}}{\pgfqpoint{4.620000in}{4.620000in}}%
\pgfusepath{clip}%
\pgfsetrectcap%
\pgfsetroundjoin%
\pgfsetlinewidth{1.204500pt}%
\definecolor{currentstroke}{rgb}{1.000000,0.576471,0.309804}%
\pgfsetstrokecolor{currentstroke}%
\pgfsetdash{}{0pt}%
\pgfpathmoveto{\pgfqpoint{3.716830in}{2.751902in}}%
\pgfusepath{stroke}%
\end{pgfscope}%
\begin{pgfscope}%
\pgfpathrectangle{\pgfqpoint{0.765000in}{0.660000in}}{\pgfqpoint{4.620000in}{4.620000in}}%
\pgfusepath{clip}%
\pgfsetbuttcap%
\pgfsetroundjoin%
\definecolor{currentfill}{rgb}{1.000000,0.576471,0.309804}%
\pgfsetfillcolor{currentfill}%
\pgfsetlinewidth{1.003750pt}%
\definecolor{currentstroke}{rgb}{1.000000,0.576471,0.309804}%
\pgfsetstrokecolor{currentstroke}%
\pgfsetdash{}{0pt}%
\pgfsys@defobject{currentmarker}{\pgfqpoint{-0.033333in}{-0.033333in}}{\pgfqpoint{0.033333in}{0.033333in}}{%
\pgfpathmoveto{\pgfqpoint{0.000000in}{-0.033333in}}%
\pgfpathcurveto{\pgfqpoint{0.008840in}{-0.033333in}}{\pgfqpoint{0.017319in}{-0.029821in}}{\pgfqpoint{0.023570in}{-0.023570in}}%
\pgfpathcurveto{\pgfqpoint{0.029821in}{-0.017319in}}{\pgfqpoint{0.033333in}{-0.008840in}}{\pgfqpoint{0.033333in}{0.000000in}}%
\pgfpathcurveto{\pgfqpoint{0.033333in}{0.008840in}}{\pgfqpoint{0.029821in}{0.017319in}}{\pgfqpoint{0.023570in}{0.023570in}}%
\pgfpathcurveto{\pgfqpoint{0.017319in}{0.029821in}}{\pgfqpoint{0.008840in}{0.033333in}}{\pgfqpoint{0.000000in}{0.033333in}}%
\pgfpathcurveto{\pgfqpoint{-0.008840in}{0.033333in}}{\pgfqpoint{-0.017319in}{0.029821in}}{\pgfqpoint{-0.023570in}{0.023570in}}%
\pgfpathcurveto{\pgfqpoint{-0.029821in}{0.017319in}}{\pgfqpoint{-0.033333in}{0.008840in}}{\pgfqpoint{-0.033333in}{0.000000in}}%
\pgfpathcurveto{\pgfqpoint{-0.033333in}{-0.008840in}}{\pgfqpoint{-0.029821in}{-0.017319in}}{\pgfqpoint{-0.023570in}{-0.023570in}}%
\pgfpathcurveto{\pgfqpoint{-0.017319in}{-0.029821in}}{\pgfqpoint{-0.008840in}{-0.033333in}}{\pgfqpoint{0.000000in}{-0.033333in}}%
\pgfpathlineto{\pgfqpoint{0.000000in}{-0.033333in}}%
\pgfpathclose%
\pgfusepath{stroke,fill}%
}%
\begin{pgfscope}%
\pgfsys@transformshift{3.716830in}{2.751902in}%
\pgfsys@useobject{currentmarker}{}%
\end{pgfscope}%
\end{pgfscope}%
\begin{pgfscope}%
\pgfpathrectangle{\pgfqpoint{0.765000in}{0.660000in}}{\pgfqpoint{4.620000in}{4.620000in}}%
\pgfusepath{clip}%
\pgfsetrectcap%
\pgfsetroundjoin%
\pgfsetlinewidth{1.204500pt}%
\definecolor{currentstroke}{rgb}{1.000000,0.576471,0.309804}%
\pgfsetstrokecolor{currentstroke}%
\pgfsetdash{}{0pt}%
\pgfpathmoveto{\pgfqpoint{3.065764in}{2.532200in}}%
\pgfusepath{stroke}%
\end{pgfscope}%
\begin{pgfscope}%
\pgfpathrectangle{\pgfqpoint{0.765000in}{0.660000in}}{\pgfqpoint{4.620000in}{4.620000in}}%
\pgfusepath{clip}%
\pgfsetbuttcap%
\pgfsetroundjoin%
\definecolor{currentfill}{rgb}{1.000000,0.576471,0.309804}%
\pgfsetfillcolor{currentfill}%
\pgfsetlinewidth{1.003750pt}%
\definecolor{currentstroke}{rgb}{1.000000,0.576471,0.309804}%
\pgfsetstrokecolor{currentstroke}%
\pgfsetdash{}{0pt}%
\pgfsys@defobject{currentmarker}{\pgfqpoint{-0.033333in}{-0.033333in}}{\pgfqpoint{0.033333in}{0.033333in}}{%
\pgfpathmoveto{\pgfqpoint{0.000000in}{-0.033333in}}%
\pgfpathcurveto{\pgfqpoint{0.008840in}{-0.033333in}}{\pgfqpoint{0.017319in}{-0.029821in}}{\pgfqpoint{0.023570in}{-0.023570in}}%
\pgfpathcurveto{\pgfqpoint{0.029821in}{-0.017319in}}{\pgfqpoint{0.033333in}{-0.008840in}}{\pgfqpoint{0.033333in}{0.000000in}}%
\pgfpathcurveto{\pgfqpoint{0.033333in}{0.008840in}}{\pgfqpoint{0.029821in}{0.017319in}}{\pgfqpoint{0.023570in}{0.023570in}}%
\pgfpathcurveto{\pgfqpoint{0.017319in}{0.029821in}}{\pgfqpoint{0.008840in}{0.033333in}}{\pgfqpoint{0.000000in}{0.033333in}}%
\pgfpathcurveto{\pgfqpoint{-0.008840in}{0.033333in}}{\pgfqpoint{-0.017319in}{0.029821in}}{\pgfqpoint{-0.023570in}{0.023570in}}%
\pgfpathcurveto{\pgfqpoint{-0.029821in}{0.017319in}}{\pgfqpoint{-0.033333in}{0.008840in}}{\pgfqpoint{-0.033333in}{0.000000in}}%
\pgfpathcurveto{\pgfqpoint{-0.033333in}{-0.008840in}}{\pgfqpoint{-0.029821in}{-0.017319in}}{\pgfqpoint{-0.023570in}{-0.023570in}}%
\pgfpathcurveto{\pgfqpoint{-0.017319in}{-0.029821in}}{\pgfqpoint{-0.008840in}{-0.033333in}}{\pgfqpoint{0.000000in}{-0.033333in}}%
\pgfpathlineto{\pgfqpoint{0.000000in}{-0.033333in}}%
\pgfpathclose%
\pgfusepath{stroke,fill}%
}%
\begin{pgfscope}%
\pgfsys@transformshift{3.065764in}{2.532200in}%
\pgfsys@useobject{currentmarker}{}%
\end{pgfscope}%
\end{pgfscope}%
\begin{pgfscope}%
\pgfpathrectangle{\pgfqpoint{0.765000in}{0.660000in}}{\pgfqpoint{4.620000in}{4.620000in}}%
\pgfusepath{clip}%
\pgfsetrectcap%
\pgfsetroundjoin%
\pgfsetlinewidth{1.204500pt}%
\definecolor{currentstroke}{rgb}{1.000000,0.576471,0.309804}%
\pgfsetstrokecolor{currentstroke}%
\pgfsetdash{}{0pt}%
\pgfpathmoveto{\pgfqpoint{2.764522in}{1.292882in}}%
\pgfusepath{stroke}%
\end{pgfscope}%
\begin{pgfscope}%
\pgfpathrectangle{\pgfqpoint{0.765000in}{0.660000in}}{\pgfqpoint{4.620000in}{4.620000in}}%
\pgfusepath{clip}%
\pgfsetbuttcap%
\pgfsetroundjoin%
\definecolor{currentfill}{rgb}{1.000000,0.576471,0.309804}%
\pgfsetfillcolor{currentfill}%
\pgfsetlinewidth{1.003750pt}%
\definecolor{currentstroke}{rgb}{1.000000,0.576471,0.309804}%
\pgfsetstrokecolor{currentstroke}%
\pgfsetdash{}{0pt}%
\pgfsys@defobject{currentmarker}{\pgfqpoint{-0.033333in}{-0.033333in}}{\pgfqpoint{0.033333in}{0.033333in}}{%
\pgfpathmoveto{\pgfqpoint{0.000000in}{-0.033333in}}%
\pgfpathcurveto{\pgfqpoint{0.008840in}{-0.033333in}}{\pgfqpoint{0.017319in}{-0.029821in}}{\pgfqpoint{0.023570in}{-0.023570in}}%
\pgfpathcurveto{\pgfqpoint{0.029821in}{-0.017319in}}{\pgfqpoint{0.033333in}{-0.008840in}}{\pgfqpoint{0.033333in}{0.000000in}}%
\pgfpathcurveto{\pgfqpoint{0.033333in}{0.008840in}}{\pgfqpoint{0.029821in}{0.017319in}}{\pgfqpoint{0.023570in}{0.023570in}}%
\pgfpathcurveto{\pgfqpoint{0.017319in}{0.029821in}}{\pgfqpoint{0.008840in}{0.033333in}}{\pgfqpoint{0.000000in}{0.033333in}}%
\pgfpathcurveto{\pgfqpoint{-0.008840in}{0.033333in}}{\pgfqpoint{-0.017319in}{0.029821in}}{\pgfqpoint{-0.023570in}{0.023570in}}%
\pgfpathcurveto{\pgfqpoint{-0.029821in}{0.017319in}}{\pgfqpoint{-0.033333in}{0.008840in}}{\pgfqpoint{-0.033333in}{0.000000in}}%
\pgfpathcurveto{\pgfqpoint{-0.033333in}{-0.008840in}}{\pgfqpoint{-0.029821in}{-0.017319in}}{\pgfqpoint{-0.023570in}{-0.023570in}}%
\pgfpathcurveto{\pgfqpoint{-0.017319in}{-0.029821in}}{\pgfqpoint{-0.008840in}{-0.033333in}}{\pgfqpoint{0.000000in}{-0.033333in}}%
\pgfpathlineto{\pgfqpoint{0.000000in}{-0.033333in}}%
\pgfpathclose%
\pgfusepath{stroke,fill}%
}%
\begin{pgfscope}%
\pgfsys@transformshift{2.764522in}{1.292882in}%
\pgfsys@useobject{currentmarker}{}%
\end{pgfscope}%
\end{pgfscope}%
\begin{pgfscope}%
\pgfpathrectangle{\pgfqpoint{0.765000in}{0.660000in}}{\pgfqpoint{4.620000in}{4.620000in}}%
\pgfusepath{clip}%
\pgfsetrectcap%
\pgfsetroundjoin%
\pgfsetlinewidth{1.204500pt}%
\definecolor{currentstroke}{rgb}{1.000000,0.576471,0.309804}%
\pgfsetstrokecolor{currentstroke}%
\pgfsetdash{}{0pt}%
\pgfpathmoveto{\pgfqpoint{4.278859in}{3.020779in}}%
\pgfusepath{stroke}%
\end{pgfscope}%
\begin{pgfscope}%
\pgfpathrectangle{\pgfqpoint{0.765000in}{0.660000in}}{\pgfqpoint{4.620000in}{4.620000in}}%
\pgfusepath{clip}%
\pgfsetbuttcap%
\pgfsetroundjoin%
\definecolor{currentfill}{rgb}{1.000000,0.576471,0.309804}%
\pgfsetfillcolor{currentfill}%
\pgfsetlinewidth{1.003750pt}%
\definecolor{currentstroke}{rgb}{1.000000,0.576471,0.309804}%
\pgfsetstrokecolor{currentstroke}%
\pgfsetdash{}{0pt}%
\pgfsys@defobject{currentmarker}{\pgfqpoint{-0.033333in}{-0.033333in}}{\pgfqpoint{0.033333in}{0.033333in}}{%
\pgfpathmoveto{\pgfqpoint{0.000000in}{-0.033333in}}%
\pgfpathcurveto{\pgfqpoint{0.008840in}{-0.033333in}}{\pgfqpoint{0.017319in}{-0.029821in}}{\pgfqpoint{0.023570in}{-0.023570in}}%
\pgfpathcurveto{\pgfqpoint{0.029821in}{-0.017319in}}{\pgfqpoint{0.033333in}{-0.008840in}}{\pgfqpoint{0.033333in}{0.000000in}}%
\pgfpathcurveto{\pgfqpoint{0.033333in}{0.008840in}}{\pgfqpoint{0.029821in}{0.017319in}}{\pgfqpoint{0.023570in}{0.023570in}}%
\pgfpathcurveto{\pgfqpoint{0.017319in}{0.029821in}}{\pgfqpoint{0.008840in}{0.033333in}}{\pgfqpoint{0.000000in}{0.033333in}}%
\pgfpathcurveto{\pgfqpoint{-0.008840in}{0.033333in}}{\pgfqpoint{-0.017319in}{0.029821in}}{\pgfqpoint{-0.023570in}{0.023570in}}%
\pgfpathcurveto{\pgfqpoint{-0.029821in}{0.017319in}}{\pgfqpoint{-0.033333in}{0.008840in}}{\pgfqpoint{-0.033333in}{0.000000in}}%
\pgfpathcurveto{\pgfqpoint{-0.033333in}{-0.008840in}}{\pgfqpoint{-0.029821in}{-0.017319in}}{\pgfqpoint{-0.023570in}{-0.023570in}}%
\pgfpathcurveto{\pgfqpoint{-0.017319in}{-0.029821in}}{\pgfqpoint{-0.008840in}{-0.033333in}}{\pgfqpoint{0.000000in}{-0.033333in}}%
\pgfpathlineto{\pgfqpoint{0.000000in}{-0.033333in}}%
\pgfpathclose%
\pgfusepath{stroke,fill}%
}%
\begin{pgfscope}%
\pgfsys@transformshift{4.278859in}{3.020779in}%
\pgfsys@useobject{currentmarker}{}%
\end{pgfscope}%
\end{pgfscope}%
\begin{pgfscope}%
\pgfpathrectangle{\pgfqpoint{0.765000in}{0.660000in}}{\pgfqpoint{4.620000in}{4.620000in}}%
\pgfusepath{clip}%
\pgfsetrectcap%
\pgfsetroundjoin%
\pgfsetlinewidth{1.204500pt}%
\definecolor{currentstroke}{rgb}{1.000000,0.576471,0.309804}%
\pgfsetstrokecolor{currentstroke}%
\pgfsetdash{}{0pt}%
\pgfpathmoveto{\pgfqpoint{2.009512in}{2.133593in}}%
\pgfusepath{stroke}%
\end{pgfscope}%
\begin{pgfscope}%
\pgfpathrectangle{\pgfqpoint{0.765000in}{0.660000in}}{\pgfqpoint{4.620000in}{4.620000in}}%
\pgfusepath{clip}%
\pgfsetbuttcap%
\pgfsetroundjoin%
\definecolor{currentfill}{rgb}{1.000000,0.576471,0.309804}%
\pgfsetfillcolor{currentfill}%
\pgfsetlinewidth{1.003750pt}%
\definecolor{currentstroke}{rgb}{1.000000,0.576471,0.309804}%
\pgfsetstrokecolor{currentstroke}%
\pgfsetdash{}{0pt}%
\pgfsys@defobject{currentmarker}{\pgfqpoint{-0.033333in}{-0.033333in}}{\pgfqpoint{0.033333in}{0.033333in}}{%
\pgfpathmoveto{\pgfqpoint{0.000000in}{-0.033333in}}%
\pgfpathcurveto{\pgfqpoint{0.008840in}{-0.033333in}}{\pgfqpoint{0.017319in}{-0.029821in}}{\pgfqpoint{0.023570in}{-0.023570in}}%
\pgfpathcurveto{\pgfqpoint{0.029821in}{-0.017319in}}{\pgfqpoint{0.033333in}{-0.008840in}}{\pgfqpoint{0.033333in}{0.000000in}}%
\pgfpathcurveto{\pgfqpoint{0.033333in}{0.008840in}}{\pgfqpoint{0.029821in}{0.017319in}}{\pgfqpoint{0.023570in}{0.023570in}}%
\pgfpathcurveto{\pgfqpoint{0.017319in}{0.029821in}}{\pgfqpoint{0.008840in}{0.033333in}}{\pgfqpoint{0.000000in}{0.033333in}}%
\pgfpathcurveto{\pgfqpoint{-0.008840in}{0.033333in}}{\pgfqpoint{-0.017319in}{0.029821in}}{\pgfqpoint{-0.023570in}{0.023570in}}%
\pgfpathcurveto{\pgfqpoint{-0.029821in}{0.017319in}}{\pgfqpoint{-0.033333in}{0.008840in}}{\pgfqpoint{-0.033333in}{0.000000in}}%
\pgfpathcurveto{\pgfqpoint{-0.033333in}{-0.008840in}}{\pgfqpoint{-0.029821in}{-0.017319in}}{\pgfqpoint{-0.023570in}{-0.023570in}}%
\pgfpathcurveto{\pgfqpoint{-0.017319in}{-0.029821in}}{\pgfqpoint{-0.008840in}{-0.033333in}}{\pgfqpoint{0.000000in}{-0.033333in}}%
\pgfpathlineto{\pgfqpoint{0.000000in}{-0.033333in}}%
\pgfpathclose%
\pgfusepath{stroke,fill}%
}%
\begin{pgfscope}%
\pgfsys@transformshift{2.009512in}{2.133593in}%
\pgfsys@useobject{currentmarker}{}%
\end{pgfscope}%
\end{pgfscope}%
\begin{pgfscope}%
\pgfpathrectangle{\pgfqpoint{0.765000in}{0.660000in}}{\pgfqpoint{4.620000in}{4.620000in}}%
\pgfusepath{clip}%
\pgfsetrectcap%
\pgfsetroundjoin%
\pgfsetlinewidth{1.204500pt}%
\definecolor{currentstroke}{rgb}{1.000000,0.576471,0.309804}%
\pgfsetstrokecolor{currentstroke}%
\pgfsetdash{}{0pt}%
\pgfpathmoveto{\pgfqpoint{2.677164in}{1.772855in}}%
\pgfusepath{stroke}%
\end{pgfscope}%
\begin{pgfscope}%
\pgfpathrectangle{\pgfqpoint{0.765000in}{0.660000in}}{\pgfqpoint{4.620000in}{4.620000in}}%
\pgfusepath{clip}%
\pgfsetbuttcap%
\pgfsetroundjoin%
\definecolor{currentfill}{rgb}{1.000000,0.576471,0.309804}%
\pgfsetfillcolor{currentfill}%
\pgfsetlinewidth{1.003750pt}%
\definecolor{currentstroke}{rgb}{1.000000,0.576471,0.309804}%
\pgfsetstrokecolor{currentstroke}%
\pgfsetdash{}{0pt}%
\pgfsys@defobject{currentmarker}{\pgfqpoint{-0.033333in}{-0.033333in}}{\pgfqpoint{0.033333in}{0.033333in}}{%
\pgfpathmoveto{\pgfqpoint{0.000000in}{-0.033333in}}%
\pgfpathcurveto{\pgfqpoint{0.008840in}{-0.033333in}}{\pgfqpoint{0.017319in}{-0.029821in}}{\pgfqpoint{0.023570in}{-0.023570in}}%
\pgfpathcurveto{\pgfqpoint{0.029821in}{-0.017319in}}{\pgfqpoint{0.033333in}{-0.008840in}}{\pgfqpoint{0.033333in}{0.000000in}}%
\pgfpathcurveto{\pgfqpoint{0.033333in}{0.008840in}}{\pgfqpoint{0.029821in}{0.017319in}}{\pgfqpoint{0.023570in}{0.023570in}}%
\pgfpathcurveto{\pgfqpoint{0.017319in}{0.029821in}}{\pgfqpoint{0.008840in}{0.033333in}}{\pgfqpoint{0.000000in}{0.033333in}}%
\pgfpathcurveto{\pgfqpoint{-0.008840in}{0.033333in}}{\pgfqpoint{-0.017319in}{0.029821in}}{\pgfqpoint{-0.023570in}{0.023570in}}%
\pgfpathcurveto{\pgfqpoint{-0.029821in}{0.017319in}}{\pgfqpoint{-0.033333in}{0.008840in}}{\pgfqpoint{-0.033333in}{0.000000in}}%
\pgfpathcurveto{\pgfqpoint{-0.033333in}{-0.008840in}}{\pgfqpoint{-0.029821in}{-0.017319in}}{\pgfqpoint{-0.023570in}{-0.023570in}}%
\pgfpathcurveto{\pgfqpoint{-0.017319in}{-0.029821in}}{\pgfqpoint{-0.008840in}{-0.033333in}}{\pgfqpoint{0.000000in}{-0.033333in}}%
\pgfpathlineto{\pgfqpoint{0.000000in}{-0.033333in}}%
\pgfpathclose%
\pgfusepath{stroke,fill}%
}%
\begin{pgfscope}%
\pgfsys@transformshift{2.677164in}{1.772855in}%
\pgfsys@useobject{currentmarker}{}%
\end{pgfscope}%
\end{pgfscope}%
\begin{pgfscope}%
\pgfpathrectangle{\pgfqpoint{0.765000in}{0.660000in}}{\pgfqpoint{4.620000in}{4.620000in}}%
\pgfusepath{clip}%
\pgfsetrectcap%
\pgfsetroundjoin%
\pgfsetlinewidth{1.204500pt}%
\definecolor{currentstroke}{rgb}{1.000000,0.576471,0.309804}%
\pgfsetstrokecolor{currentstroke}%
\pgfsetdash{}{0pt}%
\pgfpathmoveto{\pgfqpoint{2.896400in}{1.968458in}}%
\pgfusepath{stroke}%
\end{pgfscope}%
\begin{pgfscope}%
\pgfpathrectangle{\pgfqpoint{0.765000in}{0.660000in}}{\pgfqpoint{4.620000in}{4.620000in}}%
\pgfusepath{clip}%
\pgfsetbuttcap%
\pgfsetroundjoin%
\definecolor{currentfill}{rgb}{1.000000,0.576471,0.309804}%
\pgfsetfillcolor{currentfill}%
\pgfsetlinewidth{1.003750pt}%
\definecolor{currentstroke}{rgb}{1.000000,0.576471,0.309804}%
\pgfsetstrokecolor{currentstroke}%
\pgfsetdash{}{0pt}%
\pgfsys@defobject{currentmarker}{\pgfqpoint{-0.033333in}{-0.033333in}}{\pgfqpoint{0.033333in}{0.033333in}}{%
\pgfpathmoveto{\pgfqpoint{0.000000in}{-0.033333in}}%
\pgfpathcurveto{\pgfqpoint{0.008840in}{-0.033333in}}{\pgfqpoint{0.017319in}{-0.029821in}}{\pgfqpoint{0.023570in}{-0.023570in}}%
\pgfpathcurveto{\pgfqpoint{0.029821in}{-0.017319in}}{\pgfqpoint{0.033333in}{-0.008840in}}{\pgfqpoint{0.033333in}{0.000000in}}%
\pgfpathcurveto{\pgfqpoint{0.033333in}{0.008840in}}{\pgfqpoint{0.029821in}{0.017319in}}{\pgfqpoint{0.023570in}{0.023570in}}%
\pgfpathcurveto{\pgfqpoint{0.017319in}{0.029821in}}{\pgfqpoint{0.008840in}{0.033333in}}{\pgfqpoint{0.000000in}{0.033333in}}%
\pgfpathcurveto{\pgfqpoint{-0.008840in}{0.033333in}}{\pgfqpoint{-0.017319in}{0.029821in}}{\pgfqpoint{-0.023570in}{0.023570in}}%
\pgfpathcurveto{\pgfqpoint{-0.029821in}{0.017319in}}{\pgfqpoint{-0.033333in}{0.008840in}}{\pgfqpoint{-0.033333in}{0.000000in}}%
\pgfpathcurveto{\pgfqpoint{-0.033333in}{-0.008840in}}{\pgfqpoint{-0.029821in}{-0.017319in}}{\pgfqpoint{-0.023570in}{-0.023570in}}%
\pgfpathcurveto{\pgfqpoint{-0.017319in}{-0.029821in}}{\pgfqpoint{-0.008840in}{-0.033333in}}{\pgfqpoint{0.000000in}{-0.033333in}}%
\pgfpathlineto{\pgfqpoint{0.000000in}{-0.033333in}}%
\pgfpathclose%
\pgfusepath{stroke,fill}%
}%
\begin{pgfscope}%
\pgfsys@transformshift{2.896400in}{1.968458in}%
\pgfsys@useobject{currentmarker}{}%
\end{pgfscope}%
\end{pgfscope}%
\begin{pgfscope}%
\pgfpathrectangle{\pgfqpoint{0.765000in}{0.660000in}}{\pgfqpoint{4.620000in}{4.620000in}}%
\pgfusepath{clip}%
\pgfsetrectcap%
\pgfsetroundjoin%
\pgfsetlinewidth{1.204500pt}%
\definecolor{currentstroke}{rgb}{1.000000,0.576471,0.309804}%
\pgfsetstrokecolor{currentstroke}%
\pgfsetdash{}{0pt}%
\pgfpathmoveto{\pgfqpoint{3.379380in}{2.438263in}}%
\pgfusepath{stroke}%
\end{pgfscope}%
\begin{pgfscope}%
\pgfpathrectangle{\pgfqpoint{0.765000in}{0.660000in}}{\pgfqpoint{4.620000in}{4.620000in}}%
\pgfusepath{clip}%
\pgfsetbuttcap%
\pgfsetroundjoin%
\definecolor{currentfill}{rgb}{1.000000,0.576471,0.309804}%
\pgfsetfillcolor{currentfill}%
\pgfsetlinewidth{1.003750pt}%
\definecolor{currentstroke}{rgb}{1.000000,0.576471,0.309804}%
\pgfsetstrokecolor{currentstroke}%
\pgfsetdash{}{0pt}%
\pgfsys@defobject{currentmarker}{\pgfqpoint{-0.033333in}{-0.033333in}}{\pgfqpoint{0.033333in}{0.033333in}}{%
\pgfpathmoveto{\pgfqpoint{0.000000in}{-0.033333in}}%
\pgfpathcurveto{\pgfqpoint{0.008840in}{-0.033333in}}{\pgfqpoint{0.017319in}{-0.029821in}}{\pgfqpoint{0.023570in}{-0.023570in}}%
\pgfpathcurveto{\pgfqpoint{0.029821in}{-0.017319in}}{\pgfqpoint{0.033333in}{-0.008840in}}{\pgfqpoint{0.033333in}{0.000000in}}%
\pgfpathcurveto{\pgfqpoint{0.033333in}{0.008840in}}{\pgfqpoint{0.029821in}{0.017319in}}{\pgfqpoint{0.023570in}{0.023570in}}%
\pgfpathcurveto{\pgfqpoint{0.017319in}{0.029821in}}{\pgfqpoint{0.008840in}{0.033333in}}{\pgfqpoint{0.000000in}{0.033333in}}%
\pgfpathcurveto{\pgfqpoint{-0.008840in}{0.033333in}}{\pgfqpoint{-0.017319in}{0.029821in}}{\pgfqpoint{-0.023570in}{0.023570in}}%
\pgfpathcurveto{\pgfqpoint{-0.029821in}{0.017319in}}{\pgfqpoint{-0.033333in}{0.008840in}}{\pgfqpoint{-0.033333in}{0.000000in}}%
\pgfpathcurveto{\pgfqpoint{-0.033333in}{-0.008840in}}{\pgfqpoint{-0.029821in}{-0.017319in}}{\pgfqpoint{-0.023570in}{-0.023570in}}%
\pgfpathcurveto{\pgfqpoint{-0.017319in}{-0.029821in}}{\pgfqpoint{-0.008840in}{-0.033333in}}{\pgfqpoint{0.000000in}{-0.033333in}}%
\pgfpathlineto{\pgfqpoint{0.000000in}{-0.033333in}}%
\pgfpathclose%
\pgfusepath{stroke,fill}%
}%
\begin{pgfscope}%
\pgfsys@transformshift{3.379380in}{2.438263in}%
\pgfsys@useobject{currentmarker}{}%
\end{pgfscope}%
\end{pgfscope}%
\begin{pgfscope}%
\pgfpathrectangle{\pgfqpoint{0.765000in}{0.660000in}}{\pgfqpoint{4.620000in}{4.620000in}}%
\pgfusepath{clip}%
\pgfsetrectcap%
\pgfsetroundjoin%
\pgfsetlinewidth{1.204500pt}%
\definecolor{currentstroke}{rgb}{1.000000,0.576471,0.309804}%
\pgfsetstrokecolor{currentstroke}%
\pgfsetdash{}{0pt}%
\pgfpathmoveto{\pgfqpoint{3.079743in}{2.666374in}}%
\pgfusepath{stroke}%
\end{pgfscope}%
\begin{pgfscope}%
\pgfpathrectangle{\pgfqpoint{0.765000in}{0.660000in}}{\pgfqpoint{4.620000in}{4.620000in}}%
\pgfusepath{clip}%
\pgfsetbuttcap%
\pgfsetroundjoin%
\definecolor{currentfill}{rgb}{1.000000,0.576471,0.309804}%
\pgfsetfillcolor{currentfill}%
\pgfsetlinewidth{1.003750pt}%
\definecolor{currentstroke}{rgb}{1.000000,0.576471,0.309804}%
\pgfsetstrokecolor{currentstroke}%
\pgfsetdash{}{0pt}%
\pgfsys@defobject{currentmarker}{\pgfqpoint{-0.033333in}{-0.033333in}}{\pgfqpoint{0.033333in}{0.033333in}}{%
\pgfpathmoveto{\pgfqpoint{0.000000in}{-0.033333in}}%
\pgfpathcurveto{\pgfqpoint{0.008840in}{-0.033333in}}{\pgfqpoint{0.017319in}{-0.029821in}}{\pgfqpoint{0.023570in}{-0.023570in}}%
\pgfpathcurveto{\pgfqpoint{0.029821in}{-0.017319in}}{\pgfqpoint{0.033333in}{-0.008840in}}{\pgfqpoint{0.033333in}{0.000000in}}%
\pgfpathcurveto{\pgfqpoint{0.033333in}{0.008840in}}{\pgfqpoint{0.029821in}{0.017319in}}{\pgfqpoint{0.023570in}{0.023570in}}%
\pgfpathcurveto{\pgfqpoint{0.017319in}{0.029821in}}{\pgfqpoint{0.008840in}{0.033333in}}{\pgfqpoint{0.000000in}{0.033333in}}%
\pgfpathcurveto{\pgfqpoint{-0.008840in}{0.033333in}}{\pgfqpoint{-0.017319in}{0.029821in}}{\pgfqpoint{-0.023570in}{0.023570in}}%
\pgfpathcurveto{\pgfqpoint{-0.029821in}{0.017319in}}{\pgfqpoint{-0.033333in}{0.008840in}}{\pgfqpoint{-0.033333in}{0.000000in}}%
\pgfpathcurveto{\pgfqpoint{-0.033333in}{-0.008840in}}{\pgfqpoint{-0.029821in}{-0.017319in}}{\pgfqpoint{-0.023570in}{-0.023570in}}%
\pgfpathcurveto{\pgfqpoint{-0.017319in}{-0.029821in}}{\pgfqpoint{-0.008840in}{-0.033333in}}{\pgfqpoint{0.000000in}{-0.033333in}}%
\pgfpathlineto{\pgfqpoint{0.000000in}{-0.033333in}}%
\pgfpathclose%
\pgfusepath{stroke,fill}%
}%
\begin{pgfscope}%
\pgfsys@transformshift{3.079743in}{2.666374in}%
\pgfsys@useobject{currentmarker}{}%
\end{pgfscope}%
\end{pgfscope}%
\begin{pgfscope}%
\pgfpathrectangle{\pgfqpoint{0.765000in}{0.660000in}}{\pgfqpoint{4.620000in}{4.620000in}}%
\pgfusepath{clip}%
\pgfsetrectcap%
\pgfsetroundjoin%
\pgfsetlinewidth{1.204500pt}%
\definecolor{currentstroke}{rgb}{1.000000,0.576471,0.309804}%
\pgfsetstrokecolor{currentstroke}%
\pgfsetdash{}{0pt}%
\pgfpathmoveto{\pgfqpoint{4.189416in}{2.150874in}}%
\pgfusepath{stroke}%
\end{pgfscope}%
\begin{pgfscope}%
\pgfpathrectangle{\pgfqpoint{0.765000in}{0.660000in}}{\pgfqpoint{4.620000in}{4.620000in}}%
\pgfusepath{clip}%
\pgfsetbuttcap%
\pgfsetroundjoin%
\definecolor{currentfill}{rgb}{1.000000,0.576471,0.309804}%
\pgfsetfillcolor{currentfill}%
\pgfsetlinewidth{1.003750pt}%
\definecolor{currentstroke}{rgb}{1.000000,0.576471,0.309804}%
\pgfsetstrokecolor{currentstroke}%
\pgfsetdash{}{0pt}%
\pgfsys@defobject{currentmarker}{\pgfqpoint{-0.033333in}{-0.033333in}}{\pgfqpoint{0.033333in}{0.033333in}}{%
\pgfpathmoveto{\pgfqpoint{0.000000in}{-0.033333in}}%
\pgfpathcurveto{\pgfqpoint{0.008840in}{-0.033333in}}{\pgfqpoint{0.017319in}{-0.029821in}}{\pgfqpoint{0.023570in}{-0.023570in}}%
\pgfpathcurveto{\pgfqpoint{0.029821in}{-0.017319in}}{\pgfqpoint{0.033333in}{-0.008840in}}{\pgfqpoint{0.033333in}{0.000000in}}%
\pgfpathcurveto{\pgfqpoint{0.033333in}{0.008840in}}{\pgfqpoint{0.029821in}{0.017319in}}{\pgfqpoint{0.023570in}{0.023570in}}%
\pgfpathcurveto{\pgfqpoint{0.017319in}{0.029821in}}{\pgfqpoint{0.008840in}{0.033333in}}{\pgfqpoint{0.000000in}{0.033333in}}%
\pgfpathcurveto{\pgfqpoint{-0.008840in}{0.033333in}}{\pgfqpoint{-0.017319in}{0.029821in}}{\pgfqpoint{-0.023570in}{0.023570in}}%
\pgfpathcurveto{\pgfqpoint{-0.029821in}{0.017319in}}{\pgfqpoint{-0.033333in}{0.008840in}}{\pgfqpoint{-0.033333in}{0.000000in}}%
\pgfpathcurveto{\pgfqpoint{-0.033333in}{-0.008840in}}{\pgfqpoint{-0.029821in}{-0.017319in}}{\pgfqpoint{-0.023570in}{-0.023570in}}%
\pgfpathcurveto{\pgfqpoint{-0.017319in}{-0.029821in}}{\pgfqpoint{-0.008840in}{-0.033333in}}{\pgfqpoint{0.000000in}{-0.033333in}}%
\pgfpathlineto{\pgfqpoint{0.000000in}{-0.033333in}}%
\pgfpathclose%
\pgfusepath{stroke,fill}%
}%
\begin{pgfscope}%
\pgfsys@transformshift{4.189416in}{2.150874in}%
\pgfsys@useobject{currentmarker}{}%
\end{pgfscope}%
\end{pgfscope}%
\begin{pgfscope}%
\pgfpathrectangle{\pgfqpoint{0.765000in}{0.660000in}}{\pgfqpoint{4.620000in}{4.620000in}}%
\pgfusepath{clip}%
\pgfsetrectcap%
\pgfsetroundjoin%
\pgfsetlinewidth{1.204500pt}%
\definecolor{currentstroke}{rgb}{1.000000,0.576471,0.309804}%
\pgfsetstrokecolor{currentstroke}%
\pgfsetdash{}{0pt}%
\pgfpathmoveto{\pgfqpoint{3.905467in}{3.023702in}}%
\pgfusepath{stroke}%
\end{pgfscope}%
\begin{pgfscope}%
\pgfpathrectangle{\pgfqpoint{0.765000in}{0.660000in}}{\pgfqpoint{4.620000in}{4.620000in}}%
\pgfusepath{clip}%
\pgfsetbuttcap%
\pgfsetroundjoin%
\definecolor{currentfill}{rgb}{1.000000,0.576471,0.309804}%
\pgfsetfillcolor{currentfill}%
\pgfsetlinewidth{1.003750pt}%
\definecolor{currentstroke}{rgb}{1.000000,0.576471,0.309804}%
\pgfsetstrokecolor{currentstroke}%
\pgfsetdash{}{0pt}%
\pgfsys@defobject{currentmarker}{\pgfqpoint{-0.033333in}{-0.033333in}}{\pgfqpoint{0.033333in}{0.033333in}}{%
\pgfpathmoveto{\pgfqpoint{0.000000in}{-0.033333in}}%
\pgfpathcurveto{\pgfqpoint{0.008840in}{-0.033333in}}{\pgfqpoint{0.017319in}{-0.029821in}}{\pgfqpoint{0.023570in}{-0.023570in}}%
\pgfpathcurveto{\pgfqpoint{0.029821in}{-0.017319in}}{\pgfqpoint{0.033333in}{-0.008840in}}{\pgfqpoint{0.033333in}{0.000000in}}%
\pgfpathcurveto{\pgfqpoint{0.033333in}{0.008840in}}{\pgfqpoint{0.029821in}{0.017319in}}{\pgfqpoint{0.023570in}{0.023570in}}%
\pgfpathcurveto{\pgfqpoint{0.017319in}{0.029821in}}{\pgfqpoint{0.008840in}{0.033333in}}{\pgfqpoint{0.000000in}{0.033333in}}%
\pgfpathcurveto{\pgfqpoint{-0.008840in}{0.033333in}}{\pgfqpoint{-0.017319in}{0.029821in}}{\pgfqpoint{-0.023570in}{0.023570in}}%
\pgfpathcurveto{\pgfqpoint{-0.029821in}{0.017319in}}{\pgfqpoint{-0.033333in}{0.008840in}}{\pgfqpoint{-0.033333in}{0.000000in}}%
\pgfpathcurveto{\pgfqpoint{-0.033333in}{-0.008840in}}{\pgfqpoint{-0.029821in}{-0.017319in}}{\pgfqpoint{-0.023570in}{-0.023570in}}%
\pgfpathcurveto{\pgfqpoint{-0.017319in}{-0.029821in}}{\pgfqpoint{-0.008840in}{-0.033333in}}{\pgfqpoint{0.000000in}{-0.033333in}}%
\pgfpathlineto{\pgfqpoint{0.000000in}{-0.033333in}}%
\pgfpathclose%
\pgfusepath{stroke,fill}%
}%
\begin{pgfscope}%
\pgfsys@transformshift{3.905467in}{3.023702in}%
\pgfsys@useobject{currentmarker}{}%
\end{pgfscope}%
\end{pgfscope}%
\begin{pgfscope}%
\pgfpathrectangle{\pgfqpoint{0.765000in}{0.660000in}}{\pgfqpoint{4.620000in}{4.620000in}}%
\pgfusepath{clip}%
\pgfsetrectcap%
\pgfsetroundjoin%
\pgfsetlinewidth{1.204500pt}%
\definecolor{currentstroke}{rgb}{1.000000,0.576471,0.309804}%
\pgfsetstrokecolor{currentstroke}%
\pgfsetdash{}{0pt}%
\pgfpathmoveto{\pgfqpoint{2.202235in}{2.394369in}}%
\pgfusepath{stroke}%
\end{pgfscope}%
\begin{pgfscope}%
\pgfpathrectangle{\pgfqpoint{0.765000in}{0.660000in}}{\pgfqpoint{4.620000in}{4.620000in}}%
\pgfusepath{clip}%
\pgfsetbuttcap%
\pgfsetroundjoin%
\definecolor{currentfill}{rgb}{1.000000,0.576471,0.309804}%
\pgfsetfillcolor{currentfill}%
\pgfsetlinewidth{1.003750pt}%
\definecolor{currentstroke}{rgb}{1.000000,0.576471,0.309804}%
\pgfsetstrokecolor{currentstroke}%
\pgfsetdash{}{0pt}%
\pgfsys@defobject{currentmarker}{\pgfqpoint{-0.033333in}{-0.033333in}}{\pgfqpoint{0.033333in}{0.033333in}}{%
\pgfpathmoveto{\pgfqpoint{0.000000in}{-0.033333in}}%
\pgfpathcurveto{\pgfqpoint{0.008840in}{-0.033333in}}{\pgfqpoint{0.017319in}{-0.029821in}}{\pgfqpoint{0.023570in}{-0.023570in}}%
\pgfpathcurveto{\pgfqpoint{0.029821in}{-0.017319in}}{\pgfqpoint{0.033333in}{-0.008840in}}{\pgfqpoint{0.033333in}{0.000000in}}%
\pgfpathcurveto{\pgfqpoint{0.033333in}{0.008840in}}{\pgfqpoint{0.029821in}{0.017319in}}{\pgfqpoint{0.023570in}{0.023570in}}%
\pgfpathcurveto{\pgfqpoint{0.017319in}{0.029821in}}{\pgfqpoint{0.008840in}{0.033333in}}{\pgfqpoint{0.000000in}{0.033333in}}%
\pgfpathcurveto{\pgfqpoint{-0.008840in}{0.033333in}}{\pgfqpoint{-0.017319in}{0.029821in}}{\pgfqpoint{-0.023570in}{0.023570in}}%
\pgfpathcurveto{\pgfqpoint{-0.029821in}{0.017319in}}{\pgfqpoint{-0.033333in}{0.008840in}}{\pgfqpoint{-0.033333in}{0.000000in}}%
\pgfpathcurveto{\pgfqpoint{-0.033333in}{-0.008840in}}{\pgfqpoint{-0.029821in}{-0.017319in}}{\pgfqpoint{-0.023570in}{-0.023570in}}%
\pgfpathcurveto{\pgfqpoint{-0.017319in}{-0.029821in}}{\pgfqpoint{-0.008840in}{-0.033333in}}{\pgfqpoint{0.000000in}{-0.033333in}}%
\pgfpathlineto{\pgfqpoint{0.000000in}{-0.033333in}}%
\pgfpathclose%
\pgfusepath{stroke,fill}%
}%
\begin{pgfscope}%
\pgfsys@transformshift{2.202235in}{2.394369in}%
\pgfsys@useobject{currentmarker}{}%
\end{pgfscope}%
\end{pgfscope}%
\begin{pgfscope}%
\pgfpathrectangle{\pgfqpoint{0.765000in}{0.660000in}}{\pgfqpoint{4.620000in}{4.620000in}}%
\pgfusepath{clip}%
\pgfsetrectcap%
\pgfsetroundjoin%
\pgfsetlinewidth{1.204500pt}%
\definecolor{currentstroke}{rgb}{1.000000,0.576471,0.309804}%
\pgfsetstrokecolor{currentstroke}%
\pgfsetdash{}{0pt}%
\pgfpathmoveto{\pgfqpoint{2.252169in}{2.395887in}}%
\pgfusepath{stroke}%
\end{pgfscope}%
\begin{pgfscope}%
\pgfpathrectangle{\pgfqpoint{0.765000in}{0.660000in}}{\pgfqpoint{4.620000in}{4.620000in}}%
\pgfusepath{clip}%
\pgfsetbuttcap%
\pgfsetroundjoin%
\definecolor{currentfill}{rgb}{1.000000,0.576471,0.309804}%
\pgfsetfillcolor{currentfill}%
\pgfsetlinewidth{1.003750pt}%
\definecolor{currentstroke}{rgb}{1.000000,0.576471,0.309804}%
\pgfsetstrokecolor{currentstroke}%
\pgfsetdash{}{0pt}%
\pgfsys@defobject{currentmarker}{\pgfqpoint{-0.033333in}{-0.033333in}}{\pgfqpoint{0.033333in}{0.033333in}}{%
\pgfpathmoveto{\pgfqpoint{0.000000in}{-0.033333in}}%
\pgfpathcurveto{\pgfqpoint{0.008840in}{-0.033333in}}{\pgfqpoint{0.017319in}{-0.029821in}}{\pgfqpoint{0.023570in}{-0.023570in}}%
\pgfpathcurveto{\pgfqpoint{0.029821in}{-0.017319in}}{\pgfqpoint{0.033333in}{-0.008840in}}{\pgfqpoint{0.033333in}{0.000000in}}%
\pgfpathcurveto{\pgfqpoint{0.033333in}{0.008840in}}{\pgfqpoint{0.029821in}{0.017319in}}{\pgfqpoint{0.023570in}{0.023570in}}%
\pgfpathcurveto{\pgfqpoint{0.017319in}{0.029821in}}{\pgfqpoint{0.008840in}{0.033333in}}{\pgfqpoint{0.000000in}{0.033333in}}%
\pgfpathcurveto{\pgfqpoint{-0.008840in}{0.033333in}}{\pgfqpoint{-0.017319in}{0.029821in}}{\pgfqpoint{-0.023570in}{0.023570in}}%
\pgfpathcurveto{\pgfqpoint{-0.029821in}{0.017319in}}{\pgfqpoint{-0.033333in}{0.008840in}}{\pgfqpoint{-0.033333in}{0.000000in}}%
\pgfpathcurveto{\pgfqpoint{-0.033333in}{-0.008840in}}{\pgfqpoint{-0.029821in}{-0.017319in}}{\pgfqpoint{-0.023570in}{-0.023570in}}%
\pgfpathcurveto{\pgfqpoint{-0.017319in}{-0.029821in}}{\pgfqpoint{-0.008840in}{-0.033333in}}{\pgfqpoint{0.000000in}{-0.033333in}}%
\pgfpathlineto{\pgfqpoint{0.000000in}{-0.033333in}}%
\pgfpathclose%
\pgfusepath{stroke,fill}%
}%
\begin{pgfscope}%
\pgfsys@transformshift{2.252169in}{2.395887in}%
\pgfsys@useobject{currentmarker}{}%
\end{pgfscope}%
\end{pgfscope}%
\begin{pgfscope}%
\pgfpathrectangle{\pgfqpoint{0.765000in}{0.660000in}}{\pgfqpoint{4.620000in}{4.620000in}}%
\pgfusepath{clip}%
\pgfsetrectcap%
\pgfsetroundjoin%
\pgfsetlinewidth{1.204500pt}%
\definecolor{currentstroke}{rgb}{1.000000,0.576471,0.309804}%
\pgfsetstrokecolor{currentstroke}%
\pgfsetdash{}{0pt}%
\pgfpathmoveto{\pgfqpoint{2.815262in}{2.361363in}}%
\pgfusepath{stroke}%
\end{pgfscope}%
\begin{pgfscope}%
\pgfpathrectangle{\pgfqpoint{0.765000in}{0.660000in}}{\pgfqpoint{4.620000in}{4.620000in}}%
\pgfusepath{clip}%
\pgfsetbuttcap%
\pgfsetroundjoin%
\definecolor{currentfill}{rgb}{1.000000,0.576471,0.309804}%
\pgfsetfillcolor{currentfill}%
\pgfsetlinewidth{1.003750pt}%
\definecolor{currentstroke}{rgb}{1.000000,0.576471,0.309804}%
\pgfsetstrokecolor{currentstroke}%
\pgfsetdash{}{0pt}%
\pgfsys@defobject{currentmarker}{\pgfqpoint{-0.033333in}{-0.033333in}}{\pgfqpoint{0.033333in}{0.033333in}}{%
\pgfpathmoveto{\pgfqpoint{0.000000in}{-0.033333in}}%
\pgfpathcurveto{\pgfqpoint{0.008840in}{-0.033333in}}{\pgfqpoint{0.017319in}{-0.029821in}}{\pgfqpoint{0.023570in}{-0.023570in}}%
\pgfpathcurveto{\pgfqpoint{0.029821in}{-0.017319in}}{\pgfqpoint{0.033333in}{-0.008840in}}{\pgfqpoint{0.033333in}{0.000000in}}%
\pgfpathcurveto{\pgfqpoint{0.033333in}{0.008840in}}{\pgfqpoint{0.029821in}{0.017319in}}{\pgfqpoint{0.023570in}{0.023570in}}%
\pgfpathcurveto{\pgfqpoint{0.017319in}{0.029821in}}{\pgfqpoint{0.008840in}{0.033333in}}{\pgfqpoint{0.000000in}{0.033333in}}%
\pgfpathcurveto{\pgfqpoint{-0.008840in}{0.033333in}}{\pgfqpoint{-0.017319in}{0.029821in}}{\pgfqpoint{-0.023570in}{0.023570in}}%
\pgfpathcurveto{\pgfqpoint{-0.029821in}{0.017319in}}{\pgfqpoint{-0.033333in}{0.008840in}}{\pgfqpoint{-0.033333in}{0.000000in}}%
\pgfpathcurveto{\pgfqpoint{-0.033333in}{-0.008840in}}{\pgfqpoint{-0.029821in}{-0.017319in}}{\pgfqpoint{-0.023570in}{-0.023570in}}%
\pgfpathcurveto{\pgfqpoint{-0.017319in}{-0.029821in}}{\pgfqpoint{-0.008840in}{-0.033333in}}{\pgfqpoint{0.000000in}{-0.033333in}}%
\pgfpathlineto{\pgfqpoint{0.000000in}{-0.033333in}}%
\pgfpathclose%
\pgfusepath{stroke,fill}%
}%
\begin{pgfscope}%
\pgfsys@transformshift{2.815262in}{2.361363in}%
\pgfsys@useobject{currentmarker}{}%
\end{pgfscope}%
\end{pgfscope}%
\begin{pgfscope}%
\pgfpathrectangle{\pgfqpoint{0.765000in}{0.660000in}}{\pgfqpoint{4.620000in}{4.620000in}}%
\pgfusepath{clip}%
\pgfsetrectcap%
\pgfsetroundjoin%
\pgfsetlinewidth{1.204500pt}%
\definecolor{currentstroke}{rgb}{1.000000,0.576471,0.309804}%
\pgfsetstrokecolor{currentstroke}%
\pgfsetdash{}{0pt}%
\pgfpathmoveto{\pgfqpoint{3.823467in}{2.350615in}}%
\pgfusepath{stroke}%
\end{pgfscope}%
\begin{pgfscope}%
\pgfpathrectangle{\pgfqpoint{0.765000in}{0.660000in}}{\pgfqpoint{4.620000in}{4.620000in}}%
\pgfusepath{clip}%
\pgfsetbuttcap%
\pgfsetroundjoin%
\definecolor{currentfill}{rgb}{1.000000,0.576471,0.309804}%
\pgfsetfillcolor{currentfill}%
\pgfsetlinewidth{1.003750pt}%
\definecolor{currentstroke}{rgb}{1.000000,0.576471,0.309804}%
\pgfsetstrokecolor{currentstroke}%
\pgfsetdash{}{0pt}%
\pgfsys@defobject{currentmarker}{\pgfqpoint{-0.033333in}{-0.033333in}}{\pgfqpoint{0.033333in}{0.033333in}}{%
\pgfpathmoveto{\pgfqpoint{0.000000in}{-0.033333in}}%
\pgfpathcurveto{\pgfqpoint{0.008840in}{-0.033333in}}{\pgfqpoint{0.017319in}{-0.029821in}}{\pgfqpoint{0.023570in}{-0.023570in}}%
\pgfpathcurveto{\pgfqpoint{0.029821in}{-0.017319in}}{\pgfqpoint{0.033333in}{-0.008840in}}{\pgfqpoint{0.033333in}{0.000000in}}%
\pgfpathcurveto{\pgfqpoint{0.033333in}{0.008840in}}{\pgfqpoint{0.029821in}{0.017319in}}{\pgfqpoint{0.023570in}{0.023570in}}%
\pgfpathcurveto{\pgfqpoint{0.017319in}{0.029821in}}{\pgfqpoint{0.008840in}{0.033333in}}{\pgfqpoint{0.000000in}{0.033333in}}%
\pgfpathcurveto{\pgfqpoint{-0.008840in}{0.033333in}}{\pgfqpoint{-0.017319in}{0.029821in}}{\pgfqpoint{-0.023570in}{0.023570in}}%
\pgfpathcurveto{\pgfqpoint{-0.029821in}{0.017319in}}{\pgfqpoint{-0.033333in}{0.008840in}}{\pgfqpoint{-0.033333in}{0.000000in}}%
\pgfpathcurveto{\pgfqpoint{-0.033333in}{-0.008840in}}{\pgfqpoint{-0.029821in}{-0.017319in}}{\pgfqpoint{-0.023570in}{-0.023570in}}%
\pgfpathcurveto{\pgfqpoint{-0.017319in}{-0.029821in}}{\pgfqpoint{-0.008840in}{-0.033333in}}{\pgfqpoint{0.000000in}{-0.033333in}}%
\pgfpathlineto{\pgfqpoint{0.000000in}{-0.033333in}}%
\pgfpathclose%
\pgfusepath{stroke,fill}%
}%
\begin{pgfscope}%
\pgfsys@transformshift{3.823467in}{2.350615in}%
\pgfsys@useobject{currentmarker}{}%
\end{pgfscope}%
\end{pgfscope}%
\begin{pgfscope}%
\pgfpathrectangle{\pgfqpoint{0.765000in}{0.660000in}}{\pgfqpoint{4.620000in}{4.620000in}}%
\pgfusepath{clip}%
\pgfsetrectcap%
\pgfsetroundjoin%
\pgfsetlinewidth{1.204500pt}%
\definecolor{currentstroke}{rgb}{1.000000,0.576471,0.309804}%
\pgfsetstrokecolor{currentstroke}%
\pgfsetdash{}{0pt}%
\pgfpathmoveto{\pgfqpoint{4.276767in}{3.227845in}}%
\pgfusepath{stroke}%
\end{pgfscope}%
\begin{pgfscope}%
\pgfpathrectangle{\pgfqpoint{0.765000in}{0.660000in}}{\pgfqpoint{4.620000in}{4.620000in}}%
\pgfusepath{clip}%
\pgfsetbuttcap%
\pgfsetroundjoin%
\definecolor{currentfill}{rgb}{1.000000,0.576471,0.309804}%
\pgfsetfillcolor{currentfill}%
\pgfsetlinewidth{1.003750pt}%
\definecolor{currentstroke}{rgb}{1.000000,0.576471,0.309804}%
\pgfsetstrokecolor{currentstroke}%
\pgfsetdash{}{0pt}%
\pgfsys@defobject{currentmarker}{\pgfqpoint{-0.033333in}{-0.033333in}}{\pgfqpoint{0.033333in}{0.033333in}}{%
\pgfpathmoveto{\pgfqpoint{0.000000in}{-0.033333in}}%
\pgfpathcurveto{\pgfqpoint{0.008840in}{-0.033333in}}{\pgfqpoint{0.017319in}{-0.029821in}}{\pgfqpoint{0.023570in}{-0.023570in}}%
\pgfpathcurveto{\pgfqpoint{0.029821in}{-0.017319in}}{\pgfqpoint{0.033333in}{-0.008840in}}{\pgfqpoint{0.033333in}{0.000000in}}%
\pgfpathcurveto{\pgfqpoint{0.033333in}{0.008840in}}{\pgfqpoint{0.029821in}{0.017319in}}{\pgfqpoint{0.023570in}{0.023570in}}%
\pgfpathcurveto{\pgfqpoint{0.017319in}{0.029821in}}{\pgfqpoint{0.008840in}{0.033333in}}{\pgfqpoint{0.000000in}{0.033333in}}%
\pgfpathcurveto{\pgfqpoint{-0.008840in}{0.033333in}}{\pgfqpoint{-0.017319in}{0.029821in}}{\pgfqpoint{-0.023570in}{0.023570in}}%
\pgfpathcurveto{\pgfqpoint{-0.029821in}{0.017319in}}{\pgfqpoint{-0.033333in}{0.008840in}}{\pgfqpoint{-0.033333in}{0.000000in}}%
\pgfpathcurveto{\pgfqpoint{-0.033333in}{-0.008840in}}{\pgfqpoint{-0.029821in}{-0.017319in}}{\pgfqpoint{-0.023570in}{-0.023570in}}%
\pgfpathcurveto{\pgfqpoint{-0.017319in}{-0.029821in}}{\pgfqpoint{-0.008840in}{-0.033333in}}{\pgfqpoint{0.000000in}{-0.033333in}}%
\pgfpathlineto{\pgfqpoint{0.000000in}{-0.033333in}}%
\pgfpathclose%
\pgfusepath{stroke,fill}%
}%
\begin{pgfscope}%
\pgfsys@transformshift{4.276767in}{3.227845in}%
\pgfsys@useobject{currentmarker}{}%
\end{pgfscope}%
\end{pgfscope}%
\begin{pgfscope}%
\pgfpathrectangle{\pgfqpoint{0.765000in}{0.660000in}}{\pgfqpoint{4.620000in}{4.620000in}}%
\pgfusepath{clip}%
\pgfsetrectcap%
\pgfsetroundjoin%
\pgfsetlinewidth{1.204500pt}%
\definecolor{currentstroke}{rgb}{1.000000,0.576471,0.309804}%
\pgfsetstrokecolor{currentstroke}%
\pgfsetdash{}{0pt}%
\pgfpathmoveto{\pgfqpoint{3.284746in}{1.906090in}}%
\pgfusepath{stroke}%
\end{pgfscope}%
\begin{pgfscope}%
\pgfpathrectangle{\pgfqpoint{0.765000in}{0.660000in}}{\pgfqpoint{4.620000in}{4.620000in}}%
\pgfusepath{clip}%
\pgfsetbuttcap%
\pgfsetroundjoin%
\definecolor{currentfill}{rgb}{1.000000,0.576471,0.309804}%
\pgfsetfillcolor{currentfill}%
\pgfsetlinewidth{1.003750pt}%
\definecolor{currentstroke}{rgb}{1.000000,0.576471,0.309804}%
\pgfsetstrokecolor{currentstroke}%
\pgfsetdash{}{0pt}%
\pgfsys@defobject{currentmarker}{\pgfqpoint{-0.033333in}{-0.033333in}}{\pgfqpoint{0.033333in}{0.033333in}}{%
\pgfpathmoveto{\pgfqpoint{0.000000in}{-0.033333in}}%
\pgfpathcurveto{\pgfqpoint{0.008840in}{-0.033333in}}{\pgfqpoint{0.017319in}{-0.029821in}}{\pgfqpoint{0.023570in}{-0.023570in}}%
\pgfpathcurveto{\pgfqpoint{0.029821in}{-0.017319in}}{\pgfqpoint{0.033333in}{-0.008840in}}{\pgfqpoint{0.033333in}{0.000000in}}%
\pgfpathcurveto{\pgfqpoint{0.033333in}{0.008840in}}{\pgfqpoint{0.029821in}{0.017319in}}{\pgfqpoint{0.023570in}{0.023570in}}%
\pgfpathcurveto{\pgfqpoint{0.017319in}{0.029821in}}{\pgfqpoint{0.008840in}{0.033333in}}{\pgfqpoint{0.000000in}{0.033333in}}%
\pgfpathcurveto{\pgfqpoint{-0.008840in}{0.033333in}}{\pgfqpoint{-0.017319in}{0.029821in}}{\pgfqpoint{-0.023570in}{0.023570in}}%
\pgfpathcurveto{\pgfqpoint{-0.029821in}{0.017319in}}{\pgfqpoint{-0.033333in}{0.008840in}}{\pgfqpoint{-0.033333in}{0.000000in}}%
\pgfpathcurveto{\pgfqpoint{-0.033333in}{-0.008840in}}{\pgfqpoint{-0.029821in}{-0.017319in}}{\pgfqpoint{-0.023570in}{-0.023570in}}%
\pgfpathcurveto{\pgfqpoint{-0.017319in}{-0.029821in}}{\pgfqpoint{-0.008840in}{-0.033333in}}{\pgfqpoint{0.000000in}{-0.033333in}}%
\pgfpathlineto{\pgfqpoint{0.000000in}{-0.033333in}}%
\pgfpathclose%
\pgfusepath{stroke,fill}%
}%
\begin{pgfscope}%
\pgfsys@transformshift{3.284746in}{1.906090in}%
\pgfsys@useobject{currentmarker}{}%
\end{pgfscope}%
\end{pgfscope}%
\begin{pgfscope}%
\pgfpathrectangle{\pgfqpoint{0.765000in}{0.660000in}}{\pgfqpoint{4.620000in}{4.620000in}}%
\pgfusepath{clip}%
\pgfsetrectcap%
\pgfsetroundjoin%
\pgfsetlinewidth{1.204500pt}%
\definecolor{currentstroke}{rgb}{1.000000,0.576471,0.309804}%
\pgfsetstrokecolor{currentstroke}%
\pgfsetdash{}{0pt}%
\pgfpathmoveto{\pgfqpoint{3.061689in}{1.317410in}}%
\pgfusepath{stroke}%
\end{pgfscope}%
\begin{pgfscope}%
\pgfpathrectangle{\pgfqpoint{0.765000in}{0.660000in}}{\pgfqpoint{4.620000in}{4.620000in}}%
\pgfusepath{clip}%
\pgfsetbuttcap%
\pgfsetroundjoin%
\definecolor{currentfill}{rgb}{1.000000,0.576471,0.309804}%
\pgfsetfillcolor{currentfill}%
\pgfsetlinewidth{1.003750pt}%
\definecolor{currentstroke}{rgb}{1.000000,0.576471,0.309804}%
\pgfsetstrokecolor{currentstroke}%
\pgfsetdash{}{0pt}%
\pgfsys@defobject{currentmarker}{\pgfqpoint{-0.033333in}{-0.033333in}}{\pgfqpoint{0.033333in}{0.033333in}}{%
\pgfpathmoveto{\pgfqpoint{0.000000in}{-0.033333in}}%
\pgfpathcurveto{\pgfqpoint{0.008840in}{-0.033333in}}{\pgfqpoint{0.017319in}{-0.029821in}}{\pgfqpoint{0.023570in}{-0.023570in}}%
\pgfpathcurveto{\pgfqpoint{0.029821in}{-0.017319in}}{\pgfqpoint{0.033333in}{-0.008840in}}{\pgfqpoint{0.033333in}{0.000000in}}%
\pgfpathcurveto{\pgfqpoint{0.033333in}{0.008840in}}{\pgfqpoint{0.029821in}{0.017319in}}{\pgfqpoint{0.023570in}{0.023570in}}%
\pgfpathcurveto{\pgfqpoint{0.017319in}{0.029821in}}{\pgfqpoint{0.008840in}{0.033333in}}{\pgfqpoint{0.000000in}{0.033333in}}%
\pgfpathcurveto{\pgfqpoint{-0.008840in}{0.033333in}}{\pgfqpoint{-0.017319in}{0.029821in}}{\pgfqpoint{-0.023570in}{0.023570in}}%
\pgfpathcurveto{\pgfqpoint{-0.029821in}{0.017319in}}{\pgfqpoint{-0.033333in}{0.008840in}}{\pgfqpoint{-0.033333in}{0.000000in}}%
\pgfpathcurveto{\pgfqpoint{-0.033333in}{-0.008840in}}{\pgfqpoint{-0.029821in}{-0.017319in}}{\pgfqpoint{-0.023570in}{-0.023570in}}%
\pgfpathcurveto{\pgfqpoint{-0.017319in}{-0.029821in}}{\pgfqpoint{-0.008840in}{-0.033333in}}{\pgfqpoint{0.000000in}{-0.033333in}}%
\pgfpathlineto{\pgfqpoint{0.000000in}{-0.033333in}}%
\pgfpathclose%
\pgfusepath{stroke,fill}%
}%
\begin{pgfscope}%
\pgfsys@transformshift{3.061689in}{1.317410in}%
\pgfsys@useobject{currentmarker}{}%
\end{pgfscope}%
\end{pgfscope}%
\begin{pgfscope}%
\pgfpathrectangle{\pgfqpoint{0.765000in}{0.660000in}}{\pgfqpoint{4.620000in}{4.620000in}}%
\pgfusepath{clip}%
\pgfsetrectcap%
\pgfsetroundjoin%
\pgfsetlinewidth{1.204500pt}%
\definecolor{currentstroke}{rgb}{1.000000,0.576471,0.309804}%
\pgfsetstrokecolor{currentstroke}%
\pgfsetdash{}{0pt}%
\pgfpathmoveto{\pgfqpoint{2.528884in}{2.951240in}}%
\pgfusepath{stroke}%
\end{pgfscope}%
\begin{pgfscope}%
\pgfpathrectangle{\pgfqpoint{0.765000in}{0.660000in}}{\pgfqpoint{4.620000in}{4.620000in}}%
\pgfusepath{clip}%
\pgfsetbuttcap%
\pgfsetroundjoin%
\definecolor{currentfill}{rgb}{1.000000,0.576471,0.309804}%
\pgfsetfillcolor{currentfill}%
\pgfsetlinewidth{1.003750pt}%
\definecolor{currentstroke}{rgb}{1.000000,0.576471,0.309804}%
\pgfsetstrokecolor{currentstroke}%
\pgfsetdash{}{0pt}%
\pgfsys@defobject{currentmarker}{\pgfqpoint{-0.033333in}{-0.033333in}}{\pgfqpoint{0.033333in}{0.033333in}}{%
\pgfpathmoveto{\pgfqpoint{0.000000in}{-0.033333in}}%
\pgfpathcurveto{\pgfqpoint{0.008840in}{-0.033333in}}{\pgfqpoint{0.017319in}{-0.029821in}}{\pgfqpoint{0.023570in}{-0.023570in}}%
\pgfpathcurveto{\pgfqpoint{0.029821in}{-0.017319in}}{\pgfqpoint{0.033333in}{-0.008840in}}{\pgfqpoint{0.033333in}{0.000000in}}%
\pgfpathcurveto{\pgfqpoint{0.033333in}{0.008840in}}{\pgfqpoint{0.029821in}{0.017319in}}{\pgfqpoint{0.023570in}{0.023570in}}%
\pgfpathcurveto{\pgfqpoint{0.017319in}{0.029821in}}{\pgfqpoint{0.008840in}{0.033333in}}{\pgfqpoint{0.000000in}{0.033333in}}%
\pgfpathcurveto{\pgfqpoint{-0.008840in}{0.033333in}}{\pgfqpoint{-0.017319in}{0.029821in}}{\pgfqpoint{-0.023570in}{0.023570in}}%
\pgfpathcurveto{\pgfqpoint{-0.029821in}{0.017319in}}{\pgfqpoint{-0.033333in}{0.008840in}}{\pgfqpoint{-0.033333in}{0.000000in}}%
\pgfpathcurveto{\pgfqpoint{-0.033333in}{-0.008840in}}{\pgfqpoint{-0.029821in}{-0.017319in}}{\pgfqpoint{-0.023570in}{-0.023570in}}%
\pgfpathcurveto{\pgfqpoint{-0.017319in}{-0.029821in}}{\pgfqpoint{-0.008840in}{-0.033333in}}{\pgfqpoint{0.000000in}{-0.033333in}}%
\pgfpathlineto{\pgfqpoint{0.000000in}{-0.033333in}}%
\pgfpathclose%
\pgfusepath{stroke,fill}%
}%
\begin{pgfscope}%
\pgfsys@transformshift{2.528884in}{2.951240in}%
\pgfsys@useobject{currentmarker}{}%
\end{pgfscope}%
\end{pgfscope}%
\begin{pgfscope}%
\pgfpathrectangle{\pgfqpoint{0.765000in}{0.660000in}}{\pgfqpoint{4.620000in}{4.620000in}}%
\pgfusepath{clip}%
\pgfsetrectcap%
\pgfsetroundjoin%
\pgfsetlinewidth{1.204500pt}%
\definecolor{currentstroke}{rgb}{1.000000,0.576471,0.309804}%
\pgfsetstrokecolor{currentstroke}%
\pgfsetdash{}{0pt}%
\pgfpathmoveto{\pgfqpoint{3.093301in}{1.387926in}}%
\pgfusepath{stroke}%
\end{pgfscope}%
\begin{pgfscope}%
\pgfpathrectangle{\pgfqpoint{0.765000in}{0.660000in}}{\pgfqpoint{4.620000in}{4.620000in}}%
\pgfusepath{clip}%
\pgfsetbuttcap%
\pgfsetroundjoin%
\definecolor{currentfill}{rgb}{1.000000,0.576471,0.309804}%
\pgfsetfillcolor{currentfill}%
\pgfsetlinewidth{1.003750pt}%
\definecolor{currentstroke}{rgb}{1.000000,0.576471,0.309804}%
\pgfsetstrokecolor{currentstroke}%
\pgfsetdash{}{0pt}%
\pgfsys@defobject{currentmarker}{\pgfqpoint{-0.033333in}{-0.033333in}}{\pgfqpoint{0.033333in}{0.033333in}}{%
\pgfpathmoveto{\pgfqpoint{0.000000in}{-0.033333in}}%
\pgfpathcurveto{\pgfqpoint{0.008840in}{-0.033333in}}{\pgfqpoint{0.017319in}{-0.029821in}}{\pgfqpoint{0.023570in}{-0.023570in}}%
\pgfpathcurveto{\pgfqpoint{0.029821in}{-0.017319in}}{\pgfqpoint{0.033333in}{-0.008840in}}{\pgfqpoint{0.033333in}{0.000000in}}%
\pgfpathcurveto{\pgfqpoint{0.033333in}{0.008840in}}{\pgfqpoint{0.029821in}{0.017319in}}{\pgfqpoint{0.023570in}{0.023570in}}%
\pgfpathcurveto{\pgfqpoint{0.017319in}{0.029821in}}{\pgfqpoint{0.008840in}{0.033333in}}{\pgfqpoint{0.000000in}{0.033333in}}%
\pgfpathcurveto{\pgfqpoint{-0.008840in}{0.033333in}}{\pgfqpoint{-0.017319in}{0.029821in}}{\pgfqpoint{-0.023570in}{0.023570in}}%
\pgfpathcurveto{\pgfqpoint{-0.029821in}{0.017319in}}{\pgfqpoint{-0.033333in}{0.008840in}}{\pgfqpoint{-0.033333in}{0.000000in}}%
\pgfpathcurveto{\pgfqpoint{-0.033333in}{-0.008840in}}{\pgfqpoint{-0.029821in}{-0.017319in}}{\pgfqpoint{-0.023570in}{-0.023570in}}%
\pgfpathcurveto{\pgfqpoint{-0.017319in}{-0.029821in}}{\pgfqpoint{-0.008840in}{-0.033333in}}{\pgfqpoint{0.000000in}{-0.033333in}}%
\pgfpathlineto{\pgfqpoint{0.000000in}{-0.033333in}}%
\pgfpathclose%
\pgfusepath{stroke,fill}%
}%
\begin{pgfscope}%
\pgfsys@transformshift{3.093301in}{1.387926in}%
\pgfsys@useobject{currentmarker}{}%
\end{pgfscope}%
\end{pgfscope}%
\begin{pgfscope}%
\pgfpathrectangle{\pgfqpoint{0.765000in}{0.660000in}}{\pgfqpoint{4.620000in}{4.620000in}}%
\pgfusepath{clip}%
\pgfsetrectcap%
\pgfsetroundjoin%
\pgfsetlinewidth{1.204500pt}%
\definecolor{currentstroke}{rgb}{1.000000,0.576471,0.309804}%
\pgfsetstrokecolor{currentstroke}%
\pgfsetdash{}{0pt}%
\pgfpathmoveto{\pgfqpoint{2.231153in}{2.432273in}}%
\pgfusepath{stroke}%
\end{pgfscope}%
\begin{pgfscope}%
\pgfpathrectangle{\pgfqpoint{0.765000in}{0.660000in}}{\pgfqpoint{4.620000in}{4.620000in}}%
\pgfusepath{clip}%
\pgfsetbuttcap%
\pgfsetroundjoin%
\definecolor{currentfill}{rgb}{1.000000,0.576471,0.309804}%
\pgfsetfillcolor{currentfill}%
\pgfsetlinewidth{1.003750pt}%
\definecolor{currentstroke}{rgb}{1.000000,0.576471,0.309804}%
\pgfsetstrokecolor{currentstroke}%
\pgfsetdash{}{0pt}%
\pgfsys@defobject{currentmarker}{\pgfqpoint{-0.033333in}{-0.033333in}}{\pgfqpoint{0.033333in}{0.033333in}}{%
\pgfpathmoveto{\pgfqpoint{0.000000in}{-0.033333in}}%
\pgfpathcurveto{\pgfqpoint{0.008840in}{-0.033333in}}{\pgfqpoint{0.017319in}{-0.029821in}}{\pgfqpoint{0.023570in}{-0.023570in}}%
\pgfpathcurveto{\pgfqpoint{0.029821in}{-0.017319in}}{\pgfqpoint{0.033333in}{-0.008840in}}{\pgfqpoint{0.033333in}{0.000000in}}%
\pgfpathcurveto{\pgfqpoint{0.033333in}{0.008840in}}{\pgfqpoint{0.029821in}{0.017319in}}{\pgfqpoint{0.023570in}{0.023570in}}%
\pgfpathcurveto{\pgfqpoint{0.017319in}{0.029821in}}{\pgfqpoint{0.008840in}{0.033333in}}{\pgfqpoint{0.000000in}{0.033333in}}%
\pgfpathcurveto{\pgfqpoint{-0.008840in}{0.033333in}}{\pgfqpoint{-0.017319in}{0.029821in}}{\pgfqpoint{-0.023570in}{0.023570in}}%
\pgfpathcurveto{\pgfqpoint{-0.029821in}{0.017319in}}{\pgfqpoint{-0.033333in}{0.008840in}}{\pgfqpoint{-0.033333in}{0.000000in}}%
\pgfpathcurveto{\pgfqpoint{-0.033333in}{-0.008840in}}{\pgfqpoint{-0.029821in}{-0.017319in}}{\pgfqpoint{-0.023570in}{-0.023570in}}%
\pgfpathcurveto{\pgfqpoint{-0.017319in}{-0.029821in}}{\pgfqpoint{-0.008840in}{-0.033333in}}{\pgfqpoint{0.000000in}{-0.033333in}}%
\pgfpathlineto{\pgfqpoint{0.000000in}{-0.033333in}}%
\pgfpathclose%
\pgfusepath{stroke,fill}%
}%
\begin{pgfscope}%
\pgfsys@transformshift{2.231153in}{2.432273in}%
\pgfsys@useobject{currentmarker}{}%
\end{pgfscope}%
\end{pgfscope}%
\begin{pgfscope}%
\pgfpathrectangle{\pgfqpoint{0.765000in}{0.660000in}}{\pgfqpoint{4.620000in}{4.620000in}}%
\pgfusepath{clip}%
\pgfsetrectcap%
\pgfsetroundjoin%
\pgfsetlinewidth{1.204500pt}%
\definecolor{currentstroke}{rgb}{1.000000,0.576471,0.309804}%
\pgfsetstrokecolor{currentstroke}%
\pgfsetdash{}{0pt}%
\pgfpathmoveto{\pgfqpoint{3.689688in}{2.709192in}}%
\pgfusepath{stroke}%
\end{pgfscope}%
\begin{pgfscope}%
\pgfpathrectangle{\pgfqpoint{0.765000in}{0.660000in}}{\pgfqpoint{4.620000in}{4.620000in}}%
\pgfusepath{clip}%
\pgfsetbuttcap%
\pgfsetroundjoin%
\definecolor{currentfill}{rgb}{1.000000,0.576471,0.309804}%
\pgfsetfillcolor{currentfill}%
\pgfsetlinewidth{1.003750pt}%
\definecolor{currentstroke}{rgb}{1.000000,0.576471,0.309804}%
\pgfsetstrokecolor{currentstroke}%
\pgfsetdash{}{0pt}%
\pgfsys@defobject{currentmarker}{\pgfqpoint{-0.033333in}{-0.033333in}}{\pgfqpoint{0.033333in}{0.033333in}}{%
\pgfpathmoveto{\pgfqpoint{0.000000in}{-0.033333in}}%
\pgfpathcurveto{\pgfqpoint{0.008840in}{-0.033333in}}{\pgfqpoint{0.017319in}{-0.029821in}}{\pgfqpoint{0.023570in}{-0.023570in}}%
\pgfpathcurveto{\pgfqpoint{0.029821in}{-0.017319in}}{\pgfqpoint{0.033333in}{-0.008840in}}{\pgfqpoint{0.033333in}{0.000000in}}%
\pgfpathcurveto{\pgfqpoint{0.033333in}{0.008840in}}{\pgfqpoint{0.029821in}{0.017319in}}{\pgfqpoint{0.023570in}{0.023570in}}%
\pgfpathcurveto{\pgfqpoint{0.017319in}{0.029821in}}{\pgfqpoint{0.008840in}{0.033333in}}{\pgfqpoint{0.000000in}{0.033333in}}%
\pgfpathcurveto{\pgfqpoint{-0.008840in}{0.033333in}}{\pgfqpoint{-0.017319in}{0.029821in}}{\pgfqpoint{-0.023570in}{0.023570in}}%
\pgfpathcurveto{\pgfqpoint{-0.029821in}{0.017319in}}{\pgfqpoint{-0.033333in}{0.008840in}}{\pgfqpoint{-0.033333in}{0.000000in}}%
\pgfpathcurveto{\pgfqpoint{-0.033333in}{-0.008840in}}{\pgfqpoint{-0.029821in}{-0.017319in}}{\pgfqpoint{-0.023570in}{-0.023570in}}%
\pgfpathcurveto{\pgfqpoint{-0.017319in}{-0.029821in}}{\pgfqpoint{-0.008840in}{-0.033333in}}{\pgfqpoint{0.000000in}{-0.033333in}}%
\pgfpathlineto{\pgfqpoint{0.000000in}{-0.033333in}}%
\pgfpathclose%
\pgfusepath{stroke,fill}%
}%
\begin{pgfscope}%
\pgfsys@transformshift{3.689688in}{2.709192in}%
\pgfsys@useobject{currentmarker}{}%
\end{pgfscope}%
\end{pgfscope}%
\begin{pgfscope}%
\pgfpathrectangle{\pgfqpoint{0.765000in}{0.660000in}}{\pgfqpoint{4.620000in}{4.620000in}}%
\pgfusepath{clip}%
\pgfsetrectcap%
\pgfsetroundjoin%
\pgfsetlinewidth{1.204500pt}%
\definecolor{currentstroke}{rgb}{1.000000,0.576471,0.309804}%
\pgfsetstrokecolor{currentstroke}%
\pgfsetdash{}{0pt}%
\pgfpathmoveto{\pgfqpoint{1.689827in}{2.316583in}}%
\pgfusepath{stroke}%
\end{pgfscope}%
\begin{pgfscope}%
\pgfpathrectangle{\pgfqpoint{0.765000in}{0.660000in}}{\pgfqpoint{4.620000in}{4.620000in}}%
\pgfusepath{clip}%
\pgfsetbuttcap%
\pgfsetroundjoin%
\definecolor{currentfill}{rgb}{1.000000,0.576471,0.309804}%
\pgfsetfillcolor{currentfill}%
\pgfsetlinewidth{1.003750pt}%
\definecolor{currentstroke}{rgb}{1.000000,0.576471,0.309804}%
\pgfsetstrokecolor{currentstroke}%
\pgfsetdash{}{0pt}%
\pgfsys@defobject{currentmarker}{\pgfqpoint{-0.033333in}{-0.033333in}}{\pgfqpoint{0.033333in}{0.033333in}}{%
\pgfpathmoveto{\pgfqpoint{0.000000in}{-0.033333in}}%
\pgfpathcurveto{\pgfqpoint{0.008840in}{-0.033333in}}{\pgfqpoint{0.017319in}{-0.029821in}}{\pgfqpoint{0.023570in}{-0.023570in}}%
\pgfpathcurveto{\pgfqpoint{0.029821in}{-0.017319in}}{\pgfqpoint{0.033333in}{-0.008840in}}{\pgfqpoint{0.033333in}{0.000000in}}%
\pgfpathcurveto{\pgfqpoint{0.033333in}{0.008840in}}{\pgfqpoint{0.029821in}{0.017319in}}{\pgfqpoint{0.023570in}{0.023570in}}%
\pgfpathcurveto{\pgfqpoint{0.017319in}{0.029821in}}{\pgfqpoint{0.008840in}{0.033333in}}{\pgfqpoint{0.000000in}{0.033333in}}%
\pgfpathcurveto{\pgfqpoint{-0.008840in}{0.033333in}}{\pgfqpoint{-0.017319in}{0.029821in}}{\pgfqpoint{-0.023570in}{0.023570in}}%
\pgfpathcurveto{\pgfqpoint{-0.029821in}{0.017319in}}{\pgfqpoint{-0.033333in}{0.008840in}}{\pgfqpoint{-0.033333in}{0.000000in}}%
\pgfpathcurveto{\pgfqpoint{-0.033333in}{-0.008840in}}{\pgfqpoint{-0.029821in}{-0.017319in}}{\pgfqpoint{-0.023570in}{-0.023570in}}%
\pgfpathcurveto{\pgfqpoint{-0.017319in}{-0.029821in}}{\pgfqpoint{-0.008840in}{-0.033333in}}{\pgfqpoint{0.000000in}{-0.033333in}}%
\pgfpathlineto{\pgfqpoint{0.000000in}{-0.033333in}}%
\pgfpathclose%
\pgfusepath{stroke,fill}%
}%
\begin{pgfscope}%
\pgfsys@transformshift{1.689827in}{2.316583in}%
\pgfsys@useobject{currentmarker}{}%
\end{pgfscope}%
\end{pgfscope}%
\begin{pgfscope}%
\pgfpathrectangle{\pgfqpoint{0.765000in}{0.660000in}}{\pgfqpoint{4.620000in}{4.620000in}}%
\pgfusepath{clip}%
\pgfsetrectcap%
\pgfsetroundjoin%
\pgfsetlinewidth{1.204500pt}%
\definecolor{currentstroke}{rgb}{1.000000,0.576471,0.309804}%
\pgfsetstrokecolor{currentstroke}%
\pgfsetdash{}{0pt}%
\pgfpathmoveto{\pgfqpoint{3.009161in}{2.511693in}}%
\pgfusepath{stroke}%
\end{pgfscope}%
\begin{pgfscope}%
\pgfpathrectangle{\pgfqpoint{0.765000in}{0.660000in}}{\pgfqpoint{4.620000in}{4.620000in}}%
\pgfusepath{clip}%
\pgfsetbuttcap%
\pgfsetroundjoin%
\definecolor{currentfill}{rgb}{1.000000,0.576471,0.309804}%
\pgfsetfillcolor{currentfill}%
\pgfsetlinewidth{1.003750pt}%
\definecolor{currentstroke}{rgb}{1.000000,0.576471,0.309804}%
\pgfsetstrokecolor{currentstroke}%
\pgfsetdash{}{0pt}%
\pgfsys@defobject{currentmarker}{\pgfqpoint{-0.033333in}{-0.033333in}}{\pgfqpoint{0.033333in}{0.033333in}}{%
\pgfpathmoveto{\pgfqpoint{0.000000in}{-0.033333in}}%
\pgfpathcurveto{\pgfqpoint{0.008840in}{-0.033333in}}{\pgfqpoint{0.017319in}{-0.029821in}}{\pgfqpoint{0.023570in}{-0.023570in}}%
\pgfpathcurveto{\pgfqpoint{0.029821in}{-0.017319in}}{\pgfqpoint{0.033333in}{-0.008840in}}{\pgfqpoint{0.033333in}{0.000000in}}%
\pgfpathcurveto{\pgfqpoint{0.033333in}{0.008840in}}{\pgfqpoint{0.029821in}{0.017319in}}{\pgfqpoint{0.023570in}{0.023570in}}%
\pgfpathcurveto{\pgfqpoint{0.017319in}{0.029821in}}{\pgfqpoint{0.008840in}{0.033333in}}{\pgfqpoint{0.000000in}{0.033333in}}%
\pgfpathcurveto{\pgfqpoint{-0.008840in}{0.033333in}}{\pgfqpoint{-0.017319in}{0.029821in}}{\pgfqpoint{-0.023570in}{0.023570in}}%
\pgfpathcurveto{\pgfqpoint{-0.029821in}{0.017319in}}{\pgfqpoint{-0.033333in}{0.008840in}}{\pgfqpoint{-0.033333in}{0.000000in}}%
\pgfpathcurveto{\pgfqpoint{-0.033333in}{-0.008840in}}{\pgfqpoint{-0.029821in}{-0.017319in}}{\pgfqpoint{-0.023570in}{-0.023570in}}%
\pgfpathcurveto{\pgfqpoint{-0.017319in}{-0.029821in}}{\pgfqpoint{-0.008840in}{-0.033333in}}{\pgfqpoint{0.000000in}{-0.033333in}}%
\pgfpathlineto{\pgfqpoint{0.000000in}{-0.033333in}}%
\pgfpathclose%
\pgfusepath{stroke,fill}%
}%
\begin{pgfscope}%
\pgfsys@transformshift{3.009161in}{2.511693in}%
\pgfsys@useobject{currentmarker}{}%
\end{pgfscope}%
\end{pgfscope}%
\begin{pgfscope}%
\pgfpathrectangle{\pgfqpoint{0.765000in}{0.660000in}}{\pgfqpoint{4.620000in}{4.620000in}}%
\pgfusepath{clip}%
\pgfsetrectcap%
\pgfsetroundjoin%
\pgfsetlinewidth{1.204500pt}%
\definecolor{currentstroke}{rgb}{1.000000,0.576471,0.309804}%
\pgfsetstrokecolor{currentstroke}%
\pgfsetdash{}{0pt}%
\pgfpathmoveto{\pgfqpoint{1.869396in}{3.219216in}}%
\pgfusepath{stroke}%
\end{pgfscope}%
\begin{pgfscope}%
\pgfpathrectangle{\pgfqpoint{0.765000in}{0.660000in}}{\pgfqpoint{4.620000in}{4.620000in}}%
\pgfusepath{clip}%
\pgfsetbuttcap%
\pgfsetroundjoin%
\definecolor{currentfill}{rgb}{1.000000,0.576471,0.309804}%
\pgfsetfillcolor{currentfill}%
\pgfsetlinewidth{1.003750pt}%
\definecolor{currentstroke}{rgb}{1.000000,0.576471,0.309804}%
\pgfsetstrokecolor{currentstroke}%
\pgfsetdash{}{0pt}%
\pgfsys@defobject{currentmarker}{\pgfqpoint{-0.033333in}{-0.033333in}}{\pgfqpoint{0.033333in}{0.033333in}}{%
\pgfpathmoveto{\pgfqpoint{0.000000in}{-0.033333in}}%
\pgfpathcurveto{\pgfqpoint{0.008840in}{-0.033333in}}{\pgfqpoint{0.017319in}{-0.029821in}}{\pgfqpoint{0.023570in}{-0.023570in}}%
\pgfpathcurveto{\pgfqpoint{0.029821in}{-0.017319in}}{\pgfqpoint{0.033333in}{-0.008840in}}{\pgfqpoint{0.033333in}{0.000000in}}%
\pgfpathcurveto{\pgfqpoint{0.033333in}{0.008840in}}{\pgfqpoint{0.029821in}{0.017319in}}{\pgfqpoint{0.023570in}{0.023570in}}%
\pgfpathcurveto{\pgfqpoint{0.017319in}{0.029821in}}{\pgfqpoint{0.008840in}{0.033333in}}{\pgfqpoint{0.000000in}{0.033333in}}%
\pgfpathcurveto{\pgfqpoint{-0.008840in}{0.033333in}}{\pgfqpoint{-0.017319in}{0.029821in}}{\pgfqpoint{-0.023570in}{0.023570in}}%
\pgfpathcurveto{\pgfqpoint{-0.029821in}{0.017319in}}{\pgfqpoint{-0.033333in}{0.008840in}}{\pgfqpoint{-0.033333in}{0.000000in}}%
\pgfpathcurveto{\pgfqpoint{-0.033333in}{-0.008840in}}{\pgfqpoint{-0.029821in}{-0.017319in}}{\pgfqpoint{-0.023570in}{-0.023570in}}%
\pgfpathcurveto{\pgfqpoint{-0.017319in}{-0.029821in}}{\pgfqpoint{-0.008840in}{-0.033333in}}{\pgfqpoint{0.000000in}{-0.033333in}}%
\pgfpathlineto{\pgfqpoint{0.000000in}{-0.033333in}}%
\pgfpathclose%
\pgfusepath{stroke,fill}%
}%
\begin{pgfscope}%
\pgfsys@transformshift{1.869396in}{3.219216in}%
\pgfsys@useobject{currentmarker}{}%
\end{pgfscope}%
\end{pgfscope}%
\begin{pgfscope}%
\pgfpathrectangle{\pgfqpoint{0.765000in}{0.660000in}}{\pgfqpoint{4.620000in}{4.620000in}}%
\pgfusepath{clip}%
\pgfsetrectcap%
\pgfsetroundjoin%
\pgfsetlinewidth{1.204500pt}%
\definecolor{currentstroke}{rgb}{1.000000,0.576471,0.309804}%
\pgfsetstrokecolor{currentstroke}%
\pgfsetdash{}{0pt}%
\pgfpathmoveto{\pgfqpoint{1.590974in}{2.419722in}}%
\pgfusepath{stroke}%
\end{pgfscope}%
\begin{pgfscope}%
\pgfpathrectangle{\pgfqpoint{0.765000in}{0.660000in}}{\pgfqpoint{4.620000in}{4.620000in}}%
\pgfusepath{clip}%
\pgfsetbuttcap%
\pgfsetroundjoin%
\definecolor{currentfill}{rgb}{1.000000,0.576471,0.309804}%
\pgfsetfillcolor{currentfill}%
\pgfsetlinewidth{1.003750pt}%
\definecolor{currentstroke}{rgb}{1.000000,0.576471,0.309804}%
\pgfsetstrokecolor{currentstroke}%
\pgfsetdash{}{0pt}%
\pgfsys@defobject{currentmarker}{\pgfqpoint{-0.033333in}{-0.033333in}}{\pgfqpoint{0.033333in}{0.033333in}}{%
\pgfpathmoveto{\pgfqpoint{0.000000in}{-0.033333in}}%
\pgfpathcurveto{\pgfqpoint{0.008840in}{-0.033333in}}{\pgfqpoint{0.017319in}{-0.029821in}}{\pgfqpoint{0.023570in}{-0.023570in}}%
\pgfpathcurveto{\pgfqpoint{0.029821in}{-0.017319in}}{\pgfqpoint{0.033333in}{-0.008840in}}{\pgfqpoint{0.033333in}{0.000000in}}%
\pgfpathcurveto{\pgfqpoint{0.033333in}{0.008840in}}{\pgfqpoint{0.029821in}{0.017319in}}{\pgfqpoint{0.023570in}{0.023570in}}%
\pgfpathcurveto{\pgfqpoint{0.017319in}{0.029821in}}{\pgfqpoint{0.008840in}{0.033333in}}{\pgfqpoint{0.000000in}{0.033333in}}%
\pgfpathcurveto{\pgfqpoint{-0.008840in}{0.033333in}}{\pgfqpoint{-0.017319in}{0.029821in}}{\pgfqpoint{-0.023570in}{0.023570in}}%
\pgfpathcurveto{\pgfqpoint{-0.029821in}{0.017319in}}{\pgfqpoint{-0.033333in}{0.008840in}}{\pgfqpoint{-0.033333in}{0.000000in}}%
\pgfpathcurveto{\pgfqpoint{-0.033333in}{-0.008840in}}{\pgfqpoint{-0.029821in}{-0.017319in}}{\pgfqpoint{-0.023570in}{-0.023570in}}%
\pgfpathcurveto{\pgfqpoint{-0.017319in}{-0.029821in}}{\pgfqpoint{-0.008840in}{-0.033333in}}{\pgfqpoint{0.000000in}{-0.033333in}}%
\pgfpathlineto{\pgfqpoint{0.000000in}{-0.033333in}}%
\pgfpathclose%
\pgfusepath{stroke,fill}%
}%
\begin{pgfscope}%
\pgfsys@transformshift{1.590974in}{2.419722in}%
\pgfsys@useobject{currentmarker}{}%
\end{pgfscope}%
\end{pgfscope}%
\begin{pgfscope}%
\pgfpathrectangle{\pgfqpoint{0.765000in}{0.660000in}}{\pgfqpoint{4.620000in}{4.620000in}}%
\pgfusepath{clip}%
\pgfsetrectcap%
\pgfsetroundjoin%
\pgfsetlinewidth{1.204500pt}%
\definecolor{currentstroke}{rgb}{1.000000,0.576471,0.309804}%
\pgfsetstrokecolor{currentstroke}%
\pgfsetdash{}{0pt}%
\pgfpathmoveto{\pgfqpoint{3.972169in}{1.866835in}}%
\pgfusepath{stroke}%
\end{pgfscope}%
\begin{pgfscope}%
\pgfpathrectangle{\pgfqpoint{0.765000in}{0.660000in}}{\pgfqpoint{4.620000in}{4.620000in}}%
\pgfusepath{clip}%
\pgfsetbuttcap%
\pgfsetroundjoin%
\definecolor{currentfill}{rgb}{1.000000,0.576471,0.309804}%
\pgfsetfillcolor{currentfill}%
\pgfsetlinewidth{1.003750pt}%
\definecolor{currentstroke}{rgb}{1.000000,0.576471,0.309804}%
\pgfsetstrokecolor{currentstroke}%
\pgfsetdash{}{0pt}%
\pgfsys@defobject{currentmarker}{\pgfqpoint{-0.033333in}{-0.033333in}}{\pgfqpoint{0.033333in}{0.033333in}}{%
\pgfpathmoveto{\pgfqpoint{0.000000in}{-0.033333in}}%
\pgfpathcurveto{\pgfqpoint{0.008840in}{-0.033333in}}{\pgfqpoint{0.017319in}{-0.029821in}}{\pgfqpoint{0.023570in}{-0.023570in}}%
\pgfpathcurveto{\pgfqpoint{0.029821in}{-0.017319in}}{\pgfqpoint{0.033333in}{-0.008840in}}{\pgfqpoint{0.033333in}{0.000000in}}%
\pgfpathcurveto{\pgfqpoint{0.033333in}{0.008840in}}{\pgfqpoint{0.029821in}{0.017319in}}{\pgfqpoint{0.023570in}{0.023570in}}%
\pgfpathcurveto{\pgfqpoint{0.017319in}{0.029821in}}{\pgfqpoint{0.008840in}{0.033333in}}{\pgfqpoint{0.000000in}{0.033333in}}%
\pgfpathcurveto{\pgfqpoint{-0.008840in}{0.033333in}}{\pgfqpoint{-0.017319in}{0.029821in}}{\pgfqpoint{-0.023570in}{0.023570in}}%
\pgfpathcurveto{\pgfqpoint{-0.029821in}{0.017319in}}{\pgfqpoint{-0.033333in}{0.008840in}}{\pgfqpoint{-0.033333in}{0.000000in}}%
\pgfpathcurveto{\pgfqpoint{-0.033333in}{-0.008840in}}{\pgfqpoint{-0.029821in}{-0.017319in}}{\pgfqpoint{-0.023570in}{-0.023570in}}%
\pgfpathcurveto{\pgfqpoint{-0.017319in}{-0.029821in}}{\pgfqpoint{-0.008840in}{-0.033333in}}{\pgfqpoint{0.000000in}{-0.033333in}}%
\pgfpathlineto{\pgfqpoint{0.000000in}{-0.033333in}}%
\pgfpathclose%
\pgfusepath{stroke,fill}%
}%
\begin{pgfscope}%
\pgfsys@transformshift{3.972169in}{1.866835in}%
\pgfsys@useobject{currentmarker}{}%
\end{pgfscope}%
\end{pgfscope}%
\begin{pgfscope}%
\pgfpathrectangle{\pgfqpoint{0.765000in}{0.660000in}}{\pgfqpoint{4.620000in}{4.620000in}}%
\pgfusepath{clip}%
\pgfsetrectcap%
\pgfsetroundjoin%
\pgfsetlinewidth{1.204500pt}%
\definecolor{currentstroke}{rgb}{1.000000,0.576471,0.309804}%
\pgfsetstrokecolor{currentstroke}%
\pgfsetdash{}{0pt}%
\pgfpathmoveto{\pgfqpoint{2.762616in}{2.743617in}}%
\pgfusepath{stroke}%
\end{pgfscope}%
\begin{pgfscope}%
\pgfpathrectangle{\pgfqpoint{0.765000in}{0.660000in}}{\pgfqpoint{4.620000in}{4.620000in}}%
\pgfusepath{clip}%
\pgfsetbuttcap%
\pgfsetroundjoin%
\definecolor{currentfill}{rgb}{1.000000,0.576471,0.309804}%
\pgfsetfillcolor{currentfill}%
\pgfsetlinewidth{1.003750pt}%
\definecolor{currentstroke}{rgb}{1.000000,0.576471,0.309804}%
\pgfsetstrokecolor{currentstroke}%
\pgfsetdash{}{0pt}%
\pgfsys@defobject{currentmarker}{\pgfqpoint{-0.033333in}{-0.033333in}}{\pgfqpoint{0.033333in}{0.033333in}}{%
\pgfpathmoveto{\pgfqpoint{0.000000in}{-0.033333in}}%
\pgfpathcurveto{\pgfqpoint{0.008840in}{-0.033333in}}{\pgfqpoint{0.017319in}{-0.029821in}}{\pgfqpoint{0.023570in}{-0.023570in}}%
\pgfpathcurveto{\pgfqpoint{0.029821in}{-0.017319in}}{\pgfqpoint{0.033333in}{-0.008840in}}{\pgfqpoint{0.033333in}{0.000000in}}%
\pgfpathcurveto{\pgfqpoint{0.033333in}{0.008840in}}{\pgfqpoint{0.029821in}{0.017319in}}{\pgfqpoint{0.023570in}{0.023570in}}%
\pgfpathcurveto{\pgfqpoint{0.017319in}{0.029821in}}{\pgfqpoint{0.008840in}{0.033333in}}{\pgfqpoint{0.000000in}{0.033333in}}%
\pgfpathcurveto{\pgfqpoint{-0.008840in}{0.033333in}}{\pgfqpoint{-0.017319in}{0.029821in}}{\pgfqpoint{-0.023570in}{0.023570in}}%
\pgfpathcurveto{\pgfqpoint{-0.029821in}{0.017319in}}{\pgfqpoint{-0.033333in}{0.008840in}}{\pgfqpoint{-0.033333in}{0.000000in}}%
\pgfpathcurveto{\pgfqpoint{-0.033333in}{-0.008840in}}{\pgfqpoint{-0.029821in}{-0.017319in}}{\pgfqpoint{-0.023570in}{-0.023570in}}%
\pgfpathcurveto{\pgfqpoint{-0.017319in}{-0.029821in}}{\pgfqpoint{-0.008840in}{-0.033333in}}{\pgfqpoint{0.000000in}{-0.033333in}}%
\pgfpathlineto{\pgfqpoint{0.000000in}{-0.033333in}}%
\pgfpathclose%
\pgfusepath{stroke,fill}%
}%
\begin{pgfscope}%
\pgfsys@transformshift{2.762616in}{2.743617in}%
\pgfsys@useobject{currentmarker}{}%
\end{pgfscope}%
\end{pgfscope}%
\begin{pgfscope}%
\pgfpathrectangle{\pgfqpoint{0.765000in}{0.660000in}}{\pgfqpoint{4.620000in}{4.620000in}}%
\pgfusepath{clip}%
\pgfsetrectcap%
\pgfsetroundjoin%
\pgfsetlinewidth{1.204500pt}%
\definecolor{currentstroke}{rgb}{1.000000,0.576471,0.309804}%
\pgfsetstrokecolor{currentstroke}%
\pgfsetdash{}{0pt}%
\pgfpathmoveto{\pgfqpoint{2.374459in}{2.103971in}}%
\pgfusepath{stroke}%
\end{pgfscope}%
\begin{pgfscope}%
\pgfpathrectangle{\pgfqpoint{0.765000in}{0.660000in}}{\pgfqpoint{4.620000in}{4.620000in}}%
\pgfusepath{clip}%
\pgfsetbuttcap%
\pgfsetroundjoin%
\definecolor{currentfill}{rgb}{1.000000,0.576471,0.309804}%
\pgfsetfillcolor{currentfill}%
\pgfsetlinewidth{1.003750pt}%
\definecolor{currentstroke}{rgb}{1.000000,0.576471,0.309804}%
\pgfsetstrokecolor{currentstroke}%
\pgfsetdash{}{0pt}%
\pgfsys@defobject{currentmarker}{\pgfqpoint{-0.033333in}{-0.033333in}}{\pgfqpoint{0.033333in}{0.033333in}}{%
\pgfpathmoveto{\pgfqpoint{0.000000in}{-0.033333in}}%
\pgfpathcurveto{\pgfqpoint{0.008840in}{-0.033333in}}{\pgfqpoint{0.017319in}{-0.029821in}}{\pgfqpoint{0.023570in}{-0.023570in}}%
\pgfpathcurveto{\pgfqpoint{0.029821in}{-0.017319in}}{\pgfqpoint{0.033333in}{-0.008840in}}{\pgfqpoint{0.033333in}{0.000000in}}%
\pgfpathcurveto{\pgfqpoint{0.033333in}{0.008840in}}{\pgfqpoint{0.029821in}{0.017319in}}{\pgfqpoint{0.023570in}{0.023570in}}%
\pgfpathcurveto{\pgfqpoint{0.017319in}{0.029821in}}{\pgfqpoint{0.008840in}{0.033333in}}{\pgfqpoint{0.000000in}{0.033333in}}%
\pgfpathcurveto{\pgfqpoint{-0.008840in}{0.033333in}}{\pgfqpoint{-0.017319in}{0.029821in}}{\pgfqpoint{-0.023570in}{0.023570in}}%
\pgfpathcurveto{\pgfqpoint{-0.029821in}{0.017319in}}{\pgfqpoint{-0.033333in}{0.008840in}}{\pgfqpoint{-0.033333in}{0.000000in}}%
\pgfpathcurveto{\pgfqpoint{-0.033333in}{-0.008840in}}{\pgfqpoint{-0.029821in}{-0.017319in}}{\pgfqpoint{-0.023570in}{-0.023570in}}%
\pgfpathcurveto{\pgfqpoint{-0.017319in}{-0.029821in}}{\pgfqpoint{-0.008840in}{-0.033333in}}{\pgfqpoint{0.000000in}{-0.033333in}}%
\pgfpathlineto{\pgfqpoint{0.000000in}{-0.033333in}}%
\pgfpathclose%
\pgfusepath{stroke,fill}%
}%
\begin{pgfscope}%
\pgfsys@transformshift{2.374459in}{2.103971in}%
\pgfsys@useobject{currentmarker}{}%
\end{pgfscope}%
\end{pgfscope}%
\begin{pgfscope}%
\pgfpathrectangle{\pgfqpoint{0.765000in}{0.660000in}}{\pgfqpoint{4.620000in}{4.620000in}}%
\pgfusepath{clip}%
\pgfsetrectcap%
\pgfsetroundjoin%
\pgfsetlinewidth{1.204500pt}%
\definecolor{currentstroke}{rgb}{1.000000,0.576471,0.309804}%
\pgfsetstrokecolor{currentstroke}%
\pgfsetdash{}{0pt}%
\pgfpathmoveto{\pgfqpoint{2.931207in}{1.901549in}}%
\pgfusepath{stroke}%
\end{pgfscope}%
\begin{pgfscope}%
\pgfpathrectangle{\pgfqpoint{0.765000in}{0.660000in}}{\pgfqpoint{4.620000in}{4.620000in}}%
\pgfusepath{clip}%
\pgfsetbuttcap%
\pgfsetroundjoin%
\definecolor{currentfill}{rgb}{1.000000,0.576471,0.309804}%
\pgfsetfillcolor{currentfill}%
\pgfsetlinewidth{1.003750pt}%
\definecolor{currentstroke}{rgb}{1.000000,0.576471,0.309804}%
\pgfsetstrokecolor{currentstroke}%
\pgfsetdash{}{0pt}%
\pgfsys@defobject{currentmarker}{\pgfqpoint{-0.033333in}{-0.033333in}}{\pgfqpoint{0.033333in}{0.033333in}}{%
\pgfpathmoveto{\pgfqpoint{0.000000in}{-0.033333in}}%
\pgfpathcurveto{\pgfqpoint{0.008840in}{-0.033333in}}{\pgfqpoint{0.017319in}{-0.029821in}}{\pgfqpoint{0.023570in}{-0.023570in}}%
\pgfpathcurveto{\pgfqpoint{0.029821in}{-0.017319in}}{\pgfqpoint{0.033333in}{-0.008840in}}{\pgfqpoint{0.033333in}{0.000000in}}%
\pgfpathcurveto{\pgfqpoint{0.033333in}{0.008840in}}{\pgfqpoint{0.029821in}{0.017319in}}{\pgfqpoint{0.023570in}{0.023570in}}%
\pgfpathcurveto{\pgfqpoint{0.017319in}{0.029821in}}{\pgfqpoint{0.008840in}{0.033333in}}{\pgfqpoint{0.000000in}{0.033333in}}%
\pgfpathcurveto{\pgfqpoint{-0.008840in}{0.033333in}}{\pgfqpoint{-0.017319in}{0.029821in}}{\pgfqpoint{-0.023570in}{0.023570in}}%
\pgfpathcurveto{\pgfqpoint{-0.029821in}{0.017319in}}{\pgfqpoint{-0.033333in}{0.008840in}}{\pgfqpoint{-0.033333in}{0.000000in}}%
\pgfpathcurveto{\pgfqpoint{-0.033333in}{-0.008840in}}{\pgfqpoint{-0.029821in}{-0.017319in}}{\pgfqpoint{-0.023570in}{-0.023570in}}%
\pgfpathcurveto{\pgfqpoint{-0.017319in}{-0.029821in}}{\pgfqpoint{-0.008840in}{-0.033333in}}{\pgfqpoint{0.000000in}{-0.033333in}}%
\pgfpathlineto{\pgfqpoint{0.000000in}{-0.033333in}}%
\pgfpathclose%
\pgfusepath{stroke,fill}%
}%
\begin{pgfscope}%
\pgfsys@transformshift{2.931207in}{1.901549in}%
\pgfsys@useobject{currentmarker}{}%
\end{pgfscope}%
\end{pgfscope}%
\begin{pgfscope}%
\pgfpathrectangle{\pgfqpoint{0.765000in}{0.660000in}}{\pgfqpoint{4.620000in}{4.620000in}}%
\pgfusepath{clip}%
\pgfsetrectcap%
\pgfsetroundjoin%
\pgfsetlinewidth{1.204500pt}%
\definecolor{currentstroke}{rgb}{1.000000,0.576471,0.309804}%
\pgfsetstrokecolor{currentstroke}%
\pgfsetdash{}{0pt}%
\pgfpathmoveto{\pgfqpoint{4.530563in}{3.280270in}}%
\pgfusepath{stroke}%
\end{pgfscope}%
\begin{pgfscope}%
\pgfpathrectangle{\pgfqpoint{0.765000in}{0.660000in}}{\pgfqpoint{4.620000in}{4.620000in}}%
\pgfusepath{clip}%
\pgfsetbuttcap%
\pgfsetroundjoin%
\definecolor{currentfill}{rgb}{1.000000,0.576471,0.309804}%
\pgfsetfillcolor{currentfill}%
\pgfsetlinewidth{1.003750pt}%
\definecolor{currentstroke}{rgb}{1.000000,0.576471,0.309804}%
\pgfsetstrokecolor{currentstroke}%
\pgfsetdash{}{0pt}%
\pgfsys@defobject{currentmarker}{\pgfqpoint{-0.033333in}{-0.033333in}}{\pgfqpoint{0.033333in}{0.033333in}}{%
\pgfpathmoveto{\pgfqpoint{0.000000in}{-0.033333in}}%
\pgfpathcurveto{\pgfqpoint{0.008840in}{-0.033333in}}{\pgfqpoint{0.017319in}{-0.029821in}}{\pgfqpoint{0.023570in}{-0.023570in}}%
\pgfpathcurveto{\pgfqpoint{0.029821in}{-0.017319in}}{\pgfqpoint{0.033333in}{-0.008840in}}{\pgfqpoint{0.033333in}{0.000000in}}%
\pgfpathcurveto{\pgfqpoint{0.033333in}{0.008840in}}{\pgfqpoint{0.029821in}{0.017319in}}{\pgfqpoint{0.023570in}{0.023570in}}%
\pgfpathcurveto{\pgfqpoint{0.017319in}{0.029821in}}{\pgfqpoint{0.008840in}{0.033333in}}{\pgfqpoint{0.000000in}{0.033333in}}%
\pgfpathcurveto{\pgfqpoint{-0.008840in}{0.033333in}}{\pgfqpoint{-0.017319in}{0.029821in}}{\pgfqpoint{-0.023570in}{0.023570in}}%
\pgfpathcurveto{\pgfqpoint{-0.029821in}{0.017319in}}{\pgfqpoint{-0.033333in}{0.008840in}}{\pgfqpoint{-0.033333in}{0.000000in}}%
\pgfpathcurveto{\pgfqpoint{-0.033333in}{-0.008840in}}{\pgfqpoint{-0.029821in}{-0.017319in}}{\pgfqpoint{-0.023570in}{-0.023570in}}%
\pgfpathcurveto{\pgfqpoint{-0.017319in}{-0.029821in}}{\pgfqpoint{-0.008840in}{-0.033333in}}{\pgfqpoint{0.000000in}{-0.033333in}}%
\pgfpathlineto{\pgfqpoint{0.000000in}{-0.033333in}}%
\pgfpathclose%
\pgfusepath{stroke,fill}%
}%
\begin{pgfscope}%
\pgfsys@transformshift{4.530563in}{3.280270in}%
\pgfsys@useobject{currentmarker}{}%
\end{pgfscope}%
\end{pgfscope}%
\begin{pgfscope}%
\pgfpathrectangle{\pgfqpoint{0.765000in}{0.660000in}}{\pgfqpoint{4.620000in}{4.620000in}}%
\pgfusepath{clip}%
\pgfsetrectcap%
\pgfsetroundjoin%
\pgfsetlinewidth{1.204500pt}%
\definecolor{currentstroke}{rgb}{1.000000,0.576471,0.309804}%
\pgfsetstrokecolor{currentstroke}%
\pgfsetdash{}{0pt}%
\pgfpathmoveto{\pgfqpoint{2.387917in}{2.560855in}}%
\pgfusepath{stroke}%
\end{pgfscope}%
\begin{pgfscope}%
\pgfpathrectangle{\pgfqpoint{0.765000in}{0.660000in}}{\pgfqpoint{4.620000in}{4.620000in}}%
\pgfusepath{clip}%
\pgfsetbuttcap%
\pgfsetroundjoin%
\definecolor{currentfill}{rgb}{1.000000,0.576471,0.309804}%
\pgfsetfillcolor{currentfill}%
\pgfsetlinewidth{1.003750pt}%
\definecolor{currentstroke}{rgb}{1.000000,0.576471,0.309804}%
\pgfsetstrokecolor{currentstroke}%
\pgfsetdash{}{0pt}%
\pgfsys@defobject{currentmarker}{\pgfqpoint{-0.033333in}{-0.033333in}}{\pgfqpoint{0.033333in}{0.033333in}}{%
\pgfpathmoveto{\pgfqpoint{0.000000in}{-0.033333in}}%
\pgfpathcurveto{\pgfqpoint{0.008840in}{-0.033333in}}{\pgfqpoint{0.017319in}{-0.029821in}}{\pgfqpoint{0.023570in}{-0.023570in}}%
\pgfpathcurveto{\pgfqpoint{0.029821in}{-0.017319in}}{\pgfqpoint{0.033333in}{-0.008840in}}{\pgfqpoint{0.033333in}{0.000000in}}%
\pgfpathcurveto{\pgfqpoint{0.033333in}{0.008840in}}{\pgfqpoint{0.029821in}{0.017319in}}{\pgfqpoint{0.023570in}{0.023570in}}%
\pgfpathcurveto{\pgfqpoint{0.017319in}{0.029821in}}{\pgfqpoint{0.008840in}{0.033333in}}{\pgfqpoint{0.000000in}{0.033333in}}%
\pgfpathcurveto{\pgfqpoint{-0.008840in}{0.033333in}}{\pgfqpoint{-0.017319in}{0.029821in}}{\pgfqpoint{-0.023570in}{0.023570in}}%
\pgfpathcurveto{\pgfqpoint{-0.029821in}{0.017319in}}{\pgfqpoint{-0.033333in}{0.008840in}}{\pgfqpoint{-0.033333in}{0.000000in}}%
\pgfpathcurveto{\pgfqpoint{-0.033333in}{-0.008840in}}{\pgfqpoint{-0.029821in}{-0.017319in}}{\pgfqpoint{-0.023570in}{-0.023570in}}%
\pgfpathcurveto{\pgfqpoint{-0.017319in}{-0.029821in}}{\pgfqpoint{-0.008840in}{-0.033333in}}{\pgfqpoint{0.000000in}{-0.033333in}}%
\pgfpathlineto{\pgfqpoint{0.000000in}{-0.033333in}}%
\pgfpathclose%
\pgfusepath{stroke,fill}%
}%
\begin{pgfscope}%
\pgfsys@transformshift{2.387917in}{2.560855in}%
\pgfsys@useobject{currentmarker}{}%
\end{pgfscope}%
\end{pgfscope}%
\begin{pgfscope}%
\pgfpathrectangle{\pgfqpoint{0.765000in}{0.660000in}}{\pgfqpoint{4.620000in}{4.620000in}}%
\pgfusepath{clip}%
\pgfsetrectcap%
\pgfsetroundjoin%
\pgfsetlinewidth{1.204500pt}%
\definecolor{currentstroke}{rgb}{1.000000,0.576471,0.309804}%
\pgfsetstrokecolor{currentstroke}%
\pgfsetdash{}{0pt}%
\pgfpathmoveto{\pgfqpoint{2.437063in}{2.103395in}}%
\pgfusepath{stroke}%
\end{pgfscope}%
\begin{pgfscope}%
\pgfpathrectangle{\pgfqpoint{0.765000in}{0.660000in}}{\pgfqpoint{4.620000in}{4.620000in}}%
\pgfusepath{clip}%
\pgfsetbuttcap%
\pgfsetroundjoin%
\definecolor{currentfill}{rgb}{1.000000,0.576471,0.309804}%
\pgfsetfillcolor{currentfill}%
\pgfsetlinewidth{1.003750pt}%
\definecolor{currentstroke}{rgb}{1.000000,0.576471,0.309804}%
\pgfsetstrokecolor{currentstroke}%
\pgfsetdash{}{0pt}%
\pgfsys@defobject{currentmarker}{\pgfqpoint{-0.033333in}{-0.033333in}}{\pgfqpoint{0.033333in}{0.033333in}}{%
\pgfpathmoveto{\pgfqpoint{0.000000in}{-0.033333in}}%
\pgfpathcurveto{\pgfqpoint{0.008840in}{-0.033333in}}{\pgfqpoint{0.017319in}{-0.029821in}}{\pgfqpoint{0.023570in}{-0.023570in}}%
\pgfpathcurveto{\pgfqpoint{0.029821in}{-0.017319in}}{\pgfqpoint{0.033333in}{-0.008840in}}{\pgfqpoint{0.033333in}{0.000000in}}%
\pgfpathcurveto{\pgfqpoint{0.033333in}{0.008840in}}{\pgfqpoint{0.029821in}{0.017319in}}{\pgfqpoint{0.023570in}{0.023570in}}%
\pgfpathcurveto{\pgfqpoint{0.017319in}{0.029821in}}{\pgfqpoint{0.008840in}{0.033333in}}{\pgfqpoint{0.000000in}{0.033333in}}%
\pgfpathcurveto{\pgfqpoint{-0.008840in}{0.033333in}}{\pgfqpoint{-0.017319in}{0.029821in}}{\pgfqpoint{-0.023570in}{0.023570in}}%
\pgfpathcurveto{\pgfqpoint{-0.029821in}{0.017319in}}{\pgfqpoint{-0.033333in}{0.008840in}}{\pgfqpoint{-0.033333in}{0.000000in}}%
\pgfpathcurveto{\pgfqpoint{-0.033333in}{-0.008840in}}{\pgfqpoint{-0.029821in}{-0.017319in}}{\pgfqpoint{-0.023570in}{-0.023570in}}%
\pgfpathcurveto{\pgfqpoint{-0.017319in}{-0.029821in}}{\pgfqpoint{-0.008840in}{-0.033333in}}{\pgfqpoint{0.000000in}{-0.033333in}}%
\pgfpathlineto{\pgfqpoint{0.000000in}{-0.033333in}}%
\pgfpathclose%
\pgfusepath{stroke,fill}%
}%
\begin{pgfscope}%
\pgfsys@transformshift{2.437063in}{2.103395in}%
\pgfsys@useobject{currentmarker}{}%
\end{pgfscope}%
\end{pgfscope}%
\begin{pgfscope}%
\pgfpathrectangle{\pgfqpoint{0.765000in}{0.660000in}}{\pgfqpoint{4.620000in}{4.620000in}}%
\pgfusepath{clip}%
\pgfsetrectcap%
\pgfsetroundjoin%
\pgfsetlinewidth{1.204500pt}%
\definecolor{currentstroke}{rgb}{1.000000,0.576471,0.309804}%
\pgfsetstrokecolor{currentstroke}%
\pgfsetdash{}{0pt}%
\pgfpathmoveto{\pgfqpoint{2.701079in}{1.681295in}}%
\pgfusepath{stroke}%
\end{pgfscope}%
\begin{pgfscope}%
\pgfpathrectangle{\pgfqpoint{0.765000in}{0.660000in}}{\pgfqpoint{4.620000in}{4.620000in}}%
\pgfusepath{clip}%
\pgfsetbuttcap%
\pgfsetroundjoin%
\definecolor{currentfill}{rgb}{1.000000,0.576471,0.309804}%
\pgfsetfillcolor{currentfill}%
\pgfsetlinewidth{1.003750pt}%
\definecolor{currentstroke}{rgb}{1.000000,0.576471,0.309804}%
\pgfsetstrokecolor{currentstroke}%
\pgfsetdash{}{0pt}%
\pgfsys@defobject{currentmarker}{\pgfqpoint{-0.033333in}{-0.033333in}}{\pgfqpoint{0.033333in}{0.033333in}}{%
\pgfpathmoveto{\pgfqpoint{0.000000in}{-0.033333in}}%
\pgfpathcurveto{\pgfqpoint{0.008840in}{-0.033333in}}{\pgfqpoint{0.017319in}{-0.029821in}}{\pgfqpoint{0.023570in}{-0.023570in}}%
\pgfpathcurveto{\pgfqpoint{0.029821in}{-0.017319in}}{\pgfqpoint{0.033333in}{-0.008840in}}{\pgfqpoint{0.033333in}{0.000000in}}%
\pgfpathcurveto{\pgfqpoint{0.033333in}{0.008840in}}{\pgfqpoint{0.029821in}{0.017319in}}{\pgfqpoint{0.023570in}{0.023570in}}%
\pgfpathcurveto{\pgfqpoint{0.017319in}{0.029821in}}{\pgfqpoint{0.008840in}{0.033333in}}{\pgfqpoint{0.000000in}{0.033333in}}%
\pgfpathcurveto{\pgfqpoint{-0.008840in}{0.033333in}}{\pgfqpoint{-0.017319in}{0.029821in}}{\pgfqpoint{-0.023570in}{0.023570in}}%
\pgfpathcurveto{\pgfqpoint{-0.029821in}{0.017319in}}{\pgfqpoint{-0.033333in}{0.008840in}}{\pgfqpoint{-0.033333in}{0.000000in}}%
\pgfpathcurveto{\pgfqpoint{-0.033333in}{-0.008840in}}{\pgfqpoint{-0.029821in}{-0.017319in}}{\pgfqpoint{-0.023570in}{-0.023570in}}%
\pgfpathcurveto{\pgfqpoint{-0.017319in}{-0.029821in}}{\pgfqpoint{-0.008840in}{-0.033333in}}{\pgfqpoint{0.000000in}{-0.033333in}}%
\pgfpathlineto{\pgfqpoint{0.000000in}{-0.033333in}}%
\pgfpathclose%
\pgfusepath{stroke,fill}%
}%
\begin{pgfscope}%
\pgfsys@transformshift{2.701079in}{1.681295in}%
\pgfsys@useobject{currentmarker}{}%
\end{pgfscope}%
\end{pgfscope}%
\begin{pgfscope}%
\pgfpathrectangle{\pgfqpoint{0.765000in}{0.660000in}}{\pgfqpoint{4.620000in}{4.620000in}}%
\pgfusepath{clip}%
\pgfsetrectcap%
\pgfsetroundjoin%
\pgfsetlinewidth{1.204500pt}%
\definecolor{currentstroke}{rgb}{1.000000,0.576471,0.309804}%
\pgfsetstrokecolor{currentstroke}%
\pgfsetdash{}{0pt}%
\pgfpathmoveto{\pgfqpoint{3.200954in}{2.477534in}}%
\pgfusepath{stroke}%
\end{pgfscope}%
\begin{pgfscope}%
\pgfpathrectangle{\pgfqpoint{0.765000in}{0.660000in}}{\pgfqpoint{4.620000in}{4.620000in}}%
\pgfusepath{clip}%
\pgfsetbuttcap%
\pgfsetroundjoin%
\definecolor{currentfill}{rgb}{1.000000,0.576471,0.309804}%
\pgfsetfillcolor{currentfill}%
\pgfsetlinewidth{1.003750pt}%
\definecolor{currentstroke}{rgb}{1.000000,0.576471,0.309804}%
\pgfsetstrokecolor{currentstroke}%
\pgfsetdash{}{0pt}%
\pgfsys@defobject{currentmarker}{\pgfqpoint{-0.033333in}{-0.033333in}}{\pgfqpoint{0.033333in}{0.033333in}}{%
\pgfpathmoveto{\pgfqpoint{0.000000in}{-0.033333in}}%
\pgfpathcurveto{\pgfqpoint{0.008840in}{-0.033333in}}{\pgfqpoint{0.017319in}{-0.029821in}}{\pgfqpoint{0.023570in}{-0.023570in}}%
\pgfpathcurveto{\pgfqpoint{0.029821in}{-0.017319in}}{\pgfqpoint{0.033333in}{-0.008840in}}{\pgfqpoint{0.033333in}{0.000000in}}%
\pgfpathcurveto{\pgfqpoint{0.033333in}{0.008840in}}{\pgfqpoint{0.029821in}{0.017319in}}{\pgfqpoint{0.023570in}{0.023570in}}%
\pgfpathcurveto{\pgfqpoint{0.017319in}{0.029821in}}{\pgfqpoint{0.008840in}{0.033333in}}{\pgfqpoint{0.000000in}{0.033333in}}%
\pgfpathcurveto{\pgfqpoint{-0.008840in}{0.033333in}}{\pgfqpoint{-0.017319in}{0.029821in}}{\pgfqpoint{-0.023570in}{0.023570in}}%
\pgfpathcurveto{\pgfqpoint{-0.029821in}{0.017319in}}{\pgfqpoint{-0.033333in}{0.008840in}}{\pgfqpoint{-0.033333in}{0.000000in}}%
\pgfpathcurveto{\pgfqpoint{-0.033333in}{-0.008840in}}{\pgfqpoint{-0.029821in}{-0.017319in}}{\pgfqpoint{-0.023570in}{-0.023570in}}%
\pgfpathcurveto{\pgfqpoint{-0.017319in}{-0.029821in}}{\pgfqpoint{-0.008840in}{-0.033333in}}{\pgfqpoint{0.000000in}{-0.033333in}}%
\pgfpathlineto{\pgfqpoint{0.000000in}{-0.033333in}}%
\pgfpathclose%
\pgfusepath{stroke,fill}%
}%
\begin{pgfscope}%
\pgfsys@transformshift{3.200954in}{2.477534in}%
\pgfsys@useobject{currentmarker}{}%
\end{pgfscope}%
\end{pgfscope}%
\begin{pgfscope}%
\pgfpathrectangle{\pgfqpoint{0.765000in}{0.660000in}}{\pgfqpoint{4.620000in}{4.620000in}}%
\pgfusepath{clip}%
\pgfsetrectcap%
\pgfsetroundjoin%
\pgfsetlinewidth{1.204500pt}%
\definecolor{currentstroke}{rgb}{1.000000,0.576471,0.309804}%
\pgfsetstrokecolor{currentstroke}%
\pgfsetdash{}{0pt}%
\pgfpathmoveto{\pgfqpoint{3.220808in}{2.172069in}}%
\pgfusepath{stroke}%
\end{pgfscope}%
\begin{pgfscope}%
\pgfpathrectangle{\pgfqpoint{0.765000in}{0.660000in}}{\pgfqpoint{4.620000in}{4.620000in}}%
\pgfusepath{clip}%
\pgfsetbuttcap%
\pgfsetroundjoin%
\definecolor{currentfill}{rgb}{1.000000,0.576471,0.309804}%
\pgfsetfillcolor{currentfill}%
\pgfsetlinewidth{1.003750pt}%
\definecolor{currentstroke}{rgb}{1.000000,0.576471,0.309804}%
\pgfsetstrokecolor{currentstroke}%
\pgfsetdash{}{0pt}%
\pgfsys@defobject{currentmarker}{\pgfqpoint{-0.033333in}{-0.033333in}}{\pgfqpoint{0.033333in}{0.033333in}}{%
\pgfpathmoveto{\pgfqpoint{0.000000in}{-0.033333in}}%
\pgfpathcurveto{\pgfqpoint{0.008840in}{-0.033333in}}{\pgfqpoint{0.017319in}{-0.029821in}}{\pgfqpoint{0.023570in}{-0.023570in}}%
\pgfpathcurveto{\pgfqpoint{0.029821in}{-0.017319in}}{\pgfqpoint{0.033333in}{-0.008840in}}{\pgfqpoint{0.033333in}{0.000000in}}%
\pgfpathcurveto{\pgfqpoint{0.033333in}{0.008840in}}{\pgfqpoint{0.029821in}{0.017319in}}{\pgfqpoint{0.023570in}{0.023570in}}%
\pgfpathcurveto{\pgfqpoint{0.017319in}{0.029821in}}{\pgfqpoint{0.008840in}{0.033333in}}{\pgfqpoint{0.000000in}{0.033333in}}%
\pgfpathcurveto{\pgfqpoint{-0.008840in}{0.033333in}}{\pgfqpoint{-0.017319in}{0.029821in}}{\pgfqpoint{-0.023570in}{0.023570in}}%
\pgfpathcurveto{\pgfqpoint{-0.029821in}{0.017319in}}{\pgfqpoint{-0.033333in}{0.008840in}}{\pgfqpoint{-0.033333in}{0.000000in}}%
\pgfpathcurveto{\pgfqpoint{-0.033333in}{-0.008840in}}{\pgfqpoint{-0.029821in}{-0.017319in}}{\pgfqpoint{-0.023570in}{-0.023570in}}%
\pgfpathcurveto{\pgfqpoint{-0.017319in}{-0.029821in}}{\pgfqpoint{-0.008840in}{-0.033333in}}{\pgfqpoint{0.000000in}{-0.033333in}}%
\pgfpathlineto{\pgfqpoint{0.000000in}{-0.033333in}}%
\pgfpathclose%
\pgfusepath{stroke,fill}%
}%
\begin{pgfscope}%
\pgfsys@transformshift{3.220808in}{2.172069in}%
\pgfsys@useobject{currentmarker}{}%
\end{pgfscope}%
\end{pgfscope}%
\begin{pgfscope}%
\pgfpathrectangle{\pgfqpoint{0.765000in}{0.660000in}}{\pgfqpoint{4.620000in}{4.620000in}}%
\pgfusepath{clip}%
\pgfsetrectcap%
\pgfsetroundjoin%
\pgfsetlinewidth{1.204500pt}%
\definecolor{currentstroke}{rgb}{1.000000,0.576471,0.309804}%
\pgfsetstrokecolor{currentstroke}%
\pgfsetdash{}{0pt}%
\pgfpathmoveto{\pgfqpoint{2.810589in}{2.224888in}}%
\pgfusepath{stroke}%
\end{pgfscope}%
\begin{pgfscope}%
\pgfpathrectangle{\pgfqpoint{0.765000in}{0.660000in}}{\pgfqpoint{4.620000in}{4.620000in}}%
\pgfusepath{clip}%
\pgfsetbuttcap%
\pgfsetroundjoin%
\definecolor{currentfill}{rgb}{1.000000,0.576471,0.309804}%
\pgfsetfillcolor{currentfill}%
\pgfsetlinewidth{1.003750pt}%
\definecolor{currentstroke}{rgb}{1.000000,0.576471,0.309804}%
\pgfsetstrokecolor{currentstroke}%
\pgfsetdash{}{0pt}%
\pgfsys@defobject{currentmarker}{\pgfqpoint{-0.033333in}{-0.033333in}}{\pgfqpoint{0.033333in}{0.033333in}}{%
\pgfpathmoveto{\pgfqpoint{0.000000in}{-0.033333in}}%
\pgfpathcurveto{\pgfqpoint{0.008840in}{-0.033333in}}{\pgfqpoint{0.017319in}{-0.029821in}}{\pgfqpoint{0.023570in}{-0.023570in}}%
\pgfpathcurveto{\pgfqpoint{0.029821in}{-0.017319in}}{\pgfqpoint{0.033333in}{-0.008840in}}{\pgfqpoint{0.033333in}{0.000000in}}%
\pgfpathcurveto{\pgfqpoint{0.033333in}{0.008840in}}{\pgfqpoint{0.029821in}{0.017319in}}{\pgfqpoint{0.023570in}{0.023570in}}%
\pgfpathcurveto{\pgfqpoint{0.017319in}{0.029821in}}{\pgfqpoint{0.008840in}{0.033333in}}{\pgfqpoint{0.000000in}{0.033333in}}%
\pgfpathcurveto{\pgfqpoint{-0.008840in}{0.033333in}}{\pgfqpoint{-0.017319in}{0.029821in}}{\pgfqpoint{-0.023570in}{0.023570in}}%
\pgfpathcurveto{\pgfqpoint{-0.029821in}{0.017319in}}{\pgfqpoint{-0.033333in}{0.008840in}}{\pgfqpoint{-0.033333in}{0.000000in}}%
\pgfpathcurveto{\pgfqpoint{-0.033333in}{-0.008840in}}{\pgfqpoint{-0.029821in}{-0.017319in}}{\pgfqpoint{-0.023570in}{-0.023570in}}%
\pgfpathcurveto{\pgfqpoint{-0.017319in}{-0.029821in}}{\pgfqpoint{-0.008840in}{-0.033333in}}{\pgfqpoint{0.000000in}{-0.033333in}}%
\pgfpathlineto{\pgfqpoint{0.000000in}{-0.033333in}}%
\pgfpathclose%
\pgfusepath{stroke,fill}%
}%
\begin{pgfscope}%
\pgfsys@transformshift{2.810589in}{2.224888in}%
\pgfsys@useobject{currentmarker}{}%
\end{pgfscope}%
\end{pgfscope}%
\begin{pgfscope}%
\pgfpathrectangle{\pgfqpoint{0.765000in}{0.660000in}}{\pgfqpoint{4.620000in}{4.620000in}}%
\pgfusepath{clip}%
\pgfsetrectcap%
\pgfsetroundjoin%
\pgfsetlinewidth{1.204500pt}%
\definecolor{currentstroke}{rgb}{1.000000,0.576471,0.309804}%
\pgfsetstrokecolor{currentstroke}%
\pgfsetdash{}{0pt}%
\pgfpathmoveto{\pgfqpoint{1.820587in}{2.848575in}}%
\pgfusepath{stroke}%
\end{pgfscope}%
\begin{pgfscope}%
\pgfpathrectangle{\pgfqpoint{0.765000in}{0.660000in}}{\pgfqpoint{4.620000in}{4.620000in}}%
\pgfusepath{clip}%
\pgfsetbuttcap%
\pgfsetroundjoin%
\definecolor{currentfill}{rgb}{1.000000,0.576471,0.309804}%
\pgfsetfillcolor{currentfill}%
\pgfsetlinewidth{1.003750pt}%
\definecolor{currentstroke}{rgb}{1.000000,0.576471,0.309804}%
\pgfsetstrokecolor{currentstroke}%
\pgfsetdash{}{0pt}%
\pgfsys@defobject{currentmarker}{\pgfqpoint{-0.033333in}{-0.033333in}}{\pgfqpoint{0.033333in}{0.033333in}}{%
\pgfpathmoveto{\pgfqpoint{0.000000in}{-0.033333in}}%
\pgfpathcurveto{\pgfqpoint{0.008840in}{-0.033333in}}{\pgfqpoint{0.017319in}{-0.029821in}}{\pgfqpoint{0.023570in}{-0.023570in}}%
\pgfpathcurveto{\pgfqpoint{0.029821in}{-0.017319in}}{\pgfqpoint{0.033333in}{-0.008840in}}{\pgfqpoint{0.033333in}{0.000000in}}%
\pgfpathcurveto{\pgfqpoint{0.033333in}{0.008840in}}{\pgfqpoint{0.029821in}{0.017319in}}{\pgfqpoint{0.023570in}{0.023570in}}%
\pgfpathcurveto{\pgfqpoint{0.017319in}{0.029821in}}{\pgfqpoint{0.008840in}{0.033333in}}{\pgfqpoint{0.000000in}{0.033333in}}%
\pgfpathcurveto{\pgfqpoint{-0.008840in}{0.033333in}}{\pgfqpoint{-0.017319in}{0.029821in}}{\pgfqpoint{-0.023570in}{0.023570in}}%
\pgfpathcurveto{\pgfqpoint{-0.029821in}{0.017319in}}{\pgfqpoint{-0.033333in}{0.008840in}}{\pgfqpoint{-0.033333in}{0.000000in}}%
\pgfpathcurveto{\pgfqpoint{-0.033333in}{-0.008840in}}{\pgfqpoint{-0.029821in}{-0.017319in}}{\pgfqpoint{-0.023570in}{-0.023570in}}%
\pgfpathcurveto{\pgfqpoint{-0.017319in}{-0.029821in}}{\pgfqpoint{-0.008840in}{-0.033333in}}{\pgfqpoint{0.000000in}{-0.033333in}}%
\pgfpathlineto{\pgfqpoint{0.000000in}{-0.033333in}}%
\pgfpathclose%
\pgfusepath{stroke,fill}%
}%
\begin{pgfscope}%
\pgfsys@transformshift{1.820587in}{2.848575in}%
\pgfsys@useobject{currentmarker}{}%
\end{pgfscope}%
\end{pgfscope}%
\begin{pgfscope}%
\pgfpathrectangle{\pgfqpoint{0.765000in}{0.660000in}}{\pgfqpoint{4.620000in}{4.620000in}}%
\pgfusepath{clip}%
\pgfsetrectcap%
\pgfsetroundjoin%
\pgfsetlinewidth{1.204500pt}%
\definecolor{currentstroke}{rgb}{1.000000,0.576471,0.309804}%
\pgfsetstrokecolor{currentstroke}%
\pgfsetdash{}{0pt}%
\pgfpathmoveto{\pgfqpoint{2.226292in}{2.971007in}}%
\pgfusepath{stroke}%
\end{pgfscope}%
\begin{pgfscope}%
\pgfpathrectangle{\pgfqpoint{0.765000in}{0.660000in}}{\pgfqpoint{4.620000in}{4.620000in}}%
\pgfusepath{clip}%
\pgfsetbuttcap%
\pgfsetroundjoin%
\definecolor{currentfill}{rgb}{1.000000,0.576471,0.309804}%
\pgfsetfillcolor{currentfill}%
\pgfsetlinewidth{1.003750pt}%
\definecolor{currentstroke}{rgb}{1.000000,0.576471,0.309804}%
\pgfsetstrokecolor{currentstroke}%
\pgfsetdash{}{0pt}%
\pgfsys@defobject{currentmarker}{\pgfqpoint{-0.033333in}{-0.033333in}}{\pgfqpoint{0.033333in}{0.033333in}}{%
\pgfpathmoveto{\pgfqpoint{0.000000in}{-0.033333in}}%
\pgfpathcurveto{\pgfqpoint{0.008840in}{-0.033333in}}{\pgfqpoint{0.017319in}{-0.029821in}}{\pgfqpoint{0.023570in}{-0.023570in}}%
\pgfpathcurveto{\pgfqpoint{0.029821in}{-0.017319in}}{\pgfqpoint{0.033333in}{-0.008840in}}{\pgfqpoint{0.033333in}{0.000000in}}%
\pgfpathcurveto{\pgfqpoint{0.033333in}{0.008840in}}{\pgfqpoint{0.029821in}{0.017319in}}{\pgfqpoint{0.023570in}{0.023570in}}%
\pgfpathcurveto{\pgfqpoint{0.017319in}{0.029821in}}{\pgfqpoint{0.008840in}{0.033333in}}{\pgfqpoint{0.000000in}{0.033333in}}%
\pgfpathcurveto{\pgfqpoint{-0.008840in}{0.033333in}}{\pgfqpoint{-0.017319in}{0.029821in}}{\pgfqpoint{-0.023570in}{0.023570in}}%
\pgfpathcurveto{\pgfqpoint{-0.029821in}{0.017319in}}{\pgfqpoint{-0.033333in}{0.008840in}}{\pgfqpoint{-0.033333in}{0.000000in}}%
\pgfpathcurveto{\pgfqpoint{-0.033333in}{-0.008840in}}{\pgfqpoint{-0.029821in}{-0.017319in}}{\pgfqpoint{-0.023570in}{-0.023570in}}%
\pgfpathcurveto{\pgfqpoint{-0.017319in}{-0.029821in}}{\pgfqpoint{-0.008840in}{-0.033333in}}{\pgfqpoint{0.000000in}{-0.033333in}}%
\pgfpathlineto{\pgfqpoint{0.000000in}{-0.033333in}}%
\pgfpathclose%
\pgfusepath{stroke,fill}%
}%
\begin{pgfscope}%
\pgfsys@transformshift{2.226292in}{2.971007in}%
\pgfsys@useobject{currentmarker}{}%
\end{pgfscope}%
\end{pgfscope}%
\begin{pgfscope}%
\pgfpathrectangle{\pgfqpoint{0.765000in}{0.660000in}}{\pgfqpoint{4.620000in}{4.620000in}}%
\pgfusepath{clip}%
\pgfsetrectcap%
\pgfsetroundjoin%
\pgfsetlinewidth{1.204500pt}%
\definecolor{currentstroke}{rgb}{1.000000,0.576471,0.309804}%
\pgfsetstrokecolor{currentstroke}%
\pgfsetdash{}{0pt}%
\pgfpathmoveto{\pgfqpoint{2.463334in}{2.110898in}}%
\pgfusepath{stroke}%
\end{pgfscope}%
\begin{pgfscope}%
\pgfpathrectangle{\pgfqpoint{0.765000in}{0.660000in}}{\pgfqpoint{4.620000in}{4.620000in}}%
\pgfusepath{clip}%
\pgfsetbuttcap%
\pgfsetroundjoin%
\definecolor{currentfill}{rgb}{1.000000,0.576471,0.309804}%
\pgfsetfillcolor{currentfill}%
\pgfsetlinewidth{1.003750pt}%
\definecolor{currentstroke}{rgb}{1.000000,0.576471,0.309804}%
\pgfsetstrokecolor{currentstroke}%
\pgfsetdash{}{0pt}%
\pgfsys@defobject{currentmarker}{\pgfqpoint{-0.033333in}{-0.033333in}}{\pgfqpoint{0.033333in}{0.033333in}}{%
\pgfpathmoveto{\pgfqpoint{0.000000in}{-0.033333in}}%
\pgfpathcurveto{\pgfqpoint{0.008840in}{-0.033333in}}{\pgfqpoint{0.017319in}{-0.029821in}}{\pgfqpoint{0.023570in}{-0.023570in}}%
\pgfpathcurveto{\pgfqpoint{0.029821in}{-0.017319in}}{\pgfqpoint{0.033333in}{-0.008840in}}{\pgfqpoint{0.033333in}{0.000000in}}%
\pgfpathcurveto{\pgfqpoint{0.033333in}{0.008840in}}{\pgfqpoint{0.029821in}{0.017319in}}{\pgfqpoint{0.023570in}{0.023570in}}%
\pgfpathcurveto{\pgfqpoint{0.017319in}{0.029821in}}{\pgfqpoint{0.008840in}{0.033333in}}{\pgfqpoint{0.000000in}{0.033333in}}%
\pgfpathcurveto{\pgfqpoint{-0.008840in}{0.033333in}}{\pgfqpoint{-0.017319in}{0.029821in}}{\pgfqpoint{-0.023570in}{0.023570in}}%
\pgfpathcurveto{\pgfqpoint{-0.029821in}{0.017319in}}{\pgfqpoint{-0.033333in}{0.008840in}}{\pgfqpoint{-0.033333in}{0.000000in}}%
\pgfpathcurveto{\pgfqpoint{-0.033333in}{-0.008840in}}{\pgfqpoint{-0.029821in}{-0.017319in}}{\pgfqpoint{-0.023570in}{-0.023570in}}%
\pgfpathcurveto{\pgfqpoint{-0.017319in}{-0.029821in}}{\pgfqpoint{-0.008840in}{-0.033333in}}{\pgfqpoint{0.000000in}{-0.033333in}}%
\pgfpathlineto{\pgfqpoint{0.000000in}{-0.033333in}}%
\pgfpathclose%
\pgfusepath{stroke,fill}%
}%
\begin{pgfscope}%
\pgfsys@transformshift{2.463334in}{2.110898in}%
\pgfsys@useobject{currentmarker}{}%
\end{pgfscope}%
\end{pgfscope}%
\begin{pgfscope}%
\pgfpathrectangle{\pgfqpoint{0.765000in}{0.660000in}}{\pgfqpoint{4.620000in}{4.620000in}}%
\pgfusepath{clip}%
\pgfsetrectcap%
\pgfsetroundjoin%
\pgfsetlinewidth{1.204500pt}%
\definecolor{currentstroke}{rgb}{1.000000,0.576471,0.309804}%
\pgfsetstrokecolor{currentstroke}%
\pgfsetdash{}{0pt}%
\pgfpathmoveto{\pgfqpoint{2.545309in}{2.883535in}}%
\pgfusepath{stroke}%
\end{pgfscope}%
\begin{pgfscope}%
\pgfpathrectangle{\pgfqpoint{0.765000in}{0.660000in}}{\pgfqpoint{4.620000in}{4.620000in}}%
\pgfusepath{clip}%
\pgfsetbuttcap%
\pgfsetroundjoin%
\definecolor{currentfill}{rgb}{1.000000,0.576471,0.309804}%
\pgfsetfillcolor{currentfill}%
\pgfsetlinewidth{1.003750pt}%
\definecolor{currentstroke}{rgb}{1.000000,0.576471,0.309804}%
\pgfsetstrokecolor{currentstroke}%
\pgfsetdash{}{0pt}%
\pgfsys@defobject{currentmarker}{\pgfqpoint{-0.033333in}{-0.033333in}}{\pgfqpoint{0.033333in}{0.033333in}}{%
\pgfpathmoveto{\pgfqpoint{0.000000in}{-0.033333in}}%
\pgfpathcurveto{\pgfqpoint{0.008840in}{-0.033333in}}{\pgfqpoint{0.017319in}{-0.029821in}}{\pgfqpoint{0.023570in}{-0.023570in}}%
\pgfpathcurveto{\pgfqpoint{0.029821in}{-0.017319in}}{\pgfqpoint{0.033333in}{-0.008840in}}{\pgfqpoint{0.033333in}{0.000000in}}%
\pgfpathcurveto{\pgfqpoint{0.033333in}{0.008840in}}{\pgfqpoint{0.029821in}{0.017319in}}{\pgfqpoint{0.023570in}{0.023570in}}%
\pgfpathcurveto{\pgfqpoint{0.017319in}{0.029821in}}{\pgfqpoint{0.008840in}{0.033333in}}{\pgfqpoint{0.000000in}{0.033333in}}%
\pgfpathcurveto{\pgfqpoint{-0.008840in}{0.033333in}}{\pgfqpoint{-0.017319in}{0.029821in}}{\pgfqpoint{-0.023570in}{0.023570in}}%
\pgfpathcurveto{\pgfqpoint{-0.029821in}{0.017319in}}{\pgfqpoint{-0.033333in}{0.008840in}}{\pgfqpoint{-0.033333in}{0.000000in}}%
\pgfpathcurveto{\pgfqpoint{-0.033333in}{-0.008840in}}{\pgfqpoint{-0.029821in}{-0.017319in}}{\pgfqpoint{-0.023570in}{-0.023570in}}%
\pgfpathcurveto{\pgfqpoint{-0.017319in}{-0.029821in}}{\pgfqpoint{-0.008840in}{-0.033333in}}{\pgfqpoint{0.000000in}{-0.033333in}}%
\pgfpathlineto{\pgfqpoint{0.000000in}{-0.033333in}}%
\pgfpathclose%
\pgfusepath{stroke,fill}%
}%
\begin{pgfscope}%
\pgfsys@transformshift{2.545309in}{2.883535in}%
\pgfsys@useobject{currentmarker}{}%
\end{pgfscope}%
\end{pgfscope}%
\begin{pgfscope}%
\pgfpathrectangle{\pgfqpoint{0.765000in}{0.660000in}}{\pgfqpoint{4.620000in}{4.620000in}}%
\pgfusepath{clip}%
\pgfsetrectcap%
\pgfsetroundjoin%
\pgfsetlinewidth{1.204500pt}%
\definecolor{currentstroke}{rgb}{0.176471,0.192157,0.258824}%
\pgfsetstrokecolor{currentstroke}%
\pgfsetdash{}{0pt}%
\pgfpathmoveto{\pgfqpoint{2.445913in}{3.857773in}}%
\pgfusepath{stroke}%
\end{pgfscope}%
\begin{pgfscope}%
\pgfpathrectangle{\pgfqpoint{0.765000in}{0.660000in}}{\pgfqpoint{4.620000in}{4.620000in}}%
\pgfusepath{clip}%
\pgfsetbuttcap%
\pgfsetmiterjoin%
\definecolor{currentfill}{rgb}{0.176471,0.192157,0.258824}%
\pgfsetfillcolor{currentfill}%
\pgfsetlinewidth{1.003750pt}%
\definecolor{currentstroke}{rgb}{0.176471,0.192157,0.258824}%
\pgfsetstrokecolor{currentstroke}%
\pgfsetdash{}{0pt}%
\pgfsys@defobject{currentmarker}{\pgfqpoint{-0.033333in}{-0.033333in}}{\pgfqpoint{0.033333in}{0.033333in}}{%
\pgfpathmoveto{\pgfqpoint{-0.000000in}{-0.033333in}}%
\pgfpathlineto{\pgfqpoint{0.033333in}{0.033333in}}%
\pgfpathlineto{\pgfqpoint{-0.033333in}{0.033333in}}%
\pgfpathlineto{\pgfqpoint{-0.000000in}{-0.033333in}}%
\pgfpathclose%
\pgfusepath{stroke,fill}%
}%
\begin{pgfscope}%
\pgfsys@transformshift{2.445913in}{3.857773in}%
\pgfsys@useobject{currentmarker}{}%
\end{pgfscope}%
\end{pgfscope}%
\begin{pgfscope}%
\pgfpathrectangle{\pgfqpoint{0.765000in}{0.660000in}}{\pgfqpoint{4.620000in}{4.620000in}}%
\pgfusepath{clip}%
\pgfsetrectcap%
\pgfsetroundjoin%
\pgfsetlinewidth{1.204500pt}%
\definecolor{currentstroke}{rgb}{0.176471,0.192157,0.258824}%
\pgfsetstrokecolor{currentstroke}%
\pgfsetdash{}{0pt}%
\pgfpathmoveto{\pgfqpoint{3.226560in}{4.288983in}}%
\pgfusepath{stroke}%
\end{pgfscope}%
\begin{pgfscope}%
\pgfpathrectangle{\pgfqpoint{0.765000in}{0.660000in}}{\pgfqpoint{4.620000in}{4.620000in}}%
\pgfusepath{clip}%
\pgfsetbuttcap%
\pgfsetmiterjoin%
\definecolor{currentfill}{rgb}{0.176471,0.192157,0.258824}%
\pgfsetfillcolor{currentfill}%
\pgfsetlinewidth{1.003750pt}%
\definecolor{currentstroke}{rgb}{0.176471,0.192157,0.258824}%
\pgfsetstrokecolor{currentstroke}%
\pgfsetdash{}{0pt}%
\pgfsys@defobject{currentmarker}{\pgfqpoint{-0.033333in}{-0.033333in}}{\pgfqpoint{0.033333in}{0.033333in}}{%
\pgfpathmoveto{\pgfqpoint{-0.000000in}{-0.033333in}}%
\pgfpathlineto{\pgfqpoint{0.033333in}{0.033333in}}%
\pgfpathlineto{\pgfqpoint{-0.033333in}{0.033333in}}%
\pgfpathlineto{\pgfqpoint{-0.000000in}{-0.033333in}}%
\pgfpathclose%
\pgfusepath{stroke,fill}%
}%
\begin{pgfscope}%
\pgfsys@transformshift{3.226560in}{4.288983in}%
\pgfsys@useobject{currentmarker}{}%
\end{pgfscope}%
\end{pgfscope}%
\begin{pgfscope}%
\pgfpathrectangle{\pgfqpoint{0.765000in}{0.660000in}}{\pgfqpoint{4.620000in}{4.620000in}}%
\pgfusepath{clip}%
\pgfsetrectcap%
\pgfsetroundjoin%
\pgfsetlinewidth{1.204500pt}%
\definecolor{currentstroke}{rgb}{0.176471,0.192157,0.258824}%
\pgfsetstrokecolor{currentstroke}%
\pgfsetdash{}{0pt}%
\pgfpathmoveto{\pgfqpoint{3.645300in}{3.925666in}}%
\pgfusepath{stroke}%
\end{pgfscope}%
\begin{pgfscope}%
\pgfpathrectangle{\pgfqpoint{0.765000in}{0.660000in}}{\pgfqpoint{4.620000in}{4.620000in}}%
\pgfusepath{clip}%
\pgfsetbuttcap%
\pgfsetmiterjoin%
\definecolor{currentfill}{rgb}{0.176471,0.192157,0.258824}%
\pgfsetfillcolor{currentfill}%
\pgfsetlinewidth{1.003750pt}%
\definecolor{currentstroke}{rgb}{0.176471,0.192157,0.258824}%
\pgfsetstrokecolor{currentstroke}%
\pgfsetdash{}{0pt}%
\pgfsys@defobject{currentmarker}{\pgfqpoint{-0.033333in}{-0.033333in}}{\pgfqpoint{0.033333in}{0.033333in}}{%
\pgfpathmoveto{\pgfqpoint{-0.000000in}{-0.033333in}}%
\pgfpathlineto{\pgfqpoint{0.033333in}{0.033333in}}%
\pgfpathlineto{\pgfqpoint{-0.033333in}{0.033333in}}%
\pgfpathlineto{\pgfqpoint{-0.000000in}{-0.033333in}}%
\pgfpathclose%
\pgfusepath{stroke,fill}%
}%
\begin{pgfscope}%
\pgfsys@transformshift{3.645300in}{3.925666in}%
\pgfsys@useobject{currentmarker}{}%
\end{pgfscope}%
\end{pgfscope}%
\begin{pgfscope}%
\pgfpathrectangle{\pgfqpoint{0.765000in}{0.660000in}}{\pgfqpoint{4.620000in}{4.620000in}}%
\pgfusepath{clip}%
\pgfsetrectcap%
\pgfsetroundjoin%
\pgfsetlinewidth{1.204500pt}%
\definecolor{currentstroke}{rgb}{0.176471,0.192157,0.258824}%
\pgfsetstrokecolor{currentstroke}%
\pgfsetdash{}{0pt}%
\pgfpathmoveto{\pgfqpoint{3.565119in}{4.208069in}}%
\pgfusepath{stroke}%
\end{pgfscope}%
\begin{pgfscope}%
\pgfpathrectangle{\pgfqpoint{0.765000in}{0.660000in}}{\pgfqpoint{4.620000in}{4.620000in}}%
\pgfusepath{clip}%
\pgfsetbuttcap%
\pgfsetmiterjoin%
\definecolor{currentfill}{rgb}{0.176471,0.192157,0.258824}%
\pgfsetfillcolor{currentfill}%
\pgfsetlinewidth{1.003750pt}%
\definecolor{currentstroke}{rgb}{0.176471,0.192157,0.258824}%
\pgfsetstrokecolor{currentstroke}%
\pgfsetdash{}{0pt}%
\pgfsys@defobject{currentmarker}{\pgfqpoint{-0.033333in}{-0.033333in}}{\pgfqpoint{0.033333in}{0.033333in}}{%
\pgfpathmoveto{\pgfqpoint{-0.000000in}{-0.033333in}}%
\pgfpathlineto{\pgfqpoint{0.033333in}{0.033333in}}%
\pgfpathlineto{\pgfqpoint{-0.033333in}{0.033333in}}%
\pgfpathlineto{\pgfqpoint{-0.000000in}{-0.033333in}}%
\pgfpathclose%
\pgfusepath{stroke,fill}%
}%
\begin{pgfscope}%
\pgfsys@transformshift{3.565119in}{4.208069in}%
\pgfsys@useobject{currentmarker}{}%
\end{pgfscope}%
\end{pgfscope}%
\begin{pgfscope}%
\pgfpathrectangle{\pgfqpoint{0.765000in}{0.660000in}}{\pgfqpoint{4.620000in}{4.620000in}}%
\pgfusepath{clip}%
\pgfsetrectcap%
\pgfsetroundjoin%
\pgfsetlinewidth{1.204500pt}%
\definecolor{currentstroke}{rgb}{0.176471,0.192157,0.258824}%
\pgfsetstrokecolor{currentstroke}%
\pgfsetdash{}{0pt}%
\pgfpathmoveto{\pgfqpoint{2.314129in}{4.326896in}}%
\pgfusepath{stroke}%
\end{pgfscope}%
\begin{pgfscope}%
\pgfpathrectangle{\pgfqpoint{0.765000in}{0.660000in}}{\pgfqpoint{4.620000in}{4.620000in}}%
\pgfusepath{clip}%
\pgfsetbuttcap%
\pgfsetmiterjoin%
\definecolor{currentfill}{rgb}{0.176471,0.192157,0.258824}%
\pgfsetfillcolor{currentfill}%
\pgfsetlinewidth{1.003750pt}%
\definecolor{currentstroke}{rgb}{0.176471,0.192157,0.258824}%
\pgfsetstrokecolor{currentstroke}%
\pgfsetdash{}{0pt}%
\pgfsys@defobject{currentmarker}{\pgfqpoint{-0.033333in}{-0.033333in}}{\pgfqpoint{0.033333in}{0.033333in}}{%
\pgfpathmoveto{\pgfqpoint{-0.000000in}{-0.033333in}}%
\pgfpathlineto{\pgfqpoint{0.033333in}{0.033333in}}%
\pgfpathlineto{\pgfqpoint{-0.033333in}{0.033333in}}%
\pgfpathlineto{\pgfqpoint{-0.000000in}{-0.033333in}}%
\pgfpathclose%
\pgfusepath{stroke,fill}%
}%
\begin{pgfscope}%
\pgfsys@transformshift{2.314129in}{4.326896in}%
\pgfsys@useobject{currentmarker}{}%
\end{pgfscope}%
\end{pgfscope}%
\begin{pgfscope}%
\pgfpathrectangle{\pgfqpoint{0.765000in}{0.660000in}}{\pgfqpoint{4.620000in}{4.620000in}}%
\pgfusepath{clip}%
\pgfsetrectcap%
\pgfsetroundjoin%
\pgfsetlinewidth{1.204500pt}%
\definecolor{currentstroke}{rgb}{0.176471,0.192157,0.258824}%
\pgfsetstrokecolor{currentstroke}%
\pgfsetdash{}{0pt}%
\pgfpathmoveto{\pgfqpoint{2.715120in}{3.715530in}}%
\pgfusepath{stroke}%
\end{pgfscope}%
\begin{pgfscope}%
\pgfpathrectangle{\pgfqpoint{0.765000in}{0.660000in}}{\pgfqpoint{4.620000in}{4.620000in}}%
\pgfusepath{clip}%
\pgfsetbuttcap%
\pgfsetmiterjoin%
\definecolor{currentfill}{rgb}{0.176471,0.192157,0.258824}%
\pgfsetfillcolor{currentfill}%
\pgfsetlinewidth{1.003750pt}%
\definecolor{currentstroke}{rgb}{0.176471,0.192157,0.258824}%
\pgfsetstrokecolor{currentstroke}%
\pgfsetdash{}{0pt}%
\pgfsys@defobject{currentmarker}{\pgfqpoint{-0.033333in}{-0.033333in}}{\pgfqpoint{0.033333in}{0.033333in}}{%
\pgfpathmoveto{\pgfqpoint{-0.000000in}{-0.033333in}}%
\pgfpathlineto{\pgfqpoint{0.033333in}{0.033333in}}%
\pgfpathlineto{\pgfqpoint{-0.033333in}{0.033333in}}%
\pgfpathlineto{\pgfqpoint{-0.000000in}{-0.033333in}}%
\pgfpathclose%
\pgfusepath{stroke,fill}%
}%
\begin{pgfscope}%
\pgfsys@transformshift{2.715120in}{3.715530in}%
\pgfsys@useobject{currentmarker}{}%
\end{pgfscope}%
\end{pgfscope}%
\begin{pgfscope}%
\pgfpathrectangle{\pgfqpoint{0.765000in}{0.660000in}}{\pgfqpoint{4.620000in}{4.620000in}}%
\pgfusepath{clip}%
\pgfsetrectcap%
\pgfsetroundjoin%
\pgfsetlinewidth{1.204500pt}%
\definecolor{currentstroke}{rgb}{0.176471,0.192157,0.258824}%
\pgfsetstrokecolor{currentstroke}%
\pgfsetdash{}{0pt}%
\pgfpathmoveto{\pgfqpoint{3.256482in}{3.924918in}}%
\pgfusepath{stroke}%
\end{pgfscope}%
\begin{pgfscope}%
\pgfpathrectangle{\pgfqpoint{0.765000in}{0.660000in}}{\pgfqpoint{4.620000in}{4.620000in}}%
\pgfusepath{clip}%
\pgfsetbuttcap%
\pgfsetmiterjoin%
\definecolor{currentfill}{rgb}{0.176471,0.192157,0.258824}%
\pgfsetfillcolor{currentfill}%
\pgfsetlinewidth{1.003750pt}%
\definecolor{currentstroke}{rgb}{0.176471,0.192157,0.258824}%
\pgfsetstrokecolor{currentstroke}%
\pgfsetdash{}{0pt}%
\pgfsys@defobject{currentmarker}{\pgfqpoint{-0.033333in}{-0.033333in}}{\pgfqpoint{0.033333in}{0.033333in}}{%
\pgfpathmoveto{\pgfqpoint{-0.000000in}{-0.033333in}}%
\pgfpathlineto{\pgfqpoint{0.033333in}{0.033333in}}%
\pgfpathlineto{\pgfqpoint{-0.033333in}{0.033333in}}%
\pgfpathlineto{\pgfqpoint{-0.000000in}{-0.033333in}}%
\pgfpathclose%
\pgfusepath{stroke,fill}%
}%
\begin{pgfscope}%
\pgfsys@transformshift{3.256482in}{3.924918in}%
\pgfsys@useobject{currentmarker}{}%
\end{pgfscope}%
\end{pgfscope}%
\begin{pgfscope}%
\pgfpathrectangle{\pgfqpoint{0.765000in}{0.660000in}}{\pgfqpoint{4.620000in}{4.620000in}}%
\pgfusepath{clip}%
\pgfsetrectcap%
\pgfsetroundjoin%
\pgfsetlinewidth{1.204500pt}%
\definecolor{currentstroke}{rgb}{0.176471,0.192157,0.258824}%
\pgfsetstrokecolor{currentstroke}%
\pgfsetdash{}{0pt}%
\pgfpathmoveto{\pgfqpoint{3.671542in}{4.007023in}}%
\pgfusepath{stroke}%
\end{pgfscope}%
\begin{pgfscope}%
\pgfpathrectangle{\pgfqpoint{0.765000in}{0.660000in}}{\pgfqpoint{4.620000in}{4.620000in}}%
\pgfusepath{clip}%
\pgfsetbuttcap%
\pgfsetmiterjoin%
\definecolor{currentfill}{rgb}{0.176471,0.192157,0.258824}%
\pgfsetfillcolor{currentfill}%
\pgfsetlinewidth{1.003750pt}%
\definecolor{currentstroke}{rgb}{0.176471,0.192157,0.258824}%
\pgfsetstrokecolor{currentstroke}%
\pgfsetdash{}{0pt}%
\pgfsys@defobject{currentmarker}{\pgfqpoint{-0.033333in}{-0.033333in}}{\pgfqpoint{0.033333in}{0.033333in}}{%
\pgfpathmoveto{\pgfqpoint{-0.000000in}{-0.033333in}}%
\pgfpathlineto{\pgfqpoint{0.033333in}{0.033333in}}%
\pgfpathlineto{\pgfqpoint{-0.033333in}{0.033333in}}%
\pgfpathlineto{\pgfqpoint{-0.000000in}{-0.033333in}}%
\pgfpathclose%
\pgfusepath{stroke,fill}%
}%
\begin{pgfscope}%
\pgfsys@transformshift{3.671542in}{4.007023in}%
\pgfsys@useobject{currentmarker}{}%
\end{pgfscope}%
\end{pgfscope}%
\begin{pgfscope}%
\pgfpathrectangle{\pgfqpoint{0.765000in}{0.660000in}}{\pgfqpoint{4.620000in}{4.620000in}}%
\pgfusepath{clip}%
\pgfsetrectcap%
\pgfsetroundjoin%
\pgfsetlinewidth{1.204500pt}%
\definecolor{currentstroke}{rgb}{0.176471,0.192157,0.258824}%
\pgfsetstrokecolor{currentstroke}%
\pgfsetdash{}{0pt}%
\pgfpathmoveto{\pgfqpoint{2.034939in}{4.163909in}}%
\pgfusepath{stroke}%
\end{pgfscope}%
\begin{pgfscope}%
\pgfpathrectangle{\pgfqpoint{0.765000in}{0.660000in}}{\pgfqpoint{4.620000in}{4.620000in}}%
\pgfusepath{clip}%
\pgfsetbuttcap%
\pgfsetmiterjoin%
\definecolor{currentfill}{rgb}{0.176471,0.192157,0.258824}%
\pgfsetfillcolor{currentfill}%
\pgfsetlinewidth{1.003750pt}%
\definecolor{currentstroke}{rgb}{0.176471,0.192157,0.258824}%
\pgfsetstrokecolor{currentstroke}%
\pgfsetdash{}{0pt}%
\pgfsys@defobject{currentmarker}{\pgfqpoint{-0.033333in}{-0.033333in}}{\pgfqpoint{0.033333in}{0.033333in}}{%
\pgfpathmoveto{\pgfqpoint{-0.000000in}{-0.033333in}}%
\pgfpathlineto{\pgfqpoint{0.033333in}{0.033333in}}%
\pgfpathlineto{\pgfqpoint{-0.033333in}{0.033333in}}%
\pgfpathlineto{\pgfqpoint{-0.000000in}{-0.033333in}}%
\pgfpathclose%
\pgfusepath{stroke,fill}%
}%
\begin{pgfscope}%
\pgfsys@transformshift{2.034939in}{4.163909in}%
\pgfsys@useobject{currentmarker}{}%
\end{pgfscope}%
\end{pgfscope}%
\begin{pgfscope}%
\pgfpathrectangle{\pgfqpoint{0.765000in}{0.660000in}}{\pgfqpoint{4.620000in}{4.620000in}}%
\pgfusepath{clip}%
\pgfsetrectcap%
\pgfsetroundjoin%
\pgfsetlinewidth{1.204500pt}%
\definecolor{currentstroke}{rgb}{0.176471,0.192157,0.258824}%
\pgfsetstrokecolor{currentstroke}%
\pgfsetdash{}{0pt}%
\pgfpathmoveto{\pgfqpoint{3.482914in}{4.285486in}}%
\pgfusepath{stroke}%
\end{pgfscope}%
\begin{pgfscope}%
\pgfpathrectangle{\pgfqpoint{0.765000in}{0.660000in}}{\pgfqpoint{4.620000in}{4.620000in}}%
\pgfusepath{clip}%
\pgfsetbuttcap%
\pgfsetmiterjoin%
\definecolor{currentfill}{rgb}{0.176471,0.192157,0.258824}%
\pgfsetfillcolor{currentfill}%
\pgfsetlinewidth{1.003750pt}%
\definecolor{currentstroke}{rgb}{0.176471,0.192157,0.258824}%
\pgfsetstrokecolor{currentstroke}%
\pgfsetdash{}{0pt}%
\pgfsys@defobject{currentmarker}{\pgfqpoint{-0.033333in}{-0.033333in}}{\pgfqpoint{0.033333in}{0.033333in}}{%
\pgfpathmoveto{\pgfqpoint{-0.000000in}{-0.033333in}}%
\pgfpathlineto{\pgfqpoint{0.033333in}{0.033333in}}%
\pgfpathlineto{\pgfqpoint{-0.033333in}{0.033333in}}%
\pgfpathlineto{\pgfqpoint{-0.000000in}{-0.033333in}}%
\pgfpathclose%
\pgfusepath{stroke,fill}%
}%
\begin{pgfscope}%
\pgfsys@transformshift{3.482914in}{4.285486in}%
\pgfsys@useobject{currentmarker}{}%
\end{pgfscope}%
\end{pgfscope}%
\begin{pgfscope}%
\pgfpathrectangle{\pgfqpoint{0.765000in}{0.660000in}}{\pgfqpoint{4.620000in}{4.620000in}}%
\pgfusepath{clip}%
\pgfsetrectcap%
\pgfsetroundjoin%
\pgfsetlinewidth{1.204500pt}%
\definecolor{currentstroke}{rgb}{0.176471,0.192157,0.258824}%
\pgfsetstrokecolor{currentstroke}%
\pgfsetdash{}{0pt}%
\pgfpathmoveto{\pgfqpoint{4.341835in}{4.148689in}}%
\pgfusepath{stroke}%
\end{pgfscope}%
\begin{pgfscope}%
\pgfpathrectangle{\pgfqpoint{0.765000in}{0.660000in}}{\pgfqpoint{4.620000in}{4.620000in}}%
\pgfusepath{clip}%
\pgfsetbuttcap%
\pgfsetmiterjoin%
\definecolor{currentfill}{rgb}{0.176471,0.192157,0.258824}%
\pgfsetfillcolor{currentfill}%
\pgfsetlinewidth{1.003750pt}%
\definecolor{currentstroke}{rgb}{0.176471,0.192157,0.258824}%
\pgfsetstrokecolor{currentstroke}%
\pgfsetdash{}{0pt}%
\pgfsys@defobject{currentmarker}{\pgfqpoint{-0.033333in}{-0.033333in}}{\pgfqpoint{0.033333in}{0.033333in}}{%
\pgfpathmoveto{\pgfqpoint{-0.000000in}{-0.033333in}}%
\pgfpathlineto{\pgfqpoint{0.033333in}{0.033333in}}%
\pgfpathlineto{\pgfqpoint{-0.033333in}{0.033333in}}%
\pgfpathlineto{\pgfqpoint{-0.000000in}{-0.033333in}}%
\pgfpathclose%
\pgfusepath{stroke,fill}%
}%
\begin{pgfscope}%
\pgfsys@transformshift{4.341835in}{4.148689in}%
\pgfsys@useobject{currentmarker}{}%
\end{pgfscope}%
\end{pgfscope}%
\begin{pgfscope}%
\pgfpathrectangle{\pgfqpoint{0.765000in}{0.660000in}}{\pgfqpoint{4.620000in}{4.620000in}}%
\pgfusepath{clip}%
\pgfsetrectcap%
\pgfsetroundjoin%
\pgfsetlinewidth{1.204500pt}%
\definecolor{currentstroke}{rgb}{0.176471,0.192157,0.258824}%
\pgfsetstrokecolor{currentstroke}%
\pgfsetdash{}{0pt}%
\pgfpathmoveto{\pgfqpoint{3.842589in}{4.252904in}}%
\pgfusepath{stroke}%
\end{pgfscope}%
\begin{pgfscope}%
\pgfpathrectangle{\pgfqpoint{0.765000in}{0.660000in}}{\pgfqpoint{4.620000in}{4.620000in}}%
\pgfusepath{clip}%
\pgfsetbuttcap%
\pgfsetmiterjoin%
\definecolor{currentfill}{rgb}{0.176471,0.192157,0.258824}%
\pgfsetfillcolor{currentfill}%
\pgfsetlinewidth{1.003750pt}%
\definecolor{currentstroke}{rgb}{0.176471,0.192157,0.258824}%
\pgfsetstrokecolor{currentstroke}%
\pgfsetdash{}{0pt}%
\pgfsys@defobject{currentmarker}{\pgfqpoint{-0.033333in}{-0.033333in}}{\pgfqpoint{0.033333in}{0.033333in}}{%
\pgfpathmoveto{\pgfqpoint{-0.000000in}{-0.033333in}}%
\pgfpathlineto{\pgfqpoint{0.033333in}{0.033333in}}%
\pgfpathlineto{\pgfqpoint{-0.033333in}{0.033333in}}%
\pgfpathlineto{\pgfqpoint{-0.000000in}{-0.033333in}}%
\pgfpathclose%
\pgfusepath{stroke,fill}%
}%
\begin{pgfscope}%
\pgfsys@transformshift{3.842589in}{4.252904in}%
\pgfsys@useobject{currentmarker}{}%
\end{pgfscope}%
\end{pgfscope}%
\begin{pgfscope}%
\pgfpathrectangle{\pgfqpoint{0.765000in}{0.660000in}}{\pgfqpoint{4.620000in}{4.620000in}}%
\pgfusepath{clip}%
\pgfsetrectcap%
\pgfsetroundjoin%
\pgfsetlinewidth{1.204500pt}%
\definecolor{currentstroke}{rgb}{0.176471,0.192157,0.258824}%
\pgfsetstrokecolor{currentstroke}%
\pgfsetdash{}{0pt}%
\pgfpathmoveto{\pgfqpoint{2.632148in}{4.364378in}}%
\pgfusepath{stroke}%
\end{pgfscope}%
\begin{pgfscope}%
\pgfpathrectangle{\pgfqpoint{0.765000in}{0.660000in}}{\pgfqpoint{4.620000in}{4.620000in}}%
\pgfusepath{clip}%
\pgfsetbuttcap%
\pgfsetmiterjoin%
\definecolor{currentfill}{rgb}{0.176471,0.192157,0.258824}%
\pgfsetfillcolor{currentfill}%
\pgfsetlinewidth{1.003750pt}%
\definecolor{currentstroke}{rgb}{0.176471,0.192157,0.258824}%
\pgfsetstrokecolor{currentstroke}%
\pgfsetdash{}{0pt}%
\pgfsys@defobject{currentmarker}{\pgfqpoint{-0.033333in}{-0.033333in}}{\pgfqpoint{0.033333in}{0.033333in}}{%
\pgfpathmoveto{\pgfqpoint{-0.000000in}{-0.033333in}}%
\pgfpathlineto{\pgfqpoint{0.033333in}{0.033333in}}%
\pgfpathlineto{\pgfqpoint{-0.033333in}{0.033333in}}%
\pgfpathlineto{\pgfqpoint{-0.000000in}{-0.033333in}}%
\pgfpathclose%
\pgfusepath{stroke,fill}%
}%
\begin{pgfscope}%
\pgfsys@transformshift{2.632148in}{4.364378in}%
\pgfsys@useobject{currentmarker}{}%
\end{pgfscope}%
\end{pgfscope}%
\begin{pgfscope}%
\pgfpathrectangle{\pgfqpoint{0.765000in}{0.660000in}}{\pgfqpoint{4.620000in}{4.620000in}}%
\pgfusepath{clip}%
\pgfsetrectcap%
\pgfsetroundjoin%
\pgfsetlinewidth{1.204500pt}%
\definecolor{currentstroke}{rgb}{0.176471,0.192157,0.258824}%
\pgfsetstrokecolor{currentstroke}%
\pgfsetdash{}{0pt}%
\pgfpathmoveto{\pgfqpoint{3.224847in}{3.848709in}}%
\pgfusepath{stroke}%
\end{pgfscope}%
\begin{pgfscope}%
\pgfpathrectangle{\pgfqpoint{0.765000in}{0.660000in}}{\pgfqpoint{4.620000in}{4.620000in}}%
\pgfusepath{clip}%
\pgfsetbuttcap%
\pgfsetmiterjoin%
\definecolor{currentfill}{rgb}{0.176471,0.192157,0.258824}%
\pgfsetfillcolor{currentfill}%
\pgfsetlinewidth{1.003750pt}%
\definecolor{currentstroke}{rgb}{0.176471,0.192157,0.258824}%
\pgfsetstrokecolor{currentstroke}%
\pgfsetdash{}{0pt}%
\pgfsys@defobject{currentmarker}{\pgfqpoint{-0.033333in}{-0.033333in}}{\pgfqpoint{0.033333in}{0.033333in}}{%
\pgfpathmoveto{\pgfqpoint{-0.000000in}{-0.033333in}}%
\pgfpathlineto{\pgfqpoint{0.033333in}{0.033333in}}%
\pgfpathlineto{\pgfqpoint{-0.033333in}{0.033333in}}%
\pgfpathlineto{\pgfqpoint{-0.000000in}{-0.033333in}}%
\pgfpathclose%
\pgfusepath{stroke,fill}%
}%
\begin{pgfscope}%
\pgfsys@transformshift{3.224847in}{3.848709in}%
\pgfsys@useobject{currentmarker}{}%
\end{pgfscope}%
\end{pgfscope}%
\begin{pgfscope}%
\pgfpathrectangle{\pgfqpoint{0.765000in}{0.660000in}}{\pgfqpoint{4.620000in}{4.620000in}}%
\pgfusepath{clip}%
\pgfsetrectcap%
\pgfsetroundjoin%
\pgfsetlinewidth{1.204500pt}%
\definecolor{currentstroke}{rgb}{0.176471,0.192157,0.258824}%
\pgfsetstrokecolor{currentstroke}%
\pgfsetdash{}{0pt}%
\pgfpathmoveto{\pgfqpoint{2.419305in}{4.017742in}}%
\pgfusepath{stroke}%
\end{pgfscope}%
\begin{pgfscope}%
\pgfpathrectangle{\pgfqpoint{0.765000in}{0.660000in}}{\pgfqpoint{4.620000in}{4.620000in}}%
\pgfusepath{clip}%
\pgfsetbuttcap%
\pgfsetmiterjoin%
\definecolor{currentfill}{rgb}{0.176471,0.192157,0.258824}%
\pgfsetfillcolor{currentfill}%
\pgfsetlinewidth{1.003750pt}%
\definecolor{currentstroke}{rgb}{0.176471,0.192157,0.258824}%
\pgfsetstrokecolor{currentstroke}%
\pgfsetdash{}{0pt}%
\pgfsys@defobject{currentmarker}{\pgfqpoint{-0.033333in}{-0.033333in}}{\pgfqpoint{0.033333in}{0.033333in}}{%
\pgfpathmoveto{\pgfqpoint{-0.000000in}{-0.033333in}}%
\pgfpathlineto{\pgfqpoint{0.033333in}{0.033333in}}%
\pgfpathlineto{\pgfqpoint{-0.033333in}{0.033333in}}%
\pgfpathlineto{\pgfqpoint{-0.000000in}{-0.033333in}}%
\pgfpathclose%
\pgfusepath{stroke,fill}%
}%
\begin{pgfscope}%
\pgfsys@transformshift{2.419305in}{4.017742in}%
\pgfsys@useobject{currentmarker}{}%
\end{pgfscope}%
\end{pgfscope}%
\begin{pgfscope}%
\pgfpathrectangle{\pgfqpoint{0.765000in}{0.660000in}}{\pgfqpoint{4.620000in}{4.620000in}}%
\pgfusepath{clip}%
\pgfsetrectcap%
\pgfsetroundjoin%
\pgfsetlinewidth{1.204500pt}%
\definecolor{currentstroke}{rgb}{0.176471,0.192157,0.258824}%
\pgfsetstrokecolor{currentstroke}%
\pgfsetdash{}{0pt}%
\pgfpathmoveto{\pgfqpoint{3.085299in}{3.552890in}}%
\pgfusepath{stroke}%
\end{pgfscope}%
\begin{pgfscope}%
\pgfpathrectangle{\pgfqpoint{0.765000in}{0.660000in}}{\pgfqpoint{4.620000in}{4.620000in}}%
\pgfusepath{clip}%
\pgfsetbuttcap%
\pgfsetmiterjoin%
\definecolor{currentfill}{rgb}{0.176471,0.192157,0.258824}%
\pgfsetfillcolor{currentfill}%
\pgfsetlinewidth{1.003750pt}%
\definecolor{currentstroke}{rgb}{0.176471,0.192157,0.258824}%
\pgfsetstrokecolor{currentstroke}%
\pgfsetdash{}{0pt}%
\pgfsys@defobject{currentmarker}{\pgfqpoint{-0.033333in}{-0.033333in}}{\pgfqpoint{0.033333in}{0.033333in}}{%
\pgfpathmoveto{\pgfqpoint{-0.000000in}{-0.033333in}}%
\pgfpathlineto{\pgfqpoint{0.033333in}{0.033333in}}%
\pgfpathlineto{\pgfqpoint{-0.033333in}{0.033333in}}%
\pgfpathlineto{\pgfqpoint{-0.000000in}{-0.033333in}}%
\pgfpathclose%
\pgfusepath{stroke,fill}%
}%
\begin{pgfscope}%
\pgfsys@transformshift{3.085299in}{3.552890in}%
\pgfsys@useobject{currentmarker}{}%
\end{pgfscope}%
\end{pgfscope}%
\begin{pgfscope}%
\pgfpathrectangle{\pgfqpoint{0.765000in}{0.660000in}}{\pgfqpoint{4.620000in}{4.620000in}}%
\pgfusepath{clip}%
\pgfsetrectcap%
\pgfsetroundjoin%
\pgfsetlinewidth{1.204500pt}%
\definecolor{currentstroke}{rgb}{0.176471,0.192157,0.258824}%
\pgfsetstrokecolor{currentstroke}%
\pgfsetdash{}{0pt}%
\pgfpathmoveto{\pgfqpoint{2.943357in}{4.556050in}}%
\pgfusepath{stroke}%
\end{pgfscope}%
\begin{pgfscope}%
\pgfpathrectangle{\pgfqpoint{0.765000in}{0.660000in}}{\pgfqpoint{4.620000in}{4.620000in}}%
\pgfusepath{clip}%
\pgfsetbuttcap%
\pgfsetmiterjoin%
\definecolor{currentfill}{rgb}{0.176471,0.192157,0.258824}%
\pgfsetfillcolor{currentfill}%
\pgfsetlinewidth{1.003750pt}%
\definecolor{currentstroke}{rgb}{0.176471,0.192157,0.258824}%
\pgfsetstrokecolor{currentstroke}%
\pgfsetdash{}{0pt}%
\pgfsys@defobject{currentmarker}{\pgfqpoint{-0.033333in}{-0.033333in}}{\pgfqpoint{0.033333in}{0.033333in}}{%
\pgfpathmoveto{\pgfqpoint{-0.000000in}{-0.033333in}}%
\pgfpathlineto{\pgfqpoint{0.033333in}{0.033333in}}%
\pgfpathlineto{\pgfqpoint{-0.033333in}{0.033333in}}%
\pgfpathlineto{\pgfqpoint{-0.000000in}{-0.033333in}}%
\pgfpathclose%
\pgfusepath{stroke,fill}%
}%
\begin{pgfscope}%
\pgfsys@transformshift{2.943357in}{4.556050in}%
\pgfsys@useobject{currentmarker}{}%
\end{pgfscope}%
\end{pgfscope}%
\begin{pgfscope}%
\pgfpathrectangle{\pgfqpoint{0.765000in}{0.660000in}}{\pgfqpoint{4.620000in}{4.620000in}}%
\pgfusepath{clip}%
\pgfsetrectcap%
\pgfsetroundjoin%
\pgfsetlinewidth{1.204500pt}%
\definecolor{currentstroke}{rgb}{0.176471,0.192157,0.258824}%
\pgfsetstrokecolor{currentstroke}%
\pgfsetdash{}{0pt}%
\pgfpathmoveto{\pgfqpoint{3.263445in}{4.456878in}}%
\pgfusepath{stroke}%
\end{pgfscope}%
\begin{pgfscope}%
\pgfpathrectangle{\pgfqpoint{0.765000in}{0.660000in}}{\pgfqpoint{4.620000in}{4.620000in}}%
\pgfusepath{clip}%
\pgfsetbuttcap%
\pgfsetmiterjoin%
\definecolor{currentfill}{rgb}{0.176471,0.192157,0.258824}%
\pgfsetfillcolor{currentfill}%
\pgfsetlinewidth{1.003750pt}%
\definecolor{currentstroke}{rgb}{0.176471,0.192157,0.258824}%
\pgfsetstrokecolor{currentstroke}%
\pgfsetdash{}{0pt}%
\pgfsys@defobject{currentmarker}{\pgfqpoint{-0.033333in}{-0.033333in}}{\pgfqpoint{0.033333in}{0.033333in}}{%
\pgfpathmoveto{\pgfqpoint{-0.000000in}{-0.033333in}}%
\pgfpathlineto{\pgfqpoint{0.033333in}{0.033333in}}%
\pgfpathlineto{\pgfqpoint{-0.033333in}{0.033333in}}%
\pgfpathlineto{\pgfqpoint{-0.000000in}{-0.033333in}}%
\pgfpathclose%
\pgfusepath{stroke,fill}%
}%
\begin{pgfscope}%
\pgfsys@transformshift{3.263445in}{4.456878in}%
\pgfsys@useobject{currentmarker}{}%
\end{pgfscope}%
\end{pgfscope}%
\begin{pgfscope}%
\pgfpathrectangle{\pgfqpoint{0.765000in}{0.660000in}}{\pgfqpoint{4.620000in}{4.620000in}}%
\pgfusepath{clip}%
\pgfsetrectcap%
\pgfsetroundjoin%
\pgfsetlinewidth{1.204500pt}%
\definecolor{currentstroke}{rgb}{0.176471,0.192157,0.258824}%
\pgfsetstrokecolor{currentstroke}%
\pgfsetdash{}{0pt}%
\pgfpathmoveto{\pgfqpoint{2.504106in}{3.839966in}}%
\pgfusepath{stroke}%
\end{pgfscope}%
\begin{pgfscope}%
\pgfpathrectangle{\pgfqpoint{0.765000in}{0.660000in}}{\pgfqpoint{4.620000in}{4.620000in}}%
\pgfusepath{clip}%
\pgfsetbuttcap%
\pgfsetmiterjoin%
\definecolor{currentfill}{rgb}{0.176471,0.192157,0.258824}%
\pgfsetfillcolor{currentfill}%
\pgfsetlinewidth{1.003750pt}%
\definecolor{currentstroke}{rgb}{0.176471,0.192157,0.258824}%
\pgfsetstrokecolor{currentstroke}%
\pgfsetdash{}{0pt}%
\pgfsys@defobject{currentmarker}{\pgfqpoint{-0.033333in}{-0.033333in}}{\pgfqpoint{0.033333in}{0.033333in}}{%
\pgfpathmoveto{\pgfqpoint{-0.000000in}{-0.033333in}}%
\pgfpathlineto{\pgfqpoint{0.033333in}{0.033333in}}%
\pgfpathlineto{\pgfqpoint{-0.033333in}{0.033333in}}%
\pgfpathlineto{\pgfqpoint{-0.000000in}{-0.033333in}}%
\pgfpathclose%
\pgfusepath{stroke,fill}%
}%
\begin{pgfscope}%
\pgfsys@transformshift{2.504106in}{3.839966in}%
\pgfsys@useobject{currentmarker}{}%
\end{pgfscope}%
\end{pgfscope}%
\begin{pgfscope}%
\pgfpathrectangle{\pgfqpoint{0.765000in}{0.660000in}}{\pgfqpoint{4.620000in}{4.620000in}}%
\pgfusepath{clip}%
\pgfsetrectcap%
\pgfsetroundjoin%
\pgfsetlinewidth{1.204500pt}%
\definecolor{currentstroke}{rgb}{0.176471,0.192157,0.258824}%
\pgfsetstrokecolor{currentstroke}%
\pgfsetdash{}{0pt}%
\pgfpathmoveto{\pgfqpoint{3.067526in}{3.911304in}}%
\pgfusepath{stroke}%
\end{pgfscope}%
\begin{pgfscope}%
\pgfpathrectangle{\pgfqpoint{0.765000in}{0.660000in}}{\pgfqpoint{4.620000in}{4.620000in}}%
\pgfusepath{clip}%
\pgfsetbuttcap%
\pgfsetmiterjoin%
\definecolor{currentfill}{rgb}{0.176471,0.192157,0.258824}%
\pgfsetfillcolor{currentfill}%
\pgfsetlinewidth{1.003750pt}%
\definecolor{currentstroke}{rgb}{0.176471,0.192157,0.258824}%
\pgfsetstrokecolor{currentstroke}%
\pgfsetdash{}{0pt}%
\pgfsys@defobject{currentmarker}{\pgfqpoint{-0.033333in}{-0.033333in}}{\pgfqpoint{0.033333in}{0.033333in}}{%
\pgfpathmoveto{\pgfqpoint{-0.000000in}{-0.033333in}}%
\pgfpathlineto{\pgfqpoint{0.033333in}{0.033333in}}%
\pgfpathlineto{\pgfqpoint{-0.033333in}{0.033333in}}%
\pgfpathlineto{\pgfqpoint{-0.000000in}{-0.033333in}}%
\pgfpathclose%
\pgfusepath{stroke,fill}%
}%
\begin{pgfscope}%
\pgfsys@transformshift{3.067526in}{3.911304in}%
\pgfsys@useobject{currentmarker}{}%
\end{pgfscope}%
\end{pgfscope}%
\begin{pgfscope}%
\pgfpathrectangle{\pgfqpoint{0.765000in}{0.660000in}}{\pgfqpoint{4.620000in}{4.620000in}}%
\pgfusepath{clip}%
\pgfsetrectcap%
\pgfsetroundjoin%
\pgfsetlinewidth{1.204500pt}%
\definecolor{currentstroke}{rgb}{0.176471,0.192157,0.258824}%
\pgfsetstrokecolor{currentstroke}%
\pgfsetdash{}{0pt}%
\pgfpathmoveto{\pgfqpoint{3.553038in}{4.011911in}}%
\pgfusepath{stroke}%
\end{pgfscope}%
\begin{pgfscope}%
\pgfpathrectangle{\pgfqpoint{0.765000in}{0.660000in}}{\pgfqpoint{4.620000in}{4.620000in}}%
\pgfusepath{clip}%
\pgfsetbuttcap%
\pgfsetmiterjoin%
\definecolor{currentfill}{rgb}{0.176471,0.192157,0.258824}%
\pgfsetfillcolor{currentfill}%
\pgfsetlinewidth{1.003750pt}%
\definecolor{currentstroke}{rgb}{0.176471,0.192157,0.258824}%
\pgfsetstrokecolor{currentstroke}%
\pgfsetdash{}{0pt}%
\pgfsys@defobject{currentmarker}{\pgfqpoint{-0.033333in}{-0.033333in}}{\pgfqpoint{0.033333in}{0.033333in}}{%
\pgfpathmoveto{\pgfqpoint{-0.000000in}{-0.033333in}}%
\pgfpathlineto{\pgfqpoint{0.033333in}{0.033333in}}%
\pgfpathlineto{\pgfqpoint{-0.033333in}{0.033333in}}%
\pgfpathlineto{\pgfqpoint{-0.000000in}{-0.033333in}}%
\pgfpathclose%
\pgfusepath{stroke,fill}%
}%
\begin{pgfscope}%
\pgfsys@transformshift{3.553038in}{4.011911in}%
\pgfsys@useobject{currentmarker}{}%
\end{pgfscope}%
\end{pgfscope}%
\begin{pgfscope}%
\pgfpathrectangle{\pgfqpoint{0.765000in}{0.660000in}}{\pgfqpoint{4.620000in}{4.620000in}}%
\pgfusepath{clip}%
\pgfsetrectcap%
\pgfsetroundjoin%
\pgfsetlinewidth{1.204500pt}%
\definecolor{currentstroke}{rgb}{0.176471,0.192157,0.258824}%
\pgfsetstrokecolor{currentstroke}%
\pgfsetdash{}{0pt}%
\pgfpathmoveto{\pgfqpoint{2.479674in}{4.346470in}}%
\pgfusepath{stroke}%
\end{pgfscope}%
\begin{pgfscope}%
\pgfpathrectangle{\pgfqpoint{0.765000in}{0.660000in}}{\pgfqpoint{4.620000in}{4.620000in}}%
\pgfusepath{clip}%
\pgfsetbuttcap%
\pgfsetmiterjoin%
\definecolor{currentfill}{rgb}{0.176471,0.192157,0.258824}%
\pgfsetfillcolor{currentfill}%
\pgfsetlinewidth{1.003750pt}%
\definecolor{currentstroke}{rgb}{0.176471,0.192157,0.258824}%
\pgfsetstrokecolor{currentstroke}%
\pgfsetdash{}{0pt}%
\pgfsys@defobject{currentmarker}{\pgfqpoint{-0.033333in}{-0.033333in}}{\pgfqpoint{0.033333in}{0.033333in}}{%
\pgfpathmoveto{\pgfqpoint{-0.000000in}{-0.033333in}}%
\pgfpathlineto{\pgfqpoint{0.033333in}{0.033333in}}%
\pgfpathlineto{\pgfqpoint{-0.033333in}{0.033333in}}%
\pgfpathlineto{\pgfqpoint{-0.000000in}{-0.033333in}}%
\pgfpathclose%
\pgfusepath{stroke,fill}%
}%
\begin{pgfscope}%
\pgfsys@transformshift{2.479674in}{4.346470in}%
\pgfsys@useobject{currentmarker}{}%
\end{pgfscope}%
\end{pgfscope}%
\begin{pgfscope}%
\pgfpathrectangle{\pgfqpoint{0.765000in}{0.660000in}}{\pgfqpoint{4.620000in}{4.620000in}}%
\pgfusepath{clip}%
\pgfsetrectcap%
\pgfsetroundjoin%
\pgfsetlinewidth{1.204500pt}%
\definecolor{currentstroke}{rgb}{0.176471,0.192157,0.258824}%
\pgfsetstrokecolor{currentstroke}%
\pgfsetdash{}{0pt}%
\pgfpathmoveto{\pgfqpoint{3.488893in}{3.897795in}}%
\pgfusepath{stroke}%
\end{pgfscope}%
\begin{pgfscope}%
\pgfpathrectangle{\pgfqpoint{0.765000in}{0.660000in}}{\pgfqpoint{4.620000in}{4.620000in}}%
\pgfusepath{clip}%
\pgfsetbuttcap%
\pgfsetmiterjoin%
\definecolor{currentfill}{rgb}{0.176471,0.192157,0.258824}%
\pgfsetfillcolor{currentfill}%
\pgfsetlinewidth{1.003750pt}%
\definecolor{currentstroke}{rgb}{0.176471,0.192157,0.258824}%
\pgfsetstrokecolor{currentstroke}%
\pgfsetdash{}{0pt}%
\pgfsys@defobject{currentmarker}{\pgfqpoint{-0.033333in}{-0.033333in}}{\pgfqpoint{0.033333in}{0.033333in}}{%
\pgfpathmoveto{\pgfqpoint{-0.000000in}{-0.033333in}}%
\pgfpathlineto{\pgfqpoint{0.033333in}{0.033333in}}%
\pgfpathlineto{\pgfqpoint{-0.033333in}{0.033333in}}%
\pgfpathlineto{\pgfqpoint{-0.000000in}{-0.033333in}}%
\pgfpathclose%
\pgfusepath{stroke,fill}%
}%
\begin{pgfscope}%
\pgfsys@transformshift{3.488893in}{3.897795in}%
\pgfsys@useobject{currentmarker}{}%
\end{pgfscope}%
\end{pgfscope}%
\begin{pgfscope}%
\pgfpathrectangle{\pgfqpoint{0.765000in}{0.660000in}}{\pgfqpoint{4.620000in}{4.620000in}}%
\pgfusepath{clip}%
\pgfsetrectcap%
\pgfsetroundjoin%
\pgfsetlinewidth{1.204500pt}%
\definecolor{currentstroke}{rgb}{0.176471,0.192157,0.258824}%
\pgfsetstrokecolor{currentstroke}%
\pgfsetdash{}{0pt}%
\pgfpathmoveto{\pgfqpoint{4.490319in}{4.344107in}}%
\pgfusepath{stroke}%
\end{pgfscope}%
\begin{pgfscope}%
\pgfpathrectangle{\pgfqpoint{0.765000in}{0.660000in}}{\pgfqpoint{4.620000in}{4.620000in}}%
\pgfusepath{clip}%
\pgfsetbuttcap%
\pgfsetmiterjoin%
\definecolor{currentfill}{rgb}{0.176471,0.192157,0.258824}%
\pgfsetfillcolor{currentfill}%
\pgfsetlinewidth{1.003750pt}%
\definecolor{currentstroke}{rgb}{0.176471,0.192157,0.258824}%
\pgfsetstrokecolor{currentstroke}%
\pgfsetdash{}{0pt}%
\pgfsys@defobject{currentmarker}{\pgfqpoint{-0.033333in}{-0.033333in}}{\pgfqpoint{0.033333in}{0.033333in}}{%
\pgfpathmoveto{\pgfqpoint{-0.000000in}{-0.033333in}}%
\pgfpathlineto{\pgfqpoint{0.033333in}{0.033333in}}%
\pgfpathlineto{\pgfqpoint{-0.033333in}{0.033333in}}%
\pgfpathlineto{\pgfqpoint{-0.000000in}{-0.033333in}}%
\pgfpathclose%
\pgfusepath{stroke,fill}%
}%
\begin{pgfscope}%
\pgfsys@transformshift{4.490319in}{4.344107in}%
\pgfsys@useobject{currentmarker}{}%
\end{pgfscope}%
\end{pgfscope}%
\begin{pgfscope}%
\pgfpathrectangle{\pgfqpoint{0.765000in}{0.660000in}}{\pgfqpoint{4.620000in}{4.620000in}}%
\pgfusepath{clip}%
\pgfsetrectcap%
\pgfsetroundjoin%
\pgfsetlinewidth{1.204500pt}%
\definecolor{currentstroke}{rgb}{0.176471,0.192157,0.258824}%
\pgfsetstrokecolor{currentstroke}%
\pgfsetdash{}{0pt}%
\pgfpathmoveto{\pgfqpoint{3.597551in}{4.452003in}}%
\pgfusepath{stroke}%
\end{pgfscope}%
\begin{pgfscope}%
\pgfpathrectangle{\pgfqpoint{0.765000in}{0.660000in}}{\pgfqpoint{4.620000in}{4.620000in}}%
\pgfusepath{clip}%
\pgfsetbuttcap%
\pgfsetmiterjoin%
\definecolor{currentfill}{rgb}{0.176471,0.192157,0.258824}%
\pgfsetfillcolor{currentfill}%
\pgfsetlinewidth{1.003750pt}%
\definecolor{currentstroke}{rgb}{0.176471,0.192157,0.258824}%
\pgfsetstrokecolor{currentstroke}%
\pgfsetdash{}{0pt}%
\pgfsys@defobject{currentmarker}{\pgfqpoint{-0.033333in}{-0.033333in}}{\pgfqpoint{0.033333in}{0.033333in}}{%
\pgfpathmoveto{\pgfqpoint{-0.000000in}{-0.033333in}}%
\pgfpathlineto{\pgfqpoint{0.033333in}{0.033333in}}%
\pgfpathlineto{\pgfqpoint{-0.033333in}{0.033333in}}%
\pgfpathlineto{\pgfqpoint{-0.000000in}{-0.033333in}}%
\pgfpathclose%
\pgfusepath{stroke,fill}%
}%
\begin{pgfscope}%
\pgfsys@transformshift{3.597551in}{4.452003in}%
\pgfsys@useobject{currentmarker}{}%
\end{pgfscope}%
\end{pgfscope}%
\begin{pgfscope}%
\pgfpathrectangle{\pgfqpoint{0.765000in}{0.660000in}}{\pgfqpoint{4.620000in}{4.620000in}}%
\pgfusepath{clip}%
\pgfsetrectcap%
\pgfsetroundjoin%
\pgfsetlinewidth{1.204500pt}%
\definecolor{currentstroke}{rgb}{0.176471,0.192157,0.258824}%
\pgfsetstrokecolor{currentstroke}%
\pgfsetdash{}{0pt}%
\pgfpathmoveto{\pgfqpoint{3.154702in}{4.156991in}}%
\pgfusepath{stroke}%
\end{pgfscope}%
\begin{pgfscope}%
\pgfpathrectangle{\pgfqpoint{0.765000in}{0.660000in}}{\pgfqpoint{4.620000in}{4.620000in}}%
\pgfusepath{clip}%
\pgfsetbuttcap%
\pgfsetmiterjoin%
\definecolor{currentfill}{rgb}{0.176471,0.192157,0.258824}%
\pgfsetfillcolor{currentfill}%
\pgfsetlinewidth{1.003750pt}%
\definecolor{currentstroke}{rgb}{0.176471,0.192157,0.258824}%
\pgfsetstrokecolor{currentstroke}%
\pgfsetdash{}{0pt}%
\pgfsys@defobject{currentmarker}{\pgfqpoint{-0.033333in}{-0.033333in}}{\pgfqpoint{0.033333in}{0.033333in}}{%
\pgfpathmoveto{\pgfqpoint{-0.000000in}{-0.033333in}}%
\pgfpathlineto{\pgfqpoint{0.033333in}{0.033333in}}%
\pgfpathlineto{\pgfqpoint{-0.033333in}{0.033333in}}%
\pgfpathlineto{\pgfqpoint{-0.000000in}{-0.033333in}}%
\pgfpathclose%
\pgfusepath{stroke,fill}%
}%
\begin{pgfscope}%
\pgfsys@transformshift{3.154702in}{4.156991in}%
\pgfsys@useobject{currentmarker}{}%
\end{pgfscope}%
\end{pgfscope}%
\begin{pgfscope}%
\pgfpathrectangle{\pgfqpoint{0.765000in}{0.660000in}}{\pgfqpoint{4.620000in}{4.620000in}}%
\pgfusepath{clip}%
\pgfsetrectcap%
\pgfsetroundjoin%
\pgfsetlinewidth{1.204500pt}%
\definecolor{currentstroke}{rgb}{0.176471,0.192157,0.258824}%
\pgfsetstrokecolor{currentstroke}%
\pgfsetdash{}{0pt}%
\pgfpathmoveto{\pgfqpoint{2.609028in}{4.040698in}}%
\pgfusepath{stroke}%
\end{pgfscope}%
\begin{pgfscope}%
\pgfpathrectangle{\pgfqpoint{0.765000in}{0.660000in}}{\pgfqpoint{4.620000in}{4.620000in}}%
\pgfusepath{clip}%
\pgfsetbuttcap%
\pgfsetmiterjoin%
\definecolor{currentfill}{rgb}{0.176471,0.192157,0.258824}%
\pgfsetfillcolor{currentfill}%
\pgfsetlinewidth{1.003750pt}%
\definecolor{currentstroke}{rgb}{0.176471,0.192157,0.258824}%
\pgfsetstrokecolor{currentstroke}%
\pgfsetdash{}{0pt}%
\pgfsys@defobject{currentmarker}{\pgfqpoint{-0.033333in}{-0.033333in}}{\pgfqpoint{0.033333in}{0.033333in}}{%
\pgfpathmoveto{\pgfqpoint{-0.000000in}{-0.033333in}}%
\pgfpathlineto{\pgfqpoint{0.033333in}{0.033333in}}%
\pgfpathlineto{\pgfqpoint{-0.033333in}{0.033333in}}%
\pgfpathlineto{\pgfqpoint{-0.000000in}{-0.033333in}}%
\pgfpathclose%
\pgfusepath{stroke,fill}%
}%
\begin{pgfscope}%
\pgfsys@transformshift{2.609028in}{4.040698in}%
\pgfsys@useobject{currentmarker}{}%
\end{pgfscope}%
\end{pgfscope}%
\begin{pgfscope}%
\pgfpathrectangle{\pgfqpoint{0.765000in}{0.660000in}}{\pgfqpoint{4.620000in}{4.620000in}}%
\pgfusepath{clip}%
\pgfsetrectcap%
\pgfsetroundjoin%
\pgfsetlinewidth{1.204500pt}%
\definecolor{currentstroke}{rgb}{0.176471,0.192157,0.258824}%
\pgfsetstrokecolor{currentstroke}%
\pgfsetdash{}{0pt}%
\pgfpathmoveto{\pgfqpoint{3.083766in}{3.624208in}}%
\pgfusepath{stroke}%
\end{pgfscope}%
\begin{pgfscope}%
\pgfpathrectangle{\pgfqpoint{0.765000in}{0.660000in}}{\pgfqpoint{4.620000in}{4.620000in}}%
\pgfusepath{clip}%
\pgfsetbuttcap%
\pgfsetmiterjoin%
\definecolor{currentfill}{rgb}{0.176471,0.192157,0.258824}%
\pgfsetfillcolor{currentfill}%
\pgfsetlinewidth{1.003750pt}%
\definecolor{currentstroke}{rgb}{0.176471,0.192157,0.258824}%
\pgfsetstrokecolor{currentstroke}%
\pgfsetdash{}{0pt}%
\pgfsys@defobject{currentmarker}{\pgfqpoint{-0.033333in}{-0.033333in}}{\pgfqpoint{0.033333in}{0.033333in}}{%
\pgfpathmoveto{\pgfqpoint{-0.000000in}{-0.033333in}}%
\pgfpathlineto{\pgfqpoint{0.033333in}{0.033333in}}%
\pgfpathlineto{\pgfqpoint{-0.033333in}{0.033333in}}%
\pgfpathlineto{\pgfqpoint{-0.000000in}{-0.033333in}}%
\pgfpathclose%
\pgfusepath{stroke,fill}%
}%
\begin{pgfscope}%
\pgfsys@transformshift{3.083766in}{3.624208in}%
\pgfsys@useobject{currentmarker}{}%
\end{pgfscope}%
\end{pgfscope}%
\begin{pgfscope}%
\pgfpathrectangle{\pgfqpoint{0.765000in}{0.660000in}}{\pgfqpoint{4.620000in}{4.620000in}}%
\pgfusepath{clip}%
\pgfsetrectcap%
\pgfsetroundjoin%
\pgfsetlinewidth{1.204500pt}%
\definecolor{currentstroke}{rgb}{0.176471,0.192157,0.258824}%
\pgfsetstrokecolor{currentstroke}%
\pgfsetdash{}{0pt}%
\pgfpathmoveto{\pgfqpoint{2.606826in}{4.409248in}}%
\pgfusepath{stroke}%
\end{pgfscope}%
\begin{pgfscope}%
\pgfpathrectangle{\pgfqpoint{0.765000in}{0.660000in}}{\pgfqpoint{4.620000in}{4.620000in}}%
\pgfusepath{clip}%
\pgfsetbuttcap%
\pgfsetmiterjoin%
\definecolor{currentfill}{rgb}{0.176471,0.192157,0.258824}%
\pgfsetfillcolor{currentfill}%
\pgfsetlinewidth{1.003750pt}%
\definecolor{currentstroke}{rgb}{0.176471,0.192157,0.258824}%
\pgfsetstrokecolor{currentstroke}%
\pgfsetdash{}{0pt}%
\pgfsys@defobject{currentmarker}{\pgfqpoint{-0.033333in}{-0.033333in}}{\pgfqpoint{0.033333in}{0.033333in}}{%
\pgfpathmoveto{\pgfqpoint{-0.000000in}{-0.033333in}}%
\pgfpathlineto{\pgfqpoint{0.033333in}{0.033333in}}%
\pgfpathlineto{\pgfqpoint{-0.033333in}{0.033333in}}%
\pgfpathlineto{\pgfqpoint{-0.000000in}{-0.033333in}}%
\pgfpathclose%
\pgfusepath{stroke,fill}%
}%
\begin{pgfscope}%
\pgfsys@transformshift{2.606826in}{4.409248in}%
\pgfsys@useobject{currentmarker}{}%
\end{pgfscope}%
\end{pgfscope}%
\begin{pgfscope}%
\pgfpathrectangle{\pgfqpoint{0.765000in}{0.660000in}}{\pgfqpoint{4.620000in}{4.620000in}}%
\pgfusepath{clip}%
\pgfsetrectcap%
\pgfsetroundjoin%
\pgfsetlinewidth{1.204500pt}%
\definecolor{currentstroke}{rgb}{0.176471,0.192157,0.258824}%
\pgfsetstrokecolor{currentstroke}%
\pgfsetdash{}{0pt}%
\pgfpathmoveto{\pgfqpoint{3.302237in}{4.151390in}}%
\pgfusepath{stroke}%
\end{pgfscope}%
\begin{pgfscope}%
\pgfpathrectangle{\pgfqpoint{0.765000in}{0.660000in}}{\pgfqpoint{4.620000in}{4.620000in}}%
\pgfusepath{clip}%
\pgfsetbuttcap%
\pgfsetmiterjoin%
\definecolor{currentfill}{rgb}{0.176471,0.192157,0.258824}%
\pgfsetfillcolor{currentfill}%
\pgfsetlinewidth{1.003750pt}%
\definecolor{currentstroke}{rgb}{0.176471,0.192157,0.258824}%
\pgfsetstrokecolor{currentstroke}%
\pgfsetdash{}{0pt}%
\pgfsys@defobject{currentmarker}{\pgfqpoint{-0.033333in}{-0.033333in}}{\pgfqpoint{0.033333in}{0.033333in}}{%
\pgfpathmoveto{\pgfqpoint{-0.000000in}{-0.033333in}}%
\pgfpathlineto{\pgfqpoint{0.033333in}{0.033333in}}%
\pgfpathlineto{\pgfqpoint{-0.033333in}{0.033333in}}%
\pgfpathlineto{\pgfqpoint{-0.000000in}{-0.033333in}}%
\pgfpathclose%
\pgfusepath{stroke,fill}%
}%
\begin{pgfscope}%
\pgfsys@transformshift{3.302237in}{4.151390in}%
\pgfsys@useobject{currentmarker}{}%
\end{pgfscope}%
\end{pgfscope}%
\begin{pgfscope}%
\pgfpathrectangle{\pgfqpoint{0.765000in}{0.660000in}}{\pgfqpoint{4.620000in}{4.620000in}}%
\pgfusepath{clip}%
\pgfsetrectcap%
\pgfsetroundjoin%
\pgfsetlinewidth{1.204500pt}%
\definecolor{currentstroke}{rgb}{0.000000,0.000000,0.000000}%
\pgfsetstrokecolor{currentstroke}%
\pgfsetdash{}{0pt}%
\pgfpathmoveto{\pgfqpoint{3.071405in}{3.123205in}}%
\pgfusepath{stroke}%
\end{pgfscope}%
\begin{pgfscope}%
\pgfpathrectangle{\pgfqpoint{0.765000in}{0.660000in}}{\pgfqpoint{4.620000in}{4.620000in}}%
\pgfusepath{clip}%
\pgfsetbuttcap%
\pgfsetroundjoin%
\definecolor{currentfill}{rgb}{0.000000,0.000000,0.000000}%
\pgfsetfillcolor{currentfill}%
\pgfsetlinewidth{1.003750pt}%
\definecolor{currentstroke}{rgb}{0.000000,0.000000,0.000000}%
\pgfsetstrokecolor{currentstroke}%
\pgfsetdash{}{0pt}%
\pgfsys@defobject{currentmarker}{\pgfqpoint{-0.016667in}{-0.016667in}}{\pgfqpoint{0.016667in}{0.016667in}}{%
\pgfpathmoveto{\pgfqpoint{0.000000in}{-0.016667in}}%
\pgfpathcurveto{\pgfqpoint{0.004420in}{-0.016667in}}{\pgfqpoint{0.008660in}{-0.014911in}}{\pgfqpoint{0.011785in}{-0.011785in}}%
\pgfpathcurveto{\pgfqpoint{0.014911in}{-0.008660in}}{\pgfqpoint{0.016667in}{-0.004420in}}{\pgfqpoint{0.016667in}{0.000000in}}%
\pgfpathcurveto{\pgfqpoint{0.016667in}{0.004420in}}{\pgfqpoint{0.014911in}{0.008660in}}{\pgfqpoint{0.011785in}{0.011785in}}%
\pgfpathcurveto{\pgfqpoint{0.008660in}{0.014911in}}{\pgfqpoint{0.004420in}{0.016667in}}{\pgfqpoint{0.000000in}{0.016667in}}%
\pgfpathcurveto{\pgfqpoint{-0.004420in}{0.016667in}}{\pgfqpoint{-0.008660in}{0.014911in}}{\pgfqpoint{-0.011785in}{0.011785in}}%
\pgfpathcurveto{\pgfqpoint{-0.014911in}{0.008660in}}{\pgfqpoint{-0.016667in}{0.004420in}}{\pgfqpoint{-0.016667in}{0.000000in}}%
\pgfpathcurveto{\pgfqpoint{-0.016667in}{-0.004420in}}{\pgfqpoint{-0.014911in}{-0.008660in}}{\pgfqpoint{-0.011785in}{-0.011785in}}%
\pgfpathcurveto{\pgfqpoint{-0.008660in}{-0.014911in}}{\pgfqpoint{-0.004420in}{-0.016667in}}{\pgfqpoint{0.000000in}{-0.016667in}}%
\pgfpathlineto{\pgfqpoint{0.000000in}{-0.016667in}}%
\pgfpathclose%
\pgfusepath{stroke,fill}%
}%
\begin{pgfscope}%
\pgfsys@transformshift{3.071405in}{3.123205in}%
\pgfsys@useobject{currentmarker}{}%
\end{pgfscope}%
\end{pgfscope}%
\begin{pgfscope}%
\pgfpathrectangle{\pgfqpoint{0.765000in}{0.660000in}}{\pgfqpoint{4.620000in}{4.620000in}}%
\pgfusepath{clip}%
\pgfsetbuttcap%
\pgfsetroundjoin%
\pgfsetlinewidth{1.204500pt}%
\definecolor{currentstroke}{rgb}{0.827451,0.827451,0.827451}%
\pgfsetstrokecolor{currentstroke}%
\pgfsetdash{{1.200000pt}{1.980000pt}}{0.000000pt}%
\pgfpathmoveto{\pgfqpoint{3.071405in}{3.123205in}}%
\pgfpathlineto{\pgfqpoint{3.071405in}{0.650000in}}%
\pgfusepath{stroke}%
\end{pgfscope}%
\begin{pgfscope}%
\pgfpathrectangle{\pgfqpoint{0.765000in}{0.660000in}}{\pgfqpoint{4.620000in}{4.620000in}}%
\pgfusepath{clip}%
\pgfsetrectcap%
\pgfsetroundjoin%
\pgfsetlinewidth{1.204500pt}%
\definecolor{currentstroke}{rgb}{0.000000,0.000000,0.000000}%
\pgfsetstrokecolor{currentstroke}%
\pgfsetdash{}{0pt}%
\pgfpathmoveto{\pgfqpoint{3.071405in}{3.123205in}}%
\pgfpathlineto{\pgfqpoint{0.755000in}{4.208811in}}%
\pgfusepath{stroke}%
\end{pgfscope}%
\begin{pgfscope}%
\pgfpathrectangle{\pgfqpoint{0.765000in}{0.660000in}}{\pgfqpoint{4.620000in}{4.620000in}}%
\pgfusepath{clip}%
\pgfsetrectcap%
\pgfsetroundjoin%
\pgfsetlinewidth{1.204500pt}%
\definecolor{currentstroke}{rgb}{0.000000,0.000000,0.000000}%
\pgfsetstrokecolor{currentstroke}%
\pgfsetdash{}{0pt}%
\pgfpathmoveto{\pgfqpoint{3.071405in}{3.123205in}}%
\pgfpathlineto{\pgfqpoint{5.395000in}{4.212181in}}%
\pgfusepath{stroke}%
\end{pgfscope}%
\end{pgfpicture}%
\makeatother%
\endgroup%
}
}
        \caption{Maximum-margin tropical hyperplane}
        \label{fig:MaxMargin}
    \end{subfigure}
    \hfill
    \centering
    \begin{subfigure}{0.4\textwidth}
        \centering
        \resizebox{\linewidth}{!}{%
        \centering
            \clipbox{0.2\width{} 0.3\height{} 0.2\width{} 0.25\height{}}{%% Creator: Matplotlib, PGF backend
%%
%% To include the figure in your LaTeX document, write
%%   \input{<filename>.pgf}
%%
%% Make sure the required packages are loaded in your preamble
%%   \usepackage{pgf}
%%
%% Also ensure that all the required font packages are loaded; for instance,
%% the lmodern package is sometimes necessary when using math font.
%%   \usepackage{lmodern}
%%
%% Figures using additional raster images can only be included by \input if
%% they are in the same directory as the main LaTeX file. For loading figures
%% from other directories you can use the `import` package
%%   \usepackage{import}
%%
%% and then include the figures with
%%   \import{<path to file>}{<filename>.pgf}
%%
%% Matplotlib used the following preamble
%%   
%%   \usepackage{fontspec}
%%   \setmainfont{DejaVuSerif.ttf}[Path=\detokenize{/Users/sam/Library/Python/3.9/lib/python/site-packages/matplotlib/mpl-data/fonts/ttf/}]
%%   \setsansfont{Arial.ttf}[Path=\detokenize{/System/Library/Fonts/Supplemental/}]
%%   \setmonofont{DejaVuSansMono.ttf}[Path=\detokenize{/Users/sam/Library/Python/3.9/lib/python/site-packages/matplotlib/mpl-data/fonts/ttf/}]
%%   \makeatletter\@ifpackageloaded{underscore}{}{\usepackage[strings]{underscore}}\makeatother
%%
\begingroup%
\makeatletter%
\begin{pgfpicture}%
\pgfpathrectangle{\pgfpointorigin}{\pgfqpoint{6.000000in}{6.000000in}}%
\pgfusepath{use as bounding box, clip}%
\begin{pgfscope}%
\pgfsetbuttcap%
\pgfsetmiterjoin%
\definecolor{currentfill}{rgb}{1.000000,1.000000,1.000000}%
\pgfsetfillcolor{currentfill}%
\pgfsetlinewidth{0.000000pt}%
\definecolor{currentstroke}{rgb}{1.000000,1.000000,1.000000}%
\pgfsetstrokecolor{currentstroke}%
\pgfsetdash{}{0pt}%
\pgfpathmoveto{\pgfqpoint{0.000000in}{0.000000in}}%
\pgfpathlineto{\pgfqpoint{6.000000in}{0.000000in}}%
\pgfpathlineto{\pgfqpoint{6.000000in}{6.000000in}}%
\pgfpathlineto{\pgfqpoint{0.000000in}{6.000000in}}%
\pgfpathlineto{\pgfqpoint{0.000000in}{0.000000in}}%
\pgfpathclose%
\pgfusepath{fill}%
\end{pgfscope}%
\begin{pgfscope}%
\pgfsetbuttcap%
\pgfsetmiterjoin%
\definecolor{currentfill}{rgb}{1.000000,1.000000,1.000000}%
\pgfsetfillcolor{currentfill}%
\pgfsetlinewidth{0.000000pt}%
\definecolor{currentstroke}{rgb}{0.000000,0.000000,0.000000}%
\pgfsetstrokecolor{currentstroke}%
\pgfsetstrokeopacity{0.000000}%
\pgfsetdash{}{0pt}%
\pgfpathmoveto{\pgfqpoint{0.765000in}{0.660000in}}%
\pgfpathlineto{\pgfqpoint{5.385000in}{0.660000in}}%
\pgfpathlineto{\pgfqpoint{5.385000in}{5.280000in}}%
\pgfpathlineto{\pgfqpoint{0.765000in}{5.280000in}}%
\pgfpathlineto{\pgfqpoint{0.765000in}{0.660000in}}%
\pgfpathclose%
\pgfusepath{fill}%
\end{pgfscope}%
\begin{pgfscope}%
\pgfpathrectangle{\pgfqpoint{0.765000in}{0.660000in}}{\pgfqpoint{4.620000in}{4.620000in}}%
\pgfusepath{clip}%
\pgfsetroundcap%
\pgfsetroundjoin%
\pgfsetlinewidth{1.204500pt}%
\definecolor{currentstroke}{rgb}{1.000000,0.576471,0.309804}%
\pgfsetstrokecolor{currentstroke}%
\pgfsetdash{}{0pt}%
\pgfpathmoveto{\pgfqpoint{2.935620in}{2.181202in}}%
\pgfusepath{stroke}%
\end{pgfscope}%
\begin{pgfscope}%
\pgfpathrectangle{\pgfqpoint{0.765000in}{0.660000in}}{\pgfqpoint{4.620000in}{4.620000in}}%
\pgfusepath{clip}%
\pgfsetbuttcap%
\pgfsetroundjoin%
\definecolor{currentfill}{rgb}{1.000000,0.576471,0.309804}%
\pgfsetfillcolor{currentfill}%
\pgfsetlinewidth{1.003750pt}%
\definecolor{currentstroke}{rgb}{1.000000,0.576471,0.309804}%
\pgfsetstrokecolor{currentstroke}%
\pgfsetdash{}{0pt}%
\pgfsys@defobject{currentmarker}{\pgfqpoint{-0.033333in}{-0.033333in}}{\pgfqpoint{0.033333in}{0.033333in}}{%
\pgfpathmoveto{\pgfqpoint{0.000000in}{-0.033333in}}%
\pgfpathcurveto{\pgfqpoint{0.008840in}{-0.033333in}}{\pgfqpoint{0.017319in}{-0.029821in}}{\pgfqpoint{0.023570in}{-0.023570in}}%
\pgfpathcurveto{\pgfqpoint{0.029821in}{-0.017319in}}{\pgfqpoint{0.033333in}{-0.008840in}}{\pgfqpoint{0.033333in}{0.000000in}}%
\pgfpathcurveto{\pgfqpoint{0.033333in}{0.008840in}}{\pgfqpoint{0.029821in}{0.017319in}}{\pgfqpoint{0.023570in}{0.023570in}}%
\pgfpathcurveto{\pgfqpoint{0.017319in}{0.029821in}}{\pgfqpoint{0.008840in}{0.033333in}}{\pgfqpoint{0.000000in}{0.033333in}}%
\pgfpathcurveto{\pgfqpoint{-0.008840in}{0.033333in}}{\pgfqpoint{-0.017319in}{0.029821in}}{\pgfqpoint{-0.023570in}{0.023570in}}%
\pgfpathcurveto{\pgfqpoint{-0.029821in}{0.017319in}}{\pgfqpoint{-0.033333in}{0.008840in}}{\pgfqpoint{-0.033333in}{0.000000in}}%
\pgfpathcurveto{\pgfqpoint{-0.033333in}{-0.008840in}}{\pgfqpoint{-0.029821in}{-0.017319in}}{\pgfqpoint{-0.023570in}{-0.023570in}}%
\pgfpathcurveto{\pgfqpoint{-0.017319in}{-0.029821in}}{\pgfqpoint{-0.008840in}{-0.033333in}}{\pgfqpoint{0.000000in}{-0.033333in}}%
\pgfpathlineto{\pgfqpoint{0.000000in}{-0.033333in}}%
\pgfpathclose%
\pgfusepath{stroke,fill}%
}%
\begin{pgfscope}%
\pgfsys@transformshift{2.935620in}{2.181202in}%
\pgfsys@useobject{currentmarker}{}%
\end{pgfscope}%
\end{pgfscope}%
\begin{pgfscope}%
\pgfpathrectangle{\pgfqpoint{0.765000in}{0.660000in}}{\pgfqpoint{4.620000in}{4.620000in}}%
\pgfusepath{clip}%
\pgfsetroundcap%
\pgfsetroundjoin%
\pgfsetlinewidth{1.204500pt}%
\definecolor{currentstroke}{rgb}{1.000000,0.576471,0.309804}%
\pgfsetstrokecolor{currentstroke}%
\pgfsetdash{}{0pt}%
\pgfpathmoveto{\pgfqpoint{2.359014in}{2.181202in}}%
\pgfusepath{stroke}%
\end{pgfscope}%
\begin{pgfscope}%
\pgfpathrectangle{\pgfqpoint{0.765000in}{0.660000in}}{\pgfqpoint{4.620000in}{4.620000in}}%
\pgfusepath{clip}%
\pgfsetbuttcap%
\pgfsetroundjoin%
\definecolor{currentfill}{rgb}{1.000000,0.576471,0.309804}%
\pgfsetfillcolor{currentfill}%
\pgfsetlinewidth{1.003750pt}%
\definecolor{currentstroke}{rgb}{1.000000,0.576471,0.309804}%
\pgfsetstrokecolor{currentstroke}%
\pgfsetdash{}{0pt}%
\pgfsys@defobject{currentmarker}{\pgfqpoint{-0.033333in}{-0.033333in}}{\pgfqpoint{0.033333in}{0.033333in}}{%
\pgfpathmoveto{\pgfqpoint{0.000000in}{-0.033333in}}%
\pgfpathcurveto{\pgfqpoint{0.008840in}{-0.033333in}}{\pgfqpoint{0.017319in}{-0.029821in}}{\pgfqpoint{0.023570in}{-0.023570in}}%
\pgfpathcurveto{\pgfqpoint{0.029821in}{-0.017319in}}{\pgfqpoint{0.033333in}{-0.008840in}}{\pgfqpoint{0.033333in}{0.000000in}}%
\pgfpathcurveto{\pgfqpoint{0.033333in}{0.008840in}}{\pgfqpoint{0.029821in}{0.017319in}}{\pgfqpoint{0.023570in}{0.023570in}}%
\pgfpathcurveto{\pgfqpoint{0.017319in}{0.029821in}}{\pgfqpoint{0.008840in}{0.033333in}}{\pgfqpoint{0.000000in}{0.033333in}}%
\pgfpathcurveto{\pgfqpoint{-0.008840in}{0.033333in}}{\pgfqpoint{-0.017319in}{0.029821in}}{\pgfqpoint{-0.023570in}{0.023570in}}%
\pgfpathcurveto{\pgfqpoint{-0.029821in}{0.017319in}}{\pgfqpoint{-0.033333in}{0.008840in}}{\pgfqpoint{-0.033333in}{0.000000in}}%
\pgfpathcurveto{\pgfqpoint{-0.033333in}{-0.008840in}}{\pgfqpoint{-0.029821in}{-0.017319in}}{\pgfqpoint{-0.023570in}{-0.023570in}}%
\pgfpathcurveto{\pgfqpoint{-0.017319in}{-0.029821in}}{\pgfqpoint{-0.008840in}{-0.033333in}}{\pgfqpoint{0.000000in}{-0.033333in}}%
\pgfpathlineto{\pgfqpoint{0.000000in}{-0.033333in}}%
\pgfpathclose%
\pgfusepath{stroke,fill}%
}%
\begin{pgfscope}%
\pgfsys@transformshift{2.359014in}{2.181202in}%
\pgfsys@useobject{currentmarker}{}%
\end{pgfscope}%
\end{pgfscope}%
\begin{pgfscope}%
\pgfpathrectangle{\pgfqpoint{0.765000in}{0.660000in}}{\pgfqpoint{4.620000in}{4.620000in}}%
\pgfusepath{clip}%
\pgfsetroundcap%
\pgfsetroundjoin%
\pgfsetlinewidth{1.204500pt}%
\definecolor{currentstroke}{rgb}{1.000000,0.576471,0.309804}%
\pgfsetstrokecolor{currentstroke}%
\pgfsetdash{}{0pt}%
\pgfpathmoveto{\pgfqpoint{2.589657in}{2.181202in}}%
\pgfusepath{stroke}%
\end{pgfscope}%
\begin{pgfscope}%
\pgfpathrectangle{\pgfqpoint{0.765000in}{0.660000in}}{\pgfqpoint{4.620000in}{4.620000in}}%
\pgfusepath{clip}%
\pgfsetbuttcap%
\pgfsetroundjoin%
\definecolor{currentfill}{rgb}{1.000000,0.576471,0.309804}%
\pgfsetfillcolor{currentfill}%
\pgfsetlinewidth{1.003750pt}%
\definecolor{currentstroke}{rgb}{1.000000,0.576471,0.309804}%
\pgfsetstrokecolor{currentstroke}%
\pgfsetdash{}{0pt}%
\pgfsys@defobject{currentmarker}{\pgfqpoint{-0.033333in}{-0.033333in}}{\pgfqpoint{0.033333in}{0.033333in}}{%
\pgfpathmoveto{\pgfqpoint{0.000000in}{-0.033333in}}%
\pgfpathcurveto{\pgfqpoint{0.008840in}{-0.033333in}}{\pgfqpoint{0.017319in}{-0.029821in}}{\pgfqpoint{0.023570in}{-0.023570in}}%
\pgfpathcurveto{\pgfqpoint{0.029821in}{-0.017319in}}{\pgfqpoint{0.033333in}{-0.008840in}}{\pgfqpoint{0.033333in}{0.000000in}}%
\pgfpathcurveto{\pgfqpoint{0.033333in}{0.008840in}}{\pgfqpoint{0.029821in}{0.017319in}}{\pgfqpoint{0.023570in}{0.023570in}}%
\pgfpathcurveto{\pgfqpoint{0.017319in}{0.029821in}}{\pgfqpoint{0.008840in}{0.033333in}}{\pgfqpoint{0.000000in}{0.033333in}}%
\pgfpathcurveto{\pgfqpoint{-0.008840in}{0.033333in}}{\pgfqpoint{-0.017319in}{0.029821in}}{\pgfqpoint{-0.023570in}{0.023570in}}%
\pgfpathcurveto{\pgfqpoint{-0.029821in}{0.017319in}}{\pgfqpoint{-0.033333in}{0.008840in}}{\pgfqpoint{-0.033333in}{0.000000in}}%
\pgfpathcurveto{\pgfqpoint{-0.033333in}{-0.008840in}}{\pgfqpoint{-0.029821in}{-0.017319in}}{\pgfqpoint{-0.023570in}{-0.023570in}}%
\pgfpathcurveto{\pgfqpoint{-0.017319in}{-0.029821in}}{\pgfqpoint{-0.008840in}{-0.033333in}}{\pgfqpoint{0.000000in}{-0.033333in}}%
\pgfpathlineto{\pgfqpoint{0.000000in}{-0.033333in}}%
\pgfpathclose%
\pgfusepath{stroke,fill}%
}%
\begin{pgfscope}%
\pgfsys@transformshift{2.589657in}{2.181202in}%
\pgfsys@useobject{currentmarker}{}%
\end{pgfscope}%
\end{pgfscope}%
\begin{pgfscope}%
\pgfpathrectangle{\pgfqpoint{0.765000in}{0.660000in}}{\pgfqpoint{4.620000in}{4.620000in}}%
\pgfusepath{clip}%
\pgfsetroundcap%
\pgfsetroundjoin%
\pgfsetlinewidth{1.204500pt}%
\definecolor{currentstroke}{rgb}{1.000000,0.576471,0.309804}%
\pgfsetstrokecolor{currentstroke}%
\pgfsetdash{}{0pt}%
\pgfpathmoveto{\pgfqpoint{2.474336in}{2.181202in}}%
\pgfusepath{stroke}%
\end{pgfscope}%
\begin{pgfscope}%
\pgfpathrectangle{\pgfqpoint{0.765000in}{0.660000in}}{\pgfqpoint{4.620000in}{4.620000in}}%
\pgfusepath{clip}%
\pgfsetbuttcap%
\pgfsetroundjoin%
\definecolor{currentfill}{rgb}{1.000000,0.576471,0.309804}%
\pgfsetfillcolor{currentfill}%
\pgfsetlinewidth{1.003750pt}%
\definecolor{currentstroke}{rgb}{1.000000,0.576471,0.309804}%
\pgfsetstrokecolor{currentstroke}%
\pgfsetdash{}{0pt}%
\pgfsys@defobject{currentmarker}{\pgfqpoint{-0.033333in}{-0.033333in}}{\pgfqpoint{0.033333in}{0.033333in}}{%
\pgfpathmoveto{\pgfqpoint{0.000000in}{-0.033333in}}%
\pgfpathcurveto{\pgfqpoint{0.008840in}{-0.033333in}}{\pgfqpoint{0.017319in}{-0.029821in}}{\pgfqpoint{0.023570in}{-0.023570in}}%
\pgfpathcurveto{\pgfqpoint{0.029821in}{-0.017319in}}{\pgfqpoint{0.033333in}{-0.008840in}}{\pgfqpoint{0.033333in}{0.000000in}}%
\pgfpathcurveto{\pgfqpoint{0.033333in}{0.008840in}}{\pgfqpoint{0.029821in}{0.017319in}}{\pgfqpoint{0.023570in}{0.023570in}}%
\pgfpathcurveto{\pgfqpoint{0.017319in}{0.029821in}}{\pgfqpoint{0.008840in}{0.033333in}}{\pgfqpoint{0.000000in}{0.033333in}}%
\pgfpathcurveto{\pgfqpoint{-0.008840in}{0.033333in}}{\pgfqpoint{-0.017319in}{0.029821in}}{\pgfqpoint{-0.023570in}{0.023570in}}%
\pgfpathcurveto{\pgfqpoint{-0.029821in}{0.017319in}}{\pgfqpoint{-0.033333in}{0.008840in}}{\pgfqpoint{-0.033333in}{0.000000in}}%
\pgfpathcurveto{\pgfqpoint{-0.033333in}{-0.008840in}}{\pgfqpoint{-0.029821in}{-0.017319in}}{\pgfqpoint{-0.023570in}{-0.023570in}}%
\pgfpathcurveto{\pgfqpoint{-0.017319in}{-0.029821in}}{\pgfqpoint{-0.008840in}{-0.033333in}}{\pgfqpoint{0.000000in}{-0.033333in}}%
\pgfpathlineto{\pgfqpoint{0.000000in}{-0.033333in}}%
\pgfpathclose%
\pgfusepath{stroke,fill}%
}%
\begin{pgfscope}%
\pgfsys@transformshift{2.474336in}{2.181202in}%
\pgfsys@useobject{currentmarker}{}%
\end{pgfscope}%
\end{pgfscope}%
\begin{pgfscope}%
\pgfpathrectangle{\pgfqpoint{0.765000in}{0.660000in}}{\pgfqpoint{4.620000in}{4.620000in}}%
\pgfusepath{clip}%
\pgfsetroundcap%
\pgfsetroundjoin%
\pgfsetlinewidth{1.204500pt}%
\definecolor{currentstroke}{rgb}{1.000000,0.576471,0.309804}%
\pgfsetstrokecolor{currentstroke}%
\pgfsetdash{}{0pt}%
\pgfpathmoveto{\pgfqpoint{3.050942in}{2.181202in}}%
\pgfusepath{stroke}%
\end{pgfscope}%
\begin{pgfscope}%
\pgfpathrectangle{\pgfqpoint{0.765000in}{0.660000in}}{\pgfqpoint{4.620000in}{4.620000in}}%
\pgfusepath{clip}%
\pgfsetbuttcap%
\pgfsetroundjoin%
\definecolor{currentfill}{rgb}{1.000000,0.576471,0.309804}%
\pgfsetfillcolor{currentfill}%
\pgfsetlinewidth{1.003750pt}%
\definecolor{currentstroke}{rgb}{1.000000,0.576471,0.309804}%
\pgfsetstrokecolor{currentstroke}%
\pgfsetdash{}{0pt}%
\pgfsys@defobject{currentmarker}{\pgfqpoint{-0.033333in}{-0.033333in}}{\pgfqpoint{0.033333in}{0.033333in}}{%
\pgfpathmoveto{\pgfqpoint{0.000000in}{-0.033333in}}%
\pgfpathcurveto{\pgfqpoint{0.008840in}{-0.033333in}}{\pgfqpoint{0.017319in}{-0.029821in}}{\pgfqpoint{0.023570in}{-0.023570in}}%
\pgfpathcurveto{\pgfqpoint{0.029821in}{-0.017319in}}{\pgfqpoint{0.033333in}{-0.008840in}}{\pgfqpoint{0.033333in}{0.000000in}}%
\pgfpathcurveto{\pgfqpoint{0.033333in}{0.008840in}}{\pgfqpoint{0.029821in}{0.017319in}}{\pgfqpoint{0.023570in}{0.023570in}}%
\pgfpathcurveto{\pgfqpoint{0.017319in}{0.029821in}}{\pgfqpoint{0.008840in}{0.033333in}}{\pgfqpoint{0.000000in}{0.033333in}}%
\pgfpathcurveto{\pgfqpoint{-0.008840in}{0.033333in}}{\pgfqpoint{-0.017319in}{0.029821in}}{\pgfqpoint{-0.023570in}{0.023570in}}%
\pgfpathcurveto{\pgfqpoint{-0.029821in}{0.017319in}}{\pgfqpoint{-0.033333in}{0.008840in}}{\pgfqpoint{-0.033333in}{0.000000in}}%
\pgfpathcurveto{\pgfqpoint{-0.033333in}{-0.008840in}}{\pgfqpoint{-0.029821in}{-0.017319in}}{\pgfqpoint{-0.023570in}{-0.023570in}}%
\pgfpathcurveto{\pgfqpoint{-0.017319in}{-0.029821in}}{\pgfqpoint{-0.008840in}{-0.033333in}}{\pgfqpoint{0.000000in}{-0.033333in}}%
\pgfpathlineto{\pgfqpoint{0.000000in}{-0.033333in}}%
\pgfpathclose%
\pgfusepath{stroke,fill}%
}%
\begin{pgfscope}%
\pgfsys@transformshift{3.050942in}{2.181202in}%
\pgfsys@useobject{currentmarker}{}%
\end{pgfscope}%
\end{pgfscope}%
\begin{pgfscope}%
\pgfpathrectangle{\pgfqpoint{0.765000in}{0.660000in}}{\pgfqpoint{4.620000in}{4.620000in}}%
\pgfusepath{clip}%
\pgfsetroundcap%
\pgfsetroundjoin%
\pgfsetlinewidth{1.204500pt}%
\definecolor{currentstroke}{rgb}{1.000000,0.576471,0.309804}%
\pgfsetstrokecolor{currentstroke}%
\pgfsetdash{}{0pt}%
\pgfpathmoveto{\pgfqpoint{3.512226in}{2.343341in}}%
\pgfusepath{stroke}%
\end{pgfscope}%
\begin{pgfscope}%
\pgfpathrectangle{\pgfqpoint{0.765000in}{0.660000in}}{\pgfqpoint{4.620000in}{4.620000in}}%
\pgfusepath{clip}%
\pgfsetbuttcap%
\pgfsetroundjoin%
\definecolor{currentfill}{rgb}{1.000000,0.576471,0.309804}%
\pgfsetfillcolor{currentfill}%
\pgfsetlinewidth{1.003750pt}%
\definecolor{currentstroke}{rgb}{1.000000,0.576471,0.309804}%
\pgfsetstrokecolor{currentstroke}%
\pgfsetdash{}{0pt}%
\pgfsys@defobject{currentmarker}{\pgfqpoint{-0.033333in}{-0.033333in}}{\pgfqpoint{0.033333in}{0.033333in}}{%
\pgfpathmoveto{\pgfqpoint{0.000000in}{-0.033333in}}%
\pgfpathcurveto{\pgfqpoint{0.008840in}{-0.033333in}}{\pgfqpoint{0.017319in}{-0.029821in}}{\pgfqpoint{0.023570in}{-0.023570in}}%
\pgfpathcurveto{\pgfqpoint{0.029821in}{-0.017319in}}{\pgfqpoint{0.033333in}{-0.008840in}}{\pgfqpoint{0.033333in}{0.000000in}}%
\pgfpathcurveto{\pgfqpoint{0.033333in}{0.008840in}}{\pgfqpoint{0.029821in}{0.017319in}}{\pgfqpoint{0.023570in}{0.023570in}}%
\pgfpathcurveto{\pgfqpoint{0.017319in}{0.029821in}}{\pgfqpoint{0.008840in}{0.033333in}}{\pgfqpoint{0.000000in}{0.033333in}}%
\pgfpathcurveto{\pgfqpoint{-0.008840in}{0.033333in}}{\pgfqpoint{-0.017319in}{0.029821in}}{\pgfqpoint{-0.023570in}{0.023570in}}%
\pgfpathcurveto{\pgfqpoint{-0.029821in}{0.017319in}}{\pgfqpoint{-0.033333in}{0.008840in}}{\pgfqpoint{-0.033333in}{0.000000in}}%
\pgfpathcurveto{\pgfqpoint{-0.033333in}{-0.008840in}}{\pgfqpoint{-0.029821in}{-0.017319in}}{\pgfqpoint{-0.023570in}{-0.023570in}}%
\pgfpathcurveto{\pgfqpoint{-0.017319in}{-0.029821in}}{\pgfqpoint{-0.008840in}{-0.033333in}}{\pgfqpoint{0.000000in}{-0.033333in}}%
\pgfpathlineto{\pgfqpoint{0.000000in}{-0.033333in}}%
\pgfpathclose%
\pgfusepath{stroke,fill}%
}%
\begin{pgfscope}%
\pgfsys@transformshift{3.512226in}{2.343341in}%
\pgfsys@useobject{currentmarker}{}%
\end{pgfscope}%
\end{pgfscope}%
\begin{pgfscope}%
\pgfpathrectangle{\pgfqpoint{0.765000in}{0.660000in}}{\pgfqpoint{4.620000in}{4.620000in}}%
\pgfusepath{clip}%
\pgfsetroundcap%
\pgfsetroundjoin%
\pgfsetlinewidth{1.204500pt}%
\definecolor{currentstroke}{rgb}{1.000000,0.576471,0.309804}%
\pgfsetstrokecolor{currentstroke}%
\pgfsetdash{}{0pt}%
\pgfpathmoveto{\pgfqpoint{2.877960in}{2.262271in}}%
\pgfusepath{stroke}%
\end{pgfscope}%
\begin{pgfscope}%
\pgfpathrectangle{\pgfqpoint{0.765000in}{0.660000in}}{\pgfqpoint{4.620000in}{4.620000in}}%
\pgfusepath{clip}%
\pgfsetbuttcap%
\pgfsetroundjoin%
\definecolor{currentfill}{rgb}{1.000000,0.576471,0.309804}%
\pgfsetfillcolor{currentfill}%
\pgfsetlinewidth{1.003750pt}%
\definecolor{currentstroke}{rgb}{1.000000,0.576471,0.309804}%
\pgfsetstrokecolor{currentstroke}%
\pgfsetdash{}{0pt}%
\pgfsys@defobject{currentmarker}{\pgfqpoint{-0.033333in}{-0.033333in}}{\pgfqpoint{0.033333in}{0.033333in}}{%
\pgfpathmoveto{\pgfqpoint{0.000000in}{-0.033333in}}%
\pgfpathcurveto{\pgfqpoint{0.008840in}{-0.033333in}}{\pgfqpoint{0.017319in}{-0.029821in}}{\pgfqpoint{0.023570in}{-0.023570in}}%
\pgfpathcurveto{\pgfqpoint{0.029821in}{-0.017319in}}{\pgfqpoint{0.033333in}{-0.008840in}}{\pgfqpoint{0.033333in}{0.000000in}}%
\pgfpathcurveto{\pgfqpoint{0.033333in}{0.008840in}}{\pgfqpoint{0.029821in}{0.017319in}}{\pgfqpoint{0.023570in}{0.023570in}}%
\pgfpathcurveto{\pgfqpoint{0.017319in}{0.029821in}}{\pgfqpoint{0.008840in}{0.033333in}}{\pgfqpoint{0.000000in}{0.033333in}}%
\pgfpathcurveto{\pgfqpoint{-0.008840in}{0.033333in}}{\pgfqpoint{-0.017319in}{0.029821in}}{\pgfqpoint{-0.023570in}{0.023570in}}%
\pgfpathcurveto{\pgfqpoint{-0.029821in}{0.017319in}}{\pgfqpoint{-0.033333in}{0.008840in}}{\pgfqpoint{-0.033333in}{0.000000in}}%
\pgfpathcurveto{\pgfqpoint{-0.033333in}{-0.008840in}}{\pgfqpoint{-0.029821in}{-0.017319in}}{\pgfqpoint{-0.023570in}{-0.023570in}}%
\pgfpathcurveto{\pgfqpoint{-0.017319in}{-0.029821in}}{\pgfqpoint{-0.008840in}{-0.033333in}}{\pgfqpoint{0.000000in}{-0.033333in}}%
\pgfpathlineto{\pgfqpoint{0.000000in}{-0.033333in}}%
\pgfpathclose%
\pgfusepath{stroke,fill}%
}%
\begin{pgfscope}%
\pgfsys@transformshift{2.877960in}{2.262271in}%
\pgfsys@useobject{currentmarker}{}%
\end{pgfscope}%
\end{pgfscope}%
\begin{pgfscope}%
\pgfpathrectangle{\pgfqpoint{0.765000in}{0.660000in}}{\pgfqpoint{4.620000in}{4.620000in}}%
\pgfusepath{clip}%
\pgfsetroundcap%
\pgfsetroundjoin%
\pgfsetlinewidth{1.204500pt}%
\definecolor{currentstroke}{rgb}{1.000000,0.576471,0.309804}%
\pgfsetstrokecolor{currentstroke}%
\pgfsetdash{}{0pt}%
\pgfpathmoveto{\pgfqpoint{2.820299in}{2.181202in}}%
\pgfusepath{stroke}%
\end{pgfscope}%
\begin{pgfscope}%
\pgfpathrectangle{\pgfqpoint{0.765000in}{0.660000in}}{\pgfqpoint{4.620000in}{4.620000in}}%
\pgfusepath{clip}%
\pgfsetbuttcap%
\pgfsetroundjoin%
\definecolor{currentfill}{rgb}{1.000000,0.576471,0.309804}%
\pgfsetfillcolor{currentfill}%
\pgfsetlinewidth{1.003750pt}%
\definecolor{currentstroke}{rgb}{1.000000,0.576471,0.309804}%
\pgfsetstrokecolor{currentstroke}%
\pgfsetdash{}{0pt}%
\pgfsys@defobject{currentmarker}{\pgfqpoint{-0.033333in}{-0.033333in}}{\pgfqpoint{0.033333in}{0.033333in}}{%
\pgfpathmoveto{\pgfqpoint{0.000000in}{-0.033333in}}%
\pgfpathcurveto{\pgfqpoint{0.008840in}{-0.033333in}}{\pgfqpoint{0.017319in}{-0.029821in}}{\pgfqpoint{0.023570in}{-0.023570in}}%
\pgfpathcurveto{\pgfqpoint{0.029821in}{-0.017319in}}{\pgfqpoint{0.033333in}{-0.008840in}}{\pgfqpoint{0.033333in}{0.000000in}}%
\pgfpathcurveto{\pgfqpoint{0.033333in}{0.008840in}}{\pgfqpoint{0.029821in}{0.017319in}}{\pgfqpoint{0.023570in}{0.023570in}}%
\pgfpathcurveto{\pgfqpoint{0.017319in}{0.029821in}}{\pgfqpoint{0.008840in}{0.033333in}}{\pgfqpoint{0.000000in}{0.033333in}}%
\pgfpathcurveto{\pgfqpoint{-0.008840in}{0.033333in}}{\pgfqpoint{-0.017319in}{0.029821in}}{\pgfqpoint{-0.023570in}{0.023570in}}%
\pgfpathcurveto{\pgfqpoint{-0.029821in}{0.017319in}}{\pgfqpoint{-0.033333in}{0.008840in}}{\pgfqpoint{-0.033333in}{0.000000in}}%
\pgfpathcurveto{\pgfqpoint{-0.033333in}{-0.008840in}}{\pgfqpoint{-0.029821in}{-0.017319in}}{\pgfqpoint{-0.023570in}{-0.023570in}}%
\pgfpathcurveto{\pgfqpoint{-0.017319in}{-0.029821in}}{\pgfqpoint{-0.008840in}{-0.033333in}}{\pgfqpoint{0.000000in}{-0.033333in}}%
\pgfpathlineto{\pgfqpoint{0.000000in}{-0.033333in}}%
\pgfpathclose%
\pgfusepath{stroke,fill}%
}%
\begin{pgfscope}%
\pgfsys@transformshift{2.820299in}{2.181202in}%
\pgfsys@useobject{currentmarker}{}%
\end{pgfscope}%
\end{pgfscope}%
\begin{pgfscope}%
\pgfpathrectangle{\pgfqpoint{0.765000in}{0.660000in}}{\pgfqpoint{4.620000in}{4.620000in}}%
\pgfusepath{clip}%
\pgfsetroundcap%
\pgfsetroundjoin%
\pgfsetlinewidth{1.204500pt}%
\definecolor{currentstroke}{rgb}{1.000000,0.576471,0.309804}%
\pgfsetstrokecolor{currentstroke}%
\pgfsetdash{}{0pt}%
\pgfpathmoveto{\pgfqpoint{2.243693in}{2.181202in}}%
\pgfusepath{stroke}%
\end{pgfscope}%
\begin{pgfscope}%
\pgfpathrectangle{\pgfqpoint{0.765000in}{0.660000in}}{\pgfqpoint{4.620000in}{4.620000in}}%
\pgfusepath{clip}%
\pgfsetbuttcap%
\pgfsetroundjoin%
\definecolor{currentfill}{rgb}{1.000000,0.576471,0.309804}%
\pgfsetfillcolor{currentfill}%
\pgfsetlinewidth{1.003750pt}%
\definecolor{currentstroke}{rgb}{1.000000,0.576471,0.309804}%
\pgfsetstrokecolor{currentstroke}%
\pgfsetdash{}{0pt}%
\pgfsys@defobject{currentmarker}{\pgfqpoint{-0.033333in}{-0.033333in}}{\pgfqpoint{0.033333in}{0.033333in}}{%
\pgfpathmoveto{\pgfqpoint{0.000000in}{-0.033333in}}%
\pgfpathcurveto{\pgfqpoint{0.008840in}{-0.033333in}}{\pgfqpoint{0.017319in}{-0.029821in}}{\pgfqpoint{0.023570in}{-0.023570in}}%
\pgfpathcurveto{\pgfqpoint{0.029821in}{-0.017319in}}{\pgfqpoint{0.033333in}{-0.008840in}}{\pgfqpoint{0.033333in}{0.000000in}}%
\pgfpathcurveto{\pgfqpoint{0.033333in}{0.008840in}}{\pgfqpoint{0.029821in}{0.017319in}}{\pgfqpoint{0.023570in}{0.023570in}}%
\pgfpathcurveto{\pgfqpoint{0.017319in}{0.029821in}}{\pgfqpoint{0.008840in}{0.033333in}}{\pgfqpoint{0.000000in}{0.033333in}}%
\pgfpathcurveto{\pgfqpoint{-0.008840in}{0.033333in}}{\pgfqpoint{-0.017319in}{0.029821in}}{\pgfqpoint{-0.023570in}{0.023570in}}%
\pgfpathcurveto{\pgfqpoint{-0.029821in}{0.017319in}}{\pgfqpoint{-0.033333in}{0.008840in}}{\pgfqpoint{-0.033333in}{0.000000in}}%
\pgfpathcurveto{\pgfqpoint{-0.033333in}{-0.008840in}}{\pgfqpoint{-0.029821in}{-0.017319in}}{\pgfqpoint{-0.023570in}{-0.023570in}}%
\pgfpathcurveto{\pgfqpoint{-0.017319in}{-0.029821in}}{\pgfqpoint{-0.008840in}{-0.033333in}}{\pgfqpoint{0.000000in}{-0.033333in}}%
\pgfpathlineto{\pgfqpoint{0.000000in}{-0.033333in}}%
\pgfpathclose%
\pgfusepath{stroke,fill}%
}%
\begin{pgfscope}%
\pgfsys@transformshift{2.243693in}{2.181202in}%
\pgfsys@useobject{currentmarker}{}%
\end{pgfscope}%
\end{pgfscope}%
\begin{pgfscope}%
\pgfpathrectangle{\pgfqpoint{0.765000in}{0.660000in}}{\pgfqpoint{4.620000in}{4.620000in}}%
\pgfusepath{clip}%
\pgfsetroundcap%
\pgfsetroundjoin%
\pgfsetlinewidth{1.204500pt}%
\definecolor{currentstroke}{rgb}{1.000000,0.576471,0.309804}%
\pgfsetstrokecolor{currentstroke}%
\pgfsetdash{}{0pt}%
\pgfpathmoveto{\pgfqpoint{2.416675in}{2.100132in}}%
\pgfusepath{stroke}%
\end{pgfscope}%
\begin{pgfscope}%
\pgfpathrectangle{\pgfqpoint{0.765000in}{0.660000in}}{\pgfqpoint{4.620000in}{4.620000in}}%
\pgfusepath{clip}%
\pgfsetbuttcap%
\pgfsetroundjoin%
\definecolor{currentfill}{rgb}{1.000000,0.576471,0.309804}%
\pgfsetfillcolor{currentfill}%
\pgfsetlinewidth{1.003750pt}%
\definecolor{currentstroke}{rgb}{1.000000,0.576471,0.309804}%
\pgfsetstrokecolor{currentstroke}%
\pgfsetdash{}{0pt}%
\pgfsys@defobject{currentmarker}{\pgfqpoint{-0.033333in}{-0.033333in}}{\pgfqpoint{0.033333in}{0.033333in}}{%
\pgfpathmoveto{\pgfqpoint{0.000000in}{-0.033333in}}%
\pgfpathcurveto{\pgfqpoint{0.008840in}{-0.033333in}}{\pgfqpoint{0.017319in}{-0.029821in}}{\pgfqpoint{0.023570in}{-0.023570in}}%
\pgfpathcurveto{\pgfqpoint{0.029821in}{-0.017319in}}{\pgfqpoint{0.033333in}{-0.008840in}}{\pgfqpoint{0.033333in}{0.000000in}}%
\pgfpathcurveto{\pgfqpoint{0.033333in}{0.008840in}}{\pgfqpoint{0.029821in}{0.017319in}}{\pgfqpoint{0.023570in}{0.023570in}}%
\pgfpathcurveto{\pgfqpoint{0.017319in}{0.029821in}}{\pgfqpoint{0.008840in}{0.033333in}}{\pgfqpoint{0.000000in}{0.033333in}}%
\pgfpathcurveto{\pgfqpoint{-0.008840in}{0.033333in}}{\pgfqpoint{-0.017319in}{0.029821in}}{\pgfqpoint{-0.023570in}{0.023570in}}%
\pgfpathcurveto{\pgfqpoint{-0.029821in}{0.017319in}}{\pgfqpoint{-0.033333in}{0.008840in}}{\pgfqpoint{-0.033333in}{0.000000in}}%
\pgfpathcurveto{\pgfqpoint{-0.033333in}{-0.008840in}}{\pgfqpoint{-0.029821in}{-0.017319in}}{\pgfqpoint{-0.023570in}{-0.023570in}}%
\pgfpathcurveto{\pgfqpoint{-0.017319in}{-0.029821in}}{\pgfqpoint{-0.008840in}{-0.033333in}}{\pgfqpoint{0.000000in}{-0.033333in}}%
\pgfpathlineto{\pgfqpoint{0.000000in}{-0.033333in}}%
\pgfpathclose%
\pgfusepath{stroke,fill}%
}%
\begin{pgfscope}%
\pgfsys@transformshift{2.416675in}{2.100132in}%
\pgfsys@useobject{currentmarker}{}%
\end{pgfscope}%
\end{pgfscope}%
\begin{pgfscope}%
\pgfpathrectangle{\pgfqpoint{0.765000in}{0.660000in}}{\pgfqpoint{4.620000in}{4.620000in}}%
\pgfusepath{clip}%
\pgfsetroundcap%
\pgfsetroundjoin%
\pgfsetlinewidth{1.204500pt}%
\definecolor{currentstroke}{rgb}{1.000000,0.576471,0.309804}%
\pgfsetstrokecolor{currentstroke}%
\pgfsetdash{}{0pt}%
\pgfpathmoveto{\pgfqpoint{3.166263in}{2.181202in}}%
\pgfusepath{stroke}%
\end{pgfscope}%
\begin{pgfscope}%
\pgfpathrectangle{\pgfqpoint{0.765000in}{0.660000in}}{\pgfqpoint{4.620000in}{4.620000in}}%
\pgfusepath{clip}%
\pgfsetbuttcap%
\pgfsetroundjoin%
\definecolor{currentfill}{rgb}{1.000000,0.576471,0.309804}%
\pgfsetfillcolor{currentfill}%
\pgfsetlinewidth{1.003750pt}%
\definecolor{currentstroke}{rgb}{1.000000,0.576471,0.309804}%
\pgfsetstrokecolor{currentstroke}%
\pgfsetdash{}{0pt}%
\pgfsys@defobject{currentmarker}{\pgfqpoint{-0.033333in}{-0.033333in}}{\pgfqpoint{0.033333in}{0.033333in}}{%
\pgfpathmoveto{\pgfqpoint{0.000000in}{-0.033333in}}%
\pgfpathcurveto{\pgfqpoint{0.008840in}{-0.033333in}}{\pgfqpoint{0.017319in}{-0.029821in}}{\pgfqpoint{0.023570in}{-0.023570in}}%
\pgfpathcurveto{\pgfqpoint{0.029821in}{-0.017319in}}{\pgfqpoint{0.033333in}{-0.008840in}}{\pgfqpoint{0.033333in}{0.000000in}}%
\pgfpathcurveto{\pgfqpoint{0.033333in}{0.008840in}}{\pgfqpoint{0.029821in}{0.017319in}}{\pgfqpoint{0.023570in}{0.023570in}}%
\pgfpathcurveto{\pgfqpoint{0.017319in}{0.029821in}}{\pgfqpoint{0.008840in}{0.033333in}}{\pgfqpoint{0.000000in}{0.033333in}}%
\pgfpathcurveto{\pgfqpoint{-0.008840in}{0.033333in}}{\pgfqpoint{-0.017319in}{0.029821in}}{\pgfqpoint{-0.023570in}{0.023570in}}%
\pgfpathcurveto{\pgfqpoint{-0.029821in}{0.017319in}}{\pgfqpoint{-0.033333in}{0.008840in}}{\pgfqpoint{-0.033333in}{0.000000in}}%
\pgfpathcurveto{\pgfqpoint{-0.033333in}{-0.008840in}}{\pgfqpoint{-0.029821in}{-0.017319in}}{\pgfqpoint{-0.023570in}{-0.023570in}}%
\pgfpathcurveto{\pgfqpoint{-0.017319in}{-0.029821in}}{\pgfqpoint{-0.008840in}{-0.033333in}}{\pgfqpoint{0.000000in}{-0.033333in}}%
\pgfpathlineto{\pgfqpoint{0.000000in}{-0.033333in}}%
\pgfpathclose%
\pgfusepath{stroke,fill}%
}%
\begin{pgfscope}%
\pgfsys@transformshift{3.166263in}{2.181202in}%
\pgfsys@useobject{currentmarker}{}%
\end{pgfscope}%
\end{pgfscope}%
\begin{pgfscope}%
\pgfpathrectangle{\pgfqpoint{0.765000in}{0.660000in}}{\pgfqpoint{4.620000in}{4.620000in}}%
\pgfusepath{clip}%
\pgfsetroundcap%
\pgfsetroundjoin%
\pgfsetlinewidth{1.204500pt}%
\definecolor{currentstroke}{rgb}{1.000000,0.576471,0.309804}%
\pgfsetstrokecolor{currentstroke}%
\pgfsetdash{}{0pt}%
\pgfpathmoveto{\pgfqpoint{2.820299in}{2.181202in}}%
\pgfusepath{stroke}%
\end{pgfscope}%
\begin{pgfscope}%
\pgfpathrectangle{\pgfqpoint{0.765000in}{0.660000in}}{\pgfqpoint{4.620000in}{4.620000in}}%
\pgfusepath{clip}%
\pgfsetbuttcap%
\pgfsetroundjoin%
\definecolor{currentfill}{rgb}{1.000000,0.576471,0.309804}%
\pgfsetfillcolor{currentfill}%
\pgfsetlinewidth{1.003750pt}%
\definecolor{currentstroke}{rgb}{1.000000,0.576471,0.309804}%
\pgfsetstrokecolor{currentstroke}%
\pgfsetdash{}{0pt}%
\pgfsys@defobject{currentmarker}{\pgfqpoint{-0.033333in}{-0.033333in}}{\pgfqpoint{0.033333in}{0.033333in}}{%
\pgfpathmoveto{\pgfqpoint{0.000000in}{-0.033333in}}%
\pgfpathcurveto{\pgfqpoint{0.008840in}{-0.033333in}}{\pgfqpoint{0.017319in}{-0.029821in}}{\pgfqpoint{0.023570in}{-0.023570in}}%
\pgfpathcurveto{\pgfqpoint{0.029821in}{-0.017319in}}{\pgfqpoint{0.033333in}{-0.008840in}}{\pgfqpoint{0.033333in}{0.000000in}}%
\pgfpathcurveto{\pgfqpoint{0.033333in}{0.008840in}}{\pgfqpoint{0.029821in}{0.017319in}}{\pgfqpoint{0.023570in}{0.023570in}}%
\pgfpathcurveto{\pgfqpoint{0.017319in}{0.029821in}}{\pgfqpoint{0.008840in}{0.033333in}}{\pgfqpoint{0.000000in}{0.033333in}}%
\pgfpathcurveto{\pgfqpoint{-0.008840in}{0.033333in}}{\pgfqpoint{-0.017319in}{0.029821in}}{\pgfqpoint{-0.023570in}{0.023570in}}%
\pgfpathcurveto{\pgfqpoint{-0.029821in}{0.017319in}}{\pgfqpoint{-0.033333in}{0.008840in}}{\pgfqpoint{-0.033333in}{0.000000in}}%
\pgfpathcurveto{\pgfqpoint{-0.033333in}{-0.008840in}}{\pgfqpoint{-0.029821in}{-0.017319in}}{\pgfqpoint{-0.023570in}{-0.023570in}}%
\pgfpathcurveto{\pgfqpoint{-0.017319in}{-0.029821in}}{\pgfqpoint{-0.008840in}{-0.033333in}}{\pgfqpoint{0.000000in}{-0.033333in}}%
\pgfpathlineto{\pgfqpoint{0.000000in}{-0.033333in}}%
\pgfpathclose%
\pgfusepath{stroke,fill}%
}%
\begin{pgfscope}%
\pgfsys@transformshift{2.820299in}{2.181202in}%
\pgfsys@useobject{currentmarker}{}%
\end{pgfscope}%
\end{pgfscope}%
\begin{pgfscope}%
\pgfpathrectangle{\pgfqpoint{0.765000in}{0.660000in}}{\pgfqpoint{4.620000in}{4.620000in}}%
\pgfusepath{clip}%
\pgfsetroundcap%
\pgfsetroundjoin%
\pgfsetlinewidth{1.204500pt}%
\definecolor{currentstroke}{rgb}{1.000000,0.576471,0.309804}%
\pgfsetstrokecolor{currentstroke}%
\pgfsetdash{}{0pt}%
\pgfpathmoveto{\pgfqpoint{2.301354in}{2.100132in}}%
\pgfusepath{stroke}%
\end{pgfscope}%
\begin{pgfscope}%
\pgfpathrectangle{\pgfqpoint{0.765000in}{0.660000in}}{\pgfqpoint{4.620000in}{4.620000in}}%
\pgfusepath{clip}%
\pgfsetbuttcap%
\pgfsetroundjoin%
\definecolor{currentfill}{rgb}{1.000000,0.576471,0.309804}%
\pgfsetfillcolor{currentfill}%
\pgfsetlinewidth{1.003750pt}%
\definecolor{currentstroke}{rgb}{1.000000,0.576471,0.309804}%
\pgfsetstrokecolor{currentstroke}%
\pgfsetdash{}{0pt}%
\pgfsys@defobject{currentmarker}{\pgfqpoint{-0.033333in}{-0.033333in}}{\pgfqpoint{0.033333in}{0.033333in}}{%
\pgfpathmoveto{\pgfqpoint{0.000000in}{-0.033333in}}%
\pgfpathcurveto{\pgfqpoint{0.008840in}{-0.033333in}}{\pgfqpoint{0.017319in}{-0.029821in}}{\pgfqpoint{0.023570in}{-0.023570in}}%
\pgfpathcurveto{\pgfqpoint{0.029821in}{-0.017319in}}{\pgfqpoint{0.033333in}{-0.008840in}}{\pgfqpoint{0.033333in}{0.000000in}}%
\pgfpathcurveto{\pgfqpoint{0.033333in}{0.008840in}}{\pgfqpoint{0.029821in}{0.017319in}}{\pgfqpoint{0.023570in}{0.023570in}}%
\pgfpathcurveto{\pgfqpoint{0.017319in}{0.029821in}}{\pgfqpoint{0.008840in}{0.033333in}}{\pgfqpoint{0.000000in}{0.033333in}}%
\pgfpathcurveto{\pgfqpoint{-0.008840in}{0.033333in}}{\pgfqpoint{-0.017319in}{0.029821in}}{\pgfqpoint{-0.023570in}{0.023570in}}%
\pgfpathcurveto{\pgfqpoint{-0.029821in}{0.017319in}}{\pgfqpoint{-0.033333in}{0.008840in}}{\pgfqpoint{-0.033333in}{0.000000in}}%
\pgfpathcurveto{\pgfqpoint{-0.033333in}{-0.008840in}}{\pgfqpoint{-0.029821in}{-0.017319in}}{\pgfqpoint{-0.023570in}{-0.023570in}}%
\pgfpathcurveto{\pgfqpoint{-0.017319in}{-0.029821in}}{\pgfqpoint{-0.008840in}{-0.033333in}}{\pgfqpoint{0.000000in}{-0.033333in}}%
\pgfpathlineto{\pgfqpoint{0.000000in}{-0.033333in}}%
\pgfpathclose%
\pgfusepath{stroke,fill}%
}%
\begin{pgfscope}%
\pgfsys@transformshift{2.301354in}{2.100132in}%
\pgfsys@useobject{currentmarker}{}%
\end{pgfscope}%
\end{pgfscope}%
\begin{pgfscope}%
\pgfpathrectangle{\pgfqpoint{0.765000in}{0.660000in}}{\pgfqpoint{4.620000in}{4.620000in}}%
\pgfusepath{clip}%
\pgfsetroundcap%
\pgfsetroundjoin%
\pgfsetlinewidth{1.204500pt}%
\definecolor{currentstroke}{rgb}{1.000000,0.576471,0.309804}%
\pgfsetstrokecolor{currentstroke}%
\pgfsetdash{}{0pt}%
\pgfpathmoveto{\pgfqpoint{2.301354in}{2.100132in}}%
\pgfusepath{stroke}%
\end{pgfscope}%
\begin{pgfscope}%
\pgfpathrectangle{\pgfqpoint{0.765000in}{0.660000in}}{\pgfqpoint{4.620000in}{4.620000in}}%
\pgfusepath{clip}%
\pgfsetbuttcap%
\pgfsetroundjoin%
\definecolor{currentfill}{rgb}{1.000000,0.576471,0.309804}%
\pgfsetfillcolor{currentfill}%
\pgfsetlinewidth{1.003750pt}%
\definecolor{currentstroke}{rgb}{1.000000,0.576471,0.309804}%
\pgfsetstrokecolor{currentstroke}%
\pgfsetdash{}{0pt}%
\pgfsys@defobject{currentmarker}{\pgfqpoint{-0.033333in}{-0.033333in}}{\pgfqpoint{0.033333in}{0.033333in}}{%
\pgfpathmoveto{\pgfqpoint{0.000000in}{-0.033333in}}%
\pgfpathcurveto{\pgfqpoint{0.008840in}{-0.033333in}}{\pgfqpoint{0.017319in}{-0.029821in}}{\pgfqpoint{0.023570in}{-0.023570in}}%
\pgfpathcurveto{\pgfqpoint{0.029821in}{-0.017319in}}{\pgfqpoint{0.033333in}{-0.008840in}}{\pgfqpoint{0.033333in}{0.000000in}}%
\pgfpathcurveto{\pgfqpoint{0.033333in}{0.008840in}}{\pgfqpoint{0.029821in}{0.017319in}}{\pgfqpoint{0.023570in}{0.023570in}}%
\pgfpathcurveto{\pgfqpoint{0.017319in}{0.029821in}}{\pgfqpoint{0.008840in}{0.033333in}}{\pgfqpoint{0.000000in}{0.033333in}}%
\pgfpathcurveto{\pgfqpoint{-0.008840in}{0.033333in}}{\pgfqpoint{-0.017319in}{0.029821in}}{\pgfqpoint{-0.023570in}{0.023570in}}%
\pgfpathcurveto{\pgfqpoint{-0.029821in}{0.017319in}}{\pgfqpoint{-0.033333in}{0.008840in}}{\pgfqpoint{-0.033333in}{0.000000in}}%
\pgfpathcurveto{\pgfqpoint{-0.033333in}{-0.008840in}}{\pgfqpoint{-0.029821in}{-0.017319in}}{\pgfqpoint{-0.023570in}{-0.023570in}}%
\pgfpathcurveto{\pgfqpoint{-0.017319in}{-0.029821in}}{\pgfqpoint{-0.008840in}{-0.033333in}}{\pgfqpoint{0.000000in}{-0.033333in}}%
\pgfpathlineto{\pgfqpoint{0.000000in}{-0.033333in}}%
\pgfpathclose%
\pgfusepath{stroke,fill}%
}%
\begin{pgfscope}%
\pgfsys@transformshift{2.301354in}{2.100132in}%
\pgfsys@useobject{currentmarker}{}%
\end{pgfscope}%
\end{pgfscope}%
\begin{pgfscope}%
\pgfpathrectangle{\pgfqpoint{0.765000in}{0.660000in}}{\pgfqpoint{4.620000in}{4.620000in}}%
\pgfusepath{clip}%
\pgfsetroundcap%
\pgfsetroundjoin%
\pgfsetlinewidth{1.204500pt}%
\definecolor{currentstroke}{rgb}{1.000000,0.576471,0.309804}%
\pgfsetstrokecolor{currentstroke}%
\pgfsetdash{}{0pt}%
\pgfpathmoveto{\pgfqpoint{3.512226in}{2.181202in}}%
\pgfusepath{stroke}%
\end{pgfscope}%
\begin{pgfscope}%
\pgfpathrectangle{\pgfqpoint{0.765000in}{0.660000in}}{\pgfqpoint{4.620000in}{4.620000in}}%
\pgfusepath{clip}%
\pgfsetbuttcap%
\pgfsetroundjoin%
\definecolor{currentfill}{rgb}{1.000000,0.576471,0.309804}%
\pgfsetfillcolor{currentfill}%
\pgfsetlinewidth{1.003750pt}%
\definecolor{currentstroke}{rgb}{1.000000,0.576471,0.309804}%
\pgfsetstrokecolor{currentstroke}%
\pgfsetdash{}{0pt}%
\pgfsys@defobject{currentmarker}{\pgfqpoint{-0.033333in}{-0.033333in}}{\pgfqpoint{0.033333in}{0.033333in}}{%
\pgfpathmoveto{\pgfqpoint{0.000000in}{-0.033333in}}%
\pgfpathcurveto{\pgfqpoint{0.008840in}{-0.033333in}}{\pgfqpoint{0.017319in}{-0.029821in}}{\pgfqpoint{0.023570in}{-0.023570in}}%
\pgfpathcurveto{\pgfqpoint{0.029821in}{-0.017319in}}{\pgfqpoint{0.033333in}{-0.008840in}}{\pgfqpoint{0.033333in}{0.000000in}}%
\pgfpathcurveto{\pgfqpoint{0.033333in}{0.008840in}}{\pgfqpoint{0.029821in}{0.017319in}}{\pgfqpoint{0.023570in}{0.023570in}}%
\pgfpathcurveto{\pgfqpoint{0.017319in}{0.029821in}}{\pgfqpoint{0.008840in}{0.033333in}}{\pgfqpoint{0.000000in}{0.033333in}}%
\pgfpathcurveto{\pgfqpoint{-0.008840in}{0.033333in}}{\pgfqpoint{-0.017319in}{0.029821in}}{\pgfqpoint{-0.023570in}{0.023570in}}%
\pgfpathcurveto{\pgfqpoint{-0.029821in}{0.017319in}}{\pgfqpoint{-0.033333in}{0.008840in}}{\pgfqpoint{-0.033333in}{0.000000in}}%
\pgfpathcurveto{\pgfqpoint{-0.033333in}{-0.008840in}}{\pgfqpoint{-0.029821in}{-0.017319in}}{\pgfqpoint{-0.023570in}{-0.023570in}}%
\pgfpathcurveto{\pgfqpoint{-0.017319in}{-0.029821in}}{\pgfqpoint{-0.008840in}{-0.033333in}}{\pgfqpoint{0.000000in}{-0.033333in}}%
\pgfpathlineto{\pgfqpoint{0.000000in}{-0.033333in}}%
\pgfpathclose%
\pgfusepath{stroke,fill}%
}%
\begin{pgfscope}%
\pgfsys@transformshift{3.512226in}{2.181202in}%
\pgfsys@useobject{currentmarker}{}%
\end{pgfscope}%
\end{pgfscope}%
\begin{pgfscope}%
\pgfpathrectangle{\pgfqpoint{0.765000in}{0.660000in}}{\pgfqpoint{4.620000in}{4.620000in}}%
\pgfusepath{clip}%
\pgfsetroundcap%
\pgfsetroundjoin%
\pgfsetlinewidth{1.204500pt}%
\definecolor{currentstroke}{rgb}{1.000000,0.576471,0.309804}%
\pgfsetstrokecolor{currentstroke}%
\pgfsetdash{}{0pt}%
\pgfpathmoveto{\pgfqpoint{4.088832in}{2.343341in}}%
\pgfusepath{stroke}%
\end{pgfscope}%
\begin{pgfscope}%
\pgfpathrectangle{\pgfqpoint{0.765000in}{0.660000in}}{\pgfqpoint{4.620000in}{4.620000in}}%
\pgfusepath{clip}%
\pgfsetbuttcap%
\pgfsetroundjoin%
\definecolor{currentfill}{rgb}{1.000000,0.576471,0.309804}%
\pgfsetfillcolor{currentfill}%
\pgfsetlinewidth{1.003750pt}%
\definecolor{currentstroke}{rgb}{1.000000,0.576471,0.309804}%
\pgfsetstrokecolor{currentstroke}%
\pgfsetdash{}{0pt}%
\pgfsys@defobject{currentmarker}{\pgfqpoint{-0.033333in}{-0.033333in}}{\pgfqpoint{0.033333in}{0.033333in}}{%
\pgfpathmoveto{\pgfqpoint{0.000000in}{-0.033333in}}%
\pgfpathcurveto{\pgfqpoint{0.008840in}{-0.033333in}}{\pgfqpoint{0.017319in}{-0.029821in}}{\pgfqpoint{0.023570in}{-0.023570in}}%
\pgfpathcurveto{\pgfqpoint{0.029821in}{-0.017319in}}{\pgfqpoint{0.033333in}{-0.008840in}}{\pgfqpoint{0.033333in}{0.000000in}}%
\pgfpathcurveto{\pgfqpoint{0.033333in}{0.008840in}}{\pgfqpoint{0.029821in}{0.017319in}}{\pgfqpoint{0.023570in}{0.023570in}}%
\pgfpathcurveto{\pgfqpoint{0.017319in}{0.029821in}}{\pgfqpoint{0.008840in}{0.033333in}}{\pgfqpoint{0.000000in}{0.033333in}}%
\pgfpathcurveto{\pgfqpoint{-0.008840in}{0.033333in}}{\pgfqpoint{-0.017319in}{0.029821in}}{\pgfqpoint{-0.023570in}{0.023570in}}%
\pgfpathcurveto{\pgfqpoint{-0.029821in}{0.017319in}}{\pgfqpoint{-0.033333in}{0.008840in}}{\pgfqpoint{-0.033333in}{0.000000in}}%
\pgfpathcurveto{\pgfqpoint{-0.033333in}{-0.008840in}}{\pgfqpoint{-0.029821in}{-0.017319in}}{\pgfqpoint{-0.023570in}{-0.023570in}}%
\pgfpathcurveto{\pgfqpoint{-0.017319in}{-0.029821in}}{\pgfqpoint{-0.008840in}{-0.033333in}}{\pgfqpoint{0.000000in}{-0.033333in}}%
\pgfpathlineto{\pgfqpoint{0.000000in}{-0.033333in}}%
\pgfpathclose%
\pgfusepath{stroke,fill}%
}%
\begin{pgfscope}%
\pgfsys@transformshift{4.088832in}{2.343341in}%
\pgfsys@useobject{currentmarker}{}%
\end{pgfscope}%
\end{pgfscope}%
\begin{pgfscope}%
\pgfpathrectangle{\pgfqpoint{0.765000in}{0.660000in}}{\pgfqpoint{4.620000in}{4.620000in}}%
\pgfusepath{clip}%
\pgfsetroundcap%
\pgfsetroundjoin%
\pgfsetlinewidth{1.204500pt}%
\definecolor{currentstroke}{rgb}{1.000000,0.576471,0.309804}%
\pgfsetstrokecolor{currentstroke}%
\pgfsetdash{}{0pt}%
\pgfpathmoveto{\pgfqpoint{3.512226in}{2.343341in}}%
\pgfusepath{stroke}%
\end{pgfscope}%
\begin{pgfscope}%
\pgfpathrectangle{\pgfqpoint{0.765000in}{0.660000in}}{\pgfqpoint{4.620000in}{4.620000in}}%
\pgfusepath{clip}%
\pgfsetbuttcap%
\pgfsetroundjoin%
\definecolor{currentfill}{rgb}{1.000000,0.576471,0.309804}%
\pgfsetfillcolor{currentfill}%
\pgfsetlinewidth{1.003750pt}%
\definecolor{currentstroke}{rgb}{1.000000,0.576471,0.309804}%
\pgfsetstrokecolor{currentstroke}%
\pgfsetdash{}{0pt}%
\pgfsys@defobject{currentmarker}{\pgfqpoint{-0.033333in}{-0.033333in}}{\pgfqpoint{0.033333in}{0.033333in}}{%
\pgfpathmoveto{\pgfqpoint{0.000000in}{-0.033333in}}%
\pgfpathcurveto{\pgfqpoint{0.008840in}{-0.033333in}}{\pgfqpoint{0.017319in}{-0.029821in}}{\pgfqpoint{0.023570in}{-0.023570in}}%
\pgfpathcurveto{\pgfqpoint{0.029821in}{-0.017319in}}{\pgfqpoint{0.033333in}{-0.008840in}}{\pgfqpoint{0.033333in}{0.000000in}}%
\pgfpathcurveto{\pgfqpoint{0.033333in}{0.008840in}}{\pgfqpoint{0.029821in}{0.017319in}}{\pgfqpoint{0.023570in}{0.023570in}}%
\pgfpathcurveto{\pgfqpoint{0.017319in}{0.029821in}}{\pgfqpoint{0.008840in}{0.033333in}}{\pgfqpoint{0.000000in}{0.033333in}}%
\pgfpathcurveto{\pgfqpoint{-0.008840in}{0.033333in}}{\pgfqpoint{-0.017319in}{0.029821in}}{\pgfqpoint{-0.023570in}{0.023570in}}%
\pgfpathcurveto{\pgfqpoint{-0.029821in}{0.017319in}}{\pgfqpoint{-0.033333in}{0.008840in}}{\pgfqpoint{-0.033333in}{0.000000in}}%
\pgfpathcurveto{\pgfqpoint{-0.033333in}{-0.008840in}}{\pgfqpoint{-0.029821in}{-0.017319in}}{\pgfqpoint{-0.023570in}{-0.023570in}}%
\pgfpathcurveto{\pgfqpoint{-0.017319in}{-0.029821in}}{\pgfqpoint{-0.008840in}{-0.033333in}}{\pgfqpoint{0.000000in}{-0.033333in}}%
\pgfpathlineto{\pgfqpoint{0.000000in}{-0.033333in}}%
\pgfpathclose%
\pgfusepath{stroke,fill}%
}%
\begin{pgfscope}%
\pgfsys@transformshift{3.512226in}{2.343341in}%
\pgfsys@useobject{currentmarker}{}%
\end{pgfscope}%
\end{pgfscope}%
\begin{pgfscope}%
\pgfpathrectangle{\pgfqpoint{0.765000in}{0.660000in}}{\pgfqpoint{4.620000in}{4.620000in}}%
\pgfusepath{clip}%
\pgfsetroundcap%
\pgfsetroundjoin%
\pgfsetlinewidth{1.204500pt}%
\definecolor{currentstroke}{rgb}{1.000000,0.576471,0.309804}%
\pgfsetstrokecolor{currentstroke}%
\pgfsetdash{}{0pt}%
\pgfpathmoveto{\pgfqpoint{2.993281in}{2.262271in}}%
\pgfusepath{stroke}%
\end{pgfscope}%
\begin{pgfscope}%
\pgfpathrectangle{\pgfqpoint{0.765000in}{0.660000in}}{\pgfqpoint{4.620000in}{4.620000in}}%
\pgfusepath{clip}%
\pgfsetbuttcap%
\pgfsetroundjoin%
\definecolor{currentfill}{rgb}{1.000000,0.576471,0.309804}%
\pgfsetfillcolor{currentfill}%
\pgfsetlinewidth{1.003750pt}%
\definecolor{currentstroke}{rgb}{1.000000,0.576471,0.309804}%
\pgfsetstrokecolor{currentstroke}%
\pgfsetdash{}{0pt}%
\pgfsys@defobject{currentmarker}{\pgfqpoint{-0.033333in}{-0.033333in}}{\pgfqpoint{0.033333in}{0.033333in}}{%
\pgfpathmoveto{\pgfqpoint{0.000000in}{-0.033333in}}%
\pgfpathcurveto{\pgfqpoint{0.008840in}{-0.033333in}}{\pgfqpoint{0.017319in}{-0.029821in}}{\pgfqpoint{0.023570in}{-0.023570in}}%
\pgfpathcurveto{\pgfqpoint{0.029821in}{-0.017319in}}{\pgfqpoint{0.033333in}{-0.008840in}}{\pgfqpoint{0.033333in}{0.000000in}}%
\pgfpathcurveto{\pgfqpoint{0.033333in}{0.008840in}}{\pgfqpoint{0.029821in}{0.017319in}}{\pgfqpoint{0.023570in}{0.023570in}}%
\pgfpathcurveto{\pgfqpoint{0.017319in}{0.029821in}}{\pgfqpoint{0.008840in}{0.033333in}}{\pgfqpoint{0.000000in}{0.033333in}}%
\pgfpathcurveto{\pgfqpoint{-0.008840in}{0.033333in}}{\pgfqpoint{-0.017319in}{0.029821in}}{\pgfqpoint{-0.023570in}{0.023570in}}%
\pgfpathcurveto{\pgfqpoint{-0.029821in}{0.017319in}}{\pgfqpoint{-0.033333in}{0.008840in}}{\pgfqpoint{-0.033333in}{0.000000in}}%
\pgfpathcurveto{\pgfqpoint{-0.033333in}{-0.008840in}}{\pgfqpoint{-0.029821in}{-0.017319in}}{\pgfqpoint{-0.023570in}{-0.023570in}}%
\pgfpathcurveto{\pgfqpoint{-0.017319in}{-0.029821in}}{\pgfqpoint{-0.008840in}{-0.033333in}}{\pgfqpoint{0.000000in}{-0.033333in}}%
\pgfpathlineto{\pgfqpoint{0.000000in}{-0.033333in}}%
\pgfpathclose%
\pgfusepath{stroke,fill}%
}%
\begin{pgfscope}%
\pgfsys@transformshift{2.993281in}{2.262271in}%
\pgfsys@useobject{currentmarker}{}%
\end{pgfscope}%
\end{pgfscope}%
\begin{pgfscope}%
\pgfpathrectangle{\pgfqpoint{0.765000in}{0.660000in}}{\pgfqpoint{4.620000in}{4.620000in}}%
\pgfusepath{clip}%
\pgfsetroundcap%
\pgfsetroundjoin%
\pgfsetlinewidth{1.204500pt}%
\definecolor{currentstroke}{rgb}{1.000000,0.576471,0.309804}%
\pgfsetstrokecolor{currentstroke}%
\pgfsetdash{}{0pt}%
\pgfpathmoveto{\pgfqpoint{3.339245in}{2.262271in}}%
\pgfusepath{stroke}%
\end{pgfscope}%
\begin{pgfscope}%
\pgfpathrectangle{\pgfqpoint{0.765000in}{0.660000in}}{\pgfqpoint{4.620000in}{4.620000in}}%
\pgfusepath{clip}%
\pgfsetbuttcap%
\pgfsetroundjoin%
\definecolor{currentfill}{rgb}{1.000000,0.576471,0.309804}%
\pgfsetfillcolor{currentfill}%
\pgfsetlinewidth{1.003750pt}%
\definecolor{currentstroke}{rgb}{1.000000,0.576471,0.309804}%
\pgfsetstrokecolor{currentstroke}%
\pgfsetdash{}{0pt}%
\pgfsys@defobject{currentmarker}{\pgfqpoint{-0.033333in}{-0.033333in}}{\pgfqpoint{0.033333in}{0.033333in}}{%
\pgfpathmoveto{\pgfqpoint{0.000000in}{-0.033333in}}%
\pgfpathcurveto{\pgfqpoint{0.008840in}{-0.033333in}}{\pgfqpoint{0.017319in}{-0.029821in}}{\pgfqpoint{0.023570in}{-0.023570in}}%
\pgfpathcurveto{\pgfqpoint{0.029821in}{-0.017319in}}{\pgfqpoint{0.033333in}{-0.008840in}}{\pgfqpoint{0.033333in}{0.000000in}}%
\pgfpathcurveto{\pgfqpoint{0.033333in}{0.008840in}}{\pgfqpoint{0.029821in}{0.017319in}}{\pgfqpoint{0.023570in}{0.023570in}}%
\pgfpathcurveto{\pgfqpoint{0.017319in}{0.029821in}}{\pgfqpoint{0.008840in}{0.033333in}}{\pgfqpoint{0.000000in}{0.033333in}}%
\pgfpathcurveto{\pgfqpoint{-0.008840in}{0.033333in}}{\pgfqpoint{-0.017319in}{0.029821in}}{\pgfqpoint{-0.023570in}{0.023570in}}%
\pgfpathcurveto{\pgfqpoint{-0.029821in}{0.017319in}}{\pgfqpoint{-0.033333in}{0.008840in}}{\pgfqpoint{-0.033333in}{0.000000in}}%
\pgfpathcurveto{\pgfqpoint{-0.033333in}{-0.008840in}}{\pgfqpoint{-0.029821in}{-0.017319in}}{\pgfqpoint{-0.023570in}{-0.023570in}}%
\pgfpathcurveto{\pgfqpoint{-0.017319in}{-0.029821in}}{\pgfqpoint{-0.008840in}{-0.033333in}}{\pgfqpoint{0.000000in}{-0.033333in}}%
\pgfpathlineto{\pgfqpoint{0.000000in}{-0.033333in}}%
\pgfpathclose%
\pgfusepath{stroke,fill}%
}%
\begin{pgfscope}%
\pgfsys@transformshift{3.339245in}{2.262271in}%
\pgfsys@useobject{currentmarker}{}%
\end{pgfscope}%
\end{pgfscope}%
\begin{pgfscope}%
\pgfpathrectangle{\pgfqpoint{0.765000in}{0.660000in}}{\pgfqpoint{4.620000in}{4.620000in}}%
\pgfusepath{clip}%
\pgfsetroundcap%
\pgfsetroundjoin%
\pgfsetlinewidth{1.204500pt}%
\definecolor{currentstroke}{rgb}{1.000000,0.576471,0.309804}%
\pgfsetstrokecolor{currentstroke}%
\pgfsetdash{}{0pt}%
\pgfpathmoveto{\pgfqpoint{3.339245in}{2.262271in}}%
\pgfusepath{stroke}%
\end{pgfscope}%
\begin{pgfscope}%
\pgfpathrectangle{\pgfqpoint{0.765000in}{0.660000in}}{\pgfqpoint{4.620000in}{4.620000in}}%
\pgfusepath{clip}%
\pgfsetbuttcap%
\pgfsetroundjoin%
\definecolor{currentfill}{rgb}{1.000000,0.576471,0.309804}%
\pgfsetfillcolor{currentfill}%
\pgfsetlinewidth{1.003750pt}%
\definecolor{currentstroke}{rgb}{1.000000,0.576471,0.309804}%
\pgfsetstrokecolor{currentstroke}%
\pgfsetdash{}{0pt}%
\pgfsys@defobject{currentmarker}{\pgfqpoint{-0.033333in}{-0.033333in}}{\pgfqpoint{0.033333in}{0.033333in}}{%
\pgfpathmoveto{\pgfqpoint{0.000000in}{-0.033333in}}%
\pgfpathcurveto{\pgfqpoint{0.008840in}{-0.033333in}}{\pgfqpoint{0.017319in}{-0.029821in}}{\pgfqpoint{0.023570in}{-0.023570in}}%
\pgfpathcurveto{\pgfqpoint{0.029821in}{-0.017319in}}{\pgfqpoint{0.033333in}{-0.008840in}}{\pgfqpoint{0.033333in}{0.000000in}}%
\pgfpathcurveto{\pgfqpoint{0.033333in}{0.008840in}}{\pgfqpoint{0.029821in}{0.017319in}}{\pgfqpoint{0.023570in}{0.023570in}}%
\pgfpathcurveto{\pgfqpoint{0.017319in}{0.029821in}}{\pgfqpoint{0.008840in}{0.033333in}}{\pgfqpoint{0.000000in}{0.033333in}}%
\pgfpathcurveto{\pgfqpoint{-0.008840in}{0.033333in}}{\pgfqpoint{-0.017319in}{0.029821in}}{\pgfqpoint{-0.023570in}{0.023570in}}%
\pgfpathcurveto{\pgfqpoint{-0.029821in}{0.017319in}}{\pgfqpoint{-0.033333in}{0.008840in}}{\pgfqpoint{-0.033333in}{0.000000in}}%
\pgfpathcurveto{\pgfqpoint{-0.033333in}{-0.008840in}}{\pgfqpoint{-0.029821in}{-0.017319in}}{\pgfqpoint{-0.023570in}{-0.023570in}}%
\pgfpathcurveto{\pgfqpoint{-0.017319in}{-0.029821in}}{\pgfqpoint{-0.008840in}{-0.033333in}}{\pgfqpoint{0.000000in}{-0.033333in}}%
\pgfpathlineto{\pgfqpoint{0.000000in}{-0.033333in}}%
\pgfpathclose%
\pgfusepath{stroke,fill}%
}%
\begin{pgfscope}%
\pgfsys@transformshift{3.339245in}{2.262271in}%
\pgfsys@useobject{currentmarker}{}%
\end{pgfscope}%
\end{pgfscope}%
\begin{pgfscope}%
\pgfpathrectangle{\pgfqpoint{0.765000in}{0.660000in}}{\pgfqpoint{4.620000in}{4.620000in}}%
\pgfusepath{clip}%
\pgfsetroundcap%
\pgfsetroundjoin%
\pgfsetlinewidth{1.204500pt}%
\definecolor{currentstroke}{rgb}{1.000000,0.576471,0.309804}%
\pgfsetstrokecolor{currentstroke}%
\pgfsetdash{}{0pt}%
\pgfpathmoveto{\pgfqpoint{2.820299in}{2.181202in}}%
\pgfusepath{stroke}%
\end{pgfscope}%
\begin{pgfscope}%
\pgfpathrectangle{\pgfqpoint{0.765000in}{0.660000in}}{\pgfqpoint{4.620000in}{4.620000in}}%
\pgfusepath{clip}%
\pgfsetbuttcap%
\pgfsetroundjoin%
\definecolor{currentfill}{rgb}{1.000000,0.576471,0.309804}%
\pgfsetfillcolor{currentfill}%
\pgfsetlinewidth{1.003750pt}%
\definecolor{currentstroke}{rgb}{1.000000,0.576471,0.309804}%
\pgfsetstrokecolor{currentstroke}%
\pgfsetdash{}{0pt}%
\pgfsys@defobject{currentmarker}{\pgfqpoint{-0.033333in}{-0.033333in}}{\pgfqpoint{0.033333in}{0.033333in}}{%
\pgfpathmoveto{\pgfqpoint{0.000000in}{-0.033333in}}%
\pgfpathcurveto{\pgfqpoint{0.008840in}{-0.033333in}}{\pgfqpoint{0.017319in}{-0.029821in}}{\pgfqpoint{0.023570in}{-0.023570in}}%
\pgfpathcurveto{\pgfqpoint{0.029821in}{-0.017319in}}{\pgfqpoint{0.033333in}{-0.008840in}}{\pgfqpoint{0.033333in}{0.000000in}}%
\pgfpathcurveto{\pgfqpoint{0.033333in}{0.008840in}}{\pgfqpoint{0.029821in}{0.017319in}}{\pgfqpoint{0.023570in}{0.023570in}}%
\pgfpathcurveto{\pgfqpoint{0.017319in}{0.029821in}}{\pgfqpoint{0.008840in}{0.033333in}}{\pgfqpoint{0.000000in}{0.033333in}}%
\pgfpathcurveto{\pgfqpoint{-0.008840in}{0.033333in}}{\pgfqpoint{-0.017319in}{0.029821in}}{\pgfqpoint{-0.023570in}{0.023570in}}%
\pgfpathcurveto{\pgfqpoint{-0.029821in}{0.017319in}}{\pgfqpoint{-0.033333in}{0.008840in}}{\pgfqpoint{-0.033333in}{0.000000in}}%
\pgfpathcurveto{\pgfqpoint{-0.033333in}{-0.008840in}}{\pgfqpoint{-0.029821in}{-0.017319in}}{\pgfqpoint{-0.023570in}{-0.023570in}}%
\pgfpathcurveto{\pgfqpoint{-0.017319in}{-0.029821in}}{\pgfqpoint{-0.008840in}{-0.033333in}}{\pgfqpoint{0.000000in}{-0.033333in}}%
\pgfpathlineto{\pgfqpoint{0.000000in}{-0.033333in}}%
\pgfpathclose%
\pgfusepath{stroke,fill}%
}%
\begin{pgfscope}%
\pgfsys@transformshift{2.820299in}{2.181202in}%
\pgfsys@useobject{currentmarker}{}%
\end{pgfscope}%
\end{pgfscope}%
\begin{pgfscope}%
\pgfpathrectangle{\pgfqpoint{0.765000in}{0.660000in}}{\pgfqpoint{4.620000in}{4.620000in}}%
\pgfusepath{clip}%
\pgfsetroundcap%
\pgfsetroundjoin%
\pgfsetlinewidth{1.204500pt}%
\definecolor{currentstroke}{rgb}{1.000000,0.576471,0.309804}%
\pgfsetstrokecolor{currentstroke}%
\pgfsetdash{}{0pt}%
\pgfpathmoveto{\pgfqpoint{3.281584in}{2.343341in}}%
\pgfusepath{stroke}%
\end{pgfscope}%
\begin{pgfscope}%
\pgfpathrectangle{\pgfqpoint{0.765000in}{0.660000in}}{\pgfqpoint{4.620000in}{4.620000in}}%
\pgfusepath{clip}%
\pgfsetbuttcap%
\pgfsetroundjoin%
\definecolor{currentfill}{rgb}{1.000000,0.576471,0.309804}%
\pgfsetfillcolor{currentfill}%
\pgfsetlinewidth{1.003750pt}%
\definecolor{currentstroke}{rgb}{1.000000,0.576471,0.309804}%
\pgfsetstrokecolor{currentstroke}%
\pgfsetdash{}{0pt}%
\pgfsys@defobject{currentmarker}{\pgfqpoint{-0.033333in}{-0.033333in}}{\pgfqpoint{0.033333in}{0.033333in}}{%
\pgfpathmoveto{\pgfqpoint{0.000000in}{-0.033333in}}%
\pgfpathcurveto{\pgfqpoint{0.008840in}{-0.033333in}}{\pgfqpoint{0.017319in}{-0.029821in}}{\pgfqpoint{0.023570in}{-0.023570in}}%
\pgfpathcurveto{\pgfqpoint{0.029821in}{-0.017319in}}{\pgfqpoint{0.033333in}{-0.008840in}}{\pgfqpoint{0.033333in}{0.000000in}}%
\pgfpathcurveto{\pgfqpoint{0.033333in}{0.008840in}}{\pgfqpoint{0.029821in}{0.017319in}}{\pgfqpoint{0.023570in}{0.023570in}}%
\pgfpathcurveto{\pgfqpoint{0.017319in}{0.029821in}}{\pgfqpoint{0.008840in}{0.033333in}}{\pgfqpoint{0.000000in}{0.033333in}}%
\pgfpathcurveto{\pgfqpoint{-0.008840in}{0.033333in}}{\pgfqpoint{-0.017319in}{0.029821in}}{\pgfqpoint{-0.023570in}{0.023570in}}%
\pgfpathcurveto{\pgfqpoint{-0.029821in}{0.017319in}}{\pgfqpoint{-0.033333in}{0.008840in}}{\pgfqpoint{-0.033333in}{0.000000in}}%
\pgfpathcurveto{\pgfqpoint{-0.033333in}{-0.008840in}}{\pgfqpoint{-0.029821in}{-0.017319in}}{\pgfqpoint{-0.023570in}{-0.023570in}}%
\pgfpathcurveto{\pgfqpoint{-0.017319in}{-0.029821in}}{\pgfqpoint{-0.008840in}{-0.033333in}}{\pgfqpoint{0.000000in}{-0.033333in}}%
\pgfpathlineto{\pgfqpoint{0.000000in}{-0.033333in}}%
\pgfpathclose%
\pgfusepath{stroke,fill}%
}%
\begin{pgfscope}%
\pgfsys@transformshift{3.281584in}{2.343341in}%
\pgfsys@useobject{currentmarker}{}%
\end{pgfscope}%
\end{pgfscope}%
\begin{pgfscope}%
\pgfpathrectangle{\pgfqpoint{0.765000in}{0.660000in}}{\pgfqpoint{4.620000in}{4.620000in}}%
\pgfusepath{clip}%
\pgfsetroundcap%
\pgfsetroundjoin%
\pgfsetlinewidth{1.204500pt}%
\definecolor{currentstroke}{rgb}{1.000000,0.576471,0.309804}%
\pgfsetstrokecolor{currentstroke}%
\pgfsetdash{}{0pt}%
\pgfpathmoveto{\pgfqpoint{3.050942in}{2.181202in}}%
\pgfusepath{stroke}%
\end{pgfscope}%
\begin{pgfscope}%
\pgfpathrectangle{\pgfqpoint{0.765000in}{0.660000in}}{\pgfqpoint{4.620000in}{4.620000in}}%
\pgfusepath{clip}%
\pgfsetbuttcap%
\pgfsetroundjoin%
\definecolor{currentfill}{rgb}{1.000000,0.576471,0.309804}%
\pgfsetfillcolor{currentfill}%
\pgfsetlinewidth{1.003750pt}%
\definecolor{currentstroke}{rgb}{1.000000,0.576471,0.309804}%
\pgfsetstrokecolor{currentstroke}%
\pgfsetdash{}{0pt}%
\pgfsys@defobject{currentmarker}{\pgfqpoint{-0.033333in}{-0.033333in}}{\pgfqpoint{0.033333in}{0.033333in}}{%
\pgfpathmoveto{\pgfqpoint{0.000000in}{-0.033333in}}%
\pgfpathcurveto{\pgfqpoint{0.008840in}{-0.033333in}}{\pgfqpoint{0.017319in}{-0.029821in}}{\pgfqpoint{0.023570in}{-0.023570in}}%
\pgfpathcurveto{\pgfqpoint{0.029821in}{-0.017319in}}{\pgfqpoint{0.033333in}{-0.008840in}}{\pgfqpoint{0.033333in}{0.000000in}}%
\pgfpathcurveto{\pgfqpoint{0.033333in}{0.008840in}}{\pgfqpoint{0.029821in}{0.017319in}}{\pgfqpoint{0.023570in}{0.023570in}}%
\pgfpathcurveto{\pgfqpoint{0.017319in}{0.029821in}}{\pgfqpoint{0.008840in}{0.033333in}}{\pgfqpoint{0.000000in}{0.033333in}}%
\pgfpathcurveto{\pgfqpoint{-0.008840in}{0.033333in}}{\pgfqpoint{-0.017319in}{0.029821in}}{\pgfqpoint{-0.023570in}{0.023570in}}%
\pgfpathcurveto{\pgfqpoint{-0.029821in}{0.017319in}}{\pgfqpoint{-0.033333in}{0.008840in}}{\pgfqpoint{-0.033333in}{0.000000in}}%
\pgfpathcurveto{\pgfqpoint{-0.033333in}{-0.008840in}}{\pgfqpoint{-0.029821in}{-0.017319in}}{\pgfqpoint{-0.023570in}{-0.023570in}}%
\pgfpathcurveto{\pgfqpoint{-0.017319in}{-0.029821in}}{\pgfqpoint{-0.008840in}{-0.033333in}}{\pgfqpoint{0.000000in}{-0.033333in}}%
\pgfpathlineto{\pgfqpoint{0.000000in}{-0.033333in}}%
\pgfpathclose%
\pgfusepath{stroke,fill}%
}%
\begin{pgfscope}%
\pgfsys@transformshift{3.050942in}{2.181202in}%
\pgfsys@useobject{currentmarker}{}%
\end{pgfscope}%
\end{pgfscope}%
\begin{pgfscope}%
\pgfpathrectangle{\pgfqpoint{0.765000in}{0.660000in}}{\pgfqpoint{4.620000in}{4.620000in}}%
\pgfusepath{clip}%
\pgfsetroundcap%
\pgfsetroundjoin%
\pgfsetlinewidth{1.204500pt}%
\definecolor{currentstroke}{rgb}{1.000000,0.576471,0.309804}%
\pgfsetstrokecolor{currentstroke}%
\pgfsetdash{}{0pt}%
\pgfpathmoveto{\pgfqpoint{2.877960in}{2.424411in}}%
\pgfusepath{stroke}%
\end{pgfscope}%
\begin{pgfscope}%
\pgfpathrectangle{\pgfqpoint{0.765000in}{0.660000in}}{\pgfqpoint{4.620000in}{4.620000in}}%
\pgfusepath{clip}%
\pgfsetbuttcap%
\pgfsetroundjoin%
\definecolor{currentfill}{rgb}{1.000000,0.576471,0.309804}%
\pgfsetfillcolor{currentfill}%
\pgfsetlinewidth{1.003750pt}%
\definecolor{currentstroke}{rgb}{1.000000,0.576471,0.309804}%
\pgfsetstrokecolor{currentstroke}%
\pgfsetdash{}{0pt}%
\pgfsys@defobject{currentmarker}{\pgfqpoint{-0.033333in}{-0.033333in}}{\pgfqpoint{0.033333in}{0.033333in}}{%
\pgfpathmoveto{\pgfqpoint{0.000000in}{-0.033333in}}%
\pgfpathcurveto{\pgfqpoint{0.008840in}{-0.033333in}}{\pgfqpoint{0.017319in}{-0.029821in}}{\pgfqpoint{0.023570in}{-0.023570in}}%
\pgfpathcurveto{\pgfqpoint{0.029821in}{-0.017319in}}{\pgfqpoint{0.033333in}{-0.008840in}}{\pgfqpoint{0.033333in}{0.000000in}}%
\pgfpathcurveto{\pgfqpoint{0.033333in}{0.008840in}}{\pgfqpoint{0.029821in}{0.017319in}}{\pgfqpoint{0.023570in}{0.023570in}}%
\pgfpathcurveto{\pgfqpoint{0.017319in}{0.029821in}}{\pgfqpoint{0.008840in}{0.033333in}}{\pgfqpoint{0.000000in}{0.033333in}}%
\pgfpathcurveto{\pgfqpoint{-0.008840in}{0.033333in}}{\pgfqpoint{-0.017319in}{0.029821in}}{\pgfqpoint{-0.023570in}{0.023570in}}%
\pgfpathcurveto{\pgfqpoint{-0.029821in}{0.017319in}}{\pgfqpoint{-0.033333in}{0.008840in}}{\pgfqpoint{-0.033333in}{0.000000in}}%
\pgfpathcurveto{\pgfqpoint{-0.033333in}{-0.008840in}}{\pgfqpoint{-0.029821in}{-0.017319in}}{\pgfqpoint{-0.023570in}{-0.023570in}}%
\pgfpathcurveto{\pgfqpoint{-0.017319in}{-0.029821in}}{\pgfqpoint{-0.008840in}{-0.033333in}}{\pgfqpoint{0.000000in}{-0.033333in}}%
\pgfpathlineto{\pgfqpoint{0.000000in}{-0.033333in}}%
\pgfpathclose%
\pgfusepath{stroke,fill}%
}%
\begin{pgfscope}%
\pgfsys@transformshift{2.877960in}{2.424411in}%
\pgfsys@useobject{currentmarker}{}%
\end{pgfscope}%
\end{pgfscope}%
\begin{pgfscope}%
\pgfpathrectangle{\pgfqpoint{0.765000in}{0.660000in}}{\pgfqpoint{4.620000in}{4.620000in}}%
\pgfusepath{clip}%
\pgfsetroundcap%
\pgfsetroundjoin%
\pgfsetlinewidth{1.204500pt}%
\definecolor{currentstroke}{rgb}{1.000000,0.576471,0.309804}%
\pgfsetstrokecolor{currentstroke}%
\pgfsetdash{}{0pt}%
\pgfpathmoveto{\pgfqpoint{2.820299in}{2.181202in}}%
\pgfusepath{stroke}%
\end{pgfscope}%
\begin{pgfscope}%
\pgfpathrectangle{\pgfqpoint{0.765000in}{0.660000in}}{\pgfqpoint{4.620000in}{4.620000in}}%
\pgfusepath{clip}%
\pgfsetbuttcap%
\pgfsetroundjoin%
\definecolor{currentfill}{rgb}{1.000000,0.576471,0.309804}%
\pgfsetfillcolor{currentfill}%
\pgfsetlinewidth{1.003750pt}%
\definecolor{currentstroke}{rgb}{1.000000,0.576471,0.309804}%
\pgfsetstrokecolor{currentstroke}%
\pgfsetdash{}{0pt}%
\pgfsys@defobject{currentmarker}{\pgfqpoint{-0.033333in}{-0.033333in}}{\pgfqpoint{0.033333in}{0.033333in}}{%
\pgfpathmoveto{\pgfqpoint{0.000000in}{-0.033333in}}%
\pgfpathcurveto{\pgfqpoint{0.008840in}{-0.033333in}}{\pgfqpoint{0.017319in}{-0.029821in}}{\pgfqpoint{0.023570in}{-0.023570in}}%
\pgfpathcurveto{\pgfqpoint{0.029821in}{-0.017319in}}{\pgfqpoint{0.033333in}{-0.008840in}}{\pgfqpoint{0.033333in}{0.000000in}}%
\pgfpathcurveto{\pgfqpoint{0.033333in}{0.008840in}}{\pgfqpoint{0.029821in}{0.017319in}}{\pgfqpoint{0.023570in}{0.023570in}}%
\pgfpathcurveto{\pgfqpoint{0.017319in}{0.029821in}}{\pgfqpoint{0.008840in}{0.033333in}}{\pgfqpoint{0.000000in}{0.033333in}}%
\pgfpathcurveto{\pgfqpoint{-0.008840in}{0.033333in}}{\pgfqpoint{-0.017319in}{0.029821in}}{\pgfqpoint{-0.023570in}{0.023570in}}%
\pgfpathcurveto{\pgfqpoint{-0.029821in}{0.017319in}}{\pgfqpoint{-0.033333in}{0.008840in}}{\pgfqpoint{-0.033333in}{0.000000in}}%
\pgfpathcurveto{\pgfqpoint{-0.033333in}{-0.008840in}}{\pgfqpoint{-0.029821in}{-0.017319in}}{\pgfqpoint{-0.023570in}{-0.023570in}}%
\pgfpathcurveto{\pgfqpoint{-0.017319in}{-0.029821in}}{\pgfqpoint{-0.008840in}{-0.033333in}}{\pgfqpoint{0.000000in}{-0.033333in}}%
\pgfpathlineto{\pgfqpoint{0.000000in}{-0.033333in}}%
\pgfpathclose%
\pgfusepath{stroke,fill}%
}%
\begin{pgfscope}%
\pgfsys@transformshift{2.820299in}{2.181202in}%
\pgfsys@useobject{currentmarker}{}%
\end{pgfscope}%
\end{pgfscope}%
\begin{pgfscope}%
\pgfpathrectangle{\pgfqpoint{0.765000in}{0.660000in}}{\pgfqpoint{4.620000in}{4.620000in}}%
\pgfusepath{clip}%
\pgfsetroundcap%
\pgfsetroundjoin%
\pgfsetlinewidth{1.204500pt}%
\definecolor{currentstroke}{rgb}{1.000000,0.576471,0.309804}%
\pgfsetstrokecolor{currentstroke}%
\pgfsetdash{}{0pt}%
\pgfpathmoveto{\pgfqpoint{2.359014in}{2.181202in}}%
\pgfusepath{stroke}%
\end{pgfscope}%
\begin{pgfscope}%
\pgfpathrectangle{\pgfqpoint{0.765000in}{0.660000in}}{\pgfqpoint{4.620000in}{4.620000in}}%
\pgfusepath{clip}%
\pgfsetbuttcap%
\pgfsetroundjoin%
\definecolor{currentfill}{rgb}{1.000000,0.576471,0.309804}%
\pgfsetfillcolor{currentfill}%
\pgfsetlinewidth{1.003750pt}%
\definecolor{currentstroke}{rgb}{1.000000,0.576471,0.309804}%
\pgfsetstrokecolor{currentstroke}%
\pgfsetdash{}{0pt}%
\pgfsys@defobject{currentmarker}{\pgfqpoint{-0.033333in}{-0.033333in}}{\pgfqpoint{0.033333in}{0.033333in}}{%
\pgfpathmoveto{\pgfqpoint{0.000000in}{-0.033333in}}%
\pgfpathcurveto{\pgfqpoint{0.008840in}{-0.033333in}}{\pgfqpoint{0.017319in}{-0.029821in}}{\pgfqpoint{0.023570in}{-0.023570in}}%
\pgfpathcurveto{\pgfqpoint{0.029821in}{-0.017319in}}{\pgfqpoint{0.033333in}{-0.008840in}}{\pgfqpoint{0.033333in}{0.000000in}}%
\pgfpathcurveto{\pgfqpoint{0.033333in}{0.008840in}}{\pgfqpoint{0.029821in}{0.017319in}}{\pgfqpoint{0.023570in}{0.023570in}}%
\pgfpathcurveto{\pgfqpoint{0.017319in}{0.029821in}}{\pgfqpoint{0.008840in}{0.033333in}}{\pgfqpoint{0.000000in}{0.033333in}}%
\pgfpathcurveto{\pgfqpoint{-0.008840in}{0.033333in}}{\pgfqpoint{-0.017319in}{0.029821in}}{\pgfqpoint{-0.023570in}{0.023570in}}%
\pgfpathcurveto{\pgfqpoint{-0.029821in}{0.017319in}}{\pgfqpoint{-0.033333in}{0.008840in}}{\pgfqpoint{-0.033333in}{0.000000in}}%
\pgfpathcurveto{\pgfqpoint{-0.033333in}{-0.008840in}}{\pgfqpoint{-0.029821in}{-0.017319in}}{\pgfqpoint{-0.023570in}{-0.023570in}}%
\pgfpathcurveto{\pgfqpoint{-0.017319in}{-0.029821in}}{\pgfqpoint{-0.008840in}{-0.033333in}}{\pgfqpoint{0.000000in}{-0.033333in}}%
\pgfpathlineto{\pgfqpoint{0.000000in}{-0.033333in}}%
\pgfpathclose%
\pgfusepath{stroke,fill}%
}%
\begin{pgfscope}%
\pgfsys@transformshift{2.359014in}{2.181202in}%
\pgfsys@useobject{currentmarker}{}%
\end{pgfscope}%
\end{pgfscope}%
\begin{pgfscope}%
\pgfpathrectangle{\pgfqpoint{0.765000in}{0.660000in}}{\pgfqpoint{4.620000in}{4.620000in}}%
\pgfusepath{clip}%
\pgfsetroundcap%
\pgfsetroundjoin%
\pgfsetlinewidth{1.204500pt}%
\definecolor{currentstroke}{rgb}{1.000000,0.576471,0.309804}%
\pgfsetstrokecolor{currentstroke}%
\pgfsetdash{}{0pt}%
\pgfpathmoveto{\pgfqpoint{2.935620in}{2.343341in}}%
\pgfusepath{stroke}%
\end{pgfscope}%
\begin{pgfscope}%
\pgfpathrectangle{\pgfqpoint{0.765000in}{0.660000in}}{\pgfqpoint{4.620000in}{4.620000in}}%
\pgfusepath{clip}%
\pgfsetbuttcap%
\pgfsetroundjoin%
\definecolor{currentfill}{rgb}{1.000000,0.576471,0.309804}%
\pgfsetfillcolor{currentfill}%
\pgfsetlinewidth{1.003750pt}%
\definecolor{currentstroke}{rgb}{1.000000,0.576471,0.309804}%
\pgfsetstrokecolor{currentstroke}%
\pgfsetdash{}{0pt}%
\pgfsys@defobject{currentmarker}{\pgfqpoint{-0.033333in}{-0.033333in}}{\pgfqpoint{0.033333in}{0.033333in}}{%
\pgfpathmoveto{\pgfqpoint{0.000000in}{-0.033333in}}%
\pgfpathcurveto{\pgfqpoint{0.008840in}{-0.033333in}}{\pgfqpoint{0.017319in}{-0.029821in}}{\pgfqpoint{0.023570in}{-0.023570in}}%
\pgfpathcurveto{\pgfqpoint{0.029821in}{-0.017319in}}{\pgfqpoint{0.033333in}{-0.008840in}}{\pgfqpoint{0.033333in}{0.000000in}}%
\pgfpathcurveto{\pgfqpoint{0.033333in}{0.008840in}}{\pgfqpoint{0.029821in}{0.017319in}}{\pgfqpoint{0.023570in}{0.023570in}}%
\pgfpathcurveto{\pgfqpoint{0.017319in}{0.029821in}}{\pgfqpoint{0.008840in}{0.033333in}}{\pgfqpoint{0.000000in}{0.033333in}}%
\pgfpathcurveto{\pgfqpoint{-0.008840in}{0.033333in}}{\pgfqpoint{-0.017319in}{0.029821in}}{\pgfqpoint{-0.023570in}{0.023570in}}%
\pgfpathcurveto{\pgfqpoint{-0.029821in}{0.017319in}}{\pgfqpoint{-0.033333in}{0.008840in}}{\pgfqpoint{-0.033333in}{0.000000in}}%
\pgfpathcurveto{\pgfqpoint{-0.033333in}{-0.008840in}}{\pgfqpoint{-0.029821in}{-0.017319in}}{\pgfqpoint{-0.023570in}{-0.023570in}}%
\pgfpathcurveto{\pgfqpoint{-0.017319in}{-0.029821in}}{\pgfqpoint{-0.008840in}{-0.033333in}}{\pgfqpoint{0.000000in}{-0.033333in}}%
\pgfpathlineto{\pgfqpoint{0.000000in}{-0.033333in}}%
\pgfpathclose%
\pgfusepath{stroke,fill}%
}%
\begin{pgfscope}%
\pgfsys@transformshift{2.935620in}{2.343341in}%
\pgfsys@useobject{currentmarker}{}%
\end{pgfscope}%
\end{pgfscope}%
\begin{pgfscope}%
\pgfpathrectangle{\pgfqpoint{0.765000in}{0.660000in}}{\pgfqpoint{4.620000in}{4.620000in}}%
\pgfusepath{clip}%
\pgfsetroundcap%
\pgfsetroundjoin%
\pgfsetlinewidth{1.204500pt}%
\definecolor{currentstroke}{rgb}{1.000000,0.576471,0.309804}%
\pgfsetstrokecolor{currentstroke}%
\pgfsetdash{}{0pt}%
\pgfpathmoveto{\pgfqpoint{2.935620in}{2.181202in}}%
\pgfusepath{stroke}%
\end{pgfscope}%
\begin{pgfscope}%
\pgfpathrectangle{\pgfqpoint{0.765000in}{0.660000in}}{\pgfqpoint{4.620000in}{4.620000in}}%
\pgfusepath{clip}%
\pgfsetbuttcap%
\pgfsetroundjoin%
\definecolor{currentfill}{rgb}{1.000000,0.576471,0.309804}%
\pgfsetfillcolor{currentfill}%
\pgfsetlinewidth{1.003750pt}%
\definecolor{currentstroke}{rgb}{1.000000,0.576471,0.309804}%
\pgfsetstrokecolor{currentstroke}%
\pgfsetdash{}{0pt}%
\pgfsys@defobject{currentmarker}{\pgfqpoint{-0.033333in}{-0.033333in}}{\pgfqpoint{0.033333in}{0.033333in}}{%
\pgfpathmoveto{\pgfqpoint{0.000000in}{-0.033333in}}%
\pgfpathcurveto{\pgfqpoint{0.008840in}{-0.033333in}}{\pgfqpoint{0.017319in}{-0.029821in}}{\pgfqpoint{0.023570in}{-0.023570in}}%
\pgfpathcurveto{\pgfqpoint{0.029821in}{-0.017319in}}{\pgfqpoint{0.033333in}{-0.008840in}}{\pgfqpoint{0.033333in}{0.000000in}}%
\pgfpathcurveto{\pgfqpoint{0.033333in}{0.008840in}}{\pgfqpoint{0.029821in}{0.017319in}}{\pgfqpoint{0.023570in}{0.023570in}}%
\pgfpathcurveto{\pgfqpoint{0.017319in}{0.029821in}}{\pgfqpoint{0.008840in}{0.033333in}}{\pgfqpoint{0.000000in}{0.033333in}}%
\pgfpathcurveto{\pgfqpoint{-0.008840in}{0.033333in}}{\pgfqpoint{-0.017319in}{0.029821in}}{\pgfqpoint{-0.023570in}{0.023570in}}%
\pgfpathcurveto{\pgfqpoint{-0.029821in}{0.017319in}}{\pgfqpoint{-0.033333in}{0.008840in}}{\pgfqpoint{-0.033333in}{0.000000in}}%
\pgfpathcurveto{\pgfqpoint{-0.033333in}{-0.008840in}}{\pgfqpoint{-0.029821in}{-0.017319in}}{\pgfqpoint{-0.023570in}{-0.023570in}}%
\pgfpathcurveto{\pgfqpoint{-0.017319in}{-0.029821in}}{\pgfqpoint{-0.008840in}{-0.033333in}}{\pgfqpoint{0.000000in}{-0.033333in}}%
\pgfpathlineto{\pgfqpoint{0.000000in}{-0.033333in}}%
\pgfpathclose%
\pgfusepath{stroke,fill}%
}%
\begin{pgfscope}%
\pgfsys@transformshift{2.935620in}{2.181202in}%
\pgfsys@useobject{currentmarker}{}%
\end{pgfscope}%
\end{pgfscope}%
\begin{pgfscope}%
\pgfpathrectangle{\pgfqpoint{0.765000in}{0.660000in}}{\pgfqpoint{4.620000in}{4.620000in}}%
\pgfusepath{clip}%
\pgfsetroundcap%
\pgfsetroundjoin%
\pgfsetlinewidth{1.204500pt}%
\definecolor{currentstroke}{rgb}{1.000000,0.576471,0.309804}%
\pgfsetstrokecolor{currentstroke}%
\pgfsetdash{}{0pt}%
\pgfpathmoveto{\pgfqpoint{2.820299in}{2.181202in}}%
\pgfusepath{stroke}%
\end{pgfscope}%
\begin{pgfscope}%
\pgfpathrectangle{\pgfqpoint{0.765000in}{0.660000in}}{\pgfqpoint{4.620000in}{4.620000in}}%
\pgfusepath{clip}%
\pgfsetbuttcap%
\pgfsetroundjoin%
\definecolor{currentfill}{rgb}{1.000000,0.576471,0.309804}%
\pgfsetfillcolor{currentfill}%
\pgfsetlinewidth{1.003750pt}%
\definecolor{currentstroke}{rgb}{1.000000,0.576471,0.309804}%
\pgfsetstrokecolor{currentstroke}%
\pgfsetdash{}{0pt}%
\pgfsys@defobject{currentmarker}{\pgfqpoint{-0.033333in}{-0.033333in}}{\pgfqpoint{0.033333in}{0.033333in}}{%
\pgfpathmoveto{\pgfqpoint{0.000000in}{-0.033333in}}%
\pgfpathcurveto{\pgfqpoint{0.008840in}{-0.033333in}}{\pgfqpoint{0.017319in}{-0.029821in}}{\pgfqpoint{0.023570in}{-0.023570in}}%
\pgfpathcurveto{\pgfqpoint{0.029821in}{-0.017319in}}{\pgfqpoint{0.033333in}{-0.008840in}}{\pgfqpoint{0.033333in}{0.000000in}}%
\pgfpathcurveto{\pgfqpoint{0.033333in}{0.008840in}}{\pgfqpoint{0.029821in}{0.017319in}}{\pgfqpoint{0.023570in}{0.023570in}}%
\pgfpathcurveto{\pgfqpoint{0.017319in}{0.029821in}}{\pgfqpoint{0.008840in}{0.033333in}}{\pgfqpoint{0.000000in}{0.033333in}}%
\pgfpathcurveto{\pgfqpoint{-0.008840in}{0.033333in}}{\pgfqpoint{-0.017319in}{0.029821in}}{\pgfqpoint{-0.023570in}{0.023570in}}%
\pgfpathcurveto{\pgfqpoint{-0.029821in}{0.017319in}}{\pgfqpoint{-0.033333in}{0.008840in}}{\pgfqpoint{-0.033333in}{0.000000in}}%
\pgfpathcurveto{\pgfqpoint{-0.033333in}{-0.008840in}}{\pgfqpoint{-0.029821in}{-0.017319in}}{\pgfqpoint{-0.023570in}{-0.023570in}}%
\pgfpathcurveto{\pgfqpoint{-0.017319in}{-0.029821in}}{\pgfqpoint{-0.008840in}{-0.033333in}}{\pgfqpoint{0.000000in}{-0.033333in}}%
\pgfpathlineto{\pgfqpoint{0.000000in}{-0.033333in}}%
\pgfpathclose%
\pgfusepath{stroke,fill}%
}%
\begin{pgfscope}%
\pgfsys@transformshift{2.820299in}{2.181202in}%
\pgfsys@useobject{currentmarker}{}%
\end{pgfscope}%
\end{pgfscope}%
\begin{pgfscope}%
\pgfpathrectangle{\pgfqpoint{0.765000in}{0.660000in}}{\pgfqpoint{4.620000in}{4.620000in}}%
\pgfusepath{clip}%
\pgfsetroundcap%
\pgfsetroundjoin%
\pgfsetlinewidth{1.204500pt}%
\definecolor{currentstroke}{rgb}{1.000000,0.576471,0.309804}%
\pgfsetstrokecolor{currentstroke}%
\pgfsetdash{}{0pt}%
\pgfpathmoveto{\pgfqpoint{2.589657in}{2.181202in}}%
\pgfusepath{stroke}%
\end{pgfscope}%
\begin{pgfscope}%
\pgfpathrectangle{\pgfqpoint{0.765000in}{0.660000in}}{\pgfqpoint{4.620000in}{4.620000in}}%
\pgfusepath{clip}%
\pgfsetbuttcap%
\pgfsetroundjoin%
\definecolor{currentfill}{rgb}{1.000000,0.576471,0.309804}%
\pgfsetfillcolor{currentfill}%
\pgfsetlinewidth{1.003750pt}%
\definecolor{currentstroke}{rgb}{1.000000,0.576471,0.309804}%
\pgfsetstrokecolor{currentstroke}%
\pgfsetdash{}{0pt}%
\pgfsys@defobject{currentmarker}{\pgfqpoint{-0.033333in}{-0.033333in}}{\pgfqpoint{0.033333in}{0.033333in}}{%
\pgfpathmoveto{\pgfqpoint{0.000000in}{-0.033333in}}%
\pgfpathcurveto{\pgfqpoint{0.008840in}{-0.033333in}}{\pgfqpoint{0.017319in}{-0.029821in}}{\pgfqpoint{0.023570in}{-0.023570in}}%
\pgfpathcurveto{\pgfqpoint{0.029821in}{-0.017319in}}{\pgfqpoint{0.033333in}{-0.008840in}}{\pgfqpoint{0.033333in}{0.000000in}}%
\pgfpathcurveto{\pgfqpoint{0.033333in}{0.008840in}}{\pgfqpoint{0.029821in}{0.017319in}}{\pgfqpoint{0.023570in}{0.023570in}}%
\pgfpathcurveto{\pgfqpoint{0.017319in}{0.029821in}}{\pgfqpoint{0.008840in}{0.033333in}}{\pgfqpoint{0.000000in}{0.033333in}}%
\pgfpathcurveto{\pgfqpoint{-0.008840in}{0.033333in}}{\pgfqpoint{-0.017319in}{0.029821in}}{\pgfqpoint{-0.023570in}{0.023570in}}%
\pgfpathcurveto{\pgfqpoint{-0.029821in}{0.017319in}}{\pgfqpoint{-0.033333in}{0.008840in}}{\pgfqpoint{-0.033333in}{0.000000in}}%
\pgfpathcurveto{\pgfqpoint{-0.033333in}{-0.008840in}}{\pgfqpoint{-0.029821in}{-0.017319in}}{\pgfqpoint{-0.023570in}{-0.023570in}}%
\pgfpathcurveto{\pgfqpoint{-0.017319in}{-0.029821in}}{\pgfqpoint{-0.008840in}{-0.033333in}}{\pgfqpoint{0.000000in}{-0.033333in}}%
\pgfpathlineto{\pgfqpoint{0.000000in}{-0.033333in}}%
\pgfpathclose%
\pgfusepath{stroke,fill}%
}%
\begin{pgfscope}%
\pgfsys@transformshift{2.589657in}{2.181202in}%
\pgfsys@useobject{currentmarker}{}%
\end{pgfscope}%
\end{pgfscope}%
\begin{pgfscope}%
\pgfpathrectangle{\pgfqpoint{0.765000in}{0.660000in}}{\pgfqpoint{4.620000in}{4.620000in}}%
\pgfusepath{clip}%
\pgfsetroundcap%
\pgfsetroundjoin%
\pgfsetlinewidth{1.204500pt}%
\definecolor{currentstroke}{rgb}{1.000000,0.576471,0.309804}%
\pgfsetstrokecolor{currentstroke}%
\pgfsetdash{}{0pt}%
\pgfpathmoveto{\pgfqpoint{2.474336in}{2.181202in}}%
\pgfusepath{stroke}%
\end{pgfscope}%
\begin{pgfscope}%
\pgfpathrectangle{\pgfqpoint{0.765000in}{0.660000in}}{\pgfqpoint{4.620000in}{4.620000in}}%
\pgfusepath{clip}%
\pgfsetbuttcap%
\pgfsetroundjoin%
\definecolor{currentfill}{rgb}{1.000000,0.576471,0.309804}%
\pgfsetfillcolor{currentfill}%
\pgfsetlinewidth{1.003750pt}%
\definecolor{currentstroke}{rgb}{1.000000,0.576471,0.309804}%
\pgfsetstrokecolor{currentstroke}%
\pgfsetdash{}{0pt}%
\pgfsys@defobject{currentmarker}{\pgfqpoint{-0.033333in}{-0.033333in}}{\pgfqpoint{0.033333in}{0.033333in}}{%
\pgfpathmoveto{\pgfqpoint{0.000000in}{-0.033333in}}%
\pgfpathcurveto{\pgfqpoint{0.008840in}{-0.033333in}}{\pgfqpoint{0.017319in}{-0.029821in}}{\pgfqpoint{0.023570in}{-0.023570in}}%
\pgfpathcurveto{\pgfqpoint{0.029821in}{-0.017319in}}{\pgfqpoint{0.033333in}{-0.008840in}}{\pgfqpoint{0.033333in}{0.000000in}}%
\pgfpathcurveto{\pgfqpoint{0.033333in}{0.008840in}}{\pgfqpoint{0.029821in}{0.017319in}}{\pgfqpoint{0.023570in}{0.023570in}}%
\pgfpathcurveto{\pgfqpoint{0.017319in}{0.029821in}}{\pgfqpoint{0.008840in}{0.033333in}}{\pgfqpoint{0.000000in}{0.033333in}}%
\pgfpathcurveto{\pgfqpoint{-0.008840in}{0.033333in}}{\pgfqpoint{-0.017319in}{0.029821in}}{\pgfqpoint{-0.023570in}{0.023570in}}%
\pgfpathcurveto{\pgfqpoint{-0.029821in}{0.017319in}}{\pgfqpoint{-0.033333in}{0.008840in}}{\pgfqpoint{-0.033333in}{0.000000in}}%
\pgfpathcurveto{\pgfqpoint{-0.033333in}{-0.008840in}}{\pgfqpoint{-0.029821in}{-0.017319in}}{\pgfqpoint{-0.023570in}{-0.023570in}}%
\pgfpathcurveto{\pgfqpoint{-0.017319in}{-0.029821in}}{\pgfqpoint{-0.008840in}{-0.033333in}}{\pgfqpoint{0.000000in}{-0.033333in}}%
\pgfpathlineto{\pgfqpoint{0.000000in}{-0.033333in}}%
\pgfpathclose%
\pgfusepath{stroke,fill}%
}%
\begin{pgfscope}%
\pgfsys@transformshift{2.474336in}{2.181202in}%
\pgfsys@useobject{currentmarker}{}%
\end{pgfscope}%
\end{pgfscope}%
\begin{pgfscope}%
\pgfpathrectangle{\pgfqpoint{0.765000in}{0.660000in}}{\pgfqpoint{4.620000in}{4.620000in}}%
\pgfusepath{clip}%
\pgfsetroundcap%
\pgfsetroundjoin%
\pgfsetlinewidth{1.204500pt}%
\definecolor{currentstroke}{rgb}{1.000000,0.576471,0.309804}%
\pgfsetstrokecolor{currentstroke}%
\pgfsetdash{}{0pt}%
\pgfpathmoveto{\pgfqpoint{2.935620in}{2.343341in}}%
\pgfusepath{stroke}%
\end{pgfscope}%
\begin{pgfscope}%
\pgfpathrectangle{\pgfqpoint{0.765000in}{0.660000in}}{\pgfqpoint{4.620000in}{4.620000in}}%
\pgfusepath{clip}%
\pgfsetbuttcap%
\pgfsetroundjoin%
\definecolor{currentfill}{rgb}{1.000000,0.576471,0.309804}%
\pgfsetfillcolor{currentfill}%
\pgfsetlinewidth{1.003750pt}%
\definecolor{currentstroke}{rgb}{1.000000,0.576471,0.309804}%
\pgfsetstrokecolor{currentstroke}%
\pgfsetdash{}{0pt}%
\pgfsys@defobject{currentmarker}{\pgfqpoint{-0.033333in}{-0.033333in}}{\pgfqpoint{0.033333in}{0.033333in}}{%
\pgfpathmoveto{\pgfqpoint{0.000000in}{-0.033333in}}%
\pgfpathcurveto{\pgfqpoint{0.008840in}{-0.033333in}}{\pgfqpoint{0.017319in}{-0.029821in}}{\pgfqpoint{0.023570in}{-0.023570in}}%
\pgfpathcurveto{\pgfqpoint{0.029821in}{-0.017319in}}{\pgfqpoint{0.033333in}{-0.008840in}}{\pgfqpoint{0.033333in}{0.000000in}}%
\pgfpathcurveto{\pgfqpoint{0.033333in}{0.008840in}}{\pgfqpoint{0.029821in}{0.017319in}}{\pgfqpoint{0.023570in}{0.023570in}}%
\pgfpathcurveto{\pgfqpoint{0.017319in}{0.029821in}}{\pgfqpoint{0.008840in}{0.033333in}}{\pgfqpoint{0.000000in}{0.033333in}}%
\pgfpathcurveto{\pgfqpoint{-0.008840in}{0.033333in}}{\pgfqpoint{-0.017319in}{0.029821in}}{\pgfqpoint{-0.023570in}{0.023570in}}%
\pgfpathcurveto{\pgfqpoint{-0.029821in}{0.017319in}}{\pgfqpoint{-0.033333in}{0.008840in}}{\pgfqpoint{-0.033333in}{0.000000in}}%
\pgfpathcurveto{\pgfqpoint{-0.033333in}{-0.008840in}}{\pgfqpoint{-0.029821in}{-0.017319in}}{\pgfqpoint{-0.023570in}{-0.023570in}}%
\pgfpathcurveto{\pgfqpoint{-0.017319in}{-0.029821in}}{\pgfqpoint{-0.008840in}{-0.033333in}}{\pgfqpoint{0.000000in}{-0.033333in}}%
\pgfpathlineto{\pgfqpoint{0.000000in}{-0.033333in}}%
\pgfpathclose%
\pgfusepath{stroke,fill}%
}%
\begin{pgfscope}%
\pgfsys@transformshift{2.935620in}{2.343341in}%
\pgfsys@useobject{currentmarker}{}%
\end{pgfscope}%
\end{pgfscope}%
\begin{pgfscope}%
\pgfpathrectangle{\pgfqpoint{0.765000in}{0.660000in}}{\pgfqpoint{4.620000in}{4.620000in}}%
\pgfusepath{clip}%
\pgfsetroundcap%
\pgfsetroundjoin%
\pgfsetlinewidth{1.204500pt}%
\definecolor{currentstroke}{rgb}{1.000000,0.576471,0.309804}%
\pgfsetstrokecolor{currentstroke}%
\pgfsetdash{}{0pt}%
\pgfpathmoveto{\pgfqpoint{3.569887in}{2.100132in}}%
\pgfusepath{stroke}%
\end{pgfscope}%
\begin{pgfscope}%
\pgfpathrectangle{\pgfqpoint{0.765000in}{0.660000in}}{\pgfqpoint{4.620000in}{4.620000in}}%
\pgfusepath{clip}%
\pgfsetbuttcap%
\pgfsetroundjoin%
\definecolor{currentfill}{rgb}{1.000000,0.576471,0.309804}%
\pgfsetfillcolor{currentfill}%
\pgfsetlinewidth{1.003750pt}%
\definecolor{currentstroke}{rgb}{1.000000,0.576471,0.309804}%
\pgfsetstrokecolor{currentstroke}%
\pgfsetdash{}{0pt}%
\pgfsys@defobject{currentmarker}{\pgfqpoint{-0.033333in}{-0.033333in}}{\pgfqpoint{0.033333in}{0.033333in}}{%
\pgfpathmoveto{\pgfqpoint{0.000000in}{-0.033333in}}%
\pgfpathcurveto{\pgfqpoint{0.008840in}{-0.033333in}}{\pgfqpoint{0.017319in}{-0.029821in}}{\pgfqpoint{0.023570in}{-0.023570in}}%
\pgfpathcurveto{\pgfqpoint{0.029821in}{-0.017319in}}{\pgfqpoint{0.033333in}{-0.008840in}}{\pgfqpoint{0.033333in}{0.000000in}}%
\pgfpathcurveto{\pgfqpoint{0.033333in}{0.008840in}}{\pgfqpoint{0.029821in}{0.017319in}}{\pgfqpoint{0.023570in}{0.023570in}}%
\pgfpathcurveto{\pgfqpoint{0.017319in}{0.029821in}}{\pgfqpoint{0.008840in}{0.033333in}}{\pgfqpoint{0.000000in}{0.033333in}}%
\pgfpathcurveto{\pgfqpoint{-0.008840in}{0.033333in}}{\pgfqpoint{-0.017319in}{0.029821in}}{\pgfqpoint{-0.023570in}{0.023570in}}%
\pgfpathcurveto{\pgfqpoint{-0.029821in}{0.017319in}}{\pgfqpoint{-0.033333in}{0.008840in}}{\pgfqpoint{-0.033333in}{0.000000in}}%
\pgfpathcurveto{\pgfqpoint{-0.033333in}{-0.008840in}}{\pgfqpoint{-0.029821in}{-0.017319in}}{\pgfqpoint{-0.023570in}{-0.023570in}}%
\pgfpathcurveto{\pgfqpoint{-0.017319in}{-0.029821in}}{\pgfqpoint{-0.008840in}{-0.033333in}}{\pgfqpoint{0.000000in}{-0.033333in}}%
\pgfpathlineto{\pgfqpoint{0.000000in}{-0.033333in}}%
\pgfpathclose%
\pgfusepath{stroke,fill}%
}%
\begin{pgfscope}%
\pgfsys@transformshift{3.569887in}{2.100132in}%
\pgfsys@useobject{currentmarker}{}%
\end{pgfscope}%
\end{pgfscope}%
\begin{pgfscope}%
\pgfpathrectangle{\pgfqpoint{0.765000in}{0.660000in}}{\pgfqpoint{4.620000in}{4.620000in}}%
\pgfusepath{clip}%
\pgfsetroundcap%
\pgfsetroundjoin%
\pgfsetlinewidth{1.204500pt}%
\definecolor{currentstroke}{rgb}{1.000000,0.576471,0.309804}%
\pgfsetstrokecolor{currentstroke}%
\pgfsetdash{}{0pt}%
\pgfpathmoveto{\pgfqpoint{3.742869in}{2.181202in}}%
\pgfusepath{stroke}%
\end{pgfscope}%
\begin{pgfscope}%
\pgfpathrectangle{\pgfqpoint{0.765000in}{0.660000in}}{\pgfqpoint{4.620000in}{4.620000in}}%
\pgfusepath{clip}%
\pgfsetbuttcap%
\pgfsetroundjoin%
\definecolor{currentfill}{rgb}{1.000000,0.576471,0.309804}%
\pgfsetfillcolor{currentfill}%
\pgfsetlinewidth{1.003750pt}%
\definecolor{currentstroke}{rgb}{1.000000,0.576471,0.309804}%
\pgfsetstrokecolor{currentstroke}%
\pgfsetdash{}{0pt}%
\pgfsys@defobject{currentmarker}{\pgfqpoint{-0.033333in}{-0.033333in}}{\pgfqpoint{0.033333in}{0.033333in}}{%
\pgfpathmoveto{\pgfqpoint{0.000000in}{-0.033333in}}%
\pgfpathcurveto{\pgfqpoint{0.008840in}{-0.033333in}}{\pgfqpoint{0.017319in}{-0.029821in}}{\pgfqpoint{0.023570in}{-0.023570in}}%
\pgfpathcurveto{\pgfqpoint{0.029821in}{-0.017319in}}{\pgfqpoint{0.033333in}{-0.008840in}}{\pgfqpoint{0.033333in}{0.000000in}}%
\pgfpathcurveto{\pgfqpoint{0.033333in}{0.008840in}}{\pgfqpoint{0.029821in}{0.017319in}}{\pgfqpoint{0.023570in}{0.023570in}}%
\pgfpathcurveto{\pgfqpoint{0.017319in}{0.029821in}}{\pgfqpoint{0.008840in}{0.033333in}}{\pgfqpoint{0.000000in}{0.033333in}}%
\pgfpathcurveto{\pgfqpoint{-0.008840in}{0.033333in}}{\pgfqpoint{-0.017319in}{0.029821in}}{\pgfqpoint{-0.023570in}{0.023570in}}%
\pgfpathcurveto{\pgfqpoint{-0.029821in}{0.017319in}}{\pgfqpoint{-0.033333in}{0.008840in}}{\pgfqpoint{-0.033333in}{0.000000in}}%
\pgfpathcurveto{\pgfqpoint{-0.033333in}{-0.008840in}}{\pgfqpoint{-0.029821in}{-0.017319in}}{\pgfqpoint{-0.023570in}{-0.023570in}}%
\pgfpathcurveto{\pgfqpoint{-0.017319in}{-0.029821in}}{\pgfqpoint{-0.008840in}{-0.033333in}}{\pgfqpoint{0.000000in}{-0.033333in}}%
\pgfpathlineto{\pgfqpoint{0.000000in}{-0.033333in}}%
\pgfpathclose%
\pgfusepath{stroke,fill}%
}%
\begin{pgfscope}%
\pgfsys@transformshift{3.742869in}{2.181202in}%
\pgfsys@useobject{currentmarker}{}%
\end{pgfscope}%
\end{pgfscope}%
\begin{pgfscope}%
\pgfpathrectangle{\pgfqpoint{0.765000in}{0.660000in}}{\pgfqpoint{4.620000in}{4.620000in}}%
\pgfusepath{clip}%
\pgfsetroundcap%
\pgfsetroundjoin%
\pgfsetlinewidth{1.204500pt}%
\definecolor{currentstroke}{rgb}{1.000000,0.576471,0.309804}%
\pgfsetstrokecolor{currentstroke}%
\pgfsetdash{}{0pt}%
\pgfpathmoveto{\pgfqpoint{2.416675in}{2.100132in}}%
\pgfusepath{stroke}%
\end{pgfscope}%
\begin{pgfscope}%
\pgfpathrectangle{\pgfqpoint{0.765000in}{0.660000in}}{\pgfqpoint{4.620000in}{4.620000in}}%
\pgfusepath{clip}%
\pgfsetbuttcap%
\pgfsetroundjoin%
\definecolor{currentfill}{rgb}{1.000000,0.576471,0.309804}%
\pgfsetfillcolor{currentfill}%
\pgfsetlinewidth{1.003750pt}%
\definecolor{currentstroke}{rgb}{1.000000,0.576471,0.309804}%
\pgfsetstrokecolor{currentstroke}%
\pgfsetdash{}{0pt}%
\pgfsys@defobject{currentmarker}{\pgfqpoint{-0.033333in}{-0.033333in}}{\pgfqpoint{0.033333in}{0.033333in}}{%
\pgfpathmoveto{\pgfqpoint{0.000000in}{-0.033333in}}%
\pgfpathcurveto{\pgfqpoint{0.008840in}{-0.033333in}}{\pgfqpoint{0.017319in}{-0.029821in}}{\pgfqpoint{0.023570in}{-0.023570in}}%
\pgfpathcurveto{\pgfqpoint{0.029821in}{-0.017319in}}{\pgfqpoint{0.033333in}{-0.008840in}}{\pgfqpoint{0.033333in}{0.000000in}}%
\pgfpathcurveto{\pgfqpoint{0.033333in}{0.008840in}}{\pgfqpoint{0.029821in}{0.017319in}}{\pgfqpoint{0.023570in}{0.023570in}}%
\pgfpathcurveto{\pgfqpoint{0.017319in}{0.029821in}}{\pgfqpoint{0.008840in}{0.033333in}}{\pgfqpoint{0.000000in}{0.033333in}}%
\pgfpathcurveto{\pgfqpoint{-0.008840in}{0.033333in}}{\pgfqpoint{-0.017319in}{0.029821in}}{\pgfqpoint{-0.023570in}{0.023570in}}%
\pgfpathcurveto{\pgfqpoint{-0.029821in}{0.017319in}}{\pgfqpoint{-0.033333in}{0.008840in}}{\pgfqpoint{-0.033333in}{0.000000in}}%
\pgfpathcurveto{\pgfqpoint{-0.033333in}{-0.008840in}}{\pgfqpoint{-0.029821in}{-0.017319in}}{\pgfqpoint{-0.023570in}{-0.023570in}}%
\pgfpathcurveto{\pgfqpoint{-0.017319in}{-0.029821in}}{\pgfqpoint{-0.008840in}{-0.033333in}}{\pgfqpoint{0.000000in}{-0.033333in}}%
\pgfpathlineto{\pgfqpoint{0.000000in}{-0.033333in}}%
\pgfpathclose%
\pgfusepath{stroke,fill}%
}%
\begin{pgfscope}%
\pgfsys@transformshift{2.416675in}{2.100132in}%
\pgfsys@useobject{currentmarker}{}%
\end{pgfscope}%
\end{pgfscope}%
\begin{pgfscope}%
\pgfpathrectangle{\pgfqpoint{0.765000in}{0.660000in}}{\pgfqpoint{4.620000in}{4.620000in}}%
\pgfusepath{clip}%
\pgfsetroundcap%
\pgfsetroundjoin%
\pgfsetlinewidth{1.204500pt}%
\definecolor{currentstroke}{rgb}{1.000000,0.576471,0.309804}%
\pgfsetstrokecolor{currentstroke}%
\pgfsetdash{}{0pt}%
\pgfpathmoveto{\pgfqpoint{2.589657in}{2.181202in}}%
\pgfusepath{stroke}%
\end{pgfscope}%
\begin{pgfscope}%
\pgfpathrectangle{\pgfqpoint{0.765000in}{0.660000in}}{\pgfqpoint{4.620000in}{4.620000in}}%
\pgfusepath{clip}%
\pgfsetbuttcap%
\pgfsetroundjoin%
\definecolor{currentfill}{rgb}{1.000000,0.576471,0.309804}%
\pgfsetfillcolor{currentfill}%
\pgfsetlinewidth{1.003750pt}%
\definecolor{currentstroke}{rgb}{1.000000,0.576471,0.309804}%
\pgfsetstrokecolor{currentstroke}%
\pgfsetdash{}{0pt}%
\pgfsys@defobject{currentmarker}{\pgfqpoint{-0.033333in}{-0.033333in}}{\pgfqpoint{0.033333in}{0.033333in}}{%
\pgfpathmoveto{\pgfqpoint{0.000000in}{-0.033333in}}%
\pgfpathcurveto{\pgfqpoint{0.008840in}{-0.033333in}}{\pgfqpoint{0.017319in}{-0.029821in}}{\pgfqpoint{0.023570in}{-0.023570in}}%
\pgfpathcurveto{\pgfqpoint{0.029821in}{-0.017319in}}{\pgfqpoint{0.033333in}{-0.008840in}}{\pgfqpoint{0.033333in}{0.000000in}}%
\pgfpathcurveto{\pgfqpoint{0.033333in}{0.008840in}}{\pgfqpoint{0.029821in}{0.017319in}}{\pgfqpoint{0.023570in}{0.023570in}}%
\pgfpathcurveto{\pgfqpoint{0.017319in}{0.029821in}}{\pgfqpoint{0.008840in}{0.033333in}}{\pgfqpoint{0.000000in}{0.033333in}}%
\pgfpathcurveto{\pgfqpoint{-0.008840in}{0.033333in}}{\pgfqpoint{-0.017319in}{0.029821in}}{\pgfqpoint{-0.023570in}{0.023570in}}%
\pgfpathcurveto{\pgfqpoint{-0.029821in}{0.017319in}}{\pgfqpoint{-0.033333in}{0.008840in}}{\pgfqpoint{-0.033333in}{0.000000in}}%
\pgfpathcurveto{\pgfqpoint{-0.033333in}{-0.008840in}}{\pgfqpoint{-0.029821in}{-0.017319in}}{\pgfqpoint{-0.023570in}{-0.023570in}}%
\pgfpathcurveto{\pgfqpoint{-0.017319in}{-0.029821in}}{\pgfqpoint{-0.008840in}{-0.033333in}}{\pgfqpoint{0.000000in}{-0.033333in}}%
\pgfpathlineto{\pgfqpoint{0.000000in}{-0.033333in}}%
\pgfpathclose%
\pgfusepath{stroke,fill}%
}%
\begin{pgfscope}%
\pgfsys@transformshift{2.589657in}{2.181202in}%
\pgfsys@useobject{currentmarker}{}%
\end{pgfscope}%
\end{pgfscope}%
\begin{pgfscope}%
\pgfpathrectangle{\pgfqpoint{0.765000in}{0.660000in}}{\pgfqpoint{4.620000in}{4.620000in}}%
\pgfusepath{clip}%
\pgfsetroundcap%
\pgfsetroundjoin%
\pgfsetlinewidth{1.204500pt}%
\definecolor{currentstroke}{rgb}{1.000000,0.576471,0.309804}%
\pgfsetstrokecolor{currentstroke}%
\pgfsetdash{}{0pt}%
\pgfpathmoveto{\pgfqpoint{2.935620in}{2.181202in}}%
\pgfusepath{stroke}%
\end{pgfscope}%
\begin{pgfscope}%
\pgfpathrectangle{\pgfqpoint{0.765000in}{0.660000in}}{\pgfqpoint{4.620000in}{4.620000in}}%
\pgfusepath{clip}%
\pgfsetbuttcap%
\pgfsetroundjoin%
\definecolor{currentfill}{rgb}{1.000000,0.576471,0.309804}%
\pgfsetfillcolor{currentfill}%
\pgfsetlinewidth{1.003750pt}%
\definecolor{currentstroke}{rgb}{1.000000,0.576471,0.309804}%
\pgfsetstrokecolor{currentstroke}%
\pgfsetdash{}{0pt}%
\pgfsys@defobject{currentmarker}{\pgfqpoint{-0.033333in}{-0.033333in}}{\pgfqpoint{0.033333in}{0.033333in}}{%
\pgfpathmoveto{\pgfqpoint{0.000000in}{-0.033333in}}%
\pgfpathcurveto{\pgfqpoint{0.008840in}{-0.033333in}}{\pgfqpoint{0.017319in}{-0.029821in}}{\pgfqpoint{0.023570in}{-0.023570in}}%
\pgfpathcurveto{\pgfqpoint{0.029821in}{-0.017319in}}{\pgfqpoint{0.033333in}{-0.008840in}}{\pgfqpoint{0.033333in}{0.000000in}}%
\pgfpathcurveto{\pgfqpoint{0.033333in}{0.008840in}}{\pgfqpoint{0.029821in}{0.017319in}}{\pgfqpoint{0.023570in}{0.023570in}}%
\pgfpathcurveto{\pgfqpoint{0.017319in}{0.029821in}}{\pgfqpoint{0.008840in}{0.033333in}}{\pgfqpoint{0.000000in}{0.033333in}}%
\pgfpathcurveto{\pgfqpoint{-0.008840in}{0.033333in}}{\pgfqpoint{-0.017319in}{0.029821in}}{\pgfqpoint{-0.023570in}{0.023570in}}%
\pgfpathcurveto{\pgfqpoint{-0.029821in}{0.017319in}}{\pgfqpoint{-0.033333in}{0.008840in}}{\pgfqpoint{-0.033333in}{0.000000in}}%
\pgfpathcurveto{\pgfqpoint{-0.033333in}{-0.008840in}}{\pgfqpoint{-0.029821in}{-0.017319in}}{\pgfqpoint{-0.023570in}{-0.023570in}}%
\pgfpathcurveto{\pgfqpoint{-0.017319in}{-0.029821in}}{\pgfqpoint{-0.008840in}{-0.033333in}}{\pgfqpoint{0.000000in}{-0.033333in}}%
\pgfpathlineto{\pgfqpoint{0.000000in}{-0.033333in}}%
\pgfpathclose%
\pgfusepath{stroke,fill}%
}%
\begin{pgfscope}%
\pgfsys@transformshift{2.935620in}{2.181202in}%
\pgfsys@useobject{currentmarker}{}%
\end{pgfscope}%
\end{pgfscope}%
\begin{pgfscope}%
\pgfpathrectangle{\pgfqpoint{0.765000in}{0.660000in}}{\pgfqpoint{4.620000in}{4.620000in}}%
\pgfusepath{clip}%
\pgfsetroundcap%
\pgfsetroundjoin%
\pgfsetlinewidth{1.204500pt}%
\definecolor{currentstroke}{rgb}{1.000000,0.576471,0.309804}%
\pgfsetstrokecolor{currentstroke}%
\pgfsetdash{}{0pt}%
\pgfpathmoveto{\pgfqpoint{2.416675in}{2.100132in}}%
\pgfusepath{stroke}%
\end{pgfscope}%
\begin{pgfscope}%
\pgfpathrectangle{\pgfqpoint{0.765000in}{0.660000in}}{\pgfqpoint{4.620000in}{4.620000in}}%
\pgfusepath{clip}%
\pgfsetbuttcap%
\pgfsetroundjoin%
\definecolor{currentfill}{rgb}{1.000000,0.576471,0.309804}%
\pgfsetfillcolor{currentfill}%
\pgfsetlinewidth{1.003750pt}%
\definecolor{currentstroke}{rgb}{1.000000,0.576471,0.309804}%
\pgfsetstrokecolor{currentstroke}%
\pgfsetdash{}{0pt}%
\pgfsys@defobject{currentmarker}{\pgfqpoint{-0.033333in}{-0.033333in}}{\pgfqpoint{0.033333in}{0.033333in}}{%
\pgfpathmoveto{\pgfqpoint{0.000000in}{-0.033333in}}%
\pgfpathcurveto{\pgfqpoint{0.008840in}{-0.033333in}}{\pgfqpoint{0.017319in}{-0.029821in}}{\pgfqpoint{0.023570in}{-0.023570in}}%
\pgfpathcurveto{\pgfqpoint{0.029821in}{-0.017319in}}{\pgfqpoint{0.033333in}{-0.008840in}}{\pgfqpoint{0.033333in}{0.000000in}}%
\pgfpathcurveto{\pgfqpoint{0.033333in}{0.008840in}}{\pgfqpoint{0.029821in}{0.017319in}}{\pgfqpoint{0.023570in}{0.023570in}}%
\pgfpathcurveto{\pgfqpoint{0.017319in}{0.029821in}}{\pgfqpoint{0.008840in}{0.033333in}}{\pgfqpoint{0.000000in}{0.033333in}}%
\pgfpathcurveto{\pgfqpoint{-0.008840in}{0.033333in}}{\pgfqpoint{-0.017319in}{0.029821in}}{\pgfqpoint{-0.023570in}{0.023570in}}%
\pgfpathcurveto{\pgfqpoint{-0.029821in}{0.017319in}}{\pgfqpoint{-0.033333in}{0.008840in}}{\pgfqpoint{-0.033333in}{0.000000in}}%
\pgfpathcurveto{\pgfqpoint{-0.033333in}{-0.008840in}}{\pgfqpoint{-0.029821in}{-0.017319in}}{\pgfqpoint{-0.023570in}{-0.023570in}}%
\pgfpathcurveto{\pgfqpoint{-0.017319in}{-0.029821in}}{\pgfqpoint{-0.008840in}{-0.033333in}}{\pgfqpoint{0.000000in}{-0.033333in}}%
\pgfpathlineto{\pgfqpoint{0.000000in}{-0.033333in}}%
\pgfpathclose%
\pgfusepath{stroke,fill}%
}%
\begin{pgfscope}%
\pgfsys@transformshift{2.416675in}{2.100132in}%
\pgfsys@useobject{currentmarker}{}%
\end{pgfscope}%
\end{pgfscope}%
\begin{pgfscope}%
\pgfpathrectangle{\pgfqpoint{0.765000in}{0.660000in}}{\pgfqpoint{4.620000in}{4.620000in}}%
\pgfusepath{clip}%
\pgfsetroundcap%
\pgfsetroundjoin%
\pgfsetlinewidth{1.204500pt}%
\definecolor{currentstroke}{rgb}{1.000000,0.576471,0.309804}%
\pgfsetstrokecolor{currentstroke}%
\pgfsetdash{}{0pt}%
\pgfpathmoveto{\pgfqpoint{2.359014in}{2.181202in}}%
\pgfusepath{stroke}%
\end{pgfscope}%
\begin{pgfscope}%
\pgfpathrectangle{\pgfqpoint{0.765000in}{0.660000in}}{\pgfqpoint{4.620000in}{4.620000in}}%
\pgfusepath{clip}%
\pgfsetbuttcap%
\pgfsetroundjoin%
\definecolor{currentfill}{rgb}{1.000000,0.576471,0.309804}%
\pgfsetfillcolor{currentfill}%
\pgfsetlinewidth{1.003750pt}%
\definecolor{currentstroke}{rgb}{1.000000,0.576471,0.309804}%
\pgfsetstrokecolor{currentstroke}%
\pgfsetdash{}{0pt}%
\pgfsys@defobject{currentmarker}{\pgfqpoint{-0.033333in}{-0.033333in}}{\pgfqpoint{0.033333in}{0.033333in}}{%
\pgfpathmoveto{\pgfqpoint{0.000000in}{-0.033333in}}%
\pgfpathcurveto{\pgfqpoint{0.008840in}{-0.033333in}}{\pgfqpoint{0.017319in}{-0.029821in}}{\pgfqpoint{0.023570in}{-0.023570in}}%
\pgfpathcurveto{\pgfqpoint{0.029821in}{-0.017319in}}{\pgfqpoint{0.033333in}{-0.008840in}}{\pgfqpoint{0.033333in}{0.000000in}}%
\pgfpathcurveto{\pgfqpoint{0.033333in}{0.008840in}}{\pgfqpoint{0.029821in}{0.017319in}}{\pgfqpoint{0.023570in}{0.023570in}}%
\pgfpathcurveto{\pgfqpoint{0.017319in}{0.029821in}}{\pgfqpoint{0.008840in}{0.033333in}}{\pgfqpoint{0.000000in}{0.033333in}}%
\pgfpathcurveto{\pgfqpoint{-0.008840in}{0.033333in}}{\pgfqpoint{-0.017319in}{0.029821in}}{\pgfqpoint{-0.023570in}{0.023570in}}%
\pgfpathcurveto{\pgfqpoint{-0.029821in}{0.017319in}}{\pgfqpoint{-0.033333in}{0.008840in}}{\pgfqpoint{-0.033333in}{0.000000in}}%
\pgfpathcurveto{\pgfqpoint{-0.033333in}{-0.008840in}}{\pgfqpoint{-0.029821in}{-0.017319in}}{\pgfqpoint{-0.023570in}{-0.023570in}}%
\pgfpathcurveto{\pgfqpoint{-0.017319in}{-0.029821in}}{\pgfqpoint{-0.008840in}{-0.033333in}}{\pgfqpoint{0.000000in}{-0.033333in}}%
\pgfpathlineto{\pgfqpoint{0.000000in}{-0.033333in}}%
\pgfpathclose%
\pgfusepath{stroke,fill}%
}%
\begin{pgfscope}%
\pgfsys@transformshift{2.359014in}{2.181202in}%
\pgfsys@useobject{currentmarker}{}%
\end{pgfscope}%
\end{pgfscope}%
\begin{pgfscope}%
\pgfpathrectangle{\pgfqpoint{0.765000in}{0.660000in}}{\pgfqpoint{4.620000in}{4.620000in}}%
\pgfusepath{clip}%
\pgfsetroundcap%
\pgfsetroundjoin%
\pgfsetlinewidth{1.204500pt}%
\definecolor{currentstroke}{rgb}{1.000000,0.576471,0.309804}%
\pgfsetstrokecolor{currentstroke}%
\pgfsetdash{}{0pt}%
\pgfpathmoveto{\pgfqpoint{2.820299in}{2.181202in}}%
\pgfusepath{stroke}%
\end{pgfscope}%
\begin{pgfscope}%
\pgfpathrectangle{\pgfqpoint{0.765000in}{0.660000in}}{\pgfqpoint{4.620000in}{4.620000in}}%
\pgfusepath{clip}%
\pgfsetbuttcap%
\pgfsetroundjoin%
\definecolor{currentfill}{rgb}{1.000000,0.576471,0.309804}%
\pgfsetfillcolor{currentfill}%
\pgfsetlinewidth{1.003750pt}%
\definecolor{currentstroke}{rgb}{1.000000,0.576471,0.309804}%
\pgfsetstrokecolor{currentstroke}%
\pgfsetdash{}{0pt}%
\pgfsys@defobject{currentmarker}{\pgfqpoint{-0.033333in}{-0.033333in}}{\pgfqpoint{0.033333in}{0.033333in}}{%
\pgfpathmoveto{\pgfqpoint{0.000000in}{-0.033333in}}%
\pgfpathcurveto{\pgfqpoint{0.008840in}{-0.033333in}}{\pgfqpoint{0.017319in}{-0.029821in}}{\pgfqpoint{0.023570in}{-0.023570in}}%
\pgfpathcurveto{\pgfqpoint{0.029821in}{-0.017319in}}{\pgfqpoint{0.033333in}{-0.008840in}}{\pgfqpoint{0.033333in}{0.000000in}}%
\pgfpathcurveto{\pgfqpoint{0.033333in}{0.008840in}}{\pgfqpoint{0.029821in}{0.017319in}}{\pgfqpoint{0.023570in}{0.023570in}}%
\pgfpathcurveto{\pgfqpoint{0.017319in}{0.029821in}}{\pgfqpoint{0.008840in}{0.033333in}}{\pgfqpoint{0.000000in}{0.033333in}}%
\pgfpathcurveto{\pgfqpoint{-0.008840in}{0.033333in}}{\pgfqpoint{-0.017319in}{0.029821in}}{\pgfqpoint{-0.023570in}{0.023570in}}%
\pgfpathcurveto{\pgfqpoint{-0.029821in}{0.017319in}}{\pgfqpoint{-0.033333in}{0.008840in}}{\pgfqpoint{-0.033333in}{0.000000in}}%
\pgfpathcurveto{\pgfqpoint{-0.033333in}{-0.008840in}}{\pgfqpoint{-0.029821in}{-0.017319in}}{\pgfqpoint{-0.023570in}{-0.023570in}}%
\pgfpathcurveto{\pgfqpoint{-0.017319in}{-0.029821in}}{\pgfqpoint{-0.008840in}{-0.033333in}}{\pgfqpoint{0.000000in}{-0.033333in}}%
\pgfpathlineto{\pgfqpoint{0.000000in}{-0.033333in}}%
\pgfpathclose%
\pgfusepath{stroke,fill}%
}%
\begin{pgfscope}%
\pgfsys@transformshift{2.820299in}{2.181202in}%
\pgfsys@useobject{currentmarker}{}%
\end{pgfscope}%
\end{pgfscope}%
\begin{pgfscope}%
\pgfpathrectangle{\pgfqpoint{0.765000in}{0.660000in}}{\pgfqpoint{4.620000in}{4.620000in}}%
\pgfusepath{clip}%
\pgfsetroundcap%
\pgfsetroundjoin%
\pgfsetlinewidth{1.204500pt}%
\definecolor{currentstroke}{rgb}{1.000000,0.576471,0.309804}%
\pgfsetstrokecolor{currentstroke}%
\pgfsetdash{}{0pt}%
\pgfpathmoveto{\pgfqpoint{2.993281in}{2.262271in}}%
\pgfusepath{stroke}%
\end{pgfscope}%
\begin{pgfscope}%
\pgfpathrectangle{\pgfqpoint{0.765000in}{0.660000in}}{\pgfqpoint{4.620000in}{4.620000in}}%
\pgfusepath{clip}%
\pgfsetbuttcap%
\pgfsetroundjoin%
\definecolor{currentfill}{rgb}{1.000000,0.576471,0.309804}%
\pgfsetfillcolor{currentfill}%
\pgfsetlinewidth{1.003750pt}%
\definecolor{currentstroke}{rgb}{1.000000,0.576471,0.309804}%
\pgfsetstrokecolor{currentstroke}%
\pgfsetdash{}{0pt}%
\pgfsys@defobject{currentmarker}{\pgfqpoint{-0.033333in}{-0.033333in}}{\pgfqpoint{0.033333in}{0.033333in}}{%
\pgfpathmoveto{\pgfqpoint{0.000000in}{-0.033333in}}%
\pgfpathcurveto{\pgfqpoint{0.008840in}{-0.033333in}}{\pgfqpoint{0.017319in}{-0.029821in}}{\pgfqpoint{0.023570in}{-0.023570in}}%
\pgfpathcurveto{\pgfqpoint{0.029821in}{-0.017319in}}{\pgfqpoint{0.033333in}{-0.008840in}}{\pgfqpoint{0.033333in}{0.000000in}}%
\pgfpathcurveto{\pgfqpoint{0.033333in}{0.008840in}}{\pgfqpoint{0.029821in}{0.017319in}}{\pgfqpoint{0.023570in}{0.023570in}}%
\pgfpathcurveto{\pgfqpoint{0.017319in}{0.029821in}}{\pgfqpoint{0.008840in}{0.033333in}}{\pgfqpoint{0.000000in}{0.033333in}}%
\pgfpathcurveto{\pgfqpoint{-0.008840in}{0.033333in}}{\pgfqpoint{-0.017319in}{0.029821in}}{\pgfqpoint{-0.023570in}{0.023570in}}%
\pgfpathcurveto{\pgfqpoint{-0.029821in}{0.017319in}}{\pgfqpoint{-0.033333in}{0.008840in}}{\pgfqpoint{-0.033333in}{0.000000in}}%
\pgfpathcurveto{\pgfqpoint{-0.033333in}{-0.008840in}}{\pgfqpoint{-0.029821in}{-0.017319in}}{\pgfqpoint{-0.023570in}{-0.023570in}}%
\pgfpathcurveto{\pgfqpoint{-0.017319in}{-0.029821in}}{\pgfqpoint{-0.008840in}{-0.033333in}}{\pgfqpoint{0.000000in}{-0.033333in}}%
\pgfpathlineto{\pgfqpoint{0.000000in}{-0.033333in}}%
\pgfpathclose%
\pgfusepath{stroke,fill}%
}%
\begin{pgfscope}%
\pgfsys@transformshift{2.993281in}{2.262271in}%
\pgfsys@useobject{currentmarker}{}%
\end{pgfscope}%
\end{pgfscope}%
\begin{pgfscope}%
\pgfpathrectangle{\pgfqpoint{0.765000in}{0.660000in}}{\pgfqpoint{4.620000in}{4.620000in}}%
\pgfusepath{clip}%
\pgfsetroundcap%
\pgfsetroundjoin%
\pgfsetlinewidth{1.204500pt}%
\definecolor{currentstroke}{rgb}{1.000000,0.576471,0.309804}%
\pgfsetstrokecolor{currentstroke}%
\pgfsetdash{}{0pt}%
\pgfpathmoveto{\pgfqpoint{1.609427in}{2.262271in}}%
\pgfusepath{stroke}%
\end{pgfscope}%
\begin{pgfscope}%
\pgfpathrectangle{\pgfqpoint{0.765000in}{0.660000in}}{\pgfqpoint{4.620000in}{4.620000in}}%
\pgfusepath{clip}%
\pgfsetbuttcap%
\pgfsetroundjoin%
\definecolor{currentfill}{rgb}{1.000000,0.576471,0.309804}%
\pgfsetfillcolor{currentfill}%
\pgfsetlinewidth{1.003750pt}%
\definecolor{currentstroke}{rgb}{1.000000,0.576471,0.309804}%
\pgfsetstrokecolor{currentstroke}%
\pgfsetdash{}{0pt}%
\pgfsys@defobject{currentmarker}{\pgfqpoint{-0.033333in}{-0.033333in}}{\pgfqpoint{0.033333in}{0.033333in}}{%
\pgfpathmoveto{\pgfqpoint{0.000000in}{-0.033333in}}%
\pgfpathcurveto{\pgfqpoint{0.008840in}{-0.033333in}}{\pgfqpoint{0.017319in}{-0.029821in}}{\pgfqpoint{0.023570in}{-0.023570in}}%
\pgfpathcurveto{\pgfqpoint{0.029821in}{-0.017319in}}{\pgfqpoint{0.033333in}{-0.008840in}}{\pgfqpoint{0.033333in}{0.000000in}}%
\pgfpathcurveto{\pgfqpoint{0.033333in}{0.008840in}}{\pgfqpoint{0.029821in}{0.017319in}}{\pgfqpoint{0.023570in}{0.023570in}}%
\pgfpathcurveto{\pgfqpoint{0.017319in}{0.029821in}}{\pgfqpoint{0.008840in}{0.033333in}}{\pgfqpoint{0.000000in}{0.033333in}}%
\pgfpathcurveto{\pgfqpoint{-0.008840in}{0.033333in}}{\pgfqpoint{-0.017319in}{0.029821in}}{\pgfqpoint{-0.023570in}{0.023570in}}%
\pgfpathcurveto{\pgfqpoint{-0.029821in}{0.017319in}}{\pgfqpoint{-0.033333in}{0.008840in}}{\pgfqpoint{-0.033333in}{0.000000in}}%
\pgfpathcurveto{\pgfqpoint{-0.033333in}{-0.008840in}}{\pgfqpoint{-0.029821in}{-0.017319in}}{\pgfqpoint{-0.023570in}{-0.023570in}}%
\pgfpathcurveto{\pgfqpoint{-0.017319in}{-0.029821in}}{\pgfqpoint{-0.008840in}{-0.033333in}}{\pgfqpoint{0.000000in}{-0.033333in}}%
\pgfpathlineto{\pgfqpoint{0.000000in}{-0.033333in}}%
\pgfpathclose%
\pgfusepath{stroke,fill}%
}%
\begin{pgfscope}%
\pgfsys@transformshift{1.609427in}{2.262271in}%
\pgfsys@useobject{currentmarker}{}%
\end{pgfscope}%
\end{pgfscope}%
\begin{pgfscope}%
\pgfpathrectangle{\pgfqpoint{0.765000in}{0.660000in}}{\pgfqpoint{4.620000in}{4.620000in}}%
\pgfusepath{clip}%
\pgfsetroundcap%
\pgfsetroundjoin%
\pgfsetlinewidth{1.204500pt}%
\definecolor{currentstroke}{rgb}{1.000000,0.576471,0.309804}%
\pgfsetstrokecolor{currentstroke}%
\pgfsetdash{}{0pt}%
\pgfpathmoveto{\pgfqpoint{2.589657in}{2.181202in}}%
\pgfusepath{stroke}%
\end{pgfscope}%
\begin{pgfscope}%
\pgfpathrectangle{\pgfqpoint{0.765000in}{0.660000in}}{\pgfqpoint{4.620000in}{4.620000in}}%
\pgfusepath{clip}%
\pgfsetbuttcap%
\pgfsetroundjoin%
\definecolor{currentfill}{rgb}{1.000000,0.576471,0.309804}%
\pgfsetfillcolor{currentfill}%
\pgfsetlinewidth{1.003750pt}%
\definecolor{currentstroke}{rgb}{1.000000,0.576471,0.309804}%
\pgfsetstrokecolor{currentstroke}%
\pgfsetdash{}{0pt}%
\pgfsys@defobject{currentmarker}{\pgfqpoint{-0.033333in}{-0.033333in}}{\pgfqpoint{0.033333in}{0.033333in}}{%
\pgfpathmoveto{\pgfqpoint{0.000000in}{-0.033333in}}%
\pgfpathcurveto{\pgfqpoint{0.008840in}{-0.033333in}}{\pgfqpoint{0.017319in}{-0.029821in}}{\pgfqpoint{0.023570in}{-0.023570in}}%
\pgfpathcurveto{\pgfqpoint{0.029821in}{-0.017319in}}{\pgfqpoint{0.033333in}{-0.008840in}}{\pgfqpoint{0.033333in}{0.000000in}}%
\pgfpathcurveto{\pgfqpoint{0.033333in}{0.008840in}}{\pgfqpoint{0.029821in}{0.017319in}}{\pgfqpoint{0.023570in}{0.023570in}}%
\pgfpathcurveto{\pgfqpoint{0.017319in}{0.029821in}}{\pgfqpoint{0.008840in}{0.033333in}}{\pgfqpoint{0.000000in}{0.033333in}}%
\pgfpathcurveto{\pgfqpoint{-0.008840in}{0.033333in}}{\pgfqpoint{-0.017319in}{0.029821in}}{\pgfqpoint{-0.023570in}{0.023570in}}%
\pgfpathcurveto{\pgfqpoint{-0.029821in}{0.017319in}}{\pgfqpoint{-0.033333in}{0.008840in}}{\pgfqpoint{-0.033333in}{0.000000in}}%
\pgfpathcurveto{\pgfqpoint{-0.033333in}{-0.008840in}}{\pgfqpoint{-0.029821in}{-0.017319in}}{\pgfqpoint{-0.023570in}{-0.023570in}}%
\pgfpathcurveto{\pgfqpoint{-0.017319in}{-0.029821in}}{\pgfqpoint{-0.008840in}{-0.033333in}}{\pgfqpoint{0.000000in}{-0.033333in}}%
\pgfpathlineto{\pgfqpoint{0.000000in}{-0.033333in}}%
\pgfpathclose%
\pgfusepath{stroke,fill}%
}%
\begin{pgfscope}%
\pgfsys@transformshift{2.589657in}{2.181202in}%
\pgfsys@useobject{currentmarker}{}%
\end{pgfscope}%
\end{pgfscope}%
\begin{pgfscope}%
\pgfpathrectangle{\pgfqpoint{0.765000in}{0.660000in}}{\pgfqpoint{4.620000in}{4.620000in}}%
\pgfusepath{clip}%
\pgfsetroundcap%
\pgfsetroundjoin%
\pgfsetlinewidth{1.204500pt}%
\definecolor{currentstroke}{rgb}{1.000000,0.576471,0.309804}%
\pgfsetstrokecolor{currentstroke}%
\pgfsetdash{}{0pt}%
\pgfpathmoveto{\pgfqpoint{3.166263in}{2.505480in}}%
\pgfusepath{stroke}%
\end{pgfscope}%
\begin{pgfscope}%
\pgfpathrectangle{\pgfqpoint{0.765000in}{0.660000in}}{\pgfqpoint{4.620000in}{4.620000in}}%
\pgfusepath{clip}%
\pgfsetbuttcap%
\pgfsetroundjoin%
\definecolor{currentfill}{rgb}{1.000000,0.576471,0.309804}%
\pgfsetfillcolor{currentfill}%
\pgfsetlinewidth{1.003750pt}%
\definecolor{currentstroke}{rgb}{1.000000,0.576471,0.309804}%
\pgfsetstrokecolor{currentstroke}%
\pgfsetdash{}{0pt}%
\pgfsys@defobject{currentmarker}{\pgfqpoint{-0.033333in}{-0.033333in}}{\pgfqpoint{0.033333in}{0.033333in}}{%
\pgfpathmoveto{\pgfqpoint{0.000000in}{-0.033333in}}%
\pgfpathcurveto{\pgfqpoint{0.008840in}{-0.033333in}}{\pgfqpoint{0.017319in}{-0.029821in}}{\pgfqpoint{0.023570in}{-0.023570in}}%
\pgfpathcurveto{\pgfqpoint{0.029821in}{-0.017319in}}{\pgfqpoint{0.033333in}{-0.008840in}}{\pgfqpoint{0.033333in}{0.000000in}}%
\pgfpathcurveto{\pgfqpoint{0.033333in}{0.008840in}}{\pgfqpoint{0.029821in}{0.017319in}}{\pgfqpoint{0.023570in}{0.023570in}}%
\pgfpathcurveto{\pgfqpoint{0.017319in}{0.029821in}}{\pgfqpoint{0.008840in}{0.033333in}}{\pgfqpoint{0.000000in}{0.033333in}}%
\pgfpathcurveto{\pgfqpoint{-0.008840in}{0.033333in}}{\pgfqpoint{-0.017319in}{0.029821in}}{\pgfqpoint{-0.023570in}{0.023570in}}%
\pgfpathcurveto{\pgfqpoint{-0.029821in}{0.017319in}}{\pgfqpoint{-0.033333in}{0.008840in}}{\pgfqpoint{-0.033333in}{0.000000in}}%
\pgfpathcurveto{\pgfqpoint{-0.033333in}{-0.008840in}}{\pgfqpoint{-0.029821in}{-0.017319in}}{\pgfqpoint{-0.023570in}{-0.023570in}}%
\pgfpathcurveto{\pgfqpoint{-0.017319in}{-0.029821in}}{\pgfqpoint{-0.008840in}{-0.033333in}}{\pgfqpoint{0.000000in}{-0.033333in}}%
\pgfpathlineto{\pgfqpoint{0.000000in}{-0.033333in}}%
\pgfpathclose%
\pgfusepath{stroke,fill}%
}%
\begin{pgfscope}%
\pgfsys@transformshift{3.166263in}{2.505480in}%
\pgfsys@useobject{currentmarker}{}%
\end{pgfscope}%
\end{pgfscope}%
\begin{pgfscope}%
\pgfpathrectangle{\pgfqpoint{0.765000in}{0.660000in}}{\pgfqpoint{4.620000in}{4.620000in}}%
\pgfusepath{clip}%
\pgfsetroundcap%
\pgfsetroundjoin%
\pgfsetlinewidth{1.204500pt}%
\definecolor{currentstroke}{rgb}{1.000000,0.576471,0.309804}%
\pgfsetstrokecolor{currentstroke}%
\pgfsetdash{}{0pt}%
\pgfpathmoveto{\pgfqpoint{3.396905in}{2.343341in}}%
\pgfusepath{stroke}%
\end{pgfscope}%
\begin{pgfscope}%
\pgfpathrectangle{\pgfqpoint{0.765000in}{0.660000in}}{\pgfqpoint{4.620000in}{4.620000in}}%
\pgfusepath{clip}%
\pgfsetbuttcap%
\pgfsetroundjoin%
\definecolor{currentfill}{rgb}{1.000000,0.576471,0.309804}%
\pgfsetfillcolor{currentfill}%
\pgfsetlinewidth{1.003750pt}%
\definecolor{currentstroke}{rgb}{1.000000,0.576471,0.309804}%
\pgfsetstrokecolor{currentstroke}%
\pgfsetdash{}{0pt}%
\pgfsys@defobject{currentmarker}{\pgfqpoint{-0.033333in}{-0.033333in}}{\pgfqpoint{0.033333in}{0.033333in}}{%
\pgfpathmoveto{\pgfqpoint{0.000000in}{-0.033333in}}%
\pgfpathcurveto{\pgfqpoint{0.008840in}{-0.033333in}}{\pgfqpoint{0.017319in}{-0.029821in}}{\pgfqpoint{0.023570in}{-0.023570in}}%
\pgfpathcurveto{\pgfqpoint{0.029821in}{-0.017319in}}{\pgfqpoint{0.033333in}{-0.008840in}}{\pgfqpoint{0.033333in}{0.000000in}}%
\pgfpathcurveto{\pgfqpoint{0.033333in}{0.008840in}}{\pgfqpoint{0.029821in}{0.017319in}}{\pgfqpoint{0.023570in}{0.023570in}}%
\pgfpathcurveto{\pgfqpoint{0.017319in}{0.029821in}}{\pgfqpoint{0.008840in}{0.033333in}}{\pgfqpoint{0.000000in}{0.033333in}}%
\pgfpathcurveto{\pgfqpoint{-0.008840in}{0.033333in}}{\pgfqpoint{-0.017319in}{0.029821in}}{\pgfqpoint{-0.023570in}{0.023570in}}%
\pgfpathcurveto{\pgfqpoint{-0.029821in}{0.017319in}}{\pgfqpoint{-0.033333in}{0.008840in}}{\pgfqpoint{-0.033333in}{0.000000in}}%
\pgfpathcurveto{\pgfqpoint{-0.033333in}{-0.008840in}}{\pgfqpoint{-0.029821in}{-0.017319in}}{\pgfqpoint{-0.023570in}{-0.023570in}}%
\pgfpathcurveto{\pgfqpoint{-0.017319in}{-0.029821in}}{\pgfqpoint{-0.008840in}{-0.033333in}}{\pgfqpoint{0.000000in}{-0.033333in}}%
\pgfpathlineto{\pgfqpoint{0.000000in}{-0.033333in}}%
\pgfpathclose%
\pgfusepath{stroke,fill}%
}%
\begin{pgfscope}%
\pgfsys@transformshift{3.396905in}{2.343341in}%
\pgfsys@useobject{currentmarker}{}%
\end{pgfscope}%
\end{pgfscope}%
\begin{pgfscope}%
\pgfpathrectangle{\pgfqpoint{0.765000in}{0.660000in}}{\pgfqpoint{4.620000in}{4.620000in}}%
\pgfusepath{clip}%
\pgfsetroundcap%
\pgfsetroundjoin%
\pgfsetlinewidth{1.204500pt}%
\definecolor{currentstroke}{rgb}{1.000000,0.576471,0.309804}%
\pgfsetstrokecolor{currentstroke}%
\pgfsetdash{}{0pt}%
\pgfpathmoveto{\pgfqpoint{2.416675in}{2.262271in}}%
\pgfusepath{stroke}%
\end{pgfscope}%
\begin{pgfscope}%
\pgfpathrectangle{\pgfqpoint{0.765000in}{0.660000in}}{\pgfqpoint{4.620000in}{4.620000in}}%
\pgfusepath{clip}%
\pgfsetbuttcap%
\pgfsetroundjoin%
\definecolor{currentfill}{rgb}{1.000000,0.576471,0.309804}%
\pgfsetfillcolor{currentfill}%
\pgfsetlinewidth{1.003750pt}%
\definecolor{currentstroke}{rgb}{1.000000,0.576471,0.309804}%
\pgfsetstrokecolor{currentstroke}%
\pgfsetdash{}{0pt}%
\pgfsys@defobject{currentmarker}{\pgfqpoint{-0.033333in}{-0.033333in}}{\pgfqpoint{0.033333in}{0.033333in}}{%
\pgfpathmoveto{\pgfqpoint{0.000000in}{-0.033333in}}%
\pgfpathcurveto{\pgfqpoint{0.008840in}{-0.033333in}}{\pgfqpoint{0.017319in}{-0.029821in}}{\pgfqpoint{0.023570in}{-0.023570in}}%
\pgfpathcurveto{\pgfqpoint{0.029821in}{-0.017319in}}{\pgfqpoint{0.033333in}{-0.008840in}}{\pgfqpoint{0.033333in}{0.000000in}}%
\pgfpathcurveto{\pgfqpoint{0.033333in}{0.008840in}}{\pgfqpoint{0.029821in}{0.017319in}}{\pgfqpoint{0.023570in}{0.023570in}}%
\pgfpathcurveto{\pgfqpoint{0.017319in}{0.029821in}}{\pgfqpoint{0.008840in}{0.033333in}}{\pgfqpoint{0.000000in}{0.033333in}}%
\pgfpathcurveto{\pgfqpoint{-0.008840in}{0.033333in}}{\pgfqpoint{-0.017319in}{0.029821in}}{\pgfqpoint{-0.023570in}{0.023570in}}%
\pgfpathcurveto{\pgfqpoint{-0.029821in}{0.017319in}}{\pgfqpoint{-0.033333in}{0.008840in}}{\pgfqpoint{-0.033333in}{0.000000in}}%
\pgfpathcurveto{\pgfqpoint{-0.033333in}{-0.008840in}}{\pgfqpoint{-0.029821in}{-0.017319in}}{\pgfqpoint{-0.023570in}{-0.023570in}}%
\pgfpathcurveto{\pgfqpoint{-0.017319in}{-0.029821in}}{\pgfqpoint{-0.008840in}{-0.033333in}}{\pgfqpoint{0.000000in}{-0.033333in}}%
\pgfpathlineto{\pgfqpoint{0.000000in}{-0.033333in}}%
\pgfpathclose%
\pgfusepath{stroke,fill}%
}%
\begin{pgfscope}%
\pgfsys@transformshift{2.416675in}{2.262271in}%
\pgfsys@useobject{currentmarker}{}%
\end{pgfscope}%
\end{pgfscope}%
\begin{pgfscope}%
\pgfpathrectangle{\pgfqpoint{0.765000in}{0.660000in}}{\pgfqpoint{4.620000in}{4.620000in}}%
\pgfusepath{clip}%
\pgfsetroundcap%
\pgfsetroundjoin%
\pgfsetlinewidth{1.204500pt}%
\definecolor{currentstroke}{rgb}{1.000000,0.576471,0.309804}%
\pgfsetstrokecolor{currentstroke}%
\pgfsetdash{}{0pt}%
\pgfpathmoveto{\pgfqpoint{3.281584in}{2.181202in}}%
\pgfusepath{stroke}%
\end{pgfscope}%
\begin{pgfscope}%
\pgfpathrectangle{\pgfqpoint{0.765000in}{0.660000in}}{\pgfqpoint{4.620000in}{4.620000in}}%
\pgfusepath{clip}%
\pgfsetbuttcap%
\pgfsetroundjoin%
\definecolor{currentfill}{rgb}{1.000000,0.576471,0.309804}%
\pgfsetfillcolor{currentfill}%
\pgfsetlinewidth{1.003750pt}%
\definecolor{currentstroke}{rgb}{1.000000,0.576471,0.309804}%
\pgfsetstrokecolor{currentstroke}%
\pgfsetdash{}{0pt}%
\pgfsys@defobject{currentmarker}{\pgfqpoint{-0.033333in}{-0.033333in}}{\pgfqpoint{0.033333in}{0.033333in}}{%
\pgfpathmoveto{\pgfqpoint{0.000000in}{-0.033333in}}%
\pgfpathcurveto{\pgfqpoint{0.008840in}{-0.033333in}}{\pgfqpoint{0.017319in}{-0.029821in}}{\pgfqpoint{0.023570in}{-0.023570in}}%
\pgfpathcurveto{\pgfqpoint{0.029821in}{-0.017319in}}{\pgfqpoint{0.033333in}{-0.008840in}}{\pgfqpoint{0.033333in}{0.000000in}}%
\pgfpathcurveto{\pgfqpoint{0.033333in}{0.008840in}}{\pgfqpoint{0.029821in}{0.017319in}}{\pgfqpoint{0.023570in}{0.023570in}}%
\pgfpathcurveto{\pgfqpoint{0.017319in}{0.029821in}}{\pgfqpoint{0.008840in}{0.033333in}}{\pgfqpoint{0.000000in}{0.033333in}}%
\pgfpathcurveto{\pgfqpoint{-0.008840in}{0.033333in}}{\pgfqpoint{-0.017319in}{0.029821in}}{\pgfqpoint{-0.023570in}{0.023570in}}%
\pgfpathcurveto{\pgfqpoint{-0.029821in}{0.017319in}}{\pgfqpoint{-0.033333in}{0.008840in}}{\pgfqpoint{-0.033333in}{0.000000in}}%
\pgfpathcurveto{\pgfqpoint{-0.033333in}{-0.008840in}}{\pgfqpoint{-0.029821in}{-0.017319in}}{\pgfqpoint{-0.023570in}{-0.023570in}}%
\pgfpathcurveto{\pgfqpoint{-0.017319in}{-0.029821in}}{\pgfqpoint{-0.008840in}{-0.033333in}}{\pgfqpoint{0.000000in}{-0.033333in}}%
\pgfpathlineto{\pgfqpoint{0.000000in}{-0.033333in}}%
\pgfpathclose%
\pgfusepath{stroke,fill}%
}%
\begin{pgfscope}%
\pgfsys@transformshift{3.281584in}{2.181202in}%
\pgfsys@useobject{currentmarker}{}%
\end{pgfscope}%
\end{pgfscope}%
\begin{pgfscope}%
\pgfpathrectangle{\pgfqpoint{0.765000in}{0.660000in}}{\pgfqpoint{4.620000in}{4.620000in}}%
\pgfusepath{clip}%
\pgfsetroundcap%
\pgfsetroundjoin%
\pgfsetlinewidth{1.204500pt}%
\definecolor{currentstroke}{rgb}{1.000000,0.576471,0.309804}%
\pgfsetstrokecolor{currentstroke}%
\pgfsetdash{}{0pt}%
\pgfpathmoveto{\pgfqpoint{2.589657in}{2.181202in}}%
\pgfusepath{stroke}%
\end{pgfscope}%
\begin{pgfscope}%
\pgfpathrectangle{\pgfqpoint{0.765000in}{0.660000in}}{\pgfqpoint{4.620000in}{4.620000in}}%
\pgfusepath{clip}%
\pgfsetbuttcap%
\pgfsetroundjoin%
\definecolor{currentfill}{rgb}{1.000000,0.576471,0.309804}%
\pgfsetfillcolor{currentfill}%
\pgfsetlinewidth{1.003750pt}%
\definecolor{currentstroke}{rgb}{1.000000,0.576471,0.309804}%
\pgfsetstrokecolor{currentstroke}%
\pgfsetdash{}{0pt}%
\pgfsys@defobject{currentmarker}{\pgfqpoint{-0.033333in}{-0.033333in}}{\pgfqpoint{0.033333in}{0.033333in}}{%
\pgfpathmoveto{\pgfqpoint{0.000000in}{-0.033333in}}%
\pgfpathcurveto{\pgfqpoint{0.008840in}{-0.033333in}}{\pgfqpoint{0.017319in}{-0.029821in}}{\pgfqpoint{0.023570in}{-0.023570in}}%
\pgfpathcurveto{\pgfqpoint{0.029821in}{-0.017319in}}{\pgfqpoint{0.033333in}{-0.008840in}}{\pgfqpoint{0.033333in}{0.000000in}}%
\pgfpathcurveto{\pgfqpoint{0.033333in}{0.008840in}}{\pgfqpoint{0.029821in}{0.017319in}}{\pgfqpoint{0.023570in}{0.023570in}}%
\pgfpathcurveto{\pgfqpoint{0.017319in}{0.029821in}}{\pgfqpoint{0.008840in}{0.033333in}}{\pgfqpoint{0.000000in}{0.033333in}}%
\pgfpathcurveto{\pgfqpoint{-0.008840in}{0.033333in}}{\pgfqpoint{-0.017319in}{0.029821in}}{\pgfqpoint{-0.023570in}{0.023570in}}%
\pgfpathcurveto{\pgfqpoint{-0.029821in}{0.017319in}}{\pgfqpoint{-0.033333in}{0.008840in}}{\pgfqpoint{-0.033333in}{0.000000in}}%
\pgfpathcurveto{\pgfqpoint{-0.033333in}{-0.008840in}}{\pgfqpoint{-0.029821in}{-0.017319in}}{\pgfqpoint{-0.023570in}{-0.023570in}}%
\pgfpathcurveto{\pgfqpoint{-0.017319in}{-0.029821in}}{\pgfqpoint{-0.008840in}{-0.033333in}}{\pgfqpoint{0.000000in}{-0.033333in}}%
\pgfpathlineto{\pgfqpoint{0.000000in}{-0.033333in}}%
\pgfpathclose%
\pgfusepath{stroke,fill}%
}%
\begin{pgfscope}%
\pgfsys@transformshift{2.589657in}{2.181202in}%
\pgfsys@useobject{currentmarker}{}%
\end{pgfscope}%
\end{pgfscope}%
\begin{pgfscope}%
\pgfpathrectangle{\pgfqpoint{0.765000in}{0.660000in}}{\pgfqpoint{4.620000in}{4.620000in}}%
\pgfusepath{clip}%
\pgfsetroundcap%
\pgfsetroundjoin%
\pgfsetlinewidth{1.204500pt}%
\definecolor{currentstroke}{rgb}{1.000000,0.576471,0.309804}%
\pgfsetstrokecolor{currentstroke}%
\pgfsetdash{}{0pt}%
\pgfpathmoveto{\pgfqpoint{3.166263in}{2.181202in}}%
\pgfusepath{stroke}%
\end{pgfscope}%
\begin{pgfscope}%
\pgfpathrectangle{\pgfqpoint{0.765000in}{0.660000in}}{\pgfqpoint{4.620000in}{4.620000in}}%
\pgfusepath{clip}%
\pgfsetbuttcap%
\pgfsetroundjoin%
\definecolor{currentfill}{rgb}{1.000000,0.576471,0.309804}%
\pgfsetfillcolor{currentfill}%
\pgfsetlinewidth{1.003750pt}%
\definecolor{currentstroke}{rgb}{1.000000,0.576471,0.309804}%
\pgfsetstrokecolor{currentstroke}%
\pgfsetdash{}{0pt}%
\pgfsys@defobject{currentmarker}{\pgfqpoint{-0.033333in}{-0.033333in}}{\pgfqpoint{0.033333in}{0.033333in}}{%
\pgfpathmoveto{\pgfqpoint{0.000000in}{-0.033333in}}%
\pgfpathcurveto{\pgfqpoint{0.008840in}{-0.033333in}}{\pgfqpoint{0.017319in}{-0.029821in}}{\pgfqpoint{0.023570in}{-0.023570in}}%
\pgfpathcurveto{\pgfqpoint{0.029821in}{-0.017319in}}{\pgfqpoint{0.033333in}{-0.008840in}}{\pgfqpoint{0.033333in}{0.000000in}}%
\pgfpathcurveto{\pgfqpoint{0.033333in}{0.008840in}}{\pgfqpoint{0.029821in}{0.017319in}}{\pgfqpoint{0.023570in}{0.023570in}}%
\pgfpathcurveto{\pgfqpoint{0.017319in}{0.029821in}}{\pgfqpoint{0.008840in}{0.033333in}}{\pgfqpoint{0.000000in}{0.033333in}}%
\pgfpathcurveto{\pgfqpoint{-0.008840in}{0.033333in}}{\pgfqpoint{-0.017319in}{0.029821in}}{\pgfqpoint{-0.023570in}{0.023570in}}%
\pgfpathcurveto{\pgfqpoint{-0.029821in}{0.017319in}}{\pgfqpoint{-0.033333in}{0.008840in}}{\pgfqpoint{-0.033333in}{0.000000in}}%
\pgfpathcurveto{\pgfqpoint{-0.033333in}{-0.008840in}}{\pgfqpoint{-0.029821in}{-0.017319in}}{\pgfqpoint{-0.023570in}{-0.023570in}}%
\pgfpathcurveto{\pgfqpoint{-0.017319in}{-0.029821in}}{\pgfqpoint{-0.008840in}{-0.033333in}}{\pgfqpoint{0.000000in}{-0.033333in}}%
\pgfpathlineto{\pgfqpoint{0.000000in}{-0.033333in}}%
\pgfpathclose%
\pgfusepath{stroke,fill}%
}%
\begin{pgfscope}%
\pgfsys@transformshift{3.166263in}{2.181202in}%
\pgfsys@useobject{currentmarker}{}%
\end{pgfscope}%
\end{pgfscope}%
\begin{pgfscope}%
\pgfpathrectangle{\pgfqpoint{0.765000in}{0.660000in}}{\pgfqpoint{4.620000in}{4.620000in}}%
\pgfusepath{clip}%
\pgfsetroundcap%
\pgfsetroundjoin%
\pgfsetlinewidth{1.204500pt}%
\definecolor{currentstroke}{rgb}{1.000000,0.576471,0.309804}%
\pgfsetstrokecolor{currentstroke}%
\pgfsetdash{}{0pt}%
\pgfpathmoveto{\pgfqpoint{2.704978in}{2.181202in}}%
\pgfusepath{stroke}%
\end{pgfscope}%
\begin{pgfscope}%
\pgfpathrectangle{\pgfqpoint{0.765000in}{0.660000in}}{\pgfqpoint{4.620000in}{4.620000in}}%
\pgfusepath{clip}%
\pgfsetbuttcap%
\pgfsetroundjoin%
\definecolor{currentfill}{rgb}{1.000000,0.576471,0.309804}%
\pgfsetfillcolor{currentfill}%
\pgfsetlinewidth{1.003750pt}%
\definecolor{currentstroke}{rgb}{1.000000,0.576471,0.309804}%
\pgfsetstrokecolor{currentstroke}%
\pgfsetdash{}{0pt}%
\pgfsys@defobject{currentmarker}{\pgfqpoint{-0.033333in}{-0.033333in}}{\pgfqpoint{0.033333in}{0.033333in}}{%
\pgfpathmoveto{\pgfqpoint{0.000000in}{-0.033333in}}%
\pgfpathcurveto{\pgfqpoint{0.008840in}{-0.033333in}}{\pgfqpoint{0.017319in}{-0.029821in}}{\pgfqpoint{0.023570in}{-0.023570in}}%
\pgfpathcurveto{\pgfqpoint{0.029821in}{-0.017319in}}{\pgfqpoint{0.033333in}{-0.008840in}}{\pgfqpoint{0.033333in}{0.000000in}}%
\pgfpathcurveto{\pgfqpoint{0.033333in}{0.008840in}}{\pgfqpoint{0.029821in}{0.017319in}}{\pgfqpoint{0.023570in}{0.023570in}}%
\pgfpathcurveto{\pgfqpoint{0.017319in}{0.029821in}}{\pgfqpoint{0.008840in}{0.033333in}}{\pgfqpoint{0.000000in}{0.033333in}}%
\pgfpathcurveto{\pgfqpoint{-0.008840in}{0.033333in}}{\pgfqpoint{-0.017319in}{0.029821in}}{\pgfqpoint{-0.023570in}{0.023570in}}%
\pgfpathcurveto{\pgfqpoint{-0.029821in}{0.017319in}}{\pgfqpoint{-0.033333in}{0.008840in}}{\pgfqpoint{-0.033333in}{0.000000in}}%
\pgfpathcurveto{\pgfqpoint{-0.033333in}{-0.008840in}}{\pgfqpoint{-0.029821in}{-0.017319in}}{\pgfqpoint{-0.023570in}{-0.023570in}}%
\pgfpathcurveto{\pgfqpoint{-0.017319in}{-0.029821in}}{\pgfqpoint{-0.008840in}{-0.033333in}}{\pgfqpoint{0.000000in}{-0.033333in}}%
\pgfpathlineto{\pgfqpoint{0.000000in}{-0.033333in}}%
\pgfpathclose%
\pgfusepath{stroke,fill}%
}%
\begin{pgfscope}%
\pgfsys@transformshift{2.704978in}{2.181202in}%
\pgfsys@useobject{currentmarker}{}%
\end{pgfscope}%
\end{pgfscope}%
\begin{pgfscope}%
\pgfpathrectangle{\pgfqpoint{0.765000in}{0.660000in}}{\pgfqpoint{4.620000in}{4.620000in}}%
\pgfusepath{clip}%
\pgfsetroundcap%
\pgfsetroundjoin%
\pgfsetlinewidth{1.204500pt}%
\definecolor{currentstroke}{rgb}{0.176471,0.192157,0.258824}%
\pgfsetstrokecolor{currentstroke}%
\pgfsetdash{}{0pt}%
\pgfpathmoveto{\pgfqpoint{4.031172in}{4.045802in}}%
\pgfusepath{stroke}%
\end{pgfscope}%
\begin{pgfscope}%
\pgfpathrectangle{\pgfqpoint{0.765000in}{0.660000in}}{\pgfqpoint{4.620000in}{4.620000in}}%
\pgfusepath{clip}%
\pgfsetbuttcap%
\pgfsetmiterjoin%
\definecolor{currentfill}{rgb}{0.176471,0.192157,0.258824}%
\pgfsetfillcolor{currentfill}%
\pgfsetlinewidth{1.003750pt}%
\definecolor{currentstroke}{rgb}{0.176471,0.192157,0.258824}%
\pgfsetstrokecolor{currentstroke}%
\pgfsetdash{}{0pt}%
\pgfsys@defobject{currentmarker}{\pgfqpoint{-0.033333in}{-0.033333in}}{\pgfqpoint{0.033333in}{0.033333in}}{%
\pgfpathmoveto{\pgfqpoint{-0.000000in}{-0.033333in}}%
\pgfpathlineto{\pgfqpoint{0.033333in}{0.033333in}}%
\pgfpathlineto{\pgfqpoint{-0.033333in}{0.033333in}}%
\pgfpathlineto{\pgfqpoint{-0.000000in}{-0.033333in}}%
\pgfpathclose%
\pgfusepath{stroke,fill}%
}%
\begin{pgfscope}%
\pgfsys@transformshift{4.031172in}{4.045802in}%
\pgfsys@useobject{currentmarker}{}%
\end{pgfscope}%
\end{pgfscope}%
\begin{pgfscope}%
\pgfpathrectangle{\pgfqpoint{0.765000in}{0.660000in}}{\pgfqpoint{4.620000in}{4.620000in}}%
\pgfusepath{clip}%
\pgfsetroundcap%
\pgfsetroundjoin%
\pgfsetlinewidth{1.204500pt}%
\definecolor{currentstroke}{rgb}{0.176471,0.192157,0.258824}%
\pgfsetstrokecolor{currentstroke}%
\pgfsetdash{}{0pt}%
\pgfpathmoveto{\pgfqpoint{2.993281in}{3.559385in}}%
\pgfusepath{stroke}%
\end{pgfscope}%
\begin{pgfscope}%
\pgfpathrectangle{\pgfqpoint{0.765000in}{0.660000in}}{\pgfqpoint{4.620000in}{4.620000in}}%
\pgfusepath{clip}%
\pgfsetbuttcap%
\pgfsetmiterjoin%
\definecolor{currentfill}{rgb}{0.176471,0.192157,0.258824}%
\pgfsetfillcolor{currentfill}%
\pgfsetlinewidth{1.003750pt}%
\definecolor{currentstroke}{rgb}{0.176471,0.192157,0.258824}%
\pgfsetstrokecolor{currentstroke}%
\pgfsetdash{}{0pt}%
\pgfsys@defobject{currentmarker}{\pgfqpoint{-0.033333in}{-0.033333in}}{\pgfqpoint{0.033333in}{0.033333in}}{%
\pgfpathmoveto{\pgfqpoint{-0.000000in}{-0.033333in}}%
\pgfpathlineto{\pgfqpoint{0.033333in}{0.033333in}}%
\pgfpathlineto{\pgfqpoint{-0.033333in}{0.033333in}}%
\pgfpathlineto{\pgfqpoint{-0.000000in}{-0.033333in}}%
\pgfpathclose%
\pgfusepath{stroke,fill}%
}%
\begin{pgfscope}%
\pgfsys@transformshift{2.993281in}{3.559385in}%
\pgfsys@useobject{currentmarker}{}%
\end{pgfscope}%
\end{pgfscope}%
\begin{pgfscope}%
\pgfpathrectangle{\pgfqpoint{0.765000in}{0.660000in}}{\pgfqpoint{4.620000in}{4.620000in}}%
\pgfusepath{clip}%
\pgfsetroundcap%
\pgfsetroundjoin%
\pgfsetlinewidth{1.204500pt}%
\definecolor{currentstroke}{rgb}{0.176471,0.192157,0.258824}%
\pgfsetstrokecolor{currentstroke}%
\pgfsetdash{}{0pt}%
\pgfpathmoveto{\pgfqpoint{3.454566in}{3.721524in}}%
\pgfusepath{stroke}%
\end{pgfscope}%
\begin{pgfscope}%
\pgfpathrectangle{\pgfqpoint{0.765000in}{0.660000in}}{\pgfqpoint{4.620000in}{4.620000in}}%
\pgfusepath{clip}%
\pgfsetbuttcap%
\pgfsetmiterjoin%
\definecolor{currentfill}{rgb}{0.176471,0.192157,0.258824}%
\pgfsetfillcolor{currentfill}%
\pgfsetlinewidth{1.003750pt}%
\definecolor{currentstroke}{rgb}{0.176471,0.192157,0.258824}%
\pgfsetstrokecolor{currentstroke}%
\pgfsetdash{}{0pt}%
\pgfsys@defobject{currentmarker}{\pgfqpoint{-0.033333in}{-0.033333in}}{\pgfqpoint{0.033333in}{0.033333in}}{%
\pgfpathmoveto{\pgfqpoint{-0.000000in}{-0.033333in}}%
\pgfpathlineto{\pgfqpoint{0.033333in}{0.033333in}}%
\pgfpathlineto{\pgfqpoint{-0.033333in}{0.033333in}}%
\pgfpathlineto{\pgfqpoint{-0.000000in}{-0.033333in}}%
\pgfpathclose%
\pgfusepath{stroke,fill}%
}%
\begin{pgfscope}%
\pgfsys@transformshift{3.454566in}{3.721524in}%
\pgfsys@useobject{currentmarker}{}%
\end{pgfscope}%
\end{pgfscope}%
\begin{pgfscope}%
\pgfpathrectangle{\pgfqpoint{0.765000in}{0.660000in}}{\pgfqpoint{4.620000in}{4.620000in}}%
\pgfusepath{clip}%
\pgfsetroundcap%
\pgfsetroundjoin%
\pgfsetlinewidth{1.204500pt}%
\definecolor{currentstroke}{rgb}{0.176471,0.192157,0.258824}%
\pgfsetstrokecolor{currentstroke}%
\pgfsetdash{}{0pt}%
\pgfpathmoveto{\pgfqpoint{3.166263in}{3.478315in}}%
\pgfusepath{stroke}%
\end{pgfscope}%
\begin{pgfscope}%
\pgfpathrectangle{\pgfqpoint{0.765000in}{0.660000in}}{\pgfqpoint{4.620000in}{4.620000in}}%
\pgfusepath{clip}%
\pgfsetbuttcap%
\pgfsetmiterjoin%
\definecolor{currentfill}{rgb}{0.176471,0.192157,0.258824}%
\pgfsetfillcolor{currentfill}%
\pgfsetlinewidth{1.003750pt}%
\definecolor{currentstroke}{rgb}{0.176471,0.192157,0.258824}%
\pgfsetstrokecolor{currentstroke}%
\pgfsetdash{}{0pt}%
\pgfsys@defobject{currentmarker}{\pgfqpoint{-0.033333in}{-0.033333in}}{\pgfqpoint{0.033333in}{0.033333in}}{%
\pgfpathmoveto{\pgfqpoint{-0.000000in}{-0.033333in}}%
\pgfpathlineto{\pgfqpoint{0.033333in}{0.033333in}}%
\pgfpathlineto{\pgfqpoint{-0.033333in}{0.033333in}}%
\pgfpathlineto{\pgfqpoint{-0.000000in}{-0.033333in}}%
\pgfpathclose%
\pgfusepath{stroke,fill}%
}%
\begin{pgfscope}%
\pgfsys@transformshift{3.166263in}{3.478315in}%
\pgfsys@useobject{currentmarker}{}%
\end{pgfscope}%
\end{pgfscope}%
\begin{pgfscope}%
\pgfpathrectangle{\pgfqpoint{0.765000in}{0.660000in}}{\pgfqpoint{4.620000in}{4.620000in}}%
\pgfusepath{clip}%
\pgfsetroundcap%
\pgfsetroundjoin%
\pgfsetlinewidth{1.204500pt}%
\definecolor{currentstroke}{rgb}{0.176471,0.192157,0.258824}%
\pgfsetstrokecolor{currentstroke}%
\pgfsetdash{}{0pt}%
\pgfpathmoveto{\pgfqpoint{3.512226in}{3.802594in}}%
\pgfusepath{stroke}%
\end{pgfscope}%
\begin{pgfscope}%
\pgfpathrectangle{\pgfqpoint{0.765000in}{0.660000in}}{\pgfqpoint{4.620000in}{4.620000in}}%
\pgfusepath{clip}%
\pgfsetbuttcap%
\pgfsetmiterjoin%
\definecolor{currentfill}{rgb}{0.176471,0.192157,0.258824}%
\pgfsetfillcolor{currentfill}%
\pgfsetlinewidth{1.003750pt}%
\definecolor{currentstroke}{rgb}{0.176471,0.192157,0.258824}%
\pgfsetstrokecolor{currentstroke}%
\pgfsetdash{}{0pt}%
\pgfsys@defobject{currentmarker}{\pgfqpoint{-0.033333in}{-0.033333in}}{\pgfqpoint{0.033333in}{0.033333in}}{%
\pgfpathmoveto{\pgfqpoint{-0.000000in}{-0.033333in}}%
\pgfpathlineto{\pgfqpoint{0.033333in}{0.033333in}}%
\pgfpathlineto{\pgfqpoint{-0.033333in}{0.033333in}}%
\pgfpathlineto{\pgfqpoint{-0.000000in}{-0.033333in}}%
\pgfpathclose%
\pgfusepath{stroke,fill}%
}%
\begin{pgfscope}%
\pgfsys@transformshift{3.512226in}{3.802594in}%
\pgfsys@useobject{currentmarker}{}%
\end{pgfscope}%
\end{pgfscope}%
\begin{pgfscope}%
\pgfpathrectangle{\pgfqpoint{0.765000in}{0.660000in}}{\pgfqpoint{4.620000in}{4.620000in}}%
\pgfusepath{clip}%
\pgfsetroundcap%
\pgfsetroundjoin%
\pgfsetlinewidth{1.204500pt}%
\definecolor{currentstroke}{rgb}{0.176471,0.192157,0.258824}%
\pgfsetstrokecolor{currentstroke}%
\pgfsetdash{}{0pt}%
\pgfpathmoveto{\pgfqpoint{3.454566in}{3.721524in}}%
\pgfusepath{stroke}%
\end{pgfscope}%
\begin{pgfscope}%
\pgfpathrectangle{\pgfqpoint{0.765000in}{0.660000in}}{\pgfqpoint{4.620000in}{4.620000in}}%
\pgfusepath{clip}%
\pgfsetbuttcap%
\pgfsetmiterjoin%
\definecolor{currentfill}{rgb}{0.176471,0.192157,0.258824}%
\pgfsetfillcolor{currentfill}%
\pgfsetlinewidth{1.003750pt}%
\definecolor{currentstroke}{rgb}{0.176471,0.192157,0.258824}%
\pgfsetstrokecolor{currentstroke}%
\pgfsetdash{}{0pt}%
\pgfsys@defobject{currentmarker}{\pgfqpoint{-0.033333in}{-0.033333in}}{\pgfqpoint{0.033333in}{0.033333in}}{%
\pgfpathmoveto{\pgfqpoint{-0.000000in}{-0.033333in}}%
\pgfpathlineto{\pgfqpoint{0.033333in}{0.033333in}}%
\pgfpathlineto{\pgfqpoint{-0.033333in}{0.033333in}}%
\pgfpathlineto{\pgfqpoint{-0.000000in}{-0.033333in}}%
\pgfpathclose%
\pgfusepath{stroke,fill}%
}%
\begin{pgfscope}%
\pgfsys@transformshift{3.454566in}{3.721524in}%
\pgfsys@useobject{currentmarker}{}%
\end{pgfscope}%
\end{pgfscope}%
\begin{pgfscope}%
\pgfpathrectangle{\pgfqpoint{0.765000in}{0.660000in}}{\pgfqpoint{4.620000in}{4.620000in}}%
\pgfusepath{clip}%
\pgfsetroundcap%
\pgfsetroundjoin%
\pgfsetlinewidth{1.204500pt}%
\definecolor{currentstroke}{rgb}{0.176471,0.192157,0.258824}%
\pgfsetstrokecolor{currentstroke}%
\pgfsetdash{}{0pt}%
\pgfpathmoveto{\pgfqpoint{2.647317in}{3.397246in}}%
\pgfusepath{stroke}%
\end{pgfscope}%
\begin{pgfscope}%
\pgfpathrectangle{\pgfqpoint{0.765000in}{0.660000in}}{\pgfqpoint{4.620000in}{4.620000in}}%
\pgfusepath{clip}%
\pgfsetbuttcap%
\pgfsetmiterjoin%
\definecolor{currentfill}{rgb}{0.176471,0.192157,0.258824}%
\pgfsetfillcolor{currentfill}%
\pgfsetlinewidth{1.003750pt}%
\definecolor{currentstroke}{rgb}{0.176471,0.192157,0.258824}%
\pgfsetstrokecolor{currentstroke}%
\pgfsetdash{}{0pt}%
\pgfsys@defobject{currentmarker}{\pgfqpoint{-0.033333in}{-0.033333in}}{\pgfqpoint{0.033333in}{0.033333in}}{%
\pgfpathmoveto{\pgfqpoint{-0.000000in}{-0.033333in}}%
\pgfpathlineto{\pgfqpoint{0.033333in}{0.033333in}}%
\pgfpathlineto{\pgfqpoint{-0.033333in}{0.033333in}}%
\pgfpathlineto{\pgfqpoint{-0.000000in}{-0.033333in}}%
\pgfpathclose%
\pgfusepath{stroke,fill}%
}%
\begin{pgfscope}%
\pgfsys@transformshift{2.647317in}{3.397246in}%
\pgfsys@useobject{currentmarker}{}%
\end{pgfscope}%
\end{pgfscope}%
\begin{pgfscope}%
\pgfpathrectangle{\pgfqpoint{0.765000in}{0.660000in}}{\pgfqpoint{4.620000in}{4.620000in}}%
\pgfusepath{clip}%
\pgfsetroundcap%
\pgfsetroundjoin%
\pgfsetlinewidth{1.204500pt}%
\definecolor{currentstroke}{rgb}{0.176471,0.192157,0.258824}%
\pgfsetstrokecolor{currentstroke}%
\pgfsetdash{}{0pt}%
\pgfpathmoveto{\pgfqpoint{3.166263in}{3.478315in}}%
\pgfusepath{stroke}%
\end{pgfscope}%
\begin{pgfscope}%
\pgfpathrectangle{\pgfqpoint{0.765000in}{0.660000in}}{\pgfqpoint{4.620000in}{4.620000in}}%
\pgfusepath{clip}%
\pgfsetbuttcap%
\pgfsetmiterjoin%
\definecolor{currentfill}{rgb}{0.176471,0.192157,0.258824}%
\pgfsetfillcolor{currentfill}%
\pgfsetlinewidth{1.003750pt}%
\definecolor{currentstroke}{rgb}{0.176471,0.192157,0.258824}%
\pgfsetstrokecolor{currentstroke}%
\pgfsetdash{}{0pt}%
\pgfsys@defobject{currentmarker}{\pgfqpoint{-0.033333in}{-0.033333in}}{\pgfqpoint{0.033333in}{0.033333in}}{%
\pgfpathmoveto{\pgfqpoint{-0.000000in}{-0.033333in}}%
\pgfpathlineto{\pgfqpoint{0.033333in}{0.033333in}}%
\pgfpathlineto{\pgfqpoint{-0.033333in}{0.033333in}}%
\pgfpathlineto{\pgfqpoint{-0.000000in}{-0.033333in}}%
\pgfpathclose%
\pgfusepath{stroke,fill}%
}%
\begin{pgfscope}%
\pgfsys@transformshift{3.166263in}{3.478315in}%
\pgfsys@useobject{currentmarker}{}%
\end{pgfscope}%
\end{pgfscope}%
\begin{pgfscope}%
\pgfpathrectangle{\pgfqpoint{0.765000in}{0.660000in}}{\pgfqpoint{4.620000in}{4.620000in}}%
\pgfusepath{clip}%
\pgfsetroundcap%
\pgfsetroundjoin%
\pgfsetlinewidth{1.204500pt}%
\definecolor{currentstroke}{rgb}{0.176471,0.192157,0.258824}%
\pgfsetstrokecolor{currentstroke}%
\pgfsetdash{}{0pt}%
\pgfpathmoveto{\pgfqpoint{2.704978in}{3.478315in}}%
\pgfusepath{stroke}%
\end{pgfscope}%
\begin{pgfscope}%
\pgfpathrectangle{\pgfqpoint{0.765000in}{0.660000in}}{\pgfqpoint{4.620000in}{4.620000in}}%
\pgfusepath{clip}%
\pgfsetbuttcap%
\pgfsetmiterjoin%
\definecolor{currentfill}{rgb}{0.176471,0.192157,0.258824}%
\pgfsetfillcolor{currentfill}%
\pgfsetlinewidth{1.003750pt}%
\definecolor{currentstroke}{rgb}{0.176471,0.192157,0.258824}%
\pgfsetstrokecolor{currentstroke}%
\pgfsetdash{}{0pt}%
\pgfsys@defobject{currentmarker}{\pgfqpoint{-0.033333in}{-0.033333in}}{\pgfqpoint{0.033333in}{0.033333in}}{%
\pgfpathmoveto{\pgfqpoint{-0.000000in}{-0.033333in}}%
\pgfpathlineto{\pgfqpoint{0.033333in}{0.033333in}}%
\pgfpathlineto{\pgfqpoint{-0.033333in}{0.033333in}}%
\pgfpathlineto{\pgfqpoint{-0.000000in}{-0.033333in}}%
\pgfpathclose%
\pgfusepath{stroke,fill}%
}%
\begin{pgfscope}%
\pgfsys@transformshift{2.704978in}{3.478315in}%
\pgfsys@useobject{currentmarker}{}%
\end{pgfscope}%
\end{pgfscope}%
\begin{pgfscope}%
\pgfpathrectangle{\pgfqpoint{0.765000in}{0.660000in}}{\pgfqpoint{4.620000in}{4.620000in}}%
\pgfusepath{clip}%
\pgfsetroundcap%
\pgfsetroundjoin%
\pgfsetlinewidth{1.204500pt}%
\definecolor{currentstroke}{rgb}{0.176471,0.192157,0.258824}%
\pgfsetstrokecolor{currentstroke}%
\pgfsetdash{}{0pt}%
\pgfpathmoveto{\pgfqpoint{4.377135in}{4.045802in}}%
\pgfusepath{stroke}%
\end{pgfscope}%
\begin{pgfscope}%
\pgfpathrectangle{\pgfqpoint{0.765000in}{0.660000in}}{\pgfqpoint{4.620000in}{4.620000in}}%
\pgfusepath{clip}%
\pgfsetbuttcap%
\pgfsetmiterjoin%
\definecolor{currentfill}{rgb}{0.176471,0.192157,0.258824}%
\pgfsetfillcolor{currentfill}%
\pgfsetlinewidth{1.003750pt}%
\definecolor{currentstroke}{rgb}{0.176471,0.192157,0.258824}%
\pgfsetstrokecolor{currentstroke}%
\pgfsetdash{}{0pt}%
\pgfsys@defobject{currentmarker}{\pgfqpoint{-0.033333in}{-0.033333in}}{\pgfqpoint{0.033333in}{0.033333in}}{%
\pgfpathmoveto{\pgfqpoint{-0.000000in}{-0.033333in}}%
\pgfpathlineto{\pgfqpoint{0.033333in}{0.033333in}}%
\pgfpathlineto{\pgfqpoint{-0.033333in}{0.033333in}}%
\pgfpathlineto{\pgfqpoint{-0.000000in}{-0.033333in}}%
\pgfpathclose%
\pgfusepath{stroke,fill}%
}%
\begin{pgfscope}%
\pgfsys@transformshift{4.377135in}{4.045802in}%
\pgfsys@useobject{currentmarker}{}%
\end{pgfscope}%
\end{pgfscope}%
\begin{pgfscope}%
\pgfpathrectangle{\pgfqpoint{0.765000in}{0.660000in}}{\pgfqpoint{4.620000in}{4.620000in}}%
\pgfusepath{clip}%
\pgfsetroundcap%
\pgfsetroundjoin%
\pgfsetlinewidth{1.204500pt}%
\definecolor{currentstroke}{rgb}{0.176471,0.192157,0.258824}%
\pgfsetstrokecolor{currentstroke}%
\pgfsetdash{}{0pt}%
\pgfpathmoveto{\pgfqpoint{3.627548in}{3.640454in}}%
\pgfusepath{stroke}%
\end{pgfscope}%
\begin{pgfscope}%
\pgfpathrectangle{\pgfqpoint{0.765000in}{0.660000in}}{\pgfqpoint{4.620000in}{4.620000in}}%
\pgfusepath{clip}%
\pgfsetbuttcap%
\pgfsetmiterjoin%
\definecolor{currentfill}{rgb}{0.176471,0.192157,0.258824}%
\pgfsetfillcolor{currentfill}%
\pgfsetlinewidth{1.003750pt}%
\definecolor{currentstroke}{rgb}{0.176471,0.192157,0.258824}%
\pgfsetstrokecolor{currentstroke}%
\pgfsetdash{}{0pt}%
\pgfsys@defobject{currentmarker}{\pgfqpoint{-0.033333in}{-0.033333in}}{\pgfqpoint{0.033333in}{0.033333in}}{%
\pgfpathmoveto{\pgfqpoint{-0.000000in}{-0.033333in}}%
\pgfpathlineto{\pgfqpoint{0.033333in}{0.033333in}}%
\pgfpathlineto{\pgfqpoint{-0.033333in}{0.033333in}}%
\pgfpathlineto{\pgfqpoint{-0.000000in}{-0.033333in}}%
\pgfpathclose%
\pgfusepath{stroke,fill}%
}%
\begin{pgfscope}%
\pgfsys@transformshift{3.627548in}{3.640454in}%
\pgfsys@useobject{currentmarker}{}%
\end{pgfscope}%
\end{pgfscope}%
\begin{pgfscope}%
\pgfpathrectangle{\pgfqpoint{0.765000in}{0.660000in}}{\pgfqpoint{4.620000in}{4.620000in}}%
\pgfusepath{clip}%
\pgfsetroundcap%
\pgfsetroundjoin%
\pgfsetlinewidth{1.204500pt}%
\definecolor{currentstroke}{rgb}{0.176471,0.192157,0.258824}%
\pgfsetstrokecolor{currentstroke}%
\pgfsetdash{}{0pt}%
\pgfpathmoveto{\pgfqpoint{2.993281in}{3.559385in}}%
\pgfusepath{stroke}%
\end{pgfscope}%
\begin{pgfscope}%
\pgfpathrectangle{\pgfqpoint{0.765000in}{0.660000in}}{\pgfqpoint{4.620000in}{4.620000in}}%
\pgfusepath{clip}%
\pgfsetbuttcap%
\pgfsetmiterjoin%
\definecolor{currentfill}{rgb}{0.176471,0.192157,0.258824}%
\pgfsetfillcolor{currentfill}%
\pgfsetlinewidth{1.003750pt}%
\definecolor{currentstroke}{rgb}{0.176471,0.192157,0.258824}%
\pgfsetstrokecolor{currentstroke}%
\pgfsetdash{}{0pt}%
\pgfsys@defobject{currentmarker}{\pgfqpoint{-0.033333in}{-0.033333in}}{\pgfqpoint{0.033333in}{0.033333in}}{%
\pgfpathmoveto{\pgfqpoint{-0.000000in}{-0.033333in}}%
\pgfpathlineto{\pgfqpoint{0.033333in}{0.033333in}}%
\pgfpathlineto{\pgfqpoint{-0.033333in}{0.033333in}}%
\pgfpathlineto{\pgfqpoint{-0.000000in}{-0.033333in}}%
\pgfpathclose%
\pgfusepath{stroke,fill}%
}%
\begin{pgfscope}%
\pgfsys@transformshift{2.993281in}{3.559385in}%
\pgfsys@useobject{currentmarker}{}%
\end{pgfscope}%
\end{pgfscope}%
\begin{pgfscope}%
\pgfpathrectangle{\pgfqpoint{0.765000in}{0.660000in}}{\pgfqpoint{4.620000in}{4.620000in}}%
\pgfusepath{clip}%
\pgfsetroundcap%
\pgfsetroundjoin%
\pgfsetlinewidth{1.204500pt}%
\definecolor{currentstroke}{rgb}{0.176471,0.192157,0.258824}%
\pgfsetstrokecolor{currentstroke}%
\pgfsetdash{}{0pt}%
\pgfpathmoveto{\pgfqpoint{3.454566in}{3.721524in}}%
\pgfusepath{stroke}%
\end{pgfscope}%
\begin{pgfscope}%
\pgfpathrectangle{\pgfqpoint{0.765000in}{0.660000in}}{\pgfqpoint{4.620000in}{4.620000in}}%
\pgfusepath{clip}%
\pgfsetbuttcap%
\pgfsetmiterjoin%
\definecolor{currentfill}{rgb}{0.176471,0.192157,0.258824}%
\pgfsetfillcolor{currentfill}%
\pgfsetlinewidth{1.003750pt}%
\definecolor{currentstroke}{rgb}{0.176471,0.192157,0.258824}%
\pgfsetstrokecolor{currentstroke}%
\pgfsetdash{}{0pt}%
\pgfsys@defobject{currentmarker}{\pgfqpoint{-0.033333in}{-0.033333in}}{\pgfqpoint{0.033333in}{0.033333in}}{%
\pgfpathmoveto{\pgfqpoint{-0.000000in}{-0.033333in}}%
\pgfpathlineto{\pgfqpoint{0.033333in}{0.033333in}}%
\pgfpathlineto{\pgfqpoint{-0.033333in}{0.033333in}}%
\pgfpathlineto{\pgfqpoint{-0.000000in}{-0.033333in}}%
\pgfpathclose%
\pgfusepath{stroke,fill}%
}%
\begin{pgfscope}%
\pgfsys@transformshift{3.454566in}{3.721524in}%
\pgfsys@useobject{currentmarker}{}%
\end{pgfscope}%
\end{pgfscope}%
\begin{pgfscope}%
\pgfpathrectangle{\pgfqpoint{0.765000in}{0.660000in}}{\pgfqpoint{4.620000in}{4.620000in}}%
\pgfusepath{clip}%
\pgfsetroundcap%
\pgfsetroundjoin%
\pgfsetlinewidth{1.204500pt}%
\definecolor{currentstroke}{rgb}{0.176471,0.192157,0.258824}%
\pgfsetstrokecolor{currentstroke}%
\pgfsetdash{}{0pt}%
\pgfpathmoveto{\pgfqpoint{2.820299in}{3.640454in}}%
\pgfusepath{stroke}%
\end{pgfscope}%
\begin{pgfscope}%
\pgfpathrectangle{\pgfqpoint{0.765000in}{0.660000in}}{\pgfqpoint{4.620000in}{4.620000in}}%
\pgfusepath{clip}%
\pgfsetbuttcap%
\pgfsetmiterjoin%
\definecolor{currentfill}{rgb}{0.176471,0.192157,0.258824}%
\pgfsetfillcolor{currentfill}%
\pgfsetlinewidth{1.003750pt}%
\definecolor{currentstroke}{rgb}{0.176471,0.192157,0.258824}%
\pgfsetstrokecolor{currentstroke}%
\pgfsetdash{}{0pt}%
\pgfsys@defobject{currentmarker}{\pgfqpoint{-0.033333in}{-0.033333in}}{\pgfqpoint{0.033333in}{0.033333in}}{%
\pgfpathmoveto{\pgfqpoint{-0.000000in}{-0.033333in}}%
\pgfpathlineto{\pgfqpoint{0.033333in}{0.033333in}}%
\pgfpathlineto{\pgfqpoint{-0.033333in}{0.033333in}}%
\pgfpathlineto{\pgfqpoint{-0.000000in}{-0.033333in}}%
\pgfpathclose%
\pgfusepath{stroke,fill}%
}%
\begin{pgfscope}%
\pgfsys@transformshift{2.820299in}{3.640454in}%
\pgfsys@useobject{currentmarker}{}%
\end{pgfscope}%
\end{pgfscope}%
\begin{pgfscope}%
\pgfpathrectangle{\pgfqpoint{0.765000in}{0.660000in}}{\pgfqpoint{4.620000in}{4.620000in}}%
\pgfusepath{clip}%
\pgfsetroundcap%
\pgfsetroundjoin%
\pgfsetlinewidth{1.204500pt}%
\definecolor{currentstroke}{rgb}{0.176471,0.192157,0.258824}%
\pgfsetstrokecolor{currentstroke}%
\pgfsetdash{}{0pt}%
\pgfpathmoveto{\pgfqpoint{3.396905in}{3.964733in}}%
\pgfusepath{stroke}%
\end{pgfscope}%
\begin{pgfscope}%
\pgfpathrectangle{\pgfqpoint{0.765000in}{0.660000in}}{\pgfqpoint{4.620000in}{4.620000in}}%
\pgfusepath{clip}%
\pgfsetbuttcap%
\pgfsetmiterjoin%
\definecolor{currentfill}{rgb}{0.176471,0.192157,0.258824}%
\pgfsetfillcolor{currentfill}%
\pgfsetlinewidth{1.003750pt}%
\definecolor{currentstroke}{rgb}{0.176471,0.192157,0.258824}%
\pgfsetstrokecolor{currentstroke}%
\pgfsetdash{}{0pt}%
\pgfsys@defobject{currentmarker}{\pgfqpoint{-0.033333in}{-0.033333in}}{\pgfqpoint{0.033333in}{0.033333in}}{%
\pgfpathmoveto{\pgfqpoint{-0.000000in}{-0.033333in}}%
\pgfpathlineto{\pgfqpoint{0.033333in}{0.033333in}}%
\pgfpathlineto{\pgfqpoint{-0.033333in}{0.033333in}}%
\pgfpathlineto{\pgfqpoint{-0.000000in}{-0.033333in}}%
\pgfpathclose%
\pgfusepath{stroke,fill}%
}%
\begin{pgfscope}%
\pgfsys@transformshift{3.396905in}{3.964733in}%
\pgfsys@useobject{currentmarker}{}%
\end{pgfscope}%
\end{pgfscope}%
\begin{pgfscope}%
\pgfpathrectangle{\pgfqpoint{0.765000in}{0.660000in}}{\pgfqpoint{4.620000in}{4.620000in}}%
\pgfusepath{clip}%
\pgfsetroundcap%
\pgfsetroundjoin%
\pgfsetlinewidth{1.204500pt}%
\definecolor{currentstroke}{rgb}{0.176471,0.192157,0.258824}%
\pgfsetstrokecolor{currentstroke}%
\pgfsetdash{}{0pt}%
\pgfpathmoveto{\pgfqpoint{3.800529in}{3.883663in}}%
\pgfusepath{stroke}%
\end{pgfscope}%
\begin{pgfscope}%
\pgfpathrectangle{\pgfqpoint{0.765000in}{0.660000in}}{\pgfqpoint{4.620000in}{4.620000in}}%
\pgfusepath{clip}%
\pgfsetbuttcap%
\pgfsetmiterjoin%
\definecolor{currentfill}{rgb}{0.176471,0.192157,0.258824}%
\pgfsetfillcolor{currentfill}%
\pgfsetlinewidth{1.003750pt}%
\definecolor{currentstroke}{rgb}{0.176471,0.192157,0.258824}%
\pgfsetstrokecolor{currentstroke}%
\pgfsetdash{}{0pt}%
\pgfsys@defobject{currentmarker}{\pgfqpoint{-0.033333in}{-0.033333in}}{\pgfqpoint{0.033333in}{0.033333in}}{%
\pgfpathmoveto{\pgfqpoint{-0.000000in}{-0.033333in}}%
\pgfpathlineto{\pgfqpoint{0.033333in}{0.033333in}}%
\pgfpathlineto{\pgfqpoint{-0.033333in}{0.033333in}}%
\pgfpathlineto{\pgfqpoint{-0.000000in}{-0.033333in}}%
\pgfpathclose%
\pgfusepath{stroke,fill}%
}%
\begin{pgfscope}%
\pgfsys@transformshift{3.800529in}{3.883663in}%
\pgfsys@useobject{currentmarker}{}%
\end{pgfscope}%
\end{pgfscope}%
\begin{pgfscope}%
\pgfpathrectangle{\pgfqpoint{0.765000in}{0.660000in}}{\pgfqpoint{4.620000in}{4.620000in}}%
\pgfusepath{clip}%
\pgfsetroundcap%
\pgfsetroundjoin%
\pgfsetlinewidth{1.204500pt}%
\definecolor{currentstroke}{rgb}{0.176471,0.192157,0.258824}%
\pgfsetstrokecolor{currentstroke}%
\pgfsetdash{}{0pt}%
\pgfpathmoveto{\pgfqpoint{3.281584in}{3.478315in}}%
\pgfusepath{stroke}%
\end{pgfscope}%
\begin{pgfscope}%
\pgfpathrectangle{\pgfqpoint{0.765000in}{0.660000in}}{\pgfqpoint{4.620000in}{4.620000in}}%
\pgfusepath{clip}%
\pgfsetbuttcap%
\pgfsetmiterjoin%
\definecolor{currentfill}{rgb}{0.176471,0.192157,0.258824}%
\pgfsetfillcolor{currentfill}%
\pgfsetlinewidth{1.003750pt}%
\definecolor{currentstroke}{rgb}{0.176471,0.192157,0.258824}%
\pgfsetstrokecolor{currentstroke}%
\pgfsetdash{}{0pt}%
\pgfsys@defobject{currentmarker}{\pgfqpoint{-0.033333in}{-0.033333in}}{\pgfqpoint{0.033333in}{0.033333in}}{%
\pgfpathmoveto{\pgfqpoint{-0.000000in}{-0.033333in}}%
\pgfpathlineto{\pgfqpoint{0.033333in}{0.033333in}}%
\pgfpathlineto{\pgfqpoint{-0.033333in}{0.033333in}}%
\pgfpathlineto{\pgfqpoint{-0.000000in}{-0.033333in}}%
\pgfpathclose%
\pgfusepath{stroke,fill}%
}%
\begin{pgfscope}%
\pgfsys@transformshift{3.281584in}{3.478315in}%
\pgfsys@useobject{currentmarker}{}%
\end{pgfscope}%
\end{pgfscope}%
\begin{pgfscope}%
\pgfpathrectangle{\pgfqpoint{0.765000in}{0.660000in}}{\pgfqpoint{4.620000in}{4.620000in}}%
\pgfusepath{clip}%
\pgfsetroundcap%
\pgfsetroundjoin%
\pgfsetlinewidth{1.204500pt}%
\definecolor{currentstroke}{rgb}{0.176471,0.192157,0.258824}%
\pgfsetstrokecolor{currentstroke}%
\pgfsetdash{}{0pt}%
\pgfpathmoveto{\pgfqpoint{4.434796in}{3.802594in}}%
\pgfusepath{stroke}%
\end{pgfscope}%
\begin{pgfscope}%
\pgfpathrectangle{\pgfqpoint{0.765000in}{0.660000in}}{\pgfqpoint{4.620000in}{4.620000in}}%
\pgfusepath{clip}%
\pgfsetbuttcap%
\pgfsetmiterjoin%
\definecolor{currentfill}{rgb}{0.176471,0.192157,0.258824}%
\pgfsetfillcolor{currentfill}%
\pgfsetlinewidth{1.003750pt}%
\definecolor{currentstroke}{rgb}{0.176471,0.192157,0.258824}%
\pgfsetstrokecolor{currentstroke}%
\pgfsetdash{}{0pt}%
\pgfsys@defobject{currentmarker}{\pgfqpoint{-0.033333in}{-0.033333in}}{\pgfqpoint{0.033333in}{0.033333in}}{%
\pgfpathmoveto{\pgfqpoint{-0.000000in}{-0.033333in}}%
\pgfpathlineto{\pgfqpoint{0.033333in}{0.033333in}}%
\pgfpathlineto{\pgfqpoint{-0.033333in}{0.033333in}}%
\pgfpathlineto{\pgfqpoint{-0.000000in}{-0.033333in}}%
\pgfpathclose%
\pgfusepath{stroke,fill}%
}%
\begin{pgfscope}%
\pgfsys@transformshift{4.434796in}{3.802594in}%
\pgfsys@useobject{currentmarker}{}%
\end{pgfscope}%
\end{pgfscope}%
\begin{pgfscope}%
\pgfpathrectangle{\pgfqpoint{0.765000in}{0.660000in}}{\pgfqpoint{4.620000in}{4.620000in}}%
\pgfusepath{clip}%
\pgfsetroundcap%
\pgfsetroundjoin%
\pgfsetlinewidth{1.204500pt}%
\definecolor{currentstroke}{rgb}{0.176471,0.192157,0.258824}%
\pgfsetstrokecolor{currentstroke}%
\pgfsetdash{}{0pt}%
\pgfpathmoveto{\pgfqpoint{3.108602in}{3.883663in}}%
\pgfusepath{stroke}%
\end{pgfscope}%
\begin{pgfscope}%
\pgfpathrectangle{\pgfqpoint{0.765000in}{0.660000in}}{\pgfqpoint{4.620000in}{4.620000in}}%
\pgfusepath{clip}%
\pgfsetbuttcap%
\pgfsetmiterjoin%
\definecolor{currentfill}{rgb}{0.176471,0.192157,0.258824}%
\pgfsetfillcolor{currentfill}%
\pgfsetlinewidth{1.003750pt}%
\definecolor{currentstroke}{rgb}{0.176471,0.192157,0.258824}%
\pgfsetstrokecolor{currentstroke}%
\pgfsetdash{}{0pt}%
\pgfsys@defobject{currentmarker}{\pgfqpoint{-0.033333in}{-0.033333in}}{\pgfqpoint{0.033333in}{0.033333in}}{%
\pgfpathmoveto{\pgfqpoint{-0.000000in}{-0.033333in}}%
\pgfpathlineto{\pgfqpoint{0.033333in}{0.033333in}}%
\pgfpathlineto{\pgfqpoint{-0.033333in}{0.033333in}}%
\pgfpathlineto{\pgfqpoint{-0.000000in}{-0.033333in}}%
\pgfpathclose%
\pgfusepath{stroke,fill}%
}%
\begin{pgfscope}%
\pgfsys@transformshift{3.108602in}{3.883663in}%
\pgfsys@useobject{currentmarker}{}%
\end{pgfscope}%
\end{pgfscope}%
\begin{pgfscope}%
\pgfpathrectangle{\pgfqpoint{0.765000in}{0.660000in}}{\pgfqpoint{4.620000in}{4.620000in}}%
\pgfusepath{clip}%
\pgfsetroundcap%
\pgfsetroundjoin%
\pgfsetlinewidth{1.204500pt}%
\definecolor{currentstroke}{rgb}{0.176471,0.192157,0.258824}%
\pgfsetstrokecolor{currentstroke}%
\pgfsetdash{}{0pt}%
\pgfpathmoveto{\pgfqpoint{2.186033in}{3.235106in}}%
\pgfusepath{stroke}%
\end{pgfscope}%
\begin{pgfscope}%
\pgfpathrectangle{\pgfqpoint{0.765000in}{0.660000in}}{\pgfqpoint{4.620000in}{4.620000in}}%
\pgfusepath{clip}%
\pgfsetbuttcap%
\pgfsetmiterjoin%
\definecolor{currentfill}{rgb}{0.176471,0.192157,0.258824}%
\pgfsetfillcolor{currentfill}%
\pgfsetlinewidth{1.003750pt}%
\definecolor{currentstroke}{rgb}{0.176471,0.192157,0.258824}%
\pgfsetstrokecolor{currentstroke}%
\pgfsetdash{}{0pt}%
\pgfsys@defobject{currentmarker}{\pgfqpoint{-0.033333in}{-0.033333in}}{\pgfqpoint{0.033333in}{0.033333in}}{%
\pgfpathmoveto{\pgfqpoint{-0.000000in}{-0.033333in}}%
\pgfpathlineto{\pgfqpoint{0.033333in}{0.033333in}}%
\pgfpathlineto{\pgfqpoint{-0.033333in}{0.033333in}}%
\pgfpathlineto{\pgfqpoint{-0.000000in}{-0.033333in}}%
\pgfpathclose%
\pgfusepath{stroke,fill}%
}%
\begin{pgfscope}%
\pgfsys@transformshift{2.186033in}{3.235106in}%
\pgfsys@useobject{currentmarker}{}%
\end{pgfscope}%
\end{pgfscope}%
\begin{pgfscope}%
\pgfpathrectangle{\pgfqpoint{0.765000in}{0.660000in}}{\pgfqpoint{4.620000in}{4.620000in}}%
\pgfusepath{clip}%
\pgfsetroundcap%
\pgfsetroundjoin%
\pgfsetlinewidth{1.204500pt}%
\definecolor{currentstroke}{rgb}{0.176471,0.192157,0.258824}%
\pgfsetstrokecolor{currentstroke}%
\pgfsetdash{}{0pt}%
\pgfpathmoveto{\pgfqpoint{3.800529in}{3.883663in}}%
\pgfusepath{stroke}%
\end{pgfscope}%
\begin{pgfscope}%
\pgfpathrectangle{\pgfqpoint{0.765000in}{0.660000in}}{\pgfqpoint{4.620000in}{4.620000in}}%
\pgfusepath{clip}%
\pgfsetbuttcap%
\pgfsetmiterjoin%
\definecolor{currentfill}{rgb}{0.176471,0.192157,0.258824}%
\pgfsetfillcolor{currentfill}%
\pgfsetlinewidth{1.003750pt}%
\definecolor{currentstroke}{rgb}{0.176471,0.192157,0.258824}%
\pgfsetstrokecolor{currentstroke}%
\pgfsetdash{}{0pt}%
\pgfsys@defobject{currentmarker}{\pgfqpoint{-0.033333in}{-0.033333in}}{\pgfqpoint{0.033333in}{0.033333in}}{%
\pgfpathmoveto{\pgfqpoint{-0.000000in}{-0.033333in}}%
\pgfpathlineto{\pgfqpoint{0.033333in}{0.033333in}}%
\pgfpathlineto{\pgfqpoint{-0.033333in}{0.033333in}}%
\pgfpathlineto{\pgfqpoint{-0.000000in}{-0.033333in}}%
\pgfpathclose%
\pgfusepath{stroke,fill}%
}%
\begin{pgfscope}%
\pgfsys@transformshift{3.800529in}{3.883663in}%
\pgfsys@useobject{currentmarker}{}%
\end{pgfscope}%
\end{pgfscope}%
\begin{pgfscope}%
\pgfpathrectangle{\pgfqpoint{0.765000in}{0.660000in}}{\pgfqpoint{4.620000in}{4.620000in}}%
\pgfusepath{clip}%
\pgfsetroundcap%
\pgfsetroundjoin%
\pgfsetlinewidth{1.204500pt}%
\definecolor{currentstroke}{rgb}{0.176471,0.192157,0.258824}%
\pgfsetstrokecolor{currentstroke}%
\pgfsetdash{}{0pt}%
\pgfpathmoveto{\pgfqpoint{3.166263in}{3.640454in}}%
\pgfusepath{stroke}%
\end{pgfscope}%
\begin{pgfscope}%
\pgfpathrectangle{\pgfqpoint{0.765000in}{0.660000in}}{\pgfqpoint{4.620000in}{4.620000in}}%
\pgfusepath{clip}%
\pgfsetbuttcap%
\pgfsetmiterjoin%
\definecolor{currentfill}{rgb}{0.176471,0.192157,0.258824}%
\pgfsetfillcolor{currentfill}%
\pgfsetlinewidth{1.003750pt}%
\definecolor{currentstroke}{rgb}{0.176471,0.192157,0.258824}%
\pgfsetstrokecolor{currentstroke}%
\pgfsetdash{}{0pt}%
\pgfsys@defobject{currentmarker}{\pgfqpoint{-0.033333in}{-0.033333in}}{\pgfqpoint{0.033333in}{0.033333in}}{%
\pgfpathmoveto{\pgfqpoint{-0.000000in}{-0.033333in}}%
\pgfpathlineto{\pgfqpoint{0.033333in}{0.033333in}}%
\pgfpathlineto{\pgfqpoint{-0.033333in}{0.033333in}}%
\pgfpathlineto{\pgfqpoint{-0.000000in}{-0.033333in}}%
\pgfpathclose%
\pgfusepath{stroke,fill}%
}%
\begin{pgfscope}%
\pgfsys@transformshift{3.166263in}{3.640454in}%
\pgfsys@useobject{currentmarker}{}%
\end{pgfscope}%
\end{pgfscope}%
\begin{pgfscope}%
\pgfpathrectangle{\pgfqpoint{0.765000in}{0.660000in}}{\pgfqpoint{4.620000in}{4.620000in}}%
\pgfusepath{clip}%
\pgfsetroundcap%
\pgfsetroundjoin%
\pgfsetlinewidth{1.204500pt}%
\definecolor{currentstroke}{rgb}{0.176471,0.192157,0.258824}%
\pgfsetstrokecolor{currentstroke}%
\pgfsetdash{}{0pt}%
\pgfpathmoveto{\pgfqpoint{3.166263in}{3.640454in}}%
\pgfusepath{stroke}%
\end{pgfscope}%
\begin{pgfscope}%
\pgfpathrectangle{\pgfqpoint{0.765000in}{0.660000in}}{\pgfqpoint{4.620000in}{4.620000in}}%
\pgfusepath{clip}%
\pgfsetbuttcap%
\pgfsetmiterjoin%
\definecolor{currentfill}{rgb}{0.176471,0.192157,0.258824}%
\pgfsetfillcolor{currentfill}%
\pgfsetlinewidth{1.003750pt}%
\definecolor{currentstroke}{rgb}{0.176471,0.192157,0.258824}%
\pgfsetstrokecolor{currentstroke}%
\pgfsetdash{}{0pt}%
\pgfsys@defobject{currentmarker}{\pgfqpoint{-0.033333in}{-0.033333in}}{\pgfqpoint{0.033333in}{0.033333in}}{%
\pgfpathmoveto{\pgfqpoint{-0.000000in}{-0.033333in}}%
\pgfpathlineto{\pgfqpoint{0.033333in}{0.033333in}}%
\pgfpathlineto{\pgfqpoint{-0.033333in}{0.033333in}}%
\pgfpathlineto{\pgfqpoint{-0.000000in}{-0.033333in}}%
\pgfpathclose%
\pgfusepath{stroke,fill}%
}%
\begin{pgfscope}%
\pgfsys@transformshift{3.166263in}{3.640454in}%
\pgfsys@useobject{currentmarker}{}%
\end{pgfscope}%
\end{pgfscope}%
\begin{pgfscope}%
\pgfpathrectangle{\pgfqpoint{0.765000in}{0.660000in}}{\pgfqpoint{4.620000in}{4.620000in}}%
\pgfusepath{clip}%
\pgfsetroundcap%
\pgfsetroundjoin%
\pgfsetlinewidth{1.204500pt}%
\definecolor{currentstroke}{rgb}{0.176471,0.192157,0.258824}%
\pgfsetstrokecolor{currentstroke}%
\pgfsetdash{}{0pt}%
\pgfpathmoveto{\pgfqpoint{2.935620in}{3.478315in}}%
\pgfusepath{stroke}%
\end{pgfscope}%
\begin{pgfscope}%
\pgfpathrectangle{\pgfqpoint{0.765000in}{0.660000in}}{\pgfqpoint{4.620000in}{4.620000in}}%
\pgfusepath{clip}%
\pgfsetbuttcap%
\pgfsetmiterjoin%
\definecolor{currentfill}{rgb}{0.176471,0.192157,0.258824}%
\pgfsetfillcolor{currentfill}%
\pgfsetlinewidth{1.003750pt}%
\definecolor{currentstroke}{rgb}{0.176471,0.192157,0.258824}%
\pgfsetstrokecolor{currentstroke}%
\pgfsetdash{}{0pt}%
\pgfsys@defobject{currentmarker}{\pgfqpoint{-0.033333in}{-0.033333in}}{\pgfqpoint{0.033333in}{0.033333in}}{%
\pgfpathmoveto{\pgfqpoint{-0.000000in}{-0.033333in}}%
\pgfpathlineto{\pgfqpoint{0.033333in}{0.033333in}}%
\pgfpathlineto{\pgfqpoint{-0.033333in}{0.033333in}}%
\pgfpathlineto{\pgfqpoint{-0.000000in}{-0.033333in}}%
\pgfpathclose%
\pgfusepath{stroke,fill}%
}%
\begin{pgfscope}%
\pgfsys@transformshift{2.935620in}{3.478315in}%
\pgfsys@useobject{currentmarker}{}%
\end{pgfscope}%
\end{pgfscope}%
\begin{pgfscope}%
\pgfpathrectangle{\pgfqpoint{0.765000in}{0.660000in}}{\pgfqpoint{4.620000in}{4.620000in}}%
\pgfusepath{clip}%
\pgfsetroundcap%
\pgfsetroundjoin%
\pgfsetlinewidth{1.204500pt}%
\definecolor{currentstroke}{rgb}{0.176471,0.192157,0.258824}%
\pgfsetstrokecolor{currentstroke}%
\pgfsetdash{}{0pt}%
\pgfpathmoveto{\pgfqpoint{3.800529in}{3.721524in}}%
\pgfusepath{stroke}%
\end{pgfscope}%
\begin{pgfscope}%
\pgfpathrectangle{\pgfqpoint{0.765000in}{0.660000in}}{\pgfqpoint{4.620000in}{4.620000in}}%
\pgfusepath{clip}%
\pgfsetbuttcap%
\pgfsetmiterjoin%
\definecolor{currentfill}{rgb}{0.176471,0.192157,0.258824}%
\pgfsetfillcolor{currentfill}%
\pgfsetlinewidth{1.003750pt}%
\definecolor{currentstroke}{rgb}{0.176471,0.192157,0.258824}%
\pgfsetstrokecolor{currentstroke}%
\pgfsetdash{}{0pt}%
\pgfsys@defobject{currentmarker}{\pgfqpoint{-0.033333in}{-0.033333in}}{\pgfqpoint{0.033333in}{0.033333in}}{%
\pgfpathmoveto{\pgfqpoint{-0.000000in}{-0.033333in}}%
\pgfpathlineto{\pgfqpoint{0.033333in}{0.033333in}}%
\pgfpathlineto{\pgfqpoint{-0.033333in}{0.033333in}}%
\pgfpathlineto{\pgfqpoint{-0.000000in}{-0.033333in}}%
\pgfpathclose%
\pgfusepath{stroke,fill}%
}%
\begin{pgfscope}%
\pgfsys@transformshift{3.800529in}{3.721524in}%
\pgfsys@useobject{currentmarker}{}%
\end{pgfscope}%
\end{pgfscope}%
\begin{pgfscope}%
\pgfpathrectangle{\pgfqpoint{0.765000in}{0.660000in}}{\pgfqpoint{4.620000in}{4.620000in}}%
\pgfusepath{clip}%
\pgfsetroundcap%
\pgfsetroundjoin%
\pgfsetlinewidth{1.204500pt}%
\definecolor{currentstroke}{rgb}{0.176471,0.192157,0.258824}%
\pgfsetstrokecolor{currentstroke}%
\pgfsetdash{}{0pt}%
\pgfpathmoveto{\pgfqpoint{3.512226in}{3.478315in}}%
\pgfusepath{stroke}%
\end{pgfscope}%
\begin{pgfscope}%
\pgfpathrectangle{\pgfqpoint{0.765000in}{0.660000in}}{\pgfqpoint{4.620000in}{4.620000in}}%
\pgfusepath{clip}%
\pgfsetbuttcap%
\pgfsetmiterjoin%
\definecolor{currentfill}{rgb}{0.176471,0.192157,0.258824}%
\pgfsetfillcolor{currentfill}%
\pgfsetlinewidth{1.003750pt}%
\definecolor{currentstroke}{rgb}{0.176471,0.192157,0.258824}%
\pgfsetstrokecolor{currentstroke}%
\pgfsetdash{}{0pt}%
\pgfsys@defobject{currentmarker}{\pgfqpoint{-0.033333in}{-0.033333in}}{\pgfqpoint{0.033333in}{0.033333in}}{%
\pgfpathmoveto{\pgfqpoint{-0.000000in}{-0.033333in}}%
\pgfpathlineto{\pgfqpoint{0.033333in}{0.033333in}}%
\pgfpathlineto{\pgfqpoint{-0.033333in}{0.033333in}}%
\pgfpathlineto{\pgfqpoint{-0.000000in}{-0.033333in}}%
\pgfpathclose%
\pgfusepath{stroke,fill}%
}%
\begin{pgfscope}%
\pgfsys@transformshift{3.512226in}{3.478315in}%
\pgfsys@useobject{currentmarker}{}%
\end{pgfscope}%
\end{pgfscope}%
\begin{pgfscope}%
\pgfpathrectangle{\pgfqpoint{0.765000in}{0.660000in}}{\pgfqpoint{4.620000in}{4.620000in}}%
\pgfusepath{clip}%
\pgfsetroundcap%
\pgfsetroundjoin%
\pgfsetlinewidth{1.204500pt}%
\definecolor{currentstroke}{rgb}{0.176471,0.192157,0.258824}%
\pgfsetstrokecolor{currentstroke}%
\pgfsetdash{}{0pt}%
\pgfpathmoveto{\pgfqpoint{3.050942in}{3.478315in}}%
\pgfusepath{stroke}%
\end{pgfscope}%
\begin{pgfscope}%
\pgfpathrectangle{\pgfqpoint{0.765000in}{0.660000in}}{\pgfqpoint{4.620000in}{4.620000in}}%
\pgfusepath{clip}%
\pgfsetbuttcap%
\pgfsetmiterjoin%
\definecolor{currentfill}{rgb}{0.176471,0.192157,0.258824}%
\pgfsetfillcolor{currentfill}%
\pgfsetlinewidth{1.003750pt}%
\definecolor{currentstroke}{rgb}{0.176471,0.192157,0.258824}%
\pgfsetstrokecolor{currentstroke}%
\pgfsetdash{}{0pt}%
\pgfsys@defobject{currentmarker}{\pgfqpoint{-0.033333in}{-0.033333in}}{\pgfqpoint{0.033333in}{0.033333in}}{%
\pgfpathmoveto{\pgfqpoint{-0.000000in}{-0.033333in}}%
\pgfpathlineto{\pgfqpoint{0.033333in}{0.033333in}}%
\pgfpathlineto{\pgfqpoint{-0.033333in}{0.033333in}}%
\pgfpathlineto{\pgfqpoint{-0.000000in}{-0.033333in}}%
\pgfpathclose%
\pgfusepath{stroke,fill}%
}%
\begin{pgfscope}%
\pgfsys@transformshift{3.050942in}{3.478315in}%
\pgfsys@useobject{currentmarker}{}%
\end{pgfscope}%
\end{pgfscope}%
\begin{pgfscope}%
\pgfpathrectangle{\pgfqpoint{0.765000in}{0.660000in}}{\pgfqpoint{4.620000in}{4.620000in}}%
\pgfusepath{clip}%
\pgfsetroundcap%
\pgfsetroundjoin%
\pgfsetlinewidth{1.204500pt}%
\definecolor{currentstroke}{rgb}{0.176471,0.192157,0.258824}%
\pgfsetstrokecolor{currentstroke}%
\pgfsetdash{}{0pt}%
\pgfpathmoveto{\pgfqpoint{3.281584in}{3.478315in}}%
\pgfusepath{stroke}%
\end{pgfscope}%
\begin{pgfscope}%
\pgfpathrectangle{\pgfqpoint{0.765000in}{0.660000in}}{\pgfqpoint{4.620000in}{4.620000in}}%
\pgfusepath{clip}%
\pgfsetbuttcap%
\pgfsetmiterjoin%
\definecolor{currentfill}{rgb}{0.176471,0.192157,0.258824}%
\pgfsetfillcolor{currentfill}%
\pgfsetlinewidth{1.003750pt}%
\definecolor{currentstroke}{rgb}{0.176471,0.192157,0.258824}%
\pgfsetstrokecolor{currentstroke}%
\pgfsetdash{}{0pt}%
\pgfsys@defobject{currentmarker}{\pgfqpoint{-0.033333in}{-0.033333in}}{\pgfqpoint{0.033333in}{0.033333in}}{%
\pgfpathmoveto{\pgfqpoint{-0.000000in}{-0.033333in}}%
\pgfpathlineto{\pgfqpoint{0.033333in}{0.033333in}}%
\pgfpathlineto{\pgfqpoint{-0.033333in}{0.033333in}}%
\pgfpathlineto{\pgfqpoint{-0.000000in}{-0.033333in}}%
\pgfpathclose%
\pgfusepath{stroke,fill}%
}%
\begin{pgfscope}%
\pgfsys@transformshift{3.281584in}{3.478315in}%
\pgfsys@useobject{currentmarker}{}%
\end{pgfscope}%
\end{pgfscope}%
\begin{pgfscope}%
\pgfpathrectangle{\pgfqpoint{0.765000in}{0.660000in}}{\pgfqpoint{4.620000in}{4.620000in}}%
\pgfusepath{clip}%
\pgfsetroundcap%
\pgfsetroundjoin%
\pgfsetlinewidth{1.204500pt}%
\definecolor{currentstroke}{rgb}{0.176471,0.192157,0.258824}%
\pgfsetstrokecolor{currentstroke}%
\pgfsetdash{}{0pt}%
\pgfpathmoveto{\pgfqpoint{3.223923in}{3.721524in}}%
\pgfusepath{stroke}%
\end{pgfscope}%
\begin{pgfscope}%
\pgfpathrectangle{\pgfqpoint{0.765000in}{0.660000in}}{\pgfqpoint{4.620000in}{4.620000in}}%
\pgfusepath{clip}%
\pgfsetbuttcap%
\pgfsetmiterjoin%
\definecolor{currentfill}{rgb}{0.176471,0.192157,0.258824}%
\pgfsetfillcolor{currentfill}%
\pgfsetlinewidth{1.003750pt}%
\definecolor{currentstroke}{rgb}{0.176471,0.192157,0.258824}%
\pgfsetstrokecolor{currentstroke}%
\pgfsetdash{}{0pt}%
\pgfsys@defobject{currentmarker}{\pgfqpoint{-0.033333in}{-0.033333in}}{\pgfqpoint{0.033333in}{0.033333in}}{%
\pgfpathmoveto{\pgfqpoint{-0.000000in}{-0.033333in}}%
\pgfpathlineto{\pgfqpoint{0.033333in}{0.033333in}}%
\pgfpathlineto{\pgfqpoint{-0.033333in}{0.033333in}}%
\pgfpathlineto{\pgfqpoint{-0.000000in}{-0.033333in}}%
\pgfpathclose%
\pgfusepath{stroke,fill}%
}%
\begin{pgfscope}%
\pgfsys@transformshift{3.223923in}{3.721524in}%
\pgfsys@useobject{currentmarker}{}%
\end{pgfscope}%
\end{pgfscope}%
\begin{pgfscope}%
\pgfpathrectangle{\pgfqpoint{0.765000in}{0.660000in}}{\pgfqpoint{4.620000in}{4.620000in}}%
\pgfusepath{clip}%
\pgfsetroundcap%
\pgfsetroundjoin%
\pgfsetlinewidth{1.204500pt}%
\definecolor{currentstroke}{rgb}{0.176471,0.192157,0.258824}%
\pgfsetstrokecolor{currentstroke}%
\pgfsetdash{}{0pt}%
\pgfpathmoveto{\pgfqpoint{3.166263in}{3.316176in}}%
\pgfusepath{stroke}%
\end{pgfscope}%
\begin{pgfscope}%
\pgfpathrectangle{\pgfqpoint{0.765000in}{0.660000in}}{\pgfqpoint{4.620000in}{4.620000in}}%
\pgfusepath{clip}%
\pgfsetbuttcap%
\pgfsetmiterjoin%
\definecolor{currentfill}{rgb}{0.176471,0.192157,0.258824}%
\pgfsetfillcolor{currentfill}%
\pgfsetlinewidth{1.003750pt}%
\definecolor{currentstroke}{rgb}{0.176471,0.192157,0.258824}%
\pgfsetstrokecolor{currentstroke}%
\pgfsetdash{}{0pt}%
\pgfsys@defobject{currentmarker}{\pgfqpoint{-0.033333in}{-0.033333in}}{\pgfqpoint{0.033333in}{0.033333in}}{%
\pgfpathmoveto{\pgfqpoint{-0.000000in}{-0.033333in}}%
\pgfpathlineto{\pgfqpoint{0.033333in}{0.033333in}}%
\pgfpathlineto{\pgfqpoint{-0.033333in}{0.033333in}}%
\pgfpathlineto{\pgfqpoint{-0.000000in}{-0.033333in}}%
\pgfpathclose%
\pgfusepath{stroke,fill}%
}%
\begin{pgfscope}%
\pgfsys@transformshift{3.166263in}{3.316176in}%
\pgfsys@useobject{currentmarker}{}%
\end{pgfscope}%
\end{pgfscope}%
\begin{pgfscope}%
\pgfpathrectangle{\pgfqpoint{0.765000in}{0.660000in}}{\pgfqpoint{4.620000in}{4.620000in}}%
\pgfusepath{clip}%
\pgfsetroundcap%
\pgfsetroundjoin%
\pgfsetlinewidth{1.204500pt}%
\definecolor{currentstroke}{rgb}{0.176471,0.192157,0.258824}%
\pgfsetstrokecolor{currentstroke}%
\pgfsetdash{}{0pt}%
\pgfpathmoveto{\pgfqpoint{3.108602in}{3.559385in}}%
\pgfusepath{stroke}%
\end{pgfscope}%
\begin{pgfscope}%
\pgfpathrectangle{\pgfqpoint{0.765000in}{0.660000in}}{\pgfqpoint{4.620000in}{4.620000in}}%
\pgfusepath{clip}%
\pgfsetbuttcap%
\pgfsetmiterjoin%
\definecolor{currentfill}{rgb}{0.176471,0.192157,0.258824}%
\pgfsetfillcolor{currentfill}%
\pgfsetlinewidth{1.003750pt}%
\definecolor{currentstroke}{rgb}{0.176471,0.192157,0.258824}%
\pgfsetstrokecolor{currentstroke}%
\pgfsetdash{}{0pt}%
\pgfsys@defobject{currentmarker}{\pgfqpoint{-0.033333in}{-0.033333in}}{\pgfqpoint{0.033333in}{0.033333in}}{%
\pgfpathmoveto{\pgfqpoint{-0.000000in}{-0.033333in}}%
\pgfpathlineto{\pgfqpoint{0.033333in}{0.033333in}}%
\pgfpathlineto{\pgfqpoint{-0.033333in}{0.033333in}}%
\pgfpathlineto{\pgfqpoint{-0.000000in}{-0.033333in}}%
\pgfpathclose%
\pgfusepath{stroke,fill}%
}%
\begin{pgfscope}%
\pgfsys@transformshift{3.108602in}{3.559385in}%
\pgfsys@useobject{currentmarker}{}%
\end{pgfscope}%
\end{pgfscope}%
\begin{pgfscope}%
\pgfpathrectangle{\pgfqpoint{0.765000in}{0.660000in}}{\pgfqpoint{4.620000in}{4.620000in}}%
\pgfusepath{clip}%
\pgfsetroundcap%
\pgfsetroundjoin%
\pgfsetlinewidth{1.204500pt}%
\definecolor{currentstroke}{rgb}{0.176471,0.192157,0.258824}%
\pgfsetstrokecolor{currentstroke}%
\pgfsetdash{}{0pt}%
\pgfpathmoveto{\pgfqpoint{4.319475in}{3.640454in}}%
\pgfusepath{stroke}%
\end{pgfscope}%
\begin{pgfscope}%
\pgfpathrectangle{\pgfqpoint{0.765000in}{0.660000in}}{\pgfqpoint{4.620000in}{4.620000in}}%
\pgfusepath{clip}%
\pgfsetbuttcap%
\pgfsetmiterjoin%
\definecolor{currentfill}{rgb}{0.176471,0.192157,0.258824}%
\pgfsetfillcolor{currentfill}%
\pgfsetlinewidth{1.003750pt}%
\definecolor{currentstroke}{rgb}{0.176471,0.192157,0.258824}%
\pgfsetstrokecolor{currentstroke}%
\pgfsetdash{}{0pt}%
\pgfsys@defobject{currentmarker}{\pgfqpoint{-0.033333in}{-0.033333in}}{\pgfqpoint{0.033333in}{0.033333in}}{%
\pgfpathmoveto{\pgfqpoint{-0.000000in}{-0.033333in}}%
\pgfpathlineto{\pgfqpoint{0.033333in}{0.033333in}}%
\pgfpathlineto{\pgfqpoint{-0.033333in}{0.033333in}}%
\pgfpathlineto{\pgfqpoint{-0.000000in}{-0.033333in}}%
\pgfpathclose%
\pgfusepath{stroke,fill}%
}%
\begin{pgfscope}%
\pgfsys@transformshift{4.319475in}{3.640454in}%
\pgfsys@useobject{currentmarker}{}%
\end{pgfscope}%
\end{pgfscope}%
\begin{pgfscope}%
\pgfpathrectangle{\pgfqpoint{0.765000in}{0.660000in}}{\pgfqpoint{4.620000in}{4.620000in}}%
\pgfusepath{clip}%
\pgfsetroundcap%
\pgfsetroundjoin%
\pgfsetlinewidth{1.204500pt}%
\definecolor{currentstroke}{rgb}{0.176471,0.192157,0.258824}%
\pgfsetstrokecolor{currentstroke}%
\pgfsetdash{}{0pt}%
\pgfpathmoveto{\pgfqpoint{3.281584in}{3.802594in}}%
\pgfusepath{stroke}%
\end{pgfscope}%
\begin{pgfscope}%
\pgfpathrectangle{\pgfqpoint{0.765000in}{0.660000in}}{\pgfqpoint{4.620000in}{4.620000in}}%
\pgfusepath{clip}%
\pgfsetbuttcap%
\pgfsetmiterjoin%
\definecolor{currentfill}{rgb}{0.176471,0.192157,0.258824}%
\pgfsetfillcolor{currentfill}%
\pgfsetlinewidth{1.003750pt}%
\definecolor{currentstroke}{rgb}{0.176471,0.192157,0.258824}%
\pgfsetstrokecolor{currentstroke}%
\pgfsetdash{}{0pt}%
\pgfsys@defobject{currentmarker}{\pgfqpoint{-0.033333in}{-0.033333in}}{\pgfqpoint{0.033333in}{0.033333in}}{%
\pgfpathmoveto{\pgfqpoint{-0.000000in}{-0.033333in}}%
\pgfpathlineto{\pgfqpoint{0.033333in}{0.033333in}}%
\pgfpathlineto{\pgfqpoint{-0.033333in}{0.033333in}}%
\pgfpathlineto{\pgfqpoint{-0.000000in}{-0.033333in}}%
\pgfpathclose%
\pgfusepath{stroke,fill}%
}%
\begin{pgfscope}%
\pgfsys@transformshift{3.281584in}{3.802594in}%
\pgfsys@useobject{currentmarker}{}%
\end{pgfscope}%
\end{pgfscope}%
\begin{pgfscope}%
\pgfpathrectangle{\pgfqpoint{0.765000in}{0.660000in}}{\pgfqpoint{4.620000in}{4.620000in}}%
\pgfusepath{clip}%
\pgfsetroundcap%
\pgfsetroundjoin%
\pgfsetlinewidth{1.204500pt}%
\definecolor{currentstroke}{rgb}{0.176471,0.192157,0.258824}%
\pgfsetstrokecolor{currentstroke}%
\pgfsetdash{}{0pt}%
\pgfpathmoveto{\pgfqpoint{2.877960in}{3.235106in}}%
\pgfusepath{stroke}%
\end{pgfscope}%
\begin{pgfscope}%
\pgfpathrectangle{\pgfqpoint{0.765000in}{0.660000in}}{\pgfqpoint{4.620000in}{4.620000in}}%
\pgfusepath{clip}%
\pgfsetbuttcap%
\pgfsetmiterjoin%
\definecolor{currentfill}{rgb}{0.176471,0.192157,0.258824}%
\pgfsetfillcolor{currentfill}%
\pgfsetlinewidth{1.003750pt}%
\definecolor{currentstroke}{rgb}{0.176471,0.192157,0.258824}%
\pgfsetstrokecolor{currentstroke}%
\pgfsetdash{}{0pt}%
\pgfsys@defobject{currentmarker}{\pgfqpoint{-0.033333in}{-0.033333in}}{\pgfqpoint{0.033333in}{0.033333in}}{%
\pgfpathmoveto{\pgfqpoint{-0.000000in}{-0.033333in}}%
\pgfpathlineto{\pgfqpoint{0.033333in}{0.033333in}}%
\pgfpathlineto{\pgfqpoint{-0.033333in}{0.033333in}}%
\pgfpathlineto{\pgfqpoint{-0.000000in}{-0.033333in}}%
\pgfpathclose%
\pgfusepath{stroke,fill}%
}%
\begin{pgfscope}%
\pgfsys@transformshift{2.877960in}{3.235106in}%
\pgfsys@useobject{currentmarker}{}%
\end{pgfscope}%
\end{pgfscope}%
\begin{pgfscope}%
\pgfpathrectangle{\pgfqpoint{0.765000in}{0.660000in}}{\pgfqpoint{4.620000in}{4.620000in}}%
\pgfusepath{clip}%
\pgfsetroundcap%
\pgfsetroundjoin%
\pgfsetlinewidth{1.204500pt}%
\definecolor{currentstroke}{rgb}{0.176471,0.192157,0.258824}%
\pgfsetstrokecolor{currentstroke}%
\pgfsetdash{}{0pt}%
\pgfpathmoveto{\pgfqpoint{2.589657in}{3.154037in}}%
\pgfusepath{stroke}%
\end{pgfscope}%
\begin{pgfscope}%
\pgfpathrectangle{\pgfqpoint{0.765000in}{0.660000in}}{\pgfqpoint{4.620000in}{4.620000in}}%
\pgfusepath{clip}%
\pgfsetbuttcap%
\pgfsetmiterjoin%
\definecolor{currentfill}{rgb}{0.176471,0.192157,0.258824}%
\pgfsetfillcolor{currentfill}%
\pgfsetlinewidth{1.003750pt}%
\definecolor{currentstroke}{rgb}{0.176471,0.192157,0.258824}%
\pgfsetstrokecolor{currentstroke}%
\pgfsetdash{}{0pt}%
\pgfsys@defobject{currentmarker}{\pgfqpoint{-0.033333in}{-0.033333in}}{\pgfqpoint{0.033333in}{0.033333in}}{%
\pgfpathmoveto{\pgfqpoint{-0.000000in}{-0.033333in}}%
\pgfpathlineto{\pgfqpoint{0.033333in}{0.033333in}}%
\pgfpathlineto{\pgfqpoint{-0.033333in}{0.033333in}}%
\pgfpathlineto{\pgfqpoint{-0.000000in}{-0.033333in}}%
\pgfpathclose%
\pgfusepath{stroke,fill}%
}%
\begin{pgfscope}%
\pgfsys@transformshift{2.589657in}{3.154037in}%
\pgfsys@useobject{currentmarker}{}%
\end{pgfscope}%
\end{pgfscope}%
\begin{pgfscope}%
\pgfpathrectangle{\pgfqpoint{0.765000in}{0.660000in}}{\pgfqpoint{4.620000in}{4.620000in}}%
\pgfusepath{clip}%
\pgfsetroundcap%
\pgfsetroundjoin%
\pgfsetlinewidth{1.204500pt}%
\definecolor{currentstroke}{rgb}{0.176471,0.192157,0.258824}%
\pgfsetstrokecolor{currentstroke}%
\pgfsetdash{}{0pt}%
\pgfpathmoveto{\pgfqpoint{3.569887in}{3.883663in}}%
\pgfusepath{stroke}%
\end{pgfscope}%
\begin{pgfscope}%
\pgfpathrectangle{\pgfqpoint{0.765000in}{0.660000in}}{\pgfqpoint{4.620000in}{4.620000in}}%
\pgfusepath{clip}%
\pgfsetbuttcap%
\pgfsetmiterjoin%
\definecolor{currentfill}{rgb}{0.176471,0.192157,0.258824}%
\pgfsetfillcolor{currentfill}%
\pgfsetlinewidth{1.003750pt}%
\definecolor{currentstroke}{rgb}{0.176471,0.192157,0.258824}%
\pgfsetstrokecolor{currentstroke}%
\pgfsetdash{}{0pt}%
\pgfsys@defobject{currentmarker}{\pgfqpoint{-0.033333in}{-0.033333in}}{\pgfqpoint{0.033333in}{0.033333in}}{%
\pgfpathmoveto{\pgfqpoint{-0.000000in}{-0.033333in}}%
\pgfpathlineto{\pgfqpoint{0.033333in}{0.033333in}}%
\pgfpathlineto{\pgfqpoint{-0.033333in}{0.033333in}}%
\pgfpathlineto{\pgfqpoint{-0.000000in}{-0.033333in}}%
\pgfpathclose%
\pgfusepath{stroke,fill}%
}%
\begin{pgfscope}%
\pgfsys@transformshift{3.569887in}{3.883663in}%
\pgfsys@useobject{currentmarker}{}%
\end{pgfscope}%
\end{pgfscope}%
\begin{pgfscope}%
\pgfpathrectangle{\pgfqpoint{0.765000in}{0.660000in}}{\pgfqpoint{4.620000in}{4.620000in}}%
\pgfusepath{clip}%
\pgfsetroundcap%
\pgfsetroundjoin%
\pgfsetlinewidth{1.204500pt}%
\definecolor{currentstroke}{rgb}{0.176471,0.192157,0.258824}%
\pgfsetstrokecolor{currentstroke}%
\pgfsetdash{}{0pt}%
\pgfpathmoveto{\pgfqpoint{4.088832in}{3.964733in}}%
\pgfusepath{stroke}%
\end{pgfscope}%
\begin{pgfscope}%
\pgfpathrectangle{\pgfqpoint{0.765000in}{0.660000in}}{\pgfqpoint{4.620000in}{4.620000in}}%
\pgfusepath{clip}%
\pgfsetbuttcap%
\pgfsetmiterjoin%
\definecolor{currentfill}{rgb}{0.176471,0.192157,0.258824}%
\pgfsetfillcolor{currentfill}%
\pgfsetlinewidth{1.003750pt}%
\definecolor{currentstroke}{rgb}{0.176471,0.192157,0.258824}%
\pgfsetstrokecolor{currentstroke}%
\pgfsetdash{}{0pt}%
\pgfsys@defobject{currentmarker}{\pgfqpoint{-0.033333in}{-0.033333in}}{\pgfqpoint{0.033333in}{0.033333in}}{%
\pgfpathmoveto{\pgfqpoint{-0.000000in}{-0.033333in}}%
\pgfpathlineto{\pgfqpoint{0.033333in}{0.033333in}}%
\pgfpathlineto{\pgfqpoint{-0.033333in}{0.033333in}}%
\pgfpathlineto{\pgfqpoint{-0.000000in}{-0.033333in}}%
\pgfpathclose%
\pgfusepath{stroke,fill}%
}%
\begin{pgfscope}%
\pgfsys@transformshift{4.088832in}{3.964733in}%
\pgfsys@useobject{currentmarker}{}%
\end{pgfscope}%
\end{pgfscope}%
\begin{pgfscope}%
\pgfpathrectangle{\pgfqpoint{0.765000in}{0.660000in}}{\pgfqpoint{4.620000in}{4.620000in}}%
\pgfusepath{clip}%
\pgfsetroundcap%
\pgfsetroundjoin%
\pgfsetlinewidth{1.204500pt}%
\definecolor{currentstroke}{rgb}{0.176471,0.192157,0.258824}%
\pgfsetstrokecolor{currentstroke}%
\pgfsetdash{}{0pt}%
\pgfpathmoveto{\pgfqpoint{3.396905in}{3.478315in}}%
\pgfusepath{stroke}%
\end{pgfscope}%
\begin{pgfscope}%
\pgfpathrectangle{\pgfqpoint{0.765000in}{0.660000in}}{\pgfqpoint{4.620000in}{4.620000in}}%
\pgfusepath{clip}%
\pgfsetbuttcap%
\pgfsetmiterjoin%
\definecolor{currentfill}{rgb}{0.176471,0.192157,0.258824}%
\pgfsetfillcolor{currentfill}%
\pgfsetlinewidth{1.003750pt}%
\definecolor{currentstroke}{rgb}{0.176471,0.192157,0.258824}%
\pgfsetstrokecolor{currentstroke}%
\pgfsetdash{}{0pt}%
\pgfsys@defobject{currentmarker}{\pgfqpoint{-0.033333in}{-0.033333in}}{\pgfqpoint{0.033333in}{0.033333in}}{%
\pgfpathmoveto{\pgfqpoint{-0.000000in}{-0.033333in}}%
\pgfpathlineto{\pgfqpoint{0.033333in}{0.033333in}}%
\pgfpathlineto{\pgfqpoint{-0.033333in}{0.033333in}}%
\pgfpathlineto{\pgfqpoint{-0.000000in}{-0.033333in}}%
\pgfpathclose%
\pgfusepath{stroke,fill}%
}%
\begin{pgfscope}%
\pgfsys@transformshift{3.396905in}{3.478315in}%
\pgfsys@useobject{currentmarker}{}%
\end{pgfscope}%
\end{pgfscope}%
\begin{pgfscope}%
\pgfpathrectangle{\pgfqpoint{0.765000in}{0.660000in}}{\pgfqpoint{4.620000in}{4.620000in}}%
\pgfusepath{clip}%
\pgfsetroundcap%
\pgfsetroundjoin%
\pgfsetlinewidth{1.204500pt}%
\definecolor{currentstroke}{rgb}{0.176471,0.192157,0.258824}%
\pgfsetstrokecolor{currentstroke}%
\pgfsetdash{}{0pt}%
\pgfpathmoveto{\pgfqpoint{3.281584in}{3.478315in}}%
\pgfusepath{stroke}%
\end{pgfscope}%
\begin{pgfscope}%
\pgfpathrectangle{\pgfqpoint{0.765000in}{0.660000in}}{\pgfqpoint{4.620000in}{4.620000in}}%
\pgfusepath{clip}%
\pgfsetbuttcap%
\pgfsetmiterjoin%
\definecolor{currentfill}{rgb}{0.176471,0.192157,0.258824}%
\pgfsetfillcolor{currentfill}%
\pgfsetlinewidth{1.003750pt}%
\definecolor{currentstroke}{rgb}{0.176471,0.192157,0.258824}%
\pgfsetstrokecolor{currentstroke}%
\pgfsetdash{}{0pt}%
\pgfsys@defobject{currentmarker}{\pgfqpoint{-0.033333in}{-0.033333in}}{\pgfqpoint{0.033333in}{0.033333in}}{%
\pgfpathmoveto{\pgfqpoint{-0.000000in}{-0.033333in}}%
\pgfpathlineto{\pgfqpoint{0.033333in}{0.033333in}}%
\pgfpathlineto{\pgfqpoint{-0.033333in}{0.033333in}}%
\pgfpathlineto{\pgfqpoint{-0.000000in}{-0.033333in}}%
\pgfpathclose%
\pgfusepath{stroke,fill}%
}%
\begin{pgfscope}%
\pgfsys@transformshift{3.281584in}{3.478315in}%
\pgfsys@useobject{currentmarker}{}%
\end{pgfscope}%
\end{pgfscope}%
\begin{pgfscope}%
\pgfpathrectangle{\pgfqpoint{0.765000in}{0.660000in}}{\pgfqpoint{4.620000in}{4.620000in}}%
\pgfusepath{clip}%
\pgfsetroundcap%
\pgfsetroundjoin%
\pgfsetlinewidth{1.204500pt}%
\definecolor{currentstroke}{rgb}{0.176471,0.192157,0.258824}%
\pgfsetstrokecolor{currentstroke}%
\pgfsetdash{}{0pt}%
\pgfpathmoveto{\pgfqpoint{3.569887in}{3.721524in}}%
\pgfusepath{stroke}%
\end{pgfscope}%
\begin{pgfscope}%
\pgfpathrectangle{\pgfqpoint{0.765000in}{0.660000in}}{\pgfqpoint{4.620000in}{4.620000in}}%
\pgfusepath{clip}%
\pgfsetbuttcap%
\pgfsetmiterjoin%
\definecolor{currentfill}{rgb}{0.176471,0.192157,0.258824}%
\pgfsetfillcolor{currentfill}%
\pgfsetlinewidth{1.003750pt}%
\definecolor{currentstroke}{rgb}{0.176471,0.192157,0.258824}%
\pgfsetstrokecolor{currentstroke}%
\pgfsetdash{}{0pt}%
\pgfsys@defobject{currentmarker}{\pgfqpoint{-0.033333in}{-0.033333in}}{\pgfqpoint{0.033333in}{0.033333in}}{%
\pgfpathmoveto{\pgfqpoint{-0.000000in}{-0.033333in}}%
\pgfpathlineto{\pgfqpoint{0.033333in}{0.033333in}}%
\pgfpathlineto{\pgfqpoint{-0.033333in}{0.033333in}}%
\pgfpathlineto{\pgfqpoint{-0.000000in}{-0.033333in}}%
\pgfpathclose%
\pgfusepath{stroke,fill}%
}%
\begin{pgfscope}%
\pgfsys@transformshift{3.569887in}{3.721524in}%
\pgfsys@useobject{currentmarker}{}%
\end{pgfscope}%
\end{pgfscope}%
\begin{pgfscope}%
\pgfpathrectangle{\pgfqpoint{0.765000in}{0.660000in}}{\pgfqpoint{4.620000in}{4.620000in}}%
\pgfusepath{clip}%
\pgfsetroundcap%
\pgfsetroundjoin%
\pgfsetlinewidth{1.204500pt}%
\definecolor{currentstroke}{rgb}{0.176471,0.192157,0.258824}%
\pgfsetstrokecolor{currentstroke}%
\pgfsetdash{}{0pt}%
\pgfpathmoveto{\pgfqpoint{3.742869in}{3.964733in}}%
\pgfusepath{stroke}%
\end{pgfscope}%
\begin{pgfscope}%
\pgfpathrectangle{\pgfqpoint{0.765000in}{0.660000in}}{\pgfqpoint{4.620000in}{4.620000in}}%
\pgfusepath{clip}%
\pgfsetbuttcap%
\pgfsetmiterjoin%
\definecolor{currentfill}{rgb}{0.176471,0.192157,0.258824}%
\pgfsetfillcolor{currentfill}%
\pgfsetlinewidth{1.003750pt}%
\definecolor{currentstroke}{rgb}{0.176471,0.192157,0.258824}%
\pgfsetstrokecolor{currentstroke}%
\pgfsetdash{}{0pt}%
\pgfsys@defobject{currentmarker}{\pgfqpoint{-0.033333in}{-0.033333in}}{\pgfqpoint{0.033333in}{0.033333in}}{%
\pgfpathmoveto{\pgfqpoint{-0.000000in}{-0.033333in}}%
\pgfpathlineto{\pgfqpoint{0.033333in}{0.033333in}}%
\pgfpathlineto{\pgfqpoint{-0.033333in}{0.033333in}}%
\pgfpathlineto{\pgfqpoint{-0.000000in}{-0.033333in}}%
\pgfpathclose%
\pgfusepath{stroke,fill}%
}%
\begin{pgfscope}%
\pgfsys@transformshift{3.742869in}{3.964733in}%
\pgfsys@useobject{currentmarker}{}%
\end{pgfscope}%
\end{pgfscope}%
\begin{pgfscope}%
\pgfpathrectangle{\pgfqpoint{0.765000in}{0.660000in}}{\pgfqpoint{4.620000in}{4.620000in}}%
\pgfusepath{clip}%
\pgfsetroundcap%
\pgfsetroundjoin%
\pgfsetlinewidth{1.204500pt}%
\definecolor{currentstroke}{rgb}{0.176471,0.192157,0.258824}%
\pgfsetstrokecolor{currentstroke}%
\pgfsetdash{}{0pt}%
\pgfpathmoveto{\pgfqpoint{3.685208in}{3.883663in}}%
\pgfusepath{stroke}%
\end{pgfscope}%
\begin{pgfscope}%
\pgfpathrectangle{\pgfqpoint{0.765000in}{0.660000in}}{\pgfqpoint{4.620000in}{4.620000in}}%
\pgfusepath{clip}%
\pgfsetbuttcap%
\pgfsetmiterjoin%
\definecolor{currentfill}{rgb}{0.176471,0.192157,0.258824}%
\pgfsetfillcolor{currentfill}%
\pgfsetlinewidth{1.003750pt}%
\definecolor{currentstroke}{rgb}{0.176471,0.192157,0.258824}%
\pgfsetstrokecolor{currentstroke}%
\pgfsetdash{}{0pt}%
\pgfsys@defobject{currentmarker}{\pgfqpoint{-0.033333in}{-0.033333in}}{\pgfqpoint{0.033333in}{0.033333in}}{%
\pgfpathmoveto{\pgfqpoint{-0.000000in}{-0.033333in}}%
\pgfpathlineto{\pgfqpoint{0.033333in}{0.033333in}}%
\pgfpathlineto{\pgfqpoint{-0.033333in}{0.033333in}}%
\pgfpathlineto{\pgfqpoint{-0.000000in}{-0.033333in}}%
\pgfpathclose%
\pgfusepath{stroke,fill}%
}%
\begin{pgfscope}%
\pgfsys@transformshift{3.685208in}{3.883663in}%
\pgfsys@useobject{currentmarker}{}%
\end{pgfscope}%
\end{pgfscope}%
\begin{pgfscope}%
\pgfpathrectangle{\pgfqpoint{0.765000in}{0.660000in}}{\pgfqpoint{4.620000in}{4.620000in}}%
\pgfusepath{clip}%
\pgfsetroundcap%
\pgfsetroundjoin%
\pgfsetlinewidth{1.204500pt}%
\definecolor{currentstroke}{rgb}{0.176471,0.192157,0.258824}%
\pgfsetstrokecolor{currentstroke}%
\pgfsetdash{}{0pt}%
\pgfpathmoveto{\pgfqpoint{2.993281in}{3.559385in}}%
\pgfusepath{stroke}%
\end{pgfscope}%
\begin{pgfscope}%
\pgfpathrectangle{\pgfqpoint{0.765000in}{0.660000in}}{\pgfqpoint{4.620000in}{4.620000in}}%
\pgfusepath{clip}%
\pgfsetbuttcap%
\pgfsetmiterjoin%
\definecolor{currentfill}{rgb}{0.176471,0.192157,0.258824}%
\pgfsetfillcolor{currentfill}%
\pgfsetlinewidth{1.003750pt}%
\definecolor{currentstroke}{rgb}{0.176471,0.192157,0.258824}%
\pgfsetstrokecolor{currentstroke}%
\pgfsetdash{}{0pt}%
\pgfsys@defobject{currentmarker}{\pgfqpoint{-0.033333in}{-0.033333in}}{\pgfqpoint{0.033333in}{0.033333in}}{%
\pgfpathmoveto{\pgfqpoint{-0.000000in}{-0.033333in}}%
\pgfpathlineto{\pgfqpoint{0.033333in}{0.033333in}}%
\pgfpathlineto{\pgfqpoint{-0.033333in}{0.033333in}}%
\pgfpathlineto{\pgfqpoint{-0.000000in}{-0.033333in}}%
\pgfpathclose%
\pgfusepath{stroke,fill}%
}%
\begin{pgfscope}%
\pgfsys@transformshift{2.993281in}{3.559385in}%
\pgfsys@useobject{currentmarker}{}%
\end{pgfscope}%
\end{pgfscope}%
\begin{pgfscope}%
\pgfpathrectangle{\pgfqpoint{0.765000in}{0.660000in}}{\pgfqpoint{4.620000in}{4.620000in}}%
\pgfusepath{clip}%
\pgfsetroundcap%
\pgfsetroundjoin%
\pgfsetlinewidth{1.204500pt}%
\definecolor{currentstroke}{rgb}{0.176471,0.192157,0.258824}%
\pgfsetstrokecolor{currentstroke}%
\pgfsetdash{}{0pt}%
\pgfpathmoveto{\pgfqpoint{3.800529in}{3.883663in}}%
\pgfusepath{stroke}%
\end{pgfscope}%
\begin{pgfscope}%
\pgfpathrectangle{\pgfqpoint{0.765000in}{0.660000in}}{\pgfqpoint{4.620000in}{4.620000in}}%
\pgfusepath{clip}%
\pgfsetbuttcap%
\pgfsetmiterjoin%
\definecolor{currentfill}{rgb}{0.176471,0.192157,0.258824}%
\pgfsetfillcolor{currentfill}%
\pgfsetlinewidth{1.003750pt}%
\definecolor{currentstroke}{rgb}{0.176471,0.192157,0.258824}%
\pgfsetstrokecolor{currentstroke}%
\pgfsetdash{}{0pt}%
\pgfsys@defobject{currentmarker}{\pgfqpoint{-0.033333in}{-0.033333in}}{\pgfqpoint{0.033333in}{0.033333in}}{%
\pgfpathmoveto{\pgfqpoint{-0.000000in}{-0.033333in}}%
\pgfpathlineto{\pgfqpoint{0.033333in}{0.033333in}}%
\pgfpathlineto{\pgfqpoint{-0.033333in}{0.033333in}}%
\pgfpathlineto{\pgfqpoint{-0.000000in}{-0.033333in}}%
\pgfpathclose%
\pgfusepath{stroke,fill}%
}%
\begin{pgfscope}%
\pgfsys@transformshift{3.800529in}{3.883663in}%
\pgfsys@useobject{currentmarker}{}%
\end{pgfscope}%
\end{pgfscope}%
\begin{pgfscope}%
\pgfpathrectangle{\pgfqpoint{0.765000in}{0.660000in}}{\pgfqpoint{4.620000in}{4.620000in}}%
\pgfusepath{clip}%
\pgfsetroundcap%
\pgfsetroundjoin%
\pgfsetlinewidth{1.204500pt}%
\definecolor{currentstroke}{rgb}{0.176471,0.192157,0.258824}%
\pgfsetstrokecolor{currentstroke}%
\pgfsetdash{}{0pt}%
\pgfpathmoveto{\pgfqpoint{4.031172in}{4.045802in}}%
\pgfusepath{stroke}%
\end{pgfscope}%
\begin{pgfscope}%
\pgfpathrectangle{\pgfqpoint{0.765000in}{0.660000in}}{\pgfqpoint{4.620000in}{4.620000in}}%
\pgfusepath{clip}%
\pgfsetbuttcap%
\pgfsetmiterjoin%
\definecolor{currentfill}{rgb}{0.176471,0.192157,0.258824}%
\pgfsetfillcolor{currentfill}%
\pgfsetlinewidth{1.003750pt}%
\definecolor{currentstroke}{rgb}{0.176471,0.192157,0.258824}%
\pgfsetstrokecolor{currentstroke}%
\pgfsetdash{}{0pt}%
\pgfsys@defobject{currentmarker}{\pgfqpoint{-0.033333in}{-0.033333in}}{\pgfqpoint{0.033333in}{0.033333in}}{%
\pgfpathmoveto{\pgfqpoint{-0.000000in}{-0.033333in}}%
\pgfpathlineto{\pgfqpoint{0.033333in}{0.033333in}}%
\pgfpathlineto{\pgfqpoint{-0.033333in}{0.033333in}}%
\pgfpathlineto{\pgfqpoint{-0.000000in}{-0.033333in}}%
\pgfpathclose%
\pgfusepath{stroke,fill}%
}%
\begin{pgfscope}%
\pgfsys@transformshift{4.031172in}{4.045802in}%
\pgfsys@useobject{currentmarker}{}%
\end{pgfscope}%
\end{pgfscope}%
\begin{pgfscope}%
\pgfpathrectangle{\pgfqpoint{0.765000in}{0.660000in}}{\pgfqpoint{4.620000in}{4.620000in}}%
\pgfusepath{clip}%
\pgfsetroundcap%
\pgfsetroundjoin%
\pgfsetlinewidth{1.204500pt}%
\definecolor{currentstroke}{rgb}{0.176471,0.192157,0.258824}%
\pgfsetstrokecolor{currentstroke}%
\pgfsetdash{}{0pt}%
\pgfpathmoveto{\pgfqpoint{3.569887in}{3.883663in}}%
\pgfusepath{stroke}%
\end{pgfscope}%
\begin{pgfscope}%
\pgfpathrectangle{\pgfqpoint{0.765000in}{0.660000in}}{\pgfqpoint{4.620000in}{4.620000in}}%
\pgfusepath{clip}%
\pgfsetbuttcap%
\pgfsetmiterjoin%
\definecolor{currentfill}{rgb}{0.176471,0.192157,0.258824}%
\pgfsetfillcolor{currentfill}%
\pgfsetlinewidth{1.003750pt}%
\definecolor{currentstroke}{rgb}{0.176471,0.192157,0.258824}%
\pgfsetstrokecolor{currentstroke}%
\pgfsetdash{}{0pt}%
\pgfsys@defobject{currentmarker}{\pgfqpoint{-0.033333in}{-0.033333in}}{\pgfqpoint{0.033333in}{0.033333in}}{%
\pgfpathmoveto{\pgfqpoint{-0.000000in}{-0.033333in}}%
\pgfpathlineto{\pgfqpoint{0.033333in}{0.033333in}}%
\pgfpathlineto{\pgfqpoint{-0.033333in}{0.033333in}}%
\pgfpathlineto{\pgfqpoint{-0.000000in}{-0.033333in}}%
\pgfpathclose%
\pgfusepath{stroke,fill}%
}%
\begin{pgfscope}%
\pgfsys@transformshift{3.569887in}{3.883663in}%
\pgfsys@useobject{currentmarker}{}%
\end{pgfscope}%
\end{pgfscope}%
\begin{pgfscope}%
\pgfpathrectangle{\pgfqpoint{0.765000in}{0.660000in}}{\pgfqpoint{4.620000in}{4.620000in}}%
\pgfusepath{clip}%
\pgfsetroundcap%
\pgfsetroundjoin%
\pgfsetlinewidth{1.204500pt}%
\definecolor{currentstroke}{rgb}{0.176471,0.192157,0.258824}%
\pgfsetstrokecolor{currentstroke}%
\pgfsetdash{}{0pt}%
\pgfpathmoveto{\pgfqpoint{2.762639in}{3.559385in}}%
\pgfusepath{stroke}%
\end{pgfscope}%
\begin{pgfscope}%
\pgfpathrectangle{\pgfqpoint{0.765000in}{0.660000in}}{\pgfqpoint{4.620000in}{4.620000in}}%
\pgfusepath{clip}%
\pgfsetbuttcap%
\pgfsetmiterjoin%
\definecolor{currentfill}{rgb}{0.176471,0.192157,0.258824}%
\pgfsetfillcolor{currentfill}%
\pgfsetlinewidth{1.003750pt}%
\definecolor{currentstroke}{rgb}{0.176471,0.192157,0.258824}%
\pgfsetstrokecolor{currentstroke}%
\pgfsetdash{}{0pt}%
\pgfsys@defobject{currentmarker}{\pgfqpoint{-0.033333in}{-0.033333in}}{\pgfqpoint{0.033333in}{0.033333in}}{%
\pgfpathmoveto{\pgfqpoint{-0.000000in}{-0.033333in}}%
\pgfpathlineto{\pgfqpoint{0.033333in}{0.033333in}}%
\pgfpathlineto{\pgfqpoint{-0.033333in}{0.033333in}}%
\pgfpathlineto{\pgfqpoint{-0.000000in}{-0.033333in}}%
\pgfpathclose%
\pgfusepath{stroke,fill}%
}%
\begin{pgfscope}%
\pgfsys@transformshift{2.762639in}{3.559385in}%
\pgfsys@useobject{currentmarker}{}%
\end{pgfscope}%
\end{pgfscope}%
\begin{pgfscope}%
\pgfpathrectangle{\pgfqpoint{0.765000in}{0.660000in}}{\pgfqpoint{4.620000in}{4.620000in}}%
\pgfusepath{clip}%
\pgfsetroundcap%
\pgfsetroundjoin%
\pgfsetlinewidth{1.204500pt}%
\definecolor{currentstroke}{rgb}{0.176471,0.192157,0.258824}%
\pgfsetstrokecolor{currentstroke}%
\pgfsetdash{}{0pt}%
\pgfpathmoveto{\pgfqpoint{3.396905in}{3.640454in}}%
\pgfusepath{stroke}%
\end{pgfscope}%
\begin{pgfscope}%
\pgfpathrectangle{\pgfqpoint{0.765000in}{0.660000in}}{\pgfqpoint{4.620000in}{4.620000in}}%
\pgfusepath{clip}%
\pgfsetbuttcap%
\pgfsetmiterjoin%
\definecolor{currentfill}{rgb}{0.176471,0.192157,0.258824}%
\pgfsetfillcolor{currentfill}%
\pgfsetlinewidth{1.003750pt}%
\definecolor{currentstroke}{rgb}{0.176471,0.192157,0.258824}%
\pgfsetstrokecolor{currentstroke}%
\pgfsetdash{}{0pt}%
\pgfsys@defobject{currentmarker}{\pgfqpoint{-0.033333in}{-0.033333in}}{\pgfqpoint{0.033333in}{0.033333in}}{%
\pgfpathmoveto{\pgfqpoint{-0.000000in}{-0.033333in}}%
\pgfpathlineto{\pgfqpoint{0.033333in}{0.033333in}}%
\pgfpathlineto{\pgfqpoint{-0.033333in}{0.033333in}}%
\pgfpathlineto{\pgfqpoint{-0.000000in}{-0.033333in}}%
\pgfpathclose%
\pgfusepath{stroke,fill}%
}%
\begin{pgfscope}%
\pgfsys@transformshift{3.396905in}{3.640454in}%
\pgfsys@useobject{currentmarker}{}%
\end{pgfscope}%
\end{pgfscope}%
\begin{pgfscope}%
\pgfpathrectangle{\pgfqpoint{0.765000in}{0.660000in}}{\pgfqpoint{4.620000in}{4.620000in}}%
\pgfusepath{clip}%
\pgfsetroundcap%
\pgfsetroundjoin%
\pgfsetlinewidth{1.204500pt}%
\definecolor{currentstroke}{rgb}{0.176471,0.192157,0.258824}%
\pgfsetstrokecolor{currentstroke}%
\pgfsetdash{}{0pt}%
\pgfpathmoveto{\pgfqpoint{4.031172in}{3.883663in}}%
\pgfusepath{stroke}%
\end{pgfscope}%
\begin{pgfscope}%
\pgfpathrectangle{\pgfqpoint{0.765000in}{0.660000in}}{\pgfqpoint{4.620000in}{4.620000in}}%
\pgfusepath{clip}%
\pgfsetbuttcap%
\pgfsetmiterjoin%
\definecolor{currentfill}{rgb}{0.176471,0.192157,0.258824}%
\pgfsetfillcolor{currentfill}%
\pgfsetlinewidth{1.003750pt}%
\definecolor{currentstroke}{rgb}{0.176471,0.192157,0.258824}%
\pgfsetstrokecolor{currentstroke}%
\pgfsetdash{}{0pt}%
\pgfsys@defobject{currentmarker}{\pgfqpoint{-0.033333in}{-0.033333in}}{\pgfqpoint{0.033333in}{0.033333in}}{%
\pgfpathmoveto{\pgfqpoint{-0.000000in}{-0.033333in}}%
\pgfpathlineto{\pgfqpoint{0.033333in}{0.033333in}}%
\pgfpathlineto{\pgfqpoint{-0.033333in}{0.033333in}}%
\pgfpathlineto{\pgfqpoint{-0.000000in}{-0.033333in}}%
\pgfpathclose%
\pgfusepath{stroke,fill}%
}%
\begin{pgfscope}%
\pgfsys@transformshift{4.031172in}{3.883663in}%
\pgfsys@useobject{currentmarker}{}%
\end{pgfscope}%
\end{pgfscope}%
\begin{pgfscope}%
\pgfpathrectangle{\pgfqpoint{0.765000in}{0.660000in}}{\pgfqpoint{4.620000in}{4.620000in}}%
\pgfusepath{clip}%
\pgfsetroundcap%
\pgfsetroundjoin%
\pgfsetlinewidth{1.204500pt}%
\definecolor{currentstroke}{rgb}{0.176471,0.192157,0.258824}%
\pgfsetstrokecolor{currentstroke}%
\pgfsetdash{}{0pt}%
\pgfpathmoveto{\pgfqpoint{3.281584in}{3.478315in}}%
\pgfusepath{stroke}%
\end{pgfscope}%
\begin{pgfscope}%
\pgfpathrectangle{\pgfqpoint{0.765000in}{0.660000in}}{\pgfqpoint{4.620000in}{4.620000in}}%
\pgfusepath{clip}%
\pgfsetbuttcap%
\pgfsetmiterjoin%
\definecolor{currentfill}{rgb}{0.176471,0.192157,0.258824}%
\pgfsetfillcolor{currentfill}%
\pgfsetlinewidth{1.003750pt}%
\definecolor{currentstroke}{rgb}{0.176471,0.192157,0.258824}%
\pgfsetstrokecolor{currentstroke}%
\pgfsetdash{}{0pt}%
\pgfsys@defobject{currentmarker}{\pgfqpoint{-0.033333in}{-0.033333in}}{\pgfqpoint{0.033333in}{0.033333in}}{%
\pgfpathmoveto{\pgfqpoint{-0.000000in}{-0.033333in}}%
\pgfpathlineto{\pgfqpoint{0.033333in}{0.033333in}}%
\pgfpathlineto{\pgfqpoint{-0.033333in}{0.033333in}}%
\pgfpathlineto{\pgfqpoint{-0.000000in}{-0.033333in}}%
\pgfpathclose%
\pgfusepath{stroke,fill}%
}%
\begin{pgfscope}%
\pgfsys@transformshift{3.281584in}{3.478315in}%
\pgfsys@useobject{currentmarker}{}%
\end{pgfscope}%
\end{pgfscope}%
\begin{pgfscope}%
\pgfpathrectangle{\pgfqpoint{0.765000in}{0.660000in}}{\pgfqpoint{4.620000in}{4.620000in}}%
\pgfusepath{clip}%
\pgfsetroundcap%
\pgfsetroundjoin%
\pgfsetlinewidth{1.204500pt}%
\definecolor{currentstroke}{rgb}{0.019608,0.556863,0.850980}%
\pgfsetstrokecolor{currentstroke}%
\pgfsetdash{}{0pt}%
\pgfpathmoveto{\pgfqpoint{3.281584in}{3.154037in}}%
\pgfusepath{stroke}%
\end{pgfscope}%
\begin{pgfscope}%
\pgfpathrectangle{\pgfqpoint{0.765000in}{0.660000in}}{\pgfqpoint{4.620000in}{4.620000in}}%
\pgfusepath{clip}%
\pgfsetbuttcap%
\pgfsetroundjoin%
\definecolor{currentfill}{rgb}{0.019608,0.556863,0.850980}%
\pgfsetfillcolor{currentfill}%
\pgfsetlinewidth{1.003750pt}%
\definecolor{currentstroke}{rgb}{0.019608,0.556863,0.850980}%
\pgfsetstrokecolor{currentstroke}%
\pgfsetdash{}{0pt}%
\pgfsys@defobject{currentmarker}{\pgfqpoint{-0.033333in}{-0.033333in}}{\pgfqpoint{0.033333in}{0.033333in}}{%
\pgfpathmoveto{\pgfqpoint{-0.033333in}{0.000000in}}%
\pgfpathlineto{\pgfqpoint{0.033333in}{0.000000in}}%
\pgfpathmoveto{\pgfqpoint{0.000000in}{-0.033333in}}%
\pgfpathlineto{\pgfqpoint{0.000000in}{0.033333in}}%
\pgfusepath{stroke,fill}%
}%
\begin{pgfscope}%
\pgfsys@transformshift{3.281584in}{3.154037in}%
\pgfsys@useobject{currentmarker}{}%
\end{pgfscope}%
\end{pgfscope}%
\begin{pgfscope}%
\pgfpathrectangle{\pgfqpoint{0.765000in}{0.660000in}}{\pgfqpoint{4.620000in}{4.620000in}}%
\pgfusepath{clip}%
\pgfsetroundcap%
\pgfsetroundjoin%
\pgfsetlinewidth{1.204500pt}%
\definecolor{currentstroke}{rgb}{0.019608,0.556863,0.850980}%
\pgfsetstrokecolor{currentstroke}%
\pgfsetdash{}{0pt}%
\pgfpathmoveto{\pgfqpoint{3.339245in}{3.235106in}}%
\pgfusepath{stroke}%
\end{pgfscope}%
\begin{pgfscope}%
\pgfpathrectangle{\pgfqpoint{0.765000in}{0.660000in}}{\pgfqpoint{4.620000in}{4.620000in}}%
\pgfusepath{clip}%
\pgfsetbuttcap%
\pgfsetroundjoin%
\definecolor{currentfill}{rgb}{0.019608,0.556863,0.850980}%
\pgfsetfillcolor{currentfill}%
\pgfsetlinewidth{1.003750pt}%
\definecolor{currentstroke}{rgb}{0.019608,0.556863,0.850980}%
\pgfsetstrokecolor{currentstroke}%
\pgfsetdash{}{0pt}%
\pgfsys@defobject{currentmarker}{\pgfqpoint{-0.033333in}{-0.033333in}}{\pgfqpoint{0.033333in}{0.033333in}}{%
\pgfpathmoveto{\pgfqpoint{-0.033333in}{0.000000in}}%
\pgfpathlineto{\pgfqpoint{0.033333in}{0.000000in}}%
\pgfpathmoveto{\pgfqpoint{0.000000in}{-0.033333in}}%
\pgfpathlineto{\pgfqpoint{0.000000in}{0.033333in}}%
\pgfusepath{stroke,fill}%
}%
\begin{pgfscope}%
\pgfsys@transformshift{3.339245in}{3.235106in}%
\pgfsys@useobject{currentmarker}{}%
\end{pgfscope}%
\end{pgfscope}%
\begin{pgfscope}%
\pgfpathrectangle{\pgfqpoint{0.765000in}{0.660000in}}{\pgfqpoint{4.620000in}{4.620000in}}%
\pgfusepath{clip}%
\pgfsetroundcap%
\pgfsetroundjoin%
\pgfsetlinewidth{1.204500pt}%
\definecolor{currentstroke}{rgb}{0.019608,0.556863,0.850980}%
\pgfsetstrokecolor{currentstroke}%
\pgfsetdash{}{0pt}%
\pgfpathmoveto{\pgfqpoint{3.223923in}{3.235106in}}%
\pgfusepath{stroke}%
\end{pgfscope}%
\begin{pgfscope}%
\pgfpathrectangle{\pgfqpoint{0.765000in}{0.660000in}}{\pgfqpoint{4.620000in}{4.620000in}}%
\pgfusepath{clip}%
\pgfsetbuttcap%
\pgfsetroundjoin%
\definecolor{currentfill}{rgb}{0.019608,0.556863,0.850980}%
\pgfsetfillcolor{currentfill}%
\pgfsetlinewidth{1.003750pt}%
\definecolor{currentstroke}{rgb}{0.019608,0.556863,0.850980}%
\pgfsetstrokecolor{currentstroke}%
\pgfsetdash{}{0pt}%
\pgfsys@defobject{currentmarker}{\pgfqpoint{-0.033333in}{-0.033333in}}{\pgfqpoint{0.033333in}{0.033333in}}{%
\pgfpathmoveto{\pgfqpoint{-0.033333in}{0.000000in}}%
\pgfpathlineto{\pgfqpoint{0.033333in}{0.000000in}}%
\pgfpathmoveto{\pgfqpoint{0.000000in}{-0.033333in}}%
\pgfpathlineto{\pgfqpoint{0.000000in}{0.033333in}}%
\pgfusepath{stroke,fill}%
}%
\begin{pgfscope}%
\pgfsys@transformshift{3.223923in}{3.235106in}%
\pgfsys@useobject{currentmarker}{}%
\end{pgfscope}%
\end{pgfscope}%
\begin{pgfscope}%
\pgfpathrectangle{\pgfqpoint{0.765000in}{0.660000in}}{\pgfqpoint{4.620000in}{4.620000in}}%
\pgfusepath{clip}%
\pgfsetroundcap%
\pgfsetroundjoin%
\pgfsetlinewidth{1.204500pt}%
\definecolor{currentstroke}{rgb}{0.019608,0.556863,0.850980}%
\pgfsetstrokecolor{currentstroke}%
\pgfsetdash{}{0pt}%
\pgfpathmoveto{\pgfqpoint{2.186033in}{3.072967in}}%
\pgfusepath{stroke}%
\end{pgfscope}%
\begin{pgfscope}%
\pgfpathrectangle{\pgfqpoint{0.765000in}{0.660000in}}{\pgfqpoint{4.620000in}{4.620000in}}%
\pgfusepath{clip}%
\pgfsetbuttcap%
\pgfsetroundjoin%
\definecolor{currentfill}{rgb}{0.019608,0.556863,0.850980}%
\pgfsetfillcolor{currentfill}%
\pgfsetlinewidth{1.003750pt}%
\definecolor{currentstroke}{rgb}{0.019608,0.556863,0.850980}%
\pgfsetstrokecolor{currentstroke}%
\pgfsetdash{}{0pt}%
\pgfsys@defobject{currentmarker}{\pgfqpoint{-0.033333in}{-0.033333in}}{\pgfqpoint{0.033333in}{0.033333in}}{%
\pgfpathmoveto{\pgfqpoint{-0.033333in}{0.000000in}}%
\pgfpathlineto{\pgfqpoint{0.033333in}{0.000000in}}%
\pgfpathmoveto{\pgfqpoint{0.000000in}{-0.033333in}}%
\pgfpathlineto{\pgfqpoint{0.000000in}{0.033333in}}%
\pgfusepath{stroke,fill}%
}%
\begin{pgfscope}%
\pgfsys@transformshift{2.186033in}{3.072967in}%
\pgfsys@useobject{currentmarker}{}%
\end{pgfscope}%
\end{pgfscope}%
\begin{pgfscope}%
\pgfpathrectangle{\pgfqpoint{0.765000in}{0.660000in}}{\pgfqpoint{4.620000in}{4.620000in}}%
\pgfusepath{clip}%
\pgfsetroundcap%
\pgfsetroundjoin%
\pgfsetlinewidth{1.204500pt}%
\definecolor{currentstroke}{rgb}{0.019608,0.556863,0.850980}%
\pgfsetstrokecolor{currentstroke}%
\pgfsetdash{}{0pt}%
\pgfpathmoveto{\pgfqpoint{2.877960in}{3.235106in}}%
\pgfusepath{stroke}%
\end{pgfscope}%
\begin{pgfscope}%
\pgfpathrectangle{\pgfqpoint{0.765000in}{0.660000in}}{\pgfqpoint{4.620000in}{4.620000in}}%
\pgfusepath{clip}%
\pgfsetbuttcap%
\pgfsetroundjoin%
\definecolor{currentfill}{rgb}{0.019608,0.556863,0.850980}%
\pgfsetfillcolor{currentfill}%
\pgfsetlinewidth{1.003750pt}%
\definecolor{currentstroke}{rgb}{0.019608,0.556863,0.850980}%
\pgfsetstrokecolor{currentstroke}%
\pgfsetdash{}{0pt}%
\pgfsys@defobject{currentmarker}{\pgfqpoint{-0.033333in}{-0.033333in}}{\pgfqpoint{0.033333in}{0.033333in}}{%
\pgfpathmoveto{\pgfqpoint{-0.033333in}{0.000000in}}%
\pgfpathlineto{\pgfqpoint{0.033333in}{0.000000in}}%
\pgfpathmoveto{\pgfqpoint{0.000000in}{-0.033333in}}%
\pgfpathlineto{\pgfqpoint{0.000000in}{0.033333in}}%
\pgfusepath{stroke,fill}%
}%
\begin{pgfscope}%
\pgfsys@transformshift{2.877960in}{3.235106in}%
\pgfsys@useobject{currentmarker}{}%
\end{pgfscope}%
\end{pgfscope}%
\begin{pgfscope}%
\pgfpathrectangle{\pgfqpoint{0.765000in}{0.660000in}}{\pgfqpoint{4.620000in}{4.620000in}}%
\pgfusepath{clip}%
\pgfsetroundcap%
\pgfsetroundjoin%
\pgfsetlinewidth{1.204500pt}%
\definecolor{currentstroke}{rgb}{0.019608,0.556863,0.850980}%
\pgfsetstrokecolor{currentstroke}%
\pgfsetdash{}{0pt}%
\pgfpathmoveto{\pgfqpoint{2.762639in}{3.072967in}}%
\pgfusepath{stroke}%
\end{pgfscope}%
\begin{pgfscope}%
\pgfpathrectangle{\pgfqpoint{0.765000in}{0.660000in}}{\pgfqpoint{4.620000in}{4.620000in}}%
\pgfusepath{clip}%
\pgfsetbuttcap%
\pgfsetroundjoin%
\definecolor{currentfill}{rgb}{0.019608,0.556863,0.850980}%
\pgfsetfillcolor{currentfill}%
\pgfsetlinewidth{1.003750pt}%
\definecolor{currentstroke}{rgb}{0.019608,0.556863,0.850980}%
\pgfsetstrokecolor{currentstroke}%
\pgfsetdash{}{0pt}%
\pgfsys@defobject{currentmarker}{\pgfqpoint{-0.033333in}{-0.033333in}}{\pgfqpoint{0.033333in}{0.033333in}}{%
\pgfpathmoveto{\pgfqpoint{-0.033333in}{0.000000in}}%
\pgfpathlineto{\pgfqpoint{0.033333in}{0.000000in}}%
\pgfpathmoveto{\pgfqpoint{0.000000in}{-0.033333in}}%
\pgfpathlineto{\pgfqpoint{0.000000in}{0.033333in}}%
\pgfusepath{stroke,fill}%
}%
\begin{pgfscope}%
\pgfsys@transformshift{2.762639in}{3.072967in}%
\pgfsys@useobject{currentmarker}{}%
\end{pgfscope}%
\end{pgfscope}%
\begin{pgfscope}%
\pgfpathrectangle{\pgfqpoint{0.765000in}{0.660000in}}{\pgfqpoint{4.620000in}{4.620000in}}%
\pgfusepath{clip}%
\pgfsetroundcap%
\pgfsetroundjoin%
\pgfsetlinewidth{1.204500pt}%
\definecolor{currentstroke}{rgb}{0.019608,0.556863,0.850980}%
\pgfsetstrokecolor{currentstroke}%
\pgfsetdash{}{0pt}%
\pgfpathmoveto{\pgfqpoint{3.512226in}{3.316176in}}%
\pgfusepath{stroke}%
\end{pgfscope}%
\begin{pgfscope}%
\pgfpathrectangle{\pgfqpoint{0.765000in}{0.660000in}}{\pgfqpoint{4.620000in}{4.620000in}}%
\pgfusepath{clip}%
\pgfsetbuttcap%
\pgfsetroundjoin%
\definecolor{currentfill}{rgb}{0.019608,0.556863,0.850980}%
\pgfsetfillcolor{currentfill}%
\pgfsetlinewidth{1.003750pt}%
\definecolor{currentstroke}{rgb}{0.019608,0.556863,0.850980}%
\pgfsetstrokecolor{currentstroke}%
\pgfsetdash{}{0pt}%
\pgfsys@defobject{currentmarker}{\pgfqpoint{-0.033333in}{-0.033333in}}{\pgfqpoint{0.033333in}{0.033333in}}{%
\pgfpathmoveto{\pgfqpoint{-0.033333in}{0.000000in}}%
\pgfpathlineto{\pgfqpoint{0.033333in}{0.000000in}}%
\pgfpathmoveto{\pgfqpoint{0.000000in}{-0.033333in}}%
\pgfpathlineto{\pgfqpoint{0.000000in}{0.033333in}}%
\pgfusepath{stroke,fill}%
}%
\begin{pgfscope}%
\pgfsys@transformshift{3.512226in}{3.316176in}%
\pgfsys@useobject{currentmarker}{}%
\end{pgfscope}%
\end{pgfscope}%
\begin{pgfscope}%
\pgfpathrectangle{\pgfqpoint{0.765000in}{0.660000in}}{\pgfqpoint{4.620000in}{4.620000in}}%
\pgfusepath{clip}%
\pgfsetroundcap%
\pgfsetroundjoin%
\pgfsetlinewidth{1.204500pt}%
\definecolor{currentstroke}{rgb}{0.019608,0.556863,0.850980}%
\pgfsetstrokecolor{currentstroke}%
\pgfsetdash{}{0pt}%
\pgfpathmoveto{\pgfqpoint{2.128372in}{2.829758in}}%
\pgfusepath{stroke}%
\end{pgfscope}%
\begin{pgfscope}%
\pgfpathrectangle{\pgfqpoint{0.765000in}{0.660000in}}{\pgfqpoint{4.620000in}{4.620000in}}%
\pgfusepath{clip}%
\pgfsetbuttcap%
\pgfsetroundjoin%
\definecolor{currentfill}{rgb}{0.019608,0.556863,0.850980}%
\pgfsetfillcolor{currentfill}%
\pgfsetlinewidth{1.003750pt}%
\definecolor{currentstroke}{rgb}{0.019608,0.556863,0.850980}%
\pgfsetstrokecolor{currentstroke}%
\pgfsetdash{}{0pt}%
\pgfsys@defobject{currentmarker}{\pgfqpoint{-0.033333in}{-0.033333in}}{\pgfqpoint{0.033333in}{0.033333in}}{%
\pgfpathmoveto{\pgfqpoint{-0.033333in}{0.000000in}}%
\pgfpathlineto{\pgfqpoint{0.033333in}{0.000000in}}%
\pgfpathmoveto{\pgfqpoint{0.000000in}{-0.033333in}}%
\pgfpathlineto{\pgfqpoint{0.000000in}{0.033333in}}%
\pgfusepath{stroke,fill}%
}%
\begin{pgfscope}%
\pgfsys@transformshift{2.128372in}{2.829758in}%
\pgfsys@useobject{currentmarker}{}%
\end{pgfscope}%
\end{pgfscope}%
\begin{pgfscope}%
\pgfpathrectangle{\pgfqpoint{0.765000in}{0.660000in}}{\pgfqpoint{4.620000in}{4.620000in}}%
\pgfusepath{clip}%
\pgfsetroundcap%
\pgfsetroundjoin%
\pgfsetlinewidth{1.204500pt}%
\definecolor{currentstroke}{rgb}{0.019608,0.556863,0.850980}%
\pgfsetstrokecolor{currentstroke}%
\pgfsetdash{}{0pt}%
\pgfpathmoveto{\pgfqpoint{2.877960in}{3.072967in}}%
\pgfusepath{stroke}%
\end{pgfscope}%
\begin{pgfscope}%
\pgfpathrectangle{\pgfqpoint{0.765000in}{0.660000in}}{\pgfqpoint{4.620000in}{4.620000in}}%
\pgfusepath{clip}%
\pgfsetbuttcap%
\pgfsetroundjoin%
\definecolor{currentfill}{rgb}{0.019608,0.556863,0.850980}%
\pgfsetfillcolor{currentfill}%
\pgfsetlinewidth{1.003750pt}%
\definecolor{currentstroke}{rgb}{0.019608,0.556863,0.850980}%
\pgfsetstrokecolor{currentstroke}%
\pgfsetdash{}{0pt}%
\pgfsys@defobject{currentmarker}{\pgfqpoint{-0.033333in}{-0.033333in}}{\pgfqpoint{0.033333in}{0.033333in}}{%
\pgfpathmoveto{\pgfqpoint{-0.033333in}{0.000000in}}%
\pgfpathlineto{\pgfqpoint{0.033333in}{0.000000in}}%
\pgfpathmoveto{\pgfqpoint{0.000000in}{-0.033333in}}%
\pgfpathlineto{\pgfqpoint{0.000000in}{0.033333in}}%
\pgfusepath{stroke,fill}%
}%
\begin{pgfscope}%
\pgfsys@transformshift{2.877960in}{3.072967in}%
\pgfsys@useobject{currentmarker}{}%
\end{pgfscope}%
\end{pgfscope}%
\begin{pgfscope}%
\pgfpathrectangle{\pgfqpoint{0.765000in}{0.660000in}}{\pgfqpoint{4.620000in}{4.620000in}}%
\pgfusepath{clip}%
\pgfsetroundcap%
\pgfsetroundjoin%
\pgfsetlinewidth{1.204500pt}%
\definecolor{currentstroke}{rgb}{0.019608,0.556863,0.850980}%
\pgfsetstrokecolor{currentstroke}%
\pgfsetdash{}{0pt}%
\pgfpathmoveto{\pgfqpoint{2.704978in}{3.154037in}}%
\pgfusepath{stroke}%
\end{pgfscope}%
\begin{pgfscope}%
\pgfpathrectangle{\pgfqpoint{0.765000in}{0.660000in}}{\pgfqpoint{4.620000in}{4.620000in}}%
\pgfusepath{clip}%
\pgfsetbuttcap%
\pgfsetroundjoin%
\definecolor{currentfill}{rgb}{0.019608,0.556863,0.850980}%
\pgfsetfillcolor{currentfill}%
\pgfsetlinewidth{1.003750pt}%
\definecolor{currentstroke}{rgb}{0.019608,0.556863,0.850980}%
\pgfsetstrokecolor{currentstroke}%
\pgfsetdash{}{0pt}%
\pgfsys@defobject{currentmarker}{\pgfqpoint{-0.033333in}{-0.033333in}}{\pgfqpoint{0.033333in}{0.033333in}}{%
\pgfpathmoveto{\pgfqpoint{-0.033333in}{0.000000in}}%
\pgfpathlineto{\pgfqpoint{0.033333in}{0.000000in}}%
\pgfpathmoveto{\pgfqpoint{0.000000in}{-0.033333in}}%
\pgfpathlineto{\pgfqpoint{0.000000in}{0.033333in}}%
\pgfusepath{stroke,fill}%
}%
\begin{pgfscope}%
\pgfsys@transformshift{2.704978in}{3.154037in}%
\pgfsys@useobject{currentmarker}{}%
\end{pgfscope}%
\end{pgfscope}%
\begin{pgfscope}%
\pgfpathrectangle{\pgfqpoint{0.765000in}{0.660000in}}{\pgfqpoint{4.620000in}{4.620000in}}%
\pgfusepath{clip}%
\pgfsetroundcap%
\pgfsetroundjoin%
\pgfsetlinewidth{1.204500pt}%
\definecolor{currentstroke}{rgb}{0.019608,0.556863,0.850980}%
\pgfsetstrokecolor{currentstroke}%
\pgfsetdash{}{0pt}%
\pgfpathmoveto{\pgfqpoint{1.667087in}{2.829758in}}%
\pgfusepath{stroke}%
\end{pgfscope}%
\begin{pgfscope}%
\pgfpathrectangle{\pgfqpoint{0.765000in}{0.660000in}}{\pgfqpoint{4.620000in}{4.620000in}}%
\pgfusepath{clip}%
\pgfsetbuttcap%
\pgfsetroundjoin%
\definecolor{currentfill}{rgb}{0.019608,0.556863,0.850980}%
\pgfsetfillcolor{currentfill}%
\pgfsetlinewidth{1.003750pt}%
\definecolor{currentstroke}{rgb}{0.019608,0.556863,0.850980}%
\pgfsetstrokecolor{currentstroke}%
\pgfsetdash{}{0pt}%
\pgfsys@defobject{currentmarker}{\pgfqpoint{-0.033333in}{-0.033333in}}{\pgfqpoint{0.033333in}{0.033333in}}{%
\pgfpathmoveto{\pgfqpoint{-0.033333in}{0.000000in}}%
\pgfpathlineto{\pgfqpoint{0.033333in}{0.000000in}}%
\pgfpathmoveto{\pgfqpoint{0.000000in}{-0.033333in}}%
\pgfpathlineto{\pgfqpoint{0.000000in}{0.033333in}}%
\pgfusepath{stroke,fill}%
}%
\begin{pgfscope}%
\pgfsys@transformshift{1.667087in}{2.829758in}%
\pgfsys@useobject{currentmarker}{}%
\end{pgfscope}%
\end{pgfscope}%
\begin{pgfscope}%
\pgfpathrectangle{\pgfqpoint{0.765000in}{0.660000in}}{\pgfqpoint{4.620000in}{4.620000in}}%
\pgfusepath{clip}%
\pgfsetroundcap%
\pgfsetroundjoin%
\pgfsetlinewidth{1.204500pt}%
\definecolor{currentstroke}{rgb}{0.019608,0.556863,0.850980}%
\pgfsetstrokecolor{currentstroke}%
\pgfsetdash{}{0pt}%
\pgfpathmoveto{\pgfqpoint{3.108602in}{3.235106in}}%
\pgfusepath{stroke}%
\end{pgfscope}%
\begin{pgfscope}%
\pgfpathrectangle{\pgfqpoint{0.765000in}{0.660000in}}{\pgfqpoint{4.620000in}{4.620000in}}%
\pgfusepath{clip}%
\pgfsetbuttcap%
\pgfsetroundjoin%
\definecolor{currentfill}{rgb}{0.019608,0.556863,0.850980}%
\pgfsetfillcolor{currentfill}%
\pgfsetlinewidth{1.003750pt}%
\definecolor{currentstroke}{rgb}{0.019608,0.556863,0.850980}%
\pgfsetstrokecolor{currentstroke}%
\pgfsetdash{}{0pt}%
\pgfsys@defobject{currentmarker}{\pgfqpoint{-0.033333in}{-0.033333in}}{\pgfqpoint{0.033333in}{0.033333in}}{%
\pgfpathmoveto{\pgfqpoint{-0.033333in}{0.000000in}}%
\pgfpathlineto{\pgfqpoint{0.033333in}{0.000000in}}%
\pgfpathmoveto{\pgfqpoint{0.000000in}{-0.033333in}}%
\pgfpathlineto{\pgfqpoint{0.000000in}{0.033333in}}%
\pgfusepath{stroke,fill}%
}%
\begin{pgfscope}%
\pgfsys@transformshift{3.108602in}{3.235106in}%
\pgfsys@useobject{currentmarker}{}%
\end{pgfscope}%
\end{pgfscope}%
\begin{pgfscope}%
\pgfpathrectangle{\pgfqpoint{0.765000in}{0.660000in}}{\pgfqpoint{4.620000in}{4.620000in}}%
\pgfusepath{clip}%
\pgfsetroundcap%
\pgfsetroundjoin%
\pgfsetlinewidth{1.204500pt}%
\definecolor{currentstroke}{rgb}{0.019608,0.556863,0.850980}%
\pgfsetstrokecolor{currentstroke}%
\pgfsetdash{}{0pt}%
\pgfpathmoveto{\pgfqpoint{1.897730in}{2.829758in}}%
\pgfusepath{stroke}%
\end{pgfscope}%
\begin{pgfscope}%
\pgfpathrectangle{\pgfqpoint{0.765000in}{0.660000in}}{\pgfqpoint{4.620000in}{4.620000in}}%
\pgfusepath{clip}%
\pgfsetbuttcap%
\pgfsetroundjoin%
\definecolor{currentfill}{rgb}{0.019608,0.556863,0.850980}%
\pgfsetfillcolor{currentfill}%
\pgfsetlinewidth{1.003750pt}%
\definecolor{currentstroke}{rgb}{0.019608,0.556863,0.850980}%
\pgfsetstrokecolor{currentstroke}%
\pgfsetdash{}{0pt}%
\pgfsys@defobject{currentmarker}{\pgfqpoint{-0.033333in}{-0.033333in}}{\pgfqpoint{0.033333in}{0.033333in}}{%
\pgfpathmoveto{\pgfqpoint{-0.033333in}{0.000000in}}%
\pgfpathlineto{\pgfqpoint{0.033333in}{0.000000in}}%
\pgfpathmoveto{\pgfqpoint{0.000000in}{-0.033333in}}%
\pgfpathlineto{\pgfqpoint{0.000000in}{0.033333in}}%
\pgfusepath{stroke,fill}%
}%
\begin{pgfscope}%
\pgfsys@transformshift{1.897730in}{2.829758in}%
\pgfsys@useobject{currentmarker}{}%
\end{pgfscope}%
\end{pgfscope}%
\begin{pgfscope}%
\pgfpathrectangle{\pgfqpoint{0.765000in}{0.660000in}}{\pgfqpoint{4.620000in}{4.620000in}}%
\pgfusepath{clip}%
\pgfsetroundcap%
\pgfsetroundjoin%
\pgfsetlinewidth{1.204500pt}%
\definecolor{currentstroke}{rgb}{0.019608,0.556863,0.850980}%
\pgfsetstrokecolor{currentstroke}%
\pgfsetdash{}{0pt}%
\pgfpathmoveto{\pgfqpoint{2.935620in}{3.154037in}}%
\pgfusepath{stroke}%
\end{pgfscope}%
\begin{pgfscope}%
\pgfpathrectangle{\pgfqpoint{0.765000in}{0.660000in}}{\pgfqpoint{4.620000in}{4.620000in}}%
\pgfusepath{clip}%
\pgfsetbuttcap%
\pgfsetroundjoin%
\definecolor{currentfill}{rgb}{0.019608,0.556863,0.850980}%
\pgfsetfillcolor{currentfill}%
\pgfsetlinewidth{1.003750pt}%
\definecolor{currentstroke}{rgb}{0.019608,0.556863,0.850980}%
\pgfsetstrokecolor{currentstroke}%
\pgfsetdash{}{0pt}%
\pgfsys@defobject{currentmarker}{\pgfqpoint{-0.033333in}{-0.033333in}}{\pgfqpoint{0.033333in}{0.033333in}}{%
\pgfpathmoveto{\pgfqpoint{-0.033333in}{0.000000in}}%
\pgfpathlineto{\pgfqpoint{0.033333in}{0.000000in}}%
\pgfpathmoveto{\pgfqpoint{0.000000in}{-0.033333in}}%
\pgfpathlineto{\pgfqpoint{0.000000in}{0.033333in}}%
\pgfusepath{stroke,fill}%
}%
\begin{pgfscope}%
\pgfsys@transformshift{2.935620in}{3.154037in}%
\pgfsys@useobject{currentmarker}{}%
\end{pgfscope}%
\end{pgfscope}%
\begin{pgfscope}%
\pgfpathrectangle{\pgfqpoint{0.765000in}{0.660000in}}{\pgfqpoint{4.620000in}{4.620000in}}%
\pgfusepath{clip}%
\pgfsetroundcap%
\pgfsetroundjoin%
\pgfsetlinewidth{1.204500pt}%
\definecolor{currentstroke}{rgb}{0.019608,0.556863,0.850980}%
\pgfsetstrokecolor{currentstroke}%
\pgfsetdash{}{0pt}%
\pgfpathmoveto{\pgfqpoint{2.877960in}{3.072967in}}%
\pgfusepath{stroke}%
\end{pgfscope}%
\begin{pgfscope}%
\pgfpathrectangle{\pgfqpoint{0.765000in}{0.660000in}}{\pgfqpoint{4.620000in}{4.620000in}}%
\pgfusepath{clip}%
\pgfsetbuttcap%
\pgfsetroundjoin%
\definecolor{currentfill}{rgb}{0.019608,0.556863,0.850980}%
\pgfsetfillcolor{currentfill}%
\pgfsetlinewidth{1.003750pt}%
\definecolor{currentstroke}{rgb}{0.019608,0.556863,0.850980}%
\pgfsetstrokecolor{currentstroke}%
\pgfsetdash{}{0pt}%
\pgfsys@defobject{currentmarker}{\pgfqpoint{-0.033333in}{-0.033333in}}{\pgfqpoint{0.033333in}{0.033333in}}{%
\pgfpathmoveto{\pgfqpoint{-0.033333in}{0.000000in}}%
\pgfpathlineto{\pgfqpoint{0.033333in}{0.000000in}}%
\pgfpathmoveto{\pgfqpoint{0.000000in}{-0.033333in}}%
\pgfpathlineto{\pgfqpoint{0.000000in}{0.033333in}}%
\pgfusepath{stroke,fill}%
}%
\begin{pgfscope}%
\pgfsys@transformshift{2.877960in}{3.072967in}%
\pgfsys@useobject{currentmarker}{}%
\end{pgfscope}%
\end{pgfscope}%
\begin{pgfscope}%
\pgfpathrectangle{\pgfqpoint{0.765000in}{0.660000in}}{\pgfqpoint{4.620000in}{4.620000in}}%
\pgfusepath{clip}%
\pgfsetroundcap%
\pgfsetroundjoin%
\pgfsetlinewidth{1.204500pt}%
\definecolor{currentstroke}{rgb}{0.019608,0.556863,0.850980}%
\pgfsetstrokecolor{currentstroke}%
\pgfsetdash{}{0pt}%
\pgfpathmoveto{\pgfqpoint{3.166263in}{3.154037in}}%
\pgfusepath{stroke}%
\end{pgfscope}%
\begin{pgfscope}%
\pgfpathrectangle{\pgfqpoint{0.765000in}{0.660000in}}{\pgfqpoint{4.620000in}{4.620000in}}%
\pgfusepath{clip}%
\pgfsetbuttcap%
\pgfsetroundjoin%
\definecolor{currentfill}{rgb}{0.019608,0.556863,0.850980}%
\pgfsetfillcolor{currentfill}%
\pgfsetlinewidth{1.003750pt}%
\definecolor{currentstroke}{rgb}{0.019608,0.556863,0.850980}%
\pgfsetstrokecolor{currentstroke}%
\pgfsetdash{}{0pt}%
\pgfsys@defobject{currentmarker}{\pgfqpoint{-0.033333in}{-0.033333in}}{\pgfqpoint{0.033333in}{0.033333in}}{%
\pgfpathmoveto{\pgfqpoint{-0.033333in}{0.000000in}}%
\pgfpathlineto{\pgfqpoint{0.033333in}{0.000000in}}%
\pgfpathmoveto{\pgfqpoint{0.000000in}{-0.033333in}}%
\pgfpathlineto{\pgfqpoint{0.000000in}{0.033333in}}%
\pgfusepath{stroke,fill}%
}%
\begin{pgfscope}%
\pgfsys@transformshift{3.166263in}{3.154037in}%
\pgfsys@useobject{currentmarker}{}%
\end{pgfscope}%
\end{pgfscope}%
\begin{pgfscope}%
\pgfpathrectangle{\pgfqpoint{0.765000in}{0.660000in}}{\pgfqpoint{4.620000in}{4.620000in}}%
\pgfusepath{clip}%
\pgfsetroundcap%
\pgfsetroundjoin%
\pgfsetlinewidth{1.204500pt}%
\definecolor{currentstroke}{rgb}{0.019608,0.556863,0.850980}%
\pgfsetstrokecolor{currentstroke}%
\pgfsetdash{}{0pt}%
\pgfpathmoveto{\pgfqpoint{3.108602in}{3.235106in}}%
\pgfusepath{stroke}%
\end{pgfscope}%
\begin{pgfscope}%
\pgfpathrectangle{\pgfqpoint{0.765000in}{0.660000in}}{\pgfqpoint{4.620000in}{4.620000in}}%
\pgfusepath{clip}%
\pgfsetbuttcap%
\pgfsetroundjoin%
\definecolor{currentfill}{rgb}{0.019608,0.556863,0.850980}%
\pgfsetfillcolor{currentfill}%
\pgfsetlinewidth{1.003750pt}%
\definecolor{currentstroke}{rgb}{0.019608,0.556863,0.850980}%
\pgfsetstrokecolor{currentstroke}%
\pgfsetdash{}{0pt}%
\pgfsys@defobject{currentmarker}{\pgfqpoint{-0.033333in}{-0.033333in}}{\pgfqpoint{0.033333in}{0.033333in}}{%
\pgfpathmoveto{\pgfqpoint{-0.033333in}{0.000000in}}%
\pgfpathlineto{\pgfqpoint{0.033333in}{0.000000in}}%
\pgfpathmoveto{\pgfqpoint{0.000000in}{-0.033333in}}%
\pgfpathlineto{\pgfqpoint{0.000000in}{0.033333in}}%
\pgfusepath{stroke,fill}%
}%
\begin{pgfscope}%
\pgfsys@transformshift{3.108602in}{3.235106in}%
\pgfsys@useobject{currentmarker}{}%
\end{pgfscope}%
\end{pgfscope}%
\begin{pgfscope}%
\pgfpathrectangle{\pgfqpoint{0.765000in}{0.660000in}}{\pgfqpoint{4.620000in}{4.620000in}}%
\pgfusepath{clip}%
\pgfsetroundcap%
\pgfsetroundjoin%
\pgfsetlinewidth{1.204500pt}%
\definecolor{currentstroke}{rgb}{0.019608,0.556863,0.850980}%
\pgfsetstrokecolor{currentstroke}%
\pgfsetdash{}{0pt}%
\pgfpathmoveto{\pgfqpoint{2.474336in}{2.829758in}}%
\pgfusepath{stroke}%
\end{pgfscope}%
\begin{pgfscope}%
\pgfpathrectangle{\pgfqpoint{0.765000in}{0.660000in}}{\pgfqpoint{4.620000in}{4.620000in}}%
\pgfusepath{clip}%
\pgfsetbuttcap%
\pgfsetroundjoin%
\definecolor{currentfill}{rgb}{0.019608,0.556863,0.850980}%
\pgfsetfillcolor{currentfill}%
\pgfsetlinewidth{1.003750pt}%
\definecolor{currentstroke}{rgb}{0.019608,0.556863,0.850980}%
\pgfsetstrokecolor{currentstroke}%
\pgfsetdash{}{0pt}%
\pgfsys@defobject{currentmarker}{\pgfqpoint{-0.033333in}{-0.033333in}}{\pgfqpoint{0.033333in}{0.033333in}}{%
\pgfpathmoveto{\pgfqpoint{-0.033333in}{0.000000in}}%
\pgfpathlineto{\pgfqpoint{0.033333in}{0.000000in}}%
\pgfpathmoveto{\pgfqpoint{0.000000in}{-0.033333in}}%
\pgfpathlineto{\pgfqpoint{0.000000in}{0.033333in}}%
\pgfusepath{stroke,fill}%
}%
\begin{pgfscope}%
\pgfsys@transformshift{2.474336in}{2.829758in}%
\pgfsys@useobject{currentmarker}{}%
\end{pgfscope}%
\end{pgfscope}%
\begin{pgfscope}%
\pgfpathrectangle{\pgfqpoint{0.765000in}{0.660000in}}{\pgfqpoint{4.620000in}{4.620000in}}%
\pgfusepath{clip}%
\pgfsetroundcap%
\pgfsetroundjoin%
\pgfsetlinewidth{1.204500pt}%
\definecolor{currentstroke}{rgb}{0.019608,0.556863,0.850980}%
\pgfsetstrokecolor{currentstroke}%
\pgfsetdash{}{0pt}%
\pgfpathmoveto{\pgfqpoint{2.186033in}{3.235106in}}%
\pgfusepath{stroke}%
\end{pgfscope}%
\begin{pgfscope}%
\pgfpathrectangle{\pgfqpoint{0.765000in}{0.660000in}}{\pgfqpoint{4.620000in}{4.620000in}}%
\pgfusepath{clip}%
\pgfsetbuttcap%
\pgfsetroundjoin%
\definecolor{currentfill}{rgb}{0.019608,0.556863,0.850980}%
\pgfsetfillcolor{currentfill}%
\pgfsetlinewidth{1.003750pt}%
\definecolor{currentstroke}{rgb}{0.019608,0.556863,0.850980}%
\pgfsetstrokecolor{currentstroke}%
\pgfsetdash{}{0pt}%
\pgfsys@defobject{currentmarker}{\pgfqpoint{-0.033333in}{-0.033333in}}{\pgfqpoint{0.033333in}{0.033333in}}{%
\pgfpathmoveto{\pgfqpoint{-0.033333in}{0.000000in}}%
\pgfpathlineto{\pgfqpoint{0.033333in}{0.000000in}}%
\pgfpathmoveto{\pgfqpoint{0.000000in}{-0.033333in}}%
\pgfpathlineto{\pgfqpoint{0.000000in}{0.033333in}}%
\pgfusepath{stroke,fill}%
}%
\begin{pgfscope}%
\pgfsys@transformshift{2.186033in}{3.235106in}%
\pgfsys@useobject{currentmarker}{}%
\end{pgfscope}%
\end{pgfscope}%
\begin{pgfscope}%
\pgfpathrectangle{\pgfqpoint{0.765000in}{0.660000in}}{\pgfqpoint{4.620000in}{4.620000in}}%
\pgfusepath{clip}%
\pgfsetroundcap%
\pgfsetroundjoin%
\pgfsetlinewidth{1.204500pt}%
\definecolor{currentstroke}{rgb}{0.019608,0.556863,0.850980}%
\pgfsetstrokecolor{currentstroke}%
\pgfsetdash{}{0pt}%
\pgfpathmoveto{\pgfqpoint{2.301354in}{2.910828in}}%
\pgfusepath{stroke}%
\end{pgfscope}%
\begin{pgfscope}%
\pgfpathrectangle{\pgfqpoint{0.765000in}{0.660000in}}{\pgfqpoint{4.620000in}{4.620000in}}%
\pgfusepath{clip}%
\pgfsetbuttcap%
\pgfsetroundjoin%
\definecolor{currentfill}{rgb}{0.019608,0.556863,0.850980}%
\pgfsetfillcolor{currentfill}%
\pgfsetlinewidth{1.003750pt}%
\definecolor{currentstroke}{rgb}{0.019608,0.556863,0.850980}%
\pgfsetstrokecolor{currentstroke}%
\pgfsetdash{}{0pt}%
\pgfsys@defobject{currentmarker}{\pgfqpoint{-0.033333in}{-0.033333in}}{\pgfqpoint{0.033333in}{0.033333in}}{%
\pgfpathmoveto{\pgfqpoint{-0.033333in}{0.000000in}}%
\pgfpathlineto{\pgfqpoint{0.033333in}{0.000000in}}%
\pgfpathmoveto{\pgfqpoint{0.000000in}{-0.033333in}}%
\pgfpathlineto{\pgfqpoint{0.000000in}{0.033333in}}%
\pgfusepath{stroke,fill}%
}%
\begin{pgfscope}%
\pgfsys@transformshift{2.301354in}{2.910828in}%
\pgfsys@useobject{currentmarker}{}%
\end{pgfscope}%
\end{pgfscope}%
\begin{pgfscope}%
\pgfpathrectangle{\pgfqpoint{0.765000in}{0.660000in}}{\pgfqpoint{4.620000in}{4.620000in}}%
\pgfusepath{clip}%
\pgfsetroundcap%
\pgfsetroundjoin%
\pgfsetlinewidth{1.204500pt}%
\definecolor{currentstroke}{rgb}{0.019608,0.556863,0.850980}%
\pgfsetstrokecolor{currentstroke}%
\pgfsetdash{}{0pt}%
\pgfpathmoveto{\pgfqpoint{3.512226in}{3.478315in}}%
\pgfusepath{stroke}%
\end{pgfscope}%
\begin{pgfscope}%
\pgfpathrectangle{\pgfqpoint{0.765000in}{0.660000in}}{\pgfqpoint{4.620000in}{4.620000in}}%
\pgfusepath{clip}%
\pgfsetbuttcap%
\pgfsetroundjoin%
\definecolor{currentfill}{rgb}{0.019608,0.556863,0.850980}%
\pgfsetfillcolor{currentfill}%
\pgfsetlinewidth{1.003750pt}%
\definecolor{currentstroke}{rgb}{0.019608,0.556863,0.850980}%
\pgfsetstrokecolor{currentstroke}%
\pgfsetdash{}{0pt}%
\pgfsys@defobject{currentmarker}{\pgfqpoint{-0.033333in}{-0.033333in}}{\pgfqpoint{0.033333in}{0.033333in}}{%
\pgfpathmoveto{\pgfqpoint{-0.033333in}{0.000000in}}%
\pgfpathlineto{\pgfqpoint{0.033333in}{0.000000in}}%
\pgfpathmoveto{\pgfqpoint{0.000000in}{-0.033333in}}%
\pgfpathlineto{\pgfqpoint{0.000000in}{0.033333in}}%
\pgfusepath{stroke,fill}%
}%
\begin{pgfscope}%
\pgfsys@transformshift{3.512226in}{3.478315in}%
\pgfsys@useobject{currentmarker}{}%
\end{pgfscope}%
\end{pgfscope}%
\begin{pgfscope}%
\pgfpathrectangle{\pgfqpoint{0.765000in}{0.660000in}}{\pgfqpoint{4.620000in}{4.620000in}}%
\pgfusepath{clip}%
\pgfsetroundcap%
\pgfsetroundjoin%
\pgfsetlinewidth{1.204500pt}%
\definecolor{currentstroke}{rgb}{0.019608,0.556863,0.850980}%
\pgfsetstrokecolor{currentstroke}%
\pgfsetdash{}{0pt}%
\pgfpathmoveto{\pgfqpoint{2.762639in}{3.072967in}}%
\pgfusepath{stroke}%
\end{pgfscope}%
\begin{pgfscope}%
\pgfpathrectangle{\pgfqpoint{0.765000in}{0.660000in}}{\pgfqpoint{4.620000in}{4.620000in}}%
\pgfusepath{clip}%
\pgfsetbuttcap%
\pgfsetroundjoin%
\definecolor{currentfill}{rgb}{0.019608,0.556863,0.850980}%
\pgfsetfillcolor{currentfill}%
\pgfsetlinewidth{1.003750pt}%
\definecolor{currentstroke}{rgb}{0.019608,0.556863,0.850980}%
\pgfsetstrokecolor{currentstroke}%
\pgfsetdash{}{0pt}%
\pgfsys@defobject{currentmarker}{\pgfqpoint{-0.033333in}{-0.033333in}}{\pgfqpoint{0.033333in}{0.033333in}}{%
\pgfpathmoveto{\pgfqpoint{-0.033333in}{0.000000in}}%
\pgfpathlineto{\pgfqpoint{0.033333in}{0.000000in}}%
\pgfpathmoveto{\pgfqpoint{0.000000in}{-0.033333in}}%
\pgfpathlineto{\pgfqpoint{0.000000in}{0.033333in}}%
\pgfusepath{stroke,fill}%
}%
\begin{pgfscope}%
\pgfsys@transformshift{2.762639in}{3.072967in}%
\pgfsys@useobject{currentmarker}{}%
\end{pgfscope}%
\end{pgfscope}%
\begin{pgfscope}%
\pgfpathrectangle{\pgfqpoint{0.765000in}{0.660000in}}{\pgfqpoint{4.620000in}{4.620000in}}%
\pgfusepath{clip}%
\pgfsetroundcap%
\pgfsetroundjoin%
\pgfsetlinewidth{1.204500pt}%
\definecolor{currentstroke}{rgb}{0.019608,0.556863,0.850980}%
\pgfsetstrokecolor{currentstroke}%
\pgfsetdash{}{0pt}%
\pgfpathmoveto{\pgfqpoint{2.531996in}{3.235106in}}%
\pgfusepath{stroke}%
\end{pgfscope}%
\begin{pgfscope}%
\pgfpathrectangle{\pgfqpoint{0.765000in}{0.660000in}}{\pgfqpoint{4.620000in}{4.620000in}}%
\pgfusepath{clip}%
\pgfsetbuttcap%
\pgfsetroundjoin%
\definecolor{currentfill}{rgb}{0.019608,0.556863,0.850980}%
\pgfsetfillcolor{currentfill}%
\pgfsetlinewidth{1.003750pt}%
\definecolor{currentstroke}{rgb}{0.019608,0.556863,0.850980}%
\pgfsetstrokecolor{currentstroke}%
\pgfsetdash{}{0pt}%
\pgfsys@defobject{currentmarker}{\pgfqpoint{-0.033333in}{-0.033333in}}{\pgfqpoint{0.033333in}{0.033333in}}{%
\pgfpathmoveto{\pgfqpoint{-0.033333in}{0.000000in}}%
\pgfpathlineto{\pgfqpoint{0.033333in}{0.000000in}}%
\pgfpathmoveto{\pgfqpoint{0.000000in}{-0.033333in}}%
\pgfpathlineto{\pgfqpoint{0.000000in}{0.033333in}}%
\pgfusepath{stroke,fill}%
}%
\begin{pgfscope}%
\pgfsys@transformshift{2.531996in}{3.235106in}%
\pgfsys@useobject{currentmarker}{}%
\end{pgfscope}%
\end{pgfscope}%
\begin{pgfscope}%
\pgfpathrectangle{\pgfqpoint{0.765000in}{0.660000in}}{\pgfqpoint{4.620000in}{4.620000in}}%
\pgfusepath{clip}%
\pgfsetroundcap%
\pgfsetroundjoin%
\pgfsetlinewidth{1.204500pt}%
\definecolor{currentstroke}{rgb}{0.019608,0.556863,0.850980}%
\pgfsetstrokecolor{currentstroke}%
\pgfsetdash{}{0pt}%
\pgfpathmoveto{\pgfqpoint{2.704978in}{2.991898in}}%
\pgfusepath{stroke}%
\end{pgfscope}%
\begin{pgfscope}%
\pgfpathrectangle{\pgfqpoint{0.765000in}{0.660000in}}{\pgfqpoint{4.620000in}{4.620000in}}%
\pgfusepath{clip}%
\pgfsetbuttcap%
\pgfsetroundjoin%
\definecolor{currentfill}{rgb}{0.019608,0.556863,0.850980}%
\pgfsetfillcolor{currentfill}%
\pgfsetlinewidth{1.003750pt}%
\definecolor{currentstroke}{rgb}{0.019608,0.556863,0.850980}%
\pgfsetstrokecolor{currentstroke}%
\pgfsetdash{}{0pt}%
\pgfsys@defobject{currentmarker}{\pgfqpoint{-0.033333in}{-0.033333in}}{\pgfqpoint{0.033333in}{0.033333in}}{%
\pgfpathmoveto{\pgfqpoint{-0.033333in}{0.000000in}}%
\pgfpathlineto{\pgfqpoint{0.033333in}{0.000000in}}%
\pgfpathmoveto{\pgfqpoint{0.000000in}{-0.033333in}}%
\pgfpathlineto{\pgfqpoint{0.000000in}{0.033333in}}%
\pgfusepath{stroke,fill}%
}%
\begin{pgfscope}%
\pgfsys@transformshift{2.704978in}{2.991898in}%
\pgfsys@useobject{currentmarker}{}%
\end{pgfscope}%
\end{pgfscope}%
\begin{pgfscope}%
\pgfpathrectangle{\pgfqpoint{0.765000in}{0.660000in}}{\pgfqpoint{4.620000in}{4.620000in}}%
\pgfusepath{clip}%
\pgfsetroundcap%
\pgfsetroundjoin%
\pgfsetlinewidth{1.204500pt}%
\definecolor{currentstroke}{rgb}{0.019608,0.556863,0.850980}%
\pgfsetstrokecolor{currentstroke}%
\pgfsetdash{}{0pt}%
\pgfpathmoveto{\pgfqpoint{2.877960in}{3.072967in}}%
\pgfusepath{stroke}%
\end{pgfscope}%
\begin{pgfscope}%
\pgfpathrectangle{\pgfqpoint{0.765000in}{0.660000in}}{\pgfqpoint{4.620000in}{4.620000in}}%
\pgfusepath{clip}%
\pgfsetbuttcap%
\pgfsetroundjoin%
\definecolor{currentfill}{rgb}{0.019608,0.556863,0.850980}%
\pgfsetfillcolor{currentfill}%
\pgfsetlinewidth{1.003750pt}%
\definecolor{currentstroke}{rgb}{0.019608,0.556863,0.850980}%
\pgfsetstrokecolor{currentstroke}%
\pgfsetdash{}{0pt}%
\pgfsys@defobject{currentmarker}{\pgfqpoint{-0.033333in}{-0.033333in}}{\pgfqpoint{0.033333in}{0.033333in}}{%
\pgfpathmoveto{\pgfqpoint{-0.033333in}{0.000000in}}%
\pgfpathlineto{\pgfqpoint{0.033333in}{0.000000in}}%
\pgfpathmoveto{\pgfqpoint{0.000000in}{-0.033333in}}%
\pgfpathlineto{\pgfqpoint{0.000000in}{0.033333in}}%
\pgfusepath{stroke,fill}%
}%
\begin{pgfscope}%
\pgfsys@transformshift{2.877960in}{3.072967in}%
\pgfsys@useobject{currentmarker}{}%
\end{pgfscope}%
\end{pgfscope}%
\begin{pgfscope}%
\pgfpathrectangle{\pgfqpoint{0.765000in}{0.660000in}}{\pgfqpoint{4.620000in}{4.620000in}}%
\pgfusepath{clip}%
\pgfsetroundcap%
\pgfsetroundjoin%
\pgfsetlinewidth{1.204500pt}%
\definecolor{currentstroke}{rgb}{0.019608,0.556863,0.850980}%
\pgfsetstrokecolor{currentstroke}%
\pgfsetdash{}{0pt}%
\pgfpathmoveto{\pgfqpoint{3.050942in}{3.154037in}}%
\pgfusepath{stroke}%
\end{pgfscope}%
\begin{pgfscope}%
\pgfpathrectangle{\pgfqpoint{0.765000in}{0.660000in}}{\pgfqpoint{4.620000in}{4.620000in}}%
\pgfusepath{clip}%
\pgfsetbuttcap%
\pgfsetroundjoin%
\definecolor{currentfill}{rgb}{0.019608,0.556863,0.850980}%
\pgfsetfillcolor{currentfill}%
\pgfsetlinewidth{1.003750pt}%
\definecolor{currentstroke}{rgb}{0.019608,0.556863,0.850980}%
\pgfsetstrokecolor{currentstroke}%
\pgfsetdash{}{0pt}%
\pgfsys@defobject{currentmarker}{\pgfqpoint{-0.033333in}{-0.033333in}}{\pgfqpoint{0.033333in}{0.033333in}}{%
\pgfpathmoveto{\pgfqpoint{-0.033333in}{0.000000in}}%
\pgfpathlineto{\pgfqpoint{0.033333in}{0.000000in}}%
\pgfpathmoveto{\pgfqpoint{0.000000in}{-0.033333in}}%
\pgfpathlineto{\pgfqpoint{0.000000in}{0.033333in}}%
\pgfusepath{stroke,fill}%
}%
\begin{pgfscope}%
\pgfsys@transformshift{3.050942in}{3.154037in}%
\pgfsys@useobject{currentmarker}{}%
\end{pgfscope}%
\end{pgfscope}%
\begin{pgfscope}%
\pgfpathrectangle{\pgfqpoint{0.765000in}{0.660000in}}{\pgfqpoint{4.620000in}{4.620000in}}%
\pgfusepath{clip}%
\pgfsetroundcap%
\pgfsetroundjoin%
\pgfsetlinewidth{1.204500pt}%
\definecolor{currentstroke}{rgb}{0.019608,0.556863,0.850980}%
\pgfsetstrokecolor{currentstroke}%
\pgfsetdash{}{0pt}%
\pgfpathmoveto{\pgfqpoint{2.820299in}{3.154037in}}%
\pgfusepath{stroke}%
\end{pgfscope}%
\begin{pgfscope}%
\pgfpathrectangle{\pgfqpoint{0.765000in}{0.660000in}}{\pgfqpoint{4.620000in}{4.620000in}}%
\pgfusepath{clip}%
\pgfsetbuttcap%
\pgfsetroundjoin%
\definecolor{currentfill}{rgb}{0.019608,0.556863,0.850980}%
\pgfsetfillcolor{currentfill}%
\pgfsetlinewidth{1.003750pt}%
\definecolor{currentstroke}{rgb}{0.019608,0.556863,0.850980}%
\pgfsetstrokecolor{currentstroke}%
\pgfsetdash{}{0pt}%
\pgfsys@defobject{currentmarker}{\pgfqpoint{-0.033333in}{-0.033333in}}{\pgfqpoint{0.033333in}{0.033333in}}{%
\pgfpathmoveto{\pgfqpoint{-0.033333in}{0.000000in}}%
\pgfpathlineto{\pgfqpoint{0.033333in}{0.000000in}}%
\pgfpathmoveto{\pgfqpoint{0.000000in}{-0.033333in}}%
\pgfpathlineto{\pgfqpoint{0.000000in}{0.033333in}}%
\pgfusepath{stroke,fill}%
}%
\begin{pgfscope}%
\pgfsys@transformshift{2.820299in}{3.154037in}%
\pgfsys@useobject{currentmarker}{}%
\end{pgfscope}%
\end{pgfscope}%
\begin{pgfscope}%
\pgfpathrectangle{\pgfqpoint{0.765000in}{0.660000in}}{\pgfqpoint{4.620000in}{4.620000in}}%
\pgfusepath{clip}%
\pgfsetroundcap%
\pgfsetroundjoin%
\pgfsetlinewidth{1.204500pt}%
\definecolor{currentstroke}{rgb}{0.019608,0.556863,0.850980}%
\pgfsetstrokecolor{currentstroke}%
\pgfsetdash{}{0pt}%
\pgfpathmoveto{\pgfqpoint{3.223923in}{3.397246in}}%
\pgfusepath{stroke}%
\end{pgfscope}%
\begin{pgfscope}%
\pgfpathrectangle{\pgfqpoint{0.765000in}{0.660000in}}{\pgfqpoint{4.620000in}{4.620000in}}%
\pgfusepath{clip}%
\pgfsetbuttcap%
\pgfsetroundjoin%
\definecolor{currentfill}{rgb}{0.019608,0.556863,0.850980}%
\pgfsetfillcolor{currentfill}%
\pgfsetlinewidth{1.003750pt}%
\definecolor{currentstroke}{rgb}{0.019608,0.556863,0.850980}%
\pgfsetstrokecolor{currentstroke}%
\pgfsetdash{}{0pt}%
\pgfsys@defobject{currentmarker}{\pgfqpoint{-0.033333in}{-0.033333in}}{\pgfqpoint{0.033333in}{0.033333in}}{%
\pgfpathmoveto{\pgfqpoint{-0.033333in}{0.000000in}}%
\pgfpathlineto{\pgfqpoint{0.033333in}{0.000000in}}%
\pgfpathmoveto{\pgfqpoint{0.000000in}{-0.033333in}}%
\pgfpathlineto{\pgfqpoint{0.000000in}{0.033333in}}%
\pgfusepath{stroke,fill}%
}%
\begin{pgfscope}%
\pgfsys@transformshift{3.223923in}{3.397246in}%
\pgfsys@useobject{currentmarker}{}%
\end{pgfscope}%
\end{pgfscope}%
\begin{pgfscope}%
\pgfpathrectangle{\pgfqpoint{0.765000in}{0.660000in}}{\pgfqpoint{4.620000in}{4.620000in}}%
\pgfusepath{clip}%
\pgfsetroundcap%
\pgfsetroundjoin%
\pgfsetlinewidth{1.204500pt}%
\definecolor{currentstroke}{rgb}{0.019608,0.556863,0.850980}%
\pgfsetstrokecolor{currentstroke}%
\pgfsetdash{}{0pt}%
\pgfpathmoveto{\pgfqpoint{2.993281in}{3.235106in}}%
\pgfusepath{stroke}%
\end{pgfscope}%
\begin{pgfscope}%
\pgfpathrectangle{\pgfqpoint{0.765000in}{0.660000in}}{\pgfqpoint{4.620000in}{4.620000in}}%
\pgfusepath{clip}%
\pgfsetbuttcap%
\pgfsetroundjoin%
\definecolor{currentfill}{rgb}{0.019608,0.556863,0.850980}%
\pgfsetfillcolor{currentfill}%
\pgfsetlinewidth{1.003750pt}%
\definecolor{currentstroke}{rgb}{0.019608,0.556863,0.850980}%
\pgfsetstrokecolor{currentstroke}%
\pgfsetdash{}{0pt}%
\pgfsys@defobject{currentmarker}{\pgfqpoint{-0.033333in}{-0.033333in}}{\pgfqpoint{0.033333in}{0.033333in}}{%
\pgfpathmoveto{\pgfqpoint{-0.033333in}{0.000000in}}%
\pgfpathlineto{\pgfqpoint{0.033333in}{0.000000in}}%
\pgfpathmoveto{\pgfqpoint{0.000000in}{-0.033333in}}%
\pgfpathlineto{\pgfqpoint{0.000000in}{0.033333in}}%
\pgfusepath{stroke,fill}%
}%
\begin{pgfscope}%
\pgfsys@transformshift{2.993281in}{3.235106in}%
\pgfsys@useobject{currentmarker}{}%
\end{pgfscope}%
\end{pgfscope}%
\begin{pgfscope}%
\pgfpathrectangle{\pgfqpoint{0.765000in}{0.660000in}}{\pgfqpoint{4.620000in}{4.620000in}}%
\pgfusepath{clip}%
\pgfsetroundcap%
\pgfsetroundjoin%
\pgfsetlinewidth{1.204500pt}%
\definecolor{currentstroke}{rgb}{0.019608,0.556863,0.850980}%
\pgfsetstrokecolor{currentstroke}%
\pgfsetdash{}{0pt}%
\pgfpathmoveto{\pgfqpoint{2.359014in}{2.829758in}}%
\pgfusepath{stroke}%
\end{pgfscope}%
\begin{pgfscope}%
\pgfpathrectangle{\pgfqpoint{0.765000in}{0.660000in}}{\pgfqpoint{4.620000in}{4.620000in}}%
\pgfusepath{clip}%
\pgfsetbuttcap%
\pgfsetroundjoin%
\definecolor{currentfill}{rgb}{0.019608,0.556863,0.850980}%
\pgfsetfillcolor{currentfill}%
\pgfsetlinewidth{1.003750pt}%
\definecolor{currentstroke}{rgb}{0.019608,0.556863,0.850980}%
\pgfsetstrokecolor{currentstroke}%
\pgfsetdash{}{0pt}%
\pgfsys@defobject{currentmarker}{\pgfqpoint{-0.033333in}{-0.033333in}}{\pgfqpoint{0.033333in}{0.033333in}}{%
\pgfpathmoveto{\pgfqpoint{-0.033333in}{0.000000in}}%
\pgfpathlineto{\pgfqpoint{0.033333in}{0.000000in}}%
\pgfpathmoveto{\pgfqpoint{0.000000in}{-0.033333in}}%
\pgfpathlineto{\pgfqpoint{0.000000in}{0.033333in}}%
\pgfusepath{stroke,fill}%
}%
\begin{pgfscope}%
\pgfsys@transformshift{2.359014in}{2.829758in}%
\pgfsys@useobject{currentmarker}{}%
\end{pgfscope}%
\end{pgfscope}%
\begin{pgfscope}%
\pgfpathrectangle{\pgfqpoint{0.765000in}{0.660000in}}{\pgfqpoint{4.620000in}{4.620000in}}%
\pgfusepath{clip}%
\pgfsetroundcap%
\pgfsetroundjoin%
\pgfsetlinewidth{1.204500pt}%
\definecolor{currentstroke}{rgb}{0.019608,0.556863,0.850980}%
\pgfsetstrokecolor{currentstroke}%
\pgfsetdash{}{0pt}%
\pgfpathmoveto{\pgfqpoint{2.186033in}{2.910828in}}%
\pgfusepath{stroke}%
\end{pgfscope}%
\begin{pgfscope}%
\pgfpathrectangle{\pgfqpoint{0.765000in}{0.660000in}}{\pgfqpoint{4.620000in}{4.620000in}}%
\pgfusepath{clip}%
\pgfsetbuttcap%
\pgfsetroundjoin%
\definecolor{currentfill}{rgb}{0.019608,0.556863,0.850980}%
\pgfsetfillcolor{currentfill}%
\pgfsetlinewidth{1.003750pt}%
\definecolor{currentstroke}{rgb}{0.019608,0.556863,0.850980}%
\pgfsetstrokecolor{currentstroke}%
\pgfsetdash{}{0pt}%
\pgfsys@defobject{currentmarker}{\pgfqpoint{-0.033333in}{-0.033333in}}{\pgfqpoint{0.033333in}{0.033333in}}{%
\pgfpathmoveto{\pgfqpoint{-0.033333in}{0.000000in}}%
\pgfpathlineto{\pgfqpoint{0.033333in}{0.000000in}}%
\pgfpathmoveto{\pgfqpoint{0.000000in}{-0.033333in}}%
\pgfpathlineto{\pgfqpoint{0.000000in}{0.033333in}}%
\pgfusepath{stroke,fill}%
}%
\begin{pgfscope}%
\pgfsys@transformshift{2.186033in}{2.910828in}%
\pgfsys@useobject{currentmarker}{}%
\end{pgfscope}%
\end{pgfscope}%
\begin{pgfscope}%
\pgfpathrectangle{\pgfqpoint{0.765000in}{0.660000in}}{\pgfqpoint{4.620000in}{4.620000in}}%
\pgfusepath{clip}%
\pgfsetroundcap%
\pgfsetroundjoin%
\pgfsetlinewidth{1.204500pt}%
\definecolor{currentstroke}{rgb}{0.019608,0.556863,0.850980}%
\pgfsetstrokecolor{currentstroke}%
\pgfsetdash{}{0pt}%
\pgfpathmoveto{\pgfqpoint{2.128372in}{2.829758in}}%
\pgfusepath{stroke}%
\end{pgfscope}%
\begin{pgfscope}%
\pgfpathrectangle{\pgfqpoint{0.765000in}{0.660000in}}{\pgfqpoint{4.620000in}{4.620000in}}%
\pgfusepath{clip}%
\pgfsetbuttcap%
\pgfsetroundjoin%
\definecolor{currentfill}{rgb}{0.019608,0.556863,0.850980}%
\pgfsetfillcolor{currentfill}%
\pgfsetlinewidth{1.003750pt}%
\definecolor{currentstroke}{rgb}{0.019608,0.556863,0.850980}%
\pgfsetstrokecolor{currentstroke}%
\pgfsetdash{}{0pt}%
\pgfsys@defobject{currentmarker}{\pgfqpoint{-0.033333in}{-0.033333in}}{\pgfqpoint{0.033333in}{0.033333in}}{%
\pgfpathmoveto{\pgfqpoint{-0.033333in}{0.000000in}}%
\pgfpathlineto{\pgfqpoint{0.033333in}{0.000000in}}%
\pgfpathmoveto{\pgfqpoint{0.000000in}{-0.033333in}}%
\pgfpathlineto{\pgfqpoint{0.000000in}{0.033333in}}%
\pgfusepath{stroke,fill}%
}%
\begin{pgfscope}%
\pgfsys@transformshift{2.128372in}{2.829758in}%
\pgfsys@useobject{currentmarker}{}%
\end{pgfscope}%
\end{pgfscope}%
\begin{pgfscope}%
\pgfpathrectangle{\pgfqpoint{0.765000in}{0.660000in}}{\pgfqpoint{4.620000in}{4.620000in}}%
\pgfusepath{clip}%
\pgfsetroundcap%
\pgfsetroundjoin%
\pgfsetlinewidth{1.204500pt}%
\definecolor{currentstroke}{rgb}{0.019608,0.556863,0.850980}%
\pgfsetstrokecolor{currentstroke}%
\pgfsetdash{}{0pt}%
\pgfpathmoveto{\pgfqpoint{2.589657in}{2.991898in}}%
\pgfusepath{stroke}%
\end{pgfscope}%
\begin{pgfscope}%
\pgfpathrectangle{\pgfqpoint{0.765000in}{0.660000in}}{\pgfqpoint{4.620000in}{4.620000in}}%
\pgfusepath{clip}%
\pgfsetbuttcap%
\pgfsetroundjoin%
\definecolor{currentfill}{rgb}{0.019608,0.556863,0.850980}%
\pgfsetfillcolor{currentfill}%
\pgfsetlinewidth{1.003750pt}%
\definecolor{currentstroke}{rgb}{0.019608,0.556863,0.850980}%
\pgfsetstrokecolor{currentstroke}%
\pgfsetdash{}{0pt}%
\pgfsys@defobject{currentmarker}{\pgfqpoint{-0.033333in}{-0.033333in}}{\pgfqpoint{0.033333in}{0.033333in}}{%
\pgfpathmoveto{\pgfqpoint{-0.033333in}{0.000000in}}%
\pgfpathlineto{\pgfqpoint{0.033333in}{0.000000in}}%
\pgfpathmoveto{\pgfqpoint{0.000000in}{-0.033333in}}%
\pgfpathlineto{\pgfqpoint{0.000000in}{0.033333in}}%
\pgfusepath{stroke,fill}%
}%
\begin{pgfscope}%
\pgfsys@transformshift{2.589657in}{2.991898in}%
\pgfsys@useobject{currentmarker}{}%
\end{pgfscope}%
\end{pgfscope}%
\begin{pgfscope}%
\pgfpathrectangle{\pgfqpoint{0.765000in}{0.660000in}}{\pgfqpoint{4.620000in}{4.620000in}}%
\pgfusepath{clip}%
\pgfsetroundcap%
\pgfsetroundjoin%
\pgfsetlinewidth{1.204500pt}%
\definecolor{currentstroke}{rgb}{0.019608,0.556863,0.850980}%
\pgfsetstrokecolor{currentstroke}%
\pgfsetdash{}{0pt}%
\pgfpathmoveto{\pgfqpoint{2.820299in}{3.316176in}}%
\pgfusepath{stroke}%
\end{pgfscope}%
\begin{pgfscope}%
\pgfpathrectangle{\pgfqpoint{0.765000in}{0.660000in}}{\pgfqpoint{4.620000in}{4.620000in}}%
\pgfusepath{clip}%
\pgfsetbuttcap%
\pgfsetroundjoin%
\definecolor{currentfill}{rgb}{0.019608,0.556863,0.850980}%
\pgfsetfillcolor{currentfill}%
\pgfsetlinewidth{1.003750pt}%
\definecolor{currentstroke}{rgb}{0.019608,0.556863,0.850980}%
\pgfsetstrokecolor{currentstroke}%
\pgfsetdash{}{0pt}%
\pgfsys@defobject{currentmarker}{\pgfqpoint{-0.033333in}{-0.033333in}}{\pgfqpoint{0.033333in}{0.033333in}}{%
\pgfpathmoveto{\pgfqpoint{-0.033333in}{0.000000in}}%
\pgfpathlineto{\pgfqpoint{0.033333in}{0.000000in}}%
\pgfpathmoveto{\pgfqpoint{0.000000in}{-0.033333in}}%
\pgfpathlineto{\pgfqpoint{0.000000in}{0.033333in}}%
\pgfusepath{stroke,fill}%
}%
\begin{pgfscope}%
\pgfsys@transformshift{2.820299in}{3.316176in}%
\pgfsys@useobject{currentmarker}{}%
\end{pgfscope}%
\end{pgfscope}%
\begin{pgfscope}%
\pgfpathrectangle{\pgfqpoint{0.765000in}{0.660000in}}{\pgfqpoint{4.620000in}{4.620000in}}%
\pgfusepath{clip}%
\pgfsetroundcap%
\pgfsetroundjoin%
\pgfsetlinewidth{1.204500pt}%
\definecolor{currentstroke}{rgb}{0.019608,0.556863,0.850980}%
\pgfsetstrokecolor{currentstroke}%
\pgfsetdash{}{0pt}%
\pgfpathmoveto{\pgfqpoint{3.108602in}{3.235106in}}%
\pgfusepath{stroke}%
\end{pgfscope}%
\begin{pgfscope}%
\pgfpathrectangle{\pgfqpoint{0.765000in}{0.660000in}}{\pgfqpoint{4.620000in}{4.620000in}}%
\pgfusepath{clip}%
\pgfsetbuttcap%
\pgfsetroundjoin%
\definecolor{currentfill}{rgb}{0.019608,0.556863,0.850980}%
\pgfsetfillcolor{currentfill}%
\pgfsetlinewidth{1.003750pt}%
\definecolor{currentstroke}{rgb}{0.019608,0.556863,0.850980}%
\pgfsetstrokecolor{currentstroke}%
\pgfsetdash{}{0pt}%
\pgfsys@defobject{currentmarker}{\pgfqpoint{-0.033333in}{-0.033333in}}{\pgfqpoint{0.033333in}{0.033333in}}{%
\pgfpathmoveto{\pgfqpoint{-0.033333in}{0.000000in}}%
\pgfpathlineto{\pgfqpoint{0.033333in}{0.000000in}}%
\pgfpathmoveto{\pgfqpoint{0.000000in}{-0.033333in}}%
\pgfpathlineto{\pgfqpoint{0.000000in}{0.033333in}}%
\pgfusepath{stroke,fill}%
}%
\begin{pgfscope}%
\pgfsys@transformshift{3.108602in}{3.235106in}%
\pgfsys@useobject{currentmarker}{}%
\end{pgfscope}%
\end{pgfscope}%
\begin{pgfscope}%
\pgfpathrectangle{\pgfqpoint{0.765000in}{0.660000in}}{\pgfqpoint{4.620000in}{4.620000in}}%
\pgfusepath{clip}%
\pgfsetroundcap%
\pgfsetroundjoin%
\pgfsetlinewidth{1.204500pt}%
\definecolor{currentstroke}{rgb}{0.019608,0.556863,0.850980}%
\pgfsetstrokecolor{currentstroke}%
\pgfsetdash{}{0pt}%
\pgfpathmoveto{\pgfqpoint{3.627548in}{3.316176in}}%
\pgfusepath{stroke}%
\end{pgfscope}%
\begin{pgfscope}%
\pgfpathrectangle{\pgfqpoint{0.765000in}{0.660000in}}{\pgfqpoint{4.620000in}{4.620000in}}%
\pgfusepath{clip}%
\pgfsetbuttcap%
\pgfsetroundjoin%
\definecolor{currentfill}{rgb}{0.019608,0.556863,0.850980}%
\pgfsetfillcolor{currentfill}%
\pgfsetlinewidth{1.003750pt}%
\definecolor{currentstroke}{rgb}{0.019608,0.556863,0.850980}%
\pgfsetstrokecolor{currentstroke}%
\pgfsetdash{}{0pt}%
\pgfsys@defobject{currentmarker}{\pgfqpoint{-0.033333in}{-0.033333in}}{\pgfqpoint{0.033333in}{0.033333in}}{%
\pgfpathmoveto{\pgfqpoint{-0.033333in}{0.000000in}}%
\pgfpathlineto{\pgfqpoint{0.033333in}{0.000000in}}%
\pgfpathmoveto{\pgfqpoint{0.000000in}{-0.033333in}}%
\pgfpathlineto{\pgfqpoint{0.000000in}{0.033333in}}%
\pgfusepath{stroke,fill}%
}%
\begin{pgfscope}%
\pgfsys@transformshift{3.627548in}{3.316176in}%
\pgfsys@useobject{currentmarker}{}%
\end{pgfscope}%
\end{pgfscope}%
\begin{pgfscope}%
\pgfpathrectangle{\pgfqpoint{0.765000in}{0.660000in}}{\pgfqpoint{4.620000in}{4.620000in}}%
\pgfusepath{clip}%
\pgfsetroundcap%
\pgfsetroundjoin%
\pgfsetlinewidth{1.204500pt}%
\definecolor{currentstroke}{rgb}{0.019608,0.556863,0.850980}%
\pgfsetstrokecolor{currentstroke}%
\pgfsetdash{}{0pt}%
\pgfpathmoveto{\pgfqpoint{3.223923in}{3.235106in}}%
\pgfusepath{stroke}%
\end{pgfscope}%
\begin{pgfscope}%
\pgfpathrectangle{\pgfqpoint{0.765000in}{0.660000in}}{\pgfqpoint{4.620000in}{4.620000in}}%
\pgfusepath{clip}%
\pgfsetbuttcap%
\pgfsetroundjoin%
\definecolor{currentfill}{rgb}{0.019608,0.556863,0.850980}%
\pgfsetfillcolor{currentfill}%
\pgfsetlinewidth{1.003750pt}%
\definecolor{currentstroke}{rgb}{0.019608,0.556863,0.850980}%
\pgfsetstrokecolor{currentstroke}%
\pgfsetdash{}{0pt}%
\pgfsys@defobject{currentmarker}{\pgfqpoint{-0.033333in}{-0.033333in}}{\pgfqpoint{0.033333in}{0.033333in}}{%
\pgfpathmoveto{\pgfqpoint{-0.033333in}{0.000000in}}%
\pgfpathlineto{\pgfqpoint{0.033333in}{0.000000in}}%
\pgfpathmoveto{\pgfqpoint{0.000000in}{-0.033333in}}%
\pgfpathlineto{\pgfqpoint{0.000000in}{0.033333in}}%
\pgfusepath{stroke,fill}%
}%
\begin{pgfscope}%
\pgfsys@transformshift{3.223923in}{3.235106in}%
\pgfsys@useobject{currentmarker}{}%
\end{pgfscope}%
\end{pgfscope}%
\begin{pgfscope}%
\pgfpathrectangle{\pgfqpoint{0.765000in}{0.660000in}}{\pgfqpoint{4.620000in}{4.620000in}}%
\pgfusepath{clip}%
\pgfsetroundcap%
\pgfsetroundjoin%
\pgfsetlinewidth{1.204500pt}%
\definecolor{currentstroke}{rgb}{0.019608,0.556863,0.850980}%
\pgfsetstrokecolor{currentstroke}%
\pgfsetdash{}{0pt}%
\pgfpathmoveto{\pgfqpoint{2.186033in}{3.072967in}}%
\pgfusepath{stroke}%
\end{pgfscope}%
\begin{pgfscope}%
\pgfpathrectangle{\pgfqpoint{0.765000in}{0.660000in}}{\pgfqpoint{4.620000in}{4.620000in}}%
\pgfusepath{clip}%
\pgfsetbuttcap%
\pgfsetroundjoin%
\definecolor{currentfill}{rgb}{0.019608,0.556863,0.850980}%
\pgfsetfillcolor{currentfill}%
\pgfsetlinewidth{1.003750pt}%
\definecolor{currentstroke}{rgb}{0.019608,0.556863,0.850980}%
\pgfsetstrokecolor{currentstroke}%
\pgfsetdash{}{0pt}%
\pgfsys@defobject{currentmarker}{\pgfqpoint{-0.033333in}{-0.033333in}}{\pgfqpoint{0.033333in}{0.033333in}}{%
\pgfpathmoveto{\pgfqpoint{-0.033333in}{0.000000in}}%
\pgfpathlineto{\pgfqpoint{0.033333in}{0.000000in}}%
\pgfpathmoveto{\pgfqpoint{0.000000in}{-0.033333in}}%
\pgfpathlineto{\pgfqpoint{0.000000in}{0.033333in}}%
\pgfusepath{stroke,fill}%
}%
\begin{pgfscope}%
\pgfsys@transformshift{2.186033in}{3.072967in}%
\pgfsys@useobject{currentmarker}{}%
\end{pgfscope}%
\end{pgfscope}%
\begin{pgfscope}%
\pgfpathrectangle{\pgfqpoint{0.765000in}{0.660000in}}{\pgfqpoint{4.620000in}{4.620000in}}%
\pgfusepath{clip}%
\pgfsetroundcap%
\pgfsetroundjoin%
\pgfsetlinewidth{1.204500pt}%
\definecolor{currentstroke}{rgb}{0.019608,0.556863,0.850980}%
\pgfsetstrokecolor{currentstroke}%
\pgfsetdash{}{0pt}%
\pgfpathmoveto{\pgfqpoint{2.993281in}{3.072967in}}%
\pgfusepath{stroke}%
\end{pgfscope}%
\begin{pgfscope}%
\pgfpathrectangle{\pgfqpoint{0.765000in}{0.660000in}}{\pgfqpoint{4.620000in}{4.620000in}}%
\pgfusepath{clip}%
\pgfsetbuttcap%
\pgfsetroundjoin%
\definecolor{currentfill}{rgb}{0.019608,0.556863,0.850980}%
\pgfsetfillcolor{currentfill}%
\pgfsetlinewidth{1.003750pt}%
\definecolor{currentstroke}{rgb}{0.019608,0.556863,0.850980}%
\pgfsetstrokecolor{currentstroke}%
\pgfsetdash{}{0pt}%
\pgfsys@defobject{currentmarker}{\pgfqpoint{-0.033333in}{-0.033333in}}{\pgfqpoint{0.033333in}{0.033333in}}{%
\pgfpathmoveto{\pgfqpoint{-0.033333in}{0.000000in}}%
\pgfpathlineto{\pgfqpoint{0.033333in}{0.000000in}}%
\pgfpathmoveto{\pgfqpoint{0.000000in}{-0.033333in}}%
\pgfpathlineto{\pgfqpoint{0.000000in}{0.033333in}}%
\pgfusepath{stroke,fill}%
}%
\begin{pgfscope}%
\pgfsys@transformshift{2.993281in}{3.072967in}%
\pgfsys@useobject{currentmarker}{}%
\end{pgfscope}%
\end{pgfscope}%
\begin{pgfscope}%
\pgfpathrectangle{\pgfqpoint{0.765000in}{0.660000in}}{\pgfqpoint{4.620000in}{4.620000in}}%
\pgfusepath{clip}%
\pgfsetroundcap%
\pgfsetroundjoin%
\pgfsetlinewidth{1.204500pt}%
\definecolor{currentstroke}{rgb}{0.019608,0.556863,0.850980}%
\pgfsetstrokecolor{currentstroke}%
\pgfsetdash{}{0pt}%
\pgfpathmoveto{\pgfqpoint{2.416675in}{3.072967in}}%
\pgfusepath{stroke}%
\end{pgfscope}%
\begin{pgfscope}%
\pgfpathrectangle{\pgfqpoint{0.765000in}{0.660000in}}{\pgfqpoint{4.620000in}{4.620000in}}%
\pgfusepath{clip}%
\pgfsetbuttcap%
\pgfsetroundjoin%
\definecolor{currentfill}{rgb}{0.019608,0.556863,0.850980}%
\pgfsetfillcolor{currentfill}%
\pgfsetlinewidth{1.003750pt}%
\definecolor{currentstroke}{rgb}{0.019608,0.556863,0.850980}%
\pgfsetstrokecolor{currentstroke}%
\pgfsetdash{}{0pt}%
\pgfsys@defobject{currentmarker}{\pgfqpoint{-0.033333in}{-0.033333in}}{\pgfqpoint{0.033333in}{0.033333in}}{%
\pgfpathmoveto{\pgfqpoint{-0.033333in}{0.000000in}}%
\pgfpathlineto{\pgfqpoint{0.033333in}{0.000000in}}%
\pgfpathmoveto{\pgfqpoint{0.000000in}{-0.033333in}}%
\pgfpathlineto{\pgfqpoint{0.000000in}{0.033333in}}%
\pgfusepath{stroke,fill}%
}%
\begin{pgfscope}%
\pgfsys@transformshift{2.416675in}{3.072967in}%
\pgfsys@useobject{currentmarker}{}%
\end{pgfscope}%
\end{pgfscope}%
\begin{pgfscope}%
\pgfpathrectangle{\pgfqpoint{0.765000in}{0.660000in}}{\pgfqpoint{4.620000in}{4.620000in}}%
\pgfusepath{clip}%
\pgfsetroundcap%
\pgfsetroundjoin%
\pgfsetlinewidth{1.204500pt}%
\definecolor{currentstroke}{rgb}{0.019608,0.556863,0.850980}%
\pgfsetstrokecolor{currentstroke}%
\pgfsetdash{}{0pt}%
\pgfpathmoveto{\pgfqpoint{2.474336in}{2.991898in}}%
\pgfusepath{stroke}%
\end{pgfscope}%
\begin{pgfscope}%
\pgfpathrectangle{\pgfqpoint{0.765000in}{0.660000in}}{\pgfqpoint{4.620000in}{4.620000in}}%
\pgfusepath{clip}%
\pgfsetbuttcap%
\pgfsetroundjoin%
\definecolor{currentfill}{rgb}{0.019608,0.556863,0.850980}%
\pgfsetfillcolor{currentfill}%
\pgfsetlinewidth{1.003750pt}%
\definecolor{currentstroke}{rgb}{0.019608,0.556863,0.850980}%
\pgfsetstrokecolor{currentstroke}%
\pgfsetdash{}{0pt}%
\pgfsys@defobject{currentmarker}{\pgfqpoint{-0.033333in}{-0.033333in}}{\pgfqpoint{0.033333in}{0.033333in}}{%
\pgfpathmoveto{\pgfqpoint{-0.033333in}{0.000000in}}%
\pgfpathlineto{\pgfqpoint{0.033333in}{0.000000in}}%
\pgfpathmoveto{\pgfqpoint{0.000000in}{-0.033333in}}%
\pgfpathlineto{\pgfqpoint{0.000000in}{0.033333in}}%
\pgfusepath{stroke,fill}%
}%
\begin{pgfscope}%
\pgfsys@transformshift{2.474336in}{2.991898in}%
\pgfsys@useobject{currentmarker}{}%
\end{pgfscope}%
\end{pgfscope}%
\begin{pgfscope}%
\pgfpathrectangle{\pgfqpoint{0.765000in}{0.660000in}}{\pgfqpoint{4.620000in}{4.620000in}}%
\pgfusepath{clip}%
\pgfsetroundcap%
\pgfsetroundjoin%
\pgfsetlinewidth{1.204500pt}%
\definecolor{currentstroke}{rgb}{0.019608,0.556863,0.850980}%
\pgfsetstrokecolor{currentstroke}%
\pgfsetdash{}{0pt}%
\pgfpathmoveto{\pgfqpoint{3.050942in}{3.154037in}}%
\pgfusepath{stroke}%
\end{pgfscope}%
\begin{pgfscope}%
\pgfpathrectangle{\pgfqpoint{0.765000in}{0.660000in}}{\pgfqpoint{4.620000in}{4.620000in}}%
\pgfusepath{clip}%
\pgfsetbuttcap%
\pgfsetroundjoin%
\definecolor{currentfill}{rgb}{0.019608,0.556863,0.850980}%
\pgfsetfillcolor{currentfill}%
\pgfsetlinewidth{1.003750pt}%
\definecolor{currentstroke}{rgb}{0.019608,0.556863,0.850980}%
\pgfsetstrokecolor{currentstroke}%
\pgfsetdash{}{0pt}%
\pgfsys@defobject{currentmarker}{\pgfqpoint{-0.033333in}{-0.033333in}}{\pgfqpoint{0.033333in}{0.033333in}}{%
\pgfpathmoveto{\pgfqpoint{-0.033333in}{0.000000in}}%
\pgfpathlineto{\pgfqpoint{0.033333in}{0.000000in}}%
\pgfpathmoveto{\pgfqpoint{0.000000in}{-0.033333in}}%
\pgfpathlineto{\pgfqpoint{0.000000in}{0.033333in}}%
\pgfusepath{stroke,fill}%
}%
\begin{pgfscope}%
\pgfsys@transformshift{3.050942in}{3.154037in}%
\pgfsys@useobject{currentmarker}{}%
\end{pgfscope}%
\end{pgfscope}%
\begin{pgfscope}%
\pgfpathrectangle{\pgfqpoint{0.765000in}{0.660000in}}{\pgfqpoint{4.620000in}{4.620000in}}%
\pgfusepath{clip}%
\pgfsetroundcap%
\pgfsetroundjoin%
\pgfsetlinewidth{1.204500pt}%
\definecolor{currentstroke}{rgb}{0.019608,0.556863,0.850980}%
\pgfsetstrokecolor{currentstroke}%
\pgfsetdash{}{0pt}%
\pgfpathmoveto{\pgfqpoint{2.474336in}{2.991898in}}%
\pgfusepath{stroke}%
\end{pgfscope}%
\begin{pgfscope}%
\pgfpathrectangle{\pgfqpoint{0.765000in}{0.660000in}}{\pgfqpoint{4.620000in}{4.620000in}}%
\pgfusepath{clip}%
\pgfsetbuttcap%
\pgfsetroundjoin%
\definecolor{currentfill}{rgb}{0.019608,0.556863,0.850980}%
\pgfsetfillcolor{currentfill}%
\pgfsetlinewidth{1.003750pt}%
\definecolor{currentstroke}{rgb}{0.019608,0.556863,0.850980}%
\pgfsetstrokecolor{currentstroke}%
\pgfsetdash{}{0pt}%
\pgfsys@defobject{currentmarker}{\pgfqpoint{-0.033333in}{-0.033333in}}{\pgfqpoint{0.033333in}{0.033333in}}{%
\pgfpathmoveto{\pgfqpoint{-0.033333in}{0.000000in}}%
\pgfpathlineto{\pgfqpoint{0.033333in}{0.000000in}}%
\pgfpathmoveto{\pgfqpoint{0.000000in}{-0.033333in}}%
\pgfpathlineto{\pgfqpoint{0.000000in}{0.033333in}}%
\pgfusepath{stroke,fill}%
}%
\begin{pgfscope}%
\pgfsys@transformshift{2.474336in}{2.991898in}%
\pgfsys@useobject{currentmarker}{}%
\end{pgfscope}%
\end{pgfscope}%
\begin{pgfscope}%
\pgfpathrectangle{\pgfqpoint{0.765000in}{0.660000in}}{\pgfqpoint{4.620000in}{4.620000in}}%
\pgfusepath{clip}%
\pgfsetroundcap%
\pgfsetroundjoin%
\pgfsetlinewidth{1.204500pt}%
\definecolor{currentstroke}{rgb}{0.019608,0.556863,0.850980}%
\pgfsetstrokecolor{currentstroke}%
\pgfsetdash{}{0pt}%
\pgfpathmoveto{\pgfqpoint{2.013051in}{2.829758in}}%
\pgfusepath{stroke}%
\end{pgfscope}%
\begin{pgfscope}%
\pgfpathrectangle{\pgfqpoint{0.765000in}{0.660000in}}{\pgfqpoint{4.620000in}{4.620000in}}%
\pgfusepath{clip}%
\pgfsetbuttcap%
\pgfsetroundjoin%
\definecolor{currentfill}{rgb}{0.019608,0.556863,0.850980}%
\pgfsetfillcolor{currentfill}%
\pgfsetlinewidth{1.003750pt}%
\definecolor{currentstroke}{rgb}{0.019608,0.556863,0.850980}%
\pgfsetstrokecolor{currentstroke}%
\pgfsetdash{}{0pt}%
\pgfsys@defobject{currentmarker}{\pgfqpoint{-0.033333in}{-0.033333in}}{\pgfqpoint{0.033333in}{0.033333in}}{%
\pgfpathmoveto{\pgfqpoint{-0.033333in}{0.000000in}}%
\pgfpathlineto{\pgfqpoint{0.033333in}{0.000000in}}%
\pgfpathmoveto{\pgfqpoint{0.000000in}{-0.033333in}}%
\pgfpathlineto{\pgfqpoint{0.000000in}{0.033333in}}%
\pgfusepath{stroke,fill}%
}%
\begin{pgfscope}%
\pgfsys@transformshift{2.013051in}{2.829758in}%
\pgfsys@useobject{currentmarker}{}%
\end{pgfscope}%
\end{pgfscope}%
\begin{pgfscope}%
\pgfpathrectangle{\pgfqpoint{0.765000in}{0.660000in}}{\pgfqpoint{4.620000in}{4.620000in}}%
\pgfusepath{clip}%
\pgfsetroundcap%
\pgfsetroundjoin%
\pgfsetlinewidth{1.204500pt}%
\definecolor{currentstroke}{rgb}{0.019608,0.556863,0.850980}%
\pgfsetstrokecolor{currentstroke}%
\pgfsetdash{}{0pt}%
\pgfpathmoveto{\pgfqpoint{2.647317in}{3.072967in}}%
\pgfusepath{stroke}%
\end{pgfscope}%
\begin{pgfscope}%
\pgfpathrectangle{\pgfqpoint{0.765000in}{0.660000in}}{\pgfqpoint{4.620000in}{4.620000in}}%
\pgfusepath{clip}%
\pgfsetbuttcap%
\pgfsetroundjoin%
\definecolor{currentfill}{rgb}{0.019608,0.556863,0.850980}%
\pgfsetfillcolor{currentfill}%
\pgfsetlinewidth{1.003750pt}%
\definecolor{currentstroke}{rgb}{0.019608,0.556863,0.850980}%
\pgfsetstrokecolor{currentstroke}%
\pgfsetdash{}{0pt}%
\pgfsys@defobject{currentmarker}{\pgfqpoint{-0.033333in}{-0.033333in}}{\pgfqpoint{0.033333in}{0.033333in}}{%
\pgfpathmoveto{\pgfqpoint{-0.033333in}{0.000000in}}%
\pgfpathlineto{\pgfqpoint{0.033333in}{0.000000in}}%
\pgfpathmoveto{\pgfqpoint{0.000000in}{-0.033333in}}%
\pgfpathlineto{\pgfqpoint{0.000000in}{0.033333in}}%
\pgfusepath{stroke,fill}%
}%
\begin{pgfscope}%
\pgfsys@transformshift{2.647317in}{3.072967in}%
\pgfsys@useobject{currentmarker}{}%
\end{pgfscope}%
\end{pgfscope}%
\begin{pgfscope}%
\pgfpathrectangle{\pgfqpoint{0.765000in}{0.660000in}}{\pgfqpoint{4.620000in}{4.620000in}}%
\pgfusepath{clip}%
\pgfsetroundcap%
\pgfsetroundjoin%
\pgfsetlinewidth{1.204500pt}%
\definecolor{currentstroke}{rgb}{0.019608,0.556863,0.850980}%
\pgfsetstrokecolor{currentstroke}%
\pgfsetdash{}{0pt}%
\pgfpathmoveto{\pgfqpoint{2.935620in}{2.991898in}}%
\pgfusepath{stroke}%
\end{pgfscope}%
\begin{pgfscope}%
\pgfpathrectangle{\pgfqpoint{0.765000in}{0.660000in}}{\pgfqpoint{4.620000in}{4.620000in}}%
\pgfusepath{clip}%
\pgfsetbuttcap%
\pgfsetroundjoin%
\definecolor{currentfill}{rgb}{0.019608,0.556863,0.850980}%
\pgfsetfillcolor{currentfill}%
\pgfsetlinewidth{1.003750pt}%
\definecolor{currentstroke}{rgb}{0.019608,0.556863,0.850980}%
\pgfsetstrokecolor{currentstroke}%
\pgfsetdash{}{0pt}%
\pgfsys@defobject{currentmarker}{\pgfqpoint{-0.033333in}{-0.033333in}}{\pgfqpoint{0.033333in}{0.033333in}}{%
\pgfpathmoveto{\pgfqpoint{-0.033333in}{0.000000in}}%
\pgfpathlineto{\pgfqpoint{0.033333in}{0.000000in}}%
\pgfpathmoveto{\pgfqpoint{0.000000in}{-0.033333in}}%
\pgfpathlineto{\pgfqpoint{0.000000in}{0.033333in}}%
\pgfusepath{stroke,fill}%
}%
\begin{pgfscope}%
\pgfsys@transformshift{2.935620in}{2.991898in}%
\pgfsys@useobject{currentmarker}{}%
\end{pgfscope}%
\end{pgfscope}%
\begin{pgfscope}%
\pgfpathrectangle{\pgfqpoint{0.765000in}{0.660000in}}{\pgfqpoint{4.620000in}{4.620000in}}%
\pgfusepath{clip}%
\pgfsetroundcap%
\pgfsetroundjoin%
\pgfsetlinewidth{1.204500pt}%
\definecolor{currentstroke}{rgb}{0.019608,0.556863,0.850980}%
\pgfsetstrokecolor{currentstroke}%
\pgfsetdash{}{0pt}%
\pgfpathmoveto{\pgfqpoint{2.877960in}{3.072967in}}%
\pgfusepath{stroke}%
\end{pgfscope}%
\begin{pgfscope}%
\pgfpathrectangle{\pgfqpoint{0.765000in}{0.660000in}}{\pgfqpoint{4.620000in}{4.620000in}}%
\pgfusepath{clip}%
\pgfsetbuttcap%
\pgfsetroundjoin%
\definecolor{currentfill}{rgb}{0.019608,0.556863,0.850980}%
\pgfsetfillcolor{currentfill}%
\pgfsetlinewidth{1.003750pt}%
\definecolor{currentstroke}{rgb}{0.019608,0.556863,0.850980}%
\pgfsetstrokecolor{currentstroke}%
\pgfsetdash{}{0pt}%
\pgfsys@defobject{currentmarker}{\pgfqpoint{-0.033333in}{-0.033333in}}{\pgfqpoint{0.033333in}{0.033333in}}{%
\pgfpathmoveto{\pgfqpoint{-0.033333in}{0.000000in}}%
\pgfpathlineto{\pgfqpoint{0.033333in}{0.000000in}}%
\pgfpathmoveto{\pgfqpoint{0.000000in}{-0.033333in}}%
\pgfpathlineto{\pgfqpoint{0.000000in}{0.033333in}}%
\pgfusepath{stroke,fill}%
}%
\begin{pgfscope}%
\pgfsys@transformshift{2.877960in}{3.072967in}%
\pgfsys@useobject{currentmarker}{}%
\end{pgfscope}%
\end{pgfscope}%
\begin{pgfscope}%
\pgfpathrectangle{\pgfqpoint{0.765000in}{0.660000in}}{\pgfqpoint{4.620000in}{4.620000in}}%
\pgfusepath{clip}%
\pgfsetroundcap%
\pgfsetroundjoin%
\pgfsetlinewidth{1.204500pt}%
\definecolor{currentstroke}{rgb}{0.019608,0.556863,0.850980}%
\pgfsetstrokecolor{currentstroke}%
\pgfsetdash{}{0pt}%
\pgfpathmoveto{\pgfqpoint{2.877960in}{3.072967in}}%
\pgfusepath{stroke}%
\end{pgfscope}%
\begin{pgfscope}%
\pgfpathrectangle{\pgfqpoint{0.765000in}{0.660000in}}{\pgfqpoint{4.620000in}{4.620000in}}%
\pgfusepath{clip}%
\pgfsetbuttcap%
\pgfsetroundjoin%
\definecolor{currentfill}{rgb}{0.019608,0.556863,0.850980}%
\pgfsetfillcolor{currentfill}%
\pgfsetlinewidth{1.003750pt}%
\definecolor{currentstroke}{rgb}{0.019608,0.556863,0.850980}%
\pgfsetstrokecolor{currentstroke}%
\pgfsetdash{}{0pt}%
\pgfsys@defobject{currentmarker}{\pgfqpoint{-0.033333in}{-0.033333in}}{\pgfqpoint{0.033333in}{0.033333in}}{%
\pgfpathmoveto{\pgfqpoint{-0.033333in}{0.000000in}}%
\pgfpathlineto{\pgfqpoint{0.033333in}{0.000000in}}%
\pgfpathmoveto{\pgfqpoint{0.000000in}{-0.033333in}}%
\pgfpathlineto{\pgfqpoint{0.000000in}{0.033333in}}%
\pgfusepath{stroke,fill}%
}%
\begin{pgfscope}%
\pgfsys@transformshift{2.877960in}{3.072967in}%
\pgfsys@useobject{currentmarker}{}%
\end{pgfscope}%
\end{pgfscope}%
\begin{pgfscope}%
\pgfpathrectangle{\pgfqpoint{0.765000in}{0.660000in}}{\pgfqpoint{4.620000in}{4.620000in}}%
\pgfusepath{clip}%
\pgfsetroundcap%
\pgfsetroundjoin%
\pgfsetlinewidth{1.204500pt}%
\definecolor{currentstroke}{rgb}{0.019608,0.556863,0.850980}%
\pgfsetstrokecolor{currentstroke}%
\pgfsetdash{}{0pt}%
\pgfpathmoveto{\pgfqpoint{2.301354in}{2.910828in}}%
\pgfusepath{stroke}%
\end{pgfscope}%
\begin{pgfscope}%
\pgfpathrectangle{\pgfqpoint{0.765000in}{0.660000in}}{\pgfqpoint{4.620000in}{4.620000in}}%
\pgfusepath{clip}%
\pgfsetbuttcap%
\pgfsetroundjoin%
\definecolor{currentfill}{rgb}{0.019608,0.556863,0.850980}%
\pgfsetfillcolor{currentfill}%
\pgfsetlinewidth{1.003750pt}%
\definecolor{currentstroke}{rgb}{0.019608,0.556863,0.850980}%
\pgfsetstrokecolor{currentstroke}%
\pgfsetdash{}{0pt}%
\pgfsys@defobject{currentmarker}{\pgfqpoint{-0.033333in}{-0.033333in}}{\pgfqpoint{0.033333in}{0.033333in}}{%
\pgfpathmoveto{\pgfqpoint{-0.033333in}{0.000000in}}%
\pgfpathlineto{\pgfqpoint{0.033333in}{0.000000in}}%
\pgfpathmoveto{\pgfqpoint{0.000000in}{-0.033333in}}%
\pgfpathlineto{\pgfqpoint{0.000000in}{0.033333in}}%
\pgfusepath{stroke,fill}%
}%
\begin{pgfscope}%
\pgfsys@transformshift{2.301354in}{2.910828in}%
\pgfsys@useobject{currentmarker}{}%
\end{pgfscope}%
\end{pgfscope}%
\begin{pgfscope}%
\pgfpathrectangle{\pgfqpoint{0.765000in}{0.660000in}}{\pgfqpoint{4.620000in}{4.620000in}}%
\pgfusepath{clip}%
\pgfsetroundcap%
\pgfsetroundjoin%
\pgfsetlinewidth{1.204500pt}%
\definecolor{currentstroke}{rgb}{0.019608,0.556863,0.850980}%
\pgfsetstrokecolor{currentstroke}%
\pgfsetdash{}{0pt}%
\pgfpathmoveto{\pgfqpoint{2.762639in}{3.072967in}}%
\pgfusepath{stroke}%
\end{pgfscope}%
\begin{pgfscope}%
\pgfpathrectangle{\pgfqpoint{0.765000in}{0.660000in}}{\pgfqpoint{4.620000in}{4.620000in}}%
\pgfusepath{clip}%
\pgfsetbuttcap%
\pgfsetroundjoin%
\definecolor{currentfill}{rgb}{0.019608,0.556863,0.850980}%
\pgfsetfillcolor{currentfill}%
\pgfsetlinewidth{1.003750pt}%
\definecolor{currentstroke}{rgb}{0.019608,0.556863,0.850980}%
\pgfsetstrokecolor{currentstroke}%
\pgfsetdash{}{0pt}%
\pgfsys@defobject{currentmarker}{\pgfqpoint{-0.033333in}{-0.033333in}}{\pgfqpoint{0.033333in}{0.033333in}}{%
\pgfpathmoveto{\pgfqpoint{-0.033333in}{0.000000in}}%
\pgfpathlineto{\pgfqpoint{0.033333in}{0.000000in}}%
\pgfpathmoveto{\pgfqpoint{0.000000in}{-0.033333in}}%
\pgfpathlineto{\pgfqpoint{0.000000in}{0.033333in}}%
\pgfusepath{stroke,fill}%
}%
\begin{pgfscope}%
\pgfsys@transformshift{2.762639in}{3.072967in}%
\pgfsys@useobject{currentmarker}{}%
\end{pgfscope}%
\end{pgfscope}%
\begin{pgfscope}%
\pgfpathrectangle{\pgfqpoint{0.765000in}{0.660000in}}{\pgfqpoint{4.620000in}{4.620000in}}%
\pgfusepath{clip}%
\pgfsetroundcap%
\pgfsetroundjoin%
\pgfsetlinewidth{1.204500pt}%
\definecolor{currentstroke}{rgb}{0.000000,0.000000,0.000000}%
\pgfsetstrokecolor{currentstroke}%
\pgfsetdash{}{0pt}%
\pgfpathmoveto{\pgfqpoint{2.288260in}{2.879821in}}%
\pgfusepath{stroke}%
\end{pgfscope}%
\begin{pgfscope}%
\pgfpathrectangle{\pgfqpoint{0.765000in}{0.660000in}}{\pgfqpoint{4.620000in}{4.620000in}}%
\pgfusepath{clip}%
\pgfsetbuttcap%
\pgfsetroundjoin%
\definecolor{currentfill}{rgb}{0.000000,0.000000,0.000000}%
\pgfsetfillcolor{currentfill}%
\pgfsetlinewidth{1.003750pt}%
\definecolor{currentstroke}{rgb}{0.000000,0.000000,0.000000}%
\pgfsetstrokecolor{currentstroke}%
\pgfsetdash{}{0pt}%
\pgfsys@defobject{currentmarker}{\pgfqpoint{-0.016667in}{-0.016667in}}{\pgfqpoint{0.016667in}{0.016667in}}{%
\pgfpathmoveto{\pgfqpoint{0.000000in}{-0.016667in}}%
\pgfpathcurveto{\pgfqpoint{0.004420in}{-0.016667in}}{\pgfqpoint{0.008660in}{-0.014911in}}{\pgfqpoint{0.011785in}{-0.011785in}}%
\pgfpathcurveto{\pgfqpoint{0.014911in}{-0.008660in}}{\pgfqpoint{0.016667in}{-0.004420in}}{\pgfqpoint{0.016667in}{0.000000in}}%
\pgfpathcurveto{\pgfqpoint{0.016667in}{0.004420in}}{\pgfqpoint{0.014911in}{0.008660in}}{\pgfqpoint{0.011785in}{0.011785in}}%
\pgfpathcurveto{\pgfqpoint{0.008660in}{0.014911in}}{\pgfqpoint{0.004420in}{0.016667in}}{\pgfqpoint{0.000000in}{0.016667in}}%
\pgfpathcurveto{\pgfqpoint{-0.004420in}{0.016667in}}{\pgfqpoint{-0.008660in}{0.014911in}}{\pgfqpoint{-0.011785in}{0.011785in}}%
\pgfpathcurveto{\pgfqpoint{-0.014911in}{0.008660in}}{\pgfqpoint{-0.016667in}{0.004420in}}{\pgfqpoint{-0.016667in}{0.000000in}}%
\pgfpathcurveto{\pgfqpoint{-0.016667in}{-0.004420in}}{\pgfqpoint{-0.014911in}{-0.008660in}}{\pgfqpoint{-0.011785in}{-0.011785in}}%
\pgfpathcurveto{\pgfqpoint{-0.008660in}{-0.014911in}}{\pgfqpoint{-0.004420in}{-0.016667in}}{\pgfqpoint{0.000000in}{-0.016667in}}%
\pgfpathlineto{\pgfqpoint{0.000000in}{-0.016667in}}%
\pgfpathclose%
\pgfusepath{stroke,fill}%
}%
\begin{pgfscope}%
\pgfsys@transformshift{2.288260in}{2.879821in}%
\pgfsys@useobject{currentmarker}{}%
\end{pgfscope}%
\end{pgfscope}%
\begin{pgfscope}%
\pgfpathrectangle{\pgfqpoint{0.765000in}{0.660000in}}{\pgfqpoint{4.620000in}{4.620000in}}%
\pgfusepath{clip}%
\pgfsetroundcap%
\pgfsetroundjoin%
\pgfsetlinewidth{1.204500pt}%
\definecolor{currentstroke}{rgb}{0.000000,0.000000,0.000000}%
\pgfsetstrokecolor{currentstroke}%
\pgfsetdash{}{0pt}%
\pgfpathmoveto{\pgfqpoint{2.288260in}{2.879821in}}%
\pgfpathlineto{\pgfqpoint{2.398756in}{2.828036in}}%
\pgfusepath{stroke}%
\end{pgfscope}%
\begin{pgfscope}%
\pgfpathrectangle{\pgfqpoint{0.765000in}{0.660000in}}{\pgfqpoint{4.620000in}{4.620000in}}%
\pgfusepath{clip}%
\pgfsetbuttcap%
\pgfsetroundjoin%
\pgfsetlinewidth{1.204500pt}%
\definecolor{currentstroke}{rgb}{0.827451,0.827451,0.827451}%
\pgfsetstrokecolor{currentstroke}%
\pgfsetdash{{1.200000pt}{1.980000pt}}{0.000000pt}%
\pgfpathmoveto{\pgfqpoint{2.288260in}{2.879821in}}%
\pgfpathlineto{\pgfqpoint{2.288260in}{3.187196in}}%
\pgfusepath{stroke}%
\end{pgfscope}%
\begin{pgfscope}%
\pgfpathrectangle{\pgfqpoint{0.765000in}{0.660000in}}{\pgfqpoint{4.620000in}{4.620000in}}%
\pgfusepath{clip}%
\pgfsetroundcap%
\pgfsetroundjoin%
\pgfsetlinewidth{1.204500pt}%
\definecolor{currentstroke}{rgb}{0.000000,0.000000,0.000000}%
\pgfsetstrokecolor{currentstroke}%
\pgfsetdash{}{0pt}%
\pgfpathmoveto{\pgfqpoint{2.288260in}{2.879821in}}%
\pgfpathlineto{\pgfqpoint{1.909875in}{2.702487in}}%
\pgfusepath{stroke}%
\end{pgfscope}%
\begin{pgfscope}%
\pgfpathrectangle{\pgfqpoint{0.765000in}{0.660000in}}{\pgfqpoint{4.620000in}{4.620000in}}%
\pgfusepath{clip}%
\pgfsetroundcap%
\pgfsetroundjoin%
\pgfsetlinewidth{1.204500pt}%
\definecolor{currentstroke}{rgb}{0.000000,0.000000,0.000000}%
\pgfsetstrokecolor{currentstroke}%
\pgfsetdash{}{0pt}%
\pgfpathmoveto{\pgfqpoint{2.398756in}{2.828036in}}%
\pgfusepath{stroke}%
\end{pgfscope}%
\begin{pgfscope}%
\pgfpathrectangle{\pgfqpoint{0.765000in}{0.660000in}}{\pgfqpoint{4.620000in}{4.620000in}}%
\pgfusepath{clip}%
\pgfsetbuttcap%
\pgfsetroundjoin%
\definecolor{currentfill}{rgb}{0.000000,0.000000,0.000000}%
\pgfsetfillcolor{currentfill}%
\pgfsetlinewidth{1.003750pt}%
\definecolor{currentstroke}{rgb}{0.000000,0.000000,0.000000}%
\pgfsetstrokecolor{currentstroke}%
\pgfsetdash{}{0pt}%
\pgfsys@defobject{currentmarker}{\pgfqpoint{-0.016667in}{-0.016667in}}{\pgfqpoint{0.016667in}{0.016667in}}{%
\pgfpathmoveto{\pgfqpoint{0.000000in}{-0.016667in}}%
\pgfpathcurveto{\pgfqpoint{0.004420in}{-0.016667in}}{\pgfqpoint{0.008660in}{-0.014911in}}{\pgfqpoint{0.011785in}{-0.011785in}}%
\pgfpathcurveto{\pgfqpoint{0.014911in}{-0.008660in}}{\pgfqpoint{0.016667in}{-0.004420in}}{\pgfqpoint{0.016667in}{0.000000in}}%
\pgfpathcurveto{\pgfqpoint{0.016667in}{0.004420in}}{\pgfqpoint{0.014911in}{0.008660in}}{\pgfqpoint{0.011785in}{0.011785in}}%
\pgfpathcurveto{\pgfqpoint{0.008660in}{0.014911in}}{\pgfqpoint{0.004420in}{0.016667in}}{\pgfqpoint{0.000000in}{0.016667in}}%
\pgfpathcurveto{\pgfqpoint{-0.004420in}{0.016667in}}{\pgfqpoint{-0.008660in}{0.014911in}}{\pgfqpoint{-0.011785in}{0.011785in}}%
\pgfpathcurveto{\pgfqpoint{-0.014911in}{0.008660in}}{\pgfqpoint{-0.016667in}{0.004420in}}{\pgfqpoint{-0.016667in}{0.000000in}}%
\pgfpathcurveto{\pgfqpoint{-0.016667in}{-0.004420in}}{\pgfqpoint{-0.014911in}{-0.008660in}}{\pgfqpoint{-0.011785in}{-0.011785in}}%
\pgfpathcurveto{\pgfqpoint{-0.008660in}{-0.014911in}}{\pgfqpoint{-0.004420in}{-0.016667in}}{\pgfqpoint{0.000000in}{-0.016667in}}%
\pgfpathlineto{\pgfqpoint{0.000000in}{-0.016667in}}%
\pgfpathclose%
\pgfusepath{stroke,fill}%
}%
\begin{pgfscope}%
\pgfsys@transformshift{2.398756in}{2.828036in}%
\pgfsys@useobject{currentmarker}{}%
\end{pgfscope}%
\end{pgfscope}%
\begin{pgfscope}%
\pgfpathrectangle{\pgfqpoint{0.765000in}{0.660000in}}{\pgfqpoint{4.620000in}{4.620000in}}%
\pgfusepath{clip}%
\pgfsetroundcap%
\pgfsetroundjoin%
\pgfsetlinewidth{1.204500pt}%
\definecolor{currentstroke}{rgb}{0.000000,0.000000,0.000000}%
\pgfsetstrokecolor{currentstroke}%
\pgfsetdash{}{0pt}%
\pgfpathmoveto{\pgfqpoint{2.398756in}{2.828036in}}%
\pgfpathlineto{\pgfqpoint{2.288260in}{2.879821in}}%
\pgfusepath{stroke}%
\end{pgfscope}%
\begin{pgfscope}%
\pgfpathrectangle{\pgfqpoint{0.765000in}{0.660000in}}{\pgfqpoint{4.620000in}{4.620000in}}%
\pgfusepath{clip}%
\pgfsetroundcap%
\pgfsetroundjoin%
\pgfsetlinewidth{1.204500pt}%
\definecolor{currentstroke}{rgb}{0.000000,0.000000,0.000000}%
\pgfsetstrokecolor{currentstroke}%
\pgfsetdash{}{0pt}%
\pgfpathmoveto{\pgfqpoint{2.398756in}{2.828036in}}%
\pgfpathlineto{\pgfqpoint{2.713079in}{2.975346in}}%
\pgfusepath{stroke}%
\end{pgfscope}%
\begin{pgfscope}%
\pgfpathrectangle{\pgfqpoint{0.765000in}{0.660000in}}{\pgfqpoint{4.620000in}{4.620000in}}%
\pgfusepath{clip}%
\pgfsetbuttcap%
\pgfsetroundjoin%
\pgfsetlinewidth{1.204500pt}%
\definecolor{currentstroke}{rgb}{0.827451,0.827451,0.827451}%
\pgfsetstrokecolor{currentstroke}%
\pgfsetdash{{1.200000pt}{1.980000pt}}{0.000000pt}%
\pgfpathmoveto{\pgfqpoint{2.398756in}{2.828036in}}%
\pgfpathlineto{\pgfqpoint{2.398756in}{0.650000in}}%
\pgfusepath{stroke}%
\end{pgfscope}%
\begin{pgfscope}%
\pgfpathrectangle{\pgfqpoint{0.765000in}{0.660000in}}{\pgfqpoint{4.620000in}{4.620000in}}%
\pgfusepath{clip}%
\pgfsetroundcap%
\pgfsetroundjoin%
\pgfsetlinewidth{1.204500pt}%
\definecolor{currentstroke}{rgb}{0.000000,0.000000,0.000000}%
\pgfsetstrokecolor{currentstroke}%
\pgfsetdash{}{0pt}%
\pgfpathmoveto{\pgfqpoint{2.288260in}{3.187196in}}%
\pgfusepath{stroke}%
\end{pgfscope}%
\begin{pgfscope}%
\pgfpathrectangle{\pgfqpoint{0.765000in}{0.660000in}}{\pgfqpoint{4.620000in}{4.620000in}}%
\pgfusepath{clip}%
\pgfsetbuttcap%
\pgfsetroundjoin%
\definecolor{currentfill}{rgb}{0.000000,0.000000,0.000000}%
\pgfsetfillcolor{currentfill}%
\pgfsetlinewidth{1.003750pt}%
\definecolor{currentstroke}{rgb}{0.000000,0.000000,0.000000}%
\pgfsetstrokecolor{currentstroke}%
\pgfsetdash{}{0pt}%
\pgfsys@defobject{currentmarker}{\pgfqpoint{-0.016667in}{-0.016667in}}{\pgfqpoint{0.016667in}{0.016667in}}{%
\pgfpathmoveto{\pgfqpoint{0.000000in}{-0.016667in}}%
\pgfpathcurveto{\pgfqpoint{0.004420in}{-0.016667in}}{\pgfqpoint{0.008660in}{-0.014911in}}{\pgfqpoint{0.011785in}{-0.011785in}}%
\pgfpathcurveto{\pgfqpoint{0.014911in}{-0.008660in}}{\pgfqpoint{0.016667in}{-0.004420in}}{\pgfqpoint{0.016667in}{0.000000in}}%
\pgfpathcurveto{\pgfqpoint{0.016667in}{0.004420in}}{\pgfqpoint{0.014911in}{0.008660in}}{\pgfqpoint{0.011785in}{0.011785in}}%
\pgfpathcurveto{\pgfqpoint{0.008660in}{0.014911in}}{\pgfqpoint{0.004420in}{0.016667in}}{\pgfqpoint{0.000000in}{0.016667in}}%
\pgfpathcurveto{\pgfqpoint{-0.004420in}{0.016667in}}{\pgfqpoint{-0.008660in}{0.014911in}}{\pgfqpoint{-0.011785in}{0.011785in}}%
\pgfpathcurveto{\pgfqpoint{-0.014911in}{0.008660in}}{\pgfqpoint{-0.016667in}{0.004420in}}{\pgfqpoint{-0.016667in}{0.000000in}}%
\pgfpathcurveto{\pgfqpoint{-0.016667in}{-0.004420in}}{\pgfqpoint{-0.014911in}{-0.008660in}}{\pgfqpoint{-0.011785in}{-0.011785in}}%
\pgfpathcurveto{\pgfqpoint{-0.008660in}{-0.014911in}}{\pgfqpoint{-0.004420in}{-0.016667in}}{\pgfqpoint{0.000000in}{-0.016667in}}%
\pgfpathlineto{\pgfqpoint{0.000000in}{-0.016667in}}%
\pgfpathclose%
\pgfusepath{stroke,fill}%
}%
\begin{pgfscope}%
\pgfsys@transformshift{2.288260in}{3.187196in}%
\pgfsys@useobject{currentmarker}{}%
\end{pgfscope}%
\end{pgfscope}%
\begin{pgfscope}%
\pgfpathrectangle{\pgfqpoint{0.765000in}{0.660000in}}{\pgfqpoint{4.620000in}{4.620000in}}%
\pgfusepath{clip}%
\pgfsetbuttcap%
\pgfsetroundjoin%
\pgfsetlinewidth{1.204500pt}%
\definecolor{currentstroke}{rgb}{0.827451,0.827451,0.827451}%
\pgfsetstrokecolor{currentstroke}%
\pgfsetdash{{1.200000pt}{1.980000pt}}{0.000000pt}%
\pgfpathmoveto{\pgfqpoint{2.288260in}{3.187196in}}%
\pgfpathlineto{\pgfqpoint{2.288260in}{2.879821in}}%
\pgfusepath{stroke}%
\end{pgfscope}%
\begin{pgfscope}%
\pgfpathrectangle{\pgfqpoint{0.765000in}{0.660000in}}{\pgfqpoint{4.620000in}{4.620000in}}%
\pgfusepath{clip}%
\pgfsetroundcap%
\pgfsetroundjoin%
\pgfsetlinewidth{1.204500pt}%
\definecolor{currentstroke}{rgb}{0.000000,0.000000,0.000000}%
\pgfsetstrokecolor{currentstroke}%
\pgfsetdash{}{0pt}%
\pgfpathmoveto{\pgfqpoint{2.288260in}{3.187196in}}%
\pgfpathlineto{\pgfqpoint{2.334884in}{3.209047in}}%
\pgfusepath{stroke}%
\end{pgfscope}%
\begin{pgfscope}%
\pgfpathrectangle{\pgfqpoint{0.765000in}{0.660000in}}{\pgfqpoint{4.620000in}{4.620000in}}%
\pgfusepath{clip}%
\pgfsetroundcap%
\pgfsetroundjoin%
\pgfsetlinewidth{1.204500pt}%
\definecolor{currentstroke}{rgb}{0.000000,0.000000,0.000000}%
\pgfsetstrokecolor{currentstroke}%
\pgfsetdash{}{0pt}%
\pgfpathmoveto{\pgfqpoint{2.288260in}{3.187196in}}%
\pgfpathlineto{\pgfqpoint{0.755000in}{3.905774in}}%
\pgfusepath{stroke}%
\end{pgfscope}%
\begin{pgfscope}%
\pgfpathrectangle{\pgfqpoint{0.765000in}{0.660000in}}{\pgfqpoint{4.620000in}{4.620000in}}%
\pgfusepath{clip}%
\pgfsetroundcap%
\pgfsetroundjoin%
\pgfsetlinewidth{1.204500pt}%
\definecolor{currentstroke}{rgb}{0.000000,0.000000,0.000000}%
\pgfsetstrokecolor{currentstroke}%
\pgfsetdash{}{0pt}%
\pgfpathmoveto{\pgfqpoint{2.713079in}{2.975346in}}%
\pgfusepath{stroke}%
\end{pgfscope}%
\begin{pgfscope}%
\pgfpathrectangle{\pgfqpoint{0.765000in}{0.660000in}}{\pgfqpoint{4.620000in}{4.620000in}}%
\pgfusepath{clip}%
\pgfsetbuttcap%
\pgfsetroundjoin%
\definecolor{currentfill}{rgb}{0.000000,0.000000,0.000000}%
\pgfsetfillcolor{currentfill}%
\pgfsetlinewidth{1.003750pt}%
\definecolor{currentstroke}{rgb}{0.000000,0.000000,0.000000}%
\pgfsetstrokecolor{currentstroke}%
\pgfsetdash{}{0pt}%
\pgfsys@defobject{currentmarker}{\pgfqpoint{-0.016667in}{-0.016667in}}{\pgfqpoint{0.016667in}{0.016667in}}{%
\pgfpathmoveto{\pgfqpoint{0.000000in}{-0.016667in}}%
\pgfpathcurveto{\pgfqpoint{0.004420in}{-0.016667in}}{\pgfqpoint{0.008660in}{-0.014911in}}{\pgfqpoint{0.011785in}{-0.011785in}}%
\pgfpathcurveto{\pgfqpoint{0.014911in}{-0.008660in}}{\pgfqpoint{0.016667in}{-0.004420in}}{\pgfqpoint{0.016667in}{0.000000in}}%
\pgfpathcurveto{\pgfqpoint{0.016667in}{0.004420in}}{\pgfqpoint{0.014911in}{0.008660in}}{\pgfqpoint{0.011785in}{0.011785in}}%
\pgfpathcurveto{\pgfqpoint{0.008660in}{0.014911in}}{\pgfqpoint{0.004420in}{0.016667in}}{\pgfqpoint{0.000000in}{0.016667in}}%
\pgfpathcurveto{\pgfqpoint{-0.004420in}{0.016667in}}{\pgfqpoint{-0.008660in}{0.014911in}}{\pgfqpoint{-0.011785in}{0.011785in}}%
\pgfpathcurveto{\pgfqpoint{-0.014911in}{0.008660in}}{\pgfqpoint{-0.016667in}{0.004420in}}{\pgfqpoint{-0.016667in}{0.000000in}}%
\pgfpathcurveto{\pgfqpoint{-0.016667in}{-0.004420in}}{\pgfqpoint{-0.014911in}{-0.008660in}}{\pgfqpoint{-0.011785in}{-0.011785in}}%
\pgfpathcurveto{\pgfqpoint{-0.008660in}{-0.014911in}}{\pgfqpoint{-0.004420in}{-0.016667in}}{\pgfqpoint{0.000000in}{-0.016667in}}%
\pgfpathlineto{\pgfqpoint{0.000000in}{-0.016667in}}%
\pgfpathclose%
\pgfusepath{stroke,fill}%
}%
\begin{pgfscope}%
\pgfsys@transformshift{2.713079in}{2.975346in}%
\pgfsys@useobject{currentmarker}{}%
\end{pgfscope}%
\end{pgfscope}%
\begin{pgfscope}%
\pgfpathrectangle{\pgfqpoint{0.765000in}{0.660000in}}{\pgfqpoint{4.620000in}{4.620000in}}%
\pgfusepath{clip}%
\pgfsetroundcap%
\pgfsetroundjoin%
\pgfsetlinewidth{1.204500pt}%
\definecolor{currentstroke}{rgb}{0.000000,0.000000,0.000000}%
\pgfsetstrokecolor{currentstroke}%
\pgfsetdash{}{0pt}%
\pgfpathmoveto{\pgfqpoint{2.713079in}{2.975346in}}%
\pgfpathlineto{\pgfqpoint{2.398756in}{2.828036in}}%
\pgfusepath{stroke}%
\end{pgfscope}%
\begin{pgfscope}%
\pgfpathrectangle{\pgfqpoint{0.765000in}{0.660000in}}{\pgfqpoint{4.620000in}{4.620000in}}%
\pgfusepath{clip}%
\pgfsetbuttcap%
\pgfsetroundjoin%
\pgfsetlinewidth{1.204500pt}%
\definecolor{currentstroke}{rgb}{0.827451,0.827451,0.827451}%
\pgfsetstrokecolor{currentstroke}%
\pgfsetdash{{1.200000pt}{1.980000pt}}{0.000000pt}%
\pgfpathmoveto{\pgfqpoint{2.713079in}{2.975346in}}%
\pgfpathlineto{\pgfqpoint{2.713079in}{3.125145in}}%
\pgfusepath{stroke}%
\end{pgfscope}%
\begin{pgfscope}%
\pgfpathrectangle{\pgfqpoint{0.765000in}{0.660000in}}{\pgfqpoint{4.620000in}{4.620000in}}%
\pgfusepath{clip}%
\pgfsetroundcap%
\pgfsetroundjoin%
\pgfsetlinewidth{1.204500pt}%
\definecolor{currentstroke}{rgb}{0.000000,0.000000,0.000000}%
\pgfsetstrokecolor{currentstroke}%
\pgfsetdash{}{0pt}%
\pgfpathmoveto{\pgfqpoint{2.713079in}{2.975346in}}%
\pgfpathlineto{\pgfqpoint{2.749031in}{2.958497in}}%
\pgfusepath{stroke}%
\end{pgfscope}%
\begin{pgfscope}%
\pgfpathrectangle{\pgfqpoint{0.765000in}{0.660000in}}{\pgfqpoint{4.620000in}{4.620000in}}%
\pgfusepath{clip}%
\pgfsetroundcap%
\pgfsetroundjoin%
\pgfsetlinewidth{1.204500pt}%
\definecolor{currentstroke}{rgb}{0.000000,0.000000,0.000000}%
\pgfsetstrokecolor{currentstroke}%
\pgfsetdash{}{0pt}%
\pgfpathmoveto{\pgfqpoint{2.713079in}{3.125145in}}%
\pgfusepath{stroke}%
\end{pgfscope}%
\begin{pgfscope}%
\pgfpathrectangle{\pgfqpoint{0.765000in}{0.660000in}}{\pgfqpoint{4.620000in}{4.620000in}}%
\pgfusepath{clip}%
\pgfsetbuttcap%
\pgfsetroundjoin%
\definecolor{currentfill}{rgb}{0.000000,0.000000,0.000000}%
\pgfsetfillcolor{currentfill}%
\pgfsetlinewidth{1.003750pt}%
\definecolor{currentstroke}{rgb}{0.000000,0.000000,0.000000}%
\pgfsetstrokecolor{currentstroke}%
\pgfsetdash{}{0pt}%
\pgfsys@defobject{currentmarker}{\pgfqpoint{-0.016667in}{-0.016667in}}{\pgfqpoint{0.016667in}{0.016667in}}{%
\pgfpathmoveto{\pgfqpoint{0.000000in}{-0.016667in}}%
\pgfpathcurveto{\pgfqpoint{0.004420in}{-0.016667in}}{\pgfqpoint{0.008660in}{-0.014911in}}{\pgfqpoint{0.011785in}{-0.011785in}}%
\pgfpathcurveto{\pgfqpoint{0.014911in}{-0.008660in}}{\pgfqpoint{0.016667in}{-0.004420in}}{\pgfqpoint{0.016667in}{0.000000in}}%
\pgfpathcurveto{\pgfqpoint{0.016667in}{0.004420in}}{\pgfqpoint{0.014911in}{0.008660in}}{\pgfqpoint{0.011785in}{0.011785in}}%
\pgfpathcurveto{\pgfqpoint{0.008660in}{0.014911in}}{\pgfqpoint{0.004420in}{0.016667in}}{\pgfqpoint{0.000000in}{0.016667in}}%
\pgfpathcurveto{\pgfqpoint{-0.004420in}{0.016667in}}{\pgfqpoint{-0.008660in}{0.014911in}}{\pgfqpoint{-0.011785in}{0.011785in}}%
\pgfpathcurveto{\pgfqpoint{-0.014911in}{0.008660in}}{\pgfqpoint{-0.016667in}{0.004420in}}{\pgfqpoint{-0.016667in}{0.000000in}}%
\pgfpathcurveto{\pgfqpoint{-0.016667in}{-0.004420in}}{\pgfqpoint{-0.014911in}{-0.008660in}}{\pgfqpoint{-0.011785in}{-0.011785in}}%
\pgfpathcurveto{\pgfqpoint{-0.008660in}{-0.014911in}}{\pgfqpoint{-0.004420in}{-0.016667in}}{\pgfqpoint{0.000000in}{-0.016667in}}%
\pgfpathlineto{\pgfqpoint{0.000000in}{-0.016667in}}%
\pgfpathclose%
\pgfusepath{stroke,fill}%
}%
\begin{pgfscope}%
\pgfsys@transformshift{2.713079in}{3.125145in}%
\pgfsys@useobject{currentmarker}{}%
\end{pgfscope}%
\end{pgfscope}%
\begin{pgfscope}%
\pgfpathrectangle{\pgfqpoint{0.765000in}{0.660000in}}{\pgfqpoint{4.620000in}{4.620000in}}%
\pgfusepath{clip}%
\pgfsetbuttcap%
\pgfsetroundjoin%
\pgfsetlinewidth{1.204500pt}%
\definecolor{currentstroke}{rgb}{0.827451,0.827451,0.827451}%
\pgfsetstrokecolor{currentstroke}%
\pgfsetdash{{1.200000pt}{1.980000pt}}{0.000000pt}%
\pgfpathmoveto{\pgfqpoint{2.713079in}{3.125145in}}%
\pgfpathlineto{\pgfqpoint{2.713079in}{2.975346in}}%
\pgfusepath{stroke}%
\end{pgfscope}%
\begin{pgfscope}%
\pgfpathrectangle{\pgfqpoint{0.765000in}{0.660000in}}{\pgfqpoint{4.620000in}{4.620000in}}%
\pgfusepath{clip}%
\pgfsetroundcap%
\pgfsetroundjoin%
\pgfsetlinewidth{1.204500pt}%
\definecolor{currentstroke}{rgb}{0.000000,0.000000,0.000000}%
\pgfsetstrokecolor{currentstroke}%
\pgfsetdash{}{0pt}%
\pgfpathmoveto{\pgfqpoint{2.713079in}{3.125145in}}%
\pgfpathlineto{\pgfqpoint{2.534053in}{3.209047in}}%
\pgfusepath{stroke}%
\end{pgfscope}%
\begin{pgfscope}%
\pgfpathrectangle{\pgfqpoint{0.765000in}{0.660000in}}{\pgfqpoint{4.620000in}{4.620000in}}%
\pgfusepath{clip}%
\pgfsetroundcap%
\pgfsetroundjoin%
\pgfsetlinewidth{1.204500pt}%
\definecolor{currentstroke}{rgb}{0.000000,0.000000,0.000000}%
\pgfsetstrokecolor{currentstroke}%
\pgfsetdash{}{0pt}%
\pgfpathmoveto{\pgfqpoint{2.713079in}{3.125145in}}%
\pgfpathlineto{\pgfqpoint{5.395000in}{4.382053in}}%
\pgfusepath{stroke}%
\end{pgfscope}%
\begin{pgfscope}%
\pgfpathrectangle{\pgfqpoint{0.765000in}{0.660000in}}{\pgfqpoint{4.620000in}{4.620000in}}%
\pgfusepath{clip}%
\pgfsetroundcap%
\pgfsetroundjoin%
\pgfsetlinewidth{1.204500pt}%
\definecolor{currentstroke}{rgb}{0.000000,0.000000,0.000000}%
\pgfsetstrokecolor{currentstroke}%
\pgfsetdash{}{0pt}%
\pgfpathmoveto{\pgfqpoint{2.334884in}{3.209047in}}%
\pgfusepath{stroke}%
\end{pgfscope}%
\begin{pgfscope}%
\pgfpathrectangle{\pgfqpoint{0.765000in}{0.660000in}}{\pgfqpoint{4.620000in}{4.620000in}}%
\pgfusepath{clip}%
\pgfsetbuttcap%
\pgfsetroundjoin%
\definecolor{currentfill}{rgb}{0.000000,0.000000,0.000000}%
\pgfsetfillcolor{currentfill}%
\pgfsetlinewidth{1.003750pt}%
\definecolor{currentstroke}{rgb}{0.000000,0.000000,0.000000}%
\pgfsetstrokecolor{currentstroke}%
\pgfsetdash{}{0pt}%
\pgfsys@defobject{currentmarker}{\pgfqpoint{-0.016667in}{-0.016667in}}{\pgfqpoint{0.016667in}{0.016667in}}{%
\pgfpathmoveto{\pgfqpoint{0.000000in}{-0.016667in}}%
\pgfpathcurveto{\pgfqpoint{0.004420in}{-0.016667in}}{\pgfqpoint{0.008660in}{-0.014911in}}{\pgfqpoint{0.011785in}{-0.011785in}}%
\pgfpathcurveto{\pgfqpoint{0.014911in}{-0.008660in}}{\pgfqpoint{0.016667in}{-0.004420in}}{\pgfqpoint{0.016667in}{0.000000in}}%
\pgfpathcurveto{\pgfqpoint{0.016667in}{0.004420in}}{\pgfqpoint{0.014911in}{0.008660in}}{\pgfqpoint{0.011785in}{0.011785in}}%
\pgfpathcurveto{\pgfqpoint{0.008660in}{0.014911in}}{\pgfqpoint{0.004420in}{0.016667in}}{\pgfqpoint{0.000000in}{0.016667in}}%
\pgfpathcurveto{\pgfqpoint{-0.004420in}{0.016667in}}{\pgfqpoint{-0.008660in}{0.014911in}}{\pgfqpoint{-0.011785in}{0.011785in}}%
\pgfpathcurveto{\pgfqpoint{-0.014911in}{0.008660in}}{\pgfqpoint{-0.016667in}{0.004420in}}{\pgfqpoint{-0.016667in}{0.000000in}}%
\pgfpathcurveto{\pgfqpoint{-0.016667in}{-0.004420in}}{\pgfqpoint{-0.014911in}{-0.008660in}}{\pgfqpoint{-0.011785in}{-0.011785in}}%
\pgfpathcurveto{\pgfqpoint{-0.008660in}{-0.014911in}}{\pgfqpoint{-0.004420in}{-0.016667in}}{\pgfqpoint{0.000000in}{-0.016667in}}%
\pgfpathlineto{\pgfqpoint{0.000000in}{-0.016667in}}%
\pgfpathclose%
\pgfusepath{stroke,fill}%
}%
\begin{pgfscope}%
\pgfsys@transformshift{2.334884in}{3.209047in}%
\pgfsys@useobject{currentmarker}{}%
\end{pgfscope}%
\end{pgfscope}%
\begin{pgfscope}%
\pgfpathrectangle{\pgfqpoint{0.765000in}{0.660000in}}{\pgfqpoint{4.620000in}{4.620000in}}%
\pgfusepath{clip}%
\pgfsetroundcap%
\pgfsetroundjoin%
\pgfsetlinewidth{1.204500pt}%
\definecolor{currentstroke}{rgb}{0.000000,0.000000,0.000000}%
\pgfsetstrokecolor{currentstroke}%
\pgfsetdash{}{0pt}%
\pgfpathmoveto{\pgfqpoint{2.334884in}{3.209047in}}%
\pgfpathlineto{\pgfqpoint{2.288260in}{3.187196in}}%
\pgfusepath{stroke}%
\end{pgfscope}%
\begin{pgfscope}%
\pgfpathrectangle{\pgfqpoint{0.765000in}{0.660000in}}{\pgfqpoint{4.620000in}{4.620000in}}%
\pgfusepath{clip}%
\pgfsetroundcap%
\pgfsetroundjoin%
\pgfsetlinewidth{1.204500pt}%
\definecolor{currentstroke}{rgb}{0.000000,0.000000,0.000000}%
\pgfsetstrokecolor{currentstroke}%
\pgfsetdash{}{0pt}%
\pgfpathmoveto{\pgfqpoint{2.334884in}{3.209047in}}%
\pgfpathlineto{\pgfqpoint{2.534053in}{3.209047in}}%
\pgfusepath{stroke}%
\end{pgfscope}%
\begin{pgfscope}%
\pgfpathrectangle{\pgfqpoint{0.765000in}{0.660000in}}{\pgfqpoint{4.620000in}{4.620000in}}%
\pgfusepath{clip}%
\pgfsetbuttcap%
\pgfsetroundjoin%
\pgfsetlinewidth{1.204500pt}%
\definecolor{currentstroke}{rgb}{0.827451,0.827451,0.827451}%
\pgfsetstrokecolor{currentstroke}%
\pgfsetdash{{1.200000pt}{1.980000pt}}{0.000000pt}%
\pgfpathmoveto{\pgfqpoint{2.334884in}{3.209047in}}%
\pgfpathlineto{\pgfqpoint{0.755000in}{3.949475in}}%
\pgfusepath{stroke}%
\end{pgfscope}%
\begin{pgfscope}%
\pgfpathrectangle{\pgfqpoint{0.765000in}{0.660000in}}{\pgfqpoint{4.620000in}{4.620000in}}%
\pgfusepath{clip}%
\pgfsetroundcap%
\pgfsetroundjoin%
\pgfsetlinewidth{1.204500pt}%
\definecolor{currentstroke}{rgb}{0.000000,0.000000,0.000000}%
\pgfsetstrokecolor{currentstroke}%
\pgfsetdash{}{0pt}%
\pgfpathmoveto{\pgfqpoint{2.534053in}{3.209047in}}%
\pgfusepath{stroke}%
\end{pgfscope}%
\begin{pgfscope}%
\pgfpathrectangle{\pgfqpoint{0.765000in}{0.660000in}}{\pgfqpoint{4.620000in}{4.620000in}}%
\pgfusepath{clip}%
\pgfsetbuttcap%
\pgfsetroundjoin%
\definecolor{currentfill}{rgb}{0.000000,0.000000,0.000000}%
\pgfsetfillcolor{currentfill}%
\pgfsetlinewidth{1.003750pt}%
\definecolor{currentstroke}{rgb}{0.000000,0.000000,0.000000}%
\pgfsetstrokecolor{currentstroke}%
\pgfsetdash{}{0pt}%
\pgfsys@defobject{currentmarker}{\pgfqpoint{-0.016667in}{-0.016667in}}{\pgfqpoint{0.016667in}{0.016667in}}{%
\pgfpathmoveto{\pgfqpoint{0.000000in}{-0.016667in}}%
\pgfpathcurveto{\pgfqpoint{0.004420in}{-0.016667in}}{\pgfqpoint{0.008660in}{-0.014911in}}{\pgfqpoint{0.011785in}{-0.011785in}}%
\pgfpathcurveto{\pgfqpoint{0.014911in}{-0.008660in}}{\pgfqpoint{0.016667in}{-0.004420in}}{\pgfqpoint{0.016667in}{0.000000in}}%
\pgfpathcurveto{\pgfqpoint{0.016667in}{0.004420in}}{\pgfqpoint{0.014911in}{0.008660in}}{\pgfqpoint{0.011785in}{0.011785in}}%
\pgfpathcurveto{\pgfqpoint{0.008660in}{0.014911in}}{\pgfqpoint{0.004420in}{0.016667in}}{\pgfqpoint{0.000000in}{0.016667in}}%
\pgfpathcurveto{\pgfqpoint{-0.004420in}{0.016667in}}{\pgfqpoint{-0.008660in}{0.014911in}}{\pgfqpoint{-0.011785in}{0.011785in}}%
\pgfpathcurveto{\pgfqpoint{-0.014911in}{0.008660in}}{\pgfqpoint{-0.016667in}{0.004420in}}{\pgfqpoint{-0.016667in}{0.000000in}}%
\pgfpathcurveto{\pgfqpoint{-0.016667in}{-0.004420in}}{\pgfqpoint{-0.014911in}{-0.008660in}}{\pgfqpoint{-0.011785in}{-0.011785in}}%
\pgfpathcurveto{\pgfqpoint{-0.008660in}{-0.014911in}}{\pgfqpoint{-0.004420in}{-0.016667in}}{\pgfqpoint{0.000000in}{-0.016667in}}%
\pgfpathlineto{\pgfqpoint{0.000000in}{-0.016667in}}%
\pgfpathclose%
\pgfusepath{stroke,fill}%
}%
\begin{pgfscope}%
\pgfsys@transformshift{2.534053in}{3.209047in}%
\pgfsys@useobject{currentmarker}{}%
\end{pgfscope}%
\end{pgfscope}%
\begin{pgfscope}%
\pgfpathrectangle{\pgfqpoint{0.765000in}{0.660000in}}{\pgfqpoint{4.620000in}{4.620000in}}%
\pgfusepath{clip}%
\pgfsetroundcap%
\pgfsetroundjoin%
\pgfsetlinewidth{1.204500pt}%
\definecolor{currentstroke}{rgb}{0.000000,0.000000,0.000000}%
\pgfsetstrokecolor{currentstroke}%
\pgfsetdash{}{0pt}%
\pgfpathmoveto{\pgfqpoint{2.534053in}{3.209047in}}%
\pgfpathlineto{\pgfqpoint{2.713079in}{3.125145in}}%
\pgfusepath{stroke}%
\end{pgfscope}%
\begin{pgfscope}%
\pgfpathrectangle{\pgfqpoint{0.765000in}{0.660000in}}{\pgfqpoint{4.620000in}{4.620000in}}%
\pgfusepath{clip}%
\pgfsetroundcap%
\pgfsetroundjoin%
\pgfsetlinewidth{1.204500pt}%
\definecolor{currentstroke}{rgb}{0.000000,0.000000,0.000000}%
\pgfsetstrokecolor{currentstroke}%
\pgfsetdash{}{0pt}%
\pgfpathmoveto{\pgfqpoint{2.534053in}{3.209047in}}%
\pgfpathlineto{\pgfqpoint{2.334884in}{3.209047in}}%
\pgfusepath{stroke}%
\end{pgfscope}%
\begin{pgfscope}%
\pgfpathrectangle{\pgfqpoint{0.765000in}{0.660000in}}{\pgfqpoint{4.620000in}{4.620000in}}%
\pgfusepath{clip}%
\pgfsetbuttcap%
\pgfsetroundjoin%
\pgfsetlinewidth{1.204500pt}%
\definecolor{currentstroke}{rgb}{0.827451,0.827451,0.827451}%
\pgfsetstrokecolor{currentstroke}%
\pgfsetdash{{1.200000pt}{1.980000pt}}{0.000000pt}%
\pgfpathmoveto{\pgfqpoint{2.534053in}{3.209047in}}%
\pgfpathlineto{\pgfqpoint{5.395000in}{4.549858in}}%
\pgfusepath{stroke}%
\end{pgfscope}%
\begin{pgfscope}%
\pgfpathrectangle{\pgfqpoint{0.765000in}{0.660000in}}{\pgfqpoint{4.620000in}{4.620000in}}%
\pgfusepath{clip}%
\pgfsetroundcap%
\pgfsetroundjoin%
\pgfsetlinewidth{1.204500pt}%
\definecolor{currentstroke}{rgb}{0.000000,0.000000,0.000000}%
\pgfsetstrokecolor{currentstroke}%
\pgfsetdash{}{0pt}%
\pgfpathmoveto{\pgfqpoint{1.909875in}{2.702487in}}%
\pgfusepath{stroke}%
\end{pgfscope}%
\begin{pgfscope}%
\pgfpathrectangle{\pgfqpoint{0.765000in}{0.660000in}}{\pgfqpoint{4.620000in}{4.620000in}}%
\pgfusepath{clip}%
\pgfsetbuttcap%
\pgfsetroundjoin%
\definecolor{currentfill}{rgb}{0.000000,0.000000,0.000000}%
\pgfsetfillcolor{currentfill}%
\pgfsetlinewidth{1.003750pt}%
\definecolor{currentstroke}{rgb}{0.000000,0.000000,0.000000}%
\pgfsetstrokecolor{currentstroke}%
\pgfsetdash{}{0pt}%
\pgfsys@defobject{currentmarker}{\pgfqpoint{-0.016667in}{-0.016667in}}{\pgfqpoint{0.016667in}{0.016667in}}{%
\pgfpathmoveto{\pgfqpoint{0.000000in}{-0.016667in}}%
\pgfpathcurveto{\pgfqpoint{0.004420in}{-0.016667in}}{\pgfqpoint{0.008660in}{-0.014911in}}{\pgfqpoint{0.011785in}{-0.011785in}}%
\pgfpathcurveto{\pgfqpoint{0.014911in}{-0.008660in}}{\pgfqpoint{0.016667in}{-0.004420in}}{\pgfqpoint{0.016667in}{0.000000in}}%
\pgfpathcurveto{\pgfqpoint{0.016667in}{0.004420in}}{\pgfqpoint{0.014911in}{0.008660in}}{\pgfqpoint{0.011785in}{0.011785in}}%
\pgfpathcurveto{\pgfqpoint{0.008660in}{0.014911in}}{\pgfqpoint{0.004420in}{0.016667in}}{\pgfqpoint{0.000000in}{0.016667in}}%
\pgfpathcurveto{\pgfqpoint{-0.004420in}{0.016667in}}{\pgfqpoint{-0.008660in}{0.014911in}}{\pgfqpoint{-0.011785in}{0.011785in}}%
\pgfpathcurveto{\pgfqpoint{-0.014911in}{0.008660in}}{\pgfqpoint{-0.016667in}{0.004420in}}{\pgfqpoint{-0.016667in}{0.000000in}}%
\pgfpathcurveto{\pgfqpoint{-0.016667in}{-0.004420in}}{\pgfqpoint{-0.014911in}{-0.008660in}}{\pgfqpoint{-0.011785in}{-0.011785in}}%
\pgfpathcurveto{\pgfqpoint{-0.008660in}{-0.014911in}}{\pgfqpoint{-0.004420in}{-0.016667in}}{\pgfqpoint{0.000000in}{-0.016667in}}%
\pgfpathlineto{\pgfqpoint{0.000000in}{-0.016667in}}%
\pgfpathclose%
\pgfusepath{stroke,fill}%
}%
\begin{pgfscope}%
\pgfsys@transformshift{1.909875in}{2.702487in}%
\pgfsys@useobject{currentmarker}{}%
\end{pgfscope}%
\end{pgfscope}%
\begin{pgfscope}%
\pgfpathrectangle{\pgfqpoint{0.765000in}{0.660000in}}{\pgfqpoint{4.620000in}{4.620000in}}%
\pgfusepath{clip}%
\pgfsetroundcap%
\pgfsetroundjoin%
\pgfsetlinewidth{1.204500pt}%
\definecolor{currentstroke}{rgb}{0.000000,0.000000,0.000000}%
\pgfsetstrokecolor{currentstroke}%
\pgfsetdash{}{0pt}%
\pgfpathmoveto{\pgfqpoint{1.909875in}{2.702487in}}%
\pgfpathlineto{\pgfqpoint{2.288260in}{2.879821in}}%
\pgfusepath{stroke}%
\end{pgfscope}%
\begin{pgfscope}%
\pgfpathrectangle{\pgfqpoint{0.765000in}{0.660000in}}{\pgfqpoint{4.620000in}{4.620000in}}%
\pgfusepath{clip}%
\pgfsetbuttcap%
\pgfsetroundjoin%
\pgfsetlinewidth{1.204500pt}%
\definecolor{currentstroke}{rgb}{0.827451,0.827451,0.827451}%
\pgfsetstrokecolor{currentstroke}%
\pgfsetdash{{1.200000pt}{1.980000pt}}{0.000000pt}%
\pgfpathmoveto{\pgfqpoint{1.909875in}{2.702487in}}%
\pgfpathlineto{\pgfqpoint{1.909875in}{0.650000in}}%
\pgfusepath{stroke}%
\end{pgfscope}%
\begin{pgfscope}%
\pgfpathrectangle{\pgfqpoint{0.765000in}{0.660000in}}{\pgfqpoint{4.620000in}{4.620000in}}%
\pgfusepath{clip}%
\pgfsetroundcap%
\pgfsetroundjoin%
\pgfsetlinewidth{1.204500pt}%
\definecolor{currentstroke}{rgb}{0.000000,0.000000,0.000000}%
\pgfsetstrokecolor{currentstroke}%
\pgfsetdash{}{0pt}%
\pgfpathmoveto{\pgfqpoint{1.909875in}{2.702487in}}%
\pgfpathlineto{\pgfqpoint{0.755000in}{3.243730in}}%
\pgfusepath{stroke}%
\end{pgfscope}%
\begin{pgfscope}%
\pgfpathrectangle{\pgfqpoint{0.765000in}{0.660000in}}{\pgfqpoint{4.620000in}{4.620000in}}%
\pgfusepath{clip}%
\pgfsetroundcap%
\pgfsetroundjoin%
\pgfsetlinewidth{1.204500pt}%
\definecolor{currentstroke}{rgb}{0.000000,0.000000,0.000000}%
\pgfsetstrokecolor{currentstroke}%
\pgfsetdash{}{0pt}%
\pgfpathmoveto{\pgfqpoint{2.749031in}{2.958497in}}%
\pgfusepath{stroke}%
\end{pgfscope}%
\begin{pgfscope}%
\pgfpathrectangle{\pgfqpoint{0.765000in}{0.660000in}}{\pgfqpoint{4.620000in}{4.620000in}}%
\pgfusepath{clip}%
\pgfsetbuttcap%
\pgfsetroundjoin%
\definecolor{currentfill}{rgb}{0.000000,0.000000,0.000000}%
\pgfsetfillcolor{currentfill}%
\pgfsetlinewidth{1.003750pt}%
\definecolor{currentstroke}{rgb}{0.000000,0.000000,0.000000}%
\pgfsetstrokecolor{currentstroke}%
\pgfsetdash{}{0pt}%
\pgfsys@defobject{currentmarker}{\pgfqpoint{-0.016667in}{-0.016667in}}{\pgfqpoint{0.016667in}{0.016667in}}{%
\pgfpathmoveto{\pgfqpoint{0.000000in}{-0.016667in}}%
\pgfpathcurveto{\pgfqpoint{0.004420in}{-0.016667in}}{\pgfqpoint{0.008660in}{-0.014911in}}{\pgfqpoint{0.011785in}{-0.011785in}}%
\pgfpathcurveto{\pgfqpoint{0.014911in}{-0.008660in}}{\pgfqpoint{0.016667in}{-0.004420in}}{\pgfqpoint{0.016667in}{0.000000in}}%
\pgfpathcurveto{\pgfqpoint{0.016667in}{0.004420in}}{\pgfqpoint{0.014911in}{0.008660in}}{\pgfqpoint{0.011785in}{0.011785in}}%
\pgfpathcurveto{\pgfqpoint{0.008660in}{0.014911in}}{\pgfqpoint{0.004420in}{0.016667in}}{\pgfqpoint{0.000000in}{0.016667in}}%
\pgfpathcurveto{\pgfqpoint{-0.004420in}{0.016667in}}{\pgfqpoint{-0.008660in}{0.014911in}}{\pgfqpoint{-0.011785in}{0.011785in}}%
\pgfpathcurveto{\pgfqpoint{-0.014911in}{0.008660in}}{\pgfqpoint{-0.016667in}{0.004420in}}{\pgfqpoint{-0.016667in}{0.000000in}}%
\pgfpathcurveto{\pgfqpoint{-0.016667in}{-0.004420in}}{\pgfqpoint{-0.014911in}{-0.008660in}}{\pgfqpoint{-0.011785in}{-0.011785in}}%
\pgfpathcurveto{\pgfqpoint{-0.008660in}{-0.014911in}}{\pgfqpoint{-0.004420in}{-0.016667in}}{\pgfqpoint{0.000000in}{-0.016667in}}%
\pgfpathlineto{\pgfqpoint{0.000000in}{-0.016667in}}%
\pgfpathclose%
\pgfusepath{stroke,fill}%
}%
\begin{pgfscope}%
\pgfsys@transformshift{2.749031in}{2.958497in}%
\pgfsys@useobject{currentmarker}{}%
\end{pgfscope}%
\end{pgfscope}%
\begin{pgfscope}%
\pgfpathrectangle{\pgfqpoint{0.765000in}{0.660000in}}{\pgfqpoint{4.620000in}{4.620000in}}%
\pgfusepath{clip}%
\pgfsetroundcap%
\pgfsetroundjoin%
\pgfsetlinewidth{1.204500pt}%
\definecolor{currentstroke}{rgb}{0.000000,0.000000,0.000000}%
\pgfsetstrokecolor{currentstroke}%
\pgfsetdash{}{0pt}%
\pgfpathmoveto{\pgfqpoint{2.749031in}{2.958497in}}%
\pgfpathlineto{\pgfqpoint{2.713079in}{2.975346in}}%
\pgfusepath{stroke}%
\end{pgfscope}%
\begin{pgfscope}%
\pgfpathrectangle{\pgfqpoint{0.765000in}{0.660000in}}{\pgfqpoint{4.620000in}{4.620000in}}%
\pgfusepath{clip}%
\pgfsetroundcap%
\pgfsetroundjoin%
\pgfsetlinewidth{1.204500pt}%
\definecolor{currentstroke}{rgb}{0.000000,0.000000,0.000000}%
\pgfsetstrokecolor{currentstroke}%
\pgfsetdash{}{0pt}%
\pgfpathmoveto{\pgfqpoint{2.749031in}{2.958497in}}%
\pgfpathlineto{\pgfqpoint{5.395000in}{4.198556in}}%
\pgfusepath{stroke}%
\end{pgfscope}%
\begin{pgfscope}%
\pgfpathrectangle{\pgfqpoint{0.765000in}{0.660000in}}{\pgfqpoint{4.620000in}{4.620000in}}%
\pgfusepath{clip}%
\pgfsetbuttcap%
\pgfsetroundjoin%
\pgfsetlinewidth{1.204500pt}%
\definecolor{currentstroke}{rgb}{0.827451,0.827451,0.827451}%
\pgfsetstrokecolor{currentstroke}%
\pgfsetdash{{1.200000pt}{1.980000pt}}{0.000000pt}%
\pgfpathmoveto{\pgfqpoint{2.749031in}{2.958497in}}%
\pgfpathlineto{\pgfqpoint{2.749031in}{0.650000in}}%
\pgfusepath{stroke}%
\end{pgfscope}%
\end{pgfpicture}%
\makeatother%
\endgroup%
}
        }
        \caption{Iris dataset, sepal and petal width, cubic}
        \label{fig:iris}
    \end{subfigure}

    \bigskip
    \centering
    \begin{subfigure}{0.45\textwidth}
        \centering
        \resizebox{\linewidth}{!}{%
        \centering
            \clipbox{0.25\width{} 0.3\height{} 0.25\width{} 0.3\height{}}{%% Creator: Matplotlib, PGF backend
%%
%% To include the figure in your LaTeX document, write
%%   \input{<filename>.pgf}
%%
%% Make sure the required packages are loaded in your preamble
%%   \usepackage{pgf}
%%
%% Also ensure that all the required font packages are loaded; for instance,
%% the lmodern package is sometimes necessary when using math font.
%%   \usepackage{lmodern}
%%
%% Figures using additional raster images can only be included by \input if
%% they are in the same directory as the main LaTeX file. For loading figures
%% from other directories you can use the `import` package
%%   \usepackage{import}
%%
%% and then include the figures with
%%   \import{<path to file>}{<filename>.pgf}
%%
%% Matplotlib used the following preamble
%%   
%%   \usepackage{fontspec}
%%   \setmainfont{DejaVuSerif.ttf}[Path=\detokenize{/Users/sam/Library/Python/3.9/lib/python/site-packages/matplotlib/mpl-data/fonts/ttf/}]
%%   \setsansfont{Arial.ttf}[Path=\detokenize{/System/Library/Fonts/Supplemental/}]
%%   \setmonofont{DejaVuSansMono.ttf}[Path=\detokenize{/Users/sam/Library/Python/3.9/lib/python/site-packages/matplotlib/mpl-data/fonts/ttf/}]
%%   \makeatletter\@ifpackageloaded{underscore}{}{\usepackage[strings]{underscore}}\makeatother
%%
\begingroup%
\makeatletter%
\begin{pgfpicture}%
\pgfpathrectangle{\pgfpointorigin}{\pgfqpoint{6.000000in}{6.000000in}}%
\pgfusepath{use as bounding box, clip}%
\begin{pgfscope}%
\pgfsetbuttcap%
\pgfsetmiterjoin%
\definecolor{currentfill}{rgb}{1.000000,1.000000,1.000000}%
\pgfsetfillcolor{currentfill}%
\pgfsetlinewidth{0.000000pt}%
\definecolor{currentstroke}{rgb}{1.000000,1.000000,1.000000}%
\pgfsetstrokecolor{currentstroke}%
\pgfsetdash{}{0pt}%
\pgfpathmoveto{\pgfqpoint{0.000000in}{0.000000in}}%
\pgfpathlineto{\pgfqpoint{6.000000in}{0.000000in}}%
\pgfpathlineto{\pgfqpoint{6.000000in}{6.000000in}}%
\pgfpathlineto{\pgfqpoint{0.000000in}{6.000000in}}%
\pgfpathlineto{\pgfqpoint{0.000000in}{0.000000in}}%
\pgfpathclose%
\pgfusepath{fill}%
\end{pgfscope}%
\begin{pgfscope}%
\pgfsetbuttcap%
\pgfsetmiterjoin%
\definecolor{currentfill}{rgb}{1.000000,1.000000,1.000000}%
\pgfsetfillcolor{currentfill}%
\pgfsetlinewidth{0.000000pt}%
\definecolor{currentstroke}{rgb}{0.000000,0.000000,0.000000}%
\pgfsetstrokecolor{currentstroke}%
\pgfsetstrokeopacity{0.000000}%
\pgfsetdash{}{0pt}%
\pgfpathmoveto{\pgfqpoint{0.765000in}{0.660000in}}%
\pgfpathlineto{\pgfqpoint{5.385000in}{0.660000in}}%
\pgfpathlineto{\pgfqpoint{5.385000in}{5.280000in}}%
\pgfpathlineto{\pgfqpoint{0.765000in}{5.280000in}}%
\pgfpathlineto{\pgfqpoint{0.765000in}{0.660000in}}%
\pgfpathclose%
\pgfusepath{fill}%
\end{pgfscope}%
\begin{pgfscope}%
\pgfpathrectangle{\pgfqpoint{0.765000in}{0.660000in}}{\pgfqpoint{4.620000in}{4.620000in}}%
\pgfusepath{clip}%
\pgfsetbuttcap%
\pgfsetroundjoin%
\definecolor{currentfill}{rgb}{1.000000,0.576471,0.309804}%
\pgfsetfillcolor{currentfill}%
\pgfsetfillopacity{0.050000}%
\pgfsetlinewidth{0.000000pt}%
\definecolor{currentstroke}{rgb}{1.000000,0.576471,0.309804}%
\pgfsetstrokecolor{currentstroke}%
\pgfsetdash{}{0pt}%
\pgfpathmoveto{\pgfqpoint{4.096976in}{2.812171in}}%
\pgfpathlineto{\pgfqpoint{4.096976in}{2.885174in}}%
\pgfpathlineto{\pgfqpoint{4.174929in}{2.848578in}}%
\pgfpathlineto{\pgfqpoint{4.174929in}{2.775575in}}%
\pgfpathlineto{\pgfqpoint{4.096976in}{2.812171in}}%
\pgfpathclose%
\pgfusepath{fill}%
\end{pgfscope}%
\begin{pgfscope}%
\pgfpathrectangle{\pgfqpoint{0.765000in}{0.660000in}}{\pgfqpoint{4.620000in}{4.620000in}}%
\pgfusepath{clip}%
\pgfsetbuttcap%
\pgfsetroundjoin%
\definecolor{currentfill}{rgb}{1.000000,0.576471,0.309804}%
\pgfsetfillcolor{currentfill}%
\pgfsetfillopacity{0.050000}%
\pgfsetlinewidth{0.000000pt}%
\definecolor{currentstroke}{rgb}{1.000000,0.576471,0.309804}%
\pgfsetstrokecolor{currentstroke}%
\pgfsetdash{}{0pt}%
\pgfpathmoveto{\pgfqpoint{4.096976in}{2.812171in}}%
\pgfpathlineto{\pgfqpoint{4.096976in}{2.885174in}}%
\pgfpathlineto{\pgfqpoint{4.019024in}{2.848578in}}%
\pgfpathlineto{\pgfqpoint{4.019024in}{2.775575in}}%
\pgfpathlineto{\pgfqpoint{4.096976in}{2.812171in}}%
\pgfpathclose%
\pgfusepath{fill}%
\end{pgfscope}%
\begin{pgfscope}%
\pgfpathrectangle{\pgfqpoint{0.765000in}{0.660000in}}{\pgfqpoint{4.620000in}{4.620000in}}%
\pgfusepath{clip}%
\pgfsetbuttcap%
\pgfsetroundjoin%
\definecolor{currentfill}{rgb}{1.000000,0.576471,0.309804}%
\pgfsetfillcolor{currentfill}%
\pgfsetfillopacity{0.050000}%
\pgfsetlinewidth{0.000000pt}%
\definecolor{currentstroke}{rgb}{1.000000,0.576471,0.309804}%
\pgfsetstrokecolor{currentstroke}%
\pgfsetdash{}{0pt}%
\pgfpathmoveto{\pgfqpoint{4.096976in}{2.812171in}}%
\pgfpathlineto{\pgfqpoint{4.174929in}{2.775575in}}%
\pgfpathlineto{\pgfqpoint{4.096976in}{2.738978in}}%
\pgfpathlineto{\pgfqpoint{4.019024in}{2.775575in}}%
\pgfpathlineto{\pgfqpoint{4.096976in}{2.812171in}}%
\pgfpathclose%
\pgfusepath{fill}%
\end{pgfscope}%
\begin{pgfscope}%
\pgfpathrectangle{\pgfqpoint{0.765000in}{0.660000in}}{\pgfqpoint{4.620000in}{4.620000in}}%
\pgfusepath{clip}%
\pgfsetbuttcap%
\pgfsetroundjoin%
\definecolor{currentfill}{rgb}{1.000000,0.576471,0.309804}%
\pgfsetfillcolor{currentfill}%
\pgfsetfillopacity{0.050000}%
\pgfsetlinewidth{0.000000pt}%
\definecolor{currentstroke}{rgb}{1.000000,0.576471,0.309804}%
\pgfsetstrokecolor{currentstroke}%
\pgfsetdash{}{0pt}%
\pgfpathmoveto{\pgfqpoint{4.096976in}{2.885174in}}%
\pgfpathlineto{\pgfqpoint{4.174929in}{2.848578in}}%
\pgfpathlineto{\pgfqpoint{4.096976in}{2.811981in}}%
\pgfpathlineto{\pgfqpoint{4.019024in}{2.848578in}}%
\pgfpathlineto{\pgfqpoint{4.096976in}{2.885174in}}%
\pgfpathclose%
\pgfusepath{fill}%
\end{pgfscope}%
\begin{pgfscope}%
\pgfpathrectangle{\pgfqpoint{0.765000in}{0.660000in}}{\pgfqpoint{4.620000in}{4.620000in}}%
\pgfusepath{clip}%
\pgfsetbuttcap%
\pgfsetroundjoin%
\definecolor{currentfill}{rgb}{1.000000,0.576471,0.309804}%
\pgfsetfillcolor{currentfill}%
\pgfsetfillopacity{0.050000}%
\pgfsetlinewidth{0.000000pt}%
\definecolor{currentstroke}{rgb}{1.000000,0.576471,0.309804}%
\pgfsetstrokecolor{currentstroke}%
\pgfsetdash{}{0pt}%
\pgfpathmoveto{\pgfqpoint{4.019024in}{2.775575in}}%
\pgfpathlineto{\pgfqpoint{4.019024in}{2.848578in}}%
\pgfpathlineto{\pgfqpoint{4.096976in}{2.811981in}}%
\pgfpathlineto{\pgfqpoint{4.096976in}{2.738978in}}%
\pgfpathlineto{\pgfqpoint{4.019024in}{2.775575in}}%
\pgfpathclose%
\pgfusepath{fill}%
\end{pgfscope}%
\begin{pgfscope}%
\pgfpathrectangle{\pgfqpoint{0.765000in}{0.660000in}}{\pgfqpoint{4.620000in}{4.620000in}}%
\pgfusepath{clip}%
\pgfsetbuttcap%
\pgfsetroundjoin%
\definecolor{currentfill}{rgb}{1.000000,0.576471,0.309804}%
\pgfsetfillcolor{currentfill}%
\pgfsetfillopacity{0.050000}%
\pgfsetlinewidth{0.000000pt}%
\definecolor{currentstroke}{rgb}{1.000000,0.576471,0.309804}%
\pgfsetstrokecolor{currentstroke}%
\pgfsetdash{}{0pt}%
\pgfpathmoveto{\pgfqpoint{4.174929in}{2.775575in}}%
\pgfpathlineto{\pgfqpoint{4.174929in}{2.848578in}}%
\pgfpathlineto{\pgfqpoint{4.096976in}{2.811981in}}%
\pgfpathlineto{\pgfqpoint{4.096976in}{2.738978in}}%
\pgfpathlineto{\pgfqpoint{4.174929in}{2.775575in}}%
\pgfpathclose%
\pgfusepath{fill}%
\end{pgfscope}%
\begin{pgfscope}%
\pgfpathrectangle{\pgfqpoint{0.765000in}{0.660000in}}{\pgfqpoint{4.620000in}{4.620000in}}%
\pgfusepath{clip}%
\pgfsetroundcap%
\pgfsetroundjoin%
\pgfsetlinewidth{1.204500pt}%
\definecolor{currentstroke}{rgb}{1.000000,0.576471,0.309804}%
\pgfsetstrokecolor{currentstroke}%
\pgfsetdash{}{0pt}%
\pgfpathmoveto{\pgfqpoint{3.510546in}{2.285337in}}%
\pgfusepath{stroke}%
\end{pgfscope}%
\begin{pgfscope}%
\pgfpathrectangle{\pgfqpoint{0.765000in}{0.660000in}}{\pgfqpoint{4.620000in}{4.620000in}}%
\pgfusepath{clip}%
\pgfsetbuttcap%
\pgfsetroundjoin%
\definecolor{currentfill}{rgb}{1.000000,0.576471,0.309804}%
\pgfsetfillcolor{currentfill}%
\pgfsetlinewidth{1.003750pt}%
\definecolor{currentstroke}{rgb}{1.000000,0.576471,0.309804}%
\pgfsetstrokecolor{currentstroke}%
\pgfsetdash{}{0pt}%
\pgfsys@defobject{currentmarker}{\pgfqpoint{-0.033333in}{-0.033333in}}{\pgfqpoint{0.033333in}{0.033333in}}{%
\pgfpathmoveto{\pgfqpoint{0.000000in}{-0.033333in}}%
\pgfpathcurveto{\pgfqpoint{0.008840in}{-0.033333in}}{\pgfqpoint{0.017319in}{-0.029821in}}{\pgfqpoint{0.023570in}{-0.023570in}}%
\pgfpathcurveto{\pgfqpoint{0.029821in}{-0.017319in}}{\pgfqpoint{0.033333in}{-0.008840in}}{\pgfqpoint{0.033333in}{0.000000in}}%
\pgfpathcurveto{\pgfqpoint{0.033333in}{0.008840in}}{\pgfqpoint{0.029821in}{0.017319in}}{\pgfqpoint{0.023570in}{0.023570in}}%
\pgfpathcurveto{\pgfqpoint{0.017319in}{0.029821in}}{\pgfqpoint{0.008840in}{0.033333in}}{\pgfqpoint{0.000000in}{0.033333in}}%
\pgfpathcurveto{\pgfqpoint{-0.008840in}{0.033333in}}{\pgfqpoint{-0.017319in}{0.029821in}}{\pgfqpoint{-0.023570in}{0.023570in}}%
\pgfpathcurveto{\pgfqpoint{-0.029821in}{0.017319in}}{\pgfqpoint{-0.033333in}{0.008840in}}{\pgfqpoint{-0.033333in}{0.000000in}}%
\pgfpathcurveto{\pgfqpoint{-0.033333in}{-0.008840in}}{\pgfqpoint{-0.029821in}{-0.017319in}}{\pgfqpoint{-0.023570in}{-0.023570in}}%
\pgfpathcurveto{\pgfqpoint{-0.017319in}{-0.029821in}}{\pgfqpoint{-0.008840in}{-0.033333in}}{\pgfqpoint{0.000000in}{-0.033333in}}%
\pgfpathlineto{\pgfqpoint{0.000000in}{-0.033333in}}%
\pgfpathclose%
\pgfusepath{stroke,fill}%
}%
\begin{pgfscope}%
\pgfsys@transformshift{3.510546in}{2.285337in}%
\pgfsys@useobject{currentmarker}{}%
\end{pgfscope}%
\end{pgfscope}%
\begin{pgfscope}%
\pgfpathrectangle{\pgfqpoint{0.765000in}{0.660000in}}{\pgfqpoint{4.620000in}{4.620000in}}%
\pgfusepath{clip}%
\pgfsetroundcap%
\pgfsetroundjoin%
\pgfsetlinewidth{1.204500pt}%
\definecolor{currentstroke}{rgb}{1.000000,0.576471,0.309804}%
\pgfsetstrokecolor{currentstroke}%
\pgfsetdash{}{0pt}%
\pgfpathmoveto{\pgfqpoint{2.782812in}{3.874284in}}%
\pgfusepath{stroke}%
\end{pgfscope}%
\begin{pgfscope}%
\pgfpathrectangle{\pgfqpoint{0.765000in}{0.660000in}}{\pgfqpoint{4.620000in}{4.620000in}}%
\pgfusepath{clip}%
\pgfsetbuttcap%
\pgfsetroundjoin%
\definecolor{currentfill}{rgb}{1.000000,0.576471,0.309804}%
\pgfsetfillcolor{currentfill}%
\pgfsetlinewidth{1.003750pt}%
\definecolor{currentstroke}{rgb}{1.000000,0.576471,0.309804}%
\pgfsetstrokecolor{currentstroke}%
\pgfsetdash{}{0pt}%
\pgfsys@defobject{currentmarker}{\pgfqpoint{-0.033333in}{-0.033333in}}{\pgfqpoint{0.033333in}{0.033333in}}{%
\pgfpathmoveto{\pgfqpoint{0.000000in}{-0.033333in}}%
\pgfpathcurveto{\pgfqpoint{0.008840in}{-0.033333in}}{\pgfqpoint{0.017319in}{-0.029821in}}{\pgfqpoint{0.023570in}{-0.023570in}}%
\pgfpathcurveto{\pgfqpoint{0.029821in}{-0.017319in}}{\pgfqpoint{0.033333in}{-0.008840in}}{\pgfqpoint{0.033333in}{0.000000in}}%
\pgfpathcurveto{\pgfqpoint{0.033333in}{0.008840in}}{\pgfqpoint{0.029821in}{0.017319in}}{\pgfqpoint{0.023570in}{0.023570in}}%
\pgfpathcurveto{\pgfqpoint{0.017319in}{0.029821in}}{\pgfqpoint{0.008840in}{0.033333in}}{\pgfqpoint{0.000000in}{0.033333in}}%
\pgfpathcurveto{\pgfqpoint{-0.008840in}{0.033333in}}{\pgfqpoint{-0.017319in}{0.029821in}}{\pgfqpoint{-0.023570in}{0.023570in}}%
\pgfpathcurveto{\pgfqpoint{-0.029821in}{0.017319in}}{\pgfqpoint{-0.033333in}{0.008840in}}{\pgfqpoint{-0.033333in}{0.000000in}}%
\pgfpathcurveto{\pgfqpoint{-0.033333in}{-0.008840in}}{\pgfqpoint{-0.029821in}{-0.017319in}}{\pgfqpoint{-0.023570in}{-0.023570in}}%
\pgfpathcurveto{\pgfqpoint{-0.017319in}{-0.029821in}}{\pgfqpoint{-0.008840in}{-0.033333in}}{\pgfqpoint{0.000000in}{-0.033333in}}%
\pgfpathlineto{\pgfqpoint{0.000000in}{-0.033333in}}%
\pgfpathclose%
\pgfusepath{stroke,fill}%
}%
\begin{pgfscope}%
\pgfsys@transformshift{2.782812in}{3.874284in}%
\pgfsys@useobject{currentmarker}{}%
\end{pgfscope}%
\end{pgfscope}%
\begin{pgfscope}%
\pgfpathrectangle{\pgfqpoint{0.765000in}{0.660000in}}{\pgfqpoint{4.620000in}{4.620000in}}%
\pgfusepath{clip}%
\pgfsetroundcap%
\pgfsetroundjoin%
\pgfsetlinewidth{1.204500pt}%
\definecolor{currentstroke}{rgb}{1.000000,0.576471,0.309804}%
\pgfsetstrokecolor{currentstroke}%
\pgfsetdash{}{0pt}%
\pgfpathmoveto{\pgfqpoint{3.315198in}{3.726817in}}%
\pgfusepath{stroke}%
\end{pgfscope}%
\begin{pgfscope}%
\pgfpathrectangle{\pgfqpoint{0.765000in}{0.660000in}}{\pgfqpoint{4.620000in}{4.620000in}}%
\pgfusepath{clip}%
\pgfsetbuttcap%
\pgfsetroundjoin%
\definecolor{currentfill}{rgb}{1.000000,0.576471,0.309804}%
\pgfsetfillcolor{currentfill}%
\pgfsetlinewidth{1.003750pt}%
\definecolor{currentstroke}{rgb}{1.000000,0.576471,0.309804}%
\pgfsetstrokecolor{currentstroke}%
\pgfsetdash{}{0pt}%
\pgfsys@defobject{currentmarker}{\pgfqpoint{-0.033333in}{-0.033333in}}{\pgfqpoint{0.033333in}{0.033333in}}{%
\pgfpathmoveto{\pgfqpoint{0.000000in}{-0.033333in}}%
\pgfpathcurveto{\pgfqpoint{0.008840in}{-0.033333in}}{\pgfqpoint{0.017319in}{-0.029821in}}{\pgfqpoint{0.023570in}{-0.023570in}}%
\pgfpathcurveto{\pgfqpoint{0.029821in}{-0.017319in}}{\pgfqpoint{0.033333in}{-0.008840in}}{\pgfqpoint{0.033333in}{0.000000in}}%
\pgfpathcurveto{\pgfqpoint{0.033333in}{0.008840in}}{\pgfqpoint{0.029821in}{0.017319in}}{\pgfqpoint{0.023570in}{0.023570in}}%
\pgfpathcurveto{\pgfqpoint{0.017319in}{0.029821in}}{\pgfqpoint{0.008840in}{0.033333in}}{\pgfqpoint{0.000000in}{0.033333in}}%
\pgfpathcurveto{\pgfqpoint{-0.008840in}{0.033333in}}{\pgfqpoint{-0.017319in}{0.029821in}}{\pgfqpoint{-0.023570in}{0.023570in}}%
\pgfpathcurveto{\pgfqpoint{-0.029821in}{0.017319in}}{\pgfqpoint{-0.033333in}{0.008840in}}{\pgfqpoint{-0.033333in}{0.000000in}}%
\pgfpathcurveto{\pgfqpoint{-0.033333in}{-0.008840in}}{\pgfqpoint{-0.029821in}{-0.017319in}}{\pgfqpoint{-0.023570in}{-0.023570in}}%
\pgfpathcurveto{\pgfqpoint{-0.017319in}{-0.029821in}}{\pgfqpoint{-0.008840in}{-0.033333in}}{\pgfqpoint{0.000000in}{-0.033333in}}%
\pgfpathlineto{\pgfqpoint{0.000000in}{-0.033333in}}%
\pgfpathclose%
\pgfusepath{stroke,fill}%
}%
\begin{pgfscope}%
\pgfsys@transformshift{3.315198in}{3.726817in}%
\pgfsys@useobject{currentmarker}{}%
\end{pgfscope}%
\end{pgfscope}%
\begin{pgfscope}%
\pgfpathrectangle{\pgfqpoint{0.765000in}{0.660000in}}{\pgfqpoint{4.620000in}{4.620000in}}%
\pgfusepath{clip}%
\pgfsetroundcap%
\pgfsetroundjoin%
\pgfsetlinewidth{1.204500pt}%
\definecolor{currentstroke}{rgb}{1.000000,0.576471,0.309804}%
\pgfsetstrokecolor{currentstroke}%
\pgfsetdash{}{0pt}%
\pgfpathmoveto{\pgfqpoint{2.441743in}{2.161035in}}%
\pgfusepath{stroke}%
\end{pgfscope}%
\begin{pgfscope}%
\pgfpathrectangle{\pgfqpoint{0.765000in}{0.660000in}}{\pgfqpoint{4.620000in}{4.620000in}}%
\pgfusepath{clip}%
\pgfsetbuttcap%
\pgfsetroundjoin%
\definecolor{currentfill}{rgb}{1.000000,0.576471,0.309804}%
\pgfsetfillcolor{currentfill}%
\pgfsetlinewidth{1.003750pt}%
\definecolor{currentstroke}{rgb}{1.000000,0.576471,0.309804}%
\pgfsetstrokecolor{currentstroke}%
\pgfsetdash{}{0pt}%
\pgfsys@defobject{currentmarker}{\pgfqpoint{-0.033333in}{-0.033333in}}{\pgfqpoint{0.033333in}{0.033333in}}{%
\pgfpathmoveto{\pgfqpoint{0.000000in}{-0.033333in}}%
\pgfpathcurveto{\pgfqpoint{0.008840in}{-0.033333in}}{\pgfqpoint{0.017319in}{-0.029821in}}{\pgfqpoint{0.023570in}{-0.023570in}}%
\pgfpathcurveto{\pgfqpoint{0.029821in}{-0.017319in}}{\pgfqpoint{0.033333in}{-0.008840in}}{\pgfqpoint{0.033333in}{0.000000in}}%
\pgfpathcurveto{\pgfqpoint{0.033333in}{0.008840in}}{\pgfqpoint{0.029821in}{0.017319in}}{\pgfqpoint{0.023570in}{0.023570in}}%
\pgfpathcurveto{\pgfqpoint{0.017319in}{0.029821in}}{\pgfqpoint{0.008840in}{0.033333in}}{\pgfqpoint{0.000000in}{0.033333in}}%
\pgfpathcurveto{\pgfqpoint{-0.008840in}{0.033333in}}{\pgfqpoint{-0.017319in}{0.029821in}}{\pgfqpoint{-0.023570in}{0.023570in}}%
\pgfpathcurveto{\pgfqpoint{-0.029821in}{0.017319in}}{\pgfqpoint{-0.033333in}{0.008840in}}{\pgfqpoint{-0.033333in}{0.000000in}}%
\pgfpathcurveto{\pgfqpoint{-0.033333in}{-0.008840in}}{\pgfqpoint{-0.029821in}{-0.017319in}}{\pgfqpoint{-0.023570in}{-0.023570in}}%
\pgfpathcurveto{\pgfqpoint{-0.017319in}{-0.029821in}}{\pgfqpoint{-0.008840in}{-0.033333in}}{\pgfqpoint{0.000000in}{-0.033333in}}%
\pgfpathlineto{\pgfqpoint{0.000000in}{-0.033333in}}%
\pgfpathclose%
\pgfusepath{stroke,fill}%
}%
\begin{pgfscope}%
\pgfsys@transformshift{2.441743in}{2.161035in}%
\pgfsys@useobject{currentmarker}{}%
\end{pgfscope}%
\end{pgfscope}%
\begin{pgfscope}%
\pgfpathrectangle{\pgfqpoint{0.765000in}{0.660000in}}{\pgfqpoint{4.620000in}{4.620000in}}%
\pgfusepath{clip}%
\pgfsetroundcap%
\pgfsetroundjoin%
\pgfsetlinewidth{1.204500pt}%
\definecolor{currentstroke}{rgb}{1.000000,0.576471,0.309804}%
\pgfsetstrokecolor{currentstroke}%
\pgfsetdash{}{0pt}%
\pgfpathmoveto{\pgfqpoint{1.957157in}{3.643947in}}%
\pgfusepath{stroke}%
\end{pgfscope}%
\begin{pgfscope}%
\pgfpathrectangle{\pgfqpoint{0.765000in}{0.660000in}}{\pgfqpoint{4.620000in}{4.620000in}}%
\pgfusepath{clip}%
\pgfsetbuttcap%
\pgfsetroundjoin%
\definecolor{currentfill}{rgb}{1.000000,0.576471,0.309804}%
\pgfsetfillcolor{currentfill}%
\pgfsetlinewidth{1.003750pt}%
\definecolor{currentstroke}{rgb}{1.000000,0.576471,0.309804}%
\pgfsetstrokecolor{currentstroke}%
\pgfsetdash{}{0pt}%
\pgfsys@defobject{currentmarker}{\pgfqpoint{-0.033333in}{-0.033333in}}{\pgfqpoint{0.033333in}{0.033333in}}{%
\pgfpathmoveto{\pgfqpoint{0.000000in}{-0.033333in}}%
\pgfpathcurveto{\pgfqpoint{0.008840in}{-0.033333in}}{\pgfqpoint{0.017319in}{-0.029821in}}{\pgfqpoint{0.023570in}{-0.023570in}}%
\pgfpathcurveto{\pgfqpoint{0.029821in}{-0.017319in}}{\pgfqpoint{0.033333in}{-0.008840in}}{\pgfqpoint{0.033333in}{0.000000in}}%
\pgfpathcurveto{\pgfqpoint{0.033333in}{0.008840in}}{\pgfqpoint{0.029821in}{0.017319in}}{\pgfqpoint{0.023570in}{0.023570in}}%
\pgfpathcurveto{\pgfqpoint{0.017319in}{0.029821in}}{\pgfqpoint{0.008840in}{0.033333in}}{\pgfqpoint{0.000000in}{0.033333in}}%
\pgfpathcurveto{\pgfqpoint{-0.008840in}{0.033333in}}{\pgfqpoint{-0.017319in}{0.029821in}}{\pgfqpoint{-0.023570in}{0.023570in}}%
\pgfpathcurveto{\pgfqpoint{-0.029821in}{0.017319in}}{\pgfqpoint{-0.033333in}{0.008840in}}{\pgfqpoint{-0.033333in}{0.000000in}}%
\pgfpathcurveto{\pgfqpoint{-0.033333in}{-0.008840in}}{\pgfqpoint{-0.029821in}{-0.017319in}}{\pgfqpoint{-0.023570in}{-0.023570in}}%
\pgfpathcurveto{\pgfqpoint{-0.017319in}{-0.029821in}}{\pgfqpoint{-0.008840in}{-0.033333in}}{\pgfqpoint{0.000000in}{-0.033333in}}%
\pgfpathlineto{\pgfqpoint{0.000000in}{-0.033333in}}%
\pgfpathclose%
\pgfusepath{stroke,fill}%
}%
\begin{pgfscope}%
\pgfsys@transformshift{1.957157in}{3.643947in}%
\pgfsys@useobject{currentmarker}{}%
\end{pgfscope}%
\end{pgfscope}%
\begin{pgfscope}%
\pgfpathrectangle{\pgfqpoint{0.765000in}{0.660000in}}{\pgfqpoint{4.620000in}{4.620000in}}%
\pgfusepath{clip}%
\pgfsetroundcap%
\pgfsetroundjoin%
\pgfsetlinewidth{1.204500pt}%
\definecolor{currentstroke}{rgb}{1.000000,0.576471,0.309804}%
\pgfsetstrokecolor{currentstroke}%
\pgfsetdash{}{0pt}%
\pgfpathmoveto{\pgfqpoint{2.516361in}{3.336847in}}%
\pgfusepath{stroke}%
\end{pgfscope}%
\begin{pgfscope}%
\pgfpathrectangle{\pgfqpoint{0.765000in}{0.660000in}}{\pgfqpoint{4.620000in}{4.620000in}}%
\pgfusepath{clip}%
\pgfsetbuttcap%
\pgfsetroundjoin%
\definecolor{currentfill}{rgb}{1.000000,0.576471,0.309804}%
\pgfsetfillcolor{currentfill}%
\pgfsetlinewidth{1.003750pt}%
\definecolor{currentstroke}{rgb}{1.000000,0.576471,0.309804}%
\pgfsetstrokecolor{currentstroke}%
\pgfsetdash{}{0pt}%
\pgfsys@defobject{currentmarker}{\pgfqpoint{-0.033333in}{-0.033333in}}{\pgfqpoint{0.033333in}{0.033333in}}{%
\pgfpathmoveto{\pgfqpoint{0.000000in}{-0.033333in}}%
\pgfpathcurveto{\pgfqpoint{0.008840in}{-0.033333in}}{\pgfqpoint{0.017319in}{-0.029821in}}{\pgfqpoint{0.023570in}{-0.023570in}}%
\pgfpathcurveto{\pgfqpoint{0.029821in}{-0.017319in}}{\pgfqpoint{0.033333in}{-0.008840in}}{\pgfqpoint{0.033333in}{0.000000in}}%
\pgfpathcurveto{\pgfqpoint{0.033333in}{0.008840in}}{\pgfqpoint{0.029821in}{0.017319in}}{\pgfqpoint{0.023570in}{0.023570in}}%
\pgfpathcurveto{\pgfqpoint{0.017319in}{0.029821in}}{\pgfqpoint{0.008840in}{0.033333in}}{\pgfqpoint{0.000000in}{0.033333in}}%
\pgfpathcurveto{\pgfqpoint{-0.008840in}{0.033333in}}{\pgfqpoint{-0.017319in}{0.029821in}}{\pgfqpoint{-0.023570in}{0.023570in}}%
\pgfpathcurveto{\pgfqpoint{-0.029821in}{0.017319in}}{\pgfqpoint{-0.033333in}{0.008840in}}{\pgfqpoint{-0.033333in}{0.000000in}}%
\pgfpathcurveto{\pgfqpoint{-0.033333in}{-0.008840in}}{\pgfqpoint{-0.029821in}{-0.017319in}}{\pgfqpoint{-0.023570in}{-0.023570in}}%
\pgfpathcurveto{\pgfqpoint{-0.017319in}{-0.029821in}}{\pgfqpoint{-0.008840in}{-0.033333in}}{\pgfqpoint{0.000000in}{-0.033333in}}%
\pgfpathlineto{\pgfqpoint{0.000000in}{-0.033333in}}%
\pgfpathclose%
\pgfusepath{stroke,fill}%
}%
\begin{pgfscope}%
\pgfsys@transformshift{2.516361in}{3.336847in}%
\pgfsys@useobject{currentmarker}{}%
\end{pgfscope}%
\end{pgfscope}%
\begin{pgfscope}%
\pgfpathrectangle{\pgfqpoint{0.765000in}{0.660000in}}{\pgfqpoint{4.620000in}{4.620000in}}%
\pgfusepath{clip}%
\pgfsetroundcap%
\pgfsetroundjoin%
\pgfsetlinewidth{1.204500pt}%
\definecolor{currentstroke}{rgb}{1.000000,0.576471,0.309804}%
\pgfsetstrokecolor{currentstroke}%
\pgfsetdash{}{0pt}%
\pgfpathmoveto{\pgfqpoint{4.009308in}{3.134299in}}%
\pgfusepath{stroke}%
\end{pgfscope}%
\begin{pgfscope}%
\pgfpathrectangle{\pgfqpoint{0.765000in}{0.660000in}}{\pgfqpoint{4.620000in}{4.620000in}}%
\pgfusepath{clip}%
\pgfsetbuttcap%
\pgfsetroundjoin%
\definecolor{currentfill}{rgb}{1.000000,0.576471,0.309804}%
\pgfsetfillcolor{currentfill}%
\pgfsetlinewidth{1.003750pt}%
\definecolor{currentstroke}{rgb}{1.000000,0.576471,0.309804}%
\pgfsetstrokecolor{currentstroke}%
\pgfsetdash{}{0pt}%
\pgfsys@defobject{currentmarker}{\pgfqpoint{-0.033333in}{-0.033333in}}{\pgfqpoint{0.033333in}{0.033333in}}{%
\pgfpathmoveto{\pgfqpoint{0.000000in}{-0.033333in}}%
\pgfpathcurveto{\pgfqpoint{0.008840in}{-0.033333in}}{\pgfqpoint{0.017319in}{-0.029821in}}{\pgfqpoint{0.023570in}{-0.023570in}}%
\pgfpathcurveto{\pgfqpoint{0.029821in}{-0.017319in}}{\pgfqpoint{0.033333in}{-0.008840in}}{\pgfqpoint{0.033333in}{0.000000in}}%
\pgfpathcurveto{\pgfqpoint{0.033333in}{0.008840in}}{\pgfqpoint{0.029821in}{0.017319in}}{\pgfqpoint{0.023570in}{0.023570in}}%
\pgfpathcurveto{\pgfqpoint{0.017319in}{0.029821in}}{\pgfqpoint{0.008840in}{0.033333in}}{\pgfqpoint{0.000000in}{0.033333in}}%
\pgfpathcurveto{\pgfqpoint{-0.008840in}{0.033333in}}{\pgfqpoint{-0.017319in}{0.029821in}}{\pgfqpoint{-0.023570in}{0.023570in}}%
\pgfpathcurveto{\pgfqpoint{-0.029821in}{0.017319in}}{\pgfqpoint{-0.033333in}{0.008840in}}{\pgfqpoint{-0.033333in}{0.000000in}}%
\pgfpathcurveto{\pgfqpoint{-0.033333in}{-0.008840in}}{\pgfqpoint{-0.029821in}{-0.017319in}}{\pgfqpoint{-0.023570in}{-0.023570in}}%
\pgfpathcurveto{\pgfqpoint{-0.017319in}{-0.029821in}}{\pgfqpoint{-0.008840in}{-0.033333in}}{\pgfqpoint{0.000000in}{-0.033333in}}%
\pgfpathlineto{\pgfqpoint{0.000000in}{-0.033333in}}%
\pgfpathclose%
\pgfusepath{stroke,fill}%
}%
\begin{pgfscope}%
\pgfsys@transformshift{4.009308in}{3.134299in}%
\pgfsys@useobject{currentmarker}{}%
\end{pgfscope}%
\end{pgfscope}%
\begin{pgfscope}%
\pgfpathrectangle{\pgfqpoint{0.765000in}{0.660000in}}{\pgfqpoint{4.620000in}{4.620000in}}%
\pgfusepath{clip}%
\pgfsetroundcap%
\pgfsetroundjoin%
\pgfsetlinewidth{1.204500pt}%
\definecolor{currentstroke}{rgb}{1.000000,0.576471,0.309804}%
\pgfsetstrokecolor{currentstroke}%
\pgfsetdash{}{0pt}%
\pgfpathmoveto{\pgfqpoint{2.925869in}{3.718481in}}%
\pgfusepath{stroke}%
\end{pgfscope}%
\begin{pgfscope}%
\pgfpathrectangle{\pgfqpoint{0.765000in}{0.660000in}}{\pgfqpoint{4.620000in}{4.620000in}}%
\pgfusepath{clip}%
\pgfsetbuttcap%
\pgfsetroundjoin%
\definecolor{currentfill}{rgb}{1.000000,0.576471,0.309804}%
\pgfsetfillcolor{currentfill}%
\pgfsetlinewidth{1.003750pt}%
\definecolor{currentstroke}{rgb}{1.000000,0.576471,0.309804}%
\pgfsetstrokecolor{currentstroke}%
\pgfsetdash{}{0pt}%
\pgfsys@defobject{currentmarker}{\pgfqpoint{-0.033333in}{-0.033333in}}{\pgfqpoint{0.033333in}{0.033333in}}{%
\pgfpathmoveto{\pgfqpoint{0.000000in}{-0.033333in}}%
\pgfpathcurveto{\pgfqpoint{0.008840in}{-0.033333in}}{\pgfqpoint{0.017319in}{-0.029821in}}{\pgfqpoint{0.023570in}{-0.023570in}}%
\pgfpathcurveto{\pgfqpoint{0.029821in}{-0.017319in}}{\pgfqpoint{0.033333in}{-0.008840in}}{\pgfqpoint{0.033333in}{0.000000in}}%
\pgfpathcurveto{\pgfqpoint{0.033333in}{0.008840in}}{\pgfqpoint{0.029821in}{0.017319in}}{\pgfqpoint{0.023570in}{0.023570in}}%
\pgfpathcurveto{\pgfqpoint{0.017319in}{0.029821in}}{\pgfqpoint{0.008840in}{0.033333in}}{\pgfqpoint{0.000000in}{0.033333in}}%
\pgfpathcurveto{\pgfqpoint{-0.008840in}{0.033333in}}{\pgfqpoint{-0.017319in}{0.029821in}}{\pgfqpoint{-0.023570in}{0.023570in}}%
\pgfpathcurveto{\pgfqpoint{-0.029821in}{0.017319in}}{\pgfqpoint{-0.033333in}{0.008840in}}{\pgfqpoint{-0.033333in}{0.000000in}}%
\pgfpathcurveto{\pgfqpoint{-0.033333in}{-0.008840in}}{\pgfqpoint{-0.029821in}{-0.017319in}}{\pgfqpoint{-0.023570in}{-0.023570in}}%
\pgfpathcurveto{\pgfqpoint{-0.017319in}{-0.029821in}}{\pgfqpoint{-0.008840in}{-0.033333in}}{\pgfqpoint{0.000000in}{-0.033333in}}%
\pgfpathlineto{\pgfqpoint{0.000000in}{-0.033333in}}%
\pgfpathclose%
\pgfusepath{stroke,fill}%
}%
\begin{pgfscope}%
\pgfsys@transformshift{2.925869in}{3.718481in}%
\pgfsys@useobject{currentmarker}{}%
\end{pgfscope}%
\end{pgfscope}%
\begin{pgfscope}%
\pgfpathrectangle{\pgfqpoint{0.765000in}{0.660000in}}{\pgfqpoint{4.620000in}{4.620000in}}%
\pgfusepath{clip}%
\pgfsetroundcap%
\pgfsetroundjoin%
\pgfsetlinewidth{1.204500pt}%
\definecolor{currentstroke}{rgb}{1.000000,0.576471,0.309804}%
\pgfsetstrokecolor{currentstroke}%
\pgfsetdash{}{0pt}%
\pgfpathmoveto{\pgfqpoint{3.779344in}{3.637872in}}%
\pgfusepath{stroke}%
\end{pgfscope}%
\begin{pgfscope}%
\pgfpathrectangle{\pgfqpoint{0.765000in}{0.660000in}}{\pgfqpoint{4.620000in}{4.620000in}}%
\pgfusepath{clip}%
\pgfsetbuttcap%
\pgfsetroundjoin%
\definecolor{currentfill}{rgb}{1.000000,0.576471,0.309804}%
\pgfsetfillcolor{currentfill}%
\pgfsetlinewidth{1.003750pt}%
\definecolor{currentstroke}{rgb}{1.000000,0.576471,0.309804}%
\pgfsetstrokecolor{currentstroke}%
\pgfsetdash{}{0pt}%
\pgfsys@defobject{currentmarker}{\pgfqpoint{-0.033333in}{-0.033333in}}{\pgfqpoint{0.033333in}{0.033333in}}{%
\pgfpathmoveto{\pgfqpoint{0.000000in}{-0.033333in}}%
\pgfpathcurveto{\pgfqpoint{0.008840in}{-0.033333in}}{\pgfqpoint{0.017319in}{-0.029821in}}{\pgfqpoint{0.023570in}{-0.023570in}}%
\pgfpathcurveto{\pgfqpoint{0.029821in}{-0.017319in}}{\pgfqpoint{0.033333in}{-0.008840in}}{\pgfqpoint{0.033333in}{0.000000in}}%
\pgfpathcurveto{\pgfqpoint{0.033333in}{0.008840in}}{\pgfqpoint{0.029821in}{0.017319in}}{\pgfqpoint{0.023570in}{0.023570in}}%
\pgfpathcurveto{\pgfqpoint{0.017319in}{0.029821in}}{\pgfqpoint{0.008840in}{0.033333in}}{\pgfqpoint{0.000000in}{0.033333in}}%
\pgfpathcurveto{\pgfqpoint{-0.008840in}{0.033333in}}{\pgfqpoint{-0.017319in}{0.029821in}}{\pgfqpoint{-0.023570in}{0.023570in}}%
\pgfpathcurveto{\pgfqpoint{-0.029821in}{0.017319in}}{\pgfqpoint{-0.033333in}{0.008840in}}{\pgfqpoint{-0.033333in}{0.000000in}}%
\pgfpathcurveto{\pgfqpoint{-0.033333in}{-0.008840in}}{\pgfqpoint{-0.029821in}{-0.017319in}}{\pgfqpoint{-0.023570in}{-0.023570in}}%
\pgfpathcurveto{\pgfqpoint{-0.017319in}{-0.029821in}}{\pgfqpoint{-0.008840in}{-0.033333in}}{\pgfqpoint{0.000000in}{-0.033333in}}%
\pgfpathlineto{\pgfqpoint{0.000000in}{-0.033333in}}%
\pgfpathclose%
\pgfusepath{stroke,fill}%
}%
\begin{pgfscope}%
\pgfsys@transformshift{3.779344in}{3.637872in}%
\pgfsys@useobject{currentmarker}{}%
\end{pgfscope}%
\end{pgfscope}%
\begin{pgfscope}%
\pgfpathrectangle{\pgfqpoint{0.765000in}{0.660000in}}{\pgfqpoint{4.620000in}{4.620000in}}%
\pgfusepath{clip}%
\pgfsetroundcap%
\pgfsetroundjoin%
\pgfsetlinewidth{1.204500pt}%
\definecolor{currentstroke}{rgb}{1.000000,0.576471,0.309804}%
\pgfsetstrokecolor{currentstroke}%
\pgfsetdash{}{0pt}%
\pgfpathmoveto{\pgfqpoint{3.319670in}{3.877094in}}%
\pgfusepath{stroke}%
\end{pgfscope}%
\begin{pgfscope}%
\pgfpathrectangle{\pgfqpoint{0.765000in}{0.660000in}}{\pgfqpoint{4.620000in}{4.620000in}}%
\pgfusepath{clip}%
\pgfsetbuttcap%
\pgfsetroundjoin%
\definecolor{currentfill}{rgb}{1.000000,0.576471,0.309804}%
\pgfsetfillcolor{currentfill}%
\pgfsetlinewidth{1.003750pt}%
\definecolor{currentstroke}{rgb}{1.000000,0.576471,0.309804}%
\pgfsetstrokecolor{currentstroke}%
\pgfsetdash{}{0pt}%
\pgfsys@defobject{currentmarker}{\pgfqpoint{-0.033333in}{-0.033333in}}{\pgfqpoint{0.033333in}{0.033333in}}{%
\pgfpathmoveto{\pgfqpoint{0.000000in}{-0.033333in}}%
\pgfpathcurveto{\pgfqpoint{0.008840in}{-0.033333in}}{\pgfqpoint{0.017319in}{-0.029821in}}{\pgfqpoint{0.023570in}{-0.023570in}}%
\pgfpathcurveto{\pgfqpoint{0.029821in}{-0.017319in}}{\pgfqpoint{0.033333in}{-0.008840in}}{\pgfqpoint{0.033333in}{0.000000in}}%
\pgfpathcurveto{\pgfqpoint{0.033333in}{0.008840in}}{\pgfqpoint{0.029821in}{0.017319in}}{\pgfqpoint{0.023570in}{0.023570in}}%
\pgfpathcurveto{\pgfqpoint{0.017319in}{0.029821in}}{\pgfqpoint{0.008840in}{0.033333in}}{\pgfqpoint{0.000000in}{0.033333in}}%
\pgfpathcurveto{\pgfqpoint{-0.008840in}{0.033333in}}{\pgfqpoint{-0.017319in}{0.029821in}}{\pgfqpoint{-0.023570in}{0.023570in}}%
\pgfpathcurveto{\pgfqpoint{-0.029821in}{0.017319in}}{\pgfqpoint{-0.033333in}{0.008840in}}{\pgfqpoint{-0.033333in}{0.000000in}}%
\pgfpathcurveto{\pgfqpoint{-0.033333in}{-0.008840in}}{\pgfqpoint{-0.029821in}{-0.017319in}}{\pgfqpoint{-0.023570in}{-0.023570in}}%
\pgfpathcurveto{\pgfqpoint{-0.017319in}{-0.029821in}}{\pgfqpoint{-0.008840in}{-0.033333in}}{\pgfqpoint{0.000000in}{-0.033333in}}%
\pgfpathlineto{\pgfqpoint{0.000000in}{-0.033333in}}%
\pgfpathclose%
\pgfusepath{stroke,fill}%
}%
\begin{pgfscope}%
\pgfsys@transformshift{3.319670in}{3.877094in}%
\pgfsys@useobject{currentmarker}{}%
\end{pgfscope}%
\end{pgfscope}%
\begin{pgfscope}%
\pgfpathrectangle{\pgfqpoint{0.765000in}{0.660000in}}{\pgfqpoint{4.620000in}{4.620000in}}%
\pgfusepath{clip}%
\pgfsetroundcap%
\pgfsetroundjoin%
\pgfsetlinewidth{1.204500pt}%
\definecolor{currentstroke}{rgb}{1.000000,0.576471,0.309804}%
\pgfsetstrokecolor{currentstroke}%
\pgfsetdash{}{0pt}%
\pgfpathmoveto{\pgfqpoint{3.288908in}{3.839854in}}%
\pgfusepath{stroke}%
\end{pgfscope}%
\begin{pgfscope}%
\pgfpathrectangle{\pgfqpoint{0.765000in}{0.660000in}}{\pgfqpoint{4.620000in}{4.620000in}}%
\pgfusepath{clip}%
\pgfsetbuttcap%
\pgfsetroundjoin%
\definecolor{currentfill}{rgb}{1.000000,0.576471,0.309804}%
\pgfsetfillcolor{currentfill}%
\pgfsetlinewidth{1.003750pt}%
\definecolor{currentstroke}{rgb}{1.000000,0.576471,0.309804}%
\pgfsetstrokecolor{currentstroke}%
\pgfsetdash{}{0pt}%
\pgfsys@defobject{currentmarker}{\pgfqpoint{-0.033333in}{-0.033333in}}{\pgfqpoint{0.033333in}{0.033333in}}{%
\pgfpathmoveto{\pgfqpoint{0.000000in}{-0.033333in}}%
\pgfpathcurveto{\pgfqpoint{0.008840in}{-0.033333in}}{\pgfqpoint{0.017319in}{-0.029821in}}{\pgfqpoint{0.023570in}{-0.023570in}}%
\pgfpathcurveto{\pgfqpoint{0.029821in}{-0.017319in}}{\pgfqpoint{0.033333in}{-0.008840in}}{\pgfqpoint{0.033333in}{0.000000in}}%
\pgfpathcurveto{\pgfqpoint{0.033333in}{0.008840in}}{\pgfqpoint{0.029821in}{0.017319in}}{\pgfqpoint{0.023570in}{0.023570in}}%
\pgfpathcurveto{\pgfqpoint{0.017319in}{0.029821in}}{\pgfqpoint{0.008840in}{0.033333in}}{\pgfqpoint{0.000000in}{0.033333in}}%
\pgfpathcurveto{\pgfqpoint{-0.008840in}{0.033333in}}{\pgfqpoint{-0.017319in}{0.029821in}}{\pgfqpoint{-0.023570in}{0.023570in}}%
\pgfpathcurveto{\pgfqpoint{-0.029821in}{0.017319in}}{\pgfqpoint{-0.033333in}{0.008840in}}{\pgfqpoint{-0.033333in}{0.000000in}}%
\pgfpathcurveto{\pgfqpoint{-0.033333in}{-0.008840in}}{\pgfqpoint{-0.029821in}{-0.017319in}}{\pgfqpoint{-0.023570in}{-0.023570in}}%
\pgfpathcurveto{\pgfqpoint{-0.017319in}{-0.029821in}}{\pgfqpoint{-0.008840in}{-0.033333in}}{\pgfqpoint{0.000000in}{-0.033333in}}%
\pgfpathlineto{\pgfqpoint{0.000000in}{-0.033333in}}%
\pgfpathclose%
\pgfusepath{stroke,fill}%
}%
\begin{pgfscope}%
\pgfsys@transformshift{3.288908in}{3.839854in}%
\pgfsys@useobject{currentmarker}{}%
\end{pgfscope}%
\end{pgfscope}%
\begin{pgfscope}%
\pgfpathrectangle{\pgfqpoint{0.765000in}{0.660000in}}{\pgfqpoint{4.620000in}{4.620000in}}%
\pgfusepath{clip}%
\pgfsetroundcap%
\pgfsetroundjoin%
\pgfsetlinewidth{1.204500pt}%
\definecolor{currentstroke}{rgb}{1.000000,0.576471,0.309804}%
\pgfsetstrokecolor{currentstroke}%
\pgfsetdash{}{0pt}%
\pgfpathmoveto{\pgfqpoint{2.710483in}{3.473621in}}%
\pgfusepath{stroke}%
\end{pgfscope}%
\begin{pgfscope}%
\pgfpathrectangle{\pgfqpoint{0.765000in}{0.660000in}}{\pgfqpoint{4.620000in}{4.620000in}}%
\pgfusepath{clip}%
\pgfsetbuttcap%
\pgfsetroundjoin%
\definecolor{currentfill}{rgb}{1.000000,0.576471,0.309804}%
\pgfsetfillcolor{currentfill}%
\pgfsetlinewidth{1.003750pt}%
\definecolor{currentstroke}{rgb}{1.000000,0.576471,0.309804}%
\pgfsetstrokecolor{currentstroke}%
\pgfsetdash{}{0pt}%
\pgfsys@defobject{currentmarker}{\pgfqpoint{-0.033333in}{-0.033333in}}{\pgfqpoint{0.033333in}{0.033333in}}{%
\pgfpathmoveto{\pgfqpoint{0.000000in}{-0.033333in}}%
\pgfpathcurveto{\pgfqpoint{0.008840in}{-0.033333in}}{\pgfqpoint{0.017319in}{-0.029821in}}{\pgfqpoint{0.023570in}{-0.023570in}}%
\pgfpathcurveto{\pgfqpoint{0.029821in}{-0.017319in}}{\pgfqpoint{0.033333in}{-0.008840in}}{\pgfqpoint{0.033333in}{0.000000in}}%
\pgfpathcurveto{\pgfqpoint{0.033333in}{0.008840in}}{\pgfqpoint{0.029821in}{0.017319in}}{\pgfqpoint{0.023570in}{0.023570in}}%
\pgfpathcurveto{\pgfqpoint{0.017319in}{0.029821in}}{\pgfqpoint{0.008840in}{0.033333in}}{\pgfqpoint{0.000000in}{0.033333in}}%
\pgfpathcurveto{\pgfqpoint{-0.008840in}{0.033333in}}{\pgfqpoint{-0.017319in}{0.029821in}}{\pgfqpoint{-0.023570in}{0.023570in}}%
\pgfpathcurveto{\pgfqpoint{-0.029821in}{0.017319in}}{\pgfqpoint{-0.033333in}{0.008840in}}{\pgfqpoint{-0.033333in}{0.000000in}}%
\pgfpathcurveto{\pgfqpoint{-0.033333in}{-0.008840in}}{\pgfqpoint{-0.029821in}{-0.017319in}}{\pgfqpoint{-0.023570in}{-0.023570in}}%
\pgfpathcurveto{\pgfqpoint{-0.017319in}{-0.029821in}}{\pgfqpoint{-0.008840in}{-0.033333in}}{\pgfqpoint{0.000000in}{-0.033333in}}%
\pgfpathlineto{\pgfqpoint{0.000000in}{-0.033333in}}%
\pgfpathclose%
\pgfusepath{stroke,fill}%
}%
\begin{pgfscope}%
\pgfsys@transformshift{2.710483in}{3.473621in}%
\pgfsys@useobject{currentmarker}{}%
\end{pgfscope}%
\end{pgfscope}%
\begin{pgfscope}%
\pgfpathrectangle{\pgfqpoint{0.765000in}{0.660000in}}{\pgfqpoint{4.620000in}{4.620000in}}%
\pgfusepath{clip}%
\pgfsetroundcap%
\pgfsetroundjoin%
\pgfsetlinewidth{1.204500pt}%
\definecolor{currentstroke}{rgb}{1.000000,0.576471,0.309804}%
\pgfsetstrokecolor{currentstroke}%
\pgfsetdash{}{0pt}%
\pgfpathmoveto{\pgfqpoint{3.717566in}{3.664984in}}%
\pgfusepath{stroke}%
\end{pgfscope}%
\begin{pgfscope}%
\pgfpathrectangle{\pgfqpoint{0.765000in}{0.660000in}}{\pgfqpoint{4.620000in}{4.620000in}}%
\pgfusepath{clip}%
\pgfsetbuttcap%
\pgfsetroundjoin%
\definecolor{currentfill}{rgb}{1.000000,0.576471,0.309804}%
\pgfsetfillcolor{currentfill}%
\pgfsetlinewidth{1.003750pt}%
\definecolor{currentstroke}{rgb}{1.000000,0.576471,0.309804}%
\pgfsetstrokecolor{currentstroke}%
\pgfsetdash{}{0pt}%
\pgfsys@defobject{currentmarker}{\pgfqpoint{-0.033333in}{-0.033333in}}{\pgfqpoint{0.033333in}{0.033333in}}{%
\pgfpathmoveto{\pgfqpoint{0.000000in}{-0.033333in}}%
\pgfpathcurveto{\pgfqpoint{0.008840in}{-0.033333in}}{\pgfqpoint{0.017319in}{-0.029821in}}{\pgfqpoint{0.023570in}{-0.023570in}}%
\pgfpathcurveto{\pgfqpoint{0.029821in}{-0.017319in}}{\pgfqpoint{0.033333in}{-0.008840in}}{\pgfqpoint{0.033333in}{0.000000in}}%
\pgfpathcurveto{\pgfqpoint{0.033333in}{0.008840in}}{\pgfqpoint{0.029821in}{0.017319in}}{\pgfqpoint{0.023570in}{0.023570in}}%
\pgfpathcurveto{\pgfqpoint{0.017319in}{0.029821in}}{\pgfqpoint{0.008840in}{0.033333in}}{\pgfqpoint{0.000000in}{0.033333in}}%
\pgfpathcurveto{\pgfqpoint{-0.008840in}{0.033333in}}{\pgfqpoint{-0.017319in}{0.029821in}}{\pgfqpoint{-0.023570in}{0.023570in}}%
\pgfpathcurveto{\pgfqpoint{-0.029821in}{0.017319in}}{\pgfqpoint{-0.033333in}{0.008840in}}{\pgfqpoint{-0.033333in}{0.000000in}}%
\pgfpathcurveto{\pgfqpoint{-0.033333in}{-0.008840in}}{\pgfqpoint{-0.029821in}{-0.017319in}}{\pgfqpoint{-0.023570in}{-0.023570in}}%
\pgfpathcurveto{\pgfqpoint{-0.017319in}{-0.029821in}}{\pgfqpoint{-0.008840in}{-0.033333in}}{\pgfqpoint{0.000000in}{-0.033333in}}%
\pgfpathlineto{\pgfqpoint{0.000000in}{-0.033333in}}%
\pgfpathclose%
\pgfusepath{stroke,fill}%
}%
\begin{pgfscope}%
\pgfsys@transformshift{3.717566in}{3.664984in}%
\pgfsys@useobject{currentmarker}{}%
\end{pgfscope}%
\end{pgfscope}%
\begin{pgfscope}%
\pgfpathrectangle{\pgfqpoint{0.765000in}{0.660000in}}{\pgfqpoint{4.620000in}{4.620000in}}%
\pgfusepath{clip}%
\pgfsetroundcap%
\pgfsetroundjoin%
\pgfsetlinewidth{1.204500pt}%
\definecolor{currentstroke}{rgb}{1.000000,0.576471,0.309804}%
\pgfsetstrokecolor{currentstroke}%
\pgfsetdash{}{0pt}%
\pgfpathmoveto{\pgfqpoint{3.035994in}{2.387862in}}%
\pgfusepath{stroke}%
\end{pgfscope}%
\begin{pgfscope}%
\pgfpathrectangle{\pgfqpoint{0.765000in}{0.660000in}}{\pgfqpoint{4.620000in}{4.620000in}}%
\pgfusepath{clip}%
\pgfsetbuttcap%
\pgfsetroundjoin%
\definecolor{currentfill}{rgb}{1.000000,0.576471,0.309804}%
\pgfsetfillcolor{currentfill}%
\pgfsetlinewidth{1.003750pt}%
\definecolor{currentstroke}{rgb}{1.000000,0.576471,0.309804}%
\pgfsetstrokecolor{currentstroke}%
\pgfsetdash{}{0pt}%
\pgfsys@defobject{currentmarker}{\pgfqpoint{-0.033333in}{-0.033333in}}{\pgfqpoint{0.033333in}{0.033333in}}{%
\pgfpathmoveto{\pgfqpoint{0.000000in}{-0.033333in}}%
\pgfpathcurveto{\pgfqpoint{0.008840in}{-0.033333in}}{\pgfqpoint{0.017319in}{-0.029821in}}{\pgfqpoint{0.023570in}{-0.023570in}}%
\pgfpathcurveto{\pgfqpoint{0.029821in}{-0.017319in}}{\pgfqpoint{0.033333in}{-0.008840in}}{\pgfqpoint{0.033333in}{0.000000in}}%
\pgfpathcurveto{\pgfqpoint{0.033333in}{0.008840in}}{\pgfqpoint{0.029821in}{0.017319in}}{\pgfqpoint{0.023570in}{0.023570in}}%
\pgfpathcurveto{\pgfqpoint{0.017319in}{0.029821in}}{\pgfqpoint{0.008840in}{0.033333in}}{\pgfqpoint{0.000000in}{0.033333in}}%
\pgfpathcurveto{\pgfqpoint{-0.008840in}{0.033333in}}{\pgfqpoint{-0.017319in}{0.029821in}}{\pgfqpoint{-0.023570in}{0.023570in}}%
\pgfpathcurveto{\pgfqpoint{-0.029821in}{0.017319in}}{\pgfqpoint{-0.033333in}{0.008840in}}{\pgfqpoint{-0.033333in}{0.000000in}}%
\pgfpathcurveto{\pgfqpoint{-0.033333in}{-0.008840in}}{\pgfqpoint{-0.029821in}{-0.017319in}}{\pgfqpoint{-0.023570in}{-0.023570in}}%
\pgfpathcurveto{\pgfqpoint{-0.017319in}{-0.029821in}}{\pgfqpoint{-0.008840in}{-0.033333in}}{\pgfqpoint{0.000000in}{-0.033333in}}%
\pgfpathlineto{\pgfqpoint{0.000000in}{-0.033333in}}%
\pgfpathclose%
\pgfusepath{stroke,fill}%
}%
\begin{pgfscope}%
\pgfsys@transformshift{3.035994in}{2.387862in}%
\pgfsys@useobject{currentmarker}{}%
\end{pgfscope}%
\end{pgfscope}%
\begin{pgfscope}%
\pgfpathrectangle{\pgfqpoint{0.765000in}{0.660000in}}{\pgfqpoint{4.620000in}{4.620000in}}%
\pgfusepath{clip}%
\pgfsetroundcap%
\pgfsetroundjoin%
\pgfsetlinewidth{1.204500pt}%
\definecolor{currentstroke}{rgb}{1.000000,0.576471,0.309804}%
\pgfsetstrokecolor{currentstroke}%
\pgfsetdash{}{0pt}%
\pgfpathmoveto{\pgfqpoint{3.969934in}{2.866811in}}%
\pgfusepath{stroke}%
\end{pgfscope}%
\begin{pgfscope}%
\pgfpathrectangle{\pgfqpoint{0.765000in}{0.660000in}}{\pgfqpoint{4.620000in}{4.620000in}}%
\pgfusepath{clip}%
\pgfsetbuttcap%
\pgfsetroundjoin%
\definecolor{currentfill}{rgb}{1.000000,0.576471,0.309804}%
\pgfsetfillcolor{currentfill}%
\pgfsetlinewidth{1.003750pt}%
\definecolor{currentstroke}{rgb}{1.000000,0.576471,0.309804}%
\pgfsetstrokecolor{currentstroke}%
\pgfsetdash{}{0pt}%
\pgfsys@defobject{currentmarker}{\pgfqpoint{-0.033333in}{-0.033333in}}{\pgfqpoint{0.033333in}{0.033333in}}{%
\pgfpathmoveto{\pgfqpoint{0.000000in}{-0.033333in}}%
\pgfpathcurveto{\pgfqpoint{0.008840in}{-0.033333in}}{\pgfqpoint{0.017319in}{-0.029821in}}{\pgfqpoint{0.023570in}{-0.023570in}}%
\pgfpathcurveto{\pgfqpoint{0.029821in}{-0.017319in}}{\pgfqpoint{0.033333in}{-0.008840in}}{\pgfqpoint{0.033333in}{0.000000in}}%
\pgfpathcurveto{\pgfqpoint{0.033333in}{0.008840in}}{\pgfqpoint{0.029821in}{0.017319in}}{\pgfqpoint{0.023570in}{0.023570in}}%
\pgfpathcurveto{\pgfqpoint{0.017319in}{0.029821in}}{\pgfqpoint{0.008840in}{0.033333in}}{\pgfqpoint{0.000000in}{0.033333in}}%
\pgfpathcurveto{\pgfqpoint{-0.008840in}{0.033333in}}{\pgfqpoint{-0.017319in}{0.029821in}}{\pgfqpoint{-0.023570in}{0.023570in}}%
\pgfpathcurveto{\pgfqpoint{-0.029821in}{0.017319in}}{\pgfqpoint{-0.033333in}{0.008840in}}{\pgfqpoint{-0.033333in}{0.000000in}}%
\pgfpathcurveto{\pgfqpoint{-0.033333in}{-0.008840in}}{\pgfqpoint{-0.029821in}{-0.017319in}}{\pgfqpoint{-0.023570in}{-0.023570in}}%
\pgfpathcurveto{\pgfqpoint{-0.017319in}{-0.029821in}}{\pgfqpoint{-0.008840in}{-0.033333in}}{\pgfqpoint{0.000000in}{-0.033333in}}%
\pgfpathlineto{\pgfqpoint{0.000000in}{-0.033333in}}%
\pgfpathclose%
\pgfusepath{stroke,fill}%
}%
\begin{pgfscope}%
\pgfsys@transformshift{3.969934in}{2.866811in}%
\pgfsys@useobject{currentmarker}{}%
\end{pgfscope}%
\end{pgfscope}%
\begin{pgfscope}%
\pgfpathrectangle{\pgfqpoint{0.765000in}{0.660000in}}{\pgfqpoint{4.620000in}{4.620000in}}%
\pgfusepath{clip}%
\pgfsetroundcap%
\pgfsetroundjoin%
\pgfsetlinewidth{1.204500pt}%
\definecolor{currentstroke}{rgb}{1.000000,0.576471,0.309804}%
\pgfsetstrokecolor{currentstroke}%
\pgfsetdash{}{0pt}%
\pgfpathmoveto{\pgfqpoint{3.053692in}{3.568017in}}%
\pgfusepath{stroke}%
\end{pgfscope}%
\begin{pgfscope}%
\pgfpathrectangle{\pgfqpoint{0.765000in}{0.660000in}}{\pgfqpoint{4.620000in}{4.620000in}}%
\pgfusepath{clip}%
\pgfsetbuttcap%
\pgfsetroundjoin%
\definecolor{currentfill}{rgb}{1.000000,0.576471,0.309804}%
\pgfsetfillcolor{currentfill}%
\pgfsetlinewidth{1.003750pt}%
\definecolor{currentstroke}{rgb}{1.000000,0.576471,0.309804}%
\pgfsetstrokecolor{currentstroke}%
\pgfsetdash{}{0pt}%
\pgfsys@defobject{currentmarker}{\pgfqpoint{-0.033333in}{-0.033333in}}{\pgfqpoint{0.033333in}{0.033333in}}{%
\pgfpathmoveto{\pgfqpoint{0.000000in}{-0.033333in}}%
\pgfpathcurveto{\pgfqpoint{0.008840in}{-0.033333in}}{\pgfqpoint{0.017319in}{-0.029821in}}{\pgfqpoint{0.023570in}{-0.023570in}}%
\pgfpathcurveto{\pgfqpoint{0.029821in}{-0.017319in}}{\pgfqpoint{0.033333in}{-0.008840in}}{\pgfqpoint{0.033333in}{0.000000in}}%
\pgfpathcurveto{\pgfqpoint{0.033333in}{0.008840in}}{\pgfqpoint{0.029821in}{0.017319in}}{\pgfqpoint{0.023570in}{0.023570in}}%
\pgfpathcurveto{\pgfqpoint{0.017319in}{0.029821in}}{\pgfqpoint{0.008840in}{0.033333in}}{\pgfqpoint{0.000000in}{0.033333in}}%
\pgfpathcurveto{\pgfqpoint{-0.008840in}{0.033333in}}{\pgfqpoint{-0.017319in}{0.029821in}}{\pgfqpoint{-0.023570in}{0.023570in}}%
\pgfpathcurveto{\pgfqpoint{-0.029821in}{0.017319in}}{\pgfqpoint{-0.033333in}{0.008840in}}{\pgfqpoint{-0.033333in}{0.000000in}}%
\pgfpathcurveto{\pgfqpoint{-0.033333in}{-0.008840in}}{\pgfqpoint{-0.029821in}{-0.017319in}}{\pgfqpoint{-0.023570in}{-0.023570in}}%
\pgfpathcurveto{\pgfqpoint{-0.017319in}{-0.029821in}}{\pgfqpoint{-0.008840in}{-0.033333in}}{\pgfqpoint{0.000000in}{-0.033333in}}%
\pgfpathlineto{\pgfqpoint{0.000000in}{-0.033333in}}%
\pgfpathclose%
\pgfusepath{stroke,fill}%
}%
\begin{pgfscope}%
\pgfsys@transformshift{3.053692in}{3.568017in}%
\pgfsys@useobject{currentmarker}{}%
\end{pgfscope}%
\end{pgfscope}%
\begin{pgfscope}%
\pgfpathrectangle{\pgfqpoint{0.765000in}{0.660000in}}{\pgfqpoint{4.620000in}{4.620000in}}%
\pgfusepath{clip}%
\pgfsetroundcap%
\pgfsetroundjoin%
\pgfsetlinewidth{1.204500pt}%
\definecolor{currentstroke}{rgb}{1.000000,0.576471,0.309804}%
\pgfsetstrokecolor{currentstroke}%
\pgfsetdash{}{0pt}%
\pgfpathmoveto{\pgfqpoint{2.794256in}{2.614765in}}%
\pgfusepath{stroke}%
\end{pgfscope}%
\begin{pgfscope}%
\pgfpathrectangle{\pgfqpoint{0.765000in}{0.660000in}}{\pgfqpoint{4.620000in}{4.620000in}}%
\pgfusepath{clip}%
\pgfsetbuttcap%
\pgfsetroundjoin%
\definecolor{currentfill}{rgb}{1.000000,0.576471,0.309804}%
\pgfsetfillcolor{currentfill}%
\pgfsetlinewidth{1.003750pt}%
\definecolor{currentstroke}{rgb}{1.000000,0.576471,0.309804}%
\pgfsetstrokecolor{currentstroke}%
\pgfsetdash{}{0pt}%
\pgfsys@defobject{currentmarker}{\pgfqpoint{-0.033333in}{-0.033333in}}{\pgfqpoint{0.033333in}{0.033333in}}{%
\pgfpathmoveto{\pgfqpoint{0.000000in}{-0.033333in}}%
\pgfpathcurveto{\pgfqpoint{0.008840in}{-0.033333in}}{\pgfqpoint{0.017319in}{-0.029821in}}{\pgfqpoint{0.023570in}{-0.023570in}}%
\pgfpathcurveto{\pgfqpoint{0.029821in}{-0.017319in}}{\pgfqpoint{0.033333in}{-0.008840in}}{\pgfqpoint{0.033333in}{0.000000in}}%
\pgfpathcurveto{\pgfqpoint{0.033333in}{0.008840in}}{\pgfqpoint{0.029821in}{0.017319in}}{\pgfqpoint{0.023570in}{0.023570in}}%
\pgfpathcurveto{\pgfqpoint{0.017319in}{0.029821in}}{\pgfqpoint{0.008840in}{0.033333in}}{\pgfqpoint{0.000000in}{0.033333in}}%
\pgfpathcurveto{\pgfqpoint{-0.008840in}{0.033333in}}{\pgfqpoint{-0.017319in}{0.029821in}}{\pgfqpoint{-0.023570in}{0.023570in}}%
\pgfpathcurveto{\pgfqpoint{-0.029821in}{0.017319in}}{\pgfqpoint{-0.033333in}{0.008840in}}{\pgfqpoint{-0.033333in}{0.000000in}}%
\pgfpathcurveto{\pgfqpoint{-0.033333in}{-0.008840in}}{\pgfqpoint{-0.029821in}{-0.017319in}}{\pgfqpoint{-0.023570in}{-0.023570in}}%
\pgfpathcurveto{\pgfqpoint{-0.017319in}{-0.029821in}}{\pgfqpoint{-0.008840in}{-0.033333in}}{\pgfqpoint{0.000000in}{-0.033333in}}%
\pgfpathlineto{\pgfqpoint{0.000000in}{-0.033333in}}%
\pgfpathclose%
\pgfusepath{stroke,fill}%
}%
\begin{pgfscope}%
\pgfsys@transformshift{2.794256in}{2.614765in}%
\pgfsys@useobject{currentmarker}{}%
\end{pgfscope}%
\end{pgfscope}%
\begin{pgfscope}%
\pgfpathrectangle{\pgfqpoint{0.765000in}{0.660000in}}{\pgfqpoint{4.620000in}{4.620000in}}%
\pgfusepath{clip}%
\pgfsetroundcap%
\pgfsetroundjoin%
\pgfsetlinewidth{1.204500pt}%
\definecolor{currentstroke}{rgb}{1.000000,0.576471,0.309804}%
\pgfsetstrokecolor{currentstroke}%
\pgfsetdash{}{0pt}%
\pgfpathmoveto{\pgfqpoint{2.867029in}{3.407676in}}%
\pgfusepath{stroke}%
\end{pgfscope}%
\begin{pgfscope}%
\pgfpathrectangle{\pgfqpoint{0.765000in}{0.660000in}}{\pgfqpoint{4.620000in}{4.620000in}}%
\pgfusepath{clip}%
\pgfsetbuttcap%
\pgfsetroundjoin%
\definecolor{currentfill}{rgb}{1.000000,0.576471,0.309804}%
\pgfsetfillcolor{currentfill}%
\pgfsetlinewidth{1.003750pt}%
\definecolor{currentstroke}{rgb}{1.000000,0.576471,0.309804}%
\pgfsetstrokecolor{currentstroke}%
\pgfsetdash{}{0pt}%
\pgfsys@defobject{currentmarker}{\pgfqpoint{-0.033333in}{-0.033333in}}{\pgfqpoint{0.033333in}{0.033333in}}{%
\pgfpathmoveto{\pgfqpoint{0.000000in}{-0.033333in}}%
\pgfpathcurveto{\pgfqpoint{0.008840in}{-0.033333in}}{\pgfqpoint{0.017319in}{-0.029821in}}{\pgfqpoint{0.023570in}{-0.023570in}}%
\pgfpathcurveto{\pgfqpoint{0.029821in}{-0.017319in}}{\pgfqpoint{0.033333in}{-0.008840in}}{\pgfqpoint{0.033333in}{0.000000in}}%
\pgfpathcurveto{\pgfqpoint{0.033333in}{0.008840in}}{\pgfqpoint{0.029821in}{0.017319in}}{\pgfqpoint{0.023570in}{0.023570in}}%
\pgfpathcurveto{\pgfqpoint{0.017319in}{0.029821in}}{\pgfqpoint{0.008840in}{0.033333in}}{\pgfqpoint{0.000000in}{0.033333in}}%
\pgfpathcurveto{\pgfqpoint{-0.008840in}{0.033333in}}{\pgfqpoint{-0.017319in}{0.029821in}}{\pgfqpoint{-0.023570in}{0.023570in}}%
\pgfpathcurveto{\pgfqpoint{-0.029821in}{0.017319in}}{\pgfqpoint{-0.033333in}{0.008840in}}{\pgfqpoint{-0.033333in}{0.000000in}}%
\pgfpathcurveto{\pgfqpoint{-0.033333in}{-0.008840in}}{\pgfqpoint{-0.029821in}{-0.017319in}}{\pgfqpoint{-0.023570in}{-0.023570in}}%
\pgfpathcurveto{\pgfqpoint{-0.017319in}{-0.029821in}}{\pgfqpoint{-0.008840in}{-0.033333in}}{\pgfqpoint{0.000000in}{-0.033333in}}%
\pgfpathlineto{\pgfqpoint{0.000000in}{-0.033333in}}%
\pgfpathclose%
\pgfusepath{stroke,fill}%
}%
\begin{pgfscope}%
\pgfsys@transformshift{2.867029in}{3.407676in}%
\pgfsys@useobject{currentmarker}{}%
\end{pgfscope}%
\end{pgfscope}%
\begin{pgfscope}%
\pgfpathrectangle{\pgfqpoint{0.765000in}{0.660000in}}{\pgfqpoint{4.620000in}{4.620000in}}%
\pgfusepath{clip}%
\pgfsetroundcap%
\pgfsetroundjoin%
\pgfsetlinewidth{1.204500pt}%
\definecolor{currentstroke}{rgb}{1.000000,0.576471,0.309804}%
\pgfsetstrokecolor{currentstroke}%
\pgfsetdash{}{0pt}%
\pgfpathmoveto{\pgfqpoint{2.444094in}{3.580788in}}%
\pgfusepath{stroke}%
\end{pgfscope}%
\begin{pgfscope}%
\pgfpathrectangle{\pgfqpoint{0.765000in}{0.660000in}}{\pgfqpoint{4.620000in}{4.620000in}}%
\pgfusepath{clip}%
\pgfsetbuttcap%
\pgfsetroundjoin%
\definecolor{currentfill}{rgb}{1.000000,0.576471,0.309804}%
\pgfsetfillcolor{currentfill}%
\pgfsetlinewidth{1.003750pt}%
\definecolor{currentstroke}{rgb}{1.000000,0.576471,0.309804}%
\pgfsetstrokecolor{currentstroke}%
\pgfsetdash{}{0pt}%
\pgfsys@defobject{currentmarker}{\pgfqpoint{-0.033333in}{-0.033333in}}{\pgfqpoint{0.033333in}{0.033333in}}{%
\pgfpathmoveto{\pgfqpoint{0.000000in}{-0.033333in}}%
\pgfpathcurveto{\pgfqpoint{0.008840in}{-0.033333in}}{\pgfqpoint{0.017319in}{-0.029821in}}{\pgfqpoint{0.023570in}{-0.023570in}}%
\pgfpathcurveto{\pgfqpoint{0.029821in}{-0.017319in}}{\pgfqpoint{0.033333in}{-0.008840in}}{\pgfqpoint{0.033333in}{0.000000in}}%
\pgfpathcurveto{\pgfqpoint{0.033333in}{0.008840in}}{\pgfqpoint{0.029821in}{0.017319in}}{\pgfqpoint{0.023570in}{0.023570in}}%
\pgfpathcurveto{\pgfqpoint{0.017319in}{0.029821in}}{\pgfqpoint{0.008840in}{0.033333in}}{\pgfqpoint{0.000000in}{0.033333in}}%
\pgfpathcurveto{\pgfqpoint{-0.008840in}{0.033333in}}{\pgfqpoint{-0.017319in}{0.029821in}}{\pgfqpoint{-0.023570in}{0.023570in}}%
\pgfpathcurveto{\pgfqpoint{-0.029821in}{0.017319in}}{\pgfqpoint{-0.033333in}{0.008840in}}{\pgfqpoint{-0.033333in}{0.000000in}}%
\pgfpathcurveto{\pgfqpoint{-0.033333in}{-0.008840in}}{\pgfqpoint{-0.029821in}{-0.017319in}}{\pgfqpoint{-0.023570in}{-0.023570in}}%
\pgfpathcurveto{\pgfqpoint{-0.017319in}{-0.029821in}}{\pgfqpoint{-0.008840in}{-0.033333in}}{\pgfqpoint{0.000000in}{-0.033333in}}%
\pgfpathlineto{\pgfqpoint{0.000000in}{-0.033333in}}%
\pgfpathclose%
\pgfusepath{stroke,fill}%
}%
\begin{pgfscope}%
\pgfsys@transformshift{2.444094in}{3.580788in}%
\pgfsys@useobject{currentmarker}{}%
\end{pgfscope}%
\end{pgfscope}%
\begin{pgfscope}%
\pgfpathrectangle{\pgfqpoint{0.765000in}{0.660000in}}{\pgfqpoint{4.620000in}{4.620000in}}%
\pgfusepath{clip}%
\pgfsetroundcap%
\pgfsetroundjoin%
\pgfsetlinewidth{1.204500pt}%
\definecolor{currentstroke}{rgb}{1.000000,0.576471,0.309804}%
\pgfsetstrokecolor{currentstroke}%
\pgfsetdash{}{0pt}%
\pgfpathmoveto{\pgfqpoint{2.849491in}{2.575849in}}%
\pgfusepath{stroke}%
\end{pgfscope}%
\begin{pgfscope}%
\pgfpathrectangle{\pgfqpoint{0.765000in}{0.660000in}}{\pgfqpoint{4.620000in}{4.620000in}}%
\pgfusepath{clip}%
\pgfsetbuttcap%
\pgfsetroundjoin%
\definecolor{currentfill}{rgb}{1.000000,0.576471,0.309804}%
\pgfsetfillcolor{currentfill}%
\pgfsetlinewidth{1.003750pt}%
\definecolor{currentstroke}{rgb}{1.000000,0.576471,0.309804}%
\pgfsetstrokecolor{currentstroke}%
\pgfsetdash{}{0pt}%
\pgfsys@defobject{currentmarker}{\pgfqpoint{-0.033333in}{-0.033333in}}{\pgfqpoint{0.033333in}{0.033333in}}{%
\pgfpathmoveto{\pgfqpoint{0.000000in}{-0.033333in}}%
\pgfpathcurveto{\pgfqpoint{0.008840in}{-0.033333in}}{\pgfqpoint{0.017319in}{-0.029821in}}{\pgfqpoint{0.023570in}{-0.023570in}}%
\pgfpathcurveto{\pgfqpoint{0.029821in}{-0.017319in}}{\pgfqpoint{0.033333in}{-0.008840in}}{\pgfqpoint{0.033333in}{0.000000in}}%
\pgfpathcurveto{\pgfqpoint{0.033333in}{0.008840in}}{\pgfqpoint{0.029821in}{0.017319in}}{\pgfqpoint{0.023570in}{0.023570in}}%
\pgfpathcurveto{\pgfqpoint{0.017319in}{0.029821in}}{\pgfqpoint{0.008840in}{0.033333in}}{\pgfqpoint{0.000000in}{0.033333in}}%
\pgfpathcurveto{\pgfqpoint{-0.008840in}{0.033333in}}{\pgfqpoint{-0.017319in}{0.029821in}}{\pgfqpoint{-0.023570in}{0.023570in}}%
\pgfpathcurveto{\pgfqpoint{-0.029821in}{0.017319in}}{\pgfqpoint{-0.033333in}{0.008840in}}{\pgfqpoint{-0.033333in}{0.000000in}}%
\pgfpathcurveto{\pgfqpoint{-0.033333in}{-0.008840in}}{\pgfqpoint{-0.029821in}{-0.017319in}}{\pgfqpoint{-0.023570in}{-0.023570in}}%
\pgfpathcurveto{\pgfqpoint{-0.017319in}{-0.029821in}}{\pgfqpoint{-0.008840in}{-0.033333in}}{\pgfqpoint{0.000000in}{-0.033333in}}%
\pgfpathlineto{\pgfqpoint{0.000000in}{-0.033333in}}%
\pgfpathclose%
\pgfusepath{stroke,fill}%
}%
\begin{pgfscope}%
\pgfsys@transformshift{2.849491in}{2.575849in}%
\pgfsys@useobject{currentmarker}{}%
\end{pgfscope}%
\end{pgfscope}%
\begin{pgfscope}%
\pgfpathrectangle{\pgfqpoint{0.765000in}{0.660000in}}{\pgfqpoint{4.620000in}{4.620000in}}%
\pgfusepath{clip}%
\pgfsetroundcap%
\pgfsetroundjoin%
\pgfsetlinewidth{1.204500pt}%
\definecolor{currentstroke}{rgb}{1.000000,0.576471,0.309804}%
\pgfsetstrokecolor{currentstroke}%
\pgfsetdash{}{0pt}%
\pgfpathmoveto{\pgfqpoint{3.228014in}{3.505116in}}%
\pgfusepath{stroke}%
\end{pgfscope}%
\begin{pgfscope}%
\pgfpathrectangle{\pgfqpoint{0.765000in}{0.660000in}}{\pgfqpoint{4.620000in}{4.620000in}}%
\pgfusepath{clip}%
\pgfsetbuttcap%
\pgfsetroundjoin%
\definecolor{currentfill}{rgb}{1.000000,0.576471,0.309804}%
\pgfsetfillcolor{currentfill}%
\pgfsetlinewidth{1.003750pt}%
\definecolor{currentstroke}{rgb}{1.000000,0.576471,0.309804}%
\pgfsetstrokecolor{currentstroke}%
\pgfsetdash{}{0pt}%
\pgfsys@defobject{currentmarker}{\pgfqpoint{-0.033333in}{-0.033333in}}{\pgfqpoint{0.033333in}{0.033333in}}{%
\pgfpathmoveto{\pgfqpoint{0.000000in}{-0.033333in}}%
\pgfpathcurveto{\pgfqpoint{0.008840in}{-0.033333in}}{\pgfqpoint{0.017319in}{-0.029821in}}{\pgfqpoint{0.023570in}{-0.023570in}}%
\pgfpathcurveto{\pgfqpoint{0.029821in}{-0.017319in}}{\pgfqpoint{0.033333in}{-0.008840in}}{\pgfqpoint{0.033333in}{0.000000in}}%
\pgfpathcurveto{\pgfqpoint{0.033333in}{0.008840in}}{\pgfqpoint{0.029821in}{0.017319in}}{\pgfqpoint{0.023570in}{0.023570in}}%
\pgfpathcurveto{\pgfqpoint{0.017319in}{0.029821in}}{\pgfqpoint{0.008840in}{0.033333in}}{\pgfqpoint{0.000000in}{0.033333in}}%
\pgfpathcurveto{\pgfqpoint{-0.008840in}{0.033333in}}{\pgfqpoint{-0.017319in}{0.029821in}}{\pgfqpoint{-0.023570in}{0.023570in}}%
\pgfpathcurveto{\pgfqpoint{-0.029821in}{0.017319in}}{\pgfqpoint{-0.033333in}{0.008840in}}{\pgfqpoint{-0.033333in}{0.000000in}}%
\pgfpathcurveto{\pgfqpoint{-0.033333in}{-0.008840in}}{\pgfqpoint{-0.029821in}{-0.017319in}}{\pgfqpoint{-0.023570in}{-0.023570in}}%
\pgfpathcurveto{\pgfqpoint{-0.017319in}{-0.029821in}}{\pgfqpoint{-0.008840in}{-0.033333in}}{\pgfqpoint{0.000000in}{-0.033333in}}%
\pgfpathlineto{\pgfqpoint{0.000000in}{-0.033333in}}%
\pgfpathclose%
\pgfusepath{stroke,fill}%
}%
\begin{pgfscope}%
\pgfsys@transformshift{3.228014in}{3.505116in}%
\pgfsys@useobject{currentmarker}{}%
\end{pgfscope}%
\end{pgfscope}%
\begin{pgfscope}%
\pgfpathrectangle{\pgfqpoint{0.765000in}{0.660000in}}{\pgfqpoint{4.620000in}{4.620000in}}%
\pgfusepath{clip}%
\pgfsetroundcap%
\pgfsetroundjoin%
\pgfsetlinewidth{1.204500pt}%
\definecolor{currentstroke}{rgb}{1.000000,0.576471,0.309804}%
\pgfsetstrokecolor{currentstroke}%
\pgfsetdash{}{0pt}%
\pgfpathmoveto{\pgfqpoint{2.618942in}{2.638430in}}%
\pgfusepath{stroke}%
\end{pgfscope}%
\begin{pgfscope}%
\pgfpathrectangle{\pgfqpoint{0.765000in}{0.660000in}}{\pgfqpoint{4.620000in}{4.620000in}}%
\pgfusepath{clip}%
\pgfsetbuttcap%
\pgfsetroundjoin%
\definecolor{currentfill}{rgb}{1.000000,0.576471,0.309804}%
\pgfsetfillcolor{currentfill}%
\pgfsetlinewidth{1.003750pt}%
\definecolor{currentstroke}{rgb}{1.000000,0.576471,0.309804}%
\pgfsetstrokecolor{currentstroke}%
\pgfsetdash{}{0pt}%
\pgfsys@defobject{currentmarker}{\pgfqpoint{-0.033333in}{-0.033333in}}{\pgfqpoint{0.033333in}{0.033333in}}{%
\pgfpathmoveto{\pgfqpoint{0.000000in}{-0.033333in}}%
\pgfpathcurveto{\pgfqpoint{0.008840in}{-0.033333in}}{\pgfqpoint{0.017319in}{-0.029821in}}{\pgfqpoint{0.023570in}{-0.023570in}}%
\pgfpathcurveto{\pgfqpoint{0.029821in}{-0.017319in}}{\pgfqpoint{0.033333in}{-0.008840in}}{\pgfqpoint{0.033333in}{0.000000in}}%
\pgfpathcurveto{\pgfqpoint{0.033333in}{0.008840in}}{\pgfqpoint{0.029821in}{0.017319in}}{\pgfqpoint{0.023570in}{0.023570in}}%
\pgfpathcurveto{\pgfqpoint{0.017319in}{0.029821in}}{\pgfqpoint{0.008840in}{0.033333in}}{\pgfqpoint{0.000000in}{0.033333in}}%
\pgfpathcurveto{\pgfqpoint{-0.008840in}{0.033333in}}{\pgfqpoint{-0.017319in}{0.029821in}}{\pgfqpoint{-0.023570in}{0.023570in}}%
\pgfpathcurveto{\pgfqpoint{-0.029821in}{0.017319in}}{\pgfqpoint{-0.033333in}{0.008840in}}{\pgfqpoint{-0.033333in}{0.000000in}}%
\pgfpathcurveto{\pgfqpoint{-0.033333in}{-0.008840in}}{\pgfqpoint{-0.029821in}{-0.017319in}}{\pgfqpoint{-0.023570in}{-0.023570in}}%
\pgfpathcurveto{\pgfqpoint{-0.017319in}{-0.029821in}}{\pgfqpoint{-0.008840in}{-0.033333in}}{\pgfqpoint{0.000000in}{-0.033333in}}%
\pgfpathlineto{\pgfqpoint{0.000000in}{-0.033333in}}%
\pgfpathclose%
\pgfusepath{stroke,fill}%
}%
\begin{pgfscope}%
\pgfsys@transformshift{2.618942in}{2.638430in}%
\pgfsys@useobject{currentmarker}{}%
\end{pgfscope}%
\end{pgfscope}%
\begin{pgfscope}%
\pgfpathrectangle{\pgfqpoint{0.765000in}{0.660000in}}{\pgfqpoint{4.620000in}{4.620000in}}%
\pgfusepath{clip}%
\pgfsetroundcap%
\pgfsetroundjoin%
\pgfsetlinewidth{1.204500pt}%
\definecolor{currentstroke}{rgb}{1.000000,0.576471,0.309804}%
\pgfsetstrokecolor{currentstroke}%
\pgfsetdash{}{0pt}%
\pgfpathmoveto{\pgfqpoint{2.369208in}{3.565296in}}%
\pgfusepath{stroke}%
\end{pgfscope}%
\begin{pgfscope}%
\pgfpathrectangle{\pgfqpoint{0.765000in}{0.660000in}}{\pgfqpoint{4.620000in}{4.620000in}}%
\pgfusepath{clip}%
\pgfsetbuttcap%
\pgfsetroundjoin%
\definecolor{currentfill}{rgb}{1.000000,0.576471,0.309804}%
\pgfsetfillcolor{currentfill}%
\pgfsetlinewidth{1.003750pt}%
\definecolor{currentstroke}{rgb}{1.000000,0.576471,0.309804}%
\pgfsetstrokecolor{currentstroke}%
\pgfsetdash{}{0pt}%
\pgfsys@defobject{currentmarker}{\pgfqpoint{-0.033333in}{-0.033333in}}{\pgfqpoint{0.033333in}{0.033333in}}{%
\pgfpathmoveto{\pgfqpoint{0.000000in}{-0.033333in}}%
\pgfpathcurveto{\pgfqpoint{0.008840in}{-0.033333in}}{\pgfqpoint{0.017319in}{-0.029821in}}{\pgfqpoint{0.023570in}{-0.023570in}}%
\pgfpathcurveto{\pgfqpoint{0.029821in}{-0.017319in}}{\pgfqpoint{0.033333in}{-0.008840in}}{\pgfqpoint{0.033333in}{0.000000in}}%
\pgfpathcurveto{\pgfqpoint{0.033333in}{0.008840in}}{\pgfqpoint{0.029821in}{0.017319in}}{\pgfqpoint{0.023570in}{0.023570in}}%
\pgfpathcurveto{\pgfqpoint{0.017319in}{0.029821in}}{\pgfqpoint{0.008840in}{0.033333in}}{\pgfqpoint{0.000000in}{0.033333in}}%
\pgfpathcurveto{\pgfqpoint{-0.008840in}{0.033333in}}{\pgfqpoint{-0.017319in}{0.029821in}}{\pgfqpoint{-0.023570in}{0.023570in}}%
\pgfpathcurveto{\pgfqpoint{-0.029821in}{0.017319in}}{\pgfqpoint{-0.033333in}{0.008840in}}{\pgfqpoint{-0.033333in}{0.000000in}}%
\pgfpathcurveto{\pgfqpoint{-0.033333in}{-0.008840in}}{\pgfqpoint{-0.029821in}{-0.017319in}}{\pgfqpoint{-0.023570in}{-0.023570in}}%
\pgfpathcurveto{\pgfqpoint{-0.017319in}{-0.029821in}}{\pgfqpoint{-0.008840in}{-0.033333in}}{\pgfqpoint{0.000000in}{-0.033333in}}%
\pgfpathlineto{\pgfqpoint{0.000000in}{-0.033333in}}%
\pgfpathclose%
\pgfusepath{stroke,fill}%
}%
\begin{pgfscope}%
\pgfsys@transformshift{2.369208in}{3.565296in}%
\pgfsys@useobject{currentmarker}{}%
\end{pgfscope}%
\end{pgfscope}%
\begin{pgfscope}%
\pgfpathrectangle{\pgfqpoint{0.765000in}{0.660000in}}{\pgfqpoint{4.620000in}{4.620000in}}%
\pgfusepath{clip}%
\pgfsetroundcap%
\pgfsetroundjoin%
\pgfsetlinewidth{1.204500pt}%
\definecolor{currentstroke}{rgb}{1.000000,0.576471,0.309804}%
\pgfsetstrokecolor{currentstroke}%
\pgfsetdash{}{0pt}%
\pgfpathmoveto{\pgfqpoint{1.654746in}{2.274517in}}%
\pgfusepath{stroke}%
\end{pgfscope}%
\begin{pgfscope}%
\pgfpathrectangle{\pgfqpoint{0.765000in}{0.660000in}}{\pgfqpoint{4.620000in}{4.620000in}}%
\pgfusepath{clip}%
\pgfsetbuttcap%
\pgfsetroundjoin%
\definecolor{currentfill}{rgb}{1.000000,0.576471,0.309804}%
\pgfsetfillcolor{currentfill}%
\pgfsetlinewidth{1.003750pt}%
\definecolor{currentstroke}{rgb}{1.000000,0.576471,0.309804}%
\pgfsetstrokecolor{currentstroke}%
\pgfsetdash{}{0pt}%
\pgfsys@defobject{currentmarker}{\pgfqpoint{-0.033333in}{-0.033333in}}{\pgfqpoint{0.033333in}{0.033333in}}{%
\pgfpathmoveto{\pgfqpoint{0.000000in}{-0.033333in}}%
\pgfpathcurveto{\pgfqpoint{0.008840in}{-0.033333in}}{\pgfqpoint{0.017319in}{-0.029821in}}{\pgfqpoint{0.023570in}{-0.023570in}}%
\pgfpathcurveto{\pgfqpoint{0.029821in}{-0.017319in}}{\pgfqpoint{0.033333in}{-0.008840in}}{\pgfqpoint{0.033333in}{0.000000in}}%
\pgfpathcurveto{\pgfqpoint{0.033333in}{0.008840in}}{\pgfqpoint{0.029821in}{0.017319in}}{\pgfqpoint{0.023570in}{0.023570in}}%
\pgfpathcurveto{\pgfqpoint{0.017319in}{0.029821in}}{\pgfqpoint{0.008840in}{0.033333in}}{\pgfqpoint{0.000000in}{0.033333in}}%
\pgfpathcurveto{\pgfqpoint{-0.008840in}{0.033333in}}{\pgfqpoint{-0.017319in}{0.029821in}}{\pgfqpoint{-0.023570in}{0.023570in}}%
\pgfpathcurveto{\pgfqpoint{-0.029821in}{0.017319in}}{\pgfqpoint{-0.033333in}{0.008840in}}{\pgfqpoint{-0.033333in}{0.000000in}}%
\pgfpathcurveto{\pgfqpoint{-0.033333in}{-0.008840in}}{\pgfqpoint{-0.029821in}{-0.017319in}}{\pgfqpoint{-0.023570in}{-0.023570in}}%
\pgfpathcurveto{\pgfqpoint{-0.017319in}{-0.029821in}}{\pgfqpoint{-0.008840in}{-0.033333in}}{\pgfqpoint{0.000000in}{-0.033333in}}%
\pgfpathlineto{\pgfqpoint{0.000000in}{-0.033333in}}%
\pgfpathclose%
\pgfusepath{stroke,fill}%
}%
\begin{pgfscope}%
\pgfsys@transformshift{1.654746in}{2.274517in}%
\pgfsys@useobject{currentmarker}{}%
\end{pgfscope}%
\end{pgfscope}%
\begin{pgfscope}%
\pgfpathrectangle{\pgfqpoint{0.765000in}{0.660000in}}{\pgfqpoint{4.620000in}{4.620000in}}%
\pgfusepath{clip}%
\pgfsetroundcap%
\pgfsetroundjoin%
\pgfsetlinewidth{1.204500pt}%
\definecolor{currentstroke}{rgb}{1.000000,0.576471,0.309804}%
\pgfsetstrokecolor{currentstroke}%
\pgfsetdash{}{0pt}%
\pgfpathmoveto{\pgfqpoint{4.206167in}{2.348598in}}%
\pgfusepath{stroke}%
\end{pgfscope}%
\begin{pgfscope}%
\pgfpathrectangle{\pgfqpoint{0.765000in}{0.660000in}}{\pgfqpoint{4.620000in}{4.620000in}}%
\pgfusepath{clip}%
\pgfsetbuttcap%
\pgfsetroundjoin%
\definecolor{currentfill}{rgb}{1.000000,0.576471,0.309804}%
\pgfsetfillcolor{currentfill}%
\pgfsetlinewidth{1.003750pt}%
\definecolor{currentstroke}{rgb}{1.000000,0.576471,0.309804}%
\pgfsetstrokecolor{currentstroke}%
\pgfsetdash{}{0pt}%
\pgfsys@defobject{currentmarker}{\pgfqpoint{-0.033333in}{-0.033333in}}{\pgfqpoint{0.033333in}{0.033333in}}{%
\pgfpathmoveto{\pgfqpoint{0.000000in}{-0.033333in}}%
\pgfpathcurveto{\pgfqpoint{0.008840in}{-0.033333in}}{\pgfqpoint{0.017319in}{-0.029821in}}{\pgfqpoint{0.023570in}{-0.023570in}}%
\pgfpathcurveto{\pgfqpoint{0.029821in}{-0.017319in}}{\pgfqpoint{0.033333in}{-0.008840in}}{\pgfqpoint{0.033333in}{0.000000in}}%
\pgfpathcurveto{\pgfqpoint{0.033333in}{0.008840in}}{\pgfqpoint{0.029821in}{0.017319in}}{\pgfqpoint{0.023570in}{0.023570in}}%
\pgfpathcurveto{\pgfqpoint{0.017319in}{0.029821in}}{\pgfqpoint{0.008840in}{0.033333in}}{\pgfqpoint{0.000000in}{0.033333in}}%
\pgfpathcurveto{\pgfqpoint{-0.008840in}{0.033333in}}{\pgfqpoint{-0.017319in}{0.029821in}}{\pgfqpoint{-0.023570in}{0.023570in}}%
\pgfpathcurveto{\pgfqpoint{-0.029821in}{0.017319in}}{\pgfqpoint{-0.033333in}{0.008840in}}{\pgfqpoint{-0.033333in}{0.000000in}}%
\pgfpathcurveto{\pgfqpoint{-0.033333in}{-0.008840in}}{\pgfqpoint{-0.029821in}{-0.017319in}}{\pgfqpoint{-0.023570in}{-0.023570in}}%
\pgfpathcurveto{\pgfqpoint{-0.017319in}{-0.029821in}}{\pgfqpoint{-0.008840in}{-0.033333in}}{\pgfqpoint{0.000000in}{-0.033333in}}%
\pgfpathlineto{\pgfqpoint{0.000000in}{-0.033333in}}%
\pgfpathclose%
\pgfusepath{stroke,fill}%
}%
\begin{pgfscope}%
\pgfsys@transformshift{4.206167in}{2.348598in}%
\pgfsys@useobject{currentmarker}{}%
\end{pgfscope}%
\end{pgfscope}%
\begin{pgfscope}%
\pgfpathrectangle{\pgfqpoint{0.765000in}{0.660000in}}{\pgfqpoint{4.620000in}{4.620000in}}%
\pgfusepath{clip}%
\pgfsetroundcap%
\pgfsetroundjoin%
\pgfsetlinewidth{1.204500pt}%
\definecolor{currentstroke}{rgb}{1.000000,0.576471,0.309804}%
\pgfsetstrokecolor{currentstroke}%
\pgfsetdash{}{0pt}%
\pgfpathmoveto{\pgfqpoint{2.684867in}{3.374287in}}%
\pgfusepath{stroke}%
\end{pgfscope}%
\begin{pgfscope}%
\pgfpathrectangle{\pgfqpoint{0.765000in}{0.660000in}}{\pgfqpoint{4.620000in}{4.620000in}}%
\pgfusepath{clip}%
\pgfsetbuttcap%
\pgfsetroundjoin%
\definecolor{currentfill}{rgb}{1.000000,0.576471,0.309804}%
\pgfsetfillcolor{currentfill}%
\pgfsetlinewidth{1.003750pt}%
\definecolor{currentstroke}{rgb}{1.000000,0.576471,0.309804}%
\pgfsetstrokecolor{currentstroke}%
\pgfsetdash{}{0pt}%
\pgfsys@defobject{currentmarker}{\pgfqpoint{-0.033333in}{-0.033333in}}{\pgfqpoint{0.033333in}{0.033333in}}{%
\pgfpathmoveto{\pgfqpoint{0.000000in}{-0.033333in}}%
\pgfpathcurveto{\pgfqpoint{0.008840in}{-0.033333in}}{\pgfqpoint{0.017319in}{-0.029821in}}{\pgfqpoint{0.023570in}{-0.023570in}}%
\pgfpathcurveto{\pgfqpoint{0.029821in}{-0.017319in}}{\pgfqpoint{0.033333in}{-0.008840in}}{\pgfqpoint{0.033333in}{0.000000in}}%
\pgfpathcurveto{\pgfqpoint{0.033333in}{0.008840in}}{\pgfqpoint{0.029821in}{0.017319in}}{\pgfqpoint{0.023570in}{0.023570in}}%
\pgfpathcurveto{\pgfqpoint{0.017319in}{0.029821in}}{\pgfqpoint{0.008840in}{0.033333in}}{\pgfqpoint{0.000000in}{0.033333in}}%
\pgfpathcurveto{\pgfqpoint{-0.008840in}{0.033333in}}{\pgfqpoint{-0.017319in}{0.029821in}}{\pgfqpoint{-0.023570in}{0.023570in}}%
\pgfpathcurveto{\pgfqpoint{-0.029821in}{0.017319in}}{\pgfqpoint{-0.033333in}{0.008840in}}{\pgfqpoint{-0.033333in}{0.000000in}}%
\pgfpathcurveto{\pgfqpoint{-0.033333in}{-0.008840in}}{\pgfqpoint{-0.029821in}{-0.017319in}}{\pgfqpoint{-0.023570in}{-0.023570in}}%
\pgfpathcurveto{\pgfqpoint{-0.017319in}{-0.029821in}}{\pgfqpoint{-0.008840in}{-0.033333in}}{\pgfqpoint{0.000000in}{-0.033333in}}%
\pgfpathlineto{\pgfqpoint{0.000000in}{-0.033333in}}%
\pgfpathclose%
\pgfusepath{stroke,fill}%
}%
\begin{pgfscope}%
\pgfsys@transformshift{2.684867in}{3.374287in}%
\pgfsys@useobject{currentmarker}{}%
\end{pgfscope}%
\end{pgfscope}%
\begin{pgfscope}%
\pgfpathrectangle{\pgfqpoint{0.765000in}{0.660000in}}{\pgfqpoint{4.620000in}{4.620000in}}%
\pgfusepath{clip}%
\pgfsetroundcap%
\pgfsetroundjoin%
\pgfsetlinewidth{1.204500pt}%
\definecolor{currentstroke}{rgb}{1.000000,0.576471,0.309804}%
\pgfsetstrokecolor{currentstroke}%
\pgfsetdash{}{0pt}%
\pgfpathmoveto{\pgfqpoint{2.369009in}{2.619474in}}%
\pgfusepath{stroke}%
\end{pgfscope}%
\begin{pgfscope}%
\pgfpathrectangle{\pgfqpoint{0.765000in}{0.660000in}}{\pgfqpoint{4.620000in}{4.620000in}}%
\pgfusepath{clip}%
\pgfsetbuttcap%
\pgfsetroundjoin%
\definecolor{currentfill}{rgb}{1.000000,0.576471,0.309804}%
\pgfsetfillcolor{currentfill}%
\pgfsetlinewidth{1.003750pt}%
\definecolor{currentstroke}{rgb}{1.000000,0.576471,0.309804}%
\pgfsetstrokecolor{currentstroke}%
\pgfsetdash{}{0pt}%
\pgfsys@defobject{currentmarker}{\pgfqpoint{-0.033333in}{-0.033333in}}{\pgfqpoint{0.033333in}{0.033333in}}{%
\pgfpathmoveto{\pgfqpoint{0.000000in}{-0.033333in}}%
\pgfpathcurveto{\pgfqpoint{0.008840in}{-0.033333in}}{\pgfqpoint{0.017319in}{-0.029821in}}{\pgfqpoint{0.023570in}{-0.023570in}}%
\pgfpathcurveto{\pgfqpoint{0.029821in}{-0.017319in}}{\pgfqpoint{0.033333in}{-0.008840in}}{\pgfqpoint{0.033333in}{0.000000in}}%
\pgfpathcurveto{\pgfqpoint{0.033333in}{0.008840in}}{\pgfqpoint{0.029821in}{0.017319in}}{\pgfqpoint{0.023570in}{0.023570in}}%
\pgfpathcurveto{\pgfqpoint{0.017319in}{0.029821in}}{\pgfqpoint{0.008840in}{0.033333in}}{\pgfqpoint{0.000000in}{0.033333in}}%
\pgfpathcurveto{\pgfqpoint{-0.008840in}{0.033333in}}{\pgfqpoint{-0.017319in}{0.029821in}}{\pgfqpoint{-0.023570in}{0.023570in}}%
\pgfpathcurveto{\pgfqpoint{-0.029821in}{0.017319in}}{\pgfqpoint{-0.033333in}{0.008840in}}{\pgfqpoint{-0.033333in}{0.000000in}}%
\pgfpathcurveto{\pgfqpoint{-0.033333in}{-0.008840in}}{\pgfqpoint{-0.029821in}{-0.017319in}}{\pgfqpoint{-0.023570in}{-0.023570in}}%
\pgfpathcurveto{\pgfqpoint{-0.017319in}{-0.029821in}}{\pgfqpoint{-0.008840in}{-0.033333in}}{\pgfqpoint{0.000000in}{-0.033333in}}%
\pgfpathlineto{\pgfqpoint{0.000000in}{-0.033333in}}%
\pgfpathclose%
\pgfusepath{stroke,fill}%
}%
\begin{pgfscope}%
\pgfsys@transformshift{2.369009in}{2.619474in}%
\pgfsys@useobject{currentmarker}{}%
\end{pgfscope}%
\end{pgfscope}%
\begin{pgfscope}%
\pgfpathrectangle{\pgfqpoint{0.765000in}{0.660000in}}{\pgfqpoint{4.620000in}{4.620000in}}%
\pgfusepath{clip}%
\pgfsetroundcap%
\pgfsetroundjoin%
\pgfsetlinewidth{1.204500pt}%
\definecolor{currentstroke}{rgb}{1.000000,0.576471,0.309804}%
\pgfsetstrokecolor{currentstroke}%
\pgfsetdash{}{0pt}%
\pgfpathmoveto{\pgfqpoint{3.355887in}{2.348246in}}%
\pgfusepath{stroke}%
\end{pgfscope}%
\begin{pgfscope}%
\pgfpathrectangle{\pgfqpoint{0.765000in}{0.660000in}}{\pgfqpoint{4.620000in}{4.620000in}}%
\pgfusepath{clip}%
\pgfsetbuttcap%
\pgfsetroundjoin%
\definecolor{currentfill}{rgb}{1.000000,0.576471,0.309804}%
\pgfsetfillcolor{currentfill}%
\pgfsetlinewidth{1.003750pt}%
\definecolor{currentstroke}{rgb}{1.000000,0.576471,0.309804}%
\pgfsetstrokecolor{currentstroke}%
\pgfsetdash{}{0pt}%
\pgfsys@defobject{currentmarker}{\pgfqpoint{-0.033333in}{-0.033333in}}{\pgfqpoint{0.033333in}{0.033333in}}{%
\pgfpathmoveto{\pgfqpoint{0.000000in}{-0.033333in}}%
\pgfpathcurveto{\pgfqpoint{0.008840in}{-0.033333in}}{\pgfqpoint{0.017319in}{-0.029821in}}{\pgfqpoint{0.023570in}{-0.023570in}}%
\pgfpathcurveto{\pgfqpoint{0.029821in}{-0.017319in}}{\pgfqpoint{0.033333in}{-0.008840in}}{\pgfqpoint{0.033333in}{0.000000in}}%
\pgfpathcurveto{\pgfqpoint{0.033333in}{0.008840in}}{\pgfqpoint{0.029821in}{0.017319in}}{\pgfqpoint{0.023570in}{0.023570in}}%
\pgfpathcurveto{\pgfqpoint{0.017319in}{0.029821in}}{\pgfqpoint{0.008840in}{0.033333in}}{\pgfqpoint{0.000000in}{0.033333in}}%
\pgfpathcurveto{\pgfqpoint{-0.008840in}{0.033333in}}{\pgfqpoint{-0.017319in}{0.029821in}}{\pgfqpoint{-0.023570in}{0.023570in}}%
\pgfpathcurveto{\pgfqpoint{-0.029821in}{0.017319in}}{\pgfqpoint{-0.033333in}{0.008840in}}{\pgfqpoint{-0.033333in}{0.000000in}}%
\pgfpathcurveto{\pgfqpoint{-0.033333in}{-0.008840in}}{\pgfqpoint{-0.029821in}{-0.017319in}}{\pgfqpoint{-0.023570in}{-0.023570in}}%
\pgfpathcurveto{\pgfqpoint{-0.017319in}{-0.029821in}}{\pgfqpoint{-0.008840in}{-0.033333in}}{\pgfqpoint{0.000000in}{-0.033333in}}%
\pgfpathlineto{\pgfqpoint{0.000000in}{-0.033333in}}%
\pgfpathclose%
\pgfusepath{stroke,fill}%
}%
\begin{pgfscope}%
\pgfsys@transformshift{3.355887in}{2.348246in}%
\pgfsys@useobject{currentmarker}{}%
\end{pgfscope}%
\end{pgfscope}%
\begin{pgfscope}%
\pgfpathrectangle{\pgfqpoint{0.765000in}{0.660000in}}{\pgfqpoint{4.620000in}{4.620000in}}%
\pgfusepath{clip}%
\pgfsetroundcap%
\pgfsetroundjoin%
\pgfsetlinewidth{1.204500pt}%
\definecolor{currentstroke}{rgb}{1.000000,0.576471,0.309804}%
\pgfsetstrokecolor{currentstroke}%
\pgfsetdash{}{0pt}%
\pgfpathmoveto{\pgfqpoint{4.051159in}{2.799972in}}%
\pgfusepath{stroke}%
\end{pgfscope}%
\begin{pgfscope}%
\pgfpathrectangle{\pgfqpoint{0.765000in}{0.660000in}}{\pgfqpoint{4.620000in}{4.620000in}}%
\pgfusepath{clip}%
\pgfsetbuttcap%
\pgfsetroundjoin%
\definecolor{currentfill}{rgb}{1.000000,0.576471,0.309804}%
\pgfsetfillcolor{currentfill}%
\pgfsetlinewidth{1.003750pt}%
\definecolor{currentstroke}{rgb}{1.000000,0.576471,0.309804}%
\pgfsetstrokecolor{currentstroke}%
\pgfsetdash{}{0pt}%
\pgfsys@defobject{currentmarker}{\pgfqpoint{-0.033333in}{-0.033333in}}{\pgfqpoint{0.033333in}{0.033333in}}{%
\pgfpathmoveto{\pgfqpoint{0.000000in}{-0.033333in}}%
\pgfpathcurveto{\pgfqpoint{0.008840in}{-0.033333in}}{\pgfqpoint{0.017319in}{-0.029821in}}{\pgfqpoint{0.023570in}{-0.023570in}}%
\pgfpathcurveto{\pgfqpoint{0.029821in}{-0.017319in}}{\pgfqpoint{0.033333in}{-0.008840in}}{\pgfqpoint{0.033333in}{0.000000in}}%
\pgfpathcurveto{\pgfqpoint{0.033333in}{0.008840in}}{\pgfqpoint{0.029821in}{0.017319in}}{\pgfqpoint{0.023570in}{0.023570in}}%
\pgfpathcurveto{\pgfqpoint{0.017319in}{0.029821in}}{\pgfqpoint{0.008840in}{0.033333in}}{\pgfqpoint{0.000000in}{0.033333in}}%
\pgfpathcurveto{\pgfqpoint{-0.008840in}{0.033333in}}{\pgfqpoint{-0.017319in}{0.029821in}}{\pgfqpoint{-0.023570in}{0.023570in}}%
\pgfpathcurveto{\pgfqpoint{-0.029821in}{0.017319in}}{\pgfqpoint{-0.033333in}{0.008840in}}{\pgfqpoint{-0.033333in}{0.000000in}}%
\pgfpathcurveto{\pgfqpoint{-0.033333in}{-0.008840in}}{\pgfqpoint{-0.029821in}{-0.017319in}}{\pgfqpoint{-0.023570in}{-0.023570in}}%
\pgfpathcurveto{\pgfqpoint{-0.017319in}{-0.029821in}}{\pgfqpoint{-0.008840in}{-0.033333in}}{\pgfqpoint{0.000000in}{-0.033333in}}%
\pgfpathlineto{\pgfqpoint{0.000000in}{-0.033333in}}%
\pgfpathclose%
\pgfusepath{stroke,fill}%
}%
\begin{pgfscope}%
\pgfsys@transformshift{4.051159in}{2.799972in}%
\pgfsys@useobject{currentmarker}{}%
\end{pgfscope}%
\end{pgfscope}%
\begin{pgfscope}%
\pgfpathrectangle{\pgfqpoint{0.765000in}{0.660000in}}{\pgfqpoint{4.620000in}{4.620000in}}%
\pgfusepath{clip}%
\pgfsetroundcap%
\pgfsetroundjoin%
\pgfsetlinewidth{1.204500pt}%
\definecolor{currentstroke}{rgb}{1.000000,0.576471,0.309804}%
\pgfsetstrokecolor{currentstroke}%
\pgfsetdash{}{0pt}%
\pgfpathmoveto{\pgfqpoint{2.741073in}{1.922657in}}%
\pgfusepath{stroke}%
\end{pgfscope}%
\begin{pgfscope}%
\pgfpathrectangle{\pgfqpoint{0.765000in}{0.660000in}}{\pgfqpoint{4.620000in}{4.620000in}}%
\pgfusepath{clip}%
\pgfsetbuttcap%
\pgfsetroundjoin%
\definecolor{currentfill}{rgb}{1.000000,0.576471,0.309804}%
\pgfsetfillcolor{currentfill}%
\pgfsetlinewidth{1.003750pt}%
\definecolor{currentstroke}{rgb}{1.000000,0.576471,0.309804}%
\pgfsetstrokecolor{currentstroke}%
\pgfsetdash{}{0pt}%
\pgfsys@defobject{currentmarker}{\pgfqpoint{-0.033333in}{-0.033333in}}{\pgfqpoint{0.033333in}{0.033333in}}{%
\pgfpathmoveto{\pgfqpoint{0.000000in}{-0.033333in}}%
\pgfpathcurveto{\pgfqpoint{0.008840in}{-0.033333in}}{\pgfqpoint{0.017319in}{-0.029821in}}{\pgfqpoint{0.023570in}{-0.023570in}}%
\pgfpathcurveto{\pgfqpoint{0.029821in}{-0.017319in}}{\pgfqpoint{0.033333in}{-0.008840in}}{\pgfqpoint{0.033333in}{0.000000in}}%
\pgfpathcurveto{\pgfqpoint{0.033333in}{0.008840in}}{\pgfqpoint{0.029821in}{0.017319in}}{\pgfqpoint{0.023570in}{0.023570in}}%
\pgfpathcurveto{\pgfqpoint{0.017319in}{0.029821in}}{\pgfqpoint{0.008840in}{0.033333in}}{\pgfqpoint{0.000000in}{0.033333in}}%
\pgfpathcurveto{\pgfqpoint{-0.008840in}{0.033333in}}{\pgfqpoint{-0.017319in}{0.029821in}}{\pgfqpoint{-0.023570in}{0.023570in}}%
\pgfpathcurveto{\pgfqpoint{-0.029821in}{0.017319in}}{\pgfqpoint{-0.033333in}{0.008840in}}{\pgfqpoint{-0.033333in}{0.000000in}}%
\pgfpathcurveto{\pgfqpoint{-0.033333in}{-0.008840in}}{\pgfqpoint{-0.029821in}{-0.017319in}}{\pgfqpoint{-0.023570in}{-0.023570in}}%
\pgfpathcurveto{\pgfqpoint{-0.017319in}{-0.029821in}}{\pgfqpoint{-0.008840in}{-0.033333in}}{\pgfqpoint{0.000000in}{-0.033333in}}%
\pgfpathlineto{\pgfqpoint{0.000000in}{-0.033333in}}%
\pgfpathclose%
\pgfusepath{stroke,fill}%
}%
\begin{pgfscope}%
\pgfsys@transformshift{2.741073in}{1.922657in}%
\pgfsys@useobject{currentmarker}{}%
\end{pgfscope}%
\end{pgfscope}%
\begin{pgfscope}%
\pgfpathrectangle{\pgfqpoint{0.765000in}{0.660000in}}{\pgfqpoint{4.620000in}{4.620000in}}%
\pgfusepath{clip}%
\pgfsetroundcap%
\pgfsetroundjoin%
\pgfsetlinewidth{1.204500pt}%
\definecolor{currentstroke}{rgb}{1.000000,0.576471,0.309804}%
\pgfsetstrokecolor{currentstroke}%
\pgfsetdash{}{0pt}%
\pgfpathmoveto{\pgfqpoint{3.025236in}{3.672451in}}%
\pgfusepath{stroke}%
\end{pgfscope}%
\begin{pgfscope}%
\pgfpathrectangle{\pgfqpoint{0.765000in}{0.660000in}}{\pgfqpoint{4.620000in}{4.620000in}}%
\pgfusepath{clip}%
\pgfsetbuttcap%
\pgfsetroundjoin%
\definecolor{currentfill}{rgb}{1.000000,0.576471,0.309804}%
\pgfsetfillcolor{currentfill}%
\pgfsetlinewidth{1.003750pt}%
\definecolor{currentstroke}{rgb}{1.000000,0.576471,0.309804}%
\pgfsetstrokecolor{currentstroke}%
\pgfsetdash{}{0pt}%
\pgfsys@defobject{currentmarker}{\pgfqpoint{-0.033333in}{-0.033333in}}{\pgfqpoint{0.033333in}{0.033333in}}{%
\pgfpathmoveto{\pgfqpoint{0.000000in}{-0.033333in}}%
\pgfpathcurveto{\pgfqpoint{0.008840in}{-0.033333in}}{\pgfqpoint{0.017319in}{-0.029821in}}{\pgfqpoint{0.023570in}{-0.023570in}}%
\pgfpathcurveto{\pgfqpoint{0.029821in}{-0.017319in}}{\pgfqpoint{0.033333in}{-0.008840in}}{\pgfqpoint{0.033333in}{0.000000in}}%
\pgfpathcurveto{\pgfqpoint{0.033333in}{0.008840in}}{\pgfqpoint{0.029821in}{0.017319in}}{\pgfqpoint{0.023570in}{0.023570in}}%
\pgfpathcurveto{\pgfqpoint{0.017319in}{0.029821in}}{\pgfqpoint{0.008840in}{0.033333in}}{\pgfqpoint{0.000000in}{0.033333in}}%
\pgfpathcurveto{\pgfqpoint{-0.008840in}{0.033333in}}{\pgfqpoint{-0.017319in}{0.029821in}}{\pgfqpoint{-0.023570in}{0.023570in}}%
\pgfpathcurveto{\pgfqpoint{-0.029821in}{0.017319in}}{\pgfqpoint{-0.033333in}{0.008840in}}{\pgfqpoint{-0.033333in}{0.000000in}}%
\pgfpathcurveto{\pgfqpoint{-0.033333in}{-0.008840in}}{\pgfqpoint{-0.029821in}{-0.017319in}}{\pgfqpoint{-0.023570in}{-0.023570in}}%
\pgfpathcurveto{\pgfqpoint{-0.017319in}{-0.029821in}}{\pgfqpoint{-0.008840in}{-0.033333in}}{\pgfqpoint{0.000000in}{-0.033333in}}%
\pgfpathlineto{\pgfqpoint{0.000000in}{-0.033333in}}%
\pgfpathclose%
\pgfusepath{stroke,fill}%
}%
\begin{pgfscope}%
\pgfsys@transformshift{3.025236in}{3.672451in}%
\pgfsys@useobject{currentmarker}{}%
\end{pgfscope}%
\end{pgfscope}%
\begin{pgfscope}%
\pgfpathrectangle{\pgfqpoint{0.765000in}{0.660000in}}{\pgfqpoint{4.620000in}{4.620000in}}%
\pgfusepath{clip}%
\pgfsetroundcap%
\pgfsetroundjoin%
\pgfsetlinewidth{1.204500pt}%
\definecolor{currentstroke}{rgb}{1.000000,0.576471,0.309804}%
\pgfsetstrokecolor{currentstroke}%
\pgfsetdash{}{0pt}%
\pgfpathmoveto{\pgfqpoint{3.543155in}{3.553066in}}%
\pgfusepath{stroke}%
\end{pgfscope}%
\begin{pgfscope}%
\pgfpathrectangle{\pgfqpoint{0.765000in}{0.660000in}}{\pgfqpoint{4.620000in}{4.620000in}}%
\pgfusepath{clip}%
\pgfsetbuttcap%
\pgfsetroundjoin%
\definecolor{currentfill}{rgb}{1.000000,0.576471,0.309804}%
\pgfsetfillcolor{currentfill}%
\pgfsetlinewidth{1.003750pt}%
\definecolor{currentstroke}{rgb}{1.000000,0.576471,0.309804}%
\pgfsetstrokecolor{currentstroke}%
\pgfsetdash{}{0pt}%
\pgfsys@defobject{currentmarker}{\pgfqpoint{-0.033333in}{-0.033333in}}{\pgfqpoint{0.033333in}{0.033333in}}{%
\pgfpathmoveto{\pgfqpoint{0.000000in}{-0.033333in}}%
\pgfpathcurveto{\pgfqpoint{0.008840in}{-0.033333in}}{\pgfqpoint{0.017319in}{-0.029821in}}{\pgfqpoint{0.023570in}{-0.023570in}}%
\pgfpathcurveto{\pgfqpoint{0.029821in}{-0.017319in}}{\pgfqpoint{0.033333in}{-0.008840in}}{\pgfqpoint{0.033333in}{0.000000in}}%
\pgfpathcurveto{\pgfqpoint{0.033333in}{0.008840in}}{\pgfqpoint{0.029821in}{0.017319in}}{\pgfqpoint{0.023570in}{0.023570in}}%
\pgfpathcurveto{\pgfqpoint{0.017319in}{0.029821in}}{\pgfqpoint{0.008840in}{0.033333in}}{\pgfqpoint{0.000000in}{0.033333in}}%
\pgfpathcurveto{\pgfqpoint{-0.008840in}{0.033333in}}{\pgfqpoint{-0.017319in}{0.029821in}}{\pgfqpoint{-0.023570in}{0.023570in}}%
\pgfpathcurveto{\pgfqpoint{-0.029821in}{0.017319in}}{\pgfqpoint{-0.033333in}{0.008840in}}{\pgfqpoint{-0.033333in}{0.000000in}}%
\pgfpathcurveto{\pgfqpoint{-0.033333in}{-0.008840in}}{\pgfqpoint{-0.029821in}{-0.017319in}}{\pgfqpoint{-0.023570in}{-0.023570in}}%
\pgfpathcurveto{\pgfqpoint{-0.017319in}{-0.029821in}}{\pgfqpoint{-0.008840in}{-0.033333in}}{\pgfqpoint{0.000000in}{-0.033333in}}%
\pgfpathlineto{\pgfqpoint{0.000000in}{-0.033333in}}%
\pgfpathclose%
\pgfusepath{stroke,fill}%
}%
\begin{pgfscope}%
\pgfsys@transformshift{3.543155in}{3.553066in}%
\pgfsys@useobject{currentmarker}{}%
\end{pgfscope}%
\end{pgfscope}%
\begin{pgfscope}%
\pgfpathrectangle{\pgfqpoint{0.765000in}{0.660000in}}{\pgfqpoint{4.620000in}{4.620000in}}%
\pgfusepath{clip}%
\pgfsetroundcap%
\pgfsetroundjoin%
\pgfsetlinewidth{1.204500pt}%
\definecolor{currentstroke}{rgb}{1.000000,0.576471,0.309804}%
\pgfsetstrokecolor{currentstroke}%
\pgfsetdash{}{0pt}%
\pgfpathmoveto{\pgfqpoint{3.827050in}{2.761841in}}%
\pgfusepath{stroke}%
\end{pgfscope}%
\begin{pgfscope}%
\pgfpathrectangle{\pgfqpoint{0.765000in}{0.660000in}}{\pgfqpoint{4.620000in}{4.620000in}}%
\pgfusepath{clip}%
\pgfsetbuttcap%
\pgfsetroundjoin%
\definecolor{currentfill}{rgb}{1.000000,0.576471,0.309804}%
\pgfsetfillcolor{currentfill}%
\pgfsetlinewidth{1.003750pt}%
\definecolor{currentstroke}{rgb}{1.000000,0.576471,0.309804}%
\pgfsetstrokecolor{currentstroke}%
\pgfsetdash{}{0pt}%
\pgfsys@defobject{currentmarker}{\pgfqpoint{-0.033333in}{-0.033333in}}{\pgfqpoint{0.033333in}{0.033333in}}{%
\pgfpathmoveto{\pgfqpoint{0.000000in}{-0.033333in}}%
\pgfpathcurveto{\pgfqpoint{0.008840in}{-0.033333in}}{\pgfqpoint{0.017319in}{-0.029821in}}{\pgfqpoint{0.023570in}{-0.023570in}}%
\pgfpathcurveto{\pgfqpoint{0.029821in}{-0.017319in}}{\pgfqpoint{0.033333in}{-0.008840in}}{\pgfqpoint{0.033333in}{0.000000in}}%
\pgfpathcurveto{\pgfqpoint{0.033333in}{0.008840in}}{\pgfqpoint{0.029821in}{0.017319in}}{\pgfqpoint{0.023570in}{0.023570in}}%
\pgfpathcurveto{\pgfqpoint{0.017319in}{0.029821in}}{\pgfqpoint{0.008840in}{0.033333in}}{\pgfqpoint{0.000000in}{0.033333in}}%
\pgfpathcurveto{\pgfqpoint{-0.008840in}{0.033333in}}{\pgfqpoint{-0.017319in}{0.029821in}}{\pgfqpoint{-0.023570in}{0.023570in}}%
\pgfpathcurveto{\pgfqpoint{-0.029821in}{0.017319in}}{\pgfqpoint{-0.033333in}{0.008840in}}{\pgfqpoint{-0.033333in}{0.000000in}}%
\pgfpathcurveto{\pgfqpoint{-0.033333in}{-0.008840in}}{\pgfqpoint{-0.029821in}{-0.017319in}}{\pgfqpoint{-0.023570in}{-0.023570in}}%
\pgfpathcurveto{\pgfqpoint{-0.017319in}{-0.029821in}}{\pgfqpoint{-0.008840in}{-0.033333in}}{\pgfqpoint{0.000000in}{-0.033333in}}%
\pgfpathlineto{\pgfqpoint{0.000000in}{-0.033333in}}%
\pgfpathclose%
\pgfusepath{stroke,fill}%
}%
\begin{pgfscope}%
\pgfsys@transformshift{3.827050in}{2.761841in}%
\pgfsys@useobject{currentmarker}{}%
\end{pgfscope}%
\end{pgfscope}%
\begin{pgfscope}%
\pgfpathrectangle{\pgfqpoint{0.765000in}{0.660000in}}{\pgfqpoint{4.620000in}{4.620000in}}%
\pgfusepath{clip}%
\pgfsetroundcap%
\pgfsetroundjoin%
\pgfsetlinewidth{1.204500pt}%
\definecolor{currentstroke}{rgb}{1.000000,0.576471,0.309804}%
\pgfsetstrokecolor{currentstroke}%
\pgfsetdash{}{0pt}%
\pgfpathmoveto{\pgfqpoint{3.718611in}{2.633067in}}%
\pgfusepath{stroke}%
\end{pgfscope}%
\begin{pgfscope}%
\pgfpathrectangle{\pgfqpoint{0.765000in}{0.660000in}}{\pgfqpoint{4.620000in}{4.620000in}}%
\pgfusepath{clip}%
\pgfsetbuttcap%
\pgfsetroundjoin%
\definecolor{currentfill}{rgb}{1.000000,0.576471,0.309804}%
\pgfsetfillcolor{currentfill}%
\pgfsetlinewidth{1.003750pt}%
\definecolor{currentstroke}{rgb}{1.000000,0.576471,0.309804}%
\pgfsetstrokecolor{currentstroke}%
\pgfsetdash{}{0pt}%
\pgfsys@defobject{currentmarker}{\pgfqpoint{-0.033333in}{-0.033333in}}{\pgfqpoint{0.033333in}{0.033333in}}{%
\pgfpathmoveto{\pgfqpoint{0.000000in}{-0.033333in}}%
\pgfpathcurveto{\pgfqpoint{0.008840in}{-0.033333in}}{\pgfqpoint{0.017319in}{-0.029821in}}{\pgfqpoint{0.023570in}{-0.023570in}}%
\pgfpathcurveto{\pgfqpoint{0.029821in}{-0.017319in}}{\pgfqpoint{0.033333in}{-0.008840in}}{\pgfqpoint{0.033333in}{0.000000in}}%
\pgfpathcurveto{\pgfqpoint{0.033333in}{0.008840in}}{\pgfqpoint{0.029821in}{0.017319in}}{\pgfqpoint{0.023570in}{0.023570in}}%
\pgfpathcurveto{\pgfqpoint{0.017319in}{0.029821in}}{\pgfqpoint{0.008840in}{0.033333in}}{\pgfqpoint{0.000000in}{0.033333in}}%
\pgfpathcurveto{\pgfqpoint{-0.008840in}{0.033333in}}{\pgfqpoint{-0.017319in}{0.029821in}}{\pgfqpoint{-0.023570in}{0.023570in}}%
\pgfpathcurveto{\pgfqpoint{-0.029821in}{0.017319in}}{\pgfqpoint{-0.033333in}{0.008840in}}{\pgfqpoint{-0.033333in}{0.000000in}}%
\pgfpathcurveto{\pgfqpoint{-0.033333in}{-0.008840in}}{\pgfqpoint{-0.029821in}{-0.017319in}}{\pgfqpoint{-0.023570in}{-0.023570in}}%
\pgfpathcurveto{\pgfqpoint{-0.017319in}{-0.029821in}}{\pgfqpoint{-0.008840in}{-0.033333in}}{\pgfqpoint{0.000000in}{-0.033333in}}%
\pgfpathlineto{\pgfqpoint{0.000000in}{-0.033333in}}%
\pgfpathclose%
\pgfusepath{stroke,fill}%
}%
\begin{pgfscope}%
\pgfsys@transformshift{3.718611in}{2.633067in}%
\pgfsys@useobject{currentmarker}{}%
\end{pgfscope}%
\end{pgfscope}%
\begin{pgfscope}%
\pgfpathrectangle{\pgfqpoint{0.765000in}{0.660000in}}{\pgfqpoint{4.620000in}{4.620000in}}%
\pgfusepath{clip}%
\pgfsetroundcap%
\pgfsetroundjoin%
\pgfsetlinewidth{1.204500pt}%
\definecolor{currentstroke}{rgb}{1.000000,0.576471,0.309804}%
\pgfsetstrokecolor{currentstroke}%
\pgfsetdash{}{0pt}%
\pgfpathmoveto{\pgfqpoint{2.727027in}{2.510444in}}%
\pgfusepath{stroke}%
\end{pgfscope}%
\begin{pgfscope}%
\pgfpathrectangle{\pgfqpoint{0.765000in}{0.660000in}}{\pgfqpoint{4.620000in}{4.620000in}}%
\pgfusepath{clip}%
\pgfsetbuttcap%
\pgfsetroundjoin%
\definecolor{currentfill}{rgb}{1.000000,0.576471,0.309804}%
\pgfsetfillcolor{currentfill}%
\pgfsetlinewidth{1.003750pt}%
\definecolor{currentstroke}{rgb}{1.000000,0.576471,0.309804}%
\pgfsetstrokecolor{currentstroke}%
\pgfsetdash{}{0pt}%
\pgfsys@defobject{currentmarker}{\pgfqpoint{-0.033333in}{-0.033333in}}{\pgfqpoint{0.033333in}{0.033333in}}{%
\pgfpathmoveto{\pgfqpoint{0.000000in}{-0.033333in}}%
\pgfpathcurveto{\pgfqpoint{0.008840in}{-0.033333in}}{\pgfqpoint{0.017319in}{-0.029821in}}{\pgfqpoint{0.023570in}{-0.023570in}}%
\pgfpathcurveto{\pgfqpoint{0.029821in}{-0.017319in}}{\pgfqpoint{0.033333in}{-0.008840in}}{\pgfqpoint{0.033333in}{0.000000in}}%
\pgfpathcurveto{\pgfqpoint{0.033333in}{0.008840in}}{\pgfqpoint{0.029821in}{0.017319in}}{\pgfqpoint{0.023570in}{0.023570in}}%
\pgfpathcurveto{\pgfqpoint{0.017319in}{0.029821in}}{\pgfqpoint{0.008840in}{0.033333in}}{\pgfqpoint{0.000000in}{0.033333in}}%
\pgfpathcurveto{\pgfqpoint{-0.008840in}{0.033333in}}{\pgfqpoint{-0.017319in}{0.029821in}}{\pgfqpoint{-0.023570in}{0.023570in}}%
\pgfpathcurveto{\pgfqpoint{-0.029821in}{0.017319in}}{\pgfqpoint{-0.033333in}{0.008840in}}{\pgfqpoint{-0.033333in}{0.000000in}}%
\pgfpathcurveto{\pgfqpoint{-0.033333in}{-0.008840in}}{\pgfqpoint{-0.029821in}{-0.017319in}}{\pgfqpoint{-0.023570in}{-0.023570in}}%
\pgfpathcurveto{\pgfqpoint{-0.017319in}{-0.029821in}}{\pgfqpoint{-0.008840in}{-0.033333in}}{\pgfqpoint{0.000000in}{-0.033333in}}%
\pgfpathlineto{\pgfqpoint{0.000000in}{-0.033333in}}%
\pgfpathclose%
\pgfusepath{stroke,fill}%
}%
\begin{pgfscope}%
\pgfsys@transformshift{2.727027in}{2.510444in}%
\pgfsys@useobject{currentmarker}{}%
\end{pgfscope}%
\end{pgfscope}%
\begin{pgfscope}%
\pgfpathrectangle{\pgfqpoint{0.765000in}{0.660000in}}{\pgfqpoint{4.620000in}{4.620000in}}%
\pgfusepath{clip}%
\pgfsetroundcap%
\pgfsetroundjoin%
\pgfsetlinewidth{1.204500pt}%
\definecolor{currentstroke}{rgb}{1.000000,0.576471,0.309804}%
\pgfsetstrokecolor{currentstroke}%
\pgfsetdash{}{0pt}%
\pgfpathmoveto{\pgfqpoint{3.788294in}{3.302185in}}%
\pgfusepath{stroke}%
\end{pgfscope}%
\begin{pgfscope}%
\pgfpathrectangle{\pgfqpoint{0.765000in}{0.660000in}}{\pgfqpoint{4.620000in}{4.620000in}}%
\pgfusepath{clip}%
\pgfsetbuttcap%
\pgfsetroundjoin%
\definecolor{currentfill}{rgb}{1.000000,0.576471,0.309804}%
\pgfsetfillcolor{currentfill}%
\pgfsetlinewidth{1.003750pt}%
\definecolor{currentstroke}{rgb}{1.000000,0.576471,0.309804}%
\pgfsetstrokecolor{currentstroke}%
\pgfsetdash{}{0pt}%
\pgfsys@defobject{currentmarker}{\pgfqpoint{-0.033333in}{-0.033333in}}{\pgfqpoint{0.033333in}{0.033333in}}{%
\pgfpathmoveto{\pgfqpoint{0.000000in}{-0.033333in}}%
\pgfpathcurveto{\pgfqpoint{0.008840in}{-0.033333in}}{\pgfqpoint{0.017319in}{-0.029821in}}{\pgfqpoint{0.023570in}{-0.023570in}}%
\pgfpathcurveto{\pgfqpoint{0.029821in}{-0.017319in}}{\pgfqpoint{0.033333in}{-0.008840in}}{\pgfqpoint{0.033333in}{0.000000in}}%
\pgfpathcurveto{\pgfqpoint{0.033333in}{0.008840in}}{\pgfqpoint{0.029821in}{0.017319in}}{\pgfqpoint{0.023570in}{0.023570in}}%
\pgfpathcurveto{\pgfqpoint{0.017319in}{0.029821in}}{\pgfqpoint{0.008840in}{0.033333in}}{\pgfqpoint{0.000000in}{0.033333in}}%
\pgfpathcurveto{\pgfqpoint{-0.008840in}{0.033333in}}{\pgfqpoint{-0.017319in}{0.029821in}}{\pgfqpoint{-0.023570in}{0.023570in}}%
\pgfpathcurveto{\pgfqpoint{-0.029821in}{0.017319in}}{\pgfqpoint{-0.033333in}{0.008840in}}{\pgfqpoint{-0.033333in}{0.000000in}}%
\pgfpathcurveto{\pgfqpoint{-0.033333in}{-0.008840in}}{\pgfqpoint{-0.029821in}{-0.017319in}}{\pgfqpoint{-0.023570in}{-0.023570in}}%
\pgfpathcurveto{\pgfqpoint{-0.017319in}{-0.029821in}}{\pgfqpoint{-0.008840in}{-0.033333in}}{\pgfqpoint{0.000000in}{-0.033333in}}%
\pgfpathlineto{\pgfqpoint{0.000000in}{-0.033333in}}%
\pgfpathclose%
\pgfusepath{stroke,fill}%
}%
\begin{pgfscope}%
\pgfsys@transformshift{3.788294in}{3.302185in}%
\pgfsys@useobject{currentmarker}{}%
\end{pgfscope}%
\end{pgfscope}%
\begin{pgfscope}%
\pgfpathrectangle{\pgfqpoint{0.765000in}{0.660000in}}{\pgfqpoint{4.620000in}{4.620000in}}%
\pgfusepath{clip}%
\pgfsetroundcap%
\pgfsetroundjoin%
\pgfsetlinewidth{1.204500pt}%
\definecolor{currentstroke}{rgb}{1.000000,0.576471,0.309804}%
\pgfsetstrokecolor{currentstroke}%
\pgfsetdash{}{0pt}%
\pgfpathmoveto{\pgfqpoint{2.460873in}{2.800462in}}%
\pgfusepath{stroke}%
\end{pgfscope}%
\begin{pgfscope}%
\pgfpathrectangle{\pgfqpoint{0.765000in}{0.660000in}}{\pgfqpoint{4.620000in}{4.620000in}}%
\pgfusepath{clip}%
\pgfsetbuttcap%
\pgfsetroundjoin%
\definecolor{currentfill}{rgb}{1.000000,0.576471,0.309804}%
\pgfsetfillcolor{currentfill}%
\pgfsetlinewidth{1.003750pt}%
\definecolor{currentstroke}{rgb}{1.000000,0.576471,0.309804}%
\pgfsetstrokecolor{currentstroke}%
\pgfsetdash{}{0pt}%
\pgfsys@defobject{currentmarker}{\pgfqpoint{-0.033333in}{-0.033333in}}{\pgfqpoint{0.033333in}{0.033333in}}{%
\pgfpathmoveto{\pgfqpoint{0.000000in}{-0.033333in}}%
\pgfpathcurveto{\pgfqpoint{0.008840in}{-0.033333in}}{\pgfqpoint{0.017319in}{-0.029821in}}{\pgfqpoint{0.023570in}{-0.023570in}}%
\pgfpathcurveto{\pgfqpoint{0.029821in}{-0.017319in}}{\pgfqpoint{0.033333in}{-0.008840in}}{\pgfqpoint{0.033333in}{0.000000in}}%
\pgfpathcurveto{\pgfqpoint{0.033333in}{0.008840in}}{\pgfqpoint{0.029821in}{0.017319in}}{\pgfqpoint{0.023570in}{0.023570in}}%
\pgfpathcurveto{\pgfqpoint{0.017319in}{0.029821in}}{\pgfqpoint{0.008840in}{0.033333in}}{\pgfqpoint{0.000000in}{0.033333in}}%
\pgfpathcurveto{\pgfqpoint{-0.008840in}{0.033333in}}{\pgfqpoint{-0.017319in}{0.029821in}}{\pgfqpoint{-0.023570in}{0.023570in}}%
\pgfpathcurveto{\pgfqpoint{-0.029821in}{0.017319in}}{\pgfqpoint{-0.033333in}{0.008840in}}{\pgfqpoint{-0.033333in}{0.000000in}}%
\pgfpathcurveto{\pgfqpoint{-0.033333in}{-0.008840in}}{\pgfqpoint{-0.029821in}{-0.017319in}}{\pgfqpoint{-0.023570in}{-0.023570in}}%
\pgfpathcurveto{\pgfqpoint{-0.017319in}{-0.029821in}}{\pgfqpoint{-0.008840in}{-0.033333in}}{\pgfqpoint{0.000000in}{-0.033333in}}%
\pgfpathlineto{\pgfqpoint{0.000000in}{-0.033333in}}%
\pgfpathclose%
\pgfusepath{stroke,fill}%
}%
\begin{pgfscope}%
\pgfsys@transformshift{2.460873in}{2.800462in}%
\pgfsys@useobject{currentmarker}{}%
\end{pgfscope}%
\end{pgfscope}%
\begin{pgfscope}%
\pgfpathrectangle{\pgfqpoint{0.765000in}{0.660000in}}{\pgfqpoint{4.620000in}{4.620000in}}%
\pgfusepath{clip}%
\pgfsetroundcap%
\pgfsetroundjoin%
\pgfsetlinewidth{1.204500pt}%
\definecolor{currentstroke}{rgb}{1.000000,0.576471,0.309804}%
\pgfsetstrokecolor{currentstroke}%
\pgfsetdash{}{0pt}%
\pgfpathmoveto{\pgfqpoint{2.558191in}{2.818904in}}%
\pgfusepath{stroke}%
\end{pgfscope}%
\begin{pgfscope}%
\pgfpathrectangle{\pgfqpoint{0.765000in}{0.660000in}}{\pgfqpoint{4.620000in}{4.620000in}}%
\pgfusepath{clip}%
\pgfsetbuttcap%
\pgfsetroundjoin%
\definecolor{currentfill}{rgb}{1.000000,0.576471,0.309804}%
\pgfsetfillcolor{currentfill}%
\pgfsetlinewidth{1.003750pt}%
\definecolor{currentstroke}{rgb}{1.000000,0.576471,0.309804}%
\pgfsetstrokecolor{currentstroke}%
\pgfsetdash{}{0pt}%
\pgfsys@defobject{currentmarker}{\pgfqpoint{-0.033333in}{-0.033333in}}{\pgfqpoint{0.033333in}{0.033333in}}{%
\pgfpathmoveto{\pgfqpoint{0.000000in}{-0.033333in}}%
\pgfpathcurveto{\pgfqpoint{0.008840in}{-0.033333in}}{\pgfqpoint{0.017319in}{-0.029821in}}{\pgfqpoint{0.023570in}{-0.023570in}}%
\pgfpathcurveto{\pgfqpoint{0.029821in}{-0.017319in}}{\pgfqpoint{0.033333in}{-0.008840in}}{\pgfqpoint{0.033333in}{0.000000in}}%
\pgfpathcurveto{\pgfqpoint{0.033333in}{0.008840in}}{\pgfqpoint{0.029821in}{0.017319in}}{\pgfqpoint{0.023570in}{0.023570in}}%
\pgfpathcurveto{\pgfqpoint{0.017319in}{0.029821in}}{\pgfqpoint{0.008840in}{0.033333in}}{\pgfqpoint{0.000000in}{0.033333in}}%
\pgfpathcurveto{\pgfqpoint{-0.008840in}{0.033333in}}{\pgfqpoint{-0.017319in}{0.029821in}}{\pgfqpoint{-0.023570in}{0.023570in}}%
\pgfpathcurveto{\pgfqpoint{-0.029821in}{0.017319in}}{\pgfqpoint{-0.033333in}{0.008840in}}{\pgfqpoint{-0.033333in}{0.000000in}}%
\pgfpathcurveto{\pgfqpoint{-0.033333in}{-0.008840in}}{\pgfqpoint{-0.029821in}{-0.017319in}}{\pgfqpoint{-0.023570in}{-0.023570in}}%
\pgfpathcurveto{\pgfqpoint{-0.017319in}{-0.029821in}}{\pgfqpoint{-0.008840in}{-0.033333in}}{\pgfqpoint{0.000000in}{-0.033333in}}%
\pgfpathlineto{\pgfqpoint{0.000000in}{-0.033333in}}%
\pgfpathclose%
\pgfusepath{stroke,fill}%
}%
\begin{pgfscope}%
\pgfsys@transformshift{2.558191in}{2.818904in}%
\pgfsys@useobject{currentmarker}{}%
\end{pgfscope}%
\end{pgfscope}%
\begin{pgfscope}%
\pgfpathrectangle{\pgfqpoint{0.765000in}{0.660000in}}{\pgfqpoint{4.620000in}{4.620000in}}%
\pgfusepath{clip}%
\pgfsetroundcap%
\pgfsetroundjoin%
\pgfsetlinewidth{1.204500pt}%
\definecolor{currentstroke}{rgb}{1.000000,0.576471,0.309804}%
\pgfsetstrokecolor{currentstroke}%
\pgfsetdash{}{0pt}%
\pgfpathmoveto{\pgfqpoint{2.005584in}{2.611973in}}%
\pgfusepath{stroke}%
\end{pgfscope}%
\begin{pgfscope}%
\pgfpathrectangle{\pgfqpoint{0.765000in}{0.660000in}}{\pgfqpoint{4.620000in}{4.620000in}}%
\pgfusepath{clip}%
\pgfsetbuttcap%
\pgfsetroundjoin%
\definecolor{currentfill}{rgb}{1.000000,0.576471,0.309804}%
\pgfsetfillcolor{currentfill}%
\pgfsetlinewidth{1.003750pt}%
\definecolor{currentstroke}{rgb}{1.000000,0.576471,0.309804}%
\pgfsetstrokecolor{currentstroke}%
\pgfsetdash{}{0pt}%
\pgfsys@defobject{currentmarker}{\pgfqpoint{-0.033333in}{-0.033333in}}{\pgfqpoint{0.033333in}{0.033333in}}{%
\pgfpathmoveto{\pgfqpoint{0.000000in}{-0.033333in}}%
\pgfpathcurveto{\pgfqpoint{0.008840in}{-0.033333in}}{\pgfqpoint{0.017319in}{-0.029821in}}{\pgfqpoint{0.023570in}{-0.023570in}}%
\pgfpathcurveto{\pgfqpoint{0.029821in}{-0.017319in}}{\pgfqpoint{0.033333in}{-0.008840in}}{\pgfqpoint{0.033333in}{0.000000in}}%
\pgfpathcurveto{\pgfqpoint{0.033333in}{0.008840in}}{\pgfqpoint{0.029821in}{0.017319in}}{\pgfqpoint{0.023570in}{0.023570in}}%
\pgfpathcurveto{\pgfqpoint{0.017319in}{0.029821in}}{\pgfqpoint{0.008840in}{0.033333in}}{\pgfqpoint{0.000000in}{0.033333in}}%
\pgfpathcurveto{\pgfqpoint{-0.008840in}{0.033333in}}{\pgfqpoint{-0.017319in}{0.029821in}}{\pgfqpoint{-0.023570in}{0.023570in}}%
\pgfpathcurveto{\pgfqpoint{-0.029821in}{0.017319in}}{\pgfqpoint{-0.033333in}{0.008840in}}{\pgfqpoint{-0.033333in}{0.000000in}}%
\pgfpathcurveto{\pgfqpoint{-0.033333in}{-0.008840in}}{\pgfqpoint{-0.029821in}{-0.017319in}}{\pgfqpoint{-0.023570in}{-0.023570in}}%
\pgfpathcurveto{\pgfqpoint{-0.017319in}{-0.029821in}}{\pgfqpoint{-0.008840in}{-0.033333in}}{\pgfqpoint{0.000000in}{-0.033333in}}%
\pgfpathlineto{\pgfqpoint{0.000000in}{-0.033333in}}%
\pgfpathclose%
\pgfusepath{stroke,fill}%
}%
\begin{pgfscope}%
\pgfsys@transformshift{2.005584in}{2.611973in}%
\pgfsys@useobject{currentmarker}{}%
\end{pgfscope}%
\end{pgfscope}%
\begin{pgfscope}%
\pgfpathrectangle{\pgfqpoint{0.765000in}{0.660000in}}{\pgfqpoint{4.620000in}{4.620000in}}%
\pgfusepath{clip}%
\pgfsetroundcap%
\pgfsetroundjoin%
\pgfsetlinewidth{1.204500pt}%
\definecolor{currentstroke}{rgb}{1.000000,0.576471,0.309804}%
\pgfsetstrokecolor{currentstroke}%
\pgfsetdash{}{0pt}%
\pgfpathmoveto{\pgfqpoint{2.489570in}{3.127417in}}%
\pgfusepath{stroke}%
\end{pgfscope}%
\begin{pgfscope}%
\pgfpathrectangle{\pgfqpoint{0.765000in}{0.660000in}}{\pgfqpoint{4.620000in}{4.620000in}}%
\pgfusepath{clip}%
\pgfsetbuttcap%
\pgfsetroundjoin%
\definecolor{currentfill}{rgb}{1.000000,0.576471,0.309804}%
\pgfsetfillcolor{currentfill}%
\pgfsetlinewidth{1.003750pt}%
\definecolor{currentstroke}{rgb}{1.000000,0.576471,0.309804}%
\pgfsetstrokecolor{currentstroke}%
\pgfsetdash{}{0pt}%
\pgfsys@defobject{currentmarker}{\pgfqpoint{-0.033333in}{-0.033333in}}{\pgfqpoint{0.033333in}{0.033333in}}{%
\pgfpathmoveto{\pgfqpoint{0.000000in}{-0.033333in}}%
\pgfpathcurveto{\pgfqpoint{0.008840in}{-0.033333in}}{\pgfqpoint{0.017319in}{-0.029821in}}{\pgfqpoint{0.023570in}{-0.023570in}}%
\pgfpathcurveto{\pgfqpoint{0.029821in}{-0.017319in}}{\pgfqpoint{0.033333in}{-0.008840in}}{\pgfqpoint{0.033333in}{0.000000in}}%
\pgfpathcurveto{\pgfqpoint{0.033333in}{0.008840in}}{\pgfqpoint{0.029821in}{0.017319in}}{\pgfqpoint{0.023570in}{0.023570in}}%
\pgfpathcurveto{\pgfqpoint{0.017319in}{0.029821in}}{\pgfqpoint{0.008840in}{0.033333in}}{\pgfqpoint{0.000000in}{0.033333in}}%
\pgfpathcurveto{\pgfqpoint{-0.008840in}{0.033333in}}{\pgfqpoint{-0.017319in}{0.029821in}}{\pgfqpoint{-0.023570in}{0.023570in}}%
\pgfpathcurveto{\pgfqpoint{-0.029821in}{0.017319in}}{\pgfqpoint{-0.033333in}{0.008840in}}{\pgfqpoint{-0.033333in}{0.000000in}}%
\pgfpathcurveto{\pgfqpoint{-0.033333in}{-0.008840in}}{\pgfqpoint{-0.029821in}{-0.017319in}}{\pgfqpoint{-0.023570in}{-0.023570in}}%
\pgfpathcurveto{\pgfqpoint{-0.017319in}{-0.029821in}}{\pgfqpoint{-0.008840in}{-0.033333in}}{\pgfqpoint{0.000000in}{-0.033333in}}%
\pgfpathlineto{\pgfqpoint{0.000000in}{-0.033333in}}%
\pgfpathclose%
\pgfusepath{stroke,fill}%
}%
\begin{pgfscope}%
\pgfsys@transformshift{2.489570in}{3.127417in}%
\pgfsys@useobject{currentmarker}{}%
\end{pgfscope}%
\end{pgfscope}%
\begin{pgfscope}%
\pgfpathrectangle{\pgfqpoint{0.765000in}{0.660000in}}{\pgfqpoint{4.620000in}{4.620000in}}%
\pgfusepath{clip}%
\pgfsetroundcap%
\pgfsetroundjoin%
\pgfsetlinewidth{1.204500pt}%
\definecolor{currentstroke}{rgb}{1.000000,0.576471,0.309804}%
\pgfsetstrokecolor{currentstroke}%
\pgfsetdash{}{0pt}%
\pgfpathmoveto{\pgfqpoint{3.766684in}{2.840859in}}%
\pgfusepath{stroke}%
\end{pgfscope}%
\begin{pgfscope}%
\pgfpathrectangle{\pgfqpoint{0.765000in}{0.660000in}}{\pgfqpoint{4.620000in}{4.620000in}}%
\pgfusepath{clip}%
\pgfsetbuttcap%
\pgfsetroundjoin%
\definecolor{currentfill}{rgb}{1.000000,0.576471,0.309804}%
\pgfsetfillcolor{currentfill}%
\pgfsetlinewidth{1.003750pt}%
\definecolor{currentstroke}{rgb}{1.000000,0.576471,0.309804}%
\pgfsetstrokecolor{currentstroke}%
\pgfsetdash{}{0pt}%
\pgfsys@defobject{currentmarker}{\pgfqpoint{-0.033333in}{-0.033333in}}{\pgfqpoint{0.033333in}{0.033333in}}{%
\pgfpathmoveto{\pgfqpoint{0.000000in}{-0.033333in}}%
\pgfpathcurveto{\pgfqpoint{0.008840in}{-0.033333in}}{\pgfqpoint{0.017319in}{-0.029821in}}{\pgfqpoint{0.023570in}{-0.023570in}}%
\pgfpathcurveto{\pgfqpoint{0.029821in}{-0.017319in}}{\pgfqpoint{0.033333in}{-0.008840in}}{\pgfqpoint{0.033333in}{0.000000in}}%
\pgfpathcurveto{\pgfqpoint{0.033333in}{0.008840in}}{\pgfqpoint{0.029821in}{0.017319in}}{\pgfqpoint{0.023570in}{0.023570in}}%
\pgfpathcurveto{\pgfqpoint{0.017319in}{0.029821in}}{\pgfqpoint{0.008840in}{0.033333in}}{\pgfqpoint{0.000000in}{0.033333in}}%
\pgfpathcurveto{\pgfqpoint{-0.008840in}{0.033333in}}{\pgfqpoint{-0.017319in}{0.029821in}}{\pgfqpoint{-0.023570in}{0.023570in}}%
\pgfpathcurveto{\pgfqpoint{-0.029821in}{0.017319in}}{\pgfqpoint{-0.033333in}{0.008840in}}{\pgfqpoint{-0.033333in}{0.000000in}}%
\pgfpathcurveto{\pgfqpoint{-0.033333in}{-0.008840in}}{\pgfqpoint{-0.029821in}{-0.017319in}}{\pgfqpoint{-0.023570in}{-0.023570in}}%
\pgfpathcurveto{\pgfqpoint{-0.017319in}{-0.029821in}}{\pgfqpoint{-0.008840in}{-0.033333in}}{\pgfqpoint{0.000000in}{-0.033333in}}%
\pgfpathlineto{\pgfqpoint{0.000000in}{-0.033333in}}%
\pgfpathclose%
\pgfusepath{stroke,fill}%
}%
\begin{pgfscope}%
\pgfsys@transformshift{3.766684in}{2.840859in}%
\pgfsys@useobject{currentmarker}{}%
\end{pgfscope}%
\end{pgfscope}%
\begin{pgfscope}%
\pgfpathrectangle{\pgfqpoint{0.765000in}{0.660000in}}{\pgfqpoint{4.620000in}{4.620000in}}%
\pgfusepath{clip}%
\pgfsetroundcap%
\pgfsetroundjoin%
\pgfsetlinewidth{1.204500pt}%
\definecolor{currentstroke}{rgb}{1.000000,0.576471,0.309804}%
\pgfsetstrokecolor{currentstroke}%
\pgfsetdash{}{0pt}%
\pgfpathmoveto{\pgfqpoint{2.549685in}{3.052487in}}%
\pgfusepath{stroke}%
\end{pgfscope}%
\begin{pgfscope}%
\pgfpathrectangle{\pgfqpoint{0.765000in}{0.660000in}}{\pgfqpoint{4.620000in}{4.620000in}}%
\pgfusepath{clip}%
\pgfsetbuttcap%
\pgfsetroundjoin%
\definecolor{currentfill}{rgb}{1.000000,0.576471,0.309804}%
\pgfsetfillcolor{currentfill}%
\pgfsetlinewidth{1.003750pt}%
\definecolor{currentstroke}{rgb}{1.000000,0.576471,0.309804}%
\pgfsetstrokecolor{currentstroke}%
\pgfsetdash{}{0pt}%
\pgfsys@defobject{currentmarker}{\pgfqpoint{-0.033333in}{-0.033333in}}{\pgfqpoint{0.033333in}{0.033333in}}{%
\pgfpathmoveto{\pgfqpoint{0.000000in}{-0.033333in}}%
\pgfpathcurveto{\pgfqpoint{0.008840in}{-0.033333in}}{\pgfqpoint{0.017319in}{-0.029821in}}{\pgfqpoint{0.023570in}{-0.023570in}}%
\pgfpathcurveto{\pgfqpoint{0.029821in}{-0.017319in}}{\pgfqpoint{0.033333in}{-0.008840in}}{\pgfqpoint{0.033333in}{0.000000in}}%
\pgfpathcurveto{\pgfqpoint{0.033333in}{0.008840in}}{\pgfqpoint{0.029821in}{0.017319in}}{\pgfqpoint{0.023570in}{0.023570in}}%
\pgfpathcurveto{\pgfqpoint{0.017319in}{0.029821in}}{\pgfqpoint{0.008840in}{0.033333in}}{\pgfqpoint{0.000000in}{0.033333in}}%
\pgfpathcurveto{\pgfqpoint{-0.008840in}{0.033333in}}{\pgfqpoint{-0.017319in}{0.029821in}}{\pgfqpoint{-0.023570in}{0.023570in}}%
\pgfpathcurveto{\pgfqpoint{-0.029821in}{0.017319in}}{\pgfqpoint{-0.033333in}{0.008840in}}{\pgfqpoint{-0.033333in}{0.000000in}}%
\pgfpathcurveto{\pgfqpoint{-0.033333in}{-0.008840in}}{\pgfqpoint{-0.029821in}{-0.017319in}}{\pgfqpoint{-0.023570in}{-0.023570in}}%
\pgfpathcurveto{\pgfqpoint{-0.017319in}{-0.029821in}}{\pgfqpoint{-0.008840in}{-0.033333in}}{\pgfqpoint{0.000000in}{-0.033333in}}%
\pgfpathlineto{\pgfqpoint{0.000000in}{-0.033333in}}%
\pgfpathclose%
\pgfusepath{stroke,fill}%
}%
\begin{pgfscope}%
\pgfsys@transformshift{2.549685in}{3.052487in}%
\pgfsys@useobject{currentmarker}{}%
\end{pgfscope}%
\end{pgfscope}%
\begin{pgfscope}%
\pgfpathrectangle{\pgfqpoint{0.765000in}{0.660000in}}{\pgfqpoint{4.620000in}{4.620000in}}%
\pgfusepath{clip}%
\pgfsetroundcap%
\pgfsetroundjoin%
\pgfsetlinewidth{1.204500pt}%
\definecolor{currentstroke}{rgb}{1.000000,0.576471,0.309804}%
\pgfsetstrokecolor{currentstroke}%
\pgfsetdash{}{0pt}%
\pgfpathmoveto{\pgfqpoint{4.096976in}{2.812076in}}%
\pgfusepath{stroke}%
\end{pgfscope}%
\begin{pgfscope}%
\pgfpathrectangle{\pgfqpoint{0.765000in}{0.660000in}}{\pgfqpoint{4.620000in}{4.620000in}}%
\pgfusepath{clip}%
\pgfsetbuttcap%
\pgfsetroundjoin%
\definecolor{currentfill}{rgb}{1.000000,0.576471,0.309804}%
\pgfsetfillcolor{currentfill}%
\pgfsetlinewidth{1.003750pt}%
\definecolor{currentstroke}{rgb}{1.000000,0.576471,0.309804}%
\pgfsetstrokecolor{currentstroke}%
\pgfsetdash{}{0pt}%
\pgfsys@defobject{currentmarker}{\pgfqpoint{-0.033333in}{-0.033333in}}{\pgfqpoint{0.033333in}{0.033333in}}{%
\pgfpathmoveto{\pgfqpoint{0.000000in}{-0.033333in}}%
\pgfpathcurveto{\pgfqpoint{0.008840in}{-0.033333in}}{\pgfqpoint{0.017319in}{-0.029821in}}{\pgfqpoint{0.023570in}{-0.023570in}}%
\pgfpathcurveto{\pgfqpoint{0.029821in}{-0.017319in}}{\pgfqpoint{0.033333in}{-0.008840in}}{\pgfqpoint{0.033333in}{0.000000in}}%
\pgfpathcurveto{\pgfqpoint{0.033333in}{0.008840in}}{\pgfqpoint{0.029821in}{0.017319in}}{\pgfqpoint{0.023570in}{0.023570in}}%
\pgfpathcurveto{\pgfqpoint{0.017319in}{0.029821in}}{\pgfqpoint{0.008840in}{0.033333in}}{\pgfqpoint{0.000000in}{0.033333in}}%
\pgfpathcurveto{\pgfqpoint{-0.008840in}{0.033333in}}{\pgfqpoint{-0.017319in}{0.029821in}}{\pgfqpoint{-0.023570in}{0.023570in}}%
\pgfpathcurveto{\pgfqpoint{-0.029821in}{0.017319in}}{\pgfqpoint{-0.033333in}{0.008840in}}{\pgfqpoint{-0.033333in}{0.000000in}}%
\pgfpathcurveto{\pgfqpoint{-0.033333in}{-0.008840in}}{\pgfqpoint{-0.029821in}{-0.017319in}}{\pgfqpoint{-0.023570in}{-0.023570in}}%
\pgfpathcurveto{\pgfqpoint{-0.017319in}{-0.029821in}}{\pgfqpoint{-0.008840in}{-0.033333in}}{\pgfqpoint{0.000000in}{-0.033333in}}%
\pgfpathlineto{\pgfqpoint{0.000000in}{-0.033333in}}%
\pgfpathclose%
\pgfusepath{stroke,fill}%
}%
\begin{pgfscope}%
\pgfsys@transformshift{4.096976in}{2.812076in}%
\pgfsys@useobject{currentmarker}{}%
\end{pgfscope}%
\end{pgfscope}%
\begin{pgfscope}%
\pgfpathrectangle{\pgfqpoint{0.765000in}{0.660000in}}{\pgfqpoint{4.620000in}{4.620000in}}%
\pgfusepath{clip}%
\pgfsetroundcap%
\pgfsetroundjoin%
\pgfsetlinewidth{1.204500pt}%
\definecolor{currentstroke}{rgb}{1.000000,0.576471,0.309804}%
\pgfsetstrokecolor{currentstroke}%
\pgfsetdash{}{0pt}%
\pgfpathmoveto{\pgfqpoint{2.804662in}{2.478409in}}%
\pgfusepath{stroke}%
\end{pgfscope}%
\begin{pgfscope}%
\pgfpathrectangle{\pgfqpoint{0.765000in}{0.660000in}}{\pgfqpoint{4.620000in}{4.620000in}}%
\pgfusepath{clip}%
\pgfsetbuttcap%
\pgfsetroundjoin%
\definecolor{currentfill}{rgb}{1.000000,0.576471,0.309804}%
\pgfsetfillcolor{currentfill}%
\pgfsetlinewidth{1.003750pt}%
\definecolor{currentstroke}{rgb}{1.000000,0.576471,0.309804}%
\pgfsetstrokecolor{currentstroke}%
\pgfsetdash{}{0pt}%
\pgfsys@defobject{currentmarker}{\pgfqpoint{-0.033333in}{-0.033333in}}{\pgfqpoint{0.033333in}{0.033333in}}{%
\pgfpathmoveto{\pgfqpoint{0.000000in}{-0.033333in}}%
\pgfpathcurveto{\pgfqpoint{0.008840in}{-0.033333in}}{\pgfqpoint{0.017319in}{-0.029821in}}{\pgfqpoint{0.023570in}{-0.023570in}}%
\pgfpathcurveto{\pgfqpoint{0.029821in}{-0.017319in}}{\pgfqpoint{0.033333in}{-0.008840in}}{\pgfqpoint{0.033333in}{0.000000in}}%
\pgfpathcurveto{\pgfqpoint{0.033333in}{0.008840in}}{\pgfqpoint{0.029821in}{0.017319in}}{\pgfqpoint{0.023570in}{0.023570in}}%
\pgfpathcurveto{\pgfqpoint{0.017319in}{0.029821in}}{\pgfqpoint{0.008840in}{0.033333in}}{\pgfqpoint{0.000000in}{0.033333in}}%
\pgfpathcurveto{\pgfqpoint{-0.008840in}{0.033333in}}{\pgfqpoint{-0.017319in}{0.029821in}}{\pgfqpoint{-0.023570in}{0.023570in}}%
\pgfpathcurveto{\pgfqpoint{-0.029821in}{0.017319in}}{\pgfqpoint{-0.033333in}{0.008840in}}{\pgfqpoint{-0.033333in}{0.000000in}}%
\pgfpathcurveto{\pgfqpoint{-0.033333in}{-0.008840in}}{\pgfqpoint{-0.029821in}{-0.017319in}}{\pgfqpoint{-0.023570in}{-0.023570in}}%
\pgfpathcurveto{\pgfqpoint{-0.017319in}{-0.029821in}}{\pgfqpoint{-0.008840in}{-0.033333in}}{\pgfqpoint{0.000000in}{-0.033333in}}%
\pgfpathlineto{\pgfqpoint{0.000000in}{-0.033333in}}%
\pgfpathclose%
\pgfusepath{stroke,fill}%
}%
\begin{pgfscope}%
\pgfsys@transformshift{2.804662in}{2.478409in}%
\pgfsys@useobject{currentmarker}{}%
\end{pgfscope}%
\end{pgfscope}%
\begin{pgfscope}%
\pgfpathrectangle{\pgfqpoint{0.765000in}{0.660000in}}{\pgfqpoint{4.620000in}{4.620000in}}%
\pgfusepath{clip}%
\pgfsetroundcap%
\pgfsetroundjoin%
\pgfsetlinewidth{1.204500pt}%
\definecolor{currentstroke}{rgb}{1.000000,0.576471,0.309804}%
\pgfsetstrokecolor{currentstroke}%
\pgfsetdash{}{0pt}%
\pgfpathmoveto{\pgfqpoint{4.264167in}{2.841102in}}%
\pgfusepath{stroke}%
\end{pgfscope}%
\begin{pgfscope}%
\pgfpathrectangle{\pgfqpoint{0.765000in}{0.660000in}}{\pgfqpoint{4.620000in}{4.620000in}}%
\pgfusepath{clip}%
\pgfsetbuttcap%
\pgfsetroundjoin%
\definecolor{currentfill}{rgb}{1.000000,0.576471,0.309804}%
\pgfsetfillcolor{currentfill}%
\pgfsetlinewidth{1.003750pt}%
\definecolor{currentstroke}{rgb}{1.000000,0.576471,0.309804}%
\pgfsetstrokecolor{currentstroke}%
\pgfsetdash{}{0pt}%
\pgfsys@defobject{currentmarker}{\pgfqpoint{-0.033333in}{-0.033333in}}{\pgfqpoint{0.033333in}{0.033333in}}{%
\pgfpathmoveto{\pgfqpoint{0.000000in}{-0.033333in}}%
\pgfpathcurveto{\pgfqpoint{0.008840in}{-0.033333in}}{\pgfqpoint{0.017319in}{-0.029821in}}{\pgfqpoint{0.023570in}{-0.023570in}}%
\pgfpathcurveto{\pgfqpoint{0.029821in}{-0.017319in}}{\pgfqpoint{0.033333in}{-0.008840in}}{\pgfqpoint{0.033333in}{0.000000in}}%
\pgfpathcurveto{\pgfqpoint{0.033333in}{0.008840in}}{\pgfqpoint{0.029821in}{0.017319in}}{\pgfqpoint{0.023570in}{0.023570in}}%
\pgfpathcurveto{\pgfqpoint{0.017319in}{0.029821in}}{\pgfqpoint{0.008840in}{0.033333in}}{\pgfqpoint{0.000000in}{0.033333in}}%
\pgfpathcurveto{\pgfqpoint{-0.008840in}{0.033333in}}{\pgfqpoint{-0.017319in}{0.029821in}}{\pgfqpoint{-0.023570in}{0.023570in}}%
\pgfpathcurveto{\pgfqpoint{-0.029821in}{0.017319in}}{\pgfqpoint{-0.033333in}{0.008840in}}{\pgfqpoint{-0.033333in}{0.000000in}}%
\pgfpathcurveto{\pgfqpoint{-0.033333in}{-0.008840in}}{\pgfqpoint{-0.029821in}{-0.017319in}}{\pgfqpoint{-0.023570in}{-0.023570in}}%
\pgfpathcurveto{\pgfqpoint{-0.017319in}{-0.029821in}}{\pgfqpoint{-0.008840in}{-0.033333in}}{\pgfqpoint{0.000000in}{-0.033333in}}%
\pgfpathlineto{\pgfqpoint{0.000000in}{-0.033333in}}%
\pgfpathclose%
\pgfusepath{stroke,fill}%
}%
\begin{pgfscope}%
\pgfsys@transformshift{4.264167in}{2.841102in}%
\pgfsys@useobject{currentmarker}{}%
\end{pgfscope}%
\end{pgfscope}%
\begin{pgfscope}%
\pgfpathrectangle{\pgfqpoint{0.765000in}{0.660000in}}{\pgfqpoint{4.620000in}{4.620000in}}%
\pgfusepath{clip}%
\pgfsetroundcap%
\pgfsetroundjoin%
\pgfsetlinewidth{1.204500pt}%
\definecolor{currentstroke}{rgb}{1.000000,0.576471,0.309804}%
\pgfsetstrokecolor{currentstroke}%
\pgfsetdash{}{0pt}%
\pgfpathmoveto{\pgfqpoint{3.703119in}{2.413106in}}%
\pgfusepath{stroke}%
\end{pgfscope}%
\begin{pgfscope}%
\pgfpathrectangle{\pgfqpoint{0.765000in}{0.660000in}}{\pgfqpoint{4.620000in}{4.620000in}}%
\pgfusepath{clip}%
\pgfsetbuttcap%
\pgfsetroundjoin%
\definecolor{currentfill}{rgb}{1.000000,0.576471,0.309804}%
\pgfsetfillcolor{currentfill}%
\pgfsetlinewidth{1.003750pt}%
\definecolor{currentstroke}{rgb}{1.000000,0.576471,0.309804}%
\pgfsetstrokecolor{currentstroke}%
\pgfsetdash{}{0pt}%
\pgfsys@defobject{currentmarker}{\pgfqpoint{-0.033333in}{-0.033333in}}{\pgfqpoint{0.033333in}{0.033333in}}{%
\pgfpathmoveto{\pgfqpoint{0.000000in}{-0.033333in}}%
\pgfpathcurveto{\pgfqpoint{0.008840in}{-0.033333in}}{\pgfqpoint{0.017319in}{-0.029821in}}{\pgfqpoint{0.023570in}{-0.023570in}}%
\pgfpathcurveto{\pgfqpoint{0.029821in}{-0.017319in}}{\pgfqpoint{0.033333in}{-0.008840in}}{\pgfqpoint{0.033333in}{0.000000in}}%
\pgfpathcurveto{\pgfqpoint{0.033333in}{0.008840in}}{\pgfqpoint{0.029821in}{0.017319in}}{\pgfqpoint{0.023570in}{0.023570in}}%
\pgfpathcurveto{\pgfqpoint{0.017319in}{0.029821in}}{\pgfqpoint{0.008840in}{0.033333in}}{\pgfqpoint{0.000000in}{0.033333in}}%
\pgfpathcurveto{\pgfqpoint{-0.008840in}{0.033333in}}{\pgfqpoint{-0.017319in}{0.029821in}}{\pgfqpoint{-0.023570in}{0.023570in}}%
\pgfpathcurveto{\pgfqpoint{-0.029821in}{0.017319in}}{\pgfqpoint{-0.033333in}{0.008840in}}{\pgfqpoint{-0.033333in}{0.000000in}}%
\pgfpathcurveto{\pgfqpoint{-0.033333in}{-0.008840in}}{\pgfqpoint{-0.029821in}{-0.017319in}}{\pgfqpoint{-0.023570in}{-0.023570in}}%
\pgfpathcurveto{\pgfqpoint{-0.017319in}{-0.029821in}}{\pgfqpoint{-0.008840in}{-0.033333in}}{\pgfqpoint{0.000000in}{-0.033333in}}%
\pgfpathlineto{\pgfqpoint{0.000000in}{-0.033333in}}%
\pgfpathclose%
\pgfusepath{stroke,fill}%
}%
\begin{pgfscope}%
\pgfsys@transformshift{3.703119in}{2.413106in}%
\pgfsys@useobject{currentmarker}{}%
\end{pgfscope}%
\end{pgfscope}%
\begin{pgfscope}%
\pgfpathrectangle{\pgfqpoint{0.765000in}{0.660000in}}{\pgfqpoint{4.620000in}{4.620000in}}%
\pgfusepath{clip}%
\pgfsetroundcap%
\pgfsetroundjoin%
\pgfsetlinewidth{1.204500pt}%
\definecolor{currentstroke}{rgb}{1.000000,0.576471,0.309804}%
\pgfsetstrokecolor{currentstroke}%
\pgfsetdash{}{0pt}%
\pgfpathmoveto{\pgfqpoint{2.786654in}{3.408306in}}%
\pgfusepath{stroke}%
\end{pgfscope}%
\begin{pgfscope}%
\pgfpathrectangle{\pgfqpoint{0.765000in}{0.660000in}}{\pgfqpoint{4.620000in}{4.620000in}}%
\pgfusepath{clip}%
\pgfsetbuttcap%
\pgfsetroundjoin%
\definecolor{currentfill}{rgb}{1.000000,0.576471,0.309804}%
\pgfsetfillcolor{currentfill}%
\pgfsetlinewidth{1.003750pt}%
\definecolor{currentstroke}{rgb}{1.000000,0.576471,0.309804}%
\pgfsetstrokecolor{currentstroke}%
\pgfsetdash{}{0pt}%
\pgfsys@defobject{currentmarker}{\pgfqpoint{-0.033333in}{-0.033333in}}{\pgfqpoint{0.033333in}{0.033333in}}{%
\pgfpathmoveto{\pgfqpoint{0.000000in}{-0.033333in}}%
\pgfpathcurveto{\pgfqpoint{0.008840in}{-0.033333in}}{\pgfqpoint{0.017319in}{-0.029821in}}{\pgfqpoint{0.023570in}{-0.023570in}}%
\pgfpathcurveto{\pgfqpoint{0.029821in}{-0.017319in}}{\pgfqpoint{0.033333in}{-0.008840in}}{\pgfqpoint{0.033333in}{0.000000in}}%
\pgfpathcurveto{\pgfqpoint{0.033333in}{0.008840in}}{\pgfqpoint{0.029821in}{0.017319in}}{\pgfqpoint{0.023570in}{0.023570in}}%
\pgfpathcurveto{\pgfqpoint{0.017319in}{0.029821in}}{\pgfqpoint{0.008840in}{0.033333in}}{\pgfqpoint{0.000000in}{0.033333in}}%
\pgfpathcurveto{\pgfqpoint{-0.008840in}{0.033333in}}{\pgfqpoint{-0.017319in}{0.029821in}}{\pgfqpoint{-0.023570in}{0.023570in}}%
\pgfpathcurveto{\pgfqpoint{-0.029821in}{0.017319in}}{\pgfqpoint{-0.033333in}{0.008840in}}{\pgfqpoint{-0.033333in}{0.000000in}}%
\pgfpathcurveto{\pgfqpoint{-0.033333in}{-0.008840in}}{\pgfqpoint{-0.029821in}{-0.017319in}}{\pgfqpoint{-0.023570in}{-0.023570in}}%
\pgfpathcurveto{\pgfqpoint{-0.017319in}{-0.029821in}}{\pgfqpoint{-0.008840in}{-0.033333in}}{\pgfqpoint{0.000000in}{-0.033333in}}%
\pgfpathlineto{\pgfqpoint{0.000000in}{-0.033333in}}%
\pgfpathclose%
\pgfusepath{stroke,fill}%
}%
\begin{pgfscope}%
\pgfsys@transformshift{2.786654in}{3.408306in}%
\pgfsys@useobject{currentmarker}{}%
\end{pgfscope}%
\end{pgfscope}%
\begin{pgfscope}%
\pgfpathrectangle{\pgfqpoint{0.765000in}{0.660000in}}{\pgfqpoint{4.620000in}{4.620000in}}%
\pgfusepath{clip}%
\pgfsetroundcap%
\pgfsetroundjoin%
\pgfsetlinewidth{1.204500pt}%
\definecolor{currentstroke}{rgb}{1.000000,0.576471,0.309804}%
\pgfsetstrokecolor{currentstroke}%
\pgfsetdash{}{0pt}%
\pgfpathmoveto{\pgfqpoint{1.927470in}{3.089317in}}%
\pgfusepath{stroke}%
\end{pgfscope}%
\begin{pgfscope}%
\pgfpathrectangle{\pgfqpoint{0.765000in}{0.660000in}}{\pgfqpoint{4.620000in}{4.620000in}}%
\pgfusepath{clip}%
\pgfsetbuttcap%
\pgfsetroundjoin%
\definecolor{currentfill}{rgb}{1.000000,0.576471,0.309804}%
\pgfsetfillcolor{currentfill}%
\pgfsetlinewidth{1.003750pt}%
\definecolor{currentstroke}{rgb}{1.000000,0.576471,0.309804}%
\pgfsetstrokecolor{currentstroke}%
\pgfsetdash{}{0pt}%
\pgfsys@defobject{currentmarker}{\pgfqpoint{-0.033333in}{-0.033333in}}{\pgfqpoint{0.033333in}{0.033333in}}{%
\pgfpathmoveto{\pgfqpoint{0.000000in}{-0.033333in}}%
\pgfpathcurveto{\pgfqpoint{0.008840in}{-0.033333in}}{\pgfqpoint{0.017319in}{-0.029821in}}{\pgfqpoint{0.023570in}{-0.023570in}}%
\pgfpathcurveto{\pgfqpoint{0.029821in}{-0.017319in}}{\pgfqpoint{0.033333in}{-0.008840in}}{\pgfqpoint{0.033333in}{0.000000in}}%
\pgfpathcurveto{\pgfqpoint{0.033333in}{0.008840in}}{\pgfqpoint{0.029821in}{0.017319in}}{\pgfqpoint{0.023570in}{0.023570in}}%
\pgfpathcurveto{\pgfqpoint{0.017319in}{0.029821in}}{\pgfqpoint{0.008840in}{0.033333in}}{\pgfqpoint{0.000000in}{0.033333in}}%
\pgfpathcurveto{\pgfqpoint{-0.008840in}{0.033333in}}{\pgfqpoint{-0.017319in}{0.029821in}}{\pgfqpoint{-0.023570in}{0.023570in}}%
\pgfpathcurveto{\pgfqpoint{-0.029821in}{0.017319in}}{\pgfqpoint{-0.033333in}{0.008840in}}{\pgfqpoint{-0.033333in}{0.000000in}}%
\pgfpathcurveto{\pgfqpoint{-0.033333in}{-0.008840in}}{\pgfqpoint{-0.029821in}{-0.017319in}}{\pgfqpoint{-0.023570in}{-0.023570in}}%
\pgfpathcurveto{\pgfqpoint{-0.017319in}{-0.029821in}}{\pgfqpoint{-0.008840in}{-0.033333in}}{\pgfqpoint{0.000000in}{-0.033333in}}%
\pgfpathlineto{\pgfqpoint{0.000000in}{-0.033333in}}%
\pgfpathclose%
\pgfusepath{stroke,fill}%
}%
\begin{pgfscope}%
\pgfsys@transformshift{1.927470in}{3.089317in}%
\pgfsys@useobject{currentmarker}{}%
\end{pgfscope}%
\end{pgfscope}%
\begin{pgfscope}%
\pgfpathrectangle{\pgfqpoint{0.765000in}{0.660000in}}{\pgfqpoint{4.620000in}{4.620000in}}%
\pgfusepath{clip}%
\pgfsetroundcap%
\pgfsetroundjoin%
\pgfsetlinewidth{1.204500pt}%
\definecolor{currentstroke}{rgb}{1.000000,0.576471,0.309804}%
\pgfsetstrokecolor{currentstroke}%
\pgfsetdash{}{0pt}%
\pgfpathmoveto{\pgfqpoint{2.545699in}{3.168049in}}%
\pgfusepath{stroke}%
\end{pgfscope}%
\begin{pgfscope}%
\pgfpathrectangle{\pgfqpoint{0.765000in}{0.660000in}}{\pgfqpoint{4.620000in}{4.620000in}}%
\pgfusepath{clip}%
\pgfsetbuttcap%
\pgfsetroundjoin%
\definecolor{currentfill}{rgb}{1.000000,0.576471,0.309804}%
\pgfsetfillcolor{currentfill}%
\pgfsetlinewidth{1.003750pt}%
\definecolor{currentstroke}{rgb}{1.000000,0.576471,0.309804}%
\pgfsetstrokecolor{currentstroke}%
\pgfsetdash{}{0pt}%
\pgfsys@defobject{currentmarker}{\pgfqpoint{-0.033333in}{-0.033333in}}{\pgfqpoint{0.033333in}{0.033333in}}{%
\pgfpathmoveto{\pgfqpoint{0.000000in}{-0.033333in}}%
\pgfpathcurveto{\pgfqpoint{0.008840in}{-0.033333in}}{\pgfqpoint{0.017319in}{-0.029821in}}{\pgfqpoint{0.023570in}{-0.023570in}}%
\pgfpathcurveto{\pgfqpoint{0.029821in}{-0.017319in}}{\pgfqpoint{0.033333in}{-0.008840in}}{\pgfqpoint{0.033333in}{0.000000in}}%
\pgfpathcurveto{\pgfqpoint{0.033333in}{0.008840in}}{\pgfqpoint{0.029821in}{0.017319in}}{\pgfqpoint{0.023570in}{0.023570in}}%
\pgfpathcurveto{\pgfqpoint{0.017319in}{0.029821in}}{\pgfqpoint{0.008840in}{0.033333in}}{\pgfqpoint{0.000000in}{0.033333in}}%
\pgfpathcurveto{\pgfqpoint{-0.008840in}{0.033333in}}{\pgfqpoint{-0.017319in}{0.029821in}}{\pgfqpoint{-0.023570in}{0.023570in}}%
\pgfpathcurveto{\pgfqpoint{-0.029821in}{0.017319in}}{\pgfqpoint{-0.033333in}{0.008840in}}{\pgfqpoint{-0.033333in}{0.000000in}}%
\pgfpathcurveto{\pgfqpoint{-0.033333in}{-0.008840in}}{\pgfqpoint{-0.029821in}{-0.017319in}}{\pgfqpoint{-0.023570in}{-0.023570in}}%
\pgfpathcurveto{\pgfqpoint{-0.017319in}{-0.029821in}}{\pgfqpoint{-0.008840in}{-0.033333in}}{\pgfqpoint{0.000000in}{-0.033333in}}%
\pgfpathlineto{\pgfqpoint{0.000000in}{-0.033333in}}%
\pgfpathclose%
\pgfusepath{stroke,fill}%
}%
\begin{pgfscope}%
\pgfsys@transformshift{2.545699in}{3.168049in}%
\pgfsys@useobject{currentmarker}{}%
\end{pgfscope}%
\end{pgfscope}%
\begin{pgfscope}%
\pgfpathrectangle{\pgfqpoint{0.765000in}{0.660000in}}{\pgfqpoint{4.620000in}{4.620000in}}%
\pgfusepath{clip}%
\pgfsetroundcap%
\pgfsetroundjoin%
\pgfsetlinewidth{1.204500pt}%
\definecolor{currentstroke}{rgb}{1.000000,0.576471,0.309804}%
\pgfsetstrokecolor{currentstroke}%
\pgfsetdash{}{0pt}%
\pgfpathmoveto{\pgfqpoint{3.473559in}{2.566833in}}%
\pgfusepath{stroke}%
\end{pgfscope}%
\begin{pgfscope}%
\pgfpathrectangle{\pgfqpoint{0.765000in}{0.660000in}}{\pgfqpoint{4.620000in}{4.620000in}}%
\pgfusepath{clip}%
\pgfsetbuttcap%
\pgfsetroundjoin%
\definecolor{currentfill}{rgb}{1.000000,0.576471,0.309804}%
\pgfsetfillcolor{currentfill}%
\pgfsetlinewidth{1.003750pt}%
\definecolor{currentstroke}{rgb}{1.000000,0.576471,0.309804}%
\pgfsetstrokecolor{currentstroke}%
\pgfsetdash{}{0pt}%
\pgfsys@defobject{currentmarker}{\pgfqpoint{-0.033333in}{-0.033333in}}{\pgfqpoint{0.033333in}{0.033333in}}{%
\pgfpathmoveto{\pgfqpoint{0.000000in}{-0.033333in}}%
\pgfpathcurveto{\pgfqpoint{0.008840in}{-0.033333in}}{\pgfqpoint{0.017319in}{-0.029821in}}{\pgfqpoint{0.023570in}{-0.023570in}}%
\pgfpathcurveto{\pgfqpoint{0.029821in}{-0.017319in}}{\pgfqpoint{0.033333in}{-0.008840in}}{\pgfqpoint{0.033333in}{0.000000in}}%
\pgfpathcurveto{\pgfqpoint{0.033333in}{0.008840in}}{\pgfqpoint{0.029821in}{0.017319in}}{\pgfqpoint{0.023570in}{0.023570in}}%
\pgfpathcurveto{\pgfqpoint{0.017319in}{0.029821in}}{\pgfqpoint{0.008840in}{0.033333in}}{\pgfqpoint{0.000000in}{0.033333in}}%
\pgfpathcurveto{\pgfqpoint{-0.008840in}{0.033333in}}{\pgfqpoint{-0.017319in}{0.029821in}}{\pgfqpoint{-0.023570in}{0.023570in}}%
\pgfpathcurveto{\pgfqpoint{-0.029821in}{0.017319in}}{\pgfqpoint{-0.033333in}{0.008840in}}{\pgfqpoint{-0.033333in}{0.000000in}}%
\pgfpathcurveto{\pgfqpoint{-0.033333in}{-0.008840in}}{\pgfqpoint{-0.029821in}{-0.017319in}}{\pgfqpoint{-0.023570in}{-0.023570in}}%
\pgfpathcurveto{\pgfqpoint{-0.017319in}{-0.029821in}}{\pgfqpoint{-0.008840in}{-0.033333in}}{\pgfqpoint{0.000000in}{-0.033333in}}%
\pgfpathlineto{\pgfqpoint{0.000000in}{-0.033333in}}%
\pgfpathclose%
\pgfusepath{stroke,fill}%
}%
\begin{pgfscope}%
\pgfsys@transformshift{3.473559in}{2.566833in}%
\pgfsys@useobject{currentmarker}{}%
\end{pgfscope}%
\end{pgfscope}%
\begin{pgfscope}%
\pgfpathrectangle{\pgfqpoint{0.765000in}{0.660000in}}{\pgfqpoint{4.620000in}{4.620000in}}%
\pgfusepath{clip}%
\pgfsetroundcap%
\pgfsetroundjoin%
\pgfsetlinewidth{1.204500pt}%
\definecolor{currentstroke}{rgb}{1.000000,0.576471,0.309804}%
\pgfsetstrokecolor{currentstroke}%
\pgfsetdash{}{0pt}%
\pgfpathmoveto{\pgfqpoint{2.830154in}{3.438612in}}%
\pgfusepath{stroke}%
\end{pgfscope}%
\begin{pgfscope}%
\pgfpathrectangle{\pgfqpoint{0.765000in}{0.660000in}}{\pgfqpoint{4.620000in}{4.620000in}}%
\pgfusepath{clip}%
\pgfsetbuttcap%
\pgfsetroundjoin%
\definecolor{currentfill}{rgb}{1.000000,0.576471,0.309804}%
\pgfsetfillcolor{currentfill}%
\pgfsetlinewidth{1.003750pt}%
\definecolor{currentstroke}{rgb}{1.000000,0.576471,0.309804}%
\pgfsetstrokecolor{currentstroke}%
\pgfsetdash{}{0pt}%
\pgfsys@defobject{currentmarker}{\pgfqpoint{-0.033333in}{-0.033333in}}{\pgfqpoint{0.033333in}{0.033333in}}{%
\pgfpathmoveto{\pgfqpoint{0.000000in}{-0.033333in}}%
\pgfpathcurveto{\pgfqpoint{0.008840in}{-0.033333in}}{\pgfqpoint{0.017319in}{-0.029821in}}{\pgfqpoint{0.023570in}{-0.023570in}}%
\pgfpathcurveto{\pgfqpoint{0.029821in}{-0.017319in}}{\pgfqpoint{0.033333in}{-0.008840in}}{\pgfqpoint{0.033333in}{0.000000in}}%
\pgfpathcurveto{\pgfqpoint{0.033333in}{0.008840in}}{\pgfqpoint{0.029821in}{0.017319in}}{\pgfqpoint{0.023570in}{0.023570in}}%
\pgfpathcurveto{\pgfqpoint{0.017319in}{0.029821in}}{\pgfqpoint{0.008840in}{0.033333in}}{\pgfqpoint{0.000000in}{0.033333in}}%
\pgfpathcurveto{\pgfqpoint{-0.008840in}{0.033333in}}{\pgfqpoint{-0.017319in}{0.029821in}}{\pgfqpoint{-0.023570in}{0.023570in}}%
\pgfpathcurveto{\pgfqpoint{-0.029821in}{0.017319in}}{\pgfqpoint{-0.033333in}{0.008840in}}{\pgfqpoint{-0.033333in}{0.000000in}}%
\pgfpathcurveto{\pgfqpoint{-0.033333in}{-0.008840in}}{\pgfqpoint{-0.029821in}{-0.017319in}}{\pgfqpoint{-0.023570in}{-0.023570in}}%
\pgfpathcurveto{\pgfqpoint{-0.017319in}{-0.029821in}}{\pgfqpoint{-0.008840in}{-0.033333in}}{\pgfqpoint{0.000000in}{-0.033333in}}%
\pgfpathlineto{\pgfqpoint{0.000000in}{-0.033333in}}%
\pgfpathclose%
\pgfusepath{stroke,fill}%
}%
\begin{pgfscope}%
\pgfsys@transformshift{2.830154in}{3.438612in}%
\pgfsys@useobject{currentmarker}{}%
\end{pgfscope}%
\end{pgfscope}%
\begin{pgfscope}%
\pgfpathrectangle{\pgfqpoint{0.765000in}{0.660000in}}{\pgfqpoint{4.620000in}{4.620000in}}%
\pgfusepath{clip}%
\pgfsetroundcap%
\pgfsetroundjoin%
\pgfsetlinewidth{1.204500pt}%
\definecolor{currentstroke}{rgb}{1.000000,0.576471,0.309804}%
\pgfsetstrokecolor{currentstroke}%
\pgfsetdash{}{0pt}%
\pgfpathmoveto{\pgfqpoint{3.804988in}{2.449111in}}%
\pgfusepath{stroke}%
\end{pgfscope}%
\begin{pgfscope}%
\pgfpathrectangle{\pgfqpoint{0.765000in}{0.660000in}}{\pgfqpoint{4.620000in}{4.620000in}}%
\pgfusepath{clip}%
\pgfsetbuttcap%
\pgfsetroundjoin%
\definecolor{currentfill}{rgb}{1.000000,0.576471,0.309804}%
\pgfsetfillcolor{currentfill}%
\pgfsetlinewidth{1.003750pt}%
\definecolor{currentstroke}{rgb}{1.000000,0.576471,0.309804}%
\pgfsetstrokecolor{currentstroke}%
\pgfsetdash{}{0pt}%
\pgfsys@defobject{currentmarker}{\pgfqpoint{-0.033333in}{-0.033333in}}{\pgfqpoint{0.033333in}{0.033333in}}{%
\pgfpathmoveto{\pgfqpoint{0.000000in}{-0.033333in}}%
\pgfpathcurveto{\pgfqpoint{0.008840in}{-0.033333in}}{\pgfqpoint{0.017319in}{-0.029821in}}{\pgfqpoint{0.023570in}{-0.023570in}}%
\pgfpathcurveto{\pgfqpoint{0.029821in}{-0.017319in}}{\pgfqpoint{0.033333in}{-0.008840in}}{\pgfqpoint{0.033333in}{0.000000in}}%
\pgfpathcurveto{\pgfqpoint{0.033333in}{0.008840in}}{\pgfqpoint{0.029821in}{0.017319in}}{\pgfqpoint{0.023570in}{0.023570in}}%
\pgfpathcurveto{\pgfqpoint{0.017319in}{0.029821in}}{\pgfqpoint{0.008840in}{0.033333in}}{\pgfqpoint{0.000000in}{0.033333in}}%
\pgfpathcurveto{\pgfqpoint{-0.008840in}{0.033333in}}{\pgfqpoint{-0.017319in}{0.029821in}}{\pgfqpoint{-0.023570in}{0.023570in}}%
\pgfpathcurveto{\pgfqpoint{-0.029821in}{0.017319in}}{\pgfqpoint{-0.033333in}{0.008840in}}{\pgfqpoint{-0.033333in}{0.000000in}}%
\pgfpathcurveto{\pgfqpoint{-0.033333in}{-0.008840in}}{\pgfqpoint{-0.029821in}{-0.017319in}}{\pgfqpoint{-0.023570in}{-0.023570in}}%
\pgfpathcurveto{\pgfqpoint{-0.017319in}{-0.029821in}}{\pgfqpoint{-0.008840in}{-0.033333in}}{\pgfqpoint{0.000000in}{-0.033333in}}%
\pgfpathlineto{\pgfqpoint{0.000000in}{-0.033333in}}%
\pgfpathclose%
\pgfusepath{stroke,fill}%
}%
\begin{pgfscope}%
\pgfsys@transformshift{3.804988in}{2.449111in}%
\pgfsys@useobject{currentmarker}{}%
\end{pgfscope}%
\end{pgfscope}%
\begin{pgfscope}%
\pgfpathrectangle{\pgfqpoint{0.765000in}{0.660000in}}{\pgfqpoint{4.620000in}{4.620000in}}%
\pgfusepath{clip}%
\pgfsetroundcap%
\pgfsetroundjoin%
\pgfsetlinewidth{1.204500pt}%
\definecolor{currentstroke}{rgb}{1.000000,0.576471,0.309804}%
\pgfsetstrokecolor{currentstroke}%
\pgfsetdash{}{0pt}%
\pgfpathmoveto{\pgfqpoint{4.015348in}{3.691818in}}%
\pgfusepath{stroke}%
\end{pgfscope}%
\begin{pgfscope}%
\pgfpathrectangle{\pgfqpoint{0.765000in}{0.660000in}}{\pgfqpoint{4.620000in}{4.620000in}}%
\pgfusepath{clip}%
\pgfsetbuttcap%
\pgfsetroundjoin%
\definecolor{currentfill}{rgb}{1.000000,0.576471,0.309804}%
\pgfsetfillcolor{currentfill}%
\pgfsetlinewidth{1.003750pt}%
\definecolor{currentstroke}{rgb}{1.000000,0.576471,0.309804}%
\pgfsetstrokecolor{currentstroke}%
\pgfsetdash{}{0pt}%
\pgfsys@defobject{currentmarker}{\pgfqpoint{-0.033333in}{-0.033333in}}{\pgfqpoint{0.033333in}{0.033333in}}{%
\pgfpathmoveto{\pgfqpoint{0.000000in}{-0.033333in}}%
\pgfpathcurveto{\pgfqpoint{0.008840in}{-0.033333in}}{\pgfqpoint{0.017319in}{-0.029821in}}{\pgfqpoint{0.023570in}{-0.023570in}}%
\pgfpathcurveto{\pgfqpoint{0.029821in}{-0.017319in}}{\pgfqpoint{0.033333in}{-0.008840in}}{\pgfqpoint{0.033333in}{0.000000in}}%
\pgfpathcurveto{\pgfqpoint{0.033333in}{0.008840in}}{\pgfqpoint{0.029821in}{0.017319in}}{\pgfqpoint{0.023570in}{0.023570in}}%
\pgfpathcurveto{\pgfqpoint{0.017319in}{0.029821in}}{\pgfqpoint{0.008840in}{0.033333in}}{\pgfqpoint{0.000000in}{0.033333in}}%
\pgfpathcurveto{\pgfqpoint{-0.008840in}{0.033333in}}{\pgfqpoint{-0.017319in}{0.029821in}}{\pgfqpoint{-0.023570in}{0.023570in}}%
\pgfpathcurveto{\pgfqpoint{-0.029821in}{0.017319in}}{\pgfqpoint{-0.033333in}{0.008840in}}{\pgfqpoint{-0.033333in}{0.000000in}}%
\pgfpathcurveto{\pgfqpoint{-0.033333in}{-0.008840in}}{\pgfqpoint{-0.029821in}{-0.017319in}}{\pgfqpoint{-0.023570in}{-0.023570in}}%
\pgfpathcurveto{\pgfqpoint{-0.017319in}{-0.029821in}}{\pgfqpoint{-0.008840in}{-0.033333in}}{\pgfqpoint{0.000000in}{-0.033333in}}%
\pgfpathlineto{\pgfqpoint{0.000000in}{-0.033333in}}%
\pgfpathclose%
\pgfusepath{stroke,fill}%
}%
\begin{pgfscope}%
\pgfsys@transformshift{4.015348in}{3.691818in}%
\pgfsys@useobject{currentmarker}{}%
\end{pgfscope}%
\end{pgfscope}%
\begin{pgfscope}%
\pgfpathrectangle{\pgfqpoint{0.765000in}{0.660000in}}{\pgfqpoint{4.620000in}{4.620000in}}%
\pgfusepath{clip}%
\pgfsetroundcap%
\pgfsetroundjoin%
\pgfsetlinewidth{1.204500pt}%
\definecolor{currentstroke}{rgb}{1.000000,0.576471,0.309804}%
\pgfsetstrokecolor{currentstroke}%
\pgfsetdash{}{0pt}%
\pgfpathmoveto{\pgfqpoint{2.936955in}{3.580375in}}%
\pgfusepath{stroke}%
\end{pgfscope}%
\begin{pgfscope}%
\pgfpathrectangle{\pgfqpoint{0.765000in}{0.660000in}}{\pgfqpoint{4.620000in}{4.620000in}}%
\pgfusepath{clip}%
\pgfsetbuttcap%
\pgfsetroundjoin%
\definecolor{currentfill}{rgb}{1.000000,0.576471,0.309804}%
\pgfsetfillcolor{currentfill}%
\pgfsetlinewidth{1.003750pt}%
\definecolor{currentstroke}{rgb}{1.000000,0.576471,0.309804}%
\pgfsetstrokecolor{currentstroke}%
\pgfsetdash{}{0pt}%
\pgfsys@defobject{currentmarker}{\pgfqpoint{-0.033333in}{-0.033333in}}{\pgfqpoint{0.033333in}{0.033333in}}{%
\pgfpathmoveto{\pgfqpoint{0.000000in}{-0.033333in}}%
\pgfpathcurveto{\pgfqpoint{0.008840in}{-0.033333in}}{\pgfqpoint{0.017319in}{-0.029821in}}{\pgfqpoint{0.023570in}{-0.023570in}}%
\pgfpathcurveto{\pgfqpoint{0.029821in}{-0.017319in}}{\pgfqpoint{0.033333in}{-0.008840in}}{\pgfqpoint{0.033333in}{0.000000in}}%
\pgfpathcurveto{\pgfqpoint{0.033333in}{0.008840in}}{\pgfqpoint{0.029821in}{0.017319in}}{\pgfqpoint{0.023570in}{0.023570in}}%
\pgfpathcurveto{\pgfqpoint{0.017319in}{0.029821in}}{\pgfqpoint{0.008840in}{0.033333in}}{\pgfqpoint{0.000000in}{0.033333in}}%
\pgfpathcurveto{\pgfqpoint{-0.008840in}{0.033333in}}{\pgfqpoint{-0.017319in}{0.029821in}}{\pgfqpoint{-0.023570in}{0.023570in}}%
\pgfpathcurveto{\pgfqpoint{-0.029821in}{0.017319in}}{\pgfqpoint{-0.033333in}{0.008840in}}{\pgfqpoint{-0.033333in}{0.000000in}}%
\pgfpathcurveto{\pgfqpoint{-0.033333in}{-0.008840in}}{\pgfqpoint{-0.029821in}{-0.017319in}}{\pgfqpoint{-0.023570in}{-0.023570in}}%
\pgfpathcurveto{\pgfqpoint{-0.017319in}{-0.029821in}}{\pgfqpoint{-0.008840in}{-0.033333in}}{\pgfqpoint{0.000000in}{-0.033333in}}%
\pgfpathlineto{\pgfqpoint{0.000000in}{-0.033333in}}%
\pgfpathclose%
\pgfusepath{stroke,fill}%
}%
\begin{pgfscope}%
\pgfsys@transformshift{2.936955in}{3.580375in}%
\pgfsys@useobject{currentmarker}{}%
\end{pgfscope}%
\end{pgfscope}%
\begin{pgfscope}%
\pgfpathrectangle{\pgfqpoint{0.765000in}{0.660000in}}{\pgfqpoint{4.620000in}{4.620000in}}%
\pgfusepath{clip}%
\pgfsetroundcap%
\pgfsetroundjoin%
\pgfsetlinewidth{1.204500pt}%
\definecolor{currentstroke}{rgb}{0.176471,0.192157,0.258824}%
\pgfsetstrokecolor{currentstroke}%
\pgfsetdash{}{0pt}%
\pgfpathmoveto{\pgfqpoint{3.051075in}{2.864877in}}%
\pgfusepath{stroke}%
\end{pgfscope}%
\begin{pgfscope}%
\pgfpathrectangle{\pgfqpoint{0.765000in}{0.660000in}}{\pgfqpoint{4.620000in}{4.620000in}}%
\pgfusepath{clip}%
\pgfsetbuttcap%
\pgfsetmiterjoin%
\definecolor{currentfill}{rgb}{0.176471,0.192157,0.258824}%
\pgfsetfillcolor{currentfill}%
\pgfsetlinewidth{1.003750pt}%
\definecolor{currentstroke}{rgb}{0.176471,0.192157,0.258824}%
\pgfsetstrokecolor{currentstroke}%
\pgfsetdash{}{0pt}%
\pgfsys@defobject{currentmarker}{\pgfqpoint{-0.033333in}{-0.033333in}}{\pgfqpoint{0.033333in}{0.033333in}}{%
\pgfpathmoveto{\pgfqpoint{-0.000000in}{-0.033333in}}%
\pgfpathlineto{\pgfqpoint{0.033333in}{0.033333in}}%
\pgfpathlineto{\pgfqpoint{-0.033333in}{0.033333in}}%
\pgfpathlineto{\pgfqpoint{-0.000000in}{-0.033333in}}%
\pgfpathclose%
\pgfusepath{stroke,fill}%
}%
\begin{pgfscope}%
\pgfsys@transformshift{3.051075in}{2.864877in}%
\pgfsys@useobject{currentmarker}{}%
\end{pgfscope}%
\end{pgfscope}%
\begin{pgfscope}%
\pgfpathrectangle{\pgfqpoint{0.765000in}{0.660000in}}{\pgfqpoint{4.620000in}{4.620000in}}%
\pgfusepath{clip}%
\pgfsetroundcap%
\pgfsetroundjoin%
\pgfsetlinewidth{1.204500pt}%
\definecolor{currentstroke}{rgb}{0.176471,0.192157,0.258824}%
\pgfsetstrokecolor{currentstroke}%
\pgfsetdash{}{0pt}%
\pgfpathmoveto{\pgfqpoint{3.152537in}{2.887567in}}%
\pgfusepath{stroke}%
\end{pgfscope}%
\begin{pgfscope}%
\pgfpathrectangle{\pgfqpoint{0.765000in}{0.660000in}}{\pgfqpoint{4.620000in}{4.620000in}}%
\pgfusepath{clip}%
\pgfsetbuttcap%
\pgfsetmiterjoin%
\definecolor{currentfill}{rgb}{0.176471,0.192157,0.258824}%
\pgfsetfillcolor{currentfill}%
\pgfsetlinewidth{1.003750pt}%
\definecolor{currentstroke}{rgb}{0.176471,0.192157,0.258824}%
\pgfsetstrokecolor{currentstroke}%
\pgfsetdash{}{0pt}%
\pgfsys@defobject{currentmarker}{\pgfqpoint{-0.033333in}{-0.033333in}}{\pgfqpoint{0.033333in}{0.033333in}}{%
\pgfpathmoveto{\pgfqpoint{-0.000000in}{-0.033333in}}%
\pgfpathlineto{\pgfqpoint{0.033333in}{0.033333in}}%
\pgfpathlineto{\pgfqpoint{-0.033333in}{0.033333in}}%
\pgfpathlineto{\pgfqpoint{-0.000000in}{-0.033333in}}%
\pgfpathclose%
\pgfusepath{stroke,fill}%
}%
\begin{pgfscope}%
\pgfsys@transformshift{3.152537in}{2.887567in}%
\pgfsys@useobject{currentmarker}{}%
\end{pgfscope}%
\end{pgfscope}%
\begin{pgfscope}%
\pgfpathrectangle{\pgfqpoint{0.765000in}{0.660000in}}{\pgfqpoint{4.620000in}{4.620000in}}%
\pgfusepath{clip}%
\pgfsetroundcap%
\pgfsetroundjoin%
\pgfsetlinewidth{1.204500pt}%
\definecolor{currentstroke}{rgb}{0.176471,0.192157,0.258824}%
\pgfsetstrokecolor{currentstroke}%
\pgfsetdash{}{0pt}%
\pgfpathmoveto{\pgfqpoint{3.113612in}{3.034556in}}%
\pgfusepath{stroke}%
\end{pgfscope}%
\begin{pgfscope}%
\pgfpathrectangle{\pgfqpoint{0.765000in}{0.660000in}}{\pgfqpoint{4.620000in}{4.620000in}}%
\pgfusepath{clip}%
\pgfsetbuttcap%
\pgfsetmiterjoin%
\definecolor{currentfill}{rgb}{0.176471,0.192157,0.258824}%
\pgfsetfillcolor{currentfill}%
\pgfsetlinewidth{1.003750pt}%
\definecolor{currentstroke}{rgb}{0.176471,0.192157,0.258824}%
\pgfsetstrokecolor{currentstroke}%
\pgfsetdash{}{0pt}%
\pgfsys@defobject{currentmarker}{\pgfqpoint{-0.033333in}{-0.033333in}}{\pgfqpoint{0.033333in}{0.033333in}}{%
\pgfpathmoveto{\pgfqpoint{-0.000000in}{-0.033333in}}%
\pgfpathlineto{\pgfqpoint{0.033333in}{0.033333in}}%
\pgfpathlineto{\pgfqpoint{-0.033333in}{0.033333in}}%
\pgfpathlineto{\pgfqpoint{-0.000000in}{-0.033333in}}%
\pgfpathclose%
\pgfusepath{stroke,fill}%
}%
\begin{pgfscope}%
\pgfsys@transformshift{3.113612in}{3.034556in}%
\pgfsys@useobject{currentmarker}{}%
\end{pgfscope}%
\end{pgfscope}%
\begin{pgfscope}%
\pgfpathrectangle{\pgfqpoint{0.765000in}{0.660000in}}{\pgfqpoint{4.620000in}{4.620000in}}%
\pgfusepath{clip}%
\pgfsetroundcap%
\pgfsetroundjoin%
\pgfsetlinewidth{1.204500pt}%
\definecolor{currentstroke}{rgb}{0.176471,0.192157,0.258824}%
\pgfsetstrokecolor{currentstroke}%
\pgfsetdash{}{0pt}%
\pgfpathmoveto{\pgfqpoint{3.186375in}{2.986549in}}%
\pgfusepath{stroke}%
\end{pgfscope}%
\begin{pgfscope}%
\pgfpathrectangle{\pgfqpoint{0.765000in}{0.660000in}}{\pgfqpoint{4.620000in}{4.620000in}}%
\pgfusepath{clip}%
\pgfsetbuttcap%
\pgfsetmiterjoin%
\definecolor{currentfill}{rgb}{0.176471,0.192157,0.258824}%
\pgfsetfillcolor{currentfill}%
\pgfsetlinewidth{1.003750pt}%
\definecolor{currentstroke}{rgb}{0.176471,0.192157,0.258824}%
\pgfsetstrokecolor{currentstroke}%
\pgfsetdash{}{0pt}%
\pgfsys@defobject{currentmarker}{\pgfqpoint{-0.033333in}{-0.033333in}}{\pgfqpoint{0.033333in}{0.033333in}}{%
\pgfpathmoveto{\pgfqpoint{-0.000000in}{-0.033333in}}%
\pgfpathlineto{\pgfqpoint{0.033333in}{0.033333in}}%
\pgfpathlineto{\pgfqpoint{-0.033333in}{0.033333in}}%
\pgfpathlineto{\pgfqpoint{-0.000000in}{-0.033333in}}%
\pgfpathclose%
\pgfusepath{stroke,fill}%
}%
\begin{pgfscope}%
\pgfsys@transformshift{3.186375in}{2.986549in}%
\pgfsys@useobject{currentmarker}{}%
\end{pgfscope}%
\end{pgfscope}%
\begin{pgfscope}%
\pgfpathrectangle{\pgfqpoint{0.765000in}{0.660000in}}{\pgfqpoint{4.620000in}{4.620000in}}%
\pgfusepath{clip}%
\pgfsetroundcap%
\pgfsetroundjoin%
\pgfsetlinewidth{1.204500pt}%
\definecolor{currentstroke}{rgb}{0.176471,0.192157,0.258824}%
\pgfsetstrokecolor{currentstroke}%
\pgfsetdash{}{0pt}%
\pgfpathmoveto{\pgfqpoint{3.160448in}{2.858907in}}%
\pgfusepath{stroke}%
\end{pgfscope}%
\begin{pgfscope}%
\pgfpathrectangle{\pgfqpoint{0.765000in}{0.660000in}}{\pgfqpoint{4.620000in}{4.620000in}}%
\pgfusepath{clip}%
\pgfsetbuttcap%
\pgfsetmiterjoin%
\definecolor{currentfill}{rgb}{0.176471,0.192157,0.258824}%
\pgfsetfillcolor{currentfill}%
\pgfsetlinewidth{1.003750pt}%
\definecolor{currentstroke}{rgb}{0.176471,0.192157,0.258824}%
\pgfsetstrokecolor{currentstroke}%
\pgfsetdash{}{0pt}%
\pgfsys@defobject{currentmarker}{\pgfqpoint{-0.033333in}{-0.033333in}}{\pgfqpoint{0.033333in}{0.033333in}}{%
\pgfpathmoveto{\pgfqpoint{-0.000000in}{-0.033333in}}%
\pgfpathlineto{\pgfqpoint{0.033333in}{0.033333in}}%
\pgfpathlineto{\pgfqpoint{-0.033333in}{0.033333in}}%
\pgfpathlineto{\pgfqpoint{-0.000000in}{-0.033333in}}%
\pgfpathclose%
\pgfusepath{stroke,fill}%
}%
\begin{pgfscope}%
\pgfsys@transformshift{3.160448in}{2.858907in}%
\pgfsys@useobject{currentmarker}{}%
\end{pgfscope}%
\end{pgfscope}%
\begin{pgfscope}%
\pgfpathrectangle{\pgfqpoint{0.765000in}{0.660000in}}{\pgfqpoint{4.620000in}{4.620000in}}%
\pgfusepath{clip}%
\pgfsetroundcap%
\pgfsetroundjoin%
\pgfsetlinewidth{1.204500pt}%
\definecolor{currentstroke}{rgb}{0.176471,0.192157,0.258824}%
\pgfsetstrokecolor{currentstroke}%
\pgfsetdash{}{0pt}%
\pgfpathmoveto{\pgfqpoint{3.281169in}{2.954762in}}%
\pgfusepath{stroke}%
\end{pgfscope}%
\begin{pgfscope}%
\pgfpathrectangle{\pgfqpoint{0.765000in}{0.660000in}}{\pgfqpoint{4.620000in}{4.620000in}}%
\pgfusepath{clip}%
\pgfsetbuttcap%
\pgfsetmiterjoin%
\definecolor{currentfill}{rgb}{0.176471,0.192157,0.258824}%
\pgfsetfillcolor{currentfill}%
\pgfsetlinewidth{1.003750pt}%
\definecolor{currentstroke}{rgb}{0.176471,0.192157,0.258824}%
\pgfsetstrokecolor{currentstroke}%
\pgfsetdash{}{0pt}%
\pgfsys@defobject{currentmarker}{\pgfqpoint{-0.033333in}{-0.033333in}}{\pgfqpoint{0.033333in}{0.033333in}}{%
\pgfpathmoveto{\pgfqpoint{-0.000000in}{-0.033333in}}%
\pgfpathlineto{\pgfqpoint{0.033333in}{0.033333in}}%
\pgfpathlineto{\pgfqpoint{-0.033333in}{0.033333in}}%
\pgfpathlineto{\pgfqpoint{-0.000000in}{-0.033333in}}%
\pgfpathclose%
\pgfusepath{stroke,fill}%
}%
\begin{pgfscope}%
\pgfsys@transformshift{3.281169in}{2.954762in}%
\pgfsys@useobject{currentmarker}{}%
\end{pgfscope}%
\end{pgfscope}%
\begin{pgfscope}%
\pgfpathrectangle{\pgfqpoint{0.765000in}{0.660000in}}{\pgfqpoint{4.620000in}{4.620000in}}%
\pgfusepath{clip}%
\pgfsetroundcap%
\pgfsetroundjoin%
\pgfsetlinewidth{1.204500pt}%
\definecolor{currentstroke}{rgb}{0.176471,0.192157,0.258824}%
\pgfsetstrokecolor{currentstroke}%
\pgfsetdash{}{0pt}%
\pgfpathmoveto{\pgfqpoint{3.114888in}{2.980688in}}%
\pgfusepath{stroke}%
\end{pgfscope}%
\begin{pgfscope}%
\pgfpathrectangle{\pgfqpoint{0.765000in}{0.660000in}}{\pgfqpoint{4.620000in}{4.620000in}}%
\pgfusepath{clip}%
\pgfsetbuttcap%
\pgfsetmiterjoin%
\definecolor{currentfill}{rgb}{0.176471,0.192157,0.258824}%
\pgfsetfillcolor{currentfill}%
\pgfsetlinewidth{1.003750pt}%
\definecolor{currentstroke}{rgb}{0.176471,0.192157,0.258824}%
\pgfsetstrokecolor{currentstroke}%
\pgfsetdash{}{0pt}%
\pgfsys@defobject{currentmarker}{\pgfqpoint{-0.033333in}{-0.033333in}}{\pgfqpoint{0.033333in}{0.033333in}}{%
\pgfpathmoveto{\pgfqpoint{-0.000000in}{-0.033333in}}%
\pgfpathlineto{\pgfqpoint{0.033333in}{0.033333in}}%
\pgfpathlineto{\pgfqpoint{-0.033333in}{0.033333in}}%
\pgfpathlineto{\pgfqpoint{-0.000000in}{-0.033333in}}%
\pgfpathclose%
\pgfusepath{stroke,fill}%
}%
\begin{pgfscope}%
\pgfsys@transformshift{3.114888in}{2.980688in}%
\pgfsys@useobject{currentmarker}{}%
\end{pgfscope}%
\end{pgfscope}%
\begin{pgfscope}%
\pgfpathrectangle{\pgfqpoint{0.765000in}{0.660000in}}{\pgfqpoint{4.620000in}{4.620000in}}%
\pgfusepath{clip}%
\pgfsetroundcap%
\pgfsetroundjoin%
\pgfsetlinewidth{1.204500pt}%
\definecolor{currentstroke}{rgb}{0.176471,0.192157,0.258824}%
\pgfsetstrokecolor{currentstroke}%
\pgfsetdash{}{0pt}%
\pgfpathmoveto{\pgfqpoint{3.045111in}{2.990322in}}%
\pgfusepath{stroke}%
\end{pgfscope}%
\begin{pgfscope}%
\pgfpathrectangle{\pgfqpoint{0.765000in}{0.660000in}}{\pgfqpoint{4.620000in}{4.620000in}}%
\pgfusepath{clip}%
\pgfsetbuttcap%
\pgfsetmiterjoin%
\definecolor{currentfill}{rgb}{0.176471,0.192157,0.258824}%
\pgfsetfillcolor{currentfill}%
\pgfsetlinewidth{1.003750pt}%
\definecolor{currentstroke}{rgb}{0.176471,0.192157,0.258824}%
\pgfsetstrokecolor{currentstroke}%
\pgfsetdash{}{0pt}%
\pgfsys@defobject{currentmarker}{\pgfqpoint{-0.033333in}{-0.033333in}}{\pgfqpoint{0.033333in}{0.033333in}}{%
\pgfpathmoveto{\pgfqpoint{-0.000000in}{-0.033333in}}%
\pgfpathlineto{\pgfqpoint{0.033333in}{0.033333in}}%
\pgfpathlineto{\pgfqpoint{-0.033333in}{0.033333in}}%
\pgfpathlineto{\pgfqpoint{-0.000000in}{-0.033333in}}%
\pgfpathclose%
\pgfusepath{stroke,fill}%
}%
\begin{pgfscope}%
\pgfsys@transformshift{3.045111in}{2.990322in}%
\pgfsys@useobject{currentmarker}{}%
\end{pgfscope}%
\end{pgfscope}%
\begin{pgfscope}%
\pgfpathrectangle{\pgfqpoint{0.765000in}{0.660000in}}{\pgfqpoint{4.620000in}{4.620000in}}%
\pgfusepath{clip}%
\pgfsetroundcap%
\pgfsetroundjoin%
\pgfsetlinewidth{1.204500pt}%
\definecolor{currentstroke}{rgb}{0.176471,0.192157,0.258824}%
\pgfsetstrokecolor{currentstroke}%
\pgfsetdash{}{0pt}%
\pgfpathmoveto{\pgfqpoint{3.081429in}{2.862441in}}%
\pgfusepath{stroke}%
\end{pgfscope}%
\begin{pgfscope}%
\pgfpathrectangle{\pgfqpoint{0.765000in}{0.660000in}}{\pgfqpoint{4.620000in}{4.620000in}}%
\pgfusepath{clip}%
\pgfsetbuttcap%
\pgfsetmiterjoin%
\definecolor{currentfill}{rgb}{0.176471,0.192157,0.258824}%
\pgfsetfillcolor{currentfill}%
\pgfsetlinewidth{1.003750pt}%
\definecolor{currentstroke}{rgb}{0.176471,0.192157,0.258824}%
\pgfsetstrokecolor{currentstroke}%
\pgfsetdash{}{0pt}%
\pgfsys@defobject{currentmarker}{\pgfqpoint{-0.033333in}{-0.033333in}}{\pgfqpoint{0.033333in}{0.033333in}}{%
\pgfpathmoveto{\pgfqpoint{-0.000000in}{-0.033333in}}%
\pgfpathlineto{\pgfqpoint{0.033333in}{0.033333in}}%
\pgfpathlineto{\pgfqpoint{-0.033333in}{0.033333in}}%
\pgfpathlineto{\pgfqpoint{-0.000000in}{-0.033333in}}%
\pgfpathclose%
\pgfusepath{stroke,fill}%
}%
\begin{pgfscope}%
\pgfsys@transformshift{3.081429in}{2.862441in}%
\pgfsys@useobject{currentmarker}{}%
\end{pgfscope}%
\end{pgfscope}%
\begin{pgfscope}%
\pgfpathrectangle{\pgfqpoint{0.765000in}{0.660000in}}{\pgfqpoint{4.620000in}{4.620000in}}%
\pgfusepath{clip}%
\pgfsetroundcap%
\pgfsetroundjoin%
\pgfsetlinewidth{1.204500pt}%
\definecolor{currentstroke}{rgb}{0.176471,0.192157,0.258824}%
\pgfsetstrokecolor{currentstroke}%
\pgfsetdash{}{0pt}%
\pgfpathmoveto{\pgfqpoint{3.067617in}{2.953639in}}%
\pgfusepath{stroke}%
\end{pgfscope}%
\begin{pgfscope}%
\pgfpathrectangle{\pgfqpoint{0.765000in}{0.660000in}}{\pgfqpoint{4.620000in}{4.620000in}}%
\pgfusepath{clip}%
\pgfsetbuttcap%
\pgfsetmiterjoin%
\definecolor{currentfill}{rgb}{0.176471,0.192157,0.258824}%
\pgfsetfillcolor{currentfill}%
\pgfsetlinewidth{1.003750pt}%
\definecolor{currentstroke}{rgb}{0.176471,0.192157,0.258824}%
\pgfsetstrokecolor{currentstroke}%
\pgfsetdash{}{0pt}%
\pgfsys@defobject{currentmarker}{\pgfqpoint{-0.033333in}{-0.033333in}}{\pgfqpoint{0.033333in}{0.033333in}}{%
\pgfpathmoveto{\pgfqpoint{-0.000000in}{-0.033333in}}%
\pgfpathlineto{\pgfqpoint{0.033333in}{0.033333in}}%
\pgfpathlineto{\pgfqpoint{-0.033333in}{0.033333in}}%
\pgfpathlineto{\pgfqpoint{-0.000000in}{-0.033333in}}%
\pgfpathclose%
\pgfusepath{stroke,fill}%
}%
\begin{pgfscope}%
\pgfsys@transformshift{3.067617in}{2.953639in}%
\pgfsys@useobject{currentmarker}{}%
\end{pgfscope}%
\end{pgfscope}%
\begin{pgfscope}%
\pgfpathrectangle{\pgfqpoint{0.765000in}{0.660000in}}{\pgfqpoint{4.620000in}{4.620000in}}%
\pgfusepath{clip}%
\pgfsetroundcap%
\pgfsetroundjoin%
\pgfsetlinewidth{1.204500pt}%
\definecolor{currentstroke}{rgb}{0.176471,0.192157,0.258824}%
\pgfsetstrokecolor{currentstroke}%
\pgfsetdash{}{0pt}%
\pgfpathmoveto{\pgfqpoint{3.136889in}{2.948562in}}%
\pgfusepath{stroke}%
\end{pgfscope}%
\begin{pgfscope}%
\pgfpathrectangle{\pgfqpoint{0.765000in}{0.660000in}}{\pgfqpoint{4.620000in}{4.620000in}}%
\pgfusepath{clip}%
\pgfsetbuttcap%
\pgfsetmiterjoin%
\definecolor{currentfill}{rgb}{0.176471,0.192157,0.258824}%
\pgfsetfillcolor{currentfill}%
\pgfsetlinewidth{1.003750pt}%
\definecolor{currentstroke}{rgb}{0.176471,0.192157,0.258824}%
\pgfsetstrokecolor{currentstroke}%
\pgfsetdash{}{0pt}%
\pgfsys@defobject{currentmarker}{\pgfqpoint{-0.033333in}{-0.033333in}}{\pgfqpoint{0.033333in}{0.033333in}}{%
\pgfpathmoveto{\pgfqpoint{-0.000000in}{-0.033333in}}%
\pgfpathlineto{\pgfqpoint{0.033333in}{0.033333in}}%
\pgfpathlineto{\pgfqpoint{-0.033333in}{0.033333in}}%
\pgfpathlineto{\pgfqpoint{-0.000000in}{-0.033333in}}%
\pgfpathclose%
\pgfusepath{stroke,fill}%
}%
\begin{pgfscope}%
\pgfsys@transformshift{3.136889in}{2.948562in}%
\pgfsys@useobject{currentmarker}{}%
\end{pgfscope}%
\end{pgfscope}%
\begin{pgfscope}%
\pgfpathrectangle{\pgfqpoint{0.765000in}{0.660000in}}{\pgfqpoint{4.620000in}{4.620000in}}%
\pgfusepath{clip}%
\pgfsetroundcap%
\pgfsetroundjoin%
\pgfsetlinewidth{1.204500pt}%
\definecolor{currentstroke}{rgb}{0.176471,0.192157,0.258824}%
\pgfsetstrokecolor{currentstroke}%
\pgfsetdash{}{0pt}%
\pgfpathmoveto{\pgfqpoint{3.140948in}{2.989557in}}%
\pgfusepath{stroke}%
\end{pgfscope}%
\begin{pgfscope}%
\pgfpathrectangle{\pgfqpoint{0.765000in}{0.660000in}}{\pgfqpoint{4.620000in}{4.620000in}}%
\pgfusepath{clip}%
\pgfsetbuttcap%
\pgfsetmiterjoin%
\definecolor{currentfill}{rgb}{0.176471,0.192157,0.258824}%
\pgfsetfillcolor{currentfill}%
\pgfsetlinewidth{1.003750pt}%
\definecolor{currentstroke}{rgb}{0.176471,0.192157,0.258824}%
\pgfsetstrokecolor{currentstroke}%
\pgfsetdash{}{0pt}%
\pgfsys@defobject{currentmarker}{\pgfqpoint{-0.033333in}{-0.033333in}}{\pgfqpoint{0.033333in}{0.033333in}}{%
\pgfpathmoveto{\pgfqpoint{-0.000000in}{-0.033333in}}%
\pgfpathlineto{\pgfqpoint{0.033333in}{0.033333in}}%
\pgfpathlineto{\pgfqpoint{-0.033333in}{0.033333in}}%
\pgfpathlineto{\pgfqpoint{-0.000000in}{-0.033333in}}%
\pgfpathclose%
\pgfusepath{stroke,fill}%
}%
\begin{pgfscope}%
\pgfsys@transformshift{3.140948in}{2.989557in}%
\pgfsys@useobject{currentmarker}{}%
\end{pgfscope}%
\end{pgfscope}%
\begin{pgfscope}%
\pgfpathrectangle{\pgfqpoint{0.765000in}{0.660000in}}{\pgfqpoint{4.620000in}{4.620000in}}%
\pgfusepath{clip}%
\pgfsetroundcap%
\pgfsetroundjoin%
\pgfsetlinewidth{1.204500pt}%
\definecolor{currentstroke}{rgb}{0.176471,0.192157,0.258824}%
\pgfsetstrokecolor{currentstroke}%
\pgfsetdash{}{0pt}%
\pgfpathmoveto{\pgfqpoint{3.102685in}{2.899458in}}%
\pgfusepath{stroke}%
\end{pgfscope}%
\begin{pgfscope}%
\pgfpathrectangle{\pgfqpoint{0.765000in}{0.660000in}}{\pgfqpoint{4.620000in}{4.620000in}}%
\pgfusepath{clip}%
\pgfsetbuttcap%
\pgfsetmiterjoin%
\definecolor{currentfill}{rgb}{0.176471,0.192157,0.258824}%
\pgfsetfillcolor{currentfill}%
\pgfsetlinewidth{1.003750pt}%
\definecolor{currentstroke}{rgb}{0.176471,0.192157,0.258824}%
\pgfsetstrokecolor{currentstroke}%
\pgfsetdash{}{0pt}%
\pgfsys@defobject{currentmarker}{\pgfqpoint{-0.033333in}{-0.033333in}}{\pgfqpoint{0.033333in}{0.033333in}}{%
\pgfpathmoveto{\pgfqpoint{-0.000000in}{-0.033333in}}%
\pgfpathlineto{\pgfqpoint{0.033333in}{0.033333in}}%
\pgfpathlineto{\pgfqpoint{-0.033333in}{0.033333in}}%
\pgfpathlineto{\pgfqpoint{-0.000000in}{-0.033333in}}%
\pgfpathclose%
\pgfusepath{stroke,fill}%
}%
\begin{pgfscope}%
\pgfsys@transformshift{3.102685in}{2.899458in}%
\pgfsys@useobject{currentmarker}{}%
\end{pgfscope}%
\end{pgfscope}%
\begin{pgfscope}%
\pgfpathrectangle{\pgfqpoint{0.765000in}{0.660000in}}{\pgfqpoint{4.620000in}{4.620000in}}%
\pgfusepath{clip}%
\pgfsetroundcap%
\pgfsetroundjoin%
\pgfsetlinewidth{1.204500pt}%
\definecolor{currentstroke}{rgb}{0.176471,0.192157,0.258824}%
\pgfsetstrokecolor{currentstroke}%
\pgfsetdash{}{0pt}%
\pgfpathmoveto{\pgfqpoint{3.116086in}{2.982375in}}%
\pgfusepath{stroke}%
\end{pgfscope}%
\begin{pgfscope}%
\pgfpathrectangle{\pgfqpoint{0.765000in}{0.660000in}}{\pgfqpoint{4.620000in}{4.620000in}}%
\pgfusepath{clip}%
\pgfsetbuttcap%
\pgfsetmiterjoin%
\definecolor{currentfill}{rgb}{0.176471,0.192157,0.258824}%
\pgfsetfillcolor{currentfill}%
\pgfsetlinewidth{1.003750pt}%
\definecolor{currentstroke}{rgb}{0.176471,0.192157,0.258824}%
\pgfsetstrokecolor{currentstroke}%
\pgfsetdash{}{0pt}%
\pgfsys@defobject{currentmarker}{\pgfqpoint{-0.033333in}{-0.033333in}}{\pgfqpoint{0.033333in}{0.033333in}}{%
\pgfpathmoveto{\pgfqpoint{-0.000000in}{-0.033333in}}%
\pgfpathlineto{\pgfqpoint{0.033333in}{0.033333in}}%
\pgfpathlineto{\pgfqpoint{-0.033333in}{0.033333in}}%
\pgfpathlineto{\pgfqpoint{-0.000000in}{-0.033333in}}%
\pgfpathclose%
\pgfusepath{stroke,fill}%
}%
\begin{pgfscope}%
\pgfsys@transformshift{3.116086in}{2.982375in}%
\pgfsys@useobject{currentmarker}{}%
\end{pgfscope}%
\end{pgfscope}%
\begin{pgfscope}%
\pgfpathrectangle{\pgfqpoint{0.765000in}{0.660000in}}{\pgfqpoint{4.620000in}{4.620000in}}%
\pgfusepath{clip}%
\pgfsetroundcap%
\pgfsetroundjoin%
\pgfsetlinewidth{1.204500pt}%
\definecolor{currentstroke}{rgb}{0.176471,0.192157,0.258824}%
\pgfsetstrokecolor{currentstroke}%
\pgfsetdash{}{0pt}%
\pgfpathmoveto{\pgfqpoint{3.198633in}{2.902098in}}%
\pgfusepath{stroke}%
\end{pgfscope}%
\begin{pgfscope}%
\pgfpathrectangle{\pgfqpoint{0.765000in}{0.660000in}}{\pgfqpoint{4.620000in}{4.620000in}}%
\pgfusepath{clip}%
\pgfsetbuttcap%
\pgfsetmiterjoin%
\definecolor{currentfill}{rgb}{0.176471,0.192157,0.258824}%
\pgfsetfillcolor{currentfill}%
\pgfsetlinewidth{1.003750pt}%
\definecolor{currentstroke}{rgb}{0.176471,0.192157,0.258824}%
\pgfsetstrokecolor{currentstroke}%
\pgfsetdash{}{0pt}%
\pgfsys@defobject{currentmarker}{\pgfqpoint{-0.033333in}{-0.033333in}}{\pgfqpoint{0.033333in}{0.033333in}}{%
\pgfpathmoveto{\pgfqpoint{-0.000000in}{-0.033333in}}%
\pgfpathlineto{\pgfqpoint{0.033333in}{0.033333in}}%
\pgfpathlineto{\pgfqpoint{-0.033333in}{0.033333in}}%
\pgfpathlineto{\pgfqpoint{-0.000000in}{-0.033333in}}%
\pgfpathclose%
\pgfusepath{stroke,fill}%
}%
\begin{pgfscope}%
\pgfsys@transformshift{3.198633in}{2.902098in}%
\pgfsys@useobject{currentmarker}{}%
\end{pgfscope}%
\end{pgfscope}%
\begin{pgfscope}%
\pgfpathrectangle{\pgfqpoint{0.765000in}{0.660000in}}{\pgfqpoint{4.620000in}{4.620000in}}%
\pgfusepath{clip}%
\pgfsetroundcap%
\pgfsetroundjoin%
\pgfsetlinewidth{1.204500pt}%
\definecolor{currentstroke}{rgb}{0.176471,0.192157,0.258824}%
\pgfsetstrokecolor{currentstroke}%
\pgfsetdash{}{0pt}%
\pgfpathmoveto{\pgfqpoint{3.077782in}{3.059492in}}%
\pgfusepath{stroke}%
\end{pgfscope}%
\begin{pgfscope}%
\pgfpathrectangle{\pgfqpoint{0.765000in}{0.660000in}}{\pgfqpoint{4.620000in}{4.620000in}}%
\pgfusepath{clip}%
\pgfsetbuttcap%
\pgfsetmiterjoin%
\definecolor{currentfill}{rgb}{0.176471,0.192157,0.258824}%
\pgfsetfillcolor{currentfill}%
\pgfsetlinewidth{1.003750pt}%
\definecolor{currentstroke}{rgb}{0.176471,0.192157,0.258824}%
\pgfsetstrokecolor{currentstroke}%
\pgfsetdash{}{0pt}%
\pgfsys@defobject{currentmarker}{\pgfqpoint{-0.033333in}{-0.033333in}}{\pgfqpoint{0.033333in}{0.033333in}}{%
\pgfpathmoveto{\pgfqpoint{-0.000000in}{-0.033333in}}%
\pgfpathlineto{\pgfqpoint{0.033333in}{0.033333in}}%
\pgfpathlineto{\pgfqpoint{-0.033333in}{0.033333in}}%
\pgfpathlineto{\pgfqpoint{-0.000000in}{-0.033333in}}%
\pgfpathclose%
\pgfusepath{stroke,fill}%
}%
\begin{pgfscope}%
\pgfsys@transformshift{3.077782in}{3.059492in}%
\pgfsys@useobject{currentmarker}{}%
\end{pgfscope}%
\end{pgfscope}%
\begin{pgfscope}%
\pgfpathrectangle{\pgfqpoint{0.765000in}{0.660000in}}{\pgfqpoint{4.620000in}{4.620000in}}%
\pgfusepath{clip}%
\pgfsetroundcap%
\pgfsetroundjoin%
\pgfsetlinewidth{1.204500pt}%
\definecolor{currentstroke}{rgb}{0.176471,0.192157,0.258824}%
\pgfsetstrokecolor{currentstroke}%
\pgfsetdash{}{0pt}%
\pgfpathmoveto{\pgfqpoint{3.231847in}{2.958032in}}%
\pgfusepath{stroke}%
\end{pgfscope}%
\begin{pgfscope}%
\pgfpathrectangle{\pgfqpoint{0.765000in}{0.660000in}}{\pgfqpoint{4.620000in}{4.620000in}}%
\pgfusepath{clip}%
\pgfsetbuttcap%
\pgfsetmiterjoin%
\definecolor{currentfill}{rgb}{0.176471,0.192157,0.258824}%
\pgfsetfillcolor{currentfill}%
\pgfsetlinewidth{1.003750pt}%
\definecolor{currentstroke}{rgb}{0.176471,0.192157,0.258824}%
\pgfsetstrokecolor{currentstroke}%
\pgfsetdash{}{0pt}%
\pgfsys@defobject{currentmarker}{\pgfqpoint{-0.033333in}{-0.033333in}}{\pgfqpoint{0.033333in}{0.033333in}}{%
\pgfpathmoveto{\pgfqpoint{-0.000000in}{-0.033333in}}%
\pgfpathlineto{\pgfqpoint{0.033333in}{0.033333in}}%
\pgfpathlineto{\pgfqpoint{-0.033333in}{0.033333in}}%
\pgfpathlineto{\pgfqpoint{-0.000000in}{-0.033333in}}%
\pgfpathclose%
\pgfusepath{stroke,fill}%
}%
\begin{pgfscope}%
\pgfsys@transformshift{3.231847in}{2.958032in}%
\pgfsys@useobject{currentmarker}{}%
\end{pgfscope}%
\end{pgfscope}%
\begin{pgfscope}%
\pgfpathrectangle{\pgfqpoint{0.765000in}{0.660000in}}{\pgfqpoint{4.620000in}{4.620000in}}%
\pgfusepath{clip}%
\pgfsetroundcap%
\pgfsetroundjoin%
\pgfsetlinewidth{1.204500pt}%
\definecolor{currentstroke}{rgb}{0.176471,0.192157,0.258824}%
\pgfsetstrokecolor{currentstroke}%
\pgfsetdash{}{0pt}%
\pgfpathmoveto{\pgfqpoint{3.208437in}{2.846705in}}%
\pgfusepath{stroke}%
\end{pgfscope}%
\begin{pgfscope}%
\pgfpathrectangle{\pgfqpoint{0.765000in}{0.660000in}}{\pgfqpoint{4.620000in}{4.620000in}}%
\pgfusepath{clip}%
\pgfsetbuttcap%
\pgfsetmiterjoin%
\definecolor{currentfill}{rgb}{0.176471,0.192157,0.258824}%
\pgfsetfillcolor{currentfill}%
\pgfsetlinewidth{1.003750pt}%
\definecolor{currentstroke}{rgb}{0.176471,0.192157,0.258824}%
\pgfsetstrokecolor{currentstroke}%
\pgfsetdash{}{0pt}%
\pgfsys@defobject{currentmarker}{\pgfqpoint{-0.033333in}{-0.033333in}}{\pgfqpoint{0.033333in}{0.033333in}}{%
\pgfpathmoveto{\pgfqpoint{-0.000000in}{-0.033333in}}%
\pgfpathlineto{\pgfqpoint{0.033333in}{0.033333in}}%
\pgfpathlineto{\pgfqpoint{-0.033333in}{0.033333in}}%
\pgfpathlineto{\pgfqpoint{-0.000000in}{-0.033333in}}%
\pgfpathclose%
\pgfusepath{stroke,fill}%
}%
\begin{pgfscope}%
\pgfsys@transformshift{3.208437in}{2.846705in}%
\pgfsys@useobject{currentmarker}{}%
\end{pgfscope}%
\end{pgfscope}%
\begin{pgfscope}%
\pgfpathrectangle{\pgfqpoint{0.765000in}{0.660000in}}{\pgfqpoint{4.620000in}{4.620000in}}%
\pgfusepath{clip}%
\pgfsetroundcap%
\pgfsetroundjoin%
\pgfsetlinewidth{1.204500pt}%
\definecolor{currentstroke}{rgb}{0.176471,0.192157,0.258824}%
\pgfsetstrokecolor{currentstroke}%
\pgfsetdash{}{0pt}%
\pgfpathmoveto{\pgfqpoint{3.069418in}{3.025080in}}%
\pgfusepath{stroke}%
\end{pgfscope}%
\begin{pgfscope}%
\pgfpathrectangle{\pgfqpoint{0.765000in}{0.660000in}}{\pgfqpoint{4.620000in}{4.620000in}}%
\pgfusepath{clip}%
\pgfsetbuttcap%
\pgfsetmiterjoin%
\definecolor{currentfill}{rgb}{0.176471,0.192157,0.258824}%
\pgfsetfillcolor{currentfill}%
\pgfsetlinewidth{1.003750pt}%
\definecolor{currentstroke}{rgb}{0.176471,0.192157,0.258824}%
\pgfsetstrokecolor{currentstroke}%
\pgfsetdash{}{0pt}%
\pgfsys@defobject{currentmarker}{\pgfqpoint{-0.033333in}{-0.033333in}}{\pgfqpoint{0.033333in}{0.033333in}}{%
\pgfpathmoveto{\pgfqpoint{-0.000000in}{-0.033333in}}%
\pgfpathlineto{\pgfqpoint{0.033333in}{0.033333in}}%
\pgfpathlineto{\pgfqpoint{-0.033333in}{0.033333in}}%
\pgfpathlineto{\pgfqpoint{-0.000000in}{-0.033333in}}%
\pgfpathclose%
\pgfusepath{stroke,fill}%
}%
\begin{pgfscope}%
\pgfsys@transformshift{3.069418in}{3.025080in}%
\pgfsys@useobject{currentmarker}{}%
\end{pgfscope}%
\end{pgfscope}%
\begin{pgfscope}%
\pgfpathrectangle{\pgfqpoint{0.765000in}{0.660000in}}{\pgfqpoint{4.620000in}{4.620000in}}%
\pgfusepath{clip}%
\pgfsetroundcap%
\pgfsetroundjoin%
\pgfsetlinewidth{1.204500pt}%
\definecolor{currentstroke}{rgb}{0.176471,0.192157,0.258824}%
\pgfsetstrokecolor{currentstroke}%
\pgfsetdash{}{0pt}%
\pgfpathmoveto{\pgfqpoint{3.018308in}{2.873971in}}%
\pgfusepath{stroke}%
\end{pgfscope}%
\begin{pgfscope}%
\pgfpathrectangle{\pgfqpoint{0.765000in}{0.660000in}}{\pgfqpoint{4.620000in}{4.620000in}}%
\pgfusepath{clip}%
\pgfsetbuttcap%
\pgfsetmiterjoin%
\definecolor{currentfill}{rgb}{0.176471,0.192157,0.258824}%
\pgfsetfillcolor{currentfill}%
\pgfsetlinewidth{1.003750pt}%
\definecolor{currentstroke}{rgb}{0.176471,0.192157,0.258824}%
\pgfsetstrokecolor{currentstroke}%
\pgfsetdash{}{0pt}%
\pgfsys@defobject{currentmarker}{\pgfqpoint{-0.033333in}{-0.033333in}}{\pgfqpoint{0.033333in}{0.033333in}}{%
\pgfpathmoveto{\pgfqpoint{-0.000000in}{-0.033333in}}%
\pgfpathlineto{\pgfqpoint{0.033333in}{0.033333in}}%
\pgfpathlineto{\pgfqpoint{-0.033333in}{0.033333in}}%
\pgfpathlineto{\pgfqpoint{-0.000000in}{-0.033333in}}%
\pgfpathclose%
\pgfusepath{stroke,fill}%
}%
\begin{pgfscope}%
\pgfsys@transformshift{3.018308in}{2.873971in}%
\pgfsys@useobject{currentmarker}{}%
\end{pgfscope}%
\end{pgfscope}%
\begin{pgfscope}%
\pgfpathrectangle{\pgfqpoint{0.765000in}{0.660000in}}{\pgfqpoint{4.620000in}{4.620000in}}%
\pgfusepath{clip}%
\pgfsetroundcap%
\pgfsetroundjoin%
\pgfsetlinewidth{1.204500pt}%
\definecolor{currentstroke}{rgb}{0.176471,0.192157,0.258824}%
\pgfsetstrokecolor{currentstroke}%
\pgfsetdash{}{0pt}%
\pgfpathmoveto{\pgfqpoint{3.019663in}{3.025821in}}%
\pgfusepath{stroke}%
\end{pgfscope}%
\begin{pgfscope}%
\pgfpathrectangle{\pgfqpoint{0.765000in}{0.660000in}}{\pgfqpoint{4.620000in}{4.620000in}}%
\pgfusepath{clip}%
\pgfsetbuttcap%
\pgfsetmiterjoin%
\definecolor{currentfill}{rgb}{0.176471,0.192157,0.258824}%
\pgfsetfillcolor{currentfill}%
\pgfsetlinewidth{1.003750pt}%
\definecolor{currentstroke}{rgb}{0.176471,0.192157,0.258824}%
\pgfsetstrokecolor{currentstroke}%
\pgfsetdash{}{0pt}%
\pgfsys@defobject{currentmarker}{\pgfqpoint{-0.033333in}{-0.033333in}}{\pgfqpoint{0.033333in}{0.033333in}}{%
\pgfpathmoveto{\pgfqpoint{-0.000000in}{-0.033333in}}%
\pgfpathlineto{\pgfqpoint{0.033333in}{0.033333in}}%
\pgfpathlineto{\pgfqpoint{-0.033333in}{0.033333in}}%
\pgfpathlineto{\pgfqpoint{-0.000000in}{-0.033333in}}%
\pgfpathclose%
\pgfusepath{stroke,fill}%
}%
\begin{pgfscope}%
\pgfsys@transformshift{3.019663in}{3.025821in}%
\pgfsys@useobject{currentmarker}{}%
\end{pgfscope}%
\end{pgfscope}%
\begin{pgfscope}%
\pgfpathrectangle{\pgfqpoint{0.765000in}{0.660000in}}{\pgfqpoint{4.620000in}{4.620000in}}%
\pgfusepath{clip}%
\pgfsetroundcap%
\pgfsetroundjoin%
\pgfsetlinewidth{1.204500pt}%
\definecolor{currentstroke}{rgb}{0.176471,0.192157,0.258824}%
\pgfsetstrokecolor{currentstroke}%
\pgfsetdash{}{0pt}%
\pgfpathmoveto{\pgfqpoint{3.046922in}{3.003369in}}%
\pgfusepath{stroke}%
\end{pgfscope}%
\begin{pgfscope}%
\pgfpathrectangle{\pgfqpoint{0.765000in}{0.660000in}}{\pgfqpoint{4.620000in}{4.620000in}}%
\pgfusepath{clip}%
\pgfsetbuttcap%
\pgfsetmiterjoin%
\definecolor{currentfill}{rgb}{0.176471,0.192157,0.258824}%
\pgfsetfillcolor{currentfill}%
\pgfsetlinewidth{1.003750pt}%
\definecolor{currentstroke}{rgb}{0.176471,0.192157,0.258824}%
\pgfsetstrokecolor{currentstroke}%
\pgfsetdash{}{0pt}%
\pgfsys@defobject{currentmarker}{\pgfqpoint{-0.033333in}{-0.033333in}}{\pgfqpoint{0.033333in}{0.033333in}}{%
\pgfpathmoveto{\pgfqpoint{-0.000000in}{-0.033333in}}%
\pgfpathlineto{\pgfqpoint{0.033333in}{0.033333in}}%
\pgfpathlineto{\pgfqpoint{-0.033333in}{0.033333in}}%
\pgfpathlineto{\pgfqpoint{-0.000000in}{-0.033333in}}%
\pgfpathclose%
\pgfusepath{stroke,fill}%
}%
\begin{pgfscope}%
\pgfsys@transformshift{3.046922in}{3.003369in}%
\pgfsys@useobject{currentmarker}{}%
\end{pgfscope}%
\end{pgfscope}%
\begin{pgfscope}%
\pgfpathrectangle{\pgfqpoint{0.765000in}{0.660000in}}{\pgfqpoint{4.620000in}{4.620000in}}%
\pgfusepath{clip}%
\pgfsetroundcap%
\pgfsetroundjoin%
\pgfsetlinewidth{1.204500pt}%
\definecolor{currentstroke}{rgb}{0.176471,0.192157,0.258824}%
\pgfsetstrokecolor{currentstroke}%
\pgfsetdash{}{0pt}%
\pgfpathmoveto{\pgfqpoint{3.086016in}{3.018885in}}%
\pgfusepath{stroke}%
\end{pgfscope}%
\begin{pgfscope}%
\pgfpathrectangle{\pgfqpoint{0.765000in}{0.660000in}}{\pgfqpoint{4.620000in}{4.620000in}}%
\pgfusepath{clip}%
\pgfsetbuttcap%
\pgfsetmiterjoin%
\definecolor{currentfill}{rgb}{0.176471,0.192157,0.258824}%
\pgfsetfillcolor{currentfill}%
\pgfsetlinewidth{1.003750pt}%
\definecolor{currentstroke}{rgb}{0.176471,0.192157,0.258824}%
\pgfsetstrokecolor{currentstroke}%
\pgfsetdash{}{0pt}%
\pgfsys@defobject{currentmarker}{\pgfqpoint{-0.033333in}{-0.033333in}}{\pgfqpoint{0.033333in}{0.033333in}}{%
\pgfpathmoveto{\pgfqpoint{-0.000000in}{-0.033333in}}%
\pgfpathlineto{\pgfqpoint{0.033333in}{0.033333in}}%
\pgfpathlineto{\pgfqpoint{-0.033333in}{0.033333in}}%
\pgfpathlineto{\pgfqpoint{-0.000000in}{-0.033333in}}%
\pgfpathclose%
\pgfusepath{stroke,fill}%
}%
\begin{pgfscope}%
\pgfsys@transformshift{3.086016in}{3.018885in}%
\pgfsys@useobject{currentmarker}{}%
\end{pgfscope}%
\end{pgfscope}%
\begin{pgfscope}%
\pgfpathrectangle{\pgfqpoint{0.765000in}{0.660000in}}{\pgfqpoint{4.620000in}{4.620000in}}%
\pgfusepath{clip}%
\pgfsetroundcap%
\pgfsetroundjoin%
\pgfsetlinewidth{1.204500pt}%
\definecolor{currentstroke}{rgb}{0.176471,0.192157,0.258824}%
\pgfsetstrokecolor{currentstroke}%
\pgfsetdash{}{0pt}%
\pgfpathmoveto{\pgfqpoint{3.071425in}{2.926220in}}%
\pgfusepath{stroke}%
\end{pgfscope}%
\begin{pgfscope}%
\pgfpathrectangle{\pgfqpoint{0.765000in}{0.660000in}}{\pgfqpoint{4.620000in}{4.620000in}}%
\pgfusepath{clip}%
\pgfsetbuttcap%
\pgfsetmiterjoin%
\definecolor{currentfill}{rgb}{0.176471,0.192157,0.258824}%
\pgfsetfillcolor{currentfill}%
\pgfsetlinewidth{1.003750pt}%
\definecolor{currentstroke}{rgb}{0.176471,0.192157,0.258824}%
\pgfsetstrokecolor{currentstroke}%
\pgfsetdash{}{0pt}%
\pgfsys@defobject{currentmarker}{\pgfqpoint{-0.033333in}{-0.033333in}}{\pgfqpoint{0.033333in}{0.033333in}}{%
\pgfpathmoveto{\pgfqpoint{-0.000000in}{-0.033333in}}%
\pgfpathlineto{\pgfqpoint{0.033333in}{0.033333in}}%
\pgfpathlineto{\pgfqpoint{-0.033333in}{0.033333in}}%
\pgfpathlineto{\pgfqpoint{-0.000000in}{-0.033333in}}%
\pgfpathclose%
\pgfusepath{stroke,fill}%
}%
\begin{pgfscope}%
\pgfsys@transformshift{3.071425in}{2.926220in}%
\pgfsys@useobject{currentmarker}{}%
\end{pgfscope}%
\end{pgfscope}%
\begin{pgfscope}%
\pgfpathrectangle{\pgfqpoint{0.765000in}{0.660000in}}{\pgfqpoint{4.620000in}{4.620000in}}%
\pgfusepath{clip}%
\pgfsetroundcap%
\pgfsetroundjoin%
\pgfsetlinewidth{1.204500pt}%
\definecolor{currentstroke}{rgb}{0.176471,0.192157,0.258824}%
\pgfsetstrokecolor{currentstroke}%
\pgfsetdash{}{0pt}%
\pgfpathmoveto{\pgfqpoint{3.297732in}{2.926721in}}%
\pgfusepath{stroke}%
\end{pgfscope}%
\begin{pgfscope}%
\pgfpathrectangle{\pgfqpoint{0.765000in}{0.660000in}}{\pgfqpoint{4.620000in}{4.620000in}}%
\pgfusepath{clip}%
\pgfsetbuttcap%
\pgfsetmiterjoin%
\definecolor{currentfill}{rgb}{0.176471,0.192157,0.258824}%
\pgfsetfillcolor{currentfill}%
\pgfsetlinewidth{1.003750pt}%
\definecolor{currentstroke}{rgb}{0.176471,0.192157,0.258824}%
\pgfsetstrokecolor{currentstroke}%
\pgfsetdash{}{0pt}%
\pgfsys@defobject{currentmarker}{\pgfqpoint{-0.033333in}{-0.033333in}}{\pgfqpoint{0.033333in}{0.033333in}}{%
\pgfpathmoveto{\pgfqpoint{-0.000000in}{-0.033333in}}%
\pgfpathlineto{\pgfqpoint{0.033333in}{0.033333in}}%
\pgfpathlineto{\pgfqpoint{-0.033333in}{0.033333in}}%
\pgfpathlineto{\pgfqpoint{-0.000000in}{-0.033333in}}%
\pgfpathclose%
\pgfusepath{stroke,fill}%
}%
\begin{pgfscope}%
\pgfsys@transformshift{3.297732in}{2.926721in}%
\pgfsys@useobject{currentmarker}{}%
\end{pgfscope}%
\end{pgfscope}%
\begin{pgfscope}%
\pgfpathrectangle{\pgfqpoint{0.765000in}{0.660000in}}{\pgfqpoint{4.620000in}{4.620000in}}%
\pgfusepath{clip}%
\pgfsetroundcap%
\pgfsetroundjoin%
\pgfsetlinewidth{1.204500pt}%
\definecolor{currentstroke}{rgb}{0.176471,0.192157,0.258824}%
\pgfsetstrokecolor{currentstroke}%
\pgfsetdash{}{0pt}%
\pgfpathmoveto{\pgfqpoint{3.303963in}{2.945570in}}%
\pgfusepath{stroke}%
\end{pgfscope}%
\begin{pgfscope}%
\pgfpathrectangle{\pgfqpoint{0.765000in}{0.660000in}}{\pgfqpoint{4.620000in}{4.620000in}}%
\pgfusepath{clip}%
\pgfsetbuttcap%
\pgfsetmiterjoin%
\definecolor{currentfill}{rgb}{0.176471,0.192157,0.258824}%
\pgfsetfillcolor{currentfill}%
\pgfsetlinewidth{1.003750pt}%
\definecolor{currentstroke}{rgb}{0.176471,0.192157,0.258824}%
\pgfsetstrokecolor{currentstroke}%
\pgfsetdash{}{0pt}%
\pgfsys@defobject{currentmarker}{\pgfqpoint{-0.033333in}{-0.033333in}}{\pgfqpoint{0.033333in}{0.033333in}}{%
\pgfpathmoveto{\pgfqpoint{-0.000000in}{-0.033333in}}%
\pgfpathlineto{\pgfqpoint{0.033333in}{0.033333in}}%
\pgfpathlineto{\pgfqpoint{-0.033333in}{0.033333in}}%
\pgfpathlineto{\pgfqpoint{-0.000000in}{-0.033333in}}%
\pgfpathclose%
\pgfusepath{stroke,fill}%
}%
\begin{pgfscope}%
\pgfsys@transformshift{3.303963in}{2.945570in}%
\pgfsys@useobject{currentmarker}{}%
\end{pgfscope}%
\end{pgfscope}%
\begin{pgfscope}%
\pgfpathrectangle{\pgfqpoint{0.765000in}{0.660000in}}{\pgfqpoint{4.620000in}{4.620000in}}%
\pgfusepath{clip}%
\pgfsetroundcap%
\pgfsetroundjoin%
\pgfsetlinewidth{1.204500pt}%
\definecolor{currentstroke}{rgb}{0.176471,0.192157,0.258824}%
\pgfsetstrokecolor{currentstroke}%
\pgfsetdash{}{0pt}%
\pgfpathmoveto{\pgfqpoint{3.145205in}{2.871161in}}%
\pgfusepath{stroke}%
\end{pgfscope}%
\begin{pgfscope}%
\pgfpathrectangle{\pgfqpoint{0.765000in}{0.660000in}}{\pgfqpoint{4.620000in}{4.620000in}}%
\pgfusepath{clip}%
\pgfsetbuttcap%
\pgfsetmiterjoin%
\definecolor{currentfill}{rgb}{0.176471,0.192157,0.258824}%
\pgfsetfillcolor{currentfill}%
\pgfsetlinewidth{1.003750pt}%
\definecolor{currentstroke}{rgb}{0.176471,0.192157,0.258824}%
\pgfsetstrokecolor{currentstroke}%
\pgfsetdash{}{0pt}%
\pgfsys@defobject{currentmarker}{\pgfqpoint{-0.033333in}{-0.033333in}}{\pgfqpoint{0.033333in}{0.033333in}}{%
\pgfpathmoveto{\pgfqpoint{-0.000000in}{-0.033333in}}%
\pgfpathlineto{\pgfqpoint{0.033333in}{0.033333in}}%
\pgfpathlineto{\pgfqpoint{-0.033333in}{0.033333in}}%
\pgfpathlineto{\pgfqpoint{-0.000000in}{-0.033333in}}%
\pgfpathclose%
\pgfusepath{stroke,fill}%
}%
\begin{pgfscope}%
\pgfsys@transformshift{3.145205in}{2.871161in}%
\pgfsys@useobject{currentmarker}{}%
\end{pgfscope}%
\end{pgfscope}%
\begin{pgfscope}%
\pgfpathrectangle{\pgfqpoint{0.765000in}{0.660000in}}{\pgfqpoint{4.620000in}{4.620000in}}%
\pgfusepath{clip}%
\pgfsetroundcap%
\pgfsetroundjoin%
\pgfsetlinewidth{1.204500pt}%
\definecolor{currentstroke}{rgb}{0.176471,0.192157,0.258824}%
\pgfsetstrokecolor{currentstroke}%
\pgfsetdash{}{0pt}%
\pgfpathmoveto{\pgfqpoint{3.218412in}{2.838272in}}%
\pgfusepath{stroke}%
\end{pgfscope}%
\begin{pgfscope}%
\pgfpathrectangle{\pgfqpoint{0.765000in}{0.660000in}}{\pgfqpoint{4.620000in}{4.620000in}}%
\pgfusepath{clip}%
\pgfsetbuttcap%
\pgfsetmiterjoin%
\definecolor{currentfill}{rgb}{0.176471,0.192157,0.258824}%
\pgfsetfillcolor{currentfill}%
\pgfsetlinewidth{1.003750pt}%
\definecolor{currentstroke}{rgb}{0.176471,0.192157,0.258824}%
\pgfsetstrokecolor{currentstroke}%
\pgfsetdash{}{0pt}%
\pgfsys@defobject{currentmarker}{\pgfqpoint{-0.033333in}{-0.033333in}}{\pgfqpoint{0.033333in}{0.033333in}}{%
\pgfpathmoveto{\pgfqpoint{-0.000000in}{-0.033333in}}%
\pgfpathlineto{\pgfqpoint{0.033333in}{0.033333in}}%
\pgfpathlineto{\pgfqpoint{-0.033333in}{0.033333in}}%
\pgfpathlineto{\pgfqpoint{-0.000000in}{-0.033333in}}%
\pgfpathclose%
\pgfusepath{stroke,fill}%
}%
\begin{pgfscope}%
\pgfsys@transformshift{3.218412in}{2.838272in}%
\pgfsys@useobject{currentmarker}{}%
\end{pgfscope}%
\end{pgfscope}%
\begin{pgfscope}%
\pgfpathrectangle{\pgfqpoint{0.765000in}{0.660000in}}{\pgfqpoint{4.620000in}{4.620000in}}%
\pgfusepath{clip}%
\pgfsetroundcap%
\pgfsetroundjoin%
\pgfsetlinewidth{1.204500pt}%
\definecolor{currentstroke}{rgb}{0.176471,0.192157,0.258824}%
\pgfsetstrokecolor{currentstroke}%
\pgfsetdash{}{0pt}%
\pgfpathmoveto{\pgfqpoint{3.255269in}{2.810537in}}%
\pgfusepath{stroke}%
\end{pgfscope}%
\begin{pgfscope}%
\pgfpathrectangle{\pgfqpoint{0.765000in}{0.660000in}}{\pgfqpoint{4.620000in}{4.620000in}}%
\pgfusepath{clip}%
\pgfsetbuttcap%
\pgfsetmiterjoin%
\definecolor{currentfill}{rgb}{0.176471,0.192157,0.258824}%
\pgfsetfillcolor{currentfill}%
\pgfsetlinewidth{1.003750pt}%
\definecolor{currentstroke}{rgb}{0.176471,0.192157,0.258824}%
\pgfsetstrokecolor{currentstroke}%
\pgfsetdash{}{0pt}%
\pgfsys@defobject{currentmarker}{\pgfqpoint{-0.033333in}{-0.033333in}}{\pgfqpoint{0.033333in}{0.033333in}}{%
\pgfpathmoveto{\pgfqpoint{-0.000000in}{-0.033333in}}%
\pgfpathlineto{\pgfqpoint{0.033333in}{0.033333in}}%
\pgfpathlineto{\pgfqpoint{-0.033333in}{0.033333in}}%
\pgfpathlineto{\pgfqpoint{-0.000000in}{-0.033333in}}%
\pgfpathclose%
\pgfusepath{stroke,fill}%
}%
\begin{pgfscope}%
\pgfsys@transformshift{3.255269in}{2.810537in}%
\pgfsys@useobject{currentmarker}{}%
\end{pgfscope}%
\end{pgfscope}%
\begin{pgfscope}%
\pgfpathrectangle{\pgfqpoint{0.765000in}{0.660000in}}{\pgfqpoint{4.620000in}{4.620000in}}%
\pgfusepath{clip}%
\pgfsetroundcap%
\pgfsetroundjoin%
\pgfsetlinewidth{1.204500pt}%
\definecolor{currentstroke}{rgb}{0.176471,0.192157,0.258824}%
\pgfsetstrokecolor{currentstroke}%
\pgfsetdash{}{0pt}%
\pgfpathmoveto{\pgfqpoint{3.159844in}{3.032850in}}%
\pgfusepath{stroke}%
\end{pgfscope}%
\begin{pgfscope}%
\pgfpathrectangle{\pgfqpoint{0.765000in}{0.660000in}}{\pgfqpoint{4.620000in}{4.620000in}}%
\pgfusepath{clip}%
\pgfsetbuttcap%
\pgfsetmiterjoin%
\definecolor{currentfill}{rgb}{0.176471,0.192157,0.258824}%
\pgfsetfillcolor{currentfill}%
\pgfsetlinewidth{1.003750pt}%
\definecolor{currentstroke}{rgb}{0.176471,0.192157,0.258824}%
\pgfsetstrokecolor{currentstroke}%
\pgfsetdash{}{0pt}%
\pgfsys@defobject{currentmarker}{\pgfqpoint{-0.033333in}{-0.033333in}}{\pgfqpoint{0.033333in}{0.033333in}}{%
\pgfpathmoveto{\pgfqpoint{-0.000000in}{-0.033333in}}%
\pgfpathlineto{\pgfqpoint{0.033333in}{0.033333in}}%
\pgfpathlineto{\pgfqpoint{-0.033333in}{0.033333in}}%
\pgfpathlineto{\pgfqpoint{-0.000000in}{-0.033333in}}%
\pgfpathclose%
\pgfusepath{stroke,fill}%
}%
\begin{pgfscope}%
\pgfsys@transformshift{3.159844in}{3.032850in}%
\pgfsys@useobject{currentmarker}{}%
\end{pgfscope}%
\end{pgfscope}%
\begin{pgfscope}%
\pgfpathrectangle{\pgfqpoint{0.765000in}{0.660000in}}{\pgfqpoint{4.620000in}{4.620000in}}%
\pgfusepath{clip}%
\pgfsetroundcap%
\pgfsetroundjoin%
\pgfsetlinewidth{1.204500pt}%
\definecolor{currentstroke}{rgb}{0.176471,0.192157,0.258824}%
\pgfsetstrokecolor{currentstroke}%
\pgfsetdash{}{0pt}%
\pgfpathmoveto{\pgfqpoint{3.269747in}{2.856089in}}%
\pgfusepath{stroke}%
\end{pgfscope}%
\begin{pgfscope}%
\pgfpathrectangle{\pgfqpoint{0.765000in}{0.660000in}}{\pgfqpoint{4.620000in}{4.620000in}}%
\pgfusepath{clip}%
\pgfsetbuttcap%
\pgfsetmiterjoin%
\definecolor{currentfill}{rgb}{0.176471,0.192157,0.258824}%
\pgfsetfillcolor{currentfill}%
\pgfsetlinewidth{1.003750pt}%
\definecolor{currentstroke}{rgb}{0.176471,0.192157,0.258824}%
\pgfsetstrokecolor{currentstroke}%
\pgfsetdash{}{0pt}%
\pgfsys@defobject{currentmarker}{\pgfqpoint{-0.033333in}{-0.033333in}}{\pgfqpoint{0.033333in}{0.033333in}}{%
\pgfpathmoveto{\pgfqpoint{-0.000000in}{-0.033333in}}%
\pgfpathlineto{\pgfqpoint{0.033333in}{0.033333in}}%
\pgfpathlineto{\pgfqpoint{-0.033333in}{0.033333in}}%
\pgfpathlineto{\pgfqpoint{-0.000000in}{-0.033333in}}%
\pgfpathclose%
\pgfusepath{stroke,fill}%
}%
\begin{pgfscope}%
\pgfsys@transformshift{3.269747in}{2.856089in}%
\pgfsys@useobject{currentmarker}{}%
\end{pgfscope}%
\end{pgfscope}%
\begin{pgfscope}%
\pgfpathrectangle{\pgfqpoint{0.765000in}{0.660000in}}{\pgfqpoint{4.620000in}{4.620000in}}%
\pgfusepath{clip}%
\pgfsetroundcap%
\pgfsetroundjoin%
\pgfsetlinewidth{1.204500pt}%
\definecolor{currentstroke}{rgb}{0.176471,0.192157,0.258824}%
\pgfsetstrokecolor{currentstroke}%
\pgfsetdash{}{0pt}%
\pgfpathmoveto{\pgfqpoint{3.241130in}{2.835111in}}%
\pgfusepath{stroke}%
\end{pgfscope}%
\begin{pgfscope}%
\pgfpathrectangle{\pgfqpoint{0.765000in}{0.660000in}}{\pgfqpoint{4.620000in}{4.620000in}}%
\pgfusepath{clip}%
\pgfsetbuttcap%
\pgfsetmiterjoin%
\definecolor{currentfill}{rgb}{0.176471,0.192157,0.258824}%
\pgfsetfillcolor{currentfill}%
\pgfsetlinewidth{1.003750pt}%
\definecolor{currentstroke}{rgb}{0.176471,0.192157,0.258824}%
\pgfsetstrokecolor{currentstroke}%
\pgfsetdash{}{0pt}%
\pgfsys@defobject{currentmarker}{\pgfqpoint{-0.033333in}{-0.033333in}}{\pgfqpoint{0.033333in}{0.033333in}}{%
\pgfpathmoveto{\pgfqpoint{-0.000000in}{-0.033333in}}%
\pgfpathlineto{\pgfqpoint{0.033333in}{0.033333in}}%
\pgfpathlineto{\pgfqpoint{-0.033333in}{0.033333in}}%
\pgfpathlineto{\pgfqpoint{-0.000000in}{-0.033333in}}%
\pgfpathclose%
\pgfusepath{stroke,fill}%
}%
\begin{pgfscope}%
\pgfsys@transformshift{3.241130in}{2.835111in}%
\pgfsys@useobject{currentmarker}{}%
\end{pgfscope}%
\end{pgfscope}%
\begin{pgfscope}%
\pgfpathrectangle{\pgfqpoint{0.765000in}{0.660000in}}{\pgfqpoint{4.620000in}{4.620000in}}%
\pgfusepath{clip}%
\pgfsetroundcap%
\pgfsetroundjoin%
\pgfsetlinewidth{1.204500pt}%
\definecolor{currentstroke}{rgb}{0.176471,0.192157,0.258824}%
\pgfsetstrokecolor{currentstroke}%
\pgfsetdash{}{0pt}%
\pgfpathmoveto{\pgfqpoint{3.013424in}{2.970740in}}%
\pgfusepath{stroke}%
\end{pgfscope}%
\begin{pgfscope}%
\pgfpathrectangle{\pgfqpoint{0.765000in}{0.660000in}}{\pgfqpoint{4.620000in}{4.620000in}}%
\pgfusepath{clip}%
\pgfsetbuttcap%
\pgfsetmiterjoin%
\definecolor{currentfill}{rgb}{0.176471,0.192157,0.258824}%
\pgfsetfillcolor{currentfill}%
\pgfsetlinewidth{1.003750pt}%
\definecolor{currentstroke}{rgb}{0.176471,0.192157,0.258824}%
\pgfsetstrokecolor{currentstroke}%
\pgfsetdash{}{0pt}%
\pgfsys@defobject{currentmarker}{\pgfqpoint{-0.033333in}{-0.033333in}}{\pgfqpoint{0.033333in}{0.033333in}}{%
\pgfpathmoveto{\pgfqpoint{-0.000000in}{-0.033333in}}%
\pgfpathlineto{\pgfqpoint{0.033333in}{0.033333in}}%
\pgfpathlineto{\pgfqpoint{-0.033333in}{0.033333in}}%
\pgfpathlineto{\pgfqpoint{-0.000000in}{-0.033333in}}%
\pgfpathclose%
\pgfusepath{stroke,fill}%
}%
\begin{pgfscope}%
\pgfsys@transformshift{3.013424in}{2.970740in}%
\pgfsys@useobject{currentmarker}{}%
\end{pgfscope}%
\end{pgfscope}%
\begin{pgfscope}%
\pgfpathrectangle{\pgfqpoint{0.765000in}{0.660000in}}{\pgfqpoint{4.620000in}{4.620000in}}%
\pgfusepath{clip}%
\pgfsetroundcap%
\pgfsetroundjoin%
\pgfsetlinewidth{1.204500pt}%
\definecolor{currentstroke}{rgb}{0.176471,0.192157,0.258824}%
\pgfsetstrokecolor{currentstroke}%
\pgfsetdash{}{0pt}%
\pgfpathmoveto{\pgfqpoint{3.305648in}{2.948154in}}%
\pgfusepath{stroke}%
\end{pgfscope}%
\begin{pgfscope}%
\pgfpathrectangle{\pgfqpoint{0.765000in}{0.660000in}}{\pgfqpoint{4.620000in}{4.620000in}}%
\pgfusepath{clip}%
\pgfsetbuttcap%
\pgfsetmiterjoin%
\definecolor{currentfill}{rgb}{0.176471,0.192157,0.258824}%
\pgfsetfillcolor{currentfill}%
\pgfsetlinewidth{1.003750pt}%
\definecolor{currentstroke}{rgb}{0.176471,0.192157,0.258824}%
\pgfsetstrokecolor{currentstroke}%
\pgfsetdash{}{0pt}%
\pgfsys@defobject{currentmarker}{\pgfqpoint{-0.033333in}{-0.033333in}}{\pgfqpoint{0.033333in}{0.033333in}}{%
\pgfpathmoveto{\pgfqpoint{-0.000000in}{-0.033333in}}%
\pgfpathlineto{\pgfqpoint{0.033333in}{0.033333in}}%
\pgfpathlineto{\pgfqpoint{-0.033333in}{0.033333in}}%
\pgfpathlineto{\pgfqpoint{-0.000000in}{-0.033333in}}%
\pgfpathclose%
\pgfusepath{stroke,fill}%
}%
\begin{pgfscope}%
\pgfsys@transformshift{3.305648in}{2.948154in}%
\pgfsys@useobject{currentmarker}{}%
\end{pgfscope}%
\end{pgfscope}%
\begin{pgfscope}%
\pgfpathrectangle{\pgfqpoint{0.765000in}{0.660000in}}{\pgfqpoint{4.620000in}{4.620000in}}%
\pgfusepath{clip}%
\pgfsetroundcap%
\pgfsetroundjoin%
\pgfsetlinewidth{1.204500pt}%
\definecolor{currentstroke}{rgb}{0.176471,0.192157,0.258824}%
\pgfsetstrokecolor{currentstroke}%
\pgfsetdash{}{0pt}%
\pgfpathmoveto{\pgfqpoint{3.143353in}{2.966419in}}%
\pgfusepath{stroke}%
\end{pgfscope}%
\begin{pgfscope}%
\pgfpathrectangle{\pgfqpoint{0.765000in}{0.660000in}}{\pgfqpoint{4.620000in}{4.620000in}}%
\pgfusepath{clip}%
\pgfsetbuttcap%
\pgfsetmiterjoin%
\definecolor{currentfill}{rgb}{0.176471,0.192157,0.258824}%
\pgfsetfillcolor{currentfill}%
\pgfsetlinewidth{1.003750pt}%
\definecolor{currentstroke}{rgb}{0.176471,0.192157,0.258824}%
\pgfsetstrokecolor{currentstroke}%
\pgfsetdash{}{0pt}%
\pgfsys@defobject{currentmarker}{\pgfqpoint{-0.033333in}{-0.033333in}}{\pgfqpoint{0.033333in}{0.033333in}}{%
\pgfpathmoveto{\pgfqpoint{-0.000000in}{-0.033333in}}%
\pgfpathlineto{\pgfqpoint{0.033333in}{0.033333in}}%
\pgfpathlineto{\pgfqpoint{-0.033333in}{0.033333in}}%
\pgfpathlineto{\pgfqpoint{-0.000000in}{-0.033333in}}%
\pgfpathclose%
\pgfusepath{stroke,fill}%
}%
\begin{pgfscope}%
\pgfsys@transformshift{3.143353in}{2.966419in}%
\pgfsys@useobject{currentmarker}{}%
\end{pgfscope}%
\end{pgfscope}%
\begin{pgfscope}%
\pgfpathrectangle{\pgfqpoint{0.765000in}{0.660000in}}{\pgfqpoint{4.620000in}{4.620000in}}%
\pgfusepath{clip}%
\pgfsetroundcap%
\pgfsetroundjoin%
\pgfsetlinewidth{1.204500pt}%
\definecolor{currentstroke}{rgb}{0.176471,0.192157,0.258824}%
\pgfsetstrokecolor{currentstroke}%
\pgfsetdash{}{0pt}%
\pgfpathmoveto{\pgfqpoint{3.279497in}{2.967796in}}%
\pgfusepath{stroke}%
\end{pgfscope}%
\begin{pgfscope}%
\pgfpathrectangle{\pgfqpoint{0.765000in}{0.660000in}}{\pgfqpoint{4.620000in}{4.620000in}}%
\pgfusepath{clip}%
\pgfsetbuttcap%
\pgfsetmiterjoin%
\definecolor{currentfill}{rgb}{0.176471,0.192157,0.258824}%
\pgfsetfillcolor{currentfill}%
\pgfsetlinewidth{1.003750pt}%
\definecolor{currentstroke}{rgb}{0.176471,0.192157,0.258824}%
\pgfsetstrokecolor{currentstroke}%
\pgfsetdash{}{0pt}%
\pgfsys@defobject{currentmarker}{\pgfqpoint{-0.033333in}{-0.033333in}}{\pgfqpoint{0.033333in}{0.033333in}}{%
\pgfpathmoveto{\pgfqpoint{-0.000000in}{-0.033333in}}%
\pgfpathlineto{\pgfqpoint{0.033333in}{0.033333in}}%
\pgfpathlineto{\pgfqpoint{-0.033333in}{0.033333in}}%
\pgfpathlineto{\pgfqpoint{-0.000000in}{-0.033333in}}%
\pgfpathclose%
\pgfusepath{stroke,fill}%
}%
\begin{pgfscope}%
\pgfsys@transformshift{3.279497in}{2.967796in}%
\pgfsys@useobject{currentmarker}{}%
\end{pgfscope}%
\end{pgfscope}%
\begin{pgfscope}%
\pgfpathrectangle{\pgfqpoint{0.765000in}{0.660000in}}{\pgfqpoint{4.620000in}{4.620000in}}%
\pgfusepath{clip}%
\pgfsetroundcap%
\pgfsetroundjoin%
\pgfsetlinewidth{1.204500pt}%
\definecolor{currentstroke}{rgb}{0.176471,0.192157,0.258824}%
\pgfsetstrokecolor{currentstroke}%
\pgfsetdash{}{0pt}%
\pgfpathmoveto{\pgfqpoint{3.237197in}{3.018424in}}%
\pgfusepath{stroke}%
\end{pgfscope}%
\begin{pgfscope}%
\pgfpathrectangle{\pgfqpoint{0.765000in}{0.660000in}}{\pgfqpoint{4.620000in}{4.620000in}}%
\pgfusepath{clip}%
\pgfsetbuttcap%
\pgfsetmiterjoin%
\definecolor{currentfill}{rgb}{0.176471,0.192157,0.258824}%
\pgfsetfillcolor{currentfill}%
\pgfsetlinewidth{1.003750pt}%
\definecolor{currentstroke}{rgb}{0.176471,0.192157,0.258824}%
\pgfsetstrokecolor{currentstroke}%
\pgfsetdash{}{0pt}%
\pgfsys@defobject{currentmarker}{\pgfqpoint{-0.033333in}{-0.033333in}}{\pgfqpoint{0.033333in}{0.033333in}}{%
\pgfpathmoveto{\pgfqpoint{-0.000000in}{-0.033333in}}%
\pgfpathlineto{\pgfqpoint{0.033333in}{0.033333in}}%
\pgfpathlineto{\pgfqpoint{-0.033333in}{0.033333in}}%
\pgfpathlineto{\pgfqpoint{-0.000000in}{-0.033333in}}%
\pgfpathclose%
\pgfusepath{stroke,fill}%
}%
\begin{pgfscope}%
\pgfsys@transformshift{3.237197in}{3.018424in}%
\pgfsys@useobject{currentmarker}{}%
\end{pgfscope}%
\end{pgfscope}%
\begin{pgfscope}%
\pgfpathrectangle{\pgfqpoint{0.765000in}{0.660000in}}{\pgfqpoint{4.620000in}{4.620000in}}%
\pgfusepath{clip}%
\pgfsetroundcap%
\pgfsetroundjoin%
\pgfsetlinewidth{1.204500pt}%
\definecolor{currentstroke}{rgb}{0.176471,0.192157,0.258824}%
\pgfsetstrokecolor{currentstroke}%
\pgfsetdash{}{0pt}%
\pgfpathmoveto{\pgfqpoint{3.105140in}{2.958241in}}%
\pgfusepath{stroke}%
\end{pgfscope}%
\begin{pgfscope}%
\pgfpathrectangle{\pgfqpoint{0.765000in}{0.660000in}}{\pgfqpoint{4.620000in}{4.620000in}}%
\pgfusepath{clip}%
\pgfsetbuttcap%
\pgfsetmiterjoin%
\definecolor{currentfill}{rgb}{0.176471,0.192157,0.258824}%
\pgfsetfillcolor{currentfill}%
\pgfsetlinewidth{1.003750pt}%
\definecolor{currentstroke}{rgb}{0.176471,0.192157,0.258824}%
\pgfsetstrokecolor{currentstroke}%
\pgfsetdash{}{0pt}%
\pgfsys@defobject{currentmarker}{\pgfqpoint{-0.033333in}{-0.033333in}}{\pgfqpoint{0.033333in}{0.033333in}}{%
\pgfpathmoveto{\pgfqpoint{-0.000000in}{-0.033333in}}%
\pgfpathlineto{\pgfqpoint{0.033333in}{0.033333in}}%
\pgfpathlineto{\pgfqpoint{-0.033333in}{0.033333in}}%
\pgfpathlineto{\pgfqpoint{-0.000000in}{-0.033333in}}%
\pgfpathclose%
\pgfusepath{stroke,fill}%
}%
\begin{pgfscope}%
\pgfsys@transformshift{3.105140in}{2.958241in}%
\pgfsys@useobject{currentmarker}{}%
\end{pgfscope}%
\end{pgfscope}%
\begin{pgfscope}%
\pgfpathrectangle{\pgfqpoint{0.765000in}{0.660000in}}{\pgfqpoint{4.620000in}{4.620000in}}%
\pgfusepath{clip}%
\pgfsetroundcap%
\pgfsetroundjoin%
\pgfsetlinewidth{1.204500pt}%
\definecolor{currentstroke}{rgb}{0.176471,0.192157,0.258824}%
\pgfsetstrokecolor{currentstroke}%
\pgfsetdash{}{0pt}%
\pgfpathmoveto{\pgfqpoint{3.194656in}{2.829669in}}%
\pgfusepath{stroke}%
\end{pgfscope}%
\begin{pgfscope}%
\pgfpathrectangle{\pgfqpoint{0.765000in}{0.660000in}}{\pgfqpoint{4.620000in}{4.620000in}}%
\pgfusepath{clip}%
\pgfsetbuttcap%
\pgfsetmiterjoin%
\definecolor{currentfill}{rgb}{0.176471,0.192157,0.258824}%
\pgfsetfillcolor{currentfill}%
\pgfsetlinewidth{1.003750pt}%
\definecolor{currentstroke}{rgb}{0.176471,0.192157,0.258824}%
\pgfsetstrokecolor{currentstroke}%
\pgfsetdash{}{0pt}%
\pgfsys@defobject{currentmarker}{\pgfqpoint{-0.033333in}{-0.033333in}}{\pgfqpoint{0.033333in}{0.033333in}}{%
\pgfpathmoveto{\pgfqpoint{-0.000000in}{-0.033333in}}%
\pgfpathlineto{\pgfqpoint{0.033333in}{0.033333in}}%
\pgfpathlineto{\pgfqpoint{-0.033333in}{0.033333in}}%
\pgfpathlineto{\pgfqpoint{-0.000000in}{-0.033333in}}%
\pgfpathclose%
\pgfusepath{stroke,fill}%
}%
\begin{pgfscope}%
\pgfsys@transformshift{3.194656in}{2.829669in}%
\pgfsys@useobject{currentmarker}{}%
\end{pgfscope}%
\end{pgfscope}%
\begin{pgfscope}%
\pgfpathrectangle{\pgfqpoint{0.765000in}{0.660000in}}{\pgfqpoint{4.620000in}{4.620000in}}%
\pgfusepath{clip}%
\pgfsetroundcap%
\pgfsetroundjoin%
\pgfsetlinewidth{1.204500pt}%
\definecolor{currentstroke}{rgb}{0.176471,0.192157,0.258824}%
\pgfsetstrokecolor{currentstroke}%
\pgfsetdash{}{0pt}%
\pgfpathmoveto{\pgfqpoint{3.102166in}{2.988090in}}%
\pgfusepath{stroke}%
\end{pgfscope}%
\begin{pgfscope}%
\pgfpathrectangle{\pgfqpoint{0.765000in}{0.660000in}}{\pgfqpoint{4.620000in}{4.620000in}}%
\pgfusepath{clip}%
\pgfsetbuttcap%
\pgfsetmiterjoin%
\definecolor{currentfill}{rgb}{0.176471,0.192157,0.258824}%
\pgfsetfillcolor{currentfill}%
\pgfsetlinewidth{1.003750pt}%
\definecolor{currentstroke}{rgb}{0.176471,0.192157,0.258824}%
\pgfsetstrokecolor{currentstroke}%
\pgfsetdash{}{0pt}%
\pgfsys@defobject{currentmarker}{\pgfqpoint{-0.033333in}{-0.033333in}}{\pgfqpoint{0.033333in}{0.033333in}}{%
\pgfpathmoveto{\pgfqpoint{-0.000000in}{-0.033333in}}%
\pgfpathlineto{\pgfqpoint{0.033333in}{0.033333in}}%
\pgfpathlineto{\pgfqpoint{-0.033333in}{0.033333in}}%
\pgfpathlineto{\pgfqpoint{-0.000000in}{-0.033333in}}%
\pgfpathclose%
\pgfusepath{stroke,fill}%
}%
\begin{pgfscope}%
\pgfsys@transformshift{3.102166in}{2.988090in}%
\pgfsys@useobject{currentmarker}{}%
\end{pgfscope}%
\end{pgfscope}%
\begin{pgfscope}%
\pgfpathrectangle{\pgfqpoint{0.765000in}{0.660000in}}{\pgfqpoint{4.620000in}{4.620000in}}%
\pgfusepath{clip}%
\pgfsetroundcap%
\pgfsetroundjoin%
\pgfsetlinewidth{1.204500pt}%
\definecolor{currentstroke}{rgb}{0.176471,0.192157,0.258824}%
\pgfsetstrokecolor{currentstroke}%
\pgfsetdash{}{0pt}%
\pgfpathmoveto{\pgfqpoint{3.078896in}{2.967502in}}%
\pgfusepath{stroke}%
\end{pgfscope}%
\begin{pgfscope}%
\pgfpathrectangle{\pgfqpoint{0.765000in}{0.660000in}}{\pgfqpoint{4.620000in}{4.620000in}}%
\pgfusepath{clip}%
\pgfsetbuttcap%
\pgfsetmiterjoin%
\definecolor{currentfill}{rgb}{0.176471,0.192157,0.258824}%
\pgfsetfillcolor{currentfill}%
\pgfsetlinewidth{1.003750pt}%
\definecolor{currentstroke}{rgb}{0.176471,0.192157,0.258824}%
\pgfsetstrokecolor{currentstroke}%
\pgfsetdash{}{0pt}%
\pgfsys@defobject{currentmarker}{\pgfqpoint{-0.033333in}{-0.033333in}}{\pgfqpoint{0.033333in}{0.033333in}}{%
\pgfpathmoveto{\pgfqpoint{-0.000000in}{-0.033333in}}%
\pgfpathlineto{\pgfqpoint{0.033333in}{0.033333in}}%
\pgfpathlineto{\pgfqpoint{-0.033333in}{0.033333in}}%
\pgfpathlineto{\pgfqpoint{-0.000000in}{-0.033333in}}%
\pgfpathclose%
\pgfusepath{stroke,fill}%
}%
\begin{pgfscope}%
\pgfsys@transformshift{3.078896in}{2.967502in}%
\pgfsys@useobject{currentmarker}{}%
\end{pgfscope}%
\end{pgfscope}%
\begin{pgfscope}%
\pgfpathrectangle{\pgfqpoint{0.765000in}{0.660000in}}{\pgfqpoint{4.620000in}{4.620000in}}%
\pgfusepath{clip}%
\pgfsetroundcap%
\pgfsetroundjoin%
\pgfsetlinewidth{1.204500pt}%
\definecolor{currentstroke}{rgb}{0.176471,0.192157,0.258824}%
\pgfsetstrokecolor{currentstroke}%
\pgfsetdash{}{0pt}%
\pgfpathmoveto{\pgfqpoint{3.175390in}{2.854058in}}%
\pgfusepath{stroke}%
\end{pgfscope}%
\begin{pgfscope}%
\pgfpathrectangle{\pgfqpoint{0.765000in}{0.660000in}}{\pgfqpoint{4.620000in}{4.620000in}}%
\pgfusepath{clip}%
\pgfsetbuttcap%
\pgfsetmiterjoin%
\definecolor{currentfill}{rgb}{0.176471,0.192157,0.258824}%
\pgfsetfillcolor{currentfill}%
\pgfsetlinewidth{1.003750pt}%
\definecolor{currentstroke}{rgb}{0.176471,0.192157,0.258824}%
\pgfsetstrokecolor{currentstroke}%
\pgfsetdash{}{0pt}%
\pgfsys@defobject{currentmarker}{\pgfqpoint{-0.033333in}{-0.033333in}}{\pgfqpoint{0.033333in}{0.033333in}}{%
\pgfpathmoveto{\pgfqpoint{-0.000000in}{-0.033333in}}%
\pgfpathlineto{\pgfqpoint{0.033333in}{0.033333in}}%
\pgfpathlineto{\pgfqpoint{-0.033333in}{0.033333in}}%
\pgfpathlineto{\pgfqpoint{-0.000000in}{-0.033333in}}%
\pgfpathclose%
\pgfusepath{stroke,fill}%
}%
\begin{pgfscope}%
\pgfsys@transformshift{3.175390in}{2.854058in}%
\pgfsys@useobject{currentmarker}{}%
\end{pgfscope}%
\end{pgfscope}%
\begin{pgfscope}%
\pgfpathrectangle{\pgfqpoint{0.765000in}{0.660000in}}{\pgfqpoint{4.620000in}{4.620000in}}%
\pgfusepath{clip}%
\pgfsetroundcap%
\pgfsetroundjoin%
\pgfsetlinewidth{1.204500pt}%
\definecolor{currentstroke}{rgb}{0.176471,0.192157,0.258824}%
\pgfsetstrokecolor{currentstroke}%
\pgfsetdash{}{0pt}%
\pgfpathmoveto{\pgfqpoint{3.147663in}{3.037608in}}%
\pgfusepath{stroke}%
\end{pgfscope}%
\begin{pgfscope}%
\pgfpathrectangle{\pgfqpoint{0.765000in}{0.660000in}}{\pgfqpoint{4.620000in}{4.620000in}}%
\pgfusepath{clip}%
\pgfsetbuttcap%
\pgfsetmiterjoin%
\definecolor{currentfill}{rgb}{0.176471,0.192157,0.258824}%
\pgfsetfillcolor{currentfill}%
\pgfsetlinewidth{1.003750pt}%
\definecolor{currentstroke}{rgb}{0.176471,0.192157,0.258824}%
\pgfsetstrokecolor{currentstroke}%
\pgfsetdash{}{0pt}%
\pgfsys@defobject{currentmarker}{\pgfqpoint{-0.033333in}{-0.033333in}}{\pgfqpoint{0.033333in}{0.033333in}}{%
\pgfpathmoveto{\pgfqpoint{-0.000000in}{-0.033333in}}%
\pgfpathlineto{\pgfqpoint{0.033333in}{0.033333in}}%
\pgfpathlineto{\pgfqpoint{-0.033333in}{0.033333in}}%
\pgfpathlineto{\pgfqpoint{-0.000000in}{-0.033333in}}%
\pgfpathclose%
\pgfusepath{stroke,fill}%
}%
\begin{pgfscope}%
\pgfsys@transformshift{3.147663in}{3.037608in}%
\pgfsys@useobject{currentmarker}{}%
\end{pgfscope}%
\end{pgfscope}%
\begin{pgfscope}%
\pgfpathrectangle{\pgfqpoint{0.765000in}{0.660000in}}{\pgfqpoint{4.620000in}{4.620000in}}%
\pgfusepath{clip}%
\pgfsetroundcap%
\pgfsetroundjoin%
\pgfsetlinewidth{1.204500pt}%
\definecolor{currentstroke}{rgb}{0.176471,0.192157,0.258824}%
\pgfsetstrokecolor{currentstroke}%
\pgfsetdash{}{0pt}%
\pgfpathmoveto{\pgfqpoint{3.319791in}{2.901071in}}%
\pgfusepath{stroke}%
\end{pgfscope}%
\begin{pgfscope}%
\pgfpathrectangle{\pgfqpoint{0.765000in}{0.660000in}}{\pgfqpoint{4.620000in}{4.620000in}}%
\pgfusepath{clip}%
\pgfsetbuttcap%
\pgfsetmiterjoin%
\definecolor{currentfill}{rgb}{0.176471,0.192157,0.258824}%
\pgfsetfillcolor{currentfill}%
\pgfsetlinewidth{1.003750pt}%
\definecolor{currentstroke}{rgb}{0.176471,0.192157,0.258824}%
\pgfsetstrokecolor{currentstroke}%
\pgfsetdash{}{0pt}%
\pgfsys@defobject{currentmarker}{\pgfqpoint{-0.033333in}{-0.033333in}}{\pgfqpoint{0.033333in}{0.033333in}}{%
\pgfpathmoveto{\pgfqpoint{-0.000000in}{-0.033333in}}%
\pgfpathlineto{\pgfqpoint{0.033333in}{0.033333in}}%
\pgfpathlineto{\pgfqpoint{-0.033333in}{0.033333in}}%
\pgfpathlineto{\pgfqpoint{-0.000000in}{-0.033333in}}%
\pgfpathclose%
\pgfusepath{stroke,fill}%
}%
\begin{pgfscope}%
\pgfsys@transformshift{3.319791in}{2.901071in}%
\pgfsys@useobject{currentmarker}{}%
\end{pgfscope}%
\end{pgfscope}%
\begin{pgfscope}%
\pgfpathrectangle{\pgfqpoint{0.765000in}{0.660000in}}{\pgfqpoint{4.620000in}{4.620000in}}%
\pgfusepath{clip}%
\pgfsetroundcap%
\pgfsetroundjoin%
\pgfsetlinewidth{1.204500pt}%
\definecolor{currentstroke}{rgb}{0.176471,0.192157,0.258824}%
\pgfsetstrokecolor{currentstroke}%
\pgfsetdash{}{0pt}%
\pgfpathmoveto{\pgfqpoint{3.240997in}{2.920660in}}%
\pgfusepath{stroke}%
\end{pgfscope}%
\begin{pgfscope}%
\pgfpathrectangle{\pgfqpoint{0.765000in}{0.660000in}}{\pgfqpoint{4.620000in}{4.620000in}}%
\pgfusepath{clip}%
\pgfsetbuttcap%
\pgfsetmiterjoin%
\definecolor{currentfill}{rgb}{0.176471,0.192157,0.258824}%
\pgfsetfillcolor{currentfill}%
\pgfsetlinewidth{1.003750pt}%
\definecolor{currentstroke}{rgb}{0.176471,0.192157,0.258824}%
\pgfsetstrokecolor{currentstroke}%
\pgfsetdash{}{0pt}%
\pgfsys@defobject{currentmarker}{\pgfqpoint{-0.033333in}{-0.033333in}}{\pgfqpoint{0.033333in}{0.033333in}}{%
\pgfpathmoveto{\pgfqpoint{-0.000000in}{-0.033333in}}%
\pgfpathlineto{\pgfqpoint{0.033333in}{0.033333in}}%
\pgfpathlineto{\pgfqpoint{-0.033333in}{0.033333in}}%
\pgfpathlineto{\pgfqpoint{-0.000000in}{-0.033333in}}%
\pgfpathclose%
\pgfusepath{stroke,fill}%
}%
\begin{pgfscope}%
\pgfsys@transformshift{3.240997in}{2.920660in}%
\pgfsys@useobject{currentmarker}{}%
\end{pgfscope}%
\end{pgfscope}%
\begin{pgfscope}%
\pgfpathrectangle{\pgfqpoint{0.765000in}{0.660000in}}{\pgfqpoint{4.620000in}{4.620000in}}%
\pgfusepath{clip}%
\pgfsetroundcap%
\pgfsetroundjoin%
\pgfsetlinewidth{1.204500pt}%
\definecolor{currentstroke}{rgb}{0.176471,0.192157,0.258824}%
\pgfsetstrokecolor{currentstroke}%
\pgfsetdash{}{0pt}%
\pgfpathmoveto{\pgfqpoint{3.208686in}{3.083456in}}%
\pgfusepath{stroke}%
\end{pgfscope}%
\begin{pgfscope}%
\pgfpathrectangle{\pgfqpoint{0.765000in}{0.660000in}}{\pgfqpoint{4.620000in}{4.620000in}}%
\pgfusepath{clip}%
\pgfsetbuttcap%
\pgfsetmiterjoin%
\definecolor{currentfill}{rgb}{0.176471,0.192157,0.258824}%
\pgfsetfillcolor{currentfill}%
\pgfsetlinewidth{1.003750pt}%
\definecolor{currentstroke}{rgb}{0.176471,0.192157,0.258824}%
\pgfsetstrokecolor{currentstroke}%
\pgfsetdash{}{0pt}%
\pgfsys@defobject{currentmarker}{\pgfqpoint{-0.033333in}{-0.033333in}}{\pgfqpoint{0.033333in}{0.033333in}}{%
\pgfpathmoveto{\pgfqpoint{-0.000000in}{-0.033333in}}%
\pgfpathlineto{\pgfqpoint{0.033333in}{0.033333in}}%
\pgfpathlineto{\pgfqpoint{-0.033333in}{0.033333in}}%
\pgfpathlineto{\pgfqpoint{-0.000000in}{-0.033333in}}%
\pgfpathclose%
\pgfusepath{stroke,fill}%
}%
\begin{pgfscope}%
\pgfsys@transformshift{3.208686in}{3.083456in}%
\pgfsys@useobject{currentmarker}{}%
\end{pgfscope}%
\end{pgfscope}%
\begin{pgfscope}%
\pgfpathrectangle{\pgfqpoint{0.765000in}{0.660000in}}{\pgfqpoint{4.620000in}{4.620000in}}%
\pgfusepath{clip}%
\pgfsetroundcap%
\pgfsetroundjoin%
\pgfsetlinewidth{1.204500pt}%
\definecolor{currentstroke}{rgb}{0.000000,0.000000,0.000000}%
\pgfsetstrokecolor{currentstroke}%
\pgfsetdash{}{0pt}%
\pgfpathmoveto{\pgfqpoint{3.018309in}{2.654891in}}%
\pgfusepath{stroke}%
\end{pgfscope}%
\begin{pgfscope}%
\pgfpathrectangle{\pgfqpoint{0.765000in}{0.660000in}}{\pgfqpoint{4.620000in}{4.620000in}}%
\pgfusepath{clip}%
\pgfsetbuttcap%
\pgfsetroundjoin%
\definecolor{currentfill}{rgb}{0.000000,0.000000,0.000000}%
\pgfsetfillcolor{currentfill}%
\pgfsetlinewidth{1.003750pt}%
\definecolor{currentstroke}{rgb}{0.000000,0.000000,0.000000}%
\pgfsetstrokecolor{currentstroke}%
\pgfsetdash{}{0pt}%
\pgfsys@defobject{currentmarker}{\pgfqpoint{-0.016667in}{-0.016667in}}{\pgfqpoint{0.016667in}{0.016667in}}{%
\pgfpathmoveto{\pgfqpoint{0.000000in}{-0.016667in}}%
\pgfpathcurveto{\pgfqpoint{0.004420in}{-0.016667in}}{\pgfqpoint{0.008660in}{-0.014911in}}{\pgfqpoint{0.011785in}{-0.011785in}}%
\pgfpathcurveto{\pgfqpoint{0.014911in}{-0.008660in}}{\pgfqpoint{0.016667in}{-0.004420in}}{\pgfqpoint{0.016667in}{0.000000in}}%
\pgfpathcurveto{\pgfqpoint{0.016667in}{0.004420in}}{\pgfqpoint{0.014911in}{0.008660in}}{\pgfqpoint{0.011785in}{0.011785in}}%
\pgfpathcurveto{\pgfqpoint{0.008660in}{0.014911in}}{\pgfqpoint{0.004420in}{0.016667in}}{\pgfqpoint{0.000000in}{0.016667in}}%
\pgfpathcurveto{\pgfqpoint{-0.004420in}{0.016667in}}{\pgfqpoint{-0.008660in}{0.014911in}}{\pgfqpoint{-0.011785in}{0.011785in}}%
\pgfpathcurveto{\pgfqpoint{-0.014911in}{0.008660in}}{\pgfqpoint{-0.016667in}{0.004420in}}{\pgfqpoint{-0.016667in}{0.000000in}}%
\pgfpathcurveto{\pgfqpoint{-0.016667in}{-0.004420in}}{\pgfqpoint{-0.014911in}{-0.008660in}}{\pgfqpoint{-0.011785in}{-0.011785in}}%
\pgfpathcurveto{\pgfqpoint{-0.008660in}{-0.014911in}}{\pgfqpoint{-0.004420in}{-0.016667in}}{\pgfqpoint{0.000000in}{-0.016667in}}%
\pgfpathlineto{\pgfqpoint{0.000000in}{-0.016667in}}%
\pgfpathclose%
\pgfusepath{stroke,fill}%
}%
\begin{pgfscope}%
\pgfsys@transformshift{3.018309in}{2.654891in}%
\pgfsys@useobject{currentmarker}{}%
\end{pgfscope}%
\end{pgfscope}%
\begin{pgfscope}%
\pgfpathrectangle{\pgfqpoint{0.765000in}{0.660000in}}{\pgfqpoint{4.620000in}{4.620000in}}%
\pgfusepath{clip}%
\pgfsetroundcap%
\pgfsetroundjoin%
\pgfsetlinewidth{1.204500pt}%
\definecolor{currentstroke}{rgb}{0.000000,0.000000,0.000000}%
\pgfsetstrokecolor{currentstroke}%
\pgfsetdash{}{0pt}%
\pgfpathmoveto{\pgfqpoint{3.018309in}{2.654891in}}%
\pgfpathlineto{\pgfqpoint{3.230152in}{2.555609in}}%
\pgfusepath{stroke}%
\end{pgfscope}%
\begin{pgfscope}%
\pgfpathrectangle{\pgfqpoint{0.765000in}{0.660000in}}{\pgfqpoint{4.620000in}{4.620000in}}%
\pgfusepath{clip}%
\pgfsetroundcap%
\pgfsetroundjoin%
\pgfsetlinewidth{1.204500pt}%
\definecolor{currentstroke}{rgb}{0.000000,0.000000,0.000000}%
\pgfsetstrokecolor{currentstroke}%
\pgfsetdash{}{0pt}%
\pgfpathmoveto{\pgfqpoint{3.018309in}{2.654891in}}%
\pgfpathlineto{\pgfqpoint{2.779562in}{2.990565in}}%
\pgfusepath{stroke}%
\end{pgfscope}%
\begin{pgfscope}%
\pgfpathrectangle{\pgfqpoint{0.765000in}{0.660000in}}{\pgfqpoint{4.620000in}{4.620000in}}%
\pgfusepath{clip}%
\pgfsetbuttcap%
\pgfsetroundjoin%
\pgfsetlinewidth{1.204500pt}%
\definecolor{currentstroke}{rgb}{0.827451,0.827451,0.827451}%
\pgfsetstrokecolor{currentstroke}%
\pgfsetdash{{1.200000pt}{1.980000pt}}{0.000000pt}%
\pgfpathmoveto{\pgfqpoint{3.018309in}{2.654891in}}%
\pgfpathlineto{\pgfqpoint{3.018309in}{0.650000in}}%
\pgfusepath{stroke}%
\end{pgfscope}%
\begin{pgfscope}%
\pgfpathrectangle{\pgfqpoint{0.765000in}{0.660000in}}{\pgfqpoint{4.620000in}{4.620000in}}%
\pgfusepath{clip}%
\pgfsetroundcap%
\pgfsetroundjoin%
\pgfsetlinewidth{1.204500pt}%
\definecolor{currentstroke}{rgb}{0.000000,0.000000,0.000000}%
\pgfsetstrokecolor{currentstroke}%
\pgfsetdash{}{0pt}%
\pgfpathmoveto{\pgfqpoint{3.230152in}{2.555609in}}%
\pgfusepath{stroke}%
\end{pgfscope}%
\begin{pgfscope}%
\pgfpathrectangle{\pgfqpoint{0.765000in}{0.660000in}}{\pgfqpoint{4.620000in}{4.620000in}}%
\pgfusepath{clip}%
\pgfsetbuttcap%
\pgfsetroundjoin%
\definecolor{currentfill}{rgb}{0.000000,0.000000,0.000000}%
\pgfsetfillcolor{currentfill}%
\pgfsetlinewidth{1.003750pt}%
\definecolor{currentstroke}{rgb}{0.000000,0.000000,0.000000}%
\pgfsetstrokecolor{currentstroke}%
\pgfsetdash{}{0pt}%
\pgfsys@defobject{currentmarker}{\pgfqpoint{-0.016667in}{-0.016667in}}{\pgfqpoint{0.016667in}{0.016667in}}{%
\pgfpathmoveto{\pgfqpoint{0.000000in}{-0.016667in}}%
\pgfpathcurveto{\pgfqpoint{0.004420in}{-0.016667in}}{\pgfqpoint{0.008660in}{-0.014911in}}{\pgfqpoint{0.011785in}{-0.011785in}}%
\pgfpathcurveto{\pgfqpoint{0.014911in}{-0.008660in}}{\pgfqpoint{0.016667in}{-0.004420in}}{\pgfqpoint{0.016667in}{0.000000in}}%
\pgfpathcurveto{\pgfqpoint{0.016667in}{0.004420in}}{\pgfqpoint{0.014911in}{0.008660in}}{\pgfqpoint{0.011785in}{0.011785in}}%
\pgfpathcurveto{\pgfqpoint{0.008660in}{0.014911in}}{\pgfqpoint{0.004420in}{0.016667in}}{\pgfqpoint{0.000000in}{0.016667in}}%
\pgfpathcurveto{\pgfqpoint{-0.004420in}{0.016667in}}{\pgfqpoint{-0.008660in}{0.014911in}}{\pgfqpoint{-0.011785in}{0.011785in}}%
\pgfpathcurveto{\pgfqpoint{-0.014911in}{0.008660in}}{\pgfqpoint{-0.016667in}{0.004420in}}{\pgfqpoint{-0.016667in}{0.000000in}}%
\pgfpathcurveto{\pgfqpoint{-0.016667in}{-0.004420in}}{\pgfqpoint{-0.014911in}{-0.008660in}}{\pgfqpoint{-0.011785in}{-0.011785in}}%
\pgfpathcurveto{\pgfqpoint{-0.008660in}{-0.014911in}}{\pgfqpoint{-0.004420in}{-0.016667in}}{\pgfqpoint{0.000000in}{-0.016667in}}%
\pgfpathlineto{\pgfqpoint{0.000000in}{-0.016667in}}%
\pgfpathclose%
\pgfusepath{stroke,fill}%
}%
\begin{pgfscope}%
\pgfsys@transformshift{3.230152in}{2.555609in}%
\pgfsys@useobject{currentmarker}{}%
\end{pgfscope}%
\end{pgfscope}%
\begin{pgfscope}%
\pgfpathrectangle{\pgfqpoint{0.765000in}{0.660000in}}{\pgfqpoint{4.620000in}{4.620000in}}%
\pgfusepath{clip}%
\pgfsetroundcap%
\pgfsetroundjoin%
\pgfsetlinewidth{1.204500pt}%
\definecolor{currentstroke}{rgb}{0.000000,0.000000,0.000000}%
\pgfsetstrokecolor{currentstroke}%
\pgfsetdash{}{0pt}%
\pgfpathmoveto{\pgfqpoint{3.230152in}{2.555609in}}%
\pgfpathlineto{\pgfqpoint{3.018309in}{2.654891in}}%
\pgfusepath{stroke}%
\end{pgfscope}%
\begin{pgfscope}%
\pgfpathrectangle{\pgfqpoint{0.765000in}{0.660000in}}{\pgfqpoint{4.620000in}{4.620000in}}%
\pgfusepath{clip}%
\pgfsetroundcap%
\pgfsetroundjoin%
\pgfsetlinewidth{1.204500pt}%
\definecolor{currentstroke}{rgb}{0.000000,0.000000,0.000000}%
\pgfsetstrokecolor{currentstroke}%
\pgfsetdash{}{0pt}%
\pgfpathmoveto{\pgfqpoint{3.230152in}{2.555609in}}%
\pgfpathlineto{\pgfqpoint{3.623268in}{3.108321in}}%
\pgfusepath{stroke}%
\end{pgfscope}%
\begin{pgfscope}%
\pgfpathrectangle{\pgfqpoint{0.765000in}{0.660000in}}{\pgfqpoint{4.620000in}{4.620000in}}%
\pgfusepath{clip}%
\pgfsetbuttcap%
\pgfsetroundjoin%
\pgfsetlinewidth{1.204500pt}%
\definecolor{currentstroke}{rgb}{0.827451,0.827451,0.827451}%
\pgfsetstrokecolor{currentstroke}%
\pgfsetdash{{1.200000pt}{1.980000pt}}{0.000000pt}%
\pgfpathmoveto{\pgfqpoint{3.230152in}{2.555609in}}%
\pgfpathlineto{\pgfqpoint{3.230152in}{0.650000in}}%
\pgfusepath{stroke}%
\end{pgfscope}%
\begin{pgfscope}%
\pgfpathrectangle{\pgfqpoint{0.765000in}{0.660000in}}{\pgfqpoint{4.620000in}{4.620000in}}%
\pgfusepath{clip}%
\pgfsetroundcap%
\pgfsetroundjoin%
\pgfsetlinewidth{1.204500pt}%
\definecolor{currentstroke}{rgb}{0.000000,0.000000,0.000000}%
\pgfsetstrokecolor{currentstroke}%
\pgfsetdash{}{0pt}%
\pgfpathmoveto{\pgfqpoint{2.779562in}{3.139137in}}%
\pgfusepath{stroke}%
\end{pgfscope}%
\begin{pgfscope}%
\pgfpathrectangle{\pgfqpoint{0.765000in}{0.660000in}}{\pgfqpoint{4.620000in}{4.620000in}}%
\pgfusepath{clip}%
\pgfsetbuttcap%
\pgfsetroundjoin%
\definecolor{currentfill}{rgb}{0.000000,0.000000,0.000000}%
\pgfsetfillcolor{currentfill}%
\pgfsetlinewidth{1.003750pt}%
\definecolor{currentstroke}{rgb}{0.000000,0.000000,0.000000}%
\pgfsetstrokecolor{currentstroke}%
\pgfsetdash{}{0pt}%
\pgfsys@defobject{currentmarker}{\pgfqpoint{-0.016667in}{-0.016667in}}{\pgfqpoint{0.016667in}{0.016667in}}{%
\pgfpathmoveto{\pgfqpoint{0.000000in}{-0.016667in}}%
\pgfpathcurveto{\pgfqpoint{0.004420in}{-0.016667in}}{\pgfqpoint{0.008660in}{-0.014911in}}{\pgfqpoint{0.011785in}{-0.011785in}}%
\pgfpathcurveto{\pgfqpoint{0.014911in}{-0.008660in}}{\pgfqpoint{0.016667in}{-0.004420in}}{\pgfqpoint{0.016667in}{0.000000in}}%
\pgfpathcurveto{\pgfqpoint{0.016667in}{0.004420in}}{\pgfqpoint{0.014911in}{0.008660in}}{\pgfqpoint{0.011785in}{0.011785in}}%
\pgfpathcurveto{\pgfqpoint{0.008660in}{0.014911in}}{\pgfqpoint{0.004420in}{0.016667in}}{\pgfqpoint{0.000000in}{0.016667in}}%
\pgfpathcurveto{\pgfqpoint{-0.004420in}{0.016667in}}{\pgfqpoint{-0.008660in}{0.014911in}}{\pgfqpoint{-0.011785in}{0.011785in}}%
\pgfpathcurveto{\pgfqpoint{-0.014911in}{0.008660in}}{\pgfqpoint{-0.016667in}{0.004420in}}{\pgfqpoint{-0.016667in}{0.000000in}}%
\pgfpathcurveto{\pgfqpoint{-0.016667in}{-0.004420in}}{\pgfqpoint{-0.014911in}{-0.008660in}}{\pgfqpoint{-0.011785in}{-0.011785in}}%
\pgfpathcurveto{\pgfqpoint{-0.008660in}{-0.014911in}}{\pgfqpoint{-0.004420in}{-0.016667in}}{\pgfqpoint{0.000000in}{-0.016667in}}%
\pgfpathlineto{\pgfqpoint{0.000000in}{-0.016667in}}%
\pgfpathclose%
\pgfusepath{stroke,fill}%
}%
\begin{pgfscope}%
\pgfsys@transformshift{2.779562in}{3.139137in}%
\pgfsys@useobject{currentmarker}{}%
\end{pgfscope}%
\end{pgfscope}%
\begin{pgfscope}%
\pgfpathrectangle{\pgfqpoint{0.765000in}{0.660000in}}{\pgfqpoint{4.620000in}{4.620000in}}%
\pgfusepath{clip}%
\pgfsetroundcap%
\pgfsetroundjoin%
\pgfsetlinewidth{1.204500pt}%
\definecolor{currentstroke}{rgb}{0.000000,0.000000,0.000000}%
\pgfsetstrokecolor{currentstroke}%
\pgfsetdash{}{0pt}%
\pgfpathmoveto{\pgfqpoint{2.779562in}{3.139137in}}%
\pgfpathlineto{\pgfqpoint{2.779562in}{2.990565in}}%
\pgfusepath{stroke}%
\end{pgfscope}%
\begin{pgfscope}%
\pgfpathrectangle{\pgfqpoint{0.765000in}{0.660000in}}{\pgfqpoint{4.620000in}{4.620000in}}%
\pgfusepath{clip}%
\pgfsetroundcap%
\pgfsetroundjoin%
\pgfsetlinewidth{1.204500pt}%
\definecolor{currentstroke}{rgb}{0.000000,0.000000,0.000000}%
\pgfsetstrokecolor{currentstroke}%
\pgfsetdash{}{0pt}%
\pgfpathmoveto{\pgfqpoint{2.779562in}{3.139137in}}%
\pgfpathlineto{\pgfqpoint{2.894524in}{3.193015in}}%
\pgfusepath{stroke}%
\end{pgfscope}%
\begin{pgfscope}%
\pgfpathrectangle{\pgfqpoint{0.765000in}{0.660000in}}{\pgfqpoint{4.620000in}{4.620000in}}%
\pgfusepath{clip}%
\pgfsetbuttcap%
\pgfsetroundjoin%
\pgfsetlinewidth{1.204500pt}%
\definecolor{currentstroke}{rgb}{0.827451,0.827451,0.827451}%
\pgfsetstrokecolor{currentstroke}%
\pgfsetdash{{1.200000pt}{1.980000pt}}{0.000000pt}%
\pgfpathmoveto{\pgfqpoint{2.779562in}{3.139137in}}%
\pgfpathlineto{\pgfqpoint{0.755000in}{4.087967in}}%
\pgfusepath{stroke}%
\end{pgfscope}%
\begin{pgfscope}%
\pgfpathrectangle{\pgfqpoint{0.765000in}{0.660000in}}{\pgfqpoint{4.620000in}{4.620000in}}%
\pgfusepath{clip}%
\pgfsetroundcap%
\pgfsetroundjoin%
\pgfsetlinewidth{1.204500pt}%
\definecolor{currentstroke}{rgb}{0.000000,0.000000,0.000000}%
\pgfsetstrokecolor{currentstroke}%
\pgfsetdash{}{0pt}%
\pgfpathmoveto{\pgfqpoint{2.779562in}{2.990565in}}%
\pgfusepath{stroke}%
\end{pgfscope}%
\begin{pgfscope}%
\pgfpathrectangle{\pgfqpoint{0.765000in}{0.660000in}}{\pgfqpoint{4.620000in}{4.620000in}}%
\pgfusepath{clip}%
\pgfsetbuttcap%
\pgfsetroundjoin%
\definecolor{currentfill}{rgb}{0.000000,0.000000,0.000000}%
\pgfsetfillcolor{currentfill}%
\pgfsetlinewidth{1.003750pt}%
\definecolor{currentstroke}{rgb}{0.000000,0.000000,0.000000}%
\pgfsetstrokecolor{currentstroke}%
\pgfsetdash{}{0pt}%
\pgfsys@defobject{currentmarker}{\pgfqpoint{-0.016667in}{-0.016667in}}{\pgfqpoint{0.016667in}{0.016667in}}{%
\pgfpathmoveto{\pgfqpoint{0.000000in}{-0.016667in}}%
\pgfpathcurveto{\pgfqpoint{0.004420in}{-0.016667in}}{\pgfqpoint{0.008660in}{-0.014911in}}{\pgfqpoint{0.011785in}{-0.011785in}}%
\pgfpathcurveto{\pgfqpoint{0.014911in}{-0.008660in}}{\pgfqpoint{0.016667in}{-0.004420in}}{\pgfqpoint{0.016667in}{0.000000in}}%
\pgfpathcurveto{\pgfqpoint{0.016667in}{0.004420in}}{\pgfqpoint{0.014911in}{0.008660in}}{\pgfqpoint{0.011785in}{0.011785in}}%
\pgfpathcurveto{\pgfqpoint{0.008660in}{0.014911in}}{\pgfqpoint{0.004420in}{0.016667in}}{\pgfqpoint{0.000000in}{0.016667in}}%
\pgfpathcurveto{\pgfqpoint{-0.004420in}{0.016667in}}{\pgfqpoint{-0.008660in}{0.014911in}}{\pgfqpoint{-0.011785in}{0.011785in}}%
\pgfpathcurveto{\pgfqpoint{-0.014911in}{0.008660in}}{\pgfqpoint{-0.016667in}{0.004420in}}{\pgfqpoint{-0.016667in}{0.000000in}}%
\pgfpathcurveto{\pgfqpoint{-0.016667in}{-0.004420in}}{\pgfqpoint{-0.014911in}{-0.008660in}}{\pgfqpoint{-0.011785in}{-0.011785in}}%
\pgfpathcurveto{\pgfqpoint{-0.008660in}{-0.014911in}}{\pgfqpoint{-0.004420in}{-0.016667in}}{\pgfqpoint{0.000000in}{-0.016667in}}%
\pgfpathlineto{\pgfqpoint{0.000000in}{-0.016667in}}%
\pgfpathclose%
\pgfusepath{stroke,fill}%
}%
\begin{pgfscope}%
\pgfsys@transformshift{2.779562in}{2.990565in}%
\pgfsys@useobject{currentmarker}{}%
\end{pgfscope}%
\end{pgfscope}%
\begin{pgfscope}%
\pgfpathrectangle{\pgfqpoint{0.765000in}{0.660000in}}{\pgfqpoint{4.620000in}{4.620000in}}%
\pgfusepath{clip}%
\pgfsetroundcap%
\pgfsetroundjoin%
\pgfsetlinewidth{1.204500pt}%
\definecolor{currentstroke}{rgb}{0.000000,0.000000,0.000000}%
\pgfsetstrokecolor{currentstroke}%
\pgfsetdash{}{0pt}%
\pgfpathmoveto{\pgfqpoint{2.779562in}{2.990565in}}%
\pgfpathlineto{\pgfqpoint{3.018309in}{2.654891in}}%
\pgfusepath{stroke}%
\end{pgfscope}%
\begin{pgfscope}%
\pgfpathrectangle{\pgfqpoint{0.765000in}{0.660000in}}{\pgfqpoint{4.620000in}{4.620000in}}%
\pgfusepath{clip}%
\pgfsetroundcap%
\pgfsetroundjoin%
\pgfsetlinewidth{1.204500pt}%
\definecolor{currentstroke}{rgb}{0.000000,0.000000,0.000000}%
\pgfsetstrokecolor{currentstroke}%
\pgfsetdash{}{0pt}%
\pgfpathmoveto{\pgfqpoint{2.779562in}{2.990565in}}%
\pgfpathlineto{\pgfqpoint{2.779562in}{3.139137in}}%
\pgfusepath{stroke}%
\end{pgfscope}%
\begin{pgfscope}%
\pgfpathrectangle{\pgfqpoint{0.765000in}{0.660000in}}{\pgfqpoint{4.620000in}{4.620000in}}%
\pgfusepath{clip}%
\pgfsetbuttcap%
\pgfsetroundjoin%
\pgfsetlinewidth{1.204500pt}%
\definecolor{currentstroke}{rgb}{0.827451,0.827451,0.827451}%
\pgfsetstrokecolor{currentstroke}%
\pgfsetdash{{1.200000pt}{1.980000pt}}{0.000000pt}%
\pgfpathmoveto{\pgfqpoint{2.779562in}{2.990565in}}%
\pgfpathlineto{\pgfqpoint{0.755000in}{3.939395in}}%
\pgfusepath{stroke}%
\end{pgfscope}%
\begin{pgfscope}%
\pgfpathrectangle{\pgfqpoint{0.765000in}{0.660000in}}{\pgfqpoint{4.620000in}{4.620000in}}%
\pgfusepath{clip}%
\pgfsetroundcap%
\pgfsetroundjoin%
\pgfsetlinewidth{1.204500pt}%
\definecolor{currentstroke}{rgb}{0.000000,0.000000,0.000000}%
\pgfsetstrokecolor{currentstroke}%
\pgfsetdash{}{0pt}%
\pgfpathmoveto{\pgfqpoint{2.894524in}{3.193015in}}%
\pgfusepath{stroke}%
\end{pgfscope}%
\begin{pgfscope}%
\pgfpathrectangle{\pgfqpoint{0.765000in}{0.660000in}}{\pgfqpoint{4.620000in}{4.620000in}}%
\pgfusepath{clip}%
\pgfsetbuttcap%
\pgfsetroundjoin%
\definecolor{currentfill}{rgb}{0.000000,0.000000,0.000000}%
\pgfsetfillcolor{currentfill}%
\pgfsetlinewidth{1.003750pt}%
\definecolor{currentstroke}{rgb}{0.000000,0.000000,0.000000}%
\pgfsetstrokecolor{currentstroke}%
\pgfsetdash{}{0pt}%
\pgfsys@defobject{currentmarker}{\pgfqpoint{-0.016667in}{-0.016667in}}{\pgfqpoint{0.016667in}{0.016667in}}{%
\pgfpathmoveto{\pgfqpoint{0.000000in}{-0.016667in}}%
\pgfpathcurveto{\pgfqpoint{0.004420in}{-0.016667in}}{\pgfqpoint{0.008660in}{-0.014911in}}{\pgfqpoint{0.011785in}{-0.011785in}}%
\pgfpathcurveto{\pgfqpoint{0.014911in}{-0.008660in}}{\pgfqpoint{0.016667in}{-0.004420in}}{\pgfqpoint{0.016667in}{0.000000in}}%
\pgfpathcurveto{\pgfqpoint{0.016667in}{0.004420in}}{\pgfqpoint{0.014911in}{0.008660in}}{\pgfqpoint{0.011785in}{0.011785in}}%
\pgfpathcurveto{\pgfqpoint{0.008660in}{0.014911in}}{\pgfqpoint{0.004420in}{0.016667in}}{\pgfqpoint{0.000000in}{0.016667in}}%
\pgfpathcurveto{\pgfqpoint{-0.004420in}{0.016667in}}{\pgfqpoint{-0.008660in}{0.014911in}}{\pgfqpoint{-0.011785in}{0.011785in}}%
\pgfpathcurveto{\pgfqpoint{-0.014911in}{0.008660in}}{\pgfqpoint{-0.016667in}{0.004420in}}{\pgfqpoint{-0.016667in}{0.000000in}}%
\pgfpathcurveto{\pgfqpoint{-0.016667in}{-0.004420in}}{\pgfqpoint{-0.014911in}{-0.008660in}}{\pgfqpoint{-0.011785in}{-0.011785in}}%
\pgfpathcurveto{\pgfqpoint{-0.008660in}{-0.014911in}}{\pgfqpoint{-0.004420in}{-0.016667in}}{\pgfqpoint{0.000000in}{-0.016667in}}%
\pgfpathlineto{\pgfqpoint{0.000000in}{-0.016667in}}%
\pgfpathclose%
\pgfusepath{stroke,fill}%
}%
\begin{pgfscope}%
\pgfsys@transformshift{2.894524in}{3.193015in}%
\pgfsys@useobject{currentmarker}{}%
\end{pgfscope}%
\end{pgfscope}%
\begin{pgfscope}%
\pgfpathrectangle{\pgfqpoint{0.765000in}{0.660000in}}{\pgfqpoint{4.620000in}{4.620000in}}%
\pgfusepath{clip}%
\pgfsetroundcap%
\pgfsetroundjoin%
\pgfsetlinewidth{1.204500pt}%
\definecolor{currentstroke}{rgb}{0.000000,0.000000,0.000000}%
\pgfsetstrokecolor{currentstroke}%
\pgfsetdash{}{0pt}%
\pgfpathmoveto{\pgfqpoint{2.894524in}{3.193015in}}%
\pgfpathlineto{\pgfqpoint{2.779562in}{3.139137in}}%
\pgfusepath{stroke}%
\end{pgfscope}%
\begin{pgfscope}%
\pgfpathrectangle{\pgfqpoint{0.765000in}{0.660000in}}{\pgfqpoint{4.620000in}{4.620000in}}%
\pgfusepath{clip}%
\pgfsetroundcap%
\pgfsetroundjoin%
\pgfsetlinewidth{1.204500pt}%
\definecolor{currentstroke}{rgb}{0.000000,0.000000,0.000000}%
\pgfsetstrokecolor{currentstroke}%
\pgfsetdash{}{0pt}%
\pgfpathmoveto{\pgfqpoint{2.894524in}{3.193015in}}%
\pgfpathlineto{\pgfqpoint{3.442553in}{3.193015in}}%
\pgfusepath{stroke}%
\end{pgfscope}%
\begin{pgfscope}%
\pgfpathrectangle{\pgfqpoint{0.765000in}{0.660000in}}{\pgfqpoint{4.620000in}{4.620000in}}%
\pgfusepath{clip}%
\pgfsetbuttcap%
\pgfsetroundjoin%
\pgfsetlinewidth{1.204500pt}%
\definecolor{currentstroke}{rgb}{0.827451,0.827451,0.827451}%
\pgfsetstrokecolor{currentstroke}%
\pgfsetdash{{1.200000pt}{1.980000pt}}{0.000000pt}%
\pgfpathmoveto{\pgfqpoint{2.894524in}{3.193015in}}%
\pgfpathlineto{\pgfqpoint{0.755000in}{4.195724in}}%
\pgfusepath{stroke}%
\end{pgfscope}%
\begin{pgfscope}%
\pgfpathrectangle{\pgfqpoint{0.765000in}{0.660000in}}{\pgfqpoint{4.620000in}{4.620000in}}%
\pgfusepath{clip}%
\pgfsetroundcap%
\pgfsetroundjoin%
\pgfsetlinewidth{1.204500pt}%
\definecolor{currentstroke}{rgb}{0.000000,0.000000,0.000000}%
\pgfsetstrokecolor{currentstroke}%
\pgfsetdash{}{0pt}%
\pgfpathmoveto{\pgfqpoint{3.623268in}{3.108321in}}%
\pgfusepath{stroke}%
\end{pgfscope}%
\begin{pgfscope}%
\pgfpathrectangle{\pgfqpoint{0.765000in}{0.660000in}}{\pgfqpoint{4.620000in}{4.620000in}}%
\pgfusepath{clip}%
\pgfsetbuttcap%
\pgfsetroundjoin%
\definecolor{currentfill}{rgb}{0.000000,0.000000,0.000000}%
\pgfsetfillcolor{currentfill}%
\pgfsetlinewidth{1.003750pt}%
\definecolor{currentstroke}{rgb}{0.000000,0.000000,0.000000}%
\pgfsetstrokecolor{currentstroke}%
\pgfsetdash{}{0pt}%
\pgfsys@defobject{currentmarker}{\pgfqpoint{-0.016667in}{-0.016667in}}{\pgfqpoint{0.016667in}{0.016667in}}{%
\pgfpathmoveto{\pgfqpoint{0.000000in}{-0.016667in}}%
\pgfpathcurveto{\pgfqpoint{0.004420in}{-0.016667in}}{\pgfqpoint{0.008660in}{-0.014911in}}{\pgfqpoint{0.011785in}{-0.011785in}}%
\pgfpathcurveto{\pgfqpoint{0.014911in}{-0.008660in}}{\pgfqpoint{0.016667in}{-0.004420in}}{\pgfqpoint{0.016667in}{0.000000in}}%
\pgfpathcurveto{\pgfqpoint{0.016667in}{0.004420in}}{\pgfqpoint{0.014911in}{0.008660in}}{\pgfqpoint{0.011785in}{0.011785in}}%
\pgfpathcurveto{\pgfqpoint{0.008660in}{0.014911in}}{\pgfqpoint{0.004420in}{0.016667in}}{\pgfqpoint{0.000000in}{0.016667in}}%
\pgfpathcurveto{\pgfqpoint{-0.004420in}{0.016667in}}{\pgfqpoint{-0.008660in}{0.014911in}}{\pgfqpoint{-0.011785in}{0.011785in}}%
\pgfpathcurveto{\pgfqpoint{-0.014911in}{0.008660in}}{\pgfqpoint{-0.016667in}{0.004420in}}{\pgfqpoint{-0.016667in}{0.000000in}}%
\pgfpathcurveto{\pgfqpoint{-0.016667in}{-0.004420in}}{\pgfqpoint{-0.014911in}{-0.008660in}}{\pgfqpoint{-0.011785in}{-0.011785in}}%
\pgfpathcurveto{\pgfqpoint{-0.008660in}{-0.014911in}}{\pgfqpoint{-0.004420in}{-0.016667in}}{\pgfqpoint{0.000000in}{-0.016667in}}%
\pgfpathlineto{\pgfqpoint{0.000000in}{-0.016667in}}%
\pgfpathclose%
\pgfusepath{stroke,fill}%
}%
\begin{pgfscope}%
\pgfsys@transformshift{3.623268in}{3.108321in}%
\pgfsys@useobject{currentmarker}{}%
\end{pgfscope}%
\end{pgfscope}%
\begin{pgfscope}%
\pgfpathrectangle{\pgfqpoint{0.765000in}{0.660000in}}{\pgfqpoint{4.620000in}{4.620000in}}%
\pgfusepath{clip}%
\pgfsetroundcap%
\pgfsetroundjoin%
\pgfsetlinewidth{1.204500pt}%
\definecolor{currentstroke}{rgb}{0.000000,0.000000,0.000000}%
\pgfsetstrokecolor{currentstroke}%
\pgfsetdash{}{0pt}%
\pgfpathmoveto{\pgfqpoint{3.623268in}{3.108321in}}%
\pgfpathlineto{\pgfqpoint{3.230152in}{2.555609in}}%
\pgfusepath{stroke}%
\end{pgfscope}%
\begin{pgfscope}%
\pgfpathrectangle{\pgfqpoint{0.765000in}{0.660000in}}{\pgfqpoint{4.620000in}{4.620000in}}%
\pgfusepath{clip}%
\pgfsetroundcap%
\pgfsetroundjoin%
\pgfsetlinewidth{1.204500pt}%
\definecolor{currentstroke}{rgb}{0.000000,0.000000,0.000000}%
\pgfsetstrokecolor{currentstroke}%
\pgfsetdash{}{0pt}%
\pgfpathmoveto{\pgfqpoint{3.623268in}{3.108321in}}%
\pgfpathlineto{\pgfqpoint{3.442553in}{3.193015in}}%
\pgfusepath{stroke}%
\end{pgfscope}%
\begin{pgfscope}%
\pgfpathrectangle{\pgfqpoint{0.765000in}{0.660000in}}{\pgfqpoint{4.620000in}{4.620000in}}%
\pgfusepath{clip}%
\pgfsetbuttcap%
\pgfsetroundjoin%
\pgfsetlinewidth{1.204500pt}%
\definecolor{currentstroke}{rgb}{0.827451,0.827451,0.827451}%
\pgfsetstrokecolor{currentstroke}%
\pgfsetdash{{1.200000pt}{1.980000pt}}{0.000000pt}%
\pgfpathmoveto{\pgfqpoint{3.623268in}{3.108321in}}%
\pgfpathlineto{\pgfqpoint{5.395000in}{3.938661in}}%
\pgfusepath{stroke}%
\end{pgfscope}%
\begin{pgfscope}%
\pgfpathrectangle{\pgfqpoint{0.765000in}{0.660000in}}{\pgfqpoint{4.620000in}{4.620000in}}%
\pgfusepath{clip}%
\pgfsetroundcap%
\pgfsetroundjoin%
\pgfsetlinewidth{1.204500pt}%
\definecolor{currentstroke}{rgb}{0.000000,0.000000,0.000000}%
\pgfsetstrokecolor{currentstroke}%
\pgfsetdash{}{0pt}%
\pgfpathmoveto{\pgfqpoint{3.442553in}{3.193015in}}%
\pgfusepath{stroke}%
\end{pgfscope}%
\begin{pgfscope}%
\pgfpathrectangle{\pgfqpoint{0.765000in}{0.660000in}}{\pgfqpoint{4.620000in}{4.620000in}}%
\pgfusepath{clip}%
\pgfsetbuttcap%
\pgfsetroundjoin%
\definecolor{currentfill}{rgb}{0.000000,0.000000,0.000000}%
\pgfsetfillcolor{currentfill}%
\pgfsetlinewidth{1.003750pt}%
\definecolor{currentstroke}{rgb}{0.000000,0.000000,0.000000}%
\pgfsetstrokecolor{currentstroke}%
\pgfsetdash{}{0pt}%
\pgfsys@defobject{currentmarker}{\pgfqpoint{-0.016667in}{-0.016667in}}{\pgfqpoint{0.016667in}{0.016667in}}{%
\pgfpathmoveto{\pgfqpoint{0.000000in}{-0.016667in}}%
\pgfpathcurveto{\pgfqpoint{0.004420in}{-0.016667in}}{\pgfqpoint{0.008660in}{-0.014911in}}{\pgfqpoint{0.011785in}{-0.011785in}}%
\pgfpathcurveto{\pgfqpoint{0.014911in}{-0.008660in}}{\pgfqpoint{0.016667in}{-0.004420in}}{\pgfqpoint{0.016667in}{0.000000in}}%
\pgfpathcurveto{\pgfqpoint{0.016667in}{0.004420in}}{\pgfqpoint{0.014911in}{0.008660in}}{\pgfqpoint{0.011785in}{0.011785in}}%
\pgfpathcurveto{\pgfqpoint{0.008660in}{0.014911in}}{\pgfqpoint{0.004420in}{0.016667in}}{\pgfqpoint{0.000000in}{0.016667in}}%
\pgfpathcurveto{\pgfqpoint{-0.004420in}{0.016667in}}{\pgfqpoint{-0.008660in}{0.014911in}}{\pgfqpoint{-0.011785in}{0.011785in}}%
\pgfpathcurveto{\pgfqpoint{-0.014911in}{0.008660in}}{\pgfqpoint{-0.016667in}{0.004420in}}{\pgfqpoint{-0.016667in}{0.000000in}}%
\pgfpathcurveto{\pgfqpoint{-0.016667in}{-0.004420in}}{\pgfqpoint{-0.014911in}{-0.008660in}}{\pgfqpoint{-0.011785in}{-0.011785in}}%
\pgfpathcurveto{\pgfqpoint{-0.008660in}{-0.014911in}}{\pgfqpoint{-0.004420in}{-0.016667in}}{\pgfqpoint{0.000000in}{-0.016667in}}%
\pgfpathlineto{\pgfqpoint{0.000000in}{-0.016667in}}%
\pgfpathclose%
\pgfusepath{stroke,fill}%
}%
\begin{pgfscope}%
\pgfsys@transformshift{3.442553in}{3.193015in}%
\pgfsys@useobject{currentmarker}{}%
\end{pgfscope}%
\end{pgfscope}%
\begin{pgfscope}%
\pgfpathrectangle{\pgfqpoint{0.765000in}{0.660000in}}{\pgfqpoint{4.620000in}{4.620000in}}%
\pgfusepath{clip}%
\pgfsetroundcap%
\pgfsetroundjoin%
\pgfsetlinewidth{1.204500pt}%
\definecolor{currentstroke}{rgb}{0.000000,0.000000,0.000000}%
\pgfsetstrokecolor{currentstroke}%
\pgfsetdash{}{0pt}%
\pgfpathmoveto{\pgfqpoint{3.442553in}{3.193015in}}%
\pgfpathlineto{\pgfqpoint{2.894524in}{3.193015in}}%
\pgfusepath{stroke}%
\end{pgfscope}%
\begin{pgfscope}%
\pgfpathrectangle{\pgfqpoint{0.765000in}{0.660000in}}{\pgfqpoint{4.620000in}{4.620000in}}%
\pgfusepath{clip}%
\pgfsetroundcap%
\pgfsetroundjoin%
\pgfsetlinewidth{1.204500pt}%
\definecolor{currentstroke}{rgb}{0.000000,0.000000,0.000000}%
\pgfsetstrokecolor{currentstroke}%
\pgfsetdash{}{0pt}%
\pgfpathmoveto{\pgfqpoint{3.442553in}{3.193015in}}%
\pgfpathlineto{\pgfqpoint{3.623268in}{3.108321in}}%
\pgfusepath{stroke}%
\end{pgfscope}%
\begin{pgfscope}%
\pgfpathrectangle{\pgfqpoint{0.765000in}{0.660000in}}{\pgfqpoint{4.620000in}{4.620000in}}%
\pgfusepath{clip}%
\pgfsetbuttcap%
\pgfsetroundjoin%
\pgfsetlinewidth{1.204500pt}%
\definecolor{currentstroke}{rgb}{0.827451,0.827451,0.827451}%
\pgfsetstrokecolor{currentstroke}%
\pgfsetdash{{1.200000pt}{1.980000pt}}{0.000000pt}%
\pgfpathmoveto{\pgfqpoint{3.442553in}{3.193015in}}%
\pgfpathlineto{\pgfqpoint{5.395000in}{4.108048in}}%
\pgfusepath{stroke}%
\end{pgfscope}%
\begin{pgfscope}%
\pgfpathrectangle{\pgfqpoint{0.765000in}{0.660000in}}{\pgfqpoint{4.620000in}{4.620000in}}%
\pgfusepath{clip}%
\pgfsetbuttcap%
\pgfsetroundjoin%
\definecolor{currentfill}{rgb}{1.000000,0.576471,0.309804}%
\pgfsetfillcolor{currentfill}%
\pgfsetfillopacity{0.050000}%
\pgfsetlinewidth{0.000000pt}%
\definecolor{currentstroke}{rgb}{1.000000,0.576471,0.309804}%
\pgfsetstrokecolor{currentstroke}%
\pgfsetdash{}{0pt}%
\pgfpathmoveto{\pgfqpoint{2.710483in}{3.473716in}}%
\pgfpathlineto{\pgfqpoint{2.710483in}{3.546719in}}%
\pgfpathlineto{\pgfqpoint{2.788435in}{3.510123in}}%
\pgfpathlineto{\pgfqpoint{2.788435in}{3.437119in}}%
\pgfpathlineto{\pgfqpoint{2.710483in}{3.473716in}}%
\pgfpathclose%
\pgfusepath{fill}%
\end{pgfscope}%
\begin{pgfscope}%
\pgfpathrectangle{\pgfqpoint{0.765000in}{0.660000in}}{\pgfqpoint{4.620000in}{4.620000in}}%
\pgfusepath{clip}%
\pgfsetbuttcap%
\pgfsetroundjoin%
\definecolor{currentfill}{rgb}{1.000000,0.576471,0.309804}%
\pgfsetfillcolor{currentfill}%
\pgfsetfillopacity{0.050000}%
\pgfsetlinewidth{0.000000pt}%
\definecolor{currentstroke}{rgb}{1.000000,0.576471,0.309804}%
\pgfsetstrokecolor{currentstroke}%
\pgfsetdash{}{0pt}%
\pgfpathmoveto{\pgfqpoint{2.710483in}{3.473716in}}%
\pgfpathlineto{\pgfqpoint{2.710483in}{3.546719in}}%
\pgfpathlineto{\pgfqpoint{2.632530in}{3.510123in}}%
\pgfpathlineto{\pgfqpoint{2.632530in}{3.437119in}}%
\pgfpathlineto{\pgfqpoint{2.710483in}{3.473716in}}%
\pgfpathclose%
\pgfusepath{fill}%
\end{pgfscope}%
\begin{pgfscope}%
\pgfpathrectangle{\pgfqpoint{0.765000in}{0.660000in}}{\pgfqpoint{4.620000in}{4.620000in}}%
\pgfusepath{clip}%
\pgfsetbuttcap%
\pgfsetroundjoin%
\definecolor{currentfill}{rgb}{1.000000,0.576471,0.309804}%
\pgfsetfillcolor{currentfill}%
\pgfsetfillopacity{0.050000}%
\pgfsetlinewidth{0.000000pt}%
\definecolor{currentstroke}{rgb}{1.000000,0.576471,0.309804}%
\pgfsetstrokecolor{currentstroke}%
\pgfsetdash{}{0pt}%
\pgfpathmoveto{\pgfqpoint{2.710483in}{3.473716in}}%
\pgfpathlineto{\pgfqpoint{2.788435in}{3.437119in}}%
\pgfpathlineto{\pgfqpoint{2.710483in}{3.400523in}}%
\pgfpathlineto{\pgfqpoint{2.632530in}{3.437119in}}%
\pgfpathlineto{\pgfqpoint{2.710483in}{3.473716in}}%
\pgfpathclose%
\pgfusepath{fill}%
\end{pgfscope}%
\begin{pgfscope}%
\pgfpathrectangle{\pgfqpoint{0.765000in}{0.660000in}}{\pgfqpoint{4.620000in}{4.620000in}}%
\pgfusepath{clip}%
\pgfsetbuttcap%
\pgfsetroundjoin%
\definecolor{currentfill}{rgb}{1.000000,0.576471,0.309804}%
\pgfsetfillcolor{currentfill}%
\pgfsetfillopacity{0.050000}%
\pgfsetlinewidth{0.000000pt}%
\definecolor{currentstroke}{rgb}{1.000000,0.576471,0.309804}%
\pgfsetstrokecolor{currentstroke}%
\pgfsetdash{}{0pt}%
\pgfpathmoveto{\pgfqpoint{2.710483in}{3.546719in}}%
\pgfpathlineto{\pgfqpoint{2.788435in}{3.510123in}}%
\pgfpathlineto{\pgfqpoint{2.710483in}{3.473526in}}%
\pgfpathlineto{\pgfqpoint{2.632530in}{3.510123in}}%
\pgfpathlineto{\pgfqpoint{2.710483in}{3.546719in}}%
\pgfpathclose%
\pgfusepath{fill}%
\end{pgfscope}%
\begin{pgfscope}%
\pgfpathrectangle{\pgfqpoint{0.765000in}{0.660000in}}{\pgfqpoint{4.620000in}{4.620000in}}%
\pgfusepath{clip}%
\pgfsetbuttcap%
\pgfsetroundjoin%
\definecolor{currentfill}{rgb}{1.000000,0.576471,0.309804}%
\pgfsetfillcolor{currentfill}%
\pgfsetfillopacity{0.050000}%
\pgfsetlinewidth{0.000000pt}%
\definecolor{currentstroke}{rgb}{1.000000,0.576471,0.309804}%
\pgfsetstrokecolor{currentstroke}%
\pgfsetdash{}{0pt}%
\pgfpathmoveto{\pgfqpoint{2.632530in}{3.437119in}}%
\pgfpathlineto{\pgfqpoint{2.632530in}{3.510123in}}%
\pgfpathlineto{\pgfqpoint{2.710483in}{3.473526in}}%
\pgfpathlineto{\pgfqpoint{2.710483in}{3.400523in}}%
\pgfpathlineto{\pgfqpoint{2.632530in}{3.437119in}}%
\pgfpathclose%
\pgfusepath{fill}%
\end{pgfscope}%
\begin{pgfscope}%
\pgfpathrectangle{\pgfqpoint{0.765000in}{0.660000in}}{\pgfqpoint{4.620000in}{4.620000in}}%
\pgfusepath{clip}%
\pgfsetbuttcap%
\pgfsetroundjoin%
\definecolor{currentfill}{rgb}{1.000000,0.576471,0.309804}%
\pgfsetfillcolor{currentfill}%
\pgfsetfillopacity{0.050000}%
\pgfsetlinewidth{0.000000pt}%
\definecolor{currentstroke}{rgb}{1.000000,0.576471,0.309804}%
\pgfsetstrokecolor{currentstroke}%
\pgfsetdash{}{0pt}%
\pgfpathmoveto{\pgfqpoint{2.788435in}{3.437119in}}%
\pgfpathlineto{\pgfqpoint{2.788435in}{3.510123in}}%
\pgfpathlineto{\pgfqpoint{2.710483in}{3.473526in}}%
\pgfpathlineto{\pgfqpoint{2.710483in}{3.400523in}}%
\pgfpathlineto{\pgfqpoint{2.788435in}{3.437119in}}%
\pgfpathclose%
\pgfusepath{fill}%
\end{pgfscope}%
\begin{pgfscope}%
\pgfpathrectangle{\pgfqpoint{0.765000in}{0.660000in}}{\pgfqpoint{4.620000in}{4.620000in}}%
\pgfusepath{clip}%
\pgfsetbuttcap%
\pgfsetroundjoin%
\definecolor{currentfill}{rgb}{1.000000,0.576471,0.309804}%
\pgfsetfillcolor{currentfill}%
\pgfsetfillopacity{0.050000}%
\pgfsetlinewidth{0.000000pt}%
\definecolor{currentstroke}{rgb}{1.000000,0.576471,0.309804}%
\pgfsetstrokecolor{currentstroke}%
\pgfsetdash{}{0pt}%
\pgfpathmoveto{\pgfqpoint{3.319670in}{3.877189in}}%
\pgfpathlineto{\pgfqpoint{3.319670in}{3.950192in}}%
\pgfpathlineto{\pgfqpoint{3.397622in}{3.913595in}}%
\pgfpathlineto{\pgfqpoint{3.397622in}{3.840592in}}%
\pgfpathlineto{\pgfqpoint{3.319670in}{3.877189in}}%
\pgfpathclose%
\pgfusepath{fill}%
\end{pgfscope}%
\begin{pgfscope}%
\pgfpathrectangle{\pgfqpoint{0.765000in}{0.660000in}}{\pgfqpoint{4.620000in}{4.620000in}}%
\pgfusepath{clip}%
\pgfsetbuttcap%
\pgfsetroundjoin%
\definecolor{currentfill}{rgb}{1.000000,0.576471,0.309804}%
\pgfsetfillcolor{currentfill}%
\pgfsetfillopacity{0.050000}%
\pgfsetlinewidth{0.000000pt}%
\definecolor{currentstroke}{rgb}{1.000000,0.576471,0.309804}%
\pgfsetstrokecolor{currentstroke}%
\pgfsetdash{}{0pt}%
\pgfpathmoveto{\pgfqpoint{3.319670in}{3.877189in}}%
\pgfpathlineto{\pgfqpoint{3.319670in}{3.950192in}}%
\pgfpathlineto{\pgfqpoint{3.241717in}{3.913595in}}%
\pgfpathlineto{\pgfqpoint{3.241717in}{3.840592in}}%
\pgfpathlineto{\pgfqpoint{3.319670in}{3.877189in}}%
\pgfpathclose%
\pgfusepath{fill}%
\end{pgfscope}%
\begin{pgfscope}%
\pgfpathrectangle{\pgfqpoint{0.765000in}{0.660000in}}{\pgfqpoint{4.620000in}{4.620000in}}%
\pgfusepath{clip}%
\pgfsetbuttcap%
\pgfsetroundjoin%
\definecolor{currentfill}{rgb}{1.000000,0.576471,0.309804}%
\pgfsetfillcolor{currentfill}%
\pgfsetfillopacity{0.050000}%
\pgfsetlinewidth{0.000000pt}%
\definecolor{currentstroke}{rgb}{1.000000,0.576471,0.309804}%
\pgfsetstrokecolor{currentstroke}%
\pgfsetdash{}{0pt}%
\pgfpathmoveto{\pgfqpoint{3.319670in}{3.877189in}}%
\pgfpathlineto{\pgfqpoint{3.397622in}{3.840592in}}%
\pgfpathlineto{\pgfqpoint{3.319670in}{3.803995in}}%
\pgfpathlineto{\pgfqpoint{3.241717in}{3.840592in}}%
\pgfpathlineto{\pgfqpoint{3.319670in}{3.877189in}}%
\pgfpathclose%
\pgfusepath{fill}%
\end{pgfscope}%
\begin{pgfscope}%
\pgfpathrectangle{\pgfqpoint{0.765000in}{0.660000in}}{\pgfqpoint{4.620000in}{4.620000in}}%
\pgfusepath{clip}%
\pgfsetbuttcap%
\pgfsetroundjoin%
\definecolor{currentfill}{rgb}{1.000000,0.576471,0.309804}%
\pgfsetfillcolor{currentfill}%
\pgfsetfillopacity{0.050000}%
\pgfsetlinewidth{0.000000pt}%
\definecolor{currentstroke}{rgb}{1.000000,0.576471,0.309804}%
\pgfsetstrokecolor{currentstroke}%
\pgfsetdash{}{0pt}%
\pgfpathmoveto{\pgfqpoint{3.319670in}{3.950192in}}%
\pgfpathlineto{\pgfqpoint{3.397622in}{3.913595in}}%
\pgfpathlineto{\pgfqpoint{3.319670in}{3.876999in}}%
\pgfpathlineto{\pgfqpoint{3.241717in}{3.913595in}}%
\pgfpathlineto{\pgfqpoint{3.319670in}{3.950192in}}%
\pgfpathclose%
\pgfusepath{fill}%
\end{pgfscope}%
\begin{pgfscope}%
\pgfpathrectangle{\pgfqpoint{0.765000in}{0.660000in}}{\pgfqpoint{4.620000in}{4.620000in}}%
\pgfusepath{clip}%
\pgfsetbuttcap%
\pgfsetroundjoin%
\definecolor{currentfill}{rgb}{1.000000,0.576471,0.309804}%
\pgfsetfillcolor{currentfill}%
\pgfsetfillopacity{0.050000}%
\pgfsetlinewidth{0.000000pt}%
\definecolor{currentstroke}{rgb}{1.000000,0.576471,0.309804}%
\pgfsetstrokecolor{currentstroke}%
\pgfsetdash{}{0pt}%
\pgfpathmoveto{\pgfqpoint{3.241717in}{3.840592in}}%
\pgfpathlineto{\pgfqpoint{3.241717in}{3.913595in}}%
\pgfpathlineto{\pgfqpoint{3.319670in}{3.876999in}}%
\pgfpathlineto{\pgfqpoint{3.319670in}{3.803995in}}%
\pgfpathlineto{\pgfqpoint{3.241717in}{3.840592in}}%
\pgfpathclose%
\pgfusepath{fill}%
\end{pgfscope}%
\begin{pgfscope}%
\pgfpathrectangle{\pgfqpoint{0.765000in}{0.660000in}}{\pgfqpoint{4.620000in}{4.620000in}}%
\pgfusepath{clip}%
\pgfsetbuttcap%
\pgfsetroundjoin%
\definecolor{currentfill}{rgb}{1.000000,0.576471,0.309804}%
\pgfsetfillcolor{currentfill}%
\pgfsetfillopacity{0.050000}%
\pgfsetlinewidth{0.000000pt}%
\definecolor{currentstroke}{rgb}{1.000000,0.576471,0.309804}%
\pgfsetstrokecolor{currentstroke}%
\pgfsetdash{}{0pt}%
\pgfpathmoveto{\pgfqpoint{3.397622in}{3.840592in}}%
\pgfpathlineto{\pgfqpoint{3.397622in}{3.913595in}}%
\pgfpathlineto{\pgfqpoint{3.319670in}{3.876999in}}%
\pgfpathlineto{\pgfqpoint{3.319670in}{3.803995in}}%
\pgfpathlineto{\pgfqpoint{3.397622in}{3.840592in}}%
\pgfpathclose%
\pgfusepath{fill}%
\end{pgfscope}%
\begin{pgfscope}%
\pgfpathrectangle{\pgfqpoint{0.765000in}{0.660000in}}{\pgfqpoint{4.620000in}{4.620000in}}%
\pgfusepath{clip}%
\pgfsetbuttcap%
\pgfsetroundjoin%
\definecolor{currentfill}{rgb}{1.000000,0.576471,0.309804}%
\pgfsetfillcolor{currentfill}%
\pgfsetfillopacity{0.050000}%
\pgfsetlinewidth{0.000000pt}%
\definecolor{currentstroke}{rgb}{1.000000,0.576471,0.309804}%
\pgfsetstrokecolor{currentstroke}%
\pgfsetdash{}{0pt}%
\pgfpathmoveto{\pgfqpoint{3.969934in}{2.866906in}}%
\pgfpathlineto{\pgfqpoint{3.969934in}{2.939909in}}%
\pgfpathlineto{\pgfqpoint{4.047886in}{2.903313in}}%
\pgfpathlineto{\pgfqpoint{4.047886in}{2.830310in}}%
\pgfpathlineto{\pgfqpoint{3.969934in}{2.866906in}}%
\pgfpathclose%
\pgfusepath{fill}%
\end{pgfscope}%
\begin{pgfscope}%
\pgfpathrectangle{\pgfqpoint{0.765000in}{0.660000in}}{\pgfqpoint{4.620000in}{4.620000in}}%
\pgfusepath{clip}%
\pgfsetbuttcap%
\pgfsetroundjoin%
\definecolor{currentfill}{rgb}{1.000000,0.576471,0.309804}%
\pgfsetfillcolor{currentfill}%
\pgfsetfillopacity{0.050000}%
\pgfsetlinewidth{0.000000pt}%
\definecolor{currentstroke}{rgb}{1.000000,0.576471,0.309804}%
\pgfsetstrokecolor{currentstroke}%
\pgfsetdash{}{0pt}%
\pgfpathmoveto{\pgfqpoint{3.969934in}{2.866906in}}%
\pgfpathlineto{\pgfqpoint{3.969934in}{2.939909in}}%
\pgfpathlineto{\pgfqpoint{3.891981in}{2.903313in}}%
\pgfpathlineto{\pgfqpoint{3.891981in}{2.830310in}}%
\pgfpathlineto{\pgfqpoint{3.969934in}{2.866906in}}%
\pgfpathclose%
\pgfusepath{fill}%
\end{pgfscope}%
\begin{pgfscope}%
\pgfpathrectangle{\pgfqpoint{0.765000in}{0.660000in}}{\pgfqpoint{4.620000in}{4.620000in}}%
\pgfusepath{clip}%
\pgfsetbuttcap%
\pgfsetroundjoin%
\definecolor{currentfill}{rgb}{1.000000,0.576471,0.309804}%
\pgfsetfillcolor{currentfill}%
\pgfsetfillopacity{0.050000}%
\pgfsetlinewidth{0.000000pt}%
\definecolor{currentstroke}{rgb}{1.000000,0.576471,0.309804}%
\pgfsetstrokecolor{currentstroke}%
\pgfsetdash{}{0pt}%
\pgfpathmoveto{\pgfqpoint{3.969934in}{2.866906in}}%
\pgfpathlineto{\pgfqpoint{4.047886in}{2.830310in}}%
\pgfpathlineto{\pgfqpoint{3.969934in}{2.793713in}}%
\pgfpathlineto{\pgfqpoint{3.891981in}{2.830310in}}%
\pgfpathlineto{\pgfqpoint{3.969934in}{2.866906in}}%
\pgfpathclose%
\pgfusepath{fill}%
\end{pgfscope}%
\begin{pgfscope}%
\pgfpathrectangle{\pgfqpoint{0.765000in}{0.660000in}}{\pgfqpoint{4.620000in}{4.620000in}}%
\pgfusepath{clip}%
\pgfsetbuttcap%
\pgfsetroundjoin%
\definecolor{currentfill}{rgb}{1.000000,0.576471,0.309804}%
\pgfsetfillcolor{currentfill}%
\pgfsetfillopacity{0.050000}%
\pgfsetlinewidth{0.000000pt}%
\definecolor{currentstroke}{rgb}{1.000000,0.576471,0.309804}%
\pgfsetstrokecolor{currentstroke}%
\pgfsetdash{}{0pt}%
\pgfpathmoveto{\pgfqpoint{3.969934in}{2.939909in}}%
\pgfpathlineto{\pgfqpoint{4.047886in}{2.903313in}}%
\pgfpathlineto{\pgfqpoint{3.969934in}{2.866716in}}%
\pgfpathlineto{\pgfqpoint{3.891981in}{2.903313in}}%
\pgfpathlineto{\pgfqpoint{3.969934in}{2.939909in}}%
\pgfpathclose%
\pgfusepath{fill}%
\end{pgfscope}%
\begin{pgfscope}%
\pgfpathrectangle{\pgfqpoint{0.765000in}{0.660000in}}{\pgfqpoint{4.620000in}{4.620000in}}%
\pgfusepath{clip}%
\pgfsetbuttcap%
\pgfsetroundjoin%
\definecolor{currentfill}{rgb}{1.000000,0.576471,0.309804}%
\pgfsetfillcolor{currentfill}%
\pgfsetfillopacity{0.050000}%
\pgfsetlinewidth{0.000000pt}%
\definecolor{currentstroke}{rgb}{1.000000,0.576471,0.309804}%
\pgfsetstrokecolor{currentstroke}%
\pgfsetdash{}{0pt}%
\pgfpathmoveto{\pgfqpoint{3.891981in}{2.830310in}}%
\pgfpathlineto{\pgfqpoint{3.891981in}{2.903313in}}%
\pgfpathlineto{\pgfqpoint{3.969934in}{2.866716in}}%
\pgfpathlineto{\pgfqpoint{3.969934in}{2.793713in}}%
\pgfpathlineto{\pgfqpoint{3.891981in}{2.830310in}}%
\pgfpathclose%
\pgfusepath{fill}%
\end{pgfscope}%
\begin{pgfscope}%
\pgfpathrectangle{\pgfqpoint{0.765000in}{0.660000in}}{\pgfqpoint{4.620000in}{4.620000in}}%
\pgfusepath{clip}%
\pgfsetbuttcap%
\pgfsetroundjoin%
\definecolor{currentfill}{rgb}{1.000000,0.576471,0.309804}%
\pgfsetfillcolor{currentfill}%
\pgfsetfillopacity{0.050000}%
\pgfsetlinewidth{0.000000pt}%
\definecolor{currentstroke}{rgb}{1.000000,0.576471,0.309804}%
\pgfsetstrokecolor{currentstroke}%
\pgfsetdash{}{0pt}%
\pgfpathmoveto{\pgfqpoint{4.047886in}{2.830310in}}%
\pgfpathlineto{\pgfqpoint{4.047886in}{2.903313in}}%
\pgfpathlineto{\pgfqpoint{3.969934in}{2.866716in}}%
\pgfpathlineto{\pgfqpoint{3.969934in}{2.793713in}}%
\pgfpathlineto{\pgfqpoint{4.047886in}{2.830310in}}%
\pgfpathclose%
\pgfusepath{fill}%
\end{pgfscope}%
\begin{pgfscope}%
\pgfpathrectangle{\pgfqpoint{0.765000in}{0.660000in}}{\pgfqpoint{4.620000in}{4.620000in}}%
\pgfusepath{clip}%
\pgfsetbuttcap%
\pgfsetroundjoin%
\definecolor{currentfill}{rgb}{1.000000,0.576471,0.309804}%
\pgfsetfillcolor{currentfill}%
\pgfsetfillopacity{0.050000}%
\pgfsetlinewidth{0.000000pt}%
\definecolor{currentstroke}{rgb}{1.000000,0.576471,0.309804}%
\pgfsetstrokecolor{currentstroke}%
\pgfsetdash{}{0pt}%
\pgfpathmoveto{\pgfqpoint{2.618942in}{2.638525in}}%
\pgfpathlineto{\pgfqpoint{2.618942in}{2.711528in}}%
\pgfpathlineto{\pgfqpoint{2.696895in}{2.674931in}}%
\pgfpathlineto{\pgfqpoint{2.696895in}{2.601928in}}%
\pgfpathlineto{\pgfqpoint{2.618942in}{2.638525in}}%
\pgfpathclose%
\pgfusepath{fill}%
\end{pgfscope}%
\begin{pgfscope}%
\pgfpathrectangle{\pgfqpoint{0.765000in}{0.660000in}}{\pgfqpoint{4.620000in}{4.620000in}}%
\pgfusepath{clip}%
\pgfsetbuttcap%
\pgfsetroundjoin%
\definecolor{currentfill}{rgb}{1.000000,0.576471,0.309804}%
\pgfsetfillcolor{currentfill}%
\pgfsetfillopacity{0.050000}%
\pgfsetlinewidth{0.000000pt}%
\definecolor{currentstroke}{rgb}{1.000000,0.576471,0.309804}%
\pgfsetstrokecolor{currentstroke}%
\pgfsetdash{}{0pt}%
\pgfpathmoveto{\pgfqpoint{2.618942in}{2.638525in}}%
\pgfpathlineto{\pgfqpoint{2.618942in}{2.711528in}}%
\pgfpathlineto{\pgfqpoint{2.540989in}{2.674931in}}%
\pgfpathlineto{\pgfqpoint{2.540989in}{2.601928in}}%
\pgfpathlineto{\pgfqpoint{2.618942in}{2.638525in}}%
\pgfpathclose%
\pgfusepath{fill}%
\end{pgfscope}%
\begin{pgfscope}%
\pgfpathrectangle{\pgfqpoint{0.765000in}{0.660000in}}{\pgfqpoint{4.620000in}{4.620000in}}%
\pgfusepath{clip}%
\pgfsetbuttcap%
\pgfsetroundjoin%
\definecolor{currentfill}{rgb}{1.000000,0.576471,0.309804}%
\pgfsetfillcolor{currentfill}%
\pgfsetfillopacity{0.050000}%
\pgfsetlinewidth{0.000000pt}%
\definecolor{currentstroke}{rgb}{1.000000,0.576471,0.309804}%
\pgfsetstrokecolor{currentstroke}%
\pgfsetdash{}{0pt}%
\pgfpathmoveto{\pgfqpoint{2.618942in}{2.638525in}}%
\pgfpathlineto{\pgfqpoint{2.696895in}{2.601928in}}%
\pgfpathlineto{\pgfqpoint{2.618942in}{2.565331in}}%
\pgfpathlineto{\pgfqpoint{2.540989in}{2.601928in}}%
\pgfpathlineto{\pgfqpoint{2.618942in}{2.638525in}}%
\pgfpathclose%
\pgfusepath{fill}%
\end{pgfscope}%
\begin{pgfscope}%
\pgfpathrectangle{\pgfqpoint{0.765000in}{0.660000in}}{\pgfqpoint{4.620000in}{4.620000in}}%
\pgfusepath{clip}%
\pgfsetbuttcap%
\pgfsetroundjoin%
\definecolor{currentfill}{rgb}{1.000000,0.576471,0.309804}%
\pgfsetfillcolor{currentfill}%
\pgfsetfillopacity{0.050000}%
\pgfsetlinewidth{0.000000pt}%
\definecolor{currentstroke}{rgb}{1.000000,0.576471,0.309804}%
\pgfsetstrokecolor{currentstroke}%
\pgfsetdash{}{0pt}%
\pgfpathmoveto{\pgfqpoint{2.618942in}{2.711528in}}%
\pgfpathlineto{\pgfqpoint{2.696895in}{2.674931in}}%
\pgfpathlineto{\pgfqpoint{2.618942in}{2.638335in}}%
\pgfpathlineto{\pgfqpoint{2.540989in}{2.674931in}}%
\pgfpathlineto{\pgfqpoint{2.618942in}{2.711528in}}%
\pgfpathclose%
\pgfusepath{fill}%
\end{pgfscope}%
\begin{pgfscope}%
\pgfpathrectangle{\pgfqpoint{0.765000in}{0.660000in}}{\pgfqpoint{4.620000in}{4.620000in}}%
\pgfusepath{clip}%
\pgfsetbuttcap%
\pgfsetroundjoin%
\definecolor{currentfill}{rgb}{1.000000,0.576471,0.309804}%
\pgfsetfillcolor{currentfill}%
\pgfsetfillopacity{0.050000}%
\pgfsetlinewidth{0.000000pt}%
\definecolor{currentstroke}{rgb}{1.000000,0.576471,0.309804}%
\pgfsetstrokecolor{currentstroke}%
\pgfsetdash{}{0pt}%
\pgfpathmoveto{\pgfqpoint{2.540989in}{2.601928in}}%
\pgfpathlineto{\pgfqpoint{2.540989in}{2.674931in}}%
\pgfpathlineto{\pgfqpoint{2.618942in}{2.638335in}}%
\pgfpathlineto{\pgfqpoint{2.618942in}{2.565331in}}%
\pgfpathlineto{\pgfqpoint{2.540989in}{2.601928in}}%
\pgfpathclose%
\pgfusepath{fill}%
\end{pgfscope}%
\begin{pgfscope}%
\pgfpathrectangle{\pgfqpoint{0.765000in}{0.660000in}}{\pgfqpoint{4.620000in}{4.620000in}}%
\pgfusepath{clip}%
\pgfsetbuttcap%
\pgfsetroundjoin%
\definecolor{currentfill}{rgb}{1.000000,0.576471,0.309804}%
\pgfsetfillcolor{currentfill}%
\pgfsetfillopacity{0.050000}%
\pgfsetlinewidth{0.000000pt}%
\definecolor{currentstroke}{rgb}{1.000000,0.576471,0.309804}%
\pgfsetstrokecolor{currentstroke}%
\pgfsetdash{}{0pt}%
\pgfpathmoveto{\pgfqpoint{2.696895in}{2.601928in}}%
\pgfpathlineto{\pgfqpoint{2.696895in}{2.674931in}}%
\pgfpathlineto{\pgfqpoint{2.618942in}{2.638335in}}%
\pgfpathlineto{\pgfqpoint{2.618942in}{2.565331in}}%
\pgfpathlineto{\pgfqpoint{2.696895in}{2.601928in}}%
\pgfpathclose%
\pgfusepath{fill}%
\end{pgfscope}%
\begin{pgfscope}%
\pgfpathrectangle{\pgfqpoint{0.765000in}{0.660000in}}{\pgfqpoint{4.620000in}{4.620000in}}%
\pgfusepath{clip}%
\pgfsetbuttcap%
\pgfsetroundjoin%
\definecolor{currentfill}{rgb}{1.000000,0.576471,0.309804}%
\pgfsetfillcolor{currentfill}%
\pgfsetfillopacity{0.050000}%
\pgfsetlinewidth{0.000000pt}%
\definecolor{currentstroke}{rgb}{1.000000,0.576471,0.309804}%
\pgfsetstrokecolor{currentstroke}%
\pgfsetdash{}{0pt}%
\pgfpathmoveto{\pgfqpoint{1.957157in}{3.644042in}}%
\pgfpathlineto{\pgfqpoint{1.957157in}{3.717045in}}%
\pgfpathlineto{\pgfqpoint{2.035110in}{3.680448in}}%
\pgfpathlineto{\pgfqpoint{2.035110in}{3.607445in}}%
\pgfpathlineto{\pgfqpoint{1.957157in}{3.644042in}}%
\pgfpathclose%
\pgfusepath{fill}%
\end{pgfscope}%
\begin{pgfscope}%
\pgfpathrectangle{\pgfqpoint{0.765000in}{0.660000in}}{\pgfqpoint{4.620000in}{4.620000in}}%
\pgfusepath{clip}%
\pgfsetbuttcap%
\pgfsetroundjoin%
\definecolor{currentfill}{rgb}{1.000000,0.576471,0.309804}%
\pgfsetfillcolor{currentfill}%
\pgfsetfillopacity{0.050000}%
\pgfsetlinewidth{0.000000pt}%
\definecolor{currentstroke}{rgb}{1.000000,0.576471,0.309804}%
\pgfsetstrokecolor{currentstroke}%
\pgfsetdash{}{0pt}%
\pgfpathmoveto{\pgfqpoint{1.957157in}{3.644042in}}%
\pgfpathlineto{\pgfqpoint{1.957157in}{3.717045in}}%
\pgfpathlineto{\pgfqpoint{1.879205in}{3.680448in}}%
\pgfpathlineto{\pgfqpoint{1.879205in}{3.607445in}}%
\pgfpathlineto{\pgfqpoint{1.957157in}{3.644042in}}%
\pgfpathclose%
\pgfusepath{fill}%
\end{pgfscope}%
\begin{pgfscope}%
\pgfpathrectangle{\pgfqpoint{0.765000in}{0.660000in}}{\pgfqpoint{4.620000in}{4.620000in}}%
\pgfusepath{clip}%
\pgfsetbuttcap%
\pgfsetroundjoin%
\definecolor{currentfill}{rgb}{1.000000,0.576471,0.309804}%
\pgfsetfillcolor{currentfill}%
\pgfsetfillopacity{0.050000}%
\pgfsetlinewidth{0.000000pt}%
\definecolor{currentstroke}{rgb}{1.000000,0.576471,0.309804}%
\pgfsetstrokecolor{currentstroke}%
\pgfsetdash{}{0pt}%
\pgfpathmoveto{\pgfqpoint{1.957157in}{3.644042in}}%
\pgfpathlineto{\pgfqpoint{2.035110in}{3.607445in}}%
\pgfpathlineto{\pgfqpoint{1.957157in}{3.570849in}}%
\pgfpathlineto{\pgfqpoint{1.879205in}{3.607445in}}%
\pgfpathlineto{\pgfqpoint{1.957157in}{3.644042in}}%
\pgfpathclose%
\pgfusepath{fill}%
\end{pgfscope}%
\begin{pgfscope}%
\pgfpathrectangle{\pgfqpoint{0.765000in}{0.660000in}}{\pgfqpoint{4.620000in}{4.620000in}}%
\pgfusepath{clip}%
\pgfsetbuttcap%
\pgfsetroundjoin%
\definecolor{currentfill}{rgb}{1.000000,0.576471,0.309804}%
\pgfsetfillcolor{currentfill}%
\pgfsetfillopacity{0.050000}%
\pgfsetlinewidth{0.000000pt}%
\definecolor{currentstroke}{rgb}{1.000000,0.576471,0.309804}%
\pgfsetstrokecolor{currentstroke}%
\pgfsetdash{}{0pt}%
\pgfpathmoveto{\pgfqpoint{1.957157in}{3.717045in}}%
\pgfpathlineto{\pgfqpoint{2.035110in}{3.680448in}}%
\pgfpathlineto{\pgfqpoint{1.957157in}{3.643852in}}%
\pgfpathlineto{\pgfqpoint{1.879205in}{3.680448in}}%
\pgfpathlineto{\pgfqpoint{1.957157in}{3.717045in}}%
\pgfpathclose%
\pgfusepath{fill}%
\end{pgfscope}%
\begin{pgfscope}%
\pgfpathrectangle{\pgfqpoint{0.765000in}{0.660000in}}{\pgfqpoint{4.620000in}{4.620000in}}%
\pgfusepath{clip}%
\pgfsetbuttcap%
\pgfsetroundjoin%
\definecolor{currentfill}{rgb}{1.000000,0.576471,0.309804}%
\pgfsetfillcolor{currentfill}%
\pgfsetfillopacity{0.050000}%
\pgfsetlinewidth{0.000000pt}%
\definecolor{currentstroke}{rgb}{1.000000,0.576471,0.309804}%
\pgfsetstrokecolor{currentstroke}%
\pgfsetdash{}{0pt}%
\pgfpathmoveto{\pgfqpoint{1.879205in}{3.607445in}}%
\pgfpathlineto{\pgfqpoint{1.879205in}{3.680448in}}%
\pgfpathlineto{\pgfqpoint{1.957157in}{3.643852in}}%
\pgfpathlineto{\pgfqpoint{1.957157in}{3.570849in}}%
\pgfpathlineto{\pgfqpoint{1.879205in}{3.607445in}}%
\pgfpathclose%
\pgfusepath{fill}%
\end{pgfscope}%
\begin{pgfscope}%
\pgfpathrectangle{\pgfqpoint{0.765000in}{0.660000in}}{\pgfqpoint{4.620000in}{4.620000in}}%
\pgfusepath{clip}%
\pgfsetbuttcap%
\pgfsetroundjoin%
\definecolor{currentfill}{rgb}{1.000000,0.576471,0.309804}%
\pgfsetfillcolor{currentfill}%
\pgfsetfillopacity{0.050000}%
\pgfsetlinewidth{0.000000pt}%
\definecolor{currentstroke}{rgb}{1.000000,0.576471,0.309804}%
\pgfsetstrokecolor{currentstroke}%
\pgfsetdash{}{0pt}%
\pgfpathmoveto{\pgfqpoint{2.035110in}{3.607445in}}%
\pgfpathlineto{\pgfqpoint{2.035110in}{3.680448in}}%
\pgfpathlineto{\pgfqpoint{1.957157in}{3.643852in}}%
\pgfpathlineto{\pgfqpoint{1.957157in}{3.570849in}}%
\pgfpathlineto{\pgfqpoint{2.035110in}{3.607445in}}%
\pgfpathclose%
\pgfusepath{fill}%
\end{pgfscope}%
\begin{pgfscope}%
\pgfpathrectangle{\pgfqpoint{0.765000in}{0.660000in}}{\pgfqpoint{4.620000in}{4.620000in}}%
\pgfusepath{clip}%
\pgfsetbuttcap%
\pgfsetroundjoin%
\definecolor{currentfill}{rgb}{1.000000,0.576471,0.309804}%
\pgfsetfillcolor{currentfill}%
\pgfsetfillopacity{0.050000}%
\pgfsetlinewidth{0.000000pt}%
\definecolor{currentstroke}{rgb}{1.000000,0.576471,0.309804}%
\pgfsetstrokecolor{currentstroke}%
\pgfsetdash{}{0pt}%
\pgfpathmoveto{\pgfqpoint{3.510546in}{2.285432in}}%
\pgfpathlineto{\pgfqpoint{3.510546in}{2.358435in}}%
\pgfpathlineto{\pgfqpoint{3.588499in}{2.321839in}}%
\pgfpathlineto{\pgfqpoint{3.588499in}{2.248835in}}%
\pgfpathlineto{\pgfqpoint{3.510546in}{2.285432in}}%
\pgfpathclose%
\pgfusepath{fill}%
\end{pgfscope}%
\begin{pgfscope}%
\pgfpathrectangle{\pgfqpoint{0.765000in}{0.660000in}}{\pgfqpoint{4.620000in}{4.620000in}}%
\pgfusepath{clip}%
\pgfsetbuttcap%
\pgfsetroundjoin%
\definecolor{currentfill}{rgb}{1.000000,0.576471,0.309804}%
\pgfsetfillcolor{currentfill}%
\pgfsetfillopacity{0.050000}%
\pgfsetlinewidth{0.000000pt}%
\definecolor{currentstroke}{rgb}{1.000000,0.576471,0.309804}%
\pgfsetstrokecolor{currentstroke}%
\pgfsetdash{}{0pt}%
\pgfpathmoveto{\pgfqpoint{3.510546in}{2.285432in}}%
\pgfpathlineto{\pgfqpoint{3.510546in}{2.358435in}}%
\pgfpathlineto{\pgfqpoint{3.432593in}{2.321839in}}%
\pgfpathlineto{\pgfqpoint{3.432593in}{2.248835in}}%
\pgfpathlineto{\pgfqpoint{3.510546in}{2.285432in}}%
\pgfpathclose%
\pgfusepath{fill}%
\end{pgfscope}%
\begin{pgfscope}%
\pgfpathrectangle{\pgfqpoint{0.765000in}{0.660000in}}{\pgfqpoint{4.620000in}{4.620000in}}%
\pgfusepath{clip}%
\pgfsetbuttcap%
\pgfsetroundjoin%
\definecolor{currentfill}{rgb}{1.000000,0.576471,0.309804}%
\pgfsetfillcolor{currentfill}%
\pgfsetfillopacity{0.050000}%
\pgfsetlinewidth{0.000000pt}%
\definecolor{currentstroke}{rgb}{1.000000,0.576471,0.309804}%
\pgfsetstrokecolor{currentstroke}%
\pgfsetdash{}{0pt}%
\pgfpathmoveto{\pgfqpoint{3.510546in}{2.285432in}}%
\pgfpathlineto{\pgfqpoint{3.588499in}{2.248835in}}%
\pgfpathlineto{\pgfqpoint{3.510546in}{2.212239in}}%
\pgfpathlineto{\pgfqpoint{3.432593in}{2.248835in}}%
\pgfpathlineto{\pgfqpoint{3.510546in}{2.285432in}}%
\pgfpathclose%
\pgfusepath{fill}%
\end{pgfscope}%
\begin{pgfscope}%
\pgfpathrectangle{\pgfqpoint{0.765000in}{0.660000in}}{\pgfqpoint{4.620000in}{4.620000in}}%
\pgfusepath{clip}%
\pgfsetbuttcap%
\pgfsetroundjoin%
\definecolor{currentfill}{rgb}{1.000000,0.576471,0.309804}%
\pgfsetfillcolor{currentfill}%
\pgfsetfillopacity{0.050000}%
\pgfsetlinewidth{0.000000pt}%
\definecolor{currentstroke}{rgb}{1.000000,0.576471,0.309804}%
\pgfsetstrokecolor{currentstroke}%
\pgfsetdash{}{0pt}%
\pgfpathmoveto{\pgfqpoint{3.510546in}{2.358435in}}%
\pgfpathlineto{\pgfqpoint{3.588499in}{2.321839in}}%
\pgfpathlineto{\pgfqpoint{3.510546in}{2.285242in}}%
\pgfpathlineto{\pgfqpoint{3.432593in}{2.321839in}}%
\pgfpathlineto{\pgfqpoint{3.510546in}{2.358435in}}%
\pgfpathclose%
\pgfusepath{fill}%
\end{pgfscope}%
\begin{pgfscope}%
\pgfpathrectangle{\pgfqpoint{0.765000in}{0.660000in}}{\pgfqpoint{4.620000in}{4.620000in}}%
\pgfusepath{clip}%
\pgfsetbuttcap%
\pgfsetroundjoin%
\definecolor{currentfill}{rgb}{1.000000,0.576471,0.309804}%
\pgfsetfillcolor{currentfill}%
\pgfsetfillopacity{0.050000}%
\pgfsetlinewidth{0.000000pt}%
\definecolor{currentstroke}{rgb}{1.000000,0.576471,0.309804}%
\pgfsetstrokecolor{currentstroke}%
\pgfsetdash{}{0pt}%
\pgfpathmoveto{\pgfqpoint{3.432593in}{2.248835in}}%
\pgfpathlineto{\pgfqpoint{3.432593in}{2.321839in}}%
\pgfpathlineto{\pgfqpoint{3.510546in}{2.285242in}}%
\pgfpathlineto{\pgfqpoint{3.510546in}{2.212239in}}%
\pgfpathlineto{\pgfqpoint{3.432593in}{2.248835in}}%
\pgfpathclose%
\pgfusepath{fill}%
\end{pgfscope}%
\begin{pgfscope}%
\pgfpathrectangle{\pgfqpoint{0.765000in}{0.660000in}}{\pgfqpoint{4.620000in}{4.620000in}}%
\pgfusepath{clip}%
\pgfsetbuttcap%
\pgfsetroundjoin%
\definecolor{currentfill}{rgb}{1.000000,0.576471,0.309804}%
\pgfsetfillcolor{currentfill}%
\pgfsetfillopacity{0.050000}%
\pgfsetlinewidth{0.000000pt}%
\definecolor{currentstroke}{rgb}{1.000000,0.576471,0.309804}%
\pgfsetstrokecolor{currentstroke}%
\pgfsetdash{}{0pt}%
\pgfpathmoveto{\pgfqpoint{3.588499in}{2.248835in}}%
\pgfpathlineto{\pgfqpoint{3.588499in}{2.321839in}}%
\pgfpathlineto{\pgfqpoint{3.510546in}{2.285242in}}%
\pgfpathlineto{\pgfqpoint{3.510546in}{2.212239in}}%
\pgfpathlineto{\pgfqpoint{3.588499in}{2.248835in}}%
\pgfpathclose%
\pgfusepath{fill}%
\end{pgfscope}%
\begin{pgfscope}%
\pgfpathrectangle{\pgfqpoint{0.765000in}{0.660000in}}{\pgfqpoint{4.620000in}{4.620000in}}%
\pgfusepath{clip}%
\pgfsetbuttcap%
\pgfsetroundjoin%
\definecolor{currentfill}{rgb}{1.000000,0.576471,0.309804}%
\pgfsetfillcolor{currentfill}%
\pgfsetfillopacity{0.050000}%
\pgfsetlinewidth{0.000000pt}%
\definecolor{currentstroke}{rgb}{1.000000,0.576471,0.309804}%
\pgfsetstrokecolor{currentstroke}%
\pgfsetdash{}{0pt}%
\pgfpathmoveto{\pgfqpoint{2.867029in}{3.407771in}}%
\pgfpathlineto{\pgfqpoint{2.867029in}{3.480774in}}%
\pgfpathlineto{\pgfqpoint{2.944981in}{3.444177in}}%
\pgfpathlineto{\pgfqpoint{2.944981in}{3.371174in}}%
\pgfpathlineto{\pgfqpoint{2.867029in}{3.407771in}}%
\pgfpathclose%
\pgfusepath{fill}%
\end{pgfscope}%
\begin{pgfscope}%
\pgfpathrectangle{\pgfqpoint{0.765000in}{0.660000in}}{\pgfqpoint{4.620000in}{4.620000in}}%
\pgfusepath{clip}%
\pgfsetbuttcap%
\pgfsetroundjoin%
\definecolor{currentfill}{rgb}{1.000000,0.576471,0.309804}%
\pgfsetfillcolor{currentfill}%
\pgfsetfillopacity{0.050000}%
\pgfsetlinewidth{0.000000pt}%
\definecolor{currentstroke}{rgb}{1.000000,0.576471,0.309804}%
\pgfsetstrokecolor{currentstroke}%
\pgfsetdash{}{0pt}%
\pgfpathmoveto{\pgfqpoint{2.867029in}{3.407771in}}%
\pgfpathlineto{\pgfqpoint{2.867029in}{3.480774in}}%
\pgfpathlineto{\pgfqpoint{2.789076in}{3.444177in}}%
\pgfpathlineto{\pgfqpoint{2.789076in}{3.371174in}}%
\pgfpathlineto{\pgfqpoint{2.867029in}{3.407771in}}%
\pgfpathclose%
\pgfusepath{fill}%
\end{pgfscope}%
\begin{pgfscope}%
\pgfpathrectangle{\pgfqpoint{0.765000in}{0.660000in}}{\pgfqpoint{4.620000in}{4.620000in}}%
\pgfusepath{clip}%
\pgfsetbuttcap%
\pgfsetroundjoin%
\definecolor{currentfill}{rgb}{1.000000,0.576471,0.309804}%
\pgfsetfillcolor{currentfill}%
\pgfsetfillopacity{0.050000}%
\pgfsetlinewidth{0.000000pt}%
\definecolor{currentstroke}{rgb}{1.000000,0.576471,0.309804}%
\pgfsetstrokecolor{currentstroke}%
\pgfsetdash{}{0pt}%
\pgfpathmoveto{\pgfqpoint{2.867029in}{3.407771in}}%
\pgfpathlineto{\pgfqpoint{2.944981in}{3.371174in}}%
\pgfpathlineto{\pgfqpoint{2.867029in}{3.334577in}}%
\pgfpathlineto{\pgfqpoint{2.789076in}{3.371174in}}%
\pgfpathlineto{\pgfqpoint{2.867029in}{3.407771in}}%
\pgfpathclose%
\pgfusepath{fill}%
\end{pgfscope}%
\begin{pgfscope}%
\pgfpathrectangle{\pgfqpoint{0.765000in}{0.660000in}}{\pgfqpoint{4.620000in}{4.620000in}}%
\pgfusepath{clip}%
\pgfsetbuttcap%
\pgfsetroundjoin%
\definecolor{currentfill}{rgb}{1.000000,0.576471,0.309804}%
\pgfsetfillcolor{currentfill}%
\pgfsetfillopacity{0.050000}%
\pgfsetlinewidth{0.000000pt}%
\definecolor{currentstroke}{rgb}{1.000000,0.576471,0.309804}%
\pgfsetstrokecolor{currentstroke}%
\pgfsetdash{}{0pt}%
\pgfpathmoveto{\pgfqpoint{2.867029in}{3.480774in}}%
\pgfpathlineto{\pgfqpoint{2.944981in}{3.444177in}}%
\pgfpathlineto{\pgfqpoint{2.867029in}{3.407581in}}%
\pgfpathlineto{\pgfqpoint{2.789076in}{3.444177in}}%
\pgfpathlineto{\pgfqpoint{2.867029in}{3.480774in}}%
\pgfpathclose%
\pgfusepath{fill}%
\end{pgfscope}%
\begin{pgfscope}%
\pgfpathrectangle{\pgfqpoint{0.765000in}{0.660000in}}{\pgfqpoint{4.620000in}{4.620000in}}%
\pgfusepath{clip}%
\pgfsetbuttcap%
\pgfsetroundjoin%
\definecolor{currentfill}{rgb}{1.000000,0.576471,0.309804}%
\pgfsetfillcolor{currentfill}%
\pgfsetfillopacity{0.050000}%
\pgfsetlinewidth{0.000000pt}%
\definecolor{currentstroke}{rgb}{1.000000,0.576471,0.309804}%
\pgfsetstrokecolor{currentstroke}%
\pgfsetdash{}{0pt}%
\pgfpathmoveto{\pgfqpoint{2.789076in}{3.371174in}}%
\pgfpathlineto{\pgfqpoint{2.789076in}{3.444177in}}%
\pgfpathlineto{\pgfqpoint{2.867029in}{3.407581in}}%
\pgfpathlineto{\pgfqpoint{2.867029in}{3.334577in}}%
\pgfpathlineto{\pgfqpoint{2.789076in}{3.371174in}}%
\pgfpathclose%
\pgfusepath{fill}%
\end{pgfscope}%
\begin{pgfscope}%
\pgfpathrectangle{\pgfqpoint{0.765000in}{0.660000in}}{\pgfqpoint{4.620000in}{4.620000in}}%
\pgfusepath{clip}%
\pgfsetbuttcap%
\pgfsetroundjoin%
\definecolor{currentfill}{rgb}{1.000000,0.576471,0.309804}%
\pgfsetfillcolor{currentfill}%
\pgfsetfillopacity{0.050000}%
\pgfsetlinewidth{0.000000pt}%
\definecolor{currentstroke}{rgb}{1.000000,0.576471,0.309804}%
\pgfsetstrokecolor{currentstroke}%
\pgfsetdash{}{0pt}%
\pgfpathmoveto{\pgfqpoint{2.944981in}{3.371174in}}%
\pgfpathlineto{\pgfqpoint{2.944981in}{3.444177in}}%
\pgfpathlineto{\pgfqpoint{2.867029in}{3.407581in}}%
\pgfpathlineto{\pgfqpoint{2.867029in}{3.334577in}}%
\pgfpathlineto{\pgfqpoint{2.944981in}{3.371174in}}%
\pgfpathclose%
\pgfusepath{fill}%
\end{pgfscope}%
\begin{pgfscope}%
\pgfpathrectangle{\pgfqpoint{0.765000in}{0.660000in}}{\pgfqpoint{4.620000in}{4.620000in}}%
\pgfusepath{clip}%
\pgfsetbuttcap%
\pgfsetroundjoin%
\definecolor{currentfill}{rgb}{1.000000,0.576471,0.309804}%
\pgfsetfillcolor{currentfill}%
\pgfsetfillopacity{0.050000}%
\pgfsetlinewidth{0.000000pt}%
\definecolor{currentstroke}{rgb}{1.000000,0.576471,0.309804}%
\pgfsetstrokecolor{currentstroke}%
\pgfsetdash{}{0pt}%
\pgfpathmoveto{\pgfqpoint{2.925869in}{3.718576in}}%
\pgfpathlineto{\pgfqpoint{2.925869in}{3.791579in}}%
\pgfpathlineto{\pgfqpoint{3.003822in}{3.754983in}}%
\pgfpathlineto{\pgfqpoint{3.003822in}{3.681979in}}%
\pgfpathlineto{\pgfqpoint{2.925869in}{3.718576in}}%
\pgfpathclose%
\pgfusepath{fill}%
\end{pgfscope}%
\begin{pgfscope}%
\pgfpathrectangle{\pgfqpoint{0.765000in}{0.660000in}}{\pgfqpoint{4.620000in}{4.620000in}}%
\pgfusepath{clip}%
\pgfsetbuttcap%
\pgfsetroundjoin%
\definecolor{currentfill}{rgb}{1.000000,0.576471,0.309804}%
\pgfsetfillcolor{currentfill}%
\pgfsetfillopacity{0.050000}%
\pgfsetlinewidth{0.000000pt}%
\definecolor{currentstroke}{rgb}{1.000000,0.576471,0.309804}%
\pgfsetstrokecolor{currentstroke}%
\pgfsetdash{}{0pt}%
\pgfpathmoveto{\pgfqpoint{2.925869in}{3.718576in}}%
\pgfpathlineto{\pgfqpoint{2.925869in}{3.791579in}}%
\pgfpathlineto{\pgfqpoint{2.847917in}{3.754983in}}%
\pgfpathlineto{\pgfqpoint{2.847917in}{3.681979in}}%
\pgfpathlineto{\pgfqpoint{2.925869in}{3.718576in}}%
\pgfpathclose%
\pgfusepath{fill}%
\end{pgfscope}%
\begin{pgfscope}%
\pgfpathrectangle{\pgfqpoint{0.765000in}{0.660000in}}{\pgfqpoint{4.620000in}{4.620000in}}%
\pgfusepath{clip}%
\pgfsetbuttcap%
\pgfsetroundjoin%
\definecolor{currentfill}{rgb}{1.000000,0.576471,0.309804}%
\pgfsetfillcolor{currentfill}%
\pgfsetfillopacity{0.050000}%
\pgfsetlinewidth{0.000000pt}%
\definecolor{currentstroke}{rgb}{1.000000,0.576471,0.309804}%
\pgfsetstrokecolor{currentstroke}%
\pgfsetdash{}{0pt}%
\pgfpathmoveto{\pgfqpoint{2.925869in}{3.718576in}}%
\pgfpathlineto{\pgfqpoint{3.003822in}{3.681979in}}%
\pgfpathlineto{\pgfqpoint{2.925869in}{3.645383in}}%
\pgfpathlineto{\pgfqpoint{2.847917in}{3.681979in}}%
\pgfpathlineto{\pgfqpoint{2.925869in}{3.718576in}}%
\pgfpathclose%
\pgfusepath{fill}%
\end{pgfscope}%
\begin{pgfscope}%
\pgfpathrectangle{\pgfqpoint{0.765000in}{0.660000in}}{\pgfqpoint{4.620000in}{4.620000in}}%
\pgfusepath{clip}%
\pgfsetbuttcap%
\pgfsetroundjoin%
\definecolor{currentfill}{rgb}{1.000000,0.576471,0.309804}%
\pgfsetfillcolor{currentfill}%
\pgfsetfillopacity{0.050000}%
\pgfsetlinewidth{0.000000pt}%
\definecolor{currentstroke}{rgb}{1.000000,0.576471,0.309804}%
\pgfsetstrokecolor{currentstroke}%
\pgfsetdash{}{0pt}%
\pgfpathmoveto{\pgfqpoint{2.925869in}{3.791579in}}%
\pgfpathlineto{\pgfqpoint{3.003822in}{3.754983in}}%
\pgfpathlineto{\pgfqpoint{2.925869in}{3.718386in}}%
\pgfpathlineto{\pgfqpoint{2.847917in}{3.754983in}}%
\pgfpathlineto{\pgfqpoint{2.925869in}{3.791579in}}%
\pgfpathclose%
\pgfusepath{fill}%
\end{pgfscope}%
\begin{pgfscope}%
\pgfpathrectangle{\pgfqpoint{0.765000in}{0.660000in}}{\pgfqpoint{4.620000in}{4.620000in}}%
\pgfusepath{clip}%
\pgfsetbuttcap%
\pgfsetroundjoin%
\definecolor{currentfill}{rgb}{1.000000,0.576471,0.309804}%
\pgfsetfillcolor{currentfill}%
\pgfsetfillopacity{0.050000}%
\pgfsetlinewidth{0.000000pt}%
\definecolor{currentstroke}{rgb}{1.000000,0.576471,0.309804}%
\pgfsetstrokecolor{currentstroke}%
\pgfsetdash{}{0pt}%
\pgfpathmoveto{\pgfqpoint{2.847917in}{3.681979in}}%
\pgfpathlineto{\pgfqpoint{2.847917in}{3.754983in}}%
\pgfpathlineto{\pgfqpoint{2.925869in}{3.718386in}}%
\pgfpathlineto{\pgfqpoint{2.925869in}{3.645383in}}%
\pgfpathlineto{\pgfqpoint{2.847917in}{3.681979in}}%
\pgfpathclose%
\pgfusepath{fill}%
\end{pgfscope}%
\begin{pgfscope}%
\pgfpathrectangle{\pgfqpoint{0.765000in}{0.660000in}}{\pgfqpoint{4.620000in}{4.620000in}}%
\pgfusepath{clip}%
\pgfsetbuttcap%
\pgfsetroundjoin%
\definecolor{currentfill}{rgb}{1.000000,0.576471,0.309804}%
\pgfsetfillcolor{currentfill}%
\pgfsetfillopacity{0.050000}%
\pgfsetlinewidth{0.000000pt}%
\definecolor{currentstroke}{rgb}{1.000000,0.576471,0.309804}%
\pgfsetstrokecolor{currentstroke}%
\pgfsetdash{}{0pt}%
\pgfpathmoveto{\pgfqpoint{3.003822in}{3.681979in}}%
\pgfpathlineto{\pgfqpoint{3.003822in}{3.754983in}}%
\pgfpathlineto{\pgfqpoint{2.925869in}{3.718386in}}%
\pgfpathlineto{\pgfqpoint{2.925869in}{3.645383in}}%
\pgfpathlineto{\pgfqpoint{3.003822in}{3.681979in}}%
\pgfpathclose%
\pgfusepath{fill}%
\end{pgfscope}%
\begin{pgfscope}%
\pgfpathrectangle{\pgfqpoint{0.765000in}{0.660000in}}{\pgfqpoint{4.620000in}{4.620000in}}%
\pgfusepath{clip}%
\pgfsetbuttcap%
\pgfsetroundjoin%
\definecolor{currentfill}{rgb}{1.000000,0.576471,0.309804}%
\pgfsetfillcolor{currentfill}%
\pgfsetfillopacity{0.050000}%
\pgfsetlinewidth{0.000000pt}%
\definecolor{currentstroke}{rgb}{1.000000,0.576471,0.309804}%
\pgfsetstrokecolor{currentstroke}%
\pgfsetdash{}{0pt}%
\pgfpathmoveto{\pgfqpoint{2.782812in}{3.874379in}}%
\pgfpathlineto{\pgfqpoint{2.782812in}{3.947382in}}%
\pgfpathlineto{\pgfqpoint{2.860765in}{3.910786in}}%
\pgfpathlineto{\pgfqpoint{2.860765in}{3.837782in}}%
\pgfpathlineto{\pgfqpoint{2.782812in}{3.874379in}}%
\pgfpathclose%
\pgfusepath{fill}%
\end{pgfscope}%
\begin{pgfscope}%
\pgfpathrectangle{\pgfqpoint{0.765000in}{0.660000in}}{\pgfqpoint{4.620000in}{4.620000in}}%
\pgfusepath{clip}%
\pgfsetbuttcap%
\pgfsetroundjoin%
\definecolor{currentfill}{rgb}{1.000000,0.576471,0.309804}%
\pgfsetfillcolor{currentfill}%
\pgfsetfillopacity{0.050000}%
\pgfsetlinewidth{0.000000pt}%
\definecolor{currentstroke}{rgb}{1.000000,0.576471,0.309804}%
\pgfsetstrokecolor{currentstroke}%
\pgfsetdash{}{0pt}%
\pgfpathmoveto{\pgfqpoint{2.782812in}{3.874379in}}%
\pgfpathlineto{\pgfqpoint{2.782812in}{3.947382in}}%
\pgfpathlineto{\pgfqpoint{2.704860in}{3.910786in}}%
\pgfpathlineto{\pgfqpoint{2.704860in}{3.837782in}}%
\pgfpathlineto{\pgfqpoint{2.782812in}{3.874379in}}%
\pgfpathclose%
\pgfusepath{fill}%
\end{pgfscope}%
\begin{pgfscope}%
\pgfpathrectangle{\pgfqpoint{0.765000in}{0.660000in}}{\pgfqpoint{4.620000in}{4.620000in}}%
\pgfusepath{clip}%
\pgfsetbuttcap%
\pgfsetroundjoin%
\definecolor{currentfill}{rgb}{1.000000,0.576471,0.309804}%
\pgfsetfillcolor{currentfill}%
\pgfsetfillopacity{0.050000}%
\pgfsetlinewidth{0.000000pt}%
\definecolor{currentstroke}{rgb}{1.000000,0.576471,0.309804}%
\pgfsetstrokecolor{currentstroke}%
\pgfsetdash{}{0pt}%
\pgfpathmoveto{\pgfqpoint{2.782812in}{3.874379in}}%
\pgfpathlineto{\pgfqpoint{2.860765in}{3.837782in}}%
\pgfpathlineto{\pgfqpoint{2.782812in}{3.801186in}}%
\pgfpathlineto{\pgfqpoint{2.704860in}{3.837782in}}%
\pgfpathlineto{\pgfqpoint{2.782812in}{3.874379in}}%
\pgfpathclose%
\pgfusepath{fill}%
\end{pgfscope}%
\begin{pgfscope}%
\pgfpathrectangle{\pgfqpoint{0.765000in}{0.660000in}}{\pgfqpoint{4.620000in}{4.620000in}}%
\pgfusepath{clip}%
\pgfsetbuttcap%
\pgfsetroundjoin%
\definecolor{currentfill}{rgb}{1.000000,0.576471,0.309804}%
\pgfsetfillcolor{currentfill}%
\pgfsetfillopacity{0.050000}%
\pgfsetlinewidth{0.000000pt}%
\definecolor{currentstroke}{rgb}{1.000000,0.576471,0.309804}%
\pgfsetstrokecolor{currentstroke}%
\pgfsetdash{}{0pt}%
\pgfpathmoveto{\pgfqpoint{2.782812in}{3.947382in}}%
\pgfpathlineto{\pgfqpoint{2.860765in}{3.910786in}}%
\pgfpathlineto{\pgfqpoint{2.782812in}{3.874189in}}%
\pgfpathlineto{\pgfqpoint{2.704860in}{3.910786in}}%
\pgfpathlineto{\pgfqpoint{2.782812in}{3.947382in}}%
\pgfpathclose%
\pgfusepath{fill}%
\end{pgfscope}%
\begin{pgfscope}%
\pgfpathrectangle{\pgfqpoint{0.765000in}{0.660000in}}{\pgfqpoint{4.620000in}{4.620000in}}%
\pgfusepath{clip}%
\pgfsetbuttcap%
\pgfsetroundjoin%
\definecolor{currentfill}{rgb}{1.000000,0.576471,0.309804}%
\pgfsetfillcolor{currentfill}%
\pgfsetfillopacity{0.050000}%
\pgfsetlinewidth{0.000000pt}%
\definecolor{currentstroke}{rgb}{1.000000,0.576471,0.309804}%
\pgfsetstrokecolor{currentstroke}%
\pgfsetdash{}{0pt}%
\pgfpathmoveto{\pgfqpoint{2.704860in}{3.837782in}}%
\pgfpathlineto{\pgfqpoint{2.704860in}{3.910786in}}%
\pgfpathlineto{\pgfqpoint{2.782812in}{3.874189in}}%
\pgfpathlineto{\pgfqpoint{2.782812in}{3.801186in}}%
\pgfpathlineto{\pgfqpoint{2.704860in}{3.837782in}}%
\pgfpathclose%
\pgfusepath{fill}%
\end{pgfscope}%
\begin{pgfscope}%
\pgfpathrectangle{\pgfqpoint{0.765000in}{0.660000in}}{\pgfqpoint{4.620000in}{4.620000in}}%
\pgfusepath{clip}%
\pgfsetbuttcap%
\pgfsetroundjoin%
\definecolor{currentfill}{rgb}{1.000000,0.576471,0.309804}%
\pgfsetfillcolor{currentfill}%
\pgfsetfillopacity{0.050000}%
\pgfsetlinewidth{0.000000pt}%
\definecolor{currentstroke}{rgb}{1.000000,0.576471,0.309804}%
\pgfsetstrokecolor{currentstroke}%
\pgfsetdash{}{0pt}%
\pgfpathmoveto{\pgfqpoint{2.860765in}{3.837782in}}%
\pgfpathlineto{\pgfqpoint{2.860765in}{3.910786in}}%
\pgfpathlineto{\pgfqpoint{2.782812in}{3.874189in}}%
\pgfpathlineto{\pgfqpoint{2.782812in}{3.801186in}}%
\pgfpathlineto{\pgfqpoint{2.860765in}{3.837782in}}%
\pgfpathclose%
\pgfusepath{fill}%
\end{pgfscope}%
\begin{pgfscope}%
\pgfpathrectangle{\pgfqpoint{0.765000in}{0.660000in}}{\pgfqpoint{4.620000in}{4.620000in}}%
\pgfusepath{clip}%
\pgfsetbuttcap%
\pgfsetroundjoin%
\definecolor{currentfill}{rgb}{1.000000,0.576471,0.309804}%
\pgfsetfillcolor{currentfill}%
\pgfsetfillopacity{0.050000}%
\pgfsetlinewidth{0.000000pt}%
\definecolor{currentstroke}{rgb}{1.000000,0.576471,0.309804}%
\pgfsetstrokecolor{currentstroke}%
\pgfsetdash{}{0pt}%
\pgfpathmoveto{\pgfqpoint{2.516361in}{3.336942in}}%
\pgfpathlineto{\pgfqpoint{2.516361in}{3.409945in}}%
\pgfpathlineto{\pgfqpoint{2.594313in}{3.373349in}}%
\pgfpathlineto{\pgfqpoint{2.594313in}{3.300345in}}%
\pgfpathlineto{\pgfqpoint{2.516361in}{3.336942in}}%
\pgfpathclose%
\pgfusepath{fill}%
\end{pgfscope}%
\begin{pgfscope}%
\pgfpathrectangle{\pgfqpoint{0.765000in}{0.660000in}}{\pgfqpoint{4.620000in}{4.620000in}}%
\pgfusepath{clip}%
\pgfsetbuttcap%
\pgfsetroundjoin%
\definecolor{currentfill}{rgb}{1.000000,0.576471,0.309804}%
\pgfsetfillcolor{currentfill}%
\pgfsetfillopacity{0.050000}%
\pgfsetlinewidth{0.000000pt}%
\definecolor{currentstroke}{rgb}{1.000000,0.576471,0.309804}%
\pgfsetstrokecolor{currentstroke}%
\pgfsetdash{}{0pt}%
\pgfpathmoveto{\pgfqpoint{2.516361in}{3.336942in}}%
\pgfpathlineto{\pgfqpoint{2.516361in}{3.409945in}}%
\pgfpathlineto{\pgfqpoint{2.438408in}{3.373349in}}%
\pgfpathlineto{\pgfqpoint{2.438408in}{3.300345in}}%
\pgfpathlineto{\pgfqpoint{2.516361in}{3.336942in}}%
\pgfpathclose%
\pgfusepath{fill}%
\end{pgfscope}%
\begin{pgfscope}%
\pgfpathrectangle{\pgfqpoint{0.765000in}{0.660000in}}{\pgfqpoint{4.620000in}{4.620000in}}%
\pgfusepath{clip}%
\pgfsetbuttcap%
\pgfsetroundjoin%
\definecolor{currentfill}{rgb}{1.000000,0.576471,0.309804}%
\pgfsetfillcolor{currentfill}%
\pgfsetfillopacity{0.050000}%
\pgfsetlinewidth{0.000000pt}%
\definecolor{currentstroke}{rgb}{1.000000,0.576471,0.309804}%
\pgfsetstrokecolor{currentstroke}%
\pgfsetdash{}{0pt}%
\pgfpathmoveto{\pgfqpoint{2.516361in}{3.336942in}}%
\pgfpathlineto{\pgfqpoint{2.594313in}{3.300345in}}%
\pgfpathlineto{\pgfqpoint{2.516361in}{3.263749in}}%
\pgfpathlineto{\pgfqpoint{2.438408in}{3.300345in}}%
\pgfpathlineto{\pgfqpoint{2.516361in}{3.336942in}}%
\pgfpathclose%
\pgfusepath{fill}%
\end{pgfscope}%
\begin{pgfscope}%
\pgfpathrectangle{\pgfqpoint{0.765000in}{0.660000in}}{\pgfqpoint{4.620000in}{4.620000in}}%
\pgfusepath{clip}%
\pgfsetbuttcap%
\pgfsetroundjoin%
\definecolor{currentfill}{rgb}{1.000000,0.576471,0.309804}%
\pgfsetfillcolor{currentfill}%
\pgfsetfillopacity{0.050000}%
\pgfsetlinewidth{0.000000pt}%
\definecolor{currentstroke}{rgb}{1.000000,0.576471,0.309804}%
\pgfsetstrokecolor{currentstroke}%
\pgfsetdash{}{0pt}%
\pgfpathmoveto{\pgfqpoint{2.516361in}{3.409945in}}%
\pgfpathlineto{\pgfqpoint{2.594313in}{3.373349in}}%
\pgfpathlineto{\pgfqpoint{2.516361in}{3.336752in}}%
\pgfpathlineto{\pgfqpoint{2.438408in}{3.373349in}}%
\pgfpathlineto{\pgfqpoint{2.516361in}{3.409945in}}%
\pgfpathclose%
\pgfusepath{fill}%
\end{pgfscope}%
\begin{pgfscope}%
\pgfpathrectangle{\pgfqpoint{0.765000in}{0.660000in}}{\pgfqpoint{4.620000in}{4.620000in}}%
\pgfusepath{clip}%
\pgfsetbuttcap%
\pgfsetroundjoin%
\definecolor{currentfill}{rgb}{1.000000,0.576471,0.309804}%
\pgfsetfillcolor{currentfill}%
\pgfsetfillopacity{0.050000}%
\pgfsetlinewidth{0.000000pt}%
\definecolor{currentstroke}{rgb}{1.000000,0.576471,0.309804}%
\pgfsetstrokecolor{currentstroke}%
\pgfsetdash{}{0pt}%
\pgfpathmoveto{\pgfqpoint{2.438408in}{3.300345in}}%
\pgfpathlineto{\pgfqpoint{2.438408in}{3.373349in}}%
\pgfpathlineto{\pgfqpoint{2.516361in}{3.336752in}}%
\pgfpathlineto{\pgfqpoint{2.516361in}{3.263749in}}%
\pgfpathlineto{\pgfqpoint{2.438408in}{3.300345in}}%
\pgfpathclose%
\pgfusepath{fill}%
\end{pgfscope}%
\begin{pgfscope}%
\pgfpathrectangle{\pgfqpoint{0.765000in}{0.660000in}}{\pgfqpoint{4.620000in}{4.620000in}}%
\pgfusepath{clip}%
\pgfsetbuttcap%
\pgfsetroundjoin%
\definecolor{currentfill}{rgb}{1.000000,0.576471,0.309804}%
\pgfsetfillcolor{currentfill}%
\pgfsetfillopacity{0.050000}%
\pgfsetlinewidth{0.000000pt}%
\definecolor{currentstroke}{rgb}{1.000000,0.576471,0.309804}%
\pgfsetstrokecolor{currentstroke}%
\pgfsetdash{}{0pt}%
\pgfpathmoveto{\pgfqpoint{2.594313in}{3.300345in}}%
\pgfpathlineto{\pgfqpoint{2.594313in}{3.373349in}}%
\pgfpathlineto{\pgfqpoint{2.516361in}{3.336752in}}%
\pgfpathlineto{\pgfqpoint{2.516361in}{3.263749in}}%
\pgfpathlineto{\pgfqpoint{2.594313in}{3.300345in}}%
\pgfpathclose%
\pgfusepath{fill}%
\end{pgfscope}%
\begin{pgfscope}%
\pgfpathrectangle{\pgfqpoint{0.765000in}{0.660000in}}{\pgfqpoint{4.620000in}{4.620000in}}%
\pgfusepath{clip}%
\pgfsetbuttcap%
\pgfsetroundjoin%
\definecolor{currentfill}{rgb}{1.000000,0.576471,0.309804}%
\pgfsetfillcolor{currentfill}%
\pgfsetfillopacity{0.050000}%
\pgfsetlinewidth{0.000000pt}%
\definecolor{currentstroke}{rgb}{1.000000,0.576471,0.309804}%
\pgfsetstrokecolor{currentstroke}%
\pgfsetdash{}{0pt}%
\pgfpathmoveto{\pgfqpoint{2.545699in}{3.168144in}}%
\pgfpathlineto{\pgfqpoint{2.545699in}{3.241147in}}%
\pgfpathlineto{\pgfqpoint{2.623652in}{3.204551in}}%
\pgfpathlineto{\pgfqpoint{2.623652in}{3.131547in}}%
\pgfpathlineto{\pgfqpoint{2.545699in}{3.168144in}}%
\pgfpathclose%
\pgfusepath{fill}%
\end{pgfscope}%
\begin{pgfscope}%
\pgfpathrectangle{\pgfqpoint{0.765000in}{0.660000in}}{\pgfqpoint{4.620000in}{4.620000in}}%
\pgfusepath{clip}%
\pgfsetbuttcap%
\pgfsetroundjoin%
\definecolor{currentfill}{rgb}{1.000000,0.576471,0.309804}%
\pgfsetfillcolor{currentfill}%
\pgfsetfillopacity{0.050000}%
\pgfsetlinewidth{0.000000pt}%
\definecolor{currentstroke}{rgb}{1.000000,0.576471,0.309804}%
\pgfsetstrokecolor{currentstroke}%
\pgfsetdash{}{0pt}%
\pgfpathmoveto{\pgfqpoint{2.545699in}{3.168144in}}%
\pgfpathlineto{\pgfqpoint{2.545699in}{3.241147in}}%
\pgfpathlineto{\pgfqpoint{2.467747in}{3.204551in}}%
\pgfpathlineto{\pgfqpoint{2.467747in}{3.131547in}}%
\pgfpathlineto{\pgfqpoint{2.545699in}{3.168144in}}%
\pgfpathclose%
\pgfusepath{fill}%
\end{pgfscope}%
\begin{pgfscope}%
\pgfpathrectangle{\pgfqpoint{0.765000in}{0.660000in}}{\pgfqpoint{4.620000in}{4.620000in}}%
\pgfusepath{clip}%
\pgfsetbuttcap%
\pgfsetroundjoin%
\definecolor{currentfill}{rgb}{1.000000,0.576471,0.309804}%
\pgfsetfillcolor{currentfill}%
\pgfsetfillopacity{0.050000}%
\pgfsetlinewidth{0.000000pt}%
\definecolor{currentstroke}{rgb}{1.000000,0.576471,0.309804}%
\pgfsetstrokecolor{currentstroke}%
\pgfsetdash{}{0pt}%
\pgfpathmoveto{\pgfqpoint{2.545699in}{3.168144in}}%
\pgfpathlineto{\pgfqpoint{2.623652in}{3.131547in}}%
\pgfpathlineto{\pgfqpoint{2.545699in}{3.094951in}}%
\pgfpathlineto{\pgfqpoint{2.467747in}{3.131547in}}%
\pgfpathlineto{\pgfqpoint{2.545699in}{3.168144in}}%
\pgfpathclose%
\pgfusepath{fill}%
\end{pgfscope}%
\begin{pgfscope}%
\pgfpathrectangle{\pgfqpoint{0.765000in}{0.660000in}}{\pgfqpoint{4.620000in}{4.620000in}}%
\pgfusepath{clip}%
\pgfsetbuttcap%
\pgfsetroundjoin%
\definecolor{currentfill}{rgb}{1.000000,0.576471,0.309804}%
\pgfsetfillcolor{currentfill}%
\pgfsetfillopacity{0.050000}%
\pgfsetlinewidth{0.000000pt}%
\definecolor{currentstroke}{rgb}{1.000000,0.576471,0.309804}%
\pgfsetstrokecolor{currentstroke}%
\pgfsetdash{}{0pt}%
\pgfpathmoveto{\pgfqpoint{2.545699in}{3.241147in}}%
\pgfpathlineto{\pgfqpoint{2.623652in}{3.204551in}}%
\pgfpathlineto{\pgfqpoint{2.545699in}{3.167954in}}%
\pgfpathlineto{\pgfqpoint{2.467747in}{3.204551in}}%
\pgfpathlineto{\pgfqpoint{2.545699in}{3.241147in}}%
\pgfpathclose%
\pgfusepath{fill}%
\end{pgfscope}%
\begin{pgfscope}%
\pgfpathrectangle{\pgfqpoint{0.765000in}{0.660000in}}{\pgfqpoint{4.620000in}{4.620000in}}%
\pgfusepath{clip}%
\pgfsetbuttcap%
\pgfsetroundjoin%
\definecolor{currentfill}{rgb}{1.000000,0.576471,0.309804}%
\pgfsetfillcolor{currentfill}%
\pgfsetfillopacity{0.050000}%
\pgfsetlinewidth{0.000000pt}%
\definecolor{currentstroke}{rgb}{1.000000,0.576471,0.309804}%
\pgfsetstrokecolor{currentstroke}%
\pgfsetdash{}{0pt}%
\pgfpathmoveto{\pgfqpoint{2.467747in}{3.131547in}}%
\pgfpathlineto{\pgfqpoint{2.467747in}{3.204551in}}%
\pgfpathlineto{\pgfqpoint{2.545699in}{3.167954in}}%
\pgfpathlineto{\pgfqpoint{2.545699in}{3.094951in}}%
\pgfpathlineto{\pgfqpoint{2.467747in}{3.131547in}}%
\pgfpathclose%
\pgfusepath{fill}%
\end{pgfscope}%
\begin{pgfscope}%
\pgfpathrectangle{\pgfqpoint{0.765000in}{0.660000in}}{\pgfqpoint{4.620000in}{4.620000in}}%
\pgfusepath{clip}%
\pgfsetbuttcap%
\pgfsetroundjoin%
\definecolor{currentfill}{rgb}{1.000000,0.576471,0.309804}%
\pgfsetfillcolor{currentfill}%
\pgfsetfillopacity{0.050000}%
\pgfsetlinewidth{0.000000pt}%
\definecolor{currentstroke}{rgb}{1.000000,0.576471,0.309804}%
\pgfsetstrokecolor{currentstroke}%
\pgfsetdash{}{0pt}%
\pgfpathmoveto{\pgfqpoint{2.623652in}{3.131547in}}%
\pgfpathlineto{\pgfqpoint{2.623652in}{3.204551in}}%
\pgfpathlineto{\pgfqpoint{2.545699in}{3.167954in}}%
\pgfpathlineto{\pgfqpoint{2.545699in}{3.094951in}}%
\pgfpathlineto{\pgfqpoint{2.623652in}{3.131547in}}%
\pgfpathclose%
\pgfusepath{fill}%
\end{pgfscope}%
\begin{pgfscope}%
\pgfpathrectangle{\pgfqpoint{0.765000in}{0.660000in}}{\pgfqpoint{4.620000in}{4.620000in}}%
\pgfusepath{clip}%
\pgfsetbuttcap%
\pgfsetroundjoin%
\definecolor{currentfill}{rgb}{1.000000,0.576471,0.309804}%
\pgfsetfillcolor{currentfill}%
\pgfsetfillopacity{0.050000}%
\pgfsetlinewidth{0.000000pt}%
\definecolor{currentstroke}{rgb}{1.000000,0.576471,0.309804}%
\pgfsetstrokecolor{currentstroke}%
\pgfsetdash{}{0pt}%
\pgfpathmoveto{\pgfqpoint{2.830154in}{3.438707in}}%
\pgfpathlineto{\pgfqpoint{2.830154in}{3.511710in}}%
\pgfpathlineto{\pgfqpoint{2.908106in}{3.475113in}}%
\pgfpathlineto{\pgfqpoint{2.908106in}{3.402110in}}%
\pgfpathlineto{\pgfqpoint{2.830154in}{3.438707in}}%
\pgfpathclose%
\pgfusepath{fill}%
\end{pgfscope}%
\begin{pgfscope}%
\pgfpathrectangle{\pgfqpoint{0.765000in}{0.660000in}}{\pgfqpoint{4.620000in}{4.620000in}}%
\pgfusepath{clip}%
\pgfsetbuttcap%
\pgfsetroundjoin%
\definecolor{currentfill}{rgb}{1.000000,0.576471,0.309804}%
\pgfsetfillcolor{currentfill}%
\pgfsetfillopacity{0.050000}%
\pgfsetlinewidth{0.000000pt}%
\definecolor{currentstroke}{rgb}{1.000000,0.576471,0.309804}%
\pgfsetstrokecolor{currentstroke}%
\pgfsetdash{}{0pt}%
\pgfpathmoveto{\pgfqpoint{2.830154in}{3.438707in}}%
\pgfpathlineto{\pgfqpoint{2.830154in}{3.511710in}}%
\pgfpathlineto{\pgfqpoint{2.752201in}{3.475113in}}%
\pgfpathlineto{\pgfqpoint{2.752201in}{3.402110in}}%
\pgfpathlineto{\pgfqpoint{2.830154in}{3.438707in}}%
\pgfpathclose%
\pgfusepath{fill}%
\end{pgfscope}%
\begin{pgfscope}%
\pgfpathrectangle{\pgfqpoint{0.765000in}{0.660000in}}{\pgfqpoint{4.620000in}{4.620000in}}%
\pgfusepath{clip}%
\pgfsetbuttcap%
\pgfsetroundjoin%
\definecolor{currentfill}{rgb}{1.000000,0.576471,0.309804}%
\pgfsetfillcolor{currentfill}%
\pgfsetfillopacity{0.050000}%
\pgfsetlinewidth{0.000000pt}%
\definecolor{currentstroke}{rgb}{1.000000,0.576471,0.309804}%
\pgfsetstrokecolor{currentstroke}%
\pgfsetdash{}{0pt}%
\pgfpathmoveto{\pgfqpoint{2.830154in}{3.438707in}}%
\pgfpathlineto{\pgfqpoint{2.908106in}{3.402110in}}%
\pgfpathlineto{\pgfqpoint{2.830154in}{3.365513in}}%
\pgfpathlineto{\pgfqpoint{2.752201in}{3.402110in}}%
\pgfpathlineto{\pgfqpoint{2.830154in}{3.438707in}}%
\pgfpathclose%
\pgfusepath{fill}%
\end{pgfscope}%
\begin{pgfscope}%
\pgfpathrectangle{\pgfqpoint{0.765000in}{0.660000in}}{\pgfqpoint{4.620000in}{4.620000in}}%
\pgfusepath{clip}%
\pgfsetbuttcap%
\pgfsetroundjoin%
\definecolor{currentfill}{rgb}{1.000000,0.576471,0.309804}%
\pgfsetfillcolor{currentfill}%
\pgfsetfillopacity{0.050000}%
\pgfsetlinewidth{0.000000pt}%
\definecolor{currentstroke}{rgb}{1.000000,0.576471,0.309804}%
\pgfsetstrokecolor{currentstroke}%
\pgfsetdash{}{0pt}%
\pgfpathmoveto{\pgfqpoint{2.830154in}{3.511710in}}%
\pgfpathlineto{\pgfqpoint{2.908106in}{3.475113in}}%
\pgfpathlineto{\pgfqpoint{2.830154in}{3.438517in}}%
\pgfpathlineto{\pgfqpoint{2.752201in}{3.475113in}}%
\pgfpathlineto{\pgfqpoint{2.830154in}{3.511710in}}%
\pgfpathclose%
\pgfusepath{fill}%
\end{pgfscope}%
\begin{pgfscope}%
\pgfpathrectangle{\pgfqpoint{0.765000in}{0.660000in}}{\pgfqpoint{4.620000in}{4.620000in}}%
\pgfusepath{clip}%
\pgfsetbuttcap%
\pgfsetroundjoin%
\definecolor{currentfill}{rgb}{1.000000,0.576471,0.309804}%
\pgfsetfillcolor{currentfill}%
\pgfsetfillopacity{0.050000}%
\pgfsetlinewidth{0.000000pt}%
\definecolor{currentstroke}{rgb}{1.000000,0.576471,0.309804}%
\pgfsetstrokecolor{currentstroke}%
\pgfsetdash{}{0pt}%
\pgfpathmoveto{\pgfqpoint{2.752201in}{3.402110in}}%
\pgfpathlineto{\pgfqpoint{2.752201in}{3.475113in}}%
\pgfpathlineto{\pgfqpoint{2.830154in}{3.438517in}}%
\pgfpathlineto{\pgfqpoint{2.830154in}{3.365513in}}%
\pgfpathlineto{\pgfqpoint{2.752201in}{3.402110in}}%
\pgfpathclose%
\pgfusepath{fill}%
\end{pgfscope}%
\begin{pgfscope}%
\pgfpathrectangle{\pgfqpoint{0.765000in}{0.660000in}}{\pgfqpoint{4.620000in}{4.620000in}}%
\pgfusepath{clip}%
\pgfsetbuttcap%
\pgfsetroundjoin%
\definecolor{currentfill}{rgb}{1.000000,0.576471,0.309804}%
\pgfsetfillcolor{currentfill}%
\pgfsetfillopacity{0.050000}%
\pgfsetlinewidth{0.000000pt}%
\definecolor{currentstroke}{rgb}{1.000000,0.576471,0.309804}%
\pgfsetstrokecolor{currentstroke}%
\pgfsetdash{}{0pt}%
\pgfpathmoveto{\pgfqpoint{2.908106in}{3.402110in}}%
\pgfpathlineto{\pgfqpoint{2.908106in}{3.475113in}}%
\pgfpathlineto{\pgfqpoint{2.830154in}{3.438517in}}%
\pgfpathlineto{\pgfqpoint{2.830154in}{3.365513in}}%
\pgfpathlineto{\pgfqpoint{2.908106in}{3.402110in}}%
\pgfpathclose%
\pgfusepath{fill}%
\end{pgfscope}%
\begin{pgfscope}%
\pgfpathrectangle{\pgfqpoint{0.765000in}{0.660000in}}{\pgfqpoint{4.620000in}{4.620000in}}%
\pgfusepath{clip}%
\pgfsetbuttcap%
\pgfsetroundjoin%
\definecolor{currentfill}{rgb}{0.176471,0.192157,0.258824}%
\pgfsetfillcolor{currentfill}%
\pgfsetfillopacity{0.050000}%
\pgfsetlinewidth{0.000000pt}%
\definecolor{currentstroke}{rgb}{0.176471,0.192157,0.258824}%
\pgfsetstrokecolor{currentstroke}%
\pgfsetdash{}{0pt}%
\pgfpathmoveto{\pgfqpoint{3.136889in}{2.948657in}}%
\pgfpathlineto{\pgfqpoint{3.136889in}{3.021660in}}%
\pgfpathlineto{\pgfqpoint{3.214842in}{2.985063in}}%
\pgfpathlineto{\pgfqpoint{3.214842in}{2.912060in}}%
\pgfpathlineto{\pgfqpoint{3.136889in}{2.948657in}}%
\pgfpathclose%
\pgfusepath{fill}%
\end{pgfscope}%
\begin{pgfscope}%
\pgfpathrectangle{\pgfqpoint{0.765000in}{0.660000in}}{\pgfqpoint{4.620000in}{4.620000in}}%
\pgfusepath{clip}%
\pgfsetbuttcap%
\pgfsetroundjoin%
\definecolor{currentfill}{rgb}{0.176471,0.192157,0.258824}%
\pgfsetfillcolor{currentfill}%
\pgfsetfillopacity{0.050000}%
\pgfsetlinewidth{0.000000pt}%
\definecolor{currentstroke}{rgb}{0.176471,0.192157,0.258824}%
\pgfsetstrokecolor{currentstroke}%
\pgfsetdash{}{0pt}%
\pgfpathmoveto{\pgfqpoint{3.136889in}{2.948657in}}%
\pgfpathlineto{\pgfqpoint{3.136889in}{3.021660in}}%
\pgfpathlineto{\pgfqpoint{3.058937in}{2.985063in}}%
\pgfpathlineto{\pgfqpoint{3.058937in}{2.912060in}}%
\pgfpathlineto{\pgfqpoint{3.136889in}{2.948657in}}%
\pgfpathclose%
\pgfusepath{fill}%
\end{pgfscope}%
\begin{pgfscope}%
\pgfpathrectangle{\pgfqpoint{0.765000in}{0.660000in}}{\pgfqpoint{4.620000in}{4.620000in}}%
\pgfusepath{clip}%
\pgfsetbuttcap%
\pgfsetroundjoin%
\definecolor{currentfill}{rgb}{0.176471,0.192157,0.258824}%
\pgfsetfillcolor{currentfill}%
\pgfsetfillopacity{0.050000}%
\pgfsetlinewidth{0.000000pt}%
\definecolor{currentstroke}{rgb}{0.176471,0.192157,0.258824}%
\pgfsetstrokecolor{currentstroke}%
\pgfsetdash{}{0pt}%
\pgfpathmoveto{\pgfqpoint{3.136889in}{2.948657in}}%
\pgfpathlineto{\pgfqpoint{3.214842in}{2.912060in}}%
\pgfpathlineto{\pgfqpoint{3.136889in}{2.875463in}}%
\pgfpathlineto{\pgfqpoint{3.058937in}{2.912060in}}%
\pgfpathlineto{\pgfqpoint{3.136889in}{2.948657in}}%
\pgfpathclose%
\pgfusepath{fill}%
\end{pgfscope}%
\begin{pgfscope}%
\pgfpathrectangle{\pgfqpoint{0.765000in}{0.660000in}}{\pgfqpoint{4.620000in}{4.620000in}}%
\pgfusepath{clip}%
\pgfsetbuttcap%
\pgfsetroundjoin%
\definecolor{currentfill}{rgb}{0.176471,0.192157,0.258824}%
\pgfsetfillcolor{currentfill}%
\pgfsetfillopacity{0.050000}%
\pgfsetlinewidth{0.000000pt}%
\definecolor{currentstroke}{rgb}{0.176471,0.192157,0.258824}%
\pgfsetstrokecolor{currentstroke}%
\pgfsetdash{}{0pt}%
\pgfpathmoveto{\pgfqpoint{3.136889in}{3.021660in}}%
\pgfpathlineto{\pgfqpoint{3.214842in}{2.985063in}}%
\pgfpathlineto{\pgfqpoint{3.136889in}{2.948467in}}%
\pgfpathlineto{\pgfqpoint{3.058937in}{2.985063in}}%
\pgfpathlineto{\pgfqpoint{3.136889in}{3.021660in}}%
\pgfpathclose%
\pgfusepath{fill}%
\end{pgfscope}%
\begin{pgfscope}%
\pgfpathrectangle{\pgfqpoint{0.765000in}{0.660000in}}{\pgfqpoint{4.620000in}{4.620000in}}%
\pgfusepath{clip}%
\pgfsetbuttcap%
\pgfsetroundjoin%
\definecolor{currentfill}{rgb}{0.176471,0.192157,0.258824}%
\pgfsetfillcolor{currentfill}%
\pgfsetfillopacity{0.050000}%
\pgfsetlinewidth{0.000000pt}%
\definecolor{currentstroke}{rgb}{0.176471,0.192157,0.258824}%
\pgfsetstrokecolor{currentstroke}%
\pgfsetdash{}{0pt}%
\pgfpathmoveto{\pgfqpoint{3.058937in}{2.912060in}}%
\pgfpathlineto{\pgfqpoint{3.058937in}{2.985063in}}%
\pgfpathlineto{\pgfqpoint{3.136889in}{2.948467in}}%
\pgfpathlineto{\pgfqpoint{3.136889in}{2.875463in}}%
\pgfpathlineto{\pgfqpoint{3.058937in}{2.912060in}}%
\pgfpathclose%
\pgfusepath{fill}%
\end{pgfscope}%
\begin{pgfscope}%
\pgfpathrectangle{\pgfqpoint{0.765000in}{0.660000in}}{\pgfqpoint{4.620000in}{4.620000in}}%
\pgfusepath{clip}%
\pgfsetbuttcap%
\pgfsetroundjoin%
\definecolor{currentfill}{rgb}{0.176471,0.192157,0.258824}%
\pgfsetfillcolor{currentfill}%
\pgfsetfillopacity{0.050000}%
\pgfsetlinewidth{0.000000pt}%
\definecolor{currentstroke}{rgb}{0.176471,0.192157,0.258824}%
\pgfsetstrokecolor{currentstroke}%
\pgfsetdash{}{0pt}%
\pgfpathmoveto{\pgfqpoint{3.214842in}{2.912060in}}%
\pgfpathlineto{\pgfqpoint{3.214842in}{2.985063in}}%
\pgfpathlineto{\pgfqpoint{3.136889in}{2.948467in}}%
\pgfpathlineto{\pgfqpoint{3.136889in}{2.875463in}}%
\pgfpathlineto{\pgfqpoint{3.214842in}{2.912060in}}%
\pgfpathclose%
\pgfusepath{fill}%
\end{pgfscope}%
\begin{pgfscope}%
\pgfpathrectangle{\pgfqpoint{0.765000in}{0.660000in}}{\pgfqpoint{4.620000in}{4.620000in}}%
\pgfusepath{clip}%
\pgfsetbuttcap%
\pgfsetroundjoin%
\definecolor{currentfill}{rgb}{1.000000,0.576471,0.309804}%
\pgfsetfillcolor{currentfill}%
\pgfsetfillopacity{0.050000}%
\pgfsetlinewidth{0.000000pt}%
\definecolor{currentstroke}{rgb}{1.000000,0.576471,0.309804}%
\pgfsetstrokecolor{currentstroke}%
\pgfsetdash{}{0pt}%
\pgfpathmoveto{\pgfqpoint{3.788294in}{3.302280in}}%
\pgfpathlineto{\pgfqpoint{3.788294in}{3.375284in}}%
\pgfpathlineto{\pgfqpoint{3.866247in}{3.338687in}}%
\pgfpathlineto{\pgfqpoint{3.866247in}{3.265684in}}%
\pgfpathlineto{\pgfqpoint{3.788294in}{3.302280in}}%
\pgfpathclose%
\pgfusepath{fill}%
\end{pgfscope}%
\begin{pgfscope}%
\pgfpathrectangle{\pgfqpoint{0.765000in}{0.660000in}}{\pgfqpoint{4.620000in}{4.620000in}}%
\pgfusepath{clip}%
\pgfsetbuttcap%
\pgfsetroundjoin%
\definecolor{currentfill}{rgb}{1.000000,0.576471,0.309804}%
\pgfsetfillcolor{currentfill}%
\pgfsetfillopacity{0.050000}%
\pgfsetlinewidth{0.000000pt}%
\definecolor{currentstroke}{rgb}{1.000000,0.576471,0.309804}%
\pgfsetstrokecolor{currentstroke}%
\pgfsetdash{}{0pt}%
\pgfpathmoveto{\pgfqpoint{3.788294in}{3.302280in}}%
\pgfpathlineto{\pgfqpoint{3.788294in}{3.375284in}}%
\pgfpathlineto{\pgfqpoint{3.710341in}{3.338687in}}%
\pgfpathlineto{\pgfqpoint{3.710341in}{3.265684in}}%
\pgfpathlineto{\pgfqpoint{3.788294in}{3.302280in}}%
\pgfpathclose%
\pgfusepath{fill}%
\end{pgfscope}%
\begin{pgfscope}%
\pgfpathrectangle{\pgfqpoint{0.765000in}{0.660000in}}{\pgfqpoint{4.620000in}{4.620000in}}%
\pgfusepath{clip}%
\pgfsetbuttcap%
\pgfsetroundjoin%
\definecolor{currentfill}{rgb}{1.000000,0.576471,0.309804}%
\pgfsetfillcolor{currentfill}%
\pgfsetfillopacity{0.050000}%
\pgfsetlinewidth{0.000000pt}%
\definecolor{currentstroke}{rgb}{1.000000,0.576471,0.309804}%
\pgfsetstrokecolor{currentstroke}%
\pgfsetdash{}{0pt}%
\pgfpathmoveto{\pgfqpoint{3.788294in}{3.302280in}}%
\pgfpathlineto{\pgfqpoint{3.866247in}{3.265684in}}%
\pgfpathlineto{\pgfqpoint{3.788294in}{3.229087in}}%
\pgfpathlineto{\pgfqpoint{3.710341in}{3.265684in}}%
\pgfpathlineto{\pgfqpoint{3.788294in}{3.302280in}}%
\pgfpathclose%
\pgfusepath{fill}%
\end{pgfscope}%
\begin{pgfscope}%
\pgfpathrectangle{\pgfqpoint{0.765000in}{0.660000in}}{\pgfqpoint{4.620000in}{4.620000in}}%
\pgfusepath{clip}%
\pgfsetbuttcap%
\pgfsetroundjoin%
\definecolor{currentfill}{rgb}{1.000000,0.576471,0.309804}%
\pgfsetfillcolor{currentfill}%
\pgfsetfillopacity{0.050000}%
\pgfsetlinewidth{0.000000pt}%
\definecolor{currentstroke}{rgb}{1.000000,0.576471,0.309804}%
\pgfsetstrokecolor{currentstroke}%
\pgfsetdash{}{0pt}%
\pgfpathmoveto{\pgfqpoint{3.788294in}{3.375284in}}%
\pgfpathlineto{\pgfqpoint{3.866247in}{3.338687in}}%
\pgfpathlineto{\pgfqpoint{3.788294in}{3.302090in}}%
\pgfpathlineto{\pgfqpoint{3.710341in}{3.338687in}}%
\pgfpathlineto{\pgfqpoint{3.788294in}{3.375284in}}%
\pgfpathclose%
\pgfusepath{fill}%
\end{pgfscope}%
\begin{pgfscope}%
\pgfpathrectangle{\pgfqpoint{0.765000in}{0.660000in}}{\pgfqpoint{4.620000in}{4.620000in}}%
\pgfusepath{clip}%
\pgfsetbuttcap%
\pgfsetroundjoin%
\definecolor{currentfill}{rgb}{1.000000,0.576471,0.309804}%
\pgfsetfillcolor{currentfill}%
\pgfsetfillopacity{0.050000}%
\pgfsetlinewidth{0.000000pt}%
\definecolor{currentstroke}{rgb}{1.000000,0.576471,0.309804}%
\pgfsetstrokecolor{currentstroke}%
\pgfsetdash{}{0pt}%
\pgfpathmoveto{\pgfqpoint{3.710341in}{3.265684in}}%
\pgfpathlineto{\pgfqpoint{3.710341in}{3.338687in}}%
\pgfpathlineto{\pgfqpoint{3.788294in}{3.302090in}}%
\pgfpathlineto{\pgfqpoint{3.788294in}{3.229087in}}%
\pgfpathlineto{\pgfqpoint{3.710341in}{3.265684in}}%
\pgfpathclose%
\pgfusepath{fill}%
\end{pgfscope}%
\begin{pgfscope}%
\pgfpathrectangle{\pgfqpoint{0.765000in}{0.660000in}}{\pgfqpoint{4.620000in}{4.620000in}}%
\pgfusepath{clip}%
\pgfsetbuttcap%
\pgfsetroundjoin%
\definecolor{currentfill}{rgb}{1.000000,0.576471,0.309804}%
\pgfsetfillcolor{currentfill}%
\pgfsetfillopacity{0.050000}%
\pgfsetlinewidth{0.000000pt}%
\definecolor{currentstroke}{rgb}{1.000000,0.576471,0.309804}%
\pgfsetstrokecolor{currentstroke}%
\pgfsetdash{}{0pt}%
\pgfpathmoveto{\pgfqpoint{3.866247in}{3.265684in}}%
\pgfpathlineto{\pgfqpoint{3.866247in}{3.338687in}}%
\pgfpathlineto{\pgfqpoint{3.788294in}{3.302090in}}%
\pgfpathlineto{\pgfqpoint{3.788294in}{3.229087in}}%
\pgfpathlineto{\pgfqpoint{3.866247in}{3.265684in}}%
\pgfpathclose%
\pgfusepath{fill}%
\end{pgfscope}%
\begin{pgfscope}%
\pgfpathrectangle{\pgfqpoint{0.765000in}{0.660000in}}{\pgfqpoint{4.620000in}{4.620000in}}%
\pgfusepath{clip}%
\pgfsetbuttcap%
\pgfsetroundjoin%
\definecolor{currentfill}{rgb}{1.000000,0.576471,0.309804}%
\pgfsetfillcolor{currentfill}%
\pgfsetfillopacity{0.050000}%
\pgfsetlinewidth{0.000000pt}%
\definecolor{currentstroke}{rgb}{1.000000,0.576471,0.309804}%
\pgfsetstrokecolor{currentstroke}%
\pgfsetdash{}{0pt}%
\pgfpathmoveto{\pgfqpoint{2.444094in}{3.580882in}}%
\pgfpathlineto{\pgfqpoint{2.444094in}{3.653886in}}%
\pgfpathlineto{\pgfqpoint{2.522046in}{3.617289in}}%
\pgfpathlineto{\pgfqpoint{2.522046in}{3.544286in}}%
\pgfpathlineto{\pgfqpoint{2.444094in}{3.580882in}}%
\pgfpathclose%
\pgfusepath{fill}%
\end{pgfscope}%
\begin{pgfscope}%
\pgfpathrectangle{\pgfqpoint{0.765000in}{0.660000in}}{\pgfqpoint{4.620000in}{4.620000in}}%
\pgfusepath{clip}%
\pgfsetbuttcap%
\pgfsetroundjoin%
\definecolor{currentfill}{rgb}{1.000000,0.576471,0.309804}%
\pgfsetfillcolor{currentfill}%
\pgfsetfillopacity{0.050000}%
\pgfsetlinewidth{0.000000pt}%
\definecolor{currentstroke}{rgb}{1.000000,0.576471,0.309804}%
\pgfsetstrokecolor{currentstroke}%
\pgfsetdash{}{0pt}%
\pgfpathmoveto{\pgfqpoint{2.444094in}{3.580882in}}%
\pgfpathlineto{\pgfqpoint{2.444094in}{3.653886in}}%
\pgfpathlineto{\pgfqpoint{2.366141in}{3.617289in}}%
\pgfpathlineto{\pgfqpoint{2.366141in}{3.544286in}}%
\pgfpathlineto{\pgfqpoint{2.444094in}{3.580882in}}%
\pgfpathclose%
\pgfusepath{fill}%
\end{pgfscope}%
\begin{pgfscope}%
\pgfpathrectangle{\pgfqpoint{0.765000in}{0.660000in}}{\pgfqpoint{4.620000in}{4.620000in}}%
\pgfusepath{clip}%
\pgfsetbuttcap%
\pgfsetroundjoin%
\definecolor{currentfill}{rgb}{1.000000,0.576471,0.309804}%
\pgfsetfillcolor{currentfill}%
\pgfsetfillopacity{0.050000}%
\pgfsetlinewidth{0.000000pt}%
\definecolor{currentstroke}{rgb}{1.000000,0.576471,0.309804}%
\pgfsetstrokecolor{currentstroke}%
\pgfsetdash{}{0pt}%
\pgfpathmoveto{\pgfqpoint{2.444094in}{3.580882in}}%
\pgfpathlineto{\pgfqpoint{2.522046in}{3.544286in}}%
\pgfpathlineto{\pgfqpoint{2.444094in}{3.507689in}}%
\pgfpathlineto{\pgfqpoint{2.366141in}{3.544286in}}%
\pgfpathlineto{\pgfqpoint{2.444094in}{3.580882in}}%
\pgfpathclose%
\pgfusepath{fill}%
\end{pgfscope}%
\begin{pgfscope}%
\pgfpathrectangle{\pgfqpoint{0.765000in}{0.660000in}}{\pgfqpoint{4.620000in}{4.620000in}}%
\pgfusepath{clip}%
\pgfsetbuttcap%
\pgfsetroundjoin%
\definecolor{currentfill}{rgb}{1.000000,0.576471,0.309804}%
\pgfsetfillcolor{currentfill}%
\pgfsetfillopacity{0.050000}%
\pgfsetlinewidth{0.000000pt}%
\definecolor{currentstroke}{rgb}{1.000000,0.576471,0.309804}%
\pgfsetstrokecolor{currentstroke}%
\pgfsetdash{}{0pt}%
\pgfpathmoveto{\pgfqpoint{2.444094in}{3.653886in}}%
\pgfpathlineto{\pgfqpoint{2.522046in}{3.617289in}}%
\pgfpathlineto{\pgfqpoint{2.444094in}{3.580693in}}%
\pgfpathlineto{\pgfqpoint{2.366141in}{3.617289in}}%
\pgfpathlineto{\pgfqpoint{2.444094in}{3.653886in}}%
\pgfpathclose%
\pgfusepath{fill}%
\end{pgfscope}%
\begin{pgfscope}%
\pgfpathrectangle{\pgfqpoint{0.765000in}{0.660000in}}{\pgfqpoint{4.620000in}{4.620000in}}%
\pgfusepath{clip}%
\pgfsetbuttcap%
\pgfsetroundjoin%
\definecolor{currentfill}{rgb}{1.000000,0.576471,0.309804}%
\pgfsetfillcolor{currentfill}%
\pgfsetfillopacity{0.050000}%
\pgfsetlinewidth{0.000000pt}%
\definecolor{currentstroke}{rgb}{1.000000,0.576471,0.309804}%
\pgfsetstrokecolor{currentstroke}%
\pgfsetdash{}{0pt}%
\pgfpathmoveto{\pgfqpoint{2.366141in}{3.544286in}}%
\pgfpathlineto{\pgfqpoint{2.366141in}{3.617289in}}%
\pgfpathlineto{\pgfqpoint{2.444094in}{3.580693in}}%
\pgfpathlineto{\pgfqpoint{2.444094in}{3.507689in}}%
\pgfpathlineto{\pgfqpoint{2.366141in}{3.544286in}}%
\pgfpathclose%
\pgfusepath{fill}%
\end{pgfscope}%
\begin{pgfscope}%
\pgfpathrectangle{\pgfqpoint{0.765000in}{0.660000in}}{\pgfqpoint{4.620000in}{4.620000in}}%
\pgfusepath{clip}%
\pgfsetbuttcap%
\pgfsetroundjoin%
\definecolor{currentfill}{rgb}{1.000000,0.576471,0.309804}%
\pgfsetfillcolor{currentfill}%
\pgfsetfillopacity{0.050000}%
\pgfsetlinewidth{0.000000pt}%
\definecolor{currentstroke}{rgb}{1.000000,0.576471,0.309804}%
\pgfsetstrokecolor{currentstroke}%
\pgfsetdash{}{0pt}%
\pgfpathmoveto{\pgfqpoint{2.522046in}{3.544286in}}%
\pgfpathlineto{\pgfqpoint{2.522046in}{3.617289in}}%
\pgfpathlineto{\pgfqpoint{2.444094in}{3.580693in}}%
\pgfpathlineto{\pgfqpoint{2.444094in}{3.507689in}}%
\pgfpathlineto{\pgfqpoint{2.522046in}{3.544286in}}%
\pgfpathclose%
\pgfusepath{fill}%
\end{pgfscope}%
\begin{pgfscope}%
\pgfpathrectangle{\pgfqpoint{0.765000in}{0.660000in}}{\pgfqpoint{4.620000in}{4.620000in}}%
\pgfusepath{clip}%
\pgfsetbuttcap%
\pgfsetroundjoin%
\definecolor{currentfill}{rgb}{1.000000,0.576471,0.309804}%
\pgfsetfillcolor{currentfill}%
\pgfsetfillopacity{0.050000}%
\pgfsetlinewidth{0.000000pt}%
\definecolor{currentstroke}{rgb}{1.000000,0.576471,0.309804}%
\pgfsetstrokecolor{currentstroke}%
\pgfsetdash{}{0pt}%
\pgfpathmoveto{\pgfqpoint{2.727027in}{2.510539in}}%
\pgfpathlineto{\pgfqpoint{2.727027in}{2.583542in}}%
\pgfpathlineto{\pgfqpoint{2.804979in}{2.546945in}}%
\pgfpathlineto{\pgfqpoint{2.804979in}{2.473942in}}%
\pgfpathlineto{\pgfqpoint{2.727027in}{2.510539in}}%
\pgfpathclose%
\pgfusepath{fill}%
\end{pgfscope}%
\begin{pgfscope}%
\pgfpathrectangle{\pgfqpoint{0.765000in}{0.660000in}}{\pgfqpoint{4.620000in}{4.620000in}}%
\pgfusepath{clip}%
\pgfsetbuttcap%
\pgfsetroundjoin%
\definecolor{currentfill}{rgb}{1.000000,0.576471,0.309804}%
\pgfsetfillcolor{currentfill}%
\pgfsetfillopacity{0.050000}%
\pgfsetlinewidth{0.000000pt}%
\definecolor{currentstroke}{rgb}{1.000000,0.576471,0.309804}%
\pgfsetstrokecolor{currentstroke}%
\pgfsetdash{}{0pt}%
\pgfpathmoveto{\pgfqpoint{2.727027in}{2.510539in}}%
\pgfpathlineto{\pgfqpoint{2.727027in}{2.583542in}}%
\pgfpathlineto{\pgfqpoint{2.649074in}{2.546945in}}%
\pgfpathlineto{\pgfqpoint{2.649074in}{2.473942in}}%
\pgfpathlineto{\pgfqpoint{2.727027in}{2.510539in}}%
\pgfpathclose%
\pgfusepath{fill}%
\end{pgfscope}%
\begin{pgfscope}%
\pgfpathrectangle{\pgfqpoint{0.765000in}{0.660000in}}{\pgfqpoint{4.620000in}{4.620000in}}%
\pgfusepath{clip}%
\pgfsetbuttcap%
\pgfsetroundjoin%
\definecolor{currentfill}{rgb}{1.000000,0.576471,0.309804}%
\pgfsetfillcolor{currentfill}%
\pgfsetfillopacity{0.050000}%
\pgfsetlinewidth{0.000000pt}%
\definecolor{currentstroke}{rgb}{1.000000,0.576471,0.309804}%
\pgfsetstrokecolor{currentstroke}%
\pgfsetdash{}{0pt}%
\pgfpathmoveto{\pgfqpoint{2.727027in}{2.510539in}}%
\pgfpathlineto{\pgfqpoint{2.804979in}{2.473942in}}%
\pgfpathlineto{\pgfqpoint{2.727027in}{2.437346in}}%
\pgfpathlineto{\pgfqpoint{2.649074in}{2.473942in}}%
\pgfpathlineto{\pgfqpoint{2.727027in}{2.510539in}}%
\pgfpathclose%
\pgfusepath{fill}%
\end{pgfscope}%
\begin{pgfscope}%
\pgfpathrectangle{\pgfqpoint{0.765000in}{0.660000in}}{\pgfqpoint{4.620000in}{4.620000in}}%
\pgfusepath{clip}%
\pgfsetbuttcap%
\pgfsetroundjoin%
\definecolor{currentfill}{rgb}{1.000000,0.576471,0.309804}%
\pgfsetfillcolor{currentfill}%
\pgfsetfillopacity{0.050000}%
\pgfsetlinewidth{0.000000pt}%
\definecolor{currentstroke}{rgb}{1.000000,0.576471,0.309804}%
\pgfsetstrokecolor{currentstroke}%
\pgfsetdash{}{0pt}%
\pgfpathmoveto{\pgfqpoint{2.727027in}{2.583542in}}%
\pgfpathlineto{\pgfqpoint{2.804979in}{2.546945in}}%
\pgfpathlineto{\pgfqpoint{2.727027in}{2.510349in}}%
\pgfpathlineto{\pgfqpoint{2.649074in}{2.546945in}}%
\pgfpathlineto{\pgfqpoint{2.727027in}{2.583542in}}%
\pgfpathclose%
\pgfusepath{fill}%
\end{pgfscope}%
\begin{pgfscope}%
\pgfpathrectangle{\pgfqpoint{0.765000in}{0.660000in}}{\pgfqpoint{4.620000in}{4.620000in}}%
\pgfusepath{clip}%
\pgfsetbuttcap%
\pgfsetroundjoin%
\definecolor{currentfill}{rgb}{1.000000,0.576471,0.309804}%
\pgfsetfillcolor{currentfill}%
\pgfsetfillopacity{0.050000}%
\pgfsetlinewidth{0.000000pt}%
\definecolor{currentstroke}{rgb}{1.000000,0.576471,0.309804}%
\pgfsetstrokecolor{currentstroke}%
\pgfsetdash{}{0pt}%
\pgfpathmoveto{\pgfqpoint{2.649074in}{2.473942in}}%
\pgfpathlineto{\pgfqpoint{2.649074in}{2.546945in}}%
\pgfpathlineto{\pgfqpoint{2.727027in}{2.510349in}}%
\pgfpathlineto{\pgfqpoint{2.727027in}{2.437346in}}%
\pgfpathlineto{\pgfqpoint{2.649074in}{2.473942in}}%
\pgfpathclose%
\pgfusepath{fill}%
\end{pgfscope}%
\begin{pgfscope}%
\pgfpathrectangle{\pgfqpoint{0.765000in}{0.660000in}}{\pgfqpoint{4.620000in}{4.620000in}}%
\pgfusepath{clip}%
\pgfsetbuttcap%
\pgfsetroundjoin%
\definecolor{currentfill}{rgb}{1.000000,0.576471,0.309804}%
\pgfsetfillcolor{currentfill}%
\pgfsetfillopacity{0.050000}%
\pgfsetlinewidth{0.000000pt}%
\definecolor{currentstroke}{rgb}{1.000000,0.576471,0.309804}%
\pgfsetstrokecolor{currentstroke}%
\pgfsetdash{}{0pt}%
\pgfpathmoveto{\pgfqpoint{2.804979in}{2.473942in}}%
\pgfpathlineto{\pgfqpoint{2.804979in}{2.546945in}}%
\pgfpathlineto{\pgfqpoint{2.727027in}{2.510349in}}%
\pgfpathlineto{\pgfqpoint{2.727027in}{2.437346in}}%
\pgfpathlineto{\pgfqpoint{2.804979in}{2.473942in}}%
\pgfpathclose%
\pgfusepath{fill}%
\end{pgfscope}%
\begin{pgfscope}%
\pgfpathrectangle{\pgfqpoint{0.765000in}{0.660000in}}{\pgfqpoint{4.620000in}{4.620000in}}%
\pgfusepath{clip}%
\pgfsetbuttcap%
\pgfsetroundjoin%
\definecolor{currentfill}{rgb}{0.176471,0.192157,0.258824}%
\pgfsetfillcolor{currentfill}%
\pgfsetfillopacity{0.050000}%
\pgfsetlinewidth{0.000000pt}%
\definecolor{currentstroke}{rgb}{0.176471,0.192157,0.258824}%
\pgfsetstrokecolor{currentstroke}%
\pgfsetdash{}{0pt}%
\pgfpathmoveto{\pgfqpoint{3.237197in}{3.018519in}}%
\pgfpathlineto{\pgfqpoint{3.237197in}{3.091522in}}%
\pgfpathlineto{\pgfqpoint{3.315150in}{3.054926in}}%
\pgfpathlineto{\pgfqpoint{3.315150in}{2.981922in}}%
\pgfpathlineto{\pgfqpoint{3.237197in}{3.018519in}}%
\pgfpathclose%
\pgfusepath{fill}%
\end{pgfscope}%
\begin{pgfscope}%
\pgfpathrectangle{\pgfqpoint{0.765000in}{0.660000in}}{\pgfqpoint{4.620000in}{4.620000in}}%
\pgfusepath{clip}%
\pgfsetbuttcap%
\pgfsetroundjoin%
\definecolor{currentfill}{rgb}{0.176471,0.192157,0.258824}%
\pgfsetfillcolor{currentfill}%
\pgfsetfillopacity{0.050000}%
\pgfsetlinewidth{0.000000pt}%
\definecolor{currentstroke}{rgb}{0.176471,0.192157,0.258824}%
\pgfsetstrokecolor{currentstroke}%
\pgfsetdash{}{0pt}%
\pgfpathmoveto{\pgfqpoint{3.237197in}{3.018519in}}%
\pgfpathlineto{\pgfqpoint{3.237197in}{3.091522in}}%
\pgfpathlineto{\pgfqpoint{3.159244in}{3.054926in}}%
\pgfpathlineto{\pgfqpoint{3.159244in}{2.981922in}}%
\pgfpathlineto{\pgfqpoint{3.237197in}{3.018519in}}%
\pgfpathclose%
\pgfusepath{fill}%
\end{pgfscope}%
\begin{pgfscope}%
\pgfpathrectangle{\pgfqpoint{0.765000in}{0.660000in}}{\pgfqpoint{4.620000in}{4.620000in}}%
\pgfusepath{clip}%
\pgfsetbuttcap%
\pgfsetroundjoin%
\definecolor{currentfill}{rgb}{0.176471,0.192157,0.258824}%
\pgfsetfillcolor{currentfill}%
\pgfsetfillopacity{0.050000}%
\pgfsetlinewidth{0.000000pt}%
\definecolor{currentstroke}{rgb}{0.176471,0.192157,0.258824}%
\pgfsetstrokecolor{currentstroke}%
\pgfsetdash{}{0pt}%
\pgfpathmoveto{\pgfqpoint{3.237197in}{3.018519in}}%
\pgfpathlineto{\pgfqpoint{3.315150in}{2.981922in}}%
\pgfpathlineto{\pgfqpoint{3.237197in}{2.945326in}}%
\pgfpathlineto{\pgfqpoint{3.159244in}{2.981922in}}%
\pgfpathlineto{\pgfqpoint{3.237197in}{3.018519in}}%
\pgfpathclose%
\pgfusepath{fill}%
\end{pgfscope}%
\begin{pgfscope}%
\pgfpathrectangle{\pgfqpoint{0.765000in}{0.660000in}}{\pgfqpoint{4.620000in}{4.620000in}}%
\pgfusepath{clip}%
\pgfsetbuttcap%
\pgfsetroundjoin%
\definecolor{currentfill}{rgb}{0.176471,0.192157,0.258824}%
\pgfsetfillcolor{currentfill}%
\pgfsetfillopacity{0.050000}%
\pgfsetlinewidth{0.000000pt}%
\definecolor{currentstroke}{rgb}{0.176471,0.192157,0.258824}%
\pgfsetstrokecolor{currentstroke}%
\pgfsetdash{}{0pt}%
\pgfpathmoveto{\pgfqpoint{3.237197in}{3.091522in}}%
\pgfpathlineto{\pgfqpoint{3.315150in}{3.054926in}}%
\pgfpathlineto{\pgfqpoint{3.237197in}{3.018329in}}%
\pgfpathlineto{\pgfqpoint{3.159244in}{3.054926in}}%
\pgfpathlineto{\pgfqpoint{3.237197in}{3.091522in}}%
\pgfpathclose%
\pgfusepath{fill}%
\end{pgfscope}%
\begin{pgfscope}%
\pgfpathrectangle{\pgfqpoint{0.765000in}{0.660000in}}{\pgfqpoint{4.620000in}{4.620000in}}%
\pgfusepath{clip}%
\pgfsetbuttcap%
\pgfsetroundjoin%
\definecolor{currentfill}{rgb}{0.176471,0.192157,0.258824}%
\pgfsetfillcolor{currentfill}%
\pgfsetfillopacity{0.050000}%
\pgfsetlinewidth{0.000000pt}%
\definecolor{currentstroke}{rgb}{0.176471,0.192157,0.258824}%
\pgfsetstrokecolor{currentstroke}%
\pgfsetdash{}{0pt}%
\pgfpathmoveto{\pgfqpoint{3.159244in}{2.981922in}}%
\pgfpathlineto{\pgfqpoint{3.159244in}{3.054926in}}%
\pgfpathlineto{\pgfqpoint{3.237197in}{3.018329in}}%
\pgfpathlineto{\pgfqpoint{3.237197in}{2.945326in}}%
\pgfpathlineto{\pgfqpoint{3.159244in}{2.981922in}}%
\pgfpathclose%
\pgfusepath{fill}%
\end{pgfscope}%
\begin{pgfscope}%
\pgfpathrectangle{\pgfqpoint{0.765000in}{0.660000in}}{\pgfqpoint{4.620000in}{4.620000in}}%
\pgfusepath{clip}%
\pgfsetbuttcap%
\pgfsetroundjoin%
\definecolor{currentfill}{rgb}{0.176471,0.192157,0.258824}%
\pgfsetfillcolor{currentfill}%
\pgfsetfillopacity{0.050000}%
\pgfsetlinewidth{0.000000pt}%
\definecolor{currentstroke}{rgb}{0.176471,0.192157,0.258824}%
\pgfsetstrokecolor{currentstroke}%
\pgfsetdash{}{0pt}%
\pgfpathmoveto{\pgfqpoint{3.315150in}{2.981922in}}%
\pgfpathlineto{\pgfqpoint{3.315150in}{3.054926in}}%
\pgfpathlineto{\pgfqpoint{3.237197in}{3.018329in}}%
\pgfpathlineto{\pgfqpoint{3.237197in}{2.945326in}}%
\pgfpathlineto{\pgfqpoint{3.315150in}{2.981922in}}%
\pgfpathclose%
\pgfusepath{fill}%
\end{pgfscope}%
\begin{pgfscope}%
\pgfpathrectangle{\pgfqpoint{0.765000in}{0.660000in}}{\pgfqpoint{4.620000in}{4.620000in}}%
\pgfusepath{clip}%
\pgfsetbuttcap%
\pgfsetroundjoin%
\definecolor{currentfill}{rgb}{0.176471,0.192157,0.258824}%
\pgfsetfillcolor{currentfill}%
\pgfsetfillopacity{0.050000}%
\pgfsetlinewidth{0.000000pt}%
\definecolor{currentstroke}{rgb}{0.176471,0.192157,0.258824}%
\pgfsetstrokecolor{currentstroke}%
\pgfsetdash{}{0pt}%
\pgfpathmoveto{\pgfqpoint{3.319791in}{2.901166in}}%
\pgfpathlineto{\pgfqpoint{3.319791in}{2.974169in}}%
\pgfpathlineto{\pgfqpoint{3.397743in}{2.937573in}}%
\pgfpathlineto{\pgfqpoint{3.397743in}{2.864569in}}%
\pgfpathlineto{\pgfqpoint{3.319791in}{2.901166in}}%
\pgfpathclose%
\pgfusepath{fill}%
\end{pgfscope}%
\begin{pgfscope}%
\pgfpathrectangle{\pgfqpoint{0.765000in}{0.660000in}}{\pgfqpoint{4.620000in}{4.620000in}}%
\pgfusepath{clip}%
\pgfsetbuttcap%
\pgfsetroundjoin%
\definecolor{currentfill}{rgb}{0.176471,0.192157,0.258824}%
\pgfsetfillcolor{currentfill}%
\pgfsetfillopacity{0.050000}%
\pgfsetlinewidth{0.000000pt}%
\definecolor{currentstroke}{rgb}{0.176471,0.192157,0.258824}%
\pgfsetstrokecolor{currentstroke}%
\pgfsetdash{}{0pt}%
\pgfpathmoveto{\pgfqpoint{3.319791in}{2.901166in}}%
\pgfpathlineto{\pgfqpoint{3.319791in}{2.974169in}}%
\pgfpathlineto{\pgfqpoint{3.241838in}{2.937573in}}%
\pgfpathlineto{\pgfqpoint{3.241838in}{2.864569in}}%
\pgfpathlineto{\pgfqpoint{3.319791in}{2.901166in}}%
\pgfpathclose%
\pgfusepath{fill}%
\end{pgfscope}%
\begin{pgfscope}%
\pgfpathrectangle{\pgfqpoint{0.765000in}{0.660000in}}{\pgfqpoint{4.620000in}{4.620000in}}%
\pgfusepath{clip}%
\pgfsetbuttcap%
\pgfsetroundjoin%
\definecolor{currentfill}{rgb}{0.176471,0.192157,0.258824}%
\pgfsetfillcolor{currentfill}%
\pgfsetfillopacity{0.050000}%
\pgfsetlinewidth{0.000000pt}%
\definecolor{currentstroke}{rgb}{0.176471,0.192157,0.258824}%
\pgfsetstrokecolor{currentstroke}%
\pgfsetdash{}{0pt}%
\pgfpathmoveto{\pgfqpoint{3.319791in}{2.901166in}}%
\pgfpathlineto{\pgfqpoint{3.397743in}{2.864569in}}%
\pgfpathlineto{\pgfqpoint{3.319791in}{2.827973in}}%
\pgfpathlineto{\pgfqpoint{3.241838in}{2.864569in}}%
\pgfpathlineto{\pgfqpoint{3.319791in}{2.901166in}}%
\pgfpathclose%
\pgfusepath{fill}%
\end{pgfscope}%
\begin{pgfscope}%
\pgfpathrectangle{\pgfqpoint{0.765000in}{0.660000in}}{\pgfqpoint{4.620000in}{4.620000in}}%
\pgfusepath{clip}%
\pgfsetbuttcap%
\pgfsetroundjoin%
\definecolor{currentfill}{rgb}{0.176471,0.192157,0.258824}%
\pgfsetfillcolor{currentfill}%
\pgfsetfillopacity{0.050000}%
\pgfsetlinewidth{0.000000pt}%
\definecolor{currentstroke}{rgb}{0.176471,0.192157,0.258824}%
\pgfsetstrokecolor{currentstroke}%
\pgfsetdash{}{0pt}%
\pgfpathmoveto{\pgfqpoint{3.319791in}{2.974169in}}%
\pgfpathlineto{\pgfqpoint{3.397743in}{2.937573in}}%
\pgfpathlineto{\pgfqpoint{3.319791in}{2.900976in}}%
\pgfpathlineto{\pgfqpoint{3.241838in}{2.937573in}}%
\pgfpathlineto{\pgfqpoint{3.319791in}{2.974169in}}%
\pgfpathclose%
\pgfusepath{fill}%
\end{pgfscope}%
\begin{pgfscope}%
\pgfpathrectangle{\pgfqpoint{0.765000in}{0.660000in}}{\pgfqpoint{4.620000in}{4.620000in}}%
\pgfusepath{clip}%
\pgfsetbuttcap%
\pgfsetroundjoin%
\definecolor{currentfill}{rgb}{0.176471,0.192157,0.258824}%
\pgfsetfillcolor{currentfill}%
\pgfsetfillopacity{0.050000}%
\pgfsetlinewidth{0.000000pt}%
\definecolor{currentstroke}{rgb}{0.176471,0.192157,0.258824}%
\pgfsetstrokecolor{currentstroke}%
\pgfsetdash{}{0pt}%
\pgfpathmoveto{\pgfqpoint{3.241838in}{2.864569in}}%
\pgfpathlineto{\pgfqpoint{3.241838in}{2.937573in}}%
\pgfpathlineto{\pgfqpoint{3.319791in}{2.900976in}}%
\pgfpathlineto{\pgfqpoint{3.319791in}{2.827973in}}%
\pgfpathlineto{\pgfqpoint{3.241838in}{2.864569in}}%
\pgfpathclose%
\pgfusepath{fill}%
\end{pgfscope}%
\begin{pgfscope}%
\pgfpathrectangle{\pgfqpoint{0.765000in}{0.660000in}}{\pgfqpoint{4.620000in}{4.620000in}}%
\pgfusepath{clip}%
\pgfsetbuttcap%
\pgfsetroundjoin%
\definecolor{currentfill}{rgb}{0.176471,0.192157,0.258824}%
\pgfsetfillcolor{currentfill}%
\pgfsetfillopacity{0.050000}%
\pgfsetlinewidth{0.000000pt}%
\definecolor{currentstroke}{rgb}{0.176471,0.192157,0.258824}%
\pgfsetstrokecolor{currentstroke}%
\pgfsetdash{}{0pt}%
\pgfpathmoveto{\pgfqpoint{3.397743in}{2.864569in}}%
\pgfpathlineto{\pgfqpoint{3.397743in}{2.937573in}}%
\pgfpathlineto{\pgfqpoint{3.319791in}{2.900976in}}%
\pgfpathlineto{\pgfqpoint{3.319791in}{2.827973in}}%
\pgfpathlineto{\pgfqpoint{3.397743in}{2.864569in}}%
\pgfpathclose%
\pgfusepath{fill}%
\end{pgfscope}%
\begin{pgfscope}%
\pgfpathrectangle{\pgfqpoint{0.765000in}{0.660000in}}{\pgfqpoint{4.620000in}{4.620000in}}%
\pgfusepath{clip}%
\pgfsetbuttcap%
\pgfsetroundjoin%
\definecolor{currentfill}{rgb}{0.176471,0.192157,0.258824}%
\pgfsetfillcolor{currentfill}%
\pgfsetfillopacity{0.050000}%
\pgfsetlinewidth{0.000000pt}%
\definecolor{currentstroke}{rgb}{0.176471,0.192157,0.258824}%
\pgfsetstrokecolor{currentstroke}%
\pgfsetdash{}{0pt}%
\pgfpathmoveto{\pgfqpoint{3.231847in}{2.958127in}}%
\pgfpathlineto{\pgfqpoint{3.231847in}{3.031130in}}%
\pgfpathlineto{\pgfqpoint{3.309800in}{2.994534in}}%
\pgfpathlineto{\pgfqpoint{3.309800in}{2.921530in}}%
\pgfpathlineto{\pgfqpoint{3.231847in}{2.958127in}}%
\pgfpathclose%
\pgfusepath{fill}%
\end{pgfscope}%
\begin{pgfscope}%
\pgfpathrectangle{\pgfqpoint{0.765000in}{0.660000in}}{\pgfqpoint{4.620000in}{4.620000in}}%
\pgfusepath{clip}%
\pgfsetbuttcap%
\pgfsetroundjoin%
\definecolor{currentfill}{rgb}{0.176471,0.192157,0.258824}%
\pgfsetfillcolor{currentfill}%
\pgfsetfillopacity{0.050000}%
\pgfsetlinewidth{0.000000pt}%
\definecolor{currentstroke}{rgb}{0.176471,0.192157,0.258824}%
\pgfsetstrokecolor{currentstroke}%
\pgfsetdash{}{0pt}%
\pgfpathmoveto{\pgfqpoint{3.231847in}{2.958127in}}%
\pgfpathlineto{\pgfqpoint{3.231847in}{3.031130in}}%
\pgfpathlineto{\pgfqpoint{3.153895in}{2.994534in}}%
\pgfpathlineto{\pgfqpoint{3.153895in}{2.921530in}}%
\pgfpathlineto{\pgfqpoint{3.231847in}{2.958127in}}%
\pgfpathclose%
\pgfusepath{fill}%
\end{pgfscope}%
\begin{pgfscope}%
\pgfpathrectangle{\pgfqpoint{0.765000in}{0.660000in}}{\pgfqpoint{4.620000in}{4.620000in}}%
\pgfusepath{clip}%
\pgfsetbuttcap%
\pgfsetroundjoin%
\definecolor{currentfill}{rgb}{0.176471,0.192157,0.258824}%
\pgfsetfillcolor{currentfill}%
\pgfsetfillopacity{0.050000}%
\pgfsetlinewidth{0.000000pt}%
\definecolor{currentstroke}{rgb}{0.176471,0.192157,0.258824}%
\pgfsetstrokecolor{currentstroke}%
\pgfsetdash{}{0pt}%
\pgfpathmoveto{\pgfqpoint{3.231847in}{2.958127in}}%
\pgfpathlineto{\pgfqpoint{3.309800in}{2.921530in}}%
\pgfpathlineto{\pgfqpoint{3.231847in}{2.884934in}}%
\pgfpathlineto{\pgfqpoint{3.153895in}{2.921530in}}%
\pgfpathlineto{\pgfqpoint{3.231847in}{2.958127in}}%
\pgfpathclose%
\pgfusepath{fill}%
\end{pgfscope}%
\begin{pgfscope}%
\pgfpathrectangle{\pgfqpoint{0.765000in}{0.660000in}}{\pgfqpoint{4.620000in}{4.620000in}}%
\pgfusepath{clip}%
\pgfsetbuttcap%
\pgfsetroundjoin%
\definecolor{currentfill}{rgb}{0.176471,0.192157,0.258824}%
\pgfsetfillcolor{currentfill}%
\pgfsetfillopacity{0.050000}%
\pgfsetlinewidth{0.000000pt}%
\definecolor{currentstroke}{rgb}{0.176471,0.192157,0.258824}%
\pgfsetstrokecolor{currentstroke}%
\pgfsetdash{}{0pt}%
\pgfpathmoveto{\pgfqpoint{3.231847in}{3.031130in}}%
\pgfpathlineto{\pgfqpoint{3.309800in}{2.994534in}}%
\pgfpathlineto{\pgfqpoint{3.231847in}{2.957937in}}%
\pgfpathlineto{\pgfqpoint{3.153895in}{2.994534in}}%
\pgfpathlineto{\pgfqpoint{3.231847in}{3.031130in}}%
\pgfpathclose%
\pgfusepath{fill}%
\end{pgfscope}%
\begin{pgfscope}%
\pgfpathrectangle{\pgfqpoint{0.765000in}{0.660000in}}{\pgfqpoint{4.620000in}{4.620000in}}%
\pgfusepath{clip}%
\pgfsetbuttcap%
\pgfsetroundjoin%
\definecolor{currentfill}{rgb}{0.176471,0.192157,0.258824}%
\pgfsetfillcolor{currentfill}%
\pgfsetfillopacity{0.050000}%
\pgfsetlinewidth{0.000000pt}%
\definecolor{currentstroke}{rgb}{0.176471,0.192157,0.258824}%
\pgfsetstrokecolor{currentstroke}%
\pgfsetdash{}{0pt}%
\pgfpathmoveto{\pgfqpoint{3.153895in}{2.921530in}}%
\pgfpathlineto{\pgfqpoint{3.153895in}{2.994534in}}%
\pgfpathlineto{\pgfqpoint{3.231847in}{2.957937in}}%
\pgfpathlineto{\pgfqpoint{3.231847in}{2.884934in}}%
\pgfpathlineto{\pgfqpoint{3.153895in}{2.921530in}}%
\pgfpathclose%
\pgfusepath{fill}%
\end{pgfscope}%
\begin{pgfscope}%
\pgfpathrectangle{\pgfqpoint{0.765000in}{0.660000in}}{\pgfqpoint{4.620000in}{4.620000in}}%
\pgfusepath{clip}%
\pgfsetbuttcap%
\pgfsetroundjoin%
\definecolor{currentfill}{rgb}{0.176471,0.192157,0.258824}%
\pgfsetfillcolor{currentfill}%
\pgfsetfillopacity{0.050000}%
\pgfsetlinewidth{0.000000pt}%
\definecolor{currentstroke}{rgb}{0.176471,0.192157,0.258824}%
\pgfsetstrokecolor{currentstroke}%
\pgfsetdash{}{0pt}%
\pgfpathmoveto{\pgfqpoint{3.309800in}{2.921530in}}%
\pgfpathlineto{\pgfqpoint{3.309800in}{2.994534in}}%
\pgfpathlineto{\pgfqpoint{3.231847in}{2.957937in}}%
\pgfpathlineto{\pgfqpoint{3.231847in}{2.884934in}}%
\pgfpathlineto{\pgfqpoint{3.309800in}{2.921530in}}%
\pgfpathclose%
\pgfusepath{fill}%
\end{pgfscope}%
\begin{pgfscope}%
\pgfpathrectangle{\pgfqpoint{0.765000in}{0.660000in}}{\pgfqpoint{4.620000in}{4.620000in}}%
\pgfusepath{clip}%
\pgfsetbuttcap%
\pgfsetroundjoin%
\definecolor{currentfill}{rgb}{1.000000,0.576471,0.309804}%
\pgfsetfillcolor{currentfill}%
\pgfsetfillopacity{0.050000}%
\pgfsetlinewidth{0.000000pt}%
\definecolor{currentstroke}{rgb}{1.000000,0.576471,0.309804}%
\pgfsetstrokecolor{currentstroke}%
\pgfsetdash{}{0pt}%
\pgfpathmoveto{\pgfqpoint{2.804662in}{2.478504in}}%
\pgfpathlineto{\pgfqpoint{2.804662in}{2.551507in}}%
\pgfpathlineto{\pgfqpoint{2.882614in}{2.514910in}}%
\pgfpathlineto{\pgfqpoint{2.882614in}{2.441907in}}%
\pgfpathlineto{\pgfqpoint{2.804662in}{2.478504in}}%
\pgfpathclose%
\pgfusepath{fill}%
\end{pgfscope}%
\begin{pgfscope}%
\pgfpathrectangle{\pgfqpoint{0.765000in}{0.660000in}}{\pgfqpoint{4.620000in}{4.620000in}}%
\pgfusepath{clip}%
\pgfsetbuttcap%
\pgfsetroundjoin%
\definecolor{currentfill}{rgb}{1.000000,0.576471,0.309804}%
\pgfsetfillcolor{currentfill}%
\pgfsetfillopacity{0.050000}%
\pgfsetlinewidth{0.000000pt}%
\definecolor{currentstroke}{rgb}{1.000000,0.576471,0.309804}%
\pgfsetstrokecolor{currentstroke}%
\pgfsetdash{}{0pt}%
\pgfpathmoveto{\pgfqpoint{2.804662in}{2.478504in}}%
\pgfpathlineto{\pgfqpoint{2.804662in}{2.551507in}}%
\pgfpathlineto{\pgfqpoint{2.726709in}{2.514910in}}%
\pgfpathlineto{\pgfqpoint{2.726709in}{2.441907in}}%
\pgfpathlineto{\pgfqpoint{2.804662in}{2.478504in}}%
\pgfpathclose%
\pgfusepath{fill}%
\end{pgfscope}%
\begin{pgfscope}%
\pgfpathrectangle{\pgfqpoint{0.765000in}{0.660000in}}{\pgfqpoint{4.620000in}{4.620000in}}%
\pgfusepath{clip}%
\pgfsetbuttcap%
\pgfsetroundjoin%
\definecolor{currentfill}{rgb}{1.000000,0.576471,0.309804}%
\pgfsetfillcolor{currentfill}%
\pgfsetfillopacity{0.050000}%
\pgfsetlinewidth{0.000000pt}%
\definecolor{currentstroke}{rgb}{1.000000,0.576471,0.309804}%
\pgfsetstrokecolor{currentstroke}%
\pgfsetdash{}{0pt}%
\pgfpathmoveto{\pgfqpoint{2.804662in}{2.478504in}}%
\pgfpathlineto{\pgfqpoint{2.882614in}{2.441907in}}%
\pgfpathlineto{\pgfqpoint{2.804662in}{2.405310in}}%
\pgfpathlineto{\pgfqpoint{2.726709in}{2.441907in}}%
\pgfpathlineto{\pgfqpoint{2.804662in}{2.478504in}}%
\pgfpathclose%
\pgfusepath{fill}%
\end{pgfscope}%
\begin{pgfscope}%
\pgfpathrectangle{\pgfqpoint{0.765000in}{0.660000in}}{\pgfqpoint{4.620000in}{4.620000in}}%
\pgfusepath{clip}%
\pgfsetbuttcap%
\pgfsetroundjoin%
\definecolor{currentfill}{rgb}{1.000000,0.576471,0.309804}%
\pgfsetfillcolor{currentfill}%
\pgfsetfillopacity{0.050000}%
\pgfsetlinewidth{0.000000pt}%
\definecolor{currentstroke}{rgb}{1.000000,0.576471,0.309804}%
\pgfsetstrokecolor{currentstroke}%
\pgfsetdash{}{0pt}%
\pgfpathmoveto{\pgfqpoint{2.804662in}{2.551507in}}%
\pgfpathlineto{\pgfqpoint{2.882614in}{2.514910in}}%
\pgfpathlineto{\pgfqpoint{2.804662in}{2.478314in}}%
\pgfpathlineto{\pgfqpoint{2.726709in}{2.514910in}}%
\pgfpathlineto{\pgfqpoint{2.804662in}{2.551507in}}%
\pgfpathclose%
\pgfusepath{fill}%
\end{pgfscope}%
\begin{pgfscope}%
\pgfpathrectangle{\pgfqpoint{0.765000in}{0.660000in}}{\pgfqpoint{4.620000in}{4.620000in}}%
\pgfusepath{clip}%
\pgfsetbuttcap%
\pgfsetroundjoin%
\definecolor{currentfill}{rgb}{1.000000,0.576471,0.309804}%
\pgfsetfillcolor{currentfill}%
\pgfsetfillopacity{0.050000}%
\pgfsetlinewidth{0.000000pt}%
\definecolor{currentstroke}{rgb}{1.000000,0.576471,0.309804}%
\pgfsetstrokecolor{currentstroke}%
\pgfsetdash{}{0pt}%
\pgfpathmoveto{\pgfqpoint{2.726709in}{2.441907in}}%
\pgfpathlineto{\pgfqpoint{2.726709in}{2.514910in}}%
\pgfpathlineto{\pgfqpoint{2.804662in}{2.478314in}}%
\pgfpathlineto{\pgfqpoint{2.804662in}{2.405310in}}%
\pgfpathlineto{\pgfqpoint{2.726709in}{2.441907in}}%
\pgfpathclose%
\pgfusepath{fill}%
\end{pgfscope}%
\begin{pgfscope}%
\pgfpathrectangle{\pgfqpoint{0.765000in}{0.660000in}}{\pgfqpoint{4.620000in}{4.620000in}}%
\pgfusepath{clip}%
\pgfsetbuttcap%
\pgfsetroundjoin%
\definecolor{currentfill}{rgb}{1.000000,0.576471,0.309804}%
\pgfsetfillcolor{currentfill}%
\pgfsetfillopacity{0.050000}%
\pgfsetlinewidth{0.000000pt}%
\definecolor{currentstroke}{rgb}{1.000000,0.576471,0.309804}%
\pgfsetstrokecolor{currentstroke}%
\pgfsetdash{}{0pt}%
\pgfpathmoveto{\pgfqpoint{2.882614in}{2.441907in}}%
\pgfpathlineto{\pgfqpoint{2.882614in}{2.514910in}}%
\pgfpathlineto{\pgfqpoint{2.804662in}{2.478314in}}%
\pgfpathlineto{\pgfqpoint{2.804662in}{2.405310in}}%
\pgfpathlineto{\pgfqpoint{2.882614in}{2.441907in}}%
\pgfpathclose%
\pgfusepath{fill}%
\end{pgfscope}%
\begin{pgfscope}%
\pgfpathrectangle{\pgfqpoint{0.765000in}{0.660000in}}{\pgfqpoint{4.620000in}{4.620000in}}%
\pgfusepath{clip}%
\pgfsetbuttcap%
\pgfsetroundjoin%
\definecolor{currentfill}{rgb}{1.000000,0.576471,0.309804}%
\pgfsetfillcolor{currentfill}%
\pgfsetfillopacity{0.050000}%
\pgfsetlinewidth{0.000000pt}%
\definecolor{currentstroke}{rgb}{1.000000,0.576471,0.309804}%
\pgfsetstrokecolor{currentstroke}%
\pgfsetdash{}{0pt}%
\pgfpathmoveto{\pgfqpoint{3.779344in}{3.637967in}}%
\pgfpathlineto{\pgfqpoint{3.779344in}{3.710970in}}%
\pgfpathlineto{\pgfqpoint{3.857297in}{3.674373in}}%
\pgfpathlineto{\pgfqpoint{3.857297in}{3.601370in}}%
\pgfpathlineto{\pgfqpoint{3.779344in}{3.637967in}}%
\pgfpathclose%
\pgfusepath{fill}%
\end{pgfscope}%
\begin{pgfscope}%
\pgfpathrectangle{\pgfqpoint{0.765000in}{0.660000in}}{\pgfqpoint{4.620000in}{4.620000in}}%
\pgfusepath{clip}%
\pgfsetbuttcap%
\pgfsetroundjoin%
\definecolor{currentfill}{rgb}{1.000000,0.576471,0.309804}%
\pgfsetfillcolor{currentfill}%
\pgfsetfillopacity{0.050000}%
\pgfsetlinewidth{0.000000pt}%
\definecolor{currentstroke}{rgb}{1.000000,0.576471,0.309804}%
\pgfsetstrokecolor{currentstroke}%
\pgfsetdash{}{0pt}%
\pgfpathmoveto{\pgfqpoint{3.779344in}{3.637967in}}%
\pgfpathlineto{\pgfqpoint{3.779344in}{3.710970in}}%
\pgfpathlineto{\pgfqpoint{3.701392in}{3.674373in}}%
\pgfpathlineto{\pgfqpoint{3.701392in}{3.601370in}}%
\pgfpathlineto{\pgfqpoint{3.779344in}{3.637967in}}%
\pgfpathclose%
\pgfusepath{fill}%
\end{pgfscope}%
\begin{pgfscope}%
\pgfpathrectangle{\pgfqpoint{0.765000in}{0.660000in}}{\pgfqpoint{4.620000in}{4.620000in}}%
\pgfusepath{clip}%
\pgfsetbuttcap%
\pgfsetroundjoin%
\definecolor{currentfill}{rgb}{1.000000,0.576471,0.309804}%
\pgfsetfillcolor{currentfill}%
\pgfsetfillopacity{0.050000}%
\pgfsetlinewidth{0.000000pt}%
\definecolor{currentstroke}{rgb}{1.000000,0.576471,0.309804}%
\pgfsetstrokecolor{currentstroke}%
\pgfsetdash{}{0pt}%
\pgfpathmoveto{\pgfqpoint{3.779344in}{3.637967in}}%
\pgfpathlineto{\pgfqpoint{3.857297in}{3.601370in}}%
\pgfpathlineto{\pgfqpoint{3.779344in}{3.564773in}}%
\pgfpathlineto{\pgfqpoint{3.701392in}{3.601370in}}%
\pgfpathlineto{\pgfqpoint{3.779344in}{3.637967in}}%
\pgfpathclose%
\pgfusepath{fill}%
\end{pgfscope}%
\begin{pgfscope}%
\pgfpathrectangle{\pgfqpoint{0.765000in}{0.660000in}}{\pgfqpoint{4.620000in}{4.620000in}}%
\pgfusepath{clip}%
\pgfsetbuttcap%
\pgfsetroundjoin%
\definecolor{currentfill}{rgb}{1.000000,0.576471,0.309804}%
\pgfsetfillcolor{currentfill}%
\pgfsetfillopacity{0.050000}%
\pgfsetlinewidth{0.000000pt}%
\definecolor{currentstroke}{rgb}{1.000000,0.576471,0.309804}%
\pgfsetstrokecolor{currentstroke}%
\pgfsetdash{}{0pt}%
\pgfpathmoveto{\pgfqpoint{3.779344in}{3.710970in}}%
\pgfpathlineto{\pgfqpoint{3.857297in}{3.674373in}}%
\pgfpathlineto{\pgfqpoint{3.779344in}{3.637777in}}%
\pgfpathlineto{\pgfqpoint{3.701392in}{3.674373in}}%
\pgfpathlineto{\pgfqpoint{3.779344in}{3.710970in}}%
\pgfpathclose%
\pgfusepath{fill}%
\end{pgfscope}%
\begin{pgfscope}%
\pgfpathrectangle{\pgfqpoint{0.765000in}{0.660000in}}{\pgfqpoint{4.620000in}{4.620000in}}%
\pgfusepath{clip}%
\pgfsetbuttcap%
\pgfsetroundjoin%
\definecolor{currentfill}{rgb}{1.000000,0.576471,0.309804}%
\pgfsetfillcolor{currentfill}%
\pgfsetfillopacity{0.050000}%
\pgfsetlinewidth{0.000000pt}%
\definecolor{currentstroke}{rgb}{1.000000,0.576471,0.309804}%
\pgfsetstrokecolor{currentstroke}%
\pgfsetdash{}{0pt}%
\pgfpathmoveto{\pgfqpoint{3.701392in}{3.601370in}}%
\pgfpathlineto{\pgfqpoint{3.701392in}{3.674373in}}%
\pgfpathlineto{\pgfqpoint{3.779344in}{3.637777in}}%
\pgfpathlineto{\pgfqpoint{3.779344in}{3.564773in}}%
\pgfpathlineto{\pgfqpoint{3.701392in}{3.601370in}}%
\pgfpathclose%
\pgfusepath{fill}%
\end{pgfscope}%
\begin{pgfscope}%
\pgfpathrectangle{\pgfqpoint{0.765000in}{0.660000in}}{\pgfqpoint{4.620000in}{4.620000in}}%
\pgfusepath{clip}%
\pgfsetbuttcap%
\pgfsetroundjoin%
\definecolor{currentfill}{rgb}{1.000000,0.576471,0.309804}%
\pgfsetfillcolor{currentfill}%
\pgfsetfillopacity{0.050000}%
\pgfsetlinewidth{0.000000pt}%
\definecolor{currentstroke}{rgb}{1.000000,0.576471,0.309804}%
\pgfsetstrokecolor{currentstroke}%
\pgfsetdash{}{0pt}%
\pgfpathmoveto{\pgfqpoint{3.857297in}{3.601370in}}%
\pgfpathlineto{\pgfqpoint{3.857297in}{3.674373in}}%
\pgfpathlineto{\pgfqpoint{3.779344in}{3.637777in}}%
\pgfpathlineto{\pgfqpoint{3.779344in}{3.564773in}}%
\pgfpathlineto{\pgfqpoint{3.857297in}{3.601370in}}%
\pgfpathclose%
\pgfusepath{fill}%
\end{pgfscope}%
\begin{pgfscope}%
\pgfpathrectangle{\pgfqpoint{0.765000in}{0.660000in}}{\pgfqpoint{4.620000in}{4.620000in}}%
\pgfusepath{clip}%
\pgfsetbuttcap%
\pgfsetroundjoin%
\definecolor{currentfill}{rgb}{0.176471,0.192157,0.258824}%
\pgfsetfillcolor{currentfill}%
\pgfsetfillopacity{0.050000}%
\pgfsetlinewidth{0.000000pt}%
\definecolor{currentstroke}{rgb}{0.176471,0.192157,0.258824}%
\pgfsetstrokecolor{currentstroke}%
\pgfsetdash{}{0pt}%
\pgfpathmoveto{\pgfqpoint{3.140948in}{2.989652in}}%
\pgfpathlineto{\pgfqpoint{3.140948in}{3.062656in}}%
\pgfpathlineto{\pgfqpoint{3.218901in}{3.026059in}}%
\pgfpathlineto{\pgfqpoint{3.218901in}{2.953056in}}%
\pgfpathlineto{\pgfqpoint{3.140948in}{2.989652in}}%
\pgfpathclose%
\pgfusepath{fill}%
\end{pgfscope}%
\begin{pgfscope}%
\pgfpathrectangle{\pgfqpoint{0.765000in}{0.660000in}}{\pgfqpoint{4.620000in}{4.620000in}}%
\pgfusepath{clip}%
\pgfsetbuttcap%
\pgfsetroundjoin%
\definecolor{currentfill}{rgb}{0.176471,0.192157,0.258824}%
\pgfsetfillcolor{currentfill}%
\pgfsetfillopacity{0.050000}%
\pgfsetlinewidth{0.000000pt}%
\definecolor{currentstroke}{rgb}{0.176471,0.192157,0.258824}%
\pgfsetstrokecolor{currentstroke}%
\pgfsetdash{}{0pt}%
\pgfpathmoveto{\pgfqpoint{3.140948in}{2.989652in}}%
\pgfpathlineto{\pgfqpoint{3.140948in}{3.062656in}}%
\pgfpathlineto{\pgfqpoint{3.062995in}{3.026059in}}%
\pgfpathlineto{\pgfqpoint{3.062995in}{2.953056in}}%
\pgfpathlineto{\pgfqpoint{3.140948in}{2.989652in}}%
\pgfpathclose%
\pgfusepath{fill}%
\end{pgfscope}%
\begin{pgfscope}%
\pgfpathrectangle{\pgfqpoint{0.765000in}{0.660000in}}{\pgfqpoint{4.620000in}{4.620000in}}%
\pgfusepath{clip}%
\pgfsetbuttcap%
\pgfsetroundjoin%
\definecolor{currentfill}{rgb}{0.176471,0.192157,0.258824}%
\pgfsetfillcolor{currentfill}%
\pgfsetfillopacity{0.050000}%
\pgfsetlinewidth{0.000000pt}%
\definecolor{currentstroke}{rgb}{0.176471,0.192157,0.258824}%
\pgfsetstrokecolor{currentstroke}%
\pgfsetdash{}{0pt}%
\pgfpathmoveto{\pgfqpoint{3.140948in}{2.989652in}}%
\pgfpathlineto{\pgfqpoint{3.218901in}{2.953056in}}%
\pgfpathlineto{\pgfqpoint{3.140948in}{2.916459in}}%
\pgfpathlineto{\pgfqpoint{3.062995in}{2.953056in}}%
\pgfpathlineto{\pgfqpoint{3.140948in}{2.989652in}}%
\pgfpathclose%
\pgfusepath{fill}%
\end{pgfscope}%
\begin{pgfscope}%
\pgfpathrectangle{\pgfqpoint{0.765000in}{0.660000in}}{\pgfqpoint{4.620000in}{4.620000in}}%
\pgfusepath{clip}%
\pgfsetbuttcap%
\pgfsetroundjoin%
\definecolor{currentfill}{rgb}{0.176471,0.192157,0.258824}%
\pgfsetfillcolor{currentfill}%
\pgfsetfillopacity{0.050000}%
\pgfsetlinewidth{0.000000pt}%
\definecolor{currentstroke}{rgb}{0.176471,0.192157,0.258824}%
\pgfsetstrokecolor{currentstroke}%
\pgfsetdash{}{0pt}%
\pgfpathmoveto{\pgfqpoint{3.140948in}{3.062656in}}%
\pgfpathlineto{\pgfqpoint{3.218901in}{3.026059in}}%
\pgfpathlineto{\pgfqpoint{3.140948in}{2.989463in}}%
\pgfpathlineto{\pgfqpoint{3.062995in}{3.026059in}}%
\pgfpathlineto{\pgfqpoint{3.140948in}{3.062656in}}%
\pgfpathclose%
\pgfusepath{fill}%
\end{pgfscope}%
\begin{pgfscope}%
\pgfpathrectangle{\pgfqpoint{0.765000in}{0.660000in}}{\pgfqpoint{4.620000in}{4.620000in}}%
\pgfusepath{clip}%
\pgfsetbuttcap%
\pgfsetroundjoin%
\definecolor{currentfill}{rgb}{0.176471,0.192157,0.258824}%
\pgfsetfillcolor{currentfill}%
\pgfsetfillopacity{0.050000}%
\pgfsetlinewidth{0.000000pt}%
\definecolor{currentstroke}{rgb}{0.176471,0.192157,0.258824}%
\pgfsetstrokecolor{currentstroke}%
\pgfsetdash{}{0pt}%
\pgfpathmoveto{\pgfqpoint{3.062995in}{2.953056in}}%
\pgfpathlineto{\pgfqpoint{3.062995in}{3.026059in}}%
\pgfpathlineto{\pgfqpoint{3.140948in}{2.989463in}}%
\pgfpathlineto{\pgfqpoint{3.140948in}{2.916459in}}%
\pgfpathlineto{\pgfqpoint{3.062995in}{2.953056in}}%
\pgfpathclose%
\pgfusepath{fill}%
\end{pgfscope}%
\begin{pgfscope}%
\pgfpathrectangle{\pgfqpoint{0.765000in}{0.660000in}}{\pgfqpoint{4.620000in}{4.620000in}}%
\pgfusepath{clip}%
\pgfsetbuttcap%
\pgfsetroundjoin%
\definecolor{currentfill}{rgb}{0.176471,0.192157,0.258824}%
\pgfsetfillcolor{currentfill}%
\pgfsetfillopacity{0.050000}%
\pgfsetlinewidth{0.000000pt}%
\definecolor{currentstroke}{rgb}{0.176471,0.192157,0.258824}%
\pgfsetstrokecolor{currentstroke}%
\pgfsetdash{}{0pt}%
\pgfpathmoveto{\pgfqpoint{3.218901in}{2.953056in}}%
\pgfpathlineto{\pgfqpoint{3.218901in}{3.026059in}}%
\pgfpathlineto{\pgfqpoint{3.140948in}{2.989463in}}%
\pgfpathlineto{\pgfqpoint{3.140948in}{2.916459in}}%
\pgfpathlineto{\pgfqpoint{3.218901in}{2.953056in}}%
\pgfpathclose%
\pgfusepath{fill}%
\end{pgfscope}%
\begin{pgfscope}%
\pgfpathrectangle{\pgfqpoint{0.765000in}{0.660000in}}{\pgfqpoint{4.620000in}{4.620000in}}%
\pgfusepath{clip}%
\pgfsetbuttcap%
\pgfsetroundjoin%
\definecolor{currentfill}{rgb}{1.000000,0.576471,0.309804}%
\pgfsetfillcolor{currentfill}%
\pgfsetfillopacity{0.050000}%
\pgfsetlinewidth{0.000000pt}%
\definecolor{currentstroke}{rgb}{1.000000,0.576471,0.309804}%
\pgfsetstrokecolor{currentstroke}%
\pgfsetdash{}{0pt}%
\pgfpathmoveto{\pgfqpoint{2.369009in}{2.619569in}}%
\pgfpathlineto{\pgfqpoint{2.369009in}{2.692573in}}%
\pgfpathlineto{\pgfqpoint{2.446962in}{2.655976in}}%
\pgfpathlineto{\pgfqpoint{2.446962in}{2.582973in}}%
\pgfpathlineto{\pgfqpoint{2.369009in}{2.619569in}}%
\pgfpathclose%
\pgfusepath{fill}%
\end{pgfscope}%
\begin{pgfscope}%
\pgfpathrectangle{\pgfqpoint{0.765000in}{0.660000in}}{\pgfqpoint{4.620000in}{4.620000in}}%
\pgfusepath{clip}%
\pgfsetbuttcap%
\pgfsetroundjoin%
\definecolor{currentfill}{rgb}{1.000000,0.576471,0.309804}%
\pgfsetfillcolor{currentfill}%
\pgfsetfillopacity{0.050000}%
\pgfsetlinewidth{0.000000pt}%
\definecolor{currentstroke}{rgb}{1.000000,0.576471,0.309804}%
\pgfsetstrokecolor{currentstroke}%
\pgfsetdash{}{0pt}%
\pgfpathmoveto{\pgfqpoint{2.369009in}{2.619569in}}%
\pgfpathlineto{\pgfqpoint{2.369009in}{2.692573in}}%
\pgfpathlineto{\pgfqpoint{2.291056in}{2.655976in}}%
\pgfpathlineto{\pgfqpoint{2.291056in}{2.582973in}}%
\pgfpathlineto{\pgfqpoint{2.369009in}{2.619569in}}%
\pgfpathclose%
\pgfusepath{fill}%
\end{pgfscope}%
\begin{pgfscope}%
\pgfpathrectangle{\pgfqpoint{0.765000in}{0.660000in}}{\pgfqpoint{4.620000in}{4.620000in}}%
\pgfusepath{clip}%
\pgfsetbuttcap%
\pgfsetroundjoin%
\definecolor{currentfill}{rgb}{1.000000,0.576471,0.309804}%
\pgfsetfillcolor{currentfill}%
\pgfsetfillopacity{0.050000}%
\pgfsetlinewidth{0.000000pt}%
\definecolor{currentstroke}{rgb}{1.000000,0.576471,0.309804}%
\pgfsetstrokecolor{currentstroke}%
\pgfsetdash{}{0pt}%
\pgfpathmoveto{\pgfqpoint{2.369009in}{2.619569in}}%
\pgfpathlineto{\pgfqpoint{2.446962in}{2.582973in}}%
\pgfpathlineto{\pgfqpoint{2.369009in}{2.546376in}}%
\pgfpathlineto{\pgfqpoint{2.291056in}{2.582973in}}%
\pgfpathlineto{\pgfqpoint{2.369009in}{2.619569in}}%
\pgfpathclose%
\pgfusepath{fill}%
\end{pgfscope}%
\begin{pgfscope}%
\pgfpathrectangle{\pgfqpoint{0.765000in}{0.660000in}}{\pgfqpoint{4.620000in}{4.620000in}}%
\pgfusepath{clip}%
\pgfsetbuttcap%
\pgfsetroundjoin%
\definecolor{currentfill}{rgb}{1.000000,0.576471,0.309804}%
\pgfsetfillcolor{currentfill}%
\pgfsetfillopacity{0.050000}%
\pgfsetlinewidth{0.000000pt}%
\definecolor{currentstroke}{rgb}{1.000000,0.576471,0.309804}%
\pgfsetstrokecolor{currentstroke}%
\pgfsetdash{}{0pt}%
\pgfpathmoveto{\pgfqpoint{2.369009in}{2.692573in}}%
\pgfpathlineto{\pgfqpoint{2.446962in}{2.655976in}}%
\pgfpathlineto{\pgfqpoint{2.369009in}{2.619379in}}%
\pgfpathlineto{\pgfqpoint{2.291056in}{2.655976in}}%
\pgfpathlineto{\pgfqpoint{2.369009in}{2.692573in}}%
\pgfpathclose%
\pgfusepath{fill}%
\end{pgfscope}%
\begin{pgfscope}%
\pgfpathrectangle{\pgfqpoint{0.765000in}{0.660000in}}{\pgfqpoint{4.620000in}{4.620000in}}%
\pgfusepath{clip}%
\pgfsetbuttcap%
\pgfsetroundjoin%
\definecolor{currentfill}{rgb}{1.000000,0.576471,0.309804}%
\pgfsetfillcolor{currentfill}%
\pgfsetfillopacity{0.050000}%
\pgfsetlinewidth{0.000000pt}%
\definecolor{currentstroke}{rgb}{1.000000,0.576471,0.309804}%
\pgfsetstrokecolor{currentstroke}%
\pgfsetdash{}{0pt}%
\pgfpathmoveto{\pgfqpoint{2.291056in}{2.582973in}}%
\pgfpathlineto{\pgfqpoint{2.291056in}{2.655976in}}%
\pgfpathlineto{\pgfqpoint{2.369009in}{2.619379in}}%
\pgfpathlineto{\pgfqpoint{2.369009in}{2.546376in}}%
\pgfpathlineto{\pgfqpoint{2.291056in}{2.582973in}}%
\pgfpathclose%
\pgfusepath{fill}%
\end{pgfscope}%
\begin{pgfscope}%
\pgfpathrectangle{\pgfqpoint{0.765000in}{0.660000in}}{\pgfqpoint{4.620000in}{4.620000in}}%
\pgfusepath{clip}%
\pgfsetbuttcap%
\pgfsetroundjoin%
\definecolor{currentfill}{rgb}{1.000000,0.576471,0.309804}%
\pgfsetfillcolor{currentfill}%
\pgfsetfillopacity{0.050000}%
\pgfsetlinewidth{0.000000pt}%
\definecolor{currentstroke}{rgb}{1.000000,0.576471,0.309804}%
\pgfsetstrokecolor{currentstroke}%
\pgfsetdash{}{0pt}%
\pgfpathmoveto{\pgfqpoint{2.446962in}{2.582973in}}%
\pgfpathlineto{\pgfqpoint{2.446962in}{2.655976in}}%
\pgfpathlineto{\pgfqpoint{2.369009in}{2.619379in}}%
\pgfpathlineto{\pgfqpoint{2.369009in}{2.546376in}}%
\pgfpathlineto{\pgfqpoint{2.446962in}{2.582973in}}%
\pgfpathclose%
\pgfusepath{fill}%
\end{pgfscope}%
\begin{pgfscope}%
\pgfpathrectangle{\pgfqpoint{0.765000in}{0.660000in}}{\pgfqpoint{4.620000in}{4.620000in}}%
\pgfusepath{clip}%
\pgfsetbuttcap%
\pgfsetroundjoin%
\definecolor{currentfill}{rgb}{0.176471,0.192157,0.258824}%
\pgfsetfillcolor{currentfill}%
\pgfsetfillopacity{0.050000}%
\pgfsetlinewidth{0.000000pt}%
\definecolor{currentstroke}{rgb}{0.176471,0.192157,0.258824}%
\pgfsetstrokecolor{currentstroke}%
\pgfsetdash{}{0pt}%
\pgfpathmoveto{\pgfqpoint{3.186375in}{2.986644in}}%
\pgfpathlineto{\pgfqpoint{3.186375in}{3.059647in}}%
\pgfpathlineto{\pgfqpoint{3.264328in}{3.023051in}}%
\pgfpathlineto{\pgfqpoint{3.264328in}{2.950047in}}%
\pgfpathlineto{\pgfqpoint{3.186375in}{2.986644in}}%
\pgfpathclose%
\pgfusepath{fill}%
\end{pgfscope}%
\begin{pgfscope}%
\pgfpathrectangle{\pgfqpoint{0.765000in}{0.660000in}}{\pgfqpoint{4.620000in}{4.620000in}}%
\pgfusepath{clip}%
\pgfsetbuttcap%
\pgfsetroundjoin%
\definecolor{currentfill}{rgb}{0.176471,0.192157,0.258824}%
\pgfsetfillcolor{currentfill}%
\pgfsetfillopacity{0.050000}%
\pgfsetlinewidth{0.000000pt}%
\definecolor{currentstroke}{rgb}{0.176471,0.192157,0.258824}%
\pgfsetstrokecolor{currentstroke}%
\pgfsetdash{}{0pt}%
\pgfpathmoveto{\pgfqpoint{3.186375in}{2.986644in}}%
\pgfpathlineto{\pgfqpoint{3.186375in}{3.059647in}}%
\pgfpathlineto{\pgfqpoint{3.108422in}{3.023051in}}%
\pgfpathlineto{\pgfqpoint{3.108422in}{2.950047in}}%
\pgfpathlineto{\pgfqpoint{3.186375in}{2.986644in}}%
\pgfpathclose%
\pgfusepath{fill}%
\end{pgfscope}%
\begin{pgfscope}%
\pgfpathrectangle{\pgfqpoint{0.765000in}{0.660000in}}{\pgfqpoint{4.620000in}{4.620000in}}%
\pgfusepath{clip}%
\pgfsetbuttcap%
\pgfsetroundjoin%
\definecolor{currentfill}{rgb}{0.176471,0.192157,0.258824}%
\pgfsetfillcolor{currentfill}%
\pgfsetfillopacity{0.050000}%
\pgfsetlinewidth{0.000000pt}%
\definecolor{currentstroke}{rgb}{0.176471,0.192157,0.258824}%
\pgfsetstrokecolor{currentstroke}%
\pgfsetdash{}{0pt}%
\pgfpathmoveto{\pgfqpoint{3.186375in}{2.986644in}}%
\pgfpathlineto{\pgfqpoint{3.264328in}{2.950047in}}%
\pgfpathlineto{\pgfqpoint{3.186375in}{2.913451in}}%
\pgfpathlineto{\pgfqpoint{3.108422in}{2.950047in}}%
\pgfpathlineto{\pgfqpoint{3.186375in}{2.986644in}}%
\pgfpathclose%
\pgfusepath{fill}%
\end{pgfscope}%
\begin{pgfscope}%
\pgfpathrectangle{\pgfqpoint{0.765000in}{0.660000in}}{\pgfqpoint{4.620000in}{4.620000in}}%
\pgfusepath{clip}%
\pgfsetbuttcap%
\pgfsetroundjoin%
\definecolor{currentfill}{rgb}{0.176471,0.192157,0.258824}%
\pgfsetfillcolor{currentfill}%
\pgfsetfillopacity{0.050000}%
\pgfsetlinewidth{0.000000pt}%
\definecolor{currentstroke}{rgb}{0.176471,0.192157,0.258824}%
\pgfsetstrokecolor{currentstroke}%
\pgfsetdash{}{0pt}%
\pgfpathmoveto{\pgfqpoint{3.186375in}{3.059647in}}%
\pgfpathlineto{\pgfqpoint{3.264328in}{3.023051in}}%
\pgfpathlineto{\pgfqpoint{3.186375in}{2.986454in}}%
\pgfpathlineto{\pgfqpoint{3.108422in}{3.023051in}}%
\pgfpathlineto{\pgfqpoint{3.186375in}{3.059647in}}%
\pgfpathclose%
\pgfusepath{fill}%
\end{pgfscope}%
\begin{pgfscope}%
\pgfpathrectangle{\pgfqpoint{0.765000in}{0.660000in}}{\pgfqpoint{4.620000in}{4.620000in}}%
\pgfusepath{clip}%
\pgfsetbuttcap%
\pgfsetroundjoin%
\definecolor{currentfill}{rgb}{0.176471,0.192157,0.258824}%
\pgfsetfillcolor{currentfill}%
\pgfsetfillopacity{0.050000}%
\pgfsetlinewidth{0.000000pt}%
\definecolor{currentstroke}{rgb}{0.176471,0.192157,0.258824}%
\pgfsetstrokecolor{currentstroke}%
\pgfsetdash{}{0pt}%
\pgfpathmoveto{\pgfqpoint{3.108422in}{2.950047in}}%
\pgfpathlineto{\pgfqpoint{3.108422in}{3.023051in}}%
\pgfpathlineto{\pgfqpoint{3.186375in}{2.986454in}}%
\pgfpathlineto{\pgfqpoint{3.186375in}{2.913451in}}%
\pgfpathlineto{\pgfqpoint{3.108422in}{2.950047in}}%
\pgfpathclose%
\pgfusepath{fill}%
\end{pgfscope}%
\begin{pgfscope}%
\pgfpathrectangle{\pgfqpoint{0.765000in}{0.660000in}}{\pgfqpoint{4.620000in}{4.620000in}}%
\pgfusepath{clip}%
\pgfsetbuttcap%
\pgfsetroundjoin%
\definecolor{currentfill}{rgb}{0.176471,0.192157,0.258824}%
\pgfsetfillcolor{currentfill}%
\pgfsetfillopacity{0.050000}%
\pgfsetlinewidth{0.000000pt}%
\definecolor{currentstroke}{rgb}{0.176471,0.192157,0.258824}%
\pgfsetstrokecolor{currentstroke}%
\pgfsetdash{}{0pt}%
\pgfpathmoveto{\pgfqpoint{3.264328in}{2.950047in}}%
\pgfpathlineto{\pgfqpoint{3.264328in}{3.023051in}}%
\pgfpathlineto{\pgfqpoint{3.186375in}{2.986454in}}%
\pgfpathlineto{\pgfqpoint{3.186375in}{2.913451in}}%
\pgfpathlineto{\pgfqpoint{3.264328in}{2.950047in}}%
\pgfpathclose%
\pgfusepath{fill}%
\end{pgfscope}%
\begin{pgfscope}%
\pgfpathrectangle{\pgfqpoint{0.765000in}{0.660000in}}{\pgfqpoint{4.620000in}{4.620000in}}%
\pgfusepath{clip}%
\pgfsetbuttcap%
\pgfsetroundjoin%
\definecolor{currentfill}{rgb}{0.176471,0.192157,0.258824}%
\pgfsetfillcolor{currentfill}%
\pgfsetfillopacity{0.050000}%
\pgfsetlinewidth{0.000000pt}%
\definecolor{currentstroke}{rgb}{0.176471,0.192157,0.258824}%
\pgfsetstrokecolor{currentstroke}%
\pgfsetdash{}{0pt}%
\pgfpathmoveto{\pgfqpoint{3.051075in}{2.864972in}}%
\pgfpathlineto{\pgfqpoint{3.051075in}{2.937976in}}%
\pgfpathlineto{\pgfqpoint{3.129028in}{2.901379in}}%
\pgfpathlineto{\pgfqpoint{3.129028in}{2.828376in}}%
\pgfpathlineto{\pgfqpoint{3.051075in}{2.864972in}}%
\pgfpathclose%
\pgfusepath{fill}%
\end{pgfscope}%
\begin{pgfscope}%
\pgfpathrectangle{\pgfqpoint{0.765000in}{0.660000in}}{\pgfqpoint{4.620000in}{4.620000in}}%
\pgfusepath{clip}%
\pgfsetbuttcap%
\pgfsetroundjoin%
\definecolor{currentfill}{rgb}{0.176471,0.192157,0.258824}%
\pgfsetfillcolor{currentfill}%
\pgfsetfillopacity{0.050000}%
\pgfsetlinewidth{0.000000pt}%
\definecolor{currentstroke}{rgb}{0.176471,0.192157,0.258824}%
\pgfsetstrokecolor{currentstroke}%
\pgfsetdash{}{0pt}%
\pgfpathmoveto{\pgfqpoint{3.051075in}{2.864972in}}%
\pgfpathlineto{\pgfqpoint{3.051075in}{2.937976in}}%
\pgfpathlineto{\pgfqpoint{2.973123in}{2.901379in}}%
\pgfpathlineto{\pgfqpoint{2.973123in}{2.828376in}}%
\pgfpathlineto{\pgfqpoint{3.051075in}{2.864972in}}%
\pgfpathclose%
\pgfusepath{fill}%
\end{pgfscope}%
\begin{pgfscope}%
\pgfpathrectangle{\pgfqpoint{0.765000in}{0.660000in}}{\pgfqpoint{4.620000in}{4.620000in}}%
\pgfusepath{clip}%
\pgfsetbuttcap%
\pgfsetroundjoin%
\definecolor{currentfill}{rgb}{0.176471,0.192157,0.258824}%
\pgfsetfillcolor{currentfill}%
\pgfsetfillopacity{0.050000}%
\pgfsetlinewidth{0.000000pt}%
\definecolor{currentstroke}{rgb}{0.176471,0.192157,0.258824}%
\pgfsetstrokecolor{currentstroke}%
\pgfsetdash{}{0pt}%
\pgfpathmoveto{\pgfqpoint{3.051075in}{2.864972in}}%
\pgfpathlineto{\pgfqpoint{3.129028in}{2.828376in}}%
\pgfpathlineto{\pgfqpoint{3.051075in}{2.791779in}}%
\pgfpathlineto{\pgfqpoint{2.973123in}{2.828376in}}%
\pgfpathlineto{\pgfqpoint{3.051075in}{2.864972in}}%
\pgfpathclose%
\pgfusepath{fill}%
\end{pgfscope}%
\begin{pgfscope}%
\pgfpathrectangle{\pgfqpoint{0.765000in}{0.660000in}}{\pgfqpoint{4.620000in}{4.620000in}}%
\pgfusepath{clip}%
\pgfsetbuttcap%
\pgfsetroundjoin%
\definecolor{currentfill}{rgb}{0.176471,0.192157,0.258824}%
\pgfsetfillcolor{currentfill}%
\pgfsetfillopacity{0.050000}%
\pgfsetlinewidth{0.000000pt}%
\definecolor{currentstroke}{rgb}{0.176471,0.192157,0.258824}%
\pgfsetstrokecolor{currentstroke}%
\pgfsetdash{}{0pt}%
\pgfpathmoveto{\pgfqpoint{3.051075in}{2.937976in}}%
\pgfpathlineto{\pgfqpoint{3.129028in}{2.901379in}}%
\pgfpathlineto{\pgfqpoint{3.051075in}{2.864783in}}%
\pgfpathlineto{\pgfqpoint{2.973123in}{2.901379in}}%
\pgfpathlineto{\pgfqpoint{3.051075in}{2.937976in}}%
\pgfpathclose%
\pgfusepath{fill}%
\end{pgfscope}%
\begin{pgfscope}%
\pgfpathrectangle{\pgfqpoint{0.765000in}{0.660000in}}{\pgfqpoint{4.620000in}{4.620000in}}%
\pgfusepath{clip}%
\pgfsetbuttcap%
\pgfsetroundjoin%
\definecolor{currentfill}{rgb}{0.176471,0.192157,0.258824}%
\pgfsetfillcolor{currentfill}%
\pgfsetfillopacity{0.050000}%
\pgfsetlinewidth{0.000000pt}%
\definecolor{currentstroke}{rgb}{0.176471,0.192157,0.258824}%
\pgfsetstrokecolor{currentstroke}%
\pgfsetdash{}{0pt}%
\pgfpathmoveto{\pgfqpoint{2.973123in}{2.828376in}}%
\pgfpathlineto{\pgfqpoint{2.973123in}{2.901379in}}%
\pgfpathlineto{\pgfqpoint{3.051075in}{2.864783in}}%
\pgfpathlineto{\pgfqpoint{3.051075in}{2.791779in}}%
\pgfpathlineto{\pgfqpoint{2.973123in}{2.828376in}}%
\pgfpathclose%
\pgfusepath{fill}%
\end{pgfscope}%
\begin{pgfscope}%
\pgfpathrectangle{\pgfqpoint{0.765000in}{0.660000in}}{\pgfqpoint{4.620000in}{4.620000in}}%
\pgfusepath{clip}%
\pgfsetbuttcap%
\pgfsetroundjoin%
\definecolor{currentfill}{rgb}{0.176471,0.192157,0.258824}%
\pgfsetfillcolor{currentfill}%
\pgfsetfillopacity{0.050000}%
\pgfsetlinewidth{0.000000pt}%
\definecolor{currentstroke}{rgb}{0.176471,0.192157,0.258824}%
\pgfsetstrokecolor{currentstroke}%
\pgfsetdash{}{0pt}%
\pgfpathmoveto{\pgfqpoint{3.129028in}{2.828376in}}%
\pgfpathlineto{\pgfqpoint{3.129028in}{2.901379in}}%
\pgfpathlineto{\pgfqpoint{3.051075in}{2.864783in}}%
\pgfpathlineto{\pgfqpoint{3.051075in}{2.791779in}}%
\pgfpathlineto{\pgfqpoint{3.129028in}{2.828376in}}%
\pgfpathclose%
\pgfusepath{fill}%
\end{pgfscope}%
\begin{pgfscope}%
\pgfpathrectangle{\pgfqpoint{0.765000in}{0.660000in}}{\pgfqpoint{4.620000in}{4.620000in}}%
\pgfusepath{clip}%
\pgfsetbuttcap%
\pgfsetroundjoin%
\definecolor{currentfill}{rgb}{0.176471,0.192157,0.258824}%
\pgfsetfillcolor{currentfill}%
\pgfsetfillopacity{0.050000}%
\pgfsetlinewidth{0.000000pt}%
\definecolor{currentstroke}{rgb}{0.176471,0.192157,0.258824}%
\pgfsetstrokecolor{currentstroke}%
\pgfsetdash{}{0pt}%
\pgfpathmoveto{\pgfqpoint{3.046922in}{3.003464in}}%
\pgfpathlineto{\pgfqpoint{3.046922in}{3.076467in}}%
\pgfpathlineto{\pgfqpoint{3.124874in}{3.039871in}}%
\pgfpathlineto{\pgfqpoint{3.124874in}{2.966867in}}%
\pgfpathlineto{\pgfqpoint{3.046922in}{3.003464in}}%
\pgfpathclose%
\pgfusepath{fill}%
\end{pgfscope}%
\begin{pgfscope}%
\pgfpathrectangle{\pgfqpoint{0.765000in}{0.660000in}}{\pgfqpoint{4.620000in}{4.620000in}}%
\pgfusepath{clip}%
\pgfsetbuttcap%
\pgfsetroundjoin%
\definecolor{currentfill}{rgb}{0.176471,0.192157,0.258824}%
\pgfsetfillcolor{currentfill}%
\pgfsetfillopacity{0.050000}%
\pgfsetlinewidth{0.000000pt}%
\definecolor{currentstroke}{rgb}{0.176471,0.192157,0.258824}%
\pgfsetstrokecolor{currentstroke}%
\pgfsetdash{}{0pt}%
\pgfpathmoveto{\pgfqpoint{3.046922in}{3.003464in}}%
\pgfpathlineto{\pgfqpoint{3.046922in}{3.076467in}}%
\pgfpathlineto{\pgfqpoint{2.968969in}{3.039871in}}%
\pgfpathlineto{\pgfqpoint{2.968969in}{2.966867in}}%
\pgfpathlineto{\pgfqpoint{3.046922in}{3.003464in}}%
\pgfpathclose%
\pgfusepath{fill}%
\end{pgfscope}%
\begin{pgfscope}%
\pgfpathrectangle{\pgfqpoint{0.765000in}{0.660000in}}{\pgfqpoint{4.620000in}{4.620000in}}%
\pgfusepath{clip}%
\pgfsetbuttcap%
\pgfsetroundjoin%
\definecolor{currentfill}{rgb}{0.176471,0.192157,0.258824}%
\pgfsetfillcolor{currentfill}%
\pgfsetfillopacity{0.050000}%
\pgfsetlinewidth{0.000000pt}%
\definecolor{currentstroke}{rgb}{0.176471,0.192157,0.258824}%
\pgfsetstrokecolor{currentstroke}%
\pgfsetdash{}{0pt}%
\pgfpathmoveto{\pgfqpoint{3.046922in}{3.003464in}}%
\pgfpathlineto{\pgfqpoint{3.124874in}{2.966867in}}%
\pgfpathlineto{\pgfqpoint{3.046922in}{2.930271in}}%
\pgfpathlineto{\pgfqpoint{2.968969in}{2.966867in}}%
\pgfpathlineto{\pgfqpoint{3.046922in}{3.003464in}}%
\pgfpathclose%
\pgfusepath{fill}%
\end{pgfscope}%
\begin{pgfscope}%
\pgfpathrectangle{\pgfqpoint{0.765000in}{0.660000in}}{\pgfqpoint{4.620000in}{4.620000in}}%
\pgfusepath{clip}%
\pgfsetbuttcap%
\pgfsetroundjoin%
\definecolor{currentfill}{rgb}{0.176471,0.192157,0.258824}%
\pgfsetfillcolor{currentfill}%
\pgfsetfillopacity{0.050000}%
\pgfsetlinewidth{0.000000pt}%
\definecolor{currentstroke}{rgb}{0.176471,0.192157,0.258824}%
\pgfsetstrokecolor{currentstroke}%
\pgfsetdash{}{0pt}%
\pgfpathmoveto{\pgfqpoint{3.046922in}{3.076467in}}%
\pgfpathlineto{\pgfqpoint{3.124874in}{3.039871in}}%
\pgfpathlineto{\pgfqpoint{3.046922in}{3.003274in}}%
\pgfpathlineto{\pgfqpoint{2.968969in}{3.039871in}}%
\pgfpathlineto{\pgfqpoint{3.046922in}{3.076467in}}%
\pgfpathclose%
\pgfusepath{fill}%
\end{pgfscope}%
\begin{pgfscope}%
\pgfpathrectangle{\pgfqpoint{0.765000in}{0.660000in}}{\pgfqpoint{4.620000in}{4.620000in}}%
\pgfusepath{clip}%
\pgfsetbuttcap%
\pgfsetroundjoin%
\definecolor{currentfill}{rgb}{0.176471,0.192157,0.258824}%
\pgfsetfillcolor{currentfill}%
\pgfsetfillopacity{0.050000}%
\pgfsetlinewidth{0.000000pt}%
\definecolor{currentstroke}{rgb}{0.176471,0.192157,0.258824}%
\pgfsetstrokecolor{currentstroke}%
\pgfsetdash{}{0pt}%
\pgfpathmoveto{\pgfqpoint{2.968969in}{2.966867in}}%
\pgfpathlineto{\pgfqpoint{2.968969in}{3.039871in}}%
\pgfpathlineto{\pgfqpoint{3.046922in}{3.003274in}}%
\pgfpathlineto{\pgfqpoint{3.046922in}{2.930271in}}%
\pgfpathlineto{\pgfqpoint{2.968969in}{2.966867in}}%
\pgfpathclose%
\pgfusepath{fill}%
\end{pgfscope}%
\begin{pgfscope}%
\pgfpathrectangle{\pgfqpoint{0.765000in}{0.660000in}}{\pgfqpoint{4.620000in}{4.620000in}}%
\pgfusepath{clip}%
\pgfsetbuttcap%
\pgfsetroundjoin%
\definecolor{currentfill}{rgb}{0.176471,0.192157,0.258824}%
\pgfsetfillcolor{currentfill}%
\pgfsetfillopacity{0.050000}%
\pgfsetlinewidth{0.000000pt}%
\definecolor{currentstroke}{rgb}{0.176471,0.192157,0.258824}%
\pgfsetstrokecolor{currentstroke}%
\pgfsetdash{}{0pt}%
\pgfpathmoveto{\pgfqpoint{3.124874in}{2.966867in}}%
\pgfpathlineto{\pgfqpoint{3.124874in}{3.039871in}}%
\pgfpathlineto{\pgfqpoint{3.046922in}{3.003274in}}%
\pgfpathlineto{\pgfqpoint{3.046922in}{2.930271in}}%
\pgfpathlineto{\pgfqpoint{3.124874in}{2.966867in}}%
\pgfpathclose%
\pgfusepath{fill}%
\end{pgfscope}%
\begin{pgfscope}%
\pgfpathrectangle{\pgfqpoint{0.765000in}{0.660000in}}{\pgfqpoint{4.620000in}{4.620000in}}%
\pgfusepath{clip}%
\pgfsetbuttcap%
\pgfsetroundjoin%
\definecolor{currentfill}{rgb}{1.000000,0.576471,0.309804}%
\pgfsetfillcolor{currentfill}%
\pgfsetfillopacity{0.050000}%
\pgfsetlinewidth{0.000000pt}%
\definecolor{currentstroke}{rgb}{1.000000,0.576471,0.309804}%
\pgfsetstrokecolor{currentstroke}%
\pgfsetdash{}{0pt}%
\pgfpathmoveto{\pgfqpoint{3.288908in}{3.839949in}}%
\pgfpathlineto{\pgfqpoint{3.288908in}{3.912952in}}%
\pgfpathlineto{\pgfqpoint{3.366860in}{3.876356in}}%
\pgfpathlineto{\pgfqpoint{3.366860in}{3.803353in}}%
\pgfpathlineto{\pgfqpoint{3.288908in}{3.839949in}}%
\pgfpathclose%
\pgfusepath{fill}%
\end{pgfscope}%
\begin{pgfscope}%
\pgfpathrectangle{\pgfqpoint{0.765000in}{0.660000in}}{\pgfqpoint{4.620000in}{4.620000in}}%
\pgfusepath{clip}%
\pgfsetbuttcap%
\pgfsetroundjoin%
\definecolor{currentfill}{rgb}{1.000000,0.576471,0.309804}%
\pgfsetfillcolor{currentfill}%
\pgfsetfillopacity{0.050000}%
\pgfsetlinewidth{0.000000pt}%
\definecolor{currentstroke}{rgb}{1.000000,0.576471,0.309804}%
\pgfsetstrokecolor{currentstroke}%
\pgfsetdash{}{0pt}%
\pgfpathmoveto{\pgfqpoint{3.288908in}{3.839949in}}%
\pgfpathlineto{\pgfqpoint{3.288908in}{3.912952in}}%
\pgfpathlineto{\pgfqpoint{3.210955in}{3.876356in}}%
\pgfpathlineto{\pgfqpoint{3.210955in}{3.803353in}}%
\pgfpathlineto{\pgfqpoint{3.288908in}{3.839949in}}%
\pgfpathclose%
\pgfusepath{fill}%
\end{pgfscope}%
\begin{pgfscope}%
\pgfpathrectangle{\pgfqpoint{0.765000in}{0.660000in}}{\pgfqpoint{4.620000in}{4.620000in}}%
\pgfusepath{clip}%
\pgfsetbuttcap%
\pgfsetroundjoin%
\definecolor{currentfill}{rgb}{1.000000,0.576471,0.309804}%
\pgfsetfillcolor{currentfill}%
\pgfsetfillopacity{0.050000}%
\pgfsetlinewidth{0.000000pt}%
\definecolor{currentstroke}{rgb}{1.000000,0.576471,0.309804}%
\pgfsetstrokecolor{currentstroke}%
\pgfsetdash{}{0pt}%
\pgfpathmoveto{\pgfqpoint{3.288908in}{3.839949in}}%
\pgfpathlineto{\pgfqpoint{3.366860in}{3.803353in}}%
\pgfpathlineto{\pgfqpoint{3.288908in}{3.766756in}}%
\pgfpathlineto{\pgfqpoint{3.210955in}{3.803353in}}%
\pgfpathlineto{\pgfqpoint{3.288908in}{3.839949in}}%
\pgfpathclose%
\pgfusepath{fill}%
\end{pgfscope}%
\begin{pgfscope}%
\pgfpathrectangle{\pgfqpoint{0.765000in}{0.660000in}}{\pgfqpoint{4.620000in}{4.620000in}}%
\pgfusepath{clip}%
\pgfsetbuttcap%
\pgfsetroundjoin%
\definecolor{currentfill}{rgb}{1.000000,0.576471,0.309804}%
\pgfsetfillcolor{currentfill}%
\pgfsetfillopacity{0.050000}%
\pgfsetlinewidth{0.000000pt}%
\definecolor{currentstroke}{rgb}{1.000000,0.576471,0.309804}%
\pgfsetstrokecolor{currentstroke}%
\pgfsetdash{}{0pt}%
\pgfpathmoveto{\pgfqpoint{3.288908in}{3.912952in}}%
\pgfpathlineto{\pgfqpoint{3.366860in}{3.876356in}}%
\pgfpathlineto{\pgfqpoint{3.288908in}{3.839759in}}%
\pgfpathlineto{\pgfqpoint{3.210955in}{3.876356in}}%
\pgfpathlineto{\pgfqpoint{3.288908in}{3.912952in}}%
\pgfpathclose%
\pgfusepath{fill}%
\end{pgfscope}%
\begin{pgfscope}%
\pgfpathrectangle{\pgfqpoint{0.765000in}{0.660000in}}{\pgfqpoint{4.620000in}{4.620000in}}%
\pgfusepath{clip}%
\pgfsetbuttcap%
\pgfsetroundjoin%
\definecolor{currentfill}{rgb}{1.000000,0.576471,0.309804}%
\pgfsetfillcolor{currentfill}%
\pgfsetfillopacity{0.050000}%
\pgfsetlinewidth{0.000000pt}%
\definecolor{currentstroke}{rgb}{1.000000,0.576471,0.309804}%
\pgfsetstrokecolor{currentstroke}%
\pgfsetdash{}{0pt}%
\pgfpathmoveto{\pgfqpoint{3.210955in}{3.803353in}}%
\pgfpathlineto{\pgfqpoint{3.210955in}{3.876356in}}%
\pgfpathlineto{\pgfqpoint{3.288908in}{3.839759in}}%
\pgfpathlineto{\pgfqpoint{3.288908in}{3.766756in}}%
\pgfpathlineto{\pgfqpoint{3.210955in}{3.803353in}}%
\pgfpathclose%
\pgfusepath{fill}%
\end{pgfscope}%
\begin{pgfscope}%
\pgfpathrectangle{\pgfqpoint{0.765000in}{0.660000in}}{\pgfqpoint{4.620000in}{4.620000in}}%
\pgfusepath{clip}%
\pgfsetbuttcap%
\pgfsetroundjoin%
\definecolor{currentfill}{rgb}{1.000000,0.576471,0.309804}%
\pgfsetfillcolor{currentfill}%
\pgfsetfillopacity{0.050000}%
\pgfsetlinewidth{0.000000pt}%
\definecolor{currentstroke}{rgb}{1.000000,0.576471,0.309804}%
\pgfsetstrokecolor{currentstroke}%
\pgfsetdash{}{0pt}%
\pgfpathmoveto{\pgfqpoint{3.366860in}{3.803353in}}%
\pgfpathlineto{\pgfqpoint{3.366860in}{3.876356in}}%
\pgfpathlineto{\pgfqpoint{3.288908in}{3.839759in}}%
\pgfpathlineto{\pgfqpoint{3.288908in}{3.766756in}}%
\pgfpathlineto{\pgfqpoint{3.366860in}{3.803353in}}%
\pgfpathclose%
\pgfusepath{fill}%
\end{pgfscope}%
\begin{pgfscope}%
\pgfpathrectangle{\pgfqpoint{0.765000in}{0.660000in}}{\pgfqpoint{4.620000in}{4.620000in}}%
\pgfusepath{clip}%
\pgfsetbuttcap%
\pgfsetroundjoin%
\definecolor{currentfill}{rgb}{0.176471,0.192157,0.258824}%
\pgfsetfillcolor{currentfill}%
\pgfsetfillopacity{0.050000}%
\pgfsetlinewidth{0.000000pt}%
\definecolor{currentstroke}{rgb}{0.176471,0.192157,0.258824}%
\pgfsetstrokecolor{currentstroke}%
\pgfsetdash{}{0pt}%
\pgfpathmoveto{\pgfqpoint{3.303963in}{2.945665in}}%
\pgfpathlineto{\pgfqpoint{3.303963in}{3.018669in}}%
\pgfpathlineto{\pgfqpoint{3.381916in}{2.982072in}}%
\pgfpathlineto{\pgfqpoint{3.381916in}{2.909069in}}%
\pgfpathlineto{\pgfqpoint{3.303963in}{2.945665in}}%
\pgfpathclose%
\pgfusepath{fill}%
\end{pgfscope}%
\begin{pgfscope}%
\pgfpathrectangle{\pgfqpoint{0.765000in}{0.660000in}}{\pgfqpoint{4.620000in}{4.620000in}}%
\pgfusepath{clip}%
\pgfsetbuttcap%
\pgfsetroundjoin%
\definecolor{currentfill}{rgb}{0.176471,0.192157,0.258824}%
\pgfsetfillcolor{currentfill}%
\pgfsetfillopacity{0.050000}%
\pgfsetlinewidth{0.000000pt}%
\definecolor{currentstroke}{rgb}{0.176471,0.192157,0.258824}%
\pgfsetstrokecolor{currentstroke}%
\pgfsetdash{}{0pt}%
\pgfpathmoveto{\pgfqpoint{3.303963in}{2.945665in}}%
\pgfpathlineto{\pgfqpoint{3.303963in}{3.018669in}}%
\pgfpathlineto{\pgfqpoint{3.226010in}{2.982072in}}%
\pgfpathlineto{\pgfqpoint{3.226010in}{2.909069in}}%
\pgfpathlineto{\pgfqpoint{3.303963in}{2.945665in}}%
\pgfpathclose%
\pgfusepath{fill}%
\end{pgfscope}%
\begin{pgfscope}%
\pgfpathrectangle{\pgfqpoint{0.765000in}{0.660000in}}{\pgfqpoint{4.620000in}{4.620000in}}%
\pgfusepath{clip}%
\pgfsetbuttcap%
\pgfsetroundjoin%
\definecolor{currentfill}{rgb}{0.176471,0.192157,0.258824}%
\pgfsetfillcolor{currentfill}%
\pgfsetfillopacity{0.050000}%
\pgfsetlinewidth{0.000000pt}%
\definecolor{currentstroke}{rgb}{0.176471,0.192157,0.258824}%
\pgfsetstrokecolor{currentstroke}%
\pgfsetdash{}{0pt}%
\pgfpathmoveto{\pgfqpoint{3.303963in}{2.945665in}}%
\pgfpathlineto{\pgfqpoint{3.381916in}{2.909069in}}%
\pgfpathlineto{\pgfqpoint{3.303963in}{2.872472in}}%
\pgfpathlineto{\pgfqpoint{3.226010in}{2.909069in}}%
\pgfpathlineto{\pgfqpoint{3.303963in}{2.945665in}}%
\pgfpathclose%
\pgfusepath{fill}%
\end{pgfscope}%
\begin{pgfscope}%
\pgfpathrectangle{\pgfqpoint{0.765000in}{0.660000in}}{\pgfqpoint{4.620000in}{4.620000in}}%
\pgfusepath{clip}%
\pgfsetbuttcap%
\pgfsetroundjoin%
\definecolor{currentfill}{rgb}{0.176471,0.192157,0.258824}%
\pgfsetfillcolor{currentfill}%
\pgfsetfillopacity{0.050000}%
\pgfsetlinewidth{0.000000pt}%
\definecolor{currentstroke}{rgb}{0.176471,0.192157,0.258824}%
\pgfsetstrokecolor{currentstroke}%
\pgfsetdash{}{0pt}%
\pgfpathmoveto{\pgfqpoint{3.303963in}{3.018669in}}%
\pgfpathlineto{\pgfqpoint{3.381916in}{2.982072in}}%
\pgfpathlineto{\pgfqpoint{3.303963in}{2.945476in}}%
\pgfpathlineto{\pgfqpoint{3.226010in}{2.982072in}}%
\pgfpathlineto{\pgfqpoint{3.303963in}{3.018669in}}%
\pgfpathclose%
\pgfusepath{fill}%
\end{pgfscope}%
\begin{pgfscope}%
\pgfpathrectangle{\pgfqpoint{0.765000in}{0.660000in}}{\pgfqpoint{4.620000in}{4.620000in}}%
\pgfusepath{clip}%
\pgfsetbuttcap%
\pgfsetroundjoin%
\definecolor{currentfill}{rgb}{0.176471,0.192157,0.258824}%
\pgfsetfillcolor{currentfill}%
\pgfsetfillopacity{0.050000}%
\pgfsetlinewidth{0.000000pt}%
\definecolor{currentstroke}{rgb}{0.176471,0.192157,0.258824}%
\pgfsetstrokecolor{currentstroke}%
\pgfsetdash{}{0pt}%
\pgfpathmoveto{\pgfqpoint{3.226010in}{2.909069in}}%
\pgfpathlineto{\pgfqpoint{3.226010in}{2.982072in}}%
\pgfpathlineto{\pgfqpoint{3.303963in}{2.945476in}}%
\pgfpathlineto{\pgfqpoint{3.303963in}{2.872472in}}%
\pgfpathlineto{\pgfqpoint{3.226010in}{2.909069in}}%
\pgfpathclose%
\pgfusepath{fill}%
\end{pgfscope}%
\begin{pgfscope}%
\pgfpathrectangle{\pgfqpoint{0.765000in}{0.660000in}}{\pgfqpoint{4.620000in}{4.620000in}}%
\pgfusepath{clip}%
\pgfsetbuttcap%
\pgfsetroundjoin%
\definecolor{currentfill}{rgb}{0.176471,0.192157,0.258824}%
\pgfsetfillcolor{currentfill}%
\pgfsetfillopacity{0.050000}%
\pgfsetlinewidth{0.000000pt}%
\definecolor{currentstroke}{rgb}{0.176471,0.192157,0.258824}%
\pgfsetstrokecolor{currentstroke}%
\pgfsetdash{}{0pt}%
\pgfpathmoveto{\pgfqpoint{3.381916in}{2.909069in}}%
\pgfpathlineto{\pgfqpoint{3.381916in}{2.982072in}}%
\pgfpathlineto{\pgfqpoint{3.303963in}{2.945476in}}%
\pgfpathlineto{\pgfqpoint{3.303963in}{2.872472in}}%
\pgfpathlineto{\pgfqpoint{3.381916in}{2.909069in}}%
\pgfpathclose%
\pgfusepath{fill}%
\end{pgfscope}%
\begin{pgfscope}%
\pgfpathrectangle{\pgfqpoint{0.765000in}{0.660000in}}{\pgfqpoint{4.620000in}{4.620000in}}%
\pgfusepath{clip}%
\pgfsetbuttcap%
\pgfsetroundjoin%
\definecolor{currentfill}{rgb}{1.000000,0.576471,0.309804}%
\pgfsetfillcolor{currentfill}%
\pgfsetfillopacity{0.050000}%
\pgfsetlinewidth{0.000000pt}%
\definecolor{currentstroke}{rgb}{1.000000,0.576471,0.309804}%
\pgfsetstrokecolor{currentstroke}%
\pgfsetdash{}{0pt}%
\pgfpathmoveto{\pgfqpoint{4.009308in}{3.134394in}}%
\pgfpathlineto{\pgfqpoint{4.009308in}{3.207397in}}%
\pgfpathlineto{\pgfqpoint{4.087261in}{3.170801in}}%
\pgfpathlineto{\pgfqpoint{4.087261in}{3.097798in}}%
\pgfpathlineto{\pgfqpoint{4.009308in}{3.134394in}}%
\pgfpathclose%
\pgfusepath{fill}%
\end{pgfscope}%
\begin{pgfscope}%
\pgfpathrectangle{\pgfqpoint{0.765000in}{0.660000in}}{\pgfqpoint{4.620000in}{4.620000in}}%
\pgfusepath{clip}%
\pgfsetbuttcap%
\pgfsetroundjoin%
\definecolor{currentfill}{rgb}{1.000000,0.576471,0.309804}%
\pgfsetfillcolor{currentfill}%
\pgfsetfillopacity{0.050000}%
\pgfsetlinewidth{0.000000pt}%
\definecolor{currentstroke}{rgb}{1.000000,0.576471,0.309804}%
\pgfsetstrokecolor{currentstroke}%
\pgfsetdash{}{0pt}%
\pgfpathmoveto{\pgfqpoint{4.009308in}{3.134394in}}%
\pgfpathlineto{\pgfqpoint{4.009308in}{3.207397in}}%
\pgfpathlineto{\pgfqpoint{3.931355in}{3.170801in}}%
\pgfpathlineto{\pgfqpoint{3.931355in}{3.097798in}}%
\pgfpathlineto{\pgfqpoint{4.009308in}{3.134394in}}%
\pgfpathclose%
\pgfusepath{fill}%
\end{pgfscope}%
\begin{pgfscope}%
\pgfpathrectangle{\pgfqpoint{0.765000in}{0.660000in}}{\pgfqpoint{4.620000in}{4.620000in}}%
\pgfusepath{clip}%
\pgfsetbuttcap%
\pgfsetroundjoin%
\definecolor{currentfill}{rgb}{1.000000,0.576471,0.309804}%
\pgfsetfillcolor{currentfill}%
\pgfsetfillopacity{0.050000}%
\pgfsetlinewidth{0.000000pt}%
\definecolor{currentstroke}{rgb}{1.000000,0.576471,0.309804}%
\pgfsetstrokecolor{currentstroke}%
\pgfsetdash{}{0pt}%
\pgfpathmoveto{\pgfqpoint{4.009308in}{3.134394in}}%
\pgfpathlineto{\pgfqpoint{4.087261in}{3.097798in}}%
\pgfpathlineto{\pgfqpoint{4.009308in}{3.061201in}}%
\pgfpathlineto{\pgfqpoint{3.931355in}{3.097798in}}%
\pgfpathlineto{\pgfqpoint{4.009308in}{3.134394in}}%
\pgfpathclose%
\pgfusepath{fill}%
\end{pgfscope}%
\begin{pgfscope}%
\pgfpathrectangle{\pgfqpoint{0.765000in}{0.660000in}}{\pgfqpoint{4.620000in}{4.620000in}}%
\pgfusepath{clip}%
\pgfsetbuttcap%
\pgfsetroundjoin%
\definecolor{currentfill}{rgb}{1.000000,0.576471,0.309804}%
\pgfsetfillcolor{currentfill}%
\pgfsetfillopacity{0.050000}%
\pgfsetlinewidth{0.000000pt}%
\definecolor{currentstroke}{rgb}{1.000000,0.576471,0.309804}%
\pgfsetstrokecolor{currentstroke}%
\pgfsetdash{}{0pt}%
\pgfpathmoveto{\pgfqpoint{4.009308in}{3.207397in}}%
\pgfpathlineto{\pgfqpoint{4.087261in}{3.170801in}}%
\pgfpathlineto{\pgfqpoint{4.009308in}{3.134204in}}%
\pgfpathlineto{\pgfqpoint{3.931355in}{3.170801in}}%
\pgfpathlineto{\pgfqpoint{4.009308in}{3.207397in}}%
\pgfpathclose%
\pgfusepath{fill}%
\end{pgfscope}%
\begin{pgfscope}%
\pgfpathrectangle{\pgfqpoint{0.765000in}{0.660000in}}{\pgfqpoint{4.620000in}{4.620000in}}%
\pgfusepath{clip}%
\pgfsetbuttcap%
\pgfsetroundjoin%
\definecolor{currentfill}{rgb}{1.000000,0.576471,0.309804}%
\pgfsetfillcolor{currentfill}%
\pgfsetfillopacity{0.050000}%
\pgfsetlinewidth{0.000000pt}%
\definecolor{currentstroke}{rgb}{1.000000,0.576471,0.309804}%
\pgfsetstrokecolor{currentstroke}%
\pgfsetdash{}{0pt}%
\pgfpathmoveto{\pgfqpoint{3.931355in}{3.097798in}}%
\pgfpathlineto{\pgfqpoint{3.931355in}{3.170801in}}%
\pgfpathlineto{\pgfqpoint{4.009308in}{3.134204in}}%
\pgfpathlineto{\pgfqpoint{4.009308in}{3.061201in}}%
\pgfpathlineto{\pgfqpoint{3.931355in}{3.097798in}}%
\pgfpathclose%
\pgfusepath{fill}%
\end{pgfscope}%
\begin{pgfscope}%
\pgfpathrectangle{\pgfqpoint{0.765000in}{0.660000in}}{\pgfqpoint{4.620000in}{4.620000in}}%
\pgfusepath{clip}%
\pgfsetbuttcap%
\pgfsetroundjoin%
\definecolor{currentfill}{rgb}{1.000000,0.576471,0.309804}%
\pgfsetfillcolor{currentfill}%
\pgfsetfillopacity{0.050000}%
\pgfsetlinewidth{0.000000pt}%
\definecolor{currentstroke}{rgb}{1.000000,0.576471,0.309804}%
\pgfsetstrokecolor{currentstroke}%
\pgfsetdash{}{0pt}%
\pgfpathmoveto{\pgfqpoint{4.087261in}{3.097798in}}%
\pgfpathlineto{\pgfqpoint{4.087261in}{3.170801in}}%
\pgfpathlineto{\pgfqpoint{4.009308in}{3.134204in}}%
\pgfpathlineto{\pgfqpoint{4.009308in}{3.061201in}}%
\pgfpathlineto{\pgfqpoint{4.087261in}{3.097798in}}%
\pgfpathclose%
\pgfusepath{fill}%
\end{pgfscope}%
\begin{pgfscope}%
\pgfpathrectangle{\pgfqpoint{0.765000in}{0.660000in}}{\pgfqpoint{4.620000in}{4.620000in}}%
\pgfusepath{clip}%
\pgfsetbuttcap%
\pgfsetroundjoin%
\definecolor{currentfill}{rgb}{0.176471,0.192157,0.258824}%
\pgfsetfillcolor{currentfill}%
\pgfsetfillopacity{0.050000}%
\pgfsetlinewidth{0.000000pt}%
\definecolor{currentstroke}{rgb}{0.176471,0.192157,0.258824}%
\pgfsetstrokecolor{currentstroke}%
\pgfsetdash{}{0pt}%
\pgfpathmoveto{\pgfqpoint{3.281169in}{2.954857in}}%
\pgfpathlineto{\pgfqpoint{3.281169in}{3.027860in}}%
\pgfpathlineto{\pgfqpoint{3.359122in}{2.991264in}}%
\pgfpathlineto{\pgfqpoint{3.359122in}{2.918260in}}%
\pgfpathlineto{\pgfqpoint{3.281169in}{2.954857in}}%
\pgfpathclose%
\pgfusepath{fill}%
\end{pgfscope}%
\begin{pgfscope}%
\pgfpathrectangle{\pgfqpoint{0.765000in}{0.660000in}}{\pgfqpoint{4.620000in}{4.620000in}}%
\pgfusepath{clip}%
\pgfsetbuttcap%
\pgfsetroundjoin%
\definecolor{currentfill}{rgb}{0.176471,0.192157,0.258824}%
\pgfsetfillcolor{currentfill}%
\pgfsetfillopacity{0.050000}%
\pgfsetlinewidth{0.000000pt}%
\definecolor{currentstroke}{rgb}{0.176471,0.192157,0.258824}%
\pgfsetstrokecolor{currentstroke}%
\pgfsetdash{}{0pt}%
\pgfpathmoveto{\pgfqpoint{3.281169in}{2.954857in}}%
\pgfpathlineto{\pgfqpoint{3.281169in}{3.027860in}}%
\pgfpathlineto{\pgfqpoint{3.203216in}{2.991264in}}%
\pgfpathlineto{\pgfqpoint{3.203216in}{2.918260in}}%
\pgfpathlineto{\pgfqpoint{3.281169in}{2.954857in}}%
\pgfpathclose%
\pgfusepath{fill}%
\end{pgfscope}%
\begin{pgfscope}%
\pgfpathrectangle{\pgfqpoint{0.765000in}{0.660000in}}{\pgfqpoint{4.620000in}{4.620000in}}%
\pgfusepath{clip}%
\pgfsetbuttcap%
\pgfsetroundjoin%
\definecolor{currentfill}{rgb}{0.176471,0.192157,0.258824}%
\pgfsetfillcolor{currentfill}%
\pgfsetfillopacity{0.050000}%
\pgfsetlinewidth{0.000000pt}%
\definecolor{currentstroke}{rgb}{0.176471,0.192157,0.258824}%
\pgfsetstrokecolor{currentstroke}%
\pgfsetdash{}{0pt}%
\pgfpathmoveto{\pgfqpoint{3.281169in}{2.954857in}}%
\pgfpathlineto{\pgfqpoint{3.359122in}{2.918260in}}%
\pgfpathlineto{\pgfqpoint{3.281169in}{2.881664in}}%
\pgfpathlineto{\pgfqpoint{3.203216in}{2.918260in}}%
\pgfpathlineto{\pgfqpoint{3.281169in}{2.954857in}}%
\pgfpathclose%
\pgfusepath{fill}%
\end{pgfscope}%
\begin{pgfscope}%
\pgfpathrectangle{\pgfqpoint{0.765000in}{0.660000in}}{\pgfqpoint{4.620000in}{4.620000in}}%
\pgfusepath{clip}%
\pgfsetbuttcap%
\pgfsetroundjoin%
\definecolor{currentfill}{rgb}{0.176471,0.192157,0.258824}%
\pgfsetfillcolor{currentfill}%
\pgfsetfillopacity{0.050000}%
\pgfsetlinewidth{0.000000pt}%
\definecolor{currentstroke}{rgb}{0.176471,0.192157,0.258824}%
\pgfsetstrokecolor{currentstroke}%
\pgfsetdash{}{0pt}%
\pgfpathmoveto{\pgfqpoint{3.281169in}{3.027860in}}%
\pgfpathlineto{\pgfqpoint{3.359122in}{2.991264in}}%
\pgfpathlineto{\pgfqpoint{3.281169in}{2.954667in}}%
\pgfpathlineto{\pgfqpoint{3.203216in}{2.991264in}}%
\pgfpathlineto{\pgfqpoint{3.281169in}{3.027860in}}%
\pgfpathclose%
\pgfusepath{fill}%
\end{pgfscope}%
\begin{pgfscope}%
\pgfpathrectangle{\pgfqpoint{0.765000in}{0.660000in}}{\pgfqpoint{4.620000in}{4.620000in}}%
\pgfusepath{clip}%
\pgfsetbuttcap%
\pgfsetroundjoin%
\definecolor{currentfill}{rgb}{0.176471,0.192157,0.258824}%
\pgfsetfillcolor{currentfill}%
\pgfsetfillopacity{0.050000}%
\pgfsetlinewidth{0.000000pt}%
\definecolor{currentstroke}{rgb}{0.176471,0.192157,0.258824}%
\pgfsetstrokecolor{currentstroke}%
\pgfsetdash{}{0pt}%
\pgfpathmoveto{\pgfqpoint{3.203216in}{2.918260in}}%
\pgfpathlineto{\pgfqpoint{3.203216in}{2.991264in}}%
\pgfpathlineto{\pgfqpoint{3.281169in}{2.954667in}}%
\pgfpathlineto{\pgfqpoint{3.281169in}{2.881664in}}%
\pgfpathlineto{\pgfqpoint{3.203216in}{2.918260in}}%
\pgfpathclose%
\pgfusepath{fill}%
\end{pgfscope}%
\begin{pgfscope}%
\pgfpathrectangle{\pgfqpoint{0.765000in}{0.660000in}}{\pgfqpoint{4.620000in}{4.620000in}}%
\pgfusepath{clip}%
\pgfsetbuttcap%
\pgfsetroundjoin%
\definecolor{currentfill}{rgb}{0.176471,0.192157,0.258824}%
\pgfsetfillcolor{currentfill}%
\pgfsetfillopacity{0.050000}%
\pgfsetlinewidth{0.000000pt}%
\definecolor{currentstroke}{rgb}{0.176471,0.192157,0.258824}%
\pgfsetstrokecolor{currentstroke}%
\pgfsetdash{}{0pt}%
\pgfpathmoveto{\pgfqpoint{3.359122in}{2.918260in}}%
\pgfpathlineto{\pgfqpoint{3.359122in}{2.991264in}}%
\pgfpathlineto{\pgfqpoint{3.281169in}{2.954667in}}%
\pgfpathlineto{\pgfqpoint{3.281169in}{2.881664in}}%
\pgfpathlineto{\pgfqpoint{3.359122in}{2.918260in}}%
\pgfpathclose%
\pgfusepath{fill}%
\end{pgfscope}%
\begin{pgfscope}%
\pgfpathrectangle{\pgfqpoint{0.765000in}{0.660000in}}{\pgfqpoint{4.620000in}{4.620000in}}%
\pgfusepath{clip}%
\pgfsetbuttcap%
\pgfsetroundjoin%
\definecolor{currentfill}{rgb}{1.000000,0.576471,0.309804}%
\pgfsetfillcolor{currentfill}%
\pgfsetfillopacity{0.050000}%
\pgfsetlinewidth{0.000000pt}%
\definecolor{currentstroke}{rgb}{1.000000,0.576471,0.309804}%
\pgfsetstrokecolor{currentstroke}%
\pgfsetdash{}{0pt}%
\pgfpathmoveto{\pgfqpoint{2.741073in}{1.922752in}}%
\pgfpathlineto{\pgfqpoint{2.741073in}{1.995755in}}%
\pgfpathlineto{\pgfqpoint{2.819026in}{1.959159in}}%
\pgfpathlineto{\pgfqpoint{2.819026in}{1.886156in}}%
\pgfpathlineto{\pgfqpoint{2.741073in}{1.922752in}}%
\pgfpathclose%
\pgfusepath{fill}%
\end{pgfscope}%
\begin{pgfscope}%
\pgfpathrectangle{\pgfqpoint{0.765000in}{0.660000in}}{\pgfqpoint{4.620000in}{4.620000in}}%
\pgfusepath{clip}%
\pgfsetbuttcap%
\pgfsetroundjoin%
\definecolor{currentfill}{rgb}{1.000000,0.576471,0.309804}%
\pgfsetfillcolor{currentfill}%
\pgfsetfillopacity{0.050000}%
\pgfsetlinewidth{0.000000pt}%
\definecolor{currentstroke}{rgb}{1.000000,0.576471,0.309804}%
\pgfsetstrokecolor{currentstroke}%
\pgfsetdash{}{0pt}%
\pgfpathmoveto{\pgfqpoint{2.741073in}{1.922752in}}%
\pgfpathlineto{\pgfqpoint{2.741073in}{1.995755in}}%
\pgfpathlineto{\pgfqpoint{2.663121in}{1.959159in}}%
\pgfpathlineto{\pgfqpoint{2.663121in}{1.886156in}}%
\pgfpathlineto{\pgfqpoint{2.741073in}{1.922752in}}%
\pgfpathclose%
\pgfusepath{fill}%
\end{pgfscope}%
\begin{pgfscope}%
\pgfpathrectangle{\pgfqpoint{0.765000in}{0.660000in}}{\pgfqpoint{4.620000in}{4.620000in}}%
\pgfusepath{clip}%
\pgfsetbuttcap%
\pgfsetroundjoin%
\definecolor{currentfill}{rgb}{1.000000,0.576471,0.309804}%
\pgfsetfillcolor{currentfill}%
\pgfsetfillopacity{0.050000}%
\pgfsetlinewidth{0.000000pt}%
\definecolor{currentstroke}{rgb}{1.000000,0.576471,0.309804}%
\pgfsetstrokecolor{currentstroke}%
\pgfsetdash{}{0pt}%
\pgfpathmoveto{\pgfqpoint{2.741073in}{1.922752in}}%
\pgfpathlineto{\pgfqpoint{2.819026in}{1.886156in}}%
\pgfpathlineto{\pgfqpoint{2.741073in}{1.849559in}}%
\pgfpathlineto{\pgfqpoint{2.663121in}{1.886156in}}%
\pgfpathlineto{\pgfqpoint{2.741073in}{1.922752in}}%
\pgfpathclose%
\pgfusepath{fill}%
\end{pgfscope}%
\begin{pgfscope}%
\pgfpathrectangle{\pgfqpoint{0.765000in}{0.660000in}}{\pgfqpoint{4.620000in}{4.620000in}}%
\pgfusepath{clip}%
\pgfsetbuttcap%
\pgfsetroundjoin%
\definecolor{currentfill}{rgb}{1.000000,0.576471,0.309804}%
\pgfsetfillcolor{currentfill}%
\pgfsetfillopacity{0.050000}%
\pgfsetlinewidth{0.000000pt}%
\definecolor{currentstroke}{rgb}{1.000000,0.576471,0.309804}%
\pgfsetstrokecolor{currentstroke}%
\pgfsetdash{}{0pt}%
\pgfpathmoveto{\pgfqpoint{2.741073in}{1.995755in}}%
\pgfpathlineto{\pgfqpoint{2.819026in}{1.959159in}}%
\pgfpathlineto{\pgfqpoint{2.741073in}{1.922562in}}%
\pgfpathlineto{\pgfqpoint{2.663121in}{1.959159in}}%
\pgfpathlineto{\pgfqpoint{2.741073in}{1.995755in}}%
\pgfpathclose%
\pgfusepath{fill}%
\end{pgfscope}%
\begin{pgfscope}%
\pgfpathrectangle{\pgfqpoint{0.765000in}{0.660000in}}{\pgfqpoint{4.620000in}{4.620000in}}%
\pgfusepath{clip}%
\pgfsetbuttcap%
\pgfsetroundjoin%
\definecolor{currentfill}{rgb}{1.000000,0.576471,0.309804}%
\pgfsetfillcolor{currentfill}%
\pgfsetfillopacity{0.050000}%
\pgfsetlinewidth{0.000000pt}%
\definecolor{currentstroke}{rgb}{1.000000,0.576471,0.309804}%
\pgfsetstrokecolor{currentstroke}%
\pgfsetdash{}{0pt}%
\pgfpathmoveto{\pgfqpoint{2.663121in}{1.886156in}}%
\pgfpathlineto{\pgfqpoint{2.663121in}{1.959159in}}%
\pgfpathlineto{\pgfqpoint{2.741073in}{1.922562in}}%
\pgfpathlineto{\pgfqpoint{2.741073in}{1.849559in}}%
\pgfpathlineto{\pgfqpoint{2.663121in}{1.886156in}}%
\pgfpathclose%
\pgfusepath{fill}%
\end{pgfscope}%
\begin{pgfscope}%
\pgfpathrectangle{\pgfqpoint{0.765000in}{0.660000in}}{\pgfqpoint{4.620000in}{4.620000in}}%
\pgfusepath{clip}%
\pgfsetbuttcap%
\pgfsetroundjoin%
\definecolor{currentfill}{rgb}{1.000000,0.576471,0.309804}%
\pgfsetfillcolor{currentfill}%
\pgfsetfillopacity{0.050000}%
\pgfsetlinewidth{0.000000pt}%
\definecolor{currentstroke}{rgb}{1.000000,0.576471,0.309804}%
\pgfsetstrokecolor{currentstroke}%
\pgfsetdash{}{0pt}%
\pgfpathmoveto{\pgfqpoint{2.819026in}{1.886156in}}%
\pgfpathlineto{\pgfqpoint{2.819026in}{1.959159in}}%
\pgfpathlineto{\pgfqpoint{2.741073in}{1.922562in}}%
\pgfpathlineto{\pgfqpoint{2.741073in}{1.849559in}}%
\pgfpathlineto{\pgfqpoint{2.819026in}{1.886156in}}%
\pgfpathclose%
\pgfusepath{fill}%
\end{pgfscope}%
\begin{pgfscope}%
\pgfpathrectangle{\pgfqpoint{0.765000in}{0.660000in}}{\pgfqpoint{4.620000in}{4.620000in}}%
\pgfusepath{clip}%
\pgfsetbuttcap%
\pgfsetroundjoin%
\definecolor{currentfill}{rgb}{0.176471,0.192157,0.258824}%
\pgfsetfillcolor{currentfill}%
\pgfsetfillopacity{0.050000}%
\pgfsetlinewidth{0.000000pt}%
\definecolor{currentstroke}{rgb}{0.176471,0.192157,0.258824}%
\pgfsetstrokecolor{currentstroke}%
\pgfsetdash{}{0pt}%
\pgfpathmoveto{\pgfqpoint{3.143353in}{2.966514in}}%
\pgfpathlineto{\pgfqpoint{3.143353in}{3.039518in}}%
\pgfpathlineto{\pgfqpoint{3.221306in}{3.002921in}}%
\pgfpathlineto{\pgfqpoint{3.221306in}{2.929918in}}%
\pgfpathlineto{\pgfqpoint{3.143353in}{2.966514in}}%
\pgfpathclose%
\pgfusepath{fill}%
\end{pgfscope}%
\begin{pgfscope}%
\pgfpathrectangle{\pgfqpoint{0.765000in}{0.660000in}}{\pgfqpoint{4.620000in}{4.620000in}}%
\pgfusepath{clip}%
\pgfsetbuttcap%
\pgfsetroundjoin%
\definecolor{currentfill}{rgb}{0.176471,0.192157,0.258824}%
\pgfsetfillcolor{currentfill}%
\pgfsetfillopacity{0.050000}%
\pgfsetlinewidth{0.000000pt}%
\definecolor{currentstroke}{rgb}{0.176471,0.192157,0.258824}%
\pgfsetstrokecolor{currentstroke}%
\pgfsetdash{}{0pt}%
\pgfpathmoveto{\pgfqpoint{3.143353in}{2.966514in}}%
\pgfpathlineto{\pgfqpoint{3.143353in}{3.039518in}}%
\pgfpathlineto{\pgfqpoint{3.065400in}{3.002921in}}%
\pgfpathlineto{\pgfqpoint{3.065400in}{2.929918in}}%
\pgfpathlineto{\pgfqpoint{3.143353in}{2.966514in}}%
\pgfpathclose%
\pgfusepath{fill}%
\end{pgfscope}%
\begin{pgfscope}%
\pgfpathrectangle{\pgfqpoint{0.765000in}{0.660000in}}{\pgfqpoint{4.620000in}{4.620000in}}%
\pgfusepath{clip}%
\pgfsetbuttcap%
\pgfsetroundjoin%
\definecolor{currentfill}{rgb}{0.176471,0.192157,0.258824}%
\pgfsetfillcolor{currentfill}%
\pgfsetfillopacity{0.050000}%
\pgfsetlinewidth{0.000000pt}%
\definecolor{currentstroke}{rgb}{0.176471,0.192157,0.258824}%
\pgfsetstrokecolor{currentstroke}%
\pgfsetdash{}{0pt}%
\pgfpathmoveto{\pgfqpoint{3.143353in}{2.966514in}}%
\pgfpathlineto{\pgfqpoint{3.221306in}{2.929918in}}%
\pgfpathlineto{\pgfqpoint{3.143353in}{2.893321in}}%
\pgfpathlineto{\pgfqpoint{3.065400in}{2.929918in}}%
\pgfpathlineto{\pgfqpoint{3.143353in}{2.966514in}}%
\pgfpathclose%
\pgfusepath{fill}%
\end{pgfscope}%
\begin{pgfscope}%
\pgfpathrectangle{\pgfqpoint{0.765000in}{0.660000in}}{\pgfqpoint{4.620000in}{4.620000in}}%
\pgfusepath{clip}%
\pgfsetbuttcap%
\pgfsetroundjoin%
\definecolor{currentfill}{rgb}{0.176471,0.192157,0.258824}%
\pgfsetfillcolor{currentfill}%
\pgfsetfillopacity{0.050000}%
\pgfsetlinewidth{0.000000pt}%
\definecolor{currentstroke}{rgb}{0.176471,0.192157,0.258824}%
\pgfsetstrokecolor{currentstroke}%
\pgfsetdash{}{0pt}%
\pgfpathmoveto{\pgfqpoint{3.143353in}{3.039518in}}%
\pgfpathlineto{\pgfqpoint{3.221306in}{3.002921in}}%
\pgfpathlineto{\pgfqpoint{3.143353in}{2.966324in}}%
\pgfpathlineto{\pgfqpoint{3.065400in}{3.002921in}}%
\pgfpathlineto{\pgfqpoint{3.143353in}{3.039518in}}%
\pgfpathclose%
\pgfusepath{fill}%
\end{pgfscope}%
\begin{pgfscope}%
\pgfpathrectangle{\pgfqpoint{0.765000in}{0.660000in}}{\pgfqpoint{4.620000in}{4.620000in}}%
\pgfusepath{clip}%
\pgfsetbuttcap%
\pgfsetroundjoin%
\definecolor{currentfill}{rgb}{0.176471,0.192157,0.258824}%
\pgfsetfillcolor{currentfill}%
\pgfsetfillopacity{0.050000}%
\pgfsetlinewidth{0.000000pt}%
\definecolor{currentstroke}{rgb}{0.176471,0.192157,0.258824}%
\pgfsetstrokecolor{currentstroke}%
\pgfsetdash{}{0pt}%
\pgfpathmoveto{\pgfqpoint{3.065400in}{2.929918in}}%
\pgfpathlineto{\pgfqpoint{3.065400in}{3.002921in}}%
\pgfpathlineto{\pgfqpoint{3.143353in}{2.966324in}}%
\pgfpathlineto{\pgfqpoint{3.143353in}{2.893321in}}%
\pgfpathlineto{\pgfqpoint{3.065400in}{2.929918in}}%
\pgfpathclose%
\pgfusepath{fill}%
\end{pgfscope}%
\begin{pgfscope}%
\pgfpathrectangle{\pgfqpoint{0.765000in}{0.660000in}}{\pgfqpoint{4.620000in}{4.620000in}}%
\pgfusepath{clip}%
\pgfsetbuttcap%
\pgfsetroundjoin%
\definecolor{currentfill}{rgb}{0.176471,0.192157,0.258824}%
\pgfsetfillcolor{currentfill}%
\pgfsetfillopacity{0.050000}%
\pgfsetlinewidth{0.000000pt}%
\definecolor{currentstroke}{rgb}{0.176471,0.192157,0.258824}%
\pgfsetstrokecolor{currentstroke}%
\pgfsetdash{}{0pt}%
\pgfpathmoveto{\pgfqpoint{3.221306in}{2.929918in}}%
\pgfpathlineto{\pgfqpoint{3.221306in}{3.002921in}}%
\pgfpathlineto{\pgfqpoint{3.143353in}{2.966324in}}%
\pgfpathlineto{\pgfqpoint{3.143353in}{2.893321in}}%
\pgfpathlineto{\pgfqpoint{3.221306in}{2.929918in}}%
\pgfpathclose%
\pgfusepath{fill}%
\end{pgfscope}%
\begin{pgfscope}%
\pgfpathrectangle{\pgfqpoint{0.765000in}{0.660000in}}{\pgfqpoint{4.620000in}{4.620000in}}%
\pgfusepath{clip}%
\pgfsetbuttcap%
\pgfsetroundjoin%
\definecolor{currentfill}{rgb}{0.176471,0.192157,0.258824}%
\pgfsetfillcolor{currentfill}%
\pgfsetfillopacity{0.050000}%
\pgfsetlinewidth{0.000000pt}%
\definecolor{currentstroke}{rgb}{0.176471,0.192157,0.258824}%
\pgfsetstrokecolor{currentstroke}%
\pgfsetdash{}{0pt}%
\pgfpathmoveto{\pgfqpoint{3.208686in}{3.083550in}}%
\pgfpathlineto{\pgfqpoint{3.208686in}{3.156554in}}%
\pgfpathlineto{\pgfqpoint{3.286639in}{3.119957in}}%
\pgfpathlineto{\pgfqpoint{3.286639in}{3.046954in}}%
\pgfpathlineto{\pgfqpoint{3.208686in}{3.083550in}}%
\pgfpathclose%
\pgfusepath{fill}%
\end{pgfscope}%
\begin{pgfscope}%
\pgfpathrectangle{\pgfqpoint{0.765000in}{0.660000in}}{\pgfqpoint{4.620000in}{4.620000in}}%
\pgfusepath{clip}%
\pgfsetbuttcap%
\pgfsetroundjoin%
\definecolor{currentfill}{rgb}{0.176471,0.192157,0.258824}%
\pgfsetfillcolor{currentfill}%
\pgfsetfillopacity{0.050000}%
\pgfsetlinewidth{0.000000pt}%
\definecolor{currentstroke}{rgb}{0.176471,0.192157,0.258824}%
\pgfsetstrokecolor{currentstroke}%
\pgfsetdash{}{0pt}%
\pgfpathmoveto{\pgfqpoint{3.208686in}{3.083550in}}%
\pgfpathlineto{\pgfqpoint{3.208686in}{3.156554in}}%
\pgfpathlineto{\pgfqpoint{3.130734in}{3.119957in}}%
\pgfpathlineto{\pgfqpoint{3.130734in}{3.046954in}}%
\pgfpathlineto{\pgfqpoint{3.208686in}{3.083550in}}%
\pgfpathclose%
\pgfusepath{fill}%
\end{pgfscope}%
\begin{pgfscope}%
\pgfpathrectangle{\pgfqpoint{0.765000in}{0.660000in}}{\pgfqpoint{4.620000in}{4.620000in}}%
\pgfusepath{clip}%
\pgfsetbuttcap%
\pgfsetroundjoin%
\definecolor{currentfill}{rgb}{0.176471,0.192157,0.258824}%
\pgfsetfillcolor{currentfill}%
\pgfsetfillopacity{0.050000}%
\pgfsetlinewidth{0.000000pt}%
\definecolor{currentstroke}{rgb}{0.176471,0.192157,0.258824}%
\pgfsetstrokecolor{currentstroke}%
\pgfsetdash{}{0pt}%
\pgfpathmoveto{\pgfqpoint{3.208686in}{3.083550in}}%
\pgfpathlineto{\pgfqpoint{3.286639in}{3.046954in}}%
\pgfpathlineto{\pgfqpoint{3.208686in}{3.010357in}}%
\pgfpathlineto{\pgfqpoint{3.130734in}{3.046954in}}%
\pgfpathlineto{\pgfqpoint{3.208686in}{3.083550in}}%
\pgfpathclose%
\pgfusepath{fill}%
\end{pgfscope}%
\begin{pgfscope}%
\pgfpathrectangle{\pgfqpoint{0.765000in}{0.660000in}}{\pgfqpoint{4.620000in}{4.620000in}}%
\pgfusepath{clip}%
\pgfsetbuttcap%
\pgfsetroundjoin%
\definecolor{currentfill}{rgb}{0.176471,0.192157,0.258824}%
\pgfsetfillcolor{currentfill}%
\pgfsetfillopacity{0.050000}%
\pgfsetlinewidth{0.000000pt}%
\definecolor{currentstroke}{rgb}{0.176471,0.192157,0.258824}%
\pgfsetstrokecolor{currentstroke}%
\pgfsetdash{}{0pt}%
\pgfpathmoveto{\pgfqpoint{3.208686in}{3.156554in}}%
\pgfpathlineto{\pgfqpoint{3.286639in}{3.119957in}}%
\pgfpathlineto{\pgfqpoint{3.208686in}{3.083361in}}%
\pgfpathlineto{\pgfqpoint{3.130734in}{3.119957in}}%
\pgfpathlineto{\pgfqpoint{3.208686in}{3.156554in}}%
\pgfpathclose%
\pgfusepath{fill}%
\end{pgfscope}%
\begin{pgfscope}%
\pgfpathrectangle{\pgfqpoint{0.765000in}{0.660000in}}{\pgfqpoint{4.620000in}{4.620000in}}%
\pgfusepath{clip}%
\pgfsetbuttcap%
\pgfsetroundjoin%
\definecolor{currentfill}{rgb}{0.176471,0.192157,0.258824}%
\pgfsetfillcolor{currentfill}%
\pgfsetfillopacity{0.050000}%
\pgfsetlinewidth{0.000000pt}%
\definecolor{currentstroke}{rgb}{0.176471,0.192157,0.258824}%
\pgfsetstrokecolor{currentstroke}%
\pgfsetdash{}{0pt}%
\pgfpathmoveto{\pgfqpoint{3.130734in}{3.046954in}}%
\pgfpathlineto{\pgfqpoint{3.130734in}{3.119957in}}%
\pgfpathlineto{\pgfqpoint{3.208686in}{3.083361in}}%
\pgfpathlineto{\pgfqpoint{3.208686in}{3.010357in}}%
\pgfpathlineto{\pgfqpoint{3.130734in}{3.046954in}}%
\pgfpathclose%
\pgfusepath{fill}%
\end{pgfscope}%
\begin{pgfscope}%
\pgfpathrectangle{\pgfqpoint{0.765000in}{0.660000in}}{\pgfqpoint{4.620000in}{4.620000in}}%
\pgfusepath{clip}%
\pgfsetbuttcap%
\pgfsetroundjoin%
\definecolor{currentfill}{rgb}{0.176471,0.192157,0.258824}%
\pgfsetfillcolor{currentfill}%
\pgfsetfillopacity{0.050000}%
\pgfsetlinewidth{0.000000pt}%
\definecolor{currentstroke}{rgb}{0.176471,0.192157,0.258824}%
\pgfsetstrokecolor{currentstroke}%
\pgfsetdash{}{0pt}%
\pgfpathmoveto{\pgfqpoint{3.286639in}{3.046954in}}%
\pgfpathlineto{\pgfqpoint{3.286639in}{3.119957in}}%
\pgfpathlineto{\pgfqpoint{3.208686in}{3.083361in}}%
\pgfpathlineto{\pgfqpoint{3.208686in}{3.010357in}}%
\pgfpathlineto{\pgfqpoint{3.286639in}{3.046954in}}%
\pgfpathclose%
\pgfusepath{fill}%
\end{pgfscope}%
\begin{pgfscope}%
\pgfpathrectangle{\pgfqpoint{0.765000in}{0.660000in}}{\pgfqpoint{4.620000in}{4.620000in}}%
\pgfusepath{clip}%
\pgfsetbuttcap%
\pgfsetroundjoin%
\definecolor{currentfill}{rgb}{1.000000,0.576471,0.309804}%
\pgfsetfillcolor{currentfill}%
\pgfsetfillopacity{0.050000}%
\pgfsetlinewidth{0.000000pt}%
\definecolor{currentstroke}{rgb}{1.000000,0.576471,0.309804}%
\pgfsetstrokecolor{currentstroke}%
\pgfsetdash{}{0pt}%
\pgfpathmoveto{\pgfqpoint{2.369208in}{3.565390in}}%
\pgfpathlineto{\pgfqpoint{2.369208in}{3.638394in}}%
\pgfpathlineto{\pgfqpoint{2.447161in}{3.601797in}}%
\pgfpathlineto{\pgfqpoint{2.447161in}{3.528794in}}%
\pgfpathlineto{\pgfqpoint{2.369208in}{3.565390in}}%
\pgfpathclose%
\pgfusepath{fill}%
\end{pgfscope}%
\begin{pgfscope}%
\pgfpathrectangle{\pgfqpoint{0.765000in}{0.660000in}}{\pgfqpoint{4.620000in}{4.620000in}}%
\pgfusepath{clip}%
\pgfsetbuttcap%
\pgfsetroundjoin%
\definecolor{currentfill}{rgb}{1.000000,0.576471,0.309804}%
\pgfsetfillcolor{currentfill}%
\pgfsetfillopacity{0.050000}%
\pgfsetlinewidth{0.000000pt}%
\definecolor{currentstroke}{rgb}{1.000000,0.576471,0.309804}%
\pgfsetstrokecolor{currentstroke}%
\pgfsetdash{}{0pt}%
\pgfpathmoveto{\pgfqpoint{2.369208in}{3.565390in}}%
\pgfpathlineto{\pgfqpoint{2.369208in}{3.638394in}}%
\pgfpathlineto{\pgfqpoint{2.291255in}{3.601797in}}%
\pgfpathlineto{\pgfqpoint{2.291255in}{3.528794in}}%
\pgfpathlineto{\pgfqpoint{2.369208in}{3.565390in}}%
\pgfpathclose%
\pgfusepath{fill}%
\end{pgfscope}%
\begin{pgfscope}%
\pgfpathrectangle{\pgfqpoint{0.765000in}{0.660000in}}{\pgfqpoint{4.620000in}{4.620000in}}%
\pgfusepath{clip}%
\pgfsetbuttcap%
\pgfsetroundjoin%
\definecolor{currentfill}{rgb}{1.000000,0.576471,0.309804}%
\pgfsetfillcolor{currentfill}%
\pgfsetfillopacity{0.050000}%
\pgfsetlinewidth{0.000000pt}%
\definecolor{currentstroke}{rgb}{1.000000,0.576471,0.309804}%
\pgfsetstrokecolor{currentstroke}%
\pgfsetdash{}{0pt}%
\pgfpathmoveto{\pgfqpoint{2.369208in}{3.565390in}}%
\pgfpathlineto{\pgfqpoint{2.447161in}{3.528794in}}%
\pgfpathlineto{\pgfqpoint{2.369208in}{3.492197in}}%
\pgfpathlineto{\pgfqpoint{2.291255in}{3.528794in}}%
\pgfpathlineto{\pgfqpoint{2.369208in}{3.565390in}}%
\pgfpathclose%
\pgfusepath{fill}%
\end{pgfscope}%
\begin{pgfscope}%
\pgfpathrectangle{\pgfqpoint{0.765000in}{0.660000in}}{\pgfqpoint{4.620000in}{4.620000in}}%
\pgfusepath{clip}%
\pgfsetbuttcap%
\pgfsetroundjoin%
\definecolor{currentfill}{rgb}{1.000000,0.576471,0.309804}%
\pgfsetfillcolor{currentfill}%
\pgfsetfillopacity{0.050000}%
\pgfsetlinewidth{0.000000pt}%
\definecolor{currentstroke}{rgb}{1.000000,0.576471,0.309804}%
\pgfsetstrokecolor{currentstroke}%
\pgfsetdash{}{0pt}%
\pgfpathmoveto{\pgfqpoint{2.369208in}{3.638394in}}%
\pgfpathlineto{\pgfqpoint{2.447161in}{3.601797in}}%
\pgfpathlineto{\pgfqpoint{2.369208in}{3.565201in}}%
\pgfpathlineto{\pgfqpoint{2.291255in}{3.601797in}}%
\pgfpathlineto{\pgfqpoint{2.369208in}{3.638394in}}%
\pgfpathclose%
\pgfusepath{fill}%
\end{pgfscope}%
\begin{pgfscope}%
\pgfpathrectangle{\pgfqpoint{0.765000in}{0.660000in}}{\pgfqpoint{4.620000in}{4.620000in}}%
\pgfusepath{clip}%
\pgfsetbuttcap%
\pgfsetroundjoin%
\definecolor{currentfill}{rgb}{1.000000,0.576471,0.309804}%
\pgfsetfillcolor{currentfill}%
\pgfsetfillopacity{0.050000}%
\pgfsetlinewidth{0.000000pt}%
\definecolor{currentstroke}{rgb}{1.000000,0.576471,0.309804}%
\pgfsetstrokecolor{currentstroke}%
\pgfsetdash{}{0pt}%
\pgfpathmoveto{\pgfqpoint{2.291255in}{3.528794in}}%
\pgfpathlineto{\pgfqpoint{2.291255in}{3.601797in}}%
\pgfpathlineto{\pgfqpoint{2.369208in}{3.565201in}}%
\pgfpathlineto{\pgfqpoint{2.369208in}{3.492197in}}%
\pgfpathlineto{\pgfqpoint{2.291255in}{3.528794in}}%
\pgfpathclose%
\pgfusepath{fill}%
\end{pgfscope}%
\begin{pgfscope}%
\pgfpathrectangle{\pgfqpoint{0.765000in}{0.660000in}}{\pgfqpoint{4.620000in}{4.620000in}}%
\pgfusepath{clip}%
\pgfsetbuttcap%
\pgfsetroundjoin%
\definecolor{currentfill}{rgb}{1.000000,0.576471,0.309804}%
\pgfsetfillcolor{currentfill}%
\pgfsetfillopacity{0.050000}%
\pgfsetlinewidth{0.000000pt}%
\definecolor{currentstroke}{rgb}{1.000000,0.576471,0.309804}%
\pgfsetstrokecolor{currentstroke}%
\pgfsetdash{}{0pt}%
\pgfpathmoveto{\pgfqpoint{2.447161in}{3.528794in}}%
\pgfpathlineto{\pgfqpoint{2.447161in}{3.601797in}}%
\pgfpathlineto{\pgfqpoint{2.369208in}{3.565201in}}%
\pgfpathlineto{\pgfqpoint{2.369208in}{3.492197in}}%
\pgfpathlineto{\pgfqpoint{2.447161in}{3.528794in}}%
\pgfpathclose%
\pgfusepath{fill}%
\end{pgfscope}%
\begin{pgfscope}%
\pgfpathrectangle{\pgfqpoint{0.765000in}{0.660000in}}{\pgfqpoint{4.620000in}{4.620000in}}%
\pgfusepath{clip}%
\pgfsetbuttcap%
\pgfsetroundjoin%
\definecolor{currentfill}{rgb}{1.000000,0.576471,0.309804}%
\pgfsetfillcolor{currentfill}%
\pgfsetfillopacity{0.050000}%
\pgfsetlinewidth{0.000000pt}%
\definecolor{currentstroke}{rgb}{1.000000,0.576471,0.309804}%
\pgfsetstrokecolor{currentstroke}%
\pgfsetdash{}{0pt}%
\pgfpathmoveto{\pgfqpoint{2.786654in}{3.408401in}}%
\pgfpathlineto{\pgfqpoint{2.786654in}{3.481404in}}%
\pgfpathlineto{\pgfqpoint{2.864606in}{3.444808in}}%
\pgfpathlineto{\pgfqpoint{2.864606in}{3.371805in}}%
\pgfpathlineto{\pgfqpoint{2.786654in}{3.408401in}}%
\pgfpathclose%
\pgfusepath{fill}%
\end{pgfscope}%
\begin{pgfscope}%
\pgfpathrectangle{\pgfqpoint{0.765000in}{0.660000in}}{\pgfqpoint{4.620000in}{4.620000in}}%
\pgfusepath{clip}%
\pgfsetbuttcap%
\pgfsetroundjoin%
\definecolor{currentfill}{rgb}{1.000000,0.576471,0.309804}%
\pgfsetfillcolor{currentfill}%
\pgfsetfillopacity{0.050000}%
\pgfsetlinewidth{0.000000pt}%
\definecolor{currentstroke}{rgb}{1.000000,0.576471,0.309804}%
\pgfsetstrokecolor{currentstroke}%
\pgfsetdash{}{0pt}%
\pgfpathmoveto{\pgfqpoint{2.786654in}{3.408401in}}%
\pgfpathlineto{\pgfqpoint{2.786654in}{3.481404in}}%
\pgfpathlineto{\pgfqpoint{2.708701in}{3.444808in}}%
\pgfpathlineto{\pgfqpoint{2.708701in}{3.371805in}}%
\pgfpathlineto{\pgfqpoint{2.786654in}{3.408401in}}%
\pgfpathclose%
\pgfusepath{fill}%
\end{pgfscope}%
\begin{pgfscope}%
\pgfpathrectangle{\pgfqpoint{0.765000in}{0.660000in}}{\pgfqpoint{4.620000in}{4.620000in}}%
\pgfusepath{clip}%
\pgfsetbuttcap%
\pgfsetroundjoin%
\definecolor{currentfill}{rgb}{1.000000,0.576471,0.309804}%
\pgfsetfillcolor{currentfill}%
\pgfsetfillopacity{0.050000}%
\pgfsetlinewidth{0.000000pt}%
\definecolor{currentstroke}{rgb}{1.000000,0.576471,0.309804}%
\pgfsetstrokecolor{currentstroke}%
\pgfsetdash{}{0pt}%
\pgfpathmoveto{\pgfqpoint{2.786654in}{3.408401in}}%
\pgfpathlineto{\pgfqpoint{2.864606in}{3.371805in}}%
\pgfpathlineto{\pgfqpoint{2.786654in}{3.335208in}}%
\pgfpathlineto{\pgfqpoint{2.708701in}{3.371805in}}%
\pgfpathlineto{\pgfqpoint{2.786654in}{3.408401in}}%
\pgfpathclose%
\pgfusepath{fill}%
\end{pgfscope}%
\begin{pgfscope}%
\pgfpathrectangle{\pgfqpoint{0.765000in}{0.660000in}}{\pgfqpoint{4.620000in}{4.620000in}}%
\pgfusepath{clip}%
\pgfsetbuttcap%
\pgfsetroundjoin%
\definecolor{currentfill}{rgb}{1.000000,0.576471,0.309804}%
\pgfsetfillcolor{currentfill}%
\pgfsetfillopacity{0.050000}%
\pgfsetlinewidth{0.000000pt}%
\definecolor{currentstroke}{rgb}{1.000000,0.576471,0.309804}%
\pgfsetstrokecolor{currentstroke}%
\pgfsetdash{}{0pt}%
\pgfpathmoveto{\pgfqpoint{2.786654in}{3.481404in}}%
\pgfpathlineto{\pgfqpoint{2.864606in}{3.444808in}}%
\pgfpathlineto{\pgfqpoint{2.786654in}{3.408211in}}%
\pgfpathlineto{\pgfqpoint{2.708701in}{3.444808in}}%
\pgfpathlineto{\pgfqpoint{2.786654in}{3.481404in}}%
\pgfpathclose%
\pgfusepath{fill}%
\end{pgfscope}%
\begin{pgfscope}%
\pgfpathrectangle{\pgfqpoint{0.765000in}{0.660000in}}{\pgfqpoint{4.620000in}{4.620000in}}%
\pgfusepath{clip}%
\pgfsetbuttcap%
\pgfsetroundjoin%
\definecolor{currentfill}{rgb}{1.000000,0.576471,0.309804}%
\pgfsetfillcolor{currentfill}%
\pgfsetfillopacity{0.050000}%
\pgfsetlinewidth{0.000000pt}%
\definecolor{currentstroke}{rgb}{1.000000,0.576471,0.309804}%
\pgfsetstrokecolor{currentstroke}%
\pgfsetdash{}{0pt}%
\pgfpathmoveto{\pgfqpoint{2.708701in}{3.371805in}}%
\pgfpathlineto{\pgfqpoint{2.708701in}{3.444808in}}%
\pgfpathlineto{\pgfqpoint{2.786654in}{3.408211in}}%
\pgfpathlineto{\pgfqpoint{2.786654in}{3.335208in}}%
\pgfpathlineto{\pgfqpoint{2.708701in}{3.371805in}}%
\pgfpathclose%
\pgfusepath{fill}%
\end{pgfscope}%
\begin{pgfscope}%
\pgfpathrectangle{\pgfqpoint{0.765000in}{0.660000in}}{\pgfqpoint{4.620000in}{4.620000in}}%
\pgfusepath{clip}%
\pgfsetbuttcap%
\pgfsetroundjoin%
\definecolor{currentfill}{rgb}{1.000000,0.576471,0.309804}%
\pgfsetfillcolor{currentfill}%
\pgfsetfillopacity{0.050000}%
\pgfsetlinewidth{0.000000pt}%
\definecolor{currentstroke}{rgb}{1.000000,0.576471,0.309804}%
\pgfsetstrokecolor{currentstroke}%
\pgfsetdash{}{0pt}%
\pgfpathmoveto{\pgfqpoint{2.864606in}{3.371805in}}%
\pgfpathlineto{\pgfqpoint{2.864606in}{3.444808in}}%
\pgfpathlineto{\pgfqpoint{2.786654in}{3.408211in}}%
\pgfpathlineto{\pgfqpoint{2.786654in}{3.335208in}}%
\pgfpathlineto{\pgfqpoint{2.864606in}{3.371805in}}%
\pgfpathclose%
\pgfusepath{fill}%
\end{pgfscope}%
\begin{pgfscope}%
\pgfpathrectangle{\pgfqpoint{0.765000in}{0.660000in}}{\pgfqpoint{4.620000in}{4.620000in}}%
\pgfusepath{clip}%
\pgfsetbuttcap%
\pgfsetroundjoin%
\definecolor{currentfill}{rgb}{0.176471,0.192157,0.258824}%
\pgfsetfillcolor{currentfill}%
\pgfsetfillopacity{0.050000}%
\pgfsetlinewidth{0.000000pt}%
\definecolor{currentstroke}{rgb}{0.176471,0.192157,0.258824}%
\pgfsetstrokecolor{currentstroke}%
\pgfsetdash{}{0pt}%
\pgfpathmoveto{\pgfqpoint{3.114888in}{2.980783in}}%
\pgfpathlineto{\pgfqpoint{3.114888in}{3.053786in}}%
\pgfpathlineto{\pgfqpoint{3.192841in}{3.017190in}}%
\pgfpathlineto{\pgfqpoint{3.192841in}{2.944186in}}%
\pgfpathlineto{\pgfqpoint{3.114888in}{2.980783in}}%
\pgfpathclose%
\pgfusepath{fill}%
\end{pgfscope}%
\begin{pgfscope}%
\pgfpathrectangle{\pgfqpoint{0.765000in}{0.660000in}}{\pgfqpoint{4.620000in}{4.620000in}}%
\pgfusepath{clip}%
\pgfsetbuttcap%
\pgfsetroundjoin%
\definecolor{currentfill}{rgb}{0.176471,0.192157,0.258824}%
\pgfsetfillcolor{currentfill}%
\pgfsetfillopacity{0.050000}%
\pgfsetlinewidth{0.000000pt}%
\definecolor{currentstroke}{rgb}{0.176471,0.192157,0.258824}%
\pgfsetstrokecolor{currentstroke}%
\pgfsetdash{}{0pt}%
\pgfpathmoveto{\pgfqpoint{3.114888in}{2.980783in}}%
\pgfpathlineto{\pgfqpoint{3.114888in}{3.053786in}}%
\pgfpathlineto{\pgfqpoint{3.036935in}{3.017190in}}%
\pgfpathlineto{\pgfqpoint{3.036935in}{2.944186in}}%
\pgfpathlineto{\pgfqpoint{3.114888in}{2.980783in}}%
\pgfpathclose%
\pgfusepath{fill}%
\end{pgfscope}%
\begin{pgfscope}%
\pgfpathrectangle{\pgfqpoint{0.765000in}{0.660000in}}{\pgfqpoint{4.620000in}{4.620000in}}%
\pgfusepath{clip}%
\pgfsetbuttcap%
\pgfsetroundjoin%
\definecolor{currentfill}{rgb}{0.176471,0.192157,0.258824}%
\pgfsetfillcolor{currentfill}%
\pgfsetfillopacity{0.050000}%
\pgfsetlinewidth{0.000000pt}%
\definecolor{currentstroke}{rgb}{0.176471,0.192157,0.258824}%
\pgfsetstrokecolor{currentstroke}%
\pgfsetdash{}{0pt}%
\pgfpathmoveto{\pgfqpoint{3.114888in}{2.980783in}}%
\pgfpathlineto{\pgfqpoint{3.192841in}{2.944186in}}%
\pgfpathlineto{\pgfqpoint{3.114888in}{2.907590in}}%
\pgfpathlineto{\pgfqpoint{3.036935in}{2.944186in}}%
\pgfpathlineto{\pgfqpoint{3.114888in}{2.980783in}}%
\pgfpathclose%
\pgfusepath{fill}%
\end{pgfscope}%
\begin{pgfscope}%
\pgfpathrectangle{\pgfqpoint{0.765000in}{0.660000in}}{\pgfqpoint{4.620000in}{4.620000in}}%
\pgfusepath{clip}%
\pgfsetbuttcap%
\pgfsetroundjoin%
\definecolor{currentfill}{rgb}{0.176471,0.192157,0.258824}%
\pgfsetfillcolor{currentfill}%
\pgfsetfillopacity{0.050000}%
\pgfsetlinewidth{0.000000pt}%
\definecolor{currentstroke}{rgb}{0.176471,0.192157,0.258824}%
\pgfsetstrokecolor{currentstroke}%
\pgfsetdash{}{0pt}%
\pgfpathmoveto{\pgfqpoint{3.114888in}{3.053786in}}%
\pgfpathlineto{\pgfqpoint{3.192841in}{3.017190in}}%
\pgfpathlineto{\pgfqpoint{3.114888in}{2.980593in}}%
\pgfpathlineto{\pgfqpoint{3.036935in}{3.017190in}}%
\pgfpathlineto{\pgfqpoint{3.114888in}{3.053786in}}%
\pgfpathclose%
\pgfusepath{fill}%
\end{pgfscope}%
\begin{pgfscope}%
\pgfpathrectangle{\pgfqpoint{0.765000in}{0.660000in}}{\pgfqpoint{4.620000in}{4.620000in}}%
\pgfusepath{clip}%
\pgfsetbuttcap%
\pgfsetroundjoin%
\definecolor{currentfill}{rgb}{0.176471,0.192157,0.258824}%
\pgfsetfillcolor{currentfill}%
\pgfsetfillopacity{0.050000}%
\pgfsetlinewidth{0.000000pt}%
\definecolor{currentstroke}{rgb}{0.176471,0.192157,0.258824}%
\pgfsetstrokecolor{currentstroke}%
\pgfsetdash{}{0pt}%
\pgfpathmoveto{\pgfqpoint{3.036935in}{2.944186in}}%
\pgfpathlineto{\pgfqpoint{3.036935in}{3.017190in}}%
\pgfpathlineto{\pgfqpoint{3.114888in}{2.980593in}}%
\pgfpathlineto{\pgfqpoint{3.114888in}{2.907590in}}%
\pgfpathlineto{\pgfqpoint{3.036935in}{2.944186in}}%
\pgfpathclose%
\pgfusepath{fill}%
\end{pgfscope}%
\begin{pgfscope}%
\pgfpathrectangle{\pgfqpoint{0.765000in}{0.660000in}}{\pgfqpoint{4.620000in}{4.620000in}}%
\pgfusepath{clip}%
\pgfsetbuttcap%
\pgfsetroundjoin%
\definecolor{currentfill}{rgb}{0.176471,0.192157,0.258824}%
\pgfsetfillcolor{currentfill}%
\pgfsetfillopacity{0.050000}%
\pgfsetlinewidth{0.000000pt}%
\definecolor{currentstroke}{rgb}{0.176471,0.192157,0.258824}%
\pgfsetstrokecolor{currentstroke}%
\pgfsetdash{}{0pt}%
\pgfpathmoveto{\pgfqpoint{3.192841in}{2.944186in}}%
\pgfpathlineto{\pgfqpoint{3.192841in}{3.017190in}}%
\pgfpathlineto{\pgfqpoint{3.114888in}{2.980593in}}%
\pgfpathlineto{\pgfqpoint{3.114888in}{2.907590in}}%
\pgfpathlineto{\pgfqpoint{3.192841in}{2.944186in}}%
\pgfpathclose%
\pgfusepath{fill}%
\end{pgfscope}%
\begin{pgfscope}%
\pgfpathrectangle{\pgfqpoint{0.765000in}{0.660000in}}{\pgfqpoint{4.620000in}{4.620000in}}%
\pgfusepath{clip}%
\pgfsetbuttcap%
\pgfsetroundjoin%
\definecolor{currentfill}{rgb}{0.176471,0.192157,0.258824}%
\pgfsetfillcolor{currentfill}%
\pgfsetfillopacity{0.050000}%
\pgfsetlinewidth{0.000000pt}%
\definecolor{currentstroke}{rgb}{0.176471,0.192157,0.258824}%
\pgfsetstrokecolor{currentstroke}%
\pgfsetdash{}{0pt}%
\pgfpathmoveto{\pgfqpoint{3.102166in}{2.988185in}}%
\pgfpathlineto{\pgfqpoint{3.102166in}{3.061189in}}%
\pgfpathlineto{\pgfqpoint{3.180119in}{3.024592in}}%
\pgfpathlineto{\pgfqpoint{3.180119in}{2.951589in}}%
\pgfpathlineto{\pgfqpoint{3.102166in}{2.988185in}}%
\pgfpathclose%
\pgfusepath{fill}%
\end{pgfscope}%
\begin{pgfscope}%
\pgfpathrectangle{\pgfqpoint{0.765000in}{0.660000in}}{\pgfqpoint{4.620000in}{4.620000in}}%
\pgfusepath{clip}%
\pgfsetbuttcap%
\pgfsetroundjoin%
\definecolor{currentfill}{rgb}{0.176471,0.192157,0.258824}%
\pgfsetfillcolor{currentfill}%
\pgfsetfillopacity{0.050000}%
\pgfsetlinewidth{0.000000pt}%
\definecolor{currentstroke}{rgb}{0.176471,0.192157,0.258824}%
\pgfsetstrokecolor{currentstroke}%
\pgfsetdash{}{0pt}%
\pgfpathmoveto{\pgfqpoint{3.102166in}{2.988185in}}%
\pgfpathlineto{\pgfqpoint{3.102166in}{3.061189in}}%
\pgfpathlineto{\pgfqpoint{3.024213in}{3.024592in}}%
\pgfpathlineto{\pgfqpoint{3.024213in}{2.951589in}}%
\pgfpathlineto{\pgfqpoint{3.102166in}{2.988185in}}%
\pgfpathclose%
\pgfusepath{fill}%
\end{pgfscope}%
\begin{pgfscope}%
\pgfpathrectangle{\pgfqpoint{0.765000in}{0.660000in}}{\pgfqpoint{4.620000in}{4.620000in}}%
\pgfusepath{clip}%
\pgfsetbuttcap%
\pgfsetroundjoin%
\definecolor{currentfill}{rgb}{0.176471,0.192157,0.258824}%
\pgfsetfillcolor{currentfill}%
\pgfsetfillopacity{0.050000}%
\pgfsetlinewidth{0.000000pt}%
\definecolor{currentstroke}{rgb}{0.176471,0.192157,0.258824}%
\pgfsetstrokecolor{currentstroke}%
\pgfsetdash{}{0pt}%
\pgfpathmoveto{\pgfqpoint{3.102166in}{2.988185in}}%
\pgfpathlineto{\pgfqpoint{3.180119in}{2.951589in}}%
\pgfpathlineto{\pgfqpoint{3.102166in}{2.914992in}}%
\pgfpathlineto{\pgfqpoint{3.024213in}{2.951589in}}%
\pgfpathlineto{\pgfqpoint{3.102166in}{2.988185in}}%
\pgfpathclose%
\pgfusepath{fill}%
\end{pgfscope}%
\begin{pgfscope}%
\pgfpathrectangle{\pgfqpoint{0.765000in}{0.660000in}}{\pgfqpoint{4.620000in}{4.620000in}}%
\pgfusepath{clip}%
\pgfsetbuttcap%
\pgfsetroundjoin%
\definecolor{currentfill}{rgb}{0.176471,0.192157,0.258824}%
\pgfsetfillcolor{currentfill}%
\pgfsetfillopacity{0.050000}%
\pgfsetlinewidth{0.000000pt}%
\definecolor{currentstroke}{rgb}{0.176471,0.192157,0.258824}%
\pgfsetstrokecolor{currentstroke}%
\pgfsetdash{}{0pt}%
\pgfpathmoveto{\pgfqpoint{3.102166in}{3.061189in}}%
\pgfpathlineto{\pgfqpoint{3.180119in}{3.024592in}}%
\pgfpathlineto{\pgfqpoint{3.102166in}{2.987996in}}%
\pgfpathlineto{\pgfqpoint{3.024213in}{3.024592in}}%
\pgfpathlineto{\pgfqpoint{3.102166in}{3.061189in}}%
\pgfpathclose%
\pgfusepath{fill}%
\end{pgfscope}%
\begin{pgfscope}%
\pgfpathrectangle{\pgfqpoint{0.765000in}{0.660000in}}{\pgfqpoint{4.620000in}{4.620000in}}%
\pgfusepath{clip}%
\pgfsetbuttcap%
\pgfsetroundjoin%
\definecolor{currentfill}{rgb}{0.176471,0.192157,0.258824}%
\pgfsetfillcolor{currentfill}%
\pgfsetfillopacity{0.050000}%
\pgfsetlinewidth{0.000000pt}%
\definecolor{currentstroke}{rgb}{0.176471,0.192157,0.258824}%
\pgfsetstrokecolor{currentstroke}%
\pgfsetdash{}{0pt}%
\pgfpathmoveto{\pgfqpoint{3.024213in}{2.951589in}}%
\pgfpathlineto{\pgfqpoint{3.024213in}{3.024592in}}%
\pgfpathlineto{\pgfqpoint{3.102166in}{2.987996in}}%
\pgfpathlineto{\pgfqpoint{3.102166in}{2.914992in}}%
\pgfpathlineto{\pgfqpoint{3.024213in}{2.951589in}}%
\pgfpathclose%
\pgfusepath{fill}%
\end{pgfscope}%
\begin{pgfscope}%
\pgfpathrectangle{\pgfqpoint{0.765000in}{0.660000in}}{\pgfqpoint{4.620000in}{4.620000in}}%
\pgfusepath{clip}%
\pgfsetbuttcap%
\pgfsetroundjoin%
\definecolor{currentfill}{rgb}{0.176471,0.192157,0.258824}%
\pgfsetfillcolor{currentfill}%
\pgfsetfillopacity{0.050000}%
\pgfsetlinewidth{0.000000pt}%
\definecolor{currentstroke}{rgb}{0.176471,0.192157,0.258824}%
\pgfsetstrokecolor{currentstroke}%
\pgfsetdash{}{0pt}%
\pgfpathmoveto{\pgfqpoint{3.180119in}{2.951589in}}%
\pgfpathlineto{\pgfqpoint{3.180119in}{3.024592in}}%
\pgfpathlineto{\pgfqpoint{3.102166in}{2.987996in}}%
\pgfpathlineto{\pgfqpoint{3.102166in}{2.914992in}}%
\pgfpathlineto{\pgfqpoint{3.180119in}{2.951589in}}%
\pgfpathclose%
\pgfusepath{fill}%
\end{pgfscope}%
\begin{pgfscope}%
\pgfpathrectangle{\pgfqpoint{0.765000in}{0.660000in}}{\pgfqpoint{4.620000in}{4.620000in}}%
\pgfusepath{clip}%
\pgfsetbuttcap%
\pgfsetroundjoin%
\definecolor{currentfill}{rgb}{0.176471,0.192157,0.258824}%
\pgfsetfillcolor{currentfill}%
\pgfsetfillopacity{0.050000}%
\pgfsetlinewidth{0.000000pt}%
\definecolor{currentstroke}{rgb}{0.176471,0.192157,0.258824}%
\pgfsetstrokecolor{currentstroke}%
\pgfsetdash{}{0pt}%
\pgfpathmoveto{\pgfqpoint{3.241130in}{2.835206in}}%
\pgfpathlineto{\pgfqpoint{3.241130in}{2.908210in}}%
\pgfpathlineto{\pgfqpoint{3.319083in}{2.871613in}}%
\pgfpathlineto{\pgfqpoint{3.319083in}{2.798610in}}%
\pgfpathlineto{\pgfqpoint{3.241130in}{2.835206in}}%
\pgfpathclose%
\pgfusepath{fill}%
\end{pgfscope}%
\begin{pgfscope}%
\pgfpathrectangle{\pgfqpoint{0.765000in}{0.660000in}}{\pgfqpoint{4.620000in}{4.620000in}}%
\pgfusepath{clip}%
\pgfsetbuttcap%
\pgfsetroundjoin%
\definecolor{currentfill}{rgb}{0.176471,0.192157,0.258824}%
\pgfsetfillcolor{currentfill}%
\pgfsetfillopacity{0.050000}%
\pgfsetlinewidth{0.000000pt}%
\definecolor{currentstroke}{rgb}{0.176471,0.192157,0.258824}%
\pgfsetstrokecolor{currentstroke}%
\pgfsetdash{}{0pt}%
\pgfpathmoveto{\pgfqpoint{3.241130in}{2.835206in}}%
\pgfpathlineto{\pgfqpoint{3.241130in}{2.908210in}}%
\pgfpathlineto{\pgfqpoint{3.163177in}{2.871613in}}%
\pgfpathlineto{\pgfqpoint{3.163177in}{2.798610in}}%
\pgfpathlineto{\pgfqpoint{3.241130in}{2.835206in}}%
\pgfpathclose%
\pgfusepath{fill}%
\end{pgfscope}%
\begin{pgfscope}%
\pgfpathrectangle{\pgfqpoint{0.765000in}{0.660000in}}{\pgfqpoint{4.620000in}{4.620000in}}%
\pgfusepath{clip}%
\pgfsetbuttcap%
\pgfsetroundjoin%
\definecolor{currentfill}{rgb}{0.176471,0.192157,0.258824}%
\pgfsetfillcolor{currentfill}%
\pgfsetfillopacity{0.050000}%
\pgfsetlinewidth{0.000000pt}%
\definecolor{currentstroke}{rgb}{0.176471,0.192157,0.258824}%
\pgfsetstrokecolor{currentstroke}%
\pgfsetdash{}{0pt}%
\pgfpathmoveto{\pgfqpoint{3.241130in}{2.835206in}}%
\pgfpathlineto{\pgfqpoint{3.319083in}{2.798610in}}%
\pgfpathlineto{\pgfqpoint{3.241130in}{2.762013in}}%
\pgfpathlineto{\pgfqpoint{3.163177in}{2.798610in}}%
\pgfpathlineto{\pgfqpoint{3.241130in}{2.835206in}}%
\pgfpathclose%
\pgfusepath{fill}%
\end{pgfscope}%
\begin{pgfscope}%
\pgfpathrectangle{\pgfqpoint{0.765000in}{0.660000in}}{\pgfqpoint{4.620000in}{4.620000in}}%
\pgfusepath{clip}%
\pgfsetbuttcap%
\pgfsetroundjoin%
\definecolor{currentfill}{rgb}{0.176471,0.192157,0.258824}%
\pgfsetfillcolor{currentfill}%
\pgfsetfillopacity{0.050000}%
\pgfsetlinewidth{0.000000pt}%
\definecolor{currentstroke}{rgb}{0.176471,0.192157,0.258824}%
\pgfsetstrokecolor{currentstroke}%
\pgfsetdash{}{0pt}%
\pgfpathmoveto{\pgfqpoint{3.241130in}{2.908210in}}%
\pgfpathlineto{\pgfqpoint{3.319083in}{2.871613in}}%
\pgfpathlineto{\pgfqpoint{3.241130in}{2.835016in}}%
\pgfpathlineto{\pgfqpoint{3.163177in}{2.871613in}}%
\pgfpathlineto{\pgfqpoint{3.241130in}{2.908210in}}%
\pgfpathclose%
\pgfusepath{fill}%
\end{pgfscope}%
\begin{pgfscope}%
\pgfpathrectangle{\pgfqpoint{0.765000in}{0.660000in}}{\pgfqpoint{4.620000in}{4.620000in}}%
\pgfusepath{clip}%
\pgfsetbuttcap%
\pgfsetroundjoin%
\definecolor{currentfill}{rgb}{0.176471,0.192157,0.258824}%
\pgfsetfillcolor{currentfill}%
\pgfsetfillopacity{0.050000}%
\pgfsetlinewidth{0.000000pt}%
\definecolor{currentstroke}{rgb}{0.176471,0.192157,0.258824}%
\pgfsetstrokecolor{currentstroke}%
\pgfsetdash{}{0pt}%
\pgfpathmoveto{\pgfqpoint{3.163177in}{2.798610in}}%
\pgfpathlineto{\pgfqpoint{3.163177in}{2.871613in}}%
\pgfpathlineto{\pgfqpoint{3.241130in}{2.835016in}}%
\pgfpathlineto{\pgfqpoint{3.241130in}{2.762013in}}%
\pgfpathlineto{\pgfqpoint{3.163177in}{2.798610in}}%
\pgfpathclose%
\pgfusepath{fill}%
\end{pgfscope}%
\begin{pgfscope}%
\pgfpathrectangle{\pgfqpoint{0.765000in}{0.660000in}}{\pgfqpoint{4.620000in}{4.620000in}}%
\pgfusepath{clip}%
\pgfsetbuttcap%
\pgfsetroundjoin%
\definecolor{currentfill}{rgb}{0.176471,0.192157,0.258824}%
\pgfsetfillcolor{currentfill}%
\pgfsetfillopacity{0.050000}%
\pgfsetlinewidth{0.000000pt}%
\definecolor{currentstroke}{rgb}{0.176471,0.192157,0.258824}%
\pgfsetstrokecolor{currentstroke}%
\pgfsetdash{}{0pt}%
\pgfpathmoveto{\pgfqpoint{3.319083in}{2.798610in}}%
\pgfpathlineto{\pgfqpoint{3.319083in}{2.871613in}}%
\pgfpathlineto{\pgfqpoint{3.241130in}{2.835016in}}%
\pgfpathlineto{\pgfqpoint{3.241130in}{2.762013in}}%
\pgfpathlineto{\pgfqpoint{3.319083in}{2.798610in}}%
\pgfpathclose%
\pgfusepath{fill}%
\end{pgfscope}%
\begin{pgfscope}%
\pgfpathrectangle{\pgfqpoint{0.765000in}{0.660000in}}{\pgfqpoint{4.620000in}{4.620000in}}%
\pgfusepath{clip}%
\pgfsetbuttcap%
\pgfsetroundjoin%
\definecolor{currentfill}{rgb}{0.176471,0.192157,0.258824}%
\pgfsetfillcolor{currentfill}%
\pgfsetfillopacity{0.050000}%
\pgfsetlinewidth{0.000000pt}%
\definecolor{currentstroke}{rgb}{0.176471,0.192157,0.258824}%
\pgfsetstrokecolor{currentstroke}%
\pgfsetdash{}{0pt}%
\pgfpathmoveto{\pgfqpoint{3.305648in}{2.948249in}}%
\pgfpathlineto{\pgfqpoint{3.305648in}{3.021253in}}%
\pgfpathlineto{\pgfqpoint{3.383601in}{2.984656in}}%
\pgfpathlineto{\pgfqpoint{3.383601in}{2.911653in}}%
\pgfpathlineto{\pgfqpoint{3.305648in}{2.948249in}}%
\pgfpathclose%
\pgfusepath{fill}%
\end{pgfscope}%
\begin{pgfscope}%
\pgfpathrectangle{\pgfqpoint{0.765000in}{0.660000in}}{\pgfqpoint{4.620000in}{4.620000in}}%
\pgfusepath{clip}%
\pgfsetbuttcap%
\pgfsetroundjoin%
\definecolor{currentfill}{rgb}{0.176471,0.192157,0.258824}%
\pgfsetfillcolor{currentfill}%
\pgfsetfillopacity{0.050000}%
\pgfsetlinewidth{0.000000pt}%
\definecolor{currentstroke}{rgb}{0.176471,0.192157,0.258824}%
\pgfsetstrokecolor{currentstroke}%
\pgfsetdash{}{0pt}%
\pgfpathmoveto{\pgfqpoint{3.305648in}{2.948249in}}%
\pgfpathlineto{\pgfqpoint{3.305648in}{3.021253in}}%
\pgfpathlineto{\pgfqpoint{3.227696in}{2.984656in}}%
\pgfpathlineto{\pgfqpoint{3.227696in}{2.911653in}}%
\pgfpathlineto{\pgfqpoint{3.305648in}{2.948249in}}%
\pgfpathclose%
\pgfusepath{fill}%
\end{pgfscope}%
\begin{pgfscope}%
\pgfpathrectangle{\pgfqpoint{0.765000in}{0.660000in}}{\pgfqpoint{4.620000in}{4.620000in}}%
\pgfusepath{clip}%
\pgfsetbuttcap%
\pgfsetroundjoin%
\definecolor{currentfill}{rgb}{0.176471,0.192157,0.258824}%
\pgfsetfillcolor{currentfill}%
\pgfsetfillopacity{0.050000}%
\pgfsetlinewidth{0.000000pt}%
\definecolor{currentstroke}{rgb}{0.176471,0.192157,0.258824}%
\pgfsetstrokecolor{currentstroke}%
\pgfsetdash{}{0pt}%
\pgfpathmoveto{\pgfqpoint{3.305648in}{2.948249in}}%
\pgfpathlineto{\pgfqpoint{3.383601in}{2.911653in}}%
\pgfpathlineto{\pgfqpoint{3.305648in}{2.875056in}}%
\pgfpathlineto{\pgfqpoint{3.227696in}{2.911653in}}%
\pgfpathlineto{\pgfqpoint{3.305648in}{2.948249in}}%
\pgfpathclose%
\pgfusepath{fill}%
\end{pgfscope}%
\begin{pgfscope}%
\pgfpathrectangle{\pgfqpoint{0.765000in}{0.660000in}}{\pgfqpoint{4.620000in}{4.620000in}}%
\pgfusepath{clip}%
\pgfsetbuttcap%
\pgfsetroundjoin%
\definecolor{currentfill}{rgb}{0.176471,0.192157,0.258824}%
\pgfsetfillcolor{currentfill}%
\pgfsetfillopacity{0.050000}%
\pgfsetlinewidth{0.000000pt}%
\definecolor{currentstroke}{rgb}{0.176471,0.192157,0.258824}%
\pgfsetstrokecolor{currentstroke}%
\pgfsetdash{}{0pt}%
\pgfpathmoveto{\pgfqpoint{3.305648in}{3.021253in}}%
\pgfpathlineto{\pgfqpoint{3.383601in}{2.984656in}}%
\pgfpathlineto{\pgfqpoint{3.305648in}{2.948060in}}%
\pgfpathlineto{\pgfqpoint{3.227696in}{2.984656in}}%
\pgfpathlineto{\pgfqpoint{3.305648in}{3.021253in}}%
\pgfpathclose%
\pgfusepath{fill}%
\end{pgfscope}%
\begin{pgfscope}%
\pgfpathrectangle{\pgfqpoint{0.765000in}{0.660000in}}{\pgfqpoint{4.620000in}{4.620000in}}%
\pgfusepath{clip}%
\pgfsetbuttcap%
\pgfsetroundjoin%
\definecolor{currentfill}{rgb}{0.176471,0.192157,0.258824}%
\pgfsetfillcolor{currentfill}%
\pgfsetfillopacity{0.050000}%
\pgfsetlinewidth{0.000000pt}%
\definecolor{currentstroke}{rgb}{0.176471,0.192157,0.258824}%
\pgfsetstrokecolor{currentstroke}%
\pgfsetdash{}{0pt}%
\pgfpathmoveto{\pgfqpoint{3.227696in}{2.911653in}}%
\pgfpathlineto{\pgfqpoint{3.227696in}{2.984656in}}%
\pgfpathlineto{\pgfqpoint{3.305648in}{2.948060in}}%
\pgfpathlineto{\pgfqpoint{3.305648in}{2.875056in}}%
\pgfpathlineto{\pgfqpoint{3.227696in}{2.911653in}}%
\pgfpathclose%
\pgfusepath{fill}%
\end{pgfscope}%
\begin{pgfscope}%
\pgfpathrectangle{\pgfqpoint{0.765000in}{0.660000in}}{\pgfqpoint{4.620000in}{4.620000in}}%
\pgfusepath{clip}%
\pgfsetbuttcap%
\pgfsetroundjoin%
\definecolor{currentfill}{rgb}{0.176471,0.192157,0.258824}%
\pgfsetfillcolor{currentfill}%
\pgfsetfillopacity{0.050000}%
\pgfsetlinewidth{0.000000pt}%
\definecolor{currentstroke}{rgb}{0.176471,0.192157,0.258824}%
\pgfsetstrokecolor{currentstroke}%
\pgfsetdash{}{0pt}%
\pgfpathmoveto{\pgfqpoint{3.383601in}{2.911653in}}%
\pgfpathlineto{\pgfqpoint{3.383601in}{2.984656in}}%
\pgfpathlineto{\pgfqpoint{3.305648in}{2.948060in}}%
\pgfpathlineto{\pgfqpoint{3.305648in}{2.875056in}}%
\pgfpathlineto{\pgfqpoint{3.383601in}{2.911653in}}%
\pgfpathclose%
\pgfusepath{fill}%
\end{pgfscope}%
\begin{pgfscope}%
\pgfpathrectangle{\pgfqpoint{0.765000in}{0.660000in}}{\pgfqpoint{4.620000in}{4.620000in}}%
\pgfusepath{clip}%
\pgfsetbuttcap%
\pgfsetroundjoin%
\definecolor{currentfill}{rgb}{0.176471,0.192157,0.258824}%
\pgfsetfillcolor{currentfill}%
\pgfsetfillopacity{0.050000}%
\pgfsetlinewidth{0.000000pt}%
\definecolor{currentstroke}{rgb}{0.176471,0.192157,0.258824}%
\pgfsetstrokecolor{currentstroke}%
\pgfsetdash{}{0pt}%
\pgfpathmoveto{\pgfqpoint{3.105140in}{2.958336in}}%
\pgfpathlineto{\pgfqpoint{3.105140in}{3.031339in}}%
\pgfpathlineto{\pgfqpoint{3.183093in}{2.994742in}}%
\pgfpathlineto{\pgfqpoint{3.183093in}{2.921739in}}%
\pgfpathlineto{\pgfqpoint{3.105140in}{2.958336in}}%
\pgfpathclose%
\pgfusepath{fill}%
\end{pgfscope}%
\begin{pgfscope}%
\pgfpathrectangle{\pgfqpoint{0.765000in}{0.660000in}}{\pgfqpoint{4.620000in}{4.620000in}}%
\pgfusepath{clip}%
\pgfsetbuttcap%
\pgfsetroundjoin%
\definecolor{currentfill}{rgb}{0.176471,0.192157,0.258824}%
\pgfsetfillcolor{currentfill}%
\pgfsetfillopacity{0.050000}%
\pgfsetlinewidth{0.000000pt}%
\definecolor{currentstroke}{rgb}{0.176471,0.192157,0.258824}%
\pgfsetstrokecolor{currentstroke}%
\pgfsetdash{}{0pt}%
\pgfpathmoveto{\pgfqpoint{3.105140in}{2.958336in}}%
\pgfpathlineto{\pgfqpoint{3.105140in}{3.031339in}}%
\pgfpathlineto{\pgfqpoint{3.027187in}{2.994742in}}%
\pgfpathlineto{\pgfqpoint{3.027187in}{2.921739in}}%
\pgfpathlineto{\pgfqpoint{3.105140in}{2.958336in}}%
\pgfpathclose%
\pgfusepath{fill}%
\end{pgfscope}%
\begin{pgfscope}%
\pgfpathrectangle{\pgfqpoint{0.765000in}{0.660000in}}{\pgfqpoint{4.620000in}{4.620000in}}%
\pgfusepath{clip}%
\pgfsetbuttcap%
\pgfsetroundjoin%
\definecolor{currentfill}{rgb}{0.176471,0.192157,0.258824}%
\pgfsetfillcolor{currentfill}%
\pgfsetfillopacity{0.050000}%
\pgfsetlinewidth{0.000000pt}%
\definecolor{currentstroke}{rgb}{0.176471,0.192157,0.258824}%
\pgfsetstrokecolor{currentstroke}%
\pgfsetdash{}{0pt}%
\pgfpathmoveto{\pgfqpoint{3.105140in}{2.958336in}}%
\pgfpathlineto{\pgfqpoint{3.183093in}{2.921739in}}%
\pgfpathlineto{\pgfqpoint{3.105140in}{2.885142in}}%
\pgfpathlineto{\pgfqpoint{3.027187in}{2.921739in}}%
\pgfpathlineto{\pgfqpoint{3.105140in}{2.958336in}}%
\pgfpathclose%
\pgfusepath{fill}%
\end{pgfscope}%
\begin{pgfscope}%
\pgfpathrectangle{\pgfqpoint{0.765000in}{0.660000in}}{\pgfqpoint{4.620000in}{4.620000in}}%
\pgfusepath{clip}%
\pgfsetbuttcap%
\pgfsetroundjoin%
\definecolor{currentfill}{rgb}{0.176471,0.192157,0.258824}%
\pgfsetfillcolor{currentfill}%
\pgfsetfillopacity{0.050000}%
\pgfsetlinewidth{0.000000pt}%
\definecolor{currentstroke}{rgb}{0.176471,0.192157,0.258824}%
\pgfsetstrokecolor{currentstroke}%
\pgfsetdash{}{0pt}%
\pgfpathmoveto{\pgfqpoint{3.105140in}{3.031339in}}%
\pgfpathlineto{\pgfqpoint{3.183093in}{2.994742in}}%
\pgfpathlineto{\pgfqpoint{3.105140in}{2.958146in}}%
\pgfpathlineto{\pgfqpoint{3.027187in}{2.994742in}}%
\pgfpathlineto{\pgfqpoint{3.105140in}{3.031339in}}%
\pgfpathclose%
\pgfusepath{fill}%
\end{pgfscope}%
\begin{pgfscope}%
\pgfpathrectangle{\pgfqpoint{0.765000in}{0.660000in}}{\pgfqpoint{4.620000in}{4.620000in}}%
\pgfusepath{clip}%
\pgfsetbuttcap%
\pgfsetroundjoin%
\definecolor{currentfill}{rgb}{0.176471,0.192157,0.258824}%
\pgfsetfillcolor{currentfill}%
\pgfsetfillopacity{0.050000}%
\pgfsetlinewidth{0.000000pt}%
\definecolor{currentstroke}{rgb}{0.176471,0.192157,0.258824}%
\pgfsetstrokecolor{currentstroke}%
\pgfsetdash{}{0pt}%
\pgfpathmoveto{\pgfqpoint{3.027187in}{2.921739in}}%
\pgfpathlineto{\pgfqpoint{3.027187in}{2.994742in}}%
\pgfpathlineto{\pgfqpoint{3.105140in}{2.958146in}}%
\pgfpathlineto{\pgfqpoint{3.105140in}{2.885142in}}%
\pgfpathlineto{\pgfqpoint{3.027187in}{2.921739in}}%
\pgfpathclose%
\pgfusepath{fill}%
\end{pgfscope}%
\begin{pgfscope}%
\pgfpathrectangle{\pgfqpoint{0.765000in}{0.660000in}}{\pgfqpoint{4.620000in}{4.620000in}}%
\pgfusepath{clip}%
\pgfsetbuttcap%
\pgfsetroundjoin%
\definecolor{currentfill}{rgb}{0.176471,0.192157,0.258824}%
\pgfsetfillcolor{currentfill}%
\pgfsetfillopacity{0.050000}%
\pgfsetlinewidth{0.000000pt}%
\definecolor{currentstroke}{rgb}{0.176471,0.192157,0.258824}%
\pgfsetstrokecolor{currentstroke}%
\pgfsetdash{}{0pt}%
\pgfpathmoveto{\pgfqpoint{3.183093in}{2.921739in}}%
\pgfpathlineto{\pgfqpoint{3.183093in}{2.994742in}}%
\pgfpathlineto{\pgfqpoint{3.105140in}{2.958146in}}%
\pgfpathlineto{\pgfqpoint{3.105140in}{2.885142in}}%
\pgfpathlineto{\pgfqpoint{3.183093in}{2.921739in}}%
\pgfpathclose%
\pgfusepath{fill}%
\end{pgfscope}%
\begin{pgfscope}%
\pgfpathrectangle{\pgfqpoint{0.765000in}{0.660000in}}{\pgfqpoint{4.620000in}{4.620000in}}%
\pgfusepath{clip}%
\pgfsetbuttcap%
\pgfsetroundjoin%
\definecolor{currentfill}{rgb}{0.176471,0.192157,0.258824}%
\pgfsetfillcolor{currentfill}%
\pgfsetfillopacity{0.050000}%
\pgfsetlinewidth{0.000000pt}%
\definecolor{currentstroke}{rgb}{0.176471,0.192157,0.258824}%
\pgfsetstrokecolor{currentstroke}%
\pgfsetdash{}{0pt}%
\pgfpathmoveto{\pgfqpoint{3.116086in}{2.982470in}}%
\pgfpathlineto{\pgfqpoint{3.116086in}{3.055473in}}%
\pgfpathlineto{\pgfqpoint{3.194039in}{3.018876in}}%
\pgfpathlineto{\pgfqpoint{3.194039in}{2.945873in}}%
\pgfpathlineto{\pgfqpoint{3.116086in}{2.982470in}}%
\pgfpathclose%
\pgfusepath{fill}%
\end{pgfscope}%
\begin{pgfscope}%
\pgfpathrectangle{\pgfqpoint{0.765000in}{0.660000in}}{\pgfqpoint{4.620000in}{4.620000in}}%
\pgfusepath{clip}%
\pgfsetbuttcap%
\pgfsetroundjoin%
\definecolor{currentfill}{rgb}{0.176471,0.192157,0.258824}%
\pgfsetfillcolor{currentfill}%
\pgfsetfillopacity{0.050000}%
\pgfsetlinewidth{0.000000pt}%
\definecolor{currentstroke}{rgb}{0.176471,0.192157,0.258824}%
\pgfsetstrokecolor{currentstroke}%
\pgfsetdash{}{0pt}%
\pgfpathmoveto{\pgfqpoint{3.116086in}{2.982470in}}%
\pgfpathlineto{\pgfqpoint{3.116086in}{3.055473in}}%
\pgfpathlineto{\pgfqpoint{3.038134in}{3.018876in}}%
\pgfpathlineto{\pgfqpoint{3.038134in}{2.945873in}}%
\pgfpathlineto{\pgfqpoint{3.116086in}{2.982470in}}%
\pgfpathclose%
\pgfusepath{fill}%
\end{pgfscope}%
\begin{pgfscope}%
\pgfpathrectangle{\pgfqpoint{0.765000in}{0.660000in}}{\pgfqpoint{4.620000in}{4.620000in}}%
\pgfusepath{clip}%
\pgfsetbuttcap%
\pgfsetroundjoin%
\definecolor{currentfill}{rgb}{0.176471,0.192157,0.258824}%
\pgfsetfillcolor{currentfill}%
\pgfsetfillopacity{0.050000}%
\pgfsetlinewidth{0.000000pt}%
\definecolor{currentstroke}{rgb}{0.176471,0.192157,0.258824}%
\pgfsetstrokecolor{currentstroke}%
\pgfsetdash{}{0pt}%
\pgfpathmoveto{\pgfqpoint{3.116086in}{2.982470in}}%
\pgfpathlineto{\pgfqpoint{3.194039in}{2.945873in}}%
\pgfpathlineto{\pgfqpoint{3.116086in}{2.909276in}}%
\pgfpathlineto{\pgfqpoint{3.038134in}{2.945873in}}%
\pgfpathlineto{\pgfqpoint{3.116086in}{2.982470in}}%
\pgfpathclose%
\pgfusepath{fill}%
\end{pgfscope}%
\begin{pgfscope}%
\pgfpathrectangle{\pgfqpoint{0.765000in}{0.660000in}}{\pgfqpoint{4.620000in}{4.620000in}}%
\pgfusepath{clip}%
\pgfsetbuttcap%
\pgfsetroundjoin%
\definecolor{currentfill}{rgb}{0.176471,0.192157,0.258824}%
\pgfsetfillcolor{currentfill}%
\pgfsetfillopacity{0.050000}%
\pgfsetlinewidth{0.000000pt}%
\definecolor{currentstroke}{rgb}{0.176471,0.192157,0.258824}%
\pgfsetstrokecolor{currentstroke}%
\pgfsetdash{}{0pt}%
\pgfpathmoveto{\pgfqpoint{3.116086in}{3.055473in}}%
\pgfpathlineto{\pgfqpoint{3.194039in}{3.018876in}}%
\pgfpathlineto{\pgfqpoint{3.116086in}{2.982280in}}%
\pgfpathlineto{\pgfqpoint{3.038134in}{3.018876in}}%
\pgfpathlineto{\pgfqpoint{3.116086in}{3.055473in}}%
\pgfpathclose%
\pgfusepath{fill}%
\end{pgfscope}%
\begin{pgfscope}%
\pgfpathrectangle{\pgfqpoint{0.765000in}{0.660000in}}{\pgfqpoint{4.620000in}{4.620000in}}%
\pgfusepath{clip}%
\pgfsetbuttcap%
\pgfsetroundjoin%
\definecolor{currentfill}{rgb}{0.176471,0.192157,0.258824}%
\pgfsetfillcolor{currentfill}%
\pgfsetfillopacity{0.050000}%
\pgfsetlinewidth{0.000000pt}%
\definecolor{currentstroke}{rgb}{0.176471,0.192157,0.258824}%
\pgfsetstrokecolor{currentstroke}%
\pgfsetdash{}{0pt}%
\pgfpathmoveto{\pgfqpoint{3.038134in}{2.945873in}}%
\pgfpathlineto{\pgfqpoint{3.038134in}{3.018876in}}%
\pgfpathlineto{\pgfqpoint{3.116086in}{2.982280in}}%
\pgfpathlineto{\pgfqpoint{3.116086in}{2.909276in}}%
\pgfpathlineto{\pgfqpoint{3.038134in}{2.945873in}}%
\pgfpathclose%
\pgfusepath{fill}%
\end{pgfscope}%
\begin{pgfscope}%
\pgfpathrectangle{\pgfqpoint{0.765000in}{0.660000in}}{\pgfqpoint{4.620000in}{4.620000in}}%
\pgfusepath{clip}%
\pgfsetbuttcap%
\pgfsetroundjoin%
\definecolor{currentfill}{rgb}{0.176471,0.192157,0.258824}%
\pgfsetfillcolor{currentfill}%
\pgfsetfillopacity{0.050000}%
\pgfsetlinewidth{0.000000pt}%
\definecolor{currentstroke}{rgb}{0.176471,0.192157,0.258824}%
\pgfsetstrokecolor{currentstroke}%
\pgfsetdash{}{0pt}%
\pgfpathmoveto{\pgfqpoint{3.194039in}{2.945873in}}%
\pgfpathlineto{\pgfqpoint{3.194039in}{3.018876in}}%
\pgfpathlineto{\pgfqpoint{3.116086in}{2.982280in}}%
\pgfpathlineto{\pgfqpoint{3.116086in}{2.909276in}}%
\pgfpathlineto{\pgfqpoint{3.194039in}{2.945873in}}%
\pgfpathclose%
\pgfusepath{fill}%
\end{pgfscope}%
\begin{pgfscope}%
\pgfpathrectangle{\pgfqpoint{0.765000in}{0.660000in}}{\pgfqpoint{4.620000in}{4.620000in}}%
\pgfusepath{clip}%
\pgfsetbuttcap%
\pgfsetroundjoin%
\definecolor{currentfill}{rgb}{1.000000,0.576471,0.309804}%
\pgfsetfillcolor{currentfill}%
\pgfsetfillopacity{0.050000}%
\pgfsetlinewidth{0.000000pt}%
\definecolor{currentstroke}{rgb}{1.000000,0.576471,0.309804}%
\pgfsetstrokecolor{currentstroke}%
\pgfsetdash{}{0pt}%
\pgfpathmoveto{\pgfqpoint{2.684867in}{3.374382in}}%
\pgfpathlineto{\pgfqpoint{2.684867in}{3.447385in}}%
\pgfpathlineto{\pgfqpoint{2.762819in}{3.410789in}}%
\pgfpathlineto{\pgfqpoint{2.762819in}{3.337786in}}%
\pgfpathlineto{\pgfqpoint{2.684867in}{3.374382in}}%
\pgfpathclose%
\pgfusepath{fill}%
\end{pgfscope}%
\begin{pgfscope}%
\pgfpathrectangle{\pgfqpoint{0.765000in}{0.660000in}}{\pgfqpoint{4.620000in}{4.620000in}}%
\pgfusepath{clip}%
\pgfsetbuttcap%
\pgfsetroundjoin%
\definecolor{currentfill}{rgb}{1.000000,0.576471,0.309804}%
\pgfsetfillcolor{currentfill}%
\pgfsetfillopacity{0.050000}%
\pgfsetlinewidth{0.000000pt}%
\definecolor{currentstroke}{rgb}{1.000000,0.576471,0.309804}%
\pgfsetstrokecolor{currentstroke}%
\pgfsetdash{}{0pt}%
\pgfpathmoveto{\pgfqpoint{2.684867in}{3.374382in}}%
\pgfpathlineto{\pgfqpoint{2.684867in}{3.447385in}}%
\pgfpathlineto{\pgfqpoint{2.606914in}{3.410789in}}%
\pgfpathlineto{\pgfqpoint{2.606914in}{3.337786in}}%
\pgfpathlineto{\pgfqpoint{2.684867in}{3.374382in}}%
\pgfpathclose%
\pgfusepath{fill}%
\end{pgfscope}%
\begin{pgfscope}%
\pgfpathrectangle{\pgfqpoint{0.765000in}{0.660000in}}{\pgfqpoint{4.620000in}{4.620000in}}%
\pgfusepath{clip}%
\pgfsetbuttcap%
\pgfsetroundjoin%
\definecolor{currentfill}{rgb}{1.000000,0.576471,0.309804}%
\pgfsetfillcolor{currentfill}%
\pgfsetfillopacity{0.050000}%
\pgfsetlinewidth{0.000000pt}%
\definecolor{currentstroke}{rgb}{1.000000,0.576471,0.309804}%
\pgfsetstrokecolor{currentstroke}%
\pgfsetdash{}{0pt}%
\pgfpathmoveto{\pgfqpoint{2.684867in}{3.374382in}}%
\pgfpathlineto{\pgfqpoint{2.762819in}{3.337786in}}%
\pgfpathlineto{\pgfqpoint{2.684867in}{3.301189in}}%
\pgfpathlineto{\pgfqpoint{2.606914in}{3.337786in}}%
\pgfpathlineto{\pgfqpoint{2.684867in}{3.374382in}}%
\pgfpathclose%
\pgfusepath{fill}%
\end{pgfscope}%
\begin{pgfscope}%
\pgfpathrectangle{\pgfqpoint{0.765000in}{0.660000in}}{\pgfqpoint{4.620000in}{4.620000in}}%
\pgfusepath{clip}%
\pgfsetbuttcap%
\pgfsetroundjoin%
\definecolor{currentfill}{rgb}{1.000000,0.576471,0.309804}%
\pgfsetfillcolor{currentfill}%
\pgfsetfillopacity{0.050000}%
\pgfsetlinewidth{0.000000pt}%
\definecolor{currentstroke}{rgb}{1.000000,0.576471,0.309804}%
\pgfsetstrokecolor{currentstroke}%
\pgfsetdash{}{0pt}%
\pgfpathmoveto{\pgfqpoint{2.684867in}{3.447385in}}%
\pgfpathlineto{\pgfqpoint{2.762819in}{3.410789in}}%
\pgfpathlineto{\pgfqpoint{2.684867in}{3.374192in}}%
\pgfpathlineto{\pgfqpoint{2.606914in}{3.410789in}}%
\pgfpathlineto{\pgfqpoint{2.684867in}{3.447385in}}%
\pgfpathclose%
\pgfusepath{fill}%
\end{pgfscope}%
\begin{pgfscope}%
\pgfpathrectangle{\pgfqpoint{0.765000in}{0.660000in}}{\pgfqpoint{4.620000in}{4.620000in}}%
\pgfusepath{clip}%
\pgfsetbuttcap%
\pgfsetroundjoin%
\definecolor{currentfill}{rgb}{1.000000,0.576471,0.309804}%
\pgfsetfillcolor{currentfill}%
\pgfsetfillopacity{0.050000}%
\pgfsetlinewidth{0.000000pt}%
\definecolor{currentstroke}{rgb}{1.000000,0.576471,0.309804}%
\pgfsetstrokecolor{currentstroke}%
\pgfsetdash{}{0pt}%
\pgfpathmoveto{\pgfqpoint{2.606914in}{3.337786in}}%
\pgfpathlineto{\pgfqpoint{2.606914in}{3.410789in}}%
\pgfpathlineto{\pgfqpoint{2.684867in}{3.374192in}}%
\pgfpathlineto{\pgfqpoint{2.684867in}{3.301189in}}%
\pgfpathlineto{\pgfqpoint{2.606914in}{3.337786in}}%
\pgfpathclose%
\pgfusepath{fill}%
\end{pgfscope}%
\begin{pgfscope}%
\pgfpathrectangle{\pgfqpoint{0.765000in}{0.660000in}}{\pgfqpoint{4.620000in}{4.620000in}}%
\pgfusepath{clip}%
\pgfsetbuttcap%
\pgfsetroundjoin%
\definecolor{currentfill}{rgb}{1.000000,0.576471,0.309804}%
\pgfsetfillcolor{currentfill}%
\pgfsetfillopacity{0.050000}%
\pgfsetlinewidth{0.000000pt}%
\definecolor{currentstroke}{rgb}{1.000000,0.576471,0.309804}%
\pgfsetstrokecolor{currentstroke}%
\pgfsetdash{}{0pt}%
\pgfpathmoveto{\pgfqpoint{2.762819in}{3.337786in}}%
\pgfpathlineto{\pgfqpoint{2.762819in}{3.410789in}}%
\pgfpathlineto{\pgfqpoint{2.684867in}{3.374192in}}%
\pgfpathlineto{\pgfqpoint{2.684867in}{3.301189in}}%
\pgfpathlineto{\pgfqpoint{2.762819in}{3.337786in}}%
\pgfpathclose%
\pgfusepath{fill}%
\end{pgfscope}%
\begin{pgfscope}%
\pgfpathrectangle{\pgfqpoint{0.765000in}{0.660000in}}{\pgfqpoint{4.620000in}{4.620000in}}%
\pgfusepath{clip}%
\pgfsetbuttcap%
\pgfsetroundjoin%
\definecolor{currentfill}{rgb}{1.000000,0.576471,0.309804}%
\pgfsetfillcolor{currentfill}%
\pgfsetfillopacity{0.050000}%
\pgfsetlinewidth{0.000000pt}%
\definecolor{currentstroke}{rgb}{1.000000,0.576471,0.309804}%
\pgfsetstrokecolor{currentstroke}%
\pgfsetdash{}{0pt}%
\pgfpathmoveto{\pgfqpoint{1.654746in}{2.274612in}}%
\pgfpathlineto{\pgfqpoint{1.654746in}{2.347615in}}%
\pgfpathlineto{\pgfqpoint{1.732699in}{2.311019in}}%
\pgfpathlineto{\pgfqpoint{1.732699in}{2.238015in}}%
\pgfpathlineto{\pgfqpoint{1.654746in}{2.274612in}}%
\pgfpathclose%
\pgfusepath{fill}%
\end{pgfscope}%
\begin{pgfscope}%
\pgfpathrectangle{\pgfqpoint{0.765000in}{0.660000in}}{\pgfqpoint{4.620000in}{4.620000in}}%
\pgfusepath{clip}%
\pgfsetbuttcap%
\pgfsetroundjoin%
\definecolor{currentfill}{rgb}{1.000000,0.576471,0.309804}%
\pgfsetfillcolor{currentfill}%
\pgfsetfillopacity{0.050000}%
\pgfsetlinewidth{0.000000pt}%
\definecolor{currentstroke}{rgb}{1.000000,0.576471,0.309804}%
\pgfsetstrokecolor{currentstroke}%
\pgfsetdash{}{0pt}%
\pgfpathmoveto{\pgfqpoint{1.654746in}{2.274612in}}%
\pgfpathlineto{\pgfqpoint{1.654746in}{2.347615in}}%
\pgfpathlineto{\pgfqpoint{1.576793in}{2.311019in}}%
\pgfpathlineto{\pgfqpoint{1.576793in}{2.238015in}}%
\pgfpathlineto{\pgfqpoint{1.654746in}{2.274612in}}%
\pgfpathclose%
\pgfusepath{fill}%
\end{pgfscope}%
\begin{pgfscope}%
\pgfpathrectangle{\pgfqpoint{0.765000in}{0.660000in}}{\pgfqpoint{4.620000in}{4.620000in}}%
\pgfusepath{clip}%
\pgfsetbuttcap%
\pgfsetroundjoin%
\definecolor{currentfill}{rgb}{1.000000,0.576471,0.309804}%
\pgfsetfillcolor{currentfill}%
\pgfsetfillopacity{0.050000}%
\pgfsetlinewidth{0.000000pt}%
\definecolor{currentstroke}{rgb}{1.000000,0.576471,0.309804}%
\pgfsetstrokecolor{currentstroke}%
\pgfsetdash{}{0pt}%
\pgfpathmoveto{\pgfqpoint{1.654746in}{2.274612in}}%
\pgfpathlineto{\pgfqpoint{1.732699in}{2.238015in}}%
\pgfpathlineto{\pgfqpoint{1.654746in}{2.201419in}}%
\pgfpathlineto{\pgfqpoint{1.576793in}{2.238015in}}%
\pgfpathlineto{\pgfqpoint{1.654746in}{2.274612in}}%
\pgfpathclose%
\pgfusepath{fill}%
\end{pgfscope}%
\begin{pgfscope}%
\pgfpathrectangle{\pgfqpoint{0.765000in}{0.660000in}}{\pgfqpoint{4.620000in}{4.620000in}}%
\pgfusepath{clip}%
\pgfsetbuttcap%
\pgfsetroundjoin%
\definecolor{currentfill}{rgb}{1.000000,0.576471,0.309804}%
\pgfsetfillcolor{currentfill}%
\pgfsetfillopacity{0.050000}%
\pgfsetlinewidth{0.000000pt}%
\definecolor{currentstroke}{rgb}{1.000000,0.576471,0.309804}%
\pgfsetstrokecolor{currentstroke}%
\pgfsetdash{}{0pt}%
\pgfpathmoveto{\pgfqpoint{1.654746in}{2.347615in}}%
\pgfpathlineto{\pgfqpoint{1.732699in}{2.311019in}}%
\pgfpathlineto{\pgfqpoint{1.654746in}{2.274422in}}%
\pgfpathlineto{\pgfqpoint{1.576793in}{2.311019in}}%
\pgfpathlineto{\pgfqpoint{1.654746in}{2.347615in}}%
\pgfpathclose%
\pgfusepath{fill}%
\end{pgfscope}%
\begin{pgfscope}%
\pgfpathrectangle{\pgfqpoint{0.765000in}{0.660000in}}{\pgfqpoint{4.620000in}{4.620000in}}%
\pgfusepath{clip}%
\pgfsetbuttcap%
\pgfsetroundjoin%
\definecolor{currentfill}{rgb}{1.000000,0.576471,0.309804}%
\pgfsetfillcolor{currentfill}%
\pgfsetfillopacity{0.050000}%
\pgfsetlinewidth{0.000000pt}%
\definecolor{currentstroke}{rgb}{1.000000,0.576471,0.309804}%
\pgfsetstrokecolor{currentstroke}%
\pgfsetdash{}{0pt}%
\pgfpathmoveto{\pgfqpoint{1.576793in}{2.238015in}}%
\pgfpathlineto{\pgfqpoint{1.576793in}{2.311019in}}%
\pgfpathlineto{\pgfqpoint{1.654746in}{2.274422in}}%
\pgfpathlineto{\pgfqpoint{1.654746in}{2.201419in}}%
\pgfpathlineto{\pgfqpoint{1.576793in}{2.238015in}}%
\pgfpathclose%
\pgfusepath{fill}%
\end{pgfscope}%
\begin{pgfscope}%
\pgfpathrectangle{\pgfqpoint{0.765000in}{0.660000in}}{\pgfqpoint{4.620000in}{4.620000in}}%
\pgfusepath{clip}%
\pgfsetbuttcap%
\pgfsetroundjoin%
\definecolor{currentfill}{rgb}{1.000000,0.576471,0.309804}%
\pgfsetfillcolor{currentfill}%
\pgfsetfillopacity{0.050000}%
\pgfsetlinewidth{0.000000pt}%
\definecolor{currentstroke}{rgb}{1.000000,0.576471,0.309804}%
\pgfsetstrokecolor{currentstroke}%
\pgfsetdash{}{0pt}%
\pgfpathmoveto{\pgfqpoint{1.732699in}{2.238015in}}%
\pgfpathlineto{\pgfqpoint{1.732699in}{2.311019in}}%
\pgfpathlineto{\pgfqpoint{1.654746in}{2.274422in}}%
\pgfpathlineto{\pgfqpoint{1.654746in}{2.201419in}}%
\pgfpathlineto{\pgfqpoint{1.732699in}{2.238015in}}%
\pgfpathclose%
\pgfusepath{fill}%
\end{pgfscope}%
\begin{pgfscope}%
\pgfpathrectangle{\pgfqpoint{0.765000in}{0.660000in}}{\pgfqpoint{4.620000in}{4.620000in}}%
\pgfusepath{clip}%
\pgfsetbuttcap%
\pgfsetroundjoin%
\definecolor{currentfill}{rgb}{1.000000,0.576471,0.309804}%
\pgfsetfillcolor{currentfill}%
\pgfsetfillopacity{0.050000}%
\pgfsetlinewidth{0.000000pt}%
\definecolor{currentstroke}{rgb}{1.000000,0.576471,0.309804}%
\pgfsetstrokecolor{currentstroke}%
\pgfsetdash{}{0pt}%
\pgfpathmoveto{\pgfqpoint{3.228014in}{3.505211in}}%
\pgfpathlineto{\pgfqpoint{3.228014in}{3.578214in}}%
\pgfpathlineto{\pgfqpoint{3.305967in}{3.541617in}}%
\pgfpathlineto{\pgfqpoint{3.305967in}{3.468614in}}%
\pgfpathlineto{\pgfqpoint{3.228014in}{3.505211in}}%
\pgfpathclose%
\pgfusepath{fill}%
\end{pgfscope}%
\begin{pgfscope}%
\pgfpathrectangle{\pgfqpoint{0.765000in}{0.660000in}}{\pgfqpoint{4.620000in}{4.620000in}}%
\pgfusepath{clip}%
\pgfsetbuttcap%
\pgfsetroundjoin%
\definecolor{currentfill}{rgb}{1.000000,0.576471,0.309804}%
\pgfsetfillcolor{currentfill}%
\pgfsetfillopacity{0.050000}%
\pgfsetlinewidth{0.000000pt}%
\definecolor{currentstroke}{rgb}{1.000000,0.576471,0.309804}%
\pgfsetstrokecolor{currentstroke}%
\pgfsetdash{}{0pt}%
\pgfpathmoveto{\pgfqpoint{3.228014in}{3.505211in}}%
\pgfpathlineto{\pgfqpoint{3.228014in}{3.578214in}}%
\pgfpathlineto{\pgfqpoint{3.150062in}{3.541617in}}%
\pgfpathlineto{\pgfqpoint{3.150062in}{3.468614in}}%
\pgfpathlineto{\pgfqpoint{3.228014in}{3.505211in}}%
\pgfpathclose%
\pgfusepath{fill}%
\end{pgfscope}%
\begin{pgfscope}%
\pgfpathrectangle{\pgfqpoint{0.765000in}{0.660000in}}{\pgfqpoint{4.620000in}{4.620000in}}%
\pgfusepath{clip}%
\pgfsetbuttcap%
\pgfsetroundjoin%
\definecolor{currentfill}{rgb}{1.000000,0.576471,0.309804}%
\pgfsetfillcolor{currentfill}%
\pgfsetfillopacity{0.050000}%
\pgfsetlinewidth{0.000000pt}%
\definecolor{currentstroke}{rgb}{1.000000,0.576471,0.309804}%
\pgfsetstrokecolor{currentstroke}%
\pgfsetdash{}{0pt}%
\pgfpathmoveto{\pgfqpoint{3.228014in}{3.505211in}}%
\pgfpathlineto{\pgfqpoint{3.305967in}{3.468614in}}%
\pgfpathlineto{\pgfqpoint{3.228014in}{3.432017in}}%
\pgfpathlineto{\pgfqpoint{3.150062in}{3.468614in}}%
\pgfpathlineto{\pgfqpoint{3.228014in}{3.505211in}}%
\pgfpathclose%
\pgfusepath{fill}%
\end{pgfscope}%
\begin{pgfscope}%
\pgfpathrectangle{\pgfqpoint{0.765000in}{0.660000in}}{\pgfqpoint{4.620000in}{4.620000in}}%
\pgfusepath{clip}%
\pgfsetbuttcap%
\pgfsetroundjoin%
\definecolor{currentfill}{rgb}{1.000000,0.576471,0.309804}%
\pgfsetfillcolor{currentfill}%
\pgfsetfillopacity{0.050000}%
\pgfsetlinewidth{0.000000pt}%
\definecolor{currentstroke}{rgb}{1.000000,0.576471,0.309804}%
\pgfsetstrokecolor{currentstroke}%
\pgfsetdash{}{0pt}%
\pgfpathmoveto{\pgfqpoint{3.228014in}{3.578214in}}%
\pgfpathlineto{\pgfqpoint{3.305967in}{3.541617in}}%
\pgfpathlineto{\pgfqpoint{3.228014in}{3.505021in}}%
\pgfpathlineto{\pgfqpoint{3.150062in}{3.541617in}}%
\pgfpathlineto{\pgfqpoint{3.228014in}{3.578214in}}%
\pgfpathclose%
\pgfusepath{fill}%
\end{pgfscope}%
\begin{pgfscope}%
\pgfpathrectangle{\pgfqpoint{0.765000in}{0.660000in}}{\pgfqpoint{4.620000in}{4.620000in}}%
\pgfusepath{clip}%
\pgfsetbuttcap%
\pgfsetroundjoin%
\definecolor{currentfill}{rgb}{1.000000,0.576471,0.309804}%
\pgfsetfillcolor{currentfill}%
\pgfsetfillopacity{0.050000}%
\pgfsetlinewidth{0.000000pt}%
\definecolor{currentstroke}{rgb}{1.000000,0.576471,0.309804}%
\pgfsetstrokecolor{currentstroke}%
\pgfsetdash{}{0pt}%
\pgfpathmoveto{\pgfqpoint{3.150062in}{3.468614in}}%
\pgfpathlineto{\pgfqpoint{3.150062in}{3.541617in}}%
\pgfpathlineto{\pgfqpoint{3.228014in}{3.505021in}}%
\pgfpathlineto{\pgfqpoint{3.228014in}{3.432017in}}%
\pgfpathlineto{\pgfqpoint{3.150062in}{3.468614in}}%
\pgfpathclose%
\pgfusepath{fill}%
\end{pgfscope}%
\begin{pgfscope}%
\pgfpathrectangle{\pgfqpoint{0.765000in}{0.660000in}}{\pgfqpoint{4.620000in}{4.620000in}}%
\pgfusepath{clip}%
\pgfsetbuttcap%
\pgfsetroundjoin%
\definecolor{currentfill}{rgb}{1.000000,0.576471,0.309804}%
\pgfsetfillcolor{currentfill}%
\pgfsetfillopacity{0.050000}%
\pgfsetlinewidth{0.000000pt}%
\definecolor{currentstroke}{rgb}{1.000000,0.576471,0.309804}%
\pgfsetstrokecolor{currentstroke}%
\pgfsetdash{}{0pt}%
\pgfpathmoveto{\pgfqpoint{3.305967in}{3.468614in}}%
\pgfpathlineto{\pgfqpoint{3.305967in}{3.541617in}}%
\pgfpathlineto{\pgfqpoint{3.228014in}{3.505021in}}%
\pgfpathlineto{\pgfqpoint{3.228014in}{3.432017in}}%
\pgfpathlineto{\pgfqpoint{3.305967in}{3.468614in}}%
\pgfpathclose%
\pgfusepath{fill}%
\end{pgfscope}%
\begin{pgfscope}%
\pgfpathrectangle{\pgfqpoint{0.765000in}{0.660000in}}{\pgfqpoint{4.620000in}{4.620000in}}%
\pgfusepath{clip}%
\pgfsetbuttcap%
\pgfsetroundjoin%
\definecolor{currentfill}{rgb}{0.176471,0.192157,0.258824}%
\pgfsetfillcolor{currentfill}%
\pgfsetfillopacity{0.050000}%
\pgfsetlinewidth{0.000000pt}%
\definecolor{currentstroke}{rgb}{0.176471,0.192157,0.258824}%
\pgfsetstrokecolor{currentstroke}%
\pgfsetdash{}{0pt}%
\pgfpathmoveto{\pgfqpoint{3.113612in}{3.034651in}}%
\pgfpathlineto{\pgfqpoint{3.113612in}{3.107654in}}%
\pgfpathlineto{\pgfqpoint{3.191564in}{3.071058in}}%
\pgfpathlineto{\pgfqpoint{3.191564in}{2.998054in}}%
\pgfpathlineto{\pgfqpoint{3.113612in}{3.034651in}}%
\pgfpathclose%
\pgfusepath{fill}%
\end{pgfscope}%
\begin{pgfscope}%
\pgfpathrectangle{\pgfqpoint{0.765000in}{0.660000in}}{\pgfqpoint{4.620000in}{4.620000in}}%
\pgfusepath{clip}%
\pgfsetbuttcap%
\pgfsetroundjoin%
\definecolor{currentfill}{rgb}{0.176471,0.192157,0.258824}%
\pgfsetfillcolor{currentfill}%
\pgfsetfillopacity{0.050000}%
\pgfsetlinewidth{0.000000pt}%
\definecolor{currentstroke}{rgb}{0.176471,0.192157,0.258824}%
\pgfsetstrokecolor{currentstroke}%
\pgfsetdash{}{0pt}%
\pgfpathmoveto{\pgfqpoint{3.113612in}{3.034651in}}%
\pgfpathlineto{\pgfqpoint{3.113612in}{3.107654in}}%
\pgfpathlineto{\pgfqpoint{3.035659in}{3.071058in}}%
\pgfpathlineto{\pgfqpoint{3.035659in}{2.998054in}}%
\pgfpathlineto{\pgfqpoint{3.113612in}{3.034651in}}%
\pgfpathclose%
\pgfusepath{fill}%
\end{pgfscope}%
\begin{pgfscope}%
\pgfpathrectangle{\pgfqpoint{0.765000in}{0.660000in}}{\pgfqpoint{4.620000in}{4.620000in}}%
\pgfusepath{clip}%
\pgfsetbuttcap%
\pgfsetroundjoin%
\definecolor{currentfill}{rgb}{0.176471,0.192157,0.258824}%
\pgfsetfillcolor{currentfill}%
\pgfsetfillopacity{0.050000}%
\pgfsetlinewidth{0.000000pt}%
\definecolor{currentstroke}{rgb}{0.176471,0.192157,0.258824}%
\pgfsetstrokecolor{currentstroke}%
\pgfsetdash{}{0pt}%
\pgfpathmoveto{\pgfqpoint{3.113612in}{3.034651in}}%
\pgfpathlineto{\pgfqpoint{3.191564in}{2.998054in}}%
\pgfpathlineto{\pgfqpoint{3.113612in}{2.961458in}}%
\pgfpathlineto{\pgfqpoint{3.035659in}{2.998054in}}%
\pgfpathlineto{\pgfqpoint{3.113612in}{3.034651in}}%
\pgfpathclose%
\pgfusepath{fill}%
\end{pgfscope}%
\begin{pgfscope}%
\pgfpathrectangle{\pgfqpoint{0.765000in}{0.660000in}}{\pgfqpoint{4.620000in}{4.620000in}}%
\pgfusepath{clip}%
\pgfsetbuttcap%
\pgfsetroundjoin%
\definecolor{currentfill}{rgb}{0.176471,0.192157,0.258824}%
\pgfsetfillcolor{currentfill}%
\pgfsetfillopacity{0.050000}%
\pgfsetlinewidth{0.000000pt}%
\definecolor{currentstroke}{rgb}{0.176471,0.192157,0.258824}%
\pgfsetstrokecolor{currentstroke}%
\pgfsetdash{}{0pt}%
\pgfpathmoveto{\pgfqpoint{3.113612in}{3.107654in}}%
\pgfpathlineto{\pgfqpoint{3.191564in}{3.071058in}}%
\pgfpathlineto{\pgfqpoint{3.113612in}{3.034461in}}%
\pgfpathlineto{\pgfqpoint{3.035659in}{3.071058in}}%
\pgfpathlineto{\pgfqpoint{3.113612in}{3.107654in}}%
\pgfpathclose%
\pgfusepath{fill}%
\end{pgfscope}%
\begin{pgfscope}%
\pgfpathrectangle{\pgfqpoint{0.765000in}{0.660000in}}{\pgfqpoint{4.620000in}{4.620000in}}%
\pgfusepath{clip}%
\pgfsetbuttcap%
\pgfsetroundjoin%
\definecolor{currentfill}{rgb}{0.176471,0.192157,0.258824}%
\pgfsetfillcolor{currentfill}%
\pgfsetfillopacity{0.050000}%
\pgfsetlinewidth{0.000000pt}%
\definecolor{currentstroke}{rgb}{0.176471,0.192157,0.258824}%
\pgfsetstrokecolor{currentstroke}%
\pgfsetdash{}{0pt}%
\pgfpathmoveto{\pgfqpoint{3.035659in}{2.998054in}}%
\pgfpathlineto{\pgfqpoint{3.035659in}{3.071058in}}%
\pgfpathlineto{\pgfqpoint{3.113612in}{3.034461in}}%
\pgfpathlineto{\pgfqpoint{3.113612in}{2.961458in}}%
\pgfpathlineto{\pgfqpoint{3.035659in}{2.998054in}}%
\pgfpathclose%
\pgfusepath{fill}%
\end{pgfscope}%
\begin{pgfscope}%
\pgfpathrectangle{\pgfqpoint{0.765000in}{0.660000in}}{\pgfqpoint{4.620000in}{4.620000in}}%
\pgfusepath{clip}%
\pgfsetbuttcap%
\pgfsetroundjoin%
\definecolor{currentfill}{rgb}{0.176471,0.192157,0.258824}%
\pgfsetfillcolor{currentfill}%
\pgfsetfillopacity{0.050000}%
\pgfsetlinewidth{0.000000pt}%
\definecolor{currentstroke}{rgb}{0.176471,0.192157,0.258824}%
\pgfsetstrokecolor{currentstroke}%
\pgfsetdash{}{0pt}%
\pgfpathmoveto{\pgfqpoint{3.191564in}{2.998054in}}%
\pgfpathlineto{\pgfqpoint{3.191564in}{3.071058in}}%
\pgfpathlineto{\pgfqpoint{3.113612in}{3.034461in}}%
\pgfpathlineto{\pgfqpoint{3.113612in}{2.961458in}}%
\pgfpathlineto{\pgfqpoint{3.191564in}{2.998054in}}%
\pgfpathclose%
\pgfusepath{fill}%
\end{pgfscope}%
\begin{pgfscope}%
\pgfpathrectangle{\pgfqpoint{0.765000in}{0.660000in}}{\pgfqpoint{4.620000in}{4.620000in}}%
\pgfusepath{clip}%
\pgfsetbuttcap%
\pgfsetroundjoin%
\definecolor{currentfill}{rgb}{0.176471,0.192157,0.258824}%
\pgfsetfillcolor{currentfill}%
\pgfsetfillopacity{0.050000}%
\pgfsetlinewidth{0.000000pt}%
\definecolor{currentstroke}{rgb}{0.176471,0.192157,0.258824}%
\pgfsetstrokecolor{currentstroke}%
\pgfsetdash{}{0pt}%
\pgfpathmoveto{\pgfqpoint{3.071425in}{2.926315in}}%
\pgfpathlineto{\pgfqpoint{3.071425in}{2.999319in}}%
\pgfpathlineto{\pgfqpoint{3.149378in}{2.962722in}}%
\pgfpathlineto{\pgfqpoint{3.149378in}{2.889719in}}%
\pgfpathlineto{\pgfqpoint{3.071425in}{2.926315in}}%
\pgfpathclose%
\pgfusepath{fill}%
\end{pgfscope}%
\begin{pgfscope}%
\pgfpathrectangle{\pgfqpoint{0.765000in}{0.660000in}}{\pgfqpoint{4.620000in}{4.620000in}}%
\pgfusepath{clip}%
\pgfsetbuttcap%
\pgfsetroundjoin%
\definecolor{currentfill}{rgb}{0.176471,0.192157,0.258824}%
\pgfsetfillcolor{currentfill}%
\pgfsetfillopacity{0.050000}%
\pgfsetlinewidth{0.000000pt}%
\definecolor{currentstroke}{rgb}{0.176471,0.192157,0.258824}%
\pgfsetstrokecolor{currentstroke}%
\pgfsetdash{}{0pt}%
\pgfpathmoveto{\pgfqpoint{3.071425in}{2.926315in}}%
\pgfpathlineto{\pgfqpoint{3.071425in}{2.999319in}}%
\pgfpathlineto{\pgfqpoint{2.993473in}{2.962722in}}%
\pgfpathlineto{\pgfqpoint{2.993473in}{2.889719in}}%
\pgfpathlineto{\pgfqpoint{3.071425in}{2.926315in}}%
\pgfpathclose%
\pgfusepath{fill}%
\end{pgfscope}%
\begin{pgfscope}%
\pgfpathrectangle{\pgfqpoint{0.765000in}{0.660000in}}{\pgfqpoint{4.620000in}{4.620000in}}%
\pgfusepath{clip}%
\pgfsetbuttcap%
\pgfsetroundjoin%
\definecolor{currentfill}{rgb}{0.176471,0.192157,0.258824}%
\pgfsetfillcolor{currentfill}%
\pgfsetfillopacity{0.050000}%
\pgfsetlinewidth{0.000000pt}%
\definecolor{currentstroke}{rgb}{0.176471,0.192157,0.258824}%
\pgfsetstrokecolor{currentstroke}%
\pgfsetdash{}{0pt}%
\pgfpathmoveto{\pgfqpoint{3.071425in}{2.926315in}}%
\pgfpathlineto{\pgfqpoint{3.149378in}{2.889719in}}%
\pgfpathlineto{\pgfqpoint{3.071425in}{2.853122in}}%
\pgfpathlineto{\pgfqpoint{2.993473in}{2.889719in}}%
\pgfpathlineto{\pgfqpoint{3.071425in}{2.926315in}}%
\pgfpathclose%
\pgfusepath{fill}%
\end{pgfscope}%
\begin{pgfscope}%
\pgfpathrectangle{\pgfqpoint{0.765000in}{0.660000in}}{\pgfqpoint{4.620000in}{4.620000in}}%
\pgfusepath{clip}%
\pgfsetbuttcap%
\pgfsetroundjoin%
\definecolor{currentfill}{rgb}{0.176471,0.192157,0.258824}%
\pgfsetfillcolor{currentfill}%
\pgfsetfillopacity{0.050000}%
\pgfsetlinewidth{0.000000pt}%
\definecolor{currentstroke}{rgb}{0.176471,0.192157,0.258824}%
\pgfsetstrokecolor{currentstroke}%
\pgfsetdash{}{0pt}%
\pgfpathmoveto{\pgfqpoint{3.071425in}{2.999319in}}%
\pgfpathlineto{\pgfqpoint{3.149378in}{2.962722in}}%
\pgfpathlineto{\pgfqpoint{3.071425in}{2.926125in}}%
\pgfpathlineto{\pgfqpoint{2.993473in}{2.962722in}}%
\pgfpathlineto{\pgfqpoint{3.071425in}{2.999319in}}%
\pgfpathclose%
\pgfusepath{fill}%
\end{pgfscope}%
\begin{pgfscope}%
\pgfpathrectangle{\pgfqpoint{0.765000in}{0.660000in}}{\pgfqpoint{4.620000in}{4.620000in}}%
\pgfusepath{clip}%
\pgfsetbuttcap%
\pgfsetroundjoin%
\definecolor{currentfill}{rgb}{0.176471,0.192157,0.258824}%
\pgfsetfillcolor{currentfill}%
\pgfsetfillopacity{0.050000}%
\pgfsetlinewidth{0.000000pt}%
\definecolor{currentstroke}{rgb}{0.176471,0.192157,0.258824}%
\pgfsetstrokecolor{currentstroke}%
\pgfsetdash{}{0pt}%
\pgfpathmoveto{\pgfqpoint{2.993473in}{2.889719in}}%
\pgfpathlineto{\pgfqpoint{2.993473in}{2.962722in}}%
\pgfpathlineto{\pgfqpoint{3.071425in}{2.926125in}}%
\pgfpathlineto{\pgfqpoint{3.071425in}{2.853122in}}%
\pgfpathlineto{\pgfqpoint{2.993473in}{2.889719in}}%
\pgfpathclose%
\pgfusepath{fill}%
\end{pgfscope}%
\begin{pgfscope}%
\pgfpathrectangle{\pgfqpoint{0.765000in}{0.660000in}}{\pgfqpoint{4.620000in}{4.620000in}}%
\pgfusepath{clip}%
\pgfsetbuttcap%
\pgfsetroundjoin%
\definecolor{currentfill}{rgb}{0.176471,0.192157,0.258824}%
\pgfsetfillcolor{currentfill}%
\pgfsetfillopacity{0.050000}%
\pgfsetlinewidth{0.000000pt}%
\definecolor{currentstroke}{rgb}{0.176471,0.192157,0.258824}%
\pgfsetstrokecolor{currentstroke}%
\pgfsetdash{}{0pt}%
\pgfpathmoveto{\pgfqpoint{3.149378in}{2.889719in}}%
\pgfpathlineto{\pgfqpoint{3.149378in}{2.962722in}}%
\pgfpathlineto{\pgfqpoint{3.071425in}{2.926125in}}%
\pgfpathlineto{\pgfqpoint{3.071425in}{2.853122in}}%
\pgfpathlineto{\pgfqpoint{3.149378in}{2.889719in}}%
\pgfpathclose%
\pgfusepath{fill}%
\end{pgfscope}%
\begin{pgfscope}%
\pgfpathrectangle{\pgfqpoint{0.765000in}{0.660000in}}{\pgfqpoint{4.620000in}{4.620000in}}%
\pgfusepath{clip}%
\pgfsetbuttcap%
\pgfsetroundjoin%
\definecolor{currentfill}{rgb}{1.000000,0.576471,0.309804}%
\pgfsetfillcolor{currentfill}%
\pgfsetfillopacity{0.050000}%
\pgfsetlinewidth{0.000000pt}%
\definecolor{currentstroke}{rgb}{1.000000,0.576471,0.309804}%
\pgfsetstrokecolor{currentstroke}%
\pgfsetdash{}{0pt}%
\pgfpathmoveto{\pgfqpoint{2.005584in}{2.612068in}}%
\pgfpathlineto{\pgfqpoint{2.005584in}{2.685072in}}%
\pgfpathlineto{\pgfqpoint{2.083537in}{2.648475in}}%
\pgfpathlineto{\pgfqpoint{2.083537in}{2.575472in}}%
\pgfpathlineto{\pgfqpoint{2.005584in}{2.612068in}}%
\pgfpathclose%
\pgfusepath{fill}%
\end{pgfscope}%
\begin{pgfscope}%
\pgfpathrectangle{\pgfqpoint{0.765000in}{0.660000in}}{\pgfqpoint{4.620000in}{4.620000in}}%
\pgfusepath{clip}%
\pgfsetbuttcap%
\pgfsetroundjoin%
\definecolor{currentfill}{rgb}{1.000000,0.576471,0.309804}%
\pgfsetfillcolor{currentfill}%
\pgfsetfillopacity{0.050000}%
\pgfsetlinewidth{0.000000pt}%
\definecolor{currentstroke}{rgb}{1.000000,0.576471,0.309804}%
\pgfsetstrokecolor{currentstroke}%
\pgfsetdash{}{0pt}%
\pgfpathmoveto{\pgfqpoint{2.005584in}{2.612068in}}%
\pgfpathlineto{\pgfqpoint{2.005584in}{2.685072in}}%
\pgfpathlineto{\pgfqpoint{1.927631in}{2.648475in}}%
\pgfpathlineto{\pgfqpoint{1.927631in}{2.575472in}}%
\pgfpathlineto{\pgfqpoint{2.005584in}{2.612068in}}%
\pgfpathclose%
\pgfusepath{fill}%
\end{pgfscope}%
\begin{pgfscope}%
\pgfpathrectangle{\pgfqpoint{0.765000in}{0.660000in}}{\pgfqpoint{4.620000in}{4.620000in}}%
\pgfusepath{clip}%
\pgfsetbuttcap%
\pgfsetroundjoin%
\definecolor{currentfill}{rgb}{1.000000,0.576471,0.309804}%
\pgfsetfillcolor{currentfill}%
\pgfsetfillopacity{0.050000}%
\pgfsetlinewidth{0.000000pt}%
\definecolor{currentstroke}{rgb}{1.000000,0.576471,0.309804}%
\pgfsetstrokecolor{currentstroke}%
\pgfsetdash{}{0pt}%
\pgfpathmoveto{\pgfqpoint{2.005584in}{2.612068in}}%
\pgfpathlineto{\pgfqpoint{2.083537in}{2.575472in}}%
\pgfpathlineto{\pgfqpoint{2.005584in}{2.538875in}}%
\pgfpathlineto{\pgfqpoint{1.927631in}{2.575472in}}%
\pgfpathlineto{\pgfqpoint{2.005584in}{2.612068in}}%
\pgfpathclose%
\pgfusepath{fill}%
\end{pgfscope}%
\begin{pgfscope}%
\pgfpathrectangle{\pgfqpoint{0.765000in}{0.660000in}}{\pgfqpoint{4.620000in}{4.620000in}}%
\pgfusepath{clip}%
\pgfsetbuttcap%
\pgfsetroundjoin%
\definecolor{currentfill}{rgb}{1.000000,0.576471,0.309804}%
\pgfsetfillcolor{currentfill}%
\pgfsetfillopacity{0.050000}%
\pgfsetlinewidth{0.000000pt}%
\definecolor{currentstroke}{rgb}{1.000000,0.576471,0.309804}%
\pgfsetstrokecolor{currentstroke}%
\pgfsetdash{}{0pt}%
\pgfpathmoveto{\pgfqpoint{2.005584in}{2.685072in}}%
\pgfpathlineto{\pgfqpoint{2.083537in}{2.648475in}}%
\pgfpathlineto{\pgfqpoint{2.005584in}{2.611878in}}%
\pgfpathlineto{\pgfqpoint{1.927631in}{2.648475in}}%
\pgfpathlineto{\pgfqpoint{2.005584in}{2.685072in}}%
\pgfpathclose%
\pgfusepath{fill}%
\end{pgfscope}%
\begin{pgfscope}%
\pgfpathrectangle{\pgfqpoint{0.765000in}{0.660000in}}{\pgfqpoint{4.620000in}{4.620000in}}%
\pgfusepath{clip}%
\pgfsetbuttcap%
\pgfsetroundjoin%
\definecolor{currentfill}{rgb}{1.000000,0.576471,0.309804}%
\pgfsetfillcolor{currentfill}%
\pgfsetfillopacity{0.050000}%
\pgfsetlinewidth{0.000000pt}%
\definecolor{currentstroke}{rgb}{1.000000,0.576471,0.309804}%
\pgfsetstrokecolor{currentstroke}%
\pgfsetdash{}{0pt}%
\pgfpathmoveto{\pgfqpoint{1.927631in}{2.575472in}}%
\pgfpathlineto{\pgfqpoint{1.927631in}{2.648475in}}%
\pgfpathlineto{\pgfqpoint{2.005584in}{2.611878in}}%
\pgfpathlineto{\pgfqpoint{2.005584in}{2.538875in}}%
\pgfpathlineto{\pgfqpoint{1.927631in}{2.575472in}}%
\pgfpathclose%
\pgfusepath{fill}%
\end{pgfscope}%
\begin{pgfscope}%
\pgfpathrectangle{\pgfqpoint{0.765000in}{0.660000in}}{\pgfqpoint{4.620000in}{4.620000in}}%
\pgfusepath{clip}%
\pgfsetbuttcap%
\pgfsetroundjoin%
\definecolor{currentfill}{rgb}{1.000000,0.576471,0.309804}%
\pgfsetfillcolor{currentfill}%
\pgfsetfillopacity{0.050000}%
\pgfsetlinewidth{0.000000pt}%
\definecolor{currentstroke}{rgb}{1.000000,0.576471,0.309804}%
\pgfsetstrokecolor{currentstroke}%
\pgfsetdash{}{0pt}%
\pgfpathmoveto{\pgfqpoint{2.083537in}{2.575472in}}%
\pgfpathlineto{\pgfqpoint{2.083537in}{2.648475in}}%
\pgfpathlineto{\pgfqpoint{2.005584in}{2.611878in}}%
\pgfpathlineto{\pgfqpoint{2.005584in}{2.538875in}}%
\pgfpathlineto{\pgfqpoint{2.083537in}{2.575472in}}%
\pgfpathclose%
\pgfusepath{fill}%
\end{pgfscope}%
\begin{pgfscope}%
\pgfpathrectangle{\pgfqpoint{0.765000in}{0.660000in}}{\pgfqpoint{4.620000in}{4.620000in}}%
\pgfusepath{clip}%
\pgfsetbuttcap%
\pgfsetroundjoin%
\definecolor{currentfill}{rgb}{1.000000,0.576471,0.309804}%
\pgfsetfillcolor{currentfill}%
\pgfsetfillopacity{0.050000}%
\pgfsetlinewidth{0.000000pt}%
\definecolor{currentstroke}{rgb}{1.000000,0.576471,0.309804}%
\pgfsetstrokecolor{currentstroke}%
\pgfsetdash{}{0pt}%
\pgfpathmoveto{\pgfqpoint{4.264167in}{2.841197in}}%
\pgfpathlineto{\pgfqpoint{4.264167in}{2.914201in}}%
\pgfpathlineto{\pgfqpoint{4.342120in}{2.877604in}}%
\pgfpathlineto{\pgfqpoint{4.342120in}{2.804601in}}%
\pgfpathlineto{\pgfqpoint{4.264167in}{2.841197in}}%
\pgfpathclose%
\pgfusepath{fill}%
\end{pgfscope}%
\begin{pgfscope}%
\pgfpathrectangle{\pgfqpoint{0.765000in}{0.660000in}}{\pgfqpoint{4.620000in}{4.620000in}}%
\pgfusepath{clip}%
\pgfsetbuttcap%
\pgfsetroundjoin%
\definecolor{currentfill}{rgb}{1.000000,0.576471,0.309804}%
\pgfsetfillcolor{currentfill}%
\pgfsetfillopacity{0.050000}%
\pgfsetlinewidth{0.000000pt}%
\definecolor{currentstroke}{rgb}{1.000000,0.576471,0.309804}%
\pgfsetstrokecolor{currentstroke}%
\pgfsetdash{}{0pt}%
\pgfpathmoveto{\pgfqpoint{4.264167in}{2.841197in}}%
\pgfpathlineto{\pgfqpoint{4.264167in}{2.914201in}}%
\pgfpathlineto{\pgfqpoint{4.186214in}{2.877604in}}%
\pgfpathlineto{\pgfqpoint{4.186214in}{2.804601in}}%
\pgfpathlineto{\pgfqpoint{4.264167in}{2.841197in}}%
\pgfpathclose%
\pgfusepath{fill}%
\end{pgfscope}%
\begin{pgfscope}%
\pgfpathrectangle{\pgfqpoint{0.765000in}{0.660000in}}{\pgfqpoint{4.620000in}{4.620000in}}%
\pgfusepath{clip}%
\pgfsetbuttcap%
\pgfsetroundjoin%
\definecolor{currentfill}{rgb}{1.000000,0.576471,0.309804}%
\pgfsetfillcolor{currentfill}%
\pgfsetfillopacity{0.050000}%
\pgfsetlinewidth{0.000000pt}%
\definecolor{currentstroke}{rgb}{1.000000,0.576471,0.309804}%
\pgfsetstrokecolor{currentstroke}%
\pgfsetdash{}{0pt}%
\pgfpathmoveto{\pgfqpoint{4.264167in}{2.841197in}}%
\pgfpathlineto{\pgfqpoint{4.342120in}{2.804601in}}%
\pgfpathlineto{\pgfqpoint{4.264167in}{2.768004in}}%
\pgfpathlineto{\pgfqpoint{4.186214in}{2.804601in}}%
\pgfpathlineto{\pgfqpoint{4.264167in}{2.841197in}}%
\pgfpathclose%
\pgfusepath{fill}%
\end{pgfscope}%
\begin{pgfscope}%
\pgfpathrectangle{\pgfqpoint{0.765000in}{0.660000in}}{\pgfqpoint{4.620000in}{4.620000in}}%
\pgfusepath{clip}%
\pgfsetbuttcap%
\pgfsetroundjoin%
\definecolor{currentfill}{rgb}{1.000000,0.576471,0.309804}%
\pgfsetfillcolor{currentfill}%
\pgfsetfillopacity{0.050000}%
\pgfsetlinewidth{0.000000pt}%
\definecolor{currentstroke}{rgb}{1.000000,0.576471,0.309804}%
\pgfsetstrokecolor{currentstroke}%
\pgfsetdash{}{0pt}%
\pgfpathmoveto{\pgfqpoint{4.264167in}{2.914201in}}%
\pgfpathlineto{\pgfqpoint{4.342120in}{2.877604in}}%
\pgfpathlineto{\pgfqpoint{4.264167in}{2.841007in}}%
\pgfpathlineto{\pgfqpoint{4.186214in}{2.877604in}}%
\pgfpathlineto{\pgfqpoint{4.264167in}{2.914201in}}%
\pgfpathclose%
\pgfusepath{fill}%
\end{pgfscope}%
\begin{pgfscope}%
\pgfpathrectangle{\pgfqpoint{0.765000in}{0.660000in}}{\pgfqpoint{4.620000in}{4.620000in}}%
\pgfusepath{clip}%
\pgfsetbuttcap%
\pgfsetroundjoin%
\definecolor{currentfill}{rgb}{1.000000,0.576471,0.309804}%
\pgfsetfillcolor{currentfill}%
\pgfsetfillopacity{0.050000}%
\pgfsetlinewidth{0.000000pt}%
\definecolor{currentstroke}{rgb}{1.000000,0.576471,0.309804}%
\pgfsetstrokecolor{currentstroke}%
\pgfsetdash{}{0pt}%
\pgfpathmoveto{\pgfqpoint{4.186214in}{2.804601in}}%
\pgfpathlineto{\pgfqpoint{4.186214in}{2.877604in}}%
\pgfpathlineto{\pgfqpoint{4.264167in}{2.841007in}}%
\pgfpathlineto{\pgfqpoint{4.264167in}{2.768004in}}%
\pgfpathlineto{\pgfqpoint{4.186214in}{2.804601in}}%
\pgfpathclose%
\pgfusepath{fill}%
\end{pgfscope}%
\begin{pgfscope}%
\pgfpathrectangle{\pgfqpoint{0.765000in}{0.660000in}}{\pgfqpoint{4.620000in}{4.620000in}}%
\pgfusepath{clip}%
\pgfsetbuttcap%
\pgfsetroundjoin%
\definecolor{currentfill}{rgb}{1.000000,0.576471,0.309804}%
\pgfsetfillcolor{currentfill}%
\pgfsetfillopacity{0.050000}%
\pgfsetlinewidth{0.000000pt}%
\definecolor{currentstroke}{rgb}{1.000000,0.576471,0.309804}%
\pgfsetstrokecolor{currentstroke}%
\pgfsetdash{}{0pt}%
\pgfpathmoveto{\pgfqpoint{4.342120in}{2.804601in}}%
\pgfpathlineto{\pgfqpoint{4.342120in}{2.877604in}}%
\pgfpathlineto{\pgfqpoint{4.264167in}{2.841007in}}%
\pgfpathlineto{\pgfqpoint{4.264167in}{2.768004in}}%
\pgfpathlineto{\pgfqpoint{4.342120in}{2.804601in}}%
\pgfpathclose%
\pgfusepath{fill}%
\end{pgfscope}%
\begin{pgfscope}%
\pgfpathrectangle{\pgfqpoint{0.765000in}{0.660000in}}{\pgfqpoint{4.620000in}{4.620000in}}%
\pgfusepath{clip}%
\pgfsetbuttcap%
\pgfsetroundjoin%
\definecolor{currentfill}{rgb}{1.000000,0.576471,0.309804}%
\pgfsetfillcolor{currentfill}%
\pgfsetfillopacity{0.050000}%
\pgfsetlinewidth{0.000000pt}%
\definecolor{currentstroke}{rgb}{1.000000,0.576471,0.309804}%
\pgfsetstrokecolor{currentstroke}%
\pgfsetdash{}{0pt}%
\pgfpathmoveto{\pgfqpoint{4.051159in}{2.800067in}}%
\pgfpathlineto{\pgfqpoint{4.051159in}{2.873070in}}%
\pgfpathlineto{\pgfqpoint{4.129112in}{2.836474in}}%
\pgfpathlineto{\pgfqpoint{4.129112in}{2.763471in}}%
\pgfpathlineto{\pgfqpoint{4.051159in}{2.800067in}}%
\pgfpathclose%
\pgfusepath{fill}%
\end{pgfscope}%
\begin{pgfscope}%
\pgfpathrectangle{\pgfqpoint{0.765000in}{0.660000in}}{\pgfqpoint{4.620000in}{4.620000in}}%
\pgfusepath{clip}%
\pgfsetbuttcap%
\pgfsetroundjoin%
\definecolor{currentfill}{rgb}{1.000000,0.576471,0.309804}%
\pgfsetfillcolor{currentfill}%
\pgfsetfillopacity{0.050000}%
\pgfsetlinewidth{0.000000pt}%
\definecolor{currentstroke}{rgb}{1.000000,0.576471,0.309804}%
\pgfsetstrokecolor{currentstroke}%
\pgfsetdash{}{0pt}%
\pgfpathmoveto{\pgfqpoint{4.051159in}{2.800067in}}%
\pgfpathlineto{\pgfqpoint{4.051159in}{2.873070in}}%
\pgfpathlineto{\pgfqpoint{3.973206in}{2.836474in}}%
\pgfpathlineto{\pgfqpoint{3.973206in}{2.763471in}}%
\pgfpathlineto{\pgfqpoint{4.051159in}{2.800067in}}%
\pgfpathclose%
\pgfusepath{fill}%
\end{pgfscope}%
\begin{pgfscope}%
\pgfpathrectangle{\pgfqpoint{0.765000in}{0.660000in}}{\pgfqpoint{4.620000in}{4.620000in}}%
\pgfusepath{clip}%
\pgfsetbuttcap%
\pgfsetroundjoin%
\definecolor{currentfill}{rgb}{1.000000,0.576471,0.309804}%
\pgfsetfillcolor{currentfill}%
\pgfsetfillopacity{0.050000}%
\pgfsetlinewidth{0.000000pt}%
\definecolor{currentstroke}{rgb}{1.000000,0.576471,0.309804}%
\pgfsetstrokecolor{currentstroke}%
\pgfsetdash{}{0pt}%
\pgfpathmoveto{\pgfqpoint{4.051159in}{2.800067in}}%
\pgfpathlineto{\pgfqpoint{4.129112in}{2.763471in}}%
\pgfpathlineto{\pgfqpoint{4.051159in}{2.726874in}}%
\pgfpathlineto{\pgfqpoint{3.973206in}{2.763471in}}%
\pgfpathlineto{\pgfqpoint{4.051159in}{2.800067in}}%
\pgfpathclose%
\pgfusepath{fill}%
\end{pgfscope}%
\begin{pgfscope}%
\pgfpathrectangle{\pgfqpoint{0.765000in}{0.660000in}}{\pgfqpoint{4.620000in}{4.620000in}}%
\pgfusepath{clip}%
\pgfsetbuttcap%
\pgfsetroundjoin%
\definecolor{currentfill}{rgb}{1.000000,0.576471,0.309804}%
\pgfsetfillcolor{currentfill}%
\pgfsetfillopacity{0.050000}%
\pgfsetlinewidth{0.000000pt}%
\definecolor{currentstroke}{rgb}{1.000000,0.576471,0.309804}%
\pgfsetstrokecolor{currentstroke}%
\pgfsetdash{}{0pt}%
\pgfpathmoveto{\pgfqpoint{4.051159in}{2.873070in}}%
\pgfpathlineto{\pgfqpoint{4.129112in}{2.836474in}}%
\pgfpathlineto{\pgfqpoint{4.051159in}{2.799877in}}%
\pgfpathlineto{\pgfqpoint{3.973206in}{2.836474in}}%
\pgfpathlineto{\pgfqpoint{4.051159in}{2.873070in}}%
\pgfpathclose%
\pgfusepath{fill}%
\end{pgfscope}%
\begin{pgfscope}%
\pgfpathrectangle{\pgfqpoint{0.765000in}{0.660000in}}{\pgfqpoint{4.620000in}{4.620000in}}%
\pgfusepath{clip}%
\pgfsetbuttcap%
\pgfsetroundjoin%
\definecolor{currentfill}{rgb}{1.000000,0.576471,0.309804}%
\pgfsetfillcolor{currentfill}%
\pgfsetfillopacity{0.050000}%
\pgfsetlinewidth{0.000000pt}%
\definecolor{currentstroke}{rgb}{1.000000,0.576471,0.309804}%
\pgfsetstrokecolor{currentstroke}%
\pgfsetdash{}{0pt}%
\pgfpathmoveto{\pgfqpoint{3.973206in}{2.763471in}}%
\pgfpathlineto{\pgfqpoint{3.973206in}{2.836474in}}%
\pgfpathlineto{\pgfqpoint{4.051159in}{2.799877in}}%
\pgfpathlineto{\pgfqpoint{4.051159in}{2.726874in}}%
\pgfpathlineto{\pgfqpoint{3.973206in}{2.763471in}}%
\pgfpathclose%
\pgfusepath{fill}%
\end{pgfscope}%
\begin{pgfscope}%
\pgfpathrectangle{\pgfqpoint{0.765000in}{0.660000in}}{\pgfqpoint{4.620000in}{4.620000in}}%
\pgfusepath{clip}%
\pgfsetbuttcap%
\pgfsetroundjoin%
\definecolor{currentfill}{rgb}{1.000000,0.576471,0.309804}%
\pgfsetfillcolor{currentfill}%
\pgfsetfillopacity{0.050000}%
\pgfsetlinewidth{0.000000pt}%
\definecolor{currentstroke}{rgb}{1.000000,0.576471,0.309804}%
\pgfsetstrokecolor{currentstroke}%
\pgfsetdash{}{0pt}%
\pgfpathmoveto{\pgfqpoint{4.129112in}{2.763471in}}%
\pgfpathlineto{\pgfqpoint{4.129112in}{2.836474in}}%
\pgfpathlineto{\pgfqpoint{4.051159in}{2.799877in}}%
\pgfpathlineto{\pgfqpoint{4.051159in}{2.726874in}}%
\pgfpathlineto{\pgfqpoint{4.129112in}{2.763471in}}%
\pgfpathclose%
\pgfusepath{fill}%
\end{pgfscope}%
\begin{pgfscope}%
\pgfpathrectangle{\pgfqpoint{0.765000in}{0.660000in}}{\pgfqpoint{4.620000in}{4.620000in}}%
\pgfusepath{clip}%
\pgfsetbuttcap%
\pgfsetroundjoin%
\definecolor{currentfill}{rgb}{0.176471,0.192157,0.258824}%
\pgfsetfillcolor{currentfill}%
\pgfsetfillopacity{0.050000}%
\pgfsetlinewidth{0.000000pt}%
\definecolor{currentstroke}{rgb}{0.176471,0.192157,0.258824}%
\pgfsetstrokecolor{currentstroke}%
\pgfsetdash{}{0pt}%
\pgfpathmoveto{\pgfqpoint{3.045111in}{2.990417in}}%
\pgfpathlineto{\pgfqpoint{3.045111in}{3.063421in}}%
\pgfpathlineto{\pgfqpoint{3.123063in}{3.026824in}}%
\pgfpathlineto{\pgfqpoint{3.123063in}{2.953821in}}%
\pgfpathlineto{\pgfqpoint{3.045111in}{2.990417in}}%
\pgfpathclose%
\pgfusepath{fill}%
\end{pgfscope}%
\begin{pgfscope}%
\pgfpathrectangle{\pgfqpoint{0.765000in}{0.660000in}}{\pgfqpoint{4.620000in}{4.620000in}}%
\pgfusepath{clip}%
\pgfsetbuttcap%
\pgfsetroundjoin%
\definecolor{currentfill}{rgb}{0.176471,0.192157,0.258824}%
\pgfsetfillcolor{currentfill}%
\pgfsetfillopacity{0.050000}%
\pgfsetlinewidth{0.000000pt}%
\definecolor{currentstroke}{rgb}{0.176471,0.192157,0.258824}%
\pgfsetstrokecolor{currentstroke}%
\pgfsetdash{}{0pt}%
\pgfpathmoveto{\pgfqpoint{3.045111in}{2.990417in}}%
\pgfpathlineto{\pgfqpoint{3.045111in}{3.063421in}}%
\pgfpathlineto{\pgfqpoint{2.967158in}{3.026824in}}%
\pgfpathlineto{\pgfqpoint{2.967158in}{2.953821in}}%
\pgfpathlineto{\pgfqpoint{3.045111in}{2.990417in}}%
\pgfpathclose%
\pgfusepath{fill}%
\end{pgfscope}%
\begin{pgfscope}%
\pgfpathrectangle{\pgfqpoint{0.765000in}{0.660000in}}{\pgfqpoint{4.620000in}{4.620000in}}%
\pgfusepath{clip}%
\pgfsetbuttcap%
\pgfsetroundjoin%
\definecolor{currentfill}{rgb}{0.176471,0.192157,0.258824}%
\pgfsetfillcolor{currentfill}%
\pgfsetfillopacity{0.050000}%
\pgfsetlinewidth{0.000000pt}%
\definecolor{currentstroke}{rgb}{0.176471,0.192157,0.258824}%
\pgfsetstrokecolor{currentstroke}%
\pgfsetdash{}{0pt}%
\pgfpathmoveto{\pgfqpoint{3.045111in}{2.990417in}}%
\pgfpathlineto{\pgfqpoint{3.123063in}{2.953821in}}%
\pgfpathlineto{\pgfqpoint{3.045111in}{2.917224in}}%
\pgfpathlineto{\pgfqpoint{2.967158in}{2.953821in}}%
\pgfpathlineto{\pgfqpoint{3.045111in}{2.990417in}}%
\pgfpathclose%
\pgfusepath{fill}%
\end{pgfscope}%
\begin{pgfscope}%
\pgfpathrectangle{\pgfqpoint{0.765000in}{0.660000in}}{\pgfqpoint{4.620000in}{4.620000in}}%
\pgfusepath{clip}%
\pgfsetbuttcap%
\pgfsetroundjoin%
\definecolor{currentfill}{rgb}{0.176471,0.192157,0.258824}%
\pgfsetfillcolor{currentfill}%
\pgfsetfillopacity{0.050000}%
\pgfsetlinewidth{0.000000pt}%
\definecolor{currentstroke}{rgb}{0.176471,0.192157,0.258824}%
\pgfsetstrokecolor{currentstroke}%
\pgfsetdash{}{0pt}%
\pgfpathmoveto{\pgfqpoint{3.045111in}{3.063421in}}%
\pgfpathlineto{\pgfqpoint{3.123063in}{3.026824in}}%
\pgfpathlineto{\pgfqpoint{3.045111in}{2.990227in}}%
\pgfpathlineto{\pgfqpoint{2.967158in}{3.026824in}}%
\pgfpathlineto{\pgfqpoint{3.045111in}{3.063421in}}%
\pgfpathclose%
\pgfusepath{fill}%
\end{pgfscope}%
\begin{pgfscope}%
\pgfpathrectangle{\pgfqpoint{0.765000in}{0.660000in}}{\pgfqpoint{4.620000in}{4.620000in}}%
\pgfusepath{clip}%
\pgfsetbuttcap%
\pgfsetroundjoin%
\definecolor{currentfill}{rgb}{0.176471,0.192157,0.258824}%
\pgfsetfillcolor{currentfill}%
\pgfsetfillopacity{0.050000}%
\pgfsetlinewidth{0.000000pt}%
\definecolor{currentstroke}{rgb}{0.176471,0.192157,0.258824}%
\pgfsetstrokecolor{currentstroke}%
\pgfsetdash{}{0pt}%
\pgfpathmoveto{\pgfqpoint{2.967158in}{2.953821in}}%
\pgfpathlineto{\pgfqpoint{2.967158in}{3.026824in}}%
\pgfpathlineto{\pgfqpoint{3.045111in}{2.990227in}}%
\pgfpathlineto{\pgfqpoint{3.045111in}{2.917224in}}%
\pgfpathlineto{\pgfqpoint{2.967158in}{2.953821in}}%
\pgfpathclose%
\pgfusepath{fill}%
\end{pgfscope}%
\begin{pgfscope}%
\pgfpathrectangle{\pgfqpoint{0.765000in}{0.660000in}}{\pgfqpoint{4.620000in}{4.620000in}}%
\pgfusepath{clip}%
\pgfsetbuttcap%
\pgfsetroundjoin%
\definecolor{currentfill}{rgb}{0.176471,0.192157,0.258824}%
\pgfsetfillcolor{currentfill}%
\pgfsetfillopacity{0.050000}%
\pgfsetlinewidth{0.000000pt}%
\definecolor{currentstroke}{rgb}{0.176471,0.192157,0.258824}%
\pgfsetstrokecolor{currentstroke}%
\pgfsetdash{}{0pt}%
\pgfpathmoveto{\pgfqpoint{3.123063in}{2.953821in}}%
\pgfpathlineto{\pgfqpoint{3.123063in}{3.026824in}}%
\pgfpathlineto{\pgfqpoint{3.045111in}{2.990227in}}%
\pgfpathlineto{\pgfqpoint{3.045111in}{2.917224in}}%
\pgfpathlineto{\pgfqpoint{3.123063in}{2.953821in}}%
\pgfpathclose%
\pgfusepath{fill}%
\end{pgfscope}%
\begin{pgfscope}%
\pgfpathrectangle{\pgfqpoint{0.765000in}{0.660000in}}{\pgfqpoint{4.620000in}{4.620000in}}%
\pgfusepath{clip}%
\pgfsetbuttcap%
\pgfsetroundjoin%
\definecolor{currentfill}{rgb}{0.176471,0.192157,0.258824}%
\pgfsetfillcolor{currentfill}%
\pgfsetfillopacity{0.050000}%
\pgfsetlinewidth{0.000000pt}%
\definecolor{currentstroke}{rgb}{0.176471,0.192157,0.258824}%
\pgfsetstrokecolor{currentstroke}%
\pgfsetdash{}{0pt}%
\pgfpathmoveto{\pgfqpoint{3.019663in}{3.025916in}}%
\pgfpathlineto{\pgfqpoint{3.019663in}{3.098919in}}%
\pgfpathlineto{\pgfqpoint{3.097615in}{3.062323in}}%
\pgfpathlineto{\pgfqpoint{3.097615in}{2.989319in}}%
\pgfpathlineto{\pgfqpoint{3.019663in}{3.025916in}}%
\pgfpathclose%
\pgfusepath{fill}%
\end{pgfscope}%
\begin{pgfscope}%
\pgfpathrectangle{\pgfqpoint{0.765000in}{0.660000in}}{\pgfqpoint{4.620000in}{4.620000in}}%
\pgfusepath{clip}%
\pgfsetbuttcap%
\pgfsetroundjoin%
\definecolor{currentfill}{rgb}{0.176471,0.192157,0.258824}%
\pgfsetfillcolor{currentfill}%
\pgfsetfillopacity{0.050000}%
\pgfsetlinewidth{0.000000pt}%
\definecolor{currentstroke}{rgb}{0.176471,0.192157,0.258824}%
\pgfsetstrokecolor{currentstroke}%
\pgfsetdash{}{0pt}%
\pgfpathmoveto{\pgfqpoint{3.019663in}{3.025916in}}%
\pgfpathlineto{\pgfqpoint{3.019663in}{3.098919in}}%
\pgfpathlineto{\pgfqpoint{2.941710in}{3.062323in}}%
\pgfpathlineto{\pgfqpoint{2.941710in}{2.989319in}}%
\pgfpathlineto{\pgfqpoint{3.019663in}{3.025916in}}%
\pgfpathclose%
\pgfusepath{fill}%
\end{pgfscope}%
\begin{pgfscope}%
\pgfpathrectangle{\pgfqpoint{0.765000in}{0.660000in}}{\pgfqpoint{4.620000in}{4.620000in}}%
\pgfusepath{clip}%
\pgfsetbuttcap%
\pgfsetroundjoin%
\definecolor{currentfill}{rgb}{0.176471,0.192157,0.258824}%
\pgfsetfillcolor{currentfill}%
\pgfsetfillopacity{0.050000}%
\pgfsetlinewidth{0.000000pt}%
\definecolor{currentstroke}{rgb}{0.176471,0.192157,0.258824}%
\pgfsetstrokecolor{currentstroke}%
\pgfsetdash{}{0pt}%
\pgfpathmoveto{\pgfqpoint{3.019663in}{3.025916in}}%
\pgfpathlineto{\pgfqpoint{3.097615in}{2.989319in}}%
\pgfpathlineto{\pgfqpoint{3.019663in}{2.952723in}}%
\pgfpathlineto{\pgfqpoint{2.941710in}{2.989319in}}%
\pgfpathlineto{\pgfqpoint{3.019663in}{3.025916in}}%
\pgfpathclose%
\pgfusepath{fill}%
\end{pgfscope}%
\begin{pgfscope}%
\pgfpathrectangle{\pgfqpoint{0.765000in}{0.660000in}}{\pgfqpoint{4.620000in}{4.620000in}}%
\pgfusepath{clip}%
\pgfsetbuttcap%
\pgfsetroundjoin%
\definecolor{currentfill}{rgb}{0.176471,0.192157,0.258824}%
\pgfsetfillcolor{currentfill}%
\pgfsetfillopacity{0.050000}%
\pgfsetlinewidth{0.000000pt}%
\definecolor{currentstroke}{rgb}{0.176471,0.192157,0.258824}%
\pgfsetstrokecolor{currentstroke}%
\pgfsetdash{}{0pt}%
\pgfpathmoveto{\pgfqpoint{3.019663in}{3.098919in}}%
\pgfpathlineto{\pgfqpoint{3.097615in}{3.062323in}}%
\pgfpathlineto{\pgfqpoint{3.019663in}{3.025726in}}%
\pgfpathlineto{\pgfqpoint{2.941710in}{3.062323in}}%
\pgfpathlineto{\pgfqpoint{3.019663in}{3.098919in}}%
\pgfpathclose%
\pgfusepath{fill}%
\end{pgfscope}%
\begin{pgfscope}%
\pgfpathrectangle{\pgfqpoint{0.765000in}{0.660000in}}{\pgfqpoint{4.620000in}{4.620000in}}%
\pgfusepath{clip}%
\pgfsetbuttcap%
\pgfsetroundjoin%
\definecolor{currentfill}{rgb}{0.176471,0.192157,0.258824}%
\pgfsetfillcolor{currentfill}%
\pgfsetfillopacity{0.050000}%
\pgfsetlinewidth{0.000000pt}%
\definecolor{currentstroke}{rgb}{0.176471,0.192157,0.258824}%
\pgfsetstrokecolor{currentstroke}%
\pgfsetdash{}{0pt}%
\pgfpathmoveto{\pgfqpoint{2.941710in}{2.989319in}}%
\pgfpathlineto{\pgfqpoint{2.941710in}{3.062323in}}%
\pgfpathlineto{\pgfqpoint{3.019663in}{3.025726in}}%
\pgfpathlineto{\pgfqpoint{3.019663in}{2.952723in}}%
\pgfpathlineto{\pgfqpoint{2.941710in}{2.989319in}}%
\pgfpathclose%
\pgfusepath{fill}%
\end{pgfscope}%
\begin{pgfscope}%
\pgfpathrectangle{\pgfqpoint{0.765000in}{0.660000in}}{\pgfqpoint{4.620000in}{4.620000in}}%
\pgfusepath{clip}%
\pgfsetbuttcap%
\pgfsetroundjoin%
\definecolor{currentfill}{rgb}{0.176471,0.192157,0.258824}%
\pgfsetfillcolor{currentfill}%
\pgfsetfillopacity{0.050000}%
\pgfsetlinewidth{0.000000pt}%
\definecolor{currentstroke}{rgb}{0.176471,0.192157,0.258824}%
\pgfsetstrokecolor{currentstroke}%
\pgfsetdash{}{0pt}%
\pgfpathmoveto{\pgfqpoint{3.097615in}{2.989319in}}%
\pgfpathlineto{\pgfqpoint{3.097615in}{3.062323in}}%
\pgfpathlineto{\pgfqpoint{3.019663in}{3.025726in}}%
\pgfpathlineto{\pgfqpoint{3.019663in}{2.952723in}}%
\pgfpathlineto{\pgfqpoint{3.097615in}{2.989319in}}%
\pgfpathclose%
\pgfusepath{fill}%
\end{pgfscope}%
\begin{pgfscope}%
\pgfpathrectangle{\pgfqpoint{0.765000in}{0.660000in}}{\pgfqpoint{4.620000in}{4.620000in}}%
\pgfusepath{clip}%
\pgfsetbuttcap%
\pgfsetroundjoin%
\definecolor{currentfill}{rgb}{1.000000,0.576471,0.309804}%
\pgfsetfillcolor{currentfill}%
\pgfsetfillopacity{0.050000}%
\pgfsetlinewidth{0.000000pt}%
\definecolor{currentstroke}{rgb}{1.000000,0.576471,0.309804}%
\pgfsetstrokecolor{currentstroke}%
\pgfsetdash{}{0pt}%
\pgfpathmoveto{\pgfqpoint{2.489570in}{3.127512in}}%
\pgfpathlineto{\pgfqpoint{2.489570in}{3.200515in}}%
\pgfpathlineto{\pgfqpoint{2.567523in}{3.163919in}}%
\pgfpathlineto{\pgfqpoint{2.567523in}{3.090916in}}%
\pgfpathlineto{\pgfqpoint{2.489570in}{3.127512in}}%
\pgfpathclose%
\pgfusepath{fill}%
\end{pgfscope}%
\begin{pgfscope}%
\pgfpathrectangle{\pgfqpoint{0.765000in}{0.660000in}}{\pgfqpoint{4.620000in}{4.620000in}}%
\pgfusepath{clip}%
\pgfsetbuttcap%
\pgfsetroundjoin%
\definecolor{currentfill}{rgb}{1.000000,0.576471,0.309804}%
\pgfsetfillcolor{currentfill}%
\pgfsetfillopacity{0.050000}%
\pgfsetlinewidth{0.000000pt}%
\definecolor{currentstroke}{rgb}{1.000000,0.576471,0.309804}%
\pgfsetstrokecolor{currentstroke}%
\pgfsetdash{}{0pt}%
\pgfpathmoveto{\pgfqpoint{2.489570in}{3.127512in}}%
\pgfpathlineto{\pgfqpoint{2.489570in}{3.200515in}}%
\pgfpathlineto{\pgfqpoint{2.411617in}{3.163919in}}%
\pgfpathlineto{\pgfqpoint{2.411617in}{3.090916in}}%
\pgfpathlineto{\pgfqpoint{2.489570in}{3.127512in}}%
\pgfpathclose%
\pgfusepath{fill}%
\end{pgfscope}%
\begin{pgfscope}%
\pgfpathrectangle{\pgfqpoint{0.765000in}{0.660000in}}{\pgfqpoint{4.620000in}{4.620000in}}%
\pgfusepath{clip}%
\pgfsetbuttcap%
\pgfsetroundjoin%
\definecolor{currentfill}{rgb}{1.000000,0.576471,0.309804}%
\pgfsetfillcolor{currentfill}%
\pgfsetfillopacity{0.050000}%
\pgfsetlinewidth{0.000000pt}%
\definecolor{currentstroke}{rgb}{1.000000,0.576471,0.309804}%
\pgfsetstrokecolor{currentstroke}%
\pgfsetdash{}{0pt}%
\pgfpathmoveto{\pgfqpoint{2.489570in}{3.127512in}}%
\pgfpathlineto{\pgfqpoint{2.567523in}{3.090916in}}%
\pgfpathlineto{\pgfqpoint{2.489570in}{3.054319in}}%
\pgfpathlineto{\pgfqpoint{2.411617in}{3.090916in}}%
\pgfpathlineto{\pgfqpoint{2.489570in}{3.127512in}}%
\pgfpathclose%
\pgfusepath{fill}%
\end{pgfscope}%
\begin{pgfscope}%
\pgfpathrectangle{\pgfqpoint{0.765000in}{0.660000in}}{\pgfqpoint{4.620000in}{4.620000in}}%
\pgfusepath{clip}%
\pgfsetbuttcap%
\pgfsetroundjoin%
\definecolor{currentfill}{rgb}{1.000000,0.576471,0.309804}%
\pgfsetfillcolor{currentfill}%
\pgfsetfillopacity{0.050000}%
\pgfsetlinewidth{0.000000pt}%
\definecolor{currentstroke}{rgb}{1.000000,0.576471,0.309804}%
\pgfsetstrokecolor{currentstroke}%
\pgfsetdash{}{0pt}%
\pgfpathmoveto{\pgfqpoint{2.489570in}{3.200515in}}%
\pgfpathlineto{\pgfqpoint{2.567523in}{3.163919in}}%
\pgfpathlineto{\pgfqpoint{2.489570in}{3.127322in}}%
\pgfpathlineto{\pgfqpoint{2.411617in}{3.163919in}}%
\pgfpathlineto{\pgfqpoint{2.489570in}{3.200515in}}%
\pgfpathclose%
\pgfusepath{fill}%
\end{pgfscope}%
\begin{pgfscope}%
\pgfpathrectangle{\pgfqpoint{0.765000in}{0.660000in}}{\pgfqpoint{4.620000in}{4.620000in}}%
\pgfusepath{clip}%
\pgfsetbuttcap%
\pgfsetroundjoin%
\definecolor{currentfill}{rgb}{1.000000,0.576471,0.309804}%
\pgfsetfillcolor{currentfill}%
\pgfsetfillopacity{0.050000}%
\pgfsetlinewidth{0.000000pt}%
\definecolor{currentstroke}{rgb}{1.000000,0.576471,0.309804}%
\pgfsetstrokecolor{currentstroke}%
\pgfsetdash{}{0pt}%
\pgfpathmoveto{\pgfqpoint{2.411617in}{3.090916in}}%
\pgfpathlineto{\pgfqpoint{2.411617in}{3.163919in}}%
\pgfpathlineto{\pgfqpoint{2.489570in}{3.127322in}}%
\pgfpathlineto{\pgfqpoint{2.489570in}{3.054319in}}%
\pgfpathlineto{\pgfqpoint{2.411617in}{3.090916in}}%
\pgfpathclose%
\pgfusepath{fill}%
\end{pgfscope}%
\begin{pgfscope}%
\pgfpathrectangle{\pgfqpoint{0.765000in}{0.660000in}}{\pgfqpoint{4.620000in}{4.620000in}}%
\pgfusepath{clip}%
\pgfsetbuttcap%
\pgfsetroundjoin%
\definecolor{currentfill}{rgb}{1.000000,0.576471,0.309804}%
\pgfsetfillcolor{currentfill}%
\pgfsetfillopacity{0.050000}%
\pgfsetlinewidth{0.000000pt}%
\definecolor{currentstroke}{rgb}{1.000000,0.576471,0.309804}%
\pgfsetstrokecolor{currentstroke}%
\pgfsetdash{}{0pt}%
\pgfpathmoveto{\pgfqpoint{2.567523in}{3.090916in}}%
\pgfpathlineto{\pgfqpoint{2.567523in}{3.163919in}}%
\pgfpathlineto{\pgfqpoint{2.489570in}{3.127322in}}%
\pgfpathlineto{\pgfqpoint{2.489570in}{3.054319in}}%
\pgfpathlineto{\pgfqpoint{2.567523in}{3.090916in}}%
\pgfpathclose%
\pgfusepath{fill}%
\end{pgfscope}%
\begin{pgfscope}%
\pgfpathrectangle{\pgfqpoint{0.765000in}{0.660000in}}{\pgfqpoint{4.620000in}{4.620000in}}%
\pgfusepath{clip}%
\pgfsetbuttcap%
\pgfsetroundjoin%
\definecolor{currentfill}{rgb}{0.176471,0.192157,0.258824}%
\pgfsetfillcolor{currentfill}%
\pgfsetfillopacity{0.050000}%
\pgfsetlinewidth{0.000000pt}%
\definecolor{currentstroke}{rgb}{0.176471,0.192157,0.258824}%
\pgfsetstrokecolor{currentstroke}%
\pgfsetdash{}{0pt}%
\pgfpathmoveto{\pgfqpoint{3.081429in}{2.862536in}}%
\pgfpathlineto{\pgfqpoint{3.081429in}{2.935540in}}%
\pgfpathlineto{\pgfqpoint{3.159382in}{2.898943in}}%
\pgfpathlineto{\pgfqpoint{3.159382in}{2.825940in}}%
\pgfpathlineto{\pgfqpoint{3.081429in}{2.862536in}}%
\pgfpathclose%
\pgfusepath{fill}%
\end{pgfscope}%
\begin{pgfscope}%
\pgfpathrectangle{\pgfqpoint{0.765000in}{0.660000in}}{\pgfqpoint{4.620000in}{4.620000in}}%
\pgfusepath{clip}%
\pgfsetbuttcap%
\pgfsetroundjoin%
\definecolor{currentfill}{rgb}{0.176471,0.192157,0.258824}%
\pgfsetfillcolor{currentfill}%
\pgfsetfillopacity{0.050000}%
\pgfsetlinewidth{0.000000pt}%
\definecolor{currentstroke}{rgb}{0.176471,0.192157,0.258824}%
\pgfsetstrokecolor{currentstroke}%
\pgfsetdash{}{0pt}%
\pgfpathmoveto{\pgfqpoint{3.081429in}{2.862536in}}%
\pgfpathlineto{\pgfqpoint{3.081429in}{2.935540in}}%
\pgfpathlineto{\pgfqpoint{3.003477in}{2.898943in}}%
\pgfpathlineto{\pgfqpoint{3.003477in}{2.825940in}}%
\pgfpathlineto{\pgfqpoint{3.081429in}{2.862536in}}%
\pgfpathclose%
\pgfusepath{fill}%
\end{pgfscope}%
\begin{pgfscope}%
\pgfpathrectangle{\pgfqpoint{0.765000in}{0.660000in}}{\pgfqpoint{4.620000in}{4.620000in}}%
\pgfusepath{clip}%
\pgfsetbuttcap%
\pgfsetroundjoin%
\definecolor{currentfill}{rgb}{0.176471,0.192157,0.258824}%
\pgfsetfillcolor{currentfill}%
\pgfsetfillopacity{0.050000}%
\pgfsetlinewidth{0.000000pt}%
\definecolor{currentstroke}{rgb}{0.176471,0.192157,0.258824}%
\pgfsetstrokecolor{currentstroke}%
\pgfsetdash{}{0pt}%
\pgfpathmoveto{\pgfqpoint{3.081429in}{2.862536in}}%
\pgfpathlineto{\pgfqpoint{3.159382in}{2.825940in}}%
\pgfpathlineto{\pgfqpoint{3.081429in}{2.789343in}}%
\pgfpathlineto{\pgfqpoint{3.003477in}{2.825940in}}%
\pgfpathlineto{\pgfqpoint{3.081429in}{2.862536in}}%
\pgfpathclose%
\pgfusepath{fill}%
\end{pgfscope}%
\begin{pgfscope}%
\pgfpathrectangle{\pgfqpoint{0.765000in}{0.660000in}}{\pgfqpoint{4.620000in}{4.620000in}}%
\pgfusepath{clip}%
\pgfsetbuttcap%
\pgfsetroundjoin%
\definecolor{currentfill}{rgb}{0.176471,0.192157,0.258824}%
\pgfsetfillcolor{currentfill}%
\pgfsetfillopacity{0.050000}%
\pgfsetlinewidth{0.000000pt}%
\definecolor{currentstroke}{rgb}{0.176471,0.192157,0.258824}%
\pgfsetstrokecolor{currentstroke}%
\pgfsetdash{}{0pt}%
\pgfpathmoveto{\pgfqpoint{3.081429in}{2.935540in}}%
\pgfpathlineto{\pgfqpoint{3.159382in}{2.898943in}}%
\pgfpathlineto{\pgfqpoint{3.081429in}{2.862346in}}%
\pgfpathlineto{\pgfqpoint{3.003477in}{2.898943in}}%
\pgfpathlineto{\pgfqpoint{3.081429in}{2.935540in}}%
\pgfpathclose%
\pgfusepath{fill}%
\end{pgfscope}%
\begin{pgfscope}%
\pgfpathrectangle{\pgfqpoint{0.765000in}{0.660000in}}{\pgfqpoint{4.620000in}{4.620000in}}%
\pgfusepath{clip}%
\pgfsetbuttcap%
\pgfsetroundjoin%
\definecolor{currentfill}{rgb}{0.176471,0.192157,0.258824}%
\pgfsetfillcolor{currentfill}%
\pgfsetfillopacity{0.050000}%
\pgfsetlinewidth{0.000000pt}%
\definecolor{currentstroke}{rgb}{0.176471,0.192157,0.258824}%
\pgfsetstrokecolor{currentstroke}%
\pgfsetdash{}{0pt}%
\pgfpathmoveto{\pgfqpoint{3.003477in}{2.825940in}}%
\pgfpathlineto{\pgfqpoint{3.003477in}{2.898943in}}%
\pgfpathlineto{\pgfqpoint{3.081429in}{2.862346in}}%
\pgfpathlineto{\pgfqpoint{3.081429in}{2.789343in}}%
\pgfpathlineto{\pgfqpoint{3.003477in}{2.825940in}}%
\pgfpathclose%
\pgfusepath{fill}%
\end{pgfscope}%
\begin{pgfscope}%
\pgfpathrectangle{\pgfqpoint{0.765000in}{0.660000in}}{\pgfqpoint{4.620000in}{4.620000in}}%
\pgfusepath{clip}%
\pgfsetbuttcap%
\pgfsetroundjoin%
\definecolor{currentfill}{rgb}{0.176471,0.192157,0.258824}%
\pgfsetfillcolor{currentfill}%
\pgfsetfillopacity{0.050000}%
\pgfsetlinewidth{0.000000pt}%
\definecolor{currentstroke}{rgb}{0.176471,0.192157,0.258824}%
\pgfsetstrokecolor{currentstroke}%
\pgfsetdash{}{0pt}%
\pgfpathmoveto{\pgfqpoint{3.159382in}{2.825940in}}%
\pgfpathlineto{\pgfqpoint{3.159382in}{2.898943in}}%
\pgfpathlineto{\pgfqpoint{3.081429in}{2.862346in}}%
\pgfpathlineto{\pgfqpoint{3.081429in}{2.789343in}}%
\pgfpathlineto{\pgfqpoint{3.159382in}{2.825940in}}%
\pgfpathclose%
\pgfusepath{fill}%
\end{pgfscope}%
\begin{pgfscope}%
\pgfpathrectangle{\pgfqpoint{0.765000in}{0.660000in}}{\pgfqpoint{4.620000in}{4.620000in}}%
\pgfusepath{clip}%
\pgfsetbuttcap%
\pgfsetroundjoin%
\definecolor{currentfill}{rgb}{1.000000,0.576471,0.309804}%
\pgfsetfillcolor{currentfill}%
\pgfsetfillopacity{0.050000}%
\pgfsetlinewidth{0.000000pt}%
\definecolor{currentstroke}{rgb}{1.000000,0.576471,0.309804}%
\pgfsetstrokecolor{currentstroke}%
\pgfsetdash{}{0pt}%
\pgfpathmoveto{\pgfqpoint{4.015348in}{3.691913in}}%
\pgfpathlineto{\pgfqpoint{4.015348in}{3.764916in}}%
\pgfpathlineto{\pgfqpoint{4.093301in}{3.728319in}}%
\pgfpathlineto{\pgfqpoint{4.093301in}{3.655316in}}%
\pgfpathlineto{\pgfqpoint{4.015348in}{3.691913in}}%
\pgfpathclose%
\pgfusepath{fill}%
\end{pgfscope}%
\begin{pgfscope}%
\pgfpathrectangle{\pgfqpoint{0.765000in}{0.660000in}}{\pgfqpoint{4.620000in}{4.620000in}}%
\pgfusepath{clip}%
\pgfsetbuttcap%
\pgfsetroundjoin%
\definecolor{currentfill}{rgb}{1.000000,0.576471,0.309804}%
\pgfsetfillcolor{currentfill}%
\pgfsetfillopacity{0.050000}%
\pgfsetlinewidth{0.000000pt}%
\definecolor{currentstroke}{rgb}{1.000000,0.576471,0.309804}%
\pgfsetstrokecolor{currentstroke}%
\pgfsetdash{}{0pt}%
\pgfpathmoveto{\pgfqpoint{4.015348in}{3.691913in}}%
\pgfpathlineto{\pgfqpoint{4.015348in}{3.764916in}}%
\pgfpathlineto{\pgfqpoint{3.937395in}{3.728319in}}%
\pgfpathlineto{\pgfqpoint{3.937395in}{3.655316in}}%
\pgfpathlineto{\pgfqpoint{4.015348in}{3.691913in}}%
\pgfpathclose%
\pgfusepath{fill}%
\end{pgfscope}%
\begin{pgfscope}%
\pgfpathrectangle{\pgfqpoint{0.765000in}{0.660000in}}{\pgfqpoint{4.620000in}{4.620000in}}%
\pgfusepath{clip}%
\pgfsetbuttcap%
\pgfsetroundjoin%
\definecolor{currentfill}{rgb}{1.000000,0.576471,0.309804}%
\pgfsetfillcolor{currentfill}%
\pgfsetfillopacity{0.050000}%
\pgfsetlinewidth{0.000000pt}%
\definecolor{currentstroke}{rgb}{1.000000,0.576471,0.309804}%
\pgfsetstrokecolor{currentstroke}%
\pgfsetdash{}{0pt}%
\pgfpathmoveto{\pgfqpoint{4.015348in}{3.691913in}}%
\pgfpathlineto{\pgfqpoint{4.093301in}{3.655316in}}%
\pgfpathlineto{\pgfqpoint{4.015348in}{3.618720in}}%
\pgfpathlineto{\pgfqpoint{3.937395in}{3.655316in}}%
\pgfpathlineto{\pgfqpoint{4.015348in}{3.691913in}}%
\pgfpathclose%
\pgfusepath{fill}%
\end{pgfscope}%
\begin{pgfscope}%
\pgfpathrectangle{\pgfqpoint{0.765000in}{0.660000in}}{\pgfqpoint{4.620000in}{4.620000in}}%
\pgfusepath{clip}%
\pgfsetbuttcap%
\pgfsetroundjoin%
\definecolor{currentfill}{rgb}{1.000000,0.576471,0.309804}%
\pgfsetfillcolor{currentfill}%
\pgfsetfillopacity{0.050000}%
\pgfsetlinewidth{0.000000pt}%
\definecolor{currentstroke}{rgb}{1.000000,0.576471,0.309804}%
\pgfsetstrokecolor{currentstroke}%
\pgfsetdash{}{0pt}%
\pgfpathmoveto{\pgfqpoint{4.015348in}{3.764916in}}%
\pgfpathlineto{\pgfqpoint{4.093301in}{3.728319in}}%
\pgfpathlineto{\pgfqpoint{4.015348in}{3.691723in}}%
\pgfpathlineto{\pgfqpoint{3.937395in}{3.728319in}}%
\pgfpathlineto{\pgfqpoint{4.015348in}{3.764916in}}%
\pgfpathclose%
\pgfusepath{fill}%
\end{pgfscope}%
\begin{pgfscope}%
\pgfpathrectangle{\pgfqpoint{0.765000in}{0.660000in}}{\pgfqpoint{4.620000in}{4.620000in}}%
\pgfusepath{clip}%
\pgfsetbuttcap%
\pgfsetroundjoin%
\definecolor{currentfill}{rgb}{1.000000,0.576471,0.309804}%
\pgfsetfillcolor{currentfill}%
\pgfsetfillopacity{0.050000}%
\pgfsetlinewidth{0.000000pt}%
\definecolor{currentstroke}{rgb}{1.000000,0.576471,0.309804}%
\pgfsetstrokecolor{currentstroke}%
\pgfsetdash{}{0pt}%
\pgfpathmoveto{\pgfqpoint{3.937395in}{3.655316in}}%
\pgfpathlineto{\pgfqpoint{3.937395in}{3.728319in}}%
\pgfpathlineto{\pgfqpoint{4.015348in}{3.691723in}}%
\pgfpathlineto{\pgfqpoint{4.015348in}{3.618720in}}%
\pgfpathlineto{\pgfqpoint{3.937395in}{3.655316in}}%
\pgfpathclose%
\pgfusepath{fill}%
\end{pgfscope}%
\begin{pgfscope}%
\pgfpathrectangle{\pgfqpoint{0.765000in}{0.660000in}}{\pgfqpoint{4.620000in}{4.620000in}}%
\pgfusepath{clip}%
\pgfsetbuttcap%
\pgfsetroundjoin%
\definecolor{currentfill}{rgb}{1.000000,0.576471,0.309804}%
\pgfsetfillcolor{currentfill}%
\pgfsetfillopacity{0.050000}%
\pgfsetlinewidth{0.000000pt}%
\definecolor{currentstroke}{rgb}{1.000000,0.576471,0.309804}%
\pgfsetstrokecolor{currentstroke}%
\pgfsetdash{}{0pt}%
\pgfpathmoveto{\pgfqpoint{4.093301in}{3.655316in}}%
\pgfpathlineto{\pgfqpoint{4.093301in}{3.728319in}}%
\pgfpathlineto{\pgfqpoint{4.015348in}{3.691723in}}%
\pgfpathlineto{\pgfqpoint{4.015348in}{3.618720in}}%
\pgfpathlineto{\pgfqpoint{4.093301in}{3.655316in}}%
\pgfpathclose%
\pgfusepath{fill}%
\end{pgfscope}%
\begin{pgfscope}%
\pgfpathrectangle{\pgfqpoint{0.765000in}{0.660000in}}{\pgfqpoint{4.620000in}{4.620000in}}%
\pgfusepath{clip}%
\pgfsetbuttcap%
\pgfsetroundjoin%
\definecolor{currentfill}{rgb}{0.176471,0.192157,0.258824}%
\pgfsetfillcolor{currentfill}%
\pgfsetfillopacity{0.050000}%
\pgfsetlinewidth{0.000000pt}%
\definecolor{currentstroke}{rgb}{0.176471,0.192157,0.258824}%
\pgfsetstrokecolor{currentstroke}%
\pgfsetdash{}{0pt}%
\pgfpathmoveto{\pgfqpoint{3.069418in}{3.025175in}}%
\pgfpathlineto{\pgfqpoint{3.069418in}{3.098178in}}%
\pgfpathlineto{\pgfqpoint{3.147371in}{3.061582in}}%
\pgfpathlineto{\pgfqpoint{3.147371in}{2.988578in}}%
\pgfpathlineto{\pgfqpoint{3.069418in}{3.025175in}}%
\pgfpathclose%
\pgfusepath{fill}%
\end{pgfscope}%
\begin{pgfscope}%
\pgfpathrectangle{\pgfqpoint{0.765000in}{0.660000in}}{\pgfqpoint{4.620000in}{4.620000in}}%
\pgfusepath{clip}%
\pgfsetbuttcap%
\pgfsetroundjoin%
\definecolor{currentfill}{rgb}{0.176471,0.192157,0.258824}%
\pgfsetfillcolor{currentfill}%
\pgfsetfillopacity{0.050000}%
\pgfsetlinewidth{0.000000pt}%
\definecolor{currentstroke}{rgb}{0.176471,0.192157,0.258824}%
\pgfsetstrokecolor{currentstroke}%
\pgfsetdash{}{0pt}%
\pgfpathmoveto{\pgfqpoint{3.069418in}{3.025175in}}%
\pgfpathlineto{\pgfqpoint{3.069418in}{3.098178in}}%
\pgfpathlineto{\pgfqpoint{2.991466in}{3.061582in}}%
\pgfpathlineto{\pgfqpoint{2.991466in}{2.988578in}}%
\pgfpathlineto{\pgfqpoint{3.069418in}{3.025175in}}%
\pgfpathclose%
\pgfusepath{fill}%
\end{pgfscope}%
\begin{pgfscope}%
\pgfpathrectangle{\pgfqpoint{0.765000in}{0.660000in}}{\pgfqpoint{4.620000in}{4.620000in}}%
\pgfusepath{clip}%
\pgfsetbuttcap%
\pgfsetroundjoin%
\definecolor{currentfill}{rgb}{0.176471,0.192157,0.258824}%
\pgfsetfillcolor{currentfill}%
\pgfsetfillopacity{0.050000}%
\pgfsetlinewidth{0.000000pt}%
\definecolor{currentstroke}{rgb}{0.176471,0.192157,0.258824}%
\pgfsetstrokecolor{currentstroke}%
\pgfsetdash{}{0pt}%
\pgfpathmoveto{\pgfqpoint{3.069418in}{3.025175in}}%
\pgfpathlineto{\pgfqpoint{3.147371in}{2.988578in}}%
\pgfpathlineto{\pgfqpoint{3.069418in}{2.951982in}}%
\pgfpathlineto{\pgfqpoint{2.991466in}{2.988578in}}%
\pgfpathlineto{\pgfqpoint{3.069418in}{3.025175in}}%
\pgfpathclose%
\pgfusepath{fill}%
\end{pgfscope}%
\begin{pgfscope}%
\pgfpathrectangle{\pgfqpoint{0.765000in}{0.660000in}}{\pgfqpoint{4.620000in}{4.620000in}}%
\pgfusepath{clip}%
\pgfsetbuttcap%
\pgfsetroundjoin%
\definecolor{currentfill}{rgb}{0.176471,0.192157,0.258824}%
\pgfsetfillcolor{currentfill}%
\pgfsetfillopacity{0.050000}%
\pgfsetlinewidth{0.000000pt}%
\definecolor{currentstroke}{rgb}{0.176471,0.192157,0.258824}%
\pgfsetstrokecolor{currentstroke}%
\pgfsetdash{}{0pt}%
\pgfpathmoveto{\pgfqpoint{3.069418in}{3.098178in}}%
\pgfpathlineto{\pgfqpoint{3.147371in}{3.061582in}}%
\pgfpathlineto{\pgfqpoint{3.069418in}{3.024985in}}%
\pgfpathlineto{\pgfqpoint{2.991466in}{3.061582in}}%
\pgfpathlineto{\pgfqpoint{3.069418in}{3.098178in}}%
\pgfpathclose%
\pgfusepath{fill}%
\end{pgfscope}%
\begin{pgfscope}%
\pgfpathrectangle{\pgfqpoint{0.765000in}{0.660000in}}{\pgfqpoint{4.620000in}{4.620000in}}%
\pgfusepath{clip}%
\pgfsetbuttcap%
\pgfsetroundjoin%
\definecolor{currentfill}{rgb}{0.176471,0.192157,0.258824}%
\pgfsetfillcolor{currentfill}%
\pgfsetfillopacity{0.050000}%
\pgfsetlinewidth{0.000000pt}%
\definecolor{currentstroke}{rgb}{0.176471,0.192157,0.258824}%
\pgfsetstrokecolor{currentstroke}%
\pgfsetdash{}{0pt}%
\pgfpathmoveto{\pgfqpoint{2.991466in}{2.988578in}}%
\pgfpathlineto{\pgfqpoint{2.991466in}{3.061582in}}%
\pgfpathlineto{\pgfqpoint{3.069418in}{3.024985in}}%
\pgfpathlineto{\pgfqpoint{3.069418in}{2.951982in}}%
\pgfpathlineto{\pgfqpoint{2.991466in}{2.988578in}}%
\pgfpathclose%
\pgfusepath{fill}%
\end{pgfscope}%
\begin{pgfscope}%
\pgfpathrectangle{\pgfqpoint{0.765000in}{0.660000in}}{\pgfqpoint{4.620000in}{4.620000in}}%
\pgfusepath{clip}%
\pgfsetbuttcap%
\pgfsetroundjoin%
\definecolor{currentfill}{rgb}{0.176471,0.192157,0.258824}%
\pgfsetfillcolor{currentfill}%
\pgfsetfillopacity{0.050000}%
\pgfsetlinewidth{0.000000pt}%
\definecolor{currentstroke}{rgb}{0.176471,0.192157,0.258824}%
\pgfsetstrokecolor{currentstroke}%
\pgfsetdash{}{0pt}%
\pgfpathmoveto{\pgfqpoint{3.147371in}{2.988578in}}%
\pgfpathlineto{\pgfqpoint{3.147371in}{3.061582in}}%
\pgfpathlineto{\pgfqpoint{3.069418in}{3.024985in}}%
\pgfpathlineto{\pgfqpoint{3.069418in}{2.951982in}}%
\pgfpathlineto{\pgfqpoint{3.147371in}{2.988578in}}%
\pgfpathclose%
\pgfusepath{fill}%
\end{pgfscope}%
\begin{pgfscope}%
\pgfpathrectangle{\pgfqpoint{0.765000in}{0.660000in}}{\pgfqpoint{4.620000in}{4.620000in}}%
\pgfusepath{clip}%
\pgfsetbuttcap%
\pgfsetroundjoin%
\definecolor{currentfill}{rgb}{0.176471,0.192157,0.258824}%
\pgfsetfillcolor{currentfill}%
\pgfsetfillopacity{0.050000}%
\pgfsetlinewidth{0.000000pt}%
\definecolor{currentstroke}{rgb}{0.176471,0.192157,0.258824}%
\pgfsetstrokecolor{currentstroke}%
\pgfsetdash{}{0pt}%
\pgfpathmoveto{\pgfqpoint{3.078896in}{2.967597in}}%
\pgfpathlineto{\pgfqpoint{3.078896in}{3.040601in}}%
\pgfpathlineto{\pgfqpoint{3.156849in}{3.004004in}}%
\pgfpathlineto{\pgfqpoint{3.156849in}{2.931001in}}%
\pgfpathlineto{\pgfqpoint{3.078896in}{2.967597in}}%
\pgfpathclose%
\pgfusepath{fill}%
\end{pgfscope}%
\begin{pgfscope}%
\pgfpathrectangle{\pgfqpoint{0.765000in}{0.660000in}}{\pgfqpoint{4.620000in}{4.620000in}}%
\pgfusepath{clip}%
\pgfsetbuttcap%
\pgfsetroundjoin%
\definecolor{currentfill}{rgb}{0.176471,0.192157,0.258824}%
\pgfsetfillcolor{currentfill}%
\pgfsetfillopacity{0.050000}%
\pgfsetlinewidth{0.000000pt}%
\definecolor{currentstroke}{rgb}{0.176471,0.192157,0.258824}%
\pgfsetstrokecolor{currentstroke}%
\pgfsetdash{}{0pt}%
\pgfpathmoveto{\pgfqpoint{3.078896in}{2.967597in}}%
\pgfpathlineto{\pgfqpoint{3.078896in}{3.040601in}}%
\pgfpathlineto{\pgfqpoint{3.000943in}{3.004004in}}%
\pgfpathlineto{\pgfqpoint{3.000943in}{2.931001in}}%
\pgfpathlineto{\pgfqpoint{3.078896in}{2.967597in}}%
\pgfpathclose%
\pgfusepath{fill}%
\end{pgfscope}%
\begin{pgfscope}%
\pgfpathrectangle{\pgfqpoint{0.765000in}{0.660000in}}{\pgfqpoint{4.620000in}{4.620000in}}%
\pgfusepath{clip}%
\pgfsetbuttcap%
\pgfsetroundjoin%
\definecolor{currentfill}{rgb}{0.176471,0.192157,0.258824}%
\pgfsetfillcolor{currentfill}%
\pgfsetfillopacity{0.050000}%
\pgfsetlinewidth{0.000000pt}%
\definecolor{currentstroke}{rgb}{0.176471,0.192157,0.258824}%
\pgfsetstrokecolor{currentstroke}%
\pgfsetdash{}{0pt}%
\pgfpathmoveto{\pgfqpoint{3.078896in}{2.967597in}}%
\pgfpathlineto{\pgfqpoint{3.156849in}{2.931001in}}%
\pgfpathlineto{\pgfqpoint{3.078896in}{2.894404in}}%
\pgfpathlineto{\pgfqpoint{3.000943in}{2.931001in}}%
\pgfpathlineto{\pgfqpoint{3.078896in}{2.967597in}}%
\pgfpathclose%
\pgfusepath{fill}%
\end{pgfscope}%
\begin{pgfscope}%
\pgfpathrectangle{\pgfqpoint{0.765000in}{0.660000in}}{\pgfqpoint{4.620000in}{4.620000in}}%
\pgfusepath{clip}%
\pgfsetbuttcap%
\pgfsetroundjoin%
\definecolor{currentfill}{rgb}{0.176471,0.192157,0.258824}%
\pgfsetfillcolor{currentfill}%
\pgfsetfillopacity{0.050000}%
\pgfsetlinewidth{0.000000pt}%
\definecolor{currentstroke}{rgb}{0.176471,0.192157,0.258824}%
\pgfsetstrokecolor{currentstroke}%
\pgfsetdash{}{0pt}%
\pgfpathmoveto{\pgfqpoint{3.078896in}{3.040601in}}%
\pgfpathlineto{\pgfqpoint{3.156849in}{3.004004in}}%
\pgfpathlineto{\pgfqpoint{3.078896in}{2.967407in}}%
\pgfpathlineto{\pgfqpoint{3.000943in}{3.004004in}}%
\pgfpathlineto{\pgfqpoint{3.078896in}{3.040601in}}%
\pgfpathclose%
\pgfusepath{fill}%
\end{pgfscope}%
\begin{pgfscope}%
\pgfpathrectangle{\pgfqpoint{0.765000in}{0.660000in}}{\pgfqpoint{4.620000in}{4.620000in}}%
\pgfusepath{clip}%
\pgfsetbuttcap%
\pgfsetroundjoin%
\definecolor{currentfill}{rgb}{0.176471,0.192157,0.258824}%
\pgfsetfillcolor{currentfill}%
\pgfsetfillopacity{0.050000}%
\pgfsetlinewidth{0.000000pt}%
\definecolor{currentstroke}{rgb}{0.176471,0.192157,0.258824}%
\pgfsetstrokecolor{currentstroke}%
\pgfsetdash{}{0pt}%
\pgfpathmoveto{\pgfqpoint{3.000943in}{2.931001in}}%
\pgfpathlineto{\pgfqpoint{3.000943in}{3.004004in}}%
\pgfpathlineto{\pgfqpoint{3.078896in}{2.967407in}}%
\pgfpathlineto{\pgfqpoint{3.078896in}{2.894404in}}%
\pgfpathlineto{\pgfqpoint{3.000943in}{2.931001in}}%
\pgfpathclose%
\pgfusepath{fill}%
\end{pgfscope}%
\begin{pgfscope}%
\pgfpathrectangle{\pgfqpoint{0.765000in}{0.660000in}}{\pgfqpoint{4.620000in}{4.620000in}}%
\pgfusepath{clip}%
\pgfsetbuttcap%
\pgfsetroundjoin%
\definecolor{currentfill}{rgb}{0.176471,0.192157,0.258824}%
\pgfsetfillcolor{currentfill}%
\pgfsetfillopacity{0.050000}%
\pgfsetlinewidth{0.000000pt}%
\definecolor{currentstroke}{rgb}{0.176471,0.192157,0.258824}%
\pgfsetstrokecolor{currentstroke}%
\pgfsetdash{}{0pt}%
\pgfpathmoveto{\pgfqpoint{3.156849in}{2.931001in}}%
\pgfpathlineto{\pgfqpoint{3.156849in}{3.004004in}}%
\pgfpathlineto{\pgfqpoint{3.078896in}{2.967407in}}%
\pgfpathlineto{\pgfqpoint{3.078896in}{2.894404in}}%
\pgfpathlineto{\pgfqpoint{3.156849in}{2.931001in}}%
\pgfpathclose%
\pgfusepath{fill}%
\end{pgfscope}%
\begin{pgfscope}%
\pgfpathrectangle{\pgfqpoint{0.765000in}{0.660000in}}{\pgfqpoint{4.620000in}{4.620000in}}%
\pgfusepath{clip}%
\pgfsetbuttcap%
\pgfsetroundjoin%
\definecolor{currentfill}{rgb}{0.176471,0.192157,0.258824}%
\pgfsetfillcolor{currentfill}%
\pgfsetfillopacity{0.050000}%
\pgfsetlinewidth{0.000000pt}%
\definecolor{currentstroke}{rgb}{0.176471,0.192157,0.258824}%
\pgfsetstrokecolor{currentstroke}%
\pgfsetdash{}{0pt}%
\pgfpathmoveto{\pgfqpoint{3.240997in}{2.920755in}}%
\pgfpathlineto{\pgfqpoint{3.240997in}{2.993758in}}%
\pgfpathlineto{\pgfqpoint{3.318950in}{2.957161in}}%
\pgfpathlineto{\pgfqpoint{3.318950in}{2.884158in}}%
\pgfpathlineto{\pgfqpoint{3.240997in}{2.920755in}}%
\pgfpathclose%
\pgfusepath{fill}%
\end{pgfscope}%
\begin{pgfscope}%
\pgfpathrectangle{\pgfqpoint{0.765000in}{0.660000in}}{\pgfqpoint{4.620000in}{4.620000in}}%
\pgfusepath{clip}%
\pgfsetbuttcap%
\pgfsetroundjoin%
\definecolor{currentfill}{rgb}{0.176471,0.192157,0.258824}%
\pgfsetfillcolor{currentfill}%
\pgfsetfillopacity{0.050000}%
\pgfsetlinewidth{0.000000pt}%
\definecolor{currentstroke}{rgb}{0.176471,0.192157,0.258824}%
\pgfsetstrokecolor{currentstroke}%
\pgfsetdash{}{0pt}%
\pgfpathmoveto{\pgfqpoint{3.240997in}{2.920755in}}%
\pgfpathlineto{\pgfqpoint{3.240997in}{2.993758in}}%
\pgfpathlineto{\pgfqpoint{3.163045in}{2.957161in}}%
\pgfpathlineto{\pgfqpoint{3.163045in}{2.884158in}}%
\pgfpathlineto{\pgfqpoint{3.240997in}{2.920755in}}%
\pgfpathclose%
\pgfusepath{fill}%
\end{pgfscope}%
\begin{pgfscope}%
\pgfpathrectangle{\pgfqpoint{0.765000in}{0.660000in}}{\pgfqpoint{4.620000in}{4.620000in}}%
\pgfusepath{clip}%
\pgfsetbuttcap%
\pgfsetroundjoin%
\definecolor{currentfill}{rgb}{0.176471,0.192157,0.258824}%
\pgfsetfillcolor{currentfill}%
\pgfsetfillopacity{0.050000}%
\pgfsetlinewidth{0.000000pt}%
\definecolor{currentstroke}{rgb}{0.176471,0.192157,0.258824}%
\pgfsetstrokecolor{currentstroke}%
\pgfsetdash{}{0pt}%
\pgfpathmoveto{\pgfqpoint{3.240997in}{2.920755in}}%
\pgfpathlineto{\pgfqpoint{3.318950in}{2.884158in}}%
\pgfpathlineto{\pgfqpoint{3.240997in}{2.847561in}}%
\pgfpathlineto{\pgfqpoint{3.163045in}{2.884158in}}%
\pgfpathlineto{\pgfqpoint{3.240997in}{2.920755in}}%
\pgfpathclose%
\pgfusepath{fill}%
\end{pgfscope}%
\begin{pgfscope}%
\pgfpathrectangle{\pgfqpoint{0.765000in}{0.660000in}}{\pgfqpoint{4.620000in}{4.620000in}}%
\pgfusepath{clip}%
\pgfsetbuttcap%
\pgfsetroundjoin%
\definecolor{currentfill}{rgb}{0.176471,0.192157,0.258824}%
\pgfsetfillcolor{currentfill}%
\pgfsetfillopacity{0.050000}%
\pgfsetlinewidth{0.000000pt}%
\definecolor{currentstroke}{rgb}{0.176471,0.192157,0.258824}%
\pgfsetstrokecolor{currentstroke}%
\pgfsetdash{}{0pt}%
\pgfpathmoveto{\pgfqpoint{3.240997in}{2.993758in}}%
\pgfpathlineto{\pgfqpoint{3.318950in}{2.957161in}}%
\pgfpathlineto{\pgfqpoint{3.240997in}{2.920565in}}%
\pgfpathlineto{\pgfqpoint{3.163045in}{2.957161in}}%
\pgfpathlineto{\pgfqpoint{3.240997in}{2.993758in}}%
\pgfpathclose%
\pgfusepath{fill}%
\end{pgfscope}%
\begin{pgfscope}%
\pgfpathrectangle{\pgfqpoint{0.765000in}{0.660000in}}{\pgfqpoint{4.620000in}{4.620000in}}%
\pgfusepath{clip}%
\pgfsetbuttcap%
\pgfsetroundjoin%
\definecolor{currentfill}{rgb}{0.176471,0.192157,0.258824}%
\pgfsetfillcolor{currentfill}%
\pgfsetfillopacity{0.050000}%
\pgfsetlinewidth{0.000000pt}%
\definecolor{currentstroke}{rgb}{0.176471,0.192157,0.258824}%
\pgfsetstrokecolor{currentstroke}%
\pgfsetdash{}{0pt}%
\pgfpathmoveto{\pgfqpoint{3.163045in}{2.884158in}}%
\pgfpathlineto{\pgfqpoint{3.163045in}{2.957161in}}%
\pgfpathlineto{\pgfqpoint{3.240997in}{2.920565in}}%
\pgfpathlineto{\pgfqpoint{3.240997in}{2.847561in}}%
\pgfpathlineto{\pgfqpoint{3.163045in}{2.884158in}}%
\pgfpathclose%
\pgfusepath{fill}%
\end{pgfscope}%
\begin{pgfscope}%
\pgfpathrectangle{\pgfqpoint{0.765000in}{0.660000in}}{\pgfqpoint{4.620000in}{4.620000in}}%
\pgfusepath{clip}%
\pgfsetbuttcap%
\pgfsetroundjoin%
\definecolor{currentfill}{rgb}{0.176471,0.192157,0.258824}%
\pgfsetfillcolor{currentfill}%
\pgfsetfillopacity{0.050000}%
\pgfsetlinewidth{0.000000pt}%
\definecolor{currentstroke}{rgb}{0.176471,0.192157,0.258824}%
\pgfsetstrokecolor{currentstroke}%
\pgfsetdash{}{0pt}%
\pgfpathmoveto{\pgfqpoint{3.318950in}{2.884158in}}%
\pgfpathlineto{\pgfqpoint{3.318950in}{2.957161in}}%
\pgfpathlineto{\pgfqpoint{3.240997in}{2.920565in}}%
\pgfpathlineto{\pgfqpoint{3.240997in}{2.847561in}}%
\pgfpathlineto{\pgfqpoint{3.318950in}{2.884158in}}%
\pgfpathclose%
\pgfusepath{fill}%
\end{pgfscope}%
\begin{pgfscope}%
\pgfpathrectangle{\pgfqpoint{0.765000in}{0.660000in}}{\pgfqpoint{4.620000in}{4.620000in}}%
\pgfusepath{clip}%
\pgfsetbuttcap%
\pgfsetroundjoin%
\definecolor{currentfill}{rgb}{0.176471,0.192157,0.258824}%
\pgfsetfillcolor{currentfill}%
\pgfsetfillopacity{0.050000}%
\pgfsetlinewidth{0.000000pt}%
\definecolor{currentstroke}{rgb}{0.176471,0.192157,0.258824}%
\pgfsetstrokecolor{currentstroke}%
\pgfsetdash{}{0pt}%
\pgfpathmoveto{\pgfqpoint{3.102685in}{2.899553in}}%
\pgfpathlineto{\pgfqpoint{3.102685in}{2.972556in}}%
\pgfpathlineto{\pgfqpoint{3.180637in}{2.935960in}}%
\pgfpathlineto{\pgfqpoint{3.180637in}{2.862956in}}%
\pgfpathlineto{\pgfqpoint{3.102685in}{2.899553in}}%
\pgfpathclose%
\pgfusepath{fill}%
\end{pgfscope}%
\begin{pgfscope}%
\pgfpathrectangle{\pgfqpoint{0.765000in}{0.660000in}}{\pgfqpoint{4.620000in}{4.620000in}}%
\pgfusepath{clip}%
\pgfsetbuttcap%
\pgfsetroundjoin%
\definecolor{currentfill}{rgb}{0.176471,0.192157,0.258824}%
\pgfsetfillcolor{currentfill}%
\pgfsetfillopacity{0.050000}%
\pgfsetlinewidth{0.000000pt}%
\definecolor{currentstroke}{rgb}{0.176471,0.192157,0.258824}%
\pgfsetstrokecolor{currentstroke}%
\pgfsetdash{}{0pt}%
\pgfpathmoveto{\pgfqpoint{3.102685in}{2.899553in}}%
\pgfpathlineto{\pgfqpoint{3.102685in}{2.972556in}}%
\pgfpathlineto{\pgfqpoint{3.024732in}{2.935960in}}%
\pgfpathlineto{\pgfqpoint{3.024732in}{2.862956in}}%
\pgfpathlineto{\pgfqpoint{3.102685in}{2.899553in}}%
\pgfpathclose%
\pgfusepath{fill}%
\end{pgfscope}%
\begin{pgfscope}%
\pgfpathrectangle{\pgfqpoint{0.765000in}{0.660000in}}{\pgfqpoint{4.620000in}{4.620000in}}%
\pgfusepath{clip}%
\pgfsetbuttcap%
\pgfsetroundjoin%
\definecolor{currentfill}{rgb}{0.176471,0.192157,0.258824}%
\pgfsetfillcolor{currentfill}%
\pgfsetfillopacity{0.050000}%
\pgfsetlinewidth{0.000000pt}%
\definecolor{currentstroke}{rgb}{0.176471,0.192157,0.258824}%
\pgfsetstrokecolor{currentstroke}%
\pgfsetdash{}{0pt}%
\pgfpathmoveto{\pgfqpoint{3.102685in}{2.899553in}}%
\pgfpathlineto{\pgfqpoint{3.180637in}{2.862956in}}%
\pgfpathlineto{\pgfqpoint{3.102685in}{2.826360in}}%
\pgfpathlineto{\pgfqpoint{3.024732in}{2.862956in}}%
\pgfpathlineto{\pgfqpoint{3.102685in}{2.899553in}}%
\pgfpathclose%
\pgfusepath{fill}%
\end{pgfscope}%
\begin{pgfscope}%
\pgfpathrectangle{\pgfqpoint{0.765000in}{0.660000in}}{\pgfqpoint{4.620000in}{4.620000in}}%
\pgfusepath{clip}%
\pgfsetbuttcap%
\pgfsetroundjoin%
\definecolor{currentfill}{rgb}{0.176471,0.192157,0.258824}%
\pgfsetfillcolor{currentfill}%
\pgfsetfillopacity{0.050000}%
\pgfsetlinewidth{0.000000pt}%
\definecolor{currentstroke}{rgb}{0.176471,0.192157,0.258824}%
\pgfsetstrokecolor{currentstroke}%
\pgfsetdash{}{0pt}%
\pgfpathmoveto{\pgfqpoint{3.102685in}{2.972556in}}%
\pgfpathlineto{\pgfqpoint{3.180637in}{2.935960in}}%
\pgfpathlineto{\pgfqpoint{3.102685in}{2.899363in}}%
\pgfpathlineto{\pgfqpoint{3.024732in}{2.935960in}}%
\pgfpathlineto{\pgfqpoint{3.102685in}{2.972556in}}%
\pgfpathclose%
\pgfusepath{fill}%
\end{pgfscope}%
\begin{pgfscope}%
\pgfpathrectangle{\pgfqpoint{0.765000in}{0.660000in}}{\pgfqpoint{4.620000in}{4.620000in}}%
\pgfusepath{clip}%
\pgfsetbuttcap%
\pgfsetroundjoin%
\definecolor{currentfill}{rgb}{0.176471,0.192157,0.258824}%
\pgfsetfillcolor{currentfill}%
\pgfsetfillopacity{0.050000}%
\pgfsetlinewidth{0.000000pt}%
\definecolor{currentstroke}{rgb}{0.176471,0.192157,0.258824}%
\pgfsetstrokecolor{currentstroke}%
\pgfsetdash{}{0pt}%
\pgfpathmoveto{\pgfqpoint{3.024732in}{2.862956in}}%
\pgfpathlineto{\pgfqpoint{3.024732in}{2.935960in}}%
\pgfpathlineto{\pgfqpoint{3.102685in}{2.899363in}}%
\pgfpathlineto{\pgfqpoint{3.102685in}{2.826360in}}%
\pgfpathlineto{\pgfqpoint{3.024732in}{2.862956in}}%
\pgfpathclose%
\pgfusepath{fill}%
\end{pgfscope}%
\begin{pgfscope}%
\pgfpathrectangle{\pgfqpoint{0.765000in}{0.660000in}}{\pgfqpoint{4.620000in}{4.620000in}}%
\pgfusepath{clip}%
\pgfsetbuttcap%
\pgfsetroundjoin%
\definecolor{currentfill}{rgb}{0.176471,0.192157,0.258824}%
\pgfsetfillcolor{currentfill}%
\pgfsetfillopacity{0.050000}%
\pgfsetlinewidth{0.000000pt}%
\definecolor{currentstroke}{rgb}{0.176471,0.192157,0.258824}%
\pgfsetstrokecolor{currentstroke}%
\pgfsetdash{}{0pt}%
\pgfpathmoveto{\pgfqpoint{3.180637in}{2.862956in}}%
\pgfpathlineto{\pgfqpoint{3.180637in}{2.935960in}}%
\pgfpathlineto{\pgfqpoint{3.102685in}{2.899363in}}%
\pgfpathlineto{\pgfqpoint{3.102685in}{2.826360in}}%
\pgfpathlineto{\pgfqpoint{3.180637in}{2.862956in}}%
\pgfpathclose%
\pgfusepath{fill}%
\end{pgfscope}%
\begin{pgfscope}%
\pgfpathrectangle{\pgfqpoint{0.765000in}{0.660000in}}{\pgfqpoint{4.620000in}{4.620000in}}%
\pgfusepath{clip}%
\pgfsetbuttcap%
\pgfsetroundjoin%
\definecolor{currentfill}{rgb}{0.176471,0.192157,0.258824}%
\pgfsetfillcolor{currentfill}%
\pgfsetfillopacity{0.050000}%
\pgfsetlinewidth{0.000000pt}%
\definecolor{currentstroke}{rgb}{0.176471,0.192157,0.258824}%
\pgfsetstrokecolor{currentstroke}%
\pgfsetdash{}{0pt}%
\pgfpathmoveto{\pgfqpoint{3.145205in}{2.871256in}}%
\pgfpathlineto{\pgfqpoint{3.145205in}{2.944260in}}%
\pgfpathlineto{\pgfqpoint{3.223158in}{2.907663in}}%
\pgfpathlineto{\pgfqpoint{3.223158in}{2.834660in}}%
\pgfpathlineto{\pgfqpoint{3.145205in}{2.871256in}}%
\pgfpathclose%
\pgfusepath{fill}%
\end{pgfscope}%
\begin{pgfscope}%
\pgfpathrectangle{\pgfqpoint{0.765000in}{0.660000in}}{\pgfqpoint{4.620000in}{4.620000in}}%
\pgfusepath{clip}%
\pgfsetbuttcap%
\pgfsetroundjoin%
\definecolor{currentfill}{rgb}{0.176471,0.192157,0.258824}%
\pgfsetfillcolor{currentfill}%
\pgfsetfillopacity{0.050000}%
\pgfsetlinewidth{0.000000pt}%
\definecolor{currentstroke}{rgb}{0.176471,0.192157,0.258824}%
\pgfsetstrokecolor{currentstroke}%
\pgfsetdash{}{0pt}%
\pgfpathmoveto{\pgfqpoint{3.145205in}{2.871256in}}%
\pgfpathlineto{\pgfqpoint{3.145205in}{2.944260in}}%
\pgfpathlineto{\pgfqpoint{3.067252in}{2.907663in}}%
\pgfpathlineto{\pgfqpoint{3.067252in}{2.834660in}}%
\pgfpathlineto{\pgfqpoint{3.145205in}{2.871256in}}%
\pgfpathclose%
\pgfusepath{fill}%
\end{pgfscope}%
\begin{pgfscope}%
\pgfpathrectangle{\pgfqpoint{0.765000in}{0.660000in}}{\pgfqpoint{4.620000in}{4.620000in}}%
\pgfusepath{clip}%
\pgfsetbuttcap%
\pgfsetroundjoin%
\definecolor{currentfill}{rgb}{0.176471,0.192157,0.258824}%
\pgfsetfillcolor{currentfill}%
\pgfsetfillopacity{0.050000}%
\pgfsetlinewidth{0.000000pt}%
\definecolor{currentstroke}{rgb}{0.176471,0.192157,0.258824}%
\pgfsetstrokecolor{currentstroke}%
\pgfsetdash{}{0pt}%
\pgfpathmoveto{\pgfqpoint{3.145205in}{2.871256in}}%
\pgfpathlineto{\pgfqpoint{3.223158in}{2.834660in}}%
\pgfpathlineto{\pgfqpoint{3.145205in}{2.798063in}}%
\pgfpathlineto{\pgfqpoint{3.067252in}{2.834660in}}%
\pgfpathlineto{\pgfqpoint{3.145205in}{2.871256in}}%
\pgfpathclose%
\pgfusepath{fill}%
\end{pgfscope}%
\begin{pgfscope}%
\pgfpathrectangle{\pgfqpoint{0.765000in}{0.660000in}}{\pgfqpoint{4.620000in}{4.620000in}}%
\pgfusepath{clip}%
\pgfsetbuttcap%
\pgfsetroundjoin%
\definecolor{currentfill}{rgb}{0.176471,0.192157,0.258824}%
\pgfsetfillcolor{currentfill}%
\pgfsetfillopacity{0.050000}%
\pgfsetlinewidth{0.000000pt}%
\definecolor{currentstroke}{rgb}{0.176471,0.192157,0.258824}%
\pgfsetstrokecolor{currentstroke}%
\pgfsetdash{}{0pt}%
\pgfpathmoveto{\pgfqpoint{3.145205in}{2.944260in}}%
\pgfpathlineto{\pgfqpoint{3.223158in}{2.907663in}}%
\pgfpathlineto{\pgfqpoint{3.145205in}{2.871067in}}%
\pgfpathlineto{\pgfqpoint{3.067252in}{2.907663in}}%
\pgfpathlineto{\pgfqpoint{3.145205in}{2.944260in}}%
\pgfpathclose%
\pgfusepath{fill}%
\end{pgfscope}%
\begin{pgfscope}%
\pgfpathrectangle{\pgfqpoint{0.765000in}{0.660000in}}{\pgfqpoint{4.620000in}{4.620000in}}%
\pgfusepath{clip}%
\pgfsetbuttcap%
\pgfsetroundjoin%
\definecolor{currentfill}{rgb}{0.176471,0.192157,0.258824}%
\pgfsetfillcolor{currentfill}%
\pgfsetfillopacity{0.050000}%
\pgfsetlinewidth{0.000000pt}%
\definecolor{currentstroke}{rgb}{0.176471,0.192157,0.258824}%
\pgfsetstrokecolor{currentstroke}%
\pgfsetdash{}{0pt}%
\pgfpathmoveto{\pgfqpoint{3.067252in}{2.834660in}}%
\pgfpathlineto{\pgfqpoint{3.067252in}{2.907663in}}%
\pgfpathlineto{\pgfqpoint{3.145205in}{2.871067in}}%
\pgfpathlineto{\pgfqpoint{3.145205in}{2.798063in}}%
\pgfpathlineto{\pgfqpoint{3.067252in}{2.834660in}}%
\pgfpathclose%
\pgfusepath{fill}%
\end{pgfscope}%
\begin{pgfscope}%
\pgfpathrectangle{\pgfqpoint{0.765000in}{0.660000in}}{\pgfqpoint{4.620000in}{4.620000in}}%
\pgfusepath{clip}%
\pgfsetbuttcap%
\pgfsetroundjoin%
\definecolor{currentfill}{rgb}{0.176471,0.192157,0.258824}%
\pgfsetfillcolor{currentfill}%
\pgfsetfillopacity{0.050000}%
\pgfsetlinewidth{0.000000pt}%
\definecolor{currentstroke}{rgb}{0.176471,0.192157,0.258824}%
\pgfsetstrokecolor{currentstroke}%
\pgfsetdash{}{0pt}%
\pgfpathmoveto{\pgfqpoint{3.223158in}{2.834660in}}%
\pgfpathlineto{\pgfqpoint{3.223158in}{2.907663in}}%
\pgfpathlineto{\pgfqpoint{3.145205in}{2.871067in}}%
\pgfpathlineto{\pgfqpoint{3.145205in}{2.798063in}}%
\pgfpathlineto{\pgfqpoint{3.223158in}{2.834660in}}%
\pgfpathclose%
\pgfusepath{fill}%
\end{pgfscope}%
\begin{pgfscope}%
\pgfpathrectangle{\pgfqpoint{0.765000in}{0.660000in}}{\pgfqpoint{4.620000in}{4.620000in}}%
\pgfusepath{clip}%
\pgfsetbuttcap%
\pgfsetroundjoin%
\definecolor{currentfill}{rgb}{0.176471,0.192157,0.258824}%
\pgfsetfillcolor{currentfill}%
\pgfsetfillopacity{0.050000}%
\pgfsetlinewidth{0.000000pt}%
\definecolor{currentstroke}{rgb}{0.176471,0.192157,0.258824}%
\pgfsetstrokecolor{currentstroke}%
\pgfsetdash{}{0pt}%
\pgfpathmoveto{\pgfqpoint{3.159844in}{3.032945in}}%
\pgfpathlineto{\pgfqpoint{3.159844in}{3.105948in}}%
\pgfpathlineto{\pgfqpoint{3.237797in}{3.069352in}}%
\pgfpathlineto{\pgfqpoint{3.237797in}{2.996348in}}%
\pgfpathlineto{\pgfqpoint{3.159844in}{3.032945in}}%
\pgfpathclose%
\pgfusepath{fill}%
\end{pgfscope}%
\begin{pgfscope}%
\pgfpathrectangle{\pgfqpoint{0.765000in}{0.660000in}}{\pgfqpoint{4.620000in}{4.620000in}}%
\pgfusepath{clip}%
\pgfsetbuttcap%
\pgfsetroundjoin%
\definecolor{currentfill}{rgb}{0.176471,0.192157,0.258824}%
\pgfsetfillcolor{currentfill}%
\pgfsetfillopacity{0.050000}%
\pgfsetlinewidth{0.000000pt}%
\definecolor{currentstroke}{rgb}{0.176471,0.192157,0.258824}%
\pgfsetstrokecolor{currentstroke}%
\pgfsetdash{}{0pt}%
\pgfpathmoveto{\pgfqpoint{3.159844in}{3.032945in}}%
\pgfpathlineto{\pgfqpoint{3.159844in}{3.105948in}}%
\pgfpathlineto{\pgfqpoint{3.081892in}{3.069352in}}%
\pgfpathlineto{\pgfqpoint{3.081892in}{2.996348in}}%
\pgfpathlineto{\pgfqpoint{3.159844in}{3.032945in}}%
\pgfpathclose%
\pgfusepath{fill}%
\end{pgfscope}%
\begin{pgfscope}%
\pgfpathrectangle{\pgfqpoint{0.765000in}{0.660000in}}{\pgfqpoint{4.620000in}{4.620000in}}%
\pgfusepath{clip}%
\pgfsetbuttcap%
\pgfsetroundjoin%
\definecolor{currentfill}{rgb}{0.176471,0.192157,0.258824}%
\pgfsetfillcolor{currentfill}%
\pgfsetfillopacity{0.050000}%
\pgfsetlinewidth{0.000000pt}%
\definecolor{currentstroke}{rgb}{0.176471,0.192157,0.258824}%
\pgfsetstrokecolor{currentstroke}%
\pgfsetdash{}{0pt}%
\pgfpathmoveto{\pgfqpoint{3.159844in}{3.032945in}}%
\pgfpathlineto{\pgfqpoint{3.237797in}{2.996348in}}%
\pgfpathlineto{\pgfqpoint{3.159844in}{2.959752in}}%
\pgfpathlineto{\pgfqpoint{3.081892in}{2.996348in}}%
\pgfpathlineto{\pgfqpoint{3.159844in}{3.032945in}}%
\pgfpathclose%
\pgfusepath{fill}%
\end{pgfscope}%
\begin{pgfscope}%
\pgfpathrectangle{\pgfqpoint{0.765000in}{0.660000in}}{\pgfqpoint{4.620000in}{4.620000in}}%
\pgfusepath{clip}%
\pgfsetbuttcap%
\pgfsetroundjoin%
\definecolor{currentfill}{rgb}{0.176471,0.192157,0.258824}%
\pgfsetfillcolor{currentfill}%
\pgfsetfillopacity{0.050000}%
\pgfsetlinewidth{0.000000pt}%
\definecolor{currentstroke}{rgb}{0.176471,0.192157,0.258824}%
\pgfsetstrokecolor{currentstroke}%
\pgfsetdash{}{0pt}%
\pgfpathmoveto{\pgfqpoint{3.159844in}{3.105948in}}%
\pgfpathlineto{\pgfqpoint{3.237797in}{3.069352in}}%
\pgfpathlineto{\pgfqpoint{3.159844in}{3.032755in}}%
\pgfpathlineto{\pgfqpoint{3.081892in}{3.069352in}}%
\pgfpathlineto{\pgfqpoint{3.159844in}{3.105948in}}%
\pgfpathclose%
\pgfusepath{fill}%
\end{pgfscope}%
\begin{pgfscope}%
\pgfpathrectangle{\pgfqpoint{0.765000in}{0.660000in}}{\pgfqpoint{4.620000in}{4.620000in}}%
\pgfusepath{clip}%
\pgfsetbuttcap%
\pgfsetroundjoin%
\definecolor{currentfill}{rgb}{0.176471,0.192157,0.258824}%
\pgfsetfillcolor{currentfill}%
\pgfsetfillopacity{0.050000}%
\pgfsetlinewidth{0.000000pt}%
\definecolor{currentstroke}{rgb}{0.176471,0.192157,0.258824}%
\pgfsetstrokecolor{currentstroke}%
\pgfsetdash{}{0pt}%
\pgfpathmoveto{\pgfqpoint{3.081892in}{2.996348in}}%
\pgfpathlineto{\pgfqpoint{3.081892in}{3.069352in}}%
\pgfpathlineto{\pgfqpoint{3.159844in}{3.032755in}}%
\pgfpathlineto{\pgfqpoint{3.159844in}{2.959752in}}%
\pgfpathlineto{\pgfqpoint{3.081892in}{2.996348in}}%
\pgfpathclose%
\pgfusepath{fill}%
\end{pgfscope}%
\begin{pgfscope}%
\pgfpathrectangle{\pgfqpoint{0.765000in}{0.660000in}}{\pgfqpoint{4.620000in}{4.620000in}}%
\pgfusepath{clip}%
\pgfsetbuttcap%
\pgfsetroundjoin%
\definecolor{currentfill}{rgb}{0.176471,0.192157,0.258824}%
\pgfsetfillcolor{currentfill}%
\pgfsetfillopacity{0.050000}%
\pgfsetlinewidth{0.000000pt}%
\definecolor{currentstroke}{rgb}{0.176471,0.192157,0.258824}%
\pgfsetstrokecolor{currentstroke}%
\pgfsetdash{}{0pt}%
\pgfpathmoveto{\pgfqpoint{3.237797in}{2.996348in}}%
\pgfpathlineto{\pgfqpoint{3.237797in}{3.069352in}}%
\pgfpathlineto{\pgfqpoint{3.159844in}{3.032755in}}%
\pgfpathlineto{\pgfqpoint{3.159844in}{2.959752in}}%
\pgfpathlineto{\pgfqpoint{3.237797in}{2.996348in}}%
\pgfpathclose%
\pgfusepath{fill}%
\end{pgfscope}%
\begin{pgfscope}%
\pgfpathrectangle{\pgfqpoint{0.765000in}{0.660000in}}{\pgfqpoint{4.620000in}{4.620000in}}%
\pgfusepath{clip}%
\pgfsetbuttcap%
\pgfsetroundjoin%
\definecolor{currentfill}{rgb}{0.176471,0.192157,0.258824}%
\pgfsetfillcolor{currentfill}%
\pgfsetfillopacity{0.050000}%
\pgfsetlinewidth{0.000000pt}%
\definecolor{currentstroke}{rgb}{0.176471,0.192157,0.258824}%
\pgfsetstrokecolor{currentstroke}%
\pgfsetdash{}{0pt}%
\pgfpathmoveto{\pgfqpoint{3.194656in}{2.829764in}}%
\pgfpathlineto{\pgfqpoint{3.194656in}{2.902767in}}%
\pgfpathlineto{\pgfqpoint{3.272608in}{2.866171in}}%
\pgfpathlineto{\pgfqpoint{3.272608in}{2.793168in}}%
\pgfpathlineto{\pgfqpoint{3.194656in}{2.829764in}}%
\pgfpathclose%
\pgfusepath{fill}%
\end{pgfscope}%
\begin{pgfscope}%
\pgfpathrectangle{\pgfqpoint{0.765000in}{0.660000in}}{\pgfqpoint{4.620000in}{4.620000in}}%
\pgfusepath{clip}%
\pgfsetbuttcap%
\pgfsetroundjoin%
\definecolor{currentfill}{rgb}{0.176471,0.192157,0.258824}%
\pgfsetfillcolor{currentfill}%
\pgfsetfillopacity{0.050000}%
\pgfsetlinewidth{0.000000pt}%
\definecolor{currentstroke}{rgb}{0.176471,0.192157,0.258824}%
\pgfsetstrokecolor{currentstroke}%
\pgfsetdash{}{0pt}%
\pgfpathmoveto{\pgfqpoint{3.194656in}{2.829764in}}%
\pgfpathlineto{\pgfqpoint{3.194656in}{2.902767in}}%
\pgfpathlineto{\pgfqpoint{3.116703in}{2.866171in}}%
\pgfpathlineto{\pgfqpoint{3.116703in}{2.793168in}}%
\pgfpathlineto{\pgfqpoint{3.194656in}{2.829764in}}%
\pgfpathclose%
\pgfusepath{fill}%
\end{pgfscope}%
\begin{pgfscope}%
\pgfpathrectangle{\pgfqpoint{0.765000in}{0.660000in}}{\pgfqpoint{4.620000in}{4.620000in}}%
\pgfusepath{clip}%
\pgfsetbuttcap%
\pgfsetroundjoin%
\definecolor{currentfill}{rgb}{0.176471,0.192157,0.258824}%
\pgfsetfillcolor{currentfill}%
\pgfsetfillopacity{0.050000}%
\pgfsetlinewidth{0.000000pt}%
\definecolor{currentstroke}{rgb}{0.176471,0.192157,0.258824}%
\pgfsetstrokecolor{currentstroke}%
\pgfsetdash{}{0pt}%
\pgfpathmoveto{\pgfqpoint{3.194656in}{2.829764in}}%
\pgfpathlineto{\pgfqpoint{3.272608in}{2.793168in}}%
\pgfpathlineto{\pgfqpoint{3.194656in}{2.756571in}}%
\pgfpathlineto{\pgfqpoint{3.116703in}{2.793168in}}%
\pgfpathlineto{\pgfqpoint{3.194656in}{2.829764in}}%
\pgfpathclose%
\pgfusepath{fill}%
\end{pgfscope}%
\begin{pgfscope}%
\pgfpathrectangle{\pgfqpoint{0.765000in}{0.660000in}}{\pgfqpoint{4.620000in}{4.620000in}}%
\pgfusepath{clip}%
\pgfsetbuttcap%
\pgfsetroundjoin%
\definecolor{currentfill}{rgb}{0.176471,0.192157,0.258824}%
\pgfsetfillcolor{currentfill}%
\pgfsetfillopacity{0.050000}%
\pgfsetlinewidth{0.000000pt}%
\definecolor{currentstroke}{rgb}{0.176471,0.192157,0.258824}%
\pgfsetstrokecolor{currentstroke}%
\pgfsetdash{}{0pt}%
\pgfpathmoveto{\pgfqpoint{3.194656in}{2.902767in}}%
\pgfpathlineto{\pgfqpoint{3.272608in}{2.866171in}}%
\pgfpathlineto{\pgfqpoint{3.194656in}{2.829574in}}%
\pgfpathlineto{\pgfqpoint{3.116703in}{2.866171in}}%
\pgfpathlineto{\pgfqpoint{3.194656in}{2.902767in}}%
\pgfpathclose%
\pgfusepath{fill}%
\end{pgfscope}%
\begin{pgfscope}%
\pgfpathrectangle{\pgfqpoint{0.765000in}{0.660000in}}{\pgfqpoint{4.620000in}{4.620000in}}%
\pgfusepath{clip}%
\pgfsetbuttcap%
\pgfsetroundjoin%
\definecolor{currentfill}{rgb}{0.176471,0.192157,0.258824}%
\pgfsetfillcolor{currentfill}%
\pgfsetfillopacity{0.050000}%
\pgfsetlinewidth{0.000000pt}%
\definecolor{currentstroke}{rgb}{0.176471,0.192157,0.258824}%
\pgfsetstrokecolor{currentstroke}%
\pgfsetdash{}{0pt}%
\pgfpathmoveto{\pgfqpoint{3.116703in}{2.793168in}}%
\pgfpathlineto{\pgfqpoint{3.116703in}{2.866171in}}%
\pgfpathlineto{\pgfqpoint{3.194656in}{2.829574in}}%
\pgfpathlineto{\pgfqpoint{3.194656in}{2.756571in}}%
\pgfpathlineto{\pgfqpoint{3.116703in}{2.793168in}}%
\pgfpathclose%
\pgfusepath{fill}%
\end{pgfscope}%
\begin{pgfscope}%
\pgfpathrectangle{\pgfqpoint{0.765000in}{0.660000in}}{\pgfqpoint{4.620000in}{4.620000in}}%
\pgfusepath{clip}%
\pgfsetbuttcap%
\pgfsetroundjoin%
\definecolor{currentfill}{rgb}{0.176471,0.192157,0.258824}%
\pgfsetfillcolor{currentfill}%
\pgfsetfillopacity{0.050000}%
\pgfsetlinewidth{0.000000pt}%
\definecolor{currentstroke}{rgb}{0.176471,0.192157,0.258824}%
\pgfsetstrokecolor{currentstroke}%
\pgfsetdash{}{0pt}%
\pgfpathmoveto{\pgfqpoint{3.272608in}{2.793168in}}%
\pgfpathlineto{\pgfqpoint{3.272608in}{2.866171in}}%
\pgfpathlineto{\pgfqpoint{3.194656in}{2.829574in}}%
\pgfpathlineto{\pgfqpoint{3.194656in}{2.756571in}}%
\pgfpathlineto{\pgfqpoint{3.272608in}{2.793168in}}%
\pgfpathclose%
\pgfusepath{fill}%
\end{pgfscope}%
\begin{pgfscope}%
\pgfpathrectangle{\pgfqpoint{0.765000in}{0.660000in}}{\pgfqpoint{4.620000in}{4.620000in}}%
\pgfusepath{clip}%
\pgfsetbuttcap%
\pgfsetroundjoin%
\definecolor{currentfill}{rgb}{0.176471,0.192157,0.258824}%
\pgfsetfillcolor{currentfill}%
\pgfsetfillopacity{0.050000}%
\pgfsetlinewidth{0.000000pt}%
\definecolor{currentstroke}{rgb}{0.176471,0.192157,0.258824}%
\pgfsetstrokecolor{currentstroke}%
\pgfsetdash{}{0pt}%
\pgfpathmoveto{\pgfqpoint{3.147663in}{3.037703in}}%
\pgfpathlineto{\pgfqpoint{3.147663in}{3.110706in}}%
\pgfpathlineto{\pgfqpoint{3.225616in}{3.074109in}}%
\pgfpathlineto{\pgfqpoint{3.225616in}{3.001106in}}%
\pgfpathlineto{\pgfqpoint{3.147663in}{3.037703in}}%
\pgfpathclose%
\pgfusepath{fill}%
\end{pgfscope}%
\begin{pgfscope}%
\pgfpathrectangle{\pgfqpoint{0.765000in}{0.660000in}}{\pgfqpoint{4.620000in}{4.620000in}}%
\pgfusepath{clip}%
\pgfsetbuttcap%
\pgfsetroundjoin%
\definecolor{currentfill}{rgb}{0.176471,0.192157,0.258824}%
\pgfsetfillcolor{currentfill}%
\pgfsetfillopacity{0.050000}%
\pgfsetlinewidth{0.000000pt}%
\definecolor{currentstroke}{rgb}{0.176471,0.192157,0.258824}%
\pgfsetstrokecolor{currentstroke}%
\pgfsetdash{}{0pt}%
\pgfpathmoveto{\pgfqpoint{3.147663in}{3.037703in}}%
\pgfpathlineto{\pgfqpoint{3.147663in}{3.110706in}}%
\pgfpathlineto{\pgfqpoint{3.069710in}{3.074109in}}%
\pgfpathlineto{\pgfqpoint{3.069710in}{3.001106in}}%
\pgfpathlineto{\pgfqpoint{3.147663in}{3.037703in}}%
\pgfpathclose%
\pgfusepath{fill}%
\end{pgfscope}%
\begin{pgfscope}%
\pgfpathrectangle{\pgfqpoint{0.765000in}{0.660000in}}{\pgfqpoint{4.620000in}{4.620000in}}%
\pgfusepath{clip}%
\pgfsetbuttcap%
\pgfsetroundjoin%
\definecolor{currentfill}{rgb}{0.176471,0.192157,0.258824}%
\pgfsetfillcolor{currentfill}%
\pgfsetfillopacity{0.050000}%
\pgfsetlinewidth{0.000000pt}%
\definecolor{currentstroke}{rgb}{0.176471,0.192157,0.258824}%
\pgfsetstrokecolor{currentstroke}%
\pgfsetdash{}{0pt}%
\pgfpathmoveto{\pgfqpoint{3.147663in}{3.037703in}}%
\pgfpathlineto{\pgfqpoint{3.225616in}{3.001106in}}%
\pgfpathlineto{\pgfqpoint{3.147663in}{2.964509in}}%
\pgfpathlineto{\pgfqpoint{3.069710in}{3.001106in}}%
\pgfpathlineto{\pgfqpoint{3.147663in}{3.037703in}}%
\pgfpathclose%
\pgfusepath{fill}%
\end{pgfscope}%
\begin{pgfscope}%
\pgfpathrectangle{\pgfqpoint{0.765000in}{0.660000in}}{\pgfqpoint{4.620000in}{4.620000in}}%
\pgfusepath{clip}%
\pgfsetbuttcap%
\pgfsetroundjoin%
\definecolor{currentfill}{rgb}{0.176471,0.192157,0.258824}%
\pgfsetfillcolor{currentfill}%
\pgfsetfillopacity{0.050000}%
\pgfsetlinewidth{0.000000pt}%
\definecolor{currentstroke}{rgb}{0.176471,0.192157,0.258824}%
\pgfsetstrokecolor{currentstroke}%
\pgfsetdash{}{0pt}%
\pgfpathmoveto{\pgfqpoint{3.147663in}{3.110706in}}%
\pgfpathlineto{\pgfqpoint{3.225616in}{3.074109in}}%
\pgfpathlineto{\pgfqpoint{3.147663in}{3.037513in}}%
\pgfpathlineto{\pgfqpoint{3.069710in}{3.074109in}}%
\pgfpathlineto{\pgfqpoint{3.147663in}{3.110706in}}%
\pgfpathclose%
\pgfusepath{fill}%
\end{pgfscope}%
\begin{pgfscope}%
\pgfpathrectangle{\pgfqpoint{0.765000in}{0.660000in}}{\pgfqpoint{4.620000in}{4.620000in}}%
\pgfusepath{clip}%
\pgfsetbuttcap%
\pgfsetroundjoin%
\definecolor{currentfill}{rgb}{0.176471,0.192157,0.258824}%
\pgfsetfillcolor{currentfill}%
\pgfsetfillopacity{0.050000}%
\pgfsetlinewidth{0.000000pt}%
\definecolor{currentstroke}{rgb}{0.176471,0.192157,0.258824}%
\pgfsetstrokecolor{currentstroke}%
\pgfsetdash{}{0pt}%
\pgfpathmoveto{\pgfqpoint{3.069710in}{3.001106in}}%
\pgfpathlineto{\pgfqpoint{3.069710in}{3.074109in}}%
\pgfpathlineto{\pgfqpoint{3.147663in}{3.037513in}}%
\pgfpathlineto{\pgfqpoint{3.147663in}{2.964509in}}%
\pgfpathlineto{\pgfqpoint{3.069710in}{3.001106in}}%
\pgfpathclose%
\pgfusepath{fill}%
\end{pgfscope}%
\begin{pgfscope}%
\pgfpathrectangle{\pgfqpoint{0.765000in}{0.660000in}}{\pgfqpoint{4.620000in}{4.620000in}}%
\pgfusepath{clip}%
\pgfsetbuttcap%
\pgfsetroundjoin%
\definecolor{currentfill}{rgb}{0.176471,0.192157,0.258824}%
\pgfsetfillcolor{currentfill}%
\pgfsetfillopacity{0.050000}%
\pgfsetlinewidth{0.000000pt}%
\definecolor{currentstroke}{rgb}{0.176471,0.192157,0.258824}%
\pgfsetstrokecolor{currentstroke}%
\pgfsetdash{}{0pt}%
\pgfpathmoveto{\pgfqpoint{3.225616in}{3.001106in}}%
\pgfpathlineto{\pgfqpoint{3.225616in}{3.074109in}}%
\pgfpathlineto{\pgfqpoint{3.147663in}{3.037513in}}%
\pgfpathlineto{\pgfqpoint{3.147663in}{2.964509in}}%
\pgfpathlineto{\pgfqpoint{3.225616in}{3.001106in}}%
\pgfpathclose%
\pgfusepath{fill}%
\end{pgfscope}%
\begin{pgfscope}%
\pgfpathrectangle{\pgfqpoint{0.765000in}{0.660000in}}{\pgfqpoint{4.620000in}{4.620000in}}%
\pgfusepath{clip}%
\pgfsetbuttcap%
\pgfsetroundjoin%
\definecolor{currentfill}{rgb}{0.176471,0.192157,0.258824}%
\pgfsetfillcolor{currentfill}%
\pgfsetfillopacity{0.050000}%
\pgfsetlinewidth{0.000000pt}%
\definecolor{currentstroke}{rgb}{0.176471,0.192157,0.258824}%
\pgfsetstrokecolor{currentstroke}%
\pgfsetdash{}{0pt}%
\pgfpathmoveto{\pgfqpoint{3.279497in}{2.967891in}}%
\pgfpathlineto{\pgfqpoint{3.279497in}{3.040895in}}%
\pgfpathlineto{\pgfqpoint{3.357449in}{3.004298in}}%
\pgfpathlineto{\pgfqpoint{3.357449in}{2.931295in}}%
\pgfpathlineto{\pgfqpoint{3.279497in}{2.967891in}}%
\pgfpathclose%
\pgfusepath{fill}%
\end{pgfscope}%
\begin{pgfscope}%
\pgfpathrectangle{\pgfqpoint{0.765000in}{0.660000in}}{\pgfqpoint{4.620000in}{4.620000in}}%
\pgfusepath{clip}%
\pgfsetbuttcap%
\pgfsetroundjoin%
\definecolor{currentfill}{rgb}{0.176471,0.192157,0.258824}%
\pgfsetfillcolor{currentfill}%
\pgfsetfillopacity{0.050000}%
\pgfsetlinewidth{0.000000pt}%
\definecolor{currentstroke}{rgb}{0.176471,0.192157,0.258824}%
\pgfsetstrokecolor{currentstroke}%
\pgfsetdash{}{0pt}%
\pgfpathmoveto{\pgfqpoint{3.279497in}{2.967891in}}%
\pgfpathlineto{\pgfqpoint{3.279497in}{3.040895in}}%
\pgfpathlineto{\pgfqpoint{3.201544in}{3.004298in}}%
\pgfpathlineto{\pgfqpoint{3.201544in}{2.931295in}}%
\pgfpathlineto{\pgfqpoint{3.279497in}{2.967891in}}%
\pgfpathclose%
\pgfusepath{fill}%
\end{pgfscope}%
\begin{pgfscope}%
\pgfpathrectangle{\pgfqpoint{0.765000in}{0.660000in}}{\pgfqpoint{4.620000in}{4.620000in}}%
\pgfusepath{clip}%
\pgfsetbuttcap%
\pgfsetroundjoin%
\definecolor{currentfill}{rgb}{0.176471,0.192157,0.258824}%
\pgfsetfillcolor{currentfill}%
\pgfsetfillopacity{0.050000}%
\pgfsetlinewidth{0.000000pt}%
\definecolor{currentstroke}{rgb}{0.176471,0.192157,0.258824}%
\pgfsetstrokecolor{currentstroke}%
\pgfsetdash{}{0pt}%
\pgfpathmoveto{\pgfqpoint{3.279497in}{2.967891in}}%
\pgfpathlineto{\pgfqpoint{3.357449in}{2.931295in}}%
\pgfpathlineto{\pgfqpoint{3.279497in}{2.894698in}}%
\pgfpathlineto{\pgfqpoint{3.201544in}{2.931295in}}%
\pgfpathlineto{\pgfqpoint{3.279497in}{2.967891in}}%
\pgfpathclose%
\pgfusepath{fill}%
\end{pgfscope}%
\begin{pgfscope}%
\pgfpathrectangle{\pgfqpoint{0.765000in}{0.660000in}}{\pgfqpoint{4.620000in}{4.620000in}}%
\pgfusepath{clip}%
\pgfsetbuttcap%
\pgfsetroundjoin%
\definecolor{currentfill}{rgb}{0.176471,0.192157,0.258824}%
\pgfsetfillcolor{currentfill}%
\pgfsetfillopacity{0.050000}%
\pgfsetlinewidth{0.000000pt}%
\definecolor{currentstroke}{rgb}{0.176471,0.192157,0.258824}%
\pgfsetstrokecolor{currentstroke}%
\pgfsetdash{}{0pt}%
\pgfpathmoveto{\pgfqpoint{3.279497in}{3.040895in}}%
\pgfpathlineto{\pgfqpoint{3.357449in}{3.004298in}}%
\pgfpathlineto{\pgfqpoint{3.279497in}{2.967701in}}%
\pgfpathlineto{\pgfqpoint{3.201544in}{3.004298in}}%
\pgfpathlineto{\pgfqpoint{3.279497in}{3.040895in}}%
\pgfpathclose%
\pgfusepath{fill}%
\end{pgfscope}%
\begin{pgfscope}%
\pgfpathrectangle{\pgfqpoint{0.765000in}{0.660000in}}{\pgfqpoint{4.620000in}{4.620000in}}%
\pgfusepath{clip}%
\pgfsetbuttcap%
\pgfsetroundjoin%
\definecolor{currentfill}{rgb}{0.176471,0.192157,0.258824}%
\pgfsetfillcolor{currentfill}%
\pgfsetfillopacity{0.050000}%
\pgfsetlinewidth{0.000000pt}%
\definecolor{currentstroke}{rgb}{0.176471,0.192157,0.258824}%
\pgfsetstrokecolor{currentstroke}%
\pgfsetdash{}{0pt}%
\pgfpathmoveto{\pgfqpoint{3.201544in}{2.931295in}}%
\pgfpathlineto{\pgfqpoint{3.201544in}{3.004298in}}%
\pgfpathlineto{\pgfqpoint{3.279497in}{2.967701in}}%
\pgfpathlineto{\pgfqpoint{3.279497in}{2.894698in}}%
\pgfpathlineto{\pgfqpoint{3.201544in}{2.931295in}}%
\pgfpathclose%
\pgfusepath{fill}%
\end{pgfscope}%
\begin{pgfscope}%
\pgfpathrectangle{\pgfqpoint{0.765000in}{0.660000in}}{\pgfqpoint{4.620000in}{4.620000in}}%
\pgfusepath{clip}%
\pgfsetbuttcap%
\pgfsetroundjoin%
\definecolor{currentfill}{rgb}{0.176471,0.192157,0.258824}%
\pgfsetfillcolor{currentfill}%
\pgfsetfillopacity{0.050000}%
\pgfsetlinewidth{0.000000pt}%
\definecolor{currentstroke}{rgb}{0.176471,0.192157,0.258824}%
\pgfsetstrokecolor{currentstroke}%
\pgfsetdash{}{0pt}%
\pgfpathmoveto{\pgfqpoint{3.357449in}{2.931295in}}%
\pgfpathlineto{\pgfqpoint{3.357449in}{3.004298in}}%
\pgfpathlineto{\pgfqpoint{3.279497in}{2.967701in}}%
\pgfpathlineto{\pgfqpoint{3.279497in}{2.894698in}}%
\pgfpathlineto{\pgfqpoint{3.357449in}{2.931295in}}%
\pgfpathclose%
\pgfusepath{fill}%
\end{pgfscope}%
\begin{pgfscope}%
\pgfpathrectangle{\pgfqpoint{0.765000in}{0.660000in}}{\pgfqpoint{4.620000in}{4.620000in}}%
\pgfusepath{clip}%
\pgfsetbuttcap%
\pgfsetroundjoin%
\definecolor{currentfill}{rgb}{0.176471,0.192157,0.258824}%
\pgfsetfillcolor{currentfill}%
\pgfsetfillopacity{0.050000}%
\pgfsetlinewidth{0.000000pt}%
\definecolor{currentstroke}{rgb}{0.176471,0.192157,0.258824}%
\pgfsetstrokecolor{currentstroke}%
\pgfsetdash{}{0pt}%
\pgfpathmoveto{\pgfqpoint{3.198633in}{2.902193in}}%
\pgfpathlineto{\pgfqpoint{3.198633in}{2.975196in}}%
\pgfpathlineto{\pgfqpoint{3.276586in}{2.938600in}}%
\pgfpathlineto{\pgfqpoint{3.276586in}{2.865596in}}%
\pgfpathlineto{\pgfqpoint{3.198633in}{2.902193in}}%
\pgfpathclose%
\pgfusepath{fill}%
\end{pgfscope}%
\begin{pgfscope}%
\pgfpathrectangle{\pgfqpoint{0.765000in}{0.660000in}}{\pgfqpoint{4.620000in}{4.620000in}}%
\pgfusepath{clip}%
\pgfsetbuttcap%
\pgfsetroundjoin%
\definecolor{currentfill}{rgb}{0.176471,0.192157,0.258824}%
\pgfsetfillcolor{currentfill}%
\pgfsetfillopacity{0.050000}%
\pgfsetlinewidth{0.000000pt}%
\definecolor{currentstroke}{rgb}{0.176471,0.192157,0.258824}%
\pgfsetstrokecolor{currentstroke}%
\pgfsetdash{}{0pt}%
\pgfpathmoveto{\pgfqpoint{3.198633in}{2.902193in}}%
\pgfpathlineto{\pgfqpoint{3.198633in}{2.975196in}}%
\pgfpathlineto{\pgfqpoint{3.120680in}{2.938600in}}%
\pgfpathlineto{\pgfqpoint{3.120680in}{2.865596in}}%
\pgfpathlineto{\pgfqpoint{3.198633in}{2.902193in}}%
\pgfpathclose%
\pgfusepath{fill}%
\end{pgfscope}%
\begin{pgfscope}%
\pgfpathrectangle{\pgfqpoint{0.765000in}{0.660000in}}{\pgfqpoint{4.620000in}{4.620000in}}%
\pgfusepath{clip}%
\pgfsetbuttcap%
\pgfsetroundjoin%
\definecolor{currentfill}{rgb}{0.176471,0.192157,0.258824}%
\pgfsetfillcolor{currentfill}%
\pgfsetfillopacity{0.050000}%
\pgfsetlinewidth{0.000000pt}%
\definecolor{currentstroke}{rgb}{0.176471,0.192157,0.258824}%
\pgfsetstrokecolor{currentstroke}%
\pgfsetdash{}{0pt}%
\pgfpathmoveto{\pgfqpoint{3.198633in}{2.902193in}}%
\pgfpathlineto{\pgfqpoint{3.276586in}{2.865596in}}%
\pgfpathlineto{\pgfqpoint{3.198633in}{2.829000in}}%
\pgfpathlineto{\pgfqpoint{3.120680in}{2.865596in}}%
\pgfpathlineto{\pgfqpoint{3.198633in}{2.902193in}}%
\pgfpathclose%
\pgfusepath{fill}%
\end{pgfscope}%
\begin{pgfscope}%
\pgfpathrectangle{\pgfqpoint{0.765000in}{0.660000in}}{\pgfqpoint{4.620000in}{4.620000in}}%
\pgfusepath{clip}%
\pgfsetbuttcap%
\pgfsetroundjoin%
\definecolor{currentfill}{rgb}{0.176471,0.192157,0.258824}%
\pgfsetfillcolor{currentfill}%
\pgfsetfillopacity{0.050000}%
\pgfsetlinewidth{0.000000pt}%
\definecolor{currentstroke}{rgb}{0.176471,0.192157,0.258824}%
\pgfsetstrokecolor{currentstroke}%
\pgfsetdash{}{0pt}%
\pgfpathmoveto{\pgfqpoint{3.198633in}{2.975196in}}%
\pgfpathlineto{\pgfqpoint{3.276586in}{2.938600in}}%
\pgfpathlineto{\pgfqpoint{3.198633in}{2.902003in}}%
\pgfpathlineto{\pgfqpoint{3.120680in}{2.938600in}}%
\pgfpathlineto{\pgfqpoint{3.198633in}{2.975196in}}%
\pgfpathclose%
\pgfusepath{fill}%
\end{pgfscope}%
\begin{pgfscope}%
\pgfpathrectangle{\pgfqpoint{0.765000in}{0.660000in}}{\pgfqpoint{4.620000in}{4.620000in}}%
\pgfusepath{clip}%
\pgfsetbuttcap%
\pgfsetroundjoin%
\definecolor{currentfill}{rgb}{0.176471,0.192157,0.258824}%
\pgfsetfillcolor{currentfill}%
\pgfsetfillopacity{0.050000}%
\pgfsetlinewidth{0.000000pt}%
\definecolor{currentstroke}{rgb}{0.176471,0.192157,0.258824}%
\pgfsetstrokecolor{currentstroke}%
\pgfsetdash{}{0pt}%
\pgfpathmoveto{\pgfqpoint{3.120680in}{2.865596in}}%
\pgfpathlineto{\pgfqpoint{3.120680in}{2.938600in}}%
\pgfpathlineto{\pgfqpoint{3.198633in}{2.902003in}}%
\pgfpathlineto{\pgfqpoint{3.198633in}{2.829000in}}%
\pgfpathlineto{\pgfqpoint{3.120680in}{2.865596in}}%
\pgfpathclose%
\pgfusepath{fill}%
\end{pgfscope}%
\begin{pgfscope}%
\pgfpathrectangle{\pgfqpoint{0.765000in}{0.660000in}}{\pgfqpoint{4.620000in}{4.620000in}}%
\pgfusepath{clip}%
\pgfsetbuttcap%
\pgfsetroundjoin%
\definecolor{currentfill}{rgb}{0.176471,0.192157,0.258824}%
\pgfsetfillcolor{currentfill}%
\pgfsetfillopacity{0.050000}%
\pgfsetlinewidth{0.000000pt}%
\definecolor{currentstroke}{rgb}{0.176471,0.192157,0.258824}%
\pgfsetstrokecolor{currentstroke}%
\pgfsetdash{}{0pt}%
\pgfpathmoveto{\pgfqpoint{3.276586in}{2.865596in}}%
\pgfpathlineto{\pgfqpoint{3.276586in}{2.938600in}}%
\pgfpathlineto{\pgfqpoint{3.198633in}{2.902003in}}%
\pgfpathlineto{\pgfqpoint{3.198633in}{2.829000in}}%
\pgfpathlineto{\pgfqpoint{3.276586in}{2.865596in}}%
\pgfpathclose%
\pgfusepath{fill}%
\end{pgfscope}%
\begin{pgfscope}%
\pgfpathrectangle{\pgfqpoint{0.765000in}{0.660000in}}{\pgfqpoint{4.620000in}{4.620000in}}%
\pgfusepath{clip}%
\pgfsetbuttcap%
\pgfsetroundjoin%
\definecolor{currentfill}{rgb}{0.176471,0.192157,0.258824}%
\pgfsetfillcolor{currentfill}%
\pgfsetfillopacity{0.050000}%
\pgfsetlinewidth{0.000000pt}%
\definecolor{currentstroke}{rgb}{0.176471,0.192157,0.258824}%
\pgfsetstrokecolor{currentstroke}%
\pgfsetdash{}{0pt}%
\pgfpathmoveto{\pgfqpoint{3.086016in}{3.018979in}}%
\pgfpathlineto{\pgfqpoint{3.086016in}{3.091983in}}%
\pgfpathlineto{\pgfqpoint{3.163968in}{3.055386in}}%
\pgfpathlineto{\pgfqpoint{3.163968in}{2.982383in}}%
\pgfpathlineto{\pgfqpoint{3.086016in}{3.018979in}}%
\pgfpathclose%
\pgfusepath{fill}%
\end{pgfscope}%
\begin{pgfscope}%
\pgfpathrectangle{\pgfqpoint{0.765000in}{0.660000in}}{\pgfqpoint{4.620000in}{4.620000in}}%
\pgfusepath{clip}%
\pgfsetbuttcap%
\pgfsetroundjoin%
\definecolor{currentfill}{rgb}{0.176471,0.192157,0.258824}%
\pgfsetfillcolor{currentfill}%
\pgfsetfillopacity{0.050000}%
\pgfsetlinewidth{0.000000pt}%
\definecolor{currentstroke}{rgb}{0.176471,0.192157,0.258824}%
\pgfsetstrokecolor{currentstroke}%
\pgfsetdash{}{0pt}%
\pgfpathmoveto{\pgfqpoint{3.086016in}{3.018979in}}%
\pgfpathlineto{\pgfqpoint{3.086016in}{3.091983in}}%
\pgfpathlineto{\pgfqpoint{3.008063in}{3.055386in}}%
\pgfpathlineto{\pgfqpoint{3.008063in}{2.982383in}}%
\pgfpathlineto{\pgfqpoint{3.086016in}{3.018979in}}%
\pgfpathclose%
\pgfusepath{fill}%
\end{pgfscope}%
\begin{pgfscope}%
\pgfpathrectangle{\pgfqpoint{0.765000in}{0.660000in}}{\pgfqpoint{4.620000in}{4.620000in}}%
\pgfusepath{clip}%
\pgfsetbuttcap%
\pgfsetroundjoin%
\definecolor{currentfill}{rgb}{0.176471,0.192157,0.258824}%
\pgfsetfillcolor{currentfill}%
\pgfsetfillopacity{0.050000}%
\pgfsetlinewidth{0.000000pt}%
\definecolor{currentstroke}{rgb}{0.176471,0.192157,0.258824}%
\pgfsetstrokecolor{currentstroke}%
\pgfsetdash{}{0pt}%
\pgfpathmoveto{\pgfqpoint{3.086016in}{3.018979in}}%
\pgfpathlineto{\pgfqpoint{3.163968in}{2.982383in}}%
\pgfpathlineto{\pgfqpoint{3.086016in}{2.945786in}}%
\pgfpathlineto{\pgfqpoint{3.008063in}{2.982383in}}%
\pgfpathlineto{\pgfqpoint{3.086016in}{3.018979in}}%
\pgfpathclose%
\pgfusepath{fill}%
\end{pgfscope}%
\begin{pgfscope}%
\pgfpathrectangle{\pgfqpoint{0.765000in}{0.660000in}}{\pgfqpoint{4.620000in}{4.620000in}}%
\pgfusepath{clip}%
\pgfsetbuttcap%
\pgfsetroundjoin%
\definecolor{currentfill}{rgb}{0.176471,0.192157,0.258824}%
\pgfsetfillcolor{currentfill}%
\pgfsetfillopacity{0.050000}%
\pgfsetlinewidth{0.000000pt}%
\definecolor{currentstroke}{rgb}{0.176471,0.192157,0.258824}%
\pgfsetstrokecolor{currentstroke}%
\pgfsetdash{}{0pt}%
\pgfpathmoveto{\pgfqpoint{3.086016in}{3.091983in}}%
\pgfpathlineto{\pgfqpoint{3.163968in}{3.055386in}}%
\pgfpathlineto{\pgfqpoint{3.086016in}{3.018790in}}%
\pgfpathlineto{\pgfqpoint{3.008063in}{3.055386in}}%
\pgfpathlineto{\pgfqpoint{3.086016in}{3.091983in}}%
\pgfpathclose%
\pgfusepath{fill}%
\end{pgfscope}%
\begin{pgfscope}%
\pgfpathrectangle{\pgfqpoint{0.765000in}{0.660000in}}{\pgfqpoint{4.620000in}{4.620000in}}%
\pgfusepath{clip}%
\pgfsetbuttcap%
\pgfsetroundjoin%
\definecolor{currentfill}{rgb}{0.176471,0.192157,0.258824}%
\pgfsetfillcolor{currentfill}%
\pgfsetfillopacity{0.050000}%
\pgfsetlinewidth{0.000000pt}%
\definecolor{currentstroke}{rgb}{0.176471,0.192157,0.258824}%
\pgfsetstrokecolor{currentstroke}%
\pgfsetdash{}{0pt}%
\pgfpathmoveto{\pgfqpoint{3.008063in}{2.982383in}}%
\pgfpathlineto{\pgfqpoint{3.008063in}{3.055386in}}%
\pgfpathlineto{\pgfqpoint{3.086016in}{3.018790in}}%
\pgfpathlineto{\pgfqpoint{3.086016in}{2.945786in}}%
\pgfpathlineto{\pgfqpoint{3.008063in}{2.982383in}}%
\pgfpathclose%
\pgfusepath{fill}%
\end{pgfscope}%
\begin{pgfscope}%
\pgfpathrectangle{\pgfqpoint{0.765000in}{0.660000in}}{\pgfqpoint{4.620000in}{4.620000in}}%
\pgfusepath{clip}%
\pgfsetbuttcap%
\pgfsetroundjoin%
\definecolor{currentfill}{rgb}{0.176471,0.192157,0.258824}%
\pgfsetfillcolor{currentfill}%
\pgfsetfillopacity{0.050000}%
\pgfsetlinewidth{0.000000pt}%
\definecolor{currentstroke}{rgb}{0.176471,0.192157,0.258824}%
\pgfsetstrokecolor{currentstroke}%
\pgfsetdash{}{0pt}%
\pgfpathmoveto{\pgfqpoint{3.163968in}{2.982383in}}%
\pgfpathlineto{\pgfqpoint{3.163968in}{3.055386in}}%
\pgfpathlineto{\pgfqpoint{3.086016in}{3.018790in}}%
\pgfpathlineto{\pgfqpoint{3.086016in}{2.945786in}}%
\pgfpathlineto{\pgfqpoint{3.163968in}{2.982383in}}%
\pgfpathclose%
\pgfusepath{fill}%
\end{pgfscope}%
\begin{pgfscope}%
\pgfpathrectangle{\pgfqpoint{0.765000in}{0.660000in}}{\pgfqpoint{4.620000in}{4.620000in}}%
\pgfusepath{clip}%
\pgfsetbuttcap%
\pgfsetroundjoin%
\definecolor{currentfill}{rgb}{1.000000,0.576471,0.309804}%
\pgfsetfillcolor{currentfill}%
\pgfsetfillopacity{0.050000}%
\pgfsetlinewidth{0.000000pt}%
\definecolor{currentstroke}{rgb}{1.000000,0.576471,0.309804}%
\pgfsetstrokecolor{currentstroke}%
\pgfsetdash{}{0pt}%
\pgfpathmoveto{\pgfqpoint{3.717566in}{3.665079in}}%
\pgfpathlineto{\pgfqpoint{3.717566in}{3.738083in}}%
\pgfpathlineto{\pgfqpoint{3.795518in}{3.701486in}}%
\pgfpathlineto{\pgfqpoint{3.795518in}{3.628483in}}%
\pgfpathlineto{\pgfqpoint{3.717566in}{3.665079in}}%
\pgfpathclose%
\pgfusepath{fill}%
\end{pgfscope}%
\begin{pgfscope}%
\pgfpathrectangle{\pgfqpoint{0.765000in}{0.660000in}}{\pgfqpoint{4.620000in}{4.620000in}}%
\pgfusepath{clip}%
\pgfsetbuttcap%
\pgfsetroundjoin%
\definecolor{currentfill}{rgb}{1.000000,0.576471,0.309804}%
\pgfsetfillcolor{currentfill}%
\pgfsetfillopacity{0.050000}%
\pgfsetlinewidth{0.000000pt}%
\definecolor{currentstroke}{rgb}{1.000000,0.576471,0.309804}%
\pgfsetstrokecolor{currentstroke}%
\pgfsetdash{}{0pt}%
\pgfpathmoveto{\pgfqpoint{3.717566in}{3.665079in}}%
\pgfpathlineto{\pgfqpoint{3.717566in}{3.738083in}}%
\pgfpathlineto{\pgfqpoint{3.639613in}{3.701486in}}%
\pgfpathlineto{\pgfqpoint{3.639613in}{3.628483in}}%
\pgfpathlineto{\pgfqpoint{3.717566in}{3.665079in}}%
\pgfpathclose%
\pgfusepath{fill}%
\end{pgfscope}%
\begin{pgfscope}%
\pgfpathrectangle{\pgfqpoint{0.765000in}{0.660000in}}{\pgfqpoint{4.620000in}{4.620000in}}%
\pgfusepath{clip}%
\pgfsetbuttcap%
\pgfsetroundjoin%
\definecolor{currentfill}{rgb}{1.000000,0.576471,0.309804}%
\pgfsetfillcolor{currentfill}%
\pgfsetfillopacity{0.050000}%
\pgfsetlinewidth{0.000000pt}%
\definecolor{currentstroke}{rgb}{1.000000,0.576471,0.309804}%
\pgfsetstrokecolor{currentstroke}%
\pgfsetdash{}{0pt}%
\pgfpathmoveto{\pgfqpoint{3.717566in}{3.665079in}}%
\pgfpathlineto{\pgfqpoint{3.795518in}{3.628483in}}%
\pgfpathlineto{\pgfqpoint{3.717566in}{3.591886in}}%
\pgfpathlineto{\pgfqpoint{3.639613in}{3.628483in}}%
\pgfpathlineto{\pgfqpoint{3.717566in}{3.665079in}}%
\pgfpathclose%
\pgfusepath{fill}%
\end{pgfscope}%
\begin{pgfscope}%
\pgfpathrectangle{\pgfqpoint{0.765000in}{0.660000in}}{\pgfqpoint{4.620000in}{4.620000in}}%
\pgfusepath{clip}%
\pgfsetbuttcap%
\pgfsetroundjoin%
\definecolor{currentfill}{rgb}{1.000000,0.576471,0.309804}%
\pgfsetfillcolor{currentfill}%
\pgfsetfillopacity{0.050000}%
\pgfsetlinewidth{0.000000pt}%
\definecolor{currentstroke}{rgb}{1.000000,0.576471,0.309804}%
\pgfsetstrokecolor{currentstroke}%
\pgfsetdash{}{0pt}%
\pgfpathmoveto{\pgfqpoint{3.717566in}{3.738083in}}%
\pgfpathlineto{\pgfqpoint{3.795518in}{3.701486in}}%
\pgfpathlineto{\pgfqpoint{3.717566in}{3.664889in}}%
\pgfpathlineto{\pgfqpoint{3.639613in}{3.701486in}}%
\pgfpathlineto{\pgfqpoint{3.717566in}{3.738083in}}%
\pgfpathclose%
\pgfusepath{fill}%
\end{pgfscope}%
\begin{pgfscope}%
\pgfpathrectangle{\pgfqpoint{0.765000in}{0.660000in}}{\pgfqpoint{4.620000in}{4.620000in}}%
\pgfusepath{clip}%
\pgfsetbuttcap%
\pgfsetroundjoin%
\definecolor{currentfill}{rgb}{1.000000,0.576471,0.309804}%
\pgfsetfillcolor{currentfill}%
\pgfsetfillopacity{0.050000}%
\pgfsetlinewidth{0.000000pt}%
\definecolor{currentstroke}{rgb}{1.000000,0.576471,0.309804}%
\pgfsetstrokecolor{currentstroke}%
\pgfsetdash{}{0pt}%
\pgfpathmoveto{\pgfqpoint{3.639613in}{3.628483in}}%
\pgfpathlineto{\pgfqpoint{3.639613in}{3.701486in}}%
\pgfpathlineto{\pgfqpoint{3.717566in}{3.664889in}}%
\pgfpathlineto{\pgfqpoint{3.717566in}{3.591886in}}%
\pgfpathlineto{\pgfqpoint{3.639613in}{3.628483in}}%
\pgfpathclose%
\pgfusepath{fill}%
\end{pgfscope}%
\begin{pgfscope}%
\pgfpathrectangle{\pgfqpoint{0.765000in}{0.660000in}}{\pgfqpoint{4.620000in}{4.620000in}}%
\pgfusepath{clip}%
\pgfsetbuttcap%
\pgfsetroundjoin%
\definecolor{currentfill}{rgb}{1.000000,0.576471,0.309804}%
\pgfsetfillcolor{currentfill}%
\pgfsetfillopacity{0.050000}%
\pgfsetlinewidth{0.000000pt}%
\definecolor{currentstroke}{rgb}{1.000000,0.576471,0.309804}%
\pgfsetstrokecolor{currentstroke}%
\pgfsetdash{}{0pt}%
\pgfpathmoveto{\pgfqpoint{3.795518in}{3.628483in}}%
\pgfpathlineto{\pgfqpoint{3.795518in}{3.701486in}}%
\pgfpathlineto{\pgfqpoint{3.717566in}{3.664889in}}%
\pgfpathlineto{\pgfqpoint{3.717566in}{3.591886in}}%
\pgfpathlineto{\pgfqpoint{3.795518in}{3.628483in}}%
\pgfpathclose%
\pgfusepath{fill}%
\end{pgfscope}%
\begin{pgfscope}%
\pgfpathrectangle{\pgfqpoint{0.765000in}{0.660000in}}{\pgfqpoint{4.620000in}{4.620000in}}%
\pgfusepath{clip}%
\pgfsetbuttcap%
\pgfsetroundjoin%
\definecolor{currentfill}{rgb}{1.000000,0.576471,0.309804}%
\pgfsetfillcolor{currentfill}%
\pgfsetfillopacity{0.050000}%
\pgfsetlinewidth{0.000000pt}%
\definecolor{currentstroke}{rgb}{1.000000,0.576471,0.309804}%
\pgfsetstrokecolor{currentstroke}%
\pgfsetdash{}{0pt}%
\pgfpathmoveto{\pgfqpoint{2.441743in}{2.161130in}}%
\pgfpathlineto{\pgfqpoint{2.441743in}{2.234134in}}%
\pgfpathlineto{\pgfqpoint{2.519696in}{2.197537in}}%
\pgfpathlineto{\pgfqpoint{2.519696in}{2.124534in}}%
\pgfpathlineto{\pgfqpoint{2.441743in}{2.161130in}}%
\pgfpathclose%
\pgfusepath{fill}%
\end{pgfscope}%
\begin{pgfscope}%
\pgfpathrectangle{\pgfqpoint{0.765000in}{0.660000in}}{\pgfqpoint{4.620000in}{4.620000in}}%
\pgfusepath{clip}%
\pgfsetbuttcap%
\pgfsetroundjoin%
\definecolor{currentfill}{rgb}{1.000000,0.576471,0.309804}%
\pgfsetfillcolor{currentfill}%
\pgfsetfillopacity{0.050000}%
\pgfsetlinewidth{0.000000pt}%
\definecolor{currentstroke}{rgb}{1.000000,0.576471,0.309804}%
\pgfsetstrokecolor{currentstroke}%
\pgfsetdash{}{0pt}%
\pgfpathmoveto{\pgfqpoint{2.441743in}{2.161130in}}%
\pgfpathlineto{\pgfqpoint{2.441743in}{2.234134in}}%
\pgfpathlineto{\pgfqpoint{2.363790in}{2.197537in}}%
\pgfpathlineto{\pgfqpoint{2.363790in}{2.124534in}}%
\pgfpathlineto{\pgfqpoint{2.441743in}{2.161130in}}%
\pgfpathclose%
\pgfusepath{fill}%
\end{pgfscope}%
\begin{pgfscope}%
\pgfpathrectangle{\pgfqpoint{0.765000in}{0.660000in}}{\pgfqpoint{4.620000in}{4.620000in}}%
\pgfusepath{clip}%
\pgfsetbuttcap%
\pgfsetroundjoin%
\definecolor{currentfill}{rgb}{1.000000,0.576471,0.309804}%
\pgfsetfillcolor{currentfill}%
\pgfsetfillopacity{0.050000}%
\pgfsetlinewidth{0.000000pt}%
\definecolor{currentstroke}{rgb}{1.000000,0.576471,0.309804}%
\pgfsetstrokecolor{currentstroke}%
\pgfsetdash{}{0pt}%
\pgfpathmoveto{\pgfqpoint{2.441743in}{2.161130in}}%
\pgfpathlineto{\pgfqpoint{2.519696in}{2.124534in}}%
\pgfpathlineto{\pgfqpoint{2.441743in}{2.087937in}}%
\pgfpathlineto{\pgfqpoint{2.363790in}{2.124534in}}%
\pgfpathlineto{\pgfqpoint{2.441743in}{2.161130in}}%
\pgfpathclose%
\pgfusepath{fill}%
\end{pgfscope}%
\begin{pgfscope}%
\pgfpathrectangle{\pgfqpoint{0.765000in}{0.660000in}}{\pgfqpoint{4.620000in}{4.620000in}}%
\pgfusepath{clip}%
\pgfsetbuttcap%
\pgfsetroundjoin%
\definecolor{currentfill}{rgb}{1.000000,0.576471,0.309804}%
\pgfsetfillcolor{currentfill}%
\pgfsetfillopacity{0.050000}%
\pgfsetlinewidth{0.000000pt}%
\definecolor{currentstroke}{rgb}{1.000000,0.576471,0.309804}%
\pgfsetstrokecolor{currentstroke}%
\pgfsetdash{}{0pt}%
\pgfpathmoveto{\pgfqpoint{2.441743in}{2.234134in}}%
\pgfpathlineto{\pgfqpoint{2.519696in}{2.197537in}}%
\pgfpathlineto{\pgfqpoint{2.441743in}{2.160941in}}%
\pgfpathlineto{\pgfqpoint{2.363790in}{2.197537in}}%
\pgfpathlineto{\pgfqpoint{2.441743in}{2.234134in}}%
\pgfpathclose%
\pgfusepath{fill}%
\end{pgfscope}%
\begin{pgfscope}%
\pgfpathrectangle{\pgfqpoint{0.765000in}{0.660000in}}{\pgfqpoint{4.620000in}{4.620000in}}%
\pgfusepath{clip}%
\pgfsetbuttcap%
\pgfsetroundjoin%
\definecolor{currentfill}{rgb}{1.000000,0.576471,0.309804}%
\pgfsetfillcolor{currentfill}%
\pgfsetfillopacity{0.050000}%
\pgfsetlinewidth{0.000000pt}%
\definecolor{currentstroke}{rgb}{1.000000,0.576471,0.309804}%
\pgfsetstrokecolor{currentstroke}%
\pgfsetdash{}{0pt}%
\pgfpathmoveto{\pgfqpoint{2.363790in}{2.124534in}}%
\pgfpathlineto{\pgfqpoint{2.363790in}{2.197537in}}%
\pgfpathlineto{\pgfqpoint{2.441743in}{2.160941in}}%
\pgfpathlineto{\pgfqpoint{2.441743in}{2.087937in}}%
\pgfpathlineto{\pgfqpoint{2.363790in}{2.124534in}}%
\pgfpathclose%
\pgfusepath{fill}%
\end{pgfscope}%
\begin{pgfscope}%
\pgfpathrectangle{\pgfqpoint{0.765000in}{0.660000in}}{\pgfqpoint{4.620000in}{4.620000in}}%
\pgfusepath{clip}%
\pgfsetbuttcap%
\pgfsetroundjoin%
\definecolor{currentfill}{rgb}{1.000000,0.576471,0.309804}%
\pgfsetfillcolor{currentfill}%
\pgfsetfillopacity{0.050000}%
\pgfsetlinewidth{0.000000pt}%
\definecolor{currentstroke}{rgb}{1.000000,0.576471,0.309804}%
\pgfsetstrokecolor{currentstroke}%
\pgfsetdash{}{0pt}%
\pgfpathmoveto{\pgfqpoint{2.519696in}{2.124534in}}%
\pgfpathlineto{\pgfqpoint{2.519696in}{2.197537in}}%
\pgfpathlineto{\pgfqpoint{2.441743in}{2.160941in}}%
\pgfpathlineto{\pgfqpoint{2.441743in}{2.087937in}}%
\pgfpathlineto{\pgfqpoint{2.519696in}{2.124534in}}%
\pgfpathclose%
\pgfusepath{fill}%
\end{pgfscope}%
\begin{pgfscope}%
\pgfpathrectangle{\pgfqpoint{0.765000in}{0.660000in}}{\pgfqpoint{4.620000in}{4.620000in}}%
\pgfusepath{clip}%
\pgfsetbuttcap%
\pgfsetroundjoin%
\definecolor{currentfill}{rgb}{0.176471,0.192157,0.258824}%
\pgfsetfillcolor{currentfill}%
\pgfsetfillopacity{0.050000}%
\pgfsetlinewidth{0.000000pt}%
\definecolor{currentstroke}{rgb}{0.176471,0.192157,0.258824}%
\pgfsetstrokecolor{currentstroke}%
\pgfsetdash{}{0pt}%
\pgfpathmoveto{\pgfqpoint{3.067617in}{2.953734in}}%
\pgfpathlineto{\pgfqpoint{3.067617in}{3.026738in}}%
\pgfpathlineto{\pgfqpoint{3.145569in}{2.990141in}}%
\pgfpathlineto{\pgfqpoint{3.145569in}{2.917138in}}%
\pgfpathlineto{\pgfqpoint{3.067617in}{2.953734in}}%
\pgfpathclose%
\pgfusepath{fill}%
\end{pgfscope}%
\begin{pgfscope}%
\pgfpathrectangle{\pgfqpoint{0.765000in}{0.660000in}}{\pgfqpoint{4.620000in}{4.620000in}}%
\pgfusepath{clip}%
\pgfsetbuttcap%
\pgfsetroundjoin%
\definecolor{currentfill}{rgb}{0.176471,0.192157,0.258824}%
\pgfsetfillcolor{currentfill}%
\pgfsetfillopacity{0.050000}%
\pgfsetlinewidth{0.000000pt}%
\definecolor{currentstroke}{rgb}{0.176471,0.192157,0.258824}%
\pgfsetstrokecolor{currentstroke}%
\pgfsetdash{}{0pt}%
\pgfpathmoveto{\pgfqpoint{3.067617in}{2.953734in}}%
\pgfpathlineto{\pgfqpoint{3.067617in}{3.026738in}}%
\pgfpathlineto{\pgfqpoint{2.989664in}{2.990141in}}%
\pgfpathlineto{\pgfqpoint{2.989664in}{2.917138in}}%
\pgfpathlineto{\pgfqpoint{3.067617in}{2.953734in}}%
\pgfpathclose%
\pgfusepath{fill}%
\end{pgfscope}%
\begin{pgfscope}%
\pgfpathrectangle{\pgfqpoint{0.765000in}{0.660000in}}{\pgfqpoint{4.620000in}{4.620000in}}%
\pgfusepath{clip}%
\pgfsetbuttcap%
\pgfsetroundjoin%
\definecolor{currentfill}{rgb}{0.176471,0.192157,0.258824}%
\pgfsetfillcolor{currentfill}%
\pgfsetfillopacity{0.050000}%
\pgfsetlinewidth{0.000000pt}%
\definecolor{currentstroke}{rgb}{0.176471,0.192157,0.258824}%
\pgfsetstrokecolor{currentstroke}%
\pgfsetdash{}{0pt}%
\pgfpathmoveto{\pgfqpoint{3.067617in}{2.953734in}}%
\pgfpathlineto{\pgfqpoint{3.145569in}{2.917138in}}%
\pgfpathlineto{\pgfqpoint{3.067617in}{2.880541in}}%
\pgfpathlineto{\pgfqpoint{2.989664in}{2.917138in}}%
\pgfpathlineto{\pgfqpoint{3.067617in}{2.953734in}}%
\pgfpathclose%
\pgfusepath{fill}%
\end{pgfscope}%
\begin{pgfscope}%
\pgfpathrectangle{\pgfqpoint{0.765000in}{0.660000in}}{\pgfqpoint{4.620000in}{4.620000in}}%
\pgfusepath{clip}%
\pgfsetbuttcap%
\pgfsetroundjoin%
\definecolor{currentfill}{rgb}{0.176471,0.192157,0.258824}%
\pgfsetfillcolor{currentfill}%
\pgfsetfillopacity{0.050000}%
\pgfsetlinewidth{0.000000pt}%
\definecolor{currentstroke}{rgb}{0.176471,0.192157,0.258824}%
\pgfsetstrokecolor{currentstroke}%
\pgfsetdash{}{0pt}%
\pgfpathmoveto{\pgfqpoint{3.067617in}{3.026738in}}%
\pgfpathlineto{\pgfqpoint{3.145569in}{2.990141in}}%
\pgfpathlineto{\pgfqpoint{3.067617in}{2.953544in}}%
\pgfpathlineto{\pgfqpoint{2.989664in}{2.990141in}}%
\pgfpathlineto{\pgfqpoint{3.067617in}{3.026738in}}%
\pgfpathclose%
\pgfusepath{fill}%
\end{pgfscope}%
\begin{pgfscope}%
\pgfpathrectangle{\pgfqpoint{0.765000in}{0.660000in}}{\pgfqpoint{4.620000in}{4.620000in}}%
\pgfusepath{clip}%
\pgfsetbuttcap%
\pgfsetroundjoin%
\definecolor{currentfill}{rgb}{0.176471,0.192157,0.258824}%
\pgfsetfillcolor{currentfill}%
\pgfsetfillopacity{0.050000}%
\pgfsetlinewidth{0.000000pt}%
\definecolor{currentstroke}{rgb}{0.176471,0.192157,0.258824}%
\pgfsetstrokecolor{currentstroke}%
\pgfsetdash{}{0pt}%
\pgfpathmoveto{\pgfqpoint{2.989664in}{2.917138in}}%
\pgfpathlineto{\pgfqpoint{2.989664in}{2.990141in}}%
\pgfpathlineto{\pgfqpoint{3.067617in}{2.953544in}}%
\pgfpathlineto{\pgfqpoint{3.067617in}{2.880541in}}%
\pgfpathlineto{\pgfqpoint{2.989664in}{2.917138in}}%
\pgfpathclose%
\pgfusepath{fill}%
\end{pgfscope}%
\begin{pgfscope}%
\pgfpathrectangle{\pgfqpoint{0.765000in}{0.660000in}}{\pgfqpoint{4.620000in}{4.620000in}}%
\pgfusepath{clip}%
\pgfsetbuttcap%
\pgfsetroundjoin%
\definecolor{currentfill}{rgb}{0.176471,0.192157,0.258824}%
\pgfsetfillcolor{currentfill}%
\pgfsetfillopacity{0.050000}%
\pgfsetlinewidth{0.000000pt}%
\definecolor{currentstroke}{rgb}{0.176471,0.192157,0.258824}%
\pgfsetstrokecolor{currentstroke}%
\pgfsetdash{}{0pt}%
\pgfpathmoveto{\pgfqpoint{3.145569in}{2.917138in}}%
\pgfpathlineto{\pgfqpoint{3.145569in}{2.990141in}}%
\pgfpathlineto{\pgfqpoint{3.067617in}{2.953544in}}%
\pgfpathlineto{\pgfqpoint{3.067617in}{2.880541in}}%
\pgfpathlineto{\pgfqpoint{3.145569in}{2.917138in}}%
\pgfpathclose%
\pgfusepath{fill}%
\end{pgfscope}%
\begin{pgfscope}%
\pgfpathrectangle{\pgfqpoint{0.765000in}{0.660000in}}{\pgfqpoint{4.620000in}{4.620000in}}%
\pgfusepath{clip}%
\pgfsetbuttcap%
\pgfsetroundjoin%
\definecolor{currentfill}{rgb}{0.176471,0.192157,0.258824}%
\pgfsetfillcolor{currentfill}%
\pgfsetfillopacity{0.050000}%
\pgfsetlinewidth{0.000000pt}%
\definecolor{currentstroke}{rgb}{0.176471,0.192157,0.258824}%
\pgfsetstrokecolor{currentstroke}%
\pgfsetdash{}{0pt}%
\pgfpathmoveto{\pgfqpoint{3.208437in}{2.846800in}}%
\pgfpathlineto{\pgfqpoint{3.208437in}{2.919803in}}%
\pgfpathlineto{\pgfqpoint{3.286390in}{2.883207in}}%
\pgfpathlineto{\pgfqpoint{3.286390in}{2.810203in}}%
\pgfpathlineto{\pgfqpoint{3.208437in}{2.846800in}}%
\pgfpathclose%
\pgfusepath{fill}%
\end{pgfscope}%
\begin{pgfscope}%
\pgfpathrectangle{\pgfqpoint{0.765000in}{0.660000in}}{\pgfqpoint{4.620000in}{4.620000in}}%
\pgfusepath{clip}%
\pgfsetbuttcap%
\pgfsetroundjoin%
\definecolor{currentfill}{rgb}{0.176471,0.192157,0.258824}%
\pgfsetfillcolor{currentfill}%
\pgfsetfillopacity{0.050000}%
\pgfsetlinewidth{0.000000pt}%
\definecolor{currentstroke}{rgb}{0.176471,0.192157,0.258824}%
\pgfsetstrokecolor{currentstroke}%
\pgfsetdash{}{0pt}%
\pgfpathmoveto{\pgfqpoint{3.208437in}{2.846800in}}%
\pgfpathlineto{\pgfqpoint{3.208437in}{2.919803in}}%
\pgfpathlineto{\pgfqpoint{3.130484in}{2.883207in}}%
\pgfpathlineto{\pgfqpoint{3.130484in}{2.810203in}}%
\pgfpathlineto{\pgfqpoint{3.208437in}{2.846800in}}%
\pgfpathclose%
\pgfusepath{fill}%
\end{pgfscope}%
\begin{pgfscope}%
\pgfpathrectangle{\pgfqpoint{0.765000in}{0.660000in}}{\pgfqpoint{4.620000in}{4.620000in}}%
\pgfusepath{clip}%
\pgfsetbuttcap%
\pgfsetroundjoin%
\definecolor{currentfill}{rgb}{0.176471,0.192157,0.258824}%
\pgfsetfillcolor{currentfill}%
\pgfsetfillopacity{0.050000}%
\pgfsetlinewidth{0.000000pt}%
\definecolor{currentstroke}{rgb}{0.176471,0.192157,0.258824}%
\pgfsetstrokecolor{currentstroke}%
\pgfsetdash{}{0pt}%
\pgfpathmoveto{\pgfqpoint{3.208437in}{2.846800in}}%
\pgfpathlineto{\pgfqpoint{3.286390in}{2.810203in}}%
\pgfpathlineto{\pgfqpoint{3.208437in}{2.773607in}}%
\pgfpathlineto{\pgfqpoint{3.130484in}{2.810203in}}%
\pgfpathlineto{\pgfqpoint{3.208437in}{2.846800in}}%
\pgfpathclose%
\pgfusepath{fill}%
\end{pgfscope}%
\begin{pgfscope}%
\pgfpathrectangle{\pgfqpoint{0.765000in}{0.660000in}}{\pgfqpoint{4.620000in}{4.620000in}}%
\pgfusepath{clip}%
\pgfsetbuttcap%
\pgfsetroundjoin%
\definecolor{currentfill}{rgb}{0.176471,0.192157,0.258824}%
\pgfsetfillcolor{currentfill}%
\pgfsetfillopacity{0.050000}%
\pgfsetlinewidth{0.000000pt}%
\definecolor{currentstroke}{rgb}{0.176471,0.192157,0.258824}%
\pgfsetstrokecolor{currentstroke}%
\pgfsetdash{}{0pt}%
\pgfpathmoveto{\pgfqpoint{3.208437in}{2.919803in}}%
\pgfpathlineto{\pgfqpoint{3.286390in}{2.883207in}}%
\pgfpathlineto{\pgfqpoint{3.208437in}{2.846610in}}%
\pgfpathlineto{\pgfqpoint{3.130484in}{2.883207in}}%
\pgfpathlineto{\pgfqpoint{3.208437in}{2.919803in}}%
\pgfpathclose%
\pgfusepath{fill}%
\end{pgfscope}%
\begin{pgfscope}%
\pgfpathrectangle{\pgfqpoint{0.765000in}{0.660000in}}{\pgfqpoint{4.620000in}{4.620000in}}%
\pgfusepath{clip}%
\pgfsetbuttcap%
\pgfsetroundjoin%
\definecolor{currentfill}{rgb}{0.176471,0.192157,0.258824}%
\pgfsetfillcolor{currentfill}%
\pgfsetfillopacity{0.050000}%
\pgfsetlinewidth{0.000000pt}%
\definecolor{currentstroke}{rgb}{0.176471,0.192157,0.258824}%
\pgfsetstrokecolor{currentstroke}%
\pgfsetdash{}{0pt}%
\pgfpathmoveto{\pgfqpoint{3.130484in}{2.810203in}}%
\pgfpathlineto{\pgfqpoint{3.130484in}{2.883207in}}%
\pgfpathlineto{\pgfqpoint{3.208437in}{2.846610in}}%
\pgfpathlineto{\pgfqpoint{3.208437in}{2.773607in}}%
\pgfpathlineto{\pgfqpoint{3.130484in}{2.810203in}}%
\pgfpathclose%
\pgfusepath{fill}%
\end{pgfscope}%
\begin{pgfscope}%
\pgfpathrectangle{\pgfqpoint{0.765000in}{0.660000in}}{\pgfqpoint{4.620000in}{4.620000in}}%
\pgfusepath{clip}%
\pgfsetbuttcap%
\pgfsetroundjoin%
\definecolor{currentfill}{rgb}{0.176471,0.192157,0.258824}%
\pgfsetfillcolor{currentfill}%
\pgfsetfillopacity{0.050000}%
\pgfsetlinewidth{0.000000pt}%
\definecolor{currentstroke}{rgb}{0.176471,0.192157,0.258824}%
\pgfsetstrokecolor{currentstroke}%
\pgfsetdash{}{0pt}%
\pgfpathmoveto{\pgfqpoint{3.286390in}{2.810203in}}%
\pgfpathlineto{\pgfqpoint{3.286390in}{2.883207in}}%
\pgfpathlineto{\pgfqpoint{3.208437in}{2.846610in}}%
\pgfpathlineto{\pgfqpoint{3.208437in}{2.773607in}}%
\pgfpathlineto{\pgfqpoint{3.286390in}{2.810203in}}%
\pgfpathclose%
\pgfusepath{fill}%
\end{pgfscope}%
\begin{pgfscope}%
\pgfpathrectangle{\pgfqpoint{0.765000in}{0.660000in}}{\pgfqpoint{4.620000in}{4.620000in}}%
\pgfusepath{clip}%
\pgfsetbuttcap%
\pgfsetroundjoin%
\definecolor{currentfill}{rgb}{0.176471,0.192157,0.258824}%
\pgfsetfillcolor{currentfill}%
\pgfsetfillopacity{0.050000}%
\pgfsetlinewidth{0.000000pt}%
\definecolor{currentstroke}{rgb}{0.176471,0.192157,0.258824}%
\pgfsetstrokecolor{currentstroke}%
\pgfsetdash{}{0pt}%
\pgfpathmoveto{\pgfqpoint{3.160448in}{2.859002in}}%
\pgfpathlineto{\pgfqpoint{3.160448in}{2.932006in}}%
\pgfpathlineto{\pgfqpoint{3.238401in}{2.895409in}}%
\pgfpathlineto{\pgfqpoint{3.238401in}{2.822406in}}%
\pgfpathlineto{\pgfqpoint{3.160448in}{2.859002in}}%
\pgfpathclose%
\pgfusepath{fill}%
\end{pgfscope}%
\begin{pgfscope}%
\pgfpathrectangle{\pgfqpoint{0.765000in}{0.660000in}}{\pgfqpoint{4.620000in}{4.620000in}}%
\pgfusepath{clip}%
\pgfsetbuttcap%
\pgfsetroundjoin%
\definecolor{currentfill}{rgb}{0.176471,0.192157,0.258824}%
\pgfsetfillcolor{currentfill}%
\pgfsetfillopacity{0.050000}%
\pgfsetlinewidth{0.000000pt}%
\definecolor{currentstroke}{rgb}{0.176471,0.192157,0.258824}%
\pgfsetstrokecolor{currentstroke}%
\pgfsetdash{}{0pt}%
\pgfpathmoveto{\pgfqpoint{3.160448in}{2.859002in}}%
\pgfpathlineto{\pgfqpoint{3.160448in}{2.932006in}}%
\pgfpathlineto{\pgfqpoint{3.082495in}{2.895409in}}%
\pgfpathlineto{\pgfqpoint{3.082495in}{2.822406in}}%
\pgfpathlineto{\pgfqpoint{3.160448in}{2.859002in}}%
\pgfpathclose%
\pgfusepath{fill}%
\end{pgfscope}%
\begin{pgfscope}%
\pgfpathrectangle{\pgfqpoint{0.765000in}{0.660000in}}{\pgfqpoint{4.620000in}{4.620000in}}%
\pgfusepath{clip}%
\pgfsetbuttcap%
\pgfsetroundjoin%
\definecolor{currentfill}{rgb}{0.176471,0.192157,0.258824}%
\pgfsetfillcolor{currentfill}%
\pgfsetfillopacity{0.050000}%
\pgfsetlinewidth{0.000000pt}%
\definecolor{currentstroke}{rgb}{0.176471,0.192157,0.258824}%
\pgfsetstrokecolor{currentstroke}%
\pgfsetdash{}{0pt}%
\pgfpathmoveto{\pgfqpoint{3.160448in}{2.859002in}}%
\pgfpathlineto{\pgfqpoint{3.238401in}{2.822406in}}%
\pgfpathlineto{\pgfqpoint{3.160448in}{2.785809in}}%
\pgfpathlineto{\pgfqpoint{3.082495in}{2.822406in}}%
\pgfpathlineto{\pgfqpoint{3.160448in}{2.859002in}}%
\pgfpathclose%
\pgfusepath{fill}%
\end{pgfscope}%
\begin{pgfscope}%
\pgfpathrectangle{\pgfqpoint{0.765000in}{0.660000in}}{\pgfqpoint{4.620000in}{4.620000in}}%
\pgfusepath{clip}%
\pgfsetbuttcap%
\pgfsetroundjoin%
\definecolor{currentfill}{rgb}{0.176471,0.192157,0.258824}%
\pgfsetfillcolor{currentfill}%
\pgfsetfillopacity{0.050000}%
\pgfsetlinewidth{0.000000pt}%
\definecolor{currentstroke}{rgb}{0.176471,0.192157,0.258824}%
\pgfsetstrokecolor{currentstroke}%
\pgfsetdash{}{0pt}%
\pgfpathmoveto{\pgfqpoint{3.160448in}{2.932006in}}%
\pgfpathlineto{\pgfqpoint{3.238401in}{2.895409in}}%
\pgfpathlineto{\pgfqpoint{3.160448in}{2.858813in}}%
\pgfpathlineto{\pgfqpoint{3.082495in}{2.895409in}}%
\pgfpathlineto{\pgfqpoint{3.160448in}{2.932006in}}%
\pgfpathclose%
\pgfusepath{fill}%
\end{pgfscope}%
\begin{pgfscope}%
\pgfpathrectangle{\pgfqpoint{0.765000in}{0.660000in}}{\pgfqpoint{4.620000in}{4.620000in}}%
\pgfusepath{clip}%
\pgfsetbuttcap%
\pgfsetroundjoin%
\definecolor{currentfill}{rgb}{0.176471,0.192157,0.258824}%
\pgfsetfillcolor{currentfill}%
\pgfsetfillopacity{0.050000}%
\pgfsetlinewidth{0.000000pt}%
\definecolor{currentstroke}{rgb}{0.176471,0.192157,0.258824}%
\pgfsetstrokecolor{currentstroke}%
\pgfsetdash{}{0pt}%
\pgfpathmoveto{\pgfqpoint{3.082495in}{2.822406in}}%
\pgfpathlineto{\pgfqpoint{3.082495in}{2.895409in}}%
\pgfpathlineto{\pgfqpoint{3.160448in}{2.858813in}}%
\pgfpathlineto{\pgfqpoint{3.160448in}{2.785809in}}%
\pgfpathlineto{\pgfqpoint{3.082495in}{2.822406in}}%
\pgfpathclose%
\pgfusepath{fill}%
\end{pgfscope}%
\begin{pgfscope}%
\pgfpathrectangle{\pgfqpoint{0.765000in}{0.660000in}}{\pgfqpoint{4.620000in}{4.620000in}}%
\pgfusepath{clip}%
\pgfsetbuttcap%
\pgfsetroundjoin%
\definecolor{currentfill}{rgb}{0.176471,0.192157,0.258824}%
\pgfsetfillcolor{currentfill}%
\pgfsetfillopacity{0.050000}%
\pgfsetlinewidth{0.000000pt}%
\definecolor{currentstroke}{rgb}{0.176471,0.192157,0.258824}%
\pgfsetstrokecolor{currentstroke}%
\pgfsetdash{}{0pt}%
\pgfpathmoveto{\pgfqpoint{3.238401in}{2.822406in}}%
\pgfpathlineto{\pgfqpoint{3.238401in}{2.895409in}}%
\pgfpathlineto{\pgfqpoint{3.160448in}{2.858813in}}%
\pgfpathlineto{\pgfqpoint{3.160448in}{2.785809in}}%
\pgfpathlineto{\pgfqpoint{3.238401in}{2.822406in}}%
\pgfpathclose%
\pgfusepath{fill}%
\end{pgfscope}%
\begin{pgfscope}%
\pgfpathrectangle{\pgfqpoint{0.765000in}{0.660000in}}{\pgfqpoint{4.620000in}{4.620000in}}%
\pgfusepath{clip}%
\pgfsetbuttcap%
\pgfsetroundjoin%
\definecolor{currentfill}{rgb}{0.176471,0.192157,0.258824}%
\pgfsetfillcolor{currentfill}%
\pgfsetfillopacity{0.050000}%
\pgfsetlinewidth{0.000000pt}%
\definecolor{currentstroke}{rgb}{0.176471,0.192157,0.258824}%
\pgfsetstrokecolor{currentstroke}%
\pgfsetdash{}{0pt}%
\pgfpathmoveto{\pgfqpoint{3.018308in}{2.874066in}}%
\pgfpathlineto{\pgfqpoint{3.018308in}{2.947070in}}%
\pgfpathlineto{\pgfqpoint{3.096261in}{2.910473in}}%
\pgfpathlineto{\pgfqpoint{3.096261in}{2.837470in}}%
\pgfpathlineto{\pgfqpoint{3.018308in}{2.874066in}}%
\pgfpathclose%
\pgfusepath{fill}%
\end{pgfscope}%
\begin{pgfscope}%
\pgfpathrectangle{\pgfqpoint{0.765000in}{0.660000in}}{\pgfqpoint{4.620000in}{4.620000in}}%
\pgfusepath{clip}%
\pgfsetbuttcap%
\pgfsetroundjoin%
\definecolor{currentfill}{rgb}{0.176471,0.192157,0.258824}%
\pgfsetfillcolor{currentfill}%
\pgfsetfillopacity{0.050000}%
\pgfsetlinewidth{0.000000pt}%
\definecolor{currentstroke}{rgb}{0.176471,0.192157,0.258824}%
\pgfsetstrokecolor{currentstroke}%
\pgfsetdash{}{0pt}%
\pgfpathmoveto{\pgfqpoint{3.018308in}{2.874066in}}%
\pgfpathlineto{\pgfqpoint{3.018308in}{2.947070in}}%
\pgfpathlineto{\pgfqpoint{2.940356in}{2.910473in}}%
\pgfpathlineto{\pgfqpoint{2.940356in}{2.837470in}}%
\pgfpathlineto{\pgfqpoint{3.018308in}{2.874066in}}%
\pgfpathclose%
\pgfusepath{fill}%
\end{pgfscope}%
\begin{pgfscope}%
\pgfpathrectangle{\pgfqpoint{0.765000in}{0.660000in}}{\pgfqpoint{4.620000in}{4.620000in}}%
\pgfusepath{clip}%
\pgfsetbuttcap%
\pgfsetroundjoin%
\definecolor{currentfill}{rgb}{0.176471,0.192157,0.258824}%
\pgfsetfillcolor{currentfill}%
\pgfsetfillopacity{0.050000}%
\pgfsetlinewidth{0.000000pt}%
\definecolor{currentstroke}{rgb}{0.176471,0.192157,0.258824}%
\pgfsetstrokecolor{currentstroke}%
\pgfsetdash{}{0pt}%
\pgfpathmoveto{\pgfqpoint{3.018308in}{2.874066in}}%
\pgfpathlineto{\pgfqpoint{3.096261in}{2.837470in}}%
\pgfpathlineto{\pgfqpoint{3.018308in}{2.800873in}}%
\pgfpathlineto{\pgfqpoint{2.940356in}{2.837470in}}%
\pgfpathlineto{\pgfqpoint{3.018308in}{2.874066in}}%
\pgfpathclose%
\pgfusepath{fill}%
\end{pgfscope}%
\begin{pgfscope}%
\pgfpathrectangle{\pgfqpoint{0.765000in}{0.660000in}}{\pgfqpoint{4.620000in}{4.620000in}}%
\pgfusepath{clip}%
\pgfsetbuttcap%
\pgfsetroundjoin%
\definecolor{currentfill}{rgb}{0.176471,0.192157,0.258824}%
\pgfsetfillcolor{currentfill}%
\pgfsetfillopacity{0.050000}%
\pgfsetlinewidth{0.000000pt}%
\definecolor{currentstroke}{rgb}{0.176471,0.192157,0.258824}%
\pgfsetstrokecolor{currentstroke}%
\pgfsetdash{}{0pt}%
\pgfpathmoveto{\pgfqpoint{3.018308in}{2.947070in}}%
\pgfpathlineto{\pgfqpoint{3.096261in}{2.910473in}}%
\pgfpathlineto{\pgfqpoint{3.018308in}{2.873876in}}%
\pgfpathlineto{\pgfqpoint{2.940356in}{2.910473in}}%
\pgfpathlineto{\pgfqpoint{3.018308in}{2.947070in}}%
\pgfpathclose%
\pgfusepath{fill}%
\end{pgfscope}%
\begin{pgfscope}%
\pgfpathrectangle{\pgfqpoint{0.765000in}{0.660000in}}{\pgfqpoint{4.620000in}{4.620000in}}%
\pgfusepath{clip}%
\pgfsetbuttcap%
\pgfsetroundjoin%
\definecolor{currentfill}{rgb}{0.176471,0.192157,0.258824}%
\pgfsetfillcolor{currentfill}%
\pgfsetfillopacity{0.050000}%
\pgfsetlinewidth{0.000000pt}%
\definecolor{currentstroke}{rgb}{0.176471,0.192157,0.258824}%
\pgfsetstrokecolor{currentstroke}%
\pgfsetdash{}{0pt}%
\pgfpathmoveto{\pgfqpoint{2.940356in}{2.837470in}}%
\pgfpathlineto{\pgfqpoint{2.940356in}{2.910473in}}%
\pgfpathlineto{\pgfqpoint{3.018308in}{2.873876in}}%
\pgfpathlineto{\pgfqpoint{3.018308in}{2.800873in}}%
\pgfpathlineto{\pgfqpoint{2.940356in}{2.837470in}}%
\pgfpathclose%
\pgfusepath{fill}%
\end{pgfscope}%
\begin{pgfscope}%
\pgfpathrectangle{\pgfqpoint{0.765000in}{0.660000in}}{\pgfqpoint{4.620000in}{4.620000in}}%
\pgfusepath{clip}%
\pgfsetbuttcap%
\pgfsetroundjoin%
\definecolor{currentfill}{rgb}{0.176471,0.192157,0.258824}%
\pgfsetfillcolor{currentfill}%
\pgfsetfillopacity{0.050000}%
\pgfsetlinewidth{0.000000pt}%
\definecolor{currentstroke}{rgb}{0.176471,0.192157,0.258824}%
\pgfsetstrokecolor{currentstroke}%
\pgfsetdash{}{0pt}%
\pgfpathmoveto{\pgfqpoint{3.096261in}{2.837470in}}%
\pgfpathlineto{\pgfqpoint{3.096261in}{2.910473in}}%
\pgfpathlineto{\pgfqpoint{3.018308in}{2.873876in}}%
\pgfpathlineto{\pgfqpoint{3.018308in}{2.800873in}}%
\pgfpathlineto{\pgfqpoint{3.096261in}{2.837470in}}%
\pgfpathclose%
\pgfusepath{fill}%
\end{pgfscope}%
\begin{pgfscope}%
\pgfpathrectangle{\pgfqpoint{0.765000in}{0.660000in}}{\pgfqpoint{4.620000in}{4.620000in}}%
\pgfusepath{clip}%
\pgfsetbuttcap%
\pgfsetroundjoin%
\definecolor{currentfill}{rgb}{0.176471,0.192157,0.258824}%
\pgfsetfillcolor{currentfill}%
\pgfsetfillopacity{0.050000}%
\pgfsetlinewidth{0.000000pt}%
\definecolor{currentstroke}{rgb}{0.176471,0.192157,0.258824}%
\pgfsetstrokecolor{currentstroke}%
\pgfsetdash{}{0pt}%
\pgfpathmoveto{\pgfqpoint{3.218412in}{2.838367in}}%
\pgfpathlineto{\pgfqpoint{3.218412in}{2.911370in}}%
\pgfpathlineto{\pgfqpoint{3.296364in}{2.874773in}}%
\pgfpathlineto{\pgfqpoint{3.296364in}{2.801770in}}%
\pgfpathlineto{\pgfqpoint{3.218412in}{2.838367in}}%
\pgfpathclose%
\pgfusepath{fill}%
\end{pgfscope}%
\begin{pgfscope}%
\pgfpathrectangle{\pgfqpoint{0.765000in}{0.660000in}}{\pgfqpoint{4.620000in}{4.620000in}}%
\pgfusepath{clip}%
\pgfsetbuttcap%
\pgfsetroundjoin%
\definecolor{currentfill}{rgb}{0.176471,0.192157,0.258824}%
\pgfsetfillcolor{currentfill}%
\pgfsetfillopacity{0.050000}%
\pgfsetlinewidth{0.000000pt}%
\definecolor{currentstroke}{rgb}{0.176471,0.192157,0.258824}%
\pgfsetstrokecolor{currentstroke}%
\pgfsetdash{}{0pt}%
\pgfpathmoveto{\pgfqpoint{3.218412in}{2.838367in}}%
\pgfpathlineto{\pgfqpoint{3.218412in}{2.911370in}}%
\pgfpathlineto{\pgfqpoint{3.140459in}{2.874773in}}%
\pgfpathlineto{\pgfqpoint{3.140459in}{2.801770in}}%
\pgfpathlineto{\pgfqpoint{3.218412in}{2.838367in}}%
\pgfpathclose%
\pgfusepath{fill}%
\end{pgfscope}%
\begin{pgfscope}%
\pgfpathrectangle{\pgfqpoint{0.765000in}{0.660000in}}{\pgfqpoint{4.620000in}{4.620000in}}%
\pgfusepath{clip}%
\pgfsetbuttcap%
\pgfsetroundjoin%
\definecolor{currentfill}{rgb}{0.176471,0.192157,0.258824}%
\pgfsetfillcolor{currentfill}%
\pgfsetfillopacity{0.050000}%
\pgfsetlinewidth{0.000000pt}%
\definecolor{currentstroke}{rgb}{0.176471,0.192157,0.258824}%
\pgfsetstrokecolor{currentstroke}%
\pgfsetdash{}{0pt}%
\pgfpathmoveto{\pgfqpoint{3.218412in}{2.838367in}}%
\pgfpathlineto{\pgfqpoint{3.296364in}{2.801770in}}%
\pgfpathlineto{\pgfqpoint{3.218412in}{2.765173in}}%
\pgfpathlineto{\pgfqpoint{3.140459in}{2.801770in}}%
\pgfpathlineto{\pgfqpoint{3.218412in}{2.838367in}}%
\pgfpathclose%
\pgfusepath{fill}%
\end{pgfscope}%
\begin{pgfscope}%
\pgfpathrectangle{\pgfqpoint{0.765000in}{0.660000in}}{\pgfqpoint{4.620000in}{4.620000in}}%
\pgfusepath{clip}%
\pgfsetbuttcap%
\pgfsetroundjoin%
\definecolor{currentfill}{rgb}{0.176471,0.192157,0.258824}%
\pgfsetfillcolor{currentfill}%
\pgfsetfillopacity{0.050000}%
\pgfsetlinewidth{0.000000pt}%
\definecolor{currentstroke}{rgb}{0.176471,0.192157,0.258824}%
\pgfsetstrokecolor{currentstroke}%
\pgfsetdash{}{0pt}%
\pgfpathmoveto{\pgfqpoint{3.218412in}{2.911370in}}%
\pgfpathlineto{\pgfqpoint{3.296364in}{2.874773in}}%
\pgfpathlineto{\pgfqpoint{3.218412in}{2.838177in}}%
\pgfpathlineto{\pgfqpoint{3.140459in}{2.874773in}}%
\pgfpathlineto{\pgfqpoint{3.218412in}{2.911370in}}%
\pgfpathclose%
\pgfusepath{fill}%
\end{pgfscope}%
\begin{pgfscope}%
\pgfpathrectangle{\pgfqpoint{0.765000in}{0.660000in}}{\pgfqpoint{4.620000in}{4.620000in}}%
\pgfusepath{clip}%
\pgfsetbuttcap%
\pgfsetroundjoin%
\definecolor{currentfill}{rgb}{0.176471,0.192157,0.258824}%
\pgfsetfillcolor{currentfill}%
\pgfsetfillopacity{0.050000}%
\pgfsetlinewidth{0.000000pt}%
\definecolor{currentstroke}{rgb}{0.176471,0.192157,0.258824}%
\pgfsetstrokecolor{currentstroke}%
\pgfsetdash{}{0pt}%
\pgfpathmoveto{\pgfqpoint{3.140459in}{2.801770in}}%
\pgfpathlineto{\pgfqpoint{3.140459in}{2.874773in}}%
\pgfpathlineto{\pgfqpoint{3.218412in}{2.838177in}}%
\pgfpathlineto{\pgfqpoint{3.218412in}{2.765173in}}%
\pgfpathlineto{\pgfqpoint{3.140459in}{2.801770in}}%
\pgfpathclose%
\pgfusepath{fill}%
\end{pgfscope}%
\begin{pgfscope}%
\pgfpathrectangle{\pgfqpoint{0.765000in}{0.660000in}}{\pgfqpoint{4.620000in}{4.620000in}}%
\pgfusepath{clip}%
\pgfsetbuttcap%
\pgfsetroundjoin%
\definecolor{currentfill}{rgb}{0.176471,0.192157,0.258824}%
\pgfsetfillcolor{currentfill}%
\pgfsetfillopacity{0.050000}%
\pgfsetlinewidth{0.000000pt}%
\definecolor{currentstroke}{rgb}{0.176471,0.192157,0.258824}%
\pgfsetstrokecolor{currentstroke}%
\pgfsetdash{}{0pt}%
\pgfpathmoveto{\pgfqpoint{3.296364in}{2.801770in}}%
\pgfpathlineto{\pgfqpoint{3.296364in}{2.874773in}}%
\pgfpathlineto{\pgfqpoint{3.218412in}{2.838177in}}%
\pgfpathlineto{\pgfqpoint{3.218412in}{2.765173in}}%
\pgfpathlineto{\pgfqpoint{3.296364in}{2.801770in}}%
\pgfpathclose%
\pgfusepath{fill}%
\end{pgfscope}%
\begin{pgfscope}%
\pgfpathrectangle{\pgfqpoint{0.765000in}{0.660000in}}{\pgfqpoint{4.620000in}{4.620000in}}%
\pgfusepath{clip}%
\pgfsetbuttcap%
\pgfsetroundjoin%
\definecolor{currentfill}{rgb}{0.176471,0.192157,0.258824}%
\pgfsetfillcolor{currentfill}%
\pgfsetfillopacity{0.050000}%
\pgfsetlinewidth{0.000000pt}%
\definecolor{currentstroke}{rgb}{0.176471,0.192157,0.258824}%
\pgfsetstrokecolor{currentstroke}%
\pgfsetdash{}{0pt}%
\pgfpathmoveto{\pgfqpoint{3.077782in}{3.059587in}}%
\pgfpathlineto{\pgfqpoint{3.077782in}{3.132590in}}%
\pgfpathlineto{\pgfqpoint{3.155735in}{3.095994in}}%
\pgfpathlineto{\pgfqpoint{3.155735in}{3.022991in}}%
\pgfpathlineto{\pgfqpoint{3.077782in}{3.059587in}}%
\pgfpathclose%
\pgfusepath{fill}%
\end{pgfscope}%
\begin{pgfscope}%
\pgfpathrectangle{\pgfqpoint{0.765000in}{0.660000in}}{\pgfqpoint{4.620000in}{4.620000in}}%
\pgfusepath{clip}%
\pgfsetbuttcap%
\pgfsetroundjoin%
\definecolor{currentfill}{rgb}{0.176471,0.192157,0.258824}%
\pgfsetfillcolor{currentfill}%
\pgfsetfillopacity{0.050000}%
\pgfsetlinewidth{0.000000pt}%
\definecolor{currentstroke}{rgb}{0.176471,0.192157,0.258824}%
\pgfsetstrokecolor{currentstroke}%
\pgfsetdash{}{0pt}%
\pgfpathmoveto{\pgfqpoint{3.077782in}{3.059587in}}%
\pgfpathlineto{\pgfqpoint{3.077782in}{3.132590in}}%
\pgfpathlineto{\pgfqpoint{2.999829in}{3.095994in}}%
\pgfpathlineto{\pgfqpoint{2.999829in}{3.022991in}}%
\pgfpathlineto{\pgfqpoint{3.077782in}{3.059587in}}%
\pgfpathclose%
\pgfusepath{fill}%
\end{pgfscope}%
\begin{pgfscope}%
\pgfpathrectangle{\pgfqpoint{0.765000in}{0.660000in}}{\pgfqpoint{4.620000in}{4.620000in}}%
\pgfusepath{clip}%
\pgfsetbuttcap%
\pgfsetroundjoin%
\definecolor{currentfill}{rgb}{0.176471,0.192157,0.258824}%
\pgfsetfillcolor{currentfill}%
\pgfsetfillopacity{0.050000}%
\pgfsetlinewidth{0.000000pt}%
\definecolor{currentstroke}{rgb}{0.176471,0.192157,0.258824}%
\pgfsetstrokecolor{currentstroke}%
\pgfsetdash{}{0pt}%
\pgfpathmoveto{\pgfqpoint{3.077782in}{3.059587in}}%
\pgfpathlineto{\pgfqpoint{3.155735in}{3.022991in}}%
\pgfpathlineto{\pgfqpoint{3.077782in}{2.986394in}}%
\pgfpathlineto{\pgfqpoint{2.999829in}{3.022991in}}%
\pgfpathlineto{\pgfqpoint{3.077782in}{3.059587in}}%
\pgfpathclose%
\pgfusepath{fill}%
\end{pgfscope}%
\begin{pgfscope}%
\pgfpathrectangle{\pgfqpoint{0.765000in}{0.660000in}}{\pgfqpoint{4.620000in}{4.620000in}}%
\pgfusepath{clip}%
\pgfsetbuttcap%
\pgfsetroundjoin%
\definecolor{currentfill}{rgb}{0.176471,0.192157,0.258824}%
\pgfsetfillcolor{currentfill}%
\pgfsetfillopacity{0.050000}%
\pgfsetlinewidth{0.000000pt}%
\definecolor{currentstroke}{rgb}{0.176471,0.192157,0.258824}%
\pgfsetstrokecolor{currentstroke}%
\pgfsetdash{}{0pt}%
\pgfpathmoveto{\pgfqpoint{3.077782in}{3.132590in}}%
\pgfpathlineto{\pgfqpoint{3.155735in}{3.095994in}}%
\pgfpathlineto{\pgfqpoint{3.077782in}{3.059397in}}%
\pgfpathlineto{\pgfqpoint{2.999829in}{3.095994in}}%
\pgfpathlineto{\pgfqpoint{3.077782in}{3.132590in}}%
\pgfpathclose%
\pgfusepath{fill}%
\end{pgfscope}%
\begin{pgfscope}%
\pgfpathrectangle{\pgfqpoint{0.765000in}{0.660000in}}{\pgfqpoint{4.620000in}{4.620000in}}%
\pgfusepath{clip}%
\pgfsetbuttcap%
\pgfsetroundjoin%
\definecolor{currentfill}{rgb}{0.176471,0.192157,0.258824}%
\pgfsetfillcolor{currentfill}%
\pgfsetfillopacity{0.050000}%
\pgfsetlinewidth{0.000000pt}%
\definecolor{currentstroke}{rgb}{0.176471,0.192157,0.258824}%
\pgfsetstrokecolor{currentstroke}%
\pgfsetdash{}{0pt}%
\pgfpathmoveto{\pgfqpoint{2.999829in}{3.022991in}}%
\pgfpathlineto{\pgfqpoint{2.999829in}{3.095994in}}%
\pgfpathlineto{\pgfqpoint{3.077782in}{3.059397in}}%
\pgfpathlineto{\pgfqpoint{3.077782in}{2.986394in}}%
\pgfpathlineto{\pgfqpoint{2.999829in}{3.022991in}}%
\pgfpathclose%
\pgfusepath{fill}%
\end{pgfscope}%
\begin{pgfscope}%
\pgfpathrectangle{\pgfqpoint{0.765000in}{0.660000in}}{\pgfqpoint{4.620000in}{4.620000in}}%
\pgfusepath{clip}%
\pgfsetbuttcap%
\pgfsetroundjoin%
\definecolor{currentfill}{rgb}{0.176471,0.192157,0.258824}%
\pgfsetfillcolor{currentfill}%
\pgfsetfillopacity{0.050000}%
\pgfsetlinewidth{0.000000pt}%
\definecolor{currentstroke}{rgb}{0.176471,0.192157,0.258824}%
\pgfsetstrokecolor{currentstroke}%
\pgfsetdash{}{0pt}%
\pgfpathmoveto{\pgfqpoint{3.155735in}{3.022991in}}%
\pgfpathlineto{\pgfqpoint{3.155735in}{3.095994in}}%
\pgfpathlineto{\pgfqpoint{3.077782in}{3.059397in}}%
\pgfpathlineto{\pgfqpoint{3.077782in}{2.986394in}}%
\pgfpathlineto{\pgfqpoint{3.155735in}{3.022991in}}%
\pgfpathclose%
\pgfusepath{fill}%
\end{pgfscope}%
\begin{pgfscope}%
\pgfpathrectangle{\pgfqpoint{0.765000in}{0.660000in}}{\pgfqpoint{4.620000in}{4.620000in}}%
\pgfusepath{clip}%
\pgfsetbuttcap%
\pgfsetroundjoin%
\definecolor{currentfill}{rgb}{0.176471,0.192157,0.258824}%
\pgfsetfillcolor{currentfill}%
\pgfsetfillopacity{0.050000}%
\pgfsetlinewidth{0.000000pt}%
\definecolor{currentstroke}{rgb}{0.176471,0.192157,0.258824}%
\pgfsetstrokecolor{currentstroke}%
\pgfsetdash{}{0pt}%
\pgfpathmoveto{\pgfqpoint{3.269747in}{2.856184in}}%
\pgfpathlineto{\pgfqpoint{3.269747in}{2.929187in}}%
\pgfpathlineto{\pgfqpoint{3.347699in}{2.892591in}}%
\pgfpathlineto{\pgfqpoint{3.347699in}{2.819587in}}%
\pgfpathlineto{\pgfqpoint{3.269747in}{2.856184in}}%
\pgfpathclose%
\pgfusepath{fill}%
\end{pgfscope}%
\begin{pgfscope}%
\pgfpathrectangle{\pgfqpoint{0.765000in}{0.660000in}}{\pgfqpoint{4.620000in}{4.620000in}}%
\pgfusepath{clip}%
\pgfsetbuttcap%
\pgfsetroundjoin%
\definecolor{currentfill}{rgb}{0.176471,0.192157,0.258824}%
\pgfsetfillcolor{currentfill}%
\pgfsetfillopacity{0.050000}%
\pgfsetlinewidth{0.000000pt}%
\definecolor{currentstroke}{rgb}{0.176471,0.192157,0.258824}%
\pgfsetstrokecolor{currentstroke}%
\pgfsetdash{}{0pt}%
\pgfpathmoveto{\pgfqpoint{3.269747in}{2.856184in}}%
\pgfpathlineto{\pgfqpoint{3.269747in}{2.929187in}}%
\pgfpathlineto{\pgfqpoint{3.191794in}{2.892591in}}%
\pgfpathlineto{\pgfqpoint{3.191794in}{2.819587in}}%
\pgfpathlineto{\pgfqpoint{3.269747in}{2.856184in}}%
\pgfpathclose%
\pgfusepath{fill}%
\end{pgfscope}%
\begin{pgfscope}%
\pgfpathrectangle{\pgfqpoint{0.765000in}{0.660000in}}{\pgfqpoint{4.620000in}{4.620000in}}%
\pgfusepath{clip}%
\pgfsetbuttcap%
\pgfsetroundjoin%
\definecolor{currentfill}{rgb}{0.176471,0.192157,0.258824}%
\pgfsetfillcolor{currentfill}%
\pgfsetfillopacity{0.050000}%
\pgfsetlinewidth{0.000000pt}%
\definecolor{currentstroke}{rgb}{0.176471,0.192157,0.258824}%
\pgfsetstrokecolor{currentstroke}%
\pgfsetdash{}{0pt}%
\pgfpathmoveto{\pgfqpoint{3.269747in}{2.856184in}}%
\pgfpathlineto{\pgfqpoint{3.347699in}{2.819587in}}%
\pgfpathlineto{\pgfqpoint{3.269747in}{2.782991in}}%
\pgfpathlineto{\pgfqpoint{3.191794in}{2.819587in}}%
\pgfpathlineto{\pgfqpoint{3.269747in}{2.856184in}}%
\pgfpathclose%
\pgfusepath{fill}%
\end{pgfscope}%
\begin{pgfscope}%
\pgfpathrectangle{\pgfqpoint{0.765000in}{0.660000in}}{\pgfqpoint{4.620000in}{4.620000in}}%
\pgfusepath{clip}%
\pgfsetbuttcap%
\pgfsetroundjoin%
\definecolor{currentfill}{rgb}{0.176471,0.192157,0.258824}%
\pgfsetfillcolor{currentfill}%
\pgfsetfillopacity{0.050000}%
\pgfsetlinewidth{0.000000pt}%
\definecolor{currentstroke}{rgb}{0.176471,0.192157,0.258824}%
\pgfsetstrokecolor{currentstroke}%
\pgfsetdash{}{0pt}%
\pgfpathmoveto{\pgfqpoint{3.269747in}{2.929187in}}%
\pgfpathlineto{\pgfqpoint{3.347699in}{2.892591in}}%
\pgfpathlineto{\pgfqpoint{3.269747in}{2.855994in}}%
\pgfpathlineto{\pgfqpoint{3.191794in}{2.892591in}}%
\pgfpathlineto{\pgfqpoint{3.269747in}{2.929187in}}%
\pgfpathclose%
\pgfusepath{fill}%
\end{pgfscope}%
\begin{pgfscope}%
\pgfpathrectangle{\pgfqpoint{0.765000in}{0.660000in}}{\pgfqpoint{4.620000in}{4.620000in}}%
\pgfusepath{clip}%
\pgfsetbuttcap%
\pgfsetroundjoin%
\definecolor{currentfill}{rgb}{0.176471,0.192157,0.258824}%
\pgfsetfillcolor{currentfill}%
\pgfsetfillopacity{0.050000}%
\pgfsetlinewidth{0.000000pt}%
\definecolor{currentstroke}{rgb}{0.176471,0.192157,0.258824}%
\pgfsetstrokecolor{currentstroke}%
\pgfsetdash{}{0pt}%
\pgfpathmoveto{\pgfqpoint{3.191794in}{2.819587in}}%
\pgfpathlineto{\pgfqpoint{3.191794in}{2.892591in}}%
\pgfpathlineto{\pgfqpoint{3.269747in}{2.855994in}}%
\pgfpathlineto{\pgfqpoint{3.269747in}{2.782991in}}%
\pgfpathlineto{\pgfqpoint{3.191794in}{2.819587in}}%
\pgfpathclose%
\pgfusepath{fill}%
\end{pgfscope}%
\begin{pgfscope}%
\pgfpathrectangle{\pgfqpoint{0.765000in}{0.660000in}}{\pgfqpoint{4.620000in}{4.620000in}}%
\pgfusepath{clip}%
\pgfsetbuttcap%
\pgfsetroundjoin%
\definecolor{currentfill}{rgb}{0.176471,0.192157,0.258824}%
\pgfsetfillcolor{currentfill}%
\pgfsetfillopacity{0.050000}%
\pgfsetlinewidth{0.000000pt}%
\definecolor{currentstroke}{rgb}{0.176471,0.192157,0.258824}%
\pgfsetstrokecolor{currentstroke}%
\pgfsetdash{}{0pt}%
\pgfpathmoveto{\pgfqpoint{3.347699in}{2.819587in}}%
\pgfpathlineto{\pgfqpoint{3.347699in}{2.892591in}}%
\pgfpathlineto{\pgfqpoint{3.269747in}{2.855994in}}%
\pgfpathlineto{\pgfqpoint{3.269747in}{2.782991in}}%
\pgfpathlineto{\pgfqpoint{3.347699in}{2.819587in}}%
\pgfpathclose%
\pgfusepath{fill}%
\end{pgfscope}%
\begin{pgfscope}%
\pgfpathrectangle{\pgfqpoint{0.765000in}{0.660000in}}{\pgfqpoint{4.620000in}{4.620000in}}%
\pgfusepath{clip}%
\pgfsetbuttcap%
\pgfsetroundjoin%
\definecolor{currentfill}{rgb}{1.000000,0.576471,0.309804}%
\pgfsetfillcolor{currentfill}%
\pgfsetfillopacity{0.050000}%
\pgfsetlinewidth{0.000000pt}%
\definecolor{currentstroke}{rgb}{1.000000,0.576471,0.309804}%
\pgfsetstrokecolor{currentstroke}%
\pgfsetdash{}{0pt}%
\pgfpathmoveto{\pgfqpoint{4.206167in}{2.348692in}}%
\pgfpathlineto{\pgfqpoint{4.206167in}{2.421696in}}%
\pgfpathlineto{\pgfqpoint{4.284120in}{2.385099in}}%
\pgfpathlineto{\pgfqpoint{4.284120in}{2.312096in}}%
\pgfpathlineto{\pgfqpoint{4.206167in}{2.348692in}}%
\pgfpathclose%
\pgfusepath{fill}%
\end{pgfscope}%
\begin{pgfscope}%
\pgfpathrectangle{\pgfqpoint{0.765000in}{0.660000in}}{\pgfqpoint{4.620000in}{4.620000in}}%
\pgfusepath{clip}%
\pgfsetbuttcap%
\pgfsetroundjoin%
\definecolor{currentfill}{rgb}{1.000000,0.576471,0.309804}%
\pgfsetfillcolor{currentfill}%
\pgfsetfillopacity{0.050000}%
\pgfsetlinewidth{0.000000pt}%
\definecolor{currentstroke}{rgb}{1.000000,0.576471,0.309804}%
\pgfsetstrokecolor{currentstroke}%
\pgfsetdash{}{0pt}%
\pgfpathmoveto{\pgfqpoint{4.206167in}{2.348692in}}%
\pgfpathlineto{\pgfqpoint{4.206167in}{2.421696in}}%
\pgfpathlineto{\pgfqpoint{4.128214in}{2.385099in}}%
\pgfpathlineto{\pgfqpoint{4.128214in}{2.312096in}}%
\pgfpathlineto{\pgfqpoint{4.206167in}{2.348692in}}%
\pgfpathclose%
\pgfusepath{fill}%
\end{pgfscope}%
\begin{pgfscope}%
\pgfpathrectangle{\pgfqpoint{0.765000in}{0.660000in}}{\pgfqpoint{4.620000in}{4.620000in}}%
\pgfusepath{clip}%
\pgfsetbuttcap%
\pgfsetroundjoin%
\definecolor{currentfill}{rgb}{1.000000,0.576471,0.309804}%
\pgfsetfillcolor{currentfill}%
\pgfsetfillopacity{0.050000}%
\pgfsetlinewidth{0.000000pt}%
\definecolor{currentstroke}{rgb}{1.000000,0.576471,0.309804}%
\pgfsetstrokecolor{currentstroke}%
\pgfsetdash{}{0pt}%
\pgfpathmoveto{\pgfqpoint{4.206167in}{2.348692in}}%
\pgfpathlineto{\pgfqpoint{4.284120in}{2.312096in}}%
\pgfpathlineto{\pgfqpoint{4.206167in}{2.275499in}}%
\pgfpathlineto{\pgfqpoint{4.128214in}{2.312096in}}%
\pgfpathlineto{\pgfqpoint{4.206167in}{2.348692in}}%
\pgfpathclose%
\pgfusepath{fill}%
\end{pgfscope}%
\begin{pgfscope}%
\pgfpathrectangle{\pgfqpoint{0.765000in}{0.660000in}}{\pgfqpoint{4.620000in}{4.620000in}}%
\pgfusepath{clip}%
\pgfsetbuttcap%
\pgfsetroundjoin%
\definecolor{currentfill}{rgb}{1.000000,0.576471,0.309804}%
\pgfsetfillcolor{currentfill}%
\pgfsetfillopacity{0.050000}%
\pgfsetlinewidth{0.000000pt}%
\definecolor{currentstroke}{rgb}{1.000000,0.576471,0.309804}%
\pgfsetstrokecolor{currentstroke}%
\pgfsetdash{}{0pt}%
\pgfpathmoveto{\pgfqpoint{4.206167in}{2.421696in}}%
\pgfpathlineto{\pgfqpoint{4.284120in}{2.385099in}}%
\pgfpathlineto{\pgfqpoint{4.206167in}{2.348503in}}%
\pgfpathlineto{\pgfqpoint{4.128214in}{2.385099in}}%
\pgfpathlineto{\pgfqpoint{4.206167in}{2.421696in}}%
\pgfpathclose%
\pgfusepath{fill}%
\end{pgfscope}%
\begin{pgfscope}%
\pgfpathrectangle{\pgfqpoint{0.765000in}{0.660000in}}{\pgfqpoint{4.620000in}{4.620000in}}%
\pgfusepath{clip}%
\pgfsetbuttcap%
\pgfsetroundjoin%
\definecolor{currentfill}{rgb}{1.000000,0.576471,0.309804}%
\pgfsetfillcolor{currentfill}%
\pgfsetfillopacity{0.050000}%
\pgfsetlinewidth{0.000000pt}%
\definecolor{currentstroke}{rgb}{1.000000,0.576471,0.309804}%
\pgfsetstrokecolor{currentstroke}%
\pgfsetdash{}{0pt}%
\pgfpathmoveto{\pgfqpoint{4.128214in}{2.312096in}}%
\pgfpathlineto{\pgfqpoint{4.128214in}{2.385099in}}%
\pgfpathlineto{\pgfqpoint{4.206167in}{2.348503in}}%
\pgfpathlineto{\pgfqpoint{4.206167in}{2.275499in}}%
\pgfpathlineto{\pgfqpoint{4.128214in}{2.312096in}}%
\pgfpathclose%
\pgfusepath{fill}%
\end{pgfscope}%
\begin{pgfscope}%
\pgfpathrectangle{\pgfqpoint{0.765000in}{0.660000in}}{\pgfqpoint{4.620000in}{4.620000in}}%
\pgfusepath{clip}%
\pgfsetbuttcap%
\pgfsetroundjoin%
\definecolor{currentfill}{rgb}{1.000000,0.576471,0.309804}%
\pgfsetfillcolor{currentfill}%
\pgfsetfillopacity{0.050000}%
\pgfsetlinewidth{0.000000pt}%
\definecolor{currentstroke}{rgb}{1.000000,0.576471,0.309804}%
\pgfsetstrokecolor{currentstroke}%
\pgfsetdash{}{0pt}%
\pgfpathmoveto{\pgfqpoint{4.284120in}{2.312096in}}%
\pgfpathlineto{\pgfqpoint{4.284120in}{2.385099in}}%
\pgfpathlineto{\pgfqpoint{4.206167in}{2.348503in}}%
\pgfpathlineto{\pgfqpoint{4.206167in}{2.275499in}}%
\pgfpathlineto{\pgfqpoint{4.284120in}{2.312096in}}%
\pgfpathclose%
\pgfusepath{fill}%
\end{pgfscope}%
\begin{pgfscope}%
\pgfpathrectangle{\pgfqpoint{0.765000in}{0.660000in}}{\pgfqpoint{4.620000in}{4.620000in}}%
\pgfusepath{clip}%
\pgfsetbuttcap%
\pgfsetroundjoin%
\definecolor{currentfill}{rgb}{1.000000,0.576471,0.309804}%
\pgfsetfillcolor{currentfill}%
\pgfsetfillopacity{0.050000}%
\pgfsetlinewidth{0.000000pt}%
\definecolor{currentstroke}{rgb}{1.000000,0.576471,0.309804}%
\pgfsetstrokecolor{currentstroke}%
\pgfsetdash{}{0pt}%
\pgfpathmoveto{\pgfqpoint{3.035994in}{2.387957in}}%
\pgfpathlineto{\pgfqpoint{3.035994in}{2.460961in}}%
\pgfpathlineto{\pgfqpoint{3.113947in}{2.424364in}}%
\pgfpathlineto{\pgfqpoint{3.113947in}{2.351361in}}%
\pgfpathlineto{\pgfqpoint{3.035994in}{2.387957in}}%
\pgfpathclose%
\pgfusepath{fill}%
\end{pgfscope}%
\begin{pgfscope}%
\pgfpathrectangle{\pgfqpoint{0.765000in}{0.660000in}}{\pgfqpoint{4.620000in}{4.620000in}}%
\pgfusepath{clip}%
\pgfsetbuttcap%
\pgfsetroundjoin%
\definecolor{currentfill}{rgb}{1.000000,0.576471,0.309804}%
\pgfsetfillcolor{currentfill}%
\pgfsetfillopacity{0.050000}%
\pgfsetlinewidth{0.000000pt}%
\definecolor{currentstroke}{rgb}{1.000000,0.576471,0.309804}%
\pgfsetstrokecolor{currentstroke}%
\pgfsetdash{}{0pt}%
\pgfpathmoveto{\pgfqpoint{3.035994in}{2.387957in}}%
\pgfpathlineto{\pgfqpoint{3.035994in}{2.460961in}}%
\pgfpathlineto{\pgfqpoint{2.958041in}{2.424364in}}%
\pgfpathlineto{\pgfqpoint{2.958041in}{2.351361in}}%
\pgfpathlineto{\pgfqpoint{3.035994in}{2.387957in}}%
\pgfpathclose%
\pgfusepath{fill}%
\end{pgfscope}%
\begin{pgfscope}%
\pgfpathrectangle{\pgfqpoint{0.765000in}{0.660000in}}{\pgfqpoint{4.620000in}{4.620000in}}%
\pgfusepath{clip}%
\pgfsetbuttcap%
\pgfsetroundjoin%
\definecolor{currentfill}{rgb}{1.000000,0.576471,0.309804}%
\pgfsetfillcolor{currentfill}%
\pgfsetfillopacity{0.050000}%
\pgfsetlinewidth{0.000000pt}%
\definecolor{currentstroke}{rgb}{1.000000,0.576471,0.309804}%
\pgfsetstrokecolor{currentstroke}%
\pgfsetdash{}{0pt}%
\pgfpathmoveto{\pgfqpoint{3.035994in}{2.387957in}}%
\pgfpathlineto{\pgfqpoint{3.113947in}{2.351361in}}%
\pgfpathlineto{\pgfqpoint{3.035994in}{2.314764in}}%
\pgfpathlineto{\pgfqpoint{2.958041in}{2.351361in}}%
\pgfpathlineto{\pgfqpoint{3.035994in}{2.387957in}}%
\pgfpathclose%
\pgfusepath{fill}%
\end{pgfscope}%
\begin{pgfscope}%
\pgfpathrectangle{\pgfqpoint{0.765000in}{0.660000in}}{\pgfqpoint{4.620000in}{4.620000in}}%
\pgfusepath{clip}%
\pgfsetbuttcap%
\pgfsetroundjoin%
\definecolor{currentfill}{rgb}{1.000000,0.576471,0.309804}%
\pgfsetfillcolor{currentfill}%
\pgfsetfillopacity{0.050000}%
\pgfsetlinewidth{0.000000pt}%
\definecolor{currentstroke}{rgb}{1.000000,0.576471,0.309804}%
\pgfsetstrokecolor{currentstroke}%
\pgfsetdash{}{0pt}%
\pgfpathmoveto{\pgfqpoint{3.035994in}{2.460961in}}%
\pgfpathlineto{\pgfqpoint{3.113947in}{2.424364in}}%
\pgfpathlineto{\pgfqpoint{3.035994in}{2.387767in}}%
\pgfpathlineto{\pgfqpoint{2.958041in}{2.424364in}}%
\pgfpathlineto{\pgfqpoint{3.035994in}{2.460961in}}%
\pgfpathclose%
\pgfusepath{fill}%
\end{pgfscope}%
\begin{pgfscope}%
\pgfpathrectangle{\pgfqpoint{0.765000in}{0.660000in}}{\pgfqpoint{4.620000in}{4.620000in}}%
\pgfusepath{clip}%
\pgfsetbuttcap%
\pgfsetroundjoin%
\definecolor{currentfill}{rgb}{1.000000,0.576471,0.309804}%
\pgfsetfillcolor{currentfill}%
\pgfsetfillopacity{0.050000}%
\pgfsetlinewidth{0.000000pt}%
\definecolor{currentstroke}{rgb}{1.000000,0.576471,0.309804}%
\pgfsetstrokecolor{currentstroke}%
\pgfsetdash{}{0pt}%
\pgfpathmoveto{\pgfqpoint{2.958041in}{2.351361in}}%
\pgfpathlineto{\pgfqpoint{2.958041in}{2.424364in}}%
\pgfpathlineto{\pgfqpoint{3.035994in}{2.387767in}}%
\pgfpathlineto{\pgfqpoint{3.035994in}{2.314764in}}%
\pgfpathlineto{\pgfqpoint{2.958041in}{2.351361in}}%
\pgfpathclose%
\pgfusepath{fill}%
\end{pgfscope}%
\begin{pgfscope}%
\pgfpathrectangle{\pgfqpoint{0.765000in}{0.660000in}}{\pgfqpoint{4.620000in}{4.620000in}}%
\pgfusepath{clip}%
\pgfsetbuttcap%
\pgfsetroundjoin%
\definecolor{currentfill}{rgb}{1.000000,0.576471,0.309804}%
\pgfsetfillcolor{currentfill}%
\pgfsetfillopacity{0.050000}%
\pgfsetlinewidth{0.000000pt}%
\definecolor{currentstroke}{rgb}{1.000000,0.576471,0.309804}%
\pgfsetstrokecolor{currentstroke}%
\pgfsetdash{}{0pt}%
\pgfpathmoveto{\pgfqpoint{3.113947in}{2.351361in}}%
\pgfpathlineto{\pgfqpoint{3.113947in}{2.424364in}}%
\pgfpathlineto{\pgfqpoint{3.035994in}{2.387767in}}%
\pgfpathlineto{\pgfqpoint{3.035994in}{2.314764in}}%
\pgfpathlineto{\pgfqpoint{3.113947in}{2.351361in}}%
\pgfpathclose%
\pgfusepath{fill}%
\end{pgfscope}%
\begin{pgfscope}%
\pgfpathrectangle{\pgfqpoint{0.765000in}{0.660000in}}{\pgfqpoint{4.620000in}{4.620000in}}%
\pgfusepath{clip}%
\pgfsetbuttcap%
\pgfsetroundjoin%
\definecolor{currentfill}{rgb}{1.000000,0.576471,0.309804}%
\pgfsetfillcolor{currentfill}%
\pgfsetfillopacity{0.050000}%
\pgfsetlinewidth{0.000000pt}%
\definecolor{currentstroke}{rgb}{1.000000,0.576471,0.309804}%
\pgfsetstrokecolor{currentstroke}%
\pgfsetdash{}{0pt}%
\pgfpathmoveto{\pgfqpoint{3.827050in}{2.761936in}}%
\pgfpathlineto{\pgfqpoint{3.827050in}{2.834939in}}%
\pgfpathlineto{\pgfqpoint{3.905003in}{2.798342in}}%
\pgfpathlineto{\pgfqpoint{3.905003in}{2.725339in}}%
\pgfpathlineto{\pgfqpoint{3.827050in}{2.761936in}}%
\pgfpathclose%
\pgfusepath{fill}%
\end{pgfscope}%
\begin{pgfscope}%
\pgfpathrectangle{\pgfqpoint{0.765000in}{0.660000in}}{\pgfqpoint{4.620000in}{4.620000in}}%
\pgfusepath{clip}%
\pgfsetbuttcap%
\pgfsetroundjoin%
\definecolor{currentfill}{rgb}{1.000000,0.576471,0.309804}%
\pgfsetfillcolor{currentfill}%
\pgfsetfillopacity{0.050000}%
\pgfsetlinewidth{0.000000pt}%
\definecolor{currentstroke}{rgb}{1.000000,0.576471,0.309804}%
\pgfsetstrokecolor{currentstroke}%
\pgfsetdash{}{0pt}%
\pgfpathmoveto{\pgfqpoint{3.827050in}{2.761936in}}%
\pgfpathlineto{\pgfqpoint{3.827050in}{2.834939in}}%
\pgfpathlineto{\pgfqpoint{3.749097in}{2.798342in}}%
\pgfpathlineto{\pgfqpoint{3.749097in}{2.725339in}}%
\pgfpathlineto{\pgfqpoint{3.827050in}{2.761936in}}%
\pgfpathclose%
\pgfusepath{fill}%
\end{pgfscope}%
\begin{pgfscope}%
\pgfpathrectangle{\pgfqpoint{0.765000in}{0.660000in}}{\pgfqpoint{4.620000in}{4.620000in}}%
\pgfusepath{clip}%
\pgfsetbuttcap%
\pgfsetroundjoin%
\definecolor{currentfill}{rgb}{1.000000,0.576471,0.309804}%
\pgfsetfillcolor{currentfill}%
\pgfsetfillopacity{0.050000}%
\pgfsetlinewidth{0.000000pt}%
\definecolor{currentstroke}{rgb}{1.000000,0.576471,0.309804}%
\pgfsetstrokecolor{currentstroke}%
\pgfsetdash{}{0pt}%
\pgfpathmoveto{\pgfqpoint{3.827050in}{2.761936in}}%
\pgfpathlineto{\pgfqpoint{3.905003in}{2.725339in}}%
\pgfpathlineto{\pgfqpoint{3.827050in}{2.688742in}}%
\pgfpathlineto{\pgfqpoint{3.749097in}{2.725339in}}%
\pgfpathlineto{\pgfqpoint{3.827050in}{2.761936in}}%
\pgfpathclose%
\pgfusepath{fill}%
\end{pgfscope}%
\begin{pgfscope}%
\pgfpathrectangle{\pgfqpoint{0.765000in}{0.660000in}}{\pgfqpoint{4.620000in}{4.620000in}}%
\pgfusepath{clip}%
\pgfsetbuttcap%
\pgfsetroundjoin%
\definecolor{currentfill}{rgb}{1.000000,0.576471,0.309804}%
\pgfsetfillcolor{currentfill}%
\pgfsetfillopacity{0.050000}%
\pgfsetlinewidth{0.000000pt}%
\definecolor{currentstroke}{rgb}{1.000000,0.576471,0.309804}%
\pgfsetstrokecolor{currentstroke}%
\pgfsetdash{}{0pt}%
\pgfpathmoveto{\pgfqpoint{3.827050in}{2.834939in}}%
\pgfpathlineto{\pgfqpoint{3.905003in}{2.798342in}}%
\pgfpathlineto{\pgfqpoint{3.827050in}{2.761746in}}%
\pgfpathlineto{\pgfqpoint{3.749097in}{2.798342in}}%
\pgfpathlineto{\pgfqpoint{3.827050in}{2.834939in}}%
\pgfpathclose%
\pgfusepath{fill}%
\end{pgfscope}%
\begin{pgfscope}%
\pgfpathrectangle{\pgfqpoint{0.765000in}{0.660000in}}{\pgfqpoint{4.620000in}{4.620000in}}%
\pgfusepath{clip}%
\pgfsetbuttcap%
\pgfsetroundjoin%
\definecolor{currentfill}{rgb}{1.000000,0.576471,0.309804}%
\pgfsetfillcolor{currentfill}%
\pgfsetfillopacity{0.050000}%
\pgfsetlinewidth{0.000000pt}%
\definecolor{currentstroke}{rgb}{1.000000,0.576471,0.309804}%
\pgfsetstrokecolor{currentstroke}%
\pgfsetdash{}{0pt}%
\pgfpathmoveto{\pgfqpoint{3.749097in}{2.725339in}}%
\pgfpathlineto{\pgfqpoint{3.749097in}{2.798342in}}%
\pgfpathlineto{\pgfqpoint{3.827050in}{2.761746in}}%
\pgfpathlineto{\pgfqpoint{3.827050in}{2.688742in}}%
\pgfpathlineto{\pgfqpoint{3.749097in}{2.725339in}}%
\pgfpathclose%
\pgfusepath{fill}%
\end{pgfscope}%
\begin{pgfscope}%
\pgfpathrectangle{\pgfqpoint{0.765000in}{0.660000in}}{\pgfqpoint{4.620000in}{4.620000in}}%
\pgfusepath{clip}%
\pgfsetbuttcap%
\pgfsetroundjoin%
\definecolor{currentfill}{rgb}{1.000000,0.576471,0.309804}%
\pgfsetfillcolor{currentfill}%
\pgfsetfillopacity{0.050000}%
\pgfsetlinewidth{0.000000pt}%
\definecolor{currentstroke}{rgb}{1.000000,0.576471,0.309804}%
\pgfsetstrokecolor{currentstroke}%
\pgfsetdash{}{0pt}%
\pgfpathmoveto{\pgfqpoint{3.905003in}{2.725339in}}%
\pgfpathlineto{\pgfqpoint{3.905003in}{2.798342in}}%
\pgfpathlineto{\pgfqpoint{3.827050in}{2.761746in}}%
\pgfpathlineto{\pgfqpoint{3.827050in}{2.688742in}}%
\pgfpathlineto{\pgfqpoint{3.905003in}{2.725339in}}%
\pgfpathclose%
\pgfusepath{fill}%
\end{pgfscope}%
\begin{pgfscope}%
\pgfpathrectangle{\pgfqpoint{0.765000in}{0.660000in}}{\pgfqpoint{4.620000in}{4.620000in}}%
\pgfusepath{clip}%
\pgfsetbuttcap%
\pgfsetroundjoin%
\definecolor{currentfill}{rgb}{0.176471,0.192157,0.258824}%
\pgfsetfillcolor{currentfill}%
\pgfsetfillopacity{0.050000}%
\pgfsetlinewidth{0.000000pt}%
\definecolor{currentstroke}{rgb}{0.176471,0.192157,0.258824}%
\pgfsetstrokecolor{currentstroke}%
\pgfsetdash{}{0pt}%
\pgfpathmoveto{\pgfqpoint{3.175390in}{2.854152in}}%
\pgfpathlineto{\pgfqpoint{3.175390in}{2.927156in}}%
\pgfpathlineto{\pgfqpoint{3.253343in}{2.890559in}}%
\pgfpathlineto{\pgfqpoint{3.253343in}{2.817556in}}%
\pgfpathlineto{\pgfqpoint{3.175390in}{2.854152in}}%
\pgfpathclose%
\pgfusepath{fill}%
\end{pgfscope}%
\begin{pgfscope}%
\pgfpathrectangle{\pgfqpoint{0.765000in}{0.660000in}}{\pgfqpoint{4.620000in}{4.620000in}}%
\pgfusepath{clip}%
\pgfsetbuttcap%
\pgfsetroundjoin%
\definecolor{currentfill}{rgb}{0.176471,0.192157,0.258824}%
\pgfsetfillcolor{currentfill}%
\pgfsetfillopacity{0.050000}%
\pgfsetlinewidth{0.000000pt}%
\definecolor{currentstroke}{rgb}{0.176471,0.192157,0.258824}%
\pgfsetstrokecolor{currentstroke}%
\pgfsetdash{}{0pt}%
\pgfpathmoveto{\pgfqpoint{3.175390in}{2.854152in}}%
\pgfpathlineto{\pgfqpoint{3.175390in}{2.927156in}}%
\pgfpathlineto{\pgfqpoint{3.097437in}{2.890559in}}%
\pgfpathlineto{\pgfqpoint{3.097437in}{2.817556in}}%
\pgfpathlineto{\pgfqpoint{3.175390in}{2.854152in}}%
\pgfpathclose%
\pgfusepath{fill}%
\end{pgfscope}%
\begin{pgfscope}%
\pgfpathrectangle{\pgfqpoint{0.765000in}{0.660000in}}{\pgfqpoint{4.620000in}{4.620000in}}%
\pgfusepath{clip}%
\pgfsetbuttcap%
\pgfsetroundjoin%
\definecolor{currentfill}{rgb}{0.176471,0.192157,0.258824}%
\pgfsetfillcolor{currentfill}%
\pgfsetfillopacity{0.050000}%
\pgfsetlinewidth{0.000000pt}%
\definecolor{currentstroke}{rgb}{0.176471,0.192157,0.258824}%
\pgfsetstrokecolor{currentstroke}%
\pgfsetdash{}{0pt}%
\pgfpathmoveto{\pgfqpoint{3.175390in}{2.854152in}}%
\pgfpathlineto{\pgfqpoint{3.253343in}{2.817556in}}%
\pgfpathlineto{\pgfqpoint{3.175390in}{2.780959in}}%
\pgfpathlineto{\pgfqpoint{3.097437in}{2.817556in}}%
\pgfpathlineto{\pgfqpoint{3.175390in}{2.854152in}}%
\pgfpathclose%
\pgfusepath{fill}%
\end{pgfscope}%
\begin{pgfscope}%
\pgfpathrectangle{\pgfqpoint{0.765000in}{0.660000in}}{\pgfqpoint{4.620000in}{4.620000in}}%
\pgfusepath{clip}%
\pgfsetbuttcap%
\pgfsetroundjoin%
\definecolor{currentfill}{rgb}{0.176471,0.192157,0.258824}%
\pgfsetfillcolor{currentfill}%
\pgfsetfillopacity{0.050000}%
\pgfsetlinewidth{0.000000pt}%
\definecolor{currentstroke}{rgb}{0.176471,0.192157,0.258824}%
\pgfsetstrokecolor{currentstroke}%
\pgfsetdash{}{0pt}%
\pgfpathmoveto{\pgfqpoint{3.175390in}{2.927156in}}%
\pgfpathlineto{\pgfqpoint{3.253343in}{2.890559in}}%
\pgfpathlineto{\pgfqpoint{3.175390in}{2.853963in}}%
\pgfpathlineto{\pgfqpoint{3.097437in}{2.890559in}}%
\pgfpathlineto{\pgfqpoint{3.175390in}{2.927156in}}%
\pgfpathclose%
\pgfusepath{fill}%
\end{pgfscope}%
\begin{pgfscope}%
\pgfpathrectangle{\pgfqpoint{0.765000in}{0.660000in}}{\pgfqpoint{4.620000in}{4.620000in}}%
\pgfusepath{clip}%
\pgfsetbuttcap%
\pgfsetroundjoin%
\definecolor{currentfill}{rgb}{0.176471,0.192157,0.258824}%
\pgfsetfillcolor{currentfill}%
\pgfsetfillopacity{0.050000}%
\pgfsetlinewidth{0.000000pt}%
\definecolor{currentstroke}{rgb}{0.176471,0.192157,0.258824}%
\pgfsetstrokecolor{currentstroke}%
\pgfsetdash{}{0pt}%
\pgfpathmoveto{\pgfqpoint{3.097437in}{2.817556in}}%
\pgfpathlineto{\pgfqpoint{3.097437in}{2.890559in}}%
\pgfpathlineto{\pgfqpoint{3.175390in}{2.853963in}}%
\pgfpathlineto{\pgfqpoint{3.175390in}{2.780959in}}%
\pgfpathlineto{\pgfqpoint{3.097437in}{2.817556in}}%
\pgfpathclose%
\pgfusepath{fill}%
\end{pgfscope}%
\begin{pgfscope}%
\pgfpathrectangle{\pgfqpoint{0.765000in}{0.660000in}}{\pgfqpoint{4.620000in}{4.620000in}}%
\pgfusepath{clip}%
\pgfsetbuttcap%
\pgfsetroundjoin%
\definecolor{currentfill}{rgb}{0.176471,0.192157,0.258824}%
\pgfsetfillcolor{currentfill}%
\pgfsetfillopacity{0.050000}%
\pgfsetlinewidth{0.000000pt}%
\definecolor{currentstroke}{rgb}{0.176471,0.192157,0.258824}%
\pgfsetstrokecolor{currentstroke}%
\pgfsetdash{}{0pt}%
\pgfpathmoveto{\pgfqpoint{3.253343in}{2.817556in}}%
\pgfpathlineto{\pgfqpoint{3.253343in}{2.890559in}}%
\pgfpathlineto{\pgfqpoint{3.175390in}{2.853963in}}%
\pgfpathlineto{\pgfqpoint{3.175390in}{2.780959in}}%
\pgfpathlineto{\pgfqpoint{3.253343in}{2.817556in}}%
\pgfpathclose%
\pgfusepath{fill}%
\end{pgfscope}%
\begin{pgfscope}%
\pgfpathrectangle{\pgfqpoint{0.765000in}{0.660000in}}{\pgfqpoint{4.620000in}{4.620000in}}%
\pgfusepath{clip}%
\pgfsetbuttcap%
\pgfsetroundjoin%
\definecolor{currentfill}{rgb}{1.000000,0.576471,0.309804}%
\pgfsetfillcolor{currentfill}%
\pgfsetfillopacity{0.050000}%
\pgfsetlinewidth{0.000000pt}%
\definecolor{currentstroke}{rgb}{1.000000,0.576471,0.309804}%
\pgfsetstrokecolor{currentstroke}%
\pgfsetdash{}{0pt}%
\pgfpathmoveto{\pgfqpoint{3.804988in}{2.449206in}}%
\pgfpathlineto{\pgfqpoint{3.804988in}{2.522210in}}%
\pgfpathlineto{\pgfqpoint{3.882941in}{2.485613in}}%
\pgfpathlineto{\pgfqpoint{3.882941in}{2.412610in}}%
\pgfpathlineto{\pgfqpoint{3.804988in}{2.449206in}}%
\pgfpathclose%
\pgfusepath{fill}%
\end{pgfscope}%
\begin{pgfscope}%
\pgfpathrectangle{\pgfqpoint{0.765000in}{0.660000in}}{\pgfqpoint{4.620000in}{4.620000in}}%
\pgfusepath{clip}%
\pgfsetbuttcap%
\pgfsetroundjoin%
\definecolor{currentfill}{rgb}{1.000000,0.576471,0.309804}%
\pgfsetfillcolor{currentfill}%
\pgfsetfillopacity{0.050000}%
\pgfsetlinewidth{0.000000pt}%
\definecolor{currentstroke}{rgb}{1.000000,0.576471,0.309804}%
\pgfsetstrokecolor{currentstroke}%
\pgfsetdash{}{0pt}%
\pgfpathmoveto{\pgfqpoint{3.804988in}{2.449206in}}%
\pgfpathlineto{\pgfqpoint{3.804988in}{2.522210in}}%
\pgfpathlineto{\pgfqpoint{3.727035in}{2.485613in}}%
\pgfpathlineto{\pgfqpoint{3.727035in}{2.412610in}}%
\pgfpathlineto{\pgfqpoint{3.804988in}{2.449206in}}%
\pgfpathclose%
\pgfusepath{fill}%
\end{pgfscope}%
\begin{pgfscope}%
\pgfpathrectangle{\pgfqpoint{0.765000in}{0.660000in}}{\pgfqpoint{4.620000in}{4.620000in}}%
\pgfusepath{clip}%
\pgfsetbuttcap%
\pgfsetroundjoin%
\definecolor{currentfill}{rgb}{1.000000,0.576471,0.309804}%
\pgfsetfillcolor{currentfill}%
\pgfsetfillopacity{0.050000}%
\pgfsetlinewidth{0.000000pt}%
\definecolor{currentstroke}{rgb}{1.000000,0.576471,0.309804}%
\pgfsetstrokecolor{currentstroke}%
\pgfsetdash{}{0pt}%
\pgfpathmoveto{\pgfqpoint{3.804988in}{2.449206in}}%
\pgfpathlineto{\pgfqpoint{3.882941in}{2.412610in}}%
\pgfpathlineto{\pgfqpoint{3.804988in}{2.376013in}}%
\pgfpathlineto{\pgfqpoint{3.727035in}{2.412610in}}%
\pgfpathlineto{\pgfqpoint{3.804988in}{2.449206in}}%
\pgfpathclose%
\pgfusepath{fill}%
\end{pgfscope}%
\begin{pgfscope}%
\pgfpathrectangle{\pgfqpoint{0.765000in}{0.660000in}}{\pgfqpoint{4.620000in}{4.620000in}}%
\pgfusepath{clip}%
\pgfsetbuttcap%
\pgfsetroundjoin%
\definecolor{currentfill}{rgb}{1.000000,0.576471,0.309804}%
\pgfsetfillcolor{currentfill}%
\pgfsetfillopacity{0.050000}%
\pgfsetlinewidth{0.000000pt}%
\definecolor{currentstroke}{rgb}{1.000000,0.576471,0.309804}%
\pgfsetstrokecolor{currentstroke}%
\pgfsetdash{}{0pt}%
\pgfpathmoveto{\pgfqpoint{3.804988in}{2.522210in}}%
\pgfpathlineto{\pgfqpoint{3.882941in}{2.485613in}}%
\pgfpathlineto{\pgfqpoint{3.804988in}{2.449017in}}%
\pgfpathlineto{\pgfqpoint{3.727035in}{2.485613in}}%
\pgfpathlineto{\pgfqpoint{3.804988in}{2.522210in}}%
\pgfpathclose%
\pgfusepath{fill}%
\end{pgfscope}%
\begin{pgfscope}%
\pgfpathrectangle{\pgfqpoint{0.765000in}{0.660000in}}{\pgfqpoint{4.620000in}{4.620000in}}%
\pgfusepath{clip}%
\pgfsetbuttcap%
\pgfsetroundjoin%
\definecolor{currentfill}{rgb}{1.000000,0.576471,0.309804}%
\pgfsetfillcolor{currentfill}%
\pgfsetfillopacity{0.050000}%
\pgfsetlinewidth{0.000000pt}%
\definecolor{currentstroke}{rgb}{1.000000,0.576471,0.309804}%
\pgfsetstrokecolor{currentstroke}%
\pgfsetdash{}{0pt}%
\pgfpathmoveto{\pgfqpoint{3.727035in}{2.412610in}}%
\pgfpathlineto{\pgfqpoint{3.727035in}{2.485613in}}%
\pgfpathlineto{\pgfqpoint{3.804988in}{2.449017in}}%
\pgfpathlineto{\pgfqpoint{3.804988in}{2.376013in}}%
\pgfpathlineto{\pgfqpoint{3.727035in}{2.412610in}}%
\pgfpathclose%
\pgfusepath{fill}%
\end{pgfscope}%
\begin{pgfscope}%
\pgfpathrectangle{\pgfqpoint{0.765000in}{0.660000in}}{\pgfqpoint{4.620000in}{4.620000in}}%
\pgfusepath{clip}%
\pgfsetbuttcap%
\pgfsetroundjoin%
\definecolor{currentfill}{rgb}{1.000000,0.576471,0.309804}%
\pgfsetfillcolor{currentfill}%
\pgfsetfillopacity{0.050000}%
\pgfsetlinewidth{0.000000pt}%
\definecolor{currentstroke}{rgb}{1.000000,0.576471,0.309804}%
\pgfsetstrokecolor{currentstroke}%
\pgfsetdash{}{0pt}%
\pgfpathmoveto{\pgfqpoint{3.882941in}{2.412610in}}%
\pgfpathlineto{\pgfqpoint{3.882941in}{2.485613in}}%
\pgfpathlineto{\pgfqpoint{3.804988in}{2.449017in}}%
\pgfpathlineto{\pgfqpoint{3.804988in}{2.376013in}}%
\pgfpathlineto{\pgfqpoint{3.882941in}{2.412610in}}%
\pgfpathclose%
\pgfusepath{fill}%
\end{pgfscope}%
\begin{pgfscope}%
\pgfpathrectangle{\pgfqpoint{0.765000in}{0.660000in}}{\pgfqpoint{4.620000in}{4.620000in}}%
\pgfusepath{clip}%
\pgfsetbuttcap%
\pgfsetroundjoin%
\definecolor{currentfill}{rgb}{0.176471,0.192157,0.258824}%
\pgfsetfillcolor{currentfill}%
\pgfsetfillopacity{0.050000}%
\pgfsetlinewidth{0.000000pt}%
\definecolor{currentstroke}{rgb}{0.176471,0.192157,0.258824}%
\pgfsetstrokecolor{currentstroke}%
\pgfsetdash{}{0pt}%
\pgfpathmoveto{\pgfqpoint{3.152537in}{2.887662in}}%
\pgfpathlineto{\pgfqpoint{3.152537in}{2.960666in}}%
\pgfpathlineto{\pgfqpoint{3.230490in}{2.924069in}}%
\pgfpathlineto{\pgfqpoint{3.230490in}{2.851066in}}%
\pgfpathlineto{\pgfqpoint{3.152537in}{2.887662in}}%
\pgfpathclose%
\pgfusepath{fill}%
\end{pgfscope}%
\begin{pgfscope}%
\pgfpathrectangle{\pgfqpoint{0.765000in}{0.660000in}}{\pgfqpoint{4.620000in}{4.620000in}}%
\pgfusepath{clip}%
\pgfsetbuttcap%
\pgfsetroundjoin%
\definecolor{currentfill}{rgb}{0.176471,0.192157,0.258824}%
\pgfsetfillcolor{currentfill}%
\pgfsetfillopacity{0.050000}%
\pgfsetlinewidth{0.000000pt}%
\definecolor{currentstroke}{rgb}{0.176471,0.192157,0.258824}%
\pgfsetstrokecolor{currentstroke}%
\pgfsetdash{}{0pt}%
\pgfpathmoveto{\pgfqpoint{3.152537in}{2.887662in}}%
\pgfpathlineto{\pgfqpoint{3.152537in}{2.960666in}}%
\pgfpathlineto{\pgfqpoint{3.074584in}{2.924069in}}%
\pgfpathlineto{\pgfqpoint{3.074584in}{2.851066in}}%
\pgfpathlineto{\pgfqpoint{3.152537in}{2.887662in}}%
\pgfpathclose%
\pgfusepath{fill}%
\end{pgfscope}%
\begin{pgfscope}%
\pgfpathrectangle{\pgfqpoint{0.765000in}{0.660000in}}{\pgfqpoint{4.620000in}{4.620000in}}%
\pgfusepath{clip}%
\pgfsetbuttcap%
\pgfsetroundjoin%
\definecolor{currentfill}{rgb}{0.176471,0.192157,0.258824}%
\pgfsetfillcolor{currentfill}%
\pgfsetfillopacity{0.050000}%
\pgfsetlinewidth{0.000000pt}%
\definecolor{currentstroke}{rgb}{0.176471,0.192157,0.258824}%
\pgfsetstrokecolor{currentstroke}%
\pgfsetdash{}{0pt}%
\pgfpathmoveto{\pgfqpoint{3.152537in}{2.887662in}}%
\pgfpathlineto{\pgfqpoint{3.230490in}{2.851066in}}%
\pgfpathlineto{\pgfqpoint{3.152537in}{2.814469in}}%
\pgfpathlineto{\pgfqpoint{3.074584in}{2.851066in}}%
\pgfpathlineto{\pgfqpoint{3.152537in}{2.887662in}}%
\pgfpathclose%
\pgfusepath{fill}%
\end{pgfscope}%
\begin{pgfscope}%
\pgfpathrectangle{\pgfqpoint{0.765000in}{0.660000in}}{\pgfqpoint{4.620000in}{4.620000in}}%
\pgfusepath{clip}%
\pgfsetbuttcap%
\pgfsetroundjoin%
\definecolor{currentfill}{rgb}{0.176471,0.192157,0.258824}%
\pgfsetfillcolor{currentfill}%
\pgfsetfillopacity{0.050000}%
\pgfsetlinewidth{0.000000pt}%
\definecolor{currentstroke}{rgb}{0.176471,0.192157,0.258824}%
\pgfsetstrokecolor{currentstroke}%
\pgfsetdash{}{0pt}%
\pgfpathmoveto{\pgfqpoint{3.152537in}{2.960666in}}%
\pgfpathlineto{\pgfqpoint{3.230490in}{2.924069in}}%
\pgfpathlineto{\pgfqpoint{3.152537in}{2.887472in}}%
\pgfpathlineto{\pgfqpoint{3.074584in}{2.924069in}}%
\pgfpathlineto{\pgfqpoint{3.152537in}{2.960666in}}%
\pgfpathclose%
\pgfusepath{fill}%
\end{pgfscope}%
\begin{pgfscope}%
\pgfpathrectangle{\pgfqpoint{0.765000in}{0.660000in}}{\pgfqpoint{4.620000in}{4.620000in}}%
\pgfusepath{clip}%
\pgfsetbuttcap%
\pgfsetroundjoin%
\definecolor{currentfill}{rgb}{0.176471,0.192157,0.258824}%
\pgfsetfillcolor{currentfill}%
\pgfsetfillopacity{0.050000}%
\pgfsetlinewidth{0.000000pt}%
\definecolor{currentstroke}{rgb}{0.176471,0.192157,0.258824}%
\pgfsetstrokecolor{currentstroke}%
\pgfsetdash{}{0pt}%
\pgfpathmoveto{\pgfqpoint{3.074584in}{2.851066in}}%
\pgfpathlineto{\pgfqpoint{3.074584in}{2.924069in}}%
\pgfpathlineto{\pgfqpoint{3.152537in}{2.887472in}}%
\pgfpathlineto{\pgfqpoint{3.152537in}{2.814469in}}%
\pgfpathlineto{\pgfqpoint{3.074584in}{2.851066in}}%
\pgfpathclose%
\pgfusepath{fill}%
\end{pgfscope}%
\begin{pgfscope}%
\pgfpathrectangle{\pgfqpoint{0.765000in}{0.660000in}}{\pgfqpoint{4.620000in}{4.620000in}}%
\pgfusepath{clip}%
\pgfsetbuttcap%
\pgfsetroundjoin%
\definecolor{currentfill}{rgb}{0.176471,0.192157,0.258824}%
\pgfsetfillcolor{currentfill}%
\pgfsetfillopacity{0.050000}%
\pgfsetlinewidth{0.000000pt}%
\definecolor{currentstroke}{rgb}{0.176471,0.192157,0.258824}%
\pgfsetstrokecolor{currentstroke}%
\pgfsetdash{}{0pt}%
\pgfpathmoveto{\pgfqpoint{3.230490in}{2.851066in}}%
\pgfpathlineto{\pgfqpoint{3.230490in}{2.924069in}}%
\pgfpathlineto{\pgfqpoint{3.152537in}{2.887472in}}%
\pgfpathlineto{\pgfqpoint{3.152537in}{2.814469in}}%
\pgfpathlineto{\pgfqpoint{3.230490in}{2.851066in}}%
\pgfpathclose%
\pgfusepath{fill}%
\end{pgfscope}%
\begin{pgfscope}%
\pgfpathrectangle{\pgfqpoint{0.765000in}{0.660000in}}{\pgfqpoint{4.620000in}{4.620000in}}%
\pgfusepath{clip}%
\pgfsetbuttcap%
\pgfsetroundjoin%
\definecolor{currentfill}{rgb}{1.000000,0.576471,0.309804}%
\pgfsetfillcolor{currentfill}%
\pgfsetfillopacity{0.050000}%
\pgfsetlinewidth{0.000000pt}%
\definecolor{currentstroke}{rgb}{1.000000,0.576471,0.309804}%
\pgfsetstrokecolor{currentstroke}%
\pgfsetdash{}{0pt}%
\pgfpathmoveto{\pgfqpoint{3.473559in}{2.566928in}}%
\pgfpathlineto{\pgfqpoint{3.473559in}{2.639931in}}%
\pgfpathlineto{\pgfqpoint{3.551512in}{2.603335in}}%
\pgfpathlineto{\pgfqpoint{3.551512in}{2.530331in}}%
\pgfpathlineto{\pgfqpoint{3.473559in}{2.566928in}}%
\pgfpathclose%
\pgfusepath{fill}%
\end{pgfscope}%
\begin{pgfscope}%
\pgfpathrectangle{\pgfqpoint{0.765000in}{0.660000in}}{\pgfqpoint{4.620000in}{4.620000in}}%
\pgfusepath{clip}%
\pgfsetbuttcap%
\pgfsetroundjoin%
\definecolor{currentfill}{rgb}{1.000000,0.576471,0.309804}%
\pgfsetfillcolor{currentfill}%
\pgfsetfillopacity{0.050000}%
\pgfsetlinewidth{0.000000pt}%
\definecolor{currentstroke}{rgb}{1.000000,0.576471,0.309804}%
\pgfsetstrokecolor{currentstroke}%
\pgfsetdash{}{0pt}%
\pgfpathmoveto{\pgfqpoint{3.473559in}{2.566928in}}%
\pgfpathlineto{\pgfqpoint{3.473559in}{2.639931in}}%
\pgfpathlineto{\pgfqpoint{3.395606in}{2.603335in}}%
\pgfpathlineto{\pgfqpoint{3.395606in}{2.530331in}}%
\pgfpathlineto{\pgfqpoint{3.473559in}{2.566928in}}%
\pgfpathclose%
\pgfusepath{fill}%
\end{pgfscope}%
\begin{pgfscope}%
\pgfpathrectangle{\pgfqpoint{0.765000in}{0.660000in}}{\pgfqpoint{4.620000in}{4.620000in}}%
\pgfusepath{clip}%
\pgfsetbuttcap%
\pgfsetroundjoin%
\definecolor{currentfill}{rgb}{1.000000,0.576471,0.309804}%
\pgfsetfillcolor{currentfill}%
\pgfsetfillopacity{0.050000}%
\pgfsetlinewidth{0.000000pt}%
\definecolor{currentstroke}{rgb}{1.000000,0.576471,0.309804}%
\pgfsetstrokecolor{currentstroke}%
\pgfsetdash{}{0pt}%
\pgfpathmoveto{\pgfqpoint{3.473559in}{2.566928in}}%
\pgfpathlineto{\pgfqpoint{3.551512in}{2.530331in}}%
\pgfpathlineto{\pgfqpoint{3.473559in}{2.493735in}}%
\pgfpathlineto{\pgfqpoint{3.395606in}{2.530331in}}%
\pgfpathlineto{\pgfqpoint{3.473559in}{2.566928in}}%
\pgfpathclose%
\pgfusepath{fill}%
\end{pgfscope}%
\begin{pgfscope}%
\pgfpathrectangle{\pgfqpoint{0.765000in}{0.660000in}}{\pgfqpoint{4.620000in}{4.620000in}}%
\pgfusepath{clip}%
\pgfsetbuttcap%
\pgfsetroundjoin%
\definecolor{currentfill}{rgb}{1.000000,0.576471,0.309804}%
\pgfsetfillcolor{currentfill}%
\pgfsetfillopacity{0.050000}%
\pgfsetlinewidth{0.000000pt}%
\definecolor{currentstroke}{rgb}{1.000000,0.576471,0.309804}%
\pgfsetstrokecolor{currentstroke}%
\pgfsetdash{}{0pt}%
\pgfpathmoveto{\pgfqpoint{3.473559in}{2.639931in}}%
\pgfpathlineto{\pgfqpoint{3.551512in}{2.603335in}}%
\pgfpathlineto{\pgfqpoint{3.473559in}{2.566738in}}%
\pgfpathlineto{\pgfqpoint{3.395606in}{2.603335in}}%
\pgfpathlineto{\pgfqpoint{3.473559in}{2.639931in}}%
\pgfpathclose%
\pgfusepath{fill}%
\end{pgfscope}%
\begin{pgfscope}%
\pgfpathrectangle{\pgfqpoint{0.765000in}{0.660000in}}{\pgfqpoint{4.620000in}{4.620000in}}%
\pgfusepath{clip}%
\pgfsetbuttcap%
\pgfsetroundjoin%
\definecolor{currentfill}{rgb}{1.000000,0.576471,0.309804}%
\pgfsetfillcolor{currentfill}%
\pgfsetfillopacity{0.050000}%
\pgfsetlinewidth{0.000000pt}%
\definecolor{currentstroke}{rgb}{1.000000,0.576471,0.309804}%
\pgfsetstrokecolor{currentstroke}%
\pgfsetdash{}{0pt}%
\pgfpathmoveto{\pgfqpoint{3.395606in}{2.530331in}}%
\pgfpathlineto{\pgfqpoint{3.395606in}{2.603335in}}%
\pgfpathlineto{\pgfqpoint{3.473559in}{2.566738in}}%
\pgfpathlineto{\pgfqpoint{3.473559in}{2.493735in}}%
\pgfpathlineto{\pgfqpoint{3.395606in}{2.530331in}}%
\pgfpathclose%
\pgfusepath{fill}%
\end{pgfscope}%
\begin{pgfscope}%
\pgfpathrectangle{\pgfqpoint{0.765000in}{0.660000in}}{\pgfqpoint{4.620000in}{4.620000in}}%
\pgfusepath{clip}%
\pgfsetbuttcap%
\pgfsetroundjoin%
\definecolor{currentfill}{rgb}{1.000000,0.576471,0.309804}%
\pgfsetfillcolor{currentfill}%
\pgfsetfillopacity{0.050000}%
\pgfsetlinewidth{0.000000pt}%
\definecolor{currentstroke}{rgb}{1.000000,0.576471,0.309804}%
\pgfsetstrokecolor{currentstroke}%
\pgfsetdash{}{0pt}%
\pgfpathmoveto{\pgfqpoint{3.551512in}{2.530331in}}%
\pgfpathlineto{\pgfqpoint{3.551512in}{2.603335in}}%
\pgfpathlineto{\pgfqpoint{3.473559in}{2.566738in}}%
\pgfpathlineto{\pgfqpoint{3.473559in}{2.493735in}}%
\pgfpathlineto{\pgfqpoint{3.551512in}{2.530331in}}%
\pgfpathclose%
\pgfusepath{fill}%
\end{pgfscope}%
\begin{pgfscope}%
\pgfpathrectangle{\pgfqpoint{0.765000in}{0.660000in}}{\pgfqpoint{4.620000in}{4.620000in}}%
\pgfusepath{clip}%
\pgfsetbuttcap%
\pgfsetroundjoin%
\definecolor{currentfill}{rgb}{0.176471,0.192157,0.258824}%
\pgfsetfillcolor{currentfill}%
\pgfsetfillopacity{0.050000}%
\pgfsetlinewidth{0.000000pt}%
\definecolor{currentstroke}{rgb}{0.176471,0.192157,0.258824}%
\pgfsetstrokecolor{currentstroke}%
\pgfsetdash{}{0pt}%
\pgfpathmoveto{\pgfqpoint{3.255269in}{2.810632in}}%
\pgfpathlineto{\pgfqpoint{3.255269in}{2.883635in}}%
\pgfpathlineto{\pgfqpoint{3.333221in}{2.847039in}}%
\pgfpathlineto{\pgfqpoint{3.333221in}{2.774035in}}%
\pgfpathlineto{\pgfqpoint{3.255269in}{2.810632in}}%
\pgfpathclose%
\pgfusepath{fill}%
\end{pgfscope}%
\begin{pgfscope}%
\pgfpathrectangle{\pgfqpoint{0.765000in}{0.660000in}}{\pgfqpoint{4.620000in}{4.620000in}}%
\pgfusepath{clip}%
\pgfsetbuttcap%
\pgfsetroundjoin%
\definecolor{currentfill}{rgb}{0.176471,0.192157,0.258824}%
\pgfsetfillcolor{currentfill}%
\pgfsetfillopacity{0.050000}%
\pgfsetlinewidth{0.000000pt}%
\definecolor{currentstroke}{rgb}{0.176471,0.192157,0.258824}%
\pgfsetstrokecolor{currentstroke}%
\pgfsetdash{}{0pt}%
\pgfpathmoveto{\pgfqpoint{3.255269in}{2.810632in}}%
\pgfpathlineto{\pgfqpoint{3.255269in}{2.883635in}}%
\pgfpathlineto{\pgfqpoint{3.177316in}{2.847039in}}%
\pgfpathlineto{\pgfqpoint{3.177316in}{2.774035in}}%
\pgfpathlineto{\pgfqpoint{3.255269in}{2.810632in}}%
\pgfpathclose%
\pgfusepath{fill}%
\end{pgfscope}%
\begin{pgfscope}%
\pgfpathrectangle{\pgfqpoint{0.765000in}{0.660000in}}{\pgfqpoint{4.620000in}{4.620000in}}%
\pgfusepath{clip}%
\pgfsetbuttcap%
\pgfsetroundjoin%
\definecolor{currentfill}{rgb}{0.176471,0.192157,0.258824}%
\pgfsetfillcolor{currentfill}%
\pgfsetfillopacity{0.050000}%
\pgfsetlinewidth{0.000000pt}%
\definecolor{currentstroke}{rgb}{0.176471,0.192157,0.258824}%
\pgfsetstrokecolor{currentstroke}%
\pgfsetdash{}{0pt}%
\pgfpathmoveto{\pgfqpoint{3.255269in}{2.810632in}}%
\pgfpathlineto{\pgfqpoint{3.333221in}{2.774035in}}%
\pgfpathlineto{\pgfqpoint{3.255269in}{2.737439in}}%
\pgfpathlineto{\pgfqpoint{3.177316in}{2.774035in}}%
\pgfpathlineto{\pgfqpoint{3.255269in}{2.810632in}}%
\pgfpathclose%
\pgfusepath{fill}%
\end{pgfscope}%
\begin{pgfscope}%
\pgfpathrectangle{\pgfqpoint{0.765000in}{0.660000in}}{\pgfqpoint{4.620000in}{4.620000in}}%
\pgfusepath{clip}%
\pgfsetbuttcap%
\pgfsetroundjoin%
\definecolor{currentfill}{rgb}{0.176471,0.192157,0.258824}%
\pgfsetfillcolor{currentfill}%
\pgfsetfillopacity{0.050000}%
\pgfsetlinewidth{0.000000pt}%
\definecolor{currentstroke}{rgb}{0.176471,0.192157,0.258824}%
\pgfsetstrokecolor{currentstroke}%
\pgfsetdash{}{0pt}%
\pgfpathmoveto{\pgfqpoint{3.255269in}{2.883635in}}%
\pgfpathlineto{\pgfqpoint{3.333221in}{2.847039in}}%
\pgfpathlineto{\pgfqpoint{3.255269in}{2.810442in}}%
\pgfpathlineto{\pgfqpoint{3.177316in}{2.847039in}}%
\pgfpathlineto{\pgfqpoint{3.255269in}{2.883635in}}%
\pgfpathclose%
\pgfusepath{fill}%
\end{pgfscope}%
\begin{pgfscope}%
\pgfpathrectangle{\pgfqpoint{0.765000in}{0.660000in}}{\pgfqpoint{4.620000in}{4.620000in}}%
\pgfusepath{clip}%
\pgfsetbuttcap%
\pgfsetroundjoin%
\definecolor{currentfill}{rgb}{0.176471,0.192157,0.258824}%
\pgfsetfillcolor{currentfill}%
\pgfsetfillopacity{0.050000}%
\pgfsetlinewidth{0.000000pt}%
\definecolor{currentstroke}{rgb}{0.176471,0.192157,0.258824}%
\pgfsetstrokecolor{currentstroke}%
\pgfsetdash{}{0pt}%
\pgfpathmoveto{\pgfqpoint{3.177316in}{2.774035in}}%
\pgfpathlineto{\pgfqpoint{3.177316in}{2.847039in}}%
\pgfpathlineto{\pgfqpoint{3.255269in}{2.810442in}}%
\pgfpathlineto{\pgfqpoint{3.255269in}{2.737439in}}%
\pgfpathlineto{\pgfqpoint{3.177316in}{2.774035in}}%
\pgfpathclose%
\pgfusepath{fill}%
\end{pgfscope}%
\begin{pgfscope}%
\pgfpathrectangle{\pgfqpoint{0.765000in}{0.660000in}}{\pgfqpoint{4.620000in}{4.620000in}}%
\pgfusepath{clip}%
\pgfsetbuttcap%
\pgfsetroundjoin%
\definecolor{currentfill}{rgb}{0.176471,0.192157,0.258824}%
\pgfsetfillcolor{currentfill}%
\pgfsetfillopacity{0.050000}%
\pgfsetlinewidth{0.000000pt}%
\definecolor{currentstroke}{rgb}{0.176471,0.192157,0.258824}%
\pgfsetstrokecolor{currentstroke}%
\pgfsetdash{}{0pt}%
\pgfpathmoveto{\pgfqpoint{3.333221in}{2.774035in}}%
\pgfpathlineto{\pgfqpoint{3.333221in}{2.847039in}}%
\pgfpathlineto{\pgfqpoint{3.255269in}{2.810442in}}%
\pgfpathlineto{\pgfqpoint{3.255269in}{2.737439in}}%
\pgfpathlineto{\pgfqpoint{3.333221in}{2.774035in}}%
\pgfpathclose%
\pgfusepath{fill}%
\end{pgfscope}%
\begin{pgfscope}%
\pgfpathrectangle{\pgfqpoint{0.765000in}{0.660000in}}{\pgfqpoint{4.620000in}{4.620000in}}%
\pgfusepath{clip}%
\pgfsetbuttcap%
\pgfsetroundjoin%
\definecolor{currentfill}{rgb}{0.176471,0.192157,0.258824}%
\pgfsetfillcolor{currentfill}%
\pgfsetfillopacity{0.050000}%
\pgfsetlinewidth{0.000000pt}%
\definecolor{currentstroke}{rgb}{0.176471,0.192157,0.258824}%
\pgfsetstrokecolor{currentstroke}%
\pgfsetdash{}{0pt}%
\pgfpathmoveto{\pgfqpoint{3.297732in}{2.926816in}}%
\pgfpathlineto{\pgfqpoint{3.297732in}{2.999819in}}%
\pgfpathlineto{\pgfqpoint{3.375685in}{2.963223in}}%
\pgfpathlineto{\pgfqpoint{3.375685in}{2.890219in}}%
\pgfpathlineto{\pgfqpoint{3.297732in}{2.926816in}}%
\pgfpathclose%
\pgfusepath{fill}%
\end{pgfscope}%
\begin{pgfscope}%
\pgfpathrectangle{\pgfqpoint{0.765000in}{0.660000in}}{\pgfqpoint{4.620000in}{4.620000in}}%
\pgfusepath{clip}%
\pgfsetbuttcap%
\pgfsetroundjoin%
\definecolor{currentfill}{rgb}{0.176471,0.192157,0.258824}%
\pgfsetfillcolor{currentfill}%
\pgfsetfillopacity{0.050000}%
\pgfsetlinewidth{0.000000pt}%
\definecolor{currentstroke}{rgb}{0.176471,0.192157,0.258824}%
\pgfsetstrokecolor{currentstroke}%
\pgfsetdash{}{0pt}%
\pgfpathmoveto{\pgfqpoint{3.297732in}{2.926816in}}%
\pgfpathlineto{\pgfqpoint{3.297732in}{2.999819in}}%
\pgfpathlineto{\pgfqpoint{3.219780in}{2.963223in}}%
\pgfpathlineto{\pgfqpoint{3.219780in}{2.890219in}}%
\pgfpathlineto{\pgfqpoint{3.297732in}{2.926816in}}%
\pgfpathclose%
\pgfusepath{fill}%
\end{pgfscope}%
\begin{pgfscope}%
\pgfpathrectangle{\pgfqpoint{0.765000in}{0.660000in}}{\pgfqpoint{4.620000in}{4.620000in}}%
\pgfusepath{clip}%
\pgfsetbuttcap%
\pgfsetroundjoin%
\definecolor{currentfill}{rgb}{0.176471,0.192157,0.258824}%
\pgfsetfillcolor{currentfill}%
\pgfsetfillopacity{0.050000}%
\pgfsetlinewidth{0.000000pt}%
\definecolor{currentstroke}{rgb}{0.176471,0.192157,0.258824}%
\pgfsetstrokecolor{currentstroke}%
\pgfsetdash{}{0pt}%
\pgfpathmoveto{\pgfqpoint{3.297732in}{2.926816in}}%
\pgfpathlineto{\pgfqpoint{3.375685in}{2.890219in}}%
\pgfpathlineto{\pgfqpoint{3.297732in}{2.853623in}}%
\pgfpathlineto{\pgfqpoint{3.219780in}{2.890219in}}%
\pgfpathlineto{\pgfqpoint{3.297732in}{2.926816in}}%
\pgfpathclose%
\pgfusepath{fill}%
\end{pgfscope}%
\begin{pgfscope}%
\pgfpathrectangle{\pgfqpoint{0.765000in}{0.660000in}}{\pgfqpoint{4.620000in}{4.620000in}}%
\pgfusepath{clip}%
\pgfsetbuttcap%
\pgfsetroundjoin%
\definecolor{currentfill}{rgb}{0.176471,0.192157,0.258824}%
\pgfsetfillcolor{currentfill}%
\pgfsetfillopacity{0.050000}%
\pgfsetlinewidth{0.000000pt}%
\definecolor{currentstroke}{rgb}{0.176471,0.192157,0.258824}%
\pgfsetstrokecolor{currentstroke}%
\pgfsetdash{}{0pt}%
\pgfpathmoveto{\pgfqpoint{3.297732in}{2.999819in}}%
\pgfpathlineto{\pgfqpoint{3.375685in}{2.963223in}}%
\pgfpathlineto{\pgfqpoint{3.297732in}{2.926626in}}%
\pgfpathlineto{\pgfqpoint{3.219780in}{2.963223in}}%
\pgfpathlineto{\pgfqpoint{3.297732in}{2.999819in}}%
\pgfpathclose%
\pgfusepath{fill}%
\end{pgfscope}%
\begin{pgfscope}%
\pgfpathrectangle{\pgfqpoint{0.765000in}{0.660000in}}{\pgfqpoint{4.620000in}{4.620000in}}%
\pgfusepath{clip}%
\pgfsetbuttcap%
\pgfsetroundjoin%
\definecolor{currentfill}{rgb}{0.176471,0.192157,0.258824}%
\pgfsetfillcolor{currentfill}%
\pgfsetfillopacity{0.050000}%
\pgfsetlinewidth{0.000000pt}%
\definecolor{currentstroke}{rgb}{0.176471,0.192157,0.258824}%
\pgfsetstrokecolor{currentstroke}%
\pgfsetdash{}{0pt}%
\pgfpathmoveto{\pgfqpoint{3.219780in}{2.890219in}}%
\pgfpathlineto{\pgfqpoint{3.219780in}{2.963223in}}%
\pgfpathlineto{\pgfqpoint{3.297732in}{2.926626in}}%
\pgfpathlineto{\pgfqpoint{3.297732in}{2.853623in}}%
\pgfpathlineto{\pgfqpoint{3.219780in}{2.890219in}}%
\pgfpathclose%
\pgfusepath{fill}%
\end{pgfscope}%
\begin{pgfscope}%
\pgfpathrectangle{\pgfqpoint{0.765000in}{0.660000in}}{\pgfqpoint{4.620000in}{4.620000in}}%
\pgfusepath{clip}%
\pgfsetbuttcap%
\pgfsetroundjoin%
\definecolor{currentfill}{rgb}{0.176471,0.192157,0.258824}%
\pgfsetfillcolor{currentfill}%
\pgfsetfillopacity{0.050000}%
\pgfsetlinewidth{0.000000pt}%
\definecolor{currentstroke}{rgb}{0.176471,0.192157,0.258824}%
\pgfsetstrokecolor{currentstroke}%
\pgfsetdash{}{0pt}%
\pgfpathmoveto{\pgfqpoint{3.375685in}{2.890219in}}%
\pgfpathlineto{\pgfqpoint{3.375685in}{2.963223in}}%
\pgfpathlineto{\pgfqpoint{3.297732in}{2.926626in}}%
\pgfpathlineto{\pgfqpoint{3.297732in}{2.853623in}}%
\pgfpathlineto{\pgfqpoint{3.375685in}{2.890219in}}%
\pgfpathclose%
\pgfusepath{fill}%
\end{pgfscope}%
\begin{pgfscope}%
\pgfpathrectangle{\pgfqpoint{0.765000in}{0.660000in}}{\pgfqpoint{4.620000in}{4.620000in}}%
\pgfusepath{clip}%
\pgfsetbuttcap%
\pgfsetroundjoin%
\definecolor{currentfill}{rgb}{1.000000,0.576471,0.309804}%
\pgfsetfillcolor{currentfill}%
\pgfsetfillopacity{0.050000}%
\pgfsetlinewidth{0.000000pt}%
\definecolor{currentstroke}{rgb}{1.000000,0.576471,0.309804}%
\pgfsetstrokecolor{currentstroke}%
\pgfsetdash{}{0pt}%
\pgfpathmoveto{\pgfqpoint{2.936955in}{3.580470in}}%
\pgfpathlineto{\pgfqpoint{2.936955in}{3.653473in}}%
\pgfpathlineto{\pgfqpoint{3.014907in}{3.616877in}}%
\pgfpathlineto{\pgfqpoint{3.014907in}{3.543873in}}%
\pgfpathlineto{\pgfqpoint{2.936955in}{3.580470in}}%
\pgfpathclose%
\pgfusepath{fill}%
\end{pgfscope}%
\begin{pgfscope}%
\pgfpathrectangle{\pgfqpoint{0.765000in}{0.660000in}}{\pgfqpoint{4.620000in}{4.620000in}}%
\pgfusepath{clip}%
\pgfsetbuttcap%
\pgfsetroundjoin%
\definecolor{currentfill}{rgb}{1.000000,0.576471,0.309804}%
\pgfsetfillcolor{currentfill}%
\pgfsetfillopacity{0.050000}%
\pgfsetlinewidth{0.000000pt}%
\definecolor{currentstroke}{rgb}{1.000000,0.576471,0.309804}%
\pgfsetstrokecolor{currentstroke}%
\pgfsetdash{}{0pt}%
\pgfpathmoveto{\pgfqpoint{2.936955in}{3.580470in}}%
\pgfpathlineto{\pgfqpoint{2.936955in}{3.653473in}}%
\pgfpathlineto{\pgfqpoint{2.859002in}{3.616877in}}%
\pgfpathlineto{\pgfqpoint{2.859002in}{3.543873in}}%
\pgfpathlineto{\pgfqpoint{2.936955in}{3.580470in}}%
\pgfpathclose%
\pgfusepath{fill}%
\end{pgfscope}%
\begin{pgfscope}%
\pgfpathrectangle{\pgfqpoint{0.765000in}{0.660000in}}{\pgfqpoint{4.620000in}{4.620000in}}%
\pgfusepath{clip}%
\pgfsetbuttcap%
\pgfsetroundjoin%
\definecolor{currentfill}{rgb}{1.000000,0.576471,0.309804}%
\pgfsetfillcolor{currentfill}%
\pgfsetfillopacity{0.050000}%
\pgfsetlinewidth{0.000000pt}%
\definecolor{currentstroke}{rgb}{1.000000,0.576471,0.309804}%
\pgfsetstrokecolor{currentstroke}%
\pgfsetdash{}{0pt}%
\pgfpathmoveto{\pgfqpoint{2.936955in}{3.580470in}}%
\pgfpathlineto{\pgfqpoint{3.014907in}{3.543873in}}%
\pgfpathlineto{\pgfqpoint{2.936955in}{3.507277in}}%
\pgfpathlineto{\pgfqpoint{2.859002in}{3.543873in}}%
\pgfpathlineto{\pgfqpoint{2.936955in}{3.580470in}}%
\pgfpathclose%
\pgfusepath{fill}%
\end{pgfscope}%
\begin{pgfscope}%
\pgfpathrectangle{\pgfqpoint{0.765000in}{0.660000in}}{\pgfqpoint{4.620000in}{4.620000in}}%
\pgfusepath{clip}%
\pgfsetbuttcap%
\pgfsetroundjoin%
\definecolor{currentfill}{rgb}{1.000000,0.576471,0.309804}%
\pgfsetfillcolor{currentfill}%
\pgfsetfillopacity{0.050000}%
\pgfsetlinewidth{0.000000pt}%
\definecolor{currentstroke}{rgb}{1.000000,0.576471,0.309804}%
\pgfsetstrokecolor{currentstroke}%
\pgfsetdash{}{0pt}%
\pgfpathmoveto{\pgfqpoint{2.936955in}{3.653473in}}%
\pgfpathlineto{\pgfqpoint{3.014907in}{3.616877in}}%
\pgfpathlineto{\pgfqpoint{2.936955in}{3.580280in}}%
\pgfpathlineto{\pgfqpoint{2.859002in}{3.616877in}}%
\pgfpathlineto{\pgfqpoint{2.936955in}{3.653473in}}%
\pgfpathclose%
\pgfusepath{fill}%
\end{pgfscope}%
\begin{pgfscope}%
\pgfpathrectangle{\pgfqpoint{0.765000in}{0.660000in}}{\pgfqpoint{4.620000in}{4.620000in}}%
\pgfusepath{clip}%
\pgfsetbuttcap%
\pgfsetroundjoin%
\definecolor{currentfill}{rgb}{1.000000,0.576471,0.309804}%
\pgfsetfillcolor{currentfill}%
\pgfsetfillopacity{0.050000}%
\pgfsetlinewidth{0.000000pt}%
\definecolor{currentstroke}{rgb}{1.000000,0.576471,0.309804}%
\pgfsetstrokecolor{currentstroke}%
\pgfsetdash{}{0pt}%
\pgfpathmoveto{\pgfqpoint{2.859002in}{3.543873in}}%
\pgfpathlineto{\pgfqpoint{2.859002in}{3.616877in}}%
\pgfpathlineto{\pgfqpoint{2.936955in}{3.580280in}}%
\pgfpathlineto{\pgfqpoint{2.936955in}{3.507277in}}%
\pgfpathlineto{\pgfqpoint{2.859002in}{3.543873in}}%
\pgfpathclose%
\pgfusepath{fill}%
\end{pgfscope}%
\begin{pgfscope}%
\pgfpathrectangle{\pgfqpoint{0.765000in}{0.660000in}}{\pgfqpoint{4.620000in}{4.620000in}}%
\pgfusepath{clip}%
\pgfsetbuttcap%
\pgfsetroundjoin%
\definecolor{currentfill}{rgb}{1.000000,0.576471,0.309804}%
\pgfsetfillcolor{currentfill}%
\pgfsetfillopacity{0.050000}%
\pgfsetlinewidth{0.000000pt}%
\definecolor{currentstroke}{rgb}{1.000000,0.576471,0.309804}%
\pgfsetstrokecolor{currentstroke}%
\pgfsetdash{}{0pt}%
\pgfpathmoveto{\pgfqpoint{3.014907in}{3.543873in}}%
\pgfpathlineto{\pgfqpoint{3.014907in}{3.616877in}}%
\pgfpathlineto{\pgfqpoint{2.936955in}{3.580280in}}%
\pgfpathlineto{\pgfqpoint{2.936955in}{3.507277in}}%
\pgfpathlineto{\pgfqpoint{3.014907in}{3.543873in}}%
\pgfpathclose%
\pgfusepath{fill}%
\end{pgfscope}%
\begin{pgfscope}%
\pgfpathrectangle{\pgfqpoint{0.765000in}{0.660000in}}{\pgfqpoint{4.620000in}{4.620000in}}%
\pgfusepath{clip}%
\pgfsetbuttcap%
\pgfsetroundjoin%
\definecolor{currentfill}{rgb}{0.176471,0.192157,0.258824}%
\pgfsetfillcolor{currentfill}%
\pgfsetfillopacity{0.050000}%
\pgfsetlinewidth{0.000000pt}%
\definecolor{currentstroke}{rgb}{0.176471,0.192157,0.258824}%
\pgfsetstrokecolor{currentstroke}%
\pgfsetdash{}{0pt}%
\pgfpathmoveto{\pgfqpoint{3.013424in}{2.970835in}}%
\pgfpathlineto{\pgfqpoint{3.013424in}{3.043838in}}%
\pgfpathlineto{\pgfqpoint{3.091377in}{3.007241in}}%
\pgfpathlineto{\pgfqpoint{3.091377in}{2.934238in}}%
\pgfpathlineto{\pgfqpoint{3.013424in}{2.970835in}}%
\pgfpathclose%
\pgfusepath{fill}%
\end{pgfscope}%
\begin{pgfscope}%
\pgfpathrectangle{\pgfqpoint{0.765000in}{0.660000in}}{\pgfqpoint{4.620000in}{4.620000in}}%
\pgfusepath{clip}%
\pgfsetbuttcap%
\pgfsetroundjoin%
\definecolor{currentfill}{rgb}{0.176471,0.192157,0.258824}%
\pgfsetfillcolor{currentfill}%
\pgfsetfillopacity{0.050000}%
\pgfsetlinewidth{0.000000pt}%
\definecolor{currentstroke}{rgb}{0.176471,0.192157,0.258824}%
\pgfsetstrokecolor{currentstroke}%
\pgfsetdash{}{0pt}%
\pgfpathmoveto{\pgfqpoint{3.013424in}{2.970835in}}%
\pgfpathlineto{\pgfqpoint{3.013424in}{3.043838in}}%
\pgfpathlineto{\pgfqpoint{2.935471in}{3.007241in}}%
\pgfpathlineto{\pgfqpoint{2.935471in}{2.934238in}}%
\pgfpathlineto{\pgfqpoint{3.013424in}{2.970835in}}%
\pgfpathclose%
\pgfusepath{fill}%
\end{pgfscope}%
\begin{pgfscope}%
\pgfpathrectangle{\pgfqpoint{0.765000in}{0.660000in}}{\pgfqpoint{4.620000in}{4.620000in}}%
\pgfusepath{clip}%
\pgfsetbuttcap%
\pgfsetroundjoin%
\definecolor{currentfill}{rgb}{0.176471,0.192157,0.258824}%
\pgfsetfillcolor{currentfill}%
\pgfsetfillopacity{0.050000}%
\pgfsetlinewidth{0.000000pt}%
\definecolor{currentstroke}{rgb}{0.176471,0.192157,0.258824}%
\pgfsetstrokecolor{currentstroke}%
\pgfsetdash{}{0pt}%
\pgfpathmoveto{\pgfqpoint{3.013424in}{2.970835in}}%
\pgfpathlineto{\pgfqpoint{3.091377in}{2.934238in}}%
\pgfpathlineto{\pgfqpoint{3.013424in}{2.897642in}}%
\pgfpathlineto{\pgfqpoint{2.935471in}{2.934238in}}%
\pgfpathlineto{\pgfqpoint{3.013424in}{2.970835in}}%
\pgfpathclose%
\pgfusepath{fill}%
\end{pgfscope}%
\begin{pgfscope}%
\pgfpathrectangle{\pgfqpoint{0.765000in}{0.660000in}}{\pgfqpoint{4.620000in}{4.620000in}}%
\pgfusepath{clip}%
\pgfsetbuttcap%
\pgfsetroundjoin%
\definecolor{currentfill}{rgb}{0.176471,0.192157,0.258824}%
\pgfsetfillcolor{currentfill}%
\pgfsetfillopacity{0.050000}%
\pgfsetlinewidth{0.000000pt}%
\definecolor{currentstroke}{rgb}{0.176471,0.192157,0.258824}%
\pgfsetstrokecolor{currentstroke}%
\pgfsetdash{}{0pt}%
\pgfpathmoveto{\pgfqpoint{3.013424in}{3.043838in}}%
\pgfpathlineto{\pgfqpoint{3.091377in}{3.007241in}}%
\pgfpathlineto{\pgfqpoint{3.013424in}{2.970645in}}%
\pgfpathlineto{\pgfqpoint{2.935471in}{3.007241in}}%
\pgfpathlineto{\pgfqpoint{3.013424in}{3.043838in}}%
\pgfpathclose%
\pgfusepath{fill}%
\end{pgfscope}%
\begin{pgfscope}%
\pgfpathrectangle{\pgfqpoint{0.765000in}{0.660000in}}{\pgfqpoint{4.620000in}{4.620000in}}%
\pgfusepath{clip}%
\pgfsetbuttcap%
\pgfsetroundjoin%
\definecolor{currentfill}{rgb}{0.176471,0.192157,0.258824}%
\pgfsetfillcolor{currentfill}%
\pgfsetfillopacity{0.050000}%
\pgfsetlinewidth{0.000000pt}%
\definecolor{currentstroke}{rgb}{0.176471,0.192157,0.258824}%
\pgfsetstrokecolor{currentstroke}%
\pgfsetdash{}{0pt}%
\pgfpathmoveto{\pgfqpoint{2.935471in}{2.934238in}}%
\pgfpathlineto{\pgfqpoint{2.935471in}{3.007241in}}%
\pgfpathlineto{\pgfqpoint{3.013424in}{2.970645in}}%
\pgfpathlineto{\pgfqpoint{3.013424in}{2.897642in}}%
\pgfpathlineto{\pgfqpoint{2.935471in}{2.934238in}}%
\pgfpathclose%
\pgfusepath{fill}%
\end{pgfscope}%
\begin{pgfscope}%
\pgfpathrectangle{\pgfqpoint{0.765000in}{0.660000in}}{\pgfqpoint{4.620000in}{4.620000in}}%
\pgfusepath{clip}%
\pgfsetbuttcap%
\pgfsetroundjoin%
\definecolor{currentfill}{rgb}{0.176471,0.192157,0.258824}%
\pgfsetfillcolor{currentfill}%
\pgfsetfillopacity{0.050000}%
\pgfsetlinewidth{0.000000pt}%
\definecolor{currentstroke}{rgb}{0.176471,0.192157,0.258824}%
\pgfsetstrokecolor{currentstroke}%
\pgfsetdash{}{0pt}%
\pgfpathmoveto{\pgfqpoint{3.091377in}{2.934238in}}%
\pgfpathlineto{\pgfqpoint{3.091377in}{3.007241in}}%
\pgfpathlineto{\pgfqpoint{3.013424in}{2.970645in}}%
\pgfpathlineto{\pgfqpoint{3.013424in}{2.897642in}}%
\pgfpathlineto{\pgfqpoint{3.091377in}{2.934238in}}%
\pgfpathclose%
\pgfusepath{fill}%
\end{pgfscope}%
\begin{pgfscope}%
\pgfpathrectangle{\pgfqpoint{0.765000in}{0.660000in}}{\pgfqpoint{4.620000in}{4.620000in}}%
\pgfusepath{clip}%
\pgfsetbuttcap%
\pgfsetroundjoin%
\definecolor{currentfill}{rgb}{1.000000,0.576471,0.309804}%
\pgfsetfillcolor{currentfill}%
\pgfsetfillopacity{0.050000}%
\pgfsetlinewidth{0.000000pt}%
\definecolor{currentstroke}{rgb}{1.000000,0.576471,0.309804}%
\pgfsetstrokecolor{currentstroke}%
\pgfsetdash{}{0pt}%
\pgfpathmoveto{\pgfqpoint{3.543155in}{3.553161in}}%
\pgfpathlineto{\pgfqpoint{3.543155in}{3.626165in}}%
\pgfpathlineto{\pgfqpoint{3.621108in}{3.589568in}}%
\pgfpathlineto{\pgfqpoint{3.621108in}{3.516565in}}%
\pgfpathlineto{\pgfqpoint{3.543155in}{3.553161in}}%
\pgfpathclose%
\pgfusepath{fill}%
\end{pgfscope}%
\begin{pgfscope}%
\pgfpathrectangle{\pgfqpoint{0.765000in}{0.660000in}}{\pgfqpoint{4.620000in}{4.620000in}}%
\pgfusepath{clip}%
\pgfsetbuttcap%
\pgfsetroundjoin%
\definecolor{currentfill}{rgb}{1.000000,0.576471,0.309804}%
\pgfsetfillcolor{currentfill}%
\pgfsetfillopacity{0.050000}%
\pgfsetlinewidth{0.000000pt}%
\definecolor{currentstroke}{rgb}{1.000000,0.576471,0.309804}%
\pgfsetstrokecolor{currentstroke}%
\pgfsetdash{}{0pt}%
\pgfpathmoveto{\pgfqpoint{3.543155in}{3.553161in}}%
\pgfpathlineto{\pgfqpoint{3.543155in}{3.626165in}}%
\pgfpathlineto{\pgfqpoint{3.465202in}{3.589568in}}%
\pgfpathlineto{\pgfqpoint{3.465202in}{3.516565in}}%
\pgfpathlineto{\pgfqpoint{3.543155in}{3.553161in}}%
\pgfpathclose%
\pgfusepath{fill}%
\end{pgfscope}%
\begin{pgfscope}%
\pgfpathrectangle{\pgfqpoint{0.765000in}{0.660000in}}{\pgfqpoint{4.620000in}{4.620000in}}%
\pgfusepath{clip}%
\pgfsetbuttcap%
\pgfsetroundjoin%
\definecolor{currentfill}{rgb}{1.000000,0.576471,0.309804}%
\pgfsetfillcolor{currentfill}%
\pgfsetfillopacity{0.050000}%
\pgfsetlinewidth{0.000000pt}%
\definecolor{currentstroke}{rgb}{1.000000,0.576471,0.309804}%
\pgfsetstrokecolor{currentstroke}%
\pgfsetdash{}{0pt}%
\pgfpathmoveto{\pgfqpoint{3.543155in}{3.553161in}}%
\pgfpathlineto{\pgfqpoint{3.621108in}{3.516565in}}%
\pgfpathlineto{\pgfqpoint{3.543155in}{3.479968in}}%
\pgfpathlineto{\pgfqpoint{3.465202in}{3.516565in}}%
\pgfpathlineto{\pgfqpoint{3.543155in}{3.553161in}}%
\pgfpathclose%
\pgfusepath{fill}%
\end{pgfscope}%
\begin{pgfscope}%
\pgfpathrectangle{\pgfqpoint{0.765000in}{0.660000in}}{\pgfqpoint{4.620000in}{4.620000in}}%
\pgfusepath{clip}%
\pgfsetbuttcap%
\pgfsetroundjoin%
\definecolor{currentfill}{rgb}{1.000000,0.576471,0.309804}%
\pgfsetfillcolor{currentfill}%
\pgfsetfillopacity{0.050000}%
\pgfsetlinewidth{0.000000pt}%
\definecolor{currentstroke}{rgb}{1.000000,0.576471,0.309804}%
\pgfsetstrokecolor{currentstroke}%
\pgfsetdash{}{0pt}%
\pgfpathmoveto{\pgfqpoint{3.543155in}{3.626165in}}%
\pgfpathlineto{\pgfqpoint{3.621108in}{3.589568in}}%
\pgfpathlineto{\pgfqpoint{3.543155in}{3.552972in}}%
\pgfpathlineto{\pgfqpoint{3.465202in}{3.589568in}}%
\pgfpathlineto{\pgfqpoint{3.543155in}{3.626165in}}%
\pgfpathclose%
\pgfusepath{fill}%
\end{pgfscope}%
\begin{pgfscope}%
\pgfpathrectangle{\pgfqpoint{0.765000in}{0.660000in}}{\pgfqpoint{4.620000in}{4.620000in}}%
\pgfusepath{clip}%
\pgfsetbuttcap%
\pgfsetroundjoin%
\definecolor{currentfill}{rgb}{1.000000,0.576471,0.309804}%
\pgfsetfillcolor{currentfill}%
\pgfsetfillopacity{0.050000}%
\pgfsetlinewidth{0.000000pt}%
\definecolor{currentstroke}{rgb}{1.000000,0.576471,0.309804}%
\pgfsetstrokecolor{currentstroke}%
\pgfsetdash{}{0pt}%
\pgfpathmoveto{\pgfqpoint{3.465202in}{3.516565in}}%
\pgfpathlineto{\pgfqpoint{3.465202in}{3.589568in}}%
\pgfpathlineto{\pgfqpoint{3.543155in}{3.552972in}}%
\pgfpathlineto{\pgfqpoint{3.543155in}{3.479968in}}%
\pgfpathlineto{\pgfqpoint{3.465202in}{3.516565in}}%
\pgfpathclose%
\pgfusepath{fill}%
\end{pgfscope}%
\begin{pgfscope}%
\pgfpathrectangle{\pgfqpoint{0.765000in}{0.660000in}}{\pgfqpoint{4.620000in}{4.620000in}}%
\pgfusepath{clip}%
\pgfsetbuttcap%
\pgfsetroundjoin%
\definecolor{currentfill}{rgb}{1.000000,0.576471,0.309804}%
\pgfsetfillcolor{currentfill}%
\pgfsetfillopacity{0.050000}%
\pgfsetlinewidth{0.000000pt}%
\definecolor{currentstroke}{rgb}{1.000000,0.576471,0.309804}%
\pgfsetstrokecolor{currentstroke}%
\pgfsetdash{}{0pt}%
\pgfpathmoveto{\pgfqpoint{3.621108in}{3.516565in}}%
\pgfpathlineto{\pgfqpoint{3.621108in}{3.589568in}}%
\pgfpathlineto{\pgfqpoint{3.543155in}{3.552972in}}%
\pgfpathlineto{\pgfqpoint{3.543155in}{3.479968in}}%
\pgfpathlineto{\pgfqpoint{3.621108in}{3.516565in}}%
\pgfpathclose%
\pgfusepath{fill}%
\end{pgfscope}%
\begin{pgfscope}%
\pgfpathrectangle{\pgfqpoint{0.765000in}{0.660000in}}{\pgfqpoint{4.620000in}{4.620000in}}%
\pgfusepath{clip}%
\pgfsetbuttcap%
\pgfsetroundjoin%
\definecolor{currentfill}{rgb}{1.000000,0.576471,0.309804}%
\pgfsetfillcolor{currentfill}%
\pgfsetfillopacity{0.050000}%
\pgfsetlinewidth{0.000000pt}%
\definecolor{currentstroke}{rgb}{1.000000,0.576471,0.309804}%
\pgfsetstrokecolor{currentstroke}%
\pgfsetdash{}{0pt}%
\pgfpathmoveto{\pgfqpoint{3.315198in}{3.726912in}}%
\pgfpathlineto{\pgfqpoint{3.315198in}{3.799915in}}%
\pgfpathlineto{\pgfqpoint{3.393151in}{3.763318in}}%
\pgfpathlineto{\pgfqpoint{3.393151in}{3.690315in}}%
\pgfpathlineto{\pgfqpoint{3.315198in}{3.726912in}}%
\pgfpathclose%
\pgfusepath{fill}%
\end{pgfscope}%
\begin{pgfscope}%
\pgfpathrectangle{\pgfqpoint{0.765000in}{0.660000in}}{\pgfqpoint{4.620000in}{4.620000in}}%
\pgfusepath{clip}%
\pgfsetbuttcap%
\pgfsetroundjoin%
\definecolor{currentfill}{rgb}{1.000000,0.576471,0.309804}%
\pgfsetfillcolor{currentfill}%
\pgfsetfillopacity{0.050000}%
\pgfsetlinewidth{0.000000pt}%
\definecolor{currentstroke}{rgb}{1.000000,0.576471,0.309804}%
\pgfsetstrokecolor{currentstroke}%
\pgfsetdash{}{0pt}%
\pgfpathmoveto{\pgfqpoint{3.315198in}{3.726912in}}%
\pgfpathlineto{\pgfqpoint{3.315198in}{3.799915in}}%
\pgfpathlineto{\pgfqpoint{3.237245in}{3.763318in}}%
\pgfpathlineto{\pgfqpoint{3.237245in}{3.690315in}}%
\pgfpathlineto{\pgfqpoint{3.315198in}{3.726912in}}%
\pgfpathclose%
\pgfusepath{fill}%
\end{pgfscope}%
\begin{pgfscope}%
\pgfpathrectangle{\pgfqpoint{0.765000in}{0.660000in}}{\pgfqpoint{4.620000in}{4.620000in}}%
\pgfusepath{clip}%
\pgfsetbuttcap%
\pgfsetroundjoin%
\definecolor{currentfill}{rgb}{1.000000,0.576471,0.309804}%
\pgfsetfillcolor{currentfill}%
\pgfsetfillopacity{0.050000}%
\pgfsetlinewidth{0.000000pt}%
\definecolor{currentstroke}{rgb}{1.000000,0.576471,0.309804}%
\pgfsetstrokecolor{currentstroke}%
\pgfsetdash{}{0pt}%
\pgfpathmoveto{\pgfqpoint{3.315198in}{3.726912in}}%
\pgfpathlineto{\pgfqpoint{3.393151in}{3.690315in}}%
\pgfpathlineto{\pgfqpoint{3.315198in}{3.653718in}}%
\pgfpathlineto{\pgfqpoint{3.237245in}{3.690315in}}%
\pgfpathlineto{\pgfqpoint{3.315198in}{3.726912in}}%
\pgfpathclose%
\pgfusepath{fill}%
\end{pgfscope}%
\begin{pgfscope}%
\pgfpathrectangle{\pgfqpoint{0.765000in}{0.660000in}}{\pgfqpoint{4.620000in}{4.620000in}}%
\pgfusepath{clip}%
\pgfsetbuttcap%
\pgfsetroundjoin%
\definecolor{currentfill}{rgb}{1.000000,0.576471,0.309804}%
\pgfsetfillcolor{currentfill}%
\pgfsetfillopacity{0.050000}%
\pgfsetlinewidth{0.000000pt}%
\definecolor{currentstroke}{rgb}{1.000000,0.576471,0.309804}%
\pgfsetstrokecolor{currentstroke}%
\pgfsetdash{}{0pt}%
\pgfpathmoveto{\pgfqpoint{3.315198in}{3.799915in}}%
\pgfpathlineto{\pgfqpoint{3.393151in}{3.763318in}}%
\pgfpathlineto{\pgfqpoint{3.315198in}{3.726722in}}%
\pgfpathlineto{\pgfqpoint{3.237245in}{3.763318in}}%
\pgfpathlineto{\pgfqpoint{3.315198in}{3.799915in}}%
\pgfpathclose%
\pgfusepath{fill}%
\end{pgfscope}%
\begin{pgfscope}%
\pgfpathrectangle{\pgfqpoint{0.765000in}{0.660000in}}{\pgfqpoint{4.620000in}{4.620000in}}%
\pgfusepath{clip}%
\pgfsetbuttcap%
\pgfsetroundjoin%
\definecolor{currentfill}{rgb}{1.000000,0.576471,0.309804}%
\pgfsetfillcolor{currentfill}%
\pgfsetfillopacity{0.050000}%
\pgfsetlinewidth{0.000000pt}%
\definecolor{currentstroke}{rgb}{1.000000,0.576471,0.309804}%
\pgfsetstrokecolor{currentstroke}%
\pgfsetdash{}{0pt}%
\pgfpathmoveto{\pgfqpoint{3.237245in}{3.690315in}}%
\pgfpathlineto{\pgfqpoint{3.237245in}{3.763318in}}%
\pgfpathlineto{\pgfqpoint{3.315198in}{3.726722in}}%
\pgfpathlineto{\pgfqpoint{3.315198in}{3.653718in}}%
\pgfpathlineto{\pgfqpoint{3.237245in}{3.690315in}}%
\pgfpathclose%
\pgfusepath{fill}%
\end{pgfscope}%
\begin{pgfscope}%
\pgfpathrectangle{\pgfqpoint{0.765000in}{0.660000in}}{\pgfqpoint{4.620000in}{4.620000in}}%
\pgfusepath{clip}%
\pgfsetbuttcap%
\pgfsetroundjoin%
\definecolor{currentfill}{rgb}{1.000000,0.576471,0.309804}%
\pgfsetfillcolor{currentfill}%
\pgfsetfillopacity{0.050000}%
\pgfsetlinewidth{0.000000pt}%
\definecolor{currentstroke}{rgb}{1.000000,0.576471,0.309804}%
\pgfsetstrokecolor{currentstroke}%
\pgfsetdash{}{0pt}%
\pgfpathmoveto{\pgfqpoint{3.393151in}{3.690315in}}%
\pgfpathlineto{\pgfqpoint{3.393151in}{3.763318in}}%
\pgfpathlineto{\pgfqpoint{3.315198in}{3.726722in}}%
\pgfpathlineto{\pgfqpoint{3.315198in}{3.653718in}}%
\pgfpathlineto{\pgfqpoint{3.393151in}{3.690315in}}%
\pgfpathclose%
\pgfusepath{fill}%
\end{pgfscope}%
\begin{pgfscope}%
\pgfpathrectangle{\pgfqpoint{0.765000in}{0.660000in}}{\pgfqpoint{4.620000in}{4.620000in}}%
\pgfusepath{clip}%
\pgfsetbuttcap%
\pgfsetroundjoin%
\definecolor{currentfill}{rgb}{1.000000,0.576471,0.309804}%
\pgfsetfillcolor{currentfill}%
\pgfsetfillopacity{0.050000}%
\pgfsetlinewidth{0.000000pt}%
\definecolor{currentstroke}{rgb}{1.000000,0.576471,0.309804}%
\pgfsetstrokecolor{currentstroke}%
\pgfsetdash{}{0pt}%
\pgfpathmoveto{\pgfqpoint{3.703119in}{2.413201in}}%
\pgfpathlineto{\pgfqpoint{3.703119in}{2.486204in}}%
\pgfpathlineto{\pgfqpoint{3.781071in}{2.449608in}}%
\pgfpathlineto{\pgfqpoint{3.781071in}{2.376605in}}%
\pgfpathlineto{\pgfqpoint{3.703119in}{2.413201in}}%
\pgfpathclose%
\pgfusepath{fill}%
\end{pgfscope}%
\begin{pgfscope}%
\pgfpathrectangle{\pgfqpoint{0.765000in}{0.660000in}}{\pgfqpoint{4.620000in}{4.620000in}}%
\pgfusepath{clip}%
\pgfsetbuttcap%
\pgfsetroundjoin%
\definecolor{currentfill}{rgb}{1.000000,0.576471,0.309804}%
\pgfsetfillcolor{currentfill}%
\pgfsetfillopacity{0.050000}%
\pgfsetlinewidth{0.000000pt}%
\definecolor{currentstroke}{rgb}{1.000000,0.576471,0.309804}%
\pgfsetstrokecolor{currentstroke}%
\pgfsetdash{}{0pt}%
\pgfpathmoveto{\pgfqpoint{3.703119in}{2.413201in}}%
\pgfpathlineto{\pgfqpoint{3.703119in}{2.486204in}}%
\pgfpathlineto{\pgfqpoint{3.625166in}{2.449608in}}%
\pgfpathlineto{\pgfqpoint{3.625166in}{2.376605in}}%
\pgfpathlineto{\pgfqpoint{3.703119in}{2.413201in}}%
\pgfpathclose%
\pgfusepath{fill}%
\end{pgfscope}%
\begin{pgfscope}%
\pgfpathrectangle{\pgfqpoint{0.765000in}{0.660000in}}{\pgfqpoint{4.620000in}{4.620000in}}%
\pgfusepath{clip}%
\pgfsetbuttcap%
\pgfsetroundjoin%
\definecolor{currentfill}{rgb}{1.000000,0.576471,0.309804}%
\pgfsetfillcolor{currentfill}%
\pgfsetfillopacity{0.050000}%
\pgfsetlinewidth{0.000000pt}%
\definecolor{currentstroke}{rgb}{1.000000,0.576471,0.309804}%
\pgfsetstrokecolor{currentstroke}%
\pgfsetdash{}{0pt}%
\pgfpathmoveto{\pgfqpoint{3.703119in}{2.413201in}}%
\pgfpathlineto{\pgfqpoint{3.781071in}{2.376605in}}%
\pgfpathlineto{\pgfqpoint{3.703119in}{2.340008in}}%
\pgfpathlineto{\pgfqpoint{3.625166in}{2.376605in}}%
\pgfpathlineto{\pgfqpoint{3.703119in}{2.413201in}}%
\pgfpathclose%
\pgfusepath{fill}%
\end{pgfscope}%
\begin{pgfscope}%
\pgfpathrectangle{\pgfqpoint{0.765000in}{0.660000in}}{\pgfqpoint{4.620000in}{4.620000in}}%
\pgfusepath{clip}%
\pgfsetbuttcap%
\pgfsetroundjoin%
\definecolor{currentfill}{rgb}{1.000000,0.576471,0.309804}%
\pgfsetfillcolor{currentfill}%
\pgfsetfillopacity{0.050000}%
\pgfsetlinewidth{0.000000pt}%
\definecolor{currentstroke}{rgb}{1.000000,0.576471,0.309804}%
\pgfsetstrokecolor{currentstroke}%
\pgfsetdash{}{0pt}%
\pgfpathmoveto{\pgfqpoint{3.703119in}{2.486204in}}%
\pgfpathlineto{\pgfqpoint{3.781071in}{2.449608in}}%
\pgfpathlineto{\pgfqpoint{3.703119in}{2.413011in}}%
\pgfpathlineto{\pgfqpoint{3.625166in}{2.449608in}}%
\pgfpathlineto{\pgfqpoint{3.703119in}{2.486204in}}%
\pgfpathclose%
\pgfusepath{fill}%
\end{pgfscope}%
\begin{pgfscope}%
\pgfpathrectangle{\pgfqpoint{0.765000in}{0.660000in}}{\pgfqpoint{4.620000in}{4.620000in}}%
\pgfusepath{clip}%
\pgfsetbuttcap%
\pgfsetroundjoin%
\definecolor{currentfill}{rgb}{1.000000,0.576471,0.309804}%
\pgfsetfillcolor{currentfill}%
\pgfsetfillopacity{0.050000}%
\pgfsetlinewidth{0.000000pt}%
\definecolor{currentstroke}{rgb}{1.000000,0.576471,0.309804}%
\pgfsetstrokecolor{currentstroke}%
\pgfsetdash{}{0pt}%
\pgfpathmoveto{\pgfqpoint{3.625166in}{2.376605in}}%
\pgfpathlineto{\pgfqpoint{3.625166in}{2.449608in}}%
\pgfpathlineto{\pgfqpoint{3.703119in}{2.413011in}}%
\pgfpathlineto{\pgfqpoint{3.703119in}{2.340008in}}%
\pgfpathlineto{\pgfqpoint{3.625166in}{2.376605in}}%
\pgfpathclose%
\pgfusepath{fill}%
\end{pgfscope}%
\begin{pgfscope}%
\pgfpathrectangle{\pgfqpoint{0.765000in}{0.660000in}}{\pgfqpoint{4.620000in}{4.620000in}}%
\pgfusepath{clip}%
\pgfsetbuttcap%
\pgfsetroundjoin%
\definecolor{currentfill}{rgb}{1.000000,0.576471,0.309804}%
\pgfsetfillcolor{currentfill}%
\pgfsetfillopacity{0.050000}%
\pgfsetlinewidth{0.000000pt}%
\definecolor{currentstroke}{rgb}{1.000000,0.576471,0.309804}%
\pgfsetstrokecolor{currentstroke}%
\pgfsetdash{}{0pt}%
\pgfpathmoveto{\pgfqpoint{3.781071in}{2.376605in}}%
\pgfpathlineto{\pgfqpoint{3.781071in}{2.449608in}}%
\pgfpathlineto{\pgfqpoint{3.703119in}{2.413011in}}%
\pgfpathlineto{\pgfqpoint{3.703119in}{2.340008in}}%
\pgfpathlineto{\pgfqpoint{3.781071in}{2.376605in}}%
\pgfpathclose%
\pgfusepath{fill}%
\end{pgfscope}%
\begin{pgfscope}%
\pgfpathrectangle{\pgfqpoint{0.765000in}{0.660000in}}{\pgfqpoint{4.620000in}{4.620000in}}%
\pgfusepath{clip}%
\pgfsetbuttcap%
\pgfsetroundjoin%
\definecolor{currentfill}{rgb}{1.000000,0.576471,0.309804}%
\pgfsetfillcolor{currentfill}%
\pgfsetfillopacity{0.050000}%
\pgfsetlinewidth{0.000000pt}%
\definecolor{currentstroke}{rgb}{1.000000,0.576471,0.309804}%
\pgfsetstrokecolor{currentstroke}%
\pgfsetdash{}{0pt}%
\pgfpathmoveto{\pgfqpoint{2.549685in}{3.052582in}}%
\pgfpathlineto{\pgfqpoint{2.549685in}{3.125585in}}%
\pgfpathlineto{\pgfqpoint{2.627638in}{3.088989in}}%
\pgfpathlineto{\pgfqpoint{2.627638in}{3.015986in}}%
\pgfpathlineto{\pgfqpoint{2.549685in}{3.052582in}}%
\pgfpathclose%
\pgfusepath{fill}%
\end{pgfscope}%
\begin{pgfscope}%
\pgfpathrectangle{\pgfqpoint{0.765000in}{0.660000in}}{\pgfqpoint{4.620000in}{4.620000in}}%
\pgfusepath{clip}%
\pgfsetbuttcap%
\pgfsetroundjoin%
\definecolor{currentfill}{rgb}{1.000000,0.576471,0.309804}%
\pgfsetfillcolor{currentfill}%
\pgfsetfillopacity{0.050000}%
\pgfsetlinewidth{0.000000pt}%
\definecolor{currentstroke}{rgb}{1.000000,0.576471,0.309804}%
\pgfsetstrokecolor{currentstroke}%
\pgfsetdash{}{0pt}%
\pgfpathmoveto{\pgfqpoint{2.549685in}{3.052582in}}%
\pgfpathlineto{\pgfqpoint{2.549685in}{3.125585in}}%
\pgfpathlineto{\pgfqpoint{2.471732in}{3.088989in}}%
\pgfpathlineto{\pgfqpoint{2.471732in}{3.015986in}}%
\pgfpathlineto{\pgfqpoint{2.549685in}{3.052582in}}%
\pgfpathclose%
\pgfusepath{fill}%
\end{pgfscope}%
\begin{pgfscope}%
\pgfpathrectangle{\pgfqpoint{0.765000in}{0.660000in}}{\pgfqpoint{4.620000in}{4.620000in}}%
\pgfusepath{clip}%
\pgfsetbuttcap%
\pgfsetroundjoin%
\definecolor{currentfill}{rgb}{1.000000,0.576471,0.309804}%
\pgfsetfillcolor{currentfill}%
\pgfsetfillopacity{0.050000}%
\pgfsetlinewidth{0.000000pt}%
\definecolor{currentstroke}{rgb}{1.000000,0.576471,0.309804}%
\pgfsetstrokecolor{currentstroke}%
\pgfsetdash{}{0pt}%
\pgfpathmoveto{\pgfqpoint{2.549685in}{3.052582in}}%
\pgfpathlineto{\pgfqpoint{2.627638in}{3.015986in}}%
\pgfpathlineto{\pgfqpoint{2.549685in}{2.979389in}}%
\pgfpathlineto{\pgfqpoint{2.471732in}{3.015986in}}%
\pgfpathlineto{\pgfqpoint{2.549685in}{3.052582in}}%
\pgfpathclose%
\pgfusepath{fill}%
\end{pgfscope}%
\begin{pgfscope}%
\pgfpathrectangle{\pgfqpoint{0.765000in}{0.660000in}}{\pgfqpoint{4.620000in}{4.620000in}}%
\pgfusepath{clip}%
\pgfsetbuttcap%
\pgfsetroundjoin%
\definecolor{currentfill}{rgb}{1.000000,0.576471,0.309804}%
\pgfsetfillcolor{currentfill}%
\pgfsetfillopacity{0.050000}%
\pgfsetlinewidth{0.000000pt}%
\definecolor{currentstroke}{rgb}{1.000000,0.576471,0.309804}%
\pgfsetstrokecolor{currentstroke}%
\pgfsetdash{}{0pt}%
\pgfpathmoveto{\pgfqpoint{2.549685in}{3.125585in}}%
\pgfpathlineto{\pgfqpoint{2.627638in}{3.088989in}}%
\pgfpathlineto{\pgfqpoint{2.549685in}{3.052392in}}%
\pgfpathlineto{\pgfqpoint{2.471732in}{3.088989in}}%
\pgfpathlineto{\pgfqpoint{2.549685in}{3.125585in}}%
\pgfpathclose%
\pgfusepath{fill}%
\end{pgfscope}%
\begin{pgfscope}%
\pgfpathrectangle{\pgfqpoint{0.765000in}{0.660000in}}{\pgfqpoint{4.620000in}{4.620000in}}%
\pgfusepath{clip}%
\pgfsetbuttcap%
\pgfsetroundjoin%
\definecolor{currentfill}{rgb}{1.000000,0.576471,0.309804}%
\pgfsetfillcolor{currentfill}%
\pgfsetfillopacity{0.050000}%
\pgfsetlinewidth{0.000000pt}%
\definecolor{currentstroke}{rgb}{1.000000,0.576471,0.309804}%
\pgfsetstrokecolor{currentstroke}%
\pgfsetdash{}{0pt}%
\pgfpathmoveto{\pgfqpoint{2.471732in}{3.015986in}}%
\pgfpathlineto{\pgfqpoint{2.471732in}{3.088989in}}%
\pgfpathlineto{\pgfqpoint{2.549685in}{3.052392in}}%
\pgfpathlineto{\pgfqpoint{2.549685in}{2.979389in}}%
\pgfpathlineto{\pgfqpoint{2.471732in}{3.015986in}}%
\pgfpathclose%
\pgfusepath{fill}%
\end{pgfscope}%
\begin{pgfscope}%
\pgfpathrectangle{\pgfqpoint{0.765000in}{0.660000in}}{\pgfqpoint{4.620000in}{4.620000in}}%
\pgfusepath{clip}%
\pgfsetbuttcap%
\pgfsetroundjoin%
\definecolor{currentfill}{rgb}{1.000000,0.576471,0.309804}%
\pgfsetfillcolor{currentfill}%
\pgfsetfillopacity{0.050000}%
\pgfsetlinewidth{0.000000pt}%
\definecolor{currentstroke}{rgb}{1.000000,0.576471,0.309804}%
\pgfsetstrokecolor{currentstroke}%
\pgfsetdash{}{0pt}%
\pgfpathmoveto{\pgfqpoint{2.627638in}{3.015986in}}%
\pgfpathlineto{\pgfqpoint{2.627638in}{3.088989in}}%
\pgfpathlineto{\pgfqpoint{2.549685in}{3.052392in}}%
\pgfpathlineto{\pgfqpoint{2.549685in}{2.979389in}}%
\pgfpathlineto{\pgfqpoint{2.627638in}{3.015986in}}%
\pgfpathclose%
\pgfusepath{fill}%
\end{pgfscope}%
\begin{pgfscope}%
\pgfpathrectangle{\pgfqpoint{0.765000in}{0.660000in}}{\pgfqpoint{4.620000in}{4.620000in}}%
\pgfusepath{clip}%
\pgfsetbuttcap%
\pgfsetroundjoin%
\definecolor{currentfill}{rgb}{1.000000,0.576471,0.309804}%
\pgfsetfillcolor{currentfill}%
\pgfsetfillopacity{0.050000}%
\pgfsetlinewidth{0.000000pt}%
\definecolor{currentstroke}{rgb}{1.000000,0.576471,0.309804}%
\pgfsetstrokecolor{currentstroke}%
\pgfsetdash{}{0pt}%
\pgfpathmoveto{\pgfqpoint{2.794256in}{2.614860in}}%
\pgfpathlineto{\pgfqpoint{2.794256in}{2.687863in}}%
\pgfpathlineto{\pgfqpoint{2.872209in}{2.651267in}}%
\pgfpathlineto{\pgfqpoint{2.872209in}{2.578263in}}%
\pgfpathlineto{\pgfqpoint{2.794256in}{2.614860in}}%
\pgfpathclose%
\pgfusepath{fill}%
\end{pgfscope}%
\begin{pgfscope}%
\pgfpathrectangle{\pgfqpoint{0.765000in}{0.660000in}}{\pgfqpoint{4.620000in}{4.620000in}}%
\pgfusepath{clip}%
\pgfsetbuttcap%
\pgfsetroundjoin%
\definecolor{currentfill}{rgb}{1.000000,0.576471,0.309804}%
\pgfsetfillcolor{currentfill}%
\pgfsetfillopacity{0.050000}%
\pgfsetlinewidth{0.000000pt}%
\definecolor{currentstroke}{rgb}{1.000000,0.576471,0.309804}%
\pgfsetstrokecolor{currentstroke}%
\pgfsetdash{}{0pt}%
\pgfpathmoveto{\pgfqpoint{2.794256in}{2.614860in}}%
\pgfpathlineto{\pgfqpoint{2.794256in}{2.687863in}}%
\pgfpathlineto{\pgfqpoint{2.716304in}{2.651267in}}%
\pgfpathlineto{\pgfqpoint{2.716304in}{2.578263in}}%
\pgfpathlineto{\pgfqpoint{2.794256in}{2.614860in}}%
\pgfpathclose%
\pgfusepath{fill}%
\end{pgfscope}%
\begin{pgfscope}%
\pgfpathrectangle{\pgfqpoint{0.765000in}{0.660000in}}{\pgfqpoint{4.620000in}{4.620000in}}%
\pgfusepath{clip}%
\pgfsetbuttcap%
\pgfsetroundjoin%
\definecolor{currentfill}{rgb}{1.000000,0.576471,0.309804}%
\pgfsetfillcolor{currentfill}%
\pgfsetfillopacity{0.050000}%
\pgfsetlinewidth{0.000000pt}%
\definecolor{currentstroke}{rgb}{1.000000,0.576471,0.309804}%
\pgfsetstrokecolor{currentstroke}%
\pgfsetdash{}{0pt}%
\pgfpathmoveto{\pgfqpoint{2.794256in}{2.614860in}}%
\pgfpathlineto{\pgfqpoint{2.872209in}{2.578263in}}%
\pgfpathlineto{\pgfqpoint{2.794256in}{2.541667in}}%
\pgfpathlineto{\pgfqpoint{2.716304in}{2.578263in}}%
\pgfpathlineto{\pgfqpoint{2.794256in}{2.614860in}}%
\pgfpathclose%
\pgfusepath{fill}%
\end{pgfscope}%
\begin{pgfscope}%
\pgfpathrectangle{\pgfqpoint{0.765000in}{0.660000in}}{\pgfqpoint{4.620000in}{4.620000in}}%
\pgfusepath{clip}%
\pgfsetbuttcap%
\pgfsetroundjoin%
\definecolor{currentfill}{rgb}{1.000000,0.576471,0.309804}%
\pgfsetfillcolor{currentfill}%
\pgfsetfillopacity{0.050000}%
\pgfsetlinewidth{0.000000pt}%
\definecolor{currentstroke}{rgb}{1.000000,0.576471,0.309804}%
\pgfsetstrokecolor{currentstroke}%
\pgfsetdash{}{0pt}%
\pgfpathmoveto{\pgfqpoint{2.794256in}{2.687863in}}%
\pgfpathlineto{\pgfqpoint{2.872209in}{2.651267in}}%
\pgfpathlineto{\pgfqpoint{2.794256in}{2.614670in}}%
\pgfpathlineto{\pgfqpoint{2.716304in}{2.651267in}}%
\pgfpathlineto{\pgfqpoint{2.794256in}{2.687863in}}%
\pgfpathclose%
\pgfusepath{fill}%
\end{pgfscope}%
\begin{pgfscope}%
\pgfpathrectangle{\pgfqpoint{0.765000in}{0.660000in}}{\pgfqpoint{4.620000in}{4.620000in}}%
\pgfusepath{clip}%
\pgfsetbuttcap%
\pgfsetroundjoin%
\definecolor{currentfill}{rgb}{1.000000,0.576471,0.309804}%
\pgfsetfillcolor{currentfill}%
\pgfsetfillopacity{0.050000}%
\pgfsetlinewidth{0.000000pt}%
\definecolor{currentstroke}{rgb}{1.000000,0.576471,0.309804}%
\pgfsetstrokecolor{currentstroke}%
\pgfsetdash{}{0pt}%
\pgfpathmoveto{\pgfqpoint{2.716304in}{2.578263in}}%
\pgfpathlineto{\pgfqpoint{2.716304in}{2.651267in}}%
\pgfpathlineto{\pgfqpoint{2.794256in}{2.614670in}}%
\pgfpathlineto{\pgfqpoint{2.794256in}{2.541667in}}%
\pgfpathlineto{\pgfqpoint{2.716304in}{2.578263in}}%
\pgfpathclose%
\pgfusepath{fill}%
\end{pgfscope}%
\begin{pgfscope}%
\pgfpathrectangle{\pgfqpoint{0.765000in}{0.660000in}}{\pgfqpoint{4.620000in}{4.620000in}}%
\pgfusepath{clip}%
\pgfsetbuttcap%
\pgfsetroundjoin%
\definecolor{currentfill}{rgb}{1.000000,0.576471,0.309804}%
\pgfsetfillcolor{currentfill}%
\pgfsetfillopacity{0.050000}%
\pgfsetlinewidth{0.000000pt}%
\definecolor{currentstroke}{rgb}{1.000000,0.576471,0.309804}%
\pgfsetstrokecolor{currentstroke}%
\pgfsetdash{}{0pt}%
\pgfpathmoveto{\pgfqpoint{2.872209in}{2.578263in}}%
\pgfpathlineto{\pgfqpoint{2.872209in}{2.651267in}}%
\pgfpathlineto{\pgfqpoint{2.794256in}{2.614670in}}%
\pgfpathlineto{\pgfqpoint{2.794256in}{2.541667in}}%
\pgfpathlineto{\pgfqpoint{2.872209in}{2.578263in}}%
\pgfpathclose%
\pgfusepath{fill}%
\end{pgfscope}%
\begin{pgfscope}%
\pgfpathrectangle{\pgfqpoint{0.765000in}{0.660000in}}{\pgfqpoint{4.620000in}{4.620000in}}%
\pgfusepath{clip}%
\pgfsetbuttcap%
\pgfsetroundjoin%
\definecolor{currentfill}{rgb}{1.000000,0.576471,0.309804}%
\pgfsetfillcolor{currentfill}%
\pgfsetfillopacity{0.050000}%
\pgfsetlinewidth{0.000000pt}%
\definecolor{currentstroke}{rgb}{1.000000,0.576471,0.309804}%
\pgfsetstrokecolor{currentstroke}%
\pgfsetdash{}{0pt}%
\pgfpathmoveto{\pgfqpoint{3.053692in}{3.568112in}}%
\pgfpathlineto{\pgfqpoint{3.053692in}{3.641116in}}%
\pgfpathlineto{\pgfqpoint{3.131645in}{3.604519in}}%
\pgfpathlineto{\pgfqpoint{3.131645in}{3.531516in}}%
\pgfpathlineto{\pgfqpoint{3.053692in}{3.568112in}}%
\pgfpathclose%
\pgfusepath{fill}%
\end{pgfscope}%
\begin{pgfscope}%
\pgfpathrectangle{\pgfqpoint{0.765000in}{0.660000in}}{\pgfqpoint{4.620000in}{4.620000in}}%
\pgfusepath{clip}%
\pgfsetbuttcap%
\pgfsetroundjoin%
\definecolor{currentfill}{rgb}{1.000000,0.576471,0.309804}%
\pgfsetfillcolor{currentfill}%
\pgfsetfillopacity{0.050000}%
\pgfsetlinewidth{0.000000pt}%
\definecolor{currentstroke}{rgb}{1.000000,0.576471,0.309804}%
\pgfsetstrokecolor{currentstroke}%
\pgfsetdash{}{0pt}%
\pgfpathmoveto{\pgfqpoint{3.053692in}{3.568112in}}%
\pgfpathlineto{\pgfqpoint{3.053692in}{3.641116in}}%
\pgfpathlineto{\pgfqpoint{2.975739in}{3.604519in}}%
\pgfpathlineto{\pgfqpoint{2.975739in}{3.531516in}}%
\pgfpathlineto{\pgfqpoint{3.053692in}{3.568112in}}%
\pgfpathclose%
\pgfusepath{fill}%
\end{pgfscope}%
\begin{pgfscope}%
\pgfpathrectangle{\pgfqpoint{0.765000in}{0.660000in}}{\pgfqpoint{4.620000in}{4.620000in}}%
\pgfusepath{clip}%
\pgfsetbuttcap%
\pgfsetroundjoin%
\definecolor{currentfill}{rgb}{1.000000,0.576471,0.309804}%
\pgfsetfillcolor{currentfill}%
\pgfsetfillopacity{0.050000}%
\pgfsetlinewidth{0.000000pt}%
\definecolor{currentstroke}{rgb}{1.000000,0.576471,0.309804}%
\pgfsetstrokecolor{currentstroke}%
\pgfsetdash{}{0pt}%
\pgfpathmoveto{\pgfqpoint{3.053692in}{3.568112in}}%
\pgfpathlineto{\pgfqpoint{3.131645in}{3.531516in}}%
\pgfpathlineto{\pgfqpoint{3.053692in}{3.494919in}}%
\pgfpathlineto{\pgfqpoint{2.975739in}{3.531516in}}%
\pgfpathlineto{\pgfqpoint{3.053692in}{3.568112in}}%
\pgfpathclose%
\pgfusepath{fill}%
\end{pgfscope}%
\begin{pgfscope}%
\pgfpathrectangle{\pgfqpoint{0.765000in}{0.660000in}}{\pgfqpoint{4.620000in}{4.620000in}}%
\pgfusepath{clip}%
\pgfsetbuttcap%
\pgfsetroundjoin%
\definecolor{currentfill}{rgb}{1.000000,0.576471,0.309804}%
\pgfsetfillcolor{currentfill}%
\pgfsetfillopacity{0.050000}%
\pgfsetlinewidth{0.000000pt}%
\definecolor{currentstroke}{rgb}{1.000000,0.576471,0.309804}%
\pgfsetstrokecolor{currentstroke}%
\pgfsetdash{}{0pt}%
\pgfpathmoveto{\pgfqpoint{3.053692in}{3.641116in}}%
\pgfpathlineto{\pgfqpoint{3.131645in}{3.604519in}}%
\pgfpathlineto{\pgfqpoint{3.053692in}{3.567922in}}%
\pgfpathlineto{\pgfqpoint{2.975739in}{3.604519in}}%
\pgfpathlineto{\pgfqpoint{3.053692in}{3.641116in}}%
\pgfpathclose%
\pgfusepath{fill}%
\end{pgfscope}%
\begin{pgfscope}%
\pgfpathrectangle{\pgfqpoint{0.765000in}{0.660000in}}{\pgfqpoint{4.620000in}{4.620000in}}%
\pgfusepath{clip}%
\pgfsetbuttcap%
\pgfsetroundjoin%
\definecolor{currentfill}{rgb}{1.000000,0.576471,0.309804}%
\pgfsetfillcolor{currentfill}%
\pgfsetfillopacity{0.050000}%
\pgfsetlinewidth{0.000000pt}%
\definecolor{currentstroke}{rgb}{1.000000,0.576471,0.309804}%
\pgfsetstrokecolor{currentstroke}%
\pgfsetdash{}{0pt}%
\pgfpathmoveto{\pgfqpoint{2.975739in}{3.531516in}}%
\pgfpathlineto{\pgfqpoint{2.975739in}{3.604519in}}%
\pgfpathlineto{\pgfqpoint{3.053692in}{3.567922in}}%
\pgfpathlineto{\pgfqpoint{3.053692in}{3.494919in}}%
\pgfpathlineto{\pgfqpoint{2.975739in}{3.531516in}}%
\pgfpathclose%
\pgfusepath{fill}%
\end{pgfscope}%
\begin{pgfscope}%
\pgfpathrectangle{\pgfqpoint{0.765000in}{0.660000in}}{\pgfqpoint{4.620000in}{4.620000in}}%
\pgfusepath{clip}%
\pgfsetbuttcap%
\pgfsetroundjoin%
\definecolor{currentfill}{rgb}{1.000000,0.576471,0.309804}%
\pgfsetfillcolor{currentfill}%
\pgfsetfillopacity{0.050000}%
\pgfsetlinewidth{0.000000pt}%
\definecolor{currentstroke}{rgb}{1.000000,0.576471,0.309804}%
\pgfsetstrokecolor{currentstroke}%
\pgfsetdash{}{0pt}%
\pgfpathmoveto{\pgfqpoint{3.131645in}{3.531516in}}%
\pgfpathlineto{\pgfqpoint{3.131645in}{3.604519in}}%
\pgfpathlineto{\pgfqpoint{3.053692in}{3.567922in}}%
\pgfpathlineto{\pgfqpoint{3.053692in}{3.494919in}}%
\pgfpathlineto{\pgfqpoint{3.131645in}{3.531516in}}%
\pgfpathclose%
\pgfusepath{fill}%
\end{pgfscope}%
\begin{pgfscope}%
\pgfpathrectangle{\pgfqpoint{0.765000in}{0.660000in}}{\pgfqpoint{4.620000in}{4.620000in}}%
\pgfusepath{clip}%
\pgfsetbuttcap%
\pgfsetroundjoin%
\definecolor{currentfill}{rgb}{1.000000,0.576471,0.309804}%
\pgfsetfillcolor{currentfill}%
\pgfsetfillopacity{0.050000}%
\pgfsetlinewidth{0.000000pt}%
\definecolor{currentstroke}{rgb}{1.000000,0.576471,0.309804}%
\pgfsetstrokecolor{currentstroke}%
\pgfsetdash{}{0pt}%
\pgfpathmoveto{\pgfqpoint{2.460873in}{2.800557in}}%
\pgfpathlineto{\pgfqpoint{2.460873in}{2.873560in}}%
\pgfpathlineto{\pgfqpoint{2.538826in}{2.836963in}}%
\pgfpathlineto{\pgfqpoint{2.538826in}{2.763960in}}%
\pgfpathlineto{\pgfqpoint{2.460873in}{2.800557in}}%
\pgfpathclose%
\pgfusepath{fill}%
\end{pgfscope}%
\begin{pgfscope}%
\pgfpathrectangle{\pgfqpoint{0.765000in}{0.660000in}}{\pgfqpoint{4.620000in}{4.620000in}}%
\pgfusepath{clip}%
\pgfsetbuttcap%
\pgfsetroundjoin%
\definecolor{currentfill}{rgb}{1.000000,0.576471,0.309804}%
\pgfsetfillcolor{currentfill}%
\pgfsetfillopacity{0.050000}%
\pgfsetlinewidth{0.000000pt}%
\definecolor{currentstroke}{rgb}{1.000000,0.576471,0.309804}%
\pgfsetstrokecolor{currentstroke}%
\pgfsetdash{}{0pt}%
\pgfpathmoveto{\pgfqpoint{2.460873in}{2.800557in}}%
\pgfpathlineto{\pgfqpoint{2.460873in}{2.873560in}}%
\pgfpathlineto{\pgfqpoint{2.382920in}{2.836963in}}%
\pgfpathlineto{\pgfqpoint{2.382920in}{2.763960in}}%
\pgfpathlineto{\pgfqpoint{2.460873in}{2.800557in}}%
\pgfpathclose%
\pgfusepath{fill}%
\end{pgfscope}%
\begin{pgfscope}%
\pgfpathrectangle{\pgfqpoint{0.765000in}{0.660000in}}{\pgfqpoint{4.620000in}{4.620000in}}%
\pgfusepath{clip}%
\pgfsetbuttcap%
\pgfsetroundjoin%
\definecolor{currentfill}{rgb}{1.000000,0.576471,0.309804}%
\pgfsetfillcolor{currentfill}%
\pgfsetfillopacity{0.050000}%
\pgfsetlinewidth{0.000000pt}%
\definecolor{currentstroke}{rgb}{1.000000,0.576471,0.309804}%
\pgfsetstrokecolor{currentstroke}%
\pgfsetdash{}{0pt}%
\pgfpathmoveto{\pgfqpoint{2.460873in}{2.800557in}}%
\pgfpathlineto{\pgfqpoint{2.538826in}{2.763960in}}%
\pgfpathlineto{\pgfqpoint{2.460873in}{2.727364in}}%
\pgfpathlineto{\pgfqpoint{2.382920in}{2.763960in}}%
\pgfpathlineto{\pgfqpoint{2.460873in}{2.800557in}}%
\pgfpathclose%
\pgfusepath{fill}%
\end{pgfscope}%
\begin{pgfscope}%
\pgfpathrectangle{\pgfqpoint{0.765000in}{0.660000in}}{\pgfqpoint{4.620000in}{4.620000in}}%
\pgfusepath{clip}%
\pgfsetbuttcap%
\pgfsetroundjoin%
\definecolor{currentfill}{rgb}{1.000000,0.576471,0.309804}%
\pgfsetfillcolor{currentfill}%
\pgfsetfillopacity{0.050000}%
\pgfsetlinewidth{0.000000pt}%
\definecolor{currentstroke}{rgb}{1.000000,0.576471,0.309804}%
\pgfsetstrokecolor{currentstroke}%
\pgfsetdash{}{0pt}%
\pgfpathmoveto{\pgfqpoint{2.460873in}{2.873560in}}%
\pgfpathlineto{\pgfqpoint{2.538826in}{2.836963in}}%
\pgfpathlineto{\pgfqpoint{2.460873in}{2.800367in}}%
\pgfpathlineto{\pgfqpoint{2.382920in}{2.836963in}}%
\pgfpathlineto{\pgfqpoint{2.460873in}{2.873560in}}%
\pgfpathclose%
\pgfusepath{fill}%
\end{pgfscope}%
\begin{pgfscope}%
\pgfpathrectangle{\pgfqpoint{0.765000in}{0.660000in}}{\pgfqpoint{4.620000in}{4.620000in}}%
\pgfusepath{clip}%
\pgfsetbuttcap%
\pgfsetroundjoin%
\definecolor{currentfill}{rgb}{1.000000,0.576471,0.309804}%
\pgfsetfillcolor{currentfill}%
\pgfsetfillopacity{0.050000}%
\pgfsetlinewidth{0.000000pt}%
\definecolor{currentstroke}{rgb}{1.000000,0.576471,0.309804}%
\pgfsetstrokecolor{currentstroke}%
\pgfsetdash{}{0pt}%
\pgfpathmoveto{\pgfqpoint{2.382920in}{2.763960in}}%
\pgfpathlineto{\pgfqpoint{2.382920in}{2.836963in}}%
\pgfpathlineto{\pgfqpoint{2.460873in}{2.800367in}}%
\pgfpathlineto{\pgfqpoint{2.460873in}{2.727364in}}%
\pgfpathlineto{\pgfqpoint{2.382920in}{2.763960in}}%
\pgfpathclose%
\pgfusepath{fill}%
\end{pgfscope}%
\begin{pgfscope}%
\pgfpathrectangle{\pgfqpoint{0.765000in}{0.660000in}}{\pgfqpoint{4.620000in}{4.620000in}}%
\pgfusepath{clip}%
\pgfsetbuttcap%
\pgfsetroundjoin%
\definecolor{currentfill}{rgb}{1.000000,0.576471,0.309804}%
\pgfsetfillcolor{currentfill}%
\pgfsetfillopacity{0.050000}%
\pgfsetlinewidth{0.000000pt}%
\definecolor{currentstroke}{rgb}{1.000000,0.576471,0.309804}%
\pgfsetstrokecolor{currentstroke}%
\pgfsetdash{}{0pt}%
\pgfpathmoveto{\pgfqpoint{2.538826in}{2.763960in}}%
\pgfpathlineto{\pgfqpoint{2.538826in}{2.836963in}}%
\pgfpathlineto{\pgfqpoint{2.460873in}{2.800367in}}%
\pgfpathlineto{\pgfqpoint{2.460873in}{2.727364in}}%
\pgfpathlineto{\pgfqpoint{2.538826in}{2.763960in}}%
\pgfpathclose%
\pgfusepath{fill}%
\end{pgfscope}%
\begin{pgfscope}%
\pgfpathrectangle{\pgfqpoint{0.765000in}{0.660000in}}{\pgfqpoint{4.620000in}{4.620000in}}%
\pgfusepath{clip}%
\pgfsetbuttcap%
\pgfsetroundjoin%
\definecolor{currentfill}{rgb}{1.000000,0.576471,0.309804}%
\pgfsetfillcolor{currentfill}%
\pgfsetfillopacity{0.050000}%
\pgfsetlinewidth{0.000000pt}%
\definecolor{currentstroke}{rgb}{1.000000,0.576471,0.309804}%
\pgfsetstrokecolor{currentstroke}%
\pgfsetdash{}{0pt}%
\pgfpathmoveto{\pgfqpoint{3.025236in}{3.672546in}}%
\pgfpathlineto{\pgfqpoint{3.025236in}{3.745549in}}%
\pgfpathlineto{\pgfqpoint{3.103189in}{3.708953in}}%
\pgfpathlineto{\pgfqpoint{3.103189in}{3.635950in}}%
\pgfpathlineto{\pgfqpoint{3.025236in}{3.672546in}}%
\pgfpathclose%
\pgfusepath{fill}%
\end{pgfscope}%
\begin{pgfscope}%
\pgfpathrectangle{\pgfqpoint{0.765000in}{0.660000in}}{\pgfqpoint{4.620000in}{4.620000in}}%
\pgfusepath{clip}%
\pgfsetbuttcap%
\pgfsetroundjoin%
\definecolor{currentfill}{rgb}{1.000000,0.576471,0.309804}%
\pgfsetfillcolor{currentfill}%
\pgfsetfillopacity{0.050000}%
\pgfsetlinewidth{0.000000pt}%
\definecolor{currentstroke}{rgb}{1.000000,0.576471,0.309804}%
\pgfsetstrokecolor{currentstroke}%
\pgfsetdash{}{0pt}%
\pgfpathmoveto{\pgfqpoint{3.025236in}{3.672546in}}%
\pgfpathlineto{\pgfqpoint{3.025236in}{3.745549in}}%
\pgfpathlineto{\pgfqpoint{2.947283in}{3.708953in}}%
\pgfpathlineto{\pgfqpoint{2.947283in}{3.635950in}}%
\pgfpathlineto{\pgfqpoint{3.025236in}{3.672546in}}%
\pgfpathclose%
\pgfusepath{fill}%
\end{pgfscope}%
\begin{pgfscope}%
\pgfpathrectangle{\pgfqpoint{0.765000in}{0.660000in}}{\pgfqpoint{4.620000in}{4.620000in}}%
\pgfusepath{clip}%
\pgfsetbuttcap%
\pgfsetroundjoin%
\definecolor{currentfill}{rgb}{1.000000,0.576471,0.309804}%
\pgfsetfillcolor{currentfill}%
\pgfsetfillopacity{0.050000}%
\pgfsetlinewidth{0.000000pt}%
\definecolor{currentstroke}{rgb}{1.000000,0.576471,0.309804}%
\pgfsetstrokecolor{currentstroke}%
\pgfsetdash{}{0pt}%
\pgfpathmoveto{\pgfqpoint{3.025236in}{3.672546in}}%
\pgfpathlineto{\pgfqpoint{3.103189in}{3.635950in}}%
\pgfpathlineto{\pgfqpoint{3.025236in}{3.599353in}}%
\pgfpathlineto{\pgfqpoint{2.947283in}{3.635950in}}%
\pgfpathlineto{\pgfqpoint{3.025236in}{3.672546in}}%
\pgfpathclose%
\pgfusepath{fill}%
\end{pgfscope}%
\begin{pgfscope}%
\pgfpathrectangle{\pgfqpoint{0.765000in}{0.660000in}}{\pgfqpoint{4.620000in}{4.620000in}}%
\pgfusepath{clip}%
\pgfsetbuttcap%
\pgfsetroundjoin%
\definecolor{currentfill}{rgb}{1.000000,0.576471,0.309804}%
\pgfsetfillcolor{currentfill}%
\pgfsetfillopacity{0.050000}%
\pgfsetlinewidth{0.000000pt}%
\definecolor{currentstroke}{rgb}{1.000000,0.576471,0.309804}%
\pgfsetstrokecolor{currentstroke}%
\pgfsetdash{}{0pt}%
\pgfpathmoveto{\pgfqpoint{3.025236in}{3.745549in}}%
\pgfpathlineto{\pgfqpoint{3.103189in}{3.708953in}}%
\pgfpathlineto{\pgfqpoint{3.025236in}{3.672356in}}%
\pgfpathlineto{\pgfqpoint{2.947283in}{3.708953in}}%
\pgfpathlineto{\pgfqpoint{3.025236in}{3.745549in}}%
\pgfpathclose%
\pgfusepath{fill}%
\end{pgfscope}%
\begin{pgfscope}%
\pgfpathrectangle{\pgfqpoint{0.765000in}{0.660000in}}{\pgfqpoint{4.620000in}{4.620000in}}%
\pgfusepath{clip}%
\pgfsetbuttcap%
\pgfsetroundjoin%
\definecolor{currentfill}{rgb}{1.000000,0.576471,0.309804}%
\pgfsetfillcolor{currentfill}%
\pgfsetfillopacity{0.050000}%
\pgfsetlinewidth{0.000000pt}%
\definecolor{currentstroke}{rgb}{1.000000,0.576471,0.309804}%
\pgfsetstrokecolor{currentstroke}%
\pgfsetdash{}{0pt}%
\pgfpathmoveto{\pgfqpoint{2.947283in}{3.635950in}}%
\pgfpathlineto{\pgfqpoint{2.947283in}{3.708953in}}%
\pgfpathlineto{\pgfqpoint{3.025236in}{3.672356in}}%
\pgfpathlineto{\pgfqpoint{3.025236in}{3.599353in}}%
\pgfpathlineto{\pgfqpoint{2.947283in}{3.635950in}}%
\pgfpathclose%
\pgfusepath{fill}%
\end{pgfscope}%
\begin{pgfscope}%
\pgfpathrectangle{\pgfqpoint{0.765000in}{0.660000in}}{\pgfqpoint{4.620000in}{4.620000in}}%
\pgfusepath{clip}%
\pgfsetbuttcap%
\pgfsetroundjoin%
\definecolor{currentfill}{rgb}{1.000000,0.576471,0.309804}%
\pgfsetfillcolor{currentfill}%
\pgfsetfillopacity{0.050000}%
\pgfsetlinewidth{0.000000pt}%
\definecolor{currentstroke}{rgb}{1.000000,0.576471,0.309804}%
\pgfsetstrokecolor{currentstroke}%
\pgfsetdash{}{0pt}%
\pgfpathmoveto{\pgfqpoint{3.103189in}{3.635950in}}%
\pgfpathlineto{\pgfqpoint{3.103189in}{3.708953in}}%
\pgfpathlineto{\pgfqpoint{3.025236in}{3.672356in}}%
\pgfpathlineto{\pgfqpoint{3.025236in}{3.599353in}}%
\pgfpathlineto{\pgfqpoint{3.103189in}{3.635950in}}%
\pgfpathclose%
\pgfusepath{fill}%
\end{pgfscope}%
\begin{pgfscope}%
\pgfpathrectangle{\pgfqpoint{0.765000in}{0.660000in}}{\pgfqpoint{4.620000in}{4.620000in}}%
\pgfusepath{clip}%
\pgfsetbuttcap%
\pgfsetroundjoin%
\definecolor{currentfill}{rgb}{1.000000,0.576471,0.309804}%
\pgfsetfillcolor{currentfill}%
\pgfsetfillopacity{0.050000}%
\pgfsetlinewidth{0.000000pt}%
\definecolor{currentstroke}{rgb}{1.000000,0.576471,0.309804}%
\pgfsetstrokecolor{currentstroke}%
\pgfsetdash{}{0pt}%
\pgfpathmoveto{\pgfqpoint{2.558191in}{2.818999in}}%
\pgfpathlineto{\pgfqpoint{2.558191in}{2.892002in}}%
\pgfpathlineto{\pgfqpoint{2.636143in}{2.855405in}}%
\pgfpathlineto{\pgfqpoint{2.636143in}{2.782402in}}%
\pgfpathlineto{\pgfqpoint{2.558191in}{2.818999in}}%
\pgfpathclose%
\pgfusepath{fill}%
\end{pgfscope}%
\begin{pgfscope}%
\pgfpathrectangle{\pgfqpoint{0.765000in}{0.660000in}}{\pgfqpoint{4.620000in}{4.620000in}}%
\pgfusepath{clip}%
\pgfsetbuttcap%
\pgfsetroundjoin%
\definecolor{currentfill}{rgb}{1.000000,0.576471,0.309804}%
\pgfsetfillcolor{currentfill}%
\pgfsetfillopacity{0.050000}%
\pgfsetlinewidth{0.000000pt}%
\definecolor{currentstroke}{rgb}{1.000000,0.576471,0.309804}%
\pgfsetstrokecolor{currentstroke}%
\pgfsetdash{}{0pt}%
\pgfpathmoveto{\pgfqpoint{2.558191in}{2.818999in}}%
\pgfpathlineto{\pgfqpoint{2.558191in}{2.892002in}}%
\pgfpathlineto{\pgfqpoint{2.480238in}{2.855405in}}%
\pgfpathlineto{\pgfqpoint{2.480238in}{2.782402in}}%
\pgfpathlineto{\pgfqpoint{2.558191in}{2.818999in}}%
\pgfpathclose%
\pgfusepath{fill}%
\end{pgfscope}%
\begin{pgfscope}%
\pgfpathrectangle{\pgfqpoint{0.765000in}{0.660000in}}{\pgfqpoint{4.620000in}{4.620000in}}%
\pgfusepath{clip}%
\pgfsetbuttcap%
\pgfsetroundjoin%
\definecolor{currentfill}{rgb}{1.000000,0.576471,0.309804}%
\pgfsetfillcolor{currentfill}%
\pgfsetfillopacity{0.050000}%
\pgfsetlinewidth{0.000000pt}%
\definecolor{currentstroke}{rgb}{1.000000,0.576471,0.309804}%
\pgfsetstrokecolor{currentstroke}%
\pgfsetdash{}{0pt}%
\pgfpathmoveto{\pgfqpoint{2.558191in}{2.818999in}}%
\pgfpathlineto{\pgfqpoint{2.636143in}{2.782402in}}%
\pgfpathlineto{\pgfqpoint{2.558191in}{2.745806in}}%
\pgfpathlineto{\pgfqpoint{2.480238in}{2.782402in}}%
\pgfpathlineto{\pgfqpoint{2.558191in}{2.818999in}}%
\pgfpathclose%
\pgfusepath{fill}%
\end{pgfscope}%
\begin{pgfscope}%
\pgfpathrectangle{\pgfqpoint{0.765000in}{0.660000in}}{\pgfqpoint{4.620000in}{4.620000in}}%
\pgfusepath{clip}%
\pgfsetbuttcap%
\pgfsetroundjoin%
\definecolor{currentfill}{rgb}{1.000000,0.576471,0.309804}%
\pgfsetfillcolor{currentfill}%
\pgfsetfillopacity{0.050000}%
\pgfsetlinewidth{0.000000pt}%
\definecolor{currentstroke}{rgb}{1.000000,0.576471,0.309804}%
\pgfsetstrokecolor{currentstroke}%
\pgfsetdash{}{0pt}%
\pgfpathmoveto{\pgfqpoint{2.558191in}{2.892002in}}%
\pgfpathlineto{\pgfqpoint{2.636143in}{2.855405in}}%
\pgfpathlineto{\pgfqpoint{2.558191in}{2.818809in}}%
\pgfpathlineto{\pgfqpoint{2.480238in}{2.855405in}}%
\pgfpathlineto{\pgfqpoint{2.558191in}{2.892002in}}%
\pgfpathclose%
\pgfusepath{fill}%
\end{pgfscope}%
\begin{pgfscope}%
\pgfpathrectangle{\pgfqpoint{0.765000in}{0.660000in}}{\pgfqpoint{4.620000in}{4.620000in}}%
\pgfusepath{clip}%
\pgfsetbuttcap%
\pgfsetroundjoin%
\definecolor{currentfill}{rgb}{1.000000,0.576471,0.309804}%
\pgfsetfillcolor{currentfill}%
\pgfsetfillopacity{0.050000}%
\pgfsetlinewidth{0.000000pt}%
\definecolor{currentstroke}{rgb}{1.000000,0.576471,0.309804}%
\pgfsetstrokecolor{currentstroke}%
\pgfsetdash{}{0pt}%
\pgfpathmoveto{\pgfqpoint{2.480238in}{2.782402in}}%
\pgfpathlineto{\pgfqpoint{2.480238in}{2.855405in}}%
\pgfpathlineto{\pgfqpoint{2.558191in}{2.818809in}}%
\pgfpathlineto{\pgfqpoint{2.558191in}{2.745806in}}%
\pgfpathlineto{\pgfqpoint{2.480238in}{2.782402in}}%
\pgfpathclose%
\pgfusepath{fill}%
\end{pgfscope}%
\begin{pgfscope}%
\pgfpathrectangle{\pgfqpoint{0.765000in}{0.660000in}}{\pgfqpoint{4.620000in}{4.620000in}}%
\pgfusepath{clip}%
\pgfsetbuttcap%
\pgfsetroundjoin%
\definecolor{currentfill}{rgb}{1.000000,0.576471,0.309804}%
\pgfsetfillcolor{currentfill}%
\pgfsetfillopacity{0.050000}%
\pgfsetlinewidth{0.000000pt}%
\definecolor{currentstroke}{rgb}{1.000000,0.576471,0.309804}%
\pgfsetstrokecolor{currentstroke}%
\pgfsetdash{}{0pt}%
\pgfpathmoveto{\pgfqpoint{2.636143in}{2.782402in}}%
\pgfpathlineto{\pgfqpoint{2.636143in}{2.855405in}}%
\pgfpathlineto{\pgfqpoint{2.558191in}{2.818809in}}%
\pgfpathlineto{\pgfqpoint{2.558191in}{2.745806in}}%
\pgfpathlineto{\pgfqpoint{2.636143in}{2.782402in}}%
\pgfpathclose%
\pgfusepath{fill}%
\end{pgfscope}%
\begin{pgfscope}%
\pgfpathrectangle{\pgfqpoint{0.765000in}{0.660000in}}{\pgfqpoint{4.620000in}{4.620000in}}%
\pgfusepath{clip}%
\pgfsetbuttcap%
\pgfsetroundjoin%
\definecolor{currentfill}{rgb}{1.000000,0.576471,0.309804}%
\pgfsetfillcolor{currentfill}%
\pgfsetfillopacity{0.050000}%
\pgfsetlinewidth{0.000000pt}%
\definecolor{currentstroke}{rgb}{1.000000,0.576471,0.309804}%
\pgfsetstrokecolor{currentstroke}%
\pgfsetdash{}{0pt}%
\pgfpathmoveto{\pgfqpoint{3.718611in}{2.633162in}}%
\pgfpathlineto{\pgfqpoint{3.718611in}{2.706165in}}%
\pgfpathlineto{\pgfqpoint{3.796564in}{2.669569in}}%
\pgfpathlineto{\pgfqpoint{3.796564in}{2.596565in}}%
\pgfpathlineto{\pgfqpoint{3.718611in}{2.633162in}}%
\pgfpathclose%
\pgfusepath{fill}%
\end{pgfscope}%
\begin{pgfscope}%
\pgfpathrectangle{\pgfqpoint{0.765000in}{0.660000in}}{\pgfqpoint{4.620000in}{4.620000in}}%
\pgfusepath{clip}%
\pgfsetbuttcap%
\pgfsetroundjoin%
\definecolor{currentfill}{rgb}{1.000000,0.576471,0.309804}%
\pgfsetfillcolor{currentfill}%
\pgfsetfillopacity{0.050000}%
\pgfsetlinewidth{0.000000pt}%
\definecolor{currentstroke}{rgb}{1.000000,0.576471,0.309804}%
\pgfsetstrokecolor{currentstroke}%
\pgfsetdash{}{0pt}%
\pgfpathmoveto{\pgfqpoint{3.718611in}{2.633162in}}%
\pgfpathlineto{\pgfqpoint{3.718611in}{2.706165in}}%
\pgfpathlineto{\pgfqpoint{3.640659in}{2.669569in}}%
\pgfpathlineto{\pgfqpoint{3.640659in}{2.596565in}}%
\pgfpathlineto{\pgfqpoint{3.718611in}{2.633162in}}%
\pgfpathclose%
\pgfusepath{fill}%
\end{pgfscope}%
\begin{pgfscope}%
\pgfpathrectangle{\pgfqpoint{0.765000in}{0.660000in}}{\pgfqpoint{4.620000in}{4.620000in}}%
\pgfusepath{clip}%
\pgfsetbuttcap%
\pgfsetroundjoin%
\definecolor{currentfill}{rgb}{1.000000,0.576471,0.309804}%
\pgfsetfillcolor{currentfill}%
\pgfsetfillopacity{0.050000}%
\pgfsetlinewidth{0.000000pt}%
\definecolor{currentstroke}{rgb}{1.000000,0.576471,0.309804}%
\pgfsetstrokecolor{currentstroke}%
\pgfsetdash{}{0pt}%
\pgfpathmoveto{\pgfqpoint{3.718611in}{2.633162in}}%
\pgfpathlineto{\pgfqpoint{3.796564in}{2.596565in}}%
\pgfpathlineto{\pgfqpoint{3.718611in}{2.559969in}}%
\pgfpathlineto{\pgfqpoint{3.640659in}{2.596565in}}%
\pgfpathlineto{\pgfqpoint{3.718611in}{2.633162in}}%
\pgfpathclose%
\pgfusepath{fill}%
\end{pgfscope}%
\begin{pgfscope}%
\pgfpathrectangle{\pgfqpoint{0.765000in}{0.660000in}}{\pgfqpoint{4.620000in}{4.620000in}}%
\pgfusepath{clip}%
\pgfsetbuttcap%
\pgfsetroundjoin%
\definecolor{currentfill}{rgb}{1.000000,0.576471,0.309804}%
\pgfsetfillcolor{currentfill}%
\pgfsetfillopacity{0.050000}%
\pgfsetlinewidth{0.000000pt}%
\definecolor{currentstroke}{rgb}{1.000000,0.576471,0.309804}%
\pgfsetstrokecolor{currentstroke}%
\pgfsetdash{}{0pt}%
\pgfpathmoveto{\pgfqpoint{3.718611in}{2.706165in}}%
\pgfpathlineto{\pgfqpoint{3.796564in}{2.669569in}}%
\pgfpathlineto{\pgfqpoint{3.718611in}{2.632972in}}%
\pgfpathlineto{\pgfqpoint{3.640659in}{2.669569in}}%
\pgfpathlineto{\pgfqpoint{3.718611in}{2.706165in}}%
\pgfpathclose%
\pgfusepath{fill}%
\end{pgfscope}%
\begin{pgfscope}%
\pgfpathrectangle{\pgfqpoint{0.765000in}{0.660000in}}{\pgfqpoint{4.620000in}{4.620000in}}%
\pgfusepath{clip}%
\pgfsetbuttcap%
\pgfsetroundjoin%
\definecolor{currentfill}{rgb}{1.000000,0.576471,0.309804}%
\pgfsetfillcolor{currentfill}%
\pgfsetfillopacity{0.050000}%
\pgfsetlinewidth{0.000000pt}%
\definecolor{currentstroke}{rgb}{1.000000,0.576471,0.309804}%
\pgfsetstrokecolor{currentstroke}%
\pgfsetdash{}{0pt}%
\pgfpathmoveto{\pgfqpoint{3.640659in}{2.596565in}}%
\pgfpathlineto{\pgfqpoint{3.640659in}{2.669569in}}%
\pgfpathlineto{\pgfqpoint{3.718611in}{2.632972in}}%
\pgfpathlineto{\pgfqpoint{3.718611in}{2.559969in}}%
\pgfpathlineto{\pgfqpoint{3.640659in}{2.596565in}}%
\pgfpathclose%
\pgfusepath{fill}%
\end{pgfscope}%
\begin{pgfscope}%
\pgfpathrectangle{\pgfqpoint{0.765000in}{0.660000in}}{\pgfqpoint{4.620000in}{4.620000in}}%
\pgfusepath{clip}%
\pgfsetbuttcap%
\pgfsetroundjoin%
\definecolor{currentfill}{rgb}{1.000000,0.576471,0.309804}%
\pgfsetfillcolor{currentfill}%
\pgfsetfillopacity{0.050000}%
\pgfsetlinewidth{0.000000pt}%
\definecolor{currentstroke}{rgb}{1.000000,0.576471,0.309804}%
\pgfsetstrokecolor{currentstroke}%
\pgfsetdash{}{0pt}%
\pgfpathmoveto{\pgfqpoint{3.796564in}{2.596565in}}%
\pgfpathlineto{\pgfqpoint{3.796564in}{2.669569in}}%
\pgfpathlineto{\pgfqpoint{3.718611in}{2.632972in}}%
\pgfpathlineto{\pgfqpoint{3.718611in}{2.559969in}}%
\pgfpathlineto{\pgfqpoint{3.796564in}{2.596565in}}%
\pgfpathclose%
\pgfusepath{fill}%
\end{pgfscope}%
\begin{pgfscope}%
\pgfpathrectangle{\pgfqpoint{0.765000in}{0.660000in}}{\pgfqpoint{4.620000in}{4.620000in}}%
\pgfusepath{clip}%
\pgfsetbuttcap%
\pgfsetroundjoin%
\definecolor{currentfill}{rgb}{1.000000,0.576471,0.309804}%
\pgfsetfillcolor{currentfill}%
\pgfsetfillopacity{0.050000}%
\pgfsetlinewidth{0.000000pt}%
\definecolor{currentstroke}{rgb}{1.000000,0.576471,0.309804}%
\pgfsetstrokecolor{currentstroke}%
\pgfsetdash{}{0pt}%
\pgfpathmoveto{\pgfqpoint{1.927470in}{3.089412in}}%
\pgfpathlineto{\pgfqpoint{1.927470in}{3.162416in}}%
\pgfpathlineto{\pgfqpoint{2.005423in}{3.125819in}}%
\pgfpathlineto{\pgfqpoint{2.005423in}{3.052816in}}%
\pgfpathlineto{\pgfqpoint{1.927470in}{3.089412in}}%
\pgfpathclose%
\pgfusepath{fill}%
\end{pgfscope}%
\begin{pgfscope}%
\pgfpathrectangle{\pgfqpoint{0.765000in}{0.660000in}}{\pgfqpoint{4.620000in}{4.620000in}}%
\pgfusepath{clip}%
\pgfsetbuttcap%
\pgfsetroundjoin%
\definecolor{currentfill}{rgb}{1.000000,0.576471,0.309804}%
\pgfsetfillcolor{currentfill}%
\pgfsetfillopacity{0.050000}%
\pgfsetlinewidth{0.000000pt}%
\definecolor{currentstroke}{rgb}{1.000000,0.576471,0.309804}%
\pgfsetstrokecolor{currentstroke}%
\pgfsetdash{}{0pt}%
\pgfpathmoveto{\pgfqpoint{1.927470in}{3.089412in}}%
\pgfpathlineto{\pgfqpoint{1.927470in}{3.162416in}}%
\pgfpathlineto{\pgfqpoint{1.849517in}{3.125819in}}%
\pgfpathlineto{\pgfqpoint{1.849517in}{3.052816in}}%
\pgfpathlineto{\pgfqpoint{1.927470in}{3.089412in}}%
\pgfpathclose%
\pgfusepath{fill}%
\end{pgfscope}%
\begin{pgfscope}%
\pgfpathrectangle{\pgfqpoint{0.765000in}{0.660000in}}{\pgfqpoint{4.620000in}{4.620000in}}%
\pgfusepath{clip}%
\pgfsetbuttcap%
\pgfsetroundjoin%
\definecolor{currentfill}{rgb}{1.000000,0.576471,0.309804}%
\pgfsetfillcolor{currentfill}%
\pgfsetfillopacity{0.050000}%
\pgfsetlinewidth{0.000000pt}%
\definecolor{currentstroke}{rgb}{1.000000,0.576471,0.309804}%
\pgfsetstrokecolor{currentstroke}%
\pgfsetdash{}{0pt}%
\pgfpathmoveto{\pgfqpoint{1.927470in}{3.089412in}}%
\pgfpathlineto{\pgfqpoint{2.005423in}{3.052816in}}%
\pgfpathlineto{\pgfqpoint{1.927470in}{3.016219in}}%
\pgfpathlineto{\pgfqpoint{1.849517in}{3.052816in}}%
\pgfpathlineto{\pgfqpoint{1.927470in}{3.089412in}}%
\pgfpathclose%
\pgfusepath{fill}%
\end{pgfscope}%
\begin{pgfscope}%
\pgfpathrectangle{\pgfqpoint{0.765000in}{0.660000in}}{\pgfqpoint{4.620000in}{4.620000in}}%
\pgfusepath{clip}%
\pgfsetbuttcap%
\pgfsetroundjoin%
\definecolor{currentfill}{rgb}{1.000000,0.576471,0.309804}%
\pgfsetfillcolor{currentfill}%
\pgfsetfillopacity{0.050000}%
\pgfsetlinewidth{0.000000pt}%
\definecolor{currentstroke}{rgb}{1.000000,0.576471,0.309804}%
\pgfsetstrokecolor{currentstroke}%
\pgfsetdash{}{0pt}%
\pgfpathmoveto{\pgfqpoint{1.927470in}{3.162416in}}%
\pgfpathlineto{\pgfqpoint{2.005423in}{3.125819in}}%
\pgfpathlineto{\pgfqpoint{1.927470in}{3.089222in}}%
\pgfpathlineto{\pgfqpoint{1.849517in}{3.125819in}}%
\pgfpathlineto{\pgfqpoint{1.927470in}{3.162416in}}%
\pgfpathclose%
\pgfusepath{fill}%
\end{pgfscope}%
\begin{pgfscope}%
\pgfpathrectangle{\pgfqpoint{0.765000in}{0.660000in}}{\pgfqpoint{4.620000in}{4.620000in}}%
\pgfusepath{clip}%
\pgfsetbuttcap%
\pgfsetroundjoin%
\definecolor{currentfill}{rgb}{1.000000,0.576471,0.309804}%
\pgfsetfillcolor{currentfill}%
\pgfsetfillopacity{0.050000}%
\pgfsetlinewidth{0.000000pt}%
\definecolor{currentstroke}{rgb}{1.000000,0.576471,0.309804}%
\pgfsetstrokecolor{currentstroke}%
\pgfsetdash{}{0pt}%
\pgfpathmoveto{\pgfqpoint{1.849517in}{3.052816in}}%
\pgfpathlineto{\pgfqpoint{1.849517in}{3.125819in}}%
\pgfpathlineto{\pgfqpoint{1.927470in}{3.089222in}}%
\pgfpathlineto{\pgfqpoint{1.927470in}{3.016219in}}%
\pgfpathlineto{\pgfqpoint{1.849517in}{3.052816in}}%
\pgfpathclose%
\pgfusepath{fill}%
\end{pgfscope}%
\begin{pgfscope}%
\pgfpathrectangle{\pgfqpoint{0.765000in}{0.660000in}}{\pgfqpoint{4.620000in}{4.620000in}}%
\pgfusepath{clip}%
\pgfsetbuttcap%
\pgfsetroundjoin%
\definecolor{currentfill}{rgb}{1.000000,0.576471,0.309804}%
\pgfsetfillcolor{currentfill}%
\pgfsetfillopacity{0.050000}%
\pgfsetlinewidth{0.000000pt}%
\definecolor{currentstroke}{rgb}{1.000000,0.576471,0.309804}%
\pgfsetstrokecolor{currentstroke}%
\pgfsetdash{}{0pt}%
\pgfpathmoveto{\pgfqpoint{2.005423in}{3.052816in}}%
\pgfpathlineto{\pgfqpoint{2.005423in}{3.125819in}}%
\pgfpathlineto{\pgfqpoint{1.927470in}{3.089222in}}%
\pgfpathlineto{\pgfqpoint{1.927470in}{3.016219in}}%
\pgfpathlineto{\pgfqpoint{2.005423in}{3.052816in}}%
\pgfpathclose%
\pgfusepath{fill}%
\end{pgfscope}%
\begin{pgfscope}%
\pgfpathrectangle{\pgfqpoint{0.765000in}{0.660000in}}{\pgfqpoint{4.620000in}{4.620000in}}%
\pgfusepath{clip}%
\pgfsetbuttcap%
\pgfsetroundjoin%
\definecolor{currentfill}{rgb}{1.000000,0.576471,0.309804}%
\pgfsetfillcolor{currentfill}%
\pgfsetfillopacity{0.050000}%
\pgfsetlinewidth{0.000000pt}%
\definecolor{currentstroke}{rgb}{1.000000,0.576471,0.309804}%
\pgfsetstrokecolor{currentstroke}%
\pgfsetdash{}{0pt}%
\pgfpathmoveto{\pgfqpoint{3.766684in}{2.840954in}}%
\pgfpathlineto{\pgfqpoint{3.766684in}{2.913957in}}%
\pgfpathlineto{\pgfqpoint{3.844637in}{2.877361in}}%
\pgfpathlineto{\pgfqpoint{3.844637in}{2.804358in}}%
\pgfpathlineto{\pgfqpoint{3.766684in}{2.840954in}}%
\pgfpathclose%
\pgfusepath{fill}%
\end{pgfscope}%
\begin{pgfscope}%
\pgfpathrectangle{\pgfqpoint{0.765000in}{0.660000in}}{\pgfqpoint{4.620000in}{4.620000in}}%
\pgfusepath{clip}%
\pgfsetbuttcap%
\pgfsetroundjoin%
\definecolor{currentfill}{rgb}{1.000000,0.576471,0.309804}%
\pgfsetfillcolor{currentfill}%
\pgfsetfillopacity{0.050000}%
\pgfsetlinewidth{0.000000pt}%
\definecolor{currentstroke}{rgb}{1.000000,0.576471,0.309804}%
\pgfsetstrokecolor{currentstroke}%
\pgfsetdash{}{0pt}%
\pgfpathmoveto{\pgfqpoint{3.766684in}{2.840954in}}%
\pgfpathlineto{\pgfqpoint{3.766684in}{2.913957in}}%
\pgfpathlineto{\pgfqpoint{3.688732in}{2.877361in}}%
\pgfpathlineto{\pgfqpoint{3.688732in}{2.804358in}}%
\pgfpathlineto{\pgfqpoint{3.766684in}{2.840954in}}%
\pgfpathclose%
\pgfusepath{fill}%
\end{pgfscope}%
\begin{pgfscope}%
\pgfpathrectangle{\pgfqpoint{0.765000in}{0.660000in}}{\pgfqpoint{4.620000in}{4.620000in}}%
\pgfusepath{clip}%
\pgfsetbuttcap%
\pgfsetroundjoin%
\definecolor{currentfill}{rgb}{1.000000,0.576471,0.309804}%
\pgfsetfillcolor{currentfill}%
\pgfsetfillopacity{0.050000}%
\pgfsetlinewidth{0.000000pt}%
\definecolor{currentstroke}{rgb}{1.000000,0.576471,0.309804}%
\pgfsetstrokecolor{currentstroke}%
\pgfsetdash{}{0pt}%
\pgfpathmoveto{\pgfqpoint{3.766684in}{2.840954in}}%
\pgfpathlineto{\pgfqpoint{3.844637in}{2.804358in}}%
\pgfpathlineto{\pgfqpoint{3.766684in}{2.767761in}}%
\pgfpathlineto{\pgfqpoint{3.688732in}{2.804358in}}%
\pgfpathlineto{\pgfqpoint{3.766684in}{2.840954in}}%
\pgfpathclose%
\pgfusepath{fill}%
\end{pgfscope}%
\begin{pgfscope}%
\pgfpathrectangle{\pgfqpoint{0.765000in}{0.660000in}}{\pgfqpoint{4.620000in}{4.620000in}}%
\pgfusepath{clip}%
\pgfsetbuttcap%
\pgfsetroundjoin%
\definecolor{currentfill}{rgb}{1.000000,0.576471,0.309804}%
\pgfsetfillcolor{currentfill}%
\pgfsetfillopacity{0.050000}%
\pgfsetlinewidth{0.000000pt}%
\definecolor{currentstroke}{rgb}{1.000000,0.576471,0.309804}%
\pgfsetstrokecolor{currentstroke}%
\pgfsetdash{}{0pt}%
\pgfpathmoveto{\pgfqpoint{3.766684in}{2.913957in}}%
\pgfpathlineto{\pgfqpoint{3.844637in}{2.877361in}}%
\pgfpathlineto{\pgfqpoint{3.766684in}{2.840764in}}%
\pgfpathlineto{\pgfqpoint{3.688732in}{2.877361in}}%
\pgfpathlineto{\pgfqpoint{3.766684in}{2.913957in}}%
\pgfpathclose%
\pgfusepath{fill}%
\end{pgfscope}%
\begin{pgfscope}%
\pgfpathrectangle{\pgfqpoint{0.765000in}{0.660000in}}{\pgfqpoint{4.620000in}{4.620000in}}%
\pgfusepath{clip}%
\pgfsetbuttcap%
\pgfsetroundjoin%
\definecolor{currentfill}{rgb}{1.000000,0.576471,0.309804}%
\pgfsetfillcolor{currentfill}%
\pgfsetfillopacity{0.050000}%
\pgfsetlinewidth{0.000000pt}%
\definecolor{currentstroke}{rgb}{1.000000,0.576471,0.309804}%
\pgfsetstrokecolor{currentstroke}%
\pgfsetdash{}{0pt}%
\pgfpathmoveto{\pgfqpoint{3.688732in}{2.804358in}}%
\pgfpathlineto{\pgfqpoint{3.688732in}{2.877361in}}%
\pgfpathlineto{\pgfqpoint{3.766684in}{2.840764in}}%
\pgfpathlineto{\pgfqpoint{3.766684in}{2.767761in}}%
\pgfpathlineto{\pgfqpoint{3.688732in}{2.804358in}}%
\pgfpathclose%
\pgfusepath{fill}%
\end{pgfscope}%
\begin{pgfscope}%
\pgfpathrectangle{\pgfqpoint{0.765000in}{0.660000in}}{\pgfqpoint{4.620000in}{4.620000in}}%
\pgfusepath{clip}%
\pgfsetbuttcap%
\pgfsetroundjoin%
\definecolor{currentfill}{rgb}{1.000000,0.576471,0.309804}%
\pgfsetfillcolor{currentfill}%
\pgfsetfillopacity{0.050000}%
\pgfsetlinewidth{0.000000pt}%
\definecolor{currentstroke}{rgb}{1.000000,0.576471,0.309804}%
\pgfsetstrokecolor{currentstroke}%
\pgfsetdash{}{0pt}%
\pgfpathmoveto{\pgfqpoint{3.844637in}{2.804358in}}%
\pgfpathlineto{\pgfqpoint{3.844637in}{2.877361in}}%
\pgfpathlineto{\pgfqpoint{3.766684in}{2.840764in}}%
\pgfpathlineto{\pgfqpoint{3.766684in}{2.767761in}}%
\pgfpathlineto{\pgfqpoint{3.844637in}{2.804358in}}%
\pgfpathclose%
\pgfusepath{fill}%
\end{pgfscope}%
\begin{pgfscope}%
\pgfpathrectangle{\pgfqpoint{0.765000in}{0.660000in}}{\pgfqpoint{4.620000in}{4.620000in}}%
\pgfusepath{clip}%
\pgfsetbuttcap%
\pgfsetroundjoin%
\definecolor{currentfill}{rgb}{1.000000,0.576471,0.309804}%
\pgfsetfillcolor{currentfill}%
\pgfsetfillopacity{0.050000}%
\pgfsetlinewidth{0.000000pt}%
\definecolor{currentstroke}{rgb}{1.000000,0.576471,0.309804}%
\pgfsetstrokecolor{currentstroke}%
\pgfsetdash{}{0pt}%
\pgfpathmoveto{\pgfqpoint{2.849491in}{2.575944in}}%
\pgfpathlineto{\pgfqpoint{2.849491in}{2.648947in}}%
\pgfpathlineto{\pgfqpoint{2.927443in}{2.612351in}}%
\pgfpathlineto{\pgfqpoint{2.927443in}{2.539347in}}%
\pgfpathlineto{\pgfqpoint{2.849491in}{2.575944in}}%
\pgfpathclose%
\pgfusepath{fill}%
\end{pgfscope}%
\begin{pgfscope}%
\pgfpathrectangle{\pgfqpoint{0.765000in}{0.660000in}}{\pgfqpoint{4.620000in}{4.620000in}}%
\pgfusepath{clip}%
\pgfsetbuttcap%
\pgfsetroundjoin%
\definecolor{currentfill}{rgb}{1.000000,0.576471,0.309804}%
\pgfsetfillcolor{currentfill}%
\pgfsetfillopacity{0.050000}%
\pgfsetlinewidth{0.000000pt}%
\definecolor{currentstroke}{rgb}{1.000000,0.576471,0.309804}%
\pgfsetstrokecolor{currentstroke}%
\pgfsetdash{}{0pt}%
\pgfpathmoveto{\pgfqpoint{2.849491in}{2.575944in}}%
\pgfpathlineto{\pgfqpoint{2.849491in}{2.648947in}}%
\pgfpathlineto{\pgfqpoint{2.771538in}{2.612351in}}%
\pgfpathlineto{\pgfqpoint{2.771538in}{2.539347in}}%
\pgfpathlineto{\pgfqpoint{2.849491in}{2.575944in}}%
\pgfpathclose%
\pgfusepath{fill}%
\end{pgfscope}%
\begin{pgfscope}%
\pgfpathrectangle{\pgfqpoint{0.765000in}{0.660000in}}{\pgfqpoint{4.620000in}{4.620000in}}%
\pgfusepath{clip}%
\pgfsetbuttcap%
\pgfsetroundjoin%
\definecolor{currentfill}{rgb}{1.000000,0.576471,0.309804}%
\pgfsetfillcolor{currentfill}%
\pgfsetfillopacity{0.050000}%
\pgfsetlinewidth{0.000000pt}%
\definecolor{currentstroke}{rgb}{1.000000,0.576471,0.309804}%
\pgfsetstrokecolor{currentstroke}%
\pgfsetdash{}{0pt}%
\pgfpathmoveto{\pgfqpoint{2.849491in}{2.575944in}}%
\pgfpathlineto{\pgfqpoint{2.927443in}{2.539347in}}%
\pgfpathlineto{\pgfqpoint{2.849491in}{2.502751in}}%
\pgfpathlineto{\pgfqpoint{2.771538in}{2.539347in}}%
\pgfpathlineto{\pgfqpoint{2.849491in}{2.575944in}}%
\pgfpathclose%
\pgfusepath{fill}%
\end{pgfscope}%
\begin{pgfscope}%
\pgfpathrectangle{\pgfqpoint{0.765000in}{0.660000in}}{\pgfqpoint{4.620000in}{4.620000in}}%
\pgfusepath{clip}%
\pgfsetbuttcap%
\pgfsetroundjoin%
\definecolor{currentfill}{rgb}{1.000000,0.576471,0.309804}%
\pgfsetfillcolor{currentfill}%
\pgfsetfillopacity{0.050000}%
\pgfsetlinewidth{0.000000pt}%
\definecolor{currentstroke}{rgb}{1.000000,0.576471,0.309804}%
\pgfsetstrokecolor{currentstroke}%
\pgfsetdash{}{0pt}%
\pgfpathmoveto{\pgfqpoint{2.849491in}{2.648947in}}%
\pgfpathlineto{\pgfqpoint{2.927443in}{2.612351in}}%
\pgfpathlineto{\pgfqpoint{2.849491in}{2.575754in}}%
\pgfpathlineto{\pgfqpoint{2.771538in}{2.612351in}}%
\pgfpathlineto{\pgfqpoint{2.849491in}{2.648947in}}%
\pgfpathclose%
\pgfusepath{fill}%
\end{pgfscope}%
\begin{pgfscope}%
\pgfpathrectangle{\pgfqpoint{0.765000in}{0.660000in}}{\pgfqpoint{4.620000in}{4.620000in}}%
\pgfusepath{clip}%
\pgfsetbuttcap%
\pgfsetroundjoin%
\definecolor{currentfill}{rgb}{1.000000,0.576471,0.309804}%
\pgfsetfillcolor{currentfill}%
\pgfsetfillopacity{0.050000}%
\pgfsetlinewidth{0.000000pt}%
\definecolor{currentstroke}{rgb}{1.000000,0.576471,0.309804}%
\pgfsetstrokecolor{currentstroke}%
\pgfsetdash{}{0pt}%
\pgfpathmoveto{\pgfqpoint{2.771538in}{2.539347in}}%
\pgfpathlineto{\pgfqpoint{2.771538in}{2.612351in}}%
\pgfpathlineto{\pgfqpoint{2.849491in}{2.575754in}}%
\pgfpathlineto{\pgfqpoint{2.849491in}{2.502751in}}%
\pgfpathlineto{\pgfqpoint{2.771538in}{2.539347in}}%
\pgfpathclose%
\pgfusepath{fill}%
\end{pgfscope}%
\begin{pgfscope}%
\pgfpathrectangle{\pgfqpoint{0.765000in}{0.660000in}}{\pgfqpoint{4.620000in}{4.620000in}}%
\pgfusepath{clip}%
\pgfsetbuttcap%
\pgfsetroundjoin%
\definecolor{currentfill}{rgb}{1.000000,0.576471,0.309804}%
\pgfsetfillcolor{currentfill}%
\pgfsetfillopacity{0.050000}%
\pgfsetlinewidth{0.000000pt}%
\definecolor{currentstroke}{rgb}{1.000000,0.576471,0.309804}%
\pgfsetstrokecolor{currentstroke}%
\pgfsetdash{}{0pt}%
\pgfpathmoveto{\pgfqpoint{2.927443in}{2.539347in}}%
\pgfpathlineto{\pgfqpoint{2.927443in}{2.612351in}}%
\pgfpathlineto{\pgfqpoint{2.849491in}{2.575754in}}%
\pgfpathlineto{\pgfqpoint{2.849491in}{2.502751in}}%
\pgfpathlineto{\pgfqpoint{2.927443in}{2.539347in}}%
\pgfpathclose%
\pgfusepath{fill}%
\end{pgfscope}%
\begin{pgfscope}%
\pgfpathrectangle{\pgfqpoint{0.765000in}{0.660000in}}{\pgfqpoint{4.620000in}{4.620000in}}%
\pgfusepath{clip}%
\pgfsetbuttcap%
\pgfsetroundjoin%
\definecolor{currentfill}{rgb}{1.000000,0.576471,0.309804}%
\pgfsetfillcolor{currentfill}%
\pgfsetfillopacity{0.050000}%
\pgfsetlinewidth{0.000000pt}%
\definecolor{currentstroke}{rgb}{1.000000,0.576471,0.309804}%
\pgfsetstrokecolor{currentstroke}%
\pgfsetdash{}{0pt}%
\pgfpathmoveto{\pgfqpoint{3.355887in}{2.348341in}}%
\pgfpathlineto{\pgfqpoint{3.355887in}{2.421344in}}%
\pgfpathlineto{\pgfqpoint{3.433840in}{2.384747in}}%
\pgfpathlineto{\pgfqpoint{3.433840in}{2.311744in}}%
\pgfpathlineto{\pgfqpoint{3.355887in}{2.348341in}}%
\pgfpathclose%
\pgfusepath{fill}%
\end{pgfscope}%
\begin{pgfscope}%
\pgfpathrectangle{\pgfqpoint{0.765000in}{0.660000in}}{\pgfqpoint{4.620000in}{4.620000in}}%
\pgfusepath{clip}%
\pgfsetbuttcap%
\pgfsetroundjoin%
\definecolor{currentfill}{rgb}{1.000000,0.576471,0.309804}%
\pgfsetfillcolor{currentfill}%
\pgfsetfillopacity{0.050000}%
\pgfsetlinewidth{0.000000pt}%
\definecolor{currentstroke}{rgb}{1.000000,0.576471,0.309804}%
\pgfsetstrokecolor{currentstroke}%
\pgfsetdash{}{0pt}%
\pgfpathmoveto{\pgfqpoint{3.355887in}{2.348341in}}%
\pgfpathlineto{\pgfqpoint{3.355887in}{2.421344in}}%
\pgfpathlineto{\pgfqpoint{3.277935in}{2.384747in}}%
\pgfpathlineto{\pgfqpoint{3.277935in}{2.311744in}}%
\pgfpathlineto{\pgfqpoint{3.355887in}{2.348341in}}%
\pgfpathclose%
\pgfusepath{fill}%
\end{pgfscope}%
\begin{pgfscope}%
\pgfpathrectangle{\pgfqpoint{0.765000in}{0.660000in}}{\pgfqpoint{4.620000in}{4.620000in}}%
\pgfusepath{clip}%
\pgfsetbuttcap%
\pgfsetroundjoin%
\definecolor{currentfill}{rgb}{1.000000,0.576471,0.309804}%
\pgfsetfillcolor{currentfill}%
\pgfsetfillopacity{0.050000}%
\pgfsetlinewidth{0.000000pt}%
\definecolor{currentstroke}{rgb}{1.000000,0.576471,0.309804}%
\pgfsetstrokecolor{currentstroke}%
\pgfsetdash{}{0pt}%
\pgfpathmoveto{\pgfqpoint{3.355887in}{2.348341in}}%
\pgfpathlineto{\pgfqpoint{3.433840in}{2.311744in}}%
\pgfpathlineto{\pgfqpoint{3.355887in}{2.275148in}}%
\pgfpathlineto{\pgfqpoint{3.277935in}{2.311744in}}%
\pgfpathlineto{\pgfqpoint{3.355887in}{2.348341in}}%
\pgfpathclose%
\pgfusepath{fill}%
\end{pgfscope}%
\begin{pgfscope}%
\pgfpathrectangle{\pgfqpoint{0.765000in}{0.660000in}}{\pgfqpoint{4.620000in}{4.620000in}}%
\pgfusepath{clip}%
\pgfsetbuttcap%
\pgfsetroundjoin%
\definecolor{currentfill}{rgb}{1.000000,0.576471,0.309804}%
\pgfsetfillcolor{currentfill}%
\pgfsetfillopacity{0.050000}%
\pgfsetlinewidth{0.000000pt}%
\definecolor{currentstroke}{rgb}{1.000000,0.576471,0.309804}%
\pgfsetstrokecolor{currentstroke}%
\pgfsetdash{}{0pt}%
\pgfpathmoveto{\pgfqpoint{3.355887in}{2.421344in}}%
\pgfpathlineto{\pgfqpoint{3.433840in}{2.384747in}}%
\pgfpathlineto{\pgfqpoint{3.355887in}{2.348151in}}%
\pgfpathlineto{\pgfqpoint{3.277935in}{2.384747in}}%
\pgfpathlineto{\pgfqpoint{3.355887in}{2.421344in}}%
\pgfpathclose%
\pgfusepath{fill}%
\end{pgfscope}%
\begin{pgfscope}%
\pgfpathrectangle{\pgfqpoint{0.765000in}{0.660000in}}{\pgfqpoint{4.620000in}{4.620000in}}%
\pgfusepath{clip}%
\pgfsetbuttcap%
\pgfsetroundjoin%
\definecolor{currentfill}{rgb}{1.000000,0.576471,0.309804}%
\pgfsetfillcolor{currentfill}%
\pgfsetfillopacity{0.050000}%
\pgfsetlinewidth{0.000000pt}%
\definecolor{currentstroke}{rgb}{1.000000,0.576471,0.309804}%
\pgfsetstrokecolor{currentstroke}%
\pgfsetdash{}{0pt}%
\pgfpathmoveto{\pgfqpoint{3.277935in}{2.311744in}}%
\pgfpathlineto{\pgfqpoint{3.277935in}{2.384747in}}%
\pgfpathlineto{\pgfqpoint{3.355887in}{2.348151in}}%
\pgfpathlineto{\pgfqpoint{3.355887in}{2.275148in}}%
\pgfpathlineto{\pgfqpoint{3.277935in}{2.311744in}}%
\pgfpathclose%
\pgfusepath{fill}%
\end{pgfscope}%
\begin{pgfscope}%
\pgfpathrectangle{\pgfqpoint{0.765000in}{0.660000in}}{\pgfqpoint{4.620000in}{4.620000in}}%
\pgfusepath{clip}%
\pgfsetbuttcap%
\pgfsetroundjoin%
\definecolor{currentfill}{rgb}{1.000000,0.576471,0.309804}%
\pgfsetfillcolor{currentfill}%
\pgfsetfillopacity{0.050000}%
\pgfsetlinewidth{0.000000pt}%
\definecolor{currentstroke}{rgb}{1.000000,0.576471,0.309804}%
\pgfsetstrokecolor{currentstroke}%
\pgfsetdash{}{0pt}%
\pgfpathmoveto{\pgfqpoint{3.433840in}{2.311744in}}%
\pgfpathlineto{\pgfqpoint{3.433840in}{2.384747in}}%
\pgfpathlineto{\pgfqpoint{3.355887in}{2.348151in}}%
\pgfpathlineto{\pgfqpoint{3.355887in}{2.275148in}}%
\pgfpathlineto{\pgfqpoint{3.433840in}{2.311744in}}%
\pgfpathclose%
\pgfusepath{fill}%
\end{pgfscope}%
\end{pgfpicture}%
\makeatother%
\endgroup%
}
        }
        \caption{Toy binary dataset, cubic}
        \label{fig:circular}
    \end{subfigure}
    \hfill
    \centering
    \begin{subfigure}{0.35\textwidth}
        \centering
        \resizebox{\linewidth}{!}{%
        \centering
            \clipbox{0.2\width{} 0.2\height{} 0.2\width{} 0.2\height{}}{%% Creator: Matplotlib, PGF backend
%%
%% To include the figure in your LaTeX document, write
%%   \input{<filename>.pgf}
%%
%% Make sure the required packages are loaded in your preamble
%%   \usepackage{pgf}
%%
%% Also ensure that all the required font packages are loaded; for instance,
%% the lmodern package is sometimes necessary when using math font.
%%   \usepackage{lmodern}
%%
%% Figures using additional raster images can only be included by \input if
%% they are in the same directory as the main LaTeX file. For loading figures
%% from other directories you can use the `import` package
%%   \usepackage{import}
%%
%% and then include the figures with
%%   \import{<path to file>}{<filename>.pgf}
%%
%% Matplotlib used the following preamble
%%   
%%   \usepackage{fontspec}
%%   \setmainfont{DejaVuSerif.ttf}[Path=\detokenize{/Users/sam/Library/Python/3.9/lib/python/site-packages/matplotlib/mpl-data/fonts/ttf/}]
%%   \setsansfont{DejaVuSans.ttf}[Path=\detokenize{/Users/sam/Library/Python/3.9/lib/python/site-packages/matplotlib/mpl-data/fonts/ttf/}]
%%   \setmonofont{DejaVuSansMono.ttf}[Path=\detokenize{/Users/sam/Library/Python/3.9/lib/python/site-packages/matplotlib/mpl-data/fonts/ttf/}]
%%   \makeatletter\@ifpackageloaded{underscore}{}{\usepackage[strings]{underscore}}\makeatother
%%
\begingroup%
\makeatletter%
\begin{pgfpicture}%
\pgfpathrectangle{\pgfpointorigin}{\pgfqpoint{6.000000in}{6.000000in}}%
\pgfusepath{use as bounding box, clip}%
\begin{pgfscope}%
\pgfsetbuttcap%
\pgfsetmiterjoin%
\definecolor{currentfill}{rgb}{1.000000,1.000000,1.000000}%
\pgfsetfillcolor{currentfill}%
\pgfsetlinewidth{0.000000pt}%
\definecolor{currentstroke}{rgb}{1.000000,1.000000,1.000000}%
\pgfsetstrokecolor{currentstroke}%
\pgfsetdash{}{0pt}%
\pgfpathmoveto{\pgfqpoint{0.000000in}{0.000000in}}%
\pgfpathlineto{\pgfqpoint{6.000000in}{0.000000in}}%
\pgfpathlineto{\pgfqpoint{6.000000in}{6.000000in}}%
\pgfpathlineto{\pgfqpoint{0.000000in}{6.000000in}}%
\pgfpathlineto{\pgfqpoint{0.000000in}{0.000000in}}%
\pgfpathclose%
\pgfusepath{fill}%
\end{pgfscope}%
\begin{pgfscope}%
\pgfsetbuttcap%
\pgfsetmiterjoin%
\definecolor{currentfill}{rgb}{1.000000,1.000000,1.000000}%
\pgfsetfillcolor{currentfill}%
\pgfsetlinewidth{0.000000pt}%
\definecolor{currentstroke}{rgb}{0.000000,0.000000,0.000000}%
\pgfsetstrokecolor{currentstroke}%
\pgfsetstrokeopacity{0.000000}%
\pgfsetdash{}{0pt}%
\pgfpathmoveto{\pgfqpoint{0.765000in}{0.660000in}}%
\pgfpathlineto{\pgfqpoint{5.385000in}{0.660000in}}%
\pgfpathlineto{\pgfqpoint{5.385000in}{5.280000in}}%
\pgfpathlineto{\pgfqpoint{0.765000in}{5.280000in}}%
\pgfpathlineto{\pgfqpoint{0.765000in}{0.660000in}}%
\pgfpathclose%
\pgfusepath{fill}%
\end{pgfscope}%
\begin{pgfscope}%
\pgfpathrectangle{\pgfqpoint{0.765000in}{0.660000in}}{\pgfqpoint{4.620000in}{4.620000in}}%
\pgfusepath{clip}%
\pgfsetbuttcap%
\pgfsetroundjoin%
\definecolor{currentfill}{rgb}{1.000000,0.894118,0.788235}%
\pgfsetfillcolor{currentfill}%
\pgfsetlinewidth{0.000000pt}%
\definecolor{currentstroke}{rgb}{1.000000,0.894118,0.788235}%
\pgfsetstrokecolor{currentstroke}%
\pgfsetdash{}{0pt}%
\pgfpathmoveto{\pgfqpoint{2.472452in}{2.522519in}}%
\pgfpathlineto{\pgfqpoint{2.346207in}{2.449631in}}%
\pgfpathlineto{\pgfqpoint{2.472452in}{2.376743in}}%
\pgfpathlineto{\pgfqpoint{2.598697in}{2.449631in}}%
\pgfpathlineto{\pgfqpoint{2.472452in}{2.522519in}}%
\pgfpathclose%
\pgfusepath{fill}%
\end{pgfscope}%
\begin{pgfscope}%
\pgfpathrectangle{\pgfqpoint{0.765000in}{0.660000in}}{\pgfqpoint{4.620000in}{4.620000in}}%
\pgfusepath{clip}%
\pgfsetbuttcap%
\pgfsetroundjoin%
\definecolor{currentfill}{rgb}{1.000000,0.894118,0.788235}%
\pgfsetfillcolor{currentfill}%
\pgfsetlinewidth{0.000000pt}%
\definecolor{currentstroke}{rgb}{1.000000,0.894118,0.788235}%
\pgfsetstrokecolor{currentstroke}%
\pgfsetdash{}{0pt}%
\pgfpathmoveto{\pgfqpoint{2.472452in}{2.522519in}}%
\pgfpathlineto{\pgfqpoint{2.346207in}{2.449631in}}%
\pgfpathlineto{\pgfqpoint{2.346207in}{2.595406in}}%
\pgfpathlineto{\pgfqpoint{2.472452in}{2.668294in}}%
\pgfpathlineto{\pgfqpoint{2.472452in}{2.522519in}}%
\pgfpathclose%
\pgfusepath{fill}%
\end{pgfscope}%
\begin{pgfscope}%
\pgfpathrectangle{\pgfqpoint{0.765000in}{0.660000in}}{\pgfqpoint{4.620000in}{4.620000in}}%
\pgfusepath{clip}%
\pgfsetbuttcap%
\pgfsetroundjoin%
\definecolor{currentfill}{rgb}{1.000000,0.894118,0.788235}%
\pgfsetfillcolor{currentfill}%
\pgfsetlinewidth{0.000000pt}%
\definecolor{currentstroke}{rgb}{1.000000,0.894118,0.788235}%
\pgfsetstrokecolor{currentstroke}%
\pgfsetdash{}{0pt}%
\pgfpathmoveto{\pgfqpoint{2.472452in}{2.522519in}}%
\pgfpathlineto{\pgfqpoint{2.598697in}{2.449631in}}%
\pgfpathlineto{\pgfqpoint{2.598697in}{2.595406in}}%
\pgfpathlineto{\pgfqpoint{2.472452in}{2.668294in}}%
\pgfpathlineto{\pgfqpoint{2.472452in}{2.522519in}}%
\pgfpathclose%
\pgfusepath{fill}%
\end{pgfscope}%
\begin{pgfscope}%
\pgfpathrectangle{\pgfqpoint{0.765000in}{0.660000in}}{\pgfqpoint{4.620000in}{4.620000in}}%
\pgfusepath{clip}%
\pgfsetbuttcap%
\pgfsetroundjoin%
\definecolor{currentfill}{rgb}{1.000000,0.894118,0.788235}%
\pgfsetfillcolor{currentfill}%
\pgfsetlinewidth{0.000000pt}%
\definecolor{currentstroke}{rgb}{1.000000,0.894118,0.788235}%
\pgfsetstrokecolor{currentstroke}%
\pgfsetdash{}{0pt}%
\pgfpathmoveto{\pgfqpoint{3.324897in}{2.030360in}}%
\pgfpathlineto{\pgfqpoint{3.198652in}{1.957472in}}%
\pgfpathlineto{\pgfqpoint{3.324897in}{1.884585in}}%
\pgfpathlineto{\pgfqpoint{3.451141in}{1.957472in}}%
\pgfpathlineto{\pgfqpoint{3.324897in}{2.030360in}}%
\pgfpathclose%
\pgfusepath{fill}%
\end{pgfscope}%
\begin{pgfscope}%
\pgfpathrectangle{\pgfqpoint{0.765000in}{0.660000in}}{\pgfqpoint{4.620000in}{4.620000in}}%
\pgfusepath{clip}%
\pgfsetbuttcap%
\pgfsetroundjoin%
\definecolor{currentfill}{rgb}{1.000000,0.894118,0.788235}%
\pgfsetfillcolor{currentfill}%
\pgfsetlinewidth{0.000000pt}%
\definecolor{currentstroke}{rgb}{1.000000,0.894118,0.788235}%
\pgfsetstrokecolor{currentstroke}%
\pgfsetdash{}{0pt}%
\pgfpathmoveto{\pgfqpoint{3.324897in}{2.030360in}}%
\pgfpathlineto{\pgfqpoint{3.198652in}{1.957472in}}%
\pgfpathlineto{\pgfqpoint{3.198652in}{2.103247in}}%
\pgfpathlineto{\pgfqpoint{3.324897in}{2.176135in}}%
\pgfpathlineto{\pgfqpoint{3.324897in}{2.030360in}}%
\pgfpathclose%
\pgfusepath{fill}%
\end{pgfscope}%
\begin{pgfscope}%
\pgfpathrectangle{\pgfqpoint{0.765000in}{0.660000in}}{\pgfqpoint{4.620000in}{4.620000in}}%
\pgfusepath{clip}%
\pgfsetbuttcap%
\pgfsetroundjoin%
\definecolor{currentfill}{rgb}{1.000000,0.894118,0.788235}%
\pgfsetfillcolor{currentfill}%
\pgfsetlinewidth{0.000000pt}%
\definecolor{currentstroke}{rgb}{1.000000,0.894118,0.788235}%
\pgfsetstrokecolor{currentstroke}%
\pgfsetdash{}{0pt}%
\pgfpathmoveto{\pgfqpoint{3.324897in}{2.030360in}}%
\pgfpathlineto{\pgfqpoint{3.451141in}{1.957472in}}%
\pgfpathlineto{\pgfqpoint{3.451141in}{2.103247in}}%
\pgfpathlineto{\pgfqpoint{3.324897in}{2.176135in}}%
\pgfpathlineto{\pgfqpoint{3.324897in}{2.030360in}}%
\pgfpathclose%
\pgfusepath{fill}%
\end{pgfscope}%
\begin{pgfscope}%
\pgfpathrectangle{\pgfqpoint{0.765000in}{0.660000in}}{\pgfqpoint{4.620000in}{4.620000in}}%
\pgfusepath{clip}%
\pgfsetbuttcap%
\pgfsetroundjoin%
\definecolor{currentfill}{rgb}{1.000000,0.894118,0.788235}%
\pgfsetfillcolor{currentfill}%
\pgfsetlinewidth{0.000000pt}%
\definecolor{currentstroke}{rgb}{1.000000,0.894118,0.788235}%
\pgfsetstrokecolor{currentstroke}%
\pgfsetdash{}{0pt}%
\pgfpathmoveto{\pgfqpoint{2.472452in}{2.668294in}}%
\pgfpathlineto{\pgfqpoint{2.346207in}{2.595406in}}%
\pgfpathlineto{\pgfqpoint{2.472452in}{2.522519in}}%
\pgfpathlineto{\pgfqpoint{2.598697in}{2.595406in}}%
\pgfpathlineto{\pgfqpoint{2.472452in}{2.668294in}}%
\pgfpathclose%
\pgfusepath{fill}%
\end{pgfscope}%
\begin{pgfscope}%
\pgfpathrectangle{\pgfqpoint{0.765000in}{0.660000in}}{\pgfqpoint{4.620000in}{4.620000in}}%
\pgfusepath{clip}%
\pgfsetbuttcap%
\pgfsetroundjoin%
\definecolor{currentfill}{rgb}{1.000000,0.894118,0.788235}%
\pgfsetfillcolor{currentfill}%
\pgfsetlinewidth{0.000000pt}%
\definecolor{currentstroke}{rgb}{1.000000,0.894118,0.788235}%
\pgfsetstrokecolor{currentstroke}%
\pgfsetdash{}{0pt}%
\pgfpathmoveto{\pgfqpoint{2.472452in}{2.376743in}}%
\pgfpathlineto{\pgfqpoint{2.598697in}{2.449631in}}%
\pgfpathlineto{\pgfqpoint{2.598697in}{2.595406in}}%
\pgfpathlineto{\pgfqpoint{2.472452in}{2.522519in}}%
\pgfpathlineto{\pgfqpoint{2.472452in}{2.376743in}}%
\pgfpathclose%
\pgfusepath{fill}%
\end{pgfscope}%
\begin{pgfscope}%
\pgfpathrectangle{\pgfqpoint{0.765000in}{0.660000in}}{\pgfqpoint{4.620000in}{4.620000in}}%
\pgfusepath{clip}%
\pgfsetbuttcap%
\pgfsetroundjoin%
\definecolor{currentfill}{rgb}{1.000000,0.894118,0.788235}%
\pgfsetfillcolor{currentfill}%
\pgfsetlinewidth{0.000000pt}%
\definecolor{currentstroke}{rgb}{1.000000,0.894118,0.788235}%
\pgfsetstrokecolor{currentstroke}%
\pgfsetdash{}{0pt}%
\pgfpathmoveto{\pgfqpoint{2.346207in}{2.449631in}}%
\pgfpathlineto{\pgfqpoint{2.472452in}{2.376743in}}%
\pgfpathlineto{\pgfqpoint{2.472452in}{2.522519in}}%
\pgfpathlineto{\pgfqpoint{2.346207in}{2.595406in}}%
\pgfpathlineto{\pgfqpoint{2.346207in}{2.449631in}}%
\pgfpathclose%
\pgfusepath{fill}%
\end{pgfscope}%
\begin{pgfscope}%
\pgfpathrectangle{\pgfqpoint{0.765000in}{0.660000in}}{\pgfqpoint{4.620000in}{4.620000in}}%
\pgfusepath{clip}%
\pgfsetbuttcap%
\pgfsetroundjoin%
\definecolor{currentfill}{rgb}{1.000000,0.894118,0.788235}%
\pgfsetfillcolor{currentfill}%
\pgfsetlinewidth{0.000000pt}%
\definecolor{currentstroke}{rgb}{1.000000,0.894118,0.788235}%
\pgfsetstrokecolor{currentstroke}%
\pgfsetdash{}{0pt}%
\pgfpathmoveto{\pgfqpoint{3.324897in}{2.176135in}}%
\pgfpathlineto{\pgfqpoint{3.198652in}{2.103247in}}%
\pgfpathlineto{\pgfqpoint{3.324897in}{2.030360in}}%
\pgfpathlineto{\pgfqpoint{3.451141in}{2.103247in}}%
\pgfpathlineto{\pgfqpoint{3.324897in}{2.176135in}}%
\pgfpathclose%
\pgfusepath{fill}%
\end{pgfscope}%
\begin{pgfscope}%
\pgfpathrectangle{\pgfqpoint{0.765000in}{0.660000in}}{\pgfqpoint{4.620000in}{4.620000in}}%
\pgfusepath{clip}%
\pgfsetbuttcap%
\pgfsetroundjoin%
\definecolor{currentfill}{rgb}{1.000000,0.894118,0.788235}%
\pgfsetfillcolor{currentfill}%
\pgfsetlinewidth{0.000000pt}%
\definecolor{currentstroke}{rgb}{1.000000,0.894118,0.788235}%
\pgfsetstrokecolor{currentstroke}%
\pgfsetdash{}{0pt}%
\pgfpathmoveto{\pgfqpoint{3.324897in}{1.884585in}}%
\pgfpathlineto{\pgfqpoint{3.451141in}{1.957472in}}%
\pgfpathlineto{\pgfqpoint{3.451141in}{2.103247in}}%
\pgfpathlineto{\pgfqpoint{3.324897in}{2.030360in}}%
\pgfpathlineto{\pgfqpoint{3.324897in}{1.884585in}}%
\pgfpathclose%
\pgfusepath{fill}%
\end{pgfscope}%
\begin{pgfscope}%
\pgfpathrectangle{\pgfqpoint{0.765000in}{0.660000in}}{\pgfqpoint{4.620000in}{4.620000in}}%
\pgfusepath{clip}%
\pgfsetbuttcap%
\pgfsetroundjoin%
\definecolor{currentfill}{rgb}{1.000000,0.894118,0.788235}%
\pgfsetfillcolor{currentfill}%
\pgfsetlinewidth{0.000000pt}%
\definecolor{currentstroke}{rgb}{1.000000,0.894118,0.788235}%
\pgfsetstrokecolor{currentstroke}%
\pgfsetdash{}{0pt}%
\pgfpathmoveto{\pgfqpoint{3.198652in}{1.957472in}}%
\pgfpathlineto{\pgfqpoint{3.324897in}{1.884585in}}%
\pgfpathlineto{\pgfqpoint{3.324897in}{2.030360in}}%
\pgfpathlineto{\pgfqpoint{3.198652in}{2.103247in}}%
\pgfpathlineto{\pgfqpoint{3.198652in}{1.957472in}}%
\pgfpathclose%
\pgfusepath{fill}%
\end{pgfscope}%
\begin{pgfscope}%
\pgfpathrectangle{\pgfqpoint{0.765000in}{0.660000in}}{\pgfqpoint{4.620000in}{4.620000in}}%
\pgfusepath{clip}%
\pgfsetbuttcap%
\pgfsetroundjoin%
\definecolor{currentfill}{rgb}{1.000000,0.894118,0.788235}%
\pgfsetfillcolor{currentfill}%
\pgfsetlinewidth{0.000000pt}%
\definecolor{currentstroke}{rgb}{1.000000,0.894118,0.788235}%
\pgfsetstrokecolor{currentstroke}%
\pgfsetdash{}{0pt}%
\pgfpathmoveto{\pgfqpoint{2.472452in}{2.522519in}}%
\pgfpathlineto{\pgfqpoint{2.346207in}{2.449631in}}%
\pgfpathlineto{\pgfqpoint{3.198652in}{1.957472in}}%
\pgfpathlineto{\pgfqpoint{3.324897in}{2.030360in}}%
\pgfpathlineto{\pgfqpoint{2.472452in}{2.522519in}}%
\pgfpathclose%
\pgfusepath{fill}%
\end{pgfscope}%
\begin{pgfscope}%
\pgfpathrectangle{\pgfqpoint{0.765000in}{0.660000in}}{\pgfqpoint{4.620000in}{4.620000in}}%
\pgfusepath{clip}%
\pgfsetbuttcap%
\pgfsetroundjoin%
\definecolor{currentfill}{rgb}{1.000000,0.894118,0.788235}%
\pgfsetfillcolor{currentfill}%
\pgfsetlinewidth{0.000000pt}%
\definecolor{currentstroke}{rgb}{1.000000,0.894118,0.788235}%
\pgfsetstrokecolor{currentstroke}%
\pgfsetdash{}{0pt}%
\pgfpathmoveto{\pgfqpoint{2.598697in}{2.449631in}}%
\pgfpathlineto{\pgfqpoint{2.472452in}{2.522519in}}%
\pgfpathlineto{\pgfqpoint{3.324897in}{2.030360in}}%
\pgfpathlineto{\pgfqpoint{3.451141in}{1.957472in}}%
\pgfpathlineto{\pgfqpoint{2.598697in}{2.449631in}}%
\pgfpathclose%
\pgfusepath{fill}%
\end{pgfscope}%
\begin{pgfscope}%
\pgfpathrectangle{\pgfqpoint{0.765000in}{0.660000in}}{\pgfqpoint{4.620000in}{4.620000in}}%
\pgfusepath{clip}%
\pgfsetbuttcap%
\pgfsetroundjoin%
\definecolor{currentfill}{rgb}{1.000000,0.894118,0.788235}%
\pgfsetfillcolor{currentfill}%
\pgfsetlinewidth{0.000000pt}%
\definecolor{currentstroke}{rgb}{1.000000,0.894118,0.788235}%
\pgfsetstrokecolor{currentstroke}%
\pgfsetdash{}{0pt}%
\pgfpathmoveto{\pgfqpoint{2.472452in}{2.522519in}}%
\pgfpathlineto{\pgfqpoint{2.472452in}{2.668294in}}%
\pgfpathlineto{\pgfqpoint{3.324897in}{2.176135in}}%
\pgfpathlineto{\pgfqpoint{3.451141in}{1.957472in}}%
\pgfpathlineto{\pgfqpoint{2.472452in}{2.522519in}}%
\pgfpathclose%
\pgfusepath{fill}%
\end{pgfscope}%
\begin{pgfscope}%
\pgfpathrectangle{\pgfqpoint{0.765000in}{0.660000in}}{\pgfqpoint{4.620000in}{4.620000in}}%
\pgfusepath{clip}%
\pgfsetbuttcap%
\pgfsetroundjoin%
\definecolor{currentfill}{rgb}{1.000000,0.894118,0.788235}%
\pgfsetfillcolor{currentfill}%
\pgfsetlinewidth{0.000000pt}%
\definecolor{currentstroke}{rgb}{1.000000,0.894118,0.788235}%
\pgfsetstrokecolor{currentstroke}%
\pgfsetdash{}{0pt}%
\pgfpathmoveto{\pgfqpoint{2.598697in}{2.449631in}}%
\pgfpathlineto{\pgfqpoint{2.598697in}{2.595406in}}%
\pgfpathlineto{\pgfqpoint{3.451141in}{2.103247in}}%
\pgfpathlineto{\pgfqpoint{3.324897in}{2.030360in}}%
\pgfpathlineto{\pgfqpoint{2.598697in}{2.449631in}}%
\pgfpathclose%
\pgfusepath{fill}%
\end{pgfscope}%
\begin{pgfscope}%
\pgfpathrectangle{\pgfqpoint{0.765000in}{0.660000in}}{\pgfqpoint{4.620000in}{4.620000in}}%
\pgfusepath{clip}%
\pgfsetbuttcap%
\pgfsetroundjoin%
\definecolor{currentfill}{rgb}{1.000000,0.894118,0.788235}%
\pgfsetfillcolor{currentfill}%
\pgfsetlinewidth{0.000000pt}%
\definecolor{currentstroke}{rgb}{1.000000,0.894118,0.788235}%
\pgfsetstrokecolor{currentstroke}%
\pgfsetdash{}{0pt}%
\pgfpathmoveto{\pgfqpoint{2.472452in}{2.376743in}}%
\pgfpathlineto{\pgfqpoint{2.598697in}{2.449631in}}%
\pgfpathlineto{\pgfqpoint{3.451141in}{1.957472in}}%
\pgfpathlineto{\pgfqpoint{3.324897in}{1.884585in}}%
\pgfpathlineto{\pgfqpoint{2.472452in}{2.376743in}}%
\pgfpathclose%
\pgfusepath{fill}%
\end{pgfscope}%
\begin{pgfscope}%
\pgfpathrectangle{\pgfqpoint{0.765000in}{0.660000in}}{\pgfqpoint{4.620000in}{4.620000in}}%
\pgfusepath{clip}%
\pgfsetbuttcap%
\pgfsetroundjoin%
\definecolor{currentfill}{rgb}{1.000000,0.894118,0.788235}%
\pgfsetfillcolor{currentfill}%
\pgfsetlinewidth{0.000000pt}%
\definecolor{currentstroke}{rgb}{1.000000,0.894118,0.788235}%
\pgfsetstrokecolor{currentstroke}%
\pgfsetdash{}{0pt}%
\pgfpathmoveto{\pgfqpoint{2.598697in}{2.595406in}}%
\pgfpathlineto{\pgfqpoint{2.472452in}{2.668294in}}%
\pgfpathlineto{\pgfqpoint{3.324897in}{2.176135in}}%
\pgfpathlineto{\pgfqpoint{3.451141in}{2.103247in}}%
\pgfpathlineto{\pgfqpoint{2.598697in}{2.595406in}}%
\pgfpathclose%
\pgfusepath{fill}%
\end{pgfscope}%
\begin{pgfscope}%
\pgfpathrectangle{\pgfqpoint{0.765000in}{0.660000in}}{\pgfqpoint{4.620000in}{4.620000in}}%
\pgfusepath{clip}%
\pgfsetbuttcap%
\pgfsetroundjoin%
\definecolor{currentfill}{rgb}{1.000000,0.894118,0.788235}%
\pgfsetfillcolor{currentfill}%
\pgfsetlinewidth{0.000000pt}%
\definecolor{currentstroke}{rgb}{1.000000,0.894118,0.788235}%
\pgfsetstrokecolor{currentstroke}%
\pgfsetdash{}{0pt}%
\pgfpathmoveto{\pgfqpoint{2.346207in}{2.449631in}}%
\pgfpathlineto{\pgfqpoint{2.472452in}{2.376743in}}%
\pgfpathlineto{\pgfqpoint{3.324897in}{1.884585in}}%
\pgfpathlineto{\pgfqpoint{3.198652in}{1.957472in}}%
\pgfpathlineto{\pgfqpoint{2.346207in}{2.449631in}}%
\pgfpathclose%
\pgfusepath{fill}%
\end{pgfscope}%
\begin{pgfscope}%
\pgfpathrectangle{\pgfqpoint{0.765000in}{0.660000in}}{\pgfqpoint{4.620000in}{4.620000in}}%
\pgfusepath{clip}%
\pgfsetbuttcap%
\pgfsetroundjoin%
\definecolor{currentfill}{rgb}{1.000000,0.894118,0.788235}%
\pgfsetfillcolor{currentfill}%
\pgfsetlinewidth{0.000000pt}%
\definecolor{currentstroke}{rgb}{1.000000,0.894118,0.788235}%
\pgfsetstrokecolor{currentstroke}%
\pgfsetdash{}{0pt}%
\pgfpathmoveto{\pgfqpoint{2.472452in}{2.668294in}}%
\pgfpathlineto{\pgfqpoint{2.346207in}{2.595406in}}%
\pgfpathlineto{\pgfqpoint{3.198652in}{2.103247in}}%
\pgfpathlineto{\pgfqpoint{3.324897in}{2.176135in}}%
\pgfpathlineto{\pgfqpoint{2.472452in}{2.668294in}}%
\pgfpathclose%
\pgfusepath{fill}%
\end{pgfscope}%
\begin{pgfscope}%
\pgfpathrectangle{\pgfqpoint{0.765000in}{0.660000in}}{\pgfqpoint{4.620000in}{4.620000in}}%
\pgfusepath{clip}%
\pgfsetbuttcap%
\pgfsetroundjoin%
\definecolor{currentfill}{rgb}{1.000000,0.894118,0.788235}%
\pgfsetfillcolor{currentfill}%
\pgfsetlinewidth{0.000000pt}%
\definecolor{currentstroke}{rgb}{1.000000,0.894118,0.788235}%
\pgfsetstrokecolor{currentstroke}%
\pgfsetdash{}{0pt}%
\pgfpathmoveto{\pgfqpoint{2.346207in}{2.449631in}}%
\pgfpathlineto{\pgfqpoint{2.346207in}{2.595406in}}%
\pgfpathlineto{\pgfqpoint{3.198652in}{2.103247in}}%
\pgfpathlineto{\pgfqpoint{3.324897in}{1.884585in}}%
\pgfpathlineto{\pgfqpoint{2.346207in}{2.449631in}}%
\pgfpathclose%
\pgfusepath{fill}%
\end{pgfscope}%
\begin{pgfscope}%
\pgfpathrectangle{\pgfqpoint{0.765000in}{0.660000in}}{\pgfqpoint{4.620000in}{4.620000in}}%
\pgfusepath{clip}%
\pgfsetbuttcap%
\pgfsetroundjoin%
\definecolor{currentfill}{rgb}{1.000000,0.894118,0.788235}%
\pgfsetfillcolor{currentfill}%
\pgfsetlinewidth{0.000000pt}%
\definecolor{currentstroke}{rgb}{1.000000,0.894118,0.788235}%
\pgfsetstrokecolor{currentstroke}%
\pgfsetdash{}{0pt}%
\pgfpathmoveto{\pgfqpoint{2.472452in}{2.376743in}}%
\pgfpathlineto{\pgfqpoint{2.472452in}{2.522519in}}%
\pgfpathlineto{\pgfqpoint{3.324897in}{2.030360in}}%
\pgfpathlineto{\pgfqpoint{3.198652in}{1.957472in}}%
\pgfpathlineto{\pgfqpoint{2.472452in}{2.376743in}}%
\pgfpathclose%
\pgfusepath{fill}%
\end{pgfscope}%
\begin{pgfscope}%
\pgfpathrectangle{\pgfqpoint{0.765000in}{0.660000in}}{\pgfqpoint{4.620000in}{4.620000in}}%
\pgfusepath{clip}%
\pgfsetbuttcap%
\pgfsetroundjoin%
\definecolor{currentfill}{rgb}{1.000000,0.894118,0.788235}%
\pgfsetfillcolor{currentfill}%
\pgfsetlinewidth{0.000000pt}%
\definecolor{currentstroke}{rgb}{1.000000,0.894118,0.788235}%
\pgfsetstrokecolor{currentstroke}%
\pgfsetdash{}{0pt}%
\pgfpathmoveto{\pgfqpoint{2.346207in}{2.595406in}}%
\pgfpathlineto{\pgfqpoint{2.472452in}{2.522519in}}%
\pgfpathlineto{\pgfqpoint{3.324897in}{2.030360in}}%
\pgfpathlineto{\pgfqpoint{3.198652in}{2.103247in}}%
\pgfpathlineto{\pgfqpoint{2.346207in}{2.595406in}}%
\pgfpathclose%
\pgfusepath{fill}%
\end{pgfscope}%
\begin{pgfscope}%
\pgfpathrectangle{\pgfqpoint{0.765000in}{0.660000in}}{\pgfqpoint{4.620000in}{4.620000in}}%
\pgfusepath{clip}%
\pgfsetbuttcap%
\pgfsetroundjoin%
\definecolor{currentfill}{rgb}{1.000000,0.894118,0.788235}%
\pgfsetfillcolor{currentfill}%
\pgfsetlinewidth{0.000000pt}%
\definecolor{currentstroke}{rgb}{1.000000,0.894118,0.788235}%
\pgfsetstrokecolor{currentstroke}%
\pgfsetdash{}{0pt}%
\pgfpathmoveto{\pgfqpoint{2.472452in}{2.522519in}}%
\pgfpathlineto{\pgfqpoint{2.598697in}{2.595406in}}%
\pgfpathlineto{\pgfqpoint{3.451141in}{2.103247in}}%
\pgfpathlineto{\pgfqpoint{3.324897in}{2.030360in}}%
\pgfpathlineto{\pgfqpoint{2.472452in}{2.522519in}}%
\pgfpathclose%
\pgfusepath{fill}%
\end{pgfscope}%
\begin{pgfscope}%
\pgfpathrectangle{\pgfqpoint{0.765000in}{0.660000in}}{\pgfqpoint{4.620000in}{4.620000in}}%
\pgfusepath{clip}%
\pgfsetbuttcap%
\pgfsetroundjoin%
\definecolor{currentfill}{rgb}{1.000000,0.894118,0.788235}%
\pgfsetfillcolor{currentfill}%
\pgfsetlinewidth{0.000000pt}%
\definecolor{currentstroke}{rgb}{1.000000,0.894118,0.788235}%
\pgfsetstrokecolor{currentstroke}%
\pgfsetdash{}{0pt}%
\pgfpathmoveto{\pgfqpoint{2.472452in}{2.522519in}}%
\pgfpathlineto{\pgfqpoint{2.346207in}{2.449631in}}%
\pgfpathlineto{\pgfqpoint{2.472452in}{2.376743in}}%
\pgfpathlineto{\pgfqpoint{2.598697in}{2.449631in}}%
\pgfpathlineto{\pgfqpoint{2.472452in}{2.522519in}}%
\pgfpathclose%
\pgfusepath{fill}%
\end{pgfscope}%
\begin{pgfscope}%
\pgfpathrectangle{\pgfqpoint{0.765000in}{0.660000in}}{\pgfqpoint{4.620000in}{4.620000in}}%
\pgfusepath{clip}%
\pgfsetbuttcap%
\pgfsetroundjoin%
\definecolor{currentfill}{rgb}{1.000000,0.894118,0.788235}%
\pgfsetfillcolor{currentfill}%
\pgfsetlinewidth{0.000000pt}%
\definecolor{currentstroke}{rgb}{1.000000,0.894118,0.788235}%
\pgfsetstrokecolor{currentstroke}%
\pgfsetdash{}{0pt}%
\pgfpathmoveto{\pgfqpoint{2.472452in}{2.522519in}}%
\pgfpathlineto{\pgfqpoint{2.346207in}{2.449631in}}%
\pgfpathlineto{\pgfqpoint{2.346207in}{2.595406in}}%
\pgfpathlineto{\pgfqpoint{2.472452in}{2.668294in}}%
\pgfpathlineto{\pgfqpoint{2.472452in}{2.522519in}}%
\pgfpathclose%
\pgfusepath{fill}%
\end{pgfscope}%
\begin{pgfscope}%
\pgfpathrectangle{\pgfqpoint{0.765000in}{0.660000in}}{\pgfqpoint{4.620000in}{4.620000in}}%
\pgfusepath{clip}%
\pgfsetbuttcap%
\pgfsetroundjoin%
\definecolor{currentfill}{rgb}{1.000000,0.894118,0.788235}%
\pgfsetfillcolor{currentfill}%
\pgfsetlinewidth{0.000000pt}%
\definecolor{currentstroke}{rgb}{1.000000,0.894118,0.788235}%
\pgfsetstrokecolor{currentstroke}%
\pgfsetdash{}{0pt}%
\pgfpathmoveto{\pgfqpoint{2.472452in}{2.522519in}}%
\pgfpathlineto{\pgfqpoint{2.598697in}{2.449631in}}%
\pgfpathlineto{\pgfqpoint{2.598697in}{2.595406in}}%
\pgfpathlineto{\pgfqpoint{2.472452in}{2.668294in}}%
\pgfpathlineto{\pgfqpoint{2.472452in}{2.522519in}}%
\pgfpathclose%
\pgfusepath{fill}%
\end{pgfscope}%
\begin{pgfscope}%
\pgfpathrectangle{\pgfqpoint{0.765000in}{0.660000in}}{\pgfqpoint{4.620000in}{4.620000in}}%
\pgfusepath{clip}%
\pgfsetbuttcap%
\pgfsetroundjoin%
\definecolor{currentfill}{rgb}{1.000000,0.894118,0.788235}%
\pgfsetfillcolor{currentfill}%
\pgfsetlinewidth{0.000000pt}%
\definecolor{currentstroke}{rgb}{1.000000,0.894118,0.788235}%
\pgfsetstrokecolor{currentstroke}%
\pgfsetdash{}{0pt}%
\pgfpathmoveto{\pgfqpoint{2.472452in}{3.309215in}}%
\pgfpathlineto{\pgfqpoint{2.346207in}{3.236328in}}%
\pgfpathlineto{\pgfqpoint{2.472452in}{3.163440in}}%
\pgfpathlineto{\pgfqpoint{2.598697in}{3.236328in}}%
\pgfpathlineto{\pgfqpoint{2.472452in}{3.309215in}}%
\pgfpathclose%
\pgfusepath{fill}%
\end{pgfscope}%
\begin{pgfscope}%
\pgfpathrectangle{\pgfqpoint{0.765000in}{0.660000in}}{\pgfqpoint{4.620000in}{4.620000in}}%
\pgfusepath{clip}%
\pgfsetbuttcap%
\pgfsetroundjoin%
\definecolor{currentfill}{rgb}{1.000000,0.894118,0.788235}%
\pgfsetfillcolor{currentfill}%
\pgfsetlinewidth{0.000000pt}%
\definecolor{currentstroke}{rgb}{1.000000,0.894118,0.788235}%
\pgfsetstrokecolor{currentstroke}%
\pgfsetdash{}{0pt}%
\pgfpathmoveto{\pgfqpoint{2.472452in}{3.309215in}}%
\pgfpathlineto{\pgfqpoint{2.346207in}{3.236328in}}%
\pgfpathlineto{\pgfqpoint{2.346207in}{3.382103in}}%
\pgfpathlineto{\pgfqpoint{2.472452in}{3.454990in}}%
\pgfpathlineto{\pgfqpoint{2.472452in}{3.309215in}}%
\pgfpathclose%
\pgfusepath{fill}%
\end{pgfscope}%
\begin{pgfscope}%
\pgfpathrectangle{\pgfqpoint{0.765000in}{0.660000in}}{\pgfqpoint{4.620000in}{4.620000in}}%
\pgfusepath{clip}%
\pgfsetbuttcap%
\pgfsetroundjoin%
\definecolor{currentfill}{rgb}{1.000000,0.894118,0.788235}%
\pgfsetfillcolor{currentfill}%
\pgfsetlinewidth{0.000000pt}%
\definecolor{currentstroke}{rgb}{1.000000,0.894118,0.788235}%
\pgfsetstrokecolor{currentstroke}%
\pgfsetdash{}{0pt}%
\pgfpathmoveto{\pgfqpoint{2.472452in}{3.309215in}}%
\pgfpathlineto{\pgfqpoint{2.598697in}{3.236328in}}%
\pgfpathlineto{\pgfqpoint{2.598697in}{3.382103in}}%
\pgfpathlineto{\pgfqpoint{2.472452in}{3.454990in}}%
\pgfpathlineto{\pgfqpoint{2.472452in}{3.309215in}}%
\pgfpathclose%
\pgfusepath{fill}%
\end{pgfscope}%
\begin{pgfscope}%
\pgfpathrectangle{\pgfqpoint{0.765000in}{0.660000in}}{\pgfqpoint{4.620000in}{4.620000in}}%
\pgfusepath{clip}%
\pgfsetbuttcap%
\pgfsetroundjoin%
\definecolor{currentfill}{rgb}{1.000000,0.894118,0.788235}%
\pgfsetfillcolor{currentfill}%
\pgfsetlinewidth{0.000000pt}%
\definecolor{currentstroke}{rgb}{1.000000,0.894118,0.788235}%
\pgfsetstrokecolor{currentstroke}%
\pgfsetdash{}{0pt}%
\pgfpathmoveto{\pgfqpoint{2.472452in}{2.668294in}}%
\pgfpathlineto{\pgfqpoint{2.346207in}{2.595406in}}%
\pgfpathlineto{\pgfqpoint{2.472452in}{2.522519in}}%
\pgfpathlineto{\pgfqpoint{2.598697in}{2.595406in}}%
\pgfpathlineto{\pgfqpoint{2.472452in}{2.668294in}}%
\pgfpathclose%
\pgfusepath{fill}%
\end{pgfscope}%
\begin{pgfscope}%
\pgfpathrectangle{\pgfqpoint{0.765000in}{0.660000in}}{\pgfqpoint{4.620000in}{4.620000in}}%
\pgfusepath{clip}%
\pgfsetbuttcap%
\pgfsetroundjoin%
\definecolor{currentfill}{rgb}{1.000000,0.894118,0.788235}%
\pgfsetfillcolor{currentfill}%
\pgfsetlinewidth{0.000000pt}%
\definecolor{currentstroke}{rgb}{1.000000,0.894118,0.788235}%
\pgfsetstrokecolor{currentstroke}%
\pgfsetdash{}{0pt}%
\pgfpathmoveto{\pgfqpoint{2.472452in}{2.376743in}}%
\pgfpathlineto{\pgfqpoint{2.598697in}{2.449631in}}%
\pgfpathlineto{\pgfqpoint{2.598697in}{2.595406in}}%
\pgfpathlineto{\pgfqpoint{2.472452in}{2.522519in}}%
\pgfpathlineto{\pgfqpoint{2.472452in}{2.376743in}}%
\pgfpathclose%
\pgfusepath{fill}%
\end{pgfscope}%
\begin{pgfscope}%
\pgfpathrectangle{\pgfqpoint{0.765000in}{0.660000in}}{\pgfqpoint{4.620000in}{4.620000in}}%
\pgfusepath{clip}%
\pgfsetbuttcap%
\pgfsetroundjoin%
\definecolor{currentfill}{rgb}{1.000000,0.894118,0.788235}%
\pgfsetfillcolor{currentfill}%
\pgfsetlinewidth{0.000000pt}%
\definecolor{currentstroke}{rgb}{1.000000,0.894118,0.788235}%
\pgfsetstrokecolor{currentstroke}%
\pgfsetdash{}{0pt}%
\pgfpathmoveto{\pgfqpoint{2.346207in}{2.449631in}}%
\pgfpathlineto{\pgfqpoint{2.472452in}{2.376743in}}%
\pgfpathlineto{\pgfqpoint{2.472452in}{2.522519in}}%
\pgfpathlineto{\pgfqpoint{2.346207in}{2.595406in}}%
\pgfpathlineto{\pgfqpoint{2.346207in}{2.449631in}}%
\pgfpathclose%
\pgfusepath{fill}%
\end{pgfscope}%
\begin{pgfscope}%
\pgfpathrectangle{\pgfqpoint{0.765000in}{0.660000in}}{\pgfqpoint{4.620000in}{4.620000in}}%
\pgfusepath{clip}%
\pgfsetbuttcap%
\pgfsetroundjoin%
\definecolor{currentfill}{rgb}{1.000000,0.894118,0.788235}%
\pgfsetfillcolor{currentfill}%
\pgfsetlinewidth{0.000000pt}%
\definecolor{currentstroke}{rgb}{1.000000,0.894118,0.788235}%
\pgfsetstrokecolor{currentstroke}%
\pgfsetdash{}{0pt}%
\pgfpathmoveto{\pgfqpoint{2.472452in}{3.454990in}}%
\pgfpathlineto{\pgfqpoint{2.346207in}{3.382103in}}%
\pgfpathlineto{\pgfqpoint{2.472452in}{3.309215in}}%
\pgfpathlineto{\pgfqpoint{2.598697in}{3.382103in}}%
\pgfpathlineto{\pgfqpoint{2.472452in}{3.454990in}}%
\pgfpathclose%
\pgfusepath{fill}%
\end{pgfscope}%
\begin{pgfscope}%
\pgfpathrectangle{\pgfqpoint{0.765000in}{0.660000in}}{\pgfqpoint{4.620000in}{4.620000in}}%
\pgfusepath{clip}%
\pgfsetbuttcap%
\pgfsetroundjoin%
\definecolor{currentfill}{rgb}{1.000000,0.894118,0.788235}%
\pgfsetfillcolor{currentfill}%
\pgfsetlinewidth{0.000000pt}%
\definecolor{currentstroke}{rgb}{1.000000,0.894118,0.788235}%
\pgfsetstrokecolor{currentstroke}%
\pgfsetdash{}{0pt}%
\pgfpathmoveto{\pgfqpoint{2.472452in}{3.163440in}}%
\pgfpathlineto{\pgfqpoint{2.598697in}{3.236328in}}%
\pgfpathlineto{\pgfqpoint{2.598697in}{3.382103in}}%
\pgfpathlineto{\pgfqpoint{2.472452in}{3.309215in}}%
\pgfpathlineto{\pgfqpoint{2.472452in}{3.163440in}}%
\pgfpathclose%
\pgfusepath{fill}%
\end{pgfscope}%
\begin{pgfscope}%
\pgfpathrectangle{\pgfqpoint{0.765000in}{0.660000in}}{\pgfqpoint{4.620000in}{4.620000in}}%
\pgfusepath{clip}%
\pgfsetbuttcap%
\pgfsetroundjoin%
\definecolor{currentfill}{rgb}{1.000000,0.894118,0.788235}%
\pgfsetfillcolor{currentfill}%
\pgfsetlinewidth{0.000000pt}%
\definecolor{currentstroke}{rgb}{1.000000,0.894118,0.788235}%
\pgfsetstrokecolor{currentstroke}%
\pgfsetdash{}{0pt}%
\pgfpathmoveto{\pgfqpoint{2.346207in}{3.236328in}}%
\pgfpathlineto{\pgfqpoint{2.472452in}{3.163440in}}%
\pgfpathlineto{\pgfqpoint{2.472452in}{3.309215in}}%
\pgfpathlineto{\pgfqpoint{2.346207in}{3.382103in}}%
\pgfpathlineto{\pgfqpoint{2.346207in}{3.236328in}}%
\pgfpathclose%
\pgfusepath{fill}%
\end{pgfscope}%
\begin{pgfscope}%
\pgfpathrectangle{\pgfqpoint{0.765000in}{0.660000in}}{\pgfqpoint{4.620000in}{4.620000in}}%
\pgfusepath{clip}%
\pgfsetbuttcap%
\pgfsetroundjoin%
\definecolor{currentfill}{rgb}{1.000000,0.894118,0.788235}%
\pgfsetfillcolor{currentfill}%
\pgfsetlinewidth{0.000000pt}%
\definecolor{currentstroke}{rgb}{1.000000,0.894118,0.788235}%
\pgfsetstrokecolor{currentstroke}%
\pgfsetdash{}{0pt}%
\pgfpathmoveto{\pgfqpoint{2.472452in}{2.522519in}}%
\pgfpathlineto{\pgfqpoint{2.346207in}{2.449631in}}%
\pgfpathlineto{\pgfqpoint{2.346207in}{3.236328in}}%
\pgfpathlineto{\pgfqpoint{2.472452in}{3.309215in}}%
\pgfpathlineto{\pgfqpoint{2.472452in}{2.522519in}}%
\pgfpathclose%
\pgfusepath{fill}%
\end{pgfscope}%
\begin{pgfscope}%
\pgfpathrectangle{\pgfqpoint{0.765000in}{0.660000in}}{\pgfqpoint{4.620000in}{4.620000in}}%
\pgfusepath{clip}%
\pgfsetbuttcap%
\pgfsetroundjoin%
\definecolor{currentfill}{rgb}{1.000000,0.894118,0.788235}%
\pgfsetfillcolor{currentfill}%
\pgfsetlinewidth{0.000000pt}%
\definecolor{currentstroke}{rgb}{1.000000,0.894118,0.788235}%
\pgfsetstrokecolor{currentstroke}%
\pgfsetdash{}{0pt}%
\pgfpathmoveto{\pgfqpoint{2.598697in}{2.449631in}}%
\pgfpathlineto{\pgfqpoint{2.472452in}{2.522519in}}%
\pgfpathlineto{\pgfqpoint{2.472452in}{3.309215in}}%
\pgfpathlineto{\pgfqpoint{2.598697in}{3.236328in}}%
\pgfpathlineto{\pgfqpoint{2.598697in}{2.449631in}}%
\pgfpathclose%
\pgfusepath{fill}%
\end{pgfscope}%
\begin{pgfscope}%
\pgfpathrectangle{\pgfqpoint{0.765000in}{0.660000in}}{\pgfqpoint{4.620000in}{4.620000in}}%
\pgfusepath{clip}%
\pgfsetbuttcap%
\pgfsetroundjoin%
\definecolor{currentfill}{rgb}{1.000000,0.894118,0.788235}%
\pgfsetfillcolor{currentfill}%
\pgfsetlinewidth{0.000000pt}%
\definecolor{currentstroke}{rgb}{1.000000,0.894118,0.788235}%
\pgfsetstrokecolor{currentstroke}%
\pgfsetdash{}{0pt}%
\pgfpathmoveto{\pgfqpoint{2.472452in}{2.522519in}}%
\pgfpathlineto{\pgfqpoint{2.472452in}{2.668294in}}%
\pgfpathlineto{\pgfqpoint{2.472452in}{3.454990in}}%
\pgfpathlineto{\pgfqpoint{2.598697in}{3.236328in}}%
\pgfpathlineto{\pgfqpoint{2.472452in}{2.522519in}}%
\pgfpathclose%
\pgfusepath{fill}%
\end{pgfscope}%
\begin{pgfscope}%
\pgfpathrectangle{\pgfqpoint{0.765000in}{0.660000in}}{\pgfqpoint{4.620000in}{4.620000in}}%
\pgfusepath{clip}%
\pgfsetbuttcap%
\pgfsetroundjoin%
\definecolor{currentfill}{rgb}{1.000000,0.894118,0.788235}%
\pgfsetfillcolor{currentfill}%
\pgfsetlinewidth{0.000000pt}%
\definecolor{currentstroke}{rgb}{1.000000,0.894118,0.788235}%
\pgfsetstrokecolor{currentstroke}%
\pgfsetdash{}{0pt}%
\pgfpathmoveto{\pgfqpoint{2.598697in}{2.449631in}}%
\pgfpathlineto{\pgfqpoint{2.598697in}{2.595406in}}%
\pgfpathlineto{\pgfqpoint{2.598697in}{3.382103in}}%
\pgfpathlineto{\pgfqpoint{2.472452in}{3.309215in}}%
\pgfpathlineto{\pgfqpoint{2.598697in}{2.449631in}}%
\pgfpathclose%
\pgfusepath{fill}%
\end{pgfscope}%
\begin{pgfscope}%
\pgfpathrectangle{\pgfqpoint{0.765000in}{0.660000in}}{\pgfqpoint{4.620000in}{4.620000in}}%
\pgfusepath{clip}%
\pgfsetbuttcap%
\pgfsetroundjoin%
\definecolor{currentfill}{rgb}{1.000000,0.894118,0.788235}%
\pgfsetfillcolor{currentfill}%
\pgfsetlinewidth{0.000000pt}%
\definecolor{currentstroke}{rgb}{1.000000,0.894118,0.788235}%
\pgfsetstrokecolor{currentstroke}%
\pgfsetdash{}{0pt}%
\pgfpathmoveto{\pgfqpoint{2.472452in}{2.376743in}}%
\pgfpathlineto{\pgfqpoint{2.598697in}{2.449631in}}%
\pgfpathlineto{\pgfqpoint{2.598697in}{3.236328in}}%
\pgfpathlineto{\pgfqpoint{2.472452in}{3.163440in}}%
\pgfpathlineto{\pgfqpoint{2.472452in}{2.376743in}}%
\pgfpathclose%
\pgfusepath{fill}%
\end{pgfscope}%
\begin{pgfscope}%
\pgfpathrectangle{\pgfqpoint{0.765000in}{0.660000in}}{\pgfqpoint{4.620000in}{4.620000in}}%
\pgfusepath{clip}%
\pgfsetbuttcap%
\pgfsetroundjoin%
\definecolor{currentfill}{rgb}{1.000000,0.894118,0.788235}%
\pgfsetfillcolor{currentfill}%
\pgfsetlinewidth{0.000000pt}%
\definecolor{currentstroke}{rgb}{1.000000,0.894118,0.788235}%
\pgfsetstrokecolor{currentstroke}%
\pgfsetdash{}{0pt}%
\pgfpathmoveto{\pgfqpoint{2.598697in}{2.595406in}}%
\pgfpathlineto{\pgfqpoint{2.472452in}{2.668294in}}%
\pgfpathlineto{\pgfqpoint{2.472452in}{3.454990in}}%
\pgfpathlineto{\pgfqpoint{2.598697in}{3.382103in}}%
\pgfpathlineto{\pgfqpoint{2.598697in}{2.595406in}}%
\pgfpathclose%
\pgfusepath{fill}%
\end{pgfscope}%
\begin{pgfscope}%
\pgfpathrectangle{\pgfqpoint{0.765000in}{0.660000in}}{\pgfqpoint{4.620000in}{4.620000in}}%
\pgfusepath{clip}%
\pgfsetbuttcap%
\pgfsetroundjoin%
\definecolor{currentfill}{rgb}{1.000000,0.894118,0.788235}%
\pgfsetfillcolor{currentfill}%
\pgfsetlinewidth{0.000000pt}%
\definecolor{currentstroke}{rgb}{1.000000,0.894118,0.788235}%
\pgfsetstrokecolor{currentstroke}%
\pgfsetdash{}{0pt}%
\pgfpathmoveto{\pgfqpoint{2.346207in}{2.449631in}}%
\pgfpathlineto{\pgfqpoint{2.472452in}{2.376743in}}%
\pgfpathlineto{\pgfqpoint{2.472452in}{3.163440in}}%
\pgfpathlineto{\pgfqpoint{2.346207in}{3.236328in}}%
\pgfpathlineto{\pgfqpoint{2.346207in}{2.449631in}}%
\pgfpathclose%
\pgfusepath{fill}%
\end{pgfscope}%
\begin{pgfscope}%
\pgfpathrectangle{\pgfqpoint{0.765000in}{0.660000in}}{\pgfqpoint{4.620000in}{4.620000in}}%
\pgfusepath{clip}%
\pgfsetbuttcap%
\pgfsetroundjoin%
\definecolor{currentfill}{rgb}{1.000000,0.894118,0.788235}%
\pgfsetfillcolor{currentfill}%
\pgfsetlinewidth{0.000000pt}%
\definecolor{currentstroke}{rgb}{1.000000,0.894118,0.788235}%
\pgfsetstrokecolor{currentstroke}%
\pgfsetdash{}{0pt}%
\pgfpathmoveto{\pgfqpoint{2.472452in}{2.668294in}}%
\pgfpathlineto{\pgfqpoint{2.346207in}{2.595406in}}%
\pgfpathlineto{\pgfqpoint{2.346207in}{3.382103in}}%
\pgfpathlineto{\pgfqpoint{2.472452in}{3.454990in}}%
\pgfpathlineto{\pgfqpoint{2.472452in}{2.668294in}}%
\pgfpathclose%
\pgfusepath{fill}%
\end{pgfscope}%
\begin{pgfscope}%
\pgfpathrectangle{\pgfqpoint{0.765000in}{0.660000in}}{\pgfqpoint{4.620000in}{4.620000in}}%
\pgfusepath{clip}%
\pgfsetbuttcap%
\pgfsetroundjoin%
\definecolor{currentfill}{rgb}{1.000000,0.894118,0.788235}%
\pgfsetfillcolor{currentfill}%
\pgfsetlinewidth{0.000000pt}%
\definecolor{currentstroke}{rgb}{1.000000,0.894118,0.788235}%
\pgfsetstrokecolor{currentstroke}%
\pgfsetdash{}{0pt}%
\pgfpathmoveto{\pgfqpoint{2.346207in}{2.449631in}}%
\pgfpathlineto{\pgfqpoint{2.346207in}{2.595406in}}%
\pgfpathlineto{\pgfqpoint{2.346207in}{3.382103in}}%
\pgfpathlineto{\pgfqpoint{2.472452in}{3.163440in}}%
\pgfpathlineto{\pgfqpoint{2.346207in}{2.449631in}}%
\pgfpathclose%
\pgfusepath{fill}%
\end{pgfscope}%
\begin{pgfscope}%
\pgfpathrectangle{\pgfqpoint{0.765000in}{0.660000in}}{\pgfqpoint{4.620000in}{4.620000in}}%
\pgfusepath{clip}%
\pgfsetbuttcap%
\pgfsetroundjoin%
\definecolor{currentfill}{rgb}{1.000000,0.894118,0.788235}%
\pgfsetfillcolor{currentfill}%
\pgfsetlinewidth{0.000000pt}%
\definecolor{currentstroke}{rgb}{1.000000,0.894118,0.788235}%
\pgfsetstrokecolor{currentstroke}%
\pgfsetdash{}{0pt}%
\pgfpathmoveto{\pgfqpoint{2.472452in}{2.376743in}}%
\pgfpathlineto{\pgfqpoint{2.472452in}{2.522519in}}%
\pgfpathlineto{\pgfqpoint{2.472452in}{3.309215in}}%
\pgfpathlineto{\pgfqpoint{2.346207in}{3.236328in}}%
\pgfpathlineto{\pgfqpoint{2.472452in}{2.376743in}}%
\pgfpathclose%
\pgfusepath{fill}%
\end{pgfscope}%
\begin{pgfscope}%
\pgfpathrectangle{\pgfqpoint{0.765000in}{0.660000in}}{\pgfqpoint{4.620000in}{4.620000in}}%
\pgfusepath{clip}%
\pgfsetbuttcap%
\pgfsetroundjoin%
\definecolor{currentfill}{rgb}{1.000000,0.894118,0.788235}%
\pgfsetfillcolor{currentfill}%
\pgfsetlinewidth{0.000000pt}%
\definecolor{currentstroke}{rgb}{1.000000,0.894118,0.788235}%
\pgfsetstrokecolor{currentstroke}%
\pgfsetdash{}{0pt}%
\pgfpathmoveto{\pgfqpoint{2.346207in}{2.595406in}}%
\pgfpathlineto{\pgfqpoint{2.472452in}{2.522519in}}%
\pgfpathlineto{\pgfqpoint{2.472452in}{3.309215in}}%
\pgfpathlineto{\pgfqpoint{2.346207in}{3.382103in}}%
\pgfpathlineto{\pgfqpoint{2.346207in}{2.595406in}}%
\pgfpathclose%
\pgfusepath{fill}%
\end{pgfscope}%
\begin{pgfscope}%
\pgfpathrectangle{\pgfqpoint{0.765000in}{0.660000in}}{\pgfqpoint{4.620000in}{4.620000in}}%
\pgfusepath{clip}%
\pgfsetbuttcap%
\pgfsetroundjoin%
\definecolor{currentfill}{rgb}{1.000000,0.894118,0.788235}%
\pgfsetfillcolor{currentfill}%
\pgfsetlinewidth{0.000000pt}%
\definecolor{currentstroke}{rgb}{1.000000,0.894118,0.788235}%
\pgfsetstrokecolor{currentstroke}%
\pgfsetdash{}{0pt}%
\pgfpathmoveto{\pgfqpoint{2.472452in}{2.522519in}}%
\pgfpathlineto{\pgfqpoint{2.598697in}{2.595406in}}%
\pgfpathlineto{\pgfqpoint{2.598697in}{3.382103in}}%
\pgfpathlineto{\pgfqpoint{2.472452in}{3.309215in}}%
\pgfpathlineto{\pgfqpoint{2.472452in}{2.522519in}}%
\pgfpathclose%
\pgfusepath{fill}%
\end{pgfscope}%
\begin{pgfscope}%
\pgfpathrectangle{\pgfqpoint{0.765000in}{0.660000in}}{\pgfqpoint{4.620000in}{4.620000in}}%
\pgfusepath{clip}%
\pgfsetbuttcap%
\pgfsetroundjoin%
\definecolor{currentfill}{rgb}{1.000000,0.894118,0.788235}%
\pgfsetfillcolor{currentfill}%
\pgfsetlinewidth{0.000000pt}%
\definecolor{currentstroke}{rgb}{1.000000,0.894118,0.788235}%
\pgfsetstrokecolor{currentstroke}%
\pgfsetdash{}{0pt}%
\pgfpathmoveto{\pgfqpoint{2.472452in}{2.522519in}}%
\pgfpathlineto{\pgfqpoint{2.346207in}{2.449631in}}%
\pgfpathlineto{\pgfqpoint{2.472452in}{2.376743in}}%
\pgfpathlineto{\pgfqpoint{2.598697in}{2.449631in}}%
\pgfpathlineto{\pgfqpoint{2.472452in}{2.522519in}}%
\pgfpathclose%
\pgfusepath{fill}%
\end{pgfscope}%
\begin{pgfscope}%
\pgfpathrectangle{\pgfqpoint{0.765000in}{0.660000in}}{\pgfqpoint{4.620000in}{4.620000in}}%
\pgfusepath{clip}%
\pgfsetbuttcap%
\pgfsetroundjoin%
\definecolor{currentfill}{rgb}{1.000000,0.894118,0.788235}%
\pgfsetfillcolor{currentfill}%
\pgfsetlinewidth{0.000000pt}%
\definecolor{currentstroke}{rgb}{1.000000,0.894118,0.788235}%
\pgfsetstrokecolor{currentstroke}%
\pgfsetdash{}{0pt}%
\pgfpathmoveto{\pgfqpoint{2.472452in}{2.522519in}}%
\pgfpathlineto{\pgfqpoint{2.346207in}{2.449631in}}%
\pgfpathlineto{\pgfqpoint{2.346207in}{2.595406in}}%
\pgfpathlineto{\pgfqpoint{2.472452in}{2.668294in}}%
\pgfpathlineto{\pgfqpoint{2.472452in}{2.522519in}}%
\pgfpathclose%
\pgfusepath{fill}%
\end{pgfscope}%
\begin{pgfscope}%
\pgfpathrectangle{\pgfqpoint{0.765000in}{0.660000in}}{\pgfqpoint{4.620000in}{4.620000in}}%
\pgfusepath{clip}%
\pgfsetbuttcap%
\pgfsetroundjoin%
\definecolor{currentfill}{rgb}{1.000000,0.894118,0.788235}%
\pgfsetfillcolor{currentfill}%
\pgfsetlinewidth{0.000000pt}%
\definecolor{currentstroke}{rgb}{1.000000,0.894118,0.788235}%
\pgfsetstrokecolor{currentstroke}%
\pgfsetdash{}{0pt}%
\pgfpathmoveto{\pgfqpoint{2.472452in}{2.522519in}}%
\pgfpathlineto{\pgfqpoint{2.598697in}{2.449631in}}%
\pgfpathlineto{\pgfqpoint{2.598697in}{2.595406in}}%
\pgfpathlineto{\pgfqpoint{2.472452in}{2.668294in}}%
\pgfpathlineto{\pgfqpoint{2.472452in}{2.522519in}}%
\pgfpathclose%
\pgfusepath{fill}%
\end{pgfscope}%
\begin{pgfscope}%
\pgfpathrectangle{\pgfqpoint{0.765000in}{0.660000in}}{\pgfqpoint{4.620000in}{4.620000in}}%
\pgfusepath{clip}%
\pgfsetbuttcap%
\pgfsetroundjoin%
\definecolor{currentfill}{rgb}{1.000000,0.894118,0.788235}%
\pgfsetfillcolor{currentfill}%
\pgfsetlinewidth{0.000000pt}%
\definecolor{currentstroke}{rgb}{1.000000,0.894118,0.788235}%
\pgfsetstrokecolor{currentstroke}%
\pgfsetdash{}{0pt}%
\pgfpathmoveto{\pgfqpoint{-3.962385in}{-1.192637in}}%
\pgfpathlineto{\pgfqpoint{-4.088630in}{-1.265524in}}%
\pgfpathlineto{\pgfqpoint{-3.962385in}{-1.338412in}}%
\pgfpathlineto{\pgfqpoint{-3.836140in}{-1.265524in}}%
\pgfpathlineto{\pgfqpoint{-3.962385in}{-1.192637in}}%
\pgfpathclose%
\pgfusepath{fill}%
\end{pgfscope}%
\begin{pgfscope}%
\pgfpathrectangle{\pgfqpoint{0.765000in}{0.660000in}}{\pgfqpoint{4.620000in}{4.620000in}}%
\pgfusepath{clip}%
\pgfsetbuttcap%
\pgfsetroundjoin%
\definecolor{currentfill}{rgb}{1.000000,0.894118,0.788235}%
\pgfsetfillcolor{currentfill}%
\pgfsetlinewidth{0.000000pt}%
\definecolor{currentstroke}{rgb}{1.000000,0.894118,0.788235}%
\pgfsetstrokecolor{currentstroke}%
\pgfsetdash{}{0pt}%
\pgfpathmoveto{\pgfqpoint{-3.962385in}{-1.192637in}}%
\pgfpathlineto{\pgfqpoint{-4.088630in}{-1.265524in}}%
\pgfpathlineto{\pgfqpoint{-4.088630in}{-1.119749in}}%
\pgfpathlineto{\pgfqpoint{-3.962385in}{-1.046862in}}%
\pgfpathlineto{\pgfqpoint{-3.962385in}{-1.192637in}}%
\pgfpathclose%
\pgfusepath{fill}%
\end{pgfscope}%
\begin{pgfscope}%
\pgfpathrectangle{\pgfqpoint{0.765000in}{0.660000in}}{\pgfqpoint{4.620000in}{4.620000in}}%
\pgfusepath{clip}%
\pgfsetbuttcap%
\pgfsetroundjoin%
\definecolor{currentfill}{rgb}{1.000000,0.894118,0.788235}%
\pgfsetfillcolor{currentfill}%
\pgfsetlinewidth{0.000000pt}%
\definecolor{currentstroke}{rgb}{1.000000,0.894118,0.788235}%
\pgfsetstrokecolor{currentstroke}%
\pgfsetdash{}{0pt}%
\pgfpathmoveto{\pgfqpoint{-3.962385in}{-1.192637in}}%
\pgfpathlineto{\pgfqpoint{-3.836140in}{-1.265524in}}%
\pgfpathlineto{\pgfqpoint{-3.836140in}{-1.119749in}}%
\pgfpathlineto{\pgfqpoint{-3.962385in}{-1.046862in}}%
\pgfpathlineto{\pgfqpoint{-3.962385in}{-1.192637in}}%
\pgfpathclose%
\pgfusepath{fill}%
\end{pgfscope}%
\begin{pgfscope}%
\pgfpathrectangle{\pgfqpoint{0.765000in}{0.660000in}}{\pgfqpoint{4.620000in}{4.620000in}}%
\pgfusepath{clip}%
\pgfsetbuttcap%
\pgfsetroundjoin%
\definecolor{currentfill}{rgb}{1.000000,0.894118,0.788235}%
\pgfsetfillcolor{currentfill}%
\pgfsetlinewidth{0.000000pt}%
\definecolor{currentstroke}{rgb}{1.000000,0.894118,0.788235}%
\pgfsetstrokecolor{currentstroke}%
\pgfsetdash{}{0pt}%
\pgfpathmoveto{\pgfqpoint{2.472452in}{2.668294in}}%
\pgfpathlineto{\pgfqpoint{2.346207in}{2.595406in}}%
\pgfpathlineto{\pgfqpoint{2.472452in}{2.522519in}}%
\pgfpathlineto{\pgfqpoint{2.598697in}{2.595406in}}%
\pgfpathlineto{\pgfqpoint{2.472452in}{2.668294in}}%
\pgfpathclose%
\pgfusepath{fill}%
\end{pgfscope}%
\begin{pgfscope}%
\pgfpathrectangle{\pgfqpoint{0.765000in}{0.660000in}}{\pgfqpoint{4.620000in}{4.620000in}}%
\pgfusepath{clip}%
\pgfsetbuttcap%
\pgfsetroundjoin%
\definecolor{currentfill}{rgb}{1.000000,0.894118,0.788235}%
\pgfsetfillcolor{currentfill}%
\pgfsetlinewidth{0.000000pt}%
\definecolor{currentstroke}{rgb}{1.000000,0.894118,0.788235}%
\pgfsetstrokecolor{currentstroke}%
\pgfsetdash{}{0pt}%
\pgfpathmoveto{\pgfqpoint{2.472452in}{2.376743in}}%
\pgfpathlineto{\pgfqpoint{2.598697in}{2.449631in}}%
\pgfpathlineto{\pgfqpoint{2.598697in}{2.595406in}}%
\pgfpathlineto{\pgfqpoint{2.472452in}{2.522519in}}%
\pgfpathlineto{\pgfqpoint{2.472452in}{2.376743in}}%
\pgfpathclose%
\pgfusepath{fill}%
\end{pgfscope}%
\begin{pgfscope}%
\pgfpathrectangle{\pgfqpoint{0.765000in}{0.660000in}}{\pgfqpoint{4.620000in}{4.620000in}}%
\pgfusepath{clip}%
\pgfsetbuttcap%
\pgfsetroundjoin%
\definecolor{currentfill}{rgb}{1.000000,0.894118,0.788235}%
\pgfsetfillcolor{currentfill}%
\pgfsetlinewidth{0.000000pt}%
\definecolor{currentstroke}{rgb}{1.000000,0.894118,0.788235}%
\pgfsetstrokecolor{currentstroke}%
\pgfsetdash{}{0pt}%
\pgfpathmoveto{\pgfqpoint{2.346207in}{2.449631in}}%
\pgfpathlineto{\pgfqpoint{2.472452in}{2.376743in}}%
\pgfpathlineto{\pgfqpoint{2.472452in}{2.522519in}}%
\pgfpathlineto{\pgfqpoint{2.346207in}{2.595406in}}%
\pgfpathlineto{\pgfqpoint{2.346207in}{2.449631in}}%
\pgfpathclose%
\pgfusepath{fill}%
\end{pgfscope}%
\begin{pgfscope}%
\pgfpathrectangle{\pgfqpoint{0.765000in}{0.660000in}}{\pgfqpoint{4.620000in}{4.620000in}}%
\pgfusepath{clip}%
\pgfsetbuttcap%
\pgfsetroundjoin%
\definecolor{currentfill}{rgb}{1.000000,0.894118,0.788235}%
\pgfsetfillcolor{currentfill}%
\pgfsetlinewidth{0.000000pt}%
\definecolor{currentstroke}{rgb}{1.000000,0.894118,0.788235}%
\pgfsetstrokecolor{currentstroke}%
\pgfsetdash{}{0pt}%
\pgfpathmoveto{\pgfqpoint{-3.962385in}{-1.046862in}}%
\pgfpathlineto{\pgfqpoint{-4.088630in}{-1.119749in}}%
\pgfpathlineto{\pgfqpoint{-3.962385in}{-1.192637in}}%
\pgfpathlineto{\pgfqpoint{-3.836140in}{-1.119749in}}%
\pgfpathlineto{\pgfqpoint{-3.962385in}{-1.046862in}}%
\pgfpathclose%
\pgfusepath{fill}%
\end{pgfscope}%
\begin{pgfscope}%
\pgfpathrectangle{\pgfqpoint{0.765000in}{0.660000in}}{\pgfqpoint{4.620000in}{4.620000in}}%
\pgfusepath{clip}%
\pgfsetbuttcap%
\pgfsetroundjoin%
\definecolor{currentfill}{rgb}{1.000000,0.894118,0.788235}%
\pgfsetfillcolor{currentfill}%
\pgfsetlinewidth{0.000000pt}%
\definecolor{currentstroke}{rgb}{1.000000,0.894118,0.788235}%
\pgfsetstrokecolor{currentstroke}%
\pgfsetdash{}{0pt}%
\pgfpathmoveto{\pgfqpoint{-3.962385in}{-1.338412in}}%
\pgfpathlineto{\pgfqpoint{-3.836140in}{-1.265524in}}%
\pgfpathlineto{\pgfqpoint{-3.836140in}{-1.119749in}}%
\pgfpathlineto{\pgfqpoint{-3.962385in}{-1.192637in}}%
\pgfpathlineto{\pgfqpoint{-3.962385in}{-1.338412in}}%
\pgfpathclose%
\pgfusepath{fill}%
\end{pgfscope}%
\begin{pgfscope}%
\pgfpathrectangle{\pgfqpoint{0.765000in}{0.660000in}}{\pgfqpoint{4.620000in}{4.620000in}}%
\pgfusepath{clip}%
\pgfsetbuttcap%
\pgfsetroundjoin%
\definecolor{currentfill}{rgb}{1.000000,0.894118,0.788235}%
\pgfsetfillcolor{currentfill}%
\pgfsetlinewidth{0.000000pt}%
\definecolor{currentstroke}{rgb}{1.000000,0.894118,0.788235}%
\pgfsetstrokecolor{currentstroke}%
\pgfsetdash{}{0pt}%
\pgfpathmoveto{\pgfqpoint{-4.088630in}{-1.265524in}}%
\pgfpathlineto{\pgfqpoint{-3.962385in}{-1.338412in}}%
\pgfpathlineto{\pgfqpoint{-3.962385in}{-1.192637in}}%
\pgfpathlineto{\pgfqpoint{-4.088630in}{-1.119749in}}%
\pgfpathlineto{\pgfqpoint{-4.088630in}{-1.265524in}}%
\pgfpathclose%
\pgfusepath{fill}%
\end{pgfscope}%
\begin{pgfscope}%
\pgfpathrectangle{\pgfqpoint{0.765000in}{0.660000in}}{\pgfqpoint{4.620000in}{4.620000in}}%
\pgfusepath{clip}%
\pgfsetbuttcap%
\pgfsetroundjoin%
\definecolor{currentfill}{rgb}{1.000000,0.894118,0.788235}%
\pgfsetfillcolor{currentfill}%
\pgfsetlinewidth{0.000000pt}%
\definecolor{currentstroke}{rgb}{1.000000,0.894118,0.788235}%
\pgfsetstrokecolor{currentstroke}%
\pgfsetdash{}{0pt}%
\pgfpathmoveto{\pgfqpoint{2.472452in}{2.522519in}}%
\pgfpathlineto{\pgfqpoint{2.346207in}{2.449631in}}%
\pgfpathlineto{\pgfqpoint{-4.088630in}{-1.265524in}}%
\pgfpathlineto{\pgfqpoint{-3.962385in}{-1.192637in}}%
\pgfpathlineto{\pgfqpoint{2.472452in}{2.522519in}}%
\pgfpathclose%
\pgfusepath{fill}%
\end{pgfscope}%
\begin{pgfscope}%
\pgfpathrectangle{\pgfqpoint{0.765000in}{0.660000in}}{\pgfqpoint{4.620000in}{4.620000in}}%
\pgfusepath{clip}%
\pgfsetbuttcap%
\pgfsetroundjoin%
\definecolor{currentfill}{rgb}{1.000000,0.894118,0.788235}%
\pgfsetfillcolor{currentfill}%
\pgfsetlinewidth{0.000000pt}%
\definecolor{currentstroke}{rgb}{1.000000,0.894118,0.788235}%
\pgfsetstrokecolor{currentstroke}%
\pgfsetdash{}{0pt}%
\pgfpathmoveto{\pgfqpoint{2.598697in}{2.449631in}}%
\pgfpathlineto{\pgfqpoint{2.472452in}{2.522519in}}%
\pgfpathlineto{\pgfqpoint{-3.962385in}{-1.192637in}}%
\pgfpathlineto{\pgfqpoint{-3.836140in}{-1.265524in}}%
\pgfpathlineto{\pgfqpoint{2.598697in}{2.449631in}}%
\pgfpathclose%
\pgfusepath{fill}%
\end{pgfscope}%
\begin{pgfscope}%
\pgfpathrectangle{\pgfqpoint{0.765000in}{0.660000in}}{\pgfqpoint{4.620000in}{4.620000in}}%
\pgfusepath{clip}%
\pgfsetbuttcap%
\pgfsetroundjoin%
\definecolor{currentfill}{rgb}{1.000000,0.894118,0.788235}%
\pgfsetfillcolor{currentfill}%
\pgfsetlinewidth{0.000000pt}%
\definecolor{currentstroke}{rgb}{1.000000,0.894118,0.788235}%
\pgfsetstrokecolor{currentstroke}%
\pgfsetdash{}{0pt}%
\pgfpathmoveto{\pgfqpoint{2.472452in}{2.522519in}}%
\pgfpathlineto{\pgfqpoint{2.472452in}{2.668294in}}%
\pgfpathlineto{\pgfqpoint{-3.962385in}{-1.046862in}}%
\pgfpathlineto{\pgfqpoint{-3.836140in}{-1.265524in}}%
\pgfpathlineto{\pgfqpoint{2.472452in}{2.522519in}}%
\pgfpathclose%
\pgfusepath{fill}%
\end{pgfscope}%
\begin{pgfscope}%
\pgfpathrectangle{\pgfqpoint{0.765000in}{0.660000in}}{\pgfqpoint{4.620000in}{4.620000in}}%
\pgfusepath{clip}%
\pgfsetbuttcap%
\pgfsetroundjoin%
\definecolor{currentfill}{rgb}{1.000000,0.894118,0.788235}%
\pgfsetfillcolor{currentfill}%
\pgfsetlinewidth{0.000000pt}%
\definecolor{currentstroke}{rgb}{1.000000,0.894118,0.788235}%
\pgfsetstrokecolor{currentstroke}%
\pgfsetdash{}{0pt}%
\pgfpathmoveto{\pgfqpoint{2.598697in}{2.449631in}}%
\pgfpathlineto{\pgfqpoint{2.598697in}{2.595406in}}%
\pgfpathlineto{\pgfqpoint{-3.836140in}{-1.119749in}}%
\pgfpathlineto{\pgfqpoint{-3.962385in}{-1.192637in}}%
\pgfpathlineto{\pgfqpoint{2.598697in}{2.449631in}}%
\pgfpathclose%
\pgfusepath{fill}%
\end{pgfscope}%
\begin{pgfscope}%
\pgfpathrectangle{\pgfqpoint{0.765000in}{0.660000in}}{\pgfqpoint{4.620000in}{4.620000in}}%
\pgfusepath{clip}%
\pgfsetbuttcap%
\pgfsetroundjoin%
\definecolor{currentfill}{rgb}{1.000000,0.894118,0.788235}%
\pgfsetfillcolor{currentfill}%
\pgfsetlinewidth{0.000000pt}%
\definecolor{currentstroke}{rgb}{1.000000,0.894118,0.788235}%
\pgfsetstrokecolor{currentstroke}%
\pgfsetdash{}{0pt}%
\pgfpathmoveto{\pgfqpoint{2.472452in}{2.376743in}}%
\pgfpathlineto{\pgfqpoint{2.598697in}{2.449631in}}%
\pgfpathlineto{\pgfqpoint{-3.836140in}{-1.265524in}}%
\pgfpathlineto{\pgfqpoint{-3.962385in}{-1.338412in}}%
\pgfpathlineto{\pgfqpoint{2.472452in}{2.376743in}}%
\pgfpathclose%
\pgfusepath{fill}%
\end{pgfscope}%
\begin{pgfscope}%
\pgfpathrectangle{\pgfqpoint{0.765000in}{0.660000in}}{\pgfqpoint{4.620000in}{4.620000in}}%
\pgfusepath{clip}%
\pgfsetbuttcap%
\pgfsetroundjoin%
\definecolor{currentfill}{rgb}{1.000000,0.894118,0.788235}%
\pgfsetfillcolor{currentfill}%
\pgfsetlinewidth{0.000000pt}%
\definecolor{currentstroke}{rgb}{1.000000,0.894118,0.788235}%
\pgfsetstrokecolor{currentstroke}%
\pgfsetdash{}{0pt}%
\pgfpathmoveto{\pgfqpoint{2.598697in}{2.595406in}}%
\pgfpathlineto{\pgfqpoint{2.472452in}{2.668294in}}%
\pgfpathlineto{\pgfqpoint{-3.962385in}{-1.046862in}}%
\pgfpathlineto{\pgfqpoint{-3.836140in}{-1.119749in}}%
\pgfpathlineto{\pgfqpoint{2.598697in}{2.595406in}}%
\pgfpathclose%
\pgfusepath{fill}%
\end{pgfscope}%
\begin{pgfscope}%
\pgfpathrectangle{\pgfqpoint{0.765000in}{0.660000in}}{\pgfqpoint{4.620000in}{4.620000in}}%
\pgfusepath{clip}%
\pgfsetbuttcap%
\pgfsetroundjoin%
\definecolor{currentfill}{rgb}{1.000000,0.894118,0.788235}%
\pgfsetfillcolor{currentfill}%
\pgfsetlinewidth{0.000000pt}%
\definecolor{currentstroke}{rgb}{1.000000,0.894118,0.788235}%
\pgfsetstrokecolor{currentstroke}%
\pgfsetdash{}{0pt}%
\pgfpathmoveto{\pgfqpoint{2.346207in}{2.449631in}}%
\pgfpathlineto{\pgfqpoint{2.472452in}{2.376743in}}%
\pgfpathlineto{\pgfqpoint{-3.962385in}{-1.338412in}}%
\pgfpathlineto{\pgfqpoint{-4.088630in}{-1.265524in}}%
\pgfpathlineto{\pgfqpoint{2.346207in}{2.449631in}}%
\pgfpathclose%
\pgfusepath{fill}%
\end{pgfscope}%
\begin{pgfscope}%
\pgfpathrectangle{\pgfqpoint{0.765000in}{0.660000in}}{\pgfqpoint{4.620000in}{4.620000in}}%
\pgfusepath{clip}%
\pgfsetbuttcap%
\pgfsetroundjoin%
\definecolor{currentfill}{rgb}{1.000000,0.894118,0.788235}%
\pgfsetfillcolor{currentfill}%
\pgfsetlinewidth{0.000000pt}%
\definecolor{currentstroke}{rgb}{1.000000,0.894118,0.788235}%
\pgfsetstrokecolor{currentstroke}%
\pgfsetdash{}{0pt}%
\pgfpathmoveto{\pgfqpoint{2.472452in}{2.668294in}}%
\pgfpathlineto{\pgfqpoint{2.346207in}{2.595406in}}%
\pgfpathlineto{\pgfqpoint{-4.088630in}{-1.119749in}}%
\pgfpathlineto{\pgfqpoint{-3.962385in}{-1.046862in}}%
\pgfpathlineto{\pgfqpoint{2.472452in}{2.668294in}}%
\pgfpathclose%
\pgfusepath{fill}%
\end{pgfscope}%
\begin{pgfscope}%
\pgfpathrectangle{\pgfqpoint{0.765000in}{0.660000in}}{\pgfqpoint{4.620000in}{4.620000in}}%
\pgfusepath{clip}%
\pgfsetbuttcap%
\pgfsetroundjoin%
\definecolor{currentfill}{rgb}{1.000000,0.894118,0.788235}%
\pgfsetfillcolor{currentfill}%
\pgfsetlinewidth{0.000000pt}%
\definecolor{currentstroke}{rgb}{1.000000,0.894118,0.788235}%
\pgfsetstrokecolor{currentstroke}%
\pgfsetdash{}{0pt}%
\pgfpathmoveto{\pgfqpoint{2.346207in}{2.449631in}}%
\pgfpathlineto{\pgfqpoint{2.346207in}{2.595406in}}%
\pgfpathlineto{\pgfqpoint{-4.088630in}{-1.119749in}}%
\pgfpathlineto{\pgfqpoint{-3.962385in}{-1.338412in}}%
\pgfpathlineto{\pgfqpoint{2.346207in}{2.449631in}}%
\pgfpathclose%
\pgfusepath{fill}%
\end{pgfscope}%
\begin{pgfscope}%
\pgfpathrectangle{\pgfqpoint{0.765000in}{0.660000in}}{\pgfqpoint{4.620000in}{4.620000in}}%
\pgfusepath{clip}%
\pgfsetbuttcap%
\pgfsetroundjoin%
\definecolor{currentfill}{rgb}{1.000000,0.894118,0.788235}%
\pgfsetfillcolor{currentfill}%
\pgfsetlinewidth{0.000000pt}%
\definecolor{currentstroke}{rgb}{1.000000,0.894118,0.788235}%
\pgfsetstrokecolor{currentstroke}%
\pgfsetdash{}{0pt}%
\pgfpathmoveto{\pgfqpoint{2.472452in}{2.376743in}}%
\pgfpathlineto{\pgfqpoint{2.472452in}{2.522519in}}%
\pgfpathlineto{\pgfqpoint{-3.962385in}{-1.192637in}}%
\pgfpathlineto{\pgfqpoint{-4.088630in}{-1.265524in}}%
\pgfpathlineto{\pgfqpoint{2.472452in}{2.376743in}}%
\pgfpathclose%
\pgfusepath{fill}%
\end{pgfscope}%
\begin{pgfscope}%
\pgfpathrectangle{\pgfqpoint{0.765000in}{0.660000in}}{\pgfqpoint{4.620000in}{4.620000in}}%
\pgfusepath{clip}%
\pgfsetbuttcap%
\pgfsetroundjoin%
\definecolor{currentfill}{rgb}{1.000000,0.894118,0.788235}%
\pgfsetfillcolor{currentfill}%
\pgfsetlinewidth{0.000000pt}%
\definecolor{currentstroke}{rgb}{1.000000,0.894118,0.788235}%
\pgfsetstrokecolor{currentstroke}%
\pgfsetdash{}{0pt}%
\pgfpathmoveto{\pgfqpoint{2.346207in}{2.595406in}}%
\pgfpathlineto{\pgfqpoint{2.472452in}{2.522519in}}%
\pgfpathlineto{\pgfqpoint{-3.962385in}{-1.192637in}}%
\pgfpathlineto{\pgfqpoint{-4.088630in}{-1.119749in}}%
\pgfpathlineto{\pgfqpoint{2.346207in}{2.595406in}}%
\pgfpathclose%
\pgfusepath{fill}%
\end{pgfscope}%
\begin{pgfscope}%
\pgfpathrectangle{\pgfqpoint{0.765000in}{0.660000in}}{\pgfqpoint{4.620000in}{4.620000in}}%
\pgfusepath{clip}%
\pgfsetbuttcap%
\pgfsetroundjoin%
\definecolor{currentfill}{rgb}{1.000000,0.894118,0.788235}%
\pgfsetfillcolor{currentfill}%
\pgfsetlinewidth{0.000000pt}%
\definecolor{currentstroke}{rgb}{1.000000,0.894118,0.788235}%
\pgfsetstrokecolor{currentstroke}%
\pgfsetdash{}{0pt}%
\pgfpathmoveto{\pgfqpoint{2.472452in}{2.522519in}}%
\pgfpathlineto{\pgfqpoint{2.598697in}{2.595406in}}%
\pgfpathlineto{\pgfqpoint{-3.836140in}{-1.119749in}}%
\pgfpathlineto{\pgfqpoint{-3.962385in}{-1.192637in}}%
\pgfpathlineto{\pgfqpoint{2.472452in}{2.522519in}}%
\pgfpathclose%
\pgfusepath{fill}%
\end{pgfscope}%
\begin{pgfscope}%
\pgfpathrectangle{\pgfqpoint{0.765000in}{0.660000in}}{\pgfqpoint{4.620000in}{4.620000in}}%
\pgfusepath{clip}%
\pgfsetbuttcap%
\pgfsetroundjoin%
\definecolor{currentfill}{rgb}{1.000000,0.894118,0.788235}%
\pgfsetfillcolor{currentfill}%
\pgfsetlinewidth{0.000000pt}%
\definecolor{currentstroke}{rgb}{1.000000,0.894118,0.788235}%
\pgfsetstrokecolor{currentstroke}%
\pgfsetdash{}{0pt}%
\pgfpathmoveto{\pgfqpoint{3.324897in}{2.030360in}}%
\pgfpathlineto{\pgfqpoint{3.198652in}{1.957472in}}%
\pgfpathlineto{\pgfqpoint{3.324897in}{1.884585in}}%
\pgfpathlineto{\pgfqpoint{3.451141in}{1.957472in}}%
\pgfpathlineto{\pgfqpoint{3.324897in}{2.030360in}}%
\pgfpathclose%
\pgfusepath{fill}%
\end{pgfscope}%
\begin{pgfscope}%
\pgfpathrectangle{\pgfqpoint{0.765000in}{0.660000in}}{\pgfqpoint{4.620000in}{4.620000in}}%
\pgfusepath{clip}%
\pgfsetbuttcap%
\pgfsetroundjoin%
\definecolor{currentfill}{rgb}{1.000000,0.894118,0.788235}%
\pgfsetfillcolor{currentfill}%
\pgfsetlinewidth{0.000000pt}%
\definecolor{currentstroke}{rgb}{1.000000,0.894118,0.788235}%
\pgfsetstrokecolor{currentstroke}%
\pgfsetdash{}{0pt}%
\pgfpathmoveto{\pgfqpoint{3.324897in}{2.030360in}}%
\pgfpathlineto{\pgfqpoint{3.198652in}{1.957472in}}%
\pgfpathlineto{\pgfqpoint{3.198652in}{2.103247in}}%
\pgfpathlineto{\pgfqpoint{3.324897in}{2.176135in}}%
\pgfpathlineto{\pgfqpoint{3.324897in}{2.030360in}}%
\pgfpathclose%
\pgfusepath{fill}%
\end{pgfscope}%
\begin{pgfscope}%
\pgfpathrectangle{\pgfqpoint{0.765000in}{0.660000in}}{\pgfqpoint{4.620000in}{4.620000in}}%
\pgfusepath{clip}%
\pgfsetbuttcap%
\pgfsetroundjoin%
\definecolor{currentfill}{rgb}{1.000000,0.894118,0.788235}%
\pgfsetfillcolor{currentfill}%
\pgfsetlinewidth{0.000000pt}%
\definecolor{currentstroke}{rgb}{1.000000,0.894118,0.788235}%
\pgfsetstrokecolor{currentstroke}%
\pgfsetdash{}{0pt}%
\pgfpathmoveto{\pgfqpoint{3.324897in}{2.030360in}}%
\pgfpathlineto{\pgfqpoint{3.451141in}{1.957472in}}%
\pgfpathlineto{\pgfqpoint{3.451141in}{2.103247in}}%
\pgfpathlineto{\pgfqpoint{3.324897in}{2.176135in}}%
\pgfpathlineto{\pgfqpoint{3.324897in}{2.030360in}}%
\pgfpathclose%
\pgfusepath{fill}%
\end{pgfscope}%
\begin{pgfscope}%
\pgfpathrectangle{\pgfqpoint{0.765000in}{0.660000in}}{\pgfqpoint{4.620000in}{4.620000in}}%
\pgfusepath{clip}%
\pgfsetbuttcap%
\pgfsetroundjoin%
\definecolor{currentfill}{rgb}{1.000000,0.894118,0.788235}%
\pgfsetfillcolor{currentfill}%
\pgfsetlinewidth{0.000000pt}%
\definecolor{currentstroke}{rgb}{1.000000,0.894118,0.788235}%
\pgfsetstrokecolor{currentstroke}%
\pgfsetdash{}{0pt}%
\pgfpathmoveto{\pgfqpoint{2.472452in}{2.522519in}}%
\pgfpathlineto{\pgfqpoint{2.346207in}{2.449631in}}%
\pgfpathlineto{\pgfqpoint{2.472452in}{2.376743in}}%
\pgfpathlineto{\pgfqpoint{2.598697in}{2.449631in}}%
\pgfpathlineto{\pgfqpoint{2.472452in}{2.522519in}}%
\pgfpathclose%
\pgfusepath{fill}%
\end{pgfscope}%
\begin{pgfscope}%
\pgfpathrectangle{\pgfqpoint{0.765000in}{0.660000in}}{\pgfqpoint{4.620000in}{4.620000in}}%
\pgfusepath{clip}%
\pgfsetbuttcap%
\pgfsetroundjoin%
\definecolor{currentfill}{rgb}{1.000000,0.894118,0.788235}%
\pgfsetfillcolor{currentfill}%
\pgfsetlinewidth{0.000000pt}%
\definecolor{currentstroke}{rgb}{1.000000,0.894118,0.788235}%
\pgfsetstrokecolor{currentstroke}%
\pgfsetdash{}{0pt}%
\pgfpathmoveto{\pgfqpoint{2.472452in}{2.522519in}}%
\pgfpathlineto{\pgfqpoint{2.346207in}{2.449631in}}%
\pgfpathlineto{\pgfqpoint{2.346207in}{2.595406in}}%
\pgfpathlineto{\pgfqpoint{2.472452in}{2.668294in}}%
\pgfpathlineto{\pgfqpoint{2.472452in}{2.522519in}}%
\pgfpathclose%
\pgfusepath{fill}%
\end{pgfscope}%
\begin{pgfscope}%
\pgfpathrectangle{\pgfqpoint{0.765000in}{0.660000in}}{\pgfqpoint{4.620000in}{4.620000in}}%
\pgfusepath{clip}%
\pgfsetbuttcap%
\pgfsetroundjoin%
\definecolor{currentfill}{rgb}{1.000000,0.894118,0.788235}%
\pgfsetfillcolor{currentfill}%
\pgfsetlinewidth{0.000000pt}%
\definecolor{currentstroke}{rgb}{1.000000,0.894118,0.788235}%
\pgfsetstrokecolor{currentstroke}%
\pgfsetdash{}{0pt}%
\pgfpathmoveto{\pgfqpoint{2.472452in}{2.522519in}}%
\pgfpathlineto{\pgfqpoint{2.598697in}{2.449631in}}%
\pgfpathlineto{\pgfqpoint{2.598697in}{2.595406in}}%
\pgfpathlineto{\pgfqpoint{2.472452in}{2.668294in}}%
\pgfpathlineto{\pgfqpoint{2.472452in}{2.522519in}}%
\pgfpathclose%
\pgfusepath{fill}%
\end{pgfscope}%
\begin{pgfscope}%
\pgfpathrectangle{\pgfqpoint{0.765000in}{0.660000in}}{\pgfqpoint{4.620000in}{4.620000in}}%
\pgfusepath{clip}%
\pgfsetbuttcap%
\pgfsetroundjoin%
\definecolor{currentfill}{rgb}{1.000000,0.894118,0.788235}%
\pgfsetfillcolor{currentfill}%
\pgfsetlinewidth{0.000000pt}%
\definecolor{currentstroke}{rgb}{1.000000,0.894118,0.788235}%
\pgfsetstrokecolor{currentstroke}%
\pgfsetdash{}{0pt}%
\pgfpathmoveto{\pgfqpoint{3.324897in}{2.176135in}}%
\pgfpathlineto{\pgfqpoint{3.198652in}{2.103247in}}%
\pgfpathlineto{\pgfqpoint{3.324897in}{2.030360in}}%
\pgfpathlineto{\pgfqpoint{3.451141in}{2.103247in}}%
\pgfpathlineto{\pgfqpoint{3.324897in}{2.176135in}}%
\pgfpathclose%
\pgfusepath{fill}%
\end{pgfscope}%
\begin{pgfscope}%
\pgfpathrectangle{\pgfqpoint{0.765000in}{0.660000in}}{\pgfqpoint{4.620000in}{4.620000in}}%
\pgfusepath{clip}%
\pgfsetbuttcap%
\pgfsetroundjoin%
\definecolor{currentfill}{rgb}{1.000000,0.894118,0.788235}%
\pgfsetfillcolor{currentfill}%
\pgfsetlinewidth{0.000000pt}%
\definecolor{currentstroke}{rgb}{1.000000,0.894118,0.788235}%
\pgfsetstrokecolor{currentstroke}%
\pgfsetdash{}{0pt}%
\pgfpathmoveto{\pgfqpoint{3.324897in}{1.884585in}}%
\pgfpathlineto{\pgfqpoint{3.451141in}{1.957472in}}%
\pgfpathlineto{\pgfqpoint{3.451141in}{2.103247in}}%
\pgfpathlineto{\pgfqpoint{3.324897in}{2.030360in}}%
\pgfpathlineto{\pgfqpoint{3.324897in}{1.884585in}}%
\pgfpathclose%
\pgfusepath{fill}%
\end{pgfscope}%
\begin{pgfscope}%
\pgfpathrectangle{\pgfqpoint{0.765000in}{0.660000in}}{\pgfqpoint{4.620000in}{4.620000in}}%
\pgfusepath{clip}%
\pgfsetbuttcap%
\pgfsetroundjoin%
\definecolor{currentfill}{rgb}{1.000000,0.894118,0.788235}%
\pgfsetfillcolor{currentfill}%
\pgfsetlinewidth{0.000000pt}%
\definecolor{currentstroke}{rgb}{1.000000,0.894118,0.788235}%
\pgfsetstrokecolor{currentstroke}%
\pgfsetdash{}{0pt}%
\pgfpathmoveto{\pgfqpoint{3.198652in}{1.957472in}}%
\pgfpathlineto{\pgfqpoint{3.324897in}{1.884585in}}%
\pgfpathlineto{\pgfqpoint{3.324897in}{2.030360in}}%
\pgfpathlineto{\pgfqpoint{3.198652in}{2.103247in}}%
\pgfpathlineto{\pgfqpoint{3.198652in}{1.957472in}}%
\pgfpathclose%
\pgfusepath{fill}%
\end{pgfscope}%
\begin{pgfscope}%
\pgfpathrectangle{\pgfqpoint{0.765000in}{0.660000in}}{\pgfqpoint{4.620000in}{4.620000in}}%
\pgfusepath{clip}%
\pgfsetbuttcap%
\pgfsetroundjoin%
\definecolor{currentfill}{rgb}{1.000000,0.894118,0.788235}%
\pgfsetfillcolor{currentfill}%
\pgfsetlinewidth{0.000000pt}%
\definecolor{currentstroke}{rgb}{1.000000,0.894118,0.788235}%
\pgfsetstrokecolor{currentstroke}%
\pgfsetdash{}{0pt}%
\pgfpathmoveto{\pgfqpoint{2.472452in}{2.668294in}}%
\pgfpathlineto{\pgfqpoint{2.346207in}{2.595406in}}%
\pgfpathlineto{\pgfqpoint{2.472452in}{2.522519in}}%
\pgfpathlineto{\pgfqpoint{2.598697in}{2.595406in}}%
\pgfpathlineto{\pgfqpoint{2.472452in}{2.668294in}}%
\pgfpathclose%
\pgfusepath{fill}%
\end{pgfscope}%
\begin{pgfscope}%
\pgfpathrectangle{\pgfqpoint{0.765000in}{0.660000in}}{\pgfqpoint{4.620000in}{4.620000in}}%
\pgfusepath{clip}%
\pgfsetbuttcap%
\pgfsetroundjoin%
\definecolor{currentfill}{rgb}{1.000000,0.894118,0.788235}%
\pgfsetfillcolor{currentfill}%
\pgfsetlinewidth{0.000000pt}%
\definecolor{currentstroke}{rgb}{1.000000,0.894118,0.788235}%
\pgfsetstrokecolor{currentstroke}%
\pgfsetdash{}{0pt}%
\pgfpathmoveto{\pgfqpoint{2.472452in}{2.376743in}}%
\pgfpathlineto{\pgfqpoint{2.598697in}{2.449631in}}%
\pgfpathlineto{\pgfqpoint{2.598697in}{2.595406in}}%
\pgfpathlineto{\pgfqpoint{2.472452in}{2.522519in}}%
\pgfpathlineto{\pgfqpoint{2.472452in}{2.376743in}}%
\pgfpathclose%
\pgfusepath{fill}%
\end{pgfscope}%
\begin{pgfscope}%
\pgfpathrectangle{\pgfqpoint{0.765000in}{0.660000in}}{\pgfqpoint{4.620000in}{4.620000in}}%
\pgfusepath{clip}%
\pgfsetbuttcap%
\pgfsetroundjoin%
\definecolor{currentfill}{rgb}{1.000000,0.894118,0.788235}%
\pgfsetfillcolor{currentfill}%
\pgfsetlinewidth{0.000000pt}%
\definecolor{currentstroke}{rgb}{1.000000,0.894118,0.788235}%
\pgfsetstrokecolor{currentstroke}%
\pgfsetdash{}{0pt}%
\pgfpathmoveto{\pgfqpoint{2.346207in}{2.449631in}}%
\pgfpathlineto{\pgfqpoint{2.472452in}{2.376743in}}%
\pgfpathlineto{\pgfqpoint{2.472452in}{2.522519in}}%
\pgfpathlineto{\pgfqpoint{2.346207in}{2.595406in}}%
\pgfpathlineto{\pgfqpoint{2.346207in}{2.449631in}}%
\pgfpathclose%
\pgfusepath{fill}%
\end{pgfscope}%
\begin{pgfscope}%
\pgfpathrectangle{\pgfqpoint{0.765000in}{0.660000in}}{\pgfqpoint{4.620000in}{4.620000in}}%
\pgfusepath{clip}%
\pgfsetbuttcap%
\pgfsetroundjoin%
\definecolor{currentfill}{rgb}{1.000000,0.894118,0.788235}%
\pgfsetfillcolor{currentfill}%
\pgfsetlinewidth{0.000000pt}%
\definecolor{currentstroke}{rgb}{1.000000,0.894118,0.788235}%
\pgfsetstrokecolor{currentstroke}%
\pgfsetdash{}{0pt}%
\pgfpathmoveto{\pgfqpoint{3.324897in}{2.030360in}}%
\pgfpathlineto{\pgfqpoint{3.198652in}{1.957472in}}%
\pgfpathlineto{\pgfqpoint{2.346207in}{2.449631in}}%
\pgfpathlineto{\pgfqpoint{2.472452in}{2.522519in}}%
\pgfpathlineto{\pgfqpoint{3.324897in}{2.030360in}}%
\pgfpathclose%
\pgfusepath{fill}%
\end{pgfscope}%
\begin{pgfscope}%
\pgfpathrectangle{\pgfqpoint{0.765000in}{0.660000in}}{\pgfqpoint{4.620000in}{4.620000in}}%
\pgfusepath{clip}%
\pgfsetbuttcap%
\pgfsetroundjoin%
\definecolor{currentfill}{rgb}{1.000000,0.894118,0.788235}%
\pgfsetfillcolor{currentfill}%
\pgfsetlinewidth{0.000000pt}%
\definecolor{currentstroke}{rgb}{1.000000,0.894118,0.788235}%
\pgfsetstrokecolor{currentstroke}%
\pgfsetdash{}{0pt}%
\pgfpathmoveto{\pgfqpoint{3.451141in}{1.957472in}}%
\pgfpathlineto{\pgfqpoint{3.324897in}{2.030360in}}%
\pgfpathlineto{\pgfqpoint{2.472452in}{2.522519in}}%
\pgfpathlineto{\pgfqpoint{2.598697in}{2.449631in}}%
\pgfpathlineto{\pgfqpoint{3.451141in}{1.957472in}}%
\pgfpathclose%
\pgfusepath{fill}%
\end{pgfscope}%
\begin{pgfscope}%
\pgfpathrectangle{\pgfqpoint{0.765000in}{0.660000in}}{\pgfqpoint{4.620000in}{4.620000in}}%
\pgfusepath{clip}%
\pgfsetbuttcap%
\pgfsetroundjoin%
\definecolor{currentfill}{rgb}{1.000000,0.894118,0.788235}%
\pgfsetfillcolor{currentfill}%
\pgfsetlinewidth{0.000000pt}%
\definecolor{currentstroke}{rgb}{1.000000,0.894118,0.788235}%
\pgfsetstrokecolor{currentstroke}%
\pgfsetdash{}{0pt}%
\pgfpathmoveto{\pgfqpoint{3.324897in}{2.030360in}}%
\pgfpathlineto{\pgfqpoint{3.324897in}{2.176135in}}%
\pgfpathlineto{\pgfqpoint{2.472452in}{2.668294in}}%
\pgfpathlineto{\pgfqpoint{2.598697in}{2.449631in}}%
\pgfpathlineto{\pgfqpoint{3.324897in}{2.030360in}}%
\pgfpathclose%
\pgfusepath{fill}%
\end{pgfscope}%
\begin{pgfscope}%
\pgfpathrectangle{\pgfqpoint{0.765000in}{0.660000in}}{\pgfqpoint{4.620000in}{4.620000in}}%
\pgfusepath{clip}%
\pgfsetbuttcap%
\pgfsetroundjoin%
\definecolor{currentfill}{rgb}{1.000000,0.894118,0.788235}%
\pgfsetfillcolor{currentfill}%
\pgfsetlinewidth{0.000000pt}%
\definecolor{currentstroke}{rgb}{1.000000,0.894118,0.788235}%
\pgfsetstrokecolor{currentstroke}%
\pgfsetdash{}{0pt}%
\pgfpathmoveto{\pgfqpoint{3.451141in}{1.957472in}}%
\pgfpathlineto{\pgfqpoint{3.451141in}{2.103247in}}%
\pgfpathlineto{\pgfqpoint{2.598697in}{2.595406in}}%
\pgfpathlineto{\pgfqpoint{2.472452in}{2.522519in}}%
\pgfpathlineto{\pgfqpoint{3.451141in}{1.957472in}}%
\pgfpathclose%
\pgfusepath{fill}%
\end{pgfscope}%
\begin{pgfscope}%
\pgfpathrectangle{\pgfqpoint{0.765000in}{0.660000in}}{\pgfqpoint{4.620000in}{4.620000in}}%
\pgfusepath{clip}%
\pgfsetbuttcap%
\pgfsetroundjoin%
\definecolor{currentfill}{rgb}{1.000000,0.894118,0.788235}%
\pgfsetfillcolor{currentfill}%
\pgfsetlinewidth{0.000000pt}%
\definecolor{currentstroke}{rgb}{1.000000,0.894118,0.788235}%
\pgfsetstrokecolor{currentstroke}%
\pgfsetdash{}{0pt}%
\pgfpathmoveto{\pgfqpoint{3.324897in}{1.884585in}}%
\pgfpathlineto{\pgfqpoint{3.451141in}{1.957472in}}%
\pgfpathlineto{\pgfqpoint{2.598697in}{2.449631in}}%
\pgfpathlineto{\pgfqpoint{2.472452in}{2.376743in}}%
\pgfpathlineto{\pgfqpoint{3.324897in}{1.884585in}}%
\pgfpathclose%
\pgfusepath{fill}%
\end{pgfscope}%
\begin{pgfscope}%
\pgfpathrectangle{\pgfqpoint{0.765000in}{0.660000in}}{\pgfqpoint{4.620000in}{4.620000in}}%
\pgfusepath{clip}%
\pgfsetbuttcap%
\pgfsetroundjoin%
\definecolor{currentfill}{rgb}{1.000000,0.894118,0.788235}%
\pgfsetfillcolor{currentfill}%
\pgfsetlinewidth{0.000000pt}%
\definecolor{currentstroke}{rgb}{1.000000,0.894118,0.788235}%
\pgfsetstrokecolor{currentstroke}%
\pgfsetdash{}{0pt}%
\pgfpathmoveto{\pgfqpoint{3.451141in}{2.103247in}}%
\pgfpathlineto{\pgfqpoint{3.324897in}{2.176135in}}%
\pgfpathlineto{\pgfqpoint{2.472452in}{2.668294in}}%
\pgfpathlineto{\pgfqpoint{2.598697in}{2.595406in}}%
\pgfpathlineto{\pgfqpoint{3.451141in}{2.103247in}}%
\pgfpathclose%
\pgfusepath{fill}%
\end{pgfscope}%
\begin{pgfscope}%
\pgfpathrectangle{\pgfqpoint{0.765000in}{0.660000in}}{\pgfqpoint{4.620000in}{4.620000in}}%
\pgfusepath{clip}%
\pgfsetbuttcap%
\pgfsetroundjoin%
\definecolor{currentfill}{rgb}{1.000000,0.894118,0.788235}%
\pgfsetfillcolor{currentfill}%
\pgfsetlinewidth{0.000000pt}%
\definecolor{currentstroke}{rgb}{1.000000,0.894118,0.788235}%
\pgfsetstrokecolor{currentstroke}%
\pgfsetdash{}{0pt}%
\pgfpathmoveto{\pgfqpoint{3.198652in}{1.957472in}}%
\pgfpathlineto{\pgfqpoint{3.324897in}{1.884585in}}%
\pgfpathlineto{\pgfqpoint{2.472452in}{2.376743in}}%
\pgfpathlineto{\pgfqpoint{2.346207in}{2.449631in}}%
\pgfpathlineto{\pgfqpoint{3.198652in}{1.957472in}}%
\pgfpathclose%
\pgfusepath{fill}%
\end{pgfscope}%
\begin{pgfscope}%
\pgfpathrectangle{\pgfqpoint{0.765000in}{0.660000in}}{\pgfqpoint{4.620000in}{4.620000in}}%
\pgfusepath{clip}%
\pgfsetbuttcap%
\pgfsetroundjoin%
\definecolor{currentfill}{rgb}{1.000000,0.894118,0.788235}%
\pgfsetfillcolor{currentfill}%
\pgfsetlinewidth{0.000000pt}%
\definecolor{currentstroke}{rgb}{1.000000,0.894118,0.788235}%
\pgfsetstrokecolor{currentstroke}%
\pgfsetdash{}{0pt}%
\pgfpathmoveto{\pgfqpoint{3.324897in}{2.176135in}}%
\pgfpathlineto{\pgfqpoint{3.198652in}{2.103247in}}%
\pgfpathlineto{\pgfqpoint{2.346207in}{2.595406in}}%
\pgfpathlineto{\pgfqpoint{2.472452in}{2.668294in}}%
\pgfpathlineto{\pgfqpoint{3.324897in}{2.176135in}}%
\pgfpathclose%
\pgfusepath{fill}%
\end{pgfscope}%
\begin{pgfscope}%
\pgfpathrectangle{\pgfqpoint{0.765000in}{0.660000in}}{\pgfqpoint{4.620000in}{4.620000in}}%
\pgfusepath{clip}%
\pgfsetbuttcap%
\pgfsetroundjoin%
\definecolor{currentfill}{rgb}{1.000000,0.894118,0.788235}%
\pgfsetfillcolor{currentfill}%
\pgfsetlinewidth{0.000000pt}%
\definecolor{currentstroke}{rgb}{1.000000,0.894118,0.788235}%
\pgfsetstrokecolor{currentstroke}%
\pgfsetdash{}{0pt}%
\pgfpathmoveto{\pgfqpoint{3.198652in}{1.957472in}}%
\pgfpathlineto{\pgfqpoint{3.198652in}{2.103247in}}%
\pgfpathlineto{\pgfqpoint{2.346207in}{2.595406in}}%
\pgfpathlineto{\pgfqpoint{2.472452in}{2.376743in}}%
\pgfpathlineto{\pgfqpoint{3.198652in}{1.957472in}}%
\pgfpathclose%
\pgfusepath{fill}%
\end{pgfscope}%
\begin{pgfscope}%
\pgfpathrectangle{\pgfqpoint{0.765000in}{0.660000in}}{\pgfqpoint{4.620000in}{4.620000in}}%
\pgfusepath{clip}%
\pgfsetbuttcap%
\pgfsetroundjoin%
\definecolor{currentfill}{rgb}{1.000000,0.894118,0.788235}%
\pgfsetfillcolor{currentfill}%
\pgfsetlinewidth{0.000000pt}%
\definecolor{currentstroke}{rgb}{1.000000,0.894118,0.788235}%
\pgfsetstrokecolor{currentstroke}%
\pgfsetdash{}{0pt}%
\pgfpathmoveto{\pgfqpoint{3.324897in}{1.884585in}}%
\pgfpathlineto{\pgfqpoint{3.324897in}{2.030360in}}%
\pgfpathlineto{\pgfqpoint{2.472452in}{2.522519in}}%
\pgfpathlineto{\pgfqpoint{2.346207in}{2.449631in}}%
\pgfpathlineto{\pgfqpoint{3.324897in}{1.884585in}}%
\pgfpathclose%
\pgfusepath{fill}%
\end{pgfscope}%
\begin{pgfscope}%
\pgfpathrectangle{\pgfqpoint{0.765000in}{0.660000in}}{\pgfqpoint{4.620000in}{4.620000in}}%
\pgfusepath{clip}%
\pgfsetbuttcap%
\pgfsetroundjoin%
\definecolor{currentfill}{rgb}{1.000000,0.894118,0.788235}%
\pgfsetfillcolor{currentfill}%
\pgfsetlinewidth{0.000000pt}%
\definecolor{currentstroke}{rgb}{1.000000,0.894118,0.788235}%
\pgfsetstrokecolor{currentstroke}%
\pgfsetdash{}{0pt}%
\pgfpathmoveto{\pgfqpoint{3.198652in}{2.103247in}}%
\pgfpathlineto{\pgfqpoint{3.324897in}{2.030360in}}%
\pgfpathlineto{\pgfqpoint{2.472452in}{2.522519in}}%
\pgfpathlineto{\pgfqpoint{2.346207in}{2.595406in}}%
\pgfpathlineto{\pgfqpoint{3.198652in}{2.103247in}}%
\pgfpathclose%
\pgfusepath{fill}%
\end{pgfscope}%
\begin{pgfscope}%
\pgfpathrectangle{\pgfqpoint{0.765000in}{0.660000in}}{\pgfqpoint{4.620000in}{4.620000in}}%
\pgfusepath{clip}%
\pgfsetbuttcap%
\pgfsetroundjoin%
\definecolor{currentfill}{rgb}{1.000000,0.894118,0.788235}%
\pgfsetfillcolor{currentfill}%
\pgfsetlinewidth{0.000000pt}%
\definecolor{currentstroke}{rgb}{1.000000,0.894118,0.788235}%
\pgfsetstrokecolor{currentstroke}%
\pgfsetdash{}{0pt}%
\pgfpathmoveto{\pgfqpoint{3.324897in}{2.030360in}}%
\pgfpathlineto{\pgfqpoint{3.451141in}{2.103247in}}%
\pgfpathlineto{\pgfqpoint{2.598697in}{2.595406in}}%
\pgfpathlineto{\pgfqpoint{2.472452in}{2.522519in}}%
\pgfpathlineto{\pgfqpoint{3.324897in}{2.030360in}}%
\pgfpathclose%
\pgfusepath{fill}%
\end{pgfscope}%
\begin{pgfscope}%
\pgfpathrectangle{\pgfqpoint{0.765000in}{0.660000in}}{\pgfqpoint{4.620000in}{4.620000in}}%
\pgfusepath{clip}%
\pgfsetbuttcap%
\pgfsetroundjoin%
\definecolor{currentfill}{rgb}{1.000000,0.894118,0.788235}%
\pgfsetfillcolor{currentfill}%
\pgfsetlinewidth{0.000000pt}%
\definecolor{currentstroke}{rgb}{1.000000,0.894118,0.788235}%
\pgfsetstrokecolor{currentstroke}%
\pgfsetdash{}{0pt}%
\pgfpathmoveto{\pgfqpoint{3.324897in}{2.030360in}}%
\pgfpathlineto{\pgfqpoint{3.198652in}{1.957472in}}%
\pgfpathlineto{\pgfqpoint{3.324897in}{1.884585in}}%
\pgfpathlineto{\pgfqpoint{3.451141in}{1.957472in}}%
\pgfpathlineto{\pgfqpoint{3.324897in}{2.030360in}}%
\pgfpathclose%
\pgfusepath{fill}%
\end{pgfscope}%
\begin{pgfscope}%
\pgfpathrectangle{\pgfqpoint{0.765000in}{0.660000in}}{\pgfqpoint{4.620000in}{4.620000in}}%
\pgfusepath{clip}%
\pgfsetbuttcap%
\pgfsetroundjoin%
\definecolor{currentfill}{rgb}{1.000000,0.894118,0.788235}%
\pgfsetfillcolor{currentfill}%
\pgfsetlinewidth{0.000000pt}%
\definecolor{currentstroke}{rgb}{1.000000,0.894118,0.788235}%
\pgfsetstrokecolor{currentstroke}%
\pgfsetdash{}{0pt}%
\pgfpathmoveto{\pgfqpoint{3.324897in}{2.030360in}}%
\pgfpathlineto{\pgfqpoint{3.198652in}{1.957472in}}%
\pgfpathlineto{\pgfqpoint{3.198652in}{2.103247in}}%
\pgfpathlineto{\pgfqpoint{3.324897in}{2.176135in}}%
\pgfpathlineto{\pgfqpoint{3.324897in}{2.030360in}}%
\pgfpathclose%
\pgfusepath{fill}%
\end{pgfscope}%
\begin{pgfscope}%
\pgfpathrectangle{\pgfqpoint{0.765000in}{0.660000in}}{\pgfqpoint{4.620000in}{4.620000in}}%
\pgfusepath{clip}%
\pgfsetbuttcap%
\pgfsetroundjoin%
\definecolor{currentfill}{rgb}{1.000000,0.894118,0.788235}%
\pgfsetfillcolor{currentfill}%
\pgfsetlinewidth{0.000000pt}%
\definecolor{currentstroke}{rgb}{1.000000,0.894118,0.788235}%
\pgfsetstrokecolor{currentstroke}%
\pgfsetdash{}{0pt}%
\pgfpathmoveto{\pgfqpoint{3.324897in}{2.030360in}}%
\pgfpathlineto{\pgfqpoint{3.451141in}{1.957472in}}%
\pgfpathlineto{\pgfqpoint{3.451141in}{2.103247in}}%
\pgfpathlineto{\pgfqpoint{3.324897in}{2.176135in}}%
\pgfpathlineto{\pgfqpoint{3.324897in}{2.030360in}}%
\pgfpathclose%
\pgfusepath{fill}%
\end{pgfscope}%
\begin{pgfscope}%
\pgfpathrectangle{\pgfqpoint{0.765000in}{0.660000in}}{\pgfqpoint{4.620000in}{4.620000in}}%
\pgfusepath{clip}%
\pgfsetbuttcap%
\pgfsetroundjoin%
\definecolor{currentfill}{rgb}{1.000000,0.894118,0.788235}%
\pgfsetfillcolor{currentfill}%
\pgfsetlinewidth{0.000000pt}%
\definecolor{currentstroke}{rgb}{1.000000,0.894118,0.788235}%
\pgfsetstrokecolor{currentstroke}%
\pgfsetdash{}{0pt}%
\pgfpathmoveto{\pgfqpoint{3.860367in}{2.339514in}}%
\pgfpathlineto{\pgfqpoint{3.734122in}{2.266626in}}%
\pgfpathlineto{\pgfqpoint{3.860367in}{2.193739in}}%
\pgfpathlineto{\pgfqpoint{3.986612in}{2.266626in}}%
\pgfpathlineto{\pgfqpoint{3.860367in}{2.339514in}}%
\pgfpathclose%
\pgfusepath{fill}%
\end{pgfscope}%
\begin{pgfscope}%
\pgfpathrectangle{\pgfqpoint{0.765000in}{0.660000in}}{\pgfqpoint{4.620000in}{4.620000in}}%
\pgfusepath{clip}%
\pgfsetbuttcap%
\pgfsetroundjoin%
\definecolor{currentfill}{rgb}{1.000000,0.894118,0.788235}%
\pgfsetfillcolor{currentfill}%
\pgfsetlinewidth{0.000000pt}%
\definecolor{currentstroke}{rgb}{1.000000,0.894118,0.788235}%
\pgfsetstrokecolor{currentstroke}%
\pgfsetdash{}{0pt}%
\pgfpathmoveto{\pgfqpoint{3.860367in}{2.339514in}}%
\pgfpathlineto{\pgfqpoint{3.734122in}{2.266626in}}%
\pgfpathlineto{\pgfqpoint{3.734122in}{2.412401in}}%
\pgfpathlineto{\pgfqpoint{3.860367in}{2.485289in}}%
\pgfpathlineto{\pgfqpoint{3.860367in}{2.339514in}}%
\pgfpathclose%
\pgfusepath{fill}%
\end{pgfscope}%
\begin{pgfscope}%
\pgfpathrectangle{\pgfqpoint{0.765000in}{0.660000in}}{\pgfqpoint{4.620000in}{4.620000in}}%
\pgfusepath{clip}%
\pgfsetbuttcap%
\pgfsetroundjoin%
\definecolor{currentfill}{rgb}{1.000000,0.894118,0.788235}%
\pgfsetfillcolor{currentfill}%
\pgfsetlinewidth{0.000000pt}%
\definecolor{currentstroke}{rgb}{1.000000,0.894118,0.788235}%
\pgfsetstrokecolor{currentstroke}%
\pgfsetdash{}{0pt}%
\pgfpathmoveto{\pgfqpoint{3.860367in}{2.339514in}}%
\pgfpathlineto{\pgfqpoint{3.986612in}{2.266626in}}%
\pgfpathlineto{\pgfqpoint{3.986612in}{2.412401in}}%
\pgfpathlineto{\pgfqpoint{3.860367in}{2.485289in}}%
\pgfpathlineto{\pgfqpoint{3.860367in}{2.339514in}}%
\pgfpathclose%
\pgfusepath{fill}%
\end{pgfscope}%
\begin{pgfscope}%
\pgfpathrectangle{\pgfqpoint{0.765000in}{0.660000in}}{\pgfqpoint{4.620000in}{4.620000in}}%
\pgfusepath{clip}%
\pgfsetbuttcap%
\pgfsetroundjoin%
\definecolor{currentfill}{rgb}{1.000000,0.894118,0.788235}%
\pgfsetfillcolor{currentfill}%
\pgfsetlinewidth{0.000000pt}%
\definecolor{currentstroke}{rgb}{1.000000,0.894118,0.788235}%
\pgfsetstrokecolor{currentstroke}%
\pgfsetdash{}{0pt}%
\pgfpathmoveto{\pgfqpoint{3.324897in}{2.176135in}}%
\pgfpathlineto{\pgfqpoint{3.198652in}{2.103247in}}%
\pgfpathlineto{\pgfqpoint{3.324897in}{2.030360in}}%
\pgfpathlineto{\pgfqpoint{3.451141in}{2.103247in}}%
\pgfpathlineto{\pgfqpoint{3.324897in}{2.176135in}}%
\pgfpathclose%
\pgfusepath{fill}%
\end{pgfscope}%
\begin{pgfscope}%
\pgfpathrectangle{\pgfqpoint{0.765000in}{0.660000in}}{\pgfqpoint{4.620000in}{4.620000in}}%
\pgfusepath{clip}%
\pgfsetbuttcap%
\pgfsetroundjoin%
\definecolor{currentfill}{rgb}{1.000000,0.894118,0.788235}%
\pgfsetfillcolor{currentfill}%
\pgfsetlinewidth{0.000000pt}%
\definecolor{currentstroke}{rgb}{1.000000,0.894118,0.788235}%
\pgfsetstrokecolor{currentstroke}%
\pgfsetdash{}{0pt}%
\pgfpathmoveto{\pgfqpoint{3.324897in}{1.884585in}}%
\pgfpathlineto{\pgfqpoint{3.451141in}{1.957472in}}%
\pgfpathlineto{\pgfqpoint{3.451141in}{2.103247in}}%
\pgfpathlineto{\pgfqpoint{3.324897in}{2.030360in}}%
\pgfpathlineto{\pgfqpoint{3.324897in}{1.884585in}}%
\pgfpathclose%
\pgfusepath{fill}%
\end{pgfscope}%
\begin{pgfscope}%
\pgfpathrectangle{\pgfqpoint{0.765000in}{0.660000in}}{\pgfqpoint{4.620000in}{4.620000in}}%
\pgfusepath{clip}%
\pgfsetbuttcap%
\pgfsetroundjoin%
\definecolor{currentfill}{rgb}{1.000000,0.894118,0.788235}%
\pgfsetfillcolor{currentfill}%
\pgfsetlinewidth{0.000000pt}%
\definecolor{currentstroke}{rgb}{1.000000,0.894118,0.788235}%
\pgfsetstrokecolor{currentstroke}%
\pgfsetdash{}{0pt}%
\pgfpathmoveto{\pgfqpoint{3.198652in}{1.957472in}}%
\pgfpathlineto{\pgfqpoint{3.324897in}{1.884585in}}%
\pgfpathlineto{\pgfqpoint{3.324897in}{2.030360in}}%
\pgfpathlineto{\pgfqpoint{3.198652in}{2.103247in}}%
\pgfpathlineto{\pgfqpoint{3.198652in}{1.957472in}}%
\pgfpathclose%
\pgfusepath{fill}%
\end{pgfscope}%
\begin{pgfscope}%
\pgfpathrectangle{\pgfqpoint{0.765000in}{0.660000in}}{\pgfqpoint{4.620000in}{4.620000in}}%
\pgfusepath{clip}%
\pgfsetbuttcap%
\pgfsetroundjoin%
\definecolor{currentfill}{rgb}{1.000000,0.894118,0.788235}%
\pgfsetfillcolor{currentfill}%
\pgfsetlinewidth{0.000000pt}%
\definecolor{currentstroke}{rgb}{1.000000,0.894118,0.788235}%
\pgfsetstrokecolor{currentstroke}%
\pgfsetdash{}{0pt}%
\pgfpathmoveto{\pgfqpoint{3.860367in}{2.485289in}}%
\pgfpathlineto{\pgfqpoint{3.734122in}{2.412401in}}%
\pgfpathlineto{\pgfqpoint{3.860367in}{2.339514in}}%
\pgfpathlineto{\pgfqpoint{3.986612in}{2.412401in}}%
\pgfpathlineto{\pgfqpoint{3.860367in}{2.485289in}}%
\pgfpathclose%
\pgfusepath{fill}%
\end{pgfscope}%
\begin{pgfscope}%
\pgfpathrectangle{\pgfqpoint{0.765000in}{0.660000in}}{\pgfqpoint{4.620000in}{4.620000in}}%
\pgfusepath{clip}%
\pgfsetbuttcap%
\pgfsetroundjoin%
\definecolor{currentfill}{rgb}{1.000000,0.894118,0.788235}%
\pgfsetfillcolor{currentfill}%
\pgfsetlinewidth{0.000000pt}%
\definecolor{currentstroke}{rgb}{1.000000,0.894118,0.788235}%
\pgfsetstrokecolor{currentstroke}%
\pgfsetdash{}{0pt}%
\pgfpathmoveto{\pgfqpoint{3.860367in}{2.193739in}}%
\pgfpathlineto{\pgfqpoint{3.986612in}{2.266626in}}%
\pgfpathlineto{\pgfqpoint{3.986612in}{2.412401in}}%
\pgfpathlineto{\pgfqpoint{3.860367in}{2.339514in}}%
\pgfpathlineto{\pgfqpoint{3.860367in}{2.193739in}}%
\pgfpathclose%
\pgfusepath{fill}%
\end{pgfscope}%
\begin{pgfscope}%
\pgfpathrectangle{\pgfqpoint{0.765000in}{0.660000in}}{\pgfqpoint{4.620000in}{4.620000in}}%
\pgfusepath{clip}%
\pgfsetbuttcap%
\pgfsetroundjoin%
\definecolor{currentfill}{rgb}{1.000000,0.894118,0.788235}%
\pgfsetfillcolor{currentfill}%
\pgfsetlinewidth{0.000000pt}%
\definecolor{currentstroke}{rgb}{1.000000,0.894118,0.788235}%
\pgfsetstrokecolor{currentstroke}%
\pgfsetdash{}{0pt}%
\pgfpathmoveto{\pgfqpoint{3.734122in}{2.266626in}}%
\pgfpathlineto{\pgfqpoint{3.860367in}{2.193739in}}%
\pgfpathlineto{\pgfqpoint{3.860367in}{2.339514in}}%
\pgfpathlineto{\pgfqpoint{3.734122in}{2.412401in}}%
\pgfpathlineto{\pgfqpoint{3.734122in}{2.266626in}}%
\pgfpathclose%
\pgfusepath{fill}%
\end{pgfscope}%
\begin{pgfscope}%
\pgfpathrectangle{\pgfqpoint{0.765000in}{0.660000in}}{\pgfqpoint{4.620000in}{4.620000in}}%
\pgfusepath{clip}%
\pgfsetbuttcap%
\pgfsetroundjoin%
\definecolor{currentfill}{rgb}{1.000000,0.894118,0.788235}%
\pgfsetfillcolor{currentfill}%
\pgfsetlinewidth{0.000000pt}%
\definecolor{currentstroke}{rgb}{1.000000,0.894118,0.788235}%
\pgfsetstrokecolor{currentstroke}%
\pgfsetdash{}{0pt}%
\pgfpathmoveto{\pgfqpoint{3.324897in}{2.030360in}}%
\pgfpathlineto{\pgfqpoint{3.198652in}{1.957472in}}%
\pgfpathlineto{\pgfqpoint{3.734122in}{2.266626in}}%
\pgfpathlineto{\pgfqpoint{3.860367in}{2.339514in}}%
\pgfpathlineto{\pgfqpoint{3.324897in}{2.030360in}}%
\pgfpathclose%
\pgfusepath{fill}%
\end{pgfscope}%
\begin{pgfscope}%
\pgfpathrectangle{\pgfqpoint{0.765000in}{0.660000in}}{\pgfqpoint{4.620000in}{4.620000in}}%
\pgfusepath{clip}%
\pgfsetbuttcap%
\pgfsetroundjoin%
\definecolor{currentfill}{rgb}{1.000000,0.894118,0.788235}%
\pgfsetfillcolor{currentfill}%
\pgfsetlinewidth{0.000000pt}%
\definecolor{currentstroke}{rgb}{1.000000,0.894118,0.788235}%
\pgfsetstrokecolor{currentstroke}%
\pgfsetdash{}{0pt}%
\pgfpathmoveto{\pgfqpoint{3.451141in}{1.957472in}}%
\pgfpathlineto{\pgfqpoint{3.324897in}{2.030360in}}%
\pgfpathlineto{\pgfqpoint{3.860367in}{2.339514in}}%
\pgfpathlineto{\pgfqpoint{3.986612in}{2.266626in}}%
\pgfpathlineto{\pgfqpoint{3.451141in}{1.957472in}}%
\pgfpathclose%
\pgfusepath{fill}%
\end{pgfscope}%
\begin{pgfscope}%
\pgfpathrectangle{\pgfqpoint{0.765000in}{0.660000in}}{\pgfqpoint{4.620000in}{4.620000in}}%
\pgfusepath{clip}%
\pgfsetbuttcap%
\pgfsetroundjoin%
\definecolor{currentfill}{rgb}{1.000000,0.894118,0.788235}%
\pgfsetfillcolor{currentfill}%
\pgfsetlinewidth{0.000000pt}%
\definecolor{currentstroke}{rgb}{1.000000,0.894118,0.788235}%
\pgfsetstrokecolor{currentstroke}%
\pgfsetdash{}{0pt}%
\pgfpathmoveto{\pgfqpoint{3.324897in}{2.030360in}}%
\pgfpathlineto{\pgfqpoint{3.324897in}{2.176135in}}%
\pgfpathlineto{\pgfqpoint{3.860367in}{2.485289in}}%
\pgfpathlineto{\pgfqpoint{3.986612in}{2.266626in}}%
\pgfpathlineto{\pgfqpoint{3.324897in}{2.030360in}}%
\pgfpathclose%
\pgfusepath{fill}%
\end{pgfscope}%
\begin{pgfscope}%
\pgfpathrectangle{\pgfqpoint{0.765000in}{0.660000in}}{\pgfqpoint{4.620000in}{4.620000in}}%
\pgfusepath{clip}%
\pgfsetbuttcap%
\pgfsetroundjoin%
\definecolor{currentfill}{rgb}{1.000000,0.894118,0.788235}%
\pgfsetfillcolor{currentfill}%
\pgfsetlinewidth{0.000000pt}%
\definecolor{currentstroke}{rgb}{1.000000,0.894118,0.788235}%
\pgfsetstrokecolor{currentstroke}%
\pgfsetdash{}{0pt}%
\pgfpathmoveto{\pgfqpoint{3.451141in}{1.957472in}}%
\pgfpathlineto{\pgfqpoint{3.451141in}{2.103247in}}%
\pgfpathlineto{\pgfqpoint{3.986612in}{2.412401in}}%
\pgfpathlineto{\pgfqpoint{3.860367in}{2.339514in}}%
\pgfpathlineto{\pgfqpoint{3.451141in}{1.957472in}}%
\pgfpathclose%
\pgfusepath{fill}%
\end{pgfscope}%
\begin{pgfscope}%
\pgfpathrectangle{\pgfqpoint{0.765000in}{0.660000in}}{\pgfqpoint{4.620000in}{4.620000in}}%
\pgfusepath{clip}%
\pgfsetbuttcap%
\pgfsetroundjoin%
\definecolor{currentfill}{rgb}{1.000000,0.894118,0.788235}%
\pgfsetfillcolor{currentfill}%
\pgfsetlinewidth{0.000000pt}%
\definecolor{currentstroke}{rgb}{1.000000,0.894118,0.788235}%
\pgfsetstrokecolor{currentstroke}%
\pgfsetdash{}{0pt}%
\pgfpathmoveto{\pgfqpoint{3.324897in}{1.884585in}}%
\pgfpathlineto{\pgfqpoint{3.451141in}{1.957472in}}%
\pgfpathlineto{\pgfqpoint{3.986612in}{2.266626in}}%
\pgfpathlineto{\pgfqpoint{3.860367in}{2.193739in}}%
\pgfpathlineto{\pgfqpoint{3.324897in}{1.884585in}}%
\pgfpathclose%
\pgfusepath{fill}%
\end{pgfscope}%
\begin{pgfscope}%
\pgfpathrectangle{\pgfqpoint{0.765000in}{0.660000in}}{\pgfqpoint{4.620000in}{4.620000in}}%
\pgfusepath{clip}%
\pgfsetbuttcap%
\pgfsetroundjoin%
\definecolor{currentfill}{rgb}{1.000000,0.894118,0.788235}%
\pgfsetfillcolor{currentfill}%
\pgfsetlinewidth{0.000000pt}%
\definecolor{currentstroke}{rgb}{1.000000,0.894118,0.788235}%
\pgfsetstrokecolor{currentstroke}%
\pgfsetdash{}{0pt}%
\pgfpathmoveto{\pgfqpoint{3.451141in}{2.103247in}}%
\pgfpathlineto{\pgfqpoint{3.324897in}{2.176135in}}%
\pgfpathlineto{\pgfqpoint{3.860367in}{2.485289in}}%
\pgfpathlineto{\pgfqpoint{3.986612in}{2.412401in}}%
\pgfpathlineto{\pgfqpoint{3.451141in}{2.103247in}}%
\pgfpathclose%
\pgfusepath{fill}%
\end{pgfscope}%
\begin{pgfscope}%
\pgfpathrectangle{\pgfqpoint{0.765000in}{0.660000in}}{\pgfqpoint{4.620000in}{4.620000in}}%
\pgfusepath{clip}%
\pgfsetbuttcap%
\pgfsetroundjoin%
\definecolor{currentfill}{rgb}{1.000000,0.894118,0.788235}%
\pgfsetfillcolor{currentfill}%
\pgfsetlinewidth{0.000000pt}%
\definecolor{currentstroke}{rgb}{1.000000,0.894118,0.788235}%
\pgfsetstrokecolor{currentstroke}%
\pgfsetdash{}{0pt}%
\pgfpathmoveto{\pgfqpoint{3.198652in}{1.957472in}}%
\pgfpathlineto{\pgfqpoint{3.324897in}{1.884585in}}%
\pgfpathlineto{\pgfqpoint{3.860367in}{2.193739in}}%
\pgfpathlineto{\pgfqpoint{3.734122in}{2.266626in}}%
\pgfpathlineto{\pgfqpoint{3.198652in}{1.957472in}}%
\pgfpathclose%
\pgfusepath{fill}%
\end{pgfscope}%
\begin{pgfscope}%
\pgfpathrectangle{\pgfqpoint{0.765000in}{0.660000in}}{\pgfqpoint{4.620000in}{4.620000in}}%
\pgfusepath{clip}%
\pgfsetbuttcap%
\pgfsetroundjoin%
\definecolor{currentfill}{rgb}{1.000000,0.894118,0.788235}%
\pgfsetfillcolor{currentfill}%
\pgfsetlinewidth{0.000000pt}%
\definecolor{currentstroke}{rgb}{1.000000,0.894118,0.788235}%
\pgfsetstrokecolor{currentstroke}%
\pgfsetdash{}{0pt}%
\pgfpathmoveto{\pgfqpoint{3.324897in}{2.176135in}}%
\pgfpathlineto{\pgfqpoint{3.198652in}{2.103247in}}%
\pgfpathlineto{\pgfqpoint{3.734122in}{2.412401in}}%
\pgfpathlineto{\pgfqpoint{3.860367in}{2.485289in}}%
\pgfpathlineto{\pgfqpoint{3.324897in}{2.176135in}}%
\pgfpathclose%
\pgfusepath{fill}%
\end{pgfscope}%
\begin{pgfscope}%
\pgfpathrectangle{\pgfqpoint{0.765000in}{0.660000in}}{\pgfqpoint{4.620000in}{4.620000in}}%
\pgfusepath{clip}%
\pgfsetbuttcap%
\pgfsetroundjoin%
\definecolor{currentfill}{rgb}{1.000000,0.894118,0.788235}%
\pgfsetfillcolor{currentfill}%
\pgfsetlinewidth{0.000000pt}%
\definecolor{currentstroke}{rgb}{1.000000,0.894118,0.788235}%
\pgfsetstrokecolor{currentstroke}%
\pgfsetdash{}{0pt}%
\pgfpathmoveto{\pgfqpoint{3.198652in}{1.957472in}}%
\pgfpathlineto{\pgfqpoint{3.198652in}{2.103247in}}%
\pgfpathlineto{\pgfqpoint{3.734122in}{2.412401in}}%
\pgfpathlineto{\pgfqpoint{3.860367in}{2.193739in}}%
\pgfpathlineto{\pgfqpoint{3.198652in}{1.957472in}}%
\pgfpathclose%
\pgfusepath{fill}%
\end{pgfscope}%
\begin{pgfscope}%
\pgfpathrectangle{\pgfqpoint{0.765000in}{0.660000in}}{\pgfqpoint{4.620000in}{4.620000in}}%
\pgfusepath{clip}%
\pgfsetbuttcap%
\pgfsetroundjoin%
\definecolor{currentfill}{rgb}{1.000000,0.894118,0.788235}%
\pgfsetfillcolor{currentfill}%
\pgfsetlinewidth{0.000000pt}%
\definecolor{currentstroke}{rgb}{1.000000,0.894118,0.788235}%
\pgfsetstrokecolor{currentstroke}%
\pgfsetdash{}{0pt}%
\pgfpathmoveto{\pgfqpoint{3.324897in}{1.884585in}}%
\pgfpathlineto{\pgfqpoint{3.324897in}{2.030360in}}%
\pgfpathlineto{\pgfqpoint{3.860367in}{2.339514in}}%
\pgfpathlineto{\pgfqpoint{3.734122in}{2.266626in}}%
\pgfpathlineto{\pgfqpoint{3.324897in}{1.884585in}}%
\pgfpathclose%
\pgfusepath{fill}%
\end{pgfscope}%
\begin{pgfscope}%
\pgfpathrectangle{\pgfqpoint{0.765000in}{0.660000in}}{\pgfqpoint{4.620000in}{4.620000in}}%
\pgfusepath{clip}%
\pgfsetbuttcap%
\pgfsetroundjoin%
\definecolor{currentfill}{rgb}{1.000000,0.894118,0.788235}%
\pgfsetfillcolor{currentfill}%
\pgfsetlinewidth{0.000000pt}%
\definecolor{currentstroke}{rgb}{1.000000,0.894118,0.788235}%
\pgfsetstrokecolor{currentstroke}%
\pgfsetdash{}{0pt}%
\pgfpathmoveto{\pgfqpoint{3.198652in}{2.103247in}}%
\pgfpathlineto{\pgfqpoint{3.324897in}{2.030360in}}%
\pgfpathlineto{\pgfqpoint{3.860367in}{2.339514in}}%
\pgfpathlineto{\pgfqpoint{3.734122in}{2.412401in}}%
\pgfpathlineto{\pgfqpoint{3.198652in}{2.103247in}}%
\pgfpathclose%
\pgfusepath{fill}%
\end{pgfscope}%
\begin{pgfscope}%
\pgfpathrectangle{\pgfqpoint{0.765000in}{0.660000in}}{\pgfqpoint{4.620000in}{4.620000in}}%
\pgfusepath{clip}%
\pgfsetbuttcap%
\pgfsetroundjoin%
\definecolor{currentfill}{rgb}{1.000000,0.894118,0.788235}%
\pgfsetfillcolor{currentfill}%
\pgfsetlinewidth{0.000000pt}%
\definecolor{currentstroke}{rgb}{1.000000,0.894118,0.788235}%
\pgfsetstrokecolor{currentstroke}%
\pgfsetdash{}{0pt}%
\pgfpathmoveto{\pgfqpoint{3.324897in}{2.030360in}}%
\pgfpathlineto{\pgfqpoint{3.451141in}{2.103247in}}%
\pgfpathlineto{\pgfqpoint{3.986612in}{2.412401in}}%
\pgfpathlineto{\pgfqpoint{3.860367in}{2.339514in}}%
\pgfpathlineto{\pgfqpoint{3.324897in}{2.030360in}}%
\pgfpathclose%
\pgfusepath{fill}%
\end{pgfscope}%
\begin{pgfscope}%
\pgfpathrectangle{\pgfqpoint{0.765000in}{0.660000in}}{\pgfqpoint{4.620000in}{4.620000in}}%
\pgfusepath{clip}%
\pgfsetbuttcap%
\pgfsetroundjoin%
\definecolor{currentfill}{rgb}{1.000000,0.894118,0.788235}%
\pgfsetfillcolor{currentfill}%
\pgfsetlinewidth{0.000000pt}%
\definecolor{currentstroke}{rgb}{1.000000,0.894118,0.788235}%
\pgfsetstrokecolor{currentstroke}%
\pgfsetdash{}{0pt}%
\pgfpathmoveto{\pgfqpoint{2.472452in}{3.309215in}}%
\pgfpathlineto{\pgfqpoint{2.346207in}{3.236328in}}%
\pgfpathlineto{\pgfqpoint{2.472452in}{3.163440in}}%
\pgfpathlineto{\pgfqpoint{2.598697in}{3.236328in}}%
\pgfpathlineto{\pgfqpoint{2.472452in}{3.309215in}}%
\pgfpathclose%
\pgfusepath{fill}%
\end{pgfscope}%
\begin{pgfscope}%
\pgfpathrectangle{\pgfqpoint{0.765000in}{0.660000in}}{\pgfqpoint{4.620000in}{4.620000in}}%
\pgfusepath{clip}%
\pgfsetbuttcap%
\pgfsetroundjoin%
\definecolor{currentfill}{rgb}{1.000000,0.894118,0.788235}%
\pgfsetfillcolor{currentfill}%
\pgfsetlinewidth{0.000000pt}%
\definecolor{currentstroke}{rgb}{1.000000,0.894118,0.788235}%
\pgfsetstrokecolor{currentstroke}%
\pgfsetdash{}{0pt}%
\pgfpathmoveto{\pgfqpoint{2.472452in}{3.309215in}}%
\pgfpathlineto{\pgfqpoint{2.346207in}{3.236328in}}%
\pgfpathlineto{\pgfqpoint{2.346207in}{3.382103in}}%
\pgfpathlineto{\pgfqpoint{2.472452in}{3.454990in}}%
\pgfpathlineto{\pgfqpoint{2.472452in}{3.309215in}}%
\pgfpathclose%
\pgfusepath{fill}%
\end{pgfscope}%
\begin{pgfscope}%
\pgfpathrectangle{\pgfqpoint{0.765000in}{0.660000in}}{\pgfqpoint{4.620000in}{4.620000in}}%
\pgfusepath{clip}%
\pgfsetbuttcap%
\pgfsetroundjoin%
\definecolor{currentfill}{rgb}{1.000000,0.894118,0.788235}%
\pgfsetfillcolor{currentfill}%
\pgfsetlinewidth{0.000000pt}%
\definecolor{currentstroke}{rgb}{1.000000,0.894118,0.788235}%
\pgfsetstrokecolor{currentstroke}%
\pgfsetdash{}{0pt}%
\pgfpathmoveto{\pgfqpoint{2.472452in}{3.309215in}}%
\pgfpathlineto{\pgfqpoint{2.598697in}{3.236328in}}%
\pgfpathlineto{\pgfqpoint{2.598697in}{3.382103in}}%
\pgfpathlineto{\pgfqpoint{2.472452in}{3.454990in}}%
\pgfpathlineto{\pgfqpoint{2.472452in}{3.309215in}}%
\pgfpathclose%
\pgfusepath{fill}%
\end{pgfscope}%
\begin{pgfscope}%
\pgfpathrectangle{\pgfqpoint{0.765000in}{0.660000in}}{\pgfqpoint{4.620000in}{4.620000in}}%
\pgfusepath{clip}%
\pgfsetbuttcap%
\pgfsetroundjoin%
\definecolor{currentfill}{rgb}{1.000000,0.894118,0.788235}%
\pgfsetfillcolor{currentfill}%
\pgfsetlinewidth{0.000000pt}%
\definecolor{currentstroke}{rgb}{1.000000,0.894118,0.788235}%
\pgfsetstrokecolor{currentstroke}%
\pgfsetdash{}{0pt}%
\pgfpathmoveto{\pgfqpoint{2.472452in}{2.522519in}}%
\pgfpathlineto{\pgfqpoint{2.346207in}{2.449631in}}%
\pgfpathlineto{\pgfqpoint{2.472452in}{2.376743in}}%
\pgfpathlineto{\pgfqpoint{2.598697in}{2.449631in}}%
\pgfpathlineto{\pgfqpoint{2.472452in}{2.522519in}}%
\pgfpathclose%
\pgfusepath{fill}%
\end{pgfscope}%
\begin{pgfscope}%
\pgfpathrectangle{\pgfqpoint{0.765000in}{0.660000in}}{\pgfqpoint{4.620000in}{4.620000in}}%
\pgfusepath{clip}%
\pgfsetbuttcap%
\pgfsetroundjoin%
\definecolor{currentfill}{rgb}{1.000000,0.894118,0.788235}%
\pgfsetfillcolor{currentfill}%
\pgfsetlinewidth{0.000000pt}%
\definecolor{currentstroke}{rgb}{1.000000,0.894118,0.788235}%
\pgfsetstrokecolor{currentstroke}%
\pgfsetdash{}{0pt}%
\pgfpathmoveto{\pgfqpoint{2.472452in}{2.522519in}}%
\pgfpathlineto{\pgfqpoint{2.346207in}{2.449631in}}%
\pgfpathlineto{\pgfqpoint{2.346207in}{2.595406in}}%
\pgfpathlineto{\pgfqpoint{2.472452in}{2.668294in}}%
\pgfpathlineto{\pgfqpoint{2.472452in}{2.522519in}}%
\pgfpathclose%
\pgfusepath{fill}%
\end{pgfscope}%
\begin{pgfscope}%
\pgfpathrectangle{\pgfqpoint{0.765000in}{0.660000in}}{\pgfqpoint{4.620000in}{4.620000in}}%
\pgfusepath{clip}%
\pgfsetbuttcap%
\pgfsetroundjoin%
\definecolor{currentfill}{rgb}{1.000000,0.894118,0.788235}%
\pgfsetfillcolor{currentfill}%
\pgfsetlinewidth{0.000000pt}%
\definecolor{currentstroke}{rgb}{1.000000,0.894118,0.788235}%
\pgfsetstrokecolor{currentstroke}%
\pgfsetdash{}{0pt}%
\pgfpathmoveto{\pgfqpoint{2.472452in}{2.522519in}}%
\pgfpathlineto{\pgfqpoint{2.598697in}{2.449631in}}%
\pgfpathlineto{\pgfqpoint{2.598697in}{2.595406in}}%
\pgfpathlineto{\pgfqpoint{2.472452in}{2.668294in}}%
\pgfpathlineto{\pgfqpoint{2.472452in}{2.522519in}}%
\pgfpathclose%
\pgfusepath{fill}%
\end{pgfscope}%
\begin{pgfscope}%
\pgfpathrectangle{\pgfqpoint{0.765000in}{0.660000in}}{\pgfqpoint{4.620000in}{4.620000in}}%
\pgfusepath{clip}%
\pgfsetbuttcap%
\pgfsetroundjoin%
\definecolor{currentfill}{rgb}{1.000000,0.894118,0.788235}%
\pgfsetfillcolor{currentfill}%
\pgfsetlinewidth{0.000000pt}%
\definecolor{currentstroke}{rgb}{1.000000,0.894118,0.788235}%
\pgfsetstrokecolor{currentstroke}%
\pgfsetdash{}{0pt}%
\pgfpathmoveto{\pgfqpoint{2.472452in}{3.454990in}}%
\pgfpathlineto{\pgfqpoint{2.346207in}{3.382103in}}%
\pgfpathlineto{\pgfqpoint{2.472452in}{3.309215in}}%
\pgfpathlineto{\pgfqpoint{2.598697in}{3.382103in}}%
\pgfpathlineto{\pgfqpoint{2.472452in}{3.454990in}}%
\pgfpathclose%
\pgfusepath{fill}%
\end{pgfscope}%
\begin{pgfscope}%
\pgfpathrectangle{\pgfqpoint{0.765000in}{0.660000in}}{\pgfqpoint{4.620000in}{4.620000in}}%
\pgfusepath{clip}%
\pgfsetbuttcap%
\pgfsetroundjoin%
\definecolor{currentfill}{rgb}{1.000000,0.894118,0.788235}%
\pgfsetfillcolor{currentfill}%
\pgfsetlinewidth{0.000000pt}%
\definecolor{currentstroke}{rgb}{1.000000,0.894118,0.788235}%
\pgfsetstrokecolor{currentstroke}%
\pgfsetdash{}{0pt}%
\pgfpathmoveto{\pgfqpoint{2.472452in}{3.163440in}}%
\pgfpathlineto{\pgfqpoint{2.598697in}{3.236328in}}%
\pgfpathlineto{\pgfqpoint{2.598697in}{3.382103in}}%
\pgfpathlineto{\pgfqpoint{2.472452in}{3.309215in}}%
\pgfpathlineto{\pgfqpoint{2.472452in}{3.163440in}}%
\pgfpathclose%
\pgfusepath{fill}%
\end{pgfscope}%
\begin{pgfscope}%
\pgfpathrectangle{\pgfqpoint{0.765000in}{0.660000in}}{\pgfqpoint{4.620000in}{4.620000in}}%
\pgfusepath{clip}%
\pgfsetbuttcap%
\pgfsetroundjoin%
\definecolor{currentfill}{rgb}{1.000000,0.894118,0.788235}%
\pgfsetfillcolor{currentfill}%
\pgfsetlinewidth{0.000000pt}%
\definecolor{currentstroke}{rgb}{1.000000,0.894118,0.788235}%
\pgfsetstrokecolor{currentstroke}%
\pgfsetdash{}{0pt}%
\pgfpathmoveto{\pgfqpoint{2.346207in}{3.236328in}}%
\pgfpathlineto{\pgfqpoint{2.472452in}{3.163440in}}%
\pgfpathlineto{\pgfqpoint{2.472452in}{3.309215in}}%
\pgfpathlineto{\pgfqpoint{2.346207in}{3.382103in}}%
\pgfpathlineto{\pgfqpoint{2.346207in}{3.236328in}}%
\pgfpathclose%
\pgfusepath{fill}%
\end{pgfscope}%
\begin{pgfscope}%
\pgfpathrectangle{\pgfqpoint{0.765000in}{0.660000in}}{\pgfqpoint{4.620000in}{4.620000in}}%
\pgfusepath{clip}%
\pgfsetbuttcap%
\pgfsetroundjoin%
\definecolor{currentfill}{rgb}{1.000000,0.894118,0.788235}%
\pgfsetfillcolor{currentfill}%
\pgfsetlinewidth{0.000000pt}%
\definecolor{currentstroke}{rgb}{1.000000,0.894118,0.788235}%
\pgfsetstrokecolor{currentstroke}%
\pgfsetdash{}{0pt}%
\pgfpathmoveto{\pgfqpoint{2.472452in}{2.668294in}}%
\pgfpathlineto{\pgfqpoint{2.346207in}{2.595406in}}%
\pgfpathlineto{\pgfqpoint{2.472452in}{2.522519in}}%
\pgfpathlineto{\pgfqpoint{2.598697in}{2.595406in}}%
\pgfpathlineto{\pgfqpoint{2.472452in}{2.668294in}}%
\pgfpathclose%
\pgfusepath{fill}%
\end{pgfscope}%
\begin{pgfscope}%
\pgfpathrectangle{\pgfqpoint{0.765000in}{0.660000in}}{\pgfqpoint{4.620000in}{4.620000in}}%
\pgfusepath{clip}%
\pgfsetbuttcap%
\pgfsetroundjoin%
\definecolor{currentfill}{rgb}{1.000000,0.894118,0.788235}%
\pgfsetfillcolor{currentfill}%
\pgfsetlinewidth{0.000000pt}%
\definecolor{currentstroke}{rgb}{1.000000,0.894118,0.788235}%
\pgfsetstrokecolor{currentstroke}%
\pgfsetdash{}{0pt}%
\pgfpathmoveto{\pgfqpoint{2.472452in}{2.376743in}}%
\pgfpathlineto{\pgfqpoint{2.598697in}{2.449631in}}%
\pgfpathlineto{\pgfqpoint{2.598697in}{2.595406in}}%
\pgfpathlineto{\pgfqpoint{2.472452in}{2.522519in}}%
\pgfpathlineto{\pgfqpoint{2.472452in}{2.376743in}}%
\pgfpathclose%
\pgfusepath{fill}%
\end{pgfscope}%
\begin{pgfscope}%
\pgfpathrectangle{\pgfqpoint{0.765000in}{0.660000in}}{\pgfqpoint{4.620000in}{4.620000in}}%
\pgfusepath{clip}%
\pgfsetbuttcap%
\pgfsetroundjoin%
\definecolor{currentfill}{rgb}{1.000000,0.894118,0.788235}%
\pgfsetfillcolor{currentfill}%
\pgfsetlinewidth{0.000000pt}%
\definecolor{currentstroke}{rgb}{1.000000,0.894118,0.788235}%
\pgfsetstrokecolor{currentstroke}%
\pgfsetdash{}{0pt}%
\pgfpathmoveto{\pgfqpoint{2.346207in}{2.449631in}}%
\pgfpathlineto{\pgfqpoint{2.472452in}{2.376743in}}%
\pgfpathlineto{\pgfqpoint{2.472452in}{2.522519in}}%
\pgfpathlineto{\pgfqpoint{2.346207in}{2.595406in}}%
\pgfpathlineto{\pgfqpoint{2.346207in}{2.449631in}}%
\pgfpathclose%
\pgfusepath{fill}%
\end{pgfscope}%
\begin{pgfscope}%
\pgfpathrectangle{\pgfqpoint{0.765000in}{0.660000in}}{\pgfqpoint{4.620000in}{4.620000in}}%
\pgfusepath{clip}%
\pgfsetbuttcap%
\pgfsetroundjoin%
\definecolor{currentfill}{rgb}{1.000000,0.894118,0.788235}%
\pgfsetfillcolor{currentfill}%
\pgfsetlinewidth{0.000000pt}%
\definecolor{currentstroke}{rgb}{1.000000,0.894118,0.788235}%
\pgfsetstrokecolor{currentstroke}%
\pgfsetdash{}{0pt}%
\pgfpathmoveto{\pgfqpoint{2.472452in}{3.309215in}}%
\pgfpathlineto{\pgfqpoint{2.346207in}{3.236328in}}%
\pgfpathlineto{\pgfqpoint{2.346207in}{2.449631in}}%
\pgfpathlineto{\pgfqpoint{2.472452in}{2.522519in}}%
\pgfpathlineto{\pgfqpoint{2.472452in}{3.309215in}}%
\pgfpathclose%
\pgfusepath{fill}%
\end{pgfscope}%
\begin{pgfscope}%
\pgfpathrectangle{\pgfqpoint{0.765000in}{0.660000in}}{\pgfqpoint{4.620000in}{4.620000in}}%
\pgfusepath{clip}%
\pgfsetbuttcap%
\pgfsetroundjoin%
\definecolor{currentfill}{rgb}{1.000000,0.894118,0.788235}%
\pgfsetfillcolor{currentfill}%
\pgfsetlinewidth{0.000000pt}%
\definecolor{currentstroke}{rgb}{1.000000,0.894118,0.788235}%
\pgfsetstrokecolor{currentstroke}%
\pgfsetdash{}{0pt}%
\pgfpathmoveto{\pgfqpoint{2.598697in}{3.236328in}}%
\pgfpathlineto{\pgfqpoint{2.472452in}{3.309215in}}%
\pgfpathlineto{\pgfqpoint{2.472452in}{2.522519in}}%
\pgfpathlineto{\pgfqpoint{2.598697in}{2.449631in}}%
\pgfpathlineto{\pgfqpoint{2.598697in}{3.236328in}}%
\pgfpathclose%
\pgfusepath{fill}%
\end{pgfscope}%
\begin{pgfscope}%
\pgfpathrectangle{\pgfqpoint{0.765000in}{0.660000in}}{\pgfqpoint{4.620000in}{4.620000in}}%
\pgfusepath{clip}%
\pgfsetbuttcap%
\pgfsetroundjoin%
\definecolor{currentfill}{rgb}{1.000000,0.894118,0.788235}%
\pgfsetfillcolor{currentfill}%
\pgfsetlinewidth{0.000000pt}%
\definecolor{currentstroke}{rgb}{1.000000,0.894118,0.788235}%
\pgfsetstrokecolor{currentstroke}%
\pgfsetdash{}{0pt}%
\pgfpathmoveto{\pgfqpoint{2.472452in}{3.309215in}}%
\pgfpathlineto{\pgfqpoint{2.472452in}{3.454990in}}%
\pgfpathlineto{\pgfqpoint{2.472452in}{2.668294in}}%
\pgfpathlineto{\pgfqpoint{2.598697in}{2.449631in}}%
\pgfpathlineto{\pgfqpoint{2.472452in}{3.309215in}}%
\pgfpathclose%
\pgfusepath{fill}%
\end{pgfscope}%
\begin{pgfscope}%
\pgfpathrectangle{\pgfqpoint{0.765000in}{0.660000in}}{\pgfqpoint{4.620000in}{4.620000in}}%
\pgfusepath{clip}%
\pgfsetbuttcap%
\pgfsetroundjoin%
\definecolor{currentfill}{rgb}{1.000000,0.894118,0.788235}%
\pgfsetfillcolor{currentfill}%
\pgfsetlinewidth{0.000000pt}%
\definecolor{currentstroke}{rgb}{1.000000,0.894118,0.788235}%
\pgfsetstrokecolor{currentstroke}%
\pgfsetdash{}{0pt}%
\pgfpathmoveto{\pgfqpoint{2.598697in}{3.236328in}}%
\pgfpathlineto{\pgfqpoint{2.598697in}{3.382103in}}%
\pgfpathlineto{\pgfqpoint{2.598697in}{2.595406in}}%
\pgfpathlineto{\pgfqpoint{2.472452in}{2.522519in}}%
\pgfpathlineto{\pgfqpoint{2.598697in}{3.236328in}}%
\pgfpathclose%
\pgfusepath{fill}%
\end{pgfscope}%
\begin{pgfscope}%
\pgfpathrectangle{\pgfqpoint{0.765000in}{0.660000in}}{\pgfqpoint{4.620000in}{4.620000in}}%
\pgfusepath{clip}%
\pgfsetbuttcap%
\pgfsetroundjoin%
\definecolor{currentfill}{rgb}{1.000000,0.894118,0.788235}%
\pgfsetfillcolor{currentfill}%
\pgfsetlinewidth{0.000000pt}%
\definecolor{currentstroke}{rgb}{1.000000,0.894118,0.788235}%
\pgfsetstrokecolor{currentstroke}%
\pgfsetdash{}{0pt}%
\pgfpathmoveto{\pgfqpoint{2.472452in}{3.163440in}}%
\pgfpathlineto{\pgfqpoint{2.598697in}{3.236328in}}%
\pgfpathlineto{\pgfqpoint{2.598697in}{2.449631in}}%
\pgfpathlineto{\pgfqpoint{2.472452in}{2.376743in}}%
\pgfpathlineto{\pgfqpoint{2.472452in}{3.163440in}}%
\pgfpathclose%
\pgfusepath{fill}%
\end{pgfscope}%
\begin{pgfscope}%
\pgfpathrectangle{\pgfqpoint{0.765000in}{0.660000in}}{\pgfqpoint{4.620000in}{4.620000in}}%
\pgfusepath{clip}%
\pgfsetbuttcap%
\pgfsetroundjoin%
\definecolor{currentfill}{rgb}{1.000000,0.894118,0.788235}%
\pgfsetfillcolor{currentfill}%
\pgfsetlinewidth{0.000000pt}%
\definecolor{currentstroke}{rgb}{1.000000,0.894118,0.788235}%
\pgfsetstrokecolor{currentstroke}%
\pgfsetdash{}{0pt}%
\pgfpathmoveto{\pgfqpoint{2.598697in}{3.382103in}}%
\pgfpathlineto{\pgfqpoint{2.472452in}{3.454990in}}%
\pgfpathlineto{\pgfqpoint{2.472452in}{2.668294in}}%
\pgfpathlineto{\pgfqpoint{2.598697in}{2.595406in}}%
\pgfpathlineto{\pgfqpoint{2.598697in}{3.382103in}}%
\pgfpathclose%
\pgfusepath{fill}%
\end{pgfscope}%
\begin{pgfscope}%
\pgfpathrectangle{\pgfqpoint{0.765000in}{0.660000in}}{\pgfqpoint{4.620000in}{4.620000in}}%
\pgfusepath{clip}%
\pgfsetbuttcap%
\pgfsetroundjoin%
\definecolor{currentfill}{rgb}{1.000000,0.894118,0.788235}%
\pgfsetfillcolor{currentfill}%
\pgfsetlinewidth{0.000000pt}%
\definecolor{currentstroke}{rgb}{1.000000,0.894118,0.788235}%
\pgfsetstrokecolor{currentstroke}%
\pgfsetdash{}{0pt}%
\pgfpathmoveto{\pgfqpoint{2.346207in}{3.236328in}}%
\pgfpathlineto{\pgfqpoint{2.472452in}{3.163440in}}%
\pgfpathlineto{\pgfqpoint{2.472452in}{2.376743in}}%
\pgfpathlineto{\pgfqpoint{2.346207in}{2.449631in}}%
\pgfpathlineto{\pgfqpoint{2.346207in}{3.236328in}}%
\pgfpathclose%
\pgfusepath{fill}%
\end{pgfscope}%
\begin{pgfscope}%
\pgfpathrectangle{\pgfqpoint{0.765000in}{0.660000in}}{\pgfqpoint{4.620000in}{4.620000in}}%
\pgfusepath{clip}%
\pgfsetbuttcap%
\pgfsetroundjoin%
\definecolor{currentfill}{rgb}{1.000000,0.894118,0.788235}%
\pgfsetfillcolor{currentfill}%
\pgfsetlinewidth{0.000000pt}%
\definecolor{currentstroke}{rgb}{1.000000,0.894118,0.788235}%
\pgfsetstrokecolor{currentstroke}%
\pgfsetdash{}{0pt}%
\pgfpathmoveto{\pgfqpoint{2.472452in}{3.454990in}}%
\pgfpathlineto{\pgfqpoint{2.346207in}{3.382103in}}%
\pgfpathlineto{\pgfqpoint{2.346207in}{2.595406in}}%
\pgfpathlineto{\pgfqpoint{2.472452in}{2.668294in}}%
\pgfpathlineto{\pgfqpoint{2.472452in}{3.454990in}}%
\pgfpathclose%
\pgfusepath{fill}%
\end{pgfscope}%
\begin{pgfscope}%
\pgfpathrectangle{\pgfqpoint{0.765000in}{0.660000in}}{\pgfqpoint{4.620000in}{4.620000in}}%
\pgfusepath{clip}%
\pgfsetbuttcap%
\pgfsetroundjoin%
\definecolor{currentfill}{rgb}{1.000000,0.894118,0.788235}%
\pgfsetfillcolor{currentfill}%
\pgfsetlinewidth{0.000000pt}%
\definecolor{currentstroke}{rgb}{1.000000,0.894118,0.788235}%
\pgfsetstrokecolor{currentstroke}%
\pgfsetdash{}{0pt}%
\pgfpathmoveto{\pgfqpoint{2.346207in}{3.236328in}}%
\pgfpathlineto{\pgfqpoint{2.346207in}{3.382103in}}%
\pgfpathlineto{\pgfqpoint{2.346207in}{2.595406in}}%
\pgfpathlineto{\pgfqpoint{2.472452in}{2.376743in}}%
\pgfpathlineto{\pgfqpoint{2.346207in}{3.236328in}}%
\pgfpathclose%
\pgfusepath{fill}%
\end{pgfscope}%
\begin{pgfscope}%
\pgfpathrectangle{\pgfqpoint{0.765000in}{0.660000in}}{\pgfqpoint{4.620000in}{4.620000in}}%
\pgfusepath{clip}%
\pgfsetbuttcap%
\pgfsetroundjoin%
\definecolor{currentfill}{rgb}{1.000000,0.894118,0.788235}%
\pgfsetfillcolor{currentfill}%
\pgfsetlinewidth{0.000000pt}%
\definecolor{currentstroke}{rgb}{1.000000,0.894118,0.788235}%
\pgfsetstrokecolor{currentstroke}%
\pgfsetdash{}{0pt}%
\pgfpathmoveto{\pgfqpoint{2.472452in}{3.163440in}}%
\pgfpathlineto{\pgfqpoint{2.472452in}{3.309215in}}%
\pgfpathlineto{\pgfqpoint{2.472452in}{2.522519in}}%
\pgfpathlineto{\pgfqpoint{2.346207in}{2.449631in}}%
\pgfpathlineto{\pgfqpoint{2.472452in}{3.163440in}}%
\pgfpathclose%
\pgfusepath{fill}%
\end{pgfscope}%
\begin{pgfscope}%
\pgfpathrectangle{\pgfqpoint{0.765000in}{0.660000in}}{\pgfqpoint{4.620000in}{4.620000in}}%
\pgfusepath{clip}%
\pgfsetbuttcap%
\pgfsetroundjoin%
\definecolor{currentfill}{rgb}{1.000000,0.894118,0.788235}%
\pgfsetfillcolor{currentfill}%
\pgfsetlinewidth{0.000000pt}%
\definecolor{currentstroke}{rgb}{1.000000,0.894118,0.788235}%
\pgfsetstrokecolor{currentstroke}%
\pgfsetdash{}{0pt}%
\pgfpathmoveto{\pgfqpoint{2.346207in}{3.382103in}}%
\pgfpathlineto{\pgfqpoint{2.472452in}{3.309215in}}%
\pgfpathlineto{\pgfqpoint{2.472452in}{2.522519in}}%
\pgfpathlineto{\pgfqpoint{2.346207in}{2.595406in}}%
\pgfpathlineto{\pgfqpoint{2.346207in}{3.382103in}}%
\pgfpathclose%
\pgfusepath{fill}%
\end{pgfscope}%
\begin{pgfscope}%
\pgfpathrectangle{\pgfqpoint{0.765000in}{0.660000in}}{\pgfqpoint{4.620000in}{4.620000in}}%
\pgfusepath{clip}%
\pgfsetbuttcap%
\pgfsetroundjoin%
\definecolor{currentfill}{rgb}{1.000000,0.894118,0.788235}%
\pgfsetfillcolor{currentfill}%
\pgfsetlinewidth{0.000000pt}%
\definecolor{currentstroke}{rgb}{1.000000,0.894118,0.788235}%
\pgfsetstrokecolor{currentstroke}%
\pgfsetdash{}{0pt}%
\pgfpathmoveto{\pgfqpoint{2.472452in}{3.309215in}}%
\pgfpathlineto{\pgfqpoint{2.598697in}{3.382103in}}%
\pgfpathlineto{\pgfqpoint{2.598697in}{2.595406in}}%
\pgfpathlineto{\pgfqpoint{2.472452in}{2.522519in}}%
\pgfpathlineto{\pgfqpoint{2.472452in}{3.309215in}}%
\pgfpathclose%
\pgfusepath{fill}%
\end{pgfscope}%
\begin{pgfscope}%
\pgfpathrectangle{\pgfqpoint{0.765000in}{0.660000in}}{\pgfqpoint{4.620000in}{4.620000in}}%
\pgfusepath{clip}%
\pgfsetbuttcap%
\pgfsetroundjoin%
\definecolor{currentfill}{rgb}{1.000000,0.894118,0.788235}%
\pgfsetfillcolor{currentfill}%
\pgfsetlinewidth{0.000000pt}%
\definecolor{currentstroke}{rgb}{1.000000,0.894118,0.788235}%
\pgfsetstrokecolor{currentstroke}%
\pgfsetdash{}{0pt}%
\pgfpathmoveto{\pgfqpoint{2.472452in}{3.309215in}}%
\pgfpathlineto{\pgfqpoint{2.346207in}{3.236328in}}%
\pgfpathlineto{\pgfqpoint{2.472452in}{3.163440in}}%
\pgfpathlineto{\pgfqpoint{2.598697in}{3.236328in}}%
\pgfpathlineto{\pgfqpoint{2.472452in}{3.309215in}}%
\pgfpathclose%
\pgfusepath{fill}%
\end{pgfscope}%
\begin{pgfscope}%
\pgfpathrectangle{\pgfqpoint{0.765000in}{0.660000in}}{\pgfqpoint{4.620000in}{4.620000in}}%
\pgfusepath{clip}%
\pgfsetbuttcap%
\pgfsetroundjoin%
\definecolor{currentfill}{rgb}{1.000000,0.894118,0.788235}%
\pgfsetfillcolor{currentfill}%
\pgfsetlinewidth{0.000000pt}%
\definecolor{currentstroke}{rgb}{1.000000,0.894118,0.788235}%
\pgfsetstrokecolor{currentstroke}%
\pgfsetdash{}{0pt}%
\pgfpathmoveto{\pgfqpoint{2.472452in}{3.309215in}}%
\pgfpathlineto{\pgfqpoint{2.346207in}{3.236328in}}%
\pgfpathlineto{\pgfqpoint{2.346207in}{3.382103in}}%
\pgfpathlineto{\pgfqpoint{2.472452in}{3.454990in}}%
\pgfpathlineto{\pgfqpoint{2.472452in}{3.309215in}}%
\pgfpathclose%
\pgfusepath{fill}%
\end{pgfscope}%
\begin{pgfscope}%
\pgfpathrectangle{\pgfqpoint{0.765000in}{0.660000in}}{\pgfqpoint{4.620000in}{4.620000in}}%
\pgfusepath{clip}%
\pgfsetbuttcap%
\pgfsetroundjoin%
\definecolor{currentfill}{rgb}{1.000000,0.894118,0.788235}%
\pgfsetfillcolor{currentfill}%
\pgfsetlinewidth{0.000000pt}%
\definecolor{currentstroke}{rgb}{1.000000,0.894118,0.788235}%
\pgfsetstrokecolor{currentstroke}%
\pgfsetdash{}{0pt}%
\pgfpathmoveto{\pgfqpoint{2.472452in}{3.309215in}}%
\pgfpathlineto{\pgfqpoint{2.598697in}{3.236328in}}%
\pgfpathlineto{\pgfqpoint{2.598697in}{3.382103in}}%
\pgfpathlineto{\pgfqpoint{2.472452in}{3.454990in}}%
\pgfpathlineto{\pgfqpoint{2.472452in}{3.309215in}}%
\pgfpathclose%
\pgfusepath{fill}%
\end{pgfscope}%
\begin{pgfscope}%
\pgfpathrectangle{\pgfqpoint{0.765000in}{0.660000in}}{\pgfqpoint{4.620000in}{4.620000in}}%
\pgfusepath{clip}%
\pgfsetbuttcap%
\pgfsetroundjoin%
\definecolor{currentfill}{rgb}{1.000000,0.894118,0.788235}%
\pgfsetfillcolor{currentfill}%
\pgfsetlinewidth{0.000000pt}%
\definecolor{currentstroke}{rgb}{1.000000,0.894118,0.788235}%
\pgfsetstrokecolor{currentstroke}%
\pgfsetdash{}{0pt}%
\pgfpathmoveto{\pgfqpoint{3.113634in}{3.679402in}}%
\pgfpathlineto{\pgfqpoint{2.987389in}{3.606514in}}%
\pgfpathlineto{\pgfqpoint{3.113634in}{3.533627in}}%
\pgfpathlineto{\pgfqpoint{3.239879in}{3.606514in}}%
\pgfpathlineto{\pgfqpoint{3.113634in}{3.679402in}}%
\pgfpathclose%
\pgfusepath{fill}%
\end{pgfscope}%
\begin{pgfscope}%
\pgfpathrectangle{\pgfqpoint{0.765000in}{0.660000in}}{\pgfqpoint{4.620000in}{4.620000in}}%
\pgfusepath{clip}%
\pgfsetbuttcap%
\pgfsetroundjoin%
\definecolor{currentfill}{rgb}{1.000000,0.894118,0.788235}%
\pgfsetfillcolor{currentfill}%
\pgfsetlinewidth{0.000000pt}%
\definecolor{currentstroke}{rgb}{1.000000,0.894118,0.788235}%
\pgfsetstrokecolor{currentstroke}%
\pgfsetdash{}{0pt}%
\pgfpathmoveto{\pgfqpoint{3.113634in}{3.679402in}}%
\pgfpathlineto{\pgfqpoint{2.987389in}{3.606514in}}%
\pgfpathlineto{\pgfqpoint{2.987389in}{3.752289in}}%
\pgfpathlineto{\pgfqpoint{3.113634in}{3.825177in}}%
\pgfpathlineto{\pgfqpoint{3.113634in}{3.679402in}}%
\pgfpathclose%
\pgfusepath{fill}%
\end{pgfscope}%
\begin{pgfscope}%
\pgfpathrectangle{\pgfqpoint{0.765000in}{0.660000in}}{\pgfqpoint{4.620000in}{4.620000in}}%
\pgfusepath{clip}%
\pgfsetbuttcap%
\pgfsetroundjoin%
\definecolor{currentfill}{rgb}{1.000000,0.894118,0.788235}%
\pgfsetfillcolor{currentfill}%
\pgfsetlinewidth{0.000000pt}%
\definecolor{currentstroke}{rgb}{1.000000,0.894118,0.788235}%
\pgfsetstrokecolor{currentstroke}%
\pgfsetdash{}{0pt}%
\pgfpathmoveto{\pgfqpoint{3.113634in}{3.679402in}}%
\pgfpathlineto{\pgfqpoint{3.239879in}{3.606514in}}%
\pgfpathlineto{\pgfqpoint{3.239879in}{3.752289in}}%
\pgfpathlineto{\pgfqpoint{3.113634in}{3.825177in}}%
\pgfpathlineto{\pgfqpoint{3.113634in}{3.679402in}}%
\pgfpathclose%
\pgfusepath{fill}%
\end{pgfscope}%
\begin{pgfscope}%
\pgfpathrectangle{\pgfqpoint{0.765000in}{0.660000in}}{\pgfqpoint{4.620000in}{4.620000in}}%
\pgfusepath{clip}%
\pgfsetbuttcap%
\pgfsetroundjoin%
\definecolor{currentfill}{rgb}{1.000000,0.894118,0.788235}%
\pgfsetfillcolor{currentfill}%
\pgfsetlinewidth{0.000000pt}%
\definecolor{currentstroke}{rgb}{1.000000,0.894118,0.788235}%
\pgfsetstrokecolor{currentstroke}%
\pgfsetdash{}{0pt}%
\pgfpathmoveto{\pgfqpoint{2.472452in}{3.454990in}}%
\pgfpathlineto{\pgfqpoint{2.346207in}{3.382103in}}%
\pgfpathlineto{\pgfqpoint{2.472452in}{3.309215in}}%
\pgfpathlineto{\pgfqpoint{2.598697in}{3.382103in}}%
\pgfpathlineto{\pgfqpoint{2.472452in}{3.454990in}}%
\pgfpathclose%
\pgfusepath{fill}%
\end{pgfscope}%
\begin{pgfscope}%
\pgfpathrectangle{\pgfqpoint{0.765000in}{0.660000in}}{\pgfqpoint{4.620000in}{4.620000in}}%
\pgfusepath{clip}%
\pgfsetbuttcap%
\pgfsetroundjoin%
\definecolor{currentfill}{rgb}{1.000000,0.894118,0.788235}%
\pgfsetfillcolor{currentfill}%
\pgfsetlinewidth{0.000000pt}%
\definecolor{currentstroke}{rgb}{1.000000,0.894118,0.788235}%
\pgfsetstrokecolor{currentstroke}%
\pgfsetdash{}{0pt}%
\pgfpathmoveto{\pgfqpoint{2.472452in}{3.163440in}}%
\pgfpathlineto{\pgfqpoint{2.598697in}{3.236328in}}%
\pgfpathlineto{\pgfqpoint{2.598697in}{3.382103in}}%
\pgfpathlineto{\pgfqpoint{2.472452in}{3.309215in}}%
\pgfpathlineto{\pgfqpoint{2.472452in}{3.163440in}}%
\pgfpathclose%
\pgfusepath{fill}%
\end{pgfscope}%
\begin{pgfscope}%
\pgfpathrectangle{\pgfqpoint{0.765000in}{0.660000in}}{\pgfqpoint{4.620000in}{4.620000in}}%
\pgfusepath{clip}%
\pgfsetbuttcap%
\pgfsetroundjoin%
\definecolor{currentfill}{rgb}{1.000000,0.894118,0.788235}%
\pgfsetfillcolor{currentfill}%
\pgfsetlinewidth{0.000000pt}%
\definecolor{currentstroke}{rgb}{1.000000,0.894118,0.788235}%
\pgfsetstrokecolor{currentstroke}%
\pgfsetdash{}{0pt}%
\pgfpathmoveto{\pgfqpoint{2.346207in}{3.236328in}}%
\pgfpathlineto{\pgfqpoint{2.472452in}{3.163440in}}%
\pgfpathlineto{\pgfqpoint{2.472452in}{3.309215in}}%
\pgfpathlineto{\pgfqpoint{2.346207in}{3.382103in}}%
\pgfpathlineto{\pgfqpoint{2.346207in}{3.236328in}}%
\pgfpathclose%
\pgfusepath{fill}%
\end{pgfscope}%
\begin{pgfscope}%
\pgfpathrectangle{\pgfqpoint{0.765000in}{0.660000in}}{\pgfqpoint{4.620000in}{4.620000in}}%
\pgfusepath{clip}%
\pgfsetbuttcap%
\pgfsetroundjoin%
\definecolor{currentfill}{rgb}{1.000000,0.894118,0.788235}%
\pgfsetfillcolor{currentfill}%
\pgfsetlinewidth{0.000000pt}%
\definecolor{currentstroke}{rgb}{1.000000,0.894118,0.788235}%
\pgfsetstrokecolor{currentstroke}%
\pgfsetdash{}{0pt}%
\pgfpathmoveto{\pgfqpoint{3.113634in}{3.825177in}}%
\pgfpathlineto{\pgfqpoint{2.987389in}{3.752289in}}%
\pgfpathlineto{\pgfqpoint{3.113634in}{3.679402in}}%
\pgfpathlineto{\pgfqpoint{3.239879in}{3.752289in}}%
\pgfpathlineto{\pgfqpoint{3.113634in}{3.825177in}}%
\pgfpathclose%
\pgfusepath{fill}%
\end{pgfscope}%
\begin{pgfscope}%
\pgfpathrectangle{\pgfqpoint{0.765000in}{0.660000in}}{\pgfqpoint{4.620000in}{4.620000in}}%
\pgfusepath{clip}%
\pgfsetbuttcap%
\pgfsetroundjoin%
\definecolor{currentfill}{rgb}{1.000000,0.894118,0.788235}%
\pgfsetfillcolor{currentfill}%
\pgfsetlinewidth{0.000000pt}%
\definecolor{currentstroke}{rgb}{1.000000,0.894118,0.788235}%
\pgfsetstrokecolor{currentstroke}%
\pgfsetdash{}{0pt}%
\pgfpathmoveto{\pgfqpoint{3.113634in}{3.533627in}}%
\pgfpathlineto{\pgfqpoint{3.239879in}{3.606514in}}%
\pgfpathlineto{\pgfqpoint{3.239879in}{3.752289in}}%
\pgfpathlineto{\pgfqpoint{3.113634in}{3.679402in}}%
\pgfpathlineto{\pgfqpoint{3.113634in}{3.533627in}}%
\pgfpathclose%
\pgfusepath{fill}%
\end{pgfscope}%
\begin{pgfscope}%
\pgfpathrectangle{\pgfqpoint{0.765000in}{0.660000in}}{\pgfqpoint{4.620000in}{4.620000in}}%
\pgfusepath{clip}%
\pgfsetbuttcap%
\pgfsetroundjoin%
\definecolor{currentfill}{rgb}{1.000000,0.894118,0.788235}%
\pgfsetfillcolor{currentfill}%
\pgfsetlinewidth{0.000000pt}%
\definecolor{currentstroke}{rgb}{1.000000,0.894118,0.788235}%
\pgfsetstrokecolor{currentstroke}%
\pgfsetdash{}{0pt}%
\pgfpathmoveto{\pgfqpoint{2.987389in}{3.606514in}}%
\pgfpathlineto{\pgfqpoint{3.113634in}{3.533627in}}%
\pgfpathlineto{\pgfqpoint{3.113634in}{3.679402in}}%
\pgfpathlineto{\pgfqpoint{2.987389in}{3.752289in}}%
\pgfpathlineto{\pgfqpoint{2.987389in}{3.606514in}}%
\pgfpathclose%
\pgfusepath{fill}%
\end{pgfscope}%
\begin{pgfscope}%
\pgfpathrectangle{\pgfqpoint{0.765000in}{0.660000in}}{\pgfqpoint{4.620000in}{4.620000in}}%
\pgfusepath{clip}%
\pgfsetbuttcap%
\pgfsetroundjoin%
\definecolor{currentfill}{rgb}{1.000000,0.894118,0.788235}%
\pgfsetfillcolor{currentfill}%
\pgfsetlinewidth{0.000000pt}%
\definecolor{currentstroke}{rgb}{1.000000,0.894118,0.788235}%
\pgfsetstrokecolor{currentstroke}%
\pgfsetdash{}{0pt}%
\pgfpathmoveto{\pgfqpoint{2.472452in}{3.309215in}}%
\pgfpathlineto{\pgfqpoint{2.346207in}{3.236328in}}%
\pgfpathlineto{\pgfqpoint{2.987389in}{3.606514in}}%
\pgfpathlineto{\pgfqpoint{3.113634in}{3.679402in}}%
\pgfpathlineto{\pgfqpoint{2.472452in}{3.309215in}}%
\pgfpathclose%
\pgfusepath{fill}%
\end{pgfscope}%
\begin{pgfscope}%
\pgfpathrectangle{\pgfqpoint{0.765000in}{0.660000in}}{\pgfqpoint{4.620000in}{4.620000in}}%
\pgfusepath{clip}%
\pgfsetbuttcap%
\pgfsetroundjoin%
\definecolor{currentfill}{rgb}{1.000000,0.894118,0.788235}%
\pgfsetfillcolor{currentfill}%
\pgfsetlinewidth{0.000000pt}%
\definecolor{currentstroke}{rgb}{1.000000,0.894118,0.788235}%
\pgfsetstrokecolor{currentstroke}%
\pgfsetdash{}{0pt}%
\pgfpathmoveto{\pgfqpoint{2.598697in}{3.236328in}}%
\pgfpathlineto{\pgfqpoint{2.472452in}{3.309215in}}%
\pgfpathlineto{\pgfqpoint{3.113634in}{3.679402in}}%
\pgfpathlineto{\pgfqpoint{3.239879in}{3.606514in}}%
\pgfpathlineto{\pgfqpoint{2.598697in}{3.236328in}}%
\pgfpathclose%
\pgfusepath{fill}%
\end{pgfscope}%
\begin{pgfscope}%
\pgfpathrectangle{\pgfqpoint{0.765000in}{0.660000in}}{\pgfqpoint{4.620000in}{4.620000in}}%
\pgfusepath{clip}%
\pgfsetbuttcap%
\pgfsetroundjoin%
\definecolor{currentfill}{rgb}{1.000000,0.894118,0.788235}%
\pgfsetfillcolor{currentfill}%
\pgfsetlinewidth{0.000000pt}%
\definecolor{currentstroke}{rgb}{1.000000,0.894118,0.788235}%
\pgfsetstrokecolor{currentstroke}%
\pgfsetdash{}{0pt}%
\pgfpathmoveto{\pgfqpoint{2.472452in}{3.309215in}}%
\pgfpathlineto{\pgfqpoint{2.472452in}{3.454990in}}%
\pgfpathlineto{\pgfqpoint{3.113634in}{3.825177in}}%
\pgfpathlineto{\pgfqpoint{3.239879in}{3.606514in}}%
\pgfpathlineto{\pgfqpoint{2.472452in}{3.309215in}}%
\pgfpathclose%
\pgfusepath{fill}%
\end{pgfscope}%
\begin{pgfscope}%
\pgfpathrectangle{\pgfqpoint{0.765000in}{0.660000in}}{\pgfqpoint{4.620000in}{4.620000in}}%
\pgfusepath{clip}%
\pgfsetbuttcap%
\pgfsetroundjoin%
\definecolor{currentfill}{rgb}{1.000000,0.894118,0.788235}%
\pgfsetfillcolor{currentfill}%
\pgfsetlinewidth{0.000000pt}%
\definecolor{currentstroke}{rgb}{1.000000,0.894118,0.788235}%
\pgfsetstrokecolor{currentstroke}%
\pgfsetdash{}{0pt}%
\pgfpathmoveto{\pgfqpoint{2.598697in}{3.236328in}}%
\pgfpathlineto{\pgfqpoint{2.598697in}{3.382103in}}%
\pgfpathlineto{\pgfqpoint{3.239879in}{3.752289in}}%
\pgfpathlineto{\pgfqpoint{3.113634in}{3.679402in}}%
\pgfpathlineto{\pgfqpoint{2.598697in}{3.236328in}}%
\pgfpathclose%
\pgfusepath{fill}%
\end{pgfscope}%
\begin{pgfscope}%
\pgfpathrectangle{\pgfqpoint{0.765000in}{0.660000in}}{\pgfqpoint{4.620000in}{4.620000in}}%
\pgfusepath{clip}%
\pgfsetbuttcap%
\pgfsetroundjoin%
\definecolor{currentfill}{rgb}{1.000000,0.894118,0.788235}%
\pgfsetfillcolor{currentfill}%
\pgfsetlinewidth{0.000000pt}%
\definecolor{currentstroke}{rgb}{1.000000,0.894118,0.788235}%
\pgfsetstrokecolor{currentstroke}%
\pgfsetdash{}{0pt}%
\pgfpathmoveto{\pgfqpoint{2.472452in}{3.163440in}}%
\pgfpathlineto{\pgfqpoint{2.598697in}{3.236328in}}%
\pgfpathlineto{\pgfqpoint{3.239879in}{3.606514in}}%
\pgfpathlineto{\pgfqpoint{3.113634in}{3.533627in}}%
\pgfpathlineto{\pgfqpoint{2.472452in}{3.163440in}}%
\pgfpathclose%
\pgfusepath{fill}%
\end{pgfscope}%
\begin{pgfscope}%
\pgfpathrectangle{\pgfqpoint{0.765000in}{0.660000in}}{\pgfqpoint{4.620000in}{4.620000in}}%
\pgfusepath{clip}%
\pgfsetbuttcap%
\pgfsetroundjoin%
\definecolor{currentfill}{rgb}{1.000000,0.894118,0.788235}%
\pgfsetfillcolor{currentfill}%
\pgfsetlinewidth{0.000000pt}%
\definecolor{currentstroke}{rgb}{1.000000,0.894118,0.788235}%
\pgfsetstrokecolor{currentstroke}%
\pgfsetdash{}{0pt}%
\pgfpathmoveto{\pgfqpoint{2.598697in}{3.382103in}}%
\pgfpathlineto{\pgfqpoint{2.472452in}{3.454990in}}%
\pgfpathlineto{\pgfqpoint{3.113634in}{3.825177in}}%
\pgfpathlineto{\pgfqpoint{3.239879in}{3.752289in}}%
\pgfpathlineto{\pgfqpoint{2.598697in}{3.382103in}}%
\pgfpathclose%
\pgfusepath{fill}%
\end{pgfscope}%
\begin{pgfscope}%
\pgfpathrectangle{\pgfqpoint{0.765000in}{0.660000in}}{\pgfqpoint{4.620000in}{4.620000in}}%
\pgfusepath{clip}%
\pgfsetbuttcap%
\pgfsetroundjoin%
\definecolor{currentfill}{rgb}{1.000000,0.894118,0.788235}%
\pgfsetfillcolor{currentfill}%
\pgfsetlinewidth{0.000000pt}%
\definecolor{currentstroke}{rgb}{1.000000,0.894118,0.788235}%
\pgfsetstrokecolor{currentstroke}%
\pgfsetdash{}{0pt}%
\pgfpathmoveto{\pgfqpoint{2.346207in}{3.236328in}}%
\pgfpathlineto{\pgfqpoint{2.472452in}{3.163440in}}%
\pgfpathlineto{\pgfqpoint{3.113634in}{3.533627in}}%
\pgfpathlineto{\pgfqpoint{2.987389in}{3.606514in}}%
\pgfpathlineto{\pgfqpoint{2.346207in}{3.236328in}}%
\pgfpathclose%
\pgfusepath{fill}%
\end{pgfscope}%
\begin{pgfscope}%
\pgfpathrectangle{\pgfqpoint{0.765000in}{0.660000in}}{\pgfqpoint{4.620000in}{4.620000in}}%
\pgfusepath{clip}%
\pgfsetbuttcap%
\pgfsetroundjoin%
\definecolor{currentfill}{rgb}{1.000000,0.894118,0.788235}%
\pgfsetfillcolor{currentfill}%
\pgfsetlinewidth{0.000000pt}%
\definecolor{currentstroke}{rgb}{1.000000,0.894118,0.788235}%
\pgfsetstrokecolor{currentstroke}%
\pgfsetdash{}{0pt}%
\pgfpathmoveto{\pgfqpoint{2.472452in}{3.454990in}}%
\pgfpathlineto{\pgfqpoint{2.346207in}{3.382103in}}%
\pgfpathlineto{\pgfqpoint{2.987389in}{3.752289in}}%
\pgfpathlineto{\pgfqpoint{3.113634in}{3.825177in}}%
\pgfpathlineto{\pgfqpoint{2.472452in}{3.454990in}}%
\pgfpathclose%
\pgfusepath{fill}%
\end{pgfscope}%
\begin{pgfscope}%
\pgfpathrectangle{\pgfqpoint{0.765000in}{0.660000in}}{\pgfqpoint{4.620000in}{4.620000in}}%
\pgfusepath{clip}%
\pgfsetbuttcap%
\pgfsetroundjoin%
\definecolor{currentfill}{rgb}{1.000000,0.894118,0.788235}%
\pgfsetfillcolor{currentfill}%
\pgfsetlinewidth{0.000000pt}%
\definecolor{currentstroke}{rgb}{1.000000,0.894118,0.788235}%
\pgfsetstrokecolor{currentstroke}%
\pgfsetdash{}{0pt}%
\pgfpathmoveto{\pgfqpoint{2.346207in}{3.236328in}}%
\pgfpathlineto{\pgfqpoint{2.346207in}{3.382103in}}%
\pgfpathlineto{\pgfqpoint{2.987389in}{3.752289in}}%
\pgfpathlineto{\pgfqpoint{3.113634in}{3.533627in}}%
\pgfpathlineto{\pgfqpoint{2.346207in}{3.236328in}}%
\pgfpathclose%
\pgfusepath{fill}%
\end{pgfscope}%
\begin{pgfscope}%
\pgfpathrectangle{\pgfqpoint{0.765000in}{0.660000in}}{\pgfqpoint{4.620000in}{4.620000in}}%
\pgfusepath{clip}%
\pgfsetbuttcap%
\pgfsetroundjoin%
\definecolor{currentfill}{rgb}{1.000000,0.894118,0.788235}%
\pgfsetfillcolor{currentfill}%
\pgfsetlinewidth{0.000000pt}%
\definecolor{currentstroke}{rgb}{1.000000,0.894118,0.788235}%
\pgfsetstrokecolor{currentstroke}%
\pgfsetdash{}{0pt}%
\pgfpathmoveto{\pgfqpoint{2.472452in}{3.163440in}}%
\pgfpathlineto{\pgfqpoint{2.472452in}{3.309215in}}%
\pgfpathlineto{\pgfqpoint{3.113634in}{3.679402in}}%
\pgfpathlineto{\pgfqpoint{2.987389in}{3.606514in}}%
\pgfpathlineto{\pgfqpoint{2.472452in}{3.163440in}}%
\pgfpathclose%
\pgfusepath{fill}%
\end{pgfscope}%
\begin{pgfscope}%
\pgfpathrectangle{\pgfqpoint{0.765000in}{0.660000in}}{\pgfqpoint{4.620000in}{4.620000in}}%
\pgfusepath{clip}%
\pgfsetbuttcap%
\pgfsetroundjoin%
\definecolor{currentfill}{rgb}{1.000000,0.894118,0.788235}%
\pgfsetfillcolor{currentfill}%
\pgfsetlinewidth{0.000000pt}%
\definecolor{currentstroke}{rgb}{1.000000,0.894118,0.788235}%
\pgfsetstrokecolor{currentstroke}%
\pgfsetdash{}{0pt}%
\pgfpathmoveto{\pgfqpoint{2.346207in}{3.382103in}}%
\pgfpathlineto{\pgfqpoint{2.472452in}{3.309215in}}%
\pgfpathlineto{\pgfqpoint{3.113634in}{3.679402in}}%
\pgfpathlineto{\pgfqpoint{2.987389in}{3.752289in}}%
\pgfpathlineto{\pgfqpoint{2.346207in}{3.382103in}}%
\pgfpathclose%
\pgfusepath{fill}%
\end{pgfscope}%
\begin{pgfscope}%
\pgfpathrectangle{\pgfqpoint{0.765000in}{0.660000in}}{\pgfqpoint{4.620000in}{4.620000in}}%
\pgfusepath{clip}%
\pgfsetbuttcap%
\pgfsetroundjoin%
\definecolor{currentfill}{rgb}{1.000000,0.894118,0.788235}%
\pgfsetfillcolor{currentfill}%
\pgfsetlinewidth{0.000000pt}%
\definecolor{currentstroke}{rgb}{1.000000,0.894118,0.788235}%
\pgfsetstrokecolor{currentstroke}%
\pgfsetdash{}{0pt}%
\pgfpathmoveto{\pgfqpoint{2.472452in}{3.309215in}}%
\pgfpathlineto{\pgfqpoint{2.598697in}{3.382103in}}%
\pgfpathlineto{\pgfqpoint{3.239879in}{3.752289in}}%
\pgfpathlineto{\pgfqpoint{3.113634in}{3.679402in}}%
\pgfpathlineto{\pgfqpoint{2.472452in}{3.309215in}}%
\pgfpathclose%
\pgfusepath{fill}%
\end{pgfscope}%
\begin{pgfscope}%
\pgfpathrectangle{\pgfqpoint{0.765000in}{0.660000in}}{\pgfqpoint{4.620000in}{4.620000in}}%
\pgfusepath{clip}%
\pgfsetbuttcap%
\pgfsetroundjoin%
\definecolor{currentfill}{rgb}{1.000000,0.894118,0.788235}%
\pgfsetfillcolor{currentfill}%
\pgfsetlinewidth{0.000000pt}%
\definecolor{currentstroke}{rgb}{1.000000,0.894118,0.788235}%
\pgfsetstrokecolor{currentstroke}%
\pgfsetdash{}{0pt}%
\pgfpathmoveto{\pgfqpoint{3.860367in}{2.339514in}}%
\pgfpathlineto{\pgfqpoint{3.734122in}{2.266626in}}%
\pgfpathlineto{\pgfqpoint{3.860367in}{2.193739in}}%
\pgfpathlineto{\pgfqpoint{3.986612in}{2.266626in}}%
\pgfpathlineto{\pgfqpoint{3.860367in}{2.339514in}}%
\pgfpathclose%
\pgfusepath{fill}%
\end{pgfscope}%
\begin{pgfscope}%
\pgfpathrectangle{\pgfqpoint{0.765000in}{0.660000in}}{\pgfqpoint{4.620000in}{4.620000in}}%
\pgfusepath{clip}%
\pgfsetbuttcap%
\pgfsetroundjoin%
\definecolor{currentfill}{rgb}{1.000000,0.894118,0.788235}%
\pgfsetfillcolor{currentfill}%
\pgfsetlinewidth{0.000000pt}%
\definecolor{currentstroke}{rgb}{1.000000,0.894118,0.788235}%
\pgfsetstrokecolor{currentstroke}%
\pgfsetdash{}{0pt}%
\pgfpathmoveto{\pgfqpoint{3.860367in}{2.339514in}}%
\pgfpathlineto{\pgfqpoint{3.734122in}{2.266626in}}%
\pgfpathlineto{\pgfqpoint{3.734122in}{2.412401in}}%
\pgfpathlineto{\pgfqpoint{3.860367in}{2.485289in}}%
\pgfpathlineto{\pgfqpoint{3.860367in}{2.339514in}}%
\pgfpathclose%
\pgfusepath{fill}%
\end{pgfscope}%
\begin{pgfscope}%
\pgfpathrectangle{\pgfqpoint{0.765000in}{0.660000in}}{\pgfqpoint{4.620000in}{4.620000in}}%
\pgfusepath{clip}%
\pgfsetbuttcap%
\pgfsetroundjoin%
\definecolor{currentfill}{rgb}{1.000000,0.894118,0.788235}%
\pgfsetfillcolor{currentfill}%
\pgfsetlinewidth{0.000000pt}%
\definecolor{currentstroke}{rgb}{1.000000,0.894118,0.788235}%
\pgfsetstrokecolor{currentstroke}%
\pgfsetdash{}{0pt}%
\pgfpathmoveto{\pgfqpoint{3.860367in}{2.339514in}}%
\pgfpathlineto{\pgfqpoint{3.986612in}{2.266626in}}%
\pgfpathlineto{\pgfqpoint{3.986612in}{2.412401in}}%
\pgfpathlineto{\pgfqpoint{3.860367in}{2.485289in}}%
\pgfpathlineto{\pgfqpoint{3.860367in}{2.339514in}}%
\pgfpathclose%
\pgfusepath{fill}%
\end{pgfscope}%
\begin{pgfscope}%
\pgfpathrectangle{\pgfqpoint{0.765000in}{0.660000in}}{\pgfqpoint{4.620000in}{4.620000in}}%
\pgfusepath{clip}%
\pgfsetbuttcap%
\pgfsetroundjoin%
\definecolor{currentfill}{rgb}{1.000000,0.894118,0.788235}%
\pgfsetfillcolor{currentfill}%
\pgfsetlinewidth{0.000000pt}%
\definecolor{currentstroke}{rgb}{1.000000,0.894118,0.788235}%
\pgfsetstrokecolor{currentstroke}%
\pgfsetdash{}{0pt}%
\pgfpathmoveto{\pgfqpoint{3.324897in}{2.030360in}}%
\pgfpathlineto{\pgfqpoint{3.198652in}{1.957472in}}%
\pgfpathlineto{\pgfqpoint{3.324897in}{1.884585in}}%
\pgfpathlineto{\pgfqpoint{3.451141in}{1.957472in}}%
\pgfpathlineto{\pgfqpoint{3.324897in}{2.030360in}}%
\pgfpathclose%
\pgfusepath{fill}%
\end{pgfscope}%
\begin{pgfscope}%
\pgfpathrectangle{\pgfqpoint{0.765000in}{0.660000in}}{\pgfqpoint{4.620000in}{4.620000in}}%
\pgfusepath{clip}%
\pgfsetbuttcap%
\pgfsetroundjoin%
\definecolor{currentfill}{rgb}{1.000000,0.894118,0.788235}%
\pgfsetfillcolor{currentfill}%
\pgfsetlinewidth{0.000000pt}%
\definecolor{currentstroke}{rgb}{1.000000,0.894118,0.788235}%
\pgfsetstrokecolor{currentstroke}%
\pgfsetdash{}{0pt}%
\pgfpathmoveto{\pgfqpoint{3.324897in}{2.030360in}}%
\pgfpathlineto{\pgfqpoint{3.198652in}{1.957472in}}%
\pgfpathlineto{\pgfqpoint{3.198652in}{2.103247in}}%
\pgfpathlineto{\pgfqpoint{3.324897in}{2.176135in}}%
\pgfpathlineto{\pgfqpoint{3.324897in}{2.030360in}}%
\pgfpathclose%
\pgfusepath{fill}%
\end{pgfscope}%
\begin{pgfscope}%
\pgfpathrectangle{\pgfqpoint{0.765000in}{0.660000in}}{\pgfqpoint{4.620000in}{4.620000in}}%
\pgfusepath{clip}%
\pgfsetbuttcap%
\pgfsetroundjoin%
\definecolor{currentfill}{rgb}{1.000000,0.894118,0.788235}%
\pgfsetfillcolor{currentfill}%
\pgfsetlinewidth{0.000000pt}%
\definecolor{currentstroke}{rgb}{1.000000,0.894118,0.788235}%
\pgfsetstrokecolor{currentstroke}%
\pgfsetdash{}{0pt}%
\pgfpathmoveto{\pgfqpoint{3.324897in}{2.030360in}}%
\pgfpathlineto{\pgfqpoint{3.451141in}{1.957472in}}%
\pgfpathlineto{\pgfqpoint{3.451141in}{2.103247in}}%
\pgfpathlineto{\pgfqpoint{3.324897in}{2.176135in}}%
\pgfpathlineto{\pgfqpoint{3.324897in}{2.030360in}}%
\pgfpathclose%
\pgfusepath{fill}%
\end{pgfscope}%
\begin{pgfscope}%
\pgfpathrectangle{\pgfqpoint{0.765000in}{0.660000in}}{\pgfqpoint{4.620000in}{4.620000in}}%
\pgfusepath{clip}%
\pgfsetbuttcap%
\pgfsetroundjoin%
\definecolor{currentfill}{rgb}{1.000000,0.894118,0.788235}%
\pgfsetfillcolor{currentfill}%
\pgfsetlinewidth{0.000000pt}%
\definecolor{currentstroke}{rgb}{1.000000,0.894118,0.788235}%
\pgfsetstrokecolor{currentstroke}%
\pgfsetdash{}{0pt}%
\pgfpathmoveto{\pgfqpoint{3.860367in}{2.485289in}}%
\pgfpathlineto{\pgfqpoint{3.734122in}{2.412401in}}%
\pgfpathlineto{\pgfqpoint{3.860367in}{2.339514in}}%
\pgfpathlineto{\pgfqpoint{3.986612in}{2.412401in}}%
\pgfpathlineto{\pgfqpoint{3.860367in}{2.485289in}}%
\pgfpathclose%
\pgfusepath{fill}%
\end{pgfscope}%
\begin{pgfscope}%
\pgfpathrectangle{\pgfqpoint{0.765000in}{0.660000in}}{\pgfqpoint{4.620000in}{4.620000in}}%
\pgfusepath{clip}%
\pgfsetbuttcap%
\pgfsetroundjoin%
\definecolor{currentfill}{rgb}{1.000000,0.894118,0.788235}%
\pgfsetfillcolor{currentfill}%
\pgfsetlinewidth{0.000000pt}%
\definecolor{currentstroke}{rgb}{1.000000,0.894118,0.788235}%
\pgfsetstrokecolor{currentstroke}%
\pgfsetdash{}{0pt}%
\pgfpathmoveto{\pgfqpoint{3.860367in}{2.193739in}}%
\pgfpathlineto{\pgfqpoint{3.986612in}{2.266626in}}%
\pgfpathlineto{\pgfqpoint{3.986612in}{2.412401in}}%
\pgfpathlineto{\pgfqpoint{3.860367in}{2.339514in}}%
\pgfpathlineto{\pgfqpoint{3.860367in}{2.193739in}}%
\pgfpathclose%
\pgfusepath{fill}%
\end{pgfscope}%
\begin{pgfscope}%
\pgfpathrectangle{\pgfqpoint{0.765000in}{0.660000in}}{\pgfqpoint{4.620000in}{4.620000in}}%
\pgfusepath{clip}%
\pgfsetbuttcap%
\pgfsetroundjoin%
\definecolor{currentfill}{rgb}{1.000000,0.894118,0.788235}%
\pgfsetfillcolor{currentfill}%
\pgfsetlinewidth{0.000000pt}%
\definecolor{currentstroke}{rgb}{1.000000,0.894118,0.788235}%
\pgfsetstrokecolor{currentstroke}%
\pgfsetdash{}{0pt}%
\pgfpathmoveto{\pgfqpoint{3.734122in}{2.266626in}}%
\pgfpathlineto{\pgfqpoint{3.860367in}{2.193739in}}%
\pgfpathlineto{\pgfqpoint{3.860367in}{2.339514in}}%
\pgfpathlineto{\pgfqpoint{3.734122in}{2.412401in}}%
\pgfpathlineto{\pgfqpoint{3.734122in}{2.266626in}}%
\pgfpathclose%
\pgfusepath{fill}%
\end{pgfscope}%
\begin{pgfscope}%
\pgfpathrectangle{\pgfqpoint{0.765000in}{0.660000in}}{\pgfqpoint{4.620000in}{4.620000in}}%
\pgfusepath{clip}%
\pgfsetbuttcap%
\pgfsetroundjoin%
\definecolor{currentfill}{rgb}{1.000000,0.894118,0.788235}%
\pgfsetfillcolor{currentfill}%
\pgfsetlinewidth{0.000000pt}%
\definecolor{currentstroke}{rgb}{1.000000,0.894118,0.788235}%
\pgfsetstrokecolor{currentstroke}%
\pgfsetdash{}{0pt}%
\pgfpathmoveto{\pgfqpoint{3.324897in}{2.176135in}}%
\pgfpathlineto{\pgfqpoint{3.198652in}{2.103247in}}%
\pgfpathlineto{\pgfqpoint{3.324897in}{2.030360in}}%
\pgfpathlineto{\pgfqpoint{3.451141in}{2.103247in}}%
\pgfpathlineto{\pgfqpoint{3.324897in}{2.176135in}}%
\pgfpathclose%
\pgfusepath{fill}%
\end{pgfscope}%
\begin{pgfscope}%
\pgfpathrectangle{\pgfqpoint{0.765000in}{0.660000in}}{\pgfqpoint{4.620000in}{4.620000in}}%
\pgfusepath{clip}%
\pgfsetbuttcap%
\pgfsetroundjoin%
\definecolor{currentfill}{rgb}{1.000000,0.894118,0.788235}%
\pgfsetfillcolor{currentfill}%
\pgfsetlinewidth{0.000000pt}%
\definecolor{currentstroke}{rgb}{1.000000,0.894118,0.788235}%
\pgfsetstrokecolor{currentstroke}%
\pgfsetdash{}{0pt}%
\pgfpathmoveto{\pgfqpoint{3.324897in}{1.884585in}}%
\pgfpathlineto{\pgfqpoint{3.451141in}{1.957472in}}%
\pgfpathlineto{\pgfqpoint{3.451141in}{2.103247in}}%
\pgfpathlineto{\pgfqpoint{3.324897in}{2.030360in}}%
\pgfpathlineto{\pgfqpoint{3.324897in}{1.884585in}}%
\pgfpathclose%
\pgfusepath{fill}%
\end{pgfscope}%
\begin{pgfscope}%
\pgfpathrectangle{\pgfqpoint{0.765000in}{0.660000in}}{\pgfqpoint{4.620000in}{4.620000in}}%
\pgfusepath{clip}%
\pgfsetbuttcap%
\pgfsetroundjoin%
\definecolor{currentfill}{rgb}{1.000000,0.894118,0.788235}%
\pgfsetfillcolor{currentfill}%
\pgfsetlinewidth{0.000000pt}%
\definecolor{currentstroke}{rgb}{1.000000,0.894118,0.788235}%
\pgfsetstrokecolor{currentstroke}%
\pgfsetdash{}{0pt}%
\pgfpathmoveto{\pgfqpoint{3.198652in}{1.957472in}}%
\pgfpathlineto{\pgfqpoint{3.324897in}{1.884585in}}%
\pgfpathlineto{\pgfqpoint{3.324897in}{2.030360in}}%
\pgfpathlineto{\pgfqpoint{3.198652in}{2.103247in}}%
\pgfpathlineto{\pgfqpoint{3.198652in}{1.957472in}}%
\pgfpathclose%
\pgfusepath{fill}%
\end{pgfscope}%
\begin{pgfscope}%
\pgfpathrectangle{\pgfqpoint{0.765000in}{0.660000in}}{\pgfqpoint{4.620000in}{4.620000in}}%
\pgfusepath{clip}%
\pgfsetbuttcap%
\pgfsetroundjoin%
\definecolor{currentfill}{rgb}{1.000000,0.894118,0.788235}%
\pgfsetfillcolor{currentfill}%
\pgfsetlinewidth{0.000000pt}%
\definecolor{currentstroke}{rgb}{1.000000,0.894118,0.788235}%
\pgfsetstrokecolor{currentstroke}%
\pgfsetdash{}{0pt}%
\pgfpathmoveto{\pgfqpoint{3.860367in}{2.339514in}}%
\pgfpathlineto{\pgfqpoint{3.734122in}{2.266626in}}%
\pgfpathlineto{\pgfqpoint{3.198652in}{1.957472in}}%
\pgfpathlineto{\pgfqpoint{3.324897in}{2.030360in}}%
\pgfpathlineto{\pgfqpoint{3.860367in}{2.339514in}}%
\pgfpathclose%
\pgfusepath{fill}%
\end{pgfscope}%
\begin{pgfscope}%
\pgfpathrectangle{\pgfqpoint{0.765000in}{0.660000in}}{\pgfqpoint{4.620000in}{4.620000in}}%
\pgfusepath{clip}%
\pgfsetbuttcap%
\pgfsetroundjoin%
\definecolor{currentfill}{rgb}{1.000000,0.894118,0.788235}%
\pgfsetfillcolor{currentfill}%
\pgfsetlinewidth{0.000000pt}%
\definecolor{currentstroke}{rgb}{1.000000,0.894118,0.788235}%
\pgfsetstrokecolor{currentstroke}%
\pgfsetdash{}{0pt}%
\pgfpathmoveto{\pgfqpoint{3.986612in}{2.266626in}}%
\pgfpathlineto{\pgfqpoint{3.860367in}{2.339514in}}%
\pgfpathlineto{\pgfqpoint{3.324897in}{2.030360in}}%
\pgfpathlineto{\pgfqpoint{3.451141in}{1.957472in}}%
\pgfpathlineto{\pgfqpoint{3.986612in}{2.266626in}}%
\pgfpathclose%
\pgfusepath{fill}%
\end{pgfscope}%
\begin{pgfscope}%
\pgfpathrectangle{\pgfqpoint{0.765000in}{0.660000in}}{\pgfqpoint{4.620000in}{4.620000in}}%
\pgfusepath{clip}%
\pgfsetbuttcap%
\pgfsetroundjoin%
\definecolor{currentfill}{rgb}{1.000000,0.894118,0.788235}%
\pgfsetfillcolor{currentfill}%
\pgfsetlinewidth{0.000000pt}%
\definecolor{currentstroke}{rgb}{1.000000,0.894118,0.788235}%
\pgfsetstrokecolor{currentstroke}%
\pgfsetdash{}{0pt}%
\pgfpathmoveto{\pgfqpoint{3.860367in}{2.339514in}}%
\pgfpathlineto{\pgfqpoint{3.860367in}{2.485289in}}%
\pgfpathlineto{\pgfqpoint{3.324897in}{2.176135in}}%
\pgfpathlineto{\pgfqpoint{3.451141in}{1.957472in}}%
\pgfpathlineto{\pgfqpoint{3.860367in}{2.339514in}}%
\pgfpathclose%
\pgfusepath{fill}%
\end{pgfscope}%
\begin{pgfscope}%
\pgfpathrectangle{\pgfqpoint{0.765000in}{0.660000in}}{\pgfqpoint{4.620000in}{4.620000in}}%
\pgfusepath{clip}%
\pgfsetbuttcap%
\pgfsetroundjoin%
\definecolor{currentfill}{rgb}{1.000000,0.894118,0.788235}%
\pgfsetfillcolor{currentfill}%
\pgfsetlinewidth{0.000000pt}%
\definecolor{currentstroke}{rgb}{1.000000,0.894118,0.788235}%
\pgfsetstrokecolor{currentstroke}%
\pgfsetdash{}{0pt}%
\pgfpathmoveto{\pgfqpoint{3.986612in}{2.266626in}}%
\pgfpathlineto{\pgfqpoint{3.986612in}{2.412401in}}%
\pgfpathlineto{\pgfqpoint{3.451141in}{2.103247in}}%
\pgfpathlineto{\pgfqpoint{3.324897in}{2.030360in}}%
\pgfpathlineto{\pgfqpoint{3.986612in}{2.266626in}}%
\pgfpathclose%
\pgfusepath{fill}%
\end{pgfscope}%
\begin{pgfscope}%
\pgfpathrectangle{\pgfqpoint{0.765000in}{0.660000in}}{\pgfqpoint{4.620000in}{4.620000in}}%
\pgfusepath{clip}%
\pgfsetbuttcap%
\pgfsetroundjoin%
\definecolor{currentfill}{rgb}{1.000000,0.894118,0.788235}%
\pgfsetfillcolor{currentfill}%
\pgfsetlinewidth{0.000000pt}%
\definecolor{currentstroke}{rgb}{1.000000,0.894118,0.788235}%
\pgfsetstrokecolor{currentstroke}%
\pgfsetdash{}{0pt}%
\pgfpathmoveto{\pgfqpoint{3.860367in}{2.193739in}}%
\pgfpathlineto{\pgfqpoint{3.986612in}{2.266626in}}%
\pgfpathlineto{\pgfqpoint{3.451141in}{1.957472in}}%
\pgfpathlineto{\pgfqpoint{3.324897in}{1.884585in}}%
\pgfpathlineto{\pgfqpoint{3.860367in}{2.193739in}}%
\pgfpathclose%
\pgfusepath{fill}%
\end{pgfscope}%
\begin{pgfscope}%
\pgfpathrectangle{\pgfqpoint{0.765000in}{0.660000in}}{\pgfqpoint{4.620000in}{4.620000in}}%
\pgfusepath{clip}%
\pgfsetbuttcap%
\pgfsetroundjoin%
\definecolor{currentfill}{rgb}{1.000000,0.894118,0.788235}%
\pgfsetfillcolor{currentfill}%
\pgfsetlinewidth{0.000000pt}%
\definecolor{currentstroke}{rgb}{1.000000,0.894118,0.788235}%
\pgfsetstrokecolor{currentstroke}%
\pgfsetdash{}{0pt}%
\pgfpathmoveto{\pgfqpoint{3.986612in}{2.412401in}}%
\pgfpathlineto{\pgfqpoint{3.860367in}{2.485289in}}%
\pgfpathlineto{\pgfqpoint{3.324897in}{2.176135in}}%
\pgfpathlineto{\pgfqpoint{3.451141in}{2.103247in}}%
\pgfpathlineto{\pgfqpoint{3.986612in}{2.412401in}}%
\pgfpathclose%
\pgfusepath{fill}%
\end{pgfscope}%
\begin{pgfscope}%
\pgfpathrectangle{\pgfqpoint{0.765000in}{0.660000in}}{\pgfqpoint{4.620000in}{4.620000in}}%
\pgfusepath{clip}%
\pgfsetbuttcap%
\pgfsetroundjoin%
\definecolor{currentfill}{rgb}{1.000000,0.894118,0.788235}%
\pgfsetfillcolor{currentfill}%
\pgfsetlinewidth{0.000000pt}%
\definecolor{currentstroke}{rgb}{1.000000,0.894118,0.788235}%
\pgfsetstrokecolor{currentstroke}%
\pgfsetdash{}{0pt}%
\pgfpathmoveto{\pgfqpoint{3.734122in}{2.266626in}}%
\pgfpathlineto{\pgfqpoint{3.860367in}{2.193739in}}%
\pgfpathlineto{\pgfqpoint{3.324897in}{1.884585in}}%
\pgfpathlineto{\pgfqpoint{3.198652in}{1.957472in}}%
\pgfpathlineto{\pgfqpoint{3.734122in}{2.266626in}}%
\pgfpathclose%
\pgfusepath{fill}%
\end{pgfscope}%
\begin{pgfscope}%
\pgfpathrectangle{\pgfqpoint{0.765000in}{0.660000in}}{\pgfqpoint{4.620000in}{4.620000in}}%
\pgfusepath{clip}%
\pgfsetbuttcap%
\pgfsetroundjoin%
\definecolor{currentfill}{rgb}{1.000000,0.894118,0.788235}%
\pgfsetfillcolor{currentfill}%
\pgfsetlinewidth{0.000000pt}%
\definecolor{currentstroke}{rgb}{1.000000,0.894118,0.788235}%
\pgfsetstrokecolor{currentstroke}%
\pgfsetdash{}{0pt}%
\pgfpathmoveto{\pgfqpoint{3.860367in}{2.485289in}}%
\pgfpathlineto{\pgfqpoint{3.734122in}{2.412401in}}%
\pgfpathlineto{\pgfqpoint{3.198652in}{2.103247in}}%
\pgfpathlineto{\pgfqpoint{3.324897in}{2.176135in}}%
\pgfpathlineto{\pgfqpoint{3.860367in}{2.485289in}}%
\pgfpathclose%
\pgfusepath{fill}%
\end{pgfscope}%
\begin{pgfscope}%
\pgfpathrectangle{\pgfqpoint{0.765000in}{0.660000in}}{\pgfqpoint{4.620000in}{4.620000in}}%
\pgfusepath{clip}%
\pgfsetbuttcap%
\pgfsetroundjoin%
\definecolor{currentfill}{rgb}{1.000000,0.894118,0.788235}%
\pgfsetfillcolor{currentfill}%
\pgfsetlinewidth{0.000000pt}%
\definecolor{currentstroke}{rgb}{1.000000,0.894118,0.788235}%
\pgfsetstrokecolor{currentstroke}%
\pgfsetdash{}{0pt}%
\pgfpathmoveto{\pgfqpoint{3.734122in}{2.266626in}}%
\pgfpathlineto{\pgfqpoint{3.734122in}{2.412401in}}%
\pgfpathlineto{\pgfqpoint{3.198652in}{2.103247in}}%
\pgfpathlineto{\pgfqpoint{3.324897in}{1.884585in}}%
\pgfpathlineto{\pgfqpoint{3.734122in}{2.266626in}}%
\pgfpathclose%
\pgfusepath{fill}%
\end{pgfscope}%
\begin{pgfscope}%
\pgfpathrectangle{\pgfqpoint{0.765000in}{0.660000in}}{\pgfqpoint{4.620000in}{4.620000in}}%
\pgfusepath{clip}%
\pgfsetbuttcap%
\pgfsetroundjoin%
\definecolor{currentfill}{rgb}{1.000000,0.894118,0.788235}%
\pgfsetfillcolor{currentfill}%
\pgfsetlinewidth{0.000000pt}%
\definecolor{currentstroke}{rgb}{1.000000,0.894118,0.788235}%
\pgfsetstrokecolor{currentstroke}%
\pgfsetdash{}{0pt}%
\pgfpathmoveto{\pgfqpoint{3.860367in}{2.193739in}}%
\pgfpathlineto{\pgfqpoint{3.860367in}{2.339514in}}%
\pgfpathlineto{\pgfqpoint{3.324897in}{2.030360in}}%
\pgfpathlineto{\pgfqpoint{3.198652in}{1.957472in}}%
\pgfpathlineto{\pgfqpoint{3.860367in}{2.193739in}}%
\pgfpathclose%
\pgfusepath{fill}%
\end{pgfscope}%
\begin{pgfscope}%
\pgfpathrectangle{\pgfqpoint{0.765000in}{0.660000in}}{\pgfqpoint{4.620000in}{4.620000in}}%
\pgfusepath{clip}%
\pgfsetbuttcap%
\pgfsetroundjoin%
\definecolor{currentfill}{rgb}{1.000000,0.894118,0.788235}%
\pgfsetfillcolor{currentfill}%
\pgfsetlinewidth{0.000000pt}%
\definecolor{currentstroke}{rgb}{1.000000,0.894118,0.788235}%
\pgfsetstrokecolor{currentstroke}%
\pgfsetdash{}{0pt}%
\pgfpathmoveto{\pgfqpoint{3.734122in}{2.412401in}}%
\pgfpathlineto{\pgfqpoint{3.860367in}{2.339514in}}%
\pgfpathlineto{\pgfqpoint{3.324897in}{2.030360in}}%
\pgfpathlineto{\pgfqpoint{3.198652in}{2.103247in}}%
\pgfpathlineto{\pgfqpoint{3.734122in}{2.412401in}}%
\pgfpathclose%
\pgfusepath{fill}%
\end{pgfscope}%
\begin{pgfscope}%
\pgfpathrectangle{\pgfqpoint{0.765000in}{0.660000in}}{\pgfqpoint{4.620000in}{4.620000in}}%
\pgfusepath{clip}%
\pgfsetbuttcap%
\pgfsetroundjoin%
\definecolor{currentfill}{rgb}{1.000000,0.894118,0.788235}%
\pgfsetfillcolor{currentfill}%
\pgfsetlinewidth{0.000000pt}%
\definecolor{currentstroke}{rgb}{1.000000,0.894118,0.788235}%
\pgfsetstrokecolor{currentstroke}%
\pgfsetdash{}{0pt}%
\pgfpathmoveto{\pgfqpoint{3.860367in}{2.339514in}}%
\pgfpathlineto{\pgfqpoint{3.986612in}{2.412401in}}%
\pgfpathlineto{\pgfqpoint{3.451141in}{2.103247in}}%
\pgfpathlineto{\pgfqpoint{3.324897in}{2.030360in}}%
\pgfpathlineto{\pgfqpoint{3.860367in}{2.339514in}}%
\pgfpathclose%
\pgfusepath{fill}%
\end{pgfscope}%
\begin{pgfscope}%
\pgfpathrectangle{\pgfqpoint{0.765000in}{0.660000in}}{\pgfqpoint{4.620000in}{4.620000in}}%
\pgfusepath{clip}%
\pgfsetbuttcap%
\pgfsetroundjoin%
\definecolor{currentfill}{rgb}{1.000000,0.894118,0.788235}%
\pgfsetfillcolor{currentfill}%
\pgfsetlinewidth{0.000000pt}%
\definecolor{currentstroke}{rgb}{1.000000,0.894118,0.788235}%
\pgfsetstrokecolor{currentstroke}%
\pgfsetdash{}{0pt}%
\pgfpathmoveto{\pgfqpoint{3.860367in}{2.339514in}}%
\pgfpathlineto{\pgfqpoint{3.734122in}{2.266626in}}%
\pgfpathlineto{\pgfqpoint{3.860367in}{2.193739in}}%
\pgfpathlineto{\pgfqpoint{3.986612in}{2.266626in}}%
\pgfpathlineto{\pgfqpoint{3.860367in}{2.339514in}}%
\pgfpathclose%
\pgfusepath{fill}%
\end{pgfscope}%
\begin{pgfscope}%
\pgfpathrectangle{\pgfqpoint{0.765000in}{0.660000in}}{\pgfqpoint{4.620000in}{4.620000in}}%
\pgfusepath{clip}%
\pgfsetbuttcap%
\pgfsetroundjoin%
\definecolor{currentfill}{rgb}{1.000000,0.894118,0.788235}%
\pgfsetfillcolor{currentfill}%
\pgfsetlinewidth{0.000000pt}%
\definecolor{currentstroke}{rgb}{1.000000,0.894118,0.788235}%
\pgfsetstrokecolor{currentstroke}%
\pgfsetdash{}{0pt}%
\pgfpathmoveto{\pgfqpoint{3.860367in}{2.339514in}}%
\pgfpathlineto{\pgfqpoint{3.734122in}{2.266626in}}%
\pgfpathlineto{\pgfqpoint{3.734122in}{2.412401in}}%
\pgfpathlineto{\pgfqpoint{3.860367in}{2.485289in}}%
\pgfpathlineto{\pgfqpoint{3.860367in}{2.339514in}}%
\pgfpathclose%
\pgfusepath{fill}%
\end{pgfscope}%
\begin{pgfscope}%
\pgfpathrectangle{\pgfqpoint{0.765000in}{0.660000in}}{\pgfqpoint{4.620000in}{4.620000in}}%
\pgfusepath{clip}%
\pgfsetbuttcap%
\pgfsetroundjoin%
\definecolor{currentfill}{rgb}{1.000000,0.894118,0.788235}%
\pgfsetfillcolor{currentfill}%
\pgfsetlinewidth{0.000000pt}%
\definecolor{currentstroke}{rgb}{1.000000,0.894118,0.788235}%
\pgfsetstrokecolor{currentstroke}%
\pgfsetdash{}{0pt}%
\pgfpathmoveto{\pgfqpoint{3.860367in}{2.339514in}}%
\pgfpathlineto{\pgfqpoint{3.986612in}{2.266626in}}%
\pgfpathlineto{\pgfqpoint{3.986612in}{2.412401in}}%
\pgfpathlineto{\pgfqpoint{3.860367in}{2.485289in}}%
\pgfpathlineto{\pgfqpoint{3.860367in}{2.339514in}}%
\pgfpathclose%
\pgfusepath{fill}%
\end{pgfscope}%
\begin{pgfscope}%
\pgfpathrectangle{\pgfqpoint{0.765000in}{0.660000in}}{\pgfqpoint{4.620000in}{4.620000in}}%
\pgfusepath{clip}%
\pgfsetbuttcap%
\pgfsetroundjoin%
\definecolor{currentfill}{rgb}{1.000000,0.894118,0.788235}%
\pgfsetfillcolor{currentfill}%
\pgfsetlinewidth{0.000000pt}%
\definecolor{currentstroke}{rgb}{1.000000,0.894118,0.788235}%
\pgfsetstrokecolor{currentstroke}%
\pgfsetdash{}{0pt}%
\pgfpathmoveto{\pgfqpoint{3.860367in}{3.248275in}}%
\pgfpathlineto{\pgfqpoint{3.734122in}{3.175388in}}%
\pgfpathlineto{\pgfqpoint{3.860367in}{3.102500in}}%
\pgfpathlineto{\pgfqpoint{3.986612in}{3.175388in}}%
\pgfpathlineto{\pgfqpoint{3.860367in}{3.248275in}}%
\pgfpathclose%
\pgfusepath{fill}%
\end{pgfscope}%
\begin{pgfscope}%
\pgfpathrectangle{\pgfqpoint{0.765000in}{0.660000in}}{\pgfqpoint{4.620000in}{4.620000in}}%
\pgfusepath{clip}%
\pgfsetbuttcap%
\pgfsetroundjoin%
\definecolor{currentfill}{rgb}{1.000000,0.894118,0.788235}%
\pgfsetfillcolor{currentfill}%
\pgfsetlinewidth{0.000000pt}%
\definecolor{currentstroke}{rgb}{1.000000,0.894118,0.788235}%
\pgfsetstrokecolor{currentstroke}%
\pgfsetdash{}{0pt}%
\pgfpathmoveto{\pgfqpoint{3.860367in}{3.248275in}}%
\pgfpathlineto{\pgfqpoint{3.734122in}{3.175388in}}%
\pgfpathlineto{\pgfqpoint{3.734122in}{3.321163in}}%
\pgfpathlineto{\pgfqpoint{3.860367in}{3.394051in}}%
\pgfpathlineto{\pgfqpoint{3.860367in}{3.248275in}}%
\pgfpathclose%
\pgfusepath{fill}%
\end{pgfscope}%
\begin{pgfscope}%
\pgfpathrectangle{\pgfqpoint{0.765000in}{0.660000in}}{\pgfqpoint{4.620000in}{4.620000in}}%
\pgfusepath{clip}%
\pgfsetbuttcap%
\pgfsetroundjoin%
\definecolor{currentfill}{rgb}{1.000000,0.894118,0.788235}%
\pgfsetfillcolor{currentfill}%
\pgfsetlinewidth{0.000000pt}%
\definecolor{currentstroke}{rgb}{1.000000,0.894118,0.788235}%
\pgfsetstrokecolor{currentstroke}%
\pgfsetdash{}{0pt}%
\pgfpathmoveto{\pgfqpoint{3.860367in}{3.248275in}}%
\pgfpathlineto{\pgfqpoint{3.986612in}{3.175388in}}%
\pgfpathlineto{\pgfqpoint{3.986612in}{3.321163in}}%
\pgfpathlineto{\pgfqpoint{3.860367in}{3.394051in}}%
\pgfpathlineto{\pgfqpoint{3.860367in}{3.248275in}}%
\pgfpathclose%
\pgfusepath{fill}%
\end{pgfscope}%
\begin{pgfscope}%
\pgfpathrectangle{\pgfqpoint{0.765000in}{0.660000in}}{\pgfqpoint{4.620000in}{4.620000in}}%
\pgfusepath{clip}%
\pgfsetbuttcap%
\pgfsetroundjoin%
\definecolor{currentfill}{rgb}{1.000000,0.894118,0.788235}%
\pgfsetfillcolor{currentfill}%
\pgfsetlinewidth{0.000000pt}%
\definecolor{currentstroke}{rgb}{1.000000,0.894118,0.788235}%
\pgfsetstrokecolor{currentstroke}%
\pgfsetdash{}{0pt}%
\pgfpathmoveto{\pgfqpoint{3.860367in}{2.485289in}}%
\pgfpathlineto{\pgfqpoint{3.734122in}{2.412401in}}%
\pgfpathlineto{\pgfqpoint{3.860367in}{2.339514in}}%
\pgfpathlineto{\pgfqpoint{3.986612in}{2.412401in}}%
\pgfpathlineto{\pgfqpoint{3.860367in}{2.485289in}}%
\pgfpathclose%
\pgfusepath{fill}%
\end{pgfscope}%
\begin{pgfscope}%
\pgfpathrectangle{\pgfqpoint{0.765000in}{0.660000in}}{\pgfqpoint{4.620000in}{4.620000in}}%
\pgfusepath{clip}%
\pgfsetbuttcap%
\pgfsetroundjoin%
\definecolor{currentfill}{rgb}{1.000000,0.894118,0.788235}%
\pgfsetfillcolor{currentfill}%
\pgfsetlinewidth{0.000000pt}%
\definecolor{currentstroke}{rgb}{1.000000,0.894118,0.788235}%
\pgfsetstrokecolor{currentstroke}%
\pgfsetdash{}{0pt}%
\pgfpathmoveto{\pgfqpoint{3.860367in}{2.193739in}}%
\pgfpathlineto{\pgfqpoint{3.986612in}{2.266626in}}%
\pgfpathlineto{\pgfqpoint{3.986612in}{2.412401in}}%
\pgfpathlineto{\pgfqpoint{3.860367in}{2.339514in}}%
\pgfpathlineto{\pgfqpoint{3.860367in}{2.193739in}}%
\pgfpathclose%
\pgfusepath{fill}%
\end{pgfscope}%
\begin{pgfscope}%
\pgfpathrectangle{\pgfqpoint{0.765000in}{0.660000in}}{\pgfqpoint{4.620000in}{4.620000in}}%
\pgfusepath{clip}%
\pgfsetbuttcap%
\pgfsetroundjoin%
\definecolor{currentfill}{rgb}{1.000000,0.894118,0.788235}%
\pgfsetfillcolor{currentfill}%
\pgfsetlinewidth{0.000000pt}%
\definecolor{currentstroke}{rgb}{1.000000,0.894118,0.788235}%
\pgfsetstrokecolor{currentstroke}%
\pgfsetdash{}{0pt}%
\pgfpathmoveto{\pgfqpoint{3.734122in}{2.266626in}}%
\pgfpathlineto{\pgfqpoint{3.860367in}{2.193739in}}%
\pgfpathlineto{\pgfqpoint{3.860367in}{2.339514in}}%
\pgfpathlineto{\pgfqpoint{3.734122in}{2.412401in}}%
\pgfpathlineto{\pgfqpoint{3.734122in}{2.266626in}}%
\pgfpathclose%
\pgfusepath{fill}%
\end{pgfscope}%
\begin{pgfscope}%
\pgfpathrectangle{\pgfqpoint{0.765000in}{0.660000in}}{\pgfqpoint{4.620000in}{4.620000in}}%
\pgfusepath{clip}%
\pgfsetbuttcap%
\pgfsetroundjoin%
\definecolor{currentfill}{rgb}{1.000000,0.894118,0.788235}%
\pgfsetfillcolor{currentfill}%
\pgfsetlinewidth{0.000000pt}%
\definecolor{currentstroke}{rgb}{1.000000,0.894118,0.788235}%
\pgfsetstrokecolor{currentstroke}%
\pgfsetdash{}{0pt}%
\pgfpathmoveto{\pgfqpoint{3.860367in}{3.394051in}}%
\pgfpathlineto{\pgfqpoint{3.734122in}{3.321163in}}%
\pgfpathlineto{\pgfqpoint{3.860367in}{3.248275in}}%
\pgfpathlineto{\pgfqpoint{3.986612in}{3.321163in}}%
\pgfpathlineto{\pgfqpoint{3.860367in}{3.394051in}}%
\pgfpathclose%
\pgfusepath{fill}%
\end{pgfscope}%
\begin{pgfscope}%
\pgfpathrectangle{\pgfqpoint{0.765000in}{0.660000in}}{\pgfqpoint{4.620000in}{4.620000in}}%
\pgfusepath{clip}%
\pgfsetbuttcap%
\pgfsetroundjoin%
\definecolor{currentfill}{rgb}{1.000000,0.894118,0.788235}%
\pgfsetfillcolor{currentfill}%
\pgfsetlinewidth{0.000000pt}%
\definecolor{currentstroke}{rgb}{1.000000,0.894118,0.788235}%
\pgfsetstrokecolor{currentstroke}%
\pgfsetdash{}{0pt}%
\pgfpathmoveto{\pgfqpoint{3.860367in}{3.102500in}}%
\pgfpathlineto{\pgfqpoint{3.986612in}{3.175388in}}%
\pgfpathlineto{\pgfqpoint{3.986612in}{3.321163in}}%
\pgfpathlineto{\pgfqpoint{3.860367in}{3.248275in}}%
\pgfpathlineto{\pgfqpoint{3.860367in}{3.102500in}}%
\pgfpathclose%
\pgfusepath{fill}%
\end{pgfscope}%
\begin{pgfscope}%
\pgfpathrectangle{\pgfqpoint{0.765000in}{0.660000in}}{\pgfqpoint{4.620000in}{4.620000in}}%
\pgfusepath{clip}%
\pgfsetbuttcap%
\pgfsetroundjoin%
\definecolor{currentfill}{rgb}{1.000000,0.894118,0.788235}%
\pgfsetfillcolor{currentfill}%
\pgfsetlinewidth{0.000000pt}%
\definecolor{currentstroke}{rgb}{1.000000,0.894118,0.788235}%
\pgfsetstrokecolor{currentstroke}%
\pgfsetdash{}{0pt}%
\pgfpathmoveto{\pgfqpoint{3.734122in}{3.175388in}}%
\pgfpathlineto{\pgfqpoint{3.860367in}{3.102500in}}%
\pgfpathlineto{\pgfqpoint{3.860367in}{3.248275in}}%
\pgfpathlineto{\pgfqpoint{3.734122in}{3.321163in}}%
\pgfpathlineto{\pgfqpoint{3.734122in}{3.175388in}}%
\pgfpathclose%
\pgfusepath{fill}%
\end{pgfscope}%
\begin{pgfscope}%
\pgfpathrectangle{\pgfqpoint{0.765000in}{0.660000in}}{\pgfqpoint{4.620000in}{4.620000in}}%
\pgfusepath{clip}%
\pgfsetbuttcap%
\pgfsetroundjoin%
\definecolor{currentfill}{rgb}{1.000000,0.894118,0.788235}%
\pgfsetfillcolor{currentfill}%
\pgfsetlinewidth{0.000000pt}%
\definecolor{currentstroke}{rgb}{1.000000,0.894118,0.788235}%
\pgfsetstrokecolor{currentstroke}%
\pgfsetdash{}{0pt}%
\pgfpathmoveto{\pgfqpoint{3.860367in}{2.339514in}}%
\pgfpathlineto{\pgfqpoint{3.734122in}{2.266626in}}%
\pgfpathlineto{\pgfqpoint{3.734122in}{3.175388in}}%
\pgfpathlineto{\pgfqpoint{3.860367in}{3.248275in}}%
\pgfpathlineto{\pgfqpoint{3.860367in}{2.339514in}}%
\pgfpathclose%
\pgfusepath{fill}%
\end{pgfscope}%
\begin{pgfscope}%
\pgfpathrectangle{\pgfqpoint{0.765000in}{0.660000in}}{\pgfqpoint{4.620000in}{4.620000in}}%
\pgfusepath{clip}%
\pgfsetbuttcap%
\pgfsetroundjoin%
\definecolor{currentfill}{rgb}{1.000000,0.894118,0.788235}%
\pgfsetfillcolor{currentfill}%
\pgfsetlinewidth{0.000000pt}%
\definecolor{currentstroke}{rgb}{1.000000,0.894118,0.788235}%
\pgfsetstrokecolor{currentstroke}%
\pgfsetdash{}{0pt}%
\pgfpathmoveto{\pgfqpoint{3.986612in}{2.266626in}}%
\pgfpathlineto{\pgfqpoint{3.860367in}{2.339514in}}%
\pgfpathlineto{\pgfqpoint{3.860367in}{3.248275in}}%
\pgfpathlineto{\pgfqpoint{3.986612in}{3.175388in}}%
\pgfpathlineto{\pgfqpoint{3.986612in}{2.266626in}}%
\pgfpathclose%
\pgfusepath{fill}%
\end{pgfscope}%
\begin{pgfscope}%
\pgfpathrectangle{\pgfqpoint{0.765000in}{0.660000in}}{\pgfqpoint{4.620000in}{4.620000in}}%
\pgfusepath{clip}%
\pgfsetbuttcap%
\pgfsetroundjoin%
\definecolor{currentfill}{rgb}{1.000000,0.894118,0.788235}%
\pgfsetfillcolor{currentfill}%
\pgfsetlinewidth{0.000000pt}%
\definecolor{currentstroke}{rgb}{1.000000,0.894118,0.788235}%
\pgfsetstrokecolor{currentstroke}%
\pgfsetdash{}{0pt}%
\pgfpathmoveto{\pgfqpoint{3.860367in}{2.339514in}}%
\pgfpathlineto{\pgfqpoint{3.860367in}{2.485289in}}%
\pgfpathlineto{\pgfqpoint{3.860367in}{3.394051in}}%
\pgfpathlineto{\pgfqpoint{3.986612in}{3.175388in}}%
\pgfpathlineto{\pgfqpoint{3.860367in}{2.339514in}}%
\pgfpathclose%
\pgfusepath{fill}%
\end{pgfscope}%
\begin{pgfscope}%
\pgfpathrectangle{\pgfqpoint{0.765000in}{0.660000in}}{\pgfqpoint{4.620000in}{4.620000in}}%
\pgfusepath{clip}%
\pgfsetbuttcap%
\pgfsetroundjoin%
\definecolor{currentfill}{rgb}{1.000000,0.894118,0.788235}%
\pgfsetfillcolor{currentfill}%
\pgfsetlinewidth{0.000000pt}%
\definecolor{currentstroke}{rgb}{1.000000,0.894118,0.788235}%
\pgfsetstrokecolor{currentstroke}%
\pgfsetdash{}{0pt}%
\pgfpathmoveto{\pgfqpoint{3.986612in}{2.266626in}}%
\pgfpathlineto{\pgfqpoint{3.986612in}{2.412401in}}%
\pgfpathlineto{\pgfqpoint{3.986612in}{3.321163in}}%
\pgfpathlineto{\pgfqpoint{3.860367in}{3.248275in}}%
\pgfpathlineto{\pgfqpoint{3.986612in}{2.266626in}}%
\pgfpathclose%
\pgfusepath{fill}%
\end{pgfscope}%
\begin{pgfscope}%
\pgfpathrectangle{\pgfqpoint{0.765000in}{0.660000in}}{\pgfqpoint{4.620000in}{4.620000in}}%
\pgfusepath{clip}%
\pgfsetbuttcap%
\pgfsetroundjoin%
\definecolor{currentfill}{rgb}{1.000000,0.894118,0.788235}%
\pgfsetfillcolor{currentfill}%
\pgfsetlinewidth{0.000000pt}%
\definecolor{currentstroke}{rgb}{1.000000,0.894118,0.788235}%
\pgfsetstrokecolor{currentstroke}%
\pgfsetdash{}{0pt}%
\pgfpathmoveto{\pgfqpoint{3.860367in}{2.193739in}}%
\pgfpathlineto{\pgfqpoint{3.986612in}{2.266626in}}%
\pgfpathlineto{\pgfqpoint{3.986612in}{3.175388in}}%
\pgfpathlineto{\pgfqpoint{3.860367in}{3.102500in}}%
\pgfpathlineto{\pgfqpoint{3.860367in}{2.193739in}}%
\pgfpathclose%
\pgfusepath{fill}%
\end{pgfscope}%
\begin{pgfscope}%
\pgfpathrectangle{\pgfqpoint{0.765000in}{0.660000in}}{\pgfqpoint{4.620000in}{4.620000in}}%
\pgfusepath{clip}%
\pgfsetbuttcap%
\pgfsetroundjoin%
\definecolor{currentfill}{rgb}{1.000000,0.894118,0.788235}%
\pgfsetfillcolor{currentfill}%
\pgfsetlinewidth{0.000000pt}%
\definecolor{currentstroke}{rgb}{1.000000,0.894118,0.788235}%
\pgfsetstrokecolor{currentstroke}%
\pgfsetdash{}{0pt}%
\pgfpathmoveto{\pgfqpoint{3.986612in}{2.412401in}}%
\pgfpathlineto{\pgfqpoint{3.860367in}{2.485289in}}%
\pgfpathlineto{\pgfqpoint{3.860367in}{3.394051in}}%
\pgfpathlineto{\pgfqpoint{3.986612in}{3.321163in}}%
\pgfpathlineto{\pgfqpoint{3.986612in}{2.412401in}}%
\pgfpathclose%
\pgfusepath{fill}%
\end{pgfscope}%
\begin{pgfscope}%
\pgfpathrectangle{\pgfqpoint{0.765000in}{0.660000in}}{\pgfqpoint{4.620000in}{4.620000in}}%
\pgfusepath{clip}%
\pgfsetbuttcap%
\pgfsetroundjoin%
\definecolor{currentfill}{rgb}{1.000000,0.894118,0.788235}%
\pgfsetfillcolor{currentfill}%
\pgfsetlinewidth{0.000000pt}%
\definecolor{currentstroke}{rgb}{1.000000,0.894118,0.788235}%
\pgfsetstrokecolor{currentstroke}%
\pgfsetdash{}{0pt}%
\pgfpathmoveto{\pgfqpoint{3.734122in}{2.266626in}}%
\pgfpathlineto{\pgfqpoint{3.860367in}{2.193739in}}%
\pgfpathlineto{\pgfqpoint{3.860367in}{3.102500in}}%
\pgfpathlineto{\pgfqpoint{3.734122in}{3.175388in}}%
\pgfpathlineto{\pgfqpoint{3.734122in}{2.266626in}}%
\pgfpathclose%
\pgfusepath{fill}%
\end{pgfscope}%
\begin{pgfscope}%
\pgfpathrectangle{\pgfqpoint{0.765000in}{0.660000in}}{\pgfqpoint{4.620000in}{4.620000in}}%
\pgfusepath{clip}%
\pgfsetbuttcap%
\pgfsetroundjoin%
\definecolor{currentfill}{rgb}{1.000000,0.894118,0.788235}%
\pgfsetfillcolor{currentfill}%
\pgfsetlinewidth{0.000000pt}%
\definecolor{currentstroke}{rgb}{1.000000,0.894118,0.788235}%
\pgfsetstrokecolor{currentstroke}%
\pgfsetdash{}{0pt}%
\pgfpathmoveto{\pgfqpoint{3.860367in}{2.485289in}}%
\pgfpathlineto{\pgfqpoint{3.734122in}{2.412401in}}%
\pgfpathlineto{\pgfqpoint{3.734122in}{3.321163in}}%
\pgfpathlineto{\pgfqpoint{3.860367in}{3.394051in}}%
\pgfpathlineto{\pgfqpoint{3.860367in}{2.485289in}}%
\pgfpathclose%
\pgfusepath{fill}%
\end{pgfscope}%
\begin{pgfscope}%
\pgfpathrectangle{\pgfqpoint{0.765000in}{0.660000in}}{\pgfqpoint{4.620000in}{4.620000in}}%
\pgfusepath{clip}%
\pgfsetbuttcap%
\pgfsetroundjoin%
\definecolor{currentfill}{rgb}{1.000000,0.894118,0.788235}%
\pgfsetfillcolor{currentfill}%
\pgfsetlinewidth{0.000000pt}%
\definecolor{currentstroke}{rgb}{1.000000,0.894118,0.788235}%
\pgfsetstrokecolor{currentstroke}%
\pgfsetdash{}{0pt}%
\pgfpathmoveto{\pgfqpoint{3.734122in}{2.266626in}}%
\pgfpathlineto{\pgfqpoint{3.734122in}{2.412401in}}%
\pgfpathlineto{\pgfqpoint{3.734122in}{3.321163in}}%
\pgfpathlineto{\pgfqpoint{3.860367in}{3.102500in}}%
\pgfpathlineto{\pgfqpoint{3.734122in}{2.266626in}}%
\pgfpathclose%
\pgfusepath{fill}%
\end{pgfscope}%
\begin{pgfscope}%
\pgfpathrectangle{\pgfqpoint{0.765000in}{0.660000in}}{\pgfqpoint{4.620000in}{4.620000in}}%
\pgfusepath{clip}%
\pgfsetbuttcap%
\pgfsetroundjoin%
\definecolor{currentfill}{rgb}{1.000000,0.894118,0.788235}%
\pgfsetfillcolor{currentfill}%
\pgfsetlinewidth{0.000000pt}%
\definecolor{currentstroke}{rgb}{1.000000,0.894118,0.788235}%
\pgfsetstrokecolor{currentstroke}%
\pgfsetdash{}{0pt}%
\pgfpathmoveto{\pgfqpoint{3.860367in}{2.193739in}}%
\pgfpathlineto{\pgfqpoint{3.860367in}{2.339514in}}%
\pgfpathlineto{\pgfqpoint{3.860367in}{3.248275in}}%
\pgfpathlineto{\pgfqpoint{3.734122in}{3.175388in}}%
\pgfpathlineto{\pgfqpoint{3.860367in}{2.193739in}}%
\pgfpathclose%
\pgfusepath{fill}%
\end{pgfscope}%
\begin{pgfscope}%
\pgfpathrectangle{\pgfqpoint{0.765000in}{0.660000in}}{\pgfqpoint{4.620000in}{4.620000in}}%
\pgfusepath{clip}%
\pgfsetbuttcap%
\pgfsetroundjoin%
\definecolor{currentfill}{rgb}{1.000000,0.894118,0.788235}%
\pgfsetfillcolor{currentfill}%
\pgfsetlinewidth{0.000000pt}%
\definecolor{currentstroke}{rgb}{1.000000,0.894118,0.788235}%
\pgfsetstrokecolor{currentstroke}%
\pgfsetdash{}{0pt}%
\pgfpathmoveto{\pgfqpoint{3.734122in}{2.412401in}}%
\pgfpathlineto{\pgfqpoint{3.860367in}{2.339514in}}%
\pgfpathlineto{\pgfqpoint{3.860367in}{3.248275in}}%
\pgfpathlineto{\pgfqpoint{3.734122in}{3.321163in}}%
\pgfpathlineto{\pgfqpoint{3.734122in}{2.412401in}}%
\pgfpathclose%
\pgfusepath{fill}%
\end{pgfscope}%
\begin{pgfscope}%
\pgfpathrectangle{\pgfqpoint{0.765000in}{0.660000in}}{\pgfqpoint{4.620000in}{4.620000in}}%
\pgfusepath{clip}%
\pgfsetbuttcap%
\pgfsetroundjoin%
\definecolor{currentfill}{rgb}{1.000000,0.894118,0.788235}%
\pgfsetfillcolor{currentfill}%
\pgfsetlinewidth{0.000000pt}%
\definecolor{currentstroke}{rgb}{1.000000,0.894118,0.788235}%
\pgfsetstrokecolor{currentstroke}%
\pgfsetdash{}{0pt}%
\pgfpathmoveto{\pgfqpoint{3.860367in}{2.339514in}}%
\pgfpathlineto{\pgfqpoint{3.986612in}{2.412401in}}%
\pgfpathlineto{\pgfqpoint{3.986612in}{3.321163in}}%
\pgfpathlineto{\pgfqpoint{3.860367in}{3.248275in}}%
\pgfpathlineto{\pgfqpoint{3.860367in}{2.339514in}}%
\pgfpathclose%
\pgfusepath{fill}%
\end{pgfscope}%
\begin{pgfscope}%
\pgfpathrectangle{\pgfqpoint{0.765000in}{0.660000in}}{\pgfqpoint{4.620000in}{4.620000in}}%
\pgfusepath{clip}%
\pgfsetbuttcap%
\pgfsetroundjoin%
\definecolor{currentfill}{rgb}{1.000000,0.894118,0.788235}%
\pgfsetfillcolor{currentfill}%
\pgfsetlinewidth{0.000000pt}%
\definecolor{currentstroke}{rgb}{1.000000,0.894118,0.788235}%
\pgfsetstrokecolor{currentstroke}%
\pgfsetdash{}{0pt}%
\pgfpathmoveto{\pgfqpoint{3.860367in}{2.339514in}}%
\pgfpathlineto{\pgfqpoint{3.734122in}{2.266626in}}%
\pgfpathlineto{\pgfqpoint{3.860367in}{2.193739in}}%
\pgfpathlineto{\pgfqpoint{3.986612in}{2.266626in}}%
\pgfpathlineto{\pgfqpoint{3.860367in}{2.339514in}}%
\pgfpathclose%
\pgfusepath{fill}%
\end{pgfscope}%
\begin{pgfscope}%
\pgfpathrectangle{\pgfqpoint{0.765000in}{0.660000in}}{\pgfqpoint{4.620000in}{4.620000in}}%
\pgfusepath{clip}%
\pgfsetbuttcap%
\pgfsetroundjoin%
\definecolor{currentfill}{rgb}{1.000000,0.894118,0.788235}%
\pgfsetfillcolor{currentfill}%
\pgfsetlinewidth{0.000000pt}%
\definecolor{currentstroke}{rgb}{1.000000,0.894118,0.788235}%
\pgfsetstrokecolor{currentstroke}%
\pgfsetdash{}{0pt}%
\pgfpathmoveto{\pgfqpoint{3.860367in}{2.339514in}}%
\pgfpathlineto{\pgfqpoint{3.734122in}{2.266626in}}%
\pgfpathlineto{\pgfqpoint{3.734122in}{2.412401in}}%
\pgfpathlineto{\pgfqpoint{3.860367in}{2.485289in}}%
\pgfpathlineto{\pgfqpoint{3.860367in}{2.339514in}}%
\pgfpathclose%
\pgfusepath{fill}%
\end{pgfscope}%
\begin{pgfscope}%
\pgfpathrectangle{\pgfqpoint{0.765000in}{0.660000in}}{\pgfqpoint{4.620000in}{4.620000in}}%
\pgfusepath{clip}%
\pgfsetbuttcap%
\pgfsetroundjoin%
\definecolor{currentfill}{rgb}{1.000000,0.894118,0.788235}%
\pgfsetfillcolor{currentfill}%
\pgfsetlinewidth{0.000000pt}%
\definecolor{currentstroke}{rgb}{1.000000,0.894118,0.788235}%
\pgfsetstrokecolor{currentstroke}%
\pgfsetdash{}{0pt}%
\pgfpathmoveto{\pgfqpoint{3.860367in}{2.339514in}}%
\pgfpathlineto{\pgfqpoint{3.986612in}{2.266626in}}%
\pgfpathlineto{\pgfqpoint{3.986612in}{2.412401in}}%
\pgfpathlineto{\pgfqpoint{3.860367in}{2.485289in}}%
\pgfpathlineto{\pgfqpoint{3.860367in}{2.339514in}}%
\pgfpathclose%
\pgfusepath{fill}%
\end{pgfscope}%
\begin{pgfscope}%
\pgfpathrectangle{\pgfqpoint{0.765000in}{0.660000in}}{\pgfqpoint{4.620000in}{4.620000in}}%
\pgfusepath{clip}%
\pgfsetbuttcap%
\pgfsetroundjoin%
\definecolor{currentfill}{rgb}{1.000000,0.894118,0.788235}%
\pgfsetfillcolor{currentfill}%
\pgfsetlinewidth{0.000000pt}%
\definecolor{currentstroke}{rgb}{1.000000,0.894118,0.788235}%
\pgfsetstrokecolor{currentstroke}%
\pgfsetdash{}{0pt}%
\pgfpathmoveto{\pgfqpoint{10.295205in}{-1.375642in}}%
\pgfpathlineto{\pgfqpoint{10.168960in}{-1.448529in}}%
\pgfpathlineto{\pgfqpoint{10.295205in}{-1.521417in}}%
\pgfpathlineto{\pgfqpoint{10.421450in}{-1.448529in}}%
\pgfpathlineto{\pgfqpoint{10.295205in}{-1.375642in}}%
\pgfpathclose%
\pgfusepath{fill}%
\end{pgfscope}%
\begin{pgfscope}%
\pgfpathrectangle{\pgfqpoint{0.765000in}{0.660000in}}{\pgfqpoint{4.620000in}{4.620000in}}%
\pgfusepath{clip}%
\pgfsetbuttcap%
\pgfsetroundjoin%
\definecolor{currentfill}{rgb}{1.000000,0.894118,0.788235}%
\pgfsetfillcolor{currentfill}%
\pgfsetlinewidth{0.000000pt}%
\definecolor{currentstroke}{rgb}{1.000000,0.894118,0.788235}%
\pgfsetstrokecolor{currentstroke}%
\pgfsetdash{}{0pt}%
\pgfpathmoveto{\pgfqpoint{10.295205in}{-1.375642in}}%
\pgfpathlineto{\pgfqpoint{10.168960in}{-1.448529in}}%
\pgfpathlineto{\pgfqpoint{10.168960in}{-1.302754in}}%
\pgfpathlineto{\pgfqpoint{10.295205in}{-1.229867in}}%
\pgfpathlineto{\pgfqpoint{10.295205in}{-1.375642in}}%
\pgfpathclose%
\pgfusepath{fill}%
\end{pgfscope}%
\begin{pgfscope}%
\pgfpathrectangle{\pgfqpoint{0.765000in}{0.660000in}}{\pgfqpoint{4.620000in}{4.620000in}}%
\pgfusepath{clip}%
\pgfsetbuttcap%
\pgfsetroundjoin%
\definecolor{currentfill}{rgb}{1.000000,0.894118,0.788235}%
\pgfsetfillcolor{currentfill}%
\pgfsetlinewidth{0.000000pt}%
\definecolor{currentstroke}{rgb}{1.000000,0.894118,0.788235}%
\pgfsetstrokecolor{currentstroke}%
\pgfsetdash{}{0pt}%
\pgfpathmoveto{\pgfqpoint{10.295205in}{-1.375642in}}%
\pgfpathlineto{\pgfqpoint{10.421450in}{-1.448529in}}%
\pgfpathlineto{\pgfqpoint{10.421450in}{-1.302754in}}%
\pgfpathlineto{\pgfqpoint{10.295205in}{-1.229867in}}%
\pgfpathlineto{\pgfqpoint{10.295205in}{-1.375642in}}%
\pgfpathclose%
\pgfusepath{fill}%
\end{pgfscope}%
\begin{pgfscope}%
\pgfpathrectangle{\pgfqpoint{0.765000in}{0.660000in}}{\pgfqpoint{4.620000in}{4.620000in}}%
\pgfusepath{clip}%
\pgfsetbuttcap%
\pgfsetroundjoin%
\definecolor{currentfill}{rgb}{1.000000,0.894118,0.788235}%
\pgfsetfillcolor{currentfill}%
\pgfsetlinewidth{0.000000pt}%
\definecolor{currentstroke}{rgb}{1.000000,0.894118,0.788235}%
\pgfsetstrokecolor{currentstroke}%
\pgfsetdash{}{0pt}%
\pgfpathmoveto{\pgfqpoint{3.860367in}{2.485289in}}%
\pgfpathlineto{\pgfqpoint{3.734122in}{2.412401in}}%
\pgfpathlineto{\pgfqpoint{3.860367in}{2.339514in}}%
\pgfpathlineto{\pgfqpoint{3.986612in}{2.412401in}}%
\pgfpathlineto{\pgfqpoint{3.860367in}{2.485289in}}%
\pgfpathclose%
\pgfusepath{fill}%
\end{pgfscope}%
\begin{pgfscope}%
\pgfpathrectangle{\pgfqpoint{0.765000in}{0.660000in}}{\pgfqpoint{4.620000in}{4.620000in}}%
\pgfusepath{clip}%
\pgfsetbuttcap%
\pgfsetroundjoin%
\definecolor{currentfill}{rgb}{1.000000,0.894118,0.788235}%
\pgfsetfillcolor{currentfill}%
\pgfsetlinewidth{0.000000pt}%
\definecolor{currentstroke}{rgb}{1.000000,0.894118,0.788235}%
\pgfsetstrokecolor{currentstroke}%
\pgfsetdash{}{0pt}%
\pgfpathmoveto{\pgfqpoint{3.860367in}{2.193739in}}%
\pgfpathlineto{\pgfqpoint{3.986612in}{2.266626in}}%
\pgfpathlineto{\pgfqpoint{3.986612in}{2.412401in}}%
\pgfpathlineto{\pgfqpoint{3.860367in}{2.339514in}}%
\pgfpathlineto{\pgfqpoint{3.860367in}{2.193739in}}%
\pgfpathclose%
\pgfusepath{fill}%
\end{pgfscope}%
\begin{pgfscope}%
\pgfpathrectangle{\pgfqpoint{0.765000in}{0.660000in}}{\pgfqpoint{4.620000in}{4.620000in}}%
\pgfusepath{clip}%
\pgfsetbuttcap%
\pgfsetroundjoin%
\definecolor{currentfill}{rgb}{1.000000,0.894118,0.788235}%
\pgfsetfillcolor{currentfill}%
\pgfsetlinewidth{0.000000pt}%
\definecolor{currentstroke}{rgb}{1.000000,0.894118,0.788235}%
\pgfsetstrokecolor{currentstroke}%
\pgfsetdash{}{0pt}%
\pgfpathmoveto{\pgfqpoint{3.734122in}{2.266626in}}%
\pgfpathlineto{\pgfqpoint{3.860367in}{2.193739in}}%
\pgfpathlineto{\pgfqpoint{3.860367in}{2.339514in}}%
\pgfpathlineto{\pgfqpoint{3.734122in}{2.412401in}}%
\pgfpathlineto{\pgfqpoint{3.734122in}{2.266626in}}%
\pgfpathclose%
\pgfusepath{fill}%
\end{pgfscope}%
\begin{pgfscope}%
\pgfpathrectangle{\pgfqpoint{0.765000in}{0.660000in}}{\pgfqpoint{4.620000in}{4.620000in}}%
\pgfusepath{clip}%
\pgfsetbuttcap%
\pgfsetroundjoin%
\definecolor{currentfill}{rgb}{1.000000,0.894118,0.788235}%
\pgfsetfillcolor{currentfill}%
\pgfsetlinewidth{0.000000pt}%
\definecolor{currentstroke}{rgb}{1.000000,0.894118,0.788235}%
\pgfsetstrokecolor{currentstroke}%
\pgfsetdash{}{0pt}%
\pgfpathmoveto{\pgfqpoint{10.295205in}{-1.229867in}}%
\pgfpathlineto{\pgfqpoint{10.168960in}{-1.302754in}}%
\pgfpathlineto{\pgfqpoint{10.295205in}{-1.375642in}}%
\pgfpathlineto{\pgfqpoint{10.421450in}{-1.302754in}}%
\pgfpathlineto{\pgfqpoint{10.295205in}{-1.229867in}}%
\pgfpathclose%
\pgfusepath{fill}%
\end{pgfscope}%
\begin{pgfscope}%
\pgfpathrectangle{\pgfqpoint{0.765000in}{0.660000in}}{\pgfqpoint{4.620000in}{4.620000in}}%
\pgfusepath{clip}%
\pgfsetbuttcap%
\pgfsetroundjoin%
\definecolor{currentfill}{rgb}{1.000000,0.894118,0.788235}%
\pgfsetfillcolor{currentfill}%
\pgfsetlinewidth{0.000000pt}%
\definecolor{currentstroke}{rgb}{1.000000,0.894118,0.788235}%
\pgfsetstrokecolor{currentstroke}%
\pgfsetdash{}{0pt}%
\pgfpathmoveto{\pgfqpoint{10.295205in}{-1.521417in}}%
\pgfpathlineto{\pgfqpoint{10.421450in}{-1.448529in}}%
\pgfpathlineto{\pgfqpoint{10.421450in}{-1.302754in}}%
\pgfpathlineto{\pgfqpoint{10.295205in}{-1.375642in}}%
\pgfpathlineto{\pgfqpoint{10.295205in}{-1.521417in}}%
\pgfpathclose%
\pgfusepath{fill}%
\end{pgfscope}%
\begin{pgfscope}%
\pgfpathrectangle{\pgfqpoint{0.765000in}{0.660000in}}{\pgfqpoint{4.620000in}{4.620000in}}%
\pgfusepath{clip}%
\pgfsetbuttcap%
\pgfsetroundjoin%
\definecolor{currentfill}{rgb}{1.000000,0.894118,0.788235}%
\pgfsetfillcolor{currentfill}%
\pgfsetlinewidth{0.000000pt}%
\definecolor{currentstroke}{rgb}{1.000000,0.894118,0.788235}%
\pgfsetstrokecolor{currentstroke}%
\pgfsetdash{}{0pt}%
\pgfpathmoveto{\pgfqpoint{10.168960in}{-1.448529in}}%
\pgfpathlineto{\pgfqpoint{10.295205in}{-1.521417in}}%
\pgfpathlineto{\pgfqpoint{10.295205in}{-1.375642in}}%
\pgfpathlineto{\pgfqpoint{10.168960in}{-1.302754in}}%
\pgfpathlineto{\pgfqpoint{10.168960in}{-1.448529in}}%
\pgfpathclose%
\pgfusepath{fill}%
\end{pgfscope}%
\begin{pgfscope}%
\pgfpathrectangle{\pgfqpoint{0.765000in}{0.660000in}}{\pgfqpoint{4.620000in}{4.620000in}}%
\pgfusepath{clip}%
\pgfsetbuttcap%
\pgfsetroundjoin%
\definecolor{currentfill}{rgb}{1.000000,0.894118,0.788235}%
\pgfsetfillcolor{currentfill}%
\pgfsetlinewidth{0.000000pt}%
\definecolor{currentstroke}{rgb}{1.000000,0.894118,0.788235}%
\pgfsetstrokecolor{currentstroke}%
\pgfsetdash{}{0pt}%
\pgfpathmoveto{\pgfqpoint{3.860367in}{2.339514in}}%
\pgfpathlineto{\pgfqpoint{3.734122in}{2.266626in}}%
\pgfpathlineto{\pgfqpoint{10.168960in}{-1.448529in}}%
\pgfpathlineto{\pgfqpoint{10.295205in}{-1.375642in}}%
\pgfpathlineto{\pgfqpoint{3.860367in}{2.339514in}}%
\pgfpathclose%
\pgfusepath{fill}%
\end{pgfscope}%
\begin{pgfscope}%
\pgfpathrectangle{\pgfqpoint{0.765000in}{0.660000in}}{\pgfqpoint{4.620000in}{4.620000in}}%
\pgfusepath{clip}%
\pgfsetbuttcap%
\pgfsetroundjoin%
\definecolor{currentfill}{rgb}{1.000000,0.894118,0.788235}%
\pgfsetfillcolor{currentfill}%
\pgfsetlinewidth{0.000000pt}%
\definecolor{currentstroke}{rgb}{1.000000,0.894118,0.788235}%
\pgfsetstrokecolor{currentstroke}%
\pgfsetdash{}{0pt}%
\pgfpathmoveto{\pgfqpoint{3.986612in}{2.266626in}}%
\pgfpathlineto{\pgfqpoint{3.860367in}{2.339514in}}%
\pgfpathlineto{\pgfqpoint{10.295205in}{-1.375642in}}%
\pgfpathlineto{\pgfqpoint{10.421450in}{-1.448529in}}%
\pgfpathlineto{\pgfqpoint{3.986612in}{2.266626in}}%
\pgfpathclose%
\pgfusepath{fill}%
\end{pgfscope}%
\begin{pgfscope}%
\pgfpathrectangle{\pgfqpoint{0.765000in}{0.660000in}}{\pgfqpoint{4.620000in}{4.620000in}}%
\pgfusepath{clip}%
\pgfsetbuttcap%
\pgfsetroundjoin%
\definecolor{currentfill}{rgb}{1.000000,0.894118,0.788235}%
\pgfsetfillcolor{currentfill}%
\pgfsetlinewidth{0.000000pt}%
\definecolor{currentstroke}{rgb}{1.000000,0.894118,0.788235}%
\pgfsetstrokecolor{currentstroke}%
\pgfsetdash{}{0pt}%
\pgfpathmoveto{\pgfqpoint{3.860367in}{2.339514in}}%
\pgfpathlineto{\pgfqpoint{3.860367in}{2.485289in}}%
\pgfpathlineto{\pgfqpoint{10.295205in}{-1.229867in}}%
\pgfpathlineto{\pgfqpoint{10.421450in}{-1.448529in}}%
\pgfpathlineto{\pgfqpoint{3.860367in}{2.339514in}}%
\pgfpathclose%
\pgfusepath{fill}%
\end{pgfscope}%
\begin{pgfscope}%
\pgfpathrectangle{\pgfqpoint{0.765000in}{0.660000in}}{\pgfqpoint{4.620000in}{4.620000in}}%
\pgfusepath{clip}%
\pgfsetbuttcap%
\pgfsetroundjoin%
\definecolor{currentfill}{rgb}{1.000000,0.894118,0.788235}%
\pgfsetfillcolor{currentfill}%
\pgfsetlinewidth{0.000000pt}%
\definecolor{currentstroke}{rgb}{1.000000,0.894118,0.788235}%
\pgfsetstrokecolor{currentstroke}%
\pgfsetdash{}{0pt}%
\pgfpathmoveto{\pgfqpoint{3.986612in}{2.266626in}}%
\pgfpathlineto{\pgfqpoint{3.986612in}{2.412401in}}%
\pgfpathlineto{\pgfqpoint{10.421450in}{-1.302754in}}%
\pgfpathlineto{\pgfqpoint{10.295205in}{-1.375642in}}%
\pgfpathlineto{\pgfqpoint{3.986612in}{2.266626in}}%
\pgfpathclose%
\pgfusepath{fill}%
\end{pgfscope}%
\begin{pgfscope}%
\pgfpathrectangle{\pgfqpoint{0.765000in}{0.660000in}}{\pgfqpoint{4.620000in}{4.620000in}}%
\pgfusepath{clip}%
\pgfsetbuttcap%
\pgfsetroundjoin%
\definecolor{currentfill}{rgb}{1.000000,0.894118,0.788235}%
\pgfsetfillcolor{currentfill}%
\pgfsetlinewidth{0.000000pt}%
\definecolor{currentstroke}{rgb}{1.000000,0.894118,0.788235}%
\pgfsetstrokecolor{currentstroke}%
\pgfsetdash{}{0pt}%
\pgfpathmoveto{\pgfqpoint{3.860367in}{2.193739in}}%
\pgfpathlineto{\pgfqpoint{3.986612in}{2.266626in}}%
\pgfpathlineto{\pgfqpoint{10.421450in}{-1.448529in}}%
\pgfpathlineto{\pgfqpoint{10.295205in}{-1.521417in}}%
\pgfpathlineto{\pgfqpoint{3.860367in}{2.193739in}}%
\pgfpathclose%
\pgfusepath{fill}%
\end{pgfscope}%
\begin{pgfscope}%
\pgfpathrectangle{\pgfqpoint{0.765000in}{0.660000in}}{\pgfqpoint{4.620000in}{4.620000in}}%
\pgfusepath{clip}%
\pgfsetbuttcap%
\pgfsetroundjoin%
\definecolor{currentfill}{rgb}{1.000000,0.894118,0.788235}%
\pgfsetfillcolor{currentfill}%
\pgfsetlinewidth{0.000000pt}%
\definecolor{currentstroke}{rgb}{1.000000,0.894118,0.788235}%
\pgfsetstrokecolor{currentstroke}%
\pgfsetdash{}{0pt}%
\pgfpathmoveto{\pgfqpoint{3.986612in}{2.412401in}}%
\pgfpathlineto{\pgfqpoint{3.860367in}{2.485289in}}%
\pgfpathlineto{\pgfqpoint{10.295205in}{-1.229867in}}%
\pgfpathlineto{\pgfqpoint{10.421450in}{-1.302754in}}%
\pgfpathlineto{\pgfqpoint{3.986612in}{2.412401in}}%
\pgfpathclose%
\pgfusepath{fill}%
\end{pgfscope}%
\begin{pgfscope}%
\pgfpathrectangle{\pgfqpoint{0.765000in}{0.660000in}}{\pgfqpoint{4.620000in}{4.620000in}}%
\pgfusepath{clip}%
\pgfsetbuttcap%
\pgfsetroundjoin%
\definecolor{currentfill}{rgb}{1.000000,0.894118,0.788235}%
\pgfsetfillcolor{currentfill}%
\pgfsetlinewidth{0.000000pt}%
\definecolor{currentstroke}{rgb}{1.000000,0.894118,0.788235}%
\pgfsetstrokecolor{currentstroke}%
\pgfsetdash{}{0pt}%
\pgfpathmoveto{\pgfqpoint{3.734122in}{2.266626in}}%
\pgfpathlineto{\pgfqpoint{3.860367in}{2.193739in}}%
\pgfpathlineto{\pgfqpoint{10.295205in}{-1.521417in}}%
\pgfpathlineto{\pgfqpoint{10.168960in}{-1.448529in}}%
\pgfpathlineto{\pgfqpoint{3.734122in}{2.266626in}}%
\pgfpathclose%
\pgfusepath{fill}%
\end{pgfscope}%
\begin{pgfscope}%
\pgfpathrectangle{\pgfqpoint{0.765000in}{0.660000in}}{\pgfqpoint{4.620000in}{4.620000in}}%
\pgfusepath{clip}%
\pgfsetbuttcap%
\pgfsetroundjoin%
\definecolor{currentfill}{rgb}{1.000000,0.894118,0.788235}%
\pgfsetfillcolor{currentfill}%
\pgfsetlinewidth{0.000000pt}%
\definecolor{currentstroke}{rgb}{1.000000,0.894118,0.788235}%
\pgfsetstrokecolor{currentstroke}%
\pgfsetdash{}{0pt}%
\pgfpathmoveto{\pgfqpoint{3.860367in}{2.485289in}}%
\pgfpathlineto{\pgfqpoint{3.734122in}{2.412401in}}%
\pgfpathlineto{\pgfqpoint{10.168960in}{-1.302754in}}%
\pgfpathlineto{\pgfqpoint{10.295205in}{-1.229867in}}%
\pgfpathlineto{\pgfqpoint{3.860367in}{2.485289in}}%
\pgfpathclose%
\pgfusepath{fill}%
\end{pgfscope}%
\begin{pgfscope}%
\pgfpathrectangle{\pgfqpoint{0.765000in}{0.660000in}}{\pgfqpoint{4.620000in}{4.620000in}}%
\pgfusepath{clip}%
\pgfsetbuttcap%
\pgfsetroundjoin%
\definecolor{currentfill}{rgb}{1.000000,0.894118,0.788235}%
\pgfsetfillcolor{currentfill}%
\pgfsetlinewidth{0.000000pt}%
\definecolor{currentstroke}{rgb}{1.000000,0.894118,0.788235}%
\pgfsetstrokecolor{currentstroke}%
\pgfsetdash{}{0pt}%
\pgfpathmoveto{\pgfqpoint{3.734122in}{2.266626in}}%
\pgfpathlineto{\pgfqpoint{3.734122in}{2.412401in}}%
\pgfpathlineto{\pgfqpoint{10.168960in}{-1.302754in}}%
\pgfpathlineto{\pgfqpoint{10.295205in}{-1.521417in}}%
\pgfpathlineto{\pgfqpoint{3.734122in}{2.266626in}}%
\pgfpathclose%
\pgfusepath{fill}%
\end{pgfscope}%
\begin{pgfscope}%
\pgfpathrectangle{\pgfqpoint{0.765000in}{0.660000in}}{\pgfqpoint{4.620000in}{4.620000in}}%
\pgfusepath{clip}%
\pgfsetbuttcap%
\pgfsetroundjoin%
\definecolor{currentfill}{rgb}{1.000000,0.894118,0.788235}%
\pgfsetfillcolor{currentfill}%
\pgfsetlinewidth{0.000000pt}%
\definecolor{currentstroke}{rgb}{1.000000,0.894118,0.788235}%
\pgfsetstrokecolor{currentstroke}%
\pgfsetdash{}{0pt}%
\pgfpathmoveto{\pgfqpoint{3.860367in}{2.193739in}}%
\pgfpathlineto{\pgfqpoint{3.860367in}{2.339514in}}%
\pgfpathlineto{\pgfqpoint{10.295205in}{-1.375642in}}%
\pgfpathlineto{\pgfqpoint{10.168960in}{-1.448529in}}%
\pgfpathlineto{\pgfqpoint{3.860367in}{2.193739in}}%
\pgfpathclose%
\pgfusepath{fill}%
\end{pgfscope}%
\begin{pgfscope}%
\pgfpathrectangle{\pgfqpoint{0.765000in}{0.660000in}}{\pgfqpoint{4.620000in}{4.620000in}}%
\pgfusepath{clip}%
\pgfsetbuttcap%
\pgfsetroundjoin%
\definecolor{currentfill}{rgb}{1.000000,0.894118,0.788235}%
\pgfsetfillcolor{currentfill}%
\pgfsetlinewidth{0.000000pt}%
\definecolor{currentstroke}{rgb}{1.000000,0.894118,0.788235}%
\pgfsetstrokecolor{currentstroke}%
\pgfsetdash{}{0pt}%
\pgfpathmoveto{\pgfqpoint{3.734122in}{2.412401in}}%
\pgfpathlineto{\pgfqpoint{3.860367in}{2.339514in}}%
\pgfpathlineto{\pgfqpoint{10.295205in}{-1.375642in}}%
\pgfpathlineto{\pgfqpoint{10.168960in}{-1.302754in}}%
\pgfpathlineto{\pgfqpoint{3.734122in}{2.412401in}}%
\pgfpathclose%
\pgfusepath{fill}%
\end{pgfscope}%
\begin{pgfscope}%
\pgfpathrectangle{\pgfqpoint{0.765000in}{0.660000in}}{\pgfqpoint{4.620000in}{4.620000in}}%
\pgfusepath{clip}%
\pgfsetbuttcap%
\pgfsetroundjoin%
\definecolor{currentfill}{rgb}{1.000000,0.894118,0.788235}%
\pgfsetfillcolor{currentfill}%
\pgfsetlinewidth{0.000000pt}%
\definecolor{currentstroke}{rgb}{1.000000,0.894118,0.788235}%
\pgfsetstrokecolor{currentstroke}%
\pgfsetdash{}{0pt}%
\pgfpathmoveto{\pgfqpoint{3.860367in}{2.339514in}}%
\pgfpathlineto{\pgfqpoint{3.986612in}{2.412401in}}%
\pgfpathlineto{\pgfqpoint{10.421450in}{-1.302754in}}%
\pgfpathlineto{\pgfqpoint{10.295205in}{-1.375642in}}%
\pgfpathlineto{\pgfqpoint{3.860367in}{2.339514in}}%
\pgfpathclose%
\pgfusepath{fill}%
\end{pgfscope}%
\begin{pgfscope}%
\pgfpathrectangle{\pgfqpoint{0.765000in}{0.660000in}}{\pgfqpoint{4.620000in}{4.620000in}}%
\pgfusepath{clip}%
\pgfsetbuttcap%
\pgfsetroundjoin%
\definecolor{currentfill}{rgb}{1.000000,0.894118,0.788235}%
\pgfsetfillcolor{currentfill}%
\pgfsetlinewidth{0.000000pt}%
\definecolor{currentstroke}{rgb}{1.000000,0.894118,0.788235}%
\pgfsetstrokecolor{currentstroke}%
\pgfsetdash{}{0pt}%
\pgfpathmoveto{\pgfqpoint{3.860367in}{3.248275in}}%
\pgfpathlineto{\pgfqpoint{3.734122in}{3.175388in}}%
\pgfpathlineto{\pgfqpoint{3.860367in}{3.102500in}}%
\pgfpathlineto{\pgfqpoint{3.986612in}{3.175388in}}%
\pgfpathlineto{\pgfqpoint{3.860367in}{3.248275in}}%
\pgfpathclose%
\pgfusepath{fill}%
\end{pgfscope}%
\begin{pgfscope}%
\pgfpathrectangle{\pgfqpoint{0.765000in}{0.660000in}}{\pgfqpoint{4.620000in}{4.620000in}}%
\pgfusepath{clip}%
\pgfsetbuttcap%
\pgfsetroundjoin%
\definecolor{currentfill}{rgb}{1.000000,0.894118,0.788235}%
\pgfsetfillcolor{currentfill}%
\pgfsetlinewidth{0.000000pt}%
\definecolor{currentstroke}{rgb}{1.000000,0.894118,0.788235}%
\pgfsetstrokecolor{currentstroke}%
\pgfsetdash{}{0pt}%
\pgfpathmoveto{\pgfqpoint{3.860367in}{3.248275in}}%
\pgfpathlineto{\pgfqpoint{3.734122in}{3.175388in}}%
\pgfpathlineto{\pgfqpoint{3.734122in}{3.321163in}}%
\pgfpathlineto{\pgfqpoint{3.860367in}{3.394051in}}%
\pgfpathlineto{\pgfqpoint{3.860367in}{3.248275in}}%
\pgfpathclose%
\pgfusepath{fill}%
\end{pgfscope}%
\begin{pgfscope}%
\pgfpathrectangle{\pgfqpoint{0.765000in}{0.660000in}}{\pgfqpoint{4.620000in}{4.620000in}}%
\pgfusepath{clip}%
\pgfsetbuttcap%
\pgfsetroundjoin%
\definecolor{currentfill}{rgb}{1.000000,0.894118,0.788235}%
\pgfsetfillcolor{currentfill}%
\pgfsetlinewidth{0.000000pt}%
\definecolor{currentstroke}{rgb}{1.000000,0.894118,0.788235}%
\pgfsetstrokecolor{currentstroke}%
\pgfsetdash{}{0pt}%
\pgfpathmoveto{\pgfqpoint{3.860367in}{3.248275in}}%
\pgfpathlineto{\pgfqpoint{3.986612in}{3.175388in}}%
\pgfpathlineto{\pgfqpoint{3.986612in}{3.321163in}}%
\pgfpathlineto{\pgfqpoint{3.860367in}{3.394051in}}%
\pgfpathlineto{\pgfqpoint{3.860367in}{3.248275in}}%
\pgfpathclose%
\pgfusepath{fill}%
\end{pgfscope}%
\begin{pgfscope}%
\pgfpathrectangle{\pgfqpoint{0.765000in}{0.660000in}}{\pgfqpoint{4.620000in}{4.620000in}}%
\pgfusepath{clip}%
\pgfsetbuttcap%
\pgfsetroundjoin%
\definecolor{currentfill}{rgb}{1.000000,0.894118,0.788235}%
\pgfsetfillcolor{currentfill}%
\pgfsetlinewidth{0.000000pt}%
\definecolor{currentstroke}{rgb}{1.000000,0.894118,0.788235}%
\pgfsetstrokecolor{currentstroke}%
\pgfsetdash{}{0pt}%
\pgfpathmoveto{\pgfqpoint{3.860367in}{2.339514in}}%
\pgfpathlineto{\pgfqpoint{3.734122in}{2.266626in}}%
\pgfpathlineto{\pgfqpoint{3.860367in}{2.193739in}}%
\pgfpathlineto{\pgfqpoint{3.986612in}{2.266626in}}%
\pgfpathlineto{\pgfqpoint{3.860367in}{2.339514in}}%
\pgfpathclose%
\pgfusepath{fill}%
\end{pgfscope}%
\begin{pgfscope}%
\pgfpathrectangle{\pgfqpoint{0.765000in}{0.660000in}}{\pgfqpoint{4.620000in}{4.620000in}}%
\pgfusepath{clip}%
\pgfsetbuttcap%
\pgfsetroundjoin%
\definecolor{currentfill}{rgb}{1.000000,0.894118,0.788235}%
\pgfsetfillcolor{currentfill}%
\pgfsetlinewidth{0.000000pt}%
\definecolor{currentstroke}{rgb}{1.000000,0.894118,0.788235}%
\pgfsetstrokecolor{currentstroke}%
\pgfsetdash{}{0pt}%
\pgfpathmoveto{\pgfqpoint{3.860367in}{2.339514in}}%
\pgfpathlineto{\pgfqpoint{3.734122in}{2.266626in}}%
\pgfpathlineto{\pgfqpoint{3.734122in}{2.412401in}}%
\pgfpathlineto{\pgfqpoint{3.860367in}{2.485289in}}%
\pgfpathlineto{\pgfqpoint{3.860367in}{2.339514in}}%
\pgfpathclose%
\pgfusepath{fill}%
\end{pgfscope}%
\begin{pgfscope}%
\pgfpathrectangle{\pgfqpoint{0.765000in}{0.660000in}}{\pgfqpoint{4.620000in}{4.620000in}}%
\pgfusepath{clip}%
\pgfsetbuttcap%
\pgfsetroundjoin%
\definecolor{currentfill}{rgb}{1.000000,0.894118,0.788235}%
\pgfsetfillcolor{currentfill}%
\pgfsetlinewidth{0.000000pt}%
\definecolor{currentstroke}{rgb}{1.000000,0.894118,0.788235}%
\pgfsetstrokecolor{currentstroke}%
\pgfsetdash{}{0pt}%
\pgfpathmoveto{\pgfqpoint{3.860367in}{2.339514in}}%
\pgfpathlineto{\pgfqpoint{3.986612in}{2.266626in}}%
\pgfpathlineto{\pgfqpoint{3.986612in}{2.412401in}}%
\pgfpathlineto{\pgfqpoint{3.860367in}{2.485289in}}%
\pgfpathlineto{\pgfqpoint{3.860367in}{2.339514in}}%
\pgfpathclose%
\pgfusepath{fill}%
\end{pgfscope}%
\begin{pgfscope}%
\pgfpathrectangle{\pgfqpoint{0.765000in}{0.660000in}}{\pgfqpoint{4.620000in}{4.620000in}}%
\pgfusepath{clip}%
\pgfsetbuttcap%
\pgfsetroundjoin%
\definecolor{currentfill}{rgb}{1.000000,0.894118,0.788235}%
\pgfsetfillcolor{currentfill}%
\pgfsetlinewidth{0.000000pt}%
\definecolor{currentstroke}{rgb}{1.000000,0.894118,0.788235}%
\pgfsetstrokecolor{currentstroke}%
\pgfsetdash{}{0pt}%
\pgfpathmoveto{\pgfqpoint{3.860367in}{3.394051in}}%
\pgfpathlineto{\pgfqpoint{3.734122in}{3.321163in}}%
\pgfpathlineto{\pgfqpoint{3.860367in}{3.248275in}}%
\pgfpathlineto{\pgfqpoint{3.986612in}{3.321163in}}%
\pgfpathlineto{\pgfqpoint{3.860367in}{3.394051in}}%
\pgfpathclose%
\pgfusepath{fill}%
\end{pgfscope}%
\begin{pgfscope}%
\pgfpathrectangle{\pgfqpoint{0.765000in}{0.660000in}}{\pgfqpoint{4.620000in}{4.620000in}}%
\pgfusepath{clip}%
\pgfsetbuttcap%
\pgfsetroundjoin%
\definecolor{currentfill}{rgb}{1.000000,0.894118,0.788235}%
\pgfsetfillcolor{currentfill}%
\pgfsetlinewidth{0.000000pt}%
\definecolor{currentstroke}{rgb}{1.000000,0.894118,0.788235}%
\pgfsetstrokecolor{currentstroke}%
\pgfsetdash{}{0pt}%
\pgfpathmoveto{\pgfqpoint{3.860367in}{3.102500in}}%
\pgfpathlineto{\pgfqpoint{3.986612in}{3.175388in}}%
\pgfpathlineto{\pgfqpoint{3.986612in}{3.321163in}}%
\pgfpathlineto{\pgfqpoint{3.860367in}{3.248275in}}%
\pgfpathlineto{\pgfqpoint{3.860367in}{3.102500in}}%
\pgfpathclose%
\pgfusepath{fill}%
\end{pgfscope}%
\begin{pgfscope}%
\pgfpathrectangle{\pgfqpoint{0.765000in}{0.660000in}}{\pgfqpoint{4.620000in}{4.620000in}}%
\pgfusepath{clip}%
\pgfsetbuttcap%
\pgfsetroundjoin%
\definecolor{currentfill}{rgb}{1.000000,0.894118,0.788235}%
\pgfsetfillcolor{currentfill}%
\pgfsetlinewidth{0.000000pt}%
\definecolor{currentstroke}{rgb}{1.000000,0.894118,0.788235}%
\pgfsetstrokecolor{currentstroke}%
\pgfsetdash{}{0pt}%
\pgfpathmoveto{\pgfqpoint{3.734122in}{3.175388in}}%
\pgfpathlineto{\pgfqpoint{3.860367in}{3.102500in}}%
\pgfpathlineto{\pgfqpoint{3.860367in}{3.248275in}}%
\pgfpathlineto{\pgfqpoint{3.734122in}{3.321163in}}%
\pgfpathlineto{\pgfqpoint{3.734122in}{3.175388in}}%
\pgfpathclose%
\pgfusepath{fill}%
\end{pgfscope}%
\begin{pgfscope}%
\pgfpathrectangle{\pgfqpoint{0.765000in}{0.660000in}}{\pgfqpoint{4.620000in}{4.620000in}}%
\pgfusepath{clip}%
\pgfsetbuttcap%
\pgfsetroundjoin%
\definecolor{currentfill}{rgb}{1.000000,0.894118,0.788235}%
\pgfsetfillcolor{currentfill}%
\pgfsetlinewidth{0.000000pt}%
\definecolor{currentstroke}{rgb}{1.000000,0.894118,0.788235}%
\pgfsetstrokecolor{currentstroke}%
\pgfsetdash{}{0pt}%
\pgfpathmoveto{\pgfqpoint{3.860367in}{2.485289in}}%
\pgfpathlineto{\pgfqpoint{3.734122in}{2.412401in}}%
\pgfpathlineto{\pgfqpoint{3.860367in}{2.339514in}}%
\pgfpathlineto{\pgfqpoint{3.986612in}{2.412401in}}%
\pgfpathlineto{\pgfqpoint{3.860367in}{2.485289in}}%
\pgfpathclose%
\pgfusepath{fill}%
\end{pgfscope}%
\begin{pgfscope}%
\pgfpathrectangle{\pgfqpoint{0.765000in}{0.660000in}}{\pgfqpoint{4.620000in}{4.620000in}}%
\pgfusepath{clip}%
\pgfsetbuttcap%
\pgfsetroundjoin%
\definecolor{currentfill}{rgb}{1.000000,0.894118,0.788235}%
\pgfsetfillcolor{currentfill}%
\pgfsetlinewidth{0.000000pt}%
\definecolor{currentstroke}{rgb}{1.000000,0.894118,0.788235}%
\pgfsetstrokecolor{currentstroke}%
\pgfsetdash{}{0pt}%
\pgfpathmoveto{\pgfqpoint{3.860367in}{2.193739in}}%
\pgfpathlineto{\pgfqpoint{3.986612in}{2.266626in}}%
\pgfpathlineto{\pgfqpoint{3.986612in}{2.412401in}}%
\pgfpathlineto{\pgfqpoint{3.860367in}{2.339514in}}%
\pgfpathlineto{\pgfqpoint{3.860367in}{2.193739in}}%
\pgfpathclose%
\pgfusepath{fill}%
\end{pgfscope}%
\begin{pgfscope}%
\pgfpathrectangle{\pgfqpoint{0.765000in}{0.660000in}}{\pgfqpoint{4.620000in}{4.620000in}}%
\pgfusepath{clip}%
\pgfsetbuttcap%
\pgfsetroundjoin%
\definecolor{currentfill}{rgb}{1.000000,0.894118,0.788235}%
\pgfsetfillcolor{currentfill}%
\pgfsetlinewidth{0.000000pt}%
\definecolor{currentstroke}{rgb}{1.000000,0.894118,0.788235}%
\pgfsetstrokecolor{currentstroke}%
\pgfsetdash{}{0pt}%
\pgfpathmoveto{\pgfqpoint{3.734122in}{2.266626in}}%
\pgfpathlineto{\pgfqpoint{3.860367in}{2.193739in}}%
\pgfpathlineto{\pgfqpoint{3.860367in}{2.339514in}}%
\pgfpathlineto{\pgfqpoint{3.734122in}{2.412401in}}%
\pgfpathlineto{\pgfqpoint{3.734122in}{2.266626in}}%
\pgfpathclose%
\pgfusepath{fill}%
\end{pgfscope}%
\begin{pgfscope}%
\pgfpathrectangle{\pgfqpoint{0.765000in}{0.660000in}}{\pgfqpoint{4.620000in}{4.620000in}}%
\pgfusepath{clip}%
\pgfsetbuttcap%
\pgfsetroundjoin%
\definecolor{currentfill}{rgb}{1.000000,0.894118,0.788235}%
\pgfsetfillcolor{currentfill}%
\pgfsetlinewidth{0.000000pt}%
\definecolor{currentstroke}{rgb}{1.000000,0.894118,0.788235}%
\pgfsetstrokecolor{currentstroke}%
\pgfsetdash{}{0pt}%
\pgfpathmoveto{\pgfqpoint{3.860367in}{3.248275in}}%
\pgfpathlineto{\pgfqpoint{3.734122in}{3.175388in}}%
\pgfpathlineto{\pgfqpoint{3.734122in}{2.266626in}}%
\pgfpathlineto{\pgfqpoint{3.860367in}{2.339514in}}%
\pgfpathlineto{\pgfqpoint{3.860367in}{3.248275in}}%
\pgfpathclose%
\pgfusepath{fill}%
\end{pgfscope}%
\begin{pgfscope}%
\pgfpathrectangle{\pgfqpoint{0.765000in}{0.660000in}}{\pgfqpoint{4.620000in}{4.620000in}}%
\pgfusepath{clip}%
\pgfsetbuttcap%
\pgfsetroundjoin%
\definecolor{currentfill}{rgb}{1.000000,0.894118,0.788235}%
\pgfsetfillcolor{currentfill}%
\pgfsetlinewidth{0.000000pt}%
\definecolor{currentstroke}{rgb}{1.000000,0.894118,0.788235}%
\pgfsetstrokecolor{currentstroke}%
\pgfsetdash{}{0pt}%
\pgfpathmoveto{\pgfqpoint{3.986612in}{3.175388in}}%
\pgfpathlineto{\pgfqpoint{3.860367in}{3.248275in}}%
\pgfpathlineto{\pgfqpoint{3.860367in}{2.339514in}}%
\pgfpathlineto{\pgfqpoint{3.986612in}{2.266626in}}%
\pgfpathlineto{\pgfqpoint{3.986612in}{3.175388in}}%
\pgfpathclose%
\pgfusepath{fill}%
\end{pgfscope}%
\begin{pgfscope}%
\pgfpathrectangle{\pgfqpoint{0.765000in}{0.660000in}}{\pgfqpoint{4.620000in}{4.620000in}}%
\pgfusepath{clip}%
\pgfsetbuttcap%
\pgfsetroundjoin%
\definecolor{currentfill}{rgb}{1.000000,0.894118,0.788235}%
\pgfsetfillcolor{currentfill}%
\pgfsetlinewidth{0.000000pt}%
\definecolor{currentstroke}{rgb}{1.000000,0.894118,0.788235}%
\pgfsetstrokecolor{currentstroke}%
\pgfsetdash{}{0pt}%
\pgfpathmoveto{\pgfqpoint{3.860367in}{3.248275in}}%
\pgfpathlineto{\pgfqpoint{3.860367in}{3.394051in}}%
\pgfpathlineto{\pgfqpoint{3.860367in}{2.485289in}}%
\pgfpathlineto{\pgfqpoint{3.986612in}{2.266626in}}%
\pgfpathlineto{\pgfqpoint{3.860367in}{3.248275in}}%
\pgfpathclose%
\pgfusepath{fill}%
\end{pgfscope}%
\begin{pgfscope}%
\pgfpathrectangle{\pgfqpoint{0.765000in}{0.660000in}}{\pgfqpoint{4.620000in}{4.620000in}}%
\pgfusepath{clip}%
\pgfsetbuttcap%
\pgfsetroundjoin%
\definecolor{currentfill}{rgb}{1.000000,0.894118,0.788235}%
\pgfsetfillcolor{currentfill}%
\pgfsetlinewidth{0.000000pt}%
\definecolor{currentstroke}{rgb}{1.000000,0.894118,0.788235}%
\pgfsetstrokecolor{currentstroke}%
\pgfsetdash{}{0pt}%
\pgfpathmoveto{\pgfqpoint{3.986612in}{3.175388in}}%
\pgfpathlineto{\pgfqpoint{3.986612in}{3.321163in}}%
\pgfpathlineto{\pgfqpoint{3.986612in}{2.412401in}}%
\pgfpathlineto{\pgfqpoint{3.860367in}{2.339514in}}%
\pgfpathlineto{\pgfqpoint{3.986612in}{3.175388in}}%
\pgfpathclose%
\pgfusepath{fill}%
\end{pgfscope}%
\begin{pgfscope}%
\pgfpathrectangle{\pgfqpoint{0.765000in}{0.660000in}}{\pgfqpoint{4.620000in}{4.620000in}}%
\pgfusepath{clip}%
\pgfsetbuttcap%
\pgfsetroundjoin%
\definecolor{currentfill}{rgb}{1.000000,0.894118,0.788235}%
\pgfsetfillcolor{currentfill}%
\pgfsetlinewidth{0.000000pt}%
\definecolor{currentstroke}{rgb}{1.000000,0.894118,0.788235}%
\pgfsetstrokecolor{currentstroke}%
\pgfsetdash{}{0pt}%
\pgfpathmoveto{\pgfqpoint{3.860367in}{3.102500in}}%
\pgfpathlineto{\pgfqpoint{3.986612in}{3.175388in}}%
\pgfpathlineto{\pgfqpoint{3.986612in}{2.266626in}}%
\pgfpathlineto{\pgfqpoint{3.860367in}{2.193739in}}%
\pgfpathlineto{\pgfqpoint{3.860367in}{3.102500in}}%
\pgfpathclose%
\pgfusepath{fill}%
\end{pgfscope}%
\begin{pgfscope}%
\pgfpathrectangle{\pgfqpoint{0.765000in}{0.660000in}}{\pgfqpoint{4.620000in}{4.620000in}}%
\pgfusepath{clip}%
\pgfsetbuttcap%
\pgfsetroundjoin%
\definecolor{currentfill}{rgb}{1.000000,0.894118,0.788235}%
\pgfsetfillcolor{currentfill}%
\pgfsetlinewidth{0.000000pt}%
\definecolor{currentstroke}{rgb}{1.000000,0.894118,0.788235}%
\pgfsetstrokecolor{currentstroke}%
\pgfsetdash{}{0pt}%
\pgfpathmoveto{\pgfqpoint{3.986612in}{3.321163in}}%
\pgfpathlineto{\pgfqpoint{3.860367in}{3.394051in}}%
\pgfpathlineto{\pgfqpoint{3.860367in}{2.485289in}}%
\pgfpathlineto{\pgfqpoint{3.986612in}{2.412401in}}%
\pgfpathlineto{\pgfqpoint{3.986612in}{3.321163in}}%
\pgfpathclose%
\pgfusepath{fill}%
\end{pgfscope}%
\begin{pgfscope}%
\pgfpathrectangle{\pgfqpoint{0.765000in}{0.660000in}}{\pgfqpoint{4.620000in}{4.620000in}}%
\pgfusepath{clip}%
\pgfsetbuttcap%
\pgfsetroundjoin%
\definecolor{currentfill}{rgb}{1.000000,0.894118,0.788235}%
\pgfsetfillcolor{currentfill}%
\pgfsetlinewidth{0.000000pt}%
\definecolor{currentstroke}{rgb}{1.000000,0.894118,0.788235}%
\pgfsetstrokecolor{currentstroke}%
\pgfsetdash{}{0pt}%
\pgfpathmoveto{\pgfqpoint{3.734122in}{3.175388in}}%
\pgfpathlineto{\pgfqpoint{3.860367in}{3.102500in}}%
\pgfpathlineto{\pgfqpoint{3.860367in}{2.193739in}}%
\pgfpathlineto{\pgfqpoint{3.734122in}{2.266626in}}%
\pgfpathlineto{\pgfqpoint{3.734122in}{3.175388in}}%
\pgfpathclose%
\pgfusepath{fill}%
\end{pgfscope}%
\begin{pgfscope}%
\pgfpathrectangle{\pgfqpoint{0.765000in}{0.660000in}}{\pgfqpoint{4.620000in}{4.620000in}}%
\pgfusepath{clip}%
\pgfsetbuttcap%
\pgfsetroundjoin%
\definecolor{currentfill}{rgb}{1.000000,0.894118,0.788235}%
\pgfsetfillcolor{currentfill}%
\pgfsetlinewidth{0.000000pt}%
\definecolor{currentstroke}{rgb}{1.000000,0.894118,0.788235}%
\pgfsetstrokecolor{currentstroke}%
\pgfsetdash{}{0pt}%
\pgfpathmoveto{\pgfqpoint{3.860367in}{3.394051in}}%
\pgfpathlineto{\pgfqpoint{3.734122in}{3.321163in}}%
\pgfpathlineto{\pgfqpoint{3.734122in}{2.412401in}}%
\pgfpathlineto{\pgfqpoint{3.860367in}{2.485289in}}%
\pgfpathlineto{\pgfqpoint{3.860367in}{3.394051in}}%
\pgfpathclose%
\pgfusepath{fill}%
\end{pgfscope}%
\begin{pgfscope}%
\pgfpathrectangle{\pgfqpoint{0.765000in}{0.660000in}}{\pgfqpoint{4.620000in}{4.620000in}}%
\pgfusepath{clip}%
\pgfsetbuttcap%
\pgfsetroundjoin%
\definecolor{currentfill}{rgb}{1.000000,0.894118,0.788235}%
\pgfsetfillcolor{currentfill}%
\pgfsetlinewidth{0.000000pt}%
\definecolor{currentstroke}{rgb}{1.000000,0.894118,0.788235}%
\pgfsetstrokecolor{currentstroke}%
\pgfsetdash{}{0pt}%
\pgfpathmoveto{\pgfqpoint{3.734122in}{3.175388in}}%
\pgfpathlineto{\pgfqpoint{3.734122in}{3.321163in}}%
\pgfpathlineto{\pgfqpoint{3.734122in}{2.412401in}}%
\pgfpathlineto{\pgfqpoint{3.860367in}{2.193739in}}%
\pgfpathlineto{\pgfqpoint{3.734122in}{3.175388in}}%
\pgfpathclose%
\pgfusepath{fill}%
\end{pgfscope}%
\begin{pgfscope}%
\pgfpathrectangle{\pgfqpoint{0.765000in}{0.660000in}}{\pgfqpoint{4.620000in}{4.620000in}}%
\pgfusepath{clip}%
\pgfsetbuttcap%
\pgfsetroundjoin%
\definecolor{currentfill}{rgb}{1.000000,0.894118,0.788235}%
\pgfsetfillcolor{currentfill}%
\pgfsetlinewidth{0.000000pt}%
\definecolor{currentstroke}{rgb}{1.000000,0.894118,0.788235}%
\pgfsetstrokecolor{currentstroke}%
\pgfsetdash{}{0pt}%
\pgfpathmoveto{\pgfqpoint{3.860367in}{3.102500in}}%
\pgfpathlineto{\pgfqpoint{3.860367in}{3.248275in}}%
\pgfpathlineto{\pgfqpoint{3.860367in}{2.339514in}}%
\pgfpathlineto{\pgfqpoint{3.734122in}{2.266626in}}%
\pgfpathlineto{\pgfqpoint{3.860367in}{3.102500in}}%
\pgfpathclose%
\pgfusepath{fill}%
\end{pgfscope}%
\begin{pgfscope}%
\pgfpathrectangle{\pgfqpoint{0.765000in}{0.660000in}}{\pgfqpoint{4.620000in}{4.620000in}}%
\pgfusepath{clip}%
\pgfsetbuttcap%
\pgfsetroundjoin%
\definecolor{currentfill}{rgb}{1.000000,0.894118,0.788235}%
\pgfsetfillcolor{currentfill}%
\pgfsetlinewidth{0.000000pt}%
\definecolor{currentstroke}{rgb}{1.000000,0.894118,0.788235}%
\pgfsetstrokecolor{currentstroke}%
\pgfsetdash{}{0pt}%
\pgfpathmoveto{\pgfqpoint{3.734122in}{3.321163in}}%
\pgfpathlineto{\pgfqpoint{3.860367in}{3.248275in}}%
\pgfpathlineto{\pgfqpoint{3.860367in}{2.339514in}}%
\pgfpathlineto{\pgfqpoint{3.734122in}{2.412401in}}%
\pgfpathlineto{\pgfqpoint{3.734122in}{3.321163in}}%
\pgfpathclose%
\pgfusepath{fill}%
\end{pgfscope}%
\begin{pgfscope}%
\pgfpathrectangle{\pgfqpoint{0.765000in}{0.660000in}}{\pgfqpoint{4.620000in}{4.620000in}}%
\pgfusepath{clip}%
\pgfsetbuttcap%
\pgfsetroundjoin%
\definecolor{currentfill}{rgb}{1.000000,0.894118,0.788235}%
\pgfsetfillcolor{currentfill}%
\pgfsetlinewidth{0.000000pt}%
\definecolor{currentstroke}{rgb}{1.000000,0.894118,0.788235}%
\pgfsetstrokecolor{currentstroke}%
\pgfsetdash{}{0pt}%
\pgfpathmoveto{\pgfqpoint{3.860367in}{3.248275in}}%
\pgfpathlineto{\pgfqpoint{3.986612in}{3.321163in}}%
\pgfpathlineto{\pgfqpoint{3.986612in}{2.412401in}}%
\pgfpathlineto{\pgfqpoint{3.860367in}{2.339514in}}%
\pgfpathlineto{\pgfqpoint{3.860367in}{3.248275in}}%
\pgfpathclose%
\pgfusepath{fill}%
\end{pgfscope}%
\begin{pgfscope}%
\pgfpathrectangle{\pgfqpoint{0.765000in}{0.660000in}}{\pgfqpoint{4.620000in}{4.620000in}}%
\pgfusepath{clip}%
\pgfsetbuttcap%
\pgfsetroundjoin%
\definecolor{currentfill}{rgb}{1.000000,0.894118,0.788235}%
\pgfsetfillcolor{currentfill}%
\pgfsetlinewidth{0.000000pt}%
\definecolor{currentstroke}{rgb}{1.000000,0.894118,0.788235}%
\pgfsetstrokecolor{currentstroke}%
\pgfsetdash{}{0pt}%
\pgfpathmoveto{\pgfqpoint{3.860367in}{3.248275in}}%
\pgfpathlineto{\pgfqpoint{3.734122in}{3.175388in}}%
\pgfpathlineto{\pgfqpoint{3.860367in}{3.102500in}}%
\pgfpathlineto{\pgfqpoint{3.986612in}{3.175388in}}%
\pgfpathlineto{\pgfqpoint{3.860367in}{3.248275in}}%
\pgfpathclose%
\pgfusepath{fill}%
\end{pgfscope}%
\begin{pgfscope}%
\pgfpathrectangle{\pgfqpoint{0.765000in}{0.660000in}}{\pgfqpoint{4.620000in}{4.620000in}}%
\pgfusepath{clip}%
\pgfsetbuttcap%
\pgfsetroundjoin%
\definecolor{currentfill}{rgb}{1.000000,0.894118,0.788235}%
\pgfsetfillcolor{currentfill}%
\pgfsetlinewidth{0.000000pt}%
\definecolor{currentstroke}{rgb}{1.000000,0.894118,0.788235}%
\pgfsetstrokecolor{currentstroke}%
\pgfsetdash{}{0pt}%
\pgfpathmoveto{\pgfqpoint{3.860367in}{3.248275in}}%
\pgfpathlineto{\pgfqpoint{3.734122in}{3.175388in}}%
\pgfpathlineto{\pgfqpoint{3.734122in}{3.321163in}}%
\pgfpathlineto{\pgfqpoint{3.860367in}{3.394051in}}%
\pgfpathlineto{\pgfqpoint{3.860367in}{3.248275in}}%
\pgfpathclose%
\pgfusepath{fill}%
\end{pgfscope}%
\begin{pgfscope}%
\pgfpathrectangle{\pgfqpoint{0.765000in}{0.660000in}}{\pgfqpoint{4.620000in}{4.620000in}}%
\pgfusepath{clip}%
\pgfsetbuttcap%
\pgfsetroundjoin%
\definecolor{currentfill}{rgb}{1.000000,0.894118,0.788235}%
\pgfsetfillcolor{currentfill}%
\pgfsetlinewidth{0.000000pt}%
\definecolor{currentstroke}{rgb}{1.000000,0.894118,0.788235}%
\pgfsetstrokecolor{currentstroke}%
\pgfsetdash{}{0pt}%
\pgfpathmoveto{\pgfqpoint{3.860367in}{3.248275in}}%
\pgfpathlineto{\pgfqpoint{3.986612in}{3.175388in}}%
\pgfpathlineto{\pgfqpoint{3.986612in}{3.321163in}}%
\pgfpathlineto{\pgfqpoint{3.860367in}{3.394051in}}%
\pgfpathlineto{\pgfqpoint{3.860367in}{3.248275in}}%
\pgfpathclose%
\pgfusepath{fill}%
\end{pgfscope}%
\begin{pgfscope}%
\pgfpathrectangle{\pgfqpoint{0.765000in}{0.660000in}}{\pgfqpoint{4.620000in}{4.620000in}}%
\pgfusepath{clip}%
\pgfsetbuttcap%
\pgfsetroundjoin%
\definecolor{currentfill}{rgb}{1.000000,0.894118,0.788235}%
\pgfsetfillcolor{currentfill}%
\pgfsetlinewidth{0.000000pt}%
\definecolor{currentstroke}{rgb}{1.000000,0.894118,0.788235}%
\pgfsetstrokecolor{currentstroke}%
\pgfsetdash{}{0pt}%
\pgfpathmoveto{\pgfqpoint{3.113634in}{3.679402in}}%
\pgfpathlineto{\pgfqpoint{2.987389in}{3.606514in}}%
\pgfpathlineto{\pgfqpoint{3.113634in}{3.533627in}}%
\pgfpathlineto{\pgfqpoint{3.239879in}{3.606514in}}%
\pgfpathlineto{\pgfqpoint{3.113634in}{3.679402in}}%
\pgfpathclose%
\pgfusepath{fill}%
\end{pgfscope}%
\begin{pgfscope}%
\pgfpathrectangle{\pgfqpoint{0.765000in}{0.660000in}}{\pgfqpoint{4.620000in}{4.620000in}}%
\pgfusepath{clip}%
\pgfsetbuttcap%
\pgfsetroundjoin%
\definecolor{currentfill}{rgb}{1.000000,0.894118,0.788235}%
\pgfsetfillcolor{currentfill}%
\pgfsetlinewidth{0.000000pt}%
\definecolor{currentstroke}{rgb}{1.000000,0.894118,0.788235}%
\pgfsetstrokecolor{currentstroke}%
\pgfsetdash{}{0pt}%
\pgfpathmoveto{\pgfqpoint{3.113634in}{3.679402in}}%
\pgfpathlineto{\pgfqpoint{2.987389in}{3.606514in}}%
\pgfpathlineto{\pgfqpoint{2.987389in}{3.752289in}}%
\pgfpathlineto{\pgfqpoint{3.113634in}{3.825177in}}%
\pgfpathlineto{\pgfqpoint{3.113634in}{3.679402in}}%
\pgfpathclose%
\pgfusepath{fill}%
\end{pgfscope}%
\begin{pgfscope}%
\pgfpathrectangle{\pgfqpoint{0.765000in}{0.660000in}}{\pgfqpoint{4.620000in}{4.620000in}}%
\pgfusepath{clip}%
\pgfsetbuttcap%
\pgfsetroundjoin%
\definecolor{currentfill}{rgb}{1.000000,0.894118,0.788235}%
\pgfsetfillcolor{currentfill}%
\pgfsetlinewidth{0.000000pt}%
\definecolor{currentstroke}{rgb}{1.000000,0.894118,0.788235}%
\pgfsetstrokecolor{currentstroke}%
\pgfsetdash{}{0pt}%
\pgfpathmoveto{\pgfqpoint{3.113634in}{3.679402in}}%
\pgfpathlineto{\pgfqpoint{3.239879in}{3.606514in}}%
\pgfpathlineto{\pgfqpoint{3.239879in}{3.752289in}}%
\pgfpathlineto{\pgfqpoint{3.113634in}{3.825177in}}%
\pgfpathlineto{\pgfqpoint{3.113634in}{3.679402in}}%
\pgfpathclose%
\pgfusepath{fill}%
\end{pgfscope}%
\begin{pgfscope}%
\pgfpathrectangle{\pgfqpoint{0.765000in}{0.660000in}}{\pgfqpoint{4.620000in}{4.620000in}}%
\pgfusepath{clip}%
\pgfsetbuttcap%
\pgfsetroundjoin%
\definecolor{currentfill}{rgb}{1.000000,0.894118,0.788235}%
\pgfsetfillcolor{currentfill}%
\pgfsetlinewidth{0.000000pt}%
\definecolor{currentstroke}{rgb}{1.000000,0.894118,0.788235}%
\pgfsetstrokecolor{currentstroke}%
\pgfsetdash{}{0pt}%
\pgfpathmoveto{\pgfqpoint{3.860367in}{3.394051in}}%
\pgfpathlineto{\pgfqpoint{3.734122in}{3.321163in}}%
\pgfpathlineto{\pgfqpoint{3.860367in}{3.248275in}}%
\pgfpathlineto{\pgfqpoint{3.986612in}{3.321163in}}%
\pgfpathlineto{\pgfqpoint{3.860367in}{3.394051in}}%
\pgfpathclose%
\pgfusepath{fill}%
\end{pgfscope}%
\begin{pgfscope}%
\pgfpathrectangle{\pgfqpoint{0.765000in}{0.660000in}}{\pgfqpoint{4.620000in}{4.620000in}}%
\pgfusepath{clip}%
\pgfsetbuttcap%
\pgfsetroundjoin%
\definecolor{currentfill}{rgb}{1.000000,0.894118,0.788235}%
\pgfsetfillcolor{currentfill}%
\pgfsetlinewidth{0.000000pt}%
\definecolor{currentstroke}{rgb}{1.000000,0.894118,0.788235}%
\pgfsetstrokecolor{currentstroke}%
\pgfsetdash{}{0pt}%
\pgfpathmoveto{\pgfqpoint{3.860367in}{3.102500in}}%
\pgfpathlineto{\pgfqpoint{3.986612in}{3.175388in}}%
\pgfpathlineto{\pgfqpoint{3.986612in}{3.321163in}}%
\pgfpathlineto{\pgfqpoint{3.860367in}{3.248275in}}%
\pgfpathlineto{\pgfqpoint{3.860367in}{3.102500in}}%
\pgfpathclose%
\pgfusepath{fill}%
\end{pgfscope}%
\begin{pgfscope}%
\pgfpathrectangle{\pgfqpoint{0.765000in}{0.660000in}}{\pgfqpoint{4.620000in}{4.620000in}}%
\pgfusepath{clip}%
\pgfsetbuttcap%
\pgfsetroundjoin%
\definecolor{currentfill}{rgb}{1.000000,0.894118,0.788235}%
\pgfsetfillcolor{currentfill}%
\pgfsetlinewidth{0.000000pt}%
\definecolor{currentstroke}{rgb}{1.000000,0.894118,0.788235}%
\pgfsetstrokecolor{currentstroke}%
\pgfsetdash{}{0pt}%
\pgfpathmoveto{\pgfqpoint{3.734122in}{3.175388in}}%
\pgfpathlineto{\pgfqpoint{3.860367in}{3.102500in}}%
\pgfpathlineto{\pgfqpoint{3.860367in}{3.248275in}}%
\pgfpathlineto{\pgfqpoint{3.734122in}{3.321163in}}%
\pgfpathlineto{\pgfqpoint{3.734122in}{3.175388in}}%
\pgfpathclose%
\pgfusepath{fill}%
\end{pgfscope}%
\begin{pgfscope}%
\pgfpathrectangle{\pgfqpoint{0.765000in}{0.660000in}}{\pgfqpoint{4.620000in}{4.620000in}}%
\pgfusepath{clip}%
\pgfsetbuttcap%
\pgfsetroundjoin%
\definecolor{currentfill}{rgb}{1.000000,0.894118,0.788235}%
\pgfsetfillcolor{currentfill}%
\pgfsetlinewidth{0.000000pt}%
\definecolor{currentstroke}{rgb}{1.000000,0.894118,0.788235}%
\pgfsetstrokecolor{currentstroke}%
\pgfsetdash{}{0pt}%
\pgfpathmoveto{\pgfqpoint{3.113634in}{3.825177in}}%
\pgfpathlineto{\pgfqpoint{2.987389in}{3.752289in}}%
\pgfpathlineto{\pgfqpoint{3.113634in}{3.679402in}}%
\pgfpathlineto{\pgfqpoint{3.239879in}{3.752289in}}%
\pgfpathlineto{\pgfqpoint{3.113634in}{3.825177in}}%
\pgfpathclose%
\pgfusepath{fill}%
\end{pgfscope}%
\begin{pgfscope}%
\pgfpathrectangle{\pgfqpoint{0.765000in}{0.660000in}}{\pgfqpoint{4.620000in}{4.620000in}}%
\pgfusepath{clip}%
\pgfsetbuttcap%
\pgfsetroundjoin%
\definecolor{currentfill}{rgb}{1.000000,0.894118,0.788235}%
\pgfsetfillcolor{currentfill}%
\pgfsetlinewidth{0.000000pt}%
\definecolor{currentstroke}{rgb}{1.000000,0.894118,0.788235}%
\pgfsetstrokecolor{currentstroke}%
\pgfsetdash{}{0pt}%
\pgfpathmoveto{\pgfqpoint{3.113634in}{3.533627in}}%
\pgfpathlineto{\pgfqpoint{3.239879in}{3.606514in}}%
\pgfpathlineto{\pgfqpoint{3.239879in}{3.752289in}}%
\pgfpathlineto{\pgfqpoint{3.113634in}{3.679402in}}%
\pgfpathlineto{\pgfqpoint{3.113634in}{3.533627in}}%
\pgfpathclose%
\pgfusepath{fill}%
\end{pgfscope}%
\begin{pgfscope}%
\pgfpathrectangle{\pgfqpoint{0.765000in}{0.660000in}}{\pgfqpoint{4.620000in}{4.620000in}}%
\pgfusepath{clip}%
\pgfsetbuttcap%
\pgfsetroundjoin%
\definecolor{currentfill}{rgb}{1.000000,0.894118,0.788235}%
\pgfsetfillcolor{currentfill}%
\pgfsetlinewidth{0.000000pt}%
\definecolor{currentstroke}{rgb}{1.000000,0.894118,0.788235}%
\pgfsetstrokecolor{currentstroke}%
\pgfsetdash{}{0pt}%
\pgfpathmoveto{\pgfqpoint{2.987389in}{3.606514in}}%
\pgfpathlineto{\pgfqpoint{3.113634in}{3.533627in}}%
\pgfpathlineto{\pgfqpoint{3.113634in}{3.679402in}}%
\pgfpathlineto{\pgfqpoint{2.987389in}{3.752289in}}%
\pgfpathlineto{\pgfqpoint{2.987389in}{3.606514in}}%
\pgfpathclose%
\pgfusepath{fill}%
\end{pgfscope}%
\begin{pgfscope}%
\pgfpathrectangle{\pgfqpoint{0.765000in}{0.660000in}}{\pgfqpoint{4.620000in}{4.620000in}}%
\pgfusepath{clip}%
\pgfsetbuttcap%
\pgfsetroundjoin%
\definecolor{currentfill}{rgb}{1.000000,0.894118,0.788235}%
\pgfsetfillcolor{currentfill}%
\pgfsetlinewidth{0.000000pt}%
\definecolor{currentstroke}{rgb}{1.000000,0.894118,0.788235}%
\pgfsetstrokecolor{currentstroke}%
\pgfsetdash{}{0pt}%
\pgfpathmoveto{\pgfqpoint{3.860367in}{3.248275in}}%
\pgfpathlineto{\pgfqpoint{3.734122in}{3.175388in}}%
\pgfpathlineto{\pgfqpoint{2.987389in}{3.606514in}}%
\pgfpathlineto{\pgfqpoint{3.113634in}{3.679402in}}%
\pgfpathlineto{\pgfqpoint{3.860367in}{3.248275in}}%
\pgfpathclose%
\pgfusepath{fill}%
\end{pgfscope}%
\begin{pgfscope}%
\pgfpathrectangle{\pgfqpoint{0.765000in}{0.660000in}}{\pgfqpoint{4.620000in}{4.620000in}}%
\pgfusepath{clip}%
\pgfsetbuttcap%
\pgfsetroundjoin%
\definecolor{currentfill}{rgb}{1.000000,0.894118,0.788235}%
\pgfsetfillcolor{currentfill}%
\pgfsetlinewidth{0.000000pt}%
\definecolor{currentstroke}{rgb}{1.000000,0.894118,0.788235}%
\pgfsetstrokecolor{currentstroke}%
\pgfsetdash{}{0pt}%
\pgfpathmoveto{\pgfqpoint{3.986612in}{3.175388in}}%
\pgfpathlineto{\pgfqpoint{3.860367in}{3.248275in}}%
\pgfpathlineto{\pgfqpoint{3.113634in}{3.679402in}}%
\pgfpathlineto{\pgfqpoint{3.239879in}{3.606514in}}%
\pgfpathlineto{\pgfqpoint{3.986612in}{3.175388in}}%
\pgfpathclose%
\pgfusepath{fill}%
\end{pgfscope}%
\begin{pgfscope}%
\pgfpathrectangle{\pgfqpoint{0.765000in}{0.660000in}}{\pgfqpoint{4.620000in}{4.620000in}}%
\pgfusepath{clip}%
\pgfsetbuttcap%
\pgfsetroundjoin%
\definecolor{currentfill}{rgb}{1.000000,0.894118,0.788235}%
\pgfsetfillcolor{currentfill}%
\pgfsetlinewidth{0.000000pt}%
\definecolor{currentstroke}{rgb}{1.000000,0.894118,0.788235}%
\pgfsetstrokecolor{currentstroke}%
\pgfsetdash{}{0pt}%
\pgfpathmoveto{\pgfqpoint{3.860367in}{3.248275in}}%
\pgfpathlineto{\pgfqpoint{3.860367in}{3.394051in}}%
\pgfpathlineto{\pgfqpoint{3.113634in}{3.825177in}}%
\pgfpathlineto{\pgfqpoint{3.239879in}{3.606514in}}%
\pgfpathlineto{\pgfqpoint{3.860367in}{3.248275in}}%
\pgfpathclose%
\pgfusepath{fill}%
\end{pgfscope}%
\begin{pgfscope}%
\pgfpathrectangle{\pgfqpoint{0.765000in}{0.660000in}}{\pgfqpoint{4.620000in}{4.620000in}}%
\pgfusepath{clip}%
\pgfsetbuttcap%
\pgfsetroundjoin%
\definecolor{currentfill}{rgb}{1.000000,0.894118,0.788235}%
\pgfsetfillcolor{currentfill}%
\pgfsetlinewidth{0.000000pt}%
\definecolor{currentstroke}{rgb}{1.000000,0.894118,0.788235}%
\pgfsetstrokecolor{currentstroke}%
\pgfsetdash{}{0pt}%
\pgfpathmoveto{\pgfqpoint{3.986612in}{3.175388in}}%
\pgfpathlineto{\pgfqpoint{3.986612in}{3.321163in}}%
\pgfpathlineto{\pgfqpoint{3.239879in}{3.752289in}}%
\pgfpathlineto{\pgfqpoint{3.113634in}{3.679402in}}%
\pgfpathlineto{\pgfqpoint{3.986612in}{3.175388in}}%
\pgfpathclose%
\pgfusepath{fill}%
\end{pgfscope}%
\begin{pgfscope}%
\pgfpathrectangle{\pgfqpoint{0.765000in}{0.660000in}}{\pgfqpoint{4.620000in}{4.620000in}}%
\pgfusepath{clip}%
\pgfsetbuttcap%
\pgfsetroundjoin%
\definecolor{currentfill}{rgb}{1.000000,0.894118,0.788235}%
\pgfsetfillcolor{currentfill}%
\pgfsetlinewidth{0.000000pt}%
\definecolor{currentstroke}{rgb}{1.000000,0.894118,0.788235}%
\pgfsetstrokecolor{currentstroke}%
\pgfsetdash{}{0pt}%
\pgfpathmoveto{\pgfqpoint{3.860367in}{3.102500in}}%
\pgfpathlineto{\pgfqpoint{3.986612in}{3.175388in}}%
\pgfpathlineto{\pgfqpoint{3.239879in}{3.606514in}}%
\pgfpathlineto{\pgfqpoint{3.113634in}{3.533627in}}%
\pgfpathlineto{\pgfqpoint{3.860367in}{3.102500in}}%
\pgfpathclose%
\pgfusepath{fill}%
\end{pgfscope}%
\begin{pgfscope}%
\pgfpathrectangle{\pgfqpoint{0.765000in}{0.660000in}}{\pgfqpoint{4.620000in}{4.620000in}}%
\pgfusepath{clip}%
\pgfsetbuttcap%
\pgfsetroundjoin%
\definecolor{currentfill}{rgb}{1.000000,0.894118,0.788235}%
\pgfsetfillcolor{currentfill}%
\pgfsetlinewidth{0.000000pt}%
\definecolor{currentstroke}{rgb}{1.000000,0.894118,0.788235}%
\pgfsetstrokecolor{currentstroke}%
\pgfsetdash{}{0pt}%
\pgfpathmoveto{\pgfqpoint{3.986612in}{3.321163in}}%
\pgfpathlineto{\pgfqpoint{3.860367in}{3.394051in}}%
\pgfpathlineto{\pgfqpoint{3.113634in}{3.825177in}}%
\pgfpathlineto{\pgfqpoint{3.239879in}{3.752289in}}%
\pgfpathlineto{\pgfqpoint{3.986612in}{3.321163in}}%
\pgfpathclose%
\pgfusepath{fill}%
\end{pgfscope}%
\begin{pgfscope}%
\pgfpathrectangle{\pgfqpoint{0.765000in}{0.660000in}}{\pgfqpoint{4.620000in}{4.620000in}}%
\pgfusepath{clip}%
\pgfsetbuttcap%
\pgfsetroundjoin%
\definecolor{currentfill}{rgb}{1.000000,0.894118,0.788235}%
\pgfsetfillcolor{currentfill}%
\pgfsetlinewidth{0.000000pt}%
\definecolor{currentstroke}{rgb}{1.000000,0.894118,0.788235}%
\pgfsetstrokecolor{currentstroke}%
\pgfsetdash{}{0pt}%
\pgfpathmoveto{\pgfqpoint{3.734122in}{3.175388in}}%
\pgfpathlineto{\pgfqpoint{3.860367in}{3.102500in}}%
\pgfpathlineto{\pgfqpoint{3.113634in}{3.533627in}}%
\pgfpathlineto{\pgfqpoint{2.987389in}{3.606514in}}%
\pgfpathlineto{\pgfqpoint{3.734122in}{3.175388in}}%
\pgfpathclose%
\pgfusepath{fill}%
\end{pgfscope}%
\begin{pgfscope}%
\pgfpathrectangle{\pgfqpoint{0.765000in}{0.660000in}}{\pgfqpoint{4.620000in}{4.620000in}}%
\pgfusepath{clip}%
\pgfsetbuttcap%
\pgfsetroundjoin%
\definecolor{currentfill}{rgb}{1.000000,0.894118,0.788235}%
\pgfsetfillcolor{currentfill}%
\pgfsetlinewidth{0.000000pt}%
\definecolor{currentstroke}{rgb}{1.000000,0.894118,0.788235}%
\pgfsetstrokecolor{currentstroke}%
\pgfsetdash{}{0pt}%
\pgfpathmoveto{\pgfqpoint{3.860367in}{3.394051in}}%
\pgfpathlineto{\pgfqpoint{3.734122in}{3.321163in}}%
\pgfpathlineto{\pgfqpoint{2.987389in}{3.752289in}}%
\pgfpathlineto{\pgfqpoint{3.113634in}{3.825177in}}%
\pgfpathlineto{\pgfqpoint{3.860367in}{3.394051in}}%
\pgfpathclose%
\pgfusepath{fill}%
\end{pgfscope}%
\begin{pgfscope}%
\pgfpathrectangle{\pgfqpoint{0.765000in}{0.660000in}}{\pgfqpoint{4.620000in}{4.620000in}}%
\pgfusepath{clip}%
\pgfsetbuttcap%
\pgfsetroundjoin%
\definecolor{currentfill}{rgb}{1.000000,0.894118,0.788235}%
\pgfsetfillcolor{currentfill}%
\pgfsetlinewidth{0.000000pt}%
\definecolor{currentstroke}{rgb}{1.000000,0.894118,0.788235}%
\pgfsetstrokecolor{currentstroke}%
\pgfsetdash{}{0pt}%
\pgfpathmoveto{\pgfqpoint{3.734122in}{3.175388in}}%
\pgfpathlineto{\pgfqpoint{3.734122in}{3.321163in}}%
\pgfpathlineto{\pgfqpoint{2.987389in}{3.752289in}}%
\pgfpathlineto{\pgfqpoint{3.113634in}{3.533627in}}%
\pgfpathlineto{\pgfqpoint{3.734122in}{3.175388in}}%
\pgfpathclose%
\pgfusepath{fill}%
\end{pgfscope}%
\begin{pgfscope}%
\pgfpathrectangle{\pgfqpoint{0.765000in}{0.660000in}}{\pgfqpoint{4.620000in}{4.620000in}}%
\pgfusepath{clip}%
\pgfsetbuttcap%
\pgfsetroundjoin%
\definecolor{currentfill}{rgb}{1.000000,0.894118,0.788235}%
\pgfsetfillcolor{currentfill}%
\pgfsetlinewidth{0.000000pt}%
\definecolor{currentstroke}{rgb}{1.000000,0.894118,0.788235}%
\pgfsetstrokecolor{currentstroke}%
\pgfsetdash{}{0pt}%
\pgfpathmoveto{\pgfqpoint{3.860367in}{3.102500in}}%
\pgfpathlineto{\pgfqpoint{3.860367in}{3.248275in}}%
\pgfpathlineto{\pgfqpoint{3.113634in}{3.679402in}}%
\pgfpathlineto{\pgfqpoint{2.987389in}{3.606514in}}%
\pgfpathlineto{\pgfqpoint{3.860367in}{3.102500in}}%
\pgfpathclose%
\pgfusepath{fill}%
\end{pgfscope}%
\begin{pgfscope}%
\pgfpathrectangle{\pgfqpoint{0.765000in}{0.660000in}}{\pgfqpoint{4.620000in}{4.620000in}}%
\pgfusepath{clip}%
\pgfsetbuttcap%
\pgfsetroundjoin%
\definecolor{currentfill}{rgb}{1.000000,0.894118,0.788235}%
\pgfsetfillcolor{currentfill}%
\pgfsetlinewidth{0.000000pt}%
\definecolor{currentstroke}{rgb}{1.000000,0.894118,0.788235}%
\pgfsetstrokecolor{currentstroke}%
\pgfsetdash{}{0pt}%
\pgfpathmoveto{\pgfqpoint{3.734122in}{3.321163in}}%
\pgfpathlineto{\pgfqpoint{3.860367in}{3.248275in}}%
\pgfpathlineto{\pgfqpoint{3.113634in}{3.679402in}}%
\pgfpathlineto{\pgfqpoint{2.987389in}{3.752289in}}%
\pgfpathlineto{\pgfqpoint{3.734122in}{3.321163in}}%
\pgfpathclose%
\pgfusepath{fill}%
\end{pgfscope}%
\begin{pgfscope}%
\pgfpathrectangle{\pgfqpoint{0.765000in}{0.660000in}}{\pgfqpoint{4.620000in}{4.620000in}}%
\pgfusepath{clip}%
\pgfsetbuttcap%
\pgfsetroundjoin%
\definecolor{currentfill}{rgb}{1.000000,0.894118,0.788235}%
\pgfsetfillcolor{currentfill}%
\pgfsetlinewidth{0.000000pt}%
\definecolor{currentstroke}{rgb}{1.000000,0.894118,0.788235}%
\pgfsetstrokecolor{currentstroke}%
\pgfsetdash{}{0pt}%
\pgfpathmoveto{\pgfqpoint{3.860367in}{3.248275in}}%
\pgfpathlineto{\pgfqpoint{3.986612in}{3.321163in}}%
\pgfpathlineto{\pgfqpoint{3.239879in}{3.752289in}}%
\pgfpathlineto{\pgfqpoint{3.113634in}{3.679402in}}%
\pgfpathlineto{\pgfqpoint{3.860367in}{3.248275in}}%
\pgfpathclose%
\pgfusepath{fill}%
\end{pgfscope}%
\begin{pgfscope}%
\pgfpathrectangle{\pgfqpoint{0.765000in}{0.660000in}}{\pgfqpoint{4.620000in}{4.620000in}}%
\pgfusepath{clip}%
\pgfsetbuttcap%
\pgfsetroundjoin%
\definecolor{currentfill}{rgb}{1.000000,0.894118,0.788235}%
\pgfsetfillcolor{currentfill}%
\pgfsetlinewidth{0.000000pt}%
\definecolor{currentstroke}{rgb}{1.000000,0.894118,0.788235}%
\pgfsetstrokecolor{currentstroke}%
\pgfsetdash{}{0pt}%
\pgfpathmoveto{\pgfqpoint{3.113634in}{3.679402in}}%
\pgfpathlineto{\pgfqpoint{2.987389in}{3.606514in}}%
\pgfpathlineto{\pgfqpoint{3.113634in}{3.533627in}}%
\pgfpathlineto{\pgfqpoint{3.239879in}{3.606514in}}%
\pgfpathlineto{\pgfqpoint{3.113634in}{3.679402in}}%
\pgfpathclose%
\pgfusepath{fill}%
\end{pgfscope}%
\begin{pgfscope}%
\pgfpathrectangle{\pgfqpoint{0.765000in}{0.660000in}}{\pgfqpoint{4.620000in}{4.620000in}}%
\pgfusepath{clip}%
\pgfsetbuttcap%
\pgfsetroundjoin%
\definecolor{currentfill}{rgb}{1.000000,0.894118,0.788235}%
\pgfsetfillcolor{currentfill}%
\pgfsetlinewidth{0.000000pt}%
\definecolor{currentstroke}{rgb}{1.000000,0.894118,0.788235}%
\pgfsetstrokecolor{currentstroke}%
\pgfsetdash{}{0pt}%
\pgfpathmoveto{\pgfqpoint{3.113634in}{3.679402in}}%
\pgfpathlineto{\pgfqpoint{2.987389in}{3.606514in}}%
\pgfpathlineto{\pgfqpoint{2.987389in}{3.752289in}}%
\pgfpathlineto{\pgfqpoint{3.113634in}{3.825177in}}%
\pgfpathlineto{\pgfqpoint{3.113634in}{3.679402in}}%
\pgfpathclose%
\pgfusepath{fill}%
\end{pgfscope}%
\begin{pgfscope}%
\pgfpathrectangle{\pgfqpoint{0.765000in}{0.660000in}}{\pgfqpoint{4.620000in}{4.620000in}}%
\pgfusepath{clip}%
\pgfsetbuttcap%
\pgfsetroundjoin%
\definecolor{currentfill}{rgb}{1.000000,0.894118,0.788235}%
\pgfsetfillcolor{currentfill}%
\pgfsetlinewidth{0.000000pt}%
\definecolor{currentstroke}{rgb}{1.000000,0.894118,0.788235}%
\pgfsetstrokecolor{currentstroke}%
\pgfsetdash{}{0pt}%
\pgfpathmoveto{\pgfqpoint{3.113634in}{3.679402in}}%
\pgfpathlineto{\pgfqpoint{3.239879in}{3.606514in}}%
\pgfpathlineto{\pgfqpoint{3.239879in}{3.752289in}}%
\pgfpathlineto{\pgfqpoint{3.113634in}{3.825177in}}%
\pgfpathlineto{\pgfqpoint{3.113634in}{3.679402in}}%
\pgfpathclose%
\pgfusepath{fill}%
\end{pgfscope}%
\begin{pgfscope}%
\pgfpathrectangle{\pgfqpoint{0.765000in}{0.660000in}}{\pgfqpoint{4.620000in}{4.620000in}}%
\pgfusepath{clip}%
\pgfsetbuttcap%
\pgfsetroundjoin%
\definecolor{currentfill}{rgb}{1.000000,0.894118,0.788235}%
\pgfsetfillcolor{currentfill}%
\pgfsetlinewidth{0.000000pt}%
\definecolor{currentstroke}{rgb}{1.000000,0.894118,0.788235}%
\pgfsetstrokecolor{currentstroke}%
\pgfsetdash{}{0pt}%
\pgfpathmoveto{\pgfqpoint{2.472452in}{3.309215in}}%
\pgfpathlineto{\pgfqpoint{2.346207in}{3.236328in}}%
\pgfpathlineto{\pgfqpoint{2.472452in}{3.163440in}}%
\pgfpathlineto{\pgfqpoint{2.598697in}{3.236328in}}%
\pgfpathlineto{\pgfqpoint{2.472452in}{3.309215in}}%
\pgfpathclose%
\pgfusepath{fill}%
\end{pgfscope}%
\begin{pgfscope}%
\pgfpathrectangle{\pgfqpoint{0.765000in}{0.660000in}}{\pgfqpoint{4.620000in}{4.620000in}}%
\pgfusepath{clip}%
\pgfsetbuttcap%
\pgfsetroundjoin%
\definecolor{currentfill}{rgb}{1.000000,0.894118,0.788235}%
\pgfsetfillcolor{currentfill}%
\pgfsetlinewidth{0.000000pt}%
\definecolor{currentstroke}{rgb}{1.000000,0.894118,0.788235}%
\pgfsetstrokecolor{currentstroke}%
\pgfsetdash{}{0pt}%
\pgfpathmoveto{\pgfqpoint{2.472452in}{3.309215in}}%
\pgfpathlineto{\pgfqpoint{2.346207in}{3.236328in}}%
\pgfpathlineto{\pgfqpoint{2.346207in}{3.382103in}}%
\pgfpathlineto{\pgfqpoint{2.472452in}{3.454990in}}%
\pgfpathlineto{\pgfqpoint{2.472452in}{3.309215in}}%
\pgfpathclose%
\pgfusepath{fill}%
\end{pgfscope}%
\begin{pgfscope}%
\pgfpathrectangle{\pgfqpoint{0.765000in}{0.660000in}}{\pgfqpoint{4.620000in}{4.620000in}}%
\pgfusepath{clip}%
\pgfsetbuttcap%
\pgfsetroundjoin%
\definecolor{currentfill}{rgb}{1.000000,0.894118,0.788235}%
\pgfsetfillcolor{currentfill}%
\pgfsetlinewidth{0.000000pt}%
\definecolor{currentstroke}{rgb}{1.000000,0.894118,0.788235}%
\pgfsetstrokecolor{currentstroke}%
\pgfsetdash{}{0pt}%
\pgfpathmoveto{\pgfqpoint{2.472452in}{3.309215in}}%
\pgfpathlineto{\pgfqpoint{2.598697in}{3.236328in}}%
\pgfpathlineto{\pgfqpoint{2.598697in}{3.382103in}}%
\pgfpathlineto{\pgfqpoint{2.472452in}{3.454990in}}%
\pgfpathlineto{\pgfqpoint{2.472452in}{3.309215in}}%
\pgfpathclose%
\pgfusepath{fill}%
\end{pgfscope}%
\begin{pgfscope}%
\pgfpathrectangle{\pgfqpoint{0.765000in}{0.660000in}}{\pgfqpoint{4.620000in}{4.620000in}}%
\pgfusepath{clip}%
\pgfsetbuttcap%
\pgfsetroundjoin%
\definecolor{currentfill}{rgb}{1.000000,0.894118,0.788235}%
\pgfsetfillcolor{currentfill}%
\pgfsetlinewidth{0.000000pt}%
\definecolor{currentstroke}{rgb}{1.000000,0.894118,0.788235}%
\pgfsetstrokecolor{currentstroke}%
\pgfsetdash{}{0pt}%
\pgfpathmoveto{\pgfqpoint{3.113634in}{3.825177in}}%
\pgfpathlineto{\pgfqpoint{2.987389in}{3.752289in}}%
\pgfpathlineto{\pgfqpoint{3.113634in}{3.679402in}}%
\pgfpathlineto{\pgfqpoint{3.239879in}{3.752289in}}%
\pgfpathlineto{\pgfqpoint{3.113634in}{3.825177in}}%
\pgfpathclose%
\pgfusepath{fill}%
\end{pgfscope}%
\begin{pgfscope}%
\pgfpathrectangle{\pgfqpoint{0.765000in}{0.660000in}}{\pgfqpoint{4.620000in}{4.620000in}}%
\pgfusepath{clip}%
\pgfsetbuttcap%
\pgfsetroundjoin%
\definecolor{currentfill}{rgb}{1.000000,0.894118,0.788235}%
\pgfsetfillcolor{currentfill}%
\pgfsetlinewidth{0.000000pt}%
\definecolor{currentstroke}{rgb}{1.000000,0.894118,0.788235}%
\pgfsetstrokecolor{currentstroke}%
\pgfsetdash{}{0pt}%
\pgfpathmoveto{\pgfqpoint{3.113634in}{3.533627in}}%
\pgfpathlineto{\pgfqpoint{3.239879in}{3.606514in}}%
\pgfpathlineto{\pgfqpoint{3.239879in}{3.752289in}}%
\pgfpathlineto{\pgfqpoint{3.113634in}{3.679402in}}%
\pgfpathlineto{\pgfqpoint{3.113634in}{3.533627in}}%
\pgfpathclose%
\pgfusepath{fill}%
\end{pgfscope}%
\begin{pgfscope}%
\pgfpathrectangle{\pgfqpoint{0.765000in}{0.660000in}}{\pgfqpoint{4.620000in}{4.620000in}}%
\pgfusepath{clip}%
\pgfsetbuttcap%
\pgfsetroundjoin%
\definecolor{currentfill}{rgb}{1.000000,0.894118,0.788235}%
\pgfsetfillcolor{currentfill}%
\pgfsetlinewidth{0.000000pt}%
\definecolor{currentstroke}{rgb}{1.000000,0.894118,0.788235}%
\pgfsetstrokecolor{currentstroke}%
\pgfsetdash{}{0pt}%
\pgfpathmoveto{\pgfqpoint{2.987389in}{3.606514in}}%
\pgfpathlineto{\pgfqpoint{3.113634in}{3.533627in}}%
\pgfpathlineto{\pgfqpoint{3.113634in}{3.679402in}}%
\pgfpathlineto{\pgfqpoint{2.987389in}{3.752289in}}%
\pgfpathlineto{\pgfqpoint{2.987389in}{3.606514in}}%
\pgfpathclose%
\pgfusepath{fill}%
\end{pgfscope}%
\begin{pgfscope}%
\pgfpathrectangle{\pgfqpoint{0.765000in}{0.660000in}}{\pgfqpoint{4.620000in}{4.620000in}}%
\pgfusepath{clip}%
\pgfsetbuttcap%
\pgfsetroundjoin%
\definecolor{currentfill}{rgb}{1.000000,0.894118,0.788235}%
\pgfsetfillcolor{currentfill}%
\pgfsetlinewidth{0.000000pt}%
\definecolor{currentstroke}{rgb}{1.000000,0.894118,0.788235}%
\pgfsetstrokecolor{currentstroke}%
\pgfsetdash{}{0pt}%
\pgfpathmoveto{\pgfqpoint{2.472452in}{3.454990in}}%
\pgfpathlineto{\pgfqpoint{2.346207in}{3.382103in}}%
\pgfpathlineto{\pgfqpoint{2.472452in}{3.309215in}}%
\pgfpathlineto{\pgfqpoint{2.598697in}{3.382103in}}%
\pgfpathlineto{\pgfqpoint{2.472452in}{3.454990in}}%
\pgfpathclose%
\pgfusepath{fill}%
\end{pgfscope}%
\begin{pgfscope}%
\pgfpathrectangle{\pgfqpoint{0.765000in}{0.660000in}}{\pgfqpoint{4.620000in}{4.620000in}}%
\pgfusepath{clip}%
\pgfsetbuttcap%
\pgfsetroundjoin%
\definecolor{currentfill}{rgb}{1.000000,0.894118,0.788235}%
\pgfsetfillcolor{currentfill}%
\pgfsetlinewidth{0.000000pt}%
\definecolor{currentstroke}{rgb}{1.000000,0.894118,0.788235}%
\pgfsetstrokecolor{currentstroke}%
\pgfsetdash{}{0pt}%
\pgfpathmoveto{\pgfqpoint{2.472452in}{3.163440in}}%
\pgfpathlineto{\pgfqpoint{2.598697in}{3.236328in}}%
\pgfpathlineto{\pgfqpoint{2.598697in}{3.382103in}}%
\pgfpathlineto{\pgfqpoint{2.472452in}{3.309215in}}%
\pgfpathlineto{\pgfqpoint{2.472452in}{3.163440in}}%
\pgfpathclose%
\pgfusepath{fill}%
\end{pgfscope}%
\begin{pgfscope}%
\pgfpathrectangle{\pgfqpoint{0.765000in}{0.660000in}}{\pgfqpoint{4.620000in}{4.620000in}}%
\pgfusepath{clip}%
\pgfsetbuttcap%
\pgfsetroundjoin%
\definecolor{currentfill}{rgb}{1.000000,0.894118,0.788235}%
\pgfsetfillcolor{currentfill}%
\pgfsetlinewidth{0.000000pt}%
\definecolor{currentstroke}{rgb}{1.000000,0.894118,0.788235}%
\pgfsetstrokecolor{currentstroke}%
\pgfsetdash{}{0pt}%
\pgfpathmoveto{\pgfqpoint{2.346207in}{3.236328in}}%
\pgfpathlineto{\pgfqpoint{2.472452in}{3.163440in}}%
\pgfpathlineto{\pgfqpoint{2.472452in}{3.309215in}}%
\pgfpathlineto{\pgfqpoint{2.346207in}{3.382103in}}%
\pgfpathlineto{\pgfqpoint{2.346207in}{3.236328in}}%
\pgfpathclose%
\pgfusepath{fill}%
\end{pgfscope}%
\begin{pgfscope}%
\pgfpathrectangle{\pgfqpoint{0.765000in}{0.660000in}}{\pgfqpoint{4.620000in}{4.620000in}}%
\pgfusepath{clip}%
\pgfsetbuttcap%
\pgfsetroundjoin%
\definecolor{currentfill}{rgb}{1.000000,0.894118,0.788235}%
\pgfsetfillcolor{currentfill}%
\pgfsetlinewidth{0.000000pt}%
\definecolor{currentstroke}{rgb}{1.000000,0.894118,0.788235}%
\pgfsetstrokecolor{currentstroke}%
\pgfsetdash{}{0pt}%
\pgfpathmoveto{\pgfqpoint{3.113634in}{3.679402in}}%
\pgfpathlineto{\pgfqpoint{2.987389in}{3.606514in}}%
\pgfpathlineto{\pgfqpoint{2.346207in}{3.236328in}}%
\pgfpathlineto{\pgfqpoint{2.472452in}{3.309215in}}%
\pgfpathlineto{\pgfqpoint{3.113634in}{3.679402in}}%
\pgfpathclose%
\pgfusepath{fill}%
\end{pgfscope}%
\begin{pgfscope}%
\pgfpathrectangle{\pgfqpoint{0.765000in}{0.660000in}}{\pgfqpoint{4.620000in}{4.620000in}}%
\pgfusepath{clip}%
\pgfsetbuttcap%
\pgfsetroundjoin%
\definecolor{currentfill}{rgb}{1.000000,0.894118,0.788235}%
\pgfsetfillcolor{currentfill}%
\pgfsetlinewidth{0.000000pt}%
\definecolor{currentstroke}{rgb}{1.000000,0.894118,0.788235}%
\pgfsetstrokecolor{currentstroke}%
\pgfsetdash{}{0pt}%
\pgfpathmoveto{\pgfqpoint{3.239879in}{3.606514in}}%
\pgfpathlineto{\pgfqpoint{3.113634in}{3.679402in}}%
\pgfpathlineto{\pgfqpoint{2.472452in}{3.309215in}}%
\pgfpathlineto{\pgfqpoint{2.598697in}{3.236328in}}%
\pgfpathlineto{\pgfqpoint{3.239879in}{3.606514in}}%
\pgfpathclose%
\pgfusepath{fill}%
\end{pgfscope}%
\begin{pgfscope}%
\pgfpathrectangle{\pgfqpoint{0.765000in}{0.660000in}}{\pgfqpoint{4.620000in}{4.620000in}}%
\pgfusepath{clip}%
\pgfsetbuttcap%
\pgfsetroundjoin%
\definecolor{currentfill}{rgb}{1.000000,0.894118,0.788235}%
\pgfsetfillcolor{currentfill}%
\pgfsetlinewidth{0.000000pt}%
\definecolor{currentstroke}{rgb}{1.000000,0.894118,0.788235}%
\pgfsetstrokecolor{currentstroke}%
\pgfsetdash{}{0pt}%
\pgfpathmoveto{\pgfqpoint{3.113634in}{3.679402in}}%
\pgfpathlineto{\pgfqpoint{3.113634in}{3.825177in}}%
\pgfpathlineto{\pgfqpoint{2.472452in}{3.454990in}}%
\pgfpathlineto{\pgfqpoint{2.598697in}{3.236328in}}%
\pgfpathlineto{\pgfqpoint{3.113634in}{3.679402in}}%
\pgfpathclose%
\pgfusepath{fill}%
\end{pgfscope}%
\begin{pgfscope}%
\pgfpathrectangle{\pgfqpoint{0.765000in}{0.660000in}}{\pgfqpoint{4.620000in}{4.620000in}}%
\pgfusepath{clip}%
\pgfsetbuttcap%
\pgfsetroundjoin%
\definecolor{currentfill}{rgb}{1.000000,0.894118,0.788235}%
\pgfsetfillcolor{currentfill}%
\pgfsetlinewidth{0.000000pt}%
\definecolor{currentstroke}{rgb}{1.000000,0.894118,0.788235}%
\pgfsetstrokecolor{currentstroke}%
\pgfsetdash{}{0pt}%
\pgfpathmoveto{\pgfqpoint{3.239879in}{3.606514in}}%
\pgfpathlineto{\pgfqpoint{3.239879in}{3.752289in}}%
\pgfpathlineto{\pgfqpoint{2.598697in}{3.382103in}}%
\pgfpathlineto{\pgfqpoint{2.472452in}{3.309215in}}%
\pgfpathlineto{\pgfqpoint{3.239879in}{3.606514in}}%
\pgfpathclose%
\pgfusepath{fill}%
\end{pgfscope}%
\begin{pgfscope}%
\pgfpathrectangle{\pgfqpoint{0.765000in}{0.660000in}}{\pgfqpoint{4.620000in}{4.620000in}}%
\pgfusepath{clip}%
\pgfsetbuttcap%
\pgfsetroundjoin%
\definecolor{currentfill}{rgb}{1.000000,0.894118,0.788235}%
\pgfsetfillcolor{currentfill}%
\pgfsetlinewidth{0.000000pt}%
\definecolor{currentstroke}{rgb}{1.000000,0.894118,0.788235}%
\pgfsetstrokecolor{currentstroke}%
\pgfsetdash{}{0pt}%
\pgfpathmoveto{\pgfqpoint{3.113634in}{3.533627in}}%
\pgfpathlineto{\pgfqpoint{3.239879in}{3.606514in}}%
\pgfpathlineto{\pgfqpoint{2.598697in}{3.236328in}}%
\pgfpathlineto{\pgfqpoint{2.472452in}{3.163440in}}%
\pgfpathlineto{\pgfqpoint{3.113634in}{3.533627in}}%
\pgfpathclose%
\pgfusepath{fill}%
\end{pgfscope}%
\begin{pgfscope}%
\pgfpathrectangle{\pgfqpoint{0.765000in}{0.660000in}}{\pgfqpoint{4.620000in}{4.620000in}}%
\pgfusepath{clip}%
\pgfsetbuttcap%
\pgfsetroundjoin%
\definecolor{currentfill}{rgb}{1.000000,0.894118,0.788235}%
\pgfsetfillcolor{currentfill}%
\pgfsetlinewidth{0.000000pt}%
\definecolor{currentstroke}{rgb}{1.000000,0.894118,0.788235}%
\pgfsetstrokecolor{currentstroke}%
\pgfsetdash{}{0pt}%
\pgfpathmoveto{\pgfqpoint{3.239879in}{3.752289in}}%
\pgfpathlineto{\pgfqpoint{3.113634in}{3.825177in}}%
\pgfpathlineto{\pgfqpoint{2.472452in}{3.454990in}}%
\pgfpathlineto{\pgfqpoint{2.598697in}{3.382103in}}%
\pgfpathlineto{\pgfqpoint{3.239879in}{3.752289in}}%
\pgfpathclose%
\pgfusepath{fill}%
\end{pgfscope}%
\begin{pgfscope}%
\pgfpathrectangle{\pgfqpoint{0.765000in}{0.660000in}}{\pgfqpoint{4.620000in}{4.620000in}}%
\pgfusepath{clip}%
\pgfsetbuttcap%
\pgfsetroundjoin%
\definecolor{currentfill}{rgb}{1.000000,0.894118,0.788235}%
\pgfsetfillcolor{currentfill}%
\pgfsetlinewidth{0.000000pt}%
\definecolor{currentstroke}{rgb}{1.000000,0.894118,0.788235}%
\pgfsetstrokecolor{currentstroke}%
\pgfsetdash{}{0pt}%
\pgfpathmoveto{\pgfqpoint{2.987389in}{3.606514in}}%
\pgfpathlineto{\pgfqpoint{3.113634in}{3.533627in}}%
\pgfpathlineto{\pgfqpoint{2.472452in}{3.163440in}}%
\pgfpathlineto{\pgfqpoint{2.346207in}{3.236328in}}%
\pgfpathlineto{\pgfqpoint{2.987389in}{3.606514in}}%
\pgfpathclose%
\pgfusepath{fill}%
\end{pgfscope}%
\begin{pgfscope}%
\pgfpathrectangle{\pgfqpoint{0.765000in}{0.660000in}}{\pgfqpoint{4.620000in}{4.620000in}}%
\pgfusepath{clip}%
\pgfsetbuttcap%
\pgfsetroundjoin%
\definecolor{currentfill}{rgb}{1.000000,0.894118,0.788235}%
\pgfsetfillcolor{currentfill}%
\pgfsetlinewidth{0.000000pt}%
\definecolor{currentstroke}{rgb}{1.000000,0.894118,0.788235}%
\pgfsetstrokecolor{currentstroke}%
\pgfsetdash{}{0pt}%
\pgfpathmoveto{\pgfqpoint{3.113634in}{3.825177in}}%
\pgfpathlineto{\pgfqpoint{2.987389in}{3.752289in}}%
\pgfpathlineto{\pgfqpoint{2.346207in}{3.382103in}}%
\pgfpathlineto{\pgfqpoint{2.472452in}{3.454990in}}%
\pgfpathlineto{\pgfqpoint{3.113634in}{3.825177in}}%
\pgfpathclose%
\pgfusepath{fill}%
\end{pgfscope}%
\begin{pgfscope}%
\pgfpathrectangle{\pgfqpoint{0.765000in}{0.660000in}}{\pgfqpoint{4.620000in}{4.620000in}}%
\pgfusepath{clip}%
\pgfsetbuttcap%
\pgfsetroundjoin%
\definecolor{currentfill}{rgb}{1.000000,0.894118,0.788235}%
\pgfsetfillcolor{currentfill}%
\pgfsetlinewidth{0.000000pt}%
\definecolor{currentstroke}{rgb}{1.000000,0.894118,0.788235}%
\pgfsetstrokecolor{currentstroke}%
\pgfsetdash{}{0pt}%
\pgfpathmoveto{\pgfqpoint{2.987389in}{3.606514in}}%
\pgfpathlineto{\pgfqpoint{2.987389in}{3.752289in}}%
\pgfpathlineto{\pgfqpoint{2.346207in}{3.382103in}}%
\pgfpathlineto{\pgfqpoint{2.472452in}{3.163440in}}%
\pgfpathlineto{\pgfqpoint{2.987389in}{3.606514in}}%
\pgfpathclose%
\pgfusepath{fill}%
\end{pgfscope}%
\begin{pgfscope}%
\pgfpathrectangle{\pgfqpoint{0.765000in}{0.660000in}}{\pgfqpoint{4.620000in}{4.620000in}}%
\pgfusepath{clip}%
\pgfsetbuttcap%
\pgfsetroundjoin%
\definecolor{currentfill}{rgb}{1.000000,0.894118,0.788235}%
\pgfsetfillcolor{currentfill}%
\pgfsetlinewidth{0.000000pt}%
\definecolor{currentstroke}{rgb}{1.000000,0.894118,0.788235}%
\pgfsetstrokecolor{currentstroke}%
\pgfsetdash{}{0pt}%
\pgfpathmoveto{\pgfqpoint{3.113634in}{3.533627in}}%
\pgfpathlineto{\pgfqpoint{3.113634in}{3.679402in}}%
\pgfpathlineto{\pgfqpoint{2.472452in}{3.309215in}}%
\pgfpathlineto{\pgfqpoint{2.346207in}{3.236328in}}%
\pgfpathlineto{\pgfqpoint{3.113634in}{3.533627in}}%
\pgfpathclose%
\pgfusepath{fill}%
\end{pgfscope}%
\begin{pgfscope}%
\pgfpathrectangle{\pgfqpoint{0.765000in}{0.660000in}}{\pgfqpoint{4.620000in}{4.620000in}}%
\pgfusepath{clip}%
\pgfsetbuttcap%
\pgfsetroundjoin%
\definecolor{currentfill}{rgb}{1.000000,0.894118,0.788235}%
\pgfsetfillcolor{currentfill}%
\pgfsetlinewidth{0.000000pt}%
\definecolor{currentstroke}{rgb}{1.000000,0.894118,0.788235}%
\pgfsetstrokecolor{currentstroke}%
\pgfsetdash{}{0pt}%
\pgfpathmoveto{\pgfqpoint{2.987389in}{3.752289in}}%
\pgfpathlineto{\pgfqpoint{3.113634in}{3.679402in}}%
\pgfpathlineto{\pgfqpoint{2.472452in}{3.309215in}}%
\pgfpathlineto{\pgfqpoint{2.346207in}{3.382103in}}%
\pgfpathlineto{\pgfqpoint{2.987389in}{3.752289in}}%
\pgfpathclose%
\pgfusepath{fill}%
\end{pgfscope}%
\begin{pgfscope}%
\pgfpathrectangle{\pgfqpoint{0.765000in}{0.660000in}}{\pgfqpoint{4.620000in}{4.620000in}}%
\pgfusepath{clip}%
\pgfsetbuttcap%
\pgfsetroundjoin%
\definecolor{currentfill}{rgb}{1.000000,0.894118,0.788235}%
\pgfsetfillcolor{currentfill}%
\pgfsetlinewidth{0.000000pt}%
\definecolor{currentstroke}{rgb}{1.000000,0.894118,0.788235}%
\pgfsetstrokecolor{currentstroke}%
\pgfsetdash{}{0pt}%
\pgfpathmoveto{\pgfqpoint{3.113634in}{3.679402in}}%
\pgfpathlineto{\pgfqpoint{3.239879in}{3.752289in}}%
\pgfpathlineto{\pgfqpoint{2.598697in}{3.382103in}}%
\pgfpathlineto{\pgfqpoint{2.472452in}{3.309215in}}%
\pgfpathlineto{\pgfqpoint{3.113634in}{3.679402in}}%
\pgfpathclose%
\pgfusepath{fill}%
\end{pgfscope}%
\begin{pgfscope}%
\pgfpathrectangle{\pgfqpoint{0.765000in}{0.660000in}}{\pgfqpoint{4.620000in}{4.620000in}}%
\pgfusepath{clip}%
\pgfsetbuttcap%
\pgfsetroundjoin%
\definecolor{currentfill}{rgb}{1.000000,0.894118,0.788235}%
\pgfsetfillcolor{currentfill}%
\pgfsetlinewidth{0.000000pt}%
\definecolor{currentstroke}{rgb}{1.000000,0.894118,0.788235}%
\pgfsetstrokecolor{currentstroke}%
\pgfsetdash{}{0pt}%
\pgfpathmoveto{\pgfqpoint{3.113634in}{3.679402in}}%
\pgfpathlineto{\pgfqpoint{2.987389in}{3.606514in}}%
\pgfpathlineto{\pgfqpoint{3.113634in}{3.533627in}}%
\pgfpathlineto{\pgfqpoint{3.239879in}{3.606514in}}%
\pgfpathlineto{\pgfqpoint{3.113634in}{3.679402in}}%
\pgfpathclose%
\pgfusepath{fill}%
\end{pgfscope}%
\begin{pgfscope}%
\pgfpathrectangle{\pgfqpoint{0.765000in}{0.660000in}}{\pgfqpoint{4.620000in}{4.620000in}}%
\pgfusepath{clip}%
\pgfsetbuttcap%
\pgfsetroundjoin%
\definecolor{currentfill}{rgb}{1.000000,0.894118,0.788235}%
\pgfsetfillcolor{currentfill}%
\pgfsetlinewidth{0.000000pt}%
\definecolor{currentstroke}{rgb}{1.000000,0.894118,0.788235}%
\pgfsetstrokecolor{currentstroke}%
\pgfsetdash{}{0pt}%
\pgfpathmoveto{\pgfqpoint{3.113634in}{3.679402in}}%
\pgfpathlineto{\pgfqpoint{2.987389in}{3.606514in}}%
\pgfpathlineto{\pgfqpoint{2.987389in}{3.752289in}}%
\pgfpathlineto{\pgfqpoint{3.113634in}{3.825177in}}%
\pgfpathlineto{\pgfqpoint{3.113634in}{3.679402in}}%
\pgfpathclose%
\pgfusepath{fill}%
\end{pgfscope}%
\begin{pgfscope}%
\pgfpathrectangle{\pgfqpoint{0.765000in}{0.660000in}}{\pgfqpoint{4.620000in}{4.620000in}}%
\pgfusepath{clip}%
\pgfsetbuttcap%
\pgfsetroundjoin%
\definecolor{currentfill}{rgb}{1.000000,0.894118,0.788235}%
\pgfsetfillcolor{currentfill}%
\pgfsetlinewidth{0.000000pt}%
\definecolor{currentstroke}{rgb}{1.000000,0.894118,0.788235}%
\pgfsetstrokecolor{currentstroke}%
\pgfsetdash{}{0pt}%
\pgfpathmoveto{\pgfqpoint{3.113634in}{3.679402in}}%
\pgfpathlineto{\pgfqpoint{3.239879in}{3.606514in}}%
\pgfpathlineto{\pgfqpoint{3.239879in}{3.752289in}}%
\pgfpathlineto{\pgfqpoint{3.113634in}{3.825177in}}%
\pgfpathlineto{\pgfqpoint{3.113634in}{3.679402in}}%
\pgfpathclose%
\pgfusepath{fill}%
\end{pgfscope}%
\begin{pgfscope}%
\pgfpathrectangle{\pgfqpoint{0.765000in}{0.660000in}}{\pgfqpoint{4.620000in}{4.620000in}}%
\pgfusepath{clip}%
\pgfsetbuttcap%
\pgfsetroundjoin%
\definecolor{currentfill}{rgb}{1.000000,0.894118,0.788235}%
\pgfsetfillcolor{currentfill}%
\pgfsetlinewidth{0.000000pt}%
\definecolor{currentstroke}{rgb}{1.000000,0.894118,0.788235}%
\pgfsetstrokecolor{currentstroke}%
\pgfsetdash{}{0pt}%
\pgfpathmoveto{\pgfqpoint{3.860367in}{3.248275in}}%
\pgfpathlineto{\pgfqpoint{3.734122in}{3.175388in}}%
\pgfpathlineto{\pgfqpoint{3.860367in}{3.102500in}}%
\pgfpathlineto{\pgfqpoint{3.986612in}{3.175388in}}%
\pgfpathlineto{\pgfqpoint{3.860367in}{3.248275in}}%
\pgfpathclose%
\pgfusepath{fill}%
\end{pgfscope}%
\begin{pgfscope}%
\pgfpathrectangle{\pgfqpoint{0.765000in}{0.660000in}}{\pgfqpoint{4.620000in}{4.620000in}}%
\pgfusepath{clip}%
\pgfsetbuttcap%
\pgfsetroundjoin%
\definecolor{currentfill}{rgb}{1.000000,0.894118,0.788235}%
\pgfsetfillcolor{currentfill}%
\pgfsetlinewidth{0.000000pt}%
\definecolor{currentstroke}{rgb}{1.000000,0.894118,0.788235}%
\pgfsetstrokecolor{currentstroke}%
\pgfsetdash{}{0pt}%
\pgfpathmoveto{\pgfqpoint{3.860367in}{3.248275in}}%
\pgfpathlineto{\pgfqpoint{3.734122in}{3.175388in}}%
\pgfpathlineto{\pgfqpoint{3.734122in}{3.321163in}}%
\pgfpathlineto{\pgfqpoint{3.860367in}{3.394051in}}%
\pgfpathlineto{\pgfqpoint{3.860367in}{3.248275in}}%
\pgfpathclose%
\pgfusepath{fill}%
\end{pgfscope}%
\begin{pgfscope}%
\pgfpathrectangle{\pgfqpoint{0.765000in}{0.660000in}}{\pgfqpoint{4.620000in}{4.620000in}}%
\pgfusepath{clip}%
\pgfsetbuttcap%
\pgfsetroundjoin%
\definecolor{currentfill}{rgb}{1.000000,0.894118,0.788235}%
\pgfsetfillcolor{currentfill}%
\pgfsetlinewidth{0.000000pt}%
\definecolor{currentstroke}{rgb}{1.000000,0.894118,0.788235}%
\pgfsetstrokecolor{currentstroke}%
\pgfsetdash{}{0pt}%
\pgfpathmoveto{\pgfqpoint{3.860367in}{3.248275in}}%
\pgfpathlineto{\pgfqpoint{3.986612in}{3.175388in}}%
\pgfpathlineto{\pgfqpoint{3.986612in}{3.321163in}}%
\pgfpathlineto{\pgfqpoint{3.860367in}{3.394051in}}%
\pgfpathlineto{\pgfqpoint{3.860367in}{3.248275in}}%
\pgfpathclose%
\pgfusepath{fill}%
\end{pgfscope}%
\begin{pgfscope}%
\pgfpathrectangle{\pgfqpoint{0.765000in}{0.660000in}}{\pgfqpoint{4.620000in}{4.620000in}}%
\pgfusepath{clip}%
\pgfsetbuttcap%
\pgfsetroundjoin%
\definecolor{currentfill}{rgb}{1.000000,0.894118,0.788235}%
\pgfsetfillcolor{currentfill}%
\pgfsetlinewidth{0.000000pt}%
\definecolor{currentstroke}{rgb}{1.000000,0.894118,0.788235}%
\pgfsetstrokecolor{currentstroke}%
\pgfsetdash{}{0pt}%
\pgfpathmoveto{\pgfqpoint{3.113634in}{3.825177in}}%
\pgfpathlineto{\pgfqpoint{2.987389in}{3.752289in}}%
\pgfpathlineto{\pgfqpoint{3.113634in}{3.679402in}}%
\pgfpathlineto{\pgfqpoint{3.239879in}{3.752289in}}%
\pgfpathlineto{\pgfqpoint{3.113634in}{3.825177in}}%
\pgfpathclose%
\pgfusepath{fill}%
\end{pgfscope}%
\begin{pgfscope}%
\pgfpathrectangle{\pgfqpoint{0.765000in}{0.660000in}}{\pgfqpoint{4.620000in}{4.620000in}}%
\pgfusepath{clip}%
\pgfsetbuttcap%
\pgfsetroundjoin%
\definecolor{currentfill}{rgb}{1.000000,0.894118,0.788235}%
\pgfsetfillcolor{currentfill}%
\pgfsetlinewidth{0.000000pt}%
\definecolor{currentstroke}{rgb}{1.000000,0.894118,0.788235}%
\pgfsetstrokecolor{currentstroke}%
\pgfsetdash{}{0pt}%
\pgfpathmoveto{\pgfqpoint{3.113634in}{3.533627in}}%
\pgfpathlineto{\pgfqpoint{3.239879in}{3.606514in}}%
\pgfpathlineto{\pgfqpoint{3.239879in}{3.752289in}}%
\pgfpathlineto{\pgfqpoint{3.113634in}{3.679402in}}%
\pgfpathlineto{\pgfqpoint{3.113634in}{3.533627in}}%
\pgfpathclose%
\pgfusepath{fill}%
\end{pgfscope}%
\begin{pgfscope}%
\pgfpathrectangle{\pgfqpoint{0.765000in}{0.660000in}}{\pgfqpoint{4.620000in}{4.620000in}}%
\pgfusepath{clip}%
\pgfsetbuttcap%
\pgfsetroundjoin%
\definecolor{currentfill}{rgb}{1.000000,0.894118,0.788235}%
\pgfsetfillcolor{currentfill}%
\pgfsetlinewidth{0.000000pt}%
\definecolor{currentstroke}{rgb}{1.000000,0.894118,0.788235}%
\pgfsetstrokecolor{currentstroke}%
\pgfsetdash{}{0pt}%
\pgfpathmoveto{\pgfqpoint{2.987389in}{3.606514in}}%
\pgfpathlineto{\pgfqpoint{3.113634in}{3.533627in}}%
\pgfpathlineto{\pgfqpoint{3.113634in}{3.679402in}}%
\pgfpathlineto{\pgfqpoint{2.987389in}{3.752289in}}%
\pgfpathlineto{\pgfqpoint{2.987389in}{3.606514in}}%
\pgfpathclose%
\pgfusepath{fill}%
\end{pgfscope}%
\begin{pgfscope}%
\pgfpathrectangle{\pgfqpoint{0.765000in}{0.660000in}}{\pgfqpoint{4.620000in}{4.620000in}}%
\pgfusepath{clip}%
\pgfsetbuttcap%
\pgfsetroundjoin%
\definecolor{currentfill}{rgb}{1.000000,0.894118,0.788235}%
\pgfsetfillcolor{currentfill}%
\pgfsetlinewidth{0.000000pt}%
\definecolor{currentstroke}{rgb}{1.000000,0.894118,0.788235}%
\pgfsetstrokecolor{currentstroke}%
\pgfsetdash{}{0pt}%
\pgfpathmoveto{\pgfqpoint{3.860367in}{3.394051in}}%
\pgfpathlineto{\pgfqpoint{3.734122in}{3.321163in}}%
\pgfpathlineto{\pgfqpoint{3.860367in}{3.248275in}}%
\pgfpathlineto{\pgfqpoint{3.986612in}{3.321163in}}%
\pgfpathlineto{\pgfqpoint{3.860367in}{3.394051in}}%
\pgfpathclose%
\pgfusepath{fill}%
\end{pgfscope}%
\begin{pgfscope}%
\pgfpathrectangle{\pgfqpoint{0.765000in}{0.660000in}}{\pgfqpoint{4.620000in}{4.620000in}}%
\pgfusepath{clip}%
\pgfsetbuttcap%
\pgfsetroundjoin%
\definecolor{currentfill}{rgb}{1.000000,0.894118,0.788235}%
\pgfsetfillcolor{currentfill}%
\pgfsetlinewidth{0.000000pt}%
\definecolor{currentstroke}{rgb}{1.000000,0.894118,0.788235}%
\pgfsetstrokecolor{currentstroke}%
\pgfsetdash{}{0pt}%
\pgfpathmoveto{\pgfqpoint{3.860367in}{3.102500in}}%
\pgfpathlineto{\pgfqpoint{3.986612in}{3.175388in}}%
\pgfpathlineto{\pgfqpoint{3.986612in}{3.321163in}}%
\pgfpathlineto{\pgfqpoint{3.860367in}{3.248275in}}%
\pgfpathlineto{\pgfqpoint{3.860367in}{3.102500in}}%
\pgfpathclose%
\pgfusepath{fill}%
\end{pgfscope}%
\begin{pgfscope}%
\pgfpathrectangle{\pgfqpoint{0.765000in}{0.660000in}}{\pgfqpoint{4.620000in}{4.620000in}}%
\pgfusepath{clip}%
\pgfsetbuttcap%
\pgfsetroundjoin%
\definecolor{currentfill}{rgb}{1.000000,0.894118,0.788235}%
\pgfsetfillcolor{currentfill}%
\pgfsetlinewidth{0.000000pt}%
\definecolor{currentstroke}{rgb}{1.000000,0.894118,0.788235}%
\pgfsetstrokecolor{currentstroke}%
\pgfsetdash{}{0pt}%
\pgfpathmoveto{\pgfqpoint{3.734122in}{3.175388in}}%
\pgfpathlineto{\pgfqpoint{3.860367in}{3.102500in}}%
\pgfpathlineto{\pgfqpoint{3.860367in}{3.248275in}}%
\pgfpathlineto{\pgfqpoint{3.734122in}{3.321163in}}%
\pgfpathlineto{\pgfqpoint{3.734122in}{3.175388in}}%
\pgfpathclose%
\pgfusepath{fill}%
\end{pgfscope}%
\begin{pgfscope}%
\pgfpathrectangle{\pgfqpoint{0.765000in}{0.660000in}}{\pgfqpoint{4.620000in}{4.620000in}}%
\pgfusepath{clip}%
\pgfsetbuttcap%
\pgfsetroundjoin%
\definecolor{currentfill}{rgb}{1.000000,0.894118,0.788235}%
\pgfsetfillcolor{currentfill}%
\pgfsetlinewidth{0.000000pt}%
\definecolor{currentstroke}{rgb}{1.000000,0.894118,0.788235}%
\pgfsetstrokecolor{currentstroke}%
\pgfsetdash{}{0pt}%
\pgfpathmoveto{\pgfqpoint{3.113634in}{3.679402in}}%
\pgfpathlineto{\pgfqpoint{2.987389in}{3.606514in}}%
\pgfpathlineto{\pgfqpoint{3.734122in}{3.175388in}}%
\pgfpathlineto{\pgfqpoint{3.860367in}{3.248275in}}%
\pgfpathlineto{\pgfqpoint{3.113634in}{3.679402in}}%
\pgfpathclose%
\pgfusepath{fill}%
\end{pgfscope}%
\begin{pgfscope}%
\pgfpathrectangle{\pgfqpoint{0.765000in}{0.660000in}}{\pgfqpoint{4.620000in}{4.620000in}}%
\pgfusepath{clip}%
\pgfsetbuttcap%
\pgfsetroundjoin%
\definecolor{currentfill}{rgb}{1.000000,0.894118,0.788235}%
\pgfsetfillcolor{currentfill}%
\pgfsetlinewidth{0.000000pt}%
\definecolor{currentstroke}{rgb}{1.000000,0.894118,0.788235}%
\pgfsetstrokecolor{currentstroke}%
\pgfsetdash{}{0pt}%
\pgfpathmoveto{\pgfqpoint{3.239879in}{3.606514in}}%
\pgfpathlineto{\pgfqpoint{3.113634in}{3.679402in}}%
\pgfpathlineto{\pgfqpoint{3.860367in}{3.248275in}}%
\pgfpathlineto{\pgfqpoint{3.986612in}{3.175388in}}%
\pgfpathlineto{\pgfqpoint{3.239879in}{3.606514in}}%
\pgfpathclose%
\pgfusepath{fill}%
\end{pgfscope}%
\begin{pgfscope}%
\pgfpathrectangle{\pgfqpoint{0.765000in}{0.660000in}}{\pgfqpoint{4.620000in}{4.620000in}}%
\pgfusepath{clip}%
\pgfsetbuttcap%
\pgfsetroundjoin%
\definecolor{currentfill}{rgb}{1.000000,0.894118,0.788235}%
\pgfsetfillcolor{currentfill}%
\pgfsetlinewidth{0.000000pt}%
\definecolor{currentstroke}{rgb}{1.000000,0.894118,0.788235}%
\pgfsetstrokecolor{currentstroke}%
\pgfsetdash{}{0pt}%
\pgfpathmoveto{\pgfqpoint{3.113634in}{3.679402in}}%
\pgfpathlineto{\pgfqpoint{3.113634in}{3.825177in}}%
\pgfpathlineto{\pgfqpoint{3.860367in}{3.394051in}}%
\pgfpathlineto{\pgfqpoint{3.986612in}{3.175388in}}%
\pgfpathlineto{\pgfqpoint{3.113634in}{3.679402in}}%
\pgfpathclose%
\pgfusepath{fill}%
\end{pgfscope}%
\begin{pgfscope}%
\pgfpathrectangle{\pgfqpoint{0.765000in}{0.660000in}}{\pgfqpoint{4.620000in}{4.620000in}}%
\pgfusepath{clip}%
\pgfsetbuttcap%
\pgfsetroundjoin%
\definecolor{currentfill}{rgb}{1.000000,0.894118,0.788235}%
\pgfsetfillcolor{currentfill}%
\pgfsetlinewidth{0.000000pt}%
\definecolor{currentstroke}{rgb}{1.000000,0.894118,0.788235}%
\pgfsetstrokecolor{currentstroke}%
\pgfsetdash{}{0pt}%
\pgfpathmoveto{\pgfqpoint{3.239879in}{3.606514in}}%
\pgfpathlineto{\pgfqpoint{3.239879in}{3.752289in}}%
\pgfpathlineto{\pgfqpoint{3.986612in}{3.321163in}}%
\pgfpathlineto{\pgfqpoint{3.860367in}{3.248275in}}%
\pgfpathlineto{\pgfqpoint{3.239879in}{3.606514in}}%
\pgfpathclose%
\pgfusepath{fill}%
\end{pgfscope}%
\begin{pgfscope}%
\pgfpathrectangle{\pgfqpoint{0.765000in}{0.660000in}}{\pgfqpoint{4.620000in}{4.620000in}}%
\pgfusepath{clip}%
\pgfsetbuttcap%
\pgfsetroundjoin%
\definecolor{currentfill}{rgb}{1.000000,0.894118,0.788235}%
\pgfsetfillcolor{currentfill}%
\pgfsetlinewidth{0.000000pt}%
\definecolor{currentstroke}{rgb}{1.000000,0.894118,0.788235}%
\pgfsetstrokecolor{currentstroke}%
\pgfsetdash{}{0pt}%
\pgfpathmoveto{\pgfqpoint{3.113634in}{3.533627in}}%
\pgfpathlineto{\pgfqpoint{3.239879in}{3.606514in}}%
\pgfpathlineto{\pgfqpoint{3.986612in}{3.175388in}}%
\pgfpathlineto{\pgfqpoint{3.860367in}{3.102500in}}%
\pgfpathlineto{\pgfqpoint{3.113634in}{3.533627in}}%
\pgfpathclose%
\pgfusepath{fill}%
\end{pgfscope}%
\begin{pgfscope}%
\pgfpathrectangle{\pgfqpoint{0.765000in}{0.660000in}}{\pgfqpoint{4.620000in}{4.620000in}}%
\pgfusepath{clip}%
\pgfsetbuttcap%
\pgfsetroundjoin%
\definecolor{currentfill}{rgb}{1.000000,0.894118,0.788235}%
\pgfsetfillcolor{currentfill}%
\pgfsetlinewidth{0.000000pt}%
\definecolor{currentstroke}{rgb}{1.000000,0.894118,0.788235}%
\pgfsetstrokecolor{currentstroke}%
\pgfsetdash{}{0pt}%
\pgfpathmoveto{\pgfqpoint{3.239879in}{3.752289in}}%
\pgfpathlineto{\pgfqpoint{3.113634in}{3.825177in}}%
\pgfpathlineto{\pgfqpoint{3.860367in}{3.394051in}}%
\pgfpathlineto{\pgfqpoint{3.986612in}{3.321163in}}%
\pgfpathlineto{\pgfqpoint{3.239879in}{3.752289in}}%
\pgfpathclose%
\pgfusepath{fill}%
\end{pgfscope}%
\begin{pgfscope}%
\pgfpathrectangle{\pgfqpoint{0.765000in}{0.660000in}}{\pgfqpoint{4.620000in}{4.620000in}}%
\pgfusepath{clip}%
\pgfsetbuttcap%
\pgfsetroundjoin%
\definecolor{currentfill}{rgb}{1.000000,0.894118,0.788235}%
\pgfsetfillcolor{currentfill}%
\pgfsetlinewidth{0.000000pt}%
\definecolor{currentstroke}{rgb}{1.000000,0.894118,0.788235}%
\pgfsetstrokecolor{currentstroke}%
\pgfsetdash{}{0pt}%
\pgfpathmoveto{\pgfqpoint{2.987389in}{3.606514in}}%
\pgfpathlineto{\pgfqpoint{3.113634in}{3.533627in}}%
\pgfpathlineto{\pgfqpoint{3.860367in}{3.102500in}}%
\pgfpathlineto{\pgfqpoint{3.734122in}{3.175388in}}%
\pgfpathlineto{\pgfqpoint{2.987389in}{3.606514in}}%
\pgfpathclose%
\pgfusepath{fill}%
\end{pgfscope}%
\begin{pgfscope}%
\pgfpathrectangle{\pgfqpoint{0.765000in}{0.660000in}}{\pgfqpoint{4.620000in}{4.620000in}}%
\pgfusepath{clip}%
\pgfsetbuttcap%
\pgfsetroundjoin%
\definecolor{currentfill}{rgb}{1.000000,0.894118,0.788235}%
\pgfsetfillcolor{currentfill}%
\pgfsetlinewidth{0.000000pt}%
\definecolor{currentstroke}{rgb}{1.000000,0.894118,0.788235}%
\pgfsetstrokecolor{currentstroke}%
\pgfsetdash{}{0pt}%
\pgfpathmoveto{\pgfqpoint{3.113634in}{3.825177in}}%
\pgfpathlineto{\pgfqpoint{2.987389in}{3.752289in}}%
\pgfpathlineto{\pgfqpoint{3.734122in}{3.321163in}}%
\pgfpathlineto{\pgfqpoint{3.860367in}{3.394051in}}%
\pgfpathlineto{\pgfqpoint{3.113634in}{3.825177in}}%
\pgfpathclose%
\pgfusepath{fill}%
\end{pgfscope}%
\begin{pgfscope}%
\pgfpathrectangle{\pgfqpoint{0.765000in}{0.660000in}}{\pgfqpoint{4.620000in}{4.620000in}}%
\pgfusepath{clip}%
\pgfsetbuttcap%
\pgfsetroundjoin%
\definecolor{currentfill}{rgb}{1.000000,0.894118,0.788235}%
\pgfsetfillcolor{currentfill}%
\pgfsetlinewidth{0.000000pt}%
\definecolor{currentstroke}{rgb}{1.000000,0.894118,0.788235}%
\pgfsetstrokecolor{currentstroke}%
\pgfsetdash{}{0pt}%
\pgfpathmoveto{\pgfqpoint{2.987389in}{3.606514in}}%
\pgfpathlineto{\pgfqpoint{2.987389in}{3.752289in}}%
\pgfpathlineto{\pgfqpoint{3.734122in}{3.321163in}}%
\pgfpathlineto{\pgfqpoint{3.860367in}{3.102500in}}%
\pgfpathlineto{\pgfqpoint{2.987389in}{3.606514in}}%
\pgfpathclose%
\pgfusepath{fill}%
\end{pgfscope}%
\begin{pgfscope}%
\pgfpathrectangle{\pgfqpoint{0.765000in}{0.660000in}}{\pgfqpoint{4.620000in}{4.620000in}}%
\pgfusepath{clip}%
\pgfsetbuttcap%
\pgfsetroundjoin%
\definecolor{currentfill}{rgb}{1.000000,0.894118,0.788235}%
\pgfsetfillcolor{currentfill}%
\pgfsetlinewidth{0.000000pt}%
\definecolor{currentstroke}{rgb}{1.000000,0.894118,0.788235}%
\pgfsetstrokecolor{currentstroke}%
\pgfsetdash{}{0pt}%
\pgfpathmoveto{\pgfqpoint{3.113634in}{3.533627in}}%
\pgfpathlineto{\pgfqpoint{3.113634in}{3.679402in}}%
\pgfpathlineto{\pgfqpoint{3.860367in}{3.248275in}}%
\pgfpathlineto{\pgfqpoint{3.734122in}{3.175388in}}%
\pgfpathlineto{\pgfqpoint{3.113634in}{3.533627in}}%
\pgfpathclose%
\pgfusepath{fill}%
\end{pgfscope}%
\begin{pgfscope}%
\pgfpathrectangle{\pgfqpoint{0.765000in}{0.660000in}}{\pgfqpoint{4.620000in}{4.620000in}}%
\pgfusepath{clip}%
\pgfsetbuttcap%
\pgfsetroundjoin%
\definecolor{currentfill}{rgb}{1.000000,0.894118,0.788235}%
\pgfsetfillcolor{currentfill}%
\pgfsetlinewidth{0.000000pt}%
\definecolor{currentstroke}{rgb}{1.000000,0.894118,0.788235}%
\pgfsetstrokecolor{currentstroke}%
\pgfsetdash{}{0pt}%
\pgfpathmoveto{\pgfqpoint{2.987389in}{3.752289in}}%
\pgfpathlineto{\pgfqpoint{3.113634in}{3.679402in}}%
\pgfpathlineto{\pgfqpoint{3.860367in}{3.248275in}}%
\pgfpathlineto{\pgfqpoint{3.734122in}{3.321163in}}%
\pgfpathlineto{\pgfqpoint{2.987389in}{3.752289in}}%
\pgfpathclose%
\pgfusepath{fill}%
\end{pgfscope}%
\begin{pgfscope}%
\pgfpathrectangle{\pgfqpoint{0.765000in}{0.660000in}}{\pgfqpoint{4.620000in}{4.620000in}}%
\pgfusepath{clip}%
\pgfsetbuttcap%
\pgfsetroundjoin%
\definecolor{currentfill}{rgb}{1.000000,0.894118,0.788235}%
\pgfsetfillcolor{currentfill}%
\pgfsetlinewidth{0.000000pt}%
\definecolor{currentstroke}{rgb}{1.000000,0.894118,0.788235}%
\pgfsetstrokecolor{currentstroke}%
\pgfsetdash{}{0pt}%
\pgfpathmoveto{\pgfqpoint{3.113634in}{3.679402in}}%
\pgfpathlineto{\pgfqpoint{3.239879in}{3.752289in}}%
\pgfpathlineto{\pgfqpoint{3.986612in}{3.321163in}}%
\pgfpathlineto{\pgfqpoint{3.860367in}{3.248275in}}%
\pgfpathlineto{\pgfqpoint{3.113634in}{3.679402in}}%
\pgfpathclose%
\pgfusepath{fill}%
\end{pgfscope}%
\begin{pgfscope}%
\pgfpathrectangle{\pgfqpoint{0.765000in}{0.660000in}}{\pgfqpoint{4.620000in}{4.620000in}}%
\pgfusepath{clip}%
\pgfsetbuttcap%
\pgfsetroundjoin%
\definecolor{currentfill}{rgb}{1.000000,0.894118,0.788235}%
\pgfsetfillcolor{currentfill}%
\pgfsetlinewidth{0.000000pt}%
\definecolor{currentstroke}{rgb}{1.000000,0.894118,0.788235}%
\pgfsetstrokecolor{currentstroke}%
\pgfsetdash{}{0pt}%
\pgfpathmoveto{\pgfqpoint{3.113634in}{3.679402in}}%
\pgfpathlineto{\pgfqpoint{2.987389in}{3.606514in}}%
\pgfpathlineto{\pgfqpoint{3.113634in}{3.533627in}}%
\pgfpathlineto{\pgfqpoint{3.239879in}{3.606514in}}%
\pgfpathlineto{\pgfqpoint{3.113634in}{3.679402in}}%
\pgfpathclose%
\pgfusepath{fill}%
\end{pgfscope}%
\begin{pgfscope}%
\pgfpathrectangle{\pgfqpoint{0.765000in}{0.660000in}}{\pgfqpoint{4.620000in}{4.620000in}}%
\pgfusepath{clip}%
\pgfsetbuttcap%
\pgfsetroundjoin%
\definecolor{currentfill}{rgb}{1.000000,0.894118,0.788235}%
\pgfsetfillcolor{currentfill}%
\pgfsetlinewidth{0.000000pt}%
\definecolor{currentstroke}{rgb}{1.000000,0.894118,0.788235}%
\pgfsetstrokecolor{currentstroke}%
\pgfsetdash{}{0pt}%
\pgfpathmoveto{\pgfqpoint{3.113634in}{3.679402in}}%
\pgfpathlineto{\pgfqpoint{2.987389in}{3.606514in}}%
\pgfpathlineto{\pgfqpoint{2.987389in}{3.752289in}}%
\pgfpathlineto{\pgfqpoint{3.113634in}{3.825177in}}%
\pgfpathlineto{\pgfqpoint{3.113634in}{3.679402in}}%
\pgfpathclose%
\pgfusepath{fill}%
\end{pgfscope}%
\begin{pgfscope}%
\pgfpathrectangle{\pgfqpoint{0.765000in}{0.660000in}}{\pgfqpoint{4.620000in}{4.620000in}}%
\pgfusepath{clip}%
\pgfsetbuttcap%
\pgfsetroundjoin%
\definecolor{currentfill}{rgb}{1.000000,0.894118,0.788235}%
\pgfsetfillcolor{currentfill}%
\pgfsetlinewidth{0.000000pt}%
\definecolor{currentstroke}{rgb}{1.000000,0.894118,0.788235}%
\pgfsetstrokecolor{currentstroke}%
\pgfsetdash{}{0pt}%
\pgfpathmoveto{\pgfqpoint{3.113634in}{3.679402in}}%
\pgfpathlineto{\pgfqpoint{3.239879in}{3.606514in}}%
\pgfpathlineto{\pgfqpoint{3.239879in}{3.752289in}}%
\pgfpathlineto{\pgfqpoint{3.113634in}{3.825177in}}%
\pgfpathlineto{\pgfqpoint{3.113634in}{3.679402in}}%
\pgfpathclose%
\pgfusepath{fill}%
\end{pgfscope}%
\begin{pgfscope}%
\pgfpathrectangle{\pgfqpoint{0.765000in}{0.660000in}}{\pgfqpoint{4.620000in}{4.620000in}}%
\pgfusepath{clip}%
\pgfsetbuttcap%
\pgfsetroundjoin%
\definecolor{currentfill}{rgb}{1.000000,0.894118,0.788235}%
\pgfsetfillcolor{currentfill}%
\pgfsetlinewidth{0.000000pt}%
\definecolor{currentstroke}{rgb}{1.000000,0.894118,0.788235}%
\pgfsetstrokecolor{currentstroke}%
\pgfsetdash{}{0pt}%
\pgfpathmoveto{\pgfqpoint{3.113634in}{11.109713in}}%
\pgfpathlineto{\pgfqpoint{2.987389in}{11.036825in}}%
\pgfpathlineto{\pgfqpoint{3.113634in}{10.963937in}}%
\pgfpathlineto{\pgfqpoint{3.239879in}{11.036825in}}%
\pgfpathlineto{\pgfqpoint{3.113634in}{11.109713in}}%
\pgfpathclose%
\pgfusepath{fill}%
\end{pgfscope}%
\begin{pgfscope}%
\pgfpathrectangle{\pgfqpoint{0.765000in}{0.660000in}}{\pgfqpoint{4.620000in}{4.620000in}}%
\pgfusepath{clip}%
\pgfsetbuttcap%
\pgfsetroundjoin%
\definecolor{currentfill}{rgb}{1.000000,0.894118,0.788235}%
\pgfsetfillcolor{currentfill}%
\pgfsetlinewidth{0.000000pt}%
\definecolor{currentstroke}{rgb}{1.000000,0.894118,0.788235}%
\pgfsetstrokecolor{currentstroke}%
\pgfsetdash{}{0pt}%
\pgfpathmoveto{\pgfqpoint{3.113634in}{11.109713in}}%
\pgfpathlineto{\pgfqpoint{2.987389in}{11.036825in}}%
\pgfpathlineto{\pgfqpoint{2.987389in}{11.182600in}}%
\pgfpathlineto{\pgfqpoint{3.113634in}{11.255488in}}%
\pgfpathlineto{\pgfqpoint{3.113634in}{11.109713in}}%
\pgfpathclose%
\pgfusepath{fill}%
\end{pgfscope}%
\begin{pgfscope}%
\pgfpathrectangle{\pgfqpoint{0.765000in}{0.660000in}}{\pgfqpoint{4.620000in}{4.620000in}}%
\pgfusepath{clip}%
\pgfsetbuttcap%
\pgfsetroundjoin%
\definecolor{currentfill}{rgb}{1.000000,0.894118,0.788235}%
\pgfsetfillcolor{currentfill}%
\pgfsetlinewidth{0.000000pt}%
\definecolor{currentstroke}{rgb}{1.000000,0.894118,0.788235}%
\pgfsetstrokecolor{currentstroke}%
\pgfsetdash{}{0pt}%
\pgfpathmoveto{\pgfqpoint{3.113634in}{11.109713in}}%
\pgfpathlineto{\pgfqpoint{3.239879in}{11.036825in}}%
\pgfpathlineto{\pgfqpoint{3.239879in}{11.182600in}}%
\pgfpathlineto{\pgfqpoint{3.113634in}{11.255488in}}%
\pgfpathlineto{\pgfqpoint{3.113634in}{11.109713in}}%
\pgfpathclose%
\pgfusepath{fill}%
\end{pgfscope}%
\begin{pgfscope}%
\pgfpathrectangle{\pgfqpoint{0.765000in}{0.660000in}}{\pgfqpoint{4.620000in}{4.620000in}}%
\pgfusepath{clip}%
\pgfsetbuttcap%
\pgfsetroundjoin%
\definecolor{currentfill}{rgb}{1.000000,0.894118,0.788235}%
\pgfsetfillcolor{currentfill}%
\pgfsetlinewidth{0.000000pt}%
\definecolor{currentstroke}{rgb}{1.000000,0.894118,0.788235}%
\pgfsetstrokecolor{currentstroke}%
\pgfsetdash{}{0pt}%
\pgfpathmoveto{\pgfqpoint{3.113634in}{3.825177in}}%
\pgfpathlineto{\pgfqpoint{2.987389in}{3.752289in}}%
\pgfpathlineto{\pgfqpoint{3.113634in}{3.679402in}}%
\pgfpathlineto{\pgfqpoint{3.239879in}{3.752289in}}%
\pgfpathlineto{\pgfqpoint{3.113634in}{3.825177in}}%
\pgfpathclose%
\pgfusepath{fill}%
\end{pgfscope}%
\begin{pgfscope}%
\pgfpathrectangle{\pgfqpoint{0.765000in}{0.660000in}}{\pgfqpoint{4.620000in}{4.620000in}}%
\pgfusepath{clip}%
\pgfsetbuttcap%
\pgfsetroundjoin%
\definecolor{currentfill}{rgb}{1.000000,0.894118,0.788235}%
\pgfsetfillcolor{currentfill}%
\pgfsetlinewidth{0.000000pt}%
\definecolor{currentstroke}{rgb}{1.000000,0.894118,0.788235}%
\pgfsetstrokecolor{currentstroke}%
\pgfsetdash{}{0pt}%
\pgfpathmoveto{\pgfqpoint{3.113634in}{3.533627in}}%
\pgfpathlineto{\pgfqpoint{3.239879in}{3.606514in}}%
\pgfpathlineto{\pgfqpoint{3.239879in}{3.752289in}}%
\pgfpathlineto{\pgfqpoint{3.113634in}{3.679402in}}%
\pgfpathlineto{\pgfqpoint{3.113634in}{3.533627in}}%
\pgfpathclose%
\pgfusepath{fill}%
\end{pgfscope}%
\begin{pgfscope}%
\pgfpathrectangle{\pgfqpoint{0.765000in}{0.660000in}}{\pgfqpoint{4.620000in}{4.620000in}}%
\pgfusepath{clip}%
\pgfsetbuttcap%
\pgfsetroundjoin%
\definecolor{currentfill}{rgb}{1.000000,0.894118,0.788235}%
\pgfsetfillcolor{currentfill}%
\pgfsetlinewidth{0.000000pt}%
\definecolor{currentstroke}{rgb}{1.000000,0.894118,0.788235}%
\pgfsetstrokecolor{currentstroke}%
\pgfsetdash{}{0pt}%
\pgfpathmoveto{\pgfqpoint{2.987389in}{3.606514in}}%
\pgfpathlineto{\pgfqpoint{3.113634in}{3.533627in}}%
\pgfpathlineto{\pgfqpoint{3.113634in}{3.679402in}}%
\pgfpathlineto{\pgfqpoint{2.987389in}{3.752289in}}%
\pgfpathlineto{\pgfqpoint{2.987389in}{3.606514in}}%
\pgfpathclose%
\pgfusepath{fill}%
\end{pgfscope}%
\begin{pgfscope}%
\pgfpathrectangle{\pgfqpoint{0.765000in}{0.660000in}}{\pgfqpoint{4.620000in}{4.620000in}}%
\pgfusepath{clip}%
\pgfsetbuttcap%
\pgfsetroundjoin%
\definecolor{currentfill}{rgb}{1.000000,0.894118,0.788235}%
\pgfsetfillcolor{currentfill}%
\pgfsetlinewidth{0.000000pt}%
\definecolor{currentstroke}{rgb}{1.000000,0.894118,0.788235}%
\pgfsetstrokecolor{currentstroke}%
\pgfsetdash{}{0pt}%
\pgfpathmoveto{\pgfqpoint{3.113634in}{11.255488in}}%
\pgfpathlineto{\pgfqpoint{2.987389in}{11.182600in}}%
\pgfpathlineto{\pgfqpoint{3.113634in}{11.109713in}}%
\pgfpathlineto{\pgfqpoint{3.239879in}{11.182600in}}%
\pgfpathlineto{\pgfqpoint{3.113634in}{11.255488in}}%
\pgfpathclose%
\pgfusepath{fill}%
\end{pgfscope}%
\begin{pgfscope}%
\pgfpathrectangle{\pgfqpoint{0.765000in}{0.660000in}}{\pgfqpoint{4.620000in}{4.620000in}}%
\pgfusepath{clip}%
\pgfsetbuttcap%
\pgfsetroundjoin%
\definecolor{currentfill}{rgb}{1.000000,0.894118,0.788235}%
\pgfsetfillcolor{currentfill}%
\pgfsetlinewidth{0.000000pt}%
\definecolor{currentstroke}{rgb}{1.000000,0.894118,0.788235}%
\pgfsetstrokecolor{currentstroke}%
\pgfsetdash{}{0pt}%
\pgfpathmoveto{\pgfqpoint{3.113634in}{10.963937in}}%
\pgfpathlineto{\pgfqpoint{3.239879in}{11.036825in}}%
\pgfpathlineto{\pgfqpoint{3.239879in}{11.182600in}}%
\pgfpathlineto{\pgfqpoint{3.113634in}{11.109713in}}%
\pgfpathlineto{\pgfqpoint{3.113634in}{10.963937in}}%
\pgfpathclose%
\pgfusepath{fill}%
\end{pgfscope}%
\begin{pgfscope}%
\pgfpathrectangle{\pgfqpoint{0.765000in}{0.660000in}}{\pgfqpoint{4.620000in}{4.620000in}}%
\pgfusepath{clip}%
\pgfsetbuttcap%
\pgfsetroundjoin%
\definecolor{currentfill}{rgb}{1.000000,0.894118,0.788235}%
\pgfsetfillcolor{currentfill}%
\pgfsetlinewidth{0.000000pt}%
\definecolor{currentstroke}{rgb}{1.000000,0.894118,0.788235}%
\pgfsetstrokecolor{currentstroke}%
\pgfsetdash{}{0pt}%
\pgfpathmoveto{\pgfqpoint{2.987389in}{11.036825in}}%
\pgfpathlineto{\pgfqpoint{3.113634in}{10.963937in}}%
\pgfpathlineto{\pgfqpoint{3.113634in}{11.109713in}}%
\pgfpathlineto{\pgfqpoint{2.987389in}{11.182600in}}%
\pgfpathlineto{\pgfqpoint{2.987389in}{11.036825in}}%
\pgfpathclose%
\pgfusepath{fill}%
\end{pgfscope}%
\begin{pgfscope}%
\pgfpathrectangle{\pgfqpoint{0.765000in}{0.660000in}}{\pgfqpoint{4.620000in}{4.620000in}}%
\pgfusepath{clip}%
\pgfsetbuttcap%
\pgfsetroundjoin%
\definecolor{currentfill}{rgb}{1.000000,0.894118,0.788235}%
\pgfsetfillcolor{currentfill}%
\pgfsetlinewidth{0.000000pt}%
\definecolor{currentstroke}{rgb}{1.000000,0.894118,0.788235}%
\pgfsetstrokecolor{currentstroke}%
\pgfsetdash{}{0pt}%
\pgfpathmoveto{\pgfqpoint{3.113634in}{3.679402in}}%
\pgfpathlineto{\pgfqpoint{2.987389in}{3.606514in}}%
\pgfpathlineto{\pgfqpoint{2.987389in}{11.036825in}}%
\pgfpathlineto{\pgfqpoint{3.113634in}{11.109713in}}%
\pgfpathlineto{\pgfqpoint{3.113634in}{3.679402in}}%
\pgfpathclose%
\pgfusepath{fill}%
\end{pgfscope}%
\begin{pgfscope}%
\pgfpathrectangle{\pgfqpoint{0.765000in}{0.660000in}}{\pgfqpoint{4.620000in}{4.620000in}}%
\pgfusepath{clip}%
\pgfsetbuttcap%
\pgfsetroundjoin%
\definecolor{currentfill}{rgb}{1.000000,0.894118,0.788235}%
\pgfsetfillcolor{currentfill}%
\pgfsetlinewidth{0.000000pt}%
\definecolor{currentstroke}{rgb}{1.000000,0.894118,0.788235}%
\pgfsetstrokecolor{currentstroke}%
\pgfsetdash{}{0pt}%
\pgfpathmoveto{\pgfqpoint{3.239879in}{3.606514in}}%
\pgfpathlineto{\pgfqpoint{3.113634in}{3.679402in}}%
\pgfpathlineto{\pgfqpoint{3.113634in}{11.109713in}}%
\pgfpathlineto{\pgfqpoint{3.239879in}{11.036825in}}%
\pgfpathlineto{\pgfqpoint{3.239879in}{3.606514in}}%
\pgfpathclose%
\pgfusepath{fill}%
\end{pgfscope}%
\begin{pgfscope}%
\pgfpathrectangle{\pgfqpoint{0.765000in}{0.660000in}}{\pgfqpoint{4.620000in}{4.620000in}}%
\pgfusepath{clip}%
\pgfsetbuttcap%
\pgfsetroundjoin%
\definecolor{currentfill}{rgb}{1.000000,0.894118,0.788235}%
\pgfsetfillcolor{currentfill}%
\pgfsetlinewidth{0.000000pt}%
\definecolor{currentstroke}{rgb}{1.000000,0.894118,0.788235}%
\pgfsetstrokecolor{currentstroke}%
\pgfsetdash{}{0pt}%
\pgfpathmoveto{\pgfqpoint{3.113634in}{3.679402in}}%
\pgfpathlineto{\pgfqpoint{3.113634in}{3.825177in}}%
\pgfpathlineto{\pgfqpoint{3.113634in}{11.255488in}}%
\pgfpathlineto{\pgfqpoint{3.239879in}{11.036825in}}%
\pgfpathlineto{\pgfqpoint{3.113634in}{3.679402in}}%
\pgfpathclose%
\pgfusepath{fill}%
\end{pgfscope}%
\begin{pgfscope}%
\pgfpathrectangle{\pgfqpoint{0.765000in}{0.660000in}}{\pgfqpoint{4.620000in}{4.620000in}}%
\pgfusepath{clip}%
\pgfsetbuttcap%
\pgfsetroundjoin%
\definecolor{currentfill}{rgb}{1.000000,0.894118,0.788235}%
\pgfsetfillcolor{currentfill}%
\pgfsetlinewidth{0.000000pt}%
\definecolor{currentstroke}{rgb}{1.000000,0.894118,0.788235}%
\pgfsetstrokecolor{currentstroke}%
\pgfsetdash{}{0pt}%
\pgfpathmoveto{\pgfqpoint{3.239879in}{3.606514in}}%
\pgfpathlineto{\pgfqpoint{3.239879in}{3.752289in}}%
\pgfpathlineto{\pgfqpoint{3.239879in}{11.182600in}}%
\pgfpathlineto{\pgfqpoint{3.113634in}{11.109713in}}%
\pgfpathlineto{\pgfqpoint{3.239879in}{3.606514in}}%
\pgfpathclose%
\pgfusepath{fill}%
\end{pgfscope}%
\begin{pgfscope}%
\pgfpathrectangle{\pgfqpoint{0.765000in}{0.660000in}}{\pgfqpoint{4.620000in}{4.620000in}}%
\pgfusepath{clip}%
\pgfsetbuttcap%
\pgfsetroundjoin%
\definecolor{currentfill}{rgb}{1.000000,0.894118,0.788235}%
\pgfsetfillcolor{currentfill}%
\pgfsetlinewidth{0.000000pt}%
\definecolor{currentstroke}{rgb}{1.000000,0.894118,0.788235}%
\pgfsetstrokecolor{currentstroke}%
\pgfsetdash{}{0pt}%
\pgfpathmoveto{\pgfqpoint{3.113634in}{3.533627in}}%
\pgfpathlineto{\pgfqpoint{3.239879in}{3.606514in}}%
\pgfpathlineto{\pgfqpoint{3.239879in}{11.036825in}}%
\pgfpathlineto{\pgfqpoint{3.113634in}{10.963937in}}%
\pgfpathlineto{\pgfqpoint{3.113634in}{3.533627in}}%
\pgfpathclose%
\pgfusepath{fill}%
\end{pgfscope}%
\begin{pgfscope}%
\pgfpathrectangle{\pgfqpoint{0.765000in}{0.660000in}}{\pgfqpoint{4.620000in}{4.620000in}}%
\pgfusepath{clip}%
\pgfsetbuttcap%
\pgfsetroundjoin%
\definecolor{currentfill}{rgb}{1.000000,0.894118,0.788235}%
\pgfsetfillcolor{currentfill}%
\pgfsetlinewidth{0.000000pt}%
\definecolor{currentstroke}{rgb}{1.000000,0.894118,0.788235}%
\pgfsetstrokecolor{currentstroke}%
\pgfsetdash{}{0pt}%
\pgfpathmoveto{\pgfqpoint{3.239879in}{3.752289in}}%
\pgfpathlineto{\pgfqpoint{3.113634in}{3.825177in}}%
\pgfpathlineto{\pgfqpoint{3.113634in}{11.255488in}}%
\pgfpathlineto{\pgfqpoint{3.239879in}{11.182600in}}%
\pgfpathlineto{\pgfqpoint{3.239879in}{3.752289in}}%
\pgfpathclose%
\pgfusepath{fill}%
\end{pgfscope}%
\begin{pgfscope}%
\pgfpathrectangle{\pgfqpoint{0.765000in}{0.660000in}}{\pgfqpoint{4.620000in}{4.620000in}}%
\pgfusepath{clip}%
\pgfsetbuttcap%
\pgfsetroundjoin%
\definecolor{currentfill}{rgb}{1.000000,0.894118,0.788235}%
\pgfsetfillcolor{currentfill}%
\pgfsetlinewidth{0.000000pt}%
\definecolor{currentstroke}{rgb}{1.000000,0.894118,0.788235}%
\pgfsetstrokecolor{currentstroke}%
\pgfsetdash{}{0pt}%
\pgfpathmoveto{\pgfqpoint{2.987389in}{3.606514in}}%
\pgfpathlineto{\pgfqpoint{3.113634in}{3.533627in}}%
\pgfpathlineto{\pgfqpoint{3.113634in}{10.963937in}}%
\pgfpathlineto{\pgfqpoint{2.987389in}{11.036825in}}%
\pgfpathlineto{\pgfqpoint{2.987389in}{3.606514in}}%
\pgfpathclose%
\pgfusepath{fill}%
\end{pgfscope}%
\begin{pgfscope}%
\pgfpathrectangle{\pgfqpoint{0.765000in}{0.660000in}}{\pgfqpoint{4.620000in}{4.620000in}}%
\pgfusepath{clip}%
\pgfsetbuttcap%
\pgfsetroundjoin%
\definecolor{currentfill}{rgb}{1.000000,0.894118,0.788235}%
\pgfsetfillcolor{currentfill}%
\pgfsetlinewidth{0.000000pt}%
\definecolor{currentstroke}{rgb}{1.000000,0.894118,0.788235}%
\pgfsetstrokecolor{currentstroke}%
\pgfsetdash{}{0pt}%
\pgfpathmoveto{\pgfqpoint{3.113634in}{3.825177in}}%
\pgfpathlineto{\pgfqpoint{2.987389in}{3.752289in}}%
\pgfpathlineto{\pgfqpoint{2.987389in}{11.182600in}}%
\pgfpathlineto{\pgfqpoint{3.113634in}{11.255488in}}%
\pgfpathlineto{\pgfqpoint{3.113634in}{3.825177in}}%
\pgfpathclose%
\pgfusepath{fill}%
\end{pgfscope}%
\begin{pgfscope}%
\pgfpathrectangle{\pgfqpoint{0.765000in}{0.660000in}}{\pgfqpoint{4.620000in}{4.620000in}}%
\pgfusepath{clip}%
\pgfsetbuttcap%
\pgfsetroundjoin%
\definecolor{currentfill}{rgb}{1.000000,0.894118,0.788235}%
\pgfsetfillcolor{currentfill}%
\pgfsetlinewidth{0.000000pt}%
\definecolor{currentstroke}{rgb}{1.000000,0.894118,0.788235}%
\pgfsetstrokecolor{currentstroke}%
\pgfsetdash{}{0pt}%
\pgfpathmoveto{\pgfqpoint{2.987389in}{3.606514in}}%
\pgfpathlineto{\pgfqpoint{2.987389in}{3.752289in}}%
\pgfpathlineto{\pgfqpoint{2.987389in}{11.182600in}}%
\pgfpathlineto{\pgfqpoint{3.113634in}{10.963937in}}%
\pgfpathlineto{\pgfqpoint{2.987389in}{3.606514in}}%
\pgfpathclose%
\pgfusepath{fill}%
\end{pgfscope}%
\begin{pgfscope}%
\pgfpathrectangle{\pgfqpoint{0.765000in}{0.660000in}}{\pgfqpoint{4.620000in}{4.620000in}}%
\pgfusepath{clip}%
\pgfsetbuttcap%
\pgfsetroundjoin%
\definecolor{currentfill}{rgb}{1.000000,0.894118,0.788235}%
\pgfsetfillcolor{currentfill}%
\pgfsetlinewidth{0.000000pt}%
\definecolor{currentstroke}{rgb}{1.000000,0.894118,0.788235}%
\pgfsetstrokecolor{currentstroke}%
\pgfsetdash{}{0pt}%
\pgfpathmoveto{\pgfqpoint{3.113634in}{3.533627in}}%
\pgfpathlineto{\pgfqpoint{3.113634in}{3.679402in}}%
\pgfpathlineto{\pgfqpoint{3.113634in}{11.109713in}}%
\pgfpathlineto{\pgfqpoint{2.987389in}{11.036825in}}%
\pgfpathlineto{\pgfqpoint{3.113634in}{3.533627in}}%
\pgfpathclose%
\pgfusepath{fill}%
\end{pgfscope}%
\begin{pgfscope}%
\pgfpathrectangle{\pgfqpoint{0.765000in}{0.660000in}}{\pgfqpoint{4.620000in}{4.620000in}}%
\pgfusepath{clip}%
\pgfsetbuttcap%
\pgfsetroundjoin%
\definecolor{currentfill}{rgb}{1.000000,0.894118,0.788235}%
\pgfsetfillcolor{currentfill}%
\pgfsetlinewidth{0.000000pt}%
\definecolor{currentstroke}{rgb}{1.000000,0.894118,0.788235}%
\pgfsetstrokecolor{currentstroke}%
\pgfsetdash{}{0pt}%
\pgfpathmoveto{\pgfqpoint{2.987389in}{3.752289in}}%
\pgfpathlineto{\pgfqpoint{3.113634in}{3.679402in}}%
\pgfpathlineto{\pgfqpoint{3.113634in}{11.109713in}}%
\pgfpathlineto{\pgfqpoint{2.987389in}{11.182600in}}%
\pgfpathlineto{\pgfqpoint{2.987389in}{3.752289in}}%
\pgfpathclose%
\pgfusepath{fill}%
\end{pgfscope}%
\begin{pgfscope}%
\pgfpathrectangle{\pgfqpoint{0.765000in}{0.660000in}}{\pgfqpoint{4.620000in}{4.620000in}}%
\pgfusepath{clip}%
\pgfsetbuttcap%
\pgfsetroundjoin%
\definecolor{currentfill}{rgb}{1.000000,0.894118,0.788235}%
\pgfsetfillcolor{currentfill}%
\pgfsetlinewidth{0.000000pt}%
\definecolor{currentstroke}{rgb}{1.000000,0.894118,0.788235}%
\pgfsetstrokecolor{currentstroke}%
\pgfsetdash{}{0pt}%
\pgfpathmoveto{\pgfqpoint{3.113634in}{3.679402in}}%
\pgfpathlineto{\pgfqpoint{3.239879in}{3.752289in}}%
\pgfpathlineto{\pgfqpoint{3.239879in}{11.182600in}}%
\pgfpathlineto{\pgfqpoint{3.113634in}{11.109713in}}%
\pgfpathlineto{\pgfqpoint{3.113634in}{3.679402in}}%
\pgfpathclose%
\pgfusepath{fill}%
\end{pgfscope}%
\begin{pgfscope}%
\pgfpathrectangle{\pgfqpoint{0.765000in}{0.660000in}}{\pgfqpoint{4.620000in}{4.620000in}}%
\pgfusepath{clip}%
\pgfsetrectcap%
\pgfsetroundjoin%
\pgfsetlinewidth{1.204500pt}%
\definecolor{currentstroke}{rgb}{1.000000,0.576471,0.309804}%
\pgfsetstrokecolor{currentstroke}%
\pgfsetdash{}{0pt}%
\pgfpathmoveto{\pgfqpoint{4.723463in}{3.931007in}}%
\pgfusepath{stroke}%
\end{pgfscope}%
\begin{pgfscope}%
\pgfpathrectangle{\pgfqpoint{0.765000in}{0.660000in}}{\pgfqpoint{4.620000in}{4.620000in}}%
\pgfusepath{clip}%
\pgfsetbuttcap%
\pgfsetroundjoin%
\definecolor{currentfill}{rgb}{1.000000,0.576471,0.309804}%
\pgfsetfillcolor{currentfill}%
\pgfsetlinewidth{1.003750pt}%
\definecolor{currentstroke}{rgb}{1.000000,0.576471,0.309804}%
\pgfsetstrokecolor{currentstroke}%
\pgfsetdash{}{0pt}%
\pgfsys@defobject{currentmarker}{\pgfqpoint{-0.033333in}{-0.033333in}}{\pgfqpoint{0.033333in}{0.033333in}}{%
\pgfpathmoveto{\pgfqpoint{0.000000in}{-0.033333in}}%
\pgfpathcurveto{\pgfqpoint{0.008840in}{-0.033333in}}{\pgfqpoint{0.017319in}{-0.029821in}}{\pgfqpoint{0.023570in}{-0.023570in}}%
\pgfpathcurveto{\pgfqpoint{0.029821in}{-0.017319in}}{\pgfqpoint{0.033333in}{-0.008840in}}{\pgfqpoint{0.033333in}{0.000000in}}%
\pgfpathcurveto{\pgfqpoint{0.033333in}{0.008840in}}{\pgfqpoint{0.029821in}{0.017319in}}{\pgfqpoint{0.023570in}{0.023570in}}%
\pgfpathcurveto{\pgfqpoint{0.017319in}{0.029821in}}{\pgfqpoint{0.008840in}{0.033333in}}{\pgfqpoint{0.000000in}{0.033333in}}%
\pgfpathcurveto{\pgfqpoint{-0.008840in}{0.033333in}}{\pgfqpoint{-0.017319in}{0.029821in}}{\pgfqpoint{-0.023570in}{0.023570in}}%
\pgfpathcurveto{\pgfqpoint{-0.029821in}{0.017319in}}{\pgfqpoint{-0.033333in}{0.008840in}}{\pgfqpoint{-0.033333in}{0.000000in}}%
\pgfpathcurveto{\pgfqpoint{-0.033333in}{-0.008840in}}{\pgfqpoint{-0.029821in}{-0.017319in}}{\pgfqpoint{-0.023570in}{-0.023570in}}%
\pgfpathcurveto{\pgfqpoint{-0.017319in}{-0.029821in}}{\pgfqpoint{-0.008840in}{-0.033333in}}{\pgfqpoint{0.000000in}{-0.033333in}}%
\pgfpathlineto{\pgfqpoint{0.000000in}{-0.033333in}}%
\pgfpathclose%
\pgfusepath{stroke,fill}%
}%
\begin{pgfscope}%
\pgfsys@transformshift{4.723463in}{3.931007in}%
\pgfsys@useobject{currentmarker}{}%
\end{pgfscope}%
\end{pgfscope}%
\begin{pgfscope}%
\pgfpathrectangle{\pgfqpoint{0.765000in}{0.660000in}}{\pgfqpoint{4.620000in}{4.620000in}}%
\pgfusepath{clip}%
\pgfsetrectcap%
\pgfsetroundjoin%
\pgfsetlinewidth{1.204500pt}%
\definecolor{currentstroke}{rgb}{1.000000,0.576471,0.309804}%
\pgfsetstrokecolor{currentstroke}%
\pgfsetdash{}{0pt}%
\pgfpathmoveto{\pgfqpoint{5.041467in}{3.932349in}}%
\pgfusepath{stroke}%
\end{pgfscope}%
\begin{pgfscope}%
\pgfpathrectangle{\pgfqpoint{0.765000in}{0.660000in}}{\pgfqpoint{4.620000in}{4.620000in}}%
\pgfusepath{clip}%
\pgfsetbuttcap%
\pgfsetroundjoin%
\definecolor{currentfill}{rgb}{1.000000,0.576471,0.309804}%
\pgfsetfillcolor{currentfill}%
\pgfsetlinewidth{1.003750pt}%
\definecolor{currentstroke}{rgb}{1.000000,0.576471,0.309804}%
\pgfsetstrokecolor{currentstroke}%
\pgfsetdash{}{0pt}%
\pgfsys@defobject{currentmarker}{\pgfqpoint{-0.033333in}{-0.033333in}}{\pgfqpoint{0.033333in}{0.033333in}}{%
\pgfpathmoveto{\pgfqpoint{0.000000in}{-0.033333in}}%
\pgfpathcurveto{\pgfqpoint{0.008840in}{-0.033333in}}{\pgfqpoint{0.017319in}{-0.029821in}}{\pgfqpoint{0.023570in}{-0.023570in}}%
\pgfpathcurveto{\pgfqpoint{0.029821in}{-0.017319in}}{\pgfqpoint{0.033333in}{-0.008840in}}{\pgfqpoint{0.033333in}{0.000000in}}%
\pgfpathcurveto{\pgfqpoint{0.033333in}{0.008840in}}{\pgfqpoint{0.029821in}{0.017319in}}{\pgfqpoint{0.023570in}{0.023570in}}%
\pgfpathcurveto{\pgfqpoint{0.017319in}{0.029821in}}{\pgfqpoint{0.008840in}{0.033333in}}{\pgfqpoint{0.000000in}{0.033333in}}%
\pgfpathcurveto{\pgfqpoint{-0.008840in}{0.033333in}}{\pgfqpoint{-0.017319in}{0.029821in}}{\pgfqpoint{-0.023570in}{0.023570in}}%
\pgfpathcurveto{\pgfqpoint{-0.029821in}{0.017319in}}{\pgfqpoint{-0.033333in}{0.008840in}}{\pgfqpoint{-0.033333in}{0.000000in}}%
\pgfpathcurveto{\pgfqpoint{-0.033333in}{-0.008840in}}{\pgfqpoint{-0.029821in}{-0.017319in}}{\pgfqpoint{-0.023570in}{-0.023570in}}%
\pgfpathcurveto{\pgfqpoint{-0.017319in}{-0.029821in}}{\pgfqpoint{-0.008840in}{-0.033333in}}{\pgfqpoint{0.000000in}{-0.033333in}}%
\pgfpathlineto{\pgfqpoint{0.000000in}{-0.033333in}}%
\pgfpathclose%
\pgfusepath{stroke,fill}%
}%
\begin{pgfscope}%
\pgfsys@transformshift{5.041467in}{3.932349in}%
\pgfsys@useobject{currentmarker}{}%
\end{pgfscope}%
\end{pgfscope}%
\begin{pgfscope}%
\pgfpathrectangle{\pgfqpoint{0.765000in}{0.660000in}}{\pgfqpoint{4.620000in}{4.620000in}}%
\pgfusepath{clip}%
\pgfsetrectcap%
\pgfsetroundjoin%
\pgfsetlinewidth{1.204500pt}%
\definecolor{currentstroke}{rgb}{1.000000,0.576471,0.309804}%
\pgfsetstrokecolor{currentstroke}%
\pgfsetdash{}{0pt}%
\pgfpathmoveto{\pgfqpoint{4.612542in}{4.527326in}}%
\pgfusepath{stroke}%
\end{pgfscope}%
\begin{pgfscope}%
\pgfpathrectangle{\pgfqpoint{0.765000in}{0.660000in}}{\pgfqpoint{4.620000in}{4.620000in}}%
\pgfusepath{clip}%
\pgfsetbuttcap%
\pgfsetroundjoin%
\definecolor{currentfill}{rgb}{1.000000,0.576471,0.309804}%
\pgfsetfillcolor{currentfill}%
\pgfsetlinewidth{1.003750pt}%
\definecolor{currentstroke}{rgb}{1.000000,0.576471,0.309804}%
\pgfsetstrokecolor{currentstroke}%
\pgfsetdash{}{0pt}%
\pgfsys@defobject{currentmarker}{\pgfqpoint{-0.033333in}{-0.033333in}}{\pgfqpoint{0.033333in}{0.033333in}}{%
\pgfpathmoveto{\pgfqpoint{0.000000in}{-0.033333in}}%
\pgfpathcurveto{\pgfqpoint{0.008840in}{-0.033333in}}{\pgfqpoint{0.017319in}{-0.029821in}}{\pgfqpoint{0.023570in}{-0.023570in}}%
\pgfpathcurveto{\pgfqpoint{0.029821in}{-0.017319in}}{\pgfqpoint{0.033333in}{-0.008840in}}{\pgfqpoint{0.033333in}{0.000000in}}%
\pgfpathcurveto{\pgfqpoint{0.033333in}{0.008840in}}{\pgfqpoint{0.029821in}{0.017319in}}{\pgfqpoint{0.023570in}{0.023570in}}%
\pgfpathcurveto{\pgfqpoint{0.017319in}{0.029821in}}{\pgfqpoint{0.008840in}{0.033333in}}{\pgfqpoint{0.000000in}{0.033333in}}%
\pgfpathcurveto{\pgfqpoint{-0.008840in}{0.033333in}}{\pgfqpoint{-0.017319in}{0.029821in}}{\pgfqpoint{-0.023570in}{0.023570in}}%
\pgfpathcurveto{\pgfqpoint{-0.029821in}{0.017319in}}{\pgfqpoint{-0.033333in}{0.008840in}}{\pgfqpoint{-0.033333in}{0.000000in}}%
\pgfpathcurveto{\pgfqpoint{-0.033333in}{-0.008840in}}{\pgfqpoint{-0.029821in}{-0.017319in}}{\pgfqpoint{-0.023570in}{-0.023570in}}%
\pgfpathcurveto{\pgfqpoint{-0.017319in}{-0.029821in}}{\pgfqpoint{-0.008840in}{-0.033333in}}{\pgfqpoint{0.000000in}{-0.033333in}}%
\pgfpathlineto{\pgfqpoint{0.000000in}{-0.033333in}}%
\pgfpathclose%
\pgfusepath{stroke,fill}%
}%
\begin{pgfscope}%
\pgfsys@transformshift{4.612542in}{4.527326in}%
\pgfsys@useobject{currentmarker}{}%
\end{pgfscope}%
\end{pgfscope}%
\begin{pgfscope}%
\pgfpathrectangle{\pgfqpoint{0.765000in}{0.660000in}}{\pgfqpoint{4.620000in}{4.620000in}}%
\pgfusepath{clip}%
\pgfsetrectcap%
\pgfsetroundjoin%
\pgfsetlinewidth{1.204500pt}%
\definecolor{currentstroke}{rgb}{1.000000,0.576471,0.309804}%
\pgfsetstrokecolor{currentstroke}%
\pgfsetdash{}{0pt}%
\pgfpathmoveto{\pgfqpoint{4.768178in}{3.975950in}}%
\pgfusepath{stroke}%
\end{pgfscope}%
\begin{pgfscope}%
\pgfpathrectangle{\pgfqpoint{0.765000in}{0.660000in}}{\pgfqpoint{4.620000in}{4.620000in}}%
\pgfusepath{clip}%
\pgfsetbuttcap%
\pgfsetroundjoin%
\definecolor{currentfill}{rgb}{1.000000,0.576471,0.309804}%
\pgfsetfillcolor{currentfill}%
\pgfsetlinewidth{1.003750pt}%
\definecolor{currentstroke}{rgb}{1.000000,0.576471,0.309804}%
\pgfsetstrokecolor{currentstroke}%
\pgfsetdash{}{0pt}%
\pgfsys@defobject{currentmarker}{\pgfqpoint{-0.033333in}{-0.033333in}}{\pgfqpoint{0.033333in}{0.033333in}}{%
\pgfpathmoveto{\pgfqpoint{0.000000in}{-0.033333in}}%
\pgfpathcurveto{\pgfqpoint{0.008840in}{-0.033333in}}{\pgfqpoint{0.017319in}{-0.029821in}}{\pgfqpoint{0.023570in}{-0.023570in}}%
\pgfpathcurveto{\pgfqpoint{0.029821in}{-0.017319in}}{\pgfqpoint{0.033333in}{-0.008840in}}{\pgfqpoint{0.033333in}{0.000000in}}%
\pgfpathcurveto{\pgfqpoint{0.033333in}{0.008840in}}{\pgfqpoint{0.029821in}{0.017319in}}{\pgfqpoint{0.023570in}{0.023570in}}%
\pgfpathcurveto{\pgfqpoint{0.017319in}{0.029821in}}{\pgfqpoint{0.008840in}{0.033333in}}{\pgfqpoint{0.000000in}{0.033333in}}%
\pgfpathcurveto{\pgfqpoint{-0.008840in}{0.033333in}}{\pgfqpoint{-0.017319in}{0.029821in}}{\pgfqpoint{-0.023570in}{0.023570in}}%
\pgfpathcurveto{\pgfqpoint{-0.029821in}{0.017319in}}{\pgfqpoint{-0.033333in}{0.008840in}}{\pgfqpoint{-0.033333in}{0.000000in}}%
\pgfpathcurveto{\pgfqpoint{-0.033333in}{-0.008840in}}{\pgfqpoint{-0.029821in}{-0.017319in}}{\pgfqpoint{-0.023570in}{-0.023570in}}%
\pgfpathcurveto{\pgfqpoint{-0.017319in}{-0.029821in}}{\pgfqpoint{-0.008840in}{-0.033333in}}{\pgfqpoint{0.000000in}{-0.033333in}}%
\pgfpathlineto{\pgfqpoint{0.000000in}{-0.033333in}}%
\pgfpathclose%
\pgfusepath{stroke,fill}%
}%
\begin{pgfscope}%
\pgfsys@transformshift{4.768178in}{3.975950in}%
\pgfsys@useobject{currentmarker}{}%
\end{pgfscope}%
\end{pgfscope}%
\begin{pgfscope}%
\pgfpathrectangle{\pgfqpoint{0.765000in}{0.660000in}}{\pgfqpoint{4.620000in}{4.620000in}}%
\pgfusepath{clip}%
\pgfsetrectcap%
\pgfsetroundjoin%
\pgfsetlinewidth{1.204500pt}%
\definecolor{currentstroke}{rgb}{1.000000,0.576471,0.309804}%
\pgfsetstrokecolor{currentstroke}%
\pgfsetdash{}{0pt}%
\pgfpathmoveto{\pgfqpoint{4.762037in}{3.804180in}}%
\pgfusepath{stroke}%
\end{pgfscope}%
\begin{pgfscope}%
\pgfpathrectangle{\pgfqpoint{0.765000in}{0.660000in}}{\pgfqpoint{4.620000in}{4.620000in}}%
\pgfusepath{clip}%
\pgfsetbuttcap%
\pgfsetroundjoin%
\definecolor{currentfill}{rgb}{1.000000,0.576471,0.309804}%
\pgfsetfillcolor{currentfill}%
\pgfsetlinewidth{1.003750pt}%
\definecolor{currentstroke}{rgb}{1.000000,0.576471,0.309804}%
\pgfsetstrokecolor{currentstroke}%
\pgfsetdash{}{0pt}%
\pgfsys@defobject{currentmarker}{\pgfqpoint{-0.033333in}{-0.033333in}}{\pgfqpoint{0.033333in}{0.033333in}}{%
\pgfpathmoveto{\pgfqpoint{0.000000in}{-0.033333in}}%
\pgfpathcurveto{\pgfqpoint{0.008840in}{-0.033333in}}{\pgfqpoint{0.017319in}{-0.029821in}}{\pgfqpoint{0.023570in}{-0.023570in}}%
\pgfpathcurveto{\pgfqpoint{0.029821in}{-0.017319in}}{\pgfqpoint{0.033333in}{-0.008840in}}{\pgfqpoint{0.033333in}{0.000000in}}%
\pgfpathcurveto{\pgfqpoint{0.033333in}{0.008840in}}{\pgfqpoint{0.029821in}{0.017319in}}{\pgfqpoint{0.023570in}{0.023570in}}%
\pgfpathcurveto{\pgfqpoint{0.017319in}{0.029821in}}{\pgfqpoint{0.008840in}{0.033333in}}{\pgfqpoint{0.000000in}{0.033333in}}%
\pgfpathcurveto{\pgfqpoint{-0.008840in}{0.033333in}}{\pgfqpoint{-0.017319in}{0.029821in}}{\pgfqpoint{-0.023570in}{0.023570in}}%
\pgfpathcurveto{\pgfqpoint{-0.029821in}{0.017319in}}{\pgfqpoint{-0.033333in}{0.008840in}}{\pgfqpoint{-0.033333in}{0.000000in}}%
\pgfpathcurveto{\pgfqpoint{-0.033333in}{-0.008840in}}{\pgfqpoint{-0.029821in}{-0.017319in}}{\pgfqpoint{-0.023570in}{-0.023570in}}%
\pgfpathcurveto{\pgfqpoint{-0.017319in}{-0.029821in}}{\pgfqpoint{-0.008840in}{-0.033333in}}{\pgfqpoint{0.000000in}{-0.033333in}}%
\pgfpathlineto{\pgfqpoint{0.000000in}{-0.033333in}}%
\pgfpathclose%
\pgfusepath{stroke,fill}%
}%
\begin{pgfscope}%
\pgfsys@transformshift{4.762037in}{3.804180in}%
\pgfsys@useobject{currentmarker}{}%
\end{pgfscope}%
\end{pgfscope}%
\begin{pgfscope}%
\pgfpathrectangle{\pgfqpoint{0.765000in}{0.660000in}}{\pgfqpoint{4.620000in}{4.620000in}}%
\pgfusepath{clip}%
\pgfsetrectcap%
\pgfsetroundjoin%
\pgfsetlinewidth{1.204500pt}%
\definecolor{currentstroke}{rgb}{1.000000,0.576471,0.309804}%
\pgfsetstrokecolor{currentstroke}%
\pgfsetdash{}{0pt}%
\pgfpathmoveto{\pgfqpoint{4.867712in}{4.135221in}}%
\pgfusepath{stroke}%
\end{pgfscope}%
\begin{pgfscope}%
\pgfpathrectangle{\pgfqpoint{0.765000in}{0.660000in}}{\pgfqpoint{4.620000in}{4.620000in}}%
\pgfusepath{clip}%
\pgfsetbuttcap%
\pgfsetroundjoin%
\definecolor{currentfill}{rgb}{1.000000,0.576471,0.309804}%
\pgfsetfillcolor{currentfill}%
\pgfsetlinewidth{1.003750pt}%
\definecolor{currentstroke}{rgb}{1.000000,0.576471,0.309804}%
\pgfsetstrokecolor{currentstroke}%
\pgfsetdash{}{0pt}%
\pgfsys@defobject{currentmarker}{\pgfqpoint{-0.033333in}{-0.033333in}}{\pgfqpoint{0.033333in}{0.033333in}}{%
\pgfpathmoveto{\pgfqpoint{0.000000in}{-0.033333in}}%
\pgfpathcurveto{\pgfqpoint{0.008840in}{-0.033333in}}{\pgfqpoint{0.017319in}{-0.029821in}}{\pgfqpoint{0.023570in}{-0.023570in}}%
\pgfpathcurveto{\pgfqpoint{0.029821in}{-0.017319in}}{\pgfqpoint{0.033333in}{-0.008840in}}{\pgfqpoint{0.033333in}{0.000000in}}%
\pgfpathcurveto{\pgfqpoint{0.033333in}{0.008840in}}{\pgfqpoint{0.029821in}{0.017319in}}{\pgfqpoint{0.023570in}{0.023570in}}%
\pgfpathcurveto{\pgfqpoint{0.017319in}{0.029821in}}{\pgfqpoint{0.008840in}{0.033333in}}{\pgfqpoint{0.000000in}{0.033333in}}%
\pgfpathcurveto{\pgfqpoint{-0.008840in}{0.033333in}}{\pgfqpoint{-0.017319in}{0.029821in}}{\pgfqpoint{-0.023570in}{0.023570in}}%
\pgfpathcurveto{\pgfqpoint{-0.029821in}{0.017319in}}{\pgfqpoint{-0.033333in}{0.008840in}}{\pgfqpoint{-0.033333in}{0.000000in}}%
\pgfpathcurveto{\pgfqpoint{-0.033333in}{-0.008840in}}{\pgfqpoint{-0.029821in}{-0.017319in}}{\pgfqpoint{-0.023570in}{-0.023570in}}%
\pgfpathcurveto{\pgfqpoint{-0.017319in}{-0.029821in}}{\pgfqpoint{-0.008840in}{-0.033333in}}{\pgfqpoint{0.000000in}{-0.033333in}}%
\pgfpathlineto{\pgfqpoint{0.000000in}{-0.033333in}}%
\pgfpathclose%
\pgfusepath{stroke,fill}%
}%
\begin{pgfscope}%
\pgfsys@transformshift{4.867712in}{4.135221in}%
\pgfsys@useobject{currentmarker}{}%
\end{pgfscope}%
\end{pgfscope}%
\begin{pgfscope}%
\pgfpathrectangle{\pgfqpoint{0.765000in}{0.660000in}}{\pgfqpoint{4.620000in}{4.620000in}}%
\pgfusepath{clip}%
\pgfsetrectcap%
\pgfsetroundjoin%
\pgfsetlinewidth{1.204500pt}%
\definecolor{currentstroke}{rgb}{1.000000,0.576471,0.309804}%
\pgfsetstrokecolor{currentstroke}%
\pgfsetdash{}{0pt}%
\pgfpathmoveto{\pgfqpoint{5.053288in}{3.672875in}}%
\pgfusepath{stroke}%
\end{pgfscope}%
\begin{pgfscope}%
\pgfpathrectangle{\pgfqpoint{0.765000in}{0.660000in}}{\pgfqpoint{4.620000in}{4.620000in}}%
\pgfusepath{clip}%
\pgfsetbuttcap%
\pgfsetroundjoin%
\definecolor{currentfill}{rgb}{1.000000,0.576471,0.309804}%
\pgfsetfillcolor{currentfill}%
\pgfsetlinewidth{1.003750pt}%
\definecolor{currentstroke}{rgb}{1.000000,0.576471,0.309804}%
\pgfsetstrokecolor{currentstroke}%
\pgfsetdash{}{0pt}%
\pgfsys@defobject{currentmarker}{\pgfqpoint{-0.033333in}{-0.033333in}}{\pgfqpoint{0.033333in}{0.033333in}}{%
\pgfpathmoveto{\pgfqpoint{0.000000in}{-0.033333in}}%
\pgfpathcurveto{\pgfqpoint{0.008840in}{-0.033333in}}{\pgfqpoint{0.017319in}{-0.029821in}}{\pgfqpoint{0.023570in}{-0.023570in}}%
\pgfpathcurveto{\pgfqpoint{0.029821in}{-0.017319in}}{\pgfqpoint{0.033333in}{-0.008840in}}{\pgfqpoint{0.033333in}{0.000000in}}%
\pgfpathcurveto{\pgfqpoint{0.033333in}{0.008840in}}{\pgfqpoint{0.029821in}{0.017319in}}{\pgfqpoint{0.023570in}{0.023570in}}%
\pgfpathcurveto{\pgfqpoint{0.017319in}{0.029821in}}{\pgfqpoint{0.008840in}{0.033333in}}{\pgfqpoint{0.000000in}{0.033333in}}%
\pgfpathcurveto{\pgfqpoint{-0.008840in}{0.033333in}}{\pgfqpoint{-0.017319in}{0.029821in}}{\pgfqpoint{-0.023570in}{0.023570in}}%
\pgfpathcurveto{\pgfqpoint{-0.029821in}{0.017319in}}{\pgfqpoint{-0.033333in}{0.008840in}}{\pgfqpoint{-0.033333in}{0.000000in}}%
\pgfpathcurveto{\pgfqpoint{-0.033333in}{-0.008840in}}{\pgfqpoint{-0.029821in}{-0.017319in}}{\pgfqpoint{-0.023570in}{-0.023570in}}%
\pgfpathcurveto{\pgfqpoint{-0.017319in}{-0.029821in}}{\pgfqpoint{-0.008840in}{-0.033333in}}{\pgfqpoint{0.000000in}{-0.033333in}}%
\pgfpathlineto{\pgfqpoint{0.000000in}{-0.033333in}}%
\pgfpathclose%
\pgfusepath{stroke,fill}%
}%
\begin{pgfscope}%
\pgfsys@transformshift{5.053288in}{3.672875in}%
\pgfsys@useobject{currentmarker}{}%
\end{pgfscope}%
\end{pgfscope}%
\begin{pgfscope}%
\pgfpathrectangle{\pgfqpoint{0.765000in}{0.660000in}}{\pgfqpoint{4.620000in}{4.620000in}}%
\pgfusepath{clip}%
\pgfsetrectcap%
\pgfsetroundjoin%
\pgfsetlinewidth{1.204500pt}%
\definecolor{currentstroke}{rgb}{1.000000,0.576471,0.309804}%
\pgfsetstrokecolor{currentstroke}%
\pgfsetdash{}{0pt}%
\pgfpathmoveto{\pgfqpoint{4.396221in}{3.893902in}}%
\pgfusepath{stroke}%
\end{pgfscope}%
\begin{pgfscope}%
\pgfpathrectangle{\pgfqpoint{0.765000in}{0.660000in}}{\pgfqpoint{4.620000in}{4.620000in}}%
\pgfusepath{clip}%
\pgfsetbuttcap%
\pgfsetroundjoin%
\definecolor{currentfill}{rgb}{1.000000,0.576471,0.309804}%
\pgfsetfillcolor{currentfill}%
\pgfsetlinewidth{1.003750pt}%
\definecolor{currentstroke}{rgb}{1.000000,0.576471,0.309804}%
\pgfsetstrokecolor{currentstroke}%
\pgfsetdash{}{0pt}%
\pgfsys@defobject{currentmarker}{\pgfqpoint{-0.033333in}{-0.033333in}}{\pgfqpoint{0.033333in}{0.033333in}}{%
\pgfpathmoveto{\pgfqpoint{0.000000in}{-0.033333in}}%
\pgfpathcurveto{\pgfqpoint{0.008840in}{-0.033333in}}{\pgfqpoint{0.017319in}{-0.029821in}}{\pgfqpoint{0.023570in}{-0.023570in}}%
\pgfpathcurveto{\pgfqpoint{0.029821in}{-0.017319in}}{\pgfqpoint{0.033333in}{-0.008840in}}{\pgfqpoint{0.033333in}{0.000000in}}%
\pgfpathcurveto{\pgfqpoint{0.033333in}{0.008840in}}{\pgfqpoint{0.029821in}{0.017319in}}{\pgfqpoint{0.023570in}{0.023570in}}%
\pgfpathcurveto{\pgfqpoint{0.017319in}{0.029821in}}{\pgfqpoint{0.008840in}{0.033333in}}{\pgfqpoint{0.000000in}{0.033333in}}%
\pgfpathcurveto{\pgfqpoint{-0.008840in}{0.033333in}}{\pgfqpoint{-0.017319in}{0.029821in}}{\pgfqpoint{-0.023570in}{0.023570in}}%
\pgfpathcurveto{\pgfqpoint{-0.029821in}{0.017319in}}{\pgfqpoint{-0.033333in}{0.008840in}}{\pgfqpoint{-0.033333in}{0.000000in}}%
\pgfpathcurveto{\pgfqpoint{-0.033333in}{-0.008840in}}{\pgfqpoint{-0.029821in}{-0.017319in}}{\pgfqpoint{-0.023570in}{-0.023570in}}%
\pgfpathcurveto{\pgfqpoint{-0.017319in}{-0.029821in}}{\pgfqpoint{-0.008840in}{-0.033333in}}{\pgfqpoint{0.000000in}{-0.033333in}}%
\pgfpathlineto{\pgfqpoint{0.000000in}{-0.033333in}}%
\pgfpathclose%
\pgfusepath{stroke,fill}%
}%
\begin{pgfscope}%
\pgfsys@transformshift{4.396221in}{3.893902in}%
\pgfsys@useobject{currentmarker}{}%
\end{pgfscope}%
\end{pgfscope}%
\begin{pgfscope}%
\pgfpathrectangle{\pgfqpoint{0.765000in}{0.660000in}}{\pgfqpoint{4.620000in}{4.620000in}}%
\pgfusepath{clip}%
\pgfsetrectcap%
\pgfsetroundjoin%
\pgfsetlinewidth{1.204500pt}%
\definecolor{currentstroke}{rgb}{1.000000,0.576471,0.309804}%
\pgfsetstrokecolor{currentstroke}%
\pgfsetdash{}{0pt}%
\pgfpathmoveto{\pgfqpoint{4.447768in}{3.973619in}}%
\pgfusepath{stroke}%
\end{pgfscope}%
\begin{pgfscope}%
\pgfpathrectangle{\pgfqpoint{0.765000in}{0.660000in}}{\pgfqpoint{4.620000in}{4.620000in}}%
\pgfusepath{clip}%
\pgfsetbuttcap%
\pgfsetroundjoin%
\definecolor{currentfill}{rgb}{1.000000,0.576471,0.309804}%
\pgfsetfillcolor{currentfill}%
\pgfsetlinewidth{1.003750pt}%
\definecolor{currentstroke}{rgb}{1.000000,0.576471,0.309804}%
\pgfsetstrokecolor{currentstroke}%
\pgfsetdash{}{0pt}%
\pgfsys@defobject{currentmarker}{\pgfqpoint{-0.033333in}{-0.033333in}}{\pgfqpoint{0.033333in}{0.033333in}}{%
\pgfpathmoveto{\pgfqpoint{0.000000in}{-0.033333in}}%
\pgfpathcurveto{\pgfqpoint{0.008840in}{-0.033333in}}{\pgfqpoint{0.017319in}{-0.029821in}}{\pgfqpoint{0.023570in}{-0.023570in}}%
\pgfpathcurveto{\pgfqpoint{0.029821in}{-0.017319in}}{\pgfqpoint{0.033333in}{-0.008840in}}{\pgfqpoint{0.033333in}{0.000000in}}%
\pgfpathcurveto{\pgfqpoint{0.033333in}{0.008840in}}{\pgfqpoint{0.029821in}{0.017319in}}{\pgfqpoint{0.023570in}{0.023570in}}%
\pgfpathcurveto{\pgfqpoint{0.017319in}{0.029821in}}{\pgfqpoint{0.008840in}{0.033333in}}{\pgfqpoint{0.000000in}{0.033333in}}%
\pgfpathcurveto{\pgfqpoint{-0.008840in}{0.033333in}}{\pgfqpoint{-0.017319in}{0.029821in}}{\pgfqpoint{-0.023570in}{0.023570in}}%
\pgfpathcurveto{\pgfqpoint{-0.029821in}{0.017319in}}{\pgfqpoint{-0.033333in}{0.008840in}}{\pgfqpoint{-0.033333in}{0.000000in}}%
\pgfpathcurveto{\pgfqpoint{-0.033333in}{-0.008840in}}{\pgfqpoint{-0.029821in}{-0.017319in}}{\pgfqpoint{-0.023570in}{-0.023570in}}%
\pgfpathcurveto{\pgfqpoint{-0.017319in}{-0.029821in}}{\pgfqpoint{-0.008840in}{-0.033333in}}{\pgfqpoint{0.000000in}{-0.033333in}}%
\pgfpathlineto{\pgfqpoint{0.000000in}{-0.033333in}}%
\pgfpathclose%
\pgfusepath{stroke,fill}%
}%
\begin{pgfscope}%
\pgfsys@transformshift{4.447768in}{3.973619in}%
\pgfsys@useobject{currentmarker}{}%
\end{pgfscope}%
\end{pgfscope}%
\begin{pgfscope}%
\pgfpathrectangle{\pgfqpoint{0.765000in}{0.660000in}}{\pgfqpoint{4.620000in}{4.620000in}}%
\pgfusepath{clip}%
\pgfsetrectcap%
\pgfsetroundjoin%
\pgfsetlinewidth{1.204500pt}%
\definecolor{currentstroke}{rgb}{1.000000,0.576471,0.309804}%
\pgfsetstrokecolor{currentstroke}%
\pgfsetdash{}{0pt}%
\pgfpathmoveto{\pgfqpoint{4.748261in}{4.200310in}}%
\pgfusepath{stroke}%
\end{pgfscope}%
\begin{pgfscope}%
\pgfpathrectangle{\pgfqpoint{0.765000in}{0.660000in}}{\pgfqpoint{4.620000in}{4.620000in}}%
\pgfusepath{clip}%
\pgfsetbuttcap%
\pgfsetroundjoin%
\definecolor{currentfill}{rgb}{1.000000,0.576471,0.309804}%
\pgfsetfillcolor{currentfill}%
\pgfsetlinewidth{1.003750pt}%
\definecolor{currentstroke}{rgb}{1.000000,0.576471,0.309804}%
\pgfsetstrokecolor{currentstroke}%
\pgfsetdash{}{0pt}%
\pgfsys@defobject{currentmarker}{\pgfqpoint{-0.033333in}{-0.033333in}}{\pgfqpoint{0.033333in}{0.033333in}}{%
\pgfpathmoveto{\pgfqpoint{0.000000in}{-0.033333in}}%
\pgfpathcurveto{\pgfqpoint{0.008840in}{-0.033333in}}{\pgfqpoint{0.017319in}{-0.029821in}}{\pgfqpoint{0.023570in}{-0.023570in}}%
\pgfpathcurveto{\pgfqpoint{0.029821in}{-0.017319in}}{\pgfqpoint{0.033333in}{-0.008840in}}{\pgfqpoint{0.033333in}{0.000000in}}%
\pgfpathcurveto{\pgfqpoint{0.033333in}{0.008840in}}{\pgfqpoint{0.029821in}{0.017319in}}{\pgfqpoint{0.023570in}{0.023570in}}%
\pgfpathcurveto{\pgfqpoint{0.017319in}{0.029821in}}{\pgfqpoint{0.008840in}{0.033333in}}{\pgfqpoint{0.000000in}{0.033333in}}%
\pgfpathcurveto{\pgfqpoint{-0.008840in}{0.033333in}}{\pgfqpoint{-0.017319in}{0.029821in}}{\pgfqpoint{-0.023570in}{0.023570in}}%
\pgfpathcurveto{\pgfqpoint{-0.029821in}{0.017319in}}{\pgfqpoint{-0.033333in}{0.008840in}}{\pgfqpoint{-0.033333in}{0.000000in}}%
\pgfpathcurveto{\pgfqpoint{-0.033333in}{-0.008840in}}{\pgfqpoint{-0.029821in}{-0.017319in}}{\pgfqpoint{-0.023570in}{-0.023570in}}%
\pgfpathcurveto{\pgfqpoint{-0.017319in}{-0.029821in}}{\pgfqpoint{-0.008840in}{-0.033333in}}{\pgfqpoint{0.000000in}{-0.033333in}}%
\pgfpathlineto{\pgfqpoint{0.000000in}{-0.033333in}}%
\pgfpathclose%
\pgfusepath{stroke,fill}%
}%
\begin{pgfscope}%
\pgfsys@transformshift{4.748261in}{4.200310in}%
\pgfsys@useobject{currentmarker}{}%
\end{pgfscope}%
\end{pgfscope}%
\begin{pgfscope}%
\pgfpathrectangle{\pgfqpoint{0.765000in}{0.660000in}}{\pgfqpoint{4.620000in}{4.620000in}}%
\pgfusepath{clip}%
\pgfsetrectcap%
\pgfsetroundjoin%
\pgfsetlinewidth{1.204500pt}%
\definecolor{currentstroke}{rgb}{0.176471,0.192157,0.258824}%
\pgfsetstrokecolor{currentstroke}%
\pgfsetdash{}{0pt}%
\pgfpathmoveto{\pgfqpoint{1.554330in}{3.857843in}}%
\pgfusepath{stroke}%
\end{pgfscope}%
\begin{pgfscope}%
\pgfpathrectangle{\pgfqpoint{0.765000in}{0.660000in}}{\pgfqpoint{4.620000in}{4.620000in}}%
\pgfusepath{clip}%
\pgfsetbuttcap%
\pgfsetmiterjoin%
\definecolor{currentfill}{rgb}{0.176471,0.192157,0.258824}%
\pgfsetfillcolor{currentfill}%
\pgfsetlinewidth{1.003750pt}%
\definecolor{currentstroke}{rgb}{0.176471,0.192157,0.258824}%
\pgfsetstrokecolor{currentstroke}%
\pgfsetdash{}{0pt}%
\pgfsys@defobject{currentmarker}{\pgfqpoint{-0.033333in}{-0.033333in}}{\pgfqpoint{0.033333in}{0.033333in}}{%
\pgfpathmoveto{\pgfqpoint{-0.000000in}{-0.033333in}}%
\pgfpathlineto{\pgfqpoint{0.033333in}{0.033333in}}%
\pgfpathlineto{\pgfqpoint{-0.033333in}{0.033333in}}%
\pgfpathlineto{\pgfqpoint{-0.000000in}{-0.033333in}}%
\pgfpathclose%
\pgfusepath{stroke,fill}%
}%
\begin{pgfscope}%
\pgfsys@transformshift{1.554330in}{3.857843in}%
\pgfsys@useobject{currentmarker}{}%
\end{pgfscope}%
\end{pgfscope}%
\begin{pgfscope}%
\pgfpathrectangle{\pgfqpoint{0.765000in}{0.660000in}}{\pgfqpoint{4.620000in}{4.620000in}}%
\pgfusepath{clip}%
\pgfsetrectcap%
\pgfsetroundjoin%
\pgfsetlinewidth{1.204500pt}%
\definecolor{currentstroke}{rgb}{0.176471,0.192157,0.258824}%
\pgfsetstrokecolor{currentstroke}%
\pgfsetdash{}{0pt}%
\pgfpathmoveto{\pgfqpoint{1.346832in}{3.855108in}}%
\pgfusepath{stroke}%
\end{pgfscope}%
\begin{pgfscope}%
\pgfpathrectangle{\pgfqpoint{0.765000in}{0.660000in}}{\pgfqpoint{4.620000in}{4.620000in}}%
\pgfusepath{clip}%
\pgfsetbuttcap%
\pgfsetmiterjoin%
\definecolor{currentfill}{rgb}{0.176471,0.192157,0.258824}%
\pgfsetfillcolor{currentfill}%
\pgfsetlinewidth{1.003750pt}%
\definecolor{currentstroke}{rgb}{0.176471,0.192157,0.258824}%
\pgfsetstrokecolor{currentstroke}%
\pgfsetdash{}{0pt}%
\pgfsys@defobject{currentmarker}{\pgfqpoint{-0.033333in}{-0.033333in}}{\pgfqpoint{0.033333in}{0.033333in}}{%
\pgfpathmoveto{\pgfqpoint{-0.000000in}{-0.033333in}}%
\pgfpathlineto{\pgfqpoint{0.033333in}{0.033333in}}%
\pgfpathlineto{\pgfqpoint{-0.033333in}{0.033333in}}%
\pgfpathlineto{\pgfqpoint{-0.000000in}{-0.033333in}}%
\pgfpathclose%
\pgfusepath{stroke,fill}%
}%
\begin{pgfscope}%
\pgfsys@transformshift{1.346832in}{3.855108in}%
\pgfsys@useobject{currentmarker}{}%
\end{pgfscope}%
\end{pgfscope}%
\begin{pgfscope}%
\pgfpathrectangle{\pgfqpoint{0.765000in}{0.660000in}}{\pgfqpoint{4.620000in}{4.620000in}}%
\pgfusepath{clip}%
\pgfsetrectcap%
\pgfsetroundjoin%
\pgfsetlinewidth{1.204500pt}%
\definecolor{currentstroke}{rgb}{0.176471,0.192157,0.258824}%
\pgfsetstrokecolor{currentstroke}%
\pgfsetdash{}{0pt}%
\pgfpathmoveto{\pgfqpoint{1.196064in}{3.874491in}}%
\pgfusepath{stroke}%
\end{pgfscope}%
\begin{pgfscope}%
\pgfpathrectangle{\pgfqpoint{0.765000in}{0.660000in}}{\pgfqpoint{4.620000in}{4.620000in}}%
\pgfusepath{clip}%
\pgfsetbuttcap%
\pgfsetmiterjoin%
\definecolor{currentfill}{rgb}{0.176471,0.192157,0.258824}%
\pgfsetfillcolor{currentfill}%
\pgfsetlinewidth{1.003750pt}%
\definecolor{currentstroke}{rgb}{0.176471,0.192157,0.258824}%
\pgfsetstrokecolor{currentstroke}%
\pgfsetdash{}{0pt}%
\pgfsys@defobject{currentmarker}{\pgfqpoint{-0.033333in}{-0.033333in}}{\pgfqpoint{0.033333in}{0.033333in}}{%
\pgfpathmoveto{\pgfqpoint{-0.000000in}{-0.033333in}}%
\pgfpathlineto{\pgfqpoint{0.033333in}{0.033333in}}%
\pgfpathlineto{\pgfqpoint{-0.033333in}{0.033333in}}%
\pgfpathlineto{\pgfqpoint{-0.000000in}{-0.033333in}}%
\pgfpathclose%
\pgfusepath{stroke,fill}%
}%
\begin{pgfscope}%
\pgfsys@transformshift{1.196064in}{3.874491in}%
\pgfsys@useobject{currentmarker}{}%
\end{pgfscope}%
\end{pgfscope}%
\begin{pgfscope}%
\pgfpathrectangle{\pgfqpoint{0.765000in}{0.660000in}}{\pgfqpoint{4.620000in}{4.620000in}}%
\pgfusepath{clip}%
\pgfsetrectcap%
\pgfsetroundjoin%
\pgfsetlinewidth{1.204500pt}%
\definecolor{currentstroke}{rgb}{0.176471,0.192157,0.258824}%
\pgfsetstrokecolor{currentstroke}%
\pgfsetdash{}{0pt}%
\pgfpathmoveto{\pgfqpoint{1.945951in}{3.749926in}}%
\pgfusepath{stroke}%
\end{pgfscope}%
\begin{pgfscope}%
\pgfpathrectangle{\pgfqpoint{0.765000in}{0.660000in}}{\pgfqpoint{4.620000in}{4.620000in}}%
\pgfusepath{clip}%
\pgfsetbuttcap%
\pgfsetmiterjoin%
\definecolor{currentfill}{rgb}{0.176471,0.192157,0.258824}%
\pgfsetfillcolor{currentfill}%
\pgfsetlinewidth{1.003750pt}%
\definecolor{currentstroke}{rgb}{0.176471,0.192157,0.258824}%
\pgfsetstrokecolor{currentstroke}%
\pgfsetdash{}{0pt}%
\pgfsys@defobject{currentmarker}{\pgfqpoint{-0.033333in}{-0.033333in}}{\pgfqpoint{0.033333in}{0.033333in}}{%
\pgfpathmoveto{\pgfqpoint{-0.000000in}{-0.033333in}}%
\pgfpathlineto{\pgfqpoint{0.033333in}{0.033333in}}%
\pgfpathlineto{\pgfqpoint{-0.033333in}{0.033333in}}%
\pgfpathlineto{\pgfqpoint{-0.000000in}{-0.033333in}}%
\pgfpathclose%
\pgfusepath{stroke,fill}%
}%
\begin{pgfscope}%
\pgfsys@transformshift{1.945951in}{3.749926in}%
\pgfsys@useobject{currentmarker}{}%
\end{pgfscope}%
\end{pgfscope}%
\begin{pgfscope}%
\pgfpathrectangle{\pgfqpoint{0.765000in}{0.660000in}}{\pgfqpoint{4.620000in}{4.620000in}}%
\pgfusepath{clip}%
\pgfsetrectcap%
\pgfsetroundjoin%
\pgfsetlinewidth{1.204500pt}%
\definecolor{currentstroke}{rgb}{0.176471,0.192157,0.258824}%
\pgfsetstrokecolor{currentstroke}%
\pgfsetdash{}{0pt}%
\pgfpathmoveto{\pgfqpoint{1.672194in}{4.055735in}}%
\pgfusepath{stroke}%
\end{pgfscope}%
\begin{pgfscope}%
\pgfpathrectangle{\pgfqpoint{0.765000in}{0.660000in}}{\pgfqpoint{4.620000in}{4.620000in}}%
\pgfusepath{clip}%
\pgfsetbuttcap%
\pgfsetmiterjoin%
\definecolor{currentfill}{rgb}{0.176471,0.192157,0.258824}%
\pgfsetfillcolor{currentfill}%
\pgfsetlinewidth{1.003750pt}%
\definecolor{currentstroke}{rgb}{0.176471,0.192157,0.258824}%
\pgfsetstrokecolor{currentstroke}%
\pgfsetdash{}{0pt}%
\pgfsys@defobject{currentmarker}{\pgfqpoint{-0.033333in}{-0.033333in}}{\pgfqpoint{0.033333in}{0.033333in}}{%
\pgfpathmoveto{\pgfqpoint{-0.000000in}{-0.033333in}}%
\pgfpathlineto{\pgfqpoint{0.033333in}{0.033333in}}%
\pgfpathlineto{\pgfqpoint{-0.033333in}{0.033333in}}%
\pgfpathlineto{\pgfqpoint{-0.000000in}{-0.033333in}}%
\pgfpathclose%
\pgfusepath{stroke,fill}%
}%
\begin{pgfscope}%
\pgfsys@transformshift{1.672194in}{4.055735in}%
\pgfsys@useobject{currentmarker}{}%
\end{pgfscope}%
\end{pgfscope}%
\begin{pgfscope}%
\pgfpathrectangle{\pgfqpoint{0.765000in}{0.660000in}}{\pgfqpoint{4.620000in}{4.620000in}}%
\pgfusepath{clip}%
\pgfsetrectcap%
\pgfsetroundjoin%
\pgfsetlinewidth{1.204500pt}%
\definecolor{currentstroke}{rgb}{0.176471,0.192157,0.258824}%
\pgfsetstrokecolor{currentstroke}%
\pgfsetdash{}{0pt}%
\pgfpathmoveto{\pgfqpoint{1.924224in}{4.270711in}}%
\pgfusepath{stroke}%
\end{pgfscope}%
\begin{pgfscope}%
\pgfpathrectangle{\pgfqpoint{0.765000in}{0.660000in}}{\pgfqpoint{4.620000in}{4.620000in}}%
\pgfusepath{clip}%
\pgfsetbuttcap%
\pgfsetmiterjoin%
\definecolor{currentfill}{rgb}{0.176471,0.192157,0.258824}%
\pgfsetfillcolor{currentfill}%
\pgfsetlinewidth{1.003750pt}%
\definecolor{currentstroke}{rgb}{0.176471,0.192157,0.258824}%
\pgfsetstrokecolor{currentstroke}%
\pgfsetdash{}{0pt}%
\pgfsys@defobject{currentmarker}{\pgfqpoint{-0.033333in}{-0.033333in}}{\pgfqpoint{0.033333in}{0.033333in}}{%
\pgfpathmoveto{\pgfqpoint{-0.000000in}{-0.033333in}}%
\pgfpathlineto{\pgfqpoint{0.033333in}{0.033333in}}%
\pgfpathlineto{\pgfqpoint{-0.033333in}{0.033333in}}%
\pgfpathlineto{\pgfqpoint{-0.000000in}{-0.033333in}}%
\pgfpathclose%
\pgfusepath{stroke,fill}%
}%
\begin{pgfscope}%
\pgfsys@transformshift{1.924224in}{4.270711in}%
\pgfsys@useobject{currentmarker}{}%
\end{pgfscope}%
\end{pgfscope}%
\begin{pgfscope}%
\pgfpathrectangle{\pgfqpoint{0.765000in}{0.660000in}}{\pgfqpoint{4.620000in}{4.620000in}}%
\pgfusepath{clip}%
\pgfsetrectcap%
\pgfsetroundjoin%
\pgfsetlinewidth{1.204500pt}%
\definecolor{currentstroke}{rgb}{0.176471,0.192157,0.258824}%
\pgfsetstrokecolor{currentstroke}%
\pgfsetdash{}{0pt}%
\pgfpathmoveto{\pgfqpoint{1.459722in}{3.515956in}}%
\pgfusepath{stroke}%
\end{pgfscope}%
\begin{pgfscope}%
\pgfpathrectangle{\pgfqpoint{0.765000in}{0.660000in}}{\pgfqpoint{4.620000in}{4.620000in}}%
\pgfusepath{clip}%
\pgfsetbuttcap%
\pgfsetmiterjoin%
\definecolor{currentfill}{rgb}{0.176471,0.192157,0.258824}%
\pgfsetfillcolor{currentfill}%
\pgfsetlinewidth{1.003750pt}%
\definecolor{currentstroke}{rgb}{0.176471,0.192157,0.258824}%
\pgfsetstrokecolor{currentstroke}%
\pgfsetdash{}{0pt}%
\pgfsys@defobject{currentmarker}{\pgfqpoint{-0.033333in}{-0.033333in}}{\pgfqpoint{0.033333in}{0.033333in}}{%
\pgfpathmoveto{\pgfqpoint{-0.000000in}{-0.033333in}}%
\pgfpathlineto{\pgfqpoint{0.033333in}{0.033333in}}%
\pgfpathlineto{\pgfqpoint{-0.033333in}{0.033333in}}%
\pgfpathlineto{\pgfqpoint{-0.000000in}{-0.033333in}}%
\pgfpathclose%
\pgfusepath{stroke,fill}%
}%
\begin{pgfscope}%
\pgfsys@transformshift{1.459722in}{3.515956in}%
\pgfsys@useobject{currentmarker}{}%
\end{pgfscope}%
\end{pgfscope}%
\begin{pgfscope}%
\pgfpathrectangle{\pgfqpoint{0.765000in}{0.660000in}}{\pgfqpoint{4.620000in}{4.620000in}}%
\pgfusepath{clip}%
\pgfsetrectcap%
\pgfsetroundjoin%
\pgfsetlinewidth{1.204500pt}%
\definecolor{currentstroke}{rgb}{0.176471,0.192157,0.258824}%
\pgfsetstrokecolor{currentstroke}%
\pgfsetdash{}{0pt}%
\pgfpathmoveto{\pgfqpoint{1.555630in}{3.927537in}}%
\pgfusepath{stroke}%
\end{pgfscope}%
\begin{pgfscope}%
\pgfpathrectangle{\pgfqpoint{0.765000in}{0.660000in}}{\pgfqpoint{4.620000in}{4.620000in}}%
\pgfusepath{clip}%
\pgfsetbuttcap%
\pgfsetmiterjoin%
\definecolor{currentfill}{rgb}{0.176471,0.192157,0.258824}%
\pgfsetfillcolor{currentfill}%
\pgfsetlinewidth{1.003750pt}%
\definecolor{currentstroke}{rgb}{0.176471,0.192157,0.258824}%
\pgfsetstrokecolor{currentstroke}%
\pgfsetdash{}{0pt}%
\pgfsys@defobject{currentmarker}{\pgfqpoint{-0.033333in}{-0.033333in}}{\pgfqpoint{0.033333in}{0.033333in}}{%
\pgfpathmoveto{\pgfqpoint{-0.000000in}{-0.033333in}}%
\pgfpathlineto{\pgfqpoint{0.033333in}{0.033333in}}%
\pgfpathlineto{\pgfqpoint{-0.033333in}{0.033333in}}%
\pgfpathlineto{\pgfqpoint{-0.000000in}{-0.033333in}}%
\pgfpathclose%
\pgfusepath{stroke,fill}%
}%
\begin{pgfscope}%
\pgfsys@transformshift{1.555630in}{3.927537in}%
\pgfsys@useobject{currentmarker}{}%
\end{pgfscope}%
\end{pgfscope}%
\begin{pgfscope}%
\pgfpathrectangle{\pgfqpoint{0.765000in}{0.660000in}}{\pgfqpoint{4.620000in}{4.620000in}}%
\pgfusepath{clip}%
\pgfsetrectcap%
\pgfsetroundjoin%
\pgfsetlinewidth{1.204500pt}%
\definecolor{currentstroke}{rgb}{0.176471,0.192157,0.258824}%
\pgfsetstrokecolor{currentstroke}%
\pgfsetdash{}{0pt}%
\pgfpathmoveto{\pgfqpoint{1.730617in}{3.774927in}}%
\pgfusepath{stroke}%
\end{pgfscope}%
\begin{pgfscope}%
\pgfpathrectangle{\pgfqpoint{0.765000in}{0.660000in}}{\pgfqpoint{4.620000in}{4.620000in}}%
\pgfusepath{clip}%
\pgfsetbuttcap%
\pgfsetmiterjoin%
\definecolor{currentfill}{rgb}{0.176471,0.192157,0.258824}%
\pgfsetfillcolor{currentfill}%
\pgfsetlinewidth{1.003750pt}%
\definecolor{currentstroke}{rgb}{0.176471,0.192157,0.258824}%
\pgfsetstrokecolor{currentstroke}%
\pgfsetdash{}{0pt}%
\pgfsys@defobject{currentmarker}{\pgfqpoint{-0.033333in}{-0.033333in}}{\pgfqpoint{0.033333in}{0.033333in}}{%
\pgfpathmoveto{\pgfqpoint{-0.000000in}{-0.033333in}}%
\pgfpathlineto{\pgfqpoint{0.033333in}{0.033333in}}%
\pgfpathlineto{\pgfqpoint{-0.033333in}{0.033333in}}%
\pgfpathlineto{\pgfqpoint{-0.000000in}{-0.033333in}}%
\pgfpathclose%
\pgfusepath{stroke,fill}%
}%
\begin{pgfscope}%
\pgfsys@transformshift{1.730617in}{3.774927in}%
\pgfsys@useobject{currentmarker}{}%
\end{pgfscope}%
\end{pgfscope}%
\begin{pgfscope}%
\pgfpathrectangle{\pgfqpoint{0.765000in}{0.660000in}}{\pgfqpoint{4.620000in}{4.620000in}}%
\pgfusepath{clip}%
\pgfsetrectcap%
\pgfsetroundjoin%
\pgfsetlinewidth{1.204500pt}%
\definecolor{currentstroke}{rgb}{0.176471,0.192157,0.258824}%
\pgfsetstrokecolor{currentstroke}%
\pgfsetdash{}{0pt}%
\pgfpathmoveto{\pgfqpoint{1.303420in}{4.017179in}}%
\pgfusepath{stroke}%
\end{pgfscope}%
\begin{pgfscope}%
\pgfpathrectangle{\pgfqpoint{0.765000in}{0.660000in}}{\pgfqpoint{4.620000in}{4.620000in}}%
\pgfusepath{clip}%
\pgfsetbuttcap%
\pgfsetmiterjoin%
\definecolor{currentfill}{rgb}{0.176471,0.192157,0.258824}%
\pgfsetfillcolor{currentfill}%
\pgfsetlinewidth{1.003750pt}%
\definecolor{currentstroke}{rgb}{0.176471,0.192157,0.258824}%
\pgfsetstrokecolor{currentstroke}%
\pgfsetdash{}{0pt}%
\pgfsys@defobject{currentmarker}{\pgfqpoint{-0.033333in}{-0.033333in}}{\pgfqpoint{0.033333in}{0.033333in}}{%
\pgfpathmoveto{\pgfqpoint{-0.000000in}{-0.033333in}}%
\pgfpathlineto{\pgfqpoint{0.033333in}{0.033333in}}%
\pgfpathlineto{\pgfqpoint{-0.033333in}{0.033333in}}%
\pgfpathlineto{\pgfqpoint{-0.000000in}{-0.033333in}}%
\pgfpathclose%
\pgfusepath{stroke,fill}%
}%
\begin{pgfscope}%
\pgfsys@transformshift{1.303420in}{4.017179in}%
\pgfsys@useobject{currentmarker}{}%
\end{pgfscope}%
\end{pgfscope}%
\begin{pgfscope}%
\pgfpathrectangle{\pgfqpoint{0.765000in}{0.660000in}}{\pgfqpoint{4.620000in}{4.620000in}}%
\pgfusepath{clip}%
\pgfsetrectcap%
\pgfsetroundjoin%
\pgfsetlinewidth{1.204500pt}%
\definecolor{currentstroke}{rgb}{0.019608,0.556863,0.850980}%
\pgfsetstrokecolor{currentstroke}%
\pgfsetdash{}{0pt}%
\pgfpathmoveto{\pgfqpoint{3.198089in}{1.006531in}}%
\pgfusepath{stroke}%
\end{pgfscope}%
\begin{pgfscope}%
\pgfpathrectangle{\pgfqpoint{0.765000in}{0.660000in}}{\pgfqpoint{4.620000in}{4.620000in}}%
\pgfusepath{clip}%
\pgfsetbuttcap%
\pgfsetroundjoin%
\definecolor{currentfill}{rgb}{0.019608,0.556863,0.850980}%
\pgfsetfillcolor{currentfill}%
\pgfsetlinewidth{1.003750pt}%
\definecolor{currentstroke}{rgb}{0.019608,0.556863,0.850980}%
\pgfsetstrokecolor{currentstroke}%
\pgfsetdash{}{0pt}%
\pgfsys@defobject{currentmarker}{\pgfqpoint{-0.033333in}{-0.033333in}}{\pgfqpoint{0.033333in}{0.033333in}}{%
\pgfpathmoveto{\pgfqpoint{-0.033333in}{0.000000in}}%
\pgfpathlineto{\pgfqpoint{0.033333in}{0.000000in}}%
\pgfpathmoveto{\pgfqpoint{0.000000in}{-0.033333in}}%
\pgfpathlineto{\pgfqpoint{0.000000in}{0.033333in}}%
\pgfusepath{stroke,fill}%
}%
\begin{pgfscope}%
\pgfsys@transformshift{3.198089in}{1.006531in}%
\pgfsys@useobject{currentmarker}{}%
\end{pgfscope}%
\end{pgfscope}%
\begin{pgfscope}%
\pgfpathrectangle{\pgfqpoint{0.765000in}{0.660000in}}{\pgfqpoint{4.620000in}{4.620000in}}%
\pgfusepath{clip}%
\pgfsetrectcap%
\pgfsetroundjoin%
\pgfsetlinewidth{1.204500pt}%
\definecolor{currentstroke}{rgb}{0.019608,0.556863,0.850980}%
\pgfsetstrokecolor{currentstroke}%
\pgfsetdash{}{0pt}%
\pgfpathmoveto{\pgfqpoint{3.204769in}{0.812171in}}%
\pgfusepath{stroke}%
\end{pgfscope}%
\begin{pgfscope}%
\pgfpathrectangle{\pgfqpoint{0.765000in}{0.660000in}}{\pgfqpoint{4.620000in}{4.620000in}}%
\pgfusepath{clip}%
\pgfsetbuttcap%
\pgfsetroundjoin%
\definecolor{currentfill}{rgb}{0.019608,0.556863,0.850980}%
\pgfsetfillcolor{currentfill}%
\pgfsetlinewidth{1.003750pt}%
\definecolor{currentstroke}{rgb}{0.019608,0.556863,0.850980}%
\pgfsetstrokecolor{currentstroke}%
\pgfsetdash{}{0pt}%
\pgfsys@defobject{currentmarker}{\pgfqpoint{-0.033333in}{-0.033333in}}{\pgfqpoint{0.033333in}{0.033333in}}{%
\pgfpathmoveto{\pgfqpoint{-0.033333in}{0.000000in}}%
\pgfpathlineto{\pgfqpoint{0.033333in}{0.000000in}}%
\pgfpathmoveto{\pgfqpoint{0.000000in}{-0.033333in}}%
\pgfpathlineto{\pgfqpoint{0.000000in}{0.033333in}}%
\pgfusepath{stroke,fill}%
}%
\begin{pgfscope}%
\pgfsys@transformshift{3.204769in}{0.812171in}%
\pgfsys@useobject{currentmarker}{}%
\end{pgfscope}%
\end{pgfscope}%
\begin{pgfscope}%
\pgfpathrectangle{\pgfqpoint{0.765000in}{0.660000in}}{\pgfqpoint{4.620000in}{4.620000in}}%
\pgfusepath{clip}%
\pgfsetrectcap%
\pgfsetroundjoin%
\pgfsetlinewidth{1.204500pt}%
\definecolor{currentstroke}{rgb}{0.019608,0.556863,0.850980}%
\pgfsetstrokecolor{currentstroke}%
\pgfsetdash{}{0pt}%
\pgfpathmoveto{\pgfqpoint{3.030637in}{1.178001in}}%
\pgfusepath{stroke}%
\end{pgfscope}%
\begin{pgfscope}%
\pgfpathrectangle{\pgfqpoint{0.765000in}{0.660000in}}{\pgfqpoint{4.620000in}{4.620000in}}%
\pgfusepath{clip}%
\pgfsetbuttcap%
\pgfsetroundjoin%
\definecolor{currentfill}{rgb}{0.019608,0.556863,0.850980}%
\pgfsetfillcolor{currentfill}%
\pgfsetlinewidth{1.003750pt}%
\definecolor{currentstroke}{rgb}{0.019608,0.556863,0.850980}%
\pgfsetstrokecolor{currentstroke}%
\pgfsetdash{}{0pt}%
\pgfsys@defobject{currentmarker}{\pgfqpoint{-0.033333in}{-0.033333in}}{\pgfqpoint{0.033333in}{0.033333in}}{%
\pgfpathmoveto{\pgfqpoint{-0.033333in}{0.000000in}}%
\pgfpathlineto{\pgfqpoint{0.033333in}{0.000000in}}%
\pgfpathmoveto{\pgfqpoint{0.000000in}{-0.033333in}}%
\pgfpathlineto{\pgfqpoint{0.000000in}{0.033333in}}%
\pgfusepath{stroke,fill}%
}%
\begin{pgfscope}%
\pgfsys@transformshift{3.030637in}{1.178001in}%
\pgfsys@useobject{currentmarker}{}%
\end{pgfscope}%
\end{pgfscope}%
\begin{pgfscope}%
\pgfpathrectangle{\pgfqpoint{0.765000in}{0.660000in}}{\pgfqpoint{4.620000in}{4.620000in}}%
\pgfusepath{clip}%
\pgfsetrectcap%
\pgfsetroundjoin%
\pgfsetlinewidth{1.204500pt}%
\definecolor{currentstroke}{rgb}{0.019608,0.556863,0.850980}%
\pgfsetstrokecolor{currentstroke}%
\pgfsetdash{}{0pt}%
\pgfpathmoveto{\pgfqpoint{3.367510in}{1.127107in}}%
\pgfusepath{stroke}%
\end{pgfscope}%
\begin{pgfscope}%
\pgfpathrectangle{\pgfqpoint{0.765000in}{0.660000in}}{\pgfqpoint{4.620000in}{4.620000in}}%
\pgfusepath{clip}%
\pgfsetbuttcap%
\pgfsetroundjoin%
\definecolor{currentfill}{rgb}{0.019608,0.556863,0.850980}%
\pgfsetfillcolor{currentfill}%
\pgfsetlinewidth{1.003750pt}%
\definecolor{currentstroke}{rgb}{0.019608,0.556863,0.850980}%
\pgfsetstrokecolor{currentstroke}%
\pgfsetdash{}{0pt}%
\pgfsys@defobject{currentmarker}{\pgfqpoint{-0.033333in}{-0.033333in}}{\pgfqpoint{0.033333in}{0.033333in}}{%
\pgfpathmoveto{\pgfqpoint{-0.033333in}{0.000000in}}%
\pgfpathlineto{\pgfqpoint{0.033333in}{0.000000in}}%
\pgfpathmoveto{\pgfqpoint{0.000000in}{-0.033333in}}%
\pgfpathlineto{\pgfqpoint{0.000000in}{0.033333in}}%
\pgfusepath{stroke,fill}%
}%
\begin{pgfscope}%
\pgfsys@transformshift{3.367510in}{1.127107in}%
\pgfsys@useobject{currentmarker}{}%
\end{pgfscope}%
\end{pgfscope}%
\begin{pgfscope}%
\pgfpathrectangle{\pgfqpoint{0.765000in}{0.660000in}}{\pgfqpoint{4.620000in}{4.620000in}}%
\pgfusepath{clip}%
\pgfsetrectcap%
\pgfsetroundjoin%
\pgfsetlinewidth{1.204500pt}%
\definecolor{currentstroke}{rgb}{0.019608,0.556863,0.850980}%
\pgfsetstrokecolor{currentstroke}%
\pgfsetdash{}{0pt}%
\pgfpathmoveto{\pgfqpoint{3.127453in}{1.115165in}}%
\pgfusepath{stroke}%
\end{pgfscope}%
\begin{pgfscope}%
\pgfpathrectangle{\pgfqpoint{0.765000in}{0.660000in}}{\pgfqpoint{4.620000in}{4.620000in}}%
\pgfusepath{clip}%
\pgfsetbuttcap%
\pgfsetroundjoin%
\definecolor{currentfill}{rgb}{0.019608,0.556863,0.850980}%
\pgfsetfillcolor{currentfill}%
\pgfsetlinewidth{1.003750pt}%
\definecolor{currentstroke}{rgb}{0.019608,0.556863,0.850980}%
\pgfsetstrokecolor{currentstroke}%
\pgfsetdash{}{0pt}%
\pgfsys@defobject{currentmarker}{\pgfqpoint{-0.033333in}{-0.033333in}}{\pgfqpoint{0.033333in}{0.033333in}}{%
\pgfpathmoveto{\pgfqpoint{-0.033333in}{0.000000in}}%
\pgfpathlineto{\pgfqpoint{0.033333in}{0.000000in}}%
\pgfpathmoveto{\pgfqpoint{0.000000in}{-0.033333in}}%
\pgfpathlineto{\pgfqpoint{0.000000in}{0.033333in}}%
\pgfusepath{stroke,fill}%
}%
\begin{pgfscope}%
\pgfsys@transformshift{3.127453in}{1.115165in}%
\pgfsys@useobject{currentmarker}{}%
\end{pgfscope}%
\end{pgfscope}%
\begin{pgfscope}%
\pgfpathrectangle{\pgfqpoint{0.765000in}{0.660000in}}{\pgfqpoint{4.620000in}{4.620000in}}%
\pgfusepath{clip}%
\pgfsetrectcap%
\pgfsetroundjoin%
\pgfsetlinewidth{1.204500pt}%
\definecolor{currentstroke}{rgb}{0.019608,0.556863,0.850980}%
\pgfsetstrokecolor{currentstroke}%
\pgfsetdash{}{0pt}%
\pgfpathmoveto{\pgfqpoint{3.544577in}{1.721011in}}%
\pgfusepath{stroke}%
\end{pgfscope}%
\begin{pgfscope}%
\pgfpathrectangle{\pgfqpoint{0.765000in}{0.660000in}}{\pgfqpoint{4.620000in}{4.620000in}}%
\pgfusepath{clip}%
\pgfsetbuttcap%
\pgfsetroundjoin%
\definecolor{currentfill}{rgb}{0.019608,0.556863,0.850980}%
\pgfsetfillcolor{currentfill}%
\pgfsetlinewidth{1.003750pt}%
\definecolor{currentstroke}{rgb}{0.019608,0.556863,0.850980}%
\pgfsetstrokecolor{currentstroke}%
\pgfsetdash{}{0pt}%
\pgfsys@defobject{currentmarker}{\pgfqpoint{-0.033333in}{-0.033333in}}{\pgfqpoint{0.033333in}{0.033333in}}{%
\pgfpathmoveto{\pgfqpoint{-0.033333in}{0.000000in}}%
\pgfpathlineto{\pgfqpoint{0.033333in}{0.000000in}}%
\pgfpathmoveto{\pgfqpoint{0.000000in}{-0.033333in}}%
\pgfpathlineto{\pgfqpoint{0.000000in}{0.033333in}}%
\pgfusepath{stroke,fill}%
}%
\begin{pgfscope}%
\pgfsys@transformshift{3.544577in}{1.721011in}%
\pgfsys@useobject{currentmarker}{}%
\end{pgfscope}%
\end{pgfscope}%
\begin{pgfscope}%
\pgfpathrectangle{\pgfqpoint{0.765000in}{0.660000in}}{\pgfqpoint{4.620000in}{4.620000in}}%
\pgfusepath{clip}%
\pgfsetrectcap%
\pgfsetroundjoin%
\pgfsetlinewidth{1.204500pt}%
\definecolor{currentstroke}{rgb}{0.019608,0.556863,0.850980}%
\pgfsetstrokecolor{currentstroke}%
\pgfsetdash{}{0pt}%
\pgfpathmoveto{\pgfqpoint{2.995063in}{1.094653in}}%
\pgfusepath{stroke}%
\end{pgfscope}%
\begin{pgfscope}%
\pgfpathrectangle{\pgfqpoint{0.765000in}{0.660000in}}{\pgfqpoint{4.620000in}{4.620000in}}%
\pgfusepath{clip}%
\pgfsetbuttcap%
\pgfsetroundjoin%
\definecolor{currentfill}{rgb}{0.019608,0.556863,0.850980}%
\pgfsetfillcolor{currentfill}%
\pgfsetlinewidth{1.003750pt}%
\definecolor{currentstroke}{rgb}{0.019608,0.556863,0.850980}%
\pgfsetstrokecolor{currentstroke}%
\pgfsetdash{}{0pt}%
\pgfsys@defobject{currentmarker}{\pgfqpoint{-0.033333in}{-0.033333in}}{\pgfqpoint{0.033333in}{0.033333in}}{%
\pgfpathmoveto{\pgfqpoint{-0.033333in}{0.000000in}}%
\pgfpathlineto{\pgfqpoint{0.033333in}{0.000000in}}%
\pgfpathmoveto{\pgfqpoint{0.000000in}{-0.033333in}}%
\pgfpathlineto{\pgfqpoint{0.000000in}{0.033333in}}%
\pgfusepath{stroke,fill}%
}%
\begin{pgfscope}%
\pgfsys@transformshift{2.995063in}{1.094653in}%
\pgfsys@useobject{currentmarker}{}%
\end{pgfscope}%
\end{pgfscope}%
\begin{pgfscope}%
\pgfpathrectangle{\pgfqpoint{0.765000in}{0.660000in}}{\pgfqpoint{4.620000in}{4.620000in}}%
\pgfusepath{clip}%
\pgfsetrectcap%
\pgfsetroundjoin%
\pgfsetlinewidth{1.204500pt}%
\definecolor{currentstroke}{rgb}{0.019608,0.556863,0.850980}%
\pgfsetstrokecolor{currentstroke}%
\pgfsetdash{}{0pt}%
\pgfpathmoveto{\pgfqpoint{3.350107in}{1.325546in}}%
\pgfusepath{stroke}%
\end{pgfscope}%
\begin{pgfscope}%
\pgfpathrectangle{\pgfqpoint{0.765000in}{0.660000in}}{\pgfqpoint{4.620000in}{4.620000in}}%
\pgfusepath{clip}%
\pgfsetbuttcap%
\pgfsetroundjoin%
\definecolor{currentfill}{rgb}{0.019608,0.556863,0.850980}%
\pgfsetfillcolor{currentfill}%
\pgfsetlinewidth{1.003750pt}%
\definecolor{currentstroke}{rgb}{0.019608,0.556863,0.850980}%
\pgfsetstrokecolor{currentstroke}%
\pgfsetdash{}{0pt}%
\pgfsys@defobject{currentmarker}{\pgfqpoint{-0.033333in}{-0.033333in}}{\pgfqpoint{0.033333in}{0.033333in}}{%
\pgfpathmoveto{\pgfqpoint{-0.033333in}{0.000000in}}%
\pgfpathlineto{\pgfqpoint{0.033333in}{0.000000in}}%
\pgfpathmoveto{\pgfqpoint{0.000000in}{-0.033333in}}%
\pgfpathlineto{\pgfqpoint{0.000000in}{0.033333in}}%
\pgfusepath{stroke,fill}%
}%
\begin{pgfscope}%
\pgfsys@transformshift{3.350107in}{1.325546in}%
\pgfsys@useobject{currentmarker}{}%
\end{pgfscope}%
\end{pgfscope}%
\begin{pgfscope}%
\pgfpathrectangle{\pgfqpoint{0.765000in}{0.660000in}}{\pgfqpoint{4.620000in}{4.620000in}}%
\pgfusepath{clip}%
\pgfsetrectcap%
\pgfsetroundjoin%
\pgfsetlinewidth{1.204500pt}%
\definecolor{currentstroke}{rgb}{0.019608,0.556863,0.850980}%
\pgfsetstrokecolor{currentstroke}%
\pgfsetdash{}{0pt}%
\pgfpathmoveto{\pgfqpoint{3.121561in}{1.514899in}}%
\pgfusepath{stroke}%
\end{pgfscope}%
\begin{pgfscope}%
\pgfpathrectangle{\pgfqpoint{0.765000in}{0.660000in}}{\pgfqpoint{4.620000in}{4.620000in}}%
\pgfusepath{clip}%
\pgfsetbuttcap%
\pgfsetroundjoin%
\definecolor{currentfill}{rgb}{0.019608,0.556863,0.850980}%
\pgfsetfillcolor{currentfill}%
\pgfsetlinewidth{1.003750pt}%
\definecolor{currentstroke}{rgb}{0.019608,0.556863,0.850980}%
\pgfsetstrokecolor{currentstroke}%
\pgfsetdash{}{0pt}%
\pgfsys@defobject{currentmarker}{\pgfqpoint{-0.033333in}{-0.033333in}}{\pgfqpoint{0.033333in}{0.033333in}}{%
\pgfpathmoveto{\pgfqpoint{-0.033333in}{0.000000in}}%
\pgfpathlineto{\pgfqpoint{0.033333in}{0.000000in}}%
\pgfpathmoveto{\pgfqpoint{0.000000in}{-0.033333in}}%
\pgfpathlineto{\pgfqpoint{0.000000in}{0.033333in}}%
\pgfusepath{stroke,fill}%
}%
\begin{pgfscope}%
\pgfsys@transformshift{3.121561in}{1.514899in}%
\pgfsys@useobject{currentmarker}{}%
\end{pgfscope}%
\end{pgfscope}%
\begin{pgfscope}%
\pgfpathrectangle{\pgfqpoint{0.765000in}{0.660000in}}{\pgfqpoint{4.620000in}{4.620000in}}%
\pgfusepath{clip}%
\pgfsetrectcap%
\pgfsetroundjoin%
\pgfsetlinewidth{1.204500pt}%
\definecolor{currentstroke}{rgb}{0.019608,0.556863,0.850980}%
\pgfsetstrokecolor{currentstroke}%
\pgfsetdash{}{0pt}%
\pgfpathmoveto{\pgfqpoint{2.803758in}{0.869163in}}%
\pgfusepath{stroke}%
\end{pgfscope}%
\begin{pgfscope}%
\pgfpathrectangle{\pgfqpoint{0.765000in}{0.660000in}}{\pgfqpoint{4.620000in}{4.620000in}}%
\pgfusepath{clip}%
\pgfsetbuttcap%
\pgfsetroundjoin%
\definecolor{currentfill}{rgb}{0.019608,0.556863,0.850980}%
\pgfsetfillcolor{currentfill}%
\pgfsetlinewidth{1.003750pt}%
\definecolor{currentstroke}{rgb}{0.019608,0.556863,0.850980}%
\pgfsetstrokecolor{currentstroke}%
\pgfsetdash{}{0pt}%
\pgfsys@defobject{currentmarker}{\pgfqpoint{-0.033333in}{-0.033333in}}{\pgfqpoint{0.033333in}{0.033333in}}{%
\pgfpathmoveto{\pgfqpoint{-0.033333in}{0.000000in}}%
\pgfpathlineto{\pgfqpoint{0.033333in}{0.000000in}}%
\pgfpathmoveto{\pgfqpoint{0.000000in}{-0.033333in}}%
\pgfpathlineto{\pgfqpoint{0.000000in}{0.033333in}}%
\pgfusepath{stroke,fill}%
}%
\begin{pgfscope}%
\pgfsys@transformshift{2.803758in}{0.869163in}%
\pgfsys@useobject{currentmarker}{}%
\end{pgfscope}%
\end{pgfscope}%
\begin{pgfscope}%
\pgfpathrectangle{\pgfqpoint{0.765000in}{0.660000in}}{\pgfqpoint{4.620000in}{4.620000in}}%
\pgfusepath{clip}%
\pgfsetrectcap%
\pgfsetroundjoin%
\pgfsetlinewidth{1.204500pt}%
\definecolor{currentstroke}{rgb}{0.800000,0.176471,0.207843}%
\pgfsetstrokecolor{currentstroke}%
\pgfsetdash{}{0pt}%
\pgfpathmoveto{\pgfqpoint{2.971011in}{2.987645in}}%
\pgfusepath{stroke}%
\end{pgfscope}%
\begin{pgfscope}%
\pgfpathrectangle{\pgfqpoint{0.765000in}{0.660000in}}{\pgfqpoint{4.620000in}{4.620000in}}%
\pgfusepath{clip}%
\pgfsetbuttcap%
\pgfsetbeveljoin%
\definecolor{currentfill}{rgb}{0.800000,0.176471,0.207843}%
\pgfsetfillcolor{currentfill}%
\pgfsetlinewidth{1.003750pt}%
\definecolor{currentstroke}{rgb}{0.800000,0.176471,0.207843}%
\pgfsetstrokecolor{currentstroke}%
\pgfsetdash{}{0pt}%
\pgfsys@defobject{currentmarker}{\pgfqpoint{-0.031702in}{-0.026967in}}{\pgfqpoint{0.031702in}{0.033333in}}{%
\pgfpathmoveto{\pgfqpoint{0.000000in}{0.033333in}}%
\pgfpathlineto{\pgfqpoint{-0.007484in}{0.010301in}}%
\pgfpathlineto{\pgfqpoint{-0.031702in}{0.010301in}}%
\pgfpathlineto{\pgfqpoint{-0.012109in}{-0.003934in}}%
\pgfpathlineto{\pgfqpoint{-0.019593in}{-0.026967in}}%
\pgfpathlineto{\pgfqpoint{-0.000000in}{-0.012732in}}%
\pgfpathlineto{\pgfqpoint{0.019593in}{-0.026967in}}%
\pgfpathlineto{\pgfqpoint{0.012109in}{-0.003934in}}%
\pgfpathlineto{\pgfqpoint{0.031702in}{0.010301in}}%
\pgfpathlineto{\pgfqpoint{0.007484in}{0.010301in}}%
\pgfpathlineto{\pgfqpoint{0.000000in}{0.033333in}}%
\pgfpathclose%
\pgfusepath{stroke,fill}%
}%
\begin{pgfscope}%
\pgfsys@transformshift{2.971011in}{2.987645in}%
\pgfsys@useobject{currentmarker}{}%
\end{pgfscope}%
\end{pgfscope}%
\begin{pgfscope}%
\pgfpathrectangle{\pgfqpoint{0.765000in}{0.660000in}}{\pgfqpoint{4.620000in}{4.620000in}}%
\pgfusepath{clip}%
\pgfsetrectcap%
\pgfsetroundjoin%
\pgfsetlinewidth{1.204500pt}%
\definecolor{currentstroke}{rgb}{0.800000,0.176471,0.207843}%
\pgfsetstrokecolor{currentstroke}%
\pgfsetdash{}{0pt}%
\pgfpathmoveto{\pgfqpoint{3.008569in}{2.781297in}}%
\pgfusepath{stroke}%
\end{pgfscope}%
\begin{pgfscope}%
\pgfpathrectangle{\pgfqpoint{0.765000in}{0.660000in}}{\pgfqpoint{4.620000in}{4.620000in}}%
\pgfusepath{clip}%
\pgfsetbuttcap%
\pgfsetbeveljoin%
\definecolor{currentfill}{rgb}{0.800000,0.176471,0.207843}%
\pgfsetfillcolor{currentfill}%
\pgfsetlinewidth{1.003750pt}%
\definecolor{currentstroke}{rgb}{0.800000,0.176471,0.207843}%
\pgfsetstrokecolor{currentstroke}%
\pgfsetdash{}{0pt}%
\pgfsys@defobject{currentmarker}{\pgfqpoint{-0.031702in}{-0.026967in}}{\pgfqpoint{0.031702in}{0.033333in}}{%
\pgfpathmoveto{\pgfqpoint{0.000000in}{0.033333in}}%
\pgfpathlineto{\pgfqpoint{-0.007484in}{0.010301in}}%
\pgfpathlineto{\pgfqpoint{-0.031702in}{0.010301in}}%
\pgfpathlineto{\pgfqpoint{-0.012109in}{-0.003934in}}%
\pgfpathlineto{\pgfqpoint{-0.019593in}{-0.026967in}}%
\pgfpathlineto{\pgfqpoint{-0.000000in}{-0.012732in}}%
\pgfpathlineto{\pgfqpoint{0.019593in}{-0.026967in}}%
\pgfpathlineto{\pgfqpoint{0.012109in}{-0.003934in}}%
\pgfpathlineto{\pgfqpoint{0.031702in}{0.010301in}}%
\pgfpathlineto{\pgfqpoint{0.007484in}{0.010301in}}%
\pgfpathlineto{\pgfqpoint{0.000000in}{0.033333in}}%
\pgfpathclose%
\pgfusepath{stroke,fill}%
}%
\begin{pgfscope}%
\pgfsys@transformshift{3.008569in}{2.781297in}%
\pgfsys@useobject{currentmarker}{}%
\end{pgfscope}%
\end{pgfscope}%
\begin{pgfscope}%
\pgfpathrectangle{\pgfqpoint{0.765000in}{0.660000in}}{\pgfqpoint{4.620000in}{4.620000in}}%
\pgfusepath{clip}%
\pgfsetrectcap%
\pgfsetroundjoin%
\pgfsetlinewidth{1.204500pt}%
\definecolor{currentstroke}{rgb}{0.800000,0.176471,0.207843}%
\pgfsetstrokecolor{currentstroke}%
\pgfsetdash{}{0pt}%
\pgfpathmoveto{\pgfqpoint{3.325002in}{2.466601in}}%
\pgfusepath{stroke}%
\end{pgfscope}%
\begin{pgfscope}%
\pgfpathrectangle{\pgfqpoint{0.765000in}{0.660000in}}{\pgfqpoint{4.620000in}{4.620000in}}%
\pgfusepath{clip}%
\pgfsetbuttcap%
\pgfsetbeveljoin%
\definecolor{currentfill}{rgb}{0.800000,0.176471,0.207843}%
\pgfsetfillcolor{currentfill}%
\pgfsetlinewidth{1.003750pt}%
\definecolor{currentstroke}{rgb}{0.800000,0.176471,0.207843}%
\pgfsetstrokecolor{currentstroke}%
\pgfsetdash{}{0pt}%
\pgfsys@defobject{currentmarker}{\pgfqpoint{-0.031702in}{-0.026967in}}{\pgfqpoint{0.031702in}{0.033333in}}{%
\pgfpathmoveto{\pgfqpoint{0.000000in}{0.033333in}}%
\pgfpathlineto{\pgfqpoint{-0.007484in}{0.010301in}}%
\pgfpathlineto{\pgfqpoint{-0.031702in}{0.010301in}}%
\pgfpathlineto{\pgfqpoint{-0.012109in}{-0.003934in}}%
\pgfpathlineto{\pgfqpoint{-0.019593in}{-0.026967in}}%
\pgfpathlineto{\pgfqpoint{-0.000000in}{-0.012732in}}%
\pgfpathlineto{\pgfqpoint{0.019593in}{-0.026967in}}%
\pgfpathlineto{\pgfqpoint{0.012109in}{-0.003934in}}%
\pgfpathlineto{\pgfqpoint{0.031702in}{0.010301in}}%
\pgfpathlineto{\pgfqpoint{0.007484in}{0.010301in}}%
\pgfpathlineto{\pgfqpoint{0.000000in}{0.033333in}}%
\pgfpathclose%
\pgfusepath{stroke,fill}%
}%
\begin{pgfscope}%
\pgfsys@transformshift{3.325002in}{2.466601in}%
\pgfsys@useobject{currentmarker}{}%
\end{pgfscope}%
\end{pgfscope}%
\begin{pgfscope}%
\pgfpathrectangle{\pgfqpoint{0.765000in}{0.660000in}}{\pgfqpoint{4.620000in}{4.620000in}}%
\pgfusepath{clip}%
\pgfsetrectcap%
\pgfsetroundjoin%
\pgfsetlinewidth{1.204500pt}%
\definecolor{currentstroke}{rgb}{0.800000,0.176471,0.207843}%
\pgfsetstrokecolor{currentstroke}%
\pgfsetdash{}{0pt}%
\pgfpathmoveto{\pgfqpoint{3.258238in}{2.943169in}}%
\pgfusepath{stroke}%
\end{pgfscope}%
\begin{pgfscope}%
\pgfpathrectangle{\pgfqpoint{0.765000in}{0.660000in}}{\pgfqpoint{4.620000in}{4.620000in}}%
\pgfusepath{clip}%
\pgfsetbuttcap%
\pgfsetbeveljoin%
\definecolor{currentfill}{rgb}{0.800000,0.176471,0.207843}%
\pgfsetfillcolor{currentfill}%
\pgfsetlinewidth{1.003750pt}%
\definecolor{currentstroke}{rgb}{0.800000,0.176471,0.207843}%
\pgfsetstrokecolor{currentstroke}%
\pgfsetdash{}{0pt}%
\pgfsys@defobject{currentmarker}{\pgfqpoint{-0.031702in}{-0.026967in}}{\pgfqpoint{0.031702in}{0.033333in}}{%
\pgfpathmoveto{\pgfqpoint{0.000000in}{0.033333in}}%
\pgfpathlineto{\pgfqpoint{-0.007484in}{0.010301in}}%
\pgfpathlineto{\pgfqpoint{-0.031702in}{0.010301in}}%
\pgfpathlineto{\pgfqpoint{-0.012109in}{-0.003934in}}%
\pgfpathlineto{\pgfqpoint{-0.019593in}{-0.026967in}}%
\pgfpathlineto{\pgfqpoint{-0.000000in}{-0.012732in}}%
\pgfpathlineto{\pgfqpoint{0.019593in}{-0.026967in}}%
\pgfpathlineto{\pgfqpoint{0.012109in}{-0.003934in}}%
\pgfpathlineto{\pgfqpoint{0.031702in}{0.010301in}}%
\pgfpathlineto{\pgfqpoint{0.007484in}{0.010301in}}%
\pgfpathlineto{\pgfqpoint{0.000000in}{0.033333in}}%
\pgfpathclose%
\pgfusepath{stroke,fill}%
}%
\begin{pgfscope}%
\pgfsys@transformshift{3.258238in}{2.943169in}%
\pgfsys@useobject{currentmarker}{}%
\end{pgfscope}%
\end{pgfscope}%
\begin{pgfscope}%
\pgfpathrectangle{\pgfqpoint{0.765000in}{0.660000in}}{\pgfqpoint{4.620000in}{4.620000in}}%
\pgfusepath{clip}%
\pgfsetrectcap%
\pgfsetroundjoin%
\pgfsetlinewidth{1.204500pt}%
\definecolor{currentstroke}{rgb}{0.800000,0.176471,0.207843}%
\pgfsetstrokecolor{currentstroke}%
\pgfsetdash{}{0pt}%
\pgfpathmoveto{\pgfqpoint{3.074890in}{2.979035in}}%
\pgfusepath{stroke}%
\end{pgfscope}%
\begin{pgfscope}%
\pgfpathrectangle{\pgfqpoint{0.765000in}{0.660000in}}{\pgfqpoint{4.620000in}{4.620000in}}%
\pgfusepath{clip}%
\pgfsetbuttcap%
\pgfsetbeveljoin%
\definecolor{currentfill}{rgb}{0.800000,0.176471,0.207843}%
\pgfsetfillcolor{currentfill}%
\pgfsetlinewidth{1.003750pt}%
\definecolor{currentstroke}{rgb}{0.800000,0.176471,0.207843}%
\pgfsetstrokecolor{currentstroke}%
\pgfsetdash{}{0pt}%
\pgfsys@defobject{currentmarker}{\pgfqpoint{-0.031702in}{-0.026967in}}{\pgfqpoint{0.031702in}{0.033333in}}{%
\pgfpathmoveto{\pgfqpoint{0.000000in}{0.033333in}}%
\pgfpathlineto{\pgfqpoint{-0.007484in}{0.010301in}}%
\pgfpathlineto{\pgfqpoint{-0.031702in}{0.010301in}}%
\pgfpathlineto{\pgfqpoint{-0.012109in}{-0.003934in}}%
\pgfpathlineto{\pgfqpoint{-0.019593in}{-0.026967in}}%
\pgfpathlineto{\pgfqpoint{-0.000000in}{-0.012732in}}%
\pgfpathlineto{\pgfqpoint{0.019593in}{-0.026967in}}%
\pgfpathlineto{\pgfqpoint{0.012109in}{-0.003934in}}%
\pgfpathlineto{\pgfqpoint{0.031702in}{0.010301in}}%
\pgfpathlineto{\pgfqpoint{0.007484in}{0.010301in}}%
\pgfpathlineto{\pgfqpoint{0.000000in}{0.033333in}}%
\pgfpathclose%
\pgfusepath{stroke,fill}%
}%
\begin{pgfscope}%
\pgfsys@transformshift{3.074890in}{2.979035in}%
\pgfsys@useobject{currentmarker}{}%
\end{pgfscope}%
\end{pgfscope}%
\begin{pgfscope}%
\pgfpathrectangle{\pgfqpoint{0.765000in}{0.660000in}}{\pgfqpoint{4.620000in}{4.620000in}}%
\pgfusepath{clip}%
\pgfsetrectcap%
\pgfsetroundjoin%
\pgfsetlinewidth{1.204500pt}%
\definecolor{currentstroke}{rgb}{0.800000,0.176471,0.207843}%
\pgfsetstrokecolor{currentstroke}%
\pgfsetdash{}{0pt}%
\pgfpathmoveto{\pgfqpoint{3.365344in}{2.642171in}}%
\pgfusepath{stroke}%
\end{pgfscope}%
\begin{pgfscope}%
\pgfpathrectangle{\pgfqpoint{0.765000in}{0.660000in}}{\pgfqpoint{4.620000in}{4.620000in}}%
\pgfusepath{clip}%
\pgfsetbuttcap%
\pgfsetbeveljoin%
\definecolor{currentfill}{rgb}{0.800000,0.176471,0.207843}%
\pgfsetfillcolor{currentfill}%
\pgfsetlinewidth{1.003750pt}%
\definecolor{currentstroke}{rgb}{0.800000,0.176471,0.207843}%
\pgfsetstrokecolor{currentstroke}%
\pgfsetdash{}{0pt}%
\pgfsys@defobject{currentmarker}{\pgfqpoint{-0.031702in}{-0.026967in}}{\pgfqpoint{0.031702in}{0.033333in}}{%
\pgfpathmoveto{\pgfqpoint{0.000000in}{0.033333in}}%
\pgfpathlineto{\pgfqpoint{-0.007484in}{0.010301in}}%
\pgfpathlineto{\pgfqpoint{-0.031702in}{0.010301in}}%
\pgfpathlineto{\pgfqpoint{-0.012109in}{-0.003934in}}%
\pgfpathlineto{\pgfqpoint{-0.019593in}{-0.026967in}}%
\pgfpathlineto{\pgfqpoint{-0.000000in}{-0.012732in}}%
\pgfpathlineto{\pgfqpoint{0.019593in}{-0.026967in}}%
\pgfpathlineto{\pgfqpoint{0.012109in}{-0.003934in}}%
\pgfpathlineto{\pgfqpoint{0.031702in}{0.010301in}}%
\pgfpathlineto{\pgfqpoint{0.007484in}{0.010301in}}%
\pgfpathlineto{\pgfqpoint{0.000000in}{0.033333in}}%
\pgfpathclose%
\pgfusepath{stroke,fill}%
}%
\begin{pgfscope}%
\pgfsys@transformshift{3.365344in}{2.642171in}%
\pgfsys@useobject{currentmarker}{}%
\end{pgfscope}%
\end{pgfscope}%
\begin{pgfscope}%
\pgfpathrectangle{\pgfqpoint{0.765000in}{0.660000in}}{\pgfqpoint{4.620000in}{4.620000in}}%
\pgfusepath{clip}%
\pgfsetrectcap%
\pgfsetroundjoin%
\pgfsetlinewidth{1.204500pt}%
\definecolor{currentstroke}{rgb}{0.800000,0.176471,0.207843}%
\pgfsetstrokecolor{currentstroke}%
\pgfsetdash{}{0pt}%
\pgfpathmoveto{\pgfqpoint{3.229056in}{2.963433in}}%
\pgfusepath{stroke}%
\end{pgfscope}%
\begin{pgfscope}%
\pgfpathrectangle{\pgfqpoint{0.765000in}{0.660000in}}{\pgfqpoint{4.620000in}{4.620000in}}%
\pgfusepath{clip}%
\pgfsetbuttcap%
\pgfsetbeveljoin%
\definecolor{currentfill}{rgb}{0.800000,0.176471,0.207843}%
\pgfsetfillcolor{currentfill}%
\pgfsetlinewidth{1.003750pt}%
\definecolor{currentstroke}{rgb}{0.800000,0.176471,0.207843}%
\pgfsetstrokecolor{currentstroke}%
\pgfsetdash{}{0pt}%
\pgfsys@defobject{currentmarker}{\pgfqpoint{-0.031702in}{-0.026967in}}{\pgfqpoint{0.031702in}{0.033333in}}{%
\pgfpathmoveto{\pgfqpoint{0.000000in}{0.033333in}}%
\pgfpathlineto{\pgfqpoint{-0.007484in}{0.010301in}}%
\pgfpathlineto{\pgfqpoint{-0.031702in}{0.010301in}}%
\pgfpathlineto{\pgfqpoint{-0.012109in}{-0.003934in}}%
\pgfpathlineto{\pgfqpoint{-0.019593in}{-0.026967in}}%
\pgfpathlineto{\pgfqpoint{-0.000000in}{-0.012732in}}%
\pgfpathlineto{\pgfqpoint{0.019593in}{-0.026967in}}%
\pgfpathlineto{\pgfqpoint{0.012109in}{-0.003934in}}%
\pgfpathlineto{\pgfqpoint{0.031702in}{0.010301in}}%
\pgfpathlineto{\pgfqpoint{0.007484in}{0.010301in}}%
\pgfpathlineto{\pgfqpoint{0.000000in}{0.033333in}}%
\pgfpathclose%
\pgfusepath{stroke,fill}%
}%
\begin{pgfscope}%
\pgfsys@transformshift{3.229056in}{2.963433in}%
\pgfsys@useobject{currentmarker}{}%
\end{pgfscope}%
\end{pgfscope}%
\begin{pgfscope}%
\pgfpathrectangle{\pgfqpoint{0.765000in}{0.660000in}}{\pgfqpoint{4.620000in}{4.620000in}}%
\pgfusepath{clip}%
\pgfsetrectcap%
\pgfsetroundjoin%
\pgfsetlinewidth{1.204500pt}%
\definecolor{currentstroke}{rgb}{0.800000,0.176471,0.207843}%
\pgfsetstrokecolor{currentstroke}%
\pgfsetdash{}{0pt}%
\pgfpathmoveto{\pgfqpoint{3.280131in}{3.000221in}}%
\pgfusepath{stroke}%
\end{pgfscope}%
\begin{pgfscope}%
\pgfpathrectangle{\pgfqpoint{0.765000in}{0.660000in}}{\pgfqpoint{4.620000in}{4.620000in}}%
\pgfusepath{clip}%
\pgfsetbuttcap%
\pgfsetbeveljoin%
\definecolor{currentfill}{rgb}{0.800000,0.176471,0.207843}%
\pgfsetfillcolor{currentfill}%
\pgfsetlinewidth{1.003750pt}%
\definecolor{currentstroke}{rgb}{0.800000,0.176471,0.207843}%
\pgfsetstrokecolor{currentstroke}%
\pgfsetdash{}{0pt}%
\pgfsys@defobject{currentmarker}{\pgfqpoint{-0.031702in}{-0.026967in}}{\pgfqpoint{0.031702in}{0.033333in}}{%
\pgfpathmoveto{\pgfqpoint{0.000000in}{0.033333in}}%
\pgfpathlineto{\pgfqpoint{-0.007484in}{0.010301in}}%
\pgfpathlineto{\pgfqpoint{-0.031702in}{0.010301in}}%
\pgfpathlineto{\pgfqpoint{-0.012109in}{-0.003934in}}%
\pgfpathlineto{\pgfqpoint{-0.019593in}{-0.026967in}}%
\pgfpathlineto{\pgfqpoint{-0.000000in}{-0.012732in}}%
\pgfpathlineto{\pgfqpoint{0.019593in}{-0.026967in}}%
\pgfpathlineto{\pgfqpoint{0.012109in}{-0.003934in}}%
\pgfpathlineto{\pgfqpoint{0.031702in}{0.010301in}}%
\pgfpathlineto{\pgfqpoint{0.007484in}{0.010301in}}%
\pgfpathlineto{\pgfqpoint{0.000000in}{0.033333in}}%
\pgfpathclose%
\pgfusepath{stroke,fill}%
}%
\begin{pgfscope}%
\pgfsys@transformshift{3.280131in}{3.000221in}%
\pgfsys@useobject{currentmarker}{}%
\end{pgfscope}%
\end{pgfscope}%
\begin{pgfscope}%
\pgfpathrectangle{\pgfqpoint{0.765000in}{0.660000in}}{\pgfqpoint{4.620000in}{4.620000in}}%
\pgfusepath{clip}%
\pgfsetrectcap%
\pgfsetroundjoin%
\pgfsetlinewidth{1.204500pt}%
\definecolor{currentstroke}{rgb}{0.800000,0.176471,0.207843}%
\pgfsetstrokecolor{currentstroke}%
\pgfsetdash{}{0pt}%
\pgfpathmoveto{\pgfqpoint{3.482419in}{3.029945in}}%
\pgfusepath{stroke}%
\end{pgfscope}%
\begin{pgfscope}%
\pgfpathrectangle{\pgfqpoint{0.765000in}{0.660000in}}{\pgfqpoint{4.620000in}{4.620000in}}%
\pgfusepath{clip}%
\pgfsetbuttcap%
\pgfsetbeveljoin%
\definecolor{currentfill}{rgb}{0.800000,0.176471,0.207843}%
\pgfsetfillcolor{currentfill}%
\pgfsetlinewidth{1.003750pt}%
\definecolor{currentstroke}{rgb}{0.800000,0.176471,0.207843}%
\pgfsetstrokecolor{currentstroke}%
\pgfsetdash{}{0pt}%
\pgfsys@defobject{currentmarker}{\pgfqpoint{-0.031702in}{-0.026967in}}{\pgfqpoint{0.031702in}{0.033333in}}{%
\pgfpathmoveto{\pgfqpoint{0.000000in}{0.033333in}}%
\pgfpathlineto{\pgfqpoint{-0.007484in}{0.010301in}}%
\pgfpathlineto{\pgfqpoint{-0.031702in}{0.010301in}}%
\pgfpathlineto{\pgfqpoint{-0.012109in}{-0.003934in}}%
\pgfpathlineto{\pgfqpoint{-0.019593in}{-0.026967in}}%
\pgfpathlineto{\pgfqpoint{-0.000000in}{-0.012732in}}%
\pgfpathlineto{\pgfqpoint{0.019593in}{-0.026967in}}%
\pgfpathlineto{\pgfqpoint{0.012109in}{-0.003934in}}%
\pgfpathlineto{\pgfqpoint{0.031702in}{0.010301in}}%
\pgfpathlineto{\pgfqpoint{0.007484in}{0.010301in}}%
\pgfpathlineto{\pgfqpoint{0.000000in}{0.033333in}}%
\pgfpathclose%
\pgfusepath{stroke,fill}%
}%
\begin{pgfscope}%
\pgfsys@transformshift{3.482419in}{3.029945in}%
\pgfsys@useobject{currentmarker}{}%
\end{pgfscope}%
\end{pgfscope}%
\begin{pgfscope}%
\pgfpathrectangle{\pgfqpoint{0.765000in}{0.660000in}}{\pgfqpoint{4.620000in}{4.620000in}}%
\pgfusepath{clip}%
\pgfsetrectcap%
\pgfsetroundjoin%
\pgfsetlinewidth{1.204500pt}%
\definecolor{currentstroke}{rgb}{0.800000,0.176471,0.207843}%
\pgfsetstrokecolor{currentstroke}%
\pgfsetdash{}{0pt}%
\pgfpathmoveto{\pgfqpoint{2.850325in}{3.090962in}}%
\pgfusepath{stroke}%
\end{pgfscope}%
\begin{pgfscope}%
\pgfpathrectangle{\pgfqpoint{0.765000in}{0.660000in}}{\pgfqpoint{4.620000in}{4.620000in}}%
\pgfusepath{clip}%
\pgfsetbuttcap%
\pgfsetbeveljoin%
\definecolor{currentfill}{rgb}{0.800000,0.176471,0.207843}%
\pgfsetfillcolor{currentfill}%
\pgfsetlinewidth{1.003750pt}%
\definecolor{currentstroke}{rgb}{0.800000,0.176471,0.207843}%
\pgfsetstrokecolor{currentstroke}%
\pgfsetdash{}{0pt}%
\pgfsys@defobject{currentmarker}{\pgfqpoint{-0.031702in}{-0.026967in}}{\pgfqpoint{0.031702in}{0.033333in}}{%
\pgfpathmoveto{\pgfqpoint{0.000000in}{0.033333in}}%
\pgfpathlineto{\pgfqpoint{-0.007484in}{0.010301in}}%
\pgfpathlineto{\pgfqpoint{-0.031702in}{0.010301in}}%
\pgfpathlineto{\pgfqpoint{-0.012109in}{-0.003934in}}%
\pgfpathlineto{\pgfqpoint{-0.019593in}{-0.026967in}}%
\pgfpathlineto{\pgfqpoint{-0.000000in}{-0.012732in}}%
\pgfpathlineto{\pgfqpoint{0.019593in}{-0.026967in}}%
\pgfpathlineto{\pgfqpoint{0.012109in}{-0.003934in}}%
\pgfpathlineto{\pgfqpoint{0.031702in}{0.010301in}}%
\pgfpathlineto{\pgfqpoint{0.007484in}{0.010301in}}%
\pgfpathlineto{\pgfqpoint{0.000000in}{0.033333in}}%
\pgfpathclose%
\pgfusepath{stroke,fill}%
}%
\begin{pgfscope}%
\pgfsys@transformshift{2.850325in}{3.090962in}%
\pgfsys@useobject{currentmarker}{}%
\end{pgfscope}%
\end{pgfscope}%
\begin{pgfscope}%
\pgfpathrectangle{\pgfqpoint{0.765000in}{0.660000in}}{\pgfqpoint{4.620000in}{4.620000in}}%
\pgfusepath{clip}%
\pgfsetrectcap%
\pgfsetroundjoin%
\pgfsetlinewidth{1.204500pt}%
\definecolor{currentstroke}{rgb}{0.000000,0.000000,0.000000}%
\pgfsetstrokecolor{currentstroke}%
\pgfsetdash{}{0pt}%
\pgfpathmoveto{\pgfqpoint{2.472452in}{2.522519in}}%
\pgfusepath{stroke}%
\end{pgfscope}%
\begin{pgfscope}%
\pgfpathrectangle{\pgfqpoint{0.765000in}{0.660000in}}{\pgfqpoint{4.620000in}{4.620000in}}%
\pgfusepath{clip}%
\pgfsetbuttcap%
\pgfsetroundjoin%
\definecolor{currentfill}{rgb}{0.000000,0.000000,0.000000}%
\pgfsetfillcolor{currentfill}%
\pgfsetlinewidth{1.003750pt}%
\definecolor{currentstroke}{rgb}{0.000000,0.000000,0.000000}%
\pgfsetstrokecolor{currentstroke}%
\pgfsetdash{}{0pt}%
\pgfsys@defobject{currentmarker}{\pgfqpoint{-0.016667in}{-0.016667in}}{\pgfqpoint{0.016667in}{0.016667in}}{%
\pgfpathmoveto{\pgfqpoint{0.000000in}{-0.016667in}}%
\pgfpathcurveto{\pgfqpoint{0.004420in}{-0.016667in}}{\pgfqpoint{0.008660in}{-0.014911in}}{\pgfqpoint{0.011785in}{-0.011785in}}%
\pgfpathcurveto{\pgfqpoint{0.014911in}{-0.008660in}}{\pgfqpoint{0.016667in}{-0.004420in}}{\pgfqpoint{0.016667in}{0.000000in}}%
\pgfpathcurveto{\pgfqpoint{0.016667in}{0.004420in}}{\pgfqpoint{0.014911in}{0.008660in}}{\pgfqpoint{0.011785in}{0.011785in}}%
\pgfpathcurveto{\pgfqpoint{0.008660in}{0.014911in}}{\pgfqpoint{0.004420in}{0.016667in}}{\pgfqpoint{0.000000in}{0.016667in}}%
\pgfpathcurveto{\pgfqpoint{-0.004420in}{0.016667in}}{\pgfqpoint{-0.008660in}{0.014911in}}{\pgfqpoint{-0.011785in}{0.011785in}}%
\pgfpathcurveto{\pgfqpoint{-0.014911in}{0.008660in}}{\pgfqpoint{-0.016667in}{0.004420in}}{\pgfqpoint{-0.016667in}{0.000000in}}%
\pgfpathcurveto{\pgfqpoint{-0.016667in}{-0.004420in}}{\pgfqpoint{-0.014911in}{-0.008660in}}{\pgfqpoint{-0.011785in}{-0.011785in}}%
\pgfpathcurveto{\pgfqpoint{-0.008660in}{-0.014911in}}{\pgfqpoint{-0.004420in}{-0.016667in}}{\pgfqpoint{0.000000in}{-0.016667in}}%
\pgfpathlineto{\pgfqpoint{0.000000in}{-0.016667in}}%
\pgfpathclose%
\pgfusepath{stroke,fill}%
}%
\begin{pgfscope}%
\pgfsys@transformshift{2.472452in}{2.522519in}%
\pgfsys@useobject{currentmarker}{}%
\end{pgfscope}%
\end{pgfscope}%
\begin{pgfscope}%
\pgfpathrectangle{\pgfqpoint{0.765000in}{0.660000in}}{\pgfqpoint{4.620000in}{4.620000in}}%
\pgfusepath{clip}%
\pgfsetrectcap%
\pgfsetroundjoin%
\pgfsetlinewidth{1.204500pt}%
\definecolor{currentstroke}{rgb}{0.000000,0.000000,0.000000}%
\pgfsetstrokecolor{currentstroke}%
\pgfsetdash{}{0pt}%
\pgfpathmoveto{\pgfqpoint{2.472452in}{2.522519in}}%
\pgfpathlineto{\pgfqpoint{3.324897in}{2.030360in}}%
\pgfusepath{stroke}%
\end{pgfscope}%
\begin{pgfscope}%
\pgfpathrectangle{\pgfqpoint{0.765000in}{0.660000in}}{\pgfqpoint{4.620000in}{4.620000in}}%
\pgfusepath{clip}%
\pgfsetrectcap%
\pgfsetroundjoin%
\pgfsetlinewidth{1.204500pt}%
\definecolor{currentstroke}{rgb}{0.000000,0.000000,0.000000}%
\pgfsetstrokecolor{currentstroke}%
\pgfsetdash{}{0pt}%
\pgfpathmoveto{\pgfqpoint{2.472452in}{2.522519in}}%
\pgfpathlineto{\pgfqpoint{2.472452in}{3.309215in}}%
\pgfusepath{stroke}%
\end{pgfscope}%
\begin{pgfscope}%
\pgfpathrectangle{\pgfqpoint{0.765000in}{0.660000in}}{\pgfqpoint{4.620000in}{4.620000in}}%
\pgfusepath{clip}%
\pgfsetrectcap%
\pgfsetroundjoin%
\pgfsetlinewidth{1.204500pt}%
\definecolor{currentstroke}{rgb}{0.000000,0.000000,0.000000}%
\pgfsetstrokecolor{currentstroke}%
\pgfsetdash{}{0pt}%
\pgfpathmoveto{\pgfqpoint{2.472452in}{2.522519in}}%
\pgfpathlineto{\pgfqpoint{0.755000in}{1.530947in}}%
\pgfusepath{stroke}%
\end{pgfscope}%
\begin{pgfscope}%
\pgfpathrectangle{\pgfqpoint{0.765000in}{0.660000in}}{\pgfqpoint{4.620000in}{4.620000in}}%
\pgfusepath{clip}%
\pgfsetrectcap%
\pgfsetroundjoin%
\pgfsetlinewidth{1.204500pt}%
\definecolor{currentstroke}{rgb}{0.000000,0.000000,0.000000}%
\pgfsetstrokecolor{currentstroke}%
\pgfsetdash{}{0pt}%
\pgfpathmoveto{\pgfqpoint{3.324897in}{2.030360in}}%
\pgfusepath{stroke}%
\end{pgfscope}%
\begin{pgfscope}%
\pgfpathrectangle{\pgfqpoint{0.765000in}{0.660000in}}{\pgfqpoint{4.620000in}{4.620000in}}%
\pgfusepath{clip}%
\pgfsetbuttcap%
\pgfsetroundjoin%
\definecolor{currentfill}{rgb}{0.000000,0.000000,0.000000}%
\pgfsetfillcolor{currentfill}%
\pgfsetlinewidth{1.003750pt}%
\definecolor{currentstroke}{rgb}{0.000000,0.000000,0.000000}%
\pgfsetstrokecolor{currentstroke}%
\pgfsetdash{}{0pt}%
\pgfsys@defobject{currentmarker}{\pgfqpoint{-0.016667in}{-0.016667in}}{\pgfqpoint{0.016667in}{0.016667in}}{%
\pgfpathmoveto{\pgfqpoint{0.000000in}{-0.016667in}}%
\pgfpathcurveto{\pgfqpoint{0.004420in}{-0.016667in}}{\pgfqpoint{0.008660in}{-0.014911in}}{\pgfqpoint{0.011785in}{-0.011785in}}%
\pgfpathcurveto{\pgfqpoint{0.014911in}{-0.008660in}}{\pgfqpoint{0.016667in}{-0.004420in}}{\pgfqpoint{0.016667in}{0.000000in}}%
\pgfpathcurveto{\pgfqpoint{0.016667in}{0.004420in}}{\pgfqpoint{0.014911in}{0.008660in}}{\pgfqpoint{0.011785in}{0.011785in}}%
\pgfpathcurveto{\pgfqpoint{0.008660in}{0.014911in}}{\pgfqpoint{0.004420in}{0.016667in}}{\pgfqpoint{0.000000in}{0.016667in}}%
\pgfpathcurveto{\pgfqpoint{-0.004420in}{0.016667in}}{\pgfqpoint{-0.008660in}{0.014911in}}{\pgfqpoint{-0.011785in}{0.011785in}}%
\pgfpathcurveto{\pgfqpoint{-0.014911in}{0.008660in}}{\pgfqpoint{-0.016667in}{0.004420in}}{\pgfqpoint{-0.016667in}{0.000000in}}%
\pgfpathcurveto{\pgfqpoint{-0.016667in}{-0.004420in}}{\pgfqpoint{-0.014911in}{-0.008660in}}{\pgfqpoint{-0.011785in}{-0.011785in}}%
\pgfpathcurveto{\pgfqpoint{-0.008660in}{-0.014911in}}{\pgfqpoint{-0.004420in}{-0.016667in}}{\pgfqpoint{0.000000in}{-0.016667in}}%
\pgfpathlineto{\pgfqpoint{0.000000in}{-0.016667in}}%
\pgfpathclose%
\pgfusepath{stroke,fill}%
}%
\begin{pgfscope}%
\pgfsys@transformshift{3.324897in}{2.030360in}%
\pgfsys@useobject{currentmarker}{}%
\end{pgfscope}%
\end{pgfscope}%
\begin{pgfscope}%
\pgfpathrectangle{\pgfqpoint{0.765000in}{0.660000in}}{\pgfqpoint{4.620000in}{4.620000in}}%
\pgfusepath{clip}%
\pgfsetrectcap%
\pgfsetroundjoin%
\pgfsetlinewidth{1.204500pt}%
\definecolor{currentstroke}{rgb}{0.000000,0.000000,0.000000}%
\pgfsetstrokecolor{currentstroke}%
\pgfsetdash{}{0pt}%
\pgfpathmoveto{\pgfqpoint{3.324897in}{2.030360in}}%
\pgfpathlineto{\pgfqpoint{2.472452in}{2.522519in}}%
\pgfusepath{stroke}%
\end{pgfscope}%
\begin{pgfscope}%
\pgfpathrectangle{\pgfqpoint{0.765000in}{0.660000in}}{\pgfqpoint{4.620000in}{4.620000in}}%
\pgfusepath{clip}%
\pgfsetrectcap%
\pgfsetroundjoin%
\pgfsetlinewidth{1.204500pt}%
\definecolor{currentstroke}{rgb}{0.000000,0.000000,0.000000}%
\pgfsetstrokecolor{currentstroke}%
\pgfsetdash{}{0pt}%
\pgfpathmoveto{\pgfqpoint{3.324897in}{2.030360in}}%
\pgfpathlineto{\pgfqpoint{3.860367in}{2.339514in}}%
\pgfusepath{stroke}%
\end{pgfscope}%
\begin{pgfscope}%
\pgfpathrectangle{\pgfqpoint{0.765000in}{0.660000in}}{\pgfqpoint{4.620000in}{4.620000in}}%
\pgfusepath{clip}%
\pgfsetbuttcap%
\pgfsetroundjoin%
\pgfsetlinewidth{1.204500pt}%
\definecolor{currentstroke}{rgb}{0.827451,0.827451,0.827451}%
\pgfsetstrokecolor{currentstroke}%
\pgfsetdash{{1.200000pt}{1.980000pt}}{0.000000pt}%
\pgfpathmoveto{\pgfqpoint{3.324897in}{2.030360in}}%
\pgfpathlineto{\pgfqpoint{3.324897in}{0.650000in}}%
\pgfusepath{stroke}%
\end{pgfscope}%
\begin{pgfscope}%
\pgfpathrectangle{\pgfqpoint{0.765000in}{0.660000in}}{\pgfqpoint{4.620000in}{4.620000in}}%
\pgfusepath{clip}%
\pgfsetrectcap%
\pgfsetroundjoin%
\pgfsetlinewidth{1.204500pt}%
\definecolor{currentstroke}{rgb}{0.000000,0.000000,0.000000}%
\pgfsetstrokecolor{currentstroke}%
\pgfsetdash{}{0pt}%
\pgfpathmoveto{\pgfqpoint{2.472452in}{3.309215in}}%
\pgfusepath{stroke}%
\end{pgfscope}%
\begin{pgfscope}%
\pgfpathrectangle{\pgfqpoint{0.765000in}{0.660000in}}{\pgfqpoint{4.620000in}{4.620000in}}%
\pgfusepath{clip}%
\pgfsetbuttcap%
\pgfsetroundjoin%
\definecolor{currentfill}{rgb}{0.000000,0.000000,0.000000}%
\pgfsetfillcolor{currentfill}%
\pgfsetlinewidth{1.003750pt}%
\definecolor{currentstroke}{rgb}{0.000000,0.000000,0.000000}%
\pgfsetstrokecolor{currentstroke}%
\pgfsetdash{}{0pt}%
\pgfsys@defobject{currentmarker}{\pgfqpoint{-0.016667in}{-0.016667in}}{\pgfqpoint{0.016667in}{0.016667in}}{%
\pgfpathmoveto{\pgfqpoint{0.000000in}{-0.016667in}}%
\pgfpathcurveto{\pgfqpoint{0.004420in}{-0.016667in}}{\pgfqpoint{0.008660in}{-0.014911in}}{\pgfqpoint{0.011785in}{-0.011785in}}%
\pgfpathcurveto{\pgfqpoint{0.014911in}{-0.008660in}}{\pgfqpoint{0.016667in}{-0.004420in}}{\pgfqpoint{0.016667in}{0.000000in}}%
\pgfpathcurveto{\pgfqpoint{0.016667in}{0.004420in}}{\pgfqpoint{0.014911in}{0.008660in}}{\pgfqpoint{0.011785in}{0.011785in}}%
\pgfpathcurveto{\pgfqpoint{0.008660in}{0.014911in}}{\pgfqpoint{0.004420in}{0.016667in}}{\pgfqpoint{0.000000in}{0.016667in}}%
\pgfpathcurveto{\pgfqpoint{-0.004420in}{0.016667in}}{\pgfqpoint{-0.008660in}{0.014911in}}{\pgfqpoint{-0.011785in}{0.011785in}}%
\pgfpathcurveto{\pgfqpoint{-0.014911in}{0.008660in}}{\pgfqpoint{-0.016667in}{0.004420in}}{\pgfqpoint{-0.016667in}{0.000000in}}%
\pgfpathcurveto{\pgfqpoint{-0.016667in}{-0.004420in}}{\pgfqpoint{-0.014911in}{-0.008660in}}{\pgfqpoint{-0.011785in}{-0.011785in}}%
\pgfpathcurveto{\pgfqpoint{-0.008660in}{-0.014911in}}{\pgfqpoint{-0.004420in}{-0.016667in}}{\pgfqpoint{0.000000in}{-0.016667in}}%
\pgfpathlineto{\pgfqpoint{0.000000in}{-0.016667in}}%
\pgfpathclose%
\pgfusepath{stroke,fill}%
}%
\begin{pgfscope}%
\pgfsys@transformshift{2.472452in}{3.309215in}%
\pgfsys@useobject{currentmarker}{}%
\end{pgfscope}%
\end{pgfscope}%
\begin{pgfscope}%
\pgfpathrectangle{\pgfqpoint{0.765000in}{0.660000in}}{\pgfqpoint{4.620000in}{4.620000in}}%
\pgfusepath{clip}%
\pgfsetrectcap%
\pgfsetroundjoin%
\pgfsetlinewidth{1.204500pt}%
\definecolor{currentstroke}{rgb}{0.000000,0.000000,0.000000}%
\pgfsetstrokecolor{currentstroke}%
\pgfsetdash{}{0pt}%
\pgfpathmoveto{\pgfqpoint{2.472452in}{3.309215in}}%
\pgfpathlineto{\pgfqpoint{2.472452in}{2.522519in}}%
\pgfusepath{stroke}%
\end{pgfscope}%
\begin{pgfscope}%
\pgfpathrectangle{\pgfqpoint{0.765000in}{0.660000in}}{\pgfqpoint{4.620000in}{4.620000in}}%
\pgfusepath{clip}%
\pgfsetrectcap%
\pgfsetroundjoin%
\pgfsetlinewidth{1.204500pt}%
\definecolor{currentstroke}{rgb}{0.000000,0.000000,0.000000}%
\pgfsetstrokecolor{currentstroke}%
\pgfsetdash{}{0pt}%
\pgfpathmoveto{\pgfqpoint{2.472452in}{3.309215in}}%
\pgfpathlineto{\pgfqpoint{3.113634in}{3.679402in}}%
\pgfusepath{stroke}%
\end{pgfscope}%
\begin{pgfscope}%
\pgfpathrectangle{\pgfqpoint{0.765000in}{0.660000in}}{\pgfqpoint{4.620000in}{4.620000in}}%
\pgfusepath{clip}%
\pgfsetbuttcap%
\pgfsetroundjoin%
\pgfsetlinewidth{1.204500pt}%
\definecolor{currentstroke}{rgb}{0.827451,0.827451,0.827451}%
\pgfsetstrokecolor{currentstroke}%
\pgfsetdash{{1.200000pt}{1.980000pt}}{0.000000pt}%
\pgfpathmoveto{\pgfqpoint{2.472452in}{3.309215in}}%
\pgfpathlineto{\pgfqpoint{0.755000in}{4.300787in}}%
\pgfusepath{stroke}%
\end{pgfscope}%
\begin{pgfscope}%
\pgfpathrectangle{\pgfqpoint{0.765000in}{0.660000in}}{\pgfqpoint{4.620000in}{4.620000in}}%
\pgfusepath{clip}%
\pgfsetrectcap%
\pgfsetroundjoin%
\pgfsetlinewidth{1.204500pt}%
\definecolor{currentstroke}{rgb}{0.000000,0.000000,0.000000}%
\pgfsetstrokecolor{currentstroke}%
\pgfsetdash{}{0pt}%
\pgfpathmoveto{\pgfqpoint{3.860367in}{2.339514in}}%
\pgfusepath{stroke}%
\end{pgfscope}%
\begin{pgfscope}%
\pgfpathrectangle{\pgfqpoint{0.765000in}{0.660000in}}{\pgfqpoint{4.620000in}{4.620000in}}%
\pgfusepath{clip}%
\pgfsetbuttcap%
\pgfsetroundjoin%
\definecolor{currentfill}{rgb}{0.000000,0.000000,0.000000}%
\pgfsetfillcolor{currentfill}%
\pgfsetlinewidth{1.003750pt}%
\definecolor{currentstroke}{rgb}{0.000000,0.000000,0.000000}%
\pgfsetstrokecolor{currentstroke}%
\pgfsetdash{}{0pt}%
\pgfsys@defobject{currentmarker}{\pgfqpoint{-0.016667in}{-0.016667in}}{\pgfqpoint{0.016667in}{0.016667in}}{%
\pgfpathmoveto{\pgfqpoint{0.000000in}{-0.016667in}}%
\pgfpathcurveto{\pgfqpoint{0.004420in}{-0.016667in}}{\pgfqpoint{0.008660in}{-0.014911in}}{\pgfqpoint{0.011785in}{-0.011785in}}%
\pgfpathcurveto{\pgfqpoint{0.014911in}{-0.008660in}}{\pgfqpoint{0.016667in}{-0.004420in}}{\pgfqpoint{0.016667in}{0.000000in}}%
\pgfpathcurveto{\pgfqpoint{0.016667in}{0.004420in}}{\pgfqpoint{0.014911in}{0.008660in}}{\pgfqpoint{0.011785in}{0.011785in}}%
\pgfpathcurveto{\pgfqpoint{0.008660in}{0.014911in}}{\pgfqpoint{0.004420in}{0.016667in}}{\pgfqpoint{0.000000in}{0.016667in}}%
\pgfpathcurveto{\pgfqpoint{-0.004420in}{0.016667in}}{\pgfqpoint{-0.008660in}{0.014911in}}{\pgfqpoint{-0.011785in}{0.011785in}}%
\pgfpathcurveto{\pgfqpoint{-0.014911in}{0.008660in}}{\pgfqpoint{-0.016667in}{0.004420in}}{\pgfqpoint{-0.016667in}{0.000000in}}%
\pgfpathcurveto{\pgfqpoint{-0.016667in}{-0.004420in}}{\pgfqpoint{-0.014911in}{-0.008660in}}{\pgfqpoint{-0.011785in}{-0.011785in}}%
\pgfpathcurveto{\pgfqpoint{-0.008660in}{-0.014911in}}{\pgfqpoint{-0.004420in}{-0.016667in}}{\pgfqpoint{0.000000in}{-0.016667in}}%
\pgfpathlineto{\pgfqpoint{0.000000in}{-0.016667in}}%
\pgfpathclose%
\pgfusepath{stroke,fill}%
}%
\begin{pgfscope}%
\pgfsys@transformshift{3.860367in}{2.339514in}%
\pgfsys@useobject{currentmarker}{}%
\end{pgfscope}%
\end{pgfscope}%
\begin{pgfscope}%
\pgfpathrectangle{\pgfqpoint{0.765000in}{0.660000in}}{\pgfqpoint{4.620000in}{4.620000in}}%
\pgfusepath{clip}%
\pgfsetrectcap%
\pgfsetroundjoin%
\pgfsetlinewidth{1.204500pt}%
\definecolor{currentstroke}{rgb}{0.000000,0.000000,0.000000}%
\pgfsetstrokecolor{currentstroke}%
\pgfsetdash{}{0pt}%
\pgfpathmoveto{\pgfqpoint{3.860367in}{2.339514in}}%
\pgfpathlineto{\pgfqpoint{3.324897in}{2.030360in}}%
\pgfusepath{stroke}%
\end{pgfscope}%
\begin{pgfscope}%
\pgfpathrectangle{\pgfqpoint{0.765000in}{0.660000in}}{\pgfqpoint{4.620000in}{4.620000in}}%
\pgfusepath{clip}%
\pgfsetrectcap%
\pgfsetroundjoin%
\pgfsetlinewidth{1.204500pt}%
\definecolor{currentstroke}{rgb}{0.000000,0.000000,0.000000}%
\pgfsetstrokecolor{currentstroke}%
\pgfsetdash{}{0pt}%
\pgfpathmoveto{\pgfqpoint{3.860367in}{2.339514in}}%
\pgfpathlineto{\pgfqpoint{3.860367in}{3.248275in}}%
\pgfusepath{stroke}%
\end{pgfscope}%
\begin{pgfscope}%
\pgfpathrectangle{\pgfqpoint{0.765000in}{0.660000in}}{\pgfqpoint{4.620000in}{4.620000in}}%
\pgfusepath{clip}%
\pgfsetrectcap%
\pgfsetroundjoin%
\pgfsetlinewidth{1.204500pt}%
\definecolor{currentstroke}{rgb}{0.000000,0.000000,0.000000}%
\pgfsetstrokecolor{currentstroke}%
\pgfsetdash{}{0pt}%
\pgfpathmoveto{\pgfqpoint{3.860367in}{2.339514in}}%
\pgfpathlineto{\pgfqpoint{5.395000in}{1.453493in}}%
\pgfusepath{stroke}%
\end{pgfscope}%
\begin{pgfscope}%
\pgfpathrectangle{\pgfqpoint{0.765000in}{0.660000in}}{\pgfqpoint{4.620000in}{4.620000in}}%
\pgfusepath{clip}%
\pgfsetrectcap%
\pgfsetroundjoin%
\pgfsetlinewidth{1.204500pt}%
\definecolor{currentstroke}{rgb}{0.000000,0.000000,0.000000}%
\pgfsetstrokecolor{currentstroke}%
\pgfsetdash{}{0pt}%
\pgfpathmoveto{\pgfqpoint{3.860367in}{3.248275in}}%
\pgfusepath{stroke}%
\end{pgfscope}%
\begin{pgfscope}%
\pgfpathrectangle{\pgfqpoint{0.765000in}{0.660000in}}{\pgfqpoint{4.620000in}{4.620000in}}%
\pgfusepath{clip}%
\pgfsetbuttcap%
\pgfsetroundjoin%
\definecolor{currentfill}{rgb}{0.000000,0.000000,0.000000}%
\pgfsetfillcolor{currentfill}%
\pgfsetlinewidth{1.003750pt}%
\definecolor{currentstroke}{rgb}{0.000000,0.000000,0.000000}%
\pgfsetstrokecolor{currentstroke}%
\pgfsetdash{}{0pt}%
\pgfsys@defobject{currentmarker}{\pgfqpoint{-0.016667in}{-0.016667in}}{\pgfqpoint{0.016667in}{0.016667in}}{%
\pgfpathmoveto{\pgfqpoint{0.000000in}{-0.016667in}}%
\pgfpathcurveto{\pgfqpoint{0.004420in}{-0.016667in}}{\pgfqpoint{0.008660in}{-0.014911in}}{\pgfqpoint{0.011785in}{-0.011785in}}%
\pgfpathcurveto{\pgfqpoint{0.014911in}{-0.008660in}}{\pgfqpoint{0.016667in}{-0.004420in}}{\pgfqpoint{0.016667in}{0.000000in}}%
\pgfpathcurveto{\pgfqpoint{0.016667in}{0.004420in}}{\pgfqpoint{0.014911in}{0.008660in}}{\pgfqpoint{0.011785in}{0.011785in}}%
\pgfpathcurveto{\pgfqpoint{0.008660in}{0.014911in}}{\pgfqpoint{0.004420in}{0.016667in}}{\pgfqpoint{0.000000in}{0.016667in}}%
\pgfpathcurveto{\pgfqpoint{-0.004420in}{0.016667in}}{\pgfqpoint{-0.008660in}{0.014911in}}{\pgfqpoint{-0.011785in}{0.011785in}}%
\pgfpathcurveto{\pgfqpoint{-0.014911in}{0.008660in}}{\pgfqpoint{-0.016667in}{0.004420in}}{\pgfqpoint{-0.016667in}{0.000000in}}%
\pgfpathcurveto{\pgfqpoint{-0.016667in}{-0.004420in}}{\pgfqpoint{-0.014911in}{-0.008660in}}{\pgfqpoint{-0.011785in}{-0.011785in}}%
\pgfpathcurveto{\pgfqpoint{-0.008660in}{-0.014911in}}{\pgfqpoint{-0.004420in}{-0.016667in}}{\pgfqpoint{0.000000in}{-0.016667in}}%
\pgfpathlineto{\pgfqpoint{0.000000in}{-0.016667in}}%
\pgfpathclose%
\pgfusepath{stroke,fill}%
}%
\begin{pgfscope}%
\pgfsys@transformshift{3.860367in}{3.248275in}%
\pgfsys@useobject{currentmarker}{}%
\end{pgfscope}%
\end{pgfscope}%
\begin{pgfscope}%
\pgfpathrectangle{\pgfqpoint{0.765000in}{0.660000in}}{\pgfqpoint{4.620000in}{4.620000in}}%
\pgfusepath{clip}%
\pgfsetrectcap%
\pgfsetroundjoin%
\pgfsetlinewidth{1.204500pt}%
\definecolor{currentstroke}{rgb}{0.000000,0.000000,0.000000}%
\pgfsetstrokecolor{currentstroke}%
\pgfsetdash{}{0pt}%
\pgfpathmoveto{\pgfqpoint{3.860367in}{3.248275in}}%
\pgfpathlineto{\pgfqpoint{3.860367in}{2.339514in}}%
\pgfusepath{stroke}%
\end{pgfscope}%
\begin{pgfscope}%
\pgfpathrectangle{\pgfqpoint{0.765000in}{0.660000in}}{\pgfqpoint{4.620000in}{4.620000in}}%
\pgfusepath{clip}%
\pgfsetrectcap%
\pgfsetroundjoin%
\pgfsetlinewidth{1.204500pt}%
\definecolor{currentstroke}{rgb}{0.000000,0.000000,0.000000}%
\pgfsetstrokecolor{currentstroke}%
\pgfsetdash{}{0pt}%
\pgfpathmoveto{\pgfqpoint{3.860367in}{3.248275in}}%
\pgfpathlineto{\pgfqpoint{3.113634in}{3.679402in}}%
\pgfusepath{stroke}%
\end{pgfscope}%
\begin{pgfscope}%
\pgfpathrectangle{\pgfqpoint{0.765000in}{0.660000in}}{\pgfqpoint{4.620000in}{4.620000in}}%
\pgfusepath{clip}%
\pgfsetbuttcap%
\pgfsetroundjoin%
\pgfsetlinewidth{1.204500pt}%
\definecolor{currentstroke}{rgb}{0.827451,0.827451,0.827451}%
\pgfsetstrokecolor{currentstroke}%
\pgfsetdash{{1.200000pt}{1.980000pt}}{0.000000pt}%
\pgfpathmoveto{\pgfqpoint{3.860367in}{3.248275in}}%
\pgfpathlineto{\pgfqpoint{5.395000in}{4.134296in}}%
\pgfusepath{stroke}%
\end{pgfscope}%
\begin{pgfscope}%
\pgfpathrectangle{\pgfqpoint{0.765000in}{0.660000in}}{\pgfqpoint{4.620000in}{4.620000in}}%
\pgfusepath{clip}%
\pgfsetrectcap%
\pgfsetroundjoin%
\pgfsetlinewidth{1.204500pt}%
\definecolor{currentstroke}{rgb}{0.000000,0.000000,0.000000}%
\pgfsetstrokecolor{currentstroke}%
\pgfsetdash{}{0pt}%
\pgfpathmoveto{\pgfqpoint{3.113634in}{3.679402in}}%
\pgfusepath{stroke}%
\end{pgfscope}%
\begin{pgfscope}%
\pgfpathrectangle{\pgfqpoint{0.765000in}{0.660000in}}{\pgfqpoint{4.620000in}{4.620000in}}%
\pgfusepath{clip}%
\pgfsetbuttcap%
\pgfsetroundjoin%
\definecolor{currentfill}{rgb}{0.000000,0.000000,0.000000}%
\pgfsetfillcolor{currentfill}%
\pgfsetlinewidth{1.003750pt}%
\definecolor{currentstroke}{rgb}{0.000000,0.000000,0.000000}%
\pgfsetstrokecolor{currentstroke}%
\pgfsetdash{}{0pt}%
\pgfsys@defobject{currentmarker}{\pgfqpoint{-0.016667in}{-0.016667in}}{\pgfqpoint{0.016667in}{0.016667in}}{%
\pgfpathmoveto{\pgfqpoint{0.000000in}{-0.016667in}}%
\pgfpathcurveto{\pgfqpoint{0.004420in}{-0.016667in}}{\pgfqpoint{0.008660in}{-0.014911in}}{\pgfqpoint{0.011785in}{-0.011785in}}%
\pgfpathcurveto{\pgfqpoint{0.014911in}{-0.008660in}}{\pgfqpoint{0.016667in}{-0.004420in}}{\pgfqpoint{0.016667in}{0.000000in}}%
\pgfpathcurveto{\pgfqpoint{0.016667in}{0.004420in}}{\pgfqpoint{0.014911in}{0.008660in}}{\pgfqpoint{0.011785in}{0.011785in}}%
\pgfpathcurveto{\pgfqpoint{0.008660in}{0.014911in}}{\pgfqpoint{0.004420in}{0.016667in}}{\pgfqpoint{0.000000in}{0.016667in}}%
\pgfpathcurveto{\pgfqpoint{-0.004420in}{0.016667in}}{\pgfqpoint{-0.008660in}{0.014911in}}{\pgfqpoint{-0.011785in}{0.011785in}}%
\pgfpathcurveto{\pgfqpoint{-0.014911in}{0.008660in}}{\pgfqpoint{-0.016667in}{0.004420in}}{\pgfqpoint{-0.016667in}{0.000000in}}%
\pgfpathcurveto{\pgfqpoint{-0.016667in}{-0.004420in}}{\pgfqpoint{-0.014911in}{-0.008660in}}{\pgfqpoint{-0.011785in}{-0.011785in}}%
\pgfpathcurveto{\pgfqpoint{-0.008660in}{-0.014911in}}{\pgfqpoint{-0.004420in}{-0.016667in}}{\pgfqpoint{0.000000in}{-0.016667in}}%
\pgfpathlineto{\pgfqpoint{0.000000in}{-0.016667in}}%
\pgfpathclose%
\pgfusepath{stroke,fill}%
}%
\begin{pgfscope}%
\pgfsys@transformshift{3.113634in}{3.679402in}%
\pgfsys@useobject{currentmarker}{}%
\end{pgfscope}%
\end{pgfscope}%
\begin{pgfscope}%
\pgfpathrectangle{\pgfqpoint{0.765000in}{0.660000in}}{\pgfqpoint{4.620000in}{4.620000in}}%
\pgfusepath{clip}%
\pgfsetrectcap%
\pgfsetroundjoin%
\pgfsetlinewidth{1.204500pt}%
\definecolor{currentstroke}{rgb}{0.000000,0.000000,0.000000}%
\pgfsetstrokecolor{currentstroke}%
\pgfsetdash{}{0pt}%
\pgfpathmoveto{\pgfqpoint{3.113634in}{3.679402in}}%
\pgfpathlineto{\pgfqpoint{2.472452in}{3.309215in}}%
\pgfusepath{stroke}%
\end{pgfscope}%
\begin{pgfscope}%
\pgfpathrectangle{\pgfqpoint{0.765000in}{0.660000in}}{\pgfqpoint{4.620000in}{4.620000in}}%
\pgfusepath{clip}%
\pgfsetrectcap%
\pgfsetroundjoin%
\pgfsetlinewidth{1.204500pt}%
\definecolor{currentstroke}{rgb}{0.000000,0.000000,0.000000}%
\pgfsetstrokecolor{currentstroke}%
\pgfsetdash{}{0pt}%
\pgfpathmoveto{\pgfqpoint{3.113634in}{3.679402in}}%
\pgfpathlineto{\pgfqpoint{3.860367in}{3.248275in}}%
\pgfusepath{stroke}%
\end{pgfscope}%
\begin{pgfscope}%
\pgfpathrectangle{\pgfqpoint{0.765000in}{0.660000in}}{\pgfqpoint{4.620000in}{4.620000in}}%
\pgfusepath{clip}%
\pgfsetrectcap%
\pgfsetroundjoin%
\pgfsetlinewidth{1.204500pt}%
\definecolor{currentstroke}{rgb}{0.000000,0.000000,0.000000}%
\pgfsetstrokecolor{currentstroke}%
\pgfsetdash{}{0pt}%
\pgfpathmoveto{\pgfqpoint{3.113634in}{3.679402in}}%
\pgfpathlineto{\pgfqpoint{3.113634in}{5.290000in}}%
\pgfusepath{stroke}%
\end{pgfscope}%
\end{pgfpicture}%
\makeatother%
\endgroup%
}
        }
        \caption{Toy multiclass dataset, cubic}
        \label{fig:toy_reverse}
    \end{subfigure}
    
    \caption{Tropical piecewise linear classifiers}
    \label{fig:plots}
\end{figure}

\subsection*{Contribution}

We formulate the tropical classification problem as a mean-payoff game. We apply a Krasnoselskii-Mann iteration scheme to derive an $\epsilon$-approximation of the tropical classifier after ??? iterations. We prove the optimality of the margin in the separable case, and develop geometric guarantees in the opposite case. Finally, we propose a feature map for fitting tropical polynomials while retaining margin information. Thanks to their evaluation speed and the complexity of their piecewise linear hypersurfaces, we pave the way for complex classification tasks.

\newpage
\section{Tropical Support Vector Machines}\label{Sec2}

\subsection{Preliminaries on mean payoff games}

In this section, we have a finite set of points $X=(x_{1},\ldots,x_{p})\in\mathbb{R}_{\text{max}}^{d\times p}$.
We define the \emph{tropical span} of $X$ as the set of tropical
linear combinations of these points:
\[
\text{Span}(X):=\left\{\max_{1\le i\le p}(x_{i}+\lambda_{i}),\quad\lambda\in\mathbb{R}^{p}\right\}.
\]
As this is also the smallest tropically convex set containing these
points, separating several point clouds amounts to separating their
tropical spans. Given a tropically convex, compact and nonempty subset $\mathcal{V}$
of $\mathbb{R}_{\max}^{d}$, we define the projection of any vector
$x$ in $\mathbb{R}_{\max}^{d}$ by:
\[
P_{\mathcal{V}}(x):=\max\{y\in \mathcal{V},\quad y\le x\}.
\]
When $\mathcal{V}$ is the tropical span of $X$, we will also note
this projection as $P_{X}$. We know from Maclagan et al. \cite{Maclagan2015} that tropical projections
can be expressed as a mean payoff game: for $i\in[p]$ and $x\in\mathbb{R}_{\max}^{d}$,
\begin{equation}
P_{X}(x)_{i}=\max_{j\in[p]}\left\{X_{ij}+\min_{1\le k \le d}(-X_{kj}+x_{k})\right\}.\label{eq:projector}
\end{equation}

Here, we recognize the payoff of a zero-sum deterministic game with perfect
information. There are two players, ``Max'' and ``Min'' (the maximizer
and the minimizer), who alternate their actions. Starting on node
$i$, Player Max chooses to move to node $j$, and receives $X_{ij}$
from Player Min. Similarily, Player Min in turn chooses a node $k$
and has to pay $-X_{kj}$ to Player Max.

We now recall some tools from Perron-Frobenius theory, in relation
with mean payoff games. We refer the reader to \cite{AKIAN2012} for more
information. We call \emph{Shapley operator} a map $T:\mathbb{R}_{\max}^{d}\rightarrow\mathbb{R}_{\max}^{d}$
that is order preserving, continuous, and such that $T(\alpha+x)=\alpha+T(x)$
for all $\alpha\in\mathbb{R}_{\max}$ and $x\in\mathbb{R}_{\max}^{d}$. $T$ is said to be \emph{diagonal free} when $T_{i}(x)$
is independent of $x_{i}$ for all $i\in\{1,\ldots, n\}$, i.e. when for all $x,y\in\mathbb{R}_{\max}$ such that $x_{j}=y_{j}$
for all $j\neq i$, we have $T_{i}(x)=T_{i}(y)$. 

Observe that the tropical projection defined by \ref{eq:projector}
is a Shapley operator. Given a Shapley operator $T$, we also define
\[
\mathcal{S}(T):=\left\{x\in\mathbb{R}_{\max}^{d},\quad x\le T(x)\right\}.
\]
More specifically, any tropically convex set $\mathcal{V}$ is equal
to $\mathcal{S}(P_{\mathcal{V}})$. Indeed, as for any $x\in\mathbb{R}_{\text{max}}^{d}$,
$P_{\mathcal{V}}(x)\le x$, having $x\in\mathcal{S}(P_{\mathcal{V}})$
implies $x=P_{\mathcal{V}}(x)$, meaning $x$ is in
$\mathcal{V}$.


Operator $P$ can be slightly tweaked to make it \emph{diagonal-free} -- but the resulting operator isn't a projector anymore.
In the modified game, the opponent is prevented from replying eye-for-an-eye:
\[
\left[P^{\text{DF}}_X(x)\right]_{i}:=\max_{j\in[p]}\left\{ X_{ij}+\min_{k\ne i}(-X_{kj}+x_{k})\right\} .
\]

Noticeably, $\mathcal{S}(P_X^\text{DF}) = \mathcal{S}(P_X)$.

\subsection{Hard-margin binary SVM}

In this section, we introduce a practical criterion for the existence of a separating tropical hyperplane, and if it exists, we construct one with optimal margin. We also place ourselves in the binary setting, where we seek to two separate convex hulls $\mathcal{V}^{+}$
and $\mathcal{V}^{-}$. We assume we have two Shapley operators $T^{+}$
and $T^{-}$ such that $\mathcal{S}(T^{\pm})=\mathcal{V}^{\pm}$,
which is for instance the case when separating finite point clouds:
corresponding operators are projections $P_{\mathcal{V}^{\pm}}$. We define:
\[
T=\inf(T^{+},T^{-}),
\]
which is also a Shapley. We denote by $\perp$ the vector of $\mathbb{R}_{\max}^{d}$
identically equal to $-\infty$. The \emph{spectral radius} of $T$
is, by definition:
\[
\rho(T)=\sup\left\{\lambda\in\mathbb{R}_{\max},\quad\exists u\in\mathbb{R}_{\max}^{d}\backslash\{\perp\},\quad T(u)=\lambda+u\right\}.
\]

Observe that $\mathcal{S}(T)$ is equal to $\mathcal{S}(T^{+})\cap\mathcal{S}(T^{-})$,
i.e. the intersection of the convex hulls we are seeking to separate.
From Allamigeon, Gaubert et al. \cite{Allamigeon2018}, we know that $\mathcal{V}^{+}$
and $\mathcal{V^{-}}$ being disjoint is equivalent to $\rho(T)\le0$.

Given $a$ the eigenvector corresponding to $\rho(T)$, we define the sectors:
\[
I^{\pm}:=\left\{i\in\{1,\ldots, d\},\quad T^{\pm}(a)_{i}>\rho(T)+a_{i}\right\},
\]

TODO : RÉGLER QUESTION DE INF SUR LES SECTEURS ET ENVOI DE L'APEX A +INF

and the corresponding configuration $\sigma=\{I^{+},I^{-}\}$.
\begin{prop}
Hyperplane $\mathcal{H}_{a}^{\sigma}$, given the sectors defined
above, separates $\mathcal{V}^{+}$ and $\mathcal{V}^{-}$ with a
margin of $-\rho(T)$. Moreover, this margin is optimal when $T^{\pm}$
are of the form
\[
\sup_{v\in\mathcal{V}^{\pm}}\left(v_{i}+\min(-v+x)\right),
\]
which is in particular the case when separating finite point clouds.
\end{prop}

\begin{proof}
As $T^{\pm}$ is non-expansive, for $x^{\pm}\in V^{\pm}$:
\[
x_{i}^{\pm}\le T^{\pm}(x^{\pm})_{i}=\left(T^{\pm}(x^{\pm})-T^{\pm}(a)\right)_{i}+T^{\pm}(a)_{i},
\]
hence 
\begin{equation}
x_{i}^{\pm}\le\max(x^{\pm}-a)+T^{\pm}(a)_{i}.\label{eq41}
\end{equation}
For instance, let $i\in I^{-}$. Then $T^{+}(a)_{i}=\rho(T)+a_{i}$,
so for $x^{+}\in \mathcal{V}^{+}$, using equation \ref{eq41}:
\[
x_{i}^{+}-a_{i}\le\max(x^{+}-a)+\rho(T).
\]
In particular, $x_{i}^{+}-a_{i}<\max(x^{+}-a)$ and any element of
$\mathcal{V}^{+}$ cannot belong to any of sectors in $I^{-}$ with
respect to $\mathcal{H}_{a}$, from which the sectors $I^{\pm}$ are well-defined.
Finally, 
\[
d_H(\mathcal{S}^-,x^{+})=\max(x^{+}-a)-\max(x^{+}-a)_{I^{-}}\ge-\rho(T),
\]
hence the margin.

Let's finally prove that the margin is optimal in the case where $T^{\pm}$
are of the aforementioned form.
Let $i\in I^{-}$. Then, for $\varepsilon>0$, we can
find $v\in \mathcal{V}^{+}$ such that 
\[
T^{+}(a)_{i}-\varepsilon\le v_{i}-\max(v-a)\le T^{+}(a)_{i},
\]
giving us 
\[
\rho(T)-\varepsilon\le v_{i}-a_{i}-\max(v-a)\le\rho(T).
\]
Maximizing over all $i\in I^{-}$ gives that $v$ is
at most at distance $-\rho(T)+\varepsilon$ of $\mathcal{H}_{a}^{\sigma}$, hence
the optimality.
\end{proof}
%
To build a practical algorithm, we still need to compute the spectral
radius of $T$. As it amounts to solving mean-payoff games, we can apply the following Krasnoselskii-Mann iteration scheme,
given any initial point $a^{0}$:
\[
\begin{cases}
z^{k+1} & =\frac{1}{2}\left(a^{k}+T(a^{k})\right)\\
a^{k+1} & =z^{k+1}-\max_{1\le i\le d}z_{i}^{k+1}
\end{cases}
\]
The eigenpair we search is $(a^{\infty},2\cdot\max_{1\le i\le d}z_{i}^{\infty})$.
Example results of this algorithm on a toy dataset are shown in Figure
\ref{fig:MaxMargin}.

TODO : TALK ABOUT EFFICIENCY

\subsection{Geometric approach for soft-margin SVM}

We seek to separate convex hulls $\mathcal{V}^{+}$ and $\mathcal{V}^{-}$.
Assuming that the data overlap, we want to have a sense of their
non-separability. Using the operator $T$ we previously defined,  $\rho(T)$ is the
(positive) inner radius of $\mathcal{S}(T)$ . \cite{Allamigeon2018} , i.e. the supremum of
the radii of the Hilbert balls contained in the intersection between
$\mathcal{V}^{+}$ and $\mathcal{V}^{-}$. The next Proposition shows
us that this measure can be interpreted geometrically, in the case
where $\mathcal{V}^{+}$ and $\mathcal{V}^{-}$ are generated by point
clouds $X^{+}$ and $X^{-}$:
\begin{prop}
By moving some points from $X^{+}$ and $X^{-}$ by a distance of
at most $\rho(T)$, we can nullify the interior of the intersection
of the tropical spans.

More specifically, we note $a\in\mathbb{R}_{\max}^{d}$ the eigenvector
corresponding to $\rho(T)$. We project all points of $X^{+}$ and
$X^{-}$ whose distance to $\mathcal{H}_{a}$ is less than $\rho(T)$,
onto $\mathcal{H}_{a}$. Then the intersection of new convex hulls
is of empty interior.
\end{prop}

\begin{proof}
Let's denote $X$ the cloud consisting of all points (regardless of
sign), and for $x_{j}\in X$, we note $s_{j}$
its sector and $d_{j}$ the second argmax of $(x_{j}-a)$. For each
point $x_{j}=X_{\cdot j}\in X$ at distance less than $\rho(T)$ of
$\mathcal{H}_{a}$, the transformation consists in setting 
\[
W_{kj}:=\begin{cases}
X_{kj} & \text{if \ensuremath{k\ne s_{j}}}\\
X_{s_{j}j}-d_H(x_{j},\mathcal{H}_{a}) & \text{at \ensuremath{s_{j}}}
\end{cases}
\]
so that $w_{j}$ is projected on the hyperplane. As $T^{\pm}$ is
non-expansive and diagonal-free, let's remark that for $x^{\pm}\in V^{\pm}$:
\[
x_{i}^{\pm}\le T^{\pm}(x^{\pm})_{i}=\left(T^{\pm}(x^{\pm})-T^{\pm}(a)\right)_{i}+T^{\pm}(a)_{i},
\]
hence
\begin{equation}
x_{i}^{\pm}\le\max(x^{\pm}-a)_{\ne i}+T^{\pm}(a)_{i}.\label{eq31}
\end{equation}
Let $1\le i\le d$. If $x_{j}\in X$ is not in the $i$-th sector, then
for $k$ different from the sector of $x_{j}$, by definition: 
\[
(w_{j}-a)_{k}\le(x_{j}-a)_{d_{j}}\le(w_{j}-a)_{s_{j}},
\]
hence 
\[
W_{ij}-\max_{k\ne i}\left(W_{kj}-a_{k}\right)\le a_{i}.
\]
Otherwise, $x_{j}$ is in the $i$-th sector and: 
\[
\max_{k\ne i}\left(W_{kj}-a_{k}\right)=X_{d_{j}j}-a_{d_{j}},
\]
thus 
\[
W_{ij}-\max_{\ne i}\left(w_{j}-a\right)=\left(w_{j}-a\right)_{s_{j}}-\left(x_{j}-a\right)_{d_{j}}+a_{s_{j}}\ge a_{s_{j}}=a_{i},
\]
with equality iff $d_H(x_{j},\mathcal{H}_{a})\leq\rho(T)$.

Suppose by symmetry that $T(a)_{i}=T^{+}(a)_{i}=\rho(T)+a_{i}.$ We
also have $T^{-}(a)_{i}\ge\rho(T)+a_{i}$. Then, using the proof of
Theorem 22 in \cite{Akian2021TropicalLR}, we know that there exists
$j^{+},j^{-}\in[p]$ such that $x_{j^{+}}\in X^{+}$ and $x_{j^{-}}\in X^{-}$
are in sector $i$, with $x_{j^{+}}$ being at distance $\rho(T)$
from $\mathcal{H}_{a}$ and $x_{j^{-}}$ at distance greater than $\rho(T)$.
Therefore, $W_{ij^{+}}-\max_{\ne i}\left(w_{j^{+}}-a\right)=a_{i}$
and $W_{ij^{-}}-\max_{\ne i}\left(w_{j^{-}}-a\right)\geq a_{i}$.
Moreover, for any $j$ such that $x_{j}\in X^{+}$ is in sector $i$,
eq. \ref{eq31} gives $d_H(x_{j},\mathcal{H}_{a})\leq\rho(T)$.

Let $Q^{\pm}$ be the \emph{diagonal-free} projections over transformed
point clouds, and $Q=\inf(Q^{+}, Q^{-}).$ We've just shown that $Q(a)_{i}=Q^{+}(a)_{i}=a_{i}$,
and finally $Q(a)=a$.
\end{proof}
%
Thus, the spectral radius can be interpreted as a bound on the distance
under which some points have to be moved separate the interiors of convex hulls.

As $a$ is the center of the inner ball of $\mathcal{V}^{+}\cap\mathcal{V}^{-}$,
we also have a purely geometric heuristic for determining a
hyperplane in the non-separable case: we simply take the hyperplane
$\mathcal{H}_{a}$ and then assign the sectors based on dominant
population, for instance.


\subsection{Hard-margin multiclass classification}

In this section, we consider convex hulls of point clouds of $n$
classes, noted $\mathcal{V}^{k}$ for $1\le k \le n$, and described by Shapley operators
$T^{k}$. We give a simple criteria for these classes to be tropically separable in their whole, meaning that there exists a tropical signed
hyperplane such that each $\mathcal{V}^{k}$ belong to sectors of $I^{k}\subset\{1,\ldots, d\}$
with $I^{k}\cap I^{l}=\emptyset$ for $k\ne l\in\{1, \ldots, n\}.$ We then adapt
the previous results to this case.

We now consider the Shapley operator
\[
T:=\sup_{1\le k<l\le n}\inf(T^{k}, T^{l}).
\]

Morally, we can think of the union of 2 to 2 intersections of data classes, although the sup operator does not strictly speaking correspond to a union. We also remark that when $n=2$, $T$ is the same as previously
defined.

Let $(a,\rho(T))$ the eigenpair of $T$ approximated by the Kranoselskii-Mann
algorithm, verifying $T(a)=\rho(T)+a$. We define the sectors:
\[
I^{k}:=\{1\le i\le d,\quad T^{k}(a)_{i}>\rho(T)+a_{i}\}.
\]

Given the definition of $T$, it is clear that $I^{k}\cap I^{l}=\emptyset$
for $1\le k\ne l\le n,$ and we have the following result :
\begin{prop}
If $\rho(T)<0$, the tropical parametrized hyperplane $\mathcal{H}_{a}^{\sigma}$,
given the configuration $\sigma=\{I^{k}\}_{1\le k\le n}$, separates $\mathcal{V}^{k}$
and $\mathcal{V}^{l}$ for all $1 \le k\ne l \le n$ with a margin of at least $-\rho(T)$.
Moreover, this margin is exact in the case where all $T^{k}$ are
of the form
\[
T^{k}(x)=P_{\mathcal{V}^{k}}(x)=\sup_{v\in \mathcal{V}^{k}}\left(v_{i}+\min(-v+x)\right),
\]
 which is in particular the case when separating finite point clouds. 
\end{prop}

\begin{proof}
We consider class $k\in\{1,\ldots, n\}$ and sector $i\in I^\ell$, $\ell \ne k$. Using the same reasoning
as in the proof of Proposition \ref{TODO:CITE}, we obtain that for all $v^{k}\in \mathcal{V}^{k}$,
\[
d_H(\mathcal{S}^\ell,v^{k})=\max(v^{k}-a)-\max_{j\in I^{\ell}}(v^{k}-a)_j\ge-\rho(T).
\]
We now prove the margin exactness when all $T^{k}$ are of the aforementioned form. 

Let $i\in\{1,\ldots, d\}$. As $T(a)=\rho(T)+a$, there are two distinct classes $k$ and $l$ such that  $\inf(T^{k}(a)_{i}, T^{l}(a)_{i})=\rho(T)+a_{i}$. By symmetry, suppose that $T^k(a)_i=\rho(T)+a_i$, meaning that $i\in I_\ell \subset \bigsqcup_{p\ne k} I_p$. Using the form of $T^k(a)_i$ and adapting the ideas of Proposition \ref{TODO:CITE}, for $\varepsilon > 0$, we can find $v\in \mathcal{V}^k$ such that $$\max(v^{k}-a)-(v_{i}^{k}-a_{i})\le-\rho(T)+\varepsilon,$$ which means that 
\[
d_H(\mathcal{S}^\ell,\mathcal{V}^{k})\le \max(v^{k}-a)-\max_{j\in I_\ell}(v^{k}-a)_j \le \max(v^{k}-a) - (v^{k}_i - a_i) \le -\rho(T)+\varepsilon.
\]

Hence, each sector is at distance at most $-\rho(T)$ from one point of another class: the margin is exact.
\end{proof}

\section{Tropical Polynomials as Piecewise Linear Classifiers}

In this section, we extend our previous approach to piecewise linear
fitting, using tropical polynomials. 

A \emph{tropical polynomial} \emph{function} over $\mathbb{R}_{\max}^{d}$
is defined by
\[
f(x)=\bigoplus_{\alpha\in\mathbb{Z}^{d}}f_{\alpha}x^{\alpha}=\max_{\alpha\in\mathbb{Z}^{d}}\left(f_{\alpha}+\langle x,\alpha\rangle\right),
\]
where $f_{\alpha}=-\infty$ except for finitely many values of $\alpha$,
and with $\langle x,\alpha\rangle=\sum_{i}x_{i}\cdot\alpha_{i}$ under
the convention $(-\infty)\cdot0=0$. The \emph{tropical hypersurface} associated with $f$ is the set of points from
$\mathbb{R}_{\max}^{d}$ where the maximum in $f$ is attained twice.
Tropical hypersurfaces are piecewise linear and divide $\mathbb{R}_{\max}^{d}$
between sectors corresponding to the maximal monomial: they can perform
more complex classification.

If $\mathcal{A}\subset\mathbb{Z}^{d}$ is a set of vectors, we define
the \emph{Veronese embedding} of $x$ as:
\[
\text{ver}_{\mathcal{A}}(x):=\left(\langle x,\alpha\rangle\right)_{\alpha\in\mathcal{A}}\in\mathbb{R}_{\max}^{\mathcal{A}},
\]
allowing us to map our point space into a larger space, made up of
various integer combinations of features. To take a fairly exhaustive
subset of monomials, we consider the integer combinations defined
by the integer points of the dilated simplex. Noting $s\in\mathbb{N}^{*}$
a scale parameter, we define
\[
\mathcal{A}_{s}:=\left\{x\in\mathbb{N}^{d},\quad x_{1}+\cdots+x_{d}=s\right\},
\]
which grows polynomially with $s$. The Veronese embedding defined
by $\mathcal{A}_{s}$ is invertible and if $\mathcal{H}_a$ is a tropical
hyperplane of apex $a$ in $\mathbb{R}_{\max}^{\mathcal{A}}$, $\text{ver}_{\mathcal{A}_{s}}^{-1}(\mathcal{H})$
is a tropical polynomial hypersurface in the initial space, corresponding to the polynomial
\[ f_{-a, \mathcal{A_s}}(x)=\bigoplus_{\alpha\in \mathcal{A}_s} (-a_\alpha)x^\alpha \]
Hence, $\text{ver}_{\mathcal{A}_{s}}$ is a feature map
enabling us to perform piecewise linear classification in $\mathbb{R}_{\max}^{d}$. This brings us close to the framework of neural networks with piecewise linear activations, e.g. relaxed linear units \cite{zhang2018tropical}. $\mathcal{A}_1$ corresponds to tropical hyperplanes: we are generalizing the approach developed in Section \ref{sec2}.

The initial metric can be easily linked to the metric in the augmented space: for $x\in\mathbb{R}_{\max}^d$, we have
\begin{align*}
\lVert\text{ver}(x)\rVert_{H,\mathbb{R}^{\mathcal{A}_s}_{\max}}&=\max_{\alpha\in\mathcal{A}_{s}}\sum_{i}\alpha_{i}x_{i}-\min_{\alpha\in\mathcal{A}^{s}}\sum_{i}\alpha_{i}x_{i}\\&\le s(\max x-\min x)=s\lVert x\rVert_{H,\mathbb{R}_{\max}^d}.
\end{align*}

We deduce the following result:
\begin{prop}
If data is separable in the augmented space, noting $(a, \rho(T))$ the eigenpair of $T$ applied to $\mathbb{R}_{\max}^s$, data is separable in the initial space by the tropical hypersurface of $f_{-a,\mathcal{A}^s}$ with a margin of at least $-\rho(T)/s$.
\end{prop}

As the augmented data is scaled by a factor of $s$, $\rho(T)$ should often grow linearly with $s$, meaning we have a good margin guarantee at the end.

In Figures \ref{fig:iris}, \ref{fig:circular} and \ref{fig:toy_reverse}, we visualize the decision boundaries of the piecewise linear classifier. Dotted lines in light gray indicate merged sectors after assignment, and margin lower estimation is represented by the Hilbert balls around each point.

These methods obviously work in higher dimensions, but we can only visualize them in lower ones. Note that in the multi-class case, where classes are not separable in pairs, the result is consistent, although we have not provided a theoretical guarantee in this case.

\subsection*{Complexity}

The learning speed depends essentially on the dimension of the feature space, i.e. the starting dimension and the number of monomials considered. Although our iteration scheme is efficient, given that $$|\mathcal{A}^s| = \binom{s+d-1}{s},$$ the complexity grows very quickly, opening the way to feature selection issues for directly learning \emph{sparse} polynomials.

On the other hand, when the tropical polynomial is trained, since only active monomials are kept, inference is $\mathcal{O}(rd)$, where $r$ is the number of monomials considered, which is very fast.

TODO : ON LOG PAPER (quelques mots)

\bigskip
\bigskip
\bibliographystyle{plain}
\bibliography{ea}

\end{document}
