\documentclass[oneside,UKenglish,a4paper]{amsart}
\usepackage{bbold}
\usepackage[T1]{fontenc}
\usepackage{inputenc}
\setlength{\parskip}{\smallskipamount}
\setlength{\parindent}{0pt}
\usepackage{amstext}
\usepackage{amsthm}
\usepackage{amsmath}
\usepackage{pgfplots}
\usepackage{amssymb}
\usepackage{graphicx}
\usepackage{algorithm}
\usepackage{hyperref}
\usepackage{algpseudocode}
\usepackage{multirow}
\usepackage{subcaption}
\usepackage{todonotes}
\usepackage{wasysym}
\usepackage{fullpage}
\usepackage{adjustbox}
\usepackage{mathtools}


\makeatletter
\numberwithin{equation}{section}
\numberwithin{figure}{section}
\theoremstyle{plain}
\newtheorem{thm}{\protect\theoremname}
\theoremstyle{definition}
\newtheorem{defn}[thm]{\protect\definitionname}
\theoremstyle{plain}
\newtheorem{prop}[thm]{\protect\propositionname}
\theoremstyle{remark}
\newtheorem{rem}[thm]{\protect\remarkname}
\theoremstyle{plain}
\newtheorem{lem}[thm]{\protect\lemmaname}
\theoremstyle{definition}
\newtheorem{example}[thm]{\protect\examplename}
\newtheorem{cor}[thm]{\protect\corollaryname}
\theoremstyle{definition}


\floatname{algorithm}{Algorithm}

\makeatother

\usepackage{babel}
\providecommand{\definitionname}{Definition}
\providecommand{\lemmaname}{Lemma}
\providecommand{\propositionname}{Proposition}
\providecommand{\remarkname}{Remark}
\providecommand{\theoremname}{Theorem}
\providecommand{\examplename}{Example}
\providecommand{\corollaryname}{Corollary}

\newcommand{\Input}{\textbf{input}}


\begin{document}
\title{Tropical Support Vector Machines and mean payoff games}
\author{Xavier Allamigeon, Stéphane Gaubert, Samuel Boïté, Théo Molfessis}
\maketitle

\begin{abstract}
    We introduce a pseudo-polynomial algorithm for tropical support vector machines (SVMs), using mean payoff games for binary and multiclass classification. We provide a simple numerical criterion for tropical separability. In the separable case, we construct a separating tropical hyperplane, with margin guarantees. In the non-separable case, we use the same algorithm to provide a suitable classifier ; it also measures the distance to tropical separability and allows to geometrically transform the data to reduce this distance to zero. We then extend our algorithm to the multiclass case. Finally, we study a feature map method to make the data separable in higher dimensions and propose a heuristic for sparsely adding relevant dimensions to better separate the data. 
\end{abstract}


\section{Introduction}

\subsection*{Motivation and context}
\emph{Support Vector Machines} (SVMs), introduced by Boser et al. (1992) \cite{Boser92}, are used in data science to determine separating hyperplanes for binary classification tasks.

Tropical geometry recently found applications in machine learning: tropical linear regression (Akian et al., 2021) \cite{Akian2021TropicalLR}, analysis of deep neural networks and parametric statistical models (Maragos et al., 2021) \cite{Vasileios2021} and principal component analysis (Page et al., 2019) \cite{YoshidaPCA2019}.

First considered by Gärtner and Jaggi (2006) \cite{Gartner2006}, \emph{Tropical Support Vector Machines} classify data points with a tropical hyperplane and using the tropical metric. Tropical SVMs find natural applications in phylogenomics (Tang et al., 2020) \cite{Tang2020}.

Initially formulated as linear programs \cite{Gartner2006}, tropical hyperplanes can also be obtained by tropicalizing a classical hyperplane using \emph{Maslov dequantization} \cite{litvinov2005maslov} or Viro's method \cite{viro2000dequantization}. Yet, redundant features can cause difficulties in the performance of $L^2$-norm classical SVMs (Zhang and Zhou, 2010) \cite{ZHANG2010373}. Moreover, tropicalization relies on exponentialization and is subject to arithmetic errors, while not optimizing the margin directly in the initial space.

Tropical linear regression has been shown by Akian et al. (2021) \cite{Akian2021TropicalLR} as equivalent to solving mean payoff games. We consider this parallel for Tropical Support Vector Machines.

\subsection*{The tropical linear classification problem}

The tropical \emph{max-plus} semifield $\mathbb{R}_{\max}$ is the
set of real numbers, completed by $-\infty$ and equipped with the
addition $a\oplus b=\max(a,b)$ and the multiplication $a\odot b=a+b$.
A \emph{tropical hyperplane} in the $d$-dimensional tropical vector
space $\mathbb{R}_{\max}^{d}$ is a set of vectors of the form
\begin{equation}
\mathcal{H}_{a}=\left\{x\in\mathbb{R}_{\max}^{d},\quad\max_{1\le i\le d}(a_{i}+x_{i})\,\text{is achieved at least twice}\right\}.\label{eq:unsigned}
\end{equation}
Such a hyperplane is parameterized by the vector $a=(a_{1},\ldots,a_{d})\in\mathbb{R}_{\max}^{d}$,
which is required to be non-identically $-\infty$. It hence partitions
the space depending on which coordinate reaches its maximum, laying the
ground for $d$-class classification. By definition, tropical hyperplanes are invariant by translation along the constant vector $(1,\ldots, 1)$.

In this paper, we address the following tropical analogue of support
vector machines. Given $n\in\{1,\ldots, d\}$ classes of $d$-dimensional data
points $X^{1},\ldots,X^{p}$, we look for a maximum-margin separating
tropical hyperplane to build a classifier. The notion of margin depends
on the metric: the canonical choice in tropical geometry is the (additive
version of) Hilbert's projective metric. Its restriction to $\mathbb{R}^{d}$
is induced by the so-called \emph{Hilbert's seminorm} or \emph{Hopf
oscillation}
\[
\lVert x\rVert_{H}\coloneqq\max_{1\le i\le d}x_{i}-\min_{1\le i\le d}x_{i}.
\]
It is a projective metric, in the sense that the distance between
two points is zero if and only if these two points differ by an additive
constant. When dealing with hard-margin support vector machines (SVMs),
we also have to ensure that every point lands in the right sector. Each class $k$ belongs to some sectors $I^k\subset \{1,\ldots, d\}$, with each sector assigned to only one class. We then define the \emph{tropical parameterized hyperplane} of configuration
$\sigma=\left\{I^{1},\ldots,I^{n}\right\}$, where $I^{k}$ and $I^{\ell}$
are disjoint for $k\ne\ell$, as:
\[
\mathcal{H}\coloneqq\left\{x\in\mathbb{R}_{\max}^{d},\quad\exists k\ne\ell,\quad (x-a)\,\text{reaches its max coordinate in \ensuremath{I^{k}} and \ensuremath{I^{\ell}}}\right\}.
\]
We note $\mathcal{S}^k\subset \mathbb{R}_{\max}^d$ the \emph{sector} classified as class $k$, where the maximum coordinate is reached in $I^k$. $\mathcal{H}$ is said to \emph{separate} point
clouds $(X^{k})_{1\le k\le p}$ with a margin of $\nu$ when for every point
$x^{k}$ of class $k$:
\begin{enumerate}
\item $x^{k}$ is on the right sector of the hyperplane, i.e its maximum
coordinate is reached in $I^{k}$
\item $x^{k}$ is at distance at least $\nu$ from any other sector $\mathcal{S}^\ell$, $l \ne k$: %$\mathcal{H}_{a}^{\sigma}$:
\begin{equation*}
d_H(\mathcal{S}^\ell,x^{k})\coloneqq\max(x^{k}-a)-\max_{i\in I^\ell}(x^{k}-a)_i\ge\nu.
\end{equation*}
\end{enumerate}


In Figure \ref{fig:MaxMargin}, we separate two classes
of $3$-dimensional points from a toy tropical dataset using a tropical
hyperplane. The two lower sectors are assigned to one class, the upper sector corresponds to the other class. Its margin is maximal and is represented by the Hilbert
ball in gray. The figure is represented in the projective space $\mathbb{P}\left(\mathbb{R}_{\text{max}}^{3}\right)$,
i.e. the quotient of the set of non-identically $-\infty$ vectors
of $\mathbb{R}_{\text{max}}^{3}$ by the equivalence relation which
identifies tropically proportional values.

Since our separator is translation-invariant by the constant vector, in order to separate any $d$-dimensional dataset, we plunge it into the $x_1+\cdots+x_{d+1}=0$ subspace of $\mathbb{R}_{\max}^{d+1}$. This amounts to artificially adding a new coordinate to each point, equal to the opposite of the sum of the initial coordinates. In the rare case where we are dealing with an intrinsically tropical dataset, for which this translation invariance makes sense, it is better to apply the tropical classifier on the initial features.

\begin{figure}[!h]
    \centering
    \begin{subfigure}{0.55\textwidth}
        \centering
        \resizebox{0.5\textwidth}{!}{%
        \centering
    \clipbox{0.35\width{} 0.25\height{} 0.25\width{} 0.25\height{}}{%% Creator: Matplotlib, PGF backend
%%
%% To include the figure in your LaTeX document, write
%%   \input{<filename>.pgf}
%%
%% Make sure the required packages are loaded in your preamble
%%   \usepackage{pgf}
%%
%% Also ensure that all the required font packages are loaded; for instance,
%% the lmodern package is sometimes necessary when using math font.
%%   \usepackage{lmodern}
%%
%% Figures using additional raster images can only be included by \input if
%% they are in the same directory as the main LaTeX file. For loading figures
%% from other directories you can use the `import` package
%%   \usepackage{import}
%%
%% and then include the figures with
%%   \import{<path to file>}{<filename>.pgf}
%%
%% Matplotlib used the following preamble
%%   
%%   \usepackage{fontspec}
%%   \setmainfont{DejaVuSerif.ttf}[Path=\detokenize{/Users/sam/Library/Python/3.9/lib/python/site-packages/matplotlib/mpl-data/fonts/ttf/}]
%%   \setsansfont{DejaVuSans.ttf}[Path=\detokenize{/Users/sam/Library/Python/3.9/lib/python/site-packages/matplotlib/mpl-data/fonts/ttf/}]
%%   \setmonofont{DejaVuSansMono.ttf}[Path=\detokenize{/Users/sam/Library/Python/3.9/lib/python/site-packages/matplotlib/mpl-data/fonts/ttf/}]
%%   \makeatletter\@ifpackageloaded{underscore}{}{\usepackage[strings]{underscore}}\makeatother
%%
\begingroup%
\makeatletter%
\begin{pgfpicture}%
\pgfpathrectangle{\pgfpointorigin}{\pgfqpoint{6.000000in}{6.000000in}}%
\pgfusepath{use as bounding box, clip}%
\begin{pgfscope}%
\pgfsetbuttcap%
\pgfsetmiterjoin%
\definecolor{currentfill}{rgb}{1.000000,1.000000,1.000000}%
\pgfsetfillcolor{currentfill}%
\pgfsetlinewidth{0.000000pt}%
\definecolor{currentstroke}{rgb}{1.000000,1.000000,1.000000}%
\pgfsetstrokecolor{currentstroke}%
\pgfsetdash{}{0pt}%
\pgfpathmoveto{\pgfqpoint{0.000000in}{0.000000in}}%
\pgfpathlineto{\pgfqpoint{6.000000in}{0.000000in}}%
\pgfpathlineto{\pgfqpoint{6.000000in}{6.000000in}}%
\pgfpathlineto{\pgfqpoint{0.000000in}{6.000000in}}%
\pgfpathlineto{\pgfqpoint{0.000000in}{0.000000in}}%
\pgfpathclose%
\pgfusepath{fill}%
\end{pgfscope}%
\begin{pgfscope}%
\pgfsetbuttcap%
\pgfsetmiterjoin%
\definecolor{currentfill}{rgb}{1.000000,1.000000,1.000000}%
\pgfsetfillcolor{currentfill}%
\pgfsetlinewidth{0.000000pt}%
\definecolor{currentstroke}{rgb}{0.000000,0.000000,0.000000}%
\pgfsetstrokecolor{currentstroke}%
\pgfsetstrokeopacity{0.000000}%
\pgfsetdash{}{0pt}%
\pgfpathmoveto{\pgfqpoint{0.765000in}{0.660000in}}%
\pgfpathlineto{\pgfqpoint{5.385000in}{0.660000in}}%
\pgfpathlineto{\pgfqpoint{5.385000in}{5.280000in}}%
\pgfpathlineto{\pgfqpoint{0.765000in}{5.280000in}}%
\pgfpathlineto{\pgfqpoint{0.765000in}{0.660000in}}%
\pgfpathclose%
\pgfusepath{fill}%
\end{pgfscope}%
\begin{pgfscope}%
\pgfpathrectangle{\pgfqpoint{0.765000in}{0.660000in}}{\pgfqpoint{4.620000in}{4.620000in}}%
\pgfusepath{clip}%
\pgfsetbuttcap%
\pgfsetroundjoin%
\definecolor{currentfill}{rgb}{1.000000,0.894118,0.788235}%
\pgfsetfillcolor{currentfill}%
\pgfsetlinewidth{0.000000pt}%
\definecolor{currentstroke}{rgb}{1.000000,0.894118,0.788235}%
\pgfsetstrokecolor{currentstroke}%
\pgfsetdash{}{0pt}%
\pgfpathmoveto{\pgfqpoint{3.076409in}{3.135951in}}%
\pgfpathlineto{\pgfqpoint{2.657000in}{2.893805in}}%
\pgfpathlineto{\pgfqpoint{3.076409in}{2.651660in}}%
\pgfpathlineto{\pgfqpoint{3.495818in}{2.893805in}}%
\pgfpathlineto{\pgfqpoint{3.076409in}{3.135951in}}%
\pgfpathclose%
\pgfusepath{fill}%
\end{pgfscope}%
\begin{pgfscope}%
\pgfpathrectangle{\pgfqpoint{0.765000in}{0.660000in}}{\pgfqpoint{4.620000in}{4.620000in}}%
\pgfusepath{clip}%
\pgfsetbuttcap%
\pgfsetroundjoin%
\definecolor{currentfill}{rgb}{1.000000,0.894118,0.788235}%
\pgfsetfillcolor{currentfill}%
\pgfsetlinewidth{0.000000pt}%
\definecolor{currentstroke}{rgb}{1.000000,0.894118,0.788235}%
\pgfsetstrokecolor{currentstroke}%
\pgfsetdash{}{0pt}%
\pgfpathmoveto{\pgfqpoint{3.076409in}{3.135951in}}%
\pgfpathlineto{\pgfqpoint{2.657000in}{2.893805in}}%
\pgfpathlineto{\pgfqpoint{2.657000in}{3.378097in}}%
\pgfpathlineto{\pgfqpoint{3.076409in}{3.620242in}}%
\pgfpathlineto{\pgfqpoint{3.076409in}{3.135951in}}%
\pgfpathclose%
\pgfusepath{fill}%
\end{pgfscope}%
\begin{pgfscope}%
\pgfpathrectangle{\pgfqpoint{0.765000in}{0.660000in}}{\pgfqpoint{4.620000in}{4.620000in}}%
\pgfusepath{clip}%
\pgfsetbuttcap%
\pgfsetroundjoin%
\definecolor{currentfill}{rgb}{1.000000,0.894118,0.788235}%
\pgfsetfillcolor{currentfill}%
\pgfsetlinewidth{0.000000pt}%
\definecolor{currentstroke}{rgb}{1.000000,0.894118,0.788235}%
\pgfsetstrokecolor{currentstroke}%
\pgfsetdash{}{0pt}%
\pgfpathmoveto{\pgfqpoint{3.076409in}{3.135951in}}%
\pgfpathlineto{\pgfqpoint{3.495818in}{2.893805in}}%
\pgfpathlineto{\pgfqpoint{3.495818in}{3.378097in}}%
\pgfpathlineto{\pgfqpoint{3.076409in}{3.620242in}}%
\pgfpathlineto{\pgfqpoint{3.076409in}{3.135951in}}%
\pgfpathclose%
\pgfusepath{fill}%
\end{pgfscope}%
\begin{pgfscope}%
\pgfpathrectangle{\pgfqpoint{0.765000in}{0.660000in}}{\pgfqpoint{4.620000in}{4.620000in}}%
\pgfusepath{clip}%
\pgfsetbuttcap%
\pgfsetroundjoin%
\definecolor{currentfill}{rgb}{1.000000,0.894118,0.788235}%
\pgfsetfillcolor{currentfill}%
\pgfsetlinewidth{0.000000pt}%
\definecolor{currentstroke}{rgb}{1.000000,0.894118,0.788235}%
\pgfsetstrokecolor{currentstroke}%
\pgfsetdash{}{0pt}%
\pgfpathmoveto{\pgfqpoint{-3.358429in}{6.851106in}}%
\pgfpathlineto{\pgfqpoint{-3.777838in}{6.608961in}}%
\pgfpathlineto{\pgfqpoint{-3.358429in}{6.366815in}}%
\pgfpathlineto{\pgfqpoint{-2.939020in}{6.608961in}}%
\pgfpathlineto{\pgfqpoint{-3.358429in}{6.851106in}}%
\pgfpathclose%
\pgfusepath{fill}%
\end{pgfscope}%
\begin{pgfscope}%
\pgfpathrectangle{\pgfqpoint{0.765000in}{0.660000in}}{\pgfqpoint{4.620000in}{4.620000in}}%
\pgfusepath{clip}%
\pgfsetbuttcap%
\pgfsetroundjoin%
\definecolor{currentfill}{rgb}{1.000000,0.894118,0.788235}%
\pgfsetfillcolor{currentfill}%
\pgfsetlinewidth{0.000000pt}%
\definecolor{currentstroke}{rgb}{1.000000,0.894118,0.788235}%
\pgfsetstrokecolor{currentstroke}%
\pgfsetdash{}{0pt}%
\pgfpathmoveto{\pgfqpoint{-3.358429in}{6.851106in}}%
\pgfpathlineto{\pgfqpoint{-3.777838in}{6.608961in}}%
\pgfpathlineto{\pgfqpoint{-3.777838in}{7.093252in}}%
\pgfpathlineto{\pgfqpoint{-3.358429in}{7.335398in}}%
\pgfpathlineto{\pgfqpoint{-3.358429in}{6.851106in}}%
\pgfpathclose%
\pgfusepath{fill}%
\end{pgfscope}%
\begin{pgfscope}%
\pgfpathrectangle{\pgfqpoint{0.765000in}{0.660000in}}{\pgfqpoint{4.620000in}{4.620000in}}%
\pgfusepath{clip}%
\pgfsetbuttcap%
\pgfsetroundjoin%
\definecolor{currentfill}{rgb}{1.000000,0.894118,0.788235}%
\pgfsetfillcolor{currentfill}%
\pgfsetlinewidth{0.000000pt}%
\definecolor{currentstroke}{rgb}{1.000000,0.894118,0.788235}%
\pgfsetstrokecolor{currentstroke}%
\pgfsetdash{}{0pt}%
\pgfpathmoveto{\pgfqpoint{-3.358429in}{6.851106in}}%
\pgfpathlineto{\pgfqpoint{-2.939020in}{6.608961in}}%
\pgfpathlineto{\pgfqpoint{-2.939020in}{7.093252in}}%
\pgfpathlineto{\pgfqpoint{-3.358429in}{7.335398in}}%
\pgfpathlineto{\pgfqpoint{-3.358429in}{6.851106in}}%
\pgfpathclose%
\pgfusepath{fill}%
\end{pgfscope}%
\begin{pgfscope}%
\pgfpathrectangle{\pgfqpoint{0.765000in}{0.660000in}}{\pgfqpoint{4.620000in}{4.620000in}}%
\pgfusepath{clip}%
\pgfsetbuttcap%
\pgfsetroundjoin%
\definecolor{currentfill}{rgb}{1.000000,0.894118,0.788235}%
\pgfsetfillcolor{currentfill}%
\pgfsetlinewidth{0.000000pt}%
\definecolor{currentstroke}{rgb}{1.000000,0.894118,0.788235}%
\pgfsetstrokecolor{currentstroke}%
\pgfsetdash{}{0pt}%
\pgfpathmoveto{\pgfqpoint{3.076409in}{3.620242in}}%
\pgfpathlineto{\pgfqpoint{2.657000in}{3.378097in}}%
\pgfpathlineto{\pgfqpoint{3.076409in}{3.135951in}}%
\pgfpathlineto{\pgfqpoint{3.495818in}{3.378097in}}%
\pgfpathlineto{\pgfqpoint{3.076409in}{3.620242in}}%
\pgfpathclose%
\pgfusepath{fill}%
\end{pgfscope}%
\begin{pgfscope}%
\pgfpathrectangle{\pgfqpoint{0.765000in}{0.660000in}}{\pgfqpoint{4.620000in}{4.620000in}}%
\pgfusepath{clip}%
\pgfsetbuttcap%
\pgfsetroundjoin%
\definecolor{currentfill}{rgb}{1.000000,0.894118,0.788235}%
\pgfsetfillcolor{currentfill}%
\pgfsetlinewidth{0.000000pt}%
\definecolor{currentstroke}{rgb}{1.000000,0.894118,0.788235}%
\pgfsetstrokecolor{currentstroke}%
\pgfsetdash{}{0pt}%
\pgfpathmoveto{\pgfqpoint{3.076409in}{2.651660in}}%
\pgfpathlineto{\pgfqpoint{3.495818in}{2.893805in}}%
\pgfpathlineto{\pgfqpoint{3.495818in}{3.378097in}}%
\pgfpathlineto{\pgfqpoint{3.076409in}{3.135951in}}%
\pgfpathlineto{\pgfqpoint{3.076409in}{2.651660in}}%
\pgfpathclose%
\pgfusepath{fill}%
\end{pgfscope}%
\begin{pgfscope}%
\pgfpathrectangle{\pgfqpoint{0.765000in}{0.660000in}}{\pgfqpoint{4.620000in}{4.620000in}}%
\pgfusepath{clip}%
\pgfsetbuttcap%
\pgfsetroundjoin%
\definecolor{currentfill}{rgb}{1.000000,0.894118,0.788235}%
\pgfsetfillcolor{currentfill}%
\pgfsetlinewidth{0.000000pt}%
\definecolor{currentstroke}{rgb}{1.000000,0.894118,0.788235}%
\pgfsetstrokecolor{currentstroke}%
\pgfsetdash{}{0pt}%
\pgfpathmoveto{\pgfqpoint{2.657000in}{2.893805in}}%
\pgfpathlineto{\pgfqpoint{3.076409in}{2.651660in}}%
\pgfpathlineto{\pgfqpoint{3.076409in}{3.135951in}}%
\pgfpathlineto{\pgfqpoint{2.657000in}{3.378097in}}%
\pgfpathlineto{\pgfqpoint{2.657000in}{2.893805in}}%
\pgfpathclose%
\pgfusepath{fill}%
\end{pgfscope}%
\begin{pgfscope}%
\pgfpathrectangle{\pgfqpoint{0.765000in}{0.660000in}}{\pgfqpoint{4.620000in}{4.620000in}}%
\pgfusepath{clip}%
\pgfsetbuttcap%
\pgfsetroundjoin%
\definecolor{currentfill}{rgb}{1.000000,0.894118,0.788235}%
\pgfsetfillcolor{currentfill}%
\pgfsetlinewidth{0.000000pt}%
\definecolor{currentstroke}{rgb}{1.000000,0.894118,0.788235}%
\pgfsetstrokecolor{currentstroke}%
\pgfsetdash{}{0pt}%
\pgfpathmoveto{\pgfqpoint{-3.358429in}{7.335398in}}%
\pgfpathlineto{\pgfqpoint{-3.777838in}{7.093252in}}%
\pgfpathlineto{\pgfqpoint{-3.358429in}{6.851106in}}%
\pgfpathlineto{\pgfqpoint{-2.939020in}{7.093252in}}%
\pgfpathlineto{\pgfqpoint{-3.358429in}{7.335398in}}%
\pgfpathclose%
\pgfusepath{fill}%
\end{pgfscope}%
\begin{pgfscope}%
\pgfpathrectangle{\pgfqpoint{0.765000in}{0.660000in}}{\pgfqpoint{4.620000in}{4.620000in}}%
\pgfusepath{clip}%
\pgfsetbuttcap%
\pgfsetroundjoin%
\definecolor{currentfill}{rgb}{1.000000,0.894118,0.788235}%
\pgfsetfillcolor{currentfill}%
\pgfsetlinewidth{0.000000pt}%
\definecolor{currentstroke}{rgb}{1.000000,0.894118,0.788235}%
\pgfsetstrokecolor{currentstroke}%
\pgfsetdash{}{0pt}%
\pgfpathmoveto{\pgfqpoint{-3.358429in}{6.366815in}}%
\pgfpathlineto{\pgfqpoint{-2.939020in}{6.608961in}}%
\pgfpathlineto{\pgfqpoint{-2.939020in}{7.093252in}}%
\pgfpathlineto{\pgfqpoint{-3.358429in}{6.851106in}}%
\pgfpathlineto{\pgfqpoint{-3.358429in}{6.366815in}}%
\pgfpathclose%
\pgfusepath{fill}%
\end{pgfscope}%
\begin{pgfscope}%
\pgfpathrectangle{\pgfqpoint{0.765000in}{0.660000in}}{\pgfqpoint{4.620000in}{4.620000in}}%
\pgfusepath{clip}%
\pgfsetbuttcap%
\pgfsetroundjoin%
\definecolor{currentfill}{rgb}{1.000000,0.894118,0.788235}%
\pgfsetfillcolor{currentfill}%
\pgfsetlinewidth{0.000000pt}%
\definecolor{currentstroke}{rgb}{1.000000,0.894118,0.788235}%
\pgfsetstrokecolor{currentstroke}%
\pgfsetdash{}{0pt}%
\pgfpathmoveto{\pgfqpoint{-3.777838in}{6.608961in}}%
\pgfpathlineto{\pgfqpoint{-3.358429in}{6.366815in}}%
\pgfpathlineto{\pgfqpoint{-3.358429in}{6.851106in}}%
\pgfpathlineto{\pgfqpoint{-3.777838in}{7.093252in}}%
\pgfpathlineto{\pgfqpoint{-3.777838in}{6.608961in}}%
\pgfpathclose%
\pgfusepath{fill}%
\end{pgfscope}%
\begin{pgfscope}%
\pgfpathrectangle{\pgfqpoint{0.765000in}{0.660000in}}{\pgfqpoint{4.620000in}{4.620000in}}%
\pgfusepath{clip}%
\pgfsetbuttcap%
\pgfsetroundjoin%
\definecolor{currentfill}{rgb}{1.000000,0.894118,0.788235}%
\pgfsetfillcolor{currentfill}%
\pgfsetlinewidth{0.000000pt}%
\definecolor{currentstroke}{rgb}{1.000000,0.894118,0.788235}%
\pgfsetstrokecolor{currentstroke}%
\pgfsetdash{}{0pt}%
\pgfpathmoveto{\pgfqpoint{3.076409in}{3.135951in}}%
\pgfpathlineto{\pgfqpoint{2.657000in}{2.893805in}}%
\pgfpathlineto{\pgfqpoint{-3.777838in}{6.608961in}}%
\pgfpathlineto{\pgfqpoint{-3.358429in}{6.851106in}}%
\pgfpathlineto{\pgfqpoint{3.076409in}{3.135951in}}%
\pgfpathclose%
\pgfusepath{fill}%
\end{pgfscope}%
\begin{pgfscope}%
\pgfpathrectangle{\pgfqpoint{0.765000in}{0.660000in}}{\pgfqpoint{4.620000in}{4.620000in}}%
\pgfusepath{clip}%
\pgfsetbuttcap%
\pgfsetroundjoin%
\definecolor{currentfill}{rgb}{1.000000,0.894118,0.788235}%
\pgfsetfillcolor{currentfill}%
\pgfsetlinewidth{0.000000pt}%
\definecolor{currentstroke}{rgb}{1.000000,0.894118,0.788235}%
\pgfsetstrokecolor{currentstroke}%
\pgfsetdash{}{0pt}%
\pgfpathmoveto{\pgfqpoint{3.495818in}{2.893805in}}%
\pgfpathlineto{\pgfqpoint{3.076409in}{3.135951in}}%
\pgfpathlineto{\pgfqpoint{-3.358429in}{6.851106in}}%
\pgfpathlineto{\pgfqpoint{-2.939020in}{6.608961in}}%
\pgfpathlineto{\pgfqpoint{3.495818in}{2.893805in}}%
\pgfpathclose%
\pgfusepath{fill}%
\end{pgfscope}%
\begin{pgfscope}%
\pgfpathrectangle{\pgfqpoint{0.765000in}{0.660000in}}{\pgfqpoint{4.620000in}{4.620000in}}%
\pgfusepath{clip}%
\pgfsetbuttcap%
\pgfsetroundjoin%
\definecolor{currentfill}{rgb}{1.000000,0.894118,0.788235}%
\pgfsetfillcolor{currentfill}%
\pgfsetlinewidth{0.000000pt}%
\definecolor{currentstroke}{rgb}{1.000000,0.894118,0.788235}%
\pgfsetstrokecolor{currentstroke}%
\pgfsetdash{}{0pt}%
\pgfpathmoveto{\pgfqpoint{3.076409in}{3.135951in}}%
\pgfpathlineto{\pgfqpoint{3.076409in}{3.620242in}}%
\pgfpathlineto{\pgfqpoint{-3.358429in}{7.335398in}}%
\pgfpathlineto{\pgfqpoint{-2.939020in}{6.608961in}}%
\pgfpathlineto{\pgfqpoint{3.076409in}{3.135951in}}%
\pgfpathclose%
\pgfusepath{fill}%
\end{pgfscope}%
\begin{pgfscope}%
\pgfpathrectangle{\pgfqpoint{0.765000in}{0.660000in}}{\pgfqpoint{4.620000in}{4.620000in}}%
\pgfusepath{clip}%
\pgfsetbuttcap%
\pgfsetroundjoin%
\definecolor{currentfill}{rgb}{1.000000,0.894118,0.788235}%
\pgfsetfillcolor{currentfill}%
\pgfsetlinewidth{0.000000pt}%
\definecolor{currentstroke}{rgb}{1.000000,0.894118,0.788235}%
\pgfsetstrokecolor{currentstroke}%
\pgfsetdash{}{0pt}%
\pgfpathmoveto{\pgfqpoint{3.495818in}{2.893805in}}%
\pgfpathlineto{\pgfqpoint{3.495818in}{3.378097in}}%
\pgfpathlineto{\pgfqpoint{-2.939020in}{7.093252in}}%
\pgfpathlineto{\pgfqpoint{-3.358429in}{6.851106in}}%
\pgfpathlineto{\pgfqpoint{3.495818in}{2.893805in}}%
\pgfpathclose%
\pgfusepath{fill}%
\end{pgfscope}%
\begin{pgfscope}%
\pgfpathrectangle{\pgfqpoint{0.765000in}{0.660000in}}{\pgfqpoint{4.620000in}{4.620000in}}%
\pgfusepath{clip}%
\pgfsetbuttcap%
\pgfsetroundjoin%
\definecolor{currentfill}{rgb}{1.000000,0.894118,0.788235}%
\pgfsetfillcolor{currentfill}%
\pgfsetlinewidth{0.000000pt}%
\definecolor{currentstroke}{rgb}{1.000000,0.894118,0.788235}%
\pgfsetstrokecolor{currentstroke}%
\pgfsetdash{}{0pt}%
\pgfpathmoveto{\pgfqpoint{2.657000in}{2.893805in}}%
\pgfpathlineto{\pgfqpoint{3.076409in}{2.651660in}}%
\pgfpathlineto{\pgfqpoint{-3.358429in}{6.366815in}}%
\pgfpathlineto{\pgfqpoint{-3.777838in}{6.608961in}}%
\pgfpathlineto{\pgfqpoint{2.657000in}{2.893805in}}%
\pgfpathclose%
\pgfusepath{fill}%
\end{pgfscope}%
\begin{pgfscope}%
\pgfpathrectangle{\pgfqpoint{0.765000in}{0.660000in}}{\pgfqpoint{4.620000in}{4.620000in}}%
\pgfusepath{clip}%
\pgfsetbuttcap%
\pgfsetroundjoin%
\definecolor{currentfill}{rgb}{1.000000,0.894118,0.788235}%
\pgfsetfillcolor{currentfill}%
\pgfsetlinewidth{0.000000pt}%
\definecolor{currentstroke}{rgb}{1.000000,0.894118,0.788235}%
\pgfsetstrokecolor{currentstroke}%
\pgfsetdash{}{0pt}%
\pgfpathmoveto{\pgfqpoint{3.076409in}{3.620242in}}%
\pgfpathlineto{\pgfqpoint{2.657000in}{3.378097in}}%
\pgfpathlineto{\pgfqpoint{-3.777838in}{7.093252in}}%
\pgfpathlineto{\pgfqpoint{-3.358429in}{7.335398in}}%
\pgfpathlineto{\pgfqpoint{3.076409in}{3.620242in}}%
\pgfpathclose%
\pgfusepath{fill}%
\end{pgfscope}%
\begin{pgfscope}%
\pgfpathrectangle{\pgfqpoint{0.765000in}{0.660000in}}{\pgfqpoint{4.620000in}{4.620000in}}%
\pgfusepath{clip}%
\pgfsetbuttcap%
\pgfsetroundjoin%
\definecolor{currentfill}{rgb}{1.000000,0.894118,0.788235}%
\pgfsetfillcolor{currentfill}%
\pgfsetlinewidth{0.000000pt}%
\definecolor{currentstroke}{rgb}{1.000000,0.894118,0.788235}%
\pgfsetstrokecolor{currentstroke}%
\pgfsetdash{}{0pt}%
\pgfpathmoveto{\pgfqpoint{3.076409in}{2.651660in}}%
\pgfpathlineto{\pgfqpoint{3.495818in}{2.893805in}}%
\pgfpathlineto{\pgfqpoint{-2.939020in}{6.608961in}}%
\pgfpathlineto{\pgfqpoint{-3.358429in}{6.366815in}}%
\pgfpathlineto{\pgfqpoint{3.076409in}{2.651660in}}%
\pgfpathclose%
\pgfusepath{fill}%
\end{pgfscope}%
\begin{pgfscope}%
\pgfpathrectangle{\pgfqpoint{0.765000in}{0.660000in}}{\pgfqpoint{4.620000in}{4.620000in}}%
\pgfusepath{clip}%
\pgfsetbuttcap%
\pgfsetroundjoin%
\definecolor{currentfill}{rgb}{1.000000,0.894118,0.788235}%
\pgfsetfillcolor{currentfill}%
\pgfsetlinewidth{0.000000pt}%
\definecolor{currentstroke}{rgb}{1.000000,0.894118,0.788235}%
\pgfsetstrokecolor{currentstroke}%
\pgfsetdash{}{0pt}%
\pgfpathmoveto{\pgfqpoint{3.495818in}{3.378097in}}%
\pgfpathlineto{\pgfqpoint{3.076409in}{3.620242in}}%
\pgfpathlineto{\pgfqpoint{-3.358429in}{7.335398in}}%
\pgfpathlineto{\pgfqpoint{-2.939020in}{7.093252in}}%
\pgfpathlineto{\pgfqpoint{3.495818in}{3.378097in}}%
\pgfpathclose%
\pgfusepath{fill}%
\end{pgfscope}%
\begin{pgfscope}%
\pgfpathrectangle{\pgfqpoint{0.765000in}{0.660000in}}{\pgfqpoint{4.620000in}{4.620000in}}%
\pgfusepath{clip}%
\pgfsetbuttcap%
\pgfsetroundjoin%
\definecolor{currentfill}{rgb}{1.000000,0.894118,0.788235}%
\pgfsetfillcolor{currentfill}%
\pgfsetlinewidth{0.000000pt}%
\definecolor{currentstroke}{rgb}{1.000000,0.894118,0.788235}%
\pgfsetstrokecolor{currentstroke}%
\pgfsetdash{}{0pt}%
\pgfpathmoveto{\pgfqpoint{2.657000in}{2.893805in}}%
\pgfpathlineto{\pgfqpoint{2.657000in}{3.378097in}}%
\pgfpathlineto{\pgfqpoint{-3.777838in}{7.093252in}}%
\pgfpathlineto{\pgfqpoint{-3.358429in}{6.366815in}}%
\pgfpathlineto{\pgfqpoint{2.657000in}{2.893805in}}%
\pgfpathclose%
\pgfusepath{fill}%
\end{pgfscope}%
\begin{pgfscope}%
\pgfpathrectangle{\pgfqpoint{0.765000in}{0.660000in}}{\pgfqpoint{4.620000in}{4.620000in}}%
\pgfusepath{clip}%
\pgfsetbuttcap%
\pgfsetroundjoin%
\definecolor{currentfill}{rgb}{1.000000,0.894118,0.788235}%
\pgfsetfillcolor{currentfill}%
\pgfsetlinewidth{0.000000pt}%
\definecolor{currentstroke}{rgb}{1.000000,0.894118,0.788235}%
\pgfsetstrokecolor{currentstroke}%
\pgfsetdash{}{0pt}%
\pgfpathmoveto{\pgfqpoint{3.076409in}{2.651660in}}%
\pgfpathlineto{\pgfqpoint{3.076409in}{3.135951in}}%
\pgfpathlineto{\pgfqpoint{-3.358429in}{6.851106in}}%
\pgfpathlineto{\pgfqpoint{-3.777838in}{6.608961in}}%
\pgfpathlineto{\pgfqpoint{3.076409in}{2.651660in}}%
\pgfpathclose%
\pgfusepath{fill}%
\end{pgfscope}%
\begin{pgfscope}%
\pgfpathrectangle{\pgfqpoint{0.765000in}{0.660000in}}{\pgfqpoint{4.620000in}{4.620000in}}%
\pgfusepath{clip}%
\pgfsetbuttcap%
\pgfsetroundjoin%
\definecolor{currentfill}{rgb}{1.000000,0.894118,0.788235}%
\pgfsetfillcolor{currentfill}%
\pgfsetlinewidth{0.000000pt}%
\definecolor{currentstroke}{rgb}{1.000000,0.894118,0.788235}%
\pgfsetstrokecolor{currentstroke}%
\pgfsetdash{}{0pt}%
\pgfpathmoveto{\pgfqpoint{2.657000in}{3.378097in}}%
\pgfpathlineto{\pgfqpoint{3.076409in}{3.135951in}}%
\pgfpathlineto{\pgfqpoint{-3.358429in}{6.851106in}}%
\pgfpathlineto{\pgfqpoint{-3.777838in}{7.093252in}}%
\pgfpathlineto{\pgfqpoint{2.657000in}{3.378097in}}%
\pgfpathclose%
\pgfusepath{fill}%
\end{pgfscope}%
\begin{pgfscope}%
\pgfpathrectangle{\pgfqpoint{0.765000in}{0.660000in}}{\pgfqpoint{4.620000in}{4.620000in}}%
\pgfusepath{clip}%
\pgfsetbuttcap%
\pgfsetroundjoin%
\definecolor{currentfill}{rgb}{1.000000,0.894118,0.788235}%
\pgfsetfillcolor{currentfill}%
\pgfsetlinewidth{0.000000pt}%
\definecolor{currentstroke}{rgb}{1.000000,0.894118,0.788235}%
\pgfsetstrokecolor{currentstroke}%
\pgfsetdash{}{0pt}%
\pgfpathmoveto{\pgfqpoint{3.076409in}{3.135951in}}%
\pgfpathlineto{\pgfqpoint{3.495818in}{3.378097in}}%
\pgfpathlineto{\pgfqpoint{-2.939020in}{7.093252in}}%
\pgfpathlineto{\pgfqpoint{-3.358429in}{6.851106in}}%
\pgfpathlineto{\pgfqpoint{3.076409in}{3.135951in}}%
\pgfpathclose%
\pgfusepath{fill}%
\end{pgfscope}%
\begin{pgfscope}%
\pgfpathrectangle{\pgfqpoint{0.765000in}{0.660000in}}{\pgfqpoint{4.620000in}{4.620000in}}%
\pgfusepath{clip}%
\pgfsetbuttcap%
\pgfsetroundjoin%
\definecolor{currentfill}{rgb}{1.000000,0.894118,0.788235}%
\pgfsetfillcolor{currentfill}%
\pgfsetlinewidth{0.000000pt}%
\definecolor{currentstroke}{rgb}{1.000000,0.894118,0.788235}%
\pgfsetstrokecolor{currentstroke}%
\pgfsetdash{}{0pt}%
\pgfpathmoveto{\pgfqpoint{3.076409in}{3.135951in}}%
\pgfpathlineto{\pgfqpoint{2.657000in}{2.893805in}}%
\pgfpathlineto{\pgfqpoint{3.076409in}{2.651660in}}%
\pgfpathlineto{\pgfqpoint{3.495818in}{2.893805in}}%
\pgfpathlineto{\pgfqpoint{3.076409in}{3.135951in}}%
\pgfpathclose%
\pgfusepath{fill}%
\end{pgfscope}%
\begin{pgfscope}%
\pgfpathrectangle{\pgfqpoint{0.765000in}{0.660000in}}{\pgfqpoint{4.620000in}{4.620000in}}%
\pgfusepath{clip}%
\pgfsetbuttcap%
\pgfsetroundjoin%
\definecolor{currentfill}{rgb}{1.000000,0.894118,0.788235}%
\pgfsetfillcolor{currentfill}%
\pgfsetlinewidth{0.000000pt}%
\definecolor{currentstroke}{rgb}{1.000000,0.894118,0.788235}%
\pgfsetstrokecolor{currentstroke}%
\pgfsetdash{}{0pt}%
\pgfpathmoveto{\pgfqpoint{3.076409in}{3.135951in}}%
\pgfpathlineto{\pgfqpoint{2.657000in}{2.893805in}}%
\pgfpathlineto{\pgfqpoint{2.657000in}{3.378097in}}%
\pgfpathlineto{\pgfqpoint{3.076409in}{3.620242in}}%
\pgfpathlineto{\pgfqpoint{3.076409in}{3.135951in}}%
\pgfpathclose%
\pgfusepath{fill}%
\end{pgfscope}%
\begin{pgfscope}%
\pgfpathrectangle{\pgfqpoint{0.765000in}{0.660000in}}{\pgfqpoint{4.620000in}{4.620000in}}%
\pgfusepath{clip}%
\pgfsetbuttcap%
\pgfsetroundjoin%
\definecolor{currentfill}{rgb}{1.000000,0.894118,0.788235}%
\pgfsetfillcolor{currentfill}%
\pgfsetlinewidth{0.000000pt}%
\definecolor{currentstroke}{rgb}{1.000000,0.894118,0.788235}%
\pgfsetstrokecolor{currentstroke}%
\pgfsetdash{}{0pt}%
\pgfpathmoveto{\pgfqpoint{3.076409in}{3.135951in}}%
\pgfpathlineto{\pgfqpoint{3.495818in}{2.893805in}}%
\pgfpathlineto{\pgfqpoint{3.495818in}{3.378097in}}%
\pgfpathlineto{\pgfqpoint{3.076409in}{3.620242in}}%
\pgfpathlineto{\pgfqpoint{3.076409in}{3.135951in}}%
\pgfpathclose%
\pgfusepath{fill}%
\end{pgfscope}%
\begin{pgfscope}%
\pgfpathrectangle{\pgfqpoint{0.765000in}{0.660000in}}{\pgfqpoint{4.620000in}{4.620000in}}%
\pgfusepath{clip}%
\pgfsetbuttcap%
\pgfsetroundjoin%
\definecolor{currentfill}{rgb}{1.000000,0.894118,0.788235}%
\pgfsetfillcolor{currentfill}%
\pgfsetlinewidth{0.000000pt}%
\definecolor{currentstroke}{rgb}{1.000000,0.894118,0.788235}%
\pgfsetstrokecolor{currentstroke}%
\pgfsetdash{}{0pt}%
\pgfpathmoveto{\pgfqpoint{9.511247in}{6.851106in}}%
\pgfpathlineto{\pgfqpoint{9.091838in}{6.608961in}}%
\pgfpathlineto{\pgfqpoint{9.511247in}{6.366815in}}%
\pgfpathlineto{\pgfqpoint{9.930655in}{6.608961in}}%
\pgfpathlineto{\pgfqpoint{9.511247in}{6.851106in}}%
\pgfpathclose%
\pgfusepath{fill}%
\end{pgfscope}%
\begin{pgfscope}%
\pgfpathrectangle{\pgfqpoint{0.765000in}{0.660000in}}{\pgfqpoint{4.620000in}{4.620000in}}%
\pgfusepath{clip}%
\pgfsetbuttcap%
\pgfsetroundjoin%
\definecolor{currentfill}{rgb}{1.000000,0.894118,0.788235}%
\pgfsetfillcolor{currentfill}%
\pgfsetlinewidth{0.000000pt}%
\definecolor{currentstroke}{rgb}{1.000000,0.894118,0.788235}%
\pgfsetstrokecolor{currentstroke}%
\pgfsetdash{}{0pt}%
\pgfpathmoveto{\pgfqpoint{9.511247in}{6.851106in}}%
\pgfpathlineto{\pgfqpoint{9.091838in}{6.608961in}}%
\pgfpathlineto{\pgfqpoint{9.091838in}{7.093252in}}%
\pgfpathlineto{\pgfqpoint{9.511247in}{7.335398in}}%
\pgfpathlineto{\pgfqpoint{9.511247in}{6.851106in}}%
\pgfpathclose%
\pgfusepath{fill}%
\end{pgfscope}%
\begin{pgfscope}%
\pgfpathrectangle{\pgfqpoint{0.765000in}{0.660000in}}{\pgfqpoint{4.620000in}{4.620000in}}%
\pgfusepath{clip}%
\pgfsetbuttcap%
\pgfsetroundjoin%
\definecolor{currentfill}{rgb}{1.000000,0.894118,0.788235}%
\pgfsetfillcolor{currentfill}%
\pgfsetlinewidth{0.000000pt}%
\definecolor{currentstroke}{rgb}{1.000000,0.894118,0.788235}%
\pgfsetstrokecolor{currentstroke}%
\pgfsetdash{}{0pt}%
\pgfpathmoveto{\pgfqpoint{9.511247in}{6.851106in}}%
\pgfpathlineto{\pgfqpoint{9.930655in}{6.608961in}}%
\pgfpathlineto{\pgfqpoint{9.930655in}{7.093252in}}%
\pgfpathlineto{\pgfqpoint{9.511247in}{7.335398in}}%
\pgfpathlineto{\pgfqpoint{9.511247in}{6.851106in}}%
\pgfpathclose%
\pgfusepath{fill}%
\end{pgfscope}%
\begin{pgfscope}%
\pgfpathrectangle{\pgfqpoint{0.765000in}{0.660000in}}{\pgfqpoint{4.620000in}{4.620000in}}%
\pgfusepath{clip}%
\pgfsetbuttcap%
\pgfsetroundjoin%
\definecolor{currentfill}{rgb}{1.000000,0.894118,0.788235}%
\pgfsetfillcolor{currentfill}%
\pgfsetlinewidth{0.000000pt}%
\definecolor{currentstroke}{rgb}{1.000000,0.894118,0.788235}%
\pgfsetstrokecolor{currentstroke}%
\pgfsetdash{}{0pt}%
\pgfpathmoveto{\pgfqpoint{3.076409in}{3.620242in}}%
\pgfpathlineto{\pgfqpoint{2.657000in}{3.378097in}}%
\pgfpathlineto{\pgfqpoint{3.076409in}{3.135951in}}%
\pgfpathlineto{\pgfqpoint{3.495818in}{3.378097in}}%
\pgfpathlineto{\pgfqpoint{3.076409in}{3.620242in}}%
\pgfpathclose%
\pgfusepath{fill}%
\end{pgfscope}%
\begin{pgfscope}%
\pgfpathrectangle{\pgfqpoint{0.765000in}{0.660000in}}{\pgfqpoint{4.620000in}{4.620000in}}%
\pgfusepath{clip}%
\pgfsetbuttcap%
\pgfsetroundjoin%
\definecolor{currentfill}{rgb}{1.000000,0.894118,0.788235}%
\pgfsetfillcolor{currentfill}%
\pgfsetlinewidth{0.000000pt}%
\definecolor{currentstroke}{rgb}{1.000000,0.894118,0.788235}%
\pgfsetstrokecolor{currentstroke}%
\pgfsetdash{}{0pt}%
\pgfpathmoveto{\pgfqpoint{3.076409in}{2.651660in}}%
\pgfpathlineto{\pgfqpoint{3.495818in}{2.893805in}}%
\pgfpathlineto{\pgfqpoint{3.495818in}{3.378097in}}%
\pgfpathlineto{\pgfqpoint{3.076409in}{3.135951in}}%
\pgfpathlineto{\pgfqpoint{3.076409in}{2.651660in}}%
\pgfpathclose%
\pgfusepath{fill}%
\end{pgfscope}%
\begin{pgfscope}%
\pgfpathrectangle{\pgfqpoint{0.765000in}{0.660000in}}{\pgfqpoint{4.620000in}{4.620000in}}%
\pgfusepath{clip}%
\pgfsetbuttcap%
\pgfsetroundjoin%
\definecolor{currentfill}{rgb}{1.000000,0.894118,0.788235}%
\pgfsetfillcolor{currentfill}%
\pgfsetlinewidth{0.000000pt}%
\definecolor{currentstroke}{rgb}{1.000000,0.894118,0.788235}%
\pgfsetstrokecolor{currentstroke}%
\pgfsetdash{}{0pt}%
\pgfpathmoveto{\pgfqpoint{2.657000in}{2.893805in}}%
\pgfpathlineto{\pgfqpoint{3.076409in}{2.651660in}}%
\pgfpathlineto{\pgfqpoint{3.076409in}{3.135951in}}%
\pgfpathlineto{\pgfqpoint{2.657000in}{3.378097in}}%
\pgfpathlineto{\pgfqpoint{2.657000in}{2.893805in}}%
\pgfpathclose%
\pgfusepath{fill}%
\end{pgfscope}%
\begin{pgfscope}%
\pgfpathrectangle{\pgfqpoint{0.765000in}{0.660000in}}{\pgfqpoint{4.620000in}{4.620000in}}%
\pgfusepath{clip}%
\pgfsetbuttcap%
\pgfsetroundjoin%
\definecolor{currentfill}{rgb}{1.000000,0.894118,0.788235}%
\pgfsetfillcolor{currentfill}%
\pgfsetlinewidth{0.000000pt}%
\definecolor{currentstroke}{rgb}{1.000000,0.894118,0.788235}%
\pgfsetstrokecolor{currentstroke}%
\pgfsetdash{}{0pt}%
\pgfpathmoveto{\pgfqpoint{9.511247in}{7.335398in}}%
\pgfpathlineto{\pgfqpoint{9.091838in}{7.093252in}}%
\pgfpathlineto{\pgfqpoint{9.511247in}{6.851106in}}%
\pgfpathlineto{\pgfqpoint{9.930655in}{7.093252in}}%
\pgfpathlineto{\pgfqpoint{9.511247in}{7.335398in}}%
\pgfpathclose%
\pgfusepath{fill}%
\end{pgfscope}%
\begin{pgfscope}%
\pgfpathrectangle{\pgfqpoint{0.765000in}{0.660000in}}{\pgfqpoint{4.620000in}{4.620000in}}%
\pgfusepath{clip}%
\pgfsetbuttcap%
\pgfsetroundjoin%
\definecolor{currentfill}{rgb}{1.000000,0.894118,0.788235}%
\pgfsetfillcolor{currentfill}%
\pgfsetlinewidth{0.000000pt}%
\definecolor{currentstroke}{rgb}{1.000000,0.894118,0.788235}%
\pgfsetstrokecolor{currentstroke}%
\pgfsetdash{}{0pt}%
\pgfpathmoveto{\pgfqpoint{9.511247in}{6.366815in}}%
\pgfpathlineto{\pgfqpoint{9.930655in}{6.608961in}}%
\pgfpathlineto{\pgfqpoint{9.930655in}{7.093252in}}%
\pgfpathlineto{\pgfqpoint{9.511247in}{6.851106in}}%
\pgfpathlineto{\pgfqpoint{9.511247in}{6.366815in}}%
\pgfpathclose%
\pgfusepath{fill}%
\end{pgfscope}%
\begin{pgfscope}%
\pgfpathrectangle{\pgfqpoint{0.765000in}{0.660000in}}{\pgfqpoint{4.620000in}{4.620000in}}%
\pgfusepath{clip}%
\pgfsetbuttcap%
\pgfsetroundjoin%
\definecolor{currentfill}{rgb}{1.000000,0.894118,0.788235}%
\pgfsetfillcolor{currentfill}%
\pgfsetlinewidth{0.000000pt}%
\definecolor{currentstroke}{rgb}{1.000000,0.894118,0.788235}%
\pgfsetstrokecolor{currentstroke}%
\pgfsetdash{}{0pt}%
\pgfpathmoveto{\pgfqpoint{9.091838in}{6.608961in}}%
\pgfpathlineto{\pgfqpoint{9.511247in}{6.366815in}}%
\pgfpathlineto{\pgfqpoint{9.511247in}{6.851106in}}%
\pgfpathlineto{\pgfqpoint{9.091838in}{7.093252in}}%
\pgfpathlineto{\pgfqpoint{9.091838in}{6.608961in}}%
\pgfpathclose%
\pgfusepath{fill}%
\end{pgfscope}%
\begin{pgfscope}%
\pgfpathrectangle{\pgfqpoint{0.765000in}{0.660000in}}{\pgfqpoint{4.620000in}{4.620000in}}%
\pgfusepath{clip}%
\pgfsetbuttcap%
\pgfsetroundjoin%
\definecolor{currentfill}{rgb}{1.000000,0.894118,0.788235}%
\pgfsetfillcolor{currentfill}%
\pgfsetlinewidth{0.000000pt}%
\definecolor{currentstroke}{rgb}{1.000000,0.894118,0.788235}%
\pgfsetstrokecolor{currentstroke}%
\pgfsetdash{}{0pt}%
\pgfpathmoveto{\pgfqpoint{3.076409in}{3.135951in}}%
\pgfpathlineto{\pgfqpoint{2.657000in}{2.893805in}}%
\pgfpathlineto{\pgfqpoint{9.091838in}{6.608961in}}%
\pgfpathlineto{\pgfqpoint{9.511247in}{6.851106in}}%
\pgfpathlineto{\pgfqpoint{3.076409in}{3.135951in}}%
\pgfpathclose%
\pgfusepath{fill}%
\end{pgfscope}%
\begin{pgfscope}%
\pgfpathrectangle{\pgfqpoint{0.765000in}{0.660000in}}{\pgfqpoint{4.620000in}{4.620000in}}%
\pgfusepath{clip}%
\pgfsetbuttcap%
\pgfsetroundjoin%
\definecolor{currentfill}{rgb}{1.000000,0.894118,0.788235}%
\pgfsetfillcolor{currentfill}%
\pgfsetlinewidth{0.000000pt}%
\definecolor{currentstroke}{rgb}{1.000000,0.894118,0.788235}%
\pgfsetstrokecolor{currentstroke}%
\pgfsetdash{}{0pt}%
\pgfpathmoveto{\pgfqpoint{3.495818in}{2.893805in}}%
\pgfpathlineto{\pgfqpoint{3.076409in}{3.135951in}}%
\pgfpathlineto{\pgfqpoint{9.511247in}{6.851106in}}%
\pgfpathlineto{\pgfqpoint{9.930655in}{6.608961in}}%
\pgfpathlineto{\pgfqpoint{3.495818in}{2.893805in}}%
\pgfpathclose%
\pgfusepath{fill}%
\end{pgfscope}%
\begin{pgfscope}%
\pgfpathrectangle{\pgfqpoint{0.765000in}{0.660000in}}{\pgfqpoint{4.620000in}{4.620000in}}%
\pgfusepath{clip}%
\pgfsetbuttcap%
\pgfsetroundjoin%
\definecolor{currentfill}{rgb}{1.000000,0.894118,0.788235}%
\pgfsetfillcolor{currentfill}%
\pgfsetlinewidth{0.000000pt}%
\definecolor{currentstroke}{rgb}{1.000000,0.894118,0.788235}%
\pgfsetstrokecolor{currentstroke}%
\pgfsetdash{}{0pt}%
\pgfpathmoveto{\pgfqpoint{3.076409in}{3.135951in}}%
\pgfpathlineto{\pgfqpoint{3.076409in}{3.620242in}}%
\pgfpathlineto{\pgfqpoint{9.511247in}{7.335398in}}%
\pgfpathlineto{\pgfqpoint{9.930655in}{6.608961in}}%
\pgfpathlineto{\pgfqpoint{3.076409in}{3.135951in}}%
\pgfpathclose%
\pgfusepath{fill}%
\end{pgfscope}%
\begin{pgfscope}%
\pgfpathrectangle{\pgfqpoint{0.765000in}{0.660000in}}{\pgfqpoint{4.620000in}{4.620000in}}%
\pgfusepath{clip}%
\pgfsetbuttcap%
\pgfsetroundjoin%
\definecolor{currentfill}{rgb}{1.000000,0.894118,0.788235}%
\pgfsetfillcolor{currentfill}%
\pgfsetlinewidth{0.000000pt}%
\definecolor{currentstroke}{rgb}{1.000000,0.894118,0.788235}%
\pgfsetstrokecolor{currentstroke}%
\pgfsetdash{}{0pt}%
\pgfpathmoveto{\pgfqpoint{3.495818in}{2.893805in}}%
\pgfpathlineto{\pgfqpoint{3.495818in}{3.378097in}}%
\pgfpathlineto{\pgfqpoint{9.930655in}{7.093252in}}%
\pgfpathlineto{\pgfqpoint{9.511247in}{6.851106in}}%
\pgfpathlineto{\pgfqpoint{3.495818in}{2.893805in}}%
\pgfpathclose%
\pgfusepath{fill}%
\end{pgfscope}%
\begin{pgfscope}%
\pgfpathrectangle{\pgfqpoint{0.765000in}{0.660000in}}{\pgfqpoint{4.620000in}{4.620000in}}%
\pgfusepath{clip}%
\pgfsetbuttcap%
\pgfsetroundjoin%
\definecolor{currentfill}{rgb}{1.000000,0.894118,0.788235}%
\pgfsetfillcolor{currentfill}%
\pgfsetlinewidth{0.000000pt}%
\definecolor{currentstroke}{rgb}{1.000000,0.894118,0.788235}%
\pgfsetstrokecolor{currentstroke}%
\pgfsetdash{}{0pt}%
\pgfpathmoveto{\pgfqpoint{2.657000in}{2.893805in}}%
\pgfpathlineto{\pgfqpoint{3.076409in}{2.651660in}}%
\pgfpathlineto{\pgfqpoint{9.511247in}{6.366815in}}%
\pgfpathlineto{\pgfqpoint{9.091838in}{6.608961in}}%
\pgfpathlineto{\pgfqpoint{2.657000in}{2.893805in}}%
\pgfpathclose%
\pgfusepath{fill}%
\end{pgfscope}%
\begin{pgfscope}%
\pgfpathrectangle{\pgfqpoint{0.765000in}{0.660000in}}{\pgfqpoint{4.620000in}{4.620000in}}%
\pgfusepath{clip}%
\pgfsetbuttcap%
\pgfsetroundjoin%
\definecolor{currentfill}{rgb}{1.000000,0.894118,0.788235}%
\pgfsetfillcolor{currentfill}%
\pgfsetlinewidth{0.000000pt}%
\definecolor{currentstroke}{rgb}{1.000000,0.894118,0.788235}%
\pgfsetstrokecolor{currentstroke}%
\pgfsetdash{}{0pt}%
\pgfpathmoveto{\pgfqpoint{3.076409in}{3.620242in}}%
\pgfpathlineto{\pgfqpoint{2.657000in}{3.378097in}}%
\pgfpathlineto{\pgfqpoint{9.091838in}{7.093252in}}%
\pgfpathlineto{\pgfqpoint{9.511247in}{7.335398in}}%
\pgfpathlineto{\pgfqpoint{3.076409in}{3.620242in}}%
\pgfpathclose%
\pgfusepath{fill}%
\end{pgfscope}%
\begin{pgfscope}%
\pgfpathrectangle{\pgfqpoint{0.765000in}{0.660000in}}{\pgfqpoint{4.620000in}{4.620000in}}%
\pgfusepath{clip}%
\pgfsetbuttcap%
\pgfsetroundjoin%
\definecolor{currentfill}{rgb}{1.000000,0.894118,0.788235}%
\pgfsetfillcolor{currentfill}%
\pgfsetlinewidth{0.000000pt}%
\definecolor{currentstroke}{rgb}{1.000000,0.894118,0.788235}%
\pgfsetstrokecolor{currentstroke}%
\pgfsetdash{}{0pt}%
\pgfpathmoveto{\pgfqpoint{3.076409in}{2.651660in}}%
\pgfpathlineto{\pgfqpoint{3.495818in}{2.893805in}}%
\pgfpathlineto{\pgfqpoint{9.930655in}{6.608961in}}%
\pgfpathlineto{\pgfqpoint{9.511247in}{6.366815in}}%
\pgfpathlineto{\pgfqpoint{3.076409in}{2.651660in}}%
\pgfpathclose%
\pgfusepath{fill}%
\end{pgfscope}%
\begin{pgfscope}%
\pgfpathrectangle{\pgfqpoint{0.765000in}{0.660000in}}{\pgfqpoint{4.620000in}{4.620000in}}%
\pgfusepath{clip}%
\pgfsetbuttcap%
\pgfsetroundjoin%
\definecolor{currentfill}{rgb}{1.000000,0.894118,0.788235}%
\pgfsetfillcolor{currentfill}%
\pgfsetlinewidth{0.000000pt}%
\definecolor{currentstroke}{rgb}{1.000000,0.894118,0.788235}%
\pgfsetstrokecolor{currentstroke}%
\pgfsetdash{}{0pt}%
\pgfpathmoveto{\pgfqpoint{3.495818in}{3.378097in}}%
\pgfpathlineto{\pgfqpoint{3.076409in}{3.620242in}}%
\pgfpathlineto{\pgfqpoint{9.511247in}{7.335398in}}%
\pgfpathlineto{\pgfqpoint{9.930655in}{7.093252in}}%
\pgfpathlineto{\pgfqpoint{3.495818in}{3.378097in}}%
\pgfpathclose%
\pgfusepath{fill}%
\end{pgfscope}%
\begin{pgfscope}%
\pgfpathrectangle{\pgfqpoint{0.765000in}{0.660000in}}{\pgfqpoint{4.620000in}{4.620000in}}%
\pgfusepath{clip}%
\pgfsetbuttcap%
\pgfsetroundjoin%
\definecolor{currentfill}{rgb}{1.000000,0.894118,0.788235}%
\pgfsetfillcolor{currentfill}%
\pgfsetlinewidth{0.000000pt}%
\definecolor{currentstroke}{rgb}{1.000000,0.894118,0.788235}%
\pgfsetstrokecolor{currentstroke}%
\pgfsetdash{}{0pt}%
\pgfpathmoveto{\pgfqpoint{2.657000in}{2.893805in}}%
\pgfpathlineto{\pgfqpoint{2.657000in}{3.378097in}}%
\pgfpathlineto{\pgfqpoint{9.091838in}{7.093252in}}%
\pgfpathlineto{\pgfqpoint{9.511247in}{6.366815in}}%
\pgfpathlineto{\pgfqpoint{2.657000in}{2.893805in}}%
\pgfpathclose%
\pgfusepath{fill}%
\end{pgfscope}%
\begin{pgfscope}%
\pgfpathrectangle{\pgfqpoint{0.765000in}{0.660000in}}{\pgfqpoint{4.620000in}{4.620000in}}%
\pgfusepath{clip}%
\pgfsetbuttcap%
\pgfsetroundjoin%
\definecolor{currentfill}{rgb}{1.000000,0.894118,0.788235}%
\pgfsetfillcolor{currentfill}%
\pgfsetlinewidth{0.000000pt}%
\definecolor{currentstroke}{rgb}{1.000000,0.894118,0.788235}%
\pgfsetstrokecolor{currentstroke}%
\pgfsetdash{}{0pt}%
\pgfpathmoveto{\pgfqpoint{3.076409in}{2.651660in}}%
\pgfpathlineto{\pgfqpoint{3.076409in}{3.135951in}}%
\pgfpathlineto{\pgfqpoint{9.511247in}{6.851106in}}%
\pgfpathlineto{\pgfqpoint{9.091838in}{6.608961in}}%
\pgfpathlineto{\pgfqpoint{3.076409in}{2.651660in}}%
\pgfpathclose%
\pgfusepath{fill}%
\end{pgfscope}%
\begin{pgfscope}%
\pgfpathrectangle{\pgfqpoint{0.765000in}{0.660000in}}{\pgfqpoint{4.620000in}{4.620000in}}%
\pgfusepath{clip}%
\pgfsetbuttcap%
\pgfsetroundjoin%
\definecolor{currentfill}{rgb}{1.000000,0.894118,0.788235}%
\pgfsetfillcolor{currentfill}%
\pgfsetlinewidth{0.000000pt}%
\definecolor{currentstroke}{rgb}{1.000000,0.894118,0.788235}%
\pgfsetstrokecolor{currentstroke}%
\pgfsetdash{}{0pt}%
\pgfpathmoveto{\pgfqpoint{2.657000in}{3.378097in}}%
\pgfpathlineto{\pgfqpoint{3.076409in}{3.135951in}}%
\pgfpathlineto{\pgfqpoint{9.511247in}{6.851106in}}%
\pgfpathlineto{\pgfqpoint{9.091838in}{7.093252in}}%
\pgfpathlineto{\pgfqpoint{2.657000in}{3.378097in}}%
\pgfpathclose%
\pgfusepath{fill}%
\end{pgfscope}%
\begin{pgfscope}%
\pgfpathrectangle{\pgfqpoint{0.765000in}{0.660000in}}{\pgfqpoint{4.620000in}{4.620000in}}%
\pgfusepath{clip}%
\pgfsetbuttcap%
\pgfsetroundjoin%
\definecolor{currentfill}{rgb}{1.000000,0.894118,0.788235}%
\pgfsetfillcolor{currentfill}%
\pgfsetlinewidth{0.000000pt}%
\definecolor{currentstroke}{rgb}{1.000000,0.894118,0.788235}%
\pgfsetstrokecolor{currentstroke}%
\pgfsetdash{}{0pt}%
\pgfpathmoveto{\pgfqpoint{3.076409in}{3.135951in}}%
\pgfpathlineto{\pgfqpoint{3.495818in}{3.378097in}}%
\pgfpathlineto{\pgfqpoint{9.930655in}{7.093252in}}%
\pgfpathlineto{\pgfqpoint{9.511247in}{6.851106in}}%
\pgfpathlineto{\pgfqpoint{3.076409in}{3.135951in}}%
\pgfpathclose%
\pgfusepath{fill}%
\end{pgfscope}%
\begin{pgfscope}%
\pgfpathrectangle{\pgfqpoint{0.765000in}{0.660000in}}{\pgfqpoint{4.620000in}{4.620000in}}%
\pgfusepath{clip}%
\pgfsetrectcap%
\pgfsetroundjoin%
\pgfsetlinewidth{1.204500pt}%
\definecolor{currentstroke}{rgb}{1.000000,0.576471,0.309804}%
\pgfsetstrokecolor{currentstroke}%
\pgfsetdash{}{0pt}%
\pgfpathmoveto{\pgfqpoint{2.848196in}{2.392621in}}%
\pgfusepath{stroke}%
\end{pgfscope}%
\begin{pgfscope}%
\pgfpathrectangle{\pgfqpoint{0.765000in}{0.660000in}}{\pgfqpoint{4.620000in}{4.620000in}}%
\pgfusepath{clip}%
\pgfsetbuttcap%
\pgfsetroundjoin%
\definecolor{currentfill}{rgb}{1.000000,0.576471,0.309804}%
\pgfsetfillcolor{currentfill}%
\pgfsetlinewidth{1.003750pt}%
\definecolor{currentstroke}{rgb}{1.000000,0.576471,0.309804}%
\pgfsetstrokecolor{currentstroke}%
\pgfsetdash{}{0pt}%
\pgfsys@defobject{currentmarker}{\pgfqpoint{-0.033333in}{-0.033333in}}{\pgfqpoint{0.033333in}{0.033333in}}{%
\pgfpathmoveto{\pgfqpoint{0.000000in}{-0.033333in}}%
\pgfpathcurveto{\pgfqpoint{0.008840in}{-0.033333in}}{\pgfqpoint{0.017319in}{-0.029821in}}{\pgfqpoint{0.023570in}{-0.023570in}}%
\pgfpathcurveto{\pgfqpoint{0.029821in}{-0.017319in}}{\pgfqpoint{0.033333in}{-0.008840in}}{\pgfqpoint{0.033333in}{0.000000in}}%
\pgfpathcurveto{\pgfqpoint{0.033333in}{0.008840in}}{\pgfqpoint{0.029821in}{0.017319in}}{\pgfqpoint{0.023570in}{0.023570in}}%
\pgfpathcurveto{\pgfqpoint{0.017319in}{0.029821in}}{\pgfqpoint{0.008840in}{0.033333in}}{\pgfqpoint{0.000000in}{0.033333in}}%
\pgfpathcurveto{\pgfqpoint{-0.008840in}{0.033333in}}{\pgfqpoint{-0.017319in}{0.029821in}}{\pgfqpoint{-0.023570in}{0.023570in}}%
\pgfpathcurveto{\pgfqpoint{-0.029821in}{0.017319in}}{\pgfqpoint{-0.033333in}{0.008840in}}{\pgfqpoint{-0.033333in}{0.000000in}}%
\pgfpathcurveto{\pgfqpoint{-0.033333in}{-0.008840in}}{\pgfqpoint{-0.029821in}{-0.017319in}}{\pgfqpoint{-0.023570in}{-0.023570in}}%
\pgfpathcurveto{\pgfqpoint{-0.017319in}{-0.029821in}}{\pgfqpoint{-0.008840in}{-0.033333in}}{\pgfqpoint{0.000000in}{-0.033333in}}%
\pgfpathlineto{\pgfqpoint{0.000000in}{-0.033333in}}%
\pgfpathclose%
\pgfusepath{stroke,fill}%
}%
\begin{pgfscope}%
\pgfsys@transformshift{2.848196in}{2.392621in}%
\pgfsys@useobject{currentmarker}{}%
\end{pgfscope}%
\end{pgfscope}%
\begin{pgfscope}%
\pgfpathrectangle{\pgfqpoint{0.765000in}{0.660000in}}{\pgfqpoint{4.620000in}{4.620000in}}%
\pgfusepath{clip}%
\pgfsetrectcap%
\pgfsetroundjoin%
\pgfsetlinewidth{1.204500pt}%
\definecolor{currentstroke}{rgb}{1.000000,0.576471,0.309804}%
\pgfsetstrokecolor{currentstroke}%
\pgfsetdash{}{0pt}%
\pgfpathmoveto{\pgfqpoint{3.963086in}{1.777525in}}%
\pgfusepath{stroke}%
\end{pgfscope}%
\begin{pgfscope}%
\pgfpathrectangle{\pgfqpoint{0.765000in}{0.660000in}}{\pgfqpoint{4.620000in}{4.620000in}}%
\pgfusepath{clip}%
\pgfsetbuttcap%
\pgfsetroundjoin%
\definecolor{currentfill}{rgb}{1.000000,0.576471,0.309804}%
\pgfsetfillcolor{currentfill}%
\pgfsetlinewidth{1.003750pt}%
\definecolor{currentstroke}{rgb}{1.000000,0.576471,0.309804}%
\pgfsetstrokecolor{currentstroke}%
\pgfsetdash{}{0pt}%
\pgfsys@defobject{currentmarker}{\pgfqpoint{-0.033333in}{-0.033333in}}{\pgfqpoint{0.033333in}{0.033333in}}{%
\pgfpathmoveto{\pgfqpoint{0.000000in}{-0.033333in}}%
\pgfpathcurveto{\pgfqpoint{0.008840in}{-0.033333in}}{\pgfqpoint{0.017319in}{-0.029821in}}{\pgfqpoint{0.023570in}{-0.023570in}}%
\pgfpathcurveto{\pgfqpoint{0.029821in}{-0.017319in}}{\pgfqpoint{0.033333in}{-0.008840in}}{\pgfqpoint{0.033333in}{0.000000in}}%
\pgfpathcurveto{\pgfqpoint{0.033333in}{0.008840in}}{\pgfqpoint{0.029821in}{0.017319in}}{\pgfqpoint{0.023570in}{0.023570in}}%
\pgfpathcurveto{\pgfqpoint{0.017319in}{0.029821in}}{\pgfqpoint{0.008840in}{0.033333in}}{\pgfqpoint{0.000000in}{0.033333in}}%
\pgfpathcurveto{\pgfqpoint{-0.008840in}{0.033333in}}{\pgfqpoint{-0.017319in}{0.029821in}}{\pgfqpoint{-0.023570in}{0.023570in}}%
\pgfpathcurveto{\pgfqpoint{-0.029821in}{0.017319in}}{\pgfqpoint{-0.033333in}{0.008840in}}{\pgfqpoint{-0.033333in}{0.000000in}}%
\pgfpathcurveto{\pgfqpoint{-0.033333in}{-0.008840in}}{\pgfqpoint{-0.029821in}{-0.017319in}}{\pgfqpoint{-0.023570in}{-0.023570in}}%
\pgfpathcurveto{\pgfqpoint{-0.017319in}{-0.029821in}}{\pgfqpoint{-0.008840in}{-0.033333in}}{\pgfqpoint{0.000000in}{-0.033333in}}%
\pgfpathlineto{\pgfqpoint{0.000000in}{-0.033333in}}%
\pgfpathclose%
\pgfusepath{stroke,fill}%
}%
\begin{pgfscope}%
\pgfsys@transformshift{3.963086in}{1.777525in}%
\pgfsys@useobject{currentmarker}{}%
\end{pgfscope}%
\end{pgfscope}%
\begin{pgfscope}%
\pgfpathrectangle{\pgfqpoint{0.765000in}{0.660000in}}{\pgfqpoint{4.620000in}{4.620000in}}%
\pgfusepath{clip}%
\pgfsetrectcap%
\pgfsetroundjoin%
\pgfsetlinewidth{1.204500pt}%
\definecolor{currentstroke}{rgb}{1.000000,0.576471,0.309804}%
\pgfsetstrokecolor{currentstroke}%
\pgfsetdash{}{0pt}%
\pgfpathmoveto{\pgfqpoint{1.996793in}{2.971355in}}%
\pgfusepath{stroke}%
\end{pgfscope}%
\begin{pgfscope}%
\pgfpathrectangle{\pgfqpoint{0.765000in}{0.660000in}}{\pgfqpoint{4.620000in}{4.620000in}}%
\pgfusepath{clip}%
\pgfsetbuttcap%
\pgfsetroundjoin%
\definecolor{currentfill}{rgb}{1.000000,0.576471,0.309804}%
\pgfsetfillcolor{currentfill}%
\pgfsetlinewidth{1.003750pt}%
\definecolor{currentstroke}{rgb}{1.000000,0.576471,0.309804}%
\pgfsetstrokecolor{currentstroke}%
\pgfsetdash{}{0pt}%
\pgfsys@defobject{currentmarker}{\pgfqpoint{-0.033333in}{-0.033333in}}{\pgfqpoint{0.033333in}{0.033333in}}{%
\pgfpathmoveto{\pgfqpoint{0.000000in}{-0.033333in}}%
\pgfpathcurveto{\pgfqpoint{0.008840in}{-0.033333in}}{\pgfqpoint{0.017319in}{-0.029821in}}{\pgfqpoint{0.023570in}{-0.023570in}}%
\pgfpathcurveto{\pgfqpoint{0.029821in}{-0.017319in}}{\pgfqpoint{0.033333in}{-0.008840in}}{\pgfqpoint{0.033333in}{0.000000in}}%
\pgfpathcurveto{\pgfqpoint{0.033333in}{0.008840in}}{\pgfqpoint{0.029821in}{0.017319in}}{\pgfqpoint{0.023570in}{0.023570in}}%
\pgfpathcurveto{\pgfqpoint{0.017319in}{0.029821in}}{\pgfqpoint{0.008840in}{0.033333in}}{\pgfqpoint{0.000000in}{0.033333in}}%
\pgfpathcurveto{\pgfqpoint{-0.008840in}{0.033333in}}{\pgfqpoint{-0.017319in}{0.029821in}}{\pgfqpoint{-0.023570in}{0.023570in}}%
\pgfpathcurveto{\pgfqpoint{-0.029821in}{0.017319in}}{\pgfqpoint{-0.033333in}{0.008840in}}{\pgfqpoint{-0.033333in}{0.000000in}}%
\pgfpathcurveto{\pgfqpoint{-0.033333in}{-0.008840in}}{\pgfqpoint{-0.029821in}{-0.017319in}}{\pgfqpoint{-0.023570in}{-0.023570in}}%
\pgfpathcurveto{\pgfqpoint{-0.017319in}{-0.029821in}}{\pgfqpoint{-0.008840in}{-0.033333in}}{\pgfqpoint{0.000000in}{-0.033333in}}%
\pgfpathlineto{\pgfqpoint{0.000000in}{-0.033333in}}%
\pgfpathclose%
\pgfusepath{stroke,fill}%
}%
\begin{pgfscope}%
\pgfsys@transformshift{1.996793in}{2.971355in}%
\pgfsys@useobject{currentmarker}{}%
\end{pgfscope}%
\end{pgfscope}%
\begin{pgfscope}%
\pgfpathrectangle{\pgfqpoint{0.765000in}{0.660000in}}{\pgfqpoint{4.620000in}{4.620000in}}%
\pgfusepath{clip}%
\pgfsetrectcap%
\pgfsetroundjoin%
\pgfsetlinewidth{1.204500pt}%
\definecolor{currentstroke}{rgb}{1.000000,0.576471,0.309804}%
\pgfsetstrokecolor{currentstroke}%
\pgfsetdash{}{0pt}%
\pgfpathmoveto{\pgfqpoint{3.445341in}{2.661410in}}%
\pgfusepath{stroke}%
\end{pgfscope}%
\begin{pgfscope}%
\pgfpathrectangle{\pgfqpoint{0.765000in}{0.660000in}}{\pgfqpoint{4.620000in}{4.620000in}}%
\pgfusepath{clip}%
\pgfsetbuttcap%
\pgfsetroundjoin%
\definecolor{currentfill}{rgb}{1.000000,0.576471,0.309804}%
\pgfsetfillcolor{currentfill}%
\pgfsetlinewidth{1.003750pt}%
\definecolor{currentstroke}{rgb}{1.000000,0.576471,0.309804}%
\pgfsetstrokecolor{currentstroke}%
\pgfsetdash{}{0pt}%
\pgfsys@defobject{currentmarker}{\pgfqpoint{-0.033333in}{-0.033333in}}{\pgfqpoint{0.033333in}{0.033333in}}{%
\pgfpathmoveto{\pgfqpoint{0.000000in}{-0.033333in}}%
\pgfpathcurveto{\pgfqpoint{0.008840in}{-0.033333in}}{\pgfqpoint{0.017319in}{-0.029821in}}{\pgfqpoint{0.023570in}{-0.023570in}}%
\pgfpathcurveto{\pgfqpoint{0.029821in}{-0.017319in}}{\pgfqpoint{0.033333in}{-0.008840in}}{\pgfqpoint{0.033333in}{0.000000in}}%
\pgfpathcurveto{\pgfqpoint{0.033333in}{0.008840in}}{\pgfqpoint{0.029821in}{0.017319in}}{\pgfqpoint{0.023570in}{0.023570in}}%
\pgfpathcurveto{\pgfqpoint{0.017319in}{0.029821in}}{\pgfqpoint{0.008840in}{0.033333in}}{\pgfqpoint{0.000000in}{0.033333in}}%
\pgfpathcurveto{\pgfqpoint{-0.008840in}{0.033333in}}{\pgfqpoint{-0.017319in}{0.029821in}}{\pgfqpoint{-0.023570in}{0.023570in}}%
\pgfpathcurveto{\pgfqpoint{-0.029821in}{0.017319in}}{\pgfqpoint{-0.033333in}{0.008840in}}{\pgfqpoint{-0.033333in}{0.000000in}}%
\pgfpathcurveto{\pgfqpoint{-0.033333in}{-0.008840in}}{\pgfqpoint{-0.029821in}{-0.017319in}}{\pgfqpoint{-0.023570in}{-0.023570in}}%
\pgfpathcurveto{\pgfqpoint{-0.017319in}{-0.029821in}}{\pgfqpoint{-0.008840in}{-0.033333in}}{\pgfqpoint{0.000000in}{-0.033333in}}%
\pgfpathlineto{\pgfqpoint{0.000000in}{-0.033333in}}%
\pgfpathclose%
\pgfusepath{stroke,fill}%
}%
\begin{pgfscope}%
\pgfsys@transformshift{3.445341in}{2.661410in}%
\pgfsys@useobject{currentmarker}{}%
\end{pgfscope}%
\end{pgfscope}%
\begin{pgfscope}%
\pgfpathrectangle{\pgfqpoint{0.765000in}{0.660000in}}{\pgfqpoint{4.620000in}{4.620000in}}%
\pgfusepath{clip}%
\pgfsetrectcap%
\pgfsetroundjoin%
\pgfsetlinewidth{1.204500pt}%
\definecolor{currentstroke}{rgb}{1.000000,0.576471,0.309804}%
\pgfsetstrokecolor{currentstroke}%
\pgfsetdash{}{0pt}%
\pgfpathmoveto{\pgfqpoint{3.587267in}{2.885642in}}%
\pgfusepath{stroke}%
\end{pgfscope}%
\begin{pgfscope}%
\pgfpathrectangle{\pgfqpoint{0.765000in}{0.660000in}}{\pgfqpoint{4.620000in}{4.620000in}}%
\pgfusepath{clip}%
\pgfsetbuttcap%
\pgfsetroundjoin%
\definecolor{currentfill}{rgb}{1.000000,0.576471,0.309804}%
\pgfsetfillcolor{currentfill}%
\pgfsetlinewidth{1.003750pt}%
\definecolor{currentstroke}{rgb}{1.000000,0.576471,0.309804}%
\pgfsetstrokecolor{currentstroke}%
\pgfsetdash{}{0pt}%
\pgfsys@defobject{currentmarker}{\pgfqpoint{-0.033333in}{-0.033333in}}{\pgfqpoint{0.033333in}{0.033333in}}{%
\pgfpathmoveto{\pgfqpoint{0.000000in}{-0.033333in}}%
\pgfpathcurveto{\pgfqpoint{0.008840in}{-0.033333in}}{\pgfqpoint{0.017319in}{-0.029821in}}{\pgfqpoint{0.023570in}{-0.023570in}}%
\pgfpathcurveto{\pgfqpoint{0.029821in}{-0.017319in}}{\pgfqpoint{0.033333in}{-0.008840in}}{\pgfqpoint{0.033333in}{0.000000in}}%
\pgfpathcurveto{\pgfqpoint{0.033333in}{0.008840in}}{\pgfqpoint{0.029821in}{0.017319in}}{\pgfqpoint{0.023570in}{0.023570in}}%
\pgfpathcurveto{\pgfqpoint{0.017319in}{0.029821in}}{\pgfqpoint{0.008840in}{0.033333in}}{\pgfqpoint{0.000000in}{0.033333in}}%
\pgfpathcurveto{\pgfqpoint{-0.008840in}{0.033333in}}{\pgfqpoint{-0.017319in}{0.029821in}}{\pgfqpoint{-0.023570in}{0.023570in}}%
\pgfpathcurveto{\pgfqpoint{-0.029821in}{0.017319in}}{\pgfqpoint{-0.033333in}{0.008840in}}{\pgfqpoint{-0.033333in}{0.000000in}}%
\pgfpathcurveto{\pgfqpoint{-0.033333in}{-0.008840in}}{\pgfqpoint{-0.029821in}{-0.017319in}}{\pgfqpoint{-0.023570in}{-0.023570in}}%
\pgfpathcurveto{\pgfqpoint{-0.017319in}{-0.029821in}}{\pgfqpoint{-0.008840in}{-0.033333in}}{\pgfqpoint{0.000000in}{-0.033333in}}%
\pgfpathlineto{\pgfqpoint{0.000000in}{-0.033333in}}%
\pgfpathclose%
\pgfusepath{stroke,fill}%
}%
\begin{pgfscope}%
\pgfsys@transformshift{3.587267in}{2.885642in}%
\pgfsys@useobject{currentmarker}{}%
\end{pgfscope}%
\end{pgfscope}%
\begin{pgfscope}%
\pgfpathrectangle{\pgfqpoint{0.765000in}{0.660000in}}{\pgfqpoint{4.620000in}{4.620000in}}%
\pgfusepath{clip}%
\pgfsetrectcap%
\pgfsetroundjoin%
\pgfsetlinewidth{1.204500pt}%
\definecolor{currentstroke}{rgb}{1.000000,0.576471,0.309804}%
\pgfsetstrokecolor{currentstroke}%
\pgfsetdash{}{0pt}%
\pgfpathmoveto{\pgfqpoint{2.831557in}{1.637437in}}%
\pgfusepath{stroke}%
\end{pgfscope}%
\begin{pgfscope}%
\pgfpathrectangle{\pgfqpoint{0.765000in}{0.660000in}}{\pgfqpoint{4.620000in}{4.620000in}}%
\pgfusepath{clip}%
\pgfsetbuttcap%
\pgfsetroundjoin%
\definecolor{currentfill}{rgb}{1.000000,0.576471,0.309804}%
\pgfsetfillcolor{currentfill}%
\pgfsetlinewidth{1.003750pt}%
\definecolor{currentstroke}{rgb}{1.000000,0.576471,0.309804}%
\pgfsetstrokecolor{currentstroke}%
\pgfsetdash{}{0pt}%
\pgfsys@defobject{currentmarker}{\pgfqpoint{-0.033333in}{-0.033333in}}{\pgfqpoint{0.033333in}{0.033333in}}{%
\pgfpathmoveto{\pgfqpoint{0.000000in}{-0.033333in}}%
\pgfpathcurveto{\pgfqpoint{0.008840in}{-0.033333in}}{\pgfqpoint{0.017319in}{-0.029821in}}{\pgfqpoint{0.023570in}{-0.023570in}}%
\pgfpathcurveto{\pgfqpoint{0.029821in}{-0.017319in}}{\pgfqpoint{0.033333in}{-0.008840in}}{\pgfqpoint{0.033333in}{0.000000in}}%
\pgfpathcurveto{\pgfqpoint{0.033333in}{0.008840in}}{\pgfqpoint{0.029821in}{0.017319in}}{\pgfqpoint{0.023570in}{0.023570in}}%
\pgfpathcurveto{\pgfqpoint{0.017319in}{0.029821in}}{\pgfqpoint{0.008840in}{0.033333in}}{\pgfqpoint{0.000000in}{0.033333in}}%
\pgfpathcurveto{\pgfqpoint{-0.008840in}{0.033333in}}{\pgfqpoint{-0.017319in}{0.029821in}}{\pgfqpoint{-0.023570in}{0.023570in}}%
\pgfpathcurveto{\pgfqpoint{-0.029821in}{0.017319in}}{\pgfqpoint{-0.033333in}{0.008840in}}{\pgfqpoint{-0.033333in}{0.000000in}}%
\pgfpathcurveto{\pgfqpoint{-0.033333in}{-0.008840in}}{\pgfqpoint{-0.029821in}{-0.017319in}}{\pgfqpoint{-0.023570in}{-0.023570in}}%
\pgfpathcurveto{\pgfqpoint{-0.017319in}{-0.029821in}}{\pgfqpoint{-0.008840in}{-0.033333in}}{\pgfqpoint{0.000000in}{-0.033333in}}%
\pgfpathlineto{\pgfqpoint{0.000000in}{-0.033333in}}%
\pgfpathclose%
\pgfusepath{stroke,fill}%
}%
\begin{pgfscope}%
\pgfsys@transformshift{2.831557in}{1.637437in}%
\pgfsys@useobject{currentmarker}{}%
\end{pgfscope}%
\end{pgfscope}%
\begin{pgfscope}%
\pgfpathrectangle{\pgfqpoint{0.765000in}{0.660000in}}{\pgfqpoint{4.620000in}{4.620000in}}%
\pgfusepath{clip}%
\pgfsetrectcap%
\pgfsetroundjoin%
\pgfsetlinewidth{1.204500pt}%
\definecolor{currentstroke}{rgb}{1.000000,0.576471,0.309804}%
\pgfsetstrokecolor{currentstroke}%
\pgfsetdash{}{0pt}%
\pgfpathmoveto{\pgfqpoint{3.444635in}{2.730828in}}%
\pgfusepath{stroke}%
\end{pgfscope}%
\begin{pgfscope}%
\pgfpathrectangle{\pgfqpoint{0.765000in}{0.660000in}}{\pgfqpoint{4.620000in}{4.620000in}}%
\pgfusepath{clip}%
\pgfsetbuttcap%
\pgfsetroundjoin%
\definecolor{currentfill}{rgb}{1.000000,0.576471,0.309804}%
\pgfsetfillcolor{currentfill}%
\pgfsetlinewidth{1.003750pt}%
\definecolor{currentstroke}{rgb}{1.000000,0.576471,0.309804}%
\pgfsetstrokecolor{currentstroke}%
\pgfsetdash{}{0pt}%
\pgfsys@defobject{currentmarker}{\pgfqpoint{-0.033333in}{-0.033333in}}{\pgfqpoint{0.033333in}{0.033333in}}{%
\pgfpathmoveto{\pgfqpoint{0.000000in}{-0.033333in}}%
\pgfpathcurveto{\pgfqpoint{0.008840in}{-0.033333in}}{\pgfqpoint{0.017319in}{-0.029821in}}{\pgfqpoint{0.023570in}{-0.023570in}}%
\pgfpathcurveto{\pgfqpoint{0.029821in}{-0.017319in}}{\pgfqpoint{0.033333in}{-0.008840in}}{\pgfqpoint{0.033333in}{0.000000in}}%
\pgfpathcurveto{\pgfqpoint{0.033333in}{0.008840in}}{\pgfqpoint{0.029821in}{0.017319in}}{\pgfqpoint{0.023570in}{0.023570in}}%
\pgfpathcurveto{\pgfqpoint{0.017319in}{0.029821in}}{\pgfqpoint{0.008840in}{0.033333in}}{\pgfqpoint{0.000000in}{0.033333in}}%
\pgfpathcurveto{\pgfqpoint{-0.008840in}{0.033333in}}{\pgfqpoint{-0.017319in}{0.029821in}}{\pgfqpoint{-0.023570in}{0.023570in}}%
\pgfpathcurveto{\pgfqpoint{-0.029821in}{0.017319in}}{\pgfqpoint{-0.033333in}{0.008840in}}{\pgfqpoint{-0.033333in}{0.000000in}}%
\pgfpathcurveto{\pgfqpoint{-0.033333in}{-0.008840in}}{\pgfqpoint{-0.029821in}{-0.017319in}}{\pgfqpoint{-0.023570in}{-0.023570in}}%
\pgfpathcurveto{\pgfqpoint{-0.017319in}{-0.029821in}}{\pgfqpoint{-0.008840in}{-0.033333in}}{\pgfqpoint{0.000000in}{-0.033333in}}%
\pgfpathlineto{\pgfqpoint{0.000000in}{-0.033333in}}%
\pgfpathclose%
\pgfusepath{stroke,fill}%
}%
\begin{pgfscope}%
\pgfsys@transformshift{3.444635in}{2.730828in}%
\pgfsys@useobject{currentmarker}{}%
\end{pgfscope}%
\end{pgfscope}%
\begin{pgfscope}%
\pgfpathrectangle{\pgfqpoint{0.765000in}{0.660000in}}{\pgfqpoint{4.620000in}{4.620000in}}%
\pgfusepath{clip}%
\pgfsetrectcap%
\pgfsetroundjoin%
\pgfsetlinewidth{1.204500pt}%
\definecolor{currentstroke}{rgb}{1.000000,0.576471,0.309804}%
\pgfsetstrokecolor{currentstroke}%
\pgfsetdash{}{0pt}%
\pgfpathmoveto{\pgfqpoint{2.022576in}{3.041647in}}%
\pgfusepath{stroke}%
\end{pgfscope}%
\begin{pgfscope}%
\pgfpathrectangle{\pgfqpoint{0.765000in}{0.660000in}}{\pgfqpoint{4.620000in}{4.620000in}}%
\pgfusepath{clip}%
\pgfsetbuttcap%
\pgfsetroundjoin%
\definecolor{currentfill}{rgb}{1.000000,0.576471,0.309804}%
\pgfsetfillcolor{currentfill}%
\pgfsetlinewidth{1.003750pt}%
\definecolor{currentstroke}{rgb}{1.000000,0.576471,0.309804}%
\pgfsetstrokecolor{currentstroke}%
\pgfsetdash{}{0pt}%
\pgfsys@defobject{currentmarker}{\pgfqpoint{-0.033333in}{-0.033333in}}{\pgfqpoint{0.033333in}{0.033333in}}{%
\pgfpathmoveto{\pgfqpoint{0.000000in}{-0.033333in}}%
\pgfpathcurveto{\pgfqpoint{0.008840in}{-0.033333in}}{\pgfqpoint{0.017319in}{-0.029821in}}{\pgfqpoint{0.023570in}{-0.023570in}}%
\pgfpathcurveto{\pgfqpoint{0.029821in}{-0.017319in}}{\pgfqpoint{0.033333in}{-0.008840in}}{\pgfqpoint{0.033333in}{0.000000in}}%
\pgfpathcurveto{\pgfqpoint{0.033333in}{0.008840in}}{\pgfqpoint{0.029821in}{0.017319in}}{\pgfqpoint{0.023570in}{0.023570in}}%
\pgfpathcurveto{\pgfqpoint{0.017319in}{0.029821in}}{\pgfqpoint{0.008840in}{0.033333in}}{\pgfqpoint{0.000000in}{0.033333in}}%
\pgfpathcurveto{\pgfqpoint{-0.008840in}{0.033333in}}{\pgfqpoint{-0.017319in}{0.029821in}}{\pgfqpoint{-0.023570in}{0.023570in}}%
\pgfpathcurveto{\pgfqpoint{-0.029821in}{0.017319in}}{\pgfqpoint{-0.033333in}{0.008840in}}{\pgfqpoint{-0.033333in}{0.000000in}}%
\pgfpathcurveto{\pgfqpoint{-0.033333in}{-0.008840in}}{\pgfqpoint{-0.029821in}{-0.017319in}}{\pgfqpoint{-0.023570in}{-0.023570in}}%
\pgfpathcurveto{\pgfqpoint{-0.017319in}{-0.029821in}}{\pgfqpoint{-0.008840in}{-0.033333in}}{\pgfqpoint{0.000000in}{-0.033333in}}%
\pgfpathlineto{\pgfqpoint{0.000000in}{-0.033333in}}%
\pgfpathclose%
\pgfusepath{stroke,fill}%
}%
\begin{pgfscope}%
\pgfsys@transformshift{2.022576in}{3.041647in}%
\pgfsys@useobject{currentmarker}{}%
\end{pgfscope}%
\end{pgfscope}%
\begin{pgfscope}%
\pgfpathrectangle{\pgfqpoint{0.765000in}{0.660000in}}{\pgfqpoint{4.620000in}{4.620000in}}%
\pgfusepath{clip}%
\pgfsetrectcap%
\pgfsetroundjoin%
\pgfsetlinewidth{1.204500pt}%
\definecolor{currentstroke}{rgb}{1.000000,0.576471,0.309804}%
\pgfsetstrokecolor{currentstroke}%
\pgfsetdash{}{0pt}%
\pgfpathmoveto{\pgfqpoint{2.227895in}{2.740043in}}%
\pgfusepath{stroke}%
\end{pgfscope}%
\begin{pgfscope}%
\pgfpathrectangle{\pgfqpoint{0.765000in}{0.660000in}}{\pgfqpoint{4.620000in}{4.620000in}}%
\pgfusepath{clip}%
\pgfsetbuttcap%
\pgfsetroundjoin%
\definecolor{currentfill}{rgb}{1.000000,0.576471,0.309804}%
\pgfsetfillcolor{currentfill}%
\pgfsetlinewidth{1.003750pt}%
\definecolor{currentstroke}{rgb}{1.000000,0.576471,0.309804}%
\pgfsetstrokecolor{currentstroke}%
\pgfsetdash{}{0pt}%
\pgfsys@defobject{currentmarker}{\pgfqpoint{-0.033333in}{-0.033333in}}{\pgfqpoint{0.033333in}{0.033333in}}{%
\pgfpathmoveto{\pgfqpoint{0.000000in}{-0.033333in}}%
\pgfpathcurveto{\pgfqpoint{0.008840in}{-0.033333in}}{\pgfqpoint{0.017319in}{-0.029821in}}{\pgfqpoint{0.023570in}{-0.023570in}}%
\pgfpathcurveto{\pgfqpoint{0.029821in}{-0.017319in}}{\pgfqpoint{0.033333in}{-0.008840in}}{\pgfqpoint{0.033333in}{0.000000in}}%
\pgfpathcurveto{\pgfqpoint{0.033333in}{0.008840in}}{\pgfqpoint{0.029821in}{0.017319in}}{\pgfqpoint{0.023570in}{0.023570in}}%
\pgfpathcurveto{\pgfqpoint{0.017319in}{0.029821in}}{\pgfqpoint{0.008840in}{0.033333in}}{\pgfqpoint{0.000000in}{0.033333in}}%
\pgfpathcurveto{\pgfqpoint{-0.008840in}{0.033333in}}{\pgfqpoint{-0.017319in}{0.029821in}}{\pgfqpoint{-0.023570in}{0.023570in}}%
\pgfpathcurveto{\pgfqpoint{-0.029821in}{0.017319in}}{\pgfqpoint{-0.033333in}{0.008840in}}{\pgfqpoint{-0.033333in}{0.000000in}}%
\pgfpathcurveto{\pgfqpoint{-0.033333in}{-0.008840in}}{\pgfqpoint{-0.029821in}{-0.017319in}}{\pgfqpoint{-0.023570in}{-0.023570in}}%
\pgfpathcurveto{\pgfqpoint{-0.017319in}{-0.029821in}}{\pgfqpoint{-0.008840in}{-0.033333in}}{\pgfqpoint{0.000000in}{-0.033333in}}%
\pgfpathlineto{\pgfqpoint{0.000000in}{-0.033333in}}%
\pgfpathclose%
\pgfusepath{stroke,fill}%
}%
\begin{pgfscope}%
\pgfsys@transformshift{2.227895in}{2.740043in}%
\pgfsys@useobject{currentmarker}{}%
\end{pgfscope}%
\end{pgfscope}%
\begin{pgfscope}%
\pgfpathrectangle{\pgfqpoint{0.765000in}{0.660000in}}{\pgfqpoint{4.620000in}{4.620000in}}%
\pgfusepath{clip}%
\pgfsetrectcap%
\pgfsetroundjoin%
\pgfsetlinewidth{1.204500pt}%
\definecolor{currentstroke}{rgb}{1.000000,0.576471,0.309804}%
\pgfsetstrokecolor{currentstroke}%
\pgfsetdash{}{0pt}%
\pgfpathmoveto{\pgfqpoint{3.672918in}{2.713605in}}%
\pgfusepath{stroke}%
\end{pgfscope}%
\begin{pgfscope}%
\pgfpathrectangle{\pgfqpoint{0.765000in}{0.660000in}}{\pgfqpoint{4.620000in}{4.620000in}}%
\pgfusepath{clip}%
\pgfsetbuttcap%
\pgfsetroundjoin%
\definecolor{currentfill}{rgb}{1.000000,0.576471,0.309804}%
\pgfsetfillcolor{currentfill}%
\pgfsetlinewidth{1.003750pt}%
\definecolor{currentstroke}{rgb}{1.000000,0.576471,0.309804}%
\pgfsetstrokecolor{currentstroke}%
\pgfsetdash{}{0pt}%
\pgfsys@defobject{currentmarker}{\pgfqpoint{-0.033333in}{-0.033333in}}{\pgfqpoint{0.033333in}{0.033333in}}{%
\pgfpathmoveto{\pgfqpoint{0.000000in}{-0.033333in}}%
\pgfpathcurveto{\pgfqpoint{0.008840in}{-0.033333in}}{\pgfqpoint{0.017319in}{-0.029821in}}{\pgfqpoint{0.023570in}{-0.023570in}}%
\pgfpathcurveto{\pgfqpoint{0.029821in}{-0.017319in}}{\pgfqpoint{0.033333in}{-0.008840in}}{\pgfqpoint{0.033333in}{0.000000in}}%
\pgfpathcurveto{\pgfqpoint{0.033333in}{0.008840in}}{\pgfqpoint{0.029821in}{0.017319in}}{\pgfqpoint{0.023570in}{0.023570in}}%
\pgfpathcurveto{\pgfqpoint{0.017319in}{0.029821in}}{\pgfqpoint{0.008840in}{0.033333in}}{\pgfqpoint{0.000000in}{0.033333in}}%
\pgfpathcurveto{\pgfqpoint{-0.008840in}{0.033333in}}{\pgfqpoint{-0.017319in}{0.029821in}}{\pgfqpoint{-0.023570in}{0.023570in}}%
\pgfpathcurveto{\pgfqpoint{-0.029821in}{0.017319in}}{\pgfqpoint{-0.033333in}{0.008840in}}{\pgfqpoint{-0.033333in}{0.000000in}}%
\pgfpathcurveto{\pgfqpoint{-0.033333in}{-0.008840in}}{\pgfqpoint{-0.029821in}{-0.017319in}}{\pgfqpoint{-0.023570in}{-0.023570in}}%
\pgfpathcurveto{\pgfqpoint{-0.017319in}{-0.029821in}}{\pgfqpoint{-0.008840in}{-0.033333in}}{\pgfqpoint{0.000000in}{-0.033333in}}%
\pgfpathlineto{\pgfqpoint{0.000000in}{-0.033333in}}%
\pgfpathclose%
\pgfusepath{stroke,fill}%
}%
\begin{pgfscope}%
\pgfsys@transformshift{3.672918in}{2.713605in}%
\pgfsys@useobject{currentmarker}{}%
\end{pgfscope}%
\end{pgfscope}%
\begin{pgfscope}%
\pgfpathrectangle{\pgfqpoint{0.765000in}{0.660000in}}{\pgfqpoint{4.620000in}{4.620000in}}%
\pgfusepath{clip}%
\pgfsetrectcap%
\pgfsetroundjoin%
\pgfsetlinewidth{1.204500pt}%
\definecolor{currentstroke}{rgb}{1.000000,0.576471,0.309804}%
\pgfsetstrokecolor{currentstroke}%
\pgfsetdash{}{0pt}%
\pgfpathmoveto{\pgfqpoint{3.071196in}{2.463441in}}%
\pgfusepath{stroke}%
\end{pgfscope}%
\begin{pgfscope}%
\pgfpathrectangle{\pgfqpoint{0.765000in}{0.660000in}}{\pgfqpoint{4.620000in}{4.620000in}}%
\pgfusepath{clip}%
\pgfsetbuttcap%
\pgfsetroundjoin%
\definecolor{currentfill}{rgb}{1.000000,0.576471,0.309804}%
\pgfsetfillcolor{currentfill}%
\pgfsetlinewidth{1.003750pt}%
\definecolor{currentstroke}{rgb}{1.000000,0.576471,0.309804}%
\pgfsetstrokecolor{currentstroke}%
\pgfsetdash{}{0pt}%
\pgfsys@defobject{currentmarker}{\pgfqpoint{-0.033333in}{-0.033333in}}{\pgfqpoint{0.033333in}{0.033333in}}{%
\pgfpathmoveto{\pgfqpoint{0.000000in}{-0.033333in}}%
\pgfpathcurveto{\pgfqpoint{0.008840in}{-0.033333in}}{\pgfqpoint{0.017319in}{-0.029821in}}{\pgfqpoint{0.023570in}{-0.023570in}}%
\pgfpathcurveto{\pgfqpoint{0.029821in}{-0.017319in}}{\pgfqpoint{0.033333in}{-0.008840in}}{\pgfqpoint{0.033333in}{0.000000in}}%
\pgfpathcurveto{\pgfqpoint{0.033333in}{0.008840in}}{\pgfqpoint{0.029821in}{0.017319in}}{\pgfqpoint{0.023570in}{0.023570in}}%
\pgfpathcurveto{\pgfqpoint{0.017319in}{0.029821in}}{\pgfqpoint{0.008840in}{0.033333in}}{\pgfqpoint{0.000000in}{0.033333in}}%
\pgfpathcurveto{\pgfqpoint{-0.008840in}{0.033333in}}{\pgfqpoint{-0.017319in}{0.029821in}}{\pgfqpoint{-0.023570in}{0.023570in}}%
\pgfpathcurveto{\pgfqpoint{-0.029821in}{0.017319in}}{\pgfqpoint{-0.033333in}{0.008840in}}{\pgfqpoint{-0.033333in}{0.000000in}}%
\pgfpathcurveto{\pgfqpoint{-0.033333in}{-0.008840in}}{\pgfqpoint{-0.029821in}{-0.017319in}}{\pgfqpoint{-0.023570in}{-0.023570in}}%
\pgfpathcurveto{\pgfqpoint{-0.017319in}{-0.029821in}}{\pgfqpoint{-0.008840in}{-0.033333in}}{\pgfqpoint{0.000000in}{-0.033333in}}%
\pgfpathlineto{\pgfqpoint{0.000000in}{-0.033333in}}%
\pgfpathclose%
\pgfusepath{stroke,fill}%
}%
\begin{pgfscope}%
\pgfsys@transformshift{3.071196in}{2.463441in}%
\pgfsys@useobject{currentmarker}{}%
\end{pgfscope}%
\end{pgfscope}%
\begin{pgfscope}%
\pgfpathrectangle{\pgfqpoint{0.765000in}{0.660000in}}{\pgfqpoint{4.620000in}{4.620000in}}%
\pgfusepath{clip}%
\pgfsetrectcap%
\pgfsetroundjoin%
\pgfsetlinewidth{1.204500pt}%
\definecolor{currentstroke}{rgb}{1.000000,0.576471,0.309804}%
\pgfsetstrokecolor{currentstroke}%
\pgfsetdash{}{0pt}%
\pgfpathmoveto{\pgfqpoint{2.792784in}{1.052185in}}%
\pgfusepath{stroke}%
\end{pgfscope}%
\begin{pgfscope}%
\pgfpathrectangle{\pgfqpoint{0.765000in}{0.660000in}}{\pgfqpoint{4.620000in}{4.620000in}}%
\pgfusepath{clip}%
\pgfsetbuttcap%
\pgfsetroundjoin%
\definecolor{currentfill}{rgb}{1.000000,0.576471,0.309804}%
\pgfsetfillcolor{currentfill}%
\pgfsetlinewidth{1.003750pt}%
\definecolor{currentstroke}{rgb}{1.000000,0.576471,0.309804}%
\pgfsetstrokecolor{currentstroke}%
\pgfsetdash{}{0pt}%
\pgfsys@defobject{currentmarker}{\pgfqpoint{-0.033333in}{-0.033333in}}{\pgfqpoint{0.033333in}{0.033333in}}{%
\pgfpathmoveto{\pgfqpoint{0.000000in}{-0.033333in}}%
\pgfpathcurveto{\pgfqpoint{0.008840in}{-0.033333in}}{\pgfqpoint{0.017319in}{-0.029821in}}{\pgfqpoint{0.023570in}{-0.023570in}}%
\pgfpathcurveto{\pgfqpoint{0.029821in}{-0.017319in}}{\pgfqpoint{0.033333in}{-0.008840in}}{\pgfqpoint{0.033333in}{0.000000in}}%
\pgfpathcurveto{\pgfqpoint{0.033333in}{0.008840in}}{\pgfqpoint{0.029821in}{0.017319in}}{\pgfqpoint{0.023570in}{0.023570in}}%
\pgfpathcurveto{\pgfqpoint{0.017319in}{0.029821in}}{\pgfqpoint{0.008840in}{0.033333in}}{\pgfqpoint{0.000000in}{0.033333in}}%
\pgfpathcurveto{\pgfqpoint{-0.008840in}{0.033333in}}{\pgfqpoint{-0.017319in}{0.029821in}}{\pgfqpoint{-0.023570in}{0.023570in}}%
\pgfpathcurveto{\pgfqpoint{-0.029821in}{0.017319in}}{\pgfqpoint{-0.033333in}{0.008840in}}{\pgfqpoint{-0.033333in}{0.000000in}}%
\pgfpathcurveto{\pgfqpoint{-0.033333in}{-0.008840in}}{\pgfqpoint{-0.029821in}{-0.017319in}}{\pgfqpoint{-0.023570in}{-0.023570in}}%
\pgfpathcurveto{\pgfqpoint{-0.017319in}{-0.029821in}}{\pgfqpoint{-0.008840in}{-0.033333in}}{\pgfqpoint{0.000000in}{-0.033333in}}%
\pgfpathlineto{\pgfqpoint{0.000000in}{-0.033333in}}%
\pgfpathclose%
\pgfusepath{stroke,fill}%
}%
\begin{pgfscope}%
\pgfsys@transformshift{2.792784in}{1.052185in}%
\pgfsys@useobject{currentmarker}{}%
\end{pgfscope}%
\end{pgfscope}%
\begin{pgfscope}%
\pgfpathrectangle{\pgfqpoint{0.765000in}{0.660000in}}{\pgfqpoint{4.620000in}{4.620000in}}%
\pgfusepath{clip}%
\pgfsetrectcap%
\pgfsetroundjoin%
\pgfsetlinewidth{1.204500pt}%
\definecolor{currentstroke}{rgb}{1.000000,0.576471,0.309804}%
\pgfsetstrokecolor{currentstroke}%
\pgfsetdash{}{0pt}%
\pgfpathmoveto{\pgfqpoint{4.192352in}{3.019764in}}%
\pgfusepath{stroke}%
\end{pgfscope}%
\begin{pgfscope}%
\pgfpathrectangle{\pgfqpoint{0.765000in}{0.660000in}}{\pgfqpoint{4.620000in}{4.620000in}}%
\pgfusepath{clip}%
\pgfsetbuttcap%
\pgfsetroundjoin%
\definecolor{currentfill}{rgb}{1.000000,0.576471,0.309804}%
\pgfsetfillcolor{currentfill}%
\pgfsetlinewidth{1.003750pt}%
\definecolor{currentstroke}{rgb}{1.000000,0.576471,0.309804}%
\pgfsetstrokecolor{currentstroke}%
\pgfsetdash{}{0pt}%
\pgfsys@defobject{currentmarker}{\pgfqpoint{-0.033333in}{-0.033333in}}{\pgfqpoint{0.033333in}{0.033333in}}{%
\pgfpathmoveto{\pgfqpoint{0.000000in}{-0.033333in}}%
\pgfpathcurveto{\pgfqpoint{0.008840in}{-0.033333in}}{\pgfqpoint{0.017319in}{-0.029821in}}{\pgfqpoint{0.023570in}{-0.023570in}}%
\pgfpathcurveto{\pgfqpoint{0.029821in}{-0.017319in}}{\pgfqpoint{0.033333in}{-0.008840in}}{\pgfqpoint{0.033333in}{0.000000in}}%
\pgfpathcurveto{\pgfqpoint{0.033333in}{0.008840in}}{\pgfqpoint{0.029821in}{0.017319in}}{\pgfqpoint{0.023570in}{0.023570in}}%
\pgfpathcurveto{\pgfqpoint{0.017319in}{0.029821in}}{\pgfqpoint{0.008840in}{0.033333in}}{\pgfqpoint{0.000000in}{0.033333in}}%
\pgfpathcurveto{\pgfqpoint{-0.008840in}{0.033333in}}{\pgfqpoint{-0.017319in}{0.029821in}}{\pgfqpoint{-0.023570in}{0.023570in}}%
\pgfpathcurveto{\pgfqpoint{-0.029821in}{0.017319in}}{\pgfqpoint{-0.033333in}{0.008840in}}{\pgfqpoint{-0.033333in}{0.000000in}}%
\pgfpathcurveto{\pgfqpoint{-0.033333in}{-0.008840in}}{\pgfqpoint{-0.029821in}{-0.017319in}}{\pgfqpoint{-0.023570in}{-0.023570in}}%
\pgfpathcurveto{\pgfqpoint{-0.017319in}{-0.029821in}}{\pgfqpoint{-0.008840in}{-0.033333in}}{\pgfqpoint{0.000000in}{-0.033333in}}%
\pgfpathlineto{\pgfqpoint{0.000000in}{-0.033333in}}%
\pgfpathclose%
\pgfusepath{stroke,fill}%
}%
\begin{pgfscope}%
\pgfsys@transformshift{4.192352in}{3.019764in}%
\pgfsys@useobject{currentmarker}{}%
\end{pgfscope}%
\end{pgfscope}%
\begin{pgfscope}%
\pgfpathrectangle{\pgfqpoint{0.765000in}{0.660000in}}{\pgfqpoint{4.620000in}{4.620000in}}%
\pgfusepath{clip}%
\pgfsetrectcap%
\pgfsetroundjoin%
\pgfsetlinewidth{1.204500pt}%
\definecolor{currentstroke}{rgb}{1.000000,0.576471,0.309804}%
\pgfsetstrokecolor{currentstroke}%
\pgfsetdash{}{0pt}%
\pgfpathmoveto{\pgfqpoint{2.094996in}{2.008152in}}%
\pgfusepath{stroke}%
\end{pgfscope}%
\begin{pgfscope}%
\pgfpathrectangle{\pgfqpoint{0.765000in}{0.660000in}}{\pgfqpoint{4.620000in}{4.620000in}}%
\pgfusepath{clip}%
\pgfsetbuttcap%
\pgfsetroundjoin%
\definecolor{currentfill}{rgb}{1.000000,0.576471,0.309804}%
\pgfsetfillcolor{currentfill}%
\pgfsetlinewidth{1.003750pt}%
\definecolor{currentstroke}{rgb}{1.000000,0.576471,0.309804}%
\pgfsetstrokecolor{currentstroke}%
\pgfsetdash{}{0pt}%
\pgfsys@defobject{currentmarker}{\pgfqpoint{-0.033333in}{-0.033333in}}{\pgfqpoint{0.033333in}{0.033333in}}{%
\pgfpathmoveto{\pgfqpoint{0.000000in}{-0.033333in}}%
\pgfpathcurveto{\pgfqpoint{0.008840in}{-0.033333in}}{\pgfqpoint{0.017319in}{-0.029821in}}{\pgfqpoint{0.023570in}{-0.023570in}}%
\pgfpathcurveto{\pgfqpoint{0.029821in}{-0.017319in}}{\pgfqpoint{0.033333in}{-0.008840in}}{\pgfqpoint{0.033333in}{0.000000in}}%
\pgfpathcurveto{\pgfqpoint{0.033333in}{0.008840in}}{\pgfqpoint{0.029821in}{0.017319in}}{\pgfqpoint{0.023570in}{0.023570in}}%
\pgfpathcurveto{\pgfqpoint{0.017319in}{0.029821in}}{\pgfqpoint{0.008840in}{0.033333in}}{\pgfqpoint{0.000000in}{0.033333in}}%
\pgfpathcurveto{\pgfqpoint{-0.008840in}{0.033333in}}{\pgfqpoint{-0.017319in}{0.029821in}}{\pgfqpoint{-0.023570in}{0.023570in}}%
\pgfpathcurveto{\pgfqpoint{-0.029821in}{0.017319in}}{\pgfqpoint{-0.033333in}{0.008840in}}{\pgfqpoint{-0.033333in}{0.000000in}}%
\pgfpathcurveto{\pgfqpoint{-0.033333in}{-0.008840in}}{\pgfqpoint{-0.029821in}{-0.017319in}}{\pgfqpoint{-0.023570in}{-0.023570in}}%
\pgfpathcurveto{\pgfqpoint{-0.017319in}{-0.029821in}}{\pgfqpoint{-0.008840in}{-0.033333in}}{\pgfqpoint{0.000000in}{-0.033333in}}%
\pgfpathlineto{\pgfqpoint{0.000000in}{-0.033333in}}%
\pgfpathclose%
\pgfusepath{stroke,fill}%
}%
\begin{pgfscope}%
\pgfsys@transformshift{2.094996in}{2.008152in}%
\pgfsys@useobject{currentmarker}{}%
\end{pgfscope}%
\end{pgfscope}%
\begin{pgfscope}%
\pgfpathrectangle{\pgfqpoint{0.765000in}{0.660000in}}{\pgfqpoint{4.620000in}{4.620000in}}%
\pgfusepath{clip}%
\pgfsetrectcap%
\pgfsetroundjoin%
\pgfsetlinewidth{1.204500pt}%
\definecolor{currentstroke}{rgb}{1.000000,0.576471,0.309804}%
\pgfsetstrokecolor{currentstroke}%
\pgfsetdash{}{0pt}%
\pgfpathmoveto{\pgfqpoint{2.712047in}{1.598214in}}%
\pgfusepath{stroke}%
\end{pgfscope}%
\begin{pgfscope}%
\pgfpathrectangle{\pgfqpoint{0.765000in}{0.660000in}}{\pgfqpoint{4.620000in}{4.620000in}}%
\pgfusepath{clip}%
\pgfsetbuttcap%
\pgfsetroundjoin%
\definecolor{currentfill}{rgb}{1.000000,0.576471,0.309804}%
\pgfsetfillcolor{currentfill}%
\pgfsetlinewidth{1.003750pt}%
\definecolor{currentstroke}{rgb}{1.000000,0.576471,0.309804}%
\pgfsetstrokecolor{currentstroke}%
\pgfsetdash{}{0pt}%
\pgfsys@defobject{currentmarker}{\pgfqpoint{-0.033333in}{-0.033333in}}{\pgfqpoint{0.033333in}{0.033333in}}{%
\pgfpathmoveto{\pgfqpoint{0.000000in}{-0.033333in}}%
\pgfpathcurveto{\pgfqpoint{0.008840in}{-0.033333in}}{\pgfqpoint{0.017319in}{-0.029821in}}{\pgfqpoint{0.023570in}{-0.023570in}}%
\pgfpathcurveto{\pgfqpoint{0.029821in}{-0.017319in}}{\pgfqpoint{0.033333in}{-0.008840in}}{\pgfqpoint{0.033333in}{0.000000in}}%
\pgfpathcurveto{\pgfqpoint{0.033333in}{0.008840in}}{\pgfqpoint{0.029821in}{0.017319in}}{\pgfqpoint{0.023570in}{0.023570in}}%
\pgfpathcurveto{\pgfqpoint{0.017319in}{0.029821in}}{\pgfqpoint{0.008840in}{0.033333in}}{\pgfqpoint{0.000000in}{0.033333in}}%
\pgfpathcurveto{\pgfqpoint{-0.008840in}{0.033333in}}{\pgfqpoint{-0.017319in}{0.029821in}}{\pgfqpoint{-0.023570in}{0.023570in}}%
\pgfpathcurveto{\pgfqpoint{-0.029821in}{0.017319in}}{\pgfqpoint{-0.033333in}{0.008840in}}{\pgfqpoint{-0.033333in}{0.000000in}}%
\pgfpathcurveto{\pgfqpoint{-0.033333in}{-0.008840in}}{\pgfqpoint{-0.029821in}{-0.017319in}}{\pgfqpoint{-0.023570in}{-0.023570in}}%
\pgfpathcurveto{\pgfqpoint{-0.017319in}{-0.029821in}}{\pgfqpoint{-0.008840in}{-0.033333in}}{\pgfqpoint{0.000000in}{-0.033333in}}%
\pgfpathlineto{\pgfqpoint{0.000000in}{-0.033333in}}%
\pgfpathclose%
\pgfusepath{stroke,fill}%
}%
\begin{pgfscope}%
\pgfsys@transformshift{2.712047in}{1.598214in}%
\pgfsys@useobject{currentmarker}{}%
\end{pgfscope}%
\end{pgfscope}%
\begin{pgfscope}%
\pgfpathrectangle{\pgfqpoint{0.765000in}{0.660000in}}{\pgfqpoint{4.620000in}{4.620000in}}%
\pgfusepath{clip}%
\pgfsetrectcap%
\pgfsetroundjoin%
\pgfsetlinewidth{1.204500pt}%
\definecolor{currentstroke}{rgb}{1.000000,0.576471,0.309804}%
\pgfsetstrokecolor{currentstroke}%
\pgfsetdash{}{0pt}%
\pgfpathmoveto{\pgfqpoint{2.914668in}{1.820190in}}%
\pgfusepath{stroke}%
\end{pgfscope}%
\begin{pgfscope}%
\pgfpathrectangle{\pgfqpoint{0.765000in}{0.660000in}}{\pgfqpoint{4.620000in}{4.620000in}}%
\pgfusepath{clip}%
\pgfsetbuttcap%
\pgfsetroundjoin%
\definecolor{currentfill}{rgb}{1.000000,0.576471,0.309804}%
\pgfsetfillcolor{currentfill}%
\pgfsetlinewidth{1.003750pt}%
\definecolor{currentstroke}{rgb}{1.000000,0.576471,0.309804}%
\pgfsetstrokecolor{currentstroke}%
\pgfsetdash{}{0pt}%
\pgfsys@defobject{currentmarker}{\pgfqpoint{-0.033333in}{-0.033333in}}{\pgfqpoint{0.033333in}{0.033333in}}{%
\pgfpathmoveto{\pgfqpoint{0.000000in}{-0.033333in}}%
\pgfpathcurveto{\pgfqpoint{0.008840in}{-0.033333in}}{\pgfqpoint{0.017319in}{-0.029821in}}{\pgfqpoint{0.023570in}{-0.023570in}}%
\pgfpathcurveto{\pgfqpoint{0.029821in}{-0.017319in}}{\pgfqpoint{0.033333in}{-0.008840in}}{\pgfqpoint{0.033333in}{0.000000in}}%
\pgfpathcurveto{\pgfqpoint{0.033333in}{0.008840in}}{\pgfqpoint{0.029821in}{0.017319in}}{\pgfqpoint{0.023570in}{0.023570in}}%
\pgfpathcurveto{\pgfqpoint{0.017319in}{0.029821in}}{\pgfqpoint{0.008840in}{0.033333in}}{\pgfqpoint{0.000000in}{0.033333in}}%
\pgfpathcurveto{\pgfqpoint{-0.008840in}{0.033333in}}{\pgfqpoint{-0.017319in}{0.029821in}}{\pgfqpoint{-0.023570in}{0.023570in}}%
\pgfpathcurveto{\pgfqpoint{-0.029821in}{0.017319in}}{\pgfqpoint{-0.033333in}{0.008840in}}{\pgfqpoint{-0.033333in}{0.000000in}}%
\pgfpathcurveto{\pgfqpoint{-0.033333in}{-0.008840in}}{\pgfqpoint{-0.029821in}{-0.017319in}}{\pgfqpoint{-0.023570in}{-0.023570in}}%
\pgfpathcurveto{\pgfqpoint{-0.017319in}{-0.029821in}}{\pgfqpoint{-0.008840in}{-0.033333in}}{\pgfqpoint{0.000000in}{-0.033333in}}%
\pgfpathlineto{\pgfqpoint{0.000000in}{-0.033333in}}%
\pgfpathclose%
\pgfusepath{stroke,fill}%
}%
\begin{pgfscope}%
\pgfsys@transformshift{2.914668in}{1.820190in}%
\pgfsys@useobject{currentmarker}{}%
\end{pgfscope}%
\end{pgfscope}%
\begin{pgfscope}%
\pgfpathrectangle{\pgfqpoint{0.765000in}{0.660000in}}{\pgfqpoint{4.620000in}{4.620000in}}%
\pgfusepath{clip}%
\pgfsetrectcap%
\pgfsetroundjoin%
\pgfsetlinewidth{1.204500pt}%
\definecolor{currentstroke}{rgb}{1.000000,0.576471,0.309804}%
\pgfsetstrokecolor{currentstroke}%
\pgfsetdash{}{0pt}%
\pgfpathmoveto{\pgfqpoint{3.361043in}{2.355392in}}%
\pgfusepath{stroke}%
\end{pgfscope}%
\begin{pgfscope}%
\pgfpathrectangle{\pgfqpoint{0.765000in}{0.660000in}}{\pgfqpoint{4.620000in}{4.620000in}}%
\pgfusepath{clip}%
\pgfsetbuttcap%
\pgfsetroundjoin%
\definecolor{currentfill}{rgb}{1.000000,0.576471,0.309804}%
\pgfsetfillcolor{currentfill}%
\pgfsetlinewidth{1.003750pt}%
\definecolor{currentstroke}{rgb}{1.000000,0.576471,0.309804}%
\pgfsetstrokecolor{currentstroke}%
\pgfsetdash{}{0pt}%
\pgfsys@defobject{currentmarker}{\pgfqpoint{-0.033333in}{-0.033333in}}{\pgfqpoint{0.033333in}{0.033333in}}{%
\pgfpathmoveto{\pgfqpoint{0.000000in}{-0.033333in}}%
\pgfpathcurveto{\pgfqpoint{0.008840in}{-0.033333in}}{\pgfqpoint{0.017319in}{-0.029821in}}{\pgfqpoint{0.023570in}{-0.023570in}}%
\pgfpathcurveto{\pgfqpoint{0.029821in}{-0.017319in}}{\pgfqpoint{0.033333in}{-0.008840in}}{\pgfqpoint{0.033333in}{0.000000in}}%
\pgfpathcurveto{\pgfqpoint{0.033333in}{0.008840in}}{\pgfqpoint{0.029821in}{0.017319in}}{\pgfqpoint{0.023570in}{0.023570in}}%
\pgfpathcurveto{\pgfqpoint{0.017319in}{0.029821in}}{\pgfqpoint{0.008840in}{0.033333in}}{\pgfqpoint{0.000000in}{0.033333in}}%
\pgfpathcurveto{\pgfqpoint{-0.008840in}{0.033333in}}{\pgfqpoint{-0.017319in}{0.029821in}}{\pgfqpoint{-0.023570in}{0.023570in}}%
\pgfpathcurveto{\pgfqpoint{-0.029821in}{0.017319in}}{\pgfqpoint{-0.033333in}{0.008840in}}{\pgfqpoint{-0.033333in}{0.000000in}}%
\pgfpathcurveto{\pgfqpoint{-0.033333in}{-0.008840in}}{\pgfqpoint{-0.029821in}{-0.017319in}}{\pgfqpoint{-0.023570in}{-0.023570in}}%
\pgfpathcurveto{\pgfqpoint{-0.017319in}{-0.029821in}}{\pgfqpoint{-0.008840in}{-0.033333in}}{\pgfqpoint{0.000000in}{-0.033333in}}%
\pgfpathlineto{\pgfqpoint{0.000000in}{-0.033333in}}%
\pgfpathclose%
\pgfusepath{stroke,fill}%
}%
\begin{pgfscope}%
\pgfsys@transformshift{3.361043in}{2.355392in}%
\pgfsys@useobject{currentmarker}{}%
\end{pgfscope}%
\end{pgfscope}%
\begin{pgfscope}%
\pgfpathrectangle{\pgfqpoint{0.765000in}{0.660000in}}{\pgfqpoint{4.620000in}{4.620000in}}%
\pgfusepath{clip}%
\pgfsetrectcap%
\pgfsetroundjoin%
\pgfsetlinewidth{1.204500pt}%
\definecolor{currentstroke}{rgb}{1.000000,0.576471,0.309804}%
\pgfsetstrokecolor{currentstroke}%
\pgfsetdash{}{0pt}%
\pgfpathmoveto{\pgfqpoint{3.084115in}{2.616630in}}%
\pgfusepath{stroke}%
\end{pgfscope}%
\begin{pgfscope}%
\pgfpathrectangle{\pgfqpoint{0.765000in}{0.660000in}}{\pgfqpoint{4.620000in}{4.620000in}}%
\pgfusepath{clip}%
\pgfsetbuttcap%
\pgfsetroundjoin%
\definecolor{currentfill}{rgb}{1.000000,0.576471,0.309804}%
\pgfsetfillcolor{currentfill}%
\pgfsetlinewidth{1.003750pt}%
\definecolor{currentstroke}{rgb}{1.000000,0.576471,0.309804}%
\pgfsetstrokecolor{currentstroke}%
\pgfsetdash{}{0pt}%
\pgfsys@defobject{currentmarker}{\pgfqpoint{-0.033333in}{-0.033333in}}{\pgfqpoint{0.033333in}{0.033333in}}{%
\pgfpathmoveto{\pgfqpoint{0.000000in}{-0.033333in}}%
\pgfpathcurveto{\pgfqpoint{0.008840in}{-0.033333in}}{\pgfqpoint{0.017319in}{-0.029821in}}{\pgfqpoint{0.023570in}{-0.023570in}}%
\pgfpathcurveto{\pgfqpoint{0.029821in}{-0.017319in}}{\pgfqpoint{0.033333in}{-0.008840in}}{\pgfqpoint{0.033333in}{0.000000in}}%
\pgfpathcurveto{\pgfqpoint{0.033333in}{0.008840in}}{\pgfqpoint{0.029821in}{0.017319in}}{\pgfqpoint{0.023570in}{0.023570in}}%
\pgfpathcurveto{\pgfqpoint{0.017319in}{0.029821in}}{\pgfqpoint{0.008840in}{0.033333in}}{\pgfqpoint{0.000000in}{0.033333in}}%
\pgfpathcurveto{\pgfqpoint{-0.008840in}{0.033333in}}{\pgfqpoint{-0.017319in}{0.029821in}}{\pgfqpoint{-0.023570in}{0.023570in}}%
\pgfpathcurveto{\pgfqpoint{-0.029821in}{0.017319in}}{\pgfqpoint{-0.033333in}{0.008840in}}{\pgfqpoint{-0.033333in}{0.000000in}}%
\pgfpathcurveto{\pgfqpoint{-0.033333in}{-0.008840in}}{\pgfqpoint{-0.029821in}{-0.017319in}}{\pgfqpoint{-0.023570in}{-0.023570in}}%
\pgfpathcurveto{\pgfqpoint{-0.017319in}{-0.029821in}}{\pgfqpoint{-0.008840in}{-0.033333in}}{\pgfqpoint{0.000000in}{-0.033333in}}%
\pgfpathlineto{\pgfqpoint{0.000000in}{-0.033333in}}%
\pgfpathclose%
\pgfusepath{stroke,fill}%
}%
\begin{pgfscope}%
\pgfsys@transformshift{3.084115in}{2.616630in}%
\pgfsys@useobject{currentmarker}{}%
\end{pgfscope}%
\end{pgfscope}%
\begin{pgfscope}%
\pgfpathrectangle{\pgfqpoint{0.765000in}{0.660000in}}{\pgfqpoint{4.620000in}{4.620000in}}%
\pgfusepath{clip}%
\pgfsetrectcap%
\pgfsetroundjoin%
\pgfsetlinewidth{1.204500pt}%
\definecolor{currentstroke}{rgb}{1.000000,0.576471,0.309804}%
\pgfsetstrokecolor{currentstroke}%
\pgfsetdash{}{0pt}%
\pgfpathmoveto{\pgfqpoint{4.109688in}{2.028404in}}%
\pgfusepath{stroke}%
\end{pgfscope}%
\begin{pgfscope}%
\pgfpathrectangle{\pgfqpoint{0.765000in}{0.660000in}}{\pgfqpoint{4.620000in}{4.620000in}}%
\pgfusepath{clip}%
\pgfsetbuttcap%
\pgfsetroundjoin%
\definecolor{currentfill}{rgb}{1.000000,0.576471,0.309804}%
\pgfsetfillcolor{currentfill}%
\pgfsetlinewidth{1.003750pt}%
\definecolor{currentstroke}{rgb}{1.000000,0.576471,0.309804}%
\pgfsetstrokecolor{currentstroke}%
\pgfsetdash{}{0pt}%
\pgfsys@defobject{currentmarker}{\pgfqpoint{-0.033333in}{-0.033333in}}{\pgfqpoint{0.033333in}{0.033333in}}{%
\pgfpathmoveto{\pgfqpoint{0.000000in}{-0.033333in}}%
\pgfpathcurveto{\pgfqpoint{0.008840in}{-0.033333in}}{\pgfqpoint{0.017319in}{-0.029821in}}{\pgfqpoint{0.023570in}{-0.023570in}}%
\pgfpathcurveto{\pgfqpoint{0.029821in}{-0.017319in}}{\pgfqpoint{0.033333in}{-0.008840in}}{\pgfqpoint{0.033333in}{0.000000in}}%
\pgfpathcurveto{\pgfqpoint{0.033333in}{0.008840in}}{\pgfqpoint{0.029821in}{0.017319in}}{\pgfqpoint{0.023570in}{0.023570in}}%
\pgfpathcurveto{\pgfqpoint{0.017319in}{0.029821in}}{\pgfqpoint{0.008840in}{0.033333in}}{\pgfqpoint{0.000000in}{0.033333in}}%
\pgfpathcurveto{\pgfqpoint{-0.008840in}{0.033333in}}{\pgfqpoint{-0.017319in}{0.029821in}}{\pgfqpoint{-0.023570in}{0.023570in}}%
\pgfpathcurveto{\pgfqpoint{-0.029821in}{0.017319in}}{\pgfqpoint{-0.033333in}{0.008840in}}{\pgfqpoint{-0.033333in}{0.000000in}}%
\pgfpathcurveto{\pgfqpoint{-0.033333in}{-0.008840in}}{\pgfqpoint{-0.029821in}{-0.017319in}}{\pgfqpoint{-0.023570in}{-0.023570in}}%
\pgfpathcurveto{\pgfqpoint{-0.017319in}{-0.029821in}}{\pgfqpoint{-0.008840in}{-0.033333in}}{\pgfqpoint{0.000000in}{-0.033333in}}%
\pgfpathlineto{\pgfqpoint{0.000000in}{-0.033333in}}%
\pgfpathclose%
\pgfusepath{stroke,fill}%
}%
\begin{pgfscope}%
\pgfsys@transformshift{4.109688in}{2.028404in}%
\pgfsys@useobject{currentmarker}{}%
\end{pgfscope}%
\end{pgfscope}%
\begin{pgfscope}%
\pgfpathrectangle{\pgfqpoint{0.765000in}{0.660000in}}{\pgfqpoint{4.620000in}{4.620000in}}%
\pgfusepath{clip}%
\pgfsetrectcap%
\pgfsetroundjoin%
\pgfsetlinewidth{1.204500pt}%
\definecolor{currentstroke}{rgb}{1.000000,0.576471,0.309804}%
\pgfsetstrokecolor{currentstroke}%
\pgfsetdash{}{0pt}%
\pgfpathmoveto{\pgfqpoint{3.847259in}{3.022790in}}%
\pgfusepath{stroke}%
\end{pgfscope}%
\begin{pgfscope}%
\pgfpathrectangle{\pgfqpoint{0.765000in}{0.660000in}}{\pgfqpoint{4.620000in}{4.620000in}}%
\pgfusepath{clip}%
\pgfsetbuttcap%
\pgfsetroundjoin%
\definecolor{currentfill}{rgb}{1.000000,0.576471,0.309804}%
\pgfsetfillcolor{currentfill}%
\pgfsetlinewidth{1.003750pt}%
\definecolor{currentstroke}{rgb}{1.000000,0.576471,0.309804}%
\pgfsetstrokecolor{currentstroke}%
\pgfsetdash{}{0pt}%
\pgfsys@defobject{currentmarker}{\pgfqpoint{-0.033333in}{-0.033333in}}{\pgfqpoint{0.033333in}{0.033333in}}{%
\pgfpathmoveto{\pgfqpoint{0.000000in}{-0.033333in}}%
\pgfpathcurveto{\pgfqpoint{0.008840in}{-0.033333in}}{\pgfqpoint{0.017319in}{-0.029821in}}{\pgfqpoint{0.023570in}{-0.023570in}}%
\pgfpathcurveto{\pgfqpoint{0.029821in}{-0.017319in}}{\pgfqpoint{0.033333in}{-0.008840in}}{\pgfqpoint{0.033333in}{0.000000in}}%
\pgfpathcurveto{\pgfqpoint{0.033333in}{0.008840in}}{\pgfqpoint{0.029821in}{0.017319in}}{\pgfqpoint{0.023570in}{0.023570in}}%
\pgfpathcurveto{\pgfqpoint{0.017319in}{0.029821in}}{\pgfqpoint{0.008840in}{0.033333in}}{\pgfqpoint{0.000000in}{0.033333in}}%
\pgfpathcurveto{\pgfqpoint{-0.008840in}{0.033333in}}{\pgfqpoint{-0.017319in}{0.029821in}}{\pgfqpoint{-0.023570in}{0.023570in}}%
\pgfpathcurveto{\pgfqpoint{-0.029821in}{0.017319in}}{\pgfqpoint{-0.033333in}{0.008840in}}{\pgfqpoint{-0.033333in}{0.000000in}}%
\pgfpathcurveto{\pgfqpoint{-0.033333in}{-0.008840in}}{\pgfqpoint{-0.029821in}{-0.017319in}}{\pgfqpoint{-0.023570in}{-0.023570in}}%
\pgfpathcurveto{\pgfqpoint{-0.017319in}{-0.029821in}}{\pgfqpoint{-0.008840in}{-0.033333in}}{\pgfqpoint{0.000000in}{-0.033333in}}%
\pgfpathlineto{\pgfqpoint{0.000000in}{-0.033333in}}%
\pgfpathclose%
\pgfusepath{stroke,fill}%
}%
\begin{pgfscope}%
\pgfsys@transformshift{3.847259in}{3.022790in}%
\pgfsys@useobject{currentmarker}{}%
\end{pgfscope}%
\end{pgfscope}%
\begin{pgfscope}%
\pgfpathrectangle{\pgfqpoint{0.765000in}{0.660000in}}{\pgfqpoint{4.620000in}{4.620000in}}%
\pgfusepath{clip}%
\pgfsetrectcap%
\pgfsetroundjoin%
\pgfsetlinewidth{1.204500pt}%
\definecolor{currentstroke}{rgb}{1.000000,0.576471,0.309804}%
\pgfsetstrokecolor{currentstroke}%
\pgfsetdash{}{0pt}%
\pgfpathmoveto{\pgfqpoint{2.273112in}{2.305684in}}%
\pgfusepath{stroke}%
\end{pgfscope}%
\begin{pgfscope}%
\pgfpathrectangle{\pgfqpoint{0.765000in}{0.660000in}}{\pgfqpoint{4.620000in}{4.620000in}}%
\pgfusepath{clip}%
\pgfsetbuttcap%
\pgfsetroundjoin%
\definecolor{currentfill}{rgb}{1.000000,0.576471,0.309804}%
\pgfsetfillcolor{currentfill}%
\pgfsetlinewidth{1.003750pt}%
\definecolor{currentstroke}{rgb}{1.000000,0.576471,0.309804}%
\pgfsetstrokecolor{currentstroke}%
\pgfsetdash{}{0pt}%
\pgfsys@defobject{currentmarker}{\pgfqpoint{-0.033333in}{-0.033333in}}{\pgfqpoint{0.033333in}{0.033333in}}{%
\pgfpathmoveto{\pgfqpoint{0.000000in}{-0.033333in}}%
\pgfpathcurveto{\pgfqpoint{0.008840in}{-0.033333in}}{\pgfqpoint{0.017319in}{-0.029821in}}{\pgfqpoint{0.023570in}{-0.023570in}}%
\pgfpathcurveto{\pgfqpoint{0.029821in}{-0.017319in}}{\pgfqpoint{0.033333in}{-0.008840in}}{\pgfqpoint{0.033333in}{0.000000in}}%
\pgfpathcurveto{\pgfqpoint{0.033333in}{0.008840in}}{\pgfqpoint{0.029821in}{0.017319in}}{\pgfqpoint{0.023570in}{0.023570in}}%
\pgfpathcurveto{\pgfqpoint{0.017319in}{0.029821in}}{\pgfqpoint{0.008840in}{0.033333in}}{\pgfqpoint{0.000000in}{0.033333in}}%
\pgfpathcurveto{\pgfqpoint{-0.008840in}{0.033333in}}{\pgfqpoint{-0.017319in}{0.029821in}}{\pgfqpoint{-0.023570in}{0.023570in}}%
\pgfpathcurveto{\pgfqpoint{-0.029821in}{0.017319in}}{\pgfqpoint{-0.033333in}{0.008840in}}{\pgfqpoint{-0.033333in}{0.000000in}}%
\pgfpathcurveto{\pgfqpoint{-0.033333in}{-0.008840in}}{\pgfqpoint{-0.029821in}{-0.017319in}}{\pgfqpoint{-0.023570in}{-0.023570in}}%
\pgfpathcurveto{\pgfqpoint{-0.017319in}{-0.029821in}}{\pgfqpoint{-0.008840in}{-0.033333in}}{\pgfqpoint{0.000000in}{-0.033333in}}%
\pgfpathlineto{\pgfqpoint{0.000000in}{-0.033333in}}%
\pgfpathclose%
\pgfusepath{stroke,fill}%
}%
\begin{pgfscope}%
\pgfsys@transformshift{2.273112in}{2.305684in}%
\pgfsys@useobject{currentmarker}{}%
\end{pgfscope}%
\end{pgfscope}%
\begin{pgfscope}%
\pgfpathrectangle{\pgfqpoint{0.765000in}{0.660000in}}{\pgfqpoint{4.620000in}{4.620000in}}%
\pgfusepath{clip}%
\pgfsetrectcap%
\pgfsetroundjoin%
\pgfsetlinewidth{1.204500pt}%
\definecolor{currentstroke}{rgb}{1.000000,0.576471,0.309804}%
\pgfsetstrokecolor{currentstroke}%
\pgfsetdash{}{0pt}%
\pgfpathmoveto{\pgfqpoint{2.319262in}{2.307419in}}%
\pgfusepath{stroke}%
\end{pgfscope}%
\begin{pgfscope}%
\pgfpathrectangle{\pgfqpoint{0.765000in}{0.660000in}}{\pgfqpoint{4.620000in}{4.620000in}}%
\pgfusepath{clip}%
\pgfsetbuttcap%
\pgfsetroundjoin%
\definecolor{currentfill}{rgb}{1.000000,0.576471,0.309804}%
\pgfsetfillcolor{currentfill}%
\pgfsetlinewidth{1.003750pt}%
\definecolor{currentstroke}{rgb}{1.000000,0.576471,0.309804}%
\pgfsetstrokecolor{currentstroke}%
\pgfsetdash{}{0pt}%
\pgfsys@defobject{currentmarker}{\pgfqpoint{-0.033333in}{-0.033333in}}{\pgfqpoint{0.033333in}{0.033333in}}{%
\pgfpathmoveto{\pgfqpoint{0.000000in}{-0.033333in}}%
\pgfpathcurveto{\pgfqpoint{0.008840in}{-0.033333in}}{\pgfqpoint{0.017319in}{-0.029821in}}{\pgfqpoint{0.023570in}{-0.023570in}}%
\pgfpathcurveto{\pgfqpoint{0.029821in}{-0.017319in}}{\pgfqpoint{0.033333in}{-0.008840in}}{\pgfqpoint{0.033333in}{0.000000in}}%
\pgfpathcurveto{\pgfqpoint{0.033333in}{0.008840in}}{\pgfqpoint{0.029821in}{0.017319in}}{\pgfqpoint{0.023570in}{0.023570in}}%
\pgfpathcurveto{\pgfqpoint{0.017319in}{0.029821in}}{\pgfqpoint{0.008840in}{0.033333in}}{\pgfqpoint{0.000000in}{0.033333in}}%
\pgfpathcurveto{\pgfqpoint{-0.008840in}{0.033333in}}{\pgfqpoint{-0.017319in}{0.029821in}}{\pgfqpoint{-0.023570in}{0.023570in}}%
\pgfpathcurveto{\pgfqpoint{-0.029821in}{0.017319in}}{\pgfqpoint{-0.033333in}{0.008840in}}{\pgfqpoint{-0.033333in}{0.000000in}}%
\pgfpathcurveto{\pgfqpoint{-0.033333in}{-0.008840in}}{\pgfqpoint{-0.029821in}{-0.017319in}}{\pgfqpoint{-0.023570in}{-0.023570in}}%
\pgfpathcurveto{\pgfqpoint{-0.017319in}{-0.029821in}}{\pgfqpoint{-0.008840in}{-0.033333in}}{\pgfqpoint{0.000000in}{-0.033333in}}%
\pgfpathlineto{\pgfqpoint{0.000000in}{-0.033333in}}%
\pgfpathclose%
\pgfusepath{stroke,fill}%
}%
\begin{pgfscope}%
\pgfsys@transformshift{2.319262in}{2.307419in}%
\pgfsys@useobject{currentmarker}{}%
\end{pgfscope}%
\end{pgfscope}%
\begin{pgfscope}%
\pgfpathrectangle{\pgfqpoint{0.765000in}{0.660000in}}{\pgfqpoint{4.620000in}{4.620000in}}%
\pgfusepath{clip}%
\pgfsetrectcap%
\pgfsetroundjoin%
\pgfsetlinewidth{1.204500pt}%
\definecolor{currentstroke}{rgb}{1.000000,0.576471,0.309804}%
\pgfsetstrokecolor{currentstroke}%
\pgfsetdash{}{0pt}%
\pgfpathmoveto{\pgfqpoint{2.839679in}{2.267094in}}%
\pgfusepath{stroke}%
\end{pgfscope}%
\begin{pgfscope}%
\pgfpathrectangle{\pgfqpoint{0.765000in}{0.660000in}}{\pgfqpoint{4.620000in}{4.620000in}}%
\pgfusepath{clip}%
\pgfsetbuttcap%
\pgfsetroundjoin%
\definecolor{currentfill}{rgb}{1.000000,0.576471,0.309804}%
\pgfsetfillcolor{currentfill}%
\pgfsetlinewidth{1.003750pt}%
\definecolor{currentstroke}{rgb}{1.000000,0.576471,0.309804}%
\pgfsetstrokecolor{currentstroke}%
\pgfsetdash{}{0pt}%
\pgfsys@defobject{currentmarker}{\pgfqpoint{-0.033333in}{-0.033333in}}{\pgfqpoint{0.033333in}{0.033333in}}{%
\pgfpathmoveto{\pgfqpoint{0.000000in}{-0.033333in}}%
\pgfpathcurveto{\pgfqpoint{0.008840in}{-0.033333in}}{\pgfqpoint{0.017319in}{-0.029821in}}{\pgfqpoint{0.023570in}{-0.023570in}}%
\pgfpathcurveto{\pgfqpoint{0.029821in}{-0.017319in}}{\pgfqpoint{0.033333in}{-0.008840in}}{\pgfqpoint{0.033333in}{0.000000in}}%
\pgfpathcurveto{\pgfqpoint{0.033333in}{0.008840in}}{\pgfqpoint{0.029821in}{0.017319in}}{\pgfqpoint{0.023570in}{0.023570in}}%
\pgfpathcurveto{\pgfqpoint{0.017319in}{0.029821in}}{\pgfqpoint{0.008840in}{0.033333in}}{\pgfqpoint{0.000000in}{0.033333in}}%
\pgfpathcurveto{\pgfqpoint{-0.008840in}{0.033333in}}{\pgfqpoint{-0.017319in}{0.029821in}}{\pgfqpoint{-0.023570in}{0.023570in}}%
\pgfpathcurveto{\pgfqpoint{-0.029821in}{0.017319in}}{\pgfqpoint{-0.033333in}{0.008840in}}{\pgfqpoint{-0.033333in}{0.000000in}}%
\pgfpathcurveto{\pgfqpoint{-0.033333in}{-0.008840in}}{\pgfqpoint{-0.029821in}{-0.017319in}}{\pgfqpoint{-0.023570in}{-0.023570in}}%
\pgfpathcurveto{\pgfqpoint{-0.017319in}{-0.029821in}}{\pgfqpoint{-0.008840in}{-0.033333in}}{\pgfqpoint{0.000000in}{-0.033333in}}%
\pgfpathlineto{\pgfqpoint{0.000000in}{-0.033333in}}%
\pgfpathclose%
\pgfusepath{stroke,fill}%
}%
\begin{pgfscope}%
\pgfsys@transformshift{2.839679in}{2.267094in}%
\pgfsys@useobject{currentmarker}{}%
\end{pgfscope}%
\end{pgfscope}%
\begin{pgfscope}%
\pgfpathrectangle{\pgfqpoint{0.765000in}{0.660000in}}{\pgfqpoint{4.620000in}{4.620000in}}%
\pgfusepath{clip}%
\pgfsetrectcap%
\pgfsetroundjoin%
\pgfsetlinewidth{1.204500pt}%
\definecolor{currentstroke}{rgb}{1.000000,0.576471,0.309804}%
\pgfsetstrokecolor{currentstroke}%
\pgfsetdash{}{0pt}%
\pgfpathmoveto{\pgfqpoint{3.771474in}{2.256257in}}%
\pgfusepath{stroke}%
\end{pgfscope}%
\begin{pgfscope}%
\pgfpathrectangle{\pgfqpoint{0.765000in}{0.660000in}}{\pgfqpoint{4.620000in}{4.620000in}}%
\pgfusepath{clip}%
\pgfsetbuttcap%
\pgfsetroundjoin%
\definecolor{currentfill}{rgb}{1.000000,0.576471,0.309804}%
\pgfsetfillcolor{currentfill}%
\pgfsetlinewidth{1.003750pt}%
\definecolor{currentstroke}{rgb}{1.000000,0.576471,0.309804}%
\pgfsetstrokecolor{currentstroke}%
\pgfsetdash{}{0pt}%
\pgfsys@defobject{currentmarker}{\pgfqpoint{-0.033333in}{-0.033333in}}{\pgfqpoint{0.033333in}{0.033333in}}{%
\pgfpathmoveto{\pgfqpoint{0.000000in}{-0.033333in}}%
\pgfpathcurveto{\pgfqpoint{0.008840in}{-0.033333in}}{\pgfqpoint{0.017319in}{-0.029821in}}{\pgfqpoint{0.023570in}{-0.023570in}}%
\pgfpathcurveto{\pgfqpoint{0.029821in}{-0.017319in}}{\pgfqpoint{0.033333in}{-0.008840in}}{\pgfqpoint{0.033333in}{0.000000in}}%
\pgfpathcurveto{\pgfqpoint{0.033333in}{0.008840in}}{\pgfqpoint{0.029821in}{0.017319in}}{\pgfqpoint{0.023570in}{0.023570in}}%
\pgfpathcurveto{\pgfqpoint{0.017319in}{0.029821in}}{\pgfqpoint{0.008840in}{0.033333in}}{\pgfqpoint{0.000000in}{0.033333in}}%
\pgfpathcurveto{\pgfqpoint{-0.008840in}{0.033333in}}{\pgfqpoint{-0.017319in}{0.029821in}}{\pgfqpoint{-0.023570in}{0.023570in}}%
\pgfpathcurveto{\pgfqpoint{-0.029821in}{0.017319in}}{\pgfqpoint{-0.033333in}{0.008840in}}{\pgfqpoint{-0.033333in}{0.000000in}}%
\pgfpathcurveto{\pgfqpoint{-0.033333in}{-0.008840in}}{\pgfqpoint{-0.029821in}{-0.017319in}}{\pgfqpoint{-0.023570in}{-0.023570in}}%
\pgfpathcurveto{\pgfqpoint{-0.017319in}{-0.029821in}}{\pgfqpoint{-0.008840in}{-0.033333in}}{\pgfqpoint{0.000000in}{-0.033333in}}%
\pgfpathlineto{\pgfqpoint{0.000000in}{-0.033333in}}%
\pgfpathclose%
\pgfusepath{stroke,fill}%
}%
\begin{pgfscope}%
\pgfsys@transformshift{3.771474in}{2.256257in}%
\pgfsys@useobject{currentmarker}{}%
\end{pgfscope}%
\end{pgfscope}%
\begin{pgfscope}%
\pgfpathrectangle{\pgfqpoint{0.765000in}{0.660000in}}{\pgfqpoint{4.620000in}{4.620000in}}%
\pgfusepath{clip}%
\pgfsetrectcap%
\pgfsetroundjoin%
\pgfsetlinewidth{1.204500pt}%
\definecolor{currentstroke}{rgb}{1.000000,0.576471,0.309804}%
\pgfsetstrokecolor{currentstroke}%
\pgfsetdash{}{0pt}%
\pgfpathmoveto{\pgfqpoint{4.190419in}{3.254946in}}%
\pgfusepath{stroke}%
\end{pgfscope}%
\begin{pgfscope}%
\pgfpathrectangle{\pgfqpoint{0.765000in}{0.660000in}}{\pgfqpoint{4.620000in}{4.620000in}}%
\pgfusepath{clip}%
\pgfsetbuttcap%
\pgfsetroundjoin%
\definecolor{currentfill}{rgb}{1.000000,0.576471,0.309804}%
\pgfsetfillcolor{currentfill}%
\pgfsetlinewidth{1.003750pt}%
\definecolor{currentstroke}{rgb}{1.000000,0.576471,0.309804}%
\pgfsetstrokecolor{currentstroke}%
\pgfsetdash{}{0pt}%
\pgfsys@defobject{currentmarker}{\pgfqpoint{-0.033333in}{-0.033333in}}{\pgfqpoint{0.033333in}{0.033333in}}{%
\pgfpathmoveto{\pgfqpoint{0.000000in}{-0.033333in}}%
\pgfpathcurveto{\pgfqpoint{0.008840in}{-0.033333in}}{\pgfqpoint{0.017319in}{-0.029821in}}{\pgfqpoint{0.023570in}{-0.023570in}}%
\pgfpathcurveto{\pgfqpoint{0.029821in}{-0.017319in}}{\pgfqpoint{0.033333in}{-0.008840in}}{\pgfqpoint{0.033333in}{0.000000in}}%
\pgfpathcurveto{\pgfqpoint{0.033333in}{0.008840in}}{\pgfqpoint{0.029821in}{0.017319in}}{\pgfqpoint{0.023570in}{0.023570in}}%
\pgfpathcurveto{\pgfqpoint{0.017319in}{0.029821in}}{\pgfqpoint{0.008840in}{0.033333in}}{\pgfqpoint{0.000000in}{0.033333in}}%
\pgfpathcurveto{\pgfqpoint{-0.008840in}{0.033333in}}{\pgfqpoint{-0.017319in}{0.029821in}}{\pgfqpoint{-0.023570in}{0.023570in}}%
\pgfpathcurveto{\pgfqpoint{-0.029821in}{0.017319in}}{\pgfqpoint{-0.033333in}{0.008840in}}{\pgfqpoint{-0.033333in}{0.000000in}}%
\pgfpathcurveto{\pgfqpoint{-0.033333in}{-0.008840in}}{\pgfqpoint{-0.029821in}{-0.017319in}}{\pgfqpoint{-0.023570in}{-0.023570in}}%
\pgfpathcurveto{\pgfqpoint{-0.017319in}{-0.029821in}}{\pgfqpoint{-0.008840in}{-0.033333in}}{\pgfqpoint{0.000000in}{-0.033333in}}%
\pgfpathlineto{\pgfqpoint{0.000000in}{-0.033333in}}%
\pgfpathclose%
\pgfusepath{stroke,fill}%
}%
\begin{pgfscope}%
\pgfsys@transformshift{4.190419in}{3.254946in}%
\pgfsys@useobject{currentmarker}{}%
\end{pgfscope}%
\end{pgfscope}%
\begin{pgfscope}%
\pgfpathrectangle{\pgfqpoint{0.765000in}{0.660000in}}{\pgfqpoint{4.620000in}{4.620000in}}%
\pgfusepath{clip}%
\pgfsetrectcap%
\pgfsetroundjoin%
\pgfsetlinewidth{1.204500pt}%
\definecolor{currentstroke}{rgb}{1.000000,0.576471,0.309804}%
\pgfsetstrokecolor{currentstroke}%
\pgfsetdash{}{0pt}%
\pgfpathmoveto{\pgfqpoint{3.273582in}{1.749824in}}%
\pgfusepath{stroke}%
\end{pgfscope}%
\begin{pgfscope}%
\pgfpathrectangle{\pgfqpoint{0.765000in}{0.660000in}}{\pgfqpoint{4.620000in}{4.620000in}}%
\pgfusepath{clip}%
\pgfsetbuttcap%
\pgfsetroundjoin%
\definecolor{currentfill}{rgb}{1.000000,0.576471,0.309804}%
\pgfsetfillcolor{currentfill}%
\pgfsetlinewidth{1.003750pt}%
\definecolor{currentstroke}{rgb}{1.000000,0.576471,0.309804}%
\pgfsetstrokecolor{currentstroke}%
\pgfsetdash{}{0pt}%
\pgfsys@defobject{currentmarker}{\pgfqpoint{-0.033333in}{-0.033333in}}{\pgfqpoint{0.033333in}{0.033333in}}{%
\pgfpathmoveto{\pgfqpoint{0.000000in}{-0.033333in}}%
\pgfpathcurveto{\pgfqpoint{0.008840in}{-0.033333in}}{\pgfqpoint{0.017319in}{-0.029821in}}{\pgfqpoint{0.023570in}{-0.023570in}}%
\pgfpathcurveto{\pgfqpoint{0.029821in}{-0.017319in}}{\pgfqpoint{0.033333in}{-0.008840in}}{\pgfqpoint{0.033333in}{0.000000in}}%
\pgfpathcurveto{\pgfqpoint{0.033333in}{0.008840in}}{\pgfqpoint{0.029821in}{0.017319in}}{\pgfqpoint{0.023570in}{0.023570in}}%
\pgfpathcurveto{\pgfqpoint{0.017319in}{0.029821in}}{\pgfqpoint{0.008840in}{0.033333in}}{\pgfqpoint{0.000000in}{0.033333in}}%
\pgfpathcurveto{\pgfqpoint{-0.008840in}{0.033333in}}{\pgfqpoint{-0.017319in}{0.029821in}}{\pgfqpoint{-0.023570in}{0.023570in}}%
\pgfpathcurveto{\pgfqpoint{-0.029821in}{0.017319in}}{\pgfqpoint{-0.033333in}{0.008840in}}{\pgfqpoint{-0.033333in}{0.000000in}}%
\pgfpathcurveto{\pgfqpoint{-0.033333in}{-0.008840in}}{\pgfqpoint{-0.029821in}{-0.017319in}}{\pgfqpoint{-0.023570in}{-0.023570in}}%
\pgfpathcurveto{\pgfqpoint{-0.017319in}{-0.029821in}}{\pgfqpoint{-0.008840in}{-0.033333in}}{\pgfqpoint{0.000000in}{-0.033333in}}%
\pgfpathlineto{\pgfqpoint{0.000000in}{-0.033333in}}%
\pgfpathclose%
\pgfusepath{stroke,fill}%
}%
\begin{pgfscope}%
\pgfsys@transformshift{3.273582in}{1.749824in}%
\pgfsys@useobject{currentmarker}{}%
\end{pgfscope}%
\end{pgfscope}%
\begin{pgfscope}%
\pgfpathrectangle{\pgfqpoint{0.765000in}{0.660000in}}{\pgfqpoint{4.620000in}{4.620000in}}%
\pgfusepath{clip}%
\pgfsetrectcap%
\pgfsetroundjoin%
\pgfsetlinewidth{1.204500pt}%
\definecolor{currentstroke}{rgb}{1.000000,0.576471,0.309804}%
\pgfsetstrokecolor{currentstroke}%
\pgfsetdash{}{0pt}%
\pgfpathmoveto{\pgfqpoint{3.067429in}{1.079864in}}%
\pgfusepath{stroke}%
\end{pgfscope}%
\begin{pgfscope}%
\pgfpathrectangle{\pgfqpoint{0.765000in}{0.660000in}}{\pgfqpoint{4.620000in}{4.620000in}}%
\pgfusepath{clip}%
\pgfsetbuttcap%
\pgfsetroundjoin%
\definecolor{currentfill}{rgb}{1.000000,0.576471,0.309804}%
\pgfsetfillcolor{currentfill}%
\pgfsetlinewidth{1.003750pt}%
\definecolor{currentstroke}{rgb}{1.000000,0.576471,0.309804}%
\pgfsetstrokecolor{currentstroke}%
\pgfsetdash{}{0pt}%
\pgfsys@defobject{currentmarker}{\pgfqpoint{-0.033333in}{-0.033333in}}{\pgfqpoint{0.033333in}{0.033333in}}{%
\pgfpathmoveto{\pgfqpoint{0.000000in}{-0.033333in}}%
\pgfpathcurveto{\pgfqpoint{0.008840in}{-0.033333in}}{\pgfqpoint{0.017319in}{-0.029821in}}{\pgfqpoint{0.023570in}{-0.023570in}}%
\pgfpathcurveto{\pgfqpoint{0.029821in}{-0.017319in}}{\pgfqpoint{0.033333in}{-0.008840in}}{\pgfqpoint{0.033333in}{0.000000in}}%
\pgfpathcurveto{\pgfqpoint{0.033333in}{0.008840in}}{\pgfqpoint{0.029821in}{0.017319in}}{\pgfqpoint{0.023570in}{0.023570in}}%
\pgfpathcurveto{\pgfqpoint{0.017319in}{0.029821in}}{\pgfqpoint{0.008840in}{0.033333in}}{\pgfqpoint{0.000000in}{0.033333in}}%
\pgfpathcurveto{\pgfqpoint{-0.008840in}{0.033333in}}{\pgfqpoint{-0.017319in}{0.029821in}}{\pgfqpoint{-0.023570in}{0.023570in}}%
\pgfpathcurveto{\pgfqpoint{-0.029821in}{0.017319in}}{\pgfqpoint{-0.033333in}{0.008840in}}{\pgfqpoint{-0.033333in}{0.000000in}}%
\pgfpathcurveto{\pgfqpoint{-0.033333in}{-0.008840in}}{\pgfqpoint{-0.029821in}{-0.017319in}}{\pgfqpoint{-0.023570in}{-0.023570in}}%
\pgfpathcurveto{\pgfqpoint{-0.017319in}{-0.029821in}}{\pgfqpoint{-0.008840in}{-0.033333in}}{\pgfqpoint{0.000000in}{-0.033333in}}%
\pgfpathlineto{\pgfqpoint{0.000000in}{-0.033333in}}%
\pgfpathclose%
\pgfusepath{stroke,fill}%
}%
\begin{pgfscope}%
\pgfsys@transformshift{3.067429in}{1.079864in}%
\pgfsys@useobject{currentmarker}{}%
\end{pgfscope}%
\end{pgfscope}%
\begin{pgfscope}%
\pgfpathrectangle{\pgfqpoint{0.765000in}{0.660000in}}{\pgfqpoint{4.620000in}{4.620000in}}%
\pgfusepath{clip}%
\pgfsetrectcap%
\pgfsetroundjoin%
\pgfsetlinewidth{1.204500pt}%
\definecolor{currentstroke}{rgb}{1.000000,0.576471,0.309804}%
\pgfsetstrokecolor{currentstroke}%
\pgfsetdash{}{0pt}%
\pgfpathmoveto{\pgfqpoint{2.575005in}{2.941145in}}%
\pgfusepath{stroke}%
\end{pgfscope}%
\begin{pgfscope}%
\pgfpathrectangle{\pgfqpoint{0.765000in}{0.660000in}}{\pgfqpoint{4.620000in}{4.620000in}}%
\pgfusepath{clip}%
\pgfsetbuttcap%
\pgfsetroundjoin%
\definecolor{currentfill}{rgb}{1.000000,0.576471,0.309804}%
\pgfsetfillcolor{currentfill}%
\pgfsetlinewidth{1.003750pt}%
\definecolor{currentstroke}{rgb}{1.000000,0.576471,0.309804}%
\pgfsetstrokecolor{currentstroke}%
\pgfsetdash{}{0pt}%
\pgfsys@defobject{currentmarker}{\pgfqpoint{-0.033333in}{-0.033333in}}{\pgfqpoint{0.033333in}{0.033333in}}{%
\pgfpathmoveto{\pgfqpoint{0.000000in}{-0.033333in}}%
\pgfpathcurveto{\pgfqpoint{0.008840in}{-0.033333in}}{\pgfqpoint{0.017319in}{-0.029821in}}{\pgfqpoint{0.023570in}{-0.023570in}}%
\pgfpathcurveto{\pgfqpoint{0.029821in}{-0.017319in}}{\pgfqpoint{0.033333in}{-0.008840in}}{\pgfqpoint{0.033333in}{0.000000in}}%
\pgfpathcurveto{\pgfqpoint{0.033333in}{0.008840in}}{\pgfqpoint{0.029821in}{0.017319in}}{\pgfqpoint{0.023570in}{0.023570in}}%
\pgfpathcurveto{\pgfqpoint{0.017319in}{0.029821in}}{\pgfqpoint{0.008840in}{0.033333in}}{\pgfqpoint{0.000000in}{0.033333in}}%
\pgfpathcurveto{\pgfqpoint{-0.008840in}{0.033333in}}{\pgfqpoint{-0.017319in}{0.029821in}}{\pgfqpoint{-0.023570in}{0.023570in}}%
\pgfpathcurveto{\pgfqpoint{-0.029821in}{0.017319in}}{\pgfqpoint{-0.033333in}{0.008840in}}{\pgfqpoint{-0.033333in}{0.000000in}}%
\pgfpathcurveto{\pgfqpoint{-0.033333in}{-0.008840in}}{\pgfqpoint{-0.029821in}{-0.017319in}}{\pgfqpoint{-0.023570in}{-0.023570in}}%
\pgfpathcurveto{\pgfqpoint{-0.017319in}{-0.029821in}}{\pgfqpoint{-0.008840in}{-0.033333in}}{\pgfqpoint{0.000000in}{-0.033333in}}%
\pgfpathlineto{\pgfqpoint{0.000000in}{-0.033333in}}%
\pgfpathclose%
\pgfusepath{stroke,fill}%
}%
\begin{pgfscope}%
\pgfsys@transformshift{2.575005in}{2.941145in}%
\pgfsys@useobject{currentmarker}{}%
\end{pgfscope}%
\end{pgfscope}%
\begin{pgfscope}%
\pgfpathrectangle{\pgfqpoint{0.765000in}{0.660000in}}{\pgfqpoint{4.620000in}{4.620000in}}%
\pgfusepath{clip}%
\pgfsetrectcap%
\pgfsetroundjoin%
\pgfsetlinewidth{1.204500pt}%
\definecolor{currentstroke}{rgb}{1.000000,0.576471,0.309804}%
\pgfsetstrokecolor{currentstroke}%
\pgfsetdash{}{0pt}%
\pgfpathmoveto{\pgfqpoint{3.096645in}{1.160540in}}%
\pgfusepath{stroke}%
\end{pgfscope}%
\begin{pgfscope}%
\pgfpathrectangle{\pgfqpoint{0.765000in}{0.660000in}}{\pgfqpoint{4.620000in}{4.620000in}}%
\pgfusepath{clip}%
\pgfsetbuttcap%
\pgfsetroundjoin%
\definecolor{currentfill}{rgb}{1.000000,0.576471,0.309804}%
\pgfsetfillcolor{currentfill}%
\pgfsetlinewidth{1.003750pt}%
\definecolor{currentstroke}{rgb}{1.000000,0.576471,0.309804}%
\pgfsetstrokecolor{currentstroke}%
\pgfsetdash{}{0pt}%
\pgfsys@defobject{currentmarker}{\pgfqpoint{-0.033333in}{-0.033333in}}{\pgfqpoint{0.033333in}{0.033333in}}{%
\pgfpathmoveto{\pgfqpoint{0.000000in}{-0.033333in}}%
\pgfpathcurveto{\pgfqpoint{0.008840in}{-0.033333in}}{\pgfqpoint{0.017319in}{-0.029821in}}{\pgfqpoint{0.023570in}{-0.023570in}}%
\pgfpathcurveto{\pgfqpoint{0.029821in}{-0.017319in}}{\pgfqpoint{0.033333in}{-0.008840in}}{\pgfqpoint{0.033333in}{0.000000in}}%
\pgfpathcurveto{\pgfqpoint{0.033333in}{0.008840in}}{\pgfqpoint{0.029821in}{0.017319in}}{\pgfqpoint{0.023570in}{0.023570in}}%
\pgfpathcurveto{\pgfqpoint{0.017319in}{0.029821in}}{\pgfqpoint{0.008840in}{0.033333in}}{\pgfqpoint{0.000000in}{0.033333in}}%
\pgfpathcurveto{\pgfqpoint{-0.008840in}{0.033333in}}{\pgfqpoint{-0.017319in}{0.029821in}}{\pgfqpoint{-0.023570in}{0.023570in}}%
\pgfpathcurveto{\pgfqpoint{-0.029821in}{0.017319in}}{\pgfqpoint{-0.033333in}{0.008840in}}{\pgfqpoint{-0.033333in}{0.000000in}}%
\pgfpathcurveto{\pgfqpoint{-0.033333in}{-0.008840in}}{\pgfqpoint{-0.029821in}{-0.017319in}}{\pgfqpoint{-0.023570in}{-0.023570in}}%
\pgfpathcurveto{\pgfqpoint{-0.017319in}{-0.029821in}}{\pgfqpoint{-0.008840in}{-0.033333in}}{\pgfqpoint{0.000000in}{-0.033333in}}%
\pgfpathlineto{\pgfqpoint{0.000000in}{-0.033333in}}%
\pgfpathclose%
\pgfusepath{stroke,fill}%
}%
\begin{pgfscope}%
\pgfsys@transformshift{3.096645in}{1.160540in}%
\pgfsys@useobject{currentmarker}{}%
\end{pgfscope}%
\end{pgfscope}%
\begin{pgfscope}%
\pgfpathrectangle{\pgfqpoint{0.765000in}{0.660000in}}{\pgfqpoint{4.620000in}{4.620000in}}%
\pgfusepath{clip}%
\pgfsetrectcap%
\pgfsetroundjoin%
\pgfsetlinewidth{1.204500pt}%
\definecolor{currentstroke}{rgb}{1.000000,0.576471,0.309804}%
\pgfsetstrokecolor{currentstroke}%
\pgfsetdash{}{0pt}%
\pgfpathmoveto{\pgfqpoint{2.299839in}{2.349333in}}%
\pgfusepath{stroke}%
\end{pgfscope}%
\begin{pgfscope}%
\pgfpathrectangle{\pgfqpoint{0.765000in}{0.660000in}}{\pgfqpoint{4.620000in}{4.620000in}}%
\pgfusepath{clip}%
\pgfsetbuttcap%
\pgfsetroundjoin%
\definecolor{currentfill}{rgb}{1.000000,0.576471,0.309804}%
\pgfsetfillcolor{currentfill}%
\pgfsetlinewidth{1.003750pt}%
\definecolor{currentstroke}{rgb}{1.000000,0.576471,0.309804}%
\pgfsetstrokecolor{currentstroke}%
\pgfsetdash{}{0pt}%
\pgfsys@defobject{currentmarker}{\pgfqpoint{-0.033333in}{-0.033333in}}{\pgfqpoint{0.033333in}{0.033333in}}{%
\pgfpathmoveto{\pgfqpoint{0.000000in}{-0.033333in}}%
\pgfpathcurveto{\pgfqpoint{0.008840in}{-0.033333in}}{\pgfqpoint{0.017319in}{-0.029821in}}{\pgfqpoint{0.023570in}{-0.023570in}}%
\pgfpathcurveto{\pgfqpoint{0.029821in}{-0.017319in}}{\pgfqpoint{0.033333in}{-0.008840in}}{\pgfqpoint{0.033333in}{0.000000in}}%
\pgfpathcurveto{\pgfqpoint{0.033333in}{0.008840in}}{\pgfqpoint{0.029821in}{0.017319in}}{\pgfqpoint{0.023570in}{0.023570in}}%
\pgfpathcurveto{\pgfqpoint{0.017319in}{0.029821in}}{\pgfqpoint{0.008840in}{0.033333in}}{\pgfqpoint{0.000000in}{0.033333in}}%
\pgfpathcurveto{\pgfqpoint{-0.008840in}{0.033333in}}{\pgfqpoint{-0.017319in}{0.029821in}}{\pgfqpoint{-0.023570in}{0.023570in}}%
\pgfpathcurveto{\pgfqpoint{-0.029821in}{0.017319in}}{\pgfqpoint{-0.033333in}{0.008840in}}{\pgfqpoint{-0.033333in}{0.000000in}}%
\pgfpathcurveto{\pgfqpoint{-0.033333in}{-0.008840in}}{\pgfqpoint{-0.029821in}{-0.017319in}}{\pgfqpoint{-0.023570in}{-0.023570in}}%
\pgfpathcurveto{\pgfqpoint{-0.017319in}{-0.029821in}}{\pgfqpoint{-0.008840in}{-0.033333in}}{\pgfqpoint{0.000000in}{-0.033333in}}%
\pgfpathlineto{\pgfqpoint{0.000000in}{-0.033333in}}%
\pgfpathclose%
\pgfusepath{stroke,fill}%
}%
\begin{pgfscope}%
\pgfsys@transformshift{2.299839in}{2.349333in}%
\pgfsys@useobject{currentmarker}{}%
\end{pgfscope}%
\end{pgfscope}%
\begin{pgfscope}%
\pgfpathrectangle{\pgfqpoint{0.765000in}{0.660000in}}{\pgfqpoint{4.620000in}{4.620000in}}%
\pgfusepath{clip}%
\pgfsetrectcap%
\pgfsetroundjoin%
\pgfsetlinewidth{1.204500pt}%
\definecolor{currentstroke}{rgb}{1.000000,0.576471,0.309804}%
\pgfsetstrokecolor{currentstroke}%
\pgfsetdash{}{0pt}%
\pgfpathmoveto{\pgfqpoint{3.647833in}{2.665419in}}%
\pgfusepath{stroke}%
\end{pgfscope}%
\begin{pgfscope}%
\pgfpathrectangle{\pgfqpoint{0.765000in}{0.660000in}}{\pgfqpoint{4.620000in}{4.620000in}}%
\pgfusepath{clip}%
\pgfsetbuttcap%
\pgfsetroundjoin%
\definecolor{currentfill}{rgb}{1.000000,0.576471,0.309804}%
\pgfsetfillcolor{currentfill}%
\pgfsetlinewidth{1.003750pt}%
\definecolor{currentstroke}{rgb}{1.000000,0.576471,0.309804}%
\pgfsetstrokecolor{currentstroke}%
\pgfsetdash{}{0pt}%
\pgfsys@defobject{currentmarker}{\pgfqpoint{-0.033333in}{-0.033333in}}{\pgfqpoint{0.033333in}{0.033333in}}{%
\pgfpathmoveto{\pgfqpoint{0.000000in}{-0.033333in}}%
\pgfpathcurveto{\pgfqpoint{0.008840in}{-0.033333in}}{\pgfqpoint{0.017319in}{-0.029821in}}{\pgfqpoint{0.023570in}{-0.023570in}}%
\pgfpathcurveto{\pgfqpoint{0.029821in}{-0.017319in}}{\pgfqpoint{0.033333in}{-0.008840in}}{\pgfqpoint{0.033333in}{0.000000in}}%
\pgfpathcurveto{\pgfqpoint{0.033333in}{0.008840in}}{\pgfqpoint{0.029821in}{0.017319in}}{\pgfqpoint{0.023570in}{0.023570in}}%
\pgfpathcurveto{\pgfqpoint{0.017319in}{0.029821in}}{\pgfqpoint{0.008840in}{0.033333in}}{\pgfqpoint{0.000000in}{0.033333in}}%
\pgfpathcurveto{\pgfqpoint{-0.008840in}{0.033333in}}{\pgfqpoint{-0.017319in}{0.029821in}}{\pgfqpoint{-0.023570in}{0.023570in}}%
\pgfpathcurveto{\pgfqpoint{-0.029821in}{0.017319in}}{\pgfqpoint{-0.033333in}{0.008840in}}{\pgfqpoint{-0.033333in}{0.000000in}}%
\pgfpathcurveto{\pgfqpoint{-0.033333in}{-0.008840in}}{\pgfqpoint{-0.029821in}{-0.017319in}}{\pgfqpoint{-0.023570in}{-0.023570in}}%
\pgfpathcurveto{\pgfqpoint{-0.017319in}{-0.029821in}}{\pgfqpoint{-0.008840in}{-0.033333in}}{\pgfqpoint{0.000000in}{-0.033333in}}%
\pgfpathlineto{\pgfqpoint{0.000000in}{-0.033333in}}%
\pgfpathclose%
\pgfusepath{stroke,fill}%
}%
\begin{pgfscope}%
\pgfsys@transformshift{3.647833in}{2.665419in}%
\pgfsys@useobject{currentmarker}{}%
\end{pgfscope}%
\end{pgfscope}%
\begin{pgfscope}%
\pgfpathrectangle{\pgfqpoint{0.765000in}{0.660000in}}{\pgfqpoint{4.620000in}{4.620000in}}%
\pgfusepath{clip}%
\pgfsetrectcap%
\pgfsetroundjoin%
\pgfsetlinewidth{1.204500pt}%
\definecolor{currentstroke}{rgb}{1.000000,0.576471,0.309804}%
\pgfsetstrokecolor{currentstroke}%
\pgfsetdash{}{0pt}%
\pgfpathmoveto{\pgfqpoint{1.799539in}{2.217022in}}%
\pgfusepath{stroke}%
\end{pgfscope}%
\begin{pgfscope}%
\pgfpathrectangle{\pgfqpoint{0.765000in}{0.660000in}}{\pgfqpoint{4.620000in}{4.620000in}}%
\pgfusepath{clip}%
\pgfsetbuttcap%
\pgfsetroundjoin%
\definecolor{currentfill}{rgb}{1.000000,0.576471,0.309804}%
\pgfsetfillcolor{currentfill}%
\pgfsetlinewidth{1.003750pt}%
\definecolor{currentstroke}{rgb}{1.000000,0.576471,0.309804}%
\pgfsetstrokecolor{currentstroke}%
\pgfsetdash{}{0pt}%
\pgfsys@defobject{currentmarker}{\pgfqpoint{-0.033333in}{-0.033333in}}{\pgfqpoint{0.033333in}{0.033333in}}{%
\pgfpathmoveto{\pgfqpoint{0.000000in}{-0.033333in}}%
\pgfpathcurveto{\pgfqpoint{0.008840in}{-0.033333in}}{\pgfqpoint{0.017319in}{-0.029821in}}{\pgfqpoint{0.023570in}{-0.023570in}}%
\pgfpathcurveto{\pgfqpoint{0.029821in}{-0.017319in}}{\pgfqpoint{0.033333in}{-0.008840in}}{\pgfqpoint{0.033333in}{0.000000in}}%
\pgfpathcurveto{\pgfqpoint{0.033333in}{0.008840in}}{\pgfqpoint{0.029821in}{0.017319in}}{\pgfqpoint{0.023570in}{0.023570in}}%
\pgfpathcurveto{\pgfqpoint{0.017319in}{0.029821in}}{\pgfqpoint{0.008840in}{0.033333in}}{\pgfqpoint{0.000000in}{0.033333in}}%
\pgfpathcurveto{\pgfqpoint{-0.008840in}{0.033333in}}{\pgfqpoint{-0.017319in}{0.029821in}}{\pgfqpoint{-0.023570in}{0.023570in}}%
\pgfpathcurveto{\pgfqpoint{-0.029821in}{0.017319in}}{\pgfqpoint{-0.033333in}{0.008840in}}{\pgfqpoint{-0.033333in}{0.000000in}}%
\pgfpathcurveto{\pgfqpoint{-0.033333in}{-0.008840in}}{\pgfqpoint{-0.029821in}{-0.017319in}}{\pgfqpoint{-0.023570in}{-0.023570in}}%
\pgfpathcurveto{\pgfqpoint{-0.017319in}{-0.029821in}}{\pgfqpoint{-0.008840in}{-0.033333in}}{\pgfqpoint{0.000000in}{-0.033333in}}%
\pgfpathlineto{\pgfqpoint{0.000000in}{-0.033333in}}%
\pgfpathclose%
\pgfusepath{stroke,fill}%
}%
\begin{pgfscope}%
\pgfsys@transformshift{1.799539in}{2.217022in}%
\pgfsys@useobject{currentmarker}{}%
\end{pgfscope}%
\end{pgfscope}%
\begin{pgfscope}%
\pgfpathrectangle{\pgfqpoint{0.765000in}{0.660000in}}{\pgfqpoint{4.620000in}{4.620000in}}%
\pgfusepath{clip}%
\pgfsetrectcap%
\pgfsetroundjoin%
\pgfsetlinewidth{1.204500pt}%
\definecolor{currentstroke}{rgb}{1.000000,0.576471,0.309804}%
\pgfsetstrokecolor{currentstroke}%
\pgfsetdash{}{0pt}%
\pgfpathmoveto{\pgfqpoint{3.018882in}{2.440982in}}%
\pgfusepath{stroke}%
\end{pgfscope}%
\begin{pgfscope}%
\pgfpathrectangle{\pgfqpoint{0.765000in}{0.660000in}}{\pgfqpoint{4.620000in}{4.620000in}}%
\pgfusepath{clip}%
\pgfsetbuttcap%
\pgfsetroundjoin%
\definecolor{currentfill}{rgb}{1.000000,0.576471,0.309804}%
\pgfsetfillcolor{currentfill}%
\pgfsetlinewidth{1.003750pt}%
\definecolor{currentstroke}{rgb}{1.000000,0.576471,0.309804}%
\pgfsetstrokecolor{currentstroke}%
\pgfsetdash{}{0pt}%
\pgfsys@defobject{currentmarker}{\pgfqpoint{-0.033333in}{-0.033333in}}{\pgfqpoint{0.033333in}{0.033333in}}{%
\pgfpathmoveto{\pgfqpoint{0.000000in}{-0.033333in}}%
\pgfpathcurveto{\pgfqpoint{0.008840in}{-0.033333in}}{\pgfqpoint{0.017319in}{-0.029821in}}{\pgfqpoint{0.023570in}{-0.023570in}}%
\pgfpathcurveto{\pgfqpoint{0.029821in}{-0.017319in}}{\pgfqpoint{0.033333in}{-0.008840in}}{\pgfqpoint{0.033333in}{0.000000in}}%
\pgfpathcurveto{\pgfqpoint{0.033333in}{0.008840in}}{\pgfqpoint{0.029821in}{0.017319in}}{\pgfqpoint{0.023570in}{0.023570in}}%
\pgfpathcurveto{\pgfqpoint{0.017319in}{0.029821in}}{\pgfqpoint{0.008840in}{0.033333in}}{\pgfqpoint{0.000000in}{0.033333in}}%
\pgfpathcurveto{\pgfqpoint{-0.008840in}{0.033333in}}{\pgfqpoint{-0.017319in}{0.029821in}}{\pgfqpoint{-0.023570in}{0.023570in}}%
\pgfpathcurveto{\pgfqpoint{-0.029821in}{0.017319in}}{\pgfqpoint{-0.033333in}{0.008840in}}{\pgfqpoint{-0.033333in}{0.000000in}}%
\pgfpathcurveto{\pgfqpoint{-0.033333in}{-0.008840in}}{\pgfqpoint{-0.029821in}{-0.017319in}}{\pgfqpoint{-0.023570in}{-0.023570in}}%
\pgfpathcurveto{\pgfqpoint{-0.017319in}{-0.029821in}}{\pgfqpoint{-0.008840in}{-0.033333in}}{\pgfqpoint{0.000000in}{-0.033333in}}%
\pgfpathlineto{\pgfqpoint{0.000000in}{-0.033333in}}%
\pgfpathclose%
\pgfusepath{stroke,fill}%
}%
\begin{pgfscope}%
\pgfsys@transformshift{3.018882in}{2.440982in}%
\pgfsys@useobject{currentmarker}{}%
\end{pgfscope}%
\end{pgfscope}%
\begin{pgfscope}%
\pgfpathrectangle{\pgfqpoint{0.765000in}{0.660000in}}{\pgfqpoint{4.620000in}{4.620000in}}%
\pgfusepath{clip}%
\pgfsetrectcap%
\pgfsetroundjoin%
\pgfsetlinewidth{1.204500pt}%
\definecolor{currentstroke}{rgb}{1.000000,0.576471,0.309804}%
\pgfsetstrokecolor{currentstroke}%
\pgfsetdash{}{0pt}%
\pgfpathmoveto{\pgfqpoint{1.965499in}{3.245565in}}%
\pgfusepath{stroke}%
\end{pgfscope}%
\begin{pgfscope}%
\pgfpathrectangle{\pgfqpoint{0.765000in}{0.660000in}}{\pgfqpoint{4.620000in}{4.620000in}}%
\pgfusepath{clip}%
\pgfsetbuttcap%
\pgfsetroundjoin%
\definecolor{currentfill}{rgb}{1.000000,0.576471,0.309804}%
\pgfsetfillcolor{currentfill}%
\pgfsetlinewidth{1.003750pt}%
\definecolor{currentstroke}{rgb}{1.000000,0.576471,0.309804}%
\pgfsetstrokecolor{currentstroke}%
\pgfsetdash{}{0pt}%
\pgfsys@defobject{currentmarker}{\pgfqpoint{-0.033333in}{-0.033333in}}{\pgfqpoint{0.033333in}{0.033333in}}{%
\pgfpathmoveto{\pgfqpoint{0.000000in}{-0.033333in}}%
\pgfpathcurveto{\pgfqpoint{0.008840in}{-0.033333in}}{\pgfqpoint{0.017319in}{-0.029821in}}{\pgfqpoint{0.023570in}{-0.023570in}}%
\pgfpathcurveto{\pgfqpoint{0.029821in}{-0.017319in}}{\pgfqpoint{0.033333in}{-0.008840in}}{\pgfqpoint{0.033333in}{0.000000in}}%
\pgfpathcurveto{\pgfqpoint{0.033333in}{0.008840in}}{\pgfqpoint{0.029821in}{0.017319in}}{\pgfqpoint{0.023570in}{0.023570in}}%
\pgfpathcurveto{\pgfqpoint{0.017319in}{0.029821in}}{\pgfqpoint{0.008840in}{0.033333in}}{\pgfqpoint{0.000000in}{0.033333in}}%
\pgfpathcurveto{\pgfqpoint{-0.008840in}{0.033333in}}{\pgfqpoint{-0.017319in}{0.029821in}}{\pgfqpoint{-0.023570in}{0.023570in}}%
\pgfpathcurveto{\pgfqpoint{-0.029821in}{0.017319in}}{\pgfqpoint{-0.033333in}{0.008840in}}{\pgfqpoint{-0.033333in}{0.000000in}}%
\pgfpathcurveto{\pgfqpoint{-0.033333in}{-0.008840in}}{\pgfqpoint{-0.029821in}{-0.017319in}}{\pgfqpoint{-0.023570in}{-0.023570in}}%
\pgfpathcurveto{\pgfqpoint{-0.017319in}{-0.029821in}}{\pgfqpoint{-0.008840in}{-0.033333in}}{\pgfqpoint{0.000000in}{-0.033333in}}%
\pgfpathlineto{\pgfqpoint{0.000000in}{-0.033333in}}%
\pgfpathclose%
\pgfusepath{stroke,fill}%
}%
\begin{pgfscope}%
\pgfsys@transformshift{1.965499in}{3.245565in}%
\pgfsys@useobject{currentmarker}{}%
\end{pgfscope}%
\end{pgfscope}%
\begin{pgfscope}%
\pgfpathrectangle{\pgfqpoint{0.765000in}{0.660000in}}{\pgfqpoint{4.620000in}{4.620000in}}%
\pgfusepath{clip}%
\pgfsetrectcap%
\pgfsetroundjoin%
\pgfsetlinewidth{1.204500pt}%
\definecolor{currentstroke}{rgb}{1.000000,0.576471,0.309804}%
\pgfsetstrokecolor{currentstroke}%
\pgfsetdash{}{0pt}%
\pgfpathmoveto{\pgfqpoint{1.708178in}{2.334710in}}%
\pgfusepath{stroke}%
\end{pgfscope}%
\begin{pgfscope}%
\pgfpathrectangle{\pgfqpoint{0.765000in}{0.660000in}}{\pgfqpoint{4.620000in}{4.620000in}}%
\pgfusepath{clip}%
\pgfsetbuttcap%
\pgfsetroundjoin%
\definecolor{currentfill}{rgb}{1.000000,0.576471,0.309804}%
\pgfsetfillcolor{currentfill}%
\pgfsetlinewidth{1.003750pt}%
\definecolor{currentstroke}{rgb}{1.000000,0.576471,0.309804}%
\pgfsetstrokecolor{currentstroke}%
\pgfsetdash{}{0pt}%
\pgfsys@defobject{currentmarker}{\pgfqpoint{-0.033333in}{-0.033333in}}{\pgfqpoint{0.033333in}{0.033333in}}{%
\pgfpathmoveto{\pgfqpoint{0.000000in}{-0.033333in}}%
\pgfpathcurveto{\pgfqpoint{0.008840in}{-0.033333in}}{\pgfqpoint{0.017319in}{-0.029821in}}{\pgfqpoint{0.023570in}{-0.023570in}}%
\pgfpathcurveto{\pgfqpoint{0.029821in}{-0.017319in}}{\pgfqpoint{0.033333in}{-0.008840in}}{\pgfqpoint{0.033333in}{0.000000in}}%
\pgfpathcurveto{\pgfqpoint{0.033333in}{0.008840in}}{\pgfqpoint{0.029821in}{0.017319in}}{\pgfqpoint{0.023570in}{0.023570in}}%
\pgfpathcurveto{\pgfqpoint{0.017319in}{0.029821in}}{\pgfqpoint{0.008840in}{0.033333in}}{\pgfqpoint{0.000000in}{0.033333in}}%
\pgfpathcurveto{\pgfqpoint{-0.008840in}{0.033333in}}{\pgfqpoint{-0.017319in}{0.029821in}}{\pgfqpoint{-0.023570in}{0.023570in}}%
\pgfpathcurveto{\pgfqpoint{-0.029821in}{0.017319in}}{\pgfqpoint{-0.033333in}{0.008840in}}{\pgfqpoint{-0.033333in}{0.000000in}}%
\pgfpathcurveto{\pgfqpoint{-0.033333in}{-0.008840in}}{\pgfqpoint{-0.029821in}{-0.017319in}}{\pgfqpoint{-0.023570in}{-0.023570in}}%
\pgfpathcurveto{\pgfqpoint{-0.017319in}{-0.029821in}}{\pgfqpoint{-0.008840in}{-0.033333in}}{\pgfqpoint{0.000000in}{-0.033333in}}%
\pgfpathlineto{\pgfqpoint{0.000000in}{-0.033333in}}%
\pgfpathclose%
\pgfusepath{stroke,fill}%
}%
\begin{pgfscope}%
\pgfsys@transformshift{1.708178in}{2.334710in}%
\pgfsys@useobject{currentmarker}{}%
\end{pgfscope}%
\end{pgfscope}%
\begin{pgfscope}%
\pgfpathrectangle{\pgfqpoint{0.765000in}{0.660000in}}{\pgfqpoint{4.620000in}{4.620000in}}%
\pgfusepath{clip}%
\pgfsetrectcap%
\pgfsetroundjoin%
\pgfsetlinewidth{1.204500pt}%
\definecolor{currentstroke}{rgb}{1.000000,0.576471,0.309804}%
\pgfsetstrokecolor{currentstroke}%
\pgfsetdash{}{0pt}%
\pgfpathmoveto{\pgfqpoint{3.908905in}{1.704892in}}%
\pgfusepath{stroke}%
\end{pgfscope}%
\begin{pgfscope}%
\pgfpathrectangle{\pgfqpoint{0.765000in}{0.660000in}}{\pgfqpoint{4.620000in}{4.620000in}}%
\pgfusepath{clip}%
\pgfsetbuttcap%
\pgfsetroundjoin%
\definecolor{currentfill}{rgb}{1.000000,0.576471,0.309804}%
\pgfsetfillcolor{currentfill}%
\pgfsetlinewidth{1.003750pt}%
\definecolor{currentstroke}{rgb}{1.000000,0.576471,0.309804}%
\pgfsetstrokecolor{currentstroke}%
\pgfsetdash{}{0pt}%
\pgfsys@defobject{currentmarker}{\pgfqpoint{-0.033333in}{-0.033333in}}{\pgfqpoint{0.033333in}{0.033333in}}{%
\pgfpathmoveto{\pgfqpoint{0.000000in}{-0.033333in}}%
\pgfpathcurveto{\pgfqpoint{0.008840in}{-0.033333in}}{\pgfqpoint{0.017319in}{-0.029821in}}{\pgfqpoint{0.023570in}{-0.023570in}}%
\pgfpathcurveto{\pgfqpoint{0.029821in}{-0.017319in}}{\pgfqpoint{0.033333in}{-0.008840in}}{\pgfqpoint{0.033333in}{0.000000in}}%
\pgfpathcurveto{\pgfqpoint{0.033333in}{0.008840in}}{\pgfqpoint{0.029821in}{0.017319in}}{\pgfqpoint{0.023570in}{0.023570in}}%
\pgfpathcurveto{\pgfqpoint{0.017319in}{0.029821in}}{\pgfqpoint{0.008840in}{0.033333in}}{\pgfqpoint{0.000000in}{0.033333in}}%
\pgfpathcurveto{\pgfqpoint{-0.008840in}{0.033333in}}{\pgfqpoint{-0.017319in}{0.029821in}}{\pgfqpoint{-0.023570in}{0.023570in}}%
\pgfpathcurveto{\pgfqpoint{-0.029821in}{0.017319in}}{\pgfqpoint{-0.033333in}{0.008840in}}{\pgfqpoint{-0.033333in}{0.000000in}}%
\pgfpathcurveto{\pgfqpoint{-0.033333in}{-0.008840in}}{\pgfqpoint{-0.029821in}{-0.017319in}}{\pgfqpoint{-0.023570in}{-0.023570in}}%
\pgfpathcurveto{\pgfqpoint{-0.017319in}{-0.029821in}}{\pgfqpoint{-0.008840in}{-0.033333in}}{\pgfqpoint{0.000000in}{-0.033333in}}%
\pgfpathlineto{\pgfqpoint{0.000000in}{-0.033333in}}%
\pgfpathclose%
\pgfusepath{stroke,fill}%
}%
\begin{pgfscope}%
\pgfsys@transformshift{3.908905in}{1.704892in}%
\pgfsys@useobject{currentmarker}{}%
\end{pgfscope}%
\end{pgfscope}%
\begin{pgfscope}%
\pgfpathrectangle{\pgfqpoint{0.765000in}{0.660000in}}{\pgfqpoint{4.620000in}{4.620000in}}%
\pgfusepath{clip}%
\pgfsetrectcap%
\pgfsetroundjoin%
\pgfsetlinewidth{1.204500pt}%
\definecolor{currentstroke}{rgb}{1.000000,0.576471,0.309804}%
\pgfsetstrokecolor{currentstroke}%
\pgfsetdash{}{0pt}%
\pgfpathmoveto{\pgfqpoint{2.791023in}{2.704787in}}%
\pgfusepath{stroke}%
\end{pgfscope}%
\begin{pgfscope}%
\pgfpathrectangle{\pgfqpoint{0.765000in}{0.660000in}}{\pgfqpoint{4.620000in}{4.620000in}}%
\pgfusepath{clip}%
\pgfsetbuttcap%
\pgfsetroundjoin%
\definecolor{currentfill}{rgb}{1.000000,0.576471,0.309804}%
\pgfsetfillcolor{currentfill}%
\pgfsetlinewidth{1.003750pt}%
\definecolor{currentstroke}{rgb}{1.000000,0.576471,0.309804}%
\pgfsetstrokecolor{currentstroke}%
\pgfsetdash{}{0pt}%
\pgfsys@defobject{currentmarker}{\pgfqpoint{-0.033333in}{-0.033333in}}{\pgfqpoint{0.033333in}{0.033333in}}{%
\pgfpathmoveto{\pgfqpoint{0.000000in}{-0.033333in}}%
\pgfpathcurveto{\pgfqpoint{0.008840in}{-0.033333in}}{\pgfqpoint{0.017319in}{-0.029821in}}{\pgfqpoint{0.023570in}{-0.023570in}}%
\pgfpathcurveto{\pgfqpoint{0.029821in}{-0.017319in}}{\pgfqpoint{0.033333in}{-0.008840in}}{\pgfqpoint{0.033333in}{0.000000in}}%
\pgfpathcurveto{\pgfqpoint{0.033333in}{0.008840in}}{\pgfqpoint{0.029821in}{0.017319in}}{\pgfqpoint{0.023570in}{0.023570in}}%
\pgfpathcurveto{\pgfqpoint{0.017319in}{0.029821in}}{\pgfqpoint{0.008840in}{0.033333in}}{\pgfqpoint{0.000000in}{0.033333in}}%
\pgfpathcurveto{\pgfqpoint{-0.008840in}{0.033333in}}{\pgfqpoint{-0.017319in}{0.029821in}}{\pgfqpoint{-0.023570in}{0.023570in}}%
\pgfpathcurveto{\pgfqpoint{-0.029821in}{0.017319in}}{\pgfqpoint{-0.033333in}{0.008840in}}{\pgfqpoint{-0.033333in}{0.000000in}}%
\pgfpathcurveto{\pgfqpoint{-0.033333in}{-0.008840in}}{\pgfqpoint{-0.029821in}{-0.017319in}}{\pgfqpoint{-0.023570in}{-0.023570in}}%
\pgfpathcurveto{\pgfqpoint{-0.017319in}{-0.029821in}}{\pgfqpoint{-0.008840in}{-0.033333in}}{\pgfqpoint{0.000000in}{-0.033333in}}%
\pgfpathlineto{\pgfqpoint{0.000000in}{-0.033333in}}%
\pgfpathclose%
\pgfusepath{stroke,fill}%
}%
\begin{pgfscope}%
\pgfsys@transformshift{2.791023in}{2.704787in}%
\pgfsys@useobject{currentmarker}{}%
\end{pgfscope}%
\end{pgfscope}%
\begin{pgfscope}%
\pgfpathrectangle{\pgfqpoint{0.765000in}{0.660000in}}{\pgfqpoint{4.620000in}{4.620000in}}%
\pgfusepath{clip}%
\pgfsetrectcap%
\pgfsetroundjoin%
\pgfsetlinewidth{1.204500pt}%
\definecolor{currentstroke}{rgb}{1.000000,0.576471,0.309804}%
\pgfsetstrokecolor{currentstroke}%
\pgfsetdash{}{0pt}%
\pgfpathmoveto{\pgfqpoint{2.432284in}{1.975465in}}%
\pgfusepath{stroke}%
\end{pgfscope}%
\begin{pgfscope}%
\pgfpathrectangle{\pgfqpoint{0.765000in}{0.660000in}}{\pgfqpoint{4.620000in}{4.620000in}}%
\pgfusepath{clip}%
\pgfsetbuttcap%
\pgfsetroundjoin%
\definecolor{currentfill}{rgb}{1.000000,0.576471,0.309804}%
\pgfsetfillcolor{currentfill}%
\pgfsetlinewidth{1.003750pt}%
\definecolor{currentstroke}{rgb}{1.000000,0.576471,0.309804}%
\pgfsetstrokecolor{currentstroke}%
\pgfsetdash{}{0pt}%
\pgfsys@defobject{currentmarker}{\pgfqpoint{-0.033333in}{-0.033333in}}{\pgfqpoint{0.033333in}{0.033333in}}{%
\pgfpathmoveto{\pgfqpoint{0.000000in}{-0.033333in}}%
\pgfpathcurveto{\pgfqpoint{0.008840in}{-0.033333in}}{\pgfqpoint{0.017319in}{-0.029821in}}{\pgfqpoint{0.023570in}{-0.023570in}}%
\pgfpathcurveto{\pgfqpoint{0.029821in}{-0.017319in}}{\pgfqpoint{0.033333in}{-0.008840in}}{\pgfqpoint{0.033333in}{0.000000in}}%
\pgfpathcurveto{\pgfqpoint{0.033333in}{0.008840in}}{\pgfqpoint{0.029821in}{0.017319in}}{\pgfqpoint{0.023570in}{0.023570in}}%
\pgfpathcurveto{\pgfqpoint{0.017319in}{0.029821in}}{\pgfqpoint{0.008840in}{0.033333in}}{\pgfqpoint{0.000000in}{0.033333in}}%
\pgfpathcurveto{\pgfqpoint{-0.008840in}{0.033333in}}{\pgfqpoint{-0.017319in}{0.029821in}}{\pgfqpoint{-0.023570in}{0.023570in}}%
\pgfpathcurveto{\pgfqpoint{-0.029821in}{0.017319in}}{\pgfqpoint{-0.033333in}{0.008840in}}{\pgfqpoint{-0.033333in}{0.000000in}}%
\pgfpathcurveto{\pgfqpoint{-0.033333in}{-0.008840in}}{\pgfqpoint{-0.029821in}{-0.017319in}}{\pgfqpoint{-0.023570in}{-0.023570in}}%
\pgfpathcurveto{\pgfqpoint{-0.017319in}{-0.029821in}}{\pgfqpoint{-0.008840in}{-0.033333in}}{\pgfqpoint{0.000000in}{-0.033333in}}%
\pgfpathlineto{\pgfqpoint{0.000000in}{-0.033333in}}%
\pgfpathclose%
\pgfusepath{stroke,fill}%
}%
\begin{pgfscope}%
\pgfsys@transformshift{2.432284in}{1.975465in}%
\pgfsys@useobject{currentmarker}{}%
\end{pgfscope}%
\end{pgfscope}%
\begin{pgfscope}%
\pgfpathrectangle{\pgfqpoint{0.765000in}{0.660000in}}{\pgfqpoint{4.620000in}{4.620000in}}%
\pgfusepath{clip}%
\pgfsetrectcap%
\pgfsetroundjoin%
\pgfsetlinewidth{1.204500pt}%
\definecolor{currentstroke}{rgb}{1.000000,0.576471,0.309804}%
\pgfsetstrokecolor{currentstroke}%
\pgfsetdash{}{0pt}%
\pgfpathmoveto{\pgfqpoint{2.946836in}{1.744576in}}%
\pgfusepath{stroke}%
\end{pgfscope}%
\begin{pgfscope}%
\pgfpathrectangle{\pgfqpoint{0.765000in}{0.660000in}}{\pgfqpoint{4.620000in}{4.620000in}}%
\pgfusepath{clip}%
\pgfsetbuttcap%
\pgfsetroundjoin%
\definecolor{currentfill}{rgb}{1.000000,0.576471,0.309804}%
\pgfsetfillcolor{currentfill}%
\pgfsetlinewidth{1.003750pt}%
\definecolor{currentstroke}{rgb}{1.000000,0.576471,0.309804}%
\pgfsetstrokecolor{currentstroke}%
\pgfsetdash{}{0pt}%
\pgfsys@defobject{currentmarker}{\pgfqpoint{-0.033333in}{-0.033333in}}{\pgfqpoint{0.033333in}{0.033333in}}{%
\pgfpathmoveto{\pgfqpoint{0.000000in}{-0.033333in}}%
\pgfpathcurveto{\pgfqpoint{0.008840in}{-0.033333in}}{\pgfqpoint{0.017319in}{-0.029821in}}{\pgfqpoint{0.023570in}{-0.023570in}}%
\pgfpathcurveto{\pgfqpoint{0.029821in}{-0.017319in}}{\pgfqpoint{0.033333in}{-0.008840in}}{\pgfqpoint{0.033333in}{0.000000in}}%
\pgfpathcurveto{\pgfqpoint{0.033333in}{0.008840in}}{\pgfqpoint{0.029821in}{0.017319in}}{\pgfqpoint{0.023570in}{0.023570in}}%
\pgfpathcurveto{\pgfqpoint{0.017319in}{0.029821in}}{\pgfqpoint{0.008840in}{0.033333in}}{\pgfqpoint{0.000000in}{0.033333in}}%
\pgfpathcurveto{\pgfqpoint{-0.008840in}{0.033333in}}{\pgfqpoint{-0.017319in}{0.029821in}}{\pgfqpoint{-0.023570in}{0.023570in}}%
\pgfpathcurveto{\pgfqpoint{-0.029821in}{0.017319in}}{\pgfqpoint{-0.033333in}{0.008840in}}{\pgfqpoint{-0.033333in}{0.000000in}}%
\pgfpathcurveto{\pgfqpoint{-0.033333in}{-0.008840in}}{\pgfqpoint{-0.029821in}{-0.017319in}}{\pgfqpoint{-0.023570in}{-0.023570in}}%
\pgfpathcurveto{\pgfqpoint{-0.017319in}{-0.029821in}}{\pgfqpoint{-0.008840in}{-0.033333in}}{\pgfqpoint{0.000000in}{-0.033333in}}%
\pgfpathlineto{\pgfqpoint{0.000000in}{-0.033333in}}%
\pgfpathclose%
\pgfusepath{stroke,fill}%
}%
\begin{pgfscope}%
\pgfsys@transformshift{2.946836in}{1.744576in}%
\pgfsys@useobject{currentmarker}{}%
\end{pgfscope}%
\end{pgfscope}%
\begin{pgfscope}%
\pgfpathrectangle{\pgfqpoint{0.765000in}{0.660000in}}{\pgfqpoint{4.620000in}{4.620000in}}%
\pgfusepath{clip}%
\pgfsetrectcap%
\pgfsetroundjoin%
\pgfsetlinewidth{1.204500pt}%
\definecolor{currentstroke}{rgb}{1.000000,0.576471,0.309804}%
\pgfsetstrokecolor{currentstroke}%
\pgfsetdash{}{0pt}%
\pgfpathmoveto{\pgfqpoint{4.424979in}{3.314950in}}%
\pgfusepath{stroke}%
\end{pgfscope}%
\begin{pgfscope}%
\pgfpathrectangle{\pgfqpoint{0.765000in}{0.660000in}}{\pgfqpoint{4.620000in}{4.620000in}}%
\pgfusepath{clip}%
\pgfsetbuttcap%
\pgfsetroundjoin%
\definecolor{currentfill}{rgb}{1.000000,0.576471,0.309804}%
\pgfsetfillcolor{currentfill}%
\pgfsetlinewidth{1.003750pt}%
\definecolor{currentstroke}{rgb}{1.000000,0.576471,0.309804}%
\pgfsetstrokecolor{currentstroke}%
\pgfsetdash{}{0pt}%
\pgfsys@defobject{currentmarker}{\pgfqpoint{-0.033333in}{-0.033333in}}{\pgfqpoint{0.033333in}{0.033333in}}{%
\pgfpathmoveto{\pgfqpoint{0.000000in}{-0.033333in}}%
\pgfpathcurveto{\pgfqpoint{0.008840in}{-0.033333in}}{\pgfqpoint{0.017319in}{-0.029821in}}{\pgfqpoint{0.023570in}{-0.023570in}}%
\pgfpathcurveto{\pgfqpoint{0.029821in}{-0.017319in}}{\pgfqpoint{0.033333in}{-0.008840in}}{\pgfqpoint{0.033333in}{0.000000in}}%
\pgfpathcurveto{\pgfqpoint{0.033333in}{0.008840in}}{\pgfqpoint{0.029821in}{0.017319in}}{\pgfqpoint{0.023570in}{0.023570in}}%
\pgfpathcurveto{\pgfqpoint{0.017319in}{0.029821in}}{\pgfqpoint{0.008840in}{0.033333in}}{\pgfqpoint{0.000000in}{0.033333in}}%
\pgfpathcurveto{\pgfqpoint{-0.008840in}{0.033333in}}{\pgfqpoint{-0.017319in}{0.029821in}}{\pgfqpoint{-0.023570in}{0.023570in}}%
\pgfpathcurveto{\pgfqpoint{-0.029821in}{0.017319in}}{\pgfqpoint{-0.033333in}{0.008840in}}{\pgfqpoint{-0.033333in}{0.000000in}}%
\pgfpathcurveto{\pgfqpoint{-0.033333in}{-0.008840in}}{\pgfqpoint{-0.029821in}{-0.017319in}}{\pgfqpoint{-0.023570in}{-0.023570in}}%
\pgfpathcurveto{\pgfqpoint{-0.017319in}{-0.029821in}}{\pgfqpoint{-0.008840in}{-0.033333in}}{\pgfqpoint{0.000000in}{-0.033333in}}%
\pgfpathlineto{\pgfqpoint{0.000000in}{-0.033333in}}%
\pgfpathclose%
\pgfusepath{stroke,fill}%
}%
\begin{pgfscope}%
\pgfsys@transformshift{4.424979in}{3.314950in}%
\pgfsys@useobject{currentmarker}{}%
\end{pgfscope}%
\end{pgfscope}%
\begin{pgfscope}%
\pgfpathrectangle{\pgfqpoint{0.765000in}{0.660000in}}{\pgfqpoint{4.620000in}{4.620000in}}%
\pgfusepath{clip}%
\pgfsetrectcap%
\pgfsetroundjoin%
\pgfsetlinewidth{1.204500pt}%
\definecolor{currentstroke}{rgb}{1.000000,0.576471,0.309804}%
\pgfsetstrokecolor{currentstroke}%
\pgfsetdash{}{0pt}%
\pgfpathmoveto{\pgfqpoint{2.444721in}{2.494621in}}%
\pgfusepath{stroke}%
\end{pgfscope}%
\begin{pgfscope}%
\pgfpathrectangle{\pgfqpoint{0.765000in}{0.660000in}}{\pgfqpoint{4.620000in}{4.620000in}}%
\pgfusepath{clip}%
\pgfsetbuttcap%
\pgfsetroundjoin%
\definecolor{currentfill}{rgb}{1.000000,0.576471,0.309804}%
\pgfsetfillcolor{currentfill}%
\pgfsetlinewidth{1.003750pt}%
\definecolor{currentstroke}{rgb}{1.000000,0.576471,0.309804}%
\pgfsetstrokecolor{currentstroke}%
\pgfsetdash{}{0pt}%
\pgfsys@defobject{currentmarker}{\pgfqpoint{-0.033333in}{-0.033333in}}{\pgfqpoint{0.033333in}{0.033333in}}{%
\pgfpathmoveto{\pgfqpoint{0.000000in}{-0.033333in}}%
\pgfpathcurveto{\pgfqpoint{0.008840in}{-0.033333in}}{\pgfqpoint{0.017319in}{-0.029821in}}{\pgfqpoint{0.023570in}{-0.023570in}}%
\pgfpathcurveto{\pgfqpoint{0.029821in}{-0.017319in}}{\pgfqpoint{0.033333in}{-0.008840in}}{\pgfqpoint{0.033333in}{0.000000in}}%
\pgfpathcurveto{\pgfqpoint{0.033333in}{0.008840in}}{\pgfqpoint{0.029821in}{0.017319in}}{\pgfqpoint{0.023570in}{0.023570in}}%
\pgfpathcurveto{\pgfqpoint{0.017319in}{0.029821in}}{\pgfqpoint{0.008840in}{0.033333in}}{\pgfqpoint{0.000000in}{0.033333in}}%
\pgfpathcurveto{\pgfqpoint{-0.008840in}{0.033333in}}{\pgfqpoint{-0.017319in}{0.029821in}}{\pgfqpoint{-0.023570in}{0.023570in}}%
\pgfpathcurveto{\pgfqpoint{-0.029821in}{0.017319in}}{\pgfqpoint{-0.033333in}{0.008840in}}{\pgfqpoint{-0.033333in}{0.000000in}}%
\pgfpathcurveto{\pgfqpoint{-0.033333in}{-0.008840in}}{\pgfqpoint{-0.029821in}{-0.017319in}}{\pgfqpoint{-0.023570in}{-0.023570in}}%
\pgfpathcurveto{\pgfqpoint{-0.017319in}{-0.029821in}}{\pgfqpoint{-0.008840in}{-0.033333in}}{\pgfqpoint{0.000000in}{-0.033333in}}%
\pgfpathlineto{\pgfqpoint{0.000000in}{-0.033333in}}%
\pgfpathclose%
\pgfusepath{stroke,fill}%
}%
\begin{pgfscope}%
\pgfsys@transformshift{2.444721in}{2.494621in}%
\pgfsys@useobject{currentmarker}{}%
\end{pgfscope}%
\end{pgfscope}%
\begin{pgfscope}%
\pgfpathrectangle{\pgfqpoint{0.765000in}{0.660000in}}{\pgfqpoint{4.620000in}{4.620000in}}%
\pgfusepath{clip}%
\pgfsetrectcap%
\pgfsetroundjoin%
\pgfsetlinewidth{1.204500pt}%
\definecolor{currentstroke}{rgb}{1.000000,0.576471,0.309804}%
\pgfsetstrokecolor{currentstroke}%
\pgfsetdash{}{0pt}%
\pgfpathmoveto{\pgfqpoint{2.490143in}{1.973995in}}%
\pgfusepath{stroke}%
\end{pgfscope}%
\begin{pgfscope}%
\pgfpathrectangle{\pgfqpoint{0.765000in}{0.660000in}}{\pgfqpoint{4.620000in}{4.620000in}}%
\pgfusepath{clip}%
\pgfsetbuttcap%
\pgfsetroundjoin%
\definecolor{currentfill}{rgb}{1.000000,0.576471,0.309804}%
\pgfsetfillcolor{currentfill}%
\pgfsetlinewidth{1.003750pt}%
\definecolor{currentstroke}{rgb}{1.000000,0.576471,0.309804}%
\pgfsetstrokecolor{currentstroke}%
\pgfsetdash{}{0pt}%
\pgfsys@defobject{currentmarker}{\pgfqpoint{-0.033333in}{-0.033333in}}{\pgfqpoint{0.033333in}{0.033333in}}{%
\pgfpathmoveto{\pgfqpoint{0.000000in}{-0.033333in}}%
\pgfpathcurveto{\pgfqpoint{0.008840in}{-0.033333in}}{\pgfqpoint{0.017319in}{-0.029821in}}{\pgfqpoint{0.023570in}{-0.023570in}}%
\pgfpathcurveto{\pgfqpoint{0.029821in}{-0.017319in}}{\pgfqpoint{0.033333in}{-0.008840in}}{\pgfqpoint{0.033333in}{0.000000in}}%
\pgfpathcurveto{\pgfqpoint{0.033333in}{0.008840in}}{\pgfqpoint{0.029821in}{0.017319in}}{\pgfqpoint{0.023570in}{0.023570in}}%
\pgfpathcurveto{\pgfqpoint{0.017319in}{0.029821in}}{\pgfqpoint{0.008840in}{0.033333in}}{\pgfqpoint{0.000000in}{0.033333in}}%
\pgfpathcurveto{\pgfqpoint{-0.008840in}{0.033333in}}{\pgfqpoint{-0.017319in}{0.029821in}}{\pgfqpoint{-0.023570in}{0.023570in}}%
\pgfpathcurveto{\pgfqpoint{-0.029821in}{0.017319in}}{\pgfqpoint{-0.033333in}{0.008840in}}{\pgfqpoint{-0.033333in}{0.000000in}}%
\pgfpathcurveto{\pgfqpoint{-0.033333in}{-0.008840in}}{\pgfqpoint{-0.029821in}{-0.017319in}}{\pgfqpoint{-0.023570in}{-0.023570in}}%
\pgfpathcurveto{\pgfqpoint{-0.017319in}{-0.029821in}}{\pgfqpoint{-0.008840in}{-0.033333in}}{\pgfqpoint{0.000000in}{-0.033333in}}%
\pgfpathlineto{\pgfqpoint{0.000000in}{-0.033333in}}%
\pgfpathclose%
\pgfusepath{stroke,fill}%
}%
\begin{pgfscope}%
\pgfsys@transformshift{2.490143in}{1.973995in}%
\pgfsys@useobject{currentmarker}{}%
\end{pgfscope}%
\end{pgfscope}%
\begin{pgfscope}%
\pgfpathrectangle{\pgfqpoint{0.765000in}{0.660000in}}{\pgfqpoint{4.620000in}{4.620000in}}%
\pgfusepath{clip}%
\pgfsetrectcap%
\pgfsetroundjoin%
\pgfsetlinewidth{1.204500pt}%
\definecolor{currentstroke}{rgb}{1.000000,0.576471,0.309804}%
\pgfsetstrokecolor{currentstroke}%
\pgfsetdash{}{0pt}%
\pgfpathmoveto{\pgfqpoint{2.734149in}{1.494008in}}%
\pgfusepath{stroke}%
\end{pgfscope}%
\begin{pgfscope}%
\pgfpathrectangle{\pgfqpoint{0.765000in}{0.660000in}}{\pgfqpoint{4.620000in}{4.620000in}}%
\pgfusepath{clip}%
\pgfsetbuttcap%
\pgfsetroundjoin%
\definecolor{currentfill}{rgb}{1.000000,0.576471,0.309804}%
\pgfsetfillcolor{currentfill}%
\pgfsetlinewidth{1.003750pt}%
\definecolor{currentstroke}{rgb}{1.000000,0.576471,0.309804}%
\pgfsetstrokecolor{currentstroke}%
\pgfsetdash{}{0pt}%
\pgfsys@defobject{currentmarker}{\pgfqpoint{-0.033333in}{-0.033333in}}{\pgfqpoint{0.033333in}{0.033333in}}{%
\pgfpathmoveto{\pgfqpoint{0.000000in}{-0.033333in}}%
\pgfpathcurveto{\pgfqpoint{0.008840in}{-0.033333in}}{\pgfqpoint{0.017319in}{-0.029821in}}{\pgfqpoint{0.023570in}{-0.023570in}}%
\pgfpathcurveto{\pgfqpoint{0.029821in}{-0.017319in}}{\pgfqpoint{0.033333in}{-0.008840in}}{\pgfqpoint{0.033333in}{0.000000in}}%
\pgfpathcurveto{\pgfqpoint{0.033333in}{0.008840in}}{\pgfqpoint{0.029821in}{0.017319in}}{\pgfqpoint{0.023570in}{0.023570in}}%
\pgfpathcurveto{\pgfqpoint{0.017319in}{0.029821in}}{\pgfqpoint{0.008840in}{0.033333in}}{\pgfqpoint{0.000000in}{0.033333in}}%
\pgfpathcurveto{\pgfqpoint{-0.008840in}{0.033333in}}{\pgfqpoint{-0.017319in}{0.029821in}}{\pgfqpoint{-0.023570in}{0.023570in}}%
\pgfpathcurveto{\pgfqpoint{-0.029821in}{0.017319in}}{\pgfqpoint{-0.033333in}{0.008840in}}{\pgfqpoint{-0.033333in}{0.000000in}}%
\pgfpathcurveto{\pgfqpoint{-0.033333in}{-0.008840in}}{\pgfqpoint{-0.029821in}{-0.017319in}}{\pgfqpoint{-0.023570in}{-0.023570in}}%
\pgfpathcurveto{\pgfqpoint{-0.017319in}{-0.029821in}}{\pgfqpoint{-0.008840in}{-0.033333in}}{\pgfqpoint{0.000000in}{-0.033333in}}%
\pgfpathlineto{\pgfqpoint{0.000000in}{-0.033333in}}%
\pgfpathclose%
\pgfusepath{stroke,fill}%
}%
\begin{pgfscope}%
\pgfsys@transformshift{2.734149in}{1.494008in}%
\pgfsys@useobject{currentmarker}{}%
\end{pgfscope}%
\end{pgfscope}%
\begin{pgfscope}%
\pgfpathrectangle{\pgfqpoint{0.765000in}{0.660000in}}{\pgfqpoint{4.620000in}{4.620000in}}%
\pgfusepath{clip}%
\pgfsetrectcap%
\pgfsetroundjoin%
\pgfsetlinewidth{1.204500pt}%
\definecolor{currentstroke}{rgb}{1.000000,0.576471,0.309804}%
\pgfsetstrokecolor{currentstroke}%
\pgfsetdash{}{0pt}%
\pgfpathmoveto{\pgfqpoint{3.196140in}{2.399556in}}%
\pgfusepath{stroke}%
\end{pgfscope}%
\begin{pgfscope}%
\pgfpathrectangle{\pgfqpoint{0.765000in}{0.660000in}}{\pgfqpoint{4.620000in}{4.620000in}}%
\pgfusepath{clip}%
\pgfsetbuttcap%
\pgfsetroundjoin%
\definecolor{currentfill}{rgb}{1.000000,0.576471,0.309804}%
\pgfsetfillcolor{currentfill}%
\pgfsetlinewidth{1.003750pt}%
\definecolor{currentstroke}{rgb}{1.000000,0.576471,0.309804}%
\pgfsetstrokecolor{currentstroke}%
\pgfsetdash{}{0pt}%
\pgfsys@defobject{currentmarker}{\pgfqpoint{-0.033333in}{-0.033333in}}{\pgfqpoint{0.033333in}{0.033333in}}{%
\pgfpathmoveto{\pgfqpoint{0.000000in}{-0.033333in}}%
\pgfpathcurveto{\pgfqpoint{0.008840in}{-0.033333in}}{\pgfqpoint{0.017319in}{-0.029821in}}{\pgfqpoint{0.023570in}{-0.023570in}}%
\pgfpathcurveto{\pgfqpoint{0.029821in}{-0.017319in}}{\pgfqpoint{0.033333in}{-0.008840in}}{\pgfqpoint{0.033333in}{0.000000in}}%
\pgfpathcurveto{\pgfqpoint{0.033333in}{0.008840in}}{\pgfqpoint{0.029821in}{0.017319in}}{\pgfqpoint{0.023570in}{0.023570in}}%
\pgfpathcurveto{\pgfqpoint{0.017319in}{0.029821in}}{\pgfqpoint{0.008840in}{0.033333in}}{\pgfqpoint{0.000000in}{0.033333in}}%
\pgfpathcurveto{\pgfqpoint{-0.008840in}{0.033333in}}{\pgfqpoint{-0.017319in}{0.029821in}}{\pgfqpoint{-0.023570in}{0.023570in}}%
\pgfpathcurveto{\pgfqpoint{-0.029821in}{0.017319in}}{\pgfqpoint{-0.033333in}{0.008840in}}{\pgfqpoint{-0.033333in}{0.000000in}}%
\pgfpathcurveto{\pgfqpoint{-0.033333in}{-0.008840in}}{\pgfqpoint{-0.029821in}{-0.017319in}}{\pgfqpoint{-0.023570in}{-0.023570in}}%
\pgfpathcurveto{\pgfqpoint{-0.017319in}{-0.029821in}}{\pgfqpoint{-0.008840in}{-0.033333in}}{\pgfqpoint{0.000000in}{-0.033333in}}%
\pgfpathlineto{\pgfqpoint{0.000000in}{-0.033333in}}%
\pgfpathclose%
\pgfusepath{stroke,fill}%
}%
\begin{pgfscope}%
\pgfsys@transformshift{3.196140in}{2.399556in}%
\pgfsys@useobject{currentmarker}{}%
\end{pgfscope}%
\end{pgfscope}%
\begin{pgfscope}%
\pgfpathrectangle{\pgfqpoint{0.765000in}{0.660000in}}{\pgfqpoint{4.620000in}{4.620000in}}%
\pgfusepath{clip}%
\pgfsetrectcap%
\pgfsetroundjoin%
\pgfsetlinewidth{1.204500pt}%
\definecolor{currentstroke}{rgb}{1.000000,0.576471,0.309804}%
\pgfsetstrokecolor{currentstroke}%
\pgfsetdash{}{0pt}%
\pgfpathmoveto{\pgfqpoint{3.214489in}{2.051429in}}%
\pgfusepath{stroke}%
\end{pgfscope}%
\begin{pgfscope}%
\pgfpathrectangle{\pgfqpoint{0.765000in}{0.660000in}}{\pgfqpoint{4.620000in}{4.620000in}}%
\pgfusepath{clip}%
\pgfsetbuttcap%
\pgfsetroundjoin%
\definecolor{currentfill}{rgb}{1.000000,0.576471,0.309804}%
\pgfsetfillcolor{currentfill}%
\pgfsetlinewidth{1.003750pt}%
\definecolor{currentstroke}{rgb}{1.000000,0.576471,0.309804}%
\pgfsetstrokecolor{currentstroke}%
\pgfsetdash{}{0pt}%
\pgfsys@defobject{currentmarker}{\pgfqpoint{-0.033333in}{-0.033333in}}{\pgfqpoint{0.033333in}{0.033333in}}{%
\pgfpathmoveto{\pgfqpoint{0.000000in}{-0.033333in}}%
\pgfpathcurveto{\pgfqpoint{0.008840in}{-0.033333in}}{\pgfqpoint{0.017319in}{-0.029821in}}{\pgfqpoint{0.023570in}{-0.023570in}}%
\pgfpathcurveto{\pgfqpoint{0.029821in}{-0.017319in}}{\pgfqpoint{0.033333in}{-0.008840in}}{\pgfqpoint{0.033333in}{0.000000in}}%
\pgfpathcurveto{\pgfqpoint{0.033333in}{0.008840in}}{\pgfqpoint{0.029821in}{0.017319in}}{\pgfqpoint{0.023570in}{0.023570in}}%
\pgfpathcurveto{\pgfqpoint{0.017319in}{0.029821in}}{\pgfqpoint{0.008840in}{0.033333in}}{\pgfqpoint{0.000000in}{0.033333in}}%
\pgfpathcurveto{\pgfqpoint{-0.008840in}{0.033333in}}{\pgfqpoint{-0.017319in}{0.029821in}}{\pgfqpoint{-0.023570in}{0.023570in}}%
\pgfpathcurveto{\pgfqpoint{-0.029821in}{0.017319in}}{\pgfqpoint{-0.033333in}{0.008840in}}{\pgfqpoint{-0.033333in}{0.000000in}}%
\pgfpathcurveto{\pgfqpoint{-0.033333in}{-0.008840in}}{\pgfqpoint{-0.029821in}{-0.017319in}}{\pgfqpoint{-0.023570in}{-0.023570in}}%
\pgfpathcurveto{\pgfqpoint{-0.017319in}{-0.029821in}}{\pgfqpoint{-0.008840in}{-0.033333in}}{\pgfqpoint{0.000000in}{-0.033333in}}%
\pgfpathlineto{\pgfqpoint{0.000000in}{-0.033333in}}%
\pgfpathclose%
\pgfusepath{stroke,fill}%
}%
\begin{pgfscope}%
\pgfsys@transformshift{3.214489in}{2.051429in}%
\pgfsys@useobject{currentmarker}{}%
\end{pgfscope}%
\end{pgfscope}%
\begin{pgfscope}%
\pgfpathrectangle{\pgfqpoint{0.765000in}{0.660000in}}{\pgfqpoint{4.620000in}{4.620000in}}%
\pgfusepath{clip}%
\pgfsetrectcap%
\pgfsetroundjoin%
\pgfsetlinewidth{1.204500pt}%
\definecolor{currentstroke}{rgb}{1.000000,0.576471,0.309804}%
\pgfsetstrokecolor{currentstroke}%
\pgfsetdash{}{0pt}%
\pgfpathmoveto{\pgfqpoint{2.835360in}{2.112861in}}%
\pgfusepath{stroke}%
\end{pgfscope}%
\begin{pgfscope}%
\pgfpathrectangle{\pgfqpoint{0.765000in}{0.660000in}}{\pgfqpoint{4.620000in}{4.620000in}}%
\pgfusepath{clip}%
\pgfsetbuttcap%
\pgfsetroundjoin%
\definecolor{currentfill}{rgb}{1.000000,0.576471,0.309804}%
\pgfsetfillcolor{currentfill}%
\pgfsetlinewidth{1.003750pt}%
\definecolor{currentstroke}{rgb}{1.000000,0.576471,0.309804}%
\pgfsetstrokecolor{currentstroke}%
\pgfsetdash{}{0pt}%
\pgfsys@defobject{currentmarker}{\pgfqpoint{-0.033333in}{-0.033333in}}{\pgfqpoint{0.033333in}{0.033333in}}{%
\pgfpathmoveto{\pgfqpoint{0.000000in}{-0.033333in}}%
\pgfpathcurveto{\pgfqpoint{0.008840in}{-0.033333in}}{\pgfqpoint{0.017319in}{-0.029821in}}{\pgfqpoint{0.023570in}{-0.023570in}}%
\pgfpathcurveto{\pgfqpoint{0.029821in}{-0.017319in}}{\pgfqpoint{0.033333in}{-0.008840in}}{\pgfqpoint{0.033333in}{0.000000in}}%
\pgfpathcurveto{\pgfqpoint{0.033333in}{0.008840in}}{\pgfqpoint{0.029821in}{0.017319in}}{\pgfqpoint{0.023570in}{0.023570in}}%
\pgfpathcurveto{\pgfqpoint{0.017319in}{0.029821in}}{\pgfqpoint{0.008840in}{0.033333in}}{\pgfqpoint{0.000000in}{0.033333in}}%
\pgfpathcurveto{\pgfqpoint{-0.008840in}{0.033333in}}{\pgfqpoint{-0.017319in}{0.029821in}}{\pgfqpoint{-0.023570in}{0.023570in}}%
\pgfpathcurveto{\pgfqpoint{-0.029821in}{0.017319in}}{\pgfqpoint{-0.033333in}{0.008840in}}{\pgfqpoint{-0.033333in}{0.000000in}}%
\pgfpathcurveto{\pgfqpoint{-0.033333in}{-0.008840in}}{\pgfqpoint{-0.029821in}{-0.017319in}}{\pgfqpoint{-0.023570in}{-0.023570in}}%
\pgfpathcurveto{\pgfqpoint{-0.017319in}{-0.029821in}}{\pgfqpoint{-0.008840in}{-0.033333in}}{\pgfqpoint{0.000000in}{-0.033333in}}%
\pgfpathlineto{\pgfqpoint{0.000000in}{-0.033333in}}%
\pgfpathclose%
\pgfusepath{stroke,fill}%
}%
\begin{pgfscope}%
\pgfsys@transformshift{2.835360in}{2.112861in}%
\pgfsys@useobject{currentmarker}{}%
\end{pgfscope}%
\end{pgfscope}%
\begin{pgfscope}%
\pgfpathrectangle{\pgfqpoint{0.765000in}{0.660000in}}{\pgfqpoint{4.620000in}{4.620000in}}%
\pgfusepath{clip}%
\pgfsetrectcap%
\pgfsetroundjoin%
\pgfsetlinewidth{1.204500pt}%
\definecolor{currentstroke}{rgb}{1.000000,0.576471,0.309804}%
\pgfsetstrokecolor{currentstroke}%
\pgfsetdash{}{0pt}%
\pgfpathmoveto{\pgfqpoint{1.920389in}{2.822279in}}%
\pgfusepath{stroke}%
\end{pgfscope}%
\begin{pgfscope}%
\pgfpathrectangle{\pgfqpoint{0.765000in}{0.660000in}}{\pgfqpoint{4.620000in}{4.620000in}}%
\pgfusepath{clip}%
\pgfsetbuttcap%
\pgfsetroundjoin%
\definecolor{currentfill}{rgb}{1.000000,0.576471,0.309804}%
\pgfsetfillcolor{currentfill}%
\pgfsetlinewidth{1.003750pt}%
\definecolor{currentstroke}{rgb}{1.000000,0.576471,0.309804}%
\pgfsetstrokecolor{currentstroke}%
\pgfsetdash{}{0pt}%
\pgfsys@defobject{currentmarker}{\pgfqpoint{-0.033333in}{-0.033333in}}{\pgfqpoint{0.033333in}{0.033333in}}{%
\pgfpathmoveto{\pgfqpoint{0.000000in}{-0.033333in}}%
\pgfpathcurveto{\pgfqpoint{0.008840in}{-0.033333in}}{\pgfqpoint{0.017319in}{-0.029821in}}{\pgfqpoint{0.023570in}{-0.023570in}}%
\pgfpathcurveto{\pgfqpoint{0.029821in}{-0.017319in}}{\pgfqpoint{0.033333in}{-0.008840in}}{\pgfqpoint{0.033333in}{0.000000in}}%
\pgfpathcurveto{\pgfqpoint{0.033333in}{0.008840in}}{\pgfqpoint{0.029821in}{0.017319in}}{\pgfqpoint{0.023570in}{0.023570in}}%
\pgfpathcurveto{\pgfqpoint{0.017319in}{0.029821in}}{\pgfqpoint{0.008840in}{0.033333in}}{\pgfqpoint{0.000000in}{0.033333in}}%
\pgfpathcurveto{\pgfqpoint{-0.008840in}{0.033333in}}{\pgfqpoint{-0.017319in}{0.029821in}}{\pgfqpoint{-0.023570in}{0.023570in}}%
\pgfpathcurveto{\pgfqpoint{-0.029821in}{0.017319in}}{\pgfqpoint{-0.033333in}{0.008840in}}{\pgfqpoint{-0.033333in}{0.000000in}}%
\pgfpathcurveto{\pgfqpoint{-0.033333in}{-0.008840in}}{\pgfqpoint{-0.029821in}{-0.017319in}}{\pgfqpoint{-0.023570in}{-0.023570in}}%
\pgfpathcurveto{\pgfqpoint{-0.017319in}{-0.029821in}}{\pgfqpoint{-0.008840in}{-0.033333in}}{\pgfqpoint{0.000000in}{-0.033333in}}%
\pgfpathlineto{\pgfqpoint{0.000000in}{-0.033333in}}%
\pgfpathclose%
\pgfusepath{stroke,fill}%
}%
\begin{pgfscope}%
\pgfsys@transformshift{1.920389in}{2.822279in}%
\pgfsys@useobject{currentmarker}{}%
\end{pgfscope}%
\end{pgfscope}%
\begin{pgfscope}%
\pgfpathrectangle{\pgfqpoint{0.765000in}{0.660000in}}{\pgfqpoint{4.620000in}{4.620000in}}%
\pgfusepath{clip}%
\pgfsetrectcap%
\pgfsetroundjoin%
\pgfsetlinewidth{1.204500pt}%
\definecolor{currentstroke}{rgb}{1.000000,0.576471,0.309804}%
\pgfsetstrokecolor{currentstroke}%
\pgfsetdash{}{0pt}%
\pgfpathmoveto{\pgfqpoint{2.295346in}{2.961689in}}%
\pgfusepath{stroke}%
\end{pgfscope}%
\begin{pgfscope}%
\pgfpathrectangle{\pgfqpoint{0.765000in}{0.660000in}}{\pgfqpoint{4.620000in}{4.620000in}}%
\pgfusepath{clip}%
\pgfsetbuttcap%
\pgfsetroundjoin%
\definecolor{currentfill}{rgb}{1.000000,0.576471,0.309804}%
\pgfsetfillcolor{currentfill}%
\pgfsetlinewidth{1.003750pt}%
\definecolor{currentstroke}{rgb}{1.000000,0.576471,0.309804}%
\pgfsetstrokecolor{currentstroke}%
\pgfsetdash{}{0pt}%
\pgfsys@defobject{currentmarker}{\pgfqpoint{-0.033333in}{-0.033333in}}{\pgfqpoint{0.033333in}{0.033333in}}{%
\pgfpathmoveto{\pgfqpoint{0.000000in}{-0.033333in}}%
\pgfpathcurveto{\pgfqpoint{0.008840in}{-0.033333in}}{\pgfqpoint{0.017319in}{-0.029821in}}{\pgfqpoint{0.023570in}{-0.023570in}}%
\pgfpathcurveto{\pgfqpoint{0.029821in}{-0.017319in}}{\pgfqpoint{0.033333in}{-0.008840in}}{\pgfqpoint{0.033333in}{0.000000in}}%
\pgfpathcurveto{\pgfqpoint{0.033333in}{0.008840in}}{\pgfqpoint{0.029821in}{0.017319in}}{\pgfqpoint{0.023570in}{0.023570in}}%
\pgfpathcurveto{\pgfqpoint{0.017319in}{0.029821in}}{\pgfqpoint{0.008840in}{0.033333in}}{\pgfqpoint{0.000000in}{0.033333in}}%
\pgfpathcurveto{\pgfqpoint{-0.008840in}{0.033333in}}{\pgfqpoint{-0.017319in}{0.029821in}}{\pgfqpoint{-0.023570in}{0.023570in}}%
\pgfpathcurveto{\pgfqpoint{-0.029821in}{0.017319in}}{\pgfqpoint{-0.033333in}{0.008840in}}{\pgfqpoint{-0.033333in}{0.000000in}}%
\pgfpathcurveto{\pgfqpoint{-0.033333in}{-0.008840in}}{\pgfqpoint{-0.029821in}{-0.017319in}}{\pgfqpoint{-0.023570in}{-0.023570in}}%
\pgfpathcurveto{\pgfqpoint{-0.017319in}{-0.029821in}}{\pgfqpoint{-0.008840in}{-0.033333in}}{\pgfqpoint{0.000000in}{-0.033333in}}%
\pgfpathlineto{\pgfqpoint{0.000000in}{-0.033333in}}%
\pgfpathclose%
\pgfusepath{stroke,fill}%
}%
\begin{pgfscope}%
\pgfsys@transformshift{2.295346in}{2.961689in}%
\pgfsys@useobject{currentmarker}{}%
\end{pgfscope}%
\end{pgfscope}%
\begin{pgfscope}%
\pgfpathrectangle{\pgfqpoint{0.765000in}{0.660000in}}{\pgfqpoint{4.620000in}{4.620000in}}%
\pgfusepath{clip}%
\pgfsetrectcap%
\pgfsetroundjoin%
\pgfsetlinewidth{1.204500pt}%
\definecolor{currentstroke}{rgb}{1.000000,0.576471,0.309804}%
\pgfsetstrokecolor{currentstroke}%
\pgfsetdash{}{0pt}%
\pgfpathmoveto{\pgfqpoint{2.514423in}{1.983254in}}%
\pgfusepath{stroke}%
\end{pgfscope}%
\begin{pgfscope}%
\pgfpathrectangle{\pgfqpoint{0.765000in}{0.660000in}}{\pgfqpoint{4.620000in}{4.620000in}}%
\pgfusepath{clip}%
\pgfsetbuttcap%
\pgfsetroundjoin%
\definecolor{currentfill}{rgb}{1.000000,0.576471,0.309804}%
\pgfsetfillcolor{currentfill}%
\pgfsetlinewidth{1.003750pt}%
\definecolor{currentstroke}{rgb}{1.000000,0.576471,0.309804}%
\pgfsetstrokecolor{currentstroke}%
\pgfsetdash{}{0pt}%
\pgfsys@defobject{currentmarker}{\pgfqpoint{-0.033333in}{-0.033333in}}{\pgfqpoint{0.033333in}{0.033333in}}{%
\pgfpathmoveto{\pgfqpoint{0.000000in}{-0.033333in}}%
\pgfpathcurveto{\pgfqpoint{0.008840in}{-0.033333in}}{\pgfqpoint{0.017319in}{-0.029821in}}{\pgfqpoint{0.023570in}{-0.023570in}}%
\pgfpathcurveto{\pgfqpoint{0.029821in}{-0.017319in}}{\pgfqpoint{0.033333in}{-0.008840in}}{\pgfqpoint{0.033333in}{0.000000in}}%
\pgfpathcurveto{\pgfqpoint{0.033333in}{0.008840in}}{\pgfqpoint{0.029821in}{0.017319in}}{\pgfqpoint{0.023570in}{0.023570in}}%
\pgfpathcurveto{\pgfqpoint{0.017319in}{0.029821in}}{\pgfqpoint{0.008840in}{0.033333in}}{\pgfqpoint{0.000000in}{0.033333in}}%
\pgfpathcurveto{\pgfqpoint{-0.008840in}{0.033333in}}{\pgfqpoint{-0.017319in}{0.029821in}}{\pgfqpoint{-0.023570in}{0.023570in}}%
\pgfpathcurveto{\pgfqpoint{-0.029821in}{0.017319in}}{\pgfqpoint{-0.033333in}{0.008840in}}{\pgfqpoint{-0.033333in}{0.000000in}}%
\pgfpathcurveto{\pgfqpoint{-0.033333in}{-0.008840in}}{\pgfqpoint{-0.029821in}{-0.017319in}}{\pgfqpoint{-0.023570in}{-0.023570in}}%
\pgfpathcurveto{\pgfqpoint{-0.017319in}{-0.029821in}}{\pgfqpoint{-0.008840in}{-0.033333in}}{\pgfqpoint{0.000000in}{-0.033333in}}%
\pgfpathlineto{\pgfqpoint{0.000000in}{-0.033333in}}%
\pgfpathclose%
\pgfusepath{stroke,fill}%
}%
\begin{pgfscope}%
\pgfsys@transformshift{2.514423in}{1.983254in}%
\pgfsys@useobject{currentmarker}{}%
\end{pgfscope}%
\end{pgfscope}%
\begin{pgfscope}%
\pgfpathrectangle{\pgfqpoint{0.765000in}{0.660000in}}{\pgfqpoint{4.620000in}{4.620000in}}%
\pgfusepath{clip}%
\pgfsetrectcap%
\pgfsetroundjoin%
\pgfsetlinewidth{1.204500pt}%
\definecolor{currentstroke}{rgb}{1.000000,0.576471,0.309804}%
\pgfsetstrokecolor{currentstroke}%
\pgfsetdash{}{0pt}%
\pgfpathmoveto{\pgfqpoint{2.590185in}{2.863295in}}%
\pgfusepath{stroke}%
\end{pgfscope}%
\begin{pgfscope}%
\pgfpathrectangle{\pgfqpoint{0.765000in}{0.660000in}}{\pgfqpoint{4.620000in}{4.620000in}}%
\pgfusepath{clip}%
\pgfsetbuttcap%
\pgfsetroundjoin%
\definecolor{currentfill}{rgb}{1.000000,0.576471,0.309804}%
\pgfsetfillcolor{currentfill}%
\pgfsetlinewidth{1.003750pt}%
\definecolor{currentstroke}{rgb}{1.000000,0.576471,0.309804}%
\pgfsetstrokecolor{currentstroke}%
\pgfsetdash{}{0pt}%
\pgfsys@defobject{currentmarker}{\pgfqpoint{-0.033333in}{-0.033333in}}{\pgfqpoint{0.033333in}{0.033333in}}{%
\pgfpathmoveto{\pgfqpoint{0.000000in}{-0.033333in}}%
\pgfpathcurveto{\pgfqpoint{0.008840in}{-0.033333in}}{\pgfqpoint{0.017319in}{-0.029821in}}{\pgfqpoint{0.023570in}{-0.023570in}}%
\pgfpathcurveto{\pgfqpoint{0.029821in}{-0.017319in}}{\pgfqpoint{0.033333in}{-0.008840in}}{\pgfqpoint{0.033333in}{0.000000in}}%
\pgfpathcurveto{\pgfqpoint{0.033333in}{0.008840in}}{\pgfqpoint{0.029821in}{0.017319in}}{\pgfqpoint{0.023570in}{0.023570in}}%
\pgfpathcurveto{\pgfqpoint{0.017319in}{0.029821in}}{\pgfqpoint{0.008840in}{0.033333in}}{\pgfqpoint{0.000000in}{0.033333in}}%
\pgfpathcurveto{\pgfqpoint{-0.008840in}{0.033333in}}{\pgfqpoint{-0.017319in}{0.029821in}}{\pgfqpoint{-0.023570in}{0.023570in}}%
\pgfpathcurveto{\pgfqpoint{-0.029821in}{0.017319in}}{\pgfqpoint{-0.033333in}{0.008840in}}{\pgfqpoint{-0.033333in}{0.000000in}}%
\pgfpathcurveto{\pgfqpoint{-0.033333in}{-0.008840in}}{\pgfqpoint{-0.029821in}{-0.017319in}}{\pgfqpoint{-0.023570in}{-0.023570in}}%
\pgfpathcurveto{\pgfqpoint{-0.017319in}{-0.029821in}}{\pgfqpoint{-0.008840in}{-0.033333in}}{\pgfqpoint{0.000000in}{-0.033333in}}%
\pgfpathlineto{\pgfqpoint{0.000000in}{-0.033333in}}%
\pgfpathclose%
\pgfusepath{stroke,fill}%
}%
\begin{pgfscope}%
\pgfsys@transformshift{2.590185in}{2.863295in}%
\pgfsys@useobject{currentmarker}{}%
\end{pgfscope}%
\end{pgfscope}%
\begin{pgfscope}%
\pgfpathrectangle{\pgfqpoint{0.765000in}{0.660000in}}{\pgfqpoint{4.620000in}{4.620000in}}%
\pgfusepath{clip}%
\pgfsetrectcap%
\pgfsetroundjoin%
\pgfsetlinewidth{1.204500pt}%
\definecolor{currentstroke}{rgb}{0.176471,0.192157,0.258824}%
\pgfsetstrokecolor{currentstroke}%
\pgfsetdash{}{0pt}%
\pgfpathmoveto{\pgfqpoint{2.498322in}{3.973291in}}%
\pgfusepath{stroke}%
\end{pgfscope}%
\begin{pgfscope}%
\pgfpathrectangle{\pgfqpoint{0.765000in}{0.660000in}}{\pgfqpoint{4.620000in}{4.620000in}}%
\pgfusepath{clip}%
\pgfsetbuttcap%
\pgfsetmiterjoin%
\definecolor{currentfill}{rgb}{0.176471,0.192157,0.258824}%
\pgfsetfillcolor{currentfill}%
\pgfsetlinewidth{1.003750pt}%
\definecolor{currentstroke}{rgb}{0.176471,0.192157,0.258824}%
\pgfsetstrokecolor{currentstroke}%
\pgfsetdash{}{0pt}%
\pgfsys@defobject{currentmarker}{\pgfqpoint{-0.033333in}{-0.033333in}}{\pgfqpoint{0.033333in}{0.033333in}}{%
\pgfpathmoveto{\pgfqpoint{-0.000000in}{-0.033333in}}%
\pgfpathlineto{\pgfqpoint{0.033333in}{0.033333in}}%
\pgfpathlineto{\pgfqpoint{-0.033333in}{0.033333in}}%
\pgfpathlineto{\pgfqpoint{-0.000000in}{-0.033333in}}%
\pgfpathclose%
\pgfusepath{stroke,fill}%
}%
\begin{pgfscope}%
\pgfsys@transformshift{2.498322in}{3.973291in}%
\pgfsys@useobject{currentmarker}{}%
\end{pgfscope}%
\end{pgfscope}%
\begin{pgfscope}%
\pgfpathrectangle{\pgfqpoint{0.765000in}{0.660000in}}{\pgfqpoint{4.620000in}{4.620000in}}%
\pgfusepath{clip}%
\pgfsetrectcap%
\pgfsetroundjoin%
\pgfsetlinewidth{1.204500pt}%
\definecolor{currentstroke}{rgb}{0.176471,0.192157,0.258824}%
\pgfsetstrokecolor{currentstroke}%
\pgfsetdash{}{0pt}%
\pgfpathmoveto{\pgfqpoint{3.219806in}{4.462588in}}%
\pgfusepath{stroke}%
\end{pgfscope}%
\begin{pgfscope}%
\pgfpathrectangle{\pgfqpoint{0.765000in}{0.660000in}}{\pgfqpoint{4.620000in}{4.620000in}}%
\pgfusepath{clip}%
\pgfsetbuttcap%
\pgfsetmiterjoin%
\definecolor{currentfill}{rgb}{0.176471,0.192157,0.258824}%
\pgfsetfillcolor{currentfill}%
\pgfsetlinewidth{1.003750pt}%
\definecolor{currentstroke}{rgb}{0.176471,0.192157,0.258824}%
\pgfsetstrokecolor{currentstroke}%
\pgfsetdash{}{0pt}%
\pgfsys@defobject{currentmarker}{\pgfqpoint{-0.033333in}{-0.033333in}}{\pgfqpoint{0.033333in}{0.033333in}}{%
\pgfpathmoveto{\pgfqpoint{-0.000000in}{-0.033333in}}%
\pgfpathlineto{\pgfqpoint{0.033333in}{0.033333in}}%
\pgfpathlineto{\pgfqpoint{-0.033333in}{0.033333in}}%
\pgfpathlineto{\pgfqpoint{-0.000000in}{-0.033333in}}%
\pgfpathclose%
\pgfusepath{stroke,fill}%
}%
\begin{pgfscope}%
\pgfsys@transformshift{3.219806in}{4.462588in}%
\pgfsys@useobject{currentmarker}{}%
\end{pgfscope}%
\end{pgfscope}%
\begin{pgfscope}%
\pgfpathrectangle{\pgfqpoint{0.765000in}{0.660000in}}{\pgfqpoint{4.620000in}{4.620000in}}%
\pgfusepath{clip}%
\pgfsetrectcap%
\pgfsetroundjoin%
\pgfsetlinewidth{1.204500pt}%
\definecolor{currentstroke}{rgb}{0.176471,0.192157,0.258824}%
\pgfsetstrokecolor{currentstroke}%
\pgfsetdash{}{0pt}%
\pgfpathmoveto{\pgfqpoint{3.606810in}{4.049879in}}%
\pgfusepath{stroke}%
\end{pgfscope}%
\begin{pgfscope}%
\pgfpathrectangle{\pgfqpoint{0.765000in}{0.660000in}}{\pgfqpoint{4.620000in}{4.620000in}}%
\pgfusepath{clip}%
\pgfsetbuttcap%
\pgfsetmiterjoin%
\definecolor{currentfill}{rgb}{0.176471,0.192157,0.258824}%
\pgfsetfillcolor{currentfill}%
\pgfsetlinewidth{1.003750pt}%
\definecolor{currentstroke}{rgb}{0.176471,0.192157,0.258824}%
\pgfsetstrokecolor{currentstroke}%
\pgfsetdash{}{0pt}%
\pgfsys@defobject{currentmarker}{\pgfqpoint{-0.033333in}{-0.033333in}}{\pgfqpoint{0.033333in}{0.033333in}}{%
\pgfpathmoveto{\pgfqpoint{-0.000000in}{-0.033333in}}%
\pgfpathlineto{\pgfqpoint{0.033333in}{0.033333in}}%
\pgfpathlineto{\pgfqpoint{-0.033333in}{0.033333in}}%
\pgfpathlineto{\pgfqpoint{-0.000000in}{-0.033333in}}%
\pgfpathclose%
\pgfusepath{stroke,fill}%
}%
\begin{pgfscope}%
\pgfsys@transformshift{3.606810in}{4.049879in}%
\pgfsys@useobject{currentmarker}{}%
\end{pgfscope}%
\end{pgfscope}%
\begin{pgfscope}%
\pgfpathrectangle{\pgfqpoint{0.765000in}{0.660000in}}{\pgfqpoint{4.620000in}{4.620000in}}%
\pgfusepath{clip}%
\pgfsetrectcap%
\pgfsetroundjoin%
\pgfsetlinewidth{1.204500pt}%
\definecolor{currentstroke}{rgb}{0.176471,0.192157,0.258824}%
\pgfsetstrokecolor{currentstroke}%
\pgfsetdash{}{0pt}%
\pgfpathmoveto{\pgfqpoint{3.532705in}{4.371107in}}%
\pgfusepath{stroke}%
\end{pgfscope}%
\begin{pgfscope}%
\pgfpathrectangle{\pgfqpoint{0.765000in}{0.660000in}}{\pgfqpoint{4.620000in}{4.620000in}}%
\pgfusepath{clip}%
\pgfsetbuttcap%
\pgfsetmiterjoin%
\definecolor{currentfill}{rgb}{0.176471,0.192157,0.258824}%
\pgfsetfillcolor{currentfill}%
\pgfsetlinewidth{1.003750pt}%
\definecolor{currentstroke}{rgb}{0.176471,0.192157,0.258824}%
\pgfsetstrokecolor{currentstroke}%
\pgfsetdash{}{0pt}%
\pgfsys@defobject{currentmarker}{\pgfqpoint{-0.033333in}{-0.033333in}}{\pgfqpoint{0.033333in}{0.033333in}}{%
\pgfpathmoveto{\pgfqpoint{-0.000000in}{-0.033333in}}%
\pgfpathlineto{\pgfqpoint{0.033333in}{0.033333in}}%
\pgfpathlineto{\pgfqpoint{-0.033333in}{0.033333in}}%
\pgfpathlineto{\pgfqpoint{-0.000000in}{-0.033333in}}%
\pgfpathclose%
\pgfusepath{stroke,fill}%
}%
\begin{pgfscope}%
\pgfsys@transformshift{3.532705in}{4.371107in}%
\pgfsys@useobject{currentmarker}{}%
\end{pgfscope}%
\end{pgfscope}%
\begin{pgfscope}%
\pgfpathrectangle{\pgfqpoint{0.765000in}{0.660000in}}{\pgfqpoint{4.620000in}{4.620000in}}%
\pgfusepath{clip}%
\pgfsetrectcap%
\pgfsetroundjoin%
\pgfsetlinewidth{1.204500pt}%
\definecolor{currentstroke}{rgb}{0.176471,0.192157,0.258824}%
\pgfsetstrokecolor{currentstroke}%
\pgfsetdash{}{0pt}%
\pgfpathmoveto{\pgfqpoint{2.376526in}{4.507564in}}%
\pgfusepath{stroke}%
\end{pgfscope}%
\begin{pgfscope}%
\pgfpathrectangle{\pgfqpoint{0.765000in}{0.660000in}}{\pgfqpoint{4.620000in}{4.620000in}}%
\pgfusepath{clip}%
\pgfsetbuttcap%
\pgfsetmiterjoin%
\definecolor{currentfill}{rgb}{0.176471,0.192157,0.258824}%
\pgfsetfillcolor{currentfill}%
\pgfsetlinewidth{1.003750pt}%
\definecolor{currentstroke}{rgb}{0.176471,0.192157,0.258824}%
\pgfsetstrokecolor{currentstroke}%
\pgfsetdash{}{0pt}%
\pgfsys@defobject{currentmarker}{\pgfqpoint{-0.033333in}{-0.033333in}}{\pgfqpoint{0.033333in}{0.033333in}}{%
\pgfpathmoveto{\pgfqpoint{-0.000000in}{-0.033333in}}%
\pgfpathlineto{\pgfqpoint{0.033333in}{0.033333in}}%
\pgfpathlineto{\pgfqpoint{-0.033333in}{0.033333in}}%
\pgfpathlineto{\pgfqpoint{-0.000000in}{-0.033333in}}%
\pgfpathclose%
\pgfusepath{stroke,fill}%
}%
\begin{pgfscope}%
\pgfsys@transformshift{2.376526in}{4.507564in}%
\pgfsys@useobject{currentmarker}{}%
\end{pgfscope}%
\end{pgfscope}%
\begin{pgfscope}%
\pgfpathrectangle{\pgfqpoint{0.765000in}{0.660000in}}{\pgfqpoint{4.620000in}{4.620000in}}%
\pgfusepath{clip}%
\pgfsetrectcap%
\pgfsetroundjoin%
\pgfsetlinewidth{1.204500pt}%
\definecolor{currentstroke}{rgb}{0.176471,0.192157,0.258824}%
\pgfsetstrokecolor{currentstroke}%
\pgfsetdash{}{0pt}%
\pgfpathmoveto{\pgfqpoint{2.747126in}{3.810354in}}%
\pgfusepath{stroke}%
\end{pgfscope}%
\begin{pgfscope}%
\pgfpathrectangle{\pgfqpoint{0.765000in}{0.660000in}}{\pgfqpoint{4.620000in}{4.620000in}}%
\pgfusepath{clip}%
\pgfsetbuttcap%
\pgfsetmiterjoin%
\definecolor{currentfill}{rgb}{0.176471,0.192157,0.258824}%
\pgfsetfillcolor{currentfill}%
\pgfsetlinewidth{1.003750pt}%
\definecolor{currentstroke}{rgb}{0.176471,0.192157,0.258824}%
\pgfsetstrokecolor{currentstroke}%
\pgfsetdash{}{0pt}%
\pgfsys@defobject{currentmarker}{\pgfqpoint{-0.033333in}{-0.033333in}}{\pgfqpoint{0.033333in}{0.033333in}}{%
\pgfpathmoveto{\pgfqpoint{-0.000000in}{-0.033333in}}%
\pgfpathlineto{\pgfqpoint{0.033333in}{0.033333in}}%
\pgfpathlineto{\pgfqpoint{-0.033333in}{0.033333in}}%
\pgfpathlineto{\pgfqpoint{-0.000000in}{-0.033333in}}%
\pgfpathclose%
\pgfusepath{stroke,fill}%
}%
\begin{pgfscope}%
\pgfsys@transformshift{2.747126in}{3.810354in}%
\pgfsys@useobject{currentmarker}{}%
\end{pgfscope}%
\end{pgfscope}%
\begin{pgfscope}%
\pgfpathrectangle{\pgfqpoint{0.765000in}{0.660000in}}{\pgfqpoint{4.620000in}{4.620000in}}%
\pgfusepath{clip}%
\pgfsetrectcap%
\pgfsetroundjoin%
\pgfsetlinewidth{1.204500pt}%
\definecolor{currentstroke}{rgb}{0.176471,0.192157,0.258824}%
\pgfsetstrokecolor{currentstroke}%
\pgfsetdash{}{0pt}%
\pgfpathmoveto{\pgfqpoint{3.247459in}{4.050286in}}%
\pgfusepath{stroke}%
\end{pgfscope}%
\begin{pgfscope}%
\pgfpathrectangle{\pgfqpoint{0.765000in}{0.660000in}}{\pgfqpoint{4.620000in}{4.620000in}}%
\pgfusepath{clip}%
\pgfsetbuttcap%
\pgfsetmiterjoin%
\definecolor{currentfill}{rgb}{0.176471,0.192157,0.258824}%
\pgfsetfillcolor{currentfill}%
\pgfsetlinewidth{1.003750pt}%
\definecolor{currentstroke}{rgb}{0.176471,0.192157,0.258824}%
\pgfsetstrokecolor{currentstroke}%
\pgfsetdash{}{0pt}%
\pgfsys@defobject{currentmarker}{\pgfqpoint{-0.033333in}{-0.033333in}}{\pgfqpoint{0.033333in}{0.033333in}}{%
\pgfpathmoveto{\pgfqpoint{-0.000000in}{-0.033333in}}%
\pgfpathlineto{\pgfqpoint{0.033333in}{0.033333in}}%
\pgfpathlineto{\pgfqpoint{-0.033333in}{0.033333in}}%
\pgfpathlineto{\pgfqpoint{-0.000000in}{-0.033333in}}%
\pgfpathclose%
\pgfusepath{stroke,fill}%
}%
\begin{pgfscope}%
\pgfsys@transformshift{3.247459in}{4.050286in}%
\pgfsys@useobject{currentmarker}{}%
\end{pgfscope}%
\end{pgfscope}%
\begin{pgfscope}%
\pgfpathrectangle{\pgfqpoint{0.765000in}{0.660000in}}{\pgfqpoint{4.620000in}{4.620000in}}%
\pgfusepath{clip}%
\pgfsetrectcap%
\pgfsetroundjoin%
\pgfsetlinewidth{1.204500pt}%
\definecolor{currentstroke}{rgb}{0.176471,0.192157,0.258824}%
\pgfsetstrokecolor{currentstroke}%
\pgfsetdash{}{0pt}%
\pgfpathmoveto{\pgfqpoint{3.631062in}{4.142668in}}%
\pgfusepath{stroke}%
\end{pgfscope}%
\begin{pgfscope}%
\pgfpathrectangle{\pgfqpoint{0.765000in}{0.660000in}}{\pgfqpoint{4.620000in}{4.620000in}}%
\pgfusepath{clip}%
\pgfsetbuttcap%
\pgfsetmiterjoin%
\definecolor{currentfill}{rgb}{0.176471,0.192157,0.258824}%
\pgfsetfillcolor{currentfill}%
\pgfsetlinewidth{1.003750pt}%
\definecolor{currentstroke}{rgb}{0.176471,0.192157,0.258824}%
\pgfsetstrokecolor{currentstroke}%
\pgfsetdash{}{0pt}%
\pgfsys@defobject{currentmarker}{\pgfqpoint{-0.033333in}{-0.033333in}}{\pgfqpoint{0.033333in}{0.033333in}}{%
\pgfpathmoveto{\pgfqpoint{-0.000000in}{-0.033333in}}%
\pgfpathlineto{\pgfqpoint{0.033333in}{0.033333in}}%
\pgfpathlineto{\pgfqpoint{-0.033333in}{0.033333in}}%
\pgfpathlineto{\pgfqpoint{-0.000000in}{-0.033333in}}%
\pgfpathclose%
\pgfusepath{stroke,fill}%
}%
\begin{pgfscope}%
\pgfsys@transformshift{3.631062in}{4.142668in}%
\pgfsys@useobject{currentmarker}{}%
\end{pgfscope}%
\end{pgfscope}%
\begin{pgfscope}%
\pgfpathrectangle{\pgfqpoint{0.765000in}{0.660000in}}{\pgfqpoint{4.620000in}{4.620000in}}%
\pgfusepath{clip}%
\pgfsetrectcap%
\pgfsetroundjoin%
\pgfsetlinewidth{1.204500pt}%
\definecolor{currentstroke}{rgb}{0.176471,0.192157,0.258824}%
\pgfsetstrokecolor{currentstroke}%
\pgfsetdash{}{0pt}%
\pgfpathmoveto{\pgfqpoint{2.118495in}{4.321693in}}%
\pgfusepath{stroke}%
\end{pgfscope}%
\begin{pgfscope}%
\pgfpathrectangle{\pgfqpoint{0.765000in}{0.660000in}}{\pgfqpoint{4.620000in}{4.620000in}}%
\pgfusepath{clip}%
\pgfsetbuttcap%
\pgfsetmiterjoin%
\definecolor{currentfill}{rgb}{0.176471,0.192157,0.258824}%
\pgfsetfillcolor{currentfill}%
\pgfsetlinewidth{1.003750pt}%
\definecolor{currentstroke}{rgb}{0.176471,0.192157,0.258824}%
\pgfsetstrokecolor{currentstroke}%
\pgfsetdash{}{0pt}%
\pgfsys@defobject{currentmarker}{\pgfqpoint{-0.033333in}{-0.033333in}}{\pgfqpoint{0.033333in}{0.033333in}}{%
\pgfpathmoveto{\pgfqpoint{-0.000000in}{-0.033333in}}%
\pgfpathlineto{\pgfqpoint{0.033333in}{0.033333in}}%
\pgfpathlineto{\pgfqpoint{-0.033333in}{0.033333in}}%
\pgfpathlineto{\pgfqpoint{-0.000000in}{-0.033333in}}%
\pgfpathclose%
\pgfusepath{stroke,fill}%
}%
\begin{pgfscope}%
\pgfsys@transformshift{2.118495in}{4.321693in}%
\pgfsys@useobject{currentmarker}{}%
\end{pgfscope}%
\end{pgfscope}%
\begin{pgfscope}%
\pgfpathrectangle{\pgfqpoint{0.765000in}{0.660000in}}{\pgfqpoint{4.620000in}{4.620000in}}%
\pgfusepath{clip}%
\pgfsetrectcap%
\pgfsetroundjoin%
\pgfsetlinewidth{1.204500pt}%
\definecolor{currentstroke}{rgb}{0.176471,0.192157,0.258824}%
\pgfsetstrokecolor{currentstroke}%
\pgfsetdash{}{0pt}%
\pgfpathmoveto{\pgfqpoint{3.456730in}{4.459949in}}%
\pgfusepath{stroke}%
\end{pgfscope}%
\begin{pgfscope}%
\pgfpathrectangle{\pgfqpoint{0.765000in}{0.660000in}}{\pgfqpoint{4.620000in}{4.620000in}}%
\pgfusepath{clip}%
\pgfsetbuttcap%
\pgfsetmiterjoin%
\definecolor{currentfill}{rgb}{0.176471,0.192157,0.258824}%
\pgfsetfillcolor{currentfill}%
\pgfsetlinewidth{1.003750pt}%
\definecolor{currentstroke}{rgb}{0.176471,0.192157,0.258824}%
\pgfsetstrokecolor{currentstroke}%
\pgfsetdash{}{0pt}%
\pgfsys@defobject{currentmarker}{\pgfqpoint{-0.033333in}{-0.033333in}}{\pgfqpoint{0.033333in}{0.033333in}}{%
\pgfpathmoveto{\pgfqpoint{-0.000000in}{-0.033333in}}%
\pgfpathlineto{\pgfqpoint{0.033333in}{0.033333in}}%
\pgfpathlineto{\pgfqpoint{-0.033333in}{0.033333in}}%
\pgfpathlineto{\pgfqpoint{-0.000000in}{-0.033333in}}%
\pgfpathclose%
\pgfusepath{stroke,fill}%
}%
\begin{pgfscope}%
\pgfsys@transformshift{3.456730in}{4.459949in}%
\pgfsys@useobject{currentmarker}{}%
\end{pgfscope}%
\end{pgfscope}%
\begin{pgfscope}%
\pgfpathrectangle{\pgfqpoint{0.765000in}{0.660000in}}{\pgfqpoint{4.620000in}{4.620000in}}%
\pgfusepath{clip}%
\pgfsetrectcap%
\pgfsetroundjoin%
\pgfsetlinewidth{1.204500pt}%
\definecolor{currentstroke}{rgb}{0.176471,0.192157,0.258824}%
\pgfsetstrokecolor{currentstroke}%
\pgfsetdash{}{0pt}%
\pgfpathmoveto{\pgfqpoint{4.250555in}{4.304526in}}%
\pgfusepath{stroke}%
\end{pgfscope}%
\begin{pgfscope}%
\pgfpathrectangle{\pgfqpoint{0.765000in}{0.660000in}}{\pgfqpoint{4.620000in}{4.620000in}}%
\pgfusepath{clip}%
\pgfsetbuttcap%
\pgfsetmiterjoin%
\definecolor{currentfill}{rgb}{0.176471,0.192157,0.258824}%
\pgfsetfillcolor{currentfill}%
\pgfsetlinewidth{1.003750pt}%
\definecolor{currentstroke}{rgb}{0.176471,0.192157,0.258824}%
\pgfsetstrokecolor{currentstroke}%
\pgfsetdash{}{0pt}%
\pgfsys@defobject{currentmarker}{\pgfqpoint{-0.033333in}{-0.033333in}}{\pgfqpoint{0.033333in}{0.033333in}}{%
\pgfpathmoveto{\pgfqpoint{-0.000000in}{-0.033333in}}%
\pgfpathlineto{\pgfqpoint{0.033333in}{0.033333in}}%
\pgfpathlineto{\pgfqpoint{-0.033333in}{0.033333in}}%
\pgfpathlineto{\pgfqpoint{-0.000000in}{-0.033333in}}%
\pgfpathclose%
\pgfusepath{stroke,fill}%
}%
\begin{pgfscope}%
\pgfsys@transformshift{4.250555in}{4.304526in}%
\pgfsys@useobject{currentmarker}{}%
\end{pgfscope}%
\end{pgfscope}%
\begin{pgfscope}%
\pgfpathrectangle{\pgfqpoint{0.765000in}{0.660000in}}{\pgfqpoint{4.620000in}{4.620000in}}%
\pgfusepath{clip}%
\pgfsetrectcap%
\pgfsetroundjoin%
\pgfsetlinewidth{1.204500pt}%
\definecolor{currentstroke}{rgb}{0.176471,0.192157,0.258824}%
\pgfsetstrokecolor{currentstroke}%
\pgfsetdash{}{0pt}%
\pgfpathmoveto{\pgfqpoint{3.789146in}{4.423508in}}%
\pgfusepath{stroke}%
\end{pgfscope}%
\begin{pgfscope}%
\pgfpathrectangle{\pgfqpoint{0.765000in}{0.660000in}}{\pgfqpoint{4.620000in}{4.620000in}}%
\pgfusepath{clip}%
\pgfsetbuttcap%
\pgfsetmiterjoin%
\definecolor{currentfill}{rgb}{0.176471,0.192157,0.258824}%
\pgfsetfillcolor{currentfill}%
\pgfsetlinewidth{1.003750pt}%
\definecolor{currentstroke}{rgb}{0.176471,0.192157,0.258824}%
\pgfsetstrokecolor{currentstroke}%
\pgfsetdash{}{0pt}%
\pgfsys@defobject{currentmarker}{\pgfqpoint{-0.033333in}{-0.033333in}}{\pgfqpoint{0.033333in}{0.033333in}}{%
\pgfpathmoveto{\pgfqpoint{-0.000000in}{-0.033333in}}%
\pgfpathlineto{\pgfqpoint{0.033333in}{0.033333in}}%
\pgfpathlineto{\pgfqpoint{-0.033333in}{0.033333in}}%
\pgfpathlineto{\pgfqpoint{-0.000000in}{-0.033333in}}%
\pgfpathclose%
\pgfusepath{stroke,fill}%
}%
\begin{pgfscope}%
\pgfsys@transformshift{3.789146in}{4.423508in}%
\pgfsys@useobject{currentmarker}{}%
\end{pgfscope}%
\end{pgfscope}%
\begin{pgfscope}%
\pgfpathrectangle{\pgfqpoint{0.765000in}{0.660000in}}{\pgfqpoint{4.620000in}{4.620000in}}%
\pgfusepath{clip}%
\pgfsetrectcap%
\pgfsetroundjoin%
\pgfsetlinewidth{1.204500pt}%
\definecolor{currentstroke}{rgb}{0.176471,0.192157,0.258824}%
\pgfsetstrokecolor{currentstroke}%
\pgfsetdash{}{0pt}%
\pgfpathmoveto{\pgfqpoint{2.670443in}{4.549182in}}%
\pgfusepath{stroke}%
\end{pgfscope}%
\begin{pgfscope}%
\pgfpathrectangle{\pgfqpoint{0.765000in}{0.660000in}}{\pgfqpoint{4.620000in}{4.620000in}}%
\pgfusepath{clip}%
\pgfsetbuttcap%
\pgfsetmiterjoin%
\definecolor{currentfill}{rgb}{0.176471,0.192157,0.258824}%
\pgfsetfillcolor{currentfill}%
\pgfsetlinewidth{1.003750pt}%
\definecolor{currentstroke}{rgb}{0.176471,0.192157,0.258824}%
\pgfsetstrokecolor{currentstroke}%
\pgfsetdash{}{0pt}%
\pgfsys@defobject{currentmarker}{\pgfqpoint{-0.033333in}{-0.033333in}}{\pgfqpoint{0.033333in}{0.033333in}}{%
\pgfpathmoveto{\pgfqpoint{-0.000000in}{-0.033333in}}%
\pgfpathlineto{\pgfqpoint{0.033333in}{0.033333in}}%
\pgfpathlineto{\pgfqpoint{-0.033333in}{0.033333in}}%
\pgfpathlineto{\pgfqpoint{-0.000000in}{-0.033333in}}%
\pgfpathclose%
\pgfusepath{stroke,fill}%
}%
\begin{pgfscope}%
\pgfsys@transformshift{2.670443in}{4.549182in}%
\pgfsys@useobject{currentmarker}{}%
\end{pgfscope}%
\end{pgfscope}%
\begin{pgfscope}%
\pgfpathrectangle{\pgfqpoint{0.765000in}{0.660000in}}{\pgfqpoint{4.620000in}{4.620000in}}%
\pgfusepath{clip}%
\pgfsetrectcap%
\pgfsetroundjoin%
\pgfsetlinewidth{1.204500pt}%
\definecolor{currentstroke}{rgb}{0.176471,0.192157,0.258824}%
\pgfsetstrokecolor{currentstroke}%
\pgfsetdash{}{0pt}%
\pgfpathmoveto{\pgfqpoint{3.218222in}{3.961982in}}%
\pgfusepath{stroke}%
\end{pgfscope}%
\begin{pgfscope}%
\pgfpathrectangle{\pgfqpoint{0.765000in}{0.660000in}}{\pgfqpoint{4.620000in}{4.620000in}}%
\pgfusepath{clip}%
\pgfsetbuttcap%
\pgfsetmiterjoin%
\definecolor{currentfill}{rgb}{0.176471,0.192157,0.258824}%
\pgfsetfillcolor{currentfill}%
\pgfsetlinewidth{1.003750pt}%
\definecolor{currentstroke}{rgb}{0.176471,0.192157,0.258824}%
\pgfsetstrokecolor{currentstroke}%
\pgfsetdash{}{0pt}%
\pgfsys@defobject{currentmarker}{\pgfqpoint{-0.033333in}{-0.033333in}}{\pgfqpoint{0.033333in}{0.033333in}}{%
\pgfpathmoveto{\pgfqpoint{-0.000000in}{-0.033333in}}%
\pgfpathlineto{\pgfqpoint{0.033333in}{0.033333in}}%
\pgfpathlineto{\pgfqpoint{-0.033333in}{0.033333in}}%
\pgfpathlineto{\pgfqpoint{-0.000000in}{-0.033333in}}%
\pgfpathclose%
\pgfusepath{stroke,fill}%
}%
\begin{pgfscope}%
\pgfsys@transformshift{3.218222in}{3.961982in}%
\pgfsys@useobject{currentmarker}{}%
\end{pgfscope}%
\end{pgfscope}%
\begin{pgfscope}%
\pgfpathrectangle{\pgfqpoint{0.765000in}{0.660000in}}{\pgfqpoint{4.620000in}{4.620000in}}%
\pgfusepath{clip}%
\pgfsetrectcap%
\pgfsetroundjoin%
\pgfsetlinewidth{1.204500pt}%
\definecolor{currentstroke}{rgb}{0.176471,0.192157,0.258824}%
\pgfsetstrokecolor{currentstroke}%
\pgfsetdash{}{0pt}%
\pgfpathmoveto{\pgfqpoint{2.473731in}{4.154277in}}%
\pgfusepath{stroke}%
\end{pgfscope}%
\begin{pgfscope}%
\pgfpathrectangle{\pgfqpoint{0.765000in}{0.660000in}}{\pgfqpoint{4.620000in}{4.620000in}}%
\pgfusepath{clip}%
\pgfsetbuttcap%
\pgfsetmiterjoin%
\definecolor{currentfill}{rgb}{0.176471,0.192157,0.258824}%
\pgfsetfillcolor{currentfill}%
\pgfsetlinewidth{1.003750pt}%
\definecolor{currentstroke}{rgb}{0.176471,0.192157,0.258824}%
\pgfsetstrokecolor{currentstroke}%
\pgfsetdash{}{0pt}%
\pgfsys@defobject{currentmarker}{\pgfqpoint{-0.033333in}{-0.033333in}}{\pgfqpoint{0.033333in}{0.033333in}}{%
\pgfpathmoveto{\pgfqpoint{-0.000000in}{-0.033333in}}%
\pgfpathlineto{\pgfqpoint{0.033333in}{0.033333in}}%
\pgfpathlineto{\pgfqpoint{-0.033333in}{0.033333in}}%
\pgfpathlineto{\pgfqpoint{-0.000000in}{-0.033333in}}%
\pgfpathclose%
\pgfusepath{stroke,fill}%
}%
\begin{pgfscope}%
\pgfsys@transformshift{2.473731in}{4.154277in}%
\pgfsys@useobject{currentmarker}{}%
\end{pgfscope}%
\end{pgfscope}%
\begin{pgfscope}%
\pgfpathrectangle{\pgfqpoint{0.765000in}{0.660000in}}{\pgfqpoint{4.620000in}{4.620000in}}%
\pgfusepath{clip}%
\pgfsetrectcap%
\pgfsetroundjoin%
\pgfsetlinewidth{1.204500pt}%
\definecolor{currentstroke}{rgb}{0.176471,0.192157,0.258824}%
\pgfsetstrokecolor{currentstroke}%
\pgfsetdash{}{0pt}%
\pgfpathmoveto{\pgfqpoint{3.089250in}{3.627594in}}%
\pgfusepath{stroke}%
\end{pgfscope}%
\begin{pgfscope}%
\pgfpathrectangle{\pgfqpoint{0.765000in}{0.660000in}}{\pgfqpoint{4.620000in}{4.620000in}}%
\pgfusepath{clip}%
\pgfsetbuttcap%
\pgfsetmiterjoin%
\definecolor{currentfill}{rgb}{0.176471,0.192157,0.258824}%
\pgfsetfillcolor{currentfill}%
\pgfsetlinewidth{1.003750pt}%
\definecolor{currentstroke}{rgb}{0.176471,0.192157,0.258824}%
\pgfsetstrokecolor{currentstroke}%
\pgfsetdash{}{0pt}%
\pgfsys@defobject{currentmarker}{\pgfqpoint{-0.033333in}{-0.033333in}}{\pgfqpoint{0.033333in}{0.033333in}}{%
\pgfpathmoveto{\pgfqpoint{-0.000000in}{-0.033333in}}%
\pgfpathlineto{\pgfqpoint{0.033333in}{0.033333in}}%
\pgfpathlineto{\pgfqpoint{-0.033333in}{0.033333in}}%
\pgfpathlineto{\pgfqpoint{-0.000000in}{-0.033333in}}%
\pgfpathclose%
\pgfusepath{stroke,fill}%
}%
\begin{pgfscope}%
\pgfsys@transformshift{3.089250in}{3.627594in}%
\pgfsys@useobject{currentmarker}{}%
\end{pgfscope}%
\end{pgfscope}%
\begin{pgfscope}%
\pgfpathrectangle{\pgfqpoint{0.765000in}{0.660000in}}{\pgfqpoint{4.620000in}{4.620000in}}%
\pgfusepath{clip}%
\pgfsetrectcap%
\pgfsetroundjoin%
\pgfsetlinewidth{1.204500pt}%
\definecolor{currentstroke}{rgb}{0.176471,0.192157,0.258824}%
\pgfsetstrokecolor{currentstroke}%
\pgfsetdash{}{0pt}%
\pgfpathmoveto{\pgfqpoint{2.958065in}{4.767341in}}%
\pgfusepath{stroke}%
\end{pgfscope}%
\begin{pgfscope}%
\pgfpathrectangle{\pgfqpoint{0.765000in}{0.660000in}}{\pgfqpoint{4.620000in}{4.620000in}}%
\pgfusepath{clip}%
\pgfsetbuttcap%
\pgfsetmiterjoin%
\definecolor{currentfill}{rgb}{0.176471,0.192157,0.258824}%
\pgfsetfillcolor{currentfill}%
\pgfsetlinewidth{1.003750pt}%
\definecolor{currentstroke}{rgb}{0.176471,0.192157,0.258824}%
\pgfsetstrokecolor{currentstroke}%
\pgfsetdash{}{0pt}%
\pgfsys@defobject{currentmarker}{\pgfqpoint{-0.033333in}{-0.033333in}}{\pgfqpoint{0.033333in}{0.033333in}}{%
\pgfpathmoveto{\pgfqpoint{-0.000000in}{-0.033333in}}%
\pgfpathlineto{\pgfqpoint{0.033333in}{0.033333in}}%
\pgfpathlineto{\pgfqpoint{-0.033333in}{0.033333in}}%
\pgfpathlineto{\pgfqpoint{-0.000000in}{-0.033333in}}%
\pgfpathclose%
\pgfusepath{stroke,fill}%
}%
\begin{pgfscope}%
\pgfsys@transformshift{2.958065in}{4.767341in}%
\pgfsys@useobject{currentmarker}{}%
\end{pgfscope}%
\end{pgfscope}%
\begin{pgfscope}%
\pgfpathrectangle{\pgfqpoint{0.765000in}{0.660000in}}{\pgfqpoint{4.620000in}{4.620000in}}%
\pgfusepath{clip}%
\pgfsetrectcap%
\pgfsetroundjoin%
\pgfsetlinewidth{1.204500pt}%
\definecolor{currentstroke}{rgb}{0.176471,0.192157,0.258824}%
\pgfsetstrokecolor{currentstroke}%
\pgfsetdash{}{0pt}%
\pgfpathmoveto{\pgfqpoint{3.253895in}{4.654662in}}%
\pgfusepath{stroke}%
\end{pgfscope}%
\begin{pgfscope}%
\pgfpathrectangle{\pgfqpoint{0.765000in}{0.660000in}}{\pgfqpoint{4.620000in}{4.620000in}}%
\pgfusepath{clip}%
\pgfsetbuttcap%
\pgfsetmiterjoin%
\definecolor{currentfill}{rgb}{0.176471,0.192157,0.258824}%
\pgfsetfillcolor{currentfill}%
\pgfsetlinewidth{1.003750pt}%
\definecolor{currentstroke}{rgb}{0.176471,0.192157,0.258824}%
\pgfsetstrokecolor{currentstroke}%
\pgfsetdash{}{0pt}%
\pgfsys@defobject{currentmarker}{\pgfqpoint{-0.033333in}{-0.033333in}}{\pgfqpoint{0.033333in}{0.033333in}}{%
\pgfpathmoveto{\pgfqpoint{-0.000000in}{-0.033333in}}%
\pgfpathlineto{\pgfqpoint{0.033333in}{0.033333in}}%
\pgfpathlineto{\pgfqpoint{-0.033333in}{0.033333in}}%
\pgfpathlineto{\pgfqpoint{-0.000000in}{-0.033333in}}%
\pgfpathclose%
\pgfusepath{stroke,fill}%
}%
\begin{pgfscope}%
\pgfsys@transformshift{3.253895in}{4.654662in}%
\pgfsys@useobject{currentmarker}{}%
\end{pgfscope}%
\end{pgfscope}%
\begin{pgfscope}%
\pgfpathrectangle{\pgfqpoint{0.765000in}{0.660000in}}{\pgfqpoint{4.620000in}{4.620000in}}%
\pgfusepath{clip}%
\pgfsetrectcap%
\pgfsetroundjoin%
\pgfsetlinewidth{1.204500pt}%
\definecolor{currentstroke}{rgb}{0.176471,0.192157,0.258824}%
\pgfsetstrokecolor{currentstroke}%
\pgfsetdash{}{0pt}%
\pgfpathmoveto{\pgfqpoint{2.552105in}{3.952109in}}%
\pgfusepath{stroke}%
\end{pgfscope}%
\begin{pgfscope}%
\pgfpathrectangle{\pgfqpoint{0.765000in}{0.660000in}}{\pgfqpoint{4.620000in}{4.620000in}}%
\pgfusepath{clip}%
\pgfsetbuttcap%
\pgfsetmiterjoin%
\definecolor{currentfill}{rgb}{0.176471,0.192157,0.258824}%
\pgfsetfillcolor{currentfill}%
\pgfsetlinewidth{1.003750pt}%
\definecolor{currentstroke}{rgb}{0.176471,0.192157,0.258824}%
\pgfsetstrokecolor{currentstroke}%
\pgfsetdash{}{0pt}%
\pgfsys@defobject{currentmarker}{\pgfqpoint{-0.033333in}{-0.033333in}}{\pgfqpoint{0.033333in}{0.033333in}}{%
\pgfpathmoveto{\pgfqpoint{-0.000000in}{-0.033333in}}%
\pgfpathlineto{\pgfqpoint{0.033333in}{0.033333in}}%
\pgfpathlineto{\pgfqpoint{-0.033333in}{0.033333in}}%
\pgfpathlineto{\pgfqpoint{-0.000000in}{-0.033333in}}%
\pgfpathclose%
\pgfusepath{stroke,fill}%
}%
\begin{pgfscope}%
\pgfsys@transformshift{2.552105in}{3.952109in}%
\pgfsys@useobject{currentmarker}{}%
\end{pgfscope}%
\end{pgfscope}%
\begin{pgfscope}%
\pgfpathrectangle{\pgfqpoint{0.765000in}{0.660000in}}{\pgfqpoint{4.620000in}{4.620000in}}%
\pgfusepath{clip}%
\pgfsetrectcap%
\pgfsetroundjoin%
\pgfsetlinewidth{1.204500pt}%
\definecolor{currentstroke}{rgb}{0.176471,0.192157,0.258824}%
\pgfsetstrokecolor{currentstroke}%
\pgfsetdash{}{0pt}%
\pgfpathmoveto{\pgfqpoint{3.072824in}{4.032927in}}%
\pgfusepath{stroke}%
\end{pgfscope}%
\begin{pgfscope}%
\pgfpathrectangle{\pgfqpoint{0.765000in}{0.660000in}}{\pgfqpoint{4.620000in}{4.620000in}}%
\pgfusepath{clip}%
\pgfsetbuttcap%
\pgfsetmiterjoin%
\definecolor{currentfill}{rgb}{0.176471,0.192157,0.258824}%
\pgfsetfillcolor{currentfill}%
\pgfsetlinewidth{1.003750pt}%
\definecolor{currentstroke}{rgb}{0.176471,0.192157,0.258824}%
\pgfsetstrokecolor{currentstroke}%
\pgfsetdash{}{0pt}%
\pgfsys@defobject{currentmarker}{\pgfqpoint{-0.033333in}{-0.033333in}}{\pgfqpoint{0.033333in}{0.033333in}}{%
\pgfpathmoveto{\pgfqpoint{-0.000000in}{-0.033333in}}%
\pgfpathlineto{\pgfqpoint{0.033333in}{0.033333in}}%
\pgfpathlineto{\pgfqpoint{-0.033333in}{0.033333in}}%
\pgfpathlineto{\pgfqpoint{-0.000000in}{-0.033333in}}%
\pgfpathclose%
\pgfusepath{stroke,fill}%
}%
\begin{pgfscope}%
\pgfsys@transformshift{3.072824in}{4.032927in}%
\pgfsys@useobject{currentmarker}{}%
\end{pgfscope}%
\end{pgfscope}%
\begin{pgfscope}%
\pgfpathrectangle{\pgfqpoint{0.765000in}{0.660000in}}{\pgfqpoint{4.620000in}{4.620000in}}%
\pgfusepath{clip}%
\pgfsetrectcap%
\pgfsetroundjoin%
\pgfsetlinewidth{1.204500pt}%
\definecolor{currentstroke}{rgb}{0.176471,0.192157,0.258824}%
\pgfsetstrokecolor{currentstroke}%
\pgfsetdash{}{0pt}%
\pgfpathmoveto{\pgfqpoint{3.521540in}{4.147151in}}%
\pgfusepath{stroke}%
\end{pgfscope}%
\begin{pgfscope}%
\pgfpathrectangle{\pgfqpoint{0.765000in}{0.660000in}}{\pgfqpoint{4.620000in}{4.620000in}}%
\pgfusepath{clip}%
\pgfsetbuttcap%
\pgfsetmiterjoin%
\definecolor{currentfill}{rgb}{0.176471,0.192157,0.258824}%
\pgfsetfillcolor{currentfill}%
\pgfsetlinewidth{1.003750pt}%
\definecolor{currentstroke}{rgb}{0.176471,0.192157,0.258824}%
\pgfsetstrokecolor{currentstroke}%
\pgfsetdash{}{0pt}%
\pgfsys@defobject{currentmarker}{\pgfqpoint{-0.033333in}{-0.033333in}}{\pgfqpoint{0.033333in}{0.033333in}}{%
\pgfpathmoveto{\pgfqpoint{-0.000000in}{-0.033333in}}%
\pgfpathlineto{\pgfqpoint{0.033333in}{0.033333in}}%
\pgfpathlineto{\pgfqpoint{-0.033333in}{0.033333in}}%
\pgfpathlineto{\pgfqpoint{-0.000000in}{-0.033333in}}%
\pgfpathclose%
\pgfusepath{stroke,fill}%
}%
\begin{pgfscope}%
\pgfsys@transformshift{3.521540in}{4.147151in}%
\pgfsys@useobject{currentmarker}{}%
\end{pgfscope}%
\end{pgfscope}%
\begin{pgfscope}%
\pgfpathrectangle{\pgfqpoint{0.765000in}{0.660000in}}{\pgfqpoint{4.620000in}{4.620000in}}%
\pgfusepath{clip}%
\pgfsetrectcap%
\pgfsetroundjoin%
\pgfsetlinewidth{1.204500pt}%
\definecolor{currentstroke}{rgb}{0.176471,0.192157,0.258824}%
\pgfsetstrokecolor{currentstroke}%
\pgfsetdash{}{0pt}%
\pgfpathmoveto{\pgfqpoint{2.529524in}{4.529399in}}%
\pgfusepath{stroke}%
\end{pgfscope}%
\begin{pgfscope}%
\pgfpathrectangle{\pgfqpoint{0.765000in}{0.660000in}}{\pgfqpoint{4.620000in}{4.620000in}}%
\pgfusepath{clip}%
\pgfsetbuttcap%
\pgfsetmiterjoin%
\definecolor{currentfill}{rgb}{0.176471,0.192157,0.258824}%
\pgfsetfillcolor{currentfill}%
\pgfsetlinewidth{1.003750pt}%
\definecolor{currentstroke}{rgb}{0.176471,0.192157,0.258824}%
\pgfsetstrokecolor{currentstroke}%
\pgfsetdash{}{0pt}%
\pgfsys@defobject{currentmarker}{\pgfqpoint{-0.033333in}{-0.033333in}}{\pgfqpoint{0.033333in}{0.033333in}}{%
\pgfpathmoveto{\pgfqpoint{-0.000000in}{-0.033333in}}%
\pgfpathlineto{\pgfqpoint{0.033333in}{0.033333in}}%
\pgfpathlineto{\pgfqpoint{-0.033333in}{0.033333in}}%
\pgfpathlineto{\pgfqpoint{-0.000000in}{-0.033333in}}%
\pgfpathclose%
\pgfusepath{stroke,fill}%
}%
\begin{pgfscope}%
\pgfsys@transformshift{2.529524in}{4.529399in}%
\pgfsys@useobject{currentmarker}{}%
\end{pgfscope}%
\end{pgfscope}%
\begin{pgfscope}%
\pgfpathrectangle{\pgfqpoint{0.765000in}{0.660000in}}{\pgfqpoint{4.620000in}{4.620000in}}%
\pgfusepath{clip}%
\pgfsetrectcap%
\pgfsetroundjoin%
\pgfsetlinewidth{1.204500pt}%
\definecolor{currentstroke}{rgb}{0.176471,0.192157,0.258824}%
\pgfsetstrokecolor{currentstroke}%
\pgfsetdash{}{0pt}%
\pgfpathmoveto{\pgfqpoint{3.462256in}{4.018035in}}%
\pgfusepath{stroke}%
\end{pgfscope}%
\begin{pgfscope}%
\pgfpathrectangle{\pgfqpoint{0.765000in}{0.660000in}}{\pgfqpoint{4.620000in}{4.620000in}}%
\pgfusepath{clip}%
\pgfsetbuttcap%
\pgfsetmiterjoin%
\definecolor{currentfill}{rgb}{0.176471,0.192157,0.258824}%
\pgfsetfillcolor{currentfill}%
\pgfsetlinewidth{1.003750pt}%
\definecolor{currentstroke}{rgb}{0.176471,0.192157,0.258824}%
\pgfsetstrokecolor{currentstroke}%
\pgfsetdash{}{0pt}%
\pgfsys@defobject{currentmarker}{\pgfqpoint{-0.033333in}{-0.033333in}}{\pgfqpoint{0.033333in}{0.033333in}}{%
\pgfpathmoveto{\pgfqpoint{-0.000000in}{-0.033333in}}%
\pgfpathlineto{\pgfqpoint{0.033333in}{0.033333in}}%
\pgfpathlineto{\pgfqpoint{-0.033333in}{0.033333in}}%
\pgfpathlineto{\pgfqpoint{-0.000000in}{-0.033333in}}%
\pgfpathclose%
\pgfusepath{stroke,fill}%
}%
\begin{pgfscope}%
\pgfsys@transformshift{3.462256in}{4.018035in}%
\pgfsys@useobject{currentmarker}{}%
\end{pgfscope}%
\end{pgfscope}%
\begin{pgfscope}%
\pgfpathrectangle{\pgfqpoint{0.765000in}{0.660000in}}{\pgfqpoint{4.620000in}{4.620000in}}%
\pgfusepath{clip}%
\pgfsetrectcap%
\pgfsetroundjoin%
\pgfsetlinewidth{1.204500pt}%
\definecolor{currentstroke}{rgb}{0.176471,0.192157,0.258824}%
\pgfsetstrokecolor{currentstroke}%
\pgfsetdash{}{0pt}%
\pgfpathmoveto{\pgfqpoint{4.387786in}{4.527148in}}%
\pgfusepath{stroke}%
\end{pgfscope}%
\begin{pgfscope}%
\pgfpathrectangle{\pgfqpoint{0.765000in}{0.660000in}}{\pgfqpoint{4.620000in}{4.620000in}}%
\pgfusepath{clip}%
\pgfsetbuttcap%
\pgfsetmiterjoin%
\definecolor{currentfill}{rgb}{0.176471,0.192157,0.258824}%
\pgfsetfillcolor{currentfill}%
\pgfsetlinewidth{1.003750pt}%
\definecolor{currentstroke}{rgb}{0.176471,0.192157,0.258824}%
\pgfsetstrokecolor{currentstroke}%
\pgfsetdash{}{0pt}%
\pgfsys@defobject{currentmarker}{\pgfqpoint{-0.033333in}{-0.033333in}}{\pgfqpoint{0.033333in}{0.033333in}}{%
\pgfpathmoveto{\pgfqpoint{-0.000000in}{-0.033333in}}%
\pgfpathlineto{\pgfqpoint{0.033333in}{0.033333in}}%
\pgfpathlineto{\pgfqpoint{-0.033333in}{0.033333in}}%
\pgfpathlineto{\pgfqpoint{-0.000000in}{-0.033333in}}%
\pgfpathclose%
\pgfusepath{stroke,fill}%
}%
\begin{pgfscope}%
\pgfsys@transformshift{4.387786in}{4.527148in}%
\pgfsys@useobject{currentmarker}{}%
\end{pgfscope}%
\end{pgfscope}%
\begin{pgfscope}%
\pgfpathrectangle{\pgfqpoint{0.765000in}{0.660000in}}{\pgfqpoint{4.620000in}{4.620000in}}%
\pgfusepath{clip}%
\pgfsetrectcap%
\pgfsetroundjoin%
\pgfsetlinewidth{1.204500pt}%
\definecolor{currentstroke}{rgb}{0.176471,0.192157,0.258824}%
\pgfsetstrokecolor{currentstroke}%
\pgfsetdash{}{0pt}%
\pgfpathmoveto{\pgfqpoint{3.562679in}{4.649454in}}%
\pgfusepath{stroke}%
\end{pgfscope}%
\begin{pgfscope}%
\pgfpathrectangle{\pgfqpoint{0.765000in}{0.660000in}}{\pgfqpoint{4.620000in}{4.620000in}}%
\pgfusepath{clip}%
\pgfsetbuttcap%
\pgfsetmiterjoin%
\definecolor{currentfill}{rgb}{0.176471,0.192157,0.258824}%
\pgfsetfillcolor{currentfill}%
\pgfsetlinewidth{1.003750pt}%
\definecolor{currentstroke}{rgb}{0.176471,0.192157,0.258824}%
\pgfsetstrokecolor{currentstroke}%
\pgfsetdash{}{0pt}%
\pgfsys@defobject{currentmarker}{\pgfqpoint{-0.033333in}{-0.033333in}}{\pgfqpoint{0.033333in}{0.033333in}}{%
\pgfpathmoveto{\pgfqpoint{-0.000000in}{-0.033333in}}%
\pgfpathlineto{\pgfqpoint{0.033333in}{0.033333in}}%
\pgfpathlineto{\pgfqpoint{-0.033333in}{0.033333in}}%
\pgfpathlineto{\pgfqpoint{-0.000000in}{-0.033333in}}%
\pgfpathclose%
\pgfusepath{stroke,fill}%
}%
\begin{pgfscope}%
\pgfsys@transformshift{3.562679in}{4.649454in}%
\pgfsys@useobject{currentmarker}{}%
\end{pgfscope}%
\end{pgfscope}%
\begin{pgfscope}%
\pgfpathrectangle{\pgfqpoint{0.765000in}{0.660000in}}{\pgfqpoint{4.620000in}{4.620000in}}%
\pgfusepath{clip}%
\pgfsetrectcap%
\pgfsetroundjoin%
\pgfsetlinewidth{1.204500pt}%
\definecolor{currentstroke}{rgb}{0.176471,0.192157,0.258824}%
\pgfsetstrokecolor{currentstroke}%
\pgfsetdash{}{0pt}%
\pgfpathmoveto{\pgfqpoint{3.153393in}{4.313283in}}%
\pgfusepath{stroke}%
\end{pgfscope}%
\begin{pgfscope}%
\pgfpathrectangle{\pgfqpoint{0.765000in}{0.660000in}}{\pgfqpoint{4.620000in}{4.620000in}}%
\pgfusepath{clip}%
\pgfsetbuttcap%
\pgfsetmiterjoin%
\definecolor{currentfill}{rgb}{0.176471,0.192157,0.258824}%
\pgfsetfillcolor{currentfill}%
\pgfsetlinewidth{1.003750pt}%
\definecolor{currentstroke}{rgb}{0.176471,0.192157,0.258824}%
\pgfsetstrokecolor{currentstroke}%
\pgfsetdash{}{0pt}%
\pgfsys@defobject{currentmarker}{\pgfqpoint{-0.033333in}{-0.033333in}}{\pgfqpoint{0.033333in}{0.033333in}}{%
\pgfpathmoveto{\pgfqpoint{-0.000000in}{-0.033333in}}%
\pgfpathlineto{\pgfqpoint{0.033333in}{0.033333in}}%
\pgfpathlineto{\pgfqpoint{-0.033333in}{0.033333in}}%
\pgfpathlineto{\pgfqpoint{-0.000000in}{-0.033333in}}%
\pgfpathclose%
\pgfusepath{stroke,fill}%
}%
\begin{pgfscope}%
\pgfsys@transformshift{3.153393in}{4.313283in}%
\pgfsys@useobject{currentmarker}{}%
\end{pgfscope}%
\end{pgfscope}%
\begin{pgfscope}%
\pgfpathrectangle{\pgfqpoint{0.765000in}{0.660000in}}{\pgfqpoint{4.620000in}{4.620000in}}%
\pgfusepath{clip}%
\pgfsetrectcap%
\pgfsetroundjoin%
\pgfsetlinewidth{1.204500pt}%
\definecolor{currentstroke}{rgb}{0.176471,0.192157,0.258824}%
\pgfsetstrokecolor{currentstroke}%
\pgfsetdash{}{0pt}%
\pgfpathmoveto{\pgfqpoint{2.649075in}{4.180948in}}%
\pgfusepath{stroke}%
\end{pgfscope}%
\begin{pgfscope}%
\pgfpathrectangle{\pgfqpoint{0.765000in}{0.660000in}}{\pgfqpoint{4.620000in}{4.620000in}}%
\pgfusepath{clip}%
\pgfsetbuttcap%
\pgfsetmiterjoin%
\definecolor{currentfill}{rgb}{0.176471,0.192157,0.258824}%
\pgfsetfillcolor{currentfill}%
\pgfsetlinewidth{1.003750pt}%
\definecolor{currentstroke}{rgb}{0.176471,0.192157,0.258824}%
\pgfsetstrokecolor{currentstroke}%
\pgfsetdash{}{0pt}%
\pgfsys@defobject{currentmarker}{\pgfqpoint{-0.033333in}{-0.033333in}}{\pgfqpoint{0.033333in}{0.033333in}}{%
\pgfpathmoveto{\pgfqpoint{-0.000000in}{-0.033333in}}%
\pgfpathlineto{\pgfqpoint{0.033333in}{0.033333in}}%
\pgfpathlineto{\pgfqpoint{-0.033333in}{0.033333in}}%
\pgfpathlineto{\pgfqpoint{-0.000000in}{-0.033333in}}%
\pgfpathclose%
\pgfusepath{stroke,fill}%
}%
\begin{pgfscope}%
\pgfsys@transformshift{2.649075in}{4.180948in}%
\pgfsys@useobject{currentmarker}{}%
\end{pgfscope}%
\end{pgfscope}%
\begin{pgfscope}%
\pgfpathrectangle{\pgfqpoint{0.765000in}{0.660000in}}{\pgfqpoint{4.620000in}{4.620000in}}%
\pgfusepath{clip}%
\pgfsetrectcap%
\pgfsetroundjoin%
\pgfsetlinewidth{1.204500pt}%
\definecolor{currentstroke}{rgb}{0.176471,0.192157,0.258824}%
\pgfsetstrokecolor{currentstroke}%
\pgfsetdash{}{0pt}%
\pgfpathmoveto{\pgfqpoint{3.087833in}{3.708583in}}%
\pgfusepath{stroke}%
\end{pgfscope}%
\begin{pgfscope}%
\pgfpathrectangle{\pgfqpoint{0.765000in}{0.660000in}}{\pgfqpoint{4.620000in}{4.620000in}}%
\pgfusepath{clip}%
\pgfsetbuttcap%
\pgfsetmiterjoin%
\definecolor{currentfill}{rgb}{0.176471,0.192157,0.258824}%
\pgfsetfillcolor{currentfill}%
\pgfsetlinewidth{1.003750pt}%
\definecolor{currentstroke}{rgb}{0.176471,0.192157,0.258824}%
\pgfsetstrokecolor{currentstroke}%
\pgfsetdash{}{0pt}%
\pgfsys@defobject{currentmarker}{\pgfqpoint{-0.033333in}{-0.033333in}}{\pgfqpoint{0.033333in}{0.033333in}}{%
\pgfpathmoveto{\pgfqpoint{-0.000000in}{-0.033333in}}%
\pgfpathlineto{\pgfqpoint{0.033333in}{0.033333in}}%
\pgfpathlineto{\pgfqpoint{-0.033333in}{0.033333in}}%
\pgfpathlineto{\pgfqpoint{-0.000000in}{-0.033333in}}%
\pgfpathclose%
\pgfusepath{stroke,fill}%
}%
\begin{pgfscope}%
\pgfsys@transformshift{3.087833in}{3.708583in}%
\pgfsys@useobject{currentmarker}{}%
\end{pgfscope}%
\end{pgfscope}%
\begin{pgfscope}%
\pgfpathrectangle{\pgfqpoint{0.765000in}{0.660000in}}{\pgfqpoint{4.620000in}{4.620000in}}%
\pgfusepath{clip}%
\pgfsetrectcap%
\pgfsetroundjoin%
\pgfsetlinewidth{1.204500pt}%
\definecolor{currentstroke}{rgb}{0.176471,0.192157,0.258824}%
\pgfsetstrokecolor{currentstroke}%
\pgfsetdash{}{0pt}%
\pgfpathmoveto{\pgfqpoint{2.647040in}{4.600521in}}%
\pgfusepath{stroke}%
\end{pgfscope}%
\begin{pgfscope}%
\pgfpathrectangle{\pgfqpoint{0.765000in}{0.660000in}}{\pgfqpoint{4.620000in}{4.620000in}}%
\pgfusepath{clip}%
\pgfsetbuttcap%
\pgfsetmiterjoin%
\definecolor{currentfill}{rgb}{0.176471,0.192157,0.258824}%
\pgfsetfillcolor{currentfill}%
\pgfsetlinewidth{1.003750pt}%
\definecolor{currentstroke}{rgb}{0.176471,0.192157,0.258824}%
\pgfsetstrokecolor{currentstroke}%
\pgfsetdash{}{0pt}%
\pgfsys@defobject{currentmarker}{\pgfqpoint{-0.033333in}{-0.033333in}}{\pgfqpoint{0.033333in}{0.033333in}}{%
\pgfpathmoveto{\pgfqpoint{-0.000000in}{-0.033333in}}%
\pgfpathlineto{\pgfqpoint{0.033333in}{0.033333in}}%
\pgfpathlineto{\pgfqpoint{-0.033333in}{0.033333in}}%
\pgfpathlineto{\pgfqpoint{-0.000000in}{-0.033333in}}%
\pgfpathclose%
\pgfusepath{stroke,fill}%
}%
\begin{pgfscope}%
\pgfsys@transformshift{2.647040in}{4.600521in}%
\pgfsys@useobject{currentmarker}{}%
\end{pgfscope}%
\end{pgfscope}%
\begin{pgfscope}%
\pgfpathrectangle{\pgfqpoint{0.765000in}{0.660000in}}{\pgfqpoint{4.620000in}{4.620000in}}%
\pgfusepath{clip}%
\pgfsetrectcap%
\pgfsetroundjoin%
\pgfsetlinewidth{1.204500pt}%
\definecolor{currentstroke}{rgb}{0.176471,0.192157,0.258824}%
\pgfsetstrokecolor{currentstroke}%
\pgfsetdash{}{0pt}%
\pgfpathmoveto{\pgfqpoint{3.289747in}{4.306742in}}%
\pgfusepath{stroke}%
\end{pgfscope}%
\begin{pgfscope}%
\pgfpathrectangle{\pgfqpoint{0.765000in}{0.660000in}}{\pgfqpoint{4.620000in}{4.620000in}}%
\pgfusepath{clip}%
\pgfsetbuttcap%
\pgfsetmiterjoin%
\definecolor{currentfill}{rgb}{0.176471,0.192157,0.258824}%
\pgfsetfillcolor{currentfill}%
\pgfsetlinewidth{1.003750pt}%
\definecolor{currentstroke}{rgb}{0.176471,0.192157,0.258824}%
\pgfsetstrokecolor{currentstroke}%
\pgfsetdash{}{0pt}%
\pgfsys@defobject{currentmarker}{\pgfqpoint{-0.033333in}{-0.033333in}}{\pgfqpoint{0.033333in}{0.033333in}}{%
\pgfpathmoveto{\pgfqpoint{-0.000000in}{-0.033333in}}%
\pgfpathlineto{\pgfqpoint{0.033333in}{0.033333in}}%
\pgfpathlineto{\pgfqpoint{-0.033333in}{0.033333in}}%
\pgfpathlineto{\pgfqpoint{-0.000000in}{-0.033333in}}%
\pgfpathclose%
\pgfusepath{stroke,fill}%
}%
\begin{pgfscope}%
\pgfsys@transformshift{3.289747in}{4.306742in}%
\pgfsys@useobject{currentmarker}{}%
\end{pgfscope}%
\end{pgfscope}%
\begin{pgfscope}%
\pgfpathrectangle{\pgfqpoint{0.765000in}{0.660000in}}{\pgfqpoint{4.620000in}{4.620000in}}%
\pgfusepath{clip}%
\pgfsetrectcap%
\pgfsetroundjoin%
\pgfsetlinewidth{1.204500pt}%
\definecolor{currentstroke}{rgb}{0.000000,0.000000,0.000000}%
\pgfsetstrokecolor{currentstroke}%
\pgfsetdash{}{0pt}%
\pgfpathmoveto{\pgfqpoint{3.076409in}{3.135951in}}%
\pgfusepath{stroke}%
\end{pgfscope}%
\begin{pgfscope}%
\pgfpathrectangle{\pgfqpoint{0.765000in}{0.660000in}}{\pgfqpoint{4.620000in}{4.620000in}}%
\pgfusepath{clip}%
\pgfsetbuttcap%
\pgfsetroundjoin%
\definecolor{currentfill}{rgb}{0.000000,0.000000,0.000000}%
\pgfsetfillcolor{currentfill}%
\pgfsetlinewidth{1.003750pt}%
\definecolor{currentstroke}{rgb}{0.000000,0.000000,0.000000}%
\pgfsetstrokecolor{currentstroke}%
\pgfsetdash{}{0pt}%
\pgfsys@defobject{currentmarker}{\pgfqpoint{-0.016667in}{-0.016667in}}{\pgfqpoint{0.016667in}{0.016667in}}{%
\pgfpathmoveto{\pgfqpoint{0.000000in}{-0.016667in}}%
\pgfpathcurveto{\pgfqpoint{0.004420in}{-0.016667in}}{\pgfqpoint{0.008660in}{-0.014911in}}{\pgfqpoint{0.011785in}{-0.011785in}}%
\pgfpathcurveto{\pgfqpoint{0.014911in}{-0.008660in}}{\pgfqpoint{0.016667in}{-0.004420in}}{\pgfqpoint{0.016667in}{0.000000in}}%
\pgfpathcurveto{\pgfqpoint{0.016667in}{0.004420in}}{\pgfqpoint{0.014911in}{0.008660in}}{\pgfqpoint{0.011785in}{0.011785in}}%
\pgfpathcurveto{\pgfqpoint{0.008660in}{0.014911in}}{\pgfqpoint{0.004420in}{0.016667in}}{\pgfqpoint{0.000000in}{0.016667in}}%
\pgfpathcurveto{\pgfqpoint{-0.004420in}{0.016667in}}{\pgfqpoint{-0.008660in}{0.014911in}}{\pgfqpoint{-0.011785in}{0.011785in}}%
\pgfpathcurveto{\pgfqpoint{-0.014911in}{0.008660in}}{\pgfqpoint{-0.016667in}{0.004420in}}{\pgfqpoint{-0.016667in}{0.000000in}}%
\pgfpathcurveto{\pgfqpoint{-0.016667in}{-0.004420in}}{\pgfqpoint{-0.014911in}{-0.008660in}}{\pgfqpoint{-0.011785in}{-0.011785in}}%
\pgfpathcurveto{\pgfqpoint{-0.008660in}{-0.014911in}}{\pgfqpoint{-0.004420in}{-0.016667in}}{\pgfqpoint{0.000000in}{-0.016667in}}%
\pgfpathlineto{\pgfqpoint{0.000000in}{-0.016667in}}%
\pgfpathclose%
\pgfusepath{stroke,fill}%
}%
\begin{pgfscope}%
\pgfsys@transformshift{3.076409in}{3.135951in}%
\pgfsys@useobject{currentmarker}{}%
\end{pgfscope}%
\end{pgfscope}%
\begin{pgfscope}%
\pgfpathrectangle{\pgfqpoint{0.765000in}{0.660000in}}{\pgfqpoint{4.620000in}{4.620000in}}%
\pgfusepath{clip}%
\pgfsetbuttcap%
\pgfsetroundjoin%
\pgfsetlinewidth{1.204500pt}%
\definecolor{currentstroke}{rgb}{0.827451,0.827451,0.827451}%
\pgfsetstrokecolor{currentstroke}%
\pgfsetdash{{1.200000pt}{1.980000pt}}{0.000000pt}%
\pgfpathmoveto{\pgfqpoint{3.076409in}{3.135951in}}%
\pgfpathlineto{\pgfqpoint{3.076409in}{0.650000in}}%
\pgfusepath{stroke}%
\end{pgfscope}%
\begin{pgfscope}%
\pgfpathrectangle{\pgfqpoint{0.765000in}{0.660000in}}{\pgfqpoint{4.620000in}{4.620000in}}%
\pgfusepath{clip}%
\pgfsetrectcap%
\pgfsetroundjoin%
\pgfsetlinewidth{1.204500pt}%
\definecolor{currentstroke}{rgb}{0.000000,0.000000,0.000000}%
\pgfsetstrokecolor{currentstroke}%
\pgfsetdash{}{0pt}%
\pgfpathmoveto{\pgfqpoint{3.076409in}{3.135951in}}%
\pgfpathlineto{\pgfqpoint{0.755000in}{4.476217in}}%
\pgfusepath{stroke}%
\end{pgfscope}%
\begin{pgfscope}%
\pgfpathrectangle{\pgfqpoint{0.765000in}{0.660000in}}{\pgfqpoint{4.620000in}{4.620000in}}%
\pgfusepath{clip}%
\pgfsetrectcap%
\pgfsetroundjoin%
\pgfsetlinewidth{1.204500pt}%
\definecolor{currentstroke}{rgb}{0.000000,0.000000,0.000000}%
\pgfsetstrokecolor{currentstroke}%
\pgfsetdash{}{0pt}%
\pgfpathmoveto{\pgfqpoint{3.076409in}{3.135951in}}%
\pgfpathlineto{\pgfqpoint{5.395000in}{4.474590in}}%
\pgfusepath{stroke}%
\end{pgfscope}%
\end{pgfpicture}%
\makeatother%
\endgroup%
}
}
        \caption{Maximum-margin tropical hyperplane}
        \label{fig:MaxMargin}
    \end{subfigure}
    \hfill
    \centering
    \begin{subfigure}{0.4\textwidth}
        \centering
        \resizebox{\linewidth}{!}{%
        \centering
            \clipbox{0.2\width{} 0.3\height{} 0.2\width{} 0.25\height{}}{%% Creator: Matplotlib, PGF backend
%%
%% To include the figure in your LaTeX document, write
%%   \input{<filename>.pgf}
%%
%% Make sure the required packages are loaded in your preamble
%%   \usepackage{pgf}
%%
%% Also ensure that all the required font packages are loaded; for instance,
%% the lmodern package is sometimes necessary when using math font.
%%   \usepackage{lmodern}
%%
%% Figures using additional raster images can only be included by \input if
%% they are in the same directory as the main LaTeX file. For loading figures
%% from other directories you can use the `import` package
%%   \usepackage{import}
%%
%% and then include the figures with
%%   \import{<path to file>}{<filename>.pgf}
%%
%% Matplotlib used the following preamble
%%   
%%   \usepackage{fontspec}
%%   \setmainfont{DejaVuSerif.ttf}[Path=\detokenize{/Users/sam/Library/Python/3.9/lib/python/site-packages/matplotlib/mpl-data/fonts/ttf/}]
%%   \setsansfont{DejaVuSans.ttf}[Path=\detokenize{/Users/sam/Library/Python/3.9/lib/python/site-packages/matplotlib/mpl-data/fonts/ttf/}]
%%   \setmonofont{DejaVuSansMono.ttf}[Path=\detokenize{/Users/sam/Library/Python/3.9/lib/python/site-packages/matplotlib/mpl-data/fonts/ttf/}]
%%   \makeatletter\@ifpackageloaded{underscore}{}{\usepackage[strings]{underscore}}\makeatother
%%
\begingroup%
\makeatletter%
\begin{pgfpicture}%
\pgfpathrectangle{\pgfpointorigin}{\pgfqpoint{6.000000in}{6.000000in}}%
\pgfusepath{use as bounding box, clip}%
\begin{pgfscope}%
\pgfsetbuttcap%
\pgfsetmiterjoin%
\definecolor{currentfill}{rgb}{1.000000,1.000000,1.000000}%
\pgfsetfillcolor{currentfill}%
\pgfsetlinewidth{0.000000pt}%
\definecolor{currentstroke}{rgb}{1.000000,1.000000,1.000000}%
\pgfsetstrokecolor{currentstroke}%
\pgfsetdash{}{0pt}%
\pgfpathmoveto{\pgfqpoint{0.000000in}{0.000000in}}%
\pgfpathlineto{\pgfqpoint{6.000000in}{0.000000in}}%
\pgfpathlineto{\pgfqpoint{6.000000in}{6.000000in}}%
\pgfpathlineto{\pgfqpoint{0.000000in}{6.000000in}}%
\pgfpathlineto{\pgfqpoint{0.000000in}{0.000000in}}%
\pgfpathclose%
\pgfusepath{fill}%
\end{pgfscope}%
\begin{pgfscope}%
\pgfsetbuttcap%
\pgfsetmiterjoin%
\definecolor{currentfill}{rgb}{1.000000,1.000000,1.000000}%
\pgfsetfillcolor{currentfill}%
\pgfsetlinewidth{0.000000pt}%
\definecolor{currentstroke}{rgb}{0.000000,0.000000,0.000000}%
\pgfsetstrokecolor{currentstroke}%
\pgfsetstrokeopacity{0.000000}%
\pgfsetdash{}{0pt}%
\pgfpathmoveto{\pgfqpoint{0.765000in}{0.660000in}}%
\pgfpathlineto{\pgfqpoint{5.385000in}{0.660000in}}%
\pgfpathlineto{\pgfqpoint{5.385000in}{5.280000in}}%
\pgfpathlineto{\pgfqpoint{0.765000in}{5.280000in}}%
\pgfpathlineto{\pgfqpoint{0.765000in}{0.660000in}}%
\pgfpathclose%
\pgfusepath{fill}%
\end{pgfscope}%
\begin{pgfscope}%
\pgfpathrectangle{\pgfqpoint{0.765000in}{0.660000in}}{\pgfqpoint{4.620000in}{4.620000in}}%
\pgfusepath{clip}%
\pgfsetrectcap%
\pgfsetroundjoin%
\pgfsetlinewidth{1.204500pt}%
\definecolor{currentstroke}{rgb}{1.000000,0.576471,0.309804}%
\pgfsetstrokecolor{currentstroke}%
\pgfsetdash{}{0pt}%
\pgfpathmoveto{\pgfqpoint{3.985626in}{2.728727in}}%
\pgfusepath{stroke}%
\end{pgfscope}%
\begin{pgfscope}%
\pgfpathrectangle{\pgfqpoint{0.765000in}{0.660000in}}{\pgfqpoint{4.620000in}{4.620000in}}%
\pgfusepath{clip}%
\pgfsetbuttcap%
\pgfsetroundjoin%
\definecolor{currentfill}{rgb}{1.000000,0.576471,0.309804}%
\pgfsetfillcolor{currentfill}%
\pgfsetlinewidth{1.003750pt}%
\definecolor{currentstroke}{rgb}{1.000000,0.576471,0.309804}%
\pgfsetstrokecolor{currentstroke}%
\pgfsetdash{}{0pt}%
\pgfsys@defobject{currentmarker}{\pgfqpoint{-0.033333in}{-0.033333in}}{\pgfqpoint{0.033333in}{0.033333in}}{%
\pgfpathmoveto{\pgfqpoint{0.000000in}{-0.033333in}}%
\pgfpathcurveto{\pgfqpoint{0.008840in}{-0.033333in}}{\pgfqpoint{0.017319in}{-0.029821in}}{\pgfqpoint{0.023570in}{-0.023570in}}%
\pgfpathcurveto{\pgfqpoint{0.029821in}{-0.017319in}}{\pgfqpoint{0.033333in}{-0.008840in}}{\pgfqpoint{0.033333in}{0.000000in}}%
\pgfpathcurveto{\pgfqpoint{0.033333in}{0.008840in}}{\pgfqpoint{0.029821in}{0.017319in}}{\pgfqpoint{0.023570in}{0.023570in}}%
\pgfpathcurveto{\pgfqpoint{0.017319in}{0.029821in}}{\pgfqpoint{0.008840in}{0.033333in}}{\pgfqpoint{0.000000in}{0.033333in}}%
\pgfpathcurveto{\pgfqpoint{-0.008840in}{0.033333in}}{\pgfqpoint{-0.017319in}{0.029821in}}{\pgfqpoint{-0.023570in}{0.023570in}}%
\pgfpathcurveto{\pgfqpoint{-0.029821in}{0.017319in}}{\pgfqpoint{-0.033333in}{0.008840in}}{\pgfqpoint{-0.033333in}{0.000000in}}%
\pgfpathcurveto{\pgfqpoint{-0.033333in}{-0.008840in}}{\pgfqpoint{-0.029821in}{-0.017319in}}{\pgfqpoint{-0.023570in}{-0.023570in}}%
\pgfpathcurveto{\pgfqpoint{-0.017319in}{-0.029821in}}{\pgfqpoint{-0.008840in}{-0.033333in}}{\pgfqpoint{0.000000in}{-0.033333in}}%
\pgfpathlineto{\pgfqpoint{0.000000in}{-0.033333in}}%
\pgfpathclose%
\pgfusepath{stroke,fill}%
}%
\begin{pgfscope}%
\pgfsys@transformshift{3.985626in}{2.728727in}%
\pgfsys@useobject{currentmarker}{}%
\end{pgfscope}%
\end{pgfscope}%
\begin{pgfscope}%
\pgfpathrectangle{\pgfqpoint{0.765000in}{0.660000in}}{\pgfqpoint{4.620000in}{4.620000in}}%
\pgfusepath{clip}%
\pgfsetrectcap%
\pgfsetroundjoin%
\pgfsetlinewidth{1.204500pt}%
\definecolor{currentstroke}{rgb}{1.000000,0.576471,0.309804}%
\pgfsetstrokecolor{currentstroke}%
\pgfsetdash{}{0pt}%
\pgfpathmoveto{\pgfqpoint{4.372321in}{2.641107in}}%
\pgfusepath{stroke}%
\end{pgfscope}%
\begin{pgfscope}%
\pgfpathrectangle{\pgfqpoint{0.765000in}{0.660000in}}{\pgfqpoint{4.620000in}{4.620000in}}%
\pgfusepath{clip}%
\pgfsetbuttcap%
\pgfsetroundjoin%
\definecolor{currentfill}{rgb}{1.000000,0.576471,0.309804}%
\pgfsetfillcolor{currentfill}%
\pgfsetlinewidth{1.003750pt}%
\definecolor{currentstroke}{rgb}{1.000000,0.576471,0.309804}%
\pgfsetstrokecolor{currentstroke}%
\pgfsetdash{}{0pt}%
\pgfsys@defobject{currentmarker}{\pgfqpoint{-0.033333in}{-0.033333in}}{\pgfqpoint{0.033333in}{0.033333in}}{%
\pgfpathmoveto{\pgfqpoint{0.000000in}{-0.033333in}}%
\pgfpathcurveto{\pgfqpoint{0.008840in}{-0.033333in}}{\pgfqpoint{0.017319in}{-0.029821in}}{\pgfqpoint{0.023570in}{-0.023570in}}%
\pgfpathcurveto{\pgfqpoint{0.029821in}{-0.017319in}}{\pgfqpoint{0.033333in}{-0.008840in}}{\pgfqpoint{0.033333in}{0.000000in}}%
\pgfpathcurveto{\pgfqpoint{0.033333in}{0.008840in}}{\pgfqpoint{0.029821in}{0.017319in}}{\pgfqpoint{0.023570in}{0.023570in}}%
\pgfpathcurveto{\pgfqpoint{0.017319in}{0.029821in}}{\pgfqpoint{0.008840in}{0.033333in}}{\pgfqpoint{0.000000in}{0.033333in}}%
\pgfpathcurveto{\pgfqpoint{-0.008840in}{0.033333in}}{\pgfqpoint{-0.017319in}{0.029821in}}{\pgfqpoint{-0.023570in}{0.023570in}}%
\pgfpathcurveto{\pgfqpoint{-0.029821in}{0.017319in}}{\pgfqpoint{-0.033333in}{0.008840in}}{\pgfqpoint{-0.033333in}{0.000000in}}%
\pgfpathcurveto{\pgfqpoint{-0.033333in}{-0.008840in}}{\pgfqpoint{-0.029821in}{-0.017319in}}{\pgfqpoint{-0.023570in}{-0.023570in}}%
\pgfpathcurveto{\pgfqpoint{-0.017319in}{-0.029821in}}{\pgfqpoint{-0.008840in}{-0.033333in}}{\pgfqpoint{0.000000in}{-0.033333in}}%
\pgfpathlineto{\pgfqpoint{0.000000in}{-0.033333in}}%
\pgfpathclose%
\pgfusepath{stroke,fill}%
}%
\begin{pgfscope}%
\pgfsys@transformshift{4.372321in}{2.641107in}%
\pgfsys@useobject{currentmarker}{}%
\end{pgfscope}%
\end{pgfscope}%
\begin{pgfscope}%
\pgfpathrectangle{\pgfqpoint{0.765000in}{0.660000in}}{\pgfqpoint{4.620000in}{4.620000in}}%
\pgfusepath{clip}%
\pgfsetrectcap%
\pgfsetroundjoin%
\pgfsetlinewidth{1.204500pt}%
\definecolor{currentstroke}{rgb}{1.000000,0.576471,0.309804}%
\pgfsetstrokecolor{currentstroke}%
\pgfsetdash{}{0pt}%
\pgfpathmoveto{\pgfqpoint{3.797765in}{2.807778in}}%
\pgfusepath{stroke}%
\end{pgfscope}%
\begin{pgfscope}%
\pgfpathrectangle{\pgfqpoint{0.765000in}{0.660000in}}{\pgfqpoint{4.620000in}{4.620000in}}%
\pgfusepath{clip}%
\pgfsetbuttcap%
\pgfsetroundjoin%
\definecolor{currentfill}{rgb}{1.000000,0.576471,0.309804}%
\pgfsetfillcolor{currentfill}%
\pgfsetlinewidth{1.003750pt}%
\definecolor{currentstroke}{rgb}{1.000000,0.576471,0.309804}%
\pgfsetstrokecolor{currentstroke}%
\pgfsetdash{}{0pt}%
\pgfsys@defobject{currentmarker}{\pgfqpoint{-0.033333in}{-0.033333in}}{\pgfqpoint{0.033333in}{0.033333in}}{%
\pgfpathmoveto{\pgfqpoint{0.000000in}{-0.033333in}}%
\pgfpathcurveto{\pgfqpoint{0.008840in}{-0.033333in}}{\pgfqpoint{0.017319in}{-0.029821in}}{\pgfqpoint{0.023570in}{-0.023570in}}%
\pgfpathcurveto{\pgfqpoint{0.029821in}{-0.017319in}}{\pgfqpoint{0.033333in}{-0.008840in}}{\pgfqpoint{0.033333in}{0.000000in}}%
\pgfpathcurveto{\pgfqpoint{0.033333in}{0.008840in}}{\pgfqpoint{0.029821in}{0.017319in}}{\pgfqpoint{0.023570in}{0.023570in}}%
\pgfpathcurveto{\pgfqpoint{0.017319in}{0.029821in}}{\pgfqpoint{0.008840in}{0.033333in}}{\pgfqpoint{0.000000in}{0.033333in}}%
\pgfpathcurveto{\pgfqpoint{-0.008840in}{0.033333in}}{\pgfqpoint{-0.017319in}{0.029821in}}{\pgfqpoint{-0.023570in}{0.023570in}}%
\pgfpathcurveto{\pgfqpoint{-0.029821in}{0.017319in}}{\pgfqpoint{-0.033333in}{0.008840in}}{\pgfqpoint{-0.033333in}{0.000000in}}%
\pgfpathcurveto{\pgfqpoint{-0.033333in}{-0.008840in}}{\pgfqpoint{-0.029821in}{-0.017319in}}{\pgfqpoint{-0.023570in}{-0.023570in}}%
\pgfpathcurveto{\pgfqpoint{-0.017319in}{-0.029821in}}{\pgfqpoint{-0.008840in}{-0.033333in}}{\pgfqpoint{0.000000in}{-0.033333in}}%
\pgfpathlineto{\pgfqpoint{0.000000in}{-0.033333in}}%
\pgfpathclose%
\pgfusepath{stroke,fill}%
}%
\begin{pgfscope}%
\pgfsys@transformshift{3.797765in}{2.807778in}%
\pgfsys@useobject{currentmarker}{}%
\end{pgfscope}%
\end{pgfscope}%
\begin{pgfscope}%
\pgfpathrectangle{\pgfqpoint{0.765000in}{0.660000in}}{\pgfqpoint{4.620000in}{4.620000in}}%
\pgfusepath{clip}%
\pgfsetrectcap%
\pgfsetroundjoin%
\pgfsetlinewidth{1.204500pt}%
\definecolor{currentstroke}{rgb}{1.000000,0.576471,0.309804}%
\pgfsetstrokecolor{currentstroke}%
\pgfsetdash{}{0pt}%
\pgfpathmoveto{\pgfqpoint{5.369646in}{3.172612in}}%
\pgfusepath{stroke}%
\end{pgfscope}%
\begin{pgfscope}%
\pgfpathrectangle{\pgfqpoint{0.765000in}{0.660000in}}{\pgfqpoint{4.620000in}{4.620000in}}%
\pgfusepath{clip}%
\pgfsetbuttcap%
\pgfsetroundjoin%
\definecolor{currentfill}{rgb}{1.000000,0.576471,0.309804}%
\pgfsetfillcolor{currentfill}%
\pgfsetlinewidth{1.003750pt}%
\definecolor{currentstroke}{rgb}{1.000000,0.576471,0.309804}%
\pgfsetstrokecolor{currentstroke}%
\pgfsetdash{}{0pt}%
\pgfsys@defobject{currentmarker}{\pgfqpoint{-0.033333in}{-0.033333in}}{\pgfqpoint{0.033333in}{0.033333in}}{%
\pgfpathmoveto{\pgfqpoint{0.000000in}{-0.033333in}}%
\pgfpathcurveto{\pgfqpoint{0.008840in}{-0.033333in}}{\pgfqpoint{0.017319in}{-0.029821in}}{\pgfqpoint{0.023570in}{-0.023570in}}%
\pgfpathcurveto{\pgfqpoint{0.029821in}{-0.017319in}}{\pgfqpoint{0.033333in}{-0.008840in}}{\pgfqpoint{0.033333in}{0.000000in}}%
\pgfpathcurveto{\pgfqpoint{0.033333in}{0.008840in}}{\pgfqpoint{0.029821in}{0.017319in}}{\pgfqpoint{0.023570in}{0.023570in}}%
\pgfpathcurveto{\pgfqpoint{0.017319in}{0.029821in}}{\pgfqpoint{0.008840in}{0.033333in}}{\pgfqpoint{0.000000in}{0.033333in}}%
\pgfpathcurveto{\pgfqpoint{-0.008840in}{0.033333in}}{\pgfqpoint{-0.017319in}{0.029821in}}{\pgfqpoint{-0.023570in}{0.023570in}}%
\pgfpathcurveto{\pgfqpoint{-0.029821in}{0.017319in}}{\pgfqpoint{-0.033333in}{0.008840in}}{\pgfqpoint{-0.033333in}{0.000000in}}%
\pgfpathcurveto{\pgfqpoint{-0.033333in}{-0.008840in}}{\pgfqpoint{-0.029821in}{-0.017319in}}{\pgfqpoint{-0.023570in}{-0.023570in}}%
\pgfpathcurveto{\pgfqpoint{-0.017319in}{-0.029821in}}{\pgfqpoint{-0.008840in}{-0.033333in}}{\pgfqpoint{0.000000in}{-0.033333in}}%
\pgfpathlineto{\pgfqpoint{0.000000in}{-0.033333in}}%
\pgfpathclose%
\pgfusepath{stroke,fill}%
}%
\begin{pgfscope}%
\pgfsys@transformshift{5.369646in}{3.172612in}%
\pgfsys@useobject{currentmarker}{}%
\end{pgfscope}%
\end{pgfscope}%
\begin{pgfscope}%
\pgfpathrectangle{\pgfqpoint{0.765000in}{0.660000in}}{\pgfqpoint{4.620000in}{4.620000in}}%
\pgfusepath{clip}%
\pgfsetrectcap%
\pgfsetroundjoin%
\pgfsetlinewidth{1.204500pt}%
\definecolor{currentstroke}{rgb}{1.000000,0.576471,0.309804}%
\pgfsetstrokecolor{currentstroke}%
\pgfsetdash{}{0pt}%
\pgfpathmoveto{\pgfqpoint{3.995413in}{2.808234in}}%
\pgfusepath{stroke}%
\end{pgfscope}%
\begin{pgfscope}%
\pgfpathrectangle{\pgfqpoint{0.765000in}{0.660000in}}{\pgfqpoint{4.620000in}{4.620000in}}%
\pgfusepath{clip}%
\pgfsetbuttcap%
\pgfsetroundjoin%
\definecolor{currentfill}{rgb}{1.000000,0.576471,0.309804}%
\pgfsetfillcolor{currentfill}%
\pgfsetlinewidth{1.003750pt}%
\definecolor{currentstroke}{rgb}{1.000000,0.576471,0.309804}%
\pgfsetstrokecolor{currentstroke}%
\pgfsetdash{}{0pt}%
\pgfsys@defobject{currentmarker}{\pgfqpoint{-0.033333in}{-0.033333in}}{\pgfqpoint{0.033333in}{0.033333in}}{%
\pgfpathmoveto{\pgfqpoint{0.000000in}{-0.033333in}}%
\pgfpathcurveto{\pgfqpoint{0.008840in}{-0.033333in}}{\pgfqpoint{0.017319in}{-0.029821in}}{\pgfqpoint{0.023570in}{-0.023570in}}%
\pgfpathcurveto{\pgfqpoint{0.029821in}{-0.017319in}}{\pgfqpoint{0.033333in}{-0.008840in}}{\pgfqpoint{0.033333in}{0.000000in}}%
\pgfpathcurveto{\pgfqpoint{0.033333in}{0.008840in}}{\pgfqpoint{0.029821in}{0.017319in}}{\pgfqpoint{0.023570in}{0.023570in}}%
\pgfpathcurveto{\pgfqpoint{0.017319in}{0.029821in}}{\pgfqpoint{0.008840in}{0.033333in}}{\pgfqpoint{0.000000in}{0.033333in}}%
\pgfpathcurveto{\pgfqpoint{-0.008840in}{0.033333in}}{\pgfqpoint{-0.017319in}{0.029821in}}{\pgfqpoint{-0.023570in}{0.023570in}}%
\pgfpathcurveto{\pgfqpoint{-0.029821in}{0.017319in}}{\pgfqpoint{-0.033333in}{0.008840in}}{\pgfqpoint{-0.033333in}{0.000000in}}%
\pgfpathcurveto{\pgfqpoint{-0.033333in}{-0.008840in}}{\pgfqpoint{-0.029821in}{-0.017319in}}{\pgfqpoint{-0.023570in}{-0.023570in}}%
\pgfpathcurveto{\pgfqpoint{-0.017319in}{-0.029821in}}{\pgfqpoint{-0.008840in}{-0.033333in}}{\pgfqpoint{0.000000in}{-0.033333in}}%
\pgfpathlineto{\pgfqpoint{0.000000in}{-0.033333in}}%
\pgfpathclose%
\pgfusepath{stroke,fill}%
}%
\begin{pgfscope}%
\pgfsys@transformshift{3.995413in}{2.808234in}%
\pgfsys@useobject{currentmarker}{}%
\end{pgfscope}%
\end{pgfscope}%
\begin{pgfscope}%
\pgfpathrectangle{\pgfqpoint{0.765000in}{0.660000in}}{\pgfqpoint{4.620000in}{4.620000in}}%
\pgfusepath{clip}%
\pgfsetrectcap%
\pgfsetroundjoin%
\pgfsetlinewidth{1.204500pt}%
\definecolor{currentstroke}{rgb}{1.000000,0.576471,0.309804}%
\pgfsetstrokecolor{currentstroke}%
\pgfsetdash{}{0pt}%
\pgfpathmoveto{\pgfqpoint{3.214844in}{2.619659in}}%
\pgfusepath{stroke}%
\end{pgfscope}%
\begin{pgfscope}%
\pgfpathrectangle{\pgfqpoint{0.765000in}{0.660000in}}{\pgfqpoint{4.620000in}{4.620000in}}%
\pgfusepath{clip}%
\pgfsetbuttcap%
\pgfsetroundjoin%
\definecolor{currentfill}{rgb}{1.000000,0.576471,0.309804}%
\pgfsetfillcolor{currentfill}%
\pgfsetlinewidth{1.003750pt}%
\definecolor{currentstroke}{rgb}{1.000000,0.576471,0.309804}%
\pgfsetstrokecolor{currentstroke}%
\pgfsetdash{}{0pt}%
\pgfsys@defobject{currentmarker}{\pgfqpoint{-0.033333in}{-0.033333in}}{\pgfqpoint{0.033333in}{0.033333in}}{%
\pgfpathmoveto{\pgfqpoint{0.000000in}{-0.033333in}}%
\pgfpathcurveto{\pgfqpoint{0.008840in}{-0.033333in}}{\pgfqpoint{0.017319in}{-0.029821in}}{\pgfqpoint{0.023570in}{-0.023570in}}%
\pgfpathcurveto{\pgfqpoint{0.029821in}{-0.017319in}}{\pgfqpoint{0.033333in}{-0.008840in}}{\pgfqpoint{0.033333in}{0.000000in}}%
\pgfpathcurveto{\pgfqpoint{0.033333in}{0.008840in}}{\pgfqpoint{0.029821in}{0.017319in}}{\pgfqpoint{0.023570in}{0.023570in}}%
\pgfpathcurveto{\pgfqpoint{0.017319in}{0.029821in}}{\pgfqpoint{0.008840in}{0.033333in}}{\pgfqpoint{0.000000in}{0.033333in}}%
\pgfpathcurveto{\pgfqpoint{-0.008840in}{0.033333in}}{\pgfqpoint{-0.017319in}{0.029821in}}{\pgfqpoint{-0.023570in}{0.023570in}}%
\pgfpathcurveto{\pgfqpoint{-0.029821in}{0.017319in}}{\pgfqpoint{-0.033333in}{0.008840in}}{\pgfqpoint{-0.033333in}{0.000000in}}%
\pgfpathcurveto{\pgfqpoint{-0.033333in}{-0.008840in}}{\pgfqpoint{-0.029821in}{-0.017319in}}{\pgfqpoint{-0.023570in}{-0.023570in}}%
\pgfpathcurveto{\pgfqpoint{-0.017319in}{-0.029821in}}{\pgfqpoint{-0.008840in}{-0.033333in}}{\pgfqpoint{0.000000in}{-0.033333in}}%
\pgfpathlineto{\pgfqpoint{0.000000in}{-0.033333in}}%
\pgfpathclose%
\pgfusepath{stroke,fill}%
}%
\begin{pgfscope}%
\pgfsys@transformshift{3.214844in}{2.619659in}%
\pgfsys@useobject{currentmarker}{}%
\end{pgfscope}%
\end{pgfscope}%
\begin{pgfscope}%
\pgfpathrectangle{\pgfqpoint{0.765000in}{0.660000in}}{\pgfqpoint{4.620000in}{4.620000in}}%
\pgfusepath{clip}%
\pgfsetrectcap%
\pgfsetroundjoin%
\pgfsetlinewidth{1.204500pt}%
\definecolor{currentstroke}{rgb}{1.000000,0.576471,0.309804}%
\pgfsetstrokecolor{currentstroke}%
\pgfsetdash{}{0pt}%
\pgfpathmoveto{\pgfqpoint{2.840712in}{2.612490in}}%
\pgfusepath{stroke}%
\end{pgfscope}%
\begin{pgfscope}%
\pgfpathrectangle{\pgfqpoint{0.765000in}{0.660000in}}{\pgfqpoint{4.620000in}{4.620000in}}%
\pgfusepath{clip}%
\pgfsetbuttcap%
\pgfsetroundjoin%
\definecolor{currentfill}{rgb}{1.000000,0.576471,0.309804}%
\pgfsetfillcolor{currentfill}%
\pgfsetlinewidth{1.003750pt}%
\definecolor{currentstroke}{rgb}{1.000000,0.576471,0.309804}%
\pgfsetstrokecolor{currentstroke}%
\pgfsetdash{}{0pt}%
\pgfsys@defobject{currentmarker}{\pgfqpoint{-0.033333in}{-0.033333in}}{\pgfqpoint{0.033333in}{0.033333in}}{%
\pgfpathmoveto{\pgfqpoint{0.000000in}{-0.033333in}}%
\pgfpathcurveto{\pgfqpoint{0.008840in}{-0.033333in}}{\pgfqpoint{0.017319in}{-0.029821in}}{\pgfqpoint{0.023570in}{-0.023570in}}%
\pgfpathcurveto{\pgfqpoint{0.029821in}{-0.017319in}}{\pgfqpoint{0.033333in}{-0.008840in}}{\pgfqpoint{0.033333in}{0.000000in}}%
\pgfpathcurveto{\pgfqpoint{0.033333in}{0.008840in}}{\pgfqpoint{0.029821in}{0.017319in}}{\pgfqpoint{0.023570in}{0.023570in}}%
\pgfpathcurveto{\pgfqpoint{0.017319in}{0.029821in}}{\pgfqpoint{0.008840in}{0.033333in}}{\pgfqpoint{0.000000in}{0.033333in}}%
\pgfpathcurveto{\pgfqpoint{-0.008840in}{0.033333in}}{\pgfqpoint{-0.017319in}{0.029821in}}{\pgfqpoint{-0.023570in}{0.023570in}}%
\pgfpathcurveto{\pgfqpoint{-0.029821in}{0.017319in}}{\pgfqpoint{-0.033333in}{0.008840in}}{\pgfqpoint{-0.033333in}{0.000000in}}%
\pgfpathcurveto{\pgfqpoint{-0.033333in}{-0.008840in}}{\pgfqpoint{-0.029821in}{-0.017319in}}{\pgfqpoint{-0.023570in}{-0.023570in}}%
\pgfpathcurveto{\pgfqpoint{-0.017319in}{-0.029821in}}{\pgfqpoint{-0.008840in}{-0.033333in}}{\pgfqpoint{0.000000in}{-0.033333in}}%
\pgfpathlineto{\pgfqpoint{0.000000in}{-0.033333in}}%
\pgfpathclose%
\pgfusepath{stroke,fill}%
}%
\begin{pgfscope}%
\pgfsys@transformshift{2.840712in}{2.612490in}%
\pgfsys@useobject{currentmarker}{}%
\end{pgfscope}%
\end{pgfscope}%
\begin{pgfscope}%
\pgfpathrectangle{\pgfqpoint{0.765000in}{0.660000in}}{\pgfqpoint{4.620000in}{4.620000in}}%
\pgfusepath{clip}%
\pgfsetrectcap%
\pgfsetroundjoin%
\pgfsetlinewidth{1.204500pt}%
\definecolor{currentstroke}{rgb}{1.000000,0.576471,0.309804}%
\pgfsetstrokecolor{currentstroke}%
\pgfsetdash{}{0pt}%
\pgfpathmoveto{\pgfqpoint{4.655930in}{2.880516in}}%
\pgfusepath{stroke}%
\end{pgfscope}%
\begin{pgfscope}%
\pgfpathrectangle{\pgfqpoint{0.765000in}{0.660000in}}{\pgfqpoint{4.620000in}{4.620000in}}%
\pgfusepath{clip}%
\pgfsetbuttcap%
\pgfsetroundjoin%
\definecolor{currentfill}{rgb}{1.000000,0.576471,0.309804}%
\pgfsetfillcolor{currentfill}%
\pgfsetlinewidth{1.003750pt}%
\definecolor{currentstroke}{rgb}{1.000000,0.576471,0.309804}%
\pgfsetstrokecolor{currentstroke}%
\pgfsetdash{}{0pt}%
\pgfsys@defobject{currentmarker}{\pgfqpoint{-0.033333in}{-0.033333in}}{\pgfqpoint{0.033333in}{0.033333in}}{%
\pgfpathmoveto{\pgfqpoint{0.000000in}{-0.033333in}}%
\pgfpathcurveto{\pgfqpoint{0.008840in}{-0.033333in}}{\pgfqpoint{0.017319in}{-0.029821in}}{\pgfqpoint{0.023570in}{-0.023570in}}%
\pgfpathcurveto{\pgfqpoint{0.029821in}{-0.017319in}}{\pgfqpoint{0.033333in}{-0.008840in}}{\pgfqpoint{0.033333in}{0.000000in}}%
\pgfpathcurveto{\pgfqpoint{0.033333in}{0.008840in}}{\pgfqpoint{0.029821in}{0.017319in}}{\pgfqpoint{0.023570in}{0.023570in}}%
\pgfpathcurveto{\pgfqpoint{0.017319in}{0.029821in}}{\pgfqpoint{0.008840in}{0.033333in}}{\pgfqpoint{0.000000in}{0.033333in}}%
\pgfpathcurveto{\pgfqpoint{-0.008840in}{0.033333in}}{\pgfqpoint{-0.017319in}{0.029821in}}{\pgfqpoint{-0.023570in}{0.023570in}}%
\pgfpathcurveto{\pgfqpoint{-0.029821in}{0.017319in}}{\pgfqpoint{-0.033333in}{0.008840in}}{\pgfqpoint{-0.033333in}{0.000000in}}%
\pgfpathcurveto{\pgfqpoint{-0.033333in}{-0.008840in}}{\pgfqpoint{-0.029821in}{-0.017319in}}{\pgfqpoint{-0.023570in}{-0.023570in}}%
\pgfpathcurveto{\pgfqpoint{-0.017319in}{-0.029821in}}{\pgfqpoint{-0.008840in}{-0.033333in}}{\pgfqpoint{0.000000in}{-0.033333in}}%
\pgfpathlineto{\pgfqpoint{0.000000in}{-0.033333in}}%
\pgfpathclose%
\pgfusepath{stroke,fill}%
}%
\begin{pgfscope}%
\pgfsys@transformshift{4.655930in}{2.880516in}%
\pgfsys@useobject{currentmarker}{}%
\end{pgfscope}%
\end{pgfscope}%
\begin{pgfscope}%
\pgfpathrectangle{\pgfqpoint{0.765000in}{0.660000in}}{\pgfqpoint{4.620000in}{4.620000in}}%
\pgfusepath{clip}%
\pgfsetrectcap%
\pgfsetroundjoin%
\pgfsetlinewidth{1.204500pt}%
\definecolor{currentstroke}{rgb}{1.000000,0.576471,0.309804}%
\pgfsetstrokecolor{currentstroke}%
\pgfsetdash{}{0pt}%
\pgfpathmoveto{\pgfqpoint{2.981139in}{3.132003in}}%
\pgfusepath{stroke}%
\end{pgfscope}%
\begin{pgfscope}%
\pgfpathrectangle{\pgfqpoint{0.765000in}{0.660000in}}{\pgfqpoint{4.620000in}{4.620000in}}%
\pgfusepath{clip}%
\pgfsetbuttcap%
\pgfsetroundjoin%
\definecolor{currentfill}{rgb}{1.000000,0.576471,0.309804}%
\pgfsetfillcolor{currentfill}%
\pgfsetlinewidth{1.003750pt}%
\definecolor{currentstroke}{rgb}{1.000000,0.576471,0.309804}%
\pgfsetstrokecolor{currentstroke}%
\pgfsetdash{}{0pt}%
\pgfsys@defobject{currentmarker}{\pgfqpoint{-0.033333in}{-0.033333in}}{\pgfqpoint{0.033333in}{0.033333in}}{%
\pgfpathmoveto{\pgfqpoint{0.000000in}{-0.033333in}}%
\pgfpathcurveto{\pgfqpoint{0.008840in}{-0.033333in}}{\pgfqpoint{0.017319in}{-0.029821in}}{\pgfqpoint{0.023570in}{-0.023570in}}%
\pgfpathcurveto{\pgfqpoint{0.029821in}{-0.017319in}}{\pgfqpoint{0.033333in}{-0.008840in}}{\pgfqpoint{0.033333in}{0.000000in}}%
\pgfpathcurveto{\pgfqpoint{0.033333in}{0.008840in}}{\pgfqpoint{0.029821in}{0.017319in}}{\pgfqpoint{0.023570in}{0.023570in}}%
\pgfpathcurveto{\pgfqpoint{0.017319in}{0.029821in}}{\pgfqpoint{0.008840in}{0.033333in}}{\pgfqpoint{0.000000in}{0.033333in}}%
\pgfpathcurveto{\pgfqpoint{-0.008840in}{0.033333in}}{\pgfqpoint{-0.017319in}{0.029821in}}{\pgfqpoint{-0.023570in}{0.023570in}}%
\pgfpathcurveto{\pgfqpoint{-0.029821in}{0.017319in}}{\pgfqpoint{-0.033333in}{0.008840in}}{\pgfqpoint{-0.033333in}{0.000000in}}%
\pgfpathcurveto{\pgfqpoint{-0.033333in}{-0.008840in}}{\pgfqpoint{-0.029821in}{-0.017319in}}{\pgfqpoint{-0.023570in}{-0.023570in}}%
\pgfpathcurveto{\pgfqpoint{-0.017319in}{-0.029821in}}{\pgfqpoint{-0.008840in}{-0.033333in}}{\pgfqpoint{0.000000in}{-0.033333in}}%
\pgfpathlineto{\pgfqpoint{0.000000in}{-0.033333in}}%
\pgfpathclose%
\pgfusepath{stroke,fill}%
}%
\begin{pgfscope}%
\pgfsys@transformshift{2.981139in}{3.132003in}%
\pgfsys@useobject{currentmarker}{}%
\end{pgfscope}%
\end{pgfscope}%
\begin{pgfscope}%
\pgfpathrectangle{\pgfqpoint{0.765000in}{0.660000in}}{\pgfqpoint{4.620000in}{4.620000in}}%
\pgfusepath{clip}%
\pgfsetrectcap%
\pgfsetroundjoin%
\pgfsetlinewidth{1.204500pt}%
\definecolor{currentstroke}{rgb}{1.000000,0.576471,0.309804}%
\pgfsetstrokecolor{currentstroke}%
\pgfsetdash{}{0pt}%
\pgfpathmoveto{\pgfqpoint{2.938390in}{3.094295in}}%
\pgfusepath{stroke}%
\end{pgfscope}%
\begin{pgfscope}%
\pgfpathrectangle{\pgfqpoint{0.765000in}{0.660000in}}{\pgfqpoint{4.620000in}{4.620000in}}%
\pgfusepath{clip}%
\pgfsetbuttcap%
\pgfsetroundjoin%
\definecolor{currentfill}{rgb}{1.000000,0.576471,0.309804}%
\pgfsetfillcolor{currentfill}%
\pgfsetlinewidth{1.003750pt}%
\definecolor{currentstroke}{rgb}{1.000000,0.576471,0.309804}%
\pgfsetstrokecolor{currentstroke}%
\pgfsetdash{}{0pt}%
\pgfsys@defobject{currentmarker}{\pgfqpoint{-0.033333in}{-0.033333in}}{\pgfqpoint{0.033333in}{0.033333in}}{%
\pgfpathmoveto{\pgfqpoint{0.000000in}{-0.033333in}}%
\pgfpathcurveto{\pgfqpoint{0.008840in}{-0.033333in}}{\pgfqpoint{0.017319in}{-0.029821in}}{\pgfqpoint{0.023570in}{-0.023570in}}%
\pgfpathcurveto{\pgfqpoint{0.029821in}{-0.017319in}}{\pgfqpoint{0.033333in}{-0.008840in}}{\pgfqpoint{0.033333in}{0.000000in}}%
\pgfpathcurveto{\pgfqpoint{0.033333in}{0.008840in}}{\pgfqpoint{0.029821in}{0.017319in}}{\pgfqpoint{0.023570in}{0.023570in}}%
\pgfpathcurveto{\pgfqpoint{0.017319in}{0.029821in}}{\pgfqpoint{0.008840in}{0.033333in}}{\pgfqpoint{0.000000in}{0.033333in}}%
\pgfpathcurveto{\pgfqpoint{-0.008840in}{0.033333in}}{\pgfqpoint{-0.017319in}{0.029821in}}{\pgfqpoint{-0.023570in}{0.023570in}}%
\pgfpathcurveto{\pgfqpoint{-0.029821in}{0.017319in}}{\pgfqpoint{-0.033333in}{0.008840in}}{\pgfqpoint{-0.033333in}{0.000000in}}%
\pgfpathcurveto{\pgfqpoint{-0.033333in}{-0.008840in}}{\pgfqpoint{-0.029821in}{-0.017319in}}{\pgfqpoint{-0.023570in}{-0.023570in}}%
\pgfpathcurveto{\pgfqpoint{-0.017319in}{-0.029821in}}{\pgfqpoint{-0.008840in}{-0.033333in}}{\pgfqpoint{0.000000in}{-0.033333in}}%
\pgfpathlineto{\pgfqpoint{0.000000in}{-0.033333in}}%
\pgfpathclose%
\pgfusepath{stroke,fill}%
}%
\begin{pgfscope}%
\pgfsys@transformshift{2.938390in}{3.094295in}%
\pgfsys@useobject{currentmarker}{}%
\end{pgfscope}%
\end{pgfscope}%
\begin{pgfscope}%
\pgfpathrectangle{\pgfqpoint{0.765000in}{0.660000in}}{\pgfqpoint{4.620000in}{4.620000in}}%
\pgfusepath{clip}%
\pgfsetrectcap%
\pgfsetroundjoin%
\pgfsetlinewidth{1.204500pt}%
\definecolor{currentstroke}{rgb}{1.000000,0.576471,0.309804}%
\pgfsetstrokecolor{currentstroke}%
\pgfsetdash{}{0pt}%
\pgfpathmoveto{\pgfqpoint{2.751359in}{3.081578in}}%
\pgfusepath{stroke}%
\end{pgfscope}%
\begin{pgfscope}%
\pgfpathrectangle{\pgfqpoint{0.765000in}{0.660000in}}{\pgfqpoint{4.620000in}{4.620000in}}%
\pgfusepath{clip}%
\pgfsetbuttcap%
\pgfsetroundjoin%
\definecolor{currentfill}{rgb}{1.000000,0.576471,0.309804}%
\pgfsetfillcolor{currentfill}%
\pgfsetlinewidth{1.003750pt}%
\definecolor{currentstroke}{rgb}{1.000000,0.576471,0.309804}%
\pgfsetstrokecolor{currentstroke}%
\pgfsetdash{}{0pt}%
\pgfsys@defobject{currentmarker}{\pgfqpoint{-0.033333in}{-0.033333in}}{\pgfqpoint{0.033333in}{0.033333in}}{%
\pgfpathmoveto{\pgfqpoint{0.000000in}{-0.033333in}}%
\pgfpathcurveto{\pgfqpoint{0.008840in}{-0.033333in}}{\pgfqpoint{0.017319in}{-0.029821in}}{\pgfqpoint{0.023570in}{-0.023570in}}%
\pgfpathcurveto{\pgfqpoint{0.029821in}{-0.017319in}}{\pgfqpoint{0.033333in}{-0.008840in}}{\pgfqpoint{0.033333in}{0.000000in}}%
\pgfpathcurveto{\pgfqpoint{0.033333in}{0.008840in}}{\pgfqpoint{0.029821in}{0.017319in}}{\pgfqpoint{0.023570in}{0.023570in}}%
\pgfpathcurveto{\pgfqpoint{0.017319in}{0.029821in}}{\pgfqpoint{0.008840in}{0.033333in}}{\pgfqpoint{0.000000in}{0.033333in}}%
\pgfpathcurveto{\pgfqpoint{-0.008840in}{0.033333in}}{\pgfqpoint{-0.017319in}{0.029821in}}{\pgfqpoint{-0.023570in}{0.023570in}}%
\pgfpathcurveto{\pgfqpoint{-0.029821in}{0.017319in}}{\pgfqpoint{-0.033333in}{0.008840in}}{\pgfqpoint{-0.033333in}{0.000000in}}%
\pgfpathcurveto{\pgfqpoint{-0.033333in}{-0.008840in}}{\pgfqpoint{-0.029821in}{-0.017319in}}{\pgfqpoint{-0.023570in}{-0.023570in}}%
\pgfpathcurveto{\pgfqpoint{-0.017319in}{-0.029821in}}{\pgfqpoint{-0.008840in}{-0.033333in}}{\pgfqpoint{0.000000in}{-0.033333in}}%
\pgfpathlineto{\pgfqpoint{0.000000in}{-0.033333in}}%
\pgfpathclose%
\pgfusepath{stroke,fill}%
}%
\begin{pgfscope}%
\pgfsys@transformshift{2.751359in}{3.081578in}%
\pgfsys@useobject{currentmarker}{}%
\end{pgfscope}%
\end{pgfscope}%
\begin{pgfscope}%
\pgfpathrectangle{\pgfqpoint{0.765000in}{0.660000in}}{\pgfqpoint{4.620000in}{4.620000in}}%
\pgfusepath{clip}%
\pgfsetrectcap%
\pgfsetroundjoin%
\pgfsetlinewidth{1.204500pt}%
\definecolor{currentstroke}{rgb}{1.000000,0.576471,0.309804}%
\pgfsetstrokecolor{currentstroke}%
\pgfsetdash{}{0pt}%
\pgfpathmoveto{\pgfqpoint{3.588569in}{2.415841in}}%
\pgfusepath{stroke}%
\end{pgfscope}%
\begin{pgfscope}%
\pgfpathrectangle{\pgfqpoint{0.765000in}{0.660000in}}{\pgfqpoint{4.620000in}{4.620000in}}%
\pgfusepath{clip}%
\pgfsetbuttcap%
\pgfsetroundjoin%
\definecolor{currentfill}{rgb}{1.000000,0.576471,0.309804}%
\pgfsetfillcolor{currentfill}%
\pgfsetlinewidth{1.003750pt}%
\definecolor{currentstroke}{rgb}{1.000000,0.576471,0.309804}%
\pgfsetstrokecolor{currentstroke}%
\pgfsetdash{}{0pt}%
\pgfsys@defobject{currentmarker}{\pgfqpoint{-0.033333in}{-0.033333in}}{\pgfqpoint{0.033333in}{0.033333in}}{%
\pgfpathmoveto{\pgfqpoint{0.000000in}{-0.033333in}}%
\pgfpathcurveto{\pgfqpoint{0.008840in}{-0.033333in}}{\pgfqpoint{0.017319in}{-0.029821in}}{\pgfqpoint{0.023570in}{-0.023570in}}%
\pgfpathcurveto{\pgfqpoint{0.029821in}{-0.017319in}}{\pgfqpoint{0.033333in}{-0.008840in}}{\pgfqpoint{0.033333in}{0.000000in}}%
\pgfpathcurveto{\pgfqpoint{0.033333in}{0.008840in}}{\pgfqpoint{0.029821in}{0.017319in}}{\pgfqpoint{0.023570in}{0.023570in}}%
\pgfpathcurveto{\pgfqpoint{0.017319in}{0.029821in}}{\pgfqpoint{0.008840in}{0.033333in}}{\pgfqpoint{0.000000in}{0.033333in}}%
\pgfpathcurveto{\pgfqpoint{-0.008840in}{0.033333in}}{\pgfqpoint{-0.017319in}{0.029821in}}{\pgfqpoint{-0.023570in}{0.023570in}}%
\pgfpathcurveto{\pgfqpoint{-0.029821in}{0.017319in}}{\pgfqpoint{-0.033333in}{0.008840in}}{\pgfqpoint{-0.033333in}{0.000000in}}%
\pgfpathcurveto{\pgfqpoint{-0.033333in}{-0.008840in}}{\pgfqpoint{-0.029821in}{-0.017319in}}{\pgfqpoint{-0.023570in}{-0.023570in}}%
\pgfpathcurveto{\pgfqpoint{-0.017319in}{-0.029821in}}{\pgfqpoint{-0.008840in}{-0.033333in}}{\pgfqpoint{0.000000in}{-0.033333in}}%
\pgfpathlineto{\pgfqpoint{0.000000in}{-0.033333in}}%
\pgfpathclose%
\pgfusepath{stroke,fill}%
}%
\begin{pgfscope}%
\pgfsys@transformshift{3.588569in}{2.415841in}%
\pgfsys@useobject{currentmarker}{}%
\end{pgfscope}%
\end{pgfscope}%
\begin{pgfscope}%
\pgfpathrectangle{\pgfqpoint{0.765000in}{0.660000in}}{\pgfqpoint{4.620000in}{4.620000in}}%
\pgfusepath{clip}%
\pgfsetrectcap%
\pgfsetroundjoin%
\pgfsetlinewidth{1.204500pt}%
\definecolor{currentstroke}{rgb}{1.000000,0.576471,0.309804}%
\pgfsetstrokecolor{currentstroke}%
\pgfsetdash{}{0pt}%
\pgfpathmoveto{\pgfqpoint{2.987670in}{2.528670in}}%
\pgfusepath{stroke}%
\end{pgfscope}%
\begin{pgfscope}%
\pgfpathrectangle{\pgfqpoint{0.765000in}{0.660000in}}{\pgfqpoint{4.620000in}{4.620000in}}%
\pgfusepath{clip}%
\pgfsetbuttcap%
\pgfsetroundjoin%
\definecolor{currentfill}{rgb}{1.000000,0.576471,0.309804}%
\pgfsetfillcolor{currentfill}%
\pgfsetlinewidth{1.003750pt}%
\definecolor{currentstroke}{rgb}{1.000000,0.576471,0.309804}%
\pgfsetstrokecolor{currentstroke}%
\pgfsetdash{}{0pt}%
\pgfsys@defobject{currentmarker}{\pgfqpoint{-0.033333in}{-0.033333in}}{\pgfqpoint{0.033333in}{0.033333in}}{%
\pgfpathmoveto{\pgfqpoint{0.000000in}{-0.033333in}}%
\pgfpathcurveto{\pgfqpoint{0.008840in}{-0.033333in}}{\pgfqpoint{0.017319in}{-0.029821in}}{\pgfqpoint{0.023570in}{-0.023570in}}%
\pgfpathcurveto{\pgfqpoint{0.029821in}{-0.017319in}}{\pgfqpoint{0.033333in}{-0.008840in}}{\pgfqpoint{0.033333in}{0.000000in}}%
\pgfpathcurveto{\pgfqpoint{0.033333in}{0.008840in}}{\pgfqpoint{0.029821in}{0.017319in}}{\pgfqpoint{0.023570in}{0.023570in}}%
\pgfpathcurveto{\pgfqpoint{0.017319in}{0.029821in}}{\pgfqpoint{0.008840in}{0.033333in}}{\pgfqpoint{0.000000in}{0.033333in}}%
\pgfpathcurveto{\pgfqpoint{-0.008840in}{0.033333in}}{\pgfqpoint{-0.017319in}{0.029821in}}{\pgfqpoint{-0.023570in}{0.023570in}}%
\pgfpathcurveto{\pgfqpoint{-0.029821in}{0.017319in}}{\pgfqpoint{-0.033333in}{0.008840in}}{\pgfqpoint{-0.033333in}{0.000000in}}%
\pgfpathcurveto{\pgfqpoint{-0.033333in}{-0.008840in}}{\pgfqpoint{-0.029821in}{-0.017319in}}{\pgfqpoint{-0.023570in}{-0.023570in}}%
\pgfpathcurveto{\pgfqpoint{-0.017319in}{-0.029821in}}{\pgfqpoint{-0.008840in}{-0.033333in}}{\pgfqpoint{0.000000in}{-0.033333in}}%
\pgfpathlineto{\pgfqpoint{0.000000in}{-0.033333in}}%
\pgfpathclose%
\pgfusepath{stroke,fill}%
}%
\begin{pgfscope}%
\pgfsys@transformshift{2.987670in}{2.528670in}%
\pgfsys@useobject{currentmarker}{}%
\end{pgfscope}%
\end{pgfscope}%
\begin{pgfscope}%
\pgfpathrectangle{\pgfqpoint{0.765000in}{0.660000in}}{\pgfqpoint{4.620000in}{4.620000in}}%
\pgfusepath{clip}%
\pgfsetrectcap%
\pgfsetroundjoin%
\pgfsetlinewidth{1.204500pt}%
\definecolor{currentstroke}{rgb}{1.000000,0.576471,0.309804}%
\pgfsetstrokecolor{currentstroke}%
\pgfsetdash{}{0pt}%
\pgfpathmoveto{\pgfqpoint{4.969267in}{3.109766in}}%
\pgfusepath{stroke}%
\end{pgfscope}%
\begin{pgfscope}%
\pgfpathrectangle{\pgfqpoint{0.765000in}{0.660000in}}{\pgfqpoint{4.620000in}{4.620000in}}%
\pgfusepath{clip}%
\pgfsetbuttcap%
\pgfsetroundjoin%
\definecolor{currentfill}{rgb}{1.000000,0.576471,0.309804}%
\pgfsetfillcolor{currentfill}%
\pgfsetlinewidth{1.003750pt}%
\definecolor{currentstroke}{rgb}{1.000000,0.576471,0.309804}%
\pgfsetstrokecolor{currentstroke}%
\pgfsetdash{}{0pt}%
\pgfsys@defobject{currentmarker}{\pgfqpoint{-0.033333in}{-0.033333in}}{\pgfqpoint{0.033333in}{0.033333in}}{%
\pgfpathmoveto{\pgfqpoint{0.000000in}{-0.033333in}}%
\pgfpathcurveto{\pgfqpoint{0.008840in}{-0.033333in}}{\pgfqpoint{0.017319in}{-0.029821in}}{\pgfqpoint{0.023570in}{-0.023570in}}%
\pgfpathcurveto{\pgfqpoint{0.029821in}{-0.017319in}}{\pgfqpoint{0.033333in}{-0.008840in}}{\pgfqpoint{0.033333in}{0.000000in}}%
\pgfpathcurveto{\pgfqpoint{0.033333in}{0.008840in}}{\pgfqpoint{0.029821in}{0.017319in}}{\pgfqpoint{0.023570in}{0.023570in}}%
\pgfpathcurveto{\pgfqpoint{0.017319in}{0.029821in}}{\pgfqpoint{0.008840in}{0.033333in}}{\pgfqpoint{0.000000in}{0.033333in}}%
\pgfpathcurveto{\pgfqpoint{-0.008840in}{0.033333in}}{\pgfqpoint{-0.017319in}{0.029821in}}{\pgfqpoint{-0.023570in}{0.023570in}}%
\pgfpathcurveto{\pgfqpoint{-0.029821in}{0.017319in}}{\pgfqpoint{-0.033333in}{0.008840in}}{\pgfqpoint{-0.033333in}{0.000000in}}%
\pgfpathcurveto{\pgfqpoint{-0.033333in}{-0.008840in}}{\pgfqpoint{-0.029821in}{-0.017319in}}{\pgfqpoint{-0.023570in}{-0.023570in}}%
\pgfpathcurveto{\pgfqpoint{-0.017319in}{-0.029821in}}{\pgfqpoint{-0.008840in}{-0.033333in}}{\pgfqpoint{0.000000in}{-0.033333in}}%
\pgfpathlineto{\pgfqpoint{0.000000in}{-0.033333in}}%
\pgfpathclose%
\pgfusepath{stroke,fill}%
}%
\begin{pgfscope}%
\pgfsys@transformshift{4.969267in}{3.109766in}%
\pgfsys@useobject{currentmarker}{}%
\end{pgfscope}%
\end{pgfscope}%
\begin{pgfscope}%
\pgfpathrectangle{\pgfqpoint{0.765000in}{0.660000in}}{\pgfqpoint{4.620000in}{4.620000in}}%
\pgfusepath{clip}%
\pgfsetrectcap%
\pgfsetroundjoin%
\pgfsetlinewidth{1.204500pt}%
\definecolor{currentstroke}{rgb}{1.000000,0.576471,0.309804}%
\pgfsetstrokecolor{currentstroke}%
\pgfsetdash{}{0pt}%
\pgfpathmoveto{\pgfqpoint{2.565224in}{2.574721in}}%
\pgfusepath{stroke}%
\end{pgfscope}%
\begin{pgfscope}%
\pgfpathrectangle{\pgfqpoint{0.765000in}{0.660000in}}{\pgfqpoint{4.620000in}{4.620000in}}%
\pgfusepath{clip}%
\pgfsetbuttcap%
\pgfsetroundjoin%
\definecolor{currentfill}{rgb}{1.000000,0.576471,0.309804}%
\pgfsetfillcolor{currentfill}%
\pgfsetlinewidth{1.003750pt}%
\definecolor{currentstroke}{rgb}{1.000000,0.576471,0.309804}%
\pgfsetstrokecolor{currentstroke}%
\pgfsetdash{}{0pt}%
\pgfsys@defobject{currentmarker}{\pgfqpoint{-0.033333in}{-0.033333in}}{\pgfqpoint{0.033333in}{0.033333in}}{%
\pgfpathmoveto{\pgfqpoint{0.000000in}{-0.033333in}}%
\pgfpathcurveto{\pgfqpoint{0.008840in}{-0.033333in}}{\pgfqpoint{0.017319in}{-0.029821in}}{\pgfqpoint{0.023570in}{-0.023570in}}%
\pgfpathcurveto{\pgfqpoint{0.029821in}{-0.017319in}}{\pgfqpoint{0.033333in}{-0.008840in}}{\pgfqpoint{0.033333in}{0.000000in}}%
\pgfpathcurveto{\pgfqpoint{0.033333in}{0.008840in}}{\pgfqpoint{0.029821in}{0.017319in}}{\pgfqpoint{0.023570in}{0.023570in}}%
\pgfpathcurveto{\pgfqpoint{0.017319in}{0.029821in}}{\pgfqpoint{0.008840in}{0.033333in}}{\pgfqpoint{0.000000in}{0.033333in}}%
\pgfpathcurveto{\pgfqpoint{-0.008840in}{0.033333in}}{\pgfqpoint{-0.017319in}{0.029821in}}{\pgfqpoint{-0.023570in}{0.023570in}}%
\pgfpathcurveto{\pgfqpoint{-0.029821in}{0.017319in}}{\pgfqpoint{-0.033333in}{0.008840in}}{\pgfqpoint{-0.033333in}{0.000000in}}%
\pgfpathcurveto{\pgfqpoint{-0.033333in}{-0.008840in}}{\pgfqpoint{-0.029821in}{-0.017319in}}{\pgfqpoint{-0.023570in}{-0.023570in}}%
\pgfpathcurveto{\pgfqpoint{-0.017319in}{-0.029821in}}{\pgfqpoint{-0.008840in}{-0.033333in}}{\pgfqpoint{0.000000in}{-0.033333in}}%
\pgfpathlineto{\pgfqpoint{0.000000in}{-0.033333in}}%
\pgfpathclose%
\pgfusepath{stroke,fill}%
}%
\begin{pgfscope}%
\pgfsys@transformshift{2.565224in}{2.574721in}%
\pgfsys@useobject{currentmarker}{}%
\end{pgfscope}%
\end{pgfscope}%
\begin{pgfscope}%
\pgfpathrectangle{\pgfqpoint{0.765000in}{0.660000in}}{\pgfqpoint{4.620000in}{4.620000in}}%
\pgfusepath{clip}%
\pgfsetrectcap%
\pgfsetroundjoin%
\pgfsetlinewidth{1.204500pt}%
\definecolor{currentstroke}{rgb}{1.000000,0.576471,0.309804}%
\pgfsetstrokecolor{currentstroke}%
\pgfsetdash{}{0pt}%
\pgfpathmoveto{\pgfqpoint{4.583410in}{2.939705in}}%
\pgfusepath{stroke}%
\end{pgfscope}%
\begin{pgfscope}%
\pgfpathrectangle{\pgfqpoint{0.765000in}{0.660000in}}{\pgfqpoint{4.620000in}{4.620000in}}%
\pgfusepath{clip}%
\pgfsetbuttcap%
\pgfsetroundjoin%
\definecolor{currentfill}{rgb}{1.000000,0.576471,0.309804}%
\pgfsetfillcolor{currentfill}%
\pgfsetlinewidth{1.003750pt}%
\definecolor{currentstroke}{rgb}{1.000000,0.576471,0.309804}%
\pgfsetstrokecolor{currentstroke}%
\pgfsetdash{}{0pt}%
\pgfsys@defobject{currentmarker}{\pgfqpoint{-0.033333in}{-0.033333in}}{\pgfqpoint{0.033333in}{0.033333in}}{%
\pgfpathmoveto{\pgfqpoint{0.000000in}{-0.033333in}}%
\pgfpathcurveto{\pgfqpoint{0.008840in}{-0.033333in}}{\pgfqpoint{0.017319in}{-0.029821in}}{\pgfqpoint{0.023570in}{-0.023570in}}%
\pgfpathcurveto{\pgfqpoint{0.029821in}{-0.017319in}}{\pgfqpoint{0.033333in}{-0.008840in}}{\pgfqpoint{0.033333in}{0.000000in}}%
\pgfpathcurveto{\pgfqpoint{0.033333in}{0.008840in}}{\pgfqpoint{0.029821in}{0.017319in}}{\pgfqpoint{0.023570in}{0.023570in}}%
\pgfpathcurveto{\pgfqpoint{0.017319in}{0.029821in}}{\pgfqpoint{0.008840in}{0.033333in}}{\pgfqpoint{0.000000in}{0.033333in}}%
\pgfpathcurveto{\pgfqpoint{-0.008840in}{0.033333in}}{\pgfqpoint{-0.017319in}{0.029821in}}{\pgfqpoint{-0.023570in}{0.023570in}}%
\pgfpathcurveto{\pgfqpoint{-0.029821in}{0.017319in}}{\pgfqpoint{-0.033333in}{0.008840in}}{\pgfqpoint{-0.033333in}{0.000000in}}%
\pgfpathcurveto{\pgfqpoint{-0.033333in}{-0.008840in}}{\pgfqpoint{-0.029821in}{-0.017319in}}{\pgfqpoint{-0.023570in}{-0.023570in}}%
\pgfpathcurveto{\pgfqpoint{-0.017319in}{-0.029821in}}{\pgfqpoint{-0.008840in}{-0.033333in}}{\pgfqpoint{0.000000in}{-0.033333in}}%
\pgfpathlineto{\pgfqpoint{0.000000in}{-0.033333in}}%
\pgfpathclose%
\pgfusepath{stroke,fill}%
}%
\begin{pgfscope}%
\pgfsys@transformshift{4.583410in}{2.939705in}%
\pgfsys@useobject{currentmarker}{}%
\end{pgfscope}%
\end{pgfscope}%
\begin{pgfscope}%
\pgfpathrectangle{\pgfqpoint{0.765000in}{0.660000in}}{\pgfqpoint{4.620000in}{4.620000in}}%
\pgfusepath{clip}%
\pgfsetrectcap%
\pgfsetroundjoin%
\pgfsetlinewidth{1.204500pt}%
\definecolor{currentstroke}{rgb}{1.000000,0.576471,0.309804}%
\pgfsetstrokecolor{currentstroke}%
\pgfsetdash{}{0pt}%
\pgfpathmoveto{\pgfqpoint{3.307753in}{2.846877in}}%
\pgfusepath{stroke}%
\end{pgfscope}%
\begin{pgfscope}%
\pgfpathrectangle{\pgfqpoint{0.765000in}{0.660000in}}{\pgfqpoint{4.620000in}{4.620000in}}%
\pgfusepath{clip}%
\pgfsetbuttcap%
\pgfsetroundjoin%
\definecolor{currentfill}{rgb}{1.000000,0.576471,0.309804}%
\pgfsetfillcolor{currentfill}%
\pgfsetlinewidth{1.003750pt}%
\definecolor{currentstroke}{rgb}{1.000000,0.576471,0.309804}%
\pgfsetstrokecolor{currentstroke}%
\pgfsetdash{}{0pt}%
\pgfsys@defobject{currentmarker}{\pgfqpoint{-0.033333in}{-0.033333in}}{\pgfqpoint{0.033333in}{0.033333in}}{%
\pgfpathmoveto{\pgfqpoint{0.000000in}{-0.033333in}}%
\pgfpathcurveto{\pgfqpoint{0.008840in}{-0.033333in}}{\pgfqpoint{0.017319in}{-0.029821in}}{\pgfqpoint{0.023570in}{-0.023570in}}%
\pgfpathcurveto{\pgfqpoint{0.029821in}{-0.017319in}}{\pgfqpoint{0.033333in}{-0.008840in}}{\pgfqpoint{0.033333in}{0.000000in}}%
\pgfpathcurveto{\pgfqpoint{0.033333in}{0.008840in}}{\pgfqpoint{0.029821in}{0.017319in}}{\pgfqpoint{0.023570in}{0.023570in}}%
\pgfpathcurveto{\pgfqpoint{0.017319in}{0.029821in}}{\pgfqpoint{0.008840in}{0.033333in}}{\pgfqpoint{0.000000in}{0.033333in}}%
\pgfpathcurveto{\pgfqpoint{-0.008840in}{0.033333in}}{\pgfqpoint{-0.017319in}{0.029821in}}{\pgfqpoint{-0.023570in}{0.023570in}}%
\pgfpathcurveto{\pgfqpoint{-0.029821in}{0.017319in}}{\pgfqpoint{-0.033333in}{0.008840in}}{\pgfqpoint{-0.033333in}{0.000000in}}%
\pgfpathcurveto{\pgfqpoint{-0.033333in}{-0.008840in}}{\pgfqpoint{-0.029821in}{-0.017319in}}{\pgfqpoint{-0.023570in}{-0.023570in}}%
\pgfpathcurveto{\pgfqpoint{-0.017319in}{-0.029821in}}{\pgfqpoint{-0.008840in}{-0.033333in}}{\pgfqpoint{0.000000in}{-0.033333in}}%
\pgfpathlineto{\pgfqpoint{0.000000in}{-0.033333in}}%
\pgfpathclose%
\pgfusepath{stroke,fill}%
}%
\begin{pgfscope}%
\pgfsys@transformshift{3.307753in}{2.846877in}%
\pgfsys@useobject{currentmarker}{}%
\end{pgfscope}%
\end{pgfscope}%
\begin{pgfscope}%
\pgfpathrectangle{\pgfqpoint{0.765000in}{0.660000in}}{\pgfqpoint{4.620000in}{4.620000in}}%
\pgfusepath{clip}%
\pgfsetrectcap%
\pgfsetroundjoin%
\pgfsetlinewidth{1.204500pt}%
\definecolor{currentstroke}{rgb}{1.000000,0.576471,0.309804}%
\pgfsetstrokecolor{currentstroke}%
\pgfsetdash{}{0pt}%
\pgfusepath{stroke}%
\end{pgfscope}%
\begin{pgfscope}%
\pgfpathrectangle{\pgfqpoint{0.765000in}{0.660000in}}{\pgfqpoint{4.620000in}{4.620000in}}%
\pgfusepath{clip}%
\pgfsetbuttcap%
\pgfsetroundjoin%
\definecolor{currentfill}{rgb}{1.000000,0.576471,0.309804}%
\pgfsetfillcolor{currentfill}%
\pgfsetlinewidth{1.003750pt}%
\definecolor{currentstroke}{rgb}{1.000000,0.576471,0.309804}%
\pgfsetstrokecolor{currentstroke}%
\pgfsetdash{}{0pt}%
\pgfsys@defobject{currentmarker}{\pgfqpoint{-0.033333in}{-0.033333in}}{\pgfqpoint{0.033333in}{0.033333in}}{%
\pgfpathmoveto{\pgfqpoint{0.000000in}{-0.033333in}}%
\pgfpathcurveto{\pgfqpoint{0.008840in}{-0.033333in}}{\pgfqpoint{0.017319in}{-0.029821in}}{\pgfqpoint{0.023570in}{-0.023570in}}%
\pgfpathcurveto{\pgfqpoint{0.029821in}{-0.017319in}}{\pgfqpoint{0.033333in}{-0.008840in}}{\pgfqpoint{0.033333in}{0.000000in}}%
\pgfpathcurveto{\pgfqpoint{0.033333in}{0.008840in}}{\pgfqpoint{0.029821in}{0.017319in}}{\pgfqpoint{0.023570in}{0.023570in}}%
\pgfpathcurveto{\pgfqpoint{0.017319in}{0.029821in}}{\pgfqpoint{0.008840in}{0.033333in}}{\pgfqpoint{0.000000in}{0.033333in}}%
\pgfpathcurveto{\pgfqpoint{-0.008840in}{0.033333in}}{\pgfqpoint{-0.017319in}{0.029821in}}{\pgfqpoint{-0.023570in}{0.023570in}}%
\pgfpathcurveto{\pgfqpoint{-0.029821in}{0.017319in}}{\pgfqpoint{-0.033333in}{0.008840in}}{\pgfqpoint{-0.033333in}{0.000000in}}%
\pgfpathcurveto{\pgfqpoint{-0.033333in}{-0.008840in}}{\pgfqpoint{-0.029821in}{-0.017319in}}{\pgfqpoint{-0.023570in}{-0.023570in}}%
\pgfpathcurveto{\pgfqpoint{-0.017319in}{-0.029821in}}{\pgfqpoint{-0.008840in}{-0.033333in}}{\pgfqpoint{0.000000in}{-0.033333in}}%
\pgfpathlineto{\pgfqpoint{0.000000in}{-0.033333in}}%
\pgfpathclose%
\pgfusepath{stroke,fill}%
}%
\begin{pgfscope}%
\pgfsys@transformshift{5.415615in}{3.242173in}%
\pgfsys@useobject{currentmarker}{}%
\end{pgfscope}%
\end{pgfscope}%
\begin{pgfscope}%
\pgfpathrectangle{\pgfqpoint{0.765000in}{0.660000in}}{\pgfqpoint{4.620000in}{4.620000in}}%
\pgfusepath{clip}%
\pgfsetrectcap%
\pgfsetroundjoin%
\pgfsetlinewidth{1.204500pt}%
\definecolor{currentstroke}{rgb}{1.000000,0.576471,0.309804}%
\pgfsetstrokecolor{currentstroke}%
\pgfsetdash{}{0pt}%
\pgfpathmoveto{\pgfqpoint{2.144482in}{2.826821in}}%
\pgfusepath{stroke}%
\end{pgfscope}%
\begin{pgfscope}%
\pgfpathrectangle{\pgfqpoint{0.765000in}{0.660000in}}{\pgfqpoint{4.620000in}{4.620000in}}%
\pgfusepath{clip}%
\pgfsetbuttcap%
\pgfsetroundjoin%
\definecolor{currentfill}{rgb}{1.000000,0.576471,0.309804}%
\pgfsetfillcolor{currentfill}%
\pgfsetlinewidth{1.003750pt}%
\definecolor{currentstroke}{rgb}{1.000000,0.576471,0.309804}%
\pgfsetstrokecolor{currentstroke}%
\pgfsetdash{}{0pt}%
\pgfsys@defobject{currentmarker}{\pgfqpoint{-0.033333in}{-0.033333in}}{\pgfqpoint{0.033333in}{0.033333in}}{%
\pgfpathmoveto{\pgfqpoint{0.000000in}{-0.033333in}}%
\pgfpathcurveto{\pgfqpoint{0.008840in}{-0.033333in}}{\pgfqpoint{0.017319in}{-0.029821in}}{\pgfqpoint{0.023570in}{-0.023570in}}%
\pgfpathcurveto{\pgfqpoint{0.029821in}{-0.017319in}}{\pgfqpoint{0.033333in}{-0.008840in}}{\pgfqpoint{0.033333in}{0.000000in}}%
\pgfpathcurveto{\pgfqpoint{0.033333in}{0.008840in}}{\pgfqpoint{0.029821in}{0.017319in}}{\pgfqpoint{0.023570in}{0.023570in}}%
\pgfpathcurveto{\pgfqpoint{0.017319in}{0.029821in}}{\pgfqpoint{0.008840in}{0.033333in}}{\pgfqpoint{0.000000in}{0.033333in}}%
\pgfpathcurveto{\pgfqpoint{-0.008840in}{0.033333in}}{\pgfqpoint{-0.017319in}{0.029821in}}{\pgfqpoint{-0.023570in}{0.023570in}}%
\pgfpathcurveto{\pgfqpoint{-0.029821in}{0.017319in}}{\pgfqpoint{-0.033333in}{0.008840in}}{\pgfqpoint{-0.033333in}{0.000000in}}%
\pgfpathcurveto{\pgfqpoint{-0.033333in}{-0.008840in}}{\pgfqpoint{-0.029821in}{-0.017319in}}{\pgfqpoint{-0.023570in}{-0.023570in}}%
\pgfpathcurveto{\pgfqpoint{-0.017319in}{-0.029821in}}{\pgfqpoint{-0.008840in}{-0.033333in}}{\pgfqpoint{0.000000in}{-0.033333in}}%
\pgfpathlineto{\pgfqpoint{0.000000in}{-0.033333in}}%
\pgfpathclose%
\pgfusepath{stroke,fill}%
}%
\begin{pgfscope}%
\pgfsys@transformshift{2.144482in}{2.826821in}%
\pgfsys@useobject{currentmarker}{}%
\end{pgfscope}%
\end{pgfscope}%
\begin{pgfscope}%
\pgfpathrectangle{\pgfqpoint{0.765000in}{0.660000in}}{\pgfqpoint{4.620000in}{4.620000in}}%
\pgfusepath{clip}%
\pgfsetrectcap%
\pgfsetroundjoin%
\pgfsetlinewidth{1.204500pt}%
\definecolor{currentstroke}{rgb}{1.000000,0.576471,0.309804}%
\pgfsetstrokecolor{currentstroke}%
\pgfsetdash{}{0pt}%
\pgfpathmoveto{\pgfqpoint{4.664293in}{3.048055in}}%
\pgfusepath{stroke}%
\end{pgfscope}%
\begin{pgfscope}%
\pgfpathrectangle{\pgfqpoint{0.765000in}{0.660000in}}{\pgfqpoint{4.620000in}{4.620000in}}%
\pgfusepath{clip}%
\pgfsetbuttcap%
\pgfsetroundjoin%
\definecolor{currentfill}{rgb}{1.000000,0.576471,0.309804}%
\pgfsetfillcolor{currentfill}%
\pgfsetlinewidth{1.003750pt}%
\definecolor{currentstroke}{rgb}{1.000000,0.576471,0.309804}%
\pgfsetstrokecolor{currentstroke}%
\pgfsetdash{}{0pt}%
\pgfsys@defobject{currentmarker}{\pgfqpoint{-0.033333in}{-0.033333in}}{\pgfqpoint{0.033333in}{0.033333in}}{%
\pgfpathmoveto{\pgfqpoint{0.000000in}{-0.033333in}}%
\pgfpathcurveto{\pgfqpoint{0.008840in}{-0.033333in}}{\pgfqpoint{0.017319in}{-0.029821in}}{\pgfqpoint{0.023570in}{-0.023570in}}%
\pgfpathcurveto{\pgfqpoint{0.029821in}{-0.017319in}}{\pgfqpoint{0.033333in}{-0.008840in}}{\pgfqpoint{0.033333in}{0.000000in}}%
\pgfpathcurveto{\pgfqpoint{0.033333in}{0.008840in}}{\pgfqpoint{0.029821in}{0.017319in}}{\pgfqpoint{0.023570in}{0.023570in}}%
\pgfpathcurveto{\pgfqpoint{0.017319in}{0.029821in}}{\pgfqpoint{0.008840in}{0.033333in}}{\pgfqpoint{0.000000in}{0.033333in}}%
\pgfpathcurveto{\pgfqpoint{-0.008840in}{0.033333in}}{\pgfqpoint{-0.017319in}{0.029821in}}{\pgfqpoint{-0.023570in}{0.023570in}}%
\pgfpathcurveto{\pgfqpoint{-0.029821in}{0.017319in}}{\pgfqpoint{-0.033333in}{0.008840in}}{\pgfqpoint{-0.033333in}{0.000000in}}%
\pgfpathcurveto{\pgfqpoint{-0.033333in}{-0.008840in}}{\pgfqpoint{-0.029821in}{-0.017319in}}{\pgfqpoint{-0.023570in}{-0.023570in}}%
\pgfpathcurveto{\pgfqpoint{-0.017319in}{-0.029821in}}{\pgfqpoint{-0.008840in}{-0.033333in}}{\pgfqpoint{0.000000in}{-0.033333in}}%
\pgfpathlineto{\pgfqpoint{0.000000in}{-0.033333in}}%
\pgfpathclose%
\pgfusepath{stroke,fill}%
}%
\begin{pgfscope}%
\pgfsys@transformshift{4.664293in}{3.048055in}%
\pgfsys@useobject{currentmarker}{}%
\end{pgfscope}%
\end{pgfscope}%
\begin{pgfscope}%
\pgfpathrectangle{\pgfqpoint{0.765000in}{0.660000in}}{\pgfqpoint{4.620000in}{4.620000in}}%
\pgfusepath{clip}%
\pgfsetrectcap%
\pgfsetroundjoin%
\pgfsetlinewidth{1.204500pt}%
\definecolor{currentstroke}{rgb}{1.000000,0.576471,0.309804}%
\pgfsetstrokecolor{currentstroke}%
\pgfsetdash{}{0pt}%
\pgfpathmoveto{\pgfqpoint{4.749270in}{3.004696in}}%
\pgfusepath{stroke}%
\end{pgfscope}%
\begin{pgfscope}%
\pgfpathrectangle{\pgfqpoint{0.765000in}{0.660000in}}{\pgfqpoint{4.620000in}{4.620000in}}%
\pgfusepath{clip}%
\pgfsetbuttcap%
\pgfsetroundjoin%
\definecolor{currentfill}{rgb}{1.000000,0.576471,0.309804}%
\pgfsetfillcolor{currentfill}%
\pgfsetlinewidth{1.003750pt}%
\definecolor{currentstroke}{rgb}{1.000000,0.576471,0.309804}%
\pgfsetstrokecolor{currentstroke}%
\pgfsetdash{}{0pt}%
\pgfsys@defobject{currentmarker}{\pgfqpoint{-0.033333in}{-0.033333in}}{\pgfqpoint{0.033333in}{0.033333in}}{%
\pgfpathmoveto{\pgfqpoint{0.000000in}{-0.033333in}}%
\pgfpathcurveto{\pgfqpoint{0.008840in}{-0.033333in}}{\pgfqpoint{0.017319in}{-0.029821in}}{\pgfqpoint{0.023570in}{-0.023570in}}%
\pgfpathcurveto{\pgfqpoint{0.029821in}{-0.017319in}}{\pgfqpoint{0.033333in}{-0.008840in}}{\pgfqpoint{0.033333in}{0.000000in}}%
\pgfpathcurveto{\pgfqpoint{0.033333in}{0.008840in}}{\pgfqpoint{0.029821in}{0.017319in}}{\pgfqpoint{0.023570in}{0.023570in}}%
\pgfpathcurveto{\pgfqpoint{0.017319in}{0.029821in}}{\pgfqpoint{0.008840in}{0.033333in}}{\pgfqpoint{0.000000in}{0.033333in}}%
\pgfpathcurveto{\pgfqpoint{-0.008840in}{0.033333in}}{\pgfqpoint{-0.017319in}{0.029821in}}{\pgfqpoint{-0.023570in}{0.023570in}}%
\pgfpathcurveto{\pgfqpoint{-0.029821in}{0.017319in}}{\pgfqpoint{-0.033333in}{0.008840in}}{\pgfqpoint{-0.033333in}{0.000000in}}%
\pgfpathcurveto{\pgfqpoint{-0.033333in}{-0.008840in}}{\pgfqpoint{-0.029821in}{-0.017319in}}{\pgfqpoint{-0.023570in}{-0.023570in}}%
\pgfpathcurveto{\pgfqpoint{-0.017319in}{-0.029821in}}{\pgfqpoint{-0.008840in}{-0.033333in}}{\pgfqpoint{0.000000in}{-0.033333in}}%
\pgfpathlineto{\pgfqpoint{0.000000in}{-0.033333in}}%
\pgfpathclose%
\pgfusepath{stroke,fill}%
}%
\begin{pgfscope}%
\pgfsys@transformshift{4.749270in}{3.004696in}%
\pgfsys@useobject{currentmarker}{}%
\end{pgfscope}%
\end{pgfscope}%
\begin{pgfscope}%
\pgfpathrectangle{\pgfqpoint{0.765000in}{0.660000in}}{\pgfqpoint{4.620000in}{4.620000in}}%
\pgfusepath{clip}%
\pgfsetrectcap%
\pgfsetroundjoin%
\pgfsetlinewidth{1.204500pt}%
\definecolor{currentstroke}{rgb}{1.000000,0.576471,0.309804}%
\pgfsetstrokecolor{currentstroke}%
\pgfsetdash{}{0pt}%
\pgfpathmoveto{\pgfqpoint{2.328750in}{2.746484in}}%
\pgfusepath{stroke}%
\end{pgfscope}%
\begin{pgfscope}%
\pgfpathrectangle{\pgfqpoint{0.765000in}{0.660000in}}{\pgfqpoint{4.620000in}{4.620000in}}%
\pgfusepath{clip}%
\pgfsetbuttcap%
\pgfsetroundjoin%
\definecolor{currentfill}{rgb}{1.000000,0.576471,0.309804}%
\pgfsetfillcolor{currentfill}%
\pgfsetlinewidth{1.003750pt}%
\definecolor{currentstroke}{rgb}{1.000000,0.576471,0.309804}%
\pgfsetstrokecolor{currentstroke}%
\pgfsetdash{}{0pt}%
\pgfsys@defobject{currentmarker}{\pgfqpoint{-0.033333in}{-0.033333in}}{\pgfqpoint{0.033333in}{0.033333in}}{%
\pgfpathmoveto{\pgfqpoint{0.000000in}{-0.033333in}}%
\pgfpathcurveto{\pgfqpoint{0.008840in}{-0.033333in}}{\pgfqpoint{0.017319in}{-0.029821in}}{\pgfqpoint{0.023570in}{-0.023570in}}%
\pgfpathcurveto{\pgfqpoint{0.029821in}{-0.017319in}}{\pgfqpoint{0.033333in}{-0.008840in}}{\pgfqpoint{0.033333in}{0.000000in}}%
\pgfpathcurveto{\pgfqpoint{0.033333in}{0.008840in}}{\pgfqpoint{0.029821in}{0.017319in}}{\pgfqpoint{0.023570in}{0.023570in}}%
\pgfpathcurveto{\pgfqpoint{0.017319in}{0.029821in}}{\pgfqpoint{0.008840in}{0.033333in}}{\pgfqpoint{0.000000in}{0.033333in}}%
\pgfpathcurveto{\pgfqpoint{-0.008840in}{0.033333in}}{\pgfqpoint{-0.017319in}{0.029821in}}{\pgfqpoint{-0.023570in}{0.023570in}}%
\pgfpathcurveto{\pgfqpoint{-0.029821in}{0.017319in}}{\pgfqpoint{-0.033333in}{0.008840in}}{\pgfqpoint{-0.033333in}{0.000000in}}%
\pgfpathcurveto{\pgfqpoint{-0.033333in}{-0.008840in}}{\pgfqpoint{-0.029821in}{-0.017319in}}{\pgfqpoint{-0.023570in}{-0.023570in}}%
\pgfpathcurveto{\pgfqpoint{-0.017319in}{-0.029821in}}{\pgfqpoint{-0.008840in}{-0.033333in}}{\pgfqpoint{0.000000in}{-0.033333in}}%
\pgfpathlineto{\pgfqpoint{0.000000in}{-0.033333in}}%
\pgfpathclose%
\pgfusepath{stroke,fill}%
}%
\begin{pgfscope}%
\pgfsys@transformshift{2.328750in}{2.746484in}%
\pgfsys@useobject{currentmarker}{}%
\end{pgfscope}%
\end{pgfscope}%
\begin{pgfscope}%
\pgfpathrectangle{\pgfqpoint{0.765000in}{0.660000in}}{\pgfqpoint{4.620000in}{4.620000in}}%
\pgfusepath{clip}%
\pgfsetrectcap%
\pgfsetroundjoin%
\pgfsetlinewidth{1.204500pt}%
\definecolor{currentstroke}{rgb}{1.000000,0.576471,0.309804}%
\pgfsetstrokecolor{currentstroke}%
\pgfsetdash{}{0pt}%
\pgfpathmoveto{\pgfqpoint{4.435415in}{2.789876in}}%
\pgfusepath{stroke}%
\end{pgfscope}%
\begin{pgfscope}%
\pgfpathrectangle{\pgfqpoint{0.765000in}{0.660000in}}{\pgfqpoint{4.620000in}{4.620000in}}%
\pgfusepath{clip}%
\pgfsetbuttcap%
\pgfsetroundjoin%
\definecolor{currentfill}{rgb}{1.000000,0.576471,0.309804}%
\pgfsetfillcolor{currentfill}%
\pgfsetlinewidth{1.003750pt}%
\definecolor{currentstroke}{rgb}{1.000000,0.576471,0.309804}%
\pgfsetstrokecolor{currentstroke}%
\pgfsetdash{}{0pt}%
\pgfsys@defobject{currentmarker}{\pgfqpoint{-0.033333in}{-0.033333in}}{\pgfqpoint{0.033333in}{0.033333in}}{%
\pgfpathmoveto{\pgfqpoint{0.000000in}{-0.033333in}}%
\pgfpathcurveto{\pgfqpoint{0.008840in}{-0.033333in}}{\pgfqpoint{0.017319in}{-0.029821in}}{\pgfqpoint{0.023570in}{-0.023570in}}%
\pgfpathcurveto{\pgfqpoint{0.029821in}{-0.017319in}}{\pgfqpoint{0.033333in}{-0.008840in}}{\pgfqpoint{0.033333in}{0.000000in}}%
\pgfpathcurveto{\pgfqpoint{0.033333in}{0.008840in}}{\pgfqpoint{0.029821in}{0.017319in}}{\pgfqpoint{0.023570in}{0.023570in}}%
\pgfpathcurveto{\pgfqpoint{0.017319in}{0.029821in}}{\pgfqpoint{0.008840in}{0.033333in}}{\pgfqpoint{0.000000in}{0.033333in}}%
\pgfpathcurveto{\pgfqpoint{-0.008840in}{0.033333in}}{\pgfqpoint{-0.017319in}{0.029821in}}{\pgfqpoint{-0.023570in}{0.023570in}}%
\pgfpathcurveto{\pgfqpoint{-0.029821in}{0.017319in}}{\pgfqpoint{-0.033333in}{0.008840in}}{\pgfqpoint{-0.033333in}{0.000000in}}%
\pgfpathcurveto{\pgfqpoint{-0.033333in}{-0.008840in}}{\pgfqpoint{-0.029821in}{-0.017319in}}{\pgfqpoint{-0.023570in}{-0.023570in}}%
\pgfpathcurveto{\pgfqpoint{-0.017319in}{-0.029821in}}{\pgfqpoint{-0.008840in}{-0.033333in}}{\pgfqpoint{0.000000in}{-0.033333in}}%
\pgfpathlineto{\pgfqpoint{0.000000in}{-0.033333in}}%
\pgfpathclose%
\pgfusepath{stroke,fill}%
}%
\begin{pgfscope}%
\pgfsys@transformshift{4.435415in}{2.789876in}%
\pgfsys@useobject{currentmarker}{}%
\end{pgfscope}%
\end{pgfscope}%
\begin{pgfscope}%
\pgfpathrectangle{\pgfqpoint{0.765000in}{0.660000in}}{\pgfqpoint{4.620000in}{4.620000in}}%
\pgfusepath{clip}%
\pgfsetrectcap%
\pgfsetroundjoin%
\pgfsetlinewidth{1.204500pt}%
\definecolor{currentstroke}{rgb}{1.000000,0.576471,0.309804}%
\pgfsetstrokecolor{currentstroke}%
\pgfsetdash{}{0pt}%
\pgfpathmoveto{\pgfqpoint{2.704602in}{3.179003in}}%
\pgfusepath{stroke}%
\end{pgfscope}%
\begin{pgfscope}%
\pgfpathrectangle{\pgfqpoint{0.765000in}{0.660000in}}{\pgfqpoint{4.620000in}{4.620000in}}%
\pgfusepath{clip}%
\pgfsetbuttcap%
\pgfsetroundjoin%
\definecolor{currentfill}{rgb}{1.000000,0.576471,0.309804}%
\pgfsetfillcolor{currentfill}%
\pgfsetlinewidth{1.003750pt}%
\definecolor{currentstroke}{rgb}{1.000000,0.576471,0.309804}%
\pgfsetstrokecolor{currentstroke}%
\pgfsetdash{}{0pt}%
\pgfsys@defobject{currentmarker}{\pgfqpoint{-0.033333in}{-0.033333in}}{\pgfqpoint{0.033333in}{0.033333in}}{%
\pgfpathmoveto{\pgfqpoint{0.000000in}{-0.033333in}}%
\pgfpathcurveto{\pgfqpoint{0.008840in}{-0.033333in}}{\pgfqpoint{0.017319in}{-0.029821in}}{\pgfqpoint{0.023570in}{-0.023570in}}%
\pgfpathcurveto{\pgfqpoint{0.029821in}{-0.017319in}}{\pgfqpoint{0.033333in}{-0.008840in}}{\pgfqpoint{0.033333in}{0.000000in}}%
\pgfpathcurveto{\pgfqpoint{0.033333in}{0.008840in}}{\pgfqpoint{0.029821in}{0.017319in}}{\pgfqpoint{0.023570in}{0.023570in}}%
\pgfpathcurveto{\pgfqpoint{0.017319in}{0.029821in}}{\pgfqpoint{0.008840in}{0.033333in}}{\pgfqpoint{0.000000in}{0.033333in}}%
\pgfpathcurveto{\pgfqpoint{-0.008840in}{0.033333in}}{\pgfqpoint{-0.017319in}{0.029821in}}{\pgfqpoint{-0.023570in}{0.023570in}}%
\pgfpathcurveto{\pgfqpoint{-0.029821in}{0.017319in}}{\pgfqpoint{-0.033333in}{0.008840in}}{\pgfqpoint{-0.033333in}{0.000000in}}%
\pgfpathcurveto{\pgfqpoint{-0.033333in}{-0.008840in}}{\pgfqpoint{-0.029821in}{-0.017319in}}{\pgfqpoint{-0.023570in}{-0.023570in}}%
\pgfpathcurveto{\pgfqpoint{-0.017319in}{-0.029821in}}{\pgfqpoint{-0.008840in}{-0.033333in}}{\pgfqpoint{0.000000in}{-0.033333in}}%
\pgfpathlineto{\pgfqpoint{0.000000in}{-0.033333in}}%
\pgfpathclose%
\pgfusepath{stroke,fill}%
}%
\begin{pgfscope}%
\pgfsys@transformshift{2.704602in}{3.179003in}%
\pgfsys@useobject{currentmarker}{}%
\end{pgfscope}%
\end{pgfscope}%
\begin{pgfscope}%
\pgfpathrectangle{\pgfqpoint{0.765000in}{0.660000in}}{\pgfqpoint{4.620000in}{4.620000in}}%
\pgfusepath{clip}%
\pgfsetrectcap%
\pgfsetroundjoin%
\pgfsetlinewidth{1.204500pt}%
\definecolor{currentstroke}{rgb}{1.000000,0.576471,0.309804}%
\pgfsetstrokecolor{currentstroke}%
\pgfsetdash{}{0pt}%
\pgfpathmoveto{\pgfqpoint{3.816314in}{2.637750in}}%
\pgfusepath{stroke}%
\end{pgfscope}%
\begin{pgfscope}%
\pgfpathrectangle{\pgfqpoint{0.765000in}{0.660000in}}{\pgfqpoint{4.620000in}{4.620000in}}%
\pgfusepath{clip}%
\pgfsetbuttcap%
\pgfsetroundjoin%
\definecolor{currentfill}{rgb}{1.000000,0.576471,0.309804}%
\pgfsetfillcolor{currentfill}%
\pgfsetlinewidth{1.003750pt}%
\definecolor{currentstroke}{rgb}{1.000000,0.576471,0.309804}%
\pgfsetstrokecolor{currentstroke}%
\pgfsetdash{}{0pt}%
\pgfsys@defobject{currentmarker}{\pgfqpoint{-0.033333in}{-0.033333in}}{\pgfqpoint{0.033333in}{0.033333in}}{%
\pgfpathmoveto{\pgfqpoint{0.000000in}{-0.033333in}}%
\pgfpathcurveto{\pgfqpoint{0.008840in}{-0.033333in}}{\pgfqpoint{0.017319in}{-0.029821in}}{\pgfqpoint{0.023570in}{-0.023570in}}%
\pgfpathcurveto{\pgfqpoint{0.029821in}{-0.017319in}}{\pgfqpoint{0.033333in}{-0.008840in}}{\pgfqpoint{0.033333in}{0.000000in}}%
\pgfpathcurveto{\pgfqpoint{0.033333in}{0.008840in}}{\pgfqpoint{0.029821in}{0.017319in}}{\pgfqpoint{0.023570in}{0.023570in}}%
\pgfpathcurveto{\pgfqpoint{0.017319in}{0.029821in}}{\pgfqpoint{0.008840in}{0.033333in}}{\pgfqpoint{0.000000in}{0.033333in}}%
\pgfpathcurveto{\pgfqpoint{-0.008840in}{0.033333in}}{\pgfqpoint{-0.017319in}{0.029821in}}{\pgfqpoint{-0.023570in}{0.023570in}}%
\pgfpathcurveto{\pgfqpoint{-0.029821in}{0.017319in}}{\pgfqpoint{-0.033333in}{0.008840in}}{\pgfqpoint{-0.033333in}{0.000000in}}%
\pgfpathcurveto{\pgfqpoint{-0.033333in}{-0.008840in}}{\pgfqpoint{-0.029821in}{-0.017319in}}{\pgfqpoint{-0.023570in}{-0.023570in}}%
\pgfpathcurveto{\pgfqpoint{-0.017319in}{-0.029821in}}{\pgfqpoint{-0.008840in}{-0.033333in}}{\pgfqpoint{0.000000in}{-0.033333in}}%
\pgfpathlineto{\pgfqpoint{0.000000in}{-0.033333in}}%
\pgfpathclose%
\pgfusepath{stroke,fill}%
}%
\begin{pgfscope}%
\pgfsys@transformshift{3.816314in}{2.637750in}%
\pgfsys@useobject{currentmarker}{}%
\end{pgfscope}%
\end{pgfscope}%
\begin{pgfscope}%
\pgfpathrectangle{\pgfqpoint{0.765000in}{0.660000in}}{\pgfqpoint{4.620000in}{4.620000in}}%
\pgfusepath{clip}%
\pgfsetrectcap%
\pgfsetroundjoin%
\pgfsetlinewidth{1.204500pt}%
\definecolor{currentstroke}{rgb}{1.000000,0.576471,0.309804}%
\pgfsetstrokecolor{currentstroke}%
\pgfsetdash{}{0pt}%
\pgfpathmoveto{\pgfqpoint{3.754878in}{2.640924in}}%
\pgfusepath{stroke}%
\end{pgfscope}%
\begin{pgfscope}%
\pgfpathrectangle{\pgfqpoint{0.765000in}{0.660000in}}{\pgfqpoint{4.620000in}{4.620000in}}%
\pgfusepath{clip}%
\pgfsetbuttcap%
\pgfsetroundjoin%
\definecolor{currentfill}{rgb}{1.000000,0.576471,0.309804}%
\pgfsetfillcolor{currentfill}%
\pgfsetlinewidth{1.003750pt}%
\definecolor{currentstroke}{rgb}{1.000000,0.576471,0.309804}%
\pgfsetstrokecolor{currentstroke}%
\pgfsetdash{}{0pt}%
\pgfsys@defobject{currentmarker}{\pgfqpoint{-0.033333in}{-0.033333in}}{\pgfqpoint{0.033333in}{0.033333in}}{%
\pgfpathmoveto{\pgfqpoint{0.000000in}{-0.033333in}}%
\pgfpathcurveto{\pgfqpoint{0.008840in}{-0.033333in}}{\pgfqpoint{0.017319in}{-0.029821in}}{\pgfqpoint{0.023570in}{-0.023570in}}%
\pgfpathcurveto{\pgfqpoint{0.029821in}{-0.017319in}}{\pgfqpoint{0.033333in}{-0.008840in}}{\pgfqpoint{0.033333in}{0.000000in}}%
\pgfpathcurveto{\pgfqpoint{0.033333in}{0.008840in}}{\pgfqpoint{0.029821in}{0.017319in}}{\pgfqpoint{0.023570in}{0.023570in}}%
\pgfpathcurveto{\pgfqpoint{0.017319in}{0.029821in}}{\pgfqpoint{0.008840in}{0.033333in}}{\pgfqpoint{0.000000in}{0.033333in}}%
\pgfpathcurveto{\pgfqpoint{-0.008840in}{0.033333in}}{\pgfqpoint{-0.017319in}{0.029821in}}{\pgfqpoint{-0.023570in}{0.023570in}}%
\pgfpathcurveto{\pgfqpoint{-0.029821in}{0.017319in}}{\pgfqpoint{-0.033333in}{0.008840in}}{\pgfqpoint{-0.033333in}{0.000000in}}%
\pgfpathcurveto{\pgfqpoint{-0.033333in}{-0.008840in}}{\pgfqpoint{-0.029821in}{-0.017319in}}{\pgfqpoint{-0.023570in}{-0.023570in}}%
\pgfpathcurveto{\pgfqpoint{-0.017319in}{-0.029821in}}{\pgfqpoint{-0.008840in}{-0.033333in}}{\pgfqpoint{0.000000in}{-0.033333in}}%
\pgfpathlineto{\pgfqpoint{0.000000in}{-0.033333in}}%
\pgfpathclose%
\pgfusepath{stroke,fill}%
}%
\begin{pgfscope}%
\pgfsys@transformshift{3.754878in}{2.640924in}%
\pgfsys@useobject{currentmarker}{}%
\end{pgfscope}%
\end{pgfscope}%
\begin{pgfscope}%
\pgfpathrectangle{\pgfqpoint{0.765000in}{0.660000in}}{\pgfqpoint{4.620000in}{4.620000in}}%
\pgfusepath{clip}%
\pgfsetrectcap%
\pgfsetroundjoin%
\pgfsetlinewidth{1.204500pt}%
\definecolor{currentstroke}{rgb}{1.000000,0.576471,0.309804}%
\pgfsetstrokecolor{currentstroke}%
\pgfsetdash{}{0pt}%
\pgfpathmoveto{\pgfqpoint{4.498473in}{2.881780in}}%
\pgfusepath{stroke}%
\end{pgfscope}%
\begin{pgfscope}%
\pgfpathrectangle{\pgfqpoint{0.765000in}{0.660000in}}{\pgfqpoint{4.620000in}{4.620000in}}%
\pgfusepath{clip}%
\pgfsetbuttcap%
\pgfsetroundjoin%
\definecolor{currentfill}{rgb}{1.000000,0.576471,0.309804}%
\pgfsetfillcolor{currentfill}%
\pgfsetlinewidth{1.003750pt}%
\definecolor{currentstroke}{rgb}{1.000000,0.576471,0.309804}%
\pgfsetstrokecolor{currentstroke}%
\pgfsetdash{}{0pt}%
\pgfsys@defobject{currentmarker}{\pgfqpoint{-0.033333in}{-0.033333in}}{\pgfqpoint{0.033333in}{0.033333in}}{%
\pgfpathmoveto{\pgfqpoint{0.000000in}{-0.033333in}}%
\pgfpathcurveto{\pgfqpoint{0.008840in}{-0.033333in}}{\pgfqpoint{0.017319in}{-0.029821in}}{\pgfqpoint{0.023570in}{-0.023570in}}%
\pgfpathcurveto{\pgfqpoint{0.029821in}{-0.017319in}}{\pgfqpoint{0.033333in}{-0.008840in}}{\pgfqpoint{0.033333in}{0.000000in}}%
\pgfpathcurveto{\pgfqpoint{0.033333in}{0.008840in}}{\pgfqpoint{0.029821in}{0.017319in}}{\pgfqpoint{0.023570in}{0.023570in}}%
\pgfpathcurveto{\pgfqpoint{0.017319in}{0.029821in}}{\pgfqpoint{0.008840in}{0.033333in}}{\pgfqpoint{0.000000in}{0.033333in}}%
\pgfpathcurveto{\pgfqpoint{-0.008840in}{0.033333in}}{\pgfqpoint{-0.017319in}{0.029821in}}{\pgfqpoint{-0.023570in}{0.023570in}}%
\pgfpathcurveto{\pgfqpoint{-0.029821in}{0.017319in}}{\pgfqpoint{-0.033333in}{0.008840in}}{\pgfqpoint{-0.033333in}{0.000000in}}%
\pgfpathcurveto{\pgfqpoint{-0.033333in}{-0.008840in}}{\pgfqpoint{-0.029821in}{-0.017319in}}{\pgfqpoint{-0.023570in}{-0.023570in}}%
\pgfpathcurveto{\pgfqpoint{-0.017319in}{-0.029821in}}{\pgfqpoint{-0.008840in}{-0.033333in}}{\pgfqpoint{0.000000in}{-0.033333in}}%
\pgfpathlineto{\pgfqpoint{0.000000in}{-0.033333in}}%
\pgfpathclose%
\pgfusepath{stroke,fill}%
}%
\begin{pgfscope}%
\pgfsys@transformshift{4.498473in}{2.881780in}%
\pgfsys@useobject{currentmarker}{}%
\end{pgfscope}%
\end{pgfscope}%
\begin{pgfscope}%
\pgfpathrectangle{\pgfqpoint{0.765000in}{0.660000in}}{\pgfqpoint{4.620000in}{4.620000in}}%
\pgfusepath{clip}%
\pgfsetrectcap%
\pgfsetroundjoin%
\pgfsetlinewidth{1.204500pt}%
\definecolor{currentstroke}{rgb}{1.000000,0.576471,0.309804}%
\pgfsetstrokecolor{currentstroke}%
\pgfsetdash{}{0pt}%
\pgfpathmoveto{\pgfqpoint{4.857841in}{3.430614in}}%
\pgfusepath{stroke}%
\end{pgfscope}%
\begin{pgfscope}%
\pgfpathrectangle{\pgfqpoint{0.765000in}{0.660000in}}{\pgfqpoint{4.620000in}{4.620000in}}%
\pgfusepath{clip}%
\pgfsetbuttcap%
\pgfsetroundjoin%
\definecolor{currentfill}{rgb}{1.000000,0.576471,0.309804}%
\pgfsetfillcolor{currentfill}%
\pgfsetlinewidth{1.003750pt}%
\definecolor{currentstroke}{rgb}{1.000000,0.576471,0.309804}%
\pgfsetstrokecolor{currentstroke}%
\pgfsetdash{}{0pt}%
\pgfsys@defobject{currentmarker}{\pgfqpoint{-0.033333in}{-0.033333in}}{\pgfqpoint{0.033333in}{0.033333in}}{%
\pgfpathmoveto{\pgfqpoint{0.000000in}{-0.033333in}}%
\pgfpathcurveto{\pgfqpoint{0.008840in}{-0.033333in}}{\pgfqpoint{0.017319in}{-0.029821in}}{\pgfqpoint{0.023570in}{-0.023570in}}%
\pgfpathcurveto{\pgfqpoint{0.029821in}{-0.017319in}}{\pgfqpoint{0.033333in}{-0.008840in}}{\pgfqpoint{0.033333in}{0.000000in}}%
\pgfpathcurveto{\pgfqpoint{0.033333in}{0.008840in}}{\pgfqpoint{0.029821in}{0.017319in}}{\pgfqpoint{0.023570in}{0.023570in}}%
\pgfpathcurveto{\pgfqpoint{0.017319in}{0.029821in}}{\pgfqpoint{0.008840in}{0.033333in}}{\pgfqpoint{0.000000in}{0.033333in}}%
\pgfpathcurveto{\pgfqpoint{-0.008840in}{0.033333in}}{\pgfqpoint{-0.017319in}{0.029821in}}{\pgfqpoint{-0.023570in}{0.023570in}}%
\pgfpathcurveto{\pgfqpoint{-0.029821in}{0.017319in}}{\pgfqpoint{-0.033333in}{0.008840in}}{\pgfqpoint{-0.033333in}{0.000000in}}%
\pgfpathcurveto{\pgfqpoint{-0.033333in}{-0.008840in}}{\pgfqpoint{-0.029821in}{-0.017319in}}{\pgfqpoint{-0.023570in}{-0.023570in}}%
\pgfpathcurveto{\pgfqpoint{-0.017319in}{-0.029821in}}{\pgfqpoint{-0.008840in}{-0.033333in}}{\pgfqpoint{0.000000in}{-0.033333in}}%
\pgfpathlineto{\pgfqpoint{0.000000in}{-0.033333in}}%
\pgfpathclose%
\pgfusepath{stroke,fill}%
}%
\begin{pgfscope}%
\pgfsys@transformshift{4.857841in}{3.430614in}%
\pgfsys@useobject{currentmarker}{}%
\end{pgfscope}%
\end{pgfscope}%
\begin{pgfscope}%
\pgfpathrectangle{\pgfqpoint{0.765000in}{0.660000in}}{\pgfqpoint{4.620000in}{4.620000in}}%
\pgfusepath{clip}%
\pgfsetrectcap%
\pgfsetroundjoin%
\pgfsetlinewidth{1.204500pt}%
\definecolor{currentstroke}{rgb}{1.000000,0.576471,0.309804}%
\pgfsetstrokecolor{currentstroke}%
\pgfsetdash{}{0pt}%
\pgfpathmoveto{\pgfqpoint{2.628720in}{3.144735in}}%
\pgfusepath{stroke}%
\end{pgfscope}%
\begin{pgfscope}%
\pgfpathrectangle{\pgfqpoint{0.765000in}{0.660000in}}{\pgfqpoint{4.620000in}{4.620000in}}%
\pgfusepath{clip}%
\pgfsetbuttcap%
\pgfsetroundjoin%
\definecolor{currentfill}{rgb}{1.000000,0.576471,0.309804}%
\pgfsetfillcolor{currentfill}%
\pgfsetlinewidth{1.003750pt}%
\definecolor{currentstroke}{rgb}{1.000000,0.576471,0.309804}%
\pgfsetstrokecolor{currentstroke}%
\pgfsetdash{}{0pt}%
\pgfsys@defobject{currentmarker}{\pgfqpoint{-0.033333in}{-0.033333in}}{\pgfqpoint{0.033333in}{0.033333in}}{%
\pgfpathmoveto{\pgfqpoint{0.000000in}{-0.033333in}}%
\pgfpathcurveto{\pgfqpoint{0.008840in}{-0.033333in}}{\pgfqpoint{0.017319in}{-0.029821in}}{\pgfqpoint{0.023570in}{-0.023570in}}%
\pgfpathcurveto{\pgfqpoint{0.029821in}{-0.017319in}}{\pgfqpoint{0.033333in}{-0.008840in}}{\pgfqpoint{0.033333in}{0.000000in}}%
\pgfpathcurveto{\pgfqpoint{0.033333in}{0.008840in}}{\pgfqpoint{0.029821in}{0.017319in}}{\pgfqpoint{0.023570in}{0.023570in}}%
\pgfpathcurveto{\pgfqpoint{0.017319in}{0.029821in}}{\pgfqpoint{0.008840in}{0.033333in}}{\pgfqpoint{0.000000in}{0.033333in}}%
\pgfpathcurveto{\pgfqpoint{-0.008840in}{0.033333in}}{\pgfqpoint{-0.017319in}{0.029821in}}{\pgfqpoint{-0.023570in}{0.023570in}}%
\pgfpathcurveto{\pgfqpoint{-0.029821in}{0.017319in}}{\pgfqpoint{-0.033333in}{0.008840in}}{\pgfqpoint{-0.033333in}{0.000000in}}%
\pgfpathcurveto{\pgfqpoint{-0.033333in}{-0.008840in}}{\pgfqpoint{-0.029821in}{-0.017319in}}{\pgfqpoint{-0.023570in}{-0.023570in}}%
\pgfpathcurveto{\pgfqpoint{-0.017319in}{-0.029821in}}{\pgfqpoint{-0.008840in}{-0.033333in}}{\pgfqpoint{0.000000in}{-0.033333in}}%
\pgfpathlineto{\pgfqpoint{0.000000in}{-0.033333in}}%
\pgfpathclose%
\pgfusepath{stroke,fill}%
}%
\begin{pgfscope}%
\pgfsys@transformshift{2.628720in}{3.144735in}%
\pgfsys@useobject{currentmarker}{}%
\end{pgfscope}%
\end{pgfscope}%
\begin{pgfscope}%
\pgfpathrectangle{\pgfqpoint{0.765000in}{0.660000in}}{\pgfqpoint{4.620000in}{4.620000in}}%
\pgfusepath{clip}%
\pgfsetrectcap%
\pgfsetroundjoin%
\pgfsetlinewidth{1.204500pt}%
\definecolor{currentstroke}{rgb}{1.000000,0.576471,0.309804}%
\pgfsetstrokecolor{currentstroke}%
\pgfsetdash{}{0pt}%
\pgfpathmoveto{\pgfqpoint{2.865189in}{2.896017in}}%
\pgfusepath{stroke}%
\end{pgfscope}%
\begin{pgfscope}%
\pgfpathrectangle{\pgfqpoint{0.765000in}{0.660000in}}{\pgfqpoint{4.620000in}{4.620000in}}%
\pgfusepath{clip}%
\pgfsetbuttcap%
\pgfsetroundjoin%
\definecolor{currentfill}{rgb}{1.000000,0.576471,0.309804}%
\pgfsetfillcolor{currentfill}%
\pgfsetlinewidth{1.003750pt}%
\definecolor{currentstroke}{rgb}{1.000000,0.576471,0.309804}%
\pgfsetstrokecolor{currentstroke}%
\pgfsetdash{}{0pt}%
\pgfsys@defobject{currentmarker}{\pgfqpoint{-0.033333in}{-0.033333in}}{\pgfqpoint{0.033333in}{0.033333in}}{%
\pgfpathmoveto{\pgfqpoint{0.000000in}{-0.033333in}}%
\pgfpathcurveto{\pgfqpoint{0.008840in}{-0.033333in}}{\pgfqpoint{0.017319in}{-0.029821in}}{\pgfqpoint{0.023570in}{-0.023570in}}%
\pgfpathcurveto{\pgfqpoint{0.029821in}{-0.017319in}}{\pgfqpoint{0.033333in}{-0.008840in}}{\pgfqpoint{0.033333in}{0.000000in}}%
\pgfpathcurveto{\pgfqpoint{0.033333in}{0.008840in}}{\pgfqpoint{0.029821in}{0.017319in}}{\pgfqpoint{0.023570in}{0.023570in}}%
\pgfpathcurveto{\pgfqpoint{0.017319in}{0.029821in}}{\pgfqpoint{0.008840in}{0.033333in}}{\pgfqpoint{0.000000in}{0.033333in}}%
\pgfpathcurveto{\pgfqpoint{-0.008840in}{0.033333in}}{\pgfqpoint{-0.017319in}{0.029821in}}{\pgfqpoint{-0.023570in}{0.023570in}}%
\pgfpathcurveto{\pgfqpoint{-0.029821in}{0.017319in}}{\pgfqpoint{-0.033333in}{0.008840in}}{\pgfqpoint{-0.033333in}{0.000000in}}%
\pgfpathcurveto{\pgfqpoint{-0.033333in}{-0.008840in}}{\pgfqpoint{-0.029821in}{-0.017319in}}{\pgfqpoint{-0.023570in}{-0.023570in}}%
\pgfpathcurveto{\pgfqpoint{-0.017319in}{-0.029821in}}{\pgfqpoint{-0.008840in}{-0.033333in}}{\pgfqpoint{0.000000in}{-0.033333in}}%
\pgfpathlineto{\pgfqpoint{0.000000in}{-0.033333in}}%
\pgfpathclose%
\pgfusepath{stroke,fill}%
}%
\begin{pgfscope}%
\pgfsys@transformshift{2.865189in}{2.896017in}%
\pgfsys@useobject{currentmarker}{}%
\end{pgfscope}%
\end{pgfscope}%
\begin{pgfscope}%
\pgfpathrectangle{\pgfqpoint{0.765000in}{0.660000in}}{\pgfqpoint{4.620000in}{4.620000in}}%
\pgfusepath{clip}%
\pgfsetrectcap%
\pgfsetroundjoin%
\pgfsetlinewidth{1.204500pt}%
\definecolor{currentstroke}{rgb}{1.000000,0.576471,0.309804}%
\pgfsetstrokecolor{currentstroke}%
\pgfsetdash{}{0pt}%
\pgfpathmoveto{\pgfqpoint{4.524308in}{2.506842in}}%
\pgfusepath{stroke}%
\end{pgfscope}%
\begin{pgfscope}%
\pgfpathrectangle{\pgfqpoint{0.765000in}{0.660000in}}{\pgfqpoint{4.620000in}{4.620000in}}%
\pgfusepath{clip}%
\pgfsetbuttcap%
\pgfsetroundjoin%
\definecolor{currentfill}{rgb}{1.000000,0.576471,0.309804}%
\pgfsetfillcolor{currentfill}%
\pgfsetlinewidth{1.003750pt}%
\definecolor{currentstroke}{rgb}{1.000000,0.576471,0.309804}%
\pgfsetstrokecolor{currentstroke}%
\pgfsetdash{}{0pt}%
\pgfsys@defobject{currentmarker}{\pgfqpoint{-0.033333in}{-0.033333in}}{\pgfqpoint{0.033333in}{0.033333in}}{%
\pgfpathmoveto{\pgfqpoint{0.000000in}{-0.033333in}}%
\pgfpathcurveto{\pgfqpoint{0.008840in}{-0.033333in}}{\pgfqpoint{0.017319in}{-0.029821in}}{\pgfqpoint{0.023570in}{-0.023570in}}%
\pgfpathcurveto{\pgfqpoint{0.029821in}{-0.017319in}}{\pgfqpoint{0.033333in}{-0.008840in}}{\pgfqpoint{0.033333in}{0.000000in}}%
\pgfpathcurveto{\pgfqpoint{0.033333in}{0.008840in}}{\pgfqpoint{0.029821in}{0.017319in}}{\pgfqpoint{0.023570in}{0.023570in}}%
\pgfpathcurveto{\pgfqpoint{0.017319in}{0.029821in}}{\pgfqpoint{0.008840in}{0.033333in}}{\pgfqpoint{0.000000in}{0.033333in}}%
\pgfpathcurveto{\pgfqpoint{-0.008840in}{0.033333in}}{\pgfqpoint{-0.017319in}{0.029821in}}{\pgfqpoint{-0.023570in}{0.023570in}}%
\pgfpathcurveto{\pgfqpoint{-0.029821in}{0.017319in}}{\pgfqpoint{-0.033333in}{0.008840in}}{\pgfqpoint{-0.033333in}{0.000000in}}%
\pgfpathcurveto{\pgfqpoint{-0.033333in}{-0.008840in}}{\pgfqpoint{-0.029821in}{-0.017319in}}{\pgfqpoint{-0.023570in}{-0.023570in}}%
\pgfpathcurveto{\pgfqpoint{-0.017319in}{-0.029821in}}{\pgfqpoint{-0.008840in}{-0.033333in}}{\pgfqpoint{0.000000in}{-0.033333in}}%
\pgfpathlineto{\pgfqpoint{0.000000in}{-0.033333in}}%
\pgfpathclose%
\pgfusepath{stroke,fill}%
}%
\begin{pgfscope}%
\pgfsys@transformshift{4.524308in}{2.506842in}%
\pgfsys@useobject{currentmarker}{}%
\end{pgfscope}%
\end{pgfscope}%
\begin{pgfscope}%
\pgfpathrectangle{\pgfqpoint{0.765000in}{0.660000in}}{\pgfqpoint{4.620000in}{4.620000in}}%
\pgfusepath{clip}%
\pgfsetrectcap%
\pgfsetroundjoin%
\pgfsetlinewidth{1.204500pt}%
\definecolor{currentstroke}{rgb}{1.000000,0.576471,0.309804}%
\pgfsetstrokecolor{currentstroke}%
\pgfsetdash{}{0pt}%
\pgfpathmoveto{\pgfqpoint{3.380366in}{2.439375in}}%
\pgfusepath{stroke}%
\end{pgfscope}%
\begin{pgfscope}%
\pgfpathrectangle{\pgfqpoint{0.765000in}{0.660000in}}{\pgfqpoint{4.620000in}{4.620000in}}%
\pgfusepath{clip}%
\pgfsetbuttcap%
\pgfsetroundjoin%
\definecolor{currentfill}{rgb}{1.000000,0.576471,0.309804}%
\pgfsetfillcolor{currentfill}%
\pgfsetlinewidth{1.003750pt}%
\definecolor{currentstroke}{rgb}{1.000000,0.576471,0.309804}%
\pgfsetstrokecolor{currentstroke}%
\pgfsetdash{}{0pt}%
\pgfsys@defobject{currentmarker}{\pgfqpoint{-0.033333in}{-0.033333in}}{\pgfqpoint{0.033333in}{0.033333in}}{%
\pgfpathmoveto{\pgfqpoint{0.000000in}{-0.033333in}}%
\pgfpathcurveto{\pgfqpoint{0.008840in}{-0.033333in}}{\pgfqpoint{0.017319in}{-0.029821in}}{\pgfqpoint{0.023570in}{-0.023570in}}%
\pgfpathcurveto{\pgfqpoint{0.029821in}{-0.017319in}}{\pgfqpoint{0.033333in}{-0.008840in}}{\pgfqpoint{0.033333in}{0.000000in}}%
\pgfpathcurveto{\pgfqpoint{0.033333in}{0.008840in}}{\pgfqpoint{0.029821in}{0.017319in}}{\pgfqpoint{0.023570in}{0.023570in}}%
\pgfpathcurveto{\pgfqpoint{0.017319in}{0.029821in}}{\pgfqpoint{0.008840in}{0.033333in}}{\pgfqpoint{0.000000in}{0.033333in}}%
\pgfpathcurveto{\pgfqpoint{-0.008840in}{0.033333in}}{\pgfqpoint{-0.017319in}{0.029821in}}{\pgfqpoint{-0.023570in}{0.023570in}}%
\pgfpathcurveto{\pgfqpoint{-0.029821in}{0.017319in}}{\pgfqpoint{-0.033333in}{0.008840in}}{\pgfqpoint{-0.033333in}{0.000000in}}%
\pgfpathcurveto{\pgfqpoint{-0.033333in}{-0.008840in}}{\pgfqpoint{-0.029821in}{-0.017319in}}{\pgfqpoint{-0.023570in}{-0.023570in}}%
\pgfpathcurveto{\pgfqpoint{-0.017319in}{-0.029821in}}{\pgfqpoint{-0.008840in}{-0.033333in}}{\pgfqpoint{0.000000in}{-0.033333in}}%
\pgfpathlineto{\pgfqpoint{0.000000in}{-0.033333in}}%
\pgfpathclose%
\pgfusepath{stroke,fill}%
}%
\begin{pgfscope}%
\pgfsys@transformshift{3.380366in}{2.439375in}%
\pgfsys@useobject{currentmarker}{}%
\end{pgfscope}%
\end{pgfscope}%
\begin{pgfscope}%
\pgfpathrectangle{\pgfqpoint{0.765000in}{0.660000in}}{\pgfqpoint{4.620000in}{4.620000in}}%
\pgfusepath{clip}%
\pgfsetrectcap%
\pgfsetroundjoin%
\pgfsetlinewidth{1.204500pt}%
\definecolor{currentstroke}{rgb}{1.000000,0.576471,0.309804}%
\pgfsetstrokecolor{currentstroke}%
\pgfsetdash{}{0pt}%
\pgfpathmoveto{\pgfqpoint{2.925124in}{2.675622in}}%
\pgfusepath{stroke}%
\end{pgfscope}%
\begin{pgfscope}%
\pgfpathrectangle{\pgfqpoint{0.765000in}{0.660000in}}{\pgfqpoint{4.620000in}{4.620000in}}%
\pgfusepath{clip}%
\pgfsetbuttcap%
\pgfsetroundjoin%
\definecolor{currentfill}{rgb}{1.000000,0.576471,0.309804}%
\pgfsetfillcolor{currentfill}%
\pgfsetlinewidth{1.003750pt}%
\definecolor{currentstroke}{rgb}{1.000000,0.576471,0.309804}%
\pgfsetstrokecolor{currentstroke}%
\pgfsetdash{}{0pt}%
\pgfsys@defobject{currentmarker}{\pgfqpoint{-0.033333in}{-0.033333in}}{\pgfqpoint{0.033333in}{0.033333in}}{%
\pgfpathmoveto{\pgfqpoint{0.000000in}{-0.033333in}}%
\pgfpathcurveto{\pgfqpoint{0.008840in}{-0.033333in}}{\pgfqpoint{0.017319in}{-0.029821in}}{\pgfqpoint{0.023570in}{-0.023570in}}%
\pgfpathcurveto{\pgfqpoint{0.029821in}{-0.017319in}}{\pgfqpoint{0.033333in}{-0.008840in}}{\pgfqpoint{0.033333in}{0.000000in}}%
\pgfpathcurveto{\pgfqpoint{0.033333in}{0.008840in}}{\pgfqpoint{0.029821in}{0.017319in}}{\pgfqpoint{0.023570in}{0.023570in}}%
\pgfpathcurveto{\pgfqpoint{0.017319in}{0.029821in}}{\pgfqpoint{0.008840in}{0.033333in}}{\pgfqpoint{0.000000in}{0.033333in}}%
\pgfpathcurveto{\pgfqpoint{-0.008840in}{0.033333in}}{\pgfqpoint{-0.017319in}{0.029821in}}{\pgfqpoint{-0.023570in}{0.023570in}}%
\pgfpathcurveto{\pgfqpoint{-0.029821in}{0.017319in}}{\pgfqpoint{-0.033333in}{0.008840in}}{\pgfqpoint{-0.033333in}{0.000000in}}%
\pgfpathcurveto{\pgfqpoint{-0.033333in}{-0.008840in}}{\pgfqpoint{-0.029821in}{-0.017319in}}{\pgfqpoint{-0.023570in}{-0.023570in}}%
\pgfpathcurveto{\pgfqpoint{-0.017319in}{-0.029821in}}{\pgfqpoint{-0.008840in}{-0.033333in}}{\pgfqpoint{0.000000in}{-0.033333in}}%
\pgfpathlineto{\pgfqpoint{0.000000in}{-0.033333in}}%
\pgfpathclose%
\pgfusepath{stroke,fill}%
}%
\begin{pgfscope}%
\pgfsys@transformshift{2.925124in}{2.675622in}%
\pgfsys@useobject{currentmarker}{}%
\end{pgfscope}%
\end{pgfscope}%
\begin{pgfscope}%
\pgfpathrectangle{\pgfqpoint{0.765000in}{0.660000in}}{\pgfqpoint{4.620000in}{4.620000in}}%
\pgfusepath{clip}%
\pgfsetrectcap%
\pgfsetroundjoin%
\pgfsetlinewidth{1.204500pt}%
\definecolor{currentstroke}{rgb}{1.000000,0.576471,0.309804}%
\pgfsetstrokecolor{currentstroke}%
\pgfsetdash{}{0pt}%
\pgfpathmoveto{\pgfqpoint{2.896549in}{2.540388in}}%
\pgfusepath{stroke}%
\end{pgfscope}%
\begin{pgfscope}%
\pgfpathrectangle{\pgfqpoint{0.765000in}{0.660000in}}{\pgfqpoint{4.620000in}{4.620000in}}%
\pgfusepath{clip}%
\pgfsetbuttcap%
\pgfsetroundjoin%
\definecolor{currentfill}{rgb}{1.000000,0.576471,0.309804}%
\pgfsetfillcolor{currentfill}%
\pgfsetlinewidth{1.003750pt}%
\definecolor{currentstroke}{rgb}{1.000000,0.576471,0.309804}%
\pgfsetstrokecolor{currentstroke}%
\pgfsetdash{}{0pt}%
\pgfsys@defobject{currentmarker}{\pgfqpoint{-0.033333in}{-0.033333in}}{\pgfqpoint{0.033333in}{0.033333in}}{%
\pgfpathmoveto{\pgfqpoint{0.000000in}{-0.033333in}}%
\pgfpathcurveto{\pgfqpoint{0.008840in}{-0.033333in}}{\pgfqpoint{0.017319in}{-0.029821in}}{\pgfqpoint{0.023570in}{-0.023570in}}%
\pgfpathcurveto{\pgfqpoint{0.029821in}{-0.017319in}}{\pgfqpoint{0.033333in}{-0.008840in}}{\pgfqpoint{0.033333in}{0.000000in}}%
\pgfpathcurveto{\pgfqpoint{0.033333in}{0.008840in}}{\pgfqpoint{0.029821in}{0.017319in}}{\pgfqpoint{0.023570in}{0.023570in}}%
\pgfpathcurveto{\pgfqpoint{0.017319in}{0.029821in}}{\pgfqpoint{0.008840in}{0.033333in}}{\pgfqpoint{0.000000in}{0.033333in}}%
\pgfpathcurveto{\pgfqpoint{-0.008840in}{0.033333in}}{\pgfqpoint{-0.017319in}{0.029821in}}{\pgfqpoint{-0.023570in}{0.023570in}}%
\pgfpathcurveto{\pgfqpoint{-0.029821in}{0.017319in}}{\pgfqpoint{-0.033333in}{0.008840in}}{\pgfqpoint{-0.033333in}{0.000000in}}%
\pgfpathcurveto{\pgfqpoint{-0.033333in}{-0.008840in}}{\pgfqpoint{-0.029821in}{-0.017319in}}{\pgfqpoint{-0.023570in}{-0.023570in}}%
\pgfpathcurveto{\pgfqpoint{-0.017319in}{-0.029821in}}{\pgfqpoint{-0.008840in}{-0.033333in}}{\pgfqpoint{0.000000in}{-0.033333in}}%
\pgfpathlineto{\pgfqpoint{0.000000in}{-0.033333in}}%
\pgfpathclose%
\pgfusepath{stroke,fill}%
}%
\begin{pgfscope}%
\pgfsys@transformshift{2.896549in}{2.540388in}%
\pgfsys@useobject{currentmarker}{}%
\end{pgfscope}%
\end{pgfscope}%
\begin{pgfscope}%
\pgfpathrectangle{\pgfqpoint{0.765000in}{0.660000in}}{\pgfqpoint{4.620000in}{4.620000in}}%
\pgfusepath{clip}%
\pgfsetrectcap%
\pgfsetroundjoin%
\pgfsetlinewidth{1.204500pt}%
\definecolor{currentstroke}{rgb}{1.000000,0.576471,0.309804}%
\pgfsetstrokecolor{currentstroke}%
\pgfsetdash{}{0pt}%
\pgfpathmoveto{\pgfqpoint{4.697213in}{2.792804in}}%
\pgfusepath{stroke}%
\end{pgfscope}%
\begin{pgfscope}%
\pgfpathrectangle{\pgfqpoint{0.765000in}{0.660000in}}{\pgfqpoint{4.620000in}{4.620000in}}%
\pgfusepath{clip}%
\pgfsetbuttcap%
\pgfsetroundjoin%
\definecolor{currentfill}{rgb}{1.000000,0.576471,0.309804}%
\pgfsetfillcolor{currentfill}%
\pgfsetlinewidth{1.003750pt}%
\definecolor{currentstroke}{rgb}{1.000000,0.576471,0.309804}%
\pgfsetstrokecolor{currentstroke}%
\pgfsetdash{}{0pt}%
\pgfsys@defobject{currentmarker}{\pgfqpoint{-0.033333in}{-0.033333in}}{\pgfqpoint{0.033333in}{0.033333in}}{%
\pgfpathmoveto{\pgfqpoint{0.000000in}{-0.033333in}}%
\pgfpathcurveto{\pgfqpoint{0.008840in}{-0.033333in}}{\pgfqpoint{0.017319in}{-0.029821in}}{\pgfqpoint{0.023570in}{-0.023570in}}%
\pgfpathcurveto{\pgfqpoint{0.029821in}{-0.017319in}}{\pgfqpoint{0.033333in}{-0.008840in}}{\pgfqpoint{0.033333in}{0.000000in}}%
\pgfpathcurveto{\pgfqpoint{0.033333in}{0.008840in}}{\pgfqpoint{0.029821in}{0.017319in}}{\pgfqpoint{0.023570in}{0.023570in}}%
\pgfpathcurveto{\pgfqpoint{0.017319in}{0.029821in}}{\pgfqpoint{0.008840in}{0.033333in}}{\pgfqpoint{0.000000in}{0.033333in}}%
\pgfpathcurveto{\pgfqpoint{-0.008840in}{0.033333in}}{\pgfqpoint{-0.017319in}{0.029821in}}{\pgfqpoint{-0.023570in}{0.023570in}}%
\pgfpathcurveto{\pgfqpoint{-0.029821in}{0.017319in}}{\pgfqpoint{-0.033333in}{0.008840in}}{\pgfqpoint{-0.033333in}{0.000000in}}%
\pgfpathcurveto{\pgfqpoint{-0.033333in}{-0.008840in}}{\pgfqpoint{-0.029821in}{-0.017319in}}{\pgfqpoint{-0.023570in}{-0.023570in}}%
\pgfpathcurveto{\pgfqpoint{-0.017319in}{-0.029821in}}{\pgfqpoint{-0.008840in}{-0.033333in}}{\pgfqpoint{0.000000in}{-0.033333in}}%
\pgfpathlineto{\pgfqpoint{0.000000in}{-0.033333in}}%
\pgfpathclose%
\pgfusepath{stroke,fill}%
}%
\begin{pgfscope}%
\pgfsys@transformshift{4.697213in}{2.792804in}%
\pgfsys@useobject{currentmarker}{}%
\end{pgfscope}%
\end{pgfscope}%
\begin{pgfscope}%
\pgfpathrectangle{\pgfqpoint{0.765000in}{0.660000in}}{\pgfqpoint{4.620000in}{4.620000in}}%
\pgfusepath{clip}%
\pgfsetrectcap%
\pgfsetroundjoin%
\pgfsetlinewidth{1.204500pt}%
\definecolor{currentstroke}{rgb}{1.000000,0.576471,0.309804}%
\pgfsetstrokecolor{currentstroke}%
\pgfsetdash{}{0pt}%
\pgfpathmoveto{\pgfqpoint{2.742567in}{2.544196in}}%
\pgfusepath{stroke}%
\end{pgfscope}%
\begin{pgfscope}%
\pgfpathrectangle{\pgfqpoint{0.765000in}{0.660000in}}{\pgfqpoint{4.620000in}{4.620000in}}%
\pgfusepath{clip}%
\pgfsetbuttcap%
\pgfsetroundjoin%
\definecolor{currentfill}{rgb}{1.000000,0.576471,0.309804}%
\pgfsetfillcolor{currentfill}%
\pgfsetlinewidth{1.003750pt}%
\definecolor{currentstroke}{rgb}{1.000000,0.576471,0.309804}%
\pgfsetstrokecolor{currentstroke}%
\pgfsetdash{}{0pt}%
\pgfsys@defobject{currentmarker}{\pgfqpoint{-0.033333in}{-0.033333in}}{\pgfqpoint{0.033333in}{0.033333in}}{%
\pgfpathmoveto{\pgfqpoint{0.000000in}{-0.033333in}}%
\pgfpathcurveto{\pgfqpoint{0.008840in}{-0.033333in}}{\pgfqpoint{0.017319in}{-0.029821in}}{\pgfqpoint{0.023570in}{-0.023570in}}%
\pgfpathcurveto{\pgfqpoint{0.029821in}{-0.017319in}}{\pgfqpoint{0.033333in}{-0.008840in}}{\pgfqpoint{0.033333in}{0.000000in}}%
\pgfpathcurveto{\pgfqpoint{0.033333in}{0.008840in}}{\pgfqpoint{0.029821in}{0.017319in}}{\pgfqpoint{0.023570in}{0.023570in}}%
\pgfpathcurveto{\pgfqpoint{0.017319in}{0.029821in}}{\pgfqpoint{0.008840in}{0.033333in}}{\pgfqpoint{0.000000in}{0.033333in}}%
\pgfpathcurveto{\pgfqpoint{-0.008840in}{0.033333in}}{\pgfqpoint{-0.017319in}{0.029821in}}{\pgfqpoint{-0.023570in}{0.023570in}}%
\pgfpathcurveto{\pgfqpoint{-0.029821in}{0.017319in}}{\pgfqpoint{-0.033333in}{0.008840in}}{\pgfqpoint{-0.033333in}{0.000000in}}%
\pgfpathcurveto{\pgfqpoint{-0.033333in}{-0.008840in}}{\pgfqpoint{-0.029821in}{-0.017319in}}{\pgfqpoint{-0.023570in}{-0.023570in}}%
\pgfpathcurveto{\pgfqpoint{-0.017319in}{-0.029821in}}{\pgfqpoint{-0.008840in}{-0.033333in}}{\pgfqpoint{0.000000in}{-0.033333in}}%
\pgfpathlineto{\pgfqpoint{0.000000in}{-0.033333in}}%
\pgfpathclose%
\pgfusepath{stroke,fill}%
}%
\begin{pgfscope}%
\pgfsys@transformshift{2.742567in}{2.544196in}%
\pgfsys@useobject{currentmarker}{}%
\end{pgfscope}%
\end{pgfscope}%
\begin{pgfscope}%
\pgfpathrectangle{\pgfqpoint{0.765000in}{0.660000in}}{\pgfqpoint{4.620000in}{4.620000in}}%
\pgfusepath{clip}%
\pgfsetrectcap%
\pgfsetroundjoin%
\pgfsetlinewidth{1.204500pt}%
\definecolor{currentstroke}{rgb}{1.000000,0.576471,0.309804}%
\pgfsetstrokecolor{currentstroke}%
\pgfsetdash{}{0pt}%
\pgfpathmoveto{\pgfqpoint{4.376535in}{2.753859in}}%
\pgfusepath{stroke}%
\end{pgfscope}%
\begin{pgfscope}%
\pgfpathrectangle{\pgfqpoint{0.765000in}{0.660000in}}{\pgfqpoint{4.620000in}{4.620000in}}%
\pgfusepath{clip}%
\pgfsetbuttcap%
\pgfsetroundjoin%
\definecolor{currentfill}{rgb}{1.000000,0.576471,0.309804}%
\pgfsetfillcolor{currentfill}%
\pgfsetlinewidth{1.003750pt}%
\definecolor{currentstroke}{rgb}{1.000000,0.576471,0.309804}%
\pgfsetstrokecolor{currentstroke}%
\pgfsetdash{}{0pt}%
\pgfsys@defobject{currentmarker}{\pgfqpoint{-0.033333in}{-0.033333in}}{\pgfqpoint{0.033333in}{0.033333in}}{%
\pgfpathmoveto{\pgfqpoint{0.000000in}{-0.033333in}}%
\pgfpathcurveto{\pgfqpoint{0.008840in}{-0.033333in}}{\pgfqpoint{0.017319in}{-0.029821in}}{\pgfqpoint{0.023570in}{-0.023570in}}%
\pgfpathcurveto{\pgfqpoint{0.029821in}{-0.017319in}}{\pgfqpoint{0.033333in}{-0.008840in}}{\pgfqpoint{0.033333in}{0.000000in}}%
\pgfpathcurveto{\pgfqpoint{0.033333in}{0.008840in}}{\pgfqpoint{0.029821in}{0.017319in}}{\pgfqpoint{0.023570in}{0.023570in}}%
\pgfpathcurveto{\pgfqpoint{0.017319in}{0.029821in}}{\pgfqpoint{0.008840in}{0.033333in}}{\pgfqpoint{0.000000in}{0.033333in}}%
\pgfpathcurveto{\pgfqpoint{-0.008840in}{0.033333in}}{\pgfqpoint{-0.017319in}{0.029821in}}{\pgfqpoint{-0.023570in}{0.023570in}}%
\pgfpathcurveto{\pgfqpoint{-0.029821in}{0.017319in}}{\pgfqpoint{-0.033333in}{0.008840in}}{\pgfqpoint{-0.033333in}{0.000000in}}%
\pgfpathcurveto{\pgfqpoint{-0.033333in}{-0.008840in}}{\pgfqpoint{-0.029821in}{-0.017319in}}{\pgfqpoint{-0.023570in}{-0.023570in}}%
\pgfpathcurveto{\pgfqpoint{-0.017319in}{-0.029821in}}{\pgfqpoint{-0.008840in}{-0.033333in}}{\pgfqpoint{0.000000in}{-0.033333in}}%
\pgfpathlineto{\pgfqpoint{0.000000in}{-0.033333in}}%
\pgfpathclose%
\pgfusepath{stroke,fill}%
}%
\begin{pgfscope}%
\pgfsys@transformshift{4.376535in}{2.753859in}%
\pgfsys@useobject{currentmarker}{}%
\end{pgfscope}%
\end{pgfscope}%
\begin{pgfscope}%
\pgfpathrectangle{\pgfqpoint{0.765000in}{0.660000in}}{\pgfqpoint{4.620000in}{4.620000in}}%
\pgfusepath{clip}%
\pgfsetrectcap%
\pgfsetroundjoin%
\pgfsetlinewidth{1.204500pt}%
\definecolor{currentstroke}{rgb}{1.000000,0.576471,0.309804}%
\pgfsetstrokecolor{currentstroke}%
\pgfsetdash{}{0pt}%
\pgfpathmoveto{\pgfqpoint{2.695216in}{3.051958in}}%
\pgfusepath{stroke}%
\end{pgfscope}%
\begin{pgfscope}%
\pgfpathrectangle{\pgfqpoint{0.765000in}{0.660000in}}{\pgfqpoint{4.620000in}{4.620000in}}%
\pgfusepath{clip}%
\pgfsetbuttcap%
\pgfsetroundjoin%
\definecolor{currentfill}{rgb}{1.000000,0.576471,0.309804}%
\pgfsetfillcolor{currentfill}%
\pgfsetlinewidth{1.003750pt}%
\definecolor{currentstroke}{rgb}{1.000000,0.576471,0.309804}%
\pgfsetstrokecolor{currentstroke}%
\pgfsetdash{}{0pt}%
\pgfsys@defobject{currentmarker}{\pgfqpoint{-0.033333in}{-0.033333in}}{\pgfqpoint{0.033333in}{0.033333in}}{%
\pgfpathmoveto{\pgfqpoint{0.000000in}{-0.033333in}}%
\pgfpathcurveto{\pgfqpoint{0.008840in}{-0.033333in}}{\pgfqpoint{0.017319in}{-0.029821in}}{\pgfqpoint{0.023570in}{-0.023570in}}%
\pgfpathcurveto{\pgfqpoint{0.029821in}{-0.017319in}}{\pgfqpoint{0.033333in}{-0.008840in}}{\pgfqpoint{0.033333in}{0.000000in}}%
\pgfpathcurveto{\pgfqpoint{0.033333in}{0.008840in}}{\pgfqpoint{0.029821in}{0.017319in}}{\pgfqpoint{0.023570in}{0.023570in}}%
\pgfpathcurveto{\pgfqpoint{0.017319in}{0.029821in}}{\pgfqpoint{0.008840in}{0.033333in}}{\pgfqpoint{0.000000in}{0.033333in}}%
\pgfpathcurveto{\pgfqpoint{-0.008840in}{0.033333in}}{\pgfqpoint{-0.017319in}{0.029821in}}{\pgfqpoint{-0.023570in}{0.023570in}}%
\pgfpathcurveto{\pgfqpoint{-0.029821in}{0.017319in}}{\pgfqpoint{-0.033333in}{0.008840in}}{\pgfqpoint{-0.033333in}{0.000000in}}%
\pgfpathcurveto{\pgfqpoint{-0.033333in}{-0.008840in}}{\pgfqpoint{-0.029821in}{-0.017319in}}{\pgfqpoint{-0.023570in}{-0.023570in}}%
\pgfpathcurveto{\pgfqpoint{-0.017319in}{-0.029821in}}{\pgfqpoint{-0.008840in}{-0.033333in}}{\pgfqpoint{0.000000in}{-0.033333in}}%
\pgfpathlineto{\pgfqpoint{0.000000in}{-0.033333in}}%
\pgfpathclose%
\pgfusepath{stroke,fill}%
}%
\begin{pgfscope}%
\pgfsys@transformshift{2.695216in}{3.051958in}%
\pgfsys@useobject{currentmarker}{}%
\end{pgfscope}%
\end{pgfscope}%
\begin{pgfscope}%
\pgfpathrectangle{\pgfqpoint{0.765000in}{0.660000in}}{\pgfqpoint{4.620000in}{4.620000in}}%
\pgfusepath{clip}%
\pgfsetrectcap%
\pgfsetroundjoin%
\pgfsetlinewidth{1.204500pt}%
\definecolor{currentstroke}{rgb}{1.000000,0.576471,0.309804}%
\pgfsetstrokecolor{currentstroke}%
\pgfsetdash{}{0pt}%
\pgfpathmoveto{\pgfqpoint{2.697487in}{2.594797in}}%
\pgfusepath{stroke}%
\end{pgfscope}%
\begin{pgfscope}%
\pgfpathrectangle{\pgfqpoint{0.765000in}{0.660000in}}{\pgfqpoint{4.620000in}{4.620000in}}%
\pgfusepath{clip}%
\pgfsetbuttcap%
\pgfsetroundjoin%
\definecolor{currentfill}{rgb}{1.000000,0.576471,0.309804}%
\pgfsetfillcolor{currentfill}%
\pgfsetlinewidth{1.003750pt}%
\definecolor{currentstroke}{rgb}{1.000000,0.576471,0.309804}%
\pgfsetstrokecolor{currentstroke}%
\pgfsetdash{}{0pt}%
\pgfsys@defobject{currentmarker}{\pgfqpoint{-0.033333in}{-0.033333in}}{\pgfqpoint{0.033333in}{0.033333in}}{%
\pgfpathmoveto{\pgfqpoint{0.000000in}{-0.033333in}}%
\pgfpathcurveto{\pgfqpoint{0.008840in}{-0.033333in}}{\pgfqpoint{0.017319in}{-0.029821in}}{\pgfqpoint{0.023570in}{-0.023570in}}%
\pgfpathcurveto{\pgfqpoint{0.029821in}{-0.017319in}}{\pgfqpoint{0.033333in}{-0.008840in}}{\pgfqpoint{0.033333in}{0.000000in}}%
\pgfpathcurveto{\pgfqpoint{0.033333in}{0.008840in}}{\pgfqpoint{0.029821in}{0.017319in}}{\pgfqpoint{0.023570in}{0.023570in}}%
\pgfpathcurveto{\pgfqpoint{0.017319in}{0.029821in}}{\pgfqpoint{0.008840in}{0.033333in}}{\pgfqpoint{0.000000in}{0.033333in}}%
\pgfpathcurveto{\pgfqpoint{-0.008840in}{0.033333in}}{\pgfqpoint{-0.017319in}{0.029821in}}{\pgfqpoint{-0.023570in}{0.023570in}}%
\pgfpathcurveto{\pgfqpoint{-0.029821in}{0.017319in}}{\pgfqpoint{-0.033333in}{0.008840in}}{\pgfqpoint{-0.033333in}{0.000000in}}%
\pgfpathcurveto{\pgfqpoint{-0.033333in}{-0.008840in}}{\pgfqpoint{-0.029821in}{-0.017319in}}{\pgfqpoint{-0.023570in}{-0.023570in}}%
\pgfpathcurveto{\pgfqpoint{-0.017319in}{-0.029821in}}{\pgfqpoint{-0.008840in}{-0.033333in}}{\pgfqpoint{0.000000in}{-0.033333in}}%
\pgfpathlineto{\pgfqpoint{0.000000in}{-0.033333in}}%
\pgfpathclose%
\pgfusepath{stroke,fill}%
}%
\begin{pgfscope}%
\pgfsys@transformshift{2.697487in}{2.594797in}%
\pgfsys@useobject{currentmarker}{}%
\end{pgfscope}%
\end{pgfscope}%
\begin{pgfscope}%
\pgfpathrectangle{\pgfqpoint{0.765000in}{0.660000in}}{\pgfqpoint{4.620000in}{4.620000in}}%
\pgfusepath{clip}%
\pgfsetrectcap%
\pgfsetroundjoin%
\pgfsetlinewidth{1.204500pt}%
\definecolor{currentstroke}{rgb}{1.000000,0.576471,0.309804}%
\pgfsetstrokecolor{currentstroke}%
\pgfsetdash{}{0pt}%
\pgfpathmoveto{\pgfqpoint{2.717079in}{2.636880in}}%
\pgfusepath{stroke}%
\end{pgfscope}%
\begin{pgfscope}%
\pgfpathrectangle{\pgfqpoint{0.765000in}{0.660000in}}{\pgfqpoint{4.620000in}{4.620000in}}%
\pgfusepath{clip}%
\pgfsetbuttcap%
\pgfsetroundjoin%
\definecolor{currentfill}{rgb}{1.000000,0.576471,0.309804}%
\pgfsetfillcolor{currentfill}%
\pgfsetlinewidth{1.003750pt}%
\definecolor{currentstroke}{rgb}{1.000000,0.576471,0.309804}%
\pgfsetstrokecolor{currentstroke}%
\pgfsetdash{}{0pt}%
\pgfsys@defobject{currentmarker}{\pgfqpoint{-0.033333in}{-0.033333in}}{\pgfqpoint{0.033333in}{0.033333in}}{%
\pgfpathmoveto{\pgfqpoint{0.000000in}{-0.033333in}}%
\pgfpathcurveto{\pgfqpoint{0.008840in}{-0.033333in}}{\pgfqpoint{0.017319in}{-0.029821in}}{\pgfqpoint{0.023570in}{-0.023570in}}%
\pgfpathcurveto{\pgfqpoint{0.029821in}{-0.017319in}}{\pgfqpoint{0.033333in}{-0.008840in}}{\pgfqpoint{0.033333in}{0.000000in}}%
\pgfpathcurveto{\pgfqpoint{0.033333in}{0.008840in}}{\pgfqpoint{0.029821in}{0.017319in}}{\pgfqpoint{0.023570in}{0.023570in}}%
\pgfpathcurveto{\pgfqpoint{0.017319in}{0.029821in}}{\pgfqpoint{0.008840in}{0.033333in}}{\pgfqpoint{0.000000in}{0.033333in}}%
\pgfpathcurveto{\pgfqpoint{-0.008840in}{0.033333in}}{\pgfqpoint{-0.017319in}{0.029821in}}{\pgfqpoint{-0.023570in}{0.023570in}}%
\pgfpathcurveto{\pgfqpoint{-0.029821in}{0.017319in}}{\pgfqpoint{-0.033333in}{0.008840in}}{\pgfqpoint{-0.033333in}{0.000000in}}%
\pgfpathcurveto{\pgfqpoint{-0.033333in}{-0.008840in}}{\pgfqpoint{-0.029821in}{-0.017319in}}{\pgfqpoint{-0.023570in}{-0.023570in}}%
\pgfpathcurveto{\pgfqpoint{-0.017319in}{-0.029821in}}{\pgfqpoint{-0.008840in}{-0.033333in}}{\pgfqpoint{0.000000in}{-0.033333in}}%
\pgfpathlineto{\pgfqpoint{0.000000in}{-0.033333in}}%
\pgfpathclose%
\pgfusepath{stroke,fill}%
}%
\begin{pgfscope}%
\pgfsys@transformshift{2.717079in}{2.636880in}%
\pgfsys@useobject{currentmarker}{}%
\end{pgfscope}%
\end{pgfscope}%
\begin{pgfscope}%
\pgfpathrectangle{\pgfqpoint{0.765000in}{0.660000in}}{\pgfqpoint{4.620000in}{4.620000in}}%
\pgfusepath{clip}%
\pgfsetrectcap%
\pgfsetroundjoin%
\pgfsetlinewidth{1.204500pt}%
\definecolor{currentstroke}{rgb}{1.000000,0.576471,0.309804}%
\pgfsetstrokecolor{currentstroke}%
\pgfsetdash{}{0pt}%
\pgfpathmoveto{\pgfqpoint{4.923136in}{3.289740in}}%
\pgfusepath{stroke}%
\end{pgfscope}%
\begin{pgfscope}%
\pgfpathrectangle{\pgfqpoint{0.765000in}{0.660000in}}{\pgfqpoint{4.620000in}{4.620000in}}%
\pgfusepath{clip}%
\pgfsetbuttcap%
\pgfsetroundjoin%
\definecolor{currentfill}{rgb}{1.000000,0.576471,0.309804}%
\pgfsetfillcolor{currentfill}%
\pgfsetlinewidth{1.003750pt}%
\definecolor{currentstroke}{rgb}{1.000000,0.576471,0.309804}%
\pgfsetstrokecolor{currentstroke}%
\pgfsetdash{}{0pt}%
\pgfsys@defobject{currentmarker}{\pgfqpoint{-0.033333in}{-0.033333in}}{\pgfqpoint{0.033333in}{0.033333in}}{%
\pgfpathmoveto{\pgfqpoint{0.000000in}{-0.033333in}}%
\pgfpathcurveto{\pgfqpoint{0.008840in}{-0.033333in}}{\pgfqpoint{0.017319in}{-0.029821in}}{\pgfqpoint{0.023570in}{-0.023570in}}%
\pgfpathcurveto{\pgfqpoint{0.029821in}{-0.017319in}}{\pgfqpoint{0.033333in}{-0.008840in}}{\pgfqpoint{0.033333in}{0.000000in}}%
\pgfpathcurveto{\pgfqpoint{0.033333in}{0.008840in}}{\pgfqpoint{0.029821in}{0.017319in}}{\pgfqpoint{0.023570in}{0.023570in}}%
\pgfpathcurveto{\pgfqpoint{0.017319in}{0.029821in}}{\pgfqpoint{0.008840in}{0.033333in}}{\pgfqpoint{0.000000in}{0.033333in}}%
\pgfpathcurveto{\pgfqpoint{-0.008840in}{0.033333in}}{\pgfqpoint{-0.017319in}{0.029821in}}{\pgfqpoint{-0.023570in}{0.023570in}}%
\pgfpathcurveto{\pgfqpoint{-0.029821in}{0.017319in}}{\pgfqpoint{-0.033333in}{0.008840in}}{\pgfqpoint{-0.033333in}{0.000000in}}%
\pgfpathcurveto{\pgfqpoint{-0.033333in}{-0.008840in}}{\pgfqpoint{-0.029821in}{-0.017319in}}{\pgfqpoint{-0.023570in}{-0.023570in}}%
\pgfpathcurveto{\pgfqpoint{-0.017319in}{-0.029821in}}{\pgfqpoint{-0.008840in}{-0.033333in}}{\pgfqpoint{0.000000in}{-0.033333in}}%
\pgfpathlineto{\pgfqpoint{0.000000in}{-0.033333in}}%
\pgfpathclose%
\pgfusepath{stroke,fill}%
}%
\begin{pgfscope}%
\pgfsys@transformshift{4.923136in}{3.289740in}%
\pgfsys@useobject{currentmarker}{}%
\end{pgfscope}%
\end{pgfscope}%
\begin{pgfscope}%
\pgfpathrectangle{\pgfqpoint{0.765000in}{0.660000in}}{\pgfqpoint{4.620000in}{4.620000in}}%
\pgfusepath{clip}%
\pgfsetrectcap%
\pgfsetroundjoin%
\pgfsetlinewidth{1.204500pt}%
\definecolor{currentstroke}{rgb}{1.000000,0.576471,0.309804}%
\pgfsetstrokecolor{currentstroke}%
\pgfsetdash{}{0pt}%
\pgfpathmoveto{\pgfqpoint{3.196396in}{2.426574in}}%
\pgfusepath{stroke}%
\end{pgfscope}%
\begin{pgfscope}%
\pgfpathrectangle{\pgfqpoint{0.765000in}{0.660000in}}{\pgfqpoint{4.620000in}{4.620000in}}%
\pgfusepath{clip}%
\pgfsetbuttcap%
\pgfsetroundjoin%
\definecolor{currentfill}{rgb}{1.000000,0.576471,0.309804}%
\pgfsetfillcolor{currentfill}%
\pgfsetlinewidth{1.003750pt}%
\definecolor{currentstroke}{rgb}{1.000000,0.576471,0.309804}%
\pgfsetstrokecolor{currentstroke}%
\pgfsetdash{}{0pt}%
\pgfsys@defobject{currentmarker}{\pgfqpoint{-0.033333in}{-0.033333in}}{\pgfqpoint{0.033333in}{0.033333in}}{%
\pgfpathmoveto{\pgfqpoint{0.000000in}{-0.033333in}}%
\pgfpathcurveto{\pgfqpoint{0.008840in}{-0.033333in}}{\pgfqpoint{0.017319in}{-0.029821in}}{\pgfqpoint{0.023570in}{-0.023570in}}%
\pgfpathcurveto{\pgfqpoint{0.029821in}{-0.017319in}}{\pgfqpoint{0.033333in}{-0.008840in}}{\pgfqpoint{0.033333in}{0.000000in}}%
\pgfpathcurveto{\pgfqpoint{0.033333in}{0.008840in}}{\pgfqpoint{0.029821in}{0.017319in}}{\pgfqpoint{0.023570in}{0.023570in}}%
\pgfpathcurveto{\pgfqpoint{0.017319in}{0.029821in}}{\pgfqpoint{0.008840in}{0.033333in}}{\pgfqpoint{0.000000in}{0.033333in}}%
\pgfpathcurveto{\pgfqpoint{-0.008840in}{0.033333in}}{\pgfqpoint{-0.017319in}{0.029821in}}{\pgfqpoint{-0.023570in}{0.023570in}}%
\pgfpathcurveto{\pgfqpoint{-0.029821in}{0.017319in}}{\pgfqpoint{-0.033333in}{0.008840in}}{\pgfqpoint{-0.033333in}{0.000000in}}%
\pgfpathcurveto{\pgfqpoint{-0.033333in}{-0.008840in}}{\pgfqpoint{-0.029821in}{-0.017319in}}{\pgfqpoint{-0.023570in}{-0.023570in}}%
\pgfpathcurveto{\pgfqpoint{-0.017319in}{-0.029821in}}{\pgfqpoint{-0.008840in}{-0.033333in}}{\pgfqpoint{0.000000in}{-0.033333in}}%
\pgfpathlineto{\pgfqpoint{0.000000in}{-0.033333in}}%
\pgfpathclose%
\pgfusepath{stroke,fill}%
}%
\begin{pgfscope}%
\pgfsys@transformshift{3.196396in}{2.426574in}%
\pgfsys@useobject{currentmarker}{}%
\end{pgfscope}%
\end{pgfscope}%
\begin{pgfscope}%
\pgfpathrectangle{\pgfqpoint{0.765000in}{0.660000in}}{\pgfqpoint{4.620000in}{4.620000in}}%
\pgfusepath{clip}%
\pgfsetrectcap%
\pgfsetroundjoin%
\pgfsetlinewidth{1.204500pt}%
\definecolor{currentstroke}{rgb}{1.000000,0.576471,0.309804}%
\pgfsetstrokecolor{currentstroke}%
\pgfsetdash{}{0pt}%
\pgfpathmoveto{\pgfqpoint{4.169633in}{2.725769in}}%
\pgfusepath{stroke}%
\end{pgfscope}%
\begin{pgfscope}%
\pgfpathrectangle{\pgfqpoint{0.765000in}{0.660000in}}{\pgfqpoint{4.620000in}{4.620000in}}%
\pgfusepath{clip}%
\pgfsetbuttcap%
\pgfsetroundjoin%
\definecolor{currentfill}{rgb}{1.000000,0.576471,0.309804}%
\pgfsetfillcolor{currentfill}%
\pgfsetlinewidth{1.003750pt}%
\definecolor{currentstroke}{rgb}{1.000000,0.576471,0.309804}%
\pgfsetstrokecolor{currentstroke}%
\pgfsetdash{}{0pt}%
\pgfsys@defobject{currentmarker}{\pgfqpoint{-0.033333in}{-0.033333in}}{\pgfqpoint{0.033333in}{0.033333in}}{%
\pgfpathmoveto{\pgfqpoint{0.000000in}{-0.033333in}}%
\pgfpathcurveto{\pgfqpoint{0.008840in}{-0.033333in}}{\pgfqpoint{0.017319in}{-0.029821in}}{\pgfqpoint{0.023570in}{-0.023570in}}%
\pgfpathcurveto{\pgfqpoint{0.029821in}{-0.017319in}}{\pgfqpoint{0.033333in}{-0.008840in}}{\pgfqpoint{0.033333in}{0.000000in}}%
\pgfpathcurveto{\pgfqpoint{0.033333in}{0.008840in}}{\pgfqpoint{0.029821in}{0.017319in}}{\pgfqpoint{0.023570in}{0.023570in}}%
\pgfpathcurveto{\pgfqpoint{0.017319in}{0.029821in}}{\pgfqpoint{0.008840in}{0.033333in}}{\pgfqpoint{0.000000in}{0.033333in}}%
\pgfpathcurveto{\pgfqpoint{-0.008840in}{0.033333in}}{\pgfqpoint{-0.017319in}{0.029821in}}{\pgfqpoint{-0.023570in}{0.023570in}}%
\pgfpathcurveto{\pgfqpoint{-0.029821in}{0.017319in}}{\pgfqpoint{-0.033333in}{0.008840in}}{\pgfqpoint{-0.033333in}{0.000000in}}%
\pgfpathcurveto{\pgfqpoint{-0.033333in}{-0.008840in}}{\pgfqpoint{-0.029821in}{-0.017319in}}{\pgfqpoint{-0.023570in}{-0.023570in}}%
\pgfpathcurveto{\pgfqpoint{-0.017319in}{-0.029821in}}{\pgfqpoint{-0.008840in}{-0.033333in}}{\pgfqpoint{0.000000in}{-0.033333in}}%
\pgfpathlineto{\pgfqpoint{0.000000in}{-0.033333in}}%
\pgfpathclose%
\pgfusepath{stroke,fill}%
}%
\begin{pgfscope}%
\pgfsys@transformshift{4.169633in}{2.725769in}%
\pgfsys@useobject{currentmarker}{}%
\end{pgfscope}%
\end{pgfscope}%
\begin{pgfscope}%
\pgfpathrectangle{\pgfqpoint{0.765000in}{0.660000in}}{\pgfqpoint{4.620000in}{4.620000in}}%
\pgfusepath{clip}%
\pgfsetrectcap%
\pgfsetroundjoin%
\pgfsetlinewidth{1.204500pt}%
\definecolor{currentstroke}{rgb}{1.000000,0.576471,0.309804}%
\pgfsetstrokecolor{currentstroke}%
\pgfsetdash{}{0pt}%
\pgfpathmoveto{\pgfqpoint{3.047359in}{3.553061in}}%
\pgfusepath{stroke}%
\end{pgfscope}%
\begin{pgfscope}%
\pgfpathrectangle{\pgfqpoint{0.765000in}{0.660000in}}{\pgfqpoint{4.620000in}{4.620000in}}%
\pgfusepath{clip}%
\pgfsetbuttcap%
\pgfsetroundjoin%
\definecolor{currentfill}{rgb}{1.000000,0.576471,0.309804}%
\pgfsetfillcolor{currentfill}%
\pgfsetlinewidth{1.003750pt}%
\definecolor{currentstroke}{rgb}{1.000000,0.576471,0.309804}%
\pgfsetstrokecolor{currentstroke}%
\pgfsetdash{}{0pt}%
\pgfsys@defobject{currentmarker}{\pgfqpoint{-0.033333in}{-0.033333in}}{\pgfqpoint{0.033333in}{0.033333in}}{%
\pgfpathmoveto{\pgfqpoint{0.000000in}{-0.033333in}}%
\pgfpathcurveto{\pgfqpoint{0.008840in}{-0.033333in}}{\pgfqpoint{0.017319in}{-0.029821in}}{\pgfqpoint{0.023570in}{-0.023570in}}%
\pgfpathcurveto{\pgfqpoint{0.029821in}{-0.017319in}}{\pgfqpoint{0.033333in}{-0.008840in}}{\pgfqpoint{0.033333in}{0.000000in}}%
\pgfpathcurveto{\pgfqpoint{0.033333in}{0.008840in}}{\pgfqpoint{0.029821in}{0.017319in}}{\pgfqpoint{0.023570in}{0.023570in}}%
\pgfpathcurveto{\pgfqpoint{0.017319in}{0.029821in}}{\pgfqpoint{0.008840in}{0.033333in}}{\pgfqpoint{0.000000in}{0.033333in}}%
\pgfpathcurveto{\pgfqpoint{-0.008840in}{0.033333in}}{\pgfqpoint{-0.017319in}{0.029821in}}{\pgfqpoint{-0.023570in}{0.023570in}}%
\pgfpathcurveto{\pgfqpoint{-0.029821in}{0.017319in}}{\pgfqpoint{-0.033333in}{0.008840in}}{\pgfqpoint{-0.033333in}{0.000000in}}%
\pgfpathcurveto{\pgfqpoint{-0.033333in}{-0.008840in}}{\pgfqpoint{-0.029821in}{-0.017319in}}{\pgfqpoint{-0.023570in}{-0.023570in}}%
\pgfpathcurveto{\pgfqpoint{-0.017319in}{-0.029821in}}{\pgfqpoint{-0.008840in}{-0.033333in}}{\pgfqpoint{0.000000in}{-0.033333in}}%
\pgfpathlineto{\pgfqpoint{0.000000in}{-0.033333in}}%
\pgfpathclose%
\pgfusepath{stroke,fill}%
}%
\begin{pgfscope}%
\pgfsys@transformshift{3.047359in}{3.553061in}%
\pgfsys@useobject{currentmarker}{}%
\end{pgfscope}%
\end{pgfscope}%
\begin{pgfscope}%
\pgfpathrectangle{\pgfqpoint{0.765000in}{0.660000in}}{\pgfqpoint{4.620000in}{4.620000in}}%
\pgfusepath{clip}%
\pgfsetrectcap%
\pgfsetroundjoin%
\pgfsetlinewidth{1.204500pt}%
\definecolor{currentstroke}{rgb}{1.000000,0.576471,0.309804}%
\pgfsetstrokecolor{currentstroke}%
\pgfsetdash{}{0pt}%
\pgfpathmoveto{\pgfqpoint{3.424540in}{2.719085in}}%
\pgfusepath{stroke}%
\end{pgfscope}%
\begin{pgfscope}%
\pgfpathrectangle{\pgfqpoint{0.765000in}{0.660000in}}{\pgfqpoint{4.620000in}{4.620000in}}%
\pgfusepath{clip}%
\pgfsetbuttcap%
\pgfsetroundjoin%
\definecolor{currentfill}{rgb}{1.000000,0.576471,0.309804}%
\pgfsetfillcolor{currentfill}%
\pgfsetlinewidth{1.003750pt}%
\definecolor{currentstroke}{rgb}{1.000000,0.576471,0.309804}%
\pgfsetstrokecolor{currentstroke}%
\pgfsetdash{}{0pt}%
\pgfsys@defobject{currentmarker}{\pgfqpoint{-0.033333in}{-0.033333in}}{\pgfqpoint{0.033333in}{0.033333in}}{%
\pgfpathmoveto{\pgfqpoint{0.000000in}{-0.033333in}}%
\pgfpathcurveto{\pgfqpoint{0.008840in}{-0.033333in}}{\pgfqpoint{0.017319in}{-0.029821in}}{\pgfqpoint{0.023570in}{-0.023570in}}%
\pgfpathcurveto{\pgfqpoint{0.029821in}{-0.017319in}}{\pgfqpoint{0.033333in}{-0.008840in}}{\pgfqpoint{0.033333in}{0.000000in}}%
\pgfpathcurveto{\pgfqpoint{0.033333in}{0.008840in}}{\pgfqpoint{0.029821in}{0.017319in}}{\pgfqpoint{0.023570in}{0.023570in}}%
\pgfpathcurveto{\pgfqpoint{0.017319in}{0.029821in}}{\pgfqpoint{0.008840in}{0.033333in}}{\pgfqpoint{0.000000in}{0.033333in}}%
\pgfpathcurveto{\pgfqpoint{-0.008840in}{0.033333in}}{\pgfqpoint{-0.017319in}{0.029821in}}{\pgfqpoint{-0.023570in}{0.023570in}}%
\pgfpathcurveto{\pgfqpoint{-0.029821in}{0.017319in}}{\pgfqpoint{-0.033333in}{0.008840in}}{\pgfqpoint{-0.033333in}{0.000000in}}%
\pgfpathcurveto{\pgfqpoint{-0.033333in}{-0.008840in}}{\pgfqpoint{-0.029821in}{-0.017319in}}{\pgfqpoint{-0.023570in}{-0.023570in}}%
\pgfpathcurveto{\pgfqpoint{-0.017319in}{-0.029821in}}{\pgfqpoint{-0.008840in}{-0.033333in}}{\pgfqpoint{0.000000in}{-0.033333in}}%
\pgfpathlineto{\pgfqpoint{0.000000in}{-0.033333in}}%
\pgfpathclose%
\pgfusepath{stroke,fill}%
}%
\begin{pgfscope}%
\pgfsys@transformshift{3.424540in}{2.719085in}%
\pgfsys@useobject{currentmarker}{}%
\end{pgfscope}%
\end{pgfscope}%
\begin{pgfscope}%
\pgfpathrectangle{\pgfqpoint{0.765000in}{0.660000in}}{\pgfqpoint{4.620000in}{4.620000in}}%
\pgfusepath{clip}%
\pgfsetrectcap%
\pgfsetroundjoin%
\pgfsetlinewidth{1.204500pt}%
\definecolor{currentstroke}{rgb}{1.000000,0.576471,0.309804}%
\pgfsetstrokecolor{currentstroke}%
\pgfsetdash{}{0pt}%
\pgfpathmoveto{\pgfqpoint{3.811663in}{2.678737in}}%
\pgfusepath{stroke}%
\end{pgfscope}%
\begin{pgfscope}%
\pgfpathrectangle{\pgfqpoint{0.765000in}{0.660000in}}{\pgfqpoint{4.620000in}{4.620000in}}%
\pgfusepath{clip}%
\pgfsetbuttcap%
\pgfsetroundjoin%
\definecolor{currentfill}{rgb}{1.000000,0.576471,0.309804}%
\pgfsetfillcolor{currentfill}%
\pgfsetlinewidth{1.003750pt}%
\definecolor{currentstroke}{rgb}{1.000000,0.576471,0.309804}%
\pgfsetstrokecolor{currentstroke}%
\pgfsetdash{}{0pt}%
\pgfsys@defobject{currentmarker}{\pgfqpoint{-0.033333in}{-0.033333in}}{\pgfqpoint{0.033333in}{0.033333in}}{%
\pgfpathmoveto{\pgfqpoint{0.000000in}{-0.033333in}}%
\pgfpathcurveto{\pgfqpoint{0.008840in}{-0.033333in}}{\pgfqpoint{0.017319in}{-0.029821in}}{\pgfqpoint{0.023570in}{-0.023570in}}%
\pgfpathcurveto{\pgfqpoint{0.029821in}{-0.017319in}}{\pgfqpoint{0.033333in}{-0.008840in}}{\pgfqpoint{0.033333in}{0.000000in}}%
\pgfpathcurveto{\pgfqpoint{0.033333in}{0.008840in}}{\pgfqpoint{0.029821in}{0.017319in}}{\pgfqpoint{0.023570in}{0.023570in}}%
\pgfpathcurveto{\pgfqpoint{0.017319in}{0.029821in}}{\pgfqpoint{0.008840in}{0.033333in}}{\pgfqpoint{0.000000in}{0.033333in}}%
\pgfpathcurveto{\pgfqpoint{-0.008840in}{0.033333in}}{\pgfqpoint{-0.017319in}{0.029821in}}{\pgfqpoint{-0.023570in}{0.023570in}}%
\pgfpathcurveto{\pgfqpoint{-0.029821in}{0.017319in}}{\pgfqpoint{-0.033333in}{0.008840in}}{\pgfqpoint{-0.033333in}{0.000000in}}%
\pgfpathcurveto{\pgfqpoint{-0.033333in}{-0.008840in}}{\pgfqpoint{-0.029821in}{-0.017319in}}{\pgfqpoint{-0.023570in}{-0.023570in}}%
\pgfpathcurveto{\pgfqpoint{-0.017319in}{-0.029821in}}{\pgfqpoint{-0.008840in}{-0.033333in}}{\pgfqpoint{0.000000in}{-0.033333in}}%
\pgfpathlineto{\pgfqpoint{0.000000in}{-0.033333in}}%
\pgfpathclose%
\pgfusepath{stroke,fill}%
}%
\begin{pgfscope}%
\pgfsys@transformshift{3.811663in}{2.678737in}%
\pgfsys@useobject{currentmarker}{}%
\end{pgfscope}%
\end{pgfscope}%
\begin{pgfscope}%
\pgfpathrectangle{\pgfqpoint{0.765000in}{0.660000in}}{\pgfqpoint{4.620000in}{4.620000in}}%
\pgfusepath{clip}%
\pgfsetrectcap%
\pgfsetroundjoin%
\pgfsetlinewidth{1.204500pt}%
\definecolor{currentstroke}{rgb}{1.000000,0.576471,0.309804}%
\pgfsetstrokecolor{currentstroke}%
\pgfsetdash{}{0pt}%
\pgfpathmoveto{\pgfqpoint{2.707118in}{2.475054in}}%
\pgfusepath{stroke}%
\end{pgfscope}%
\begin{pgfscope}%
\pgfpathrectangle{\pgfqpoint{0.765000in}{0.660000in}}{\pgfqpoint{4.620000in}{4.620000in}}%
\pgfusepath{clip}%
\pgfsetbuttcap%
\pgfsetroundjoin%
\definecolor{currentfill}{rgb}{1.000000,0.576471,0.309804}%
\pgfsetfillcolor{currentfill}%
\pgfsetlinewidth{1.003750pt}%
\definecolor{currentstroke}{rgb}{1.000000,0.576471,0.309804}%
\pgfsetstrokecolor{currentstroke}%
\pgfsetdash{}{0pt}%
\pgfsys@defobject{currentmarker}{\pgfqpoint{-0.033333in}{-0.033333in}}{\pgfqpoint{0.033333in}{0.033333in}}{%
\pgfpathmoveto{\pgfqpoint{0.000000in}{-0.033333in}}%
\pgfpathcurveto{\pgfqpoint{0.008840in}{-0.033333in}}{\pgfqpoint{0.017319in}{-0.029821in}}{\pgfqpoint{0.023570in}{-0.023570in}}%
\pgfpathcurveto{\pgfqpoint{0.029821in}{-0.017319in}}{\pgfqpoint{0.033333in}{-0.008840in}}{\pgfqpoint{0.033333in}{0.000000in}}%
\pgfpathcurveto{\pgfqpoint{0.033333in}{0.008840in}}{\pgfqpoint{0.029821in}{0.017319in}}{\pgfqpoint{0.023570in}{0.023570in}}%
\pgfpathcurveto{\pgfqpoint{0.017319in}{0.029821in}}{\pgfqpoint{0.008840in}{0.033333in}}{\pgfqpoint{0.000000in}{0.033333in}}%
\pgfpathcurveto{\pgfqpoint{-0.008840in}{0.033333in}}{\pgfqpoint{-0.017319in}{0.029821in}}{\pgfqpoint{-0.023570in}{0.023570in}}%
\pgfpathcurveto{\pgfqpoint{-0.029821in}{0.017319in}}{\pgfqpoint{-0.033333in}{0.008840in}}{\pgfqpoint{-0.033333in}{0.000000in}}%
\pgfpathcurveto{\pgfqpoint{-0.033333in}{-0.008840in}}{\pgfqpoint{-0.029821in}{-0.017319in}}{\pgfqpoint{-0.023570in}{-0.023570in}}%
\pgfpathcurveto{\pgfqpoint{-0.017319in}{-0.029821in}}{\pgfqpoint{-0.008840in}{-0.033333in}}{\pgfqpoint{0.000000in}{-0.033333in}}%
\pgfpathlineto{\pgfqpoint{0.000000in}{-0.033333in}}%
\pgfpathclose%
\pgfusepath{stroke,fill}%
}%
\begin{pgfscope}%
\pgfsys@transformshift{2.707118in}{2.475054in}%
\pgfsys@useobject{currentmarker}{}%
\end{pgfscope}%
\end{pgfscope}%
\begin{pgfscope}%
\pgfpathrectangle{\pgfqpoint{0.765000in}{0.660000in}}{\pgfqpoint{4.620000in}{4.620000in}}%
\pgfusepath{clip}%
\pgfsetrectcap%
\pgfsetroundjoin%
\pgfsetlinewidth{1.204500pt}%
\definecolor{currentstroke}{rgb}{1.000000,0.576471,0.309804}%
\pgfsetstrokecolor{currentstroke}%
\pgfsetdash{}{0pt}%
\pgfpathmoveto{\pgfqpoint{3.857217in}{2.697247in}}%
\pgfusepath{stroke}%
\end{pgfscope}%
\begin{pgfscope}%
\pgfpathrectangle{\pgfqpoint{0.765000in}{0.660000in}}{\pgfqpoint{4.620000in}{4.620000in}}%
\pgfusepath{clip}%
\pgfsetbuttcap%
\pgfsetroundjoin%
\definecolor{currentfill}{rgb}{1.000000,0.576471,0.309804}%
\pgfsetfillcolor{currentfill}%
\pgfsetlinewidth{1.003750pt}%
\definecolor{currentstroke}{rgb}{1.000000,0.576471,0.309804}%
\pgfsetstrokecolor{currentstroke}%
\pgfsetdash{}{0pt}%
\pgfsys@defobject{currentmarker}{\pgfqpoint{-0.033333in}{-0.033333in}}{\pgfqpoint{0.033333in}{0.033333in}}{%
\pgfpathmoveto{\pgfqpoint{0.000000in}{-0.033333in}}%
\pgfpathcurveto{\pgfqpoint{0.008840in}{-0.033333in}}{\pgfqpoint{0.017319in}{-0.029821in}}{\pgfqpoint{0.023570in}{-0.023570in}}%
\pgfpathcurveto{\pgfqpoint{0.029821in}{-0.017319in}}{\pgfqpoint{0.033333in}{-0.008840in}}{\pgfqpoint{0.033333in}{0.000000in}}%
\pgfpathcurveto{\pgfqpoint{0.033333in}{0.008840in}}{\pgfqpoint{0.029821in}{0.017319in}}{\pgfqpoint{0.023570in}{0.023570in}}%
\pgfpathcurveto{\pgfqpoint{0.017319in}{0.029821in}}{\pgfqpoint{0.008840in}{0.033333in}}{\pgfqpoint{0.000000in}{0.033333in}}%
\pgfpathcurveto{\pgfqpoint{-0.008840in}{0.033333in}}{\pgfqpoint{-0.017319in}{0.029821in}}{\pgfqpoint{-0.023570in}{0.023570in}}%
\pgfpathcurveto{\pgfqpoint{-0.029821in}{0.017319in}}{\pgfqpoint{-0.033333in}{0.008840in}}{\pgfqpoint{-0.033333in}{0.000000in}}%
\pgfpathcurveto{\pgfqpoint{-0.033333in}{-0.008840in}}{\pgfqpoint{-0.029821in}{-0.017319in}}{\pgfqpoint{-0.023570in}{-0.023570in}}%
\pgfpathcurveto{\pgfqpoint{-0.017319in}{-0.029821in}}{\pgfqpoint{-0.008840in}{-0.033333in}}{\pgfqpoint{0.000000in}{-0.033333in}}%
\pgfpathlineto{\pgfqpoint{0.000000in}{-0.033333in}}%
\pgfpathclose%
\pgfusepath{stroke,fill}%
}%
\begin{pgfscope}%
\pgfsys@transformshift{3.857217in}{2.697247in}%
\pgfsys@useobject{currentmarker}{}%
\end{pgfscope}%
\end{pgfscope}%
\begin{pgfscope}%
\pgfpathrectangle{\pgfqpoint{0.765000in}{0.660000in}}{\pgfqpoint{4.620000in}{4.620000in}}%
\pgfusepath{clip}%
\pgfsetrectcap%
\pgfsetroundjoin%
\pgfsetlinewidth{1.204500pt}%
\definecolor{currentstroke}{rgb}{1.000000,0.576471,0.309804}%
\pgfsetstrokecolor{currentstroke}%
\pgfsetdash{}{0pt}%
\pgfpathmoveto{\pgfqpoint{3.358780in}{2.339945in}}%
\pgfusepath{stroke}%
\end{pgfscope}%
\begin{pgfscope}%
\pgfpathrectangle{\pgfqpoint{0.765000in}{0.660000in}}{\pgfqpoint{4.620000in}{4.620000in}}%
\pgfusepath{clip}%
\pgfsetbuttcap%
\pgfsetroundjoin%
\definecolor{currentfill}{rgb}{1.000000,0.576471,0.309804}%
\pgfsetfillcolor{currentfill}%
\pgfsetlinewidth{1.003750pt}%
\definecolor{currentstroke}{rgb}{1.000000,0.576471,0.309804}%
\pgfsetstrokecolor{currentstroke}%
\pgfsetdash{}{0pt}%
\pgfsys@defobject{currentmarker}{\pgfqpoint{-0.033333in}{-0.033333in}}{\pgfqpoint{0.033333in}{0.033333in}}{%
\pgfpathmoveto{\pgfqpoint{0.000000in}{-0.033333in}}%
\pgfpathcurveto{\pgfqpoint{0.008840in}{-0.033333in}}{\pgfqpoint{0.017319in}{-0.029821in}}{\pgfqpoint{0.023570in}{-0.023570in}}%
\pgfpathcurveto{\pgfqpoint{0.029821in}{-0.017319in}}{\pgfqpoint{0.033333in}{-0.008840in}}{\pgfqpoint{0.033333in}{0.000000in}}%
\pgfpathcurveto{\pgfqpoint{0.033333in}{0.008840in}}{\pgfqpoint{0.029821in}{0.017319in}}{\pgfqpoint{0.023570in}{0.023570in}}%
\pgfpathcurveto{\pgfqpoint{0.017319in}{0.029821in}}{\pgfqpoint{0.008840in}{0.033333in}}{\pgfqpoint{0.000000in}{0.033333in}}%
\pgfpathcurveto{\pgfqpoint{-0.008840in}{0.033333in}}{\pgfqpoint{-0.017319in}{0.029821in}}{\pgfqpoint{-0.023570in}{0.023570in}}%
\pgfpathcurveto{\pgfqpoint{-0.029821in}{0.017319in}}{\pgfqpoint{-0.033333in}{0.008840in}}{\pgfqpoint{-0.033333in}{0.000000in}}%
\pgfpathcurveto{\pgfqpoint{-0.033333in}{-0.008840in}}{\pgfqpoint{-0.029821in}{-0.017319in}}{\pgfqpoint{-0.023570in}{-0.023570in}}%
\pgfpathcurveto{\pgfqpoint{-0.017319in}{-0.029821in}}{\pgfqpoint{-0.008840in}{-0.033333in}}{\pgfqpoint{0.000000in}{-0.033333in}}%
\pgfpathlineto{\pgfqpoint{0.000000in}{-0.033333in}}%
\pgfpathclose%
\pgfusepath{stroke,fill}%
}%
\begin{pgfscope}%
\pgfsys@transformshift{3.358780in}{2.339945in}%
\pgfsys@useobject{currentmarker}{}%
\end{pgfscope}%
\end{pgfscope}%
\begin{pgfscope}%
\pgfpathrectangle{\pgfqpoint{0.765000in}{0.660000in}}{\pgfqpoint{4.620000in}{4.620000in}}%
\pgfusepath{clip}%
\pgfsetrectcap%
\pgfsetroundjoin%
\pgfsetlinewidth{1.204500pt}%
\definecolor{currentstroke}{rgb}{1.000000,0.576471,0.309804}%
\pgfsetstrokecolor{currentstroke}%
\pgfsetdash{}{0pt}%
\pgfpathmoveto{\pgfqpoint{2.731879in}{2.093345in}}%
\pgfusepath{stroke}%
\end{pgfscope}%
\begin{pgfscope}%
\pgfpathrectangle{\pgfqpoint{0.765000in}{0.660000in}}{\pgfqpoint{4.620000in}{4.620000in}}%
\pgfusepath{clip}%
\pgfsetbuttcap%
\pgfsetroundjoin%
\definecolor{currentfill}{rgb}{1.000000,0.576471,0.309804}%
\pgfsetfillcolor{currentfill}%
\pgfsetlinewidth{1.003750pt}%
\definecolor{currentstroke}{rgb}{1.000000,0.576471,0.309804}%
\pgfsetstrokecolor{currentstroke}%
\pgfsetdash{}{0pt}%
\pgfsys@defobject{currentmarker}{\pgfqpoint{-0.033333in}{-0.033333in}}{\pgfqpoint{0.033333in}{0.033333in}}{%
\pgfpathmoveto{\pgfqpoint{0.000000in}{-0.033333in}}%
\pgfpathcurveto{\pgfqpoint{0.008840in}{-0.033333in}}{\pgfqpoint{0.017319in}{-0.029821in}}{\pgfqpoint{0.023570in}{-0.023570in}}%
\pgfpathcurveto{\pgfqpoint{0.029821in}{-0.017319in}}{\pgfqpoint{0.033333in}{-0.008840in}}{\pgfqpoint{0.033333in}{0.000000in}}%
\pgfpathcurveto{\pgfqpoint{0.033333in}{0.008840in}}{\pgfqpoint{0.029821in}{0.017319in}}{\pgfqpoint{0.023570in}{0.023570in}}%
\pgfpathcurveto{\pgfqpoint{0.017319in}{0.029821in}}{\pgfqpoint{0.008840in}{0.033333in}}{\pgfqpoint{0.000000in}{0.033333in}}%
\pgfpathcurveto{\pgfqpoint{-0.008840in}{0.033333in}}{\pgfqpoint{-0.017319in}{0.029821in}}{\pgfqpoint{-0.023570in}{0.023570in}}%
\pgfpathcurveto{\pgfqpoint{-0.029821in}{0.017319in}}{\pgfqpoint{-0.033333in}{0.008840in}}{\pgfqpoint{-0.033333in}{0.000000in}}%
\pgfpathcurveto{\pgfqpoint{-0.033333in}{-0.008840in}}{\pgfqpoint{-0.029821in}{-0.017319in}}{\pgfqpoint{-0.023570in}{-0.023570in}}%
\pgfpathcurveto{\pgfqpoint{-0.017319in}{-0.029821in}}{\pgfqpoint{-0.008840in}{-0.033333in}}{\pgfqpoint{0.000000in}{-0.033333in}}%
\pgfpathlineto{\pgfqpoint{0.000000in}{-0.033333in}}%
\pgfpathclose%
\pgfusepath{stroke,fill}%
}%
\begin{pgfscope}%
\pgfsys@transformshift{2.731879in}{2.093345in}%
\pgfsys@useobject{currentmarker}{}%
\end{pgfscope}%
\end{pgfscope}%
\begin{pgfscope}%
\pgfpathrectangle{\pgfqpoint{0.765000in}{0.660000in}}{\pgfqpoint{4.620000in}{4.620000in}}%
\pgfusepath{clip}%
\pgfsetrectcap%
\pgfsetroundjoin%
\pgfsetlinewidth{1.204500pt}%
\definecolor{currentstroke}{rgb}{0.176471,0.192157,0.258824}%
\pgfsetstrokecolor{currentstroke}%
\pgfsetdash{}{0pt}%
\pgfpathmoveto{\pgfqpoint{2.376513in}{3.241847in}}%
\pgfusepath{stroke}%
\end{pgfscope}%
\begin{pgfscope}%
\pgfpathrectangle{\pgfqpoint{0.765000in}{0.660000in}}{\pgfqpoint{4.620000in}{4.620000in}}%
\pgfusepath{clip}%
\pgfsetbuttcap%
\pgfsetmiterjoin%
\definecolor{currentfill}{rgb}{0.176471,0.192157,0.258824}%
\pgfsetfillcolor{currentfill}%
\pgfsetlinewidth{1.003750pt}%
\definecolor{currentstroke}{rgb}{0.176471,0.192157,0.258824}%
\pgfsetstrokecolor{currentstroke}%
\pgfsetdash{}{0pt}%
\pgfsys@defobject{currentmarker}{\pgfqpoint{-0.033333in}{-0.033333in}}{\pgfqpoint{0.033333in}{0.033333in}}{%
\pgfpathmoveto{\pgfqpoint{-0.000000in}{-0.033333in}}%
\pgfpathlineto{\pgfqpoint{0.033333in}{0.033333in}}%
\pgfpathlineto{\pgfqpoint{-0.033333in}{0.033333in}}%
\pgfpathlineto{\pgfqpoint{-0.000000in}{-0.033333in}}%
\pgfpathclose%
\pgfusepath{stroke,fill}%
}%
\begin{pgfscope}%
\pgfsys@transformshift{2.376513in}{3.241847in}%
\pgfsys@useobject{currentmarker}{}%
\end{pgfscope}%
\end{pgfscope}%
\begin{pgfscope}%
\pgfpathrectangle{\pgfqpoint{0.765000in}{0.660000in}}{\pgfqpoint{4.620000in}{4.620000in}}%
\pgfusepath{clip}%
\pgfsetrectcap%
\pgfsetroundjoin%
\pgfsetlinewidth{1.204500pt}%
\definecolor{currentstroke}{rgb}{0.176471,0.192157,0.258824}%
\pgfsetstrokecolor{currentstroke}%
\pgfsetdash{}{0pt}%
\pgfpathmoveto{\pgfqpoint{3.598776in}{3.501804in}}%
\pgfusepath{stroke}%
\end{pgfscope}%
\begin{pgfscope}%
\pgfpathrectangle{\pgfqpoint{0.765000in}{0.660000in}}{\pgfqpoint{4.620000in}{4.620000in}}%
\pgfusepath{clip}%
\pgfsetbuttcap%
\pgfsetmiterjoin%
\definecolor{currentfill}{rgb}{0.176471,0.192157,0.258824}%
\pgfsetfillcolor{currentfill}%
\pgfsetlinewidth{1.003750pt}%
\definecolor{currentstroke}{rgb}{0.176471,0.192157,0.258824}%
\pgfsetstrokecolor{currentstroke}%
\pgfsetdash{}{0pt}%
\pgfsys@defobject{currentmarker}{\pgfqpoint{-0.033333in}{-0.033333in}}{\pgfqpoint{0.033333in}{0.033333in}}{%
\pgfpathmoveto{\pgfqpoint{-0.000000in}{-0.033333in}}%
\pgfpathlineto{\pgfqpoint{0.033333in}{0.033333in}}%
\pgfpathlineto{\pgfqpoint{-0.033333in}{0.033333in}}%
\pgfpathlineto{\pgfqpoint{-0.000000in}{-0.033333in}}%
\pgfpathclose%
\pgfusepath{stroke,fill}%
}%
\begin{pgfscope}%
\pgfsys@transformshift{3.598776in}{3.501804in}%
\pgfsys@useobject{currentmarker}{}%
\end{pgfscope}%
\end{pgfscope}%
\begin{pgfscope}%
\pgfpathrectangle{\pgfqpoint{0.765000in}{0.660000in}}{\pgfqpoint{4.620000in}{4.620000in}}%
\pgfusepath{clip}%
\pgfsetrectcap%
\pgfsetroundjoin%
\pgfsetlinewidth{1.204500pt}%
\definecolor{currentstroke}{rgb}{0.176471,0.192157,0.258824}%
\pgfsetstrokecolor{currentstroke}%
\pgfsetdash{}{0pt}%
\pgfpathmoveto{\pgfqpoint{2.083911in}{3.716298in}}%
\pgfusepath{stroke}%
\end{pgfscope}%
\begin{pgfscope}%
\pgfpathrectangle{\pgfqpoint{0.765000in}{0.660000in}}{\pgfqpoint{4.620000in}{4.620000in}}%
\pgfusepath{clip}%
\pgfsetbuttcap%
\pgfsetmiterjoin%
\definecolor{currentfill}{rgb}{0.176471,0.192157,0.258824}%
\pgfsetfillcolor{currentfill}%
\pgfsetlinewidth{1.003750pt}%
\definecolor{currentstroke}{rgb}{0.176471,0.192157,0.258824}%
\pgfsetstrokecolor{currentstroke}%
\pgfsetdash{}{0pt}%
\pgfsys@defobject{currentmarker}{\pgfqpoint{-0.033333in}{-0.033333in}}{\pgfqpoint{0.033333in}{0.033333in}}{%
\pgfpathmoveto{\pgfqpoint{-0.000000in}{-0.033333in}}%
\pgfpathlineto{\pgfqpoint{0.033333in}{0.033333in}}%
\pgfpathlineto{\pgfqpoint{-0.033333in}{0.033333in}}%
\pgfpathlineto{\pgfqpoint{-0.000000in}{-0.033333in}}%
\pgfpathclose%
\pgfusepath{stroke,fill}%
}%
\begin{pgfscope}%
\pgfsys@transformshift{2.083911in}{3.716298in}%
\pgfsys@useobject{currentmarker}{}%
\end{pgfscope}%
\end{pgfscope}%
\begin{pgfscope}%
\pgfpathrectangle{\pgfqpoint{0.765000in}{0.660000in}}{\pgfqpoint{4.620000in}{4.620000in}}%
\pgfusepath{clip}%
\pgfsetrectcap%
\pgfsetroundjoin%
\pgfsetlinewidth{1.204500pt}%
\definecolor{currentstroke}{rgb}{0.176471,0.192157,0.258824}%
\pgfsetstrokecolor{currentstroke}%
\pgfsetdash{}{0pt}%
\pgfpathmoveto{\pgfqpoint{2.034318in}{3.482362in}}%
\pgfusepath{stroke}%
\end{pgfscope}%
\begin{pgfscope}%
\pgfpathrectangle{\pgfqpoint{0.765000in}{0.660000in}}{\pgfqpoint{4.620000in}{4.620000in}}%
\pgfusepath{clip}%
\pgfsetbuttcap%
\pgfsetmiterjoin%
\definecolor{currentfill}{rgb}{0.176471,0.192157,0.258824}%
\pgfsetfillcolor{currentfill}%
\pgfsetlinewidth{1.003750pt}%
\definecolor{currentstroke}{rgb}{0.176471,0.192157,0.258824}%
\pgfsetstrokecolor{currentstroke}%
\pgfsetdash{}{0pt}%
\pgfsys@defobject{currentmarker}{\pgfqpoint{-0.033333in}{-0.033333in}}{\pgfqpoint{0.033333in}{0.033333in}}{%
\pgfpathmoveto{\pgfqpoint{-0.000000in}{-0.033333in}}%
\pgfpathlineto{\pgfqpoint{0.033333in}{0.033333in}}%
\pgfpathlineto{\pgfqpoint{-0.033333in}{0.033333in}}%
\pgfpathlineto{\pgfqpoint{-0.000000in}{-0.033333in}}%
\pgfpathclose%
\pgfusepath{stroke,fill}%
}%
\begin{pgfscope}%
\pgfsys@transformshift{2.034318in}{3.482362in}%
\pgfsys@useobject{currentmarker}{}%
\end{pgfscope}%
\end{pgfscope}%
\begin{pgfscope}%
\pgfpathrectangle{\pgfqpoint{0.765000in}{0.660000in}}{\pgfqpoint{4.620000in}{4.620000in}}%
\pgfusepath{clip}%
\pgfsetrectcap%
\pgfsetroundjoin%
\pgfsetlinewidth{1.204500pt}%
\definecolor{currentstroke}{rgb}{0.176471,0.192157,0.258824}%
\pgfsetstrokecolor{currentstroke}%
\pgfsetdash{}{0pt}%
\pgfpathmoveto{\pgfqpoint{2.888262in}{3.592853in}}%
\pgfusepath{stroke}%
\end{pgfscope}%
\begin{pgfscope}%
\pgfpathrectangle{\pgfqpoint{0.765000in}{0.660000in}}{\pgfqpoint{4.620000in}{4.620000in}}%
\pgfusepath{clip}%
\pgfsetbuttcap%
\pgfsetmiterjoin%
\definecolor{currentfill}{rgb}{0.176471,0.192157,0.258824}%
\pgfsetfillcolor{currentfill}%
\pgfsetlinewidth{1.003750pt}%
\definecolor{currentstroke}{rgb}{0.176471,0.192157,0.258824}%
\pgfsetstrokecolor{currentstroke}%
\pgfsetdash{}{0pt}%
\pgfsys@defobject{currentmarker}{\pgfqpoint{-0.033333in}{-0.033333in}}{\pgfqpoint{0.033333in}{0.033333in}}{%
\pgfpathmoveto{\pgfqpoint{-0.000000in}{-0.033333in}}%
\pgfpathlineto{\pgfqpoint{0.033333in}{0.033333in}}%
\pgfpathlineto{\pgfqpoint{-0.033333in}{0.033333in}}%
\pgfpathlineto{\pgfqpoint{-0.000000in}{-0.033333in}}%
\pgfpathclose%
\pgfusepath{stroke,fill}%
}%
\begin{pgfscope}%
\pgfsys@transformshift{2.888262in}{3.592853in}%
\pgfsys@useobject{currentmarker}{}%
\end{pgfscope}%
\end{pgfscope}%
\begin{pgfscope}%
\pgfpathrectangle{\pgfqpoint{0.765000in}{0.660000in}}{\pgfqpoint{4.620000in}{4.620000in}}%
\pgfusepath{clip}%
\pgfsetrectcap%
\pgfsetroundjoin%
\pgfsetlinewidth{1.204500pt}%
\definecolor{currentstroke}{rgb}{0.176471,0.192157,0.258824}%
\pgfsetstrokecolor{currentstroke}%
\pgfsetdash{}{0pt}%
\pgfpathmoveto{\pgfqpoint{2.213511in}{3.603999in}}%
\pgfusepath{stroke}%
\end{pgfscope}%
\begin{pgfscope}%
\pgfpathrectangle{\pgfqpoint{0.765000in}{0.660000in}}{\pgfqpoint{4.620000in}{4.620000in}}%
\pgfusepath{clip}%
\pgfsetbuttcap%
\pgfsetmiterjoin%
\definecolor{currentfill}{rgb}{0.176471,0.192157,0.258824}%
\pgfsetfillcolor{currentfill}%
\pgfsetlinewidth{1.003750pt}%
\definecolor{currentstroke}{rgb}{0.176471,0.192157,0.258824}%
\pgfsetstrokecolor{currentstroke}%
\pgfsetdash{}{0pt}%
\pgfsys@defobject{currentmarker}{\pgfqpoint{-0.033333in}{-0.033333in}}{\pgfqpoint{0.033333in}{0.033333in}}{%
\pgfpathmoveto{\pgfqpoint{-0.000000in}{-0.033333in}}%
\pgfpathlineto{\pgfqpoint{0.033333in}{0.033333in}}%
\pgfpathlineto{\pgfqpoint{-0.033333in}{0.033333in}}%
\pgfpathlineto{\pgfqpoint{-0.000000in}{-0.033333in}}%
\pgfpathclose%
\pgfusepath{stroke,fill}%
}%
\begin{pgfscope}%
\pgfsys@transformshift{2.213511in}{3.603999in}%
\pgfsys@useobject{currentmarker}{}%
\end{pgfscope}%
\end{pgfscope}%
\begin{pgfscope}%
\pgfpathrectangle{\pgfqpoint{0.765000in}{0.660000in}}{\pgfqpoint{4.620000in}{4.620000in}}%
\pgfusepath{clip}%
\pgfsetrectcap%
\pgfsetroundjoin%
\pgfsetlinewidth{1.204500pt}%
\definecolor{currentstroke}{rgb}{0.176471,0.192157,0.258824}%
\pgfsetstrokecolor{currentstroke}%
\pgfsetdash{}{0pt}%
\pgfpathmoveto{\pgfqpoint{1.862606in}{3.364769in}}%
\pgfusepath{stroke}%
\end{pgfscope}%
\begin{pgfscope}%
\pgfpathrectangle{\pgfqpoint{0.765000in}{0.660000in}}{\pgfqpoint{4.620000in}{4.620000in}}%
\pgfusepath{clip}%
\pgfsetbuttcap%
\pgfsetmiterjoin%
\definecolor{currentfill}{rgb}{0.176471,0.192157,0.258824}%
\pgfsetfillcolor{currentfill}%
\pgfsetlinewidth{1.003750pt}%
\definecolor{currentstroke}{rgb}{0.176471,0.192157,0.258824}%
\pgfsetstrokecolor{currentstroke}%
\pgfsetdash{}{0pt}%
\pgfsys@defobject{currentmarker}{\pgfqpoint{-0.033333in}{-0.033333in}}{\pgfqpoint{0.033333in}{0.033333in}}{%
\pgfpathmoveto{\pgfqpoint{-0.000000in}{-0.033333in}}%
\pgfpathlineto{\pgfqpoint{0.033333in}{0.033333in}}%
\pgfpathlineto{\pgfqpoint{-0.033333in}{0.033333in}}%
\pgfpathlineto{\pgfqpoint{-0.000000in}{-0.033333in}}%
\pgfpathclose%
\pgfusepath{stroke,fill}%
}%
\begin{pgfscope}%
\pgfsys@transformshift{1.862606in}{3.364769in}%
\pgfsys@useobject{currentmarker}{}%
\end{pgfscope}%
\end{pgfscope}%
\begin{pgfscope}%
\pgfpathrectangle{\pgfqpoint{0.765000in}{0.660000in}}{\pgfqpoint{4.620000in}{4.620000in}}%
\pgfusepath{clip}%
\pgfsetrectcap%
\pgfsetroundjoin%
\pgfsetlinewidth{1.204500pt}%
\definecolor{currentstroke}{rgb}{0.176471,0.192157,0.258824}%
\pgfsetstrokecolor{currentstroke}%
\pgfsetdash{}{0pt}%
\pgfpathmoveto{\pgfqpoint{3.711268in}{3.488344in}}%
\pgfusepath{stroke}%
\end{pgfscope}%
\begin{pgfscope}%
\pgfpathrectangle{\pgfqpoint{0.765000in}{0.660000in}}{\pgfqpoint{4.620000in}{4.620000in}}%
\pgfusepath{clip}%
\pgfsetbuttcap%
\pgfsetmiterjoin%
\definecolor{currentfill}{rgb}{0.176471,0.192157,0.258824}%
\pgfsetfillcolor{currentfill}%
\pgfsetlinewidth{1.003750pt}%
\definecolor{currentstroke}{rgb}{0.176471,0.192157,0.258824}%
\pgfsetstrokecolor{currentstroke}%
\pgfsetdash{}{0pt}%
\pgfsys@defobject{currentmarker}{\pgfqpoint{-0.033333in}{-0.033333in}}{\pgfqpoint{0.033333in}{0.033333in}}{%
\pgfpathmoveto{\pgfqpoint{-0.000000in}{-0.033333in}}%
\pgfpathlineto{\pgfqpoint{0.033333in}{0.033333in}}%
\pgfpathlineto{\pgfqpoint{-0.033333in}{0.033333in}}%
\pgfpathlineto{\pgfqpoint{-0.000000in}{-0.033333in}}%
\pgfpathclose%
\pgfusepath{stroke,fill}%
}%
\begin{pgfscope}%
\pgfsys@transformshift{3.711268in}{3.488344in}%
\pgfsys@useobject{currentmarker}{}%
\end{pgfscope}%
\end{pgfscope}%
\begin{pgfscope}%
\pgfpathrectangle{\pgfqpoint{0.765000in}{0.660000in}}{\pgfqpoint{4.620000in}{4.620000in}}%
\pgfusepath{clip}%
\pgfsetrectcap%
\pgfsetroundjoin%
\pgfsetlinewidth{1.204500pt}%
\definecolor{currentstroke}{rgb}{0.176471,0.192157,0.258824}%
\pgfsetstrokecolor{currentstroke}%
\pgfsetdash{}{0pt}%
\pgfpathmoveto{\pgfqpoint{3.393700in}{2.984977in}}%
\pgfusepath{stroke}%
\end{pgfscope}%
\begin{pgfscope}%
\pgfpathrectangle{\pgfqpoint{0.765000in}{0.660000in}}{\pgfqpoint{4.620000in}{4.620000in}}%
\pgfusepath{clip}%
\pgfsetbuttcap%
\pgfsetmiterjoin%
\definecolor{currentfill}{rgb}{0.176471,0.192157,0.258824}%
\pgfsetfillcolor{currentfill}%
\pgfsetlinewidth{1.003750pt}%
\definecolor{currentstroke}{rgb}{0.176471,0.192157,0.258824}%
\pgfsetstrokecolor{currentstroke}%
\pgfsetdash{}{0pt}%
\pgfsys@defobject{currentmarker}{\pgfqpoint{-0.033333in}{-0.033333in}}{\pgfqpoint{0.033333in}{0.033333in}}{%
\pgfpathmoveto{\pgfqpoint{-0.000000in}{-0.033333in}}%
\pgfpathlineto{\pgfqpoint{0.033333in}{0.033333in}}%
\pgfpathlineto{\pgfqpoint{-0.033333in}{0.033333in}}%
\pgfpathlineto{\pgfqpoint{-0.000000in}{-0.033333in}}%
\pgfpathclose%
\pgfusepath{stroke,fill}%
}%
\begin{pgfscope}%
\pgfsys@transformshift{3.393700in}{2.984977in}%
\pgfsys@useobject{currentmarker}{}%
\end{pgfscope}%
\end{pgfscope}%
\begin{pgfscope}%
\pgfpathrectangle{\pgfqpoint{0.765000in}{0.660000in}}{\pgfqpoint{4.620000in}{4.620000in}}%
\pgfusepath{clip}%
\pgfsetrectcap%
\pgfsetroundjoin%
\pgfsetlinewidth{1.204500pt}%
\definecolor{currentstroke}{rgb}{0.176471,0.192157,0.258824}%
\pgfsetstrokecolor{currentstroke}%
\pgfsetdash{}{0pt}%
\pgfpathmoveto{\pgfqpoint{3.388538in}{3.505737in}}%
\pgfusepath{stroke}%
\end{pgfscope}%
\begin{pgfscope}%
\pgfpathrectangle{\pgfqpoint{0.765000in}{0.660000in}}{\pgfqpoint{4.620000in}{4.620000in}}%
\pgfusepath{clip}%
\pgfsetbuttcap%
\pgfsetmiterjoin%
\definecolor{currentfill}{rgb}{0.176471,0.192157,0.258824}%
\pgfsetfillcolor{currentfill}%
\pgfsetlinewidth{1.003750pt}%
\definecolor{currentstroke}{rgb}{0.176471,0.192157,0.258824}%
\pgfsetstrokecolor{currentstroke}%
\pgfsetdash{}{0pt}%
\pgfsys@defobject{currentmarker}{\pgfqpoint{-0.033333in}{-0.033333in}}{\pgfqpoint{0.033333in}{0.033333in}}{%
\pgfpathmoveto{\pgfqpoint{-0.000000in}{-0.033333in}}%
\pgfpathlineto{\pgfqpoint{0.033333in}{0.033333in}}%
\pgfpathlineto{\pgfqpoint{-0.033333in}{0.033333in}}%
\pgfpathlineto{\pgfqpoint{-0.000000in}{-0.033333in}}%
\pgfpathclose%
\pgfusepath{stroke,fill}%
}%
\begin{pgfscope}%
\pgfsys@transformshift{3.388538in}{3.505737in}%
\pgfsys@useobject{currentmarker}{}%
\end{pgfscope}%
\end{pgfscope}%
\begin{pgfscope}%
\pgfpathrectangle{\pgfqpoint{0.765000in}{0.660000in}}{\pgfqpoint{4.620000in}{4.620000in}}%
\pgfusepath{clip}%
\pgfsetrectcap%
\pgfsetroundjoin%
\pgfsetlinewidth{1.204500pt}%
\definecolor{currentstroke}{rgb}{0.176471,0.192157,0.258824}%
\pgfsetstrokecolor{currentstroke}%
\pgfsetdash{}{0pt}%
\pgfpathmoveto{\pgfqpoint{2.176875in}{3.446714in}}%
\pgfusepath{stroke}%
\end{pgfscope}%
\begin{pgfscope}%
\pgfpathrectangle{\pgfqpoint{0.765000in}{0.660000in}}{\pgfqpoint{4.620000in}{4.620000in}}%
\pgfusepath{clip}%
\pgfsetbuttcap%
\pgfsetmiterjoin%
\definecolor{currentfill}{rgb}{0.176471,0.192157,0.258824}%
\pgfsetfillcolor{currentfill}%
\pgfsetlinewidth{1.003750pt}%
\definecolor{currentstroke}{rgb}{0.176471,0.192157,0.258824}%
\pgfsetstrokecolor{currentstroke}%
\pgfsetdash{}{0pt}%
\pgfsys@defobject{currentmarker}{\pgfqpoint{-0.033333in}{-0.033333in}}{\pgfqpoint{0.033333in}{0.033333in}}{%
\pgfpathmoveto{\pgfqpoint{-0.000000in}{-0.033333in}}%
\pgfpathlineto{\pgfqpoint{0.033333in}{0.033333in}}%
\pgfpathlineto{\pgfqpoint{-0.033333in}{0.033333in}}%
\pgfpathlineto{\pgfqpoint{-0.000000in}{-0.033333in}}%
\pgfpathclose%
\pgfusepath{stroke,fill}%
}%
\begin{pgfscope}%
\pgfsys@transformshift{2.176875in}{3.446714in}%
\pgfsys@useobject{currentmarker}{}%
\end{pgfscope}%
\end{pgfscope}%
\begin{pgfscope}%
\pgfpathrectangle{\pgfqpoint{0.765000in}{0.660000in}}{\pgfqpoint{4.620000in}{4.620000in}}%
\pgfusepath{clip}%
\pgfsetrectcap%
\pgfsetroundjoin%
\pgfsetlinewidth{1.204500pt}%
\definecolor{currentstroke}{rgb}{0.176471,0.192157,0.258824}%
\pgfsetstrokecolor{currentstroke}%
\pgfsetdash{}{0pt}%
\pgfpathmoveto{\pgfqpoint{2.887136in}{3.564853in}}%
\pgfusepath{stroke}%
\end{pgfscope}%
\begin{pgfscope}%
\pgfpathrectangle{\pgfqpoint{0.765000in}{0.660000in}}{\pgfqpoint{4.620000in}{4.620000in}}%
\pgfusepath{clip}%
\pgfsetbuttcap%
\pgfsetmiterjoin%
\definecolor{currentfill}{rgb}{0.176471,0.192157,0.258824}%
\pgfsetfillcolor{currentfill}%
\pgfsetlinewidth{1.003750pt}%
\definecolor{currentstroke}{rgb}{0.176471,0.192157,0.258824}%
\pgfsetstrokecolor{currentstroke}%
\pgfsetdash{}{0pt}%
\pgfsys@defobject{currentmarker}{\pgfqpoint{-0.033333in}{-0.033333in}}{\pgfqpoint{0.033333in}{0.033333in}}{%
\pgfpathmoveto{\pgfqpoint{-0.000000in}{-0.033333in}}%
\pgfpathlineto{\pgfqpoint{0.033333in}{0.033333in}}%
\pgfpathlineto{\pgfqpoint{-0.033333in}{0.033333in}}%
\pgfpathlineto{\pgfqpoint{-0.000000in}{-0.033333in}}%
\pgfpathclose%
\pgfusepath{stroke,fill}%
}%
\begin{pgfscope}%
\pgfsys@transformshift{2.887136in}{3.564853in}%
\pgfsys@useobject{currentmarker}{}%
\end{pgfscope}%
\end{pgfscope}%
\begin{pgfscope}%
\pgfpathrectangle{\pgfqpoint{0.765000in}{0.660000in}}{\pgfqpoint{4.620000in}{4.620000in}}%
\pgfusepath{clip}%
\pgfsetrectcap%
\pgfsetroundjoin%
\pgfsetlinewidth{1.204500pt}%
\definecolor{currentstroke}{rgb}{0.176471,0.192157,0.258824}%
\pgfsetstrokecolor{currentstroke}%
\pgfsetdash{}{0pt}%
\pgfpathmoveto{\pgfqpoint{1.026858in}{2.886571in}}%
\pgfusepath{stroke}%
\end{pgfscope}%
\begin{pgfscope}%
\pgfpathrectangle{\pgfqpoint{0.765000in}{0.660000in}}{\pgfqpoint{4.620000in}{4.620000in}}%
\pgfusepath{clip}%
\pgfsetbuttcap%
\pgfsetmiterjoin%
\definecolor{currentfill}{rgb}{0.176471,0.192157,0.258824}%
\pgfsetfillcolor{currentfill}%
\pgfsetlinewidth{1.003750pt}%
\definecolor{currentstroke}{rgb}{0.176471,0.192157,0.258824}%
\pgfsetstrokecolor{currentstroke}%
\pgfsetdash{}{0pt}%
\pgfsys@defobject{currentmarker}{\pgfqpoint{-0.033333in}{-0.033333in}}{\pgfqpoint{0.033333in}{0.033333in}}{%
\pgfpathmoveto{\pgfqpoint{-0.000000in}{-0.033333in}}%
\pgfpathlineto{\pgfqpoint{0.033333in}{0.033333in}}%
\pgfpathlineto{\pgfqpoint{-0.033333in}{0.033333in}}%
\pgfpathlineto{\pgfqpoint{-0.000000in}{-0.033333in}}%
\pgfpathclose%
\pgfusepath{stroke,fill}%
}%
\begin{pgfscope}%
\pgfsys@transformshift{1.026858in}{2.886571in}%
\pgfsys@useobject{currentmarker}{}%
\end{pgfscope}%
\end{pgfscope}%
\begin{pgfscope}%
\pgfpathrectangle{\pgfqpoint{0.765000in}{0.660000in}}{\pgfqpoint{4.620000in}{4.620000in}}%
\pgfusepath{clip}%
\pgfsetrectcap%
\pgfsetroundjoin%
\pgfsetlinewidth{1.204500pt}%
\definecolor{currentstroke}{rgb}{0.176471,0.192157,0.258824}%
\pgfsetstrokecolor{currentstroke}%
\pgfsetdash{}{0pt}%
\pgfpathmoveto{\pgfqpoint{2.044090in}{3.571430in}}%
\pgfusepath{stroke}%
\end{pgfscope}%
\begin{pgfscope}%
\pgfpathrectangle{\pgfqpoint{0.765000in}{0.660000in}}{\pgfqpoint{4.620000in}{4.620000in}}%
\pgfusepath{clip}%
\pgfsetbuttcap%
\pgfsetmiterjoin%
\definecolor{currentfill}{rgb}{0.176471,0.192157,0.258824}%
\pgfsetfillcolor{currentfill}%
\pgfsetlinewidth{1.003750pt}%
\definecolor{currentstroke}{rgb}{0.176471,0.192157,0.258824}%
\pgfsetstrokecolor{currentstroke}%
\pgfsetdash{}{0pt}%
\pgfsys@defobject{currentmarker}{\pgfqpoint{-0.033333in}{-0.033333in}}{\pgfqpoint{0.033333in}{0.033333in}}{%
\pgfpathmoveto{\pgfqpoint{-0.000000in}{-0.033333in}}%
\pgfpathlineto{\pgfqpoint{0.033333in}{0.033333in}}%
\pgfpathlineto{\pgfqpoint{-0.033333in}{0.033333in}}%
\pgfpathlineto{\pgfqpoint{-0.000000in}{-0.033333in}}%
\pgfpathclose%
\pgfusepath{stroke,fill}%
}%
\begin{pgfscope}%
\pgfsys@transformshift{2.044090in}{3.571430in}%
\pgfsys@useobject{currentmarker}{}%
\end{pgfscope}%
\end{pgfscope}%
\begin{pgfscope}%
\pgfpathrectangle{\pgfqpoint{0.765000in}{0.660000in}}{\pgfqpoint{4.620000in}{4.620000in}}%
\pgfusepath{clip}%
\pgfsetrectcap%
\pgfsetroundjoin%
\pgfsetlinewidth{1.204500pt}%
\definecolor{currentstroke}{rgb}{0.176471,0.192157,0.258824}%
\pgfsetstrokecolor{currentstroke}%
\pgfsetdash{}{0pt}%
\pgfpathmoveto{\pgfqpoint{1.922885in}{3.304537in}}%
\pgfusepath{stroke}%
\end{pgfscope}%
\begin{pgfscope}%
\pgfpathrectangle{\pgfqpoint{0.765000in}{0.660000in}}{\pgfqpoint{4.620000in}{4.620000in}}%
\pgfusepath{clip}%
\pgfsetbuttcap%
\pgfsetmiterjoin%
\definecolor{currentfill}{rgb}{0.176471,0.192157,0.258824}%
\pgfsetfillcolor{currentfill}%
\pgfsetlinewidth{1.003750pt}%
\definecolor{currentstroke}{rgb}{0.176471,0.192157,0.258824}%
\pgfsetstrokecolor{currentstroke}%
\pgfsetdash{}{0pt}%
\pgfsys@defobject{currentmarker}{\pgfqpoint{-0.033333in}{-0.033333in}}{\pgfqpoint{0.033333in}{0.033333in}}{%
\pgfpathmoveto{\pgfqpoint{-0.000000in}{-0.033333in}}%
\pgfpathlineto{\pgfqpoint{0.033333in}{0.033333in}}%
\pgfpathlineto{\pgfqpoint{-0.033333in}{0.033333in}}%
\pgfpathlineto{\pgfqpoint{-0.000000in}{-0.033333in}}%
\pgfpathclose%
\pgfusepath{stroke,fill}%
}%
\begin{pgfscope}%
\pgfsys@transformshift{1.922885in}{3.304537in}%
\pgfsys@useobject{currentmarker}{}%
\end{pgfscope}%
\end{pgfscope}%
\begin{pgfscope}%
\pgfpathrectangle{\pgfqpoint{0.765000in}{0.660000in}}{\pgfqpoint{4.620000in}{4.620000in}}%
\pgfusepath{clip}%
\pgfsetrectcap%
\pgfsetroundjoin%
\pgfsetlinewidth{1.204500pt}%
\definecolor{currentstroke}{rgb}{0.176471,0.192157,0.258824}%
\pgfsetstrokecolor{currentstroke}%
\pgfsetdash{}{0pt}%
\pgfpathmoveto{\pgfqpoint{1.310540in}{2.853537in}}%
\pgfusepath{stroke}%
\end{pgfscope}%
\begin{pgfscope}%
\pgfpathrectangle{\pgfqpoint{0.765000in}{0.660000in}}{\pgfqpoint{4.620000in}{4.620000in}}%
\pgfusepath{clip}%
\pgfsetbuttcap%
\pgfsetmiterjoin%
\definecolor{currentfill}{rgb}{0.176471,0.192157,0.258824}%
\pgfsetfillcolor{currentfill}%
\pgfsetlinewidth{1.003750pt}%
\definecolor{currentstroke}{rgb}{0.176471,0.192157,0.258824}%
\pgfsetstrokecolor{currentstroke}%
\pgfsetdash{}{0pt}%
\pgfsys@defobject{currentmarker}{\pgfqpoint{-0.033333in}{-0.033333in}}{\pgfqpoint{0.033333in}{0.033333in}}{%
\pgfpathmoveto{\pgfqpoint{-0.000000in}{-0.033333in}}%
\pgfpathlineto{\pgfqpoint{0.033333in}{0.033333in}}%
\pgfpathlineto{\pgfqpoint{-0.033333in}{0.033333in}}%
\pgfpathlineto{\pgfqpoint{-0.000000in}{-0.033333in}}%
\pgfpathclose%
\pgfusepath{stroke,fill}%
}%
\begin{pgfscope}%
\pgfsys@transformshift{1.310540in}{2.853537in}%
\pgfsys@useobject{currentmarker}{}%
\end{pgfscope}%
\end{pgfscope}%
\begin{pgfscope}%
\pgfpathrectangle{\pgfqpoint{0.765000in}{0.660000in}}{\pgfqpoint{4.620000in}{4.620000in}}%
\pgfusepath{clip}%
\pgfsetrectcap%
\pgfsetroundjoin%
\pgfsetlinewidth{1.204500pt}%
\definecolor{currentstroke}{rgb}{0.176471,0.192157,0.258824}%
\pgfsetstrokecolor{currentstroke}%
\pgfsetdash{}{0pt}%
\pgfpathmoveto{\pgfqpoint{3.116587in}{3.684682in}}%
\pgfusepath{stroke}%
\end{pgfscope}%
\begin{pgfscope}%
\pgfpathrectangle{\pgfqpoint{0.765000in}{0.660000in}}{\pgfqpoint{4.620000in}{4.620000in}}%
\pgfusepath{clip}%
\pgfsetbuttcap%
\pgfsetmiterjoin%
\definecolor{currentfill}{rgb}{0.176471,0.192157,0.258824}%
\pgfsetfillcolor{currentfill}%
\pgfsetlinewidth{1.003750pt}%
\definecolor{currentstroke}{rgb}{0.176471,0.192157,0.258824}%
\pgfsetstrokecolor{currentstroke}%
\pgfsetdash{}{0pt}%
\pgfsys@defobject{currentmarker}{\pgfqpoint{-0.033333in}{-0.033333in}}{\pgfqpoint{0.033333in}{0.033333in}}{%
\pgfpathmoveto{\pgfqpoint{-0.000000in}{-0.033333in}}%
\pgfpathlineto{\pgfqpoint{0.033333in}{0.033333in}}%
\pgfpathlineto{\pgfqpoint{-0.033333in}{0.033333in}}%
\pgfpathlineto{\pgfqpoint{-0.000000in}{-0.033333in}}%
\pgfpathclose%
\pgfusepath{stroke,fill}%
}%
\begin{pgfscope}%
\pgfsys@transformshift{3.116587in}{3.684682in}%
\pgfsys@useobject{currentmarker}{}%
\end{pgfscope}%
\end{pgfscope}%
\begin{pgfscope}%
\pgfpathrectangle{\pgfqpoint{0.765000in}{0.660000in}}{\pgfqpoint{4.620000in}{4.620000in}}%
\pgfusepath{clip}%
\pgfsetrectcap%
\pgfsetroundjoin%
\pgfsetlinewidth{1.204500pt}%
\definecolor{currentstroke}{rgb}{0.176471,0.192157,0.258824}%
\pgfsetstrokecolor{currentstroke}%
\pgfsetdash{}{0pt}%
\pgfpathmoveto{\pgfqpoint{2.310226in}{3.734334in}}%
\pgfusepath{stroke}%
\end{pgfscope}%
\begin{pgfscope}%
\pgfpathrectangle{\pgfqpoint{0.765000in}{0.660000in}}{\pgfqpoint{4.620000in}{4.620000in}}%
\pgfusepath{clip}%
\pgfsetbuttcap%
\pgfsetmiterjoin%
\definecolor{currentfill}{rgb}{0.176471,0.192157,0.258824}%
\pgfsetfillcolor{currentfill}%
\pgfsetlinewidth{1.003750pt}%
\definecolor{currentstroke}{rgb}{0.176471,0.192157,0.258824}%
\pgfsetstrokecolor{currentstroke}%
\pgfsetdash{}{0pt}%
\pgfsys@defobject{currentmarker}{\pgfqpoint{-0.033333in}{-0.033333in}}{\pgfqpoint{0.033333in}{0.033333in}}{%
\pgfpathmoveto{\pgfqpoint{-0.000000in}{-0.033333in}}%
\pgfpathlineto{\pgfqpoint{0.033333in}{0.033333in}}%
\pgfpathlineto{\pgfqpoint{-0.033333in}{0.033333in}}%
\pgfpathlineto{\pgfqpoint{-0.000000in}{-0.033333in}}%
\pgfpathclose%
\pgfusepath{stroke,fill}%
}%
\begin{pgfscope}%
\pgfsys@transformshift{2.310226in}{3.734334in}%
\pgfsys@useobject{currentmarker}{}%
\end{pgfscope}%
\end{pgfscope}%
\begin{pgfscope}%
\pgfpathrectangle{\pgfqpoint{0.765000in}{0.660000in}}{\pgfqpoint{4.620000in}{4.620000in}}%
\pgfusepath{clip}%
\pgfsetrectcap%
\pgfsetroundjoin%
\pgfsetlinewidth{1.204500pt}%
\definecolor{currentstroke}{rgb}{0.176471,0.192157,0.258824}%
\pgfsetstrokecolor{currentstroke}%
\pgfsetdash{}{0pt}%
\pgfpathmoveto{\pgfqpoint{3.799586in}{3.182808in}}%
\pgfusepath{stroke}%
\end{pgfscope}%
\begin{pgfscope}%
\pgfpathrectangle{\pgfqpoint{0.765000in}{0.660000in}}{\pgfqpoint{4.620000in}{4.620000in}}%
\pgfusepath{clip}%
\pgfsetbuttcap%
\pgfsetmiterjoin%
\definecolor{currentfill}{rgb}{0.176471,0.192157,0.258824}%
\pgfsetfillcolor{currentfill}%
\pgfsetlinewidth{1.003750pt}%
\definecolor{currentstroke}{rgb}{0.176471,0.192157,0.258824}%
\pgfsetstrokecolor{currentstroke}%
\pgfsetdash{}{0pt}%
\pgfsys@defobject{currentmarker}{\pgfqpoint{-0.033333in}{-0.033333in}}{\pgfqpoint{0.033333in}{0.033333in}}{%
\pgfpathmoveto{\pgfqpoint{-0.000000in}{-0.033333in}}%
\pgfpathlineto{\pgfqpoint{0.033333in}{0.033333in}}%
\pgfpathlineto{\pgfqpoint{-0.033333in}{0.033333in}}%
\pgfpathlineto{\pgfqpoint{-0.000000in}{-0.033333in}}%
\pgfpathclose%
\pgfusepath{stroke,fill}%
}%
\begin{pgfscope}%
\pgfsys@transformshift{3.799586in}{3.182808in}%
\pgfsys@useobject{currentmarker}{}%
\end{pgfscope}%
\end{pgfscope}%
\begin{pgfscope}%
\pgfpathrectangle{\pgfqpoint{0.765000in}{0.660000in}}{\pgfqpoint{4.620000in}{4.620000in}}%
\pgfusepath{clip}%
\pgfsetrectcap%
\pgfsetroundjoin%
\pgfsetlinewidth{1.204500pt}%
\definecolor{currentstroke}{rgb}{0.176471,0.192157,0.258824}%
\pgfsetstrokecolor{currentstroke}%
\pgfsetdash{}{0pt}%
\pgfpathmoveto{\pgfqpoint{1.942951in}{3.439868in}}%
\pgfusepath{stroke}%
\end{pgfscope}%
\begin{pgfscope}%
\pgfpathrectangle{\pgfqpoint{0.765000in}{0.660000in}}{\pgfqpoint{4.620000in}{4.620000in}}%
\pgfusepath{clip}%
\pgfsetbuttcap%
\pgfsetmiterjoin%
\definecolor{currentfill}{rgb}{0.176471,0.192157,0.258824}%
\pgfsetfillcolor{currentfill}%
\pgfsetlinewidth{1.003750pt}%
\definecolor{currentstroke}{rgb}{0.176471,0.192157,0.258824}%
\pgfsetstrokecolor{currentstroke}%
\pgfsetdash{}{0pt}%
\pgfsys@defobject{currentmarker}{\pgfqpoint{-0.033333in}{-0.033333in}}{\pgfqpoint{0.033333in}{0.033333in}}{%
\pgfpathmoveto{\pgfqpoint{-0.000000in}{-0.033333in}}%
\pgfpathlineto{\pgfqpoint{0.033333in}{0.033333in}}%
\pgfpathlineto{\pgfqpoint{-0.033333in}{0.033333in}}%
\pgfpathlineto{\pgfqpoint{-0.000000in}{-0.033333in}}%
\pgfpathclose%
\pgfusepath{stroke,fill}%
}%
\begin{pgfscope}%
\pgfsys@transformshift{1.942951in}{3.439868in}%
\pgfsys@useobject{currentmarker}{}%
\end{pgfscope}%
\end{pgfscope}%
\begin{pgfscope}%
\pgfpathrectangle{\pgfqpoint{0.765000in}{0.660000in}}{\pgfqpoint{4.620000in}{4.620000in}}%
\pgfusepath{clip}%
\pgfsetrectcap%
\pgfsetroundjoin%
\pgfsetlinewidth{1.204500pt}%
\definecolor{currentstroke}{rgb}{0.176471,0.192157,0.258824}%
\pgfsetstrokecolor{currentstroke}%
\pgfsetdash{}{0pt}%
\pgfpathmoveto{\pgfqpoint{1.647734in}{3.036230in}}%
\pgfusepath{stroke}%
\end{pgfscope}%
\begin{pgfscope}%
\pgfpathrectangle{\pgfqpoint{0.765000in}{0.660000in}}{\pgfqpoint{4.620000in}{4.620000in}}%
\pgfusepath{clip}%
\pgfsetbuttcap%
\pgfsetmiterjoin%
\definecolor{currentfill}{rgb}{0.176471,0.192157,0.258824}%
\pgfsetfillcolor{currentfill}%
\pgfsetlinewidth{1.003750pt}%
\definecolor{currentstroke}{rgb}{0.176471,0.192157,0.258824}%
\pgfsetstrokecolor{currentstroke}%
\pgfsetdash{}{0pt}%
\pgfsys@defobject{currentmarker}{\pgfqpoint{-0.033333in}{-0.033333in}}{\pgfqpoint{0.033333in}{0.033333in}}{%
\pgfpathmoveto{\pgfqpoint{-0.000000in}{-0.033333in}}%
\pgfpathlineto{\pgfqpoint{0.033333in}{0.033333in}}%
\pgfpathlineto{\pgfqpoint{-0.033333in}{0.033333in}}%
\pgfpathlineto{\pgfqpoint{-0.000000in}{-0.033333in}}%
\pgfpathclose%
\pgfusepath{stroke,fill}%
}%
\begin{pgfscope}%
\pgfsys@transformshift{1.647734in}{3.036230in}%
\pgfsys@useobject{currentmarker}{}%
\end{pgfscope}%
\end{pgfscope}%
\begin{pgfscope}%
\pgfpathrectangle{\pgfqpoint{0.765000in}{0.660000in}}{\pgfqpoint{4.620000in}{4.620000in}}%
\pgfusepath{clip}%
\pgfsetrectcap%
\pgfsetroundjoin%
\pgfsetlinewidth{1.204500pt}%
\definecolor{currentstroke}{rgb}{0.176471,0.192157,0.258824}%
\pgfsetstrokecolor{currentstroke}%
\pgfsetdash{}{0pt}%
\pgfpathmoveto{\pgfqpoint{3.322501in}{3.749250in}}%
\pgfusepath{stroke}%
\end{pgfscope}%
\begin{pgfscope}%
\pgfpathrectangle{\pgfqpoint{0.765000in}{0.660000in}}{\pgfqpoint{4.620000in}{4.620000in}}%
\pgfusepath{clip}%
\pgfsetbuttcap%
\pgfsetmiterjoin%
\definecolor{currentfill}{rgb}{0.176471,0.192157,0.258824}%
\pgfsetfillcolor{currentfill}%
\pgfsetlinewidth{1.003750pt}%
\definecolor{currentstroke}{rgb}{0.176471,0.192157,0.258824}%
\pgfsetstrokecolor{currentstroke}%
\pgfsetdash{}{0pt}%
\pgfsys@defobject{currentmarker}{\pgfqpoint{-0.033333in}{-0.033333in}}{\pgfqpoint{0.033333in}{0.033333in}}{%
\pgfpathmoveto{\pgfqpoint{-0.000000in}{-0.033333in}}%
\pgfpathlineto{\pgfqpoint{0.033333in}{0.033333in}}%
\pgfpathlineto{\pgfqpoint{-0.033333in}{0.033333in}}%
\pgfpathlineto{\pgfqpoint{-0.000000in}{-0.033333in}}%
\pgfpathclose%
\pgfusepath{stroke,fill}%
}%
\begin{pgfscope}%
\pgfsys@transformshift{3.322501in}{3.749250in}%
\pgfsys@useobject{currentmarker}{}%
\end{pgfscope}%
\end{pgfscope}%
\begin{pgfscope}%
\pgfpathrectangle{\pgfqpoint{0.765000in}{0.660000in}}{\pgfqpoint{4.620000in}{4.620000in}}%
\pgfusepath{clip}%
\pgfsetrectcap%
\pgfsetroundjoin%
\pgfsetlinewidth{1.204500pt}%
\definecolor{currentstroke}{rgb}{0.176471,0.192157,0.258824}%
\pgfsetstrokecolor{currentstroke}%
\pgfsetdash{}{0pt}%
\pgfpathmoveto{\pgfqpoint{1.975323in}{3.493702in}}%
\pgfusepath{stroke}%
\end{pgfscope}%
\begin{pgfscope}%
\pgfpathrectangle{\pgfqpoint{0.765000in}{0.660000in}}{\pgfqpoint{4.620000in}{4.620000in}}%
\pgfusepath{clip}%
\pgfsetbuttcap%
\pgfsetmiterjoin%
\definecolor{currentfill}{rgb}{0.176471,0.192157,0.258824}%
\pgfsetfillcolor{currentfill}%
\pgfsetlinewidth{1.003750pt}%
\definecolor{currentstroke}{rgb}{0.176471,0.192157,0.258824}%
\pgfsetstrokecolor{currentstroke}%
\pgfsetdash{}{0pt}%
\pgfsys@defobject{currentmarker}{\pgfqpoint{-0.033333in}{-0.033333in}}{\pgfqpoint{0.033333in}{0.033333in}}{%
\pgfpathmoveto{\pgfqpoint{-0.000000in}{-0.033333in}}%
\pgfpathlineto{\pgfqpoint{0.033333in}{0.033333in}}%
\pgfpathlineto{\pgfqpoint{-0.033333in}{0.033333in}}%
\pgfpathlineto{\pgfqpoint{-0.000000in}{-0.033333in}}%
\pgfpathclose%
\pgfusepath{stroke,fill}%
}%
\begin{pgfscope}%
\pgfsys@transformshift{1.975323in}{3.493702in}%
\pgfsys@useobject{currentmarker}{}%
\end{pgfscope}%
\end{pgfscope}%
\begin{pgfscope}%
\pgfpathrectangle{\pgfqpoint{0.765000in}{0.660000in}}{\pgfqpoint{4.620000in}{4.620000in}}%
\pgfusepath{clip}%
\pgfsetrectcap%
\pgfsetroundjoin%
\pgfsetlinewidth{1.204500pt}%
\definecolor{currentstroke}{rgb}{0.176471,0.192157,0.258824}%
\pgfsetstrokecolor{currentstroke}%
\pgfsetdash{}{0pt}%
\pgfpathmoveto{\pgfqpoint{3.218012in}{3.993230in}}%
\pgfusepath{stroke}%
\end{pgfscope}%
\begin{pgfscope}%
\pgfpathrectangle{\pgfqpoint{0.765000in}{0.660000in}}{\pgfqpoint{4.620000in}{4.620000in}}%
\pgfusepath{clip}%
\pgfsetbuttcap%
\pgfsetmiterjoin%
\definecolor{currentfill}{rgb}{0.176471,0.192157,0.258824}%
\pgfsetfillcolor{currentfill}%
\pgfsetlinewidth{1.003750pt}%
\definecolor{currentstroke}{rgb}{0.176471,0.192157,0.258824}%
\pgfsetstrokecolor{currentstroke}%
\pgfsetdash{}{0pt}%
\pgfsys@defobject{currentmarker}{\pgfqpoint{-0.033333in}{-0.033333in}}{\pgfqpoint{0.033333in}{0.033333in}}{%
\pgfpathmoveto{\pgfqpoint{-0.000000in}{-0.033333in}}%
\pgfpathlineto{\pgfqpoint{0.033333in}{0.033333in}}%
\pgfpathlineto{\pgfqpoint{-0.033333in}{0.033333in}}%
\pgfpathlineto{\pgfqpoint{-0.000000in}{-0.033333in}}%
\pgfpathclose%
\pgfusepath{stroke,fill}%
}%
\begin{pgfscope}%
\pgfsys@transformshift{3.218012in}{3.993230in}%
\pgfsys@useobject{currentmarker}{}%
\end{pgfscope}%
\end{pgfscope}%
\begin{pgfscope}%
\pgfpathrectangle{\pgfqpoint{0.765000in}{0.660000in}}{\pgfqpoint{4.620000in}{4.620000in}}%
\pgfusepath{clip}%
\pgfsetrectcap%
\pgfsetroundjoin%
\pgfsetlinewidth{1.204500pt}%
\definecolor{currentstroke}{rgb}{0.176471,0.192157,0.258824}%
\pgfsetstrokecolor{currentstroke}%
\pgfsetdash{}{0pt}%
\pgfpathmoveto{\pgfqpoint{4.109341in}{3.247221in}}%
\pgfusepath{stroke}%
\end{pgfscope}%
\begin{pgfscope}%
\pgfpathrectangle{\pgfqpoint{0.765000in}{0.660000in}}{\pgfqpoint{4.620000in}{4.620000in}}%
\pgfusepath{clip}%
\pgfsetbuttcap%
\pgfsetmiterjoin%
\definecolor{currentfill}{rgb}{0.176471,0.192157,0.258824}%
\pgfsetfillcolor{currentfill}%
\pgfsetlinewidth{1.003750pt}%
\definecolor{currentstroke}{rgb}{0.176471,0.192157,0.258824}%
\pgfsetstrokecolor{currentstroke}%
\pgfsetdash{}{0pt}%
\pgfsys@defobject{currentmarker}{\pgfqpoint{-0.033333in}{-0.033333in}}{\pgfqpoint{0.033333in}{0.033333in}}{%
\pgfpathmoveto{\pgfqpoint{-0.000000in}{-0.033333in}}%
\pgfpathlineto{\pgfqpoint{0.033333in}{0.033333in}}%
\pgfpathlineto{\pgfqpoint{-0.033333in}{0.033333in}}%
\pgfpathlineto{\pgfqpoint{-0.000000in}{-0.033333in}}%
\pgfpathclose%
\pgfusepath{stroke,fill}%
}%
\begin{pgfscope}%
\pgfsys@transformshift{4.109341in}{3.247221in}%
\pgfsys@useobject{currentmarker}{}%
\end{pgfscope}%
\end{pgfscope}%
\begin{pgfscope}%
\pgfpathrectangle{\pgfqpoint{0.765000in}{0.660000in}}{\pgfqpoint{4.620000in}{4.620000in}}%
\pgfusepath{clip}%
\pgfsetrectcap%
\pgfsetroundjoin%
\pgfsetlinewidth{1.204500pt}%
\definecolor{currentstroke}{rgb}{0.176471,0.192157,0.258824}%
\pgfsetstrokecolor{currentstroke}%
\pgfsetdash{}{0pt}%
\pgfpathmoveto{\pgfqpoint{1.443774in}{2.982997in}}%
\pgfusepath{stroke}%
\end{pgfscope}%
\begin{pgfscope}%
\pgfpathrectangle{\pgfqpoint{0.765000in}{0.660000in}}{\pgfqpoint{4.620000in}{4.620000in}}%
\pgfusepath{clip}%
\pgfsetbuttcap%
\pgfsetmiterjoin%
\definecolor{currentfill}{rgb}{0.176471,0.192157,0.258824}%
\pgfsetfillcolor{currentfill}%
\pgfsetlinewidth{1.003750pt}%
\definecolor{currentstroke}{rgb}{0.176471,0.192157,0.258824}%
\pgfsetstrokecolor{currentstroke}%
\pgfsetdash{}{0pt}%
\pgfsys@defobject{currentmarker}{\pgfqpoint{-0.033333in}{-0.033333in}}{\pgfqpoint{0.033333in}{0.033333in}}{%
\pgfpathmoveto{\pgfqpoint{-0.000000in}{-0.033333in}}%
\pgfpathlineto{\pgfqpoint{0.033333in}{0.033333in}}%
\pgfpathlineto{\pgfqpoint{-0.033333in}{0.033333in}}%
\pgfpathlineto{\pgfqpoint{-0.000000in}{-0.033333in}}%
\pgfpathclose%
\pgfusepath{stroke,fill}%
}%
\begin{pgfscope}%
\pgfsys@transformshift{1.443774in}{2.982997in}%
\pgfsys@useobject{currentmarker}{}%
\end{pgfscope}%
\end{pgfscope}%
\begin{pgfscope}%
\pgfpathrectangle{\pgfqpoint{0.765000in}{0.660000in}}{\pgfqpoint{4.620000in}{4.620000in}}%
\pgfusepath{clip}%
\pgfsetrectcap%
\pgfsetroundjoin%
\pgfsetlinewidth{1.204500pt}%
\definecolor{currentstroke}{rgb}{0.176471,0.192157,0.258824}%
\pgfsetstrokecolor{currentstroke}%
\pgfsetdash{}{0pt}%
\pgfpathmoveto{\pgfqpoint{3.658773in}{3.657330in}}%
\pgfusepath{stroke}%
\end{pgfscope}%
\begin{pgfscope}%
\pgfpathrectangle{\pgfqpoint{0.765000in}{0.660000in}}{\pgfqpoint{4.620000in}{4.620000in}}%
\pgfusepath{clip}%
\pgfsetbuttcap%
\pgfsetmiterjoin%
\definecolor{currentfill}{rgb}{0.176471,0.192157,0.258824}%
\pgfsetfillcolor{currentfill}%
\pgfsetlinewidth{1.003750pt}%
\definecolor{currentstroke}{rgb}{0.176471,0.192157,0.258824}%
\pgfsetstrokecolor{currentstroke}%
\pgfsetdash{}{0pt}%
\pgfsys@defobject{currentmarker}{\pgfqpoint{-0.033333in}{-0.033333in}}{\pgfqpoint{0.033333in}{0.033333in}}{%
\pgfpathmoveto{\pgfqpoint{-0.000000in}{-0.033333in}}%
\pgfpathlineto{\pgfqpoint{0.033333in}{0.033333in}}%
\pgfpathlineto{\pgfqpoint{-0.033333in}{0.033333in}}%
\pgfpathlineto{\pgfqpoint{-0.000000in}{-0.033333in}}%
\pgfpathclose%
\pgfusepath{stroke,fill}%
}%
\begin{pgfscope}%
\pgfsys@transformshift{3.658773in}{3.657330in}%
\pgfsys@useobject{currentmarker}{}%
\end{pgfscope}%
\end{pgfscope}%
\begin{pgfscope}%
\pgfpathrectangle{\pgfqpoint{0.765000in}{0.660000in}}{\pgfqpoint{4.620000in}{4.620000in}}%
\pgfusepath{clip}%
\pgfsetrectcap%
\pgfsetroundjoin%
\pgfsetlinewidth{1.204500pt}%
\definecolor{currentstroke}{rgb}{0.176471,0.192157,0.258824}%
\pgfsetstrokecolor{currentstroke}%
\pgfsetdash{}{0pt}%
\pgfpathmoveto{\pgfqpoint{3.734816in}{3.019265in}}%
\pgfusepath{stroke}%
\end{pgfscope}%
\begin{pgfscope}%
\pgfpathrectangle{\pgfqpoint{0.765000in}{0.660000in}}{\pgfqpoint{4.620000in}{4.620000in}}%
\pgfusepath{clip}%
\pgfsetbuttcap%
\pgfsetmiterjoin%
\definecolor{currentfill}{rgb}{0.176471,0.192157,0.258824}%
\pgfsetfillcolor{currentfill}%
\pgfsetlinewidth{1.003750pt}%
\definecolor{currentstroke}{rgb}{0.176471,0.192157,0.258824}%
\pgfsetstrokecolor{currentstroke}%
\pgfsetdash{}{0pt}%
\pgfsys@defobject{currentmarker}{\pgfqpoint{-0.033333in}{-0.033333in}}{\pgfqpoint{0.033333in}{0.033333in}}{%
\pgfpathmoveto{\pgfqpoint{-0.000000in}{-0.033333in}}%
\pgfpathlineto{\pgfqpoint{0.033333in}{0.033333in}}%
\pgfpathlineto{\pgfqpoint{-0.033333in}{0.033333in}}%
\pgfpathlineto{\pgfqpoint{-0.000000in}{-0.033333in}}%
\pgfpathclose%
\pgfusepath{stroke,fill}%
}%
\begin{pgfscope}%
\pgfsys@transformshift{3.734816in}{3.019265in}%
\pgfsys@useobject{currentmarker}{}%
\end{pgfscope}%
\end{pgfscope}%
\begin{pgfscope}%
\pgfpathrectangle{\pgfqpoint{0.765000in}{0.660000in}}{\pgfqpoint{4.620000in}{4.620000in}}%
\pgfusepath{clip}%
\pgfsetrectcap%
\pgfsetroundjoin%
\pgfsetlinewidth{1.204500pt}%
\definecolor{currentstroke}{rgb}{0.176471,0.192157,0.258824}%
\pgfsetstrokecolor{currentstroke}%
\pgfsetdash{}{0pt}%
\pgfpathmoveto{\pgfqpoint{2.673707in}{3.614585in}}%
\pgfusepath{stroke}%
\end{pgfscope}%
\begin{pgfscope}%
\pgfpathrectangle{\pgfqpoint{0.765000in}{0.660000in}}{\pgfqpoint{4.620000in}{4.620000in}}%
\pgfusepath{clip}%
\pgfsetbuttcap%
\pgfsetmiterjoin%
\definecolor{currentfill}{rgb}{0.176471,0.192157,0.258824}%
\pgfsetfillcolor{currentfill}%
\pgfsetlinewidth{1.003750pt}%
\definecolor{currentstroke}{rgb}{0.176471,0.192157,0.258824}%
\pgfsetstrokecolor{currentstroke}%
\pgfsetdash{}{0pt}%
\pgfsys@defobject{currentmarker}{\pgfqpoint{-0.033333in}{-0.033333in}}{\pgfqpoint{0.033333in}{0.033333in}}{%
\pgfpathmoveto{\pgfqpoint{-0.000000in}{-0.033333in}}%
\pgfpathlineto{\pgfqpoint{0.033333in}{0.033333in}}%
\pgfpathlineto{\pgfqpoint{-0.033333in}{0.033333in}}%
\pgfpathlineto{\pgfqpoint{-0.000000in}{-0.033333in}}%
\pgfpathclose%
\pgfusepath{stroke,fill}%
}%
\begin{pgfscope}%
\pgfsys@transformshift{2.673707in}{3.614585in}%
\pgfsys@useobject{currentmarker}{}%
\end{pgfscope}%
\end{pgfscope}%
\begin{pgfscope}%
\pgfpathrectangle{\pgfqpoint{0.765000in}{0.660000in}}{\pgfqpoint{4.620000in}{4.620000in}}%
\pgfusepath{clip}%
\pgfsetrectcap%
\pgfsetroundjoin%
\pgfsetlinewidth{1.204500pt}%
\definecolor{currentstroke}{rgb}{0.176471,0.192157,0.258824}%
\pgfsetstrokecolor{currentstroke}%
\pgfsetdash{}{0pt}%
\pgfpathmoveto{\pgfqpoint{2.030724in}{3.262091in}}%
\pgfusepath{stroke}%
\end{pgfscope}%
\begin{pgfscope}%
\pgfpathrectangle{\pgfqpoint{0.765000in}{0.660000in}}{\pgfqpoint{4.620000in}{4.620000in}}%
\pgfusepath{clip}%
\pgfsetbuttcap%
\pgfsetmiterjoin%
\definecolor{currentfill}{rgb}{0.176471,0.192157,0.258824}%
\pgfsetfillcolor{currentfill}%
\pgfsetlinewidth{1.003750pt}%
\definecolor{currentstroke}{rgb}{0.176471,0.192157,0.258824}%
\pgfsetstrokecolor{currentstroke}%
\pgfsetdash{}{0pt}%
\pgfsys@defobject{currentmarker}{\pgfqpoint{-0.033333in}{-0.033333in}}{\pgfqpoint{0.033333in}{0.033333in}}{%
\pgfpathmoveto{\pgfqpoint{-0.000000in}{-0.033333in}}%
\pgfpathlineto{\pgfqpoint{0.033333in}{0.033333in}}%
\pgfpathlineto{\pgfqpoint{-0.033333in}{0.033333in}}%
\pgfpathlineto{\pgfqpoint{-0.000000in}{-0.033333in}}%
\pgfpathclose%
\pgfusepath{stroke,fill}%
}%
\begin{pgfscope}%
\pgfsys@transformshift{2.030724in}{3.262091in}%
\pgfsys@useobject{currentmarker}{}%
\end{pgfscope}%
\end{pgfscope}%
\begin{pgfscope}%
\pgfpathrectangle{\pgfqpoint{0.765000in}{0.660000in}}{\pgfqpoint{4.620000in}{4.620000in}}%
\pgfusepath{clip}%
\pgfsetrectcap%
\pgfsetroundjoin%
\pgfsetlinewidth{1.204500pt}%
\definecolor{currentstroke}{rgb}{0.176471,0.192157,0.258824}%
\pgfsetstrokecolor{currentstroke}%
\pgfsetdash{}{0pt}%
\pgfpathmoveto{\pgfqpoint{1.151863in}{3.062159in}}%
\pgfusepath{stroke}%
\end{pgfscope}%
\begin{pgfscope}%
\pgfpathrectangle{\pgfqpoint{0.765000in}{0.660000in}}{\pgfqpoint{4.620000in}{4.620000in}}%
\pgfusepath{clip}%
\pgfsetbuttcap%
\pgfsetmiterjoin%
\definecolor{currentfill}{rgb}{0.176471,0.192157,0.258824}%
\pgfsetfillcolor{currentfill}%
\pgfsetlinewidth{1.003750pt}%
\definecolor{currentstroke}{rgb}{0.176471,0.192157,0.258824}%
\pgfsetstrokecolor{currentstroke}%
\pgfsetdash{}{0pt}%
\pgfsys@defobject{currentmarker}{\pgfqpoint{-0.033333in}{-0.033333in}}{\pgfqpoint{0.033333in}{0.033333in}}{%
\pgfpathmoveto{\pgfqpoint{-0.000000in}{-0.033333in}}%
\pgfpathlineto{\pgfqpoint{0.033333in}{0.033333in}}%
\pgfpathlineto{\pgfqpoint{-0.033333in}{0.033333in}}%
\pgfpathlineto{\pgfqpoint{-0.000000in}{-0.033333in}}%
\pgfpathclose%
\pgfusepath{stroke,fill}%
}%
\begin{pgfscope}%
\pgfsys@transformshift{1.151863in}{3.062159in}%
\pgfsys@useobject{currentmarker}{}%
\end{pgfscope}%
\end{pgfscope}%
\begin{pgfscope}%
\pgfpathrectangle{\pgfqpoint{0.765000in}{0.660000in}}{\pgfqpoint{4.620000in}{4.620000in}}%
\pgfusepath{clip}%
\pgfsetrectcap%
\pgfsetroundjoin%
\pgfsetlinewidth{1.204500pt}%
\definecolor{currentstroke}{rgb}{0.176471,0.192157,0.258824}%
\pgfsetstrokecolor{currentstroke}%
\pgfsetdash{}{0pt}%
\pgfpathmoveto{\pgfqpoint{3.873934in}{3.382896in}}%
\pgfusepath{stroke}%
\end{pgfscope}%
\begin{pgfscope}%
\pgfpathrectangle{\pgfqpoint{0.765000in}{0.660000in}}{\pgfqpoint{4.620000in}{4.620000in}}%
\pgfusepath{clip}%
\pgfsetbuttcap%
\pgfsetmiterjoin%
\definecolor{currentfill}{rgb}{0.176471,0.192157,0.258824}%
\pgfsetfillcolor{currentfill}%
\pgfsetlinewidth{1.003750pt}%
\definecolor{currentstroke}{rgb}{0.176471,0.192157,0.258824}%
\pgfsetstrokecolor{currentstroke}%
\pgfsetdash{}{0pt}%
\pgfsys@defobject{currentmarker}{\pgfqpoint{-0.033333in}{-0.033333in}}{\pgfqpoint{0.033333in}{0.033333in}}{%
\pgfpathmoveto{\pgfqpoint{-0.000000in}{-0.033333in}}%
\pgfpathlineto{\pgfqpoint{0.033333in}{0.033333in}}%
\pgfpathlineto{\pgfqpoint{-0.033333in}{0.033333in}}%
\pgfpathlineto{\pgfqpoint{-0.000000in}{-0.033333in}}%
\pgfpathclose%
\pgfusepath{stroke,fill}%
}%
\begin{pgfscope}%
\pgfsys@transformshift{3.873934in}{3.382896in}%
\pgfsys@useobject{currentmarker}{}%
\end{pgfscope}%
\end{pgfscope}%
\begin{pgfscope}%
\pgfpathrectangle{\pgfqpoint{0.765000in}{0.660000in}}{\pgfqpoint{4.620000in}{4.620000in}}%
\pgfusepath{clip}%
\pgfsetrectcap%
\pgfsetroundjoin%
\pgfsetlinewidth{1.204500pt}%
\definecolor{currentstroke}{rgb}{0.176471,0.192157,0.258824}%
\pgfsetstrokecolor{currentstroke}%
\pgfsetdash{}{0pt}%
\pgfpathmoveto{\pgfqpoint{3.751687in}{2.898405in}}%
\pgfusepath{stroke}%
\end{pgfscope}%
\begin{pgfscope}%
\pgfpathrectangle{\pgfqpoint{0.765000in}{0.660000in}}{\pgfqpoint{4.620000in}{4.620000in}}%
\pgfusepath{clip}%
\pgfsetbuttcap%
\pgfsetmiterjoin%
\definecolor{currentfill}{rgb}{0.176471,0.192157,0.258824}%
\pgfsetfillcolor{currentfill}%
\pgfsetlinewidth{1.003750pt}%
\definecolor{currentstroke}{rgb}{0.176471,0.192157,0.258824}%
\pgfsetstrokecolor{currentstroke}%
\pgfsetdash{}{0pt}%
\pgfsys@defobject{currentmarker}{\pgfqpoint{-0.033333in}{-0.033333in}}{\pgfqpoint{0.033333in}{0.033333in}}{%
\pgfpathmoveto{\pgfqpoint{-0.000000in}{-0.033333in}}%
\pgfpathlineto{\pgfqpoint{0.033333in}{0.033333in}}%
\pgfpathlineto{\pgfqpoint{-0.033333in}{0.033333in}}%
\pgfpathlineto{\pgfqpoint{-0.000000in}{-0.033333in}}%
\pgfpathclose%
\pgfusepath{stroke,fill}%
}%
\begin{pgfscope}%
\pgfsys@transformshift{3.751687in}{2.898405in}%
\pgfsys@useobject{currentmarker}{}%
\end{pgfscope}%
\end{pgfscope}%
\begin{pgfscope}%
\pgfpathrectangle{\pgfqpoint{0.765000in}{0.660000in}}{\pgfqpoint{4.620000in}{4.620000in}}%
\pgfusepath{clip}%
\pgfsetrectcap%
\pgfsetroundjoin%
\pgfsetlinewidth{1.204500pt}%
\definecolor{currentstroke}{rgb}{0.176471,0.192157,0.258824}%
\pgfsetstrokecolor{currentstroke}%
\pgfsetdash{}{0pt}%
\pgfpathmoveto{\pgfqpoint{3.504135in}{3.912421in}}%
\pgfusepath{stroke}%
\end{pgfscope}%
\begin{pgfscope}%
\pgfpathrectangle{\pgfqpoint{0.765000in}{0.660000in}}{\pgfqpoint{4.620000in}{4.620000in}}%
\pgfusepath{clip}%
\pgfsetbuttcap%
\pgfsetmiterjoin%
\definecolor{currentfill}{rgb}{0.176471,0.192157,0.258824}%
\pgfsetfillcolor{currentfill}%
\pgfsetlinewidth{1.003750pt}%
\definecolor{currentstroke}{rgb}{0.176471,0.192157,0.258824}%
\pgfsetstrokecolor{currentstroke}%
\pgfsetdash{}{0pt}%
\pgfsys@defobject{currentmarker}{\pgfqpoint{-0.033333in}{-0.033333in}}{\pgfqpoint{0.033333in}{0.033333in}}{%
\pgfpathmoveto{\pgfqpoint{-0.000000in}{-0.033333in}}%
\pgfpathlineto{\pgfqpoint{0.033333in}{0.033333in}}%
\pgfpathlineto{\pgfqpoint{-0.033333in}{0.033333in}}%
\pgfpathlineto{\pgfqpoint{-0.000000in}{-0.033333in}}%
\pgfpathclose%
\pgfusepath{stroke,fill}%
}%
\begin{pgfscope}%
\pgfsys@transformshift{3.504135in}{3.912421in}%
\pgfsys@useobject{currentmarker}{}%
\end{pgfscope}%
\end{pgfscope}%
\begin{pgfscope}%
\pgfpathrectangle{\pgfqpoint{0.765000in}{0.660000in}}{\pgfqpoint{4.620000in}{4.620000in}}%
\pgfusepath{clip}%
\pgfsetrectcap%
\pgfsetroundjoin%
\pgfsetlinewidth{1.204500pt}%
\definecolor{currentstroke}{rgb}{0.176471,0.192157,0.258824}%
\pgfsetstrokecolor{currentstroke}%
\pgfsetdash{}{0pt}%
\pgfpathmoveto{\pgfqpoint{4.098812in}{3.295655in}}%
\pgfusepath{stroke}%
\end{pgfscope}%
\begin{pgfscope}%
\pgfpathrectangle{\pgfqpoint{0.765000in}{0.660000in}}{\pgfqpoint{4.620000in}{4.620000in}}%
\pgfusepath{clip}%
\pgfsetbuttcap%
\pgfsetmiterjoin%
\definecolor{currentfill}{rgb}{0.176471,0.192157,0.258824}%
\pgfsetfillcolor{currentfill}%
\pgfsetlinewidth{1.003750pt}%
\definecolor{currentstroke}{rgb}{0.176471,0.192157,0.258824}%
\pgfsetstrokecolor{currentstroke}%
\pgfsetdash{}{0pt}%
\pgfsys@defobject{currentmarker}{\pgfqpoint{-0.033333in}{-0.033333in}}{\pgfqpoint{0.033333in}{0.033333in}}{%
\pgfpathmoveto{\pgfqpoint{-0.000000in}{-0.033333in}}%
\pgfpathlineto{\pgfqpoint{0.033333in}{0.033333in}}%
\pgfpathlineto{\pgfqpoint{-0.033333in}{0.033333in}}%
\pgfpathlineto{\pgfqpoint{-0.000000in}{-0.033333in}}%
\pgfpathclose%
\pgfusepath{stroke,fill}%
}%
\begin{pgfscope}%
\pgfsys@transformshift{4.098812in}{3.295655in}%
\pgfsys@useobject{currentmarker}{}%
\end{pgfscope}%
\end{pgfscope}%
\begin{pgfscope}%
\pgfpathrectangle{\pgfqpoint{0.765000in}{0.660000in}}{\pgfqpoint{4.620000in}{4.620000in}}%
\pgfusepath{clip}%
\pgfsetrectcap%
\pgfsetroundjoin%
\pgfsetlinewidth{1.204500pt}%
\definecolor{currentstroke}{rgb}{0.176471,0.192157,0.258824}%
\pgfsetstrokecolor{currentstroke}%
\pgfsetdash{}{0pt}%
\pgfpathmoveto{\pgfqpoint{3.509425in}{3.287215in}}%
\pgfusepath{stroke}%
\end{pgfscope}%
\begin{pgfscope}%
\pgfpathrectangle{\pgfqpoint{0.765000in}{0.660000in}}{\pgfqpoint{4.620000in}{4.620000in}}%
\pgfusepath{clip}%
\pgfsetbuttcap%
\pgfsetmiterjoin%
\definecolor{currentfill}{rgb}{0.176471,0.192157,0.258824}%
\pgfsetfillcolor{currentfill}%
\pgfsetlinewidth{1.003750pt}%
\definecolor{currentstroke}{rgb}{0.176471,0.192157,0.258824}%
\pgfsetstrokecolor{currentstroke}%
\pgfsetdash{}{0pt}%
\pgfsys@defobject{currentmarker}{\pgfqpoint{-0.033333in}{-0.033333in}}{\pgfqpoint{0.033333in}{0.033333in}}{%
\pgfpathmoveto{\pgfqpoint{-0.000000in}{-0.033333in}}%
\pgfpathlineto{\pgfqpoint{0.033333in}{0.033333in}}%
\pgfpathlineto{\pgfqpoint{-0.033333in}{0.033333in}}%
\pgfpathlineto{\pgfqpoint{-0.000000in}{-0.033333in}}%
\pgfpathclose%
\pgfusepath{stroke,fill}%
}%
\begin{pgfscope}%
\pgfsys@transformshift{3.509425in}{3.287215in}%
\pgfsys@useobject{currentmarker}{}%
\end{pgfscope}%
\end{pgfscope}%
\begin{pgfscope}%
\pgfpathrectangle{\pgfqpoint{0.765000in}{0.660000in}}{\pgfqpoint{4.620000in}{4.620000in}}%
\pgfusepath{clip}%
\pgfsetrectcap%
\pgfsetroundjoin%
\pgfsetlinewidth{1.204500pt}%
\definecolor{currentstroke}{rgb}{0.176471,0.192157,0.258824}%
\pgfsetstrokecolor{currentstroke}%
\pgfsetdash{}{0pt}%
\pgfpathmoveto{\pgfqpoint{3.473620in}{3.266050in}}%
\pgfusepath{stroke}%
\end{pgfscope}%
\begin{pgfscope}%
\pgfpathrectangle{\pgfqpoint{0.765000in}{0.660000in}}{\pgfqpoint{4.620000in}{4.620000in}}%
\pgfusepath{clip}%
\pgfsetbuttcap%
\pgfsetmiterjoin%
\definecolor{currentfill}{rgb}{0.176471,0.192157,0.258824}%
\pgfsetfillcolor{currentfill}%
\pgfsetlinewidth{1.003750pt}%
\definecolor{currentstroke}{rgb}{0.176471,0.192157,0.258824}%
\pgfsetstrokecolor{currentstroke}%
\pgfsetdash{}{0pt}%
\pgfsys@defobject{currentmarker}{\pgfqpoint{-0.033333in}{-0.033333in}}{\pgfqpoint{0.033333in}{0.033333in}}{%
\pgfpathmoveto{\pgfqpoint{-0.000000in}{-0.033333in}}%
\pgfpathlineto{\pgfqpoint{0.033333in}{0.033333in}}%
\pgfpathlineto{\pgfqpoint{-0.033333in}{0.033333in}}%
\pgfpathlineto{\pgfqpoint{-0.000000in}{-0.033333in}}%
\pgfpathclose%
\pgfusepath{stroke,fill}%
}%
\begin{pgfscope}%
\pgfsys@transformshift{3.473620in}{3.266050in}%
\pgfsys@useobject{currentmarker}{}%
\end{pgfscope}%
\end{pgfscope}%
\begin{pgfscope}%
\pgfpathrectangle{\pgfqpoint{0.765000in}{0.660000in}}{\pgfqpoint{4.620000in}{4.620000in}}%
\pgfusepath{clip}%
\pgfsetrectcap%
\pgfsetroundjoin%
\pgfsetlinewidth{1.204500pt}%
\definecolor{currentstroke}{rgb}{0.176471,0.192157,0.258824}%
\pgfsetstrokecolor{currentstroke}%
\pgfsetdash{}{0pt}%
\pgfpathmoveto{\pgfqpoint{2.426243in}{3.568364in}}%
\pgfusepath{stroke}%
\end{pgfscope}%
\begin{pgfscope}%
\pgfpathrectangle{\pgfqpoint{0.765000in}{0.660000in}}{\pgfqpoint{4.620000in}{4.620000in}}%
\pgfusepath{clip}%
\pgfsetbuttcap%
\pgfsetmiterjoin%
\definecolor{currentfill}{rgb}{0.176471,0.192157,0.258824}%
\pgfsetfillcolor{currentfill}%
\pgfsetlinewidth{1.003750pt}%
\definecolor{currentstroke}{rgb}{0.176471,0.192157,0.258824}%
\pgfsetstrokecolor{currentstroke}%
\pgfsetdash{}{0pt}%
\pgfsys@defobject{currentmarker}{\pgfqpoint{-0.033333in}{-0.033333in}}{\pgfqpoint{0.033333in}{0.033333in}}{%
\pgfpathmoveto{\pgfqpoint{-0.000000in}{-0.033333in}}%
\pgfpathlineto{\pgfqpoint{0.033333in}{0.033333in}}%
\pgfpathlineto{\pgfqpoint{-0.033333in}{0.033333in}}%
\pgfpathlineto{\pgfqpoint{-0.000000in}{-0.033333in}}%
\pgfpathclose%
\pgfusepath{stroke,fill}%
}%
\begin{pgfscope}%
\pgfsys@transformshift{2.426243in}{3.568364in}%
\pgfsys@useobject{currentmarker}{}%
\end{pgfscope}%
\end{pgfscope}%
\begin{pgfscope}%
\pgfpathrectangle{\pgfqpoint{0.765000in}{0.660000in}}{\pgfqpoint{4.620000in}{4.620000in}}%
\pgfusepath{clip}%
\pgfsetrectcap%
\pgfsetroundjoin%
\pgfsetlinewidth{1.204500pt}%
\definecolor{currentstroke}{rgb}{0.176471,0.192157,0.258824}%
\pgfsetstrokecolor{currentstroke}%
\pgfsetdash{}{0pt}%
\pgfpathmoveto{\pgfqpoint{2.830877in}{3.623423in}}%
\pgfusepath{stroke}%
\end{pgfscope}%
\begin{pgfscope}%
\pgfpathrectangle{\pgfqpoint{0.765000in}{0.660000in}}{\pgfqpoint{4.620000in}{4.620000in}}%
\pgfusepath{clip}%
\pgfsetbuttcap%
\pgfsetmiterjoin%
\definecolor{currentfill}{rgb}{0.176471,0.192157,0.258824}%
\pgfsetfillcolor{currentfill}%
\pgfsetlinewidth{1.003750pt}%
\definecolor{currentstroke}{rgb}{0.176471,0.192157,0.258824}%
\pgfsetstrokecolor{currentstroke}%
\pgfsetdash{}{0pt}%
\pgfsys@defobject{currentmarker}{\pgfqpoint{-0.033333in}{-0.033333in}}{\pgfqpoint{0.033333in}{0.033333in}}{%
\pgfpathmoveto{\pgfqpoint{-0.000000in}{-0.033333in}}%
\pgfpathlineto{\pgfqpoint{0.033333in}{0.033333in}}%
\pgfpathlineto{\pgfqpoint{-0.033333in}{0.033333in}}%
\pgfpathlineto{\pgfqpoint{-0.000000in}{-0.033333in}}%
\pgfpathclose%
\pgfusepath{stroke,fill}%
}%
\begin{pgfscope}%
\pgfsys@transformshift{2.830877in}{3.623423in}%
\pgfsys@useobject{currentmarker}{}%
\end{pgfscope}%
\end{pgfscope}%
\begin{pgfscope}%
\pgfpathrectangle{\pgfqpoint{0.765000in}{0.660000in}}{\pgfqpoint{4.620000in}{4.620000in}}%
\pgfusepath{clip}%
\pgfsetrectcap%
\pgfsetroundjoin%
\pgfsetlinewidth{1.204500pt}%
\definecolor{currentstroke}{rgb}{0.176471,0.192157,0.258824}%
\pgfsetstrokecolor{currentstroke}%
\pgfsetdash{}{0pt}%
\pgfpathmoveto{\pgfqpoint{3.720555in}{3.262853in}}%
\pgfusepath{stroke}%
\end{pgfscope}%
\begin{pgfscope}%
\pgfpathrectangle{\pgfqpoint{0.765000in}{0.660000in}}{\pgfqpoint{4.620000in}{4.620000in}}%
\pgfusepath{clip}%
\pgfsetbuttcap%
\pgfsetmiterjoin%
\definecolor{currentfill}{rgb}{0.176471,0.192157,0.258824}%
\pgfsetfillcolor{currentfill}%
\pgfsetlinewidth{1.003750pt}%
\definecolor{currentstroke}{rgb}{0.176471,0.192157,0.258824}%
\pgfsetstrokecolor{currentstroke}%
\pgfsetdash{}{0pt}%
\pgfsys@defobject{currentmarker}{\pgfqpoint{-0.033333in}{-0.033333in}}{\pgfqpoint{0.033333in}{0.033333in}}{%
\pgfpathmoveto{\pgfqpoint{-0.000000in}{-0.033333in}}%
\pgfpathlineto{\pgfqpoint{0.033333in}{0.033333in}}%
\pgfpathlineto{\pgfqpoint{-0.033333in}{0.033333in}}%
\pgfpathlineto{\pgfqpoint{-0.000000in}{-0.033333in}}%
\pgfpathclose%
\pgfusepath{stroke,fill}%
}%
\begin{pgfscope}%
\pgfsys@transformshift{3.720555in}{3.262853in}%
\pgfsys@useobject{currentmarker}{}%
\end{pgfscope}%
\end{pgfscope}%
\begin{pgfscope}%
\pgfpathrectangle{\pgfqpoint{0.765000in}{0.660000in}}{\pgfqpoint{4.620000in}{4.620000in}}%
\pgfusepath{clip}%
\pgfsetrectcap%
\pgfsetroundjoin%
\pgfsetlinewidth{1.204500pt}%
\definecolor{currentstroke}{rgb}{0.176471,0.192157,0.258824}%
\pgfsetstrokecolor{currentstroke}%
\pgfsetdash{}{0pt}%
\pgfpathmoveto{\pgfqpoint{3.347171in}{3.451717in}}%
\pgfusepath{stroke}%
\end{pgfscope}%
\begin{pgfscope}%
\pgfpathrectangle{\pgfqpoint{0.765000in}{0.660000in}}{\pgfqpoint{4.620000in}{4.620000in}}%
\pgfusepath{clip}%
\pgfsetbuttcap%
\pgfsetmiterjoin%
\definecolor{currentfill}{rgb}{0.176471,0.192157,0.258824}%
\pgfsetfillcolor{currentfill}%
\pgfsetlinewidth{1.003750pt}%
\definecolor{currentstroke}{rgb}{0.176471,0.192157,0.258824}%
\pgfsetstrokecolor{currentstroke}%
\pgfsetdash{}{0pt}%
\pgfsys@defobject{currentmarker}{\pgfqpoint{-0.033333in}{-0.033333in}}{\pgfqpoint{0.033333in}{0.033333in}}{%
\pgfpathmoveto{\pgfqpoint{-0.000000in}{-0.033333in}}%
\pgfpathlineto{\pgfqpoint{0.033333in}{0.033333in}}%
\pgfpathlineto{\pgfqpoint{-0.033333in}{0.033333in}}%
\pgfpathlineto{\pgfqpoint{-0.000000in}{-0.033333in}}%
\pgfpathclose%
\pgfusepath{stroke,fill}%
}%
\begin{pgfscope}%
\pgfsys@transformshift{3.347171in}{3.451717in}%
\pgfsys@useobject{currentmarker}{}%
\end{pgfscope}%
\end{pgfscope}%
\begin{pgfscope}%
\pgfpathrectangle{\pgfqpoint{0.765000in}{0.660000in}}{\pgfqpoint{4.620000in}{4.620000in}}%
\pgfusepath{clip}%
\pgfsetrectcap%
\pgfsetroundjoin%
\pgfsetlinewidth{1.204500pt}%
\definecolor{currentstroke}{rgb}{0.176471,0.192157,0.258824}%
\pgfsetstrokecolor{currentstroke}%
\pgfsetdash{}{0pt}%
\pgfpathmoveto{\pgfqpoint{3.284686in}{3.825545in}}%
\pgfusepath{stroke}%
\end{pgfscope}%
\begin{pgfscope}%
\pgfpathrectangle{\pgfqpoint{0.765000in}{0.660000in}}{\pgfqpoint{4.620000in}{4.620000in}}%
\pgfusepath{clip}%
\pgfsetbuttcap%
\pgfsetmiterjoin%
\definecolor{currentfill}{rgb}{0.176471,0.192157,0.258824}%
\pgfsetfillcolor{currentfill}%
\pgfsetlinewidth{1.003750pt}%
\definecolor{currentstroke}{rgb}{0.176471,0.192157,0.258824}%
\pgfsetstrokecolor{currentstroke}%
\pgfsetdash{}{0pt}%
\pgfsys@defobject{currentmarker}{\pgfqpoint{-0.033333in}{-0.033333in}}{\pgfqpoint{0.033333in}{0.033333in}}{%
\pgfpathmoveto{\pgfqpoint{-0.000000in}{-0.033333in}}%
\pgfpathlineto{\pgfqpoint{0.033333in}{0.033333in}}%
\pgfpathlineto{\pgfqpoint{-0.033333in}{0.033333in}}%
\pgfpathlineto{\pgfqpoint{-0.000000in}{-0.033333in}}%
\pgfpathclose%
\pgfusepath{stroke,fill}%
}%
\begin{pgfscope}%
\pgfsys@transformshift{3.284686in}{3.825545in}%
\pgfsys@useobject{currentmarker}{}%
\end{pgfscope}%
\end{pgfscope}%
\begin{pgfscope}%
\pgfpathrectangle{\pgfqpoint{0.765000in}{0.660000in}}{\pgfqpoint{4.620000in}{4.620000in}}%
\pgfusepath{clip}%
\pgfsetrectcap%
\pgfsetroundjoin%
\pgfsetlinewidth{1.204500pt}%
\definecolor{currentstroke}{rgb}{0.176471,0.192157,0.258824}%
\pgfsetstrokecolor{currentstroke}%
\pgfsetdash{}{0pt}%
\pgfpathmoveto{\pgfqpoint{3.371455in}{3.730738in}}%
\pgfusepath{stroke}%
\end{pgfscope}%
\begin{pgfscope}%
\pgfpathrectangle{\pgfqpoint{0.765000in}{0.660000in}}{\pgfqpoint{4.620000in}{4.620000in}}%
\pgfusepath{clip}%
\pgfsetbuttcap%
\pgfsetmiterjoin%
\definecolor{currentfill}{rgb}{0.176471,0.192157,0.258824}%
\pgfsetfillcolor{currentfill}%
\pgfsetlinewidth{1.003750pt}%
\definecolor{currentstroke}{rgb}{0.176471,0.192157,0.258824}%
\pgfsetstrokecolor{currentstroke}%
\pgfsetdash{}{0pt}%
\pgfsys@defobject{currentmarker}{\pgfqpoint{-0.033333in}{-0.033333in}}{\pgfqpoint{0.033333in}{0.033333in}}{%
\pgfpathmoveto{\pgfqpoint{-0.000000in}{-0.033333in}}%
\pgfpathlineto{\pgfqpoint{0.033333in}{0.033333in}}%
\pgfpathlineto{\pgfqpoint{-0.033333in}{0.033333in}}%
\pgfpathlineto{\pgfqpoint{-0.000000in}{-0.033333in}}%
\pgfpathclose%
\pgfusepath{stroke,fill}%
}%
\begin{pgfscope}%
\pgfsys@transformshift{3.371455in}{3.730738in}%
\pgfsys@useobject{currentmarker}{}%
\end{pgfscope}%
\end{pgfscope}%
\begin{pgfscope}%
\pgfpathrectangle{\pgfqpoint{0.765000in}{0.660000in}}{\pgfqpoint{4.620000in}{4.620000in}}%
\pgfusepath{clip}%
\pgfsetrectcap%
\pgfsetroundjoin%
\pgfsetlinewidth{1.204500pt}%
\definecolor{currentstroke}{rgb}{0.176471,0.192157,0.258824}%
\pgfsetstrokecolor{currentstroke}%
\pgfsetdash{}{0pt}%
\pgfpathmoveto{\pgfqpoint{3.751612in}{3.122098in}}%
\pgfusepath{stroke}%
\end{pgfscope}%
\begin{pgfscope}%
\pgfpathrectangle{\pgfqpoint{0.765000in}{0.660000in}}{\pgfqpoint{4.620000in}{4.620000in}}%
\pgfusepath{clip}%
\pgfsetbuttcap%
\pgfsetmiterjoin%
\definecolor{currentfill}{rgb}{0.176471,0.192157,0.258824}%
\pgfsetfillcolor{currentfill}%
\pgfsetlinewidth{1.003750pt}%
\definecolor{currentstroke}{rgb}{0.176471,0.192157,0.258824}%
\pgfsetstrokecolor{currentstroke}%
\pgfsetdash{}{0pt}%
\pgfsys@defobject{currentmarker}{\pgfqpoint{-0.033333in}{-0.033333in}}{\pgfqpoint{0.033333in}{0.033333in}}{%
\pgfpathmoveto{\pgfqpoint{-0.000000in}{-0.033333in}}%
\pgfpathlineto{\pgfqpoint{0.033333in}{0.033333in}}%
\pgfpathlineto{\pgfqpoint{-0.033333in}{0.033333in}}%
\pgfpathlineto{\pgfqpoint{-0.000000in}{-0.033333in}}%
\pgfpathclose%
\pgfusepath{stroke,fill}%
}%
\begin{pgfscope}%
\pgfsys@transformshift{3.751612in}{3.122098in}%
\pgfsys@useobject{currentmarker}{}%
\end{pgfscope}%
\end{pgfscope}%
\begin{pgfscope}%
\pgfpathrectangle{\pgfqpoint{0.765000in}{0.660000in}}{\pgfqpoint{4.620000in}{4.620000in}}%
\pgfusepath{clip}%
\pgfsetrectcap%
\pgfsetroundjoin%
\pgfsetlinewidth{1.204500pt}%
\definecolor{currentstroke}{rgb}{0.176471,0.192157,0.258824}%
\pgfsetstrokecolor{currentstroke}%
\pgfsetdash{}{0pt}%
\pgfpathmoveto{\pgfqpoint{3.846694in}{2.901213in}}%
\pgfusepath{stroke}%
\end{pgfscope}%
\begin{pgfscope}%
\pgfpathrectangle{\pgfqpoint{0.765000in}{0.660000in}}{\pgfqpoint{4.620000in}{4.620000in}}%
\pgfusepath{clip}%
\pgfsetbuttcap%
\pgfsetmiterjoin%
\definecolor{currentfill}{rgb}{0.176471,0.192157,0.258824}%
\pgfsetfillcolor{currentfill}%
\pgfsetlinewidth{1.003750pt}%
\definecolor{currentstroke}{rgb}{0.176471,0.192157,0.258824}%
\pgfsetstrokecolor{currentstroke}%
\pgfsetdash{}{0pt}%
\pgfsys@defobject{currentmarker}{\pgfqpoint{-0.033333in}{-0.033333in}}{\pgfqpoint{0.033333in}{0.033333in}}{%
\pgfpathmoveto{\pgfqpoint{-0.000000in}{-0.033333in}}%
\pgfpathlineto{\pgfqpoint{0.033333in}{0.033333in}}%
\pgfpathlineto{\pgfqpoint{-0.033333in}{0.033333in}}%
\pgfpathlineto{\pgfqpoint{-0.000000in}{-0.033333in}}%
\pgfpathclose%
\pgfusepath{stroke,fill}%
}%
\begin{pgfscope}%
\pgfsys@transformshift{3.846694in}{2.901213in}%
\pgfsys@useobject{currentmarker}{}%
\end{pgfscope}%
\end{pgfscope}%
\begin{pgfscope}%
\pgfpathrectangle{\pgfqpoint{0.765000in}{0.660000in}}{\pgfqpoint{4.620000in}{4.620000in}}%
\pgfusepath{clip}%
\pgfsetrectcap%
\pgfsetroundjoin%
\pgfsetlinewidth{1.204500pt}%
\definecolor{currentstroke}{rgb}{0.176471,0.192157,0.258824}%
\pgfsetstrokecolor{currentstroke}%
\pgfsetdash{}{0pt}%
\pgfpathmoveto{\pgfqpoint{3.626640in}{3.505332in}}%
\pgfusepath{stroke}%
\end{pgfscope}%
\begin{pgfscope}%
\pgfpathrectangle{\pgfqpoint{0.765000in}{0.660000in}}{\pgfqpoint{4.620000in}{4.620000in}}%
\pgfusepath{clip}%
\pgfsetbuttcap%
\pgfsetmiterjoin%
\definecolor{currentfill}{rgb}{0.176471,0.192157,0.258824}%
\pgfsetfillcolor{currentfill}%
\pgfsetlinewidth{1.003750pt}%
\definecolor{currentstroke}{rgb}{0.176471,0.192157,0.258824}%
\pgfsetstrokecolor{currentstroke}%
\pgfsetdash{}{0pt}%
\pgfsys@defobject{currentmarker}{\pgfqpoint{-0.033333in}{-0.033333in}}{\pgfqpoint{0.033333in}{0.033333in}}{%
\pgfpathmoveto{\pgfqpoint{-0.000000in}{-0.033333in}}%
\pgfpathlineto{\pgfqpoint{0.033333in}{0.033333in}}%
\pgfpathlineto{\pgfqpoint{-0.033333in}{0.033333in}}%
\pgfpathlineto{\pgfqpoint{-0.000000in}{-0.033333in}}%
\pgfpathclose%
\pgfusepath{stroke,fill}%
}%
\begin{pgfscope}%
\pgfsys@transformshift{3.626640in}{3.505332in}%
\pgfsys@useobject{currentmarker}{}%
\end{pgfscope}%
\end{pgfscope}%
\begin{pgfscope}%
\pgfpathrectangle{\pgfqpoint{0.765000in}{0.660000in}}{\pgfqpoint{4.620000in}{4.620000in}}%
\pgfusepath{clip}%
\pgfsetrectcap%
\pgfsetroundjoin%
\pgfsetlinewidth{1.204500pt}%
\definecolor{currentstroke}{rgb}{0.176471,0.192157,0.258824}%
\pgfsetstrokecolor{currentstroke}%
\pgfsetdash{}{0pt}%
\pgfpathmoveto{\pgfqpoint{3.907771in}{3.206128in}}%
\pgfusepath{stroke}%
\end{pgfscope}%
\begin{pgfscope}%
\pgfpathrectangle{\pgfqpoint{0.765000in}{0.660000in}}{\pgfqpoint{4.620000in}{4.620000in}}%
\pgfusepath{clip}%
\pgfsetbuttcap%
\pgfsetmiterjoin%
\definecolor{currentfill}{rgb}{0.176471,0.192157,0.258824}%
\pgfsetfillcolor{currentfill}%
\pgfsetlinewidth{1.003750pt}%
\definecolor{currentstroke}{rgb}{0.176471,0.192157,0.258824}%
\pgfsetstrokecolor{currentstroke}%
\pgfsetdash{}{0pt}%
\pgfsys@defobject{currentmarker}{\pgfqpoint{-0.033333in}{-0.033333in}}{\pgfqpoint{0.033333in}{0.033333in}}{%
\pgfpathmoveto{\pgfqpoint{-0.000000in}{-0.033333in}}%
\pgfpathlineto{\pgfqpoint{0.033333in}{0.033333in}}%
\pgfpathlineto{\pgfqpoint{-0.033333in}{0.033333in}}%
\pgfpathlineto{\pgfqpoint{-0.000000in}{-0.033333in}}%
\pgfpathclose%
\pgfusepath{stroke,fill}%
}%
\begin{pgfscope}%
\pgfsys@transformshift{3.907771in}{3.206128in}%
\pgfsys@useobject{currentmarker}{}%
\end{pgfscope}%
\end{pgfscope}%
\begin{pgfscope}%
\pgfpathrectangle{\pgfqpoint{0.765000in}{0.660000in}}{\pgfqpoint{4.620000in}{4.620000in}}%
\pgfusepath{clip}%
\pgfsetrectcap%
\pgfsetroundjoin%
\pgfsetlinewidth{1.204500pt}%
\definecolor{currentstroke}{rgb}{0.176471,0.192157,0.258824}%
\pgfsetstrokecolor{currentstroke}%
\pgfsetdash{}{0pt}%
\pgfpathmoveto{\pgfqpoint{3.470588in}{3.321701in}}%
\pgfusepath{stroke}%
\end{pgfscope}%
\begin{pgfscope}%
\pgfpathrectangle{\pgfqpoint{0.765000in}{0.660000in}}{\pgfqpoint{4.620000in}{4.620000in}}%
\pgfusepath{clip}%
\pgfsetbuttcap%
\pgfsetmiterjoin%
\definecolor{currentfill}{rgb}{0.176471,0.192157,0.258824}%
\pgfsetfillcolor{currentfill}%
\pgfsetlinewidth{1.003750pt}%
\definecolor{currentstroke}{rgb}{0.176471,0.192157,0.258824}%
\pgfsetstrokecolor{currentstroke}%
\pgfsetdash{}{0pt}%
\pgfsys@defobject{currentmarker}{\pgfqpoint{-0.033333in}{-0.033333in}}{\pgfqpoint{0.033333in}{0.033333in}}{%
\pgfpathmoveto{\pgfqpoint{-0.000000in}{-0.033333in}}%
\pgfpathlineto{\pgfqpoint{0.033333in}{0.033333in}}%
\pgfpathlineto{\pgfqpoint{-0.033333in}{0.033333in}}%
\pgfpathlineto{\pgfqpoint{-0.000000in}{-0.033333in}}%
\pgfpathclose%
\pgfusepath{stroke,fill}%
}%
\begin{pgfscope}%
\pgfsys@transformshift{3.470588in}{3.321701in}%
\pgfsys@useobject{currentmarker}{}%
\end{pgfscope}%
\end{pgfscope}%
\begin{pgfscope}%
\pgfpathrectangle{\pgfqpoint{0.765000in}{0.660000in}}{\pgfqpoint{4.620000in}{4.620000in}}%
\pgfusepath{clip}%
\pgfsetrectcap%
\pgfsetroundjoin%
\pgfsetlinewidth{1.204500pt}%
\definecolor{currentstroke}{rgb}{0.176471,0.192157,0.258824}%
\pgfsetstrokecolor{currentstroke}%
\pgfsetdash{}{0pt}%
\pgfpathmoveto{\pgfqpoint{4.251402in}{3.479248in}}%
\pgfusepath{stroke}%
\end{pgfscope}%
\begin{pgfscope}%
\pgfpathrectangle{\pgfqpoint{0.765000in}{0.660000in}}{\pgfqpoint{4.620000in}{4.620000in}}%
\pgfusepath{clip}%
\pgfsetbuttcap%
\pgfsetmiterjoin%
\definecolor{currentfill}{rgb}{0.176471,0.192157,0.258824}%
\pgfsetfillcolor{currentfill}%
\pgfsetlinewidth{1.003750pt}%
\definecolor{currentstroke}{rgb}{0.176471,0.192157,0.258824}%
\pgfsetstrokecolor{currentstroke}%
\pgfsetdash{}{0pt}%
\pgfsys@defobject{currentmarker}{\pgfqpoint{-0.033333in}{-0.033333in}}{\pgfqpoint{0.033333in}{0.033333in}}{%
\pgfpathmoveto{\pgfqpoint{-0.000000in}{-0.033333in}}%
\pgfpathlineto{\pgfqpoint{0.033333in}{0.033333in}}%
\pgfpathlineto{\pgfqpoint{-0.033333in}{0.033333in}}%
\pgfpathlineto{\pgfqpoint{-0.000000in}{-0.033333in}}%
\pgfpathclose%
\pgfusepath{stroke,fill}%
}%
\begin{pgfscope}%
\pgfsys@transformshift{4.251402in}{3.479248in}%
\pgfsys@useobject{currentmarker}{}%
\end{pgfscope}%
\end{pgfscope}%
\begin{pgfscope}%
\pgfpathrectangle{\pgfqpoint{0.765000in}{0.660000in}}{\pgfqpoint{4.620000in}{4.620000in}}%
\pgfusepath{clip}%
\pgfsetrectcap%
\pgfsetroundjoin%
\pgfsetlinewidth{1.204500pt}%
\definecolor{currentstroke}{rgb}{0.176471,0.192157,0.258824}%
\pgfsetstrokecolor{currentstroke}%
\pgfsetdash{}{0pt}%
\pgfpathmoveto{\pgfqpoint{1.894935in}{3.618149in}}%
\pgfusepath{stroke}%
\end{pgfscope}%
\begin{pgfscope}%
\pgfpathrectangle{\pgfqpoint{0.765000in}{0.660000in}}{\pgfqpoint{4.620000in}{4.620000in}}%
\pgfusepath{clip}%
\pgfsetbuttcap%
\pgfsetmiterjoin%
\definecolor{currentfill}{rgb}{0.176471,0.192157,0.258824}%
\pgfsetfillcolor{currentfill}%
\pgfsetlinewidth{1.003750pt}%
\definecolor{currentstroke}{rgb}{0.176471,0.192157,0.258824}%
\pgfsetstrokecolor{currentstroke}%
\pgfsetdash{}{0pt}%
\pgfsys@defobject{currentmarker}{\pgfqpoint{-0.033333in}{-0.033333in}}{\pgfqpoint{0.033333in}{0.033333in}}{%
\pgfpathmoveto{\pgfqpoint{-0.000000in}{-0.033333in}}%
\pgfpathlineto{\pgfqpoint{0.033333in}{0.033333in}}%
\pgfpathlineto{\pgfqpoint{-0.033333in}{0.033333in}}%
\pgfpathlineto{\pgfqpoint{-0.000000in}{-0.033333in}}%
\pgfpathclose%
\pgfusepath{stroke,fill}%
}%
\begin{pgfscope}%
\pgfsys@transformshift{1.894935in}{3.618149in}%
\pgfsys@useobject{currentmarker}{}%
\end{pgfscope}%
\end{pgfscope}%
\begin{pgfscope}%
\pgfpathrectangle{\pgfqpoint{0.765000in}{0.660000in}}{\pgfqpoint{4.620000in}{4.620000in}}%
\pgfusepath{clip}%
\pgfsetrectcap%
\pgfsetroundjoin%
\pgfsetlinewidth{1.204500pt}%
\definecolor{currentstroke}{rgb}{0.000000,0.000000,0.000000}%
\pgfsetstrokecolor{currentstroke}%
\pgfsetdash{}{0pt}%
\pgfpathmoveto{\pgfqpoint{2.840481in}{3.244213in}}%
\pgfusepath{stroke}%
\end{pgfscope}%
\begin{pgfscope}%
\pgfpathrectangle{\pgfqpoint{0.765000in}{0.660000in}}{\pgfqpoint{4.620000in}{4.620000in}}%
\pgfusepath{clip}%
\pgfsetbuttcap%
\pgfsetroundjoin%
\definecolor{currentfill}{rgb}{0.000000,0.000000,0.000000}%
\pgfsetfillcolor{currentfill}%
\pgfsetlinewidth{1.003750pt}%
\definecolor{currentstroke}{rgb}{0.000000,0.000000,0.000000}%
\pgfsetstrokecolor{currentstroke}%
\pgfsetdash{}{0pt}%
\pgfsys@defobject{currentmarker}{\pgfqpoint{-0.016667in}{-0.016667in}}{\pgfqpoint{0.016667in}{0.016667in}}{%
\pgfpathmoveto{\pgfqpoint{0.000000in}{-0.016667in}}%
\pgfpathcurveto{\pgfqpoint{0.004420in}{-0.016667in}}{\pgfqpoint{0.008660in}{-0.014911in}}{\pgfqpoint{0.011785in}{-0.011785in}}%
\pgfpathcurveto{\pgfqpoint{0.014911in}{-0.008660in}}{\pgfqpoint{0.016667in}{-0.004420in}}{\pgfqpoint{0.016667in}{0.000000in}}%
\pgfpathcurveto{\pgfqpoint{0.016667in}{0.004420in}}{\pgfqpoint{0.014911in}{0.008660in}}{\pgfqpoint{0.011785in}{0.011785in}}%
\pgfpathcurveto{\pgfqpoint{0.008660in}{0.014911in}}{\pgfqpoint{0.004420in}{0.016667in}}{\pgfqpoint{0.000000in}{0.016667in}}%
\pgfpathcurveto{\pgfqpoint{-0.004420in}{0.016667in}}{\pgfqpoint{-0.008660in}{0.014911in}}{\pgfqpoint{-0.011785in}{0.011785in}}%
\pgfpathcurveto{\pgfqpoint{-0.014911in}{0.008660in}}{\pgfqpoint{-0.016667in}{0.004420in}}{\pgfqpoint{-0.016667in}{0.000000in}}%
\pgfpathcurveto{\pgfqpoint{-0.016667in}{-0.004420in}}{\pgfqpoint{-0.014911in}{-0.008660in}}{\pgfqpoint{-0.011785in}{-0.011785in}}%
\pgfpathcurveto{\pgfqpoint{-0.008660in}{-0.014911in}}{\pgfqpoint{-0.004420in}{-0.016667in}}{\pgfqpoint{0.000000in}{-0.016667in}}%
\pgfpathlineto{\pgfqpoint{0.000000in}{-0.016667in}}%
\pgfpathclose%
\pgfusepath{stroke,fill}%
}%
\begin{pgfscope}%
\pgfsys@transformshift{2.840481in}{3.244213in}%
\pgfsys@useobject{currentmarker}{}%
\end{pgfscope}%
\end{pgfscope}%
\begin{pgfscope}%
\pgfpathrectangle{\pgfqpoint{0.765000in}{0.660000in}}{\pgfqpoint{4.620000in}{4.620000in}}%
\pgfusepath{clip}%
\pgfsetrectcap%
\pgfsetroundjoin%
\pgfsetlinewidth{1.204500pt}%
\definecolor{currentstroke}{rgb}{0.000000,0.000000,0.000000}%
\pgfsetstrokecolor{currentstroke}%
\pgfsetdash{}{0pt}%
\pgfpathmoveto{\pgfqpoint{2.840481in}{3.244213in}}%
\pgfpathlineto{\pgfqpoint{3.665320in}{2.857644in}}%
\pgfusepath{stroke}%
\end{pgfscope}%
\begin{pgfscope}%
\pgfpathrectangle{\pgfqpoint{0.765000in}{0.660000in}}{\pgfqpoint{4.620000in}{4.620000in}}%
\pgfusepath{clip}%
\pgfsetbuttcap%
\pgfsetroundjoin%
\pgfsetlinewidth{1.204500pt}%
\definecolor{currentstroke}{rgb}{0.827451,0.827451,0.827451}%
\pgfsetstrokecolor{currentstroke}%
\pgfsetdash{{1.200000pt}{1.980000pt}}{0.000000pt}%
\pgfpathmoveto{\pgfqpoint{2.840481in}{3.244213in}}%
\pgfpathlineto{\pgfqpoint{2.840481in}{3.456103in}}%
\pgfusepath{stroke}%
\end{pgfscope}%
\begin{pgfscope}%
\pgfpathrectangle{\pgfqpoint{0.765000in}{0.660000in}}{\pgfqpoint{4.620000in}{4.620000in}}%
\pgfusepath{clip}%
\pgfsetrectcap%
\pgfsetroundjoin%
\pgfsetlinewidth{1.204500pt}%
\definecolor{currentstroke}{rgb}{0.000000,0.000000,0.000000}%
\pgfsetstrokecolor{currentstroke}%
\pgfsetdash{}{0pt}%
\pgfpathmoveto{\pgfqpoint{2.840481in}{3.244213in}}%
\pgfpathlineto{\pgfqpoint{2.144233in}{2.917910in}}%
\pgfusepath{stroke}%
\end{pgfscope}%
\begin{pgfscope}%
\pgfpathrectangle{\pgfqpoint{0.765000in}{0.660000in}}{\pgfqpoint{4.620000in}{4.620000in}}%
\pgfusepath{clip}%
\pgfsetrectcap%
\pgfsetroundjoin%
\pgfsetlinewidth{1.204500pt}%
\definecolor{currentstroke}{rgb}{0.000000,0.000000,0.000000}%
\pgfsetstrokecolor{currentstroke}%
\pgfsetdash{}{0pt}%
\pgfpathmoveto{\pgfqpoint{3.665320in}{2.857644in}}%
\pgfusepath{stroke}%
\end{pgfscope}%
\begin{pgfscope}%
\pgfpathrectangle{\pgfqpoint{0.765000in}{0.660000in}}{\pgfqpoint{4.620000in}{4.620000in}}%
\pgfusepath{clip}%
\pgfsetbuttcap%
\pgfsetroundjoin%
\definecolor{currentfill}{rgb}{0.000000,0.000000,0.000000}%
\pgfsetfillcolor{currentfill}%
\pgfsetlinewidth{1.003750pt}%
\definecolor{currentstroke}{rgb}{0.000000,0.000000,0.000000}%
\pgfsetstrokecolor{currentstroke}%
\pgfsetdash{}{0pt}%
\pgfsys@defobject{currentmarker}{\pgfqpoint{-0.016667in}{-0.016667in}}{\pgfqpoint{0.016667in}{0.016667in}}{%
\pgfpathmoveto{\pgfqpoint{0.000000in}{-0.016667in}}%
\pgfpathcurveto{\pgfqpoint{0.004420in}{-0.016667in}}{\pgfqpoint{0.008660in}{-0.014911in}}{\pgfqpoint{0.011785in}{-0.011785in}}%
\pgfpathcurveto{\pgfqpoint{0.014911in}{-0.008660in}}{\pgfqpoint{0.016667in}{-0.004420in}}{\pgfqpoint{0.016667in}{0.000000in}}%
\pgfpathcurveto{\pgfqpoint{0.016667in}{0.004420in}}{\pgfqpoint{0.014911in}{0.008660in}}{\pgfqpoint{0.011785in}{0.011785in}}%
\pgfpathcurveto{\pgfqpoint{0.008660in}{0.014911in}}{\pgfqpoint{0.004420in}{0.016667in}}{\pgfqpoint{0.000000in}{0.016667in}}%
\pgfpathcurveto{\pgfqpoint{-0.004420in}{0.016667in}}{\pgfqpoint{-0.008660in}{0.014911in}}{\pgfqpoint{-0.011785in}{0.011785in}}%
\pgfpathcurveto{\pgfqpoint{-0.014911in}{0.008660in}}{\pgfqpoint{-0.016667in}{0.004420in}}{\pgfqpoint{-0.016667in}{0.000000in}}%
\pgfpathcurveto{\pgfqpoint{-0.016667in}{-0.004420in}}{\pgfqpoint{-0.014911in}{-0.008660in}}{\pgfqpoint{-0.011785in}{-0.011785in}}%
\pgfpathcurveto{\pgfqpoint{-0.008660in}{-0.014911in}}{\pgfqpoint{-0.004420in}{-0.016667in}}{\pgfqpoint{0.000000in}{-0.016667in}}%
\pgfpathlineto{\pgfqpoint{0.000000in}{-0.016667in}}%
\pgfpathclose%
\pgfusepath{stroke,fill}%
}%
\begin{pgfscope}%
\pgfsys@transformshift{3.665320in}{2.857644in}%
\pgfsys@useobject{currentmarker}{}%
\end{pgfscope}%
\end{pgfscope}%
\begin{pgfscope}%
\pgfpathrectangle{\pgfqpoint{0.765000in}{0.660000in}}{\pgfqpoint{4.620000in}{4.620000in}}%
\pgfusepath{clip}%
\pgfsetrectcap%
\pgfsetroundjoin%
\pgfsetlinewidth{1.204500pt}%
\definecolor{currentstroke}{rgb}{0.000000,0.000000,0.000000}%
\pgfsetstrokecolor{currentstroke}%
\pgfsetdash{}{0pt}%
\pgfpathmoveto{\pgfqpoint{3.665320in}{2.857644in}}%
\pgfpathlineto{\pgfqpoint{2.840481in}{3.244213in}}%
\pgfusepath{stroke}%
\end{pgfscope}%
\begin{pgfscope}%
\pgfpathrectangle{\pgfqpoint{0.765000in}{0.660000in}}{\pgfqpoint{4.620000in}{4.620000in}}%
\pgfusepath{clip}%
\pgfsetrectcap%
\pgfsetroundjoin%
\pgfsetlinewidth{1.204500pt}%
\definecolor{currentstroke}{rgb}{0.000000,0.000000,0.000000}%
\pgfsetstrokecolor{currentstroke}%
\pgfsetdash{}{0pt}%
\pgfpathmoveto{\pgfqpoint{3.665320in}{2.857644in}}%
\pgfpathlineto{\pgfqpoint{3.846694in}{2.942647in}}%
\pgfusepath{stroke}%
\end{pgfscope}%
\begin{pgfscope}%
\pgfpathrectangle{\pgfqpoint{0.765000in}{0.660000in}}{\pgfqpoint{4.620000in}{4.620000in}}%
\pgfusepath{clip}%
\pgfsetbuttcap%
\pgfsetroundjoin%
\pgfsetlinewidth{1.204500pt}%
\definecolor{currentstroke}{rgb}{0.827451,0.827451,0.827451}%
\pgfsetstrokecolor{currentstroke}%
\pgfsetdash{{1.200000pt}{1.980000pt}}{0.000000pt}%
\pgfpathmoveto{\pgfqpoint{3.665320in}{2.857644in}}%
\pgfpathlineto{\pgfqpoint{3.665320in}{0.650000in}}%
\pgfusepath{stroke}%
\end{pgfscope}%
\begin{pgfscope}%
\pgfpathrectangle{\pgfqpoint{0.765000in}{0.660000in}}{\pgfqpoint{4.620000in}{4.620000in}}%
\pgfusepath{clip}%
\pgfsetrectcap%
\pgfsetroundjoin%
\pgfsetlinewidth{1.204500pt}%
\definecolor{currentstroke}{rgb}{0.000000,0.000000,0.000000}%
\pgfsetstrokecolor{currentstroke}%
\pgfsetdash{}{0pt}%
\pgfpathmoveto{\pgfqpoint{2.840481in}{3.456103in}}%
\pgfusepath{stroke}%
\end{pgfscope}%
\begin{pgfscope}%
\pgfpathrectangle{\pgfqpoint{0.765000in}{0.660000in}}{\pgfqpoint{4.620000in}{4.620000in}}%
\pgfusepath{clip}%
\pgfsetbuttcap%
\pgfsetroundjoin%
\definecolor{currentfill}{rgb}{0.000000,0.000000,0.000000}%
\pgfsetfillcolor{currentfill}%
\pgfsetlinewidth{1.003750pt}%
\definecolor{currentstroke}{rgb}{0.000000,0.000000,0.000000}%
\pgfsetstrokecolor{currentstroke}%
\pgfsetdash{}{0pt}%
\pgfsys@defobject{currentmarker}{\pgfqpoint{-0.016667in}{-0.016667in}}{\pgfqpoint{0.016667in}{0.016667in}}{%
\pgfpathmoveto{\pgfqpoint{0.000000in}{-0.016667in}}%
\pgfpathcurveto{\pgfqpoint{0.004420in}{-0.016667in}}{\pgfqpoint{0.008660in}{-0.014911in}}{\pgfqpoint{0.011785in}{-0.011785in}}%
\pgfpathcurveto{\pgfqpoint{0.014911in}{-0.008660in}}{\pgfqpoint{0.016667in}{-0.004420in}}{\pgfqpoint{0.016667in}{0.000000in}}%
\pgfpathcurveto{\pgfqpoint{0.016667in}{0.004420in}}{\pgfqpoint{0.014911in}{0.008660in}}{\pgfqpoint{0.011785in}{0.011785in}}%
\pgfpathcurveto{\pgfqpoint{0.008660in}{0.014911in}}{\pgfqpoint{0.004420in}{0.016667in}}{\pgfqpoint{0.000000in}{0.016667in}}%
\pgfpathcurveto{\pgfqpoint{-0.004420in}{0.016667in}}{\pgfqpoint{-0.008660in}{0.014911in}}{\pgfqpoint{-0.011785in}{0.011785in}}%
\pgfpathcurveto{\pgfqpoint{-0.014911in}{0.008660in}}{\pgfqpoint{-0.016667in}{0.004420in}}{\pgfqpoint{-0.016667in}{0.000000in}}%
\pgfpathcurveto{\pgfqpoint{-0.016667in}{-0.004420in}}{\pgfqpoint{-0.014911in}{-0.008660in}}{\pgfqpoint{-0.011785in}{-0.011785in}}%
\pgfpathcurveto{\pgfqpoint{-0.008660in}{-0.014911in}}{\pgfqpoint{-0.004420in}{-0.016667in}}{\pgfqpoint{0.000000in}{-0.016667in}}%
\pgfpathlineto{\pgfqpoint{0.000000in}{-0.016667in}}%
\pgfpathclose%
\pgfusepath{stroke,fill}%
}%
\begin{pgfscope}%
\pgfsys@transformshift{2.840481in}{3.456103in}%
\pgfsys@useobject{currentmarker}{}%
\end{pgfscope}%
\end{pgfscope}%
\begin{pgfscope}%
\pgfpathrectangle{\pgfqpoint{0.765000in}{0.660000in}}{\pgfqpoint{4.620000in}{4.620000in}}%
\pgfusepath{clip}%
\pgfsetbuttcap%
\pgfsetroundjoin%
\pgfsetlinewidth{1.204500pt}%
\definecolor{currentstroke}{rgb}{0.827451,0.827451,0.827451}%
\pgfsetstrokecolor{currentstroke}%
\pgfsetdash{{1.200000pt}{1.980000pt}}{0.000000pt}%
\pgfpathmoveto{\pgfqpoint{2.840481in}{3.456103in}}%
\pgfpathlineto{\pgfqpoint{2.840481in}{3.244213in}}%
\pgfusepath{stroke}%
\end{pgfscope}%
\begin{pgfscope}%
\pgfpathrectangle{\pgfqpoint{0.765000in}{0.660000in}}{\pgfqpoint{4.620000in}{4.620000in}}%
\pgfusepath{clip}%
\pgfsetbuttcap%
\pgfsetroundjoin%
\pgfsetlinewidth{1.204500pt}%
\definecolor{currentstroke}{rgb}{0.827451,0.827451,0.827451}%
\pgfsetstrokecolor{currentstroke}%
\pgfsetdash{{1.200000pt}{1.980000pt}}{0.000000pt}%
\pgfpathmoveto{\pgfqpoint{2.840481in}{3.456103in}}%
\pgfpathlineto{\pgfqpoint{3.047359in}{3.553059in}}%
\pgfusepath{stroke}%
\end{pgfscope}%
\begin{pgfscope}%
\pgfpathrectangle{\pgfqpoint{0.765000in}{0.660000in}}{\pgfqpoint{4.620000in}{4.620000in}}%
\pgfusepath{clip}%
\pgfsetbuttcap%
\pgfsetroundjoin%
\pgfsetlinewidth{1.204500pt}%
\definecolor{currentstroke}{rgb}{0.827451,0.827451,0.827451}%
\pgfsetstrokecolor{currentstroke}%
\pgfsetdash{{1.200000pt}{1.980000pt}}{0.000000pt}%
\pgfpathmoveto{\pgfqpoint{2.840481in}{3.456103in}}%
\pgfpathlineto{\pgfqpoint{3.047359in}{3.553061in}}%
\pgfusepath{stroke}%
\end{pgfscope}%
\begin{pgfscope}%
\pgfpathrectangle{\pgfqpoint{0.765000in}{0.660000in}}{\pgfqpoint{4.620000in}{4.620000in}}%
\pgfusepath{clip}%
\pgfsetbuttcap%
\pgfsetroundjoin%
\pgfsetlinewidth{1.204500pt}%
\definecolor{currentstroke}{rgb}{0.827451,0.827451,0.827451}%
\pgfsetstrokecolor{currentstroke}%
\pgfsetdash{{1.200000pt}{1.980000pt}}{0.000000pt}%
\pgfpathmoveto{\pgfqpoint{2.840481in}{3.456103in}}%
\pgfpathlineto{\pgfqpoint{5.395000in}{4.653303in}}%
\pgfusepath{stroke}%
\end{pgfscope}%
\begin{pgfscope}%
\pgfpathrectangle{\pgfqpoint{0.765000in}{0.660000in}}{\pgfqpoint{4.620000in}{4.620000in}}%
\pgfusepath{clip}%
\pgfsetbuttcap%
\pgfsetroundjoin%
\pgfsetlinewidth{1.204500pt}%
\definecolor{currentstroke}{rgb}{0.827451,0.827451,0.827451}%
\pgfsetstrokecolor{currentstroke}%
\pgfsetdash{{1.200000pt}{1.980000pt}}{0.000000pt}%
\pgfpathmoveto{\pgfqpoint{2.840481in}{3.456103in}}%
\pgfpathlineto{\pgfqpoint{0.755000in}{4.433484in}}%
\pgfusepath{stroke}%
\end{pgfscope}%
\begin{pgfscope}%
\pgfpathrectangle{\pgfqpoint{0.765000in}{0.660000in}}{\pgfqpoint{4.620000in}{4.620000in}}%
\pgfusepath{clip}%
\pgfsetrectcap%
\pgfsetroundjoin%
\pgfsetlinewidth{1.204500pt}%
\definecolor{currentstroke}{rgb}{0.000000,0.000000,0.000000}%
\pgfsetstrokecolor{currentstroke}%
\pgfsetdash{}{0pt}%
\pgfpathmoveto{\pgfqpoint{3.861519in}{3.171494in}}%
\pgfusepath{stroke}%
\end{pgfscope}%
\begin{pgfscope}%
\pgfpathrectangle{\pgfqpoint{0.765000in}{0.660000in}}{\pgfqpoint{4.620000in}{4.620000in}}%
\pgfusepath{clip}%
\pgfsetbuttcap%
\pgfsetroundjoin%
\definecolor{currentfill}{rgb}{0.000000,0.000000,0.000000}%
\pgfsetfillcolor{currentfill}%
\pgfsetlinewidth{1.003750pt}%
\definecolor{currentstroke}{rgb}{0.000000,0.000000,0.000000}%
\pgfsetstrokecolor{currentstroke}%
\pgfsetdash{}{0pt}%
\pgfsys@defobject{currentmarker}{\pgfqpoint{-0.016667in}{-0.016667in}}{\pgfqpoint{0.016667in}{0.016667in}}{%
\pgfpathmoveto{\pgfqpoint{0.000000in}{-0.016667in}}%
\pgfpathcurveto{\pgfqpoint{0.004420in}{-0.016667in}}{\pgfqpoint{0.008660in}{-0.014911in}}{\pgfqpoint{0.011785in}{-0.011785in}}%
\pgfpathcurveto{\pgfqpoint{0.014911in}{-0.008660in}}{\pgfqpoint{0.016667in}{-0.004420in}}{\pgfqpoint{0.016667in}{0.000000in}}%
\pgfpathcurveto{\pgfqpoint{0.016667in}{0.004420in}}{\pgfqpoint{0.014911in}{0.008660in}}{\pgfqpoint{0.011785in}{0.011785in}}%
\pgfpathcurveto{\pgfqpoint{0.008660in}{0.014911in}}{\pgfqpoint{0.004420in}{0.016667in}}{\pgfqpoint{0.000000in}{0.016667in}}%
\pgfpathcurveto{\pgfqpoint{-0.004420in}{0.016667in}}{\pgfqpoint{-0.008660in}{0.014911in}}{\pgfqpoint{-0.011785in}{0.011785in}}%
\pgfpathcurveto{\pgfqpoint{-0.014911in}{0.008660in}}{\pgfqpoint{-0.016667in}{0.004420in}}{\pgfqpoint{-0.016667in}{0.000000in}}%
\pgfpathcurveto{\pgfqpoint{-0.016667in}{-0.004420in}}{\pgfqpoint{-0.014911in}{-0.008660in}}{\pgfqpoint{-0.011785in}{-0.011785in}}%
\pgfpathcurveto{\pgfqpoint{-0.008660in}{-0.014911in}}{\pgfqpoint{-0.004420in}{-0.016667in}}{\pgfqpoint{0.000000in}{-0.016667in}}%
\pgfpathlineto{\pgfqpoint{0.000000in}{-0.016667in}}%
\pgfpathclose%
\pgfusepath{stroke,fill}%
}%
\begin{pgfscope}%
\pgfsys@transformshift{3.861519in}{3.171494in}%
\pgfsys@useobject{currentmarker}{}%
\end{pgfscope}%
\end{pgfscope}%
\begin{pgfscope}%
\pgfpathrectangle{\pgfqpoint{0.765000in}{0.660000in}}{\pgfqpoint{4.620000in}{4.620000in}}%
\pgfusepath{clip}%
\pgfsetrectcap%
\pgfsetroundjoin%
\pgfsetlinewidth{1.204500pt}%
\definecolor{currentstroke}{rgb}{0.000000,0.000000,0.000000}%
\pgfsetstrokecolor{currentstroke}%
\pgfsetdash{}{0pt}%
\pgfpathmoveto{\pgfqpoint{3.861519in}{3.171494in}}%
\pgfpathlineto{\pgfqpoint{3.861519in}{2.963491in}}%
\pgfusepath{stroke}%
\end{pgfscope}%
\begin{pgfscope}%
\pgfpathrectangle{\pgfqpoint{0.765000in}{0.660000in}}{\pgfqpoint{4.620000in}{4.620000in}}%
\pgfusepath{clip}%
\pgfsetbuttcap%
\pgfsetroundjoin%
\pgfsetlinewidth{1.204500pt}%
\definecolor{currentstroke}{rgb}{0.827451,0.827451,0.827451}%
\pgfsetstrokecolor{currentstroke}%
\pgfsetdash{{1.200000pt}{1.980000pt}}{0.000000pt}%
\pgfpathmoveto{\pgfqpoint{3.861519in}{3.171494in}}%
\pgfpathlineto{\pgfqpoint{3.047359in}{3.553059in}}%
\pgfusepath{stroke}%
\end{pgfscope}%
\begin{pgfscope}%
\pgfpathrectangle{\pgfqpoint{0.765000in}{0.660000in}}{\pgfqpoint{4.620000in}{4.620000in}}%
\pgfusepath{clip}%
\pgfsetbuttcap%
\pgfsetroundjoin%
\pgfsetlinewidth{1.204500pt}%
\definecolor{currentstroke}{rgb}{0.827451,0.827451,0.827451}%
\pgfsetstrokecolor{currentstroke}%
\pgfsetdash{{1.200000pt}{1.980000pt}}{0.000000pt}%
\pgfpathmoveto{\pgfqpoint{3.861519in}{3.171494in}}%
\pgfpathlineto{\pgfqpoint{3.047359in}{3.553061in}}%
\pgfusepath{stroke}%
\end{pgfscope}%
\begin{pgfscope}%
\pgfpathrectangle{\pgfqpoint{0.765000in}{0.660000in}}{\pgfqpoint{4.620000in}{4.620000in}}%
\pgfusepath{clip}%
\pgfsetbuttcap%
\pgfsetroundjoin%
\pgfsetlinewidth{1.204500pt}%
\definecolor{currentstroke}{rgb}{0.827451,0.827451,0.827451}%
\pgfsetstrokecolor{currentstroke}%
\pgfsetdash{{1.200000pt}{1.980000pt}}{0.000000pt}%
\pgfpathmoveto{\pgfqpoint{3.861519in}{3.171494in}}%
\pgfpathlineto{\pgfqpoint{0.755000in}{4.627395in}}%
\pgfusepath{stroke}%
\end{pgfscope}%
\begin{pgfscope}%
\pgfpathrectangle{\pgfqpoint{0.765000in}{0.660000in}}{\pgfqpoint{4.620000in}{4.620000in}}%
\pgfusepath{clip}%
\pgfsetrectcap%
\pgfsetroundjoin%
\pgfsetlinewidth{1.204500pt}%
\definecolor{currentstroke}{rgb}{0.000000,0.000000,0.000000}%
\pgfsetstrokecolor{currentstroke}%
\pgfsetdash{}{0pt}%
\pgfpathmoveto{\pgfqpoint{3.861519in}{3.171494in}}%
\pgfpathlineto{\pgfqpoint{5.395000in}{3.890175in}}%
\pgfusepath{stroke}%
\end{pgfscope}%
\begin{pgfscope}%
\pgfpathrectangle{\pgfqpoint{0.765000in}{0.660000in}}{\pgfqpoint{4.620000in}{4.620000in}}%
\pgfusepath{clip}%
\pgfsetrectcap%
\pgfsetroundjoin%
\pgfsetlinewidth{1.204500pt}%
\definecolor{currentstroke}{rgb}{0.000000,0.000000,0.000000}%
\pgfsetstrokecolor{currentstroke}%
\pgfsetdash{}{0pt}%
\pgfpathmoveto{\pgfqpoint{3.861519in}{2.963491in}}%
\pgfusepath{stroke}%
\end{pgfscope}%
\begin{pgfscope}%
\pgfpathrectangle{\pgfqpoint{0.765000in}{0.660000in}}{\pgfqpoint{4.620000in}{4.620000in}}%
\pgfusepath{clip}%
\pgfsetbuttcap%
\pgfsetroundjoin%
\definecolor{currentfill}{rgb}{0.000000,0.000000,0.000000}%
\pgfsetfillcolor{currentfill}%
\pgfsetlinewidth{1.003750pt}%
\definecolor{currentstroke}{rgb}{0.000000,0.000000,0.000000}%
\pgfsetstrokecolor{currentstroke}%
\pgfsetdash{}{0pt}%
\pgfsys@defobject{currentmarker}{\pgfqpoint{-0.016667in}{-0.016667in}}{\pgfqpoint{0.016667in}{0.016667in}}{%
\pgfpathmoveto{\pgfqpoint{0.000000in}{-0.016667in}}%
\pgfpathcurveto{\pgfqpoint{0.004420in}{-0.016667in}}{\pgfqpoint{0.008660in}{-0.014911in}}{\pgfqpoint{0.011785in}{-0.011785in}}%
\pgfpathcurveto{\pgfqpoint{0.014911in}{-0.008660in}}{\pgfqpoint{0.016667in}{-0.004420in}}{\pgfqpoint{0.016667in}{0.000000in}}%
\pgfpathcurveto{\pgfqpoint{0.016667in}{0.004420in}}{\pgfqpoint{0.014911in}{0.008660in}}{\pgfqpoint{0.011785in}{0.011785in}}%
\pgfpathcurveto{\pgfqpoint{0.008660in}{0.014911in}}{\pgfqpoint{0.004420in}{0.016667in}}{\pgfqpoint{0.000000in}{0.016667in}}%
\pgfpathcurveto{\pgfqpoint{-0.004420in}{0.016667in}}{\pgfqpoint{-0.008660in}{0.014911in}}{\pgfqpoint{-0.011785in}{0.011785in}}%
\pgfpathcurveto{\pgfqpoint{-0.014911in}{0.008660in}}{\pgfqpoint{-0.016667in}{0.004420in}}{\pgfqpoint{-0.016667in}{0.000000in}}%
\pgfpathcurveto{\pgfqpoint{-0.016667in}{-0.004420in}}{\pgfqpoint{-0.014911in}{-0.008660in}}{\pgfqpoint{-0.011785in}{-0.011785in}}%
\pgfpathcurveto{\pgfqpoint{-0.008660in}{-0.014911in}}{\pgfqpoint{-0.004420in}{-0.016667in}}{\pgfqpoint{0.000000in}{-0.016667in}}%
\pgfpathlineto{\pgfqpoint{0.000000in}{-0.016667in}}%
\pgfpathclose%
\pgfusepath{stroke,fill}%
}%
\begin{pgfscope}%
\pgfsys@transformshift{3.861519in}{2.963491in}%
\pgfsys@useobject{currentmarker}{}%
\end{pgfscope}%
\end{pgfscope}%
\begin{pgfscope}%
\pgfpathrectangle{\pgfqpoint{0.765000in}{0.660000in}}{\pgfqpoint{4.620000in}{4.620000in}}%
\pgfusepath{clip}%
\pgfsetrectcap%
\pgfsetroundjoin%
\pgfsetlinewidth{1.204500pt}%
\definecolor{currentstroke}{rgb}{0.000000,0.000000,0.000000}%
\pgfsetstrokecolor{currentstroke}%
\pgfsetdash{}{0pt}%
\pgfpathmoveto{\pgfqpoint{3.861519in}{2.963491in}}%
\pgfpathlineto{\pgfqpoint{3.861519in}{3.171494in}}%
\pgfusepath{stroke}%
\end{pgfscope}%
\begin{pgfscope}%
\pgfpathrectangle{\pgfqpoint{0.765000in}{0.660000in}}{\pgfqpoint{4.620000in}{4.620000in}}%
\pgfusepath{clip}%
\pgfsetrectcap%
\pgfsetroundjoin%
\pgfsetlinewidth{1.204500pt}%
\definecolor{currentstroke}{rgb}{0.000000,0.000000,0.000000}%
\pgfsetstrokecolor{currentstroke}%
\pgfsetdash{}{0pt}%
\pgfpathmoveto{\pgfqpoint{3.861519in}{2.963491in}}%
\pgfpathlineto{\pgfqpoint{3.846694in}{2.942647in}}%
\pgfusepath{stroke}%
\end{pgfscope}%
\begin{pgfscope}%
\pgfpathrectangle{\pgfqpoint{0.765000in}{0.660000in}}{\pgfqpoint{4.620000in}{4.620000in}}%
\pgfusepath{clip}%
\pgfsetbuttcap%
\pgfsetroundjoin%
\pgfsetlinewidth{1.204500pt}%
\definecolor{currentstroke}{rgb}{0.827451,0.827451,0.827451}%
\pgfsetstrokecolor{currentstroke}%
\pgfsetdash{{1.200000pt}{1.980000pt}}{0.000000pt}%
\pgfpathmoveto{\pgfqpoint{3.861519in}{2.963491in}}%
\pgfpathlineto{\pgfqpoint{5.395000in}{3.682172in}}%
\pgfusepath{stroke}%
\end{pgfscope}%
\begin{pgfscope}%
\pgfpathrectangle{\pgfqpoint{0.765000in}{0.660000in}}{\pgfqpoint{4.620000in}{4.620000in}}%
\pgfusepath{clip}%
\pgfsetrectcap%
\pgfsetroundjoin%
\pgfsetlinewidth{1.204500pt}%
\definecolor{currentstroke}{rgb}{0.000000,0.000000,0.000000}%
\pgfsetstrokecolor{currentstroke}%
\pgfsetdash{}{0pt}%
\pgfpathmoveto{\pgfqpoint{3.846694in}{2.942647in}}%
\pgfusepath{stroke}%
\end{pgfscope}%
\begin{pgfscope}%
\pgfpathrectangle{\pgfqpoint{0.765000in}{0.660000in}}{\pgfqpoint{4.620000in}{4.620000in}}%
\pgfusepath{clip}%
\pgfsetbuttcap%
\pgfsetroundjoin%
\definecolor{currentfill}{rgb}{0.000000,0.000000,0.000000}%
\pgfsetfillcolor{currentfill}%
\pgfsetlinewidth{1.003750pt}%
\definecolor{currentstroke}{rgb}{0.000000,0.000000,0.000000}%
\pgfsetstrokecolor{currentstroke}%
\pgfsetdash{}{0pt}%
\pgfsys@defobject{currentmarker}{\pgfqpoint{-0.016667in}{-0.016667in}}{\pgfqpoint{0.016667in}{0.016667in}}{%
\pgfpathmoveto{\pgfqpoint{0.000000in}{-0.016667in}}%
\pgfpathcurveto{\pgfqpoint{0.004420in}{-0.016667in}}{\pgfqpoint{0.008660in}{-0.014911in}}{\pgfqpoint{0.011785in}{-0.011785in}}%
\pgfpathcurveto{\pgfqpoint{0.014911in}{-0.008660in}}{\pgfqpoint{0.016667in}{-0.004420in}}{\pgfqpoint{0.016667in}{0.000000in}}%
\pgfpathcurveto{\pgfqpoint{0.016667in}{0.004420in}}{\pgfqpoint{0.014911in}{0.008660in}}{\pgfqpoint{0.011785in}{0.011785in}}%
\pgfpathcurveto{\pgfqpoint{0.008660in}{0.014911in}}{\pgfqpoint{0.004420in}{0.016667in}}{\pgfqpoint{0.000000in}{0.016667in}}%
\pgfpathcurveto{\pgfqpoint{-0.004420in}{0.016667in}}{\pgfqpoint{-0.008660in}{0.014911in}}{\pgfqpoint{-0.011785in}{0.011785in}}%
\pgfpathcurveto{\pgfqpoint{-0.014911in}{0.008660in}}{\pgfqpoint{-0.016667in}{0.004420in}}{\pgfqpoint{-0.016667in}{0.000000in}}%
\pgfpathcurveto{\pgfqpoint{-0.016667in}{-0.004420in}}{\pgfqpoint{-0.014911in}{-0.008660in}}{\pgfqpoint{-0.011785in}{-0.011785in}}%
\pgfpathcurveto{\pgfqpoint{-0.008660in}{-0.014911in}}{\pgfqpoint{-0.004420in}{-0.016667in}}{\pgfqpoint{0.000000in}{-0.016667in}}%
\pgfpathlineto{\pgfqpoint{0.000000in}{-0.016667in}}%
\pgfpathclose%
\pgfusepath{stroke,fill}%
}%
\begin{pgfscope}%
\pgfsys@transformshift{3.846694in}{2.942647in}%
\pgfsys@useobject{currentmarker}{}%
\end{pgfscope}%
\end{pgfscope}%
\begin{pgfscope}%
\pgfpathrectangle{\pgfqpoint{0.765000in}{0.660000in}}{\pgfqpoint{4.620000in}{4.620000in}}%
\pgfusepath{clip}%
\pgfsetrectcap%
\pgfsetroundjoin%
\pgfsetlinewidth{1.204500pt}%
\definecolor{currentstroke}{rgb}{0.000000,0.000000,0.000000}%
\pgfsetstrokecolor{currentstroke}%
\pgfsetdash{}{0pt}%
\pgfpathmoveto{\pgfqpoint{3.846694in}{2.942647in}}%
\pgfpathlineto{\pgfqpoint{3.665320in}{2.857644in}}%
\pgfusepath{stroke}%
\end{pgfscope}%
\begin{pgfscope}%
\pgfpathrectangle{\pgfqpoint{0.765000in}{0.660000in}}{\pgfqpoint{4.620000in}{4.620000in}}%
\pgfusepath{clip}%
\pgfsetrectcap%
\pgfsetroundjoin%
\pgfsetlinewidth{1.204500pt}%
\definecolor{currentstroke}{rgb}{0.000000,0.000000,0.000000}%
\pgfsetstrokecolor{currentstroke}%
\pgfsetdash{}{0pt}%
\pgfpathmoveto{\pgfqpoint{3.846694in}{2.942647in}}%
\pgfpathlineto{\pgfqpoint{3.861519in}{2.963491in}}%
\pgfusepath{stroke}%
\end{pgfscope}%
\begin{pgfscope}%
\pgfpathrectangle{\pgfqpoint{0.765000in}{0.660000in}}{\pgfqpoint{4.620000in}{4.620000in}}%
\pgfusepath{clip}%
\pgfsetbuttcap%
\pgfsetroundjoin%
\pgfsetlinewidth{1.204500pt}%
\definecolor{currentstroke}{rgb}{0.827451,0.827451,0.827451}%
\pgfsetstrokecolor{currentstroke}%
\pgfsetdash{{1.200000pt}{1.980000pt}}{0.000000pt}%
\pgfpathmoveto{\pgfqpoint{3.846694in}{2.942647in}}%
\pgfpathlineto{\pgfqpoint{3.846694in}{0.650000in}}%
\pgfusepath{stroke}%
\end{pgfscope}%
\begin{pgfscope}%
\pgfpathrectangle{\pgfqpoint{0.765000in}{0.660000in}}{\pgfqpoint{4.620000in}{4.620000in}}%
\pgfusepath{clip}%
\pgfsetrectcap%
\pgfsetroundjoin%
\pgfsetlinewidth{1.204500pt}%
\definecolor{currentstroke}{rgb}{0.000000,0.000000,0.000000}%
\pgfsetstrokecolor{currentstroke}%
\pgfsetdash{}{0pt}%
\pgfpathmoveto{\pgfqpoint{3.047359in}{3.553059in}}%
\pgfusepath{stroke}%
\end{pgfscope}%
\begin{pgfscope}%
\pgfpathrectangle{\pgfqpoint{0.765000in}{0.660000in}}{\pgfqpoint{4.620000in}{4.620000in}}%
\pgfusepath{clip}%
\pgfsetbuttcap%
\pgfsetroundjoin%
\definecolor{currentfill}{rgb}{0.000000,0.000000,0.000000}%
\pgfsetfillcolor{currentfill}%
\pgfsetlinewidth{1.003750pt}%
\definecolor{currentstroke}{rgb}{0.000000,0.000000,0.000000}%
\pgfsetstrokecolor{currentstroke}%
\pgfsetdash{}{0pt}%
\pgfsys@defobject{currentmarker}{\pgfqpoint{-0.016667in}{-0.016667in}}{\pgfqpoint{0.016667in}{0.016667in}}{%
\pgfpathmoveto{\pgfqpoint{0.000000in}{-0.016667in}}%
\pgfpathcurveto{\pgfqpoint{0.004420in}{-0.016667in}}{\pgfqpoint{0.008660in}{-0.014911in}}{\pgfqpoint{0.011785in}{-0.011785in}}%
\pgfpathcurveto{\pgfqpoint{0.014911in}{-0.008660in}}{\pgfqpoint{0.016667in}{-0.004420in}}{\pgfqpoint{0.016667in}{0.000000in}}%
\pgfpathcurveto{\pgfqpoint{0.016667in}{0.004420in}}{\pgfqpoint{0.014911in}{0.008660in}}{\pgfqpoint{0.011785in}{0.011785in}}%
\pgfpathcurveto{\pgfqpoint{0.008660in}{0.014911in}}{\pgfqpoint{0.004420in}{0.016667in}}{\pgfqpoint{0.000000in}{0.016667in}}%
\pgfpathcurveto{\pgfqpoint{-0.004420in}{0.016667in}}{\pgfqpoint{-0.008660in}{0.014911in}}{\pgfqpoint{-0.011785in}{0.011785in}}%
\pgfpathcurveto{\pgfqpoint{-0.014911in}{0.008660in}}{\pgfqpoint{-0.016667in}{0.004420in}}{\pgfqpoint{-0.016667in}{0.000000in}}%
\pgfpathcurveto{\pgfqpoint{-0.016667in}{-0.004420in}}{\pgfqpoint{-0.014911in}{-0.008660in}}{\pgfqpoint{-0.011785in}{-0.011785in}}%
\pgfpathcurveto{\pgfqpoint{-0.008660in}{-0.014911in}}{\pgfqpoint{-0.004420in}{-0.016667in}}{\pgfqpoint{0.000000in}{-0.016667in}}%
\pgfpathlineto{\pgfqpoint{0.000000in}{-0.016667in}}%
\pgfpathclose%
\pgfusepath{stroke,fill}%
}%
\begin{pgfscope}%
\pgfsys@transformshift{3.047359in}{3.553059in}%
\pgfsys@useobject{currentmarker}{}%
\end{pgfscope}%
\end{pgfscope}%
\begin{pgfscope}%
\pgfpathrectangle{\pgfqpoint{0.765000in}{0.660000in}}{\pgfqpoint{4.620000in}{4.620000in}}%
\pgfusepath{clip}%
\pgfsetbuttcap%
\pgfsetroundjoin%
\pgfsetlinewidth{1.204500pt}%
\definecolor{currentstroke}{rgb}{0.827451,0.827451,0.827451}%
\pgfsetstrokecolor{currentstroke}%
\pgfsetdash{{1.200000pt}{1.980000pt}}{0.000000pt}%
\pgfpathmoveto{\pgfqpoint{3.047359in}{3.553059in}}%
\pgfpathlineto{\pgfqpoint{2.840481in}{3.456103in}}%
\pgfusepath{stroke}%
\end{pgfscope}%
\begin{pgfscope}%
\pgfpathrectangle{\pgfqpoint{0.765000in}{0.660000in}}{\pgfqpoint{4.620000in}{4.620000in}}%
\pgfusepath{clip}%
\pgfsetbuttcap%
\pgfsetroundjoin%
\pgfsetlinewidth{1.204500pt}%
\definecolor{currentstroke}{rgb}{0.827451,0.827451,0.827451}%
\pgfsetstrokecolor{currentstroke}%
\pgfsetdash{{1.200000pt}{1.980000pt}}{0.000000pt}%
\pgfpathmoveto{\pgfqpoint{3.047359in}{3.553059in}}%
\pgfpathlineto{\pgfqpoint{3.861519in}{3.171494in}}%
\pgfusepath{stroke}%
\end{pgfscope}%
\begin{pgfscope}%
\pgfpathrectangle{\pgfqpoint{0.765000in}{0.660000in}}{\pgfqpoint{4.620000in}{4.620000in}}%
\pgfusepath{clip}%
\pgfsetbuttcap%
\pgfsetroundjoin%
\pgfsetlinewidth{1.204500pt}%
\definecolor{currentstroke}{rgb}{0.827451,0.827451,0.827451}%
\pgfsetstrokecolor{currentstroke}%
\pgfsetdash{{1.200000pt}{1.980000pt}}{0.000000pt}%
\pgfpathmoveto{\pgfqpoint{3.047359in}{3.553059in}}%
\pgfpathlineto{\pgfqpoint{3.047359in}{3.553061in}}%
\pgfusepath{stroke}%
\end{pgfscope}%
\begin{pgfscope}%
\pgfpathrectangle{\pgfqpoint{0.765000in}{0.660000in}}{\pgfqpoint{4.620000in}{4.620000in}}%
\pgfusepath{clip}%
\pgfsetbuttcap%
\pgfsetroundjoin%
\pgfsetlinewidth{1.204500pt}%
\definecolor{currentstroke}{rgb}{0.827451,0.827451,0.827451}%
\pgfsetstrokecolor{currentstroke}%
\pgfsetdash{{1.200000pt}{1.980000pt}}{0.000000pt}%
\pgfpathmoveto{\pgfqpoint{3.047359in}{3.553059in}}%
\pgfpathlineto{\pgfqpoint{5.395000in}{4.653303in}}%
\pgfusepath{stroke}%
\end{pgfscope}%
\begin{pgfscope}%
\pgfpathrectangle{\pgfqpoint{0.765000in}{0.660000in}}{\pgfqpoint{4.620000in}{4.620000in}}%
\pgfusepath{clip}%
\pgfsetbuttcap%
\pgfsetroundjoin%
\pgfsetlinewidth{1.204500pt}%
\definecolor{currentstroke}{rgb}{0.827451,0.827451,0.827451}%
\pgfsetstrokecolor{currentstroke}%
\pgfsetdash{{1.200000pt}{1.980000pt}}{0.000000pt}%
\pgfpathmoveto{\pgfqpoint{3.047359in}{3.553059in}}%
\pgfpathlineto{\pgfqpoint{0.755000in}{4.627395in}}%
\pgfusepath{stroke}%
\end{pgfscope}%
\begin{pgfscope}%
\pgfpathrectangle{\pgfqpoint{0.765000in}{0.660000in}}{\pgfqpoint{4.620000in}{4.620000in}}%
\pgfusepath{clip}%
\pgfsetrectcap%
\pgfsetroundjoin%
\pgfsetlinewidth{1.204500pt}%
\definecolor{currentstroke}{rgb}{0.000000,0.000000,0.000000}%
\pgfsetstrokecolor{currentstroke}%
\pgfsetdash{}{0pt}%
\pgfpathmoveto{\pgfqpoint{2.144233in}{2.917910in}}%
\pgfusepath{stroke}%
\end{pgfscope}%
\begin{pgfscope}%
\pgfpathrectangle{\pgfqpoint{0.765000in}{0.660000in}}{\pgfqpoint{4.620000in}{4.620000in}}%
\pgfusepath{clip}%
\pgfsetbuttcap%
\pgfsetroundjoin%
\definecolor{currentfill}{rgb}{0.000000,0.000000,0.000000}%
\pgfsetfillcolor{currentfill}%
\pgfsetlinewidth{1.003750pt}%
\definecolor{currentstroke}{rgb}{0.000000,0.000000,0.000000}%
\pgfsetstrokecolor{currentstroke}%
\pgfsetdash{}{0pt}%
\pgfsys@defobject{currentmarker}{\pgfqpoint{-0.016667in}{-0.016667in}}{\pgfqpoint{0.016667in}{0.016667in}}{%
\pgfpathmoveto{\pgfqpoint{0.000000in}{-0.016667in}}%
\pgfpathcurveto{\pgfqpoint{0.004420in}{-0.016667in}}{\pgfqpoint{0.008660in}{-0.014911in}}{\pgfqpoint{0.011785in}{-0.011785in}}%
\pgfpathcurveto{\pgfqpoint{0.014911in}{-0.008660in}}{\pgfqpoint{0.016667in}{-0.004420in}}{\pgfqpoint{0.016667in}{0.000000in}}%
\pgfpathcurveto{\pgfqpoint{0.016667in}{0.004420in}}{\pgfqpoint{0.014911in}{0.008660in}}{\pgfqpoint{0.011785in}{0.011785in}}%
\pgfpathcurveto{\pgfqpoint{0.008660in}{0.014911in}}{\pgfqpoint{0.004420in}{0.016667in}}{\pgfqpoint{0.000000in}{0.016667in}}%
\pgfpathcurveto{\pgfqpoint{-0.004420in}{0.016667in}}{\pgfqpoint{-0.008660in}{0.014911in}}{\pgfqpoint{-0.011785in}{0.011785in}}%
\pgfpathcurveto{\pgfqpoint{-0.014911in}{0.008660in}}{\pgfqpoint{-0.016667in}{0.004420in}}{\pgfqpoint{-0.016667in}{0.000000in}}%
\pgfpathcurveto{\pgfqpoint{-0.016667in}{-0.004420in}}{\pgfqpoint{-0.014911in}{-0.008660in}}{\pgfqpoint{-0.011785in}{-0.011785in}}%
\pgfpathcurveto{\pgfqpoint{-0.008660in}{-0.014911in}}{\pgfqpoint{-0.004420in}{-0.016667in}}{\pgfqpoint{0.000000in}{-0.016667in}}%
\pgfpathlineto{\pgfqpoint{0.000000in}{-0.016667in}}%
\pgfpathclose%
\pgfusepath{stroke,fill}%
}%
\begin{pgfscope}%
\pgfsys@transformshift{2.144233in}{2.917910in}%
\pgfsys@useobject{currentmarker}{}%
\end{pgfscope}%
\end{pgfscope}%
\begin{pgfscope}%
\pgfpathrectangle{\pgfqpoint{0.765000in}{0.660000in}}{\pgfqpoint{4.620000in}{4.620000in}}%
\pgfusepath{clip}%
\pgfsetrectcap%
\pgfsetroundjoin%
\pgfsetlinewidth{1.204500pt}%
\definecolor{currentstroke}{rgb}{0.000000,0.000000,0.000000}%
\pgfsetstrokecolor{currentstroke}%
\pgfsetdash{}{0pt}%
\pgfpathmoveto{\pgfqpoint{2.144233in}{2.917910in}}%
\pgfpathlineto{\pgfqpoint{2.840481in}{3.244213in}}%
\pgfusepath{stroke}%
\end{pgfscope}%
\begin{pgfscope}%
\pgfpathrectangle{\pgfqpoint{0.765000in}{0.660000in}}{\pgfqpoint{4.620000in}{4.620000in}}%
\pgfusepath{clip}%
\pgfsetrectcap%
\pgfsetroundjoin%
\pgfsetlinewidth{1.204500pt}%
\definecolor{currentstroke}{rgb}{0.000000,0.000000,0.000000}%
\pgfsetstrokecolor{currentstroke}%
\pgfsetdash{}{0pt}%
\pgfpathmoveto{\pgfqpoint{2.144233in}{2.917910in}}%
\pgfpathlineto{\pgfqpoint{2.144233in}{0.650000in}}%
\pgfusepath{stroke}%
\end{pgfscope}%
\begin{pgfscope}%
\pgfpathrectangle{\pgfqpoint{0.765000in}{0.660000in}}{\pgfqpoint{4.620000in}{4.620000in}}%
\pgfusepath{clip}%
\pgfsetbuttcap%
\pgfsetroundjoin%
\pgfsetlinewidth{1.204500pt}%
\definecolor{currentstroke}{rgb}{0.827451,0.827451,0.827451}%
\pgfsetstrokecolor{currentstroke}%
\pgfsetdash{{1.200000pt}{1.980000pt}}{0.000000pt}%
\pgfpathmoveto{\pgfqpoint{2.144233in}{2.917910in}}%
\pgfpathlineto{\pgfqpoint{0.755000in}{3.568987in}}%
\pgfusepath{stroke}%
\end{pgfscope}%
\begin{pgfscope}%
\pgfpathrectangle{\pgfqpoint{0.765000in}{0.660000in}}{\pgfqpoint{4.620000in}{4.620000in}}%
\pgfusepath{clip}%
\pgfsetrectcap%
\pgfsetroundjoin%
\pgfsetlinewidth{1.204500pt}%
\definecolor{currentstroke}{rgb}{0.000000,0.000000,0.000000}%
\pgfsetstrokecolor{currentstroke}%
\pgfsetdash{}{0pt}%
\pgfpathmoveto{\pgfqpoint{3.047359in}{3.553061in}}%
\pgfusepath{stroke}%
\end{pgfscope}%
\begin{pgfscope}%
\pgfpathrectangle{\pgfqpoint{0.765000in}{0.660000in}}{\pgfqpoint{4.620000in}{4.620000in}}%
\pgfusepath{clip}%
\pgfsetbuttcap%
\pgfsetroundjoin%
\definecolor{currentfill}{rgb}{0.000000,0.000000,0.000000}%
\pgfsetfillcolor{currentfill}%
\pgfsetlinewidth{1.003750pt}%
\definecolor{currentstroke}{rgb}{0.000000,0.000000,0.000000}%
\pgfsetstrokecolor{currentstroke}%
\pgfsetdash{}{0pt}%
\pgfsys@defobject{currentmarker}{\pgfqpoint{-0.016667in}{-0.016667in}}{\pgfqpoint{0.016667in}{0.016667in}}{%
\pgfpathmoveto{\pgfqpoint{0.000000in}{-0.016667in}}%
\pgfpathcurveto{\pgfqpoint{0.004420in}{-0.016667in}}{\pgfqpoint{0.008660in}{-0.014911in}}{\pgfqpoint{0.011785in}{-0.011785in}}%
\pgfpathcurveto{\pgfqpoint{0.014911in}{-0.008660in}}{\pgfqpoint{0.016667in}{-0.004420in}}{\pgfqpoint{0.016667in}{0.000000in}}%
\pgfpathcurveto{\pgfqpoint{0.016667in}{0.004420in}}{\pgfqpoint{0.014911in}{0.008660in}}{\pgfqpoint{0.011785in}{0.011785in}}%
\pgfpathcurveto{\pgfqpoint{0.008660in}{0.014911in}}{\pgfqpoint{0.004420in}{0.016667in}}{\pgfqpoint{0.000000in}{0.016667in}}%
\pgfpathcurveto{\pgfqpoint{-0.004420in}{0.016667in}}{\pgfqpoint{-0.008660in}{0.014911in}}{\pgfqpoint{-0.011785in}{0.011785in}}%
\pgfpathcurveto{\pgfqpoint{-0.014911in}{0.008660in}}{\pgfqpoint{-0.016667in}{0.004420in}}{\pgfqpoint{-0.016667in}{0.000000in}}%
\pgfpathcurveto{\pgfqpoint{-0.016667in}{-0.004420in}}{\pgfqpoint{-0.014911in}{-0.008660in}}{\pgfqpoint{-0.011785in}{-0.011785in}}%
\pgfpathcurveto{\pgfqpoint{-0.008660in}{-0.014911in}}{\pgfqpoint{-0.004420in}{-0.016667in}}{\pgfqpoint{0.000000in}{-0.016667in}}%
\pgfpathlineto{\pgfqpoint{0.000000in}{-0.016667in}}%
\pgfpathclose%
\pgfusepath{stroke,fill}%
}%
\begin{pgfscope}%
\pgfsys@transformshift{3.047359in}{3.553061in}%
\pgfsys@useobject{currentmarker}{}%
\end{pgfscope}%
\end{pgfscope}%
\begin{pgfscope}%
\pgfpathrectangle{\pgfqpoint{0.765000in}{0.660000in}}{\pgfqpoint{4.620000in}{4.620000in}}%
\pgfusepath{clip}%
\pgfsetbuttcap%
\pgfsetroundjoin%
\pgfsetlinewidth{1.204500pt}%
\definecolor{currentstroke}{rgb}{0.827451,0.827451,0.827451}%
\pgfsetstrokecolor{currentstroke}%
\pgfsetdash{{1.200000pt}{1.980000pt}}{0.000000pt}%
\pgfpathmoveto{\pgfqpoint{3.047359in}{3.553061in}}%
\pgfpathlineto{\pgfqpoint{2.840481in}{3.456103in}}%
\pgfusepath{stroke}%
\end{pgfscope}%
\begin{pgfscope}%
\pgfpathrectangle{\pgfqpoint{0.765000in}{0.660000in}}{\pgfqpoint{4.620000in}{4.620000in}}%
\pgfusepath{clip}%
\pgfsetbuttcap%
\pgfsetroundjoin%
\pgfsetlinewidth{1.204500pt}%
\definecolor{currentstroke}{rgb}{0.827451,0.827451,0.827451}%
\pgfsetstrokecolor{currentstroke}%
\pgfsetdash{{1.200000pt}{1.980000pt}}{0.000000pt}%
\pgfpathmoveto{\pgfqpoint{3.047359in}{3.553061in}}%
\pgfpathlineto{\pgfqpoint{3.861519in}{3.171494in}}%
\pgfusepath{stroke}%
\end{pgfscope}%
\begin{pgfscope}%
\pgfpathrectangle{\pgfqpoint{0.765000in}{0.660000in}}{\pgfqpoint{4.620000in}{4.620000in}}%
\pgfusepath{clip}%
\pgfsetbuttcap%
\pgfsetroundjoin%
\pgfsetlinewidth{1.204500pt}%
\definecolor{currentstroke}{rgb}{0.827451,0.827451,0.827451}%
\pgfsetstrokecolor{currentstroke}%
\pgfsetdash{{1.200000pt}{1.980000pt}}{0.000000pt}%
\pgfpathmoveto{\pgfqpoint{3.047359in}{3.553061in}}%
\pgfpathlineto{\pgfqpoint{3.047359in}{3.553059in}}%
\pgfusepath{stroke}%
\end{pgfscope}%
\begin{pgfscope}%
\pgfpathrectangle{\pgfqpoint{0.765000in}{0.660000in}}{\pgfqpoint{4.620000in}{4.620000in}}%
\pgfusepath{clip}%
\pgfsetbuttcap%
\pgfsetroundjoin%
\pgfsetlinewidth{1.204500pt}%
\definecolor{currentstroke}{rgb}{0.827451,0.827451,0.827451}%
\pgfsetstrokecolor{currentstroke}%
\pgfsetdash{{1.200000pt}{1.980000pt}}{0.000000pt}%
\pgfpathmoveto{\pgfqpoint{3.047359in}{3.553061in}}%
\pgfpathlineto{\pgfqpoint{0.755000in}{4.627397in}}%
\pgfusepath{stroke}%
\end{pgfscope}%
\begin{pgfscope}%
\pgfpathrectangle{\pgfqpoint{0.765000in}{0.660000in}}{\pgfqpoint{4.620000in}{4.620000in}}%
\pgfusepath{clip}%
\pgfsetbuttcap%
\pgfsetroundjoin%
\pgfsetlinewidth{1.204500pt}%
\definecolor{currentstroke}{rgb}{0.827451,0.827451,0.827451}%
\pgfsetstrokecolor{currentstroke}%
\pgfsetdash{{1.200000pt}{1.980000pt}}{0.000000pt}%
\pgfpathmoveto{\pgfqpoint{3.047359in}{3.553061in}}%
\pgfpathlineto{\pgfqpoint{5.395000in}{4.653306in}}%
\pgfusepath{stroke}%
\end{pgfscope}%
\end{pgfpicture}%
\makeatother%
\endgroup%
}
        }
        \caption{Moons dataset from Scikit-learn}
        \label{fig:moon}
    \end{subfigure}

    \bigskip
    \centering
    \begin{subfigure}{0.45\textwidth}
        \centering
        \resizebox{\linewidth}{!}{%
        \centering
            \clipbox{0.25\width{} 0.3\height{} 0.25\width{} 0.3\height{}}{%% Creator: Matplotlib, PGF backend
%%
%% To include the figure in your LaTeX document, write
%%   \input{<filename>.pgf}
%%
%% Make sure the required packages are loaded in your preamble
%%   \usepackage{pgf}
%%
%% Also ensure that all the required font packages are loaded; for instance,
%% the lmodern package is sometimes necessary when using math font.
%%   \usepackage{lmodern}
%%
%% Figures using additional raster images can only be included by \input if
%% they are in the same directory as the main LaTeX file. For loading figures
%% from other directories you can use the `import` package
%%   \usepackage{import}
%%
%% and then include the figures with
%%   \import{<path to file>}{<filename>.pgf}
%%
%% Matplotlib used the following preamble
%%   
%%   \usepackage{fontspec}
%%   \setmainfont{DejaVuSerif.ttf}[Path=\detokenize{/Users/sam/Library/Python/3.9/lib/python/site-packages/matplotlib/mpl-data/fonts/ttf/}]
%%   \setsansfont{DejaVuSans.ttf}[Path=\detokenize{/Users/sam/Library/Python/3.9/lib/python/site-packages/matplotlib/mpl-data/fonts/ttf/}]
%%   \setmonofont{DejaVuSansMono.ttf}[Path=\detokenize{/Users/sam/Library/Python/3.9/lib/python/site-packages/matplotlib/mpl-data/fonts/ttf/}]
%%   \makeatletter\@ifpackageloaded{underscore}{}{\usepackage[strings]{underscore}}\makeatother
%%
\begingroup%
\makeatletter%
\begin{pgfpicture}%
\pgfpathrectangle{\pgfpointorigin}{\pgfqpoint{6.000000in}{6.000000in}}%
\pgfusepath{use as bounding box, clip}%
\begin{pgfscope}%
\pgfsetbuttcap%
\pgfsetmiterjoin%
\definecolor{currentfill}{rgb}{1.000000,1.000000,1.000000}%
\pgfsetfillcolor{currentfill}%
\pgfsetlinewidth{0.000000pt}%
\definecolor{currentstroke}{rgb}{1.000000,1.000000,1.000000}%
\pgfsetstrokecolor{currentstroke}%
\pgfsetdash{}{0pt}%
\pgfpathmoveto{\pgfqpoint{0.000000in}{0.000000in}}%
\pgfpathlineto{\pgfqpoint{6.000000in}{0.000000in}}%
\pgfpathlineto{\pgfqpoint{6.000000in}{6.000000in}}%
\pgfpathlineto{\pgfqpoint{0.000000in}{6.000000in}}%
\pgfpathlineto{\pgfqpoint{0.000000in}{0.000000in}}%
\pgfpathclose%
\pgfusepath{fill}%
\end{pgfscope}%
\begin{pgfscope}%
\pgfsetbuttcap%
\pgfsetmiterjoin%
\definecolor{currentfill}{rgb}{1.000000,1.000000,1.000000}%
\pgfsetfillcolor{currentfill}%
\pgfsetlinewidth{0.000000pt}%
\definecolor{currentstroke}{rgb}{0.000000,0.000000,0.000000}%
\pgfsetstrokecolor{currentstroke}%
\pgfsetstrokeopacity{0.000000}%
\pgfsetdash{}{0pt}%
\pgfpathmoveto{\pgfqpoint{0.765000in}{0.660000in}}%
\pgfpathlineto{\pgfqpoint{5.385000in}{0.660000in}}%
\pgfpathlineto{\pgfqpoint{5.385000in}{5.280000in}}%
\pgfpathlineto{\pgfqpoint{0.765000in}{5.280000in}}%
\pgfpathlineto{\pgfqpoint{0.765000in}{0.660000in}}%
\pgfpathclose%
\pgfusepath{fill}%
\end{pgfscope}%
\begin{pgfscope}%
\pgfpathrectangle{\pgfqpoint{0.765000in}{0.660000in}}{\pgfqpoint{4.620000in}{4.620000in}}%
\pgfusepath{clip}%
\pgfsetbuttcap%
\pgfsetroundjoin%
\definecolor{currentfill}{rgb}{1.000000,0.894118,0.788235}%
\pgfsetfillcolor{currentfill}%
\pgfsetlinewidth{0.000000pt}%
\definecolor{currentstroke}{rgb}{1.000000,0.894118,0.788235}%
\pgfsetstrokecolor{currentstroke}%
\pgfsetdash{}{0pt}%
\pgfpathmoveto{\pgfqpoint{3.149560in}{2.514906in}}%
\pgfpathlineto{\pgfqpoint{3.068601in}{2.468164in}}%
\pgfpathlineto{\pgfqpoint{3.149560in}{2.421422in}}%
\pgfpathlineto{\pgfqpoint{3.230519in}{2.468164in}}%
\pgfpathlineto{\pgfqpoint{3.149560in}{2.514906in}}%
\pgfpathclose%
\pgfusepath{fill}%
\end{pgfscope}%
\begin{pgfscope}%
\pgfpathrectangle{\pgfqpoint{0.765000in}{0.660000in}}{\pgfqpoint{4.620000in}{4.620000in}}%
\pgfusepath{clip}%
\pgfsetbuttcap%
\pgfsetroundjoin%
\definecolor{currentfill}{rgb}{1.000000,0.894118,0.788235}%
\pgfsetfillcolor{currentfill}%
\pgfsetlinewidth{0.000000pt}%
\definecolor{currentstroke}{rgb}{1.000000,0.894118,0.788235}%
\pgfsetstrokecolor{currentstroke}%
\pgfsetdash{}{0pt}%
\pgfpathmoveto{\pgfqpoint{3.149560in}{2.514906in}}%
\pgfpathlineto{\pgfqpoint{3.068601in}{2.468164in}}%
\pgfpathlineto{\pgfqpoint{3.068601in}{2.561648in}}%
\pgfpathlineto{\pgfqpoint{3.149560in}{2.608390in}}%
\pgfpathlineto{\pgfqpoint{3.149560in}{2.514906in}}%
\pgfpathclose%
\pgfusepath{fill}%
\end{pgfscope}%
\begin{pgfscope}%
\pgfpathrectangle{\pgfqpoint{0.765000in}{0.660000in}}{\pgfqpoint{4.620000in}{4.620000in}}%
\pgfusepath{clip}%
\pgfsetbuttcap%
\pgfsetroundjoin%
\definecolor{currentfill}{rgb}{1.000000,0.894118,0.788235}%
\pgfsetfillcolor{currentfill}%
\pgfsetlinewidth{0.000000pt}%
\definecolor{currentstroke}{rgb}{1.000000,0.894118,0.788235}%
\pgfsetstrokecolor{currentstroke}%
\pgfsetdash{}{0pt}%
\pgfpathmoveto{\pgfqpoint{3.149560in}{2.514906in}}%
\pgfpathlineto{\pgfqpoint{3.230519in}{2.468164in}}%
\pgfpathlineto{\pgfqpoint{3.230519in}{2.561648in}}%
\pgfpathlineto{\pgfqpoint{3.149560in}{2.608390in}}%
\pgfpathlineto{\pgfqpoint{3.149560in}{2.514906in}}%
\pgfpathclose%
\pgfusepath{fill}%
\end{pgfscope}%
\begin{pgfscope}%
\pgfpathrectangle{\pgfqpoint{0.765000in}{0.660000in}}{\pgfqpoint{4.620000in}{4.620000in}}%
\pgfusepath{clip}%
\pgfsetbuttcap%
\pgfsetroundjoin%
\definecolor{currentfill}{rgb}{1.000000,0.894118,0.788235}%
\pgfsetfillcolor{currentfill}%
\pgfsetlinewidth{0.000000pt}%
\definecolor{currentstroke}{rgb}{1.000000,0.894118,0.788235}%
\pgfsetstrokecolor{currentstroke}%
\pgfsetdash{}{0pt}%
\pgfpathmoveto{\pgfqpoint{3.025552in}{2.586502in}}%
\pgfpathlineto{\pgfqpoint{2.944593in}{2.539760in}}%
\pgfpathlineto{\pgfqpoint{3.025552in}{2.493018in}}%
\pgfpathlineto{\pgfqpoint{3.106512in}{2.539760in}}%
\pgfpathlineto{\pgfqpoint{3.025552in}{2.586502in}}%
\pgfpathclose%
\pgfusepath{fill}%
\end{pgfscope}%
\begin{pgfscope}%
\pgfpathrectangle{\pgfqpoint{0.765000in}{0.660000in}}{\pgfqpoint{4.620000in}{4.620000in}}%
\pgfusepath{clip}%
\pgfsetbuttcap%
\pgfsetroundjoin%
\definecolor{currentfill}{rgb}{1.000000,0.894118,0.788235}%
\pgfsetfillcolor{currentfill}%
\pgfsetlinewidth{0.000000pt}%
\definecolor{currentstroke}{rgb}{1.000000,0.894118,0.788235}%
\pgfsetstrokecolor{currentstroke}%
\pgfsetdash{}{0pt}%
\pgfpathmoveto{\pgfqpoint{3.025552in}{2.586502in}}%
\pgfpathlineto{\pgfqpoint{2.944593in}{2.539760in}}%
\pgfpathlineto{\pgfqpoint{2.944593in}{2.633244in}}%
\pgfpathlineto{\pgfqpoint{3.025552in}{2.679986in}}%
\pgfpathlineto{\pgfqpoint{3.025552in}{2.586502in}}%
\pgfpathclose%
\pgfusepath{fill}%
\end{pgfscope}%
\begin{pgfscope}%
\pgfpathrectangle{\pgfqpoint{0.765000in}{0.660000in}}{\pgfqpoint{4.620000in}{4.620000in}}%
\pgfusepath{clip}%
\pgfsetbuttcap%
\pgfsetroundjoin%
\definecolor{currentfill}{rgb}{1.000000,0.894118,0.788235}%
\pgfsetfillcolor{currentfill}%
\pgfsetlinewidth{0.000000pt}%
\definecolor{currentstroke}{rgb}{1.000000,0.894118,0.788235}%
\pgfsetstrokecolor{currentstroke}%
\pgfsetdash{}{0pt}%
\pgfpathmoveto{\pgfqpoint{3.025552in}{2.586502in}}%
\pgfpathlineto{\pgfqpoint{3.106512in}{2.539760in}}%
\pgfpathlineto{\pgfqpoint{3.106512in}{2.633244in}}%
\pgfpathlineto{\pgfqpoint{3.025552in}{2.679986in}}%
\pgfpathlineto{\pgfqpoint{3.025552in}{2.586502in}}%
\pgfpathclose%
\pgfusepath{fill}%
\end{pgfscope}%
\begin{pgfscope}%
\pgfpathrectangle{\pgfqpoint{0.765000in}{0.660000in}}{\pgfqpoint{4.620000in}{4.620000in}}%
\pgfusepath{clip}%
\pgfsetbuttcap%
\pgfsetroundjoin%
\definecolor{currentfill}{rgb}{1.000000,0.894118,0.788235}%
\pgfsetfillcolor{currentfill}%
\pgfsetlinewidth{0.000000pt}%
\definecolor{currentstroke}{rgb}{1.000000,0.894118,0.788235}%
\pgfsetstrokecolor{currentstroke}%
\pgfsetdash{}{0pt}%
\pgfpathmoveto{\pgfqpoint{3.149560in}{2.421422in}}%
\pgfpathlineto{\pgfqpoint{3.230519in}{2.468164in}}%
\pgfpathlineto{\pgfqpoint{3.230519in}{2.561648in}}%
\pgfpathlineto{\pgfqpoint{3.149560in}{2.514906in}}%
\pgfpathlineto{\pgfqpoint{3.149560in}{2.421422in}}%
\pgfpathclose%
\pgfusepath{fill}%
\end{pgfscope}%
\begin{pgfscope}%
\pgfpathrectangle{\pgfqpoint{0.765000in}{0.660000in}}{\pgfqpoint{4.620000in}{4.620000in}}%
\pgfusepath{clip}%
\pgfsetbuttcap%
\pgfsetroundjoin%
\definecolor{currentfill}{rgb}{1.000000,0.894118,0.788235}%
\pgfsetfillcolor{currentfill}%
\pgfsetlinewidth{0.000000pt}%
\definecolor{currentstroke}{rgb}{1.000000,0.894118,0.788235}%
\pgfsetstrokecolor{currentstroke}%
\pgfsetdash{}{0pt}%
\pgfpathmoveto{\pgfqpoint{3.025552in}{2.493018in}}%
\pgfpathlineto{\pgfqpoint{3.106512in}{2.539760in}}%
\pgfpathlineto{\pgfqpoint{3.106512in}{2.633244in}}%
\pgfpathlineto{\pgfqpoint{3.025552in}{2.586502in}}%
\pgfpathlineto{\pgfqpoint{3.025552in}{2.493018in}}%
\pgfpathclose%
\pgfusepath{fill}%
\end{pgfscope}%
\begin{pgfscope}%
\pgfpathrectangle{\pgfqpoint{0.765000in}{0.660000in}}{\pgfqpoint{4.620000in}{4.620000in}}%
\pgfusepath{clip}%
\pgfsetbuttcap%
\pgfsetroundjoin%
\definecolor{currentfill}{rgb}{1.000000,0.894118,0.788235}%
\pgfsetfillcolor{currentfill}%
\pgfsetlinewidth{0.000000pt}%
\definecolor{currentstroke}{rgb}{1.000000,0.894118,0.788235}%
\pgfsetstrokecolor{currentstroke}%
\pgfsetdash{}{0pt}%
\pgfpathmoveto{\pgfqpoint{3.149560in}{2.608390in}}%
\pgfpathlineto{\pgfqpoint{3.068601in}{2.561648in}}%
\pgfpathlineto{\pgfqpoint{3.149560in}{2.514906in}}%
\pgfpathlineto{\pgfqpoint{3.230519in}{2.561648in}}%
\pgfpathlineto{\pgfqpoint{3.149560in}{2.608390in}}%
\pgfpathclose%
\pgfusepath{fill}%
\end{pgfscope}%
\begin{pgfscope}%
\pgfpathrectangle{\pgfqpoint{0.765000in}{0.660000in}}{\pgfqpoint{4.620000in}{4.620000in}}%
\pgfusepath{clip}%
\pgfsetbuttcap%
\pgfsetroundjoin%
\definecolor{currentfill}{rgb}{1.000000,0.894118,0.788235}%
\pgfsetfillcolor{currentfill}%
\pgfsetlinewidth{0.000000pt}%
\definecolor{currentstroke}{rgb}{1.000000,0.894118,0.788235}%
\pgfsetstrokecolor{currentstroke}%
\pgfsetdash{}{0pt}%
\pgfpathmoveto{\pgfqpoint{3.068601in}{2.468164in}}%
\pgfpathlineto{\pgfqpoint{3.149560in}{2.421422in}}%
\pgfpathlineto{\pgfqpoint{3.149560in}{2.514906in}}%
\pgfpathlineto{\pgfqpoint{3.068601in}{2.561648in}}%
\pgfpathlineto{\pgfqpoint{3.068601in}{2.468164in}}%
\pgfpathclose%
\pgfusepath{fill}%
\end{pgfscope}%
\begin{pgfscope}%
\pgfpathrectangle{\pgfqpoint{0.765000in}{0.660000in}}{\pgfqpoint{4.620000in}{4.620000in}}%
\pgfusepath{clip}%
\pgfsetbuttcap%
\pgfsetroundjoin%
\definecolor{currentfill}{rgb}{1.000000,0.894118,0.788235}%
\pgfsetfillcolor{currentfill}%
\pgfsetlinewidth{0.000000pt}%
\definecolor{currentstroke}{rgb}{1.000000,0.894118,0.788235}%
\pgfsetstrokecolor{currentstroke}%
\pgfsetdash{}{0pt}%
\pgfpathmoveto{\pgfqpoint{3.025552in}{2.679986in}}%
\pgfpathlineto{\pgfqpoint{2.944593in}{2.633244in}}%
\pgfpathlineto{\pgfqpoint{3.025552in}{2.586502in}}%
\pgfpathlineto{\pgfqpoint{3.106512in}{2.633244in}}%
\pgfpathlineto{\pgfqpoint{3.025552in}{2.679986in}}%
\pgfpathclose%
\pgfusepath{fill}%
\end{pgfscope}%
\begin{pgfscope}%
\pgfpathrectangle{\pgfqpoint{0.765000in}{0.660000in}}{\pgfqpoint{4.620000in}{4.620000in}}%
\pgfusepath{clip}%
\pgfsetbuttcap%
\pgfsetroundjoin%
\definecolor{currentfill}{rgb}{1.000000,0.894118,0.788235}%
\pgfsetfillcolor{currentfill}%
\pgfsetlinewidth{0.000000pt}%
\definecolor{currentstroke}{rgb}{1.000000,0.894118,0.788235}%
\pgfsetstrokecolor{currentstroke}%
\pgfsetdash{}{0pt}%
\pgfpathmoveto{\pgfqpoint{2.944593in}{2.539760in}}%
\pgfpathlineto{\pgfqpoint{3.025552in}{2.493018in}}%
\pgfpathlineto{\pgfqpoint{3.025552in}{2.586502in}}%
\pgfpathlineto{\pgfqpoint{2.944593in}{2.633244in}}%
\pgfpathlineto{\pgfqpoint{2.944593in}{2.539760in}}%
\pgfpathclose%
\pgfusepath{fill}%
\end{pgfscope}%
\begin{pgfscope}%
\pgfpathrectangle{\pgfqpoint{0.765000in}{0.660000in}}{\pgfqpoint{4.620000in}{4.620000in}}%
\pgfusepath{clip}%
\pgfsetbuttcap%
\pgfsetroundjoin%
\definecolor{currentfill}{rgb}{1.000000,0.894118,0.788235}%
\pgfsetfillcolor{currentfill}%
\pgfsetlinewidth{0.000000pt}%
\definecolor{currentstroke}{rgb}{1.000000,0.894118,0.788235}%
\pgfsetstrokecolor{currentstroke}%
\pgfsetdash{}{0pt}%
\pgfpathmoveto{\pgfqpoint{3.149560in}{2.514906in}}%
\pgfpathlineto{\pgfqpoint{3.068601in}{2.468164in}}%
\pgfpathlineto{\pgfqpoint{2.944593in}{2.539760in}}%
\pgfpathlineto{\pgfqpoint{3.025552in}{2.586502in}}%
\pgfpathlineto{\pgfqpoint{3.149560in}{2.514906in}}%
\pgfpathclose%
\pgfusepath{fill}%
\end{pgfscope}%
\begin{pgfscope}%
\pgfpathrectangle{\pgfqpoint{0.765000in}{0.660000in}}{\pgfqpoint{4.620000in}{4.620000in}}%
\pgfusepath{clip}%
\pgfsetbuttcap%
\pgfsetroundjoin%
\definecolor{currentfill}{rgb}{1.000000,0.894118,0.788235}%
\pgfsetfillcolor{currentfill}%
\pgfsetlinewidth{0.000000pt}%
\definecolor{currentstroke}{rgb}{1.000000,0.894118,0.788235}%
\pgfsetstrokecolor{currentstroke}%
\pgfsetdash{}{0pt}%
\pgfpathmoveto{\pgfqpoint{3.230519in}{2.468164in}}%
\pgfpathlineto{\pgfqpoint{3.149560in}{2.514906in}}%
\pgfpathlineto{\pgfqpoint{3.025552in}{2.586502in}}%
\pgfpathlineto{\pgfqpoint{3.106512in}{2.539760in}}%
\pgfpathlineto{\pgfqpoint{3.230519in}{2.468164in}}%
\pgfpathclose%
\pgfusepath{fill}%
\end{pgfscope}%
\begin{pgfscope}%
\pgfpathrectangle{\pgfqpoint{0.765000in}{0.660000in}}{\pgfqpoint{4.620000in}{4.620000in}}%
\pgfusepath{clip}%
\pgfsetbuttcap%
\pgfsetroundjoin%
\definecolor{currentfill}{rgb}{1.000000,0.894118,0.788235}%
\pgfsetfillcolor{currentfill}%
\pgfsetlinewidth{0.000000pt}%
\definecolor{currentstroke}{rgb}{1.000000,0.894118,0.788235}%
\pgfsetstrokecolor{currentstroke}%
\pgfsetdash{}{0pt}%
\pgfpathmoveto{\pgfqpoint{3.149560in}{2.514906in}}%
\pgfpathlineto{\pgfqpoint{3.149560in}{2.608390in}}%
\pgfpathlineto{\pgfqpoint{3.025552in}{2.679986in}}%
\pgfpathlineto{\pgfqpoint{3.106512in}{2.539760in}}%
\pgfpathlineto{\pgfqpoint{3.149560in}{2.514906in}}%
\pgfpathclose%
\pgfusepath{fill}%
\end{pgfscope}%
\begin{pgfscope}%
\pgfpathrectangle{\pgfqpoint{0.765000in}{0.660000in}}{\pgfqpoint{4.620000in}{4.620000in}}%
\pgfusepath{clip}%
\pgfsetbuttcap%
\pgfsetroundjoin%
\definecolor{currentfill}{rgb}{1.000000,0.894118,0.788235}%
\pgfsetfillcolor{currentfill}%
\pgfsetlinewidth{0.000000pt}%
\definecolor{currentstroke}{rgb}{1.000000,0.894118,0.788235}%
\pgfsetstrokecolor{currentstroke}%
\pgfsetdash{}{0pt}%
\pgfpathmoveto{\pgfqpoint{3.230519in}{2.468164in}}%
\pgfpathlineto{\pgfqpoint{3.230519in}{2.561648in}}%
\pgfpathlineto{\pgfqpoint{3.106512in}{2.633244in}}%
\pgfpathlineto{\pgfqpoint{3.025552in}{2.586502in}}%
\pgfpathlineto{\pgfqpoint{3.230519in}{2.468164in}}%
\pgfpathclose%
\pgfusepath{fill}%
\end{pgfscope}%
\begin{pgfscope}%
\pgfpathrectangle{\pgfqpoint{0.765000in}{0.660000in}}{\pgfqpoint{4.620000in}{4.620000in}}%
\pgfusepath{clip}%
\pgfsetbuttcap%
\pgfsetroundjoin%
\definecolor{currentfill}{rgb}{1.000000,0.894118,0.788235}%
\pgfsetfillcolor{currentfill}%
\pgfsetlinewidth{0.000000pt}%
\definecolor{currentstroke}{rgb}{1.000000,0.894118,0.788235}%
\pgfsetstrokecolor{currentstroke}%
\pgfsetdash{}{0pt}%
\pgfpathmoveto{\pgfqpoint{3.068601in}{2.468164in}}%
\pgfpathlineto{\pgfqpoint{3.149560in}{2.421422in}}%
\pgfpathlineto{\pgfqpoint{3.025552in}{2.493018in}}%
\pgfpathlineto{\pgfqpoint{2.944593in}{2.539760in}}%
\pgfpathlineto{\pgfqpoint{3.068601in}{2.468164in}}%
\pgfpathclose%
\pgfusepath{fill}%
\end{pgfscope}%
\begin{pgfscope}%
\pgfpathrectangle{\pgfqpoint{0.765000in}{0.660000in}}{\pgfqpoint{4.620000in}{4.620000in}}%
\pgfusepath{clip}%
\pgfsetbuttcap%
\pgfsetroundjoin%
\definecolor{currentfill}{rgb}{1.000000,0.894118,0.788235}%
\pgfsetfillcolor{currentfill}%
\pgfsetlinewidth{0.000000pt}%
\definecolor{currentstroke}{rgb}{1.000000,0.894118,0.788235}%
\pgfsetstrokecolor{currentstroke}%
\pgfsetdash{}{0pt}%
\pgfpathmoveto{\pgfqpoint{3.149560in}{2.421422in}}%
\pgfpathlineto{\pgfqpoint{3.230519in}{2.468164in}}%
\pgfpathlineto{\pgfqpoint{3.106512in}{2.539760in}}%
\pgfpathlineto{\pgfqpoint{3.025552in}{2.493018in}}%
\pgfpathlineto{\pgfqpoint{3.149560in}{2.421422in}}%
\pgfpathclose%
\pgfusepath{fill}%
\end{pgfscope}%
\begin{pgfscope}%
\pgfpathrectangle{\pgfqpoint{0.765000in}{0.660000in}}{\pgfqpoint{4.620000in}{4.620000in}}%
\pgfusepath{clip}%
\pgfsetbuttcap%
\pgfsetroundjoin%
\definecolor{currentfill}{rgb}{1.000000,0.894118,0.788235}%
\pgfsetfillcolor{currentfill}%
\pgfsetlinewidth{0.000000pt}%
\definecolor{currentstroke}{rgb}{1.000000,0.894118,0.788235}%
\pgfsetstrokecolor{currentstroke}%
\pgfsetdash{}{0pt}%
\pgfpathmoveto{\pgfqpoint{3.149560in}{2.608390in}}%
\pgfpathlineto{\pgfqpoint{3.068601in}{2.561648in}}%
\pgfpathlineto{\pgfqpoint{2.944593in}{2.633244in}}%
\pgfpathlineto{\pgfqpoint{3.025552in}{2.679986in}}%
\pgfpathlineto{\pgfqpoint{3.149560in}{2.608390in}}%
\pgfpathclose%
\pgfusepath{fill}%
\end{pgfscope}%
\begin{pgfscope}%
\pgfpathrectangle{\pgfqpoint{0.765000in}{0.660000in}}{\pgfqpoint{4.620000in}{4.620000in}}%
\pgfusepath{clip}%
\pgfsetbuttcap%
\pgfsetroundjoin%
\definecolor{currentfill}{rgb}{1.000000,0.894118,0.788235}%
\pgfsetfillcolor{currentfill}%
\pgfsetlinewidth{0.000000pt}%
\definecolor{currentstroke}{rgb}{1.000000,0.894118,0.788235}%
\pgfsetstrokecolor{currentstroke}%
\pgfsetdash{}{0pt}%
\pgfpathmoveto{\pgfqpoint{3.230519in}{2.561648in}}%
\pgfpathlineto{\pgfqpoint{3.149560in}{2.608390in}}%
\pgfpathlineto{\pgfqpoint{3.025552in}{2.679986in}}%
\pgfpathlineto{\pgfqpoint{3.106512in}{2.633244in}}%
\pgfpathlineto{\pgfqpoint{3.230519in}{2.561648in}}%
\pgfpathclose%
\pgfusepath{fill}%
\end{pgfscope}%
\begin{pgfscope}%
\pgfpathrectangle{\pgfqpoint{0.765000in}{0.660000in}}{\pgfqpoint{4.620000in}{4.620000in}}%
\pgfusepath{clip}%
\pgfsetbuttcap%
\pgfsetroundjoin%
\definecolor{currentfill}{rgb}{1.000000,0.894118,0.788235}%
\pgfsetfillcolor{currentfill}%
\pgfsetlinewidth{0.000000pt}%
\definecolor{currentstroke}{rgb}{1.000000,0.894118,0.788235}%
\pgfsetstrokecolor{currentstroke}%
\pgfsetdash{}{0pt}%
\pgfpathmoveto{\pgfqpoint{3.068601in}{2.468164in}}%
\pgfpathlineto{\pgfqpoint{3.068601in}{2.561648in}}%
\pgfpathlineto{\pgfqpoint{2.944593in}{2.633244in}}%
\pgfpathlineto{\pgfqpoint{3.025552in}{2.493018in}}%
\pgfpathlineto{\pgfqpoint{3.068601in}{2.468164in}}%
\pgfpathclose%
\pgfusepath{fill}%
\end{pgfscope}%
\begin{pgfscope}%
\pgfpathrectangle{\pgfqpoint{0.765000in}{0.660000in}}{\pgfqpoint{4.620000in}{4.620000in}}%
\pgfusepath{clip}%
\pgfsetbuttcap%
\pgfsetroundjoin%
\definecolor{currentfill}{rgb}{1.000000,0.894118,0.788235}%
\pgfsetfillcolor{currentfill}%
\pgfsetlinewidth{0.000000pt}%
\definecolor{currentstroke}{rgb}{1.000000,0.894118,0.788235}%
\pgfsetstrokecolor{currentstroke}%
\pgfsetdash{}{0pt}%
\pgfpathmoveto{\pgfqpoint{3.149560in}{2.421422in}}%
\pgfpathlineto{\pgfqpoint{3.149560in}{2.514906in}}%
\pgfpathlineto{\pgfqpoint{3.025552in}{2.586502in}}%
\pgfpathlineto{\pgfqpoint{2.944593in}{2.539760in}}%
\pgfpathlineto{\pgfqpoint{3.149560in}{2.421422in}}%
\pgfpathclose%
\pgfusepath{fill}%
\end{pgfscope}%
\begin{pgfscope}%
\pgfpathrectangle{\pgfqpoint{0.765000in}{0.660000in}}{\pgfqpoint{4.620000in}{4.620000in}}%
\pgfusepath{clip}%
\pgfsetbuttcap%
\pgfsetroundjoin%
\definecolor{currentfill}{rgb}{1.000000,0.894118,0.788235}%
\pgfsetfillcolor{currentfill}%
\pgfsetlinewidth{0.000000pt}%
\definecolor{currentstroke}{rgb}{1.000000,0.894118,0.788235}%
\pgfsetstrokecolor{currentstroke}%
\pgfsetdash{}{0pt}%
\pgfpathmoveto{\pgfqpoint{3.068601in}{2.561648in}}%
\pgfpathlineto{\pgfqpoint{3.149560in}{2.514906in}}%
\pgfpathlineto{\pgfqpoint{3.025552in}{2.586502in}}%
\pgfpathlineto{\pgfqpoint{2.944593in}{2.633244in}}%
\pgfpathlineto{\pgfqpoint{3.068601in}{2.561648in}}%
\pgfpathclose%
\pgfusepath{fill}%
\end{pgfscope}%
\begin{pgfscope}%
\pgfpathrectangle{\pgfqpoint{0.765000in}{0.660000in}}{\pgfqpoint{4.620000in}{4.620000in}}%
\pgfusepath{clip}%
\pgfsetbuttcap%
\pgfsetroundjoin%
\definecolor{currentfill}{rgb}{1.000000,0.894118,0.788235}%
\pgfsetfillcolor{currentfill}%
\pgfsetlinewidth{0.000000pt}%
\definecolor{currentstroke}{rgb}{1.000000,0.894118,0.788235}%
\pgfsetstrokecolor{currentstroke}%
\pgfsetdash{}{0pt}%
\pgfpathmoveto{\pgfqpoint{3.149560in}{2.514906in}}%
\pgfpathlineto{\pgfqpoint{3.230519in}{2.561648in}}%
\pgfpathlineto{\pgfqpoint{3.106512in}{2.633244in}}%
\pgfpathlineto{\pgfqpoint{3.025552in}{2.586502in}}%
\pgfpathlineto{\pgfqpoint{3.149560in}{2.514906in}}%
\pgfpathclose%
\pgfusepath{fill}%
\end{pgfscope}%
\begin{pgfscope}%
\pgfpathrectangle{\pgfqpoint{0.765000in}{0.660000in}}{\pgfqpoint{4.620000in}{4.620000in}}%
\pgfusepath{clip}%
\pgfsetbuttcap%
\pgfsetroundjoin%
\definecolor{currentfill}{rgb}{1.000000,0.894118,0.788235}%
\pgfsetfillcolor{currentfill}%
\pgfsetlinewidth{0.000000pt}%
\definecolor{currentstroke}{rgb}{1.000000,0.894118,0.788235}%
\pgfsetstrokecolor{currentstroke}%
\pgfsetdash{}{0pt}%
\pgfpathmoveto{\pgfqpoint{3.149560in}{2.514906in}}%
\pgfpathlineto{\pgfqpoint{3.068601in}{2.468164in}}%
\pgfpathlineto{\pgfqpoint{3.149560in}{2.421422in}}%
\pgfpathlineto{\pgfqpoint{3.230519in}{2.468164in}}%
\pgfpathlineto{\pgfqpoint{3.149560in}{2.514906in}}%
\pgfpathclose%
\pgfusepath{fill}%
\end{pgfscope}%
\begin{pgfscope}%
\pgfpathrectangle{\pgfqpoint{0.765000in}{0.660000in}}{\pgfqpoint{4.620000in}{4.620000in}}%
\pgfusepath{clip}%
\pgfsetbuttcap%
\pgfsetroundjoin%
\definecolor{currentfill}{rgb}{1.000000,0.894118,0.788235}%
\pgfsetfillcolor{currentfill}%
\pgfsetlinewidth{0.000000pt}%
\definecolor{currentstroke}{rgb}{1.000000,0.894118,0.788235}%
\pgfsetstrokecolor{currentstroke}%
\pgfsetdash{}{0pt}%
\pgfpathmoveto{\pgfqpoint{3.149560in}{2.514906in}}%
\pgfpathlineto{\pgfqpoint{3.068601in}{2.468164in}}%
\pgfpathlineto{\pgfqpoint{3.068601in}{2.561648in}}%
\pgfpathlineto{\pgfqpoint{3.149560in}{2.608390in}}%
\pgfpathlineto{\pgfqpoint{3.149560in}{2.514906in}}%
\pgfpathclose%
\pgfusepath{fill}%
\end{pgfscope}%
\begin{pgfscope}%
\pgfpathrectangle{\pgfqpoint{0.765000in}{0.660000in}}{\pgfqpoint{4.620000in}{4.620000in}}%
\pgfusepath{clip}%
\pgfsetbuttcap%
\pgfsetroundjoin%
\definecolor{currentfill}{rgb}{1.000000,0.894118,0.788235}%
\pgfsetfillcolor{currentfill}%
\pgfsetlinewidth{0.000000pt}%
\definecolor{currentstroke}{rgb}{1.000000,0.894118,0.788235}%
\pgfsetstrokecolor{currentstroke}%
\pgfsetdash{}{0pt}%
\pgfpathmoveto{\pgfqpoint{3.149560in}{2.514906in}}%
\pgfpathlineto{\pgfqpoint{3.230519in}{2.468164in}}%
\pgfpathlineto{\pgfqpoint{3.230519in}{2.561648in}}%
\pgfpathlineto{\pgfqpoint{3.149560in}{2.608390in}}%
\pgfpathlineto{\pgfqpoint{3.149560in}{2.514906in}}%
\pgfpathclose%
\pgfusepath{fill}%
\end{pgfscope}%
\begin{pgfscope}%
\pgfpathrectangle{\pgfqpoint{0.765000in}{0.660000in}}{\pgfqpoint{4.620000in}{4.620000in}}%
\pgfusepath{clip}%
\pgfsetbuttcap%
\pgfsetroundjoin%
\definecolor{currentfill}{rgb}{1.000000,0.894118,0.788235}%
\pgfsetfillcolor{currentfill}%
\pgfsetlinewidth{0.000000pt}%
\definecolor{currentstroke}{rgb}{1.000000,0.894118,0.788235}%
\pgfsetstrokecolor{currentstroke}%
\pgfsetdash{}{0pt}%
\pgfpathmoveto{\pgfqpoint{3.337483in}{2.623404in}}%
\pgfpathlineto{\pgfqpoint{3.256524in}{2.576662in}}%
\pgfpathlineto{\pgfqpoint{3.337483in}{2.529920in}}%
\pgfpathlineto{\pgfqpoint{3.418443in}{2.576662in}}%
\pgfpathlineto{\pgfqpoint{3.337483in}{2.623404in}}%
\pgfpathclose%
\pgfusepath{fill}%
\end{pgfscope}%
\begin{pgfscope}%
\pgfpathrectangle{\pgfqpoint{0.765000in}{0.660000in}}{\pgfqpoint{4.620000in}{4.620000in}}%
\pgfusepath{clip}%
\pgfsetbuttcap%
\pgfsetroundjoin%
\definecolor{currentfill}{rgb}{1.000000,0.894118,0.788235}%
\pgfsetfillcolor{currentfill}%
\pgfsetlinewidth{0.000000pt}%
\definecolor{currentstroke}{rgb}{1.000000,0.894118,0.788235}%
\pgfsetstrokecolor{currentstroke}%
\pgfsetdash{}{0pt}%
\pgfpathmoveto{\pgfqpoint{3.337483in}{2.623404in}}%
\pgfpathlineto{\pgfqpoint{3.256524in}{2.576662in}}%
\pgfpathlineto{\pgfqpoint{3.256524in}{2.670145in}}%
\pgfpathlineto{\pgfqpoint{3.337483in}{2.716887in}}%
\pgfpathlineto{\pgfqpoint{3.337483in}{2.623404in}}%
\pgfpathclose%
\pgfusepath{fill}%
\end{pgfscope}%
\begin{pgfscope}%
\pgfpathrectangle{\pgfqpoint{0.765000in}{0.660000in}}{\pgfqpoint{4.620000in}{4.620000in}}%
\pgfusepath{clip}%
\pgfsetbuttcap%
\pgfsetroundjoin%
\definecolor{currentfill}{rgb}{1.000000,0.894118,0.788235}%
\pgfsetfillcolor{currentfill}%
\pgfsetlinewidth{0.000000pt}%
\definecolor{currentstroke}{rgb}{1.000000,0.894118,0.788235}%
\pgfsetstrokecolor{currentstroke}%
\pgfsetdash{}{0pt}%
\pgfpathmoveto{\pgfqpoint{3.337483in}{2.623404in}}%
\pgfpathlineto{\pgfqpoint{3.418443in}{2.576662in}}%
\pgfpathlineto{\pgfqpoint{3.418443in}{2.670145in}}%
\pgfpathlineto{\pgfqpoint{3.337483in}{2.716887in}}%
\pgfpathlineto{\pgfqpoint{3.337483in}{2.623404in}}%
\pgfpathclose%
\pgfusepath{fill}%
\end{pgfscope}%
\begin{pgfscope}%
\pgfpathrectangle{\pgfqpoint{0.765000in}{0.660000in}}{\pgfqpoint{4.620000in}{4.620000in}}%
\pgfusepath{clip}%
\pgfsetbuttcap%
\pgfsetroundjoin%
\definecolor{currentfill}{rgb}{1.000000,0.894118,0.788235}%
\pgfsetfillcolor{currentfill}%
\pgfsetlinewidth{0.000000pt}%
\definecolor{currentstroke}{rgb}{1.000000,0.894118,0.788235}%
\pgfsetstrokecolor{currentstroke}%
\pgfsetdash{}{0pt}%
\pgfpathmoveto{\pgfqpoint{3.149560in}{2.421422in}}%
\pgfpathlineto{\pgfqpoint{3.230519in}{2.468164in}}%
\pgfpathlineto{\pgfqpoint{3.230519in}{2.561648in}}%
\pgfpathlineto{\pgfqpoint{3.149560in}{2.514906in}}%
\pgfpathlineto{\pgfqpoint{3.149560in}{2.421422in}}%
\pgfpathclose%
\pgfusepath{fill}%
\end{pgfscope}%
\begin{pgfscope}%
\pgfpathrectangle{\pgfqpoint{0.765000in}{0.660000in}}{\pgfqpoint{4.620000in}{4.620000in}}%
\pgfusepath{clip}%
\pgfsetbuttcap%
\pgfsetroundjoin%
\definecolor{currentfill}{rgb}{1.000000,0.894118,0.788235}%
\pgfsetfillcolor{currentfill}%
\pgfsetlinewidth{0.000000pt}%
\definecolor{currentstroke}{rgb}{1.000000,0.894118,0.788235}%
\pgfsetstrokecolor{currentstroke}%
\pgfsetdash{}{0pt}%
\pgfpathmoveto{\pgfqpoint{3.337483in}{2.529920in}}%
\pgfpathlineto{\pgfqpoint{3.418443in}{2.576662in}}%
\pgfpathlineto{\pgfqpoint{3.418443in}{2.670145in}}%
\pgfpathlineto{\pgfqpoint{3.337483in}{2.623404in}}%
\pgfpathlineto{\pgfqpoint{3.337483in}{2.529920in}}%
\pgfpathclose%
\pgfusepath{fill}%
\end{pgfscope}%
\begin{pgfscope}%
\pgfpathrectangle{\pgfqpoint{0.765000in}{0.660000in}}{\pgfqpoint{4.620000in}{4.620000in}}%
\pgfusepath{clip}%
\pgfsetbuttcap%
\pgfsetroundjoin%
\definecolor{currentfill}{rgb}{1.000000,0.894118,0.788235}%
\pgfsetfillcolor{currentfill}%
\pgfsetlinewidth{0.000000pt}%
\definecolor{currentstroke}{rgb}{1.000000,0.894118,0.788235}%
\pgfsetstrokecolor{currentstroke}%
\pgfsetdash{}{0pt}%
\pgfpathmoveto{\pgfqpoint{3.149560in}{2.608390in}}%
\pgfpathlineto{\pgfqpoint{3.068601in}{2.561648in}}%
\pgfpathlineto{\pgfqpoint{3.149560in}{2.514906in}}%
\pgfpathlineto{\pgfqpoint{3.230519in}{2.561648in}}%
\pgfpathlineto{\pgfqpoint{3.149560in}{2.608390in}}%
\pgfpathclose%
\pgfusepath{fill}%
\end{pgfscope}%
\begin{pgfscope}%
\pgfpathrectangle{\pgfqpoint{0.765000in}{0.660000in}}{\pgfqpoint{4.620000in}{4.620000in}}%
\pgfusepath{clip}%
\pgfsetbuttcap%
\pgfsetroundjoin%
\definecolor{currentfill}{rgb}{1.000000,0.894118,0.788235}%
\pgfsetfillcolor{currentfill}%
\pgfsetlinewidth{0.000000pt}%
\definecolor{currentstroke}{rgb}{1.000000,0.894118,0.788235}%
\pgfsetstrokecolor{currentstroke}%
\pgfsetdash{}{0pt}%
\pgfpathmoveto{\pgfqpoint{3.068601in}{2.468164in}}%
\pgfpathlineto{\pgfqpoint{3.149560in}{2.421422in}}%
\pgfpathlineto{\pgfqpoint{3.149560in}{2.514906in}}%
\pgfpathlineto{\pgfqpoint{3.068601in}{2.561648in}}%
\pgfpathlineto{\pgfqpoint{3.068601in}{2.468164in}}%
\pgfpathclose%
\pgfusepath{fill}%
\end{pgfscope}%
\begin{pgfscope}%
\pgfpathrectangle{\pgfqpoint{0.765000in}{0.660000in}}{\pgfqpoint{4.620000in}{4.620000in}}%
\pgfusepath{clip}%
\pgfsetbuttcap%
\pgfsetroundjoin%
\definecolor{currentfill}{rgb}{1.000000,0.894118,0.788235}%
\pgfsetfillcolor{currentfill}%
\pgfsetlinewidth{0.000000pt}%
\definecolor{currentstroke}{rgb}{1.000000,0.894118,0.788235}%
\pgfsetstrokecolor{currentstroke}%
\pgfsetdash{}{0pt}%
\pgfpathmoveto{\pgfqpoint{3.337483in}{2.716887in}}%
\pgfpathlineto{\pgfqpoint{3.256524in}{2.670145in}}%
\pgfpathlineto{\pgfqpoint{3.337483in}{2.623404in}}%
\pgfpathlineto{\pgfqpoint{3.418443in}{2.670145in}}%
\pgfpathlineto{\pgfqpoint{3.337483in}{2.716887in}}%
\pgfpathclose%
\pgfusepath{fill}%
\end{pgfscope}%
\begin{pgfscope}%
\pgfpathrectangle{\pgfqpoint{0.765000in}{0.660000in}}{\pgfqpoint{4.620000in}{4.620000in}}%
\pgfusepath{clip}%
\pgfsetbuttcap%
\pgfsetroundjoin%
\definecolor{currentfill}{rgb}{1.000000,0.894118,0.788235}%
\pgfsetfillcolor{currentfill}%
\pgfsetlinewidth{0.000000pt}%
\definecolor{currentstroke}{rgb}{1.000000,0.894118,0.788235}%
\pgfsetstrokecolor{currentstroke}%
\pgfsetdash{}{0pt}%
\pgfpathmoveto{\pgfqpoint{3.256524in}{2.576662in}}%
\pgfpathlineto{\pgfqpoint{3.337483in}{2.529920in}}%
\pgfpathlineto{\pgfqpoint{3.337483in}{2.623404in}}%
\pgfpathlineto{\pgfqpoint{3.256524in}{2.670145in}}%
\pgfpathlineto{\pgfqpoint{3.256524in}{2.576662in}}%
\pgfpathclose%
\pgfusepath{fill}%
\end{pgfscope}%
\begin{pgfscope}%
\pgfpathrectangle{\pgfqpoint{0.765000in}{0.660000in}}{\pgfqpoint{4.620000in}{4.620000in}}%
\pgfusepath{clip}%
\pgfsetbuttcap%
\pgfsetroundjoin%
\definecolor{currentfill}{rgb}{1.000000,0.894118,0.788235}%
\pgfsetfillcolor{currentfill}%
\pgfsetlinewidth{0.000000pt}%
\definecolor{currentstroke}{rgb}{1.000000,0.894118,0.788235}%
\pgfsetstrokecolor{currentstroke}%
\pgfsetdash{}{0pt}%
\pgfpathmoveto{\pgfqpoint{3.149560in}{2.514906in}}%
\pgfpathlineto{\pgfqpoint{3.068601in}{2.468164in}}%
\pgfpathlineto{\pgfqpoint{3.256524in}{2.576662in}}%
\pgfpathlineto{\pgfqpoint{3.337483in}{2.623404in}}%
\pgfpathlineto{\pgfqpoint{3.149560in}{2.514906in}}%
\pgfpathclose%
\pgfusepath{fill}%
\end{pgfscope}%
\begin{pgfscope}%
\pgfpathrectangle{\pgfqpoint{0.765000in}{0.660000in}}{\pgfqpoint{4.620000in}{4.620000in}}%
\pgfusepath{clip}%
\pgfsetbuttcap%
\pgfsetroundjoin%
\definecolor{currentfill}{rgb}{1.000000,0.894118,0.788235}%
\pgfsetfillcolor{currentfill}%
\pgfsetlinewidth{0.000000pt}%
\definecolor{currentstroke}{rgb}{1.000000,0.894118,0.788235}%
\pgfsetstrokecolor{currentstroke}%
\pgfsetdash{}{0pt}%
\pgfpathmoveto{\pgfqpoint{3.230519in}{2.468164in}}%
\pgfpathlineto{\pgfqpoint{3.149560in}{2.514906in}}%
\pgfpathlineto{\pgfqpoint{3.337483in}{2.623404in}}%
\pgfpathlineto{\pgfqpoint{3.418443in}{2.576662in}}%
\pgfpathlineto{\pgfqpoint{3.230519in}{2.468164in}}%
\pgfpathclose%
\pgfusepath{fill}%
\end{pgfscope}%
\begin{pgfscope}%
\pgfpathrectangle{\pgfqpoint{0.765000in}{0.660000in}}{\pgfqpoint{4.620000in}{4.620000in}}%
\pgfusepath{clip}%
\pgfsetbuttcap%
\pgfsetroundjoin%
\definecolor{currentfill}{rgb}{1.000000,0.894118,0.788235}%
\pgfsetfillcolor{currentfill}%
\pgfsetlinewidth{0.000000pt}%
\definecolor{currentstroke}{rgb}{1.000000,0.894118,0.788235}%
\pgfsetstrokecolor{currentstroke}%
\pgfsetdash{}{0pt}%
\pgfpathmoveto{\pgfqpoint{3.149560in}{2.514906in}}%
\pgfpathlineto{\pgfqpoint{3.149560in}{2.608390in}}%
\pgfpathlineto{\pgfqpoint{3.337483in}{2.716887in}}%
\pgfpathlineto{\pgfqpoint{3.418443in}{2.576662in}}%
\pgfpathlineto{\pgfqpoint{3.149560in}{2.514906in}}%
\pgfpathclose%
\pgfusepath{fill}%
\end{pgfscope}%
\begin{pgfscope}%
\pgfpathrectangle{\pgfqpoint{0.765000in}{0.660000in}}{\pgfqpoint{4.620000in}{4.620000in}}%
\pgfusepath{clip}%
\pgfsetbuttcap%
\pgfsetroundjoin%
\definecolor{currentfill}{rgb}{1.000000,0.894118,0.788235}%
\pgfsetfillcolor{currentfill}%
\pgfsetlinewidth{0.000000pt}%
\definecolor{currentstroke}{rgb}{1.000000,0.894118,0.788235}%
\pgfsetstrokecolor{currentstroke}%
\pgfsetdash{}{0pt}%
\pgfpathmoveto{\pgfqpoint{3.230519in}{2.468164in}}%
\pgfpathlineto{\pgfqpoint{3.230519in}{2.561648in}}%
\pgfpathlineto{\pgfqpoint{3.418443in}{2.670145in}}%
\pgfpathlineto{\pgfqpoint{3.337483in}{2.623404in}}%
\pgfpathlineto{\pgfqpoint{3.230519in}{2.468164in}}%
\pgfpathclose%
\pgfusepath{fill}%
\end{pgfscope}%
\begin{pgfscope}%
\pgfpathrectangle{\pgfqpoint{0.765000in}{0.660000in}}{\pgfqpoint{4.620000in}{4.620000in}}%
\pgfusepath{clip}%
\pgfsetbuttcap%
\pgfsetroundjoin%
\definecolor{currentfill}{rgb}{1.000000,0.894118,0.788235}%
\pgfsetfillcolor{currentfill}%
\pgfsetlinewidth{0.000000pt}%
\definecolor{currentstroke}{rgb}{1.000000,0.894118,0.788235}%
\pgfsetstrokecolor{currentstroke}%
\pgfsetdash{}{0pt}%
\pgfpathmoveto{\pgfqpoint{3.068601in}{2.468164in}}%
\pgfpathlineto{\pgfqpoint{3.149560in}{2.421422in}}%
\pgfpathlineto{\pgfqpoint{3.337483in}{2.529920in}}%
\pgfpathlineto{\pgfqpoint{3.256524in}{2.576662in}}%
\pgfpathlineto{\pgfqpoint{3.068601in}{2.468164in}}%
\pgfpathclose%
\pgfusepath{fill}%
\end{pgfscope}%
\begin{pgfscope}%
\pgfpathrectangle{\pgfqpoint{0.765000in}{0.660000in}}{\pgfqpoint{4.620000in}{4.620000in}}%
\pgfusepath{clip}%
\pgfsetbuttcap%
\pgfsetroundjoin%
\definecolor{currentfill}{rgb}{1.000000,0.894118,0.788235}%
\pgfsetfillcolor{currentfill}%
\pgfsetlinewidth{0.000000pt}%
\definecolor{currentstroke}{rgb}{1.000000,0.894118,0.788235}%
\pgfsetstrokecolor{currentstroke}%
\pgfsetdash{}{0pt}%
\pgfpathmoveto{\pgfqpoint{3.149560in}{2.421422in}}%
\pgfpathlineto{\pgfqpoint{3.230519in}{2.468164in}}%
\pgfpathlineto{\pgfqpoint{3.418443in}{2.576662in}}%
\pgfpathlineto{\pgfqpoint{3.337483in}{2.529920in}}%
\pgfpathlineto{\pgfqpoint{3.149560in}{2.421422in}}%
\pgfpathclose%
\pgfusepath{fill}%
\end{pgfscope}%
\begin{pgfscope}%
\pgfpathrectangle{\pgfqpoint{0.765000in}{0.660000in}}{\pgfqpoint{4.620000in}{4.620000in}}%
\pgfusepath{clip}%
\pgfsetbuttcap%
\pgfsetroundjoin%
\definecolor{currentfill}{rgb}{1.000000,0.894118,0.788235}%
\pgfsetfillcolor{currentfill}%
\pgfsetlinewidth{0.000000pt}%
\definecolor{currentstroke}{rgb}{1.000000,0.894118,0.788235}%
\pgfsetstrokecolor{currentstroke}%
\pgfsetdash{}{0pt}%
\pgfpathmoveto{\pgfqpoint{3.149560in}{2.608390in}}%
\pgfpathlineto{\pgfqpoint{3.068601in}{2.561648in}}%
\pgfpathlineto{\pgfqpoint{3.256524in}{2.670145in}}%
\pgfpathlineto{\pgfqpoint{3.337483in}{2.716887in}}%
\pgfpathlineto{\pgfqpoint{3.149560in}{2.608390in}}%
\pgfpathclose%
\pgfusepath{fill}%
\end{pgfscope}%
\begin{pgfscope}%
\pgfpathrectangle{\pgfqpoint{0.765000in}{0.660000in}}{\pgfqpoint{4.620000in}{4.620000in}}%
\pgfusepath{clip}%
\pgfsetbuttcap%
\pgfsetroundjoin%
\definecolor{currentfill}{rgb}{1.000000,0.894118,0.788235}%
\pgfsetfillcolor{currentfill}%
\pgfsetlinewidth{0.000000pt}%
\definecolor{currentstroke}{rgb}{1.000000,0.894118,0.788235}%
\pgfsetstrokecolor{currentstroke}%
\pgfsetdash{}{0pt}%
\pgfpathmoveto{\pgfqpoint{3.230519in}{2.561648in}}%
\pgfpathlineto{\pgfqpoint{3.149560in}{2.608390in}}%
\pgfpathlineto{\pgfqpoint{3.337483in}{2.716887in}}%
\pgfpathlineto{\pgfqpoint{3.418443in}{2.670145in}}%
\pgfpathlineto{\pgfqpoint{3.230519in}{2.561648in}}%
\pgfpathclose%
\pgfusepath{fill}%
\end{pgfscope}%
\begin{pgfscope}%
\pgfpathrectangle{\pgfqpoint{0.765000in}{0.660000in}}{\pgfqpoint{4.620000in}{4.620000in}}%
\pgfusepath{clip}%
\pgfsetbuttcap%
\pgfsetroundjoin%
\definecolor{currentfill}{rgb}{1.000000,0.894118,0.788235}%
\pgfsetfillcolor{currentfill}%
\pgfsetlinewidth{0.000000pt}%
\definecolor{currentstroke}{rgb}{1.000000,0.894118,0.788235}%
\pgfsetstrokecolor{currentstroke}%
\pgfsetdash{}{0pt}%
\pgfpathmoveto{\pgfqpoint{3.068601in}{2.468164in}}%
\pgfpathlineto{\pgfqpoint{3.068601in}{2.561648in}}%
\pgfpathlineto{\pgfqpoint{3.256524in}{2.670145in}}%
\pgfpathlineto{\pgfqpoint{3.337483in}{2.529920in}}%
\pgfpathlineto{\pgfqpoint{3.068601in}{2.468164in}}%
\pgfpathclose%
\pgfusepath{fill}%
\end{pgfscope}%
\begin{pgfscope}%
\pgfpathrectangle{\pgfqpoint{0.765000in}{0.660000in}}{\pgfqpoint{4.620000in}{4.620000in}}%
\pgfusepath{clip}%
\pgfsetbuttcap%
\pgfsetroundjoin%
\definecolor{currentfill}{rgb}{1.000000,0.894118,0.788235}%
\pgfsetfillcolor{currentfill}%
\pgfsetlinewidth{0.000000pt}%
\definecolor{currentstroke}{rgb}{1.000000,0.894118,0.788235}%
\pgfsetstrokecolor{currentstroke}%
\pgfsetdash{}{0pt}%
\pgfpathmoveto{\pgfqpoint{3.149560in}{2.421422in}}%
\pgfpathlineto{\pgfqpoint{3.149560in}{2.514906in}}%
\pgfpathlineto{\pgfqpoint{3.337483in}{2.623404in}}%
\pgfpathlineto{\pgfqpoint{3.256524in}{2.576662in}}%
\pgfpathlineto{\pgfqpoint{3.149560in}{2.421422in}}%
\pgfpathclose%
\pgfusepath{fill}%
\end{pgfscope}%
\begin{pgfscope}%
\pgfpathrectangle{\pgfqpoint{0.765000in}{0.660000in}}{\pgfqpoint{4.620000in}{4.620000in}}%
\pgfusepath{clip}%
\pgfsetbuttcap%
\pgfsetroundjoin%
\definecolor{currentfill}{rgb}{1.000000,0.894118,0.788235}%
\pgfsetfillcolor{currentfill}%
\pgfsetlinewidth{0.000000pt}%
\definecolor{currentstroke}{rgb}{1.000000,0.894118,0.788235}%
\pgfsetstrokecolor{currentstroke}%
\pgfsetdash{}{0pt}%
\pgfpathmoveto{\pgfqpoint{3.068601in}{2.561648in}}%
\pgfpathlineto{\pgfqpoint{3.149560in}{2.514906in}}%
\pgfpathlineto{\pgfqpoint{3.337483in}{2.623404in}}%
\pgfpathlineto{\pgfqpoint{3.256524in}{2.670145in}}%
\pgfpathlineto{\pgfqpoint{3.068601in}{2.561648in}}%
\pgfpathclose%
\pgfusepath{fill}%
\end{pgfscope}%
\begin{pgfscope}%
\pgfpathrectangle{\pgfqpoint{0.765000in}{0.660000in}}{\pgfqpoint{4.620000in}{4.620000in}}%
\pgfusepath{clip}%
\pgfsetbuttcap%
\pgfsetroundjoin%
\definecolor{currentfill}{rgb}{1.000000,0.894118,0.788235}%
\pgfsetfillcolor{currentfill}%
\pgfsetlinewidth{0.000000pt}%
\definecolor{currentstroke}{rgb}{1.000000,0.894118,0.788235}%
\pgfsetstrokecolor{currentstroke}%
\pgfsetdash{}{0pt}%
\pgfpathmoveto{\pgfqpoint{3.149560in}{2.514906in}}%
\pgfpathlineto{\pgfqpoint{3.230519in}{2.561648in}}%
\pgfpathlineto{\pgfqpoint{3.418443in}{2.670145in}}%
\pgfpathlineto{\pgfqpoint{3.337483in}{2.623404in}}%
\pgfpathlineto{\pgfqpoint{3.149560in}{2.514906in}}%
\pgfpathclose%
\pgfusepath{fill}%
\end{pgfscope}%
\begin{pgfscope}%
\pgfpathrectangle{\pgfqpoint{0.765000in}{0.660000in}}{\pgfqpoint{4.620000in}{4.620000in}}%
\pgfusepath{clip}%
\pgfsetbuttcap%
\pgfsetroundjoin%
\definecolor{currentfill}{rgb}{1.000000,0.894118,0.788235}%
\pgfsetfillcolor{currentfill}%
\pgfsetlinewidth{0.000000pt}%
\definecolor{currentstroke}{rgb}{1.000000,0.894118,0.788235}%
\pgfsetstrokecolor{currentstroke}%
\pgfsetdash{}{0pt}%
\pgfpathmoveto{\pgfqpoint{3.025552in}{2.586502in}}%
\pgfpathlineto{\pgfqpoint{2.944593in}{2.539760in}}%
\pgfpathlineto{\pgfqpoint{3.025552in}{2.493018in}}%
\pgfpathlineto{\pgfqpoint{3.106512in}{2.539760in}}%
\pgfpathlineto{\pgfqpoint{3.025552in}{2.586502in}}%
\pgfpathclose%
\pgfusepath{fill}%
\end{pgfscope}%
\begin{pgfscope}%
\pgfpathrectangle{\pgfqpoint{0.765000in}{0.660000in}}{\pgfqpoint{4.620000in}{4.620000in}}%
\pgfusepath{clip}%
\pgfsetbuttcap%
\pgfsetroundjoin%
\definecolor{currentfill}{rgb}{1.000000,0.894118,0.788235}%
\pgfsetfillcolor{currentfill}%
\pgfsetlinewidth{0.000000pt}%
\definecolor{currentstroke}{rgb}{1.000000,0.894118,0.788235}%
\pgfsetstrokecolor{currentstroke}%
\pgfsetdash{}{0pt}%
\pgfpathmoveto{\pgfqpoint{3.025552in}{2.586502in}}%
\pgfpathlineto{\pgfqpoint{2.944593in}{2.539760in}}%
\pgfpathlineto{\pgfqpoint{2.944593in}{2.633244in}}%
\pgfpathlineto{\pgfqpoint{3.025552in}{2.679986in}}%
\pgfpathlineto{\pgfqpoint{3.025552in}{2.586502in}}%
\pgfpathclose%
\pgfusepath{fill}%
\end{pgfscope}%
\begin{pgfscope}%
\pgfpathrectangle{\pgfqpoint{0.765000in}{0.660000in}}{\pgfqpoint{4.620000in}{4.620000in}}%
\pgfusepath{clip}%
\pgfsetbuttcap%
\pgfsetroundjoin%
\definecolor{currentfill}{rgb}{1.000000,0.894118,0.788235}%
\pgfsetfillcolor{currentfill}%
\pgfsetlinewidth{0.000000pt}%
\definecolor{currentstroke}{rgb}{1.000000,0.894118,0.788235}%
\pgfsetstrokecolor{currentstroke}%
\pgfsetdash{}{0pt}%
\pgfpathmoveto{\pgfqpoint{3.025552in}{2.586502in}}%
\pgfpathlineto{\pgfqpoint{3.106512in}{2.539760in}}%
\pgfpathlineto{\pgfqpoint{3.106512in}{2.633244in}}%
\pgfpathlineto{\pgfqpoint{3.025552in}{2.679986in}}%
\pgfpathlineto{\pgfqpoint{3.025552in}{2.586502in}}%
\pgfpathclose%
\pgfusepath{fill}%
\end{pgfscope}%
\begin{pgfscope}%
\pgfpathrectangle{\pgfqpoint{0.765000in}{0.660000in}}{\pgfqpoint{4.620000in}{4.620000in}}%
\pgfusepath{clip}%
\pgfsetbuttcap%
\pgfsetroundjoin%
\definecolor{currentfill}{rgb}{1.000000,0.894118,0.788235}%
\pgfsetfillcolor{currentfill}%
\pgfsetlinewidth{0.000000pt}%
\definecolor{currentstroke}{rgb}{1.000000,0.894118,0.788235}%
\pgfsetstrokecolor{currentstroke}%
\pgfsetdash{}{0pt}%
\pgfpathmoveto{\pgfqpoint{3.149560in}{2.514906in}}%
\pgfpathlineto{\pgfqpoint{3.068601in}{2.468164in}}%
\pgfpathlineto{\pgfqpoint{3.149560in}{2.421422in}}%
\pgfpathlineto{\pgfqpoint{3.230519in}{2.468164in}}%
\pgfpathlineto{\pgfqpoint{3.149560in}{2.514906in}}%
\pgfpathclose%
\pgfusepath{fill}%
\end{pgfscope}%
\begin{pgfscope}%
\pgfpathrectangle{\pgfqpoint{0.765000in}{0.660000in}}{\pgfqpoint{4.620000in}{4.620000in}}%
\pgfusepath{clip}%
\pgfsetbuttcap%
\pgfsetroundjoin%
\definecolor{currentfill}{rgb}{1.000000,0.894118,0.788235}%
\pgfsetfillcolor{currentfill}%
\pgfsetlinewidth{0.000000pt}%
\definecolor{currentstroke}{rgb}{1.000000,0.894118,0.788235}%
\pgfsetstrokecolor{currentstroke}%
\pgfsetdash{}{0pt}%
\pgfpathmoveto{\pgfqpoint{3.149560in}{2.514906in}}%
\pgfpathlineto{\pgfqpoint{3.068601in}{2.468164in}}%
\pgfpathlineto{\pgfqpoint{3.068601in}{2.561648in}}%
\pgfpathlineto{\pgfqpoint{3.149560in}{2.608390in}}%
\pgfpathlineto{\pgfqpoint{3.149560in}{2.514906in}}%
\pgfpathclose%
\pgfusepath{fill}%
\end{pgfscope}%
\begin{pgfscope}%
\pgfpathrectangle{\pgfqpoint{0.765000in}{0.660000in}}{\pgfqpoint{4.620000in}{4.620000in}}%
\pgfusepath{clip}%
\pgfsetbuttcap%
\pgfsetroundjoin%
\definecolor{currentfill}{rgb}{1.000000,0.894118,0.788235}%
\pgfsetfillcolor{currentfill}%
\pgfsetlinewidth{0.000000pt}%
\definecolor{currentstroke}{rgb}{1.000000,0.894118,0.788235}%
\pgfsetstrokecolor{currentstroke}%
\pgfsetdash{}{0pt}%
\pgfpathmoveto{\pgfqpoint{3.149560in}{2.514906in}}%
\pgfpathlineto{\pgfqpoint{3.230519in}{2.468164in}}%
\pgfpathlineto{\pgfqpoint{3.230519in}{2.561648in}}%
\pgfpathlineto{\pgfqpoint{3.149560in}{2.608390in}}%
\pgfpathlineto{\pgfqpoint{3.149560in}{2.514906in}}%
\pgfpathclose%
\pgfusepath{fill}%
\end{pgfscope}%
\begin{pgfscope}%
\pgfpathrectangle{\pgfqpoint{0.765000in}{0.660000in}}{\pgfqpoint{4.620000in}{4.620000in}}%
\pgfusepath{clip}%
\pgfsetbuttcap%
\pgfsetroundjoin%
\definecolor{currentfill}{rgb}{1.000000,0.894118,0.788235}%
\pgfsetfillcolor{currentfill}%
\pgfsetlinewidth{0.000000pt}%
\definecolor{currentstroke}{rgb}{1.000000,0.894118,0.788235}%
\pgfsetstrokecolor{currentstroke}%
\pgfsetdash{}{0pt}%
\pgfpathmoveto{\pgfqpoint{3.025552in}{2.493018in}}%
\pgfpathlineto{\pgfqpoint{3.106512in}{2.539760in}}%
\pgfpathlineto{\pgfqpoint{3.106512in}{2.633244in}}%
\pgfpathlineto{\pgfqpoint{3.025552in}{2.586502in}}%
\pgfpathlineto{\pgfqpoint{3.025552in}{2.493018in}}%
\pgfpathclose%
\pgfusepath{fill}%
\end{pgfscope}%
\begin{pgfscope}%
\pgfpathrectangle{\pgfqpoint{0.765000in}{0.660000in}}{\pgfqpoint{4.620000in}{4.620000in}}%
\pgfusepath{clip}%
\pgfsetbuttcap%
\pgfsetroundjoin%
\definecolor{currentfill}{rgb}{1.000000,0.894118,0.788235}%
\pgfsetfillcolor{currentfill}%
\pgfsetlinewidth{0.000000pt}%
\definecolor{currentstroke}{rgb}{1.000000,0.894118,0.788235}%
\pgfsetstrokecolor{currentstroke}%
\pgfsetdash{}{0pt}%
\pgfpathmoveto{\pgfqpoint{3.149560in}{2.421422in}}%
\pgfpathlineto{\pgfqpoint{3.230519in}{2.468164in}}%
\pgfpathlineto{\pgfqpoint{3.230519in}{2.561648in}}%
\pgfpathlineto{\pgfqpoint{3.149560in}{2.514906in}}%
\pgfpathlineto{\pgfqpoint{3.149560in}{2.421422in}}%
\pgfpathclose%
\pgfusepath{fill}%
\end{pgfscope}%
\begin{pgfscope}%
\pgfpathrectangle{\pgfqpoint{0.765000in}{0.660000in}}{\pgfqpoint{4.620000in}{4.620000in}}%
\pgfusepath{clip}%
\pgfsetbuttcap%
\pgfsetroundjoin%
\definecolor{currentfill}{rgb}{1.000000,0.894118,0.788235}%
\pgfsetfillcolor{currentfill}%
\pgfsetlinewidth{0.000000pt}%
\definecolor{currentstroke}{rgb}{1.000000,0.894118,0.788235}%
\pgfsetstrokecolor{currentstroke}%
\pgfsetdash{}{0pt}%
\pgfpathmoveto{\pgfqpoint{3.025552in}{2.679986in}}%
\pgfpathlineto{\pgfqpoint{2.944593in}{2.633244in}}%
\pgfpathlineto{\pgfqpoint{3.025552in}{2.586502in}}%
\pgfpathlineto{\pgfqpoint{3.106512in}{2.633244in}}%
\pgfpathlineto{\pgfqpoint{3.025552in}{2.679986in}}%
\pgfpathclose%
\pgfusepath{fill}%
\end{pgfscope}%
\begin{pgfscope}%
\pgfpathrectangle{\pgfqpoint{0.765000in}{0.660000in}}{\pgfqpoint{4.620000in}{4.620000in}}%
\pgfusepath{clip}%
\pgfsetbuttcap%
\pgfsetroundjoin%
\definecolor{currentfill}{rgb}{1.000000,0.894118,0.788235}%
\pgfsetfillcolor{currentfill}%
\pgfsetlinewidth{0.000000pt}%
\definecolor{currentstroke}{rgb}{1.000000,0.894118,0.788235}%
\pgfsetstrokecolor{currentstroke}%
\pgfsetdash{}{0pt}%
\pgfpathmoveto{\pgfqpoint{2.944593in}{2.539760in}}%
\pgfpathlineto{\pgfqpoint{3.025552in}{2.493018in}}%
\pgfpathlineto{\pgfqpoint{3.025552in}{2.586502in}}%
\pgfpathlineto{\pgfqpoint{2.944593in}{2.633244in}}%
\pgfpathlineto{\pgfqpoint{2.944593in}{2.539760in}}%
\pgfpathclose%
\pgfusepath{fill}%
\end{pgfscope}%
\begin{pgfscope}%
\pgfpathrectangle{\pgfqpoint{0.765000in}{0.660000in}}{\pgfqpoint{4.620000in}{4.620000in}}%
\pgfusepath{clip}%
\pgfsetbuttcap%
\pgfsetroundjoin%
\definecolor{currentfill}{rgb}{1.000000,0.894118,0.788235}%
\pgfsetfillcolor{currentfill}%
\pgfsetlinewidth{0.000000pt}%
\definecolor{currentstroke}{rgb}{1.000000,0.894118,0.788235}%
\pgfsetstrokecolor{currentstroke}%
\pgfsetdash{}{0pt}%
\pgfpathmoveto{\pgfqpoint{3.149560in}{2.608390in}}%
\pgfpathlineto{\pgfqpoint{3.068601in}{2.561648in}}%
\pgfpathlineto{\pgfqpoint{3.149560in}{2.514906in}}%
\pgfpathlineto{\pgfqpoint{3.230519in}{2.561648in}}%
\pgfpathlineto{\pgfqpoint{3.149560in}{2.608390in}}%
\pgfpathclose%
\pgfusepath{fill}%
\end{pgfscope}%
\begin{pgfscope}%
\pgfpathrectangle{\pgfqpoint{0.765000in}{0.660000in}}{\pgfqpoint{4.620000in}{4.620000in}}%
\pgfusepath{clip}%
\pgfsetbuttcap%
\pgfsetroundjoin%
\definecolor{currentfill}{rgb}{1.000000,0.894118,0.788235}%
\pgfsetfillcolor{currentfill}%
\pgfsetlinewidth{0.000000pt}%
\definecolor{currentstroke}{rgb}{1.000000,0.894118,0.788235}%
\pgfsetstrokecolor{currentstroke}%
\pgfsetdash{}{0pt}%
\pgfpathmoveto{\pgfqpoint{3.068601in}{2.468164in}}%
\pgfpathlineto{\pgfqpoint{3.149560in}{2.421422in}}%
\pgfpathlineto{\pgfqpoint{3.149560in}{2.514906in}}%
\pgfpathlineto{\pgfqpoint{3.068601in}{2.561648in}}%
\pgfpathlineto{\pgfqpoint{3.068601in}{2.468164in}}%
\pgfpathclose%
\pgfusepath{fill}%
\end{pgfscope}%
\begin{pgfscope}%
\pgfpathrectangle{\pgfqpoint{0.765000in}{0.660000in}}{\pgfqpoint{4.620000in}{4.620000in}}%
\pgfusepath{clip}%
\pgfsetbuttcap%
\pgfsetroundjoin%
\definecolor{currentfill}{rgb}{1.000000,0.894118,0.788235}%
\pgfsetfillcolor{currentfill}%
\pgfsetlinewidth{0.000000pt}%
\definecolor{currentstroke}{rgb}{1.000000,0.894118,0.788235}%
\pgfsetstrokecolor{currentstroke}%
\pgfsetdash{}{0pt}%
\pgfpathmoveto{\pgfqpoint{3.025552in}{2.586502in}}%
\pgfpathlineto{\pgfqpoint{2.944593in}{2.539760in}}%
\pgfpathlineto{\pgfqpoint{3.068601in}{2.468164in}}%
\pgfpathlineto{\pgfqpoint{3.149560in}{2.514906in}}%
\pgfpathlineto{\pgfqpoint{3.025552in}{2.586502in}}%
\pgfpathclose%
\pgfusepath{fill}%
\end{pgfscope}%
\begin{pgfscope}%
\pgfpathrectangle{\pgfqpoint{0.765000in}{0.660000in}}{\pgfqpoint{4.620000in}{4.620000in}}%
\pgfusepath{clip}%
\pgfsetbuttcap%
\pgfsetroundjoin%
\definecolor{currentfill}{rgb}{1.000000,0.894118,0.788235}%
\pgfsetfillcolor{currentfill}%
\pgfsetlinewidth{0.000000pt}%
\definecolor{currentstroke}{rgb}{1.000000,0.894118,0.788235}%
\pgfsetstrokecolor{currentstroke}%
\pgfsetdash{}{0pt}%
\pgfpathmoveto{\pgfqpoint{3.106512in}{2.539760in}}%
\pgfpathlineto{\pgfqpoint{3.025552in}{2.586502in}}%
\pgfpathlineto{\pgfqpoint{3.149560in}{2.514906in}}%
\pgfpathlineto{\pgfqpoint{3.230519in}{2.468164in}}%
\pgfpathlineto{\pgfqpoint{3.106512in}{2.539760in}}%
\pgfpathclose%
\pgfusepath{fill}%
\end{pgfscope}%
\begin{pgfscope}%
\pgfpathrectangle{\pgfqpoint{0.765000in}{0.660000in}}{\pgfqpoint{4.620000in}{4.620000in}}%
\pgfusepath{clip}%
\pgfsetbuttcap%
\pgfsetroundjoin%
\definecolor{currentfill}{rgb}{1.000000,0.894118,0.788235}%
\pgfsetfillcolor{currentfill}%
\pgfsetlinewidth{0.000000pt}%
\definecolor{currentstroke}{rgb}{1.000000,0.894118,0.788235}%
\pgfsetstrokecolor{currentstroke}%
\pgfsetdash{}{0pt}%
\pgfpathmoveto{\pgfqpoint{3.025552in}{2.586502in}}%
\pgfpathlineto{\pgfqpoint{3.025552in}{2.679986in}}%
\pgfpathlineto{\pgfqpoint{3.149560in}{2.608390in}}%
\pgfpathlineto{\pgfqpoint{3.230519in}{2.468164in}}%
\pgfpathlineto{\pgfqpoint{3.025552in}{2.586502in}}%
\pgfpathclose%
\pgfusepath{fill}%
\end{pgfscope}%
\begin{pgfscope}%
\pgfpathrectangle{\pgfqpoint{0.765000in}{0.660000in}}{\pgfqpoint{4.620000in}{4.620000in}}%
\pgfusepath{clip}%
\pgfsetbuttcap%
\pgfsetroundjoin%
\definecolor{currentfill}{rgb}{1.000000,0.894118,0.788235}%
\pgfsetfillcolor{currentfill}%
\pgfsetlinewidth{0.000000pt}%
\definecolor{currentstroke}{rgb}{1.000000,0.894118,0.788235}%
\pgfsetstrokecolor{currentstroke}%
\pgfsetdash{}{0pt}%
\pgfpathmoveto{\pgfqpoint{3.106512in}{2.539760in}}%
\pgfpathlineto{\pgfqpoint{3.106512in}{2.633244in}}%
\pgfpathlineto{\pgfqpoint{3.230519in}{2.561648in}}%
\pgfpathlineto{\pgfqpoint{3.149560in}{2.514906in}}%
\pgfpathlineto{\pgfqpoint{3.106512in}{2.539760in}}%
\pgfpathclose%
\pgfusepath{fill}%
\end{pgfscope}%
\begin{pgfscope}%
\pgfpathrectangle{\pgfqpoint{0.765000in}{0.660000in}}{\pgfqpoint{4.620000in}{4.620000in}}%
\pgfusepath{clip}%
\pgfsetbuttcap%
\pgfsetroundjoin%
\definecolor{currentfill}{rgb}{1.000000,0.894118,0.788235}%
\pgfsetfillcolor{currentfill}%
\pgfsetlinewidth{0.000000pt}%
\definecolor{currentstroke}{rgb}{1.000000,0.894118,0.788235}%
\pgfsetstrokecolor{currentstroke}%
\pgfsetdash{}{0pt}%
\pgfpathmoveto{\pgfqpoint{2.944593in}{2.539760in}}%
\pgfpathlineto{\pgfqpoint{3.025552in}{2.493018in}}%
\pgfpathlineto{\pgfqpoint{3.149560in}{2.421422in}}%
\pgfpathlineto{\pgfqpoint{3.068601in}{2.468164in}}%
\pgfpathlineto{\pgfqpoint{2.944593in}{2.539760in}}%
\pgfpathclose%
\pgfusepath{fill}%
\end{pgfscope}%
\begin{pgfscope}%
\pgfpathrectangle{\pgfqpoint{0.765000in}{0.660000in}}{\pgfqpoint{4.620000in}{4.620000in}}%
\pgfusepath{clip}%
\pgfsetbuttcap%
\pgfsetroundjoin%
\definecolor{currentfill}{rgb}{1.000000,0.894118,0.788235}%
\pgfsetfillcolor{currentfill}%
\pgfsetlinewidth{0.000000pt}%
\definecolor{currentstroke}{rgb}{1.000000,0.894118,0.788235}%
\pgfsetstrokecolor{currentstroke}%
\pgfsetdash{}{0pt}%
\pgfpathmoveto{\pgfqpoint{3.025552in}{2.493018in}}%
\pgfpathlineto{\pgfqpoint{3.106512in}{2.539760in}}%
\pgfpathlineto{\pgfqpoint{3.230519in}{2.468164in}}%
\pgfpathlineto{\pgfqpoint{3.149560in}{2.421422in}}%
\pgfpathlineto{\pgfqpoint{3.025552in}{2.493018in}}%
\pgfpathclose%
\pgfusepath{fill}%
\end{pgfscope}%
\begin{pgfscope}%
\pgfpathrectangle{\pgfqpoint{0.765000in}{0.660000in}}{\pgfqpoint{4.620000in}{4.620000in}}%
\pgfusepath{clip}%
\pgfsetbuttcap%
\pgfsetroundjoin%
\definecolor{currentfill}{rgb}{1.000000,0.894118,0.788235}%
\pgfsetfillcolor{currentfill}%
\pgfsetlinewidth{0.000000pt}%
\definecolor{currentstroke}{rgb}{1.000000,0.894118,0.788235}%
\pgfsetstrokecolor{currentstroke}%
\pgfsetdash{}{0pt}%
\pgfpathmoveto{\pgfqpoint{3.025552in}{2.679986in}}%
\pgfpathlineto{\pgfqpoint{2.944593in}{2.633244in}}%
\pgfpathlineto{\pgfqpoint{3.068601in}{2.561648in}}%
\pgfpathlineto{\pgfqpoint{3.149560in}{2.608390in}}%
\pgfpathlineto{\pgfqpoint{3.025552in}{2.679986in}}%
\pgfpathclose%
\pgfusepath{fill}%
\end{pgfscope}%
\begin{pgfscope}%
\pgfpathrectangle{\pgfqpoint{0.765000in}{0.660000in}}{\pgfqpoint{4.620000in}{4.620000in}}%
\pgfusepath{clip}%
\pgfsetbuttcap%
\pgfsetroundjoin%
\definecolor{currentfill}{rgb}{1.000000,0.894118,0.788235}%
\pgfsetfillcolor{currentfill}%
\pgfsetlinewidth{0.000000pt}%
\definecolor{currentstroke}{rgb}{1.000000,0.894118,0.788235}%
\pgfsetstrokecolor{currentstroke}%
\pgfsetdash{}{0pt}%
\pgfpathmoveto{\pgfqpoint{3.106512in}{2.633244in}}%
\pgfpathlineto{\pgfqpoint{3.025552in}{2.679986in}}%
\pgfpathlineto{\pgfqpoint{3.149560in}{2.608390in}}%
\pgfpathlineto{\pgfqpoint{3.230519in}{2.561648in}}%
\pgfpathlineto{\pgfqpoint{3.106512in}{2.633244in}}%
\pgfpathclose%
\pgfusepath{fill}%
\end{pgfscope}%
\begin{pgfscope}%
\pgfpathrectangle{\pgfqpoint{0.765000in}{0.660000in}}{\pgfqpoint{4.620000in}{4.620000in}}%
\pgfusepath{clip}%
\pgfsetbuttcap%
\pgfsetroundjoin%
\definecolor{currentfill}{rgb}{1.000000,0.894118,0.788235}%
\pgfsetfillcolor{currentfill}%
\pgfsetlinewidth{0.000000pt}%
\definecolor{currentstroke}{rgb}{1.000000,0.894118,0.788235}%
\pgfsetstrokecolor{currentstroke}%
\pgfsetdash{}{0pt}%
\pgfpathmoveto{\pgfqpoint{2.944593in}{2.539760in}}%
\pgfpathlineto{\pgfqpoint{2.944593in}{2.633244in}}%
\pgfpathlineto{\pgfqpoint{3.068601in}{2.561648in}}%
\pgfpathlineto{\pgfqpoint{3.149560in}{2.421422in}}%
\pgfpathlineto{\pgfqpoint{2.944593in}{2.539760in}}%
\pgfpathclose%
\pgfusepath{fill}%
\end{pgfscope}%
\begin{pgfscope}%
\pgfpathrectangle{\pgfqpoint{0.765000in}{0.660000in}}{\pgfqpoint{4.620000in}{4.620000in}}%
\pgfusepath{clip}%
\pgfsetbuttcap%
\pgfsetroundjoin%
\definecolor{currentfill}{rgb}{1.000000,0.894118,0.788235}%
\pgfsetfillcolor{currentfill}%
\pgfsetlinewidth{0.000000pt}%
\definecolor{currentstroke}{rgb}{1.000000,0.894118,0.788235}%
\pgfsetstrokecolor{currentstroke}%
\pgfsetdash{}{0pt}%
\pgfpathmoveto{\pgfqpoint{3.025552in}{2.493018in}}%
\pgfpathlineto{\pgfqpoint{3.025552in}{2.586502in}}%
\pgfpathlineto{\pgfqpoint{3.149560in}{2.514906in}}%
\pgfpathlineto{\pgfqpoint{3.068601in}{2.468164in}}%
\pgfpathlineto{\pgfqpoint{3.025552in}{2.493018in}}%
\pgfpathclose%
\pgfusepath{fill}%
\end{pgfscope}%
\begin{pgfscope}%
\pgfpathrectangle{\pgfqpoint{0.765000in}{0.660000in}}{\pgfqpoint{4.620000in}{4.620000in}}%
\pgfusepath{clip}%
\pgfsetbuttcap%
\pgfsetroundjoin%
\definecolor{currentfill}{rgb}{1.000000,0.894118,0.788235}%
\pgfsetfillcolor{currentfill}%
\pgfsetlinewidth{0.000000pt}%
\definecolor{currentstroke}{rgb}{1.000000,0.894118,0.788235}%
\pgfsetstrokecolor{currentstroke}%
\pgfsetdash{}{0pt}%
\pgfpathmoveto{\pgfqpoint{2.944593in}{2.633244in}}%
\pgfpathlineto{\pgfqpoint{3.025552in}{2.586502in}}%
\pgfpathlineto{\pgfqpoint{3.149560in}{2.514906in}}%
\pgfpathlineto{\pgfqpoint{3.068601in}{2.561648in}}%
\pgfpathlineto{\pgfqpoint{2.944593in}{2.633244in}}%
\pgfpathclose%
\pgfusepath{fill}%
\end{pgfscope}%
\begin{pgfscope}%
\pgfpathrectangle{\pgfqpoint{0.765000in}{0.660000in}}{\pgfqpoint{4.620000in}{4.620000in}}%
\pgfusepath{clip}%
\pgfsetbuttcap%
\pgfsetroundjoin%
\definecolor{currentfill}{rgb}{1.000000,0.894118,0.788235}%
\pgfsetfillcolor{currentfill}%
\pgfsetlinewidth{0.000000pt}%
\definecolor{currentstroke}{rgb}{1.000000,0.894118,0.788235}%
\pgfsetstrokecolor{currentstroke}%
\pgfsetdash{}{0pt}%
\pgfpathmoveto{\pgfqpoint{3.025552in}{2.586502in}}%
\pgfpathlineto{\pgfqpoint{3.106512in}{2.633244in}}%
\pgfpathlineto{\pgfqpoint{3.230519in}{2.561648in}}%
\pgfpathlineto{\pgfqpoint{3.149560in}{2.514906in}}%
\pgfpathlineto{\pgfqpoint{3.025552in}{2.586502in}}%
\pgfpathclose%
\pgfusepath{fill}%
\end{pgfscope}%
\begin{pgfscope}%
\pgfpathrectangle{\pgfqpoint{0.765000in}{0.660000in}}{\pgfqpoint{4.620000in}{4.620000in}}%
\pgfusepath{clip}%
\pgfsetbuttcap%
\pgfsetroundjoin%
\definecolor{currentfill}{rgb}{1.000000,0.894118,0.788235}%
\pgfsetfillcolor{currentfill}%
\pgfsetlinewidth{0.000000pt}%
\definecolor{currentstroke}{rgb}{1.000000,0.894118,0.788235}%
\pgfsetstrokecolor{currentstroke}%
\pgfsetdash{}{0pt}%
\pgfpathmoveto{\pgfqpoint{3.025552in}{2.586502in}}%
\pgfpathlineto{\pgfqpoint{2.944593in}{2.539760in}}%
\pgfpathlineto{\pgfqpoint{3.025552in}{2.493018in}}%
\pgfpathlineto{\pgfqpoint{3.106512in}{2.539760in}}%
\pgfpathlineto{\pgfqpoint{3.025552in}{2.586502in}}%
\pgfpathclose%
\pgfusepath{fill}%
\end{pgfscope}%
\begin{pgfscope}%
\pgfpathrectangle{\pgfqpoint{0.765000in}{0.660000in}}{\pgfqpoint{4.620000in}{4.620000in}}%
\pgfusepath{clip}%
\pgfsetbuttcap%
\pgfsetroundjoin%
\definecolor{currentfill}{rgb}{1.000000,0.894118,0.788235}%
\pgfsetfillcolor{currentfill}%
\pgfsetlinewidth{0.000000pt}%
\definecolor{currentstroke}{rgb}{1.000000,0.894118,0.788235}%
\pgfsetstrokecolor{currentstroke}%
\pgfsetdash{}{0pt}%
\pgfpathmoveto{\pgfqpoint{3.025552in}{2.586502in}}%
\pgfpathlineto{\pgfqpoint{2.944593in}{2.539760in}}%
\pgfpathlineto{\pgfqpoint{2.944593in}{2.633244in}}%
\pgfpathlineto{\pgfqpoint{3.025552in}{2.679986in}}%
\pgfpathlineto{\pgfqpoint{3.025552in}{2.586502in}}%
\pgfpathclose%
\pgfusepath{fill}%
\end{pgfscope}%
\begin{pgfscope}%
\pgfpathrectangle{\pgfqpoint{0.765000in}{0.660000in}}{\pgfqpoint{4.620000in}{4.620000in}}%
\pgfusepath{clip}%
\pgfsetbuttcap%
\pgfsetroundjoin%
\definecolor{currentfill}{rgb}{1.000000,0.894118,0.788235}%
\pgfsetfillcolor{currentfill}%
\pgfsetlinewidth{0.000000pt}%
\definecolor{currentstroke}{rgb}{1.000000,0.894118,0.788235}%
\pgfsetstrokecolor{currentstroke}%
\pgfsetdash{}{0pt}%
\pgfpathmoveto{\pgfqpoint{3.025552in}{2.586502in}}%
\pgfpathlineto{\pgfqpoint{3.106512in}{2.539760in}}%
\pgfpathlineto{\pgfqpoint{3.106512in}{2.633244in}}%
\pgfpathlineto{\pgfqpoint{3.025552in}{2.679986in}}%
\pgfpathlineto{\pgfqpoint{3.025552in}{2.586502in}}%
\pgfpathclose%
\pgfusepath{fill}%
\end{pgfscope}%
\begin{pgfscope}%
\pgfpathrectangle{\pgfqpoint{0.765000in}{0.660000in}}{\pgfqpoint{4.620000in}{4.620000in}}%
\pgfusepath{clip}%
\pgfsetbuttcap%
\pgfsetroundjoin%
\definecolor{currentfill}{rgb}{1.000000,0.894118,0.788235}%
\pgfsetfillcolor{currentfill}%
\pgfsetlinewidth{0.000000pt}%
\definecolor{currentstroke}{rgb}{1.000000,0.894118,0.788235}%
\pgfsetstrokecolor{currentstroke}%
\pgfsetdash{}{0pt}%
\pgfpathmoveto{\pgfqpoint{2.654247in}{3.229621in}}%
\pgfpathlineto{\pgfqpoint{2.573288in}{3.182879in}}%
\pgfpathlineto{\pgfqpoint{2.654247in}{3.136137in}}%
\pgfpathlineto{\pgfqpoint{2.735207in}{3.182879in}}%
\pgfpathlineto{\pgfqpoint{2.654247in}{3.229621in}}%
\pgfpathclose%
\pgfusepath{fill}%
\end{pgfscope}%
\begin{pgfscope}%
\pgfpathrectangle{\pgfqpoint{0.765000in}{0.660000in}}{\pgfqpoint{4.620000in}{4.620000in}}%
\pgfusepath{clip}%
\pgfsetbuttcap%
\pgfsetroundjoin%
\definecolor{currentfill}{rgb}{1.000000,0.894118,0.788235}%
\pgfsetfillcolor{currentfill}%
\pgfsetlinewidth{0.000000pt}%
\definecolor{currentstroke}{rgb}{1.000000,0.894118,0.788235}%
\pgfsetstrokecolor{currentstroke}%
\pgfsetdash{}{0pt}%
\pgfpathmoveto{\pgfqpoint{2.654247in}{3.229621in}}%
\pgfpathlineto{\pgfqpoint{2.573288in}{3.182879in}}%
\pgfpathlineto{\pgfqpoint{2.573288in}{3.276363in}}%
\pgfpathlineto{\pgfqpoint{2.654247in}{3.323105in}}%
\pgfpathlineto{\pgfqpoint{2.654247in}{3.229621in}}%
\pgfpathclose%
\pgfusepath{fill}%
\end{pgfscope}%
\begin{pgfscope}%
\pgfpathrectangle{\pgfqpoint{0.765000in}{0.660000in}}{\pgfqpoint{4.620000in}{4.620000in}}%
\pgfusepath{clip}%
\pgfsetbuttcap%
\pgfsetroundjoin%
\definecolor{currentfill}{rgb}{1.000000,0.894118,0.788235}%
\pgfsetfillcolor{currentfill}%
\pgfsetlinewidth{0.000000pt}%
\definecolor{currentstroke}{rgb}{1.000000,0.894118,0.788235}%
\pgfsetstrokecolor{currentstroke}%
\pgfsetdash{}{0pt}%
\pgfpathmoveto{\pgfqpoint{2.654247in}{3.229621in}}%
\pgfpathlineto{\pgfqpoint{2.735207in}{3.182879in}}%
\pgfpathlineto{\pgfqpoint{2.735207in}{3.276363in}}%
\pgfpathlineto{\pgfqpoint{2.654247in}{3.323105in}}%
\pgfpathlineto{\pgfqpoint{2.654247in}{3.229621in}}%
\pgfpathclose%
\pgfusepath{fill}%
\end{pgfscope}%
\begin{pgfscope}%
\pgfpathrectangle{\pgfqpoint{0.765000in}{0.660000in}}{\pgfqpoint{4.620000in}{4.620000in}}%
\pgfusepath{clip}%
\pgfsetbuttcap%
\pgfsetroundjoin%
\definecolor{currentfill}{rgb}{1.000000,0.894118,0.788235}%
\pgfsetfillcolor{currentfill}%
\pgfsetlinewidth{0.000000pt}%
\definecolor{currentstroke}{rgb}{1.000000,0.894118,0.788235}%
\pgfsetstrokecolor{currentstroke}%
\pgfsetdash{}{0pt}%
\pgfpathmoveto{\pgfqpoint{3.025552in}{2.493018in}}%
\pgfpathlineto{\pgfqpoint{3.106512in}{2.539760in}}%
\pgfpathlineto{\pgfqpoint{3.106512in}{2.633244in}}%
\pgfpathlineto{\pgfqpoint{3.025552in}{2.586502in}}%
\pgfpathlineto{\pgfqpoint{3.025552in}{2.493018in}}%
\pgfpathclose%
\pgfusepath{fill}%
\end{pgfscope}%
\begin{pgfscope}%
\pgfpathrectangle{\pgfqpoint{0.765000in}{0.660000in}}{\pgfqpoint{4.620000in}{4.620000in}}%
\pgfusepath{clip}%
\pgfsetbuttcap%
\pgfsetroundjoin%
\definecolor{currentfill}{rgb}{1.000000,0.894118,0.788235}%
\pgfsetfillcolor{currentfill}%
\pgfsetlinewidth{0.000000pt}%
\definecolor{currentstroke}{rgb}{1.000000,0.894118,0.788235}%
\pgfsetstrokecolor{currentstroke}%
\pgfsetdash{}{0pt}%
\pgfpathmoveto{\pgfqpoint{2.654247in}{3.136137in}}%
\pgfpathlineto{\pgfqpoint{2.735207in}{3.182879in}}%
\pgfpathlineto{\pgfqpoint{2.735207in}{3.276363in}}%
\pgfpathlineto{\pgfqpoint{2.654247in}{3.229621in}}%
\pgfpathlineto{\pgfqpoint{2.654247in}{3.136137in}}%
\pgfpathclose%
\pgfusepath{fill}%
\end{pgfscope}%
\begin{pgfscope}%
\pgfpathrectangle{\pgfqpoint{0.765000in}{0.660000in}}{\pgfqpoint{4.620000in}{4.620000in}}%
\pgfusepath{clip}%
\pgfsetbuttcap%
\pgfsetroundjoin%
\definecolor{currentfill}{rgb}{1.000000,0.894118,0.788235}%
\pgfsetfillcolor{currentfill}%
\pgfsetlinewidth{0.000000pt}%
\definecolor{currentstroke}{rgb}{1.000000,0.894118,0.788235}%
\pgfsetstrokecolor{currentstroke}%
\pgfsetdash{}{0pt}%
\pgfpathmoveto{\pgfqpoint{3.025552in}{2.679986in}}%
\pgfpathlineto{\pgfqpoint{2.944593in}{2.633244in}}%
\pgfpathlineto{\pgfqpoint{3.025552in}{2.586502in}}%
\pgfpathlineto{\pgfqpoint{3.106512in}{2.633244in}}%
\pgfpathlineto{\pgfqpoint{3.025552in}{2.679986in}}%
\pgfpathclose%
\pgfusepath{fill}%
\end{pgfscope}%
\begin{pgfscope}%
\pgfpathrectangle{\pgfqpoint{0.765000in}{0.660000in}}{\pgfqpoint{4.620000in}{4.620000in}}%
\pgfusepath{clip}%
\pgfsetbuttcap%
\pgfsetroundjoin%
\definecolor{currentfill}{rgb}{1.000000,0.894118,0.788235}%
\pgfsetfillcolor{currentfill}%
\pgfsetlinewidth{0.000000pt}%
\definecolor{currentstroke}{rgb}{1.000000,0.894118,0.788235}%
\pgfsetstrokecolor{currentstroke}%
\pgfsetdash{}{0pt}%
\pgfpathmoveto{\pgfqpoint{2.944593in}{2.539760in}}%
\pgfpathlineto{\pgfqpoint{3.025552in}{2.493018in}}%
\pgfpathlineto{\pgfqpoint{3.025552in}{2.586502in}}%
\pgfpathlineto{\pgfqpoint{2.944593in}{2.633244in}}%
\pgfpathlineto{\pgfqpoint{2.944593in}{2.539760in}}%
\pgfpathclose%
\pgfusepath{fill}%
\end{pgfscope}%
\begin{pgfscope}%
\pgfpathrectangle{\pgfqpoint{0.765000in}{0.660000in}}{\pgfqpoint{4.620000in}{4.620000in}}%
\pgfusepath{clip}%
\pgfsetbuttcap%
\pgfsetroundjoin%
\definecolor{currentfill}{rgb}{1.000000,0.894118,0.788235}%
\pgfsetfillcolor{currentfill}%
\pgfsetlinewidth{0.000000pt}%
\definecolor{currentstroke}{rgb}{1.000000,0.894118,0.788235}%
\pgfsetstrokecolor{currentstroke}%
\pgfsetdash{}{0pt}%
\pgfpathmoveto{\pgfqpoint{2.654247in}{3.323105in}}%
\pgfpathlineto{\pgfqpoint{2.573288in}{3.276363in}}%
\pgfpathlineto{\pgfqpoint{2.654247in}{3.229621in}}%
\pgfpathlineto{\pgfqpoint{2.735207in}{3.276363in}}%
\pgfpathlineto{\pgfqpoint{2.654247in}{3.323105in}}%
\pgfpathclose%
\pgfusepath{fill}%
\end{pgfscope}%
\begin{pgfscope}%
\pgfpathrectangle{\pgfqpoint{0.765000in}{0.660000in}}{\pgfqpoint{4.620000in}{4.620000in}}%
\pgfusepath{clip}%
\pgfsetbuttcap%
\pgfsetroundjoin%
\definecolor{currentfill}{rgb}{1.000000,0.894118,0.788235}%
\pgfsetfillcolor{currentfill}%
\pgfsetlinewidth{0.000000pt}%
\definecolor{currentstroke}{rgb}{1.000000,0.894118,0.788235}%
\pgfsetstrokecolor{currentstroke}%
\pgfsetdash{}{0pt}%
\pgfpathmoveto{\pgfqpoint{2.573288in}{3.182879in}}%
\pgfpathlineto{\pgfqpoint{2.654247in}{3.136137in}}%
\pgfpathlineto{\pgfqpoint{2.654247in}{3.229621in}}%
\pgfpathlineto{\pgfqpoint{2.573288in}{3.276363in}}%
\pgfpathlineto{\pgfqpoint{2.573288in}{3.182879in}}%
\pgfpathclose%
\pgfusepath{fill}%
\end{pgfscope}%
\begin{pgfscope}%
\pgfpathrectangle{\pgfqpoint{0.765000in}{0.660000in}}{\pgfqpoint{4.620000in}{4.620000in}}%
\pgfusepath{clip}%
\pgfsetbuttcap%
\pgfsetroundjoin%
\definecolor{currentfill}{rgb}{1.000000,0.894118,0.788235}%
\pgfsetfillcolor{currentfill}%
\pgfsetlinewidth{0.000000pt}%
\definecolor{currentstroke}{rgb}{1.000000,0.894118,0.788235}%
\pgfsetstrokecolor{currentstroke}%
\pgfsetdash{}{0pt}%
\pgfpathmoveto{\pgfqpoint{3.025552in}{2.586502in}}%
\pgfpathlineto{\pgfqpoint{2.944593in}{2.539760in}}%
\pgfpathlineto{\pgfqpoint{2.573288in}{3.182879in}}%
\pgfpathlineto{\pgfqpoint{2.654247in}{3.229621in}}%
\pgfpathlineto{\pgfqpoint{3.025552in}{2.586502in}}%
\pgfpathclose%
\pgfusepath{fill}%
\end{pgfscope}%
\begin{pgfscope}%
\pgfpathrectangle{\pgfqpoint{0.765000in}{0.660000in}}{\pgfqpoint{4.620000in}{4.620000in}}%
\pgfusepath{clip}%
\pgfsetbuttcap%
\pgfsetroundjoin%
\definecolor{currentfill}{rgb}{1.000000,0.894118,0.788235}%
\pgfsetfillcolor{currentfill}%
\pgfsetlinewidth{0.000000pt}%
\definecolor{currentstroke}{rgb}{1.000000,0.894118,0.788235}%
\pgfsetstrokecolor{currentstroke}%
\pgfsetdash{}{0pt}%
\pgfpathmoveto{\pgfqpoint{3.106512in}{2.539760in}}%
\pgfpathlineto{\pgfqpoint{3.025552in}{2.586502in}}%
\pgfpathlineto{\pgfqpoint{2.654247in}{3.229621in}}%
\pgfpathlineto{\pgfqpoint{2.735207in}{3.182879in}}%
\pgfpathlineto{\pgfqpoint{3.106512in}{2.539760in}}%
\pgfpathclose%
\pgfusepath{fill}%
\end{pgfscope}%
\begin{pgfscope}%
\pgfpathrectangle{\pgfqpoint{0.765000in}{0.660000in}}{\pgfqpoint{4.620000in}{4.620000in}}%
\pgfusepath{clip}%
\pgfsetbuttcap%
\pgfsetroundjoin%
\definecolor{currentfill}{rgb}{1.000000,0.894118,0.788235}%
\pgfsetfillcolor{currentfill}%
\pgfsetlinewidth{0.000000pt}%
\definecolor{currentstroke}{rgb}{1.000000,0.894118,0.788235}%
\pgfsetstrokecolor{currentstroke}%
\pgfsetdash{}{0pt}%
\pgfpathmoveto{\pgfqpoint{3.025552in}{2.586502in}}%
\pgfpathlineto{\pgfqpoint{3.025552in}{2.679986in}}%
\pgfpathlineto{\pgfqpoint{2.654247in}{3.323105in}}%
\pgfpathlineto{\pgfqpoint{2.735207in}{3.182879in}}%
\pgfpathlineto{\pgfqpoint{3.025552in}{2.586502in}}%
\pgfpathclose%
\pgfusepath{fill}%
\end{pgfscope}%
\begin{pgfscope}%
\pgfpathrectangle{\pgfqpoint{0.765000in}{0.660000in}}{\pgfqpoint{4.620000in}{4.620000in}}%
\pgfusepath{clip}%
\pgfsetbuttcap%
\pgfsetroundjoin%
\definecolor{currentfill}{rgb}{1.000000,0.894118,0.788235}%
\pgfsetfillcolor{currentfill}%
\pgfsetlinewidth{0.000000pt}%
\definecolor{currentstroke}{rgb}{1.000000,0.894118,0.788235}%
\pgfsetstrokecolor{currentstroke}%
\pgfsetdash{}{0pt}%
\pgfpathmoveto{\pgfqpoint{3.106512in}{2.539760in}}%
\pgfpathlineto{\pgfqpoint{3.106512in}{2.633244in}}%
\pgfpathlineto{\pgfqpoint{2.735207in}{3.276363in}}%
\pgfpathlineto{\pgfqpoint{2.654247in}{3.229621in}}%
\pgfpathlineto{\pgfqpoint{3.106512in}{2.539760in}}%
\pgfpathclose%
\pgfusepath{fill}%
\end{pgfscope}%
\begin{pgfscope}%
\pgfpathrectangle{\pgfqpoint{0.765000in}{0.660000in}}{\pgfqpoint{4.620000in}{4.620000in}}%
\pgfusepath{clip}%
\pgfsetbuttcap%
\pgfsetroundjoin%
\definecolor{currentfill}{rgb}{1.000000,0.894118,0.788235}%
\pgfsetfillcolor{currentfill}%
\pgfsetlinewidth{0.000000pt}%
\definecolor{currentstroke}{rgb}{1.000000,0.894118,0.788235}%
\pgfsetstrokecolor{currentstroke}%
\pgfsetdash{}{0pt}%
\pgfpathmoveto{\pgfqpoint{2.944593in}{2.539760in}}%
\pgfpathlineto{\pgfqpoint{3.025552in}{2.493018in}}%
\pgfpathlineto{\pgfqpoint{2.654247in}{3.136137in}}%
\pgfpathlineto{\pgfqpoint{2.573288in}{3.182879in}}%
\pgfpathlineto{\pgfqpoint{2.944593in}{2.539760in}}%
\pgfpathclose%
\pgfusepath{fill}%
\end{pgfscope}%
\begin{pgfscope}%
\pgfpathrectangle{\pgfqpoint{0.765000in}{0.660000in}}{\pgfqpoint{4.620000in}{4.620000in}}%
\pgfusepath{clip}%
\pgfsetbuttcap%
\pgfsetroundjoin%
\definecolor{currentfill}{rgb}{1.000000,0.894118,0.788235}%
\pgfsetfillcolor{currentfill}%
\pgfsetlinewidth{0.000000pt}%
\definecolor{currentstroke}{rgb}{1.000000,0.894118,0.788235}%
\pgfsetstrokecolor{currentstroke}%
\pgfsetdash{}{0pt}%
\pgfpathmoveto{\pgfqpoint{3.025552in}{2.493018in}}%
\pgfpathlineto{\pgfqpoint{3.106512in}{2.539760in}}%
\pgfpathlineto{\pgfqpoint{2.735207in}{3.182879in}}%
\pgfpathlineto{\pgfqpoint{2.654247in}{3.136137in}}%
\pgfpathlineto{\pgfqpoint{3.025552in}{2.493018in}}%
\pgfpathclose%
\pgfusepath{fill}%
\end{pgfscope}%
\begin{pgfscope}%
\pgfpathrectangle{\pgfqpoint{0.765000in}{0.660000in}}{\pgfqpoint{4.620000in}{4.620000in}}%
\pgfusepath{clip}%
\pgfsetbuttcap%
\pgfsetroundjoin%
\definecolor{currentfill}{rgb}{1.000000,0.894118,0.788235}%
\pgfsetfillcolor{currentfill}%
\pgfsetlinewidth{0.000000pt}%
\definecolor{currentstroke}{rgb}{1.000000,0.894118,0.788235}%
\pgfsetstrokecolor{currentstroke}%
\pgfsetdash{}{0pt}%
\pgfpathmoveto{\pgfqpoint{3.025552in}{2.679986in}}%
\pgfpathlineto{\pgfqpoint{2.944593in}{2.633244in}}%
\pgfpathlineto{\pgfqpoint{2.573288in}{3.276363in}}%
\pgfpathlineto{\pgfqpoint{2.654247in}{3.323105in}}%
\pgfpathlineto{\pgfqpoint{3.025552in}{2.679986in}}%
\pgfpathclose%
\pgfusepath{fill}%
\end{pgfscope}%
\begin{pgfscope}%
\pgfpathrectangle{\pgfqpoint{0.765000in}{0.660000in}}{\pgfqpoint{4.620000in}{4.620000in}}%
\pgfusepath{clip}%
\pgfsetbuttcap%
\pgfsetroundjoin%
\definecolor{currentfill}{rgb}{1.000000,0.894118,0.788235}%
\pgfsetfillcolor{currentfill}%
\pgfsetlinewidth{0.000000pt}%
\definecolor{currentstroke}{rgb}{1.000000,0.894118,0.788235}%
\pgfsetstrokecolor{currentstroke}%
\pgfsetdash{}{0pt}%
\pgfpathmoveto{\pgfqpoint{3.106512in}{2.633244in}}%
\pgfpathlineto{\pgfqpoint{3.025552in}{2.679986in}}%
\pgfpathlineto{\pgfqpoint{2.654247in}{3.323105in}}%
\pgfpathlineto{\pgfqpoint{2.735207in}{3.276363in}}%
\pgfpathlineto{\pgfqpoint{3.106512in}{2.633244in}}%
\pgfpathclose%
\pgfusepath{fill}%
\end{pgfscope}%
\begin{pgfscope}%
\pgfpathrectangle{\pgfqpoint{0.765000in}{0.660000in}}{\pgfqpoint{4.620000in}{4.620000in}}%
\pgfusepath{clip}%
\pgfsetbuttcap%
\pgfsetroundjoin%
\definecolor{currentfill}{rgb}{1.000000,0.894118,0.788235}%
\pgfsetfillcolor{currentfill}%
\pgfsetlinewidth{0.000000pt}%
\definecolor{currentstroke}{rgb}{1.000000,0.894118,0.788235}%
\pgfsetstrokecolor{currentstroke}%
\pgfsetdash{}{0pt}%
\pgfpathmoveto{\pgfqpoint{2.944593in}{2.539760in}}%
\pgfpathlineto{\pgfqpoint{2.944593in}{2.633244in}}%
\pgfpathlineto{\pgfqpoint{2.573288in}{3.276363in}}%
\pgfpathlineto{\pgfqpoint{2.654247in}{3.136137in}}%
\pgfpathlineto{\pgfqpoint{2.944593in}{2.539760in}}%
\pgfpathclose%
\pgfusepath{fill}%
\end{pgfscope}%
\begin{pgfscope}%
\pgfpathrectangle{\pgfqpoint{0.765000in}{0.660000in}}{\pgfqpoint{4.620000in}{4.620000in}}%
\pgfusepath{clip}%
\pgfsetbuttcap%
\pgfsetroundjoin%
\definecolor{currentfill}{rgb}{1.000000,0.894118,0.788235}%
\pgfsetfillcolor{currentfill}%
\pgfsetlinewidth{0.000000pt}%
\definecolor{currentstroke}{rgb}{1.000000,0.894118,0.788235}%
\pgfsetstrokecolor{currentstroke}%
\pgfsetdash{}{0pt}%
\pgfpathmoveto{\pgfqpoint{3.025552in}{2.493018in}}%
\pgfpathlineto{\pgfqpoint{3.025552in}{2.586502in}}%
\pgfpathlineto{\pgfqpoint{2.654247in}{3.229621in}}%
\pgfpathlineto{\pgfqpoint{2.573288in}{3.182879in}}%
\pgfpathlineto{\pgfqpoint{3.025552in}{2.493018in}}%
\pgfpathclose%
\pgfusepath{fill}%
\end{pgfscope}%
\begin{pgfscope}%
\pgfpathrectangle{\pgfqpoint{0.765000in}{0.660000in}}{\pgfqpoint{4.620000in}{4.620000in}}%
\pgfusepath{clip}%
\pgfsetbuttcap%
\pgfsetroundjoin%
\definecolor{currentfill}{rgb}{1.000000,0.894118,0.788235}%
\pgfsetfillcolor{currentfill}%
\pgfsetlinewidth{0.000000pt}%
\definecolor{currentstroke}{rgb}{1.000000,0.894118,0.788235}%
\pgfsetstrokecolor{currentstroke}%
\pgfsetdash{}{0pt}%
\pgfpathmoveto{\pgfqpoint{2.944593in}{2.633244in}}%
\pgfpathlineto{\pgfqpoint{3.025552in}{2.586502in}}%
\pgfpathlineto{\pgfqpoint{2.654247in}{3.229621in}}%
\pgfpathlineto{\pgfqpoint{2.573288in}{3.276363in}}%
\pgfpathlineto{\pgfqpoint{2.944593in}{2.633244in}}%
\pgfpathclose%
\pgfusepath{fill}%
\end{pgfscope}%
\begin{pgfscope}%
\pgfpathrectangle{\pgfqpoint{0.765000in}{0.660000in}}{\pgfqpoint{4.620000in}{4.620000in}}%
\pgfusepath{clip}%
\pgfsetbuttcap%
\pgfsetroundjoin%
\definecolor{currentfill}{rgb}{1.000000,0.894118,0.788235}%
\pgfsetfillcolor{currentfill}%
\pgfsetlinewidth{0.000000pt}%
\definecolor{currentstroke}{rgb}{1.000000,0.894118,0.788235}%
\pgfsetstrokecolor{currentstroke}%
\pgfsetdash{}{0pt}%
\pgfpathmoveto{\pgfqpoint{3.025552in}{2.586502in}}%
\pgfpathlineto{\pgfqpoint{3.106512in}{2.633244in}}%
\pgfpathlineto{\pgfqpoint{2.735207in}{3.276363in}}%
\pgfpathlineto{\pgfqpoint{2.654247in}{3.229621in}}%
\pgfpathlineto{\pgfqpoint{3.025552in}{2.586502in}}%
\pgfpathclose%
\pgfusepath{fill}%
\end{pgfscope}%
\begin{pgfscope}%
\pgfpathrectangle{\pgfqpoint{0.765000in}{0.660000in}}{\pgfqpoint{4.620000in}{4.620000in}}%
\pgfusepath{clip}%
\pgfsetbuttcap%
\pgfsetroundjoin%
\definecolor{currentfill}{rgb}{1.000000,0.894118,0.788235}%
\pgfsetfillcolor{currentfill}%
\pgfsetlinewidth{0.000000pt}%
\definecolor{currentstroke}{rgb}{1.000000,0.894118,0.788235}%
\pgfsetstrokecolor{currentstroke}%
\pgfsetdash{}{0pt}%
\pgfpathmoveto{\pgfqpoint{2.654247in}{3.328328in}}%
\pgfpathlineto{\pgfqpoint{2.573288in}{3.281587in}}%
\pgfpathlineto{\pgfqpoint{2.654247in}{3.234845in}}%
\pgfpathlineto{\pgfqpoint{2.735207in}{3.281587in}}%
\pgfpathlineto{\pgfqpoint{2.654247in}{3.328328in}}%
\pgfpathclose%
\pgfusepath{fill}%
\end{pgfscope}%
\begin{pgfscope}%
\pgfpathrectangle{\pgfqpoint{0.765000in}{0.660000in}}{\pgfqpoint{4.620000in}{4.620000in}}%
\pgfusepath{clip}%
\pgfsetbuttcap%
\pgfsetroundjoin%
\definecolor{currentfill}{rgb}{1.000000,0.894118,0.788235}%
\pgfsetfillcolor{currentfill}%
\pgfsetlinewidth{0.000000pt}%
\definecolor{currentstroke}{rgb}{1.000000,0.894118,0.788235}%
\pgfsetstrokecolor{currentstroke}%
\pgfsetdash{}{0pt}%
\pgfpathmoveto{\pgfqpoint{2.654247in}{3.328328in}}%
\pgfpathlineto{\pgfqpoint{2.573288in}{3.281587in}}%
\pgfpathlineto{\pgfqpoint{2.573288in}{3.375070in}}%
\pgfpathlineto{\pgfqpoint{2.654247in}{3.421812in}}%
\pgfpathlineto{\pgfqpoint{2.654247in}{3.328328in}}%
\pgfpathclose%
\pgfusepath{fill}%
\end{pgfscope}%
\begin{pgfscope}%
\pgfpathrectangle{\pgfqpoint{0.765000in}{0.660000in}}{\pgfqpoint{4.620000in}{4.620000in}}%
\pgfusepath{clip}%
\pgfsetbuttcap%
\pgfsetroundjoin%
\definecolor{currentfill}{rgb}{1.000000,0.894118,0.788235}%
\pgfsetfillcolor{currentfill}%
\pgfsetlinewidth{0.000000pt}%
\definecolor{currentstroke}{rgb}{1.000000,0.894118,0.788235}%
\pgfsetstrokecolor{currentstroke}%
\pgfsetdash{}{0pt}%
\pgfpathmoveto{\pgfqpoint{2.654247in}{3.328328in}}%
\pgfpathlineto{\pgfqpoint{2.735207in}{3.281587in}}%
\pgfpathlineto{\pgfqpoint{2.735207in}{3.375070in}}%
\pgfpathlineto{\pgfqpoint{2.654247in}{3.421812in}}%
\pgfpathlineto{\pgfqpoint{2.654247in}{3.328328in}}%
\pgfpathclose%
\pgfusepath{fill}%
\end{pgfscope}%
\begin{pgfscope}%
\pgfpathrectangle{\pgfqpoint{0.765000in}{0.660000in}}{\pgfqpoint{4.620000in}{4.620000in}}%
\pgfusepath{clip}%
\pgfsetbuttcap%
\pgfsetroundjoin%
\definecolor{currentfill}{rgb}{1.000000,0.894118,0.788235}%
\pgfsetfillcolor{currentfill}%
\pgfsetlinewidth{0.000000pt}%
\definecolor{currentstroke}{rgb}{1.000000,0.894118,0.788235}%
\pgfsetstrokecolor{currentstroke}%
\pgfsetdash{}{0pt}%
\pgfpathmoveto{\pgfqpoint{2.654247in}{3.229621in}}%
\pgfpathlineto{\pgfqpoint{2.573288in}{3.182879in}}%
\pgfpathlineto{\pgfqpoint{2.654247in}{3.136137in}}%
\pgfpathlineto{\pgfqpoint{2.735207in}{3.182879in}}%
\pgfpathlineto{\pgfqpoint{2.654247in}{3.229621in}}%
\pgfpathclose%
\pgfusepath{fill}%
\end{pgfscope}%
\begin{pgfscope}%
\pgfpathrectangle{\pgfqpoint{0.765000in}{0.660000in}}{\pgfqpoint{4.620000in}{4.620000in}}%
\pgfusepath{clip}%
\pgfsetbuttcap%
\pgfsetroundjoin%
\definecolor{currentfill}{rgb}{1.000000,0.894118,0.788235}%
\pgfsetfillcolor{currentfill}%
\pgfsetlinewidth{0.000000pt}%
\definecolor{currentstroke}{rgb}{1.000000,0.894118,0.788235}%
\pgfsetstrokecolor{currentstroke}%
\pgfsetdash{}{0pt}%
\pgfpathmoveto{\pgfqpoint{2.654247in}{3.229621in}}%
\pgfpathlineto{\pgfqpoint{2.573288in}{3.182879in}}%
\pgfpathlineto{\pgfqpoint{2.573288in}{3.276363in}}%
\pgfpathlineto{\pgfqpoint{2.654247in}{3.323105in}}%
\pgfpathlineto{\pgfqpoint{2.654247in}{3.229621in}}%
\pgfpathclose%
\pgfusepath{fill}%
\end{pgfscope}%
\begin{pgfscope}%
\pgfpathrectangle{\pgfqpoint{0.765000in}{0.660000in}}{\pgfqpoint{4.620000in}{4.620000in}}%
\pgfusepath{clip}%
\pgfsetbuttcap%
\pgfsetroundjoin%
\definecolor{currentfill}{rgb}{1.000000,0.894118,0.788235}%
\pgfsetfillcolor{currentfill}%
\pgfsetlinewidth{0.000000pt}%
\definecolor{currentstroke}{rgb}{1.000000,0.894118,0.788235}%
\pgfsetstrokecolor{currentstroke}%
\pgfsetdash{}{0pt}%
\pgfpathmoveto{\pgfqpoint{2.654247in}{3.229621in}}%
\pgfpathlineto{\pgfqpoint{2.735207in}{3.182879in}}%
\pgfpathlineto{\pgfqpoint{2.735207in}{3.276363in}}%
\pgfpathlineto{\pgfqpoint{2.654247in}{3.323105in}}%
\pgfpathlineto{\pgfqpoint{2.654247in}{3.229621in}}%
\pgfpathclose%
\pgfusepath{fill}%
\end{pgfscope}%
\begin{pgfscope}%
\pgfpathrectangle{\pgfqpoint{0.765000in}{0.660000in}}{\pgfqpoint{4.620000in}{4.620000in}}%
\pgfusepath{clip}%
\pgfsetbuttcap%
\pgfsetroundjoin%
\definecolor{currentfill}{rgb}{1.000000,0.894118,0.788235}%
\pgfsetfillcolor{currentfill}%
\pgfsetlinewidth{0.000000pt}%
\definecolor{currentstroke}{rgb}{1.000000,0.894118,0.788235}%
\pgfsetstrokecolor{currentstroke}%
\pgfsetdash{}{0pt}%
\pgfpathmoveto{\pgfqpoint{2.654247in}{3.234845in}}%
\pgfpathlineto{\pgfqpoint{2.735207in}{3.281587in}}%
\pgfpathlineto{\pgfqpoint{2.735207in}{3.375070in}}%
\pgfpathlineto{\pgfqpoint{2.654247in}{3.328328in}}%
\pgfpathlineto{\pgfqpoint{2.654247in}{3.234845in}}%
\pgfpathclose%
\pgfusepath{fill}%
\end{pgfscope}%
\begin{pgfscope}%
\pgfpathrectangle{\pgfqpoint{0.765000in}{0.660000in}}{\pgfqpoint{4.620000in}{4.620000in}}%
\pgfusepath{clip}%
\pgfsetbuttcap%
\pgfsetroundjoin%
\definecolor{currentfill}{rgb}{1.000000,0.894118,0.788235}%
\pgfsetfillcolor{currentfill}%
\pgfsetlinewidth{0.000000pt}%
\definecolor{currentstroke}{rgb}{1.000000,0.894118,0.788235}%
\pgfsetstrokecolor{currentstroke}%
\pgfsetdash{}{0pt}%
\pgfpathmoveto{\pgfqpoint{2.654247in}{3.136137in}}%
\pgfpathlineto{\pgfqpoint{2.735207in}{3.182879in}}%
\pgfpathlineto{\pgfqpoint{2.735207in}{3.276363in}}%
\pgfpathlineto{\pgfqpoint{2.654247in}{3.229621in}}%
\pgfpathlineto{\pgfqpoint{2.654247in}{3.136137in}}%
\pgfpathclose%
\pgfusepath{fill}%
\end{pgfscope}%
\begin{pgfscope}%
\pgfpathrectangle{\pgfqpoint{0.765000in}{0.660000in}}{\pgfqpoint{4.620000in}{4.620000in}}%
\pgfusepath{clip}%
\pgfsetbuttcap%
\pgfsetroundjoin%
\definecolor{currentfill}{rgb}{1.000000,0.894118,0.788235}%
\pgfsetfillcolor{currentfill}%
\pgfsetlinewidth{0.000000pt}%
\definecolor{currentstroke}{rgb}{1.000000,0.894118,0.788235}%
\pgfsetstrokecolor{currentstroke}%
\pgfsetdash{}{0pt}%
\pgfpathmoveto{\pgfqpoint{2.654247in}{3.421812in}}%
\pgfpathlineto{\pgfqpoint{2.573288in}{3.375070in}}%
\pgfpathlineto{\pgfqpoint{2.654247in}{3.328328in}}%
\pgfpathlineto{\pgfqpoint{2.735207in}{3.375070in}}%
\pgfpathlineto{\pgfqpoint{2.654247in}{3.421812in}}%
\pgfpathclose%
\pgfusepath{fill}%
\end{pgfscope}%
\begin{pgfscope}%
\pgfpathrectangle{\pgfqpoint{0.765000in}{0.660000in}}{\pgfqpoint{4.620000in}{4.620000in}}%
\pgfusepath{clip}%
\pgfsetbuttcap%
\pgfsetroundjoin%
\definecolor{currentfill}{rgb}{1.000000,0.894118,0.788235}%
\pgfsetfillcolor{currentfill}%
\pgfsetlinewidth{0.000000pt}%
\definecolor{currentstroke}{rgb}{1.000000,0.894118,0.788235}%
\pgfsetstrokecolor{currentstroke}%
\pgfsetdash{}{0pt}%
\pgfpathmoveto{\pgfqpoint{2.573288in}{3.281587in}}%
\pgfpathlineto{\pgfqpoint{2.654247in}{3.234845in}}%
\pgfpathlineto{\pgfqpoint{2.654247in}{3.328328in}}%
\pgfpathlineto{\pgfqpoint{2.573288in}{3.375070in}}%
\pgfpathlineto{\pgfqpoint{2.573288in}{3.281587in}}%
\pgfpathclose%
\pgfusepath{fill}%
\end{pgfscope}%
\begin{pgfscope}%
\pgfpathrectangle{\pgfqpoint{0.765000in}{0.660000in}}{\pgfqpoint{4.620000in}{4.620000in}}%
\pgfusepath{clip}%
\pgfsetbuttcap%
\pgfsetroundjoin%
\definecolor{currentfill}{rgb}{1.000000,0.894118,0.788235}%
\pgfsetfillcolor{currentfill}%
\pgfsetlinewidth{0.000000pt}%
\definecolor{currentstroke}{rgb}{1.000000,0.894118,0.788235}%
\pgfsetstrokecolor{currentstroke}%
\pgfsetdash{}{0pt}%
\pgfpathmoveto{\pgfqpoint{2.654247in}{3.323105in}}%
\pgfpathlineto{\pgfqpoint{2.573288in}{3.276363in}}%
\pgfpathlineto{\pgfqpoint{2.654247in}{3.229621in}}%
\pgfpathlineto{\pgfqpoint{2.735207in}{3.276363in}}%
\pgfpathlineto{\pgfqpoint{2.654247in}{3.323105in}}%
\pgfpathclose%
\pgfusepath{fill}%
\end{pgfscope}%
\begin{pgfscope}%
\pgfpathrectangle{\pgfqpoint{0.765000in}{0.660000in}}{\pgfqpoint{4.620000in}{4.620000in}}%
\pgfusepath{clip}%
\pgfsetbuttcap%
\pgfsetroundjoin%
\definecolor{currentfill}{rgb}{1.000000,0.894118,0.788235}%
\pgfsetfillcolor{currentfill}%
\pgfsetlinewidth{0.000000pt}%
\definecolor{currentstroke}{rgb}{1.000000,0.894118,0.788235}%
\pgfsetstrokecolor{currentstroke}%
\pgfsetdash{}{0pt}%
\pgfpathmoveto{\pgfqpoint{2.573288in}{3.182879in}}%
\pgfpathlineto{\pgfqpoint{2.654247in}{3.136137in}}%
\pgfpathlineto{\pgfqpoint{2.654247in}{3.229621in}}%
\pgfpathlineto{\pgfqpoint{2.573288in}{3.276363in}}%
\pgfpathlineto{\pgfqpoint{2.573288in}{3.182879in}}%
\pgfpathclose%
\pgfusepath{fill}%
\end{pgfscope}%
\begin{pgfscope}%
\pgfpathrectangle{\pgfqpoint{0.765000in}{0.660000in}}{\pgfqpoint{4.620000in}{4.620000in}}%
\pgfusepath{clip}%
\pgfsetbuttcap%
\pgfsetroundjoin%
\definecolor{currentfill}{rgb}{1.000000,0.894118,0.788235}%
\pgfsetfillcolor{currentfill}%
\pgfsetlinewidth{0.000000pt}%
\definecolor{currentstroke}{rgb}{1.000000,0.894118,0.788235}%
\pgfsetstrokecolor{currentstroke}%
\pgfsetdash{}{0pt}%
\pgfpathmoveto{\pgfqpoint{2.654247in}{3.328328in}}%
\pgfpathlineto{\pgfqpoint{2.573288in}{3.281587in}}%
\pgfpathlineto{\pgfqpoint{2.573288in}{3.182879in}}%
\pgfpathlineto{\pgfqpoint{2.654247in}{3.229621in}}%
\pgfpathlineto{\pgfqpoint{2.654247in}{3.328328in}}%
\pgfpathclose%
\pgfusepath{fill}%
\end{pgfscope}%
\begin{pgfscope}%
\pgfpathrectangle{\pgfqpoint{0.765000in}{0.660000in}}{\pgfqpoint{4.620000in}{4.620000in}}%
\pgfusepath{clip}%
\pgfsetbuttcap%
\pgfsetroundjoin%
\definecolor{currentfill}{rgb}{1.000000,0.894118,0.788235}%
\pgfsetfillcolor{currentfill}%
\pgfsetlinewidth{0.000000pt}%
\definecolor{currentstroke}{rgb}{1.000000,0.894118,0.788235}%
\pgfsetstrokecolor{currentstroke}%
\pgfsetdash{}{0pt}%
\pgfpathmoveto{\pgfqpoint{2.735207in}{3.281587in}}%
\pgfpathlineto{\pgfqpoint{2.654247in}{3.328328in}}%
\pgfpathlineto{\pgfqpoint{2.654247in}{3.229621in}}%
\pgfpathlineto{\pgfqpoint{2.735207in}{3.182879in}}%
\pgfpathlineto{\pgfqpoint{2.735207in}{3.281587in}}%
\pgfpathclose%
\pgfusepath{fill}%
\end{pgfscope}%
\begin{pgfscope}%
\pgfpathrectangle{\pgfqpoint{0.765000in}{0.660000in}}{\pgfqpoint{4.620000in}{4.620000in}}%
\pgfusepath{clip}%
\pgfsetbuttcap%
\pgfsetroundjoin%
\definecolor{currentfill}{rgb}{1.000000,0.894118,0.788235}%
\pgfsetfillcolor{currentfill}%
\pgfsetlinewidth{0.000000pt}%
\definecolor{currentstroke}{rgb}{1.000000,0.894118,0.788235}%
\pgfsetstrokecolor{currentstroke}%
\pgfsetdash{}{0pt}%
\pgfpathmoveto{\pgfqpoint{2.654247in}{3.328328in}}%
\pgfpathlineto{\pgfqpoint{2.654247in}{3.421812in}}%
\pgfpathlineto{\pgfqpoint{2.654247in}{3.323105in}}%
\pgfpathlineto{\pgfqpoint{2.735207in}{3.182879in}}%
\pgfpathlineto{\pgfqpoint{2.654247in}{3.328328in}}%
\pgfpathclose%
\pgfusepath{fill}%
\end{pgfscope}%
\begin{pgfscope}%
\pgfpathrectangle{\pgfqpoint{0.765000in}{0.660000in}}{\pgfqpoint{4.620000in}{4.620000in}}%
\pgfusepath{clip}%
\pgfsetbuttcap%
\pgfsetroundjoin%
\definecolor{currentfill}{rgb}{1.000000,0.894118,0.788235}%
\pgfsetfillcolor{currentfill}%
\pgfsetlinewidth{0.000000pt}%
\definecolor{currentstroke}{rgb}{1.000000,0.894118,0.788235}%
\pgfsetstrokecolor{currentstroke}%
\pgfsetdash{}{0pt}%
\pgfpathmoveto{\pgfqpoint{2.735207in}{3.281587in}}%
\pgfpathlineto{\pgfqpoint{2.735207in}{3.375070in}}%
\pgfpathlineto{\pgfqpoint{2.735207in}{3.276363in}}%
\pgfpathlineto{\pgfqpoint{2.654247in}{3.229621in}}%
\pgfpathlineto{\pgfqpoint{2.735207in}{3.281587in}}%
\pgfpathclose%
\pgfusepath{fill}%
\end{pgfscope}%
\begin{pgfscope}%
\pgfpathrectangle{\pgfqpoint{0.765000in}{0.660000in}}{\pgfqpoint{4.620000in}{4.620000in}}%
\pgfusepath{clip}%
\pgfsetbuttcap%
\pgfsetroundjoin%
\definecolor{currentfill}{rgb}{1.000000,0.894118,0.788235}%
\pgfsetfillcolor{currentfill}%
\pgfsetlinewidth{0.000000pt}%
\definecolor{currentstroke}{rgb}{1.000000,0.894118,0.788235}%
\pgfsetstrokecolor{currentstroke}%
\pgfsetdash{}{0pt}%
\pgfpathmoveto{\pgfqpoint{2.573288in}{3.281587in}}%
\pgfpathlineto{\pgfqpoint{2.654247in}{3.234845in}}%
\pgfpathlineto{\pgfqpoint{2.654247in}{3.136137in}}%
\pgfpathlineto{\pgfqpoint{2.573288in}{3.182879in}}%
\pgfpathlineto{\pgfqpoint{2.573288in}{3.281587in}}%
\pgfpathclose%
\pgfusepath{fill}%
\end{pgfscope}%
\begin{pgfscope}%
\pgfpathrectangle{\pgfqpoint{0.765000in}{0.660000in}}{\pgfqpoint{4.620000in}{4.620000in}}%
\pgfusepath{clip}%
\pgfsetbuttcap%
\pgfsetroundjoin%
\definecolor{currentfill}{rgb}{1.000000,0.894118,0.788235}%
\pgfsetfillcolor{currentfill}%
\pgfsetlinewidth{0.000000pt}%
\definecolor{currentstroke}{rgb}{1.000000,0.894118,0.788235}%
\pgfsetstrokecolor{currentstroke}%
\pgfsetdash{}{0pt}%
\pgfpathmoveto{\pgfqpoint{2.654247in}{3.234845in}}%
\pgfpathlineto{\pgfqpoint{2.735207in}{3.281587in}}%
\pgfpathlineto{\pgfqpoint{2.735207in}{3.182879in}}%
\pgfpathlineto{\pgfqpoint{2.654247in}{3.136137in}}%
\pgfpathlineto{\pgfqpoint{2.654247in}{3.234845in}}%
\pgfpathclose%
\pgfusepath{fill}%
\end{pgfscope}%
\begin{pgfscope}%
\pgfpathrectangle{\pgfqpoint{0.765000in}{0.660000in}}{\pgfqpoint{4.620000in}{4.620000in}}%
\pgfusepath{clip}%
\pgfsetbuttcap%
\pgfsetroundjoin%
\definecolor{currentfill}{rgb}{1.000000,0.894118,0.788235}%
\pgfsetfillcolor{currentfill}%
\pgfsetlinewidth{0.000000pt}%
\definecolor{currentstroke}{rgb}{1.000000,0.894118,0.788235}%
\pgfsetstrokecolor{currentstroke}%
\pgfsetdash{}{0pt}%
\pgfpathmoveto{\pgfqpoint{2.654247in}{3.421812in}}%
\pgfpathlineto{\pgfqpoint{2.573288in}{3.375070in}}%
\pgfpathlineto{\pgfqpoint{2.573288in}{3.276363in}}%
\pgfpathlineto{\pgfqpoint{2.654247in}{3.323105in}}%
\pgfpathlineto{\pgfqpoint{2.654247in}{3.421812in}}%
\pgfpathclose%
\pgfusepath{fill}%
\end{pgfscope}%
\begin{pgfscope}%
\pgfpathrectangle{\pgfqpoint{0.765000in}{0.660000in}}{\pgfqpoint{4.620000in}{4.620000in}}%
\pgfusepath{clip}%
\pgfsetbuttcap%
\pgfsetroundjoin%
\definecolor{currentfill}{rgb}{1.000000,0.894118,0.788235}%
\pgfsetfillcolor{currentfill}%
\pgfsetlinewidth{0.000000pt}%
\definecolor{currentstroke}{rgb}{1.000000,0.894118,0.788235}%
\pgfsetstrokecolor{currentstroke}%
\pgfsetdash{}{0pt}%
\pgfpathmoveto{\pgfqpoint{2.735207in}{3.375070in}}%
\pgfpathlineto{\pgfqpoint{2.654247in}{3.421812in}}%
\pgfpathlineto{\pgfqpoint{2.654247in}{3.323105in}}%
\pgfpathlineto{\pgfqpoint{2.735207in}{3.276363in}}%
\pgfpathlineto{\pgfqpoint{2.735207in}{3.375070in}}%
\pgfpathclose%
\pgfusepath{fill}%
\end{pgfscope}%
\begin{pgfscope}%
\pgfpathrectangle{\pgfqpoint{0.765000in}{0.660000in}}{\pgfqpoint{4.620000in}{4.620000in}}%
\pgfusepath{clip}%
\pgfsetbuttcap%
\pgfsetroundjoin%
\definecolor{currentfill}{rgb}{1.000000,0.894118,0.788235}%
\pgfsetfillcolor{currentfill}%
\pgfsetlinewidth{0.000000pt}%
\definecolor{currentstroke}{rgb}{1.000000,0.894118,0.788235}%
\pgfsetstrokecolor{currentstroke}%
\pgfsetdash{}{0pt}%
\pgfpathmoveto{\pgfqpoint{2.573288in}{3.281587in}}%
\pgfpathlineto{\pgfqpoint{2.573288in}{3.375070in}}%
\pgfpathlineto{\pgfqpoint{2.573288in}{3.276363in}}%
\pgfpathlineto{\pgfqpoint{2.654247in}{3.136137in}}%
\pgfpathlineto{\pgfqpoint{2.573288in}{3.281587in}}%
\pgfpathclose%
\pgfusepath{fill}%
\end{pgfscope}%
\begin{pgfscope}%
\pgfpathrectangle{\pgfqpoint{0.765000in}{0.660000in}}{\pgfqpoint{4.620000in}{4.620000in}}%
\pgfusepath{clip}%
\pgfsetbuttcap%
\pgfsetroundjoin%
\definecolor{currentfill}{rgb}{1.000000,0.894118,0.788235}%
\pgfsetfillcolor{currentfill}%
\pgfsetlinewidth{0.000000pt}%
\definecolor{currentstroke}{rgb}{1.000000,0.894118,0.788235}%
\pgfsetstrokecolor{currentstroke}%
\pgfsetdash{}{0pt}%
\pgfpathmoveto{\pgfqpoint{2.654247in}{3.234845in}}%
\pgfpathlineto{\pgfqpoint{2.654247in}{3.328328in}}%
\pgfpathlineto{\pgfqpoint{2.654247in}{3.229621in}}%
\pgfpathlineto{\pgfqpoint{2.573288in}{3.182879in}}%
\pgfpathlineto{\pgfqpoint{2.654247in}{3.234845in}}%
\pgfpathclose%
\pgfusepath{fill}%
\end{pgfscope}%
\begin{pgfscope}%
\pgfpathrectangle{\pgfqpoint{0.765000in}{0.660000in}}{\pgfqpoint{4.620000in}{4.620000in}}%
\pgfusepath{clip}%
\pgfsetbuttcap%
\pgfsetroundjoin%
\definecolor{currentfill}{rgb}{1.000000,0.894118,0.788235}%
\pgfsetfillcolor{currentfill}%
\pgfsetlinewidth{0.000000pt}%
\definecolor{currentstroke}{rgb}{1.000000,0.894118,0.788235}%
\pgfsetstrokecolor{currentstroke}%
\pgfsetdash{}{0pt}%
\pgfpathmoveto{\pgfqpoint{2.573288in}{3.375070in}}%
\pgfpathlineto{\pgfqpoint{2.654247in}{3.328328in}}%
\pgfpathlineto{\pgfqpoint{2.654247in}{3.229621in}}%
\pgfpathlineto{\pgfqpoint{2.573288in}{3.276363in}}%
\pgfpathlineto{\pgfqpoint{2.573288in}{3.375070in}}%
\pgfpathclose%
\pgfusepath{fill}%
\end{pgfscope}%
\begin{pgfscope}%
\pgfpathrectangle{\pgfqpoint{0.765000in}{0.660000in}}{\pgfqpoint{4.620000in}{4.620000in}}%
\pgfusepath{clip}%
\pgfsetbuttcap%
\pgfsetroundjoin%
\definecolor{currentfill}{rgb}{1.000000,0.894118,0.788235}%
\pgfsetfillcolor{currentfill}%
\pgfsetlinewidth{0.000000pt}%
\definecolor{currentstroke}{rgb}{1.000000,0.894118,0.788235}%
\pgfsetstrokecolor{currentstroke}%
\pgfsetdash{}{0pt}%
\pgfpathmoveto{\pgfqpoint{2.654247in}{3.328328in}}%
\pgfpathlineto{\pgfqpoint{2.735207in}{3.375070in}}%
\pgfpathlineto{\pgfqpoint{2.735207in}{3.276363in}}%
\pgfpathlineto{\pgfqpoint{2.654247in}{3.229621in}}%
\pgfpathlineto{\pgfqpoint{2.654247in}{3.328328in}}%
\pgfpathclose%
\pgfusepath{fill}%
\end{pgfscope}%
\begin{pgfscope}%
\pgfpathrectangle{\pgfqpoint{0.765000in}{0.660000in}}{\pgfqpoint{4.620000in}{4.620000in}}%
\pgfusepath{clip}%
\pgfsetbuttcap%
\pgfsetroundjoin%
\definecolor{currentfill}{rgb}{1.000000,0.894118,0.788235}%
\pgfsetfillcolor{currentfill}%
\pgfsetlinewidth{0.000000pt}%
\definecolor{currentstroke}{rgb}{1.000000,0.894118,0.788235}%
\pgfsetstrokecolor{currentstroke}%
\pgfsetdash{}{0pt}%
\pgfpathmoveto{\pgfqpoint{2.654247in}{3.328328in}}%
\pgfpathlineto{\pgfqpoint{2.573288in}{3.281587in}}%
\pgfpathlineto{\pgfqpoint{2.654247in}{3.234845in}}%
\pgfpathlineto{\pgfqpoint{2.735207in}{3.281587in}}%
\pgfpathlineto{\pgfqpoint{2.654247in}{3.328328in}}%
\pgfpathclose%
\pgfusepath{fill}%
\end{pgfscope}%
\begin{pgfscope}%
\pgfpathrectangle{\pgfqpoint{0.765000in}{0.660000in}}{\pgfqpoint{4.620000in}{4.620000in}}%
\pgfusepath{clip}%
\pgfsetbuttcap%
\pgfsetroundjoin%
\definecolor{currentfill}{rgb}{1.000000,0.894118,0.788235}%
\pgfsetfillcolor{currentfill}%
\pgfsetlinewidth{0.000000pt}%
\definecolor{currentstroke}{rgb}{1.000000,0.894118,0.788235}%
\pgfsetstrokecolor{currentstroke}%
\pgfsetdash{}{0pt}%
\pgfpathmoveto{\pgfqpoint{2.654247in}{3.328328in}}%
\pgfpathlineto{\pgfqpoint{2.573288in}{3.281587in}}%
\pgfpathlineto{\pgfqpoint{2.573288in}{3.375070in}}%
\pgfpathlineto{\pgfqpoint{2.654247in}{3.421812in}}%
\pgfpathlineto{\pgfqpoint{2.654247in}{3.328328in}}%
\pgfpathclose%
\pgfusepath{fill}%
\end{pgfscope}%
\begin{pgfscope}%
\pgfpathrectangle{\pgfqpoint{0.765000in}{0.660000in}}{\pgfqpoint{4.620000in}{4.620000in}}%
\pgfusepath{clip}%
\pgfsetbuttcap%
\pgfsetroundjoin%
\definecolor{currentfill}{rgb}{1.000000,0.894118,0.788235}%
\pgfsetfillcolor{currentfill}%
\pgfsetlinewidth{0.000000pt}%
\definecolor{currentstroke}{rgb}{1.000000,0.894118,0.788235}%
\pgfsetstrokecolor{currentstroke}%
\pgfsetdash{}{0pt}%
\pgfpathmoveto{\pgfqpoint{2.654247in}{3.328328in}}%
\pgfpathlineto{\pgfqpoint{2.735207in}{3.281587in}}%
\pgfpathlineto{\pgfqpoint{2.735207in}{3.375070in}}%
\pgfpathlineto{\pgfqpoint{2.654247in}{3.421812in}}%
\pgfpathlineto{\pgfqpoint{2.654247in}{3.328328in}}%
\pgfpathclose%
\pgfusepath{fill}%
\end{pgfscope}%
\begin{pgfscope}%
\pgfpathrectangle{\pgfqpoint{0.765000in}{0.660000in}}{\pgfqpoint{4.620000in}{4.620000in}}%
\pgfusepath{clip}%
\pgfsetbuttcap%
\pgfsetroundjoin%
\definecolor{currentfill}{rgb}{1.000000,0.894118,0.788235}%
\pgfsetfillcolor{currentfill}%
\pgfsetlinewidth{0.000000pt}%
\definecolor{currentstroke}{rgb}{1.000000,0.894118,0.788235}%
\pgfsetstrokecolor{currentstroke}%
\pgfsetdash{}{0pt}%
\pgfpathmoveto{\pgfqpoint{2.869318in}{3.452500in}}%
\pgfpathlineto{\pgfqpoint{2.788359in}{3.405758in}}%
\pgfpathlineto{\pgfqpoint{2.869318in}{3.359016in}}%
\pgfpathlineto{\pgfqpoint{2.950278in}{3.405758in}}%
\pgfpathlineto{\pgfqpoint{2.869318in}{3.452500in}}%
\pgfpathclose%
\pgfusepath{fill}%
\end{pgfscope}%
\begin{pgfscope}%
\pgfpathrectangle{\pgfqpoint{0.765000in}{0.660000in}}{\pgfqpoint{4.620000in}{4.620000in}}%
\pgfusepath{clip}%
\pgfsetbuttcap%
\pgfsetroundjoin%
\definecolor{currentfill}{rgb}{1.000000,0.894118,0.788235}%
\pgfsetfillcolor{currentfill}%
\pgfsetlinewidth{0.000000pt}%
\definecolor{currentstroke}{rgb}{1.000000,0.894118,0.788235}%
\pgfsetstrokecolor{currentstroke}%
\pgfsetdash{}{0pt}%
\pgfpathmoveto{\pgfqpoint{2.869318in}{3.452500in}}%
\pgfpathlineto{\pgfqpoint{2.788359in}{3.405758in}}%
\pgfpathlineto{\pgfqpoint{2.788359in}{3.499242in}}%
\pgfpathlineto{\pgfqpoint{2.869318in}{3.545983in}}%
\pgfpathlineto{\pgfqpoint{2.869318in}{3.452500in}}%
\pgfpathclose%
\pgfusepath{fill}%
\end{pgfscope}%
\begin{pgfscope}%
\pgfpathrectangle{\pgfqpoint{0.765000in}{0.660000in}}{\pgfqpoint{4.620000in}{4.620000in}}%
\pgfusepath{clip}%
\pgfsetbuttcap%
\pgfsetroundjoin%
\definecolor{currentfill}{rgb}{1.000000,0.894118,0.788235}%
\pgfsetfillcolor{currentfill}%
\pgfsetlinewidth{0.000000pt}%
\definecolor{currentstroke}{rgb}{1.000000,0.894118,0.788235}%
\pgfsetstrokecolor{currentstroke}%
\pgfsetdash{}{0pt}%
\pgfpathmoveto{\pgfqpoint{2.869318in}{3.452500in}}%
\pgfpathlineto{\pgfqpoint{2.950278in}{3.405758in}}%
\pgfpathlineto{\pgfqpoint{2.950278in}{3.499242in}}%
\pgfpathlineto{\pgfqpoint{2.869318in}{3.545983in}}%
\pgfpathlineto{\pgfqpoint{2.869318in}{3.452500in}}%
\pgfpathclose%
\pgfusepath{fill}%
\end{pgfscope}%
\begin{pgfscope}%
\pgfpathrectangle{\pgfqpoint{0.765000in}{0.660000in}}{\pgfqpoint{4.620000in}{4.620000in}}%
\pgfusepath{clip}%
\pgfsetbuttcap%
\pgfsetroundjoin%
\definecolor{currentfill}{rgb}{1.000000,0.894118,0.788235}%
\pgfsetfillcolor{currentfill}%
\pgfsetlinewidth{0.000000pt}%
\definecolor{currentstroke}{rgb}{1.000000,0.894118,0.788235}%
\pgfsetstrokecolor{currentstroke}%
\pgfsetdash{}{0pt}%
\pgfpathmoveto{\pgfqpoint{2.654247in}{3.234845in}}%
\pgfpathlineto{\pgfqpoint{2.735207in}{3.281587in}}%
\pgfpathlineto{\pgfqpoint{2.735207in}{3.375070in}}%
\pgfpathlineto{\pgfqpoint{2.654247in}{3.328328in}}%
\pgfpathlineto{\pgfqpoint{2.654247in}{3.234845in}}%
\pgfpathclose%
\pgfusepath{fill}%
\end{pgfscope}%
\begin{pgfscope}%
\pgfpathrectangle{\pgfqpoint{0.765000in}{0.660000in}}{\pgfqpoint{4.620000in}{4.620000in}}%
\pgfusepath{clip}%
\pgfsetbuttcap%
\pgfsetroundjoin%
\definecolor{currentfill}{rgb}{1.000000,0.894118,0.788235}%
\pgfsetfillcolor{currentfill}%
\pgfsetlinewidth{0.000000pt}%
\definecolor{currentstroke}{rgb}{1.000000,0.894118,0.788235}%
\pgfsetstrokecolor{currentstroke}%
\pgfsetdash{}{0pt}%
\pgfpathmoveto{\pgfqpoint{2.869318in}{3.359016in}}%
\pgfpathlineto{\pgfqpoint{2.950278in}{3.405758in}}%
\pgfpathlineto{\pgfqpoint{2.950278in}{3.499242in}}%
\pgfpathlineto{\pgfqpoint{2.869318in}{3.452500in}}%
\pgfpathlineto{\pgfqpoint{2.869318in}{3.359016in}}%
\pgfpathclose%
\pgfusepath{fill}%
\end{pgfscope}%
\begin{pgfscope}%
\pgfpathrectangle{\pgfqpoint{0.765000in}{0.660000in}}{\pgfqpoint{4.620000in}{4.620000in}}%
\pgfusepath{clip}%
\pgfsetbuttcap%
\pgfsetroundjoin%
\definecolor{currentfill}{rgb}{1.000000,0.894118,0.788235}%
\pgfsetfillcolor{currentfill}%
\pgfsetlinewidth{0.000000pt}%
\definecolor{currentstroke}{rgb}{1.000000,0.894118,0.788235}%
\pgfsetstrokecolor{currentstroke}%
\pgfsetdash{}{0pt}%
\pgfpathmoveto{\pgfqpoint{2.654247in}{3.421812in}}%
\pgfpathlineto{\pgfqpoint{2.573288in}{3.375070in}}%
\pgfpathlineto{\pgfqpoint{2.654247in}{3.328328in}}%
\pgfpathlineto{\pgfqpoint{2.735207in}{3.375070in}}%
\pgfpathlineto{\pgfqpoint{2.654247in}{3.421812in}}%
\pgfpathclose%
\pgfusepath{fill}%
\end{pgfscope}%
\begin{pgfscope}%
\pgfpathrectangle{\pgfqpoint{0.765000in}{0.660000in}}{\pgfqpoint{4.620000in}{4.620000in}}%
\pgfusepath{clip}%
\pgfsetbuttcap%
\pgfsetroundjoin%
\definecolor{currentfill}{rgb}{1.000000,0.894118,0.788235}%
\pgfsetfillcolor{currentfill}%
\pgfsetlinewidth{0.000000pt}%
\definecolor{currentstroke}{rgb}{1.000000,0.894118,0.788235}%
\pgfsetstrokecolor{currentstroke}%
\pgfsetdash{}{0pt}%
\pgfpathmoveto{\pgfqpoint{2.573288in}{3.281587in}}%
\pgfpathlineto{\pgfqpoint{2.654247in}{3.234845in}}%
\pgfpathlineto{\pgfqpoint{2.654247in}{3.328328in}}%
\pgfpathlineto{\pgfqpoint{2.573288in}{3.375070in}}%
\pgfpathlineto{\pgfqpoint{2.573288in}{3.281587in}}%
\pgfpathclose%
\pgfusepath{fill}%
\end{pgfscope}%
\begin{pgfscope}%
\pgfpathrectangle{\pgfqpoint{0.765000in}{0.660000in}}{\pgfqpoint{4.620000in}{4.620000in}}%
\pgfusepath{clip}%
\pgfsetbuttcap%
\pgfsetroundjoin%
\definecolor{currentfill}{rgb}{1.000000,0.894118,0.788235}%
\pgfsetfillcolor{currentfill}%
\pgfsetlinewidth{0.000000pt}%
\definecolor{currentstroke}{rgb}{1.000000,0.894118,0.788235}%
\pgfsetstrokecolor{currentstroke}%
\pgfsetdash{}{0pt}%
\pgfpathmoveto{\pgfqpoint{2.869318in}{3.545983in}}%
\pgfpathlineto{\pgfqpoint{2.788359in}{3.499242in}}%
\pgfpathlineto{\pgfqpoint{2.869318in}{3.452500in}}%
\pgfpathlineto{\pgfqpoint{2.950278in}{3.499242in}}%
\pgfpathlineto{\pgfqpoint{2.869318in}{3.545983in}}%
\pgfpathclose%
\pgfusepath{fill}%
\end{pgfscope}%
\begin{pgfscope}%
\pgfpathrectangle{\pgfqpoint{0.765000in}{0.660000in}}{\pgfqpoint{4.620000in}{4.620000in}}%
\pgfusepath{clip}%
\pgfsetbuttcap%
\pgfsetroundjoin%
\definecolor{currentfill}{rgb}{1.000000,0.894118,0.788235}%
\pgfsetfillcolor{currentfill}%
\pgfsetlinewidth{0.000000pt}%
\definecolor{currentstroke}{rgb}{1.000000,0.894118,0.788235}%
\pgfsetstrokecolor{currentstroke}%
\pgfsetdash{}{0pt}%
\pgfpathmoveto{\pgfqpoint{2.788359in}{3.405758in}}%
\pgfpathlineto{\pgfqpoint{2.869318in}{3.359016in}}%
\pgfpathlineto{\pgfqpoint{2.869318in}{3.452500in}}%
\pgfpathlineto{\pgfqpoint{2.788359in}{3.499242in}}%
\pgfpathlineto{\pgfqpoint{2.788359in}{3.405758in}}%
\pgfpathclose%
\pgfusepath{fill}%
\end{pgfscope}%
\begin{pgfscope}%
\pgfpathrectangle{\pgfqpoint{0.765000in}{0.660000in}}{\pgfqpoint{4.620000in}{4.620000in}}%
\pgfusepath{clip}%
\pgfsetbuttcap%
\pgfsetroundjoin%
\definecolor{currentfill}{rgb}{1.000000,0.894118,0.788235}%
\pgfsetfillcolor{currentfill}%
\pgfsetlinewidth{0.000000pt}%
\definecolor{currentstroke}{rgb}{1.000000,0.894118,0.788235}%
\pgfsetstrokecolor{currentstroke}%
\pgfsetdash{}{0pt}%
\pgfpathmoveto{\pgfqpoint{2.654247in}{3.328328in}}%
\pgfpathlineto{\pgfqpoint{2.573288in}{3.281587in}}%
\pgfpathlineto{\pgfqpoint{2.788359in}{3.405758in}}%
\pgfpathlineto{\pgfqpoint{2.869318in}{3.452500in}}%
\pgfpathlineto{\pgfqpoint{2.654247in}{3.328328in}}%
\pgfpathclose%
\pgfusepath{fill}%
\end{pgfscope}%
\begin{pgfscope}%
\pgfpathrectangle{\pgfqpoint{0.765000in}{0.660000in}}{\pgfqpoint{4.620000in}{4.620000in}}%
\pgfusepath{clip}%
\pgfsetbuttcap%
\pgfsetroundjoin%
\definecolor{currentfill}{rgb}{1.000000,0.894118,0.788235}%
\pgfsetfillcolor{currentfill}%
\pgfsetlinewidth{0.000000pt}%
\definecolor{currentstroke}{rgb}{1.000000,0.894118,0.788235}%
\pgfsetstrokecolor{currentstroke}%
\pgfsetdash{}{0pt}%
\pgfpathmoveto{\pgfqpoint{2.735207in}{3.281587in}}%
\pgfpathlineto{\pgfqpoint{2.654247in}{3.328328in}}%
\pgfpathlineto{\pgfqpoint{2.869318in}{3.452500in}}%
\pgfpathlineto{\pgfqpoint{2.950278in}{3.405758in}}%
\pgfpathlineto{\pgfqpoint{2.735207in}{3.281587in}}%
\pgfpathclose%
\pgfusepath{fill}%
\end{pgfscope}%
\begin{pgfscope}%
\pgfpathrectangle{\pgfqpoint{0.765000in}{0.660000in}}{\pgfqpoint{4.620000in}{4.620000in}}%
\pgfusepath{clip}%
\pgfsetbuttcap%
\pgfsetroundjoin%
\definecolor{currentfill}{rgb}{1.000000,0.894118,0.788235}%
\pgfsetfillcolor{currentfill}%
\pgfsetlinewidth{0.000000pt}%
\definecolor{currentstroke}{rgb}{1.000000,0.894118,0.788235}%
\pgfsetstrokecolor{currentstroke}%
\pgfsetdash{}{0pt}%
\pgfpathmoveto{\pgfqpoint{2.654247in}{3.328328in}}%
\pgfpathlineto{\pgfqpoint{2.654247in}{3.421812in}}%
\pgfpathlineto{\pgfqpoint{2.869318in}{3.545983in}}%
\pgfpathlineto{\pgfqpoint{2.950278in}{3.405758in}}%
\pgfpathlineto{\pgfqpoint{2.654247in}{3.328328in}}%
\pgfpathclose%
\pgfusepath{fill}%
\end{pgfscope}%
\begin{pgfscope}%
\pgfpathrectangle{\pgfqpoint{0.765000in}{0.660000in}}{\pgfqpoint{4.620000in}{4.620000in}}%
\pgfusepath{clip}%
\pgfsetbuttcap%
\pgfsetroundjoin%
\definecolor{currentfill}{rgb}{1.000000,0.894118,0.788235}%
\pgfsetfillcolor{currentfill}%
\pgfsetlinewidth{0.000000pt}%
\definecolor{currentstroke}{rgb}{1.000000,0.894118,0.788235}%
\pgfsetstrokecolor{currentstroke}%
\pgfsetdash{}{0pt}%
\pgfpathmoveto{\pgfqpoint{2.735207in}{3.281587in}}%
\pgfpathlineto{\pgfqpoint{2.735207in}{3.375070in}}%
\pgfpathlineto{\pgfqpoint{2.950278in}{3.499242in}}%
\pgfpathlineto{\pgfqpoint{2.869318in}{3.452500in}}%
\pgfpathlineto{\pgfqpoint{2.735207in}{3.281587in}}%
\pgfpathclose%
\pgfusepath{fill}%
\end{pgfscope}%
\begin{pgfscope}%
\pgfpathrectangle{\pgfqpoint{0.765000in}{0.660000in}}{\pgfqpoint{4.620000in}{4.620000in}}%
\pgfusepath{clip}%
\pgfsetbuttcap%
\pgfsetroundjoin%
\definecolor{currentfill}{rgb}{1.000000,0.894118,0.788235}%
\pgfsetfillcolor{currentfill}%
\pgfsetlinewidth{0.000000pt}%
\definecolor{currentstroke}{rgb}{1.000000,0.894118,0.788235}%
\pgfsetstrokecolor{currentstroke}%
\pgfsetdash{}{0pt}%
\pgfpathmoveto{\pgfqpoint{2.573288in}{3.281587in}}%
\pgfpathlineto{\pgfqpoint{2.654247in}{3.234845in}}%
\pgfpathlineto{\pgfqpoint{2.869318in}{3.359016in}}%
\pgfpathlineto{\pgfqpoint{2.788359in}{3.405758in}}%
\pgfpathlineto{\pgfqpoint{2.573288in}{3.281587in}}%
\pgfpathclose%
\pgfusepath{fill}%
\end{pgfscope}%
\begin{pgfscope}%
\pgfpathrectangle{\pgfqpoint{0.765000in}{0.660000in}}{\pgfqpoint{4.620000in}{4.620000in}}%
\pgfusepath{clip}%
\pgfsetbuttcap%
\pgfsetroundjoin%
\definecolor{currentfill}{rgb}{1.000000,0.894118,0.788235}%
\pgfsetfillcolor{currentfill}%
\pgfsetlinewidth{0.000000pt}%
\definecolor{currentstroke}{rgb}{1.000000,0.894118,0.788235}%
\pgfsetstrokecolor{currentstroke}%
\pgfsetdash{}{0pt}%
\pgfpathmoveto{\pgfqpoint{2.654247in}{3.234845in}}%
\pgfpathlineto{\pgfqpoint{2.735207in}{3.281587in}}%
\pgfpathlineto{\pgfqpoint{2.950278in}{3.405758in}}%
\pgfpathlineto{\pgfqpoint{2.869318in}{3.359016in}}%
\pgfpathlineto{\pgfqpoint{2.654247in}{3.234845in}}%
\pgfpathclose%
\pgfusepath{fill}%
\end{pgfscope}%
\begin{pgfscope}%
\pgfpathrectangle{\pgfqpoint{0.765000in}{0.660000in}}{\pgfqpoint{4.620000in}{4.620000in}}%
\pgfusepath{clip}%
\pgfsetbuttcap%
\pgfsetroundjoin%
\definecolor{currentfill}{rgb}{1.000000,0.894118,0.788235}%
\pgfsetfillcolor{currentfill}%
\pgfsetlinewidth{0.000000pt}%
\definecolor{currentstroke}{rgb}{1.000000,0.894118,0.788235}%
\pgfsetstrokecolor{currentstroke}%
\pgfsetdash{}{0pt}%
\pgfpathmoveto{\pgfqpoint{2.654247in}{3.421812in}}%
\pgfpathlineto{\pgfqpoint{2.573288in}{3.375070in}}%
\pgfpathlineto{\pgfqpoint{2.788359in}{3.499242in}}%
\pgfpathlineto{\pgfqpoint{2.869318in}{3.545983in}}%
\pgfpathlineto{\pgfqpoint{2.654247in}{3.421812in}}%
\pgfpathclose%
\pgfusepath{fill}%
\end{pgfscope}%
\begin{pgfscope}%
\pgfpathrectangle{\pgfqpoint{0.765000in}{0.660000in}}{\pgfqpoint{4.620000in}{4.620000in}}%
\pgfusepath{clip}%
\pgfsetbuttcap%
\pgfsetroundjoin%
\definecolor{currentfill}{rgb}{1.000000,0.894118,0.788235}%
\pgfsetfillcolor{currentfill}%
\pgfsetlinewidth{0.000000pt}%
\definecolor{currentstroke}{rgb}{1.000000,0.894118,0.788235}%
\pgfsetstrokecolor{currentstroke}%
\pgfsetdash{}{0pt}%
\pgfpathmoveto{\pgfqpoint{2.735207in}{3.375070in}}%
\pgfpathlineto{\pgfqpoint{2.654247in}{3.421812in}}%
\pgfpathlineto{\pgfqpoint{2.869318in}{3.545983in}}%
\pgfpathlineto{\pgfqpoint{2.950278in}{3.499242in}}%
\pgfpathlineto{\pgfqpoint{2.735207in}{3.375070in}}%
\pgfpathclose%
\pgfusepath{fill}%
\end{pgfscope}%
\begin{pgfscope}%
\pgfpathrectangle{\pgfqpoint{0.765000in}{0.660000in}}{\pgfqpoint{4.620000in}{4.620000in}}%
\pgfusepath{clip}%
\pgfsetbuttcap%
\pgfsetroundjoin%
\definecolor{currentfill}{rgb}{1.000000,0.894118,0.788235}%
\pgfsetfillcolor{currentfill}%
\pgfsetlinewidth{0.000000pt}%
\definecolor{currentstroke}{rgb}{1.000000,0.894118,0.788235}%
\pgfsetstrokecolor{currentstroke}%
\pgfsetdash{}{0pt}%
\pgfpathmoveto{\pgfqpoint{2.573288in}{3.281587in}}%
\pgfpathlineto{\pgfqpoint{2.573288in}{3.375070in}}%
\pgfpathlineto{\pgfqpoint{2.788359in}{3.499242in}}%
\pgfpathlineto{\pgfqpoint{2.869318in}{3.359016in}}%
\pgfpathlineto{\pgfqpoint{2.573288in}{3.281587in}}%
\pgfpathclose%
\pgfusepath{fill}%
\end{pgfscope}%
\begin{pgfscope}%
\pgfpathrectangle{\pgfqpoint{0.765000in}{0.660000in}}{\pgfqpoint{4.620000in}{4.620000in}}%
\pgfusepath{clip}%
\pgfsetbuttcap%
\pgfsetroundjoin%
\definecolor{currentfill}{rgb}{1.000000,0.894118,0.788235}%
\pgfsetfillcolor{currentfill}%
\pgfsetlinewidth{0.000000pt}%
\definecolor{currentstroke}{rgb}{1.000000,0.894118,0.788235}%
\pgfsetstrokecolor{currentstroke}%
\pgfsetdash{}{0pt}%
\pgfpathmoveto{\pgfqpoint{2.654247in}{3.234845in}}%
\pgfpathlineto{\pgfqpoint{2.654247in}{3.328328in}}%
\pgfpathlineto{\pgfqpoint{2.869318in}{3.452500in}}%
\pgfpathlineto{\pgfqpoint{2.788359in}{3.405758in}}%
\pgfpathlineto{\pgfqpoint{2.654247in}{3.234845in}}%
\pgfpathclose%
\pgfusepath{fill}%
\end{pgfscope}%
\begin{pgfscope}%
\pgfpathrectangle{\pgfqpoint{0.765000in}{0.660000in}}{\pgfqpoint{4.620000in}{4.620000in}}%
\pgfusepath{clip}%
\pgfsetbuttcap%
\pgfsetroundjoin%
\definecolor{currentfill}{rgb}{1.000000,0.894118,0.788235}%
\pgfsetfillcolor{currentfill}%
\pgfsetlinewidth{0.000000pt}%
\definecolor{currentstroke}{rgb}{1.000000,0.894118,0.788235}%
\pgfsetstrokecolor{currentstroke}%
\pgfsetdash{}{0pt}%
\pgfpathmoveto{\pgfqpoint{2.573288in}{3.375070in}}%
\pgfpathlineto{\pgfqpoint{2.654247in}{3.328328in}}%
\pgfpathlineto{\pgfqpoint{2.869318in}{3.452500in}}%
\pgfpathlineto{\pgfqpoint{2.788359in}{3.499242in}}%
\pgfpathlineto{\pgfqpoint{2.573288in}{3.375070in}}%
\pgfpathclose%
\pgfusepath{fill}%
\end{pgfscope}%
\begin{pgfscope}%
\pgfpathrectangle{\pgfqpoint{0.765000in}{0.660000in}}{\pgfqpoint{4.620000in}{4.620000in}}%
\pgfusepath{clip}%
\pgfsetbuttcap%
\pgfsetroundjoin%
\definecolor{currentfill}{rgb}{1.000000,0.894118,0.788235}%
\pgfsetfillcolor{currentfill}%
\pgfsetlinewidth{0.000000pt}%
\definecolor{currentstroke}{rgb}{1.000000,0.894118,0.788235}%
\pgfsetstrokecolor{currentstroke}%
\pgfsetdash{}{0pt}%
\pgfpathmoveto{\pgfqpoint{2.654247in}{3.328328in}}%
\pgfpathlineto{\pgfqpoint{2.735207in}{3.375070in}}%
\pgfpathlineto{\pgfqpoint{2.950278in}{3.499242in}}%
\pgfpathlineto{\pgfqpoint{2.869318in}{3.452500in}}%
\pgfpathlineto{\pgfqpoint{2.654247in}{3.328328in}}%
\pgfpathclose%
\pgfusepath{fill}%
\end{pgfscope}%
\begin{pgfscope}%
\pgfpathrectangle{\pgfqpoint{0.765000in}{0.660000in}}{\pgfqpoint{4.620000in}{4.620000in}}%
\pgfusepath{clip}%
\pgfsetbuttcap%
\pgfsetroundjoin%
\definecolor{currentfill}{rgb}{1.000000,0.894118,0.788235}%
\pgfsetfillcolor{currentfill}%
\pgfsetlinewidth{0.000000pt}%
\definecolor{currentstroke}{rgb}{1.000000,0.894118,0.788235}%
\pgfsetstrokecolor{currentstroke}%
\pgfsetdash{}{0pt}%
\pgfpathmoveto{\pgfqpoint{2.654247in}{3.229621in}}%
\pgfpathlineto{\pgfqpoint{2.573288in}{3.182879in}}%
\pgfpathlineto{\pgfqpoint{2.654247in}{3.136137in}}%
\pgfpathlineto{\pgfqpoint{2.735207in}{3.182879in}}%
\pgfpathlineto{\pgfqpoint{2.654247in}{3.229621in}}%
\pgfpathclose%
\pgfusepath{fill}%
\end{pgfscope}%
\begin{pgfscope}%
\pgfpathrectangle{\pgfqpoint{0.765000in}{0.660000in}}{\pgfqpoint{4.620000in}{4.620000in}}%
\pgfusepath{clip}%
\pgfsetbuttcap%
\pgfsetroundjoin%
\definecolor{currentfill}{rgb}{1.000000,0.894118,0.788235}%
\pgfsetfillcolor{currentfill}%
\pgfsetlinewidth{0.000000pt}%
\definecolor{currentstroke}{rgb}{1.000000,0.894118,0.788235}%
\pgfsetstrokecolor{currentstroke}%
\pgfsetdash{}{0pt}%
\pgfpathmoveto{\pgfqpoint{2.654247in}{3.229621in}}%
\pgfpathlineto{\pgfqpoint{2.573288in}{3.182879in}}%
\pgfpathlineto{\pgfqpoint{2.573288in}{3.276363in}}%
\pgfpathlineto{\pgfqpoint{2.654247in}{3.323105in}}%
\pgfpathlineto{\pgfqpoint{2.654247in}{3.229621in}}%
\pgfpathclose%
\pgfusepath{fill}%
\end{pgfscope}%
\begin{pgfscope}%
\pgfpathrectangle{\pgfqpoint{0.765000in}{0.660000in}}{\pgfqpoint{4.620000in}{4.620000in}}%
\pgfusepath{clip}%
\pgfsetbuttcap%
\pgfsetroundjoin%
\definecolor{currentfill}{rgb}{1.000000,0.894118,0.788235}%
\pgfsetfillcolor{currentfill}%
\pgfsetlinewidth{0.000000pt}%
\definecolor{currentstroke}{rgb}{1.000000,0.894118,0.788235}%
\pgfsetstrokecolor{currentstroke}%
\pgfsetdash{}{0pt}%
\pgfpathmoveto{\pgfqpoint{2.654247in}{3.229621in}}%
\pgfpathlineto{\pgfqpoint{2.735207in}{3.182879in}}%
\pgfpathlineto{\pgfqpoint{2.735207in}{3.276363in}}%
\pgfpathlineto{\pgfqpoint{2.654247in}{3.323105in}}%
\pgfpathlineto{\pgfqpoint{2.654247in}{3.229621in}}%
\pgfpathclose%
\pgfusepath{fill}%
\end{pgfscope}%
\begin{pgfscope}%
\pgfpathrectangle{\pgfqpoint{0.765000in}{0.660000in}}{\pgfqpoint{4.620000in}{4.620000in}}%
\pgfusepath{clip}%
\pgfsetbuttcap%
\pgfsetroundjoin%
\definecolor{currentfill}{rgb}{1.000000,0.894118,0.788235}%
\pgfsetfillcolor{currentfill}%
\pgfsetlinewidth{0.000000pt}%
\definecolor{currentstroke}{rgb}{1.000000,0.894118,0.788235}%
\pgfsetstrokecolor{currentstroke}%
\pgfsetdash{}{0pt}%
\pgfpathmoveto{\pgfqpoint{3.025552in}{2.586502in}}%
\pgfpathlineto{\pgfqpoint{2.944593in}{2.539760in}}%
\pgfpathlineto{\pgfqpoint{3.025552in}{2.493018in}}%
\pgfpathlineto{\pgfqpoint{3.106512in}{2.539760in}}%
\pgfpathlineto{\pgfqpoint{3.025552in}{2.586502in}}%
\pgfpathclose%
\pgfusepath{fill}%
\end{pgfscope}%
\begin{pgfscope}%
\pgfpathrectangle{\pgfqpoint{0.765000in}{0.660000in}}{\pgfqpoint{4.620000in}{4.620000in}}%
\pgfusepath{clip}%
\pgfsetbuttcap%
\pgfsetroundjoin%
\definecolor{currentfill}{rgb}{1.000000,0.894118,0.788235}%
\pgfsetfillcolor{currentfill}%
\pgfsetlinewidth{0.000000pt}%
\definecolor{currentstroke}{rgb}{1.000000,0.894118,0.788235}%
\pgfsetstrokecolor{currentstroke}%
\pgfsetdash{}{0pt}%
\pgfpathmoveto{\pgfqpoint{3.025552in}{2.586502in}}%
\pgfpathlineto{\pgfqpoint{2.944593in}{2.539760in}}%
\pgfpathlineto{\pgfqpoint{2.944593in}{2.633244in}}%
\pgfpathlineto{\pgfqpoint{3.025552in}{2.679986in}}%
\pgfpathlineto{\pgfqpoint{3.025552in}{2.586502in}}%
\pgfpathclose%
\pgfusepath{fill}%
\end{pgfscope}%
\begin{pgfscope}%
\pgfpathrectangle{\pgfqpoint{0.765000in}{0.660000in}}{\pgfqpoint{4.620000in}{4.620000in}}%
\pgfusepath{clip}%
\pgfsetbuttcap%
\pgfsetroundjoin%
\definecolor{currentfill}{rgb}{1.000000,0.894118,0.788235}%
\pgfsetfillcolor{currentfill}%
\pgfsetlinewidth{0.000000pt}%
\definecolor{currentstroke}{rgb}{1.000000,0.894118,0.788235}%
\pgfsetstrokecolor{currentstroke}%
\pgfsetdash{}{0pt}%
\pgfpathmoveto{\pgfqpoint{3.025552in}{2.586502in}}%
\pgfpathlineto{\pgfqpoint{3.106512in}{2.539760in}}%
\pgfpathlineto{\pgfqpoint{3.106512in}{2.633244in}}%
\pgfpathlineto{\pgfqpoint{3.025552in}{2.679986in}}%
\pgfpathlineto{\pgfqpoint{3.025552in}{2.586502in}}%
\pgfpathclose%
\pgfusepath{fill}%
\end{pgfscope}%
\begin{pgfscope}%
\pgfpathrectangle{\pgfqpoint{0.765000in}{0.660000in}}{\pgfqpoint{4.620000in}{4.620000in}}%
\pgfusepath{clip}%
\pgfsetbuttcap%
\pgfsetroundjoin%
\definecolor{currentfill}{rgb}{1.000000,0.894118,0.788235}%
\pgfsetfillcolor{currentfill}%
\pgfsetlinewidth{0.000000pt}%
\definecolor{currentstroke}{rgb}{1.000000,0.894118,0.788235}%
\pgfsetstrokecolor{currentstroke}%
\pgfsetdash{}{0pt}%
\pgfpathmoveto{\pgfqpoint{2.654247in}{3.136137in}}%
\pgfpathlineto{\pgfqpoint{2.735207in}{3.182879in}}%
\pgfpathlineto{\pgfqpoint{2.735207in}{3.276363in}}%
\pgfpathlineto{\pgfqpoint{2.654247in}{3.229621in}}%
\pgfpathlineto{\pgfqpoint{2.654247in}{3.136137in}}%
\pgfpathclose%
\pgfusepath{fill}%
\end{pgfscope}%
\begin{pgfscope}%
\pgfpathrectangle{\pgfqpoint{0.765000in}{0.660000in}}{\pgfqpoint{4.620000in}{4.620000in}}%
\pgfusepath{clip}%
\pgfsetbuttcap%
\pgfsetroundjoin%
\definecolor{currentfill}{rgb}{1.000000,0.894118,0.788235}%
\pgfsetfillcolor{currentfill}%
\pgfsetlinewidth{0.000000pt}%
\definecolor{currentstroke}{rgb}{1.000000,0.894118,0.788235}%
\pgfsetstrokecolor{currentstroke}%
\pgfsetdash{}{0pt}%
\pgfpathmoveto{\pgfqpoint{3.025552in}{2.493018in}}%
\pgfpathlineto{\pgfqpoint{3.106512in}{2.539760in}}%
\pgfpathlineto{\pgfqpoint{3.106512in}{2.633244in}}%
\pgfpathlineto{\pgfqpoint{3.025552in}{2.586502in}}%
\pgfpathlineto{\pgfqpoint{3.025552in}{2.493018in}}%
\pgfpathclose%
\pgfusepath{fill}%
\end{pgfscope}%
\begin{pgfscope}%
\pgfpathrectangle{\pgfqpoint{0.765000in}{0.660000in}}{\pgfqpoint{4.620000in}{4.620000in}}%
\pgfusepath{clip}%
\pgfsetbuttcap%
\pgfsetroundjoin%
\definecolor{currentfill}{rgb}{1.000000,0.894118,0.788235}%
\pgfsetfillcolor{currentfill}%
\pgfsetlinewidth{0.000000pt}%
\definecolor{currentstroke}{rgb}{1.000000,0.894118,0.788235}%
\pgfsetstrokecolor{currentstroke}%
\pgfsetdash{}{0pt}%
\pgfpathmoveto{\pgfqpoint{2.654247in}{3.323105in}}%
\pgfpathlineto{\pgfqpoint{2.573288in}{3.276363in}}%
\pgfpathlineto{\pgfqpoint{2.654247in}{3.229621in}}%
\pgfpathlineto{\pgfqpoint{2.735207in}{3.276363in}}%
\pgfpathlineto{\pgfqpoint{2.654247in}{3.323105in}}%
\pgfpathclose%
\pgfusepath{fill}%
\end{pgfscope}%
\begin{pgfscope}%
\pgfpathrectangle{\pgfqpoint{0.765000in}{0.660000in}}{\pgfqpoint{4.620000in}{4.620000in}}%
\pgfusepath{clip}%
\pgfsetbuttcap%
\pgfsetroundjoin%
\definecolor{currentfill}{rgb}{1.000000,0.894118,0.788235}%
\pgfsetfillcolor{currentfill}%
\pgfsetlinewidth{0.000000pt}%
\definecolor{currentstroke}{rgb}{1.000000,0.894118,0.788235}%
\pgfsetstrokecolor{currentstroke}%
\pgfsetdash{}{0pt}%
\pgfpathmoveto{\pgfqpoint{2.573288in}{3.182879in}}%
\pgfpathlineto{\pgfqpoint{2.654247in}{3.136137in}}%
\pgfpathlineto{\pgfqpoint{2.654247in}{3.229621in}}%
\pgfpathlineto{\pgfqpoint{2.573288in}{3.276363in}}%
\pgfpathlineto{\pgfqpoint{2.573288in}{3.182879in}}%
\pgfpathclose%
\pgfusepath{fill}%
\end{pgfscope}%
\begin{pgfscope}%
\pgfpathrectangle{\pgfqpoint{0.765000in}{0.660000in}}{\pgfqpoint{4.620000in}{4.620000in}}%
\pgfusepath{clip}%
\pgfsetbuttcap%
\pgfsetroundjoin%
\definecolor{currentfill}{rgb}{1.000000,0.894118,0.788235}%
\pgfsetfillcolor{currentfill}%
\pgfsetlinewidth{0.000000pt}%
\definecolor{currentstroke}{rgb}{1.000000,0.894118,0.788235}%
\pgfsetstrokecolor{currentstroke}%
\pgfsetdash{}{0pt}%
\pgfpathmoveto{\pgfqpoint{3.025552in}{2.679986in}}%
\pgfpathlineto{\pgfqpoint{2.944593in}{2.633244in}}%
\pgfpathlineto{\pgfqpoint{3.025552in}{2.586502in}}%
\pgfpathlineto{\pgfqpoint{3.106512in}{2.633244in}}%
\pgfpathlineto{\pgfqpoint{3.025552in}{2.679986in}}%
\pgfpathclose%
\pgfusepath{fill}%
\end{pgfscope}%
\begin{pgfscope}%
\pgfpathrectangle{\pgfqpoint{0.765000in}{0.660000in}}{\pgfqpoint{4.620000in}{4.620000in}}%
\pgfusepath{clip}%
\pgfsetbuttcap%
\pgfsetroundjoin%
\definecolor{currentfill}{rgb}{1.000000,0.894118,0.788235}%
\pgfsetfillcolor{currentfill}%
\pgfsetlinewidth{0.000000pt}%
\definecolor{currentstroke}{rgb}{1.000000,0.894118,0.788235}%
\pgfsetstrokecolor{currentstroke}%
\pgfsetdash{}{0pt}%
\pgfpathmoveto{\pgfqpoint{2.944593in}{2.539760in}}%
\pgfpathlineto{\pgfqpoint{3.025552in}{2.493018in}}%
\pgfpathlineto{\pgfqpoint{3.025552in}{2.586502in}}%
\pgfpathlineto{\pgfqpoint{2.944593in}{2.633244in}}%
\pgfpathlineto{\pgfqpoint{2.944593in}{2.539760in}}%
\pgfpathclose%
\pgfusepath{fill}%
\end{pgfscope}%
\begin{pgfscope}%
\pgfpathrectangle{\pgfqpoint{0.765000in}{0.660000in}}{\pgfqpoint{4.620000in}{4.620000in}}%
\pgfusepath{clip}%
\pgfsetbuttcap%
\pgfsetroundjoin%
\definecolor{currentfill}{rgb}{1.000000,0.894118,0.788235}%
\pgfsetfillcolor{currentfill}%
\pgfsetlinewidth{0.000000pt}%
\definecolor{currentstroke}{rgb}{1.000000,0.894118,0.788235}%
\pgfsetstrokecolor{currentstroke}%
\pgfsetdash{}{0pt}%
\pgfpathmoveto{\pgfqpoint{2.654247in}{3.229621in}}%
\pgfpathlineto{\pgfqpoint{2.573288in}{3.182879in}}%
\pgfpathlineto{\pgfqpoint{2.944593in}{2.539760in}}%
\pgfpathlineto{\pgfqpoint{3.025552in}{2.586502in}}%
\pgfpathlineto{\pgfqpoint{2.654247in}{3.229621in}}%
\pgfpathclose%
\pgfusepath{fill}%
\end{pgfscope}%
\begin{pgfscope}%
\pgfpathrectangle{\pgfqpoint{0.765000in}{0.660000in}}{\pgfqpoint{4.620000in}{4.620000in}}%
\pgfusepath{clip}%
\pgfsetbuttcap%
\pgfsetroundjoin%
\definecolor{currentfill}{rgb}{1.000000,0.894118,0.788235}%
\pgfsetfillcolor{currentfill}%
\pgfsetlinewidth{0.000000pt}%
\definecolor{currentstroke}{rgb}{1.000000,0.894118,0.788235}%
\pgfsetstrokecolor{currentstroke}%
\pgfsetdash{}{0pt}%
\pgfpathmoveto{\pgfqpoint{2.735207in}{3.182879in}}%
\pgfpathlineto{\pgfqpoint{2.654247in}{3.229621in}}%
\pgfpathlineto{\pgfqpoint{3.025552in}{2.586502in}}%
\pgfpathlineto{\pgfqpoint{3.106512in}{2.539760in}}%
\pgfpathlineto{\pgfqpoint{2.735207in}{3.182879in}}%
\pgfpathclose%
\pgfusepath{fill}%
\end{pgfscope}%
\begin{pgfscope}%
\pgfpathrectangle{\pgfqpoint{0.765000in}{0.660000in}}{\pgfqpoint{4.620000in}{4.620000in}}%
\pgfusepath{clip}%
\pgfsetbuttcap%
\pgfsetroundjoin%
\definecolor{currentfill}{rgb}{1.000000,0.894118,0.788235}%
\pgfsetfillcolor{currentfill}%
\pgfsetlinewidth{0.000000pt}%
\definecolor{currentstroke}{rgb}{1.000000,0.894118,0.788235}%
\pgfsetstrokecolor{currentstroke}%
\pgfsetdash{}{0pt}%
\pgfpathmoveto{\pgfqpoint{2.654247in}{3.229621in}}%
\pgfpathlineto{\pgfqpoint{2.654247in}{3.323105in}}%
\pgfpathlineto{\pgfqpoint{3.025552in}{2.679986in}}%
\pgfpathlineto{\pgfqpoint{3.106512in}{2.539760in}}%
\pgfpathlineto{\pgfqpoint{2.654247in}{3.229621in}}%
\pgfpathclose%
\pgfusepath{fill}%
\end{pgfscope}%
\begin{pgfscope}%
\pgfpathrectangle{\pgfqpoint{0.765000in}{0.660000in}}{\pgfqpoint{4.620000in}{4.620000in}}%
\pgfusepath{clip}%
\pgfsetbuttcap%
\pgfsetroundjoin%
\definecolor{currentfill}{rgb}{1.000000,0.894118,0.788235}%
\pgfsetfillcolor{currentfill}%
\pgfsetlinewidth{0.000000pt}%
\definecolor{currentstroke}{rgb}{1.000000,0.894118,0.788235}%
\pgfsetstrokecolor{currentstroke}%
\pgfsetdash{}{0pt}%
\pgfpathmoveto{\pgfqpoint{2.735207in}{3.182879in}}%
\pgfpathlineto{\pgfqpoint{2.735207in}{3.276363in}}%
\pgfpathlineto{\pgfqpoint{3.106512in}{2.633244in}}%
\pgfpathlineto{\pgfqpoint{3.025552in}{2.586502in}}%
\pgfpathlineto{\pgfqpoint{2.735207in}{3.182879in}}%
\pgfpathclose%
\pgfusepath{fill}%
\end{pgfscope}%
\begin{pgfscope}%
\pgfpathrectangle{\pgfqpoint{0.765000in}{0.660000in}}{\pgfqpoint{4.620000in}{4.620000in}}%
\pgfusepath{clip}%
\pgfsetbuttcap%
\pgfsetroundjoin%
\definecolor{currentfill}{rgb}{1.000000,0.894118,0.788235}%
\pgfsetfillcolor{currentfill}%
\pgfsetlinewidth{0.000000pt}%
\definecolor{currentstroke}{rgb}{1.000000,0.894118,0.788235}%
\pgfsetstrokecolor{currentstroke}%
\pgfsetdash{}{0pt}%
\pgfpathmoveto{\pgfqpoint{2.573288in}{3.182879in}}%
\pgfpathlineto{\pgfqpoint{2.654247in}{3.136137in}}%
\pgfpathlineto{\pgfqpoint{3.025552in}{2.493018in}}%
\pgfpathlineto{\pgfqpoint{2.944593in}{2.539760in}}%
\pgfpathlineto{\pgfqpoint{2.573288in}{3.182879in}}%
\pgfpathclose%
\pgfusepath{fill}%
\end{pgfscope}%
\begin{pgfscope}%
\pgfpathrectangle{\pgfqpoint{0.765000in}{0.660000in}}{\pgfqpoint{4.620000in}{4.620000in}}%
\pgfusepath{clip}%
\pgfsetbuttcap%
\pgfsetroundjoin%
\definecolor{currentfill}{rgb}{1.000000,0.894118,0.788235}%
\pgfsetfillcolor{currentfill}%
\pgfsetlinewidth{0.000000pt}%
\definecolor{currentstroke}{rgb}{1.000000,0.894118,0.788235}%
\pgfsetstrokecolor{currentstroke}%
\pgfsetdash{}{0pt}%
\pgfpathmoveto{\pgfqpoint{2.654247in}{3.136137in}}%
\pgfpathlineto{\pgfqpoint{2.735207in}{3.182879in}}%
\pgfpathlineto{\pgfqpoint{3.106512in}{2.539760in}}%
\pgfpathlineto{\pgfqpoint{3.025552in}{2.493018in}}%
\pgfpathlineto{\pgfqpoint{2.654247in}{3.136137in}}%
\pgfpathclose%
\pgfusepath{fill}%
\end{pgfscope}%
\begin{pgfscope}%
\pgfpathrectangle{\pgfqpoint{0.765000in}{0.660000in}}{\pgfqpoint{4.620000in}{4.620000in}}%
\pgfusepath{clip}%
\pgfsetbuttcap%
\pgfsetroundjoin%
\definecolor{currentfill}{rgb}{1.000000,0.894118,0.788235}%
\pgfsetfillcolor{currentfill}%
\pgfsetlinewidth{0.000000pt}%
\definecolor{currentstroke}{rgb}{1.000000,0.894118,0.788235}%
\pgfsetstrokecolor{currentstroke}%
\pgfsetdash{}{0pt}%
\pgfpathmoveto{\pgfqpoint{2.654247in}{3.323105in}}%
\pgfpathlineto{\pgfqpoint{2.573288in}{3.276363in}}%
\pgfpathlineto{\pgfqpoint{2.944593in}{2.633244in}}%
\pgfpathlineto{\pgfqpoint{3.025552in}{2.679986in}}%
\pgfpathlineto{\pgfqpoint{2.654247in}{3.323105in}}%
\pgfpathclose%
\pgfusepath{fill}%
\end{pgfscope}%
\begin{pgfscope}%
\pgfpathrectangle{\pgfqpoint{0.765000in}{0.660000in}}{\pgfqpoint{4.620000in}{4.620000in}}%
\pgfusepath{clip}%
\pgfsetbuttcap%
\pgfsetroundjoin%
\definecolor{currentfill}{rgb}{1.000000,0.894118,0.788235}%
\pgfsetfillcolor{currentfill}%
\pgfsetlinewidth{0.000000pt}%
\definecolor{currentstroke}{rgb}{1.000000,0.894118,0.788235}%
\pgfsetstrokecolor{currentstroke}%
\pgfsetdash{}{0pt}%
\pgfpathmoveto{\pgfqpoint{2.735207in}{3.276363in}}%
\pgfpathlineto{\pgfqpoint{2.654247in}{3.323105in}}%
\pgfpathlineto{\pgfqpoint{3.025552in}{2.679986in}}%
\pgfpathlineto{\pgfqpoint{3.106512in}{2.633244in}}%
\pgfpathlineto{\pgfqpoint{2.735207in}{3.276363in}}%
\pgfpathclose%
\pgfusepath{fill}%
\end{pgfscope}%
\begin{pgfscope}%
\pgfpathrectangle{\pgfqpoint{0.765000in}{0.660000in}}{\pgfqpoint{4.620000in}{4.620000in}}%
\pgfusepath{clip}%
\pgfsetbuttcap%
\pgfsetroundjoin%
\definecolor{currentfill}{rgb}{1.000000,0.894118,0.788235}%
\pgfsetfillcolor{currentfill}%
\pgfsetlinewidth{0.000000pt}%
\definecolor{currentstroke}{rgb}{1.000000,0.894118,0.788235}%
\pgfsetstrokecolor{currentstroke}%
\pgfsetdash{}{0pt}%
\pgfpathmoveto{\pgfqpoint{2.573288in}{3.182879in}}%
\pgfpathlineto{\pgfqpoint{2.573288in}{3.276363in}}%
\pgfpathlineto{\pgfqpoint{2.944593in}{2.633244in}}%
\pgfpathlineto{\pgfqpoint{3.025552in}{2.493018in}}%
\pgfpathlineto{\pgfqpoint{2.573288in}{3.182879in}}%
\pgfpathclose%
\pgfusepath{fill}%
\end{pgfscope}%
\begin{pgfscope}%
\pgfpathrectangle{\pgfqpoint{0.765000in}{0.660000in}}{\pgfqpoint{4.620000in}{4.620000in}}%
\pgfusepath{clip}%
\pgfsetbuttcap%
\pgfsetroundjoin%
\definecolor{currentfill}{rgb}{1.000000,0.894118,0.788235}%
\pgfsetfillcolor{currentfill}%
\pgfsetlinewidth{0.000000pt}%
\definecolor{currentstroke}{rgb}{1.000000,0.894118,0.788235}%
\pgfsetstrokecolor{currentstroke}%
\pgfsetdash{}{0pt}%
\pgfpathmoveto{\pgfqpoint{2.654247in}{3.136137in}}%
\pgfpathlineto{\pgfqpoint{2.654247in}{3.229621in}}%
\pgfpathlineto{\pgfqpoint{3.025552in}{2.586502in}}%
\pgfpathlineto{\pgfqpoint{2.944593in}{2.539760in}}%
\pgfpathlineto{\pgfqpoint{2.654247in}{3.136137in}}%
\pgfpathclose%
\pgfusepath{fill}%
\end{pgfscope}%
\begin{pgfscope}%
\pgfpathrectangle{\pgfqpoint{0.765000in}{0.660000in}}{\pgfqpoint{4.620000in}{4.620000in}}%
\pgfusepath{clip}%
\pgfsetbuttcap%
\pgfsetroundjoin%
\definecolor{currentfill}{rgb}{1.000000,0.894118,0.788235}%
\pgfsetfillcolor{currentfill}%
\pgfsetlinewidth{0.000000pt}%
\definecolor{currentstroke}{rgb}{1.000000,0.894118,0.788235}%
\pgfsetstrokecolor{currentstroke}%
\pgfsetdash{}{0pt}%
\pgfpathmoveto{\pgfqpoint{2.573288in}{3.276363in}}%
\pgfpathlineto{\pgfqpoint{2.654247in}{3.229621in}}%
\pgfpathlineto{\pgfqpoint{3.025552in}{2.586502in}}%
\pgfpathlineto{\pgfqpoint{2.944593in}{2.633244in}}%
\pgfpathlineto{\pgfqpoint{2.573288in}{3.276363in}}%
\pgfpathclose%
\pgfusepath{fill}%
\end{pgfscope}%
\begin{pgfscope}%
\pgfpathrectangle{\pgfqpoint{0.765000in}{0.660000in}}{\pgfqpoint{4.620000in}{4.620000in}}%
\pgfusepath{clip}%
\pgfsetbuttcap%
\pgfsetroundjoin%
\definecolor{currentfill}{rgb}{1.000000,0.894118,0.788235}%
\pgfsetfillcolor{currentfill}%
\pgfsetlinewidth{0.000000pt}%
\definecolor{currentstroke}{rgb}{1.000000,0.894118,0.788235}%
\pgfsetstrokecolor{currentstroke}%
\pgfsetdash{}{0pt}%
\pgfpathmoveto{\pgfqpoint{2.654247in}{3.229621in}}%
\pgfpathlineto{\pgfqpoint{2.735207in}{3.276363in}}%
\pgfpathlineto{\pgfqpoint{3.106512in}{2.633244in}}%
\pgfpathlineto{\pgfqpoint{3.025552in}{2.586502in}}%
\pgfpathlineto{\pgfqpoint{2.654247in}{3.229621in}}%
\pgfpathclose%
\pgfusepath{fill}%
\end{pgfscope}%
\begin{pgfscope}%
\pgfpathrectangle{\pgfqpoint{0.765000in}{0.660000in}}{\pgfqpoint{4.620000in}{4.620000in}}%
\pgfusepath{clip}%
\pgfsetbuttcap%
\pgfsetroundjoin%
\definecolor{currentfill}{rgb}{1.000000,0.894118,0.788235}%
\pgfsetfillcolor{currentfill}%
\pgfsetlinewidth{0.000000pt}%
\definecolor{currentstroke}{rgb}{1.000000,0.894118,0.788235}%
\pgfsetstrokecolor{currentstroke}%
\pgfsetdash{}{0pt}%
\pgfpathmoveto{\pgfqpoint{2.654247in}{3.229621in}}%
\pgfpathlineto{\pgfqpoint{2.573288in}{3.182879in}}%
\pgfpathlineto{\pgfqpoint{2.654247in}{3.136137in}}%
\pgfpathlineto{\pgfqpoint{2.735207in}{3.182879in}}%
\pgfpathlineto{\pgfqpoint{2.654247in}{3.229621in}}%
\pgfpathclose%
\pgfusepath{fill}%
\end{pgfscope}%
\begin{pgfscope}%
\pgfpathrectangle{\pgfqpoint{0.765000in}{0.660000in}}{\pgfqpoint{4.620000in}{4.620000in}}%
\pgfusepath{clip}%
\pgfsetbuttcap%
\pgfsetroundjoin%
\definecolor{currentfill}{rgb}{1.000000,0.894118,0.788235}%
\pgfsetfillcolor{currentfill}%
\pgfsetlinewidth{0.000000pt}%
\definecolor{currentstroke}{rgb}{1.000000,0.894118,0.788235}%
\pgfsetstrokecolor{currentstroke}%
\pgfsetdash{}{0pt}%
\pgfpathmoveto{\pgfqpoint{2.654247in}{3.229621in}}%
\pgfpathlineto{\pgfqpoint{2.573288in}{3.182879in}}%
\pgfpathlineto{\pgfqpoint{2.573288in}{3.276363in}}%
\pgfpathlineto{\pgfqpoint{2.654247in}{3.323105in}}%
\pgfpathlineto{\pgfqpoint{2.654247in}{3.229621in}}%
\pgfpathclose%
\pgfusepath{fill}%
\end{pgfscope}%
\begin{pgfscope}%
\pgfpathrectangle{\pgfqpoint{0.765000in}{0.660000in}}{\pgfqpoint{4.620000in}{4.620000in}}%
\pgfusepath{clip}%
\pgfsetbuttcap%
\pgfsetroundjoin%
\definecolor{currentfill}{rgb}{1.000000,0.894118,0.788235}%
\pgfsetfillcolor{currentfill}%
\pgfsetlinewidth{0.000000pt}%
\definecolor{currentstroke}{rgb}{1.000000,0.894118,0.788235}%
\pgfsetstrokecolor{currentstroke}%
\pgfsetdash{}{0pt}%
\pgfpathmoveto{\pgfqpoint{2.654247in}{3.229621in}}%
\pgfpathlineto{\pgfqpoint{2.735207in}{3.182879in}}%
\pgfpathlineto{\pgfqpoint{2.735207in}{3.276363in}}%
\pgfpathlineto{\pgfqpoint{2.654247in}{3.323105in}}%
\pgfpathlineto{\pgfqpoint{2.654247in}{3.229621in}}%
\pgfpathclose%
\pgfusepath{fill}%
\end{pgfscope}%
\begin{pgfscope}%
\pgfpathrectangle{\pgfqpoint{0.765000in}{0.660000in}}{\pgfqpoint{4.620000in}{4.620000in}}%
\pgfusepath{clip}%
\pgfsetbuttcap%
\pgfsetroundjoin%
\definecolor{currentfill}{rgb}{1.000000,0.894118,0.788235}%
\pgfsetfillcolor{currentfill}%
\pgfsetlinewidth{0.000000pt}%
\definecolor{currentstroke}{rgb}{1.000000,0.894118,0.788235}%
\pgfsetstrokecolor{currentstroke}%
\pgfsetdash{}{0pt}%
\pgfpathmoveto{\pgfqpoint{2.654247in}{3.328328in}}%
\pgfpathlineto{\pgfqpoint{2.573288in}{3.281587in}}%
\pgfpathlineto{\pgfqpoint{2.654247in}{3.234845in}}%
\pgfpathlineto{\pgfqpoint{2.735207in}{3.281587in}}%
\pgfpathlineto{\pgfqpoint{2.654247in}{3.328328in}}%
\pgfpathclose%
\pgfusepath{fill}%
\end{pgfscope}%
\begin{pgfscope}%
\pgfpathrectangle{\pgfqpoint{0.765000in}{0.660000in}}{\pgfqpoint{4.620000in}{4.620000in}}%
\pgfusepath{clip}%
\pgfsetbuttcap%
\pgfsetroundjoin%
\definecolor{currentfill}{rgb}{1.000000,0.894118,0.788235}%
\pgfsetfillcolor{currentfill}%
\pgfsetlinewidth{0.000000pt}%
\definecolor{currentstroke}{rgb}{1.000000,0.894118,0.788235}%
\pgfsetstrokecolor{currentstroke}%
\pgfsetdash{}{0pt}%
\pgfpathmoveto{\pgfqpoint{2.654247in}{3.328328in}}%
\pgfpathlineto{\pgfqpoint{2.573288in}{3.281587in}}%
\pgfpathlineto{\pgfqpoint{2.573288in}{3.375070in}}%
\pgfpathlineto{\pgfqpoint{2.654247in}{3.421812in}}%
\pgfpathlineto{\pgfqpoint{2.654247in}{3.328328in}}%
\pgfpathclose%
\pgfusepath{fill}%
\end{pgfscope}%
\begin{pgfscope}%
\pgfpathrectangle{\pgfqpoint{0.765000in}{0.660000in}}{\pgfqpoint{4.620000in}{4.620000in}}%
\pgfusepath{clip}%
\pgfsetbuttcap%
\pgfsetroundjoin%
\definecolor{currentfill}{rgb}{1.000000,0.894118,0.788235}%
\pgfsetfillcolor{currentfill}%
\pgfsetlinewidth{0.000000pt}%
\definecolor{currentstroke}{rgb}{1.000000,0.894118,0.788235}%
\pgfsetstrokecolor{currentstroke}%
\pgfsetdash{}{0pt}%
\pgfpathmoveto{\pgfqpoint{2.654247in}{3.328328in}}%
\pgfpathlineto{\pgfqpoint{2.735207in}{3.281587in}}%
\pgfpathlineto{\pgfqpoint{2.735207in}{3.375070in}}%
\pgfpathlineto{\pgfqpoint{2.654247in}{3.421812in}}%
\pgfpathlineto{\pgfqpoint{2.654247in}{3.328328in}}%
\pgfpathclose%
\pgfusepath{fill}%
\end{pgfscope}%
\begin{pgfscope}%
\pgfpathrectangle{\pgfqpoint{0.765000in}{0.660000in}}{\pgfqpoint{4.620000in}{4.620000in}}%
\pgfusepath{clip}%
\pgfsetbuttcap%
\pgfsetroundjoin%
\definecolor{currentfill}{rgb}{1.000000,0.894118,0.788235}%
\pgfsetfillcolor{currentfill}%
\pgfsetlinewidth{0.000000pt}%
\definecolor{currentstroke}{rgb}{1.000000,0.894118,0.788235}%
\pgfsetstrokecolor{currentstroke}%
\pgfsetdash{}{0pt}%
\pgfpathmoveto{\pgfqpoint{2.654247in}{3.136137in}}%
\pgfpathlineto{\pgfqpoint{2.735207in}{3.182879in}}%
\pgfpathlineto{\pgfqpoint{2.735207in}{3.276363in}}%
\pgfpathlineto{\pgfqpoint{2.654247in}{3.229621in}}%
\pgfpathlineto{\pgfqpoint{2.654247in}{3.136137in}}%
\pgfpathclose%
\pgfusepath{fill}%
\end{pgfscope}%
\begin{pgfscope}%
\pgfpathrectangle{\pgfqpoint{0.765000in}{0.660000in}}{\pgfqpoint{4.620000in}{4.620000in}}%
\pgfusepath{clip}%
\pgfsetbuttcap%
\pgfsetroundjoin%
\definecolor{currentfill}{rgb}{1.000000,0.894118,0.788235}%
\pgfsetfillcolor{currentfill}%
\pgfsetlinewidth{0.000000pt}%
\definecolor{currentstroke}{rgb}{1.000000,0.894118,0.788235}%
\pgfsetstrokecolor{currentstroke}%
\pgfsetdash{}{0pt}%
\pgfpathmoveto{\pgfqpoint{2.654247in}{3.234845in}}%
\pgfpathlineto{\pgfqpoint{2.735207in}{3.281587in}}%
\pgfpathlineto{\pgfqpoint{2.735207in}{3.375070in}}%
\pgfpathlineto{\pgfqpoint{2.654247in}{3.328328in}}%
\pgfpathlineto{\pgfqpoint{2.654247in}{3.234845in}}%
\pgfpathclose%
\pgfusepath{fill}%
\end{pgfscope}%
\begin{pgfscope}%
\pgfpathrectangle{\pgfqpoint{0.765000in}{0.660000in}}{\pgfqpoint{4.620000in}{4.620000in}}%
\pgfusepath{clip}%
\pgfsetbuttcap%
\pgfsetroundjoin%
\definecolor{currentfill}{rgb}{1.000000,0.894118,0.788235}%
\pgfsetfillcolor{currentfill}%
\pgfsetlinewidth{0.000000pt}%
\definecolor{currentstroke}{rgb}{1.000000,0.894118,0.788235}%
\pgfsetstrokecolor{currentstroke}%
\pgfsetdash{}{0pt}%
\pgfpathmoveto{\pgfqpoint{2.654247in}{3.323105in}}%
\pgfpathlineto{\pgfqpoint{2.573288in}{3.276363in}}%
\pgfpathlineto{\pgfqpoint{2.654247in}{3.229621in}}%
\pgfpathlineto{\pgfqpoint{2.735207in}{3.276363in}}%
\pgfpathlineto{\pgfqpoint{2.654247in}{3.323105in}}%
\pgfpathclose%
\pgfusepath{fill}%
\end{pgfscope}%
\begin{pgfscope}%
\pgfpathrectangle{\pgfqpoint{0.765000in}{0.660000in}}{\pgfqpoint{4.620000in}{4.620000in}}%
\pgfusepath{clip}%
\pgfsetbuttcap%
\pgfsetroundjoin%
\definecolor{currentfill}{rgb}{1.000000,0.894118,0.788235}%
\pgfsetfillcolor{currentfill}%
\pgfsetlinewidth{0.000000pt}%
\definecolor{currentstroke}{rgb}{1.000000,0.894118,0.788235}%
\pgfsetstrokecolor{currentstroke}%
\pgfsetdash{}{0pt}%
\pgfpathmoveto{\pgfqpoint{2.573288in}{3.182879in}}%
\pgfpathlineto{\pgfqpoint{2.654247in}{3.136137in}}%
\pgfpathlineto{\pgfqpoint{2.654247in}{3.229621in}}%
\pgfpathlineto{\pgfqpoint{2.573288in}{3.276363in}}%
\pgfpathlineto{\pgfqpoint{2.573288in}{3.182879in}}%
\pgfpathclose%
\pgfusepath{fill}%
\end{pgfscope}%
\begin{pgfscope}%
\pgfpathrectangle{\pgfqpoint{0.765000in}{0.660000in}}{\pgfqpoint{4.620000in}{4.620000in}}%
\pgfusepath{clip}%
\pgfsetbuttcap%
\pgfsetroundjoin%
\definecolor{currentfill}{rgb}{1.000000,0.894118,0.788235}%
\pgfsetfillcolor{currentfill}%
\pgfsetlinewidth{0.000000pt}%
\definecolor{currentstroke}{rgb}{1.000000,0.894118,0.788235}%
\pgfsetstrokecolor{currentstroke}%
\pgfsetdash{}{0pt}%
\pgfpathmoveto{\pgfqpoint{2.654247in}{3.421812in}}%
\pgfpathlineto{\pgfqpoint{2.573288in}{3.375070in}}%
\pgfpathlineto{\pgfqpoint{2.654247in}{3.328328in}}%
\pgfpathlineto{\pgfqpoint{2.735207in}{3.375070in}}%
\pgfpathlineto{\pgfqpoint{2.654247in}{3.421812in}}%
\pgfpathclose%
\pgfusepath{fill}%
\end{pgfscope}%
\begin{pgfscope}%
\pgfpathrectangle{\pgfqpoint{0.765000in}{0.660000in}}{\pgfqpoint{4.620000in}{4.620000in}}%
\pgfusepath{clip}%
\pgfsetbuttcap%
\pgfsetroundjoin%
\definecolor{currentfill}{rgb}{1.000000,0.894118,0.788235}%
\pgfsetfillcolor{currentfill}%
\pgfsetlinewidth{0.000000pt}%
\definecolor{currentstroke}{rgb}{1.000000,0.894118,0.788235}%
\pgfsetstrokecolor{currentstroke}%
\pgfsetdash{}{0pt}%
\pgfpathmoveto{\pgfqpoint{2.573288in}{3.281587in}}%
\pgfpathlineto{\pgfqpoint{2.654247in}{3.234845in}}%
\pgfpathlineto{\pgfqpoint{2.654247in}{3.328328in}}%
\pgfpathlineto{\pgfqpoint{2.573288in}{3.375070in}}%
\pgfpathlineto{\pgfqpoint{2.573288in}{3.281587in}}%
\pgfpathclose%
\pgfusepath{fill}%
\end{pgfscope}%
\begin{pgfscope}%
\pgfpathrectangle{\pgfqpoint{0.765000in}{0.660000in}}{\pgfqpoint{4.620000in}{4.620000in}}%
\pgfusepath{clip}%
\pgfsetbuttcap%
\pgfsetroundjoin%
\definecolor{currentfill}{rgb}{1.000000,0.894118,0.788235}%
\pgfsetfillcolor{currentfill}%
\pgfsetlinewidth{0.000000pt}%
\definecolor{currentstroke}{rgb}{1.000000,0.894118,0.788235}%
\pgfsetstrokecolor{currentstroke}%
\pgfsetdash{}{0pt}%
\pgfpathmoveto{\pgfqpoint{2.654247in}{3.229621in}}%
\pgfpathlineto{\pgfqpoint{2.573288in}{3.182879in}}%
\pgfpathlineto{\pgfqpoint{2.573288in}{3.281587in}}%
\pgfpathlineto{\pgfqpoint{2.654247in}{3.328328in}}%
\pgfpathlineto{\pgfqpoint{2.654247in}{3.229621in}}%
\pgfpathclose%
\pgfusepath{fill}%
\end{pgfscope}%
\begin{pgfscope}%
\pgfpathrectangle{\pgfqpoint{0.765000in}{0.660000in}}{\pgfqpoint{4.620000in}{4.620000in}}%
\pgfusepath{clip}%
\pgfsetbuttcap%
\pgfsetroundjoin%
\definecolor{currentfill}{rgb}{1.000000,0.894118,0.788235}%
\pgfsetfillcolor{currentfill}%
\pgfsetlinewidth{0.000000pt}%
\definecolor{currentstroke}{rgb}{1.000000,0.894118,0.788235}%
\pgfsetstrokecolor{currentstroke}%
\pgfsetdash{}{0pt}%
\pgfpathmoveto{\pgfqpoint{2.735207in}{3.182879in}}%
\pgfpathlineto{\pgfqpoint{2.654247in}{3.229621in}}%
\pgfpathlineto{\pgfqpoint{2.654247in}{3.328328in}}%
\pgfpathlineto{\pgfqpoint{2.735207in}{3.281587in}}%
\pgfpathlineto{\pgfqpoint{2.735207in}{3.182879in}}%
\pgfpathclose%
\pgfusepath{fill}%
\end{pgfscope}%
\begin{pgfscope}%
\pgfpathrectangle{\pgfqpoint{0.765000in}{0.660000in}}{\pgfqpoint{4.620000in}{4.620000in}}%
\pgfusepath{clip}%
\pgfsetbuttcap%
\pgfsetroundjoin%
\definecolor{currentfill}{rgb}{1.000000,0.894118,0.788235}%
\pgfsetfillcolor{currentfill}%
\pgfsetlinewidth{0.000000pt}%
\definecolor{currentstroke}{rgb}{1.000000,0.894118,0.788235}%
\pgfsetstrokecolor{currentstroke}%
\pgfsetdash{}{0pt}%
\pgfpathmoveto{\pgfqpoint{2.654247in}{3.229621in}}%
\pgfpathlineto{\pgfqpoint{2.654247in}{3.323105in}}%
\pgfpathlineto{\pgfqpoint{2.654247in}{3.421812in}}%
\pgfpathlineto{\pgfqpoint{2.735207in}{3.281587in}}%
\pgfpathlineto{\pgfqpoint{2.654247in}{3.229621in}}%
\pgfpathclose%
\pgfusepath{fill}%
\end{pgfscope}%
\begin{pgfscope}%
\pgfpathrectangle{\pgfqpoint{0.765000in}{0.660000in}}{\pgfqpoint{4.620000in}{4.620000in}}%
\pgfusepath{clip}%
\pgfsetbuttcap%
\pgfsetroundjoin%
\definecolor{currentfill}{rgb}{1.000000,0.894118,0.788235}%
\pgfsetfillcolor{currentfill}%
\pgfsetlinewidth{0.000000pt}%
\definecolor{currentstroke}{rgb}{1.000000,0.894118,0.788235}%
\pgfsetstrokecolor{currentstroke}%
\pgfsetdash{}{0pt}%
\pgfpathmoveto{\pgfqpoint{2.735207in}{3.182879in}}%
\pgfpathlineto{\pgfqpoint{2.735207in}{3.276363in}}%
\pgfpathlineto{\pgfqpoint{2.735207in}{3.375070in}}%
\pgfpathlineto{\pgfqpoint{2.654247in}{3.328328in}}%
\pgfpathlineto{\pgfqpoint{2.735207in}{3.182879in}}%
\pgfpathclose%
\pgfusepath{fill}%
\end{pgfscope}%
\begin{pgfscope}%
\pgfpathrectangle{\pgfqpoint{0.765000in}{0.660000in}}{\pgfqpoint{4.620000in}{4.620000in}}%
\pgfusepath{clip}%
\pgfsetbuttcap%
\pgfsetroundjoin%
\definecolor{currentfill}{rgb}{1.000000,0.894118,0.788235}%
\pgfsetfillcolor{currentfill}%
\pgfsetlinewidth{0.000000pt}%
\definecolor{currentstroke}{rgb}{1.000000,0.894118,0.788235}%
\pgfsetstrokecolor{currentstroke}%
\pgfsetdash{}{0pt}%
\pgfpathmoveto{\pgfqpoint{2.573288in}{3.182879in}}%
\pgfpathlineto{\pgfqpoint{2.654247in}{3.136137in}}%
\pgfpathlineto{\pgfqpoint{2.654247in}{3.234845in}}%
\pgfpathlineto{\pgfqpoint{2.573288in}{3.281587in}}%
\pgfpathlineto{\pgfqpoint{2.573288in}{3.182879in}}%
\pgfpathclose%
\pgfusepath{fill}%
\end{pgfscope}%
\begin{pgfscope}%
\pgfpathrectangle{\pgfqpoint{0.765000in}{0.660000in}}{\pgfqpoint{4.620000in}{4.620000in}}%
\pgfusepath{clip}%
\pgfsetbuttcap%
\pgfsetroundjoin%
\definecolor{currentfill}{rgb}{1.000000,0.894118,0.788235}%
\pgfsetfillcolor{currentfill}%
\pgfsetlinewidth{0.000000pt}%
\definecolor{currentstroke}{rgb}{1.000000,0.894118,0.788235}%
\pgfsetstrokecolor{currentstroke}%
\pgfsetdash{}{0pt}%
\pgfpathmoveto{\pgfqpoint{2.654247in}{3.136137in}}%
\pgfpathlineto{\pgfqpoint{2.735207in}{3.182879in}}%
\pgfpathlineto{\pgfqpoint{2.735207in}{3.281587in}}%
\pgfpathlineto{\pgfqpoint{2.654247in}{3.234845in}}%
\pgfpathlineto{\pgfqpoint{2.654247in}{3.136137in}}%
\pgfpathclose%
\pgfusepath{fill}%
\end{pgfscope}%
\begin{pgfscope}%
\pgfpathrectangle{\pgfqpoint{0.765000in}{0.660000in}}{\pgfqpoint{4.620000in}{4.620000in}}%
\pgfusepath{clip}%
\pgfsetbuttcap%
\pgfsetroundjoin%
\definecolor{currentfill}{rgb}{1.000000,0.894118,0.788235}%
\pgfsetfillcolor{currentfill}%
\pgfsetlinewidth{0.000000pt}%
\definecolor{currentstroke}{rgb}{1.000000,0.894118,0.788235}%
\pgfsetstrokecolor{currentstroke}%
\pgfsetdash{}{0pt}%
\pgfpathmoveto{\pgfqpoint{2.654247in}{3.323105in}}%
\pgfpathlineto{\pgfqpoint{2.573288in}{3.276363in}}%
\pgfpathlineto{\pgfqpoint{2.573288in}{3.375070in}}%
\pgfpathlineto{\pgfqpoint{2.654247in}{3.421812in}}%
\pgfpathlineto{\pgfqpoint{2.654247in}{3.323105in}}%
\pgfpathclose%
\pgfusepath{fill}%
\end{pgfscope}%
\begin{pgfscope}%
\pgfpathrectangle{\pgfqpoint{0.765000in}{0.660000in}}{\pgfqpoint{4.620000in}{4.620000in}}%
\pgfusepath{clip}%
\pgfsetbuttcap%
\pgfsetroundjoin%
\definecolor{currentfill}{rgb}{1.000000,0.894118,0.788235}%
\pgfsetfillcolor{currentfill}%
\pgfsetlinewidth{0.000000pt}%
\definecolor{currentstroke}{rgb}{1.000000,0.894118,0.788235}%
\pgfsetstrokecolor{currentstroke}%
\pgfsetdash{}{0pt}%
\pgfpathmoveto{\pgfqpoint{2.735207in}{3.276363in}}%
\pgfpathlineto{\pgfqpoint{2.654247in}{3.323105in}}%
\pgfpathlineto{\pgfqpoint{2.654247in}{3.421812in}}%
\pgfpathlineto{\pgfqpoint{2.735207in}{3.375070in}}%
\pgfpathlineto{\pgfqpoint{2.735207in}{3.276363in}}%
\pgfpathclose%
\pgfusepath{fill}%
\end{pgfscope}%
\begin{pgfscope}%
\pgfpathrectangle{\pgfqpoint{0.765000in}{0.660000in}}{\pgfqpoint{4.620000in}{4.620000in}}%
\pgfusepath{clip}%
\pgfsetbuttcap%
\pgfsetroundjoin%
\definecolor{currentfill}{rgb}{1.000000,0.894118,0.788235}%
\pgfsetfillcolor{currentfill}%
\pgfsetlinewidth{0.000000pt}%
\definecolor{currentstroke}{rgb}{1.000000,0.894118,0.788235}%
\pgfsetstrokecolor{currentstroke}%
\pgfsetdash{}{0pt}%
\pgfpathmoveto{\pgfqpoint{2.573288in}{3.182879in}}%
\pgfpathlineto{\pgfqpoint{2.573288in}{3.276363in}}%
\pgfpathlineto{\pgfqpoint{2.573288in}{3.375070in}}%
\pgfpathlineto{\pgfqpoint{2.654247in}{3.234845in}}%
\pgfpathlineto{\pgfqpoint{2.573288in}{3.182879in}}%
\pgfpathclose%
\pgfusepath{fill}%
\end{pgfscope}%
\begin{pgfscope}%
\pgfpathrectangle{\pgfqpoint{0.765000in}{0.660000in}}{\pgfqpoint{4.620000in}{4.620000in}}%
\pgfusepath{clip}%
\pgfsetbuttcap%
\pgfsetroundjoin%
\definecolor{currentfill}{rgb}{1.000000,0.894118,0.788235}%
\pgfsetfillcolor{currentfill}%
\pgfsetlinewidth{0.000000pt}%
\definecolor{currentstroke}{rgb}{1.000000,0.894118,0.788235}%
\pgfsetstrokecolor{currentstroke}%
\pgfsetdash{}{0pt}%
\pgfpathmoveto{\pgfqpoint{2.654247in}{3.136137in}}%
\pgfpathlineto{\pgfqpoint{2.654247in}{3.229621in}}%
\pgfpathlineto{\pgfqpoint{2.654247in}{3.328328in}}%
\pgfpathlineto{\pgfqpoint{2.573288in}{3.281587in}}%
\pgfpathlineto{\pgfqpoint{2.654247in}{3.136137in}}%
\pgfpathclose%
\pgfusepath{fill}%
\end{pgfscope}%
\begin{pgfscope}%
\pgfpathrectangle{\pgfqpoint{0.765000in}{0.660000in}}{\pgfqpoint{4.620000in}{4.620000in}}%
\pgfusepath{clip}%
\pgfsetbuttcap%
\pgfsetroundjoin%
\definecolor{currentfill}{rgb}{1.000000,0.894118,0.788235}%
\pgfsetfillcolor{currentfill}%
\pgfsetlinewidth{0.000000pt}%
\definecolor{currentstroke}{rgb}{1.000000,0.894118,0.788235}%
\pgfsetstrokecolor{currentstroke}%
\pgfsetdash{}{0pt}%
\pgfpathmoveto{\pgfqpoint{2.573288in}{3.276363in}}%
\pgfpathlineto{\pgfqpoint{2.654247in}{3.229621in}}%
\pgfpathlineto{\pgfqpoint{2.654247in}{3.328328in}}%
\pgfpathlineto{\pgfqpoint{2.573288in}{3.375070in}}%
\pgfpathlineto{\pgfqpoint{2.573288in}{3.276363in}}%
\pgfpathclose%
\pgfusepath{fill}%
\end{pgfscope}%
\begin{pgfscope}%
\pgfpathrectangle{\pgfqpoint{0.765000in}{0.660000in}}{\pgfqpoint{4.620000in}{4.620000in}}%
\pgfusepath{clip}%
\pgfsetbuttcap%
\pgfsetroundjoin%
\definecolor{currentfill}{rgb}{1.000000,0.894118,0.788235}%
\pgfsetfillcolor{currentfill}%
\pgfsetlinewidth{0.000000pt}%
\definecolor{currentstroke}{rgb}{1.000000,0.894118,0.788235}%
\pgfsetstrokecolor{currentstroke}%
\pgfsetdash{}{0pt}%
\pgfpathmoveto{\pgfqpoint{2.654247in}{3.229621in}}%
\pgfpathlineto{\pgfqpoint{2.735207in}{3.276363in}}%
\pgfpathlineto{\pgfqpoint{2.735207in}{3.375070in}}%
\pgfpathlineto{\pgfqpoint{2.654247in}{3.328328in}}%
\pgfpathlineto{\pgfqpoint{2.654247in}{3.229621in}}%
\pgfpathclose%
\pgfusepath{fill}%
\end{pgfscope}%
\begin{pgfscope}%
\pgfpathrectangle{\pgfqpoint{0.765000in}{0.660000in}}{\pgfqpoint{4.620000in}{4.620000in}}%
\pgfusepath{clip}%
\pgfsetbuttcap%
\pgfsetroundjoin%
\definecolor{currentfill}{rgb}{1.000000,0.894118,0.788235}%
\pgfsetfillcolor{currentfill}%
\pgfsetlinewidth{0.000000pt}%
\definecolor{currentstroke}{rgb}{1.000000,0.894118,0.788235}%
\pgfsetstrokecolor{currentstroke}%
\pgfsetdash{}{0pt}%
\pgfpathmoveto{\pgfqpoint{3.593460in}{3.066768in}}%
\pgfpathlineto{\pgfqpoint{3.512500in}{3.020026in}}%
\pgfpathlineto{\pgfqpoint{3.593460in}{2.973284in}}%
\pgfpathlineto{\pgfqpoint{3.674419in}{3.020026in}}%
\pgfpathlineto{\pgfqpoint{3.593460in}{3.066768in}}%
\pgfpathclose%
\pgfusepath{fill}%
\end{pgfscope}%
\begin{pgfscope}%
\pgfpathrectangle{\pgfqpoint{0.765000in}{0.660000in}}{\pgfqpoint{4.620000in}{4.620000in}}%
\pgfusepath{clip}%
\pgfsetbuttcap%
\pgfsetroundjoin%
\definecolor{currentfill}{rgb}{1.000000,0.894118,0.788235}%
\pgfsetfillcolor{currentfill}%
\pgfsetlinewidth{0.000000pt}%
\definecolor{currentstroke}{rgb}{1.000000,0.894118,0.788235}%
\pgfsetstrokecolor{currentstroke}%
\pgfsetdash{}{0pt}%
\pgfpathmoveto{\pgfqpoint{3.593460in}{3.066768in}}%
\pgfpathlineto{\pgfqpoint{3.512500in}{3.020026in}}%
\pgfpathlineto{\pgfqpoint{3.512500in}{3.113510in}}%
\pgfpathlineto{\pgfqpoint{3.593460in}{3.160252in}}%
\pgfpathlineto{\pgfqpoint{3.593460in}{3.066768in}}%
\pgfpathclose%
\pgfusepath{fill}%
\end{pgfscope}%
\begin{pgfscope}%
\pgfpathrectangle{\pgfqpoint{0.765000in}{0.660000in}}{\pgfqpoint{4.620000in}{4.620000in}}%
\pgfusepath{clip}%
\pgfsetbuttcap%
\pgfsetroundjoin%
\definecolor{currentfill}{rgb}{1.000000,0.894118,0.788235}%
\pgfsetfillcolor{currentfill}%
\pgfsetlinewidth{0.000000pt}%
\definecolor{currentstroke}{rgb}{1.000000,0.894118,0.788235}%
\pgfsetstrokecolor{currentstroke}%
\pgfsetdash{}{0pt}%
\pgfpathmoveto{\pgfqpoint{3.593460in}{3.066768in}}%
\pgfpathlineto{\pgfqpoint{3.674419in}{3.020026in}}%
\pgfpathlineto{\pgfqpoint{3.674419in}{3.113510in}}%
\pgfpathlineto{\pgfqpoint{3.593460in}{3.160252in}}%
\pgfpathlineto{\pgfqpoint{3.593460in}{3.066768in}}%
\pgfpathclose%
\pgfusepath{fill}%
\end{pgfscope}%
\begin{pgfscope}%
\pgfpathrectangle{\pgfqpoint{0.765000in}{0.660000in}}{\pgfqpoint{4.620000in}{4.620000in}}%
\pgfusepath{clip}%
\pgfsetbuttcap%
\pgfsetroundjoin%
\definecolor{currentfill}{rgb}{1.000000,0.894118,0.788235}%
\pgfsetfillcolor{currentfill}%
\pgfsetlinewidth{0.000000pt}%
\definecolor{currentstroke}{rgb}{1.000000,0.894118,0.788235}%
\pgfsetstrokecolor{currentstroke}%
\pgfsetdash{}{0pt}%
\pgfpathmoveto{\pgfqpoint{3.593460in}{3.452500in}}%
\pgfpathlineto{\pgfqpoint{3.512500in}{3.405758in}}%
\pgfpathlineto{\pgfqpoint{3.593460in}{3.359016in}}%
\pgfpathlineto{\pgfqpoint{3.674419in}{3.405758in}}%
\pgfpathlineto{\pgfqpoint{3.593460in}{3.452500in}}%
\pgfpathclose%
\pgfusepath{fill}%
\end{pgfscope}%
\begin{pgfscope}%
\pgfpathrectangle{\pgfqpoint{0.765000in}{0.660000in}}{\pgfqpoint{4.620000in}{4.620000in}}%
\pgfusepath{clip}%
\pgfsetbuttcap%
\pgfsetroundjoin%
\definecolor{currentfill}{rgb}{1.000000,0.894118,0.788235}%
\pgfsetfillcolor{currentfill}%
\pgfsetlinewidth{0.000000pt}%
\definecolor{currentstroke}{rgb}{1.000000,0.894118,0.788235}%
\pgfsetstrokecolor{currentstroke}%
\pgfsetdash{}{0pt}%
\pgfpathmoveto{\pgfqpoint{3.593460in}{3.452500in}}%
\pgfpathlineto{\pgfqpoint{3.512500in}{3.405758in}}%
\pgfpathlineto{\pgfqpoint{3.512500in}{3.499242in}}%
\pgfpathlineto{\pgfqpoint{3.593460in}{3.545983in}}%
\pgfpathlineto{\pgfqpoint{3.593460in}{3.452500in}}%
\pgfpathclose%
\pgfusepath{fill}%
\end{pgfscope}%
\begin{pgfscope}%
\pgfpathrectangle{\pgfqpoint{0.765000in}{0.660000in}}{\pgfqpoint{4.620000in}{4.620000in}}%
\pgfusepath{clip}%
\pgfsetbuttcap%
\pgfsetroundjoin%
\definecolor{currentfill}{rgb}{1.000000,0.894118,0.788235}%
\pgfsetfillcolor{currentfill}%
\pgfsetlinewidth{0.000000pt}%
\definecolor{currentstroke}{rgb}{1.000000,0.894118,0.788235}%
\pgfsetstrokecolor{currentstroke}%
\pgfsetdash{}{0pt}%
\pgfpathmoveto{\pgfqpoint{3.593460in}{3.452500in}}%
\pgfpathlineto{\pgfqpoint{3.674419in}{3.405758in}}%
\pgfpathlineto{\pgfqpoint{3.674419in}{3.499242in}}%
\pgfpathlineto{\pgfqpoint{3.593460in}{3.545983in}}%
\pgfpathlineto{\pgfqpoint{3.593460in}{3.452500in}}%
\pgfpathclose%
\pgfusepath{fill}%
\end{pgfscope}%
\begin{pgfscope}%
\pgfpathrectangle{\pgfqpoint{0.765000in}{0.660000in}}{\pgfqpoint{4.620000in}{4.620000in}}%
\pgfusepath{clip}%
\pgfsetbuttcap%
\pgfsetroundjoin%
\definecolor{currentfill}{rgb}{1.000000,0.894118,0.788235}%
\pgfsetfillcolor{currentfill}%
\pgfsetlinewidth{0.000000pt}%
\definecolor{currentstroke}{rgb}{1.000000,0.894118,0.788235}%
\pgfsetstrokecolor{currentstroke}%
\pgfsetdash{}{0pt}%
\pgfpathmoveto{\pgfqpoint{3.593460in}{2.973284in}}%
\pgfpathlineto{\pgfqpoint{3.674419in}{3.020026in}}%
\pgfpathlineto{\pgfqpoint{3.674419in}{3.113510in}}%
\pgfpathlineto{\pgfqpoint{3.593460in}{3.066768in}}%
\pgfpathlineto{\pgfqpoint{3.593460in}{2.973284in}}%
\pgfpathclose%
\pgfusepath{fill}%
\end{pgfscope}%
\begin{pgfscope}%
\pgfpathrectangle{\pgfqpoint{0.765000in}{0.660000in}}{\pgfqpoint{4.620000in}{4.620000in}}%
\pgfusepath{clip}%
\pgfsetbuttcap%
\pgfsetroundjoin%
\definecolor{currentfill}{rgb}{1.000000,0.894118,0.788235}%
\pgfsetfillcolor{currentfill}%
\pgfsetlinewidth{0.000000pt}%
\definecolor{currentstroke}{rgb}{1.000000,0.894118,0.788235}%
\pgfsetstrokecolor{currentstroke}%
\pgfsetdash{}{0pt}%
\pgfpathmoveto{\pgfqpoint{3.593460in}{3.359016in}}%
\pgfpathlineto{\pgfqpoint{3.674419in}{3.405758in}}%
\pgfpathlineto{\pgfqpoint{3.674419in}{3.499242in}}%
\pgfpathlineto{\pgfqpoint{3.593460in}{3.452500in}}%
\pgfpathlineto{\pgfqpoint{3.593460in}{3.359016in}}%
\pgfpathclose%
\pgfusepath{fill}%
\end{pgfscope}%
\begin{pgfscope}%
\pgfpathrectangle{\pgfqpoint{0.765000in}{0.660000in}}{\pgfqpoint{4.620000in}{4.620000in}}%
\pgfusepath{clip}%
\pgfsetbuttcap%
\pgfsetroundjoin%
\definecolor{currentfill}{rgb}{1.000000,0.894118,0.788235}%
\pgfsetfillcolor{currentfill}%
\pgfsetlinewidth{0.000000pt}%
\definecolor{currentstroke}{rgb}{1.000000,0.894118,0.788235}%
\pgfsetstrokecolor{currentstroke}%
\pgfsetdash{}{0pt}%
\pgfpathmoveto{\pgfqpoint{3.593460in}{3.160252in}}%
\pgfpathlineto{\pgfqpoint{3.512500in}{3.113510in}}%
\pgfpathlineto{\pgfqpoint{3.593460in}{3.066768in}}%
\pgfpathlineto{\pgfqpoint{3.674419in}{3.113510in}}%
\pgfpathlineto{\pgfqpoint{3.593460in}{3.160252in}}%
\pgfpathclose%
\pgfusepath{fill}%
\end{pgfscope}%
\begin{pgfscope}%
\pgfpathrectangle{\pgfqpoint{0.765000in}{0.660000in}}{\pgfqpoint{4.620000in}{4.620000in}}%
\pgfusepath{clip}%
\pgfsetbuttcap%
\pgfsetroundjoin%
\definecolor{currentfill}{rgb}{1.000000,0.894118,0.788235}%
\pgfsetfillcolor{currentfill}%
\pgfsetlinewidth{0.000000pt}%
\definecolor{currentstroke}{rgb}{1.000000,0.894118,0.788235}%
\pgfsetstrokecolor{currentstroke}%
\pgfsetdash{}{0pt}%
\pgfpathmoveto{\pgfqpoint{3.512500in}{3.020026in}}%
\pgfpathlineto{\pgfqpoint{3.593460in}{2.973284in}}%
\pgfpathlineto{\pgfqpoint{3.593460in}{3.066768in}}%
\pgfpathlineto{\pgfqpoint{3.512500in}{3.113510in}}%
\pgfpathlineto{\pgfqpoint{3.512500in}{3.020026in}}%
\pgfpathclose%
\pgfusepath{fill}%
\end{pgfscope}%
\begin{pgfscope}%
\pgfpathrectangle{\pgfqpoint{0.765000in}{0.660000in}}{\pgfqpoint{4.620000in}{4.620000in}}%
\pgfusepath{clip}%
\pgfsetbuttcap%
\pgfsetroundjoin%
\definecolor{currentfill}{rgb}{1.000000,0.894118,0.788235}%
\pgfsetfillcolor{currentfill}%
\pgfsetlinewidth{0.000000pt}%
\definecolor{currentstroke}{rgb}{1.000000,0.894118,0.788235}%
\pgfsetstrokecolor{currentstroke}%
\pgfsetdash{}{0pt}%
\pgfpathmoveto{\pgfqpoint{3.593460in}{3.545983in}}%
\pgfpathlineto{\pgfqpoint{3.512500in}{3.499242in}}%
\pgfpathlineto{\pgfqpoint{3.593460in}{3.452500in}}%
\pgfpathlineto{\pgfqpoint{3.674419in}{3.499242in}}%
\pgfpathlineto{\pgfqpoint{3.593460in}{3.545983in}}%
\pgfpathclose%
\pgfusepath{fill}%
\end{pgfscope}%
\begin{pgfscope}%
\pgfpathrectangle{\pgfqpoint{0.765000in}{0.660000in}}{\pgfqpoint{4.620000in}{4.620000in}}%
\pgfusepath{clip}%
\pgfsetbuttcap%
\pgfsetroundjoin%
\definecolor{currentfill}{rgb}{1.000000,0.894118,0.788235}%
\pgfsetfillcolor{currentfill}%
\pgfsetlinewidth{0.000000pt}%
\definecolor{currentstroke}{rgb}{1.000000,0.894118,0.788235}%
\pgfsetstrokecolor{currentstroke}%
\pgfsetdash{}{0pt}%
\pgfpathmoveto{\pgfqpoint{3.512500in}{3.405758in}}%
\pgfpathlineto{\pgfqpoint{3.593460in}{3.359016in}}%
\pgfpathlineto{\pgfqpoint{3.593460in}{3.452500in}}%
\pgfpathlineto{\pgfqpoint{3.512500in}{3.499242in}}%
\pgfpathlineto{\pgfqpoint{3.512500in}{3.405758in}}%
\pgfpathclose%
\pgfusepath{fill}%
\end{pgfscope}%
\begin{pgfscope}%
\pgfpathrectangle{\pgfqpoint{0.765000in}{0.660000in}}{\pgfqpoint{4.620000in}{4.620000in}}%
\pgfusepath{clip}%
\pgfsetbuttcap%
\pgfsetroundjoin%
\definecolor{currentfill}{rgb}{1.000000,0.894118,0.788235}%
\pgfsetfillcolor{currentfill}%
\pgfsetlinewidth{0.000000pt}%
\definecolor{currentstroke}{rgb}{1.000000,0.894118,0.788235}%
\pgfsetstrokecolor{currentstroke}%
\pgfsetdash{}{0pt}%
\pgfpathmoveto{\pgfqpoint{3.593460in}{3.066768in}}%
\pgfpathlineto{\pgfqpoint{3.512500in}{3.020026in}}%
\pgfpathlineto{\pgfqpoint{3.512500in}{3.405758in}}%
\pgfpathlineto{\pgfqpoint{3.593460in}{3.452500in}}%
\pgfpathlineto{\pgfqpoint{3.593460in}{3.066768in}}%
\pgfpathclose%
\pgfusepath{fill}%
\end{pgfscope}%
\begin{pgfscope}%
\pgfpathrectangle{\pgfqpoint{0.765000in}{0.660000in}}{\pgfqpoint{4.620000in}{4.620000in}}%
\pgfusepath{clip}%
\pgfsetbuttcap%
\pgfsetroundjoin%
\definecolor{currentfill}{rgb}{1.000000,0.894118,0.788235}%
\pgfsetfillcolor{currentfill}%
\pgfsetlinewidth{0.000000pt}%
\definecolor{currentstroke}{rgb}{1.000000,0.894118,0.788235}%
\pgfsetstrokecolor{currentstroke}%
\pgfsetdash{}{0pt}%
\pgfpathmoveto{\pgfqpoint{3.674419in}{3.020026in}}%
\pgfpathlineto{\pgfqpoint{3.593460in}{3.066768in}}%
\pgfpathlineto{\pgfqpoint{3.593460in}{3.452500in}}%
\pgfpathlineto{\pgfqpoint{3.674419in}{3.405758in}}%
\pgfpathlineto{\pgfqpoint{3.674419in}{3.020026in}}%
\pgfpathclose%
\pgfusepath{fill}%
\end{pgfscope}%
\begin{pgfscope}%
\pgfpathrectangle{\pgfqpoint{0.765000in}{0.660000in}}{\pgfqpoint{4.620000in}{4.620000in}}%
\pgfusepath{clip}%
\pgfsetbuttcap%
\pgfsetroundjoin%
\definecolor{currentfill}{rgb}{1.000000,0.894118,0.788235}%
\pgfsetfillcolor{currentfill}%
\pgfsetlinewidth{0.000000pt}%
\definecolor{currentstroke}{rgb}{1.000000,0.894118,0.788235}%
\pgfsetstrokecolor{currentstroke}%
\pgfsetdash{}{0pt}%
\pgfpathmoveto{\pgfqpoint{3.593460in}{3.066768in}}%
\pgfpathlineto{\pgfqpoint{3.593460in}{3.160252in}}%
\pgfpathlineto{\pgfqpoint{3.593460in}{3.545983in}}%
\pgfpathlineto{\pgfqpoint{3.674419in}{3.405758in}}%
\pgfpathlineto{\pgfqpoint{3.593460in}{3.066768in}}%
\pgfpathclose%
\pgfusepath{fill}%
\end{pgfscope}%
\begin{pgfscope}%
\pgfpathrectangle{\pgfqpoint{0.765000in}{0.660000in}}{\pgfqpoint{4.620000in}{4.620000in}}%
\pgfusepath{clip}%
\pgfsetbuttcap%
\pgfsetroundjoin%
\definecolor{currentfill}{rgb}{1.000000,0.894118,0.788235}%
\pgfsetfillcolor{currentfill}%
\pgfsetlinewidth{0.000000pt}%
\definecolor{currentstroke}{rgb}{1.000000,0.894118,0.788235}%
\pgfsetstrokecolor{currentstroke}%
\pgfsetdash{}{0pt}%
\pgfpathmoveto{\pgfqpoint{3.674419in}{3.020026in}}%
\pgfpathlineto{\pgfqpoint{3.674419in}{3.113510in}}%
\pgfpathlineto{\pgfqpoint{3.674419in}{3.499242in}}%
\pgfpathlineto{\pgfqpoint{3.593460in}{3.452500in}}%
\pgfpathlineto{\pgfqpoint{3.674419in}{3.020026in}}%
\pgfpathclose%
\pgfusepath{fill}%
\end{pgfscope}%
\begin{pgfscope}%
\pgfpathrectangle{\pgfqpoint{0.765000in}{0.660000in}}{\pgfqpoint{4.620000in}{4.620000in}}%
\pgfusepath{clip}%
\pgfsetbuttcap%
\pgfsetroundjoin%
\definecolor{currentfill}{rgb}{1.000000,0.894118,0.788235}%
\pgfsetfillcolor{currentfill}%
\pgfsetlinewidth{0.000000pt}%
\definecolor{currentstroke}{rgb}{1.000000,0.894118,0.788235}%
\pgfsetstrokecolor{currentstroke}%
\pgfsetdash{}{0pt}%
\pgfpathmoveto{\pgfqpoint{3.512500in}{3.020026in}}%
\pgfpathlineto{\pgfqpoint{3.593460in}{2.973284in}}%
\pgfpathlineto{\pgfqpoint{3.593460in}{3.359016in}}%
\pgfpathlineto{\pgfqpoint{3.512500in}{3.405758in}}%
\pgfpathlineto{\pgfqpoint{3.512500in}{3.020026in}}%
\pgfpathclose%
\pgfusepath{fill}%
\end{pgfscope}%
\begin{pgfscope}%
\pgfpathrectangle{\pgfqpoint{0.765000in}{0.660000in}}{\pgfqpoint{4.620000in}{4.620000in}}%
\pgfusepath{clip}%
\pgfsetbuttcap%
\pgfsetroundjoin%
\definecolor{currentfill}{rgb}{1.000000,0.894118,0.788235}%
\pgfsetfillcolor{currentfill}%
\pgfsetlinewidth{0.000000pt}%
\definecolor{currentstroke}{rgb}{1.000000,0.894118,0.788235}%
\pgfsetstrokecolor{currentstroke}%
\pgfsetdash{}{0pt}%
\pgfpathmoveto{\pgfqpoint{3.593460in}{2.973284in}}%
\pgfpathlineto{\pgfqpoint{3.674419in}{3.020026in}}%
\pgfpathlineto{\pgfqpoint{3.674419in}{3.405758in}}%
\pgfpathlineto{\pgfqpoint{3.593460in}{3.359016in}}%
\pgfpathlineto{\pgfqpoint{3.593460in}{2.973284in}}%
\pgfpathclose%
\pgfusepath{fill}%
\end{pgfscope}%
\begin{pgfscope}%
\pgfpathrectangle{\pgfqpoint{0.765000in}{0.660000in}}{\pgfqpoint{4.620000in}{4.620000in}}%
\pgfusepath{clip}%
\pgfsetbuttcap%
\pgfsetroundjoin%
\definecolor{currentfill}{rgb}{1.000000,0.894118,0.788235}%
\pgfsetfillcolor{currentfill}%
\pgfsetlinewidth{0.000000pt}%
\definecolor{currentstroke}{rgb}{1.000000,0.894118,0.788235}%
\pgfsetstrokecolor{currentstroke}%
\pgfsetdash{}{0pt}%
\pgfpathmoveto{\pgfqpoint{3.593460in}{3.160252in}}%
\pgfpathlineto{\pgfqpoint{3.512500in}{3.113510in}}%
\pgfpathlineto{\pgfqpoint{3.512500in}{3.499242in}}%
\pgfpathlineto{\pgfqpoint{3.593460in}{3.545983in}}%
\pgfpathlineto{\pgfqpoint{3.593460in}{3.160252in}}%
\pgfpathclose%
\pgfusepath{fill}%
\end{pgfscope}%
\begin{pgfscope}%
\pgfpathrectangle{\pgfqpoint{0.765000in}{0.660000in}}{\pgfqpoint{4.620000in}{4.620000in}}%
\pgfusepath{clip}%
\pgfsetbuttcap%
\pgfsetroundjoin%
\definecolor{currentfill}{rgb}{1.000000,0.894118,0.788235}%
\pgfsetfillcolor{currentfill}%
\pgfsetlinewidth{0.000000pt}%
\definecolor{currentstroke}{rgb}{1.000000,0.894118,0.788235}%
\pgfsetstrokecolor{currentstroke}%
\pgfsetdash{}{0pt}%
\pgfpathmoveto{\pgfqpoint{3.674419in}{3.113510in}}%
\pgfpathlineto{\pgfqpoint{3.593460in}{3.160252in}}%
\pgfpathlineto{\pgfqpoint{3.593460in}{3.545983in}}%
\pgfpathlineto{\pgfqpoint{3.674419in}{3.499242in}}%
\pgfpathlineto{\pgfqpoint{3.674419in}{3.113510in}}%
\pgfpathclose%
\pgfusepath{fill}%
\end{pgfscope}%
\begin{pgfscope}%
\pgfpathrectangle{\pgfqpoint{0.765000in}{0.660000in}}{\pgfqpoint{4.620000in}{4.620000in}}%
\pgfusepath{clip}%
\pgfsetbuttcap%
\pgfsetroundjoin%
\definecolor{currentfill}{rgb}{1.000000,0.894118,0.788235}%
\pgfsetfillcolor{currentfill}%
\pgfsetlinewidth{0.000000pt}%
\definecolor{currentstroke}{rgb}{1.000000,0.894118,0.788235}%
\pgfsetstrokecolor{currentstroke}%
\pgfsetdash{}{0pt}%
\pgfpathmoveto{\pgfqpoint{3.512500in}{3.020026in}}%
\pgfpathlineto{\pgfqpoint{3.512500in}{3.113510in}}%
\pgfpathlineto{\pgfqpoint{3.512500in}{3.499242in}}%
\pgfpathlineto{\pgfqpoint{3.593460in}{3.359016in}}%
\pgfpathlineto{\pgfqpoint{3.512500in}{3.020026in}}%
\pgfpathclose%
\pgfusepath{fill}%
\end{pgfscope}%
\begin{pgfscope}%
\pgfpathrectangle{\pgfqpoint{0.765000in}{0.660000in}}{\pgfqpoint{4.620000in}{4.620000in}}%
\pgfusepath{clip}%
\pgfsetbuttcap%
\pgfsetroundjoin%
\definecolor{currentfill}{rgb}{1.000000,0.894118,0.788235}%
\pgfsetfillcolor{currentfill}%
\pgfsetlinewidth{0.000000pt}%
\definecolor{currentstroke}{rgb}{1.000000,0.894118,0.788235}%
\pgfsetstrokecolor{currentstroke}%
\pgfsetdash{}{0pt}%
\pgfpathmoveto{\pgfqpoint{3.593460in}{2.973284in}}%
\pgfpathlineto{\pgfqpoint{3.593460in}{3.066768in}}%
\pgfpathlineto{\pgfqpoint{3.593460in}{3.452500in}}%
\pgfpathlineto{\pgfqpoint{3.512500in}{3.405758in}}%
\pgfpathlineto{\pgfqpoint{3.593460in}{2.973284in}}%
\pgfpathclose%
\pgfusepath{fill}%
\end{pgfscope}%
\begin{pgfscope}%
\pgfpathrectangle{\pgfqpoint{0.765000in}{0.660000in}}{\pgfqpoint{4.620000in}{4.620000in}}%
\pgfusepath{clip}%
\pgfsetbuttcap%
\pgfsetroundjoin%
\definecolor{currentfill}{rgb}{1.000000,0.894118,0.788235}%
\pgfsetfillcolor{currentfill}%
\pgfsetlinewidth{0.000000pt}%
\definecolor{currentstroke}{rgb}{1.000000,0.894118,0.788235}%
\pgfsetstrokecolor{currentstroke}%
\pgfsetdash{}{0pt}%
\pgfpathmoveto{\pgfqpoint{3.512500in}{3.113510in}}%
\pgfpathlineto{\pgfqpoint{3.593460in}{3.066768in}}%
\pgfpathlineto{\pgfqpoint{3.593460in}{3.452500in}}%
\pgfpathlineto{\pgfqpoint{3.512500in}{3.499242in}}%
\pgfpathlineto{\pgfqpoint{3.512500in}{3.113510in}}%
\pgfpathclose%
\pgfusepath{fill}%
\end{pgfscope}%
\begin{pgfscope}%
\pgfpathrectangle{\pgfqpoint{0.765000in}{0.660000in}}{\pgfqpoint{4.620000in}{4.620000in}}%
\pgfusepath{clip}%
\pgfsetbuttcap%
\pgfsetroundjoin%
\definecolor{currentfill}{rgb}{1.000000,0.894118,0.788235}%
\pgfsetfillcolor{currentfill}%
\pgfsetlinewidth{0.000000pt}%
\definecolor{currentstroke}{rgb}{1.000000,0.894118,0.788235}%
\pgfsetstrokecolor{currentstroke}%
\pgfsetdash{}{0pt}%
\pgfpathmoveto{\pgfqpoint{3.593460in}{3.066768in}}%
\pgfpathlineto{\pgfqpoint{3.674419in}{3.113510in}}%
\pgfpathlineto{\pgfqpoint{3.674419in}{3.499242in}}%
\pgfpathlineto{\pgfqpoint{3.593460in}{3.452500in}}%
\pgfpathlineto{\pgfqpoint{3.593460in}{3.066768in}}%
\pgfpathclose%
\pgfusepath{fill}%
\end{pgfscope}%
\begin{pgfscope}%
\pgfpathrectangle{\pgfqpoint{0.765000in}{0.660000in}}{\pgfqpoint{4.620000in}{4.620000in}}%
\pgfusepath{clip}%
\pgfsetbuttcap%
\pgfsetroundjoin%
\definecolor{currentfill}{rgb}{1.000000,0.894118,0.788235}%
\pgfsetfillcolor{currentfill}%
\pgfsetlinewidth{0.000000pt}%
\definecolor{currentstroke}{rgb}{1.000000,0.894118,0.788235}%
\pgfsetstrokecolor{currentstroke}%
\pgfsetdash{}{0pt}%
\pgfpathmoveto{\pgfqpoint{3.593460in}{3.066768in}}%
\pgfpathlineto{\pgfqpoint{3.512500in}{3.020026in}}%
\pgfpathlineto{\pgfqpoint{3.593460in}{2.973284in}}%
\pgfpathlineto{\pgfqpoint{3.674419in}{3.020026in}}%
\pgfpathlineto{\pgfqpoint{3.593460in}{3.066768in}}%
\pgfpathclose%
\pgfusepath{fill}%
\end{pgfscope}%
\begin{pgfscope}%
\pgfpathrectangle{\pgfqpoint{0.765000in}{0.660000in}}{\pgfqpoint{4.620000in}{4.620000in}}%
\pgfusepath{clip}%
\pgfsetbuttcap%
\pgfsetroundjoin%
\definecolor{currentfill}{rgb}{1.000000,0.894118,0.788235}%
\pgfsetfillcolor{currentfill}%
\pgfsetlinewidth{0.000000pt}%
\definecolor{currentstroke}{rgb}{1.000000,0.894118,0.788235}%
\pgfsetstrokecolor{currentstroke}%
\pgfsetdash{}{0pt}%
\pgfpathmoveto{\pgfqpoint{3.593460in}{3.066768in}}%
\pgfpathlineto{\pgfqpoint{3.512500in}{3.020026in}}%
\pgfpathlineto{\pgfqpoint{3.512500in}{3.113510in}}%
\pgfpathlineto{\pgfqpoint{3.593460in}{3.160252in}}%
\pgfpathlineto{\pgfqpoint{3.593460in}{3.066768in}}%
\pgfpathclose%
\pgfusepath{fill}%
\end{pgfscope}%
\begin{pgfscope}%
\pgfpathrectangle{\pgfqpoint{0.765000in}{0.660000in}}{\pgfqpoint{4.620000in}{4.620000in}}%
\pgfusepath{clip}%
\pgfsetbuttcap%
\pgfsetroundjoin%
\definecolor{currentfill}{rgb}{1.000000,0.894118,0.788235}%
\pgfsetfillcolor{currentfill}%
\pgfsetlinewidth{0.000000pt}%
\definecolor{currentstroke}{rgb}{1.000000,0.894118,0.788235}%
\pgfsetstrokecolor{currentstroke}%
\pgfsetdash{}{0pt}%
\pgfpathmoveto{\pgfqpoint{3.593460in}{3.066768in}}%
\pgfpathlineto{\pgfqpoint{3.674419in}{3.020026in}}%
\pgfpathlineto{\pgfqpoint{3.674419in}{3.113510in}}%
\pgfpathlineto{\pgfqpoint{3.593460in}{3.160252in}}%
\pgfpathlineto{\pgfqpoint{3.593460in}{3.066768in}}%
\pgfpathclose%
\pgfusepath{fill}%
\end{pgfscope}%
\begin{pgfscope}%
\pgfpathrectangle{\pgfqpoint{0.765000in}{0.660000in}}{\pgfqpoint{4.620000in}{4.620000in}}%
\pgfusepath{clip}%
\pgfsetbuttcap%
\pgfsetroundjoin%
\definecolor{currentfill}{rgb}{1.000000,0.894118,0.788235}%
\pgfsetfillcolor{currentfill}%
\pgfsetlinewidth{0.000000pt}%
\definecolor{currentstroke}{rgb}{1.000000,0.894118,0.788235}%
\pgfsetstrokecolor{currentstroke}%
\pgfsetdash{}{0pt}%
\pgfpathmoveto{\pgfqpoint{3.337483in}{2.623404in}}%
\pgfpathlineto{\pgfqpoint{3.256524in}{2.576662in}}%
\pgfpathlineto{\pgfqpoint{3.337483in}{2.529920in}}%
\pgfpathlineto{\pgfqpoint{3.418443in}{2.576662in}}%
\pgfpathlineto{\pgfqpoint{3.337483in}{2.623404in}}%
\pgfpathclose%
\pgfusepath{fill}%
\end{pgfscope}%
\begin{pgfscope}%
\pgfpathrectangle{\pgfqpoint{0.765000in}{0.660000in}}{\pgfqpoint{4.620000in}{4.620000in}}%
\pgfusepath{clip}%
\pgfsetbuttcap%
\pgfsetroundjoin%
\definecolor{currentfill}{rgb}{1.000000,0.894118,0.788235}%
\pgfsetfillcolor{currentfill}%
\pgfsetlinewidth{0.000000pt}%
\definecolor{currentstroke}{rgb}{1.000000,0.894118,0.788235}%
\pgfsetstrokecolor{currentstroke}%
\pgfsetdash{}{0pt}%
\pgfpathmoveto{\pgfqpoint{3.337483in}{2.623404in}}%
\pgfpathlineto{\pgfqpoint{3.256524in}{2.576662in}}%
\pgfpathlineto{\pgfqpoint{3.256524in}{2.670145in}}%
\pgfpathlineto{\pgfqpoint{3.337483in}{2.716887in}}%
\pgfpathlineto{\pgfqpoint{3.337483in}{2.623404in}}%
\pgfpathclose%
\pgfusepath{fill}%
\end{pgfscope}%
\begin{pgfscope}%
\pgfpathrectangle{\pgfqpoint{0.765000in}{0.660000in}}{\pgfqpoint{4.620000in}{4.620000in}}%
\pgfusepath{clip}%
\pgfsetbuttcap%
\pgfsetroundjoin%
\definecolor{currentfill}{rgb}{1.000000,0.894118,0.788235}%
\pgfsetfillcolor{currentfill}%
\pgfsetlinewidth{0.000000pt}%
\definecolor{currentstroke}{rgb}{1.000000,0.894118,0.788235}%
\pgfsetstrokecolor{currentstroke}%
\pgfsetdash{}{0pt}%
\pgfpathmoveto{\pgfqpoint{3.337483in}{2.623404in}}%
\pgfpathlineto{\pgfqpoint{3.418443in}{2.576662in}}%
\pgfpathlineto{\pgfqpoint{3.418443in}{2.670145in}}%
\pgfpathlineto{\pgfqpoint{3.337483in}{2.716887in}}%
\pgfpathlineto{\pgfqpoint{3.337483in}{2.623404in}}%
\pgfpathclose%
\pgfusepath{fill}%
\end{pgfscope}%
\begin{pgfscope}%
\pgfpathrectangle{\pgfqpoint{0.765000in}{0.660000in}}{\pgfqpoint{4.620000in}{4.620000in}}%
\pgfusepath{clip}%
\pgfsetbuttcap%
\pgfsetroundjoin%
\definecolor{currentfill}{rgb}{1.000000,0.894118,0.788235}%
\pgfsetfillcolor{currentfill}%
\pgfsetlinewidth{0.000000pt}%
\definecolor{currentstroke}{rgb}{1.000000,0.894118,0.788235}%
\pgfsetstrokecolor{currentstroke}%
\pgfsetdash{}{0pt}%
\pgfpathmoveto{\pgfqpoint{3.593460in}{2.973284in}}%
\pgfpathlineto{\pgfqpoint{3.674419in}{3.020026in}}%
\pgfpathlineto{\pgfqpoint{3.674419in}{3.113510in}}%
\pgfpathlineto{\pgfqpoint{3.593460in}{3.066768in}}%
\pgfpathlineto{\pgfqpoint{3.593460in}{2.973284in}}%
\pgfpathclose%
\pgfusepath{fill}%
\end{pgfscope}%
\begin{pgfscope}%
\pgfpathrectangle{\pgfqpoint{0.765000in}{0.660000in}}{\pgfqpoint{4.620000in}{4.620000in}}%
\pgfusepath{clip}%
\pgfsetbuttcap%
\pgfsetroundjoin%
\definecolor{currentfill}{rgb}{1.000000,0.894118,0.788235}%
\pgfsetfillcolor{currentfill}%
\pgfsetlinewidth{0.000000pt}%
\definecolor{currentstroke}{rgb}{1.000000,0.894118,0.788235}%
\pgfsetstrokecolor{currentstroke}%
\pgfsetdash{}{0pt}%
\pgfpathmoveto{\pgfqpoint{3.337483in}{2.529920in}}%
\pgfpathlineto{\pgfqpoint{3.418443in}{2.576662in}}%
\pgfpathlineto{\pgfqpoint{3.418443in}{2.670145in}}%
\pgfpathlineto{\pgfqpoint{3.337483in}{2.623404in}}%
\pgfpathlineto{\pgfqpoint{3.337483in}{2.529920in}}%
\pgfpathclose%
\pgfusepath{fill}%
\end{pgfscope}%
\begin{pgfscope}%
\pgfpathrectangle{\pgfqpoint{0.765000in}{0.660000in}}{\pgfqpoint{4.620000in}{4.620000in}}%
\pgfusepath{clip}%
\pgfsetbuttcap%
\pgfsetroundjoin%
\definecolor{currentfill}{rgb}{1.000000,0.894118,0.788235}%
\pgfsetfillcolor{currentfill}%
\pgfsetlinewidth{0.000000pt}%
\definecolor{currentstroke}{rgb}{1.000000,0.894118,0.788235}%
\pgfsetstrokecolor{currentstroke}%
\pgfsetdash{}{0pt}%
\pgfpathmoveto{\pgfqpoint{3.593460in}{3.160252in}}%
\pgfpathlineto{\pgfqpoint{3.512500in}{3.113510in}}%
\pgfpathlineto{\pgfqpoint{3.593460in}{3.066768in}}%
\pgfpathlineto{\pgfqpoint{3.674419in}{3.113510in}}%
\pgfpathlineto{\pgfqpoint{3.593460in}{3.160252in}}%
\pgfpathclose%
\pgfusepath{fill}%
\end{pgfscope}%
\begin{pgfscope}%
\pgfpathrectangle{\pgfqpoint{0.765000in}{0.660000in}}{\pgfqpoint{4.620000in}{4.620000in}}%
\pgfusepath{clip}%
\pgfsetbuttcap%
\pgfsetroundjoin%
\definecolor{currentfill}{rgb}{1.000000,0.894118,0.788235}%
\pgfsetfillcolor{currentfill}%
\pgfsetlinewidth{0.000000pt}%
\definecolor{currentstroke}{rgb}{1.000000,0.894118,0.788235}%
\pgfsetstrokecolor{currentstroke}%
\pgfsetdash{}{0pt}%
\pgfpathmoveto{\pgfqpoint{3.512500in}{3.020026in}}%
\pgfpathlineto{\pgfqpoint{3.593460in}{2.973284in}}%
\pgfpathlineto{\pgfqpoint{3.593460in}{3.066768in}}%
\pgfpathlineto{\pgfqpoint{3.512500in}{3.113510in}}%
\pgfpathlineto{\pgfqpoint{3.512500in}{3.020026in}}%
\pgfpathclose%
\pgfusepath{fill}%
\end{pgfscope}%
\begin{pgfscope}%
\pgfpathrectangle{\pgfqpoint{0.765000in}{0.660000in}}{\pgfqpoint{4.620000in}{4.620000in}}%
\pgfusepath{clip}%
\pgfsetbuttcap%
\pgfsetroundjoin%
\definecolor{currentfill}{rgb}{1.000000,0.894118,0.788235}%
\pgfsetfillcolor{currentfill}%
\pgfsetlinewidth{0.000000pt}%
\definecolor{currentstroke}{rgb}{1.000000,0.894118,0.788235}%
\pgfsetstrokecolor{currentstroke}%
\pgfsetdash{}{0pt}%
\pgfpathmoveto{\pgfqpoint{3.337483in}{2.716887in}}%
\pgfpathlineto{\pgfqpoint{3.256524in}{2.670145in}}%
\pgfpathlineto{\pgfqpoint{3.337483in}{2.623404in}}%
\pgfpathlineto{\pgfqpoint{3.418443in}{2.670145in}}%
\pgfpathlineto{\pgfqpoint{3.337483in}{2.716887in}}%
\pgfpathclose%
\pgfusepath{fill}%
\end{pgfscope}%
\begin{pgfscope}%
\pgfpathrectangle{\pgfqpoint{0.765000in}{0.660000in}}{\pgfqpoint{4.620000in}{4.620000in}}%
\pgfusepath{clip}%
\pgfsetbuttcap%
\pgfsetroundjoin%
\definecolor{currentfill}{rgb}{1.000000,0.894118,0.788235}%
\pgfsetfillcolor{currentfill}%
\pgfsetlinewidth{0.000000pt}%
\definecolor{currentstroke}{rgb}{1.000000,0.894118,0.788235}%
\pgfsetstrokecolor{currentstroke}%
\pgfsetdash{}{0pt}%
\pgfpathmoveto{\pgfqpoint{3.256524in}{2.576662in}}%
\pgfpathlineto{\pgfqpoint{3.337483in}{2.529920in}}%
\pgfpathlineto{\pgfqpoint{3.337483in}{2.623404in}}%
\pgfpathlineto{\pgfqpoint{3.256524in}{2.670145in}}%
\pgfpathlineto{\pgfqpoint{3.256524in}{2.576662in}}%
\pgfpathclose%
\pgfusepath{fill}%
\end{pgfscope}%
\begin{pgfscope}%
\pgfpathrectangle{\pgfqpoint{0.765000in}{0.660000in}}{\pgfqpoint{4.620000in}{4.620000in}}%
\pgfusepath{clip}%
\pgfsetbuttcap%
\pgfsetroundjoin%
\definecolor{currentfill}{rgb}{1.000000,0.894118,0.788235}%
\pgfsetfillcolor{currentfill}%
\pgfsetlinewidth{0.000000pt}%
\definecolor{currentstroke}{rgb}{1.000000,0.894118,0.788235}%
\pgfsetstrokecolor{currentstroke}%
\pgfsetdash{}{0pt}%
\pgfpathmoveto{\pgfqpoint{3.593460in}{3.066768in}}%
\pgfpathlineto{\pgfqpoint{3.512500in}{3.020026in}}%
\pgfpathlineto{\pgfqpoint{3.256524in}{2.576662in}}%
\pgfpathlineto{\pgfqpoint{3.337483in}{2.623404in}}%
\pgfpathlineto{\pgfqpoint{3.593460in}{3.066768in}}%
\pgfpathclose%
\pgfusepath{fill}%
\end{pgfscope}%
\begin{pgfscope}%
\pgfpathrectangle{\pgfqpoint{0.765000in}{0.660000in}}{\pgfqpoint{4.620000in}{4.620000in}}%
\pgfusepath{clip}%
\pgfsetbuttcap%
\pgfsetroundjoin%
\definecolor{currentfill}{rgb}{1.000000,0.894118,0.788235}%
\pgfsetfillcolor{currentfill}%
\pgfsetlinewidth{0.000000pt}%
\definecolor{currentstroke}{rgb}{1.000000,0.894118,0.788235}%
\pgfsetstrokecolor{currentstroke}%
\pgfsetdash{}{0pt}%
\pgfpathmoveto{\pgfqpoint{3.674419in}{3.020026in}}%
\pgfpathlineto{\pgfqpoint{3.593460in}{3.066768in}}%
\pgfpathlineto{\pgfqpoint{3.337483in}{2.623404in}}%
\pgfpathlineto{\pgfqpoint{3.418443in}{2.576662in}}%
\pgfpathlineto{\pgfqpoint{3.674419in}{3.020026in}}%
\pgfpathclose%
\pgfusepath{fill}%
\end{pgfscope}%
\begin{pgfscope}%
\pgfpathrectangle{\pgfqpoint{0.765000in}{0.660000in}}{\pgfqpoint{4.620000in}{4.620000in}}%
\pgfusepath{clip}%
\pgfsetbuttcap%
\pgfsetroundjoin%
\definecolor{currentfill}{rgb}{1.000000,0.894118,0.788235}%
\pgfsetfillcolor{currentfill}%
\pgfsetlinewidth{0.000000pt}%
\definecolor{currentstroke}{rgb}{1.000000,0.894118,0.788235}%
\pgfsetstrokecolor{currentstroke}%
\pgfsetdash{}{0pt}%
\pgfpathmoveto{\pgfqpoint{3.593460in}{3.066768in}}%
\pgfpathlineto{\pgfqpoint{3.593460in}{3.160252in}}%
\pgfpathlineto{\pgfqpoint{3.337483in}{2.716887in}}%
\pgfpathlineto{\pgfqpoint{3.418443in}{2.576662in}}%
\pgfpathlineto{\pgfqpoint{3.593460in}{3.066768in}}%
\pgfpathclose%
\pgfusepath{fill}%
\end{pgfscope}%
\begin{pgfscope}%
\pgfpathrectangle{\pgfqpoint{0.765000in}{0.660000in}}{\pgfqpoint{4.620000in}{4.620000in}}%
\pgfusepath{clip}%
\pgfsetbuttcap%
\pgfsetroundjoin%
\definecolor{currentfill}{rgb}{1.000000,0.894118,0.788235}%
\pgfsetfillcolor{currentfill}%
\pgfsetlinewidth{0.000000pt}%
\definecolor{currentstroke}{rgb}{1.000000,0.894118,0.788235}%
\pgfsetstrokecolor{currentstroke}%
\pgfsetdash{}{0pt}%
\pgfpathmoveto{\pgfqpoint{3.674419in}{3.020026in}}%
\pgfpathlineto{\pgfqpoint{3.674419in}{3.113510in}}%
\pgfpathlineto{\pgfqpoint{3.418443in}{2.670145in}}%
\pgfpathlineto{\pgfqpoint{3.337483in}{2.623404in}}%
\pgfpathlineto{\pgfqpoint{3.674419in}{3.020026in}}%
\pgfpathclose%
\pgfusepath{fill}%
\end{pgfscope}%
\begin{pgfscope}%
\pgfpathrectangle{\pgfqpoint{0.765000in}{0.660000in}}{\pgfqpoint{4.620000in}{4.620000in}}%
\pgfusepath{clip}%
\pgfsetbuttcap%
\pgfsetroundjoin%
\definecolor{currentfill}{rgb}{1.000000,0.894118,0.788235}%
\pgfsetfillcolor{currentfill}%
\pgfsetlinewidth{0.000000pt}%
\definecolor{currentstroke}{rgb}{1.000000,0.894118,0.788235}%
\pgfsetstrokecolor{currentstroke}%
\pgfsetdash{}{0pt}%
\pgfpathmoveto{\pgfqpoint{3.512500in}{3.020026in}}%
\pgfpathlineto{\pgfqpoint{3.593460in}{2.973284in}}%
\pgfpathlineto{\pgfqpoint{3.337483in}{2.529920in}}%
\pgfpathlineto{\pgfqpoint{3.256524in}{2.576662in}}%
\pgfpathlineto{\pgfqpoint{3.512500in}{3.020026in}}%
\pgfpathclose%
\pgfusepath{fill}%
\end{pgfscope}%
\begin{pgfscope}%
\pgfpathrectangle{\pgfqpoint{0.765000in}{0.660000in}}{\pgfqpoint{4.620000in}{4.620000in}}%
\pgfusepath{clip}%
\pgfsetbuttcap%
\pgfsetroundjoin%
\definecolor{currentfill}{rgb}{1.000000,0.894118,0.788235}%
\pgfsetfillcolor{currentfill}%
\pgfsetlinewidth{0.000000pt}%
\definecolor{currentstroke}{rgb}{1.000000,0.894118,0.788235}%
\pgfsetstrokecolor{currentstroke}%
\pgfsetdash{}{0pt}%
\pgfpathmoveto{\pgfqpoint{3.593460in}{2.973284in}}%
\pgfpathlineto{\pgfqpoint{3.674419in}{3.020026in}}%
\pgfpathlineto{\pgfqpoint{3.418443in}{2.576662in}}%
\pgfpathlineto{\pgfqpoint{3.337483in}{2.529920in}}%
\pgfpathlineto{\pgfqpoint{3.593460in}{2.973284in}}%
\pgfpathclose%
\pgfusepath{fill}%
\end{pgfscope}%
\begin{pgfscope}%
\pgfpathrectangle{\pgfqpoint{0.765000in}{0.660000in}}{\pgfqpoint{4.620000in}{4.620000in}}%
\pgfusepath{clip}%
\pgfsetbuttcap%
\pgfsetroundjoin%
\definecolor{currentfill}{rgb}{1.000000,0.894118,0.788235}%
\pgfsetfillcolor{currentfill}%
\pgfsetlinewidth{0.000000pt}%
\definecolor{currentstroke}{rgb}{1.000000,0.894118,0.788235}%
\pgfsetstrokecolor{currentstroke}%
\pgfsetdash{}{0pt}%
\pgfpathmoveto{\pgfqpoint{3.593460in}{3.160252in}}%
\pgfpathlineto{\pgfqpoint{3.512500in}{3.113510in}}%
\pgfpathlineto{\pgfqpoint{3.256524in}{2.670145in}}%
\pgfpathlineto{\pgfqpoint{3.337483in}{2.716887in}}%
\pgfpathlineto{\pgfqpoint{3.593460in}{3.160252in}}%
\pgfpathclose%
\pgfusepath{fill}%
\end{pgfscope}%
\begin{pgfscope}%
\pgfpathrectangle{\pgfqpoint{0.765000in}{0.660000in}}{\pgfqpoint{4.620000in}{4.620000in}}%
\pgfusepath{clip}%
\pgfsetbuttcap%
\pgfsetroundjoin%
\definecolor{currentfill}{rgb}{1.000000,0.894118,0.788235}%
\pgfsetfillcolor{currentfill}%
\pgfsetlinewidth{0.000000pt}%
\definecolor{currentstroke}{rgb}{1.000000,0.894118,0.788235}%
\pgfsetstrokecolor{currentstroke}%
\pgfsetdash{}{0pt}%
\pgfpathmoveto{\pgfqpoint{3.674419in}{3.113510in}}%
\pgfpathlineto{\pgfqpoint{3.593460in}{3.160252in}}%
\pgfpathlineto{\pgfqpoint{3.337483in}{2.716887in}}%
\pgfpathlineto{\pgfqpoint{3.418443in}{2.670145in}}%
\pgfpathlineto{\pgfqpoint{3.674419in}{3.113510in}}%
\pgfpathclose%
\pgfusepath{fill}%
\end{pgfscope}%
\begin{pgfscope}%
\pgfpathrectangle{\pgfqpoint{0.765000in}{0.660000in}}{\pgfqpoint{4.620000in}{4.620000in}}%
\pgfusepath{clip}%
\pgfsetbuttcap%
\pgfsetroundjoin%
\definecolor{currentfill}{rgb}{1.000000,0.894118,0.788235}%
\pgfsetfillcolor{currentfill}%
\pgfsetlinewidth{0.000000pt}%
\definecolor{currentstroke}{rgb}{1.000000,0.894118,0.788235}%
\pgfsetstrokecolor{currentstroke}%
\pgfsetdash{}{0pt}%
\pgfpathmoveto{\pgfqpoint{3.512500in}{3.020026in}}%
\pgfpathlineto{\pgfqpoint{3.512500in}{3.113510in}}%
\pgfpathlineto{\pgfqpoint{3.256524in}{2.670145in}}%
\pgfpathlineto{\pgfqpoint{3.337483in}{2.529920in}}%
\pgfpathlineto{\pgfqpoint{3.512500in}{3.020026in}}%
\pgfpathclose%
\pgfusepath{fill}%
\end{pgfscope}%
\begin{pgfscope}%
\pgfpathrectangle{\pgfqpoint{0.765000in}{0.660000in}}{\pgfqpoint{4.620000in}{4.620000in}}%
\pgfusepath{clip}%
\pgfsetbuttcap%
\pgfsetroundjoin%
\definecolor{currentfill}{rgb}{1.000000,0.894118,0.788235}%
\pgfsetfillcolor{currentfill}%
\pgfsetlinewidth{0.000000pt}%
\definecolor{currentstroke}{rgb}{1.000000,0.894118,0.788235}%
\pgfsetstrokecolor{currentstroke}%
\pgfsetdash{}{0pt}%
\pgfpathmoveto{\pgfqpoint{3.593460in}{2.973284in}}%
\pgfpathlineto{\pgfqpoint{3.593460in}{3.066768in}}%
\pgfpathlineto{\pgfqpoint{3.337483in}{2.623404in}}%
\pgfpathlineto{\pgfqpoint{3.256524in}{2.576662in}}%
\pgfpathlineto{\pgfqpoint{3.593460in}{2.973284in}}%
\pgfpathclose%
\pgfusepath{fill}%
\end{pgfscope}%
\begin{pgfscope}%
\pgfpathrectangle{\pgfqpoint{0.765000in}{0.660000in}}{\pgfqpoint{4.620000in}{4.620000in}}%
\pgfusepath{clip}%
\pgfsetbuttcap%
\pgfsetroundjoin%
\definecolor{currentfill}{rgb}{1.000000,0.894118,0.788235}%
\pgfsetfillcolor{currentfill}%
\pgfsetlinewidth{0.000000pt}%
\definecolor{currentstroke}{rgb}{1.000000,0.894118,0.788235}%
\pgfsetstrokecolor{currentstroke}%
\pgfsetdash{}{0pt}%
\pgfpathmoveto{\pgfqpoint{3.512500in}{3.113510in}}%
\pgfpathlineto{\pgfqpoint{3.593460in}{3.066768in}}%
\pgfpathlineto{\pgfqpoint{3.337483in}{2.623404in}}%
\pgfpathlineto{\pgfqpoint{3.256524in}{2.670145in}}%
\pgfpathlineto{\pgfqpoint{3.512500in}{3.113510in}}%
\pgfpathclose%
\pgfusepath{fill}%
\end{pgfscope}%
\begin{pgfscope}%
\pgfpathrectangle{\pgfqpoint{0.765000in}{0.660000in}}{\pgfqpoint{4.620000in}{4.620000in}}%
\pgfusepath{clip}%
\pgfsetbuttcap%
\pgfsetroundjoin%
\definecolor{currentfill}{rgb}{1.000000,0.894118,0.788235}%
\pgfsetfillcolor{currentfill}%
\pgfsetlinewidth{0.000000pt}%
\definecolor{currentstroke}{rgb}{1.000000,0.894118,0.788235}%
\pgfsetstrokecolor{currentstroke}%
\pgfsetdash{}{0pt}%
\pgfpathmoveto{\pgfqpoint{3.593460in}{3.066768in}}%
\pgfpathlineto{\pgfqpoint{3.674419in}{3.113510in}}%
\pgfpathlineto{\pgfqpoint{3.418443in}{2.670145in}}%
\pgfpathlineto{\pgfqpoint{3.337483in}{2.623404in}}%
\pgfpathlineto{\pgfqpoint{3.593460in}{3.066768in}}%
\pgfpathclose%
\pgfusepath{fill}%
\end{pgfscope}%
\begin{pgfscope}%
\pgfpathrectangle{\pgfqpoint{0.765000in}{0.660000in}}{\pgfqpoint{4.620000in}{4.620000in}}%
\pgfusepath{clip}%
\pgfsetbuttcap%
\pgfsetroundjoin%
\definecolor{currentfill}{rgb}{1.000000,0.894118,0.788235}%
\pgfsetfillcolor{currentfill}%
\pgfsetlinewidth{0.000000pt}%
\definecolor{currentstroke}{rgb}{1.000000,0.894118,0.788235}%
\pgfsetstrokecolor{currentstroke}%
\pgfsetdash{}{0pt}%
\pgfpathmoveto{\pgfqpoint{3.593460in}{3.452500in}}%
\pgfpathlineto{\pgfqpoint{3.512500in}{3.405758in}}%
\pgfpathlineto{\pgfqpoint{3.593460in}{3.359016in}}%
\pgfpathlineto{\pgfqpoint{3.674419in}{3.405758in}}%
\pgfpathlineto{\pgfqpoint{3.593460in}{3.452500in}}%
\pgfpathclose%
\pgfusepath{fill}%
\end{pgfscope}%
\begin{pgfscope}%
\pgfpathrectangle{\pgfqpoint{0.765000in}{0.660000in}}{\pgfqpoint{4.620000in}{4.620000in}}%
\pgfusepath{clip}%
\pgfsetbuttcap%
\pgfsetroundjoin%
\definecolor{currentfill}{rgb}{1.000000,0.894118,0.788235}%
\pgfsetfillcolor{currentfill}%
\pgfsetlinewidth{0.000000pt}%
\definecolor{currentstroke}{rgb}{1.000000,0.894118,0.788235}%
\pgfsetstrokecolor{currentstroke}%
\pgfsetdash{}{0pt}%
\pgfpathmoveto{\pgfqpoint{3.593460in}{3.452500in}}%
\pgfpathlineto{\pgfqpoint{3.512500in}{3.405758in}}%
\pgfpathlineto{\pgfqpoint{3.512500in}{3.499242in}}%
\pgfpathlineto{\pgfqpoint{3.593460in}{3.545983in}}%
\pgfpathlineto{\pgfqpoint{3.593460in}{3.452500in}}%
\pgfpathclose%
\pgfusepath{fill}%
\end{pgfscope}%
\begin{pgfscope}%
\pgfpathrectangle{\pgfqpoint{0.765000in}{0.660000in}}{\pgfqpoint{4.620000in}{4.620000in}}%
\pgfusepath{clip}%
\pgfsetbuttcap%
\pgfsetroundjoin%
\definecolor{currentfill}{rgb}{1.000000,0.894118,0.788235}%
\pgfsetfillcolor{currentfill}%
\pgfsetlinewidth{0.000000pt}%
\definecolor{currentstroke}{rgb}{1.000000,0.894118,0.788235}%
\pgfsetstrokecolor{currentstroke}%
\pgfsetdash{}{0pt}%
\pgfpathmoveto{\pgfqpoint{3.593460in}{3.452500in}}%
\pgfpathlineto{\pgfqpoint{3.674419in}{3.405758in}}%
\pgfpathlineto{\pgfqpoint{3.674419in}{3.499242in}}%
\pgfpathlineto{\pgfqpoint{3.593460in}{3.545983in}}%
\pgfpathlineto{\pgfqpoint{3.593460in}{3.452500in}}%
\pgfpathclose%
\pgfusepath{fill}%
\end{pgfscope}%
\begin{pgfscope}%
\pgfpathrectangle{\pgfqpoint{0.765000in}{0.660000in}}{\pgfqpoint{4.620000in}{4.620000in}}%
\pgfusepath{clip}%
\pgfsetbuttcap%
\pgfsetroundjoin%
\definecolor{currentfill}{rgb}{1.000000,0.894118,0.788235}%
\pgfsetfillcolor{currentfill}%
\pgfsetlinewidth{0.000000pt}%
\definecolor{currentstroke}{rgb}{1.000000,0.894118,0.788235}%
\pgfsetstrokecolor{currentstroke}%
\pgfsetdash{}{0pt}%
\pgfpathmoveto{\pgfqpoint{3.593460in}{3.066768in}}%
\pgfpathlineto{\pgfqpoint{3.512500in}{3.020026in}}%
\pgfpathlineto{\pgfqpoint{3.593460in}{2.973284in}}%
\pgfpathlineto{\pgfqpoint{3.674419in}{3.020026in}}%
\pgfpathlineto{\pgfqpoint{3.593460in}{3.066768in}}%
\pgfpathclose%
\pgfusepath{fill}%
\end{pgfscope}%
\begin{pgfscope}%
\pgfpathrectangle{\pgfqpoint{0.765000in}{0.660000in}}{\pgfqpoint{4.620000in}{4.620000in}}%
\pgfusepath{clip}%
\pgfsetbuttcap%
\pgfsetroundjoin%
\definecolor{currentfill}{rgb}{1.000000,0.894118,0.788235}%
\pgfsetfillcolor{currentfill}%
\pgfsetlinewidth{0.000000pt}%
\definecolor{currentstroke}{rgb}{1.000000,0.894118,0.788235}%
\pgfsetstrokecolor{currentstroke}%
\pgfsetdash{}{0pt}%
\pgfpathmoveto{\pgfqpoint{3.593460in}{3.066768in}}%
\pgfpathlineto{\pgfqpoint{3.512500in}{3.020026in}}%
\pgfpathlineto{\pgfqpoint{3.512500in}{3.113510in}}%
\pgfpathlineto{\pgfqpoint{3.593460in}{3.160252in}}%
\pgfpathlineto{\pgfqpoint{3.593460in}{3.066768in}}%
\pgfpathclose%
\pgfusepath{fill}%
\end{pgfscope}%
\begin{pgfscope}%
\pgfpathrectangle{\pgfqpoint{0.765000in}{0.660000in}}{\pgfqpoint{4.620000in}{4.620000in}}%
\pgfusepath{clip}%
\pgfsetbuttcap%
\pgfsetroundjoin%
\definecolor{currentfill}{rgb}{1.000000,0.894118,0.788235}%
\pgfsetfillcolor{currentfill}%
\pgfsetlinewidth{0.000000pt}%
\definecolor{currentstroke}{rgb}{1.000000,0.894118,0.788235}%
\pgfsetstrokecolor{currentstroke}%
\pgfsetdash{}{0pt}%
\pgfpathmoveto{\pgfqpoint{3.593460in}{3.066768in}}%
\pgfpathlineto{\pgfqpoint{3.674419in}{3.020026in}}%
\pgfpathlineto{\pgfqpoint{3.674419in}{3.113510in}}%
\pgfpathlineto{\pgfqpoint{3.593460in}{3.160252in}}%
\pgfpathlineto{\pgfqpoint{3.593460in}{3.066768in}}%
\pgfpathclose%
\pgfusepath{fill}%
\end{pgfscope}%
\begin{pgfscope}%
\pgfpathrectangle{\pgfqpoint{0.765000in}{0.660000in}}{\pgfqpoint{4.620000in}{4.620000in}}%
\pgfusepath{clip}%
\pgfsetbuttcap%
\pgfsetroundjoin%
\definecolor{currentfill}{rgb}{1.000000,0.894118,0.788235}%
\pgfsetfillcolor{currentfill}%
\pgfsetlinewidth{0.000000pt}%
\definecolor{currentstroke}{rgb}{1.000000,0.894118,0.788235}%
\pgfsetstrokecolor{currentstroke}%
\pgfsetdash{}{0pt}%
\pgfpathmoveto{\pgfqpoint{3.593460in}{3.359016in}}%
\pgfpathlineto{\pgfqpoint{3.674419in}{3.405758in}}%
\pgfpathlineto{\pgfqpoint{3.674419in}{3.499242in}}%
\pgfpathlineto{\pgfqpoint{3.593460in}{3.452500in}}%
\pgfpathlineto{\pgfqpoint{3.593460in}{3.359016in}}%
\pgfpathclose%
\pgfusepath{fill}%
\end{pgfscope}%
\begin{pgfscope}%
\pgfpathrectangle{\pgfqpoint{0.765000in}{0.660000in}}{\pgfqpoint{4.620000in}{4.620000in}}%
\pgfusepath{clip}%
\pgfsetbuttcap%
\pgfsetroundjoin%
\definecolor{currentfill}{rgb}{1.000000,0.894118,0.788235}%
\pgfsetfillcolor{currentfill}%
\pgfsetlinewidth{0.000000pt}%
\definecolor{currentstroke}{rgb}{1.000000,0.894118,0.788235}%
\pgfsetstrokecolor{currentstroke}%
\pgfsetdash{}{0pt}%
\pgfpathmoveto{\pgfqpoint{3.593460in}{2.973284in}}%
\pgfpathlineto{\pgfqpoint{3.674419in}{3.020026in}}%
\pgfpathlineto{\pgfqpoint{3.674419in}{3.113510in}}%
\pgfpathlineto{\pgfqpoint{3.593460in}{3.066768in}}%
\pgfpathlineto{\pgfqpoint{3.593460in}{2.973284in}}%
\pgfpathclose%
\pgfusepath{fill}%
\end{pgfscope}%
\begin{pgfscope}%
\pgfpathrectangle{\pgfqpoint{0.765000in}{0.660000in}}{\pgfqpoint{4.620000in}{4.620000in}}%
\pgfusepath{clip}%
\pgfsetbuttcap%
\pgfsetroundjoin%
\definecolor{currentfill}{rgb}{1.000000,0.894118,0.788235}%
\pgfsetfillcolor{currentfill}%
\pgfsetlinewidth{0.000000pt}%
\definecolor{currentstroke}{rgb}{1.000000,0.894118,0.788235}%
\pgfsetstrokecolor{currentstroke}%
\pgfsetdash{}{0pt}%
\pgfpathmoveto{\pgfqpoint{3.593460in}{3.545983in}}%
\pgfpathlineto{\pgfqpoint{3.512500in}{3.499242in}}%
\pgfpathlineto{\pgfqpoint{3.593460in}{3.452500in}}%
\pgfpathlineto{\pgfqpoint{3.674419in}{3.499242in}}%
\pgfpathlineto{\pgfqpoint{3.593460in}{3.545983in}}%
\pgfpathclose%
\pgfusepath{fill}%
\end{pgfscope}%
\begin{pgfscope}%
\pgfpathrectangle{\pgfqpoint{0.765000in}{0.660000in}}{\pgfqpoint{4.620000in}{4.620000in}}%
\pgfusepath{clip}%
\pgfsetbuttcap%
\pgfsetroundjoin%
\definecolor{currentfill}{rgb}{1.000000,0.894118,0.788235}%
\pgfsetfillcolor{currentfill}%
\pgfsetlinewidth{0.000000pt}%
\definecolor{currentstroke}{rgb}{1.000000,0.894118,0.788235}%
\pgfsetstrokecolor{currentstroke}%
\pgfsetdash{}{0pt}%
\pgfpathmoveto{\pgfqpoint{3.512500in}{3.405758in}}%
\pgfpathlineto{\pgfqpoint{3.593460in}{3.359016in}}%
\pgfpathlineto{\pgfqpoint{3.593460in}{3.452500in}}%
\pgfpathlineto{\pgfqpoint{3.512500in}{3.499242in}}%
\pgfpathlineto{\pgfqpoint{3.512500in}{3.405758in}}%
\pgfpathclose%
\pgfusepath{fill}%
\end{pgfscope}%
\begin{pgfscope}%
\pgfpathrectangle{\pgfqpoint{0.765000in}{0.660000in}}{\pgfqpoint{4.620000in}{4.620000in}}%
\pgfusepath{clip}%
\pgfsetbuttcap%
\pgfsetroundjoin%
\definecolor{currentfill}{rgb}{1.000000,0.894118,0.788235}%
\pgfsetfillcolor{currentfill}%
\pgfsetlinewidth{0.000000pt}%
\definecolor{currentstroke}{rgb}{1.000000,0.894118,0.788235}%
\pgfsetstrokecolor{currentstroke}%
\pgfsetdash{}{0pt}%
\pgfpathmoveto{\pgfqpoint{3.593460in}{3.160252in}}%
\pgfpathlineto{\pgfqpoint{3.512500in}{3.113510in}}%
\pgfpathlineto{\pgfqpoint{3.593460in}{3.066768in}}%
\pgfpathlineto{\pgfqpoint{3.674419in}{3.113510in}}%
\pgfpathlineto{\pgfqpoint{3.593460in}{3.160252in}}%
\pgfpathclose%
\pgfusepath{fill}%
\end{pgfscope}%
\begin{pgfscope}%
\pgfpathrectangle{\pgfqpoint{0.765000in}{0.660000in}}{\pgfqpoint{4.620000in}{4.620000in}}%
\pgfusepath{clip}%
\pgfsetbuttcap%
\pgfsetroundjoin%
\definecolor{currentfill}{rgb}{1.000000,0.894118,0.788235}%
\pgfsetfillcolor{currentfill}%
\pgfsetlinewidth{0.000000pt}%
\definecolor{currentstroke}{rgb}{1.000000,0.894118,0.788235}%
\pgfsetstrokecolor{currentstroke}%
\pgfsetdash{}{0pt}%
\pgfpathmoveto{\pgfqpoint{3.512500in}{3.020026in}}%
\pgfpathlineto{\pgfqpoint{3.593460in}{2.973284in}}%
\pgfpathlineto{\pgfqpoint{3.593460in}{3.066768in}}%
\pgfpathlineto{\pgfqpoint{3.512500in}{3.113510in}}%
\pgfpathlineto{\pgfqpoint{3.512500in}{3.020026in}}%
\pgfpathclose%
\pgfusepath{fill}%
\end{pgfscope}%
\begin{pgfscope}%
\pgfpathrectangle{\pgfqpoint{0.765000in}{0.660000in}}{\pgfqpoint{4.620000in}{4.620000in}}%
\pgfusepath{clip}%
\pgfsetbuttcap%
\pgfsetroundjoin%
\definecolor{currentfill}{rgb}{1.000000,0.894118,0.788235}%
\pgfsetfillcolor{currentfill}%
\pgfsetlinewidth{0.000000pt}%
\definecolor{currentstroke}{rgb}{1.000000,0.894118,0.788235}%
\pgfsetstrokecolor{currentstroke}%
\pgfsetdash{}{0pt}%
\pgfpathmoveto{\pgfqpoint{3.593460in}{3.452500in}}%
\pgfpathlineto{\pgfqpoint{3.512500in}{3.405758in}}%
\pgfpathlineto{\pgfqpoint{3.512500in}{3.020026in}}%
\pgfpathlineto{\pgfqpoint{3.593460in}{3.066768in}}%
\pgfpathlineto{\pgfqpoint{3.593460in}{3.452500in}}%
\pgfpathclose%
\pgfusepath{fill}%
\end{pgfscope}%
\begin{pgfscope}%
\pgfpathrectangle{\pgfqpoint{0.765000in}{0.660000in}}{\pgfqpoint{4.620000in}{4.620000in}}%
\pgfusepath{clip}%
\pgfsetbuttcap%
\pgfsetroundjoin%
\definecolor{currentfill}{rgb}{1.000000,0.894118,0.788235}%
\pgfsetfillcolor{currentfill}%
\pgfsetlinewidth{0.000000pt}%
\definecolor{currentstroke}{rgb}{1.000000,0.894118,0.788235}%
\pgfsetstrokecolor{currentstroke}%
\pgfsetdash{}{0pt}%
\pgfpathmoveto{\pgfqpoint{3.674419in}{3.405758in}}%
\pgfpathlineto{\pgfqpoint{3.593460in}{3.452500in}}%
\pgfpathlineto{\pgfqpoint{3.593460in}{3.066768in}}%
\pgfpathlineto{\pgfqpoint{3.674419in}{3.020026in}}%
\pgfpathlineto{\pgfqpoint{3.674419in}{3.405758in}}%
\pgfpathclose%
\pgfusepath{fill}%
\end{pgfscope}%
\begin{pgfscope}%
\pgfpathrectangle{\pgfqpoint{0.765000in}{0.660000in}}{\pgfqpoint{4.620000in}{4.620000in}}%
\pgfusepath{clip}%
\pgfsetbuttcap%
\pgfsetroundjoin%
\definecolor{currentfill}{rgb}{1.000000,0.894118,0.788235}%
\pgfsetfillcolor{currentfill}%
\pgfsetlinewidth{0.000000pt}%
\definecolor{currentstroke}{rgb}{1.000000,0.894118,0.788235}%
\pgfsetstrokecolor{currentstroke}%
\pgfsetdash{}{0pt}%
\pgfpathmoveto{\pgfqpoint{3.593460in}{3.452500in}}%
\pgfpathlineto{\pgfqpoint{3.593460in}{3.545983in}}%
\pgfpathlineto{\pgfqpoint{3.593460in}{3.160252in}}%
\pgfpathlineto{\pgfqpoint{3.674419in}{3.020026in}}%
\pgfpathlineto{\pgfqpoint{3.593460in}{3.452500in}}%
\pgfpathclose%
\pgfusepath{fill}%
\end{pgfscope}%
\begin{pgfscope}%
\pgfpathrectangle{\pgfqpoint{0.765000in}{0.660000in}}{\pgfqpoint{4.620000in}{4.620000in}}%
\pgfusepath{clip}%
\pgfsetbuttcap%
\pgfsetroundjoin%
\definecolor{currentfill}{rgb}{1.000000,0.894118,0.788235}%
\pgfsetfillcolor{currentfill}%
\pgfsetlinewidth{0.000000pt}%
\definecolor{currentstroke}{rgb}{1.000000,0.894118,0.788235}%
\pgfsetstrokecolor{currentstroke}%
\pgfsetdash{}{0pt}%
\pgfpathmoveto{\pgfqpoint{3.674419in}{3.405758in}}%
\pgfpathlineto{\pgfqpoint{3.674419in}{3.499242in}}%
\pgfpathlineto{\pgfqpoint{3.674419in}{3.113510in}}%
\pgfpathlineto{\pgfqpoint{3.593460in}{3.066768in}}%
\pgfpathlineto{\pgfqpoint{3.674419in}{3.405758in}}%
\pgfpathclose%
\pgfusepath{fill}%
\end{pgfscope}%
\begin{pgfscope}%
\pgfpathrectangle{\pgfqpoint{0.765000in}{0.660000in}}{\pgfqpoint{4.620000in}{4.620000in}}%
\pgfusepath{clip}%
\pgfsetbuttcap%
\pgfsetroundjoin%
\definecolor{currentfill}{rgb}{1.000000,0.894118,0.788235}%
\pgfsetfillcolor{currentfill}%
\pgfsetlinewidth{0.000000pt}%
\definecolor{currentstroke}{rgb}{1.000000,0.894118,0.788235}%
\pgfsetstrokecolor{currentstroke}%
\pgfsetdash{}{0pt}%
\pgfpathmoveto{\pgfqpoint{3.512500in}{3.405758in}}%
\pgfpathlineto{\pgfqpoint{3.593460in}{3.359016in}}%
\pgfpathlineto{\pgfqpoint{3.593460in}{2.973284in}}%
\pgfpathlineto{\pgfqpoint{3.512500in}{3.020026in}}%
\pgfpathlineto{\pgfqpoint{3.512500in}{3.405758in}}%
\pgfpathclose%
\pgfusepath{fill}%
\end{pgfscope}%
\begin{pgfscope}%
\pgfpathrectangle{\pgfqpoint{0.765000in}{0.660000in}}{\pgfqpoint{4.620000in}{4.620000in}}%
\pgfusepath{clip}%
\pgfsetbuttcap%
\pgfsetroundjoin%
\definecolor{currentfill}{rgb}{1.000000,0.894118,0.788235}%
\pgfsetfillcolor{currentfill}%
\pgfsetlinewidth{0.000000pt}%
\definecolor{currentstroke}{rgb}{1.000000,0.894118,0.788235}%
\pgfsetstrokecolor{currentstroke}%
\pgfsetdash{}{0pt}%
\pgfpathmoveto{\pgfqpoint{3.593460in}{3.359016in}}%
\pgfpathlineto{\pgfqpoint{3.674419in}{3.405758in}}%
\pgfpathlineto{\pgfqpoint{3.674419in}{3.020026in}}%
\pgfpathlineto{\pgfqpoint{3.593460in}{2.973284in}}%
\pgfpathlineto{\pgfqpoint{3.593460in}{3.359016in}}%
\pgfpathclose%
\pgfusepath{fill}%
\end{pgfscope}%
\begin{pgfscope}%
\pgfpathrectangle{\pgfqpoint{0.765000in}{0.660000in}}{\pgfqpoint{4.620000in}{4.620000in}}%
\pgfusepath{clip}%
\pgfsetbuttcap%
\pgfsetroundjoin%
\definecolor{currentfill}{rgb}{1.000000,0.894118,0.788235}%
\pgfsetfillcolor{currentfill}%
\pgfsetlinewidth{0.000000pt}%
\definecolor{currentstroke}{rgb}{1.000000,0.894118,0.788235}%
\pgfsetstrokecolor{currentstroke}%
\pgfsetdash{}{0pt}%
\pgfpathmoveto{\pgfqpoint{3.593460in}{3.545983in}}%
\pgfpathlineto{\pgfqpoint{3.512500in}{3.499242in}}%
\pgfpathlineto{\pgfqpoint{3.512500in}{3.113510in}}%
\pgfpathlineto{\pgfqpoint{3.593460in}{3.160252in}}%
\pgfpathlineto{\pgfqpoint{3.593460in}{3.545983in}}%
\pgfpathclose%
\pgfusepath{fill}%
\end{pgfscope}%
\begin{pgfscope}%
\pgfpathrectangle{\pgfqpoint{0.765000in}{0.660000in}}{\pgfqpoint{4.620000in}{4.620000in}}%
\pgfusepath{clip}%
\pgfsetbuttcap%
\pgfsetroundjoin%
\definecolor{currentfill}{rgb}{1.000000,0.894118,0.788235}%
\pgfsetfillcolor{currentfill}%
\pgfsetlinewidth{0.000000pt}%
\definecolor{currentstroke}{rgb}{1.000000,0.894118,0.788235}%
\pgfsetstrokecolor{currentstroke}%
\pgfsetdash{}{0pt}%
\pgfpathmoveto{\pgfqpoint{3.674419in}{3.499242in}}%
\pgfpathlineto{\pgfqpoint{3.593460in}{3.545983in}}%
\pgfpathlineto{\pgfqpoint{3.593460in}{3.160252in}}%
\pgfpathlineto{\pgfqpoint{3.674419in}{3.113510in}}%
\pgfpathlineto{\pgfqpoint{3.674419in}{3.499242in}}%
\pgfpathclose%
\pgfusepath{fill}%
\end{pgfscope}%
\begin{pgfscope}%
\pgfpathrectangle{\pgfqpoint{0.765000in}{0.660000in}}{\pgfqpoint{4.620000in}{4.620000in}}%
\pgfusepath{clip}%
\pgfsetbuttcap%
\pgfsetroundjoin%
\definecolor{currentfill}{rgb}{1.000000,0.894118,0.788235}%
\pgfsetfillcolor{currentfill}%
\pgfsetlinewidth{0.000000pt}%
\definecolor{currentstroke}{rgb}{1.000000,0.894118,0.788235}%
\pgfsetstrokecolor{currentstroke}%
\pgfsetdash{}{0pt}%
\pgfpathmoveto{\pgfqpoint{3.512500in}{3.405758in}}%
\pgfpathlineto{\pgfqpoint{3.512500in}{3.499242in}}%
\pgfpathlineto{\pgfqpoint{3.512500in}{3.113510in}}%
\pgfpathlineto{\pgfqpoint{3.593460in}{2.973284in}}%
\pgfpathlineto{\pgfqpoint{3.512500in}{3.405758in}}%
\pgfpathclose%
\pgfusepath{fill}%
\end{pgfscope}%
\begin{pgfscope}%
\pgfpathrectangle{\pgfqpoint{0.765000in}{0.660000in}}{\pgfqpoint{4.620000in}{4.620000in}}%
\pgfusepath{clip}%
\pgfsetbuttcap%
\pgfsetroundjoin%
\definecolor{currentfill}{rgb}{1.000000,0.894118,0.788235}%
\pgfsetfillcolor{currentfill}%
\pgfsetlinewidth{0.000000pt}%
\definecolor{currentstroke}{rgb}{1.000000,0.894118,0.788235}%
\pgfsetstrokecolor{currentstroke}%
\pgfsetdash{}{0pt}%
\pgfpathmoveto{\pgfqpoint{3.593460in}{3.359016in}}%
\pgfpathlineto{\pgfqpoint{3.593460in}{3.452500in}}%
\pgfpathlineto{\pgfqpoint{3.593460in}{3.066768in}}%
\pgfpathlineto{\pgfqpoint{3.512500in}{3.020026in}}%
\pgfpathlineto{\pgfqpoint{3.593460in}{3.359016in}}%
\pgfpathclose%
\pgfusepath{fill}%
\end{pgfscope}%
\begin{pgfscope}%
\pgfpathrectangle{\pgfqpoint{0.765000in}{0.660000in}}{\pgfqpoint{4.620000in}{4.620000in}}%
\pgfusepath{clip}%
\pgfsetbuttcap%
\pgfsetroundjoin%
\definecolor{currentfill}{rgb}{1.000000,0.894118,0.788235}%
\pgfsetfillcolor{currentfill}%
\pgfsetlinewidth{0.000000pt}%
\definecolor{currentstroke}{rgb}{1.000000,0.894118,0.788235}%
\pgfsetstrokecolor{currentstroke}%
\pgfsetdash{}{0pt}%
\pgfpathmoveto{\pgfqpoint{3.512500in}{3.499242in}}%
\pgfpathlineto{\pgfqpoint{3.593460in}{3.452500in}}%
\pgfpathlineto{\pgfqpoint{3.593460in}{3.066768in}}%
\pgfpathlineto{\pgfqpoint{3.512500in}{3.113510in}}%
\pgfpathlineto{\pgfqpoint{3.512500in}{3.499242in}}%
\pgfpathclose%
\pgfusepath{fill}%
\end{pgfscope}%
\begin{pgfscope}%
\pgfpathrectangle{\pgfqpoint{0.765000in}{0.660000in}}{\pgfqpoint{4.620000in}{4.620000in}}%
\pgfusepath{clip}%
\pgfsetbuttcap%
\pgfsetroundjoin%
\definecolor{currentfill}{rgb}{1.000000,0.894118,0.788235}%
\pgfsetfillcolor{currentfill}%
\pgfsetlinewidth{0.000000pt}%
\definecolor{currentstroke}{rgb}{1.000000,0.894118,0.788235}%
\pgfsetstrokecolor{currentstroke}%
\pgfsetdash{}{0pt}%
\pgfpathmoveto{\pgfqpoint{3.593460in}{3.452500in}}%
\pgfpathlineto{\pgfqpoint{3.674419in}{3.499242in}}%
\pgfpathlineto{\pgfqpoint{3.674419in}{3.113510in}}%
\pgfpathlineto{\pgfqpoint{3.593460in}{3.066768in}}%
\pgfpathlineto{\pgfqpoint{3.593460in}{3.452500in}}%
\pgfpathclose%
\pgfusepath{fill}%
\end{pgfscope}%
\begin{pgfscope}%
\pgfpathrectangle{\pgfqpoint{0.765000in}{0.660000in}}{\pgfqpoint{4.620000in}{4.620000in}}%
\pgfusepath{clip}%
\pgfsetbuttcap%
\pgfsetroundjoin%
\definecolor{currentfill}{rgb}{1.000000,0.894118,0.788235}%
\pgfsetfillcolor{currentfill}%
\pgfsetlinewidth{0.000000pt}%
\definecolor{currentstroke}{rgb}{1.000000,0.894118,0.788235}%
\pgfsetstrokecolor{currentstroke}%
\pgfsetdash{}{0pt}%
\pgfpathmoveto{\pgfqpoint{3.593460in}{3.452500in}}%
\pgfpathlineto{\pgfqpoint{3.512500in}{3.405758in}}%
\pgfpathlineto{\pgfqpoint{3.593460in}{3.359016in}}%
\pgfpathlineto{\pgfqpoint{3.674419in}{3.405758in}}%
\pgfpathlineto{\pgfqpoint{3.593460in}{3.452500in}}%
\pgfpathclose%
\pgfusepath{fill}%
\end{pgfscope}%
\begin{pgfscope}%
\pgfpathrectangle{\pgfqpoint{0.765000in}{0.660000in}}{\pgfqpoint{4.620000in}{4.620000in}}%
\pgfusepath{clip}%
\pgfsetbuttcap%
\pgfsetroundjoin%
\definecolor{currentfill}{rgb}{1.000000,0.894118,0.788235}%
\pgfsetfillcolor{currentfill}%
\pgfsetlinewidth{0.000000pt}%
\definecolor{currentstroke}{rgb}{1.000000,0.894118,0.788235}%
\pgfsetstrokecolor{currentstroke}%
\pgfsetdash{}{0pt}%
\pgfpathmoveto{\pgfqpoint{3.593460in}{3.452500in}}%
\pgfpathlineto{\pgfqpoint{3.512500in}{3.405758in}}%
\pgfpathlineto{\pgfqpoint{3.512500in}{3.499242in}}%
\pgfpathlineto{\pgfqpoint{3.593460in}{3.545983in}}%
\pgfpathlineto{\pgfqpoint{3.593460in}{3.452500in}}%
\pgfpathclose%
\pgfusepath{fill}%
\end{pgfscope}%
\begin{pgfscope}%
\pgfpathrectangle{\pgfqpoint{0.765000in}{0.660000in}}{\pgfqpoint{4.620000in}{4.620000in}}%
\pgfusepath{clip}%
\pgfsetbuttcap%
\pgfsetroundjoin%
\definecolor{currentfill}{rgb}{1.000000,0.894118,0.788235}%
\pgfsetfillcolor{currentfill}%
\pgfsetlinewidth{0.000000pt}%
\definecolor{currentstroke}{rgb}{1.000000,0.894118,0.788235}%
\pgfsetstrokecolor{currentstroke}%
\pgfsetdash{}{0pt}%
\pgfpathmoveto{\pgfqpoint{3.593460in}{3.452500in}}%
\pgfpathlineto{\pgfqpoint{3.674419in}{3.405758in}}%
\pgfpathlineto{\pgfqpoint{3.674419in}{3.499242in}}%
\pgfpathlineto{\pgfqpoint{3.593460in}{3.545983in}}%
\pgfpathlineto{\pgfqpoint{3.593460in}{3.452500in}}%
\pgfpathclose%
\pgfusepath{fill}%
\end{pgfscope}%
\begin{pgfscope}%
\pgfpathrectangle{\pgfqpoint{0.765000in}{0.660000in}}{\pgfqpoint{4.620000in}{4.620000in}}%
\pgfusepath{clip}%
\pgfsetbuttcap%
\pgfsetroundjoin%
\definecolor{currentfill}{rgb}{1.000000,0.894118,0.788235}%
\pgfsetfillcolor{currentfill}%
\pgfsetlinewidth{0.000000pt}%
\definecolor{currentstroke}{rgb}{1.000000,0.894118,0.788235}%
\pgfsetstrokecolor{currentstroke}%
\pgfsetdash{}{0pt}%
\pgfpathmoveto{\pgfqpoint{2.869318in}{3.452500in}}%
\pgfpathlineto{\pgfqpoint{2.788359in}{3.405758in}}%
\pgfpathlineto{\pgfqpoint{2.869318in}{3.359016in}}%
\pgfpathlineto{\pgfqpoint{2.950278in}{3.405758in}}%
\pgfpathlineto{\pgfqpoint{2.869318in}{3.452500in}}%
\pgfpathclose%
\pgfusepath{fill}%
\end{pgfscope}%
\begin{pgfscope}%
\pgfpathrectangle{\pgfqpoint{0.765000in}{0.660000in}}{\pgfqpoint{4.620000in}{4.620000in}}%
\pgfusepath{clip}%
\pgfsetbuttcap%
\pgfsetroundjoin%
\definecolor{currentfill}{rgb}{1.000000,0.894118,0.788235}%
\pgfsetfillcolor{currentfill}%
\pgfsetlinewidth{0.000000pt}%
\definecolor{currentstroke}{rgb}{1.000000,0.894118,0.788235}%
\pgfsetstrokecolor{currentstroke}%
\pgfsetdash{}{0pt}%
\pgfpathmoveto{\pgfqpoint{2.869318in}{3.452500in}}%
\pgfpathlineto{\pgfqpoint{2.788359in}{3.405758in}}%
\pgfpathlineto{\pgfqpoint{2.788359in}{3.499242in}}%
\pgfpathlineto{\pgfqpoint{2.869318in}{3.545983in}}%
\pgfpathlineto{\pgfqpoint{2.869318in}{3.452500in}}%
\pgfpathclose%
\pgfusepath{fill}%
\end{pgfscope}%
\begin{pgfscope}%
\pgfpathrectangle{\pgfqpoint{0.765000in}{0.660000in}}{\pgfqpoint{4.620000in}{4.620000in}}%
\pgfusepath{clip}%
\pgfsetbuttcap%
\pgfsetroundjoin%
\definecolor{currentfill}{rgb}{1.000000,0.894118,0.788235}%
\pgfsetfillcolor{currentfill}%
\pgfsetlinewidth{0.000000pt}%
\definecolor{currentstroke}{rgb}{1.000000,0.894118,0.788235}%
\pgfsetstrokecolor{currentstroke}%
\pgfsetdash{}{0pt}%
\pgfpathmoveto{\pgfqpoint{2.869318in}{3.452500in}}%
\pgfpathlineto{\pgfqpoint{2.950278in}{3.405758in}}%
\pgfpathlineto{\pgfqpoint{2.950278in}{3.499242in}}%
\pgfpathlineto{\pgfqpoint{2.869318in}{3.545983in}}%
\pgfpathlineto{\pgfqpoint{2.869318in}{3.452500in}}%
\pgfpathclose%
\pgfusepath{fill}%
\end{pgfscope}%
\begin{pgfscope}%
\pgfpathrectangle{\pgfqpoint{0.765000in}{0.660000in}}{\pgfqpoint{4.620000in}{4.620000in}}%
\pgfusepath{clip}%
\pgfsetbuttcap%
\pgfsetroundjoin%
\definecolor{currentfill}{rgb}{1.000000,0.894118,0.788235}%
\pgfsetfillcolor{currentfill}%
\pgfsetlinewidth{0.000000pt}%
\definecolor{currentstroke}{rgb}{1.000000,0.894118,0.788235}%
\pgfsetstrokecolor{currentstroke}%
\pgfsetdash{}{0pt}%
\pgfpathmoveto{\pgfqpoint{3.593460in}{3.359016in}}%
\pgfpathlineto{\pgfqpoint{3.674419in}{3.405758in}}%
\pgfpathlineto{\pgfqpoint{3.674419in}{3.499242in}}%
\pgfpathlineto{\pgfqpoint{3.593460in}{3.452500in}}%
\pgfpathlineto{\pgfqpoint{3.593460in}{3.359016in}}%
\pgfpathclose%
\pgfusepath{fill}%
\end{pgfscope}%
\begin{pgfscope}%
\pgfpathrectangle{\pgfqpoint{0.765000in}{0.660000in}}{\pgfqpoint{4.620000in}{4.620000in}}%
\pgfusepath{clip}%
\pgfsetbuttcap%
\pgfsetroundjoin%
\definecolor{currentfill}{rgb}{1.000000,0.894118,0.788235}%
\pgfsetfillcolor{currentfill}%
\pgfsetlinewidth{0.000000pt}%
\definecolor{currentstroke}{rgb}{1.000000,0.894118,0.788235}%
\pgfsetstrokecolor{currentstroke}%
\pgfsetdash{}{0pt}%
\pgfpathmoveto{\pgfqpoint{2.869318in}{3.359016in}}%
\pgfpathlineto{\pgfqpoint{2.950278in}{3.405758in}}%
\pgfpathlineto{\pgfqpoint{2.950278in}{3.499242in}}%
\pgfpathlineto{\pgfqpoint{2.869318in}{3.452500in}}%
\pgfpathlineto{\pgfqpoint{2.869318in}{3.359016in}}%
\pgfpathclose%
\pgfusepath{fill}%
\end{pgfscope}%
\begin{pgfscope}%
\pgfpathrectangle{\pgfqpoint{0.765000in}{0.660000in}}{\pgfqpoint{4.620000in}{4.620000in}}%
\pgfusepath{clip}%
\pgfsetbuttcap%
\pgfsetroundjoin%
\definecolor{currentfill}{rgb}{1.000000,0.894118,0.788235}%
\pgfsetfillcolor{currentfill}%
\pgfsetlinewidth{0.000000pt}%
\definecolor{currentstroke}{rgb}{1.000000,0.894118,0.788235}%
\pgfsetstrokecolor{currentstroke}%
\pgfsetdash{}{0pt}%
\pgfpathmoveto{\pgfqpoint{3.593460in}{3.545983in}}%
\pgfpathlineto{\pgfqpoint{3.512500in}{3.499242in}}%
\pgfpathlineto{\pgfqpoint{3.593460in}{3.452500in}}%
\pgfpathlineto{\pgfqpoint{3.674419in}{3.499242in}}%
\pgfpathlineto{\pgfqpoint{3.593460in}{3.545983in}}%
\pgfpathclose%
\pgfusepath{fill}%
\end{pgfscope}%
\begin{pgfscope}%
\pgfpathrectangle{\pgfqpoint{0.765000in}{0.660000in}}{\pgfqpoint{4.620000in}{4.620000in}}%
\pgfusepath{clip}%
\pgfsetbuttcap%
\pgfsetroundjoin%
\definecolor{currentfill}{rgb}{1.000000,0.894118,0.788235}%
\pgfsetfillcolor{currentfill}%
\pgfsetlinewidth{0.000000pt}%
\definecolor{currentstroke}{rgb}{1.000000,0.894118,0.788235}%
\pgfsetstrokecolor{currentstroke}%
\pgfsetdash{}{0pt}%
\pgfpathmoveto{\pgfqpoint{3.512500in}{3.405758in}}%
\pgfpathlineto{\pgfqpoint{3.593460in}{3.359016in}}%
\pgfpathlineto{\pgfqpoint{3.593460in}{3.452500in}}%
\pgfpathlineto{\pgfqpoint{3.512500in}{3.499242in}}%
\pgfpathlineto{\pgfqpoint{3.512500in}{3.405758in}}%
\pgfpathclose%
\pgfusepath{fill}%
\end{pgfscope}%
\begin{pgfscope}%
\pgfpathrectangle{\pgfqpoint{0.765000in}{0.660000in}}{\pgfqpoint{4.620000in}{4.620000in}}%
\pgfusepath{clip}%
\pgfsetbuttcap%
\pgfsetroundjoin%
\definecolor{currentfill}{rgb}{1.000000,0.894118,0.788235}%
\pgfsetfillcolor{currentfill}%
\pgfsetlinewidth{0.000000pt}%
\definecolor{currentstroke}{rgb}{1.000000,0.894118,0.788235}%
\pgfsetstrokecolor{currentstroke}%
\pgfsetdash{}{0pt}%
\pgfpathmoveto{\pgfqpoint{2.869318in}{3.545983in}}%
\pgfpathlineto{\pgfqpoint{2.788359in}{3.499242in}}%
\pgfpathlineto{\pgfqpoint{2.869318in}{3.452500in}}%
\pgfpathlineto{\pgfqpoint{2.950278in}{3.499242in}}%
\pgfpathlineto{\pgfqpoint{2.869318in}{3.545983in}}%
\pgfpathclose%
\pgfusepath{fill}%
\end{pgfscope}%
\begin{pgfscope}%
\pgfpathrectangle{\pgfqpoint{0.765000in}{0.660000in}}{\pgfqpoint{4.620000in}{4.620000in}}%
\pgfusepath{clip}%
\pgfsetbuttcap%
\pgfsetroundjoin%
\definecolor{currentfill}{rgb}{1.000000,0.894118,0.788235}%
\pgfsetfillcolor{currentfill}%
\pgfsetlinewidth{0.000000pt}%
\definecolor{currentstroke}{rgb}{1.000000,0.894118,0.788235}%
\pgfsetstrokecolor{currentstroke}%
\pgfsetdash{}{0pt}%
\pgfpathmoveto{\pgfqpoint{2.788359in}{3.405758in}}%
\pgfpathlineto{\pgfqpoint{2.869318in}{3.359016in}}%
\pgfpathlineto{\pgfqpoint{2.869318in}{3.452500in}}%
\pgfpathlineto{\pgfqpoint{2.788359in}{3.499242in}}%
\pgfpathlineto{\pgfqpoint{2.788359in}{3.405758in}}%
\pgfpathclose%
\pgfusepath{fill}%
\end{pgfscope}%
\begin{pgfscope}%
\pgfpathrectangle{\pgfqpoint{0.765000in}{0.660000in}}{\pgfqpoint{4.620000in}{4.620000in}}%
\pgfusepath{clip}%
\pgfsetbuttcap%
\pgfsetroundjoin%
\definecolor{currentfill}{rgb}{1.000000,0.894118,0.788235}%
\pgfsetfillcolor{currentfill}%
\pgfsetlinewidth{0.000000pt}%
\definecolor{currentstroke}{rgb}{1.000000,0.894118,0.788235}%
\pgfsetstrokecolor{currentstroke}%
\pgfsetdash{}{0pt}%
\pgfpathmoveto{\pgfqpoint{3.593460in}{3.452500in}}%
\pgfpathlineto{\pgfqpoint{3.512500in}{3.405758in}}%
\pgfpathlineto{\pgfqpoint{2.788359in}{3.405758in}}%
\pgfpathlineto{\pgfqpoint{2.869318in}{3.452500in}}%
\pgfpathlineto{\pgfqpoint{3.593460in}{3.452500in}}%
\pgfpathclose%
\pgfusepath{fill}%
\end{pgfscope}%
\begin{pgfscope}%
\pgfpathrectangle{\pgfqpoint{0.765000in}{0.660000in}}{\pgfqpoint{4.620000in}{4.620000in}}%
\pgfusepath{clip}%
\pgfsetbuttcap%
\pgfsetroundjoin%
\definecolor{currentfill}{rgb}{1.000000,0.894118,0.788235}%
\pgfsetfillcolor{currentfill}%
\pgfsetlinewidth{0.000000pt}%
\definecolor{currentstroke}{rgb}{1.000000,0.894118,0.788235}%
\pgfsetstrokecolor{currentstroke}%
\pgfsetdash{}{0pt}%
\pgfpathmoveto{\pgfqpoint{3.674419in}{3.405758in}}%
\pgfpathlineto{\pgfqpoint{3.593460in}{3.452500in}}%
\pgfpathlineto{\pgfqpoint{2.869318in}{3.452500in}}%
\pgfpathlineto{\pgfqpoint{2.950278in}{3.405758in}}%
\pgfpathlineto{\pgfqpoint{3.674419in}{3.405758in}}%
\pgfpathclose%
\pgfusepath{fill}%
\end{pgfscope}%
\begin{pgfscope}%
\pgfpathrectangle{\pgfqpoint{0.765000in}{0.660000in}}{\pgfqpoint{4.620000in}{4.620000in}}%
\pgfusepath{clip}%
\pgfsetbuttcap%
\pgfsetroundjoin%
\definecolor{currentfill}{rgb}{1.000000,0.894118,0.788235}%
\pgfsetfillcolor{currentfill}%
\pgfsetlinewidth{0.000000pt}%
\definecolor{currentstroke}{rgb}{1.000000,0.894118,0.788235}%
\pgfsetstrokecolor{currentstroke}%
\pgfsetdash{}{0pt}%
\pgfpathmoveto{\pgfqpoint{3.593460in}{3.452500in}}%
\pgfpathlineto{\pgfqpoint{3.593460in}{3.545983in}}%
\pgfpathlineto{\pgfqpoint{2.869318in}{3.545983in}}%
\pgfpathlineto{\pgfqpoint{2.950278in}{3.405758in}}%
\pgfpathlineto{\pgfqpoint{3.593460in}{3.452500in}}%
\pgfpathclose%
\pgfusepath{fill}%
\end{pgfscope}%
\begin{pgfscope}%
\pgfpathrectangle{\pgfqpoint{0.765000in}{0.660000in}}{\pgfqpoint{4.620000in}{4.620000in}}%
\pgfusepath{clip}%
\pgfsetbuttcap%
\pgfsetroundjoin%
\definecolor{currentfill}{rgb}{1.000000,0.894118,0.788235}%
\pgfsetfillcolor{currentfill}%
\pgfsetlinewidth{0.000000pt}%
\definecolor{currentstroke}{rgb}{1.000000,0.894118,0.788235}%
\pgfsetstrokecolor{currentstroke}%
\pgfsetdash{}{0pt}%
\pgfpathmoveto{\pgfqpoint{3.674419in}{3.405758in}}%
\pgfpathlineto{\pgfqpoint{3.674419in}{3.499242in}}%
\pgfpathlineto{\pgfqpoint{2.950278in}{3.499242in}}%
\pgfpathlineto{\pgfqpoint{2.869318in}{3.452500in}}%
\pgfpathlineto{\pgfqpoint{3.674419in}{3.405758in}}%
\pgfpathclose%
\pgfusepath{fill}%
\end{pgfscope}%
\begin{pgfscope}%
\pgfpathrectangle{\pgfqpoint{0.765000in}{0.660000in}}{\pgfqpoint{4.620000in}{4.620000in}}%
\pgfusepath{clip}%
\pgfsetbuttcap%
\pgfsetroundjoin%
\definecolor{currentfill}{rgb}{1.000000,0.894118,0.788235}%
\pgfsetfillcolor{currentfill}%
\pgfsetlinewidth{0.000000pt}%
\definecolor{currentstroke}{rgb}{1.000000,0.894118,0.788235}%
\pgfsetstrokecolor{currentstroke}%
\pgfsetdash{}{0pt}%
\pgfpathmoveto{\pgfqpoint{3.512500in}{3.405758in}}%
\pgfpathlineto{\pgfqpoint{3.593460in}{3.359016in}}%
\pgfpathlineto{\pgfqpoint{2.869318in}{3.359016in}}%
\pgfpathlineto{\pgfqpoint{2.788359in}{3.405758in}}%
\pgfpathlineto{\pgfqpoint{3.512500in}{3.405758in}}%
\pgfpathclose%
\pgfusepath{fill}%
\end{pgfscope}%
\begin{pgfscope}%
\pgfpathrectangle{\pgfqpoint{0.765000in}{0.660000in}}{\pgfqpoint{4.620000in}{4.620000in}}%
\pgfusepath{clip}%
\pgfsetbuttcap%
\pgfsetroundjoin%
\definecolor{currentfill}{rgb}{1.000000,0.894118,0.788235}%
\pgfsetfillcolor{currentfill}%
\pgfsetlinewidth{0.000000pt}%
\definecolor{currentstroke}{rgb}{1.000000,0.894118,0.788235}%
\pgfsetstrokecolor{currentstroke}%
\pgfsetdash{}{0pt}%
\pgfpathmoveto{\pgfqpoint{3.593460in}{3.359016in}}%
\pgfpathlineto{\pgfqpoint{3.674419in}{3.405758in}}%
\pgfpathlineto{\pgfqpoint{2.950278in}{3.405758in}}%
\pgfpathlineto{\pgfqpoint{2.869318in}{3.359016in}}%
\pgfpathlineto{\pgfqpoint{3.593460in}{3.359016in}}%
\pgfpathclose%
\pgfusepath{fill}%
\end{pgfscope}%
\begin{pgfscope}%
\pgfpathrectangle{\pgfqpoint{0.765000in}{0.660000in}}{\pgfqpoint{4.620000in}{4.620000in}}%
\pgfusepath{clip}%
\pgfsetbuttcap%
\pgfsetroundjoin%
\definecolor{currentfill}{rgb}{1.000000,0.894118,0.788235}%
\pgfsetfillcolor{currentfill}%
\pgfsetlinewidth{0.000000pt}%
\definecolor{currentstroke}{rgb}{1.000000,0.894118,0.788235}%
\pgfsetstrokecolor{currentstroke}%
\pgfsetdash{}{0pt}%
\pgfpathmoveto{\pgfqpoint{3.593460in}{3.545983in}}%
\pgfpathlineto{\pgfqpoint{3.512500in}{3.499242in}}%
\pgfpathlineto{\pgfqpoint{2.788359in}{3.499242in}}%
\pgfpathlineto{\pgfqpoint{2.869318in}{3.545983in}}%
\pgfpathlineto{\pgfqpoint{3.593460in}{3.545983in}}%
\pgfpathclose%
\pgfusepath{fill}%
\end{pgfscope}%
\begin{pgfscope}%
\pgfpathrectangle{\pgfqpoint{0.765000in}{0.660000in}}{\pgfqpoint{4.620000in}{4.620000in}}%
\pgfusepath{clip}%
\pgfsetbuttcap%
\pgfsetroundjoin%
\definecolor{currentfill}{rgb}{1.000000,0.894118,0.788235}%
\pgfsetfillcolor{currentfill}%
\pgfsetlinewidth{0.000000pt}%
\definecolor{currentstroke}{rgb}{1.000000,0.894118,0.788235}%
\pgfsetstrokecolor{currentstroke}%
\pgfsetdash{}{0pt}%
\pgfpathmoveto{\pgfqpoint{3.674419in}{3.499242in}}%
\pgfpathlineto{\pgfqpoint{3.593460in}{3.545983in}}%
\pgfpathlineto{\pgfqpoint{2.869318in}{3.545983in}}%
\pgfpathlineto{\pgfqpoint{2.950278in}{3.499242in}}%
\pgfpathlineto{\pgfqpoint{3.674419in}{3.499242in}}%
\pgfpathclose%
\pgfusepath{fill}%
\end{pgfscope}%
\begin{pgfscope}%
\pgfpathrectangle{\pgfqpoint{0.765000in}{0.660000in}}{\pgfqpoint{4.620000in}{4.620000in}}%
\pgfusepath{clip}%
\pgfsetbuttcap%
\pgfsetroundjoin%
\definecolor{currentfill}{rgb}{1.000000,0.894118,0.788235}%
\pgfsetfillcolor{currentfill}%
\pgfsetlinewidth{0.000000pt}%
\definecolor{currentstroke}{rgb}{1.000000,0.894118,0.788235}%
\pgfsetstrokecolor{currentstroke}%
\pgfsetdash{}{0pt}%
\pgfpathmoveto{\pgfqpoint{3.512500in}{3.405758in}}%
\pgfpathlineto{\pgfqpoint{3.512500in}{3.499242in}}%
\pgfpathlineto{\pgfqpoint{2.788359in}{3.499242in}}%
\pgfpathlineto{\pgfqpoint{2.869318in}{3.359016in}}%
\pgfpathlineto{\pgfqpoint{3.512500in}{3.405758in}}%
\pgfpathclose%
\pgfusepath{fill}%
\end{pgfscope}%
\begin{pgfscope}%
\pgfpathrectangle{\pgfqpoint{0.765000in}{0.660000in}}{\pgfqpoint{4.620000in}{4.620000in}}%
\pgfusepath{clip}%
\pgfsetbuttcap%
\pgfsetroundjoin%
\definecolor{currentfill}{rgb}{1.000000,0.894118,0.788235}%
\pgfsetfillcolor{currentfill}%
\pgfsetlinewidth{0.000000pt}%
\definecolor{currentstroke}{rgb}{1.000000,0.894118,0.788235}%
\pgfsetstrokecolor{currentstroke}%
\pgfsetdash{}{0pt}%
\pgfpathmoveto{\pgfqpoint{3.593460in}{3.359016in}}%
\pgfpathlineto{\pgfqpoint{3.593460in}{3.452500in}}%
\pgfpathlineto{\pgfqpoint{2.869318in}{3.452500in}}%
\pgfpathlineto{\pgfqpoint{2.788359in}{3.405758in}}%
\pgfpathlineto{\pgfqpoint{3.593460in}{3.359016in}}%
\pgfpathclose%
\pgfusepath{fill}%
\end{pgfscope}%
\begin{pgfscope}%
\pgfpathrectangle{\pgfqpoint{0.765000in}{0.660000in}}{\pgfqpoint{4.620000in}{4.620000in}}%
\pgfusepath{clip}%
\pgfsetbuttcap%
\pgfsetroundjoin%
\definecolor{currentfill}{rgb}{1.000000,0.894118,0.788235}%
\pgfsetfillcolor{currentfill}%
\pgfsetlinewidth{0.000000pt}%
\definecolor{currentstroke}{rgb}{1.000000,0.894118,0.788235}%
\pgfsetstrokecolor{currentstroke}%
\pgfsetdash{}{0pt}%
\pgfpathmoveto{\pgfqpoint{3.512500in}{3.499242in}}%
\pgfpathlineto{\pgfqpoint{3.593460in}{3.452500in}}%
\pgfpathlineto{\pgfqpoint{2.869318in}{3.452500in}}%
\pgfpathlineto{\pgfqpoint{2.788359in}{3.499242in}}%
\pgfpathlineto{\pgfqpoint{3.512500in}{3.499242in}}%
\pgfpathclose%
\pgfusepath{fill}%
\end{pgfscope}%
\begin{pgfscope}%
\pgfpathrectangle{\pgfqpoint{0.765000in}{0.660000in}}{\pgfqpoint{4.620000in}{4.620000in}}%
\pgfusepath{clip}%
\pgfsetbuttcap%
\pgfsetroundjoin%
\definecolor{currentfill}{rgb}{1.000000,0.894118,0.788235}%
\pgfsetfillcolor{currentfill}%
\pgfsetlinewidth{0.000000pt}%
\definecolor{currentstroke}{rgb}{1.000000,0.894118,0.788235}%
\pgfsetstrokecolor{currentstroke}%
\pgfsetdash{}{0pt}%
\pgfpathmoveto{\pgfqpoint{3.593460in}{3.452500in}}%
\pgfpathlineto{\pgfqpoint{3.674419in}{3.499242in}}%
\pgfpathlineto{\pgfqpoint{2.950278in}{3.499242in}}%
\pgfpathlineto{\pgfqpoint{2.869318in}{3.452500in}}%
\pgfpathlineto{\pgfqpoint{3.593460in}{3.452500in}}%
\pgfpathclose%
\pgfusepath{fill}%
\end{pgfscope}%
\begin{pgfscope}%
\pgfpathrectangle{\pgfqpoint{0.765000in}{0.660000in}}{\pgfqpoint{4.620000in}{4.620000in}}%
\pgfusepath{clip}%
\pgfsetbuttcap%
\pgfsetroundjoin%
\definecolor{currentfill}{rgb}{1.000000,0.894118,0.788235}%
\pgfsetfillcolor{currentfill}%
\pgfsetlinewidth{0.000000pt}%
\definecolor{currentstroke}{rgb}{1.000000,0.894118,0.788235}%
\pgfsetstrokecolor{currentstroke}%
\pgfsetdash{}{0pt}%
\pgfpathmoveto{\pgfqpoint{3.337483in}{2.623404in}}%
\pgfpathlineto{\pgfqpoint{3.256524in}{2.576662in}}%
\pgfpathlineto{\pgfqpoint{3.337483in}{2.529920in}}%
\pgfpathlineto{\pgfqpoint{3.418443in}{2.576662in}}%
\pgfpathlineto{\pgfqpoint{3.337483in}{2.623404in}}%
\pgfpathclose%
\pgfusepath{fill}%
\end{pgfscope}%
\begin{pgfscope}%
\pgfpathrectangle{\pgfqpoint{0.765000in}{0.660000in}}{\pgfqpoint{4.620000in}{4.620000in}}%
\pgfusepath{clip}%
\pgfsetbuttcap%
\pgfsetroundjoin%
\definecolor{currentfill}{rgb}{1.000000,0.894118,0.788235}%
\pgfsetfillcolor{currentfill}%
\pgfsetlinewidth{0.000000pt}%
\definecolor{currentstroke}{rgb}{1.000000,0.894118,0.788235}%
\pgfsetstrokecolor{currentstroke}%
\pgfsetdash{}{0pt}%
\pgfpathmoveto{\pgfqpoint{3.337483in}{2.623404in}}%
\pgfpathlineto{\pgfqpoint{3.256524in}{2.576662in}}%
\pgfpathlineto{\pgfqpoint{3.256524in}{2.670145in}}%
\pgfpathlineto{\pgfqpoint{3.337483in}{2.716887in}}%
\pgfpathlineto{\pgfqpoint{3.337483in}{2.623404in}}%
\pgfpathclose%
\pgfusepath{fill}%
\end{pgfscope}%
\begin{pgfscope}%
\pgfpathrectangle{\pgfqpoint{0.765000in}{0.660000in}}{\pgfqpoint{4.620000in}{4.620000in}}%
\pgfusepath{clip}%
\pgfsetbuttcap%
\pgfsetroundjoin%
\definecolor{currentfill}{rgb}{1.000000,0.894118,0.788235}%
\pgfsetfillcolor{currentfill}%
\pgfsetlinewidth{0.000000pt}%
\definecolor{currentstroke}{rgb}{1.000000,0.894118,0.788235}%
\pgfsetstrokecolor{currentstroke}%
\pgfsetdash{}{0pt}%
\pgfpathmoveto{\pgfqpoint{3.337483in}{2.623404in}}%
\pgfpathlineto{\pgfqpoint{3.418443in}{2.576662in}}%
\pgfpathlineto{\pgfqpoint{3.418443in}{2.670145in}}%
\pgfpathlineto{\pgfqpoint{3.337483in}{2.716887in}}%
\pgfpathlineto{\pgfqpoint{3.337483in}{2.623404in}}%
\pgfpathclose%
\pgfusepath{fill}%
\end{pgfscope}%
\begin{pgfscope}%
\pgfpathrectangle{\pgfqpoint{0.765000in}{0.660000in}}{\pgfqpoint{4.620000in}{4.620000in}}%
\pgfusepath{clip}%
\pgfsetbuttcap%
\pgfsetroundjoin%
\definecolor{currentfill}{rgb}{1.000000,0.894118,0.788235}%
\pgfsetfillcolor{currentfill}%
\pgfsetlinewidth{0.000000pt}%
\definecolor{currentstroke}{rgb}{1.000000,0.894118,0.788235}%
\pgfsetstrokecolor{currentstroke}%
\pgfsetdash{}{0pt}%
\pgfpathmoveto{\pgfqpoint{3.149560in}{2.514906in}}%
\pgfpathlineto{\pgfqpoint{3.068601in}{2.468164in}}%
\pgfpathlineto{\pgfqpoint{3.149560in}{2.421422in}}%
\pgfpathlineto{\pgfqpoint{3.230519in}{2.468164in}}%
\pgfpathlineto{\pgfqpoint{3.149560in}{2.514906in}}%
\pgfpathclose%
\pgfusepath{fill}%
\end{pgfscope}%
\begin{pgfscope}%
\pgfpathrectangle{\pgfqpoint{0.765000in}{0.660000in}}{\pgfqpoint{4.620000in}{4.620000in}}%
\pgfusepath{clip}%
\pgfsetbuttcap%
\pgfsetroundjoin%
\definecolor{currentfill}{rgb}{1.000000,0.894118,0.788235}%
\pgfsetfillcolor{currentfill}%
\pgfsetlinewidth{0.000000pt}%
\definecolor{currentstroke}{rgb}{1.000000,0.894118,0.788235}%
\pgfsetstrokecolor{currentstroke}%
\pgfsetdash{}{0pt}%
\pgfpathmoveto{\pgfqpoint{3.149560in}{2.514906in}}%
\pgfpathlineto{\pgfqpoint{3.068601in}{2.468164in}}%
\pgfpathlineto{\pgfqpoint{3.068601in}{2.561648in}}%
\pgfpathlineto{\pgfqpoint{3.149560in}{2.608390in}}%
\pgfpathlineto{\pgfqpoint{3.149560in}{2.514906in}}%
\pgfpathclose%
\pgfusepath{fill}%
\end{pgfscope}%
\begin{pgfscope}%
\pgfpathrectangle{\pgfqpoint{0.765000in}{0.660000in}}{\pgfqpoint{4.620000in}{4.620000in}}%
\pgfusepath{clip}%
\pgfsetbuttcap%
\pgfsetroundjoin%
\definecolor{currentfill}{rgb}{1.000000,0.894118,0.788235}%
\pgfsetfillcolor{currentfill}%
\pgfsetlinewidth{0.000000pt}%
\definecolor{currentstroke}{rgb}{1.000000,0.894118,0.788235}%
\pgfsetstrokecolor{currentstroke}%
\pgfsetdash{}{0pt}%
\pgfpathmoveto{\pgfqpoint{3.149560in}{2.514906in}}%
\pgfpathlineto{\pgfqpoint{3.230519in}{2.468164in}}%
\pgfpathlineto{\pgfqpoint{3.230519in}{2.561648in}}%
\pgfpathlineto{\pgfqpoint{3.149560in}{2.608390in}}%
\pgfpathlineto{\pgfqpoint{3.149560in}{2.514906in}}%
\pgfpathclose%
\pgfusepath{fill}%
\end{pgfscope}%
\begin{pgfscope}%
\pgfpathrectangle{\pgfqpoint{0.765000in}{0.660000in}}{\pgfqpoint{4.620000in}{4.620000in}}%
\pgfusepath{clip}%
\pgfsetbuttcap%
\pgfsetroundjoin%
\definecolor{currentfill}{rgb}{1.000000,0.894118,0.788235}%
\pgfsetfillcolor{currentfill}%
\pgfsetlinewidth{0.000000pt}%
\definecolor{currentstroke}{rgb}{1.000000,0.894118,0.788235}%
\pgfsetstrokecolor{currentstroke}%
\pgfsetdash{}{0pt}%
\pgfpathmoveto{\pgfqpoint{3.337483in}{2.529920in}}%
\pgfpathlineto{\pgfqpoint{3.418443in}{2.576662in}}%
\pgfpathlineto{\pgfqpoint{3.418443in}{2.670145in}}%
\pgfpathlineto{\pgfqpoint{3.337483in}{2.623404in}}%
\pgfpathlineto{\pgfqpoint{3.337483in}{2.529920in}}%
\pgfpathclose%
\pgfusepath{fill}%
\end{pgfscope}%
\begin{pgfscope}%
\pgfpathrectangle{\pgfqpoint{0.765000in}{0.660000in}}{\pgfqpoint{4.620000in}{4.620000in}}%
\pgfusepath{clip}%
\pgfsetbuttcap%
\pgfsetroundjoin%
\definecolor{currentfill}{rgb}{1.000000,0.894118,0.788235}%
\pgfsetfillcolor{currentfill}%
\pgfsetlinewidth{0.000000pt}%
\definecolor{currentstroke}{rgb}{1.000000,0.894118,0.788235}%
\pgfsetstrokecolor{currentstroke}%
\pgfsetdash{}{0pt}%
\pgfpathmoveto{\pgfqpoint{3.149560in}{2.421422in}}%
\pgfpathlineto{\pgfqpoint{3.230519in}{2.468164in}}%
\pgfpathlineto{\pgfqpoint{3.230519in}{2.561648in}}%
\pgfpathlineto{\pgfqpoint{3.149560in}{2.514906in}}%
\pgfpathlineto{\pgfqpoint{3.149560in}{2.421422in}}%
\pgfpathclose%
\pgfusepath{fill}%
\end{pgfscope}%
\begin{pgfscope}%
\pgfpathrectangle{\pgfqpoint{0.765000in}{0.660000in}}{\pgfqpoint{4.620000in}{4.620000in}}%
\pgfusepath{clip}%
\pgfsetbuttcap%
\pgfsetroundjoin%
\definecolor{currentfill}{rgb}{1.000000,0.894118,0.788235}%
\pgfsetfillcolor{currentfill}%
\pgfsetlinewidth{0.000000pt}%
\definecolor{currentstroke}{rgb}{1.000000,0.894118,0.788235}%
\pgfsetstrokecolor{currentstroke}%
\pgfsetdash{}{0pt}%
\pgfpathmoveto{\pgfqpoint{3.337483in}{2.716887in}}%
\pgfpathlineto{\pgfqpoint{3.256524in}{2.670145in}}%
\pgfpathlineto{\pgfqpoint{3.337483in}{2.623404in}}%
\pgfpathlineto{\pgfqpoint{3.418443in}{2.670145in}}%
\pgfpathlineto{\pgfqpoint{3.337483in}{2.716887in}}%
\pgfpathclose%
\pgfusepath{fill}%
\end{pgfscope}%
\begin{pgfscope}%
\pgfpathrectangle{\pgfqpoint{0.765000in}{0.660000in}}{\pgfqpoint{4.620000in}{4.620000in}}%
\pgfusepath{clip}%
\pgfsetbuttcap%
\pgfsetroundjoin%
\definecolor{currentfill}{rgb}{1.000000,0.894118,0.788235}%
\pgfsetfillcolor{currentfill}%
\pgfsetlinewidth{0.000000pt}%
\definecolor{currentstroke}{rgb}{1.000000,0.894118,0.788235}%
\pgfsetstrokecolor{currentstroke}%
\pgfsetdash{}{0pt}%
\pgfpathmoveto{\pgfqpoint{3.256524in}{2.576662in}}%
\pgfpathlineto{\pgfqpoint{3.337483in}{2.529920in}}%
\pgfpathlineto{\pgfqpoint{3.337483in}{2.623404in}}%
\pgfpathlineto{\pgfqpoint{3.256524in}{2.670145in}}%
\pgfpathlineto{\pgfqpoint{3.256524in}{2.576662in}}%
\pgfpathclose%
\pgfusepath{fill}%
\end{pgfscope}%
\begin{pgfscope}%
\pgfpathrectangle{\pgfqpoint{0.765000in}{0.660000in}}{\pgfqpoint{4.620000in}{4.620000in}}%
\pgfusepath{clip}%
\pgfsetbuttcap%
\pgfsetroundjoin%
\definecolor{currentfill}{rgb}{1.000000,0.894118,0.788235}%
\pgfsetfillcolor{currentfill}%
\pgfsetlinewidth{0.000000pt}%
\definecolor{currentstroke}{rgb}{1.000000,0.894118,0.788235}%
\pgfsetstrokecolor{currentstroke}%
\pgfsetdash{}{0pt}%
\pgfpathmoveto{\pgfqpoint{3.149560in}{2.608390in}}%
\pgfpathlineto{\pgfqpoint{3.068601in}{2.561648in}}%
\pgfpathlineto{\pgfqpoint{3.149560in}{2.514906in}}%
\pgfpathlineto{\pgfqpoint{3.230519in}{2.561648in}}%
\pgfpathlineto{\pgfqpoint{3.149560in}{2.608390in}}%
\pgfpathclose%
\pgfusepath{fill}%
\end{pgfscope}%
\begin{pgfscope}%
\pgfpathrectangle{\pgfqpoint{0.765000in}{0.660000in}}{\pgfqpoint{4.620000in}{4.620000in}}%
\pgfusepath{clip}%
\pgfsetbuttcap%
\pgfsetroundjoin%
\definecolor{currentfill}{rgb}{1.000000,0.894118,0.788235}%
\pgfsetfillcolor{currentfill}%
\pgfsetlinewidth{0.000000pt}%
\definecolor{currentstroke}{rgb}{1.000000,0.894118,0.788235}%
\pgfsetstrokecolor{currentstroke}%
\pgfsetdash{}{0pt}%
\pgfpathmoveto{\pgfqpoint{3.068601in}{2.468164in}}%
\pgfpathlineto{\pgfqpoint{3.149560in}{2.421422in}}%
\pgfpathlineto{\pgfqpoint{3.149560in}{2.514906in}}%
\pgfpathlineto{\pgfqpoint{3.068601in}{2.561648in}}%
\pgfpathlineto{\pgfqpoint{3.068601in}{2.468164in}}%
\pgfpathclose%
\pgfusepath{fill}%
\end{pgfscope}%
\begin{pgfscope}%
\pgfpathrectangle{\pgfqpoint{0.765000in}{0.660000in}}{\pgfqpoint{4.620000in}{4.620000in}}%
\pgfusepath{clip}%
\pgfsetbuttcap%
\pgfsetroundjoin%
\definecolor{currentfill}{rgb}{1.000000,0.894118,0.788235}%
\pgfsetfillcolor{currentfill}%
\pgfsetlinewidth{0.000000pt}%
\definecolor{currentstroke}{rgb}{1.000000,0.894118,0.788235}%
\pgfsetstrokecolor{currentstroke}%
\pgfsetdash{}{0pt}%
\pgfpathmoveto{\pgfqpoint{3.337483in}{2.623404in}}%
\pgfpathlineto{\pgfqpoint{3.256524in}{2.576662in}}%
\pgfpathlineto{\pgfqpoint{3.068601in}{2.468164in}}%
\pgfpathlineto{\pgfqpoint{3.149560in}{2.514906in}}%
\pgfpathlineto{\pgfqpoint{3.337483in}{2.623404in}}%
\pgfpathclose%
\pgfusepath{fill}%
\end{pgfscope}%
\begin{pgfscope}%
\pgfpathrectangle{\pgfqpoint{0.765000in}{0.660000in}}{\pgfqpoint{4.620000in}{4.620000in}}%
\pgfusepath{clip}%
\pgfsetbuttcap%
\pgfsetroundjoin%
\definecolor{currentfill}{rgb}{1.000000,0.894118,0.788235}%
\pgfsetfillcolor{currentfill}%
\pgfsetlinewidth{0.000000pt}%
\definecolor{currentstroke}{rgb}{1.000000,0.894118,0.788235}%
\pgfsetstrokecolor{currentstroke}%
\pgfsetdash{}{0pt}%
\pgfpathmoveto{\pgfqpoint{3.418443in}{2.576662in}}%
\pgfpathlineto{\pgfqpoint{3.337483in}{2.623404in}}%
\pgfpathlineto{\pgfqpoint{3.149560in}{2.514906in}}%
\pgfpathlineto{\pgfqpoint{3.230519in}{2.468164in}}%
\pgfpathlineto{\pgfqpoint{3.418443in}{2.576662in}}%
\pgfpathclose%
\pgfusepath{fill}%
\end{pgfscope}%
\begin{pgfscope}%
\pgfpathrectangle{\pgfqpoint{0.765000in}{0.660000in}}{\pgfqpoint{4.620000in}{4.620000in}}%
\pgfusepath{clip}%
\pgfsetbuttcap%
\pgfsetroundjoin%
\definecolor{currentfill}{rgb}{1.000000,0.894118,0.788235}%
\pgfsetfillcolor{currentfill}%
\pgfsetlinewidth{0.000000pt}%
\definecolor{currentstroke}{rgb}{1.000000,0.894118,0.788235}%
\pgfsetstrokecolor{currentstroke}%
\pgfsetdash{}{0pt}%
\pgfpathmoveto{\pgfqpoint{3.337483in}{2.623404in}}%
\pgfpathlineto{\pgfqpoint{3.337483in}{2.716887in}}%
\pgfpathlineto{\pgfqpoint{3.149560in}{2.608390in}}%
\pgfpathlineto{\pgfqpoint{3.230519in}{2.468164in}}%
\pgfpathlineto{\pgfqpoint{3.337483in}{2.623404in}}%
\pgfpathclose%
\pgfusepath{fill}%
\end{pgfscope}%
\begin{pgfscope}%
\pgfpathrectangle{\pgfqpoint{0.765000in}{0.660000in}}{\pgfqpoint{4.620000in}{4.620000in}}%
\pgfusepath{clip}%
\pgfsetbuttcap%
\pgfsetroundjoin%
\definecolor{currentfill}{rgb}{1.000000,0.894118,0.788235}%
\pgfsetfillcolor{currentfill}%
\pgfsetlinewidth{0.000000pt}%
\definecolor{currentstroke}{rgb}{1.000000,0.894118,0.788235}%
\pgfsetstrokecolor{currentstroke}%
\pgfsetdash{}{0pt}%
\pgfpathmoveto{\pgfqpoint{3.418443in}{2.576662in}}%
\pgfpathlineto{\pgfqpoint{3.418443in}{2.670145in}}%
\pgfpathlineto{\pgfqpoint{3.230519in}{2.561648in}}%
\pgfpathlineto{\pgfqpoint{3.149560in}{2.514906in}}%
\pgfpathlineto{\pgfqpoint{3.418443in}{2.576662in}}%
\pgfpathclose%
\pgfusepath{fill}%
\end{pgfscope}%
\begin{pgfscope}%
\pgfpathrectangle{\pgfqpoint{0.765000in}{0.660000in}}{\pgfqpoint{4.620000in}{4.620000in}}%
\pgfusepath{clip}%
\pgfsetbuttcap%
\pgfsetroundjoin%
\definecolor{currentfill}{rgb}{1.000000,0.894118,0.788235}%
\pgfsetfillcolor{currentfill}%
\pgfsetlinewidth{0.000000pt}%
\definecolor{currentstroke}{rgb}{1.000000,0.894118,0.788235}%
\pgfsetstrokecolor{currentstroke}%
\pgfsetdash{}{0pt}%
\pgfpathmoveto{\pgfqpoint{3.256524in}{2.576662in}}%
\pgfpathlineto{\pgfqpoint{3.337483in}{2.529920in}}%
\pgfpathlineto{\pgfqpoint{3.149560in}{2.421422in}}%
\pgfpathlineto{\pgfqpoint{3.068601in}{2.468164in}}%
\pgfpathlineto{\pgfqpoint{3.256524in}{2.576662in}}%
\pgfpathclose%
\pgfusepath{fill}%
\end{pgfscope}%
\begin{pgfscope}%
\pgfpathrectangle{\pgfqpoint{0.765000in}{0.660000in}}{\pgfqpoint{4.620000in}{4.620000in}}%
\pgfusepath{clip}%
\pgfsetbuttcap%
\pgfsetroundjoin%
\definecolor{currentfill}{rgb}{1.000000,0.894118,0.788235}%
\pgfsetfillcolor{currentfill}%
\pgfsetlinewidth{0.000000pt}%
\definecolor{currentstroke}{rgb}{1.000000,0.894118,0.788235}%
\pgfsetstrokecolor{currentstroke}%
\pgfsetdash{}{0pt}%
\pgfpathmoveto{\pgfqpoint{3.337483in}{2.529920in}}%
\pgfpathlineto{\pgfqpoint{3.418443in}{2.576662in}}%
\pgfpathlineto{\pgfqpoint{3.230519in}{2.468164in}}%
\pgfpathlineto{\pgfqpoint{3.149560in}{2.421422in}}%
\pgfpathlineto{\pgfqpoint{3.337483in}{2.529920in}}%
\pgfpathclose%
\pgfusepath{fill}%
\end{pgfscope}%
\begin{pgfscope}%
\pgfpathrectangle{\pgfqpoint{0.765000in}{0.660000in}}{\pgfqpoint{4.620000in}{4.620000in}}%
\pgfusepath{clip}%
\pgfsetbuttcap%
\pgfsetroundjoin%
\definecolor{currentfill}{rgb}{1.000000,0.894118,0.788235}%
\pgfsetfillcolor{currentfill}%
\pgfsetlinewidth{0.000000pt}%
\definecolor{currentstroke}{rgb}{1.000000,0.894118,0.788235}%
\pgfsetstrokecolor{currentstroke}%
\pgfsetdash{}{0pt}%
\pgfpathmoveto{\pgfqpoint{3.337483in}{2.716887in}}%
\pgfpathlineto{\pgfqpoint{3.256524in}{2.670145in}}%
\pgfpathlineto{\pgfqpoint{3.068601in}{2.561648in}}%
\pgfpathlineto{\pgfqpoint{3.149560in}{2.608390in}}%
\pgfpathlineto{\pgfqpoint{3.337483in}{2.716887in}}%
\pgfpathclose%
\pgfusepath{fill}%
\end{pgfscope}%
\begin{pgfscope}%
\pgfpathrectangle{\pgfqpoint{0.765000in}{0.660000in}}{\pgfqpoint{4.620000in}{4.620000in}}%
\pgfusepath{clip}%
\pgfsetbuttcap%
\pgfsetroundjoin%
\definecolor{currentfill}{rgb}{1.000000,0.894118,0.788235}%
\pgfsetfillcolor{currentfill}%
\pgfsetlinewidth{0.000000pt}%
\definecolor{currentstroke}{rgb}{1.000000,0.894118,0.788235}%
\pgfsetstrokecolor{currentstroke}%
\pgfsetdash{}{0pt}%
\pgfpathmoveto{\pgfqpoint{3.418443in}{2.670145in}}%
\pgfpathlineto{\pgfqpoint{3.337483in}{2.716887in}}%
\pgfpathlineto{\pgfqpoint{3.149560in}{2.608390in}}%
\pgfpathlineto{\pgfqpoint{3.230519in}{2.561648in}}%
\pgfpathlineto{\pgfqpoint{3.418443in}{2.670145in}}%
\pgfpathclose%
\pgfusepath{fill}%
\end{pgfscope}%
\begin{pgfscope}%
\pgfpathrectangle{\pgfqpoint{0.765000in}{0.660000in}}{\pgfqpoint{4.620000in}{4.620000in}}%
\pgfusepath{clip}%
\pgfsetbuttcap%
\pgfsetroundjoin%
\definecolor{currentfill}{rgb}{1.000000,0.894118,0.788235}%
\pgfsetfillcolor{currentfill}%
\pgfsetlinewidth{0.000000pt}%
\definecolor{currentstroke}{rgb}{1.000000,0.894118,0.788235}%
\pgfsetstrokecolor{currentstroke}%
\pgfsetdash{}{0pt}%
\pgfpathmoveto{\pgfqpoint{3.256524in}{2.576662in}}%
\pgfpathlineto{\pgfqpoint{3.256524in}{2.670145in}}%
\pgfpathlineto{\pgfqpoint{3.068601in}{2.561648in}}%
\pgfpathlineto{\pgfqpoint{3.149560in}{2.421422in}}%
\pgfpathlineto{\pgfqpoint{3.256524in}{2.576662in}}%
\pgfpathclose%
\pgfusepath{fill}%
\end{pgfscope}%
\begin{pgfscope}%
\pgfpathrectangle{\pgfqpoint{0.765000in}{0.660000in}}{\pgfqpoint{4.620000in}{4.620000in}}%
\pgfusepath{clip}%
\pgfsetbuttcap%
\pgfsetroundjoin%
\definecolor{currentfill}{rgb}{1.000000,0.894118,0.788235}%
\pgfsetfillcolor{currentfill}%
\pgfsetlinewidth{0.000000pt}%
\definecolor{currentstroke}{rgb}{1.000000,0.894118,0.788235}%
\pgfsetstrokecolor{currentstroke}%
\pgfsetdash{}{0pt}%
\pgfpathmoveto{\pgfqpoint{3.337483in}{2.529920in}}%
\pgfpathlineto{\pgfqpoint{3.337483in}{2.623404in}}%
\pgfpathlineto{\pgfqpoint{3.149560in}{2.514906in}}%
\pgfpathlineto{\pgfqpoint{3.068601in}{2.468164in}}%
\pgfpathlineto{\pgfqpoint{3.337483in}{2.529920in}}%
\pgfpathclose%
\pgfusepath{fill}%
\end{pgfscope}%
\begin{pgfscope}%
\pgfpathrectangle{\pgfqpoint{0.765000in}{0.660000in}}{\pgfqpoint{4.620000in}{4.620000in}}%
\pgfusepath{clip}%
\pgfsetbuttcap%
\pgfsetroundjoin%
\definecolor{currentfill}{rgb}{1.000000,0.894118,0.788235}%
\pgfsetfillcolor{currentfill}%
\pgfsetlinewidth{0.000000pt}%
\definecolor{currentstroke}{rgb}{1.000000,0.894118,0.788235}%
\pgfsetstrokecolor{currentstroke}%
\pgfsetdash{}{0pt}%
\pgfpathmoveto{\pgfqpoint{3.256524in}{2.670145in}}%
\pgfpathlineto{\pgfqpoint{3.337483in}{2.623404in}}%
\pgfpathlineto{\pgfqpoint{3.149560in}{2.514906in}}%
\pgfpathlineto{\pgfqpoint{3.068601in}{2.561648in}}%
\pgfpathlineto{\pgfqpoint{3.256524in}{2.670145in}}%
\pgfpathclose%
\pgfusepath{fill}%
\end{pgfscope}%
\begin{pgfscope}%
\pgfpathrectangle{\pgfqpoint{0.765000in}{0.660000in}}{\pgfqpoint{4.620000in}{4.620000in}}%
\pgfusepath{clip}%
\pgfsetbuttcap%
\pgfsetroundjoin%
\definecolor{currentfill}{rgb}{1.000000,0.894118,0.788235}%
\pgfsetfillcolor{currentfill}%
\pgfsetlinewidth{0.000000pt}%
\definecolor{currentstroke}{rgb}{1.000000,0.894118,0.788235}%
\pgfsetstrokecolor{currentstroke}%
\pgfsetdash{}{0pt}%
\pgfpathmoveto{\pgfqpoint{3.337483in}{2.623404in}}%
\pgfpathlineto{\pgfqpoint{3.418443in}{2.670145in}}%
\pgfpathlineto{\pgfqpoint{3.230519in}{2.561648in}}%
\pgfpathlineto{\pgfqpoint{3.149560in}{2.514906in}}%
\pgfpathlineto{\pgfqpoint{3.337483in}{2.623404in}}%
\pgfpathclose%
\pgfusepath{fill}%
\end{pgfscope}%
\begin{pgfscope}%
\pgfpathrectangle{\pgfqpoint{0.765000in}{0.660000in}}{\pgfqpoint{4.620000in}{4.620000in}}%
\pgfusepath{clip}%
\pgfsetbuttcap%
\pgfsetroundjoin%
\definecolor{currentfill}{rgb}{1.000000,0.894118,0.788235}%
\pgfsetfillcolor{currentfill}%
\pgfsetlinewidth{0.000000pt}%
\definecolor{currentstroke}{rgb}{1.000000,0.894118,0.788235}%
\pgfsetstrokecolor{currentstroke}%
\pgfsetdash{}{0pt}%
\pgfpathmoveto{\pgfqpoint{3.337483in}{2.623404in}}%
\pgfpathlineto{\pgfqpoint{3.256524in}{2.576662in}}%
\pgfpathlineto{\pgfqpoint{3.337483in}{2.529920in}}%
\pgfpathlineto{\pgfqpoint{3.418443in}{2.576662in}}%
\pgfpathlineto{\pgfqpoint{3.337483in}{2.623404in}}%
\pgfpathclose%
\pgfusepath{fill}%
\end{pgfscope}%
\begin{pgfscope}%
\pgfpathrectangle{\pgfqpoint{0.765000in}{0.660000in}}{\pgfqpoint{4.620000in}{4.620000in}}%
\pgfusepath{clip}%
\pgfsetbuttcap%
\pgfsetroundjoin%
\definecolor{currentfill}{rgb}{1.000000,0.894118,0.788235}%
\pgfsetfillcolor{currentfill}%
\pgfsetlinewidth{0.000000pt}%
\definecolor{currentstroke}{rgb}{1.000000,0.894118,0.788235}%
\pgfsetstrokecolor{currentstroke}%
\pgfsetdash{}{0pt}%
\pgfpathmoveto{\pgfqpoint{3.337483in}{2.623404in}}%
\pgfpathlineto{\pgfqpoint{3.256524in}{2.576662in}}%
\pgfpathlineto{\pgfqpoint{3.256524in}{2.670145in}}%
\pgfpathlineto{\pgfqpoint{3.337483in}{2.716887in}}%
\pgfpathlineto{\pgfqpoint{3.337483in}{2.623404in}}%
\pgfpathclose%
\pgfusepath{fill}%
\end{pgfscope}%
\begin{pgfscope}%
\pgfpathrectangle{\pgfqpoint{0.765000in}{0.660000in}}{\pgfqpoint{4.620000in}{4.620000in}}%
\pgfusepath{clip}%
\pgfsetbuttcap%
\pgfsetroundjoin%
\definecolor{currentfill}{rgb}{1.000000,0.894118,0.788235}%
\pgfsetfillcolor{currentfill}%
\pgfsetlinewidth{0.000000pt}%
\definecolor{currentstroke}{rgb}{1.000000,0.894118,0.788235}%
\pgfsetstrokecolor{currentstroke}%
\pgfsetdash{}{0pt}%
\pgfpathmoveto{\pgfqpoint{3.337483in}{2.623404in}}%
\pgfpathlineto{\pgfqpoint{3.418443in}{2.576662in}}%
\pgfpathlineto{\pgfqpoint{3.418443in}{2.670145in}}%
\pgfpathlineto{\pgfqpoint{3.337483in}{2.716887in}}%
\pgfpathlineto{\pgfqpoint{3.337483in}{2.623404in}}%
\pgfpathclose%
\pgfusepath{fill}%
\end{pgfscope}%
\begin{pgfscope}%
\pgfpathrectangle{\pgfqpoint{0.765000in}{0.660000in}}{\pgfqpoint{4.620000in}{4.620000in}}%
\pgfusepath{clip}%
\pgfsetbuttcap%
\pgfsetroundjoin%
\definecolor{currentfill}{rgb}{1.000000,0.894118,0.788235}%
\pgfsetfillcolor{currentfill}%
\pgfsetlinewidth{0.000000pt}%
\definecolor{currentstroke}{rgb}{1.000000,0.894118,0.788235}%
\pgfsetstrokecolor{currentstroke}%
\pgfsetdash{}{0pt}%
\pgfpathmoveto{\pgfqpoint{3.593460in}{3.066768in}}%
\pgfpathlineto{\pgfqpoint{3.512500in}{3.020026in}}%
\pgfpathlineto{\pgfqpoint{3.593460in}{2.973284in}}%
\pgfpathlineto{\pgfqpoint{3.674419in}{3.020026in}}%
\pgfpathlineto{\pgfqpoint{3.593460in}{3.066768in}}%
\pgfpathclose%
\pgfusepath{fill}%
\end{pgfscope}%
\begin{pgfscope}%
\pgfpathrectangle{\pgfqpoint{0.765000in}{0.660000in}}{\pgfqpoint{4.620000in}{4.620000in}}%
\pgfusepath{clip}%
\pgfsetbuttcap%
\pgfsetroundjoin%
\definecolor{currentfill}{rgb}{1.000000,0.894118,0.788235}%
\pgfsetfillcolor{currentfill}%
\pgfsetlinewidth{0.000000pt}%
\definecolor{currentstroke}{rgb}{1.000000,0.894118,0.788235}%
\pgfsetstrokecolor{currentstroke}%
\pgfsetdash{}{0pt}%
\pgfpathmoveto{\pgfqpoint{3.593460in}{3.066768in}}%
\pgfpathlineto{\pgfqpoint{3.512500in}{3.020026in}}%
\pgfpathlineto{\pgfqpoint{3.512500in}{3.113510in}}%
\pgfpathlineto{\pgfqpoint{3.593460in}{3.160252in}}%
\pgfpathlineto{\pgfqpoint{3.593460in}{3.066768in}}%
\pgfpathclose%
\pgfusepath{fill}%
\end{pgfscope}%
\begin{pgfscope}%
\pgfpathrectangle{\pgfqpoint{0.765000in}{0.660000in}}{\pgfqpoint{4.620000in}{4.620000in}}%
\pgfusepath{clip}%
\pgfsetbuttcap%
\pgfsetroundjoin%
\definecolor{currentfill}{rgb}{1.000000,0.894118,0.788235}%
\pgfsetfillcolor{currentfill}%
\pgfsetlinewidth{0.000000pt}%
\definecolor{currentstroke}{rgb}{1.000000,0.894118,0.788235}%
\pgfsetstrokecolor{currentstroke}%
\pgfsetdash{}{0pt}%
\pgfpathmoveto{\pgfqpoint{3.593460in}{3.066768in}}%
\pgfpathlineto{\pgfqpoint{3.674419in}{3.020026in}}%
\pgfpathlineto{\pgfqpoint{3.674419in}{3.113510in}}%
\pgfpathlineto{\pgfqpoint{3.593460in}{3.160252in}}%
\pgfpathlineto{\pgfqpoint{3.593460in}{3.066768in}}%
\pgfpathclose%
\pgfusepath{fill}%
\end{pgfscope}%
\begin{pgfscope}%
\pgfpathrectangle{\pgfqpoint{0.765000in}{0.660000in}}{\pgfqpoint{4.620000in}{4.620000in}}%
\pgfusepath{clip}%
\pgfsetbuttcap%
\pgfsetroundjoin%
\definecolor{currentfill}{rgb}{1.000000,0.894118,0.788235}%
\pgfsetfillcolor{currentfill}%
\pgfsetlinewidth{0.000000pt}%
\definecolor{currentstroke}{rgb}{1.000000,0.894118,0.788235}%
\pgfsetstrokecolor{currentstroke}%
\pgfsetdash{}{0pt}%
\pgfpathmoveto{\pgfqpoint{3.337483in}{2.529920in}}%
\pgfpathlineto{\pgfqpoint{3.418443in}{2.576662in}}%
\pgfpathlineto{\pgfqpoint{3.418443in}{2.670145in}}%
\pgfpathlineto{\pgfqpoint{3.337483in}{2.623404in}}%
\pgfpathlineto{\pgfqpoint{3.337483in}{2.529920in}}%
\pgfpathclose%
\pgfusepath{fill}%
\end{pgfscope}%
\begin{pgfscope}%
\pgfpathrectangle{\pgfqpoint{0.765000in}{0.660000in}}{\pgfqpoint{4.620000in}{4.620000in}}%
\pgfusepath{clip}%
\pgfsetbuttcap%
\pgfsetroundjoin%
\definecolor{currentfill}{rgb}{1.000000,0.894118,0.788235}%
\pgfsetfillcolor{currentfill}%
\pgfsetlinewidth{0.000000pt}%
\definecolor{currentstroke}{rgb}{1.000000,0.894118,0.788235}%
\pgfsetstrokecolor{currentstroke}%
\pgfsetdash{}{0pt}%
\pgfpathmoveto{\pgfqpoint{3.593460in}{2.973284in}}%
\pgfpathlineto{\pgfqpoint{3.674419in}{3.020026in}}%
\pgfpathlineto{\pgfqpoint{3.674419in}{3.113510in}}%
\pgfpathlineto{\pgfqpoint{3.593460in}{3.066768in}}%
\pgfpathlineto{\pgfqpoint{3.593460in}{2.973284in}}%
\pgfpathclose%
\pgfusepath{fill}%
\end{pgfscope}%
\begin{pgfscope}%
\pgfpathrectangle{\pgfqpoint{0.765000in}{0.660000in}}{\pgfqpoint{4.620000in}{4.620000in}}%
\pgfusepath{clip}%
\pgfsetbuttcap%
\pgfsetroundjoin%
\definecolor{currentfill}{rgb}{1.000000,0.894118,0.788235}%
\pgfsetfillcolor{currentfill}%
\pgfsetlinewidth{0.000000pt}%
\definecolor{currentstroke}{rgb}{1.000000,0.894118,0.788235}%
\pgfsetstrokecolor{currentstroke}%
\pgfsetdash{}{0pt}%
\pgfpathmoveto{\pgfqpoint{3.337483in}{2.716887in}}%
\pgfpathlineto{\pgfqpoint{3.256524in}{2.670145in}}%
\pgfpathlineto{\pgfqpoint{3.337483in}{2.623404in}}%
\pgfpathlineto{\pgfqpoint{3.418443in}{2.670145in}}%
\pgfpathlineto{\pgfqpoint{3.337483in}{2.716887in}}%
\pgfpathclose%
\pgfusepath{fill}%
\end{pgfscope}%
\begin{pgfscope}%
\pgfpathrectangle{\pgfqpoint{0.765000in}{0.660000in}}{\pgfqpoint{4.620000in}{4.620000in}}%
\pgfusepath{clip}%
\pgfsetbuttcap%
\pgfsetroundjoin%
\definecolor{currentfill}{rgb}{1.000000,0.894118,0.788235}%
\pgfsetfillcolor{currentfill}%
\pgfsetlinewidth{0.000000pt}%
\definecolor{currentstroke}{rgb}{1.000000,0.894118,0.788235}%
\pgfsetstrokecolor{currentstroke}%
\pgfsetdash{}{0pt}%
\pgfpathmoveto{\pgfqpoint{3.256524in}{2.576662in}}%
\pgfpathlineto{\pgfqpoint{3.337483in}{2.529920in}}%
\pgfpathlineto{\pgfqpoint{3.337483in}{2.623404in}}%
\pgfpathlineto{\pgfqpoint{3.256524in}{2.670145in}}%
\pgfpathlineto{\pgfqpoint{3.256524in}{2.576662in}}%
\pgfpathclose%
\pgfusepath{fill}%
\end{pgfscope}%
\begin{pgfscope}%
\pgfpathrectangle{\pgfqpoint{0.765000in}{0.660000in}}{\pgfqpoint{4.620000in}{4.620000in}}%
\pgfusepath{clip}%
\pgfsetbuttcap%
\pgfsetroundjoin%
\definecolor{currentfill}{rgb}{1.000000,0.894118,0.788235}%
\pgfsetfillcolor{currentfill}%
\pgfsetlinewidth{0.000000pt}%
\definecolor{currentstroke}{rgb}{1.000000,0.894118,0.788235}%
\pgfsetstrokecolor{currentstroke}%
\pgfsetdash{}{0pt}%
\pgfpathmoveto{\pgfqpoint{3.593460in}{3.160252in}}%
\pgfpathlineto{\pgfqpoint{3.512500in}{3.113510in}}%
\pgfpathlineto{\pgfqpoint{3.593460in}{3.066768in}}%
\pgfpathlineto{\pgfqpoint{3.674419in}{3.113510in}}%
\pgfpathlineto{\pgfqpoint{3.593460in}{3.160252in}}%
\pgfpathclose%
\pgfusepath{fill}%
\end{pgfscope}%
\begin{pgfscope}%
\pgfpathrectangle{\pgfqpoint{0.765000in}{0.660000in}}{\pgfqpoint{4.620000in}{4.620000in}}%
\pgfusepath{clip}%
\pgfsetbuttcap%
\pgfsetroundjoin%
\definecolor{currentfill}{rgb}{1.000000,0.894118,0.788235}%
\pgfsetfillcolor{currentfill}%
\pgfsetlinewidth{0.000000pt}%
\definecolor{currentstroke}{rgb}{1.000000,0.894118,0.788235}%
\pgfsetstrokecolor{currentstroke}%
\pgfsetdash{}{0pt}%
\pgfpathmoveto{\pgfqpoint{3.512500in}{3.020026in}}%
\pgfpathlineto{\pgfqpoint{3.593460in}{2.973284in}}%
\pgfpathlineto{\pgfqpoint{3.593460in}{3.066768in}}%
\pgfpathlineto{\pgfqpoint{3.512500in}{3.113510in}}%
\pgfpathlineto{\pgfqpoint{3.512500in}{3.020026in}}%
\pgfpathclose%
\pgfusepath{fill}%
\end{pgfscope}%
\begin{pgfscope}%
\pgfpathrectangle{\pgfqpoint{0.765000in}{0.660000in}}{\pgfqpoint{4.620000in}{4.620000in}}%
\pgfusepath{clip}%
\pgfsetbuttcap%
\pgfsetroundjoin%
\definecolor{currentfill}{rgb}{1.000000,0.894118,0.788235}%
\pgfsetfillcolor{currentfill}%
\pgfsetlinewidth{0.000000pt}%
\definecolor{currentstroke}{rgb}{1.000000,0.894118,0.788235}%
\pgfsetstrokecolor{currentstroke}%
\pgfsetdash{}{0pt}%
\pgfpathmoveto{\pgfqpoint{3.337483in}{2.623404in}}%
\pgfpathlineto{\pgfqpoint{3.256524in}{2.576662in}}%
\pgfpathlineto{\pgfqpoint{3.512500in}{3.020026in}}%
\pgfpathlineto{\pgfqpoint{3.593460in}{3.066768in}}%
\pgfpathlineto{\pgfqpoint{3.337483in}{2.623404in}}%
\pgfpathclose%
\pgfusepath{fill}%
\end{pgfscope}%
\begin{pgfscope}%
\pgfpathrectangle{\pgfqpoint{0.765000in}{0.660000in}}{\pgfqpoint{4.620000in}{4.620000in}}%
\pgfusepath{clip}%
\pgfsetbuttcap%
\pgfsetroundjoin%
\definecolor{currentfill}{rgb}{1.000000,0.894118,0.788235}%
\pgfsetfillcolor{currentfill}%
\pgfsetlinewidth{0.000000pt}%
\definecolor{currentstroke}{rgb}{1.000000,0.894118,0.788235}%
\pgfsetstrokecolor{currentstroke}%
\pgfsetdash{}{0pt}%
\pgfpathmoveto{\pgfqpoint{3.418443in}{2.576662in}}%
\pgfpathlineto{\pgfqpoint{3.337483in}{2.623404in}}%
\pgfpathlineto{\pgfqpoint{3.593460in}{3.066768in}}%
\pgfpathlineto{\pgfqpoint{3.674419in}{3.020026in}}%
\pgfpathlineto{\pgfqpoint{3.418443in}{2.576662in}}%
\pgfpathclose%
\pgfusepath{fill}%
\end{pgfscope}%
\begin{pgfscope}%
\pgfpathrectangle{\pgfqpoint{0.765000in}{0.660000in}}{\pgfqpoint{4.620000in}{4.620000in}}%
\pgfusepath{clip}%
\pgfsetbuttcap%
\pgfsetroundjoin%
\definecolor{currentfill}{rgb}{1.000000,0.894118,0.788235}%
\pgfsetfillcolor{currentfill}%
\pgfsetlinewidth{0.000000pt}%
\definecolor{currentstroke}{rgb}{1.000000,0.894118,0.788235}%
\pgfsetstrokecolor{currentstroke}%
\pgfsetdash{}{0pt}%
\pgfpathmoveto{\pgfqpoint{3.337483in}{2.623404in}}%
\pgfpathlineto{\pgfqpoint{3.337483in}{2.716887in}}%
\pgfpathlineto{\pgfqpoint{3.593460in}{3.160252in}}%
\pgfpathlineto{\pgfqpoint{3.674419in}{3.020026in}}%
\pgfpathlineto{\pgfqpoint{3.337483in}{2.623404in}}%
\pgfpathclose%
\pgfusepath{fill}%
\end{pgfscope}%
\begin{pgfscope}%
\pgfpathrectangle{\pgfqpoint{0.765000in}{0.660000in}}{\pgfqpoint{4.620000in}{4.620000in}}%
\pgfusepath{clip}%
\pgfsetbuttcap%
\pgfsetroundjoin%
\definecolor{currentfill}{rgb}{1.000000,0.894118,0.788235}%
\pgfsetfillcolor{currentfill}%
\pgfsetlinewidth{0.000000pt}%
\definecolor{currentstroke}{rgb}{1.000000,0.894118,0.788235}%
\pgfsetstrokecolor{currentstroke}%
\pgfsetdash{}{0pt}%
\pgfpathmoveto{\pgfqpoint{3.418443in}{2.576662in}}%
\pgfpathlineto{\pgfqpoint{3.418443in}{2.670145in}}%
\pgfpathlineto{\pgfqpoint{3.674419in}{3.113510in}}%
\pgfpathlineto{\pgfqpoint{3.593460in}{3.066768in}}%
\pgfpathlineto{\pgfqpoint{3.418443in}{2.576662in}}%
\pgfpathclose%
\pgfusepath{fill}%
\end{pgfscope}%
\begin{pgfscope}%
\pgfpathrectangle{\pgfqpoint{0.765000in}{0.660000in}}{\pgfqpoint{4.620000in}{4.620000in}}%
\pgfusepath{clip}%
\pgfsetbuttcap%
\pgfsetroundjoin%
\definecolor{currentfill}{rgb}{1.000000,0.894118,0.788235}%
\pgfsetfillcolor{currentfill}%
\pgfsetlinewidth{0.000000pt}%
\definecolor{currentstroke}{rgb}{1.000000,0.894118,0.788235}%
\pgfsetstrokecolor{currentstroke}%
\pgfsetdash{}{0pt}%
\pgfpathmoveto{\pgfqpoint{3.256524in}{2.576662in}}%
\pgfpathlineto{\pgfqpoint{3.337483in}{2.529920in}}%
\pgfpathlineto{\pgfqpoint{3.593460in}{2.973284in}}%
\pgfpathlineto{\pgfqpoint{3.512500in}{3.020026in}}%
\pgfpathlineto{\pgfqpoint{3.256524in}{2.576662in}}%
\pgfpathclose%
\pgfusepath{fill}%
\end{pgfscope}%
\begin{pgfscope}%
\pgfpathrectangle{\pgfqpoint{0.765000in}{0.660000in}}{\pgfqpoint{4.620000in}{4.620000in}}%
\pgfusepath{clip}%
\pgfsetbuttcap%
\pgfsetroundjoin%
\definecolor{currentfill}{rgb}{1.000000,0.894118,0.788235}%
\pgfsetfillcolor{currentfill}%
\pgfsetlinewidth{0.000000pt}%
\definecolor{currentstroke}{rgb}{1.000000,0.894118,0.788235}%
\pgfsetstrokecolor{currentstroke}%
\pgfsetdash{}{0pt}%
\pgfpathmoveto{\pgfqpoint{3.337483in}{2.529920in}}%
\pgfpathlineto{\pgfqpoint{3.418443in}{2.576662in}}%
\pgfpathlineto{\pgfqpoint{3.674419in}{3.020026in}}%
\pgfpathlineto{\pgfqpoint{3.593460in}{2.973284in}}%
\pgfpathlineto{\pgfqpoint{3.337483in}{2.529920in}}%
\pgfpathclose%
\pgfusepath{fill}%
\end{pgfscope}%
\begin{pgfscope}%
\pgfpathrectangle{\pgfqpoint{0.765000in}{0.660000in}}{\pgfqpoint{4.620000in}{4.620000in}}%
\pgfusepath{clip}%
\pgfsetbuttcap%
\pgfsetroundjoin%
\definecolor{currentfill}{rgb}{1.000000,0.894118,0.788235}%
\pgfsetfillcolor{currentfill}%
\pgfsetlinewidth{0.000000pt}%
\definecolor{currentstroke}{rgb}{1.000000,0.894118,0.788235}%
\pgfsetstrokecolor{currentstroke}%
\pgfsetdash{}{0pt}%
\pgfpathmoveto{\pgfqpoint{3.337483in}{2.716887in}}%
\pgfpathlineto{\pgfqpoint{3.256524in}{2.670145in}}%
\pgfpathlineto{\pgfqpoint{3.512500in}{3.113510in}}%
\pgfpathlineto{\pgfqpoint{3.593460in}{3.160252in}}%
\pgfpathlineto{\pgfqpoint{3.337483in}{2.716887in}}%
\pgfpathclose%
\pgfusepath{fill}%
\end{pgfscope}%
\begin{pgfscope}%
\pgfpathrectangle{\pgfqpoint{0.765000in}{0.660000in}}{\pgfqpoint{4.620000in}{4.620000in}}%
\pgfusepath{clip}%
\pgfsetbuttcap%
\pgfsetroundjoin%
\definecolor{currentfill}{rgb}{1.000000,0.894118,0.788235}%
\pgfsetfillcolor{currentfill}%
\pgfsetlinewidth{0.000000pt}%
\definecolor{currentstroke}{rgb}{1.000000,0.894118,0.788235}%
\pgfsetstrokecolor{currentstroke}%
\pgfsetdash{}{0pt}%
\pgfpathmoveto{\pgfqpoint{3.418443in}{2.670145in}}%
\pgfpathlineto{\pgfqpoint{3.337483in}{2.716887in}}%
\pgfpathlineto{\pgfqpoint{3.593460in}{3.160252in}}%
\pgfpathlineto{\pgfqpoint{3.674419in}{3.113510in}}%
\pgfpathlineto{\pgfqpoint{3.418443in}{2.670145in}}%
\pgfpathclose%
\pgfusepath{fill}%
\end{pgfscope}%
\begin{pgfscope}%
\pgfpathrectangle{\pgfqpoint{0.765000in}{0.660000in}}{\pgfqpoint{4.620000in}{4.620000in}}%
\pgfusepath{clip}%
\pgfsetbuttcap%
\pgfsetroundjoin%
\definecolor{currentfill}{rgb}{1.000000,0.894118,0.788235}%
\pgfsetfillcolor{currentfill}%
\pgfsetlinewidth{0.000000pt}%
\definecolor{currentstroke}{rgb}{1.000000,0.894118,0.788235}%
\pgfsetstrokecolor{currentstroke}%
\pgfsetdash{}{0pt}%
\pgfpathmoveto{\pgfqpoint{3.256524in}{2.576662in}}%
\pgfpathlineto{\pgfqpoint{3.256524in}{2.670145in}}%
\pgfpathlineto{\pgfqpoint{3.512500in}{3.113510in}}%
\pgfpathlineto{\pgfqpoint{3.593460in}{2.973284in}}%
\pgfpathlineto{\pgfqpoint{3.256524in}{2.576662in}}%
\pgfpathclose%
\pgfusepath{fill}%
\end{pgfscope}%
\begin{pgfscope}%
\pgfpathrectangle{\pgfqpoint{0.765000in}{0.660000in}}{\pgfqpoint{4.620000in}{4.620000in}}%
\pgfusepath{clip}%
\pgfsetbuttcap%
\pgfsetroundjoin%
\definecolor{currentfill}{rgb}{1.000000,0.894118,0.788235}%
\pgfsetfillcolor{currentfill}%
\pgfsetlinewidth{0.000000pt}%
\definecolor{currentstroke}{rgb}{1.000000,0.894118,0.788235}%
\pgfsetstrokecolor{currentstroke}%
\pgfsetdash{}{0pt}%
\pgfpathmoveto{\pgfqpoint{3.337483in}{2.529920in}}%
\pgfpathlineto{\pgfqpoint{3.337483in}{2.623404in}}%
\pgfpathlineto{\pgfqpoint{3.593460in}{3.066768in}}%
\pgfpathlineto{\pgfqpoint{3.512500in}{3.020026in}}%
\pgfpathlineto{\pgfqpoint{3.337483in}{2.529920in}}%
\pgfpathclose%
\pgfusepath{fill}%
\end{pgfscope}%
\begin{pgfscope}%
\pgfpathrectangle{\pgfqpoint{0.765000in}{0.660000in}}{\pgfqpoint{4.620000in}{4.620000in}}%
\pgfusepath{clip}%
\pgfsetbuttcap%
\pgfsetroundjoin%
\definecolor{currentfill}{rgb}{1.000000,0.894118,0.788235}%
\pgfsetfillcolor{currentfill}%
\pgfsetlinewidth{0.000000pt}%
\definecolor{currentstroke}{rgb}{1.000000,0.894118,0.788235}%
\pgfsetstrokecolor{currentstroke}%
\pgfsetdash{}{0pt}%
\pgfpathmoveto{\pgfqpoint{3.256524in}{2.670145in}}%
\pgfpathlineto{\pgfqpoint{3.337483in}{2.623404in}}%
\pgfpathlineto{\pgfqpoint{3.593460in}{3.066768in}}%
\pgfpathlineto{\pgfqpoint{3.512500in}{3.113510in}}%
\pgfpathlineto{\pgfqpoint{3.256524in}{2.670145in}}%
\pgfpathclose%
\pgfusepath{fill}%
\end{pgfscope}%
\begin{pgfscope}%
\pgfpathrectangle{\pgfqpoint{0.765000in}{0.660000in}}{\pgfqpoint{4.620000in}{4.620000in}}%
\pgfusepath{clip}%
\pgfsetbuttcap%
\pgfsetroundjoin%
\definecolor{currentfill}{rgb}{1.000000,0.894118,0.788235}%
\pgfsetfillcolor{currentfill}%
\pgfsetlinewidth{0.000000pt}%
\definecolor{currentstroke}{rgb}{1.000000,0.894118,0.788235}%
\pgfsetstrokecolor{currentstroke}%
\pgfsetdash{}{0pt}%
\pgfpathmoveto{\pgfqpoint{3.337483in}{2.623404in}}%
\pgfpathlineto{\pgfqpoint{3.418443in}{2.670145in}}%
\pgfpathlineto{\pgfqpoint{3.674419in}{3.113510in}}%
\pgfpathlineto{\pgfqpoint{3.593460in}{3.066768in}}%
\pgfpathlineto{\pgfqpoint{3.337483in}{2.623404in}}%
\pgfpathclose%
\pgfusepath{fill}%
\end{pgfscope}%
\begin{pgfscope}%
\pgfpathrectangle{\pgfqpoint{0.765000in}{0.660000in}}{\pgfqpoint{4.620000in}{4.620000in}}%
\pgfusepath{clip}%
\pgfsetbuttcap%
\pgfsetroundjoin%
\definecolor{currentfill}{rgb}{1.000000,0.894118,0.788235}%
\pgfsetfillcolor{currentfill}%
\pgfsetlinewidth{0.000000pt}%
\definecolor{currentstroke}{rgb}{1.000000,0.894118,0.788235}%
\pgfsetstrokecolor{currentstroke}%
\pgfsetdash{}{0pt}%
\pgfpathmoveto{\pgfqpoint{2.869318in}{3.452500in}}%
\pgfpathlineto{\pgfqpoint{2.788359in}{3.405758in}}%
\pgfpathlineto{\pgfqpoint{2.869318in}{3.359016in}}%
\pgfpathlineto{\pgfqpoint{2.950278in}{3.405758in}}%
\pgfpathlineto{\pgfqpoint{2.869318in}{3.452500in}}%
\pgfpathclose%
\pgfusepath{fill}%
\end{pgfscope}%
\begin{pgfscope}%
\pgfpathrectangle{\pgfqpoint{0.765000in}{0.660000in}}{\pgfqpoint{4.620000in}{4.620000in}}%
\pgfusepath{clip}%
\pgfsetbuttcap%
\pgfsetroundjoin%
\definecolor{currentfill}{rgb}{1.000000,0.894118,0.788235}%
\pgfsetfillcolor{currentfill}%
\pgfsetlinewidth{0.000000pt}%
\definecolor{currentstroke}{rgb}{1.000000,0.894118,0.788235}%
\pgfsetstrokecolor{currentstroke}%
\pgfsetdash{}{0pt}%
\pgfpathmoveto{\pgfqpoint{2.869318in}{3.452500in}}%
\pgfpathlineto{\pgfqpoint{2.788359in}{3.405758in}}%
\pgfpathlineto{\pgfqpoint{2.788359in}{3.499242in}}%
\pgfpathlineto{\pgfqpoint{2.869318in}{3.545983in}}%
\pgfpathlineto{\pgfqpoint{2.869318in}{3.452500in}}%
\pgfpathclose%
\pgfusepath{fill}%
\end{pgfscope}%
\begin{pgfscope}%
\pgfpathrectangle{\pgfqpoint{0.765000in}{0.660000in}}{\pgfqpoint{4.620000in}{4.620000in}}%
\pgfusepath{clip}%
\pgfsetbuttcap%
\pgfsetroundjoin%
\definecolor{currentfill}{rgb}{1.000000,0.894118,0.788235}%
\pgfsetfillcolor{currentfill}%
\pgfsetlinewidth{0.000000pt}%
\definecolor{currentstroke}{rgb}{1.000000,0.894118,0.788235}%
\pgfsetstrokecolor{currentstroke}%
\pgfsetdash{}{0pt}%
\pgfpathmoveto{\pgfqpoint{2.869318in}{3.452500in}}%
\pgfpathlineto{\pgfqpoint{2.950278in}{3.405758in}}%
\pgfpathlineto{\pgfqpoint{2.950278in}{3.499242in}}%
\pgfpathlineto{\pgfqpoint{2.869318in}{3.545983in}}%
\pgfpathlineto{\pgfqpoint{2.869318in}{3.452500in}}%
\pgfpathclose%
\pgfusepath{fill}%
\end{pgfscope}%
\begin{pgfscope}%
\pgfpathrectangle{\pgfqpoint{0.765000in}{0.660000in}}{\pgfqpoint{4.620000in}{4.620000in}}%
\pgfusepath{clip}%
\pgfsetbuttcap%
\pgfsetroundjoin%
\definecolor{currentfill}{rgb}{1.000000,0.894118,0.788235}%
\pgfsetfillcolor{currentfill}%
\pgfsetlinewidth{0.000000pt}%
\definecolor{currentstroke}{rgb}{1.000000,0.894118,0.788235}%
\pgfsetstrokecolor{currentstroke}%
\pgfsetdash{}{0pt}%
\pgfpathmoveto{\pgfqpoint{2.654247in}{3.328328in}}%
\pgfpathlineto{\pgfqpoint{2.573288in}{3.281587in}}%
\pgfpathlineto{\pgfqpoint{2.654247in}{3.234845in}}%
\pgfpathlineto{\pgfqpoint{2.735207in}{3.281587in}}%
\pgfpathlineto{\pgfqpoint{2.654247in}{3.328328in}}%
\pgfpathclose%
\pgfusepath{fill}%
\end{pgfscope}%
\begin{pgfscope}%
\pgfpathrectangle{\pgfqpoint{0.765000in}{0.660000in}}{\pgfqpoint{4.620000in}{4.620000in}}%
\pgfusepath{clip}%
\pgfsetbuttcap%
\pgfsetroundjoin%
\definecolor{currentfill}{rgb}{1.000000,0.894118,0.788235}%
\pgfsetfillcolor{currentfill}%
\pgfsetlinewidth{0.000000pt}%
\definecolor{currentstroke}{rgb}{1.000000,0.894118,0.788235}%
\pgfsetstrokecolor{currentstroke}%
\pgfsetdash{}{0pt}%
\pgfpathmoveto{\pgfqpoint{2.654247in}{3.328328in}}%
\pgfpathlineto{\pgfqpoint{2.573288in}{3.281587in}}%
\pgfpathlineto{\pgfqpoint{2.573288in}{3.375070in}}%
\pgfpathlineto{\pgfqpoint{2.654247in}{3.421812in}}%
\pgfpathlineto{\pgfqpoint{2.654247in}{3.328328in}}%
\pgfpathclose%
\pgfusepath{fill}%
\end{pgfscope}%
\begin{pgfscope}%
\pgfpathrectangle{\pgfqpoint{0.765000in}{0.660000in}}{\pgfqpoint{4.620000in}{4.620000in}}%
\pgfusepath{clip}%
\pgfsetbuttcap%
\pgfsetroundjoin%
\definecolor{currentfill}{rgb}{1.000000,0.894118,0.788235}%
\pgfsetfillcolor{currentfill}%
\pgfsetlinewidth{0.000000pt}%
\definecolor{currentstroke}{rgb}{1.000000,0.894118,0.788235}%
\pgfsetstrokecolor{currentstroke}%
\pgfsetdash{}{0pt}%
\pgfpathmoveto{\pgfqpoint{2.654247in}{3.328328in}}%
\pgfpathlineto{\pgfqpoint{2.735207in}{3.281587in}}%
\pgfpathlineto{\pgfqpoint{2.735207in}{3.375070in}}%
\pgfpathlineto{\pgfqpoint{2.654247in}{3.421812in}}%
\pgfpathlineto{\pgfqpoint{2.654247in}{3.328328in}}%
\pgfpathclose%
\pgfusepath{fill}%
\end{pgfscope}%
\begin{pgfscope}%
\pgfpathrectangle{\pgfqpoint{0.765000in}{0.660000in}}{\pgfqpoint{4.620000in}{4.620000in}}%
\pgfusepath{clip}%
\pgfsetbuttcap%
\pgfsetroundjoin%
\definecolor{currentfill}{rgb}{1.000000,0.894118,0.788235}%
\pgfsetfillcolor{currentfill}%
\pgfsetlinewidth{0.000000pt}%
\definecolor{currentstroke}{rgb}{1.000000,0.894118,0.788235}%
\pgfsetstrokecolor{currentstroke}%
\pgfsetdash{}{0pt}%
\pgfpathmoveto{\pgfqpoint{2.869318in}{3.359016in}}%
\pgfpathlineto{\pgfqpoint{2.950278in}{3.405758in}}%
\pgfpathlineto{\pgfqpoint{2.950278in}{3.499242in}}%
\pgfpathlineto{\pgfqpoint{2.869318in}{3.452500in}}%
\pgfpathlineto{\pgfqpoint{2.869318in}{3.359016in}}%
\pgfpathclose%
\pgfusepath{fill}%
\end{pgfscope}%
\begin{pgfscope}%
\pgfpathrectangle{\pgfqpoint{0.765000in}{0.660000in}}{\pgfqpoint{4.620000in}{4.620000in}}%
\pgfusepath{clip}%
\pgfsetbuttcap%
\pgfsetroundjoin%
\definecolor{currentfill}{rgb}{1.000000,0.894118,0.788235}%
\pgfsetfillcolor{currentfill}%
\pgfsetlinewidth{0.000000pt}%
\definecolor{currentstroke}{rgb}{1.000000,0.894118,0.788235}%
\pgfsetstrokecolor{currentstroke}%
\pgfsetdash{}{0pt}%
\pgfpathmoveto{\pgfqpoint{2.654247in}{3.234845in}}%
\pgfpathlineto{\pgfqpoint{2.735207in}{3.281587in}}%
\pgfpathlineto{\pgfqpoint{2.735207in}{3.375070in}}%
\pgfpathlineto{\pgfqpoint{2.654247in}{3.328328in}}%
\pgfpathlineto{\pgfqpoint{2.654247in}{3.234845in}}%
\pgfpathclose%
\pgfusepath{fill}%
\end{pgfscope}%
\begin{pgfscope}%
\pgfpathrectangle{\pgfqpoint{0.765000in}{0.660000in}}{\pgfqpoint{4.620000in}{4.620000in}}%
\pgfusepath{clip}%
\pgfsetbuttcap%
\pgfsetroundjoin%
\definecolor{currentfill}{rgb}{1.000000,0.894118,0.788235}%
\pgfsetfillcolor{currentfill}%
\pgfsetlinewidth{0.000000pt}%
\definecolor{currentstroke}{rgb}{1.000000,0.894118,0.788235}%
\pgfsetstrokecolor{currentstroke}%
\pgfsetdash{}{0pt}%
\pgfpathmoveto{\pgfqpoint{2.869318in}{3.545983in}}%
\pgfpathlineto{\pgfqpoint{2.788359in}{3.499242in}}%
\pgfpathlineto{\pgfqpoint{2.869318in}{3.452500in}}%
\pgfpathlineto{\pgfqpoint{2.950278in}{3.499242in}}%
\pgfpathlineto{\pgfqpoint{2.869318in}{3.545983in}}%
\pgfpathclose%
\pgfusepath{fill}%
\end{pgfscope}%
\begin{pgfscope}%
\pgfpathrectangle{\pgfqpoint{0.765000in}{0.660000in}}{\pgfqpoint{4.620000in}{4.620000in}}%
\pgfusepath{clip}%
\pgfsetbuttcap%
\pgfsetroundjoin%
\definecolor{currentfill}{rgb}{1.000000,0.894118,0.788235}%
\pgfsetfillcolor{currentfill}%
\pgfsetlinewidth{0.000000pt}%
\definecolor{currentstroke}{rgb}{1.000000,0.894118,0.788235}%
\pgfsetstrokecolor{currentstroke}%
\pgfsetdash{}{0pt}%
\pgfpathmoveto{\pgfqpoint{2.788359in}{3.405758in}}%
\pgfpathlineto{\pgfqpoint{2.869318in}{3.359016in}}%
\pgfpathlineto{\pgfqpoint{2.869318in}{3.452500in}}%
\pgfpathlineto{\pgfqpoint{2.788359in}{3.499242in}}%
\pgfpathlineto{\pgfqpoint{2.788359in}{3.405758in}}%
\pgfpathclose%
\pgfusepath{fill}%
\end{pgfscope}%
\begin{pgfscope}%
\pgfpathrectangle{\pgfqpoint{0.765000in}{0.660000in}}{\pgfqpoint{4.620000in}{4.620000in}}%
\pgfusepath{clip}%
\pgfsetbuttcap%
\pgfsetroundjoin%
\definecolor{currentfill}{rgb}{1.000000,0.894118,0.788235}%
\pgfsetfillcolor{currentfill}%
\pgfsetlinewidth{0.000000pt}%
\definecolor{currentstroke}{rgb}{1.000000,0.894118,0.788235}%
\pgfsetstrokecolor{currentstroke}%
\pgfsetdash{}{0pt}%
\pgfpathmoveto{\pgfqpoint{2.654247in}{3.421812in}}%
\pgfpathlineto{\pgfqpoint{2.573288in}{3.375070in}}%
\pgfpathlineto{\pgfqpoint{2.654247in}{3.328328in}}%
\pgfpathlineto{\pgfqpoint{2.735207in}{3.375070in}}%
\pgfpathlineto{\pgfqpoint{2.654247in}{3.421812in}}%
\pgfpathclose%
\pgfusepath{fill}%
\end{pgfscope}%
\begin{pgfscope}%
\pgfpathrectangle{\pgfqpoint{0.765000in}{0.660000in}}{\pgfqpoint{4.620000in}{4.620000in}}%
\pgfusepath{clip}%
\pgfsetbuttcap%
\pgfsetroundjoin%
\definecolor{currentfill}{rgb}{1.000000,0.894118,0.788235}%
\pgfsetfillcolor{currentfill}%
\pgfsetlinewidth{0.000000pt}%
\definecolor{currentstroke}{rgb}{1.000000,0.894118,0.788235}%
\pgfsetstrokecolor{currentstroke}%
\pgfsetdash{}{0pt}%
\pgfpathmoveto{\pgfqpoint{2.573288in}{3.281587in}}%
\pgfpathlineto{\pgfqpoint{2.654247in}{3.234845in}}%
\pgfpathlineto{\pgfqpoint{2.654247in}{3.328328in}}%
\pgfpathlineto{\pgfqpoint{2.573288in}{3.375070in}}%
\pgfpathlineto{\pgfqpoint{2.573288in}{3.281587in}}%
\pgfpathclose%
\pgfusepath{fill}%
\end{pgfscope}%
\begin{pgfscope}%
\pgfpathrectangle{\pgfqpoint{0.765000in}{0.660000in}}{\pgfqpoint{4.620000in}{4.620000in}}%
\pgfusepath{clip}%
\pgfsetbuttcap%
\pgfsetroundjoin%
\definecolor{currentfill}{rgb}{1.000000,0.894118,0.788235}%
\pgfsetfillcolor{currentfill}%
\pgfsetlinewidth{0.000000pt}%
\definecolor{currentstroke}{rgb}{1.000000,0.894118,0.788235}%
\pgfsetstrokecolor{currentstroke}%
\pgfsetdash{}{0pt}%
\pgfpathmoveto{\pgfqpoint{2.869318in}{3.452500in}}%
\pgfpathlineto{\pgfqpoint{2.788359in}{3.405758in}}%
\pgfpathlineto{\pgfqpoint{2.573288in}{3.281587in}}%
\pgfpathlineto{\pgfqpoint{2.654247in}{3.328328in}}%
\pgfpathlineto{\pgfqpoint{2.869318in}{3.452500in}}%
\pgfpathclose%
\pgfusepath{fill}%
\end{pgfscope}%
\begin{pgfscope}%
\pgfpathrectangle{\pgfqpoint{0.765000in}{0.660000in}}{\pgfqpoint{4.620000in}{4.620000in}}%
\pgfusepath{clip}%
\pgfsetbuttcap%
\pgfsetroundjoin%
\definecolor{currentfill}{rgb}{1.000000,0.894118,0.788235}%
\pgfsetfillcolor{currentfill}%
\pgfsetlinewidth{0.000000pt}%
\definecolor{currentstroke}{rgb}{1.000000,0.894118,0.788235}%
\pgfsetstrokecolor{currentstroke}%
\pgfsetdash{}{0pt}%
\pgfpathmoveto{\pgfqpoint{2.950278in}{3.405758in}}%
\pgfpathlineto{\pgfqpoint{2.869318in}{3.452500in}}%
\pgfpathlineto{\pgfqpoint{2.654247in}{3.328328in}}%
\pgfpathlineto{\pgfqpoint{2.735207in}{3.281587in}}%
\pgfpathlineto{\pgfqpoint{2.950278in}{3.405758in}}%
\pgfpathclose%
\pgfusepath{fill}%
\end{pgfscope}%
\begin{pgfscope}%
\pgfpathrectangle{\pgfqpoint{0.765000in}{0.660000in}}{\pgfqpoint{4.620000in}{4.620000in}}%
\pgfusepath{clip}%
\pgfsetbuttcap%
\pgfsetroundjoin%
\definecolor{currentfill}{rgb}{1.000000,0.894118,0.788235}%
\pgfsetfillcolor{currentfill}%
\pgfsetlinewidth{0.000000pt}%
\definecolor{currentstroke}{rgb}{1.000000,0.894118,0.788235}%
\pgfsetstrokecolor{currentstroke}%
\pgfsetdash{}{0pt}%
\pgfpathmoveto{\pgfqpoint{2.869318in}{3.452500in}}%
\pgfpathlineto{\pgfqpoint{2.869318in}{3.545983in}}%
\pgfpathlineto{\pgfqpoint{2.654247in}{3.421812in}}%
\pgfpathlineto{\pgfqpoint{2.735207in}{3.281587in}}%
\pgfpathlineto{\pgfqpoint{2.869318in}{3.452500in}}%
\pgfpathclose%
\pgfusepath{fill}%
\end{pgfscope}%
\begin{pgfscope}%
\pgfpathrectangle{\pgfqpoint{0.765000in}{0.660000in}}{\pgfqpoint{4.620000in}{4.620000in}}%
\pgfusepath{clip}%
\pgfsetbuttcap%
\pgfsetroundjoin%
\definecolor{currentfill}{rgb}{1.000000,0.894118,0.788235}%
\pgfsetfillcolor{currentfill}%
\pgfsetlinewidth{0.000000pt}%
\definecolor{currentstroke}{rgb}{1.000000,0.894118,0.788235}%
\pgfsetstrokecolor{currentstroke}%
\pgfsetdash{}{0pt}%
\pgfpathmoveto{\pgfqpoint{2.950278in}{3.405758in}}%
\pgfpathlineto{\pgfqpoint{2.950278in}{3.499242in}}%
\pgfpathlineto{\pgfqpoint{2.735207in}{3.375070in}}%
\pgfpathlineto{\pgfqpoint{2.654247in}{3.328328in}}%
\pgfpathlineto{\pgfqpoint{2.950278in}{3.405758in}}%
\pgfpathclose%
\pgfusepath{fill}%
\end{pgfscope}%
\begin{pgfscope}%
\pgfpathrectangle{\pgfqpoint{0.765000in}{0.660000in}}{\pgfqpoint{4.620000in}{4.620000in}}%
\pgfusepath{clip}%
\pgfsetbuttcap%
\pgfsetroundjoin%
\definecolor{currentfill}{rgb}{1.000000,0.894118,0.788235}%
\pgfsetfillcolor{currentfill}%
\pgfsetlinewidth{0.000000pt}%
\definecolor{currentstroke}{rgb}{1.000000,0.894118,0.788235}%
\pgfsetstrokecolor{currentstroke}%
\pgfsetdash{}{0pt}%
\pgfpathmoveto{\pgfqpoint{2.788359in}{3.405758in}}%
\pgfpathlineto{\pgfqpoint{2.869318in}{3.359016in}}%
\pgfpathlineto{\pgfqpoint{2.654247in}{3.234845in}}%
\pgfpathlineto{\pgfqpoint{2.573288in}{3.281587in}}%
\pgfpathlineto{\pgfqpoint{2.788359in}{3.405758in}}%
\pgfpathclose%
\pgfusepath{fill}%
\end{pgfscope}%
\begin{pgfscope}%
\pgfpathrectangle{\pgfqpoint{0.765000in}{0.660000in}}{\pgfqpoint{4.620000in}{4.620000in}}%
\pgfusepath{clip}%
\pgfsetbuttcap%
\pgfsetroundjoin%
\definecolor{currentfill}{rgb}{1.000000,0.894118,0.788235}%
\pgfsetfillcolor{currentfill}%
\pgfsetlinewidth{0.000000pt}%
\definecolor{currentstroke}{rgb}{1.000000,0.894118,0.788235}%
\pgfsetstrokecolor{currentstroke}%
\pgfsetdash{}{0pt}%
\pgfpathmoveto{\pgfqpoint{2.869318in}{3.359016in}}%
\pgfpathlineto{\pgfqpoint{2.950278in}{3.405758in}}%
\pgfpathlineto{\pgfqpoint{2.735207in}{3.281587in}}%
\pgfpathlineto{\pgfqpoint{2.654247in}{3.234845in}}%
\pgfpathlineto{\pgfqpoint{2.869318in}{3.359016in}}%
\pgfpathclose%
\pgfusepath{fill}%
\end{pgfscope}%
\begin{pgfscope}%
\pgfpathrectangle{\pgfqpoint{0.765000in}{0.660000in}}{\pgfqpoint{4.620000in}{4.620000in}}%
\pgfusepath{clip}%
\pgfsetbuttcap%
\pgfsetroundjoin%
\definecolor{currentfill}{rgb}{1.000000,0.894118,0.788235}%
\pgfsetfillcolor{currentfill}%
\pgfsetlinewidth{0.000000pt}%
\definecolor{currentstroke}{rgb}{1.000000,0.894118,0.788235}%
\pgfsetstrokecolor{currentstroke}%
\pgfsetdash{}{0pt}%
\pgfpathmoveto{\pgfqpoint{2.869318in}{3.545983in}}%
\pgfpathlineto{\pgfqpoint{2.788359in}{3.499242in}}%
\pgfpathlineto{\pgfqpoint{2.573288in}{3.375070in}}%
\pgfpathlineto{\pgfqpoint{2.654247in}{3.421812in}}%
\pgfpathlineto{\pgfqpoint{2.869318in}{3.545983in}}%
\pgfpathclose%
\pgfusepath{fill}%
\end{pgfscope}%
\begin{pgfscope}%
\pgfpathrectangle{\pgfqpoint{0.765000in}{0.660000in}}{\pgfqpoint{4.620000in}{4.620000in}}%
\pgfusepath{clip}%
\pgfsetbuttcap%
\pgfsetroundjoin%
\definecolor{currentfill}{rgb}{1.000000,0.894118,0.788235}%
\pgfsetfillcolor{currentfill}%
\pgfsetlinewidth{0.000000pt}%
\definecolor{currentstroke}{rgb}{1.000000,0.894118,0.788235}%
\pgfsetstrokecolor{currentstroke}%
\pgfsetdash{}{0pt}%
\pgfpathmoveto{\pgfqpoint{2.950278in}{3.499242in}}%
\pgfpathlineto{\pgfqpoint{2.869318in}{3.545983in}}%
\pgfpathlineto{\pgfqpoint{2.654247in}{3.421812in}}%
\pgfpathlineto{\pgfqpoint{2.735207in}{3.375070in}}%
\pgfpathlineto{\pgfqpoint{2.950278in}{3.499242in}}%
\pgfpathclose%
\pgfusepath{fill}%
\end{pgfscope}%
\begin{pgfscope}%
\pgfpathrectangle{\pgfqpoint{0.765000in}{0.660000in}}{\pgfqpoint{4.620000in}{4.620000in}}%
\pgfusepath{clip}%
\pgfsetbuttcap%
\pgfsetroundjoin%
\definecolor{currentfill}{rgb}{1.000000,0.894118,0.788235}%
\pgfsetfillcolor{currentfill}%
\pgfsetlinewidth{0.000000pt}%
\definecolor{currentstroke}{rgb}{1.000000,0.894118,0.788235}%
\pgfsetstrokecolor{currentstroke}%
\pgfsetdash{}{0pt}%
\pgfpathmoveto{\pgfqpoint{2.788359in}{3.405758in}}%
\pgfpathlineto{\pgfqpoint{2.788359in}{3.499242in}}%
\pgfpathlineto{\pgfqpoint{2.573288in}{3.375070in}}%
\pgfpathlineto{\pgfqpoint{2.654247in}{3.234845in}}%
\pgfpathlineto{\pgfqpoint{2.788359in}{3.405758in}}%
\pgfpathclose%
\pgfusepath{fill}%
\end{pgfscope}%
\begin{pgfscope}%
\pgfpathrectangle{\pgfqpoint{0.765000in}{0.660000in}}{\pgfqpoint{4.620000in}{4.620000in}}%
\pgfusepath{clip}%
\pgfsetbuttcap%
\pgfsetroundjoin%
\definecolor{currentfill}{rgb}{1.000000,0.894118,0.788235}%
\pgfsetfillcolor{currentfill}%
\pgfsetlinewidth{0.000000pt}%
\definecolor{currentstroke}{rgb}{1.000000,0.894118,0.788235}%
\pgfsetstrokecolor{currentstroke}%
\pgfsetdash{}{0pt}%
\pgfpathmoveto{\pgfqpoint{2.869318in}{3.359016in}}%
\pgfpathlineto{\pgfqpoint{2.869318in}{3.452500in}}%
\pgfpathlineto{\pgfqpoint{2.654247in}{3.328328in}}%
\pgfpathlineto{\pgfqpoint{2.573288in}{3.281587in}}%
\pgfpathlineto{\pgfqpoint{2.869318in}{3.359016in}}%
\pgfpathclose%
\pgfusepath{fill}%
\end{pgfscope}%
\begin{pgfscope}%
\pgfpathrectangle{\pgfqpoint{0.765000in}{0.660000in}}{\pgfqpoint{4.620000in}{4.620000in}}%
\pgfusepath{clip}%
\pgfsetbuttcap%
\pgfsetroundjoin%
\definecolor{currentfill}{rgb}{1.000000,0.894118,0.788235}%
\pgfsetfillcolor{currentfill}%
\pgfsetlinewidth{0.000000pt}%
\definecolor{currentstroke}{rgb}{1.000000,0.894118,0.788235}%
\pgfsetstrokecolor{currentstroke}%
\pgfsetdash{}{0pt}%
\pgfpathmoveto{\pgfqpoint{2.788359in}{3.499242in}}%
\pgfpathlineto{\pgfqpoint{2.869318in}{3.452500in}}%
\pgfpathlineto{\pgfqpoint{2.654247in}{3.328328in}}%
\pgfpathlineto{\pgfqpoint{2.573288in}{3.375070in}}%
\pgfpathlineto{\pgfqpoint{2.788359in}{3.499242in}}%
\pgfpathclose%
\pgfusepath{fill}%
\end{pgfscope}%
\begin{pgfscope}%
\pgfpathrectangle{\pgfqpoint{0.765000in}{0.660000in}}{\pgfqpoint{4.620000in}{4.620000in}}%
\pgfusepath{clip}%
\pgfsetbuttcap%
\pgfsetroundjoin%
\definecolor{currentfill}{rgb}{1.000000,0.894118,0.788235}%
\pgfsetfillcolor{currentfill}%
\pgfsetlinewidth{0.000000pt}%
\definecolor{currentstroke}{rgb}{1.000000,0.894118,0.788235}%
\pgfsetstrokecolor{currentstroke}%
\pgfsetdash{}{0pt}%
\pgfpathmoveto{\pgfqpoint{2.869318in}{3.452500in}}%
\pgfpathlineto{\pgfqpoint{2.950278in}{3.499242in}}%
\pgfpathlineto{\pgfqpoint{2.735207in}{3.375070in}}%
\pgfpathlineto{\pgfqpoint{2.654247in}{3.328328in}}%
\pgfpathlineto{\pgfqpoint{2.869318in}{3.452500in}}%
\pgfpathclose%
\pgfusepath{fill}%
\end{pgfscope}%
\begin{pgfscope}%
\pgfpathrectangle{\pgfqpoint{0.765000in}{0.660000in}}{\pgfqpoint{4.620000in}{4.620000in}}%
\pgfusepath{clip}%
\pgfsetbuttcap%
\pgfsetroundjoin%
\definecolor{currentfill}{rgb}{1.000000,0.894118,0.788235}%
\pgfsetfillcolor{currentfill}%
\pgfsetlinewidth{0.000000pt}%
\definecolor{currentstroke}{rgb}{1.000000,0.894118,0.788235}%
\pgfsetstrokecolor{currentstroke}%
\pgfsetdash{}{0pt}%
\pgfpathmoveto{\pgfqpoint{2.869318in}{3.452500in}}%
\pgfpathlineto{\pgfqpoint{2.788359in}{3.405758in}}%
\pgfpathlineto{\pgfqpoint{2.869318in}{3.359016in}}%
\pgfpathlineto{\pgfqpoint{2.950278in}{3.405758in}}%
\pgfpathlineto{\pgfqpoint{2.869318in}{3.452500in}}%
\pgfpathclose%
\pgfusepath{fill}%
\end{pgfscope}%
\begin{pgfscope}%
\pgfpathrectangle{\pgfqpoint{0.765000in}{0.660000in}}{\pgfqpoint{4.620000in}{4.620000in}}%
\pgfusepath{clip}%
\pgfsetbuttcap%
\pgfsetroundjoin%
\definecolor{currentfill}{rgb}{1.000000,0.894118,0.788235}%
\pgfsetfillcolor{currentfill}%
\pgfsetlinewidth{0.000000pt}%
\definecolor{currentstroke}{rgb}{1.000000,0.894118,0.788235}%
\pgfsetstrokecolor{currentstroke}%
\pgfsetdash{}{0pt}%
\pgfpathmoveto{\pgfqpoint{2.869318in}{3.452500in}}%
\pgfpathlineto{\pgfqpoint{2.788359in}{3.405758in}}%
\pgfpathlineto{\pgfqpoint{2.788359in}{3.499242in}}%
\pgfpathlineto{\pgfqpoint{2.869318in}{3.545983in}}%
\pgfpathlineto{\pgfqpoint{2.869318in}{3.452500in}}%
\pgfpathclose%
\pgfusepath{fill}%
\end{pgfscope}%
\begin{pgfscope}%
\pgfpathrectangle{\pgfqpoint{0.765000in}{0.660000in}}{\pgfqpoint{4.620000in}{4.620000in}}%
\pgfusepath{clip}%
\pgfsetbuttcap%
\pgfsetroundjoin%
\definecolor{currentfill}{rgb}{1.000000,0.894118,0.788235}%
\pgfsetfillcolor{currentfill}%
\pgfsetlinewidth{0.000000pt}%
\definecolor{currentstroke}{rgb}{1.000000,0.894118,0.788235}%
\pgfsetstrokecolor{currentstroke}%
\pgfsetdash{}{0pt}%
\pgfpathmoveto{\pgfqpoint{2.869318in}{3.452500in}}%
\pgfpathlineto{\pgfqpoint{2.950278in}{3.405758in}}%
\pgfpathlineto{\pgfqpoint{2.950278in}{3.499242in}}%
\pgfpathlineto{\pgfqpoint{2.869318in}{3.545983in}}%
\pgfpathlineto{\pgfqpoint{2.869318in}{3.452500in}}%
\pgfpathclose%
\pgfusepath{fill}%
\end{pgfscope}%
\begin{pgfscope}%
\pgfpathrectangle{\pgfqpoint{0.765000in}{0.660000in}}{\pgfqpoint{4.620000in}{4.620000in}}%
\pgfusepath{clip}%
\pgfsetbuttcap%
\pgfsetroundjoin%
\definecolor{currentfill}{rgb}{1.000000,0.894118,0.788235}%
\pgfsetfillcolor{currentfill}%
\pgfsetlinewidth{0.000000pt}%
\definecolor{currentstroke}{rgb}{1.000000,0.894118,0.788235}%
\pgfsetstrokecolor{currentstroke}%
\pgfsetdash{}{0pt}%
\pgfpathmoveto{\pgfqpoint{3.593460in}{3.452500in}}%
\pgfpathlineto{\pgfqpoint{3.512500in}{3.405758in}}%
\pgfpathlineto{\pgfqpoint{3.593460in}{3.359016in}}%
\pgfpathlineto{\pgfqpoint{3.674419in}{3.405758in}}%
\pgfpathlineto{\pgfqpoint{3.593460in}{3.452500in}}%
\pgfpathclose%
\pgfusepath{fill}%
\end{pgfscope}%
\begin{pgfscope}%
\pgfpathrectangle{\pgfqpoint{0.765000in}{0.660000in}}{\pgfqpoint{4.620000in}{4.620000in}}%
\pgfusepath{clip}%
\pgfsetbuttcap%
\pgfsetroundjoin%
\definecolor{currentfill}{rgb}{1.000000,0.894118,0.788235}%
\pgfsetfillcolor{currentfill}%
\pgfsetlinewidth{0.000000pt}%
\definecolor{currentstroke}{rgb}{1.000000,0.894118,0.788235}%
\pgfsetstrokecolor{currentstroke}%
\pgfsetdash{}{0pt}%
\pgfpathmoveto{\pgfqpoint{3.593460in}{3.452500in}}%
\pgfpathlineto{\pgfqpoint{3.512500in}{3.405758in}}%
\pgfpathlineto{\pgfqpoint{3.512500in}{3.499242in}}%
\pgfpathlineto{\pgfqpoint{3.593460in}{3.545983in}}%
\pgfpathlineto{\pgfqpoint{3.593460in}{3.452500in}}%
\pgfpathclose%
\pgfusepath{fill}%
\end{pgfscope}%
\begin{pgfscope}%
\pgfpathrectangle{\pgfqpoint{0.765000in}{0.660000in}}{\pgfqpoint{4.620000in}{4.620000in}}%
\pgfusepath{clip}%
\pgfsetbuttcap%
\pgfsetroundjoin%
\definecolor{currentfill}{rgb}{1.000000,0.894118,0.788235}%
\pgfsetfillcolor{currentfill}%
\pgfsetlinewidth{0.000000pt}%
\definecolor{currentstroke}{rgb}{1.000000,0.894118,0.788235}%
\pgfsetstrokecolor{currentstroke}%
\pgfsetdash{}{0pt}%
\pgfpathmoveto{\pgfqpoint{3.593460in}{3.452500in}}%
\pgfpathlineto{\pgfqpoint{3.674419in}{3.405758in}}%
\pgfpathlineto{\pgfqpoint{3.674419in}{3.499242in}}%
\pgfpathlineto{\pgfqpoint{3.593460in}{3.545983in}}%
\pgfpathlineto{\pgfqpoint{3.593460in}{3.452500in}}%
\pgfpathclose%
\pgfusepath{fill}%
\end{pgfscope}%
\begin{pgfscope}%
\pgfpathrectangle{\pgfqpoint{0.765000in}{0.660000in}}{\pgfqpoint{4.620000in}{4.620000in}}%
\pgfusepath{clip}%
\pgfsetbuttcap%
\pgfsetroundjoin%
\definecolor{currentfill}{rgb}{1.000000,0.894118,0.788235}%
\pgfsetfillcolor{currentfill}%
\pgfsetlinewidth{0.000000pt}%
\definecolor{currentstroke}{rgb}{1.000000,0.894118,0.788235}%
\pgfsetstrokecolor{currentstroke}%
\pgfsetdash{}{0pt}%
\pgfpathmoveto{\pgfqpoint{2.869318in}{3.359016in}}%
\pgfpathlineto{\pgfqpoint{2.950278in}{3.405758in}}%
\pgfpathlineto{\pgfqpoint{2.950278in}{3.499242in}}%
\pgfpathlineto{\pgfqpoint{2.869318in}{3.452500in}}%
\pgfpathlineto{\pgfqpoint{2.869318in}{3.359016in}}%
\pgfpathclose%
\pgfusepath{fill}%
\end{pgfscope}%
\begin{pgfscope}%
\pgfpathrectangle{\pgfqpoint{0.765000in}{0.660000in}}{\pgfqpoint{4.620000in}{4.620000in}}%
\pgfusepath{clip}%
\pgfsetbuttcap%
\pgfsetroundjoin%
\definecolor{currentfill}{rgb}{1.000000,0.894118,0.788235}%
\pgfsetfillcolor{currentfill}%
\pgfsetlinewidth{0.000000pt}%
\definecolor{currentstroke}{rgb}{1.000000,0.894118,0.788235}%
\pgfsetstrokecolor{currentstroke}%
\pgfsetdash{}{0pt}%
\pgfpathmoveto{\pgfqpoint{3.593460in}{3.359016in}}%
\pgfpathlineto{\pgfqpoint{3.674419in}{3.405758in}}%
\pgfpathlineto{\pgfqpoint{3.674419in}{3.499242in}}%
\pgfpathlineto{\pgfqpoint{3.593460in}{3.452500in}}%
\pgfpathlineto{\pgfqpoint{3.593460in}{3.359016in}}%
\pgfpathclose%
\pgfusepath{fill}%
\end{pgfscope}%
\begin{pgfscope}%
\pgfpathrectangle{\pgfqpoint{0.765000in}{0.660000in}}{\pgfqpoint{4.620000in}{4.620000in}}%
\pgfusepath{clip}%
\pgfsetbuttcap%
\pgfsetroundjoin%
\definecolor{currentfill}{rgb}{1.000000,0.894118,0.788235}%
\pgfsetfillcolor{currentfill}%
\pgfsetlinewidth{0.000000pt}%
\definecolor{currentstroke}{rgb}{1.000000,0.894118,0.788235}%
\pgfsetstrokecolor{currentstroke}%
\pgfsetdash{}{0pt}%
\pgfpathmoveto{\pgfqpoint{2.869318in}{3.545983in}}%
\pgfpathlineto{\pgfqpoint{2.788359in}{3.499242in}}%
\pgfpathlineto{\pgfqpoint{2.869318in}{3.452500in}}%
\pgfpathlineto{\pgfqpoint{2.950278in}{3.499242in}}%
\pgfpathlineto{\pgfqpoint{2.869318in}{3.545983in}}%
\pgfpathclose%
\pgfusepath{fill}%
\end{pgfscope}%
\begin{pgfscope}%
\pgfpathrectangle{\pgfqpoint{0.765000in}{0.660000in}}{\pgfqpoint{4.620000in}{4.620000in}}%
\pgfusepath{clip}%
\pgfsetbuttcap%
\pgfsetroundjoin%
\definecolor{currentfill}{rgb}{1.000000,0.894118,0.788235}%
\pgfsetfillcolor{currentfill}%
\pgfsetlinewidth{0.000000pt}%
\definecolor{currentstroke}{rgb}{1.000000,0.894118,0.788235}%
\pgfsetstrokecolor{currentstroke}%
\pgfsetdash{}{0pt}%
\pgfpathmoveto{\pgfqpoint{2.788359in}{3.405758in}}%
\pgfpathlineto{\pgfqpoint{2.869318in}{3.359016in}}%
\pgfpathlineto{\pgfqpoint{2.869318in}{3.452500in}}%
\pgfpathlineto{\pgfqpoint{2.788359in}{3.499242in}}%
\pgfpathlineto{\pgfqpoint{2.788359in}{3.405758in}}%
\pgfpathclose%
\pgfusepath{fill}%
\end{pgfscope}%
\begin{pgfscope}%
\pgfpathrectangle{\pgfqpoint{0.765000in}{0.660000in}}{\pgfqpoint{4.620000in}{4.620000in}}%
\pgfusepath{clip}%
\pgfsetbuttcap%
\pgfsetroundjoin%
\definecolor{currentfill}{rgb}{1.000000,0.894118,0.788235}%
\pgfsetfillcolor{currentfill}%
\pgfsetlinewidth{0.000000pt}%
\definecolor{currentstroke}{rgb}{1.000000,0.894118,0.788235}%
\pgfsetstrokecolor{currentstroke}%
\pgfsetdash{}{0pt}%
\pgfpathmoveto{\pgfqpoint{3.593460in}{3.545983in}}%
\pgfpathlineto{\pgfqpoint{3.512500in}{3.499242in}}%
\pgfpathlineto{\pgfqpoint{3.593460in}{3.452500in}}%
\pgfpathlineto{\pgfqpoint{3.674419in}{3.499242in}}%
\pgfpathlineto{\pgfqpoint{3.593460in}{3.545983in}}%
\pgfpathclose%
\pgfusepath{fill}%
\end{pgfscope}%
\begin{pgfscope}%
\pgfpathrectangle{\pgfqpoint{0.765000in}{0.660000in}}{\pgfqpoint{4.620000in}{4.620000in}}%
\pgfusepath{clip}%
\pgfsetbuttcap%
\pgfsetroundjoin%
\definecolor{currentfill}{rgb}{1.000000,0.894118,0.788235}%
\pgfsetfillcolor{currentfill}%
\pgfsetlinewidth{0.000000pt}%
\definecolor{currentstroke}{rgb}{1.000000,0.894118,0.788235}%
\pgfsetstrokecolor{currentstroke}%
\pgfsetdash{}{0pt}%
\pgfpathmoveto{\pgfqpoint{3.512500in}{3.405758in}}%
\pgfpathlineto{\pgfqpoint{3.593460in}{3.359016in}}%
\pgfpathlineto{\pgfqpoint{3.593460in}{3.452500in}}%
\pgfpathlineto{\pgfqpoint{3.512500in}{3.499242in}}%
\pgfpathlineto{\pgfqpoint{3.512500in}{3.405758in}}%
\pgfpathclose%
\pgfusepath{fill}%
\end{pgfscope}%
\begin{pgfscope}%
\pgfpathrectangle{\pgfqpoint{0.765000in}{0.660000in}}{\pgfqpoint{4.620000in}{4.620000in}}%
\pgfusepath{clip}%
\pgfsetbuttcap%
\pgfsetroundjoin%
\definecolor{currentfill}{rgb}{1.000000,0.894118,0.788235}%
\pgfsetfillcolor{currentfill}%
\pgfsetlinewidth{0.000000pt}%
\definecolor{currentstroke}{rgb}{1.000000,0.894118,0.788235}%
\pgfsetstrokecolor{currentstroke}%
\pgfsetdash{}{0pt}%
\pgfpathmoveto{\pgfqpoint{2.869318in}{3.452500in}}%
\pgfpathlineto{\pgfqpoint{2.788359in}{3.405758in}}%
\pgfpathlineto{\pgfqpoint{3.512500in}{3.405758in}}%
\pgfpathlineto{\pgfqpoint{3.593460in}{3.452500in}}%
\pgfpathlineto{\pgfqpoint{2.869318in}{3.452500in}}%
\pgfpathclose%
\pgfusepath{fill}%
\end{pgfscope}%
\begin{pgfscope}%
\pgfpathrectangle{\pgfqpoint{0.765000in}{0.660000in}}{\pgfqpoint{4.620000in}{4.620000in}}%
\pgfusepath{clip}%
\pgfsetbuttcap%
\pgfsetroundjoin%
\definecolor{currentfill}{rgb}{1.000000,0.894118,0.788235}%
\pgfsetfillcolor{currentfill}%
\pgfsetlinewidth{0.000000pt}%
\definecolor{currentstroke}{rgb}{1.000000,0.894118,0.788235}%
\pgfsetstrokecolor{currentstroke}%
\pgfsetdash{}{0pt}%
\pgfpathmoveto{\pgfqpoint{2.950278in}{3.405758in}}%
\pgfpathlineto{\pgfqpoint{2.869318in}{3.452500in}}%
\pgfpathlineto{\pgfqpoint{3.593460in}{3.452500in}}%
\pgfpathlineto{\pgfqpoint{3.674419in}{3.405758in}}%
\pgfpathlineto{\pgfqpoint{2.950278in}{3.405758in}}%
\pgfpathclose%
\pgfusepath{fill}%
\end{pgfscope}%
\begin{pgfscope}%
\pgfpathrectangle{\pgfqpoint{0.765000in}{0.660000in}}{\pgfqpoint{4.620000in}{4.620000in}}%
\pgfusepath{clip}%
\pgfsetbuttcap%
\pgfsetroundjoin%
\definecolor{currentfill}{rgb}{1.000000,0.894118,0.788235}%
\pgfsetfillcolor{currentfill}%
\pgfsetlinewidth{0.000000pt}%
\definecolor{currentstroke}{rgb}{1.000000,0.894118,0.788235}%
\pgfsetstrokecolor{currentstroke}%
\pgfsetdash{}{0pt}%
\pgfpathmoveto{\pgfqpoint{2.869318in}{3.452500in}}%
\pgfpathlineto{\pgfqpoint{2.869318in}{3.545983in}}%
\pgfpathlineto{\pgfqpoint{3.593460in}{3.545983in}}%
\pgfpathlineto{\pgfqpoint{3.674419in}{3.405758in}}%
\pgfpathlineto{\pgfqpoint{2.869318in}{3.452500in}}%
\pgfpathclose%
\pgfusepath{fill}%
\end{pgfscope}%
\begin{pgfscope}%
\pgfpathrectangle{\pgfqpoint{0.765000in}{0.660000in}}{\pgfqpoint{4.620000in}{4.620000in}}%
\pgfusepath{clip}%
\pgfsetbuttcap%
\pgfsetroundjoin%
\definecolor{currentfill}{rgb}{1.000000,0.894118,0.788235}%
\pgfsetfillcolor{currentfill}%
\pgfsetlinewidth{0.000000pt}%
\definecolor{currentstroke}{rgb}{1.000000,0.894118,0.788235}%
\pgfsetstrokecolor{currentstroke}%
\pgfsetdash{}{0pt}%
\pgfpathmoveto{\pgfqpoint{2.950278in}{3.405758in}}%
\pgfpathlineto{\pgfqpoint{2.950278in}{3.499242in}}%
\pgfpathlineto{\pgfqpoint{3.674419in}{3.499242in}}%
\pgfpathlineto{\pgfqpoint{3.593460in}{3.452500in}}%
\pgfpathlineto{\pgfqpoint{2.950278in}{3.405758in}}%
\pgfpathclose%
\pgfusepath{fill}%
\end{pgfscope}%
\begin{pgfscope}%
\pgfpathrectangle{\pgfqpoint{0.765000in}{0.660000in}}{\pgfqpoint{4.620000in}{4.620000in}}%
\pgfusepath{clip}%
\pgfsetbuttcap%
\pgfsetroundjoin%
\definecolor{currentfill}{rgb}{1.000000,0.894118,0.788235}%
\pgfsetfillcolor{currentfill}%
\pgfsetlinewidth{0.000000pt}%
\definecolor{currentstroke}{rgb}{1.000000,0.894118,0.788235}%
\pgfsetstrokecolor{currentstroke}%
\pgfsetdash{}{0pt}%
\pgfpathmoveto{\pgfqpoint{2.788359in}{3.405758in}}%
\pgfpathlineto{\pgfqpoint{2.869318in}{3.359016in}}%
\pgfpathlineto{\pgfqpoint{3.593460in}{3.359016in}}%
\pgfpathlineto{\pgfqpoint{3.512500in}{3.405758in}}%
\pgfpathlineto{\pgfqpoint{2.788359in}{3.405758in}}%
\pgfpathclose%
\pgfusepath{fill}%
\end{pgfscope}%
\begin{pgfscope}%
\pgfpathrectangle{\pgfqpoint{0.765000in}{0.660000in}}{\pgfqpoint{4.620000in}{4.620000in}}%
\pgfusepath{clip}%
\pgfsetbuttcap%
\pgfsetroundjoin%
\definecolor{currentfill}{rgb}{1.000000,0.894118,0.788235}%
\pgfsetfillcolor{currentfill}%
\pgfsetlinewidth{0.000000pt}%
\definecolor{currentstroke}{rgb}{1.000000,0.894118,0.788235}%
\pgfsetstrokecolor{currentstroke}%
\pgfsetdash{}{0pt}%
\pgfpathmoveto{\pgfqpoint{2.869318in}{3.359016in}}%
\pgfpathlineto{\pgfqpoint{2.950278in}{3.405758in}}%
\pgfpathlineto{\pgfqpoint{3.674419in}{3.405758in}}%
\pgfpathlineto{\pgfqpoint{3.593460in}{3.359016in}}%
\pgfpathlineto{\pgfqpoint{2.869318in}{3.359016in}}%
\pgfpathclose%
\pgfusepath{fill}%
\end{pgfscope}%
\begin{pgfscope}%
\pgfpathrectangle{\pgfqpoint{0.765000in}{0.660000in}}{\pgfqpoint{4.620000in}{4.620000in}}%
\pgfusepath{clip}%
\pgfsetbuttcap%
\pgfsetroundjoin%
\definecolor{currentfill}{rgb}{1.000000,0.894118,0.788235}%
\pgfsetfillcolor{currentfill}%
\pgfsetlinewidth{0.000000pt}%
\definecolor{currentstroke}{rgb}{1.000000,0.894118,0.788235}%
\pgfsetstrokecolor{currentstroke}%
\pgfsetdash{}{0pt}%
\pgfpathmoveto{\pgfqpoint{2.869318in}{3.545983in}}%
\pgfpathlineto{\pgfqpoint{2.788359in}{3.499242in}}%
\pgfpathlineto{\pgfqpoint{3.512500in}{3.499242in}}%
\pgfpathlineto{\pgfqpoint{3.593460in}{3.545983in}}%
\pgfpathlineto{\pgfqpoint{2.869318in}{3.545983in}}%
\pgfpathclose%
\pgfusepath{fill}%
\end{pgfscope}%
\begin{pgfscope}%
\pgfpathrectangle{\pgfqpoint{0.765000in}{0.660000in}}{\pgfqpoint{4.620000in}{4.620000in}}%
\pgfusepath{clip}%
\pgfsetbuttcap%
\pgfsetroundjoin%
\definecolor{currentfill}{rgb}{1.000000,0.894118,0.788235}%
\pgfsetfillcolor{currentfill}%
\pgfsetlinewidth{0.000000pt}%
\definecolor{currentstroke}{rgb}{1.000000,0.894118,0.788235}%
\pgfsetstrokecolor{currentstroke}%
\pgfsetdash{}{0pt}%
\pgfpathmoveto{\pgfqpoint{2.950278in}{3.499242in}}%
\pgfpathlineto{\pgfqpoint{2.869318in}{3.545983in}}%
\pgfpathlineto{\pgfqpoint{3.593460in}{3.545983in}}%
\pgfpathlineto{\pgfqpoint{3.674419in}{3.499242in}}%
\pgfpathlineto{\pgfqpoint{2.950278in}{3.499242in}}%
\pgfpathclose%
\pgfusepath{fill}%
\end{pgfscope}%
\begin{pgfscope}%
\pgfpathrectangle{\pgfqpoint{0.765000in}{0.660000in}}{\pgfqpoint{4.620000in}{4.620000in}}%
\pgfusepath{clip}%
\pgfsetbuttcap%
\pgfsetroundjoin%
\definecolor{currentfill}{rgb}{1.000000,0.894118,0.788235}%
\pgfsetfillcolor{currentfill}%
\pgfsetlinewidth{0.000000pt}%
\definecolor{currentstroke}{rgb}{1.000000,0.894118,0.788235}%
\pgfsetstrokecolor{currentstroke}%
\pgfsetdash{}{0pt}%
\pgfpathmoveto{\pgfqpoint{2.788359in}{3.405758in}}%
\pgfpathlineto{\pgfqpoint{2.788359in}{3.499242in}}%
\pgfpathlineto{\pgfqpoint{3.512500in}{3.499242in}}%
\pgfpathlineto{\pgfqpoint{3.593460in}{3.359016in}}%
\pgfpathlineto{\pgfqpoint{2.788359in}{3.405758in}}%
\pgfpathclose%
\pgfusepath{fill}%
\end{pgfscope}%
\begin{pgfscope}%
\pgfpathrectangle{\pgfqpoint{0.765000in}{0.660000in}}{\pgfqpoint{4.620000in}{4.620000in}}%
\pgfusepath{clip}%
\pgfsetbuttcap%
\pgfsetroundjoin%
\definecolor{currentfill}{rgb}{1.000000,0.894118,0.788235}%
\pgfsetfillcolor{currentfill}%
\pgfsetlinewidth{0.000000pt}%
\definecolor{currentstroke}{rgb}{1.000000,0.894118,0.788235}%
\pgfsetstrokecolor{currentstroke}%
\pgfsetdash{}{0pt}%
\pgfpathmoveto{\pgfqpoint{2.869318in}{3.359016in}}%
\pgfpathlineto{\pgfqpoint{2.869318in}{3.452500in}}%
\pgfpathlineto{\pgfqpoint{3.593460in}{3.452500in}}%
\pgfpathlineto{\pgfqpoint{3.512500in}{3.405758in}}%
\pgfpathlineto{\pgfqpoint{2.869318in}{3.359016in}}%
\pgfpathclose%
\pgfusepath{fill}%
\end{pgfscope}%
\begin{pgfscope}%
\pgfpathrectangle{\pgfqpoint{0.765000in}{0.660000in}}{\pgfqpoint{4.620000in}{4.620000in}}%
\pgfusepath{clip}%
\pgfsetbuttcap%
\pgfsetroundjoin%
\definecolor{currentfill}{rgb}{1.000000,0.894118,0.788235}%
\pgfsetfillcolor{currentfill}%
\pgfsetlinewidth{0.000000pt}%
\definecolor{currentstroke}{rgb}{1.000000,0.894118,0.788235}%
\pgfsetstrokecolor{currentstroke}%
\pgfsetdash{}{0pt}%
\pgfpathmoveto{\pgfqpoint{2.788359in}{3.499242in}}%
\pgfpathlineto{\pgfqpoint{2.869318in}{3.452500in}}%
\pgfpathlineto{\pgfqpoint{3.593460in}{3.452500in}}%
\pgfpathlineto{\pgfqpoint{3.512500in}{3.499242in}}%
\pgfpathlineto{\pgfqpoint{2.788359in}{3.499242in}}%
\pgfpathclose%
\pgfusepath{fill}%
\end{pgfscope}%
\begin{pgfscope}%
\pgfpathrectangle{\pgfqpoint{0.765000in}{0.660000in}}{\pgfqpoint{4.620000in}{4.620000in}}%
\pgfusepath{clip}%
\pgfsetbuttcap%
\pgfsetroundjoin%
\definecolor{currentfill}{rgb}{1.000000,0.894118,0.788235}%
\pgfsetfillcolor{currentfill}%
\pgfsetlinewidth{0.000000pt}%
\definecolor{currentstroke}{rgb}{1.000000,0.894118,0.788235}%
\pgfsetstrokecolor{currentstroke}%
\pgfsetdash{}{0pt}%
\pgfpathmoveto{\pgfqpoint{2.869318in}{3.452500in}}%
\pgfpathlineto{\pgfqpoint{2.950278in}{3.499242in}}%
\pgfpathlineto{\pgfqpoint{3.674419in}{3.499242in}}%
\pgfpathlineto{\pgfqpoint{3.593460in}{3.452500in}}%
\pgfpathlineto{\pgfqpoint{2.869318in}{3.452500in}}%
\pgfpathclose%
\pgfusepath{fill}%
\end{pgfscope}%
\begin{pgfscope}%
\pgfpathrectangle{\pgfqpoint{0.765000in}{0.660000in}}{\pgfqpoint{4.620000in}{4.620000in}}%
\pgfusepath{clip}%
\pgfsetrectcap%
\pgfsetroundjoin%
\pgfsetlinewidth{1.204500pt}%
\definecolor{currentstroke}{rgb}{1.000000,0.576471,0.309804}%
\pgfsetstrokecolor{currentstroke}%
\pgfsetdash{}{0pt}%
\pgfpathmoveto{\pgfqpoint{2.291739in}{2.573807in}}%
\pgfusepath{stroke}%
\end{pgfscope}%
\begin{pgfscope}%
\pgfpathrectangle{\pgfqpoint{0.765000in}{0.660000in}}{\pgfqpoint{4.620000in}{4.620000in}}%
\pgfusepath{clip}%
\pgfsetbuttcap%
\pgfsetroundjoin%
\definecolor{currentfill}{rgb}{1.000000,0.576471,0.309804}%
\pgfsetfillcolor{currentfill}%
\pgfsetlinewidth{1.003750pt}%
\definecolor{currentstroke}{rgb}{1.000000,0.576471,0.309804}%
\pgfsetstrokecolor{currentstroke}%
\pgfsetdash{}{0pt}%
\pgfsys@defobject{currentmarker}{\pgfqpoint{-0.033333in}{-0.033333in}}{\pgfqpoint{0.033333in}{0.033333in}}{%
\pgfpathmoveto{\pgfqpoint{0.000000in}{-0.033333in}}%
\pgfpathcurveto{\pgfqpoint{0.008840in}{-0.033333in}}{\pgfqpoint{0.017319in}{-0.029821in}}{\pgfqpoint{0.023570in}{-0.023570in}}%
\pgfpathcurveto{\pgfqpoint{0.029821in}{-0.017319in}}{\pgfqpoint{0.033333in}{-0.008840in}}{\pgfqpoint{0.033333in}{0.000000in}}%
\pgfpathcurveto{\pgfqpoint{0.033333in}{0.008840in}}{\pgfqpoint{0.029821in}{0.017319in}}{\pgfqpoint{0.023570in}{0.023570in}}%
\pgfpathcurveto{\pgfqpoint{0.017319in}{0.029821in}}{\pgfqpoint{0.008840in}{0.033333in}}{\pgfqpoint{0.000000in}{0.033333in}}%
\pgfpathcurveto{\pgfqpoint{-0.008840in}{0.033333in}}{\pgfqpoint{-0.017319in}{0.029821in}}{\pgfqpoint{-0.023570in}{0.023570in}}%
\pgfpathcurveto{\pgfqpoint{-0.029821in}{0.017319in}}{\pgfqpoint{-0.033333in}{0.008840in}}{\pgfqpoint{-0.033333in}{0.000000in}}%
\pgfpathcurveto{\pgfqpoint{-0.033333in}{-0.008840in}}{\pgfqpoint{-0.029821in}{-0.017319in}}{\pgfqpoint{-0.023570in}{-0.023570in}}%
\pgfpathcurveto{\pgfqpoint{-0.017319in}{-0.029821in}}{\pgfqpoint{-0.008840in}{-0.033333in}}{\pgfqpoint{0.000000in}{-0.033333in}}%
\pgfpathlineto{\pgfqpoint{0.000000in}{-0.033333in}}%
\pgfpathclose%
\pgfusepath{stroke,fill}%
}%
\begin{pgfscope}%
\pgfsys@transformshift{2.291739in}{2.573807in}%
\pgfsys@useobject{currentmarker}{}%
\end{pgfscope}%
\end{pgfscope}%
\begin{pgfscope}%
\pgfpathrectangle{\pgfqpoint{0.765000in}{0.660000in}}{\pgfqpoint{4.620000in}{4.620000in}}%
\pgfusepath{clip}%
\pgfsetrectcap%
\pgfsetroundjoin%
\pgfsetlinewidth{1.204500pt}%
\definecolor{currentstroke}{rgb}{1.000000,0.576471,0.309804}%
\pgfsetstrokecolor{currentstroke}%
\pgfsetdash{}{0pt}%
\pgfpathmoveto{\pgfqpoint{2.742212in}{2.153332in}}%
\pgfusepath{stroke}%
\end{pgfscope}%
\begin{pgfscope}%
\pgfpathrectangle{\pgfqpoint{0.765000in}{0.660000in}}{\pgfqpoint{4.620000in}{4.620000in}}%
\pgfusepath{clip}%
\pgfsetbuttcap%
\pgfsetroundjoin%
\definecolor{currentfill}{rgb}{1.000000,0.576471,0.309804}%
\pgfsetfillcolor{currentfill}%
\pgfsetlinewidth{1.003750pt}%
\definecolor{currentstroke}{rgb}{1.000000,0.576471,0.309804}%
\pgfsetstrokecolor{currentstroke}%
\pgfsetdash{}{0pt}%
\pgfsys@defobject{currentmarker}{\pgfqpoint{-0.033333in}{-0.033333in}}{\pgfqpoint{0.033333in}{0.033333in}}{%
\pgfpathmoveto{\pgfqpoint{0.000000in}{-0.033333in}}%
\pgfpathcurveto{\pgfqpoint{0.008840in}{-0.033333in}}{\pgfqpoint{0.017319in}{-0.029821in}}{\pgfqpoint{0.023570in}{-0.023570in}}%
\pgfpathcurveto{\pgfqpoint{0.029821in}{-0.017319in}}{\pgfqpoint{0.033333in}{-0.008840in}}{\pgfqpoint{0.033333in}{0.000000in}}%
\pgfpathcurveto{\pgfqpoint{0.033333in}{0.008840in}}{\pgfqpoint{0.029821in}{0.017319in}}{\pgfqpoint{0.023570in}{0.023570in}}%
\pgfpathcurveto{\pgfqpoint{0.017319in}{0.029821in}}{\pgfqpoint{0.008840in}{0.033333in}}{\pgfqpoint{0.000000in}{0.033333in}}%
\pgfpathcurveto{\pgfqpoint{-0.008840in}{0.033333in}}{\pgfqpoint{-0.017319in}{0.029821in}}{\pgfqpoint{-0.023570in}{0.023570in}}%
\pgfpathcurveto{\pgfqpoint{-0.029821in}{0.017319in}}{\pgfqpoint{-0.033333in}{0.008840in}}{\pgfqpoint{-0.033333in}{0.000000in}}%
\pgfpathcurveto{\pgfqpoint{-0.033333in}{-0.008840in}}{\pgfqpoint{-0.029821in}{-0.017319in}}{\pgfqpoint{-0.023570in}{-0.023570in}}%
\pgfpathcurveto{\pgfqpoint{-0.017319in}{-0.029821in}}{\pgfqpoint{-0.008840in}{-0.033333in}}{\pgfqpoint{0.000000in}{-0.033333in}}%
\pgfpathlineto{\pgfqpoint{0.000000in}{-0.033333in}}%
\pgfpathclose%
\pgfusepath{stroke,fill}%
}%
\begin{pgfscope}%
\pgfsys@transformshift{2.742212in}{2.153332in}%
\pgfsys@useobject{currentmarker}{}%
\end{pgfscope}%
\end{pgfscope}%
\begin{pgfscope}%
\pgfpathrectangle{\pgfqpoint{0.765000in}{0.660000in}}{\pgfqpoint{4.620000in}{4.620000in}}%
\pgfusepath{clip}%
\pgfsetrectcap%
\pgfsetroundjoin%
\pgfsetlinewidth{1.204500pt}%
\definecolor{currentstroke}{rgb}{1.000000,0.576471,0.309804}%
\pgfsetstrokecolor{currentstroke}%
\pgfsetdash{}{0pt}%
\pgfpathmoveto{\pgfqpoint{2.102070in}{2.567929in}}%
\pgfusepath{stroke}%
\end{pgfscope}%
\begin{pgfscope}%
\pgfpathrectangle{\pgfqpoint{0.765000in}{0.660000in}}{\pgfqpoint{4.620000in}{4.620000in}}%
\pgfusepath{clip}%
\pgfsetbuttcap%
\pgfsetroundjoin%
\definecolor{currentfill}{rgb}{1.000000,0.576471,0.309804}%
\pgfsetfillcolor{currentfill}%
\pgfsetlinewidth{1.003750pt}%
\definecolor{currentstroke}{rgb}{1.000000,0.576471,0.309804}%
\pgfsetstrokecolor{currentstroke}%
\pgfsetdash{}{0pt}%
\pgfsys@defobject{currentmarker}{\pgfqpoint{-0.033333in}{-0.033333in}}{\pgfqpoint{0.033333in}{0.033333in}}{%
\pgfpathmoveto{\pgfqpoint{0.000000in}{-0.033333in}}%
\pgfpathcurveto{\pgfqpoint{0.008840in}{-0.033333in}}{\pgfqpoint{0.017319in}{-0.029821in}}{\pgfqpoint{0.023570in}{-0.023570in}}%
\pgfpathcurveto{\pgfqpoint{0.029821in}{-0.017319in}}{\pgfqpoint{0.033333in}{-0.008840in}}{\pgfqpoint{0.033333in}{0.000000in}}%
\pgfpathcurveto{\pgfqpoint{0.033333in}{0.008840in}}{\pgfqpoint{0.029821in}{0.017319in}}{\pgfqpoint{0.023570in}{0.023570in}}%
\pgfpathcurveto{\pgfqpoint{0.017319in}{0.029821in}}{\pgfqpoint{0.008840in}{0.033333in}}{\pgfqpoint{0.000000in}{0.033333in}}%
\pgfpathcurveto{\pgfqpoint{-0.008840in}{0.033333in}}{\pgfqpoint{-0.017319in}{0.029821in}}{\pgfqpoint{-0.023570in}{0.023570in}}%
\pgfpathcurveto{\pgfqpoint{-0.029821in}{0.017319in}}{\pgfqpoint{-0.033333in}{0.008840in}}{\pgfqpoint{-0.033333in}{0.000000in}}%
\pgfpathcurveto{\pgfqpoint{-0.033333in}{-0.008840in}}{\pgfqpoint{-0.029821in}{-0.017319in}}{\pgfqpoint{-0.023570in}{-0.023570in}}%
\pgfpathcurveto{\pgfqpoint{-0.017319in}{-0.029821in}}{\pgfqpoint{-0.008840in}{-0.033333in}}{\pgfqpoint{0.000000in}{-0.033333in}}%
\pgfpathlineto{\pgfqpoint{0.000000in}{-0.033333in}}%
\pgfpathclose%
\pgfusepath{stroke,fill}%
}%
\begin{pgfscope}%
\pgfsys@transformshift{2.102070in}{2.567929in}%
\pgfsys@useobject{currentmarker}{}%
\end{pgfscope}%
\end{pgfscope}%
\begin{pgfscope}%
\pgfpathrectangle{\pgfqpoint{0.765000in}{0.660000in}}{\pgfqpoint{4.620000in}{4.620000in}}%
\pgfusepath{clip}%
\pgfsetrectcap%
\pgfsetroundjoin%
\pgfsetlinewidth{1.204500pt}%
\definecolor{currentstroke}{rgb}{1.000000,0.576471,0.309804}%
\pgfsetstrokecolor{currentstroke}%
\pgfsetdash{}{0pt}%
\pgfpathmoveto{\pgfqpoint{2.888730in}{4.501916in}}%
\pgfusepath{stroke}%
\end{pgfscope}%
\begin{pgfscope}%
\pgfpathrectangle{\pgfqpoint{0.765000in}{0.660000in}}{\pgfqpoint{4.620000in}{4.620000in}}%
\pgfusepath{clip}%
\pgfsetbuttcap%
\pgfsetroundjoin%
\definecolor{currentfill}{rgb}{1.000000,0.576471,0.309804}%
\pgfsetfillcolor{currentfill}%
\pgfsetlinewidth{1.003750pt}%
\definecolor{currentstroke}{rgb}{1.000000,0.576471,0.309804}%
\pgfsetstrokecolor{currentstroke}%
\pgfsetdash{}{0pt}%
\pgfsys@defobject{currentmarker}{\pgfqpoint{-0.033333in}{-0.033333in}}{\pgfqpoint{0.033333in}{0.033333in}}{%
\pgfpathmoveto{\pgfqpoint{0.000000in}{-0.033333in}}%
\pgfpathcurveto{\pgfqpoint{0.008840in}{-0.033333in}}{\pgfqpoint{0.017319in}{-0.029821in}}{\pgfqpoint{0.023570in}{-0.023570in}}%
\pgfpathcurveto{\pgfqpoint{0.029821in}{-0.017319in}}{\pgfqpoint{0.033333in}{-0.008840in}}{\pgfqpoint{0.033333in}{0.000000in}}%
\pgfpathcurveto{\pgfqpoint{0.033333in}{0.008840in}}{\pgfqpoint{0.029821in}{0.017319in}}{\pgfqpoint{0.023570in}{0.023570in}}%
\pgfpathcurveto{\pgfqpoint{0.017319in}{0.029821in}}{\pgfqpoint{0.008840in}{0.033333in}}{\pgfqpoint{0.000000in}{0.033333in}}%
\pgfpathcurveto{\pgfqpoint{-0.008840in}{0.033333in}}{\pgfqpoint{-0.017319in}{0.029821in}}{\pgfqpoint{-0.023570in}{0.023570in}}%
\pgfpathcurveto{\pgfqpoint{-0.029821in}{0.017319in}}{\pgfqpoint{-0.033333in}{0.008840in}}{\pgfqpoint{-0.033333in}{0.000000in}}%
\pgfpathcurveto{\pgfqpoint{-0.033333in}{-0.008840in}}{\pgfqpoint{-0.029821in}{-0.017319in}}{\pgfqpoint{-0.023570in}{-0.023570in}}%
\pgfpathcurveto{\pgfqpoint{-0.017319in}{-0.029821in}}{\pgfqpoint{-0.008840in}{-0.033333in}}{\pgfqpoint{0.000000in}{-0.033333in}}%
\pgfpathlineto{\pgfqpoint{0.000000in}{-0.033333in}}%
\pgfpathclose%
\pgfusepath{stroke,fill}%
}%
\begin{pgfscope}%
\pgfsys@transformshift{2.888730in}{4.501916in}%
\pgfsys@useobject{currentmarker}{}%
\end{pgfscope}%
\end{pgfscope}%
\begin{pgfscope}%
\pgfpathrectangle{\pgfqpoint{0.765000in}{0.660000in}}{\pgfqpoint{4.620000in}{4.620000in}}%
\pgfusepath{clip}%
\pgfsetrectcap%
\pgfsetroundjoin%
\pgfsetlinewidth{1.204500pt}%
\definecolor{currentstroke}{rgb}{1.000000,0.576471,0.309804}%
\pgfsetstrokecolor{currentstroke}%
\pgfsetdash{}{0pt}%
\pgfpathmoveto{\pgfqpoint{4.298302in}{2.705740in}}%
\pgfusepath{stroke}%
\end{pgfscope}%
\begin{pgfscope}%
\pgfpathrectangle{\pgfqpoint{0.765000in}{0.660000in}}{\pgfqpoint{4.620000in}{4.620000in}}%
\pgfusepath{clip}%
\pgfsetbuttcap%
\pgfsetroundjoin%
\definecolor{currentfill}{rgb}{1.000000,0.576471,0.309804}%
\pgfsetfillcolor{currentfill}%
\pgfsetlinewidth{1.003750pt}%
\definecolor{currentstroke}{rgb}{1.000000,0.576471,0.309804}%
\pgfsetstrokecolor{currentstroke}%
\pgfsetdash{}{0pt}%
\pgfsys@defobject{currentmarker}{\pgfqpoint{-0.033333in}{-0.033333in}}{\pgfqpoint{0.033333in}{0.033333in}}{%
\pgfpathmoveto{\pgfqpoint{0.000000in}{-0.033333in}}%
\pgfpathcurveto{\pgfqpoint{0.008840in}{-0.033333in}}{\pgfqpoint{0.017319in}{-0.029821in}}{\pgfqpoint{0.023570in}{-0.023570in}}%
\pgfpathcurveto{\pgfqpoint{0.029821in}{-0.017319in}}{\pgfqpoint{0.033333in}{-0.008840in}}{\pgfqpoint{0.033333in}{0.000000in}}%
\pgfpathcurveto{\pgfqpoint{0.033333in}{0.008840in}}{\pgfqpoint{0.029821in}{0.017319in}}{\pgfqpoint{0.023570in}{0.023570in}}%
\pgfpathcurveto{\pgfqpoint{0.017319in}{0.029821in}}{\pgfqpoint{0.008840in}{0.033333in}}{\pgfqpoint{0.000000in}{0.033333in}}%
\pgfpathcurveto{\pgfqpoint{-0.008840in}{0.033333in}}{\pgfqpoint{-0.017319in}{0.029821in}}{\pgfqpoint{-0.023570in}{0.023570in}}%
\pgfpathcurveto{\pgfqpoint{-0.029821in}{0.017319in}}{\pgfqpoint{-0.033333in}{0.008840in}}{\pgfqpoint{-0.033333in}{0.000000in}}%
\pgfpathcurveto{\pgfqpoint{-0.033333in}{-0.008840in}}{\pgfqpoint{-0.029821in}{-0.017319in}}{\pgfqpoint{-0.023570in}{-0.023570in}}%
\pgfpathcurveto{\pgfqpoint{-0.017319in}{-0.029821in}}{\pgfqpoint{-0.008840in}{-0.033333in}}{\pgfqpoint{0.000000in}{-0.033333in}}%
\pgfpathlineto{\pgfqpoint{0.000000in}{-0.033333in}}%
\pgfpathclose%
\pgfusepath{stroke,fill}%
}%
\begin{pgfscope}%
\pgfsys@transformshift{4.298302in}{2.705740in}%
\pgfsys@useobject{currentmarker}{}%
\end{pgfscope}%
\end{pgfscope}%
\begin{pgfscope}%
\pgfpathrectangle{\pgfqpoint{0.765000in}{0.660000in}}{\pgfqpoint{4.620000in}{4.620000in}}%
\pgfusepath{clip}%
\pgfsetrectcap%
\pgfsetroundjoin%
\pgfsetlinewidth{1.204500pt}%
\definecolor{currentstroke}{rgb}{1.000000,0.576471,0.309804}%
\pgfsetstrokecolor{currentstroke}%
\pgfsetdash{}{0pt}%
\pgfpathmoveto{\pgfqpoint{4.024928in}{2.450185in}}%
\pgfusepath{stroke}%
\end{pgfscope}%
\begin{pgfscope}%
\pgfpathrectangle{\pgfqpoint{0.765000in}{0.660000in}}{\pgfqpoint{4.620000in}{4.620000in}}%
\pgfusepath{clip}%
\pgfsetbuttcap%
\pgfsetroundjoin%
\definecolor{currentfill}{rgb}{1.000000,0.576471,0.309804}%
\pgfsetfillcolor{currentfill}%
\pgfsetlinewidth{1.003750pt}%
\definecolor{currentstroke}{rgb}{1.000000,0.576471,0.309804}%
\pgfsetstrokecolor{currentstroke}%
\pgfsetdash{}{0pt}%
\pgfsys@defobject{currentmarker}{\pgfqpoint{-0.033333in}{-0.033333in}}{\pgfqpoint{0.033333in}{0.033333in}}{%
\pgfpathmoveto{\pgfqpoint{0.000000in}{-0.033333in}}%
\pgfpathcurveto{\pgfqpoint{0.008840in}{-0.033333in}}{\pgfqpoint{0.017319in}{-0.029821in}}{\pgfqpoint{0.023570in}{-0.023570in}}%
\pgfpathcurveto{\pgfqpoint{0.029821in}{-0.017319in}}{\pgfqpoint{0.033333in}{-0.008840in}}{\pgfqpoint{0.033333in}{0.000000in}}%
\pgfpathcurveto{\pgfqpoint{0.033333in}{0.008840in}}{\pgfqpoint{0.029821in}{0.017319in}}{\pgfqpoint{0.023570in}{0.023570in}}%
\pgfpathcurveto{\pgfqpoint{0.017319in}{0.029821in}}{\pgfqpoint{0.008840in}{0.033333in}}{\pgfqpoint{0.000000in}{0.033333in}}%
\pgfpathcurveto{\pgfqpoint{-0.008840in}{0.033333in}}{\pgfqpoint{-0.017319in}{0.029821in}}{\pgfqpoint{-0.023570in}{0.023570in}}%
\pgfpathcurveto{\pgfqpoint{-0.029821in}{0.017319in}}{\pgfqpoint{-0.033333in}{0.008840in}}{\pgfqpoint{-0.033333in}{0.000000in}}%
\pgfpathcurveto{\pgfqpoint{-0.033333in}{-0.008840in}}{\pgfqpoint{-0.029821in}{-0.017319in}}{\pgfqpoint{-0.023570in}{-0.023570in}}%
\pgfpathcurveto{\pgfqpoint{-0.017319in}{-0.029821in}}{\pgfqpoint{-0.008840in}{-0.033333in}}{\pgfqpoint{0.000000in}{-0.033333in}}%
\pgfpathlineto{\pgfqpoint{0.000000in}{-0.033333in}}%
\pgfpathclose%
\pgfusepath{stroke,fill}%
}%
\begin{pgfscope}%
\pgfsys@transformshift{4.024928in}{2.450185in}%
\pgfsys@useobject{currentmarker}{}%
\end{pgfscope}%
\end{pgfscope}%
\begin{pgfscope}%
\pgfpathrectangle{\pgfqpoint{0.765000in}{0.660000in}}{\pgfqpoint{4.620000in}{4.620000in}}%
\pgfusepath{clip}%
\pgfsetrectcap%
\pgfsetroundjoin%
\pgfsetlinewidth{1.204500pt}%
\definecolor{currentstroke}{rgb}{1.000000,0.576471,0.309804}%
\pgfsetstrokecolor{currentstroke}%
\pgfsetdash{}{0pt}%
\pgfpathmoveto{\pgfqpoint{2.095736in}{2.808077in}}%
\pgfusepath{stroke}%
\end{pgfscope}%
\begin{pgfscope}%
\pgfpathrectangle{\pgfqpoint{0.765000in}{0.660000in}}{\pgfqpoint{4.620000in}{4.620000in}}%
\pgfusepath{clip}%
\pgfsetbuttcap%
\pgfsetroundjoin%
\definecolor{currentfill}{rgb}{1.000000,0.576471,0.309804}%
\pgfsetfillcolor{currentfill}%
\pgfsetlinewidth{1.003750pt}%
\definecolor{currentstroke}{rgb}{1.000000,0.576471,0.309804}%
\pgfsetstrokecolor{currentstroke}%
\pgfsetdash{}{0pt}%
\pgfsys@defobject{currentmarker}{\pgfqpoint{-0.033333in}{-0.033333in}}{\pgfqpoint{0.033333in}{0.033333in}}{%
\pgfpathmoveto{\pgfqpoint{0.000000in}{-0.033333in}}%
\pgfpathcurveto{\pgfqpoint{0.008840in}{-0.033333in}}{\pgfqpoint{0.017319in}{-0.029821in}}{\pgfqpoint{0.023570in}{-0.023570in}}%
\pgfpathcurveto{\pgfqpoint{0.029821in}{-0.017319in}}{\pgfqpoint{0.033333in}{-0.008840in}}{\pgfqpoint{0.033333in}{0.000000in}}%
\pgfpathcurveto{\pgfqpoint{0.033333in}{0.008840in}}{\pgfqpoint{0.029821in}{0.017319in}}{\pgfqpoint{0.023570in}{0.023570in}}%
\pgfpathcurveto{\pgfqpoint{0.017319in}{0.029821in}}{\pgfqpoint{0.008840in}{0.033333in}}{\pgfqpoint{0.000000in}{0.033333in}}%
\pgfpathcurveto{\pgfqpoint{-0.008840in}{0.033333in}}{\pgfqpoint{-0.017319in}{0.029821in}}{\pgfqpoint{-0.023570in}{0.023570in}}%
\pgfpathcurveto{\pgfqpoint{-0.029821in}{0.017319in}}{\pgfqpoint{-0.033333in}{0.008840in}}{\pgfqpoint{-0.033333in}{0.000000in}}%
\pgfpathcurveto{\pgfqpoint{-0.033333in}{-0.008840in}}{\pgfqpoint{-0.029821in}{-0.017319in}}{\pgfqpoint{-0.023570in}{-0.023570in}}%
\pgfpathcurveto{\pgfqpoint{-0.017319in}{-0.029821in}}{\pgfqpoint{-0.008840in}{-0.033333in}}{\pgfqpoint{0.000000in}{-0.033333in}}%
\pgfpathlineto{\pgfqpoint{0.000000in}{-0.033333in}}%
\pgfpathclose%
\pgfusepath{stroke,fill}%
}%
\begin{pgfscope}%
\pgfsys@transformshift{2.095736in}{2.808077in}%
\pgfsys@useobject{currentmarker}{}%
\end{pgfscope}%
\end{pgfscope}%
\begin{pgfscope}%
\pgfpathrectangle{\pgfqpoint{0.765000in}{0.660000in}}{\pgfqpoint{4.620000in}{4.620000in}}%
\pgfusepath{clip}%
\pgfsetrectcap%
\pgfsetroundjoin%
\pgfsetlinewidth{1.204500pt}%
\definecolor{currentstroke}{rgb}{1.000000,0.576471,0.309804}%
\pgfsetstrokecolor{currentstroke}%
\pgfsetdash{}{0pt}%
\pgfpathmoveto{\pgfqpoint{3.252520in}{3.963575in}}%
\pgfusepath{stroke}%
\end{pgfscope}%
\begin{pgfscope}%
\pgfpathrectangle{\pgfqpoint{0.765000in}{0.660000in}}{\pgfqpoint{4.620000in}{4.620000in}}%
\pgfusepath{clip}%
\pgfsetbuttcap%
\pgfsetroundjoin%
\definecolor{currentfill}{rgb}{1.000000,0.576471,0.309804}%
\pgfsetfillcolor{currentfill}%
\pgfsetlinewidth{1.003750pt}%
\definecolor{currentstroke}{rgb}{1.000000,0.576471,0.309804}%
\pgfsetstrokecolor{currentstroke}%
\pgfsetdash{}{0pt}%
\pgfsys@defobject{currentmarker}{\pgfqpoint{-0.033333in}{-0.033333in}}{\pgfqpoint{0.033333in}{0.033333in}}{%
\pgfpathmoveto{\pgfqpoint{0.000000in}{-0.033333in}}%
\pgfpathcurveto{\pgfqpoint{0.008840in}{-0.033333in}}{\pgfqpoint{0.017319in}{-0.029821in}}{\pgfqpoint{0.023570in}{-0.023570in}}%
\pgfpathcurveto{\pgfqpoint{0.029821in}{-0.017319in}}{\pgfqpoint{0.033333in}{-0.008840in}}{\pgfqpoint{0.033333in}{0.000000in}}%
\pgfpathcurveto{\pgfqpoint{0.033333in}{0.008840in}}{\pgfqpoint{0.029821in}{0.017319in}}{\pgfqpoint{0.023570in}{0.023570in}}%
\pgfpathcurveto{\pgfqpoint{0.017319in}{0.029821in}}{\pgfqpoint{0.008840in}{0.033333in}}{\pgfqpoint{0.000000in}{0.033333in}}%
\pgfpathcurveto{\pgfqpoint{-0.008840in}{0.033333in}}{\pgfqpoint{-0.017319in}{0.029821in}}{\pgfqpoint{-0.023570in}{0.023570in}}%
\pgfpathcurveto{\pgfqpoint{-0.029821in}{0.017319in}}{\pgfqpoint{-0.033333in}{0.008840in}}{\pgfqpoint{-0.033333in}{0.000000in}}%
\pgfpathcurveto{\pgfqpoint{-0.033333in}{-0.008840in}}{\pgfqpoint{-0.029821in}{-0.017319in}}{\pgfqpoint{-0.023570in}{-0.023570in}}%
\pgfpathcurveto{\pgfqpoint{-0.017319in}{-0.029821in}}{\pgfqpoint{-0.008840in}{-0.033333in}}{\pgfqpoint{0.000000in}{-0.033333in}}%
\pgfpathlineto{\pgfqpoint{0.000000in}{-0.033333in}}%
\pgfpathclose%
\pgfusepath{stroke,fill}%
}%
\begin{pgfscope}%
\pgfsys@transformshift{3.252520in}{3.963575in}%
\pgfsys@useobject{currentmarker}{}%
\end{pgfscope}%
\end{pgfscope}%
\begin{pgfscope}%
\pgfpathrectangle{\pgfqpoint{0.765000in}{0.660000in}}{\pgfqpoint{4.620000in}{4.620000in}}%
\pgfusepath{clip}%
\pgfsetrectcap%
\pgfsetroundjoin%
\pgfsetlinewidth{1.204500pt}%
\definecolor{currentstroke}{rgb}{1.000000,0.576471,0.309804}%
\pgfsetstrokecolor{currentstroke}%
\pgfsetdash{}{0pt}%
\pgfpathmoveto{\pgfqpoint{2.125219in}{3.625990in}}%
\pgfusepath{stroke}%
\end{pgfscope}%
\begin{pgfscope}%
\pgfpathrectangle{\pgfqpoint{0.765000in}{0.660000in}}{\pgfqpoint{4.620000in}{4.620000in}}%
\pgfusepath{clip}%
\pgfsetbuttcap%
\pgfsetroundjoin%
\definecolor{currentfill}{rgb}{1.000000,0.576471,0.309804}%
\pgfsetfillcolor{currentfill}%
\pgfsetlinewidth{1.003750pt}%
\definecolor{currentstroke}{rgb}{1.000000,0.576471,0.309804}%
\pgfsetstrokecolor{currentstroke}%
\pgfsetdash{}{0pt}%
\pgfsys@defobject{currentmarker}{\pgfqpoint{-0.033333in}{-0.033333in}}{\pgfqpoint{0.033333in}{0.033333in}}{%
\pgfpathmoveto{\pgfqpoint{0.000000in}{-0.033333in}}%
\pgfpathcurveto{\pgfqpoint{0.008840in}{-0.033333in}}{\pgfqpoint{0.017319in}{-0.029821in}}{\pgfqpoint{0.023570in}{-0.023570in}}%
\pgfpathcurveto{\pgfqpoint{0.029821in}{-0.017319in}}{\pgfqpoint{0.033333in}{-0.008840in}}{\pgfqpoint{0.033333in}{0.000000in}}%
\pgfpathcurveto{\pgfqpoint{0.033333in}{0.008840in}}{\pgfqpoint{0.029821in}{0.017319in}}{\pgfqpoint{0.023570in}{0.023570in}}%
\pgfpathcurveto{\pgfqpoint{0.017319in}{0.029821in}}{\pgfqpoint{0.008840in}{0.033333in}}{\pgfqpoint{0.000000in}{0.033333in}}%
\pgfpathcurveto{\pgfqpoint{-0.008840in}{0.033333in}}{\pgfqpoint{-0.017319in}{0.029821in}}{\pgfqpoint{-0.023570in}{0.023570in}}%
\pgfpathcurveto{\pgfqpoint{-0.029821in}{0.017319in}}{\pgfqpoint{-0.033333in}{0.008840in}}{\pgfqpoint{-0.033333in}{0.000000in}}%
\pgfpathcurveto{\pgfqpoint{-0.033333in}{-0.008840in}}{\pgfqpoint{-0.029821in}{-0.017319in}}{\pgfqpoint{-0.023570in}{-0.023570in}}%
\pgfpathcurveto{\pgfqpoint{-0.017319in}{-0.029821in}}{\pgfqpoint{-0.008840in}{-0.033333in}}{\pgfqpoint{0.000000in}{-0.033333in}}%
\pgfpathlineto{\pgfqpoint{0.000000in}{-0.033333in}}%
\pgfpathclose%
\pgfusepath{stroke,fill}%
}%
\begin{pgfscope}%
\pgfsys@transformshift{2.125219in}{3.625990in}%
\pgfsys@useobject{currentmarker}{}%
\end{pgfscope}%
\end{pgfscope}%
\begin{pgfscope}%
\pgfpathrectangle{\pgfqpoint{0.765000in}{0.660000in}}{\pgfqpoint{4.620000in}{4.620000in}}%
\pgfusepath{clip}%
\pgfsetrectcap%
\pgfsetroundjoin%
\pgfsetlinewidth{1.204500pt}%
\definecolor{currentstroke}{rgb}{1.000000,0.576471,0.309804}%
\pgfsetstrokecolor{currentstroke}%
\pgfsetdash{}{0pt}%
\pgfpathmoveto{\pgfqpoint{3.081498in}{2.055640in}}%
\pgfusepath{stroke}%
\end{pgfscope}%
\begin{pgfscope}%
\pgfpathrectangle{\pgfqpoint{0.765000in}{0.660000in}}{\pgfqpoint{4.620000in}{4.620000in}}%
\pgfusepath{clip}%
\pgfsetbuttcap%
\pgfsetroundjoin%
\definecolor{currentfill}{rgb}{1.000000,0.576471,0.309804}%
\pgfsetfillcolor{currentfill}%
\pgfsetlinewidth{1.003750pt}%
\definecolor{currentstroke}{rgb}{1.000000,0.576471,0.309804}%
\pgfsetstrokecolor{currentstroke}%
\pgfsetdash{}{0pt}%
\pgfsys@defobject{currentmarker}{\pgfqpoint{-0.033333in}{-0.033333in}}{\pgfqpoint{0.033333in}{0.033333in}}{%
\pgfpathmoveto{\pgfqpoint{0.000000in}{-0.033333in}}%
\pgfpathcurveto{\pgfqpoint{0.008840in}{-0.033333in}}{\pgfqpoint{0.017319in}{-0.029821in}}{\pgfqpoint{0.023570in}{-0.023570in}}%
\pgfpathcurveto{\pgfqpoint{0.029821in}{-0.017319in}}{\pgfqpoint{0.033333in}{-0.008840in}}{\pgfqpoint{0.033333in}{0.000000in}}%
\pgfpathcurveto{\pgfqpoint{0.033333in}{0.008840in}}{\pgfqpoint{0.029821in}{0.017319in}}{\pgfqpoint{0.023570in}{0.023570in}}%
\pgfpathcurveto{\pgfqpoint{0.017319in}{0.029821in}}{\pgfqpoint{0.008840in}{0.033333in}}{\pgfqpoint{0.000000in}{0.033333in}}%
\pgfpathcurveto{\pgfqpoint{-0.008840in}{0.033333in}}{\pgfqpoint{-0.017319in}{0.029821in}}{\pgfqpoint{-0.023570in}{0.023570in}}%
\pgfpathcurveto{\pgfqpoint{-0.029821in}{0.017319in}}{\pgfqpoint{-0.033333in}{0.008840in}}{\pgfqpoint{-0.033333in}{0.000000in}}%
\pgfpathcurveto{\pgfqpoint{-0.033333in}{-0.008840in}}{\pgfqpoint{-0.029821in}{-0.017319in}}{\pgfqpoint{-0.023570in}{-0.023570in}}%
\pgfpathcurveto{\pgfqpoint{-0.017319in}{-0.029821in}}{\pgfqpoint{-0.008840in}{-0.033333in}}{\pgfqpoint{0.000000in}{-0.033333in}}%
\pgfpathlineto{\pgfqpoint{0.000000in}{-0.033333in}}%
\pgfpathclose%
\pgfusepath{stroke,fill}%
}%
\begin{pgfscope}%
\pgfsys@transformshift{3.081498in}{2.055640in}%
\pgfsys@useobject{currentmarker}{}%
\end{pgfscope}%
\end{pgfscope}%
\begin{pgfscope}%
\pgfpathrectangle{\pgfqpoint{0.765000in}{0.660000in}}{\pgfqpoint{4.620000in}{4.620000in}}%
\pgfusepath{clip}%
\pgfsetrectcap%
\pgfsetroundjoin%
\pgfsetlinewidth{1.204500pt}%
\definecolor{currentstroke}{rgb}{1.000000,0.576471,0.309804}%
\pgfsetstrokecolor{currentstroke}%
\pgfsetdash{}{0pt}%
\pgfpathmoveto{\pgfqpoint{4.086145in}{3.908950in}}%
\pgfusepath{stroke}%
\end{pgfscope}%
\begin{pgfscope}%
\pgfpathrectangle{\pgfqpoint{0.765000in}{0.660000in}}{\pgfqpoint{4.620000in}{4.620000in}}%
\pgfusepath{clip}%
\pgfsetbuttcap%
\pgfsetroundjoin%
\definecolor{currentfill}{rgb}{1.000000,0.576471,0.309804}%
\pgfsetfillcolor{currentfill}%
\pgfsetlinewidth{1.003750pt}%
\definecolor{currentstroke}{rgb}{1.000000,0.576471,0.309804}%
\pgfsetstrokecolor{currentstroke}%
\pgfsetdash{}{0pt}%
\pgfsys@defobject{currentmarker}{\pgfqpoint{-0.033333in}{-0.033333in}}{\pgfqpoint{0.033333in}{0.033333in}}{%
\pgfpathmoveto{\pgfqpoint{0.000000in}{-0.033333in}}%
\pgfpathcurveto{\pgfqpoint{0.008840in}{-0.033333in}}{\pgfqpoint{0.017319in}{-0.029821in}}{\pgfqpoint{0.023570in}{-0.023570in}}%
\pgfpathcurveto{\pgfqpoint{0.029821in}{-0.017319in}}{\pgfqpoint{0.033333in}{-0.008840in}}{\pgfqpoint{0.033333in}{0.000000in}}%
\pgfpathcurveto{\pgfqpoint{0.033333in}{0.008840in}}{\pgfqpoint{0.029821in}{0.017319in}}{\pgfqpoint{0.023570in}{0.023570in}}%
\pgfpathcurveto{\pgfqpoint{0.017319in}{0.029821in}}{\pgfqpoint{0.008840in}{0.033333in}}{\pgfqpoint{0.000000in}{0.033333in}}%
\pgfpathcurveto{\pgfqpoint{-0.008840in}{0.033333in}}{\pgfqpoint{-0.017319in}{0.029821in}}{\pgfqpoint{-0.023570in}{0.023570in}}%
\pgfpathcurveto{\pgfqpoint{-0.029821in}{0.017319in}}{\pgfqpoint{-0.033333in}{0.008840in}}{\pgfqpoint{-0.033333in}{0.000000in}}%
\pgfpathcurveto{\pgfqpoint{-0.033333in}{-0.008840in}}{\pgfqpoint{-0.029821in}{-0.017319in}}{\pgfqpoint{-0.023570in}{-0.023570in}}%
\pgfpathcurveto{\pgfqpoint{-0.017319in}{-0.029821in}}{\pgfqpoint{-0.008840in}{-0.033333in}}{\pgfqpoint{0.000000in}{-0.033333in}}%
\pgfpathlineto{\pgfqpoint{0.000000in}{-0.033333in}}%
\pgfpathclose%
\pgfusepath{stroke,fill}%
}%
\begin{pgfscope}%
\pgfsys@transformshift{4.086145in}{3.908950in}%
\pgfsys@useobject{currentmarker}{}%
\end{pgfscope}%
\end{pgfscope}%
\begin{pgfscope}%
\pgfpathrectangle{\pgfqpoint{0.765000in}{0.660000in}}{\pgfqpoint{4.620000in}{4.620000in}}%
\pgfusepath{clip}%
\pgfsetrectcap%
\pgfsetroundjoin%
\pgfsetlinewidth{1.204500pt}%
\definecolor{currentstroke}{rgb}{1.000000,0.576471,0.309804}%
\pgfsetstrokecolor{currentstroke}%
\pgfsetdash{}{0pt}%
\pgfpathmoveto{\pgfqpoint{3.608603in}{3.981476in}}%
\pgfusepath{stroke}%
\end{pgfscope}%
\begin{pgfscope}%
\pgfpathrectangle{\pgfqpoint{0.765000in}{0.660000in}}{\pgfqpoint{4.620000in}{4.620000in}}%
\pgfusepath{clip}%
\pgfsetbuttcap%
\pgfsetroundjoin%
\definecolor{currentfill}{rgb}{1.000000,0.576471,0.309804}%
\pgfsetfillcolor{currentfill}%
\pgfsetlinewidth{1.003750pt}%
\definecolor{currentstroke}{rgb}{1.000000,0.576471,0.309804}%
\pgfsetstrokecolor{currentstroke}%
\pgfsetdash{}{0pt}%
\pgfsys@defobject{currentmarker}{\pgfqpoint{-0.033333in}{-0.033333in}}{\pgfqpoint{0.033333in}{0.033333in}}{%
\pgfpathmoveto{\pgfqpoint{0.000000in}{-0.033333in}}%
\pgfpathcurveto{\pgfqpoint{0.008840in}{-0.033333in}}{\pgfqpoint{0.017319in}{-0.029821in}}{\pgfqpoint{0.023570in}{-0.023570in}}%
\pgfpathcurveto{\pgfqpoint{0.029821in}{-0.017319in}}{\pgfqpoint{0.033333in}{-0.008840in}}{\pgfqpoint{0.033333in}{0.000000in}}%
\pgfpathcurveto{\pgfqpoint{0.033333in}{0.008840in}}{\pgfqpoint{0.029821in}{0.017319in}}{\pgfqpoint{0.023570in}{0.023570in}}%
\pgfpathcurveto{\pgfqpoint{0.017319in}{0.029821in}}{\pgfqpoint{0.008840in}{0.033333in}}{\pgfqpoint{0.000000in}{0.033333in}}%
\pgfpathcurveto{\pgfqpoint{-0.008840in}{0.033333in}}{\pgfqpoint{-0.017319in}{0.029821in}}{\pgfqpoint{-0.023570in}{0.023570in}}%
\pgfpathcurveto{\pgfqpoint{-0.029821in}{0.017319in}}{\pgfqpoint{-0.033333in}{0.008840in}}{\pgfqpoint{-0.033333in}{0.000000in}}%
\pgfpathcurveto{\pgfqpoint{-0.033333in}{-0.008840in}}{\pgfqpoint{-0.029821in}{-0.017319in}}{\pgfqpoint{-0.023570in}{-0.023570in}}%
\pgfpathcurveto{\pgfqpoint{-0.017319in}{-0.029821in}}{\pgfqpoint{-0.008840in}{-0.033333in}}{\pgfqpoint{0.000000in}{-0.033333in}}%
\pgfpathlineto{\pgfqpoint{0.000000in}{-0.033333in}}%
\pgfpathclose%
\pgfusepath{stroke,fill}%
}%
\begin{pgfscope}%
\pgfsys@transformshift{3.608603in}{3.981476in}%
\pgfsys@useobject{currentmarker}{}%
\end{pgfscope}%
\end{pgfscope}%
\begin{pgfscope}%
\pgfpathrectangle{\pgfqpoint{0.765000in}{0.660000in}}{\pgfqpoint{4.620000in}{4.620000in}}%
\pgfusepath{clip}%
\pgfsetrectcap%
\pgfsetroundjoin%
\pgfsetlinewidth{1.204500pt}%
\definecolor{currentstroke}{rgb}{1.000000,0.576471,0.309804}%
\pgfsetstrokecolor{currentstroke}%
\pgfsetdash{}{0pt}%
\pgfpathmoveto{\pgfqpoint{3.891614in}{1.195241in}}%
\pgfusepath{stroke}%
\end{pgfscope}%
\begin{pgfscope}%
\pgfpathrectangle{\pgfqpoint{0.765000in}{0.660000in}}{\pgfqpoint{4.620000in}{4.620000in}}%
\pgfusepath{clip}%
\pgfsetbuttcap%
\pgfsetroundjoin%
\definecolor{currentfill}{rgb}{1.000000,0.576471,0.309804}%
\pgfsetfillcolor{currentfill}%
\pgfsetlinewidth{1.003750pt}%
\definecolor{currentstroke}{rgb}{1.000000,0.576471,0.309804}%
\pgfsetstrokecolor{currentstroke}%
\pgfsetdash{}{0pt}%
\pgfsys@defobject{currentmarker}{\pgfqpoint{-0.033333in}{-0.033333in}}{\pgfqpoint{0.033333in}{0.033333in}}{%
\pgfpathmoveto{\pgfqpoint{0.000000in}{-0.033333in}}%
\pgfpathcurveto{\pgfqpoint{0.008840in}{-0.033333in}}{\pgfqpoint{0.017319in}{-0.029821in}}{\pgfqpoint{0.023570in}{-0.023570in}}%
\pgfpathcurveto{\pgfqpoint{0.029821in}{-0.017319in}}{\pgfqpoint{0.033333in}{-0.008840in}}{\pgfqpoint{0.033333in}{0.000000in}}%
\pgfpathcurveto{\pgfqpoint{0.033333in}{0.008840in}}{\pgfqpoint{0.029821in}{0.017319in}}{\pgfqpoint{0.023570in}{0.023570in}}%
\pgfpathcurveto{\pgfqpoint{0.017319in}{0.029821in}}{\pgfqpoint{0.008840in}{0.033333in}}{\pgfqpoint{0.000000in}{0.033333in}}%
\pgfpathcurveto{\pgfqpoint{-0.008840in}{0.033333in}}{\pgfqpoint{-0.017319in}{0.029821in}}{\pgfqpoint{-0.023570in}{0.023570in}}%
\pgfpathcurveto{\pgfqpoint{-0.029821in}{0.017319in}}{\pgfqpoint{-0.033333in}{0.008840in}}{\pgfqpoint{-0.033333in}{0.000000in}}%
\pgfpathcurveto{\pgfqpoint{-0.033333in}{-0.008840in}}{\pgfqpoint{-0.029821in}{-0.017319in}}{\pgfqpoint{-0.023570in}{-0.023570in}}%
\pgfpathcurveto{\pgfqpoint{-0.017319in}{-0.029821in}}{\pgfqpoint{-0.008840in}{-0.033333in}}{\pgfqpoint{0.000000in}{-0.033333in}}%
\pgfpathlineto{\pgfqpoint{0.000000in}{-0.033333in}}%
\pgfpathclose%
\pgfusepath{stroke,fill}%
}%
\begin{pgfscope}%
\pgfsys@transformshift{3.891614in}{1.195241in}%
\pgfsys@useobject{currentmarker}{}%
\end{pgfscope}%
\end{pgfscope}%
\begin{pgfscope}%
\pgfpathrectangle{\pgfqpoint{0.765000in}{0.660000in}}{\pgfqpoint{4.620000in}{4.620000in}}%
\pgfusepath{clip}%
\pgfsetrectcap%
\pgfsetroundjoin%
\pgfsetlinewidth{1.204500pt}%
\definecolor{currentstroke}{rgb}{1.000000,0.576471,0.309804}%
\pgfsetstrokecolor{currentstroke}%
\pgfsetdash{}{0pt}%
\pgfpathmoveto{\pgfqpoint{2.138241in}{3.457870in}}%
\pgfusepath{stroke}%
\end{pgfscope}%
\begin{pgfscope}%
\pgfpathrectangle{\pgfqpoint{0.765000in}{0.660000in}}{\pgfqpoint{4.620000in}{4.620000in}}%
\pgfusepath{clip}%
\pgfsetbuttcap%
\pgfsetroundjoin%
\definecolor{currentfill}{rgb}{1.000000,0.576471,0.309804}%
\pgfsetfillcolor{currentfill}%
\pgfsetlinewidth{1.003750pt}%
\definecolor{currentstroke}{rgb}{1.000000,0.576471,0.309804}%
\pgfsetstrokecolor{currentstroke}%
\pgfsetdash{}{0pt}%
\pgfsys@defobject{currentmarker}{\pgfqpoint{-0.033333in}{-0.033333in}}{\pgfqpoint{0.033333in}{0.033333in}}{%
\pgfpathmoveto{\pgfqpoint{0.000000in}{-0.033333in}}%
\pgfpathcurveto{\pgfqpoint{0.008840in}{-0.033333in}}{\pgfqpoint{0.017319in}{-0.029821in}}{\pgfqpoint{0.023570in}{-0.023570in}}%
\pgfpathcurveto{\pgfqpoint{0.029821in}{-0.017319in}}{\pgfqpoint{0.033333in}{-0.008840in}}{\pgfqpoint{0.033333in}{0.000000in}}%
\pgfpathcurveto{\pgfqpoint{0.033333in}{0.008840in}}{\pgfqpoint{0.029821in}{0.017319in}}{\pgfqpoint{0.023570in}{0.023570in}}%
\pgfpathcurveto{\pgfqpoint{0.017319in}{0.029821in}}{\pgfqpoint{0.008840in}{0.033333in}}{\pgfqpoint{0.000000in}{0.033333in}}%
\pgfpathcurveto{\pgfqpoint{-0.008840in}{0.033333in}}{\pgfqpoint{-0.017319in}{0.029821in}}{\pgfqpoint{-0.023570in}{0.023570in}}%
\pgfpathcurveto{\pgfqpoint{-0.029821in}{0.017319in}}{\pgfqpoint{-0.033333in}{0.008840in}}{\pgfqpoint{-0.033333in}{0.000000in}}%
\pgfpathcurveto{\pgfqpoint{-0.033333in}{-0.008840in}}{\pgfqpoint{-0.029821in}{-0.017319in}}{\pgfqpoint{-0.023570in}{-0.023570in}}%
\pgfpathcurveto{\pgfqpoint{-0.017319in}{-0.029821in}}{\pgfqpoint{-0.008840in}{-0.033333in}}{\pgfqpoint{0.000000in}{-0.033333in}}%
\pgfpathlineto{\pgfqpoint{0.000000in}{-0.033333in}}%
\pgfpathclose%
\pgfusepath{stroke,fill}%
}%
\begin{pgfscope}%
\pgfsys@transformshift{2.138241in}{3.457870in}%
\pgfsys@useobject{currentmarker}{}%
\end{pgfscope}%
\end{pgfscope}%
\begin{pgfscope}%
\pgfpathrectangle{\pgfqpoint{0.765000in}{0.660000in}}{\pgfqpoint{4.620000in}{4.620000in}}%
\pgfusepath{clip}%
\pgfsetrectcap%
\pgfsetroundjoin%
\pgfsetlinewidth{1.204500pt}%
\definecolor{currentstroke}{rgb}{1.000000,0.576471,0.309804}%
\pgfsetstrokecolor{currentstroke}%
\pgfsetdash{}{0pt}%
\pgfpathmoveto{\pgfqpoint{3.663049in}{2.267237in}}%
\pgfusepath{stroke}%
\end{pgfscope}%
\begin{pgfscope}%
\pgfpathrectangle{\pgfqpoint{0.765000in}{0.660000in}}{\pgfqpoint{4.620000in}{4.620000in}}%
\pgfusepath{clip}%
\pgfsetbuttcap%
\pgfsetroundjoin%
\definecolor{currentfill}{rgb}{1.000000,0.576471,0.309804}%
\pgfsetfillcolor{currentfill}%
\pgfsetlinewidth{1.003750pt}%
\definecolor{currentstroke}{rgb}{1.000000,0.576471,0.309804}%
\pgfsetstrokecolor{currentstroke}%
\pgfsetdash{}{0pt}%
\pgfsys@defobject{currentmarker}{\pgfqpoint{-0.033333in}{-0.033333in}}{\pgfqpoint{0.033333in}{0.033333in}}{%
\pgfpathmoveto{\pgfqpoint{0.000000in}{-0.033333in}}%
\pgfpathcurveto{\pgfqpoint{0.008840in}{-0.033333in}}{\pgfqpoint{0.017319in}{-0.029821in}}{\pgfqpoint{0.023570in}{-0.023570in}}%
\pgfpathcurveto{\pgfqpoint{0.029821in}{-0.017319in}}{\pgfqpoint{0.033333in}{-0.008840in}}{\pgfqpoint{0.033333in}{0.000000in}}%
\pgfpathcurveto{\pgfqpoint{0.033333in}{0.008840in}}{\pgfqpoint{0.029821in}{0.017319in}}{\pgfqpoint{0.023570in}{0.023570in}}%
\pgfpathcurveto{\pgfqpoint{0.017319in}{0.029821in}}{\pgfqpoint{0.008840in}{0.033333in}}{\pgfqpoint{0.000000in}{0.033333in}}%
\pgfpathcurveto{\pgfqpoint{-0.008840in}{0.033333in}}{\pgfqpoint{-0.017319in}{0.029821in}}{\pgfqpoint{-0.023570in}{0.023570in}}%
\pgfpathcurveto{\pgfqpoint{-0.029821in}{0.017319in}}{\pgfqpoint{-0.033333in}{0.008840in}}{\pgfqpoint{-0.033333in}{0.000000in}}%
\pgfpathcurveto{\pgfqpoint{-0.033333in}{-0.008840in}}{\pgfqpoint{-0.029821in}{-0.017319in}}{\pgfqpoint{-0.023570in}{-0.023570in}}%
\pgfpathcurveto{\pgfqpoint{-0.017319in}{-0.029821in}}{\pgfqpoint{-0.008840in}{-0.033333in}}{\pgfqpoint{0.000000in}{-0.033333in}}%
\pgfpathlineto{\pgfqpoint{0.000000in}{-0.033333in}}%
\pgfpathclose%
\pgfusepath{stroke,fill}%
}%
\begin{pgfscope}%
\pgfsys@transformshift{3.663049in}{2.267237in}%
\pgfsys@useobject{currentmarker}{}%
\end{pgfscope}%
\end{pgfscope}%
\begin{pgfscope}%
\pgfpathrectangle{\pgfqpoint{0.765000in}{0.660000in}}{\pgfqpoint{4.620000in}{4.620000in}}%
\pgfusepath{clip}%
\pgfsetrectcap%
\pgfsetroundjoin%
\pgfsetlinewidth{1.204500pt}%
\definecolor{currentstroke}{rgb}{1.000000,0.576471,0.309804}%
\pgfsetstrokecolor{currentstroke}%
\pgfsetdash{}{0pt}%
\pgfpathmoveto{\pgfqpoint{3.275168in}{3.921300in}}%
\pgfusepath{stroke}%
\end{pgfscope}%
\begin{pgfscope}%
\pgfpathrectangle{\pgfqpoint{0.765000in}{0.660000in}}{\pgfqpoint{4.620000in}{4.620000in}}%
\pgfusepath{clip}%
\pgfsetbuttcap%
\pgfsetroundjoin%
\definecolor{currentfill}{rgb}{1.000000,0.576471,0.309804}%
\pgfsetfillcolor{currentfill}%
\pgfsetlinewidth{1.003750pt}%
\definecolor{currentstroke}{rgb}{1.000000,0.576471,0.309804}%
\pgfsetstrokecolor{currentstroke}%
\pgfsetdash{}{0pt}%
\pgfsys@defobject{currentmarker}{\pgfqpoint{-0.033333in}{-0.033333in}}{\pgfqpoint{0.033333in}{0.033333in}}{%
\pgfpathmoveto{\pgfqpoint{0.000000in}{-0.033333in}}%
\pgfpathcurveto{\pgfqpoint{0.008840in}{-0.033333in}}{\pgfqpoint{0.017319in}{-0.029821in}}{\pgfqpoint{0.023570in}{-0.023570in}}%
\pgfpathcurveto{\pgfqpoint{0.029821in}{-0.017319in}}{\pgfqpoint{0.033333in}{-0.008840in}}{\pgfqpoint{0.033333in}{0.000000in}}%
\pgfpathcurveto{\pgfqpoint{0.033333in}{0.008840in}}{\pgfqpoint{0.029821in}{0.017319in}}{\pgfqpoint{0.023570in}{0.023570in}}%
\pgfpathcurveto{\pgfqpoint{0.017319in}{0.029821in}}{\pgfqpoint{0.008840in}{0.033333in}}{\pgfqpoint{0.000000in}{0.033333in}}%
\pgfpathcurveto{\pgfqpoint{-0.008840in}{0.033333in}}{\pgfqpoint{-0.017319in}{0.029821in}}{\pgfqpoint{-0.023570in}{0.023570in}}%
\pgfpathcurveto{\pgfqpoint{-0.029821in}{0.017319in}}{\pgfqpoint{-0.033333in}{0.008840in}}{\pgfqpoint{-0.033333in}{0.000000in}}%
\pgfpathcurveto{\pgfqpoint{-0.033333in}{-0.008840in}}{\pgfqpoint{-0.029821in}{-0.017319in}}{\pgfqpoint{-0.023570in}{-0.023570in}}%
\pgfpathcurveto{\pgfqpoint{-0.017319in}{-0.029821in}}{\pgfqpoint{-0.008840in}{-0.033333in}}{\pgfqpoint{0.000000in}{-0.033333in}}%
\pgfpathlineto{\pgfqpoint{0.000000in}{-0.033333in}}%
\pgfpathclose%
\pgfusepath{stroke,fill}%
}%
\begin{pgfscope}%
\pgfsys@transformshift{3.275168in}{3.921300in}%
\pgfsys@useobject{currentmarker}{}%
\end{pgfscope}%
\end{pgfscope}%
\begin{pgfscope}%
\pgfpathrectangle{\pgfqpoint{0.765000in}{0.660000in}}{\pgfqpoint{4.620000in}{4.620000in}}%
\pgfusepath{clip}%
\pgfsetrectcap%
\pgfsetroundjoin%
\pgfsetlinewidth{1.204500pt}%
\definecolor{currentstroke}{rgb}{1.000000,0.576471,0.309804}%
\pgfsetstrokecolor{currentstroke}%
\pgfsetdash{}{0pt}%
\pgfpathmoveto{\pgfqpoint{3.544302in}{2.377750in}}%
\pgfusepath{stroke}%
\end{pgfscope}%
\begin{pgfscope}%
\pgfpathrectangle{\pgfqpoint{0.765000in}{0.660000in}}{\pgfqpoint{4.620000in}{4.620000in}}%
\pgfusepath{clip}%
\pgfsetbuttcap%
\pgfsetroundjoin%
\definecolor{currentfill}{rgb}{1.000000,0.576471,0.309804}%
\pgfsetfillcolor{currentfill}%
\pgfsetlinewidth{1.003750pt}%
\definecolor{currentstroke}{rgb}{1.000000,0.576471,0.309804}%
\pgfsetstrokecolor{currentstroke}%
\pgfsetdash{}{0pt}%
\pgfsys@defobject{currentmarker}{\pgfqpoint{-0.033333in}{-0.033333in}}{\pgfqpoint{0.033333in}{0.033333in}}{%
\pgfpathmoveto{\pgfqpoint{0.000000in}{-0.033333in}}%
\pgfpathcurveto{\pgfqpoint{0.008840in}{-0.033333in}}{\pgfqpoint{0.017319in}{-0.029821in}}{\pgfqpoint{0.023570in}{-0.023570in}}%
\pgfpathcurveto{\pgfqpoint{0.029821in}{-0.017319in}}{\pgfqpoint{0.033333in}{-0.008840in}}{\pgfqpoint{0.033333in}{0.000000in}}%
\pgfpathcurveto{\pgfqpoint{0.033333in}{0.008840in}}{\pgfqpoint{0.029821in}{0.017319in}}{\pgfqpoint{0.023570in}{0.023570in}}%
\pgfpathcurveto{\pgfqpoint{0.017319in}{0.029821in}}{\pgfqpoint{0.008840in}{0.033333in}}{\pgfqpoint{0.000000in}{0.033333in}}%
\pgfpathcurveto{\pgfqpoint{-0.008840in}{0.033333in}}{\pgfqpoint{-0.017319in}{0.029821in}}{\pgfqpoint{-0.023570in}{0.023570in}}%
\pgfpathcurveto{\pgfqpoint{-0.029821in}{0.017319in}}{\pgfqpoint{-0.033333in}{0.008840in}}{\pgfqpoint{-0.033333in}{0.000000in}}%
\pgfpathcurveto{\pgfqpoint{-0.033333in}{-0.008840in}}{\pgfqpoint{-0.029821in}{-0.017319in}}{\pgfqpoint{-0.023570in}{-0.023570in}}%
\pgfpathcurveto{\pgfqpoint{-0.017319in}{-0.029821in}}{\pgfqpoint{-0.008840in}{-0.033333in}}{\pgfqpoint{0.000000in}{-0.033333in}}%
\pgfpathlineto{\pgfqpoint{0.000000in}{-0.033333in}}%
\pgfpathclose%
\pgfusepath{stroke,fill}%
}%
\begin{pgfscope}%
\pgfsys@transformshift{3.544302in}{2.377750in}%
\pgfsys@useobject{currentmarker}{}%
\end{pgfscope}%
\end{pgfscope}%
\begin{pgfscope}%
\pgfpathrectangle{\pgfqpoint{0.765000in}{0.660000in}}{\pgfqpoint{4.620000in}{4.620000in}}%
\pgfusepath{clip}%
\pgfsetrectcap%
\pgfsetroundjoin%
\pgfsetlinewidth{1.204500pt}%
\definecolor{currentstroke}{rgb}{1.000000,0.576471,0.309804}%
\pgfsetstrokecolor{currentstroke}%
\pgfsetdash{}{0pt}%
\pgfpathmoveto{\pgfqpoint{3.835229in}{2.523208in}}%
\pgfusepath{stroke}%
\end{pgfscope}%
\begin{pgfscope}%
\pgfpathrectangle{\pgfqpoint{0.765000in}{0.660000in}}{\pgfqpoint{4.620000in}{4.620000in}}%
\pgfusepath{clip}%
\pgfsetbuttcap%
\pgfsetroundjoin%
\definecolor{currentfill}{rgb}{1.000000,0.576471,0.309804}%
\pgfsetfillcolor{currentfill}%
\pgfsetlinewidth{1.003750pt}%
\definecolor{currentstroke}{rgb}{1.000000,0.576471,0.309804}%
\pgfsetstrokecolor{currentstroke}%
\pgfsetdash{}{0pt}%
\pgfsys@defobject{currentmarker}{\pgfqpoint{-0.033333in}{-0.033333in}}{\pgfqpoint{0.033333in}{0.033333in}}{%
\pgfpathmoveto{\pgfqpoint{0.000000in}{-0.033333in}}%
\pgfpathcurveto{\pgfqpoint{0.008840in}{-0.033333in}}{\pgfqpoint{0.017319in}{-0.029821in}}{\pgfqpoint{0.023570in}{-0.023570in}}%
\pgfpathcurveto{\pgfqpoint{0.029821in}{-0.017319in}}{\pgfqpoint{0.033333in}{-0.008840in}}{\pgfqpoint{0.033333in}{0.000000in}}%
\pgfpathcurveto{\pgfqpoint{0.033333in}{0.008840in}}{\pgfqpoint{0.029821in}{0.017319in}}{\pgfqpoint{0.023570in}{0.023570in}}%
\pgfpathcurveto{\pgfqpoint{0.017319in}{0.029821in}}{\pgfqpoint{0.008840in}{0.033333in}}{\pgfqpoint{0.000000in}{0.033333in}}%
\pgfpathcurveto{\pgfqpoint{-0.008840in}{0.033333in}}{\pgfqpoint{-0.017319in}{0.029821in}}{\pgfqpoint{-0.023570in}{0.023570in}}%
\pgfpathcurveto{\pgfqpoint{-0.029821in}{0.017319in}}{\pgfqpoint{-0.033333in}{0.008840in}}{\pgfqpoint{-0.033333in}{0.000000in}}%
\pgfpathcurveto{\pgfqpoint{-0.033333in}{-0.008840in}}{\pgfqpoint{-0.029821in}{-0.017319in}}{\pgfqpoint{-0.023570in}{-0.023570in}}%
\pgfpathcurveto{\pgfqpoint{-0.017319in}{-0.029821in}}{\pgfqpoint{-0.008840in}{-0.033333in}}{\pgfqpoint{0.000000in}{-0.033333in}}%
\pgfpathlineto{\pgfqpoint{0.000000in}{-0.033333in}}%
\pgfpathclose%
\pgfusepath{stroke,fill}%
}%
\begin{pgfscope}%
\pgfsys@transformshift{3.835229in}{2.523208in}%
\pgfsys@useobject{currentmarker}{}%
\end{pgfscope}%
\end{pgfscope}%
\begin{pgfscope}%
\pgfpathrectangle{\pgfqpoint{0.765000in}{0.660000in}}{\pgfqpoint{4.620000in}{4.620000in}}%
\pgfusepath{clip}%
\pgfsetrectcap%
\pgfsetroundjoin%
\pgfsetlinewidth{1.204500pt}%
\definecolor{currentstroke}{rgb}{1.000000,0.576471,0.309804}%
\pgfsetstrokecolor{currentstroke}%
\pgfsetdash{}{0pt}%
\pgfpathmoveto{\pgfqpoint{2.970821in}{1.951121in}}%
\pgfusepath{stroke}%
\end{pgfscope}%
\begin{pgfscope}%
\pgfpathrectangle{\pgfqpoint{0.765000in}{0.660000in}}{\pgfqpoint{4.620000in}{4.620000in}}%
\pgfusepath{clip}%
\pgfsetbuttcap%
\pgfsetroundjoin%
\definecolor{currentfill}{rgb}{1.000000,0.576471,0.309804}%
\pgfsetfillcolor{currentfill}%
\pgfsetlinewidth{1.003750pt}%
\definecolor{currentstroke}{rgb}{1.000000,0.576471,0.309804}%
\pgfsetstrokecolor{currentstroke}%
\pgfsetdash{}{0pt}%
\pgfsys@defobject{currentmarker}{\pgfqpoint{-0.033333in}{-0.033333in}}{\pgfqpoint{0.033333in}{0.033333in}}{%
\pgfpathmoveto{\pgfqpoint{0.000000in}{-0.033333in}}%
\pgfpathcurveto{\pgfqpoint{0.008840in}{-0.033333in}}{\pgfqpoint{0.017319in}{-0.029821in}}{\pgfqpoint{0.023570in}{-0.023570in}}%
\pgfpathcurveto{\pgfqpoint{0.029821in}{-0.017319in}}{\pgfqpoint{0.033333in}{-0.008840in}}{\pgfqpoint{0.033333in}{0.000000in}}%
\pgfpathcurveto{\pgfqpoint{0.033333in}{0.008840in}}{\pgfqpoint{0.029821in}{0.017319in}}{\pgfqpoint{0.023570in}{0.023570in}}%
\pgfpathcurveto{\pgfqpoint{0.017319in}{0.029821in}}{\pgfqpoint{0.008840in}{0.033333in}}{\pgfqpoint{0.000000in}{0.033333in}}%
\pgfpathcurveto{\pgfqpoint{-0.008840in}{0.033333in}}{\pgfqpoint{-0.017319in}{0.029821in}}{\pgfqpoint{-0.023570in}{0.023570in}}%
\pgfpathcurveto{\pgfqpoint{-0.029821in}{0.017319in}}{\pgfqpoint{-0.033333in}{0.008840in}}{\pgfqpoint{-0.033333in}{0.000000in}}%
\pgfpathcurveto{\pgfqpoint{-0.033333in}{-0.008840in}}{\pgfqpoint{-0.029821in}{-0.017319in}}{\pgfqpoint{-0.023570in}{-0.023570in}}%
\pgfpathcurveto{\pgfqpoint{-0.017319in}{-0.029821in}}{\pgfqpoint{-0.008840in}{-0.033333in}}{\pgfqpoint{0.000000in}{-0.033333in}}%
\pgfpathlineto{\pgfqpoint{0.000000in}{-0.033333in}}%
\pgfpathclose%
\pgfusepath{stroke,fill}%
}%
\begin{pgfscope}%
\pgfsys@transformshift{2.970821in}{1.951121in}%
\pgfsys@useobject{currentmarker}{}%
\end{pgfscope}%
\end{pgfscope}%
\begin{pgfscope}%
\pgfpathrectangle{\pgfqpoint{0.765000in}{0.660000in}}{\pgfqpoint{4.620000in}{4.620000in}}%
\pgfusepath{clip}%
\pgfsetrectcap%
\pgfsetroundjoin%
\pgfsetlinewidth{1.204500pt}%
\definecolor{currentstroke}{rgb}{1.000000,0.576471,0.309804}%
\pgfsetstrokecolor{currentstroke}%
\pgfsetdash{}{0pt}%
\pgfpathmoveto{\pgfqpoint{3.170345in}{4.403195in}}%
\pgfusepath{stroke}%
\end{pgfscope}%
\begin{pgfscope}%
\pgfpathrectangle{\pgfqpoint{0.765000in}{0.660000in}}{\pgfqpoint{4.620000in}{4.620000in}}%
\pgfusepath{clip}%
\pgfsetbuttcap%
\pgfsetroundjoin%
\definecolor{currentfill}{rgb}{1.000000,0.576471,0.309804}%
\pgfsetfillcolor{currentfill}%
\pgfsetlinewidth{1.003750pt}%
\definecolor{currentstroke}{rgb}{1.000000,0.576471,0.309804}%
\pgfsetstrokecolor{currentstroke}%
\pgfsetdash{}{0pt}%
\pgfsys@defobject{currentmarker}{\pgfqpoint{-0.033333in}{-0.033333in}}{\pgfqpoint{0.033333in}{0.033333in}}{%
\pgfpathmoveto{\pgfqpoint{0.000000in}{-0.033333in}}%
\pgfpathcurveto{\pgfqpoint{0.008840in}{-0.033333in}}{\pgfqpoint{0.017319in}{-0.029821in}}{\pgfqpoint{0.023570in}{-0.023570in}}%
\pgfpathcurveto{\pgfqpoint{0.029821in}{-0.017319in}}{\pgfqpoint{0.033333in}{-0.008840in}}{\pgfqpoint{0.033333in}{0.000000in}}%
\pgfpathcurveto{\pgfqpoint{0.033333in}{0.008840in}}{\pgfqpoint{0.029821in}{0.017319in}}{\pgfqpoint{0.023570in}{0.023570in}}%
\pgfpathcurveto{\pgfqpoint{0.017319in}{0.029821in}}{\pgfqpoint{0.008840in}{0.033333in}}{\pgfqpoint{0.000000in}{0.033333in}}%
\pgfpathcurveto{\pgfqpoint{-0.008840in}{0.033333in}}{\pgfqpoint{-0.017319in}{0.029821in}}{\pgfqpoint{-0.023570in}{0.023570in}}%
\pgfpathcurveto{\pgfqpoint{-0.029821in}{0.017319in}}{\pgfqpoint{-0.033333in}{0.008840in}}{\pgfqpoint{-0.033333in}{0.000000in}}%
\pgfpathcurveto{\pgfqpoint{-0.033333in}{-0.008840in}}{\pgfqpoint{-0.029821in}{-0.017319in}}{\pgfqpoint{-0.023570in}{-0.023570in}}%
\pgfpathcurveto{\pgfqpoint{-0.017319in}{-0.029821in}}{\pgfqpoint{-0.008840in}{-0.033333in}}{\pgfqpoint{0.000000in}{-0.033333in}}%
\pgfpathlineto{\pgfqpoint{0.000000in}{-0.033333in}}%
\pgfpathclose%
\pgfusepath{stroke,fill}%
}%
\begin{pgfscope}%
\pgfsys@transformshift{3.170345in}{4.403195in}%
\pgfsys@useobject{currentmarker}{}%
\end{pgfscope}%
\end{pgfscope}%
\begin{pgfscope}%
\pgfpathrectangle{\pgfqpoint{0.765000in}{0.660000in}}{\pgfqpoint{4.620000in}{4.620000in}}%
\pgfusepath{clip}%
\pgfsetrectcap%
\pgfsetroundjoin%
\pgfsetlinewidth{1.204500pt}%
\definecolor{currentstroke}{rgb}{1.000000,0.576471,0.309804}%
\pgfsetstrokecolor{currentstroke}%
\pgfsetdash{}{0pt}%
\pgfpathmoveto{\pgfqpoint{4.230410in}{3.478023in}}%
\pgfusepath{stroke}%
\end{pgfscope}%
\begin{pgfscope}%
\pgfpathrectangle{\pgfqpoint{0.765000in}{0.660000in}}{\pgfqpoint{4.620000in}{4.620000in}}%
\pgfusepath{clip}%
\pgfsetbuttcap%
\pgfsetroundjoin%
\definecolor{currentfill}{rgb}{1.000000,0.576471,0.309804}%
\pgfsetfillcolor{currentfill}%
\pgfsetlinewidth{1.003750pt}%
\definecolor{currentstroke}{rgb}{1.000000,0.576471,0.309804}%
\pgfsetstrokecolor{currentstroke}%
\pgfsetdash{}{0pt}%
\pgfsys@defobject{currentmarker}{\pgfqpoint{-0.033333in}{-0.033333in}}{\pgfqpoint{0.033333in}{0.033333in}}{%
\pgfpathmoveto{\pgfqpoint{0.000000in}{-0.033333in}}%
\pgfpathcurveto{\pgfqpoint{0.008840in}{-0.033333in}}{\pgfqpoint{0.017319in}{-0.029821in}}{\pgfqpoint{0.023570in}{-0.023570in}}%
\pgfpathcurveto{\pgfqpoint{0.029821in}{-0.017319in}}{\pgfqpoint{0.033333in}{-0.008840in}}{\pgfqpoint{0.033333in}{0.000000in}}%
\pgfpathcurveto{\pgfqpoint{0.033333in}{0.008840in}}{\pgfqpoint{0.029821in}{0.017319in}}{\pgfqpoint{0.023570in}{0.023570in}}%
\pgfpathcurveto{\pgfqpoint{0.017319in}{0.029821in}}{\pgfqpoint{0.008840in}{0.033333in}}{\pgfqpoint{0.000000in}{0.033333in}}%
\pgfpathcurveto{\pgfqpoint{-0.008840in}{0.033333in}}{\pgfqpoint{-0.017319in}{0.029821in}}{\pgfqpoint{-0.023570in}{0.023570in}}%
\pgfpathcurveto{\pgfqpoint{-0.029821in}{0.017319in}}{\pgfqpoint{-0.033333in}{0.008840in}}{\pgfqpoint{-0.033333in}{0.000000in}}%
\pgfpathcurveto{\pgfqpoint{-0.033333in}{-0.008840in}}{\pgfqpoint{-0.029821in}{-0.017319in}}{\pgfqpoint{-0.023570in}{-0.023570in}}%
\pgfpathcurveto{\pgfqpoint{-0.017319in}{-0.029821in}}{\pgfqpoint{-0.008840in}{-0.033333in}}{\pgfqpoint{0.000000in}{-0.033333in}}%
\pgfpathlineto{\pgfqpoint{0.000000in}{-0.033333in}}%
\pgfpathclose%
\pgfusepath{stroke,fill}%
}%
\begin{pgfscope}%
\pgfsys@transformshift{4.230410in}{3.478023in}%
\pgfsys@useobject{currentmarker}{}%
\end{pgfscope}%
\end{pgfscope}%
\begin{pgfscope}%
\pgfpathrectangle{\pgfqpoint{0.765000in}{0.660000in}}{\pgfqpoint{4.620000in}{4.620000in}}%
\pgfusepath{clip}%
\pgfsetrectcap%
\pgfsetroundjoin%
\pgfsetlinewidth{1.204500pt}%
\definecolor{currentstroke}{rgb}{1.000000,0.576471,0.309804}%
\pgfsetstrokecolor{currentstroke}%
\pgfsetdash{}{0pt}%
\pgfpathmoveto{\pgfqpoint{2.318786in}{3.861573in}}%
\pgfusepath{stroke}%
\end{pgfscope}%
\begin{pgfscope}%
\pgfpathrectangle{\pgfqpoint{0.765000in}{0.660000in}}{\pgfqpoint{4.620000in}{4.620000in}}%
\pgfusepath{clip}%
\pgfsetbuttcap%
\pgfsetroundjoin%
\definecolor{currentfill}{rgb}{1.000000,0.576471,0.309804}%
\pgfsetfillcolor{currentfill}%
\pgfsetlinewidth{1.003750pt}%
\definecolor{currentstroke}{rgb}{1.000000,0.576471,0.309804}%
\pgfsetstrokecolor{currentstroke}%
\pgfsetdash{}{0pt}%
\pgfsys@defobject{currentmarker}{\pgfqpoint{-0.033333in}{-0.033333in}}{\pgfqpoint{0.033333in}{0.033333in}}{%
\pgfpathmoveto{\pgfqpoint{0.000000in}{-0.033333in}}%
\pgfpathcurveto{\pgfqpoint{0.008840in}{-0.033333in}}{\pgfqpoint{0.017319in}{-0.029821in}}{\pgfqpoint{0.023570in}{-0.023570in}}%
\pgfpathcurveto{\pgfqpoint{0.029821in}{-0.017319in}}{\pgfqpoint{0.033333in}{-0.008840in}}{\pgfqpoint{0.033333in}{0.000000in}}%
\pgfpathcurveto{\pgfqpoint{0.033333in}{0.008840in}}{\pgfqpoint{0.029821in}{0.017319in}}{\pgfqpoint{0.023570in}{0.023570in}}%
\pgfpathcurveto{\pgfqpoint{0.017319in}{0.029821in}}{\pgfqpoint{0.008840in}{0.033333in}}{\pgfqpoint{0.000000in}{0.033333in}}%
\pgfpathcurveto{\pgfqpoint{-0.008840in}{0.033333in}}{\pgfqpoint{-0.017319in}{0.029821in}}{\pgfqpoint{-0.023570in}{0.023570in}}%
\pgfpathcurveto{\pgfqpoint{-0.029821in}{0.017319in}}{\pgfqpoint{-0.033333in}{0.008840in}}{\pgfqpoint{-0.033333in}{0.000000in}}%
\pgfpathcurveto{\pgfqpoint{-0.033333in}{-0.008840in}}{\pgfqpoint{-0.029821in}{-0.017319in}}{\pgfqpoint{-0.023570in}{-0.023570in}}%
\pgfpathcurveto{\pgfqpoint{-0.017319in}{-0.029821in}}{\pgfqpoint{-0.008840in}{-0.033333in}}{\pgfqpoint{0.000000in}{-0.033333in}}%
\pgfpathlineto{\pgfqpoint{0.000000in}{-0.033333in}}%
\pgfpathclose%
\pgfusepath{stroke,fill}%
}%
\begin{pgfscope}%
\pgfsys@transformshift{2.318786in}{3.861573in}%
\pgfsys@useobject{currentmarker}{}%
\end{pgfscope}%
\end{pgfscope}%
\begin{pgfscope}%
\pgfpathrectangle{\pgfqpoint{0.765000in}{0.660000in}}{\pgfqpoint{4.620000in}{4.620000in}}%
\pgfusepath{clip}%
\pgfsetrectcap%
\pgfsetroundjoin%
\pgfsetlinewidth{1.204500pt}%
\definecolor{currentstroke}{rgb}{1.000000,0.576471,0.309804}%
\pgfsetstrokecolor{currentstroke}%
\pgfsetdash{}{0pt}%
\pgfpathmoveto{\pgfqpoint{4.076604in}{4.486588in}}%
\pgfusepath{stroke}%
\end{pgfscope}%
\begin{pgfscope}%
\pgfpathrectangle{\pgfqpoint{0.765000in}{0.660000in}}{\pgfqpoint{4.620000in}{4.620000in}}%
\pgfusepath{clip}%
\pgfsetbuttcap%
\pgfsetroundjoin%
\definecolor{currentfill}{rgb}{1.000000,0.576471,0.309804}%
\pgfsetfillcolor{currentfill}%
\pgfsetlinewidth{1.003750pt}%
\definecolor{currentstroke}{rgb}{1.000000,0.576471,0.309804}%
\pgfsetstrokecolor{currentstroke}%
\pgfsetdash{}{0pt}%
\pgfsys@defobject{currentmarker}{\pgfqpoint{-0.033333in}{-0.033333in}}{\pgfqpoint{0.033333in}{0.033333in}}{%
\pgfpathmoveto{\pgfqpoint{0.000000in}{-0.033333in}}%
\pgfpathcurveto{\pgfqpoint{0.008840in}{-0.033333in}}{\pgfqpoint{0.017319in}{-0.029821in}}{\pgfqpoint{0.023570in}{-0.023570in}}%
\pgfpathcurveto{\pgfqpoint{0.029821in}{-0.017319in}}{\pgfqpoint{0.033333in}{-0.008840in}}{\pgfqpoint{0.033333in}{0.000000in}}%
\pgfpathcurveto{\pgfqpoint{0.033333in}{0.008840in}}{\pgfqpoint{0.029821in}{0.017319in}}{\pgfqpoint{0.023570in}{0.023570in}}%
\pgfpathcurveto{\pgfqpoint{0.017319in}{0.029821in}}{\pgfqpoint{0.008840in}{0.033333in}}{\pgfqpoint{0.000000in}{0.033333in}}%
\pgfpathcurveto{\pgfqpoint{-0.008840in}{0.033333in}}{\pgfqpoint{-0.017319in}{0.029821in}}{\pgfqpoint{-0.023570in}{0.023570in}}%
\pgfpathcurveto{\pgfqpoint{-0.029821in}{0.017319in}}{\pgfqpoint{-0.033333in}{0.008840in}}{\pgfqpoint{-0.033333in}{0.000000in}}%
\pgfpathcurveto{\pgfqpoint{-0.033333in}{-0.008840in}}{\pgfqpoint{-0.029821in}{-0.017319in}}{\pgfqpoint{-0.023570in}{-0.023570in}}%
\pgfpathcurveto{\pgfqpoint{-0.017319in}{-0.029821in}}{\pgfqpoint{-0.008840in}{-0.033333in}}{\pgfqpoint{0.000000in}{-0.033333in}}%
\pgfpathlineto{\pgfqpoint{0.000000in}{-0.033333in}}%
\pgfpathclose%
\pgfusepath{stroke,fill}%
}%
\begin{pgfscope}%
\pgfsys@transformshift{4.076604in}{4.486588in}%
\pgfsys@useobject{currentmarker}{}%
\end{pgfscope}%
\end{pgfscope}%
\begin{pgfscope}%
\pgfpathrectangle{\pgfqpoint{0.765000in}{0.660000in}}{\pgfqpoint{4.620000in}{4.620000in}}%
\pgfusepath{clip}%
\pgfsetrectcap%
\pgfsetroundjoin%
\pgfsetlinewidth{1.204500pt}%
\definecolor{currentstroke}{rgb}{1.000000,0.576471,0.309804}%
\pgfsetstrokecolor{currentstroke}%
\pgfsetdash{}{0pt}%
\pgfpathmoveto{\pgfqpoint{2.630042in}{3.660013in}}%
\pgfusepath{stroke}%
\end{pgfscope}%
\begin{pgfscope}%
\pgfpathrectangle{\pgfqpoint{0.765000in}{0.660000in}}{\pgfqpoint{4.620000in}{4.620000in}}%
\pgfusepath{clip}%
\pgfsetbuttcap%
\pgfsetroundjoin%
\definecolor{currentfill}{rgb}{1.000000,0.576471,0.309804}%
\pgfsetfillcolor{currentfill}%
\pgfsetlinewidth{1.003750pt}%
\definecolor{currentstroke}{rgb}{1.000000,0.576471,0.309804}%
\pgfsetstrokecolor{currentstroke}%
\pgfsetdash{}{0pt}%
\pgfsys@defobject{currentmarker}{\pgfqpoint{-0.033333in}{-0.033333in}}{\pgfqpoint{0.033333in}{0.033333in}}{%
\pgfpathmoveto{\pgfqpoint{0.000000in}{-0.033333in}}%
\pgfpathcurveto{\pgfqpoint{0.008840in}{-0.033333in}}{\pgfqpoint{0.017319in}{-0.029821in}}{\pgfqpoint{0.023570in}{-0.023570in}}%
\pgfpathcurveto{\pgfqpoint{0.029821in}{-0.017319in}}{\pgfqpoint{0.033333in}{-0.008840in}}{\pgfqpoint{0.033333in}{0.000000in}}%
\pgfpathcurveto{\pgfqpoint{0.033333in}{0.008840in}}{\pgfqpoint{0.029821in}{0.017319in}}{\pgfqpoint{0.023570in}{0.023570in}}%
\pgfpathcurveto{\pgfqpoint{0.017319in}{0.029821in}}{\pgfqpoint{0.008840in}{0.033333in}}{\pgfqpoint{0.000000in}{0.033333in}}%
\pgfpathcurveto{\pgfqpoint{-0.008840in}{0.033333in}}{\pgfqpoint{-0.017319in}{0.029821in}}{\pgfqpoint{-0.023570in}{0.023570in}}%
\pgfpathcurveto{\pgfqpoint{-0.029821in}{0.017319in}}{\pgfqpoint{-0.033333in}{0.008840in}}{\pgfqpoint{-0.033333in}{0.000000in}}%
\pgfpathcurveto{\pgfqpoint{-0.033333in}{-0.008840in}}{\pgfqpoint{-0.029821in}{-0.017319in}}{\pgfqpoint{-0.023570in}{-0.023570in}}%
\pgfpathcurveto{\pgfqpoint{-0.017319in}{-0.029821in}}{\pgfqpoint{-0.008840in}{-0.033333in}}{\pgfqpoint{0.000000in}{-0.033333in}}%
\pgfpathlineto{\pgfqpoint{0.000000in}{-0.033333in}}%
\pgfpathclose%
\pgfusepath{stroke,fill}%
}%
\begin{pgfscope}%
\pgfsys@transformshift{2.630042in}{3.660013in}%
\pgfsys@useobject{currentmarker}{}%
\end{pgfscope}%
\end{pgfscope}%
\begin{pgfscope}%
\pgfpathrectangle{\pgfqpoint{0.765000in}{0.660000in}}{\pgfqpoint{4.620000in}{4.620000in}}%
\pgfusepath{clip}%
\pgfsetrectcap%
\pgfsetroundjoin%
\pgfsetlinewidth{1.204500pt}%
\definecolor{currentstroke}{rgb}{1.000000,0.576471,0.309804}%
\pgfsetstrokecolor{currentstroke}%
\pgfsetdash{}{0pt}%
\pgfpathmoveto{\pgfqpoint{3.860845in}{2.713286in}}%
\pgfusepath{stroke}%
\end{pgfscope}%
\begin{pgfscope}%
\pgfpathrectangle{\pgfqpoint{0.765000in}{0.660000in}}{\pgfqpoint{4.620000in}{4.620000in}}%
\pgfusepath{clip}%
\pgfsetbuttcap%
\pgfsetroundjoin%
\definecolor{currentfill}{rgb}{1.000000,0.576471,0.309804}%
\pgfsetfillcolor{currentfill}%
\pgfsetlinewidth{1.003750pt}%
\definecolor{currentstroke}{rgb}{1.000000,0.576471,0.309804}%
\pgfsetstrokecolor{currentstroke}%
\pgfsetdash{}{0pt}%
\pgfsys@defobject{currentmarker}{\pgfqpoint{-0.033333in}{-0.033333in}}{\pgfqpoint{0.033333in}{0.033333in}}{%
\pgfpathmoveto{\pgfqpoint{0.000000in}{-0.033333in}}%
\pgfpathcurveto{\pgfqpoint{0.008840in}{-0.033333in}}{\pgfqpoint{0.017319in}{-0.029821in}}{\pgfqpoint{0.023570in}{-0.023570in}}%
\pgfpathcurveto{\pgfqpoint{0.029821in}{-0.017319in}}{\pgfqpoint{0.033333in}{-0.008840in}}{\pgfqpoint{0.033333in}{0.000000in}}%
\pgfpathcurveto{\pgfqpoint{0.033333in}{0.008840in}}{\pgfqpoint{0.029821in}{0.017319in}}{\pgfqpoint{0.023570in}{0.023570in}}%
\pgfpathcurveto{\pgfqpoint{0.017319in}{0.029821in}}{\pgfqpoint{0.008840in}{0.033333in}}{\pgfqpoint{0.000000in}{0.033333in}}%
\pgfpathcurveto{\pgfqpoint{-0.008840in}{0.033333in}}{\pgfqpoint{-0.017319in}{0.029821in}}{\pgfqpoint{-0.023570in}{0.023570in}}%
\pgfpathcurveto{\pgfqpoint{-0.029821in}{0.017319in}}{\pgfqpoint{-0.033333in}{0.008840in}}{\pgfqpoint{-0.033333in}{0.000000in}}%
\pgfpathcurveto{\pgfqpoint{-0.033333in}{-0.008840in}}{\pgfqpoint{-0.029821in}{-0.017319in}}{\pgfqpoint{-0.023570in}{-0.023570in}}%
\pgfpathcurveto{\pgfqpoint{-0.017319in}{-0.029821in}}{\pgfqpoint{-0.008840in}{-0.033333in}}{\pgfqpoint{0.000000in}{-0.033333in}}%
\pgfpathlineto{\pgfqpoint{0.000000in}{-0.033333in}}%
\pgfpathclose%
\pgfusepath{stroke,fill}%
}%
\begin{pgfscope}%
\pgfsys@transformshift{3.860845in}{2.713286in}%
\pgfsys@useobject{currentmarker}{}%
\end{pgfscope}%
\end{pgfscope}%
\begin{pgfscope}%
\pgfpathrectangle{\pgfqpoint{0.765000in}{0.660000in}}{\pgfqpoint{4.620000in}{4.620000in}}%
\pgfusepath{clip}%
\pgfsetrectcap%
\pgfsetroundjoin%
\pgfsetlinewidth{1.204500pt}%
\definecolor{currentstroke}{rgb}{1.000000,0.576471,0.309804}%
\pgfsetstrokecolor{currentstroke}%
\pgfsetdash{}{0pt}%
\pgfpathmoveto{\pgfqpoint{2.017281in}{2.669087in}}%
\pgfusepath{stroke}%
\end{pgfscope}%
\begin{pgfscope}%
\pgfpathrectangle{\pgfqpoint{0.765000in}{0.660000in}}{\pgfqpoint{4.620000in}{4.620000in}}%
\pgfusepath{clip}%
\pgfsetbuttcap%
\pgfsetroundjoin%
\definecolor{currentfill}{rgb}{1.000000,0.576471,0.309804}%
\pgfsetfillcolor{currentfill}%
\pgfsetlinewidth{1.003750pt}%
\definecolor{currentstroke}{rgb}{1.000000,0.576471,0.309804}%
\pgfsetstrokecolor{currentstroke}%
\pgfsetdash{}{0pt}%
\pgfsys@defobject{currentmarker}{\pgfqpoint{-0.033333in}{-0.033333in}}{\pgfqpoint{0.033333in}{0.033333in}}{%
\pgfpathmoveto{\pgfqpoint{0.000000in}{-0.033333in}}%
\pgfpathcurveto{\pgfqpoint{0.008840in}{-0.033333in}}{\pgfqpoint{0.017319in}{-0.029821in}}{\pgfqpoint{0.023570in}{-0.023570in}}%
\pgfpathcurveto{\pgfqpoint{0.029821in}{-0.017319in}}{\pgfqpoint{0.033333in}{-0.008840in}}{\pgfqpoint{0.033333in}{0.000000in}}%
\pgfpathcurveto{\pgfqpoint{0.033333in}{0.008840in}}{\pgfqpoint{0.029821in}{0.017319in}}{\pgfqpoint{0.023570in}{0.023570in}}%
\pgfpathcurveto{\pgfqpoint{0.017319in}{0.029821in}}{\pgfqpoint{0.008840in}{0.033333in}}{\pgfqpoint{0.000000in}{0.033333in}}%
\pgfpathcurveto{\pgfqpoint{-0.008840in}{0.033333in}}{\pgfqpoint{-0.017319in}{0.029821in}}{\pgfqpoint{-0.023570in}{0.023570in}}%
\pgfpathcurveto{\pgfqpoint{-0.029821in}{0.017319in}}{\pgfqpoint{-0.033333in}{0.008840in}}{\pgfqpoint{-0.033333in}{0.000000in}}%
\pgfpathcurveto{\pgfqpoint{-0.033333in}{-0.008840in}}{\pgfqpoint{-0.029821in}{-0.017319in}}{\pgfqpoint{-0.023570in}{-0.023570in}}%
\pgfpathcurveto{\pgfqpoint{-0.017319in}{-0.029821in}}{\pgfqpoint{-0.008840in}{-0.033333in}}{\pgfqpoint{0.000000in}{-0.033333in}}%
\pgfpathlineto{\pgfqpoint{0.000000in}{-0.033333in}}%
\pgfpathclose%
\pgfusepath{stroke,fill}%
}%
\begin{pgfscope}%
\pgfsys@transformshift{2.017281in}{2.669087in}%
\pgfsys@useobject{currentmarker}{}%
\end{pgfscope}%
\end{pgfscope}%
\begin{pgfscope}%
\pgfpathrectangle{\pgfqpoint{0.765000in}{0.660000in}}{\pgfqpoint{4.620000in}{4.620000in}}%
\pgfusepath{clip}%
\pgfsetrectcap%
\pgfsetroundjoin%
\pgfsetlinewidth{1.204500pt}%
\definecolor{currentstroke}{rgb}{1.000000,0.576471,0.309804}%
\pgfsetstrokecolor{currentstroke}%
\pgfsetdash{}{0pt}%
\pgfpathmoveto{\pgfqpoint{2.154893in}{3.297158in}}%
\pgfusepath{stroke}%
\end{pgfscope}%
\begin{pgfscope}%
\pgfpathrectangle{\pgfqpoint{0.765000in}{0.660000in}}{\pgfqpoint{4.620000in}{4.620000in}}%
\pgfusepath{clip}%
\pgfsetbuttcap%
\pgfsetroundjoin%
\definecolor{currentfill}{rgb}{1.000000,0.576471,0.309804}%
\pgfsetfillcolor{currentfill}%
\pgfsetlinewidth{1.003750pt}%
\definecolor{currentstroke}{rgb}{1.000000,0.576471,0.309804}%
\pgfsetstrokecolor{currentstroke}%
\pgfsetdash{}{0pt}%
\pgfsys@defobject{currentmarker}{\pgfqpoint{-0.033333in}{-0.033333in}}{\pgfqpoint{0.033333in}{0.033333in}}{%
\pgfpathmoveto{\pgfqpoint{0.000000in}{-0.033333in}}%
\pgfpathcurveto{\pgfqpoint{0.008840in}{-0.033333in}}{\pgfqpoint{0.017319in}{-0.029821in}}{\pgfqpoint{0.023570in}{-0.023570in}}%
\pgfpathcurveto{\pgfqpoint{0.029821in}{-0.017319in}}{\pgfqpoint{0.033333in}{-0.008840in}}{\pgfqpoint{0.033333in}{0.000000in}}%
\pgfpathcurveto{\pgfqpoint{0.033333in}{0.008840in}}{\pgfqpoint{0.029821in}{0.017319in}}{\pgfqpoint{0.023570in}{0.023570in}}%
\pgfpathcurveto{\pgfqpoint{0.017319in}{0.029821in}}{\pgfqpoint{0.008840in}{0.033333in}}{\pgfqpoint{0.000000in}{0.033333in}}%
\pgfpathcurveto{\pgfqpoint{-0.008840in}{0.033333in}}{\pgfqpoint{-0.017319in}{0.029821in}}{\pgfqpoint{-0.023570in}{0.023570in}}%
\pgfpathcurveto{\pgfqpoint{-0.029821in}{0.017319in}}{\pgfqpoint{-0.033333in}{0.008840in}}{\pgfqpoint{-0.033333in}{0.000000in}}%
\pgfpathcurveto{\pgfqpoint{-0.033333in}{-0.008840in}}{\pgfqpoint{-0.029821in}{-0.017319in}}{\pgfqpoint{-0.023570in}{-0.023570in}}%
\pgfpathcurveto{\pgfqpoint{-0.017319in}{-0.029821in}}{\pgfqpoint{-0.008840in}{-0.033333in}}{\pgfqpoint{0.000000in}{-0.033333in}}%
\pgfpathlineto{\pgfqpoint{0.000000in}{-0.033333in}}%
\pgfpathclose%
\pgfusepath{stroke,fill}%
}%
\begin{pgfscope}%
\pgfsys@transformshift{2.154893in}{3.297158in}%
\pgfsys@useobject{currentmarker}{}%
\end{pgfscope}%
\end{pgfscope}%
\begin{pgfscope}%
\pgfpathrectangle{\pgfqpoint{0.765000in}{0.660000in}}{\pgfqpoint{4.620000in}{4.620000in}}%
\pgfusepath{clip}%
\pgfsetrectcap%
\pgfsetroundjoin%
\pgfsetlinewidth{1.204500pt}%
\definecolor{currentstroke}{rgb}{1.000000,0.576471,0.309804}%
\pgfsetstrokecolor{currentstroke}%
\pgfsetdash{}{0pt}%
\pgfpathmoveto{\pgfqpoint{1.740234in}{3.241407in}}%
\pgfusepath{stroke}%
\end{pgfscope}%
\begin{pgfscope}%
\pgfpathrectangle{\pgfqpoint{0.765000in}{0.660000in}}{\pgfqpoint{4.620000in}{4.620000in}}%
\pgfusepath{clip}%
\pgfsetbuttcap%
\pgfsetroundjoin%
\definecolor{currentfill}{rgb}{1.000000,0.576471,0.309804}%
\pgfsetfillcolor{currentfill}%
\pgfsetlinewidth{1.003750pt}%
\definecolor{currentstroke}{rgb}{1.000000,0.576471,0.309804}%
\pgfsetstrokecolor{currentstroke}%
\pgfsetdash{}{0pt}%
\pgfsys@defobject{currentmarker}{\pgfqpoint{-0.033333in}{-0.033333in}}{\pgfqpoint{0.033333in}{0.033333in}}{%
\pgfpathmoveto{\pgfqpoint{0.000000in}{-0.033333in}}%
\pgfpathcurveto{\pgfqpoint{0.008840in}{-0.033333in}}{\pgfqpoint{0.017319in}{-0.029821in}}{\pgfqpoint{0.023570in}{-0.023570in}}%
\pgfpathcurveto{\pgfqpoint{0.029821in}{-0.017319in}}{\pgfqpoint{0.033333in}{-0.008840in}}{\pgfqpoint{0.033333in}{0.000000in}}%
\pgfpathcurveto{\pgfqpoint{0.033333in}{0.008840in}}{\pgfqpoint{0.029821in}{0.017319in}}{\pgfqpoint{0.023570in}{0.023570in}}%
\pgfpathcurveto{\pgfqpoint{0.017319in}{0.029821in}}{\pgfqpoint{0.008840in}{0.033333in}}{\pgfqpoint{0.000000in}{0.033333in}}%
\pgfpathcurveto{\pgfqpoint{-0.008840in}{0.033333in}}{\pgfqpoint{-0.017319in}{0.029821in}}{\pgfqpoint{-0.023570in}{0.023570in}}%
\pgfpathcurveto{\pgfqpoint{-0.029821in}{0.017319in}}{\pgfqpoint{-0.033333in}{0.008840in}}{\pgfqpoint{-0.033333in}{0.000000in}}%
\pgfpathcurveto{\pgfqpoint{-0.033333in}{-0.008840in}}{\pgfqpoint{-0.029821in}{-0.017319in}}{\pgfqpoint{-0.023570in}{-0.023570in}}%
\pgfpathcurveto{\pgfqpoint{-0.017319in}{-0.029821in}}{\pgfqpoint{-0.008840in}{-0.033333in}}{\pgfqpoint{0.000000in}{-0.033333in}}%
\pgfpathlineto{\pgfqpoint{0.000000in}{-0.033333in}}%
\pgfpathclose%
\pgfusepath{stroke,fill}%
}%
\begin{pgfscope}%
\pgfsys@transformshift{1.740234in}{3.241407in}%
\pgfsys@useobject{currentmarker}{}%
\end{pgfscope}%
\end{pgfscope}%
\begin{pgfscope}%
\pgfpathrectangle{\pgfqpoint{0.765000in}{0.660000in}}{\pgfqpoint{4.620000in}{4.620000in}}%
\pgfusepath{clip}%
\pgfsetrectcap%
\pgfsetroundjoin%
\pgfsetlinewidth{1.204500pt}%
\definecolor{currentstroke}{rgb}{1.000000,0.576471,0.309804}%
\pgfsetstrokecolor{currentstroke}%
\pgfsetdash{}{0pt}%
\pgfpathmoveto{\pgfqpoint{3.377710in}{2.090057in}}%
\pgfusepath{stroke}%
\end{pgfscope}%
\begin{pgfscope}%
\pgfpathrectangle{\pgfqpoint{0.765000in}{0.660000in}}{\pgfqpoint{4.620000in}{4.620000in}}%
\pgfusepath{clip}%
\pgfsetbuttcap%
\pgfsetroundjoin%
\definecolor{currentfill}{rgb}{1.000000,0.576471,0.309804}%
\pgfsetfillcolor{currentfill}%
\pgfsetlinewidth{1.003750pt}%
\definecolor{currentstroke}{rgb}{1.000000,0.576471,0.309804}%
\pgfsetstrokecolor{currentstroke}%
\pgfsetdash{}{0pt}%
\pgfsys@defobject{currentmarker}{\pgfqpoint{-0.033333in}{-0.033333in}}{\pgfqpoint{0.033333in}{0.033333in}}{%
\pgfpathmoveto{\pgfqpoint{0.000000in}{-0.033333in}}%
\pgfpathcurveto{\pgfqpoint{0.008840in}{-0.033333in}}{\pgfqpoint{0.017319in}{-0.029821in}}{\pgfqpoint{0.023570in}{-0.023570in}}%
\pgfpathcurveto{\pgfqpoint{0.029821in}{-0.017319in}}{\pgfqpoint{0.033333in}{-0.008840in}}{\pgfqpoint{0.033333in}{0.000000in}}%
\pgfpathcurveto{\pgfqpoint{0.033333in}{0.008840in}}{\pgfqpoint{0.029821in}{0.017319in}}{\pgfqpoint{0.023570in}{0.023570in}}%
\pgfpathcurveto{\pgfqpoint{0.017319in}{0.029821in}}{\pgfqpoint{0.008840in}{0.033333in}}{\pgfqpoint{0.000000in}{0.033333in}}%
\pgfpathcurveto{\pgfqpoint{-0.008840in}{0.033333in}}{\pgfqpoint{-0.017319in}{0.029821in}}{\pgfqpoint{-0.023570in}{0.023570in}}%
\pgfpathcurveto{\pgfqpoint{-0.029821in}{0.017319in}}{\pgfqpoint{-0.033333in}{0.008840in}}{\pgfqpoint{-0.033333in}{0.000000in}}%
\pgfpathcurveto{\pgfqpoint{-0.033333in}{-0.008840in}}{\pgfqpoint{-0.029821in}{-0.017319in}}{\pgfqpoint{-0.023570in}{-0.023570in}}%
\pgfpathcurveto{\pgfqpoint{-0.017319in}{-0.029821in}}{\pgfqpoint{-0.008840in}{-0.033333in}}{\pgfqpoint{0.000000in}{-0.033333in}}%
\pgfpathlineto{\pgfqpoint{0.000000in}{-0.033333in}}%
\pgfpathclose%
\pgfusepath{stroke,fill}%
}%
\begin{pgfscope}%
\pgfsys@transformshift{3.377710in}{2.090057in}%
\pgfsys@useobject{currentmarker}{}%
\end{pgfscope}%
\end{pgfscope}%
\begin{pgfscope}%
\pgfpathrectangle{\pgfqpoint{0.765000in}{0.660000in}}{\pgfqpoint{4.620000in}{4.620000in}}%
\pgfusepath{clip}%
\pgfsetrectcap%
\pgfsetroundjoin%
\pgfsetlinewidth{1.204500pt}%
\definecolor{currentstroke}{rgb}{1.000000,0.576471,0.309804}%
\pgfsetstrokecolor{currentstroke}%
\pgfsetdash{}{0pt}%
\pgfpathmoveto{\pgfqpoint{4.006889in}{3.440269in}}%
\pgfusepath{stroke}%
\end{pgfscope}%
\begin{pgfscope}%
\pgfpathrectangle{\pgfqpoint{0.765000in}{0.660000in}}{\pgfqpoint{4.620000in}{4.620000in}}%
\pgfusepath{clip}%
\pgfsetbuttcap%
\pgfsetroundjoin%
\definecolor{currentfill}{rgb}{1.000000,0.576471,0.309804}%
\pgfsetfillcolor{currentfill}%
\pgfsetlinewidth{1.003750pt}%
\definecolor{currentstroke}{rgb}{1.000000,0.576471,0.309804}%
\pgfsetstrokecolor{currentstroke}%
\pgfsetdash{}{0pt}%
\pgfsys@defobject{currentmarker}{\pgfqpoint{-0.033333in}{-0.033333in}}{\pgfqpoint{0.033333in}{0.033333in}}{%
\pgfpathmoveto{\pgfqpoint{0.000000in}{-0.033333in}}%
\pgfpathcurveto{\pgfqpoint{0.008840in}{-0.033333in}}{\pgfqpoint{0.017319in}{-0.029821in}}{\pgfqpoint{0.023570in}{-0.023570in}}%
\pgfpathcurveto{\pgfqpoint{0.029821in}{-0.017319in}}{\pgfqpoint{0.033333in}{-0.008840in}}{\pgfqpoint{0.033333in}{0.000000in}}%
\pgfpathcurveto{\pgfqpoint{0.033333in}{0.008840in}}{\pgfqpoint{0.029821in}{0.017319in}}{\pgfqpoint{0.023570in}{0.023570in}}%
\pgfpathcurveto{\pgfqpoint{0.017319in}{0.029821in}}{\pgfqpoint{0.008840in}{0.033333in}}{\pgfqpoint{0.000000in}{0.033333in}}%
\pgfpathcurveto{\pgfqpoint{-0.008840in}{0.033333in}}{\pgfqpoint{-0.017319in}{0.029821in}}{\pgfqpoint{-0.023570in}{0.023570in}}%
\pgfpathcurveto{\pgfqpoint{-0.029821in}{0.017319in}}{\pgfqpoint{-0.033333in}{0.008840in}}{\pgfqpoint{-0.033333in}{0.000000in}}%
\pgfpathcurveto{\pgfqpoint{-0.033333in}{-0.008840in}}{\pgfqpoint{-0.029821in}{-0.017319in}}{\pgfqpoint{-0.023570in}{-0.023570in}}%
\pgfpathcurveto{\pgfqpoint{-0.017319in}{-0.029821in}}{\pgfqpoint{-0.008840in}{-0.033333in}}{\pgfqpoint{0.000000in}{-0.033333in}}%
\pgfpathlineto{\pgfqpoint{0.000000in}{-0.033333in}}%
\pgfpathclose%
\pgfusepath{stroke,fill}%
}%
\begin{pgfscope}%
\pgfsys@transformshift{4.006889in}{3.440269in}%
\pgfsys@useobject{currentmarker}{}%
\end{pgfscope}%
\end{pgfscope}%
\begin{pgfscope}%
\pgfpathrectangle{\pgfqpoint{0.765000in}{0.660000in}}{\pgfqpoint{4.620000in}{4.620000in}}%
\pgfusepath{clip}%
\pgfsetrectcap%
\pgfsetroundjoin%
\pgfsetlinewidth{1.204500pt}%
\definecolor{currentstroke}{rgb}{1.000000,0.576471,0.309804}%
\pgfsetstrokecolor{currentstroke}%
\pgfsetdash{}{0pt}%
\pgfpathmoveto{\pgfqpoint{2.465772in}{1.758120in}}%
\pgfusepath{stroke}%
\end{pgfscope}%
\begin{pgfscope}%
\pgfpathrectangle{\pgfqpoint{0.765000in}{0.660000in}}{\pgfqpoint{4.620000in}{4.620000in}}%
\pgfusepath{clip}%
\pgfsetbuttcap%
\pgfsetroundjoin%
\definecolor{currentfill}{rgb}{1.000000,0.576471,0.309804}%
\pgfsetfillcolor{currentfill}%
\pgfsetlinewidth{1.003750pt}%
\definecolor{currentstroke}{rgb}{1.000000,0.576471,0.309804}%
\pgfsetstrokecolor{currentstroke}%
\pgfsetdash{}{0pt}%
\pgfsys@defobject{currentmarker}{\pgfqpoint{-0.033333in}{-0.033333in}}{\pgfqpoint{0.033333in}{0.033333in}}{%
\pgfpathmoveto{\pgfqpoint{0.000000in}{-0.033333in}}%
\pgfpathcurveto{\pgfqpoint{0.008840in}{-0.033333in}}{\pgfqpoint{0.017319in}{-0.029821in}}{\pgfqpoint{0.023570in}{-0.023570in}}%
\pgfpathcurveto{\pgfqpoint{0.029821in}{-0.017319in}}{\pgfqpoint{0.033333in}{-0.008840in}}{\pgfqpoint{0.033333in}{0.000000in}}%
\pgfpathcurveto{\pgfqpoint{0.033333in}{0.008840in}}{\pgfqpoint{0.029821in}{0.017319in}}{\pgfqpoint{0.023570in}{0.023570in}}%
\pgfpathcurveto{\pgfqpoint{0.017319in}{0.029821in}}{\pgfqpoint{0.008840in}{0.033333in}}{\pgfqpoint{0.000000in}{0.033333in}}%
\pgfpathcurveto{\pgfqpoint{-0.008840in}{0.033333in}}{\pgfqpoint{-0.017319in}{0.029821in}}{\pgfqpoint{-0.023570in}{0.023570in}}%
\pgfpathcurveto{\pgfqpoint{-0.029821in}{0.017319in}}{\pgfqpoint{-0.033333in}{0.008840in}}{\pgfqpoint{-0.033333in}{0.000000in}}%
\pgfpathcurveto{\pgfqpoint{-0.033333in}{-0.008840in}}{\pgfqpoint{-0.029821in}{-0.017319in}}{\pgfqpoint{-0.023570in}{-0.023570in}}%
\pgfpathcurveto{\pgfqpoint{-0.017319in}{-0.029821in}}{\pgfqpoint{-0.008840in}{-0.033333in}}{\pgfqpoint{0.000000in}{-0.033333in}}%
\pgfpathlineto{\pgfqpoint{0.000000in}{-0.033333in}}%
\pgfpathclose%
\pgfusepath{stroke,fill}%
}%
\begin{pgfscope}%
\pgfsys@transformshift{2.465772in}{1.758120in}%
\pgfsys@useobject{currentmarker}{}%
\end{pgfscope}%
\end{pgfscope}%
\begin{pgfscope}%
\pgfpathrectangle{\pgfqpoint{0.765000in}{0.660000in}}{\pgfqpoint{4.620000in}{4.620000in}}%
\pgfusepath{clip}%
\pgfsetrectcap%
\pgfsetroundjoin%
\pgfsetlinewidth{1.204500pt}%
\definecolor{currentstroke}{rgb}{1.000000,0.576471,0.309804}%
\pgfsetstrokecolor{currentstroke}%
\pgfsetdash{}{0pt}%
\pgfpathmoveto{\pgfqpoint{2.351752in}{3.577768in}}%
\pgfusepath{stroke}%
\end{pgfscope}%
\begin{pgfscope}%
\pgfpathrectangle{\pgfqpoint{0.765000in}{0.660000in}}{\pgfqpoint{4.620000in}{4.620000in}}%
\pgfusepath{clip}%
\pgfsetbuttcap%
\pgfsetroundjoin%
\definecolor{currentfill}{rgb}{1.000000,0.576471,0.309804}%
\pgfsetfillcolor{currentfill}%
\pgfsetlinewidth{1.003750pt}%
\definecolor{currentstroke}{rgb}{1.000000,0.576471,0.309804}%
\pgfsetstrokecolor{currentstroke}%
\pgfsetdash{}{0pt}%
\pgfsys@defobject{currentmarker}{\pgfqpoint{-0.033333in}{-0.033333in}}{\pgfqpoint{0.033333in}{0.033333in}}{%
\pgfpathmoveto{\pgfqpoint{0.000000in}{-0.033333in}}%
\pgfpathcurveto{\pgfqpoint{0.008840in}{-0.033333in}}{\pgfqpoint{0.017319in}{-0.029821in}}{\pgfqpoint{0.023570in}{-0.023570in}}%
\pgfpathcurveto{\pgfqpoint{0.029821in}{-0.017319in}}{\pgfqpoint{0.033333in}{-0.008840in}}{\pgfqpoint{0.033333in}{0.000000in}}%
\pgfpathcurveto{\pgfqpoint{0.033333in}{0.008840in}}{\pgfqpoint{0.029821in}{0.017319in}}{\pgfqpoint{0.023570in}{0.023570in}}%
\pgfpathcurveto{\pgfqpoint{0.017319in}{0.029821in}}{\pgfqpoint{0.008840in}{0.033333in}}{\pgfqpoint{0.000000in}{0.033333in}}%
\pgfpathcurveto{\pgfqpoint{-0.008840in}{0.033333in}}{\pgfqpoint{-0.017319in}{0.029821in}}{\pgfqpoint{-0.023570in}{0.023570in}}%
\pgfpathcurveto{\pgfqpoint{-0.029821in}{0.017319in}}{\pgfqpoint{-0.033333in}{0.008840in}}{\pgfqpoint{-0.033333in}{0.000000in}}%
\pgfpathcurveto{\pgfqpoint{-0.033333in}{-0.008840in}}{\pgfqpoint{-0.029821in}{-0.017319in}}{\pgfqpoint{-0.023570in}{-0.023570in}}%
\pgfpathcurveto{\pgfqpoint{-0.017319in}{-0.029821in}}{\pgfqpoint{-0.008840in}{-0.033333in}}{\pgfqpoint{0.000000in}{-0.033333in}}%
\pgfpathlineto{\pgfqpoint{0.000000in}{-0.033333in}}%
\pgfpathclose%
\pgfusepath{stroke,fill}%
}%
\begin{pgfscope}%
\pgfsys@transformshift{2.351752in}{3.577768in}%
\pgfsys@useobject{currentmarker}{}%
\end{pgfscope}%
\end{pgfscope}%
\begin{pgfscope}%
\pgfpathrectangle{\pgfqpoint{0.765000in}{0.660000in}}{\pgfqpoint{4.620000in}{4.620000in}}%
\pgfusepath{clip}%
\pgfsetrectcap%
\pgfsetroundjoin%
\pgfsetlinewidth{1.204500pt}%
\definecolor{currentstroke}{rgb}{1.000000,0.576471,0.309804}%
\pgfsetstrokecolor{currentstroke}%
\pgfsetdash{}{0pt}%
\pgfpathmoveto{\pgfqpoint{2.185914in}{3.494286in}}%
\pgfusepath{stroke}%
\end{pgfscope}%
\begin{pgfscope}%
\pgfpathrectangle{\pgfqpoint{0.765000in}{0.660000in}}{\pgfqpoint{4.620000in}{4.620000in}}%
\pgfusepath{clip}%
\pgfsetbuttcap%
\pgfsetroundjoin%
\definecolor{currentfill}{rgb}{1.000000,0.576471,0.309804}%
\pgfsetfillcolor{currentfill}%
\pgfsetlinewidth{1.003750pt}%
\definecolor{currentstroke}{rgb}{1.000000,0.576471,0.309804}%
\pgfsetstrokecolor{currentstroke}%
\pgfsetdash{}{0pt}%
\pgfsys@defobject{currentmarker}{\pgfqpoint{-0.033333in}{-0.033333in}}{\pgfqpoint{0.033333in}{0.033333in}}{%
\pgfpathmoveto{\pgfqpoint{0.000000in}{-0.033333in}}%
\pgfpathcurveto{\pgfqpoint{0.008840in}{-0.033333in}}{\pgfqpoint{0.017319in}{-0.029821in}}{\pgfqpoint{0.023570in}{-0.023570in}}%
\pgfpathcurveto{\pgfqpoint{0.029821in}{-0.017319in}}{\pgfqpoint{0.033333in}{-0.008840in}}{\pgfqpoint{0.033333in}{0.000000in}}%
\pgfpathcurveto{\pgfqpoint{0.033333in}{0.008840in}}{\pgfqpoint{0.029821in}{0.017319in}}{\pgfqpoint{0.023570in}{0.023570in}}%
\pgfpathcurveto{\pgfqpoint{0.017319in}{0.029821in}}{\pgfqpoint{0.008840in}{0.033333in}}{\pgfqpoint{0.000000in}{0.033333in}}%
\pgfpathcurveto{\pgfqpoint{-0.008840in}{0.033333in}}{\pgfqpoint{-0.017319in}{0.029821in}}{\pgfqpoint{-0.023570in}{0.023570in}}%
\pgfpathcurveto{\pgfqpoint{-0.029821in}{0.017319in}}{\pgfqpoint{-0.033333in}{0.008840in}}{\pgfqpoint{-0.033333in}{0.000000in}}%
\pgfpathcurveto{\pgfqpoint{-0.033333in}{-0.008840in}}{\pgfqpoint{-0.029821in}{-0.017319in}}{\pgfqpoint{-0.023570in}{-0.023570in}}%
\pgfpathcurveto{\pgfqpoint{-0.017319in}{-0.029821in}}{\pgfqpoint{-0.008840in}{-0.033333in}}{\pgfqpoint{0.000000in}{-0.033333in}}%
\pgfpathlineto{\pgfqpoint{0.000000in}{-0.033333in}}%
\pgfpathclose%
\pgfusepath{stroke,fill}%
}%
\begin{pgfscope}%
\pgfsys@transformshift{2.185914in}{3.494286in}%
\pgfsys@useobject{currentmarker}{}%
\end{pgfscope}%
\end{pgfscope}%
\begin{pgfscope}%
\pgfpathrectangle{\pgfqpoint{0.765000in}{0.660000in}}{\pgfqpoint{4.620000in}{4.620000in}}%
\pgfusepath{clip}%
\pgfsetrectcap%
\pgfsetroundjoin%
\pgfsetlinewidth{1.204500pt}%
\definecolor{currentstroke}{rgb}{1.000000,0.576471,0.309804}%
\pgfsetstrokecolor{currentstroke}%
\pgfsetdash{}{0pt}%
\pgfpathmoveto{\pgfqpoint{2.310318in}{4.226772in}}%
\pgfusepath{stroke}%
\end{pgfscope}%
\begin{pgfscope}%
\pgfpathrectangle{\pgfqpoint{0.765000in}{0.660000in}}{\pgfqpoint{4.620000in}{4.620000in}}%
\pgfusepath{clip}%
\pgfsetbuttcap%
\pgfsetroundjoin%
\definecolor{currentfill}{rgb}{1.000000,0.576471,0.309804}%
\pgfsetfillcolor{currentfill}%
\pgfsetlinewidth{1.003750pt}%
\definecolor{currentstroke}{rgb}{1.000000,0.576471,0.309804}%
\pgfsetstrokecolor{currentstroke}%
\pgfsetdash{}{0pt}%
\pgfsys@defobject{currentmarker}{\pgfqpoint{-0.033333in}{-0.033333in}}{\pgfqpoint{0.033333in}{0.033333in}}{%
\pgfpathmoveto{\pgfqpoint{0.000000in}{-0.033333in}}%
\pgfpathcurveto{\pgfqpoint{0.008840in}{-0.033333in}}{\pgfqpoint{0.017319in}{-0.029821in}}{\pgfqpoint{0.023570in}{-0.023570in}}%
\pgfpathcurveto{\pgfqpoint{0.029821in}{-0.017319in}}{\pgfqpoint{0.033333in}{-0.008840in}}{\pgfqpoint{0.033333in}{0.000000in}}%
\pgfpathcurveto{\pgfqpoint{0.033333in}{0.008840in}}{\pgfqpoint{0.029821in}{0.017319in}}{\pgfqpoint{0.023570in}{0.023570in}}%
\pgfpathcurveto{\pgfqpoint{0.017319in}{0.029821in}}{\pgfqpoint{0.008840in}{0.033333in}}{\pgfqpoint{0.000000in}{0.033333in}}%
\pgfpathcurveto{\pgfqpoint{-0.008840in}{0.033333in}}{\pgfqpoint{-0.017319in}{0.029821in}}{\pgfqpoint{-0.023570in}{0.023570in}}%
\pgfpathcurveto{\pgfqpoint{-0.029821in}{0.017319in}}{\pgfqpoint{-0.033333in}{0.008840in}}{\pgfqpoint{-0.033333in}{0.000000in}}%
\pgfpathcurveto{\pgfqpoint{-0.033333in}{-0.008840in}}{\pgfqpoint{-0.029821in}{-0.017319in}}{\pgfqpoint{-0.023570in}{-0.023570in}}%
\pgfpathcurveto{\pgfqpoint{-0.017319in}{-0.029821in}}{\pgfqpoint{-0.008840in}{-0.033333in}}{\pgfqpoint{0.000000in}{-0.033333in}}%
\pgfpathlineto{\pgfqpoint{0.000000in}{-0.033333in}}%
\pgfpathclose%
\pgfusepath{stroke,fill}%
}%
\begin{pgfscope}%
\pgfsys@transformshift{2.310318in}{4.226772in}%
\pgfsys@useobject{currentmarker}{}%
\end{pgfscope}%
\end{pgfscope}%
\begin{pgfscope}%
\pgfpathrectangle{\pgfqpoint{0.765000in}{0.660000in}}{\pgfqpoint{4.620000in}{4.620000in}}%
\pgfusepath{clip}%
\pgfsetrectcap%
\pgfsetroundjoin%
\pgfsetlinewidth{1.204500pt}%
\definecolor{currentstroke}{rgb}{1.000000,0.576471,0.309804}%
\pgfsetstrokecolor{currentstroke}%
\pgfsetdash{}{0pt}%
\pgfpathmoveto{\pgfqpoint{2.422467in}{2.531677in}}%
\pgfusepath{stroke}%
\end{pgfscope}%
\begin{pgfscope}%
\pgfpathrectangle{\pgfqpoint{0.765000in}{0.660000in}}{\pgfqpoint{4.620000in}{4.620000in}}%
\pgfusepath{clip}%
\pgfsetbuttcap%
\pgfsetroundjoin%
\definecolor{currentfill}{rgb}{1.000000,0.576471,0.309804}%
\pgfsetfillcolor{currentfill}%
\pgfsetlinewidth{1.003750pt}%
\definecolor{currentstroke}{rgb}{1.000000,0.576471,0.309804}%
\pgfsetstrokecolor{currentstroke}%
\pgfsetdash{}{0pt}%
\pgfsys@defobject{currentmarker}{\pgfqpoint{-0.033333in}{-0.033333in}}{\pgfqpoint{0.033333in}{0.033333in}}{%
\pgfpathmoveto{\pgfqpoint{0.000000in}{-0.033333in}}%
\pgfpathcurveto{\pgfqpoint{0.008840in}{-0.033333in}}{\pgfqpoint{0.017319in}{-0.029821in}}{\pgfqpoint{0.023570in}{-0.023570in}}%
\pgfpathcurveto{\pgfqpoint{0.029821in}{-0.017319in}}{\pgfqpoint{0.033333in}{-0.008840in}}{\pgfqpoint{0.033333in}{0.000000in}}%
\pgfpathcurveto{\pgfqpoint{0.033333in}{0.008840in}}{\pgfqpoint{0.029821in}{0.017319in}}{\pgfqpoint{0.023570in}{0.023570in}}%
\pgfpathcurveto{\pgfqpoint{0.017319in}{0.029821in}}{\pgfqpoint{0.008840in}{0.033333in}}{\pgfqpoint{0.000000in}{0.033333in}}%
\pgfpathcurveto{\pgfqpoint{-0.008840in}{0.033333in}}{\pgfqpoint{-0.017319in}{0.029821in}}{\pgfqpoint{-0.023570in}{0.023570in}}%
\pgfpathcurveto{\pgfqpoint{-0.029821in}{0.017319in}}{\pgfqpoint{-0.033333in}{0.008840in}}{\pgfqpoint{-0.033333in}{0.000000in}}%
\pgfpathcurveto{\pgfqpoint{-0.033333in}{-0.008840in}}{\pgfqpoint{-0.029821in}{-0.017319in}}{\pgfqpoint{-0.023570in}{-0.023570in}}%
\pgfpathcurveto{\pgfqpoint{-0.017319in}{-0.029821in}}{\pgfqpoint{-0.008840in}{-0.033333in}}{\pgfqpoint{0.000000in}{-0.033333in}}%
\pgfpathlineto{\pgfqpoint{0.000000in}{-0.033333in}}%
\pgfpathclose%
\pgfusepath{stroke,fill}%
}%
\begin{pgfscope}%
\pgfsys@transformshift{2.422467in}{2.531677in}%
\pgfsys@useobject{currentmarker}{}%
\end{pgfscope}%
\end{pgfscope}%
\begin{pgfscope}%
\pgfpathrectangle{\pgfqpoint{0.765000in}{0.660000in}}{\pgfqpoint{4.620000in}{4.620000in}}%
\pgfusepath{clip}%
\pgfsetrectcap%
\pgfsetroundjoin%
\pgfsetlinewidth{1.204500pt}%
\definecolor{currentstroke}{rgb}{1.000000,0.576471,0.309804}%
\pgfsetstrokecolor{currentstroke}%
\pgfsetdash{}{0pt}%
\pgfpathmoveto{\pgfqpoint{3.638058in}{4.067957in}}%
\pgfusepath{stroke}%
\end{pgfscope}%
\begin{pgfscope}%
\pgfpathrectangle{\pgfqpoint{0.765000in}{0.660000in}}{\pgfqpoint{4.620000in}{4.620000in}}%
\pgfusepath{clip}%
\pgfsetbuttcap%
\pgfsetroundjoin%
\definecolor{currentfill}{rgb}{1.000000,0.576471,0.309804}%
\pgfsetfillcolor{currentfill}%
\pgfsetlinewidth{1.003750pt}%
\definecolor{currentstroke}{rgb}{1.000000,0.576471,0.309804}%
\pgfsetstrokecolor{currentstroke}%
\pgfsetdash{}{0pt}%
\pgfsys@defobject{currentmarker}{\pgfqpoint{-0.033333in}{-0.033333in}}{\pgfqpoint{0.033333in}{0.033333in}}{%
\pgfpathmoveto{\pgfqpoint{0.000000in}{-0.033333in}}%
\pgfpathcurveto{\pgfqpoint{0.008840in}{-0.033333in}}{\pgfqpoint{0.017319in}{-0.029821in}}{\pgfqpoint{0.023570in}{-0.023570in}}%
\pgfpathcurveto{\pgfqpoint{0.029821in}{-0.017319in}}{\pgfqpoint{0.033333in}{-0.008840in}}{\pgfqpoint{0.033333in}{0.000000in}}%
\pgfpathcurveto{\pgfqpoint{0.033333in}{0.008840in}}{\pgfqpoint{0.029821in}{0.017319in}}{\pgfqpoint{0.023570in}{0.023570in}}%
\pgfpathcurveto{\pgfqpoint{0.017319in}{0.029821in}}{\pgfqpoint{0.008840in}{0.033333in}}{\pgfqpoint{0.000000in}{0.033333in}}%
\pgfpathcurveto{\pgfqpoint{-0.008840in}{0.033333in}}{\pgfqpoint{-0.017319in}{0.029821in}}{\pgfqpoint{-0.023570in}{0.023570in}}%
\pgfpathcurveto{\pgfqpoint{-0.029821in}{0.017319in}}{\pgfqpoint{-0.033333in}{0.008840in}}{\pgfqpoint{-0.033333in}{0.000000in}}%
\pgfpathcurveto{\pgfqpoint{-0.033333in}{-0.008840in}}{\pgfqpoint{-0.029821in}{-0.017319in}}{\pgfqpoint{-0.023570in}{-0.023570in}}%
\pgfpathcurveto{\pgfqpoint{-0.017319in}{-0.029821in}}{\pgfqpoint{-0.008840in}{-0.033333in}}{\pgfqpoint{0.000000in}{-0.033333in}}%
\pgfpathlineto{\pgfqpoint{0.000000in}{-0.033333in}}%
\pgfpathclose%
\pgfusepath{stroke,fill}%
}%
\begin{pgfscope}%
\pgfsys@transformshift{3.638058in}{4.067957in}%
\pgfsys@useobject{currentmarker}{}%
\end{pgfscope}%
\end{pgfscope}%
\begin{pgfscope}%
\pgfpathrectangle{\pgfqpoint{0.765000in}{0.660000in}}{\pgfqpoint{4.620000in}{4.620000in}}%
\pgfusepath{clip}%
\pgfsetrectcap%
\pgfsetroundjoin%
\pgfsetlinewidth{1.204500pt}%
\definecolor{currentstroke}{rgb}{1.000000,0.576471,0.309804}%
\pgfsetstrokecolor{currentstroke}%
\pgfsetdash{}{0pt}%
\pgfpathmoveto{\pgfqpoint{2.556667in}{3.552469in}}%
\pgfusepath{stroke}%
\end{pgfscope}%
\begin{pgfscope}%
\pgfpathrectangle{\pgfqpoint{0.765000in}{0.660000in}}{\pgfqpoint{4.620000in}{4.620000in}}%
\pgfusepath{clip}%
\pgfsetbuttcap%
\pgfsetroundjoin%
\definecolor{currentfill}{rgb}{1.000000,0.576471,0.309804}%
\pgfsetfillcolor{currentfill}%
\pgfsetlinewidth{1.003750pt}%
\definecolor{currentstroke}{rgb}{1.000000,0.576471,0.309804}%
\pgfsetstrokecolor{currentstroke}%
\pgfsetdash{}{0pt}%
\pgfsys@defobject{currentmarker}{\pgfqpoint{-0.033333in}{-0.033333in}}{\pgfqpoint{0.033333in}{0.033333in}}{%
\pgfpathmoveto{\pgfqpoint{0.000000in}{-0.033333in}}%
\pgfpathcurveto{\pgfqpoint{0.008840in}{-0.033333in}}{\pgfqpoint{0.017319in}{-0.029821in}}{\pgfqpoint{0.023570in}{-0.023570in}}%
\pgfpathcurveto{\pgfqpoint{0.029821in}{-0.017319in}}{\pgfqpoint{0.033333in}{-0.008840in}}{\pgfqpoint{0.033333in}{0.000000in}}%
\pgfpathcurveto{\pgfqpoint{0.033333in}{0.008840in}}{\pgfqpoint{0.029821in}{0.017319in}}{\pgfqpoint{0.023570in}{0.023570in}}%
\pgfpathcurveto{\pgfqpoint{0.017319in}{0.029821in}}{\pgfqpoint{0.008840in}{0.033333in}}{\pgfqpoint{0.000000in}{0.033333in}}%
\pgfpathcurveto{\pgfqpoint{-0.008840in}{0.033333in}}{\pgfqpoint{-0.017319in}{0.029821in}}{\pgfqpoint{-0.023570in}{0.023570in}}%
\pgfpathcurveto{\pgfqpoint{-0.029821in}{0.017319in}}{\pgfqpoint{-0.033333in}{0.008840in}}{\pgfqpoint{-0.033333in}{0.000000in}}%
\pgfpathcurveto{\pgfqpoint{-0.033333in}{-0.008840in}}{\pgfqpoint{-0.029821in}{-0.017319in}}{\pgfqpoint{-0.023570in}{-0.023570in}}%
\pgfpathcurveto{\pgfqpoint{-0.017319in}{-0.029821in}}{\pgfqpoint{-0.008840in}{-0.033333in}}{\pgfqpoint{0.000000in}{-0.033333in}}%
\pgfpathlineto{\pgfqpoint{0.000000in}{-0.033333in}}%
\pgfpathclose%
\pgfusepath{stroke,fill}%
}%
\begin{pgfscope}%
\pgfsys@transformshift{2.556667in}{3.552469in}%
\pgfsys@useobject{currentmarker}{}%
\end{pgfscope}%
\end{pgfscope}%
\begin{pgfscope}%
\pgfpathrectangle{\pgfqpoint{0.765000in}{0.660000in}}{\pgfqpoint{4.620000in}{4.620000in}}%
\pgfusepath{clip}%
\pgfsetrectcap%
\pgfsetroundjoin%
\pgfsetlinewidth{1.204500pt}%
\definecolor{currentstroke}{rgb}{1.000000,0.576471,0.309804}%
\pgfsetstrokecolor{currentstroke}%
\pgfsetdash{}{0pt}%
\pgfpathmoveto{\pgfqpoint{3.338592in}{3.977561in}}%
\pgfusepath{stroke}%
\end{pgfscope}%
\begin{pgfscope}%
\pgfpathrectangle{\pgfqpoint{0.765000in}{0.660000in}}{\pgfqpoint{4.620000in}{4.620000in}}%
\pgfusepath{clip}%
\pgfsetbuttcap%
\pgfsetroundjoin%
\definecolor{currentfill}{rgb}{1.000000,0.576471,0.309804}%
\pgfsetfillcolor{currentfill}%
\pgfsetlinewidth{1.003750pt}%
\definecolor{currentstroke}{rgb}{1.000000,0.576471,0.309804}%
\pgfsetstrokecolor{currentstroke}%
\pgfsetdash{}{0pt}%
\pgfsys@defobject{currentmarker}{\pgfqpoint{-0.033333in}{-0.033333in}}{\pgfqpoint{0.033333in}{0.033333in}}{%
\pgfpathmoveto{\pgfqpoint{0.000000in}{-0.033333in}}%
\pgfpathcurveto{\pgfqpoint{0.008840in}{-0.033333in}}{\pgfqpoint{0.017319in}{-0.029821in}}{\pgfqpoint{0.023570in}{-0.023570in}}%
\pgfpathcurveto{\pgfqpoint{0.029821in}{-0.017319in}}{\pgfqpoint{0.033333in}{-0.008840in}}{\pgfqpoint{0.033333in}{0.000000in}}%
\pgfpathcurveto{\pgfqpoint{0.033333in}{0.008840in}}{\pgfqpoint{0.029821in}{0.017319in}}{\pgfqpoint{0.023570in}{0.023570in}}%
\pgfpathcurveto{\pgfqpoint{0.017319in}{0.029821in}}{\pgfqpoint{0.008840in}{0.033333in}}{\pgfqpoint{0.000000in}{0.033333in}}%
\pgfpathcurveto{\pgfqpoint{-0.008840in}{0.033333in}}{\pgfqpoint{-0.017319in}{0.029821in}}{\pgfqpoint{-0.023570in}{0.023570in}}%
\pgfpathcurveto{\pgfqpoint{-0.029821in}{0.017319in}}{\pgfqpoint{-0.033333in}{0.008840in}}{\pgfqpoint{-0.033333in}{0.000000in}}%
\pgfpathcurveto{\pgfqpoint{-0.033333in}{-0.008840in}}{\pgfqpoint{-0.029821in}{-0.017319in}}{\pgfqpoint{-0.023570in}{-0.023570in}}%
\pgfpathcurveto{\pgfqpoint{-0.017319in}{-0.029821in}}{\pgfqpoint{-0.008840in}{-0.033333in}}{\pgfqpoint{0.000000in}{-0.033333in}}%
\pgfpathlineto{\pgfqpoint{0.000000in}{-0.033333in}}%
\pgfpathclose%
\pgfusepath{stroke,fill}%
}%
\begin{pgfscope}%
\pgfsys@transformshift{3.338592in}{3.977561in}%
\pgfsys@useobject{currentmarker}{}%
\end{pgfscope}%
\end{pgfscope}%
\begin{pgfscope}%
\pgfpathrectangle{\pgfqpoint{0.765000in}{0.660000in}}{\pgfqpoint{4.620000in}{4.620000in}}%
\pgfusepath{clip}%
\pgfsetrectcap%
\pgfsetroundjoin%
\pgfsetlinewidth{1.204500pt}%
\definecolor{currentstroke}{rgb}{1.000000,0.576471,0.309804}%
\pgfsetstrokecolor{currentstroke}%
\pgfsetdash{}{0pt}%
\pgfpathmoveto{\pgfqpoint{3.129614in}{4.893934in}}%
\pgfusepath{stroke}%
\end{pgfscope}%
\begin{pgfscope}%
\pgfpathrectangle{\pgfqpoint{0.765000in}{0.660000in}}{\pgfqpoint{4.620000in}{4.620000in}}%
\pgfusepath{clip}%
\pgfsetbuttcap%
\pgfsetroundjoin%
\definecolor{currentfill}{rgb}{1.000000,0.576471,0.309804}%
\pgfsetfillcolor{currentfill}%
\pgfsetlinewidth{1.003750pt}%
\definecolor{currentstroke}{rgb}{1.000000,0.576471,0.309804}%
\pgfsetstrokecolor{currentstroke}%
\pgfsetdash{}{0pt}%
\pgfsys@defobject{currentmarker}{\pgfqpoint{-0.033333in}{-0.033333in}}{\pgfqpoint{0.033333in}{0.033333in}}{%
\pgfpathmoveto{\pgfqpoint{0.000000in}{-0.033333in}}%
\pgfpathcurveto{\pgfqpoint{0.008840in}{-0.033333in}}{\pgfqpoint{0.017319in}{-0.029821in}}{\pgfqpoint{0.023570in}{-0.023570in}}%
\pgfpathcurveto{\pgfqpoint{0.029821in}{-0.017319in}}{\pgfqpoint{0.033333in}{-0.008840in}}{\pgfqpoint{0.033333in}{0.000000in}}%
\pgfpathcurveto{\pgfqpoint{0.033333in}{0.008840in}}{\pgfqpoint{0.029821in}{0.017319in}}{\pgfqpoint{0.023570in}{0.023570in}}%
\pgfpathcurveto{\pgfqpoint{0.017319in}{0.029821in}}{\pgfqpoint{0.008840in}{0.033333in}}{\pgfqpoint{0.000000in}{0.033333in}}%
\pgfpathcurveto{\pgfqpoint{-0.008840in}{0.033333in}}{\pgfqpoint{-0.017319in}{0.029821in}}{\pgfqpoint{-0.023570in}{0.023570in}}%
\pgfpathcurveto{\pgfqpoint{-0.029821in}{0.017319in}}{\pgfqpoint{-0.033333in}{0.008840in}}{\pgfqpoint{-0.033333in}{0.000000in}}%
\pgfpathcurveto{\pgfqpoint{-0.033333in}{-0.008840in}}{\pgfqpoint{-0.029821in}{-0.017319in}}{\pgfqpoint{-0.023570in}{-0.023570in}}%
\pgfpathcurveto{\pgfqpoint{-0.017319in}{-0.029821in}}{\pgfqpoint{-0.008840in}{-0.033333in}}{\pgfqpoint{0.000000in}{-0.033333in}}%
\pgfpathlineto{\pgfqpoint{0.000000in}{-0.033333in}}%
\pgfpathclose%
\pgfusepath{stroke,fill}%
}%
\begin{pgfscope}%
\pgfsys@transformshift{3.129614in}{4.893934in}%
\pgfsys@useobject{currentmarker}{}%
\end{pgfscope}%
\end{pgfscope}%
\begin{pgfscope}%
\pgfpathrectangle{\pgfqpoint{0.765000in}{0.660000in}}{\pgfqpoint{4.620000in}{4.620000in}}%
\pgfusepath{clip}%
\pgfsetrectcap%
\pgfsetroundjoin%
\pgfsetlinewidth{1.204500pt}%
\definecolor{currentstroke}{rgb}{1.000000,0.576471,0.309804}%
\pgfsetstrokecolor{currentstroke}%
\pgfsetdash{}{0pt}%
\pgfpathmoveto{\pgfqpoint{4.192903in}{1.370546in}}%
\pgfusepath{stroke}%
\end{pgfscope}%
\begin{pgfscope}%
\pgfpathrectangle{\pgfqpoint{0.765000in}{0.660000in}}{\pgfqpoint{4.620000in}{4.620000in}}%
\pgfusepath{clip}%
\pgfsetbuttcap%
\pgfsetroundjoin%
\definecolor{currentfill}{rgb}{1.000000,0.576471,0.309804}%
\pgfsetfillcolor{currentfill}%
\pgfsetlinewidth{1.003750pt}%
\definecolor{currentstroke}{rgb}{1.000000,0.576471,0.309804}%
\pgfsetstrokecolor{currentstroke}%
\pgfsetdash{}{0pt}%
\pgfsys@defobject{currentmarker}{\pgfqpoint{-0.033333in}{-0.033333in}}{\pgfqpoint{0.033333in}{0.033333in}}{%
\pgfpathmoveto{\pgfqpoint{0.000000in}{-0.033333in}}%
\pgfpathcurveto{\pgfqpoint{0.008840in}{-0.033333in}}{\pgfqpoint{0.017319in}{-0.029821in}}{\pgfqpoint{0.023570in}{-0.023570in}}%
\pgfpathcurveto{\pgfqpoint{0.029821in}{-0.017319in}}{\pgfqpoint{0.033333in}{-0.008840in}}{\pgfqpoint{0.033333in}{0.000000in}}%
\pgfpathcurveto{\pgfqpoint{0.033333in}{0.008840in}}{\pgfqpoint{0.029821in}{0.017319in}}{\pgfqpoint{0.023570in}{0.023570in}}%
\pgfpathcurveto{\pgfqpoint{0.017319in}{0.029821in}}{\pgfqpoint{0.008840in}{0.033333in}}{\pgfqpoint{0.000000in}{0.033333in}}%
\pgfpathcurveto{\pgfqpoint{-0.008840in}{0.033333in}}{\pgfqpoint{-0.017319in}{0.029821in}}{\pgfqpoint{-0.023570in}{0.023570in}}%
\pgfpathcurveto{\pgfqpoint{-0.029821in}{0.017319in}}{\pgfqpoint{-0.033333in}{0.008840in}}{\pgfqpoint{-0.033333in}{0.000000in}}%
\pgfpathcurveto{\pgfqpoint{-0.033333in}{-0.008840in}}{\pgfqpoint{-0.029821in}{-0.017319in}}{\pgfqpoint{-0.023570in}{-0.023570in}}%
\pgfpathcurveto{\pgfqpoint{-0.017319in}{-0.029821in}}{\pgfqpoint{-0.008840in}{-0.033333in}}{\pgfqpoint{0.000000in}{-0.033333in}}%
\pgfpathlineto{\pgfqpoint{0.000000in}{-0.033333in}}%
\pgfpathclose%
\pgfusepath{stroke,fill}%
}%
\begin{pgfscope}%
\pgfsys@transformshift{4.192903in}{1.370546in}%
\pgfsys@useobject{currentmarker}{}%
\end{pgfscope}%
\end{pgfscope}%
\begin{pgfscope}%
\pgfpathrectangle{\pgfqpoint{0.765000in}{0.660000in}}{\pgfqpoint{4.620000in}{4.620000in}}%
\pgfusepath{clip}%
\pgfsetrectcap%
\pgfsetroundjoin%
\pgfsetlinewidth{1.204500pt}%
\definecolor{currentstroke}{rgb}{1.000000,0.576471,0.309804}%
\pgfsetstrokecolor{currentstroke}%
\pgfsetdash{}{0pt}%
\pgfpathmoveto{\pgfqpoint{2.033582in}{3.052409in}}%
\pgfusepath{stroke}%
\end{pgfscope}%
\begin{pgfscope}%
\pgfpathrectangle{\pgfqpoint{0.765000in}{0.660000in}}{\pgfqpoint{4.620000in}{4.620000in}}%
\pgfusepath{clip}%
\pgfsetbuttcap%
\pgfsetroundjoin%
\definecolor{currentfill}{rgb}{1.000000,0.576471,0.309804}%
\pgfsetfillcolor{currentfill}%
\pgfsetlinewidth{1.003750pt}%
\definecolor{currentstroke}{rgb}{1.000000,0.576471,0.309804}%
\pgfsetstrokecolor{currentstroke}%
\pgfsetdash{}{0pt}%
\pgfsys@defobject{currentmarker}{\pgfqpoint{-0.033333in}{-0.033333in}}{\pgfqpoint{0.033333in}{0.033333in}}{%
\pgfpathmoveto{\pgfqpoint{0.000000in}{-0.033333in}}%
\pgfpathcurveto{\pgfqpoint{0.008840in}{-0.033333in}}{\pgfqpoint{0.017319in}{-0.029821in}}{\pgfqpoint{0.023570in}{-0.023570in}}%
\pgfpathcurveto{\pgfqpoint{0.029821in}{-0.017319in}}{\pgfqpoint{0.033333in}{-0.008840in}}{\pgfqpoint{0.033333in}{0.000000in}}%
\pgfpathcurveto{\pgfqpoint{0.033333in}{0.008840in}}{\pgfqpoint{0.029821in}{0.017319in}}{\pgfqpoint{0.023570in}{0.023570in}}%
\pgfpathcurveto{\pgfqpoint{0.017319in}{0.029821in}}{\pgfqpoint{0.008840in}{0.033333in}}{\pgfqpoint{0.000000in}{0.033333in}}%
\pgfpathcurveto{\pgfqpoint{-0.008840in}{0.033333in}}{\pgfqpoint{-0.017319in}{0.029821in}}{\pgfqpoint{-0.023570in}{0.023570in}}%
\pgfpathcurveto{\pgfqpoint{-0.029821in}{0.017319in}}{\pgfqpoint{-0.033333in}{0.008840in}}{\pgfqpoint{-0.033333in}{0.000000in}}%
\pgfpathcurveto{\pgfqpoint{-0.033333in}{-0.008840in}}{\pgfqpoint{-0.029821in}{-0.017319in}}{\pgfqpoint{-0.023570in}{-0.023570in}}%
\pgfpathcurveto{\pgfqpoint{-0.017319in}{-0.029821in}}{\pgfqpoint{-0.008840in}{-0.033333in}}{\pgfqpoint{0.000000in}{-0.033333in}}%
\pgfpathlineto{\pgfqpoint{0.000000in}{-0.033333in}}%
\pgfpathclose%
\pgfusepath{stroke,fill}%
}%
\begin{pgfscope}%
\pgfsys@transformshift{2.033582in}{3.052409in}%
\pgfsys@useobject{currentmarker}{}%
\end{pgfscope}%
\end{pgfscope}%
\begin{pgfscope}%
\pgfpathrectangle{\pgfqpoint{0.765000in}{0.660000in}}{\pgfqpoint{4.620000in}{4.620000in}}%
\pgfusepath{clip}%
\pgfsetrectcap%
\pgfsetroundjoin%
\pgfsetlinewidth{1.204500pt}%
\definecolor{currentstroke}{rgb}{1.000000,0.576471,0.309804}%
\pgfsetstrokecolor{currentstroke}%
\pgfsetdash{}{0pt}%
\pgfpathmoveto{\pgfqpoint{2.640771in}{2.146971in}}%
\pgfusepath{stroke}%
\end{pgfscope}%
\begin{pgfscope}%
\pgfpathrectangle{\pgfqpoint{0.765000in}{0.660000in}}{\pgfqpoint{4.620000in}{4.620000in}}%
\pgfusepath{clip}%
\pgfsetbuttcap%
\pgfsetroundjoin%
\definecolor{currentfill}{rgb}{1.000000,0.576471,0.309804}%
\pgfsetfillcolor{currentfill}%
\pgfsetlinewidth{1.003750pt}%
\definecolor{currentstroke}{rgb}{1.000000,0.576471,0.309804}%
\pgfsetstrokecolor{currentstroke}%
\pgfsetdash{}{0pt}%
\pgfsys@defobject{currentmarker}{\pgfqpoint{-0.033333in}{-0.033333in}}{\pgfqpoint{0.033333in}{0.033333in}}{%
\pgfpathmoveto{\pgfqpoint{0.000000in}{-0.033333in}}%
\pgfpathcurveto{\pgfqpoint{0.008840in}{-0.033333in}}{\pgfqpoint{0.017319in}{-0.029821in}}{\pgfqpoint{0.023570in}{-0.023570in}}%
\pgfpathcurveto{\pgfqpoint{0.029821in}{-0.017319in}}{\pgfqpoint{0.033333in}{-0.008840in}}{\pgfqpoint{0.033333in}{0.000000in}}%
\pgfpathcurveto{\pgfqpoint{0.033333in}{0.008840in}}{\pgfqpoint{0.029821in}{0.017319in}}{\pgfqpoint{0.023570in}{0.023570in}}%
\pgfpathcurveto{\pgfqpoint{0.017319in}{0.029821in}}{\pgfqpoint{0.008840in}{0.033333in}}{\pgfqpoint{0.000000in}{0.033333in}}%
\pgfpathcurveto{\pgfqpoint{-0.008840in}{0.033333in}}{\pgfqpoint{-0.017319in}{0.029821in}}{\pgfqpoint{-0.023570in}{0.023570in}}%
\pgfpathcurveto{\pgfqpoint{-0.029821in}{0.017319in}}{\pgfqpoint{-0.033333in}{0.008840in}}{\pgfqpoint{-0.033333in}{0.000000in}}%
\pgfpathcurveto{\pgfqpoint{-0.033333in}{-0.008840in}}{\pgfqpoint{-0.029821in}{-0.017319in}}{\pgfqpoint{-0.023570in}{-0.023570in}}%
\pgfpathcurveto{\pgfqpoint{-0.017319in}{-0.029821in}}{\pgfqpoint{-0.008840in}{-0.033333in}}{\pgfqpoint{0.000000in}{-0.033333in}}%
\pgfpathlineto{\pgfqpoint{0.000000in}{-0.033333in}}%
\pgfpathclose%
\pgfusepath{stroke,fill}%
}%
\begin{pgfscope}%
\pgfsys@transformshift{2.640771in}{2.146971in}%
\pgfsys@useobject{currentmarker}{}%
\end{pgfscope}%
\end{pgfscope}%
\begin{pgfscope}%
\pgfpathrectangle{\pgfqpoint{0.765000in}{0.660000in}}{\pgfqpoint{4.620000in}{4.620000in}}%
\pgfusepath{clip}%
\pgfsetrectcap%
\pgfsetroundjoin%
\pgfsetlinewidth{1.204500pt}%
\definecolor{currentstroke}{rgb}{1.000000,0.576471,0.309804}%
\pgfsetstrokecolor{currentstroke}%
\pgfsetdash{}{0pt}%
\pgfpathmoveto{\pgfqpoint{2.409507in}{1.178779in}}%
\pgfusepath{stroke}%
\end{pgfscope}%
\begin{pgfscope}%
\pgfpathrectangle{\pgfqpoint{0.765000in}{0.660000in}}{\pgfqpoint{4.620000in}{4.620000in}}%
\pgfusepath{clip}%
\pgfsetbuttcap%
\pgfsetroundjoin%
\definecolor{currentfill}{rgb}{1.000000,0.576471,0.309804}%
\pgfsetfillcolor{currentfill}%
\pgfsetlinewidth{1.003750pt}%
\definecolor{currentstroke}{rgb}{1.000000,0.576471,0.309804}%
\pgfsetstrokecolor{currentstroke}%
\pgfsetdash{}{0pt}%
\pgfsys@defobject{currentmarker}{\pgfqpoint{-0.033333in}{-0.033333in}}{\pgfqpoint{0.033333in}{0.033333in}}{%
\pgfpathmoveto{\pgfqpoint{0.000000in}{-0.033333in}}%
\pgfpathcurveto{\pgfqpoint{0.008840in}{-0.033333in}}{\pgfqpoint{0.017319in}{-0.029821in}}{\pgfqpoint{0.023570in}{-0.023570in}}%
\pgfpathcurveto{\pgfqpoint{0.029821in}{-0.017319in}}{\pgfqpoint{0.033333in}{-0.008840in}}{\pgfqpoint{0.033333in}{0.000000in}}%
\pgfpathcurveto{\pgfqpoint{0.033333in}{0.008840in}}{\pgfqpoint{0.029821in}{0.017319in}}{\pgfqpoint{0.023570in}{0.023570in}}%
\pgfpathcurveto{\pgfqpoint{0.017319in}{0.029821in}}{\pgfqpoint{0.008840in}{0.033333in}}{\pgfqpoint{0.000000in}{0.033333in}}%
\pgfpathcurveto{\pgfqpoint{-0.008840in}{0.033333in}}{\pgfqpoint{-0.017319in}{0.029821in}}{\pgfqpoint{-0.023570in}{0.023570in}}%
\pgfpathcurveto{\pgfqpoint{-0.029821in}{0.017319in}}{\pgfqpoint{-0.033333in}{0.008840in}}{\pgfqpoint{-0.033333in}{0.000000in}}%
\pgfpathcurveto{\pgfqpoint{-0.033333in}{-0.008840in}}{\pgfqpoint{-0.029821in}{-0.017319in}}{\pgfqpoint{-0.023570in}{-0.023570in}}%
\pgfpathcurveto{\pgfqpoint{-0.017319in}{-0.029821in}}{\pgfqpoint{-0.008840in}{-0.033333in}}{\pgfqpoint{0.000000in}{-0.033333in}}%
\pgfpathlineto{\pgfqpoint{0.000000in}{-0.033333in}}%
\pgfpathclose%
\pgfusepath{stroke,fill}%
}%
\begin{pgfscope}%
\pgfsys@transformshift{2.409507in}{1.178779in}%
\pgfsys@useobject{currentmarker}{}%
\end{pgfscope}%
\end{pgfscope}%
\begin{pgfscope}%
\pgfpathrectangle{\pgfqpoint{0.765000in}{0.660000in}}{\pgfqpoint{4.620000in}{4.620000in}}%
\pgfusepath{clip}%
\pgfsetrectcap%
\pgfsetroundjoin%
\pgfsetlinewidth{1.204500pt}%
\definecolor{currentstroke}{rgb}{1.000000,0.576471,0.309804}%
\pgfsetstrokecolor{currentstroke}%
\pgfsetdash{}{0pt}%
\pgfpathmoveto{\pgfqpoint{2.724838in}{3.656969in}}%
\pgfusepath{stroke}%
\end{pgfscope}%
\begin{pgfscope}%
\pgfpathrectangle{\pgfqpoint{0.765000in}{0.660000in}}{\pgfqpoint{4.620000in}{4.620000in}}%
\pgfusepath{clip}%
\pgfsetbuttcap%
\pgfsetroundjoin%
\definecolor{currentfill}{rgb}{1.000000,0.576471,0.309804}%
\pgfsetfillcolor{currentfill}%
\pgfsetlinewidth{1.003750pt}%
\definecolor{currentstroke}{rgb}{1.000000,0.576471,0.309804}%
\pgfsetstrokecolor{currentstroke}%
\pgfsetdash{}{0pt}%
\pgfsys@defobject{currentmarker}{\pgfqpoint{-0.033333in}{-0.033333in}}{\pgfqpoint{0.033333in}{0.033333in}}{%
\pgfpathmoveto{\pgfqpoint{0.000000in}{-0.033333in}}%
\pgfpathcurveto{\pgfqpoint{0.008840in}{-0.033333in}}{\pgfqpoint{0.017319in}{-0.029821in}}{\pgfqpoint{0.023570in}{-0.023570in}}%
\pgfpathcurveto{\pgfqpoint{0.029821in}{-0.017319in}}{\pgfqpoint{0.033333in}{-0.008840in}}{\pgfqpoint{0.033333in}{0.000000in}}%
\pgfpathcurveto{\pgfqpoint{0.033333in}{0.008840in}}{\pgfqpoint{0.029821in}{0.017319in}}{\pgfqpoint{0.023570in}{0.023570in}}%
\pgfpathcurveto{\pgfqpoint{0.017319in}{0.029821in}}{\pgfqpoint{0.008840in}{0.033333in}}{\pgfqpoint{0.000000in}{0.033333in}}%
\pgfpathcurveto{\pgfqpoint{-0.008840in}{0.033333in}}{\pgfqpoint{-0.017319in}{0.029821in}}{\pgfqpoint{-0.023570in}{0.023570in}}%
\pgfpathcurveto{\pgfqpoint{-0.029821in}{0.017319in}}{\pgfqpoint{-0.033333in}{0.008840in}}{\pgfqpoint{-0.033333in}{0.000000in}}%
\pgfpathcurveto{\pgfqpoint{-0.033333in}{-0.008840in}}{\pgfqpoint{-0.029821in}{-0.017319in}}{\pgfqpoint{-0.023570in}{-0.023570in}}%
\pgfpathcurveto{\pgfqpoint{-0.017319in}{-0.029821in}}{\pgfqpoint{-0.008840in}{-0.033333in}}{\pgfqpoint{0.000000in}{-0.033333in}}%
\pgfpathlineto{\pgfqpoint{0.000000in}{-0.033333in}}%
\pgfpathclose%
\pgfusepath{stroke,fill}%
}%
\begin{pgfscope}%
\pgfsys@transformshift{2.724838in}{3.656969in}%
\pgfsys@useobject{currentmarker}{}%
\end{pgfscope}%
\end{pgfscope}%
\begin{pgfscope}%
\pgfpathrectangle{\pgfqpoint{0.765000in}{0.660000in}}{\pgfqpoint{4.620000in}{4.620000in}}%
\pgfusepath{clip}%
\pgfsetrectcap%
\pgfsetroundjoin%
\pgfsetlinewidth{1.204500pt}%
\definecolor{currentstroke}{rgb}{1.000000,0.576471,0.309804}%
\pgfsetstrokecolor{currentstroke}%
\pgfsetdash{}{0pt}%
\pgfpathmoveto{\pgfqpoint{2.144568in}{2.965188in}}%
\pgfusepath{stroke}%
\end{pgfscope}%
\begin{pgfscope}%
\pgfpathrectangle{\pgfqpoint{0.765000in}{0.660000in}}{\pgfqpoint{4.620000in}{4.620000in}}%
\pgfusepath{clip}%
\pgfsetbuttcap%
\pgfsetroundjoin%
\definecolor{currentfill}{rgb}{1.000000,0.576471,0.309804}%
\pgfsetfillcolor{currentfill}%
\pgfsetlinewidth{1.003750pt}%
\definecolor{currentstroke}{rgb}{1.000000,0.576471,0.309804}%
\pgfsetstrokecolor{currentstroke}%
\pgfsetdash{}{0pt}%
\pgfsys@defobject{currentmarker}{\pgfqpoint{-0.033333in}{-0.033333in}}{\pgfqpoint{0.033333in}{0.033333in}}{%
\pgfpathmoveto{\pgfqpoint{0.000000in}{-0.033333in}}%
\pgfpathcurveto{\pgfqpoint{0.008840in}{-0.033333in}}{\pgfqpoint{0.017319in}{-0.029821in}}{\pgfqpoint{0.023570in}{-0.023570in}}%
\pgfpathcurveto{\pgfqpoint{0.029821in}{-0.017319in}}{\pgfqpoint{0.033333in}{-0.008840in}}{\pgfqpoint{0.033333in}{0.000000in}}%
\pgfpathcurveto{\pgfqpoint{0.033333in}{0.008840in}}{\pgfqpoint{0.029821in}{0.017319in}}{\pgfqpoint{0.023570in}{0.023570in}}%
\pgfpathcurveto{\pgfqpoint{0.017319in}{0.029821in}}{\pgfqpoint{0.008840in}{0.033333in}}{\pgfqpoint{0.000000in}{0.033333in}}%
\pgfpathcurveto{\pgfqpoint{-0.008840in}{0.033333in}}{\pgfqpoint{-0.017319in}{0.029821in}}{\pgfqpoint{-0.023570in}{0.023570in}}%
\pgfpathcurveto{\pgfqpoint{-0.029821in}{0.017319in}}{\pgfqpoint{-0.033333in}{0.008840in}}{\pgfqpoint{-0.033333in}{0.000000in}}%
\pgfpathcurveto{\pgfqpoint{-0.033333in}{-0.008840in}}{\pgfqpoint{-0.029821in}{-0.017319in}}{\pgfqpoint{-0.023570in}{-0.023570in}}%
\pgfpathcurveto{\pgfqpoint{-0.017319in}{-0.029821in}}{\pgfqpoint{-0.008840in}{-0.033333in}}{\pgfqpoint{0.000000in}{-0.033333in}}%
\pgfpathlineto{\pgfqpoint{0.000000in}{-0.033333in}}%
\pgfpathclose%
\pgfusepath{stroke,fill}%
}%
\begin{pgfscope}%
\pgfsys@transformshift{2.144568in}{2.965188in}%
\pgfsys@useobject{currentmarker}{}%
\end{pgfscope}%
\end{pgfscope}%
\begin{pgfscope}%
\pgfpathrectangle{\pgfqpoint{0.765000in}{0.660000in}}{\pgfqpoint{4.620000in}{4.620000in}}%
\pgfusepath{clip}%
\pgfsetrectcap%
\pgfsetroundjoin%
\pgfsetlinewidth{1.204500pt}%
\definecolor{currentstroke}{rgb}{1.000000,0.576471,0.309804}%
\pgfsetstrokecolor{currentstroke}%
\pgfsetdash{}{0pt}%
\pgfpathmoveto{\pgfqpoint{3.044015in}{4.338936in}}%
\pgfusepath{stroke}%
\end{pgfscope}%
\begin{pgfscope}%
\pgfpathrectangle{\pgfqpoint{0.765000in}{0.660000in}}{\pgfqpoint{4.620000in}{4.620000in}}%
\pgfusepath{clip}%
\pgfsetbuttcap%
\pgfsetroundjoin%
\definecolor{currentfill}{rgb}{1.000000,0.576471,0.309804}%
\pgfsetfillcolor{currentfill}%
\pgfsetlinewidth{1.003750pt}%
\definecolor{currentstroke}{rgb}{1.000000,0.576471,0.309804}%
\pgfsetstrokecolor{currentstroke}%
\pgfsetdash{}{0pt}%
\pgfsys@defobject{currentmarker}{\pgfqpoint{-0.033333in}{-0.033333in}}{\pgfqpoint{0.033333in}{0.033333in}}{%
\pgfpathmoveto{\pgfqpoint{0.000000in}{-0.033333in}}%
\pgfpathcurveto{\pgfqpoint{0.008840in}{-0.033333in}}{\pgfqpoint{0.017319in}{-0.029821in}}{\pgfqpoint{0.023570in}{-0.023570in}}%
\pgfpathcurveto{\pgfqpoint{0.029821in}{-0.017319in}}{\pgfqpoint{0.033333in}{-0.008840in}}{\pgfqpoint{0.033333in}{0.000000in}}%
\pgfpathcurveto{\pgfqpoint{0.033333in}{0.008840in}}{\pgfqpoint{0.029821in}{0.017319in}}{\pgfqpoint{0.023570in}{0.023570in}}%
\pgfpathcurveto{\pgfqpoint{0.017319in}{0.029821in}}{\pgfqpoint{0.008840in}{0.033333in}}{\pgfqpoint{0.000000in}{0.033333in}}%
\pgfpathcurveto{\pgfqpoint{-0.008840in}{0.033333in}}{\pgfqpoint{-0.017319in}{0.029821in}}{\pgfqpoint{-0.023570in}{0.023570in}}%
\pgfpathcurveto{\pgfqpoint{-0.029821in}{0.017319in}}{\pgfqpoint{-0.033333in}{0.008840in}}{\pgfqpoint{-0.033333in}{0.000000in}}%
\pgfpathcurveto{\pgfqpoint{-0.033333in}{-0.008840in}}{\pgfqpoint{-0.029821in}{-0.017319in}}{\pgfqpoint{-0.023570in}{-0.023570in}}%
\pgfpathcurveto{\pgfqpoint{-0.017319in}{-0.029821in}}{\pgfqpoint{-0.008840in}{-0.033333in}}{\pgfqpoint{0.000000in}{-0.033333in}}%
\pgfpathlineto{\pgfqpoint{0.000000in}{-0.033333in}}%
\pgfpathclose%
\pgfusepath{stroke,fill}%
}%
\begin{pgfscope}%
\pgfsys@transformshift{3.044015in}{4.338936in}%
\pgfsys@useobject{currentmarker}{}%
\end{pgfscope}%
\end{pgfscope}%
\begin{pgfscope}%
\pgfpathrectangle{\pgfqpoint{0.765000in}{0.660000in}}{\pgfqpoint{4.620000in}{4.620000in}}%
\pgfusepath{clip}%
\pgfsetrectcap%
\pgfsetroundjoin%
\pgfsetlinewidth{1.204500pt}%
\definecolor{currentstroke}{rgb}{1.000000,0.576471,0.309804}%
\pgfsetstrokecolor{currentstroke}%
\pgfsetdash{}{0pt}%
\pgfpathmoveto{\pgfqpoint{2.335251in}{3.797662in}}%
\pgfusepath{stroke}%
\end{pgfscope}%
\begin{pgfscope}%
\pgfpathrectangle{\pgfqpoint{0.765000in}{0.660000in}}{\pgfqpoint{4.620000in}{4.620000in}}%
\pgfusepath{clip}%
\pgfsetbuttcap%
\pgfsetroundjoin%
\definecolor{currentfill}{rgb}{1.000000,0.576471,0.309804}%
\pgfsetfillcolor{currentfill}%
\pgfsetlinewidth{1.003750pt}%
\definecolor{currentstroke}{rgb}{1.000000,0.576471,0.309804}%
\pgfsetstrokecolor{currentstroke}%
\pgfsetdash{}{0pt}%
\pgfsys@defobject{currentmarker}{\pgfqpoint{-0.033333in}{-0.033333in}}{\pgfqpoint{0.033333in}{0.033333in}}{%
\pgfpathmoveto{\pgfqpoint{0.000000in}{-0.033333in}}%
\pgfpathcurveto{\pgfqpoint{0.008840in}{-0.033333in}}{\pgfqpoint{0.017319in}{-0.029821in}}{\pgfqpoint{0.023570in}{-0.023570in}}%
\pgfpathcurveto{\pgfqpoint{0.029821in}{-0.017319in}}{\pgfqpoint{0.033333in}{-0.008840in}}{\pgfqpoint{0.033333in}{0.000000in}}%
\pgfpathcurveto{\pgfqpoint{0.033333in}{0.008840in}}{\pgfqpoint{0.029821in}{0.017319in}}{\pgfqpoint{0.023570in}{0.023570in}}%
\pgfpathcurveto{\pgfqpoint{0.017319in}{0.029821in}}{\pgfqpoint{0.008840in}{0.033333in}}{\pgfqpoint{0.000000in}{0.033333in}}%
\pgfpathcurveto{\pgfqpoint{-0.008840in}{0.033333in}}{\pgfqpoint{-0.017319in}{0.029821in}}{\pgfqpoint{-0.023570in}{0.023570in}}%
\pgfpathcurveto{\pgfqpoint{-0.029821in}{0.017319in}}{\pgfqpoint{-0.033333in}{0.008840in}}{\pgfqpoint{-0.033333in}{0.000000in}}%
\pgfpathcurveto{\pgfqpoint{-0.033333in}{-0.008840in}}{\pgfqpoint{-0.029821in}{-0.017319in}}{\pgfqpoint{-0.023570in}{-0.023570in}}%
\pgfpathcurveto{\pgfqpoint{-0.017319in}{-0.029821in}}{\pgfqpoint{-0.008840in}{-0.033333in}}{\pgfqpoint{0.000000in}{-0.033333in}}%
\pgfpathlineto{\pgfqpoint{0.000000in}{-0.033333in}}%
\pgfpathclose%
\pgfusepath{stroke,fill}%
}%
\begin{pgfscope}%
\pgfsys@transformshift{2.335251in}{3.797662in}%
\pgfsys@useobject{currentmarker}{}%
\end{pgfscope}%
\end{pgfscope}%
\begin{pgfscope}%
\pgfpathrectangle{\pgfqpoint{0.765000in}{0.660000in}}{\pgfqpoint{4.620000in}{4.620000in}}%
\pgfusepath{clip}%
\pgfsetrectcap%
\pgfsetroundjoin%
\pgfsetlinewidth{1.204500pt}%
\definecolor{currentstroke}{rgb}{1.000000,0.576471,0.309804}%
\pgfsetstrokecolor{currentstroke}%
\pgfsetdash{}{0pt}%
\pgfpathmoveto{\pgfqpoint{2.609790in}{1.353408in}}%
\pgfusepath{stroke}%
\end{pgfscope}%
\begin{pgfscope}%
\pgfpathrectangle{\pgfqpoint{0.765000in}{0.660000in}}{\pgfqpoint{4.620000in}{4.620000in}}%
\pgfusepath{clip}%
\pgfsetbuttcap%
\pgfsetroundjoin%
\definecolor{currentfill}{rgb}{1.000000,0.576471,0.309804}%
\pgfsetfillcolor{currentfill}%
\pgfsetlinewidth{1.003750pt}%
\definecolor{currentstroke}{rgb}{1.000000,0.576471,0.309804}%
\pgfsetstrokecolor{currentstroke}%
\pgfsetdash{}{0pt}%
\pgfsys@defobject{currentmarker}{\pgfqpoint{-0.033333in}{-0.033333in}}{\pgfqpoint{0.033333in}{0.033333in}}{%
\pgfpathmoveto{\pgfqpoint{0.000000in}{-0.033333in}}%
\pgfpathcurveto{\pgfqpoint{0.008840in}{-0.033333in}}{\pgfqpoint{0.017319in}{-0.029821in}}{\pgfqpoint{0.023570in}{-0.023570in}}%
\pgfpathcurveto{\pgfqpoint{0.029821in}{-0.017319in}}{\pgfqpoint{0.033333in}{-0.008840in}}{\pgfqpoint{0.033333in}{0.000000in}}%
\pgfpathcurveto{\pgfqpoint{0.033333in}{0.008840in}}{\pgfqpoint{0.029821in}{0.017319in}}{\pgfqpoint{0.023570in}{0.023570in}}%
\pgfpathcurveto{\pgfqpoint{0.017319in}{0.029821in}}{\pgfqpoint{0.008840in}{0.033333in}}{\pgfqpoint{0.000000in}{0.033333in}}%
\pgfpathcurveto{\pgfqpoint{-0.008840in}{0.033333in}}{\pgfqpoint{-0.017319in}{0.029821in}}{\pgfqpoint{-0.023570in}{0.023570in}}%
\pgfpathcurveto{\pgfqpoint{-0.029821in}{0.017319in}}{\pgfqpoint{-0.033333in}{0.008840in}}{\pgfqpoint{-0.033333in}{0.000000in}}%
\pgfpathcurveto{\pgfqpoint{-0.033333in}{-0.008840in}}{\pgfqpoint{-0.029821in}{-0.017319in}}{\pgfqpoint{-0.023570in}{-0.023570in}}%
\pgfpathcurveto{\pgfqpoint{-0.017319in}{-0.029821in}}{\pgfqpoint{-0.008840in}{-0.033333in}}{\pgfqpoint{0.000000in}{-0.033333in}}%
\pgfpathlineto{\pgfqpoint{0.000000in}{-0.033333in}}%
\pgfpathclose%
\pgfusepath{stroke,fill}%
}%
\begin{pgfscope}%
\pgfsys@transformshift{2.609790in}{1.353408in}%
\pgfsys@useobject{currentmarker}{}%
\end{pgfscope}%
\end{pgfscope}%
\begin{pgfscope}%
\pgfpathrectangle{\pgfqpoint{0.765000in}{0.660000in}}{\pgfqpoint{4.620000in}{4.620000in}}%
\pgfusepath{clip}%
\pgfsetrectcap%
\pgfsetroundjoin%
\pgfsetlinewidth{1.204500pt}%
\definecolor{currentstroke}{rgb}{1.000000,0.576471,0.309804}%
\pgfsetstrokecolor{currentstroke}%
\pgfsetdash{}{0pt}%
\pgfpathmoveto{\pgfqpoint{3.963409in}{2.452525in}}%
\pgfusepath{stroke}%
\end{pgfscope}%
\begin{pgfscope}%
\pgfpathrectangle{\pgfqpoint{0.765000in}{0.660000in}}{\pgfqpoint{4.620000in}{4.620000in}}%
\pgfusepath{clip}%
\pgfsetbuttcap%
\pgfsetroundjoin%
\definecolor{currentfill}{rgb}{1.000000,0.576471,0.309804}%
\pgfsetfillcolor{currentfill}%
\pgfsetlinewidth{1.003750pt}%
\definecolor{currentstroke}{rgb}{1.000000,0.576471,0.309804}%
\pgfsetstrokecolor{currentstroke}%
\pgfsetdash{}{0pt}%
\pgfsys@defobject{currentmarker}{\pgfqpoint{-0.033333in}{-0.033333in}}{\pgfqpoint{0.033333in}{0.033333in}}{%
\pgfpathmoveto{\pgfqpoint{0.000000in}{-0.033333in}}%
\pgfpathcurveto{\pgfqpoint{0.008840in}{-0.033333in}}{\pgfqpoint{0.017319in}{-0.029821in}}{\pgfqpoint{0.023570in}{-0.023570in}}%
\pgfpathcurveto{\pgfqpoint{0.029821in}{-0.017319in}}{\pgfqpoint{0.033333in}{-0.008840in}}{\pgfqpoint{0.033333in}{0.000000in}}%
\pgfpathcurveto{\pgfqpoint{0.033333in}{0.008840in}}{\pgfqpoint{0.029821in}{0.017319in}}{\pgfqpoint{0.023570in}{0.023570in}}%
\pgfpathcurveto{\pgfqpoint{0.017319in}{0.029821in}}{\pgfqpoint{0.008840in}{0.033333in}}{\pgfqpoint{0.000000in}{0.033333in}}%
\pgfpathcurveto{\pgfqpoint{-0.008840in}{0.033333in}}{\pgfqpoint{-0.017319in}{0.029821in}}{\pgfqpoint{-0.023570in}{0.023570in}}%
\pgfpathcurveto{\pgfqpoint{-0.029821in}{0.017319in}}{\pgfqpoint{-0.033333in}{0.008840in}}{\pgfqpoint{-0.033333in}{0.000000in}}%
\pgfpathcurveto{\pgfqpoint{-0.033333in}{-0.008840in}}{\pgfqpoint{-0.029821in}{-0.017319in}}{\pgfqpoint{-0.023570in}{-0.023570in}}%
\pgfpathcurveto{\pgfqpoint{-0.017319in}{-0.029821in}}{\pgfqpoint{-0.008840in}{-0.033333in}}{\pgfqpoint{0.000000in}{-0.033333in}}%
\pgfpathlineto{\pgfqpoint{0.000000in}{-0.033333in}}%
\pgfpathclose%
\pgfusepath{stroke,fill}%
}%
\begin{pgfscope}%
\pgfsys@transformshift{3.963409in}{2.452525in}%
\pgfsys@useobject{currentmarker}{}%
\end{pgfscope}%
\end{pgfscope}%
\begin{pgfscope}%
\pgfpathrectangle{\pgfqpoint{0.765000in}{0.660000in}}{\pgfqpoint{4.620000in}{4.620000in}}%
\pgfusepath{clip}%
\pgfsetrectcap%
\pgfsetroundjoin%
\pgfsetlinewidth{1.204500pt}%
\definecolor{currentstroke}{rgb}{1.000000,0.576471,0.309804}%
\pgfsetstrokecolor{currentstroke}%
\pgfsetdash{}{0pt}%
\pgfpathmoveto{\pgfqpoint{3.286641in}{2.212006in}}%
\pgfusepath{stroke}%
\end{pgfscope}%
\begin{pgfscope}%
\pgfpathrectangle{\pgfqpoint{0.765000in}{0.660000in}}{\pgfqpoint{4.620000in}{4.620000in}}%
\pgfusepath{clip}%
\pgfsetbuttcap%
\pgfsetroundjoin%
\definecolor{currentfill}{rgb}{1.000000,0.576471,0.309804}%
\pgfsetfillcolor{currentfill}%
\pgfsetlinewidth{1.003750pt}%
\definecolor{currentstroke}{rgb}{1.000000,0.576471,0.309804}%
\pgfsetstrokecolor{currentstroke}%
\pgfsetdash{}{0pt}%
\pgfsys@defobject{currentmarker}{\pgfqpoint{-0.033333in}{-0.033333in}}{\pgfqpoint{0.033333in}{0.033333in}}{%
\pgfpathmoveto{\pgfqpoint{0.000000in}{-0.033333in}}%
\pgfpathcurveto{\pgfqpoint{0.008840in}{-0.033333in}}{\pgfqpoint{0.017319in}{-0.029821in}}{\pgfqpoint{0.023570in}{-0.023570in}}%
\pgfpathcurveto{\pgfqpoint{0.029821in}{-0.017319in}}{\pgfqpoint{0.033333in}{-0.008840in}}{\pgfqpoint{0.033333in}{0.000000in}}%
\pgfpathcurveto{\pgfqpoint{0.033333in}{0.008840in}}{\pgfqpoint{0.029821in}{0.017319in}}{\pgfqpoint{0.023570in}{0.023570in}}%
\pgfpathcurveto{\pgfqpoint{0.017319in}{0.029821in}}{\pgfqpoint{0.008840in}{0.033333in}}{\pgfqpoint{0.000000in}{0.033333in}}%
\pgfpathcurveto{\pgfqpoint{-0.008840in}{0.033333in}}{\pgfqpoint{-0.017319in}{0.029821in}}{\pgfqpoint{-0.023570in}{0.023570in}}%
\pgfpathcurveto{\pgfqpoint{-0.029821in}{0.017319in}}{\pgfqpoint{-0.033333in}{0.008840in}}{\pgfqpoint{-0.033333in}{0.000000in}}%
\pgfpathcurveto{\pgfqpoint{-0.033333in}{-0.008840in}}{\pgfqpoint{-0.029821in}{-0.017319in}}{\pgfqpoint{-0.023570in}{-0.023570in}}%
\pgfpathcurveto{\pgfqpoint{-0.017319in}{-0.029821in}}{\pgfqpoint{-0.008840in}{-0.033333in}}{\pgfqpoint{0.000000in}{-0.033333in}}%
\pgfpathlineto{\pgfqpoint{0.000000in}{-0.033333in}}%
\pgfpathclose%
\pgfusepath{stroke,fill}%
}%
\begin{pgfscope}%
\pgfsys@transformshift{3.286641in}{2.212006in}%
\pgfsys@useobject{currentmarker}{}%
\end{pgfscope}%
\end{pgfscope}%
\begin{pgfscope}%
\pgfpathrectangle{\pgfqpoint{0.765000in}{0.660000in}}{\pgfqpoint{4.620000in}{4.620000in}}%
\pgfusepath{clip}%
\pgfsetrectcap%
\pgfsetroundjoin%
\pgfsetlinewidth{1.204500pt}%
\definecolor{currentstroke}{rgb}{1.000000,0.576471,0.309804}%
\pgfsetstrokecolor{currentstroke}%
\pgfsetdash{}{0pt}%
\pgfpathmoveto{\pgfqpoint{1.612957in}{3.401627in}}%
\pgfusepath{stroke}%
\end{pgfscope}%
\begin{pgfscope}%
\pgfpathrectangle{\pgfqpoint{0.765000in}{0.660000in}}{\pgfqpoint{4.620000in}{4.620000in}}%
\pgfusepath{clip}%
\pgfsetbuttcap%
\pgfsetroundjoin%
\definecolor{currentfill}{rgb}{1.000000,0.576471,0.309804}%
\pgfsetfillcolor{currentfill}%
\pgfsetlinewidth{1.003750pt}%
\definecolor{currentstroke}{rgb}{1.000000,0.576471,0.309804}%
\pgfsetstrokecolor{currentstroke}%
\pgfsetdash{}{0pt}%
\pgfsys@defobject{currentmarker}{\pgfqpoint{-0.033333in}{-0.033333in}}{\pgfqpoint{0.033333in}{0.033333in}}{%
\pgfpathmoveto{\pgfqpoint{0.000000in}{-0.033333in}}%
\pgfpathcurveto{\pgfqpoint{0.008840in}{-0.033333in}}{\pgfqpoint{0.017319in}{-0.029821in}}{\pgfqpoint{0.023570in}{-0.023570in}}%
\pgfpathcurveto{\pgfqpoint{0.029821in}{-0.017319in}}{\pgfqpoint{0.033333in}{-0.008840in}}{\pgfqpoint{0.033333in}{0.000000in}}%
\pgfpathcurveto{\pgfqpoint{0.033333in}{0.008840in}}{\pgfqpoint{0.029821in}{0.017319in}}{\pgfqpoint{0.023570in}{0.023570in}}%
\pgfpathcurveto{\pgfqpoint{0.017319in}{0.029821in}}{\pgfqpoint{0.008840in}{0.033333in}}{\pgfqpoint{0.000000in}{0.033333in}}%
\pgfpathcurveto{\pgfqpoint{-0.008840in}{0.033333in}}{\pgfqpoint{-0.017319in}{0.029821in}}{\pgfqpoint{-0.023570in}{0.023570in}}%
\pgfpathcurveto{\pgfqpoint{-0.029821in}{0.017319in}}{\pgfqpoint{-0.033333in}{0.008840in}}{\pgfqpoint{-0.033333in}{0.000000in}}%
\pgfpathcurveto{\pgfqpoint{-0.033333in}{-0.008840in}}{\pgfqpoint{-0.029821in}{-0.017319in}}{\pgfqpoint{-0.023570in}{-0.023570in}}%
\pgfpathcurveto{\pgfqpoint{-0.017319in}{-0.029821in}}{\pgfqpoint{-0.008840in}{-0.033333in}}{\pgfqpoint{0.000000in}{-0.033333in}}%
\pgfpathlineto{\pgfqpoint{0.000000in}{-0.033333in}}%
\pgfpathclose%
\pgfusepath{stroke,fill}%
}%
\begin{pgfscope}%
\pgfsys@transformshift{1.612957in}{3.401627in}%
\pgfsys@useobject{currentmarker}{}%
\end{pgfscope}%
\end{pgfscope}%
\begin{pgfscope}%
\pgfpathrectangle{\pgfqpoint{0.765000in}{0.660000in}}{\pgfqpoint{4.620000in}{4.620000in}}%
\pgfusepath{clip}%
\pgfsetrectcap%
\pgfsetroundjoin%
\pgfsetlinewidth{1.204500pt}%
\definecolor{currentstroke}{rgb}{1.000000,0.576471,0.309804}%
\pgfsetstrokecolor{currentstroke}%
\pgfsetdash{}{0pt}%
\pgfpathmoveto{\pgfqpoint{2.654806in}{4.368094in}}%
\pgfusepath{stroke}%
\end{pgfscope}%
\begin{pgfscope}%
\pgfpathrectangle{\pgfqpoint{0.765000in}{0.660000in}}{\pgfqpoint{4.620000in}{4.620000in}}%
\pgfusepath{clip}%
\pgfsetbuttcap%
\pgfsetroundjoin%
\definecolor{currentfill}{rgb}{1.000000,0.576471,0.309804}%
\pgfsetfillcolor{currentfill}%
\pgfsetlinewidth{1.003750pt}%
\definecolor{currentstroke}{rgb}{1.000000,0.576471,0.309804}%
\pgfsetstrokecolor{currentstroke}%
\pgfsetdash{}{0pt}%
\pgfsys@defobject{currentmarker}{\pgfqpoint{-0.033333in}{-0.033333in}}{\pgfqpoint{0.033333in}{0.033333in}}{%
\pgfpathmoveto{\pgfqpoint{0.000000in}{-0.033333in}}%
\pgfpathcurveto{\pgfqpoint{0.008840in}{-0.033333in}}{\pgfqpoint{0.017319in}{-0.029821in}}{\pgfqpoint{0.023570in}{-0.023570in}}%
\pgfpathcurveto{\pgfqpoint{0.029821in}{-0.017319in}}{\pgfqpoint{0.033333in}{-0.008840in}}{\pgfqpoint{0.033333in}{0.000000in}}%
\pgfpathcurveto{\pgfqpoint{0.033333in}{0.008840in}}{\pgfqpoint{0.029821in}{0.017319in}}{\pgfqpoint{0.023570in}{0.023570in}}%
\pgfpathcurveto{\pgfqpoint{0.017319in}{0.029821in}}{\pgfqpoint{0.008840in}{0.033333in}}{\pgfqpoint{0.000000in}{0.033333in}}%
\pgfpathcurveto{\pgfqpoint{-0.008840in}{0.033333in}}{\pgfqpoint{-0.017319in}{0.029821in}}{\pgfqpoint{-0.023570in}{0.023570in}}%
\pgfpathcurveto{\pgfqpoint{-0.029821in}{0.017319in}}{\pgfqpoint{-0.033333in}{0.008840in}}{\pgfqpoint{-0.033333in}{0.000000in}}%
\pgfpathcurveto{\pgfqpoint{-0.033333in}{-0.008840in}}{\pgfqpoint{-0.029821in}{-0.017319in}}{\pgfqpoint{-0.023570in}{-0.023570in}}%
\pgfpathcurveto{\pgfqpoint{-0.017319in}{-0.029821in}}{\pgfqpoint{-0.008840in}{-0.033333in}}{\pgfqpoint{0.000000in}{-0.033333in}}%
\pgfpathlineto{\pgfqpoint{0.000000in}{-0.033333in}}%
\pgfpathclose%
\pgfusepath{stroke,fill}%
}%
\begin{pgfscope}%
\pgfsys@transformshift{2.654806in}{4.368094in}%
\pgfsys@useobject{currentmarker}{}%
\end{pgfscope}%
\end{pgfscope}%
\begin{pgfscope}%
\pgfpathrectangle{\pgfqpoint{0.765000in}{0.660000in}}{\pgfqpoint{4.620000in}{4.620000in}}%
\pgfusepath{clip}%
\pgfsetrectcap%
\pgfsetroundjoin%
\pgfsetlinewidth{1.204500pt}%
\definecolor{currentstroke}{rgb}{1.000000,0.576471,0.309804}%
\pgfsetstrokecolor{currentstroke}%
\pgfsetdash{}{0pt}%
\pgfpathmoveto{\pgfqpoint{3.600780in}{3.643618in}}%
\pgfusepath{stroke}%
\end{pgfscope}%
\begin{pgfscope}%
\pgfpathrectangle{\pgfqpoint{0.765000in}{0.660000in}}{\pgfqpoint{4.620000in}{4.620000in}}%
\pgfusepath{clip}%
\pgfsetbuttcap%
\pgfsetroundjoin%
\definecolor{currentfill}{rgb}{1.000000,0.576471,0.309804}%
\pgfsetfillcolor{currentfill}%
\pgfsetlinewidth{1.003750pt}%
\definecolor{currentstroke}{rgb}{1.000000,0.576471,0.309804}%
\pgfsetstrokecolor{currentstroke}%
\pgfsetdash{}{0pt}%
\pgfsys@defobject{currentmarker}{\pgfqpoint{-0.033333in}{-0.033333in}}{\pgfqpoint{0.033333in}{0.033333in}}{%
\pgfpathmoveto{\pgfqpoint{0.000000in}{-0.033333in}}%
\pgfpathcurveto{\pgfqpoint{0.008840in}{-0.033333in}}{\pgfqpoint{0.017319in}{-0.029821in}}{\pgfqpoint{0.023570in}{-0.023570in}}%
\pgfpathcurveto{\pgfqpoint{0.029821in}{-0.017319in}}{\pgfqpoint{0.033333in}{-0.008840in}}{\pgfqpoint{0.033333in}{0.000000in}}%
\pgfpathcurveto{\pgfqpoint{0.033333in}{0.008840in}}{\pgfqpoint{0.029821in}{0.017319in}}{\pgfqpoint{0.023570in}{0.023570in}}%
\pgfpathcurveto{\pgfqpoint{0.017319in}{0.029821in}}{\pgfqpoint{0.008840in}{0.033333in}}{\pgfqpoint{0.000000in}{0.033333in}}%
\pgfpathcurveto{\pgfqpoint{-0.008840in}{0.033333in}}{\pgfqpoint{-0.017319in}{0.029821in}}{\pgfqpoint{-0.023570in}{0.023570in}}%
\pgfpathcurveto{\pgfqpoint{-0.029821in}{0.017319in}}{\pgfqpoint{-0.033333in}{0.008840in}}{\pgfqpoint{-0.033333in}{0.000000in}}%
\pgfpathcurveto{\pgfqpoint{-0.033333in}{-0.008840in}}{\pgfqpoint{-0.029821in}{-0.017319in}}{\pgfqpoint{-0.023570in}{-0.023570in}}%
\pgfpathcurveto{\pgfqpoint{-0.017319in}{-0.029821in}}{\pgfqpoint{-0.008840in}{-0.033333in}}{\pgfqpoint{0.000000in}{-0.033333in}}%
\pgfpathlineto{\pgfqpoint{0.000000in}{-0.033333in}}%
\pgfpathclose%
\pgfusepath{stroke,fill}%
}%
\begin{pgfscope}%
\pgfsys@transformshift{3.600780in}{3.643618in}%
\pgfsys@useobject{currentmarker}{}%
\end{pgfscope}%
\end{pgfscope}%
\begin{pgfscope}%
\pgfpathrectangle{\pgfqpoint{0.765000in}{0.660000in}}{\pgfqpoint{4.620000in}{4.620000in}}%
\pgfusepath{clip}%
\pgfsetrectcap%
\pgfsetroundjoin%
\pgfsetlinewidth{1.204500pt}%
\definecolor{currentstroke}{rgb}{1.000000,0.576471,0.309804}%
\pgfsetstrokecolor{currentstroke}%
\pgfsetdash{}{0pt}%
\pgfpathmoveto{\pgfqpoint{1.758098in}{3.695680in}}%
\pgfusepath{stroke}%
\end{pgfscope}%
\begin{pgfscope}%
\pgfpathrectangle{\pgfqpoint{0.765000in}{0.660000in}}{\pgfqpoint{4.620000in}{4.620000in}}%
\pgfusepath{clip}%
\pgfsetbuttcap%
\pgfsetroundjoin%
\definecolor{currentfill}{rgb}{1.000000,0.576471,0.309804}%
\pgfsetfillcolor{currentfill}%
\pgfsetlinewidth{1.003750pt}%
\definecolor{currentstroke}{rgb}{1.000000,0.576471,0.309804}%
\pgfsetstrokecolor{currentstroke}%
\pgfsetdash{}{0pt}%
\pgfsys@defobject{currentmarker}{\pgfqpoint{-0.033333in}{-0.033333in}}{\pgfqpoint{0.033333in}{0.033333in}}{%
\pgfpathmoveto{\pgfqpoint{0.000000in}{-0.033333in}}%
\pgfpathcurveto{\pgfqpoint{0.008840in}{-0.033333in}}{\pgfqpoint{0.017319in}{-0.029821in}}{\pgfqpoint{0.023570in}{-0.023570in}}%
\pgfpathcurveto{\pgfqpoint{0.029821in}{-0.017319in}}{\pgfqpoint{0.033333in}{-0.008840in}}{\pgfqpoint{0.033333in}{0.000000in}}%
\pgfpathcurveto{\pgfqpoint{0.033333in}{0.008840in}}{\pgfqpoint{0.029821in}{0.017319in}}{\pgfqpoint{0.023570in}{0.023570in}}%
\pgfpathcurveto{\pgfqpoint{0.017319in}{0.029821in}}{\pgfqpoint{0.008840in}{0.033333in}}{\pgfqpoint{0.000000in}{0.033333in}}%
\pgfpathcurveto{\pgfqpoint{-0.008840in}{0.033333in}}{\pgfqpoint{-0.017319in}{0.029821in}}{\pgfqpoint{-0.023570in}{0.023570in}}%
\pgfpathcurveto{\pgfqpoint{-0.029821in}{0.017319in}}{\pgfqpoint{-0.033333in}{0.008840in}}{\pgfqpoint{-0.033333in}{0.000000in}}%
\pgfpathcurveto{\pgfqpoint{-0.033333in}{-0.008840in}}{\pgfqpoint{-0.029821in}{-0.017319in}}{\pgfqpoint{-0.023570in}{-0.023570in}}%
\pgfpathcurveto{\pgfqpoint{-0.017319in}{-0.029821in}}{\pgfqpoint{-0.008840in}{-0.033333in}}{\pgfqpoint{0.000000in}{-0.033333in}}%
\pgfpathlineto{\pgfqpoint{0.000000in}{-0.033333in}}%
\pgfpathclose%
\pgfusepath{stroke,fill}%
}%
\begin{pgfscope}%
\pgfsys@transformshift{1.758098in}{3.695680in}%
\pgfsys@useobject{currentmarker}{}%
\end{pgfscope}%
\end{pgfscope}%
\begin{pgfscope}%
\pgfpathrectangle{\pgfqpoint{0.765000in}{0.660000in}}{\pgfqpoint{4.620000in}{4.620000in}}%
\pgfusepath{clip}%
\pgfsetrectcap%
\pgfsetroundjoin%
\pgfsetlinewidth{1.204500pt}%
\definecolor{currentstroke}{rgb}{1.000000,0.576471,0.309804}%
\pgfsetstrokecolor{currentstroke}%
\pgfsetdash{}{0pt}%
\pgfpathmoveto{\pgfqpoint{1.783496in}{4.135000in}}%
\pgfusepath{stroke}%
\end{pgfscope}%
\begin{pgfscope}%
\pgfpathrectangle{\pgfqpoint{0.765000in}{0.660000in}}{\pgfqpoint{4.620000in}{4.620000in}}%
\pgfusepath{clip}%
\pgfsetbuttcap%
\pgfsetroundjoin%
\definecolor{currentfill}{rgb}{1.000000,0.576471,0.309804}%
\pgfsetfillcolor{currentfill}%
\pgfsetlinewidth{1.003750pt}%
\definecolor{currentstroke}{rgb}{1.000000,0.576471,0.309804}%
\pgfsetstrokecolor{currentstroke}%
\pgfsetdash{}{0pt}%
\pgfsys@defobject{currentmarker}{\pgfqpoint{-0.033333in}{-0.033333in}}{\pgfqpoint{0.033333in}{0.033333in}}{%
\pgfpathmoveto{\pgfqpoint{0.000000in}{-0.033333in}}%
\pgfpathcurveto{\pgfqpoint{0.008840in}{-0.033333in}}{\pgfqpoint{0.017319in}{-0.029821in}}{\pgfqpoint{0.023570in}{-0.023570in}}%
\pgfpathcurveto{\pgfqpoint{0.029821in}{-0.017319in}}{\pgfqpoint{0.033333in}{-0.008840in}}{\pgfqpoint{0.033333in}{0.000000in}}%
\pgfpathcurveto{\pgfqpoint{0.033333in}{0.008840in}}{\pgfqpoint{0.029821in}{0.017319in}}{\pgfqpoint{0.023570in}{0.023570in}}%
\pgfpathcurveto{\pgfqpoint{0.017319in}{0.029821in}}{\pgfqpoint{0.008840in}{0.033333in}}{\pgfqpoint{0.000000in}{0.033333in}}%
\pgfpathcurveto{\pgfqpoint{-0.008840in}{0.033333in}}{\pgfqpoint{-0.017319in}{0.029821in}}{\pgfqpoint{-0.023570in}{0.023570in}}%
\pgfpathcurveto{\pgfqpoint{-0.029821in}{0.017319in}}{\pgfqpoint{-0.033333in}{0.008840in}}{\pgfqpoint{-0.033333in}{0.000000in}}%
\pgfpathcurveto{\pgfqpoint{-0.033333in}{-0.008840in}}{\pgfqpoint{-0.029821in}{-0.017319in}}{\pgfqpoint{-0.023570in}{-0.023570in}}%
\pgfpathcurveto{\pgfqpoint{-0.017319in}{-0.029821in}}{\pgfqpoint{-0.008840in}{-0.033333in}}{\pgfqpoint{0.000000in}{-0.033333in}}%
\pgfpathlineto{\pgfqpoint{0.000000in}{-0.033333in}}%
\pgfpathclose%
\pgfusepath{stroke,fill}%
}%
\begin{pgfscope}%
\pgfsys@transformshift{1.783496in}{4.135000in}%
\pgfsys@useobject{currentmarker}{}%
\end{pgfscope}%
\end{pgfscope}%
\begin{pgfscope}%
\pgfpathrectangle{\pgfqpoint{0.765000in}{0.660000in}}{\pgfqpoint{4.620000in}{4.620000in}}%
\pgfusepath{clip}%
\pgfsetrectcap%
\pgfsetroundjoin%
\pgfsetlinewidth{1.204500pt}%
\definecolor{currentstroke}{rgb}{1.000000,0.576471,0.309804}%
\pgfsetstrokecolor{currentstroke}%
\pgfsetdash{}{0pt}%
\pgfpathmoveto{\pgfqpoint{3.986832in}{3.959663in}}%
\pgfusepath{stroke}%
\end{pgfscope}%
\begin{pgfscope}%
\pgfpathrectangle{\pgfqpoint{0.765000in}{0.660000in}}{\pgfqpoint{4.620000in}{4.620000in}}%
\pgfusepath{clip}%
\pgfsetbuttcap%
\pgfsetroundjoin%
\definecolor{currentfill}{rgb}{1.000000,0.576471,0.309804}%
\pgfsetfillcolor{currentfill}%
\pgfsetlinewidth{1.003750pt}%
\definecolor{currentstroke}{rgb}{1.000000,0.576471,0.309804}%
\pgfsetstrokecolor{currentstroke}%
\pgfsetdash{}{0pt}%
\pgfsys@defobject{currentmarker}{\pgfqpoint{-0.033333in}{-0.033333in}}{\pgfqpoint{0.033333in}{0.033333in}}{%
\pgfpathmoveto{\pgfqpoint{0.000000in}{-0.033333in}}%
\pgfpathcurveto{\pgfqpoint{0.008840in}{-0.033333in}}{\pgfqpoint{0.017319in}{-0.029821in}}{\pgfqpoint{0.023570in}{-0.023570in}}%
\pgfpathcurveto{\pgfqpoint{0.029821in}{-0.017319in}}{\pgfqpoint{0.033333in}{-0.008840in}}{\pgfqpoint{0.033333in}{0.000000in}}%
\pgfpathcurveto{\pgfqpoint{0.033333in}{0.008840in}}{\pgfqpoint{0.029821in}{0.017319in}}{\pgfqpoint{0.023570in}{0.023570in}}%
\pgfpathcurveto{\pgfqpoint{0.017319in}{0.029821in}}{\pgfqpoint{0.008840in}{0.033333in}}{\pgfqpoint{0.000000in}{0.033333in}}%
\pgfpathcurveto{\pgfqpoint{-0.008840in}{0.033333in}}{\pgfqpoint{-0.017319in}{0.029821in}}{\pgfqpoint{-0.023570in}{0.023570in}}%
\pgfpathcurveto{\pgfqpoint{-0.029821in}{0.017319in}}{\pgfqpoint{-0.033333in}{0.008840in}}{\pgfqpoint{-0.033333in}{0.000000in}}%
\pgfpathcurveto{\pgfqpoint{-0.033333in}{-0.008840in}}{\pgfqpoint{-0.029821in}{-0.017319in}}{\pgfqpoint{-0.023570in}{-0.023570in}}%
\pgfpathcurveto{\pgfqpoint{-0.017319in}{-0.029821in}}{\pgfqpoint{-0.008840in}{-0.033333in}}{\pgfqpoint{0.000000in}{-0.033333in}}%
\pgfpathlineto{\pgfqpoint{0.000000in}{-0.033333in}}%
\pgfpathclose%
\pgfusepath{stroke,fill}%
}%
\begin{pgfscope}%
\pgfsys@transformshift{3.986832in}{3.959663in}%
\pgfsys@useobject{currentmarker}{}%
\end{pgfscope}%
\end{pgfscope}%
\begin{pgfscope}%
\pgfpathrectangle{\pgfqpoint{0.765000in}{0.660000in}}{\pgfqpoint{4.620000in}{4.620000in}}%
\pgfusepath{clip}%
\pgfsetrectcap%
\pgfsetroundjoin%
\pgfsetlinewidth{1.204500pt}%
\definecolor{currentstroke}{rgb}{1.000000,0.576471,0.309804}%
\pgfsetstrokecolor{currentstroke}%
\pgfsetdash{}{0pt}%
\pgfpathmoveto{\pgfqpoint{2.628738in}{3.644561in}}%
\pgfusepath{stroke}%
\end{pgfscope}%
\begin{pgfscope}%
\pgfpathrectangle{\pgfqpoint{0.765000in}{0.660000in}}{\pgfqpoint{4.620000in}{4.620000in}}%
\pgfusepath{clip}%
\pgfsetbuttcap%
\pgfsetroundjoin%
\definecolor{currentfill}{rgb}{1.000000,0.576471,0.309804}%
\pgfsetfillcolor{currentfill}%
\pgfsetlinewidth{1.003750pt}%
\definecolor{currentstroke}{rgb}{1.000000,0.576471,0.309804}%
\pgfsetstrokecolor{currentstroke}%
\pgfsetdash{}{0pt}%
\pgfsys@defobject{currentmarker}{\pgfqpoint{-0.033333in}{-0.033333in}}{\pgfqpoint{0.033333in}{0.033333in}}{%
\pgfpathmoveto{\pgfqpoint{0.000000in}{-0.033333in}}%
\pgfpathcurveto{\pgfqpoint{0.008840in}{-0.033333in}}{\pgfqpoint{0.017319in}{-0.029821in}}{\pgfqpoint{0.023570in}{-0.023570in}}%
\pgfpathcurveto{\pgfqpoint{0.029821in}{-0.017319in}}{\pgfqpoint{0.033333in}{-0.008840in}}{\pgfqpoint{0.033333in}{0.000000in}}%
\pgfpathcurveto{\pgfqpoint{0.033333in}{0.008840in}}{\pgfqpoint{0.029821in}{0.017319in}}{\pgfqpoint{0.023570in}{0.023570in}}%
\pgfpathcurveto{\pgfqpoint{0.017319in}{0.029821in}}{\pgfqpoint{0.008840in}{0.033333in}}{\pgfqpoint{0.000000in}{0.033333in}}%
\pgfpathcurveto{\pgfqpoint{-0.008840in}{0.033333in}}{\pgfqpoint{-0.017319in}{0.029821in}}{\pgfqpoint{-0.023570in}{0.023570in}}%
\pgfpathcurveto{\pgfqpoint{-0.029821in}{0.017319in}}{\pgfqpoint{-0.033333in}{0.008840in}}{\pgfqpoint{-0.033333in}{0.000000in}}%
\pgfpathcurveto{\pgfqpoint{-0.033333in}{-0.008840in}}{\pgfqpoint{-0.029821in}{-0.017319in}}{\pgfqpoint{-0.023570in}{-0.023570in}}%
\pgfpathcurveto{\pgfqpoint{-0.017319in}{-0.029821in}}{\pgfqpoint{-0.008840in}{-0.033333in}}{\pgfqpoint{0.000000in}{-0.033333in}}%
\pgfpathlineto{\pgfqpoint{0.000000in}{-0.033333in}}%
\pgfpathclose%
\pgfusepath{stroke,fill}%
}%
\begin{pgfscope}%
\pgfsys@transformshift{2.628738in}{3.644561in}%
\pgfsys@useobject{currentmarker}{}%
\end{pgfscope}%
\end{pgfscope}%
\begin{pgfscope}%
\pgfpathrectangle{\pgfqpoint{0.765000in}{0.660000in}}{\pgfqpoint{4.620000in}{4.620000in}}%
\pgfusepath{clip}%
\pgfsetrectcap%
\pgfsetroundjoin%
\pgfsetlinewidth{1.204500pt}%
\definecolor{currentstroke}{rgb}{0.176471,0.192157,0.258824}%
\pgfsetstrokecolor{currentstroke}%
\pgfsetdash{}{0pt}%
\pgfpathmoveto{\pgfqpoint{3.028160in}{3.143114in}}%
\pgfusepath{stroke}%
\end{pgfscope}%
\begin{pgfscope}%
\pgfpathrectangle{\pgfqpoint{0.765000in}{0.660000in}}{\pgfqpoint{4.620000in}{4.620000in}}%
\pgfusepath{clip}%
\pgfsetbuttcap%
\pgfsetmiterjoin%
\definecolor{currentfill}{rgb}{0.176471,0.192157,0.258824}%
\pgfsetfillcolor{currentfill}%
\pgfsetlinewidth{1.003750pt}%
\definecolor{currentstroke}{rgb}{0.176471,0.192157,0.258824}%
\pgfsetstrokecolor{currentstroke}%
\pgfsetdash{}{0pt}%
\pgfsys@defobject{currentmarker}{\pgfqpoint{-0.033333in}{-0.033333in}}{\pgfqpoint{0.033333in}{0.033333in}}{%
\pgfpathmoveto{\pgfqpoint{-0.000000in}{-0.033333in}}%
\pgfpathlineto{\pgfqpoint{0.033333in}{0.033333in}}%
\pgfpathlineto{\pgfqpoint{-0.033333in}{0.033333in}}%
\pgfpathlineto{\pgfqpoint{-0.000000in}{-0.033333in}}%
\pgfpathclose%
\pgfusepath{stroke,fill}%
}%
\begin{pgfscope}%
\pgfsys@transformshift{3.028160in}{3.143114in}%
\pgfsys@useobject{currentmarker}{}%
\end{pgfscope}%
\end{pgfscope}%
\begin{pgfscope}%
\pgfpathrectangle{\pgfqpoint{0.765000in}{0.660000in}}{\pgfqpoint{4.620000in}{4.620000in}}%
\pgfusepath{clip}%
\pgfsetrectcap%
\pgfsetroundjoin%
\pgfsetlinewidth{1.204500pt}%
\definecolor{currentstroke}{rgb}{0.176471,0.192157,0.258824}%
\pgfsetstrokecolor{currentstroke}%
\pgfsetdash{}{0pt}%
\pgfpathmoveto{\pgfqpoint{2.969234in}{2.964526in}}%
\pgfusepath{stroke}%
\end{pgfscope}%
\begin{pgfscope}%
\pgfpathrectangle{\pgfqpoint{0.765000in}{0.660000in}}{\pgfqpoint{4.620000in}{4.620000in}}%
\pgfusepath{clip}%
\pgfsetbuttcap%
\pgfsetmiterjoin%
\definecolor{currentfill}{rgb}{0.176471,0.192157,0.258824}%
\pgfsetfillcolor{currentfill}%
\pgfsetlinewidth{1.003750pt}%
\definecolor{currentstroke}{rgb}{0.176471,0.192157,0.258824}%
\pgfsetstrokecolor{currentstroke}%
\pgfsetdash{}{0pt}%
\pgfsys@defobject{currentmarker}{\pgfqpoint{-0.033333in}{-0.033333in}}{\pgfqpoint{0.033333in}{0.033333in}}{%
\pgfpathmoveto{\pgfqpoint{-0.000000in}{-0.033333in}}%
\pgfpathlineto{\pgfqpoint{0.033333in}{0.033333in}}%
\pgfpathlineto{\pgfqpoint{-0.033333in}{0.033333in}}%
\pgfpathlineto{\pgfqpoint{-0.000000in}{-0.033333in}}%
\pgfpathclose%
\pgfusepath{stroke,fill}%
}%
\begin{pgfscope}%
\pgfsys@transformshift{2.969234in}{2.964526in}%
\pgfsys@useobject{currentmarker}{}%
\end{pgfscope}%
\end{pgfscope}%
\begin{pgfscope}%
\pgfpathrectangle{\pgfqpoint{0.765000in}{0.660000in}}{\pgfqpoint{4.620000in}{4.620000in}}%
\pgfusepath{clip}%
\pgfsetrectcap%
\pgfsetroundjoin%
\pgfsetlinewidth{1.204500pt}%
\definecolor{currentstroke}{rgb}{0.176471,0.192157,0.258824}%
\pgfsetstrokecolor{currentstroke}%
\pgfsetdash{}{0pt}%
\pgfpathmoveto{\pgfqpoint{3.057453in}{3.259269in}}%
\pgfusepath{stroke}%
\end{pgfscope}%
\begin{pgfscope}%
\pgfpathrectangle{\pgfqpoint{0.765000in}{0.660000in}}{\pgfqpoint{4.620000in}{4.620000in}}%
\pgfusepath{clip}%
\pgfsetbuttcap%
\pgfsetmiterjoin%
\definecolor{currentfill}{rgb}{0.176471,0.192157,0.258824}%
\pgfsetfillcolor{currentfill}%
\pgfsetlinewidth{1.003750pt}%
\definecolor{currentstroke}{rgb}{0.176471,0.192157,0.258824}%
\pgfsetstrokecolor{currentstroke}%
\pgfsetdash{}{0pt}%
\pgfsys@defobject{currentmarker}{\pgfqpoint{-0.033333in}{-0.033333in}}{\pgfqpoint{0.033333in}{0.033333in}}{%
\pgfpathmoveto{\pgfqpoint{-0.000000in}{-0.033333in}}%
\pgfpathlineto{\pgfqpoint{0.033333in}{0.033333in}}%
\pgfpathlineto{\pgfqpoint{-0.033333in}{0.033333in}}%
\pgfpathlineto{\pgfqpoint{-0.000000in}{-0.033333in}}%
\pgfpathclose%
\pgfusepath{stroke,fill}%
}%
\begin{pgfscope}%
\pgfsys@transformshift{3.057453in}{3.259269in}%
\pgfsys@useobject{currentmarker}{}%
\end{pgfscope}%
\end{pgfscope}%
\begin{pgfscope}%
\pgfpathrectangle{\pgfqpoint{0.765000in}{0.660000in}}{\pgfqpoint{4.620000in}{4.620000in}}%
\pgfusepath{clip}%
\pgfsetrectcap%
\pgfsetroundjoin%
\pgfsetlinewidth{1.204500pt}%
\definecolor{currentstroke}{rgb}{0.176471,0.192157,0.258824}%
\pgfsetstrokecolor{currentstroke}%
\pgfsetdash{}{0pt}%
\pgfpathmoveto{\pgfqpoint{3.175598in}{2.810411in}}%
\pgfusepath{stroke}%
\end{pgfscope}%
\begin{pgfscope}%
\pgfpathrectangle{\pgfqpoint{0.765000in}{0.660000in}}{\pgfqpoint{4.620000in}{4.620000in}}%
\pgfusepath{clip}%
\pgfsetbuttcap%
\pgfsetmiterjoin%
\definecolor{currentfill}{rgb}{0.176471,0.192157,0.258824}%
\pgfsetfillcolor{currentfill}%
\pgfsetlinewidth{1.003750pt}%
\definecolor{currentstroke}{rgb}{0.176471,0.192157,0.258824}%
\pgfsetstrokecolor{currentstroke}%
\pgfsetdash{}{0pt}%
\pgfsys@defobject{currentmarker}{\pgfqpoint{-0.033333in}{-0.033333in}}{\pgfqpoint{0.033333in}{0.033333in}}{%
\pgfpathmoveto{\pgfqpoint{-0.000000in}{-0.033333in}}%
\pgfpathlineto{\pgfqpoint{0.033333in}{0.033333in}}%
\pgfpathlineto{\pgfqpoint{-0.033333in}{0.033333in}}%
\pgfpathlineto{\pgfqpoint{-0.000000in}{-0.033333in}}%
\pgfpathclose%
\pgfusepath{stroke,fill}%
}%
\begin{pgfscope}%
\pgfsys@transformshift{3.175598in}{2.810411in}%
\pgfsys@useobject{currentmarker}{}%
\end{pgfscope}%
\end{pgfscope}%
\begin{pgfscope}%
\pgfpathrectangle{\pgfqpoint{0.765000in}{0.660000in}}{\pgfqpoint{4.620000in}{4.620000in}}%
\pgfusepath{clip}%
\pgfsetrectcap%
\pgfsetroundjoin%
\pgfsetlinewidth{1.204500pt}%
\definecolor{currentstroke}{rgb}{0.176471,0.192157,0.258824}%
\pgfsetstrokecolor{currentstroke}%
\pgfsetdash{}{0pt}%
\pgfpathmoveto{\pgfqpoint{3.233094in}{2.885854in}}%
\pgfusepath{stroke}%
\end{pgfscope}%
\begin{pgfscope}%
\pgfpathrectangle{\pgfqpoint{0.765000in}{0.660000in}}{\pgfqpoint{4.620000in}{4.620000in}}%
\pgfusepath{clip}%
\pgfsetbuttcap%
\pgfsetmiterjoin%
\definecolor{currentfill}{rgb}{0.176471,0.192157,0.258824}%
\pgfsetfillcolor{currentfill}%
\pgfsetlinewidth{1.003750pt}%
\definecolor{currentstroke}{rgb}{0.176471,0.192157,0.258824}%
\pgfsetstrokecolor{currentstroke}%
\pgfsetdash{}{0pt}%
\pgfsys@defobject{currentmarker}{\pgfqpoint{-0.033333in}{-0.033333in}}{\pgfqpoint{0.033333in}{0.033333in}}{%
\pgfpathmoveto{\pgfqpoint{-0.000000in}{-0.033333in}}%
\pgfpathlineto{\pgfqpoint{0.033333in}{0.033333in}}%
\pgfpathlineto{\pgfqpoint{-0.033333in}{0.033333in}}%
\pgfpathlineto{\pgfqpoint{-0.000000in}{-0.033333in}}%
\pgfpathclose%
\pgfusepath{stroke,fill}%
}%
\begin{pgfscope}%
\pgfsys@transformshift{3.233094in}{2.885854in}%
\pgfsys@useobject{currentmarker}{}%
\end{pgfscope}%
\end{pgfscope}%
\begin{pgfscope}%
\pgfpathrectangle{\pgfqpoint{0.765000in}{0.660000in}}{\pgfqpoint{4.620000in}{4.620000in}}%
\pgfusepath{clip}%
\pgfsetrectcap%
\pgfsetroundjoin%
\pgfsetlinewidth{1.204500pt}%
\definecolor{currentstroke}{rgb}{0.176471,0.192157,0.258824}%
\pgfsetstrokecolor{currentstroke}%
\pgfsetdash{}{0pt}%
\pgfpathmoveto{\pgfqpoint{2.973942in}{3.024328in}}%
\pgfusepath{stroke}%
\end{pgfscope}%
\begin{pgfscope}%
\pgfpathrectangle{\pgfqpoint{0.765000in}{0.660000in}}{\pgfqpoint{4.620000in}{4.620000in}}%
\pgfusepath{clip}%
\pgfsetbuttcap%
\pgfsetmiterjoin%
\definecolor{currentfill}{rgb}{0.176471,0.192157,0.258824}%
\pgfsetfillcolor{currentfill}%
\pgfsetlinewidth{1.003750pt}%
\definecolor{currentstroke}{rgb}{0.176471,0.192157,0.258824}%
\pgfsetstrokecolor{currentstroke}%
\pgfsetdash{}{0pt}%
\pgfsys@defobject{currentmarker}{\pgfqpoint{-0.033333in}{-0.033333in}}{\pgfqpoint{0.033333in}{0.033333in}}{%
\pgfpathmoveto{\pgfqpoint{-0.000000in}{-0.033333in}}%
\pgfpathlineto{\pgfqpoint{0.033333in}{0.033333in}}%
\pgfpathlineto{\pgfqpoint{-0.033333in}{0.033333in}}%
\pgfpathlineto{\pgfqpoint{-0.000000in}{-0.033333in}}%
\pgfpathclose%
\pgfusepath{stroke,fill}%
}%
\begin{pgfscope}%
\pgfsys@transformshift{2.973942in}{3.024328in}%
\pgfsys@useobject{currentmarker}{}%
\end{pgfscope}%
\end{pgfscope}%
\begin{pgfscope}%
\pgfpathrectangle{\pgfqpoint{0.765000in}{0.660000in}}{\pgfqpoint{4.620000in}{4.620000in}}%
\pgfusepath{clip}%
\pgfsetrectcap%
\pgfsetroundjoin%
\pgfsetlinewidth{1.204500pt}%
\definecolor{currentstroke}{rgb}{0.176471,0.192157,0.258824}%
\pgfsetstrokecolor{currentstroke}%
\pgfsetdash{}{0pt}%
\pgfpathmoveto{\pgfqpoint{3.215035in}{3.002857in}}%
\pgfusepath{stroke}%
\end{pgfscope}%
\begin{pgfscope}%
\pgfpathrectangle{\pgfqpoint{0.765000in}{0.660000in}}{\pgfqpoint{4.620000in}{4.620000in}}%
\pgfusepath{clip}%
\pgfsetbuttcap%
\pgfsetmiterjoin%
\definecolor{currentfill}{rgb}{0.176471,0.192157,0.258824}%
\pgfsetfillcolor{currentfill}%
\pgfsetlinewidth{1.003750pt}%
\definecolor{currentstroke}{rgb}{0.176471,0.192157,0.258824}%
\pgfsetstrokecolor{currentstroke}%
\pgfsetdash{}{0pt}%
\pgfsys@defobject{currentmarker}{\pgfqpoint{-0.033333in}{-0.033333in}}{\pgfqpoint{0.033333in}{0.033333in}}{%
\pgfpathmoveto{\pgfqpoint{-0.000000in}{-0.033333in}}%
\pgfpathlineto{\pgfqpoint{0.033333in}{0.033333in}}%
\pgfpathlineto{\pgfqpoint{-0.033333in}{0.033333in}}%
\pgfpathlineto{\pgfqpoint{-0.000000in}{-0.033333in}}%
\pgfpathclose%
\pgfusepath{stroke,fill}%
}%
\begin{pgfscope}%
\pgfsys@transformshift{3.215035in}{3.002857in}%
\pgfsys@useobject{currentmarker}{}%
\end{pgfscope}%
\end{pgfscope}%
\begin{pgfscope}%
\pgfpathrectangle{\pgfqpoint{0.765000in}{0.660000in}}{\pgfqpoint{4.620000in}{4.620000in}}%
\pgfusepath{clip}%
\pgfsetrectcap%
\pgfsetroundjoin%
\pgfsetlinewidth{1.204500pt}%
\definecolor{currentstroke}{rgb}{0.176471,0.192157,0.258824}%
\pgfsetstrokecolor{currentstroke}%
\pgfsetdash{}{0pt}%
\pgfpathmoveto{\pgfqpoint{3.099557in}{2.824254in}}%
\pgfusepath{stroke}%
\end{pgfscope}%
\begin{pgfscope}%
\pgfpathrectangle{\pgfqpoint{0.765000in}{0.660000in}}{\pgfqpoint{4.620000in}{4.620000in}}%
\pgfusepath{clip}%
\pgfsetbuttcap%
\pgfsetmiterjoin%
\definecolor{currentfill}{rgb}{0.176471,0.192157,0.258824}%
\pgfsetfillcolor{currentfill}%
\pgfsetlinewidth{1.003750pt}%
\definecolor{currentstroke}{rgb}{0.176471,0.192157,0.258824}%
\pgfsetstrokecolor{currentstroke}%
\pgfsetdash{}{0pt}%
\pgfsys@defobject{currentmarker}{\pgfqpoint{-0.033333in}{-0.033333in}}{\pgfqpoint{0.033333in}{0.033333in}}{%
\pgfpathmoveto{\pgfqpoint{-0.000000in}{-0.033333in}}%
\pgfpathlineto{\pgfqpoint{0.033333in}{0.033333in}}%
\pgfpathlineto{\pgfqpoint{-0.033333in}{0.033333in}}%
\pgfpathlineto{\pgfqpoint{-0.000000in}{-0.033333in}}%
\pgfpathclose%
\pgfusepath{stroke,fill}%
}%
\begin{pgfscope}%
\pgfsys@transformshift{3.099557in}{2.824254in}%
\pgfsys@useobject{currentmarker}{}%
\end{pgfscope}%
\end{pgfscope}%
\begin{pgfscope}%
\pgfpathrectangle{\pgfqpoint{0.765000in}{0.660000in}}{\pgfqpoint{4.620000in}{4.620000in}}%
\pgfusepath{clip}%
\pgfsetrectcap%
\pgfsetroundjoin%
\pgfsetlinewidth{1.204500pt}%
\definecolor{currentstroke}{rgb}{0.176471,0.192157,0.258824}%
\pgfsetstrokecolor{currentstroke}%
\pgfsetdash{}{0pt}%
\pgfpathmoveto{\pgfqpoint{3.082660in}{3.266765in}}%
\pgfusepath{stroke}%
\end{pgfscope}%
\begin{pgfscope}%
\pgfpathrectangle{\pgfqpoint{0.765000in}{0.660000in}}{\pgfqpoint{4.620000in}{4.620000in}}%
\pgfusepath{clip}%
\pgfsetbuttcap%
\pgfsetmiterjoin%
\definecolor{currentfill}{rgb}{0.176471,0.192157,0.258824}%
\pgfsetfillcolor{currentfill}%
\pgfsetlinewidth{1.003750pt}%
\definecolor{currentstroke}{rgb}{0.176471,0.192157,0.258824}%
\pgfsetstrokecolor{currentstroke}%
\pgfsetdash{}{0pt}%
\pgfsys@defobject{currentmarker}{\pgfqpoint{-0.033333in}{-0.033333in}}{\pgfqpoint{0.033333in}{0.033333in}}{%
\pgfpathmoveto{\pgfqpoint{-0.000000in}{-0.033333in}}%
\pgfpathlineto{\pgfqpoint{0.033333in}{0.033333in}}%
\pgfpathlineto{\pgfqpoint{-0.033333in}{0.033333in}}%
\pgfpathlineto{\pgfqpoint{-0.000000in}{-0.033333in}}%
\pgfpathclose%
\pgfusepath{stroke,fill}%
}%
\begin{pgfscope}%
\pgfsys@transformshift{3.082660in}{3.266765in}%
\pgfsys@useobject{currentmarker}{}%
\end{pgfscope}%
\end{pgfscope}%
\begin{pgfscope}%
\pgfpathrectangle{\pgfqpoint{0.765000in}{0.660000in}}{\pgfqpoint{4.620000in}{4.620000in}}%
\pgfusepath{clip}%
\pgfsetrectcap%
\pgfsetroundjoin%
\pgfsetlinewidth{1.204500pt}%
\definecolor{currentstroke}{rgb}{0.176471,0.192157,0.258824}%
\pgfsetstrokecolor{currentstroke}%
\pgfsetdash{}{0pt}%
\pgfpathmoveto{\pgfqpoint{3.230738in}{3.234090in}}%
\pgfusepath{stroke}%
\end{pgfscope}%
\begin{pgfscope}%
\pgfpathrectangle{\pgfqpoint{0.765000in}{0.660000in}}{\pgfqpoint{4.620000in}{4.620000in}}%
\pgfusepath{clip}%
\pgfsetbuttcap%
\pgfsetmiterjoin%
\definecolor{currentfill}{rgb}{0.176471,0.192157,0.258824}%
\pgfsetfillcolor{currentfill}%
\pgfsetlinewidth{1.003750pt}%
\definecolor{currentstroke}{rgb}{0.176471,0.192157,0.258824}%
\pgfsetstrokecolor{currentstroke}%
\pgfsetdash{}{0pt}%
\pgfsys@defobject{currentmarker}{\pgfqpoint{-0.033333in}{-0.033333in}}{\pgfqpoint{0.033333in}{0.033333in}}{%
\pgfpathmoveto{\pgfqpoint{-0.000000in}{-0.033333in}}%
\pgfpathlineto{\pgfqpoint{0.033333in}{0.033333in}}%
\pgfpathlineto{\pgfqpoint{-0.033333in}{0.033333in}}%
\pgfpathlineto{\pgfqpoint{-0.000000in}{-0.033333in}}%
\pgfpathclose%
\pgfusepath{stroke,fill}%
}%
\begin{pgfscope}%
\pgfsys@transformshift{3.230738in}{3.234090in}%
\pgfsys@useobject{currentmarker}{}%
\end{pgfscope}%
\end{pgfscope}%
\begin{pgfscope}%
\pgfpathrectangle{\pgfqpoint{0.765000in}{0.660000in}}{\pgfqpoint{4.620000in}{4.620000in}}%
\pgfusepath{clip}%
\pgfsetrectcap%
\pgfsetroundjoin%
\pgfsetlinewidth{1.204500pt}%
\definecolor{currentstroke}{rgb}{0.176471,0.192157,0.258824}%
\pgfsetstrokecolor{currentstroke}%
\pgfsetdash{}{0pt}%
\pgfpathmoveto{\pgfqpoint{3.175631in}{2.915262in}}%
\pgfusepath{stroke}%
\end{pgfscope}%
\begin{pgfscope}%
\pgfpathrectangle{\pgfqpoint{0.765000in}{0.660000in}}{\pgfqpoint{4.620000in}{4.620000in}}%
\pgfusepath{clip}%
\pgfsetbuttcap%
\pgfsetmiterjoin%
\definecolor{currentfill}{rgb}{0.176471,0.192157,0.258824}%
\pgfsetfillcolor{currentfill}%
\pgfsetlinewidth{1.003750pt}%
\definecolor{currentstroke}{rgb}{0.176471,0.192157,0.258824}%
\pgfsetstrokecolor{currentstroke}%
\pgfsetdash{}{0pt}%
\pgfsys@defobject{currentmarker}{\pgfqpoint{-0.033333in}{-0.033333in}}{\pgfqpoint{0.033333in}{0.033333in}}{%
\pgfpathmoveto{\pgfqpoint{-0.000000in}{-0.033333in}}%
\pgfpathlineto{\pgfqpoint{0.033333in}{0.033333in}}%
\pgfpathlineto{\pgfqpoint{-0.033333in}{0.033333in}}%
\pgfpathlineto{\pgfqpoint{-0.000000in}{-0.033333in}}%
\pgfpathclose%
\pgfusepath{stroke,fill}%
}%
\begin{pgfscope}%
\pgfsys@transformshift{3.175631in}{2.915262in}%
\pgfsys@useobject{currentmarker}{}%
\end{pgfscope}%
\end{pgfscope}%
\begin{pgfscope}%
\pgfpathrectangle{\pgfqpoint{0.765000in}{0.660000in}}{\pgfqpoint{4.620000in}{4.620000in}}%
\pgfusepath{clip}%
\pgfsetrectcap%
\pgfsetroundjoin%
\pgfsetlinewidth{1.204500pt}%
\definecolor{currentstroke}{rgb}{0.176471,0.192157,0.258824}%
\pgfsetstrokecolor{currentstroke}%
\pgfsetdash{}{0pt}%
\pgfpathmoveto{\pgfqpoint{3.299840in}{3.038101in}}%
\pgfusepath{stroke}%
\end{pgfscope}%
\begin{pgfscope}%
\pgfpathrectangle{\pgfqpoint{0.765000in}{0.660000in}}{\pgfqpoint{4.620000in}{4.620000in}}%
\pgfusepath{clip}%
\pgfsetbuttcap%
\pgfsetmiterjoin%
\definecolor{currentfill}{rgb}{0.176471,0.192157,0.258824}%
\pgfsetfillcolor{currentfill}%
\pgfsetlinewidth{1.003750pt}%
\definecolor{currentstroke}{rgb}{0.176471,0.192157,0.258824}%
\pgfsetstrokecolor{currentstroke}%
\pgfsetdash{}{0pt}%
\pgfsys@defobject{currentmarker}{\pgfqpoint{-0.033333in}{-0.033333in}}{\pgfqpoint{0.033333in}{0.033333in}}{%
\pgfpathmoveto{\pgfqpoint{-0.000000in}{-0.033333in}}%
\pgfpathlineto{\pgfqpoint{0.033333in}{0.033333in}}%
\pgfpathlineto{\pgfqpoint{-0.033333in}{0.033333in}}%
\pgfpathlineto{\pgfqpoint{-0.000000in}{-0.033333in}}%
\pgfpathclose%
\pgfusepath{stroke,fill}%
}%
\begin{pgfscope}%
\pgfsys@transformshift{3.299840in}{3.038101in}%
\pgfsys@useobject{currentmarker}{}%
\end{pgfscope}%
\end{pgfscope}%
\begin{pgfscope}%
\pgfpathrectangle{\pgfqpoint{0.765000in}{0.660000in}}{\pgfqpoint{4.620000in}{4.620000in}}%
\pgfusepath{clip}%
\pgfsetrectcap%
\pgfsetroundjoin%
\pgfsetlinewidth{1.204500pt}%
\definecolor{currentstroke}{rgb}{0.176471,0.192157,0.258824}%
\pgfsetstrokecolor{currentstroke}%
\pgfsetdash{}{0pt}%
\pgfpathmoveto{\pgfqpoint{3.012296in}{2.919010in}}%
\pgfusepath{stroke}%
\end{pgfscope}%
\begin{pgfscope}%
\pgfpathrectangle{\pgfqpoint{0.765000in}{0.660000in}}{\pgfqpoint{4.620000in}{4.620000in}}%
\pgfusepath{clip}%
\pgfsetbuttcap%
\pgfsetmiterjoin%
\definecolor{currentfill}{rgb}{0.176471,0.192157,0.258824}%
\pgfsetfillcolor{currentfill}%
\pgfsetlinewidth{1.003750pt}%
\definecolor{currentstroke}{rgb}{0.176471,0.192157,0.258824}%
\pgfsetstrokecolor{currentstroke}%
\pgfsetdash{}{0pt}%
\pgfsys@defobject{currentmarker}{\pgfqpoint{-0.033333in}{-0.033333in}}{\pgfqpoint{0.033333in}{0.033333in}}{%
\pgfpathmoveto{\pgfqpoint{-0.000000in}{-0.033333in}}%
\pgfpathlineto{\pgfqpoint{0.033333in}{0.033333in}}%
\pgfpathlineto{\pgfqpoint{-0.033333in}{0.033333in}}%
\pgfpathlineto{\pgfqpoint{-0.000000in}{-0.033333in}}%
\pgfpathclose%
\pgfusepath{stroke,fill}%
}%
\begin{pgfscope}%
\pgfsys@transformshift{3.012296in}{2.919010in}%
\pgfsys@useobject{currentmarker}{}%
\end{pgfscope}%
\end{pgfscope}%
\begin{pgfscope}%
\pgfpathrectangle{\pgfqpoint{0.765000in}{0.660000in}}{\pgfqpoint{4.620000in}{4.620000in}}%
\pgfusepath{clip}%
\pgfsetrectcap%
\pgfsetroundjoin%
\pgfsetlinewidth{1.204500pt}%
\definecolor{currentstroke}{rgb}{0.176471,0.192157,0.258824}%
\pgfsetstrokecolor{currentstroke}%
\pgfsetdash{}{0pt}%
\pgfpathmoveto{\pgfqpoint{2.992656in}{3.169763in}}%
\pgfusepath{stroke}%
\end{pgfscope}%
\begin{pgfscope}%
\pgfpathrectangle{\pgfqpoint{0.765000in}{0.660000in}}{\pgfqpoint{4.620000in}{4.620000in}}%
\pgfusepath{clip}%
\pgfsetbuttcap%
\pgfsetmiterjoin%
\definecolor{currentfill}{rgb}{0.176471,0.192157,0.258824}%
\pgfsetfillcolor{currentfill}%
\pgfsetlinewidth{1.003750pt}%
\definecolor{currentstroke}{rgb}{0.176471,0.192157,0.258824}%
\pgfsetstrokecolor{currentstroke}%
\pgfsetdash{}{0pt}%
\pgfsys@defobject{currentmarker}{\pgfqpoint{-0.033333in}{-0.033333in}}{\pgfqpoint{0.033333in}{0.033333in}}{%
\pgfpathmoveto{\pgfqpoint{-0.000000in}{-0.033333in}}%
\pgfpathlineto{\pgfqpoint{0.033333in}{0.033333in}}%
\pgfpathlineto{\pgfqpoint{-0.033333in}{0.033333in}}%
\pgfpathlineto{\pgfqpoint{-0.000000in}{-0.033333in}}%
\pgfpathclose%
\pgfusepath{stroke,fill}%
}%
\begin{pgfscope}%
\pgfsys@transformshift{2.992656in}{3.169763in}%
\pgfsys@useobject{currentmarker}{}%
\end{pgfscope}%
\end{pgfscope}%
\begin{pgfscope}%
\pgfpathrectangle{\pgfqpoint{0.765000in}{0.660000in}}{\pgfqpoint{4.620000in}{4.620000in}}%
\pgfusepath{clip}%
\pgfsetrectcap%
\pgfsetroundjoin%
\pgfsetlinewidth{1.204500pt}%
\definecolor{currentstroke}{rgb}{0.176471,0.192157,0.258824}%
\pgfsetstrokecolor{currentstroke}%
\pgfsetdash{}{0pt}%
\pgfpathmoveto{\pgfqpoint{3.118777in}{2.854793in}}%
\pgfusepath{stroke}%
\end{pgfscope}%
\begin{pgfscope}%
\pgfpathrectangle{\pgfqpoint{0.765000in}{0.660000in}}{\pgfqpoint{4.620000in}{4.620000in}}%
\pgfusepath{clip}%
\pgfsetbuttcap%
\pgfsetmiterjoin%
\definecolor{currentfill}{rgb}{0.176471,0.192157,0.258824}%
\pgfsetfillcolor{currentfill}%
\pgfsetlinewidth{1.003750pt}%
\definecolor{currentstroke}{rgb}{0.176471,0.192157,0.258824}%
\pgfsetstrokecolor{currentstroke}%
\pgfsetdash{}{0pt}%
\pgfsys@defobject{currentmarker}{\pgfqpoint{-0.033333in}{-0.033333in}}{\pgfqpoint{0.033333in}{0.033333in}}{%
\pgfpathmoveto{\pgfqpoint{-0.000000in}{-0.033333in}}%
\pgfpathlineto{\pgfqpoint{0.033333in}{0.033333in}}%
\pgfpathlineto{\pgfqpoint{-0.033333in}{0.033333in}}%
\pgfpathlineto{\pgfqpoint{-0.000000in}{-0.033333in}}%
\pgfpathclose%
\pgfusepath{stroke,fill}%
}%
\begin{pgfscope}%
\pgfsys@transformshift{3.118777in}{2.854793in}%
\pgfsys@useobject{currentmarker}{}%
\end{pgfscope}%
\end{pgfscope}%
\begin{pgfscope}%
\pgfpathrectangle{\pgfqpoint{0.765000in}{0.660000in}}{\pgfqpoint{4.620000in}{4.620000in}}%
\pgfusepath{clip}%
\pgfsetrectcap%
\pgfsetroundjoin%
\pgfsetlinewidth{1.204500pt}%
\definecolor{currentstroke}{rgb}{0.176471,0.192157,0.258824}%
\pgfsetstrokecolor{currentstroke}%
\pgfsetdash{}{0pt}%
\pgfpathmoveto{\pgfqpoint{3.123443in}{3.007043in}}%
\pgfusepath{stroke}%
\end{pgfscope}%
\begin{pgfscope}%
\pgfpathrectangle{\pgfqpoint{0.765000in}{0.660000in}}{\pgfqpoint{4.620000in}{4.620000in}}%
\pgfusepath{clip}%
\pgfsetbuttcap%
\pgfsetmiterjoin%
\definecolor{currentfill}{rgb}{0.176471,0.192157,0.258824}%
\pgfsetfillcolor{currentfill}%
\pgfsetlinewidth{1.003750pt}%
\definecolor{currentstroke}{rgb}{0.176471,0.192157,0.258824}%
\pgfsetstrokecolor{currentstroke}%
\pgfsetdash{}{0pt}%
\pgfsys@defobject{currentmarker}{\pgfqpoint{-0.033333in}{-0.033333in}}{\pgfqpoint{0.033333in}{0.033333in}}{%
\pgfpathmoveto{\pgfqpoint{-0.000000in}{-0.033333in}}%
\pgfpathlineto{\pgfqpoint{0.033333in}{0.033333in}}%
\pgfpathlineto{\pgfqpoint{-0.033333in}{0.033333in}}%
\pgfpathlineto{\pgfqpoint{-0.000000in}{-0.033333in}}%
\pgfpathclose%
\pgfusepath{stroke,fill}%
}%
\begin{pgfscope}%
\pgfsys@transformshift{3.123443in}{3.007043in}%
\pgfsys@useobject{currentmarker}{}%
\end{pgfscope}%
\end{pgfscope}%
\begin{pgfscope}%
\pgfpathrectangle{\pgfqpoint{0.765000in}{0.660000in}}{\pgfqpoint{4.620000in}{4.620000in}}%
\pgfusepath{clip}%
\pgfsetrectcap%
\pgfsetroundjoin%
\pgfsetlinewidth{1.204500pt}%
\definecolor{currentstroke}{rgb}{0.176471,0.192157,0.258824}%
\pgfsetstrokecolor{currentstroke}%
\pgfsetdash{}{0pt}%
\pgfpathmoveto{\pgfqpoint{2.941470in}{3.131352in}}%
\pgfusepath{stroke}%
\end{pgfscope}%
\begin{pgfscope}%
\pgfpathrectangle{\pgfqpoint{0.765000in}{0.660000in}}{\pgfqpoint{4.620000in}{4.620000in}}%
\pgfusepath{clip}%
\pgfsetbuttcap%
\pgfsetmiterjoin%
\definecolor{currentfill}{rgb}{0.176471,0.192157,0.258824}%
\pgfsetfillcolor{currentfill}%
\pgfsetlinewidth{1.003750pt}%
\definecolor{currentstroke}{rgb}{0.176471,0.192157,0.258824}%
\pgfsetstrokecolor{currentstroke}%
\pgfsetdash{}{0pt}%
\pgfsys@defobject{currentmarker}{\pgfqpoint{-0.033333in}{-0.033333in}}{\pgfqpoint{0.033333in}{0.033333in}}{%
\pgfpathmoveto{\pgfqpoint{-0.000000in}{-0.033333in}}%
\pgfpathlineto{\pgfqpoint{0.033333in}{0.033333in}}%
\pgfpathlineto{\pgfqpoint{-0.033333in}{0.033333in}}%
\pgfpathlineto{\pgfqpoint{-0.000000in}{-0.033333in}}%
\pgfpathclose%
\pgfusepath{stroke,fill}%
}%
\begin{pgfscope}%
\pgfsys@transformshift{2.941470in}{3.131352in}%
\pgfsys@useobject{currentmarker}{}%
\end{pgfscope}%
\end{pgfscope}%
\begin{pgfscope}%
\pgfpathrectangle{\pgfqpoint{0.765000in}{0.660000in}}{\pgfqpoint{4.620000in}{4.620000in}}%
\pgfusepath{clip}%
\pgfsetrectcap%
\pgfsetroundjoin%
\pgfsetlinewidth{1.204500pt}%
\definecolor{currentstroke}{rgb}{0.176471,0.192157,0.258824}%
\pgfsetstrokecolor{currentstroke}%
\pgfsetdash{}{0pt}%
\pgfpathmoveto{\pgfqpoint{3.056021in}{3.132614in}}%
\pgfusepath{stroke}%
\end{pgfscope}%
\begin{pgfscope}%
\pgfpathrectangle{\pgfqpoint{0.765000in}{0.660000in}}{\pgfqpoint{4.620000in}{4.620000in}}%
\pgfusepath{clip}%
\pgfsetbuttcap%
\pgfsetmiterjoin%
\definecolor{currentfill}{rgb}{0.176471,0.192157,0.258824}%
\pgfsetfillcolor{currentfill}%
\pgfsetlinewidth{1.003750pt}%
\definecolor{currentstroke}{rgb}{0.176471,0.192157,0.258824}%
\pgfsetstrokecolor{currentstroke}%
\pgfsetdash{}{0pt}%
\pgfsys@defobject{currentmarker}{\pgfqpoint{-0.033333in}{-0.033333in}}{\pgfqpoint{0.033333in}{0.033333in}}{%
\pgfpathmoveto{\pgfqpoint{-0.000000in}{-0.033333in}}%
\pgfpathlineto{\pgfqpoint{0.033333in}{0.033333in}}%
\pgfpathlineto{\pgfqpoint{-0.033333in}{0.033333in}}%
\pgfpathlineto{\pgfqpoint{-0.000000in}{-0.033333in}}%
\pgfpathclose%
\pgfusepath{stroke,fill}%
}%
\begin{pgfscope}%
\pgfsys@transformshift{3.056021in}{3.132614in}%
\pgfsys@useobject{currentmarker}{}%
\end{pgfscope}%
\end{pgfscope}%
\begin{pgfscope}%
\pgfpathrectangle{\pgfqpoint{0.765000in}{0.660000in}}{\pgfqpoint{4.620000in}{4.620000in}}%
\pgfusepath{clip}%
\pgfsetrectcap%
\pgfsetroundjoin%
\pgfsetlinewidth{1.204500pt}%
\definecolor{currentstroke}{rgb}{0.176471,0.192157,0.258824}%
\pgfsetstrokecolor{currentstroke}%
\pgfsetdash{}{0pt}%
\pgfpathmoveto{\pgfqpoint{3.207459in}{3.040819in}}%
\pgfusepath{stroke}%
\end{pgfscope}%
\begin{pgfscope}%
\pgfpathrectangle{\pgfqpoint{0.765000in}{0.660000in}}{\pgfqpoint{4.620000in}{4.620000in}}%
\pgfusepath{clip}%
\pgfsetbuttcap%
\pgfsetmiterjoin%
\definecolor{currentfill}{rgb}{0.176471,0.192157,0.258824}%
\pgfsetfillcolor{currentfill}%
\pgfsetlinewidth{1.003750pt}%
\definecolor{currentstroke}{rgb}{0.176471,0.192157,0.258824}%
\pgfsetstrokecolor{currentstroke}%
\pgfsetdash{}{0pt}%
\pgfsys@defobject{currentmarker}{\pgfqpoint{-0.033333in}{-0.033333in}}{\pgfqpoint{0.033333in}{0.033333in}}{%
\pgfpathmoveto{\pgfqpoint{-0.000000in}{-0.033333in}}%
\pgfpathlineto{\pgfqpoint{0.033333in}{0.033333in}}%
\pgfpathlineto{\pgfqpoint{-0.033333in}{0.033333in}}%
\pgfpathlineto{\pgfqpoint{-0.000000in}{-0.033333in}}%
\pgfpathclose%
\pgfusepath{stroke,fill}%
}%
\begin{pgfscope}%
\pgfsys@transformshift{3.207459in}{3.040819in}%
\pgfsys@useobject{currentmarker}{}%
\end{pgfscope}%
\end{pgfscope}%
\begin{pgfscope}%
\pgfpathrectangle{\pgfqpoint{0.765000in}{0.660000in}}{\pgfqpoint{4.620000in}{4.620000in}}%
\pgfusepath{clip}%
\pgfsetrectcap%
\pgfsetroundjoin%
\pgfsetlinewidth{1.204500pt}%
\definecolor{currentstroke}{rgb}{0.176471,0.192157,0.258824}%
\pgfsetstrokecolor{currentstroke}%
\pgfsetdash{}{0pt}%
\pgfpathmoveto{\pgfqpoint{3.196394in}{3.218235in}}%
\pgfusepath{stroke}%
\end{pgfscope}%
\begin{pgfscope}%
\pgfpathrectangle{\pgfqpoint{0.765000in}{0.660000in}}{\pgfqpoint{4.620000in}{4.620000in}}%
\pgfusepath{clip}%
\pgfsetbuttcap%
\pgfsetmiterjoin%
\definecolor{currentfill}{rgb}{0.176471,0.192157,0.258824}%
\pgfsetfillcolor{currentfill}%
\pgfsetlinewidth{1.003750pt}%
\definecolor{currentstroke}{rgb}{0.176471,0.192157,0.258824}%
\pgfsetstrokecolor{currentstroke}%
\pgfsetdash{}{0pt}%
\pgfsys@defobject{currentmarker}{\pgfqpoint{-0.033333in}{-0.033333in}}{\pgfqpoint{0.033333in}{0.033333in}}{%
\pgfpathmoveto{\pgfqpoint{-0.000000in}{-0.033333in}}%
\pgfpathlineto{\pgfqpoint{0.033333in}{0.033333in}}%
\pgfpathlineto{\pgfqpoint{-0.033333in}{0.033333in}}%
\pgfpathlineto{\pgfqpoint{-0.000000in}{-0.033333in}}%
\pgfpathclose%
\pgfusepath{stroke,fill}%
}%
\begin{pgfscope}%
\pgfsys@transformshift{3.196394in}{3.218235in}%
\pgfsys@useobject{currentmarker}{}%
\end{pgfscope}%
\end{pgfscope}%
\begin{pgfscope}%
\pgfpathrectangle{\pgfqpoint{0.765000in}{0.660000in}}{\pgfqpoint{4.620000in}{4.620000in}}%
\pgfusepath{clip}%
\pgfsetrectcap%
\pgfsetroundjoin%
\pgfsetlinewidth{1.204500pt}%
\definecolor{currentstroke}{rgb}{0.176471,0.192157,0.258824}%
\pgfsetstrokecolor{currentstroke}%
\pgfsetdash{}{0pt}%
\pgfpathmoveto{\pgfqpoint{3.144253in}{2.877742in}}%
\pgfusepath{stroke}%
\end{pgfscope}%
\begin{pgfscope}%
\pgfpathrectangle{\pgfqpoint{0.765000in}{0.660000in}}{\pgfqpoint{4.620000in}{4.620000in}}%
\pgfusepath{clip}%
\pgfsetbuttcap%
\pgfsetmiterjoin%
\definecolor{currentfill}{rgb}{0.176471,0.192157,0.258824}%
\pgfsetfillcolor{currentfill}%
\pgfsetlinewidth{1.003750pt}%
\definecolor{currentstroke}{rgb}{0.176471,0.192157,0.258824}%
\pgfsetstrokecolor{currentstroke}%
\pgfsetdash{}{0pt}%
\pgfsys@defobject{currentmarker}{\pgfqpoint{-0.033333in}{-0.033333in}}{\pgfqpoint{0.033333in}{0.033333in}}{%
\pgfpathmoveto{\pgfqpoint{-0.000000in}{-0.033333in}}%
\pgfpathlineto{\pgfqpoint{0.033333in}{0.033333in}}%
\pgfpathlineto{\pgfqpoint{-0.033333in}{0.033333in}}%
\pgfpathlineto{\pgfqpoint{-0.000000in}{-0.033333in}}%
\pgfpathclose%
\pgfusepath{stroke,fill}%
}%
\begin{pgfscope}%
\pgfsys@transformshift{3.144253in}{2.877742in}%
\pgfsys@useobject{currentmarker}{}%
\end{pgfscope}%
\end{pgfscope}%
\begin{pgfscope}%
\pgfpathrectangle{\pgfqpoint{0.765000in}{0.660000in}}{\pgfqpoint{4.620000in}{4.620000in}}%
\pgfusepath{clip}%
\pgfsetrectcap%
\pgfsetroundjoin%
\pgfsetlinewidth{1.204500pt}%
\definecolor{currentstroke}{rgb}{0.176471,0.192157,0.258824}%
\pgfsetstrokecolor{currentstroke}%
\pgfsetdash{}{0pt}%
\pgfpathmoveto{\pgfqpoint{3.138672in}{2.878014in}}%
\pgfusepath{stroke}%
\end{pgfscope}%
\begin{pgfscope}%
\pgfpathrectangle{\pgfqpoint{0.765000in}{0.660000in}}{\pgfqpoint{4.620000in}{4.620000in}}%
\pgfusepath{clip}%
\pgfsetbuttcap%
\pgfsetmiterjoin%
\definecolor{currentfill}{rgb}{0.176471,0.192157,0.258824}%
\pgfsetfillcolor{currentfill}%
\pgfsetlinewidth{1.003750pt}%
\definecolor{currentstroke}{rgb}{0.176471,0.192157,0.258824}%
\pgfsetstrokecolor{currentstroke}%
\pgfsetdash{}{0pt}%
\pgfsys@defobject{currentmarker}{\pgfqpoint{-0.033333in}{-0.033333in}}{\pgfqpoint{0.033333in}{0.033333in}}{%
\pgfpathmoveto{\pgfqpoint{-0.000000in}{-0.033333in}}%
\pgfpathlineto{\pgfqpoint{0.033333in}{0.033333in}}%
\pgfpathlineto{\pgfqpoint{-0.033333in}{0.033333in}}%
\pgfpathlineto{\pgfqpoint{-0.000000in}{-0.033333in}}%
\pgfpathclose%
\pgfusepath{stroke,fill}%
}%
\begin{pgfscope}%
\pgfsys@transformshift{3.138672in}{2.878014in}%
\pgfsys@useobject{currentmarker}{}%
\end{pgfscope}%
\end{pgfscope}%
\begin{pgfscope}%
\pgfpathrectangle{\pgfqpoint{0.765000in}{0.660000in}}{\pgfqpoint{4.620000in}{4.620000in}}%
\pgfusepath{clip}%
\pgfsetrectcap%
\pgfsetroundjoin%
\pgfsetlinewidth{1.204500pt}%
\definecolor{currentstroke}{rgb}{0.176471,0.192157,0.258824}%
\pgfsetstrokecolor{currentstroke}%
\pgfsetdash{}{0pt}%
\pgfpathmoveto{\pgfqpoint{3.194018in}{3.291834in}}%
\pgfusepath{stroke}%
\end{pgfscope}%
\begin{pgfscope}%
\pgfpathrectangle{\pgfqpoint{0.765000in}{0.660000in}}{\pgfqpoint{4.620000in}{4.620000in}}%
\pgfusepath{clip}%
\pgfsetbuttcap%
\pgfsetmiterjoin%
\definecolor{currentfill}{rgb}{0.176471,0.192157,0.258824}%
\pgfsetfillcolor{currentfill}%
\pgfsetlinewidth{1.003750pt}%
\definecolor{currentstroke}{rgb}{0.176471,0.192157,0.258824}%
\pgfsetstrokecolor{currentstroke}%
\pgfsetdash{}{0pt}%
\pgfsys@defobject{currentmarker}{\pgfqpoint{-0.033333in}{-0.033333in}}{\pgfqpoint{0.033333in}{0.033333in}}{%
\pgfpathmoveto{\pgfqpoint{-0.000000in}{-0.033333in}}%
\pgfpathlineto{\pgfqpoint{0.033333in}{0.033333in}}%
\pgfpathlineto{\pgfqpoint{-0.033333in}{0.033333in}}%
\pgfpathlineto{\pgfqpoint{-0.000000in}{-0.033333in}}%
\pgfpathclose%
\pgfusepath{stroke,fill}%
}%
\begin{pgfscope}%
\pgfsys@transformshift{3.194018in}{3.291834in}%
\pgfsys@useobject{currentmarker}{}%
\end{pgfscope}%
\end{pgfscope}%
\begin{pgfscope}%
\pgfpathrectangle{\pgfqpoint{0.765000in}{0.660000in}}{\pgfqpoint{4.620000in}{4.620000in}}%
\pgfusepath{clip}%
\pgfsetrectcap%
\pgfsetroundjoin%
\pgfsetlinewidth{1.204500pt}%
\definecolor{currentstroke}{rgb}{0.176471,0.192157,0.258824}%
\pgfsetstrokecolor{currentstroke}%
\pgfsetdash{}{0pt}%
\pgfpathmoveto{\pgfqpoint{3.048272in}{2.985681in}}%
\pgfusepath{stroke}%
\end{pgfscope}%
\begin{pgfscope}%
\pgfpathrectangle{\pgfqpoint{0.765000in}{0.660000in}}{\pgfqpoint{4.620000in}{4.620000in}}%
\pgfusepath{clip}%
\pgfsetbuttcap%
\pgfsetmiterjoin%
\definecolor{currentfill}{rgb}{0.176471,0.192157,0.258824}%
\pgfsetfillcolor{currentfill}%
\pgfsetlinewidth{1.003750pt}%
\definecolor{currentstroke}{rgb}{0.176471,0.192157,0.258824}%
\pgfsetstrokecolor{currentstroke}%
\pgfsetdash{}{0pt}%
\pgfsys@defobject{currentmarker}{\pgfqpoint{-0.033333in}{-0.033333in}}{\pgfqpoint{0.033333in}{0.033333in}}{%
\pgfpathmoveto{\pgfqpoint{-0.000000in}{-0.033333in}}%
\pgfpathlineto{\pgfqpoint{0.033333in}{0.033333in}}%
\pgfpathlineto{\pgfqpoint{-0.033333in}{0.033333in}}%
\pgfpathlineto{\pgfqpoint{-0.000000in}{-0.033333in}}%
\pgfpathclose%
\pgfusepath{stroke,fill}%
}%
\begin{pgfscope}%
\pgfsys@transformshift{3.048272in}{2.985681in}%
\pgfsys@useobject{currentmarker}{}%
\end{pgfscope}%
\end{pgfscope}%
\begin{pgfscope}%
\pgfpathrectangle{\pgfqpoint{0.765000in}{0.660000in}}{\pgfqpoint{4.620000in}{4.620000in}}%
\pgfusepath{clip}%
\pgfsetrectcap%
\pgfsetroundjoin%
\pgfsetlinewidth{1.204500pt}%
\definecolor{currentstroke}{rgb}{0.176471,0.192157,0.258824}%
\pgfsetstrokecolor{currentstroke}%
\pgfsetdash{}{0pt}%
\pgfpathmoveto{\pgfqpoint{3.246341in}{3.201645in}}%
\pgfusepath{stroke}%
\end{pgfscope}%
\begin{pgfscope}%
\pgfpathrectangle{\pgfqpoint{0.765000in}{0.660000in}}{\pgfqpoint{4.620000in}{4.620000in}}%
\pgfusepath{clip}%
\pgfsetbuttcap%
\pgfsetmiterjoin%
\definecolor{currentfill}{rgb}{0.176471,0.192157,0.258824}%
\pgfsetfillcolor{currentfill}%
\pgfsetlinewidth{1.003750pt}%
\definecolor{currentstroke}{rgb}{0.176471,0.192157,0.258824}%
\pgfsetstrokecolor{currentstroke}%
\pgfsetdash{}{0pt}%
\pgfsys@defobject{currentmarker}{\pgfqpoint{-0.033333in}{-0.033333in}}{\pgfqpoint{0.033333in}{0.033333in}}{%
\pgfpathmoveto{\pgfqpoint{-0.000000in}{-0.033333in}}%
\pgfpathlineto{\pgfqpoint{0.033333in}{0.033333in}}%
\pgfpathlineto{\pgfqpoint{-0.033333in}{0.033333in}}%
\pgfpathlineto{\pgfqpoint{-0.000000in}{-0.033333in}}%
\pgfpathclose%
\pgfusepath{stroke,fill}%
}%
\begin{pgfscope}%
\pgfsys@transformshift{3.246341in}{3.201645in}%
\pgfsys@useobject{currentmarker}{}%
\end{pgfscope}%
\end{pgfscope}%
\begin{pgfscope}%
\pgfpathrectangle{\pgfqpoint{0.765000in}{0.660000in}}{\pgfqpoint{4.620000in}{4.620000in}}%
\pgfusepath{clip}%
\pgfsetrectcap%
\pgfsetroundjoin%
\pgfsetlinewidth{1.204500pt}%
\definecolor{currentstroke}{rgb}{0.176471,0.192157,0.258824}%
\pgfsetstrokecolor{currentstroke}%
\pgfsetdash{}{0pt}%
\pgfpathmoveto{\pgfqpoint{3.216446in}{2.958357in}}%
\pgfusepath{stroke}%
\end{pgfscope}%
\begin{pgfscope}%
\pgfpathrectangle{\pgfqpoint{0.765000in}{0.660000in}}{\pgfqpoint{4.620000in}{4.620000in}}%
\pgfusepath{clip}%
\pgfsetbuttcap%
\pgfsetmiterjoin%
\definecolor{currentfill}{rgb}{0.176471,0.192157,0.258824}%
\pgfsetfillcolor{currentfill}%
\pgfsetlinewidth{1.003750pt}%
\definecolor{currentstroke}{rgb}{0.176471,0.192157,0.258824}%
\pgfsetstrokecolor{currentstroke}%
\pgfsetdash{}{0pt}%
\pgfsys@defobject{currentmarker}{\pgfqpoint{-0.033333in}{-0.033333in}}{\pgfqpoint{0.033333in}{0.033333in}}{%
\pgfpathmoveto{\pgfqpoint{-0.000000in}{-0.033333in}}%
\pgfpathlineto{\pgfqpoint{0.033333in}{0.033333in}}%
\pgfpathlineto{\pgfqpoint{-0.033333in}{0.033333in}}%
\pgfpathlineto{\pgfqpoint{-0.000000in}{-0.033333in}}%
\pgfpathclose%
\pgfusepath{stroke,fill}%
}%
\begin{pgfscope}%
\pgfsys@transformshift{3.216446in}{2.958357in}%
\pgfsys@useobject{currentmarker}{}%
\end{pgfscope}%
\end{pgfscope}%
\begin{pgfscope}%
\pgfpathrectangle{\pgfqpoint{0.765000in}{0.660000in}}{\pgfqpoint{4.620000in}{4.620000in}}%
\pgfusepath{clip}%
\pgfsetrectcap%
\pgfsetroundjoin%
\pgfsetlinewidth{1.204500pt}%
\definecolor{currentstroke}{rgb}{0.176471,0.192157,0.258824}%
\pgfsetstrokecolor{currentstroke}%
\pgfsetdash{}{0pt}%
\pgfpathmoveto{\pgfqpoint{3.016890in}{3.164394in}}%
\pgfusepath{stroke}%
\end{pgfscope}%
\begin{pgfscope}%
\pgfpathrectangle{\pgfqpoint{0.765000in}{0.660000in}}{\pgfqpoint{4.620000in}{4.620000in}}%
\pgfusepath{clip}%
\pgfsetbuttcap%
\pgfsetmiterjoin%
\definecolor{currentfill}{rgb}{0.176471,0.192157,0.258824}%
\pgfsetfillcolor{currentfill}%
\pgfsetlinewidth{1.003750pt}%
\definecolor{currentstroke}{rgb}{0.176471,0.192157,0.258824}%
\pgfsetstrokecolor{currentstroke}%
\pgfsetdash{}{0pt}%
\pgfsys@defobject{currentmarker}{\pgfqpoint{-0.033333in}{-0.033333in}}{\pgfqpoint{0.033333in}{0.033333in}}{%
\pgfpathmoveto{\pgfqpoint{-0.000000in}{-0.033333in}}%
\pgfpathlineto{\pgfqpoint{0.033333in}{0.033333in}}%
\pgfpathlineto{\pgfqpoint{-0.033333in}{0.033333in}}%
\pgfpathlineto{\pgfqpoint{-0.000000in}{-0.033333in}}%
\pgfpathclose%
\pgfusepath{stroke,fill}%
}%
\begin{pgfscope}%
\pgfsys@transformshift{3.016890in}{3.164394in}%
\pgfsys@useobject{currentmarker}{}%
\end{pgfscope}%
\end{pgfscope}%
\begin{pgfscope}%
\pgfpathrectangle{\pgfqpoint{0.765000in}{0.660000in}}{\pgfqpoint{4.620000in}{4.620000in}}%
\pgfusepath{clip}%
\pgfsetrectcap%
\pgfsetroundjoin%
\pgfsetlinewidth{1.204500pt}%
\definecolor{currentstroke}{rgb}{0.176471,0.192157,0.258824}%
\pgfsetstrokecolor{currentstroke}%
\pgfsetdash{}{0pt}%
\pgfpathmoveto{\pgfqpoint{3.042281in}{2.865179in}}%
\pgfusepath{stroke}%
\end{pgfscope}%
\begin{pgfscope}%
\pgfpathrectangle{\pgfqpoint{0.765000in}{0.660000in}}{\pgfqpoint{4.620000in}{4.620000in}}%
\pgfusepath{clip}%
\pgfsetbuttcap%
\pgfsetmiterjoin%
\definecolor{currentfill}{rgb}{0.176471,0.192157,0.258824}%
\pgfsetfillcolor{currentfill}%
\pgfsetlinewidth{1.003750pt}%
\definecolor{currentstroke}{rgb}{0.176471,0.192157,0.258824}%
\pgfsetstrokecolor{currentstroke}%
\pgfsetdash{}{0pt}%
\pgfsys@defobject{currentmarker}{\pgfqpoint{-0.033333in}{-0.033333in}}{\pgfqpoint{0.033333in}{0.033333in}}{%
\pgfpathmoveto{\pgfqpoint{-0.000000in}{-0.033333in}}%
\pgfpathlineto{\pgfqpoint{0.033333in}{0.033333in}}%
\pgfpathlineto{\pgfqpoint{-0.033333in}{0.033333in}}%
\pgfpathlineto{\pgfqpoint{-0.000000in}{-0.033333in}}%
\pgfpathclose%
\pgfusepath{stroke,fill}%
}%
\begin{pgfscope}%
\pgfsys@transformshift{3.042281in}{2.865179in}%
\pgfsys@useobject{currentmarker}{}%
\end{pgfscope}%
\end{pgfscope}%
\begin{pgfscope}%
\pgfpathrectangle{\pgfqpoint{0.765000in}{0.660000in}}{\pgfqpoint{4.620000in}{4.620000in}}%
\pgfusepath{clip}%
\pgfsetrectcap%
\pgfsetroundjoin%
\pgfsetlinewidth{1.204500pt}%
\definecolor{currentstroke}{rgb}{0.176471,0.192157,0.258824}%
\pgfsetstrokecolor{currentstroke}%
\pgfsetdash{}{0pt}%
\pgfpathmoveto{\pgfqpoint{2.897149in}{3.183561in}}%
\pgfusepath{stroke}%
\end{pgfscope}%
\begin{pgfscope}%
\pgfpathrectangle{\pgfqpoint{0.765000in}{0.660000in}}{\pgfqpoint{4.620000in}{4.620000in}}%
\pgfusepath{clip}%
\pgfsetbuttcap%
\pgfsetmiterjoin%
\definecolor{currentfill}{rgb}{0.176471,0.192157,0.258824}%
\pgfsetfillcolor{currentfill}%
\pgfsetlinewidth{1.003750pt}%
\definecolor{currentstroke}{rgb}{0.176471,0.192157,0.258824}%
\pgfsetstrokecolor{currentstroke}%
\pgfsetdash{}{0pt}%
\pgfsys@defobject{currentmarker}{\pgfqpoint{-0.033333in}{-0.033333in}}{\pgfqpoint{0.033333in}{0.033333in}}{%
\pgfpathmoveto{\pgfqpoint{-0.000000in}{-0.033333in}}%
\pgfpathlineto{\pgfqpoint{0.033333in}{0.033333in}}%
\pgfpathlineto{\pgfqpoint{-0.033333in}{0.033333in}}%
\pgfpathlineto{\pgfqpoint{-0.000000in}{-0.033333in}}%
\pgfpathclose%
\pgfusepath{stroke,fill}%
}%
\begin{pgfscope}%
\pgfsys@transformshift{2.897149in}{3.183561in}%
\pgfsys@useobject{currentmarker}{}%
\end{pgfscope}%
\end{pgfscope}%
\begin{pgfscope}%
\pgfpathrectangle{\pgfqpoint{0.765000in}{0.660000in}}{\pgfqpoint{4.620000in}{4.620000in}}%
\pgfusepath{clip}%
\pgfsetrectcap%
\pgfsetroundjoin%
\pgfsetlinewidth{1.204500pt}%
\definecolor{currentstroke}{rgb}{0.176471,0.192157,0.258824}%
\pgfsetstrokecolor{currentstroke}%
\pgfsetdash{}{0pt}%
\pgfpathmoveto{\pgfqpoint{2.927431in}{3.131107in}}%
\pgfusepath{stroke}%
\end{pgfscope}%
\begin{pgfscope}%
\pgfpathrectangle{\pgfqpoint{0.765000in}{0.660000in}}{\pgfqpoint{4.620000in}{4.620000in}}%
\pgfusepath{clip}%
\pgfsetbuttcap%
\pgfsetmiterjoin%
\definecolor{currentfill}{rgb}{0.176471,0.192157,0.258824}%
\pgfsetfillcolor{currentfill}%
\pgfsetlinewidth{1.003750pt}%
\definecolor{currentstroke}{rgb}{0.176471,0.192157,0.258824}%
\pgfsetstrokecolor{currentstroke}%
\pgfsetdash{}{0pt}%
\pgfsys@defobject{currentmarker}{\pgfqpoint{-0.033333in}{-0.033333in}}{\pgfqpoint{0.033333in}{0.033333in}}{%
\pgfpathmoveto{\pgfqpoint{-0.000000in}{-0.033333in}}%
\pgfpathlineto{\pgfqpoint{0.033333in}{0.033333in}}%
\pgfpathlineto{\pgfqpoint{-0.033333in}{0.033333in}}%
\pgfpathlineto{\pgfqpoint{-0.000000in}{-0.033333in}}%
\pgfpathclose%
\pgfusepath{stroke,fill}%
}%
\begin{pgfscope}%
\pgfsys@transformshift{2.927431in}{3.131107in}%
\pgfsys@useobject{currentmarker}{}%
\end{pgfscope}%
\end{pgfscope}%
\begin{pgfscope}%
\pgfpathrectangle{\pgfqpoint{0.765000in}{0.660000in}}{\pgfqpoint{4.620000in}{4.620000in}}%
\pgfusepath{clip}%
\pgfsetrectcap%
\pgfsetroundjoin%
\pgfsetlinewidth{1.204500pt}%
\definecolor{currentstroke}{rgb}{0.176471,0.192157,0.258824}%
\pgfsetstrokecolor{currentstroke}%
\pgfsetdash{}{0pt}%
\pgfpathmoveto{\pgfqpoint{3.211963in}{3.099725in}}%
\pgfusepath{stroke}%
\end{pgfscope}%
\begin{pgfscope}%
\pgfpathrectangle{\pgfqpoint{0.765000in}{0.660000in}}{\pgfqpoint{4.620000in}{4.620000in}}%
\pgfusepath{clip}%
\pgfsetbuttcap%
\pgfsetmiterjoin%
\definecolor{currentfill}{rgb}{0.176471,0.192157,0.258824}%
\pgfsetfillcolor{currentfill}%
\pgfsetlinewidth{1.003750pt}%
\definecolor{currentstroke}{rgb}{0.176471,0.192157,0.258824}%
\pgfsetstrokecolor{currentstroke}%
\pgfsetdash{}{0pt}%
\pgfsys@defobject{currentmarker}{\pgfqpoint{-0.033333in}{-0.033333in}}{\pgfqpoint{0.033333in}{0.033333in}}{%
\pgfpathmoveto{\pgfqpoint{-0.000000in}{-0.033333in}}%
\pgfpathlineto{\pgfqpoint{0.033333in}{0.033333in}}%
\pgfpathlineto{\pgfqpoint{-0.033333in}{0.033333in}}%
\pgfpathlineto{\pgfqpoint{-0.000000in}{-0.033333in}}%
\pgfpathclose%
\pgfusepath{stroke,fill}%
}%
\begin{pgfscope}%
\pgfsys@transformshift{3.211963in}{3.099725in}%
\pgfsys@useobject{currentmarker}{}%
\end{pgfscope}%
\end{pgfscope}%
\begin{pgfscope}%
\pgfpathrectangle{\pgfqpoint{0.765000in}{0.660000in}}{\pgfqpoint{4.620000in}{4.620000in}}%
\pgfusepath{clip}%
\pgfsetrectcap%
\pgfsetroundjoin%
\pgfsetlinewidth{1.204500pt}%
\definecolor{currentstroke}{rgb}{0.176471,0.192157,0.258824}%
\pgfsetstrokecolor{currentstroke}%
\pgfsetdash{}{0pt}%
\pgfpathmoveto{\pgfqpoint{3.218699in}{3.075680in}}%
\pgfusepath{stroke}%
\end{pgfscope}%
\begin{pgfscope}%
\pgfpathrectangle{\pgfqpoint{0.765000in}{0.660000in}}{\pgfqpoint{4.620000in}{4.620000in}}%
\pgfusepath{clip}%
\pgfsetbuttcap%
\pgfsetmiterjoin%
\definecolor{currentfill}{rgb}{0.176471,0.192157,0.258824}%
\pgfsetfillcolor{currentfill}%
\pgfsetlinewidth{1.003750pt}%
\definecolor{currentstroke}{rgb}{0.176471,0.192157,0.258824}%
\pgfsetstrokecolor{currentstroke}%
\pgfsetdash{}{0pt}%
\pgfsys@defobject{currentmarker}{\pgfqpoint{-0.033333in}{-0.033333in}}{\pgfqpoint{0.033333in}{0.033333in}}{%
\pgfpathmoveto{\pgfqpoint{-0.000000in}{-0.033333in}}%
\pgfpathlineto{\pgfqpoint{0.033333in}{0.033333in}}%
\pgfpathlineto{\pgfqpoint{-0.033333in}{0.033333in}}%
\pgfpathlineto{\pgfqpoint{-0.000000in}{-0.033333in}}%
\pgfpathclose%
\pgfusepath{stroke,fill}%
}%
\begin{pgfscope}%
\pgfsys@transformshift{3.218699in}{3.075680in}%
\pgfsys@useobject{currentmarker}{}%
\end{pgfscope}%
\end{pgfscope}%
\begin{pgfscope}%
\pgfpathrectangle{\pgfqpoint{0.765000in}{0.660000in}}{\pgfqpoint{4.620000in}{4.620000in}}%
\pgfusepath{clip}%
\pgfsetrectcap%
\pgfsetroundjoin%
\pgfsetlinewidth{1.204500pt}%
\definecolor{currentstroke}{rgb}{0.176471,0.192157,0.258824}%
\pgfsetstrokecolor{currentstroke}%
\pgfsetdash{}{0pt}%
\pgfpathmoveto{\pgfqpoint{2.975479in}{3.165570in}}%
\pgfusepath{stroke}%
\end{pgfscope}%
\begin{pgfscope}%
\pgfpathrectangle{\pgfqpoint{0.765000in}{0.660000in}}{\pgfqpoint{4.620000in}{4.620000in}}%
\pgfusepath{clip}%
\pgfsetbuttcap%
\pgfsetmiterjoin%
\definecolor{currentfill}{rgb}{0.176471,0.192157,0.258824}%
\pgfsetfillcolor{currentfill}%
\pgfsetlinewidth{1.003750pt}%
\definecolor{currentstroke}{rgb}{0.176471,0.192157,0.258824}%
\pgfsetstrokecolor{currentstroke}%
\pgfsetdash{}{0pt}%
\pgfsys@defobject{currentmarker}{\pgfqpoint{-0.033333in}{-0.033333in}}{\pgfqpoint{0.033333in}{0.033333in}}{%
\pgfpathmoveto{\pgfqpoint{-0.000000in}{-0.033333in}}%
\pgfpathlineto{\pgfqpoint{0.033333in}{0.033333in}}%
\pgfpathlineto{\pgfqpoint{-0.033333in}{0.033333in}}%
\pgfpathlineto{\pgfqpoint{-0.000000in}{-0.033333in}}%
\pgfpathclose%
\pgfusepath{stroke,fill}%
}%
\begin{pgfscope}%
\pgfsys@transformshift{2.975479in}{3.165570in}%
\pgfsys@useobject{currentmarker}{}%
\end{pgfscope}%
\end{pgfscope}%
\begin{pgfscope}%
\pgfpathrectangle{\pgfqpoint{0.765000in}{0.660000in}}{\pgfqpoint{4.620000in}{4.620000in}}%
\pgfusepath{clip}%
\pgfsetrectcap%
\pgfsetroundjoin%
\pgfsetlinewidth{1.204500pt}%
\definecolor{currentstroke}{rgb}{0.176471,0.192157,0.258824}%
\pgfsetstrokecolor{currentstroke}%
\pgfsetdash{}{0pt}%
\pgfpathmoveto{\pgfqpoint{3.214146in}{3.060766in}}%
\pgfusepath{stroke}%
\end{pgfscope}%
\begin{pgfscope}%
\pgfpathrectangle{\pgfqpoint{0.765000in}{0.660000in}}{\pgfqpoint{4.620000in}{4.620000in}}%
\pgfusepath{clip}%
\pgfsetbuttcap%
\pgfsetmiterjoin%
\definecolor{currentfill}{rgb}{0.176471,0.192157,0.258824}%
\pgfsetfillcolor{currentfill}%
\pgfsetlinewidth{1.003750pt}%
\definecolor{currentstroke}{rgb}{0.176471,0.192157,0.258824}%
\pgfsetstrokecolor{currentstroke}%
\pgfsetdash{}{0pt}%
\pgfsys@defobject{currentmarker}{\pgfqpoint{-0.033333in}{-0.033333in}}{\pgfqpoint{0.033333in}{0.033333in}}{%
\pgfpathmoveto{\pgfqpoint{-0.000000in}{-0.033333in}}%
\pgfpathlineto{\pgfqpoint{0.033333in}{0.033333in}}%
\pgfpathlineto{\pgfqpoint{-0.033333in}{0.033333in}}%
\pgfpathlineto{\pgfqpoint{-0.000000in}{-0.033333in}}%
\pgfpathclose%
\pgfusepath{stroke,fill}%
}%
\begin{pgfscope}%
\pgfsys@transformshift{3.214146in}{3.060766in}%
\pgfsys@useobject{currentmarker}{}%
\end{pgfscope}%
\end{pgfscope}%
\begin{pgfscope}%
\pgfpathrectangle{\pgfqpoint{0.765000in}{0.660000in}}{\pgfqpoint{4.620000in}{4.620000in}}%
\pgfusepath{clip}%
\pgfsetrectcap%
\pgfsetroundjoin%
\pgfsetlinewidth{1.204500pt}%
\definecolor{currentstroke}{rgb}{0.176471,0.192157,0.258824}%
\pgfsetstrokecolor{currentstroke}%
\pgfsetdash{}{0pt}%
\pgfpathmoveto{\pgfqpoint{3.350563in}{2.926523in}}%
\pgfusepath{stroke}%
\end{pgfscope}%
\begin{pgfscope}%
\pgfpathrectangle{\pgfqpoint{0.765000in}{0.660000in}}{\pgfqpoint{4.620000in}{4.620000in}}%
\pgfusepath{clip}%
\pgfsetbuttcap%
\pgfsetmiterjoin%
\definecolor{currentfill}{rgb}{0.176471,0.192157,0.258824}%
\pgfsetfillcolor{currentfill}%
\pgfsetlinewidth{1.003750pt}%
\definecolor{currentstroke}{rgb}{0.176471,0.192157,0.258824}%
\pgfsetstrokecolor{currentstroke}%
\pgfsetdash{}{0pt}%
\pgfsys@defobject{currentmarker}{\pgfqpoint{-0.033333in}{-0.033333in}}{\pgfqpoint{0.033333in}{0.033333in}}{%
\pgfpathmoveto{\pgfqpoint{-0.000000in}{-0.033333in}}%
\pgfpathlineto{\pgfqpoint{0.033333in}{0.033333in}}%
\pgfpathlineto{\pgfqpoint{-0.033333in}{0.033333in}}%
\pgfpathlineto{\pgfqpoint{-0.000000in}{-0.033333in}}%
\pgfpathclose%
\pgfusepath{stroke,fill}%
}%
\begin{pgfscope}%
\pgfsys@transformshift{3.350563in}{2.926523in}%
\pgfsys@useobject{currentmarker}{}%
\end{pgfscope}%
\end{pgfscope}%
\begin{pgfscope}%
\pgfpathrectangle{\pgfqpoint{0.765000in}{0.660000in}}{\pgfqpoint{4.620000in}{4.620000in}}%
\pgfusepath{clip}%
\pgfsetrectcap%
\pgfsetroundjoin%
\pgfsetlinewidth{1.204500pt}%
\definecolor{currentstroke}{rgb}{0.176471,0.192157,0.258824}%
\pgfsetstrokecolor{currentstroke}%
\pgfsetdash{}{0pt}%
\pgfpathmoveto{\pgfqpoint{3.101871in}{2.913666in}}%
\pgfusepath{stroke}%
\end{pgfscope}%
\begin{pgfscope}%
\pgfpathrectangle{\pgfqpoint{0.765000in}{0.660000in}}{\pgfqpoint{4.620000in}{4.620000in}}%
\pgfusepath{clip}%
\pgfsetbuttcap%
\pgfsetmiterjoin%
\definecolor{currentfill}{rgb}{0.176471,0.192157,0.258824}%
\pgfsetfillcolor{currentfill}%
\pgfsetlinewidth{1.003750pt}%
\definecolor{currentstroke}{rgb}{0.176471,0.192157,0.258824}%
\pgfsetstrokecolor{currentstroke}%
\pgfsetdash{}{0pt}%
\pgfsys@defobject{currentmarker}{\pgfqpoint{-0.033333in}{-0.033333in}}{\pgfqpoint{0.033333in}{0.033333in}}{%
\pgfpathmoveto{\pgfqpoint{-0.000000in}{-0.033333in}}%
\pgfpathlineto{\pgfqpoint{0.033333in}{0.033333in}}%
\pgfpathlineto{\pgfqpoint{-0.033333in}{0.033333in}}%
\pgfpathlineto{\pgfqpoint{-0.000000in}{-0.033333in}}%
\pgfpathclose%
\pgfusepath{stroke,fill}%
}%
\begin{pgfscope}%
\pgfsys@transformshift{3.101871in}{2.913666in}%
\pgfsys@useobject{currentmarker}{}%
\end{pgfscope}%
\end{pgfscope}%
\begin{pgfscope}%
\pgfpathrectangle{\pgfqpoint{0.765000in}{0.660000in}}{\pgfqpoint{4.620000in}{4.620000in}}%
\pgfusepath{clip}%
\pgfsetrectcap%
\pgfsetroundjoin%
\pgfsetlinewidth{1.204500pt}%
\definecolor{currentstroke}{rgb}{0.176471,0.192157,0.258824}%
\pgfsetstrokecolor{currentstroke}%
\pgfsetdash{}{0pt}%
\pgfpathmoveto{\pgfqpoint{3.236664in}{3.233008in}}%
\pgfusepath{stroke}%
\end{pgfscope}%
\begin{pgfscope}%
\pgfpathrectangle{\pgfqpoint{0.765000in}{0.660000in}}{\pgfqpoint{4.620000in}{4.620000in}}%
\pgfusepath{clip}%
\pgfsetbuttcap%
\pgfsetmiterjoin%
\definecolor{currentfill}{rgb}{0.176471,0.192157,0.258824}%
\pgfsetfillcolor{currentfill}%
\pgfsetlinewidth{1.003750pt}%
\definecolor{currentstroke}{rgb}{0.176471,0.192157,0.258824}%
\pgfsetstrokecolor{currentstroke}%
\pgfsetdash{}{0pt}%
\pgfsys@defobject{currentmarker}{\pgfqpoint{-0.033333in}{-0.033333in}}{\pgfqpoint{0.033333in}{0.033333in}}{%
\pgfpathmoveto{\pgfqpoint{-0.000000in}{-0.033333in}}%
\pgfpathlineto{\pgfqpoint{0.033333in}{0.033333in}}%
\pgfpathlineto{\pgfqpoint{-0.033333in}{0.033333in}}%
\pgfpathlineto{\pgfqpoint{-0.000000in}{-0.033333in}}%
\pgfpathclose%
\pgfusepath{stroke,fill}%
}%
\begin{pgfscope}%
\pgfsys@transformshift{3.236664in}{3.233008in}%
\pgfsys@useobject{currentmarker}{}%
\end{pgfscope}%
\end{pgfscope}%
\begin{pgfscope}%
\pgfpathrectangle{\pgfqpoint{0.765000in}{0.660000in}}{\pgfqpoint{4.620000in}{4.620000in}}%
\pgfusepath{clip}%
\pgfsetrectcap%
\pgfsetroundjoin%
\pgfsetlinewidth{1.204500pt}%
\definecolor{currentstroke}{rgb}{0.176471,0.192157,0.258824}%
\pgfsetstrokecolor{currentstroke}%
\pgfsetdash{}{0pt}%
\pgfpathmoveto{\pgfqpoint{2.971495in}{3.231013in}}%
\pgfusepath{stroke}%
\end{pgfscope}%
\begin{pgfscope}%
\pgfpathrectangle{\pgfqpoint{0.765000in}{0.660000in}}{\pgfqpoint{4.620000in}{4.620000in}}%
\pgfusepath{clip}%
\pgfsetbuttcap%
\pgfsetmiterjoin%
\definecolor{currentfill}{rgb}{0.176471,0.192157,0.258824}%
\pgfsetfillcolor{currentfill}%
\pgfsetlinewidth{1.003750pt}%
\definecolor{currentstroke}{rgb}{0.176471,0.192157,0.258824}%
\pgfsetstrokecolor{currentstroke}%
\pgfsetdash{}{0pt}%
\pgfsys@defobject{currentmarker}{\pgfqpoint{-0.033333in}{-0.033333in}}{\pgfqpoint{0.033333in}{0.033333in}}{%
\pgfpathmoveto{\pgfqpoint{-0.000000in}{-0.033333in}}%
\pgfpathlineto{\pgfqpoint{0.033333in}{0.033333in}}%
\pgfpathlineto{\pgfqpoint{-0.033333in}{0.033333in}}%
\pgfpathlineto{\pgfqpoint{-0.000000in}{-0.033333in}}%
\pgfpathclose%
\pgfusepath{stroke,fill}%
}%
\begin{pgfscope}%
\pgfsys@transformshift{2.971495in}{3.231013in}%
\pgfsys@useobject{currentmarker}{}%
\end{pgfscope}%
\end{pgfscope}%
\begin{pgfscope}%
\pgfpathrectangle{\pgfqpoint{0.765000in}{0.660000in}}{\pgfqpoint{4.620000in}{4.620000in}}%
\pgfusepath{clip}%
\pgfsetrectcap%
\pgfsetroundjoin%
\pgfsetlinewidth{1.204500pt}%
\definecolor{currentstroke}{rgb}{0.176471,0.192157,0.258824}%
\pgfsetstrokecolor{currentstroke}%
\pgfsetdash{}{0pt}%
\pgfpathmoveto{\pgfqpoint{3.002064in}{3.106786in}}%
\pgfusepath{stroke}%
\end{pgfscope}%
\begin{pgfscope}%
\pgfpathrectangle{\pgfqpoint{0.765000in}{0.660000in}}{\pgfqpoint{4.620000in}{4.620000in}}%
\pgfusepath{clip}%
\pgfsetbuttcap%
\pgfsetmiterjoin%
\definecolor{currentfill}{rgb}{0.176471,0.192157,0.258824}%
\pgfsetfillcolor{currentfill}%
\pgfsetlinewidth{1.003750pt}%
\definecolor{currentstroke}{rgb}{0.176471,0.192157,0.258824}%
\pgfsetstrokecolor{currentstroke}%
\pgfsetdash{}{0pt}%
\pgfsys@defobject{currentmarker}{\pgfqpoint{-0.033333in}{-0.033333in}}{\pgfqpoint{0.033333in}{0.033333in}}{%
\pgfpathmoveto{\pgfqpoint{-0.000000in}{-0.033333in}}%
\pgfpathlineto{\pgfqpoint{0.033333in}{0.033333in}}%
\pgfpathlineto{\pgfqpoint{-0.033333in}{0.033333in}}%
\pgfpathlineto{\pgfqpoint{-0.000000in}{-0.033333in}}%
\pgfpathclose%
\pgfusepath{stroke,fill}%
}%
\begin{pgfscope}%
\pgfsys@transformshift{3.002064in}{3.106786in}%
\pgfsys@useobject{currentmarker}{}%
\end{pgfscope}%
\end{pgfscope}%
\begin{pgfscope}%
\pgfpathrectangle{\pgfqpoint{0.765000in}{0.660000in}}{\pgfqpoint{4.620000in}{4.620000in}}%
\pgfusepath{clip}%
\pgfsetrectcap%
\pgfsetroundjoin%
\pgfsetlinewidth{1.204500pt}%
\definecolor{currentstroke}{rgb}{0.176471,0.192157,0.258824}%
\pgfsetstrokecolor{currentstroke}%
\pgfsetdash{}{0pt}%
\pgfpathmoveto{\pgfqpoint{3.208241in}{3.203556in}}%
\pgfusepath{stroke}%
\end{pgfscope}%
\begin{pgfscope}%
\pgfpathrectangle{\pgfqpoint{0.765000in}{0.660000in}}{\pgfqpoint{4.620000in}{4.620000in}}%
\pgfusepath{clip}%
\pgfsetbuttcap%
\pgfsetmiterjoin%
\definecolor{currentfill}{rgb}{0.176471,0.192157,0.258824}%
\pgfsetfillcolor{currentfill}%
\pgfsetlinewidth{1.003750pt}%
\definecolor{currentstroke}{rgb}{0.176471,0.192157,0.258824}%
\pgfsetstrokecolor{currentstroke}%
\pgfsetdash{}{0pt}%
\pgfsys@defobject{currentmarker}{\pgfqpoint{-0.033333in}{-0.033333in}}{\pgfqpoint{0.033333in}{0.033333in}}{%
\pgfpathmoveto{\pgfqpoint{-0.000000in}{-0.033333in}}%
\pgfpathlineto{\pgfqpoint{0.033333in}{0.033333in}}%
\pgfpathlineto{\pgfqpoint{-0.033333in}{0.033333in}}%
\pgfpathlineto{\pgfqpoint{-0.000000in}{-0.033333in}}%
\pgfpathclose%
\pgfusepath{stroke,fill}%
}%
\begin{pgfscope}%
\pgfsys@transformshift{3.208241in}{3.203556in}%
\pgfsys@useobject{currentmarker}{}%
\end{pgfscope}%
\end{pgfscope}%
\begin{pgfscope}%
\pgfpathrectangle{\pgfqpoint{0.765000in}{0.660000in}}{\pgfqpoint{4.620000in}{4.620000in}}%
\pgfusepath{clip}%
\pgfsetrectcap%
\pgfsetroundjoin%
\pgfsetlinewidth{1.204500pt}%
\definecolor{currentstroke}{rgb}{0.176471,0.192157,0.258824}%
\pgfsetstrokecolor{currentstroke}%
\pgfsetdash{}{0pt}%
\pgfpathmoveto{\pgfqpoint{2.985480in}{3.050363in}}%
\pgfusepath{stroke}%
\end{pgfscope}%
\begin{pgfscope}%
\pgfpathrectangle{\pgfqpoint{0.765000in}{0.660000in}}{\pgfqpoint{4.620000in}{4.620000in}}%
\pgfusepath{clip}%
\pgfsetbuttcap%
\pgfsetmiterjoin%
\definecolor{currentfill}{rgb}{0.176471,0.192157,0.258824}%
\pgfsetfillcolor{currentfill}%
\pgfsetlinewidth{1.003750pt}%
\definecolor{currentstroke}{rgb}{0.176471,0.192157,0.258824}%
\pgfsetstrokecolor{currentstroke}%
\pgfsetdash{}{0pt}%
\pgfsys@defobject{currentmarker}{\pgfqpoint{-0.033333in}{-0.033333in}}{\pgfqpoint{0.033333in}{0.033333in}}{%
\pgfpathmoveto{\pgfqpoint{-0.000000in}{-0.033333in}}%
\pgfpathlineto{\pgfqpoint{0.033333in}{0.033333in}}%
\pgfpathlineto{\pgfqpoint{-0.033333in}{0.033333in}}%
\pgfpathlineto{\pgfqpoint{-0.000000in}{-0.033333in}}%
\pgfpathclose%
\pgfusepath{stroke,fill}%
}%
\begin{pgfscope}%
\pgfsys@transformshift{2.985480in}{3.050363in}%
\pgfsys@useobject{currentmarker}{}%
\end{pgfscope}%
\end{pgfscope}%
\begin{pgfscope}%
\pgfpathrectangle{\pgfqpoint{0.765000in}{0.660000in}}{\pgfqpoint{4.620000in}{4.620000in}}%
\pgfusepath{clip}%
\pgfsetrectcap%
\pgfsetroundjoin%
\pgfsetlinewidth{1.204500pt}%
\definecolor{currentstroke}{rgb}{0.176471,0.192157,0.258824}%
\pgfsetstrokecolor{currentstroke}%
\pgfsetdash{}{0pt}%
\pgfpathmoveto{\pgfqpoint{3.113707in}{3.312267in}}%
\pgfusepath{stroke}%
\end{pgfscope}%
\begin{pgfscope}%
\pgfpathrectangle{\pgfqpoint{0.765000in}{0.660000in}}{\pgfqpoint{4.620000in}{4.620000in}}%
\pgfusepath{clip}%
\pgfsetbuttcap%
\pgfsetmiterjoin%
\definecolor{currentfill}{rgb}{0.176471,0.192157,0.258824}%
\pgfsetfillcolor{currentfill}%
\pgfsetlinewidth{1.003750pt}%
\definecolor{currentstroke}{rgb}{0.176471,0.192157,0.258824}%
\pgfsetstrokecolor{currentstroke}%
\pgfsetdash{}{0pt}%
\pgfsys@defobject{currentmarker}{\pgfqpoint{-0.033333in}{-0.033333in}}{\pgfqpoint{0.033333in}{0.033333in}}{%
\pgfpathmoveto{\pgfqpoint{-0.000000in}{-0.033333in}}%
\pgfpathlineto{\pgfqpoint{0.033333in}{0.033333in}}%
\pgfpathlineto{\pgfqpoint{-0.033333in}{0.033333in}}%
\pgfpathlineto{\pgfqpoint{-0.000000in}{-0.033333in}}%
\pgfpathclose%
\pgfusepath{stroke,fill}%
}%
\begin{pgfscope}%
\pgfsys@transformshift{3.113707in}{3.312267in}%
\pgfsys@useobject{currentmarker}{}%
\end{pgfscope}%
\end{pgfscope}%
\begin{pgfscope}%
\pgfpathrectangle{\pgfqpoint{0.765000in}{0.660000in}}{\pgfqpoint{4.620000in}{4.620000in}}%
\pgfusepath{clip}%
\pgfsetrectcap%
\pgfsetroundjoin%
\pgfsetlinewidth{1.204500pt}%
\definecolor{currentstroke}{rgb}{0.176471,0.192157,0.258824}%
\pgfsetstrokecolor{currentstroke}%
\pgfsetdash{}{0pt}%
\pgfpathmoveto{\pgfqpoint{3.209310in}{3.069953in}}%
\pgfusepath{stroke}%
\end{pgfscope}%
\begin{pgfscope}%
\pgfpathrectangle{\pgfqpoint{0.765000in}{0.660000in}}{\pgfqpoint{4.620000in}{4.620000in}}%
\pgfusepath{clip}%
\pgfsetbuttcap%
\pgfsetmiterjoin%
\definecolor{currentfill}{rgb}{0.176471,0.192157,0.258824}%
\pgfsetfillcolor{currentfill}%
\pgfsetlinewidth{1.003750pt}%
\definecolor{currentstroke}{rgb}{0.176471,0.192157,0.258824}%
\pgfsetstrokecolor{currentstroke}%
\pgfsetdash{}{0pt}%
\pgfsys@defobject{currentmarker}{\pgfqpoint{-0.033333in}{-0.033333in}}{\pgfqpoint{0.033333in}{0.033333in}}{%
\pgfpathmoveto{\pgfqpoint{-0.000000in}{-0.033333in}}%
\pgfpathlineto{\pgfqpoint{0.033333in}{0.033333in}}%
\pgfpathlineto{\pgfqpoint{-0.033333in}{0.033333in}}%
\pgfpathlineto{\pgfqpoint{-0.000000in}{-0.033333in}}%
\pgfpathclose%
\pgfusepath{stroke,fill}%
}%
\begin{pgfscope}%
\pgfsys@transformshift{3.209310in}{3.069953in}%
\pgfsys@useobject{currentmarker}{}%
\end{pgfscope}%
\end{pgfscope}%
\begin{pgfscope}%
\pgfpathrectangle{\pgfqpoint{0.765000in}{0.660000in}}{\pgfqpoint{4.620000in}{4.620000in}}%
\pgfusepath{clip}%
\pgfsetrectcap%
\pgfsetroundjoin%
\pgfsetlinewidth{1.204500pt}%
\definecolor{currentstroke}{rgb}{0.000000,0.000000,0.000000}%
\pgfsetstrokecolor{currentstroke}%
\pgfsetdash{}{0pt}%
\pgfpathmoveto{\pgfqpoint{3.149560in}{2.514906in}}%
\pgfusepath{stroke}%
\end{pgfscope}%
\begin{pgfscope}%
\pgfpathrectangle{\pgfqpoint{0.765000in}{0.660000in}}{\pgfqpoint{4.620000in}{4.620000in}}%
\pgfusepath{clip}%
\pgfsetbuttcap%
\pgfsetroundjoin%
\definecolor{currentfill}{rgb}{0.000000,0.000000,0.000000}%
\pgfsetfillcolor{currentfill}%
\pgfsetlinewidth{1.003750pt}%
\definecolor{currentstroke}{rgb}{0.000000,0.000000,0.000000}%
\pgfsetstrokecolor{currentstroke}%
\pgfsetdash{}{0pt}%
\pgfsys@defobject{currentmarker}{\pgfqpoint{-0.016667in}{-0.016667in}}{\pgfqpoint{0.016667in}{0.016667in}}{%
\pgfpathmoveto{\pgfqpoint{0.000000in}{-0.016667in}}%
\pgfpathcurveto{\pgfqpoint{0.004420in}{-0.016667in}}{\pgfqpoint{0.008660in}{-0.014911in}}{\pgfqpoint{0.011785in}{-0.011785in}}%
\pgfpathcurveto{\pgfqpoint{0.014911in}{-0.008660in}}{\pgfqpoint{0.016667in}{-0.004420in}}{\pgfqpoint{0.016667in}{0.000000in}}%
\pgfpathcurveto{\pgfqpoint{0.016667in}{0.004420in}}{\pgfqpoint{0.014911in}{0.008660in}}{\pgfqpoint{0.011785in}{0.011785in}}%
\pgfpathcurveto{\pgfqpoint{0.008660in}{0.014911in}}{\pgfqpoint{0.004420in}{0.016667in}}{\pgfqpoint{0.000000in}{0.016667in}}%
\pgfpathcurveto{\pgfqpoint{-0.004420in}{0.016667in}}{\pgfqpoint{-0.008660in}{0.014911in}}{\pgfqpoint{-0.011785in}{0.011785in}}%
\pgfpathcurveto{\pgfqpoint{-0.014911in}{0.008660in}}{\pgfqpoint{-0.016667in}{0.004420in}}{\pgfqpoint{-0.016667in}{0.000000in}}%
\pgfpathcurveto{\pgfqpoint{-0.016667in}{-0.004420in}}{\pgfqpoint{-0.014911in}{-0.008660in}}{\pgfqpoint{-0.011785in}{-0.011785in}}%
\pgfpathcurveto{\pgfqpoint{-0.008660in}{-0.014911in}}{\pgfqpoint{-0.004420in}{-0.016667in}}{\pgfqpoint{0.000000in}{-0.016667in}}%
\pgfpathlineto{\pgfqpoint{0.000000in}{-0.016667in}}%
\pgfpathclose%
\pgfusepath{stroke,fill}%
}%
\begin{pgfscope}%
\pgfsys@transformshift{3.149560in}{2.514906in}%
\pgfsys@useobject{currentmarker}{}%
\end{pgfscope}%
\end{pgfscope}%
\begin{pgfscope}%
\pgfpathrectangle{\pgfqpoint{0.765000in}{0.660000in}}{\pgfqpoint{4.620000in}{4.620000in}}%
\pgfusepath{clip}%
\pgfsetrectcap%
\pgfsetroundjoin%
\pgfsetlinewidth{1.204500pt}%
\definecolor{currentstroke}{rgb}{0.000000,0.000000,0.000000}%
\pgfsetstrokecolor{currentstroke}%
\pgfsetdash{}{0pt}%
\pgfpathmoveto{\pgfqpoint{3.149560in}{2.514906in}}%
\pgfpathlineto{\pgfqpoint{3.025552in}{2.586502in}}%
\pgfusepath{stroke}%
\end{pgfscope}%
\begin{pgfscope}%
\pgfpathrectangle{\pgfqpoint{0.765000in}{0.660000in}}{\pgfqpoint{4.620000in}{4.620000in}}%
\pgfusepath{clip}%
\pgfsetrectcap%
\pgfsetroundjoin%
\pgfsetlinewidth{1.204500pt}%
\definecolor{currentstroke}{rgb}{0.000000,0.000000,0.000000}%
\pgfsetstrokecolor{currentstroke}%
\pgfsetdash{}{0pt}%
\pgfpathmoveto{\pgfqpoint{3.149560in}{2.514906in}}%
\pgfpathlineto{\pgfqpoint{3.337483in}{2.623404in}}%
\pgfusepath{stroke}%
\end{pgfscope}%
\begin{pgfscope}%
\pgfpathrectangle{\pgfqpoint{0.765000in}{0.660000in}}{\pgfqpoint{4.620000in}{4.620000in}}%
\pgfusepath{clip}%
\pgfsetbuttcap%
\pgfsetroundjoin%
\pgfsetlinewidth{1.204500pt}%
\definecolor{currentstroke}{rgb}{0.827451,0.827451,0.827451}%
\pgfsetstrokecolor{currentstroke}%
\pgfsetdash{{1.200000pt}{1.980000pt}}{0.000000pt}%
\pgfpathmoveto{\pgfqpoint{3.149560in}{2.514906in}}%
\pgfpathlineto{\pgfqpoint{3.149560in}{0.650000in}}%
\pgfusepath{stroke}%
\end{pgfscope}%
\begin{pgfscope}%
\pgfpathrectangle{\pgfqpoint{0.765000in}{0.660000in}}{\pgfqpoint{4.620000in}{4.620000in}}%
\pgfusepath{clip}%
\pgfsetrectcap%
\pgfsetroundjoin%
\pgfsetlinewidth{1.204500pt}%
\definecolor{currentstroke}{rgb}{0.000000,0.000000,0.000000}%
\pgfsetstrokecolor{currentstroke}%
\pgfsetdash{}{0pt}%
\pgfpathmoveto{\pgfqpoint{3.025552in}{2.586502in}}%
\pgfusepath{stroke}%
\end{pgfscope}%
\begin{pgfscope}%
\pgfpathrectangle{\pgfqpoint{0.765000in}{0.660000in}}{\pgfqpoint{4.620000in}{4.620000in}}%
\pgfusepath{clip}%
\pgfsetbuttcap%
\pgfsetroundjoin%
\definecolor{currentfill}{rgb}{0.000000,0.000000,0.000000}%
\pgfsetfillcolor{currentfill}%
\pgfsetlinewidth{1.003750pt}%
\definecolor{currentstroke}{rgb}{0.000000,0.000000,0.000000}%
\pgfsetstrokecolor{currentstroke}%
\pgfsetdash{}{0pt}%
\pgfsys@defobject{currentmarker}{\pgfqpoint{-0.016667in}{-0.016667in}}{\pgfqpoint{0.016667in}{0.016667in}}{%
\pgfpathmoveto{\pgfqpoint{0.000000in}{-0.016667in}}%
\pgfpathcurveto{\pgfqpoint{0.004420in}{-0.016667in}}{\pgfqpoint{0.008660in}{-0.014911in}}{\pgfqpoint{0.011785in}{-0.011785in}}%
\pgfpathcurveto{\pgfqpoint{0.014911in}{-0.008660in}}{\pgfqpoint{0.016667in}{-0.004420in}}{\pgfqpoint{0.016667in}{0.000000in}}%
\pgfpathcurveto{\pgfqpoint{0.016667in}{0.004420in}}{\pgfqpoint{0.014911in}{0.008660in}}{\pgfqpoint{0.011785in}{0.011785in}}%
\pgfpathcurveto{\pgfqpoint{0.008660in}{0.014911in}}{\pgfqpoint{0.004420in}{0.016667in}}{\pgfqpoint{0.000000in}{0.016667in}}%
\pgfpathcurveto{\pgfqpoint{-0.004420in}{0.016667in}}{\pgfqpoint{-0.008660in}{0.014911in}}{\pgfqpoint{-0.011785in}{0.011785in}}%
\pgfpathcurveto{\pgfqpoint{-0.014911in}{0.008660in}}{\pgfqpoint{-0.016667in}{0.004420in}}{\pgfqpoint{-0.016667in}{0.000000in}}%
\pgfpathcurveto{\pgfqpoint{-0.016667in}{-0.004420in}}{\pgfqpoint{-0.014911in}{-0.008660in}}{\pgfqpoint{-0.011785in}{-0.011785in}}%
\pgfpathcurveto{\pgfqpoint{-0.008660in}{-0.014911in}}{\pgfqpoint{-0.004420in}{-0.016667in}}{\pgfqpoint{0.000000in}{-0.016667in}}%
\pgfpathlineto{\pgfqpoint{0.000000in}{-0.016667in}}%
\pgfpathclose%
\pgfusepath{stroke,fill}%
}%
\begin{pgfscope}%
\pgfsys@transformshift{3.025552in}{2.586502in}%
\pgfsys@useobject{currentmarker}{}%
\end{pgfscope}%
\end{pgfscope}%
\begin{pgfscope}%
\pgfpathrectangle{\pgfqpoint{0.765000in}{0.660000in}}{\pgfqpoint{4.620000in}{4.620000in}}%
\pgfusepath{clip}%
\pgfsetrectcap%
\pgfsetroundjoin%
\pgfsetlinewidth{1.204500pt}%
\definecolor{currentstroke}{rgb}{0.000000,0.000000,0.000000}%
\pgfsetstrokecolor{currentstroke}%
\pgfsetdash{}{0pt}%
\pgfpathmoveto{\pgfqpoint{3.025552in}{2.586502in}}%
\pgfpathlineto{\pgfqpoint{3.149560in}{2.514906in}}%
\pgfusepath{stroke}%
\end{pgfscope}%
\begin{pgfscope}%
\pgfpathrectangle{\pgfqpoint{0.765000in}{0.660000in}}{\pgfqpoint{4.620000in}{4.620000in}}%
\pgfusepath{clip}%
\pgfsetrectcap%
\pgfsetroundjoin%
\pgfsetlinewidth{1.204500pt}%
\definecolor{currentstroke}{rgb}{0.000000,0.000000,0.000000}%
\pgfsetstrokecolor{currentstroke}%
\pgfsetdash{}{0pt}%
\pgfpathmoveto{\pgfqpoint{3.025552in}{2.586502in}}%
\pgfpathlineto{\pgfqpoint{2.654247in}{3.229621in}}%
\pgfusepath{stroke}%
\end{pgfscope}%
\begin{pgfscope}%
\pgfpathrectangle{\pgfqpoint{0.765000in}{0.660000in}}{\pgfqpoint{4.620000in}{4.620000in}}%
\pgfusepath{clip}%
\pgfsetbuttcap%
\pgfsetroundjoin%
\pgfsetlinewidth{1.204500pt}%
\definecolor{currentstroke}{rgb}{0.827451,0.827451,0.827451}%
\pgfsetstrokecolor{currentstroke}%
\pgfsetdash{{1.200000pt}{1.980000pt}}{0.000000pt}%
\pgfpathmoveto{\pgfqpoint{3.025552in}{2.586502in}}%
\pgfpathlineto{\pgfqpoint{3.025552in}{0.650000in}}%
\pgfusepath{stroke}%
\end{pgfscope}%
\begin{pgfscope}%
\pgfpathrectangle{\pgfqpoint{0.765000in}{0.660000in}}{\pgfqpoint{4.620000in}{4.620000in}}%
\pgfusepath{clip}%
\pgfsetrectcap%
\pgfsetroundjoin%
\pgfsetlinewidth{1.204500pt}%
\definecolor{currentstroke}{rgb}{0.000000,0.000000,0.000000}%
\pgfsetstrokecolor{currentstroke}%
\pgfsetdash{}{0pt}%
\pgfpathmoveto{\pgfqpoint{2.654247in}{3.328328in}}%
\pgfusepath{stroke}%
\end{pgfscope}%
\begin{pgfscope}%
\pgfpathrectangle{\pgfqpoint{0.765000in}{0.660000in}}{\pgfqpoint{4.620000in}{4.620000in}}%
\pgfusepath{clip}%
\pgfsetbuttcap%
\pgfsetroundjoin%
\definecolor{currentfill}{rgb}{0.000000,0.000000,0.000000}%
\pgfsetfillcolor{currentfill}%
\pgfsetlinewidth{1.003750pt}%
\definecolor{currentstroke}{rgb}{0.000000,0.000000,0.000000}%
\pgfsetstrokecolor{currentstroke}%
\pgfsetdash{}{0pt}%
\pgfsys@defobject{currentmarker}{\pgfqpoint{-0.016667in}{-0.016667in}}{\pgfqpoint{0.016667in}{0.016667in}}{%
\pgfpathmoveto{\pgfqpoint{0.000000in}{-0.016667in}}%
\pgfpathcurveto{\pgfqpoint{0.004420in}{-0.016667in}}{\pgfqpoint{0.008660in}{-0.014911in}}{\pgfqpoint{0.011785in}{-0.011785in}}%
\pgfpathcurveto{\pgfqpoint{0.014911in}{-0.008660in}}{\pgfqpoint{0.016667in}{-0.004420in}}{\pgfqpoint{0.016667in}{0.000000in}}%
\pgfpathcurveto{\pgfqpoint{0.016667in}{0.004420in}}{\pgfqpoint{0.014911in}{0.008660in}}{\pgfqpoint{0.011785in}{0.011785in}}%
\pgfpathcurveto{\pgfqpoint{0.008660in}{0.014911in}}{\pgfqpoint{0.004420in}{0.016667in}}{\pgfqpoint{0.000000in}{0.016667in}}%
\pgfpathcurveto{\pgfqpoint{-0.004420in}{0.016667in}}{\pgfqpoint{-0.008660in}{0.014911in}}{\pgfqpoint{-0.011785in}{0.011785in}}%
\pgfpathcurveto{\pgfqpoint{-0.014911in}{0.008660in}}{\pgfqpoint{-0.016667in}{0.004420in}}{\pgfqpoint{-0.016667in}{0.000000in}}%
\pgfpathcurveto{\pgfqpoint{-0.016667in}{-0.004420in}}{\pgfqpoint{-0.014911in}{-0.008660in}}{\pgfqpoint{-0.011785in}{-0.011785in}}%
\pgfpathcurveto{\pgfqpoint{-0.008660in}{-0.014911in}}{\pgfqpoint{-0.004420in}{-0.016667in}}{\pgfqpoint{0.000000in}{-0.016667in}}%
\pgfpathlineto{\pgfqpoint{0.000000in}{-0.016667in}}%
\pgfpathclose%
\pgfusepath{stroke,fill}%
}%
\begin{pgfscope}%
\pgfsys@transformshift{2.654247in}{3.328328in}%
\pgfsys@useobject{currentmarker}{}%
\end{pgfscope}%
\end{pgfscope}%
\begin{pgfscope}%
\pgfpathrectangle{\pgfqpoint{0.765000in}{0.660000in}}{\pgfqpoint{4.620000in}{4.620000in}}%
\pgfusepath{clip}%
\pgfsetrectcap%
\pgfsetroundjoin%
\pgfsetlinewidth{1.204500pt}%
\definecolor{currentstroke}{rgb}{0.000000,0.000000,0.000000}%
\pgfsetstrokecolor{currentstroke}%
\pgfsetdash{}{0pt}%
\pgfpathmoveto{\pgfqpoint{2.654247in}{3.328328in}}%
\pgfpathlineto{\pgfqpoint{2.654247in}{3.229621in}}%
\pgfusepath{stroke}%
\end{pgfscope}%
\begin{pgfscope}%
\pgfpathrectangle{\pgfqpoint{0.765000in}{0.660000in}}{\pgfqpoint{4.620000in}{4.620000in}}%
\pgfusepath{clip}%
\pgfsetrectcap%
\pgfsetroundjoin%
\pgfsetlinewidth{1.204500pt}%
\definecolor{currentstroke}{rgb}{0.000000,0.000000,0.000000}%
\pgfsetstrokecolor{currentstroke}%
\pgfsetdash{}{0pt}%
\pgfpathmoveto{\pgfqpoint{2.654247in}{3.328328in}}%
\pgfpathlineto{\pgfqpoint{2.869318in}{3.452500in}}%
\pgfusepath{stroke}%
\end{pgfscope}%
\begin{pgfscope}%
\pgfpathrectangle{\pgfqpoint{0.765000in}{0.660000in}}{\pgfqpoint{4.620000in}{4.620000in}}%
\pgfusepath{clip}%
\pgfsetbuttcap%
\pgfsetroundjoin%
\pgfsetlinewidth{1.204500pt}%
\definecolor{currentstroke}{rgb}{0.827451,0.827451,0.827451}%
\pgfsetstrokecolor{currentstroke}%
\pgfsetdash{{1.200000pt}{1.980000pt}}{0.000000pt}%
\pgfpathmoveto{\pgfqpoint{2.654247in}{3.328328in}}%
\pgfpathlineto{\pgfqpoint{0.755000in}{4.424859in}}%
\pgfusepath{stroke}%
\end{pgfscope}%
\begin{pgfscope}%
\pgfpathrectangle{\pgfqpoint{0.765000in}{0.660000in}}{\pgfqpoint{4.620000in}{4.620000in}}%
\pgfusepath{clip}%
\pgfsetrectcap%
\pgfsetroundjoin%
\pgfsetlinewidth{1.204500pt}%
\definecolor{currentstroke}{rgb}{0.000000,0.000000,0.000000}%
\pgfsetstrokecolor{currentstroke}%
\pgfsetdash{}{0pt}%
\pgfpathmoveto{\pgfqpoint{2.654247in}{3.229621in}}%
\pgfusepath{stroke}%
\end{pgfscope}%
\begin{pgfscope}%
\pgfpathrectangle{\pgfqpoint{0.765000in}{0.660000in}}{\pgfqpoint{4.620000in}{4.620000in}}%
\pgfusepath{clip}%
\pgfsetbuttcap%
\pgfsetroundjoin%
\definecolor{currentfill}{rgb}{0.000000,0.000000,0.000000}%
\pgfsetfillcolor{currentfill}%
\pgfsetlinewidth{1.003750pt}%
\definecolor{currentstroke}{rgb}{0.000000,0.000000,0.000000}%
\pgfsetstrokecolor{currentstroke}%
\pgfsetdash{}{0pt}%
\pgfsys@defobject{currentmarker}{\pgfqpoint{-0.016667in}{-0.016667in}}{\pgfqpoint{0.016667in}{0.016667in}}{%
\pgfpathmoveto{\pgfqpoint{0.000000in}{-0.016667in}}%
\pgfpathcurveto{\pgfqpoint{0.004420in}{-0.016667in}}{\pgfqpoint{0.008660in}{-0.014911in}}{\pgfqpoint{0.011785in}{-0.011785in}}%
\pgfpathcurveto{\pgfqpoint{0.014911in}{-0.008660in}}{\pgfqpoint{0.016667in}{-0.004420in}}{\pgfqpoint{0.016667in}{0.000000in}}%
\pgfpathcurveto{\pgfqpoint{0.016667in}{0.004420in}}{\pgfqpoint{0.014911in}{0.008660in}}{\pgfqpoint{0.011785in}{0.011785in}}%
\pgfpathcurveto{\pgfqpoint{0.008660in}{0.014911in}}{\pgfqpoint{0.004420in}{0.016667in}}{\pgfqpoint{0.000000in}{0.016667in}}%
\pgfpathcurveto{\pgfqpoint{-0.004420in}{0.016667in}}{\pgfqpoint{-0.008660in}{0.014911in}}{\pgfqpoint{-0.011785in}{0.011785in}}%
\pgfpathcurveto{\pgfqpoint{-0.014911in}{0.008660in}}{\pgfqpoint{-0.016667in}{0.004420in}}{\pgfqpoint{-0.016667in}{0.000000in}}%
\pgfpathcurveto{\pgfqpoint{-0.016667in}{-0.004420in}}{\pgfqpoint{-0.014911in}{-0.008660in}}{\pgfqpoint{-0.011785in}{-0.011785in}}%
\pgfpathcurveto{\pgfqpoint{-0.008660in}{-0.014911in}}{\pgfqpoint{-0.004420in}{-0.016667in}}{\pgfqpoint{0.000000in}{-0.016667in}}%
\pgfpathlineto{\pgfqpoint{0.000000in}{-0.016667in}}%
\pgfpathclose%
\pgfusepath{stroke,fill}%
}%
\begin{pgfscope}%
\pgfsys@transformshift{2.654247in}{3.229621in}%
\pgfsys@useobject{currentmarker}{}%
\end{pgfscope}%
\end{pgfscope}%
\begin{pgfscope}%
\pgfpathrectangle{\pgfqpoint{0.765000in}{0.660000in}}{\pgfqpoint{4.620000in}{4.620000in}}%
\pgfusepath{clip}%
\pgfsetrectcap%
\pgfsetroundjoin%
\pgfsetlinewidth{1.204500pt}%
\definecolor{currentstroke}{rgb}{0.000000,0.000000,0.000000}%
\pgfsetstrokecolor{currentstroke}%
\pgfsetdash{}{0pt}%
\pgfpathmoveto{\pgfqpoint{2.654247in}{3.229621in}}%
\pgfpathlineto{\pgfqpoint{3.025552in}{2.586502in}}%
\pgfusepath{stroke}%
\end{pgfscope}%
\begin{pgfscope}%
\pgfpathrectangle{\pgfqpoint{0.765000in}{0.660000in}}{\pgfqpoint{4.620000in}{4.620000in}}%
\pgfusepath{clip}%
\pgfsetrectcap%
\pgfsetroundjoin%
\pgfsetlinewidth{1.204500pt}%
\definecolor{currentstroke}{rgb}{0.000000,0.000000,0.000000}%
\pgfsetstrokecolor{currentstroke}%
\pgfsetdash{}{0pt}%
\pgfpathmoveto{\pgfqpoint{2.654247in}{3.229621in}}%
\pgfpathlineto{\pgfqpoint{2.654247in}{3.328328in}}%
\pgfusepath{stroke}%
\end{pgfscope}%
\begin{pgfscope}%
\pgfpathrectangle{\pgfqpoint{0.765000in}{0.660000in}}{\pgfqpoint{4.620000in}{4.620000in}}%
\pgfusepath{clip}%
\pgfsetbuttcap%
\pgfsetroundjoin%
\pgfsetlinewidth{1.204500pt}%
\definecolor{currentstroke}{rgb}{0.827451,0.827451,0.827451}%
\pgfsetstrokecolor{currentstroke}%
\pgfsetdash{{1.200000pt}{1.980000pt}}{0.000000pt}%
\pgfpathmoveto{\pgfqpoint{2.654247in}{3.229621in}}%
\pgfpathlineto{\pgfqpoint{0.755000in}{4.326152in}}%
\pgfusepath{stroke}%
\end{pgfscope}%
\begin{pgfscope}%
\pgfpathrectangle{\pgfqpoint{0.765000in}{0.660000in}}{\pgfqpoint{4.620000in}{4.620000in}}%
\pgfusepath{clip}%
\pgfsetrectcap%
\pgfsetroundjoin%
\pgfsetlinewidth{1.204500pt}%
\definecolor{currentstroke}{rgb}{0.000000,0.000000,0.000000}%
\pgfsetstrokecolor{currentstroke}%
\pgfsetdash{}{0pt}%
\pgfpathmoveto{\pgfqpoint{3.593460in}{3.066768in}}%
\pgfusepath{stroke}%
\end{pgfscope}%
\begin{pgfscope}%
\pgfpathrectangle{\pgfqpoint{0.765000in}{0.660000in}}{\pgfqpoint{4.620000in}{4.620000in}}%
\pgfusepath{clip}%
\pgfsetbuttcap%
\pgfsetroundjoin%
\definecolor{currentfill}{rgb}{0.000000,0.000000,0.000000}%
\pgfsetfillcolor{currentfill}%
\pgfsetlinewidth{1.003750pt}%
\definecolor{currentstroke}{rgb}{0.000000,0.000000,0.000000}%
\pgfsetstrokecolor{currentstroke}%
\pgfsetdash{}{0pt}%
\pgfsys@defobject{currentmarker}{\pgfqpoint{-0.016667in}{-0.016667in}}{\pgfqpoint{0.016667in}{0.016667in}}{%
\pgfpathmoveto{\pgfqpoint{0.000000in}{-0.016667in}}%
\pgfpathcurveto{\pgfqpoint{0.004420in}{-0.016667in}}{\pgfqpoint{0.008660in}{-0.014911in}}{\pgfqpoint{0.011785in}{-0.011785in}}%
\pgfpathcurveto{\pgfqpoint{0.014911in}{-0.008660in}}{\pgfqpoint{0.016667in}{-0.004420in}}{\pgfqpoint{0.016667in}{0.000000in}}%
\pgfpathcurveto{\pgfqpoint{0.016667in}{0.004420in}}{\pgfqpoint{0.014911in}{0.008660in}}{\pgfqpoint{0.011785in}{0.011785in}}%
\pgfpathcurveto{\pgfqpoint{0.008660in}{0.014911in}}{\pgfqpoint{0.004420in}{0.016667in}}{\pgfqpoint{0.000000in}{0.016667in}}%
\pgfpathcurveto{\pgfqpoint{-0.004420in}{0.016667in}}{\pgfqpoint{-0.008660in}{0.014911in}}{\pgfqpoint{-0.011785in}{0.011785in}}%
\pgfpathcurveto{\pgfqpoint{-0.014911in}{0.008660in}}{\pgfqpoint{-0.016667in}{0.004420in}}{\pgfqpoint{-0.016667in}{0.000000in}}%
\pgfpathcurveto{\pgfqpoint{-0.016667in}{-0.004420in}}{\pgfqpoint{-0.014911in}{-0.008660in}}{\pgfqpoint{-0.011785in}{-0.011785in}}%
\pgfpathcurveto{\pgfqpoint{-0.008660in}{-0.014911in}}{\pgfqpoint{-0.004420in}{-0.016667in}}{\pgfqpoint{0.000000in}{-0.016667in}}%
\pgfpathlineto{\pgfqpoint{0.000000in}{-0.016667in}}%
\pgfpathclose%
\pgfusepath{stroke,fill}%
}%
\begin{pgfscope}%
\pgfsys@transformshift{3.593460in}{3.066768in}%
\pgfsys@useobject{currentmarker}{}%
\end{pgfscope}%
\end{pgfscope}%
\begin{pgfscope}%
\pgfpathrectangle{\pgfqpoint{0.765000in}{0.660000in}}{\pgfqpoint{4.620000in}{4.620000in}}%
\pgfusepath{clip}%
\pgfsetrectcap%
\pgfsetroundjoin%
\pgfsetlinewidth{1.204500pt}%
\definecolor{currentstroke}{rgb}{0.000000,0.000000,0.000000}%
\pgfsetstrokecolor{currentstroke}%
\pgfsetdash{}{0pt}%
\pgfpathmoveto{\pgfqpoint{3.593460in}{3.066768in}}%
\pgfpathlineto{\pgfqpoint{3.593460in}{3.452500in}}%
\pgfusepath{stroke}%
\end{pgfscope}%
\begin{pgfscope}%
\pgfpathrectangle{\pgfqpoint{0.765000in}{0.660000in}}{\pgfqpoint{4.620000in}{4.620000in}}%
\pgfusepath{clip}%
\pgfsetrectcap%
\pgfsetroundjoin%
\pgfsetlinewidth{1.204500pt}%
\definecolor{currentstroke}{rgb}{0.000000,0.000000,0.000000}%
\pgfsetstrokecolor{currentstroke}%
\pgfsetdash{}{0pt}%
\pgfpathmoveto{\pgfqpoint{3.593460in}{3.066768in}}%
\pgfpathlineto{\pgfqpoint{3.337483in}{2.623404in}}%
\pgfusepath{stroke}%
\end{pgfscope}%
\begin{pgfscope}%
\pgfpathrectangle{\pgfqpoint{0.765000in}{0.660000in}}{\pgfqpoint{4.620000in}{4.620000in}}%
\pgfusepath{clip}%
\pgfsetbuttcap%
\pgfsetroundjoin%
\pgfsetlinewidth{1.204500pt}%
\definecolor{currentstroke}{rgb}{0.827451,0.827451,0.827451}%
\pgfsetstrokecolor{currentstroke}%
\pgfsetdash{{1.200000pt}{1.980000pt}}{0.000000pt}%
\pgfpathmoveto{\pgfqpoint{3.593460in}{3.066768in}}%
\pgfpathlineto{\pgfqpoint{5.395000in}{4.106888in}}%
\pgfusepath{stroke}%
\end{pgfscope}%
\begin{pgfscope}%
\pgfpathrectangle{\pgfqpoint{0.765000in}{0.660000in}}{\pgfqpoint{4.620000in}{4.620000in}}%
\pgfusepath{clip}%
\pgfsetrectcap%
\pgfsetroundjoin%
\pgfsetlinewidth{1.204500pt}%
\definecolor{currentstroke}{rgb}{0.000000,0.000000,0.000000}%
\pgfsetstrokecolor{currentstroke}%
\pgfsetdash{}{0pt}%
\pgfpathmoveto{\pgfqpoint{3.593460in}{3.452500in}}%
\pgfusepath{stroke}%
\end{pgfscope}%
\begin{pgfscope}%
\pgfpathrectangle{\pgfqpoint{0.765000in}{0.660000in}}{\pgfqpoint{4.620000in}{4.620000in}}%
\pgfusepath{clip}%
\pgfsetbuttcap%
\pgfsetroundjoin%
\definecolor{currentfill}{rgb}{0.000000,0.000000,0.000000}%
\pgfsetfillcolor{currentfill}%
\pgfsetlinewidth{1.003750pt}%
\definecolor{currentstroke}{rgb}{0.000000,0.000000,0.000000}%
\pgfsetstrokecolor{currentstroke}%
\pgfsetdash{}{0pt}%
\pgfsys@defobject{currentmarker}{\pgfqpoint{-0.016667in}{-0.016667in}}{\pgfqpoint{0.016667in}{0.016667in}}{%
\pgfpathmoveto{\pgfqpoint{0.000000in}{-0.016667in}}%
\pgfpathcurveto{\pgfqpoint{0.004420in}{-0.016667in}}{\pgfqpoint{0.008660in}{-0.014911in}}{\pgfqpoint{0.011785in}{-0.011785in}}%
\pgfpathcurveto{\pgfqpoint{0.014911in}{-0.008660in}}{\pgfqpoint{0.016667in}{-0.004420in}}{\pgfqpoint{0.016667in}{0.000000in}}%
\pgfpathcurveto{\pgfqpoint{0.016667in}{0.004420in}}{\pgfqpoint{0.014911in}{0.008660in}}{\pgfqpoint{0.011785in}{0.011785in}}%
\pgfpathcurveto{\pgfqpoint{0.008660in}{0.014911in}}{\pgfqpoint{0.004420in}{0.016667in}}{\pgfqpoint{0.000000in}{0.016667in}}%
\pgfpathcurveto{\pgfqpoint{-0.004420in}{0.016667in}}{\pgfqpoint{-0.008660in}{0.014911in}}{\pgfqpoint{-0.011785in}{0.011785in}}%
\pgfpathcurveto{\pgfqpoint{-0.014911in}{0.008660in}}{\pgfqpoint{-0.016667in}{0.004420in}}{\pgfqpoint{-0.016667in}{0.000000in}}%
\pgfpathcurveto{\pgfqpoint{-0.016667in}{-0.004420in}}{\pgfqpoint{-0.014911in}{-0.008660in}}{\pgfqpoint{-0.011785in}{-0.011785in}}%
\pgfpathcurveto{\pgfqpoint{-0.008660in}{-0.014911in}}{\pgfqpoint{-0.004420in}{-0.016667in}}{\pgfqpoint{0.000000in}{-0.016667in}}%
\pgfpathlineto{\pgfqpoint{0.000000in}{-0.016667in}}%
\pgfpathclose%
\pgfusepath{stroke,fill}%
}%
\begin{pgfscope}%
\pgfsys@transformshift{3.593460in}{3.452500in}%
\pgfsys@useobject{currentmarker}{}%
\end{pgfscope}%
\end{pgfscope}%
\begin{pgfscope}%
\pgfpathrectangle{\pgfqpoint{0.765000in}{0.660000in}}{\pgfqpoint{4.620000in}{4.620000in}}%
\pgfusepath{clip}%
\pgfsetrectcap%
\pgfsetroundjoin%
\pgfsetlinewidth{1.204500pt}%
\definecolor{currentstroke}{rgb}{0.000000,0.000000,0.000000}%
\pgfsetstrokecolor{currentstroke}%
\pgfsetdash{}{0pt}%
\pgfpathmoveto{\pgfqpoint{3.593460in}{3.452500in}}%
\pgfpathlineto{\pgfqpoint{3.593460in}{3.066768in}}%
\pgfusepath{stroke}%
\end{pgfscope}%
\begin{pgfscope}%
\pgfpathrectangle{\pgfqpoint{0.765000in}{0.660000in}}{\pgfqpoint{4.620000in}{4.620000in}}%
\pgfusepath{clip}%
\pgfsetrectcap%
\pgfsetroundjoin%
\pgfsetlinewidth{1.204500pt}%
\definecolor{currentstroke}{rgb}{0.000000,0.000000,0.000000}%
\pgfsetstrokecolor{currentstroke}%
\pgfsetdash{}{0pt}%
\pgfpathmoveto{\pgfqpoint{3.593460in}{3.452500in}}%
\pgfpathlineto{\pgfqpoint{2.869318in}{3.452500in}}%
\pgfusepath{stroke}%
\end{pgfscope}%
\begin{pgfscope}%
\pgfpathrectangle{\pgfqpoint{0.765000in}{0.660000in}}{\pgfqpoint{4.620000in}{4.620000in}}%
\pgfusepath{clip}%
\pgfsetbuttcap%
\pgfsetroundjoin%
\pgfsetlinewidth{1.204500pt}%
\definecolor{currentstroke}{rgb}{0.827451,0.827451,0.827451}%
\pgfsetstrokecolor{currentstroke}%
\pgfsetdash{{1.200000pt}{1.980000pt}}{0.000000pt}%
\pgfpathmoveto{\pgfqpoint{3.593460in}{3.452500in}}%
\pgfpathlineto{\pgfqpoint{5.395000in}{4.492619in}}%
\pgfusepath{stroke}%
\end{pgfscope}%
\begin{pgfscope}%
\pgfpathrectangle{\pgfqpoint{0.765000in}{0.660000in}}{\pgfqpoint{4.620000in}{4.620000in}}%
\pgfusepath{clip}%
\pgfsetrectcap%
\pgfsetroundjoin%
\pgfsetlinewidth{1.204500pt}%
\definecolor{currentstroke}{rgb}{0.000000,0.000000,0.000000}%
\pgfsetstrokecolor{currentstroke}%
\pgfsetdash{}{0pt}%
\pgfpathmoveto{\pgfqpoint{3.337483in}{2.623404in}}%
\pgfusepath{stroke}%
\end{pgfscope}%
\begin{pgfscope}%
\pgfpathrectangle{\pgfqpoint{0.765000in}{0.660000in}}{\pgfqpoint{4.620000in}{4.620000in}}%
\pgfusepath{clip}%
\pgfsetbuttcap%
\pgfsetroundjoin%
\definecolor{currentfill}{rgb}{0.000000,0.000000,0.000000}%
\pgfsetfillcolor{currentfill}%
\pgfsetlinewidth{1.003750pt}%
\definecolor{currentstroke}{rgb}{0.000000,0.000000,0.000000}%
\pgfsetstrokecolor{currentstroke}%
\pgfsetdash{}{0pt}%
\pgfsys@defobject{currentmarker}{\pgfqpoint{-0.016667in}{-0.016667in}}{\pgfqpoint{0.016667in}{0.016667in}}{%
\pgfpathmoveto{\pgfqpoint{0.000000in}{-0.016667in}}%
\pgfpathcurveto{\pgfqpoint{0.004420in}{-0.016667in}}{\pgfqpoint{0.008660in}{-0.014911in}}{\pgfqpoint{0.011785in}{-0.011785in}}%
\pgfpathcurveto{\pgfqpoint{0.014911in}{-0.008660in}}{\pgfqpoint{0.016667in}{-0.004420in}}{\pgfqpoint{0.016667in}{0.000000in}}%
\pgfpathcurveto{\pgfqpoint{0.016667in}{0.004420in}}{\pgfqpoint{0.014911in}{0.008660in}}{\pgfqpoint{0.011785in}{0.011785in}}%
\pgfpathcurveto{\pgfqpoint{0.008660in}{0.014911in}}{\pgfqpoint{0.004420in}{0.016667in}}{\pgfqpoint{0.000000in}{0.016667in}}%
\pgfpathcurveto{\pgfqpoint{-0.004420in}{0.016667in}}{\pgfqpoint{-0.008660in}{0.014911in}}{\pgfqpoint{-0.011785in}{0.011785in}}%
\pgfpathcurveto{\pgfqpoint{-0.014911in}{0.008660in}}{\pgfqpoint{-0.016667in}{0.004420in}}{\pgfqpoint{-0.016667in}{0.000000in}}%
\pgfpathcurveto{\pgfqpoint{-0.016667in}{-0.004420in}}{\pgfqpoint{-0.014911in}{-0.008660in}}{\pgfqpoint{-0.011785in}{-0.011785in}}%
\pgfpathcurveto{\pgfqpoint{-0.008660in}{-0.014911in}}{\pgfqpoint{-0.004420in}{-0.016667in}}{\pgfqpoint{0.000000in}{-0.016667in}}%
\pgfpathlineto{\pgfqpoint{0.000000in}{-0.016667in}}%
\pgfpathclose%
\pgfusepath{stroke,fill}%
}%
\begin{pgfscope}%
\pgfsys@transformshift{3.337483in}{2.623404in}%
\pgfsys@useobject{currentmarker}{}%
\end{pgfscope}%
\end{pgfscope}%
\begin{pgfscope}%
\pgfpathrectangle{\pgfqpoint{0.765000in}{0.660000in}}{\pgfqpoint{4.620000in}{4.620000in}}%
\pgfusepath{clip}%
\pgfsetrectcap%
\pgfsetroundjoin%
\pgfsetlinewidth{1.204500pt}%
\definecolor{currentstroke}{rgb}{0.000000,0.000000,0.000000}%
\pgfsetstrokecolor{currentstroke}%
\pgfsetdash{}{0pt}%
\pgfpathmoveto{\pgfqpoint{3.337483in}{2.623404in}}%
\pgfpathlineto{\pgfqpoint{3.149560in}{2.514906in}}%
\pgfusepath{stroke}%
\end{pgfscope}%
\begin{pgfscope}%
\pgfpathrectangle{\pgfqpoint{0.765000in}{0.660000in}}{\pgfqpoint{4.620000in}{4.620000in}}%
\pgfusepath{clip}%
\pgfsetrectcap%
\pgfsetroundjoin%
\pgfsetlinewidth{1.204500pt}%
\definecolor{currentstroke}{rgb}{0.000000,0.000000,0.000000}%
\pgfsetstrokecolor{currentstroke}%
\pgfsetdash{}{0pt}%
\pgfpathmoveto{\pgfqpoint{3.337483in}{2.623404in}}%
\pgfpathlineto{\pgfqpoint{3.593460in}{3.066768in}}%
\pgfusepath{stroke}%
\end{pgfscope}%
\begin{pgfscope}%
\pgfpathrectangle{\pgfqpoint{0.765000in}{0.660000in}}{\pgfqpoint{4.620000in}{4.620000in}}%
\pgfusepath{clip}%
\pgfsetbuttcap%
\pgfsetroundjoin%
\pgfsetlinewidth{1.204500pt}%
\definecolor{currentstroke}{rgb}{0.827451,0.827451,0.827451}%
\pgfsetstrokecolor{currentstroke}%
\pgfsetdash{{1.200000pt}{1.980000pt}}{0.000000pt}%
\pgfpathmoveto{\pgfqpoint{3.337483in}{2.623404in}}%
\pgfpathlineto{\pgfqpoint{3.337483in}{0.650000in}}%
\pgfusepath{stroke}%
\end{pgfscope}%
\begin{pgfscope}%
\pgfpathrectangle{\pgfqpoint{0.765000in}{0.660000in}}{\pgfqpoint{4.620000in}{4.620000in}}%
\pgfusepath{clip}%
\pgfsetrectcap%
\pgfsetroundjoin%
\pgfsetlinewidth{1.204500pt}%
\definecolor{currentstroke}{rgb}{0.000000,0.000000,0.000000}%
\pgfsetstrokecolor{currentstroke}%
\pgfsetdash{}{0pt}%
\pgfpathmoveto{\pgfqpoint{2.869318in}{3.452500in}}%
\pgfusepath{stroke}%
\end{pgfscope}%
\begin{pgfscope}%
\pgfpathrectangle{\pgfqpoint{0.765000in}{0.660000in}}{\pgfqpoint{4.620000in}{4.620000in}}%
\pgfusepath{clip}%
\pgfsetbuttcap%
\pgfsetroundjoin%
\definecolor{currentfill}{rgb}{0.000000,0.000000,0.000000}%
\pgfsetfillcolor{currentfill}%
\pgfsetlinewidth{1.003750pt}%
\definecolor{currentstroke}{rgb}{0.000000,0.000000,0.000000}%
\pgfsetstrokecolor{currentstroke}%
\pgfsetdash{}{0pt}%
\pgfsys@defobject{currentmarker}{\pgfqpoint{-0.016667in}{-0.016667in}}{\pgfqpoint{0.016667in}{0.016667in}}{%
\pgfpathmoveto{\pgfqpoint{0.000000in}{-0.016667in}}%
\pgfpathcurveto{\pgfqpoint{0.004420in}{-0.016667in}}{\pgfqpoint{0.008660in}{-0.014911in}}{\pgfqpoint{0.011785in}{-0.011785in}}%
\pgfpathcurveto{\pgfqpoint{0.014911in}{-0.008660in}}{\pgfqpoint{0.016667in}{-0.004420in}}{\pgfqpoint{0.016667in}{0.000000in}}%
\pgfpathcurveto{\pgfqpoint{0.016667in}{0.004420in}}{\pgfqpoint{0.014911in}{0.008660in}}{\pgfqpoint{0.011785in}{0.011785in}}%
\pgfpathcurveto{\pgfqpoint{0.008660in}{0.014911in}}{\pgfqpoint{0.004420in}{0.016667in}}{\pgfqpoint{0.000000in}{0.016667in}}%
\pgfpathcurveto{\pgfqpoint{-0.004420in}{0.016667in}}{\pgfqpoint{-0.008660in}{0.014911in}}{\pgfqpoint{-0.011785in}{0.011785in}}%
\pgfpathcurveto{\pgfqpoint{-0.014911in}{0.008660in}}{\pgfqpoint{-0.016667in}{0.004420in}}{\pgfqpoint{-0.016667in}{0.000000in}}%
\pgfpathcurveto{\pgfqpoint{-0.016667in}{-0.004420in}}{\pgfqpoint{-0.014911in}{-0.008660in}}{\pgfqpoint{-0.011785in}{-0.011785in}}%
\pgfpathcurveto{\pgfqpoint{-0.008660in}{-0.014911in}}{\pgfqpoint{-0.004420in}{-0.016667in}}{\pgfqpoint{0.000000in}{-0.016667in}}%
\pgfpathlineto{\pgfqpoint{0.000000in}{-0.016667in}}%
\pgfpathclose%
\pgfusepath{stroke,fill}%
}%
\begin{pgfscope}%
\pgfsys@transformshift{2.869318in}{3.452500in}%
\pgfsys@useobject{currentmarker}{}%
\end{pgfscope}%
\end{pgfscope}%
\begin{pgfscope}%
\pgfpathrectangle{\pgfqpoint{0.765000in}{0.660000in}}{\pgfqpoint{4.620000in}{4.620000in}}%
\pgfusepath{clip}%
\pgfsetrectcap%
\pgfsetroundjoin%
\pgfsetlinewidth{1.204500pt}%
\definecolor{currentstroke}{rgb}{0.000000,0.000000,0.000000}%
\pgfsetstrokecolor{currentstroke}%
\pgfsetdash{}{0pt}%
\pgfpathmoveto{\pgfqpoint{2.869318in}{3.452500in}}%
\pgfpathlineto{\pgfqpoint{2.654247in}{3.328328in}}%
\pgfusepath{stroke}%
\end{pgfscope}%
\begin{pgfscope}%
\pgfpathrectangle{\pgfqpoint{0.765000in}{0.660000in}}{\pgfqpoint{4.620000in}{4.620000in}}%
\pgfusepath{clip}%
\pgfsetrectcap%
\pgfsetroundjoin%
\pgfsetlinewidth{1.204500pt}%
\definecolor{currentstroke}{rgb}{0.000000,0.000000,0.000000}%
\pgfsetstrokecolor{currentstroke}%
\pgfsetdash{}{0pt}%
\pgfpathmoveto{\pgfqpoint{2.869318in}{3.452500in}}%
\pgfpathlineto{\pgfqpoint{3.593460in}{3.452500in}}%
\pgfusepath{stroke}%
\end{pgfscope}%
\begin{pgfscope}%
\pgfpathrectangle{\pgfqpoint{0.765000in}{0.660000in}}{\pgfqpoint{4.620000in}{4.620000in}}%
\pgfusepath{clip}%
\pgfsetbuttcap%
\pgfsetroundjoin%
\pgfsetlinewidth{1.204500pt}%
\definecolor{currentstroke}{rgb}{0.827451,0.827451,0.827451}%
\pgfsetstrokecolor{currentstroke}%
\pgfsetdash{{1.200000pt}{1.980000pt}}{0.000000pt}%
\pgfpathmoveto{\pgfqpoint{2.869318in}{3.452500in}}%
\pgfpathlineto{\pgfqpoint{0.755000in}{4.673202in}}%
\pgfusepath{stroke}%
\end{pgfscope}%
\end{pgfpicture}%
\makeatother%
\endgroup%
}
        }
        \caption{Toy binary dataset, cubic}
        \label{fig:circular}
    \end{subfigure}
    \hfill
    \centering
    \begin{subfigure}{0.35\textwidth}
        \centering
        \resizebox{\linewidth}{!}{%
        \centering
            \clipbox{0.2\width{} 0.2\height{} 0.2\width{} 0.2\height{}}{%% Creator: Matplotlib, PGF backend
%%
%% To include the figure in your LaTeX document, write
%%   \input{<filename>.pgf}
%%
%% Make sure the required packages are loaded in your preamble
%%   \usepackage{pgf}
%%
%% Also ensure that all the required font packages are loaded; for instance,
%% the lmodern package is sometimes necessary when using math font.
%%   \usepackage{lmodern}
%%
%% Figures using additional raster images can only be included by \input if
%% they are in the same directory as the main LaTeX file. For loading figures
%% from other directories you can use the `import` package
%%   \usepackage{import}
%%
%% and then include the figures with
%%   \import{<path to file>}{<filename>.pgf}
%%
%% Matplotlib used the following preamble
%%   
%%   \usepackage{fontspec}
%%   \setmainfont{DejaVuSerif.ttf}[Path=\detokenize{/Users/sam/Library/Python/3.9/lib/python/site-packages/matplotlib/mpl-data/fonts/ttf/}]
%%   \setsansfont{Arial.ttf}[Path=\detokenize{/System/Library/Fonts/Supplemental/}]
%%   \setmonofont{DejaVuSansMono.ttf}[Path=\detokenize{/Users/sam/Library/Python/3.9/lib/python/site-packages/matplotlib/mpl-data/fonts/ttf/}]
%%   \makeatletter\@ifpackageloaded{underscore}{}{\usepackage[strings]{underscore}}\makeatother
%%
\begingroup%
\makeatletter%
\begin{pgfpicture}%
\pgfpathrectangle{\pgfpointorigin}{\pgfqpoint{6.000000in}{6.000000in}}%
\pgfusepath{use as bounding box, clip}%
\begin{pgfscope}%
\pgfsetbuttcap%
\pgfsetmiterjoin%
\definecolor{currentfill}{rgb}{1.000000,1.000000,1.000000}%
\pgfsetfillcolor{currentfill}%
\pgfsetlinewidth{0.000000pt}%
\definecolor{currentstroke}{rgb}{1.000000,1.000000,1.000000}%
\pgfsetstrokecolor{currentstroke}%
\pgfsetdash{}{0pt}%
\pgfpathmoveto{\pgfqpoint{0.000000in}{0.000000in}}%
\pgfpathlineto{\pgfqpoint{6.000000in}{0.000000in}}%
\pgfpathlineto{\pgfqpoint{6.000000in}{6.000000in}}%
\pgfpathlineto{\pgfqpoint{0.000000in}{6.000000in}}%
\pgfpathlineto{\pgfqpoint{0.000000in}{0.000000in}}%
\pgfpathclose%
\pgfusepath{fill}%
\end{pgfscope}%
\begin{pgfscope}%
\pgfsetbuttcap%
\pgfsetmiterjoin%
\definecolor{currentfill}{rgb}{1.000000,1.000000,1.000000}%
\pgfsetfillcolor{currentfill}%
\pgfsetlinewidth{0.000000pt}%
\definecolor{currentstroke}{rgb}{0.000000,0.000000,0.000000}%
\pgfsetstrokecolor{currentstroke}%
\pgfsetstrokeopacity{0.000000}%
\pgfsetdash{}{0pt}%
\pgfpathmoveto{\pgfqpoint{0.765000in}{0.660000in}}%
\pgfpathlineto{\pgfqpoint{5.385000in}{0.660000in}}%
\pgfpathlineto{\pgfqpoint{5.385000in}{5.280000in}}%
\pgfpathlineto{\pgfqpoint{0.765000in}{5.280000in}}%
\pgfpathlineto{\pgfqpoint{0.765000in}{0.660000in}}%
\pgfpathclose%
\pgfusepath{fill}%
\end{pgfscope}%
\begin{pgfscope}%
\pgfpathrectangle{\pgfqpoint{0.765000in}{0.660000in}}{\pgfqpoint{4.620000in}{4.620000in}}%
\pgfusepath{clip}%
\pgfsetbuttcap%
\pgfsetroundjoin%
\definecolor{currentfill}{rgb}{1.000000,0.576471,0.309804}%
\pgfsetfillcolor{currentfill}%
\pgfsetfillopacity{0.050000}%
\pgfsetlinewidth{0.000000pt}%
\definecolor{currentstroke}{rgb}{1.000000,0.576471,0.309804}%
\pgfsetstrokecolor{currentstroke}%
\pgfsetdash{}{0pt}%
\pgfpathmoveto{\pgfqpoint{4.575692in}{3.931540in}}%
\pgfpathlineto{\pgfqpoint{4.575692in}{4.037129in}}%
\pgfpathlineto{\pgfqpoint{4.688439in}{3.984197in}}%
\pgfpathlineto{\pgfqpoint{4.688439in}{3.878608in}}%
\pgfpathlineto{\pgfqpoint{4.575692in}{3.931540in}}%
\pgfpathclose%
\pgfusepath{fill}%
\end{pgfscope}%
\begin{pgfscope}%
\pgfpathrectangle{\pgfqpoint{0.765000in}{0.660000in}}{\pgfqpoint{4.620000in}{4.620000in}}%
\pgfusepath{clip}%
\pgfsetbuttcap%
\pgfsetroundjoin%
\definecolor{currentfill}{rgb}{1.000000,0.576471,0.309804}%
\pgfsetfillcolor{currentfill}%
\pgfsetfillopacity{0.050000}%
\pgfsetlinewidth{0.000000pt}%
\definecolor{currentstroke}{rgb}{1.000000,0.576471,0.309804}%
\pgfsetstrokecolor{currentstroke}%
\pgfsetdash{}{0pt}%
\pgfpathmoveto{\pgfqpoint{4.575692in}{3.931540in}}%
\pgfpathlineto{\pgfqpoint{4.575692in}{4.037129in}}%
\pgfpathlineto{\pgfqpoint{4.462944in}{3.984197in}}%
\pgfpathlineto{\pgfqpoint{4.462944in}{3.878608in}}%
\pgfpathlineto{\pgfqpoint{4.575692in}{3.931540in}}%
\pgfpathclose%
\pgfusepath{fill}%
\end{pgfscope}%
\begin{pgfscope}%
\pgfpathrectangle{\pgfqpoint{0.765000in}{0.660000in}}{\pgfqpoint{4.620000in}{4.620000in}}%
\pgfusepath{clip}%
\pgfsetbuttcap%
\pgfsetroundjoin%
\definecolor{currentfill}{rgb}{1.000000,0.576471,0.309804}%
\pgfsetfillcolor{currentfill}%
\pgfsetfillopacity{0.050000}%
\pgfsetlinewidth{0.000000pt}%
\definecolor{currentstroke}{rgb}{1.000000,0.576471,0.309804}%
\pgfsetstrokecolor{currentstroke}%
\pgfsetdash{}{0pt}%
\pgfpathmoveto{\pgfqpoint{4.575692in}{3.931540in}}%
\pgfpathlineto{\pgfqpoint{4.688439in}{3.878608in}}%
\pgfpathlineto{\pgfqpoint{4.575692in}{3.825677in}}%
\pgfpathlineto{\pgfqpoint{4.462944in}{3.878608in}}%
\pgfpathlineto{\pgfqpoint{4.575692in}{3.931540in}}%
\pgfpathclose%
\pgfusepath{fill}%
\end{pgfscope}%
\begin{pgfscope}%
\pgfpathrectangle{\pgfqpoint{0.765000in}{0.660000in}}{\pgfqpoint{4.620000in}{4.620000in}}%
\pgfusepath{clip}%
\pgfsetbuttcap%
\pgfsetroundjoin%
\definecolor{currentfill}{rgb}{1.000000,0.576471,0.309804}%
\pgfsetfillcolor{currentfill}%
\pgfsetfillopacity{0.050000}%
\pgfsetlinewidth{0.000000pt}%
\definecolor{currentstroke}{rgb}{1.000000,0.576471,0.309804}%
\pgfsetstrokecolor{currentstroke}%
\pgfsetdash{}{0pt}%
\pgfpathmoveto{\pgfqpoint{4.575692in}{4.037129in}}%
\pgfpathlineto{\pgfqpoint{4.688439in}{3.984197in}}%
\pgfpathlineto{\pgfqpoint{4.575692in}{3.931265in}}%
\pgfpathlineto{\pgfqpoint{4.462944in}{3.984197in}}%
\pgfpathlineto{\pgfqpoint{4.575692in}{4.037129in}}%
\pgfpathclose%
\pgfusepath{fill}%
\end{pgfscope}%
\begin{pgfscope}%
\pgfpathrectangle{\pgfqpoint{0.765000in}{0.660000in}}{\pgfqpoint{4.620000in}{4.620000in}}%
\pgfusepath{clip}%
\pgfsetbuttcap%
\pgfsetroundjoin%
\definecolor{currentfill}{rgb}{1.000000,0.576471,0.309804}%
\pgfsetfillcolor{currentfill}%
\pgfsetfillopacity{0.050000}%
\pgfsetlinewidth{0.000000pt}%
\definecolor{currentstroke}{rgb}{1.000000,0.576471,0.309804}%
\pgfsetstrokecolor{currentstroke}%
\pgfsetdash{}{0pt}%
\pgfpathmoveto{\pgfqpoint{4.462944in}{3.878608in}}%
\pgfpathlineto{\pgfqpoint{4.462944in}{3.984197in}}%
\pgfpathlineto{\pgfqpoint{4.575692in}{3.931265in}}%
\pgfpathlineto{\pgfqpoint{4.575692in}{3.825677in}}%
\pgfpathlineto{\pgfqpoint{4.462944in}{3.878608in}}%
\pgfpathclose%
\pgfusepath{fill}%
\end{pgfscope}%
\begin{pgfscope}%
\pgfpathrectangle{\pgfqpoint{0.765000in}{0.660000in}}{\pgfqpoint{4.620000in}{4.620000in}}%
\pgfusepath{clip}%
\pgfsetbuttcap%
\pgfsetroundjoin%
\definecolor{currentfill}{rgb}{1.000000,0.576471,0.309804}%
\pgfsetfillcolor{currentfill}%
\pgfsetfillopacity{0.050000}%
\pgfsetlinewidth{0.000000pt}%
\definecolor{currentstroke}{rgb}{1.000000,0.576471,0.309804}%
\pgfsetstrokecolor{currentstroke}%
\pgfsetdash{}{0pt}%
\pgfpathmoveto{\pgfqpoint{4.688439in}{3.878608in}}%
\pgfpathlineto{\pgfqpoint{4.688439in}{3.984197in}}%
\pgfpathlineto{\pgfqpoint{4.575692in}{3.931265in}}%
\pgfpathlineto{\pgfqpoint{4.575692in}{3.825677in}}%
\pgfpathlineto{\pgfqpoint{4.688439in}{3.878608in}}%
\pgfpathclose%
\pgfusepath{fill}%
\end{pgfscope}%
\begin{pgfscope}%
\pgfpathrectangle{\pgfqpoint{0.765000in}{0.660000in}}{\pgfqpoint{4.620000in}{4.620000in}}%
\pgfusepath{clip}%
\pgfsetroundcap%
\pgfsetroundjoin%
\pgfsetlinewidth{1.204500pt}%
\definecolor{currentstroke}{rgb}{1.000000,0.576471,0.309804}%
\pgfsetstrokecolor{currentstroke}%
\pgfsetdash{}{0pt}%
\pgfpathmoveto{\pgfqpoint{4.663634in}{3.772752in}}%
\pgfusepath{stroke}%
\end{pgfscope}%
\begin{pgfscope}%
\pgfpathrectangle{\pgfqpoint{0.765000in}{0.660000in}}{\pgfqpoint{4.620000in}{4.620000in}}%
\pgfusepath{clip}%
\pgfsetbuttcap%
\pgfsetroundjoin%
\definecolor{currentfill}{rgb}{1.000000,0.576471,0.309804}%
\pgfsetfillcolor{currentfill}%
\pgfsetlinewidth{1.003750pt}%
\definecolor{currentstroke}{rgb}{1.000000,0.576471,0.309804}%
\pgfsetstrokecolor{currentstroke}%
\pgfsetdash{}{0pt}%
\pgfsys@defobject{currentmarker}{\pgfqpoint{-0.033333in}{-0.033333in}}{\pgfqpoint{0.033333in}{0.033333in}}{%
\pgfpathmoveto{\pgfqpoint{0.000000in}{-0.033333in}}%
\pgfpathcurveto{\pgfqpoint{0.008840in}{-0.033333in}}{\pgfqpoint{0.017319in}{-0.029821in}}{\pgfqpoint{0.023570in}{-0.023570in}}%
\pgfpathcurveto{\pgfqpoint{0.029821in}{-0.017319in}}{\pgfqpoint{0.033333in}{-0.008840in}}{\pgfqpoint{0.033333in}{0.000000in}}%
\pgfpathcurveto{\pgfqpoint{0.033333in}{0.008840in}}{\pgfqpoint{0.029821in}{0.017319in}}{\pgfqpoint{0.023570in}{0.023570in}}%
\pgfpathcurveto{\pgfqpoint{0.017319in}{0.029821in}}{\pgfqpoint{0.008840in}{0.033333in}}{\pgfqpoint{0.000000in}{0.033333in}}%
\pgfpathcurveto{\pgfqpoint{-0.008840in}{0.033333in}}{\pgfqpoint{-0.017319in}{0.029821in}}{\pgfqpoint{-0.023570in}{0.023570in}}%
\pgfpathcurveto{\pgfqpoint{-0.029821in}{0.017319in}}{\pgfqpoint{-0.033333in}{0.008840in}}{\pgfqpoint{-0.033333in}{0.000000in}}%
\pgfpathcurveto{\pgfqpoint{-0.033333in}{-0.008840in}}{\pgfqpoint{-0.029821in}{-0.017319in}}{\pgfqpoint{-0.023570in}{-0.023570in}}%
\pgfpathcurveto{\pgfqpoint{-0.017319in}{-0.029821in}}{\pgfqpoint{-0.008840in}{-0.033333in}}{\pgfqpoint{0.000000in}{-0.033333in}}%
\pgfpathlineto{\pgfqpoint{0.000000in}{-0.033333in}}%
\pgfpathclose%
\pgfusepath{stroke,fill}%
}%
\begin{pgfscope}%
\pgfsys@transformshift{4.663634in}{3.772752in}%
\pgfsys@useobject{currentmarker}{}%
\end{pgfscope}%
\end{pgfscope}%
\begin{pgfscope}%
\pgfpathrectangle{\pgfqpoint{0.765000in}{0.660000in}}{\pgfqpoint{4.620000in}{4.620000in}}%
\pgfusepath{clip}%
\pgfsetroundcap%
\pgfsetroundjoin%
\pgfsetlinewidth{1.204500pt}%
\definecolor{currentstroke}{rgb}{1.000000,0.576471,0.309804}%
\pgfsetstrokecolor{currentstroke}%
\pgfsetdash{}{0pt}%
\pgfpathmoveto{\pgfqpoint{4.866516in}{3.823379in}}%
\pgfusepath{stroke}%
\end{pgfscope}%
\begin{pgfscope}%
\pgfpathrectangle{\pgfqpoint{0.765000in}{0.660000in}}{\pgfqpoint{4.620000in}{4.620000in}}%
\pgfusepath{clip}%
\pgfsetbuttcap%
\pgfsetroundjoin%
\definecolor{currentfill}{rgb}{1.000000,0.576471,0.309804}%
\pgfsetfillcolor{currentfill}%
\pgfsetlinewidth{1.003750pt}%
\definecolor{currentstroke}{rgb}{1.000000,0.576471,0.309804}%
\pgfsetstrokecolor{currentstroke}%
\pgfsetdash{}{0pt}%
\pgfsys@defobject{currentmarker}{\pgfqpoint{-0.033333in}{-0.033333in}}{\pgfqpoint{0.033333in}{0.033333in}}{%
\pgfpathmoveto{\pgfqpoint{0.000000in}{-0.033333in}}%
\pgfpathcurveto{\pgfqpoint{0.008840in}{-0.033333in}}{\pgfqpoint{0.017319in}{-0.029821in}}{\pgfqpoint{0.023570in}{-0.023570in}}%
\pgfpathcurveto{\pgfqpoint{0.029821in}{-0.017319in}}{\pgfqpoint{0.033333in}{-0.008840in}}{\pgfqpoint{0.033333in}{0.000000in}}%
\pgfpathcurveto{\pgfqpoint{0.033333in}{0.008840in}}{\pgfqpoint{0.029821in}{0.017319in}}{\pgfqpoint{0.023570in}{0.023570in}}%
\pgfpathcurveto{\pgfqpoint{0.017319in}{0.029821in}}{\pgfqpoint{0.008840in}{0.033333in}}{\pgfqpoint{0.000000in}{0.033333in}}%
\pgfpathcurveto{\pgfqpoint{-0.008840in}{0.033333in}}{\pgfqpoint{-0.017319in}{0.029821in}}{\pgfqpoint{-0.023570in}{0.023570in}}%
\pgfpathcurveto{\pgfqpoint{-0.029821in}{0.017319in}}{\pgfqpoint{-0.033333in}{0.008840in}}{\pgfqpoint{-0.033333in}{0.000000in}}%
\pgfpathcurveto{\pgfqpoint{-0.033333in}{-0.008840in}}{\pgfqpoint{-0.029821in}{-0.017319in}}{\pgfqpoint{-0.023570in}{-0.023570in}}%
\pgfpathcurveto{\pgfqpoint{-0.017319in}{-0.029821in}}{\pgfqpoint{-0.008840in}{-0.033333in}}{\pgfqpoint{0.000000in}{-0.033333in}}%
\pgfpathlineto{\pgfqpoint{0.000000in}{-0.033333in}}%
\pgfpathclose%
\pgfusepath{stroke,fill}%
}%
\begin{pgfscope}%
\pgfsys@transformshift{4.866516in}{3.823379in}%
\pgfsys@useobject{currentmarker}{}%
\end{pgfscope}%
\end{pgfscope}%
\begin{pgfscope}%
\pgfpathrectangle{\pgfqpoint{0.765000in}{0.660000in}}{\pgfqpoint{4.620000in}{4.620000in}}%
\pgfusepath{clip}%
\pgfsetroundcap%
\pgfsetroundjoin%
\pgfsetlinewidth{1.204500pt}%
\definecolor{currentstroke}{rgb}{1.000000,0.576471,0.309804}%
\pgfsetstrokecolor{currentstroke}%
\pgfsetdash{}{0pt}%
\pgfpathmoveto{\pgfqpoint{4.575692in}{3.931403in}}%
\pgfusepath{stroke}%
\end{pgfscope}%
\begin{pgfscope}%
\pgfpathrectangle{\pgfqpoint{0.765000in}{0.660000in}}{\pgfqpoint{4.620000in}{4.620000in}}%
\pgfusepath{clip}%
\pgfsetbuttcap%
\pgfsetroundjoin%
\definecolor{currentfill}{rgb}{1.000000,0.576471,0.309804}%
\pgfsetfillcolor{currentfill}%
\pgfsetlinewidth{1.003750pt}%
\definecolor{currentstroke}{rgb}{1.000000,0.576471,0.309804}%
\pgfsetstrokecolor{currentstroke}%
\pgfsetdash{}{0pt}%
\pgfsys@defobject{currentmarker}{\pgfqpoint{-0.033333in}{-0.033333in}}{\pgfqpoint{0.033333in}{0.033333in}}{%
\pgfpathmoveto{\pgfqpoint{0.000000in}{-0.033333in}}%
\pgfpathcurveto{\pgfqpoint{0.008840in}{-0.033333in}}{\pgfqpoint{0.017319in}{-0.029821in}}{\pgfqpoint{0.023570in}{-0.023570in}}%
\pgfpathcurveto{\pgfqpoint{0.029821in}{-0.017319in}}{\pgfqpoint{0.033333in}{-0.008840in}}{\pgfqpoint{0.033333in}{0.000000in}}%
\pgfpathcurveto{\pgfqpoint{0.033333in}{0.008840in}}{\pgfqpoint{0.029821in}{0.017319in}}{\pgfqpoint{0.023570in}{0.023570in}}%
\pgfpathcurveto{\pgfqpoint{0.017319in}{0.029821in}}{\pgfqpoint{0.008840in}{0.033333in}}{\pgfqpoint{0.000000in}{0.033333in}}%
\pgfpathcurveto{\pgfqpoint{-0.008840in}{0.033333in}}{\pgfqpoint{-0.017319in}{0.029821in}}{\pgfqpoint{-0.023570in}{0.023570in}}%
\pgfpathcurveto{\pgfqpoint{-0.029821in}{0.017319in}}{\pgfqpoint{-0.033333in}{0.008840in}}{\pgfqpoint{-0.033333in}{0.000000in}}%
\pgfpathcurveto{\pgfqpoint{-0.033333in}{-0.008840in}}{\pgfqpoint{-0.029821in}{-0.017319in}}{\pgfqpoint{-0.023570in}{-0.023570in}}%
\pgfpathcurveto{\pgfqpoint{-0.017319in}{-0.029821in}}{\pgfqpoint{-0.008840in}{-0.033333in}}{\pgfqpoint{0.000000in}{-0.033333in}}%
\pgfpathlineto{\pgfqpoint{0.000000in}{-0.033333in}}%
\pgfpathclose%
\pgfusepath{stroke,fill}%
}%
\begin{pgfscope}%
\pgfsys@transformshift{4.575692in}{3.931403in}%
\pgfsys@useobject{currentmarker}{}%
\end{pgfscope}%
\end{pgfscope}%
\begin{pgfscope}%
\pgfpathrectangle{\pgfqpoint{0.765000in}{0.660000in}}{\pgfqpoint{4.620000in}{4.620000in}}%
\pgfusepath{clip}%
\pgfsetroundcap%
\pgfsetroundjoin%
\pgfsetlinewidth{1.204500pt}%
\definecolor{currentstroke}{rgb}{1.000000,0.576471,0.309804}%
\pgfsetstrokecolor{currentstroke}%
\pgfsetdash{}{0pt}%
\pgfpathmoveto{\pgfqpoint{4.727949in}{3.450437in}}%
\pgfusepath{stroke}%
\end{pgfscope}%
\begin{pgfscope}%
\pgfpathrectangle{\pgfqpoint{0.765000in}{0.660000in}}{\pgfqpoint{4.620000in}{4.620000in}}%
\pgfusepath{clip}%
\pgfsetbuttcap%
\pgfsetroundjoin%
\definecolor{currentfill}{rgb}{1.000000,0.576471,0.309804}%
\pgfsetfillcolor{currentfill}%
\pgfsetlinewidth{1.003750pt}%
\definecolor{currentstroke}{rgb}{1.000000,0.576471,0.309804}%
\pgfsetstrokecolor{currentstroke}%
\pgfsetdash{}{0pt}%
\pgfsys@defobject{currentmarker}{\pgfqpoint{-0.033333in}{-0.033333in}}{\pgfqpoint{0.033333in}{0.033333in}}{%
\pgfpathmoveto{\pgfqpoint{0.000000in}{-0.033333in}}%
\pgfpathcurveto{\pgfqpoint{0.008840in}{-0.033333in}}{\pgfqpoint{0.017319in}{-0.029821in}}{\pgfqpoint{0.023570in}{-0.023570in}}%
\pgfpathcurveto{\pgfqpoint{0.029821in}{-0.017319in}}{\pgfqpoint{0.033333in}{-0.008840in}}{\pgfqpoint{0.033333in}{0.000000in}}%
\pgfpathcurveto{\pgfqpoint{0.033333in}{0.008840in}}{\pgfqpoint{0.029821in}{0.017319in}}{\pgfqpoint{0.023570in}{0.023570in}}%
\pgfpathcurveto{\pgfqpoint{0.017319in}{0.029821in}}{\pgfqpoint{0.008840in}{0.033333in}}{\pgfqpoint{0.000000in}{0.033333in}}%
\pgfpathcurveto{\pgfqpoint{-0.008840in}{0.033333in}}{\pgfqpoint{-0.017319in}{0.029821in}}{\pgfqpoint{-0.023570in}{0.023570in}}%
\pgfpathcurveto{\pgfqpoint{-0.029821in}{0.017319in}}{\pgfqpoint{-0.033333in}{0.008840in}}{\pgfqpoint{-0.033333in}{0.000000in}}%
\pgfpathcurveto{\pgfqpoint{-0.033333in}{-0.008840in}}{\pgfqpoint{-0.029821in}{-0.017319in}}{\pgfqpoint{-0.023570in}{-0.023570in}}%
\pgfpathcurveto{\pgfqpoint{-0.017319in}{-0.029821in}}{\pgfqpoint{-0.008840in}{-0.033333in}}{\pgfqpoint{0.000000in}{-0.033333in}}%
\pgfpathlineto{\pgfqpoint{0.000000in}{-0.033333in}}%
\pgfpathclose%
\pgfusepath{stroke,fill}%
}%
\begin{pgfscope}%
\pgfsys@transformshift{4.727949in}{3.450437in}%
\pgfsys@useobject{currentmarker}{}%
\end{pgfscope}%
\end{pgfscope}%
\begin{pgfscope}%
\pgfpathrectangle{\pgfqpoint{0.765000in}{0.660000in}}{\pgfqpoint{4.620000in}{4.620000in}}%
\pgfusepath{clip}%
\pgfsetroundcap%
\pgfsetroundjoin%
\pgfsetlinewidth{1.204500pt}%
\definecolor{currentstroke}{rgb}{1.000000,0.576471,0.309804}%
\pgfsetstrokecolor{currentstroke}%
\pgfsetdash{}{0pt}%
\pgfpathmoveto{\pgfqpoint{4.684886in}{3.947482in}}%
\pgfusepath{stroke}%
\end{pgfscope}%
\begin{pgfscope}%
\pgfpathrectangle{\pgfqpoint{0.765000in}{0.660000in}}{\pgfqpoint{4.620000in}{4.620000in}}%
\pgfusepath{clip}%
\pgfsetbuttcap%
\pgfsetroundjoin%
\definecolor{currentfill}{rgb}{1.000000,0.576471,0.309804}%
\pgfsetfillcolor{currentfill}%
\pgfsetlinewidth{1.003750pt}%
\definecolor{currentstroke}{rgb}{1.000000,0.576471,0.309804}%
\pgfsetstrokecolor{currentstroke}%
\pgfsetdash{}{0pt}%
\pgfsys@defobject{currentmarker}{\pgfqpoint{-0.033333in}{-0.033333in}}{\pgfqpoint{0.033333in}{0.033333in}}{%
\pgfpathmoveto{\pgfqpoint{0.000000in}{-0.033333in}}%
\pgfpathcurveto{\pgfqpoint{0.008840in}{-0.033333in}}{\pgfqpoint{0.017319in}{-0.029821in}}{\pgfqpoint{0.023570in}{-0.023570in}}%
\pgfpathcurveto{\pgfqpoint{0.029821in}{-0.017319in}}{\pgfqpoint{0.033333in}{-0.008840in}}{\pgfqpoint{0.033333in}{0.000000in}}%
\pgfpathcurveto{\pgfqpoint{0.033333in}{0.008840in}}{\pgfqpoint{0.029821in}{0.017319in}}{\pgfqpoint{0.023570in}{0.023570in}}%
\pgfpathcurveto{\pgfqpoint{0.017319in}{0.029821in}}{\pgfqpoint{0.008840in}{0.033333in}}{\pgfqpoint{0.000000in}{0.033333in}}%
\pgfpathcurveto{\pgfqpoint{-0.008840in}{0.033333in}}{\pgfqpoint{-0.017319in}{0.029821in}}{\pgfqpoint{-0.023570in}{0.023570in}}%
\pgfpathcurveto{\pgfqpoint{-0.029821in}{0.017319in}}{\pgfqpoint{-0.033333in}{0.008840in}}{\pgfqpoint{-0.033333in}{0.000000in}}%
\pgfpathcurveto{\pgfqpoint{-0.033333in}{-0.008840in}}{\pgfqpoint{-0.029821in}{-0.017319in}}{\pgfqpoint{-0.023570in}{-0.023570in}}%
\pgfpathcurveto{\pgfqpoint{-0.017319in}{-0.029821in}}{\pgfqpoint{-0.008840in}{-0.033333in}}{\pgfqpoint{0.000000in}{-0.033333in}}%
\pgfpathlineto{\pgfqpoint{0.000000in}{-0.033333in}}%
\pgfpathclose%
\pgfusepath{stroke,fill}%
}%
\begin{pgfscope}%
\pgfsys@transformshift{4.684886in}{3.947482in}%
\pgfsys@useobject{currentmarker}{}%
\end{pgfscope}%
\end{pgfscope}%
\begin{pgfscope}%
\pgfpathrectangle{\pgfqpoint{0.765000in}{0.660000in}}{\pgfqpoint{4.620000in}{4.620000in}}%
\pgfusepath{clip}%
\pgfsetroundcap%
\pgfsetroundjoin%
\pgfsetlinewidth{1.204500pt}%
\definecolor{currentstroke}{rgb}{1.000000,0.576471,0.309804}%
\pgfsetstrokecolor{currentstroke}%
\pgfsetdash{}{0pt}%
\pgfpathmoveto{\pgfqpoint{4.444824in}{3.671711in}}%
\pgfusepath{stroke}%
\end{pgfscope}%
\begin{pgfscope}%
\pgfpathrectangle{\pgfqpoint{0.765000in}{0.660000in}}{\pgfqpoint{4.620000in}{4.620000in}}%
\pgfusepath{clip}%
\pgfsetbuttcap%
\pgfsetroundjoin%
\definecolor{currentfill}{rgb}{1.000000,0.576471,0.309804}%
\pgfsetfillcolor{currentfill}%
\pgfsetlinewidth{1.003750pt}%
\definecolor{currentstroke}{rgb}{1.000000,0.576471,0.309804}%
\pgfsetstrokecolor{currentstroke}%
\pgfsetdash{}{0pt}%
\pgfsys@defobject{currentmarker}{\pgfqpoint{-0.033333in}{-0.033333in}}{\pgfqpoint{0.033333in}{0.033333in}}{%
\pgfpathmoveto{\pgfqpoint{0.000000in}{-0.033333in}}%
\pgfpathcurveto{\pgfqpoint{0.008840in}{-0.033333in}}{\pgfqpoint{0.017319in}{-0.029821in}}{\pgfqpoint{0.023570in}{-0.023570in}}%
\pgfpathcurveto{\pgfqpoint{0.029821in}{-0.017319in}}{\pgfqpoint{0.033333in}{-0.008840in}}{\pgfqpoint{0.033333in}{0.000000in}}%
\pgfpathcurveto{\pgfqpoint{0.033333in}{0.008840in}}{\pgfqpoint{0.029821in}{0.017319in}}{\pgfqpoint{0.023570in}{0.023570in}}%
\pgfpathcurveto{\pgfqpoint{0.017319in}{0.029821in}}{\pgfqpoint{0.008840in}{0.033333in}}{\pgfqpoint{0.000000in}{0.033333in}}%
\pgfpathcurveto{\pgfqpoint{-0.008840in}{0.033333in}}{\pgfqpoint{-0.017319in}{0.029821in}}{\pgfqpoint{-0.023570in}{0.023570in}}%
\pgfpathcurveto{\pgfqpoint{-0.029821in}{0.017319in}}{\pgfqpoint{-0.033333in}{0.008840in}}{\pgfqpoint{-0.033333in}{0.000000in}}%
\pgfpathcurveto{\pgfqpoint{-0.033333in}{-0.008840in}}{\pgfqpoint{-0.029821in}{-0.017319in}}{\pgfqpoint{-0.023570in}{-0.023570in}}%
\pgfpathcurveto{\pgfqpoint{-0.017319in}{-0.029821in}}{\pgfqpoint{-0.008840in}{-0.033333in}}{\pgfqpoint{0.000000in}{-0.033333in}}%
\pgfpathlineto{\pgfqpoint{0.000000in}{-0.033333in}}%
\pgfpathclose%
\pgfusepath{stroke,fill}%
}%
\begin{pgfscope}%
\pgfsys@transformshift{4.444824in}{3.671711in}%
\pgfsys@useobject{currentmarker}{}%
\end{pgfscope}%
\end{pgfscope}%
\begin{pgfscope}%
\pgfpathrectangle{\pgfqpoint{0.765000in}{0.660000in}}{\pgfqpoint{4.620000in}{4.620000in}}%
\pgfusepath{clip}%
\pgfsetroundcap%
\pgfsetroundjoin%
\pgfsetlinewidth{1.204500pt}%
\definecolor{currentstroke}{rgb}{1.000000,0.576471,0.309804}%
\pgfsetstrokecolor{currentstroke}%
\pgfsetdash{}{0pt}%
\pgfpathmoveto{\pgfqpoint{4.327018in}{3.679960in}}%
\pgfusepath{stroke}%
\end{pgfscope}%
\begin{pgfscope}%
\pgfpathrectangle{\pgfqpoint{0.765000in}{0.660000in}}{\pgfqpoint{4.620000in}{4.620000in}}%
\pgfusepath{clip}%
\pgfsetbuttcap%
\pgfsetroundjoin%
\definecolor{currentfill}{rgb}{1.000000,0.576471,0.309804}%
\pgfsetfillcolor{currentfill}%
\pgfsetlinewidth{1.003750pt}%
\definecolor{currentstroke}{rgb}{1.000000,0.576471,0.309804}%
\pgfsetstrokecolor{currentstroke}%
\pgfsetdash{}{0pt}%
\pgfsys@defobject{currentmarker}{\pgfqpoint{-0.033333in}{-0.033333in}}{\pgfqpoint{0.033333in}{0.033333in}}{%
\pgfpathmoveto{\pgfqpoint{0.000000in}{-0.033333in}}%
\pgfpathcurveto{\pgfqpoint{0.008840in}{-0.033333in}}{\pgfqpoint{0.017319in}{-0.029821in}}{\pgfqpoint{0.023570in}{-0.023570in}}%
\pgfpathcurveto{\pgfqpoint{0.029821in}{-0.017319in}}{\pgfqpoint{0.033333in}{-0.008840in}}{\pgfqpoint{0.033333in}{0.000000in}}%
\pgfpathcurveto{\pgfqpoint{0.033333in}{0.008840in}}{\pgfqpoint{0.029821in}{0.017319in}}{\pgfqpoint{0.023570in}{0.023570in}}%
\pgfpathcurveto{\pgfqpoint{0.017319in}{0.029821in}}{\pgfqpoint{0.008840in}{0.033333in}}{\pgfqpoint{0.000000in}{0.033333in}}%
\pgfpathcurveto{\pgfqpoint{-0.008840in}{0.033333in}}{\pgfqpoint{-0.017319in}{0.029821in}}{\pgfqpoint{-0.023570in}{0.023570in}}%
\pgfpathcurveto{\pgfqpoint{-0.029821in}{0.017319in}}{\pgfqpoint{-0.033333in}{0.008840in}}{\pgfqpoint{-0.033333in}{0.000000in}}%
\pgfpathcurveto{\pgfqpoint{-0.033333in}{-0.008840in}}{\pgfqpoint{-0.029821in}{-0.017319in}}{\pgfqpoint{-0.023570in}{-0.023570in}}%
\pgfpathcurveto{\pgfqpoint{-0.017319in}{-0.029821in}}{\pgfqpoint{-0.008840in}{-0.033333in}}{\pgfqpoint{0.000000in}{-0.033333in}}%
\pgfpathlineto{\pgfqpoint{0.000000in}{-0.033333in}}%
\pgfpathclose%
\pgfusepath{stroke,fill}%
}%
\begin{pgfscope}%
\pgfsys@transformshift{4.327018in}{3.679960in}%
\pgfsys@useobject{currentmarker}{}%
\end{pgfscope}%
\end{pgfscope}%
\begin{pgfscope}%
\pgfpathrectangle{\pgfqpoint{0.765000in}{0.660000in}}{\pgfqpoint{4.620000in}{4.620000in}}%
\pgfusepath{clip}%
\pgfsetroundcap%
\pgfsetroundjoin%
\pgfsetlinewidth{1.204500pt}%
\definecolor{currentstroke}{rgb}{1.000000,0.576471,0.309804}%
\pgfsetstrokecolor{currentstroke}%
\pgfsetdash{}{0pt}%
\pgfpathmoveto{\pgfqpoint{4.527043in}{3.591971in}}%
\pgfusepath{stroke}%
\end{pgfscope}%
\begin{pgfscope}%
\pgfpathrectangle{\pgfqpoint{0.765000in}{0.660000in}}{\pgfqpoint{4.620000in}{4.620000in}}%
\pgfusepath{clip}%
\pgfsetbuttcap%
\pgfsetroundjoin%
\definecolor{currentfill}{rgb}{1.000000,0.576471,0.309804}%
\pgfsetfillcolor{currentfill}%
\pgfsetlinewidth{1.003750pt}%
\definecolor{currentstroke}{rgb}{1.000000,0.576471,0.309804}%
\pgfsetstrokecolor{currentstroke}%
\pgfsetdash{}{0pt}%
\pgfsys@defobject{currentmarker}{\pgfqpoint{-0.033333in}{-0.033333in}}{\pgfqpoint{0.033333in}{0.033333in}}{%
\pgfpathmoveto{\pgfqpoint{0.000000in}{-0.033333in}}%
\pgfpathcurveto{\pgfqpoint{0.008840in}{-0.033333in}}{\pgfqpoint{0.017319in}{-0.029821in}}{\pgfqpoint{0.023570in}{-0.023570in}}%
\pgfpathcurveto{\pgfqpoint{0.029821in}{-0.017319in}}{\pgfqpoint{0.033333in}{-0.008840in}}{\pgfqpoint{0.033333in}{0.000000in}}%
\pgfpathcurveto{\pgfqpoint{0.033333in}{0.008840in}}{\pgfqpoint{0.029821in}{0.017319in}}{\pgfqpoint{0.023570in}{0.023570in}}%
\pgfpathcurveto{\pgfqpoint{0.017319in}{0.029821in}}{\pgfqpoint{0.008840in}{0.033333in}}{\pgfqpoint{0.000000in}{0.033333in}}%
\pgfpathcurveto{\pgfqpoint{-0.008840in}{0.033333in}}{\pgfqpoint{-0.017319in}{0.029821in}}{\pgfqpoint{-0.023570in}{0.023570in}}%
\pgfpathcurveto{\pgfqpoint{-0.029821in}{0.017319in}}{\pgfqpoint{-0.033333in}{0.008840in}}{\pgfqpoint{-0.033333in}{0.000000in}}%
\pgfpathcurveto{\pgfqpoint{-0.033333in}{-0.008840in}}{\pgfqpoint{-0.029821in}{-0.017319in}}{\pgfqpoint{-0.023570in}{-0.023570in}}%
\pgfpathcurveto{\pgfqpoint{-0.017319in}{-0.029821in}}{\pgfqpoint{-0.008840in}{-0.033333in}}{\pgfqpoint{0.000000in}{-0.033333in}}%
\pgfpathlineto{\pgfqpoint{0.000000in}{-0.033333in}}%
\pgfpathclose%
\pgfusepath{stroke,fill}%
}%
\begin{pgfscope}%
\pgfsys@transformshift{4.527043in}{3.591971in}%
\pgfsys@useobject{currentmarker}{}%
\end{pgfscope}%
\end{pgfscope}%
\begin{pgfscope}%
\pgfpathrectangle{\pgfqpoint{0.765000in}{0.660000in}}{\pgfqpoint{4.620000in}{4.620000in}}%
\pgfusepath{clip}%
\pgfsetroundcap%
\pgfsetroundjoin%
\pgfsetlinewidth{1.204500pt}%
\definecolor{currentstroke}{rgb}{1.000000,0.576471,0.309804}%
\pgfsetstrokecolor{currentstroke}%
\pgfsetdash{}{0pt}%
\pgfpathmoveto{\pgfqpoint{4.633827in}{3.677788in}}%
\pgfusepath{stroke}%
\end{pgfscope}%
\begin{pgfscope}%
\pgfpathrectangle{\pgfqpoint{0.765000in}{0.660000in}}{\pgfqpoint{4.620000in}{4.620000in}}%
\pgfusepath{clip}%
\pgfsetbuttcap%
\pgfsetroundjoin%
\definecolor{currentfill}{rgb}{1.000000,0.576471,0.309804}%
\pgfsetfillcolor{currentfill}%
\pgfsetlinewidth{1.003750pt}%
\definecolor{currentstroke}{rgb}{1.000000,0.576471,0.309804}%
\pgfsetstrokecolor{currentstroke}%
\pgfsetdash{}{0pt}%
\pgfsys@defobject{currentmarker}{\pgfqpoint{-0.033333in}{-0.033333in}}{\pgfqpoint{0.033333in}{0.033333in}}{%
\pgfpathmoveto{\pgfqpoint{0.000000in}{-0.033333in}}%
\pgfpathcurveto{\pgfqpoint{0.008840in}{-0.033333in}}{\pgfqpoint{0.017319in}{-0.029821in}}{\pgfqpoint{0.023570in}{-0.023570in}}%
\pgfpathcurveto{\pgfqpoint{0.029821in}{-0.017319in}}{\pgfqpoint{0.033333in}{-0.008840in}}{\pgfqpoint{0.033333in}{0.000000in}}%
\pgfpathcurveto{\pgfqpoint{0.033333in}{0.008840in}}{\pgfqpoint{0.029821in}{0.017319in}}{\pgfqpoint{0.023570in}{0.023570in}}%
\pgfpathcurveto{\pgfqpoint{0.017319in}{0.029821in}}{\pgfqpoint{0.008840in}{0.033333in}}{\pgfqpoint{0.000000in}{0.033333in}}%
\pgfpathcurveto{\pgfqpoint{-0.008840in}{0.033333in}}{\pgfqpoint{-0.017319in}{0.029821in}}{\pgfqpoint{-0.023570in}{0.023570in}}%
\pgfpathcurveto{\pgfqpoint{-0.029821in}{0.017319in}}{\pgfqpoint{-0.033333in}{0.008840in}}{\pgfqpoint{-0.033333in}{0.000000in}}%
\pgfpathcurveto{\pgfqpoint{-0.033333in}{-0.008840in}}{\pgfqpoint{-0.029821in}{-0.017319in}}{\pgfqpoint{-0.023570in}{-0.023570in}}%
\pgfpathcurveto{\pgfqpoint{-0.017319in}{-0.029821in}}{\pgfqpoint{-0.008840in}{-0.033333in}}{\pgfqpoint{0.000000in}{-0.033333in}}%
\pgfpathlineto{\pgfqpoint{0.000000in}{-0.033333in}}%
\pgfpathclose%
\pgfusepath{stroke,fill}%
}%
\begin{pgfscope}%
\pgfsys@transformshift{4.633827in}{3.677788in}%
\pgfsys@useobject{currentmarker}{}%
\end{pgfscope}%
\end{pgfscope}%
\begin{pgfscope}%
\pgfpathrectangle{\pgfqpoint{0.765000in}{0.660000in}}{\pgfqpoint{4.620000in}{4.620000in}}%
\pgfusepath{clip}%
\pgfsetroundcap%
\pgfsetroundjoin%
\pgfsetlinewidth{1.204500pt}%
\definecolor{currentstroke}{rgb}{1.000000,0.576471,0.309804}%
\pgfsetstrokecolor{currentstroke}%
\pgfsetdash{}{0pt}%
\pgfpathmoveto{\pgfqpoint{4.711174in}{3.562883in}}%
\pgfusepath{stroke}%
\end{pgfscope}%
\begin{pgfscope}%
\pgfpathrectangle{\pgfqpoint{0.765000in}{0.660000in}}{\pgfqpoint{4.620000in}{4.620000in}}%
\pgfusepath{clip}%
\pgfsetbuttcap%
\pgfsetroundjoin%
\definecolor{currentfill}{rgb}{1.000000,0.576471,0.309804}%
\pgfsetfillcolor{currentfill}%
\pgfsetlinewidth{1.003750pt}%
\definecolor{currentstroke}{rgb}{1.000000,0.576471,0.309804}%
\pgfsetstrokecolor{currentstroke}%
\pgfsetdash{}{0pt}%
\pgfsys@defobject{currentmarker}{\pgfqpoint{-0.033333in}{-0.033333in}}{\pgfqpoint{0.033333in}{0.033333in}}{%
\pgfpathmoveto{\pgfqpoint{0.000000in}{-0.033333in}}%
\pgfpathcurveto{\pgfqpoint{0.008840in}{-0.033333in}}{\pgfqpoint{0.017319in}{-0.029821in}}{\pgfqpoint{0.023570in}{-0.023570in}}%
\pgfpathcurveto{\pgfqpoint{0.029821in}{-0.017319in}}{\pgfqpoint{0.033333in}{-0.008840in}}{\pgfqpoint{0.033333in}{0.000000in}}%
\pgfpathcurveto{\pgfqpoint{0.033333in}{0.008840in}}{\pgfqpoint{0.029821in}{0.017319in}}{\pgfqpoint{0.023570in}{0.023570in}}%
\pgfpathcurveto{\pgfqpoint{0.017319in}{0.029821in}}{\pgfqpoint{0.008840in}{0.033333in}}{\pgfqpoint{0.000000in}{0.033333in}}%
\pgfpathcurveto{\pgfqpoint{-0.008840in}{0.033333in}}{\pgfqpoint{-0.017319in}{0.029821in}}{\pgfqpoint{-0.023570in}{0.023570in}}%
\pgfpathcurveto{\pgfqpoint{-0.029821in}{0.017319in}}{\pgfqpoint{-0.033333in}{0.008840in}}{\pgfqpoint{-0.033333in}{0.000000in}}%
\pgfpathcurveto{\pgfqpoint{-0.033333in}{-0.008840in}}{\pgfqpoint{-0.029821in}{-0.017319in}}{\pgfqpoint{-0.023570in}{-0.023570in}}%
\pgfpathcurveto{\pgfqpoint{-0.017319in}{-0.029821in}}{\pgfqpoint{-0.008840in}{-0.033333in}}{\pgfqpoint{0.000000in}{-0.033333in}}%
\pgfpathlineto{\pgfqpoint{0.000000in}{-0.033333in}}%
\pgfpathclose%
\pgfusepath{stroke,fill}%
}%
\begin{pgfscope}%
\pgfsys@transformshift{4.711174in}{3.562883in}%
\pgfsys@useobject{currentmarker}{}%
\end{pgfscope}%
\end{pgfscope}%
\begin{pgfscope}%
\pgfpathrectangle{\pgfqpoint{0.765000in}{0.660000in}}{\pgfqpoint{4.620000in}{4.620000in}}%
\pgfusepath{clip}%
\pgfsetroundcap%
\pgfsetroundjoin%
\pgfsetlinewidth{1.204500pt}%
\definecolor{currentstroke}{rgb}{0.176471,0.192157,0.258824}%
\pgfsetstrokecolor{currentstroke}%
\pgfsetdash{}{0pt}%
\pgfpathmoveto{\pgfqpoint{1.576906in}{3.852795in}}%
\pgfusepath{stroke}%
\end{pgfscope}%
\begin{pgfscope}%
\pgfpathrectangle{\pgfqpoint{0.765000in}{0.660000in}}{\pgfqpoint{4.620000in}{4.620000in}}%
\pgfusepath{clip}%
\pgfsetbuttcap%
\pgfsetmiterjoin%
\definecolor{currentfill}{rgb}{0.176471,0.192157,0.258824}%
\pgfsetfillcolor{currentfill}%
\pgfsetlinewidth{1.003750pt}%
\definecolor{currentstroke}{rgb}{0.176471,0.192157,0.258824}%
\pgfsetstrokecolor{currentstroke}%
\pgfsetdash{}{0pt}%
\pgfsys@defobject{currentmarker}{\pgfqpoint{-0.033333in}{-0.033333in}}{\pgfqpoint{0.033333in}{0.033333in}}{%
\pgfpathmoveto{\pgfqpoint{-0.000000in}{-0.033333in}}%
\pgfpathlineto{\pgfqpoint{0.033333in}{0.033333in}}%
\pgfpathlineto{\pgfqpoint{-0.033333in}{0.033333in}}%
\pgfpathlineto{\pgfqpoint{-0.000000in}{-0.033333in}}%
\pgfpathclose%
\pgfusepath{stroke,fill}%
}%
\begin{pgfscope}%
\pgfsys@transformshift{1.576906in}{3.852795in}%
\pgfsys@useobject{currentmarker}{}%
\end{pgfscope}%
\end{pgfscope}%
\begin{pgfscope}%
\pgfpathrectangle{\pgfqpoint{0.765000in}{0.660000in}}{\pgfqpoint{4.620000in}{4.620000in}}%
\pgfusepath{clip}%
\pgfsetroundcap%
\pgfsetroundjoin%
\pgfsetlinewidth{1.204500pt}%
\definecolor{currentstroke}{rgb}{0.176471,0.192157,0.258824}%
\pgfsetstrokecolor{currentstroke}%
\pgfsetdash{}{0pt}%
\pgfpathmoveto{\pgfqpoint{1.882907in}{3.849578in}}%
\pgfusepath{stroke}%
\end{pgfscope}%
\begin{pgfscope}%
\pgfpathrectangle{\pgfqpoint{0.765000in}{0.660000in}}{\pgfqpoint{4.620000in}{4.620000in}}%
\pgfusepath{clip}%
\pgfsetbuttcap%
\pgfsetmiterjoin%
\definecolor{currentfill}{rgb}{0.176471,0.192157,0.258824}%
\pgfsetfillcolor{currentfill}%
\pgfsetlinewidth{1.003750pt}%
\definecolor{currentstroke}{rgb}{0.176471,0.192157,0.258824}%
\pgfsetstrokecolor{currentstroke}%
\pgfsetdash{}{0pt}%
\pgfsys@defobject{currentmarker}{\pgfqpoint{-0.033333in}{-0.033333in}}{\pgfqpoint{0.033333in}{0.033333in}}{%
\pgfpathmoveto{\pgfqpoint{-0.000000in}{-0.033333in}}%
\pgfpathlineto{\pgfqpoint{0.033333in}{0.033333in}}%
\pgfpathlineto{\pgfqpoint{-0.033333in}{0.033333in}}%
\pgfpathlineto{\pgfqpoint{-0.000000in}{-0.033333in}}%
\pgfpathclose%
\pgfusepath{stroke,fill}%
}%
\begin{pgfscope}%
\pgfsys@transformshift{1.882907in}{3.849578in}%
\pgfsys@useobject{currentmarker}{}%
\end{pgfscope}%
\end{pgfscope}%
\begin{pgfscope}%
\pgfpathrectangle{\pgfqpoint{0.765000in}{0.660000in}}{\pgfqpoint{4.620000in}{4.620000in}}%
\pgfusepath{clip}%
\pgfsetroundcap%
\pgfsetroundjoin%
\pgfsetlinewidth{1.204500pt}%
\definecolor{currentstroke}{rgb}{0.176471,0.192157,0.258824}%
\pgfsetstrokecolor{currentstroke}%
\pgfsetdash{}{0pt}%
\pgfpathmoveto{\pgfqpoint{1.889868in}{3.777574in}}%
\pgfusepath{stroke}%
\end{pgfscope}%
\begin{pgfscope}%
\pgfpathrectangle{\pgfqpoint{0.765000in}{0.660000in}}{\pgfqpoint{4.620000in}{4.620000in}}%
\pgfusepath{clip}%
\pgfsetbuttcap%
\pgfsetmiterjoin%
\definecolor{currentfill}{rgb}{0.176471,0.192157,0.258824}%
\pgfsetfillcolor{currentfill}%
\pgfsetlinewidth{1.003750pt}%
\definecolor{currentstroke}{rgb}{0.176471,0.192157,0.258824}%
\pgfsetstrokecolor{currentstroke}%
\pgfsetdash{}{0pt}%
\pgfsys@defobject{currentmarker}{\pgfqpoint{-0.033333in}{-0.033333in}}{\pgfqpoint{0.033333in}{0.033333in}}{%
\pgfpathmoveto{\pgfqpoint{-0.000000in}{-0.033333in}}%
\pgfpathlineto{\pgfqpoint{0.033333in}{0.033333in}}%
\pgfpathlineto{\pgfqpoint{-0.033333in}{0.033333in}}%
\pgfpathlineto{\pgfqpoint{-0.000000in}{-0.033333in}}%
\pgfpathclose%
\pgfusepath{stroke,fill}%
}%
\begin{pgfscope}%
\pgfsys@transformshift{1.889868in}{3.777574in}%
\pgfsys@useobject{currentmarker}{}%
\end{pgfscope}%
\end{pgfscope}%
\begin{pgfscope}%
\pgfpathrectangle{\pgfqpoint{0.765000in}{0.660000in}}{\pgfqpoint{4.620000in}{4.620000in}}%
\pgfusepath{clip}%
\pgfsetroundcap%
\pgfsetroundjoin%
\pgfsetlinewidth{1.204500pt}%
\definecolor{currentstroke}{rgb}{0.176471,0.192157,0.258824}%
\pgfsetstrokecolor{currentstroke}%
\pgfsetdash{}{0pt}%
\pgfpathmoveto{\pgfqpoint{1.917679in}{3.687971in}}%
\pgfusepath{stroke}%
\end{pgfscope}%
\begin{pgfscope}%
\pgfpathrectangle{\pgfqpoint{0.765000in}{0.660000in}}{\pgfqpoint{4.620000in}{4.620000in}}%
\pgfusepath{clip}%
\pgfsetbuttcap%
\pgfsetmiterjoin%
\definecolor{currentfill}{rgb}{0.176471,0.192157,0.258824}%
\pgfsetfillcolor{currentfill}%
\pgfsetlinewidth{1.003750pt}%
\definecolor{currentstroke}{rgb}{0.176471,0.192157,0.258824}%
\pgfsetstrokecolor{currentstroke}%
\pgfsetdash{}{0pt}%
\pgfsys@defobject{currentmarker}{\pgfqpoint{-0.033333in}{-0.033333in}}{\pgfqpoint{0.033333in}{0.033333in}}{%
\pgfpathmoveto{\pgfqpoint{-0.000000in}{-0.033333in}}%
\pgfpathlineto{\pgfqpoint{0.033333in}{0.033333in}}%
\pgfpathlineto{\pgfqpoint{-0.033333in}{0.033333in}}%
\pgfpathlineto{\pgfqpoint{-0.000000in}{-0.033333in}}%
\pgfpathclose%
\pgfusepath{stroke,fill}%
}%
\begin{pgfscope}%
\pgfsys@transformshift{1.917679in}{3.687971in}%
\pgfsys@useobject{currentmarker}{}%
\end{pgfscope}%
\end{pgfscope}%
\begin{pgfscope}%
\pgfpathrectangle{\pgfqpoint{0.765000in}{0.660000in}}{\pgfqpoint{4.620000in}{4.620000in}}%
\pgfusepath{clip}%
\pgfsetroundcap%
\pgfsetroundjoin%
\pgfsetlinewidth{1.204500pt}%
\definecolor{currentstroke}{rgb}{0.176471,0.192157,0.258824}%
\pgfsetstrokecolor{currentstroke}%
\pgfsetdash{}{0pt}%
\pgfpathmoveto{\pgfqpoint{1.323944in}{3.676791in}}%
\pgfusepath{stroke}%
\end{pgfscope}%
\begin{pgfscope}%
\pgfpathrectangle{\pgfqpoint{0.765000in}{0.660000in}}{\pgfqpoint{4.620000in}{4.620000in}}%
\pgfusepath{clip}%
\pgfsetbuttcap%
\pgfsetmiterjoin%
\definecolor{currentfill}{rgb}{0.176471,0.192157,0.258824}%
\pgfsetfillcolor{currentfill}%
\pgfsetlinewidth{1.003750pt}%
\definecolor{currentstroke}{rgb}{0.176471,0.192157,0.258824}%
\pgfsetstrokecolor{currentstroke}%
\pgfsetdash{}{0pt}%
\pgfsys@defobject{currentmarker}{\pgfqpoint{-0.033333in}{-0.033333in}}{\pgfqpoint{0.033333in}{0.033333in}}{%
\pgfpathmoveto{\pgfqpoint{-0.000000in}{-0.033333in}}%
\pgfpathlineto{\pgfqpoint{0.033333in}{0.033333in}}%
\pgfpathlineto{\pgfqpoint{-0.033333in}{0.033333in}}%
\pgfpathlineto{\pgfqpoint{-0.000000in}{-0.033333in}}%
\pgfpathclose%
\pgfusepath{stroke,fill}%
}%
\begin{pgfscope}%
\pgfsys@transformshift{1.323944in}{3.676791in}%
\pgfsys@useobject{currentmarker}{}%
\end{pgfscope}%
\end{pgfscope}%
\begin{pgfscope}%
\pgfpathrectangle{\pgfqpoint{0.765000in}{0.660000in}}{\pgfqpoint{4.620000in}{4.620000in}}%
\pgfusepath{clip}%
\pgfsetroundcap%
\pgfsetroundjoin%
\pgfsetlinewidth{1.204500pt}%
\definecolor{currentstroke}{rgb}{0.176471,0.192157,0.258824}%
\pgfsetstrokecolor{currentstroke}%
\pgfsetdash{}{0pt}%
\pgfpathmoveto{\pgfqpoint{1.615209in}{3.639494in}}%
\pgfusepath{stroke}%
\end{pgfscope}%
\begin{pgfscope}%
\pgfpathrectangle{\pgfqpoint{0.765000in}{0.660000in}}{\pgfqpoint{4.620000in}{4.620000in}}%
\pgfusepath{clip}%
\pgfsetbuttcap%
\pgfsetmiterjoin%
\definecolor{currentfill}{rgb}{0.176471,0.192157,0.258824}%
\pgfsetfillcolor{currentfill}%
\pgfsetlinewidth{1.003750pt}%
\definecolor{currentstroke}{rgb}{0.176471,0.192157,0.258824}%
\pgfsetstrokecolor{currentstroke}%
\pgfsetdash{}{0pt}%
\pgfsys@defobject{currentmarker}{\pgfqpoint{-0.033333in}{-0.033333in}}{\pgfqpoint{0.033333in}{0.033333in}}{%
\pgfpathmoveto{\pgfqpoint{-0.000000in}{-0.033333in}}%
\pgfpathlineto{\pgfqpoint{0.033333in}{0.033333in}}%
\pgfpathlineto{\pgfqpoint{-0.033333in}{0.033333in}}%
\pgfpathlineto{\pgfqpoint{-0.000000in}{-0.033333in}}%
\pgfpathclose%
\pgfusepath{stroke,fill}%
}%
\begin{pgfscope}%
\pgfsys@transformshift{1.615209in}{3.639494in}%
\pgfsys@useobject{currentmarker}{}%
\end{pgfscope}%
\end{pgfscope}%
\begin{pgfscope}%
\pgfpathrectangle{\pgfqpoint{0.765000in}{0.660000in}}{\pgfqpoint{4.620000in}{4.620000in}}%
\pgfusepath{clip}%
\pgfsetroundcap%
\pgfsetroundjoin%
\pgfsetlinewidth{1.204500pt}%
\definecolor{currentstroke}{rgb}{0.176471,0.192157,0.258824}%
\pgfsetstrokecolor{currentstroke}%
\pgfsetdash{}{0pt}%
\pgfpathmoveto{\pgfqpoint{1.820042in}{3.572116in}}%
\pgfusepath{stroke}%
\end{pgfscope}%
\begin{pgfscope}%
\pgfpathrectangle{\pgfqpoint{0.765000in}{0.660000in}}{\pgfqpoint{4.620000in}{4.620000in}}%
\pgfusepath{clip}%
\pgfsetbuttcap%
\pgfsetmiterjoin%
\definecolor{currentfill}{rgb}{0.176471,0.192157,0.258824}%
\pgfsetfillcolor{currentfill}%
\pgfsetlinewidth{1.003750pt}%
\definecolor{currentstroke}{rgb}{0.176471,0.192157,0.258824}%
\pgfsetstrokecolor{currentstroke}%
\pgfsetdash{}{0pt}%
\pgfsys@defobject{currentmarker}{\pgfqpoint{-0.033333in}{-0.033333in}}{\pgfqpoint{0.033333in}{0.033333in}}{%
\pgfpathmoveto{\pgfqpoint{-0.000000in}{-0.033333in}}%
\pgfpathlineto{\pgfqpoint{0.033333in}{0.033333in}}%
\pgfpathlineto{\pgfqpoint{-0.033333in}{0.033333in}}%
\pgfpathlineto{\pgfqpoint{-0.000000in}{-0.033333in}}%
\pgfpathclose%
\pgfusepath{stroke,fill}%
}%
\begin{pgfscope}%
\pgfsys@transformshift{1.820042in}{3.572116in}%
\pgfsys@useobject{currentmarker}{}%
\end{pgfscope}%
\end{pgfscope}%
\begin{pgfscope}%
\pgfpathrectangle{\pgfqpoint{0.765000in}{0.660000in}}{\pgfqpoint{4.620000in}{4.620000in}}%
\pgfusepath{clip}%
\pgfsetroundcap%
\pgfsetroundjoin%
\pgfsetlinewidth{1.204500pt}%
\definecolor{currentstroke}{rgb}{0.176471,0.192157,0.258824}%
\pgfsetstrokecolor{currentstroke}%
\pgfsetdash{}{0pt}%
\pgfpathmoveto{\pgfqpoint{2.118666in}{3.732888in}}%
\pgfusepath{stroke}%
\end{pgfscope}%
\begin{pgfscope}%
\pgfpathrectangle{\pgfqpoint{0.765000in}{0.660000in}}{\pgfqpoint{4.620000in}{4.620000in}}%
\pgfusepath{clip}%
\pgfsetbuttcap%
\pgfsetmiterjoin%
\definecolor{currentfill}{rgb}{0.176471,0.192157,0.258824}%
\pgfsetfillcolor{currentfill}%
\pgfsetlinewidth{1.003750pt}%
\definecolor{currentstroke}{rgb}{0.176471,0.192157,0.258824}%
\pgfsetstrokecolor{currentstroke}%
\pgfsetdash{}{0pt}%
\pgfsys@defobject{currentmarker}{\pgfqpoint{-0.033333in}{-0.033333in}}{\pgfqpoint{0.033333in}{0.033333in}}{%
\pgfpathmoveto{\pgfqpoint{-0.000000in}{-0.033333in}}%
\pgfpathlineto{\pgfqpoint{0.033333in}{0.033333in}}%
\pgfpathlineto{\pgfqpoint{-0.033333in}{0.033333in}}%
\pgfpathlineto{\pgfqpoint{-0.000000in}{-0.033333in}}%
\pgfpathclose%
\pgfusepath{stroke,fill}%
}%
\begin{pgfscope}%
\pgfsys@transformshift{2.118666in}{3.732888in}%
\pgfsys@useobject{currentmarker}{}%
\end{pgfscope}%
\end{pgfscope}%
\begin{pgfscope}%
\pgfpathrectangle{\pgfqpoint{0.765000in}{0.660000in}}{\pgfqpoint{4.620000in}{4.620000in}}%
\pgfusepath{clip}%
\pgfsetroundcap%
\pgfsetroundjoin%
\pgfsetlinewidth{1.204500pt}%
\definecolor{currentstroke}{rgb}{0.176471,0.192157,0.258824}%
\pgfsetstrokecolor{currentstroke}%
\pgfsetdash{}{0pt}%
\pgfpathmoveto{\pgfqpoint{1.569094in}{3.883383in}}%
\pgfusepath{stroke}%
\end{pgfscope}%
\begin{pgfscope}%
\pgfpathrectangle{\pgfqpoint{0.765000in}{0.660000in}}{\pgfqpoint{4.620000in}{4.620000in}}%
\pgfusepath{clip}%
\pgfsetbuttcap%
\pgfsetmiterjoin%
\definecolor{currentfill}{rgb}{0.176471,0.192157,0.258824}%
\pgfsetfillcolor{currentfill}%
\pgfsetlinewidth{1.003750pt}%
\definecolor{currentstroke}{rgb}{0.176471,0.192157,0.258824}%
\pgfsetstrokecolor{currentstroke}%
\pgfsetdash{}{0pt}%
\pgfsys@defobject{currentmarker}{\pgfqpoint{-0.033333in}{-0.033333in}}{\pgfqpoint{0.033333in}{0.033333in}}{%
\pgfpathmoveto{\pgfqpoint{-0.000000in}{-0.033333in}}%
\pgfpathlineto{\pgfqpoint{0.033333in}{0.033333in}}%
\pgfpathlineto{\pgfqpoint{-0.033333in}{0.033333in}}%
\pgfpathlineto{\pgfqpoint{-0.000000in}{-0.033333in}}%
\pgfpathclose%
\pgfusepath{stroke,fill}%
}%
\begin{pgfscope}%
\pgfsys@transformshift{1.569094in}{3.883383in}%
\pgfsys@useobject{currentmarker}{}%
\end{pgfscope}%
\end{pgfscope}%
\begin{pgfscope}%
\pgfpathrectangle{\pgfqpoint{0.765000in}{0.660000in}}{\pgfqpoint{4.620000in}{4.620000in}}%
\pgfusepath{clip}%
\pgfsetroundcap%
\pgfsetroundjoin%
\pgfsetlinewidth{1.204500pt}%
\definecolor{currentstroke}{rgb}{0.176471,0.192157,0.258824}%
\pgfsetstrokecolor{currentstroke}%
\pgfsetdash{}{0pt}%
\pgfpathmoveto{\pgfqpoint{2.124708in}{3.303623in}}%
\pgfusepath{stroke}%
\end{pgfscope}%
\begin{pgfscope}%
\pgfpathrectangle{\pgfqpoint{0.765000in}{0.660000in}}{\pgfqpoint{4.620000in}{4.620000in}}%
\pgfusepath{clip}%
\pgfsetbuttcap%
\pgfsetmiterjoin%
\definecolor{currentfill}{rgb}{0.176471,0.192157,0.258824}%
\pgfsetfillcolor{currentfill}%
\pgfsetlinewidth{1.003750pt}%
\definecolor{currentstroke}{rgb}{0.176471,0.192157,0.258824}%
\pgfsetstrokecolor{currentstroke}%
\pgfsetdash{}{0pt}%
\pgfsys@defobject{currentmarker}{\pgfqpoint{-0.033333in}{-0.033333in}}{\pgfqpoint{0.033333in}{0.033333in}}{%
\pgfpathmoveto{\pgfqpoint{-0.000000in}{-0.033333in}}%
\pgfpathlineto{\pgfqpoint{0.033333in}{0.033333in}}%
\pgfpathlineto{\pgfqpoint{-0.033333in}{0.033333in}}%
\pgfpathlineto{\pgfqpoint{-0.000000in}{-0.033333in}}%
\pgfpathclose%
\pgfusepath{stroke,fill}%
}%
\begin{pgfscope}%
\pgfsys@transformshift{2.124708in}{3.303623in}%
\pgfsys@useobject{currentmarker}{}%
\end{pgfscope}%
\end{pgfscope}%
\begin{pgfscope}%
\pgfpathrectangle{\pgfqpoint{0.765000in}{0.660000in}}{\pgfqpoint{4.620000in}{4.620000in}}%
\pgfusepath{clip}%
\pgfsetroundcap%
\pgfsetroundjoin%
\pgfsetlinewidth{1.204500pt}%
\definecolor{currentstroke}{rgb}{0.019608,0.556863,0.850980}%
\pgfsetstrokecolor{currentstroke}%
\pgfsetdash{}{0pt}%
\pgfpathmoveto{\pgfqpoint{2.927916in}{1.756540in}}%
\pgfusepath{stroke}%
\end{pgfscope}%
\begin{pgfscope}%
\pgfpathrectangle{\pgfqpoint{0.765000in}{0.660000in}}{\pgfqpoint{4.620000in}{4.620000in}}%
\pgfusepath{clip}%
\pgfsetbuttcap%
\pgfsetroundjoin%
\definecolor{currentfill}{rgb}{0.019608,0.556863,0.850980}%
\pgfsetfillcolor{currentfill}%
\pgfsetlinewidth{1.003750pt}%
\definecolor{currentstroke}{rgb}{0.019608,0.556863,0.850980}%
\pgfsetstrokecolor{currentstroke}%
\pgfsetdash{}{0pt}%
\pgfsys@defobject{currentmarker}{\pgfqpoint{-0.033333in}{-0.033333in}}{\pgfqpoint{0.033333in}{0.033333in}}{%
\pgfpathmoveto{\pgfqpoint{-0.033333in}{0.000000in}}%
\pgfpathlineto{\pgfqpoint{0.033333in}{0.000000in}}%
\pgfpathmoveto{\pgfqpoint{0.000000in}{-0.033333in}}%
\pgfpathlineto{\pgfqpoint{0.000000in}{0.033333in}}%
\pgfusepath{stroke,fill}%
}%
\begin{pgfscope}%
\pgfsys@transformshift{2.927916in}{1.756540in}%
\pgfsys@useobject{currentmarker}{}%
\end{pgfscope}%
\end{pgfscope}%
\begin{pgfscope}%
\pgfpathrectangle{\pgfqpoint{0.765000in}{0.660000in}}{\pgfqpoint{4.620000in}{4.620000in}}%
\pgfusepath{clip}%
\pgfsetroundcap%
\pgfsetroundjoin%
\pgfsetlinewidth{1.204500pt}%
\definecolor{currentstroke}{rgb}{0.019608,0.556863,0.850980}%
\pgfsetstrokecolor{currentstroke}%
\pgfsetdash{}{0pt}%
\pgfpathmoveto{\pgfqpoint{3.045502in}{1.713353in}}%
\pgfusepath{stroke}%
\end{pgfscope}%
\begin{pgfscope}%
\pgfpathrectangle{\pgfqpoint{0.765000in}{0.660000in}}{\pgfqpoint{4.620000in}{4.620000in}}%
\pgfusepath{clip}%
\pgfsetbuttcap%
\pgfsetroundjoin%
\definecolor{currentfill}{rgb}{0.019608,0.556863,0.850980}%
\pgfsetfillcolor{currentfill}%
\pgfsetlinewidth{1.003750pt}%
\definecolor{currentstroke}{rgb}{0.019608,0.556863,0.850980}%
\pgfsetstrokecolor{currentstroke}%
\pgfsetdash{}{0pt}%
\pgfsys@defobject{currentmarker}{\pgfqpoint{-0.033333in}{-0.033333in}}{\pgfqpoint{0.033333in}{0.033333in}}{%
\pgfpathmoveto{\pgfqpoint{-0.033333in}{0.000000in}}%
\pgfpathlineto{\pgfqpoint{0.033333in}{0.000000in}}%
\pgfpathmoveto{\pgfqpoint{0.000000in}{-0.033333in}}%
\pgfpathlineto{\pgfqpoint{0.000000in}{0.033333in}}%
\pgfusepath{stroke,fill}%
}%
\begin{pgfscope}%
\pgfsys@transformshift{3.045502in}{1.713353in}%
\pgfsys@useobject{currentmarker}{}%
\end{pgfscope}%
\end{pgfscope}%
\begin{pgfscope}%
\pgfpathrectangle{\pgfqpoint{0.765000in}{0.660000in}}{\pgfqpoint{4.620000in}{4.620000in}}%
\pgfusepath{clip}%
\pgfsetroundcap%
\pgfsetroundjoin%
\pgfsetlinewidth{1.204500pt}%
\definecolor{currentstroke}{rgb}{0.019608,0.556863,0.850980}%
\pgfsetstrokecolor{currentstroke}%
\pgfsetdash{}{0pt}%
\pgfpathmoveto{\pgfqpoint{3.390528in}{1.727114in}}%
\pgfusepath{stroke}%
\end{pgfscope}%
\begin{pgfscope}%
\pgfpathrectangle{\pgfqpoint{0.765000in}{0.660000in}}{\pgfqpoint{4.620000in}{4.620000in}}%
\pgfusepath{clip}%
\pgfsetbuttcap%
\pgfsetroundjoin%
\definecolor{currentfill}{rgb}{0.019608,0.556863,0.850980}%
\pgfsetfillcolor{currentfill}%
\pgfsetlinewidth{1.003750pt}%
\definecolor{currentstroke}{rgb}{0.019608,0.556863,0.850980}%
\pgfsetstrokecolor{currentstroke}%
\pgfsetdash{}{0pt}%
\pgfsys@defobject{currentmarker}{\pgfqpoint{-0.033333in}{-0.033333in}}{\pgfqpoint{0.033333in}{0.033333in}}{%
\pgfpathmoveto{\pgfqpoint{-0.033333in}{0.000000in}}%
\pgfpathlineto{\pgfqpoint{0.033333in}{0.000000in}}%
\pgfpathmoveto{\pgfqpoint{0.000000in}{-0.033333in}}%
\pgfpathlineto{\pgfqpoint{0.000000in}{0.033333in}}%
\pgfusepath{stroke,fill}%
}%
\begin{pgfscope}%
\pgfsys@transformshift{3.390528in}{1.727114in}%
\pgfsys@useobject{currentmarker}{}%
\end{pgfscope}%
\end{pgfscope}%
\begin{pgfscope}%
\pgfpathrectangle{\pgfqpoint{0.765000in}{0.660000in}}{\pgfqpoint{4.620000in}{4.620000in}}%
\pgfusepath{clip}%
\pgfsetroundcap%
\pgfsetroundjoin%
\pgfsetlinewidth{1.204500pt}%
\definecolor{currentstroke}{rgb}{0.019608,0.556863,0.850980}%
\pgfsetstrokecolor{currentstroke}%
\pgfsetdash{}{0pt}%
\pgfpathmoveto{\pgfqpoint{3.159366in}{1.424664in}}%
\pgfusepath{stroke}%
\end{pgfscope}%
\begin{pgfscope}%
\pgfpathrectangle{\pgfqpoint{0.765000in}{0.660000in}}{\pgfqpoint{4.620000in}{4.620000in}}%
\pgfusepath{clip}%
\pgfsetbuttcap%
\pgfsetroundjoin%
\definecolor{currentfill}{rgb}{0.019608,0.556863,0.850980}%
\pgfsetfillcolor{currentfill}%
\pgfsetlinewidth{1.003750pt}%
\definecolor{currentstroke}{rgb}{0.019608,0.556863,0.850980}%
\pgfsetstrokecolor{currentstroke}%
\pgfsetdash{}{0pt}%
\pgfsys@defobject{currentmarker}{\pgfqpoint{-0.033333in}{-0.033333in}}{\pgfqpoint{0.033333in}{0.033333in}}{%
\pgfpathmoveto{\pgfqpoint{-0.033333in}{0.000000in}}%
\pgfpathlineto{\pgfqpoint{0.033333in}{0.000000in}}%
\pgfpathmoveto{\pgfqpoint{0.000000in}{-0.033333in}}%
\pgfpathlineto{\pgfqpoint{0.000000in}{0.033333in}}%
\pgfusepath{stroke,fill}%
}%
\begin{pgfscope}%
\pgfsys@transformshift{3.159366in}{1.424664in}%
\pgfsys@useobject{currentmarker}{}%
\end{pgfscope}%
\end{pgfscope}%
\begin{pgfscope}%
\pgfpathrectangle{\pgfqpoint{0.765000in}{0.660000in}}{\pgfqpoint{4.620000in}{4.620000in}}%
\pgfusepath{clip}%
\pgfsetroundcap%
\pgfsetroundjoin%
\pgfsetlinewidth{1.204500pt}%
\definecolor{currentstroke}{rgb}{0.019608,0.556863,0.850980}%
\pgfsetstrokecolor{currentstroke}%
\pgfsetdash{}{0pt}%
\pgfpathmoveto{\pgfqpoint{3.233804in}{1.810278in}}%
\pgfusepath{stroke}%
\end{pgfscope}%
\begin{pgfscope}%
\pgfpathrectangle{\pgfqpoint{0.765000in}{0.660000in}}{\pgfqpoint{4.620000in}{4.620000in}}%
\pgfusepath{clip}%
\pgfsetbuttcap%
\pgfsetroundjoin%
\definecolor{currentfill}{rgb}{0.019608,0.556863,0.850980}%
\pgfsetfillcolor{currentfill}%
\pgfsetlinewidth{1.003750pt}%
\definecolor{currentstroke}{rgb}{0.019608,0.556863,0.850980}%
\pgfsetstrokecolor{currentstroke}%
\pgfsetdash{}{0pt}%
\pgfsys@defobject{currentmarker}{\pgfqpoint{-0.033333in}{-0.033333in}}{\pgfqpoint{0.033333in}{0.033333in}}{%
\pgfpathmoveto{\pgfqpoint{-0.033333in}{0.000000in}}%
\pgfpathlineto{\pgfqpoint{0.033333in}{0.000000in}}%
\pgfpathmoveto{\pgfqpoint{0.000000in}{-0.033333in}}%
\pgfpathlineto{\pgfqpoint{0.000000in}{0.033333in}}%
\pgfusepath{stroke,fill}%
}%
\begin{pgfscope}%
\pgfsys@transformshift{3.233804in}{1.810278in}%
\pgfsys@useobject{currentmarker}{}%
\end{pgfscope}%
\end{pgfscope}%
\begin{pgfscope}%
\pgfpathrectangle{\pgfqpoint{0.765000in}{0.660000in}}{\pgfqpoint{4.620000in}{4.620000in}}%
\pgfusepath{clip}%
\pgfsetroundcap%
\pgfsetroundjoin%
\pgfsetlinewidth{1.204500pt}%
\definecolor{currentstroke}{rgb}{0.019608,0.556863,0.850980}%
\pgfsetstrokecolor{currentstroke}%
\pgfsetdash{}{0pt}%
\pgfpathmoveto{\pgfqpoint{3.366704in}{1.759629in}}%
\pgfusepath{stroke}%
\end{pgfscope}%
\begin{pgfscope}%
\pgfpathrectangle{\pgfqpoint{0.765000in}{0.660000in}}{\pgfqpoint{4.620000in}{4.620000in}}%
\pgfusepath{clip}%
\pgfsetbuttcap%
\pgfsetroundjoin%
\definecolor{currentfill}{rgb}{0.019608,0.556863,0.850980}%
\pgfsetfillcolor{currentfill}%
\pgfsetlinewidth{1.003750pt}%
\definecolor{currentstroke}{rgb}{0.019608,0.556863,0.850980}%
\pgfsetstrokecolor{currentstroke}%
\pgfsetdash{}{0pt}%
\pgfsys@defobject{currentmarker}{\pgfqpoint{-0.033333in}{-0.033333in}}{\pgfqpoint{0.033333in}{0.033333in}}{%
\pgfpathmoveto{\pgfqpoint{-0.033333in}{0.000000in}}%
\pgfpathlineto{\pgfqpoint{0.033333in}{0.000000in}}%
\pgfpathmoveto{\pgfqpoint{0.000000in}{-0.033333in}}%
\pgfpathlineto{\pgfqpoint{0.000000in}{0.033333in}}%
\pgfusepath{stroke,fill}%
}%
\begin{pgfscope}%
\pgfsys@transformshift{3.366704in}{1.759629in}%
\pgfsys@useobject{currentmarker}{}%
\end{pgfscope}%
\end{pgfscope}%
\begin{pgfscope}%
\pgfpathrectangle{\pgfqpoint{0.765000in}{0.660000in}}{\pgfqpoint{4.620000in}{4.620000in}}%
\pgfusepath{clip}%
\pgfsetroundcap%
\pgfsetroundjoin%
\pgfsetlinewidth{1.204500pt}%
\definecolor{currentstroke}{rgb}{0.019608,0.556863,0.850980}%
\pgfsetstrokecolor{currentstroke}%
\pgfsetdash{}{0pt}%
\pgfpathmoveto{\pgfqpoint{3.370235in}{1.591225in}}%
\pgfusepath{stroke}%
\end{pgfscope}%
\begin{pgfscope}%
\pgfpathrectangle{\pgfqpoint{0.765000in}{0.660000in}}{\pgfqpoint{4.620000in}{4.620000in}}%
\pgfusepath{clip}%
\pgfsetbuttcap%
\pgfsetroundjoin%
\definecolor{currentfill}{rgb}{0.019608,0.556863,0.850980}%
\pgfsetfillcolor{currentfill}%
\pgfsetlinewidth{1.003750pt}%
\definecolor{currentstroke}{rgb}{0.019608,0.556863,0.850980}%
\pgfsetstrokecolor{currentstroke}%
\pgfsetdash{}{0pt}%
\pgfsys@defobject{currentmarker}{\pgfqpoint{-0.033333in}{-0.033333in}}{\pgfqpoint{0.033333in}{0.033333in}}{%
\pgfpathmoveto{\pgfqpoint{-0.033333in}{0.000000in}}%
\pgfpathlineto{\pgfqpoint{0.033333in}{0.000000in}}%
\pgfpathmoveto{\pgfqpoint{0.000000in}{-0.033333in}}%
\pgfpathlineto{\pgfqpoint{0.000000in}{0.033333in}}%
\pgfusepath{stroke,fill}%
}%
\begin{pgfscope}%
\pgfsys@transformshift{3.370235in}{1.591225in}%
\pgfsys@useobject{currentmarker}{}%
\end{pgfscope}%
\end{pgfscope}%
\begin{pgfscope}%
\pgfpathrectangle{\pgfqpoint{0.765000in}{0.660000in}}{\pgfqpoint{4.620000in}{4.620000in}}%
\pgfusepath{clip}%
\pgfsetroundcap%
\pgfsetroundjoin%
\pgfsetlinewidth{1.204500pt}%
\definecolor{currentstroke}{rgb}{0.019608,0.556863,0.850980}%
\pgfsetstrokecolor{currentstroke}%
\pgfsetdash{}{0pt}%
\pgfpathmoveto{\pgfqpoint{3.111211in}{1.827890in}}%
\pgfusepath{stroke}%
\end{pgfscope}%
\begin{pgfscope}%
\pgfpathrectangle{\pgfqpoint{0.765000in}{0.660000in}}{\pgfqpoint{4.620000in}{4.620000in}}%
\pgfusepath{clip}%
\pgfsetbuttcap%
\pgfsetroundjoin%
\definecolor{currentfill}{rgb}{0.019608,0.556863,0.850980}%
\pgfsetfillcolor{currentfill}%
\pgfsetlinewidth{1.003750pt}%
\definecolor{currentstroke}{rgb}{0.019608,0.556863,0.850980}%
\pgfsetstrokecolor{currentstroke}%
\pgfsetdash{}{0pt}%
\pgfsys@defobject{currentmarker}{\pgfqpoint{-0.033333in}{-0.033333in}}{\pgfqpoint{0.033333in}{0.033333in}}{%
\pgfpathmoveto{\pgfqpoint{-0.033333in}{0.000000in}}%
\pgfpathlineto{\pgfqpoint{0.033333in}{0.000000in}}%
\pgfpathmoveto{\pgfqpoint{0.000000in}{-0.033333in}}%
\pgfpathlineto{\pgfqpoint{0.000000in}{0.033333in}}%
\pgfusepath{stroke,fill}%
}%
\begin{pgfscope}%
\pgfsys@transformshift{3.111211in}{1.827890in}%
\pgfsys@useobject{currentmarker}{}%
\end{pgfscope}%
\end{pgfscope}%
\begin{pgfscope}%
\pgfpathrectangle{\pgfqpoint{0.765000in}{0.660000in}}{\pgfqpoint{4.620000in}{4.620000in}}%
\pgfusepath{clip}%
\pgfsetroundcap%
\pgfsetroundjoin%
\pgfsetlinewidth{1.204500pt}%
\definecolor{currentstroke}{rgb}{0.019608,0.556863,0.850980}%
\pgfsetstrokecolor{currentstroke}%
\pgfsetdash{}{0pt}%
\pgfpathmoveto{\pgfqpoint{3.429697in}{1.829944in}}%
\pgfusepath{stroke}%
\end{pgfscope}%
\begin{pgfscope}%
\pgfpathrectangle{\pgfqpoint{0.765000in}{0.660000in}}{\pgfqpoint{4.620000in}{4.620000in}}%
\pgfusepath{clip}%
\pgfsetbuttcap%
\pgfsetroundjoin%
\definecolor{currentfill}{rgb}{0.019608,0.556863,0.850980}%
\pgfsetfillcolor{currentfill}%
\pgfsetlinewidth{1.003750pt}%
\definecolor{currentstroke}{rgb}{0.019608,0.556863,0.850980}%
\pgfsetstrokecolor{currentstroke}%
\pgfsetdash{}{0pt}%
\pgfsys@defobject{currentmarker}{\pgfqpoint{-0.033333in}{-0.033333in}}{\pgfqpoint{0.033333in}{0.033333in}}{%
\pgfpathmoveto{\pgfqpoint{-0.033333in}{0.000000in}}%
\pgfpathlineto{\pgfqpoint{0.033333in}{0.000000in}}%
\pgfpathmoveto{\pgfqpoint{0.000000in}{-0.033333in}}%
\pgfpathlineto{\pgfqpoint{0.000000in}{0.033333in}}%
\pgfusepath{stroke,fill}%
}%
\begin{pgfscope}%
\pgfsys@transformshift{3.429697in}{1.829944in}%
\pgfsys@useobject{currentmarker}{}%
\end{pgfscope}%
\end{pgfscope}%
\begin{pgfscope}%
\pgfpathrectangle{\pgfqpoint{0.765000in}{0.660000in}}{\pgfqpoint{4.620000in}{4.620000in}}%
\pgfusepath{clip}%
\pgfsetroundcap%
\pgfsetroundjoin%
\pgfsetlinewidth{1.204500pt}%
\definecolor{currentstroke}{rgb}{0.019608,0.556863,0.850980}%
\pgfsetstrokecolor{currentstroke}%
\pgfsetdash{}{0pt}%
\pgfpathmoveto{\pgfqpoint{3.334439in}{1.678885in}}%
\pgfusepath{stroke}%
\end{pgfscope}%
\begin{pgfscope}%
\pgfpathrectangle{\pgfqpoint{0.765000in}{0.660000in}}{\pgfqpoint{4.620000in}{4.620000in}}%
\pgfusepath{clip}%
\pgfsetbuttcap%
\pgfsetroundjoin%
\definecolor{currentfill}{rgb}{0.019608,0.556863,0.850980}%
\pgfsetfillcolor{currentfill}%
\pgfsetlinewidth{1.003750pt}%
\definecolor{currentstroke}{rgb}{0.019608,0.556863,0.850980}%
\pgfsetstrokecolor{currentstroke}%
\pgfsetdash{}{0pt}%
\pgfsys@defobject{currentmarker}{\pgfqpoint{-0.033333in}{-0.033333in}}{\pgfqpoint{0.033333in}{0.033333in}}{%
\pgfpathmoveto{\pgfqpoint{-0.033333in}{0.000000in}}%
\pgfpathlineto{\pgfqpoint{0.033333in}{0.000000in}}%
\pgfpathmoveto{\pgfqpoint{0.000000in}{-0.033333in}}%
\pgfpathlineto{\pgfqpoint{0.000000in}{0.033333in}}%
\pgfusepath{stroke,fill}%
}%
\begin{pgfscope}%
\pgfsys@transformshift{3.334439in}{1.678885in}%
\pgfsys@useobject{currentmarker}{}%
\end{pgfscope}%
\end{pgfscope}%
\begin{pgfscope}%
\pgfpathrectangle{\pgfqpoint{0.765000in}{0.660000in}}{\pgfqpoint{4.620000in}{4.620000in}}%
\pgfusepath{clip}%
\pgfsetroundcap%
\pgfsetroundjoin%
\pgfsetlinewidth{1.204500pt}%
\definecolor{currentstroke}{rgb}{0.800000,0.176471,0.207843}%
\pgfsetstrokecolor{currentstroke}%
\pgfsetdash{}{0pt}%
\pgfpathmoveto{\pgfqpoint{2.909201in}{2.967083in}}%
\pgfusepath{stroke}%
\end{pgfscope}%
\begin{pgfscope}%
\pgfpathrectangle{\pgfqpoint{0.765000in}{0.660000in}}{\pgfqpoint{4.620000in}{4.620000in}}%
\pgfusepath{clip}%
\pgfsetbuttcap%
\pgfsetbeveljoin%
\definecolor{currentfill}{rgb}{0.800000,0.176471,0.207843}%
\pgfsetfillcolor{currentfill}%
\pgfsetlinewidth{1.003750pt}%
\definecolor{currentstroke}{rgb}{0.800000,0.176471,0.207843}%
\pgfsetstrokecolor{currentstroke}%
\pgfsetdash{}{0pt}%
\pgfsys@defobject{currentmarker}{\pgfqpoint{-0.031702in}{-0.026967in}}{\pgfqpoint{0.031702in}{0.033333in}}{%
\pgfpathmoveto{\pgfqpoint{0.000000in}{0.033333in}}%
\pgfpathlineto{\pgfqpoint{-0.007484in}{0.010301in}}%
\pgfpathlineto{\pgfqpoint{-0.031702in}{0.010301in}}%
\pgfpathlineto{\pgfqpoint{-0.012109in}{-0.003934in}}%
\pgfpathlineto{\pgfqpoint{-0.019593in}{-0.026967in}}%
\pgfpathlineto{\pgfqpoint{-0.000000in}{-0.012732in}}%
\pgfpathlineto{\pgfqpoint{0.019593in}{-0.026967in}}%
\pgfpathlineto{\pgfqpoint{0.012109in}{-0.003934in}}%
\pgfpathlineto{\pgfqpoint{0.031702in}{0.010301in}}%
\pgfpathlineto{\pgfqpoint{0.007484in}{0.010301in}}%
\pgfpathlineto{\pgfqpoint{0.000000in}{0.033333in}}%
\pgfpathclose%
\pgfusepath{stroke,fill}%
}%
\begin{pgfscope}%
\pgfsys@transformshift{2.909201in}{2.967083in}%
\pgfsys@useobject{currentmarker}{}%
\end{pgfscope}%
\end{pgfscope}%
\begin{pgfscope}%
\pgfpathrectangle{\pgfqpoint{0.765000in}{0.660000in}}{\pgfqpoint{4.620000in}{4.620000in}}%
\pgfusepath{clip}%
\pgfsetroundcap%
\pgfsetroundjoin%
\pgfsetlinewidth{1.204500pt}%
\definecolor{currentstroke}{rgb}{0.800000,0.176471,0.207843}%
\pgfsetstrokecolor{currentstroke}%
\pgfsetdash{}{0pt}%
\pgfpathmoveto{\pgfqpoint{2.801327in}{2.979236in}}%
\pgfusepath{stroke}%
\end{pgfscope}%
\begin{pgfscope}%
\pgfpathrectangle{\pgfqpoint{0.765000in}{0.660000in}}{\pgfqpoint{4.620000in}{4.620000in}}%
\pgfusepath{clip}%
\pgfsetbuttcap%
\pgfsetbeveljoin%
\definecolor{currentfill}{rgb}{0.800000,0.176471,0.207843}%
\pgfsetfillcolor{currentfill}%
\pgfsetlinewidth{1.003750pt}%
\definecolor{currentstroke}{rgb}{0.800000,0.176471,0.207843}%
\pgfsetstrokecolor{currentstroke}%
\pgfsetdash{}{0pt}%
\pgfsys@defobject{currentmarker}{\pgfqpoint{-0.031702in}{-0.026967in}}{\pgfqpoint{0.031702in}{0.033333in}}{%
\pgfpathmoveto{\pgfqpoint{0.000000in}{0.033333in}}%
\pgfpathlineto{\pgfqpoint{-0.007484in}{0.010301in}}%
\pgfpathlineto{\pgfqpoint{-0.031702in}{0.010301in}}%
\pgfpathlineto{\pgfqpoint{-0.012109in}{-0.003934in}}%
\pgfpathlineto{\pgfqpoint{-0.019593in}{-0.026967in}}%
\pgfpathlineto{\pgfqpoint{-0.000000in}{-0.012732in}}%
\pgfpathlineto{\pgfqpoint{0.019593in}{-0.026967in}}%
\pgfpathlineto{\pgfqpoint{0.012109in}{-0.003934in}}%
\pgfpathlineto{\pgfqpoint{0.031702in}{0.010301in}}%
\pgfpathlineto{\pgfqpoint{0.007484in}{0.010301in}}%
\pgfpathlineto{\pgfqpoint{0.000000in}{0.033333in}}%
\pgfpathclose%
\pgfusepath{stroke,fill}%
}%
\begin{pgfscope}%
\pgfsys@transformshift{2.801327in}{2.979236in}%
\pgfsys@useobject{currentmarker}{}%
\end{pgfscope}%
\end{pgfscope}%
\begin{pgfscope}%
\pgfpathrectangle{\pgfqpoint{0.765000in}{0.660000in}}{\pgfqpoint{4.620000in}{4.620000in}}%
\pgfusepath{clip}%
\pgfsetroundcap%
\pgfsetroundjoin%
\pgfsetlinewidth{1.204500pt}%
\definecolor{currentstroke}{rgb}{0.800000,0.176471,0.207843}%
\pgfsetstrokecolor{currentstroke}%
\pgfsetdash{}{0pt}%
\pgfpathmoveto{\pgfqpoint{3.241047in}{3.194937in}}%
\pgfusepath{stroke}%
\end{pgfscope}%
\begin{pgfscope}%
\pgfpathrectangle{\pgfqpoint{0.765000in}{0.660000in}}{\pgfqpoint{4.620000in}{4.620000in}}%
\pgfusepath{clip}%
\pgfsetbuttcap%
\pgfsetbeveljoin%
\definecolor{currentfill}{rgb}{0.800000,0.176471,0.207843}%
\pgfsetfillcolor{currentfill}%
\pgfsetlinewidth{1.003750pt}%
\definecolor{currentstroke}{rgb}{0.800000,0.176471,0.207843}%
\pgfsetstrokecolor{currentstroke}%
\pgfsetdash{}{0pt}%
\pgfsys@defobject{currentmarker}{\pgfqpoint{-0.031702in}{-0.026967in}}{\pgfqpoint{0.031702in}{0.033333in}}{%
\pgfpathmoveto{\pgfqpoint{0.000000in}{0.033333in}}%
\pgfpathlineto{\pgfqpoint{-0.007484in}{0.010301in}}%
\pgfpathlineto{\pgfqpoint{-0.031702in}{0.010301in}}%
\pgfpathlineto{\pgfqpoint{-0.012109in}{-0.003934in}}%
\pgfpathlineto{\pgfqpoint{-0.019593in}{-0.026967in}}%
\pgfpathlineto{\pgfqpoint{-0.000000in}{-0.012732in}}%
\pgfpathlineto{\pgfqpoint{0.019593in}{-0.026967in}}%
\pgfpathlineto{\pgfqpoint{0.012109in}{-0.003934in}}%
\pgfpathlineto{\pgfqpoint{0.031702in}{0.010301in}}%
\pgfpathlineto{\pgfqpoint{0.007484in}{0.010301in}}%
\pgfpathlineto{\pgfqpoint{0.000000in}{0.033333in}}%
\pgfpathclose%
\pgfusepath{stroke,fill}%
}%
\begin{pgfscope}%
\pgfsys@transformshift{3.241047in}{3.194937in}%
\pgfsys@useobject{currentmarker}{}%
\end{pgfscope}%
\end{pgfscope}%
\begin{pgfscope}%
\pgfpathrectangle{\pgfqpoint{0.765000in}{0.660000in}}{\pgfqpoint{4.620000in}{4.620000in}}%
\pgfusepath{clip}%
\pgfsetroundcap%
\pgfsetroundjoin%
\pgfsetlinewidth{1.204500pt}%
\definecolor{currentstroke}{rgb}{0.800000,0.176471,0.207843}%
\pgfsetstrokecolor{currentstroke}%
\pgfsetdash{}{0pt}%
\pgfpathmoveto{\pgfqpoint{3.362238in}{2.913253in}}%
\pgfusepath{stroke}%
\end{pgfscope}%
\begin{pgfscope}%
\pgfpathrectangle{\pgfqpoint{0.765000in}{0.660000in}}{\pgfqpoint{4.620000in}{4.620000in}}%
\pgfusepath{clip}%
\pgfsetbuttcap%
\pgfsetbeveljoin%
\definecolor{currentfill}{rgb}{0.800000,0.176471,0.207843}%
\pgfsetfillcolor{currentfill}%
\pgfsetlinewidth{1.003750pt}%
\definecolor{currentstroke}{rgb}{0.800000,0.176471,0.207843}%
\pgfsetstrokecolor{currentstroke}%
\pgfsetdash{}{0pt}%
\pgfsys@defobject{currentmarker}{\pgfqpoint{-0.031702in}{-0.026967in}}{\pgfqpoint{0.031702in}{0.033333in}}{%
\pgfpathmoveto{\pgfqpoint{0.000000in}{0.033333in}}%
\pgfpathlineto{\pgfqpoint{-0.007484in}{0.010301in}}%
\pgfpathlineto{\pgfqpoint{-0.031702in}{0.010301in}}%
\pgfpathlineto{\pgfqpoint{-0.012109in}{-0.003934in}}%
\pgfpathlineto{\pgfqpoint{-0.019593in}{-0.026967in}}%
\pgfpathlineto{\pgfqpoint{-0.000000in}{-0.012732in}}%
\pgfpathlineto{\pgfqpoint{0.019593in}{-0.026967in}}%
\pgfpathlineto{\pgfqpoint{0.012109in}{-0.003934in}}%
\pgfpathlineto{\pgfqpoint{0.031702in}{0.010301in}}%
\pgfpathlineto{\pgfqpoint{0.007484in}{0.010301in}}%
\pgfpathlineto{\pgfqpoint{0.000000in}{0.033333in}}%
\pgfpathclose%
\pgfusepath{stroke,fill}%
}%
\begin{pgfscope}%
\pgfsys@transformshift{3.362238in}{2.913253in}%
\pgfsys@useobject{currentmarker}{}%
\end{pgfscope}%
\end{pgfscope}%
\begin{pgfscope}%
\pgfpathrectangle{\pgfqpoint{0.765000in}{0.660000in}}{\pgfqpoint{4.620000in}{4.620000in}}%
\pgfusepath{clip}%
\pgfsetroundcap%
\pgfsetroundjoin%
\pgfsetlinewidth{1.204500pt}%
\definecolor{currentstroke}{rgb}{0.800000,0.176471,0.207843}%
\pgfsetstrokecolor{currentstroke}%
\pgfsetdash{}{0pt}%
\pgfpathmoveto{\pgfqpoint{3.158413in}{3.132506in}}%
\pgfusepath{stroke}%
\end{pgfscope}%
\begin{pgfscope}%
\pgfpathrectangle{\pgfqpoint{0.765000in}{0.660000in}}{\pgfqpoint{4.620000in}{4.620000in}}%
\pgfusepath{clip}%
\pgfsetbuttcap%
\pgfsetbeveljoin%
\definecolor{currentfill}{rgb}{0.800000,0.176471,0.207843}%
\pgfsetfillcolor{currentfill}%
\pgfsetlinewidth{1.003750pt}%
\definecolor{currentstroke}{rgb}{0.800000,0.176471,0.207843}%
\pgfsetstrokecolor{currentstroke}%
\pgfsetdash{}{0pt}%
\pgfsys@defobject{currentmarker}{\pgfqpoint{-0.031702in}{-0.026967in}}{\pgfqpoint{0.031702in}{0.033333in}}{%
\pgfpathmoveto{\pgfqpoint{0.000000in}{0.033333in}}%
\pgfpathlineto{\pgfqpoint{-0.007484in}{0.010301in}}%
\pgfpathlineto{\pgfqpoint{-0.031702in}{0.010301in}}%
\pgfpathlineto{\pgfqpoint{-0.012109in}{-0.003934in}}%
\pgfpathlineto{\pgfqpoint{-0.019593in}{-0.026967in}}%
\pgfpathlineto{\pgfqpoint{-0.000000in}{-0.012732in}}%
\pgfpathlineto{\pgfqpoint{0.019593in}{-0.026967in}}%
\pgfpathlineto{\pgfqpoint{0.012109in}{-0.003934in}}%
\pgfpathlineto{\pgfqpoint{0.031702in}{0.010301in}}%
\pgfpathlineto{\pgfqpoint{0.007484in}{0.010301in}}%
\pgfpathlineto{\pgfqpoint{0.000000in}{0.033333in}}%
\pgfpathclose%
\pgfusepath{stroke,fill}%
}%
\begin{pgfscope}%
\pgfsys@transformshift{3.158413in}{3.132506in}%
\pgfsys@useobject{currentmarker}{}%
\end{pgfscope}%
\end{pgfscope}%
\begin{pgfscope}%
\pgfpathrectangle{\pgfqpoint{0.765000in}{0.660000in}}{\pgfqpoint{4.620000in}{4.620000in}}%
\pgfusepath{clip}%
\pgfsetroundcap%
\pgfsetroundjoin%
\pgfsetlinewidth{1.204500pt}%
\definecolor{currentstroke}{rgb}{0.800000,0.176471,0.207843}%
\pgfsetstrokecolor{currentstroke}%
\pgfsetdash{}{0pt}%
\pgfpathmoveto{\pgfqpoint{2.899889in}{2.708067in}}%
\pgfusepath{stroke}%
\end{pgfscope}%
\begin{pgfscope}%
\pgfpathrectangle{\pgfqpoint{0.765000in}{0.660000in}}{\pgfqpoint{4.620000in}{4.620000in}}%
\pgfusepath{clip}%
\pgfsetbuttcap%
\pgfsetbeveljoin%
\definecolor{currentfill}{rgb}{0.800000,0.176471,0.207843}%
\pgfsetfillcolor{currentfill}%
\pgfsetlinewidth{1.003750pt}%
\definecolor{currentstroke}{rgb}{0.800000,0.176471,0.207843}%
\pgfsetstrokecolor{currentstroke}%
\pgfsetdash{}{0pt}%
\pgfsys@defobject{currentmarker}{\pgfqpoint{-0.031702in}{-0.026967in}}{\pgfqpoint{0.031702in}{0.033333in}}{%
\pgfpathmoveto{\pgfqpoint{0.000000in}{0.033333in}}%
\pgfpathlineto{\pgfqpoint{-0.007484in}{0.010301in}}%
\pgfpathlineto{\pgfqpoint{-0.031702in}{0.010301in}}%
\pgfpathlineto{\pgfqpoint{-0.012109in}{-0.003934in}}%
\pgfpathlineto{\pgfqpoint{-0.019593in}{-0.026967in}}%
\pgfpathlineto{\pgfqpoint{-0.000000in}{-0.012732in}}%
\pgfpathlineto{\pgfqpoint{0.019593in}{-0.026967in}}%
\pgfpathlineto{\pgfqpoint{0.012109in}{-0.003934in}}%
\pgfpathlineto{\pgfqpoint{0.031702in}{0.010301in}}%
\pgfpathlineto{\pgfqpoint{0.007484in}{0.010301in}}%
\pgfpathlineto{\pgfqpoint{0.000000in}{0.033333in}}%
\pgfpathclose%
\pgfusepath{stroke,fill}%
}%
\begin{pgfscope}%
\pgfsys@transformshift{2.899889in}{2.708067in}%
\pgfsys@useobject{currentmarker}{}%
\end{pgfscope}%
\end{pgfscope}%
\begin{pgfscope}%
\pgfpathrectangle{\pgfqpoint{0.765000in}{0.660000in}}{\pgfqpoint{4.620000in}{4.620000in}}%
\pgfusepath{clip}%
\pgfsetroundcap%
\pgfsetroundjoin%
\pgfsetlinewidth{1.204500pt}%
\definecolor{currentstroke}{rgb}{0.800000,0.176471,0.207843}%
\pgfsetstrokecolor{currentstroke}%
\pgfsetdash{}{0pt}%
\pgfpathmoveto{\pgfqpoint{3.089948in}{2.662213in}}%
\pgfusepath{stroke}%
\end{pgfscope}%
\begin{pgfscope}%
\pgfpathrectangle{\pgfqpoint{0.765000in}{0.660000in}}{\pgfqpoint{4.620000in}{4.620000in}}%
\pgfusepath{clip}%
\pgfsetbuttcap%
\pgfsetbeveljoin%
\definecolor{currentfill}{rgb}{0.800000,0.176471,0.207843}%
\pgfsetfillcolor{currentfill}%
\pgfsetlinewidth{1.003750pt}%
\definecolor{currentstroke}{rgb}{0.800000,0.176471,0.207843}%
\pgfsetstrokecolor{currentstroke}%
\pgfsetdash{}{0pt}%
\pgfsys@defobject{currentmarker}{\pgfqpoint{-0.031702in}{-0.026967in}}{\pgfqpoint{0.031702in}{0.033333in}}{%
\pgfpathmoveto{\pgfqpoint{0.000000in}{0.033333in}}%
\pgfpathlineto{\pgfqpoint{-0.007484in}{0.010301in}}%
\pgfpathlineto{\pgfqpoint{-0.031702in}{0.010301in}}%
\pgfpathlineto{\pgfqpoint{-0.012109in}{-0.003934in}}%
\pgfpathlineto{\pgfqpoint{-0.019593in}{-0.026967in}}%
\pgfpathlineto{\pgfqpoint{-0.000000in}{-0.012732in}}%
\pgfpathlineto{\pgfqpoint{0.019593in}{-0.026967in}}%
\pgfpathlineto{\pgfqpoint{0.012109in}{-0.003934in}}%
\pgfpathlineto{\pgfqpoint{0.031702in}{0.010301in}}%
\pgfpathlineto{\pgfqpoint{0.007484in}{0.010301in}}%
\pgfpathlineto{\pgfqpoint{0.000000in}{0.033333in}}%
\pgfpathclose%
\pgfusepath{stroke,fill}%
}%
\begin{pgfscope}%
\pgfsys@transformshift{3.089948in}{2.662213in}%
\pgfsys@useobject{currentmarker}{}%
\end{pgfscope}%
\end{pgfscope}%
\begin{pgfscope}%
\pgfpathrectangle{\pgfqpoint{0.765000in}{0.660000in}}{\pgfqpoint{4.620000in}{4.620000in}}%
\pgfusepath{clip}%
\pgfsetroundcap%
\pgfsetroundjoin%
\pgfsetlinewidth{1.204500pt}%
\definecolor{currentstroke}{rgb}{0.800000,0.176471,0.207843}%
\pgfsetstrokecolor{currentstroke}%
\pgfsetdash{}{0pt}%
\pgfpathmoveto{\pgfqpoint{3.327877in}{3.072645in}}%
\pgfusepath{stroke}%
\end{pgfscope}%
\begin{pgfscope}%
\pgfpathrectangle{\pgfqpoint{0.765000in}{0.660000in}}{\pgfqpoint{4.620000in}{4.620000in}}%
\pgfusepath{clip}%
\pgfsetbuttcap%
\pgfsetbeveljoin%
\definecolor{currentfill}{rgb}{0.800000,0.176471,0.207843}%
\pgfsetfillcolor{currentfill}%
\pgfsetlinewidth{1.003750pt}%
\definecolor{currentstroke}{rgb}{0.800000,0.176471,0.207843}%
\pgfsetstrokecolor{currentstroke}%
\pgfsetdash{}{0pt}%
\pgfsys@defobject{currentmarker}{\pgfqpoint{-0.031702in}{-0.026967in}}{\pgfqpoint{0.031702in}{0.033333in}}{%
\pgfpathmoveto{\pgfqpoint{0.000000in}{0.033333in}}%
\pgfpathlineto{\pgfqpoint{-0.007484in}{0.010301in}}%
\pgfpathlineto{\pgfqpoint{-0.031702in}{0.010301in}}%
\pgfpathlineto{\pgfqpoint{-0.012109in}{-0.003934in}}%
\pgfpathlineto{\pgfqpoint{-0.019593in}{-0.026967in}}%
\pgfpathlineto{\pgfqpoint{-0.000000in}{-0.012732in}}%
\pgfpathlineto{\pgfqpoint{0.019593in}{-0.026967in}}%
\pgfpathlineto{\pgfqpoint{0.012109in}{-0.003934in}}%
\pgfpathlineto{\pgfqpoint{0.031702in}{0.010301in}}%
\pgfpathlineto{\pgfqpoint{0.007484in}{0.010301in}}%
\pgfpathlineto{\pgfqpoint{0.000000in}{0.033333in}}%
\pgfpathclose%
\pgfusepath{stroke,fill}%
}%
\begin{pgfscope}%
\pgfsys@transformshift{3.327877in}{3.072645in}%
\pgfsys@useobject{currentmarker}{}%
\end{pgfscope}%
\end{pgfscope}%
\begin{pgfscope}%
\pgfpathrectangle{\pgfqpoint{0.765000in}{0.660000in}}{\pgfqpoint{4.620000in}{4.620000in}}%
\pgfusepath{clip}%
\pgfsetroundcap%
\pgfsetroundjoin%
\pgfsetlinewidth{1.204500pt}%
\definecolor{currentstroke}{rgb}{0.800000,0.176471,0.207843}%
\pgfsetstrokecolor{currentstroke}%
\pgfsetdash{}{0pt}%
\pgfpathmoveto{\pgfqpoint{2.947475in}{3.368506in}}%
\pgfusepath{stroke}%
\end{pgfscope}%
\begin{pgfscope}%
\pgfpathrectangle{\pgfqpoint{0.765000in}{0.660000in}}{\pgfqpoint{4.620000in}{4.620000in}}%
\pgfusepath{clip}%
\pgfsetbuttcap%
\pgfsetbeveljoin%
\definecolor{currentfill}{rgb}{0.800000,0.176471,0.207843}%
\pgfsetfillcolor{currentfill}%
\pgfsetlinewidth{1.003750pt}%
\definecolor{currentstroke}{rgb}{0.800000,0.176471,0.207843}%
\pgfsetstrokecolor{currentstroke}%
\pgfsetdash{}{0pt}%
\pgfsys@defobject{currentmarker}{\pgfqpoint{-0.031702in}{-0.026967in}}{\pgfqpoint{0.031702in}{0.033333in}}{%
\pgfpathmoveto{\pgfqpoint{0.000000in}{0.033333in}}%
\pgfpathlineto{\pgfqpoint{-0.007484in}{0.010301in}}%
\pgfpathlineto{\pgfqpoint{-0.031702in}{0.010301in}}%
\pgfpathlineto{\pgfqpoint{-0.012109in}{-0.003934in}}%
\pgfpathlineto{\pgfqpoint{-0.019593in}{-0.026967in}}%
\pgfpathlineto{\pgfqpoint{-0.000000in}{-0.012732in}}%
\pgfpathlineto{\pgfqpoint{0.019593in}{-0.026967in}}%
\pgfpathlineto{\pgfqpoint{0.012109in}{-0.003934in}}%
\pgfpathlineto{\pgfqpoint{0.031702in}{0.010301in}}%
\pgfpathlineto{\pgfqpoint{0.007484in}{0.010301in}}%
\pgfpathlineto{\pgfqpoint{0.000000in}{0.033333in}}%
\pgfpathclose%
\pgfusepath{stroke,fill}%
}%
\begin{pgfscope}%
\pgfsys@transformshift{2.947475in}{3.368506in}%
\pgfsys@useobject{currentmarker}{}%
\end{pgfscope}%
\end{pgfscope}%
\begin{pgfscope}%
\pgfpathrectangle{\pgfqpoint{0.765000in}{0.660000in}}{\pgfqpoint{4.620000in}{4.620000in}}%
\pgfusepath{clip}%
\pgfsetroundcap%
\pgfsetroundjoin%
\pgfsetlinewidth{1.204500pt}%
\definecolor{currentstroke}{rgb}{0.800000,0.176471,0.207843}%
\pgfsetstrokecolor{currentstroke}%
\pgfsetdash{}{0pt}%
\pgfpathmoveto{\pgfqpoint{3.551432in}{3.127490in}}%
\pgfusepath{stroke}%
\end{pgfscope}%
\begin{pgfscope}%
\pgfpathrectangle{\pgfqpoint{0.765000in}{0.660000in}}{\pgfqpoint{4.620000in}{4.620000in}}%
\pgfusepath{clip}%
\pgfsetbuttcap%
\pgfsetbeveljoin%
\definecolor{currentfill}{rgb}{0.800000,0.176471,0.207843}%
\pgfsetfillcolor{currentfill}%
\pgfsetlinewidth{1.003750pt}%
\definecolor{currentstroke}{rgb}{0.800000,0.176471,0.207843}%
\pgfsetstrokecolor{currentstroke}%
\pgfsetdash{}{0pt}%
\pgfsys@defobject{currentmarker}{\pgfqpoint{-0.031702in}{-0.026967in}}{\pgfqpoint{0.031702in}{0.033333in}}{%
\pgfpathmoveto{\pgfqpoint{0.000000in}{0.033333in}}%
\pgfpathlineto{\pgfqpoint{-0.007484in}{0.010301in}}%
\pgfpathlineto{\pgfqpoint{-0.031702in}{0.010301in}}%
\pgfpathlineto{\pgfqpoint{-0.012109in}{-0.003934in}}%
\pgfpathlineto{\pgfqpoint{-0.019593in}{-0.026967in}}%
\pgfpathlineto{\pgfqpoint{-0.000000in}{-0.012732in}}%
\pgfpathlineto{\pgfqpoint{0.019593in}{-0.026967in}}%
\pgfpathlineto{\pgfqpoint{0.012109in}{-0.003934in}}%
\pgfpathlineto{\pgfqpoint{0.031702in}{0.010301in}}%
\pgfpathlineto{\pgfqpoint{0.007484in}{0.010301in}}%
\pgfpathlineto{\pgfqpoint{0.000000in}{0.033333in}}%
\pgfpathclose%
\pgfusepath{stroke,fill}%
}%
\begin{pgfscope}%
\pgfsys@transformshift{3.551432in}{3.127490in}%
\pgfsys@useobject{currentmarker}{}%
\end{pgfscope}%
\end{pgfscope}%
\begin{pgfscope}%
\pgfpathrectangle{\pgfqpoint{0.765000in}{0.660000in}}{\pgfqpoint{4.620000in}{4.620000in}}%
\pgfusepath{clip}%
\pgfsetroundcap%
\pgfsetroundjoin%
\pgfsetlinewidth{1.204500pt}%
\definecolor{currentstroke}{rgb}{0.000000,0.000000,0.000000}%
\pgfsetstrokecolor{currentstroke}%
\pgfsetdash{}{0pt}%
\pgfpathmoveto{\pgfqpoint{2.463018in}{2.599135in}}%
\pgfusepath{stroke}%
\end{pgfscope}%
\begin{pgfscope}%
\pgfpathrectangle{\pgfqpoint{0.765000in}{0.660000in}}{\pgfqpoint{4.620000in}{4.620000in}}%
\pgfusepath{clip}%
\pgfsetbuttcap%
\pgfsetroundjoin%
\definecolor{currentfill}{rgb}{0.000000,0.000000,0.000000}%
\pgfsetfillcolor{currentfill}%
\pgfsetlinewidth{1.003750pt}%
\definecolor{currentstroke}{rgb}{0.000000,0.000000,0.000000}%
\pgfsetstrokecolor{currentstroke}%
\pgfsetdash{}{0pt}%
\pgfsys@defobject{currentmarker}{\pgfqpoint{-0.016667in}{-0.016667in}}{\pgfqpoint{0.016667in}{0.016667in}}{%
\pgfpathmoveto{\pgfqpoint{0.000000in}{-0.016667in}}%
\pgfpathcurveto{\pgfqpoint{0.004420in}{-0.016667in}}{\pgfqpoint{0.008660in}{-0.014911in}}{\pgfqpoint{0.011785in}{-0.011785in}}%
\pgfpathcurveto{\pgfqpoint{0.014911in}{-0.008660in}}{\pgfqpoint{0.016667in}{-0.004420in}}{\pgfqpoint{0.016667in}{0.000000in}}%
\pgfpathcurveto{\pgfqpoint{0.016667in}{0.004420in}}{\pgfqpoint{0.014911in}{0.008660in}}{\pgfqpoint{0.011785in}{0.011785in}}%
\pgfpathcurveto{\pgfqpoint{0.008660in}{0.014911in}}{\pgfqpoint{0.004420in}{0.016667in}}{\pgfqpoint{0.000000in}{0.016667in}}%
\pgfpathcurveto{\pgfqpoint{-0.004420in}{0.016667in}}{\pgfqpoint{-0.008660in}{0.014911in}}{\pgfqpoint{-0.011785in}{0.011785in}}%
\pgfpathcurveto{\pgfqpoint{-0.014911in}{0.008660in}}{\pgfqpoint{-0.016667in}{0.004420in}}{\pgfqpoint{-0.016667in}{0.000000in}}%
\pgfpathcurveto{\pgfqpoint{-0.016667in}{-0.004420in}}{\pgfqpoint{-0.014911in}{-0.008660in}}{\pgfqpoint{-0.011785in}{-0.011785in}}%
\pgfpathcurveto{\pgfqpoint{-0.008660in}{-0.014911in}}{\pgfqpoint{-0.004420in}{-0.016667in}}{\pgfqpoint{0.000000in}{-0.016667in}}%
\pgfpathlineto{\pgfqpoint{0.000000in}{-0.016667in}}%
\pgfpathclose%
\pgfusepath{stroke,fill}%
}%
\begin{pgfscope}%
\pgfsys@transformshift{2.463018in}{2.599135in}%
\pgfsys@useobject{currentmarker}{}%
\end{pgfscope}%
\end{pgfscope}%
\begin{pgfscope}%
\pgfpathrectangle{\pgfqpoint{0.765000in}{0.660000in}}{\pgfqpoint{4.620000in}{4.620000in}}%
\pgfusepath{clip}%
\pgfsetroundcap%
\pgfsetroundjoin%
\pgfsetlinewidth{1.204500pt}%
\definecolor{currentstroke}{rgb}{0.000000,0.000000,0.000000}%
\pgfsetstrokecolor{currentstroke}%
\pgfsetdash{}{0pt}%
\pgfpathmoveto{\pgfqpoint{2.463018in}{2.599135in}}%
\pgfpathlineto{\pgfqpoint{3.043939in}{2.326881in}}%
\pgfusepath{stroke}%
\end{pgfscope}%
\begin{pgfscope}%
\pgfpathrectangle{\pgfqpoint{0.765000in}{0.660000in}}{\pgfqpoint{4.620000in}{4.620000in}}%
\pgfusepath{clip}%
\pgfsetroundcap%
\pgfsetroundjoin%
\pgfsetlinewidth{1.204500pt}%
\definecolor{currentstroke}{rgb}{0.000000,0.000000,0.000000}%
\pgfsetstrokecolor{currentstroke}%
\pgfsetdash{}{0pt}%
\pgfpathmoveto{\pgfqpoint{2.463018in}{2.599135in}}%
\pgfpathlineto{\pgfqpoint{2.463018in}{3.462142in}}%
\pgfusepath{stroke}%
\end{pgfscope}%
\begin{pgfscope}%
\pgfpathrectangle{\pgfqpoint{0.765000in}{0.660000in}}{\pgfqpoint{4.620000in}{4.620000in}}%
\pgfusepath{clip}%
\pgfsetroundcap%
\pgfsetroundjoin%
\pgfsetlinewidth{1.204500pt}%
\definecolor{currentstroke}{rgb}{0.000000,0.000000,0.000000}%
\pgfsetstrokecolor{currentstroke}%
\pgfsetdash{}{0pt}%
\pgfpathmoveto{\pgfqpoint{2.463018in}{2.599135in}}%
\pgfpathlineto{\pgfqpoint{0.755000in}{1.798656in}}%
\pgfusepath{stroke}%
\end{pgfscope}%
\begin{pgfscope}%
\pgfpathrectangle{\pgfqpoint{0.765000in}{0.660000in}}{\pgfqpoint{4.620000in}{4.620000in}}%
\pgfusepath{clip}%
\pgfsetroundcap%
\pgfsetroundjoin%
\pgfsetlinewidth{1.204500pt}%
\definecolor{currentstroke}{rgb}{0.000000,0.000000,0.000000}%
\pgfsetstrokecolor{currentstroke}%
\pgfsetdash{}{0pt}%
\pgfpathmoveto{\pgfqpoint{3.043939in}{2.326881in}}%
\pgfusepath{stroke}%
\end{pgfscope}%
\begin{pgfscope}%
\pgfpathrectangle{\pgfqpoint{0.765000in}{0.660000in}}{\pgfqpoint{4.620000in}{4.620000in}}%
\pgfusepath{clip}%
\pgfsetbuttcap%
\pgfsetroundjoin%
\definecolor{currentfill}{rgb}{0.000000,0.000000,0.000000}%
\pgfsetfillcolor{currentfill}%
\pgfsetlinewidth{1.003750pt}%
\definecolor{currentstroke}{rgb}{0.000000,0.000000,0.000000}%
\pgfsetstrokecolor{currentstroke}%
\pgfsetdash{}{0pt}%
\pgfsys@defobject{currentmarker}{\pgfqpoint{-0.016667in}{-0.016667in}}{\pgfqpoint{0.016667in}{0.016667in}}{%
\pgfpathmoveto{\pgfqpoint{0.000000in}{-0.016667in}}%
\pgfpathcurveto{\pgfqpoint{0.004420in}{-0.016667in}}{\pgfqpoint{0.008660in}{-0.014911in}}{\pgfqpoint{0.011785in}{-0.011785in}}%
\pgfpathcurveto{\pgfqpoint{0.014911in}{-0.008660in}}{\pgfqpoint{0.016667in}{-0.004420in}}{\pgfqpoint{0.016667in}{0.000000in}}%
\pgfpathcurveto{\pgfqpoint{0.016667in}{0.004420in}}{\pgfqpoint{0.014911in}{0.008660in}}{\pgfqpoint{0.011785in}{0.011785in}}%
\pgfpathcurveto{\pgfqpoint{0.008660in}{0.014911in}}{\pgfqpoint{0.004420in}{0.016667in}}{\pgfqpoint{0.000000in}{0.016667in}}%
\pgfpathcurveto{\pgfqpoint{-0.004420in}{0.016667in}}{\pgfqpoint{-0.008660in}{0.014911in}}{\pgfqpoint{-0.011785in}{0.011785in}}%
\pgfpathcurveto{\pgfqpoint{-0.014911in}{0.008660in}}{\pgfqpoint{-0.016667in}{0.004420in}}{\pgfqpoint{-0.016667in}{0.000000in}}%
\pgfpathcurveto{\pgfqpoint{-0.016667in}{-0.004420in}}{\pgfqpoint{-0.014911in}{-0.008660in}}{\pgfqpoint{-0.011785in}{-0.011785in}}%
\pgfpathcurveto{\pgfqpoint{-0.008660in}{-0.014911in}}{\pgfqpoint{-0.004420in}{-0.016667in}}{\pgfqpoint{0.000000in}{-0.016667in}}%
\pgfpathlineto{\pgfqpoint{0.000000in}{-0.016667in}}%
\pgfpathclose%
\pgfusepath{stroke,fill}%
}%
\begin{pgfscope}%
\pgfsys@transformshift{3.043939in}{2.326881in}%
\pgfsys@useobject{currentmarker}{}%
\end{pgfscope}%
\end{pgfscope}%
\begin{pgfscope}%
\pgfpathrectangle{\pgfqpoint{0.765000in}{0.660000in}}{\pgfqpoint{4.620000in}{4.620000in}}%
\pgfusepath{clip}%
\pgfsetroundcap%
\pgfsetroundjoin%
\pgfsetlinewidth{1.204500pt}%
\definecolor{currentstroke}{rgb}{0.000000,0.000000,0.000000}%
\pgfsetstrokecolor{currentstroke}%
\pgfsetdash{}{0pt}%
\pgfpathmoveto{\pgfqpoint{3.043939in}{2.326881in}}%
\pgfpathlineto{\pgfqpoint{2.463018in}{2.599135in}}%
\pgfusepath{stroke}%
\end{pgfscope}%
\begin{pgfscope}%
\pgfpathrectangle{\pgfqpoint{0.765000in}{0.660000in}}{\pgfqpoint{4.620000in}{4.620000in}}%
\pgfusepath{clip}%
\pgfsetroundcap%
\pgfsetroundjoin%
\pgfsetlinewidth{1.204500pt}%
\definecolor{currentstroke}{rgb}{0.000000,0.000000,0.000000}%
\pgfsetstrokecolor{currentstroke}%
\pgfsetdash{}{0pt}%
\pgfpathmoveto{\pgfqpoint{3.043939in}{2.326881in}}%
\pgfpathlineto{\pgfqpoint{3.889745in}{2.723276in}}%
\pgfusepath{stroke}%
\end{pgfscope}%
\begin{pgfscope}%
\pgfpathrectangle{\pgfqpoint{0.765000in}{0.660000in}}{\pgfqpoint{4.620000in}{4.620000in}}%
\pgfusepath{clip}%
\pgfsetbuttcap%
\pgfsetroundjoin%
\pgfsetlinewidth{1.204500pt}%
\definecolor{currentstroke}{rgb}{0.827451,0.827451,0.827451}%
\pgfsetstrokecolor{currentstroke}%
\pgfsetdash{{1.200000pt}{1.980000pt}}{0.000000pt}%
\pgfpathmoveto{\pgfqpoint{3.043939in}{2.326881in}}%
\pgfpathlineto{\pgfqpoint{3.043939in}{0.650000in}}%
\pgfusepath{stroke}%
\end{pgfscope}%
\begin{pgfscope}%
\pgfpathrectangle{\pgfqpoint{0.765000in}{0.660000in}}{\pgfqpoint{4.620000in}{4.620000in}}%
\pgfusepath{clip}%
\pgfsetroundcap%
\pgfsetroundjoin%
\pgfsetlinewidth{1.204500pt}%
\definecolor{currentstroke}{rgb}{0.000000,0.000000,0.000000}%
\pgfsetstrokecolor{currentstroke}%
\pgfsetdash{}{0pt}%
\pgfpathmoveto{\pgfqpoint{2.463018in}{3.462142in}}%
\pgfusepath{stroke}%
\end{pgfscope}%
\begin{pgfscope}%
\pgfpathrectangle{\pgfqpoint{0.765000in}{0.660000in}}{\pgfqpoint{4.620000in}{4.620000in}}%
\pgfusepath{clip}%
\pgfsetbuttcap%
\pgfsetroundjoin%
\definecolor{currentfill}{rgb}{0.000000,0.000000,0.000000}%
\pgfsetfillcolor{currentfill}%
\pgfsetlinewidth{1.003750pt}%
\definecolor{currentstroke}{rgb}{0.000000,0.000000,0.000000}%
\pgfsetstrokecolor{currentstroke}%
\pgfsetdash{}{0pt}%
\pgfsys@defobject{currentmarker}{\pgfqpoint{-0.016667in}{-0.016667in}}{\pgfqpoint{0.016667in}{0.016667in}}{%
\pgfpathmoveto{\pgfqpoint{0.000000in}{-0.016667in}}%
\pgfpathcurveto{\pgfqpoint{0.004420in}{-0.016667in}}{\pgfqpoint{0.008660in}{-0.014911in}}{\pgfqpoint{0.011785in}{-0.011785in}}%
\pgfpathcurveto{\pgfqpoint{0.014911in}{-0.008660in}}{\pgfqpoint{0.016667in}{-0.004420in}}{\pgfqpoint{0.016667in}{0.000000in}}%
\pgfpathcurveto{\pgfqpoint{0.016667in}{0.004420in}}{\pgfqpoint{0.014911in}{0.008660in}}{\pgfqpoint{0.011785in}{0.011785in}}%
\pgfpathcurveto{\pgfqpoint{0.008660in}{0.014911in}}{\pgfqpoint{0.004420in}{0.016667in}}{\pgfqpoint{0.000000in}{0.016667in}}%
\pgfpathcurveto{\pgfqpoint{-0.004420in}{0.016667in}}{\pgfqpoint{-0.008660in}{0.014911in}}{\pgfqpoint{-0.011785in}{0.011785in}}%
\pgfpathcurveto{\pgfqpoint{-0.014911in}{0.008660in}}{\pgfqpoint{-0.016667in}{0.004420in}}{\pgfqpoint{-0.016667in}{0.000000in}}%
\pgfpathcurveto{\pgfqpoint{-0.016667in}{-0.004420in}}{\pgfqpoint{-0.014911in}{-0.008660in}}{\pgfqpoint{-0.011785in}{-0.011785in}}%
\pgfpathcurveto{\pgfqpoint{-0.008660in}{-0.014911in}}{\pgfqpoint{-0.004420in}{-0.016667in}}{\pgfqpoint{0.000000in}{-0.016667in}}%
\pgfpathlineto{\pgfqpoint{0.000000in}{-0.016667in}}%
\pgfpathclose%
\pgfusepath{stroke,fill}%
}%
\begin{pgfscope}%
\pgfsys@transformshift{2.463018in}{3.462142in}%
\pgfsys@useobject{currentmarker}{}%
\end{pgfscope}%
\end{pgfscope}%
\begin{pgfscope}%
\pgfpathrectangle{\pgfqpoint{0.765000in}{0.660000in}}{\pgfqpoint{4.620000in}{4.620000in}}%
\pgfusepath{clip}%
\pgfsetroundcap%
\pgfsetroundjoin%
\pgfsetlinewidth{1.204500pt}%
\definecolor{currentstroke}{rgb}{0.000000,0.000000,0.000000}%
\pgfsetstrokecolor{currentstroke}%
\pgfsetdash{}{0pt}%
\pgfpathmoveto{\pgfqpoint{2.463018in}{3.462142in}}%
\pgfpathlineto{\pgfqpoint{2.463018in}{2.599135in}}%
\pgfusepath{stroke}%
\end{pgfscope}%
\begin{pgfscope}%
\pgfpathrectangle{\pgfqpoint{0.765000in}{0.660000in}}{\pgfqpoint{4.620000in}{4.620000in}}%
\pgfusepath{clip}%
\pgfsetroundcap%
\pgfsetroundjoin%
\pgfsetlinewidth{1.204500pt}%
\definecolor{currentstroke}{rgb}{0.000000,0.000000,0.000000}%
\pgfsetstrokecolor{currentstroke}%
\pgfsetdash{}{0pt}%
\pgfpathmoveto{\pgfqpoint{2.463018in}{3.462142in}}%
\pgfpathlineto{\pgfqpoint{2.992075in}{3.710089in}}%
\pgfusepath{stroke}%
\end{pgfscope}%
\begin{pgfscope}%
\pgfpathrectangle{\pgfqpoint{0.765000in}{0.660000in}}{\pgfqpoint{4.620000in}{4.620000in}}%
\pgfusepath{clip}%
\pgfsetbuttcap%
\pgfsetroundjoin%
\pgfsetlinewidth{1.204500pt}%
\definecolor{currentstroke}{rgb}{0.827451,0.827451,0.827451}%
\pgfsetstrokecolor{currentstroke}%
\pgfsetdash{{1.200000pt}{1.980000pt}}{0.000000pt}%
\pgfpathmoveto{\pgfqpoint{2.463018in}{3.462142in}}%
\pgfpathlineto{\pgfqpoint{0.755000in}{4.262620in}}%
\pgfusepath{stroke}%
\end{pgfscope}%
\begin{pgfscope}%
\pgfpathrectangle{\pgfqpoint{0.765000in}{0.660000in}}{\pgfqpoint{4.620000in}{4.620000in}}%
\pgfusepath{clip}%
\pgfsetroundcap%
\pgfsetroundjoin%
\pgfsetlinewidth{1.204500pt}%
\definecolor{currentstroke}{rgb}{0.000000,0.000000,0.000000}%
\pgfsetstrokecolor{currentstroke}%
\pgfsetdash{}{0pt}%
\pgfpathmoveto{\pgfqpoint{3.889745in}{2.723276in}}%
\pgfusepath{stroke}%
\end{pgfscope}%
\begin{pgfscope}%
\pgfpathrectangle{\pgfqpoint{0.765000in}{0.660000in}}{\pgfqpoint{4.620000in}{4.620000in}}%
\pgfusepath{clip}%
\pgfsetbuttcap%
\pgfsetroundjoin%
\definecolor{currentfill}{rgb}{0.000000,0.000000,0.000000}%
\pgfsetfillcolor{currentfill}%
\pgfsetlinewidth{1.003750pt}%
\definecolor{currentstroke}{rgb}{0.000000,0.000000,0.000000}%
\pgfsetstrokecolor{currentstroke}%
\pgfsetdash{}{0pt}%
\pgfsys@defobject{currentmarker}{\pgfqpoint{-0.016667in}{-0.016667in}}{\pgfqpoint{0.016667in}{0.016667in}}{%
\pgfpathmoveto{\pgfqpoint{0.000000in}{-0.016667in}}%
\pgfpathcurveto{\pgfqpoint{0.004420in}{-0.016667in}}{\pgfqpoint{0.008660in}{-0.014911in}}{\pgfqpoint{0.011785in}{-0.011785in}}%
\pgfpathcurveto{\pgfqpoint{0.014911in}{-0.008660in}}{\pgfqpoint{0.016667in}{-0.004420in}}{\pgfqpoint{0.016667in}{0.000000in}}%
\pgfpathcurveto{\pgfqpoint{0.016667in}{0.004420in}}{\pgfqpoint{0.014911in}{0.008660in}}{\pgfqpoint{0.011785in}{0.011785in}}%
\pgfpathcurveto{\pgfqpoint{0.008660in}{0.014911in}}{\pgfqpoint{0.004420in}{0.016667in}}{\pgfqpoint{0.000000in}{0.016667in}}%
\pgfpathcurveto{\pgfqpoint{-0.004420in}{0.016667in}}{\pgfqpoint{-0.008660in}{0.014911in}}{\pgfqpoint{-0.011785in}{0.011785in}}%
\pgfpathcurveto{\pgfqpoint{-0.014911in}{0.008660in}}{\pgfqpoint{-0.016667in}{0.004420in}}{\pgfqpoint{-0.016667in}{0.000000in}}%
\pgfpathcurveto{\pgfqpoint{-0.016667in}{-0.004420in}}{\pgfqpoint{-0.014911in}{-0.008660in}}{\pgfqpoint{-0.011785in}{-0.011785in}}%
\pgfpathcurveto{\pgfqpoint{-0.008660in}{-0.014911in}}{\pgfqpoint{-0.004420in}{-0.016667in}}{\pgfqpoint{0.000000in}{-0.016667in}}%
\pgfpathlineto{\pgfqpoint{0.000000in}{-0.016667in}}%
\pgfpathclose%
\pgfusepath{stroke,fill}%
}%
\begin{pgfscope}%
\pgfsys@transformshift{3.889745in}{2.723276in}%
\pgfsys@useobject{currentmarker}{}%
\end{pgfscope}%
\end{pgfscope}%
\begin{pgfscope}%
\pgfpathrectangle{\pgfqpoint{0.765000in}{0.660000in}}{\pgfqpoint{4.620000in}{4.620000in}}%
\pgfusepath{clip}%
\pgfsetroundcap%
\pgfsetroundjoin%
\pgfsetlinewidth{1.204500pt}%
\definecolor{currentstroke}{rgb}{0.000000,0.000000,0.000000}%
\pgfsetstrokecolor{currentstroke}%
\pgfsetdash{}{0pt}%
\pgfpathmoveto{\pgfqpoint{3.889745in}{2.723276in}}%
\pgfpathlineto{\pgfqpoint{3.043939in}{2.326881in}}%
\pgfusepath{stroke}%
\end{pgfscope}%
\begin{pgfscope}%
\pgfpathrectangle{\pgfqpoint{0.765000in}{0.660000in}}{\pgfqpoint{4.620000in}{4.620000in}}%
\pgfusepath{clip}%
\pgfsetroundcap%
\pgfsetroundjoin%
\pgfsetlinewidth{1.204500pt}%
\definecolor{currentstroke}{rgb}{0.000000,0.000000,0.000000}%
\pgfsetstrokecolor{currentstroke}%
\pgfsetdash{}{0pt}%
\pgfpathmoveto{\pgfqpoint{3.889745in}{2.723276in}}%
\pgfpathlineto{\pgfqpoint{3.889745in}{3.289387in}}%
\pgfusepath{stroke}%
\end{pgfscope}%
\begin{pgfscope}%
\pgfpathrectangle{\pgfqpoint{0.765000in}{0.660000in}}{\pgfqpoint{4.620000in}{4.620000in}}%
\pgfusepath{clip}%
\pgfsetroundcap%
\pgfsetroundjoin%
\pgfsetlinewidth{1.204500pt}%
\definecolor{currentstroke}{rgb}{0.000000,0.000000,0.000000}%
\pgfsetstrokecolor{currentstroke}%
\pgfsetdash{}{0pt}%
\pgfpathmoveto{\pgfqpoint{3.889745in}{2.723276in}}%
\pgfpathlineto{\pgfqpoint{5.395000in}{2.017824in}}%
\pgfusepath{stroke}%
\end{pgfscope}%
\begin{pgfscope}%
\pgfpathrectangle{\pgfqpoint{0.765000in}{0.660000in}}{\pgfqpoint{4.620000in}{4.620000in}}%
\pgfusepath{clip}%
\pgfsetroundcap%
\pgfsetroundjoin%
\pgfsetlinewidth{1.204500pt}%
\definecolor{currentstroke}{rgb}{0.000000,0.000000,0.000000}%
\pgfsetstrokecolor{currentstroke}%
\pgfsetdash{}{0pt}%
\pgfpathmoveto{\pgfqpoint{3.889745in}{3.289387in}}%
\pgfusepath{stroke}%
\end{pgfscope}%
\begin{pgfscope}%
\pgfpathrectangle{\pgfqpoint{0.765000in}{0.660000in}}{\pgfqpoint{4.620000in}{4.620000in}}%
\pgfusepath{clip}%
\pgfsetbuttcap%
\pgfsetroundjoin%
\definecolor{currentfill}{rgb}{0.000000,0.000000,0.000000}%
\pgfsetfillcolor{currentfill}%
\pgfsetlinewidth{1.003750pt}%
\definecolor{currentstroke}{rgb}{0.000000,0.000000,0.000000}%
\pgfsetstrokecolor{currentstroke}%
\pgfsetdash{}{0pt}%
\pgfsys@defobject{currentmarker}{\pgfqpoint{-0.016667in}{-0.016667in}}{\pgfqpoint{0.016667in}{0.016667in}}{%
\pgfpathmoveto{\pgfqpoint{0.000000in}{-0.016667in}}%
\pgfpathcurveto{\pgfqpoint{0.004420in}{-0.016667in}}{\pgfqpoint{0.008660in}{-0.014911in}}{\pgfqpoint{0.011785in}{-0.011785in}}%
\pgfpathcurveto{\pgfqpoint{0.014911in}{-0.008660in}}{\pgfqpoint{0.016667in}{-0.004420in}}{\pgfqpoint{0.016667in}{0.000000in}}%
\pgfpathcurveto{\pgfqpoint{0.016667in}{0.004420in}}{\pgfqpoint{0.014911in}{0.008660in}}{\pgfqpoint{0.011785in}{0.011785in}}%
\pgfpathcurveto{\pgfqpoint{0.008660in}{0.014911in}}{\pgfqpoint{0.004420in}{0.016667in}}{\pgfqpoint{0.000000in}{0.016667in}}%
\pgfpathcurveto{\pgfqpoint{-0.004420in}{0.016667in}}{\pgfqpoint{-0.008660in}{0.014911in}}{\pgfqpoint{-0.011785in}{0.011785in}}%
\pgfpathcurveto{\pgfqpoint{-0.014911in}{0.008660in}}{\pgfqpoint{-0.016667in}{0.004420in}}{\pgfqpoint{-0.016667in}{0.000000in}}%
\pgfpathcurveto{\pgfqpoint{-0.016667in}{-0.004420in}}{\pgfqpoint{-0.014911in}{-0.008660in}}{\pgfqpoint{-0.011785in}{-0.011785in}}%
\pgfpathcurveto{\pgfqpoint{-0.008660in}{-0.014911in}}{\pgfqpoint{-0.004420in}{-0.016667in}}{\pgfqpoint{0.000000in}{-0.016667in}}%
\pgfpathlineto{\pgfqpoint{0.000000in}{-0.016667in}}%
\pgfpathclose%
\pgfusepath{stroke,fill}%
}%
\begin{pgfscope}%
\pgfsys@transformshift{3.889745in}{3.289387in}%
\pgfsys@useobject{currentmarker}{}%
\end{pgfscope}%
\end{pgfscope}%
\begin{pgfscope}%
\pgfpathrectangle{\pgfqpoint{0.765000in}{0.660000in}}{\pgfqpoint{4.620000in}{4.620000in}}%
\pgfusepath{clip}%
\pgfsetroundcap%
\pgfsetroundjoin%
\pgfsetlinewidth{1.204500pt}%
\definecolor{currentstroke}{rgb}{0.000000,0.000000,0.000000}%
\pgfsetstrokecolor{currentstroke}%
\pgfsetdash{}{0pt}%
\pgfpathmoveto{\pgfqpoint{3.889745in}{3.289387in}}%
\pgfpathlineto{\pgfqpoint{3.889745in}{2.723276in}}%
\pgfusepath{stroke}%
\end{pgfscope}%
\begin{pgfscope}%
\pgfpathrectangle{\pgfqpoint{0.765000in}{0.660000in}}{\pgfqpoint{4.620000in}{4.620000in}}%
\pgfusepath{clip}%
\pgfsetroundcap%
\pgfsetroundjoin%
\pgfsetlinewidth{1.204500pt}%
\definecolor{currentstroke}{rgb}{0.000000,0.000000,0.000000}%
\pgfsetstrokecolor{currentstroke}%
\pgfsetdash{}{0pt}%
\pgfpathmoveto{\pgfqpoint{3.889745in}{3.289387in}}%
\pgfpathlineto{\pgfqpoint{2.992075in}{3.710089in}}%
\pgfusepath{stroke}%
\end{pgfscope}%
\begin{pgfscope}%
\pgfpathrectangle{\pgfqpoint{0.765000in}{0.660000in}}{\pgfqpoint{4.620000in}{4.620000in}}%
\pgfusepath{clip}%
\pgfsetbuttcap%
\pgfsetroundjoin%
\pgfsetlinewidth{1.204500pt}%
\definecolor{currentstroke}{rgb}{0.827451,0.827451,0.827451}%
\pgfsetstrokecolor{currentstroke}%
\pgfsetdash{{1.200000pt}{1.980000pt}}{0.000000pt}%
\pgfpathmoveto{\pgfqpoint{3.889745in}{3.289387in}}%
\pgfpathlineto{\pgfqpoint{5.395000in}{3.994840in}}%
\pgfusepath{stroke}%
\end{pgfscope}%
\begin{pgfscope}%
\pgfpathrectangle{\pgfqpoint{0.765000in}{0.660000in}}{\pgfqpoint{4.620000in}{4.620000in}}%
\pgfusepath{clip}%
\pgfsetroundcap%
\pgfsetroundjoin%
\pgfsetlinewidth{1.204500pt}%
\definecolor{currentstroke}{rgb}{0.000000,0.000000,0.000000}%
\pgfsetstrokecolor{currentstroke}%
\pgfsetdash{}{0pt}%
\pgfpathmoveto{\pgfqpoint{2.992075in}{3.710089in}}%
\pgfusepath{stroke}%
\end{pgfscope}%
\begin{pgfscope}%
\pgfpathrectangle{\pgfqpoint{0.765000in}{0.660000in}}{\pgfqpoint{4.620000in}{4.620000in}}%
\pgfusepath{clip}%
\pgfsetbuttcap%
\pgfsetroundjoin%
\definecolor{currentfill}{rgb}{0.000000,0.000000,0.000000}%
\pgfsetfillcolor{currentfill}%
\pgfsetlinewidth{1.003750pt}%
\definecolor{currentstroke}{rgb}{0.000000,0.000000,0.000000}%
\pgfsetstrokecolor{currentstroke}%
\pgfsetdash{}{0pt}%
\pgfsys@defobject{currentmarker}{\pgfqpoint{-0.016667in}{-0.016667in}}{\pgfqpoint{0.016667in}{0.016667in}}{%
\pgfpathmoveto{\pgfqpoint{0.000000in}{-0.016667in}}%
\pgfpathcurveto{\pgfqpoint{0.004420in}{-0.016667in}}{\pgfqpoint{0.008660in}{-0.014911in}}{\pgfqpoint{0.011785in}{-0.011785in}}%
\pgfpathcurveto{\pgfqpoint{0.014911in}{-0.008660in}}{\pgfqpoint{0.016667in}{-0.004420in}}{\pgfqpoint{0.016667in}{0.000000in}}%
\pgfpathcurveto{\pgfqpoint{0.016667in}{0.004420in}}{\pgfqpoint{0.014911in}{0.008660in}}{\pgfqpoint{0.011785in}{0.011785in}}%
\pgfpathcurveto{\pgfqpoint{0.008660in}{0.014911in}}{\pgfqpoint{0.004420in}{0.016667in}}{\pgfqpoint{0.000000in}{0.016667in}}%
\pgfpathcurveto{\pgfqpoint{-0.004420in}{0.016667in}}{\pgfqpoint{-0.008660in}{0.014911in}}{\pgfqpoint{-0.011785in}{0.011785in}}%
\pgfpathcurveto{\pgfqpoint{-0.014911in}{0.008660in}}{\pgfqpoint{-0.016667in}{0.004420in}}{\pgfqpoint{-0.016667in}{0.000000in}}%
\pgfpathcurveto{\pgfqpoint{-0.016667in}{-0.004420in}}{\pgfqpoint{-0.014911in}{-0.008660in}}{\pgfqpoint{-0.011785in}{-0.011785in}}%
\pgfpathcurveto{\pgfqpoint{-0.008660in}{-0.014911in}}{\pgfqpoint{-0.004420in}{-0.016667in}}{\pgfqpoint{0.000000in}{-0.016667in}}%
\pgfpathlineto{\pgfqpoint{0.000000in}{-0.016667in}}%
\pgfpathclose%
\pgfusepath{stroke,fill}%
}%
\begin{pgfscope}%
\pgfsys@transformshift{2.992075in}{3.710089in}%
\pgfsys@useobject{currentmarker}{}%
\end{pgfscope}%
\end{pgfscope}%
\begin{pgfscope}%
\pgfpathrectangle{\pgfqpoint{0.765000in}{0.660000in}}{\pgfqpoint{4.620000in}{4.620000in}}%
\pgfusepath{clip}%
\pgfsetroundcap%
\pgfsetroundjoin%
\pgfsetlinewidth{1.204500pt}%
\definecolor{currentstroke}{rgb}{0.000000,0.000000,0.000000}%
\pgfsetstrokecolor{currentstroke}%
\pgfsetdash{}{0pt}%
\pgfpathmoveto{\pgfqpoint{2.992075in}{3.710089in}}%
\pgfpathlineto{\pgfqpoint{2.463018in}{3.462142in}}%
\pgfusepath{stroke}%
\end{pgfscope}%
\begin{pgfscope}%
\pgfpathrectangle{\pgfqpoint{0.765000in}{0.660000in}}{\pgfqpoint{4.620000in}{4.620000in}}%
\pgfusepath{clip}%
\pgfsetroundcap%
\pgfsetroundjoin%
\pgfsetlinewidth{1.204500pt}%
\definecolor{currentstroke}{rgb}{0.000000,0.000000,0.000000}%
\pgfsetstrokecolor{currentstroke}%
\pgfsetdash{}{0pt}%
\pgfpathmoveto{\pgfqpoint{2.992075in}{3.710089in}}%
\pgfpathlineto{\pgfqpoint{3.889745in}{3.289387in}}%
\pgfusepath{stroke}%
\end{pgfscope}%
\begin{pgfscope}%
\pgfpathrectangle{\pgfqpoint{0.765000in}{0.660000in}}{\pgfqpoint{4.620000in}{4.620000in}}%
\pgfusepath{clip}%
\pgfsetroundcap%
\pgfsetroundjoin%
\pgfsetlinewidth{1.204500pt}%
\definecolor{currentstroke}{rgb}{0.000000,0.000000,0.000000}%
\pgfsetstrokecolor{currentstroke}%
\pgfsetdash{}{0pt}%
\pgfpathmoveto{\pgfqpoint{2.992075in}{3.710089in}}%
\pgfpathlineto{\pgfqpoint{2.992075in}{5.290000in}}%
\pgfusepath{stroke}%
\end{pgfscope}%
\begin{pgfscope}%
\pgfpathrectangle{\pgfqpoint{0.765000in}{0.660000in}}{\pgfqpoint{4.620000in}{4.620000in}}%
\pgfusepath{clip}%
\pgfsetbuttcap%
\pgfsetroundjoin%
\definecolor{currentfill}{rgb}{0.176471,0.192157,0.258824}%
\pgfsetfillcolor{currentfill}%
\pgfsetfillopacity{0.050000}%
\pgfsetlinewidth{0.000000pt}%
\definecolor{currentstroke}{rgb}{0.176471,0.192157,0.258824}%
\pgfsetstrokecolor{currentstroke}%
\pgfsetdash{}{0pt}%
\pgfpathmoveto{\pgfqpoint{1.576906in}{3.852932in}}%
\pgfpathlineto{\pgfqpoint{1.576906in}{3.958521in}}%
\pgfpathlineto{\pgfqpoint{1.689653in}{3.905589in}}%
\pgfpathlineto{\pgfqpoint{1.689653in}{3.800001in}}%
\pgfpathlineto{\pgfqpoint{1.576906in}{3.852932in}}%
\pgfpathclose%
\pgfusepath{fill}%
\end{pgfscope}%
\begin{pgfscope}%
\pgfpathrectangle{\pgfqpoint{0.765000in}{0.660000in}}{\pgfqpoint{4.620000in}{4.620000in}}%
\pgfusepath{clip}%
\pgfsetbuttcap%
\pgfsetroundjoin%
\definecolor{currentfill}{rgb}{0.176471,0.192157,0.258824}%
\pgfsetfillcolor{currentfill}%
\pgfsetfillopacity{0.050000}%
\pgfsetlinewidth{0.000000pt}%
\definecolor{currentstroke}{rgb}{0.176471,0.192157,0.258824}%
\pgfsetstrokecolor{currentstroke}%
\pgfsetdash{}{0pt}%
\pgfpathmoveto{\pgfqpoint{1.576906in}{3.852932in}}%
\pgfpathlineto{\pgfqpoint{1.576906in}{3.958521in}}%
\pgfpathlineto{\pgfqpoint{1.464158in}{3.905589in}}%
\pgfpathlineto{\pgfqpoint{1.464158in}{3.800001in}}%
\pgfpathlineto{\pgfqpoint{1.576906in}{3.852932in}}%
\pgfpathclose%
\pgfusepath{fill}%
\end{pgfscope}%
\begin{pgfscope}%
\pgfpathrectangle{\pgfqpoint{0.765000in}{0.660000in}}{\pgfqpoint{4.620000in}{4.620000in}}%
\pgfusepath{clip}%
\pgfsetbuttcap%
\pgfsetroundjoin%
\definecolor{currentfill}{rgb}{0.176471,0.192157,0.258824}%
\pgfsetfillcolor{currentfill}%
\pgfsetfillopacity{0.050000}%
\pgfsetlinewidth{0.000000pt}%
\definecolor{currentstroke}{rgb}{0.176471,0.192157,0.258824}%
\pgfsetstrokecolor{currentstroke}%
\pgfsetdash{}{0pt}%
\pgfpathmoveto{\pgfqpoint{1.576906in}{3.852932in}}%
\pgfpathlineto{\pgfqpoint{1.689653in}{3.800001in}}%
\pgfpathlineto{\pgfqpoint{1.576906in}{3.747069in}}%
\pgfpathlineto{\pgfqpoint{1.464158in}{3.800001in}}%
\pgfpathlineto{\pgfqpoint{1.576906in}{3.852932in}}%
\pgfpathclose%
\pgfusepath{fill}%
\end{pgfscope}%
\begin{pgfscope}%
\pgfpathrectangle{\pgfqpoint{0.765000in}{0.660000in}}{\pgfqpoint{4.620000in}{4.620000in}}%
\pgfusepath{clip}%
\pgfsetbuttcap%
\pgfsetroundjoin%
\definecolor{currentfill}{rgb}{0.176471,0.192157,0.258824}%
\pgfsetfillcolor{currentfill}%
\pgfsetfillopacity{0.050000}%
\pgfsetlinewidth{0.000000pt}%
\definecolor{currentstroke}{rgb}{0.176471,0.192157,0.258824}%
\pgfsetstrokecolor{currentstroke}%
\pgfsetdash{}{0pt}%
\pgfpathmoveto{\pgfqpoint{1.576906in}{3.958521in}}%
\pgfpathlineto{\pgfqpoint{1.689653in}{3.905589in}}%
\pgfpathlineto{\pgfqpoint{1.576906in}{3.852658in}}%
\pgfpathlineto{\pgfqpoint{1.464158in}{3.905589in}}%
\pgfpathlineto{\pgfqpoint{1.576906in}{3.958521in}}%
\pgfpathclose%
\pgfusepath{fill}%
\end{pgfscope}%
\begin{pgfscope}%
\pgfpathrectangle{\pgfqpoint{0.765000in}{0.660000in}}{\pgfqpoint{4.620000in}{4.620000in}}%
\pgfusepath{clip}%
\pgfsetbuttcap%
\pgfsetroundjoin%
\definecolor{currentfill}{rgb}{0.176471,0.192157,0.258824}%
\pgfsetfillcolor{currentfill}%
\pgfsetfillopacity{0.050000}%
\pgfsetlinewidth{0.000000pt}%
\definecolor{currentstroke}{rgb}{0.176471,0.192157,0.258824}%
\pgfsetstrokecolor{currentstroke}%
\pgfsetdash{}{0pt}%
\pgfpathmoveto{\pgfqpoint{1.464158in}{3.800001in}}%
\pgfpathlineto{\pgfqpoint{1.464158in}{3.905589in}}%
\pgfpathlineto{\pgfqpoint{1.576906in}{3.852658in}}%
\pgfpathlineto{\pgfqpoint{1.576906in}{3.747069in}}%
\pgfpathlineto{\pgfqpoint{1.464158in}{3.800001in}}%
\pgfpathclose%
\pgfusepath{fill}%
\end{pgfscope}%
\begin{pgfscope}%
\pgfpathrectangle{\pgfqpoint{0.765000in}{0.660000in}}{\pgfqpoint{4.620000in}{4.620000in}}%
\pgfusepath{clip}%
\pgfsetbuttcap%
\pgfsetroundjoin%
\definecolor{currentfill}{rgb}{0.176471,0.192157,0.258824}%
\pgfsetfillcolor{currentfill}%
\pgfsetfillopacity{0.050000}%
\pgfsetlinewidth{0.000000pt}%
\definecolor{currentstroke}{rgb}{0.176471,0.192157,0.258824}%
\pgfsetstrokecolor{currentstroke}%
\pgfsetdash{}{0pt}%
\pgfpathmoveto{\pgfqpoint{1.689653in}{3.800001in}}%
\pgfpathlineto{\pgfqpoint{1.689653in}{3.905589in}}%
\pgfpathlineto{\pgfqpoint{1.576906in}{3.852658in}}%
\pgfpathlineto{\pgfqpoint{1.576906in}{3.747069in}}%
\pgfpathlineto{\pgfqpoint{1.689653in}{3.800001in}}%
\pgfpathclose%
\pgfusepath{fill}%
\end{pgfscope}%
\begin{pgfscope}%
\pgfpathrectangle{\pgfqpoint{0.765000in}{0.660000in}}{\pgfqpoint{4.620000in}{4.620000in}}%
\pgfusepath{clip}%
\pgfsetbuttcap%
\pgfsetroundjoin%
\definecolor{currentfill}{rgb}{0.176471,0.192157,0.258824}%
\pgfsetfillcolor{currentfill}%
\pgfsetfillopacity{0.050000}%
\pgfsetlinewidth{0.000000pt}%
\definecolor{currentstroke}{rgb}{0.176471,0.192157,0.258824}%
\pgfsetstrokecolor{currentstroke}%
\pgfsetdash{}{0pt}%
\pgfpathmoveto{\pgfqpoint{1.889868in}{3.777711in}}%
\pgfpathlineto{\pgfqpoint{1.889868in}{3.883300in}}%
\pgfpathlineto{\pgfqpoint{2.002615in}{3.830368in}}%
\pgfpathlineto{\pgfqpoint{2.002615in}{3.724779in}}%
\pgfpathlineto{\pgfqpoint{1.889868in}{3.777711in}}%
\pgfpathclose%
\pgfusepath{fill}%
\end{pgfscope}%
\begin{pgfscope}%
\pgfpathrectangle{\pgfqpoint{0.765000in}{0.660000in}}{\pgfqpoint{4.620000in}{4.620000in}}%
\pgfusepath{clip}%
\pgfsetbuttcap%
\pgfsetroundjoin%
\definecolor{currentfill}{rgb}{0.176471,0.192157,0.258824}%
\pgfsetfillcolor{currentfill}%
\pgfsetfillopacity{0.050000}%
\pgfsetlinewidth{0.000000pt}%
\definecolor{currentstroke}{rgb}{0.176471,0.192157,0.258824}%
\pgfsetstrokecolor{currentstroke}%
\pgfsetdash{}{0pt}%
\pgfpathmoveto{\pgfqpoint{1.889868in}{3.777711in}}%
\pgfpathlineto{\pgfqpoint{1.889868in}{3.883300in}}%
\pgfpathlineto{\pgfqpoint{1.777121in}{3.830368in}}%
\pgfpathlineto{\pgfqpoint{1.777121in}{3.724779in}}%
\pgfpathlineto{\pgfqpoint{1.889868in}{3.777711in}}%
\pgfpathclose%
\pgfusepath{fill}%
\end{pgfscope}%
\begin{pgfscope}%
\pgfpathrectangle{\pgfqpoint{0.765000in}{0.660000in}}{\pgfqpoint{4.620000in}{4.620000in}}%
\pgfusepath{clip}%
\pgfsetbuttcap%
\pgfsetroundjoin%
\definecolor{currentfill}{rgb}{0.176471,0.192157,0.258824}%
\pgfsetfillcolor{currentfill}%
\pgfsetfillopacity{0.050000}%
\pgfsetlinewidth{0.000000pt}%
\definecolor{currentstroke}{rgb}{0.176471,0.192157,0.258824}%
\pgfsetstrokecolor{currentstroke}%
\pgfsetdash{}{0pt}%
\pgfpathmoveto{\pgfqpoint{1.889868in}{3.777711in}}%
\pgfpathlineto{\pgfqpoint{2.002615in}{3.724779in}}%
\pgfpathlineto{\pgfqpoint{1.889868in}{3.671848in}}%
\pgfpathlineto{\pgfqpoint{1.777121in}{3.724779in}}%
\pgfpathlineto{\pgfqpoint{1.889868in}{3.777711in}}%
\pgfpathclose%
\pgfusepath{fill}%
\end{pgfscope}%
\begin{pgfscope}%
\pgfpathrectangle{\pgfqpoint{0.765000in}{0.660000in}}{\pgfqpoint{4.620000in}{4.620000in}}%
\pgfusepath{clip}%
\pgfsetbuttcap%
\pgfsetroundjoin%
\definecolor{currentfill}{rgb}{0.176471,0.192157,0.258824}%
\pgfsetfillcolor{currentfill}%
\pgfsetfillopacity{0.050000}%
\pgfsetlinewidth{0.000000pt}%
\definecolor{currentstroke}{rgb}{0.176471,0.192157,0.258824}%
\pgfsetstrokecolor{currentstroke}%
\pgfsetdash{}{0pt}%
\pgfpathmoveto{\pgfqpoint{1.889868in}{3.883300in}}%
\pgfpathlineto{\pgfqpoint{2.002615in}{3.830368in}}%
\pgfpathlineto{\pgfqpoint{1.889868in}{3.777436in}}%
\pgfpathlineto{\pgfqpoint{1.777121in}{3.830368in}}%
\pgfpathlineto{\pgfqpoint{1.889868in}{3.883300in}}%
\pgfpathclose%
\pgfusepath{fill}%
\end{pgfscope}%
\begin{pgfscope}%
\pgfpathrectangle{\pgfqpoint{0.765000in}{0.660000in}}{\pgfqpoint{4.620000in}{4.620000in}}%
\pgfusepath{clip}%
\pgfsetbuttcap%
\pgfsetroundjoin%
\definecolor{currentfill}{rgb}{0.176471,0.192157,0.258824}%
\pgfsetfillcolor{currentfill}%
\pgfsetfillopacity{0.050000}%
\pgfsetlinewidth{0.000000pt}%
\definecolor{currentstroke}{rgb}{0.176471,0.192157,0.258824}%
\pgfsetstrokecolor{currentstroke}%
\pgfsetdash{}{0pt}%
\pgfpathmoveto{\pgfqpoint{1.777121in}{3.724779in}}%
\pgfpathlineto{\pgfqpoint{1.777121in}{3.830368in}}%
\pgfpathlineto{\pgfqpoint{1.889868in}{3.777436in}}%
\pgfpathlineto{\pgfqpoint{1.889868in}{3.671848in}}%
\pgfpathlineto{\pgfqpoint{1.777121in}{3.724779in}}%
\pgfpathclose%
\pgfusepath{fill}%
\end{pgfscope}%
\begin{pgfscope}%
\pgfpathrectangle{\pgfqpoint{0.765000in}{0.660000in}}{\pgfqpoint{4.620000in}{4.620000in}}%
\pgfusepath{clip}%
\pgfsetbuttcap%
\pgfsetroundjoin%
\definecolor{currentfill}{rgb}{0.176471,0.192157,0.258824}%
\pgfsetfillcolor{currentfill}%
\pgfsetfillopacity{0.050000}%
\pgfsetlinewidth{0.000000pt}%
\definecolor{currentstroke}{rgb}{0.176471,0.192157,0.258824}%
\pgfsetstrokecolor{currentstroke}%
\pgfsetdash{}{0pt}%
\pgfpathmoveto{\pgfqpoint{2.002615in}{3.724779in}}%
\pgfpathlineto{\pgfqpoint{2.002615in}{3.830368in}}%
\pgfpathlineto{\pgfqpoint{1.889868in}{3.777436in}}%
\pgfpathlineto{\pgfqpoint{1.889868in}{3.671848in}}%
\pgfpathlineto{\pgfqpoint{2.002615in}{3.724779in}}%
\pgfpathclose%
\pgfusepath{fill}%
\end{pgfscope}%
\begin{pgfscope}%
\pgfpathrectangle{\pgfqpoint{0.765000in}{0.660000in}}{\pgfqpoint{4.620000in}{4.620000in}}%
\pgfusepath{clip}%
\pgfsetbuttcap%
\pgfsetroundjoin%
\definecolor{currentfill}{rgb}{1.000000,0.576471,0.309804}%
\pgfsetfillcolor{currentfill}%
\pgfsetfillopacity{0.050000}%
\pgfsetlinewidth{0.000000pt}%
\definecolor{currentstroke}{rgb}{1.000000,0.576471,0.309804}%
\pgfsetstrokecolor{currentstroke}%
\pgfsetdash{}{0pt}%
\pgfpathmoveto{\pgfqpoint{4.711174in}{3.563020in}}%
\pgfpathlineto{\pgfqpoint{4.711174in}{3.668609in}}%
\pgfpathlineto{\pgfqpoint{4.823922in}{3.615677in}}%
\pgfpathlineto{\pgfqpoint{4.823922in}{3.510088in}}%
\pgfpathlineto{\pgfqpoint{4.711174in}{3.563020in}}%
\pgfpathclose%
\pgfusepath{fill}%
\end{pgfscope}%
\begin{pgfscope}%
\pgfpathrectangle{\pgfqpoint{0.765000in}{0.660000in}}{\pgfqpoint{4.620000in}{4.620000in}}%
\pgfusepath{clip}%
\pgfsetbuttcap%
\pgfsetroundjoin%
\definecolor{currentfill}{rgb}{1.000000,0.576471,0.309804}%
\pgfsetfillcolor{currentfill}%
\pgfsetfillopacity{0.050000}%
\pgfsetlinewidth{0.000000pt}%
\definecolor{currentstroke}{rgb}{1.000000,0.576471,0.309804}%
\pgfsetstrokecolor{currentstroke}%
\pgfsetdash{}{0pt}%
\pgfpathmoveto{\pgfqpoint{4.711174in}{3.563020in}}%
\pgfpathlineto{\pgfqpoint{4.711174in}{3.668609in}}%
\pgfpathlineto{\pgfqpoint{4.598427in}{3.615677in}}%
\pgfpathlineto{\pgfqpoint{4.598427in}{3.510088in}}%
\pgfpathlineto{\pgfqpoint{4.711174in}{3.563020in}}%
\pgfpathclose%
\pgfusepath{fill}%
\end{pgfscope}%
\begin{pgfscope}%
\pgfpathrectangle{\pgfqpoint{0.765000in}{0.660000in}}{\pgfqpoint{4.620000in}{4.620000in}}%
\pgfusepath{clip}%
\pgfsetbuttcap%
\pgfsetroundjoin%
\definecolor{currentfill}{rgb}{1.000000,0.576471,0.309804}%
\pgfsetfillcolor{currentfill}%
\pgfsetfillopacity{0.050000}%
\pgfsetlinewidth{0.000000pt}%
\definecolor{currentstroke}{rgb}{1.000000,0.576471,0.309804}%
\pgfsetstrokecolor{currentstroke}%
\pgfsetdash{}{0pt}%
\pgfpathmoveto{\pgfqpoint{4.711174in}{3.563020in}}%
\pgfpathlineto{\pgfqpoint{4.823922in}{3.510088in}}%
\pgfpathlineto{\pgfqpoint{4.711174in}{3.457157in}}%
\pgfpathlineto{\pgfqpoint{4.598427in}{3.510088in}}%
\pgfpathlineto{\pgfqpoint{4.711174in}{3.563020in}}%
\pgfpathclose%
\pgfusepath{fill}%
\end{pgfscope}%
\begin{pgfscope}%
\pgfpathrectangle{\pgfqpoint{0.765000in}{0.660000in}}{\pgfqpoint{4.620000in}{4.620000in}}%
\pgfusepath{clip}%
\pgfsetbuttcap%
\pgfsetroundjoin%
\definecolor{currentfill}{rgb}{1.000000,0.576471,0.309804}%
\pgfsetfillcolor{currentfill}%
\pgfsetfillopacity{0.050000}%
\pgfsetlinewidth{0.000000pt}%
\definecolor{currentstroke}{rgb}{1.000000,0.576471,0.309804}%
\pgfsetstrokecolor{currentstroke}%
\pgfsetdash{}{0pt}%
\pgfpathmoveto{\pgfqpoint{4.711174in}{3.668609in}}%
\pgfpathlineto{\pgfqpoint{4.823922in}{3.615677in}}%
\pgfpathlineto{\pgfqpoint{4.711174in}{3.562746in}}%
\pgfpathlineto{\pgfqpoint{4.598427in}{3.615677in}}%
\pgfpathlineto{\pgfqpoint{4.711174in}{3.668609in}}%
\pgfpathclose%
\pgfusepath{fill}%
\end{pgfscope}%
\begin{pgfscope}%
\pgfpathrectangle{\pgfqpoint{0.765000in}{0.660000in}}{\pgfqpoint{4.620000in}{4.620000in}}%
\pgfusepath{clip}%
\pgfsetbuttcap%
\pgfsetroundjoin%
\definecolor{currentfill}{rgb}{1.000000,0.576471,0.309804}%
\pgfsetfillcolor{currentfill}%
\pgfsetfillopacity{0.050000}%
\pgfsetlinewidth{0.000000pt}%
\definecolor{currentstroke}{rgb}{1.000000,0.576471,0.309804}%
\pgfsetstrokecolor{currentstroke}%
\pgfsetdash{}{0pt}%
\pgfpathmoveto{\pgfqpoint{4.598427in}{3.510088in}}%
\pgfpathlineto{\pgfqpoint{4.598427in}{3.615677in}}%
\pgfpathlineto{\pgfqpoint{4.711174in}{3.562746in}}%
\pgfpathlineto{\pgfqpoint{4.711174in}{3.457157in}}%
\pgfpathlineto{\pgfqpoint{4.598427in}{3.510088in}}%
\pgfpathclose%
\pgfusepath{fill}%
\end{pgfscope}%
\begin{pgfscope}%
\pgfpathrectangle{\pgfqpoint{0.765000in}{0.660000in}}{\pgfqpoint{4.620000in}{4.620000in}}%
\pgfusepath{clip}%
\pgfsetbuttcap%
\pgfsetroundjoin%
\definecolor{currentfill}{rgb}{1.000000,0.576471,0.309804}%
\pgfsetfillcolor{currentfill}%
\pgfsetfillopacity{0.050000}%
\pgfsetlinewidth{0.000000pt}%
\definecolor{currentstroke}{rgb}{1.000000,0.576471,0.309804}%
\pgfsetstrokecolor{currentstroke}%
\pgfsetdash{}{0pt}%
\pgfpathmoveto{\pgfqpoint{4.823922in}{3.510088in}}%
\pgfpathlineto{\pgfqpoint{4.823922in}{3.615677in}}%
\pgfpathlineto{\pgfqpoint{4.711174in}{3.562746in}}%
\pgfpathlineto{\pgfqpoint{4.711174in}{3.457157in}}%
\pgfpathlineto{\pgfqpoint{4.823922in}{3.510088in}}%
\pgfpathclose%
\pgfusepath{fill}%
\end{pgfscope}%
\begin{pgfscope}%
\pgfpathrectangle{\pgfqpoint{0.765000in}{0.660000in}}{\pgfqpoint{4.620000in}{4.620000in}}%
\pgfusepath{clip}%
\pgfsetbuttcap%
\pgfsetroundjoin%
\definecolor{currentfill}{rgb}{1.000000,0.576471,0.309804}%
\pgfsetfillcolor{currentfill}%
\pgfsetfillopacity{0.050000}%
\pgfsetlinewidth{0.000000pt}%
\definecolor{currentstroke}{rgb}{1.000000,0.576471,0.309804}%
\pgfsetstrokecolor{currentstroke}%
\pgfsetdash{}{0pt}%
\pgfpathmoveto{\pgfqpoint{4.633827in}{3.677925in}}%
\pgfpathlineto{\pgfqpoint{4.633827in}{3.783514in}}%
\pgfpathlineto{\pgfqpoint{4.746574in}{3.730582in}}%
\pgfpathlineto{\pgfqpoint{4.746574in}{3.624993in}}%
\pgfpathlineto{\pgfqpoint{4.633827in}{3.677925in}}%
\pgfpathclose%
\pgfusepath{fill}%
\end{pgfscope}%
\begin{pgfscope}%
\pgfpathrectangle{\pgfqpoint{0.765000in}{0.660000in}}{\pgfqpoint{4.620000in}{4.620000in}}%
\pgfusepath{clip}%
\pgfsetbuttcap%
\pgfsetroundjoin%
\definecolor{currentfill}{rgb}{1.000000,0.576471,0.309804}%
\pgfsetfillcolor{currentfill}%
\pgfsetfillopacity{0.050000}%
\pgfsetlinewidth{0.000000pt}%
\definecolor{currentstroke}{rgb}{1.000000,0.576471,0.309804}%
\pgfsetstrokecolor{currentstroke}%
\pgfsetdash{}{0pt}%
\pgfpathmoveto{\pgfqpoint{4.633827in}{3.677925in}}%
\pgfpathlineto{\pgfqpoint{4.633827in}{3.783514in}}%
\pgfpathlineto{\pgfqpoint{4.521079in}{3.730582in}}%
\pgfpathlineto{\pgfqpoint{4.521079in}{3.624993in}}%
\pgfpathlineto{\pgfqpoint{4.633827in}{3.677925in}}%
\pgfpathclose%
\pgfusepath{fill}%
\end{pgfscope}%
\begin{pgfscope}%
\pgfpathrectangle{\pgfqpoint{0.765000in}{0.660000in}}{\pgfqpoint{4.620000in}{4.620000in}}%
\pgfusepath{clip}%
\pgfsetbuttcap%
\pgfsetroundjoin%
\definecolor{currentfill}{rgb}{1.000000,0.576471,0.309804}%
\pgfsetfillcolor{currentfill}%
\pgfsetfillopacity{0.050000}%
\pgfsetlinewidth{0.000000pt}%
\definecolor{currentstroke}{rgb}{1.000000,0.576471,0.309804}%
\pgfsetstrokecolor{currentstroke}%
\pgfsetdash{}{0pt}%
\pgfpathmoveto{\pgfqpoint{4.633827in}{3.677925in}}%
\pgfpathlineto{\pgfqpoint{4.746574in}{3.624993in}}%
\pgfpathlineto{\pgfqpoint{4.633827in}{3.572062in}}%
\pgfpathlineto{\pgfqpoint{4.521079in}{3.624993in}}%
\pgfpathlineto{\pgfqpoint{4.633827in}{3.677925in}}%
\pgfpathclose%
\pgfusepath{fill}%
\end{pgfscope}%
\begin{pgfscope}%
\pgfpathrectangle{\pgfqpoint{0.765000in}{0.660000in}}{\pgfqpoint{4.620000in}{4.620000in}}%
\pgfusepath{clip}%
\pgfsetbuttcap%
\pgfsetroundjoin%
\definecolor{currentfill}{rgb}{1.000000,0.576471,0.309804}%
\pgfsetfillcolor{currentfill}%
\pgfsetfillopacity{0.050000}%
\pgfsetlinewidth{0.000000pt}%
\definecolor{currentstroke}{rgb}{1.000000,0.576471,0.309804}%
\pgfsetstrokecolor{currentstroke}%
\pgfsetdash{}{0pt}%
\pgfpathmoveto{\pgfqpoint{4.633827in}{3.783514in}}%
\pgfpathlineto{\pgfqpoint{4.746574in}{3.730582in}}%
\pgfpathlineto{\pgfqpoint{4.633827in}{3.677650in}}%
\pgfpathlineto{\pgfqpoint{4.521079in}{3.730582in}}%
\pgfpathlineto{\pgfqpoint{4.633827in}{3.783514in}}%
\pgfpathclose%
\pgfusepath{fill}%
\end{pgfscope}%
\begin{pgfscope}%
\pgfpathrectangle{\pgfqpoint{0.765000in}{0.660000in}}{\pgfqpoint{4.620000in}{4.620000in}}%
\pgfusepath{clip}%
\pgfsetbuttcap%
\pgfsetroundjoin%
\definecolor{currentfill}{rgb}{1.000000,0.576471,0.309804}%
\pgfsetfillcolor{currentfill}%
\pgfsetfillopacity{0.050000}%
\pgfsetlinewidth{0.000000pt}%
\definecolor{currentstroke}{rgb}{1.000000,0.576471,0.309804}%
\pgfsetstrokecolor{currentstroke}%
\pgfsetdash{}{0pt}%
\pgfpathmoveto{\pgfqpoint{4.521079in}{3.624993in}}%
\pgfpathlineto{\pgfqpoint{4.521079in}{3.730582in}}%
\pgfpathlineto{\pgfqpoint{4.633827in}{3.677650in}}%
\pgfpathlineto{\pgfqpoint{4.633827in}{3.572062in}}%
\pgfpathlineto{\pgfqpoint{4.521079in}{3.624993in}}%
\pgfpathclose%
\pgfusepath{fill}%
\end{pgfscope}%
\begin{pgfscope}%
\pgfpathrectangle{\pgfqpoint{0.765000in}{0.660000in}}{\pgfqpoint{4.620000in}{4.620000in}}%
\pgfusepath{clip}%
\pgfsetbuttcap%
\pgfsetroundjoin%
\definecolor{currentfill}{rgb}{1.000000,0.576471,0.309804}%
\pgfsetfillcolor{currentfill}%
\pgfsetfillopacity{0.050000}%
\pgfsetlinewidth{0.000000pt}%
\definecolor{currentstroke}{rgb}{1.000000,0.576471,0.309804}%
\pgfsetstrokecolor{currentstroke}%
\pgfsetdash{}{0pt}%
\pgfpathmoveto{\pgfqpoint{4.746574in}{3.624993in}}%
\pgfpathlineto{\pgfqpoint{4.746574in}{3.730582in}}%
\pgfpathlineto{\pgfqpoint{4.633827in}{3.677650in}}%
\pgfpathlineto{\pgfqpoint{4.633827in}{3.572062in}}%
\pgfpathlineto{\pgfqpoint{4.746574in}{3.624993in}}%
\pgfpathclose%
\pgfusepath{fill}%
\end{pgfscope}%
\begin{pgfscope}%
\pgfpathrectangle{\pgfqpoint{0.765000in}{0.660000in}}{\pgfqpoint{4.620000in}{4.620000in}}%
\pgfusepath{clip}%
\pgfsetbuttcap%
\pgfsetroundjoin%
\definecolor{currentfill}{rgb}{0.176471,0.192157,0.258824}%
\pgfsetfillcolor{currentfill}%
\pgfsetfillopacity{0.050000}%
\pgfsetlinewidth{0.000000pt}%
\definecolor{currentstroke}{rgb}{0.176471,0.192157,0.258824}%
\pgfsetstrokecolor{currentstroke}%
\pgfsetdash{}{0pt}%
\pgfpathmoveto{\pgfqpoint{2.118666in}{3.733026in}}%
\pgfpathlineto{\pgfqpoint{2.118666in}{3.838615in}}%
\pgfpathlineto{\pgfqpoint{2.231414in}{3.785683in}}%
\pgfpathlineto{\pgfqpoint{2.231414in}{3.680094in}}%
\pgfpathlineto{\pgfqpoint{2.118666in}{3.733026in}}%
\pgfpathclose%
\pgfusepath{fill}%
\end{pgfscope}%
\begin{pgfscope}%
\pgfpathrectangle{\pgfqpoint{0.765000in}{0.660000in}}{\pgfqpoint{4.620000in}{4.620000in}}%
\pgfusepath{clip}%
\pgfsetbuttcap%
\pgfsetroundjoin%
\definecolor{currentfill}{rgb}{0.176471,0.192157,0.258824}%
\pgfsetfillcolor{currentfill}%
\pgfsetfillopacity{0.050000}%
\pgfsetlinewidth{0.000000pt}%
\definecolor{currentstroke}{rgb}{0.176471,0.192157,0.258824}%
\pgfsetstrokecolor{currentstroke}%
\pgfsetdash{}{0pt}%
\pgfpathmoveto{\pgfqpoint{2.118666in}{3.733026in}}%
\pgfpathlineto{\pgfqpoint{2.118666in}{3.838615in}}%
\pgfpathlineto{\pgfqpoint{2.005919in}{3.785683in}}%
\pgfpathlineto{\pgfqpoint{2.005919in}{3.680094in}}%
\pgfpathlineto{\pgfqpoint{2.118666in}{3.733026in}}%
\pgfpathclose%
\pgfusepath{fill}%
\end{pgfscope}%
\begin{pgfscope}%
\pgfpathrectangle{\pgfqpoint{0.765000in}{0.660000in}}{\pgfqpoint{4.620000in}{4.620000in}}%
\pgfusepath{clip}%
\pgfsetbuttcap%
\pgfsetroundjoin%
\definecolor{currentfill}{rgb}{0.176471,0.192157,0.258824}%
\pgfsetfillcolor{currentfill}%
\pgfsetfillopacity{0.050000}%
\pgfsetlinewidth{0.000000pt}%
\definecolor{currentstroke}{rgb}{0.176471,0.192157,0.258824}%
\pgfsetstrokecolor{currentstroke}%
\pgfsetdash{}{0pt}%
\pgfpathmoveto{\pgfqpoint{2.118666in}{3.733026in}}%
\pgfpathlineto{\pgfqpoint{2.231414in}{3.680094in}}%
\pgfpathlineto{\pgfqpoint{2.118666in}{3.627162in}}%
\pgfpathlineto{\pgfqpoint{2.005919in}{3.680094in}}%
\pgfpathlineto{\pgfqpoint{2.118666in}{3.733026in}}%
\pgfpathclose%
\pgfusepath{fill}%
\end{pgfscope}%
\begin{pgfscope}%
\pgfpathrectangle{\pgfqpoint{0.765000in}{0.660000in}}{\pgfqpoint{4.620000in}{4.620000in}}%
\pgfusepath{clip}%
\pgfsetbuttcap%
\pgfsetroundjoin%
\definecolor{currentfill}{rgb}{0.176471,0.192157,0.258824}%
\pgfsetfillcolor{currentfill}%
\pgfsetfillopacity{0.050000}%
\pgfsetlinewidth{0.000000pt}%
\definecolor{currentstroke}{rgb}{0.176471,0.192157,0.258824}%
\pgfsetstrokecolor{currentstroke}%
\pgfsetdash{}{0pt}%
\pgfpathmoveto{\pgfqpoint{2.118666in}{3.838615in}}%
\pgfpathlineto{\pgfqpoint{2.231414in}{3.785683in}}%
\pgfpathlineto{\pgfqpoint{2.118666in}{3.732751in}}%
\pgfpathlineto{\pgfqpoint{2.005919in}{3.785683in}}%
\pgfpathlineto{\pgfqpoint{2.118666in}{3.838615in}}%
\pgfpathclose%
\pgfusepath{fill}%
\end{pgfscope}%
\begin{pgfscope}%
\pgfpathrectangle{\pgfqpoint{0.765000in}{0.660000in}}{\pgfqpoint{4.620000in}{4.620000in}}%
\pgfusepath{clip}%
\pgfsetbuttcap%
\pgfsetroundjoin%
\definecolor{currentfill}{rgb}{0.176471,0.192157,0.258824}%
\pgfsetfillcolor{currentfill}%
\pgfsetfillopacity{0.050000}%
\pgfsetlinewidth{0.000000pt}%
\definecolor{currentstroke}{rgb}{0.176471,0.192157,0.258824}%
\pgfsetstrokecolor{currentstroke}%
\pgfsetdash{}{0pt}%
\pgfpathmoveto{\pgfqpoint{2.005919in}{3.680094in}}%
\pgfpathlineto{\pgfqpoint{2.005919in}{3.785683in}}%
\pgfpathlineto{\pgfqpoint{2.118666in}{3.732751in}}%
\pgfpathlineto{\pgfqpoint{2.118666in}{3.627162in}}%
\pgfpathlineto{\pgfqpoint{2.005919in}{3.680094in}}%
\pgfpathclose%
\pgfusepath{fill}%
\end{pgfscope}%
\begin{pgfscope}%
\pgfpathrectangle{\pgfqpoint{0.765000in}{0.660000in}}{\pgfqpoint{4.620000in}{4.620000in}}%
\pgfusepath{clip}%
\pgfsetbuttcap%
\pgfsetroundjoin%
\definecolor{currentfill}{rgb}{0.176471,0.192157,0.258824}%
\pgfsetfillcolor{currentfill}%
\pgfsetfillopacity{0.050000}%
\pgfsetlinewidth{0.000000pt}%
\definecolor{currentstroke}{rgb}{0.176471,0.192157,0.258824}%
\pgfsetstrokecolor{currentstroke}%
\pgfsetdash{}{0pt}%
\pgfpathmoveto{\pgfqpoint{2.231414in}{3.680094in}}%
\pgfpathlineto{\pgfqpoint{2.231414in}{3.785683in}}%
\pgfpathlineto{\pgfqpoint{2.118666in}{3.732751in}}%
\pgfpathlineto{\pgfqpoint{2.118666in}{3.627162in}}%
\pgfpathlineto{\pgfqpoint{2.231414in}{3.680094in}}%
\pgfpathclose%
\pgfusepath{fill}%
\end{pgfscope}%
\begin{pgfscope}%
\pgfpathrectangle{\pgfqpoint{0.765000in}{0.660000in}}{\pgfqpoint{4.620000in}{4.620000in}}%
\pgfusepath{clip}%
\pgfsetbuttcap%
\pgfsetroundjoin%
\definecolor{currentfill}{rgb}{1.000000,0.576471,0.309804}%
\pgfsetfillcolor{currentfill}%
\pgfsetfillopacity{0.050000}%
\pgfsetlinewidth{0.000000pt}%
\definecolor{currentstroke}{rgb}{1.000000,0.576471,0.309804}%
\pgfsetstrokecolor{currentstroke}%
\pgfsetdash{}{0pt}%
\pgfpathmoveto{\pgfqpoint{4.684886in}{3.947620in}}%
\pgfpathlineto{\pgfqpoint{4.684886in}{4.053208in}}%
\pgfpathlineto{\pgfqpoint{4.797634in}{4.000277in}}%
\pgfpathlineto{\pgfqpoint{4.797634in}{3.894688in}}%
\pgfpathlineto{\pgfqpoint{4.684886in}{3.947620in}}%
\pgfpathclose%
\pgfusepath{fill}%
\end{pgfscope}%
\begin{pgfscope}%
\pgfpathrectangle{\pgfqpoint{0.765000in}{0.660000in}}{\pgfqpoint{4.620000in}{4.620000in}}%
\pgfusepath{clip}%
\pgfsetbuttcap%
\pgfsetroundjoin%
\definecolor{currentfill}{rgb}{1.000000,0.576471,0.309804}%
\pgfsetfillcolor{currentfill}%
\pgfsetfillopacity{0.050000}%
\pgfsetlinewidth{0.000000pt}%
\definecolor{currentstroke}{rgb}{1.000000,0.576471,0.309804}%
\pgfsetstrokecolor{currentstroke}%
\pgfsetdash{}{0pt}%
\pgfpathmoveto{\pgfqpoint{4.684886in}{3.947620in}}%
\pgfpathlineto{\pgfqpoint{4.684886in}{4.053208in}}%
\pgfpathlineto{\pgfqpoint{4.572139in}{4.000277in}}%
\pgfpathlineto{\pgfqpoint{4.572139in}{3.894688in}}%
\pgfpathlineto{\pgfqpoint{4.684886in}{3.947620in}}%
\pgfpathclose%
\pgfusepath{fill}%
\end{pgfscope}%
\begin{pgfscope}%
\pgfpathrectangle{\pgfqpoint{0.765000in}{0.660000in}}{\pgfqpoint{4.620000in}{4.620000in}}%
\pgfusepath{clip}%
\pgfsetbuttcap%
\pgfsetroundjoin%
\definecolor{currentfill}{rgb}{1.000000,0.576471,0.309804}%
\pgfsetfillcolor{currentfill}%
\pgfsetfillopacity{0.050000}%
\pgfsetlinewidth{0.000000pt}%
\definecolor{currentstroke}{rgb}{1.000000,0.576471,0.309804}%
\pgfsetstrokecolor{currentstroke}%
\pgfsetdash{}{0pt}%
\pgfpathmoveto{\pgfqpoint{4.684886in}{3.947620in}}%
\pgfpathlineto{\pgfqpoint{4.797634in}{3.894688in}}%
\pgfpathlineto{\pgfqpoint{4.684886in}{3.841756in}}%
\pgfpathlineto{\pgfqpoint{4.572139in}{3.894688in}}%
\pgfpathlineto{\pgfqpoint{4.684886in}{3.947620in}}%
\pgfpathclose%
\pgfusepath{fill}%
\end{pgfscope}%
\begin{pgfscope}%
\pgfpathrectangle{\pgfqpoint{0.765000in}{0.660000in}}{\pgfqpoint{4.620000in}{4.620000in}}%
\pgfusepath{clip}%
\pgfsetbuttcap%
\pgfsetroundjoin%
\definecolor{currentfill}{rgb}{1.000000,0.576471,0.309804}%
\pgfsetfillcolor{currentfill}%
\pgfsetfillopacity{0.050000}%
\pgfsetlinewidth{0.000000pt}%
\definecolor{currentstroke}{rgb}{1.000000,0.576471,0.309804}%
\pgfsetstrokecolor{currentstroke}%
\pgfsetdash{}{0pt}%
\pgfpathmoveto{\pgfqpoint{4.684886in}{4.053208in}}%
\pgfpathlineto{\pgfqpoint{4.797634in}{4.000277in}}%
\pgfpathlineto{\pgfqpoint{4.684886in}{3.947345in}}%
\pgfpathlineto{\pgfqpoint{4.572139in}{4.000277in}}%
\pgfpathlineto{\pgfqpoint{4.684886in}{4.053208in}}%
\pgfpathclose%
\pgfusepath{fill}%
\end{pgfscope}%
\begin{pgfscope}%
\pgfpathrectangle{\pgfqpoint{0.765000in}{0.660000in}}{\pgfqpoint{4.620000in}{4.620000in}}%
\pgfusepath{clip}%
\pgfsetbuttcap%
\pgfsetroundjoin%
\definecolor{currentfill}{rgb}{1.000000,0.576471,0.309804}%
\pgfsetfillcolor{currentfill}%
\pgfsetfillopacity{0.050000}%
\pgfsetlinewidth{0.000000pt}%
\definecolor{currentstroke}{rgb}{1.000000,0.576471,0.309804}%
\pgfsetstrokecolor{currentstroke}%
\pgfsetdash{}{0pt}%
\pgfpathmoveto{\pgfqpoint{4.572139in}{3.894688in}}%
\pgfpathlineto{\pgfqpoint{4.572139in}{4.000277in}}%
\pgfpathlineto{\pgfqpoint{4.684886in}{3.947345in}}%
\pgfpathlineto{\pgfqpoint{4.684886in}{3.841756in}}%
\pgfpathlineto{\pgfqpoint{4.572139in}{3.894688in}}%
\pgfpathclose%
\pgfusepath{fill}%
\end{pgfscope}%
\begin{pgfscope}%
\pgfpathrectangle{\pgfqpoint{0.765000in}{0.660000in}}{\pgfqpoint{4.620000in}{4.620000in}}%
\pgfusepath{clip}%
\pgfsetbuttcap%
\pgfsetroundjoin%
\definecolor{currentfill}{rgb}{1.000000,0.576471,0.309804}%
\pgfsetfillcolor{currentfill}%
\pgfsetfillopacity{0.050000}%
\pgfsetlinewidth{0.000000pt}%
\definecolor{currentstroke}{rgb}{1.000000,0.576471,0.309804}%
\pgfsetstrokecolor{currentstroke}%
\pgfsetdash{}{0pt}%
\pgfpathmoveto{\pgfqpoint{4.797634in}{3.894688in}}%
\pgfpathlineto{\pgfqpoint{4.797634in}{4.000277in}}%
\pgfpathlineto{\pgfqpoint{4.684886in}{3.947345in}}%
\pgfpathlineto{\pgfqpoint{4.684886in}{3.841756in}}%
\pgfpathlineto{\pgfqpoint{4.797634in}{3.894688in}}%
\pgfpathclose%
\pgfusepath{fill}%
\end{pgfscope}%
\begin{pgfscope}%
\pgfpathrectangle{\pgfqpoint{0.765000in}{0.660000in}}{\pgfqpoint{4.620000in}{4.620000in}}%
\pgfusepath{clip}%
\pgfsetbuttcap%
\pgfsetroundjoin%
\definecolor{currentfill}{rgb}{0.176471,0.192157,0.258824}%
\pgfsetfillcolor{currentfill}%
\pgfsetfillopacity{0.050000}%
\pgfsetlinewidth{0.000000pt}%
\definecolor{currentstroke}{rgb}{0.176471,0.192157,0.258824}%
\pgfsetstrokecolor{currentstroke}%
\pgfsetdash{}{0pt}%
\pgfpathmoveto{\pgfqpoint{2.124708in}{3.303760in}}%
\pgfpathlineto{\pgfqpoint{2.124708in}{3.409349in}}%
\pgfpathlineto{\pgfqpoint{2.237456in}{3.356417in}}%
\pgfpathlineto{\pgfqpoint{2.237456in}{3.250829in}}%
\pgfpathlineto{\pgfqpoint{2.124708in}{3.303760in}}%
\pgfpathclose%
\pgfusepath{fill}%
\end{pgfscope}%
\begin{pgfscope}%
\pgfpathrectangle{\pgfqpoint{0.765000in}{0.660000in}}{\pgfqpoint{4.620000in}{4.620000in}}%
\pgfusepath{clip}%
\pgfsetbuttcap%
\pgfsetroundjoin%
\definecolor{currentfill}{rgb}{0.176471,0.192157,0.258824}%
\pgfsetfillcolor{currentfill}%
\pgfsetfillopacity{0.050000}%
\pgfsetlinewidth{0.000000pt}%
\definecolor{currentstroke}{rgb}{0.176471,0.192157,0.258824}%
\pgfsetstrokecolor{currentstroke}%
\pgfsetdash{}{0pt}%
\pgfpathmoveto{\pgfqpoint{2.124708in}{3.303760in}}%
\pgfpathlineto{\pgfqpoint{2.124708in}{3.409349in}}%
\pgfpathlineto{\pgfqpoint{2.011961in}{3.356417in}}%
\pgfpathlineto{\pgfqpoint{2.011961in}{3.250829in}}%
\pgfpathlineto{\pgfqpoint{2.124708in}{3.303760in}}%
\pgfpathclose%
\pgfusepath{fill}%
\end{pgfscope}%
\begin{pgfscope}%
\pgfpathrectangle{\pgfqpoint{0.765000in}{0.660000in}}{\pgfqpoint{4.620000in}{4.620000in}}%
\pgfusepath{clip}%
\pgfsetbuttcap%
\pgfsetroundjoin%
\definecolor{currentfill}{rgb}{0.176471,0.192157,0.258824}%
\pgfsetfillcolor{currentfill}%
\pgfsetfillopacity{0.050000}%
\pgfsetlinewidth{0.000000pt}%
\definecolor{currentstroke}{rgb}{0.176471,0.192157,0.258824}%
\pgfsetstrokecolor{currentstroke}%
\pgfsetdash{}{0pt}%
\pgfpathmoveto{\pgfqpoint{2.124708in}{3.303760in}}%
\pgfpathlineto{\pgfqpoint{2.237456in}{3.250829in}}%
\pgfpathlineto{\pgfqpoint{2.124708in}{3.197897in}}%
\pgfpathlineto{\pgfqpoint{2.011961in}{3.250829in}}%
\pgfpathlineto{\pgfqpoint{2.124708in}{3.303760in}}%
\pgfpathclose%
\pgfusepath{fill}%
\end{pgfscope}%
\begin{pgfscope}%
\pgfpathrectangle{\pgfqpoint{0.765000in}{0.660000in}}{\pgfqpoint{4.620000in}{4.620000in}}%
\pgfusepath{clip}%
\pgfsetbuttcap%
\pgfsetroundjoin%
\definecolor{currentfill}{rgb}{0.176471,0.192157,0.258824}%
\pgfsetfillcolor{currentfill}%
\pgfsetfillopacity{0.050000}%
\pgfsetlinewidth{0.000000pt}%
\definecolor{currentstroke}{rgb}{0.176471,0.192157,0.258824}%
\pgfsetstrokecolor{currentstroke}%
\pgfsetdash{}{0pt}%
\pgfpathmoveto{\pgfqpoint{2.124708in}{3.409349in}}%
\pgfpathlineto{\pgfqpoint{2.237456in}{3.356417in}}%
\pgfpathlineto{\pgfqpoint{2.124708in}{3.303486in}}%
\pgfpathlineto{\pgfqpoint{2.011961in}{3.356417in}}%
\pgfpathlineto{\pgfqpoint{2.124708in}{3.409349in}}%
\pgfpathclose%
\pgfusepath{fill}%
\end{pgfscope}%
\begin{pgfscope}%
\pgfpathrectangle{\pgfqpoint{0.765000in}{0.660000in}}{\pgfqpoint{4.620000in}{4.620000in}}%
\pgfusepath{clip}%
\pgfsetbuttcap%
\pgfsetroundjoin%
\definecolor{currentfill}{rgb}{0.176471,0.192157,0.258824}%
\pgfsetfillcolor{currentfill}%
\pgfsetfillopacity{0.050000}%
\pgfsetlinewidth{0.000000pt}%
\definecolor{currentstroke}{rgb}{0.176471,0.192157,0.258824}%
\pgfsetstrokecolor{currentstroke}%
\pgfsetdash{}{0pt}%
\pgfpathmoveto{\pgfqpoint{2.011961in}{3.250829in}}%
\pgfpathlineto{\pgfqpoint{2.011961in}{3.356417in}}%
\pgfpathlineto{\pgfqpoint{2.124708in}{3.303486in}}%
\pgfpathlineto{\pgfqpoint{2.124708in}{3.197897in}}%
\pgfpathlineto{\pgfqpoint{2.011961in}{3.250829in}}%
\pgfpathclose%
\pgfusepath{fill}%
\end{pgfscope}%
\begin{pgfscope}%
\pgfpathrectangle{\pgfqpoint{0.765000in}{0.660000in}}{\pgfqpoint{4.620000in}{4.620000in}}%
\pgfusepath{clip}%
\pgfsetbuttcap%
\pgfsetroundjoin%
\definecolor{currentfill}{rgb}{0.176471,0.192157,0.258824}%
\pgfsetfillcolor{currentfill}%
\pgfsetfillopacity{0.050000}%
\pgfsetlinewidth{0.000000pt}%
\definecolor{currentstroke}{rgb}{0.176471,0.192157,0.258824}%
\pgfsetstrokecolor{currentstroke}%
\pgfsetdash{}{0pt}%
\pgfpathmoveto{\pgfqpoint{2.237456in}{3.250829in}}%
\pgfpathlineto{\pgfqpoint{2.237456in}{3.356417in}}%
\pgfpathlineto{\pgfqpoint{2.124708in}{3.303486in}}%
\pgfpathlineto{\pgfqpoint{2.124708in}{3.197897in}}%
\pgfpathlineto{\pgfqpoint{2.237456in}{3.250829in}}%
\pgfpathclose%
\pgfusepath{fill}%
\end{pgfscope}%
\begin{pgfscope}%
\pgfpathrectangle{\pgfqpoint{0.765000in}{0.660000in}}{\pgfqpoint{4.620000in}{4.620000in}}%
\pgfusepath{clip}%
\pgfsetbuttcap%
\pgfsetroundjoin%
\definecolor{currentfill}{rgb}{0.176471,0.192157,0.258824}%
\pgfsetfillcolor{currentfill}%
\pgfsetfillopacity{0.050000}%
\pgfsetlinewidth{0.000000pt}%
\definecolor{currentstroke}{rgb}{0.176471,0.192157,0.258824}%
\pgfsetstrokecolor{currentstroke}%
\pgfsetdash{}{0pt}%
\pgfpathmoveto{\pgfqpoint{1.615209in}{3.639632in}}%
\pgfpathlineto{\pgfqpoint{1.615209in}{3.745221in}}%
\pgfpathlineto{\pgfqpoint{1.727957in}{3.692289in}}%
\pgfpathlineto{\pgfqpoint{1.727957in}{3.586700in}}%
\pgfpathlineto{\pgfqpoint{1.615209in}{3.639632in}}%
\pgfpathclose%
\pgfusepath{fill}%
\end{pgfscope}%
\begin{pgfscope}%
\pgfpathrectangle{\pgfqpoint{0.765000in}{0.660000in}}{\pgfqpoint{4.620000in}{4.620000in}}%
\pgfusepath{clip}%
\pgfsetbuttcap%
\pgfsetroundjoin%
\definecolor{currentfill}{rgb}{0.176471,0.192157,0.258824}%
\pgfsetfillcolor{currentfill}%
\pgfsetfillopacity{0.050000}%
\pgfsetlinewidth{0.000000pt}%
\definecolor{currentstroke}{rgb}{0.176471,0.192157,0.258824}%
\pgfsetstrokecolor{currentstroke}%
\pgfsetdash{}{0pt}%
\pgfpathmoveto{\pgfqpoint{1.615209in}{3.639632in}}%
\pgfpathlineto{\pgfqpoint{1.615209in}{3.745221in}}%
\pgfpathlineto{\pgfqpoint{1.502462in}{3.692289in}}%
\pgfpathlineto{\pgfqpoint{1.502462in}{3.586700in}}%
\pgfpathlineto{\pgfqpoint{1.615209in}{3.639632in}}%
\pgfpathclose%
\pgfusepath{fill}%
\end{pgfscope}%
\begin{pgfscope}%
\pgfpathrectangle{\pgfqpoint{0.765000in}{0.660000in}}{\pgfqpoint{4.620000in}{4.620000in}}%
\pgfusepath{clip}%
\pgfsetbuttcap%
\pgfsetroundjoin%
\definecolor{currentfill}{rgb}{0.176471,0.192157,0.258824}%
\pgfsetfillcolor{currentfill}%
\pgfsetfillopacity{0.050000}%
\pgfsetlinewidth{0.000000pt}%
\definecolor{currentstroke}{rgb}{0.176471,0.192157,0.258824}%
\pgfsetstrokecolor{currentstroke}%
\pgfsetdash{}{0pt}%
\pgfpathmoveto{\pgfqpoint{1.615209in}{3.639632in}}%
\pgfpathlineto{\pgfqpoint{1.727957in}{3.586700in}}%
\pgfpathlineto{\pgfqpoint{1.615209in}{3.533768in}}%
\pgfpathlineto{\pgfqpoint{1.502462in}{3.586700in}}%
\pgfpathlineto{\pgfqpoint{1.615209in}{3.639632in}}%
\pgfpathclose%
\pgfusepath{fill}%
\end{pgfscope}%
\begin{pgfscope}%
\pgfpathrectangle{\pgfqpoint{0.765000in}{0.660000in}}{\pgfqpoint{4.620000in}{4.620000in}}%
\pgfusepath{clip}%
\pgfsetbuttcap%
\pgfsetroundjoin%
\definecolor{currentfill}{rgb}{0.176471,0.192157,0.258824}%
\pgfsetfillcolor{currentfill}%
\pgfsetfillopacity{0.050000}%
\pgfsetlinewidth{0.000000pt}%
\definecolor{currentstroke}{rgb}{0.176471,0.192157,0.258824}%
\pgfsetstrokecolor{currentstroke}%
\pgfsetdash{}{0pt}%
\pgfpathmoveto{\pgfqpoint{1.615209in}{3.745221in}}%
\pgfpathlineto{\pgfqpoint{1.727957in}{3.692289in}}%
\pgfpathlineto{\pgfqpoint{1.615209in}{3.639357in}}%
\pgfpathlineto{\pgfqpoint{1.502462in}{3.692289in}}%
\pgfpathlineto{\pgfqpoint{1.615209in}{3.745221in}}%
\pgfpathclose%
\pgfusepath{fill}%
\end{pgfscope}%
\begin{pgfscope}%
\pgfpathrectangle{\pgfqpoint{0.765000in}{0.660000in}}{\pgfqpoint{4.620000in}{4.620000in}}%
\pgfusepath{clip}%
\pgfsetbuttcap%
\pgfsetroundjoin%
\definecolor{currentfill}{rgb}{0.176471,0.192157,0.258824}%
\pgfsetfillcolor{currentfill}%
\pgfsetfillopacity{0.050000}%
\pgfsetlinewidth{0.000000pt}%
\definecolor{currentstroke}{rgb}{0.176471,0.192157,0.258824}%
\pgfsetstrokecolor{currentstroke}%
\pgfsetdash{}{0pt}%
\pgfpathmoveto{\pgfqpoint{1.502462in}{3.586700in}}%
\pgfpathlineto{\pgfqpoint{1.502462in}{3.692289in}}%
\pgfpathlineto{\pgfqpoint{1.615209in}{3.639357in}}%
\pgfpathlineto{\pgfqpoint{1.615209in}{3.533768in}}%
\pgfpathlineto{\pgfqpoint{1.502462in}{3.586700in}}%
\pgfpathclose%
\pgfusepath{fill}%
\end{pgfscope}%
\begin{pgfscope}%
\pgfpathrectangle{\pgfqpoint{0.765000in}{0.660000in}}{\pgfqpoint{4.620000in}{4.620000in}}%
\pgfusepath{clip}%
\pgfsetbuttcap%
\pgfsetroundjoin%
\definecolor{currentfill}{rgb}{0.176471,0.192157,0.258824}%
\pgfsetfillcolor{currentfill}%
\pgfsetfillopacity{0.050000}%
\pgfsetlinewidth{0.000000pt}%
\definecolor{currentstroke}{rgb}{0.176471,0.192157,0.258824}%
\pgfsetstrokecolor{currentstroke}%
\pgfsetdash{}{0pt}%
\pgfpathmoveto{\pgfqpoint{1.727957in}{3.586700in}}%
\pgfpathlineto{\pgfqpoint{1.727957in}{3.692289in}}%
\pgfpathlineto{\pgfqpoint{1.615209in}{3.639357in}}%
\pgfpathlineto{\pgfqpoint{1.615209in}{3.533768in}}%
\pgfpathlineto{\pgfqpoint{1.727957in}{3.586700in}}%
\pgfpathclose%
\pgfusepath{fill}%
\end{pgfscope}%
\begin{pgfscope}%
\pgfpathrectangle{\pgfqpoint{0.765000in}{0.660000in}}{\pgfqpoint{4.620000in}{4.620000in}}%
\pgfusepath{clip}%
\pgfsetbuttcap%
\pgfsetroundjoin%
\definecolor{currentfill}{rgb}{0.176471,0.192157,0.258824}%
\pgfsetfillcolor{currentfill}%
\pgfsetfillopacity{0.050000}%
\pgfsetlinewidth{0.000000pt}%
\definecolor{currentstroke}{rgb}{0.176471,0.192157,0.258824}%
\pgfsetstrokecolor{currentstroke}%
\pgfsetdash{}{0pt}%
\pgfpathmoveto{\pgfqpoint{1.569094in}{3.883520in}}%
\pgfpathlineto{\pgfqpoint{1.569094in}{3.989109in}}%
\pgfpathlineto{\pgfqpoint{1.681842in}{3.936177in}}%
\pgfpathlineto{\pgfqpoint{1.681842in}{3.830588in}}%
\pgfpathlineto{\pgfqpoint{1.569094in}{3.883520in}}%
\pgfpathclose%
\pgfusepath{fill}%
\end{pgfscope}%
\begin{pgfscope}%
\pgfpathrectangle{\pgfqpoint{0.765000in}{0.660000in}}{\pgfqpoint{4.620000in}{4.620000in}}%
\pgfusepath{clip}%
\pgfsetbuttcap%
\pgfsetroundjoin%
\definecolor{currentfill}{rgb}{0.176471,0.192157,0.258824}%
\pgfsetfillcolor{currentfill}%
\pgfsetfillopacity{0.050000}%
\pgfsetlinewidth{0.000000pt}%
\definecolor{currentstroke}{rgb}{0.176471,0.192157,0.258824}%
\pgfsetstrokecolor{currentstroke}%
\pgfsetdash{}{0pt}%
\pgfpathmoveto{\pgfqpoint{1.569094in}{3.883520in}}%
\pgfpathlineto{\pgfqpoint{1.569094in}{3.989109in}}%
\pgfpathlineto{\pgfqpoint{1.456347in}{3.936177in}}%
\pgfpathlineto{\pgfqpoint{1.456347in}{3.830588in}}%
\pgfpathlineto{\pgfqpoint{1.569094in}{3.883520in}}%
\pgfpathclose%
\pgfusepath{fill}%
\end{pgfscope}%
\begin{pgfscope}%
\pgfpathrectangle{\pgfqpoint{0.765000in}{0.660000in}}{\pgfqpoint{4.620000in}{4.620000in}}%
\pgfusepath{clip}%
\pgfsetbuttcap%
\pgfsetroundjoin%
\definecolor{currentfill}{rgb}{0.176471,0.192157,0.258824}%
\pgfsetfillcolor{currentfill}%
\pgfsetfillopacity{0.050000}%
\pgfsetlinewidth{0.000000pt}%
\definecolor{currentstroke}{rgb}{0.176471,0.192157,0.258824}%
\pgfsetstrokecolor{currentstroke}%
\pgfsetdash{}{0pt}%
\pgfpathmoveto{\pgfqpoint{1.569094in}{3.883520in}}%
\pgfpathlineto{\pgfqpoint{1.681842in}{3.830588in}}%
\pgfpathlineto{\pgfqpoint{1.569094in}{3.777656in}}%
\pgfpathlineto{\pgfqpoint{1.456347in}{3.830588in}}%
\pgfpathlineto{\pgfqpoint{1.569094in}{3.883520in}}%
\pgfpathclose%
\pgfusepath{fill}%
\end{pgfscope}%
\begin{pgfscope}%
\pgfpathrectangle{\pgfqpoint{0.765000in}{0.660000in}}{\pgfqpoint{4.620000in}{4.620000in}}%
\pgfusepath{clip}%
\pgfsetbuttcap%
\pgfsetroundjoin%
\definecolor{currentfill}{rgb}{0.176471,0.192157,0.258824}%
\pgfsetfillcolor{currentfill}%
\pgfsetfillopacity{0.050000}%
\pgfsetlinewidth{0.000000pt}%
\definecolor{currentstroke}{rgb}{0.176471,0.192157,0.258824}%
\pgfsetstrokecolor{currentstroke}%
\pgfsetdash{}{0pt}%
\pgfpathmoveto{\pgfqpoint{1.569094in}{3.989109in}}%
\pgfpathlineto{\pgfqpoint{1.681842in}{3.936177in}}%
\pgfpathlineto{\pgfqpoint{1.569094in}{3.883245in}}%
\pgfpathlineto{\pgfqpoint{1.456347in}{3.936177in}}%
\pgfpathlineto{\pgfqpoint{1.569094in}{3.989109in}}%
\pgfpathclose%
\pgfusepath{fill}%
\end{pgfscope}%
\begin{pgfscope}%
\pgfpathrectangle{\pgfqpoint{0.765000in}{0.660000in}}{\pgfqpoint{4.620000in}{4.620000in}}%
\pgfusepath{clip}%
\pgfsetbuttcap%
\pgfsetroundjoin%
\definecolor{currentfill}{rgb}{0.176471,0.192157,0.258824}%
\pgfsetfillcolor{currentfill}%
\pgfsetfillopacity{0.050000}%
\pgfsetlinewidth{0.000000pt}%
\definecolor{currentstroke}{rgb}{0.176471,0.192157,0.258824}%
\pgfsetstrokecolor{currentstroke}%
\pgfsetdash{}{0pt}%
\pgfpathmoveto{\pgfqpoint{1.456347in}{3.830588in}}%
\pgfpathlineto{\pgfqpoint{1.456347in}{3.936177in}}%
\pgfpathlineto{\pgfqpoint{1.569094in}{3.883245in}}%
\pgfpathlineto{\pgfqpoint{1.569094in}{3.777656in}}%
\pgfpathlineto{\pgfqpoint{1.456347in}{3.830588in}}%
\pgfpathclose%
\pgfusepath{fill}%
\end{pgfscope}%
\begin{pgfscope}%
\pgfpathrectangle{\pgfqpoint{0.765000in}{0.660000in}}{\pgfqpoint{4.620000in}{4.620000in}}%
\pgfusepath{clip}%
\pgfsetbuttcap%
\pgfsetroundjoin%
\definecolor{currentfill}{rgb}{0.176471,0.192157,0.258824}%
\pgfsetfillcolor{currentfill}%
\pgfsetfillopacity{0.050000}%
\pgfsetlinewidth{0.000000pt}%
\definecolor{currentstroke}{rgb}{0.176471,0.192157,0.258824}%
\pgfsetstrokecolor{currentstroke}%
\pgfsetdash{}{0pt}%
\pgfpathmoveto{\pgfqpoint{1.681842in}{3.830588in}}%
\pgfpathlineto{\pgfqpoint{1.681842in}{3.936177in}}%
\pgfpathlineto{\pgfqpoint{1.569094in}{3.883245in}}%
\pgfpathlineto{\pgfqpoint{1.569094in}{3.777656in}}%
\pgfpathlineto{\pgfqpoint{1.681842in}{3.830588in}}%
\pgfpathclose%
\pgfusepath{fill}%
\end{pgfscope}%
\begin{pgfscope}%
\pgfpathrectangle{\pgfqpoint{0.765000in}{0.660000in}}{\pgfqpoint{4.620000in}{4.620000in}}%
\pgfusepath{clip}%
\pgfsetbuttcap%
\pgfsetroundjoin%
\definecolor{currentfill}{rgb}{1.000000,0.576471,0.309804}%
\pgfsetfillcolor{currentfill}%
\pgfsetfillopacity{0.050000}%
\pgfsetlinewidth{0.000000pt}%
\definecolor{currentstroke}{rgb}{1.000000,0.576471,0.309804}%
\pgfsetstrokecolor{currentstroke}%
\pgfsetdash{}{0pt}%
\pgfpathmoveto{\pgfqpoint{4.866516in}{3.823516in}}%
\pgfpathlineto{\pgfqpoint{4.866516in}{3.929105in}}%
\pgfpathlineto{\pgfqpoint{4.979263in}{3.876173in}}%
\pgfpathlineto{\pgfqpoint{4.979263in}{3.770584in}}%
\pgfpathlineto{\pgfqpoint{4.866516in}{3.823516in}}%
\pgfpathclose%
\pgfusepath{fill}%
\end{pgfscope}%
\begin{pgfscope}%
\pgfpathrectangle{\pgfqpoint{0.765000in}{0.660000in}}{\pgfqpoint{4.620000in}{4.620000in}}%
\pgfusepath{clip}%
\pgfsetbuttcap%
\pgfsetroundjoin%
\definecolor{currentfill}{rgb}{1.000000,0.576471,0.309804}%
\pgfsetfillcolor{currentfill}%
\pgfsetfillopacity{0.050000}%
\pgfsetlinewidth{0.000000pt}%
\definecolor{currentstroke}{rgb}{1.000000,0.576471,0.309804}%
\pgfsetstrokecolor{currentstroke}%
\pgfsetdash{}{0pt}%
\pgfpathmoveto{\pgfqpoint{4.866516in}{3.823516in}}%
\pgfpathlineto{\pgfqpoint{4.866516in}{3.929105in}}%
\pgfpathlineto{\pgfqpoint{4.753768in}{3.876173in}}%
\pgfpathlineto{\pgfqpoint{4.753768in}{3.770584in}}%
\pgfpathlineto{\pgfqpoint{4.866516in}{3.823516in}}%
\pgfpathclose%
\pgfusepath{fill}%
\end{pgfscope}%
\begin{pgfscope}%
\pgfpathrectangle{\pgfqpoint{0.765000in}{0.660000in}}{\pgfqpoint{4.620000in}{4.620000in}}%
\pgfusepath{clip}%
\pgfsetbuttcap%
\pgfsetroundjoin%
\definecolor{currentfill}{rgb}{1.000000,0.576471,0.309804}%
\pgfsetfillcolor{currentfill}%
\pgfsetfillopacity{0.050000}%
\pgfsetlinewidth{0.000000pt}%
\definecolor{currentstroke}{rgb}{1.000000,0.576471,0.309804}%
\pgfsetstrokecolor{currentstroke}%
\pgfsetdash{}{0pt}%
\pgfpathmoveto{\pgfqpoint{4.866516in}{3.823516in}}%
\pgfpathlineto{\pgfqpoint{4.979263in}{3.770584in}}%
\pgfpathlineto{\pgfqpoint{4.866516in}{3.717652in}}%
\pgfpathlineto{\pgfqpoint{4.753768in}{3.770584in}}%
\pgfpathlineto{\pgfqpoint{4.866516in}{3.823516in}}%
\pgfpathclose%
\pgfusepath{fill}%
\end{pgfscope}%
\begin{pgfscope}%
\pgfpathrectangle{\pgfqpoint{0.765000in}{0.660000in}}{\pgfqpoint{4.620000in}{4.620000in}}%
\pgfusepath{clip}%
\pgfsetbuttcap%
\pgfsetroundjoin%
\definecolor{currentfill}{rgb}{1.000000,0.576471,0.309804}%
\pgfsetfillcolor{currentfill}%
\pgfsetfillopacity{0.050000}%
\pgfsetlinewidth{0.000000pt}%
\definecolor{currentstroke}{rgb}{1.000000,0.576471,0.309804}%
\pgfsetstrokecolor{currentstroke}%
\pgfsetdash{}{0pt}%
\pgfpathmoveto{\pgfqpoint{4.866516in}{3.929105in}}%
\pgfpathlineto{\pgfqpoint{4.979263in}{3.876173in}}%
\pgfpathlineto{\pgfqpoint{4.866516in}{3.823241in}}%
\pgfpathlineto{\pgfqpoint{4.753768in}{3.876173in}}%
\pgfpathlineto{\pgfqpoint{4.866516in}{3.929105in}}%
\pgfpathclose%
\pgfusepath{fill}%
\end{pgfscope}%
\begin{pgfscope}%
\pgfpathrectangle{\pgfqpoint{0.765000in}{0.660000in}}{\pgfqpoint{4.620000in}{4.620000in}}%
\pgfusepath{clip}%
\pgfsetbuttcap%
\pgfsetroundjoin%
\definecolor{currentfill}{rgb}{1.000000,0.576471,0.309804}%
\pgfsetfillcolor{currentfill}%
\pgfsetfillopacity{0.050000}%
\pgfsetlinewidth{0.000000pt}%
\definecolor{currentstroke}{rgb}{1.000000,0.576471,0.309804}%
\pgfsetstrokecolor{currentstroke}%
\pgfsetdash{}{0pt}%
\pgfpathmoveto{\pgfqpoint{4.753768in}{3.770584in}}%
\pgfpathlineto{\pgfqpoint{4.753768in}{3.876173in}}%
\pgfpathlineto{\pgfqpoint{4.866516in}{3.823241in}}%
\pgfpathlineto{\pgfqpoint{4.866516in}{3.717652in}}%
\pgfpathlineto{\pgfqpoint{4.753768in}{3.770584in}}%
\pgfpathclose%
\pgfusepath{fill}%
\end{pgfscope}%
\begin{pgfscope}%
\pgfpathrectangle{\pgfqpoint{0.765000in}{0.660000in}}{\pgfqpoint{4.620000in}{4.620000in}}%
\pgfusepath{clip}%
\pgfsetbuttcap%
\pgfsetroundjoin%
\definecolor{currentfill}{rgb}{1.000000,0.576471,0.309804}%
\pgfsetfillcolor{currentfill}%
\pgfsetfillopacity{0.050000}%
\pgfsetlinewidth{0.000000pt}%
\definecolor{currentstroke}{rgb}{1.000000,0.576471,0.309804}%
\pgfsetstrokecolor{currentstroke}%
\pgfsetdash{}{0pt}%
\pgfpathmoveto{\pgfqpoint{4.979263in}{3.770584in}}%
\pgfpathlineto{\pgfqpoint{4.979263in}{3.876173in}}%
\pgfpathlineto{\pgfqpoint{4.866516in}{3.823241in}}%
\pgfpathlineto{\pgfqpoint{4.866516in}{3.717652in}}%
\pgfpathlineto{\pgfqpoint{4.979263in}{3.770584in}}%
\pgfpathclose%
\pgfusepath{fill}%
\end{pgfscope}%
\begin{pgfscope}%
\pgfpathrectangle{\pgfqpoint{0.765000in}{0.660000in}}{\pgfqpoint{4.620000in}{4.620000in}}%
\pgfusepath{clip}%
\pgfsetbuttcap%
\pgfsetroundjoin%
\definecolor{currentfill}{rgb}{1.000000,0.576471,0.309804}%
\pgfsetfillcolor{currentfill}%
\pgfsetfillopacity{0.050000}%
\pgfsetlinewidth{0.000000pt}%
\definecolor{currentstroke}{rgb}{1.000000,0.576471,0.309804}%
\pgfsetstrokecolor{currentstroke}%
\pgfsetdash{}{0pt}%
\pgfpathmoveto{\pgfqpoint{4.444824in}{3.671848in}}%
\pgfpathlineto{\pgfqpoint{4.444824in}{3.777437in}}%
\pgfpathlineto{\pgfqpoint{4.557571in}{3.724505in}}%
\pgfpathlineto{\pgfqpoint{4.557571in}{3.618917in}}%
\pgfpathlineto{\pgfqpoint{4.444824in}{3.671848in}}%
\pgfpathclose%
\pgfusepath{fill}%
\end{pgfscope}%
\begin{pgfscope}%
\pgfpathrectangle{\pgfqpoint{0.765000in}{0.660000in}}{\pgfqpoint{4.620000in}{4.620000in}}%
\pgfusepath{clip}%
\pgfsetbuttcap%
\pgfsetroundjoin%
\definecolor{currentfill}{rgb}{1.000000,0.576471,0.309804}%
\pgfsetfillcolor{currentfill}%
\pgfsetfillopacity{0.050000}%
\pgfsetlinewidth{0.000000pt}%
\definecolor{currentstroke}{rgb}{1.000000,0.576471,0.309804}%
\pgfsetstrokecolor{currentstroke}%
\pgfsetdash{}{0pt}%
\pgfpathmoveto{\pgfqpoint{4.444824in}{3.671848in}}%
\pgfpathlineto{\pgfqpoint{4.444824in}{3.777437in}}%
\pgfpathlineto{\pgfqpoint{4.332076in}{3.724505in}}%
\pgfpathlineto{\pgfqpoint{4.332076in}{3.618917in}}%
\pgfpathlineto{\pgfqpoint{4.444824in}{3.671848in}}%
\pgfpathclose%
\pgfusepath{fill}%
\end{pgfscope}%
\begin{pgfscope}%
\pgfpathrectangle{\pgfqpoint{0.765000in}{0.660000in}}{\pgfqpoint{4.620000in}{4.620000in}}%
\pgfusepath{clip}%
\pgfsetbuttcap%
\pgfsetroundjoin%
\definecolor{currentfill}{rgb}{1.000000,0.576471,0.309804}%
\pgfsetfillcolor{currentfill}%
\pgfsetfillopacity{0.050000}%
\pgfsetlinewidth{0.000000pt}%
\definecolor{currentstroke}{rgb}{1.000000,0.576471,0.309804}%
\pgfsetstrokecolor{currentstroke}%
\pgfsetdash{}{0pt}%
\pgfpathmoveto{\pgfqpoint{4.444824in}{3.671848in}}%
\pgfpathlineto{\pgfqpoint{4.557571in}{3.618917in}}%
\pgfpathlineto{\pgfqpoint{4.444824in}{3.565985in}}%
\pgfpathlineto{\pgfqpoint{4.332076in}{3.618917in}}%
\pgfpathlineto{\pgfqpoint{4.444824in}{3.671848in}}%
\pgfpathclose%
\pgfusepath{fill}%
\end{pgfscope}%
\begin{pgfscope}%
\pgfpathrectangle{\pgfqpoint{0.765000in}{0.660000in}}{\pgfqpoint{4.620000in}{4.620000in}}%
\pgfusepath{clip}%
\pgfsetbuttcap%
\pgfsetroundjoin%
\definecolor{currentfill}{rgb}{1.000000,0.576471,0.309804}%
\pgfsetfillcolor{currentfill}%
\pgfsetfillopacity{0.050000}%
\pgfsetlinewidth{0.000000pt}%
\definecolor{currentstroke}{rgb}{1.000000,0.576471,0.309804}%
\pgfsetstrokecolor{currentstroke}%
\pgfsetdash{}{0pt}%
\pgfpathmoveto{\pgfqpoint{4.444824in}{3.777437in}}%
\pgfpathlineto{\pgfqpoint{4.557571in}{3.724505in}}%
\pgfpathlineto{\pgfqpoint{4.444824in}{3.671574in}}%
\pgfpathlineto{\pgfqpoint{4.332076in}{3.724505in}}%
\pgfpathlineto{\pgfqpoint{4.444824in}{3.777437in}}%
\pgfpathclose%
\pgfusepath{fill}%
\end{pgfscope}%
\begin{pgfscope}%
\pgfpathrectangle{\pgfqpoint{0.765000in}{0.660000in}}{\pgfqpoint{4.620000in}{4.620000in}}%
\pgfusepath{clip}%
\pgfsetbuttcap%
\pgfsetroundjoin%
\definecolor{currentfill}{rgb}{1.000000,0.576471,0.309804}%
\pgfsetfillcolor{currentfill}%
\pgfsetfillopacity{0.050000}%
\pgfsetlinewidth{0.000000pt}%
\definecolor{currentstroke}{rgb}{1.000000,0.576471,0.309804}%
\pgfsetstrokecolor{currentstroke}%
\pgfsetdash{}{0pt}%
\pgfpathmoveto{\pgfqpoint{4.332076in}{3.618917in}}%
\pgfpathlineto{\pgfqpoint{4.332076in}{3.724505in}}%
\pgfpathlineto{\pgfqpoint{4.444824in}{3.671574in}}%
\pgfpathlineto{\pgfqpoint{4.444824in}{3.565985in}}%
\pgfpathlineto{\pgfqpoint{4.332076in}{3.618917in}}%
\pgfpathclose%
\pgfusepath{fill}%
\end{pgfscope}%
\begin{pgfscope}%
\pgfpathrectangle{\pgfqpoint{0.765000in}{0.660000in}}{\pgfqpoint{4.620000in}{4.620000in}}%
\pgfusepath{clip}%
\pgfsetbuttcap%
\pgfsetroundjoin%
\definecolor{currentfill}{rgb}{1.000000,0.576471,0.309804}%
\pgfsetfillcolor{currentfill}%
\pgfsetfillopacity{0.050000}%
\pgfsetlinewidth{0.000000pt}%
\definecolor{currentstroke}{rgb}{1.000000,0.576471,0.309804}%
\pgfsetstrokecolor{currentstroke}%
\pgfsetdash{}{0pt}%
\pgfpathmoveto{\pgfqpoint{4.557571in}{3.618917in}}%
\pgfpathlineto{\pgfqpoint{4.557571in}{3.724505in}}%
\pgfpathlineto{\pgfqpoint{4.444824in}{3.671574in}}%
\pgfpathlineto{\pgfqpoint{4.444824in}{3.565985in}}%
\pgfpathlineto{\pgfqpoint{4.557571in}{3.618917in}}%
\pgfpathclose%
\pgfusepath{fill}%
\end{pgfscope}%
\begin{pgfscope}%
\pgfpathrectangle{\pgfqpoint{0.765000in}{0.660000in}}{\pgfqpoint{4.620000in}{4.620000in}}%
\pgfusepath{clip}%
\pgfsetbuttcap%
\pgfsetroundjoin%
\definecolor{currentfill}{rgb}{1.000000,0.576471,0.309804}%
\pgfsetfillcolor{currentfill}%
\pgfsetfillopacity{0.050000}%
\pgfsetlinewidth{0.000000pt}%
\definecolor{currentstroke}{rgb}{1.000000,0.576471,0.309804}%
\pgfsetstrokecolor{currentstroke}%
\pgfsetdash{}{0pt}%
\pgfpathmoveto{\pgfqpoint{4.327018in}{3.680097in}}%
\pgfpathlineto{\pgfqpoint{4.327018in}{3.785686in}}%
\pgfpathlineto{\pgfqpoint{4.439765in}{3.732754in}}%
\pgfpathlineto{\pgfqpoint{4.439765in}{3.627165in}}%
\pgfpathlineto{\pgfqpoint{4.327018in}{3.680097in}}%
\pgfpathclose%
\pgfusepath{fill}%
\end{pgfscope}%
\begin{pgfscope}%
\pgfpathrectangle{\pgfqpoint{0.765000in}{0.660000in}}{\pgfqpoint{4.620000in}{4.620000in}}%
\pgfusepath{clip}%
\pgfsetbuttcap%
\pgfsetroundjoin%
\definecolor{currentfill}{rgb}{1.000000,0.576471,0.309804}%
\pgfsetfillcolor{currentfill}%
\pgfsetfillopacity{0.050000}%
\pgfsetlinewidth{0.000000pt}%
\definecolor{currentstroke}{rgb}{1.000000,0.576471,0.309804}%
\pgfsetstrokecolor{currentstroke}%
\pgfsetdash{}{0pt}%
\pgfpathmoveto{\pgfqpoint{4.327018in}{3.680097in}}%
\pgfpathlineto{\pgfqpoint{4.327018in}{3.785686in}}%
\pgfpathlineto{\pgfqpoint{4.214270in}{3.732754in}}%
\pgfpathlineto{\pgfqpoint{4.214270in}{3.627165in}}%
\pgfpathlineto{\pgfqpoint{4.327018in}{3.680097in}}%
\pgfpathclose%
\pgfusepath{fill}%
\end{pgfscope}%
\begin{pgfscope}%
\pgfpathrectangle{\pgfqpoint{0.765000in}{0.660000in}}{\pgfqpoint{4.620000in}{4.620000in}}%
\pgfusepath{clip}%
\pgfsetbuttcap%
\pgfsetroundjoin%
\definecolor{currentfill}{rgb}{1.000000,0.576471,0.309804}%
\pgfsetfillcolor{currentfill}%
\pgfsetfillopacity{0.050000}%
\pgfsetlinewidth{0.000000pt}%
\definecolor{currentstroke}{rgb}{1.000000,0.576471,0.309804}%
\pgfsetstrokecolor{currentstroke}%
\pgfsetdash{}{0pt}%
\pgfpathmoveto{\pgfqpoint{4.327018in}{3.680097in}}%
\pgfpathlineto{\pgfqpoint{4.439765in}{3.627165in}}%
\pgfpathlineto{\pgfqpoint{4.327018in}{3.574234in}}%
\pgfpathlineto{\pgfqpoint{4.214270in}{3.627165in}}%
\pgfpathlineto{\pgfqpoint{4.327018in}{3.680097in}}%
\pgfpathclose%
\pgfusepath{fill}%
\end{pgfscope}%
\begin{pgfscope}%
\pgfpathrectangle{\pgfqpoint{0.765000in}{0.660000in}}{\pgfqpoint{4.620000in}{4.620000in}}%
\pgfusepath{clip}%
\pgfsetbuttcap%
\pgfsetroundjoin%
\definecolor{currentfill}{rgb}{1.000000,0.576471,0.309804}%
\pgfsetfillcolor{currentfill}%
\pgfsetfillopacity{0.050000}%
\pgfsetlinewidth{0.000000pt}%
\definecolor{currentstroke}{rgb}{1.000000,0.576471,0.309804}%
\pgfsetstrokecolor{currentstroke}%
\pgfsetdash{}{0pt}%
\pgfpathmoveto{\pgfqpoint{4.327018in}{3.785686in}}%
\pgfpathlineto{\pgfqpoint{4.439765in}{3.732754in}}%
\pgfpathlineto{\pgfqpoint{4.327018in}{3.679822in}}%
\pgfpathlineto{\pgfqpoint{4.214270in}{3.732754in}}%
\pgfpathlineto{\pgfqpoint{4.327018in}{3.785686in}}%
\pgfpathclose%
\pgfusepath{fill}%
\end{pgfscope}%
\begin{pgfscope}%
\pgfpathrectangle{\pgfqpoint{0.765000in}{0.660000in}}{\pgfqpoint{4.620000in}{4.620000in}}%
\pgfusepath{clip}%
\pgfsetbuttcap%
\pgfsetroundjoin%
\definecolor{currentfill}{rgb}{1.000000,0.576471,0.309804}%
\pgfsetfillcolor{currentfill}%
\pgfsetfillopacity{0.050000}%
\pgfsetlinewidth{0.000000pt}%
\definecolor{currentstroke}{rgb}{1.000000,0.576471,0.309804}%
\pgfsetstrokecolor{currentstroke}%
\pgfsetdash{}{0pt}%
\pgfpathmoveto{\pgfqpoint{4.214270in}{3.627165in}}%
\pgfpathlineto{\pgfqpoint{4.214270in}{3.732754in}}%
\pgfpathlineto{\pgfqpoint{4.327018in}{3.679822in}}%
\pgfpathlineto{\pgfqpoint{4.327018in}{3.574234in}}%
\pgfpathlineto{\pgfqpoint{4.214270in}{3.627165in}}%
\pgfpathclose%
\pgfusepath{fill}%
\end{pgfscope}%
\begin{pgfscope}%
\pgfpathrectangle{\pgfqpoint{0.765000in}{0.660000in}}{\pgfqpoint{4.620000in}{4.620000in}}%
\pgfusepath{clip}%
\pgfsetbuttcap%
\pgfsetroundjoin%
\definecolor{currentfill}{rgb}{1.000000,0.576471,0.309804}%
\pgfsetfillcolor{currentfill}%
\pgfsetfillopacity{0.050000}%
\pgfsetlinewidth{0.000000pt}%
\definecolor{currentstroke}{rgb}{1.000000,0.576471,0.309804}%
\pgfsetstrokecolor{currentstroke}%
\pgfsetdash{}{0pt}%
\pgfpathmoveto{\pgfqpoint{4.439765in}{3.627165in}}%
\pgfpathlineto{\pgfqpoint{4.439765in}{3.732754in}}%
\pgfpathlineto{\pgfqpoint{4.327018in}{3.679822in}}%
\pgfpathlineto{\pgfqpoint{4.327018in}{3.574234in}}%
\pgfpathlineto{\pgfqpoint{4.439765in}{3.627165in}}%
\pgfpathclose%
\pgfusepath{fill}%
\end{pgfscope}%
\begin{pgfscope}%
\pgfpathrectangle{\pgfqpoint{0.765000in}{0.660000in}}{\pgfqpoint{4.620000in}{4.620000in}}%
\pgfusepath{clip}%
\pgfsetbuttcap%
\pgfsetroundjoin%
\definecolor{currentfill}{rgb}{1.000000,0.576471,0.309804}%
\pgfsetfillcolor{currentfill}%
\pgfsetfillopacity{0.050000}%
\pgfsetlinewidth{0.000000pt}%
\definecolor{currentstroke}{rgb}{1.000000,0.576471,0.309804}%
\pgfsetstrokecolor{currentstroke}%
\pgfsetdash{}{0pt}%
\pgfpathmoveto{\pgfqpoint{4.727949in}{3.450575in}}%
\pgfpathlineto{\pgfqpoint{4.727949in}{3.556163in}}%
\pgfpathlineto{\pgfqpoint{4.840697in}{3.503232in}}%
\pgfpathlineto{\pgfqpoint{4.840697in}{3.397643in}}%
\pgfpathlineto{\pgfqpoint{4.727949in}{3.450575in}}%
\pgfpathclose%
\pgfusepath{fill}%
\end{pgfscope}%
\begin{pgfscope}%
\pgfpathrectangle{\pgfqpoint{0.765000in}{0.660000in}}{\pgfqpoint{4.620000in}{4.620000in}}%
\pgfusepath{clip}%
\pgfsetbuttcap%
\pgfsetroundjoin%
\definecolor{currentfill}{rgb}{1.000000,0.576471,0.309804}%
\pgfsetfillcolor{currentfill}%
\pgfsetfillopacity{0.050000}%
\pgfsetlinewidth{0.000000pt}%
\definecolor{currentstroke}{rgb}{1.000000,0.576471,0.309804}%
\pgfsetstrokecolor{currentstroke}%
\pgfsetdash{}{0pt}%
\pgfpathmoveto{\pgfqpoint{4.727949in}{3.450575in}}%
\pgfpathlineto{\pgfqpoint{4.727949in}{3.556163in}}%
\pgfpathlineto{\pgfqpoint{4.615202in}{3.503232in}}%
\pgfpathlineto{\pgfqpoint{4.615202in}{3.397643in}}%
\pgfpathlineto{\pgfqpoint{4.727949in}{3.450575in}}%
\pgfpathclose%
\pgfusepath{fill}%
\end{pgfscope}%
\begin{pgfscope}%
\pgfpathrectangle{\pgfqpoint{0.765000in}{0.660000in}}{\pgfqpoint{4.620000in}{4.620000in}}%
\pgfusepath{clip}%
\pgfsetbuttcap%
\pgfsetroundjoin%
\definecolor{currentfill}{rgb}{1.000000,0.576471,0.309804}%
\pgfsetfillcolor{currentfill}%
\pgfsetfillopacity{0.050000}%
\pgfsetlinewidth{0.000000pt}%
\definecolor{currentstroke}{rgb}{1.000000,0.576471,0.309804}%
\pgfsetstrokecolor{currentstroke}%
\pgfsetdash{}{0pt}%
\pgfpathmoveto{\pgfqpoint{4.727949in}{3.450575in}}%
\pgfpathlineto{\pgfqpoint{4.840697in}{3.397643in}}%
\pgfpathlineto{\pgfqpoint{4.727949in}{3.344711in}}%
\pgfpathlineto{\pgfqpoint{4.615202in}{3.397643in}}%
\pgfpathlineto{\pgfqpoint{4.727949in}{3.450575in}}%
\pgfpathclose%
\pgfusepath{fill}%
\end{pgfscope}%
\begin{pgfscope}%
\pgfpathrectangle{\pgfqpoint{0.765000in}{0.660000in}}{\pgfqpoint{4.620000in}{4.620000in}}%
\pgfusepath{clip}%
\pgfsetbuttcap%
\pgfsetroundjoin%
\definecolor{currentfill}{rgb}{1.000000,0.576471,0.309804}%
\pgfsetfillcolor{currentfill}%
\pgfsetfillopacity{0.050000}%
\pgfsetlinewidth{0.000000pt}%
\definecolor{currentstroke}{rgb}{1.000000,0.576471,0.309804}%
\pgfsetstrokecolor{currentstroke}%
\pgfsetdash{}{0pt}%
\pgfpathmoveto{\pgfqpoint{4.727949in}{3.556163in}}%
\pgfpathlineto{\pgfqpoint{4.840697in}{3.503232in}}%
\pgfpathlineto{\pgfqpoint{4.727949in}{3.450300in}}%
\pgfpathlineto{\pgfqpoint{4.615202in}{3.503232in}}%
\pgfpathlineto{\pgfqpoint{4.727949in}{3.556163in}}%
\pgfpathclose%
\pgfusepath{fill}%
\end{pgfscope}%
\begin{pgfscope}%
\pgfpathrectangle{\pgfqpoint{0.765000in}{0.660000in}}{\pgfqpoint{4.620000in}{4.620000in}}%
\pgfusepath{clip}%
\pgfsetbuttcap%
\pgfsetroundjoin%
\definecolor{currentfill}{rgb}{1.000000,0.576471,0.309804}%
\pgfsetfillcolor{currentfill}%
\pgfsetfillopacity{0.050000}%
\pgfsetlinewidth{0.000000pt}%
\definecolor{currentstroke}{rgb}{1.000000,0.576471,0.309804}%
\pgfsetstrokecolor{currentstroke}%
\pgfsetdash{}{0pt}%
\pgfpathmoveto{\pgfqpoint{4.615202in}{3.397643in}}%
\pgfpathlineto{\pgfqpoint{4.615202in}{3.503232in}}%
\pgfpathlineto{\pgfqpoint{4.727949in}{3.450300in}}%
\pgfpathlineto{\pgfqpoint{4.727949in}{3.344711in}}%
\pgfpathlineto{\pgfqpoint{4.615202in}{3.397643in}}%
\pgfpathclose%
\pgfusepath{fill}%
\end{pgfscope}%
\begin{pgfscope}%
\pgfpathrectangle{\pgfqpoint{0.765000in}{0.660000in}}{\pgfqpoint{4.620000in}{4.620000in}}%
\pgfusepath{clip}%
\pgfsetbuttcap%
\pgfsetroundjoin%
\definecolor{currentfill}{rgb}{1.000000,0.576471,0.309804}%
\pgfsetfillcolor{currentfill}%
\pgfsetfillopacity{0.050000}%
\pgfsetlinewidth{0.000000pt}%
\definecolor{currentstroke}{rgb}{1.000000,0.576471,0.309804}%
\pgfsetstrokecolor{currentstroke}%
\pgfsetdash{}{0pt}%
\pgfpathmoveto{\pgfqpoint{4.840697in}{3.397643in}}%
\pgfpathlineto{\pgfqpoint{4.840697in}{3.503232in}}%
\pgfpathlineto{\pgfqpoint{4.727949in}{3.450300in}}%
\pgfpathlineto{\pgfqpoint{4.727949in}{3.344711in}}%
\pgfpathlineto{\pgfqpoint{4.840697in}{3.397643in}}%
\pgfpathclose%
\pgfusepath{fill}%
\end{pgfscope}%
\begin{pgfscope}%
\pgfpathrectangle{\pgfqpoint{0.765000in}{0.660000in}}{\pgfqpoint{4.620000in}{4.620000in}}%
\pgfusepath{clip}%
\pgfsetbuttcap%
\pgfsetroundjoin%
\definecolor{currentfill}{rgb}{0.176471,0.192157,0.258824}%
\pgfsetfillcolor{currentfill}%
\pgfsetfillopacity{0.050000}%
\pgfsetlinewidth{0.000000pt}%
\definecolor{currentstroke}{rgb}{0.176471,0.192157,0.258824}%
\pgfsetstrokecolor{currentstroke}%
\pgfsetdash{}{0pt}%
\pgfpathmoveto{\pgfqpoint{1.917679in}{3.688108in}}%
\pgfpathlineto{\pgfqpoint{1.917679in}{3.793697in}}%
\pgfpathlineto{\pgfqpoint{2.030427in}{3.740765in}}%
\pgfpathlineto{\pgfqpoint{2.030427in}{3.635176in}}%
\pgfpathlineto{\pgfqpoint{1.917679in}{3.688108in}}%
\pgfpathclose%
\pgfusepath{fill}%
\end{pgfscope}%
\begin{pgfscope}%
\pgfpathrectangle{\pgfqpoint{0.765000in}{0.660000in}}{\pgfqpoint{4.620000in}{4.620000in}}%
\pgfusepath{clip}%
\pgfsetbuttcap%
\pgfsetroundjoin%
\definecolor{currentfill}{rgb}{0.176471,0.192157,0.258824}%
\pgfsetfillcolor{currentfill}%
\pgfsetfillopacity{0.050000}%
\pgfsetlinewidth{0.000000pt}%
\definecolor{currentstroke}{rgb}{0.176471,0.192157,0.258824}%
\pgfsetstrokecolor{currentstroke}%
\pgfsetdash{}{0pt}%
\pgfpathmoveto{\pgfqpoint{1.917679in}{3.688108in}}%
\pgfpathlineto{\pgfqpoint{1.917679in}{3.793697in}}%
\pgfpathlineto{\pgfqpoint{1.804932in}{3.740765in}}%
\pgfpathlineto{\pgfqpoint{1.804932in}{3.635176in}}%
\pgfpathlineto{\pgfqpoint{1.917679in}{3.688108in}}%
\pgfpathclose%
\pgfusepath{fill}%
\end{pgfscope}%
\begin{pgfscope}%
\pgfpathrectangle{\pgfqpoint{0.765000in}{0.660000in}}{\pgfqpoint{4.620000in}{4.620000in}}%
\pgfusepath{clip}%
\pgfsetbuttcap%
\pgfsetroundjoin%
\definecolor{currentfill}{rgb}{0.176471,0.192157,0.258824}%
\pgfsetfillcolor{currentfill}%
\pgfsetfillopacity{0.050000}%
\pgfsetlinewidth{0.000000pt}%
\definecolor{currentstroke}{rgb}{0.176471,0.192157,0.258824}%
\pgfsetstrokecolor{currentstroke}%
\pgfsetdash{}{0pt}%
\pgfpathmoveto{\pgfqpoint{1.917679in}{3.688108in}}%
\pgfpathlineto{\pgfqpoint{2.030427in}{3.635176in}}%
\pgfpathlineto{\pgfqpoint{1.917679in}{3.582245in}}%
\pgfpathlineto{\pgfqpoint{1.804932in}{3.635176in}}%
\pgfpathlineto{\pgfqpoint{1.917679in}{3.688108in}}%
\pgfpathclose%
\pgfusepath{fill}%
\end{pgfscope}%
\begin{pgfscope}%
\pgfpathrectangle{\pgfqpoint{0.765000in}{0.660000in}}{\pgfqpoint{4.620000in}{4.620000in}}%
\pgfusepath{clip}%
\pgfsetbuttcap%
\pgfsetroundjoin%
\definecolor{currentfill}{rgb}{0.176471,0.192157,0.258824}%
\pgfsetfillcolor{currentfill}%
\pgfsetfillopacity{0.050000}%
\pgfsetlinewidth{0.000000pt}%
\definecolor{currentstroke}{rgb}{0.176471,0.192157,0.258824}%
\pgfsetstrokecolor{currentstroke}%
\pgfsetdash{}{0pt}%
\pgfpathmoveto{\pgfqpoint{1.917679in}{3.793697in}}%
\pgfpathlineto{\pgfqpoint{2.030427in}{3.740765in}}%
\pgfpathlineto{\pgfqpoint{1.917679in}{3.687834in}}%
\pgfpathlineto{\pgfqpoint{1.804932in}{3.740765in}}%
\pgfpathlineto{\pgfqpoint{1.917679in}{3.793697in}}%
\pgfpathclose%
\pgfusepath{fill}%
\end{pgfscope}%
\begin{pgfscope}%
\pgfpathrectangle{\pgfqpoint{0.765000in}{0.660000in}}{\pgfqpoint{4.620000in}{4.620000in}}%
\pgfusepath{clip}%
\pgfsetbuttcap%
\pgfsetroundjoin%
\definecolor{currentfill}{rgb}{0.176471,0.192157,0.258824}%
\pgfsetfillcolor{currentfill}%
\pgfsetfillopacity{0.050000}%
\pgfsetlinewidth{0.000000pt}%
\definecolor{currentstroke}{rgb}{0.176471,0.192157,0.258824}%
\pgfsetstrokecolor{currentstroke}%
\pgfsetdash{}{0pt}%
\pgfpathmoveto{\pgfqpoint{1.804932in}{3.635176in}}%
\pgfpathlineto{\pgfqpoint{1.804932in}{3.740765in}}%
\pgfpathlineto{\pgfqpoint{1.917679in}{3.687834in}}%
\pgfpathlineto{\pgfqpoint{1.917679in}{3.582245in}}%
\pgfpathlineto{\pgfqpoint{1.804932in}{3.635176in}}%
\pgfpathclose%
\pgfusepath{fill}%
\end{pgfscope}%
\begin{pgfscope}%
\pgfpathrectangle{\pgfqpoint{0.765000in}{0.660000in}}{\pgfqpoint{4.620000in}{4.620000in}}%
\pgfusepath{clip}%
\pgfsetbuttcap%
\pgfsetroundjoin%
\definecolor{currentfill}{rgb}{0.176471,0.192157,0.258824}%
\pgfsetfillcolor{currentfill}%
\pgfsetfillopacity{0.050000}%
\pgfsetlinewidth{0.000000pt}%
\definecolor{currentstroke}{rgb}{0.176471,0.192157,0.258824}%
\pgfsetstrokecolor{currentstroke}%
\pgfsetdash{}{0pt}%
\pgfpathmoveto{\pgfqpoint{2.030427in}{3.635176in}}%
\pgfpathlineto{\pgfqpoint{2.030427in}{3.740765in}}%
\pgfpathlineto{\pgfqpoint{1.917679in}{3.687834in}}%
\pgfpathlineto{\pgfqpoint{1.917679in}{3.582245in}}%
\pgfpathlineto{\pgfqpoint{2.030427in}{3.635176in}}%
\pgfpathclose%
\pgfusepath{fill}%
\end{pgfscope}%
\begin{pgfscope}%
\pgfpathrectangle{\pgfqpoint{0.765000in}{0.660000in}}{\pgfqpoint{4.620000in}{4.620000in}}%
\pgfusepath{clip}%
\pgfsetbuttcap%
\pgfsetroundjoin%
\definecolor{currentfill}{rgb}{0.176471,0.192157,0.258824}%
\pgfsetfillcolor{currentfill}%
\pgfsetfillopacity{0.050000}%
\pgfsetlinewidth{0.000000pt}%
\definecolor{currentstroke}{rgb}{0.176471,0.192157,0.258824}%
\pgfsetstrokecolor{currentstroke}%
\pgfsetdash{}{0pt}%
\pgfpathmoveto{\pgfqpoint{1.323944in}{3.676928in}}%
\pgfpathlineto{\pgfqpoint{1.323944in}{3.782517in}}%
\pgfpathlineto{\pgfqpoint{1.436691in}{3.729585in}}%
\pgfpathlineto{\pgfqpoint{1.436691in}{3.623996in}}%
\pgfpathlineto{\pgfqpoint{1.323944in}{3.676928in}}%
\pgfpathclose%
\pgfusepath{fill}%
\end{pgfscope}%
\begin{pgfscope}%
\pgfpathrectangle{\pgfqpoint{0.765000in}{0.660000in}}{\pgfqpoint{4.620000in}{4.620000in}}%
\pgfusepath{clip}%
\pgfsetbuttcap%
\pgfsetroundjoin%
\definecolor{currentfill}{rgb}{0.176471,0.192157,0.258824}%
\pgfsetfillcolor{currentfill}%
\pgfsetfillopacity{0.050000}%
\pgfsetlinewidth{0.000000pt}%
\definecolor{currentstroke}{rgb}{0.176471,0.192157,0.258824}%
\pgfsetstrokecolor{currentstroke}%
\pgfsetdash{}{0pt}%
\pgfpathmoveto{\pgfqpoint{1.323944in}{3.676928in}}%
\pgfpathlineto{\pgfqpoint{1.323944in}{3.782517in}}%
\pgfpathlineto{\pgfqpoint{1.211196in}{3.729585in}}%
\pgfpathlineto{\pgfqpoint{1.211196in}{3.623996in}}%
\pgfpathlineto{\pgfqpoint{1.323944in}{3.676928in}}%
\pgfpathclose%
\pgfusepath{fill}%
\end{pgfscope}%
\begin{pgfscope}%
\pgfpathrectangle{\pgfqpoint{0.765000in}{0.660000in}}{\pgfqpoint{4.620000in}{4.620000in}}%
\pgfusepath{clip}%
\pgfsetbuttcap%
\pgfsetroundjoin%
\definecolor{currentfill}{rgb}{0.176471,0.192157,0.258824}%
\pgfsetfillcolor{currentfill}%
\pgfsetfillopacity{0.050000}%
\pgfsetlinewidth{0.000000pt}%
\definecolor{currentstroke}{rgb}{0.176471,0.192157,0.258824}%
\pgfsetstrokecolor{currentstroke}%
\pgfsetdash{}{0pt}%
\pgfpathmoveto{\pgfqpoint{1.323944in}{3.676928in}}%
\pgfpathlineto{\pgfqpoint{1.436691in}{3.623996in}}%
\pgfpathlineto{\pgfqpoint{1.323944in}{3.571065in}}%
\pgfpathlineto{\pgfqpoint{1.211196in}{3.623996in}}%
\pgfpathlineto{\pgfqpoint{1.323944in}{3.676928in}}%
\pgfpathclose%
\pgfusepath{fill}%
\end{pgfscope}%
\begin{pgfscope}%
\pgfpathrectangle{\pgfqpoint{0.765000in}{0.660000in}}{\pgfqpoint{4.620000in}{4.620000in}}%
\pgfusepath{clip}%
\pgfsetbuttcap%
\pgfsetroundjoin%
\definecolor{currentfill}{rgb}{0.176471,0.192157,0.258824}%
\pgfsetfillcolor{currentfill}%
\pgfsetfillopacity{0.050000}%
\pgfsetlinewidth{0.000000pt}%
\definecolor{currentstroke}{rgb}{0.176471,0.192157,0.258824}%
\pgfsetstrokecolor{currentstroke}%
\pgfsetdash{}{0pt}%
\pgfpathmoveto{\pgfqpoint{1.323944in}{3.782517in}}%
\pgfpathlineto{\pgfqpoint{1.436691in}{3.729585in}}%
\pgfpathlineto{\pgfqpoint{1.323944in}{3.676653in}}%
\pgfpathlineto{\pgfqpoint{1.211196in}{3.729585in}}%
\pgfpathlineto{\pgfqpoint{1.323944in}{3.782517in}}%
\pgfpathclose%
\pgfusepath{fill}%
\end{pgfscope}%
\begin{pgfscope}%
\pgfpathrectangle{\pgfqpoint{0.765000in}{0.660000in}}{\pgfqpoint{4.620000in}{4.620000in}}%
\pgfusepath{clip}%
\pgfsetbuttcap%
\pgfsetroundjoin%
\definecolor{currentfill}{rgb}{0.176471,0.192157,0.258824}%
\pgfsetfillcolor{currentfill}%
\pgfsetfillopacity{0.050000}%
\pgfsetlinewidth{0.000000pt}%
\definecolor{currentstroke}{rgb}{0.176471,0.192157,0.258824}%
\pgfsetstrokecolor{currentstroke}%
\pgfsetdash{}{0pt}%
\pgfpathmoveto{\pgfqpoint{1.211196in}{3.623996in}}%
\pgfpathlineto{\pgfqpoint{1.211196in}{3.729585in}}%
\pgfpathlineto{\pgfqpoint{1.323944in}{3.676653in}}%
\pgfpathlineto{\pgfqpoint{1.323944in}{3.571065in}}%
\pgfpathlineto{\pgfqpoint{1.211196in}{3.623996in}}%
\pgfpathclose%
\pgfusepath{fill}%
\end{pgfscope}%
\begin{pgfscope}%
\pgfpathrectangle{\pgfqpoint{0.765000in}{0.660000in}}{\pgfqpoint{4.620000in}{4.620000in}}%
\pgfusepath{clip}%
\pgfsetbuttcap%
\pgfsetroundjoin%
\definecolor{currentfill}{rgb}{0.176471,0.192157,0.258824}%
\pgfsetfillcolor{currentfill}%
\pgfsetfillopacity{0.050000}%
\pgfsetlinewidth{0.000000pt}%
\definecolor{currentstroke}{rgb}{0.176471,0.192157,0.258824}%
\pgfsetstrokecolor{currentstroke}%
\pgfsetdash{}{0pt}%
\pgfpathmoveto{\pgfqpoint{1.436691in}{3.623996in}}%
\pgfpathlineto{\pgfqpoint{1.436691in}{3.729585in}}%
\pgfpathlineto{\pgfqpoint{1.323944in}{3.676653in}}%
\pgfpathlineto{\pgfqpoint{1.323944in}{3.571065in}}%
\pgfpathlineto{\pgfqpoint{1.436691in}{3.623996in}}%
\pgfpathclose%
\pgfusepath{fill}%
\end{pgfscope}%
\begin{pgfscope}%
\pgfpathrectangle{\pgfqpoint{0.765000in}{0.660000in}}{\pgfqpoint{4.620000in}{4.620000in}}%
\pgfusepath{clip}%
\pgfsetbuttcap%
\pgfsetroundjoin%
\definecolor{currentfill}{rgb}{0.176471,0.192157,0.258824}%
\pgfsetfillcolor{currentfill}%
\pgfsetfillopacity{0.050000}%
\pgfsetlinewidth{0.000000pt}%
\definecolor{currentstroke}{rgb}{0.176471,0.192157,0.258824}%
\pgfsetstrokecolor{currentstroke}%
\pgfsetdash{}{0pt}%
\pgfpathmoveto{\pgfqpoint{1.820042in}{3.572254in}}%
\pgfpathlineto{\pgfqpoint{1.820042in}{3.677842in}}%
\pgfpathlineto{\pgfqpoint{1.932790in}{3.624911in}}%
\pgfpathlineto{\pgfqpoint{1.932790in}{3.519322in}}%
\pgfpathlineto{\pgfqpoint{1.820042in}{3.572254in}}%
\pgfpathclose%
\pgfusepath{fill}%
\end{pgfscope}%
\begin{pgfscope}%
\pgfpathrectangle{\pgfqpoint{0.765000in}{0.660000in}}{\pgfqpoint{4.620000in}{4.620000in}}%
\pgfusepath{clip}%
\pgfsetbuttcap%
\pgfsetroundjoin%
\definecolor{currentfill}{rgb}{0.176471,0.192157,0.258824}%
\pgfsetfillcolor{currentfill}%
\pgfsetfillopacity{0.050000}%
\pgfsetlinewidth{0.000000pt}%
\definecolor{currentstroke}{rgb}{0.176471,0.192157,0.258824}%
\pgfsetstrokecolor{currentstroke}%
\pgfsetdash{}{0pt}%
\pgfpathmoveto{\pgfqpoint{1.820042in}{3.572254in}}%
\pgfpathlineto{\pgfqpoint{1.820042in}{3.677842in}}%
\pgfpathlineto{\pgfqpoint{1.707295in}{3.624911in}}%
\pgfpathlineto{\pgfqpoint{1.707295in}{3.519322in}}%
\pgfpathlineto{\pgfqpoint{1.820042in}{3.572254in}}%
\pgfpathclose%
\pgfusepath{fill}%
\end{pgfscope}%
\begin{pgfscope}%
\pgfpathrectangle{\pgfqpoint{0.765000in}{0.660000in}}{\pgfqpoint{4.620000in}{4.620000in}}%
\pgfusepath{clip}%
\pgfsetbuttcap%
\pgfsetroundjoin%
\definecolor{currentfill}{rgb}{0.176471,0.192157,0.258824}%
\pgfsetfillcolor{currentfill}%
\pgfsetfillopacity{0.050000}%
\pgfsetlinewidth{0.000000pt}%
\definecolor{currentstroke}{rgb}{0.176471,0.192157,0.258824}%
\pgfsetstrokecolor{currentstroke}%
\pgfsetdash{}{0pt}%
\pgfpathmoveto{\pgfqpoint{1.820042in}{3.572254in}}%
\pgfpathlineto{\pgfqpoint{1.932790in}{3.519322in}}%
\pgfpathlineto{\pgfqpoint{1.820042in}{3.466390in}}%
\pgfpathlineto{\pgfqpoint{1.707295in}{3.519322in}}%
\pgfpathlineto{\pgfqpoint{1.820042in}{3.572254in}}%
\pgfpathclose%
\pgfusepath{fill}%
\end{pgfscope}%
\begin{pgfscope}%
\pgfpathrectangle{\pgfqpoint{0.765000in}{0.660000in}}{\pgfqpoint{4.620000in}{4.620000in}}%
\pgfusepath{clip}%
\pgfsetbuttcap%
\pgfsetroundjoin%
\definecolor{currentfill}{rgb}{0.176471,0.192157,0.258824}%
\pgfsetfillcolor{currentfill}%
\pgfsetfillopacity{0.050000}%
\pgfsetlinewidth{0.000000pt}%
\definecolor{currentstroke}{rgb}{0.176471,0.192157,0.258824}%
\pgfsetstrokecolor{currentstroke}%
\pgfsetdash{}{0pt}%
\pgfpathmoveto{\pgfqpoint{1.820042in}{3.677842in}}%
\pgfpathlineto{\pgfqpoint{1.932790in}{3.624911in}}%
\pgfpathlineto{\pgfqpoint{1.820042in}{3.571979in}}%
\pgfpathlineto{\pgfqpoint{1.707295in}{3.624911in}}%
\pgfpathlineto{\pgfqpoint{1.820042in}{3.677842in}}%
\pgfpathclose%
\pgfusepath{fill}%
\end{pgfscope}%
\begin{pgfscope}%
\pgfpathrectangle{\pgfqpoint{0.765000in}{0.660000in}}{\pgfqpoint{4.620000in}{4.620000in}}%
\pgfusepath{clip}%
\pgfsetbuttcap%
\pgfsetroundjoin%
\definecolor{currentfill}{rgb}{0.176471,0.192157,0.258824}%
\pgfsetfillcolor{currentfill}%
\pgfsetfillopacity{0.050000}%
\pgfsetlinewidth{0.000000pt}%
\definecolor{currentstroke}{rgb}{0.176471,0.192157,0.258824}%
\pgfsetstrokecolor{currentstroke}%
\pgfsetdash{}{0pt}%
\pgfpathmoveto{\pgfqpoint{1.707295in}{3.519322in}}%
\pgfpathlineto{\pgfqpoint{1.707295in}{3.624911in}}%
\pgfpathlineto{\pgfqpoint{1.820042in}{3.571979in}}%
\pgfpathlineto{\pgfqpoint{1.820042in}{3.466390in}}%
\pgfpathlineto{\pgfqpoint{1.707295in}{3.519322in}}%
\pgfpathclose%
\pgfusepath{fill}%
\end{pgfscope}%
\begin{pgfscope}%
\pgfpathrectangle{\pgfqpoint{0.765000in}{0.660000in}}{\pgfqpoint{4.620000in}{4.620000in}}%
\pgfusepath{clip}%
\pgfsetbuttcap%
\pgfsetroundjoin%
\definecolor{currentfill}{rgb}{0.176471,0.192157,0.258824}%
\pgfsetfillcolor{currentfill}%
\pgfsetfillopacity{0.050000}%
\pgfsetlinewidth{0.000000pt}%
\definecolor{currentstroke}{rgb}{0.176471,0.192157,0.258824}%
\pgfsetstrokecolor{currentstroke}%
\pgfsetdash{}{0pt}%
\pgfpathmoveto{\pgfqpoint{1.932790in}{3.519322in}}%
\pgfpathlineto{\pgfqpoint{1.932790in}{3.624911in}}%
\pgfpathlineto{\pgfqpoint{1.820042in}{3.571979in}}%
\pgfpathlineto{\pgfqpoint{1.820042in}{3.466390in}}%
\pgfpathlineto{\pgfqpoint{1.932790in}{3.519322in}}%
\pgfpathclose%
\pgfusepath{fill}%
\end{pgfscope}%
\begin{pgfscope}%
\pgfpathrectangle{\pgfqpoint{0.765000in}{0.660000in}}{\pgfqpoint{4.620000in}{4.620000in}}%
\pgfusepath{clip}%
\pgfsetbuttcap%
\pgfsetroundjoin%
\definecolor{currentfill}{rgb}{1.000000,0.576471,0.309804}%
\pgfsetfillcolor{currentfill}%
\pgfsetfillopacity{0.050000}%
\pgfsetlinewidth{0.000000pt}%
\definecolor{currentstroke}{rgb}{1.000000,0.576471,0.309804}%
\pgfsetstrokecolor{currentstroke}%
\pgfsetdash{}{0pt}%
\pgfpathmoveto{\pgfqpoint{4.527043in}{3.592108in}}%
\pgfpathlineto{\pgfqpoint{4.527043in}{3.697697in}}%
\pgfpathlineto{\pgfqpoint{4.639791in}{3.644765in}}%
\pgfpathlineto{\pgfqpoint{4.639791in}{3.539176in}}%
\pgfpathlineto{\pgfqpoint{4.527043in}{3.592108in}}%
\pgfpathclose%
\pgfusepath{fill}%
\end{pgfscope}%
\begin{pgfscope}%
\pgfpathrectangle{\pgfqpoint{0.765000in}{0.660000in}}{\pgfqpoint{4.620000in}{4.620000in}}%
\pgfusepath{clip}%
\pgfsetbuttcap%
\pgfsetroundjoin%
\definecolor{currentfill}{rgb}{1.000000,0.576471,0.309804}%
\pgfsetfillcolor{currentfill}%
\pgfsetfillopacity{0.050000}%
\pgfsetlinewidth{0.000000pt}%
\definecolor{currentstroke}{rgb}{1.000000,0.576471,0.309804}%
\pgfsetstrokecolor{currentstroke}%
\pgfsetdash{}{0pt}%
\pgfpathmoveto{\pgfqpoint{4.527043in}{3.592108in}}%
\pgfpathlineto{\pgfqpoint{4.527043in}{3.697697in}}%
\pgfpathlineto{\pgfqpoint{4.414296in}{3.644765in}}%
\pgfpathlineto{\pgfqpoint{4.414296in}{3.539176in}}%
\pgfpathlineto{\pgfqpoint{4.527043in}{3.592108in}}%
\pgfpathclose%
\pgfusepath{fill}%
\end{pgfscope}%
\begin{pgfscope}%
\pgfpathrectangle{\pgfqpoint{0.765000in}{0.660000in}}{\pgfqpoint{4.620000in}{4.620000in}}%
\pgfusepath{clip}%
\pgfsetbuttcap%
\pgfsetroundjoin%
\definecolor{currentfill}{rgb}{1.000000,0.576471,0.309804}%
\pgfsetfillcolor{currentfill}%
\pgfsetfillopacity{0.050000}%
\pgfsetlinewidth{0.000000pt}%
\definecolor{currentstroke}{rgb}{1.000000,0.576471,0.309804}%
\pgfsetstrokecolor{currentstroke}%
\pgfsetdash{}{0pt}%
\pgfpathmoveto{\pgfqpoint{4.527043in}{3.592108in}}%
\pgfpathlineto{\pgfqpoint{4.639791in}{3.539176in}}%
\pgfpathlineto{\pgfqpoint{4.527043in}{3.486244in}}%
\pgfpathlineto{\pgfqpoint{4.414296in}{3.539176in}}%
\pgfpathlineto{\pgfqpoint{4.527043in}{3.592108in}}%
\pgfpathclose%
\pgfusepath{fill}%
\end{pgfscope}%
\begin{pgfscope}%
\pgfpathrectangle{\pgfqpoint{0.765000in}{0.660000in}}{\pgfqpoint{4.620000in}{4.620000in}}%
\pgfusepath{clip}%
\pgfsetbuttcap%
\pgfsetroundjoin%
\definecolor{currentfill}{rgb}{1.000000,0.576471,0.309804}%
\pgfsetfillcolor{currentfill}%
\pgfsetfillopacity{0.050000}%
\pgfsetlinewidth{0.000000pt}%
\definecolor{currentstroke}{rgb}{1.000000,0.576471,0.309804}%
\pgfsetstrokecolor{currentstroke}%
\pgfsetdash{}{0pt}%
\pgfpathmoveto{\pgfqpoint{4.527043in}{3.697697in}}%
\pgfpathlineto{\pgfqpoint{4.639791in}{3.644765in}}%
\pgfpathlineto{\pgfqpoint{4.527043in}{3.591833in}}%
\pgfpathlineto{\pgfqpoint{4.414296in}{3.644765in}}%
\pgfpathlineto{\pgfqpoint{4.527043in}{3.697697in}}%
\pgfpathclose%
\pgfusepath{fill}%
\end{pgfscope}%
\begin{pgfscope}%
\pgfpathrectangle{\pgfqpoint{0.765000in}{0.660000in}}{\pgfqpoint{4.620000in}{4.620000in}}%
\pgfusepath{clip}%
\pgfsetbuttcap%
\pgfsetroundjoin%
\definecolor{currentfill}{rgb}{1.000000,0.576471,0.309804}%
\pgfsetfillcolor{currentfill}%
\pgfsetfillopacity{0.050000}%
\pgfsetlinewidth{0.000000pt}%
\definecolor{currentstroke}{rgb}{1.000000,0.576471,0.309804}%
\pgfsetstrokecolor{currentstroke}%
\pgfsetdash{}{0pt}%
\pgfpathmoveto{\pgfqpoint{4.414296in}{3.539176in}}%
\pgfpathlineto{\pgfqpoint{4.414296in}{3.644765in}}%
\pgfpathlineto{\pgfqpoint{4.527043in}{3.591833in}}%
\pgfpathlineto{\pgfqpoint{4.527043in}{3.486244in}}%
\pgfpathlineto{\pgfqpoint{4.414296in}{3.539176in}}%
\pgfpathclose%
\pgfusepath{fill}%
\end{pgfscope}%
\begin{pgfscope}%
\pgfpathrectangle{\pgfqpoint{0.765000in}{0.660000in}}{\pgfqpoint{4.620000in}{4.620000in}}%
\pgfusepath{clip}%
\pgfsetbuttcap%
\pgfsetroundjoin%
\definecolor{currentfill}{rgb}{1.000000,0.576471,0.309804}%
\pgfsetfillcolor{currentfill}%
\pgfsetfillopacity{0.050000}%
\pgfsetlinewidth{0.000000pt}%
\definecolor{currentstroke}{rgb}{1.000000,0.576471,0.309804}%
\pgfsetstrokecolor{currentstroke}%
\pgfsetdash{}{0pt}%
\pgfpathmoveto{\pgfqpoint{4.639791in}{3.539176in}}%
\pgfpathlineto{\pgfqpoint{4.639791in}{3.644765in}}%
\pgfpathlineto{\pgfqpoint{4.527043in}{3.591833in}}%
\pgfpathlineto{\pgfqpoint{4.527043in}{3.486244in}}%
\pgfpathlineto{\pgfqpoint{4.639791in}{3.539176in}}%
\pgfpathclose%
\pgfusepath{fill}%
\end{pgfscope}%
\begin{pgfscope}%
\pgfpathrectangle{\pgfqpoint{0.765000in}{0.660000in}}{\pgfqpoint{4.620000in}{4.620000in}}%
\pgfusepath{clip}%
\pgfsetbuttcap%
\pgfsetroundjoin%
\definecolor{currentfill}{rgb}{1.000000,0.576471,0.309804}%
\pgfsetfillcolor{currentfill}%
\pgfsetfillopacity{0.050000}%
\pgfsetlinewidth{0.000000pt}%
\definecolor{currentstroke}{rgb}{1.000000,0.576471,0.309804}%
\pgfsetstrokecolor{currentstroke}%
\pgfsetdash{}{0pt}%
\pgfpathmoveto{\pgfqpoint{4.663634in}{3.772889in}}%
\pgfpathlineto{\pgfqpoint{4.663634in}{3.878478in}}%
\pgfpathlineto{\pgfqpoint{4.776381in}{3.825546in}}%
\pgfpathlineto{\pgfqpoint{4.776381in}{3.719957in}}%
\pgfpathlineto{\pgfqpoint{4.663634in}{3.772889in}}%
\pgfpathclose%
\pgfusepath{fill}%
\end{pgfscope}%
\begin{pgfscope}%
\pgfpathrectangle{\pgfqpoint{0.765000in}{0.660000in}}{\pgfqpoint{4.620000in}{4.620000in}}%
\pgfusepath{clip}%
\pgfsetbuttcap%
\pgfsetroundjoin%
\definecolor{currentfill}{rgb}{1.000000,0.576471,0.309804}%
\pgfsetfillcolor{currentfill}%
\pgfsetfillopacity{0.050000}%
\pgfsetlinewidth{0.000000pt}%
\definecolor{currentstroke}{rgb}{1.000000,0.576471,0.309804}%
\pgfsetstrokecolor{currentstroke}%
\pgfsetdash{}{0pt}%
\pgfpathmoveto{\pgfqpoint{4.663634in}{3.772889in}}%
\pgfpathlineto{\pgfqpoint{4.663634in}{3.878478in}}%
\pgfpathlineto{\pgfqpoint{4.550886in}{3.825546in}}%
\pgfpathlineto{\pgfqpoint{4.550886in}{3.719957in}}%
\pgfpathlineto{\pgfqpoint{4.663634in}{3.772889in}}%
\pgfpathclose%
\pgfusepath{fill}%
\end{pgfscope}%
\begin{pgfscope}%
\pgfpathrectangle{\pgfqpoint{0.765000in}{0.660000in}}{\pgfqpoint{4.620000in}{4.620000in}}%
\pgfusepath{clip}%
\pgfsetbuttcap%
\pgfsetroundjoin%
\definecolor{currentfill}{rgb}{1.000000,0.576471,0.309804}%
\pgfsetfillcolor{currentfill}%
\pgfsetfillopacity{0.050000}%
\pgfsetlinewidth{0.000000pt}%
\definecolor{currentstroke}{rgb}{1.000000,0.576471,0.309804}%
\pgfsetstrokecolor{currentstroke}%
\pgfsetdash{}{0pt}%
\pgfpathmoveto{\pgfqpoint{4.663634in}{3.772889in}}%
\pgfpathlineto{\pgfqpoint{4.776381in}{3.719957in}}%
\pgfpathlineto{\pgfqpoint{4.663634in}{3.667025in}}%
\pgfpathlineto{\pgfqpoint{4.550886in}{3.719957in}}%
\pgfpathlineto{\pgfqpoint{4.663634in}{3.772889in}}%
\pgfpathclose%
\pgfusepath{fill}%
\end{pgfscope}%
\begin{pgfscope}%
\pgfpathrectangle{\pgfqpoint{0.765000in}{0.660000in}}{\pgfqpoint{4.620000in}{4.620000in}}%
\pgfusepath{clip}%
\pgfsetbuttcap%
\pgfsetroundjoin%
\definecolor{currentfill}{rgb}{1.000000,0.576471,0.309804}%
\pgfsetfillcolor{currentfill}%
\pgfsetfillopacity{0.050000}%
\pgfsetlinewidth{0.000000pt}%
\definecolor{currentstroke}{rgb}{1.000000,0.576471,0.309804}%
\pgfsetstrokecolor{currentstroke}%
\pgfsetdash{}{0pt}%
\pgfpathmoveto{\pgfqpoint{4.663634in}{3.878478in}}%
\pgfpathlineto{\pgfqpoint{4.776381in}{3.825546in}}%
\pgfpathlineto{\pgfqpoint{4.663634in}{3.772614in}}%
\pgfpathlineto{\pgfqpoint{4.550886in}{3.825546in}}%
\pgfpathlineto{\pgfqpoint{4.663634in}{3.878478in}}%
\pgfpathclose%
\pgfusepath{fill}%
\end{pgfscope}%
\begin{pgfscope}%
\pgfpathrectangle{\pgfqpoint{0.765000in}{0.660000in}}{\pgfqpoint{4.620000in}{4.620000in}}%
\pgfusepath{clip}%
\pgfsetbuttcap%
\pgfsetroundjoin%
\definecolor{currentfill}{rgb}{1.000000,0.576471,0.309804}%
\pgfsetfillcolor{currentfill}%
\pgfsetfillopacity{0.050000}%
\pgfsetlinewidth{0.000000pt}%
\definecolor{currentstroke}{rgb}{1.000000,0.576471,0.309804}%
\pgfsetstrokecolor{currentstroke}%
\pgfsetdash{}{0pt}%
\pgfpathmoveto{\pgfqpoint{4.550886in}{3.719957in}}%
\pgfpathlineto{\pgfqpoint{4.550886in}{3.825546in}}%
\pgfpathlineto{\pgfqpoint{4.663634in}{3.772614in}}%
\pgfpathlineto{\pgfqpoint{4.663634in}{3.667025in}}%
\pgfpathlineto{\pgfqpoint{4.550886in}{3.719957in}}%
\pgfpathclose%
\pgfusepath{fill}%
\end{pgfscope}%
\begin{pgfscope}%
\pgfpathrectangle{\pgfqpoint{0.765000in}{0.660000in}}{\pgfqpoint{4.620000in}{4.620000in}}%
\pgfusepath{clip}%
\pgfsetbuttcap%
\pgfsetroundjoin%
\definecolor{currentfill}{rgb}{1.000000,0.576471,0.309804}%
\pgfsetfillcolor{currentfill}%
\pgfsetfillopacity{0.050000}%
\pgfsetlinewidth{0.000000pt}%
\definecolor{currentstroke}{rgb}{1.000000,0.576471,0.309804}%
\pgfsetstrokecolor{currentstroke}%
\pgfsetdash{}{0pt}%
\pgfpathmoveto{\pgfqpoint{4.776381in}{3.719957in}}%
\pgfpathlineto{\pgfqpoint{4.776381in}{3.825546in}}%
\pgfpathlineto{\pgfqpoint{4.663634in}{3.772614in}}%
\pgfpathlineto{\pgfqpoint{4.663634in}{3.667025in}}%
\pgfpathlineto{\pgfqpoint{4.776381in}{3.719957in}}%
\pgfpathclose%
\pgfusepath{fill}%
\end{pgfscope}%
\begin{pgfscope}%
\pgfpathrectangle{\pgfqpoint{0.765000in}{0.660000in}}{\pgfqpoint{4.620000in}{4.620000in}}%
\pgfusepath{clip}%
\pgfsetbuttcap%
\pgfsetroundjoin%
\definecolor{currentfill}{rgb}{0.176471,0.192157,0.258824}%
\pgfsetfillcolor{currentfill}%
\pgfsetfillopacity{0.050000}%
\pgfsetlinewidth{0.000000pt}%
\definecolor{currentstroke}{rgb}{0.176471,0.192157,0.258824}%
\pgfsetstrokecolor{currentstroke}%
\pgfsetdash{}{0pt}%
\pgfpathmoveto{\pgfqpoint{1.882907in}{3.849716in}}%
\pgfpathlineto{\pgfqpoint{1.882907in}{3.955304in}}%
\pgfpathlineto{\pgfqpoint{1.995655in}{3.902373in}}%
\pgfpathlineto{\pgfqpoint{1.995655in}{3.796784in}}%
\pgfpathlineto{\pgfqpoint{1.882907in}{3.849716in}}%
\pgfpathclose%
\pgfusepath{fill}%
\end{pgfscope}%
\begin{pgfscope}%
\pgfpathrectangle{\pgfqpoint{0.765000in}{0.660000in}}{\pgfqpoint{4.620000in}{4.620000in}}%
\pgfusepath{clip}%
\pgfsetbuttcap%
\pgfsetroundjoin%
\definecolor{currentfill}{rgb}{0.176471,0.192157,0.258824}%
\pgfsetfillcolor{currentfill}%
\pgfsetfillopacity{0.050000}%
\pgfsetlinewidth{0.000000pt}%
\definecolor{currentstroke}{rgb}{0.176471,0.192157,0.258824}%
\pgfsetstrokecolor{currentstroke}%
\pgfsetdash{}{0pt}%
\pgfpathmoveto{\pgfqpoint{1.882907in}{3.849716in}}%
\pgfpathlineto{\pgfqpoint{1.882907in}{3.955304in}}%
\pgfpathlineto{\pgfqpoint{1.770160in}{3.902373in}}%
\pgfpathlineto{\pgfqpoint{1.770160in}{3.796784in}}%
\pgfpathlineto{\pgfqpoint{1.882907in}{3.849716in}}%
\pgfpathclose%
\pgfusepath{fill}%
\end{pgfscope}%
\begin{pgfscope}%
\pgfpathrectangle{\pgfqpoint{0.765000in}{0.660000in}}{\pgfqpoint{4.620000in}{4.620000in}}%
\pgfusepath{clip}%
\pgfsetbuttcap%
\pgfsetroundjoin%
\definecolor{currentfill}{rgb}{0.176471,0.192157,0.258824}%
\pgfsetfillcolor{currentfill}%
\pgfsetfillopacity{0.050000}%
\pgfsetlinewidth{0.000000pt}%
\definecolor{currentstroke}{rgb}{0.176471,0.192157,0.258824}%
\pgfsetstrokecolor{currentstroke}%
\pgfsetdash{}{0pt}%
\pgfpathmoveto{\pgfqpoint{1.882907in}{3.849716in}}%
\pgfpathlineto{\pgfqpoint{1.995655in}{3.796784in}}%
\pgfpathlineto{\pgfqpoint{1.882907in}{3.743852in}}%
\pgfpathlineto{\pgfqpoint{1.770160in}{3.796784in}}%
\pgfpathlineto{\pgfqpoint{1.882907in}{3.849716in}}%
\pgfpathclose%
\pgfusepath{fill}%
\end{pgfscope}%
\begin{pgfscope}%
\pgfpathrectangle{\pgfqpoint{0.765000in}{0.660000in}}{\pgfqpoint{4.620000in}{4.620000in}}%
\pgfusepath{clip}%
\pgfsetbuttcap%
\pgfsetroundjoin%
\definecolor{currentfill}{rgb}{0.176471,0.192157,0.258824}%
\pgfsetfillcolor{currentfill}%
\pgfsetfillopacity{0.050000}%
\pgfsetlinewidth{0.000000pt}%
\definecolor{currentstroke}{rgb}{0.176471,0.192157,0.258824}%
\pgfsetstrokecolor{currentstroke}%
\pgfsetdash{}{0pt}%
\pgfpathmoveto{\pgfqpoint{1.882907in}{3.955304in}}%
\pgfpathlineto{\pgfqpoint{1.995655in}{3.902373in}}%
\pgfpathlineto{\pgfqpoint{1.882907in}{3.849441in}}%
\pgfpathlineto{\pgfqpoint{1.770160in}{3.902373in}}%
\pgfpathlineto{\pgfqpoint{1.882907in}{3.955304in}}%
\pgfpathclose%
\pgfusepath{fill}%
\end{pgfscope}%
\begin{pgfscope}%
\pgfpathrectangle{\pgfqpoint{0.765000in}{0.660000in}}{\pgfqpoint{4.620000in}{4.620000in}}%
\pgfusepath{clip}%
\pgfsetbuttcap%
\pgfsetroundjoin%
\definecolor{currentfill}{rgb}{0.176471,0.192157,0.258824}%
\pgfsetfillcolor{currentfill}%
\pgfsetfillopacity{0.050000}%
\pgfsetlinewidth{0.000000pt}%
\definecolor{currentstroke}{rgb}{0.176471,0.192157,0.258824}%
\pgfsetstrokecolor{currentstroke}%
\pgfsetdash{}{0pt}%
\pgfpathmoveto{\pgfqpoint{1.770160in}{3.796784in}}%
\pgfpathlineto{\pgfqpoint{1.770160in}{3.902373in}}%
\pgfpathlineto{\pgfqpoint{1.882907in}{3.849441in}}%
\pgfpathlineto{\pgfqpoint{1.882907in}{3.743852in}}%
\pgfpathlineto{\pgfqpoint{1.770160in}{3.796784in}}%
\pgfpathclose%
\pgfusepath{fill}%
\end{pgfscope}%
\begin{pgfscope}%
\pgfpathrectangle{\pgfqpoint{0.765000in}{0.660000in}}{\pgfqpoint{4.620000in}{4.620000in}}%
\pgfusepath{clip}%
\pgfsetbuttcap%
\pgfsetroundjoin%
\definecolor{currentfill}{rgb}{0.176471,0.192157,0.258824}%
\pgfsetfillcolor{currentfill}%
\pgfsetfillopacity{0.050000}%
\pgfsetlinewidth{0.000000pt}%
\definecolor{currentstroke}{rgb}{0.176471,0.192157,0.258824}%
\pgfsetstrokecolor{currentstroke}%
\pgfsetdash{}{0pt}%
\pgfpathmoveto{\pgfqpoint{1.995655in}{3.796784in}}%
\pgfpathlineto{\pgfqpoint{1.995655in}{3.902373in}}%
\pgfpathlineto{\pgfqpoint{1.882907in}{3.849441in}}%
\pgfpathlineto{\pgfqpoint{1.882907in}{3.743852in}}%
\pgfpathlineto{\pgfqpoint{1.995655in}{3.796784in}}%
\pgfpathclose%
\pgfusepath{fill}%
\end{pgfscope}%
\begin{pgfscope}%
\pgfpathrectangle{\pgfqpoint{0.765000in}{0.660000in}}{\pgfqpoint{4.620000in}{4.620000in}}%
\pgfusepath{clip}%
\pgfsetbuttcap%
\pgfsetroundjoin%
\definecolor{currentfill}{rgb}{0.019608,0.556863,0.850980}%
\pgfsetfillcolor{currentfill}%
\pgfsetfillopacity{0.050000}%
\pgfsetlinewidth{0.000000pt}%
\definecolor{currentstroke}{rgb}{0.019608,0.556863,0.850980}%
\pgfsetstrokecolor{currentstroke}%
\pgfsetdash{}{0pt}%
\pgfpathmoveto{\pgfqpoint{3.159366in}{1.424802in}}%
\pgfpathlineto{\pgfqpoint{3.159366in}{1.530390in}}%
\pgfpathlineto{\pgfqpoint{3.272113in}{1.477459in}}%
\pgfpathlineto{\pgfqpoint{3.272113in}{1.371870in}}%
\pgfpathlineto{\pgfqpoint{3.159366in}{1.424802in}}%
\pgfpathclose%
\pgfusepath{fill}%
\end{pgfscope}%
\begin{pgfscope}%
\pgfpathrectangle{\pgfqpoint{0.765000in}{0.660000in}}{\pgfqpoint{4.620000in}{4.620000in}}%
\pgfusepath{clip}%
\pgfsetbuttcap%
\pgfsetroundjoin%
\definecolor{currentfill}{rgb}{0.019608,0.556863,0.850980}%
\pgfsetfillcolor{currentfill}%
\pgfsetfillopacity{0.050000}%
\pgfsetlinewidth{0.000000pt}%
\definecolor{currentstroke}{rgb}{0.019608,0.556863,0.850980}%
\pgfsetstrokecolor{currentstroke}%
\pgfsetdash{}{0pt}%
\pgfpathmoveto{\pgfqpoint{3.159366in}{1.424802in}}%
\pgfpathlineto{\pgfqpoint{3.159366in}{1.530390in}}%
\pgfpathlineto{\pgfqpoint{3.046618in}{1.477459in}}%
\pgfpathlineto{\pgfqpoint{3.046618in}{1.371870in}}%
\pgfpathlineto{\pgfqpoint{3.159366in}{1.424802in}}%
\pgfpathclose%
\pgfusepath{fill}%
\end{pgfscope}%
\begin{pgfscope}%
\pgfpathrectangle{\pgfqpoint{0.765000in}{0.660000in}}{\pgfqpoint{4.620000in}{4.620000in}}%
\pgfusepath{clip}%
\pgfsetbuttcap%
\pgfsetroundjoin%
\definecolor{currentfill}{rgb}{0.019608,0.556863,0.850980}%
\pgfsetfillcolor{currentfill}%
\pgfsetfillopacity{0.050000}%
\pgfsetlinewidth{0.000000pt}%
\definecolor{currentstroke}{rgb}{0.019608,0.556863,0.850980}%
\pgfsetstrokecolor{currentstroke}%
\pgfsetdash{}{0pt}%
\pgfpathmoveto{\pgfqpoint{3.159366in}{1.424802in}}%
\pgfpathlineto{\pgfqpoint{3.272113in}{1.371870in}}%
\pgfpathlineto{\pgfqpoint{3.159366in}{1.318938in}}%
\pgfpathlineto{\pgfqpoint{3.046618in}{1.371870in}}%
\pgfpathlineto{\pgfqpoint{3.159366in}{1.424802in}}%
\pgfpathclose%
\pgfusepath{fill}%
\end{pgfscope}%
\begin{pgfscope}%
\pgfpathrectangle{\pgfqpoint{0.765000in}{0.660000in}}{\pgfqpoint{4.620000in}{4.620000in}}%
\pgfusepath{clip}%
\pgfsetbuttcap%
\pgfsetroundjoin%
\definecolor{currentfill}{rgb}{0.019608,0.556863,0.850980}%
\pgfsetfillcolor{currentfill}%
\pgfsetfillopacity{0.050000}%
\pgfsetlinewidth{0.000000pt}%
\definecolor{currentstroke}{rgb}{0.019608,0.556863,0.850980}%
\pgfsetstrokecolor{currentstroke}%
\pgfsetdash{}{0pt}%
\pgfpathmoveto{\pgfqpoint{3.159366in}{1.530390in}}%
\pgfpathlineto{\pgfqpoint{3.272113in}{1.477459in}}%
\pgfpathlineto{\pgfqpoint{3.159366in}{1.424527in}}%
\pgfpathlineto{\pgfqpoint{3.046618in}{1.477459in}}%
\pgfpathlineto{\pgfqpoint{3.159366in}{1.530390in}}%
\pgfpathclose%
\pgfusepath{fill}%
\end{pgfscope}%
\begin{pgfscope}%
\pgfpathrectangle{\pgfqpoint{0.765000in}{0.660000in}}{\pgfqpoint{4.620000in}{4.620000in}}%
\pgfusepath{clip}%
\pgfsetbuttcap%
\pgfsetroundjoin%
\definecolor{currentfill}{rgb}{0.019608,0.556863,0.850980}%
\pgfsetfillcolor{currentfill}%
\pgfsetfillopacity{0.050000}%
\pgfsetlinewidth{0.000000pt}%
\definecolor{currentstroke}{rgb}{0.019608,0.556863,0.850980}%
\pgfsetstrokecolor{currentstroke}%
\pgfsetdash{}{0pt}%
\pgfpathmoveto{\pgfqpoint{3.046618in}{1.371870in}}%
\pgfpathlineto{\pgfqpoint{3.046618in}{1.477459in}}%
\pgfpathlineto{\pgfqpoint{3.159366in}{1.424527in}}%
\pgfpathlineto{\pgfqpoint{3.159366in}{1.318938in}}%
\pgfpathlineto{\pgfqpoint{3.046618in}{1.371870in}}%
\pgfpathclose%
\pgfusepath{fill}%
\end{pgfscope}%
\begin{pgfscope}%
\pgfpathrectangle{\pgfqpoint{0.765000in}{0.660000in}}{\pgfqpoint{4.620000in}{4.620000in}}%
\pgfusepath{clip}%
\pgfsetbuttcap%
\pgfsetroundjoin%
\definecolor{currentfill}{rgb}{0.019608,0.556863,0.850980}%
\pgfsetfillcolor{currentfill}%
\pgfsetfillopacity{0.050000}%
\pgfsetlinewidth{0.000000pt}%
\definecolor{currentstroke}{rgb}{0.019608,0.556863,0.850980}%
\pgfsetstrokecolor{currentstroke}%
\pgfsetdash{}{0pt}%
\pgfpathmoveto{\pgfqpoint{3.272113in}{1.371870in}}%
\pgfpathlineto{\pgfqpoint{3.272113in}{1.477459in}}%
\pgfpathlineto{\pgfqpoint{3.159366in}{1.424527in}}%
\pgfpathlineto{\pgfqpoint{3.159366in}{1.318938in}}%
\pgfpathlineto{\pgfqpoint{3.272113in}{1.371870in}}%
\pgfpathclose%
\pgfusepath{fill}%
\end{pgfscope}%
\begin{pgfscope}%
\pgfpathrectangle{\pgfqpoint{0.765000in}{0.660000in}}{\pgfqpoint{4.620000in}{4.620000in}}%
\pgfusepath{clip}%
\pgfsetbuttcap%
\pgfsetroundjoin%
\definecolor{currentfill}{rgb}{0.019608,0.556863,0.850980}%
\pgfsetfillcolor{currentfill}%
\pgfsetfillopacity{0.050000}%
\pgfsetlinewidth{0.000000pt}%
\definecolor{currentstroke}{rgb}{0.019608,0.556863,0.850980}%
\pgfsetstrokecolor{currentstroke}%
\pgfsetdash{}{0pt}%
\pgfpathmoveto{\pgfqpoint{3.370235in}{1.591363in}}%
\pgfpathlineto{\pgfqpoint{3.370235in}{1.696952in}}%
\pgfpathlineto{\pgfqpoint{3.482982in}{1.644020in}}%
\pgfpathlineto{\pgfqpoint{3.482982in}{1.538431in}}%
\pgfpathlineto{\pgfqpoint{3.370235in}{1.591363in}}%
\pgfpathclose%
\pgfusepath{fill}%
\end{pgfscope}%
\begin{pgfscope}%
\pgfpathrectangle{\pgfqpoint{0.765000in}{0.660000in}}{\pgfqpoint{4.620000in}{4.620000in}}%
\pgfusepath{clip}%
\pgfsetbuttcap%
\pgfsetroundjoin%
\definecolor{currentfill}{rgb}{0.019608,0.556863,0.850980}%
\pgfsetfillcolor{currentfill}%
\pgfsetfillopacity{0.050000}%
\pgfsetlinewidth{0.000000pt}%
\definecolor{currentstroke}{rgb}{0.019608,0.556863,0.850980}%
\pgfsetstrokecolor{currentstroke}%
\pgfsetdash{}{0pt}%
\pgfpathmoveto{\pgfqpoint{3.370235in}{1.591363in}}%
\pgfpathlineto{\pgfqpoint{3.370235in}{1.696952in}}%
\pgfpathlineto{\pgfqpoint{3.257487in}{1.644020in}}%
\pgfpathlineto{\pgfqpoint{3.257487in}{1.538431in}}%
\pgfpathlineto{\pgfqpoint{3.370235in}{1.591363in}}%
\pgfpathclose%
\pgfusepath{fill}%
\end{pgfscope}%
\begin{pgfscope}%
\pgfpathrectangle{\pgfqpoint{0.765000in}{0.660000in}}{\pgfqpoint{4.620000in}{4.620000in}}%
\pgfusepath{clip}%
\pgfsetbuttcap%
\pgfsetroundjoin%
\definecolor{currentfill}{rgb}{0.019608,0.556863,0.850980}%
\pgfsetfillcolor{currentfill}%
\pgfsetfillopacity{0.050000}%
\pgfsetlinewidth{0.000000pt}%
\definecolor{currentstroke}{rgb}{0.019608,0.556863,0.850980}%
\pgfsetstrokecolor{currentstroke}%
\pgfsetdash{}{0pt}%
\pgfpathmoveto{\pgfqpoint{3.370235in}{1.591363in}}%
\pgfpathlineto{\pgfqpoint{3.482982in}{1.538431in}}%
\pgfpathlineto{\pgfqpoint{3.370235in}{1.485499in}}%
\pgfpathlineto{\pgfqpoint{3.257487in}{1.538431in}}%
\pgfpathlineto{\pgfqpoint{3.370235in}{1.591363in}}%
\pgfpathclose%
\pgfusepath{fill}%
\end{pgfscope}%
\begin{pgfscope}%
\pgfpathrectangle{\pgfqpoint{0.765000in}{0.660000in}}{\pgfqpoint{4.620000in}{4.620000in}}%
\pgfusepath{clip}%
\pgfsetbuttcap%
\pgfsetroundjoin%
\definecolor{currentfill}{rgb}{0.019608,0.556863,0.850980}%
\pgfsetfillcolor{currentfill}%
\pgfsetfillopacity{0.050000}%
\pgfsetlinewidth{0.000000pt}%
\definecolor{currentstroke}{rgb}{0.019608,0.556863,0.850980}%
\pgfsetstrokecolor{currentstroke}%
\pgfsetdash{}{0pt}%
\pgfpathmoveto{\pgfqpoint{3.370235in}{1.696952in}}%
\pgfpathlineto{\pgfqpoint{3.482982in}{1.644020in}}%
\pgfpathlineto{\pgfqpoint{3.370235in}{1.591088in}}%
\pgfpathlineto{\pgfqpoint{3.257487in}{1.644020in}}%
\pgfpathlineto{\pgfqpoint{3.370235in}{1.696952in}}%
\pgfpathclose%
\pgfusepath{fill}%
\end{pgfscope}%
\begin{pgfscope}%
\pgfpathrectangle{\pgfqpoint{0.765000in}{0.660000in}}{\pgfqpoint{4.620000in}{4.620000in}}%
\pgfusepath{clip}%
\pgfsetbuttcap%
\pgfsetroundjoin%
\definecolor{currentfill}{rgb}{0.019608,0.556863,0.850980}%
\pgfsetfillcolor{currentfill}%
\pgfsetfillopacity{0.050000}%
\pgfsetlinewidth{0.000000pt}%
\definecolor{currentstroke}{rgb}{0.019608,0.556863,0.850980}%
\pgfsetstrokecolor{currentstroke}%
\pgfsetdash{}{0pt}%
\pgfpathmoveto{\pgfqpoint{3.257487in}{1.538431in}}%
\pgfpathlineto{\pgfqpoint{3.257487in}{1.644020in}}%
\pgfpathlineto{\pgfqpoint{3.370235in}{1.591088in}}%
\pgfpathlineto{\pgfqpoint{3.370235in}{1.485499in}}%
\pgfpathlineto{\pgfqpoint{3.257487in}{1.538431in}}%
\pgfpathclose%
\pgfusepath{fill}%
\end{pgfscope}%
\begin{pgfscope}%
\pgfpathrectangle{\pgfqpoint{0.765000in}{0.660000in}}{\pgfqpoint{4.620000in}{4.620000in}}%
\pgfusepath{clip}%
\pgfsetbuttcap%
\pgfsetroundjoin%
\definecolor{currentfill}{rgb}{0.019608,0.556863,0.850980}%
\pgfsetfillcolor{currentfill}%
\pgfsetfillopacity{0.050000}%
\pgfsetlinewidth{0.000000pt}%
\definecolor{currentstroke}{rgb}{0.019608,0.556863,0.850980}%
\pgfsetstrokecolor{currentstroke}%
\pgfsetdash{}{0pt}%
\pgfpathmoveto{\pgfqpoint{3.482982in}{1.538431in}}%
\pgfpathlineto{\pgfqpoint{3.482982in}{1.644020in}}%
\pgfpathlineto{\pgfqpoint{3.370235in}{1.591088in}}%
\pgfpathlineto{\pgfqpoint{3.370235in}{1.485499in}}%
\pgfpathlineto{\pgfqpoint{3.482982in}{1.538431in}}%
\pgfpathclose%
\pgfusepath{fill}%
\end{pgfscope}%
\begin{pgfscope}%
\pgfpathrectangle{\pgfqpoint{0.765000in}{0.660000in}}{\pgfqpoint{4.620000in}{4.620000in}}%
\pgfusepath{clip}%
\pgfsetbuttcap%
\pgfsetroundjoin%
\definecolor{currentfill}{rgb}{0.019608,0.556863,0.850980}%
\pgfsetfillcolor{currentfill}%
\pgfsetfillopacity{0.050000}%
\pgfsetlinewidth{0.000000pt}%
\definecolor{currentstroke}{rgb}{0.019608,0.556863,0.850980}%
\pgfsetstrokecolor{currentstroke}%
\pgfsetdash{}{0pt}%
\pgfpathmoveto{\pgfqpoint{3.390528in}{1.727252in}}%
\pgfpathlineto{\pgfqpoint{3.390528in}{1.832840in}}%
\pgfpathlineto{\pgfqpoint{3.503276in}{1.779909in}}%
\pgfpathlineto{\pgfqpoint{3.503276in}{1.674320in}}%
\pgfpathlineto{\pgfqpoint{3.390528in}{1.727252in}}%
\pgfpathclose%
\pgfusepath{fill}%
\end{pgfscope}%
\begin{pgfscope}%
\pgfpathrectangle{\pgfqpoint{0.765000in}{0.660000in}}{\pgfqpoint{4.620000in}{4.620000in}}%
\pgfusepath{clip}%
\pgfsetbuttcap%
\pgfsetroundjoin%
\definecolor{currentfill}{rgb}{0.019608,0.556863,0.850980}%
\pgfsetfillcolor{currentfill}%
\pgfsetfillopacity{0.050000}%
\pgfsetlinewidth{0.000000pt}%
\definecolor{currentstroke}{rgb}{0.019608,0.556863,0.850980}%
\pgfsetstrokecolor{currentstroke}%
\pgfsetdash{}{0pt}%
\pgfpathmoveto{\pgfqpoint{3.390528in}{1.727252in}}%
\pgfpathlineto{\pgfqpoint{3.390528in}{1.832840in}}%
\pgfpathlineto{\pgfqpoint{3.277781in}{1.779909in}}%
\pgfpathlineto{\pgfqpoint{3.277781in}{1.674320in}}%
\pgfpathlineto{\pgfqpoint{3.390528in}{1.727252in}}%
\pgfpathclose%
\pgfusepath{fill}%
\end{pgfscope}%
\begin{pgfscope}%
\pgfpathrectangle{\pgfqpoint{0.765000in}{0.660000in}}{\pgfqpoint{4.620000in}{4.620000in}}%
\pgfusepath{clip}%
\pgfsetbuttcap%
\pgfsetroundjoin%
\definecolor{currentfill}{rgb}{0.019608,0.556863,0.850980}%
\pgfsetfillcolor{currentfill}%
\pgfsetfillopacity{0.050000}%
\pgfsetlinewidth{0.000000pt}%
\definecolor{currentstroke}{rgb}{0.019608,0.556863,0.850980}%
\pgfsetstrokecolor{currentstroke}%
\pgfsetdash{}{0pt}%
\pgfpathmoveto{\pgfqpoint{3.390528in}{1.727252in}}%
\pgfpathlineto{\pgfqpoint{3.503276in}{1.674320in}}%
\pgfpathlineto{\pgfqpoint{3.390528in}{1.621388in}}%
\pgfpathlineto{\pgfqpoint{3.277781in}{1.674320in}}%
\pgfpathlineto{\pgfqpoint{3.390528in}{1.727252in}}%
\pgfpathclose%
\pgfusepath{fill}%
\end{pgfscope}%
\begin{pgfscope}%
\pgfpathrectangle{\pgfqpoint{0.765000in}{0.660000in}}{\pgfqpoint{4.620000in}{4.620000in}}%
\pgfusepath{clip}%
\pgfsetbuttcap%
\pgfsetroundjoin%
\definecolor{currentfill}{rgb}{0.019608,0.556863,0.850980}%
\pgfsetfillcolor{currentfill}%
\pgfsetfillopacity{0.050000}%
\pgfsetlinewidth{0.000000pt}%
\definecolor{currentstroke}{rgb}{0.019608,0.556863,0.850980}%
\pgfsetstrokecolor{currentstroke}%
\pgfsetdash{}{0pt}%
\pgfpathmoveto{\pgfqpoint{3.390528in}{1.832840in}}%
\pgfpathlineto{\pgfqpoint{3.503276in}{1.779909in}}%
\pgfpathlineto{\pgfqpoint{3.390528in}{1.726977in}}%
\pgfpathlineto{\pgfqpoint{3.277781in}{1.779909in}}%
\pgfpathlineto{\pgfqpoint{3.390528in}{1.832840in}}%
\pgfpathclose%
\pgfusepath{fill}%
\end{pgfscope}%
\begin{pgfscope}%
\pgfpathrectangle{\pgfqpoint{0.765000in}{0.660000in}}{\pgfqpoint{4.620000in}{4.620000in}}%
\pgfusepath{clip}%
\pgfsetbuttcap%
\pgfsetroundjoin%
\definecolor{currentfill}{rgb}{0.019608,0.556863,0.850980}%
\pgfsetfillcolor{currentfill}%
\pgfsetfillopacity{0.050000}%
\pgfsetlinewidth{0.000000pt}%
\definecolor{currentstroke}{rgb}{0.019608,0.556863,0.850980}%
\pgfsetstrokecolor{currentstroke}%
\pgfsetdash{}{0pt}%
\pgfpathmoveto{\pgfqpoint{3.277781in}{1.674320in}}%
\pgfpathlineto{\pgfqpoint{3.277781in}{1.779909in}}%
\pgfpathlineto{\pgfqpoint{3.390528in}{1.726977in}}%
\pgfpathlineto{\pgfqpoint{3.390528in}{1.621388in}}%
\pgfpathlineto{\pgfqpoint{3.277781in}{1.674320in}}%
\pgfpathclose%
\pgfusepath{fill}%
\end{pgfscope}%
\begin{pgfscope}%
\pgfpathrectangle{\pgfqpoint{0.765000in}{0.660000in}}{\pgfqpoint{4.620000in}{4.620000in}}%
\pgfusepath{clip}%
\pgfsetbuttcap%
\pgfsetroundjoin%
\definecolor{currentfill}{rgb}{0.019608,0.556863,0.850980}%
\pgfsetfillcolor{currentfill}%
\pgfsetfillopacity{0.050000}%
\pgfsetlinewidth{0.000000pt}%
\definecolor{currentstroke}{rgb}{0.019608,0.556863,0.850980}%
\pgfsetstrokecolor{currentstroke}%
\pgfsetdash{}{0pt}%
\pgfpathmoveto{\pgfqpoint{3.503276in}{1.674320in}}%
\pgfpathlineto{\pgfqpoint{3.503276in}{1.779909in}}%
\pgfpathlineto{\pgfqpoint{3.390528in}{1.726977in}}%
\pgfpathlineto{\pgfqpoint{3.390528in}{1.621388in}}%
\pgfpathlineto{\pgfqpoint{3.503276in}{1.674320in}}%
\pgfpathclose%
\pgfusepath{fill}%
\end{pgfscope}%
\begin{pgfscope}%
\pgfpathrectangle{\pgfqpoint{0.765000in}{0.660000in}}{\pgfqpoint{4.620000in}{4.620000in}}%
\pgfusepath{clip}%
\pgfsetbuttcap%
\pgfsetroundjoin%
\definecolor{currentfill}{rgb}{0.019608,0.556863,0.850980}%
\pgfsetfillcolor{currentfill}%
\pgfsetfillopacity{0.050000}%
\pgfsetlinewidth{0.000000pt}%
\definecolor{currentstroke}{rgb}{0.019608,0.556863,0.850980}%
\pgfsetstrokecolor{currentstroke}%
\pgfsetdash{}{0pt}%
\pgfpathmoveto{\pgfqpoint{3.429697in}{1.830082in}}%
\pgfpathlineto{\pgfqpoint{3.429697in}{1.935670in}}%
\pgfpathlineto{\pgfqpoint{3.542445in}{1.882739in}}%
\pgfpathlineto{\pgfqpoint{3.542445in}{1.777150in}}%
\pgfpathlineto{\pgfqpoint{3.429697in}{1.830082in}}%
\pgfpathclose%
\pgfusepath{fill}%
\end{pgfscope}%
\begin{pgfscope}%
\pgfpathrectangle{\pgfqpoint{0.765000in}{0.660000in}}{\pgfqpoint{4.620000in}{4.620000in}}%
\pgfusepath{clip}%
\pgfsetbuttcap%
\pgfsetroundjoin%
\definecolor{currentfill}{rgb}{0.019608,0.556863,0.850980}%
\pgfsetfillcolor{currentfill}%
\pgfsetfillopacity{0.050000}%
\pgfsetlinewidth{0.000000pt}%
\definecolor{currentstroke}{rgb}{0.019608,0.556863,0.850980}%
\pgfsetstrokecolor{currentstroke}%
\pgfsetdash{}{0pt}%
\pgfpathmoveto{\pgfqpoint{3.429697in}{1.830082in}}%
\pgfpathlineto{\pgfqpoint{3.429697in}{1.935670in}}%
\pgfpathlineto{\pgfqpoint{3.316950in}{1.882739in}}%
\pgfpathlineto{\pgfqpoint{3.316950in}{1.777150in}}%
\pgfpathlineto{\pgfqpoint{3.429697in}{1.830082in}}%
\pgfpathclose%
\pgfusepath{fill}%
\end{pgfscope}%
\begin{pgfscope}%
\pgfpathrectangle{\pgfqpoint{0.765000in}{0.660000in}}{\pgfqpoint{4.620000in}{4.620000in}}%
\pgfusepath{clip}%
\pgfsetbuttcap%
\pgfsetroundjoin%
\definecolor{currentfill}{rgb}{0.019608,0.556863,0.850980}%
\pgfsetfillcolor{currentfill}%
\pgfsetfillopacity{0.050000}%
\pgfsetlinewidth{0.000000pt}%
\definecolor{currentstroke}{rgb}{0.019608,0.556863,0.850980}%
\pgfsetstrokecolor{currentstroke}%
\pgfsetdash{}{0pt}%
\pgfpathmoveto{\pgfqpoint{3.429697in}{1.830082in}}%
\pgfpathlineto{\pgfqpoint{3.542445in}{1.777150in}}%
\pgfpathlineto{\pgfqpoint{3.429697in}{1.724218in}}%
\pgfpathlineto{\pgfqpoint{3.316950in}{1.777150in}}%
\pgfpathlineto{\pgfqpoint{3.429697in}{1.830082in}}%
\pgfpathclose%
\pgfusepath{fill}%
\end{pgfscope}%
\begin{pgfscope}%
\pgfpathrectangle{\pgfqpoint{0.765000in}{0.660000in}}{\pgfqpoint{4.620000in}{4.620000in}}%
\pgfusepath{clip}%
\pgfsetbuttcap%
\pgfsetroundjoin%
\definecolor{currentfill}{rgb}{0.019608,0.556863,0.850980}%
\pgfsetfillcolor{currentfill}%
\pgfsetfillopacity{0.050000}%
\pgfsetlinewidth{0.000000pt}%
\definecolor{currentstroke}{rgb}{0.019608,0.556863,0.850980}%
\pgfsetstrokecolor{currentstroke}%
\pgfsetdash{}{0pt}%
\pgfpathmoveto{\pgfqpoint{3.429697in}{1.935670in}}%
\pgfpathlineto{\pgfqpoint{3.542445in}{1.882739in}}%
\pgfpathlineto{\pgfqpoint{3.429697in}{1.829807in}}%
\pgfpathlineto{\pgfqpoint{3.316950in}{1.882739in}}%
\pgfpathlineto{\pgfqpoint{3.429697in}{1.935670in}}%
\pgfpathclose%
\pgfusepath{fill}%
\end{pgfscope}%
\begin{pgfscope}%
\pgfpathrectangle{\pgfqpoint{0.765000in}{0.660000in}}{\pgfqpoint{4.620000in}{4.620000in}}%
\pgfusepath{clip}%
\pgfsetbuttcap%
\pgfsetroundjoin%
\definecolor{currentfill}{rgb}{0.019608,0.556863,0.850980}%
\pgfsetfillcolor{currentfill}%
\pgfsetfillopacity{0.050000}%
\pgfsetlinewidth{0.000000pt}%
\definecolor{currentstroke}{rgb}{0.019608,0.556863,0.850980}%
\pgfsetstrokecolor{currentstroke}%
\pgfsetdash{}{0pt}%
\pgfpathmoveto{\pgfqpoint{3.316950in}{1.777150in}}%
\pgfpathlineto{\pgfqpoint{3.316950in}{1.882739in}}%
\pgfpathlineto{\pgfqpoint{3.429697in}{1.829807in}}%
\pgfpathlineto{\pgfqpoint{3.429697in}{1.724218in}}%
\pgfpathlineto{\pgfqpoint{3.316950in}{1.777150in}}%
\pgfpathclose%
\pgfusepath{fill}%
\end{pgfscope}%
\begin{pgfscope}%
\pgfpathrectangle{\pgfqpoint{0.765000in}{0.660000in}}{\pgfqpoint{4.620000in}{4.620000in}}%
\pgfusepath{clip}%
\pgfsetbuttcap%
\pgfsetroundjoin%
\definecolor{currentfill}{rgb}{0.019608,0.556863,0.850980}%
\pgfsetfillcolor{currentfill}%
\pgfsetfillopacity{0.050000}%
\pgfsetlinewidth{0.000000pt}%
\definecolor{currentstroke}{rgb}{0.019608,0.556863,0.850980}%
\pgfsetstrokecolor{currentstroke}%
\pgfsetdash{}{0pt}%
\pgfpathmoveto{\pgfqpoint{3.542445in}{1.777150in}}%
\pgfpathlineto{\pgfqpoint{3.542445in}{1.882739in}}%
\pgfpathlineto{\pgfqpoint{3.429697in}{1.829807in}}%
\pgfpathlineto{\pgfqpoint{3.429697in}{1.724218in}}%
\pgfpathlineto{\pgfqpoint{3.542445in}{1.777150in}}%
\pgfpathclose%
\pgfusepath{fill}%
\end{pgfscope}%
\begin{pgfscope}%
\pgfpathrectangle{\pgfqpoint{0.765000in}{0.660000in}}{\pgfqpoint{4.620000in}{4.620000in}}%
\pgfusepath{clip}%
\pgfsetbuttcap%
\pgfsetroundjoin%
\definecolor{currentfill}{rgb}{0.019608,0.556863,0.850980}%
\pgfsetfillcolor{currentfill}%
\pgfsetfillopacity{0.050000}%
\pgfsetlinewidth{0.000000pt}%
\definecolor{currentstroke}{rgb}{0.019608,0.556863,0.850980}%
\pgfsetstrokecolor{currentstroke}%
\pgfsetdash{}{0pt}%
\pgfpathmoveto{\pgfqpoint{3.233804in}{1.810415in}}%
\pgfpathlineto{\pgfqpoint{3.233804in}{1.916004in}}%
\pgfpathlineto{\pgfqpoint{3.346551in}{1.863072in}}%
\pgfpathlineto{\pgfqpoint{3.346551in}{1.757483in}}%
\pgfpathlineto{\pgfqpoint{3.233804in}{1.810415in}}%
\pgfpathclose%
\pgfusepath{fill}%
\end{pgfscope}%
\begin{pgfscope}%
\pgfpathrectangle{\pgfqpoint{0.765000in}{0.660000in}}{\pgfqpoint{4.620000in}{4.620000in}}%
\pgfusepath{clip}%
\pgfsetbuttcap%
\pgfsetroundjoin%
\definecolor{currentfill}{rgb}{0.019608,0.556863,0.850980}%
\pgfsetfillcolor{currentfill}%
\pgfsetfillopacity{0.050000}%
\pgfsetlinewidth{0.000000pt}%
\definecolor{currentstroke}{rgb}{0.019608,0.556863,0.850980}%
\pgfsetstrokecolor{currentstroke}%
\pgfsetdash{}{0pt}%
\pgfpathmoveto{\pgfqpoint{3.233804in}{1.810415in}}%
\pgfpathlineto{\pgfqpoint{3.233804in}{1.916004in}}%
\pgfpathlineto{\pgfqpoint{3.121056in}{1.863072in}}%
\pgfpathlineto{\pgfqpoint{3.121056in}{1.757483in}}%
\pgfpathlineto{\pgfqpoint{3.233804in}{1.810415in}}%
\pgfpathclose%
\pgfusepath{fill}%
\end{pgfscope}%
\begin{pgfscope}%
\pgfpathrectangle{\pgfqpoint{0.765000in}{0.660000in}}{\pgfqpoint{4.620000in}{4.620000in}}%
\pgfusepath{clip}%
\pgfsetbuttcap%
\pgfsetroundjoin%
\definecolor{currentfill}{rgb}{0.019608,0.556863,0.850980}%
\pgfsetfillcolor{currentfill}%
\pgfsetfillopacity{0.050000}%
\pgfsetlinewidth{0.000000pt}%
\definecolor{currentstroke}{rgb}{0.019608,0.556863,0.850980}%
\pgfsetstrokecolor{currentstroke}%
\pgfsetdash{}{0pt}%
\pgfpathmoveto{\pgfqpoint{3.233804in}{1.810415in}}%
\pgfpathlineto{\pgfqpoint{3.346551in}{1.757483in}}%
\pgfpathlineto{\pgfqpoint{3.233804in}{1.704552in}}%
\pgfpathlineto{\pgfqpoint{3.121056in}{1.757483in}}%
\pgfpathlineto{\pgfqpoint{3.233804in}{1.810415in}}%
\pgfpathclose%
\pgfusepath{fill}%
\end{pgfscope}%
\begin{pgfscope}%
\pgfpathrectangle{\pgfqpoint{0.765000in}{0.660000in}}{\pgfqpoint{4.620000in}{4.620000in}}%
\pgfusepath{clip}%
\pgfsetbuttcap%
\pgfsetroundjoin%
\definecolor{currentfill}{rgb}{0.019608,0.556863,0.850980}%
\pgfsetfillcolor{currentfill}%
\pgfsetfillopacity{0.050000}%
\pgfsetlinewidth{0.000000pt}%
\definecolor{currentstroke}{rgb}{0.019608,0.556863,0.850980}%
\pgfsetstrokecolor{currentstroke}%
\pgfsetdash{}{0pt}%
\pgfpathmoveto{\pgfqpoint{3.233804in}{1.916004in}}%
\pgfpathlineto{\pgfqpoint{3.346551in}{1.863072in}}%
\pgfpathlineto{\pgfqpoint{3.233804in}{1.810141in}}%
\pgfpathlineto{\pgfqpoint{3.121056in}{1.863072in}}%
\pgfpathlineto{\pgfqpoint{3.233804in}{1.916004in}}%
\pgfpathclose%
\pgfusepath{fill}%
\end{pgfscope}%
\begin{pgfscope}%
\pgfpathrectangle{\pgfqpoint{0.765000in}{0.660000in}}{\pgfqpoint{4.620000in}{4.620000in}}%
\pgfusepath{clip}%
\pgfsetbuttcap%
\pgfsetroundjoin%
\definecolor{currentfill}{rgb}{0.019608,0.556863,0.850980}%
\pgfsetfillcolor{currentfill}%
\pgfsetfillopacity{0.050000}%
\pgfsetlinewidth{0.000000pt}%
\definecolor{currentstroke}{rgb}{0.019608,0.556863,0.850980}%
\pgfsetstrokecolor{currentstroke}%
\pgfsetdash{}{0pt}%
\pgfpathmoveto{\pgfqpoint{3.121056in}{1.757483in}}%
\pgfpathlineto{\pgfqpoint{3.121056in}{1.863072in}}%
\pgfpathlineto{\pgfqpoint{3.233804in}{1.810141in}}%
\pgfpathlineto{\pgfqpoint{3.233804in}{1.704552in}}%
\pgfpathlineto{\pgfqpoint{3.121056in}{1.757483in}}%
\pgfpathclose%
\pgfusepath{fill}%
\end{pgfscope}%
\begin{pgfscope}%
\pgfpathrectangle{\pgfqpoint{0.765000in}{0.660000in}}{\pgfqpoint{4.620000in}{4.620000in}}%
\pgfusepath{clip}%
\pgfsetbuttcap%
\pgfsetroundjoin%
\definecolor{currentfill}{rgb}{0.019608,0.556863,0.850980}%
\pgfsetfillcolor{currentfill}%
\pgfsetfillopacity{0.050000}%
\pgfsetlinewidth{0.000000pt}%
\definecolor{currentstroke}{rgb}{0.019608,0.556863,0.850980}%
\pgfsetstrokecolor{currentstroke}%
\pgfsetdash{}{0pt}%
\pgfpathmoveto{\pgfqpoint{3.346551in}{1.757483in}}%
\pgfpathlineto{\pgfqpoint{3.346551in}{1.863072in}}%
\pgfpathlineto{\pgfqpoint{3.233804in}{1.810141in}}%
\pgfpathlineto{\pgfqpoint{3.233804in}{1.704552in}}%
\pgfpathlineto{\pgfqpoint{3.346551in}{1.757483in}}%
\pgfpathclose%
\pgfusepath{fill}%
\end{pgfscope}%
\begin{pgfscope}%
\pgfpathrectangle{\pgfqpoint{0.765000in}{0.660000in}}{\pgfqpoint{4.620000in}{4.620000in}}%
\pgfusepath{clip}%
\pgfsetbuttcap%
\pgfsetroundjoin%
\definecolor{currentfill}{rgb}{0.019608,0.556863,0.850980}%
\pgfsetfillcolor{currentfill}%
\pgfsetfillopacity{0.050000}%
\pgfsetlinewidth{0.000000pt}%
\definecolor{currentstroke}{rgb}{0.019608,0.556863,0.850980}%
\pgfsetstrokecolor{currentstroke}%
\pgfsetdash{}{0pt}%
\pgfpathmoveto{\pgfqpoint{3.111211in}{1.828028in}}%
\pgfpathlineto{\pgfqpoint{3.111211in}{1.933616in}}%
\pgfpathlineto{\pgfqpoint{3.223958in}{1.880685in}}%
\pgfpathlineto{\pgfqpoint{3.223958in}{1.775096in}}%
\pgfpathlineto{\pgfqpoint{3.111211in}{1.828028in}}%
\pgfpathclose%
\pgfusepath{fill}%
\end{pgfscope}%
\begin{pgfscope}%
\pgfpathrectangle{\pgfqpoint{0.765000in}{0.660000in}}{\pgfqpoint{4.620000in}{4.620000in}}%
\pgfusepath{clip}%
\pgfsetbuttcap%
\pgfsetroundjoin%
\definecolor{currentfill}{rgb}{0.019608,0.556863,0.850980}%
\pgfsetfillcolor{currentfill}%
\pgfsetfillopacity{0.050000}%
\pgfsetlinewidth{0.000000pt}%
\definecolor{currentstroke}{rgb}{0.019608,0.556863,0.850980}%
\pgfsetstrokecolor{currentstroke}%
\pgfsetdash{}{0pt}%
\pgfpathmoveto{\pgfqpoint{3.111211in}{1.828028in}}%
\pgfpathlineto{\pgfqpoint{3.111211in}{1.933616in}}%
\pgfpathlineto{\pgfqpoint{2.998463in}{1.880685in}}%
\pgfpathlineto{\pgfqpoint{2.998463in}{1.775096in}}%
\pgfpathlineto{\pgfqpoint{3.111211in}{1.828028in}}%
\pgfpathclose%
\pgfusepath{fill}%
\end{pgfscope}%
\begin{pgfscope}%
\pgfpathrectangle{\pgfqpoint{0.765000in}{0.660000in}}{\pgfqpoint{4.620000in}{4.620000in}}%
\pgfusepath{clip}%
\pgfsetbuttcap%
\pgfsetroundjoin%
\definecolor{currentfill}{rgb}{0.019608,0.556863,0.850980}%
\pgfsetfillcolor{currentfill}%
\pgfsetfillopacity{0.050000}%
\pgfsetlinewidth{0.000000pt}%
\definecolor{currentstroke}{rgb}{0.019608,0.556863,0.850980}%
\pgfsetstrokecolor{currentstroke}%
\pgfsetdash{}{0pt}%
\pgfpathmoveto{\pgfqpoint{3.111211in}{1.828028in}}%
\pgfpathlineto{\pgfqpoint{3.223958in}{1.775096in}}%
\pgfpathlineto{\pgfqpoint{3.111211in}{1.722164in}}%
\pgfpathlineto{\pgfqpoint{2.998463in}{1.775096in}}%
\pgfpathlineto{\pgfqpoint{3.111211in}{1.828028in}}%
\pgfpathclose%
\pgfusepath{fill}%
\end{pgfscope}%
\begin{pgfscope}%
\pgfpathrectangle{\pgfqpoint{0.765000in}{0.660000in}}{\pgfqpoint{4.620000in}{4.620000in}}%
\pgfusepath{clip}%
\pgfsetbuttcap%
\pgfsetroundjoin%
\definecolor{currentfill}{rgb}{0.019608,0.556863,0.850980}%
\pgfsetfillcolor{currentfill}%
\pgfsetfillopacity{0.050000}%
\pgfsetlinewidth{0.000000pt}%
\definecolor{currentstroke}{rgb}{0.019608,0.556863,0.850980}%
\pgfsetstrokecolor{currentstroke}%
\pgfsetdash{}{0pt}%
\pgfpathmoveto{\pgfqpoint{3.111211in}{1.933616in}}%
\pgfpathlineto{\pgfqpoint{3.223958in}{1.880685in}}%
\pgfpathlineto{\pgfqpoint{3.111211in}{1.827753in}}%
\pgfpathlineto{\pgfqpoint{2.998463in}{1.880685in}}%
\pgfpathlineto{\pgfqpoint{3.111211in}{1.933616in}}%
\pgfpathclose%
\pgfusepath{fill}%
\end{pgfscope}%
\begin{pgfscope}%
\pgfpathrectangle{\pgfqpoint{0.765000in}{0.660000in}}{\pgfqpoint{4.620000in}{4.620000in}}%
\pgfusepath{clip}%
\pgfsetbuttcap%
\pgfsetroundjoin%
\definecolor{currentfill}{rgb}{0.019608,0.556863,0.850980}%
\pgfsetfillcolor{currentfill}%
\pgfsetfillopacity{0.050000}%
\pgfsetlinewidth{0.000000pt}%
\definecolor{currentstroke}{rgb}{0.019608,0.556863,0.850980}%
\pgfsetstrokecolor{currentstroke}%
\pgfsetdash{}{0pt}%
\pgfpathmoveto{\pgfqpoint{2.998463in}{1.775096in}}%
\pgfpathlineto{\pgfqpoint{2.998463in}{1.880685in}}%
\pgfpathlineto{\pgfqpoint{3.111211in}{1.827753in}}%
\pgfpathlineto{\pgfqpoint{3.111211in}{1.722164in}}%
\pgfpathlineto{\pgfqpoint{2.998463in}{1.775096in}}%
\pgfpathclose%
\pgfusepath{fill}%
\end{pgfscope}%
\begin{pgfscope}%
\pgfpathrectangle{\pgfqpoint{0.765000in}{0.660000in}}{\pgfqpoint{4.620000in}{4.620000in}}%
\pgfusepath{clip}%
\pgfsetbuttcap%
\pgfsetroundjoin%
\definecolor{currentfill}{rgb}{0.019608,0.556863,0.850980}%
\pgfsetfillcolor{currentfill}%
\pgfsetfillopacity{0.050000}%
\pgfsetlinewidth{0.000000pt}%
\definecolor{currentstroke}{rgb}{0.019608,0.556863,0.850980}%
\pgfsetstrokecolor{currentstroke}%
\pgfsetdash{}{0pt}%
\pgfpathmoveto{\pgfqpoint{3.223958in}{1.775096in}}%
\pgfpathlineto{\pgfqpoint{3.223958in}{1.880685in}}%
\pgfpathlineto{\pgfqpoint{3.111211in}{1.827753in}}%
\pgfpathlineto{\pgfqpoint{3.111211in}{1.722164in}}%
\pgfpathlineto{\pgfqpoint{3.223958in}{1.775096in}}%
\pgfpathclose%
\pgfusepath{fill}%
\end{pgfscope}%
\begin{pgfscope}%
\pgfpathrectangle{\pgfqpoint{0.765000in}{0.660000in}}{\pgfqpoint{4.620000in}{4.620000in}}%
\pgfusepath{clip}%
\pgfsetbuttcap%
\pgfsetroundjoin%
\definecolor{currentfill}{rgb}{0.019608,0.556863,0.850980}%
\pgfsetfillcolor{currentfill}%
\pgfsetfillopacity{0.050000}%
\pgfsetlinewidth{0.000000pt}%
\definecolor{currentstroke}{rgb}{0.019608,0.556863,0.850980}%
\pgfsetstrokecolor{currentstroke}%
\pgfsetdash{}{0pt}%
\pgfpathmoveto{\pgfqpoint{3.334439in}{1.679022in}}%
\pgfpathlineto{\pgfqpoint{3.334439in}{1.784611in}}%
\pgfpathlineto{\pgfqpoint{3.447187in}{1.731679in}}%
\pgfpathlineto{\pgfqpoint{3.447187in}{1.626091in}}%
\pgfpathlineto{\pgfqpoint{3.334439in}{1.679022in}}%
\pgfpathclose%
\pgfusepath{fill}%
\end{pgfscope}%
\begin{pgfscope}%
\pgfpathrectangle{\pgfqpoint{0.765000in}{0.660000in}}{\pgfqpoint{4.620000in}{4.620000in}}%
\pgfusepath{clip}%
\pgfsetbuttcap%
\pgfsetroundjoin%
\definecolor{currentfill}{rgb}{0.019608,0.556863,0.850980}%
\pgfsetfillcolor{currentfill}%
\pgfsetfillopacity{0.050000}%
\pgfsetlinewidth{0.000000pt}%
\definecolor{currentstroke}{rgb}{0.019608,0.556863,0.850980}%
\pgfsetstrokecolor{currentstroke}%
\pgfsetdash{}{0pt}%
\pgfpathmoveto{\pgfqpoint{3.334439in}{1.679022in}}%
\pgfpathlineto{\pgfqpoint{3.334439in}{1.784611in}}%
\pgfpathlineto{\pgfqpoint{3.221692in}{1.731679in}}%
\pgfpathlineto{\pgfqpoint{3.221692in}{1.626091in}}%
\pgfpathlineto{\pgfqpoint{3.334439in}{1.679022in}}%
\pgfpathclose%
\pgfusepath{fill}%
\end{pgfscope}%
\begin{pgfscope}%
\pgfpathrectangle{\pgfqpoint{0.765000in}{0.660000in}}{\pgfqpoint{4.620000in}{4.620000in}}%
\pgfusepath{clip}%
\pgfsetbuttcap%
\pgfsetroundjoin%
\definecolor{currentfill}{rgb}{0.019608,0.556863,0.850980}%
\pgfsetfillcolor{currentfill}%
\pgfsetfillopacity{0.050000}%
\pgfsetlinewidth{0.000000pt}%
\definecolor{currentstroke}{rgb}{0.019608,0.556863,0.850980}%
\pgfsetstrokecolor{currentstroke}%
\pgfsetdash{}{0pt}%
\pgfpathmoveto{\pgfqpoint{3.334439in}{1.679022in}}%
\pgfpathlineto{\pgfqpoint{3.447187in}{1.626091in}}%
\pgfpathlineto{\pgfqpoint{3.334439in}{1.573159in}}%
\pgfpathlineto{\pgfqpoint{3.221692in}{1.626091in}}%
\pgfpathlineto{\pgfqpoint{3.334439in}{1.679022in}}%
\pgfpathclose%
\pgfusepath{fill}%
\end{pgfscope}%
\begin{pgfscope}%
\pgfpathrectangle{\pgfqpoint{0.765000in}{0.660000in}}{\pgfqpoint{4.620000in}{4.620000in}}%
\pgfusepath{clip}%
\pgfsetbuttcap%
\pgfsetroundjoin%
\definecolor{currentfill}{rgb}{0.019608,0.556863,0.850980}%
\pgfsetfillcolor{currentfill}%
\pgfsetfillopacity{0.050000}%
\pgfsetlinewidth{0.000000pt}%
\definecolor{currentstroke}{rgb}{0.019608,0.556863,0.850980}%
\pgfsetstrokecolor{currentstroke}%
\pgfsetdash{}{0pt}%
\pgfpathmoveto{\pgfqpoint{3.334439in}{1.784611in}}%
\pgfpathlineto{\pgfqpoint{3.447187in}{1.731679in}}%
\pgfpathlineto{\pgfqpoint{3.334439in}{1.678748in}}%
\pgfpathlineto{\pgfqpoint{3.221692in}{1.731679in}}%
\pgfpathlineto{\pgfqpoint{3.334439in}{1.784611in}}%
\pgfpathclose%
\pgfusepath{fill}%
\end{pgfscope}%
\begin{pgfscope}%
\pgfpathrectangle{\pgfqpoint{0.765000in}{0.660000in}}{\pgfqpoint{4.620000in}{4.620000in}}%
\pgfusepath{clip}%
\pgfsetbuttcap%
\pgfsetroundjoin%
\definecolor{currentfill}{rgb}{0.019608,0.556863,0.850980}%
\pgfsetfillcolor{currentfill}%
\pgfsetfillopacity{0.050000}%
\pgfsetlinewidth{0.000000pt}%
\definecolor{currentstroke}{rgb}{0.019608,0.556863,0.850980}%
\pgfsetstrokecolor{currentstroke}%
\pgfsetdash{}{0pt}%
\pgfpathmoveto{\pgfqpoint{3.221692in}{1.626091in}}%
\pgfpathlineto{\pgfqpoint{3.221692in}{1.731679in}}%
\pgfpathlineto{\pgfqpoint{3.334439in}{1.678748in}}%
\pgfpathlineto{\pgfqpoint{3.334439in}{1.573159in}}%
\pgfpathlineto{\pgfqpoint{3.221692in}{1.626091in}}%
\pgfpathclose%
\pgfusepath{fill}%
\end{pgfscope}%
\begin{pgfscope}%
\pgfpathrectangle{\pgfqpoint{0.765000in}{0.660000in}}{\pgfqpoint{4.620000in}{4.620000in}}%
\pgfusepath{clip}%
\pgfsetbuttcap%
\pgfsetroundjoin%
\definecolor{currentfill}{rgb}{0.019608,0.556863,0.850980}%
\pgfsetfillcolor{currentfill}%
\pgfsetfillopacity{0.050000}%
\pgfsetlinewidth{0.000000pt}%
\definecolor{currentstroke}{rgb}{0.019608,0.556863,0.850980}%
\pgfsetstrokecolor{currentstroke}%
\pgfsetdash{}{0pt}%
\pgfpathmoveto{\pgfqpoint{3.447187in}{1.626091in}}%
\pgfpathlineto{\pgfqpoint{3.447187in}{1.731679in}}%
\pgfpathlineto{\pgfqpoint{3.334439in}{1.678748in}}%
\pgfpathlineto{\pgfqpoint{3.334439in}{1.573159in}}%
\pgfpathlineto{\pgfqpoint{3.447187in}{1.626091in}}%
\pgfpathclose%
\pgfusepath{fill}%
\end{pgfscope}%
\begin{pgfscope}%
\pgfpathrectangle{\pgfqpoint{0.765000in}{0.660000in}}{\pgfqpoint{4.620000in}{4.620000in}}%
\pgfusepath{clip}%
\pgfsetbuttcap%
\pgfsetroundjoin%
\definecolor{currentfill}{rgb}{0.019608,0.556863,0.850980}%
\pgfsetfillcolor{currentfill}%
\pgfsetfillopacity{0.050000}%
\pgfsetlinewidth{0.000000pt}%
\definecolor{currentstroke}{rgb}{0.019608,0.556863,0.850980}%
\pgfsetstrokecolor{currentstroke}%
\pgfsetdash{}{0pt}%
\pgfpathmoveto{\pgfqpoint{3.366704in}{1.759766in}}%
\pgfpathlineto{\pgfqpoint{3.366704in}{1.865355in}}%
\pgfpathlineto{\pgfqpoint{3.479451in}{1.812423in}}%
\pgfpathlineto{\pgfqpoint{3.479451in}{1.706835in}}%
\pgfpathlineto{\pgfqpoint{3.366704in}{1.759766in}}%
\pgfpathclose%
\pgfusepath{fill}%
\end{pgfscope}%
\begin{pgfscope}%
\pgfpathrectangle{\pgfqpoint{0.765000in}{0.660000in}}{\pgfqpoint{4.620000in}{4.620000in}}%
\pgfusepath{clip}%
\pgfsetbuttcap%
\pgfsetroundjoin%
\definecolor{currentfill}{rgb}{0.019608,0.556863,0.850980}%
\pgfsetfillcolor{currentfill}%
\pgfsetfillopacity{0.050000}%
\pgfsetlinewidth{0.000000pt}%
\definecolor{currentstroke}{rgb}{0.019608,0.556863,0.850980}%
\pgfsetstrokecolor{currentstroke}%
\pgfsetdash{}{0pt}%
\pgfpathmoveto{\pgfqpoint{3.366704in}{1.759766in}}%
\pgfpathlineto{\pgfqpoint{3.366704in}{1.865355in}}%
\pgfpathlineto{\pgfqpoint{3.253956in}{1.812423in}}%
\pgfpathlineto{\pgfqpoint{3.253956in}{1.706835in}}%
\pgfpathlineto{\pgfqpoint{3.366704in}{1.759766in}}%
\pgfpathclose%
\pgfusepath{fill}%
\end{pgfscope}%
\begin{pgfscope}%
\pgfpathrectangle{\pgfqpoint{0.765000in}{0.660000in}}{\pgfqpoint{4.620000in}{4.620000in}}%
\pgfusepath{clip}%
\pgfsetbuttcap%
\pgfsetroundjoin%
\definecolor{currentfill}{rgb}{0.019608,0.556863,0.850980}%
\pgfsetfillcolor{currentfill}%
\pgfsetfillopacity{0.050000}%
\pgfsetlinewidth{0.000000pt}%
\definecolor{currentstroke}{rgb}{0.019608,0.556863,0.850980}%
\pgfsetstrokecolor{currentstroke}%
\pgfsetdash{}{0pt}%
\pgfpathmoveto{\pgfqpoint{3.366704in}{1.759766in}}%
\pgfpathlineto{\pgfqpoint{3.479451in}{1.706835in}}%
\pgfpathlineto{\pgfqpoint{3.366704in}{1.653903in}}%
\pgfpathlineto{\pgfqpoint{3.253956in}{1.706835in}}%
\pgfpathlineto{\pgfqpoint{3.366704in}{1.759766in}}%
\pgfpathclose%
\pgfusepath{fill}%
\end{pgfscope}%
\begin{pgfscope}%
\pgfpathrectangle{\pgfqpoint{0.765000in}{0.660000in}}{\pgfqpoint{4.620000in}{4.620000in}}%
\pgfusepath{clip}%
\pgfsetbuttcap%
\pgfsetroundjoin%
\definecolor{currentfill}{rgb}{0.019608,0.556863,0.850980}%
\pgfsetfillcolor{currentfill}%
\pgfsetfillopacity{0.050000}%
\pgfsetlinewidth{0.000000pt}%
\definecolor{currentstroke}{rgb}{0.019608,0.556863,0.850980}%
\pgfsetstrokecolor{currentstroke}%
\pgfsetdash{}{0pt}%
\pgfpathmoveto{\pgfqpoint{3.366704in}{1.865355in}}%
\pgfpathlineto{\pgfqpoint{3.479451in}{1.812423in}}%
\pgfpathlineto{\pgfqpoint{3.366704in}{1.759492in}}%
\pgfpathlineto{\pgfqpoint{3.253956in}{1.812423in}}%
\pgfpathlineto{\pgfqpoint{3.366704in}{1.865355in}}%
\pgfpathclose%
\pgfusepath{fill}%
\end{pgfscope}%
\begin{pgfscope}%
\pgfpathrectangle{\pgfqpoint{0.765000in}{0.660000in}}{\pgfqpoint{4.620000in}{4.620000in}}%
\pgfusepath{clip}%
\pgfsetbuttcap%
\pgfsetroundjoin%
\definecolor{currentfill}{rgb}{0.019608,0.556863,0.850980}%
\pgfsetfillcolor{currentfill}%
\pgfsetfillopacity{0.050000}%
\pgfsetlinewidth{0.000000pt}%
\definecolor{currentstroke}{rgb}{0.019608,0.556863,0.850980}%
\pgfsetstrokecolor{currentstroke}%
\pgfsetdash{}{0pt}%
\pgfpathmoveto{\pgfqpoint{3.253956in}{1.706835in}}%
\pgfpathlineto{\pgfqpoint{3.253956in}{1.812423in}}%
\pgfpathlineto{\pgfqpoint{3.366704in}{1.759492in}}%
\pgfpathlineto{\pgfqpoint{3.366704in}{1.653903in}}%
\pgfpathlineto{\pgfqpoint{3.253956in}{1.706835in}}%
\pgfpathclose%
\pgfusepath{fill}%
\end{pgfscope}%
\begin{pgfscope}%
\pgfpathrectangle{\pgfqpoint{0.765000in}{0.660000in}}{\pgfqpoint{4.620000in}{4.620000in}}%
\pgfusepath{clip}%
\pgfsetbuttcap%
\pgfsetroundjoin%
\definecolor{currentfill}{rgb}{0.019608,0.556863,0.850980}%
\pgfsetfillcolor{currentfill}%
\pgfsetfillopacity{0.050000}%
\pgfsetlinewidth{0.000000pt}%
\definecolor{currentstroke}{rgb}{0.019608,0.556863,0.850980}%
\pgfsetstrokecolor{currentstroke}%
\pgfsetdash{}{0pt}%
\pgfpathmoveto{\pgfqpoint{3.479451in}{1.706835in}}%
\pgfpathlineto{\pgfqpoint{3.479451in}{1.812423in}}%
\pgfpathlineto{\pgfqpoint{3.366704in}{1.759492in}}%
\pgfpathlineto{\pgfqpoint{3.366704in}{1.653903in}}%
\pgfpathlineto{\pgfqpoint{3.479451in}{1.706835in}}%
\pgfpathclose%
\pgfusepath{fill}%
\end{pgfscope}%
\begin{pgfscope}%
\pgfpathrectangle{\pgfqpoint{0.765000in}{0.660000in}}{\pgfqpoint{4.620000in}{4.620000in}}%
\pgfusepath{clip}%
\pgfsetbuttcap%
\pgfsetroundjoin%
\definecolor{currentfill}{rgb}{0.019608,0.556863,0.850980}%
\pgfsetfillcolor{currentfill}%
\pgfsetfillopacity{0.050000}%
\pgfsetlinewidth{0.000000pt}%
\definecolor{currentstroke}{rgb}{0.019608,0.556863,0.850980}%
\pgfsetstrokecolor{currentstroke}%
\pgfsetdash{}{0pt}%
\pgfpathmoveto{\pgfqpoint{2.927916in}{1.756677in}}%
\pgfpathlineto{\pgfqpoint{2.927916in}{1.862266in}}%
\pgfpathlineto{\pgfqpoint{3.040663in}{1.809335in}}%
\pgfpathlineto{\pgfqpoint{3.040663in}{1.703746in}}%
\pgfpathlineto{\pgfqpoint{2.927916in}{1.756677in}}%
\pgfpathclose%
\pgfusepath{fill}%
\end{pgfscope}%
\begin{pgfscope}%
\pgfpathrectangle{\pgfqpoint{0.765000in}{0.660000in}}{\pgfqpoint{4.620000in}{4.620000in}}%
\pgfusepath{clip}%
\pgfsetbuttcap%
\pgfsetroundjoin%
\definecolor{currentfill}{rgb}{0.019608,0.556863,0.850980}%
\pgfsetfillcolor{currentfill}%
\pgfsetfillopacity{0.050000}%
\pgfsetlinewidth{0.000000pt}%
\definecolor{currentstroke}{rgb}{0.019608,0.556863,0.850980}%
\pgfsetstrokecolor{currentstroke}%
\pgfsetdash{}{0pt}%
\pgfpathmoveto{\pgfqpoint{2.927916in}{1.756677in}}%
\pgfpathlineto{\pgfqpoint{2.927916in}{1.862266in}}%
\pgfpathlineto{\pgfqpoint{2.815168in}{1.809335in}}%
\pgfpathlineto{\pgfqpoint{2.815168in}{1.703746in}}%
\pgfpathlineto{\pgfqpoint{2.927916in}{1.756677in}}%
\pgfpathclose%
\pgfusepath{fill}%
\end{pgfscope}%
\begin{pgfscope}%
\pgfpathrectangle{\pgfqpoint{0.765000in}{0.660000in}}{\pgfqpoint{4.620000in}{4.620000in}}%
\pgfusepath{clip}%
\pgfsetbuttcap%
\pgfsetroundjoin%
\definecolor{currentfill}{rgb}{0.019608,0.556863,0.850980}%
\pgfsetfillcolor{currentfill}%
\pgfsetfillopacity{0.050000}%
\pgfsetlinewidth{0.000000pt}%
\definecolor{currentstroke}{rgb}{0.019608,0.556863,0.850980}%
\pgfsetstrokecolor{currentstroke}%
\pgfsetdash{}{0pt}%
\pgfpathmoveto{\pgfqpoint{2.927916in}{1.756677in}}%
\pgfpathlineto{\pgfqpoint{3.040663in}{1.703746in}}%
\pgfpathlineto{\pgfqpoint{2.927916in}{1.650814in}}%
\pgfpathlineto{\pgfqpoint{2.815168in}{1.703746in}}%
\pgfpathlineto{\pgfqpoint{2.927916in}{1.756677in}}%
\pgfpathclose%
\pgfusepath{fill}%
\end{pgfscope}%
\begin{pgfscope}%
\pgfpathrectangle{\pgfqpoint{0.765000in}{0.660000in}}{\pgfqpoint{4.620000in}{4.620000in}}%
\pgfusepath{clip}%
\pgfsetbuttcap%
\pgfsetroundjoin%
\definecolor{currentfill}{rgb}{0.019608,0.556863,0.850980}%
\pgfsetfillcolor{currentfill}%
\pgfsetfillopacity{0.050000}%
\pgfsetlinewidth{0.000000pt}%
\definecolor{currentstroke}{rgb}{0.019608,0.556863,0.850980}%
\pgfsetstrokecolor{currentstroke}%
\pgfsetdash{}{0pt}%
\pgfpathmoveto{\pgfqpoint{2.927916in}{1.862266in}}%
\pgfpathlineto{\pgfqpoint{3.040663in}{1.809335in}}%
\pgfpathlineto{\pgfqpoint{2.927916in}{1.756403in}}%
\pgfpathlineto{\pgfqpoint{2.815168in}{1.809335in}}%
\pgfpathlineto{\pgfqpoint{2.927916in}{1.862266in}}%
\pgfpathclose%
\pgfusepath{fill}%
\end{pgfscope}%
\begin{pgfscope}%
\pgfpathrectangle{\pgfqpoint{0.765000in}{0.660000in}}{\pgfqpoint{4.620000in}{4.620000in}}%
\pgfusepath{clip}%
\pgfsetbuttcap%
\pgfsetroundjoin%
\definecolor{currentfill}{rgb}{0.019608,0.556863,0.850980}%
\pgfsetfillcolor{currentfill}%
\pgfsetfillopacity{0.050000}%
\pgfsetlinewidth{0.000000pt}%
\definecolor{currentstroke}{rgb}{0.019608,0.556863,0.850980}%
\pgfsetstrokecolor{currentstroke}%
\pgfsetdash{}{0pt}%
\pgfpathmoveto{\pgfqpoint{2.815168in}{1.703746in}}%
\pgfpathlineto{\pgfqpoint{2.815168in}{1.809335in}}%
\pgfpathlineto{\pgfqpoint{2.927916in}{1.756403in}}%
\pgfpathlineto{\pgfqpoint{2.927916in}{1.650814in}}%
\pgfpathlineto{\pgfqpoint{2.815168in}{1.703746in}}%
\pgfpathclose%
\pgfusepath{fill}%
\end{pgfscope}%
\begin{pgfscope}%
\pgfpathrectangle{\pgfqpoint{0.765000in}{0.660000in}}{\pgfqpoint{4.620000in}{4.620000in}}%
\pgfusepath{clip}%
\pgfsetbuttcap%
\pgfsetroundjoin%
\definecolor{currentfill}{rgb}{0.019608,0.556863,0.850980}%
\pgfsetfillcolor{currentfill}%
\pgfsetfillopacity{0.050000}%
\pgfsetlinewidth{0.000000pt}%
\definecolor{currentstroke}{rgb}{0.019608,0.556863,0.850980}%
\pgfsetstrokecolor{currentstroke}%
\pgfsetdash{}{0pt}%
\pgfpathmoveto{\pgfqpoint{3.040663in}{1.703746in}}%
\pgfpathlineto{\pgfqpoint{3.040663in}{1.809335in}}%
\pgfpathlineto{\pgfqpoint{2.927916in}{1.756403in}}%
\pgfpathlineto{\pgfqpoint{2.927916in}{1.650814in}}%
\pgfpathlineto{\pgfqpoint{3.040663in}{1.703746in}}%
\pgfpathclose%
\pgfusepath{fill}%
\end{pgfscope}%
\begin{pgfscope}%
\pgfpathrectangle{\pgfqpoint{0.765000in}{0.660000in}}{\pgfqpoint{4.620000in}{4.620000in}}%
\pgfusepath{clip}%
\pgfsetbuttcap%
\pgfsetroundjoin%
\definecolor{currentfill}{rgb}{0.019608,0.556863,0.850980}%
\pgfsetfillcolor{currentfill}%
\pgfsetfillopacity{0.050000}%
\pgfsetlinewidth{0.000000pt}%
\definecolor{currentstroke}{rgb}{0.019608,0.556863,0.850980}%
\pgfsetstrokecolor{currentstroke}%
\pgfsetdash{}{0pt}%
\pgfpathmoveto{\pgfqpoint{3.045502in}{1.713490in}}%
\pgfpathlineto{\pgfqpoint{3.045502in}{1.819079in}}%
\pgfpathlineto{\pgfqpoint{3.158250in}{1.766147in}}%
\pgfpathlineto{\pgfqpoint{3.158250in}{1.660558in}}%
\pgfpathlineto{\pgfqpoint{3.045502in}{1.713490in}}%
\pgfpathclose%
\pgfusepath{fill}%
\end{pgfscope}%
\begin{pgfscope}%
\pgfpathrectangle{\pgfqpoint{0.765000in}{0.660000in}}{\pgfqpoint{4.620000in}{4.620000in}}%
\pgfusepath{clip}%
\pgfsetbuttcap%
\pgfsetroundjoin%
\definecolor{currentfill}{rgb}{0.019608,0.556863,0.850980}%
\pgfsetfillcolor{currentfill}%
\pgfsetfillopacity{0.050000}%
\pgfsetlinewidth{0.000000pt}%
\definecolor{currentstroke}{rgb}{0.019608,0.556863,0.850980}%
\pgfsetstrokecolor{currentstroke}%
\pgfsetdash{}{0pt}%
\pgfpathmoveto{\pgfqpoint{3.045502in}{1.713490in}}%
\pgfpathlineto{\pgfqpoint{3.045502in}{1.819079in}}%
\pgfpathlineto{\pgfqpoint{2.932755in}{1.766147in}}%
\pgfpathlineto{\pgfqpoint{2.932755in}{1.660558in}}%
\pgfpathlineto{\pgfqpoint{3.045502in}{1.713490in}}%
\pgfpathclose%
\pgfusepath{fill}%
\end{pgfscope}%
\begin{pgfscope}%
\pgfpathrectangle{\pgfqpoint{0.765000in}{0.660000in}}{\pgfqpoint{4.620000in}{4.620000in}}%
\pgfusepath{clip}%
\pgfsetbuttcap%
\pgfsetroundjoin%
\definecolor{currentfill}{rgb}{0.019608,0.556863,0.850980}%
\pgfsetfillcolor{currentfill}%
\pgfsetfillopacity{0.050000}%
\pgfsetlinewidth{0.000000pt}%
\definecolor{currentstroke}{rgb}{0.019608,0.556863,0.850980}%
\pgfsetstrokecolor{currentstroke}%
\pgfsetdash{}{0pt}%
\pgfpathmoveto{\pgfqpoint{3.045502in}{1.713490in}}%
\pgfpathlineto{\pgfqpoint{3.158250in}{1.660558in}}%
\pgfpathlineto{\pgfqpoint{3.045502in}{1.607626in}}%
\pgfpathlineto{\pgfqpoint{2.932755in}{1.660558in}}%
\pgfpathlineto{\pgfqpoint{3.045502in}{1.713490in}}%
\pgfpathclose%
\pgfusepath{fill}%
\end{pgfscope}%
\begin{pgfscope}%
\pgfpathrectangle{\pgfqpoint{0.765000in}{0.660000in}}{\pgfqpoint{4.620000in}{4.620000in}}%
\pgfusepath{clip}%
\pgfsetbuttcap%
\pgfsetroundjoin%
\definecolor{currentfill}{rgb}{0.019608,0.556863,0.850980}%
\pgfsetfillcolor{currentfill}%
\pgfsetfillopacity{0.050000}%
\pgfsetlinewidth{0.000000pt}%
\definecolor{currentstroke}{rgb}{0.019608,0.556863,0.850980}%
\pgfsetstrokecolor{currentstroke}%
\pgfsetdash{}{0pt}%
\pgfpathmoveto{\pgfqpoint{3.045502in}{1.819079in}}%
\pgfpathlineto{\pgfqpoint{3.158250in}{1.766147in}}%
\pgfpathlineto{\pgfqpoint{3.045502in}{1.713215in}}%
\pgfpathlineto{\pgfqpoint{2.932755in}{1.766147in}}%
\pgfpathlineto{\pgfqpoint{3.045502in}{1.819079in}}%
\pgfpathclose%
\pgfusepath{fill}%
\end{pgfscope}%
\begin{pgfscope}%
\pgfpathrectangle{\pgfqpoint{0.765000in}{0.660000in}}{\pgfqpoint{4.620000in}{4.620000in}}%
\pgfusepath{clip}%
\pgfsetbuttcap%
\pgfsetroundjoin%
\definecolor{currentfill}{rgb}{0.019608,0.556863,0.850980}%
\pgfsetfillcolor{currentfill}%
\pgfsetfillopacity{0.050000}%
\pgfsetlinewidth{0.000000pt}%
\definecolor{currentstroke}{rgb}{0.019608,0.556863,0.850980}%
\pgfsetstrokecolor{currentstroke}%
\pgfsetdash{}{0pt}%
\pgfpathmoveto{\pgfqpoint{2.932755in}{1.660558in}}%
\pgfpathlineto{\pgfqpoint{2.932755in}{1.766147in}}%
\pgfpathlineto{\pgfqpoint{3.045502in}{1.713215in}}%
\pgfpathlineto{\pgfqpoint{3.045502in}{1.607626in}}%
\pgfpathlineto{\pgfqpoint{2.932755in}{1.660558in}}%
\pgfpathclose%
\pgfusepath{fill}%
\end{pgfscope}%
\begin{pgfscope}%
\pgfpathrectangle{\pgfqpoint{0.765000in}{0.660000in}}{\pgfqpoint{4.620000in}{4.620000in}}%
\pgfusepath{clip}%
\pgfsetbuttcap%
\pgfsetroundjoin%
\definecolor{currentfill}{rgb}{0.019608,0.556863,0.850980}%
\pgfsetfillcolor{currentfill}%
\pgfsetfillopacity{0.050000}%
\pgfsetlinewidth{0.000000pt}%
\definecolor{currentstroke}{rgb}{0.019608,0.556863,0.850980}%
\pgfsetstrokecolor{currentstroke}%
\pgfsetdash{}{0pt}%
\pgfpathmoveto{\pgfqpoint{3.158250in}{1.660558in}}%
\pgfpathlineto{\pgfqpoint{3.158250in}{1.766147in}}%
\pgfpathlineto{\pgfqpoint{3.045502in}{1.713215in}}%
\pgfpathlineto{\pgfqpoint{3.045502in}{1.607626in}}%
\pgfpathlineto{\pgfqpoint{3.158250in}{1.660558in}}%
\pgfpathclose%
\pgfusepath{fill}%
\end{pgfscope}%
\begin{pgfscope}%
\pgfpathrectangle{\pgfqpoint{0.765000in}{0.660000in}}{\pgfqpoint{4.620000in}{4.620000in}}%
\pgfusepath{clip}%
\pgfsetbuttcap%
\pgfsetroundjoin%
\definecolor{currentfill}{rgb}{0.800000,0.176471,0.207843}%
\pgfsetfillcolor{currentfill}%
\pgfsetfillopacity{0.050000}%
\pgfsetlinewidth{0.000000pt}%
\definecolor{currentstroke}{rgb}{0.800000,0.176471,0.207843}%
\pgfsetstrokecolor{currentstroke}%
\pgfsetdash{}{0pt}%
\pgfpathmoveto{\pgfqpoint{3.551432in}{3.127628in}}%
\pgfpathlineto{\pgfqpoint{3.551432in}{3.233216in}}%
\pgfpathlineto{\pgfqpoint{3.664180in}{3.180285in}}%
\pgfpathlineto{\pgfqpoint{3.664180in}{3.074696in}}%
\pgfpathlineto{\pgfqpoint{3.551432in}{3.127628in}}%
\pgfpathclose%
\pgfusepath{fill}%
\end{pgfscope}%
\begin{pgfscope}%
\pgfpathrectangle{\pgfqpoint{0.765000in}{0.660000in}}{\pgfqpoint{4.620000in}{4.620000in}}%
\pgfusepath{clip}%
\pgfsetbuttcap%
\pgfsetroundjoin%
\definecolor{currentfill}{rgb}{0.800000,0.176471,0.207843}%
\pgfsetfillcolor{currentfill}%
\pgfsetfillopacity{0.050000}%
\pgfsetlinewidth{0.000000pt}%
\definecolor{currentstroke}{rgb}{0.800000,0.176471,0.207843}%
\pgfsetstrokecolor{currentstroke}%
\pgfsetdash{}{0pt}%
\pgfpathmoveto{\pgfqpoint{3.551432in}{3.127628in}}%
\pgfpathlineto{\pgfqpoint{3.551432in}{3.233216in}}%
\pgfpathlineto{\pgfqpoint{3.438685in}{3.180285in}}%
\pgfpathlineto{\pgfqpoint{3.438685in}{3.074696in}}%
\pgfpathlineto{\pgfqpoint{3.551432in}{3.127628in}}%
\pgfpathclose%
\pgfusepath{fill}%
\end{pgfscope}%
\begin{pgfscope}%
\pgfpathrectangle{\pgfqpoint{0.765000in}{0.660000in}}{\pgfqpoint{4.620000in}{4.620000in}}%
\pgfusepath{clip}%
\pgfsetbuttcap%
\pgfsetroundjoin%
\definecolor{currentfill}{rgb}{0.800000,0.176471,0.207843}%
\pgfsetfillcolor{currentfill}%
\pgfsetfillopacity{0.050000}%
\pgfsetlinewidth{0.000000pt}%
\definecolor{currentstroke}{rgb}{0.800000,0.176471,0.207843}%
\pgfsetstrokecolor{currentstroke}%
\pgfsetdash{}{0pt}%
\pgfpathmoveto{\pgfqpoint{3.551432in}{3.127628in}}%
\pgfpathlineto{\pgfqpoint{3.664180in}{3.074696in}}%
\pgfpathlineto{\pgfqpoint{3.551432in}{3.021764in}}%
\pgfpathlineto{\pgfqpoint{3.438685in}{3.074696in}}%
\pgfpathlineto{\pgfqpoint{3.551432in}{3.127628in}}%
\pgfpathclose%
\pgfusepath{fill}%
\end{pgfscope}%
\begin{pgfscope}%
\pgfpathrectangle{\pgfqpoint{0.765000in}{0.660000in}}{\pgfqpoint{4.620000in}{4.620000in}}%
\pgfusepath{clip}%
\pgfsetbuttcap%
\pgfsetroundjoin%
\definecolor{currentfill}{rgb}{0.800000,0.176471,0.207843}%
\pgfsetfillcolor{currentfill}%
\pgfsetfillopacity{0.050000}%
\pgfsetlinewidth{0.000000pt}%
\definecolor{currentstroke}{rgb}{0.800000,0.176471,0.207843}%
\pgfsetstrokecolor{currentstroke}%
\pgfsetdash{}{0pt}%
\pgfpathmoveto{\pgfqpoint{3.551432in}{3.233216in}}%
\pgfpathlineto{\pgfqpoint{3.664180in}{3.180285in}}%
\pgfpathlineto{\pgfqpoint{3.551432in}{3.127353in}}%
\pgfpathlineto{\pgfqpoint{3.438685in}{3.180285in}}%
\pgfpathlineto{\pgfqpoint{3.551432in}{3.233216in}}%
\pgfpathclose%
\pgfusepath{fill}%
\end{pgfscope}%
\begin{pgfscope}%
\pgfpathrectangle{\pgfqpoint{0.765000in}{0.660000in}}{\pgfqpoint{4.620000in}{4.620000in}}%
\pgfusepath{clip}%
\pgfsetbuttcap%
\pgfsetroundjoin%
\definecolor{currentfill}{rgb}{0.800000,0.176471,0.207843}%
\pgfsetfillcolor{currentfill}%
\pgfsetfillopacity{0.050000}%
\pgfsetlinewidth{0.000000pt}%
\definecolor{currentstroke}{rgb}{0.800000,0.176471,0.207843}%
\pgfsetstrokecolor{currentstroke}%
\pgfsetdash{}{0pt}%
\pgfpathmoveto{\pgfqpoint{3.438685in}{3.074696in}}%
\pgfpathlineto{\pgfqpoint{3.438685in}{3.180285in}}%
\pgfpathlineto{\pgfqpoint{3.551432in}{3.127353in}}%
\pgfpathlineto{\pgfqpoint{3.551432in}{3.021764in}}%
\pgfpathlineto{\pgfqpoint{3.438685in}{3.074696in}}%
\pgfpathclose%
\pgfusepath{fill}%
\end{pgfscope}%
\begin{pgfscope}%
\pgfpathrectangle{\pgfqpoint{0.765000in}{0.660000in}}{\pgfqpoint{4.620000in}{4.620000in}}%
\pgfusepath{clip}%
\pgfsetbuttcap%
\pgfsetroundjoin%
\definecolor{currentfill}{rgb}{0.800000,0.176471,0.207843}%
\pgfsetfillcolor{currentfill}%
\pgfsetfillopacity{0.050000}%
\pgfsetlinewidth{0.000000pt}%
\definecolor{currentstroke}{rgb}{0.800000,0.176471,0.207843}%
\pgfsetstrokecolor{currentstroke}%
\pgfsetdash{}{0pt}%
\pgfpathmoveto{\pgfqpoint{3.664180in}{3.074696in}}%
\pgfpathlineto{\pgfqpoint{3.664180in}{3.180285in}}%
\pgfpathlineto{\pgfqpoint{3.551432in}{3.127353in}}%
\pgfpathlineto{\pgfqpoint{3.551432in}{3.021764in}}%
\pgfpathlineto{\pgfqpoint{3.664180in}{3.074696in}}%
\pgfpathclose%
\pgfusepath{fill}%
\end{pgfscope}%
\begin{pgfscope}%
\pgfpathrectangle{\pgfqpoint{0.765000in}{0.660000in}}{\pgfqpoint{4.620000in}{4.620000in}}%
\pgfusepath{clip}%
\pgfsetbuttcap%
\pgfsetroundjoin%
\definecolor{currentfill}{rgb}{0.800000,0.176471,0.207843}%
\pgfsetfillcolor{currentfill}%
\pgfsetfillopacity{0.050000}%
\pgfsetlinewidth{0.000000pt}%
\definecolor{currentstroke}{rgb}{0.800000,0.176471,0.207843}%
\pgfsetstrokecolor{currentstroke}%
\pgfsetdash{}{0pt}%
\pgfpathmoveto{\pgfqpoint{3.362238in}{2.913390in}}%
\pgfpathlineto{\pgfqpoint{3.362238in}{3.018979in}}%
\pgfpathlineto{\pgfqpoint{3.474985in}{2.966047in}}%
\pgfpathlineto{\pgfqpoint{3.474985in}{2.860458in}}%
\pgfpathlineto{\pgfqpoint{3.362238in}{2.913390in}}%
\pgfpathclose%
\pgfusepath{fill}%
\end{pgfscope}%
\begin{pgfscope}%
\pgfpathrectangle{\pgfqpoint{0.765000in}{0.660000in}}{\pgfqpoint{4.620000in}{4.620000in}}%
\pgfusepath{clip}%
\pgfsetbuttcap%
\pgfsetroundjoin%
\definecolor{currentfill}{rgb}{0.800000,0.176471,0.207843}%
\pgfsetfillcolor{currentfill}%
\pgfsetfillopacity{0.050000}%
\pgfsetlinewidth{0.000000pt}%
\definecolor{currentstroke}{rgb}{0.800000,0.176471,0.207843}%
\pgfsetstrokecolor{currentstroke}%
\pgfsetdash{}{0pt}%
\pgfpathmoveto{\pgfqpoint{3.362238in}{2.913390in}}%
\pgfpathlineto{\pgfqpoint{3.362238in}{3.018979in}}%
\pgfpathlineto{\pgfqpoint{3.249490in}{2.966047in}}%
\pgfpathlineto{\pgfqpoint{3.249490in}{2.860458in}}%
\pgfpathlineto{\pgfqpoint{3.362238in}{2.913390in}}%
\pgfpathclose%
\pgfusepath{fill}%
\end{pgfscope}%
\begin{pgfscope}%
\pgfpathrectangle{\pgfqpoint{0.765000in}{0.660000in}}{\pgfqpoint{4.620000in}{4.620000in}}%
\pgfusepath{clip}%
\pgfsetbuttcap%
\pgfsetroundjoin%
\definecolor{currentfill}{rgb}{0.800000,0.176471,0.207843}%
\pgfsetfillcolor{currentfill}%
\pgfsetfillopacity{0.050000}%
\pgfsetlinewidth{0.000000pt}%
\definecolor{currentstroke}{rgb}{0.800000,0.176471,0.207843}%
\pgfsetstrokecolor{currentstroke}%
\pgfsetdash{}{0pt}%
\pgfpathmoveto{\pgfqpoint{3.362238in}{2.913390in}}%
\pgfpathlineto{\pgfqpoint{3.474985in}{2.860458in}}%
\pgfpathlineto{\pgfqpoint{3.362238in}{2.807527in}}%
\pgfpathlineto{\pgfqpoint{3.249490in}{2.860458in}}%
\pgfpathlineto{\pgfqpoint{3.362238in}{2.913390in}}%
\pgfpathclose%
\pgfusepath{fill}%
\end{pgfscope}%
\begin{pgfscope}%
\pgfpathrectangle{\pgfqpoint{0.765000in}{0.660000in}}{\pgfqpoint{4.620000in}{4.620000in}}%
\pgfusepath{clip}%
\pgfsetbuttcap%
\pgfsetroundjoin%
\definecolor{currentfill}{rgb}{0.800000,0.176471,0.207843}%
\pgfsetfillcolor{currentfill}%
\pgfsetfillopacity{0.050000}%
\pgfsetlinewidth{0.000000pt}%
\definecolor{currentstroke}{rgb}{0.800000,0.176471,0.207843}%
\pgfsetstrokecolor{currentstroke}%
\pgfsetdash{}{0pt}%
\pgfpathmoveto{\pgfqpoint{3.362238in}{3.018979in}}%
\pgfpathlineto{\pgfqpoint{3.474985in}{2.966047in}}%
\pgfpathlineto{\pgfqpoint{3.362238in}{2.913115in}}%
\pgfpathlineto{\pgfqpoint{3.249490in}{2.966047in}}%
\pgfpathlineto{\pgfqpoint{3.362238in}{3.018979in}}%
\pgfpathclose%
\pgfusepath{fill}%
\end{pgfscope}%
\begin{pgfscope}%
\pgfpathrectangle{\pgfqpoint{0.765000in}{0.660000in}}{\pgfqpoint{4.620000in}{4.620000in}}%
\pgfusepath{clip}%
\pgfsetbuttcap%
\pgfsetroundjoin%
\definecolor{currentfill}{rgb}{0.800000,0.176471,0.207843}%
\pgfsetfillcolor{currentfill}%
\pgfsetfillopacity{0.050000}%
\pgfsetlinewidth{0.000000pt}%
\definecolor{currentstroke}{rgb}{0.800000,0.176471,0.207843}%
\pgfsetstrokecolor{currentstroke}%
\pgfsetdash{}{0pt}%
\pgfpathmoveto{\pgfqpoint{3.249490in}{2.860458in}}%
\pgfpathlineto{\pgfqpoint{3.249490in}{2.966047in}}%
\pgfpathlineto{\pgfqpoint{3.362238in}{2.913115in}}%
\pgfpathlineto{\pgfqpoint{3.362238in}{2.807527in}}%
\pgfpathlineto{\pgfqpoint{3.249490in}{2.860458in}}%
\pgfpathclose%
\pgfusepath{fill}%
\end{pgfscope}%
\begin{pgfscope}%
\pgfpathrectangle{\pgfqpoint{0.765000in}{0.660000in}}{\pgfqpoint{4.620000in}{4.620000in}}%
\pgfusepath{clip}%
\pgfsetbuttcap%
\pgfsetroundjoin%
\definecolor{currentfill}{rgb}{0.800000,0.176471,0.207843}%
\pgfsetfillcolor{currentfill}%
\pgfsetfillopacity{0.050000}%
\pgfsetlinewidth{0.000000pt}%
\definecolor{currentstroke}{rgb}{0.800000,0.176471,0.207843}%
\pgfsetstrokecolor{currentstroke}%
\pgfsetdash{}{0pt}%
\pgfpathmoveto{\pgfqpoint{3.474985in}{2.860458in}}%
\pgfpathlineto{\pgfqpoint{3.474985in}{2.966047in}}%
\pgfpathlineto{\pgfqpoint{3.362238in}{2.913115in}}%
\pgfpathlineto{\pgfqpoint{3.362238in}{2.807527in}}%
\pgfpathlineto{\pgfqpoint{3.474985in}{2.860458in}}%
\pgfpathclose%
\pgfusepath{fill}%
\end{pgfscope}%
\begin{pgfscope}%
\pgfpathrectangle{\pgfqpoint{0.765000in}{0.660000in}}{\pgfqpoint{4.620000in}{4.620000in}}%
\pgfusepath{clip}%
\pgfsetbuttcap%
\pgfsetroundjoin%
\definecolor{currentfill}{rgb}{0.800000,0.176471,0.207843}%
\pgfsetfillcolor{currentfill}%
\pgfsetfillopacity{0.050000}%
\pgfsetlinewidth{0.000000pt}%
\definecolor{currentstroke}{rgb}{0.800000,0.176471,0.207843}%
\pgfsetstrokecolor{currentstroke}%
\pgfsetdash{}{0pt}%
\pgfpathmoveto{\pgfqpoint{2.801327in}{2.979373in}}%
\pgfpathlineto{\pgfqpoint{2.801327in}{3.084962in}}%
\pgfpathlineto{\pgfqpoint{2.914074in}{3.032030in}}%
\pgfpathlineto{\pgfqpoint{2.914074in}{2.926441in}}%
\pgfpathlineto{\pgfqpoint{2.801327in}{2.979373in}}%
\pgfpathclose%
\pgfusepath{fill}%
\end{pgfscope}%
\begin{pgfscope}%
\pgfpathrectangle{\pgfqpoint{0.765000in}{0.660000in}}{\pgfqpoint{4.620000in}{4.620000in}}%
\pgfusepath{clip}%
\pgfsetbuttcap%
\pgfsetroundjoin%
\definecolor{currentfill}{rgb}{0.800000,0.176471,0.207843}%
\pgfsetfillcolor{currentfill}%
\pgfsetfillopacity{0.050000}%
\pgfsetlinewidth{0.000000pt}%
\definecolor{currentstroke}{rgb}{0.800000,0.176471,0.207843}%
\pgfsetstrokecolor{currentstroke}%
\pgfsetdash{}{0pt}%
\pgfpathmoveto{\pgfqpoint{2.801327in}{2.979373in}}%
\pgfpathlineto{\pgfqpoint{2.801327in}{3.084962in}}%
\pgfpathlineto{\pgfqpoint{2.688580in}{3.032030in}}%
\pgfpathlineto{\pgfqpoint{2.688580in}{2.926441in}}%
\pgfpathlineto{\pgfqpoint{2.801327in}{2.979373in}}%
\pgfpathclose%
\pgfusepath{fill}%
\end{pgfscope}%
\begin{pgfscope}%
\pgfpathrectangle{\pgfqpoint{0.765000in}{0.660000in}}{\pgfqpoint{4.620000in}{4.620000in}}%
\pgfusepath{clip}%
\pgfsetbuttcap%
\pgfsetroundjoin%
\definecolor{currentfill}{rgb}{0.800000,0.176471,0.207843}%
\pgfsetfillcolor{currentfill}%
\pgfsetfillopacity{0.050000}%
\pgfsetlinewidth{0.000000pt}%
\definecolor{currentstroke}{rgb}{0.800000,0.176471,0.207843}%
\pgfsetstrokecolor{currentstroke}%
\pgfsetdash{}{0pt}%
\pgfpathmoveto{\pgfqpoint{2.801327in}{2.979373in}}%
\pgfpathlineto{\pgfqpoint{2.914074in}{2.926441in}}%
\pgfpathlineto{\pgfqpoint{2.801327in}{2.873510in}}%
\pgfpathlineto{\pgfqpoint{2.688580in}{2.926441in}}%
\pgfpathlineto{\pgfqpoint{2.801327in}{2.979373in}}%
\pgfpathclose%
\pgfusepath{fill}%
\end{pgfscope}%
\begin{pgfscope}%
\pgfpathrectangle{\pgfqpoint{0.765000in}{0.660000in}}{\pgfqpoint{4.620000in}{4.620000in}}%
\pgfusepath{clip}%
\pgfsetbuttcap%
\pgfsetroundjoin%
\definecolor{currentfill}{rgb}{0.800000,0.176471,0.207843}%
\pgfsetfillcolor{currentfill}%
\pgfsetfillopacity{0.050000}%
\pgfsetlinewidth{0.000000pt}%
\definecolor{currentstroke}{rgb}{0.800000,0.176471,0.207843}%
\pgfsetstrokecolor{currentstroke}%
\pgfsetdash{}{0pt}%
\pgfpathmoveto{\pgfqpoint{2.801327in}{3.084962in}}%
\pgfpathlineto{\pgfqpoint{2.914074in}{3.032030in}}%
\pgfpathlineto{\pgfqpoint{2.801327in}{2.979098in}}%
\pgfpathlineto{\pgfqpoint{2.688580in}{3.032030in}}%
\pgfpathlineto{\pgfqpoint{2.801327in}{3.084962in}}%
\pgfpathclose%
\pgfusepath{fill}%
\end{pgfscope}%
\begin{pgfscope}%
\pgfpathrectangle{\pgfqpoint{0.765000in}{0.660000in}}{\pgfqpoint{4.620000in}{4.620000in}}%
\pgfusepath{clip}%
\pgfsetbuttcap%
\pgfsetroundjoin%
\definecolor{currentfill}{rgb}{0.800000,0.176471,0.207843}%
\pgfsetfillcolor{currentfill}%
\pgfsetfillopacity{0.050000}%
\pgfsetlinewidth{0.000000pt}%
\definecolor{currentstroke}{rgb}{0.800000,0.176471,0.207843}%
\pgfsetstrokecolor{currentstroke}%
\pgfsetdash{}{0pt}%
\pgfpathmoveto{\pgfqpoint{2.688580in}{2.926441in}}%
\pgfpathlineto{\pgfqpoint{2.688580in}{3.032030in}}%
\pgfpathlineto{\pgfqpoint{2.801327in}{2.979098in}}%
\pgfpathlineto{\pgfqpoint{2.801327in}{2.873510in}}%
\pgfpathlineto{\pgfqpoint{2.688580in}{2.926441in}}%
\pgfpathclose%
\pgfusepath{fill}%
\end{pgfscope}%
\begin{pgfscope}%
\pgfpathrectangle{\pgfqpoint{0.765000in}{0.660000in}}{\pgfqpoint{4.620000in}{4.620000in}}%
\pgfusepath{clip}%
\pgfsetbuttcap%
\pgfsetroundjoin%
\definecolor{currentfill}{rgb}{0.800000,0.176471,0.207843}%
\pgfsetfillcolor{currentfill}%
\pgfsetfillopacity{0.050000}%
\pgfsetlinewidth{0.000000pt}%
\definecolor{currentstroke}{rgb}{0.800000,0.176471,0.207843}%
\pgfsetstrokecolor{currentstroke}%
\pgfsetdash{}{0pt}%
\pgfpathmoveto{\pgfqpoint{2.914074in}{2.926441in}}%
\pgfpathlineto{\pgfqpoint{2.914074in}{3.032030in}}%
\pgfpathlineto{\pgfqpoint{2.801327in}{2.979098in}}%
\pgfpathlineto{\pgfqpoint{2.801327in}{2.873510in}}%
\pgfpathlineto{\pgfqpoint{2.914074in}{2.926441in}}%
\pgfpathclose%
\pgfusepath{fill}%
\end{pgfscope}%
\begin{pgfscope}%
\pgfpathrectangle{\pgfqpoint{0.765000in}{0.660000in}}{\pgfqpoint{4.620000in}{4.620000in}}%
\pgfusepath{clip}%
\pgfsetbuttcap%
\pgfsetroundjoin%
\definecolor{currentfill}{rgb}{0.800000,0.176471,0.207843}%
\pgfsetfillcolor{currentfill}%
\pgfsetfillopacity{0.050000}%
\pgfsetlinewidth{0.000000pt}%
\definecolor{currentstroke}{rgb}{0.800000,0.176471,0.207843}%
\pgfsetstrokecolor{currentstroke}%
\pgfsetdash{}{0pt}%
\pgfpathmoveto{\pgfqpoint{3.089948in}{2.662350in}}%
\pgfpathlineto{\pgfqpoint{3.089948in}{2.767939in}}%
\pgfpathlineto{\pgfqpoint{3.202695in}{2.715007in}}%
\pgfpathlineto{\pgfqpoint{3.202695in}{2.609418in}}%
\pgfpathlineto{\pgfqpoint{3.089948in}{2.662350in}}%
\pgfpathclose%
\pgfusepath{fill}%
\end{pgfscope}%
\begin{pgfscope}%
\pgfpathrectangle{\pgfqpoint{0.765000in}{0.660000in}}{\pgfqpoint{4.620000in}{4.620000in}}%
\pgfusepath{clip}%
\pgfsetbuttcap%
\pgfsetroundjoin%
\definecolor{currentfill}{rgb}{0.800000,0.176471,0.207843}%
\pgfsetfillcolor{currentfill}%
\pgfsetfillopacity{0.050000}%
\pgfsetlinewidth{0.000000pt}%
\definecolor{currentstroke}{rgb}{0.800000,0.176471,0.207843}%
\pgfsetstrokecolor{currentstroke}%
\pgfsetdash{}{0pt}%
\pgfpathmoveto{\pgfqpoint{3.089948in}{2.662350in}}%
\pgfpathlineto{\pgfqpoint{3.089948in}{2.767939in}}%
\pgfpathlineto{\pgfqpoint{2.977200in}{2.715007in}}%
\pgfpathlineto{\pgfqpoint{2.977200in}{2.609418in}}%
\pgfpathlineto{\pgfqpoint{3.089948in}{2.662350in}}%
\pgfpathclose%
\pgfusepath{fill}%
\end{pgfscope}%
\begin{pgfscope}%
\pgfpathrectangle{\pgfqpoint{0.765000in}{0.660000in}}{\pgfqpoint{4.620000in}{4.620000in}}%
\pgfusepath{clip}%
\pgfsetbuttcap%
\pgfsetroundjoin%
\definecolor{currentfill}{rgb}{0.800000,0.176471,0.207843}%
\pgfsetfillcolor{currentfill}%
\pgfsetfillopacity{0.050000}%
\pgfsetlinewidth{0.000000pt}%
\definecolor{currentstroke}{rgb}{0.800000,0.176471,0.207843}%
\pgfsetstrokecolor{currentstroke}%
\pgfsetdash{}{0pt}%
\pgfpathmoveto{\pgfqpoint{3.089948in}{2.662350in}}%
\pgfpathlineto{\pgfqpoint{3.202695in}{2.609418in}}%
\pgfpathlineto{\pgfqpoint{3.089948in}{2.556486in}}%
\pgfpathlineto{\pgfqpoint{2.977200in}{2.609418in}}%
\pgfpathlineto{\pgfqpoint{3.089948in}{2.662350in}}%
\pgfpathclose%
\pgfusepath{fill}%
\end{pgfscope}%
\begin{pgfscope}%
\pgfpathrectangle{\pgfqpoint{0.765000in}{0.660000in}}{\pgfqpoint{4.620000in}{4.620000in}}%
\pgfusepath{clip}%
\pgfsetbuttcap%
\pgfsetroundjoin%
\definecolor{currentfill}{rgb}{0.800000,0.176471,0.207843}%
\pgfsetfillcolor{currentfill}%
\pgfsetfillopacity{0.050000}%
\pgfsetlinewidth{0.000000pt}%
\definecolor{currentstroke}{rgb}{0.800000,0.176471,0.207843}%
\pgfsetstrokecolor{currentstroke}%
\pgfsetdash{}{0pt}%
\pgfpathmoveto{\pgfqpoint{3.089948in}{2.767939in}}%
\pgfpathlineto{\pgfqpoint{3.202695in}{2.715007in}}%
\pgfpathlineto{\pgfqpoint{3.089948in}{2.662075in}}%
\pgfpathlineto{\pgfqpoint{2.977200in}{2.715007in}}%
\pgfpathlineto{\pgfqpoint{3.089948in}{2.767939in}}%
\pgfpathclose%
\pgfusepath{fill}%
\end{pgfscope}%
\begin{pgfscope}%
\pgfpathrectangle{\pgfqpoint{0.765000in}{0.660000in}}{\pgfqpoint{4.620000in}{4.620000in}}%
\pgfusepath{clip}%
\pgfsetbuttcap%
\pgfsetroundjoin%
\definecolor{currentfill}{rgb}{0.800000,0.176471,0.207843}%
\pgfsetfillcolor{currentfill}%
\pgfsetfillopacity{0.050000}%
\pgfsetlinewidth{0.000000pt}%
\definecolor{currentstroke}{rgb}{0.800000,0.176471,0.207843}%
\pgfsetstrokecolor{currentstroke}%
\pgfsetdash{}{0pt}%
\pgfpathmoveto{\pgfqpoint{2.977200in}{2.609418in}}%
\pgfpathlineto{\pgfqpoint{2.977200in}{2.715007in}}%
\pgfpathlineto{\pgfqpoint{3.089948in}{2.662075in}}%
\pgfpathlineto{\pgfqpoint{3.089948in}{2.556486in}}%
\pgfpathlineto{\pgfqpoint{2.977200in}{2.609418in}}%
\pgfpathclose%
\pgfusepath{fill}%
\end{pgfscope}%
\begin{pgfscope}%
\pgfpathrectangle{\pgfqpoint{0.765000in}{0.660000in}}{\pgfqpoint{4.620000in}{4.620000in}}%
\pgfusepath{clip}%
\pgfsetbuttcap%
\pgfsetroundjoin%
\definecolor{currentfill}{rgb}{0.800000,0.176471,0.207843}%
\pgfsetfillcolor{currentfill}%
\pgfsetfillopacity{0.050000}%
\pgfsetlinewidth{0.000000pt}%
\definecolor{currentstroke}{rgb}{0.800000,0.176471,0.207843}%
\pgfsetstrokecolor{currentstroke}%
\pgfsetdash{}{0pt}%
\pgfpathmoveto{\pgfqpoint{3.202695in}{2.609418in}}%
\pgfpathlineto{\pgfqpoint{3.202695in}{2.715007in}}%
\pgfpathlineto{\pgfqpoint{3.089948in}{2.662075in}}%
\pgfpathlineto{\pgfqpoint{3.089948in}{2.556486in}}%
\pgfpathlineto{\pgfqpoint{3.202695in}{2.609418in}}%
\pgfpathclose%
\pgfusepath{fill}%
\end{pgfscope}%
\begin{pgfscope}%
\pgfpathrectangle{\pgfqpoint{0.765000in}{0.660000in}}{\pgfqpoint{4.620000in}{4.620000in}}%
\pgfusepath{clip}%
\pgfsetbuttcap%
\pgfsetroundjoin%
\definecolor{currentfill}{rgb}{0.800000,0.176471,0.207843}%
\pgfsetfillcolor{currentfill}%
\pgfsetfillopacity{0.050000}%
\pgfsetlinewidth{0.000000pt}%
\definecolor{currentstroke}{rgb}{0.800000,0.176471,0.207843}%
\pgfsetstrokecolor{currentstroke}%
\pgfsetdash{}{0pt}%
\pgfpathmoveto{\pgfqpoint{2.909201in}{2.967220in}}%
\pgfpathlineto{\pgfqpoint{2.909201in}{3.072809in}}%
\pgfpathlineto{\pgfqpoint{3.021948in}{3.019877in}}%
\pgfpathlineto{\pgfqpoint{3.021948in}{2.914288in}}%
\pgfpathlineto{\pgfqpoint{2.909201in}{2.967220in}}%
\pgfpathclose%
\pgfusepath{fill}%
\end{pgfscope}%
\begin{pgfscope}%
\pgfpathrectangle{\pgfqpoint{0.765000in}{0.660000in}}{\pgfqpoint{4.620000in}{4.620000in}}%
\pgfusepath{clip}%
\pgfsetbuttcap%
\pgfsetroundjoin%
\definecolor{currentfill}{rgb}{0.800000,0.176471,0.207843}%
\pgfsetfillcolor{currentfill}%
\pgfsetfillopacity{0.050000}%
\pgfsetlinewidth{0.000000pt}%
\definecolor{currentstroke}{rgb}{0.800000,0.176471,0.207843}%
\pgfsetstrokecolor{currentstroke}%
\pgfsetdash{}{0pt}%
\pgfpathmoveto{\pgfqpoint{2.909201in}{2.967220in}}%
\pgfpathlineto{\pgfqpoint{2.909201in}{3.072809in}}%
\pgfpathlineto{\pgfqpoint{2.796454in}{3.019877in}}%
\pgfpathlineto{\pgfqpoint{2.796454in}{2.914288in}}%
\pgfpathlineto{\pgfqpoint{2.909201in}{2.967220in}}%
\pgfpathclose%
\pgfusepath{fill}%
\end{pgfscope}%
\begin{pgfscope}%
\pgfpathrectangle{\pgfqpoint{0.765000in}{0.660000in}}{\pgfqpoint{4.620000in}{4.620000in}}%
\pgfusepath{clip}%
\pgfsetbuttcap%
\pgfsetroundjoin%
\definecolor{currentfill}{rgb}{0.800000,0.176471,0.207843}%
\pgfsetfillcolor{currentfill}%
\pgfsetfillopacity{0.050000}%
\pgfsetlinewidth{0.000000pt}%
\definecolor{currentstroke}{rgb}{0.800000,0.176471,0.207843}%
\pgfsetstrokecolor{currentstroke}%
\pgfsetdash{}{0pt}%
\pgfpathmoveto{\pgfqpoint{2.909201in}{2.967220in}}%
\pgfpathlineto{\pgfqpoint{3.021948in}{2.914288in}}%
\pgfpathlineto{\pgfqpoint{2.909201in}{2.861357in}}%
\pgfpathlineto{\pgfqpoint{2.796454in}{2.914288in}}%
\pgfpathlineto{\pgfqpoint{2.909201in}{2.967220in}}%
\pgfpathclose%
\pgfusepath{fill}%
\end{pgfscope}%
\begin{pgfscope}%
\pgfpathrectangle{\pgfqpoint{0.765000in}{0.660000in}}{\pgfqpoint{4.620000in}{4.620000in}}%
\pgfusepath{clip}%
\pgfsetbuttcap%
\pgfsetroundjoin%
\definecolor{currentfill}{rgb}{0.800000,0.176471,0.207843}%
\pgfsetfillcolor{currentfill}%
\pgfsetfillopacity{0.050000}%
\pgfsetlinewidth{0.000000pt}%
\definecolor{currentstroke}{rgb}{0.800000,0.176471,0.207843}%
\pgfsetstrokecolor{currentstroke}%
\pgfsetdash{}{0pt}%
\pgfpathmoveto{\pgfqpoint{2.909201in}{3.072809in}}%
\pgfpathlineto{\pgfqpoint{3.021948in}{3.019877in}}%
\pgfpathlineto{\pgfqpoint{2.909201in}{2.966945in}}%
\pgfpathlineto{\pgfqpoint{2.796454in}{3.019877in}}%
\pgfpathlineto{\pgfqpoint{2.909201in}{3.072809in}}%
\pgfpathclose%
\pgfusepath{fill}%
\end{pgfscope}%
\begin{pgfscope}%
\pgfpathrectangle{\pgfqpoint{0.765000in}{0.660000in}}{\pgfqpoint{4.620000in}{4.620000in}}%
\pgfusepath{clip}%
\pgfsetbuttcap%
\pgfsetroundjoin%
\definecolor{currentfill}{rgb}{0.800000,0.176471,0.207843}%
\pgfsetfillcolor{currentfill}%
\pgfsetfillopacity{0.050000}%
\pgfsetlinewidth{0.000000pt}%
\definecolor{currentstroke}{rgb}{0.800000,0.176471,0.207843}%
\pgfsetstrokecolor{currentstroke}%
\pgfsetdash{}{0pt}%
\pgfpathmoveto{\pgfqpoint{2.796454in}{2.914288in}}%
\pgfpathlineto{\pgfqpoint{2.796454in}{3.019877in}}%
\pgfpathlineto{\pgfqpoint{2.909201in}{2.966945in}}%
\pgfpathlineto{\pgfqpoint{2.909201in}{2.861357in}}%
\pgfpathlineto{\pgfqpoint{2.796454in}{2.914288in}}%
\pgfpathclose%
\pgfusepath{fill}%
\end{pgfscope}%
\begin{pgfscope}%
\pgfpathrectangle{\pgfqpoint{0.765000in}{0.660000in}}{\pgfqpoint{4.620000in}{4.620000in}}%
\pgfusepath{clip}%
\pgfsetbuttcap%
\pgfsetroundjoin%
\definecolor{currentfill}{rgb}{0.800000,0.176471,0.207843}%
\pgfsetfillcolor{currentfill}%
\pgfsetfillopacity{0.050000}%
\pgfsetlinewidth{0.000000pt}%
\definecolor{currentstroke}{rgb}{0.800000,0.176471,0.207843}%
\pgfsetstrokecolor{currentstroke}%
\pgfsetdash{}{0pt}%
\pgfpathmoveto{\pgfqpoint{3.021948in}{2.914288in}}%
\pgfpathlineto{\pgfqpoint{3.021948in}{3.019877in}}%
\pgfpathlineto{\pgfqpoint{2.909201in}{2.966945in}}%
\pgfpathlineto{\pgfqpoint{2.909201in}{2.861357in}}%
\pgfpathlineto{\pgfqpoint{3.021948in}{2.914288in}}%
\pgfpathclose%
\pgfusepath{fill}%
\end{pgfscope}%
\begin{pgfscope}%
\pgfpathrectangle{\pgfqpoint{0.765000in}{0.660000in}}{\pgfqpoint{4.620000in}{4.620000in}}%
\pgfusepath{clip}%
\pgfsetbuttcap%
\pgfsetroundjoin%
\definecolor{currentfill}{rgb}{0.800000,0.176471,0.207843}%
\pgfsetfillcolor{currentfill}%
\pgfsetfillopacity{0.050000}%
\pgfsetlinewidth{0.000000pt}%
\definecolor{currentstroke}{rgb}{0.800000,0.176471,0.207843}%
\pgfsetstrokecolor{currentstroke}%
\pgfsetdash{}{0pt}%
\pgfpathmoveto{\pgfqpoint{2.947475in}{3.368644in}}%
\pgfpathlineto{\pgfqpoint{2.947475in}{3.474233in}}%
\pgfpathlineto{\pgfqpoint{3.060223in}{3.421301in}}%
\pgfpathlineto{\pgfqpoint{3.060223in}{3.315712in}}%
\pgfpathlineto{\pgfqpoint{2.947475in}{3.368644in}}%
\pgfpathclose%
\pgfusepath{fill}%
\end{pgfscope}%
\begin{pgfscope}%
\pgfpathrectangle{\pgfqpoint{0.765000in}{0.660000in}}{\pgfqpoint{4.620000in}{4.620000in}}%
\pgfusepath{clip}%
\pgfsetbuttcap%
\pgfsetroundjoin%
\definecolor{currentfill}{rgb}{0.800000,0.176471,0.207843}%
\pgfsetfillcolor{currentfill}%
\pgfsetfillopacity{0.050000}%
\pgfsetlinewidth{0.000000pt}%
\definecolor{currentstroke}{rgb}{0.800000,0.176471,0.207843}%
\pgfsetstrokecolor{currentstroke}%
\pgfsetdash{}{0pt}%
\pgfpathmoveto{\pgfqpoint{2.947475in}{3.368644in}}%
\pgfpathlineto{\pgfqpoint{2.947475in}{3.474233in}}%
\pgfpathlineto{\pgfqpoint{2.834728in}{3.421301in}}%
\pgfpathlineto{\pgfqpoint{2.834728in}{3.315712in}}%
\pgfpathlineto{\pgfqpoint{2.947475in}{3.368644in}}%
\pgfpathclose%
\pgfusepath{fill}%
\end{pgfscope}%
\begin{pgfscope}%
\pgfpathrectangle{\pgfqpoint{0.765000in}{0.660000in}}{\pgfqpoint{4.620000in}{4.620000in}}%
\pgfusepath{clip}%
\pgfsetbuttcap%
\pgfsetroundjoin%
\definecolor{currentfill}{rgb}{0.800000,0.176471,0.207843}%
\pgfsetfillcolor{currentfill}%
\pgfsetfillopacity{0.050000}%
\pgfsetlinewidth{0.000000pt}%
\definecolor{currentstroke}{rgb}{0.800000,0.176471,0.207843}%
\pgfsetstrokecolor{currentstroke}%
\pgfsetdash{}{0pt}%
\pgfpathmoveto{\pgfqpoint{2.947475in}{3.368644in}}%
\pgfpathlineto{\pgfqpoint{3.060223in}{3.315712in}}%
\pgfpathlineto{\pgfqpoint{2.947475in}{3.262780in}}%
\pgfpathlineto{\pgfqpoint{2.834728in}{3.315712in}}%
\pgfpathlineto{\pgfqpoint{2.947475in}{3.368644in}}%
\pgfpathclose%
\pgfusepath{fill}%
\end{pgfscope}%
\begin{pgfscope}%
\pgfpathrectangle{\pgfqpoint{0.765000in}{0.660000in}}{\pgfqpoint{4.620000in}{4.620000in}}%
\pgfusepath{clip}%
\pgfsetbuttcap%
\pgfsetroundjoin%
\definecolor{currentfill}{rgb}{0.800000,0.176471,0.207843}%
\pgfsetfillcolor{currentfill}%
\pgfsetfillopacity{0.050000}%
\pgfsetlinewidth{0.000000pt}%
\definecolor{currentstroke}{rgb}{0.800000,0.176471,0.207843}%
\pgfsetstrokecolor{currentstroke}%
\pgfsetdash{}{0pt}%
\pgfpathmoveto{\pgfqpoint{2.947475in}{3.474233in}}%
\pgfpathlineto{\pgfqpoint{3.060223in}{3.421301in}}%
\pgfpathlineto{\pgfqpoint{2.947475in}{3.368369in}}%
\pgfpathlineto{\pgfqpoint{2.834728in}{3.421301in}}%
\pgfpathlineto{\pgfqpoint{2.947475in}{3.474233in}}%
\pgfpathclose%
\pgfusepath{fill}%
\end{pgfscope}%
\begin{pgfscope}%
\pgfpathrectangle{\pgfqpoint{0.765000in}{0.660000in}}{\pgfqpoint{4.620000in}{4.620000in}}%
\pgfusepath{clip}%
\pgfsetbuttcap%
\pgfsetroundjoin%
\definecolor{currentfill}{rgb}{0.800000,0.176471,0.207843}%
\pgfsetfillcolor{currentfill}%
\pgfsetfillopacity{0.050000}%
\pgfsetlinewidth{0.000000pt}%
\definecolor{currentstroke}{rgb}{0.800000,0.176471,0.207843}%
\pgfsetstrokecolor{currentstroke}%
\pgfsetdash{}{0pt}%
\pgfpathmoveto{\pgfqpoint{2.834728in}{3.315712in}}%
\pgfpathlineto{\pgfqpoint{2.834728in}{3.421301in}}%
\pgfpathlineto{\pgfqpoint{2.947475in}{3.368369in}}%
\pgfpathlineto{\pgfqpoint{2.947475in}{3.262780in}}%
\pgfpathlineto{\pgfqpoint{2.834728in}{3.315712in}}%
\pgfpathclose%
\pgfusepath{fill}%
\end{pgfscope}%
\begin{pgfscope}%
\pgfpathrectangle{\pgfqpoint{0.765000in}{0.660000in}}{\pgfqpoint{4.620000in}{4.620000in}}%
\pgfusepath{clip}%
\pgfsetbuttcap%
\pgfsetroundjoin%
\definecolor{currentfill}{rgb}{0.800000,0.176471,0.207843}%
\pgfsetfillcolor{currentfill}%
\pgfsetfillopacity{0.050000}%
\pgfsetlinewidth{0.000000pt}%
\definecolor{currentstroke}{rgb}{0.800000,0.176471,0.207843}%
\pgfsetstrokecolor{currentstroke}%
\pgfsetdash{}{0pt}%
\pgfpathmoveto{\pgfqpoint{3.060223in}{3.315712in}}%
\pgfpathlineto{\pgfqpoint{3.060223in}{3.421301in}}%
\pgfpathlineto{\pgfqpoint{2.947475in}{3.368369in}}%
\pgfpathlineto{\pgfqpoint{2.947475in}{3.262780in}}%
\pgfpathlineto{\pgfqpoint{3.060223in}{3.315712in}}%
\pgfpathclose%
\pgfusepath{fill}%
\end{pgfscope}%
\begin{pgfscope}%
\pgfpathrectangle{\pgfqpoint{0.765000in}{0.660000in}}{\pgfqpoint{4.620000in}{4.620000in}}%
\pgfusepath{clip}%
\pgfsetbuttcap%
\pgfsetroundjoin%
\definecolor{currentfill}{rgb}{0.800000,0.176471,0.207843}%
\pgfsetfillcolor{currentfill}%
\pgfsetfillopacity{0.050000}%
\pgfsetlinewidth{0.000000pt}%
\definecolor{currentstroke}{rgb}{0.800000,0.176471,0.207843}%
\pgfsetstrokecolor{currentstroke}%
\pgfsetdash{}{0pt}%
\pgfpathmoveto{\pgfqpoint{2.899889in}{2.708204in}}%
\pgfpathlineto{\pgfqpoint{2.899889in}{2.813793in}}%
\pgfpathlineto{\pgfqpoint{3.012636in}{2.760861in}}%
\pgfpathlineto{\pgfqpoint{3.012636in}{2.655272in}}%
\pgfpathlineto{\pgfqpoint{2.899889in}{2.708204in}}%
\pgfpathclose%
\pgfusepath{fill}%
\end{pgfscope}%
\begin{pgfscope}%
\pgfpathrectangle{\pgfqpoint{0.765000in}{0.660000in}}{\pgfqpoint{4.620000in}{4.620000in}}%
\pgfusepath{clip}%
\pgfsetbuttcap%
\pgfsetroundjoin%
\definecolor{currentfill}{rgb}{0.800000,0.176471,0.207843}%
\pgfsetfillcolor{currentfill}%
\pgfsetfillopacity{0.050000}%
\pgfsetlinewidth{0.000000pt}%
\definecolor{currentstroke}{rgb}{0.800000,0.176471,0.207843}%
\pgfsetstrokecolor{currentstroke}%
\pgfsetdash{}{0pt}%
\pgfpathmoveto{\pgfqpoint{2.899889in}{2.708204in}}%
\pgfpathlineto{\pgfqpoint{2.899889in}{2.813793in}}%
\pgfpathlineto{\pgfqpoint{2.787141in}{2.760861in}}%
\pgfpathlineto{\pgfqpoint{2.787141in}{2.655272in}}%
\pgfpathlineto{\pgfqpoint{2.899889in}{2.708204in}}%
\pgfpathclose%
\pgfusepath{fill}%
\end{pgfscope}%
\begin{pgfscope}%
\pgfpathrectangle{\pgfqpoint{0.765000in}{0.660000in}}{\pgfqpoint{4.620000in}{4.620000in}}%
\pgfusepath{clip}%
\pgfsetbuttcap%
\pgfsetroundjoin%
\definecolor{currentfill}{rgb}{0.800000,0.176471,0.207843}%
\pgfsetfillcolor{currentfill}%
\pgfsetfillopacity{0.050000}%
\pgfsetlinewidth{0.000000pt}%
\definecolor{currentstroke}{rgb}{0.800000,0.176471,0.207843}%
\pgfsetstrokecolor{currentstroke}%
\pgfsetdash{}{0pt}%
\pgfpathmoveto{\pgfqpoint{2.899889in}{2.708204in}}%
\pgfpathlineto{\pgfqpoint{3.012636in}{2.655272in}}%
\pgfpathlineto{\pgfqpoint{2.899889in}{2.602341in}}%
\pgfpathlineto{\pgfqpoint{2.787141in}{2.655272in}}%
\pgfpathlineto{\pgfqpoint{2.899889in}{2.708204in}}%
\pgfpathclose%
\pgfusepath{fill}%
\end{pgfscope}%
\begin{pgfscope}%
\pgfpathrectangle{\pgfqpoint{0.765000in}{0.660000in}}{\pgfqpoint{4.620000in}{4.620000in}}%
\pgfusepath{clip}%
\pgfsetbuttcap%
\pgfsetroundjoin%
\definecolor{currentfill}{rgb}{0.800000,0.176471,0.207843}%
\pgfsetfillcolor{currentfill}%
\pgfsetfillopacity{0.050000}%
\pgfsetlinewidth{0.000000pt}%
\definecolor{currentstroke}{rgb}{0.800000,0.176471,0.207843}%
\pgfsetstrokecolor{currentstroke}%
\pgfsetdash{}{0pt}%
\pgfpathmoveto{\pgfqpoint{2.899889in}{2.813793in}}%
\pgfpathlineto{\pgfqpoint{3.012636in}{2.760861in}}%
\pgfpathlineto{\pgfqpoint{2.899889in}{2.707929in}}%
\pgfpathlineto{\pgfqpoint{2.787141in}{2.760861in}}%
\pgfpathlineto{\pgfqpoint{2.899889in}{2.813793in}}%
\pgfpathclose%
\pgfusepath{fill}%
\end{pgfscope}%
\begin{pgfscope}%
\pgfpathrectangle{\pgfqpoint{0.765000in}{0.660000in}}{\pgfqpoint{4.620000in}{4.620000in}}%
\pgfusepath{clip}%
\pgfsetbuttcap%
\pgfsetroundjoin%
\definecolor{currentfill}{rgb}{0.800000,0.176471,0.207843}%
\pgfsetfillcolor{currentfill}%
\pgfsetfillopacity{0.050000}%
\pgfsetlinewidth{0.000000pt}%
\definecolor{currentstroke}{rgb}{0.800000,0.176471,0.207843}%
\pgfsetstrokecolor{currentstroke}%
\pgfsetdash{}{0pt}%
\pgfpathmoveto{\pgfqpoint{2.787141in}{2.655272in}}%
\pgfpathlineto{\pgfqpoint{2.787141in}{2.760861in}}%
\pgfpathlineto{\pgfqpoint{2.899889in}{2.707929in}}%
\pgfpathlineto{\pgfqpoint{2.899889in}{2.602341in}}%
\pgfpathlineto{\pgfqpoint{2.787141in}{2.655272in}}%
\pgfpathclose%
\pgfusepath{fill}%
\end{pgfscope}%
\begin{pgfscope}%
\pgfpathrectangle{\pgfqpoint{0.765000in}{0.660000in}}{\pgfqpoint{4.620000in}{4.620000in}}%
\pgfusepath{clip}%
\pgfsetbuttcap%
\pgfsetroundjoin%
\definecolor{currentfill}{rgb}{0.800000,0.176471,0.207843}%
\pgfsetfillcolor{currentfill}%
\pgfsetfillopacity{0.050000}%
\pgfsetlinewidth{0.000000pt}%
\definecolor{currentstroke}{rgb}{0.800000,0.176471,0.207843}%
\pgfsetstrokecolor{currentstroke}%
\pgfsetdash{}{0pt}%
\pgfpathmoveto{\pgfqpoint{3.012636in}{2.655272in}}%
\pgfpathlineto{\pgfqpoint{3.012636in}{2.760861in}}%
\pgfpathlineto{\pgfqpoint{2.899889in}{2.707929in}}%
\pgfpathlineto{\pgfqpoint{2.899889in}{2.602341in}}%
\pgfpathlineto{\pgfqpoint{3.012636in}{2.655272in}}%
\pgfpathclose%
\pgfusepath{fill}%
\end{pgfscope}%
\begin{pgfscope}%
\pgfpathrectangle{\pgfqpoint{0.765000in}{0.660000in}}{\pgfqpoint{4.620000in}{4.620000in}}%
\pgfusepath{clip}%
\pgfsetbuttcap%
\pgfsetroundjoin%
\definecolor{currentfill}{rgb}{0.800000,0.176471,0.207843}%
\pgfsetfillcolor{currentfill}%
\pgfsetfillopacity{0.050000}%
\pgfsetlinewidth{0.000000pt}%
\definecolor{currentstroke}{rgb}{0.800000,0.176471,0.207843}%
\pgfsetstrokecolor{currentstroke}%
\pgfsetdash{}{0pt}%
\pgfpathmoveto{\pgfqpoint{3.158413in}{3.132643in}}%
\pgfpathlineto{\pgfqpoint{3.158413in}{3.238232in}}%
\pgfpathlineto{\pgfqpoint{3.271160in}{3.185300in}}%
\pgfpathlineto{\pgfqpoint{3.271160in}{3.079711in}}%
\pgfpathlineto{\pgfqpoint{3.158413in}{3.132643in}}%
\pgfpathclose%
\pgfusepath{fill}%
\end{pgfscope}%
\begin{pgfscope}%
\pgfpathrectangle{\pgfqpoint{0.765000in}{0.660000in}}{\pgfqpoint{4.620000in}{4.620000in}}%
\pgfusepath{clip}%
\pgfsetbuttcap%
\pgfsetroundjoin%
\definecolor{currentfill}{rgb}{0.800000,0.176471,0.207843}%
\pgfsetfillcolor{currentfill}%
\pgfsetfillopacity{0.050000}%
\pgfsetlinewidth{0.000000pt}%
\definecolor{currentstroke}{rgb}{0.800000,0.176471,0.207843}%
\pgfsetstrokecolor{currentstroke}%
\pgfsetdash{}{0pt}%
\pgfpathmoveto{\pgfqpoint{3.158413in}{3.132643in}}%
\pgfpathlineto{\pgfqpoint{3.158413in}{3.238232in}}%
\pgfpathlineto{\pgfqpoint{3.045665in}{3.185300in}}%
\pgfpathlineto{\pgfqpoint{3.045665in}{3.079711in}}%
\pgfpathlineto{\pgfqpoint{3.158413in}{3.132643in}}%
\pgfpathclose%
\pgfusepath{fill}%
\end{pgfscope}%
\begin{pgfscope}%
\pgfpathrectangle{\pgfqpoint{0.765000in}{0.660000in}}{\pgfqpoint{4.620000in}{4.620000in}}%
\pgfusepath{clip}%
\pgfsetbuttcap%
\pgfsetroundjoin%
\definecolor{currentfill}{rgb}{0.800000,0.176471,0.207843}%
\pgfsetfillcolor{currentfill}%
\pgfsetfillopacity{0.050000}%
\pgfsetlinewidth{0.000000pt}%
\definecolor{currentstroke}{rgb}{0.800000,0.176471,0.207843}%
\pgfsetstrokecolor{currentstroke}%
\pgfsetdash{}{0pt}%
\pgfpathmoveto{\pgfqpoint{3.158413in}{3.132643in}}%
\pgfpathlineto{\pgfqpoint{3.271160in}{3.079711in}}%
\pgfpathlineto{\pgfqpoint{3.158413in}{3.026780in}}%
\pgfpathlineto{\pgfqpoint{3.045665in}{3.079711in}}%
\pgfpathlineto{\pgfqpoint{3.158413in}{3.132643in}}%
\pgfpathclose%
\pgfusepath{fill}%
\end{pgfscope}%
\begin{pgfscope}%
\pgfpathrectangle{\pgfqpoint{0.765000in}{0.660000in}}{\pgfqpoint{4.620000in}{4.620000in}}%
\pgfusepath{clip}%
\pgfsetbuttcap%
\pgfsetroundjoin%
\definecolor{currentfill}{rgb}{0.800000,0.176471,0.207843}%
\pgfsetfillcolor{currentfill}%
\pgfsetfillopacity{0.050000}%
\pgfsetlinewidth{0.000000pt}%
\definecolor{currentstroke}{rgb}{0.800000,0.176471,0.207843}%
\pgfsetstrokecolor{currentstroke}%
\pgfsetdash{}{0pt}%
\pgfpathmoveto{\pgfqpoint{3.158413in}{3.238232in}}%
\pgfpathlineto{\pgfqpoint{3.271160in}{3.185300in}}%
\pgfpathlineto{\pgfqpoint{3.158413in}{3.132368in}}%
\pgfpathlineto{\pgfqpoint{3.045665in}{3.185300in}}%
\pgfpathlineto{\pgfqpoint{3.158413in}{3.238232in}}%
\pgfpathclose%
\pgfusepath{fill}%
\end{pgfscope}%
\begin{pgfscope}%
\pgfpathrectangle{\pgfqpoint{0.765000in}{0.660000in}}{\pgfqpoint{4.620000in}{4.620000in}}%
\pgfusepath{clip}%
\pgfsetbuttcap%
\pgfsetroundjoin%
\definecolor{currentfill}{rgb}{0.800000,0.176471,0.207843}%
\pgfsetfillcolor{currentfill}%
\pgfsetfillopacity{0.050000}%
\pgfsetlinewidth{0.000000pt}%
\definecolor{currentstroke}{rgb}{0.800000,0.176471,0.207843}%
\pgfsetstrokecolor{currentstroke}%
\pgfsetdash{}{0pt}%
\pgfpathmoveto{\pgfqpoint{3.045665in}{3.079711in}}%
\pgfpathlineto{\pgfqpoint{3.045665in}{3.185300in}}%
\pgfpathlineto{\pgfqpoint{3.158413in}{3.132368in}}%
\pgfpathlineto{\pgfqpoint{3.158413in}{3.026780in}}%
\pgfpathlineto{\pgfqpoint{3.045665in}{3.079711in}}%
\pgfpathclose%
\pgfusepath{fill}%
\end{pgfscope}%
\begin{pgfscope}%
\pgfpathrectangle{\pgfqpoint{0.765000in}{0.660000in}}{\pgfqpoint{4.620000in}{4.620000in}}%
\pgfusepath{clip}%
\pgfsetbuttcap%
\pgfsetroundjoin%
\definecolor{currentfill}{rgb}{0.800000,0.176471,0.207843}%
\pgfsetfillcolor{currentfill}%
\pgfsetfillopacity{0.050000}%
\pgfsetlinewidth{0.000000pt}%
\definecolor{currentstroke}{rgb}{0.800000,0.176471,0.207843}%
\pgfsetstrokecolor{currentstroke}%
\pgfsetdash{}{0pt}%
\pgfpathmoveto{\pgfqpoint{3.271160in}{3.079711in}}%
\pgfpathlineto{\pgfqpoint{3.271160in}{3.185300in}}%
\pgfpathlineto{\pgfqpoint{3.158413in}{3.132368in}}%
\pgfpathlineto{\pgfqpoint{3.158413in}{3.026780in}}%
\pgfpathlineto{\pgfqpoint{3.271160in}{3.079711in}}%
\pgfpathclose%
\pgfusepath{fill}%
\end{pgfscope}%
\begin{pgfscope}%
\pgfpathrectangle{\pgfqpoint{0.765000in}{0.660000in}}{\pgfqpoint{4.620000in}{4.620000in}}%
\pgfusepath{clip}%
\pgfsetbuttcap%
\pgfsetroundjoin%
\definecolor{currentfill}{rgb}{0.800000,0.176471,0.207843}%
\pgfsetfillcolor{currentfill}%
\pgfsetfillopacity{0.050000}%
\pgfsetlinewidth{0.000000pt}%
\definecolor{currentstroke}{rgb}{0.800000,0.176471,0.207843}%
\pgfsetstrokecolor{currentstroke}%
\pgfsetdash{}{0pt}%
\pgfpathmoveto{\pgfqpoint{3.241047in}{3.195074in}}%
\pgfpathlineto{\pgfqpoint{3.241047in}{3.300663in}}%
\pgfpathlineto{\pgfqpoint{3.353795in}{3.247731in}}%
\pgfpathlineto{\pgfqpoint{3.353795in}{3.142142in}}%
\pgfpathlineto{\pgfqpoint{3.241047in}{3.195074in}}%
\pgfpathclose%
\pgfusepath{fill}%
\end{pgfscope}%
\begin{pgfscope}%
\pgfpathrectangle{\pgfqpoint{0.765000in}{0.660000in}}{\pgfqpoint{4.620000in}{4.620000in}}%
\pgfusepath{clip}%
\pgfsetbuttcap%
\pgfsetroundjoin%
\definecolor{currentfill}{rgb}{0.800000,0.176471,0.207843}%
\pgfsetfillcolor{currentfill}%
\pgfsetfillopacity{0.050000}%
\pgfsetlinewidth{0.000000pt}%
\definecolor{currentstroke}{rgb}{0.800000,0.176471,0.207843}%
\pgfsetstrokecolor{currentstroke}%
\pgfsetdash{}{0pt}%
\pgfpathmoveto{\pgfqpoint{3.241047in}{3.195074in}}%
\pgfpathlineto{\pgfqpoint{3.241047in}{3.300663in}}%
\pgfpathlineto{\pgfqpoint{3.128300in}{3.247731in}}%
\pgfpathlineto{\pgfqpoint{3.128300in}{3.142142in}}%
\pgfpathlineto{\pgfqpoint{3.241047in}{3.195074in}}%
\pgfpathclose%
\pgfusepath{fill}%
\end{pgfscope}%
\begin{pgfscope}%
\pgfpathrectangle{\pgfqpoint{0.765000in}{0.660000in}}{\pgfqpoint{4.620000in}{4.620000in}}%
\pgfusepath{clip}%
\pgfsetbuttcap%
\pgfsetroundjoin%
\definecolor{currentfill}{rgb}{0.800000,0.176471,0.207843}%
\pgfsetfillcolor{currentfill}%
\pgfsetfillopacity{0.050000}%
\pgfsetlinewidth{0.000000pt}%
\definecolor{currentstroke}{rgb}{0.800000,0.176471,0.207843}%
\pgfsetstrokecolor{currentstroke}%
\pgfsetdash{}{0pt}%
\pgfpathmoveto{\pgfqpoint{3.241047in}{3.195074in}}%
\pgfpathlineto{\pgfqpoint{3.353795in}{3.142142in}}%
\pgfpathlineto{\pgfqpoint{3.241047in}{3.089211in}}%
\pgfpathlineto{\pgfqpoint{3.128300in}{3.142142in}}%
\pgfpathlineto{\pgfqpoint{3.241047in}{3.195074in}}%
\pgfpathclose%
\pgfusepath{fill}%
\end{pgfscope}%
\begin{pgfscope}%
\pgfpathrectangle{\pgfqpoint{0.765000in}{0.660000in}}{\pgfqpoint{4.620000in}{4.620000in}}%
\pgfusepath{clip}%
\pgfsetbuttcap%
\pgfsetroundjoin%
\definecolor{currentfill}{rgb}{0.800000,0.176471,0.207843}%
\pgfsetfillcolor{currentfill}%
\pgfsetfillopacity{0.050000}%
\pgfsetlinewidth{0.000000pt}%
\definecolor{currentstroke}{rgb}{0.800000,0.176471,0.207843}%
\pgfsetstrokecolor{currentstroke}%
\pgfsetdash{}{0pt}%
\pgfpathmoveto{\pgfqpoint{3.241047in}{3.300663in}}%
\pgfpathlineto{\pgfqpoint{3.353795in}{3.247731in}}%
\pgfpathlineto{\pgfqpoint{3.241047in}{3.194799in}}%
\pgfpathlineto{\pgfqpoint{3.128300in}{3.247731in}}%
\pgfpathlineto{\pgfqpoint{3.241047in}{3.300663in}}%
\pgfpathclose%
\pgfusepath{fill}%
\end{pgfscope}%
\begin{pgfscope}%
\pgfpathrectangle{\pgfqpoint{0.765000in}{0.660000in}}{\pgfqpoint{4.620000in}{4.620000in}}%
\pgfusepath{clip}%
\pgfsetbuttcap%
\pgfsetroundjoin%
\definecolor{currentfill}{rgb}{0.800000,0.176471,0.207843}%
\pgfsetfillcolor{currentfill}%
\pgfsetfillopacity{0.050000}%
\pgfsetlinewidth{0.000000pt}%
\definecolor{currentstroke}{rgb}{0.800000,0.176471,0.207843}%
\pgfsetstrokecolor{currentstroke}%
\pgfsetdash{}{0pt}%
\pgfpathmoveto{\pgfqpoint{3.353795in}{3.142142in}}%
\pgfpathlineto{\pgfqpoint{3.353795in}{3.247731in}}%
\pgfpathlineto{\pgfqpoint{3.241047in}{3.194799in}}%
\pgfpathlineto{\pgfqpoint{3.241047in}{3.089211in}}%
\pgfpathlineto{\pgfqpoint{3.353795in}{3.142142in}}%
\pgfpathclose%
\pgfusepath{fill}%
\end{pgfscope}%
\begin{pgfscope}%
\pgfpathrectangle{\pgfqpoint{0.765000in}{0.660000in}}{\pgfqpoint{4.620000in}{4.620000in}}%
\pgfusepath{clip}%
\pgfsetbuttcap%
\pgfsetroundjoin%
\definecolor{currentfill}{rgb}{0.800000,0.176471,0.207843}%
\pgfsetfillcolor{currentfill}%
\pgfsetfillopacity{0.050000}%
\pgfsetlinewidth{0.000000pt}%
\definecolor{currentstroke}{rgb}{0.800000,0.176471,0.207843}%
\pgfsetstrokecolor{currentstroke}%
\pgfsetdash{}{0pt}%
\pgfpathmoveto{\pgfqpoint{3.128300in}{3.142142in}}%
\pgfpathlineto{\pgfqpoint{3.128300in}{3.247731in}}%
\pgfpathlineto{\pgfqpoint{3.241047in}{3.194799in}}%
\pgfpathlineto{\pgfqpoint{3.241047in}{3.089211in}}%
\pgfpathlineto{\pgfqpoint{3.128300in}{3.142142in}}%
\pgfpathclose%
\pgfusepath{fill}%
\end{pgfscope}%
\begin{pgfscope}%
\pgfpathrectangle{\pgfqpoint{0.765000in}{0.660000in}}{\pgfqpoint{4.620000in}{4.620000in}}%
\pgfusepath{clip}%
\pgfsetbuttcap%
\pgfsetroundjoin%
\definecolor{currentfill}{rgb}{0.800000,0.176471,0.207843}%
\pgfsetfillcolor{currentfill}%
\pgfsetfillopacity{0.050000}%
\pgfsetlinewidth{0.000000pt}%
\definecolor{currentstroke}{rgb}{0.800000,0.176471,0.207843}%
\pgfsetstrokecolor{currentstroke}%
\pgfsetdash{}{0pt}%
\pgfpathmoveto{\pgfqpoint{3.327877in}{3.072782in}}%
\pgfpathlineto{\pgfqpoint{3.327877in}{3.178371in}}%
\pgfpathlineto{\pgfqpoint{3.440625in}{3.125440in}}%
\pgfpathlineto{\pgfqpoint{3.440625in}{3.019851in}}%
\pgfpathlineto{\pgfqpoint{3.327877in}{3.072782in}}%
\pgfpathclose%
\pgfusepath{fill}%
\end{pgfscope}%
\begin{pgfscope}%
\pgfpathrectangle{\pgfqpoint{0.765000in}{0.660000in}}{\pgfqpoint{4.620000in}{4.620000in}}%
\pgfusepath{clip}%
\pgfsetbuttcap%
\pgfsetroundjoin%
\definecolor{currentfill}{rgb}{0.800000,0.176471,0.207843}%
\pgfsetfillcolor{currentfill}%
\pgfsetfillopacity{0.050000}%
\pgfsetlinewidth{0.000000pt}%
\definecolor{currentstroke}{rgb}{0.800000,0.176471,0.207843}%
\pgfsetstrokecolor{currentstroke}%
\pgfsetdash{}{0pt}%
\pgfpathmoveto{\pgfqpoint{3.327877in}{3.072782in}}%
\pgfpathlineto{\pgfqpoint{3.327877in}{3.178371in}}%
\pgfpathlineto{\pgfqpoint{3.215130in}{3.125440in}}%
\pgfpathlineto{\pgfqpoint{3.215130in}{3.019851in}}%
\pgfpathlineto{\pgfqpoint{3.327877in}{3.072782in}}%
\pgfpathclose%
\pgfusepath{fill}%
\end{pgfscope}%
\begin{pgfscope}%
\pgfpathrectangle{\pgfqpoint{0.765000in}{0.660000in}}{\pgfqpoint{4.620000in}{4.620000in}}%
\pgfusepath{clip}%
\pgfsetbuttcap%
\pgfsetroundjoin%
\definecolor{currentfill}{rgb}{0.800000,0.176471,0.207843}%
\pgfsetfillcolor{currentfill}%
\pgfsetfillopacity{0.050000}%
\pgfsetlinewidth{0.000000pt}%
\definecolor{currentstroke}{rgb}{0.800000,0.176471,0.207843}%
\pgfsetstrokecolor{currentstroke}%
\pgfsetdash{}{0pt}%
\pgfpathmoveto{\pgfqpoint{3.327877in}{3.072782in}}%
\pgfpathlineto{\pgfqpoint{3.440625in}{3.019851in}}%
\pgfpathlineto{\pgfqpoint{3.327877in}{2.966919in}}%
\pgfpathlineto{\pgfqpoint{3.215130in}{3.019851in}}%
\pgfpathlineto{\pgfqpoint{3.327877in}{3.072782in}}%
\pgfpathclose%
\pgfusepath{fill}%
\end{pgfscope}%
\begin{pgfscope}%
\pgfpathrectangle{\pgfqpoint{0.765000in}{0.660000in}}{\pgfqpoint{4.620000in}{4.620000in}}%
\pgfusepath{clip}%
\pgfsetbuttcap%
\pgfsetroundjoin%
\definecolor{currentfill}{rgb}{0.800000,0.176471,0.207843}%
\pgfsetfillcolor{currentfill}%
\pgfsetfillopacity{0.050000}%
\pgfsetlinewidth{0.000000pt}%
\definecolor{currentstroke}{rgb}{0.800000,0.176471,0.207843}%
\pgfsetstrokecolor{currentstroke}%
\pgfsetdash{}{0pt}%
\pgfpathmoveto{\pgfqpoint{3.327877in}{3.178371in}}%
\pgfpathlineto{\pgfqpoint{3.440625in}{3.125440in}}%
\pgfpathlineto{\pgfqpoint{3.327877in}{3.072508in}}%
\pgfpathlineto{\pgfqpoint{3.215130in}{3.125440in}}%
\pgfpathlineto{\pgfqpoint{3.327877in}{3.178371in}}%
\pgfpathclose%
\pgfusepath{fill}%
\end{pgfscope}%
\begin{pgfscope}%
\pgfpathrectangle{\pgfqpoint{0.765000in}{0.660000in}}{\pgfqpoint{4.620000in}{4.620000in}}%
\pgfusepath{clip}%
\pgfsetbuttcap%
\pgfsetroundjoin%
\definecolor{currentfill}{rgb}{0.800000,0.176471,0.207843}%
\pgfsetfillcolor{currentfill}%
\pgfsetfillopacity{0.050000}%
\pgfsetlinewidth{0.000000pt}%
\definecolor{currentstroke}{rgb}{0.800000,0.176471,0.207843}%
\pgfsetstrokecolor{currentstroke}%
\pgfsetdash{}{0pt}%
\pgfpathmoveto{\pgfqpoint{3.215130in}{3.019851in}}%
\pgfpathlineto{\pgfqpoint{3.215130in}{3.125440in}}%
\pgfpathlineto{\pgfqpoint{3.327877in}{3.072508in}}%
\pgfpathlineto{\pgfqpoint{3.327877in}{2.966919in}}%
\pgfpathlineto{\pgfqpoint{3.215130in}{3.019851in}}%
\pgfpathclose%
\pgfusepath{fill}%
\end{pgfscope}%
\begin{pgfscope}%
\pgfpathrectangle{\pgfqpoint{0.765000in}{0.660000in}}{\pgfqpoint{4.620000in}{4.620000in}}%
\pgfusepath{clip}%
\pgfsetbuttcap%
\pgfsetroundjoin%
\definecolor{currentfill}{rgb}{0.800000,0.176471,0.207843}%
\pgfsetfillcolor{currentfill}%
\pgfsetfillopacity{0.050000}%
\pgfsetlinewidth{0.000000pt}%
\definecolor{currentstroke}{rgb}{0.800000,0.176471,0.207843}%
\pgfsetstrokecolor{currentstroke}%
\pgfsetdash{}{0pt}%
\pgfpathmoveto{\pgfqpoint{3.440625in}{3.019851in}}%
\pgfpathlineto{\pgfqpoint{3.440625in}{3.125440in}}%
\pgfpathlineto{\pgfqpoint{3.327877in}{3.072508in}}%
\pgfpathlineto{\pgfqpoint{3.327877in}{2.966919in}}%
\pgfpathlineto{\pgfqpoint{3.440625in}{3.019851in}}%
\pgfpathclose%
\pgfusepath{fill}%
\end{pgfscope}%
\end{pgfpicture}%
\makeatother%
\endgroup%
}
        }
        \caption{Toy multiclass dataset, cubic}
        \label{fig:toy_reverse}
    \end{subfigure}
    
    \caption{Tropical piecewise linear classifiers}
    \label{fig:plots}
\end{figure}

\subsection*{Contribution}

We formulate the tropical classification problem as a mean payoff game. We apply a Krasnoselskii-Mann iteration scheme to derive an $\varepsilon$-approximation of the tropical classifier in pseudo-polynomial time, and we develop geometrical guarantees, e.g. on the margin in the separable case. Finally, we propose a feature map for fitting tropical polynomials while retaining margin information. Thanks to their evaluation speed and their piecewise linear hypersurfaces, we enable more complex classification tasks. All our methods are implemented in \emph{Python} in the following repository: \url{https://gitlab.inria.fr/tropical/tropical-svm}.

\section{Shapley operators and mean payoff games}

In this section, we have a finite set of points $X=(x_{1},\ldots,x_{p})\in\mathbb{R}_{\text{max}}^{d\times p}$.
We define the \emph{tropical span} of $X$ as the set of tropical
linear combinations of these points:
\[
\text{Span}(X)\coloneqq\left\{\max_{1\le i\le p}(x_{i}+\lambda_{i}),\quad\lambda\in\mathbb{R}^{p}\right\}.
\]
As this is also the smallest tropically convex set containing these
points, separating several point clouds amounts to separating their
tropical spans. Given a tropically convex, compact and nonempty subset $\mathcal{V}$
of $\mathbb{R}_{\max}^{d}$, we define the projection of
$x$ in $\mathbb{R}_{\max}^{d}$ by:
\[
P_{\mathcal{V}}(x)\coloneqq\max\{y\in \mathcal{V},\quad y\le x\}.
\]
When $\mathcal{V}$ is the tropical span of $X$, we will also note
this projection as $P_{X}$. Tropical projections
can be expressed as a mean payoff game: for $i\in\{1,\ldots, p\}$ and $x$ in $\mathbb{R}_{\max}^{d}$,   \cite{Maclagan2015}
\begin{equation*}
P_{X}(x)_{i}=\max_{1\le j\le p}\left\{X_{ij}+\min_{1\le k \le d}(-X_{kj}+x_{k})\right\}.
\end{equation*}

Here, we recognize the payoff of a zero-sum deterministic game with perfect
information. There are two players, ``Max'' and ``Min'' (the maximizer
and the minimizer), who alternate their actions. Starting on node
$i$, Player Max chooses to move to node $j$, and receives $X_{ij}$
from Player Min. Similarily, Player Min in turn chooses a node $k$
and has to pay $-X_{kj}$ to Player Max.

We now recall some tools from Perron-Frobenius theory, in relation
with mean payoff games. We refer the reader to \cite{AKIAN2012} for more
information.

We call \emph{Shapley operator} a map $T:\mathbb{R}_{\max}^{d}\rightarrow\mathbb{R}_{\max}^{d}$
that is \emph{order-preserving} and \emph{additively homogenous}, i.e. $T(\alpha+x)=\alpha+T(x)$
for all $\alpha$ in $\mathbb{R}_{\max}$ and $x$ in $\mathbb{R}_{\max}^{d}$. $T$ is said to be \emph{diagonal free} when $T_{i}(x)$
is independent of $x_{i}$ for all $i\in\{1,\ldots, n\}$, i.e. when for all $x$, $y$ in $\mathbb{R}_{\max}$ which coincide on all coordinates except the $i$-th, we have $T_{i}(x)=T_{i}(y)$. 

Tropical projections are Shapley operators. Given a Shapley operator $T$, we also define
\[
\mathcal{S}(T)\coloneqq\left\{x\in\mathbb{R}_{\max}^{d},\quad x\le T(x)\right\}.
\]
More specifically, any tropically convex set $\mathcal{V}$ is equal
to $\mathcal{S}(P_{\mathcal{V}})$. Indeed, as for any $x\in\mathbb{R}_{\text{max}}^{d}$,
$P_{\mathcal{V}}(x)\le x$, having $x\in\mathcal{S}(P_{\mathcal{V}})$
implies $x=P_{\mathcal{V}}(x)$, meaning $x$ is in
$\mathcal{V}$.


Operator $P$ can be slightly tweaked to make it \emph{diagonal-free} -- but the resulting operator isn't a projector anymore.
In the modified game, the opponent is prevented from replying eye-for-an-eye:
\[
\left[P^{\text{DF}}_X(x)\right]_{i}\coloneqq\max_{1\le j\le p}\left\{ X_{ij}+\min_{k\ne i}(-X_{kj}+x_{k})\right\} .
\]

Noticeably, $\mathcal{S}(P_X^\text{DF}) = \mathcal{S}(P_X)$. $P$ and $P^\text{DF}$ can be evaluated in time $\mathcal{O}(dp)$ as they involve computing the two greatest values of $(-X_{kj}+x_k)_{1 \le k \le d}$ for $j\in\{1,\ldots, p\}$.

\section{Binary Tropical Support Vector Machines}

\subsection{Hard-Margin SVM}
In this section, we introduce a practical criterion for the existence of a separating tropical hyperplane, and if it exists, we construct one with optimal margin. We also place ourselves in the binary setting, where we seek to separate two convex hulls $\mathcal{V}^{+}$
and $\mathcal{V}^{-}$. We assume we have two Shapley operators $T^{+}$
and $T^{-}$ such that $\mathcal{S}(T^{\pm})=\mathcal{V}^{\pm}$,
which is for instance the case when separating finite point clouds:
corresponding operators are projections $P_{\mathcal{V}^{\pm}}$. We define:
\[
T=\min(T^{+},T^{-}),
\]
which is also a Shapley operator. We denote by $\perp$ the vector of $\mathbb{R}_{\max}^{d}$
identically equal to $-\infty$. The \emph{spectral radius} of $T$
is, by definition:
\[
\rho(T)=\sup\left\{\lambda\in\mathbb{R}_{\max},\quad\exists u\in\mathbb{R}_{\max}^{d}\backslash\{\perp\},\quad T(u)=\lambda+u\right\}.
\]

Note that $\mathcal{S}(T)$ is equal to the intersection of $\mathcal{S}(T^{+})$ and $\mathcal{S}(T^{-})$, the convex hulls we are seeking to separate.
From Allamigeon et al. \cite{Allamigeon2018}, we know that $\mathcal{V}^{+}$
and $\mathcal{V^{-}}$ being disjoint is equivalent to $\rho(T)<0$.

Given $a$ the eigenvector corresponding to $\rho(T)$, we define the sectors:
\[
I^{\pm}\coloneqq\left\{i\in\{1,\ldots, d\},\quad T^{\pm}(a)_{i}>\rho(T)+a_{i}\right\},
\]
and the corresponding configuration $\sigma=\{I^{+},I^{-}\}$.

\begin{prop} \label{prop:BinaryHardMargin}
If $\rho(T) < 0$, the tropical hyperplane of apex $a$ and configuration $\sigma$, given the sectors defined
above, separates $\mathcal{V}^{+}$ and $\mathcal{V}^{-}$ with a
margin of at least $-\rho(T)$. Moreover, this margin is exact when $T^{\pm}$
are of the form
\[
\sup_{v\in\mathcal{V}^{\pm}}\left(v_{i}+\min(-v+x)\right),
\]
which is in particular the case when separating finite point clouds.
\end{prop}

\begin{proof}
For instance, let $i\in I^{-}$. As $T^{+}$ is non-expansive, for $v^{+}\in \mathcal{V}^{+}$:
\[
v_{i}^{+}\le T^{+}(v^{+})_{i}=\left(T^{+}(v^{+})-T^{+}(a)\right)_{i}+T^{+}(a)_{i},
\]
hence 
\begin{equation*}
v_{i}^{+}\le\max(v^{+}-a)+T^{+}(a)_{i}.
\end{equation*}

As $i\in I^-$, $T^{+}(a)_{i}\le \rho(T)+a_{i}$,
and
\[
v_{i}^{+}-a_{i}\le\max(v^{+}-a)+\rho(T).
\]
In particular, $v_{i}^{+}-a_{i}<\max(v^{+}-a)$ and any element of
$\mathcal{V}^{+}$ cannot belong to any of the sectors in $I^{-}$ with
respect to $\mathcal{H}_{a}$, from which the sectors are well-defined.
Finally, 
\[
d_H(\mathcal{S}^-,v^{+})=\max(v^{+}-a)-\max(v^{+}-a)_{I^{-}}\ge-\rho(T),
\]
hence the margin is at least $-\rho(T)$ which is positive in the separable case, as shown in \cite{Allamigeon2018}.

Finally, when $T^{\pm}$
are of the aforementioned form, for  $i\in I^{-}$ and $\varepsilon>0$, we can
find $v\in \mathcal{V}^{+}$ such that 
\[
T^{+}(a)_{i}-\varepsilon\le v_{i}-\max(v-a)\le T^{+}(a)_{i},
\]
giving us 
\[
\rho(T)-\varepsilon\le v_{i}-a_{i}-\max(v-a)\le\rho(T).
\]

Maximizing over all $i\in I^{-}$ gives that $v$ is
at most at distance $-\rho(T)+\varepsilon$ of $\mathcal{S}^-$, hence
the margin.
\end{proof}

\begin{prop}\label{prop:SectorInterpretation}
TBD
\end{prop}
\todo{Caractérisation des secteurs en terme de présence de points témoins.}

\begin{proof}
TBD
\end{proof}

\begin{rem}\label{rem:UnreachedSectors}
To save on calculations, especially in large dimensions, we don't evaluate the coordinates associated with empty sectors. In fact, this is equivalent to defining corresponding apex coordinates to $+\inf$, which only increases the margin. For example, for $x^+\in X^+$, and $i$ a sector not assigned to the positive class, $$d_H(\mathcal{S}^-, x^+)=\max(x^+-a) - (x^+_i - a_i)$$ is an increasing quantity in $a_i$, as the $\max$ is not reached in the $i$ coordinate. This reasoning also applies to the multiclass case.
\end{rem}

%
To build a practical algorithm, we still need to compute the spectral
radius of $T$. As it amounts to solving mean payoff games, we can apply the Krasnoselskii-Mann iteration scheme described in Algorithm \ref{RVI}.
\begin{algorithm}[h!]
\caption{Krasnoselskii-Mann iteration scheme for finding Shapley eigenpairs}\label{RVI}
\begin{algorithmic}[1]
  \State \Input: A Shapley operator $T$ and a relative convergence criterion $\eta>0$.
\State $ x \coloneqq  \mathbf{1}_d \in \mathbb{R}_{\max}^d $
%
\Repeat \State $z \coloneqq  \left(x + T(x)\right)/2$
\State $x \coloneqq z - \max(z)\cdot\textbf{1}_d$
%
\Until { relative change of $x$ is lower than $\eta$ }
\State $a \coloneqq  x-\overline{x}; \, \lambda \coloneqq 2 \max(z)$\\

\Return $(a, \lambda)$
\end{algorithmic}
\end{algorithm}

\todo{quote a reference \& talk about efficiency}

\subsection{Geometric approach for soft-margin SVM}

We seek to classify convex hulls $\mathcal{V}^{+}$ and $\mathcal{V}^{-}$.
Assuming that the data overlap, we want to have a sense of their
non-separability. Using the operator $T$ we previously defined,  $\rho(T)$ is the
(positive) inner radius of $\mathcal{S}(T)$, i.e. the supremum of
the radii of the Hilbert balls contained in the intersection between
$\mathcal{V}^{+}$ and $\mathcal{V}^{-}$, as shown by Allamigeon et al. (2018) \cite{Allamigeon2018} and as seen in Figure \ref{fig:softmargin_ex}.

\begin{figure}[!h]
    \centering
    \begin{subfigure}{0.5\textwidth}
        \centering
        \resizebox{\linewidth}{!}{%
        \centering
            \clipbox{0.2\width{} 0.2\height{} 0.2\width{} 0.2\height{}}{%% Creator: Matplotlib, PGF backend
%%
%% To include the figure in your LaTeX document, write
%%   \input{<filename>.pgf}
%%
%% Make sure the required packages are loaded in your preamble
%%   \usepackage{pgf}
%%
%% Also ensure that all the required font packages are loaded; for instance,
%% the lmodern package is sometimes necessary when using math font.
%%   \usepackage{lmodern}
%%
%% Figures using additional raster images can only be included by \input if
%% they are in the same directory as the main LaTeX file. For loading figures
%% from other directories you can use the `import` package
%%   \usepackage{import}
%%
%% and then include the figures with
%%   \import{<path to file>}{<filename>.pgf}
%%
%% Matplotlib used the following preamble
%%   
%%   \usepackage{fontspec}
%%   \setmainfont{DejaVuSerif.ttf}[Path=\detokenize{/Users/sam/Library/Python/3.9/lib/python/site-packages/matplotlib/mpl-data/fonts/ttf/}]
%%   \setsansfont{Arial.ttf}[Path=\detokenize{/System/Library/Fonts/Supplemental/}]
%%   \setmonofont{DejaVuSansMono.ttf}[Path=\detokenize{/Users/sam/Library/Python/3.9/lib/python/site-packages/matplotlib/mpl-data/fonts/ttf/}]
%%   \makeatletter\@ifpackageloaded{underscore}{}{\usepackage[strings]{underscore}}\makeatother
%%
\begingroup%
\makeatletter%
\begin{pgfpicture}%
\pgfpathrectangle{\pgfpointorigin}{\pgfqpoint{6.400000in}{4.800000in}}%
\pgfusepath{use as bounding box, clip}%
\begin{pgfscope}%
\pgfsetbuttcap%
\pgfsetmiterjoin%
\definecolor{currentfill}{rgb}{1.000000,1.000000,1.000000}%
\pgfsetfillcolor{currentfill}%
\pgfsetlinewidth{0.000000pt}%
\definecolor{currentstroke}{rgb}{1.000000,1.000000,1.000000}%
\pgfsetstrokecolor{currentstroke}%
\pgfsetdash{}{0pt}%
\pgfpathmoveto{\pgfqpoint{0.000000in}{0.000000in}}%
\pgfpathlineto{\pgfqpoint{6.400000in}{0.000000in}}%
\pgfpathlineto{\pgfqpoint{6.400000in}{4.800000in}}%
\pgfpathlineto{\pgfqpoint{0.000000in}{4.800000in}}%
\pgfpathlineto{\pgfqpoint{0.000000in}{0.000000in}}%
\pgfpathclose%
\pgfusepath{fill}%
\end{pgfscope}%
\begin{pgfscope}%
\pgfsetbuttcap%
\pgfsetmiterjoin%
\definecolor{currentfill}{rgb}{1.000000,1.000000,1.000000}%
\pgfsetfillcolor{currentfill}%
\pgfsetlinewidth{0.000000pt}%
\definecolor{currentstroke}{rgb}{0.000000,0.000000,0.000000}%
\pgfsetstrokecolor{currentstroke}%
\pgfsetstrokeopacity{0.000000}%
\pgfsetdash{}{0pt}%
\pgfpathmoveto{\pgfqpoint{1.432000in}{0.528000in}}%
\pgfpathlineto{\pgfqpoint{5.128000in}{0.528000in}}%
\pgfpathlineto{\pgfqpoint{5.128000in}{4.224000in}}%
\pgfpathlineto{\pgfqpoint{1.432000in}{4.224000in}}%
\pgfpathlineto{\pgfqpoint{1.432000in}{0.528000in}}%
\pgfpathclose%
\pgfusepath{fill}%
\end{pgfscope}%
\begin{pgfscope}%
\pgfpathrectangle{\pgfqpoint{1.432000in}{0.528000in}}{\pgfqpoint{3.696000in}{3.696000in}}%
\pgfusepath{clip}%
\pgfsetbuttcap%
\pgfsetroundjoin%
\definecolor{currentfill}{rgb}{0.501961,0.501961,0.501961}%
\pgfsetfillcolor{currentfill}%
\pgfsetfillopacity{0.100000}%
\pgfsetlinewidth{0.000000pt}%
\definecolor{currentstroke}{rgb}{0.501961,0.501961,0.501961}%
\pgfsetstrokecolor{currentstroke}%
\pgfsetdash{}{0pt}%
\pgfpathmoveto{\pgfqpoint{3.150162in}{2.175329in}}%
\pgfpathlineto{\pgfqpoint{3.150162in}{2.416870in}}%
\pgfpathlineto{\pgfqpoint{3.408079in}{2.295785in}}%
\pgfpathlineto{\pgfqpoint{3.408079in}{2.054244in}}%
\pgfpathlineto{\pgfqpoint{3.150162in}{2.175329in}}%
\pgfpathclose%
\pgfusepath{fill}%
\end{pgfscope}%
\begin{pgfscope}%
\pgfpathrectangle{\pgfqpoint{1.432000in}{0.528000in}}{\pgfqpoint{3.696000in}{3.696000in}}%
\pgfusepath{clip}%
\pgfsetbuttcap%
\pgfsetroundjoin%
\definecolor{currentfill}{rgb}{0.501961,0.501961,0.501961}%
\pgfsetfillcolor{currentfill}%
\pgfsetfillopacity{0.100000}%
\pgfsetlinewidth{0.000000pt}%
\definecolor{currentstroke}{rgb}{0.501961,0.501961,0.501961}%
\pgfsetstrokecolor{currentstroke}%
\pgfsetdash{}{0pt}%
\pgfpathmoveto{\pgfqpoint{3.150162in}{2.175329in}}%
\pgfpathlineto{\pgfqpoint{3.150162in}{2.416870in}}%
\pgfpathlineto{\pgfqpoint{2.892245in}{2.295785in}}%
\pgfpathlineto{\pgfqpoint{2.892245in}{2.054244in}}%
\pgfpathlineto{\pgfqpoint{3.150162in}{2.175329in}}%
\pgfpathclose%
\pgfusepath{fill}%
\end{pgfscope}%
\begin{pgfscope}%
\pgfpathrectangle{\pgfqpoint{1.432000in}{0.528000in}}{\pgfqpoint{3.696000in}{3.696000in}}%
\pgfusepath{clip}%
\pgfsetbuttcap%
\pgfsetroundjoin%
\definecolor{currentfill}{rgb}{0.501961,0.501961,0.501961}%
\pgfsetfillcolor{currentfill}%
\pgfsetfillopacity{0.100000}%
\pgfsetlinewidth{0.000000pt}%
\definecolor{currentstroke}{rgb}{0.501961,0.501961,0.501961}%
\pgfsetstrokecolor{currentstroke}%
\pgfsetdash{}{0pt}%
\pgfpathmoveto{\pgfqpoint{3.150162in}{2.175329in}}%
\pgfpathlineto{\pgfqpoint{3.408079in}{2.054244in}}%
\pgfpathlineto{\pgfqpoint{3.150162in}{1.933159in}}%
\pgfpathlineto{\pgfqpoint{2.892245in}{2.054244in}}%
\pgfpathlineto{\pgfqpoint{3.150162in}{2.175329in}}%
\pgfpathclose%
\pgfusepath{fill}%
\end{pgfscope}%
\begin{pgfscope}%
\pgfpathrectangle{\pgfqpoint{1.432000in}{0.528000in}}{\pgfqpoint{3.696000in}{3.696000in}}%
\pgfusepath{clip}%
\pgfsetbuttcap%
\pgfsetroundjoin%
\definecolor{currentfill}{rgb}{0.501961,0.501961,0.501961}%
\pgfsetfillcolor{currentfill}%
\pgfsetfillopacity{0.100000}%
\pgfsetlinewidth{0.000000pt}%
\definecolor{currentstroke}{rgb}{0.501961,0.501961,0.501961}%
\pgfsetstrokecolor{currentstroke}%
\pgfsetdash{}{0pt}%
\pgfpathmoveto{\pgfqpoint{3.150162in}{2.416870in}}%
\pgfpathlineto{\pgfqpoint{3.408079in}{2.295785in}}%
\pgfpathlineto{\pgfqpoint{3.150162in}{2.174700in}}%
\pgfpathlineto{\pgfqpoint{2.892245in}{2.295785in}}%
\pgfpathlineto{\pgfqpoint{3.150162in}{2.416870in}}%
\pgfpathclose%
\pgfusepath{fill}%
\end{pgfscope}%
\begin{pgfscope}%
\pgfpathrectangle{\pgfqpoint{1.432000in}{0.528000in}}{\pgfqpoint{3.696000in}{3.696000in}}%
\pgfusepath{clip}%
\pgfsetbuttcap%
\pgfsetroundjoin%
\definecolor{currentfill}{rgb}{0.501961,0.501961,0.501961}%
\pgfsetfillcolor{currentfill}%
\pgfsetfillopacity{0.100000}%
\pgfsetlinewidth{0.000000pt}%
\definecolor{currentstroke}{rgb}{0.501961,0.501961,0.501961}%
\pgfsetstrokecolor{currentstroke}%
\pgfsetdash{}{0pt}%
\pgfpathmoveto{\pgfqpoint{2.892245in}{2.054244in}}%
\pgfpathlineto{\pgfqpoint{2.892245in}{2.295785in}}%
\pgfpathlineto{\pgfqpoint{3.150162in}{2.174700in}}%
\pgfpathlineto{\pgfqpoint{3.150162in}{1.933159in}}%
\pgfpathlineto{\pgfqpoint{2.892245in}{2.054244in}}%
\pgfpathclose%
\pgfusepath{fill}%
\end{pgfscope}%
\begin{pgfscope}%
\pgfpathrectangle{\pgfqpoint{1.432000in}{0.528000in}}{\pgfqpoint{3.696000in}{3.696000in}}%
\pgfusepath{clip}%
\pgfsetbuttcap%
\pgfsetroundjoin%
\definecolor{currentfill}{rgb}{0.501961,0.501961,0.501961}%
\pgfsetfillcolor{currentfill}%
\pgfsetfillopacity{0.100000}%
\pgfsetlinewidth{0.000000pt}%
\definecolor{currentstroke}{rgb}{0.501961,0.501961,0.501961}%
\pgfsetstrokecolor{currentstroke}%
\pgfsetdash{}{0pt}%
\pgfpathmoveto{\pgfqpoint{3.408079in}{2.054244in}}%
\pgfpathlineto{\pgfqpoint{3.408079in}{2.295785in}}%
\pgfpathlineto{\pgfqpoint{3.150162in}{2.174700in}}%
\pgfpathlineto{\pgfqpoint{3.150162in}{1.933159in}}%
\pgfpathlineto{\pgfqpoint{3.408079in}{2.054244in}}%
\pgfpathclose%
\pgfusepath{fill}%
\end{pgfscope}%
\begin{pgfscope}%
\pgfpathrectangle{\pgfqpoint{1.432000in}{0.528000in}}{\pgfqpoint{3.696000in}{3.696000in}}%
\pgfusepath{clip}%
\pgfsetbuttcap%
\pgfsetroundjoin%
\pgfsetlinewidth{1.204500pt}%
\definecolor{currentstroke}{rgb}{1.000000,0.576471,0.309804}%
\pgfsetstrokecolor{currentstroke}%
\pgfsetdash{{1.200000pt}{1.980000pt}}{0.000000pt}%
\pgfpathmoveto{\pgfqpoint{2.232471in}{2.823001in}}%
\pgfpathlineto{\pgfqpoint{1.422000in}{2.442508in}}%
\pgfusepath{stroke}%
\end{pgfscope}%
\begin{pgfscope}%
\pgfpathrectangle{\pgfqpoint{1.432000in}{0.528000in}}{\pgfqpoint{3.696000in}{3.696000in}}%
\pgfusepath{clip}%
\pgfsetbuttcap%
\pgfsetroundjoin%
\pgfsetlinewidth{1.204500pt}%
\definecolor{currentstroke}{rgb}{1.000000,0.576471,0.309804}%
\pgfsetstrokecolor{currentstroke}%
\pgfsetdash{{1.200000pt}{1.980000pt}}{0.000000pt}%
\pgfpathmoveto{\pgfqpoint{2.232471in}{2.823001in}}%
\pgfpathlineto{\pgfqpoint{3.382353in}{2.283164in}}%
\pgfusepath{stroke}%
\end{pgfscope}%
\begin{pgfscope}%
\pgfpathrectangle{\pgfqpoint{1.432000in}{0.528000in}}{\pgfqpoint{3.696000in}{3.696000in}}%
\pgfusepath{clip}%
\pgfsetbuttcap%
\pgfsetroundjoin%
\pgfsetlinewidth{1.204500pt}%
\definecolor{currentstroke}{rgb}{1.000000,0.576471,0.309804}%
\pgfsetstrokecolor{currentstroke}%
\pgfsetdash{{1.200000pt}{1.980000pt}}{0.000000pt}%
\pgfpathmoveto{\pgfqpoint{2.232471in}{2.823001in}}%
\pgfpathlineto{\pgfqpoint{2.232471in}{3.899874in}}%
\pgfusepath{stroke}%
\end{pgfscope}%
\begin{pgfscope}%
\pgfpathrectangle{\pgfqpoint{1.432000in}{0.528000in}}{\pgfqpoint{3.696000in}{3.696000in}}%
\pgfusepath{clip}%
\pgfsetroundcap%
\pgfsetroundjoin%
\pgfsetlinewidth{1.204500pt}%
\definecolor{currentstroke}{rgb}{1.000000,0.576471,0.309804}%
\pgfsetstrokecolor{currentstroke}%
\pgfsetdash{}{0pt}%
\pgfpathmoveto{\pgfqpoint{2.232471in}{2.823001in}}%
\pgfusepath{stroke}%
\end{pgfscope}%
\begin{pgfscope}%
\pgfpathrectangle{\pgfqpoint{1.432000in}{0.528000in}}{\pgfqpoint{3.696000in}{3.696000in}}%
\pgfusepath{clip}%
\pgfsetbuttcap%
\pgfsetroundjoin%
\definecolor{currentfill}{rgb}{1.000000,0.576471,0.309804}%
\pgfsetfillcolor{currentfill}%
\pgfsetlinewidth{1.003750pt}%
\definecolor{currentstroke}{rgb}{1.000000,0.576471,0.309804}%
\pgfsetstrokecolor{currentstroke}%
\pgfsetdash{}{0pt}%
\pgfsys@defobject{currentmarker}{\pgfqpoint{-0.033333in}{-0.033333in}}{\pgfqpoint{0.033333in}{0.033333in}}{%
\pgfpathmoveto{\pgfqpoint{0.000000in}{-0.033333in}}%
\pgfpathcurveto{\pgfqpoint{0.008840in}{-0.033333in}}{\pgfqpoint{0.017319in}{-0.029821in}}{\pgfqpoint{0.023570in}{-0.023570in}}%
\pgfpathcurveto{\pgfqpoint{0.029821in}{-0.017319in}}{\pgfqpoint{0.033333in}{-0.008840in}}{\pgfqpoint{0.033333in}{0.000000in}}%
\pgfpathcurveto{\pgfqpoint{0.033333in}{0.008840in}}{\pgfqpoint{0.029821in}{0.017319in}}{\pgfqpoint{0.023570in}{0.023570in}}%
\pgfpathcurveto{\pgfqpoint{0.017319in}{0.029821in}}{\pgfqpoint{0.008840in}{0.033333in}}{\pgfqpoint{0.000000in}{0.033333in}}%
\pgfpathcurveto{\pgfqpoint{-0.008840in}{0.033333in}}{\pgfqpoint{-0.017319in}{0.029821in}}{\pgfqpoint{-0.023570in}{0.023570in}}%
\pgfpathcurveto{\pgfqpoint{-0.029821in}{0.017319in}}{\pgfqpoint{-0.033333in}{0.008840in}}{\pgfqpoint{-0.033333in}{0.000000in}}%
\pgfpathcurveto{\pgfqpoint{-0.033333in}{-0.008840in}}{\pgfqpoint{-0.029821in}{-0.017319in}}{\pgfqpoint{-0.023570in}{-0.023570in}}%
\pgfpathcurveto{\pgfqpoint{-0.017319in}{-0.029821in}}{\pgfqpoint{-0.008840in}{-0.033333in}}{\pgfqpoint{0.000000in}{-0.033333in}}%
\pgfpathlineto{\pgfqpoint{0.000000in}{-0.033333in}}%
\pgfpathclose%
\pgfusepath{stroke,fill}%
}%
\begin{pgfscope}%
\pgfsys@transformshift{2.232471in}{2.823001in}%
\pgfsys@useobject{currentmarker}{}%
\end{pgfscope}%
\end{pgfscope}%
\begin{pgfscope}%
\pgfpathrectangle{\pgfqpoint{1.432000in}{0.528000in}}{\pgfqpoint{3.696000in}{3.696000in}}%
\pgfusepath{clip}%
\pgfsetbuttcap%
\pgfsetroundjoin%
\pgfsetlinewidth{1.204500pt}%
\definecolor{currentstroke}{rgb}{1.000000,0.576471,0.309804}%
\pgfsetstrokecolor{currentstroke}%
\pgfsetdash{{1.200000pt}{1.980000pt}}{0.000000pt}%
\pgfpathmoveto{\pgfqpoint{1.936960in}{2.778370in}}%
\pgfpathlineto{\pgfqpoint{1.422000in}{2.536611in}}%
\pgfusepath{stroke}%
\end{pgfscope}%
\begin{pgfscope}%
\pgfpathrectangle{\pgfqpoint{1.432000in}{0.528000in}}{\pgfqpoint{3.696000in}{3.696000in}}%
\pgfusepath{clip}%
\pgfsetbuttcap%
\pgfsetroundjoin%
\pgfsetlinewidth{1.204500pt}%
\definecolor{currentstroke}{rgb}{1.000000,0.576471,0.309804}%
\pgfsetstrokecolor{currentstroke}%
\pgfsetdash{{1.200000pt}{1.980000pt}}{0.000000pt}%
\pgfpathmoveto{\pgfqpoint{1.936960in}{2.778370in}}%
\pgfpathlineto{\pgfqpoint{3.086843in}{2.238533in}}%
\pgfusepath{stroke}%
\end{pgfscope}%
\begin{pgfscope}%
\pgfpathrectangle{\pgfqpoint{1.432000in}{0.528000in}}{\pgfqpoint{3.696000in}{3.696000in}}%
\pgfusepath{clip}%
\pgfsetbuttcap%
\pgfsetroundjoin%
\pgfsetlinewidth{1.204500pt}%
\definecolor{currentstroke}{rgb}{1.000000,0.576471,0.309804}%
\pgfsetstrokecolor{currentstroke}%
\pgfsetdash{{1.200000pt}{1.980000pt}}{0.000000pt}%
\pgfpathmoveto{\pgfqpoint{1.936960in}{2.778370in}}%
\pgfpathlineto{\pgfqpoint{1.936960in}{3.855244in}}%
\pgfusepath{stroke}%
\end{pgfscope}%
\begin{pgfscope}%
\pgfpathrectangle{\pgfqpoint{1.432000in}{0.528000in}}{\pgfqpoint{3.696000in}{3.696000in}}%
\pgfusepath{clip}%
\pgfsetroundcap%
\pgfsetroundjoin%
\pgfsetlinewidth{1.204500pt}%
\definecolor{currentstroke}{rgb}{1.000000,0.576471,0.309804}%
\pgfsetstrokecolor{currentstroke}%
\pgfsetdash{}{0pt}%
\pgfpathmoveto{\pgfqpoint{1.936960in}{2.778370in}}%
\pgfusepath{stroke}%
\end{pgfscope}%
\begin{pgfscope}%
\pgfpathrectangle{\pgfqpoint{1.432000in}{0.528000in}}{\pgfqpoint{3.696000in}{3.696000in}}%
\pgfusepath{clip}%
\pgfsetbuttcap%
\pgfsetroundjoin%
\definecolor{currentfill}{rgb}{1.000000,0.576471,0.309804}%
\pgfsetfillcolor{currentfill}%
\pgfsetlinewidth{1.003750pt}%
\definecolor{currentstroke}{rgb}{1.000000,0.576471,0.309804}%
\pgfsetstrokecolor{currentstroke}%
\pgfsetdash{}{0pt}%
\pgfsys@defobject{currentmarker}{\pgfqpoint{-0.033333in}{-0.033333in}}{\pgfqpoint{0.033333in}{0.033333in}}{%
\pgfpathmoveto{\pgfqpoint{0.000000in}{-0.033333in}}%
\pgfpathcurveto{\pgfqpoint{0.008840in}{-0.033333in}}{\pgfqpoint{0.017319in}{-0.029821in}}{\pgfqpoint{0.023570in}{-0.023570in}}%
\pgfpathcurveto{\pgfqpoint{0.029821in}{-0.017319in}}{\pgfqpoint{0.033333in}{-0.008840in}}{\pgfqpoint{0.033333in}{0.000000in}}%
\pgfpathcurveto{\pgfqpoint{0.033333in}{0.008840in}}{\pgfqpoint{0.029821in}{0.017319in}}{\pgfqpoint{0.023570in}{0.023570in}}%
\pgfpathcurveto{\pgfqpoint{0.017319in}{0.029821in}}{\pgfqpoint{0.008840in}{0.033333in}}{\pgfqpoint{0.000000in}{0.033333in}}%
\pgfpathcurveto{\pgfqpoint{-0.008840in}{0.033333in}}{\pgfqpoint{-0.017319in}{0.029821in}}{\pgfqpoint{-0.023570in}{0.023570in}}%
\pgfpathcurveto{\pgfqpoint{-0.029821in}{0.017319in}}{\pgfqpoint{-0.033333in}{0.008840in}}{\pgfqpoint{-0.033333in}{0.000000in}}%
\pgfpathcurveto{\pgfqpoint{-0.033333in}{-0.008840in}}{\pgfqpoint{-0.029821in}{-0.017319in}}{\pgfqpoint{-0.023570in}{-0.023570in}}%
\pgfpathcurveto{\pgfqpoint{-0.017319in}{-0.029821in}}{\pgfqpoint{-0.008840in}{-0.033333in}}{\pgfqpoint{0.000000in}{-0.033333in}}%
\pgfpathlineto{\pgfqpoint{0.000000in}{-0.033333in}}%
\pgfpathclose%
\pgfusepath{stroke,fill}%
}%
\begin{pgfscope}%
\pgfsys@transformshift{1.936960in}{2.778370in}%
\pgfsys@useobject{currentmarker}{}%
\end{pgfscope}%
\end{pgfscope}%
\begin{pgfscope}%
\pgfpathrectangle{\pgfqpoint{1.432000in}{0.528000in}}{\pgfqpoint{3.696000in}{3.696000in}}%
\pgfusepath{clip}%
\pgfsetbuttcap%
\pgfsetroundjoin%
\pgfsetlinewidth{1.204500pt}%
\definecolor{currentstroke}{rgb}{1.000000,0.576471,0.309804}%
\pgfsetstrokecolor{currentstroke}%
\pgfsetdash{{1.200000pt}{1.980000pt}}{0.000000pt}%
\pgfpathmoveto{\pgfqpoint{2.448280in}{1.844986in}}%
\pgfpathlineto{\pgfqpoint{1.422000in}{1.363177in}}%
\pgfusepath{stroke}%
\end{pgfscope}%
\begin{pgfscope}%
\pgfpathrectangle{\pgfqpoint{1.432000in}{0.528000in}}{\pgfqpoint{3.696000in}{3.696000in}}%
\pgfusepath{clip}%
\pgfsetbuttcap%
\pgfsetroundjoin%
\pgfsetlinewidth{1.204500pt}%
\definecolor{currentstroke}{rgb}{1.000000,0.576471,0.309804}%
\pgfsetstrokecolor{currentstroke}%
\pgfsetdash{{1.200000pt}{1.980000pt}}{0.000000pt}%
\pgfpathmoveto{\pgfqpoint{2.448280in}{1.844986in}}%
\pgfpathlineto{\pgfqpoint{3.598162in}{1.305148in}}%
\pgfusepath{stroke}%
\end{pgfscope}%
\begin{pgfscope}%
\pgfpathrectangle{\pgfqpoint{1.432000in}{0.528000in}}{\pgfqpoint{3.696000in}{3.696000in}}%
\pgfusepath{clip}%
\pgfsetbuttcap%
\pgfsetroundjoin%
\pgfsetlinewidth{1.204500pt}%
\definecolor{currentstroke}{rgb}{1.000000,0.576471,0.309804}%
\pgfsetstrokecolor{currentstroke}%
\pgfsetdash{{1.200000pt}{1.980000pt}}{0.000000pt}%
\pgfpathmoveto{\pgfqpoint{2.448280in}{1.844986in}}%
\pgfpathlineto{\pgfqpoint{2.448280in}{2.921859in}}%
\pgfusepath{stroke}%
\end{pgfscope}%
\begin{pgfscope}%
\pgfpathrectangle{\pgfqpoint{1.432000in}{0.528000in}}{\pgfqpoint{3.696000in}{3.696000in}}%
\pgfusepath{clip}%
\pgfsetroundcap%
\pgfsetroundjoin%
\pgfsetlinewidth{1.204500pt}%
\definecolor{currentstroke}{rgb}{1.000000,0.576471,0.309804}%
\pgfsetstrokecolor{currentstroke}%
\pgfsetdash{}{0pt}%
\pgfpathmoveto{\pgfqpoint{2.448280in}{1.844986in}}%
\pgfusepath{stroke}%
\end{pgfscope}%
\begin{pgfscope}%
\pgfpathrectangle{\pgfqpoint{1.432000in}{0.528000in}}{\pgfqpoint{3.696000in}{3.696000in}}%
\pgfusepath{clip}%
\pgfsetbuttcap%
\pgfsetroundjoin%
\definecolor{currentfill}{rgb}{1.000000,0.576471,0.309804}%
\pgfsetfillcolor{currentfill}%
\pgfsetlinewidth{1.003750pt}%
\definecolor{currentstroke}{rgb}{1.000000,0.576471,0.309804}%
\pgfsetstrokecolor{currentstroke}%
\pgfsetdash{}{0pt}%
\pgfsys@defobject{currentmarker}{\pgfqpoint{-0.033333in}{-0.033333in}}{\pgfqpoint{0.033333in}{0.033333in}}{%
\pgfpathmoveto{\pgfqpoint{0.000000in}{-0.033333in}}%
\pgfpathcurveto{\pgfqpoint{0.008840in}{-0.033333in}}{\pgfqpoint{0.017319in}{-0.029821in}}{\pgfqpoint{0.023570in}{-0.023570in}}%
\pgfpathcurveto{\pgfqpoint{0.029821in}{-0.017319in}}{\pgfqpoint{0.033333in}{-0.008840in}}{\pgfqpoint{0.033333in}{0.000000in}}%
\pgfpathcurveto{\pgfqpoint{0.033333in}{0.008840in}}{\pgfqpoint{0.029821in}{0.017319in}}{\pgfqpoint{0.023570in}{0.023570in}}%
\pgfpathcurveto{\pgfqpoint{0.017319in}{0.029821in}}{\pgfqpoint{0.008840in}{0.033333in}}{\pgfqpoint{0.000000in}{0.033333in}}%
\pgfpathcurveto{\pgfqpoint{-0.008840in}{0.033333in}}{\pgfqpoint{-0.017319in}{0.029821in}}{\pgfqpoint{-0.023570in}{0.023570in}}%
\pgfpathcurveto{\pgfqpoint{-0.029821in}{0.017319in}}{\pgfqpoint{-0.033333in}{0.008840in}}{\pgfqpoint{-0.033333in}{0.000000in}}%
\pgfpathcurveto{\pgfqpoint{-0.033333in}{-0.008840in}}{\pgfqpoint{-0.029821in}{-0.017319in}}{\pgfqpoint{-0.023570in}{-0.023570in}}%
\pgfpathcurveto{\pgfqpoint{-0.017319in}{-0.029821in}}{\pgfqpoint{-0.008840in}{-0.033333in}}{\pgfqpoint{0.000000in}{-0.033333in}}%
\pgfpathlineto{\pgfqpoint{0.000000in}{-0.033333in}}%
\pgfpathclose%
\pgfusepath{stroke,fill}%
}%
\begin{pgfscope}%
\pgfsys@transformshift{2.448280in}{1.844986in}%
\pgfsys@useobject{currentmarker}{}%
\end{pgfscope}%
\end{pgfscope}%
\begin{pgfscope}%
\pgfpathrectangle{\pgfqpoint{1.432000in}{0.528000in}}{\pgfqpoint{3.696000in}{3.696000in}}%
\pgfusepath{clip}%
\pgfsetbuttcap%
\pgfsetroundjoin%
\pgfsetlinewidth{1.204500pt}%
\definecolor{currentstroke}{rgb}{1.000000,0.576471,0.309804}%
\pgfsetstrokecolor{currentstroke}%
\pgfsetdash{{1.200000pt}{1.980000pt}}{0.000000pt}%
\pgfpathmoveto{\pgfqpoint{3.408079in}{1.420377in}}%
\pgfpathlineto{\pgfqpoint{2.258196in}{0.880540in}}%
\pgfusepath{stroke}%
\end{pgfscope}%
\begin{pgfscope}%
\pgfpathrectangle{\pgfqpoint{1.432000in}{0.528000in}}{\pgfqpoint{3.696000in}{3.696000in}}%
\pgfusepath{clip}%
\pgfsetbuttcap%
\pgfsetroundjoin%
\pgfsetlinewidth{1.204500pt}%
\definecolor{currentstroke}{rgb}{1.000000,0.576471,0.309804}%
\pgfsetstrokecolor{currentstroke}%
\pgfsetdash{{1.200000pt}{1.980000pt}}{0.000000pt}%
\pgfpathmoveto{\pgfqpoint{3.408079in}{1.420377in}}%
\pgfpathlineto{\pgfqpoint{4.557961in}{0.880540in}}%
\pgfusepath{stroke}%
\end{pgfscope}%
\begin{pgfscope}%
\pgfpathrectangle{\pgfqpoint{1.432000in}{0.528000in}}{\pgfqpoint{3.696000in}{3.696000in}}%
\pgfusepath{clip}%
\pgfsetbuttcap%
\pgfsetroundjoin%
\pgfsetlinewidth{1.204500pt}%
\definecolor{currentstroke}{rgb}{1.000000,0.576471,0.309804}%
\pgfsetstrokecolor{currentstroke}%
\pgfsetdash{{1.200000pt}{1.980000pt}}{0.000000pt}%
\pgfpathmoveto{\pgfqpoint{3.408079in}{1.420377in}}%
\pgfpathlineto{\pgfqpoint{3.408079in}{2.497251in}}%
\pgfusepath{stroke}%
\end{pgfscope}%
\begin{pgfscope}%
\pgfpathrectangle{\pgfqpoint{1.432000in}{0.528000in}}{\pgfqpoint{3.696000in}{3.696000in}}%
\pgfusepath{clip}%
\pgfsetroundcap%
\pgfsetroundjoin%
\pgfsetlinewidth{1.204500pt}%
\definecolor{currentstroke}{rgb}{1.000000,0.576471,0.309804}%
\pgfsetstrokecolor{currentstroke}%
\pgfsetdash{}{0pt}%
\pgfpathmoveto{\pgfqpoint{3.408079in}{1.420377in}}%
\pgfusepath{stroke}%
\end{pgfscope}%
\begin{pgfscope}%
\pgfpathrectangle{\pgfqpoint{1.432000in}{0.528000in}}{\pgfqpoint{3.696000in}{3.696000in}}%
\pgfusepath{clip}%
\pgfsetbuttcap%
\pgfsetroundjoin%
\definecolor{currentfill}{rgb}{1.000000,0.576471,0.309804}%
\pgfsetfillcolor{currentfill}%
\pgfsetlinewidth{1.003750pt}%
\definecolor{currentstroke}{rgb}{1.000000,0.576471,0.309804}%
\pgfsetstrokecolor{currentstroke}%
\pgfsetdash{}{0pt}%
\pgfsys@defobject{currentmarker}{\pgfqpoint{-0.033333in}{-0.033333in}}{\pgfqpoint{0.033333in}{0.033333in}}{%
\pgfpathmoveto{\pgfqpoint{0.000000in}{-0.033333in}}%
\pgfpathcurveto{\pgfqpoint{0.008840in}{-0.033333in}}{\pgfqpoint{0.017319in}{-0.029821in}}{\pgfqpoint{0.023570in}{-0.023570in}}%
\pgfpathcurveto{\pgfqpoint{0.029821in}{-0.017319in}}{\pgfqpoint{0.033333in}{-0.008840in}}{\pgfqpoint{0.033333in}{0.000000in}}%
\pgfpathcurveto{\pgfqpoint{0.033333in}{0.008840in}}{\pgfqpoint{0.029821in}{0.017319in}}{\pgfqpoint{0.023570in}{0.023570in}}%
\pgfpathcurveto{\pgfqpoint{0.017319in}{0.029821in}}{\pgfqpoint{0.008840in}{0.033333in}}{\pgfqpoint{0.000000in}{0.033333in}}%
\pgfpathcurveto{\pgfqpoint{-0.008840in}{0.033333in}}{\pgfqpoint{-0.017319in}{0.029821in}}{\pgfqpoint{-0.023570in}{0.023570in}}%
\pgfpathcurveto{\pgfqpoint{-0.029821in}{0.017319in}}{\pgfqpoint{-0.033333in}{0.008840in}}{\pgfqpoint{-0.033333in}{0.000000in}}%
\pgfpathcurveto{\pgfqpoint{-0.033333in}{-0.008840in}}{\pgfqpoint{-0.029821in}{-0.017319in}}{\pgfqpoint{-0.023570in}{-0.023570in}}%
\pgfpathcurveto{\pgfqpoint{-0.017319in}{-0.029821in}}{\pgfqpoint{-0.008840in}{-0.033333in}}{\pgfqpoint{0.000000in}{-0.033333in}}%
\pgfpathlineto{\pgfqpoint{0.000000in}{-0.033333in}}%
\pgfpathclose%
\pgfusepath{stroke,fill}%
}%
\begin{pgfscope}%
\pgfsys@transformshift{3.408079in}{1.420377in}%
\pgfsys@useobject{currentmarker}{}%
\end{pgfscope}%
\end{pgfscope}%
\begin{pgfscope}%
\pgfpathrectangle{\pgfqpoint{1.432000in}{0.528000in}}{\pgfqpoint{3.696000in}{3.696000in}}%
\pgfusepath{clip}%
\pgfsetbuttcap%
\pgfsetroundjoin%
\pgfsetlinewidth{1.204500pt}%
\definecolor{currentstroke}{rgb}{1.000000,0.576471,0.309804}%
\pgfsetstrokecolor{currentstroke}%
\pgfsetdash{{1.200000pt}{1.980000pt}}{0.000000pt}%
\pgfpathmoveto{\pgfqpoint{3.605286in}{2.630009in}}%
\pgfpathlineto{\pgfqpoint{2.455403in}{2.090172in}}%
\pgfusepath{stroke}%
\end{pgfscope}%
\begin{pgfscope}%
\pgfpathrectangle{\pgfqpoint{1.432000in}{0.528000in}}{\pgfqpoint{3.696000in}{3.696000in}}%
\pgfusepath{clip}%
\pgfsetbuttcap%
\pgfsetroundjoin%
\pgfsetlinewidth{1.204500pt}%
\definecolor{currentstroke}{rgb}{1.000000,0.576471,0.309804}%
\pgfsetstrokecolor{currentstroke}%
\pgfsetdash{{1.200000pt}{1.980000pt}}{0.000000pt}%
\pgfpathmoveto{\pgfqpoint{3.605286in}{2.630009in}}%
\pgfpathlineto{\pgfqpoint{4.755169in}{2.090172in}}%
\pgfusepath{stroke}%
\end{pgfscope}%
\begin{pgfscope}%
\pgfpathrectangle{\pgfqpoint{1.432000in}{0.528000in}}{\pgfqpoint{3.696000in}{3.696000in}}%
\pgfusepath{clip}%
\pgfsetbuttcap%
\pgfsetroundjoin%
\pgfsetlinewidth{1.204500pt}%
\definecolor{currentstroke}{rgb}{1.000000,0.576471,0.309804}%
\pgfsetstrokecolor{currentstroke}%
\pgfsetdash{{1.200000pt}{1.980000pt}}{0.000000pt}%
\pgfpathmoveto{\pgfqpoint{3.605286in}{2.630009in}}%
\pgfpathlineto{\pgfqpoint{3.605286in}{3.706883in}}%
\pgfusepath{stroke}%
\end{pgfscope}%
\begin{pgfscope}%
\pgfpathrectangle{\pgfqpoint{1.432000in}{0.528000in}}{\pgfqpoint{3.696000in}{3.696000in}}%
\pgfusepath{clip}%
\pgfsetroundcap%
\pgfsetroundjoin%
\pgfsetlinewidth{1.204500pt}%
\definecolor{currentstroke}{rgb}{1.000000,0.576471,0.309804}%
\pgfsetstrokecolor{currentstroke}%
\pgfsetdash{}{0pt}%
\pgfpathmoveto{\pgfqpoint{3.605286in}{2.630009in}}%
\pgfusepath{stroke}%
\end{pgfscope}%
\begin{pgfscope}%
\pgfpathrectangle{\pgfqpoint{1.432000in}{0.528000in}}{\pgfqpoint{3.696000in}{3.696000in}}%
\pgfusepath{clip}%
\pgfsetbuttcap%
\pgfsetroundjoin%
\definecolor{currentfill}{rgb}{1.000000,0.576471,0.309804}%
\pgfsetfillcolor{currentfill}%
\pgfsetlinewidth{1.003750pt}%
\definecolor{currentstroke}{rgb}{1.000000,0.576471,0.309804}%
\pgfsetstrokecolor{currentstroke}%
\pgfsetdash{}{0pt}%
\pgfsys@defobject{currentmarker}{\pgfqpoint{-0.033333in}{-0.033333in}}{\pgfqpoint{0.033333in}{0.033333in}}{%
\pgfpathmoveto{\pgfqpoint{0.000000in}{-0.033333in}}%
\pgfpathcurveto{\pgfqpoint{0.008840in}{-0.033333in}}{\pgfqpoint{0.017319in}{-0.029821in}}{\pgfqpoint{0.023570in}{-0.023570in}}%
\pgfpathcurveto{\pgfqpoint{0.029821in}{-0.017319in}}{\pgfqpoint{0.033333in}{-0.008840in}}{\pgfqpoint{0.033333in}{0.000000in}}%
\pgfpathcurveto{\pgfqpoint{0.033333in}{0.008840in}}{\pgfqpoint{0.029821in}{0.017319in}}{\pgfqpoint{0.023570in}{0.023570in}}%
\pgfpathcurveto{\pgfqpoint{0.017319in}{0.029821in}}{\pgfqpoint{0.008840in}{0.033333in}}{\pgfqpoint{0.000000in}{0.033333in}}%
\pgfpathcurveto{\pgfqpoint{-0.008840in}{0.033333in}}{\pgfqpoint{-0.017319in}{0.029821in}}{\pgfqpoint{-0.023570in}{0.023570in}}%
\pgfpathcurveto{\pgfqpoint{-0.029821in}{0.017319in}}{\pgfqpoint{-0.033333in}{0.008840in}}{\pgfqpoint{-0.033333in}{0.000000in}}%
\pgfpathcurveto{\pgfqpoint{-0.033333in}{-0.008840in}}{\pgfqpoint{-0.029821in}{-0.017319in}}{\pgfqpoint{-0.023570in}{-0.023570in}}%
\pgfpathcurveto{\pgfqpoint{-0.017319in}{-0.029821in}}{\pgfqpoint{-0.008840in}{-0.033333in}}{\pgfqpoint{0.000000in}{-0.033333in}}%
\pgfpathlineto{\pgfqpoint{0.000000in}{-0.033333in}}%
\pgfpathclose%
\pgfusepath{stroke,fill}%
}%
\begin{pgfscope}%
\pgfsys@transformshift{3.605286in}{2.630009in}%
\pgfsys@useobject{currentmarker}{}%
\end{pgfscope}%
\end{pgfscope}%
\begin{pgfscope}%
\pgfpathrectangle{\pgfqpoint{1.432000in}{0.528000in}}{\pgfqpoint{3.696000in}{3.696000in}}%
\pgfusepath{clip}%
\pgfsetbuttcap%
\pgfsetroundjoin%
\pgfsetlinewidth{1.204500pt}%
\definecolor{currentstroke}{rgb}{0.176471,0.192157,0.258824}%
\pgfsetstrokecolor{currentstroke}%
\pgfsetdash{{4.440000pt}{1.920000pt}}{0.000000pt}%
\pgfpathmoveto{\pgfqpoint{2.892245in}{2.003575in}}%
\pgfpathlineto{\pgfqpoint{1.742362in}{1.463737in}}%
\pgfusepath{stroke}%
\end{pgfscope}%
\begin{pgfscope}%
\pgfpathrectangle{\pgfqpoint{1.432000in}{0.528000in}}{\pgfqpoint{3.696000in}{3.696000in}}%
\pgfusepath{clip}%
\pgfsetbuttcap%
\pgfsetroundjoin%
\pgfsetlinewidth{1.204500pt}%
\definecolor{currentstroke}{rgb}{0.176471,0.192157,0.258824}%
\pgfsetstrokecolor{currentstroke}%
\pgfsetdash{{4.440000pt}{1.920000pt}}{0.000000pt}%
\pgfpathmoveto{\pgfqpoint{2.892245in}{2.003575in}}%
\pgfpathlineto{\pgfqpoint{4.042127in}{1.463737in}}%
\pgfusepath{stroke}%
\end{pgfscope}%
\begin{pgfscope}%
\pgfpathrectangle{\pgfqpoint{1.432000in}{0.528000in}}{\pgfqpoint{3.696000in}{3.696000in}}%
\pgfusepath{clip}%
\pgfsetbuttcap%
\pgfsetroundjoin%
\pgfsetlinewidth{1.204500pt}%
\definecolor{currentstroke}{rgb}{0.176471,0.192157,0.258824}%
\pgfsetstrokecolor{currentstroke}%
\pgfsetdash{{4.440000pt}{1.920000pt}}{0.000000pt}%
\pgfpathmoveto{\pgfqpoint{2.892245in}{2.003575in}}%
\pgfpathlineto{\pgfqpoint{2.892245in}{3.080448in}}%
\pgfusepath{stroke}%
\end{pgfscope}%
\begin{pgfscope}%
\pgfpathrectangle{\pgfqpoint{1.432000in}{0.528000in}}{\pgfqpoint{3.696000in}{3.696000in}}%
\pgfusepath{clip}%
\pgfsetroundcap%
\pgfsetroundjoin%
\pgfsetlinewidth{1.204500pt}%
\definecolor{currentstroke}{rgb}{0.176471,0.192157,0.258824}%
\pgfsetstrokecolor{currentstroke}%
\pgfsetdash{}{0pt}%
\pgfpathmoveto{\pgfqpoint{2.892245in}{2.003575in}}%
\pgfusepath{stroke}%
\end{pgfscope}%
\begin{pgfscope}%
\pgfpathrectangle{\pgfqpoint{1.432000in}{0.528000in}}{\pgfqpoint{3.696000in}{3.696000in}}%
\pgfusepath{clip}%
\pgfsetbuttcap%
\pgfsetmiterjoin%
\definecolor{currentfill}{rgb}{0.176471,0.192157,0.258824}%
\pgfsetfillcolor{currentfill}%
\pgfsetlinewidth{1.003750pt}%
\definecolor{currentstroke}{rgb}{0.176471,0.192157,0.258824}%
\pgfsetstrokecolor{currentstroke}%
\pgfsetdash{}{0pt}%
\pgfsys@defobject{currentmarker}{\pgfqpoint{-0.033333in}{-0.033333in}}{\pgfqpoint{0.033333in}{0.033333in}}{%
\pgfpathmoveto{\pgfqpoint{-0.000000in}{-0.033333in}}%
\pgfpathlineto{\pgfqpoint{0.033333in}{0.033333in}}%
\pgfpathlineto{\pgfqpoint{-0.033333in}{0.033333in}}%
\pgfpathlineto{\pgfqpoint{-0.000000in}{-0.033333in}}%
\pgfpathclose%
\pgfusepath{stroke,fill}%
}%
\begin{pgfscope}%
\pgfsys@transformshift{2.892245in}{2.003575in}%
\pgfsys@useobject{currentmarker}{}%
\end{pgfscope}%
\end{pgfscope}%
\begin{pgfscope}%
\pgfpathrectangle{\pgfqpoint{1.432000in}{0.528000in}}{\pgfqpoint{3.696000in}{3.696000in}}%
\pgfusepath{clip}%
\pgfsetbuttcap%
\pgfsetroundjoin%
\pgfsetlinewidth{1.204500pt}%
\definecolor{currentstroke}{rgb}{0.176471,0.192157,0.258824}%
\pgfsetstrokecolor{currentstroke}%
\pgfsetdash{{4.440000pt}{1.920000pt}}{0.000000pt}%
\pgfpathmoveto{\pgfqpoint{4.777998in}{2.245549in}}%
\pgfpathlineto{\pgfqpoint{3.628116in}{1.705712in}}%
\pgfusepath{stroke}%
\end{pgfscope}%
\begin{pgfscope}%
\pgfpathrectangle{\pgfqpoint{1.432000in}{0.528000in}}{\pgfqpoint{3.696000in}{3.696000in}}%
\pgfusepath{clip}%
\pgfsetbuttcap%
\pgfsetroundjoin%
\pgfsetlinewidth{1.204500pt}%
\definecolor{currentstroke}{rgb}{0.176471,0.192157,0.258824}%
\pgfsetstrokecolor{currentstroke}%
\pgfsetdash{{4.440000pt}{1.920000pt}}{0.000000pt}%
\pgfpathmoveto{\pgfqpoint{4.777998in}{2.245549in}}%
\pgfpathlineto{\pgfqpoint{5.138000in}{2.076538in}}%
\pgfusepath{stroke}%
\end{pgfscope}%
\begin{pgfscope}%
\pgfpathrectangle{\pgfqpoint{1.432000in}{0.528000in}}{\pgfqpoint{3.696000in}{3.696000in}}%
\pgfusepath{clip}%
\pgfsetbuttcap%
\pgfsetroundjoin%
\pgfsetlinewidth{1.204500pt}%
\definecolor{currentstroke}{rgb}{0.176471,0.192157,0.258824}%
\pgfsetstrokecolor{currentstroke}%
\pgfsetdash{{4.440000pt}{1.920000pt}}{0.000000pt}%
\pgfpathmoveto{\pgfqpoint{4.777998in}{2.245549in}}%
\pgfpathlineto{\pgfqpoint{4.777998in}{3.322422in}}%
\pgfusepath{stroke}%
\end{pgfscope}%
\begin{pgfscope}%
\pgfpathrectangle{\pgfqpoint{1.432000in}{0.528000in}}{\pgfqpoint{3.696000in}{3.696000in}}%
\pgfusepath{clip}%
\pgfsetroundcap%
\pgfsetroundjoin%
\pgfsetlinewidth{1.204500pt}%
\definecolor{currentstroke}{rgb}{0.176471,0.192157,0.258824}%
\pgfsetstrokecolor{currentstroke}%
\pgfsetdash{}{0pt}%
\pgfpathmoveto{\pgfqpoint{4.777998in}{2.245549in}}%
\pgfusepath{stroke}%
\end{pgfscope}%
\begin{pgfscope}%
\pgfpathrectangle{\pgfqpoint{1.432000in}{0.528000in}}{\pgfqpoint{3.696000in}{3.696000in}}%
\pgfusepath{clip}%
\pgfsetbuttcap%
\pgfsetmiterjoin%
\definecolor{currentfill}{rgb}{0.176471,0.192157,0.258824}%
\pgfsetfillcolor{currentfill}%
\pgfsetlinewidth{1.003750pt}%
\definecolor{currentstroke}{rgb}{0.176471,0.192157,0.258824}%
\pgfsetstrokecolor{currentstroke}%
\pgfsetdash{}{0pt}%
\pgfsys@defobject{currentmarker}{\pgfqpoint{-0.033333in}{-0.033333in}}{\pgfqpoint{0.033333in}{0.033333in}}{%
\pgfpathmoveto{\pgfqpoint{-0.000000in}{-0.033333in}}%
\pgfpathlineto{\pgfqpoint{0.033333in}{0.033333in}}%
\pgfpathlineto{\pgfqpoint{-0.033333in}{0.033333in}}%
\pgfpathlineto{\pgfqpoint{-0.000000in}{-0.033333in}}%
\pgfpathclose%
\pgfusepath{stroke,fill}%
}%
\begin{pgfscope}%
\pgfsys@transformshift{4.777998in}{2.245549in}%
\pgfsys@useobject{currentmarker}{}%
\end{pgfscope}%
\end{pgfscope}%
\begin{pgfscope}%
\pgfpathrectangle{\pgfqpoint{1.432000in}{0.528000in}}{\pgfqpoint{3.696000in}{3.696000in}}%
\pgfusepath{clip}%
\pgfsetbuttcap%
\pgfsetroundjoin%
\pgfsetlinewidth{1.204500pt}%
\definecolor{currentstroke}{rgb}{0.176471,0.192157,0.258824}%
\pgfsetstrokecolor{currentstroke}%
\pgfsetdash{{4.440000pt}{1.920000pt}}{0.000000pt}%
\pgfpathmoveto{\pgfqpoint{3.885302in}{3.369306in}}%
\pgfpathlineto{\pgfqpoint{2.735420in}{2.829469in}}%
\pgfusepath{stroke}%
\end{pgfscope}%
\begin{pgfscope}%
\pgfpathrectangle{\pgfqpoint{1.432000in}{0.528000in}}{\pgfqpoint{3.696000in}{3.696000in}}%
\pgfusepath{clip}%
\pgfsetbuttcap%
\pgfsetroundjoin%
\pgfsetlinewidth{1.204500pt}%
\definecolor{currentstroke}{rgb}{0.176471,0.192157,0.258824}%
\pgfsetstrokecolor{currentstroke}%
\pgfsetdash{{4.440000pt}{1.920000pt}}{0.000000pt}%
\pgfpathmoveto{\pgfqpoint{3.885302in}{3.369306in}}%
\pgfpathlineto{\pgfqpoint{5.035185in}{2.829469in}}%
\pgfusepath{stroke}%
\end{pgfscope}%
\begin{pgfscope}%
\pgfpathrectangle{\pgfqpoint{1.432000in}{0.528000in}}{\pgfqpoint{3.696000in}{3.696000in}}%
\pgfusepath{clip}%
\pgfsetbuttcap%
\pgfsetroundjoin%
\pgfsetlinewidth{1.204500pt}%
\definecolor{currentstroke}{rgb}{0.176471,0.192157,0.258824}%
\pgfsetstrokecolor{currentstroke}%
\pgfsetdash{{4.440000pt}{1.920000pt}}{0.000000pt}%
\pgfpathmoveto{\pgfqpoint{3.885302in}{3.369306in}}%
\pgfpathlineto{\pgfqpoint{3.885302in}{4.234000in}}%
\pgfusepath{stroke}%
\end{pgfscope}%
\begin{pgfscope}%
\pgfpathrectangle{\pgfqpoint{1.432000in}{0.528000in}}{\pgfqpoint{3.696000in}{3.696000in}}%
\pgfusepath{clip}%
\pgfsetroundcap%
\pgfsetroundjoin%
\pgfsetlinewidth{1.204500pt}%
\definecolor{currentstroke}{rgb}{0.176471,0.192157,0.258824}%
\pgfsetstrokecolor{currentstroke}%
\pgfsetdash{}{0pt}%
\pgfpathmoveto{\pgfqpoint{3.885302in}{3.369306in}}%
\pgfusepath{stroke}%
\end{pgfscope}%
\begin{pgfscope}%
\pgfpathrectangle{\pgfqpoint{1.432000in}{0.528000in}}{\pgfqpoint{3.696000in}{3.696000in}}%
\pgfusepath{clip}%
\pgfsetbuttcap%
\pgfsetmiterjoin%
\definecolor{currentfill}{rgb}{0.176471,0.192157,0.258824}%
\pgfsetfillcolor{currentfill}%
\pgfsetlinewidth{1.003750pt}%
\definecolor{currentstroke}{rgb}{0.176471,0.192157,0.258824}%
\pgfsetstrokecolor{currentstroke}%
\pgfsetdash{}{0pt}%
\pgfsys@defobject{currentmarker}{\pgfqpoint{-0.033333in}{-0.033333in}}{\pgfqpoint{0.033333in}{0.033333in}}{%
\pgfpathmoveto{\pgfqpoint{-0.000000in}{-0.033333in}}%
\pgfpathlineto{\pgfqpoint{0.033333in}{0.033333in}}%
\pgfpathlineto{\pgfqpoint{-0.033333in}{0.033333in}}%
\pgfpathlineto{\pgfqpoint{-0.000000in}{-0.033333in}}%
\pgfpathclose%
\pgfusepath{stroke,fill}%
}%
\begin{pgfscope}%
\pgfsys@transformshift{3.885302in}{3.369306in}%
\pgfsys@useobject{currentmarker}{}%
\end{pgfscope}%
\end{pgfscope}%
\begin{pgfscope}%
\pgfpathrectangle{\pgfqpoint{1.432000in}{0.528000in}}{\pgfqpoint{3.696000in}{3.696000in}}%
\pgfusepath{clip}%
\pgfsetroundcap%
\pgfsetroundjoin%
\pgfsetlinewidth{1.204500pt}%
\definecolor{currentstroke}{rgb}{0.000000,0.000000,0.000000}%
\pgfsetstrokecolor{currentstroke}%
\pgfsetdash{}{0pt}%
\pgfpathmoveto{\pgfqpoint{3.150162in}{2.175015in}}%
\pgfpathlineto{\pgfqpoint{5.138000in}{3.108248in}}%
\pgfusepath{stroke}%
\end{pgfscope}%
\begin{pgfscope}%
\pgfpathrectangle{\pgfqpoint{1.432000in}{0.528000in}}{\pgfqpoint{3.696000in}{3.696000in}}%
\pgfusepath{clip}%
\pgfsetroundcap%
\pgfsetroundjoin%
\pgfsetlinewidth{1.204500pt}%
\definecolor{currentstroke}{rgb}{0.000000,0.000000,0.000000}%
\pgfsetstrokecolor{currentstroke}%
\pgfsetdash{}{0pt}%
\pgfpathmoveto{\pgfqpoint{3.150162in}{2.175015in}}%
\pgfpathlineto{\pgfqpoint{1.422000in}{2.986337in}}%
\pgfusepath{stroke}%
\end{pgfscope}%
\begin{pgfscope}%
\pgfpathrectangle{\pgfqpoint{1.432000in}{0.528000in}}{\pgfqpoint{3.696000in}{3.696000in}}%
\pgfusepath{clip}%
\pgfsetroundcap%
\pgfsetroundjoin%
\pgfsetlinewidth{1.204500pt}%
\definecolor{currentstroke}{rgb}{0.000000,0.000000,0.000000}%
\pgfsetstrokecolor{currentstroke}%
\pgfsetdash{}{0pt}%
\pgfpathmoveto{\pgfqpoint{3.150162in}{2.175015in}}%
\pgfpathlineto{\pgfqpoint{3.150162in}{0.518000in}}%
\pgfusepath{stroke}%
\end{pgfscope}%
\end{pgfpicture}%
\makeatother%
\endgroup%
}
        }
    \end{subfigure}
    
    \caption{Eigenvalue is inner radius of intersection}
    \label{fig:softmargin_ex}
\end{figure}

The next Proposition shows
us that this measure can be interpreted geometrically, in the case
where $\mathcal{V}^{+}$ and $\mathcal{V}^{-}$ are generated by point
clouds $X^{+}$ and $X^{-}$:
\begin{prop}\label{prop:BinarySoftMargin}
By moving some points from $X^{+}$ and $X^{-}$ by a distance of
at most $\rho(T)$, we can nullify the interior of the intersection
of the tropical spans.

More specifically, we note $a\in\mathbb{R}_{\max}^{d}$ the eigenvector
corresponding to $\rho(T)$, and $\mathcal{H}_a$ the (unparameterized) tropical hyperplane as defined by equation \ref{eq:unsigned}. We project all points of $X^{+}$ and
$X^{-}$ whose distance to $\mathcal{H}_{a}$ is less than $\rho(T)$,
onto $\mathcal{H}_{a}$. Then the intersection of new convex hulls
is of empty interior.
\end{prop}

\begin{proof}
The distance between any point $x$ in $\mathbb{R}_{\max}^d$ and the tropical hyperplane of apex $a$ is:
\[ d_H(x, \mathcal{H}_a) = \max(x-a) - \text{max}_2 (x-a), \]
where $\max_2$ stands for the second maximal value of any vector.

Considering $X$ the cloud consisting of all points (regardless of
sign), and for $x_{j}\in X$, we note $s_{j}$
its sector and $d_{j}$ the second argmax of $(x_{j}-a)$. For each
point $x_{j}=X_{\cdot j}\in X$ at distance less than $\rho(T)$ of
$\mathcal{H}_{a}$, the transformation described above consists in setting 
\[
W_{kj}\coloneqq\begin{cases}
X_{kj} & \text{if \ensuremath{k\ne s_{j}}}\\
X_{s_{j}j}-d_H(x_{j},\mathcal{H}_{a}) & \text{at \ensuremath{s_{j}}}
\end{cases}
\]
so that $w_{j}$ is projected on the hyperplane. As $T^{\pm}$ is
non-expansive and diagonal-free, let's remark that for $x^{\pm}\in V^{\pm}$:
\[
x_{i}^{\pm}\le T^{\pm}(x^{\pm})_{i}=\left(T^{\pm}(x^{\pm})-T^{\pm}(a)\right)_{i}+T^{\pm}(a)_{i},
\]
hence
\begin{equation}
\label{eq31}
x_{i}^{\pm}\le\max(x^{\pm}-a)_{\ne i}+T^{\pm}(a)_{i}. 
\end{equation}
Let $1\le i\le d$. If $x_{j}\in X$ is not in the $i$-th sector, then
for $k$ different from the sector of $x_{j}$, by definition: 
\[
(w_{j}-a)_{k}\le(x_{j}-a)_{d_{j}}\le(w_{j}-a)_{s_{j}},
\]
hence 
\[
W_{ij}-\max_{k\ne i}\left(W_{kj}-a_{k}\right)\le a_{i}.
\]
Otherwise, $x_{j}$ is in the $i$-th sector and: 
\[
\max_{k\ne i}\left(W_{kj}-a_{k}\right)=X_{d_{j}j}-a_{d_{j}},
\]
thus 
\[
W_{ij}-\max_{\ne i}\left(w_{j}-a\right)=\left(w_{j}-a\right)_{s_{j}}-\left(x_{j}-a\right)_{d_{j}}+a_{s_{j}}\ge a_{s_{j}}=a_{i},
\]
with equality iff $d_H(x_{j},\mathcal{H}_{a})\leq\rho(T)$.

Suppose by symmetry that $T(a)_{i}=T^{+}(a)_{i}=\rho(T)+a_{i}.$ We
also have $T^{-}(a)_{i}\ge\rho(T)+a_{i}$. Then, using the proof of
Theorem 22 in \cite{Akian2021TropicalLR}, we know that there is
$j^{+}$, $j^{-}$ in $\{1,\ldots, p\}$ such that $x_{j^{+}}\in X^{+}$ and $x_{j^{-}}\in X^{-}$
are in sector $i$, with $x_{j^{+}}$ being at distance $\rho(T)$
from $\mathcal{H}_{a}$ and $x_{j^{-}}$ at distance greater than $\rho(T)$.
Therefore, $W_{ij^{+}}-\max_{\ne i}\left(w_{j^{+}}-a\right)=a_{i}$
and $W_{ij^{-}}-\max_{\ne i}\left(w_{j^{-}}-a\right)\geq a_{i}$.
Moreover, for any $j$ such that $x_{j}\in X^{+}$ is in sector $i$,
equation \ref{eq31} gives $d_H(x_{j},\mathcal{H}_{a})\leq\rho(T)$.

Let $Q^{\pm}$ be the \emph{diagonal-free} projections over transformed
point clouds, and $Q=\inf(Q^{+}, Q^{-}).$ We've just shown that $Q(a)_{i}=Q^{+}(a)_{i}=a_{i}$,
and finally $Q(a)=a$.
\end{proof}
%
Thus, the spectral radius can be interpreted as a bound on the distance
under which some points have to be moved to separate the interiors of convex hulls.

As $a$ is the center of the inner ball of $\mathcal{V}^{+}\cap\mathcal{V}^{-}$,
we also have a purely geometric heuristic for determining a
hyperplane in the non-separable case: we simply take the hyperplane
$\mathcal{H}$ of apex $a$ and then assign the sectors based on dominant
population, for instance.


\section{Multiclass classification}

In this section, we consider convex hulls of $n$
point classes, noted $(\mathcal{V}^{k})_{1\le k\le n}$ and described by Shapley operators
$T^{k}$. We give a simple criterion for these classes to be separable by a tropical hyperplane, meaning that there exists a tropical signed
hyperplane such that each $\mathcal{V}^{k}$ belong to sectors of $I^{k}\subset\{1,\ldots, d\}$
with $(I^{k})_{1\le k \le n}$ being pairwaise disjoint. We then characterize the margin as another spectral radius in this case.

We now consider the Shapley operator
\[
T\coloneqq\max_{1\le k<l\le n}\min(T^{k}, T^{l}),
\]

which can be rewritten, for any $x$ in $\mathbb{R}_{\max}^d$ and $i$ in $\{1,\ldots, d\}$:
\[
T(x)_i\coloneqq\text{max}_2\, \{T^k(x)_i, \,\, 1\le k\le n\}.
\]


Morally, we can think of $\mathcal{S}(T)$ as the union of pairwise intersections of data classes, although the max operator does not strictly speaking correspond to a union. We also remark that when $n=2$, $T$ is the same as previously
defined.

Let $(a,\rho(T))$ be the eigenpair of $T$ approximated by the Kranoselskii-Mann
algorithm, verifying $T(a)=\rho(T)+a$. We define the sectors:
\[
I^{k}\coloneqq\{1\le i\le d,\quad T^{k}(a)_{i}>\rho(T)+a_{i}\}.
\]

Given the definition of $T$, it is clear that these sets are disjoint, and we have the following result:
\begin{prop}\label{prop:MultiClass}
If $\rho(T)\le 0$, the tropical parameterized hyperplane of apex $a$ and configuration $\sigma$,
given the configuration $\sigma=\{I^{k}\}_{1\le k\le n}$, separates all point clouds with a margin of at least $-\rho(T)$.

%Moreover, this margin is reached somewhere in the case where all $T^{k}$ are
%of the form
%\[
%T^{k}(x)=P_{\mathcal{V}^{k}}(x)=\sup_{v\in \mathcal{V}^{k}}\left(v_{i}+\min(-v+x)\right),
%\]
% which is in particular the case when separating finite point clouds. 
\end{prop}

\begin{proof}
We consider class $k\in\{1,\ldots, n\}$ and sector $\mathcal{S}^\ell$ with $\ell \ne k$. For $j \in I^{\ell}$, we have $T^{k}(a)_i\le a_i + \rho(T)$ and therefore can use the same reasoning
as in the proof of Proposition \ref{prop:BinaryHardMargin} to obtain that for all $v^{k}$ in $\mathcal{V}^{k}$,
\[
d_H(\mathcal{S}^\ell,v^{k})=\max(v^{k}-a)-\max_{j\in I^{\ell}}(v^{k}-a)_j\ge-\rho(T),
\]
and $v^k$ cannot belong to sector $\mathcal{S}^\ell$, from which, as again, the sectors are well-defined.
\end{proof}

In practice, our method tends to minimize the minimum margin between two classes. For example, if two classes are very close, and the third is far away, the method won't bother separating the latter from the others as much as possible.

But don't assume that the margin is reached, even between two classes. Indeed, let $i\in\{1,\ldots, d\}$. As $T(a)=\rho(T)+a$, 
From the expression of $T$, noting $k$ and $\ell$ respectively the second argmax and argmax of the family $\left(T^{j}(a)_{i}\right)_{1\le j\le n}$, we have
\[
T^{k}(a)_{i} =\lambda+a_{i}\qquad \text{and}\qquad T^{\ell}(a)_{i} >\lambda+a_{i},
\] %\todo{ceci est vrai si l'on a envoyé l'apex sur +inf dans les coordonnées i où max Tk= max2 Tk %Il faudrait alors modifier le "this margin is reached somewhere" en disant que pour chaque secteur (attribué à une classe), il y a un point d'une classe non attribuée à ce secteur à distance presque la marge}
meaning $i\in \mathcal{I}^{\ell}$. When separating finite point clouds, $T^k$
is of the form
\[
\sup_{v\in\mathcal{V}^{k}}\left(v_{i}+\min(-v+x)\right).
\]
Hence, adapting the ideas of Proposition \ref{prop:BinaryHardMargin}, for $\varepsilon > 0$, we can find $v\in \mathcal{V}^k$ such that $$\max(v^{k}-a)-(v_{i}^{k}-a_{i})\le-\rho(T)+\varepsilon,$$ which means that 
\[
d_H\left(\mathcal{V}^{k}, \mathcal{S}^\ell \right)\le \max(v^{k}-a)-\max_{j\in I_l}(v^{k}-a)_j \le \max(v^{k}-a) - (v^{k}_i - a_i) \le -\rho(T)+\varepsilon.
\]

So, for every sector (assigned to a certain class), there's a point of a class not assigned to that sector within margin distance. In general, nothing says that this situation will be symmetrical between 2 sectors in particular, which would achieve the margin between them. In simple, general cases, more complex relationships can be observed.

Since this method optimizes the minimum margin, it works best on previously balanced data. If necessary, it's still possible to use the binary setting of the previous section with one-vs-all approaches, in order to maximize all margins 2 by 2.

\begin{rem}\label{rem:SectorAssign}
According to Proposition 2, we extend the sector assignment policy as follows: the sector corresponding to the maximum coordinate $i$ is assigned to the class with the greatest number of points at distance $\max(0,-\rho(T))$ from it. Unreached sectors are removed for more sparsity, which preserves the margin, as explained in Remark \ref{rem:UnreachedSectors}
\end{rem}

\subsection*{Conclusion}

We therefore introduce Algorithm \ref{algo:Final} to solve the tropical linear classification problem.

\begin{algorithm}[h!]
\caption{Determining tropical linear classifiers by solving mean payoff games}
\label{algo:Final}
\begin{algorithmic}[1]
  \State \Input: Data classes $X^1,\ldots, X^n$ corresponding to labels $1, \ldots, n$, and a relative convergence criterion $\eta$ for the iteration scheme.
\State $T\coloneqq\max_2 \left(T^k\right)_{1\le k \le n}$
\State $(a,\lambda) \coloneqq \textsc{Krasnoselskii-Mann}(T, \eta)$ \label{step:3}
\State \textbf{assign} sectors $\sigma$ to classes based on Remark \ref{rem:SectorAssign}\\
\Return classifier $H_a^\sigma$ and margin $\max(0, -\lambda)$
\end{algorithmic}
\end{algorithm}

In the separable case, we have margin information, which is exact in the binary setting (as it is reached by the two classes, see Proposition \ref{prop:BinaryHardMargin} and \ref{prop:MultiClass}). In the non-separable case, by Proposition \ref{prop:BinarySoftMargin}, the overlap zone is contained around the (non-parameterized) hyperplane $H_a$ in a tubular envelope of radius $\lambda$, the (positive) eigenvalue approximated by the algorithm.

Let's analyze the complexity of Algorithm \ref{algo:Final}. Step \ref{step:3} typically involves a few dozen iterations, in which $n$ tropical projections have to be calculated each time, for a total cost in $\mathcal{O}(dp)$, where $p$ is the total number of points in the dataset. We therefore have a pseudo-polynomial algorithm for calculating an $\varepsilon$-approximation of the classifier described above.

This gain in efficiency is achieved at the cost that tropical SVMs are ultimately very sensitive to outliers, which count as much as any point in the margin calculation.

\subsection*{Why not tropicalizing classical SVMs?}

Tropical hyperplanes can be obtained as the limit of classical hyperplanes seen on log-log paper, thanks to what we call \emph{Maslov dequantization} \cite{litvinov2005maslov} or Viro's method \cite{viro2000dequantization}. A natural way of doing Tropical SVM is therefore to take our data to the power $\beta > 0$, apply a classical SVM (which would maximize the margin in the new space), then go back to the logarithm. As $\beta$ tends towards infinity, we are approaching some limiting tropical hyperplane. However, this method is a source of significant numerical error, especially when $\beta$ is large. Moreover, it does not guarantee a good margin in the starting space, as illustrated in Figure \ref{fig:LogLog}, where the hypersurface in yellow tends towards the suboptimal tropical one in red.

\begin{figure}[!h]
    \centering
    \begin{subfigure}{0.5\textwidth}
        \centering
        \resizebox{\linewidth}{!}{%
        \centering
            \clipbox{0.3\width{} 0.3\height{} 0.2\width{} 0.3\height{}}{%% Creator: Matplotlib, PGF backend
%%
%% To include the figure in your LaTeX document, write
%%   \input{<filename>.pgf}
%%
%% Make sure the required packages are loaded in your preamble
%%   \usepackage{pgf}
%%
%% Also ensure that all the required font packages are loaded; for instance,
%% the lmodern package is sometimes necessary when using math font.
%%   \usepackage{lmodern}
%%
%% Figures using additional raster images can only be included by \input if
%% they are in the same directory as the main LaTeX file. For loading figures
%% from other directories you can use the `import` package
%%   \usepackage{import}
%%
%% and then include the figures with
%%   \import{<path to file>}{<filename>.pgf}
%%
%% Matplotlib used the following preamble
%%   
%%   \usepackage{fontspec}
%%   \setmainfont{DejaVuSerif.ttf}[Path=\detokenize{/Users/sam/Library/Python/3.9/lib/python/site-packages/matplotlib/mpl-data/fonts/ttf/}]
%%   \setsansfont{DejaVuSans.ttf}[Path=\detokenize{/Users/sam/Library/Python/3.9/lib/python/site-packages/matplotlib/mpl-data/fonts/ttf/}]
%%   \setmonofont{DejaVuSansMono.ttf}[Path=\detokenize{/Users/sam/Library/Python/3.9/lib/python/site-packages/matplotlib/mpl-data/fonts/ttf/}]
%%   \makeatletter\@ifpackageloaded{underscore}{}{\usepackage[strings]{underscore}}\makeatother
%%
\begingroup%
\makeatletter%
\begin{pgfpicture}%
\pgfpathrectangle{\pgfpointorigin}{\pgfqpoint{5.000000in}{5.000000in}}%
\pgfusepath{use as bounding box, clip}%
\begin{pgfscope}%
\pgfsetbuttcap%
\pgfsetmiterjoin%
\definecolor{currentfill}{rgb}{1.000000,1.000000,1.000000}%
\pgfsetfillcolor{currentfill}%
\pgfsetlinewidth{0.000000pt}%
\definecolor{currentstroke}{rgb}{1.000000,1.000000,1.000000}%
\pgfsetstrokecolor{currentstroke}%
\pgfsetdash{}{0pt}%
\pgfpathmoveto{\pgfqpoint{0.000000in}{0.000000in}}%
\pgfpathlineto{\pgfqpoint{5.000000in}{0.000000in}}%
\pgfpathlineto{\pgfqpoint{5.000000in}{5.000000in}}%
\pgfpathlineto{\pgfqpoint{0.000000in}{5.000000in}}%
\pgfpathlineto{\pgfqpoint{0.000000in}{0.000000in}}%
\pgfpathclose%
\pgfusepath{fill}%
\end{pgfscope}%
\begin{pgfscope}%
\pgfsetbuttcap%
\pgfsetmiterjoin%
\definecolor{currentfill}{rgb}{1.000000,1.000000,1.000000}%
\pgfsetfillcolor{currentfill}%
\pgfsetlinewidth{0.000000pt}%
\definecolor{currentstroke}{rgb}{0.000000,0.000000,0.000000}%
\pgfsetstrokecolor{currentstroke}%
\pgfsetstrokeopacity{0.000000}%
\pgfsetdash{}{0pt}%
\pgfpathmoveto{\pgfqpoint{0.637500in}{0.550000in}}%
\pgfpathlineto{\pgfqpoint{4.487500in}{0.550000in}}%
\pgfpathlineto{\pgfqpoint{4.487500in}{4.400000in}}%
\pgfpathlineto{\pgfqpoint{0.637500in}{4.400000in}}%
\pgfpathlineto{\pgfqpoint{0.637500in}{0.550000in}}%
\pgfpathclose%
\pgfusepath{fill}%
\end{pgfscope}%
\begin{pgfscope}%
\pgfpathrectangle{\pgfqpoint{0.637500in}{0.550000in}}{\pgfqpoint{3.850000in}{3.850000in}}%
\pgfusepath{clip}%
\pgfsetbuttcap%
\pgfsetroundjoin%
\definecolor{currentfill}{rgb}{0.935716,0.605463,0.000000}%
\pgfsetfillcolor{currentfill}%
\pgfsetfillopacity{0.800000}%
\pgfsetlinewidth{0.000000pt}%
\definecolor{currentstroke}{rgb}{0.000000,0.000000,0.000000}%
\pgfsetstrokecolor{currentstroke}%
\pgfsetdash{}{0pt}%
\pgfpathmoveto{\pgfqpoint{2.522784in}{2.951455in}}%
\pgfpathlineto{\pgfqpoint{2.452866in}{2.948963in}}%
\pgfpathlineto{\pgfqpoint{2.522784in}{2.949257in}}%
\pgfpathlineto{\pgfqpoint{2.592703in}{2.958294in}}%
\pgfpathlineto{\pgfqpoint{2.522784in}{2.951455in}}%
\pgfpathclose%
\pgfusepath{fill}%
\end{pgfscope}%
\begin{pgfscope}%
\pgfpathrectangle{\pgfqpoint{0.637500in}{0.550000in}}{\pgfqpoint{3.850000in}{3.850000in}}%
\pgfusepath{clip}%
\pgfsetbuttcap%
\pgfsetroundjoin%
\definecolor{currentfill}{rgb}{0.934964,0.604977,0.000000}%
\pgfsetfillcolor{currentfill}%
\pgfsetfillopacity{0.800000}%
\pgfsetlinewidth{0.000000pt}%
\definecolor{currentstroke}{rgb}{0.000000,0.000000,0.000000}%
\pgfsetstrokecolor{currentstroke}%
\pgfsetdash{}{0pt}%
\pgfpathmoveto{\pgfqpoint{2.452866in}{2.948963in}}%
\pgfpathlineto{\pgfqpoint{2.382947in}{2.953090in}}%
\pgfpathlineto{\pgfqpoint{2.452866in}{2.946764in}}%
\pgfpathlineto{\pgfqpoint{2.522784in}{2.949257in}}%
\pgfpathlineto{\pgfqpoint{2.452866in}{2.948963in}}%
\pgfpathclose%
\pgfusepath{fill}%
\end{pgfscope}%
\begin{pgfscope}%
\pgfpathrectangle{\pgfqpoint{0.637500in}{0.550000in}}{\pgfqpoint{3.850000in}{3.850000in}}%
\pgfusepath{clip}%
\pgfsetbuttcap%
\pgfsetroundjoin%
\definecolor{currentfill}{rgb}{0.933916,0.604299,0.000000}%
\pgfsetfillcolor{currentfill}%
\pgfsetfillopacity{0.800000}%
\pgfsetlinewidth{0.000000pt}%
\definecolor{currentstroke}{rgb}{0.000000,0.000000,0.000000}%
\pgfsetstrokecolor{currentstroke}%
\pgfsetdash{}{0pt}%
\pgfpathmoveto{\pgfqpoint{2.592703in}{2.958294in}}%
\pgfpathlineto{\pgfqpoint{2.522784in}{2.949257in}}%
\pgfpathlineto{\pgfqpoint{2.592703in}{2.956095in}}%
\pgfpathlineto{\pgfqpoint{2.662622in}{2.971264in}}%
\pgfpathlineto{\pgfqpoint{2.592703in}{2.958294in}}%
\pgfpathclose%
\pgfusepath{fill}%
\end{pgfscope}%
\begin{pgfscope}%
\pgfpathrectangle{\pgfqpoint{0.637500in}{0.550000in}}{\pgfqpoint{3.850000in}{3.850000in}}%
\pgfusepath{clip}%
\pgfsetbuttcap%
\pgfsetroundjoin%
\definecolor{currentfill}{rgb}{0.935716,0.605463,0.000000}%
\pgfsetfillcolor{currentfill}%
\pgfsetfillopacity{0.800000}%
\pgfsetlinewidth{0.000000pt}%
\definecolor{currentstroke}{rgb}{0.000000,0.000000,0.000000}%
\pgfsetstrokecolor{currentstroke}%
\pgfsetdash{}{0pt}%
\pgfpathmoveto{\pgfqpoint{2.522784in}{2.949257in}}%
\pgfpathlineto{\pgfqpoint{2.452866in}{2.946764in}}%
\pgfpathlineto{\pgfqpoint{2.522784in}{2.947058in}}%
\pgfpathlineto{\pgfqpoint{2.592703in}{2.956095in}}%
\pgfpathlineto{\pgfqpoint{2.522784in}{2.949257in}}%
\pgfpathclose%
\pgfusepath{fill}%
\end{pgfscope}%
\begin{pgfscope}%
\pgfpathrectangle{\pgfqpoint{0.637500in}{0.550000in}}{\pgfqpoint{3.850000in}{3.850000in}}%
\pgfusepath{clip}%
\pgfsetbuttcap%
\pgfsetroundjoin%
\definecolor{currentfill}{rgb}{0.931871,0.602975,0.000000}%
\pgfsetfillcolor{currentfill}%
\pgfsetfillopacity{0.800000}%
\pgfsetlinewidth{0.000000pt}%
\definecolor{currentstroke}{rgb}{0.000000,0.000000,0.000000}%
\pgfsetstrokecolor{currentstroke}%
\pgfsetdash{}{0pt}%
\pgfpathmoveto{\pgfqpoint{2.382947in}{2.953090in}}%
\pgfpathlineto{\pgfqpoint{2.313028in}{2.963560in}}%
\pgfpathlineto{\pgfqpoint{2.382947in}{2.950891in}}%
\pgfpathlineto{\pgfqpoint{2.452866in}{2.946764in}}%
\pgfpathlineto{\pgfqpoint{2.382947in}{2.953090in}}%
\pgfpathclose%
\pgfusepath{fill}%
\end{pgfscope}%
\begin{pgfscope}%
\pgfpathrectangle{\pgfqpoint{0.637500in}{0.550000in}}{\pgfqpoint{3.850000in}{3.850000in}}%
\pgfusepath{clip}%
\pgfsetbuttcap%
\pgfsetroundjoin%
\definecolor{currentfill}{rgb}{0.930057,0.601801,0.000000}%
\pgfsetfillcolor{currentfill}%
\pgfsetfillopacity{0.800000}%
\pgfsetlinewidth{0.000000pt}%
\definecolor{currentstroke}{rgb}{0.000000,0.000000,0.000000}%
\pgfsetstrokecolor{currentstroke}%
\pgfsetdash{}{0pt}%
\pgfpathmoveto{\pgfqpoint{2.662622in}{2.971264in}}%
\pgfpathlineto{\pgfqpoint{2.592703in}{2.956095in}}%
\pgfpathlineto{\pgfqpoint{2.662622in}{2.969066in}}%
\pgfpathlineto{\pgfqpoint{2.732540in}{2.989695in}}%
\pgfpathlineto{\pgfqpoint{2.662622in}{2.971264in}}%
\pgfpathclose%
\pgfusepath{fill}%
\end{pgfscope}%
\begin{pgfscope}%
\pgfpathrectangle{\pgfqpoint{0.637500in}{0.550000in}}{\pgfqpoint{3.850000in}{3.850000in}}%
\pgfusepath{clip}%
\pgfsetbuttcap%
\pgfsetroundjoin%
\definecolor{currentfill}{rgb}{0.934964,0.604977,0.000000}%
\pgfsetfillcolor{currentfill}%
\pgfsetfillopacity{0.800000}%
\pgfsetlinewidth{0.000000pt}%
\definecolor{currentstroke}{rgb}{0.000000,0.000000,0.000000}%
\pgfsetstrokecolor{currentstroke}%
\pgfsetdash{}{0pt}%
\pgfpathmoveto{\pgfqpoint{2.452866in}{2.946764in}}%
\pgfpathlineto{\pgfqpoint{2.382947in}{2.950891in}}%
\pgfpathlineto{\pgfqpoint{2.452866in}{2.944565in}}%
\pgfpathlineto{\pgfqpoint{2.522784in}{2.947058in}}%
\pgfpathlineto{\pgfqpoint{2.452866in}{2.946764in}}%
\pgfpathclose%
\pgfusepath{fill}%
\end{pgfscope}%
\begin{pgfscope}%
\pgfpathrectangle{\pgfqpoint{0.637500in}{0.550000in}}{\pgfqpoint{3.850000in}{3.850000in}}%
\pgfusepath{clip}%
\pgfsetbuttcap%
\pgfsetroundjoin%
\definecolor{currentfill}{rgb}{0.933916,0.604299,0.000000}%
\pgfsetfillcolor{currentfill}%
\pgfsetfillopacity{0.800000}%
\pgfsetlinewidth{0.000000pt}%
\definecolor{currentstroke}{rgb}{0.000000,0.000000,0.000000}%
\pgfsetstrokecolor{currentstroke}%
\pgfsetdash{}{0pt}%
\pgfpathmoveto{\pgfqpoint{2.592703in}{2.956095in}}%
\pgfpathlineto{\pgfqpoint{2.522784in}{2.947058in}}%
\pgfpathlineto{\pgfqpoint{2.592703in}{2.953897in}}%
\pgfpathlineto{\pgfqpoint{2.662622in}{2.969066in}}%
\pgfpathlineto{\pgfqpoint{2.592703in}{2.956095in}}%
\pgfpathclose%
\pgfusepath{fill}%
\end{pgfscope}%
\begin{pgfscope}%
\pgfpathrectangle{\pgfqpoint{0.637500in}{0.550000in}}{\pgfqpoint{3.850000in}{3.850000in}}%
\pgfusepath{clip}%
\pgfsetbuttcap%
\pgfsetroundjoin%
\definecolor{currentfill}{rgb}{0.927227,0.599970,0.000000}%
\pgfsetfillcolor{currentfill}%
\pgfsetfillopacity{0.800000}%
\pgfsetlinewidth{0.000000pt}%
\definecolor{currentstroke}{rgb}{0.000000,0.000000,0.000000}%
\pgfsetstrokecolor{currentstroke}%
\pgfsetdash{}{0pt}%
\pgfpathmoveto{\pgfqpoint{2.313028in}{2.963560in}}%
\pgfpathlineto{\pgfqpoint{2.243110in}{2.979795in}}%
\pgfpathlineto{\pgfqpoint{2.313028in}{2.961361in}}%
\pgfpathlineto{\pgfqpoint{2.382947in}{2.950891in}}%
\pgfpathlineto{\pgfqpoint{2.313028in}{2.963560in}}%
\pgfpathclose%
\pgfusepath{fill}%
\end{pgfscope}%
\begin{pgfscope}%
\pgfpathrectangle{\pgfqpoint{0.637500in}{0.550000in}}{\pgfqpoint{3.850000in}{3.850000in}}%
\pgfusepath{clip}%
\pgfsetbuttcap%
\pgfsetroundjoin%
\definecolor{currentfill}{rgb}{0.925053,0.598564,0.000000}%
\pgfsetfillcolor{currentfill}%
\pgfsetfillopacity{0.800000}%
\pgfsetlinewidth{0.000000pt}%
\definecolor{currentstroke}{rgb}{0.000000,0.000000,0.000000}%
\pgfsetstrokecolor{currentstroke}%
\pgfsetdash{}{0pt}%
\pgfpathmoveto{\pgfqpoint{2.732540in}{2.989695in}}%
\pgfpathlineto{\pgfqpoint{2.662622in}{2.969066in}}%
\pgfpathlineto{\pgfqpoint{2.732540in}{2.987496in}}%
\pgfpathlineto{\pgfqpoint{2.802459in}{3.012774in}}%
\pgfpathlineto{\pgfqpoint{2.732540in}{2.989695in}}%
\pgfpathclose%
\pgfusepath{fill}%
\end{pgfscope}%
\begin{pgfscope}%
\pgfpathrectangle{\pgfqpoint{0.637500in}{0.550000in}}{\pgfqpoint{3.850000in}{3.850000in}}%
\pgfusepath{clip}%
\pgfsetbuttcap%
\pgfsetroundjoin%
\definecolor{currentfill}{rgb}{0.935716,0.605463,0.000000}%
\pgfsetfillcolor{currentfill}%
\pgfsetfillopacity{0.800000}%
\pgfsetlinewidth{0.000000pt}%
\definecolor{currentstroke}{rgb}{0.000000,0.000000,0.000000}%
\pgfsetstrokecolor{currentstroke}%
\pgfsetdash{}{0pt}%
\pgfpathmoveto{\pgfqpoint{2.522784in}{2.947058in}}%
\pgfpathlineto{\pgfqpoint{2.452866in}{2.944565in}}%
\pgfpathlineto{\pgfqpoint{2.522784in}{2.944859in}}%
\pgfpathlineto{\pgfqpoint{2.592703in}{2.953897in}}%
\pgfpathlineto{\pgfqpoint{2.522784in}{2.947058in}}%
\pgfpathclose%
\pgfusepath{fill}%
\end{pgfscope}%
\begin{pgfscope}%
\pgfpathrectangle{\pgfqpoint{0.637500in}{0.550000in}}{\pgfqpoint{3.850000in}{3.850000in}}%
\pgfusepath{clip}%
\pgfsetbuttcap%
\pgfsetroundjoin%
\definecolor{currentfill}{rgb}{0.931871,0.602975,0.000000}%
\pgfsetfillcolor{currentfill}%
\pgfsetfillopacity{0.800000}%
\pgfsetlinewidth{0.000000pt}%
\definecolor{currentstroke}{rgb}{0.000000,0.000000,0.000000}%
\pgfsetstrokecolor{currentstroke}%
\pgfsetdash{}{0pt}%
\pgfpathmoveto{\pgfqpoint{2.382947in}{2.950891in}}%
\pgfpathlineto{\pgfqpoint{2.313028in}{2.961361in}}%
\pgfpathlineto{\pgfqpoint{2.382947in}{2.948693in}}%
\pgfpathlineto{\pgfqpoint{2.452866in}{2.944565in}}%
\pgfpathlineto{\pgfqpoint{2.382947in}{2.950891in}}%
\pgfpathclose%
\pgfusepath{fill}%
\end{pgfscope}%
\begin{pgfscope}%
\pgfpathrectangle{\pgfqpoint{0.637500in}{0.550000in}}{\pgfqpoint{3.850000in}{3.850000in}}%
\pgfusepath{clip}%
\pgfsetbuttcap%
\pgfsetroundjoin%
\definecolor{currentfill}{rgb}{0.930057,0.601801,0.000000}%
\pgfsetfillcolor{currentfill}%
\pgfsetfillopacity{0.800000}%
\pgfsetlinewidth{0.000000pt}%
\definecolor{currentstroke}{rgb}{0.000000,0.000000,0.000000}%
\pgfsetstrokecolor{currentstroke}%
\pgfsetdash{}{0pt}%
\pgfpathmoveto{\pgfqpoint{2.662622in}{2.969066in}}%
\pgfpathlineto{\pgfqpoint{2.592703in}{2.953897in}}%
\pgfpathlineto{\pgfqpoint{2.662622in}{2.966867in}}%
\pgfpathlineto{\pgfqpoint{2.732540in}{2.987496in}}%
\pgfpathlineto{\pgfqpoint{2.662622in}{2.969066in}}%
\pgfpathclose%
\pgfusepath{fill}%
\end{pgfscope}%
\begin{pgfscope}%
\pgfpathrectangle{\pgfqpoint{0.637500in}{0.550000in}}{\pgfqpoint{3.850000in}{3.850000in}}%
\pgfusepath{clip}%
\pgfsetbuttcap%
\pgfsetroundjoin%
\definecolor{currentfill}{rgb}{0.921990,0.596582,0.000000}%
\pgfsetfillcolor{currentfill}%
\pgfsetfillopacity{0.800000}%
\pgfsetlinewidth{0.000000pt}%
\definecolor{currentstroke}{rgb}{0.000000,0.000000,0.000000}%
\pgfsetstrokecolor{currentstroke}%
\pgfsetdash{}{0pt}%
\pgfpathmoveto{\pgfqpoint{2.243110in}{2.979795in}}%
\pgfpathlineto{\pgfqpoint{2.173191in}{3.001029in}}%
\pgfpathlineto{\pgfqpoint{2.243110in}{2.977596in}}%
\pgfpathlineto{\pgfqpoint{2.313028in}{2.961361in}}%
\pgfpathlineto{\pgfqpoint{2.243110in}{2.979795in}}%
\pgfpathclose%
\pgfusepath{fill}%
\end{pgfscope}%
\begin{pgfscope}%
\pgfpathrectangle{\pgfqpoint{0.637500in}{0.550000in}}{\pgfqpoint{3.850000in}{3.850000in}}%
\pgfusepath{clip}%
\pgfsetbuttcap%
\pgfsetroundjoin%
\definecolor{currentfill}{rgb}{0.919814,0.595174,0.000000}%
\pgfsetfillcolor{currentfill}%
\pgfsetfillopacity{0.800000}%
\pgfsetlinewidth{0.000000pt}%
\definecolor{currentstroke}{rgb}{0.000000,0.000000,0.000000}%
\pgfsetstrokecolor{currentstroke}%
\pgfsetdash{}{0pt}%
\pgfpathmoveto{\pgfqpoint{2.802459in}{3.012774in}}%
\pgfpathlineto{\pgfqpoint{2.732540in}{2.987496in}}%
\pgfpathlineto{\pgfqpoint{2.802459in}{3.010575in}}%
\pgfpathlineto{\pgfqpoint{2.872378in}{3.039660in}}%
\pgfpathlineto{\pgfqpoint{2.802459in}{3.012774in}}%
\pgfpathclose%
\pgfusepath{fill}%
\end{pgfscope}%
\begin{pgfscope}%
\pgfpathrectangle{\pgfqpoint{0.637500in}{0.550000in}}{\pgfqpoint{3.850000in}{3.850000in}}%
\pgfusepath{clip}%
\pgfsetbuttcap%
\pgfsetroundjoin%
\definecolor{currentfill}{rgb}{0.934964,0.604977,0.000000}%
\pgfsetfillcolor{currentfill}%
\pgfsetfillopacity{0.800000}%
\pgfsetlinewidth{0.000000pt}%
\definecolor{currentstroke}{rgb}{0.000000,0.000000,0.000000}%
\pgfsetstrokecolor{currentstroke}%
\pgfsetdash{}{0pt}%
\pgfpathmoveto{\pgfqpoint{2.452866in}{2.944565in}}%
\pgfpathlineto{\pgfqpoint{2.382947in}{2.948693in}}%
\pgfpathlineto{\pgfqpoint{2.452866in}{2.942366in}}%
\pgfpathlineto{\pgfqpoint{2.522784in}{2.944859in}}%
\pgfpathlineto{\pgfqpoint{2.452866in}{2.944565in}}%
\pgfpathclose%
\pgfusepath{fill}%
\end{pgfscope}%
\begin{pgfscope}%
\pgfpathrectangle{\pgfqpoint{0.637500in}{0.550000in}}{\pgfqpoint{3.850000in}{3.850000in}}%
\pgfusepath{clip}%
\pgfsetbuttcap%
\pgfsetroundjoin%
\definecolor{currentfill}{rgb}{0.933916,0.604299,0.000000}%
\pgfsetfillcolor{currentfill}%
\pgfsetfillopacity{0.800000}%
\pgfsetlinewidth{0.000000pt}%
\definecolor{currentstroke}{rgb}{0.000000,0.000000,0.000000}%
\pgfsetstrokecolor{currentstroke}%
\pgfsetdash{}{0pt}%
\pgfpathmoveto{\pgfqpoint{2.592703in}{2.953897in}}%
\pgfpathlineto{\pgfqpoint{2.522784in}{2.944859in}}%
\pgfpathlineto{\pgfqpoint{2.592703in}{2.951698in}}%
\pgfpathlineto{\pgfqpoint{2.662622in}{2.966867in}}%
\pgfpathlineto{\pgfqpoint{2.592703in}{2.953897in}}%
\pgfpathclose%
\pgfusepath{fill}%
\end{pgfscope}%
\begin{pgfscope}%
\pgfpathrectangle{\pgfqpoint{0.637500in}{0.550000in}}{\pgfqpoint{3.850000in}{3.850000in}}%
\pgfusepath{clip}%
\pgfsetbuttcap%
\pgfsetroundjoin%
\definecolor{currentfill}{rgb}{0.927227,0.599970,0.000000}%
\pgfsetfillcolor{currentfill}%
\pgfsetfillopacity{0.800000}%
\pgfsetlinewidth{0.000000pt}%
\definecolor{currentstroke}{rgb}{0.000000,0.000000,0.000000}%
\pgfsetstrokecolor{currentstroke}%
\pgfsetdash{}{0pt}%
\pgfpathmoveto{\pgfqpoint{2.313028in}{2.961361in}}%
\pgfpathlineto{\pgfqpoint{2.243110in}{2.977596in}}%
\pgfpathlineto{\pgfqpoint{2.313028in}{2.959162in}}%
\pgfpathlineto{\pgfqpoint{2.382947in}{2.948693in}}%
\pgfpathlineto{\pgfqpoint{2.313028in}{2.961361in}}%
\pgfpathclose%
\pgfusepath{fill}%
\end{pgfscope}%
\begin{pgfscope}%
\pgfpathrectangle{\pgfqpoint{0.637500in}{0.550000in}}{\pgfqpoint{3.850000in}{3.850000in}}%
\pgfusepath{clip}%
\pgfsetbuttcap%
\pgfsetroundjoin%
\definecolor{currentfill}{rgb}{0.925053,0.598564,0.000000}%
\pgfsetfillcolor{currentfill}%
\pgfsetfillopacity{0.800000}%
\pgfsetlinewidth{0.000000pt}%
\definecolor{currentstroke}{rgb}{0.000000,0.000000,0.000000}%
\pgfsetstrokecolor{currentstroke}%
\pgfsetdash{}{0pt}%
\pgfpathmoveto{\pgfqpoint{2.732540in}{2.987496in}}%
\pgfpathlineto{\pgfqpoint{2.662622in}{2.966867in}}%
\pgfpathlineto{\pgfqpoint{2.732540in}{2.985297in}}%
\pgfpathlineto{\pgfqpoint{2.802459in}{3.010575in}}%
\pgfpathlineto{\pgfqpoint{2.732540in}{2.987496in}}%
\pgfpathclose%
\pgfusepath{fill}%
\end{pgfscope}%
\begin{pgfscope}%
\pgfpathrectangle{\pgfqpoint{0.637500in}{0.550000in}}{\pgfqpoint{3.850000in}{3.850000in}}%
\pgfusepath{clip}%
\pgfsetbuttcap%
\pgfsetroundjoin%
\definecolor{currentfill}{rgb}{0.916926,0.593305,0.000000}%
\pgfsetfillcolor{currentfill}%
\pgfsetfillopacity{0.800000}%
\pgfsetlinewidth{0.000000pt}%
\definecolor{currentstroke}{rgb}{0.000000,0.000000,0.000000}%
\pgfsetstrokecolor{currentstroke}%
\pgfsetdash{}{0pt}%
\pgfpathmoveto{\pgfqpoint{2.173191in}{3.001029in}}%
\pgfpathlineto{\pgfqpoint{2.103272in}{3.026419in}}%
\pgfpathlineto{\pgfqpoint{2.173191in}{2.998830in}}%
\pgfpathlineto{\pgfqpoint{2.243110in}{2.977596in}}%
\pgfpathlineto{\pgfqpoint{2.173191in}{3.001029in}}%
\pgfpathclose%
\pgfusepath{fill}%
\end{pgfscope}%
\begin{pgfscope}%
\pgfpathrectangle{\pgfqpoint{0.637500in}{0.550000in}}{\pgfqpoint{3.850000in}{3.850000in}}%
\pgfusepath{clip}%
\pgfsetbuttcap%
\pgfsetroundjoin%
\definecolor{currentfill}{rgb}{0.914972,0.592040,0.000000}%
\pgfsetfillcolor{currentfill}%
\pgfsetfillopacity{0.800000}%
\pgfsetlinewidth{0.000000pt}%
\definecolor{currentstroke}{rgb}{0.000000,0.000000,0.000000}%
\pgfsetstrokecolor{currentstroke}%
\pgfsetdash{}{0pt}%
\pgfpathmoveto{\pgfqpoint{2.872378in}{3.039660in}}%
\pgfpathlineto{\pgfqpoint{2.802459in}{3.010575in}}%
\pgfpathlineto{\pgfqpoint{2.872378in}{3.037461in}}%
\pgfpathlineto{\pgfqpoint{2.942296in}{3.069565in}}%
\pgfpathlineto{\pgfqpoint{2.872378in}{3.039660in}}%
\pgfpathclose%
\pgfusepath{fill}%
\end{pgfscope}%
\begin{pgfscope}%
\pgfpathrectangle{\pgfqpoint{0.637500in}{0.550000in}}{\pgfqpoint{3.850000in}{3.850000in}}%
\pgfusepath{clip}%
\pgfsetbuttcap%
\pgfsetroundjoin%
\definecolor{currentfill}{rgb}{0.935716,0.605463,0.000000}%
\pgfsetfillcolor{currentfill}%
\pgfsetfillopacity{0.800000}%
\pgfsetlinewidth{0.000000pt}%
\definecolor{currentstroke}{rgb}{0.000000,0.000000,0.000000}%
\pgfsetstrokecolor{currentstroke}%
\pgfsetdash{}{0pt}%
\pgfpathmoveto{\pgfqpoint{2.522784in}{2.944859in}}%
\pgfpathlineto{\pgfqpoint{2.452866in}{2.942366in}}%
\pgfpathlineto{\pgfqpoint{2.522784in}{2.942660in}}%
\pgfpathlineto{\pgfqpoint{2.592703in}{2.951698in}}%
\pgfpathlineto{\pgfqpoint{2.522784in}{2.944859in}}%
\pgfpathclose%
\pgfusepath{fill}%
\end{pgfscope}%
\begin{pgfscope}%
\pgfpathrectangle{\pgfqpoint{0.637500in}{0.550000in}}{\pgfqpoint{3.850000in}{3.850000in}}%
\pgfusepath{clip}%
\pgfsetbuttcap%
\pgfsetroundjoin%
\definecolor{currentfill}{rgb}{0.931871,0.602975,0.000000}%
\pgfsetfillcolor{currentfill}%
\pgfsetfillopacity{0.800000}%
\pgfsetlinewidth{0.000000pt}%
\definecolor{currentstroke}{rgb}{0.000000,0.000000,0.000000}%
\pgfsetstrokecolor{currentstroke}%
\pgfsetdash{}{0pt}%
\pgfpathmoveto{\pgfqpoint{2.382947in}{2.948693in}}%
\pgfpathlineto{\pgfqpoint{2.313028in}{2.959162in}}%
\pgfpathlineto{\pgfqpoint{2.382947in}{2.946494in}}%
\pgfpathlineto{\pgfqpoint{2.452866in}{2.942366in}}%
\pgfpathlineto{\pgfqpoint{2.382947in}{2.948693in}}%
\pgfpathclose%
\pgfusepath{fill}%
\end{pgfscope}%
\begin{pgfscope}%
\pgfpathrectangle{\pgfqpoint{0.637500in}{0.550000in}}{\pgfqpoint{3.850000in}{3.850000in}}%
\pgfusepath{clip}%
\pgfsetbuttcap%
\pgfsetroundjoin%
\definecolor{currentfill}{rgb}{0.930057,0.601801,0.000000}%
\pgfsetfillcolor{currentfill}%
\pgfsetfillopacity{0.800000}%
\pgfsetlinewidth{0.000000pt}%
\definecolor{currentstroke}{rgb}{0.000000,0.000000,0.000000}%
\pgfsetstrokecolor{currentstroke}%
\pgfsetdash{}{0pt}%
\pgfpathmoveto{\pgfqpoint{2.662622in}{2.966867in}}%
\pgfpathlineto{\pgfqpoint{2.592703in}{2.951698in}}%
\pgfpathlineto{\pgfqpoint{2.662622in}{2.964668in}}%
\pgfpathlineto{\pgfqpoint{2.732540in}{2.985297in}}%
\pgfpathlineto{\pgfqpoint{2.662622in}{2.966867in}}%
\pgfpathclose%
\pgfusepath{fill}%
\end{pgfscope}%
\begin{pgfscope}%
\pgfpathrectangle{\pgfqpoint{0.637500in}{0.550000in}}{\pgfqpoint{3.850000in}{3.850000in}}%
\pgfusepath{clip}%
\pgfsetbuttcap%
\pgfsetroundjoin%
\definecolor{currentfill}{rgb}{0.921990,0.596582,0.000000}%
\pgfsetfillcolor{currentfill}%
\pgfsetfillopacity{0.800000}%
\pgfsetlinewidth{0.000000pt}%
\definecolor{currentstroke}{rgb}{0.000000,0.000000,0.000000}%
\pgfsetstrokecolor{currentstroke}%
\pgfsetdash{}{0pt}%
\pgfpathmoveto{\pgfqpoint{2.243110in}{2.977596in}}%
\pgfpathlineto{\pgfqpoint{2.173191in}{2.998830in}}%
\pgfpathlineto{\pgfqpoint{2.243110in}{2.975398in}}%
\pgfpathlineto{\pgfqpoint{2.313028in}{2.959162in}}%
\pgfpathlineto{\pgfqpoint{2.243110in}{2.977596in}}%
\pgfpathclose%
\pgfusepath{fill}%
\end{pgfscope}%
\begin{pgfscope}%
\pgfpathrectangle{\pgfqpoint{0.637500in}{0.550000in}}{\pgfqpoint{3.850000in}{3.850000in}}%
\pgfusepath{clip}%
\pgfsetbuttcap%
\pgfsetroundjoin%
\definecolor{currentfill}{rgb}{0.919814,0.595174,0.000000}%
\pgfsetfillcolor{currentfill}%
\pgfsetfillopacity{0.800000}%
\pgfsetlinewidth{0.000000pt}%
\definecolor{currentstroke}{rgb}{0.000000,0.000000,0.000000}%
\pgfsetstrokecolor{currentstroke}%
\pgfsetdash{}{0pt}%
\pgfpathmoveto{\pgfqpoint{2.802459in}{3.010575in}}%
\pgfpathlineto{\pgfqpoint{2.732540in}{2.985297in}}%
\pgfpathlineto{\pgfqpoint{2.802459in}{3.008376in}}%
\pgfpathlineto{\pgfqpoint{2.872378in}{3.037461in}}%
\pgfpathlineto{\pgfqpoint{2.802459in}{3.010575in}}%
\pgfpathclose%
\pgfusepath{fill}%
\end{pgfscope}%
\begin{pgfscope}%
\pgfpathrectangle{\pgfqpoint{0.637500in}{0.550000in}}{\pgfqpoint{3.850000in}{3.850000in}}%
\pgfusepath{clip}%
\pgfsetbuttcap%
\pgfsetroundjoin%
\definecolor{currentfill}{rgb}{0.912477,0.590427,0.000000}%
\pgfsetfillcolor{currentfill}%
\pgfsetfillopacity{0.800000}%
\pgfsetlinewidth{0.000000pt}%
\definecolor{currentstroke}{rgb}{0.000000,0.000000,0.000000}%
\pgfsetstrokecolor{currentstroke}%
\pgfsetdash{}{0pt}%
\pgfpathmoveto{\pgfqpoint{2.103272in}{3.026419in}}%
\pgfpathlineto{\pgfqpoint{2.033354in}{3.055149in}}%
\pgfpathlineto{\pgfqpoint{2.103272in}{3.024221in}}%
\pgfpathlineto{\pgfqpoint{2.173191in}{2.998830in}}%
\pgfpathlineto{\pgfqpoint{2.103272in}{3.026419in}}%
\pgfpathclose%
\pgfusepath{fill}%
\end{pgfscope}%
\begin{pgfscope}%
\pgfpathrectangle{\pgfqpoint{0.637500in}{0.550000in}}{\pgfqpoint{3.850000in}{3.850000in}}%
\pgfusepath{clip}%
\pgfsetbuttcap%
\pgfsetroundjoin%
\definecolor{currentfill}{rgb}{0.934964,0.604977,0.000000}%
\pgfsetfillcolor{currentfill}%
\pgfsetfillopacity{0.800000}%
\pgfsetlinewidth{0.000000pt}%
\definecolor{currentstroke}{rgb}{0.000000,0.000000,0.000000}%
\pgfsetstrokecolor{currentstroke}%
\pgfsetdash{}{0pt}%
\pgfpathmoveto{\pgfqpoint{2.452866in}{2.942366in}}%
\pgfpathlineto{\pgfqpoint{2.382947in}{2.946494in}}%
\pgfpathlineto{\pgfqpoint{2.452866in}{2.940167in}}%
\pgfpathlineto{\pgfqpoint{2.522784in}{2.942660in}}%
\pgfpathlineto{\pgfqpoint{2.452866in}{2.942366in}}%
\pgfpathclose%
\pgfusepath{fill}%
\end{pgfscope}%
\begin{pgfscope}%
\pgfpathrectangle{\pgfqpoint{0.637500in}{0.550000in}}{\pgfqpoint{3.850000in}{3.850000in}}%
\pgfusepath{clip}%
\pgfsetbuttcap%
\pgfsetroundjoin%
\definecolor{currentfill}{rgb}{0.910843,0.589369,0.000000}%
\pgfsetfillcolor{currentfill}%
\pgfsetfillopacity{0.800000}%
\pgfsetlinewidth{0.000000pt}%
\definecolor{currentstroke}{rgb}{0.000000,0.000000,0.000000}%
\pgfsetstrokecolor{currentstroke}%
\pgfsetdash{}{0pt}%
\pgfpathmoveto{\pgfqpoint{2.942296in}{3.069565in}}%
\pgfpathlineto{\pgfqpoint{2.872378in}{3.037461in}}%
\pgfpathlineto{\pgfqpoint{2.942296in}{3.067366in}}%
\pgfpathlineto{\pgfqpoint{3.012215in}{3.101806in}}%
\pgfpathlineto{\pgfqpoint{2.942296in}{3.069565in}}%
\pgfpathclose%
\pgfusepath{fill}%
\end{pgfscope}%
\begin{pgfscope}%
\pgfpathrectangle{\pgfqpoint{0.637500in}{0.550000in}}{\pgfqpoint{3.850000in}{3.850000in}}%
\pgfusepath{clip}%
\pgfsetbuttcap%
\pgfsetroundjoin%
\definecolor{currentfill}{rgb}{0.933916,0.604299,0.000000}%
\pgfsetfillcolor{currentfill}%
\pgfsetfillopacity{0.800000}%
\pgfsetlinewidth{0.000000pt}%
\definecolor{currentstroke}{rgb}{0.000000,0.000000,0.000000}%
\pgfsetstrokecolor{currentstroke}%
\pgfsetdash{}{0pt}%
\pgfpathmoveto{\pgfqpoint{2.592703in}{2.951698in}}%
\pgfpathlineto{\pgfqpoint{2.522784in}{2.942660in}}%
\pgfpathlineto{\pgfqpoint{2.592703in}{2.949499in}}%
\pgfpathlineto{\pgfqpoint{2.662622in}{2.964668in}}%
\pgfpathlineto{\pgfqpoint{2.592703in}{2.951698in}}%
\pgfpathclose%
\pgfusepath{fill}%
\end{pgfscope}%
\begin{pgfscope}%
\pgfpathrectangle{\pgfqpoint{0.637500in}{0.550000in}}{\pgfqpoint{3.850000in}{3.850000in}}%
\pgfusepath{clip}%
\pgfsetbuttcap%
\pgfsetroundjoin%
\definecolor{currentfill}{rgb}{0.927227,0.599970,0.000000}%
\pgfsetfillcolor{currentfill}%
\pgfsetfillopacity{0.800000}%
\pgfsetlinewidth{0.000000pt}%
\definecolor{currentstroke}{rgb}{0.000000,0.000000,0.000000}%
\pgfsetstrokecolor{currentstroke}%
\pgfsetdash{}{0pt}%
\pgfpathmoveto{\pgfqpoint{2.313028in}{2.959162in}}%
\pgfpathlineto{\pgfqpoint{2.243110in}{2.975398in}}%
\pgfpathlineto{\pgfqpoint{2.313028in}{2.956963in}}%
\pgfpathlineto{\pgfqpoint{2.382947in}{2.946494in}}%
\pgfpathlineto{\pgfqpoint{2.313028in}{2.959162in}}%
\pgfpathclose%
\pgfusepath{fill}%
\end{pgfscope}%
\begin{pgfscope}%
\pgfpathrectangle{\pgfqpoint{0.637500in}{0.550000in}}{\pgfqpoint{3.850000in}{3.850000in}}%
\pgfusepath{clip}%
\pgfsetbuttcap%
\pgfsetroundjoin%
\definecolor{currentfill}{rgb}{0.925053,0.598564,0.000000}%
\pgfsetfillcolor{currentfill}%
\pgfsetfillopacity{0.800000}%
\pgfsetlinewidth{0.000000pt}%
\definecolor{currentstroke}{rgb}{0.000000,0.000000,0.000000}%
\pgfsetstrokecolor{currentstroke}%
\pgfsetdash{}{0pt}%
\pgfpathmoveto{\pgfqpoint{2.732540in}{2.985297in}}%
\pgfpathlineto{\pgfqpoint{2.662622in}{2.964668in}}%
\pgfpathlineto{\pgfqpoint{2.732540in}{2.983099in}}%
\pgfpathlineto{\pgfqpoint{2.802459in}{3.008376in}}%
\pgfpathlineto{\pgfqpoint{2.732540in}{2.985297in}}%
\pgfpathclose%
\pgfusepath{fill}%
\end{pgfscope}%
\begin{pgfscope}%
\pgfpathrectangle{\pgfqpoint{0.637500in}{0.550000in}}{\pgfqpoint{3.850000in}{3.850000in}}%
\pgfusepath{clip}%
\pgfsetbuttcap%
\pgfsetroundjoin%
\definecolor{currentfill}{rgb}{0.916926,0.593305,0.000000}%
\pgfsetfillcolor{currentfill}%
\pgfsetfillopacity{0.800000}%
\pgfsetlinewidth{0.000000pt}%
\definecolor{currentstroke}{rgb}{0.000000,0.000000,0.000000}%
\pgfsetstrokecolor{currentstroke}%
\pgfsetdash{}{0pt}%
\pgfpathmoveto{\pgfqpoint{2.173191in}{2.998830in}}%
\pgfpathlineto{\pgfqpoint{2.103272in}{3.024221in}}%
\pgfpathlineto{\pgfqpoint{2.173191in}{2.996631in}}%
\pgfpathlineto{\pgfqpoint{2.243110in}{2.975398in}}%
\pgfpathlineto{\pgfqpoint{2.173191in}{2.998830in}}%
\pgfpathclose%
\pgfusepath{fill}%
\end{pgfscope}%
\begin{pgfscope}%
\pgfpathrectangle{\pgfqpoint{0.637500in}{0.550000in}}{\pgfqpoint{3.850000in}{3.850000in}}%
\pgfusepath{clip}%
\pgfsetbuttcap%
\pgfsetroundjoin%
\definecolor{currentfill}{rgb}{0.914972,0.592040,0.000000}%
\pgfsetfillcolor{currentfill}%
\pgfsetfillopacity{0.800000}%
\pgfsetlinewidth{0.000000pt}%
\definecolor{currentstroke}{rgb}{0.000000,0.000000,0.000000}%
\pgfsetstrokecolor{currentstroke}%
\pgfsetdash{}{0pt}%
\pgfpathmoveto{\pgfqpoint{2.872378in}{3.037461in}}%
\pgfpathlineto{\pgfqpoint{2.802459in}{3.008376in}}%
\pgfpathlineto{\pgfqpoint{2.872378in}{3.035262in}}%
\pgfpathlineto{\pgfqpoint{2.942296in}{3.067366in}}%
\pgfpathlineto{\pgfqpoint{2.872378in}{3.037461in}}%
\pgfpathclose%
\pgfusepath{fill}%
\end{pgfscope}%
\begin{pgfscope}%
\pgfpathrectangle{\pgfqpoint{0.637500in}{0.550000in}}{\pgfqpoint{3.850000in}{3.850000in}}%
\pgfusepath{clip}%
\pgfsetbuttcap%
\pgfsetroundjoin%
\definecolor{currentfill}{rgb}{0.908810,0.588053,0.000000}%
\pgfsetfillcolor{currentfill}%
\pgfsetfillopacity{0.800000}%
\pgfsetlinewidth{0.000000pt}%
\definecolor{currentstroke}{rgb}{0.000000,0.000000,0.000000}%
\pgfsetstrokecolor{currentstroke}%
\pgfsetdash{}{0pt}%
\pgfpathmoveto{\pgfqpoint{2.033354in}{3.055149in}}%
\pgfpathlineto{\pgfqpoint{1.963435in}{3.086487in}}%
\pgfpathlineto{\pgfqpoint{2.033354in}{3.052950in}}%
\pgfpathlineto{\pgfqpoint{2.103272in}{3.024221in}}%
\pgfpathlineto{\pgfqpoint{2.033354in}{3.055149in}}%
\pgfpathclose%
\pgfusepath{fill}%
\end{pgfscope}%
\begin{pgfscope}%
\pgfpathrectangle{\pgfqpoint{0.637500in}{0.550000in}}{\pgfqpoint{3.850000in}{3.850000in}}%
\pgfusepath{clip}%
\pgfsetbuttcap%
\pgfsetroundjoin%
\definecolor{currentfill}{rgb}{0.935716,0.605463,0.000000}%
\pgfsetfillcolor{currentfill}%
\pgfsetfillopacity{0.800000}%
\pgfsetlinewidth{0.000000pt}%
\definecolor{currentstroke}{rgb}{0.000000,0.000000,0.000000}%
\pgfsetstrokecolor{currentstroke}%
\pgfsetdash{}{0pt}%
\pgfpathmoveto{\pgfqpoint{2.522784in}{2.942660in}}%
\pgfpathlineto{\pgfqpoint{2.452866in}{2.940167in}}%
\pgfpathlineto{\pgfqpoint{2.522784in}{2.940461in}}%
\pgfpathlineto{\pgfqpoint{2.592703in}{2.949499in}}%
\pgfpathlineto{\pgfqpoint{2.522784in}{2.942660in}}%
\pgfpathclose%
\pgfusepath{fill}%
\end{pgfscope}%
\begin{pgfscope}%
\pgfpathrectangle{\pgfqpoint{0.637500in}{0.550000in}}{\pgfqpoint{3.850000in}{3.850000in}}%
\pgfusepath{clip}%
\pgfsetbuttcap%
\pgfsetroundjoin%
\definecolor{currentfill}{rgb}{0.931871,0.602975,0.000000}%
\pgfsetfillcolor{currentfill}%
\pgfsetfillopacity{0.800000}%
\pgfsetlinewidth{0.000000pt}%
\definecolor{currentstroke}{rgb}{0.000000,0.000000,0.000000}%
\pgfsetstrokecolor{currentstroke}%
\pgfsetdash{}{0pt}%
\pgfpathmoveto{\pgfqpoint{2.382947in}{2.946494in}}%
\pgfpathlineto{\pgfqpoint{2.313028in}{2.956963in}}%
\pgfpathlineto{\pgfqpoint{2.382947in}{2.944295in}}%
\pgfpathlineto{\pgfqpoint{2.452866in}{2.940167in}}%
\pgfpathlineto{\pgfqpoint{2.382947in}{2.946494in}}%
\pgfpathclose%
\pgfusepath{fill}%
\end{pgfscope}%
\begin{pgfscope}%
\pgfpathrectangle{\pgfqpoint{0.637500in}{0.550000in}}{\pgfqpoint{3.850000in}{3.850000in}}%
\pgfusepath{clip}%
\pgfsetbuttcap%
\pgfsetroundjoin%
\definecolor{currentfill}{rgb}{0.930057,0.601801,0.000000}%
\pgfsetfillcolor{currentfill}%
\pgfsetfillopacity{0.800000}%
\pgfsetlinewidth{0.000000pt}%
\definecolor{currentstroke}{rgb}{0.000000,0.000000,0.000000}%
\pgfsetstrokecolor{currentstroke}%
\pgfsetdash{}{0pt}%
\pgfpathmoveto{\pgfqpoint{2.662622in}{2.964668in}}%
\pgfpathlineto{\pgfqpoint{2.592703in}{2.949499in}}%
\pgfpathlineto{\pgfqpoint{2.662622in}{2.962469in}}%
\pgfpathlineto{\pgfqpoint{2.732540in}{2.983099in}}%
\pgfpathlineto{\pgfqpoint{2.662622in}{2.964668in}}%
\pgfpathclose%
\pgfusepath{fill}%
\end{pgfscope}%
\begin{pgfscope}%
\pgfpathrectangle{\pgfqpoint{0.637500in}{0.550000in}}{\pgfqpoint{3.850000in}{3.850000in}}%
\pgfusepath{clip}%
\pgfsetbuttcap%
\pgfsetroundjoin%
\definecolor{currentfill}{rgb}{0.921990,0.596582,0.000000}%
\pgfsetfillcolor{currentfill}%
\pgfsetfillopacity{0.800000}%
\pgfsetlinewidth{0.000000pt}%
\definecolor{currentstroke}{rgb}{0.000000,0.000000,0.000000}%
\pgfsetstrokecolor{currentstroke}%
\pgfsetdash{}{0pt}%
\pgfpathmoveto{\pgfqpoint{2.243110in}{2.975398in}}%
\pgfpathlineto{\pgfqpoint{2.173191in}{2.996631in}}%
\pgfpathlineto{\pgfqpoint{2.243110in}{2.973199in}}%
\pgfpathlineto{\pgfqpoint{2.313028in}{2.956963in}}%
\pgfpathlineto{\pgfqpoint{2.243110in}{2.975398in}}%
\pgfpathclose%
\pgfusepath{fill}%
\end{pgfscope}%
\begin{pgfscope}%
\pgfpathrectangle{\pgfqpoint{0.637500in}{0.550000in}}{\pgfqpoint{3.850000in}{3.850000in}}%
\pgfusepath{clip}%
\pgfsetbuttcap%
\pgfsetroundjoin%
\definecolor{currentfill}{rgb}{0.907506,0.587210,0.000000}%
\pgfsetfillcolor{currentfill}%
\pgfsetfillopacity{0.800000}%
\pgfsetlinewidth{0.000000pt}%
\definecolor{currentstroke}{rgb}{0.000000,0.000000,0.000000}%
\pgfsetstrokecolor{currentstroke}%
\pgfsetdash{}{0pt}%
\pgfpathmoveto{\pgfqpoint{3.012215in}{3.101806in}}%
\pgfpathlineto{\pgfqpoint{2.942296in}{3.067366in}}%
\pgfpathlineto{\pgfqpoint{3.012215in}{3.099607in}}%
\pgfpathlineto{\pgfqpoint{3.082134in}{3.135818in}}%
\pgfpathlineto{\pgfqpoint{3.012215in}{3.101806in}}%
\pgfpathclose%
\pgfusepath{fill}%
\end{pgfscope}%
\begin{pgfscope}%
\pgfpathrectangle{\pgfqpoint{0.637500in}{0.550000in}}{\pgfqpoint{3.850000in}{3.850000in}}%
\pgfusepath{clip}%
\pgfsetbuttcap%
\pgfsetroundjoin%
\definecolor{currentfill}{rgb}{0.919814,0.595174,0.000000}%
\pgfsetfillcolor{currentfill}%
\pgfsetfillopacity{0.800000}%
\pgfsetlinewidth{0.000000pt}%
\definecolor{currentstroke}{rgb}{0.000000,0.000000,0.000000}%
\pgfsetstrokecolor{currentstroke}%
\pgfsetdash{}{0pt}%
\pgfpathmoveto{\pgfqpoint{2.802459in}{3.008376in}}%
\pgfpathlineto{\pgfqpoint{2.732540in}{2.983099in}}%
\pgfpathlineto{\pgfqpoint{2.802459in}{3.006178in}}%
\pgfpathlineto{\pgfqpoint{2.872378in}{3.035262in}}%
\pgfpathlineto{\pgfqpoint{2.802459in}{3.008376in}}%
\pgfpathclose%
\pgfusepath{fill}%
\end{pgfscope}%
\begin{pgfscope}%
\pgfpathrectangle{\pgfqpoint{0.637500in}{0.550000in}}{\pgfqpoint{3.850000in}{3.850000in}}%
\pgfusepath{clip}%
\pgfsetbuttcap%
\pgfsetroundjoin%
\definecolor{currentfill}{rgb}{0.912477,0.590427,0.000000}%
\pgfsetfillcolor{currentfill}%
\pgfsetfillopacity{0.800000}%
\pgfsetlinewidth{0.000000pt}%
\definecolor{currentstroke}{rgb}{0.000000,0.000000,0.000000}%
\pgfsetstrokecolor{currentstroke}%
\pgfsetdash{}{0pt}%
\pgfpathmoveto{\pgfqpoint{2.103272in}{3.024221in}}%
\pgfpathlineto{\pgfqpoint{2.033354in}{3.052950in}}%
\pgfpathlineto{\pgfqpoint{2.103272in}{3.022022in}}%
\pgfpathlineto{\pgfqpoint{2.173191in}{2.996631in}}%
\pgfpathlineto{\pgfqpoint{2.103272in}{3.024221in}}%
\pgfpathclose%
\pgfusepath{fill}%
\end{pgfscope}%
\begin{pgfscope}%
\pgfpathrectangle{\pgfqpoint{0.637500in}{0.550000in}}{\pgfqpoint{3.850000in}{3.850000in}}%
\pgfusepath{clip}%
\pgfsetbuttcap%
\pgfsetroundjoin%
\definecolor{currentfill}{rgb}{0.934964,0.604977,0.000000}%
\pgfsetfillcolor{currentfill}%
\pgfsetfillopacity{0.800000}%
\pgfsetlinewidth{0.000000pt}%
\definecolor{currentstroke}{rgb}{0.000000,0.000000,0.000000}%
\pgfsetstrokecolor{currentstroke}%
\pgfsetdash{}{0pt}%
\pgfpathmoveto{\pgfqpoint{2.452866in}{2.940167in}}%
\pgfpathlineto{\pgfqpoint{2.382947in}{2.944295in}}%
\pgfpathlineto{\pgfqpoint{2.452866in}{2.937969in}}%
\pgfpathlineto{\pgfqpoint{2.522784in}{2.940461in}}%
\pgfpathlineto{\pgfqpoint{2.452866in}{2.940167in}}%
\pgfpathclose%
\pgfusepath{fill}%
\end{pgfscope}%
\begin{pgfscope}%
\pgfpathrectangle{\pgfqpoint{0.637500in}{0.550000in}}{\pgfqpoint{3.850000in}{3.850000in}}%
\pgfusepath{clip}%
\pgfsetbuttcap%
\pgfsetroundjoin%
\definecolor{currentfill}{rgb}{0.910843,0.589369,0.000000}%
\pgfsetfillcolor{currentfill}%
\pgfsetfillopacity{0.800000}%
\pgfsetlinewidth{0.000000pt}%
\definecolor{currentstroke}{rgb}{0.000000,0.000000,0.000000}%
\pgfsetstrokecolor{currentstroke}%
\pgfsetdash{}{0pt}%
\pgfpathmoveto{\pgfqpoint{2.942296in}{3.067366in}}%
\pgfpathlineto{\pgfqpoint{2.872378in}{3.035262in}}%
\pgfpathlineto{\pgfqpoint{2.942296in}{3.065167in}}%
\pgfpathlineto{\pgfqpoint{3.012215in}{3.099607in}}%
\pgfpathlineto{\pgfqpoint{2.942296in}{3.067366in}}%
\pgfpathclose%
\pgfusepath{fill}%
\end{pgfscope}%
\begin{pgfscope}%
\pgfpathrectangle{\pgfqpoint{0.637500in}{0.550000in}}{\pgfqpoint{3.850000in}{3.850000in}}%
\pgfusepath{clip}%
\pgfsetbuttcap%
\pgfsetroundjoin%
\definecolor{currentfill}{rgb}{0.933916,0.604299,0.000000}%
\pgfsetfillcolor{currentfill}%
\pgfsetfillopacity{0.800000}%
\pgfsetlinewidth{0.000000pt}%
\definecolor{currentstroke}{rgb}{0.000000,0.000000,0.000000}%
\pgfsetstrokecolor{currentstroke}%
\pgfsetdash{}{0pt}%
\pgfpathmoveto{\pgfqpoint{2.592703in}{2.949499in}}%
\pgfpathlineto{\pgfqpoint{2.522784in}{2.940461in}}%
\pgfpathlineto{\pgfqpoint{2.592703in}{2.947300in}}%
\pgfpathlineto{\pgfqpoint{2.662622in}{2.962469in}}%
\pgfpathlineto{\pgfqpoint{2.592703in}{2.949499in}}%
\pgfpathclose%
\pgfusepath{fill}%
\end{pgfscope}%
\begin{pgfscope}%
\pgfpathrectangle{\pgfqpoint{0.637500in}{0.550000in}}{\pgfqpoint{3.850000in}{3.850000in}}%
\pgfusepath{clip}%
\pgfsetbuttcap%
\pgfsetroundjoin%
\definecolor{currentfill}{rgb}{0.927227,0.599970,0.000000}%
\pgfsetfillcolor{currentfill}%
\pgfsetfillopacity{0.800000}%
\pgfsetlinewidth{0.000000pt}%
\definecolor{currentstroke}{rgb}{0.000000,0.000000,0.000000}%
\pgfsetstrokecolor{currentstroke}%
\pgfsetdash{}{0pt}%
\pgfpathmoveto{\pgfqpoint{2.313028in}{2.956963in}}%
\pgfpathlineto{\pgfqpoint{2.243110in}{2.973199in}}%
\pgfpathlineto{\pgfqpoint{2.313028in}{2.954765in}}%
\pgfpathlineto{\pgfqpoint{2.382947in}{2.944295in}}%
\pgfpathlineto{\pgfqpoint{2.313028in}{2.956963in}}%
\pgfpathclose%
\pgfusepath{fill}%
\end{pgfscope}%
\begin{pgfscope}%
\pgfpathrectangle{\pgfqpoint{0.637500in}{0.550000in}}{\pgfqpoint{3.850000in}{3.850000in}}%
\pgfusepath{clip}%
\pgfsetbuttcap%
\pgfsetroundjoin%
\definecolor{currentfill}{rgb}{0.905912,0.586178,0.000000}%
\pgfsetfillcolor{currentfill}%
\pgfsetfillopacity{0.800000}%
\pgfsetlinewidth{0.000000pt}%
\definecolor{currentstroke}{rgb}{0.000000,0.000000,0.000000}%
\pgfsetstrokecolor{currentstroke}%
\pgfsetdash{}{0pt}%
\pgfpathmoveto{\pgfqpoint{1.963435in}{3.086487in}}%
\pgfpathlineto{\pgfqpoint{1.893516in}{3.119817in}}%
\pgfpathlineto{\pgfqpoint{1.963435in}{3.084288in}}%
\pgfpathlineto{\pgfqpoint{2.033354in}{3.052950in}}%
\pgfpathlineto{\pgfqpoint{1.963435in}{3.086487in}}%
\pgfpathclose%
\pgfusepath{fill}%
\end{pgfscope}%
\begin{pgfscope}%
\pgfpathrectangle{\pgfqpoint{0.637500in}{0.550000in}}{\pgfqpoint{3.850000in}{3.850000in}}%
\pgfusepath{clip}%
\pgfsetbuttcap%
\pgfsetroundjoin%
\definecolor{currentfill}{rgb}{0.925053,0.598564,0.000000}%
\pgfsetfillcolor{currentfill}%
\pgfsetfillopacity{0.800000}%
\pgfsetlinewidth{0.000000pt}%
\definecolor{currentstroke}{rgb}{0.000000,0.000000,0.000000}%
\pgfsetstrokecolor{currentstroke}%
\pgfsetdash{}{0pt}%
\pgfpathmoveto{\pgfqpoint{2.732540in}{2.983099in}}%
\pgfpathlineto{\pgfqpoint{2.662622in}{2.962469in}}%
\pgfpathlineto{\pgfqpoint{2.732540in}{2.980900in}}%
\pgfpathlineto{\pgfqpoint{2.802459in}{3.006178in}}%
\pgfpathlineto{\pgfqpoint{2.732540in}{2.983099in}}%
\pgfpathclose%
\pgfusepath{fill}%
\end{pgfscope}%
\begin{pgfscope}%
\pgfpathrectangle{\pgfqpoint{0.637500in}{0.550000in}}{\pgfqpoint{3.850000in}{3.850000in}}%
\pgfusepath{clip}%
\pgfsetbuttcap%
\pgfsetroundjoin%
\definecolor{currentfill}{rgb}{0.916926,0.593305,0.000000}%
\pgfsetfillcolor{currentfill}%
\pgfsetfillopacity{0.800000}%
\pgfsetlinewidth{0.000000pt}%
\definecolor{currentstroke}{rgb}{0.000000,0.000000,0.000000}%
\pgfsetstrokecolor{currentstroke}%
\pgfsetdash{}{0pt}%
\pgfpathmoveto{\pgfqpoint{2.173191in}{2.996631in}}%
\pgfpathlineto{\pgfqpoint{2.103272in}{3.022022in}}%
\pgfpathlineto{\pgfqpoint{2.173191in}{2.994433in}}%
\pgfpathlineto{\pgfqpoint{2.243110in}{2.973199in}}%
\pgfpathlineto{\pgfqpoint{2.173191in}{2.996631in}}%
\pgfpathclose%
\pgfusepath{fill}%
\end{pgfscope}%
\begin{pgfscope}%
\pgfpathrectangle{\pgfqpoint{0.637500in}{0.550000in}}{\pgfqpoint{3.850000in}{3.850000in}}%
\pgfusepath{clip}%
\pgfsetbuttcap%
\pgfsetroundjoin%
\definecolor{currentfill}{rgb}{0.904905,0.585527,0.000000}%
\pgfsetfillcolor{currentfill}%
\pgfsetfillopacity{0.800000}%
\pgfsetlinewidth{0.000000pt}%
\definecolor{currentstroke}{rgb}{0.000000,0.000000,0.000000}%
\pgfsetstrokecolor{currentstroke}%
\pgfsetdash{}{0pt}%
\pgfpathmoveto{\pgfqpoint{3.082134in}{3.135818in}}%
\pgfpathlineto{\pgfqpoint{3.012215in}{3.099607in}}%
\pgfpathlineto{\pgfqpoint{3.082134in}{3.133619in}}%
\pgfpathlineto{\pgfqpoint{3.152052in}{3.171153in}}%
\pgfpathlineto{\pgfqpoint{3.082134in}{3.135818in}}%
\pgfpathclose%
\pgfusepath{fill}%
\end{pgfscope}%
\begin{pgfscope}%
\pgfpathrectangle{\pgfqpoint{0.637500in}{0.550000in}}{\pgfqpoint{3.850000in}{3.850000in}}%
\pgfusepath{clip}%
\pgfsetbuttcap%
\pgfsetroundjoin%
\definecolor{currentfill}{rgb}{0.914972,0.592040,0.000000}%
\pgfsetfillcolor{currentfill}%
\pgfsetfillopacity{0.800000}%
\pgfsetlinewidth{0.000000pt}%
\definecolor{currentstroke}{rgb}{0.000000,0.000000,0.000000}%
\pgfsetstrokecolor{currentstroke}%
\pgfsetdash{}{0pt}%
\pgfpathmoveto{\pgfqpoint{2.872378in}{3.035262in}}%
\pgfpathlineto{\pgfqpoint{2.802459in}{3.006178in}}%
\pgfpathlineto{\pgfqpoint{2.872378in}{3.033063in}}%
\pgfpathlineto{\pgfqpoint{2.942296in}{3.065167in}}%
\pgfpathlineto{\pgfqpoint{2.872378in}{3.035262in}}%
\pgfpathclose%
\pgfusepath{fill}%
\end{pgfscope}%
\begin{pgfscope}%
\pgfpathrectangle{\pgfqpoint{0.637500in}{0.550000in}}{\pgfqpoint{3.850000in}{3.850000in}}%
\pgfusepath{clip}%
\pgfsetbuttcap%
\pgfsetroundjoin%
\definecolor{currentfill}{rgb}{0.908810,0.588053,0.000000}%
\pgfsetfillcolor{currentfill}%
\pgfsetfillopacity{0.800000}%
\pgfsetlinewidth{0.000000pt}%
\definecolor{currentstroke}{rgb}{0.000000,0.000000,0.000000}%
\pgfsetstrokecolor{currentstroke}%
\pgfsetdash{}{0pt}%
\pgfpathmoveto{\pgfqpoint{2.033354in}{3.052950in}}%
\pgfpathlineto{\pgfqpoint{1.963435in}{3.084288in}}%
\pgfpathlineto{\pgfqpoint{2.033354in}{3.050751in}}%
\pgfpathlineto{\pgfqpoint{2.103272in}{3.022022in}}%
\pgfpathlineto{\pgfqpoint{2.033354in}{3.052950in}}%
\pgfpathclose%
\pgfusepath{fill}%
\end{pgfscope}%
\begin{pgfscope}%
\pgfpathrectangle{\pgfqpoint{0.637500in}{0.550000in}}{\pgfqpoint{3.850000in}{3.850000in}}%
\pgfusepath{clip}%
\pgfsetbuttcap%
\pgfsetroundjoin%
\definecolor{currentfill}{rgb}{0.935716,0.605463,0.000000}%
\pgfsetfillcolor{currentfill}%
\pgfsetfillopacity{0.800000}%
\pgfsetlinewidth{0.000000pt}%
\definecolor{currentstroke}{rgb}{0.000000,0.000000,0.000000}%
\pgfsetstrokecolor{currentstroke}%
\pgfsetdash{}{0pt}%
\pgfpathmoveto{\pgfqpoint{2.522784in}{2.940461in}}%
\pgfpathlineto{\pgfqpoint{2.452866in}{2.937969in}}%
\pgfpathlineto{\pgfqpoint{2.522784in}{2.938263in}}%
\pgfpathlineto{\pgfqpoint{2.592703in}{2.947300in}}%
\pgfpathlineto{\pgfqpoint{2.522784in}{2.940461in}}%
\pgfpathclose%
\pgfusepath{fill}%
\end{pgfscope}%
\begin{pgfscope}%
\pgfpathrectangle{\pgfqpoint{0.637500in}{0.550000in}}{\pgfqpoint{3.850000in}{3.850000in}}%
\pgfusepath{clip}%
\pgfsetbuttcap%
\pgfsetroundjoin%
\definecolor{currentfill}{rgb}{0.931871,0.602975,0.000000}%
\pgfsetfillcolor{currentfill}%
\pgfsetfillopacity{0.800000}%
\pgfsetlinewidth{0.000000pt}%
\definecolor{currentstroke}{rgb}{0.000000,0.000000,0.000000}%
\pgfsetstrokecolor{currentstroke}%
\pgfsetdash{}{0pt}%
\pgfpathmoveto{\pgfqpoint{2.382947in}{2.944295in}}%
\pgfpathlineto{\pgfqpoint{2.313028in}{2.954765in}}%
\pgfpathlineto{\pgfqpoint{2.382947in}{2.942096in}}%
\pgfpathlineto{\pgfqpoint{2.452866in}{2.937969in}}%
\pgfpathlineto{\pgfqpoint{2.382947in}{2.944295in}}%
\pgfpathclose%
\pgfusepath{fill}%
\end{pgfscope}%
\begin{pgfscope}%
\pgfpathrectangle{\pgfqpoint{0.637500in}{0.550000in}}{\pgfqpoint{3.850000in}{3.850000in}}%
\pgfusepath{clip}%
\pgfsetbuttcap%
\pgfsetroundjoin%
\definecolor{currentfill}{rgb}{0.930057,0.601801,0.000000}%
\pgfsetfillcolor{currentfill}%
\pgfsetfillopacity{0.800000}%
\pgfsetlinewidth{0.000000pt}%
\definecolor{currentstroke}{rgb}{0.000000,0.000000,0.000000}%
\pgfsetstrokecolor{currentstroke}%
\pgfsetdash{}{0pt}%
\pgfpathmoveto{\pgfqpoint{2.662622in}{2.962469in}}%
\pgfpathlineto{\pgfqpoint{2.592703in}{2.947300in}}%
\pgfpathlineto{\pgfqpoint{2.662622in}{2.960270in}}%
\pgfpathlineto{\pgfqpoint{2.732540in}{2.980900in}}%
\pgfpathlineto{\pgfqpoint{2.662622in}{2.962469in}}%
\pgfpathclose%
\pgfusepath{fill}%
\end{pgfscope}%
\begin{pgfscope}%
\pgfpathrectangle{\pgfqpoint{0.637500in}{0.550000in}}{\pgfqpoint{3.850000in}{3.850000in}}%
\pgfusepath{clip}%
\pgfsetbuttcap%
\pgfsetroundjoin%
\definecolor{currentfill}{rgb}{0.921990,0.596582,0.000000}%
\pgfsetfillcolor{currentfill}%
\pgfsetfillopacity{0.800000}%
\pgfsetlinewidth{0.000000pt}%
\definecolor{currentstroke}{rgb}{0.000000,0.000000,0.000000}%
\pgfsetstrokecolor{currentstroke}%
\pgfsetdash{}{0pt}%
\pgfpathmoveto{\pgfqpoint{2.243110in}{2.973199in}}%
\pgfpathlineto{\pgfqpoint{2.173191in}{2.994433in}}%
\pgfpathlineto{\pgfqpoint{2.243110in}{2.971000in}}%
\pgfpathlineto{\pgfqpoint{2.313028in}{2.954765in}}%
\pgfpathlineto{\pgfqpoint{2.243110in}{2.973199in}}%
\pgfpathclose%
\pgfusepath{fill}%
\end{pgfscope}%
\begin{pgfscope}%
\pgfpathrectangle{\pgfqpoint{0.637500in}{0.550000in}}{\pgfqpoint{3.850000in}{3.850000in}}%
\pgfusepath{clip}%
\pgfsetbuttcap%
\pgfsetroundjoin%
\definecolor{currentfill}{rgb}{0.907506,0.587210,0.000000}%
\pgfsetfillcolor{currentfill}%
\pgfsetfillopacity{0.800000}%
\pgfsetlinewidth{0.000000pt}%
\definecolor{currentstroke}{rgb}{0.000000,0.000000,0.000000}%
\pgfsetstrokecolor{currentstroke}%
\pgfsetdash{}{0pt}%
\pgfpathmoveto{\pgfqpoint{3.012215in}{3.099607in}}%
\pgfpathlineto{\pgfqpoint{2.942296in}{3.065167in}}%
\pgfpathlineto{\pgfqpoint{3.012215in}{3.097408in}}%
\pgfpathlineto{\pgfqpoint{3.082134in}{3.133619in}}%
\pgfpathlineto{\pgfqpoint{3.012215in}{3.099607in}}%
\pgfpathclose%
\pgfusepath{fill}%
\end{pgfscope}%
\begin{pgfscope}%
\pgfpathrectangle{\pgfqpoint{0.637500in}{0.550000in}}{\pgfqpoint{3.850000in}{3.850000in}}%
\pgfusepath{clip}%
\pgfsetbuttcap%
\pgfsetroundjoin%
\definecolor{currentfill}{rgb}{0.903689,0.584740,0.000000}%
\pgfsetfillcolor{currentfill}%
\pgfsetfillopacity{0.800000}%
\pgfsetlinewidth{0.000000pt}%
\definecolor{currentstroke}{rgb}{0.000000,0.000000,0.000000}%
\pgfsetstrokecolor{currentstroke}%
\pgfsetdash{}{0pt}%
\pgfpathmoveto{\pgfqpoint{1.893516in}{3.119817in}}%
\pgfpathlineto{\pgfqpoint{1.823598in}{3.154645in}}%
\pgfpathlineto{\pgfqpoint{1.893516in}{3.117618in}}%
\pgfpathlineto{\pgfqpoint{1.963435in}{3.084288in}}%
\pgfpathlineto{\pgfqpoint{1.893516in}{3.119817in}}%
\pgfpathclose%
\pgfusepath{fill}%
\end{pgfscope}%
\begin{pgfscope}%
\pgfpathrectangle{\pgfqpoint{0.637500in}{0.550000in}}{\pgfqpoint{3.850000in}{3.850000in}}%
\pgfusepath{clip}%
\pgfsetbuttcap%
\pgfsetroundjoin%
\definecolor{currentfill}{rgb}{0.919814,0.595174,0.000000}%
\pgfsetfillcolor{currentfill}%
\pgfsetfillopacity{0.800000}%
\pgfsetlinewidth{0.000000pt}%
\definecolor{currentstroke}{rgb}{0.000000,0.000000,0.000000}%
\pgfsetstrokecolor{currentstroke}%
\pgfsetdash{}{0pt}%
\pgfpathmoveto{\pgfqpoint{2.802459in}{3.006178in}}%
\pgfpathlineto{\pgfqpoint{2.732540in}{2.980900in}}%
\pgfpathlineto{\pgfqpoint{2.802459in}{3.003979in}}%
\pgfpathlineto{\pgfqpoint{2.872378in}{3.033063in}}%
\pgfpathlineto{\pgfqpoint{2.802459in}{3.006178in}}%
\pgfpathclose%
\pgfusepath{fill}%
\end{pgfscope}%
\begin{pgfscope}%
\pgfpathrectangle{\pgfqpoint{0.637500in}{0.550000in}}{\pgfqpoint{3.850000in}{3.850000in}}%
\pgfusepath{clip}%
\pgfsetbuttcap%
\pgfsetroundjoin%
\definecolor{currentfill}{rgb}{0.912477,0.590427,0.000000}%
\pgfsetfillcolor{currentfill}%
\pgfsetfillopacity{0.800000}%
\pgfsetlinewidth{0.000000pt}%
\definecolor{currentstroke}{rgb}{0.000000,0.000000,0.000000}%
\pgfsetstrokecolor{currentstroke}%
\pgfsetdash{}{0pt}%
\pgfpathmoveto{\pgfqpoint{2.103272in}{3.022022in}}%
\pgfpathlineto{\pgfqpoint{2.033354in}{3.050751in}}%
\pgfpathlineto{\pgfqpoint{2.103272in}{3.019823in}}%
\pgfpathlineto{\pgfqpoint{2.173191in}{2.994433in}}%
\pgfpathlineto{\pgfqpoint{2.103272in}{3.022022in}}%
\pgfpathclose%
\pgfusepath{fill}%
\end{pgfscope}%
\begin{pgfscope}%
\pgfpathrectangle{\pgfqpoint{0.637500in}{0.550000in}}{\pgfqpoint{3.850000in}{3.850000in}}%
\pgfusepath{clip}%
\pgfsetbuttcap%
\pgfsetroundjoin%
\definecolor{currentfill}{rgb}{0.902928,0.584248,0.000000}%
\pgfsetfillcolor{currentfill}%
\pgfsetfillopacity{0.800000}%
\pgfsetlinewidth{0.000000pt}%
\definecolor{currentstroke}{rgb}{0.000000,0.000000,0.000000}%
\pgfsetstrokecolor{currentstroke}%
\pgfsetdash{}{0pt}%
\pgfpathmoveto{\pgfqpoint{3.152052in}{3.171153in}}%
\pgfpathlineto{\pgfqpoint{3.082134in}{3.133619in}}%
\pgfpathlineto{\pgfqpoint{3.152052in}{3.168954in}}%
\pgfpathlineto{\pgfqpoint{3.221971in}{3.207465in}}%
\pgfpathlineto{\pgfqpoint{3.152052in}{3.171153in}}%
\pgfpathclose%
\pgfusepath{fill}%
\end{pgfscope}%
\begin{pgfscope}%
\pgfpathrectangle{\pgfqpoint{0.637500in}{0.550000in}}{\pgfqpoint{3.850000in}{3.850000in}}%
\pgfusepath{clip}%
\pgfsetbuttcap%
\pgfsetroundjoin%
\definecolor{currentfill}{rgb}{0.934964,0.604977,0.000000}%
\pgfsetfillcolor{currentfill}%
\pgfsetfillopacity{0.800000}%
\pgfsetlinewidth{0.000000pt}%
\definecolor{currentstroke}{rgb}{0.000000,0.000000,0.000000}%
\pgfsetstrokecolor{currentstroke}%
\pgfsetdash{}{0pt}%
\pgfpathmoveto{\pgfqpoint{2.452866in}{2.937969in}}%
\pgfpathlineto{\pgfqpoint{2.382947in}{2.942096in}}%
\pgfpathlineto{\pgfqpoint{2.452866in}{2.935770in}}%
\pgfpathlineto{\pgfqpoint{2.522784in}{2.938263in}}%
\pgfpathlineto{\pgfqpoint{2.452866in}{2.937969in}}%
\pgfpathclose%
\pgfusepath{fill}%
\end{pgfscope}%
\begin{pgfscope}%
\pgfpathrectangle{\pgfqpoint{0.637500in}{0.550000in}}{\pgfqpoint{3.850000in}{3.850000in}}%
\pgfusepath{clip}%
\pgfsetbuttcap%
\pgfsetroundjoin%
\definecolor{currentfill}{rgb}{0.910843,0.589369,0.000000}%
\pgfsetfillcolor{currentfill}%
\pgfsetfillopacity{0.800000}%
\pgfsetlinewidth{0.000000pt}%
\definecolor{currentstroke}{rgb}{0.000000,0.000000,0.000000}%
\pgfsetstrokecolor{currentstroke}%
\pgfsetdash{}{0pt}%
\pgfpathmoveto{\pgfqpoint{2.942296in}{3.065167in}}%
\pgfpathlineto{\pgfqpoint{2.872378in}{3.033063in}}%
\pgfpathlineto{\pgfqpoint{2.942296in}{3.062969in}}%
\pgfpathlineto{\pgfqpoint{3.012215in}{3.097408in}}%
\pgfpathlineto{\pgfqpoint{2.942296in}{3.065167in}}%
\pgfpathclose%
\pgfusepath{fill}%
\end{pgfscope}%
\begin{pgfscope}%
\pgfpathrectangle{\pgfqpoint{0.637500in}{0.550000in}}{\pgfqpoint{3.850000in}{3.850000in}}%
\pgfusepath{clip}%
\pgfsetbuttcap%
\pgfsetroundjoin%
\definecolor{currentfill}{rgb}{0.933916,0.604299,0.000000}%
\pgfsetfillcolor{currentfill}%
\pgfsetfillopacity{0.800000}%
\pgfsetlinewidth{0.000000pt}%
\definecolor{currentstroke}{rgb}{0.000000,0.000000,0.000000}%
\pgfsetstrokecolor{currentstroke}%
\pgfsetdash{}{0pt}%
\pgfpathmoveto{\pgfqpoint{2.592703in}{2.947300in}}%
\pgfpathlineto{\pgfqpoint{2.522784in}{2.938263in}}%
\pgfpathlineto{\pgfqpoint{2.592703in}{2.945101in}}%
\pgfpathlineto{\pgfqpoint{2.662622in}{2.960270in}}%
\pgfpathlineto{\pgfqpoint{2.592703in}{2.947300in}}%
\pgfpathclose%
\pgfusepath{fill}%
\end{pgfscope}%
\begin{pgfscope}%
\pgfpathrectangle{\pgfqpoint{0.637500in}{0.550000in}}{\pgfqpoint{3.850000in}{3.850000in}}%
\pgfusepath{clip}%
\pgfsetbuttcap%
\pgfsetroundjoin%
\definecolor{currentfill}{rgb}{0.927227,0.599970,0.000000}%
\pgfsetfillcolor{currentfill}%
\pgfsetfillopacity{0.800000}%
\pgfsetlinewidth{0.000000pt}%
\definecolor{currentstroke}{rgb}{0.000000,0.000000,0.000000}%
\pgfsetstrokecolor{currentstroke}%
\pgfsetdash{}{0pt}%
\pgfpathmoveto{\pgfqpoint{2.313028in}{2.954765in}}%
\pgfpathlineto{\pgfqpoint{2.243110in}{2.971000in}}%
\pgfpathlineto{\pgfqpoint{2.313028in}{2.952566in}}%
\pgfpathlineto{\pgfqpoint{2.382947in}{2.942096in}}%
\pgfpathlineto{\pgfqpoint{2.313028in}{2.954765in}}%
\pgfpathclose%
\pgfusepath{fill}%
\end{pgfscope}%
\begin{pgfscope}%
\pgfpathrectangle{\pgfqpoint{0.637500in}{0.550000in}}{\pgfqpoint{3.850000in}{3.850000in}}%
\pgfusepath{clip}%
\pgfsetbuttcap%
\pgfsetroundjoin%
\definecolor{currentfill}{rgb}{0.905912,0.586178,0.000000}%
\pgfsetfillcolor{currentfill}%
\pgfsetfillopacity{0.800000}%
\pgfsetlinewidth{0.000000pt}%
\definecolor{currentstroke}{rgb}{0.000000,0.000000,0.000000}%
\pgfsetstrokecolor{currentstroke}%
\pgfsetdash{}{0pt}%
\pgfpathmoveto{\pgfqpoint{1.963435in}{3.084288in}}%
\pgfpathlineto{\pgfqpoint{1.893516in}{3.117618in}}%
\pgfpathlineto{\pgfqpoint{1.963435in}{3.082089in}}%
\pgfpathlineto{\pgfqpoint{2.033354in}{3.050751in}}%
\pgfpathlineto{\pgfqpoint{1.963435in}{3.084288in}}%
\pgfpathclose%
\pgfusepath{fill}%
\end{pgfscope}%
\begin{pgfscope}%
\pgfpathrectangle{\pgfqpoint{0.637500in}{0.550000in}}{\pgfqpoint{3.850000in}{3.850000in}}%
\pgfusepath{clip}%
\pgfsetbuttcap%
\pgfsetroundjoin%
\definecolor{currentfill}{rgb}{0.925053,0.598564,0.000000}%
\pgfsetfillcolor{currentfill}%
\pgfsetfillopacity{0.800000}%
\pgfsetlinewidth{0.000000pt}%
\definecolor{currentstroke}{rgb}{0.000000,0.000000,0.000000}%
\pgfsetstrokecolor{currentstroke}%
\pgfsetdash{}{0pt}%
\pgfpathmoveto{\pgfqpoint{2.732540in}{2.980900in}}%
\pgfpathlineto{\pgfqpoint{2.662622in}{2.960270in}}%
\pgfpathlineto{\pgfqpoint{2.732540in}{2.978701in}}%
\pgfpathlineto{\pgfqpoint{2.802459in}{3.003979in}}%
\pgfpathlineto{\pgfqpoint{2.732540in}{2.980900in}}%
\pgfpathclose%
\pgfusepath{fill}%
\end{pgfscope}%
\begin{pgfscope}%
\pgfpathrectangle{\pgfqpoint{0.637500in}{0.550000in}}{\pgfqpoint{3.850000in}{3.850000in}}%
\pgfusepath{clip}%
\pgfsetbuttcap%
\pgfsetroundjoin%
\definecolor{currentfill}{rgb}{0.916926,0.593305,0.000000}%
\pgfsetfillcolor{currentfill}%
\pgfsetfillopacity{0.800000}%
\pgfsetlinewidth{0.000000pt}%
\definecolor{currentstroke}{rgb}{0.000000,0.000000,0.000000}%
\pgfsetstrokecolor{currentstroke}%
\pgfsetdash{}{0pt}%
\pgfpathmoveto{\pgfqpoint{2.173191in}{2.994433in}}%
\pgfpathlineto{\pgfqpoint{2.103272in}{3.019823in}}%
\pgfpathlineto{\pgfqpoint{2.173191in}{2.992234in}}%
\pgfpathlineto{\pgfqpoint{2.243110in}{2.971000in}}%
\pgfpathlineto{\pgfqpoint{2.173191in}{2.994433in}}%
\pgfpathclose%
\pgfusepath{fill}%
\end{pgfscope}%
\begin{pgfscope}%
\pgfpathrectangle{\pgfqpoint{0.637500in}{0.550000in}}{\pgfqpoint{3.850000in}{3.850000in}}%
\pgfusepath{clip}%
\pgfsetbuttcap%
\pgfsetroundjoin%
\definecolor{currentfill}{rgb}{0.904905,0.585527,0.000000}%
\pgfsetfillcolor{currentfill}%
\pgfsetfillopacity{0.800000}%
\pgfsetlinewidth{0.000000pt}%
\definecolor{currentstroke}{rgb}{0.000000,0.000000,0.000000}%
\pgfsetstrokecolor{currentstroke}%
\pgfsetdash{}{0pt}%
\pgfpathmoveto{\pgfqpoint{3.082134in}{3.133619in}}%
\pgfpathlineto{\pgfqpoint{3.012215in}{3.097408in}}%
\pgfpathlineto{\pgfqpoint{3.082134in}{3.131420in}}%
\pgfpathlineto{\pgfqpoint{3.152052in}{3.168954in}}%
\pgfpathlineto{\pgfqpoint{3.082134in}{3.133619in}}%
\pgfpathclose%
\pgfusepath{fill}%
\end{pgfscope}%
\begin{pgfscope}%
\pgfpathrectangle{\pgfqpoint{0.637500in}{0.550000in}}{\pgfqpoint{3.850000in}{3.850000in}}%
\pgfusepath{clip}%
\pgfsetbuttcap%
\pgfsetroundjoin%
\definecolor{currentfill}{rgb}{0.914972,0.592040,0.000000}%
\pgfsetfillcolor{currentfill}%
\pgfsetfillopacity{0.800000}%
\pgfsetlinewidth{0.000000pt}%
\definecolor{currentstroke}{rgb}{0.000000,0.000000,0.000000}%
\pgfsetstrokecolor{currentstroke}%
\pgfsetdash{}{0pt}%
\pgfpathmoveto{\pgfqpoint{2.872378in}{3.033063in}}%
\pgfpathlineto{\pgfqpoint{2.802459in}{3.003979in}}%
\pgfpathlineto{\pgfqpoint{2.872378in}{3.030865in}}%
\pgfpathlineto{\pgfqpoint{2.942296in}{3.062969in}}%
\pgfpathlineto{\pgfqpoint{2.872378in}{3.033063in}}%
\pgfpathclose%
\pgfusepath{fill}%
\end{pgfscope}%
\begin{pgfscope}%
\pgfpathrectangle{\pgfqpoint{0.637500in}{0.550000in}}{\pgfqpoint{3.850000in}{3.850000in}}%
\pgfusepath{clip}%
\pgfsetbuttcap%
\pgfsetroundjoin%
\definecolor{currentfill}{rgb}{0.902018,0.583658,0.000000}%
\pgfsetfillcolor{currentfill}%
\pgfsetfillopacity{0.800000}%
\pgfsetlinewidth{0.000000pt}%
\definecolor{currentstroke}{rgb}{0.000000,0.000000,0.000000}%
\pgfsetstrokecolor{currentstroke}%
\pgfsetdash{}{0pt}%
\pgfpathmoveto{\pgfqpoint{1.823598in}{3.154645in}}%
\pgfpathlineto{\pgfqpoint{1.753679in}{3.190584in}}%
\pgfpathlineto{\pgfqpoint{1.823598in}{3.152446in}}%
\pgfpathlineto{\pgfqpoint{1.893516in}{3.117618in}}%
\pgfpathlineto{\pgfqpoint{1.823598in}{3.154645in}}%
\pgfpathclose%
\pgfusepath{fill}%
\end{pgfscope}%
\begin{pgfscope}%
\pgfpathrectangle{\pgfqpoint{0.637500in}{0.550000in}}{\pgfqpoint{3.850000in}{3.850000in}}%
\pgfusepath{clip}%
\pgfsetbuttcap%
\pgfsetroundjoin%
\definecolor{currentfill}{rgb}{0.908810,0.588053,0.000000}%
\pgfsetfillcolor{currentfill}%
\pgfsetfillopacity{0.800000}%
\pgfsetlinewidth{0.000000pt}%
\definecolor{currentstroke}{rgb}{0.000000,0.000000,0.000000}%
\pgfsetstrokecolor{currentstroke}%
\pgfsetdash{}{0pt}%
\pgfpathmoveto{\pgfqpoint{2.033354in}{3.050751in}}%
\pgfpathlineto{\pgfqpoint{1.963435in}{3.082089in}}%
\pgfpathlineto{\pgfqpoint{2.033354in}{3.048553in}}%
\pgfpathlineto{\pgfqpoint{2.103272in}{3.019823in}}%
\pgfpathlineto{\pgfqpoint{2.033354in}{3.050751in}}%
\pgfpathclose%
\pgfusepath{fill}%
\end{pgfscope}%
\begin{pgfscope}%
\pgfpathrectangle{\pgfqpoint{0.637500in}{0.550000in}}{\pgfqpoint{3.850000in}{3.850000in}}%
\pgfusepath{clip}%
\pgfsetbuttcap%
\pgfsetroundjoin%
\definecolor{currentfill}{rgb}{0.935716,0.605463,0.000000}%
\pgfsetfillcolor{currentfill}%
\pgfsetfillopacity{0.800000}%
\pgfsetlinewidth{0.000000pt}%
\definecolor{currentstroke}{rgb}{0.000000,0.000000,0.000000}%
\pgfsetstrokecolor{currentstroke}%
\pgfsetdash{}{0pt}%
\pgfpathmoveto{\pgfqpoint{2.522784in}{2.938263in}}%
\pgfpathlineto{\pgfqpoint{2.452866in}{2.935770in}}%
\pgfpathlineto{\pgfqpoint{2.522784in}{2.936064in}}%
\pgfpathlineto{\pgfqpoint{2.592703in}{2.945101in}}%
\pgfpathlineto{\pgfqpoint{2.522784in}{2.938263in}}%
\pgfpathclose%
\pgfusepath{fill}%
\end{pgfscope}%
\begin{pgfscope}%
\pgfpathrectangle{\pgfqpoint{0.637500in}{0.550000in}}{\pgfqpoint{3.850000in}{3.850000in}}%
\pgfusepath{clip}%
\pgfsetbuttcap%
\pgfsetroundjoin%
\definecolor{currentfill}{rgb}{0.931871,0.602975,0.000000}%
\pgfsetfillcolor{currentfill}%
\pgfsetfillopacity{0.800000}%
\pgfsetlinewidth{0.000000pt}%
\definecolor{currentstroke}{rgb}{0.000000,0.000000,0.000000}%
\pgfsetstrokecolor{currentstroke}%
\pgfsetdash{}{0pt}%
\pgfpathmoveto{\pgfqpoint{2.382947in}{2.942096in}}%
\pgfpathlineto{\pgfqpoint{2.313028in}{2.952566in}}%
\pgfpathlineto{\pgfqpoint{2.382947in}{2.939898in}}%
\pgfpathlineto{\pgfqpoint{2.452866in}{2.935770in}}%
\pgfpathlineto{\pgfqpoint{2.382947in}{2.942096in}}%
\pgfpathclose%
\pgfusepath{fill}%
\end{pgfscope}%
\begin{pgfscope}%
\pgfpathrectangle{\pgfqpoint{0.637500in}{0.550000in}}{\pgfqpoint{3.850000in}{3.850000in}}%
\pgfusepath{clip}%
\pgfsetbuttcap%
\pgfsetroundjoin%
\definecolor{currentfill}{rgb}{0.930057,0.601801,0.000000}%
\pgfsetfillcolor{currentfill}%
\pgfsetfillopacity{0.800000}%
\pgfsetlinewidth{0.000000pt}%
\definecolor{currentstroke}{rgb}{0.000000,0.000000,0.000000}%
\pgfsetstrokecolor{currentstroke}%
\pgfsetdash{}{0pt}%
\pgfpathmoveto{\pgfqpoint{2.662622in}{2.960270in}}%
\pgfpathlineto{\pgfqpoint{2.592703in}{2.945101in}}%
\pgfpathlineto{\pgfqpoint{2.662622in}{2.958072in}}%
\pgfpathlineto{\pgfqpoint{2.732540in}{2.978701in}}%
\pgfpathlineto{\pgfqpoint{2.662622in}{2.960270in}}%
\pgfpathclose%
\pgfusepath{fill}%
\end{pgfscope}%
\begin{pgfscope}%
\pgfpathrectangle{\pgfqpoint{0.637500in}{0.550000in}}{\pgfqpoint{3.850000in}{3.850000in}}%
\pgfusepath{clip}%
\pgfsetbuttcap%
\pgfsetroundjoin%
\definecolor{currentfill}{rgb}{0.901452,0.583292,0.000000}%
\pgfsetfillcolor{currentfill}%
\pgfsetfillopacity{0.800000}%
\pgfsetlinewidth{0.000000pt}%
\definecolor{currentstroke}{rgb}{0.000000,0.000000,0.000000}%
\pgfsetstrokecolor{currentstroke}%
\pgfsetdash{}{0pt}%
\pgfpathmoveto{\pgfqpoint{3.221971in}{3.207465in}}%
\pgfpathlineto{\pgfqpoint{3.152052in}{3.168954in}}%
\pgfpathlineto{\pgfqpoint{3.221971in}{3.205266in}}%
\pgfpathlineto{\pgfqpoint{3.291890in}{3.244493in}}%
\pgfpathlineto{\pgfqpoint{3.221971in}{3.207465in}}%
\pgfpathclose%
\pgfusepath{fill}%
\end{pgfscope}%
\begin{pgfscope}%
\pgfpathrectangle{\pgfqpoint{0.637500in}{0.550000in}}{\pgfqpoint{3.850000in}{3.850000in}}%
\pgfusepath{clip}%
\pgfsetbuttcap%
\pgfsetroundjoin%
\definecolor{currentfill}{rgb}{0.921990,0.596582,0.000000}%
\pgfsetfillcolor{currentfill}%
\pgfsetfillopacity{0.800000}%
\pgfsetlinewidth{0.000000pt}%
\definecolor{currentstroke}{rgb}{0.000000,0.000000,0.000000}%
\pgfsetstrokecolor{currentstroke}%
\pgfsetdash{}{0pt}%
\pgfpathmoveto{\pgfqpoint{2.243110in}{2.971000in}}%
\pgfpathlineto{\pgfqpoint{2.173191in}{2.992234in}}%
\pgfpathlineto{\pgfqpoint{2.243110in}{2.968801in}}%
\pgfpathlineto{\pgfqpoint{2.313028in}{2.952566in}}%
\pgfpathlineto{\pgfqpoint{2.243110in}{2.971000in}}%
\pgfpathclose%
\pgfusepath{fill}%
\end{pgfscope}%
\begin{pgfscope}%
\pgfpathrectangle{\pgfqpoint{0.637500in}{0.550000in}}{\pgfqpoint{3.850000in}{3.850000in}}%
\pgfusepath{clip}%
\pgfsetbuttcap%
\pgfsetroundjoin%
\definecolor{currentfill}{rgb}{0.907506,0.587210,0.000000}%
\pgfsetfillcolor{currentfill}%
\pgfsetfillopacity{0.800000}%
\pgfsetlinewidth{0.000000pt}%
\definecolor{currentstroke}{rgb}{0.000000,0.000000,0.000000}%
\pgfsetstrokecolor{currentstroke}%
\pgfsetdash{}{0pt}%
\pgfpathmoveto{\pgfqpoint{3.012215in}{3.097408in}}%
\pgfpathlineto{\pgfqpoint{2.942296in}{3.062969in}}%
\pgfpathlineto{\pgfqpoint{3.012215in}{3.095210in}}%
\pgfpathlineto{\pgfqpoint{3.082134in}{3.131420in}}%
\pgfpathlineto{\pgfqpoint{3.012215in}{3.097408in}}%
\pgfpathclose%
\pgfusepath{fill}%
\end{pgfscope}%
\begin{pgfscope}%
\pgfpathrectangle{\pgfqpoint{0.637500in}{0.550000in}}{\pgfqpoint{3.850000in}{3.850000in}}%
\pgfusepath{clip}%
\pgfsetbuttcap%
\pgfsetroundjoin%
\definecolor{currentfill}{rgb}{0.903689,0.584740,0.000000}%
\pgfsetfillcolor{currentfill}%
\pgfsetfillopacity{0.800000}%
\pgfsetlinewidth{0.000000pt}%
\definecolor{currentstroke}{rgb}{0.000000,0.000000,0.000000}%
\pgfsetstrokecolor{currentstroke}%
\pgfsetdash{}{0pt}%
\pgfpathmoveto{\pgfqpoint{1.893516in}{3.117618in}}%
\pgfpathlineto{\pgfqpoint{1.823598in}{3.152446in}}%
\pgfpathlineto{\pgfqpoint{1.893516in}{3.115420in}}%
\pgfpathlineto{\pgfqpoint{1.963435in}{3.082089in}}%
\pgfpathlineto{\pgfqpoint{1.893516in}{3.117618in}}%
\pgfpathclose%
\pgfusepath{fill}%
\end{pgfscope}%
\begin{pgfscope}%
\pgfpathrectangle{\pgfqpoint{0.637500in}{0.550000in}}{\pgfqpoint{3.850000in}{3.850000in}}%
\pgfusepath{clip}%
\pgfsetbuttcap%
\pgfsetroundjoin%
\definecolor{currentfill}{rgb}{0.919814,0.595174,0.000000}%
\pgfsetfillcolor{currentfill}%
\pgfsetfillopacity{0.800000}%
\pgfsetlinewidth{0.000000pt}%
\definecolor{currentstroke}{rgb}{0.000000,0.000000,0.000000}%
\pgfsetstrokecolor{currentstroke}%
\pgfsetdash{}{0pt}%
\pgfpathmoveto{\pgfqpoint{2.802459in}{3.003979in}}%
\pgfpathlineto{\pgfqpoint{2.732540in}{2.978701in}}%
\pgfpathlineto{\pgfqpoint{2.802459in}{3.001780in}}%
\pgfpathlineto{\pgfqpoint{2.872378in}{3.030865in}}%
\pgfpathlineto{\pgfqpoint{2.802459in}{3.003979in}}%
\pgfpathclose%
\pgfusepath{fill}%
\end{pgfscope}%
\begin{pgfscope}%
\pgfpathrectangle{\pgfqpoint{0.637500in}{0.550000in}}{\pgfqpoint{3.850000in}{3.850000in}}%
\pgfusepath{clip}%
\pgfsetbuttcap%
\pgfsetroundjoin%
\definecolor{currentfill}{rgb}{0.912477,0.590427,0.000000}%
\pgfsetfillcolor{currentfill}%
\pgfsetfillopacity{0.800000}%
\pgfsetlinewidth{0.000000pt}%
\definecolor{currentstroke}{rgb}{0.000000,0.000000,0.000000}%
\pgfsetstrokecolor{currentstroke}%
\pgfsetdash{}{0pt}%
\pgfpathmoveto{\pgfqpoint{2.103272in}{3.019823in}}%
\pgfpathlineto{\pgfqpoint{2.033354in}{3.048553in}}%
\pgfpathlineto{\pgfqpoint{2.103272in}{3.017624in}}%
\pgfpathlineto{\pgfqpoint{2.173191in}{2.992234in}}%
\pgfpathlineto{\pgfqpoint{2.103272in}{3.019823in}}%
\pgfpathclose%
\pgfusepath{fill}%
\end{pgfscope}%
\begin{pgfscope}%
\pgfpathrectangle{\pgfqpoint{0.637500in}{0.550000in}}{\pgfqpoint{3.850000in}{3.850000in}}%
\pgfusepath{clip}%
\pgfsetbuttcap%
\pgfsetroundjoin%
\definecolor{currentfill}{rgb}{0.902928,0.584248,0.000000}%
\pgfsetfillcolor{currentfill}%
\pgfsetfillopacity{0.800000}%
\pgfsetlinewidth{0.000000pt}%
\definecolor{currentstroke}{rgb}{0.000000,0.000000,0.000000}%
\pgfsetstrokecolor{currentstroke}%
\pgfsetdash{}{0pt}%
\pgfpathmoveto{\pgfqpoint{3.152052in}{3.168954in}}%
\pgfpathlineto{\pgfqpoint{3.082134in}{3.131420in}}%
\pgfpathlineto{\pgfqpoint{3.152052in}{3.166755in}}%
\pgfpathlineto{\pgfqpoint{3.221971in}{3.205266in}}%
\pgfpathlineto{\pgfqpoint{3.152052in}{3.168954in}}%
\pgfpathclose%
\pgfusepath{fill}%
\end{pgfscope}%
\begin{pgfscope}%
\pgfpathrectangle{\pgfqpoint{0.637500in}{0.550000in}}{\pgfqpoint{3.850000in}{3.850000in}}%
\pgfusepath{clip}%
\pgfsetbuttcap%
\pgfsetroundjoin%
\definecolor{currentfill}{rgb}{0.934964,0.604977,0.000000}%
\pgfsetfillcolor{currentfill}%
\pgfsetfillopacity{0.800000}%
\pgfsetlinewidth{0.000000pt}%
\definecolor{currentstroke}{rgb}{0.000000,0.000000,0.000000}%
\pgfsetstrokecolor{currentstroke}%
\pgfsetdash{}{0pt}%
\pgfpathmoveto{\pgfqpoint{2.452866in}{2.935770in}}%
\pgfpathlineto{\pgfqpoint{2.382947in}{2.939898in}}%
\pgfpathlineto{\pgfqpoint{2.452866in}{2.933571in}}%
\pgfpathlineto{\pgfqpoint{2.522784in}{2.936064in}}%
\pgfpathlineto{\pgfqpoint{2.452866in}{2.935770in}}%
\pgfpathclose%
\pgfusepath{fill}%
\end{pgfscope}%
\begin{pgfscope}%
\pgfpathrectangle{\pgfqpoint{0.637500in}{0.550000in}}{\pgfqpoint{3.850000in}{3.850000in}}%
\pgfusepath{clip}%
\pgfsetbuttcap%
\pgfsetroundjoin%
\definecolor{currentfill}{rgb}{0.910843,0.589369,0.000000}%
\pgfsetfillcolor{currentfill}%
\pgfsetfillopacity{0.800000}%
\pgfsetlinewidth{0.000000pt}%
\definecolor{currentstroke}{rgb}{0.000000,0.000000,0.000000}%
\pgfsetstrokecolor{currentstroke}%
\pgfsetdash{}{0pt}%
\pgfpathmoveto{\pgfqpoint{2.942296in}{3.062969in}}%
\pgfpathlineto{\pgfqpoint{2.872378in}{3.030865in}}%
\pgfpathlineto{\pgfqpoint{2.942296in}{3.060770in}}%
\pgfpathlineto{\pgfqpoint{3.012215in}{3.095210in}}%
\pgfpathlineto{\pgfqpoint{2.942296in}{3.062969in}}%
\pgfpathclose%
\pgfusepath{fill}%
\end{pgfscope}%
\begin{pgfscope}%
\pgfpathrectangle{\pgfqpoint{0.637500in}{0.550000in}}{\pgfqpoint{3.850000in}{3.850000in}}%
\pgfusepath{clip}%
\pgfsetbuttcap%
\pgfsetroundjoin%
\definecolor{currentfill}{rgb}{0.933916,0.604299,0.000000}%
\pgfsetfillcolor{currentfill}%
\pgfsetfillopacity{0.800000}%
\pgfsetlinewidth{0.000000pt}%
\definecolor{currentstroke}{rgb}{0.000000,0.000000,0.000000}%
\pgfsetstrokecolor{currentstroke}%
\pgfsetdash{}{0pt}%
\pgfpathmoveto{\pgfqpoint{2.592703in}{2.945101in}}%
\pgfpathlineto{\pgfqpoint{2.522784in}{2.936064in}}%
\pgfpathlineto{\pgfqpoint{2.592703in}{2.942903in}}%
\pgfpathlineto{\pgfqpoint{2.662622in}{2.958072in}}%
\pgfpathlineto{\pgfqpoint{2.592703in}{2.945101in}}%
\pgfpathclose%
\pgfusepath{fill}%
\end{pgfscope}%
\begin{pgfscope}%
\pgfpathrectangle{\pgfqpoint{0.637500in}{0.550000in}}{\pgfqpoint{3.850000in}{3.850000in}}%
\pgfusepath{clip}%
\pgfsetbuttcap%
\pgfsetroundjoin%
\definecolor{currentfill}{rgb}{0.900779,0.582857,0.000000}%
\pgfsetfillcolor{currentfill}%
\pgfsetfillopacity{0.800000}%
\pgfsetlinewidth{0.000000pt}%
\definecolor{currentstroke}{rgb}{0.000000,0.000000,0.000000}%
\pgfsetstrokecolor{currentstroke}%
\pgfsetdash{}{0pt}%
\pgfpathmoveto{\pgfqpoint{1.753679in}{3.190584in}}%
\pgfpathlineto{\pgfqpoint{1.683761in}{3.227339in}}%
\pgfpathlineto{\pgfqpoint{1.753679in}{3.188385in}}%
\pgfpathlineto{\pgfqpoint{1.823598in}{3.152446in}}%
\pgfpathlineto{\pgfqpoint{1.753679in}{3.190584in}}%
\pgfpathclose%
\pgfusepath{fill}%
\end{pgfscope}%
\begin{pgfscope}%
\pgfpathrectangle{\pgfqpoint{0.637500in}{0.550000in}}{\pgfqpoint{3.850000in}{3.850000in}}%
\pgfusepath{clip}%
\pgfsetbuttcap%
\pgfsetroundjoin%
\definecolor{currentfill}{rgb}{0.927227,0.599970,0.000000}%
\pgfsetfillcolor{currentfill}%
\pgfsetfillopacity{0.800000}%
\pgfsetlinewidth{0.000000pt}%
\definecolor{currentstroke}{rgb}{0.000000,0.000000,0.000000}%
\pgfsetstrokecolor{currentstroke}%
\pgfsetdash{}{0pt}%
\pgfpathmoveto{\pgfqpoint{2.313028in}{2.952566in}}%
\pgfpathlineto{\pgfqpoint{2.243110in}{2.968801in}}%
\pgfpathlineto{\pgfqpoint{2.313028in}{2.950367in}}%
\pgfpathlineto{\pgfqpoint{2.382947in}{2.939898in}}%
\pgfpathlineto{\pgfqpoint{2.313028in}{2.952566in}}%
\pgfpathclose%
\pgfusepath{fill}%
\end{pgfscope}%
\begin{pgfscope}%
\pgfpathrectangle{\pgfqpoint{0.637500in}{0.550000in}}{\pgfqpoint{3.850000in}{3.850000in}}%
\pgfusepath{clip}%
\pgfsetbuttcap%
\pgfsetroundjoin%
\definecolor{currentfill}{rgb}{0.905912,0.586178,0.000000}%
\pgfsetfillcolor{currentfill}%
\pgfsetfillopacity{0.800000}%
\pgfsetlinewidth{0.000000pt}%
\definecolor{currentstroke}{rgb}{0.000000,0.000000,0.000000}%
\pgfsetstrokecolor{currentstroke}%
\pgfsetdash{}{0pt}%
\pgfpathmoveto{\pgfqpoint{1.963435in}{3.082089in}}%
\pgfpathlineto{\pgfqpoint{1.893516in}{3.115420in}}%
\pgfpathlineto{\pgfqpoint{1.963435in}{3.079890in}}%
\pgfpathlineto{\pgfqpoint{2.033354in}{3.048553in}}%
\pgfpathlineto{\pgfqpoint{1.963435in}{3.082089in}}%
\pgfpathclose%
\pgfusepath{fill}%
\end{pgfscope}%
\begin{pgfscope}%
\pgfpathrectangle{\pgfqpoint{0.637500in}{0.550000in}}{\pgfqpoint{3.850000in}{3.850000in}}%
\pgfusepath{clip}%
\pgfsetbuttcap%
\pgfsetroundjoin%
\definecolor{currentfill}{rgb}{0.925053,0.598564,0.000000}%
\pgfsetfillcolor{currentfill}%
\pgfsetfillopacity{0.800000}%
\pgfsetlinewidth{0.000000pt}%
\definecolor{currentstroke}{rgb}{0.000000,0.000000,0.000000}%
\pgfsetstrokecolor{currentstroke}%
\pgfsetdash{}{0pt}%
\pgfpathmoveto{\pgfqpoint{2.732540in}{2.978701in}}%
\pgfpathlineto{\pgfqpoint{2.662622in}{2.958072in}}%
\pgfpathlineto{\pgfqpoint{2.732540in}{2.976502in}}%
\pgfpathlineto{\pgfqpoint{2.802459in}{3.001780in}}%
\pgfpathlineto{\pgfqpoint{2.732540in}{2.978701in}}%
\pgfpathclose%
\pgfusepath{fill}%
\end{pgfscope}%
\begin{pgfscope}%
\pgfpathrectangle{\pgfqpoint{0.637500in}{0.550000in}}{\pgfqpoint{3.850000in}{3.850000in}}%
\pgfusepath{clip}%
\pgfsetbuttcap%
\pgfsetroundjoin%
\definecolor{currentfill}{rgb}{0.916926,0.593305,0.000000}%
\pgfsetfillcolor{currentfill}%
\pgfsetfillopacity{0.800000}%
\pgfsetlinewidth{0.000000pt}%
\definecolor{currentstroke}{rgb}{0.000000,0.000000,0.000000}%
\pgfsetstrokecolor{currentstroke}%
\pgfsetdash{}{0pt}%
\pgfpathmoveto{\pgfqpoint{2.173191in}{2.992234in}}%
\pgfpathlineto{\pgfqpoint{2.103272in}{3.017624in}}%
\pgfpathlineto{\pgfqpoint{2.173191in}{2.990035in}}%
\pgfpathlineto{\pgfqpoint{2.243110in}{2.968801in}}%
\pgfpathlineto{\pgfqpoint{2.173191in}{2.992234in}}%
\pgfpathclose%
\pgfusepath{fill}%
\end{pgfscope}%
\begin{pgfscope}%
\pgfpathrectangle{\pgfqpoint{0.637500in}{0.550000in}}{\pgfqpoint{3.850000in}{3.850000in}}%
\pgfusepath{clip}%
\pgfsetbuttcap%
\pgfsetroundjoin%
\definecolor{currentfill}{rgb}{0.900362,0.582587,0.000000}%
\pgfsetfillcolor{currentfill}%
\pgfsetfillopacity{0.800000}%
\pgfsetlinewidth{0.000000pt}%
\definecolor{currentstroke}{rgb}{0.000000,0.000000,0.000000}%
\pgfsetstrokecolor{currentstroke}%
\pgfsetdash{}{0pt}%
\pgfpathmoveto{\pgfqpoint{3.291890in}{3.244493in}}%
\pgfpathlineto{\pgfqpoint{3.221971in}{3.205266in}}%
\pgfpathlineto{\pgfqpoint{3.291890in}{3.242294in}}%
\pgfpathlineto{\pgfqpoint{3.361808in}{3.282042in}}%
\pgfpathlineto{\pgfqpoint{3.291890in}{3.244493in}}%
\pgfpathclose%
\pgfusepath{fill}%
\end{pgfscope}%
\begin{pgfscope}%
\pgfpathrectangle{\pgfqpoint{0.637500in}{0.550000in}}{\pgfqpoint{3.850000in}{3.850000in}}%
\pgfusepath{clip}%
\pgfsetbuttcap%
\pgfsetroundjoin%
\definecolor{currentfill}{rgb}{0.904905,0.585527,0.000000}%
\pgfsetfillcolor{currentfill}%
\pgfsetfillopacity{0.800000}%
\pgfsetlinewidth{0.000000pt}%
\definecolor{currentstroke}{rgb}{0.000000,0.000000,0.000000}%
\pgfsetstrokecolor{currentstroke}%
\pgfsetdash{}{0pt}%
\pgfpathmoveto{\pgfqpoint{3.082134in}{3.131420in}}%
\pgfpathlineto{\pgfqpoint{3.012215in}{3.095210in}}%
\pgfpathlineto{\pgfqpoint{3.082134in}{3.129221in}}%
\pgfpathlineto{\pgfqpoint{3.152052in}{3.166755in}}%
\pgfpathlineto{\pgfqpoint{3.082134in}{3.131420in}}%
\pgfpathclose%
\pgfusepath{fill}%
\end{pgfscope}%
\begin{pgfscope}%
\pgfpathrectangle{\pgfqpoint{0.637500in}{0.550000in}}{\pgfqpoint{3.850000in}{3.850000in}}%
\pgfusepath{clip}%
\pgfsetbuttcap%
\pgfsetroundjoin%
\definecolor{currentfill}{rgb}{0.914972,0.592040,0.000000}%
\pgfsetfillcolor{currentfill}%
\pgfsetfillopacity{0.800000}%
\pgfsetlinewidth{0.000000pt}%
\definecolor{currentstroke}{rgb}{0.000000,0.000000,0.000000}%
\pgfsetstrokecolor{currentstroke}%
\pgfsetdash{}{0pt}%
\pgfpathmoveto{\pgfqpoint{2.872378in}{3.030865in}}%
\pgfpathlineto{\pgfqpoint{2.802459in}{3.001780in}}%
\pgfpathlineto{\pgfqpoint{2.872378in}{3.028666in}}%
\pgfpathlineto{\pgfqpoint{2.942296in}{3.060770in}}%
\pgfpathlineto{\pgfqpoint{2.872378in}{3.030865in}}%
\pgfpathclose%
\pgfusepath{fill}%
\end{pgfscope}%
\begin{pgfscope}%
\pgfpathrectangle{\pgfqpoint{0.637500in}{0.550000in}}{\pgfqpoint{3.850000in}{3.850000in}}%
\pgfusepath{clip}%
\pgfsetbuttcap%
\pgfsetroundjoin%
\definecolor{currentfill}{rgb}{0.902018,0.583658,0.000000}%
\pgfsetfillcolor{currentfill}%
\pgfsetfillopacity{0.800000}%
\pgfsetlinewidth{0.000000pt}%
\definecolor{currentstroke}{rgb}{0.000000,0.000000,0.000000}%
\pgfsetstrokecolor{currentstroke}%
\pgfsetdash{}{0pt}%
\pgfpathmoveto{\pgfqpoint{1.823598in}{3.152446in}}%
\pgfpathlineto{\pgfqpoint{1.753679in}{3.188385in}}%
\pgfpathlineto{\pgfqpoint{1.823598in}{3.150248in}}%
\pgfpathlineto{\pgfqpoint{1.893516in}{3.115420in}}%
\pgfpathlineto{\pgfqpoint{1.823598in}{3.152446in}}%
\pgfpathclose%
\pgfusepath{fill}%
\end{pgfscope}%
\begin{pgfscope}%
\pgfpathrectangle{\pgfqpoint{0.637500in}{0.550000in}}{\pgfqpoint{3.850000in}{3.850000in}}%
\pgfusepath{clip}%
\pgfsetbuttcap%
\pgfsetroundjoin%
\definecolor{currentfill}{rgb}{0.908810,0.588053,0.000000}%
\pgfsetfillcolor{currentfill}%
\pgfsetfillopacity{0.800000}%
\pgfsetlinewidth{0.000000pt}%
\definecolor{currentstroke}{rgb}{0.000000,0.000000,0.000000}%
\pgfsetstrokecolor{currentstroke}%
\pgfsetdash{}{0pt}%
\pgfpathmoveto{\pgfqpoint{2.033354in}{3.048553in}}%
\pgfpathlineto{\pgfqpoint{1.963435in}{3.079890in}}%
\pgfpathlineto{\pgfqpoint{2.033354in}{3.046354in}}%
\pgfpathlineto{\pgfqpoint{2.103272in}{3.017624in}}%
\pgfpathlineto{\pgfqpoint{2.033354in}{3.048553in}}%
\pgfpathclose%
\pgfusepath{fill}%
\end{pgfscope}%
\begin{pgfscope}%
\pgfpathrectangle{\pgfqpoint{0.637500in}{0.550000in}}{\pgfqpoint{3.850000in}{3.850000in}}%
\pgfusepath{clip}%
\pgfsetbuttcap%
\pgfsetroundjoin%
\definecolor{currentfill}{rgb}{0.935716,0.605463,0.000000}%
\pgfsetfillcolor{currentfill}%
\pgfsetfillopacity{0.800000}%
\pgfsetlinewidth{0.000000pt}%
\definecolor{currentstroke}{rgb}{0.000000,0.000000,0.000000}%
\pgfsetstrokecolor{currentstroke}%
\pgfsetdash{}{0pt}%
\pgfpathmoveto{\pgfqpoint{2.522784in}{2.936064in}}%
\pgfpathlineto{\pgfqpoint{2.452866in}{2.933571in}}%
\pgfpathlineto{\pgfqpoint{2.522784in}{2.933865in}}%
\pgfpathlineto{\pgfqpoint{2.592703in}{2.942903in}}%
\pgfpathlineto{\pgfqpoint{2.522784in}{2.936064in}}%
\pgfpathclose%
\pgfusepath{fill}%
\end{pgfscope}%
\begin{pgfscope}%
\pgfpathrectangle{\pgfqpoint{0.637500in}{0.550000in}}{\pgfqpoint{3.850000in}{3.850000in}}%
\pgfusepath{clip}%
\pgfsetbuttcap%
\pgfsetroundjoin%
\definecolor{currentfill}{rgb}{0.931871,0.602975,0.000000}%
\pgfsetfillcolor{currentfill}%
\pgfsetfillopacity{0.800000}%
\pgfsetlinewidth{0.000000pt}%
\definecolor{currentstroke}{rgb}{0.000000,0.000000,0.000000}%
\pgfsetstrokecolor{currentstroke}%
\pgfsetdash{}{0pt}%
\pgfpathmoveto{\pgfqpoint{2.382947in}{2.939898in}}%
\pgfpathlineto{\pgfqpoint{2.313028in}{2.950367in}}%
\pgfpathlineto{\pgfqpoint{2.382947in}{2.937699in}}%
\pgfpathlineto{\pgfqpoint{2.452866in}{2.933571in}}%
\pgfpathlineto{\pgfqpoint{2.382947in}{2.939898in}}%
\pgfpathclose%
\pgfusepath{fill}%
\end{pgfscope}%
\begin{pgfscope}%
\pgfpathrectangle{\pgfqpoint{0.637500in}{0.550000in}}{\pgfqpoint{3.850000in}{3.850000in}}%
\pgfusepath{clip}%
\pgfsetbuttcap%
\pgfsetroundjoin%
\definecolor{currentfill}{rgb}{0.930057,0.601801,0.000000}%
\pgfsetfillcolor{currentfill}%
\pgfsetfillopacity{0.800000}%
\pgfsetlinewidth{0.000000pt}%
\definecolor{currentstroke}{rgb}{0.000000,0.000000,0.000000}%
\pgfsetstrokecolor{currentstroke}%
\pgfsetdash{}{0pt}%
\pgfpathmoveto{\pgfqpoint{2.662622in}{2.958072in}}%
\pgfpathlineto{\pgfqpoint{2.592703in}{2.942903in}}%
\pgfpathlineto{\pgfqpoint{2.662622in}{2.955873in}}%
\pgfpathlineto{\pgfqpoint{2.732540in}{2.976502in}}%
\pgfpathlineto{\pgfqpoint{2.662622in}{2.958072in}}%
\pgfpathclose%
\pgfusepath{fill}%
\end{pgfscope}%
\begin{pgfscope}%
\pgfpathrectangle{\pgfqpoint{0.637500in}{0.550000in}}{\pgfqpoint{3.850000in}{3.850000in}}%
\pgfusepath{clip}%
\pgfsetbuttcap%
\pgfsetroundjoin%
\definecolor{currentfill}{rgb}{0.901452,0.583292,0.000000}%
\pgfsetfillcolor{currentfill}%
\pgfsetfillopacity{0.800000}%
\pgfsetlinewidth{0.000000pt}%
\definecolor{currentstroke}{rgb}{0.000000,0.000000,0.000000}%
\pgfsetstrokecolor{currentstroke}%
\pgfsetdash{}{0pt}%
\pgfpathmoveto{\pgfqpoint{3.221971in}{3.205266in}}%
\pgfpathlineto{\pgfqpoint{3.152052in}{3.166755in}}%
\pgfpathlineto{\pgfqpoint{3.221971in}{3.203067in}}%
\pgfpathlineto{\pgfqpoint{3.291890in}{3.242294in}}%
\pgfpathlineto{\pgfqpoint{3.221971in}{3.205266in}}%
\pgfpathclose%
\pgfusepath{fill}%
\end{pgfscope}%
\begin{pgfscope}%
\pgfpathrectangle{\pgfqpoint{0.637500in}{0.550000in}}{\pgfqpoint{3.850000in}{3.850000in}}%
\pgfusepath{clip}%
\pgfsetbuttcap%
\pgfsetroundjoin%
\definecolor{currentfill}{rgb}{0.921990,0.596582,0.000000}%
\pgfsetfillcolor{currentfill}%
\pgfsetfillopacity{0.800000}%
\pgfsetlinewidth{0.000000pt}%
\definecolor{currentstroke}{rgb}{0.000000,0.000000,0.000000}%
\pgfsetstrokecolor{currentstroke}%
\pgfsetdash{}{0pt}%
\pgfpathmoveto{\pgfqpoint{2.243110in}{2.968801in}}%
\pgfpathlineto{\pgfqpoint{2.173191in}{2.990035in}}%
\pgfpathlineto{\pgfqpoint{2.243110in}{2.966602in}}%
\pgfpathlineto{\pgfqpoint{2.313028in}{2.950367in}}%
\pgfpathlineto{\pgfqpoint{2.243110in}{2.968801in}}%
\pgfpathclose%
\pgfusepath{fill}%
\end{pgfscope}%
\begin{pgfscope}%
\pgfpathrectangle{\pgfqpoint{0.637500in}{0.550000in}}{\pgfqpoint{3.850000in}{3.850000in}}%
\pgfusepath{clip}%
\pgfsetbuttcap%
\pgfsetroundjoin%
\definecolor{currentfill}{rgb}{0.907506,0.587210,0.000000}%
\pgfsetfillcolor{currentfill}%
\pgfsetfillopacity{0.800000}%
\pgfsetlinewidth{0.000000pt}%
\definecolor{currentstroke}{rgb}{0.000000,0.000000,0.000000}%
\pgfsetstrokecolor{currentstroke}%
\pgfsetdash{}{0pt}%
\pgfpathmoveto{\pgfqpoint{3.012215in}{3.095210in}}%
\pgfpathlineto{\pgfqpoint{2.942296in}{3.060770in}}%
\pgfpathlineto{\pgfqpoint{3.012215in}{3.093011in}}%
\pgfpathlineto{\pgfqpoint{3.082134in}{3.129221in}}%
\pgfpathlineto{\pgfqpoint{3.012215in}{3.095210in}}%
\pgfpathclose%
\pgfusepath{fill}%
\end{pgfscope}%
\begin{pgfscope}%
\pgfpathrectangle{\pgfqpoint{0.637500in}{0.550000in}}{\pgfqpoint{3.850000in}{3.850000in}}%
\pgfusepath{clip}%
\pgfsetbuttcap%
\pgfsetroundjoin%
\definecolor{currentfill}{rgb}{0.899869,0.582268,0.000000}%
\pgfsetfillcolor{currentfill}%
\pgfsetfillopacity{0.800000}%
\pgfsetlinewidth{0.000000pt}%
\definecolor{currentstroke}{rgb}{0.000000,0.000000,0.000000}%
\pgfsetstrokecolor{currentstroke}%
\pgfsetdash{}{0pt}%
\pgfpathmoveto{\pgfqpoint{1.683761in}{3.227339in}}%
\pgfpathlineto{\pgfqpoint{1.613842in}{3.264690in}}%
\pgfpathlineto{\pgfqpoint{1.683761in}{3.225140in}}%
\pgfpathlineto{\pgfqpoint{1.753679in}{3.188385in}}%
\pgfpathlineto{\pgfqpoint{1.683761in}{3.227339in}}%
\pgfpathclose%
\pgfusepath{fill}%
\end{pgfscope}%
\begin{pgfscope}%
\pgfpathrectangle{\pgfqpoint{0.637500in}{0.550000in}}{\pgfqpoint{3.850000in}{3.850000in}}%
\pgfusepath{clip}%
\pgfsetbuttcap%
\pgfsetroundjoin%
\definecolor{currentfill}{rgb}{0.903689,0.584740,0.000000}%
\pgfsetfillcolor{currentfill}%
\pgfsetfillopacity{0.800000}%
\pgfsetlinewidth{0.000000pt}%
\definecolor{currentstroke}{rgb}{0.000000,0.000000,0.000000}%
\pgfsetstrokecolor{currentstroke}%
\pgfsetdash{}{0pt}%
\pgfpathmoveto{\pgfqpoint{1.893516in}{3.115420in}}%
\pgfpathlineto{\pgfqpoint{1.823598in}{3.150248in}}%
\pgfpathlineto{\pgfqpoint{1.893516in}{3.113221in}}%
\pgfpathlineto{\pgfqpoint{1.963435in}{3.079890in}}%
\pgfpathlineto{\pgfqpoint{1.893516in}{3.115420in}}%
\pgfpathclose%
\pgfusepath{fill}%
\end{pgfscope}%
\begin{pgfscope}%
\pgfpathrectangle{\pgfqpoint{0.637500in}{0.550000in}}{\pgfqpoint{3.850000in}{3.850000in}}%
\pgfusepath{clip}%
\pgfsetbuttcap%
\pgfsetroundjoin%
\definecolor{currentfill}{rgb}{0.919814,0.595174,0.000000}%
\pgfsetfillcolor{currentfill}%
\pgfsetfillopacity{0.800000}%
\pgfsetlinewidth{0.000000pt}%
\definecolor{currentstroke}{rgb}{0.000000,0.000000,0.000000}%
\pgfsetstrokecolor{currentstroke}%
\pgfsetdash{}{0pt}%
\pgfpathmoveto{\pgfqpoint{2.802459in}{3.001780in}}%
\pgfpathlineto{\pgfqpoint{2.732540in}{2.976502in}}%
\pgfpathlineto{\pgfqpoint{2.802459in}{2.999581in}}%
\pgfpathlineto{\pgfqpoint{2.872378in}{3.028666in}}%
\pgfpathlineto{\pgfqpoint{2.802459in}{3.001780in}}%
\pgfpathclose%
\pgfusepath{fill}%
\end{pgfscope}%
\begin{pgfscope}%
\pgfpathrectangle{\pgfqpoint{0.637500in}{0.550000in}}{\pgfqpoint{3.850000in}{3.850000in}}%
\pgfusepath{clip}%
\pgfsetbuttcap%
\pgfsetroundjoin%
\definecolor{currentfill}{rgb}{0.912477,0.590427,0.000000}%
\pgfsetfillcolor{currentfill}%
\pgfsetfillopacity{0.800000}%
\pgfsetlinewidth{0.000000pt}%
\definecolor{currentstroke}{rgb}{0.000000,0.000000,0.000000}%
\pgfsetstrokecolor{currentstroke}%
\pgfsetdash{}{0pt}%
\pgfpathmoveto{\pgfqpoint{2.103272in}{3.017624in}}%
\pgfpathlineto{\pgfqpoint{2.033354in}{3.046354in}}%
\pgfpathlineto{\pgfqpoint{2.103272in}{3.015426in}}%
\pgfpathlineto{\pgfqpoint{2.173191in}{2.990035in}}%
\pgfpathlineto{\pgfqpoint{2.103272in}{3.017624in}}%
\pgfpathclose%
\pgfusepath{fill}%
\end{pgfscope}%
\begin{pgfscope}%
\pgfpathrectangle{\pgfqpoint{0.637500in}{0.550000in}}{\pgfqpoint{3.850000in}{3.850000in}}%
\pgfusepath{clip}%
\pgfsetbuttcap%
\pgfsetroundjoin%
\definecolor{currentfill}{rgb}{0.899565,0.582072,0.000000}%
\pgfsetfillcolor{currentfill}%
\pgfsetfillopacity{0.800000}%
\pgfsetlinewidth{0.000000pt}%
\definecolor{currentstroke}{rgb}{0.000000,0.000000,0.000000}%
\pgfsetstrokecolor{currentstroke}%
\pgfsetdash{}{0pt}%
\pgfpathmoveto{\pgfqpoint{3.361808in}{3.282042in}}%
\pgfpathlineto{\pgfqpoint{3.291890in}{3.242294in}}%
\pgfpathlineto{\pgfqpoint{3.361808in}{3.279843in}}%
\pgfpathlineto{\pgfqpoint{3.431727in}{3.319969in}}%
\pgfpathlineto{\pgfqpoint{3.361808in}{3.282042in}}%
\pgfpathclose%
\pgfusepath{fill}%
\end{pgfscope}%
\begin{pgfscope}%
\pgfpathrectangle{\pgfqpoint{0.637500in}{0.550000in}}{\pgfqpoint{3.850000in}{3.850000in}}%
\pgfusepath{clip}%
\pgfsetbuttcap%
\pgfsetroundjoin%
\definecolor{currentfill}{rgb}{0.902928,0.584248,0.000000}%
\pgfsetfillcolor{currentfill}%
\pgfsetfillopacity{0.800000}%
\pgfsetlinewidth{0.000000pt}%
\definecolor{currentstroke}{rgb}{0.000000,0.000000,0.000000}%
\pgfsetstrokecolor{currentstroke}%
\pgfsetdash{}{0pt}%
\pgfpathmoveto{\pgfqpoint{3.152052in}{3.166755in}}%
\pgfpathlineto{\pgfqpoint{3.082134in}{3.129221in}}%
\pgfpathlineto{\pgfqpoint{3.152052in}{3.164556in}}%
\pgfpathlineto{\pgfqpoint{3.221971in}{3.203067in}}%
\pgfpathlineto{\pgfqpoint{3.152052in}{3.166755in}}%
\pgfpathclose%
\pgfusepath{fill}%
\end{pgfscope}%
\begin{pgfscope}%
\pgfpathrectangle{\pgfqpoint{0.637500in}{0.550000in}}{\pgfqpoint{3.850000in}{3.850000in}}%
\pgfusepath{clip}%
\pgfsetbuttcap%
\pgfsetroundjoin%
\definecolor{currentfill}{rgb}{0.934964,0.604977,0.000000}%
\pgfsetfillcolor{currentfill}%
\pgfsetfillopacity{0.800000}%
\pgfsetlinewidth{0.000000pt}%
\definecolor{currentstroke}{rgb}{0.000000,0.000000,0.000000}%
\pgfsetstrokecolor{currentstroke}%
\pgfsetdash{}{0pt}%
\pgfpathmoveto{\pgfqpoint{2.452866in}{2.933571in}}%
\pgfpathlineto{\pgfqpoint{2.382947in}{2.937699in}}%
\pgfpathlineto{\pgfqpoint{2.452866in}{2.931372in}}%
\pgfpathlineto{\pgfqpoint{2.522784in}{2.933865in}}%
\pgfpathlineto{\pgfqpoint{2.452866in}{2.933571in}}%
\pgfpathclose%
\pgfusepath{fill}%
\end{pgfscope}%
\begin{pgfscope}%
\pgfpathrectangle{\pgfqpoint{0.637500in}{0.550000in}}{\pgfqpoint{3.850000in}{3.850000in}}%
\pgfusepath{clip}%
\pgfsetbuttcap%
\pgfsetroundjoin%
\definecolor{currentfill}{rgb}{0.910843,0.589369,0.000000}%
\pgfsetfillcolor{currentfill}%
\pgfsetfillopacity{0.800000}%
\pgfsetlinewidth{0.000000pt}%
\definecolor{currentstroke}{rgb}{0.000000,0.000000,0.000000}%
\pgfsetstrokecolor{currentstroke}%
\pgfsetdash{}{0pt}%
\pgfpathmoveto{\pgfqpoint{2.942296in}{3.060770in}}%
\pgfpathlineto{\pgfqpoint{2.872378in}{3.028666in}}%
\pgfpathlineto{\pgfqpoint{2.942296in}{3.058571in}}%
\pgfpathlineto{\pgfqpoint{3.012215in}{3.093011in}}%
\pgfpathlineto{\pgfqpoint{2.942296in}{3.060770in}}%
\pgfpathclose%
\pgfusepath{fill}%
\end{pgfscope}%
\begin{pgfscope}%
\pgfpathrectangle{\pgfqpoint{0.637500in}{0.550000in}}{\pgfqpoint{3.850000in}{3.850000in}}%
\pgfusepath{clip}%
\pgfsetbuttcap%
\pgfsetroundjoin%
\definecolor{currentfill}{rgb}{0.933916,0.604299,0.000000}%
\pgfsetfillcolor{currentfill}%
\pgfsetfillopacity{0.800000}%
\pgfsetlinewidth{0.000000pt}%
\definecolor{currentstroke}{rgb}{0.000000,0.000000,0.000000}%
\pgfsetstrokecolor{currentstroke}%
\pgfsetdash{}{0pt}%
\pgfpathmoveto{\pgfqpoint{2.592703in}{2.942903in}}%
\pgfpathlineto{\pgfqpoint{2.522784in}{2.933865in}}%
\pgfpathlineto{\pgfqpoint{2.592703in}{2.940704in}}%
\pgfpathlineto{\pgfqpoint{2.662622in}{2.955873in}}%
\pgfpathlineto{\pgfqpoint{2.592703in}{2.942903in}}%
\pgfpathclose%
\pgfusepath{fill}%
\end{pgfscope}%
\begin{pgfscope}%
\pgfpathrectangle{\pgfqpoint{0.637500in}{0.550000in}}{\pgfqpoint{3.850000in}{3.850000in}}%
\pgfusepath{clip}%
\pgfsetbuttcap%
\pgfsetroundjoin%
\definecolor{currentfill}{rgb}{0.900779,0.582857,0.000000}%
\pgfsetfillcolor{currentfill}%
\pgfsetfillopacity{0.800000}%
\pgfsetlinewidth{0.000000pt}%
\definecolor{currentstroke}{rgb}{0.000000,0.000000,0.000000}%
\pgfsetstrokecolor{currentstroke}%
\pgfsetdash{}{0pt}%
\pgfpathmoveto{\pgfqpoint{1.753679in}{3.188385in}}%
\pgfpathlineto{\pgfqpoint{1.683761in}{3.225140in}}%
\pgfpathlineto{\pgfqpoint{1.753679in}{3.186186in}}%
\pgfpathlineto{\pgfqpoint{1.823598in}{3.150248in}}%
\pgfpathlineto{\pgfqpoint{1.753679in}{3.188385in}}%
\pgfpathclose%
\pgfusepath{fill}%
\end{pgfscope}%
\begin{pgfscope}%
\pgfpathrectangle{\pgfqpoint{0.637500in}{0.550000in}}{\pgfqpoint{3.850000in}{3.850000in}}%
\pgfusepath{clip}%
\pgfsetbuttcap%
\pgfsetroundjoin%
\definecolor{currentfill}{rgb}{0.927227,0.599970,0.000000}%
\pgfsetfillcolor{currentfill}%
\pgfsetfillopacity{0.800000}%
\pgfsetlinewidth{0.000000pt}%
\definecolor{currentstroke}{rgb}{0.000000,0.000000,0.000000}%
\pgfsetstrokecolor{currentstroke}%
\pgfsetdash{}{0pt}%
\pgfpathmoveto{\pgfqpoint{2.313028in}{2.950367in}}%
\pgfpathlineto{\pgfqpoint{2.243110in}{2.966602in}}%
\pgfpathlineto{\pgfqpoint{2.313028in}{2.948168in}}%
\pgfpathlineto{\pgfqpoint{2.382947in}{2.937699in}}%
\pgfpathlineto{\pgfqpoint{2.313028in}{2.950367in}}%
\pgfpathclose%
\pgfusepath{fill}%
\end{pgfscope}%
\begin{pgfscope}%
\pgfpathrectangle{\pgfqpoint{0.637500in}{0.550000in}}{\pgfqpoint{3.850000in}{3.850000in}}%
\pgfusepath{clip}%
\pgfsetbuttcap%
\pgfsetroundjoin%
\definecolor{currentfill}{rgb}{0.905912,0.586178,0.000000}%
\pgfsetfillcolor{currentfill}%
\pgfsetfillopacity{0.800000}%
\pgfsetlinewidth{0.000000pt}%
\definecolor{currentstroke}{rgb}{0.000000,0.000000,0.000000}%
\pgfsetstrokecolor{currentstroke}%
\pgfsetdash{}{0pt}%
\pgfpathmoveto{\pgfqpoint{1.963435in}{3.079890in}}%
\pgfpathlineto{\pgfqpoint{1.893516in}{3.113221in}}%
\pgfpathlineto{\pgfqpoint{1.963435in}{3.077692in}}%
\pgfpathlineto{\pgfqpoint{2.033354in}{3.046354in}}%
\pgfpathlineto{\pgfqpoint{1.963435in}{3.079890in}}%
\pgfpathclose%
\pgfusepath{fill}%
\end{pgfscope}%
\begin{pgfscope}%
\pgfpathrectangle{\pgfqpoint{0.637500in}{0.550000in}}{\pgfqpoint{3.850000in}{3.850000in}}%
\pgfusepath{clip}%
\pgfsetbuttcap%
\pgfsetroundjoin%
\definecolor{currentfill}{rgb}{0.925053,0.598564,0.000000}%
\pgfsetfillcolor{currentfill}%
\pgfsetfillopacity{0.800000}%
\pgfsetlinewidth{0.000000pt}%
\definecolor{currentstroke}{rgb}{0.000000,0.000000,0.000000}%
\pgfsetstrokecolor{currentstroke}%
\pgfsetdash{}{0pt}%
\pgfpathmoveto{\pgfqpoint{2.732540in}{2.976502in}}%
\pgfpathlineto{\pgfqpoint{2.662622in}{2.955873in}}%
\pgfpathlineto{\pgfqpoint{2.732540in}{2.974304in}}%
\pgfpathlineto{\pgfqpoint{2.802459in}{2.999581in}}%
\pgfpathlineto{\pgfqpoint{2.732540in}{2.976502in}}%
\pgfpathclose%
\pgfusepath{fill}%
\end{pgfscope}%
\begin{pgfscope}%
\pgfpathrectangle{\pgfqpoint{0.637500in}{0.550000in}}{\pgfqpoint{3.850000in}{3.850000in}}%
\pgfusepath{clip}%
\pgfsetbuttcap%
\pgfsetroundjoin%
\definecolor{currentfill}{rgb}{0.916926,0.593305,0.000000}%
\pgfsetfillcolor{currentfill}%
\pgfsetfillopacity{0.800000}%
\pgfsetlinewidth{0.000000pt}%
\definecolor{currentstroke}{rgb}{0.000000,0.000000,0.000000}%
\pgfsetstrokecolor{currentstroke}%
\pgfsetdash{}{0pt}%
\pgfpathmoveto{\pgfqpoint{2.173191in}{2.990035in}}%
\pgfpathlineto{\pgfqpoint{2.103272in}{3.015426in}}%
\pgfpathlineto{\pgfqpoint{2.173191in}{2.987836in}}%
\pgfpathlineto{\pgfqpoint{2.243110in}{2.966602in}}%
\pgfpathlineto{\pgfqpoint{2.173191in}{2.990035in}}%
\pgfpathclose%
\pgfusepath{fill}%
\end{pgfscope}%
\begin{pgfscope}%
\pgfpathrectangle{\pgfqpoint{0.637500in}{0.550000in}}{\pgfqpoint{3.850000in}{3.850000in}}%
\pgfusepath{clip}%
\pgfsetbuttcap%
\pgfsetroundjoin%
\definecolor{currentfill}{rgb}{0.900362,0.582587,0.000000}%
\pgfsetfillcolor{currentfill}%
\pgfsetfillopacity{0.800000}%
\pgfsetlinewidth{0.000000pt}%
\definecolor{currentstroke}{rgb}{0.000000,0.000000,0.000000}%
\pgfsetstrokecolor{currentstroke}%
\pgfsetdash{}{0pt}%
\pgfpathmoveto{\pgfqpoint{3.291890in}{3.242294in}}%
\pgfpathlineto{\pgfqpoint{3.221971in}{3.203067in}}%
\pgfpathlineto{\pgfqpoint{3.291890in}{3.240095in}}%
\pgfpathlineto{\pgfqpoint{3.361808in}{3.279843in}}%
\pgfpathlineto{\pgfqpoint{3.291890in}{3.242294in}}%
\pgfpathclose%
\pgfusepath{fill}%
\end{pgfscope}%
\begin{pgfscope}%
\pgfpathrectangle{\pgfqpoint{0.637500in}{0.550000in}}{\pgfqpoint{3.850000in}{3.850000in}}%
\pgfusepath{clip}%
\pgfsetbuttcap%
\pgfsetroundjoin%
\definecolor{currentfill}{rgb}{0.904905,0.585527,0.000000}%
\pgfsetfillcolor{currentfill}%
\pgfsetfillopacity{0.800000}%
\pgfsetlinewidth{0.000000pt}%
\definecolor{currentstroke}{rgb}{0.000000,0.000000,0.000000}%
\pgfsetstrokecolor{currentstroke}%
\pgfsetdash{}{0pt}%
\pgfpathmoveto{\pgfqpoint{3.082134in}{3.129221in}}%
\pgfpathlineto{\pgfqpoint{3.012215in}{3.093011in}}%
\pgfpathlineto{\pgfqpoint{3.082134in}{3.127023in}}%
\pgfpathlineto{\pgfqpoint{3.152052in}{3.164556in}}%
\pgfpathlineto{\pgfqpoint{3.082134in}{3.129221in}}%
\pgfpathclose%
\pgfusepath{fill}%
\end{pgfscope}%
\begin{pgfscope}%
\pgfpathrectangle{\pgfqpoint{0.637500in}{0.550000in}}{\pgfqpoint{3.850000in}{3.850000in}}%
\pgfusepath{clip}%
\pgfsetbuttcap%
\pgfsetroundjoin%
\definecolor{currentfill}{rgb}{0.899206,0.581839,0.000000}%
\pgfsetfillcolor{currentfill}%
\pgfsetfillopacity{0.800000}%
\pgfsetlinewidth{0.000000pt}%
\definecolor{currentstroke}{rgb}{0.000000,0.000000,0.000000}%
\pgfsetstrokecolor{currentstroke}%
\pgfsetdash{}{0pt}%
\pgfpathmoveto{\pgfqpoint{1.613842in}{3.264690in}}%
\pgfpathlineto{\pgfqpoint{1.543923in}{3.302473in}}%
\pgfpathlineto{\pgfqpoint{1.613842in}{3.262491in}}%
\pgfpathlineto{\pgfqpoint{1.683761in}{3.225140in}}%
\pgfpathlineto{\pgfqpoint{1.613842in}{3.264690in}}%
\pgfpathclose%
\pgfusepath{fill}%
\end{pgfscope}%
\begin{pgfscope}%
\pgfpathrectangle{\pgfqpoint{0.637500in}{0.550000in}}{\pgfqpoint{3.850000in}{3.850000in}}%
\pgfusepath{clip}%
\pgfsetbuttcap%
\pgfsetroundjoin%
\definecolor{currentfill}{rgb}{0.914972,0.592040,0.000000}%
\pgfsetfillcolor{currentfill}%
\pgfsetfillopacity{0.800000}%
\pgfsetlinewidth{0.000000pt}%
\definecolor{currentstroke}{rgb}{0.000000,0.000000,0.000000}%
\pgfsetstrokecolor{currentstroke}%
\pgfsetdash{}{0pt}%
\pgfpathmoveto{\pgfqpoint{2.872378in}{3.028666in}}%
\pgfpathlineto{\pgfqpoint{2.802459in}{2.999581in}}%
\pgfpathlineto{\pgfqpoint{2.872378in}{3.026467in}}%
\pgfpathlineto{\pgfqpoint{2.942296in}{3.058571in}}%
\pgfpathlineto{\pgfqpoint{2.872378in}{3.028666in}}%
\pgfpathclose%
\pgfusepath{fill}%
\end{pgfscope}%
\begin{pgfscope}%
\pgfpathrectangle{\pgfqpoint{0.637500in}{0.550000in}}{\pgfqpoint{3.850000in}{3.850000in}}%
\pgfusepath{clip}%
\pgfsetbuttcap%
\pgfsetroundjoin%
\definecolor{currentfill}{rgb}{0.902018,0.583658,0.000000}%
\pgfsetfillcolor{currentfill}%
\pgfsetfillopacity{0.800000}%
\pgfsetlinewidth{0.000000pt}%
\definecolor{currentstroke}{rgb}{0.000000,0.000000,0.000000}%
\pgfsetstrokecolor{currentstroke}%
\pgfsetdash{}{0pt}%
\pgfpathmoveto{\pgfqpoint{1.823598in}{3.150248in}}%
\pgfpathlineto{\pgfqpoint{1.753679in}{3.186186in}}%
\pgfpathlineto{\pgfqpoint{1.823598in}{3.148049in}}%
\pgfpathlineto{\pgfqpoint{1.893516in}{3.113221in}}%
\pgfpathlineto{\pgfqpoint{1.823598in}{3.150248in}}%
\pgfpathclose%
\pgfusepath{fill}%
\end{pgfscope}%
\begin{pgfscope}%
\pgfpathrectangle{\pgfqpoint{0.637500in}{0.550000in}}{\pgfqpoint{3.850000in}{3.850000in}}%
\pgfusepath{clip}%
\pgfsetbuttcap%
\pgfsetroundjoin%
\definecolor{currentfill}{rgb}{0.908810,0.588053,0.000000}%
\pgfsetfillcolor{currentfill}%
\pgfsetfillopacity{0.800000}%
\pgfsetlinewidth{0.000000pt}%
\definecolor{currentstroke}{rgb}{0.000000,0.000000,0.000000}%
\pgfsetstrokecolor{currentstroke}%
\pgfsetdash{}{0pt}%
\pgfpathmoveto{\pgfqpoint{2.033354in}{3.046354in}}%
\pgfpathlineto{\pgfqpoint{1.963435in}{3.077692in}}%
\pgfpathlineto{\pgfqpoint{2.033354in}{3.044155in}}%
\pgfpathlineto{\pgfqpoint{2.103272in}{3.015426in}}%
\pgfpathlineto{\pgfqpoint{2.033354in}{3.046354in}}%
\pgfpathclose%
\pgfusepath{fill}%
\end{pgfscope}%
\begin{pgfscope}%
\pgfpathrectangle{\pgfqpoint{0.637500in}{0.550000in}}{\pgfqpoint{3.850000in}{3.850000in}}%
\pgfusepath{clip}%
\pgfsetbuttcap%
\pgfsetroundjoin%
\definecolor{currentfill}{rgb}{0.935716,0.605463,0.000000}%
\pgfsetfillcolor{currentfill}%
\pgfsetfillopacity{0.800000}%
\pgfsetlinewidth{0.000000pt}%
\definecolor{currentstroke}{rgb}{0.000000,0.000000,0.000000}%
\pgfsetstrokecolor{currentstroke}%
\pgfsetdash{}{0pt}%
\pgfpathmoveto{\pgfqpoint{2.522784in}{2.933865in}}%
\pgfpathlineto{\pgfqpoint{2.452866in}{2.931372in}}%
\pgfpathlineto{\pgfqpoint{2.522784in}{2.931666in}}%
\pgfpathlineto{\pgfqpoint{2.592703in}{2.940704in}}%
\pgfpathlineto{\pgfqpoint{2.522784in}{2.933865in}}%
\pgfpathclose%
\pgfusepath{fill}%
\end{pgfscope}%
\begin{pgfscope}%
\pgfpathrectangle{\pgfqpoint{0.637500in}{0.550000in}}{\pgfqpoint{3.850000in}{3.850000in}}%
\pgfusepath{clip}%
\pgfsetbuttcap%
\pgfsetroundjoin%
\definecolor{currentfill}{rgb}{0.931871,0.602975,0.000000}%
\pgfsetfillcolor{currentfill}%
\pgfsetfillopacity{0.800000}%
\pgfsetlinewidth{0.000000pt}%
\definecolor{currentstroke}{rgb}{0.000000,0.000000,0.000000}%
\pgfsetstrokecolor{currentstroke}%
\pgfsetdash{}{0pt}%
\pgfpathmoveto{\pgfqpoint{2.382947in}{2.937699in}}%
\pgfpathlineto{\pgfqpoint{2.313028in}{2.948168in}}%
\pgfpathlineto{\pgfqpoint{2.382947in}{2.935500in}}%
\pgfpathlineto{\pgfqpoint{2.452866in}{2.931372in}}%
\pgfpathlineto{\pgfqpoint{2.382947in}{2.937699in}}%
\pgfpathclose%
\pgfusepath{fill}%
\end{pgfscope}%
\begin{pgfscope}%
\pgfpathrectangle{\pgfqpoint{0.637500in}{0.550000in}}{\pgfqpoint{3.850000in}{3.850000in}}%
\pgfusepath{clip}%
\pgfsetbuttcap%
\pgfsetroundjoin%
\definecolor{currentfill}{rgb}{0.930057,0.601801,0.000000}%
\pgfsetfillcolor{currentfill}%
\pgfsetfillopacity{0.800000}%
\pgfsetlinewidth{0.000000pt}%
\definecolor{currentstroke}{rgb}{0.000000,0.000000,0.000000}%
\pgfsetstrokecolor{currentstroke}%
\pgfsetdash{}{0pt}%
\pgfpathmoveto{\pgfqpoint{2.662622in}{2.955873in}}%
\pgfpathlineto{\pgfqpoint{2.592703in}{2.940704in}}%
\pgfpathlineto{\pgfqpoint{2.662622in}{2.953674in}}%
\pgfpathlineto{\pgfqpoint{2.732540in}{2.974304in}}%
\pgfpathlineto{\pgfqpoint{2.662622in}{2.955873in}}%
\pgfpathclose%
\pgfusepath{fill}%
\end{pgfscope}%
\begin{pgfscope}%
\pgfpathrectangle{\pgfqpoint{0.637500in}{0.550000in}}{\pgfqpoint{3.850000in}{3.850000in}}%
\pgfusepath{clip}%
\pgfsetbuttcap%
\pgfsetroundjoin%
\definecolor{currentfill}{rgb}{0.898985,0.581696,0.000000}%
\pgfsetfillcolor{currentfill}%
\pgfsetfillopacity{0.800000}%
\pgfsetlinewidth{0.000000pt}%
\definecolor{currentstroke}{rgb}{0.000000,0.000000,0.000000}%
\pgfsetstrokecolor{currentstroke}%
\pgfsetdash{}{0pt}%
\pgfpathmoveto{\pgfqpoint{3.431727in}{3.319969in}}%
\pgfpathlineto{\pgfqpoint{3.361808in}{3.279843in}}%
\pgfpathlineto{\pgfqpoint{3.431727in}{3.317770in}}%
\pgfpathlineto{\pgfqpoint{3.501646in}{3.358168in}}%
\pgfpathlineto{\pgfqpoint{3.431727in}{3.319969in}}%
\pgfpathclose%
\pgfusepath{fill}%
\end{pgfscope}%
\begin{pgfscope}%
\pgfpathrectangle{\pgfqpoint{0.637500in}{0.550000in}}{\pgfqpoint{3.850000in}{3.850000in}}%
\pgfusepath{clip}%
\pgfsetbuttcap%
\pgfsetroundjoin%
\definecolor{currentfill}{rgb}{0.901452,0.583292,0.000000}%
\pgfsetfillcolor{currentfill}%
\pgfsetfillopacity{0.800000}%
\pgfsetlinewidth{0.000000pt}%
\definecolor{currentstroke}{rgb}{0.000000,0.000000,0.000000}%
\pgfsetstrokecolor{currentstroke}%
\pgfsetdash{}{0pt}%
\pgfpathmoveto{\pgfqpoint{3.221971in}{3.203067in}}%
\pgfpathlineto{\pgfqpoint{3.152052in}{3.164556in}}%
\pgfpathlineto{\pgfqpoint{3.221971in}{3.200868in}}%
\pgfpathlineto{\pgfqpoint{3.291890in}{3.240095in}}%
\pgfpathlineto{\pgfqpoint{3.221971in}{3.203067in}}%
\pgfpathclose%
\pgfusepath{fill}%
\end{pgfscope}%
\begin{pgfscope}%
\pgfpathrectangle{\pgfqpoint{0.637500in}{0.550000in}}{\pgfqpoint{3.850000in}{3.850000in}}%
\pgfusepath{clip}%
\pgfsetbuttcap%
\pgfsetroundjoin%
\definecolor{currentfill}{rgb}{0.921990,0.596582,0.000000}%
\pgfsetfillcolor{currentfill}%
\pgfsetfillopacity{0.800000}%
\pgfsetlinewidth{0.000000pt}%
\definecolor{currentstroke}{rgb}{0.000000,0.000000,0.000000}%
\pgfsetstrokecolor{currentstroke}%
\pgfsetdash{}{0pt}%
\pgfpathmoveto{\pgfqpoint{2.243110in}{2.966602in}}%
\pgfpathlineto{\pgfqpoint{2.173191in}{2.987836in}}%
\pgfpathlineto{\pgfqpoint{2.243110in}{2.964404in}}%
\pgfpathlineto{\pgfqpoint{2.313028in}{2.948168in}}%
\pgfpathlineto{\pgfqpoint{2.243110in}{2.966602in}}%
\pgfpathclose%
\pgfusepath{fill}%
\end{pgfscope}%
\begin{pgfscope}%
\pgfpathrectangle{\pgfqpoint{0.637500in}{0.550000in}}{\pgfqpoint{3.850000in}{3.850000in}}%
\pgfusepath{clip}%
\pgfsetbuttcap%
\pgfsetroundjoin%
\definecolor{currentfill}{rgb}{0.907506,0.587210,0.000000}%
\pgfsetfillcolor{currentfill}%
\pgfsetfillopacity{0.800000}%
\pgfsetlinewidth{0.000000pt}%
\definecolor{currentstroke}{rgb}{0.000000,0.000000,0.000000}%
\pgfsetstrokecolor{currentstroke}%
\pgfsetdash{}{0pt}%
\pgfpathmoveto{\pgfqpoint{3.012215in}{3.093011in}}%
\pgfpathlineto{\pgfqpoint{2.942296in}{3.058571in}}%
\pgfpathlineto{\pgfqpoint{3.012215in}{3.090812in}}%
\pgfpathlineto{\pgfqpoint{3.082134in}{3.127023in}}%
\pgfpathlineto{\pgfqpoint{3.012215in}{3.093011in}}%
\pgfpathclose%
\pgfusepath{fill}%
\end{pgfscope}%
\begin{pgfscope}%
\pgfpathrectangle{\pgfqpoint{0.637500in}{0.550000in}}{\pgfqpoint{3.850000in}{3.850000in}}%
\pgfusepath{clip}%
\pgfsetbuttcap%
\pgfsetroundjoin%
\definecolor{currentfill}{rgb}{0.899869,0.582268,0.000000}%
\pgfsetfillcolor{currentfill}%
\pgfsetfillopacity{0.800000}%
\pgfsetlinewidth{0.000000pt}%
\definecolor{currentstroke}{rgb}{0.000000,0.000000,0.000000}%
\pgfsetstrokecolor{currentstroke}%
\pgfsetdash{}{0pt}%
\pgfpathmoveto{\pgfqpoint{1.683761in}{3.225140in}}%
\pgfpathlineto{\pgfqpoint{1.613842in}{3.262491in}}%
\pgfpathlineto{\pgfqpoint{1.683761in}{3.222941in}}%
\pgfpathlineto{\pgfqpoint{1.753679in}{3.186186in}}%
\pgfpathlineto{\pgfqpoint{1.683761in}{3.225140in}}%
\pgfpathclose%
\pgfusepath{fill}%
\end{pgfscope}%
\begin{pgfscope}%
\pgfpathrectangle{\pgfqpoint{0.637500in}{0.550000in}}{\pgfqpoint{3.850000in}{3.850000in}}%
\pgfusepath{clip}%
\pgfsetbuttcap%
\pgfsetroundjoin%
\definecolor{currentfill}{rgb}{0.903689,0.584740,0.000000}%
\pgfsetfillcolor{currentfill}%
\pgfsetfillopacity{0.800000}%
\pgfsetlinewidth{0.000000pt}%
\definecolor{currentstroke}{rgb}{0.000000,0.000000,0.000000}%
\pgfsetstrokecolor{currentstroke}%
\pgfsetdash{}{0pt}%
\pgfpathmoveto{\pgfqpoint{1.893516in}{3.113221in}}%
\pgfpathlineto{\pgfqpoint{1.823598in}{3.148049in}}%
\pgfpathlineto{\pgfqpoint{1.893516in}{3.111022in}}%
\pgfpathlineto{\pgfqpoint{1.963435in}{3.077692in}}%
\pgfpathlineto{\pgfqpoint{1.893516in}{3.113221in}}%
\pgfpathclose%
\pgfusepath{fill}%
\end{pgfscope}%
\begin{pgfscope}%
\pgfpathrectangle{\pgfqpoint{0.637500in}{0.550000in}}{\pgfqpoint{3.850000in}{3.850000in}}%
\pgfusepath{clip}%
\pgfsetbuttcap%
\pgfsetroundjoin%
\definecolor{currentfill}{rgb}{0.919814,0.595174,0.000000}%
\pgfsetfillcolor{currentfill}%
\pgfsetfillopacity{0.800000}%
\pgfsetlinewidth{0.000000pt}%
\definecolor{currentstroke}{rgb}{0.000000,0.000000,0.000000}%
\pgfsetstrokecolor{currentstroke}%
\pgfsetdash{}{0pt}%
\pgfpathmoveto{\pgfqpoint{2.802459in}{2.999581in}}%
\pgfpathlineto{\pgfqpoint{2.732540in}{2.974304in}}%
\pgfpathlineto{\pgfqpoint{2.802459in}{2.997383in}}%
\pgfpathlineto{\pgfqpoint{2.872378in}{3.026467in}}%
\pgfpathlineto{\pgfqpoint{2.802459in}{2.999581in}}%
\pgfpathclose%
\pgfusepath{fill}%
\end{pgfscope}%
\begin{pgfscope}%
\pgfpathrectangle{\pgfqpoint{0.637500in}{0.550000in}}{\pgfqpoint{3.850000in}{3.850000in}}%
\pgfusepath{clip}%
\pgfsetbuttcap%
\pgfsetroundjoin%
\definecolor{currentfill}{rgb}{0.912477,0.590427,0.000000}%
\pgfsetfillcolor{currentfill}%
\pgfsetfillopacity{0.800000}%
\pgfsetlinewidth{0.000000pt}%
\definecolor{currentstroke}{rgb}{0.000000,0.000000,0.000000}%
\pgfsetstrokecolor{currentstroke}%
\pgfsetdash{}{0pt}%
\pgfpathmoveto{\pgfqpoint{2.103272in}{3.015426in}}%
\pgfpathlineto{\pgfqpoint{2.033354in}{3.044155in}}%
\pgfpathlineto{\pgfqpoint{2.103272in}{3.013227in}}%
\pgfpathlineto{\pgfqpoint{2.173191in}{2.987836in}}%
\pgfpathlineto{\pgfqpoint{2.103272in}{3.015426in}}%
\pgfpathclose%
\pgfusepath{fill}%
\end{pgfscope}%
\begin{pgfscope}%
\pgfpathrectangle{\pgfqpoint{0.637500in}{0.550000in}}{\pgfqpoint{3.850000in}{3.850000in}}%
\pgfusepath{clip}%
\pgfsetbuttcap%
\pgfsetroundjoin%
\definecolor{currentfill}{rgb}{0.899565,0.582072,0.000000}%
\pgfsetfillcolor{currentfill}%
\pgfsetfillopacity{0.800000}%
\pgfsetlinewidth{0.000000pt}%
\definecolor{currentstroke}{rgb}{0.000000,0.000000,0.000000}%
\pgfsetstrokecolor{currentstroke}%
\pgfsetdash{}{0pt}%
\pgfpathmoveto{\pgfqpoint{3.361808in}{3.279843in}}%
\pgfpathlineto{\pgfqpoint{3.291890in}{3.240095in}}%
\pgfpathlineto{\pgfqpoint{3.361808in}{3.277644in}}%
\pgfpathlineto{\pgfqpoint{3.431727in}{3.317770in}}%
\pgfpathlineto{\pgfqpoint{3.361808in}{3.279843in}}%
\pgfpathclose%
\pgfusepath{fill}%
\end{pgfscope}%
\begin{pgfscope}%
\pgfpathrectangle{\pgfqpoint{0.637500in}{0.550000in}}{\pgfqpoint{3.850000in}{3.850000in}}%
\pgfusepath{clip}%
\pgfsetbuttcap%
\pgfsetroundjoin%
\definecolor{currentfill}{rgb}{0.902928,0.584248,0.000000}%
\pgfsetfillcolor{currentfill}%
\pgfsetfillopacity{0.800000}%
\pgfsetlinewidth{0.000000pt}%
\definecolor{currentstroke}{rgb}{0.000000,0.000000,0.000000}%
\pgfsetstrokecolor{currentstroke}%
\pgfsetdash{}{0pt}%
\pgfpathmoveto{\pgfqpoint{3.152052in}{3.164556in}}%
\pgfpathlineto{\pgfqpoint{3.082134in}{3.127023in}}%
\pgfpathlineto{\pgfqpoint{3.152052in}{3.162357in}}%
\pgfpathlineto{\pgfqpoint{3.221971in}{3.200868in}}%
\pgfpathlineto{\pgfqpoint{3.152052in}{3.164556in}}%
\pgfpathclose%
\pgfusepath{fill}%
\end{pgfscope}%
\begin{pgfscope}%
\pgfpathrectangle{\pgfqpoint{0.637500in}{0.550000in}}{\pgfqpoint{3.850000in}{3.850000in}}%
\pgfusepath{clip}%
\pgfsetbuttcap%
\pgfsetroundjoin%
\definecolor{currentfill}{rgb}{0.934964,0.604977,0.000000}%
\pgfsetfillcolor{currentfill}%
\pgfsetfillopacity{0.800000}%
\pgfsetlinewidth{0.000000pt}%
\definecolor{currentstroke}{rgb}{0.000000,0.000000,0.000000}%
\pgfsetstrokecolor{currentstroke}%
\pgfsetdash{}{0pt}%
\pgfpathmoveto{\pgfqpoint{2.452866in}{2.931372in}}%
\pgfpathlineto{\pgfqpoint{2.382947in}{2.935500in}}%
\pgfpathlineto{\pgfqpoint{2.452866in}{2.929174in}}%
\pgfpathlineto{\pgfqpoint{2.522784in}{2.931666in}}%
\pgfpathlineto{\pgfqpoint{2.452866in}{2.931372in}}%
\pgfpathclose%
\pgfusepath{fill}%
\end{pgfscope}%
\begin{pgfscope}%
\pgfpathrectangle{\pgfqpoint{0.637500in}{0.550000in}}{\pgfqpoint{3.850000in}{3.850000in}}%
\pgfusepath{clip}%
\pgfsetbuttcap%
\pgfsetroundjoin%
\definecolor{currentfill}{rgb}{0.910843,0.589369,0.000000}%
\pgfsetfillcolor{currentfill}%
\pgfsetfillopacity{0.800000}%
\pgfsetlinewidth{0.000000pt}%
\definecolor{currentstroke}{rgb}{0.000000,0.000000,0.000000}%
\pgfsetstrokecolor{currentstroke}%
\pgfsetdash{}{0pt}%
\pgfpathmoveto{\pgfqpoint{2.942296in}{3.058571in}}%
\pgfpathlineto{\pgfqpoint{2.872378in}{3.026467in}}%
\pgfpathlineto{\pgfqpoint{2.942296in}{3.056372in}}%
\pgfpathlineto{\pgfqpoint{3.012215in}{3.090812in}}%
\pgfpathlineto{\pgfqpoint{2.942296in}{3.058571in}}%
\pgfpathclose%
\pgfusepath{fill}%
\end{pgfscope}%
\begin{pgfscope}%
\pgfpathrectangle{\pgfqpoint{0.637500in}{0.550000in}}{\pgfqpoint{3.850000in}{3.850000in}}%
\pgfusepath{clip}%
\pgfsetbuttcap%
\pgfsetroundjoin%
\definecolor{currentfill}{rgb}{0.933916,0.604299,0.000000}%
\pgfsetfillcolor{currentfill}%
\pgfsetfillopacity{0.800000}%
\pgfsetlinewidth{0.000000pt}%
\definecolor{currentstroke}{rgb}{0.000000,0.000000,0.000000}%
\pgfsetstrokecolor{currentstroke}%
\pgfsetdash{}{0pt}%
\pgfpathmoveto{\pgfqpoint{2.592703in}{2.940704in}}%
\pgfpathlineto{\pgfqpoint{2.522784in}{2.931666in}}%
\pgfpathlineto{\pgfqpoint{2.592703in}{2.938505in}}%
\pgfpathlineto{\pgfqpoint{2.662622in}{2.953674in}}%
\pgfpathlineto{\pgfqpoint{2.592703in}{2.940704in}}%
\pgfpathclose%
\pgfusepath{fill}%
\end{pgfscope}%
\begin{pgfscope}%
\pgfpathrectangle{\pgfqpoint{0.637500in}{0.550000in}}{\pgfqpoint{3.850000in}{3.850000in}}%
\pgfusepath{clip}%
\pgfsetbuttcap%
\pgfsetroundjoin%
\definecolor{currentfill}{rgb}{0.898725,0.581528,0.000000}%
\pgfsetfillcolor{currentfill}%
\pgfsetfillopacity{0.800000}%
\pgfsetlinewidth{0.000000pt}%
\definecolor{currentstroke}{rgb}{0.000000,0.000000,0.000000}%
\pgfsetstrokecolor{currentstroke}%
\pgfsetdash{}{0pt}%
\pgfpathmoveto{\pgfqpoint{1.543923in}{3.302473in}}%
\pgfpathlineto{\pgfqpoint{1.474005in}{3.340569in}}%
\pgfpathlineto{\pgfqpoint{1.543923in}{3.300274in}}%
\pgfpathlineto{\pgfqpoint{1.613842in}{3.262491in}}%
\pgfpathlineto{\pgfqpoint{1.543923in}{3.302473in}}%
\pgfpathclose%
\pgfusepath{fill}%
\end{pgfscope}%
\begin{pgfscope}%
\pgfpathrectangle{\pgfqpoint{0.637500in}{0.550000in}}{\pgfqpoint{3.850000in}{3.850000in}}%
\pgfusepath{clip}%
\pgfsetbuttcap%
\pgfsetroundjoin%
\definecolor{currentfill}{rgb}{0.900779,0.582857,0.000000}%
\pgfsetfillcolor{currentfill}%
\pgfsetfillopacity{0.800000}%
\pgfsetlinewidth{0.000000pt}%
\definecolor{currentstroke}{rgb}{0.000000,0.000000,0.000000}%
\pgfsetstrokecolor{currentstroke}%
\pgfsetdash{}{0pt}%
\pgfpathmoveto{\pgfqpoint{1.753679in}{3.186186in}}%
\pgfpathlineto{\pgfqpoint{1.683761in}{3.222941in}}%
\pgfpathlineto{\pgfqpoint{1.753679in}{3.183988in}}%
\pgfpathlineto{\pgfqpoint{1.823598in}{3.148049in}}%
\pgfpathlineto{\pgfqpoint{1.753679in}{3.186186in}}%
\pgfpathclose%
\pgfusepath{fill}%
\end{pgfscope}%
\begin{pgfscope}%
\pgfpathrectangle{\pgfqpoint{0.637500in}{0.550000in}}{\pgfqpoint{3.850000in}{3.850000in}}%
\pgfusepath{clip}%
\pgfsetbuttcap%
\pgfsetroundjoin%
\definecolor{currentfill}{rgb}{0.927227,0.599970,0.000000}%
\pgfsetfillcolor{currentfill}%
\pgfsetfillopacity{0.800000}%
\pgfsetlinewidth{0.000000pt}%
\definecolor{currentstroke}{rgb}{0.000000,0.000000,0.000000}%
\pgfsetstrokecolor{currentstroke}%
\pgfsetdash{}{0pt}%
\pgfpathmoveto{\pgfqpoint{2.313028in}{2.948168in}}%
\pgfpathlineto{\pgfqpoint{2.243110in}{2.964404in}}%
\pgfpathlineto{\pgfqpoint{2.313028in}{2.945970in}}%
\pgfpathlineto{\pgfqpoint{2.382947in}{2.935500in}}%
\pgfpathlineto{\pgfqpoint{2.313028in}{2.948168in}}%
\pgfpathclose%
\pgfusepath{fill}%
\end{pgfscope}%
\begin{pgfscope}%
\pgfpathrectangle{\pgfqpoint{0.637500in}{0.550000in}}{\pgfqpoint{3.850000in}{3.850000in}}%
\pgfusepath{clip}%
\pgfsetbuttcap%
\pgfsetroundjoin%
\definecolor{currentfill}{rgb}{0.905912,0.586178,0.000000}%
\pgfsetfillcolor{currentfill}%
\pgfsetfillopacity{0.800000}%
\pgfsetlinewidth{0.000000pt}%
\definecolor{currentstroke}{rgb}{0.000000,0.000000,0.000000}%
\pgfsetstrokecolor{currentstroke}%
\pgfsetdash{}{0pt}%
\pgfpathmoveto{\pgfqpoint{1.963435in}{3.077692in}}%
\pgfpathlineto{\pgfqpoint{1.893516in}{3.111022in}}%
\pgfpathlineto{\pgfqpoint{1.963435in}{3.075493in}}%
\pgfpathlineto{\pgfqpoint{2.033354in}{3.044155in}}%
\pgfpathlineto{\pgfqpoint{1.963435in}{3.077692in}}%
\pgfpathclose%
\pgfusepath{fill}%
\end{pgfscope}%
\begin{pgfscope}%
\pgfpathrectangle{\pgfqpoint{0.637500in}{0.550000in}}{\pgfqpoint{3.850000in}{3.850000in}}%
\pgfusepath{clip}%
\pgfsetbuttcap%
\pgfsetroundjoin%
\definecolor{currentfill}{rgb}{0.925053,0.598564,0.000000}%
\pgfsetfillcolor{currentfill}%
\pgfsetfillopacity{0.800000}%
\pgfsetlinewidth{0.000000pt}%
\definecolor{currentstroke}{rgb}{0.000000,0.000000,0.000000}%
\pgfsetstrokecolor{currentstroke}%
\pgfsetdash{}{0pt}%
\pgfpathmoveto{\pgfqpoint{2.732540in}{2.974304in}}%
\pgfpathlineto{\pgfqpoint{2.662622in}{2.953674in}}%
\pgfpathlineto{\pgfqpoint{2.732540in}{2.972105in}}%
\pgfpathlineto{\pgfqpoint{2.802459in}{2.997383in}}%
\pgfpathlineto{\pgfqpoint{2.732540in}{2.974304in}}%
\pgfpathclose%
\pgfusepath{fill}%
\end{pgfscope}%
\begin{pgfscope}%
\pgfpathrectangle{\pgfqpoint{0.637500in}{0.550000in}}{\pgfqpoint{3.850000in}{3.850000in}}%
\pgfusepath{clip}%
\pgfsetbuttcap%
\pgfsetroundjoin%
\definecolor{currentfill}{rgb}{0.916926,0.593305,0.000000}%
\pgfsetfillcolor{currentfill}%
\pgfsetfillopacity{0.800000}%
\pgfsetlinewidth{0.000000pt}%
\definecolor{currentstroke}{rgb}{0.000000,0.000000,0.000000}%
\pgfsetstrokecolor{currentstroke}%
\pgfsetdash{}{0pt}%
\pgfpathmoveto{\pgfqpoint{2.173191in}{2.987836in}}%
\pgfpathlineto{\pgfqpoint{2.103272in}{3.013227in}}%
\pgfpathlineto{\pgfqpoint{2.173191in}{2.985637in}}%
\pgfpathlineto{\pgfqpoint{2.243110in}{2.964404in}}%
\pgfpathlineto{\pgfqpoint{2.173191in}{2.987836in}}%
\pgfpathclose%
\pgfusepath{fill}%
\end{pgfscope}%
\begin{pgfscope}%
\pgfpathrectangle{\pgfqpoint{0.637500in}{0.550000in}}{\pgfqpoint{3.850000in}{3.850000in}}%
\pgfusepath{clip}%
\pgfsetbuttcap%
\pgfsetroundjoin%
\definecolor{currentfill}{rgb}{0.898565,0.581425,0.000000}%
\pgfsetfillcolor{currentfill}%
\pgfsetfillopacity{0.800000}%
\pgfsetlinewidth{0.000000pt}%
\definecolor{currentstroke}{rgb}{0.000000,0.000000,0.000000}%
\pgfsetstrokecolor{currentstroke}%
\pgfsetdash{}{0pt}%
\pgfpathmoveto{\pgfqpoint{3.501646in}{3.358168in}}%
\pgfpathlineto{\pgfqpoint{3.431727in}{3.317770in}}%
\pgfpathlineto{\pgfqpoint{3.501646in}{3.355970in}}%
\pgfpathlineto{\pgfqpoint{3.571564in}{3.396565in}}%
\pgfpathlineto{\pgfqpoint{3.501646in}{3.358168in}}%
\pgfpathclose%
\pgfusepath{fill}%
\end{pgfscope}%
\begin{pgfscope}%
\pgfpathrectangle{\pgfqpoint{0.637500in}{0.550000in}}{\pgfqpoint{3.850000in}{3.850000in}}%
\pgfusepath{clip}%
\pgfsetbuttcap%
\pgfsetroundjoin%
\definecolor{currentfill}{rgb}{0.900362,0.582587,0.000000}%
\pgfsetfillcolor{currentfill}%
\pgfsetfillopacity{0.800000}%
\pgfsetlinewidth{0.000000pt}%
\definecolor{currentstroke}{rgb}{0.000000,0.000000,0.000000}%
\pgfsetstrokecolor{currentstroke}%
\pgfsetdash{}{0pt}%
\pgfpathmoveto{\pgfqpoint{3.291890in}{3.240095in}}%
\pgfpathlineto{\pgfqpoint{3.221971in}{3.200868in}}%
\pgfpathlineto{\pgfqpoint{3.291890in}{3.237896in}}%
\pgfpathlineto{\pgfqpoint{3.361808in}{3.277644in}}%
\pgfpathlineto{\pgfqpoint{3.291890in}{3.240095in}}%
\pgfpathclose%
\pgfusepath{fill}%
\end{pgfscope}%
\begin{pgfscope}%
\pgfpathrectangle{\pgfqpoint{0.637500in}{0.550000in}}{\pgfqpoint{3.850000in}{3.850000in}}%
\pgfusepath{clip}%
\pgfsetbuttcap%
\pgfsetroundjoin%
\definecolor{currentfill}{rgb}{0.904905,0.585527,0.000000}%
\pgfsetfillcolor{currentfill}%
\pgfsetfillopacity{0.800000}%
\pgfsetlinewidth{0.000000pt}%
\definecolor{currentstroke}{rgb}{0.000000,0.000000,0.000000}%
\pgfsetstrokecolor{currentstroke}%
\pgfsetdash{}{0pt}%
\pgfpathmoveto{\pgfqpoint{3.082134in}{3.127023in}}%
\pgfpathlineto{\pgfqpoint{3.012215in}{3.090812in}}%
\pgfpathlineto{\pgfqpoint{3.082134in}{3.124824in}}%
\pgfpathlineto{\pgfqpoint{3.152052in}{3.162357in}}%
\pgfpathlineto{\pgfqpoint{3.082134in}{3.127023in}}%
\pgfpathclose%
\pgfusepath{fill}%
\end{pgfscope}%
\begin{pgfscope}%
\pgfpathrectangle{\pgfqpoint{0.637500in}{0.550000in}}{\pgfqpoint{3.850000in}{3.850000in}}%
\pgfusepath{clip}%
\pgfsetbuttcap%
\pgfsetroundjoin%
\definecolor{currentfill}{rgb}{0.899206,0.581839,0.000000}%
\pgfsetfillcolor{currentfill}%
\pgfsetfillopacity{0.800000}%
\pgfsetlinewidth{0.000000pt}%
\definecolor{currentstroke}{rgb}{0.000000,0.000000,0.000000}%
\pgfsetstrokecolor{currentstroke}%
\pgfsetdash{}{0pt}%
\pgfpathmoveto{\pgfqpoint{1.613842in}{3.262491in}}%
\pgfpathlineto{\pgfqpoint{1.543923in}{3.300274in}}%
\pgfpathlineto{\pgfqpoint{1.613842in}{3.260292in}}%
\pgfpathlineto{\pgfqpoint{1.683761in}{3.222941in}}%
\pgfpathlineto{\pgfqpoint{1.613842in}{3.262491in}}%
\pgfpathclose%
\pgfusepath{fill}%
\end{pgfscope}%
\begin{pgfscope}%
\pgfpathrectangle{\pgfqpoint{0.637500in}{0.550000in}}{\pgfqpoint{3.850000in}{3.850000in}}%
\pgfusepath{clip}%
\pgfsetbuttcap%
\pgfsetroundjoin%
\definecolor{currentfill}{rgb}{0.914972,0.592040,0.000000}%
\pgfsetfillcolor{currentfill}%
\pgfsetfillopacity{0.800000}%
\pgfsetlinewidth{0.000000pt}%
\definecolor{currentstroke}{rgb}{0.000000,0.000000,0.000000}%
\pgfsetstrokecolor{currentstroke}%
\pgfsetdash{}{0pt}%
\pgfpathmoveto{\pgfqpoint{2.872378in}{3.026467in}}%
\pgfpathlineto{\pgfqpoint{2.802459in}{2.997383in}}%
\pgfpathlineto{\pgfqpoint{2.872378in}{3.024268in}}%
\pgfpathlineto{\pgfqpoint{2.942296in}{3.056372in}}%
\pgfpathlineto{\pgfqpoint{2.872378in}{3.026467in}}%
\pgfpathclose%
\pgfusepath{fill}%
\end{pgfscope}%
\begin{pgfscope}%
\pgfpathrectangle{\pgfqpoint{0.637500in}{0.550000in}}{\pgfqpoint{3.850000in}{3.850000in}}%
\pgfusepath{clip}%
\pgfsetbuttcap%
\pgfsetroundjoin%
\definecolor{currentfill}{rgb}{0.902018,0.583658,0.000000}%
\pgfsetfillcolor{currentfill}%
\pgfsetfillopacity{0.800000}%
\pgfsetlinewidth{0.000000pt}%
\definecolor{currentstroke}{rgb}{0.000000,0.000000,0.000000}%
\pgfsetstrokecolor{currentstroke}%
\pgfsetdash{}{0pt}%
\pgfpathmoveto{\pgfqpoint{1.823598in}{3.148049in}}%
\pgfpathlineto{\pgfqpoint{1.753679in}{3.183988in}}%
\pgfpathlineto{\pgfqpoint{1.823598in}{3.145850in}}%
\pgfpathlineto{\pgfqpoint{1.893516in}{3.111022in}}%
\pgfpathlineto{\pgfqpoint{1.823598in}{3.148049in}}%
\pgfpathclose%
\pgfusepath{fill}%
\end{pgfscope}%
\begin{pgfscope}%
\pgfpathrectangle{\pgfqpoint{0.637500in}{0.550000in}}{\pgfqpoint{3.850000in}{3.850000in}}%
\pgfusepath{clip}%
\pgfsetbuttcap%
\pgfsetroundjoin%
\definecolor{currentfill}{rgb}{0.908810,0.588053,0.000000}%
\pgfsetfillcolor{currentfill}%
\pgfsetfillopacity{0.800000}%
\pgfsetlinewidth{0.000000pt}%
\definecolor{currentstroke}{rgb}{0.000000,0.000000,0.000000}%
\pgfsetstrokecolor{currentstroke}%
\pgfsetdash{}{0pt}%
\pgfpathmoveto{\pgfqpoint{2.033354in}{3.044155in}}%
\pgfpathlineto{\pgfqpoint{1.963435in}{3.075493in}}%
\pgfpathlineto{\pgfqpoint{2.033354in}{3.041956in}}%
\pgfpathlineto{\pgfqpoint{2.103272in}{3.013227in}}%
\pgfpathlineto{\pgfqpoint{2.033354in}{3.044155in}}%
\pgfpathclose%
\pgfusepath{fill}%
\end{pgfscope}%
\begin{pgfscope}%
\pgfpathrectangle{\pgfqpoint{0.637500in}{0.550000in}}{\pgfqpoint{3.850000in}{3.850000in}}%
\pgfusepath{clip}%
\pgfsetbuttcap%
\pgfsetroundjoin%
\definecolor{currentfill}{rgb}{0.935716,0.605463,0.000000}%
\pgfsetfillcolor{currentfill}%
\pgfsetfillopacity{0.800000}%
\pgfsetlinewidth{0.000000pt}%
\definecolor{currentstroke}{rgb}{0.000000,0.000000,0.000000}%
\pgfsetstrokecolor{currentstroke}%
\pgfsetdash{}{0pt}%
\pgfpathmoveto{\pgfqpoint{2.522784in}{2.931666in}}%
\pgfpathlineto{\pgfqpoint{2.452866in}{2.929174in}}%
\pgfpathlineto{\pgfqpoint{2.522784in}{2.929467in}}%
\pgfpathlineto{\pgfqpoint{2.592703in}{2.938505in}}%
\pgfpathlineto{\pgfqpoint{2.522784in}{2.931666in}}%
\pgfpathclose%
\pgfusepath{fill}%
\end{pgfscope}%
\begin{pgfscope}%
\pgfpathrectangle{\pgfqpoint{0.637500in}{0.550000in}}{\pgfqpoint{3.850000in}{3.850000in}}%
\pgfusepath{clip}%
\pgfsetbuttcap%
\pgfsetroundjoin%
\definecolor{currentfill}{rgb}{0.931871,0.602975,0.000000}%
\pgfsetfillcolor{currentfill}%
\pgfsetfillopacity{0.800000}%
\pgfsetlinewidth{0.000000pt}%
\definecolor{currentstroke}{rgb}{0.000000,0.000000,0.000000}%
\pgfsetstrokecolor{currentstroke}%
\pgfsetdash{}{0pt}%
\pgfpathmoveto{\pgfqpoint{2.382947in}{2.935500in}}%
\pgfpathlineto{\pgfqpoint{2.313028in}{2.945970in}}%
\pgfpathlineto{\pgfqpoint{2.382947in}{2.933301in}}%
\pgfpathlineto{\pgfqpoint{2.452866in}{2.929174in}}%
\pgfpathlineto{\pgfqpoint{2.382947in}{2.935500in}}%
\pgfpathclose%
\pgfusepath{fill}%
\end{pgfscope}%
\begin{pgfscope}%
\pgfpathrectangle{\pgfqpoint{0.637500in}{0.550000in}}{\pgfqpoint{3.850000in}{3.850000in}}%
\pgfusepath{clip}%
\pgfsetbuttcap%
\pgfsetroundjoin%
\definecolor{currentfill}{rgb}{0.930057,0.601801,0.000000}%
\pgfsetfillcolor{currentfill}%
\pgfsetfillopacity{0.800000}%
\pgfsetlinewidth{0.000000pt}%
\definecolor{currentstroke}{rgb}{0.000000,0.000000,0.000000}%
\pgfsetstrokecolor{currentstroke}%
\pgfsetdash{}{0pt}%
\pgfpathmoveto{\pgfqpoint{2.662622in}{2.953674in}}%
\pgfpathlineto{\pgfqpoint{2.592703in}{2.938505in}}%
\pgfpathlineto{\pgfqpoint{2.662622in}{2.951475in}}%
\pgfpathlineto{\pgfqpoint{2.732540in}{2.972105in}}%
\pgfpathlineto{\pgfqpoint{2.662622in}{2.953674in}}%
\pgfpathclose%
\pgfusepath{fill}%
\end{pgfscope}%
\begin{pgfscope}%
\pgfpathrectangle{\pgfqpoint{0.637500in}{0.550000in}}{\pgfqpoint{3.850000in}{3.850000in}}%
\pgfusepath{clip}%
\pgfsetbuttcap%
\pgfsetroundjoin%
\definecolor{currentfill}{rgb}{0.898985,0.581696,0.000000}%
\pgfsetfillcolor{currentfill}%
\pgfsetfillopacity{0.800000}%
\pgfsetlinewidth{0.000000pt}%
\definecolor{currentstroke}{rgb}{0.000000,0.000000,0.000000}%
\pgfsetstrokecolor{currentstroke}%
\pgfsetdash{}{0pt}%
\pgfpathmoveto{\pgfqpoint{3.431727in}{3.317770in}}%
\pgfpathlineto{\pgfqpoint{3.361808in}{3.277644in}}%
\pgfpathlineto{\pgfqpoint{3.431727in}{3.315571in}}%
\pgfpathlineto{\pgfqpoint{3.501646in}{3.355970in}}%
\pgfpathlineto{\pgfqpoint{3.431727in}{3.317770in}}%
\pgfpathclose%
\pgfusepath{fill}%
\end{pgfscope}%
\begin{pgfscope}%
\pgfpathrectangle{\pgfqpoint{0.637500in}{0.550000in}}{\pgfqpoint{3.850000in}{3.850000in}}%
\pgfusepath{clip}%
\pgfsetbuttcap%
\pgfsetroundjoin%
\definecolor{currentfill}{rgb}{0.901452,0.583292,0.000000}%
\pgfsetfillcolor{currentfill}%
\pgfsetfillopacity{0.800000}%
\pgfsetlinewidth{0.000000pt}%
\definecolor{currentstroke}{rgb}{0.000000,0.000000,0.000000}%
\pgfsetstrokecolor{currentstroke}%
\pgfsetdash{}{0pt}%
\pgfpathmoveto{\pgfqpoint{3.221971in}{3.200868in}}%
\pgfpathlineto{\pgfqpoint{3.152052in}{3.162357in}}%
\pgfpathlineto{\pgfqpoint{3.221971in}{3.198670in}}%
\pgfpathlineto{\pgfqpoint{3.291890in}{3.237896in}}%
\pgfpathlineto{\pgfqpoint{3.221971in}{3.200868in}}%
\pgfpathclose%
\pgfusepath{fill}%
\end{pgfscope}%
\begin{pgfscope}%
\pgfpathrectangle{\pgfqpoint{0.637500in}{0.550000in}}{\pgfqpoint{3.850000in}{3.850000in}}%
\pgfusepath{clip}%
\pgfsetbuttcap%
\pgfsetroundjoin%
\definecolor{currentfill}{rgb}{0.921990,0.596582,0.000000}%
\pgfsetfillcolor{currentfill}%
\pgfsetfillopacity{0.800000}%
\pgfsetlinewidth{0.000000pt}%
\definecolor{currentstroke}{rgb}{0.000000,0.000000,0.000000}%
\pgfsetstrokecolor{currentstroke}%
\pgfsetdash{}{0pt}%
\pgfpathmoveto{\pgfqpoint{2.243110in}{2.964404in}}%
\pgfpathlineto{\pgfqpoint{2.173191in}{2.985637in}}%
\pgfpathlineto{\pgfqpoint{2.243110in}{2.962205in}}%
\pgfpathlineto{\pgfqpoint{2.313028in}{2.945970in}}%
\pgfpathlineto{\pgfqpoint{2.243110in}{2.964404in}}%
\pgfpathclose%
\pgfusepath{fill}%
\end{pgfscope}%
\begin{pgfscope}%
\pgfpathrectangle{\pgfqpoint{0.637500in}{0.550000in}}{\pgfqpoint{3.850000in}{3.850000in}}%
\pgfusepath{clip}%
\pgfsetbuttcap%
\pgfsetroundjoin%
\definecolor{currentfill}{rgb}{0.907506,0.587210,0.000000}%
\pgfsetfillcolor{currentfill}%
\pgfsetfillopacity{0.800000}%
\pgfsetlinewidth{0.000000pt}%
\definecolor{currentstroke}{rgb}{0.000000,0.000000,0.000000}%
\pgfsetstrokecolor{currentstroke}%
\pgfsetdash{}{0pt}%
\pgfpathmoveto{\pgfqpoint{3.012215in}{3.090812in}}%
\pgfpathlineto{\pgfqpoint{2.942296in}{3.056372in}}%
\pgfpathlineto{\pgfqpoint{3.012215in}{3.088613in}}%
\pgfpathlineto{\pgfqpoint{3.082134in}{3.124824in}}%
\pgfpathlineto{\pgfqpoint{3.012215in}{3.090812in}}%
\pgfpathclose%
\pgfusepath{fill}%
\end{pgfscope}%
\begin{pgfscope}%
\pgfpathrectangle{\pgfqpoint{0.637500in}{0.550000in}}{\pgfqpoint{3.850000in}{3.850000in}}%
\pgfusepath{clip}%
\pgfsetbuttcap%
\pgfsetroundjoin%
\definecolor{currentfill}{rgb}{0.898377,0.581303,0.000000}%
\pgfsetfillcolor{currentfill}%
\pgfsetfillopacity{0.800000}%
\pgfsetlinewidth{0.000000pt}%
\definecolor{currentstroke}{rgb}{0.000000,0.000000,0.000000}%
\pgfsetstrokecolor{currentstroke}%
\pgfsetdash{}{0pt}%
\pgfpathmoveto{\pgfqpoint{1.474005in}{3.340569in}}%
\pgfpathlineto{\pgfqpoint{1.404086in}{3.378891in}}%
\pgfpathlineto{\pgfqpoint{1.474005in}{3.338370in}}%
\pgfpathlineto{\pgfqpoint{1.543923in}{3.300274in}}%
\pgfpathlineto{\pgfqpoint{1.474005in}{3.340569in}}%
\pgfpathclose%
\pgfusepath{fill}%
\end{pgfscope}%
\begin{pgfscope}%
\pgfpathrectangle{\pgfqpoint{0.637500in}{0.550000in}}{\pgfqpoint{3.850000in}{3.850000in}}%
\pgfusepath{clip}%
\pgfsetbuttcap%
\pgfsetroundjoin%
\definecolor{currentfill}{rgb}{0.899869,0.582268,0.000000}%
\pgfsetfillcolor{currentfill}%
\pgfsetfillopacity{0.800000}%
\pgfsetlinewidth{0.000000pt}%
\definecolor{currentstroke}{rgb}{0.000000,0.000000,0.000000}%
\pgfsetstrokecolor{currentstroke}%
\pgfsetdash{}{0pt}%
\pgfpathmoveto{\pgfqpoint{1.683761in}{3.222941in}}%
\pgfpathlineto{\pgfqpoint{1.613842in}{3.260292in}}%
\pgfpathlineto{\pgfqpoint{1.683761in}{3.220743in}}%
\pgfpathlineto{\pgfqpoint{1.753679in}{3.183988in}}%
\pgfpathlineto{\pgfqpoint{1.683761in}{3.222941in}}%
\pgfpathclose%
\pgfusepath{fill}%
\end{pgfscope}%
\begin{pgfscope}%
\pgfpathrectangle{\pgfqpoint{0.637500in}{0.550000in}}{\pgfqpoint{3.850000in}{3.850000in}}%
\pgfusepath{clip}%
\pgfsetbuttcap%
\pgfsetroundjoin%
\definecolor{currentfill}{rgb}{0.903689,0.584740,0.000000}%
\pgfsetfillcolor{currentfill}%
\pgfsetfillopacity{0.800000}%
\pgfsetlinewidth{0.000000pt}%
\definecolor{currentstroke}{rgb}{0.000000,0.000000,0.000000}%
\pgfsetstrokecolor{currentstroke}%
\pgfsetdash{}{0pt}%
\pgfpathmoveto{\pgfqpoint{1.893516in}{3.111022in}}%
\pgfpathlineto{\pgfqpoint{1.823598in}{3.145850in}}%
\pgfpathlineto{\pgfqpoint{1.893516in}{3.108823in}}%
\pgfpathlineto{\pgfqpoint{1.963435in}{3.075493in}}%
\pgfpathlineto{\pgfqpoint{1.893516in}{3.111022in}}%
\pgfpathclose%
\pgfusepath{fill}%
\end{pgfscope}%
\begin{pgfscope}%
\pgfpathrectangle{\pgfqpoint{0.637500in}{0.550000in}}{\pgfqpoint{3.850000in}{3.850000in}}%
\pgfusepath{clip}%
\pgfsetbuttcap%
\pgfsetroundjoin%
\definecolor{currentfill}{rgb}{0.919814,0.595174,0.000000}%
\pgfsetfillcolor{currentfill}%
\pgfsetfillopacity{0.800000}%
\pgfsetlinewidth{0.000000pt}%
\definecolor{currentstroke}{rgb}{0.000000,0.000000,0.000000}%
\pgfsetstrokecolor{currentstroke}%
\pgfsetdash{}{0pt}%
\pgfpathmoveto{\pgfqpoint{2.802459in}{2.997383in}}%
\pgfpathlineto{\pgfqpoint{2.732540in}{2.972105in}}%
\pgfpathlineto{\pgfqpoint{2.802459in}{2.995184in}}%
\pgfpathlineto{\pgfqpoint{2.872378in}{3.024268in}}%
\pgfpathlineto{\pgfqpoint{2.802459in}{2.997383in}}%
\pgfpathclose%
\pgfusepath{fill}%
\end{pgfscope}%
\begin{pgfscope}%
\pgfpathrectangle{\pgfqpoint{0.637500in}{0.550000in}}{\pgfqpoint{3.850000in}{3.850000in}}%
\pgfusepath{clip}%
\pgfsetbuttcap%
\pgfsetroundjoin%
\definecolor{currentfill}{rgb}{0.912477,0.590427,0.000000}%
\pgfsetfillcolor{currentfill}%
\pgfsetfillopacity{0.800000}%
\pgfsetlinewidth{0.000000pt}%
\definecolor{currentstroke}{rgb}{0.000000,0.000000,0.000000}%
\pgfsetstrokecolor{currentstroke}%
\pgfsetdash{}{0pt}%
\pgfpathmoveto{\pgfqpoint{2.103272in}{3.013227in}}%
\pgfpathlineto{\pgfqpoint{2.033354in}{3.041956in}}%
\pgfpathlineto{\pgfqpoint{2.103272in}{3.011028in}}%
\pgfpathlineto{\pgfqpoint{2.173191in}{2.985637in}}%
\pgfpathlineto{\pgfqpoint{2.103272in}{3.013227in}}%
\pgfpathclose%
\pgfusepath{fill}%
\end{pgfscope}%
\begin{pgfscope}%
\pgfpathrectangle{\pgfqpoint{0.637500in}{0.550000in}}{\pgfqpoint{3.850000in}{3.850000in}}%
\pgfusepath{clip}%
\pgfsetbuttcap%
\pgfsetroundjoin%
\definecolor{currentfill}{rgb}{0.898262,0.581229,0.000000}%
\pgfsetfillcolor{currentfill}%
\pgfsetfillopacity{0.800000}%
\pgfsetlinewidth{0.000000pt}%
\definecolor{currentstroke}{rgb}{0.000000,0.000000,0.000000}%
\pgfsetstrokecolor{currentstroke}%
\pgfsetdash{}{0pt}%
\pgfpathmoveto{\pgfqpoint{3.571564in}{3.396565in}}%
\pgfpathlineto{\pgfqpoint{3.501646in}{3.355970in}}%
\pgfpathlineto{\pgfqpoint{3.571564in}{3.394366in}}%
\pgfpathlineto{\pgfqpoint{3.641483in}{3.435103in}}%
\pgfpathlineto{\pgfqpoint{3.571564in}{3.396565in}}%
\pgfpathclose%
\pgfusepath{fill}%
\end{pgfscope}%
\begin{pgfscope}%
\pgfpathrectangle{\pgfqpoint{0.637500in}{0.550000in}}{\pgfqpoint{3.850000in}{3.850000in}}%
\pgfusepath{clip}%
\pgfsetbuttcap%
\pgfsetroundjoin%
\definecolor{currentfill}{rgb}{0.899565,0.582072,0.000000}%
\pgfsetfillcolor{currentfill}%
\pgfsetfillopacity{0.800000}%
\pgfsetlinewidth{0.000000pt}%
\definecolor{currentstroke}{rgb}{0.000000,0.000000,0.000000}%
\pgfsetstrokecolor{currentstroke}%
\pgfsetdash{}{0pt}%
\pgfpathmoveto{\pgfqpoint{3.361808in}{3.277644in}}%
\pgfpathlineto{\pgfqpoint{3.291890in}{3.237896in}}%
\pgfpathlineto{\pgfqpoint{3.361808in}{3.275445in}}%
\pgfpathlineto{\pgfqpoint{3.431727in}{3.315571in}}%
\pgfpathlineto{\pgfqpoint{3.361808in}{3.277644in}}%
\pgfpathclose%
\pgfusepath{fill}%
\end{pgfscope}%
\begin{pgfscope}%
\pgfpathrectangle{\pgfqpoint{0.637500in}{0.550000in}}{\pgfqpoint{3.850000in}{3.850000in}}%
\pgfusepath{clip}%
\pgfsetbuttcap%
\pgfsetroundjoin%
\definecolor{currentfill}{rgb}{0.902928,0.584248,0.000000}%
\pgfsetfillcolor{currentfill}%
\pgfsetfillopacity{0.800000}%
\pgfsetlinewidth{0.000000pt}%
\definecolor{currentstroke}{rgb}{0.000000,0.000000,0.000000}%
\pgfsetstrokecolor{currentstroke}%
\pgfsetdash{}{0pt}%
\pgfpathmoveto{\pgfqpoint{3.152052in}{3.162357in}}%
\pgfpathlineto{\pgfqpoint{3.082134in}{3.124824in}}%
\pgfpathlineto{\pgfqpoint{3.152052in}{3.160159in}}%
\pgfpathlineto{\pgfqpoint{3.221971in}{3.198670in}}%
\pgfpathlineto{\pgfqpoint{3.152052in}{3.162357in}}%
\pgfpathclose%
\pgfusepath{fill}%
\end{pgfscope}%
\begin{pgfscope}%
\pgfpathrectangle{\pgfqpoint{0.637500in}{0.550000in}}{\pgfqpoint{3.850000in}{3.850000in}}%
\pgfusepath{clip}%
\pgfsetbuttcap%
\pgfsetroundjoin%
\definecolor{currentfill}{rgb}{0.934964,0.604977,0.000000}%
\pgfsetfillcolor{currentfill}%
\pgfsetfillopacity{0.800000}%
\pgfsetlinewidth{0.000000pt}%
\definecolor{currentstroke}{rgb}{0.000000,0.000000,0.000000}%
\pgfsetstrokecolor{currentstroke}%
\pgfsetdash{}{0pt}%
\pgfpathmoveto{\pgfqpoint{2.452866in}{2.929174in}}%
\pgfpathlineto{\pgfqpoint{2.382947in}{2.933301in}}%
\pgfpathlineto{\pgfqpoint{2.452866in}{2.926975in}}%
\pgfpathlineto{\pgfqpoint{2.522784in}{2.929467in}}%
\pgfpathlineto{\pgfqpoint{2.452866in}{2.929174in}}%
\pgfpathclose%
\pgfusepath{fill}%
\end{pgfscope}%
\begin{pgfscope}%
\pgfpathrectangle{\pgfqpoint{0.637500in}{0.550000in}}{\pgfqpoint{3.850000in}{3.850000in}}%
\pgfusepath{clip}%
\pgfsetbuttcap%
\pgfsetroundjoin%
\definecolor{currentfill}{rgb}{0.910843,0.589369,0.000000}%
\pgfsetfillcolor{currentfill}%
\pgfsetfillopacity{0.800000}%
\pgfsetlinewidth{0.000000pt}%
\definecolor{currentstroke}{rgb}{0.000000,0.000000,0.000000}%
\pgfsetstrokecolor{currentstroke}%
\pgfsetdash{}{0pt}%
\pgfpathmoveto{\pgfqpoint{2.942296in}{3.056372in}}%
\pgfpathlineto{\pgfqpoint{2.872378in}{3.024268in}}%
\pgfpathlineto{\pgfqpoint{2.942296in}{3.054174in}}%
\pgfpathlineto{\pgfqpoint{3.012215in}{3.088613in}}%
\pgfpathlineto{\pgfqpoint{2.942296in}{3.056372in}}%
\pgfpathclose%
\pgfusepath{fill}%
\end{pgfscope}%
\begin{pgfscope}%
\pgfpathrectangle{\pgfqpoint{0.637500in}{0.550000in}}{\pgfqpoint{3.850000in}{3.850000in}}%
\pgfusepath{clip}%
\pgfsetbuttcap%
\pgfsetroundjoin%
\definecolor{currentfill}{rgb}{0.933916,0.604299,0.000000}%
\pgfsetfillcolor{currentfill}%
\pgfsetfillopacity{0.800000}%
\pgfsetlinewidth{0.000000pt}%
\definecolor{currentstroke}{rgb}{0.000000,0.000000,0.000000}%
\pgfsetstrokecolor{currentstroke}%
\pgfsetdash{}{0pt}%
\pgfpathmoveto{\pgfqpoint{2.592703in}{2.938505in}}%
\pgfpathlineto{\pgfqpoint{2.522784in}{2.929467in}}%
\pgfpathlineto{\pgfqpoint{2.592703in}{2.936306in}}%
\pgfpathlineto{\pgfqpoint{2.662622in}{2.951475in}}%
\pgfpathlineto{\pgfqpoint{2.592703in}{2.938505in}}%
\pgfpathclose%
\pgfusepath{fill}%
\end{pgfscope}%
\begin{pgfscope}%
\pgfpathrectangle{\pgfqpoint{0.637500in}{0.550000in}}{\pgfqpoint{3.850000in}{3.850000in}}%
\pgfusepath{clip}%
\pgfsetbuttcap%
\pgfsetroundjoin%
\definecolor{currentfill}{rgb}{0.898725,0.581528,0.000000}%
\pgfsetfillcolor{currentfill}%
\pgfsetfillopacity{0.800000}%
\pgfsetlinewidth{0.000000pt}%
\definecolor{currentstroke}{rgb}{0.000000,0.000000,0.000000}%
\pgfsetstrokecolor{currentstroke}%
\pgfsetdash{}{0pt}%
\pgfpathmoveto{\pgfqpoint{1.543923in}{3.300274in}}%
\pgfpathlineto{\pgfqpoint{1.474005in}{3.338370in}}%
\pgfpathlineto{\pgfqpoint{1.543923in}{3.298075in}}%
\pgfpathlineto{\pgfqpoint{1.613842in}{3.260292in}}%
\pgfpathlineto{\pgfqpoint{1.543923in}{3.300274in}}%
\pgfpathclose%
\pgfusepath{fill}%
\end{pgfscope}%
\begin{pgfscope}%
\pgfpathrectangle{\pgfqpoint{0.637500in}{0.550000in}}{\pgfqpoint{3.850000in}{3.850000in}}%
\pgfusepath{clip}%
\pgfsetbuttcap%
\pgfsetroundjoin%
\definecolor{currentfill}{rgb}{0.900779,0.582857,0.000000}%
\pgfsetfillcolor{currentfill}%
\pgfsetfillopacity{0.800000}%
\pgfsetlinewidth{0.000000pt}%
\definecolor{currentstroke}{rgb}{0.000000,0.000000,0.000000}%
\pgfsetstrokecolor{currentstroke}%
\pgfsetdash{}{0pt}%
\pgfpathmoveto{\pgfqpoint{1.753679in}{3.183988in}}%
\pgfpathlineto{\pgfqpoint{1.683761in}{3.220743in}}%
\pgfpathlineto{\pgfqpoint{1.753679in}{3.181789in}}%
\pgfpathlineto{\pgfqpoint{1.823598in}{3.145850in}}%
\pgfpathlineto{\pgfqpoint{1.753679in}{3.183988in}}%
\pgfpathclose%
\pgfusepath{fill}%
\end{pgfscope}%
\begin{pgfscope}%
\pgfpathrectangle{\pgfqpoint{0.637500in}{0.550000in}}{\pgfqpoint{3.850000in}{3.850000in}}%
\pgfusepath{clip}%
\pgfsetbuttcap%
\pgfsetroundjoin%
\definecolor{currentfill}{rgb}{0.927227,0.599970,0.000000}%
\pgfsetfillcolor{currentfill}%
\pgfsetfillopacity{0.800000}%
\pgfsetlinewidth{0.000000pt}%
\definecolor{currentstroke}{rgb}{0.000000,0.000000,0.000000}%
\pgfsetstrokecolor{currentstroke}%
\pgfsetdash{}{0pt}%
\pgfpathmoveto{\pgfqpoint{2.313028in}{2.945970in}}%
\pgfpathlineto{\pgfqpoint{2.243110in}{2.962205in}}%
\pgfpathlineto{\pgfqpoint{2.313028in}{2.943771in}}%
\pgfpathlineto{\pgfqpoint{2.382947in}{2.933301in}}%
\pgfpathlineto{\pgfqpoint{2.313028in}{2.945970in}}%
\pgfpathclose%
\pgfusepath{fill}%
\end{pgfscope}%
\begin{pgfscope}%
\pgfpathrectangle{\pgfqpoint{0.637500in}{0.550000in}}{\pgfqpoint{3.850000in}{3.850000in}}%
\pgfusepath{clip}%
\pgfsetbuttcap%
\pgfsetroundjoin%
\definecolor{currentfill}{rgb}{0.905912,0.586178,0.000000}%
\pgfsetfillcolor{currentfill}%
\pgfsetfillopacity{0.800000}%
\pgfsetlinewidth{0.000000pt}%
\definecolor{currentstroke}{rgb}{0.000000,0.000000,0.000000}%
\pgfsetstrokecolor{currentstroke}%
\pgfsetdash{}{0pt}%
\pgfpathmoveto{\pgfqpoint{1.963435in}{3.075493in}}%
\pgfpathlineto{\pgfqpoint{1.893516in}{3.108823in}}%
\pgfpathlineto{\pgfqpoint{1.963435in}{3.073294in}}%
\pgfpathlineto{\pgfqpoint{2.033354in}{3.041956in}}%
\pgfpathlineto{\pgfqpoint{1.963435in}{3.075493in}}%
\pgfpathclose%
\pgfusepath{fill}%
\end{pgfscope}%
\begin{pgfscope}%
\pgfpathrectangle{\pgfqpoint{0.637500in}{0.550000in}}{\pgfqpoint{3.850000in}{3.850000in}}%
\pgfusepath{clip}%
\pgfsetbuttcap%
\pgfsetroundjoin%
\definecolor{currentfill}{rgb}{0.925053,0.598564,0.000000}%
\pgfsetfillcolor{currentfill}%
\pgfsetfillopacity{0.800000}%
\pgfsetlinewidth{0.000000pt}%
\definecolor{currentstroke}{rgb}{0.000000,0.000000,0.000000}%
\pgfsetstrokecolor{currentstroke}%
\pgfsetdash{}{0pt}%
\pgfpathmoveto{\pgfqpoint{2.732540in}{2.972105in}}%
\pgfpathlineto{\pgfqpoint{2.662622in}{2.951475in}}%
\pgfpathlineto{\pgfqpoint{2.732540in}{2.969906in}}%
\pgfpathlineto{\pgfqpoint{2.802459in}{2.995184in}}%
\pgfpathlineto{\pgfqpoint{2.732540in}{2.972105in}}%
\pgfpathclose%
\pgfusepath{fill}%
\end{pgfscope}%
\begin{pgfscope}%
\pgfpathrectangle{\pgfqpoint{0.637500in}{0.550000in}}{\pgfqpoint{3.850000in}{3.850000in}}%
\pgfusepath{clip}%
\pgfsetbuttcap%
\pgfsetroundjoin%
\definecolor{currentfill}{rgb}{0.916926,0.593305,0.000000}%
\pgfsetfillcolor{currentfill}%
\pgfsetfillopacity{0.800000}%
\pgfsetlinewidth{0.000000pt}%
\definecolor{currentstroke}{rgb}{0.000000,0.000000,0.000000}%
\pgfsetstrokecolor{currentstroke}%
\pgfsetdash{}{0pt}%
\pgfpathmoveto{\pgfqpoint{2.173191in}{2.985637in}}%
\pgfpathlineto{\pgfqpoint{2.103272in}{3.011028in}}%
\pgfpathlineto{\pgfqpoint{2.173191in}{2.983439in}}%
\pgfpathlineto{\pgfqpoint{2.243110in}{2.962205in}}%
\pgfpathlineto{\pgfqpoint{2.173191in}{2.985637in}}%
\pgfpathclose%
\pgfusepath{fill}%
\end{pgfscope}%
\begin{pgfscope}%
\pgfpathrectangle{\pgfqpoint{0.637500in}{0.550000in}}{\pgfqpoint{3.850000in}{3.850000in}}%
\pgfusepath{clip}%
\pgfsetbuttcap%
\pgfsetroundjoin%
\definecolor{currentfill}{rgb}{0.898565,0.581425,0.000000}%
\pgfsetfillcolor{currentfill}%
\pgfsetfillopacity{0.800000}%
\pgfsetlinewidth{0.000000pt}%
\definecolor{currentstroke}{rgb}{0.000000,0.000000,0.000000}%
\pgfsetstrokecolor{currentstroke}%
\pgfsetdash{}{0pt}%
\pgfpathmoveto{\pgfqpoint{3.501646in}{3.355970in}}%
\pgfpathlineto{\pgfqpoint{3.431727in}{3.315571in}}%
\pgfpathlineto{\pgfqpoint{3.501646in}{3.353771in}}%
\pgfpathlineto{\pgfqpoint{3.571564in}{3.394366in}}%
\pgfpathlineto{\pgfqpoint{3.501646in}{3.355970in}}%
\pgfpathclose%
\pgfusepath{fill}%
\end{pgfscope}%
\begin{pgfscope}%
\pgfpathrectangle{\pgfqpoint{0.637500in}{0.550000in}}{\pgfqpoint{3.850000in}{3.850000in}}%
\pgfusepath{clip}%
\pgfsetbuttcap%
\pgfsetroundjoin%
\definecolor{currentfill}{rgb}{0.900362,0.582587,0.000000}%
\pgfsetfillcolor{currentfill}%
\pgfsetfillopacity{0.800000}%
\pgfsetlinewidth{0.000000pt}%
\definecolor{currentstroke}{rgb}{0.000000,0.000000,0.000000}%
\pgfsetstrokecolor{currentstroke}%
\pgfsetdash{}{0pt}%
\pgfpathmoveto{\pgfqpoint{3.291890in}{3.237896in}}%
\pgfpathlineto{\pgfqpoint{3.221971in}{3.198670in}}%
\pgfpathlineto{\pgfqpoint{3.291890in}{3.235697in}}%
\pgfpathlineto{\pgfqpoint{3.361808in}{3.275445in}}%
\pgfpathlineto{\pgfqpoint{3.291890in}{3.237896in}}%
\pgfpathclose%
\pgfusepath{fill}%
\end{pgfscope}%
\begin{pgfscope}%
\pgfpathrectangle{\pgfqpoint{0.637500in}{0.550000in}}{\pgfqpoint{3.850000in}{3.850000in}}%
\pgfusepath{clip}%
\pgfsetbuttcap%
\pgfsetroundjoin%
\definecolor{currentfill}{rgb}{0.904905,0.585527,0.000000}%
\pgfsetfillcolor{currentfill}%
\pgfsetfillopacity{0.800000}%
\pgfsetlinewidth{0.000000pt}%
\definecolor{currentstroke}{rgb}{0.000000,0.000000,0.000000}%
\pgfsetstrokecolor{currentstroke}%
\pgfsetdash{}{0pt}%
\pgfpathmoveto{\pgfqpoint{3.082134in}{3.124824in}}%
\pgfpathlineto{\pgfqpoint{3.012215in}{3.088613in}}%
\pgfpathlineto{\pgfqpoint{3.082134in}{3.122625in}}%
\pgfpathlineto{\pgfqpoint{3.152052in}{3.160159in}}%
\pgfpathlineto{\pgfqpoint{3.082134in}{3.124824in}}%
\pgfpathclose%
\pgfusepath{fill}%
\end{pgfscope}%
\begin{pgfscope}%
\pgfpathrectangle{\pgfqpoint{0.637500in}{0.550000in}}{\pgfqpoint{3.850000in}{3.850000in}}%
\pgfusepath{clip}%
\pgfsetbuttcap%
\pgfsetroundjoin%
\definecolor{currentfill}{rgb}{0.898127,0.581141,0.000000}%
\pgfsetfillcolor{currentfill}%
\pgfsetfillopacity{0.800000}%
\pgfsetlinewidth{0.000000pt}%
\definecolor{currentstroke}{rgb}{0.000000,0.000000,0.000000}%
\pgfsetstrokecolor{currentstroke}%
\pgfsetdash{}{0pt}%
\pgfpathmoveto{\pgfqpoint{1.404086in}{3.378891in}}%
\pgfpathlineto{\pgfqpoint{1.334167in}{3.417375in}}%
\pgfpathlineto{\pgfqpoint{1.404086in}{3.376692in}}%
\pgfpathlineto{\pgfqpoint{1.474005in}{3.338370in}}%
\pgfpathlineto{\pgfqpoint{1.404086in}{3.378891in}}%
\pgfpathclose%
\pgfusepath{fill}%
\end{pgfscope}%
\begin{pgfscope}%
\pgfpathrectangle{\pgfqpoint{0.637500in}{0.550000in}}{\pgfqpoint{3.850000in}{3.850000in}}%
\pgfusepath{clip}%
\pgfsetbuttcap%
\pgfsetroundjoin%
\definecolor{currentfill}{rgb}{0.899206,0.581839,0.000000}%
\pgfsetfillcolor{currentfill}%
\pgfsetfillopacity{0.800000}%
\pgfsetlinewidth{0.000000pt}%
\definecolor{currentstroke}{rgb}{0.000000,0.000000,0.000000}%
\pgfsetstrokecolor{currentstroke}%
\pgfsetdash{}{0pt}%
\pgfpathmoveto{\pgfqpoint{1.613842in}{3.260292in}}%
\pgfpathlineto{\pgfqpoint{1.543923in}{3.298075in}}%
\pgfpathlineto{\pgfqpoint{1.613842in}{3.258093in}}%
\pgfpathlineto{\pgfqpoint{1.683761in}{3.220743in}}%
\pgfpathlineto{\pgfqpoint{1.613842in}{3.260292in}}%
\pgfpathclose%
\pgfusepath{fill}%
\end{pgfscope}%
\begin{pgfscope}%
\pgfpathrectangle{\pgfqpoint{0.637500in}{0.550000in}}{\pgfqpoint{3.850000in}{3.850000in}}%
\pgfusepath{clip}%
\pgfsetbuttcap%
\pgfsetroundjoin%
\definecolor{currentfill}{rgb}{0.914972,0.592040,0.000000}%
\pgfsetfillcolor{currentfill}%
\pgfsetfillopacity{0.800000}%
\pgfsetlinewidth{0.000000pt}%
\definecolor{currentstroke}{rgb}{0.000000,0.000000,0.000000}%
\pgfsetstrokecolor{currentstroke}%
\pgfsetdash{}{0pt}%
\pgfpathmoveto{\pgfqpoint{2.872378in}{3.024268in}}%
\pgfpathlineto{\pgfqpoint{2.802459in}{2.995184in}}%
\pgfpathlineto{\pgfqpoint{2.872378in}{3.022069in}}%
\pgfpathlineto{\pgfqpoint{2.942296in}{3.054174in}}%
\pgfpathlineto{\pgfqpoint{2.872378in}{3.024268in}}%
\pgfpathclose%
\pgfusepath{fill}%
\end{pgfscope}%
\begin{pgfscope}%
\pgfpathrectangle{\pgfqpoint{0.637500in}{0.550000in}}{\pgfqpoint{3.850000in}{3.850000in}}%
\pgfusepath{clip}%
\pgfsetbuttcap%
\pgfsetroundjoin%
\definecolor{currentfill}{rgb}{0.902018,0.583658,0.000000}%
\pgfsetfillcolor{currentfill}%
\pgfsetfillopacity{0.800000}%
\pgfsetlinewidth{0.000000pt}%
\definecolor{currentstroke}{rgb}{0.000000,0.000000,0.000000}%
\pgfsetstrokecolor{currentstroke}%
\pgfsetdash{}{0pt}%
\pgfpathmoveto{\pgfqpoint{1.823598in}{3.145850in}}%
\pgfpathlineto{\pgfqpoint{1.753679in}{3.181789in}}%
\pgfpathlineto{\pgfqpoint{1.823598in}{3.143651in}}%
\pgfpathlineto{\pgfqpoint{1.893516in}{3.108823in}}%
\pgfpathlineto{\pgfqpoint{1.823598in}{3.145850in}}%
\pgfpathclose%
\pgfusepath{fill}%
\end{pgfscope}%
\begin{pgfscope}%
\pgfpathrectangle{\pgfqpoint{0.637500in}{0.550000in}}{\pgfqpoint{3.850000in}{3.850000in}}%
\pgfusepath{clip}%
\pgfsetbuttcap%
\pgfsetroundjoin%
\definecolor{currentfill}{rgb}{0.908810,0.588053,0.000000}%
\pgfsetfillcolor{currentfill}%
\pgfsetfillopacity{0.800000}%
\pgfsetlinewidth{0.000000pt}%
\definecolor{currentstroke}{rgb}{0.000000,0.000000,0.000000}%
\pgfsetstrokecolor{currentstroke}%
\pgfsetdash{}{0pt}%
\pgfpathmoveto{\pgfqpoint{2.033354in}{3.041956in}}%
\pgfpathlineto{\pgfqpoint{1.963435in}{3.073294in}}%
\pgfpathlineto{\pgfqpoint{2.033354in}{3.039758in}}%
\pgfpathlineto{\pgfqpoint{2.103272in}{3.011028in}}%
\pgfpathlineto{\pgfqpoint{2.033354in}{3.041956in}}%
\pgfpathclose%
\pgfusepath{fill}%
\end{pgfscope}%
\begin{pgfscope}%
\pgfpathrectangle{\pgfqpoint{0.637500in}{0.550000in}}{\pgfqpoint{3.850000in}{3.850000in}}%
\pgfusepath{clip}%
\pgfsetbuttcap%
\pgfsetroundjoin%
\definecolor{currentfill}{rgb}{0.935716,0.605463,0.000000}%
\pgfsetfillcolor{currentfill}%
\pgfsetfillopacity{0.800000}%
\pgfsetlinewidth{0.000000pt}%
\definecolor{currentstroke}{rgb}{0.000000,0.000000,0.000000}%
\pgfsetstrokecolor{currentstroke}%
\pgfsetdash{}{0pt}%
\pgfpathmoveto{\pgfqpoint{2.522784in}{2.929467in}}%
\pgfpathlineto{\pgfqpoint{2.452866in}{2.926975in}}%
\pgfpathlineto{\pgfqpoint{2.522784in}{2.927269in}}%
\pgfpathlineto{\pgfqpoint{2.592703in}{2.936306in}}%
\pgfpathlineto{\pgfqpoint{2.522784in}{2.929467in}}%
\pgfpathclose%
\pgfusepath{fill}%
\end{pgfscope}%
\begin{pgfscope}%
\pgfpathrectangle{\pgfqpoint{0.637500in}{0.550000in}}{\pgfqpoint{3.850000in}{3.850000in}}%
\pgfusepath{clip}%
\pgfsetbuttcap%
\pgfsetroundjoin%
\definecolor{currentfill}{rgb}{0.931871,0.602975,0.000000}%
\pgfsetfillcolor{currentfill}%
\pgfsetfillopacity{0.800000}%
\pgfsetlinewidth{0.000000pt}%
\definecolor{currentstroke}{rgb}{0.000000,0.000000,0.000000}%
\pgfsetstrokecolor{currentstroke}%
\pgfsetdash{}{0pt}%
\pgfpathmoveto{\pgfqpoint{2.382947in}{2.933301in}}%
\pgfpathlineto{\pgfqpoint{2.313028in}{2.943771in}}%
\pgfpathlineto{\pgfqpoint{2.382947in}{2.931102in}}%
\pgfpathlineto{\pgfqpoint{2.452866in}{2.926975in}}%
\pgfpathlineto{\pgfqpoint{2.382947in}{2.933301in}}%
\pgfpathclose%
\pgfusepath{fill}%
\end{pgfscope}%
\begin{pgfscope}%
\pgfpathrectangle{\pgfqpoint{0.637500in}{0.550000in}}{\pgfqpoint{3.850000in}{3.850000in}}%
\pgfusepath{clip}%
\pgfsetbuttcap%
\pgfsetroundjoin%
\definecolor{currentfill}{rgb}{0.930057,0.601801,0.000000}%
\pgfsetfillcolor{currentfill}%
\pgfsetfillopacity{0.800000}%
\pgfsetlinewidth{0.000000pt}%
\definecolor{currentstroke}{rgb}{0.000000,0.000000,0.000000}%
\pgfsetstrokecolor{currentstroke}%
\pgfsetdash{}{0pt}%
\pgfpathmoveto{\pgfqpoint{2.662622in}{2.951475in}}%
\pgfpathlineto{\pgfqpoint{2.592703in}{2.936306in}}%
\pgfpathlineto{\pgfqpoint{2.662622in}{2.949276in}}%
\pgfpathlineto{\pgfqpoint{2.732540in}{2.969906in}}%
\pgfpathlineto{\pgfqpoint{2.662622in}{2.951475in}}%
\pgfpathclose%
\pgfusepath{fill}%
\end{pgfscope}%
\begin{pgfscope}%
\pgfpathrectangle{\pgfqpoint{0.637500in}{0.550000in}}{\pgfqpoint{3.850000in}{3.850000in}}%
\pgfusepath{clip}%
\pgfsetbuttcap%
\pgfsetroundjoin%
\definecolor{currentfill}{rgb}{0.898985,0.581696,0.000000}%
\pgfsetfillcolor{currentfill}%
\pgfsetfillopacity{0.800000}%
\pgfsetlinewidth{0.000000pt}%
\definecolor{currentstroke}{rgb}{0.000000,0.000000,0.000000}%
\pgfsetstrokecolor{currentstroke}%
\pgfsetdash{}{0pt}%
\pgfpathmoveto{\pgfqpoint{3.431727in}{3.315571in}}%
\pgfpathlineto{\pgfqpoint{3.361808in}{3.275445in}}%
\pgfpathlineto{\pgfqpoint{3.431727in}{3.313372in}}%
\pgfpathlineto{\pgfqpoint{3.501646in}{3.353771in}}%
\pgfpathlineto{\pgfqpoint{3.431727in}{3.315571in}}%
\pgfpathclose%
\pgfusepath{fill}%
\end{pgfscope}%
\begin{pgfscope}%
\pgfpathrectangle{\pgfqpoint{0.637500in}{0.550000in}}{\pgfqpoint{3.850000in}{3.850000in}}%
\pgfusepath{clip}%
\pgfsetbuttcap%
\pgfsetroundjoin%
\definecolor{currentfill}{rgb}{0.898044,0.581087,0.000000}%
\pgfsetfillcolor{currentfill}%
\pgfsetfillopacity{0.800000}%
\pgfsetlinewidth{0.000000pt}%
\definecolor{currentstroke}{rgb}{0.000000,0.000000,0.000000}%
\pgfsetstrokecolor{currentstroke}%
\pgfsetdash{}{0pt}%
\pgfpathmoveto{\pgfqpoint{3.641483in}{3.435103in}}%
\pgfpathlineto{\pgfqpoint{3.571564in}{3.394366in}}%
\pgfpathlineto{\pgfqpoint{3.641483in}{3.432904in}}%
\pgfpathlineto{\pgfqpoint{3.711402in}{3.473743in}}%
\pgfpathlineto{\pgfqpoint{3.641483in}{3.435103in}}%
\pgfpathclose%
\pgfusepath{fill}%
\end{pgfscope}%
\begin{pgfscope}%
\pgfpathrectangle{\pgfqpoint{0.637500in}{0.550000in}}{\pgfqpoint{3.850000in}{3.850000in}}%
\pgfusepath{clip}%
\pgfsetbuttcap%
\pgfsetroundjoin%
\definecolor{currentfill}{rgb}{0.901452,0.583292,0.000000}%
\pgfsetfillcolor{currentfill}%
\pgfsetfillopacity{0.800000}%
\pgfsetlinewidth{0.000000pt}%
\definecolor{currentstroke}{rgb}{0.000000,0.000000,0.000000}%
\pgfsetstrokecolor{currentstroke}%
\pgfsetdash{}{0pt}%
\pgfpathmoveto{\pgfqpoint{3.221971in}{3.198670in}}%
\pgfpathlineto{\pgfqpoint{3.152052in}{3.160159in}}%
\pgfpathlineto{\pgfqpoint{3.221971in}{3.196471in}}%
\pgfpathlineto{\pgfqpoint{3.291890in}{3.235697in}}%
\pgfpathlineto{\pgfqpoint{3.221971in}{3.198670in}}%
\pgfpathclose%
\pgfusepath{fill}%
\end{pgfscope}%
\begin{pgfscope}%
\pgfpathrectangle{\pgfqpoint{0.637500in}{0.550000in}}{\pgfqpoint{3.850000in}{3.850000in}}%
\pgfusepath{clip}%
\pgfsetbuttcap%
\pgfsetroundjoin%
\definecolor{currentfill}{rgb}{0.921990,0.596582,0.000000}%
\pgfsetfillcolor{currentfill}%
\pgfsetfillopacity{0.800000}%
\pgfsetlinewidth{0.000000pt}%
\definecolor{currentstroke}{rgb}{0.000000,0.000000,0.000000}%
\pgfsetstrokecolor{currentstroke}%
\pgfsetdash{}{0pt}%
\pgfpathmoveto{\pgfqpoint{2.243110in}{2.962205in}}%
\pgfpathlineto{\pgfqpoint{2.173191in}{2.983439in}}%
\pgfpathlineto{\pgfqpoint{2.243110in}{2.960006in}}%
\pgfpathlineto{\pgfqpoint{2.313028in}{2.943771in}}%
\pgfpathlineto{\pgfqpoint{2.243110in}{2.962205in}}%
\pgfpathclose%
\pgfusepath{fill}%
\end{pgfscope}%
\begin{pgfscope}%
\pgfpathrectangle{\pgfqpoint{0.637500in}{0.550000in}}{\pgfqpoint{3.850000in}{3.850000in}}%
\pgfusepath{clip}%
\pgfsetbuttcap%
\pgfsetroundjoin%
\definecolor{currentfill}{rgb}{0.907506,0.587210,0.000000}%
\pgfsetfillcolor{currentfill}%
\pgfsetfillopacity{0.800000}%
\pgfsetlinewidth{0.000000pt}%
\definecolor{currentstroke}{rgb}{0.000000,0.000000,0.000000}%
\pgfsetstrokecolor{currentstroke}%
\pgfsetdash{}{0pt}%
\pgfpathmoveto{\pgfqpoint{3.012215in}{3.088613in}}%
\pgfpathlineto{\pgfqpoint{2.942296in}{3.054174in}}%
\pgfpathlineto{\pgfqpoint{3.012215in}{3.086414in}}%
\pgfpathlineto{\pgfqpoint{3.082134in}{3.122625in}}%
\pgfpathlineto{\pgfqpoint{3.012215in}{3.088613in}}%
\pgfpathclose%
\pgfusepath{fill}%
\end{pgfscope}%
\begin{pgfscope}%
\pgfpathrectangle{\pgfqpoint{0.637500in}{0.550000in}}{\pgfqpoint{3.850000in}{3.850000in}}%
\pgfusepath{clip}%
\pgfsetbuttcap%
\pgfsetroundjoin%
\definecolor{currentfill}{rgb}{0.898377,0.581303,0.000000}%
\pgfsetfillcolor{currentfill}%
\pgfsetfillopacity{0.800000}%
\pgfsetlinewidth{0.000000pt}%
\definecolor{currentstroke}{rgb}{0.000000,0.000000,0.000000}%
\pgfsetstrokecolor{currentstroke}%
\pgfsetdash{}{0pt}%
\pgfpathmoveto{\pgfqpoint{1.474005in}{3.338370in}}%
\pgfpathlineto{\pgfqpoint{1.404086in}{3.376692in}}%
\pgfpathlineto{\pgfqpoint{1.474005in}{3.336171in}}%
\pgfpathlineto{\pgfqpoint{1.543923in}{3.298075in}}%
\pgfpathlineto{\pgfqpoint{1.474005in}{3.338370in}}%
\pgfpathclose%
\pgfusepath{fill}%
\end{pgfscope}%
\begin{pgfscope}%
\pgfpathrectangle{\pgfqpoint{0.637500in}{0.550000in}}{\pgfqpoint{3.850000in}{3.850000in}}%
\pgfusepath{clip}%
\pgfsetbuttcap%
\pgfsetroundjoin%
\definecolor{currentfill}{rgb}{0.899869,0.582268,0.000000}%
\pgfsetfillcolor{currentfill}%
\pgfsetfillopacity{0.800000}%
\pgfsetlinewidth{0.000000pt}%
\definecolor{currentstroke}{rgb}{0.000000,0.000000,0.000000}%
\pgfsetstrokecolor{currentstroke}%
\pgfsetdash{}{0pt}%
\pgfpathmoveto{\pgfqpoint{1.683761in}{3.220743in}}%
\pgfpathlineto{\pgfqpoint{1.613842in}{3.258093in}}%
\pgfpathlineto{\pgfqpoint{1.683761in}{3.218544in}}%
\pgfpathlineto{\pgfqpoint{1.753679in}{3.181789in}}%
\pgfpathlineto{\pgfqpoint{1.683761in}{3.220743in}}%
\pgfpathclose%
\pgfusepath{fill}%
\end{pgfscope}%
\begin{pgfscope}%
\pgfpathrectangle{\pgfqpoint{0.637500in}{0.550000in}}{\pgfqpoint{3.850000in}{3.850000in}}%
\pgfusepath{clip}%
\pgfsetbuttcap%
\pgfsetroundjoin%
\definecolor{currentfill}{rgb}{0.903689,0.584740,0.000000}%
\pgfsetfillcolor{currentfill}%
\pgfsetfillopacity{0.800000}%
\pgfsetlinewidth{0.000000pt}%
\definecolor{currentstroke}{rgb}{0.000000,0.000000,0.000000}%
\pgfsetstrokecolor{currentstroke}%
\pgfsetdash{}{0pt}%
\pgfpathmoveto{\pgfqpoint{1.893516in}{3.108823in}}%
\pgfpathlineto{\pgfqpoint{1.823598in}{3.143651in}}%
\pgfpathlineto{\pgfqpoint{1.893516in}{3.106625in}}%
\pgfpathlineto{\pgfqpoint{1.963435in}{3.073294in}}%
\pgfpathlineto{\pgfqpoint{1.893516in}{3.108823in}}%
\pgfpathclose%
\pgfusepath{fill}%
\end{pgfscope}%
\begin{pgfscope}%
\pgfpathrectangle{\pgfqpoint{0.637500in}{0.550000in}}{\pgfqpoint{3.850000in}{3.850000in}}%
\pgfusepath{clip}%
\pgfsetbuttcap%
\pgfsetroundjoin%
\definecolor{currentfill}{rgb}{0.919814,0.595174,0.000000}%
\pgfsetfillcolor{currentfill}%
\pgfsetfillopacity{0.800000}%
\pgfsetlinewidth{0.000000pt}%
\definecolor{currentstroke}{rgb}{0.000000,0.000000,0.000000}%
\pgfsetstrokecolor{currentstroke}%
\pgfsetdash{}{0pt}%
\pgfpathmoveto{\pgfqpoint{2.802459in}{2.995184in}}%
\pgfpathlineto{\pgfqpoint{2.732540in}{2.969906in}}%
\pgfpathlineto{\pgfqpoint{2.802459in}{2.992985in}}%
\pgfpathlineto{\pgfqpoint{2.872378in}{3.022069in}}%
\pgfpathlineto{\pgfqpoint{2.802459in}{2.995184in}}%
\pgfpathclose%
\pgfusepath{fill}%
\end{pgfscope}%
\begin{pgfscope}%
\pgfpathrectangle{\pgfqpoint{0.637500in}{0.550000in}}{\pgfqpoint{3.850000in}{3.850000in}}%
\pgfusepath{clip}%
\pgfsetbuttcap%
\pgfsetroundjoin%
\definecolor{currentfill}{rgb}{0.912477,0.590427,0.000000}%
\pgfsetfillcolor{currentfill}%
\pgfsetfillopacity{0.800000}%
\pgfsetlinewidth{0.000000pt}%
\definecolor{currentstroke}{rgb}{0.000000,0.000000,0.000000}%
\pgfsetstrokecolor{currentstroke}%
\pgfsetdash{}{0pt}%
\pgfpathmoveto{\pgfqpoint{2.103272in}{3.011028in}}%
\pgfpathlineto{\pgfqpoint{2.033354in}{3.039758in}}%
\pgfpathlineto{\pgfqpoint{2.103272in}{3.008829in}}%
\pgfpathlineto{\pgfqpoint{2.173191in}{2.983439in}}%
\pgfpathlineto{\pgfqpoint{2.103272in}{3.011028in}}%
\pgfpathclose%
\pgfusepath{fill}%
\end{pgfscope}%
\begin{pgfscope}%
\pgfpathrectangle{\pgfqpoint{0.637500in}{0.550000in}}{\pgfqpoint{3.850000in}{3.850000in}}%
\pgfusepath{clip}%
\pgfsetbuttcap%
\pgfsetroundjoin%
\definecolor{currentfill}{rgb}{0.898262,0.581229,0.000000}%
\pgfsetfillcolor{currentfill}%
\pgfsetfillopacity{0.800000}%
\pgfsetlinewidth{0.000000pt}%
\definecolor{currentstroke}{rgb}{0.000000,0.000000,0.000000}%
\pgfsetstrokecolor{currentstroke}%
\pgfsetdash{}{0pt}%
\pgfpathmoveto{\pgfqpoint{3.571564in}{3.394366in}}%
\pgfpathlineto{\pgfqpoint{3.501646in}{3.353771in}}%
\pgfpathlineto{\pgfqpoint{3.571564in}{3.392167in}}%
\pgfpathlineto{\pgfqpoint{3.641483in}{3.432904in}}%
\pgfpathlineto{\pgfqpoint{3.571564in}{3.394366in}}%
\pgfpathclose%
\pgfusepath{fill}%
\end{pgfscope}%
\begin{pgfscope}%
\pgfpathrectangle{\pgfqpoint{0.637500in}{0.550000in}}{\pgfqpoint{3.850000in}{3.850000in}}%
\pgfusepath{clip}%
\pgfsetbuttcap%
\pgfsetroundjoin%
\definecolor{currentfill}{rgb}{0.899565,0.582072,0.000000}%
\pgfsetfillcolor{currentfill}%
\pgfsetfillopacity{0.800000}%
\pgfsetlinewidth{0.000000pt}%
\definecolor{currentstroke}{rgb}{0.000000,0.000000,0.000000}%
\pgfsetstrokecolor{currentstroke}%
\pgfsetdash{}{0pt}%
\pgfpathmoveto{\pgfqpoint{3.361808in}{3.275445in}}%
\pgfpathlineto{\pgfqpoint{3.291890in}{3.235697in}}%
\pgfpathlineto{\pgfqpoint{3.361808in}{3.273247in}}%
\pgfpathlineto{\pgfqpoint{3.431727in}{3.313372in}}%
\pgfpathlineto{\pgfqpoint{3.361808in}{3.275445in}}%
\pgfpathclose%
\pgfusepath{fill}%
\end{pgfscope}%
\begin{pgfscope}%
\pgfpathrectangle{\pgfqpoint{0.637500in}{0.550000in}}{\pgfqpoint{3.850000in}{3.850000in}}%
\pgfusepath{clip}%
\pgfsetbuttcap%
\pgfsetroundjoin%
\definecolor{currentfill}{rgb}{0.902928,0.584248,0.000000}%
\pgfsetfillcolor{currentfill}%
\pgfsetfillopacity{0.800000}%
\pgfsetlinewidth{0.000000pt}%
\definecolor{currentstroke}{rgb}{0.000000,0.000000,0.000000}%
\pgfsetstrokecolor{currentstroke}%
\pgfsetdash{}{0pt}%
\pgfpathmoveto{\pgfqpoint{3.152052in}{3.160159in}}%
\pgfpathlineto{\pgfqpoint{3.082134in}{3.122625in}}%
\pgfpathlineto{\pgfqpoint{3.152052in}{3.157960in}}%
\pgfpathlineto{\pgfqpoint{3.221971in}{3.196471in}}%
\pgfpathlineto{\pgfqpoint{3.152052in}{3.160159in}}%
\pgfpathclose%
\pgfusepath{fill}%
\end{pgfscope}%
\begin{pgfscope}%
\pgfpathrectangle{\pgfqpoint{0.637500in}{0.550000in}}{\pgfqpoint{3.850000in}{3.850000in}}%
\pgfusepath{clip}%
\pgfsetbuttcap%
\pgfsetroundjoin%
\definecolor{currentfill}{rgb}{0.934964,0.604977,0.000000}%
\pgfsetfillcolor{currentfill}%
\pgfsetfillopacity{0.800000}%
\pgfsetlinewidth{0.000000pt}%
\definecolor{currentstroke}{rgb}{0.000000,0.000000,0.000000}%
\pgfsetstrokecolor{currentstroke}%
\pgfsetdash{}{0pt}%
\pgfpathmoveto{\pgfqpoint{2.452866in}{2.926975in}}%
\pgfpathlineto{\pgfqpoint{2.382947in}{2.931102in}}%
\pgfpathlineto{\pgfqpoint{2.452866in}{2.924776in}}%
\pgfpathlineto{\pgfqpoint{2.522784in}{2.927269in}}%
\pgfpathlineto{\pgfqpoint{2.452866in}{2.926975in}}%
\pgfpathclose%
\pgfusepath{fill}%
\end{pgfscope}%
\begin{pgfscope}%
\pgfpathrectangle{\pgfqpoint{0.637500in}{0.550000in}}{\pgfqpoint{3.850000in}{3.850000in}}%
\pgfusepath{clip}%
\pgfsetbuttcap%
\pgfsetroundjoin%
\definecolor{currentfill}{rgb}{0.910843,0.589369,0.000000}%
\pgfsetfillcolor{currentfill}%
\pgfsetfillopacity{0.800000}%
\pgfsetlinewidth{0.000000pt}%
\definecolor{currentstroke}{rgb}{0.000000,0.000000,0.000000}%
\pgfsetstrokecolor{currentstroke}%
\pgfsetdash{}{0pt}%
\pgfpathmoveto{\pgfqpoint{2.942296in}{3.054174in}}%
\pgfpathlineto{\pgfqpoint{2.872378in}{3.022069in}}%
\pgfpathlineto{\pgfqpoint{2.942296in}{3.051975in}}%
\pgfpathlineto{\pgfqpoint{3.012215in}{3.086414in}}%
\pgfpathlineto{\pgfqpoint{2.942296in}{3.054174in}}%
\pgfpathclose%
\pgfusepath{fill}%
\end{pgfscope}%
\begin{pgfscope}%
\pgfpathrectangle{\pgfqpoint{0.637500in}{0.550000in}}{\pgfqpoint{3.850000in}{3.850000in}}%
\pgfusepath{clip}%
\pgfsetbuttcap%
\pgfsetroundjoin%
\definecolor{currentfill}{rgb}{0.933916,0.604299,0.000000}%
\pgfsetfillcolor{currentfill}%
\pgfsetfillopacity{0.800000}%
\pgfsetlinewidth{0.000000pt}%
\definecolor{currentstroke}{rgb}{0.000000,0.000000,0.000000}%
\pgfsetstrokecolor{currentstroke}%
\pgfsetdash{}{0pt}%
\pgfpathmoveto{\pgfqpoint{2.592703in}{2.936306in}}%
\pgfpathlineto{\pgfqpoint{2.522784in}{2.927269in}}%
\pgfpathlineto{\pgfqpoint{2.592703in}{2.934108in}}%
\pgfpathlineto{\pgfqpoint{2.662622in}{2.949276in}}%
\pgfpathlineto{\pgfqpoint{2.592703in}{2.936306in}}%
\pgfpathclose%
\pgfusepath{fill}%
\end{pgfscope}%
\begin{pgfscope}%
\pgfpathrectangle{\pgfqpoint{0.637500in}{0.550000in}}{\pgfqpoint{3.850000in}{3.850000in}}%
\pgfusepath{clip}%
\pgfsetbuttcap%
\pgfsetroundjoin%
\definecolor{currentfill}{rgb}{0.898725,0.581528,0.000000}%
\pgfsetfillcolor{currentfill}%
\pgfsetfillopacity{0.800000}%
\pgfsetlinewidth{0.000000pt}%
\definecolor{currentstroke}{rgb}{0.000000,0.000000,0.000000}%
\pgfsetstrokecolor{currentstroke}%
\pgfsetdash{}{0pt}%
\pgfpathmoveto{\pgfqpoint{1.543923in}{3.298075in}}%
\pgfpathlineto{\pgfqpoint{1.474005in}{3.336171in}}%
\pgfpathlineto{\pgfqpoint{1.543923in}{3.295877in}}%
\pgfpathlineto{\pgfqpoint{1.613842in}{3.258093in}}%
\pgfpathlineto{\pgfqpoint{1.543923in}{3.298075in}}%
\pgfpathclose%
\pgfusepath{fill}%
\end{pgfscope}%
\begin{pgfscope}%
\pgfpathrectangle{\pgfqpoint{0.637500in}{0.550000in}}{\pgfqpoint{3.850000in}{3.850000in}}%
\pgfusepath{clip}%
\pgfsetbuttcap%
\pgfsetroundjoin%
\definecolor{currentfill}{rgb}{0.897946,0.581024,0.000000}%
\pgfsetfillcolor{currentfill}%
\pgfsetfillopacity{0.800000}%
\pgfsetlinewidth{0.000000pt}%
\definecolor{currentstroke}{rgb}{0.000000,0.000000,0.000000}%
\pgfsetstrokecolor{currentstroke}%
\pgfsetdash{}{0pt}%
\pgfpathmoveto{\pgfqpoint{1.334167in}{3.417375in}}%
\pgfpathlineto{\pgfqpoint{1.264249in}{3.455977in}}%
\pgfpathlineto{\pgfqpoint{1.334167in}{3.415177in}}%
\pgfpathlineto{\pgfqpoint{1.404086in}{3.376692in}}%
\pgfpathlineto{\pgfqpoint{1.334167in}{3.417375in}}%
\pgfpathclose%
\pgfusepath{fill}%
\end{pgfscope}%
\begin{pgfscope}%
\pgfpathrectangle{\pgfqpoint{0.637500in}{0.550000in}}{\pgfqpoint{3.850000in}{3.850000in}}%
\pgfusepath{clip}%
\pgfsetbuttcap%
\pgfsetroundjoin%
\definecolor{currentfill}{rgb}{0.900779,0.582857,0.000000}%
\pgfsetfillcolor{currentfill}%
\pgfsetfillopacity{0.800000}%
\pgfsetlinewidth{0.000000pt}%
\definecolor{currentstroke}{rgb}{0.000000,0.000000,0.000000}%
\pgfsetstrokecolor{currentstroke}%
\pgfsetdash{}{0pt}%
\pgfpathmoveto{\pgfqpoint{1.753679in}{3.181789in}}%
\pgfpathlineto{\pgfqpoint{1.683761in}{3.218544in}}%
\pgfpathlineto{\pgfqpoint{1.753679in}{3.179590in}}%
\pgfpathlineto{\pgfqpoint{1.823598in}{3.143651in}}%
\pgfpathlineto{\pgfqpoint{1.753679in}{3.181789in}}%
\pgfpathclose%
\pgfusepath{fill}%
\end{pgfscope}%
\begin{pgfscope}%
\pgfpathrectangle{\pgfqpoint{0.637500in}{0.550000in}}{\pgfqpoint{3.850000in}{3.850000in}}%
\pgfusepath{clip}%
\pgfsetbuttcap%
\pgfsetroundjoin%
\definecolor{currentfill}{rgb}{0.927227,0.599970,0.000000}%
\pgfsetfillcolor{currentfill}%
\pgfsetfillopacity{0.800000}%
\pgfsetlinewidth{0.000000pt}%
\definecolor{currentstroke}{rgb}{0.000000,0.000000,0.000000}%
\pgfsetstrokecolor{currentstroke}%
\pgfsetdash{}{0pt}%
\pgfpathmoveto{\pgfqpoint{2.313028in}{2.943771in}}%
\pgfpathlineto{\pgfqpoint{2.243110in}{2.960006in}}%
\pgfpathlineto{\pgfqpoint{2.313028in}{2.941572in}}%
\pgfpathlineto{\pgfqpoint{2.382947in}{2.931102in}}%
\pgfpathlineto{\pgfqpoint{2.313028in}{2.943771in}}%
\pgfpathclose%
\pgfusepath{fill}%
\end{pgfscope}%
\begin{pgfscope}%
\pgfpathrectangle{\pgfqpoint{0.637500in}{0.550000in}}{\pgfqpoint{3.850000in}{3.850000in}}%
\pgfusepath{clip}%
\pgfsetbuttcap%
\pgfsetroundjoin%
\definecolor{currentfill}{rgb}{0.905912,0.586178,0.000000}%
\pgfsetfillcolor{currentfill}%
\pgfsetfillopacity{0.800000}%
\pgfsetlinewidth{0.000000pt}%
\definecolor{currentstroke}{rgb}{0.000000,0.000000,0.000000}%
\pgfsetstrokecolor{currentstroke}%
\pgfsetdash{}{0pt}%
\pgfpathmoveto{\pgfqpoint{1.963435in}{3.073294in}}%
\pgfpathlineto{\pgfqpoint{1.893516in}{3.106625in}}%
\pgfpathlineto{\pgfqpoint{1.963435in}{3.071095in}}%
\pgfpathlineto{\pgfqpoint{2.033354in}{3.039758in}}%
\pgfpathlineto{\pgfqpoint{1.963435in}{3.073294in}}%
\pgfpathclose%
\pgfusepath{fill}%
\end{pgfscope}%
\begin{pgfscope}%
\pgfpathrectangle{\pgfqpoint{0.637500in}{0.550000in}}{\pgfqpoint{3.850000in}{3.850000in}}%
\pgfusepath{clip}%
\pgfsetbuttcap%
\pgfsetroundjoin%
\definecolor{currentfill}{rgb}{0.925053,0.598564,0.000000}%
\pgfsetfillcolor{currentfill}%
\pgfsetfillopacity{0.800000}%
\pgfsetlinewidth{0.000000pt}%
\definecolor{currentstroke}{rgb}{0.000000,0.000000,0.000000}%
\pgfsetstrokecolor{currentstroke}%
\pgfsetdash{}{0pt}%
\pgfpathmoveto{\pgfqpoint{2.732540in}{2.969906in}}%
\pgfpathlineto{\pgfqpoint{2.662622in}{2.949276in}}%
\pgfpathlineto{\pgfqpoint{2.732540in}{2.967707in}}%
\pgfpathlineto{\pgfqpoint{2.802459in}{2.992985in}}%
\pgfpathlineto{\pgfqpoint{2.732540in}{2.969906in}}%
\pgfpathclose%
\pgfusepath{fill}%
\end{pgfscope}%
\begin{pgfscope}%
\pgfpathrectangle{\pgfqpoint{0.637500in}{0.550000in}}{\pgfqpoint{3.850000in}{3.850000in}}%
\pgfusepath{clip}%
\pgfsetbuttcap%
\pgfsetroundjoin%
\definecolor{currentfill}{rgb}{0.916926,0.593305,0.000000}%
\pgfsetfillcolor{currentfill}%
\pgfsetfillopacity{0.800000}%
\pgfsetlinewidth{0.000000pt}%
\definecolor{currentstroke}{rgb}{0.000000,0.000000,0.000000}%
\pgfsetstrokecolor{currentstroke}%
\pgfsetdash{}{0pt}%
\pgfpathmoveto{\pgfqpoint{2.173191in}{2.983439in}}%
\pgfpathlineto{\pgfqpoint{2.103272in}{3.008829in}}%
\pgfpathlineto{\pgfqpoint{2.173191in}{2.981240in}}%
\pgfpathlineto{\pgfqpoint{2.243110in}{2.960006in}}%
\pgfpathlineto{\pgfqpoint{2.173191in}{2.983439in}}%
\pgfpathclose%
\pgfusepath{fill}%
\end{pgfscope}%
\begin{pgfscope}%
\pgfpathrectangle{\pgfqpoint{0.637500in}{0.550000in}}{\pgfqpoint{3.850000in}{3.850000in}}%
\pgfusepath{clip}%
\pgfsetbuttcap%
\pgfsetroundjoin%
\definecolor{currentfill}{rgb}{0.898565,0.581425,0.000000}%
\pgfsetfillcolor{currentfill}%
\pgfsetfillopacity{0.800000}%
\pgfsetlinewidth{0.000000pt}%
\definecolor{currentstroke}{rgb}{0.000000,0.000000,0.000000}%
\pgfsetstrokecolor{currentstroke}%
\pgfsetdash{}{0pt}%
\pgfpathmoveto{\pgfqpoint{3.501646in}{3.353771in}}%
\pgfpathlineto{\pgfqpoint{3.431727in}{3.313372in}}%
\pgfpathlineto{\pgfqpoint{3.501646in}{3.351572in}}%
\pgfpathlineto{\pgfqpoint{3.571564in}{3.392167in}}%
\pgfpathlineto{\pgfqpoint{3.501646in}{3.353771in}}%
\pgfpathclose%
\pgfusepath{fill}%
\end{pgfscope}%
\begin{pgfscope}%
\pgfpathrectangle{\pgfqpoint{0.637500in}{0.550000in}}{\pgfqpoint{3.850000in}{3.850000in}}%
\pgfusepath{clip}%
\pgfsetbuttcap%
\pgfsetroundjoin%
\definecolor{currentfill}{rgb}{0.900362,0.582587,0.000000}%
\pgfsetfillcolor{currentfill}%
\pgfsetfillopacity{0.800000}%
\pgfsetlinewidth{0.000000pt}%
\definecolor{currentstroke}{rgb}{0.000000,0.000000,0.000000}%
\pgfsetstrokecolor{currentstroke}%
\pgfsetdash{}{0pt}%
\pgfpathmoveto{\pgfqpoint{3.291890in}{3.235697in}}%
\pgfpathlineto{\pgfqpoint{3.221971in}{3.196471in}}%
\pgfpathlineto{\pgfqpoint{3.291890in}{3.233499in}}%
\pgfpathlineto{\pgfqpoint{3.361808in}{3.273247in}}%
\pgfpathlineto{\pgfqpoint{3.291890in}{3.235697in}}%
\pgfpathclose%
\pgfusepath{fill}%
\end{pgfscope}%
\begin{pgfscope}%
\pgfpathrectangle{\pgfqpoint{0.637500in}{0.550000in}}{\pgfqpoint{3.850000in}{3.850000in}}%
\pgfusepath{clip}%
\pgfsetbuttcap%
\pgfsetroundjoin%
\definecolor{currentfill}{rgb}{0.897887,0.580986,0.000000}%
\pgfsetfillcolor{currentfill}%
\pgfsetfillopacity{0.800000}%
\pgfsetlinewidth{0.000000pt}%
\definecolor{currentstroke}{rgb}{0.000000,0.000000,0.000000}%
\pgfsetstrokecolor{currentstroke}%
\pgfsetdash{}{0pt}%
\pgfpathmoveto{\pgfqpoint{3.711402in}{3.473743in}}%
\pgfpathlineto{\pgfqpoint{3.641483in}{3.432904in}}%
\pgfpathlineto{\pgfqpoint{3.711402in}{3.471544in}}%
\pgfpathlineto{\pgfqpoint{3.781320in}{3.512456in}}%
\pgfpathlineto{\pgfqpoint{3.711402in}{3.473743in}}%
\pgfpathclose%
\pgfusepath{fill}%
\end{pgfscope}%
\begin{pgfscope}%
\pgfpathrectangle{\pgfqpoint{0.637500in}{0.550000in}}{\pgfqpoint{3.850000in}{3.850000in}}%
\pgfusepath{clip}%
\pgfsetbuttcap%
\pgfsetroundjoin%
\definecolor{currentfill}{rgb}{0.904905,0.585527,0.000000}%
\pgfsetfillcolor{currentfill}%
\pgfsetfillopacity{0.800000}%
\pgfsetlinewidth{0.000000pt}%
\definecolor{currentstroke}{rgb}{0.000000,0.000000,0.000000}%
\pgfsetstrokecolor{currentstroke}%
\pgfsetdash{}{0pt}%
\pgfpathmoveto{\pgfqpoint{3.082134in}{3.122625in}}%
\pgfpathlineto{\pgfqpoint{3.012215in}{3.086414in}}%
\pgfpathlineto{\pgfqpoint{3.082134in}{3.120426in}}%
\pgfpathlineto{\pgfqpoint{3.152052in}{3.157960in}}%
\pgfpathlineto{\pgfqpoint{3.082134in}{3.122625in}}%
\pgfpathclose%
\pgfusepath{fill}%
\end{pgfscope}%
\begin{pgfscope}%
\pgfpathrectangle{\pgfqpoint{0.637500in}{0.550000in}}{\pgfqpoint{3.850000in}{3.850000in}}%
\pgfusepath{clip}%
\pgfsetbuttcap%
\pgfsetroundjoin%
\definecolor{currentfill}{rgb}{0.898127,0.581141,0.000000}%
\pgfsetfillcolor{currentfill}%
\pgfsetfillopacity{0.800000}%
\pgfsetlinewidth{0.000000pt}%
\definecolor{currentstroke}{rgb}{0.000000,0.000000,0.000000}%
\pgfsetstrokecolor{currentstroke}%
\pgfsetdash{}{0pt}%
\pgfpathmoveto{\pgfqpoint{1.404086in}{3.376692in}}%
\pgfpathlineto{\pgfqpoint{1.334167in}{3.415177in}}%
\pgfpathlineto{\pgfqpoint{1.404086in}{3.374493in}}%
\pgfpathlineto{\pgfqpoint{1.474005in}{3.336171in}}%
\pgfpathlineto{\pgfqpoint{1.404086in}{3.376692in}}%
\pgfpathclose%
\pgfusepath{fill}%
\end{pgfscope}%
\begin{pgfscope}%
\pgfpathrectangle{\pgfqpoint{0.637500in}{0.550000in}}{\pgfqpoint{3.850000in}{3.850000in}}%
\pgfusepath{clip}%
\pgfsetbuttcap%
\pgfsetroundjoin%
\definecolor{currentfill}{rgb}{0.899206,0.581839,0.000000}%
\pgfsetfillcolor{currentfill}%
\pgfsetfillopacity{0.800000}%
\pgfsetlinewidth{0.000000pt}%
\definecolor{currentstroke}{rgb}{0.000000,0.000000,0.000000}%
\pgfsetstrokecolor{currentstroke}%
\pgfsetdash{}{0pt}%
\pgfpathmoveto{\pgfqpoint{1.613842in}{3.258093in}}%
\pgfpathlineto{\pgfqpoint{1.543923in}{3.295877in}}%
\pgfpathlineto{\pgfqpoint{1.613842in}{3.255894in}}%
\pgfpathlineto{\pgfqpoint{1.683761in}{3.218544in}}%
\pgfpathlineto{\pgfqpoint{1.613842in}{3.258093in}}%
\pgfpathclose%
\pgfusepath{fill}%
\end{pgfscope}%
\begin{pgfscope}%
\pgfpathrectangle{\pgfqpoint{0.637500in}{0.550000in}}{\pgfqpoint{3.850000in}{3.850000in}}%
\pgfusepath{clip}%
\pgfsetbuttcap%
\pgfsetroundjoin%
\definecolor{currentfill}{rgb}{0.914972,0.592040,0.000000}%
\pgfsetfillcolor{currentfill}%
\pgfsetfillopacity{0.800000}%
\pgfsetlinewidth{0.000000pt}%
\definecolor{currentstroke}{rgb}{0.000000,0.000000,0.000000}%
\pgfsetstrokecolor{currentstroke}%
\pgfsetdash{}{0pt}%
\pgfpathmoveto{\pgfqpoint{2.872378in}{3.022069in}}%
\pgfpathlineto{\pgfqpoint{2.802459in}{2.992985in}}%
\pgfpathlineto{\pgfqpoint{2.872378in}{3.019871in}}%
\pgfpathlineto{\pgfqpoint{2.942296in}{3.051975in}}%
\pgfpathlineto{\pgfqpoint{2.872378in}{3.022069in}}%
\pgfpathclose%
\pgfusepath{fill}%
\end{pgfscope}%
\begin{pgfscope}%
\pgfpathrectangle{\pgfqpoint{0.637500in}{0.550000in}}{\pgfqpoint{3.850000in}{3.850000in}}%
\pgfusepath{clip}%
\pgfsetbuttcap%
\pgfsetroundjoin%
\definecolor{currentfill}{rgb}{0.902018,0.583658,0.000000}%
\pgfsetfillcolor{currentfill}%
\pgfsetfillopacity{0.800000}%
\pgfsetlinewidth{0.000000pt}%
\definecolor{currentstroke}{rgb}{0.000000,0.000000,0.000000}%
\pgfsetstrokecolor{currentstroke}%
\pgfsetdash{}{0pt}%
\pgfpathmoveto{\pgfqpoint{1.823598in}{3.143651in}}%
\pgfpathlineto{\pgfqpoint{1.753679in}{3.179590in}}%
\pgfpathlineto{\pgfqpoint{1.823598in}{3.141452in}}%
\pgfpathlineto{\pgfqpoint{1.893516in}{3.106625in}}%
\pgfpathlineto{\pgfqpoint{1.823598in}{3.143651in}}%
\pgfpathclose%
\pgfusepath{fill}%
\end{pgfscope}%
\begin{pgfscope}%
\pgfpathrectangle{\pgfqpoint{0.637500in}{0.550000in}}{\pgfqpoint{3.850000in}{3.850000in}}%
\pgfusepath{clip}%
\pgfsetbuttcap%
\pgfsetroundjoin%
\definecolor{currentfill}{rgb}{0.908810,0.588053,0.000000}%
\pgfsetfillcolor{currentfill}%
\pgfsetfillopacity{0.800000}%
\pgfsetlinewidth{0.000000pt}%
\definecolor{currentstroke}{rgb}{0.000000,0.000000,0.000000}%
\pgfsetstrokecolor{currentstroke}%
\pgfsetdash{}{0pt}%
\pgfpathmoveto{\pgfqpoint{2.033354in}{3.039758in}}%
\pgfpathlineto{\pgfqpoint{1.963435in}{3.071095in}}%
\pgfpathlineto{\pgfqpoint{2.033354in}{3.037559in}}%
\pgfpathlineto{\pgfqpoint{2.103272in}{3.008829in}}%
\pgfpathlineto{\pgfqpoint{2.033354in}{3.039758in}}%
\pgfpathclose%
\pgfusepath{fill}%
\end{pgfscope}%
\begin{pgfscope}%
\pgfpathrectangle{\pgfqpoint{0.637500in}{0.550000in}}{\pgfqpoint{3.850000in}{3.850000in}}%
\pgfusepath{clip}%
\pgfsetbuttcap%
\pgfsetroundjoin%
\definecolor{currentfill}{rgb}{0.935716,0.605463,0.000000}%
\pgfsetfillcolor{currentfill}%
\pgfsetfillopacity{0.800000}%
\pgfsetlinewidth{0.000000pt}%
\definecolor{currentstroke}{rgb}{0.000000,0.000000,0.000000}%
\pgfsetstrokecolor{currentstroke}%
\pgfsetdash{}{0pt}%
\pgfpathmoveto{\pgfqpoint{2.522784in}{2.927269in}}%
\pgfpathlineto{\pgfqpoint{2.452866in}{2.924776in}}%
\pgfpathlineto{\pgfqpoint{2.522784in}{2.925070in}}%
\pgfpathlineto{\pgfqpoint{2.592703in}{2.934108in}}%
\pgfpathlineto{\pgfqpoint{2.522784in}{2.927269in}}%
\pgfpathclose%
\pgfusepath{fill}%
\end{pgfscope}%
\begin{pgfscope}%
\pgfpathrectangle{\pgfqpoint{0.637500in}{0.550000in}}{\pgfqpoint{3.850000in}{3.850000in}}%
\pgfusepath{clip}%
\pgfsetbuttcap%
\pgfsetroundjoin%
\definecolor{currentfill}{rgb}{0.931871,0.602975,0.000000}%
\pgfsetfillcolor{currentfill}%
\pgfsetfillopacity{0.800000}%
\pgfsetlinewidth{0.000000pt}%
\definecolor{currentstroke}{rgb}{0.000000,0.000000,0.000000}%
\pgfsetstrokecolor{currentstroke}%
\pgfsetdash{}{0pt}%
\pgfpathmoveto{\pgfqpoint{2.382947in}{2.931102in}}%
\pgfpathlineto{\pgfqpoint{2.313028in}{2.941572in}}%
\pgfpathlineto{\pgfqpoint{2.382947in}{2.928904in}}%
\pgfpathlineto{\pgfqpoint{2.452866in}{2.924776in}}%
\pgfpathlineto{\pgfqpoint{2.382947in}{2.931102in}}%
\pgfpathclose%
\pgfusepath{fill}%
\end{pgfscope}%
\begin{pgfscope}%
\pgfpathrectangle{\pgfqpoint{0.637500in}{0.550000in}}{\pgfqpoint{3.850000in}{3.850000in}}%
\pgfusepath{clip}%
\pgfsetbuttcap%
\pgfsetroundjoin%
\definecolor{currentfill}{rgb}{0.930057,0.601801,0.000000}%
\pgfsetfillcolor{currentfill}%
\pgfsetfillopacity{0.800000}%
\pgfsetlinewidth{0.000000pt}%
\definecolor{currentstroke}{rgb}{0.000000,0.000000,0.000000}%
\pgfsetstrokecolor{currentstroke}%
\pgfsetdash{}{0pt}%
\pgfpathmoveto{\pgfqpoint{2.662622in}{2.949276in}}%
\pgfpathlineto{\pgfqpoint{2.592703in}{2.934108in}}%
\pgfpathlineto{\pgfqpoint{2.662622in}{2.947078in}}%
\pgfpathlineto{\pgfqpoint{2.732540in}{2.967707in}}%
\pgfpathlineto{\pgfqpoint{2.662622in}{2.949276in}}%
\pgfpathclose%
\pgfusepath{fill}%
\end{pgfscope}%
\begin{pgfscope}%
\pgfpathrectangle{\pgfqpoint{0.637500in}{0.550000in}}{\pgfqpoint{3.850000in}{3.850000in}}%
\pgfusepath{clip}%
\pgfsetbuttcap%
\pgfsetroundjoin%
\definecolor{currentfill}{rgb}{0.898985,0.581696,0.000000}%
\pgfsetfillcolor{currentfill}%
\pgfsetfillopacity{0.800000}%
\pgfsetlinewidth{0.000000pt}%
\definecolor{currentstroke}{rgb}{0.000000,0.000000,0.000000}%
\pgfsetstrokecolor{currentstroke}%
\pgfsetdash{}{0pt}%
\pgfpathmoveto{\pgfqpoint{3.431727in}{3.313372in}}%
\pgfpathlineto{\pgfqpoint{3.361808in}{3.273247in}}%
\pgfpathlineto{\pgfqpoint{3.431727in}{3.311173in}}%
\pgfpathlineto{\pgfqpoint{3.501646in}{3.351572in}}%
\pgfpathlineto{\pgfqpoint{3.431727in}{3.313372in}}%
\pgfpathclose%
\pgfusepath{fill}%
\end{pgfscope}%
\begin{pgfscope}%
\pgfpathrectangle{\pgfqpoint{0.637500in}{0.550000in}}{\pgfqpoint{3.850000in}{3.850000in}}%
\pgfusepath{clip}%
\pgfsetbuttcap%
\pgfsetroundjoin%
\definecolor{currentfill}{rgb}{0.898044,0.581087,0.000000}%
\pgfsetfillcolor{currentfill}%
\pgfsetfillopacity{0.800000}%
\pgfsetlinewidth{0.000000pt}%
\definecolor{currentstroke}{rgb}{0.000000,0.000000,0.000000}%
\pgfsetstrokecolor{currentstroke}%
\pgfsetdash{}{0pt}%
\pgfpathmoveto{\pgfqpoint{3.641483in}{3.432904in}}%
\pgfpathlineto{\pgfqpoint{3.571564in}{3.392167in}}%
\pgfpathlineto{\pgfqpoint{3.641483in}{3.430706in}}%
\pgfpathlineto{\pgfqpoint{3.711402in}{3.471544in}}%
\pgfpathlineto{\pgfqpoint{3.641483in}{3.432904in}}%
\pgfpathclose%
\pgfusepath{fill}%
\end{pgfscope}%
\begin{pgfscope}%
\pgfpathrectangle{\pgfqpoint{0.637500in}{0.550000in}}{\pgfqpoint{3.850000in}{3.850000in}}%
\pgfusepath{clip}%
\pgfsetbuttcap%
\pgfsetroundjoin%
\definecolor{currentfill}{rgb}{0.901452,0.583292,0.000000}%
\pgfsetfillcolor{currentfill}%
\pgfsetfillopacity{0.800000}%
\pgfsetlinewidth{0.000000pt}%
\definecolor{currentstroke}{rgb}{0.000000,0.000000,0.000000}%
\pgfsetstrokecolor{currentstroke}%
\pgfsetdash{}{0pt}%
\pgfpathmoveto{\pgfqpoint{3.221971in}{3.196471in}}%
\pgfpathlineto{\pgfqpoint{3.152052in}{3.157960in}}%
\pgfpathlineto{\pgfqpoint{3.221971in}{3.194272in}}%
\pgfpathlineto{\pgfqpoint{3.291890in}{3.233499in}}%
\pgfpathlineto{\pgfqpoint{3.221971in}{3.196471in}}%
\pgfpathclose%
\pgfusepath{fill}%
\end{pgfscope}%
\begin{pgfscope}%
\pgfpathrectangle{\pgfqpoint{0.637500in}{0.550000in}}{\pgfqpoint{3.850000in}{3.850000in}}%
\pgfusepath{clip}%
\pgfsetbuttcap%
\pgfsetroundjoin%
\definecolor{currentfill}{rgb}{0.921990,0.596582,0.000000}%
\pgfsetfillcolor{currentfill}%
\pgfsetfillopacity{0.800000}%
\pgfsetlinewidth{0.000000pt}%
\definecolor{currentstroke}{rgb}{0.000000,0.000000,0.000000}%
\pgfsetstrokecolor{currentstroke}%
\pgfsetdash{}{0pt}%
\pgfpathmoveto{\pgfqpoint{2.243110in}{2.960006in}}%
\pgfpathlineto{\pgfqpoint{2.173191in}{2.981240in}}%
\pgfpathlineto{\pgfqpoint{2.243110in}{2.957807in}}%
\pgfpathlineto{\pgfqpoint{2.313028in}{2.941572in}}%
\pgfpathlineto{\pgfqpoint{2.243110in}{2.960006in}}%
\pgfpathclose%
\pgfusepath{fill}%
\end{pgfscope}%
\begin{pgfscope}%
\pgfpathrectangle{\pgfqpoint{0.637500in}{0.550000in}}{\pgfqpoint{3.850000in}{3.850000in}}%
\pgfusepath{clip}%
\pgfsetbuttcap%
\pgfsetroundjoin%
\definecolor{currentfill}{rgb}{0.907506,0.587210,0.000000}%
\pgfsetfillcolor{currentfill}%
\pgfsetfillopacity{0.800000}%
\pgfsetlinewidth{0.000000pt}%
\definecolor{currentstroke}{rgb}{0.000000,0.000000,0.000000}%
\pgfsetstrokecolor{currentstroke}%
\pgfsetdash{}{0pt}%
\pgfpathmoveto{\pgfqpoint{3.012215in}{3.086414in}}%
\pgfpathlineto{\pgfqpoint{2.942296in}{3.051975in}}%
\pgfpathlineto{\pgfqpoint{3.012215in}{3.084216in}}%
\pgfpathlineto{\pgfqpoint{3.082134in}{3.120426in}}%
\pgfpathlineto{\pgfqpoint{3.012215in}{3.086414in}}%
\pgfpathclose%
\pgfusepath{fill}%
\end{pgfscope}%
\begin{pgfscope}%
\pgfpathrectangle{\pgfqpoint{0.637500in}{0.550000in}}{\pgfqpoint{3.850000in}{3.850000in}}%
\pgfusepath{clip}%
\pgfsetbuttcap%
\pgfsetroundjoin%
\definecolor{currentfill}{rgb}{0.898377,0.581303,0.000000}%
\pgfsetfillcolor{currentfill}%
\pgfsetfillopacity{0.800000}%
\pgfsetlinewidth{0.000000pt}%
\definecolor{currentstroke}{rgb}{0.000000,0.000000,0.000000}%
\pgfsetstrokecolor{currentstroke}%
\pgfsetdash{}{0pt}%
\pgfpathmoveto{\pgfqpoint{1.474005in}{3.336171in}}%
\pgfpathlineto{\pgfqpoint{1.404086in}{3.374493in}}%
\pgfpathlineto{\pgfqpoint{1.474005in}{3.333973in}}%
\pgfpathlineto{\pgfqpoint{1.543923in}{3.295877in}}%
\pgfpathlineto{\pgfqpoint{1.474005in}{3.336171in}}%
\pgfpathclose%
\pgfusepath{fill}%
\end{pgfscope}%
\begin{pgfscope}%
\pgfpathrectangle{\pgfqpoint{0.637500in}{0.550000in}}{\pgfqpoint{3.850000in}{3.850000in}}%
\pgfusepath{clip}%
\pgfsetbuttcap%
\pgfsetroundjoin%
\definecolor{currentfill}{rgb}{0.899869,0.582268,0.000000}%
\pgfsetfillcolor{currentfill}%
\pgfsetfillopacity{0.800000}%
\pgfsetlinewidth{0.000000pt}%
\definecolor{currentstroke}{rgb}{0.000000,0.000000,0.000000}%
\pgfsetstrokecolor{currentstroke}%
\pgfsetdash{}{0pt}%
\pgfpathmoveto{\pgfqpoint{1.683761in}{3.218544in}}%
\pgfpathlineto{\pgfqpoint{1.613842in}{3.255894in}}%
\pgfpathlineto{\pgfqpoint{1.683761in}{3.216345in}}%
\pgfpathlineto{\pgfqpoint{1.753679in}{3.179590in}}%
\pgfpathlineto{\pgfqpoint{1.683761in}{3.218544in}}%
\pgfpathclose%
\pgfusepath{fill}%
\end{pgfscope}%
\begin{pgfscope}%
\pgfpathrectangle{\pgfqpoint{0.637500in}{0.550000in}}{\pgfqpoint{3.850000in}{3.850000in}}%
\pgfusepath{clip}%
\pgfsetbuttcap%
\pgfsetroundjoin%
\definecolor{currentfill}{rgb}{0.903689,0.584740,0.000000}%
\pgfsetfillcolor{currentfill}%
\pgfsetfillopacity{0.800000}%
\pgfsetlinewidth{0.000000pt}%
\definecolor{currentstroke}{rgb}{0.000000,0.000000,0.000000}%
\pgfsetstrokecolor{currentstroke}%
\pgfsetdash{}{0pt}%
\pgfpathmoveto{\pgfqpoint{1.893516in}{3.106625in}}%
\pgfpathlineto{\pgfqpoint{1.823598in}{3.141452in}}%
\pgfpathlineto{\pgfqpoint{1.893516in}{3.104426in}}%
\pgfpathlineto{\pgfqpoint{1.963435in}{3.071095in}}%
\pgfpathlineto{\pgfqpoint{1.893516in}{3.106625in}}%
\pgfpathclose%
\pgfusepath{fill}%
\end{pgfscope}%
\begin{pgfscope}%
\pgfpathrectangle{\pgfqpoint{0.637500in}{0.550000in}}{\pgfqpoint{3.850000in}{3.850000in}}%
\pgfusepath{clip}%
\pgfsetbuttcap%
\pgfsetroundjoin%
\definecolor{currentfill}{rgb}{0.919814,0.595174,0.000000}%
\pgfsetfillcolor{currentfill}%
\pgfsetfillopacity{0.800000}%
\pgfsetlinewidth{0.000000pt}%
\definecolor{currentstroke}{rgb}{0.000000,0.000000,0.000000}%
\pgfsetstrokecolor{currentstroke}%
\pgfsetdash{}{0pt}%
\pgfpathmoveto{\pgfqpoint{2.802459in}{2.992985in}}%
\pgfpathlineto{\pgfqpoint{2.732540in}{2.967707in}}%
\pgfpathlineto{\pgfqpoint{2.802459in}{2.990786in}}%
\pgfpathlineto{\pgfqpoint{2.872378in}{3.019871in}}%
\pgfpathlineto{\pgfqpoint{2.802459in}{2.992985in}}%
\pgfpathclose%
\pgfusepath{fill}%
\end{pgfscope}%
\begin{pgfscope}%
\pgfpathrectangle{\pgfqpoint{0.637500in}{0.550000in}}{\pgfqpoint{3.850000in}{3.850000in}}%
\pgfusepath{clip}%
\pgfsetbuttcap%
\pgfsetroundjoin%
\definecolor{currentfill}{rgb}{0.912477,0.590427,0.000000}%
\pgfsetfillcolor{currentfill}%
\pgfsetfillopacity{0.800000}%
\pgfsetlinewidth{0.000000pt}%
\definecolor{currentstroke}{rgb}{0.000000,0.000000,0.000000}%
\pgfsetstrokecolor{currentstroke}%
\pgfsetdash{}{0pt}%
\pgfpathmoveto{\pgfqpoint{2.103272in}{3.008829in}}%
\pgfpathlineto{\pgfqpoint{2.033354in}{3.037559in}}%
\pgfpathlineto{\pgfqpoint{2.103272in}{3.006630in}}%
\pgfpathlineto{\pgfqpoint{2.173191in}{2.981240in}}%
\pgfpathlineto{\pgfqpoint{2.103272in}{3.008829in}}%
\pgfpathclose%
\pgfusepath{fill}%
\end{pgfscope}%
\begin{pgfscope}%
\pgfpathrectangle{\pgfqpoint{0.637500in}{0.550000in}}{\pgfqpoint{3.850000in}{3.850000in}}%
\pgfusepath{clip}%
\pgfsetbuttcap%
\pgfsetroundjoin%
\definecolor{currentfill}{rgb}{0.898262,0.581229,0.000000}%
\pgfsetfillcolor{currentfill}%
\pgfsetfillopacity{0.800000}%
\pgfsetlinewidth{0.000000pt}%
\definecolor{currentstroke}{rgb}{0.000000,0.000000,0.000000}%
\pgfsetstrokecolor{currentstroke}%
\pgfsetdash{}{0pt}%
\pgfpathmoveto{\pgfqpoint{3.571564in}{3.392167in}}%
\pgfpathlineto{\pgfqpoint{3.501646in}{3.351572in}}%
\pgfpathlineto{\pgfqpoint{3.571564in}{3.389969in}}%
\pgfpathlineto{\pgfqpoint{3.641483in}{3.430706in}}%
\pgfpathlineto{\pgfqpoint{3.571564in}{3.392167in}}%
\pgfpathclose%
\pgfusepath{fill}%
\end{pgfscope}%
\begin{pgfscope}%
\pgfpathrectangle{\pgfqpoint{0.637500in}{0.550000in}}{\pgfqpoint{3.850000in}{3.850000in}}%
\pgfusepath{clip}%
\pgfsetbuttcap%
\pgfsetroundjoin%
\definecolor{currentfill}{rgb}{0.899565,0.582072,0.000000}%
\pgfsetfillcolor{currentfill}%
\pgfsetfillopacity{0.800000}%
\pgfsetlinewidth{0.000000pt}%
\definecolor{currentstroke}{rgb}{0.000000,0.000000,0.000000}%
\pgfsetstrokecolor{currentstroke}%
\pgfsetdash{}{0pt}%
\pgfpathmoveto{\pgfqpoint{3.361808in}{3.273247in}}%
\pgfpathlineto{\pgfqpoint{3.291890in}{3.233499in}}%
\pgfpathlineto{\pgfqpoint{3.361808in}{3.271048in}}%
\pgfpathlineto{\pgfqpoint{3.431727in}{3.311173in}}%
\pgfpathlineto{\pgfqpoint{3.361808in}{3.273247in}}%
\pgfpathclose%
\pgfusepath{fill}%
\end{pgfscope}%
\begin{pgfscope}%
\pgfpathrectangle{\pgfqpoint{0.637500in}{0.550000in}}{\pgfqpoint{3.850000in}{3.850000in}}%
\pgfusepath{clip}%
\pgfsetbuttcap%
\pgfsetroundjoin%
\definecolor{currentfill}{rgb}{0.902928,0.584248,0.000000}%
\pgfsetfillcolor{currentfill}%
\pgfsetfillopacity{0.800000}%
\pgfsetlinewidth{0.000000pt}%
\definecolor{currentstroke}{rgb}{0.000000,0.000000,0.000000}%
\pgfsetstrokecolor{currentstroke}%
\pgfsetdash{}{0pt}%
\pgfpathmoveto{\pgfqpoint{3.152052in}{3.157960in}}%
\pgfpathlineto{\pgfqpoint{3.082134in}{3.120426in}}%
\pgfpathlineto{\pgfqpoint{3.152052in}{3.155761in}}%
\pgfpathlineto{\pgfqpoint{3.221971in}{3.194272in}}%
\pgfpathlineto{\pgfqpoint{3.152052in}{3.157960in}}%
\pgfpathclose%
\pgfusepath{fill}%
\end{pgfscope}%
\begin{pgfscope}%
\pgfpathrectangle{\pgfqpoint{0.637500in}{0.550000in}}{\pgfqpoint{3.850000in}{3.850000in}}%
\pgfusepath{clip}%
\pgfsetbuttcap%
\pgfsetroundjoin%
\definecolor{currentfill}{rgb}{0.934964,0.604977,0.000000}%
\pgfsetfillcolor{currentfill}%
\pgfsetfillopacity{0.800000}%
\pgfsetlinewidth{0.000000pt}%
\definecolor{currentstroke}{rgb}{0.000000,0.000000,0.000000}%
\pgfsetstrokecolor{currentstroke}%
\pgfsetdash{}{0pt}%
\pgfpathmoveto{\pgfqpoint{2.452866in}{2.924776in}}%
\pgfpathlineto{\pgfqpoint{2.382947in}{2.928904in}}%
\pgfpathlineto{\pgfqpoint{2.452866in}{2.922577in}}%
\pgfpathlineto{\pgfqpoint{2.522784in}{2.925070in}}%
\pgfpathlineto{\pgfqpoint{2.452866in}{2.924776in}}%
\pgfpathclose%
\pgfusepath{fill}%
\end{pgfscope}%
\begin{pgfscope}%
\pgfpathrectangle{\pgfqpoint{0.637500in}{0.550000in}}{\pgfqpoint{3.850000in}{3.850000in}}%
\pgfusepath{clip}%
\pgfsetbuttcap%
\pgfsetroundjoin%
\definecolor{currentfill}{rgb}{0.910843,0.589369,0.000000}%
\pgfsetfillcolor{currentfill}%
\pgfsetfillopacity{0.800000}%
\pgfsetlinewidth{0.000000pt}%
\definecolor{currentstroke}{rgb}{0.000000,0.000000,0.000000}%
\pgfsetstrokecolor{currentstroke}%
\pgfsetdash{}{0pt}%
\pgfpathmoveto{\pgfqpoint{2.942296in}{3.051975in}}%
\pgfpathlineto{\pgfqpoint{2.872378in}{3.019871in}}%
\pgfpathlineto{\pgfqpoint{2.942296in}{3.049776in}}%
\pgfpathlineto{\pgfqpoint{3.012215in}{3.084216in}}%
\pgfpathlineto{\pgfqpoint{2.942296in}{3.051975in}}%
\pgfpathclose%
\pgfusepath{fill}%
\end{pgfscope}%
\begin{pgfscope}%
\pgfpathrectangle{\pgfqpoint{0.637500in}{0.550000in}}{\pgfqpoint{3.850000in}{3.850000in}}%
\pgfusepath{clip}%
\pgfsetbuttcap%
\pgfsetroundjoin%
\definecolor{currentfill}{rgb}{0.933916,0.604299,0.000000}%
\pgfsetfillcolor{currentfill}%
\pgfsetfillopacity{0.800000}%
\pgfsetlinewidth{0.000000pt}%
\definecolor{currentstroke}{rgb}{0.000000,0.000000,0.000000}%
\pgfsetstrokecolor{currentstroke}%
\pgfsetdash{}{0pt}%
\pgfpathmoveto{\pgfqpoint{2.592703in}{2.934108in}}%
\pgfpathlineto{\pgfqpoint{2.522784in}{2.925070in}}%
\pgfpathlineto{\pgfqpoint{2.592703in}{2.931909in}}%
\pgfpathlineto{\pgfqpoint{2.662622in}{2.947078in}}%
\pgfpathlineto{\pgfqpoint{2.592703in}{2.934108in}}%
\pgfpathclose%
\pgfusepath{fill}%
\end{pgfscope}%
\begin{pgfscope}%
\pgfpathrectangle{\pgfqpoint{0.637500in}{0.550000in}}{\pgfqpoint{3.850000in}{3.850000in}}%
\pgfusepath{clip}%
\pgfsetbuttcap%
\pgfsetroundjoin%
\definecolor{currentfill}{rgb}{0.898725,0.581528,0.000000}%
\pgfsetfillcolor{currentfill}%
\pgfsetfillopacity{0.800000}%
\pgfsetlinewidth{0.000000pt}%
\definecolor{currentstroke}{rgb}{0.000000,0.000000,0.000000}%
\pgfsetstrokecolor{currentstroke}%
\pgfsetdash{}{0pt}%
\pgfpathmoveto{\pgfqpoint{1.543923in}{3.295877in}}%
\pgfpathlineto{\pgfqpoint{1.474005in}{3.333973in}}%
\pgfpathlineto{\pgfqpoint{1.543923in}{3.293678in}}%
\pgfpathlineto{\pgfqpoint{1.613842in}{3.255894in}}%
\pgfpathlineto{\pgfqpoint{1.543923in}{3.295877in}}%
\pgfpathclose%
\pgfusepath{fill}%
\end{pgfscope}%
\begin{pgfscope}%
\pgfpathrectangle{\pgfqpoint{0.637500in}{0.550000in}}{\pgfqpoint{3.850000in}{3.850000in}}%
\pgfusepath{clip}%
\pgfsetbuttcap%
\pgfsetroundjoin%
\definecolor{currentfill}{rgb}{0.900779,0.582857,0.000000}%
\pgfsetfillcolor{currentfill}%
\pgfsetfillopacity{0.800000}%
\pgfsetlinewidth{0.000000pt}%
\definecolor{currentstroke}{rgb}{0.000000,0.000000,0.000000}%
\pgfsetstrokecolor{currentstroke}%
\pgfsetdash{}{0pt}%
\pgfpathmoveto{\pgfqpoint{1.753679in}{3.179590in}}%
\pgfpathlineto{\pgfqpoint{1.683761in}{3.216345in}}%
\pgfpathlineto{\pgfqpoint{1.753679in}{3.177391in}}%
\pgfpathlineto{\pgfqpoint{1.823598in}{3.141452in}}%
\pgfpathlineto{\pgfqpoint{1.753679in}{3.179590in}}%
\pgfpathclose%
\pgfusepath{fill}%
\end{pgfscope}%
\begin{pgfscope}%
\pgfpathrectangle{\pgfqpoint{0.637500in}{0.550000in}}{\pgfqpoint{3.850000in}{3.850000in}}%
\pgfusepath{clip}%
\pgfsetbuttcap%
\pgfsetroundjoin%
\definecolor{currentfill}{rgb}{0.927227,0.599970,0.000000}%
\pgfsetfillcolor{currentfill}%
\pgfsetfillopacity{0.800000}%
\pgfsetlinewidth{0.000000pt}%
\definecolor{currentstroke}{rgb}{0.000000,0.000000,0.000000}%
\pgfsetstrokecolor{currentstroke}%
\pgfsetdash{}{0pt}%
\pgfpathmoveto{\pgfqpoint{2.313028in}{2.941572in}}%
\pgfpathlineto{\pgfqpoint{2.243110in}{2.957807in}}%
\pgfpathlineto{\pgfqpoint{2.313028in}{2.939373in}}%
\pgfpathlineto{\pgfqpoint{2.382947in}{2.928904in}}%
\pgfpathlineto{\pgfqpoint{2.313028in}{2.941572in}}%
\pgfpathclose%
\pgfusepath{fill}%
\end{pgfscope}%
\begin{pgfscope}%
\pgfpathrectangle{\pgfqpoint{0.637500in}{0.550000in}}{\pgfqpoint{3.850000in}{3.850000in}}%
\pgfusepath{clip}%
\pgfsetbuttcap%
\pgfsetroundjoin%
\definecolor{currentfill}{rgb}{0.905912,0.586178,0.000000}%
\pgfsetfillcolor{currentfill}%
\pgfsetfillopacity{0.800000}%
\pgfsetlinewidth{0.000000pt}%
\definecolor{currentstroke}{rgb}{0.000000,0.000000,0.000000}%
\pgfsetstrokecolor{currentstroke}%
\pgfsetdash{}{0pt}%
\pgfpathmoveto{\pgfqpoint{1.963435in}{3.071095in}}%
\pgfpathlineto{\pgfqpoint{1.893516in}{3.104426in}}%
\pgfpathlineto{\pgfqpoint{1.963435in}{3.068896in}}%
\pgfpathlineto{\pgfqpoint{2.033354in}{3.037559in}}%
\pgfpathlineto{\pgfqpoint{1.963435in}{3.071095in}}%
\pgfpathclose%
\pgfusepath{fill}%
\end{pgfscope}%
\begin{pgfscope}%
\pgfpathrectangle{\pgfqpoint{0.637500in}{0.550000in}}{\pgfqpoint{3.850000in}{3.850000in}}%
\pgfusepath{clip}%
\pgfsetbuttcap%
\pgfsetroundjoin%
\definecolor{currentfill}{rgb}{0.925053,0.598564,0.000000}%
\pgfsetfillcolor{currentfill}%
\pgfsetfillopacity{0.800000}%
\pgfsetlinewidth{0.000000pt}%
\definecolor{currentstroke}{rgb}{0.000000,0.000000,0.000000}%
\pgfsetstrokecolor{currentstroke}%
\pgfsetdash{}{0pt}%
\pgfpathmoveto{\pgfqpoint{2.732540in}{2.967707in}}%
\pgfpathlineto{\pgfqpoint{2.662622in}{2.947078in}}%
\pgfpathlineto{\pgfqpoint{2.732540in}{2.965508in}}%
\pgfpathlineto{\pgfqpoint{2.802459in}{2.990786in}}%
\pgfpathlineto{\pgfqpoint{2.732540in}{2.967707in}}%
\pgfpathclose%
\pgfusepath{fill}%
\end{pgfscope}%
\begin{pgfscope}%
\pgfpathrectangle{\pgfqpoint{0.637500in}{0.550000in}}{\pgfqpoint{3.850000in}{3.850000in}}%
\pgfusepath{clip}%
\pgfsetbuttcap%
\pgfsetroundjoin%
\definecolor{currentfill}{rgb}{0.916926,0.593305,0.000000}%
\pgfsetfillcolor{currentfill}%
\pgfsetfillopacity{0.800000}%
\pgfsetlinewidth{0.000000pt}%
\definecolor{currentstroke}{rgb}{0.000000,0.000000,0.000000}%
\pgfsetstrokecolor{currentstroke}%
\pgfsetdash{}{0pt}%
\pgfpathmoveto{\pgfqpoint{2.173191in}{2.981240in}}%
\pgfpathlineto{\pgfqpoint{2.103272in}{3.006630in}}%
\pgfpathlineto{\pgfqpoint{2.173191in}{2.979041in}}%
\pgfpathlineto{\pgfqpoint{2.243110in}{2.957807in}}%
\pgfpathlineto{\pgfqpoint{2.173191in}{2.981240in}}%
\pgfpathclose%
\pgfusepath{fill}%
\end{pgfscope}%
\begin{pgfscope}%
\pgfpathrectangle{\pgfqpoint{0.637500in}{0.550000in}}{\pgfqpoint{3.850000in}{3.850000in}}%
\pgfusepath{clip}%
\pgfsetbuttcap%
\pgfsetroundjoin%
\definecolor{currentfill}{rgb}{0.898565,0.581425,0.000000}%
\pgfsetfillcolor{currentfill}%
\pgfsetfillopacity{0.800000}%
\pgfsetlinewidth{0.000000pt}%
\definecolor{currentstroke}{rgb}{0.000000,0.000000,0.000000}%
\pgfsetstrokecolor{currentstroke}%
\pgfsetdash{}{0pt}%
\pgfpathmoveto{\pgfqpoint{3.501646in}{3.351572in}}%
\pgfpathlineto{\pgfqpoint{3.431727in}{3.311173in}}%
\pgfpathlineto{\pgfqpoint{3.501646in}{3.349373in}}%
\pgfpathlineto{\pgfqpoint{3.571564in}{3.389969in}}%
\pgfpathlineto{\pgfqpoint{3.501646in}{3.351572in}}%
\pgfpathclose%
\pgfusepath{fill}%
\end{pgfscope}%
\begin{pgfscope}%
\pgfpathrectangle{\pgfqpoint{0.637500in}{0.550000in}}{\pgfqpoint{3.850000in}{3.850000in}}%
\pgfusepath{clip}%
\pgfsetbuttcap%
\pgfsetroundjoin%
\definecolor{currentfill}{rgb}{0.900362,0.582587,0.000000}%
\pgfsetfillcolor{currentfill}%
\pgfsetfillopacity{0.800000}%
\pgfsetlinewidth{0.000000pt}%
\definecolor{currentstroke}{rgb}{0.000000,0.000000,0.000000}%
\pgfsetstrokecolor{currentstroke}%
\pgfsetdash{}{0pt}%
\pgfpathmoveto{\pgfqpoint{3.291890in}{3.233499in}}%
\pgfpathlineto{\pgfqpoint{3.221971in}{3.194272in}}%
\pgfpathlineto{\pgfqpoint{3.291890in}{3.231300in}}%
\pgfpathlineto{\pgfqpoint{3.361808in}{3.271048in}}%
\pgfpathlineto{\pgfqpoint{3.291890in}{3.233499in}}%
\pgfpathclose%
\pgfusepath{fill}%
\end{pgfscope}%
\begin{pgfscope}%
\pgfpathrectangle{\pgfqpoint{0.637500in}{0.550000in}}{\pgfqpoint{3.850000in}{3.850000in}}%
\pgfusepath{clip}%
\pgfsetbuttcap%
\pgfsetroundjoin%
\definecolor{currentfill}{rgb}{0.904905,0.585527,0.000000}%
\pgfsetfillcolor{currentfill}%
\pgfsetfillopacity{0.800000}%
\pgfsetlinewidth{0.000000pt}%
\definecolor{currentstroke}{rgb}{0.000000,0.000000,0.000000}%
\pgfsetstrokecolor{currentstroke}%
\pgfsetdash{}{0pt}%
\pgfpathmoveto{\pgfqpoint{3.082134in}{3.120426in}}%
\pgfpathlineto{\pgfqpoint{3.012215in}{3.084216in}}%
\pgfpathlineto{\pgfqpoint{3.082134in}{3.118228in}}%
\pgfpathlineto{\pgfqpoint{3.152052in}{3.155761in}}%
\pgfpathlineto{\pgfqpoint{3.082134in}{3.120426in}}%
\pgfpathclose%
\pgfusepath{fill}%
\end{pgfscope}%
\begin{pgfscope}%
\pgfpathrectangle{\pgfqpoint{0.637500in}{0.550000in}}{\pgfqpoint{3.850000in}{3.850000in}}%
\pgfusepath{clip}%
\pgfsetbuttcap%
\pgfsetroundjoin%
\definecolor{currentfill}{rgb}{0.899206,0.581839,0.000000}%
\pgfsetfillcolor{currentfill}%
\pgfsetfillopacity{0.800000}%
\pgfsetlinewidth{0.000000pt}%
\definecolor{currentstroke}{rgb}{0.000000,0.000000,0.000000}%
\pgfsetstrokecolor{currentstroke}%
\pgfsetdash{}{0pt}%
\pgfpathmoveto{\pgfqpoint{1.613842in}{3.255894in}}%
\pgfpathlineto{\pgfqpoint{1.543923in}{3.293678in}}%
\pgfpathlineto{\pgfqpoint{1.613842in}{3.253696in}}%
\pgfpathlineto{\pgfqpoint{1.683761in}{3.216345in}}%
\pgfpathlineto{\pgfqpoint{1.613842in}{3.255894in}}%
\pgfpathclose%
\pgfusepath{fill}%
\end{pgfscope}%
\begin{pgfscope}%
\pgfpathrectangle{\pgfqpoint{0.637500in}{0.550000in}}{\pgfqpoint{3.850000in}{3.850000in}}%
\pgfusepath{clip}%
\pgfsetbuttcap%
\pgfsetroundjoin%
\definecolor{currentfill}{rgb}{0.914972,0.592040,0.000000}%
\pgfsetfillcolor{currentfill}%
\pgfsetfillopacity{0.800000}%
\pgfsetlinewidth{0.000000pt}%
\definecolor{currentstroke}{rgb}{0.000000,0.000000,0.000000}%
\pgfsetstrokecolor{currentstroke}%
\pgfsetdash{}{0pt}%
\pgfpathmoveto{\pgfqpoint{2.872378in}{3.019871in}}%
\pgfpathlineto{\pgfqpoint{2.802459in}{2.990786in}}%
\pgfpathlineto{\pgfqpoint{2.872378in}{3.017672in}}%
\pgfpathlineto{\pgfqpoint{2.942296in}{3.049776in}}%
\pgfpathlineto{\pgfqpoint{2.872378in}{3.019871in}}%
\pgfpathclose%
\pgfusepath{fill}%
\end{pgfscope}%
\begin{pgfscope}%
\pgfpathrectangle{\pgfqpoint{0.637500in}{0.550000in}}{\pgfqpoint{3.850000in}{3.850000in}}%
\pgfusepath{clip}%
\pgfsetbuttcap%
\pgfsetroundjoin%
\definecolor{currentfill}{rgb}{0.902018,0.583658,0.000000}%
\pgfsetfillcolor{currentfill}%
\pgfsetfillopacity{0.800000}%
\pgfsetlinewidth{0.000000pt}%
\definecolor{currentstroke}{rgb}{0.000000,0.000000,0.000000}%
\pgfsetstrokecolor{currentstroke}%
\pgfsetdash{}{0pt}%
\pgfpathmoveto{\pgfqpoint{1.823598in}{3.141452in}}%
\pgfpathlineto{\pgfqpoint{1.753679in}{3.177391in}}%
\pgfpathlineto{\pgfqpoint{1.823598in}{3.139254in}}%
\pgfpathlineto{\pgfqpoint{1.893516in}{3.104426in}}%
\pgfpathlineto{\pgfqpoint{1.823598in}{3.141452in}}%
\pgfpathclose%
\pgfusepath{fill}%
\end{pgfscope}%
\begin{pgfscope}%
\pgfpathrectangle{\pgfqpoint{0.637500in}{0.550000in}}{\pgfqpoint{3.850000in}{3.850000in}}%
\pgfusepath{clip}%
\pgfsetbuttcap%
\pgfsetroundjoin%
\definecolor{currentfill}{rgb}{0.908810,0.588053,0.000000}%
\pgfsetfillcolor{currentfill}%
\pgfsetfillopacity{0.800000}%
\pgfsetlinewidth{0.000000pt}%
\definecolor{currentstroke}{rgb}{0.000000,0.000000,0.000000}%
\pgfsetstrokecolor{currentstroke}%
\pgfsetdash{}{0pt}%
\pgfpathmoveto{\pgfqpoint{2.033354in}{3.037559in}}%
\pgfpathlineto{\pgfqpoint{1.963435in}{3.068896in}}%
\pgfpathlineto{\pgfqpoint{2.033354in}{3.035360in}}%
\pgfpathlineto{\pgfqpoint{2.103272in}{3.006630in}}%
\pgfpathlineto{\pgfqpoint{2.033354in}{3.037559in}}%
\pgfpathclose%
\pgfusepath{fill}%
\end{pgfscope}%
\begin{pgfscope}%
\pgfpathrectangle{\pgfqpoint{0.637500in}{0.550000in}}{\pgfqpoint{3.850000in}{3.850000in}}%
\pgfusepath{clip}%
\pgfsetbuttcap%
\pgfsetroundjoin%
\definecolor{currentfill}{rgb}{0.935716,0.605463,0.000000}%
\pgfsetfillcolor{currentfill}%
\pgfsetfillopacity{0.800000}%
\pgfsetlinewidth{0.000000pt}%
\definecolor{currentstroke}{rgb}{0.000000,0.000000,0.000000}%
\pgfsetstrokecolor{currentstroke}%
\pgfsetdash{}{0pt}%
\pgfpathmoveto{\pgfqpoint{2.522784in}{2.925070in}}%
\pgfpathlineto{\pgfqpoint{2.452866in}{2.922577in}}%
\pgfpathlineto{\pgfqpoint{2.522784in}{2.922871in}}%
\pgfpathlineto{\pgfqpoint{2.592703in}{2.931909in}}%
\pgfpathlineto{\pgfqpoint{2.522784in}{2.925070in}}%
\pgfpathclose%
\pgfusepath{fill}%
\end{pgfscope}%
\begin{pgfscope}%
\pgfpathrectangle{\pgfqpoint{0.637500in}{0.550000in}}{\pgfqpoint{3.850000in}{3.850000in}}%
\pgfusepath{clip}%
\pgfsetbuttcap%
\pgfsetroundjoin%
\definecolor{currentfill}{rgb}{0.931871,0.602975,0.000000}%
\pgfsetfillcolor{currentfill}%
\pgfsetfillopacity{0.800000}%
\pgfsetlinewidth{0.000000pt}%
\definecolor{currentstroke}{rgb}{0.000000,0.000000,0.000000}%
\pgfsetstrokecolor{currentstroke}%
\pgfsetdash{}{0pt}%
\pgfpathmoveto{\pgfqpoint{2.382947in}{2.928904in}}%
\pgfpathlineto{\pgfqpoint{2.313028in}{2.939373in}}%
\pgfpathlineto{\pgfqpoint{2.382947in}{2.926705in}}%
\pgfpathlineto{\pgfqpoint{2.452866in}{2.922577in}}%
\pgfpathlineto{\pgfqpoint{2.382947in}{2.928904in}}%
\pgfpathclose%
\pgfusepath{fill}%
\end{pgfscope}%
\begin{pgfscope}%
\pgfpathrectangle{\pgfqpoint{0.637500in}{0.550000in}}{\pgfqpoint{3.850000in}{3.850000in}}%
\pgfusepath{clip}%
\pgfsetbuttcap%
\pgfsetroundjoin%
\definecolor{currentfill}{rgb}{0.930057,0.601801,0.000000}%
\pgfsetfillcolor{currentfill}%
\pgfsetfillopacity{0.800000}%
\pgfsetlinewidth{0.000000pt}%
\definecolor{currentstroke}{rgb}{0.000000,0.000000,0.000000}%
\pgfsetstrokecolor{currentstroke}%
\pgfsetdash{}{0pt}%
\pgfpathmoveto{\pgfqpoint{2.662622in}{2.947078in}}%
\pgfpathlineto{\pgfqpoint{2.592703in}{2.931909in}}%
\pgfpathlineto{\pgfqpoint{2.662622in}{2.944879in}}%
\pgfpathlineto{\pgfqpoint{2.732540in}{2.965508in}}%
\pgfpathlineto{\pgfqpoint{2.662622in}{2.947078in}}%
\pgfpathclose%
\pgfusepath{fill}%
\end{pgfscope}%
\begin{pgfscope}%
\pgfpathrectangle{\pgfqpoint{0.637500in}{0.550000in}}{\pgfqpoint{3.850000in}{3.850000in}}%
\pgfusepath{clip}%
\pgfsetbuttcap%
\pgfsetroundjoin%
\definecolor{currentfill}{rgb}{0.898985,0.581696,0.000000}%
\pgfsetfillcolor{currentfill}%
\pgfsetfillopacity{0.800000}%
\pgfsetlinewidth{0.000000pt}%
\definecolor{currentstroke}{rgb}{0.000000,0.000000,0.000000}%
\pgfsetstrokecolor{currentstroke}%
\pgfsetdash{}{0pt}%
\pgfpathmoveto{\pgfqpoint{3.431727in}{3.311173in}}%
\pgfpathlineto{\pgfqpoint{3.361808in}{3.271048in}}%
\pgfpathlineto{\pgfqpoint{3.431727in}{3.308975in}}%
\pgfpathlineto{\pgfqpoint{3.501646in}{3.349373in}}%
\pgfpathlineto{\pgfqpoint{3.431727in}{3.311173in}}%
\pgfpathclose%
\pgfusepath{fill}%
\end{pgfscope}%
\begin{pgfscope}%
\pgfpathrectangle{\pgfqpoint{0.637500in}{0.550000in}}{\pgfqpoint{3.850000in}{3.850000in}}%
\pgfusepath{clip}%
\pgfsetbuttcap%
\pgfsetroundjoin%
\definecolor{currentfill}{rgb}{0.901452,0.583292,0.000000}%
\pgfsetfillcolor{currentfill}%
\pgfsetfillopacity{0.800000}%
\pgfsetlinewidth{0.000000pt}%
\definecolor{currentstroke}{rgb}{0.000000,0.000000,0.000000}%
\pgfsetstrokecolor{currentstroke}%
\pgfsetdash{}{0pt}%
\pgfpathmoveto{\pgfqpoint{3.221971in}{3.194272in}}%
\pgfpathlineto{\pgfqpoint{3.152052in}{3.155761in}}%
\pgfpathlineto{\pgfqpoint{3.221971in}{3.192073in}}%
\pgfpathlineto{\pgfqpoint{3.291890in}{3.231300in}}%
\pgfpathlineto{\pgfqpoint{3.221971in}{3.194272in}}%
\pgfpathclose%
\pgfusepath{fill}%
\end{pgfscope}%
\begin{pgfscope}%
\pgfpathrectangle{\pgfqpoint{0.637500in}{0.550000in}}{\pgfqpoint{3.850000in}{3.850000in}}%
\pgfusepath{clip}%
\pgfsetbuttcap%
\pgfsetroundjoin%
\definecolor{currentfill}{rgb}{0.921990,0.596582,0.000000}%
\pgfsetfillcolor{currentfill}%
\pgfsetfillopacity{0.800000}%
\pgfsetlinewidth{0.000000pt}%
\definecolor{currentstroke}{rgb}{0.000000,0.000000,0.000000}%
\pgfsetstrokecolor{currentstroke}%
\pgfsetdash{}{0pt}%
\pgfpathmoveto{\pgfqpoint{2.243110in}{2.957807in}}%
\pgfpathlineto{\pgfqpoint{2.173191in}{2.979041in}}%
\pgfpathlineto{\pgfqpoint{2.243110in}{2.955609in}}%
\pgfpathlineto{\pgfqpoint{2.313028in}{2.939373in}}%
\pgfpathlineto{\pgfqpoint{2.243110in}{2.957807in}}%
\pgfpathclose%
\pgfusepath{fill}%
\end{pgfscope}%
\begin{pgfscope}%
\pgfpathrectangle{\pgfqpoint{0.637500in}{0.550000in}}{\pgfqpoint{3.850000in}{3.850000in}}%
\pgfusepath{clip}%
\pgfsetbuttcap%
\pgfsetroundjoin%
\definecolor{currentfill}{rgb}{0.907506,0.587210,0.000000}%
\pgfsetfillcolor{currentfill}%
\pgfsetfillopacity{0.800000}%
\pgfsetlinewidth{0.000000pt}%
\definecolor{currentstroke}{rgb}{0.000000,0.000000,0.000000}%
\pgfsetstrokecolor{currentstroke}%
\pgfsetdash{}{0pt}%
\pgfpathmoveto{\pgfqpoint{3.012215in}{3.084216in}}%
\pgfpathlineto{\pgfqpoint{2.942296in}{3.049776in}}%
\pgfpathlineto{\pgfqpoint{3.012215in}{3.082017in}}%
\pgfpathlineto{\pgfqpoint{3.082134in}{3.118228in}}%
\pgfpathlineto{\pgfqpoint{3.012215in}{3.084216in}}%
\pgfpathclose%
\pgfusepath{fill}%
\end{pgfscope}%
\begin{pgfscope}%
\pgfpathrectangle{\pgfqpoint{0.637500in}{0.550000in}}{\pgfqpoint{3.850000in}{3.850000in}}%
\pgfusepath{clip}%
\pgfsetbuttcap%
\pgfsetroundjoin%
\definecolor{currentfill}{rgb}{0.899869,0.582268,0.000000}%
\pgfsetfillcolor{currentfill}%
\pgfsetfillopacity{0.800000}%
\pgfsetlinewidth{0.000000pt}%
\definecolor{currentstroke}{rgb}{0.000000,0.000000,0.000000}%
\pgfsetstrokecolor{currentstroke}%
\pgfsetdash{}{0pt}%
\pgfpathmoveto{\pgfqpoint{1.683761in}{3.216345in}}%
\pgfpathlineto{\pgfqpoint{1.613842in}{3.253696in}}%
\pgfpathlineto{\pgfqpoint{1.683761in}{3.214146in}}%
\pgfpathlineto{\pgfqpoint{1.753679in}{3.177391in}}%
\pgfpathlineto{\pgfqpoint{1.683761in}{3.216345in}}%
\pgfpathclose%
\pgfusepath{fill}%
\end{pgfscope}%
\begin{pgfscope}%
\pgfpathrectangle{\pgfqpoint{0.637500in}{0.550000in}}{\pgfqpoint{3.850000in}{3.850000in}}%
\pgfusepath{clip}%
\pgfsetbuttcap%
\pgfsetroundjoin%
\definecolor{currentfill}{rgb}{0.903689,0.584740,0.000000}%
\pgfsetfillcolor{currentfill}%
\pgfsetfillopacity{0.800000}%
\pgfsetlinewidth{0.000000pt}%
\definecolor{currentstroke}{rgb}{0.000000,0.000000,0.000000}%
\pgfsetstrokecolor{currentstroke}%
\pgfsetdash{}{0pt}%
\pgfpathmoveto{\pgfqpoint{1.893516in}{3.104426in}}%
\pgfpathlineto{\pgfqpoint{1.823598in}{3.139254in}}%
\pgfpathlineto{\pgfqpoint{1.893516in}{3.102227in}}%
\pgfpathlineto{\pgfqpoint{1.963435in}{3.068896in}}%
\pgfpathlineto{\pgfqpoint{1.893516in}{3.104426in}}%
\pgfpathclose%
\pgfusepath{fill}%
\end{pgfscope}%
\begin{pgfscope}%
\pgfpathrectangle{\pgfqpoint{0.637500in}{0.550000in}}{\pgfqpoint{3.850000in}{3.850000in}}%
\pgfusepath{clip}%
\pgfsetbuttcap%
\pgfsetroundjoin%
\definecolor{currentfill}{rgb}{0.919814,0.595174,0.000000}%
\pgfsetfillcolor{currentfill}%
\pgfsetfillopacity{0.800000}%
\pgfsetlinewidth{0.000000pt}%
\definecolor{currentstroke}{rgb}{0.000000,0.000000,0.000000}%
\pgfsetstrokecolor{currentstroke}%
\pgfsetdash{}{0pt}%
\pgfpathmoveto{\pgfqpoint{2.802459in}{2.990786in}}%
\pgfpathlineto{\pgfqpoint{2.732540in}{2.965508in}}%
\pgfpathlineto{\pgfqpoint{2.802459in}{2.988587in}}%
\pgfpathlineto{\pgfqpoint{2.872378in}{3.017672in}}%
\pgfpathlineto{\pgfqpoint{2.802459in}{2.990786in}}%
\pgfpathclose%
\pgfusepath{fill}%
\end{pgfscope}%
\begin{pgfscope}%
\pgfpathrectangle{\pgfqpoint{0.637500in}{0.550000in}}{\pgfqpoint{3.850000in}{3.850000in}}%
\pgfusepath{clip}%
\pgfsetbuttcap%
\pgfsetroundjoin%
\definecolor{currentfill}{rgb}{0.912477,0.590427,0.000000}%
\pgfsetfillcolor{currentfill}%
\pgfsetfillopacity{0.800000}%
\pgfsetlinewidth{0.000000pt}%
\definecolor{currentstroke}{rgb}{0.000000,0.000000,0.000000}%
\pgfsetstrokecolor{currentstroke}%
\pgfsetdash{}{0pt}%
\pgfpathmoveto{\pgfqpoint{2.103272in}{3.006630in}}%
\pgfpathlineto{\pgfqpoint{2.033354in}{3.035360in}}%
\pgfpathlineto{\pgfqpoint{2.103272in}{3.004432in}}%
\pgfpathlineto{\pgfqpoint{2.173191in}{2.979041in}}%
\pgfpathlineto{\pgfqpoint{2.103272in}{3.006630in}}%
\pgfpathclose%
\pgfusepath{fill}%
\end{pgfscope}%
\begin{pgfscope}%
\pgfpathrectangle{\pgfqpoint{0.637500in}{0.550000in}}{\pgfqpoint{3.850000in}{3.850000in}}%
\pgfusepath{clip}%
\pgfsetbuttcap%
\pgfsetroundjoin%
\definecolor{currentfill}{rgb}{0.899565,0.582072,0.000000}%
\pgfsetfillcolor{currentfill}%
\pgfsetfillopacity{0.800000}%
\pgfsetlinewidth{0.000000pt}%
\definecolor{currentstroke}{rgb}{0.000000,0.000000,0.000000}%
\pgfsetstrokecolor{currentstroke}%
\pgfsetdash{}{0pt}%
\pgfpathmoveto{\pgfqpoint{3.361808in}{3.271048in}}%
\pgfpathlineto{\pgfqpoint{3.291890in}{3.231300in}}%
\pgfpathlineto{\pgfqpoint{3.361808in}{3.268849in}}%
\pgfpathlineto{\pgfqpoint{3.431727in}{3.308975in}}%
\pgfpathlineto{\pgfqpoint{3.361808in}{3.271048in}}%
\pgfpathclose%
\pgfusepath{fill}%
\end{pgfscope}%
\begin{pgfscope}%
\pgfpathrectangle{\pgfqpoint{0.637500in}{0.550000in}}{\pgfqpoint{3.850000in}{3.850000in}}%
\pgfusepath{clip}%
\pgfsetbuttcap%
\pgfsetroundjoin%
\definecolor{currentfill}{rgb}{0.902928,0.584248,0.000000}%
\pgfsetfillcolor{currentfill}%
\pgfsetfillopacity{0.800000}%
\pgfsetlinewidth{0.000000pt}%
\definecolor{currentstroke}{rgb}{0.000000,0.000000,0.000000}%
\pgfsetstrokecolor{currentstroke}%
\pgfsetdash{}{0pt}%
\pgfpathmoveto{\pgfqpoint{3.152052in}{3.155761in}}%
\pgfpathlineto{\pgfqpoint{3.082134in}{3.118228in}}%
\pgfpathlineto{\pgfqpoint{3.152052in}{3.153562in}}%
\pgfpathlineto{\pgfqpoint{3.221971in}{3.192073in}}%
\pgfpathlineto{\pgfqpoint{3.152052in}{3.155761in}}%
\pgfpathclose%
\pgfusepath{fill}%
\end{pgfscope}%
\begin{pgfscope}%
\pgfpathrectangle{\pgfqpoint{0.637500in}{0.550000in}}{\pgfqpoint{3.850000in}{3.850000in}}%
\pgfusepath{clip}%
\pgfsetbuttcap%
\pgfsetroundjoin%
\definecolor{currentfill}{rgb}{0.934964,0.604977,0.000000}%
\pgfsetfillcolor{currentfill}%
\pgfsetfillopacity{0.800000}%
\pgfsetlinewidth{0.000000pt}%
\definecolor{currentstroke}{rgb}{0.000000,0.000000,0.000000}%
\pgfsetstrokecolor{currentstroke}%
\pgfsetdash{}{0pt}%
\pgfpathmoveto{\pgfqpoint{2.452866in}{2.922577in}}%
\pgfpathlineto{\pgfqpoint{2.382947in}{2.926705in}}%
\pgfpathlineto{\pgfqpoint{2.452866in}{2.920378in}}%
\pgfpathlineto{\pgfqpoint{2.522784in}{2.922871in}}%
\pgfpathlineto{\pgfqpoint{2.452866in}{2.922577in}}%
\pgfpathclose%
\pgfusepath{fill}%
\end{pgfscope}%
\begin{pgfscope}%
\pgfpathrectangle{\pgfqpoint{0.637500in}{0.550000in}}{\pgfqpoint{3.850000in}{3.850000in}}%
\pgfusepath{clip}%
\pgfsetbuttcap%
\pgfsetroundjoin%
\definecolor{currentfill}{rgb}{0.910843,0.589369,0.000000}%
\pgfsetfillcolor{currentfill}%
\pgfsetfillopacity{0.800000}%
\pgfsetlinewidth{0.000000pt}%
\definecolor{currentstroke}{rgb}{0.000000,0.000000,0.000000}%
\pgfsetstrokecolor{currentstroke}%
\pgfsetdash{}{0pt}%
\pgfpathmoveto{\pgfqpoint{2.942296in}{3.049776in}}%
\pgfpathlineto{\pgfqpoint{2.872378in}{3.017672in}}%
\pgfpathlineto{\pgfqpoint{2.942296in}{3.047577in}}%
\pgfpathlineto{\pgfqpoint{3.012215in}{3.082017in}}%
\pgfpathlineto{\pgfqpoint{2.942296in}{3.049776in}}%
\pgfpathclose%
\pgfusepath{fill}%
\end{pgfscope}%
\begin{pgfscope}%
\pgfpathrectangle{\pgfqpoint{0.637500in}{0.550000in}}{\pgfqpoint{3.850000in}{3.850000in}}%
\pgfusepath{clip}%
\pgfsetbuttcap%
\pgfsetroundjoin%
\definecolor{currentfill}{rgb}{0.933916,0.604299,0.000000}%
\pgfsetfillcolor{currentfill}%
\pgfsetfillopacity{0.800000}%
\pgfsetlinewidth{0.000000pt}%
\definecolor{currentstroke}{rgb}{0.000000,0.000000,0.000000}%
\pgfsetstrokecolor{currentstroke}%
\pgfsetdash{}{0pt}%
\pgfpathmoveto{\pgfqpoint{2.592703in}{2.931909in}}%
\pgfpathlineto{\pgfqpoint{2.522784in}{2.922871in}}%
\pgfpathlineto{\pgfqpoint{2.592703in}{2.929710in}}%
\pgfpathlineto{\pgfqpoint{2.662622in}{2.944879in}}%
\pgfpathlineto{\pgfqpoint{2.592703in}{2.931909in}}%
\pgfpathclose%
\pgfusepath{fill}%
\end{pgfscope}%
\begin{pgfscope}%
\pgfpathrectangle{\pgfqpoint{0.637500in}{0.550000in}}{\pgfqpoint{3.850000in}{3.850000in}}%
\pgfusepath{clip}%
\pgfsetbuttcap%
\pgfsetroundjoin%
\definecolor{currentfill}{rgb}{0.900779,0.582857,0.000000}%
\pgfsetfillcolor{currentfill}%
\pgfsetfillopacity{0.800000}%
\pgfsetlinewidth{0.000000pt}%
\definecolor{currentstroke}{rgb}{0.000000,0.000000,0.000000}%
\pgfsetstrokecolor{currentstroke}%
\pgfsetdash{}{0pt}%
\pgfpathmoveto{\pgfqpoint{1.753679in}{3.177391in}}%
\pgfpathlineto{\pgfqpoint{1.683761in}{3.214146in}}%
\pgfpathlineto{\pgfqpoint{1.753679in}{3.175192in}}%
\pgfpathlineto{\pgfqpoint{1.823598in}{3.139254in}}%
\pgfpathlineto{\pgfqpoint{1.753679in}{3.177391in}}%
\pgfpathclose%
\pgfusepath{fill}%
\end{pgfscope}%
\begin{pgfscope}%
\pgfpathrectangle{\pgfqpoint{0.637500in}{0.550000in}}{\pgfqpoint{3.850000in}{3.850000in}}%
\pgfusepath{clip}%
\pgfsetbuttcap%
\pgfsetroundjoin%
\definecolor{currentfill}{rgb}{0.927227,0.599970,0.000000}%
\pgfsetfillcolor{currentfill}%
\pgfsetfillopacity{0.800000}%
\pgfsetlinewidth{0.000000pt}%
\definecolor{currentstroke}{rgb}{0.000000,0.000000,0.000000}%
\pgfsetstrokecolor{currentstroke}%
\pgfsetdash{}{0pt}%
\pgfpathmoveto{\pgfqpoint{2.313028in}{2.939373in}}%
\pgfpathlineto{\pgfqpoint{2.243110in}{2.955609in}}%
\pgfpathlineto{\pgfqpoint{2.313028in}{2.937174in}}%
\pgfpathlineto{\pgfqpoint{2.382947in}{2.926705in}}%
\pgfpathlineto{\pgfqpoint{2.313028in}{2.939373in}}%
\pgfpathclose%
\pgfusepath{fill}%
\end{pgfscope}%
\begin{pgfscope}%
\pgfpathrectangle{\pgfqpoint{0.637500in}{0.550000in}}{\pgfqpoint{3.850000in}{3.850000in}}%
\pgfusepath{clip}%
\pgfsetbuttcap%
\pgfsetroundjoin%
\definecolor{currentfill}{rgb}{0.905912,0.586178,0.000000}%
\pgfsetfillcolor{currentfill}%
\pgfsetfillopacity{0.800000}%
\pgfsetlinewidth{0.000000pt}%
\definecolor{currentstroke}{rgb}{0.000000,0.000000,0.000000}%
\pgfsetstrokecolor{currentstroke}%
\pgfsetdash{}{0pt}%
\pgfpathmoveto{\pgfqpoint{1.963435in}{3.068896in}}%
\pgfpathlineto{\pgfqpoint{1.893516in}{3.102227in}}%
\pgfpathlineto{\pgfqpoint{1.963435in}{3.066698in}}%
\pgfpathlineto{\pgfqpoint{2.033354in}{3.035360in}}%
\pgfpathlineto{\pgfqpoint{1.963435in}{3.068896in}}%
\pgfpathclose%
\pgfusepath{fill}%
\end{pgfscope}%
\begin{pgfscope}%
\pgfpathrectangle{\pgfqpoint{0.637500in}{0.550000in}}{\pgfqpoint{3.850000in}{3.850000in}}%
\pgfusepath{clip}%
\pgfsetbuttcap%
\pgfsetroundjoin%
\definecolor{currentfill}{rgb}{0.925053,0.598564,0.000000}%
\pgfsetfillcolor{currentfill}%
\pgfsetfillopacity{0.800000}%
\pgfsetlinewidth{0.000000pt}%
\definecolor{currentstroke}{rgb}{0.000000,0.000000,0.000000}%
\pgfsetstrokecolor{currentstroke}%
\pgfsetdash{}{0pt}%
\pgfpathmoveto{\pgfqpoint{2.732540in}{2.965508in}}%
\pgfpathlineto{\pgfqpoint{2.662622in}{2.944879in}}%
\pgfpathlineto{\pgfqpoint{2.732540in}{2.963310in}}%
\pgfpathlineto{\pgfqpoint{2.802459in}{2.988587in}}%
\pgfpathlineto{\pgfqpoint{2.732540in}{2.965508in}}%
\pgfpathclose%
\pgfusepath{fill}%
\end{pgfscope}%
\begin{pgfscope}%
\pgfpathrectangle{\pgfqpoint{0.637500in}{0.550000in}}{\pgfqpoint{3.850000in}{3.850000in}}%
\pgfusepath{clip}%
\pgfsetbuttcap%
\pgfsetroundjoin%
\definecolor{currentfill}{rgb}{0.916926,0.593305,0.000000}%
\pgfsetfillcolor{currentfill}%
\pgfsetfillopacity{0.800000}%
\pgfsetlinewidth{0.000000pt}%
\definecolor{currentstroke}{rgb}{0.000000,0.000000,0.000000}%
\pgfsetstrokecolor{currentstroke}%
\pgfsetdash{}{0pt}%
\pgfpathmoveto{\pgfqpoint{2.173191in}{2.979041in}}%
\pgfpathlineto{\pgfqpoint{2.103272in}{3.004432in}}%
\pgfpathlineto{\pgfqpoint{2.173191in}{2.976842in}}%
\pgfpathlineto{\pgfqpoint{2.243110in}{2.955609in}}%
\pgfpathlineto{\pgfqpoint{2.173191in}{2.979041in}}%
\pgfpathclose%
\pgfusepath{fill}%
\end{pgfscope}%
\begin{pgfscope}%
\pgfpathrectangle{\pgfqpoint{0.637500in}{0.550000in}}{\pgfqpoint{3.850000in}{3.850000in}}%
\pgfusepath{clip}%
\pgfsetbuttcap%
\pgfsetroundjoin%
\definecolor{currentfill}{rgb}{0.900362,0.582587,0.000000}%
\pgfsetfillcolor{currentfill}%
\pgfsetfillopacity{0.800000}%
\pgfsetlinewidth{0.000000pt}%
\definecolor{currentstroke}{rgb}{0.000000,0.000000,0.000000}%
\pgfsetstrokecolor{currentstroke}%
\pgfsetdash{}{0pt}%
\pgfpathmoveto{\pgfqpoint{3.291890in}{3.231300in}}%
\pgfpathlineto{\pgfqpoint{3.221971in}{3.192073in}}%
\pgfpathlineto{\pgfqpoint{3.291890in}{3.229101in}}%
\pgfpathlineto{\pgfqpoint{3.361808in}{3.268849in}}%
\pgfpathlineto{\pgfqpoint{3.291890in}{3.231300in}}%
\pgfpathclose%
\pgfusepath{fill}%
\end{pgfscope}%
\begin{pgfscope}%
\pgfpathrectangle{\pgfqpoint{0.637500in}{0.550000in}}{\pgfqpoint{3.850000in}{3.850000in}}%
\pgfusepath{clip}%
\pgfsetbuttcap%
\pgfsetroundjoin%
\definecolor{currentfill}{rgb}{0.904905,0.585527,0.000000}%
\pgfsetfillcolor{currentfill}%
\pgfsetfillopacity{0.800000}%
\pgfsetlinewidth{0.000000pt}%
\definecolor{currentstroke}{rgb}{0.000000,0.000000,0.000000}%
\pgfsetstrokecolor{currentstroke}%
\pgfsetdash{}{0pt}%
\pgfpathmoveto{\pgfqpoint{3.082134in}{3.118228in}}%
\pgfpathlineto{\pgfqpoint{3.012215in}{3.082017in}}%
\pgfpathlineto{\pgfqpoint{3.082134in}{3.116029in}}%
\pgfpathlineto{\pgfqpoint{3.152052in}{3.153562in}}%
\pgfpathlineto{\pgfqpoint{3.082134in}{3.118228in}}%
\pgfpathclose%
\pgfusepath{fill}%
\end{pgfscope}%
\begin{pgfscope}%
\pgfpathrectangle{\pgfqpoint{0.637500in}{0.550000in}}{\pgfqpoint{3.850000in}{3.850000in}}%
\pgfusepath{clip}%
\pgfsetbuttcap%
\pgfsetroundjoin%
\definecolor{currentfill}{rgb}{0.914972,0.592040,0.000000}%
\pgfsetfillcolor{currentfill}%
\pgfsetfillopacity{0.800000}%
\pgfsetlinewidth{0.000000pt}%
\definecolor{currentstroke}{rgb}{0.000000,0.000000,0.000000}%
\pgfsetstrokecolor{currentstroke}%
\pgfsetdash{}{0pt}%
\pgfpathmoveto{\pgfqpoint{2.872378in}{3.017672in}}%
\pgfpathlineto{\pgfqpoint{2.802459in}{2.988587in}}%
\pgfpathlineto{\pgfqpoint{2.872378in}{3.015473in}}%
\pgfpathlineto{\pgfqpoint{2.942296in}{3.047577in}}%
\pgfpathlineto{\pgfqpoint{2.872378in}{3.017672in}}%
\pgfpathclose%
\pgfusepath{fill}%
\end{pgfscope}%
\begin{pgfscope}%
\pgfpathrectangle{\pgfqpoint{0.637500in}{0.550000in}}{\pgfqpoint{3.850000in}{3.850000in}}%
\pgfusepath{clip}%
\pgfsetbuttcap%
\pgfsetroundjoin%
\definecolor{currentfill}{rgb}{0.902018,0.583658,0.000000}%
\pgfsetfillcolor{currentfill}%
\pgfsetfillopacity{0.800000}%
\pgfsetlinewidth{0.000000pt}%
\definecolor{currentstroke}{rgb}{0.000000,0.000000,0.000000}%
\pgfsetstrokecolor{currentstroke}%
\pgfsetdash{}{0pt}%
\pgfpathmoveto{\pgfqpoint{1.823598in}{3.139254in}}%
\pgfpathlineto{\pgfqpoint{1.753679in}{3.175192in}}%
\pgfpathlineto{\pgfqpoint{1.823598in}{3.137055in}}%
\pgfpathlineto{\pgfqpoint{1.893516in}{3.102227in}}%
\pgfpathlineto{\pgfqpoint{1.823598in}{3.139254in}}%
\pgfpathclose%
\pgfusepath{fill}%
\end{pgfscope}%
\begin{pgfscope}%
\pgfpathrectangle{\pgfqpoint{0.637500in}{0.550000in}}{\pgfqpoint{3.850000in}{3.850000in}}%
\pgfusepath{clip}%
\pgfsetbuttcap%
\pgfsetroundjoin%
\definecolor{currentfill}{rgb}{0.908810,0.588053,0.000000}%
\pgfsetfillcolor{currentfill}%
\pgfsetfillopacity{0.800000}%
\pgfsetlinewidth{0.000000pt}%
\definecolor{currentstroke}{rgb}{0.000000,0.000000,0.000000}%
\pgfsetstrokecolor{currentstroke}%
\pgfsetdash{}{0pt}%
\pgfpathmoveto{\pgfqpoint{2.033354in}{3.035360in}}%
\pgfpathlineto{\pgfqpoint{1.963435in}{3.066698in}}%
\pgfpathlineto{\pgfqpoint{2.033354in}{3.033161in}}%
\pgfpathlineto{\pgfqpoint{2.103272in}{3.004432in}}%
\pgfpathlineto{\pgfqpoint{2.033354in}{3.035360in}}%
\pgfpathclose%
\pgfusepath{fill}%
\end{pgfscope}%
\begin{pgfscope}%
\pgfpathrectangle{\pgfqpoint{0.637500in}{0.550000in}}{\pgfqpoint{3.850000in}{3.850000in}}%
\pgfusepath{clip}%
\pgfsetbuttcap%
\pgfsetroundjoin%
\definecolor{currentfill}{rgb}{0.935716,0.605463,0.000000}%
\pgfsetfillcolor{currentfill}%
\pgfsetfillopacity{0.800000}%
\pgfsetlinewidth{0.000000pt}%
\definecolor{currentstroke}{rgb}{0.000000,0.000000,0.000000}%
\pgfsetstrokecolor{currentstroke}%
\pgfsetdash{}{0pt}%
\pgfpathmoveto{\pgfqpoint{2.522784in}{2.922871in}}%
\pgfpathlineto{\pgfqpoint{2.452866in}{2.920378in}}%
\pgfpathlineto{\pgfqpoint{2.522784in}{2.920672in}}%
\pgfpathlineto{\pgfqpoint{2.592703in}{2.929710in}}%
\pgfpathlineto{\pgfqpoint{2.522784in}{2.922871in}}%
\pgfpathclose%
\pgfusepath{fill}%
\end{pgfscope}%
\begin{pgfscope}%
\pgfpathrectangle{\pgfqpoint{0.637500in}{0.550000in}}{\pgfqpoint{3.850000in}{3.850000in}}%
\pgfusepath{clip}%
\pgfsetbuttcap%
\pgfsetroundjoin%
\definecolor{currentfill}{rgb}{0.931871,0.602975,0.000000}%
\pgfsetfillcolor{currentfill}%
\pgfsetfillopacity{0.800000}%
\pgfsetlinewidth{0.000000pt}%
\definecolor{currentstroke}{rgb}{0.000000,0.000000,0.000000}%
\pgfsetstrokecolor{currentstroke}%
\pgfsetdash{}{0pt}%
\pgfpathmoveto{\pgfqpoint{2.382947in}{2.926705in}}%
\pgfpathlineto{\pgfqpoint{2.313028in}{2.937174in}}%
\pgfpathlineto{\pgfqpoint{2.382947in}{2.924506in}}%
\pgfpathlineto{\pgfqpoint{2.452866in}{2.920378in}}%
\pgfpathlineto{\pgfqpoint{2.382947in}{2.926705in}}%
\pgfpathclose%
\pgfusepath{fill}%
\end{pgfscope}%
\begin{pgfscope}%
\pgfpathrectangle{\pgfqpoint{0.637500in}{0.550000in}}{\pgfqpoint{3.850000in}{3.850000in}}%
\pgfusepath{clip}%
\pgfsetbuttcap%
\pgfsetroundjoin%
\definecolor{currentfill}{rgb}{0.930057,0.601801,0.000000}%
\pgfsetfillcolor{currentfill}%
\pgfsetfillopacity{0.800000}%
\pgfsetlinewidth{0.000000pt}%
\definecolor{currentstroke}{rgb}{0.000000,0.000000,0.000000}%
\pgfsetstrokecolor{currentstroke}%
\pgfsetdash{}{0pt}%
\pgfpathmoveto{\pgfqpoint{2.662622in}{2.944879in}}%
\pgfpathlineto{\pgfqpoint{2.592703in}{2.929710in}}%
\pgfpathlineto{\pgfqpoint{2.662622in}{2.942680in}}%
\pgfpathlineto{\pgfqpoint{2.732540in}{2.963310in}}%
\pgfpathlineto{\pgfqpoint{2.662622in}{2.944879in}}%
\pgfpathclose%
\pgfusepath{fill}%
\end{pgfscope}%
\begin{pgfscope}%
\pgfpathrectangle{\pgfqpoint{0.637500in}{0.550000in}}{\pgfqpoint{3.850000in}{3.850000in}}%
\pgfusepath{clip}%
\pgfsetbuttcap%
\pgfsetroundjoin%
\definecolor{currentfill}{rgb}{0.901452,0.583292,0.000000}%
\pgfsetfillcolor{currentfill}%
\pgfsetfillopacity{0.800000}%
\pgfsetlinewidth{0.000000pt}%
\definecolor{currentstroke}{rgb}{0.000000,0.000000,0.000000}%
\pgfsetstrokecolor{currentstroke}%
\pgfsetdash{}{0pt}%
\pgfpathmoveto{\pgfqpoint{3.221971in}{3.192073in}}%
\pgfpathlineto{\pgfqpoint{3.152052in}{3.153562in}}%
\pgfpathlineto{\pgfqpoint{3.221971in}{3.189874in}}%
\pgfpathlineto{\pgfqpoint{3.291890in}{3.229101in}}%
\pgfpathlineto{\pgfqpoint{3.221971in}{3.192073in}}%
\pgfpathclose%
\pgfusepath{fill}%
\end{pgfscope}%
\begin{pgfscope}%
\pgfpathrectangle{\pgfqpoint{0.637500in}{0.550000in}}{\pgfqpoint{3.850000in}{3.850000in}}%
\pgfusepath{clip}%
\pgfsetbuttcap%
\pgfsetroundjoin%
\definecolor{currentfill}{rgb}{0.921990,0.596582,0.000000}%
\pgfsetfillcolor{currentfill}%
\pgfsetfillopacity{0.800000}%
\pgfsetlinewidth{0.000000pt}%
\definecolor{currentstroke}{rgb}{0.000000,0.000000,0.000000}%
\pgfsetstrokecolor{currentstroke}%
\pgfsetdash{}{0pt}%
\pgfpathmoveto{\pgfqpoint{2.243110in}{2.955609in}}%
\pgfpathlineto{\pgfqpoint{2.173191in}{2.976842in}}%
\pgfpathlineto{\pgfqpoint{2.243110in}{2.953410in}}%
\pgfpathlineto{\pgfqpoint{2.313028in}{2.937174in}}%
\pgfpathlineto{\pgfqpoint{2.243110in}{2.955609in}}%
\pgfpathclose%
\pgfusepath{fill}%
\end{pgfscope}%
\begin{pgfscope}%
\pgfpathrectangle{\pgfqpoint{0.637500in}{0.550000in}}{\pgfqpoint{3.850000in}{3.850000in}}%
\pgfusepath{clip}%
\pgfsetbuttcap%
\pgfsetroundjoin%
\definecolor{currentfill}{rgb}{0.907506,0.587210,0.000000}%
\pgfsetfillcolor{currentfill}%
\pgfsetfillopacity{0.800000}%
\pgfsetlinewidth{0.000000pt}%
\definecolor{currentstroke}{rgb}{0.000000,0.000000,0.000000}%
\pgfsetstrokecolor{currentstroke}%
\pgfsetdash{}{0pt}%
\pgfpathmoveto{\pgfqpoint{3.012215in}{3.082017in}}%
\pgfpathlineto{\pgfqpoint{2.942296in}{3.047577in}}%
\pgfpathlineto{\pgfqpoint{3.012215in}{3.079818in}}%
\pgfpathlineto{\pgfqpoint{3.082134in}{3.116029in}}%
\pgfpathlineto{\pgfqpoint{3.012215in}{3.082017in}}%
\pgfpathclose%
\pgfusepath{fill}%
\end{pgfscope}%
\begin{pgfscope}%
\pgfpathrectangle{\pgfqpoint{0.637500in}{0.550000in}}{\pgfqpoint{3.850000in}{3.850000in}}%
\pgfusepath{clip}%
\pgfsetbuttcap%
\pgfsetroundjoin%
\definecolor{currentfill}{rgb}{0.903689,0.584740,0.000000}%
\pgfsetfillcolor{currentfill}%
\pgfsetfillopacity{0.800000}%
\pgfsetlinewidth{0.000000pt}%
\definecolor{currentstroke}{rgb}{0.000000,0.000000,0.000000}%
\pgfsetstrokecolor{currentstroke}%
\pgfsetdash{}{0pt}%
\pgfpathmoveto{\pgfqpoint{1.893516in}{3.102227in}}%
\pgfpathlineto{\pgfqpoint{1.823598in}{3.137055in}}%
\pgfpathlineto{\pgfqpoint{1.893516in}{3.100028in}}%
\pgfpathlineto{\pgfqpoint{1.963435in}{3.066698in}}%
\pgfpathlineto{\pgfqpoint{1.893516in}{3.102227in}}%
\pgfpathclose%
\pgfusepath{fill}%
\end{pgfscope}%
\begin{pgfscope}%
\pgfpathrectangle{\pgfqpoint{0.637500in}{0.550000in}}{\pgfqpoint{3.850000in}{3.850000in}}%
\pgfusepath{clip}%
\pgfsetbuttcap%
\pgfsetroundjoin%
\definecolor{currentfill}{rgb}{0.919814,0.595174,0.000000}%
\pgfsetfillcolor{currentfill}%
\pgfsetfillopacity{0.800000}%
\pgfsetlinewidth{0.000000pt}%
\definecolor{currentstroke}{rgb}{0.000000,0.000000,0.000000}%
\pgfsetstrokecolor{currentstroke}%
\pgfsetdash{}{0pt}%
\pgfpathmoveto{\pgfqpoint{2.802459in}{2.988587in}}%
\pgfpathlineto{\pgfqpoint{2.732540in}{2.963310in}}%
\pgfpathlineto{\pgfqpoint{2.802459in}{2.986389in}}%
\pgfpathlineto{\pgfqpoint{2.872378in}{3.015473in}}%
\pgfpathlineto{\pgfqpoint{2.802459in}{2.988587in}}%
\pgfpathclose%
\pgfusepath{fill}%
\end{pgfscope}%
\begin{pgfscope}%
\pgfpathrectangle{\pgfqpoint{0.637500in}{0.550000in}}{\pgfqpoint{3.850000in}{3.850000in}}%
\pgfusepath{clip}%
\pgfsetbuttcap%
\pgfsetroundjoin%
\definecolor{currentfill}{rgb}{0.912477,0.590427,0.000000}%
\pgfsetfillcolor{currentfill}%
\pgfsetfillopacity{0.800000}%
\pgfsetlinewidth{0.000000pt}%
\definecolor{currentstroke}{rgb}{0.000000,0.000000,0.000000}%
\pgfsetstrokecolor{currentstroke}%
\pgfsetdash{}{0pt}%
\pgfpathmoveto{\pgfqpoint{2.103272in}{3.004432in}}%
\pgfpathlineto{\pgfqpoint{2.033354in}{3.033161in}}%
\pgfpathlineto{\pgfqpoint{2.103272in}{3.002233in}}%
\pgfpathlineto{\pgfqpoint{2.173191in}{2.976842in}}%
\pgfpathlineto{\pgfqpoint{2.103272in}{3.004432in}}%
\pgfpathclose%
\pgfusepath{fill}%
\end{pgfscope}%
\begin{pgfscope}%
\pgfpathrectangle{\pgfqpoint{0.637500in}{0.550000in}}{\pgfqpoint{3.850000in}{3.850000in}}%
\pgfusepath{clip}%
\pgfsetbuttcap%
\pgfsetroundjoin%
\definecolor{currentfill}{rgb}{0.902928,0.584248,0.000000}%
\pgfsetfillcolor{currentfill}%
\pgfsetfillopacity{0.800000}%
\pgfsetlinewidth{0.000000pt}%
\definecolor{currentstroke}{rgb}{0.000000,0.000000,0.000000}%
\pgfsetstrokecolor{currentstroke}%
\pgfsetdash{}{0pt}%
\pgfpathmoveto{\pgfqpoint{3.152052in}{3.153562in}}%
\pgfpathlineto{\pgfqpoint{3.082134in}{3.116029in}}%
\pgfpathlineto{\pgfqpoint{3.152052in}{3.151364in}}%
\pgfpathlineto{\pgfqpoint{3.221971in}{3.189874in}}%
\pgfpathlineto{\pgfqpoint{3.152052in}{3.153562in}}%
\pgfpathclose%
\pgfusepath{fill}%
\end{pgfscope}%
\begin{pgfscope}%
\pgfpathrectangle{\pgfqpoint{0.637500in}{0.550000in}}{\pgfqpoint{3.850000in}{3.850000in}}%
\pgfusepath{clip}%
\pgfsetbuttcap%
\pgfsetroundjoin%
\definecolor{currentfill}{rgb}{0.934964,0.604977,0.000000}%
\pgfsetfillcolor{currentfill}%
\pgfsetfillopacity{0.800000}%
\pgfsetlinewidth{0.000000pt}%
\definecolor{currentstroke}{rgb}{0.000000,0.000000,0.000000}%
\pgfsetstrokecolor{currentstroke}%
\pgfsetdash{}{0pt}%
\pgfpathmoveto{\pgfqpoint{2.452866in}{2.920378in}}%
\pgfpathlineto{\pgfqpoint{2.382947in}{2.924506in}}%
\pgfpathlineto{\pgfqpoint{2.452866in}{2.918180in}}%
\pgfpathlineto{\pgfqpoint{2.522784in}{2.920672in}}%
\pgfpathlineto{\pgfqpoint{2.452866in}{2.920378in}}%
\pgfpathclose%
\pgfusepath{fill}%
\end{pgfscope}%
\begin{pgfscope}%
\pgfpathrectangle{\pgfqpoint{0.637500in}{0.550000in}}{\pgfqpoint{3.850000in}{3.850000in}}%
\pgfusepath{clip}%
\pgfsetbuttcap%
\pgfsetroundjoin%
\definecolor{currentfill}{rgb}{0.910843,0.589369,0.000000}%
\pgfsetfillcolor{currentfill}%
\pgfsetfillopacity{0.800000}%
\pgfsetlinewidth{0.000000pt}%
\definecolor{currentstroke}{rgb}{0.000000,0.000000,0.000000}%
\pgfsetstrokecolor{currentstroke}%
\pgfsetdash{}{0pt}%
\pgfpathmoveto{\pgfqpoint{2.942296in}{3.047577in}}%
\pgfpathlineto{\pgfqpoint{2.872378in}{3.015473in}}%
\pgfpathlineto{\pgfqpoint{2.942296in}{3.045378in}}%
\pgfpathlineto{\pgfqpoint{3.012215in}{3.079818in}}%
\pgfpathlineto{\pgfqpoint{2.942296in}{3.047577in}}%
\pgfpathclose%
\pgfusepath{fill}%
\end{pgfscope}%
\begin{pgfscope}%
\pgfpathrectangle{\pgfqpoint{0.637500in}{0.550000in}}{\pgfqpoint{3.850000in}{3.850000in}}%
\pgfusepath{clip}%
\pgfsetbuttcap%
\pgfsetroundjoin%
\definecolor{currentfill}{rgb}{0.933916,0.604299,0.000000}%
\pgfsetfillcolor{currentfill}%
\pgfsetfillopacity{0.800000}%
\pgfsetlinewidth{0.000000pt}%
\definecolor{currentstroke}{rgb}{0.000000,0.000000,0.000000}%
\pgfsetstrokecolor{currentstroke}%
\pgfsetdash{}{0pt}%
\pgfpathmoveto{\pgfqpoint{2.592703in}{2.929710in}}%
\pgfpathlineto{\pgfqpoint{2.522784in}{2.920672in}}%
\pgfpathlineto{\pgfqpoint{2.592703in}{2.927511in}}%
\pgfpathlineto{\pgfqpoint{2.662622in}{2.942680in}}%
\pgfpathlineto{\pgfqpoint{2.592703in}{2.929710in}}%
\pgfpathclose%
\pgfusepath{fill}%
\end{pgfscope}%
\begin{pgfscope}%
\pgfpathrectangle{\pgfqpoint{0.637500in}{0.550000in}}{\pgfqpoint{3.850000in}{3.850000in}}%
\pgfusepath{clip}%
\pgfsetbuttcap%
\pgfsetroundjoin%
\definecolor{currentfill}{rgb}{0.927227,0.599970,0.000000}%
\pgfsetfillcolor{currentfill}%
\pgfsetfillopacity{0.800000}%
\pgfsetlinewidth{0.000000pt}%
\definecolor{currentstroke}{rgb}{0.000000,0.000000,0.000000}%
\pgfsetstrokecolor{currentstroke}%
\pgfsetdash{}{0pt}%
\pgfpathmoveto{\pgfqpoint{2.313028in}{2.937174in}}%
\pgfpathlineto{\pgfqpoint{2.243110in}{2.953410in}}%
\pgfpathlineto{\pgfqpoint{2.313028in}{2.934976in}}%
\pgfpathlineto{\pgfqpoint{2.382947in}{2.924506in}}%
\pgfpathlineto{\pgfqpoint{2.313028in}{2.937174in}}%
\pgfpathclose%
\pgfusepath{fill}%
\end{pgfscope}%
\begin{pgfscope}%
\pgfpathrectangle{\pgfqpoint{0.637500in}{0.550000in}}{\pgfqpoint{3.850000in}{3.850000in}}%
\pgfusepath{clip}%
\pgfsetbuttcap%
\pgfsetroundjoin%
\definecolor{currentfill}{rgb}{0.905912,0.586178,0.000000}%
\pgfsetfillcolor{currentfill}%
\pgfsetfillopacity{0.800000}%
\pgfsetlinewidth{0.000000pt}%
\definecolor{currentstroke}{rgb}{0.000000,0.000000,0.000000}%
\pgfsetstrokecolor{currentstroke}%
\pgfsetdash{}{0pt}%
\pgfpathmoveto{\pgfqpoint{1.963435in}{3.066698in}}%
\pgfpathlineto{\pgfqpoint{1.893516in}{3.100028in}}%
\pgfpathlineto{\pgfqpoint{1.963435in}{3.064499in}}%
\pgfpathlineto{\pgfqpoint{2.033354in}{3.033161in}}%
\pgfpathlineto{\pgfqpoint{1.963435in}{3.066698in}}%
\pgfpathclose%
\pgfusepath{fill}%
\end{pgfscope}%
\begin{pgfscope}%
\pgfpathrectangle{\pgfqpoint{0.637500in}{0.550000in}}{\pgfqpoint{3.850000in}{3.850000in}}%
\pgfusepath{clip}%
\pgfsetbuttcap%
\pgfsetroundjoin%
\definecolor{currentfill}{rgb}{0.925053,0.598564,0.000000}%
\pgfsetfillcolor{currentfill}%
\pgfsetfillopacity{0.800000}%
\pgfsetlinewidth{0.000000pt}%
\definecolor{currentstroke}{rgb}{0.000000,0.000000,0.000000}%
\pgfsetstrokecolor{currentstroke}%
\pgfsetdash{}{0pt}%
\pgfpathmoveto{\pgfqpoint{2.732540in}{2.963310in}}%
\pgfpathlineto{\pgfqpoint{2.662622in}{2.942680in}}%
\pgfpathlineto{\pgfqpoint{2.732540in}{2.961111in}}%
\pgfpathlineto{\pgfqpoint{2.802459in}{2.986389in}}%
\pgfpathlineto{\pgfqpoint{2.732540in}{2.963310in}}%
\pgfpathclose%
\pgfusepath{fill}%
\end{pgfscope}%
\begin{pgfscope}%
\pgfpathrectangle{\pgfqpoint{0.637500in}{0.550000in}}{\pgfqpoint{3.850000in}{3.850000in}}%
\pgfusepath{clip}%
\pgfsetbuttcap%
\pgfsetroundjoin%
\definecolor{currentfill}{rgb}{0.916926,0.593305,0.000000}%
\pgfsetfillcolor{currentfill}%
\pgfsetfillopacity{0.800000}%
\pgfsetlinewidth{0.000000pt}%
\definecolor{currentstroke}{rgb}{0.000000,0.000000,0.000000}%
\pgfsetstrokecolor{currentstroke}%
\pgfsetdash{}{0pt}%
\pgfpathmoveto{\pgfqpoint{2.173191in}{2.976842in}}%
\pgfpathlineto{\pgfqpoint{2.103272in}{3.002233in}}%
\pgfpathlineto{\pgfqpoint{2.173191in}{2.974644in}}%
\pgfpathlineto{\pgfqpoint{2.243110in}{2.953410in}}%
\pgfpathlineto{\pgfqpoint{2.173191in}{2.976842in}}%
\pgfpathclose%
\pgfusepath{fill}%
\end{pgfscope}%
\begin{pgfscope}%
\pgfpathrectangle{\pgfqpoint{0.637500in}{0.550000in}}{\pgfqpoint{3.850000in}{3.850000in}}%
\pgfusepath{clip}%
\pgfsetbuttcap%
\pgfsetroundjoin%
\definecolor{currentfill}{rgb}{0.904905,0.585527,0.000000}%
\pgfsetfillcolor{currentfill}%
\pgfsetfillopacity{0.800000}%
\pgfsetlinewidth{0.000000pt}%
\definecolor{currentstroke}{rgb}{0.000000,0.000000,0.000000}%
\pgfsetstrokecolor{currentstroke}%
\pgfsetdash{}{0pt}%
\pgfpathmoveto{\pgfqpoint{3.082134in}{3.116029in}}%
\pgfpathlineto{\pgfqpoint{3.012215in}{3.079818in}}%
\pgfpathlineto{\pgfqpoint{3.082134in}{3.113830in}}%
\pgfpathlineto{\pgfqpoint{3.152052in}{3.151364in}}%
\pgfpathlineto{\pgfqpoint{3.082134in}{3.116029in}}%
\pgfpathclose%
\pgfusepath{fill}%
\end{pgfscope}%
\begin{pgfscope}%
\pgfpathrectangle{\pgfqpoint{0.637500in}{0.550000in}}{\pgfqpoint{3.850000in}{3.850000in}}%
\pgfusepath{clip}%
\pgfsetbuttcap%
\pgfsetroundjoin%
\definecolor{currentfill}{rgb}{0.914972,0.592040,0.000000}%
\pgfsetfillcolor{currentfill}%
\pgfsetfillopacity{0.800000}%
\pgfsetlinewidth{0.000000pt}%
\definecolor{currentstroke}{rgb}{0.000000,0.000000,0.000000}%
\pgfsetstrokecolor{currentstroke}%
\pgfsetdash{}{0pt}%
\pgfpathmoveto{\pgfqpoint{2.872378in}{3.015473in}}%
\pgfpathlineto{\pgfqpoint{2.802459in}{2.986389in}}%
\pgfpathlineto{\pgfqpoint{2.872378in}{3.013274in}}%
\pgfpathlineto{\pgfqpoint{2.942296in}{3.045378in}}%
\pgfpathlineto{\pgfqpoint{2.872378in}{3.015473in}}%
\pgfpathclose%
\pgfusepath{fill}%
\end{pgfscope}%
\begin{pgfscope}%
\pgfpathrectangle{\pgfqpoint{0.637500in}{0.550000in}}{\pgfqpoint{3.850000in}{3.850000in}}%
\pgfusepath{clip}%
\pgfsetbuttcap%
\pgfsetroundjoin%
\definecolor{currentfill}{rgb}{0.908810,0.588053,0.000000}%
\pgfsetfillcolor{currentfill}%
\pgfsetfillopacity{0.800000}%
\pgfsetlinewidth{0.000000pt}%
\definecolor{currentstroke}{rgb}{0.000000,0.000000,0.000000}%
\pgfsetstrokecolor{currentstroke}%
\pgfsetdash{}{0pt}%
\pgfpathmoveto{\pgfqpoint{2.033354in}{3.033161in}}%
\pgfpathlineto{\pgfqpoint{1.963435in}{3.064499in}}%
\pgfpathlineto{\pgfqpoint{2.033354in}{3.030962in}}%
\pgfpathlineto{\pgfqpoint{2.103272in}{3.002233in}}%
\pgfpathlineto{\pgfqpoint{2.033354in}{3.033161in}}%
\pgfpathclose%
\pgfusepath{fill}%
\end{pgfscope}%
\begin{pgfscope}%
\pgfpathrectangle{\pgfqpoint{0.637500in}{0.550000in}}{\pgfqpoint{3.850000in}{3.850000in}}%
\pgfusepath{clip}%
\pgfsetbuttcap%
\pgfsetroundjoin%
\definecolor{currentfill}{rgb}{0.935716,0.605463,0.000000}%
\pgfsetfillcolor{currentfill}%
\pgfsetfillopacity{0.800000}%
\pgfsetlinewidth{0.000000pt}%
\definecolor{currentstroke}{rgb}{0.000000,0.000000,0.000000}%
\pgfsetstrokecolor{currentstroke}%
\pgfsetdash{}{0pt}%
\pgfpathmoveto{\pgfqpoint{2.522784in}{2.920672in}}%
\pgfpathlineto{\pgfqpoint{2.452866in}{2.918180in}}%
\pgfpathlineto{\pgfqpoint{2.522784in}{2.918474in}}%
\pgfpathlineto{\pgfqpoint{2.592703in}{2.927511in}}%
\pgfpathlineto{\pgfqpoint{2.522784in}{2.920672in}}%
\pgfpathclose%
\pgfusepath{fill}%
\end{pgfscope}%
\begin{pgfscope}%
\pgfpathrectangle{\pgfqpoint{0.637500in}{0.550000in}}{\pgfqpoint{3.850000in}{3.850000in}}%
\pgfusepath{clip}%
\pgfsetbuttcap%
\pgfsetroundjoin%
\definecolor{currentfill}{rgb}{0.931871,0.602975,0.000000}%
\pgfsetfillcolor{currentfill}%
\pgfsetfillopacity{0.800000}%
\pgfsetlinewidth{0.000000pt}%
\definecolor{currentstroke}{rgb}{0.000000,0.000000,0.000000}%
\pgfsetstrokecolor{currentstroke}%
\pgfsetdash{}{0pt}%
\pgfpathmoveto{\pgfqpoint{2.382947in}{2.924506in}}%
\pgfpathlineto{\pgfqpoint{2.313028in}{2.934976in}}%
\pgfpathlineto{\pgfqpoint{2.382947in}{2.922307in}}%
\pgfpathlineto{\pgfqpoint{2.452866in}{2.918180in}}%
\pgfpathlineto{\pgfqpoint{2.382947in}{2.924506in}}%
\pgfpathclose%
\pgfusepath{fill}%
\end{pgfscope}%
\begin{pgfscope}%
\pgfpathrectangle{\pgfqpoint{0.637500in}{0.550000in}}{\pgfqpoint{3.850000in}{3.850000in}}%
\pgfusepath{clip}%
\pgfsetbuttcap%
\pgfsetroundjoin%
\definecolor{currentfill}{rgb}{0.930057,0.601801,0.000000}%
\pgfsetfillcolor{currentfill}%
\pgfsetfillopacity{0.800000}%
\pgfsetlinewidth{0.000000pt}%
\definecolor{currentstroke}{rgb}{0.000000,0.000000,0.000000}%
\pgfsetstrokecolor{currentstroke}%
\pgfsetdash{}{0pt}%
\pgfpathmoveto{\pgfqpoint{2.662622in}{2.942680in}}%
\pgfpathlineto{\pgfqpoint{2.592703in}{2.927511in}}%
\pgfpathlineto{\pgfqpoint{2.662622in}{2.940481in}}%
\pgfpathlineto{\pgfqpoint{2.732540in}{2.961111in}}%
\pgfpathlineto{\pgfqpoint{2.662622in}{2.942680in}}%
\pgfpathclose%
\pgfusepath{fill}%
\end{pgfscope}%
\begin{pgfscope}%
\pgfpathrectangle{\pgfqpoint{0.637500in}{0.550000in}}{\pgfqpoint{3.850000in}{3.850000in}}%
\pgfusepath{clip}%
\pgfsetbuttcap%
\pgfsetroundjoin%
\definecolor{currentfill}{rgb}{0.921990,0.596582,0.000000}%
\pgfsetfillcolor{currentfill}%
\pgfsetfillopacity{0.800000}%
\pgfsetlinewidth{0.000000pt}%
\definecolor{currentstroke}{rgb}{0.000000,0.000000,0.000000}%
\pgfsetstrokecolor{currentstroke}%
\pgfsetdash{}{0pt}%
\pgfpathmoveto{\pgfqpoint{2.243110in}{2.953410in}}%
\pgfpathlineto{\pgfqpoint{2.173191in}{2.974644in}}%
\pgfpathlineto{\pgfqpoint{2.243110in}{2.951211in}}%
\pgfpathlineto{\pgfqpoint{2.313028in}{2.934976in}}%
\pgfpathlineto{\pgfqpoint{2.243110in}{2.953410in}}%
\pgfpathclose%
\pgfusepath{fill}%
\end{pgfscope}%
\begin{pgfscope}%
\pgfpathrectangle{\pgfqpoint{0.637500in}{0.550000in}}{\pgfqpoint{3.850000in}{3.850000in}}%
\pgfusepath{clip}%
\pgfsetbuttcap%
\pgfsetroundjoin%
\definecolor{currentfill}{rgb}{0.907506,0.587210,0.000000}%
\pgfsetfillcolor{currentfill}%
\pgfsetfillopacity{0.800000}%
\pgfsetlinewidth{0.000000pt}%
\definecolor{currentstroke}{rgb}{0.000000,0.000000,0.000000}%
\pgfsetstrokecolor{currentstroke}%
\pgfsetdash{}{0pt}%
\pgfpathmoveto{\pgfqpoint{3.012215in}{3.079818in}}%
\pgfpathlineto{\pgfqpoint{2.942296in}{3.045378in}}%
\pgfpathlineto{\pgfqpoint{3.012215in}{3.077619in}}%
\pgfpathlineto{\pgfqpoint{3.082134in}{3.113830in}}%
\pgfpathlineto{\pgfqpoint{3.012215in}{3.079818in}}%
\pgfpathclose%
\pgfusepath{fill}%
\end{pgfscope}%
\begin{pgfscope}%
\pgfpathrectangle{\pgfqpoint{0.637500in}{0.550000in}}{\pgfqpoint{3.850000in}{3.850000in}}%
\pgfusepath{clip}%
\pgfsetbuttcap%
\pgfsetroundjoin%
\definecolor{currentfill}{rgb}{0.919814,0.595174,0.000000}%
\pgfsetfillcolor{currentfill}%
\pgfsetfillopacity{0.800000}%
\pgfsetlinewidth{0.000000pt}%
\definecolor{currentstroke}{rgb}{0.000000,0.000000,0.000000}%
\pgfsetstrokecolor{currentstroke}%
\pgfsetdash{}{0pt}%
\pgfpathmoveto{\pgfqpoint{2.802459in}{2.986389in}}%
\pgfpathlineto{\pgfqpoint{2.732540in}{2.961111in}}%
\pgfpathlineto{\pgfqpoint{2.802459in}{2.984190in}}%
\pgfpathlineto{\pgfqpoint{2.872378in}{3.013274in}}%
\pgfpathlineto{\pgfqpoint{2.802459in}{2.986389in}}%
\pgfpathclose%
\pgfusepath{fill}%
\end{pgfscope}%
\begin{pgfscope}%
\pgfpathrectangle{\pgfqpoint{0.637500in}{0.550000in}}{\pgfqpoint{3.850000in}{3.850000in}}%
\pgfusepath{clip}%
\pgfsetbuttcap%
\pgfsetroundjoin%
\definecolor{currentfill}{rgb}{0.912477,0.590427,0.000000}%
\pgfsetfillcolor{currentfill}%
\pgfsetfillopacity{0.800000}%
\pgfsetlinewidth{0.000000pt}%
\definecolor{currentstroke}{rgb}{0.000000,0.000000,0.000000}%
\pgfsetstrokecolor{currentstroke}%
\pgfsetdash{}{0pt}%
\pgfpathmoveto{\pgfqpoint{2.103272in}{3.002233in}}%
\pgfpathlineto{\pgfqpoint{2.033354in}{3.030962in}}%
\pgfpathlineto{\pgfqpoint{2.103272in}{3.000034in}}%
\pgfpathlineto{\pgfqpoint{2.173191in}{2.974644in}}%
\pgfpathlineto{\pgfqpoint{2.103272in}{3.002233in}}%
\pgfpathclose%
\pgfusepath{fill}%
\end{pgfscope}%
\begin{pgfscope}%
\pgfpathrectangle{\pgfqpoint{0.637500in}{0.550000in}}{\pgfqpoint{3.850000in}{3.850000in}}%
\pgfusepath{clip}%
\pgfsetbuttcap%
\pgfsetroundjoin%
\definecolor{currentfill}{rgb}{0.934964,0.604977,0.000000}%
\pgfsetfillcolor{currentfill}%
\pgfsetfillopacity{0.800000}%
\pgfsetlinewidth{0.000000pt}%
\definecolor{currentstroke}{rgb}{0.000000,0.000000,0.000000}%
\pgfsetstrokecolor{currentstroke}%
\pgfsetdash{}{0pt}%
\pgfpathmoveto{\pgfqpoint{2.452866in}{2.918180in}}%
\pgfpathlineto{\pgfqpoint{2.382947in}{2.922307in}}%
\pgfpathlineto{\pgfqpoint{2.452866in}{2.915981in}}%
\pgfpathlineto{\pgfqpoint{2.522784in}{2.918474in}}%
\pgfpathlineto{\pgfqpoint{2.452866in}{2.918180in}}%
\pgfpathclose%
\pgfusepath{fill}%
\end{pgfscope}%
\begin{pgfscope}%
\pgfpathrectangle{\pgfqpoint{0.637500in}{0.550000in}}{\pgfqpoint{3.850000in}{3.850000in}}%
\pgfusepath{clip}%
\pgfsetbuttcap%
\pgfsetroundjoin%
\definecolor{currentfill}{rgb}{0.910843,0.589369,0.000000}%
\pgfsetfillcolor{currentfill}%
\pgfsetfillopacity{0.800000}%
\pgfsetlinewidth{0.000000pt}%
\definecolor{currentstroke}{rgb}{0.000000,0.000000,0.000000}%
\pgfsetstrokecolor{currentstroke}%
\pgfsetdash{}{0pt}%
\pgfpathmoveto{\pgfqpoint{2.942296in}{3.045378in}}%
\pgfpathlineto{\pgfqpoint{2.872378in}{3.013274in}}%
\pgfpathlineto{\pgfqpoint{2.942296in}{3.043180in}}%
\pgfpathlineto{\pgfqpoint{3.012215in}{3.077619in}}%
\pgfpathlineto{\pgfqpoint{2.942296in}{3.045378in}}%
\pgfpathclose%
\pgfusepath{fill}%
\end{pgfscope}%
\begin{pgfscope}%
\pgfpathrectangle{\pgfqpoint{0.637500in}{0.550000in}}{\pgfqpoint{3.850000in}{3.850000in}}%
\pgfusepath{clip}%
\pgfsetbuttcap%
\pgfsetroundjoin%
\definecolor{currentfill}{rgb}{0.933916,0.604299,0.000000}%
\pgfsetfillcolor{currentfill}%
\pgfsetfillopacity{0.800000}%
\pgfsetlinewidth{0.000000pt}%
\definecolor{currentstroke}{rgb}{0.000000,0.000000,0.000000}%
\pgfsetstrokecolor{currentstroke}%
\pgfsetdash{}{0pt}%
\pgfpathmoveto{\pgfqpoint{2.592703in}{2.927511in}}%
\pgfpathlineto{\pgfqpoint{2.522784in}{2.918474in}}%
\pgfpathlineto{\pgfqpoint{2.592703in}{2.925312in}}%
\pgfpathlineto{\pgfqpoint{2.662622in}{2.940481in}}%
\pgfpathlineto{\pgfqpoint{2.592703in}{2.927511in}}%
\pgfpathclose%
\pgfusepath{fill}%
\end{pgfscope}%
\begin{pgfscope}%
\pgfpathrectangle{\pgfqpoint{0.637500in}{0.550000in}}{\pgfqpoint{3.850000in}{3.850000in}}%
\pgfusepath{clip}%
\pgfsetbuttcap%
\pgfsetroundjoin%
\definecolor{currentfill}{rgb}{0.927227,0.599970,0.000000}%
\pgfsetfillcolor{currentfill}%
\pgfsetfillopacity{0.800000}%
\pgfsetlinewidth{0.000000pt}%
\definecolor{currentstroke}{rgb}{0.000000,0.000000,0.000000}%
\pgfsetstrokecolor{currentstroke}%
\pgfsetdash{}{0pt}%
\pgfpathmoveto{\pgfqpoint{2.313028in}{2.934976in}}%
\pgfpathlineto{\pgfqpoint{2.243110in}{2.951211in}}%
\pgfpathlineto{\pgfqpoint{2.313028in}{2.932777in}}%
\pgfpathlineto{\pgfqpoint{2.382947in}{2.922307in}}%
\pgfpathlineto{\pgfqpoint{2.313028in}{2.934976in}}%
\pgfpathclose%
\pgfusepath{fill}%
\end{pgfscope}%
\begin{pgfscope}%
\pgfpathrectangle{\pgfqpoint{0.637500in}{0.550000in}}{\pgfqpoint{3.850000in}{3.850000in}}%
\pgfusepath{clip}%
\pgfsetbuttcap%
\pgfsetroundjoin%
\definecolor{currentfill}{rgb}{0.925053,0.598564,0.000000}%
\pgfsetfillcolor{currentfill}%
\pgfsetfillopacity{0.800000}%
\pgfsetlinewidth{0.000000pt}%
\definecolor{currentstroke}{rgb}{0.000000,0.000000,0.000000}%
\pgfsetstrokecolor{currentstroke}%
\pgfsetdash{}{0pt}%
\pgfpathmoveto{\pgfqpoint{2.732540in}{2.961111in}}%
\pgfpathlineto{\pgfqpoint{2.662622in}{2.940481in}}%
\pgfpathlineto{\pgfqpoint{2.732540in}{2.958912in}}%
\pgfpathlineto{\pgfqpoint{2.802459in}{2.984190in}}%
\pgfpathlineto{\pgfqpoint{2.732540in}{2.961111in}}%
\pgfpathclose%
\pgfusepath{fill}%
\end{pgfscope}%
\begin{pgfscope}%
\pgfpathrectangle{\pgfqpoint{0.637500in}{0.550000in}}{\pgfqpoint{3.850000in}{3.850000in}}%
\pgfusepath{clip}%
\pgfsetbuttcap%
\pgfsetroundjoin%
\definecolor{currentfill}{rgb}{0.916926,0.593305,0.000000}%
\pgfsetfillcolor{currentfill}%
\pgfsetfillopacity{0.800000}%
\pgfsetlinewidth{0.000000pt}%
\definecolor{currentstroke}{rgb}{0.000000,0.000000,0.000000}%
\pgfsetstrokecolor{currentstroke}%
\pgfsetdash{}{0pt}%
\pgfpathmoveto{\pgfqpoint{2.173191in}{2.974644in}}%
\pgfpathlineto{\pgfqpoint{2.103272in}{3.000034in}}%
\pgfpathlineto{\pgfqpoint{2.173191in}{2.972445in}}%
\pgfpathlineto{\pgfqpoint{2.243110in}{2.951211in}}%
\pgfpathlineto{\pgfqpoint{2.173191in}{2.974644in}}%
\pgfpathclose%
\pgfusepath{fill}%
\end{pgfscope}%
\begin{pgfscope}%
\pgfpathrectangle{\pgfqpoint{0.637500in}{0.550000in}}{\pgfqpoint{3.850000in}{3.850000in}}%
\pgfusepath{clip}%
\pgfsetbuttcap%
\pgfsetroundjoin%
\definecolor{currentfill}{rgb}{0.914972,0.592040,0.000000}%
\pgfsetfillcolor{currentfill}%
\pgfsetfillopacity{0.800000}%
\pgfsetlinewidth{0.000000pt}%
\definecolor{currentstroke}{rgb}{0.000000,0.000000,0.000000}%
\pgfsetstrokecolor{currentstroke}%
\pgfsetdash{}{0pt}%
\pgfpathmoveto{\pgfqpoint{2.872378in}{3.013274in}}%
\pgfpathlineto{\pgfqpoint{2.802459in}{2.984190in}}%
\pgfpathlineto{\pgfqpoint{2.872378in}{3.011075in}}%
\pgfpathlineto{\pgfqpoint{2.942296in}{3.043180in}}%
\pgfpathlineto{\pgfqpoint{2.872378in}{3.013274in}}%
\pgfpathclose%
\pgfusepath{fill}%
\end{pgfscope}%
\begin{pgfscope}%
\pgfpathrectangle{\pgfqpoint{0.637500in}{0.550000in}}{\pgfqpoint{3.850000in}{3.850000in}}%
\pgfusepath{clip}%
\pgfsetbuttcap%
\pgfsetroundjoin%
\definecolor{currentfill}{rgb}{0.935716,0.605463,0.000000}%
\pgfsetfillcolor{currentfill}%
\pgfsetfillopacity{0.800000}%
\pgfsetlinewidth{0.000000pt}%
\definecolor{currentstroke}{rgb}{0.000000,0.000000,0.000000}%
\pgfsetstrokecolor{currentstroke}%
\pgfsetdash{}{0pt}%
\pgfpathmoveto{\pgfqpoint{2.522784in}{2.918474in}}%
\pgfpathlineto{\pgfqpoint{2.452866in}{2.915981in}}%
\pgfpathlineto{\pgfqpoint{2.522784in}{2.916275in}}%
\pgfpathlineto{\pgfqpoint{2.592703in}{2.925312in}}%
\pgfpathlineto{\pgfqpoint{2.522784in}{2.918474in}}%
\pgfpathclose%
\pgfusepath{fill}%
\end{pgfscope}%
\begin{pgfscope}%
\pgfpathrectangle{\pgfqpoint{0.637500in}{0.550000in}}{\pgfqpoint{3.850000in}{3.850000in}}%
\pgfusepath{clip}%
\pgfsetbuttcap%
\pgfsetroundjoin%
\definecolor{currentfill}{rgb}{0.931871,0.602975,0.000000}%
\pgfsetfillcolor{currentfill}%
\pgfsetfillopacity{0.800000}%
\pgfsetlinewidth{0.000000pt}%
\definecolor{currentstroke}{rgb}{0.000000,0.000000,0.000000}%
\pgfsetstrokecolor{currentstroke}%
\pgfsetdash{}{0pt}%
\pgfpathmoveto{\pgfqpoint{2.382947in}{2.922307in}}%
\pgfpathlineto{\pgfqpoint{2.313028in}{2.932777in}}%
\pgfpathlineto{\pgfqpoint{2.382947in}{2.920109in}}%
\pgfpathlineto{\pgfqpoint{2.452866in}{2.915981in}}%
\pgfpathlineto{\pgfqpoint{2.382947in}{2.922307in}}%
\pgfpathclose%
\pgfusepath{fill}%
\end{pgfscope}%
\begin{pgfscope}%
\pgfpathrectangle{\pgfqpoint{0.637500in}{0.550000in}}{\pgfqpoint{3.850000in}{3.850000in}}%
\pgfusepath{clip}%
\pgfsetbuttcap%
\pgfsetroundjoin%
\definecolor{currentfill}{rgb}{0.930057,0.601801,0.000000}%
\pgfsetfillcolor{currentfill}%
\pgfsetfillopacity{0.800000}%
\pgfsetlinewidth{0.000000pt}%
\definecolor{currentstroke}{rgb}{0.000000,0.000000,0.000000}%
\pgfsetstrokecolor{currentstroke}%
\pgfsetdash{}{0pt}%
\pgfpathmoveto{\pgfqpoint{2.662622in}{2.940481in}}%
\pgfpathlineto{\pgfqpoint{2.592703in}{2.925312in}}%
\pgfpathlineto{\pgfqpoint{2.662622in}{2.938283in}}%
\pgfpathlineto{\pgfqpoint{2.732540in}{2.958912in}}%
\pgfpathlineto{\pgfqpoint{2.662622in}{2.940481in}}%
\pgfpathclose%
\pgfusepath{fill}%
\end{pgfscope}%
\begin{pgfscope}%
\pgfpathrectangle{\pgfqpoint{0.637500in}{0.550000in}}{\pgfqpoint{3.850000in}{3.850000in}}%
\pgfusepath{clip}%
\pgfsetbuttcap%
\pgfsetroundjoin%
\definecolor{currentfill}{rgb}{0.921990,0.596582,0.000000}%
\pgfsetfillcolor{currentfill}%
\pgfsetfillopacity{0.800000}%
\pgfsetlinewidth{0.000000pt}%
\definecolor{currentstroke}{rgb}{0.000000,0.000000,0.000000}%
\pgfsetstrokecolor{currentstroke}%
\pgfsetdash{}{0pt}%
\pgfpathmoveto{\pgfqpoint{2.243110in}{2.951211in}}%
\pgfpathlineto{\pgfqpoint{2.173191in}{2.972445in}}%
\pgfpathlineto{\pgfqpoint{2.243110in}{2.949012in}}%
\pgfpathlineto{\pgfqpoint{2.313028in}{2.932777in}}%
\pgfpathlineto{\pgfqpoint{2.243110in}{2.951211in}}%
\pgfpathclose%
\pgfusepath{fill}%
\end{pgfscope}%
\begin{pgfscope}%
\pgfpathrectangle{\pgfqpoint{0.637500in}{0.550000in}}{\pgfqpoint{3.850000in}{3.850000in}}%
\pgfusepath{clip}%
\pgfsetbuttcap%
\pgfsetroundjoin%
\definecolor{currentfill}{rgb}{0.919814,0.595174,0.000000}%
\pgfsetfillcolor{currentfill}%
\pgfsetfillopacity{0.800000}%
\pgfsetlinewidth{0.000000pt}%
\definecolor{currentstroke}{rgb}{0.000000,0.000000,0.000000}%
\pgfsetstrokecolor{currentstroke}%
\pgfsetdash{}{0pt}%
\pgfpathmoveto{\pgfqpoint{2.802459in}{2.984190in}}%
\pgfpathlineto{\pgfqpoint{2.732540in}{2.958912in}}%
\pgfpathlineto{\pgfqpoint{2.802459in}{2.981991in}}%
\pgfpathlineto{\pgfqpoint{2.872378in}{3.011075in}}%
\pgfpathlineto{\pgfqpoint{2.802459in}{2.984190in}}%
\pgfpathclose%
\pgfusepath{fill}%
\end{pgfscope}%
\begin{pgfscope}%
\pgfpathrectangle{\pgfqpoint{0.637500in}{0.550000in}}{\pgfqpoint{3.850000in}{3.850000in}}%
\pgfusepath{clip}%
\pgfsetbuttcap%
\pgfsetroundjoin%
\definecolor{currentfill}{rgb}{0.934964,0.604977,0.000000}%
\pgfsetfillcolor{currentfill}%
\pgfsetfillopacity{0.800000}%
\pgfsetlinewidth{0.000000pt}%
\definecolor{currentstroke}{rgb}{0.000000,0.000000,0.000000}%
\pgfsetstrokecolor{currentstroke}%
\pgfsetdash{}{0pt}%
\pgfpathmoveto{\pgfqpoint{2.452866in}{2.915981in}}%
\pgfpathlineto{\pgfqpoint{2.382947in}{2.920109in}}%
\pgfpathlineto{\pgfqpoint{2.452866in}{2.913782in}}%
\pgfpathlineto{\pgfqpoint{2.522784in}{2.916275in}}%
\pgfpathlineto{\pgfqpoint{2.452866in}{2.915981in}}%
\pgfpathclose%
\pgfusepath{fill}%
\end{pgfscope}%
\begin{pgfscope}%
\pgfpathrectangle{\pgfqpoint{0.637500in}{0.550000in}}{\pgfqpoint{3.850000in}{3.850000in}}%
\pgfusepath{clip}%
\pgfsetbuttcap%
\pgfsetroundjoin%
\definecolor{currentfill}{rgb}{0.933916,0.604299,0.000000}%
\pgfsetfillcolor{currentfill}%
\pgfsetfillopacity{0.800000}%
\pgfsetlinewidth{0.000000pt}%
\definecolor{currentstroke}{rgb}{0.000000,0.000000,0.000000}%
\pgfsetstrokecolor{currentstroke}%
\pgfsetdash{}{0pt}%
\pgfpathmoveto{\pgfqpoint{2.592703in}{2.925312in}}%
\pgfpathlineto{\pgfqpoint{2.522784in}{2.916275in}}%
\pgfpathlineto{\pgfqpoint{2.592703in}{2.923114in}}%
\pgfpathlineto{\pgfqpoint{2.662622in}{2.938283in}}%
\pgfpathlineto{\pgfqpoint{2.592703in}{2.925312in}}%
\pgfpathclose%
\pgfusepath{fill}%
\end{pgfscope}%
\begin{pgfscope}%
\pgfpathrectangle{\pgfqpoint{0.637500in}{0.550000in}}{\pgfqpoint{3.850000in}{3.850000in}}%
\pgfusepath{clip}%
\pgfsetbuttcap%
\pgfsetroundjoin%
\definecolor{currentfill}{rgb}{0.927227,0.599970,0.000000}%
\pgfsetfillcolor{currentfill}%
\pgfsetfillopacity{0.800000}%
\pgfsetlinewidth{0.000000pt}%
\definecolor{currentstroke}{rgb}{0.000000,0.000000,0.000000}%
\pgfsetstrokecolor{currentstroke}%
\pgfsetdash{}{0pt}%
\pgfpathmoveto{\pgfqpoint{2.313028in}{2.932777in}}%
\pgfpathlineto{\pgfqpoint{2.243110in}{2.949012in}}%
\pgfpathlineto{\pgfqpoint{2.313028in}{2.930578in}}%
\pgfpathlineto{\pgfqpoint{2.382947in}{2.920109in}}%
\pgfpathlineto{\pgfqpoint{2.313028in}{2.932777in}}%
\pgfpathclose%
\pgfusepath{fill}%
\end{pgfscope}%
\begin{pgfscope}%
\pgfpathrectangle{\pgfqpoint{0.637500in}{0.550000in}}{\pgfqpoint{3.850000in}{3.850000in}}%
\pgfusepath{clip}%
\pgfsetbuttcap%
\pgfsetroundjoin%
\definecolor{currentfill}{rgb}{0.925053,0.598564,0.000000}%
\pgfsetfillcolor{currentfill}%
\pgfsetfillopacity{0.800000}%
\pgfsetlinewidth{0.000000pt}%
\definecolor{currentstroke}{rgb}{0.000000,0.000000,0.000000}%
\pgfsetstrokecolor{currentstroke}%
\pgfsetdash{}{0pt}%
\pgfpathmoveto{\pgfqpoint{2.732540in}{2.958912in}}%
\pgfpathlineto{\pgfqpoint{2.662622in}{2.938283in}}%
\pgfpathlineto{\pgfqpoint{2.732540in}{2.956713in}}%
\pgfpathlineto{\pgfqpoint{2.802459in}{2.981991in}}%
\pgfpathlineto{\pgfqpoint{2.732540in}{2.958912in}}%
\pgfpathclose%
\pgfusepath{fill}%
\end{pgfscope}%
\begin{pgfscope}%
\pgfpathrectangle{\pgfqpoint{0.637500in}{0.550000in}}{\pgfqpoint{3.850000in}{3.850000in}}%
\pgfusepath{clip}%
\pgfsetbuttcap%
\pgfsetroundjoin%
\definecolor{currentfill}{rgb}{0.935716,0.605463,0.000000}%
\pgfsetfillcolor{currentfill}%
\pgfsetfillopacity{0.800000}%
\pgfsetlinewidth{0.000000pt}%
\definecolor{currentstroke}{rgb}{0.000000,0.000000,0.000000}%
\pgfsetstrokecolor{currentstroke}%
\pgfsetdash{}{0pt}%
\pgfpathmoveto{\pgfqpoint{2.522784in}{2.916275in}}%
\pgfpathlineto{\pgfqpoint{2.452866in}{2.913782in}}%
\pgfpathlineto{\pgfqpoint{2.522784in}{2.914076in}}%
\pgfpathlineto{\pgfqpoint{2.592703in}{2.923114in}}%
\pgfpathlineto{\pgfqpoint{2.522784in}{2.916275in}}%
\pgfpathclose%
\pgfusepath{fill}%
\end{pgfscope}%
\begin{pgfscope}%
\pgfpathrectangle{\pgfqpoint{0.637500in}{0.550000in}}{\pgfqpoint{3.850000in}{3.850000in}}%
\pgfusepath{clip}%
\pgfsetbuttcap%
\pgfsetroundjoin%
\definecolor{currentfill}{rgb}{0.931871,0.602975,0.000000}%
\pgfsetfillcolor{currentfill}%
\pgfsetfillopacity{0.800000}%
\pgfsetlinewidth{0.000000pt}%
\definecolor{currentstroke}{rgb}{0.000000,0.000000,0.000000}%
\pgfsetstrokecolor{currentstroke}%
\pgfsetdash{}{0pt}%
\pgfpathmoveto{\pgfqpoint{2.382947in}{2.920109in}}%
\pgfpathlineto{\pgfqpoint{2.313028in}{2.930578in}}%
\pgfpathlineto{\pgfqpoint{2.382947in}{2.917910in}}%
\pgfpathlineto{\pgfqpoint{2.452866in}{2.913782in}}%
\pgfpathlineto{\pgfqpoint{2.382947in}{2.920109in}}%
\pgfpathclose%
\pgfusepath{fill}%
\end{pgfscope}%
\begin{pgfscope}%
\pgfpathrectangle{\pgfqpoint{0.637500in}{0.550000in}}{\pgfqpoint{3.850000in}{3.850000in}}%
\pgfusepath{clip}%
\pgfsetbuttcap%
\pgfsetroundjoin%
\definecolor{currentfill}{rgb}{0.930057,0.601801,0.000000}%
\pgfsetfillcolor{currentfill}%
\pgfsetfillopacity{0.800000}%
\pgfsetlinewidth{0.000000pt}%
\definecolor{currentstroke}{rgb}{0.000000,0.000000,0.000000}%
\pgfsetstrokecolor{currentstroke}%
\pgfsetdash{}{0pt}%
\pgfpathmoveto{\pgfqpoint{2.662622in}{2.938283in}}%
\pgfpathlineto{\pgfqpoint{2.592703in}{2.923114in}}%
\pgfpathlineto{\pgfqpoint{2.662622in}{2.936084in}}%
\pgfpathlineto{\pgfqpoint{2.732540in}{2.956713in}}%
\pgfpathlineto{\pgfqpoint{2.662622in}{2.938283in}}%
\pgfpathclose%
\pgfusepath{fill}%
\end{pgfscope}%
\begin{pgfscope}%
\pgfpathrectangle{\pgfqpoint{0.637500in}{0.550000in}}{\pgfqpoint{3.850000in}{3.850000in}}%
\pgfusepath{clip}%
\pgfsetbuttcap%
\pgfsetroundjoin%
\definecolor{currentfill}{rgb}{0.934964,0.604977,0.000000}%
\pgfsetfillcolor{currentfill}%
\pgfsetfillopacity{0.800000}%
\pgfsetlinewidth{0.000000pt}%
\definecolor{currentstroke}{rgb}{0.000000,0.000000,0.000000}%
\pgfsetstrokecolor{currentstroke}%
\pgfsetdash{}{0pt}%
\pgfpathmoveto{\pgfqpoint{2.452866in}{2.913782in}}%
\pgfpathlineto{\pgfqpoint{2.382947in}{2.917910in}}%
\pgfpathlineto{\pgfqpoint{2.452866in}{2.911583in}}%
\pgfpathlineto{\pgfqpoint{2.522784in}{2.914076in}}%
\pgfpathlineto{\pgfqpoint{2.452866in}{2.913782in}}%
\pgfpathclose%
\pgfusepath{fill}%
\end{pgfscope}%
\begin{pgfscope}%
\pgfpathrectangle{\pgfqpoint{0.637500in}{0.550000in}}{\pgfqpoint{3.850000in}{3.850000in}}%
\pgfusepath{clip}%
\pgfsetbuttcap%
\pgfsetroundjoin%
\definecolor{currentfill}{rgb}{0.933916,0.604299,0.000000}%
\pgfsetfillcolor{currentfill}%
\pgfsetfillopacity{0.800000}%
\pgfsetlinewidth{0.000000pt}%
\definecolor{currentstroke}{rgb}{0.000000,0.000000,0.000000}%
\pgfsetstrokecolor{currentstroke}%
\pgfsetdash{}{0pt}%
\pgfpathmoveto{\pgfqpoint{2.592703in}{2.923114in}}%
\pgfpathlineto{\pgfqpoint{2.522784in}{2.914076in}}%
\pgfpathlineto{\pgfqpoint{2.592703in}{2.920915in}}%
\pgfpathlineto{\pgfqpoint{2.662622in}{2.936084in}}%
\pgfpathlineto{\pgfqpoint{2.592703in}{2.923114in}}%
\pgfpathclose%
\pgfusepath{fill}%
\end{pgfscope}%
\begin{pgfscope}%
\pgfpathrectangle{\pgfqpoint{0.637500in}{0.550000in}}{\pgfqpoint{3.850000in}{3.850000in}}%
\pgfusepath{clip}%
\pgfsetbuttcap%
\pgfsetroundjoin%
\definecolor{currentfill}{rgb}{0.935716,0.605463,0.000000}%
\pgfsetfillcolor{currentfill}%
\pgfsetfillopacity{0.800000}%
\pgfsetlinewidth{0.000000pt}%
\definecolor{currentstroke}{rgb}{0.000000,0.000000,0.000000}%
\pgfsetstrokecolor{currentstroke}%
\pgfsetdash{}{0pt}%
\pgfpathmoveto{\pgfqpoint{2.522784in}{2.914076in}}%
\pgfpathlineto{\pgfqpoint{2.452866in}{2.911583in}}%
\pgfpathlineto{\pgfqpoint{2.522784in}{2.911877in}}%
\pgfpathlineto{\pgfqpoint{2.592703in}{2.920915in}}%
\pgfpathlineto{\pgfqpoint{2.522784in}{2.914076in}}%
\pgfpathclose%
\pgfusepath{fill}%
\end{pgfscope}%
\begin{pgfscope}%
\pgfpathrectangle{\pgfqpoint{0.637500in}{0.550000in}}{\pgfqpoint{3.850000in}{3.850000in}}%
\pgfusepath{clip}%
\pgfsetbuttcap%
\pgfsetroundjoin%
\definecolor{currentfill}{rgb}{0.909646,0.000000,0.000000}%
\pgfsetfillcolor{currentfill}%
\pgfsetfillopacity{0.800000}%
\pgfsetlinewidth{0.000000pt}%
\definecolor{currentstroke}{rgb}{0.000000,0.000000,0.000000}%
\pgfsetstrokecolor{currentstroke}%
\pgfsetdash{}{0pt}%
\pgfpathmoveto{\pgfqpoint{2.522784in}{2.833740in}}%
\pgfpathlineto{\pgfqpoint{2.452866in}{2.800789in}}%
\pgfpathlineto{\pgfqpoint{2.522784in}{2.831541in}}%
\pgfpathlineto{\pgfqpoint{2.592703in}{2.872638in}}%
\pgfpathlineto{\pgfqpoint{2.522784in}{2.833740in}}%
\pgfpathclose%
\pgfusepath{fill}%
\end{pgfscope}%
\begin{pgfscope}%
\pgfpathrectangle{\pgfqpoint{0.637500in}{0.550000in}}{\pgfqpoint{3.850000in}{3.850000in}}%
\pgfusepath{clip}%
\pgfsetbuttcap%
\pgfsetroundjoin%
\definecolor{currentfill}{rgb}{0.898098,0.000000,0.000000}%
\pgfsetfillcolor{currentfill}%
\pgfsetfillopacity{0.800000}%
\pgfsetlinewidth{0.000000pt}%
\definecolor{currentstroke}{rgb}{0.000000,0.000000,0.000000}%
\pgfsetstrokecolor{currentstroke}%
\pgfsetdash{}{0pt}%
\pgfpathmoveto{\pgfqpoint{2.452866in}{2.800789in}}%
\pgfpathlineto{\pgfqpoint{2.382947in}{2.839292in}}%
\pgfpathlineto{\pgfqpoint{2.452866in}{2.798590in}}%
\pgfpathlineto{\pgfqpoint{2.522784in}{2.831541in}}%
\pgfpathlineto{\pgfqpoint{2.452866in}{2.800789in}}%
\pgfpathclose%
\pgfusepath{fill}%
\end{pgfscope}%
\begin{pgfscope}%
\pgfpathrectangle{\pgfqpoint{0.637500in}{0.550000in}}{\pgfqpoint{3.850000in}{3.850000in}}%
\pgfusepath{clip}%
\pgfsetbuttcap%
\pgfsetroundjoin%
\definecolor{currentfill}{rgb}{0.897487,0.000000,0.000000}%
\pgfsetfillcolor{currentfill}%
\pgfsetfillopacity{0.800000}%
\pgfsetlinewidth{0.000000pt}%
\definecolor{currentstroke}{rgb}{0.000000,0.000000,0.000000}%
\pgfsetstrokecolor{currentstroke}%
\pgfsetdash{}{0pt}%
\pgfpathmoveto{\pgfqpoint{2.592703in}{2.872638in}}%
\pgfpathlineto{\pgfqpoint{2.522784in}{2.831541in}}%
\pgfpathlineto{\pgfqpoint{2.592703in}{2.870440in}}%
\pgfpathlineto{\pgfqpoint{2.662622in}{2.911537in}}%
\pgfpathlineto{\pgfqpoint{2.592703in}{2.872638in}}%
\pgfpathclose%
\pgfusepath{fill}%
\end{pgfscope}%
\begin{pgfscope}%
\pgfpathrectangle{\pgfqpoint{0.637500in}{0.550000in}}{\pgfqpoint{3.850000in}{3.850000in}}%
\pgfusepath{clip}%
\pgfsetbuttcap%
\pgfsetroundjoin%
\definecolor{currentfill}{rgb}{0.909646,0.000000,0.000000}%
\pgfsetfillcolor{currentfill}%
\pgfsetfillopacity{0.800000}%
\pgfsetlinewidth{0.000000pt}%
\definecolor{currentstroke}{rgb}{0.000000,0.000000,0.000000}%
\pgfsetstrokecolor{currentstroke}%
\pgfsetdash{}{0pt}%
\pgfpathmoveto{\pgfqpoint{2.522784in}{2.831541in}}%
\pgfpathlineto{\pgfqpoint{2.452866in}{2.798590in}}%
\pgfpathlineto{\pgfqpoint{2.522784in}{2.829342in}}%
\pgfpathlineto{\pgfqpoint{2.592703in}{2.870440in}}%
\pgfpathlineto{\pgfqpoint{2.522784in}{2.831541in}}%
\pgfpathclose%
\pgfusepath{fill}%
\end{pgfscope}%
\begin{pgfscope}%
\pgfpathrectangle{\pgfqpoint{0.637500in}{0.550000in}}{\pgfqpoint{3.850000in}{3.850000in}}%
\pgfusepath{clip}%
\pgfsetbuttcap%
\pgfsetroundjoin%
\definecolor{currentfill}{rgb}{0.897487,0.000000,0.000000}%
\pgfsetfillcolor{currentfill}%
\pgfsetfillopacity{0.800000}%
\pgfsetlinewidth{0.000000pt}%
\definecolor{currentstroke}{rgb}{0.000000,0.000000,0.000000}%
\pgfsetstrokecolor{currentstroke}%
\pgfsetdash{}{0pt}%
\pgfpathmoveto{\pgfqpoint{2.382947in}{2.839292in}}%
\pgfpathlineto{\pgfqpoint{2.313028in}{2.878190in}}%
\pgfpathlineto{\pgfqpoint{2.382947in}{2.837093in}}%
\pgfpathlineto{\pgfqpoint{2.452866in}{2.798590in}}%
\pgfpathlineto{\pgfqpoint{2.382947in}{2.839292in}}%
\pgfpathclose%
\pgfusepath{fill}%
\end{pgfscope}%
\begin{pgfscope}%
\pgfpathrectangle{\pgfqpoint{0.637500in}{0.550000in}}{\pgfqpoint{3.850000in}{3.850000in}}%
\pgfusepath{clip}%
\pgfsetbuttcap%
\pgfsetroundjoin%
\definecolor{currentfill}{rgb}{0.897487,0.000000,0.000000}%
\pgfsetfillcolor{currentfill}%
\pgfsetfillopacity{0.800000}%
\pgfsetlinewidth{0.000000pt}%
\definecolor{currentstroke}{rgb}{0.000000,0.000000,0.000000}%
\pgfsetstrokecolor{currentstroke}%
\pgfsetdash{}{0pt}%
\pgfpathmoveto{\pgfqpoint{2.662622in}{2.911537in}}%
\pgfpathlineto{\pgfqpoint{2.592703in}{2.870440in}}%
\pgfpathlineto{\pgfqpoint{2.662622in}{2.909338in}}%
\pgfpathlineto{\pgfqpoint{2.732540in}{2.950435in}}%
\pgfpathlineto{\pgfqpoint{2.662622in}{2.911537in}}%
\pgfpathclose%
\pgfusepath{fill}%
\end{pgfscope}%
\begin{pgfscope}%
\pgfpathrectangle{\pgfqpoint{0.637500in}{0.550000in}}{\pgfqpoint{3.850000in}{3.850000in}}%
\pgfusepath{clip}%
\pgfsetbuttcap%
\pgfsetroundjoin%
\definecolor{currentfill}{rgb}{0.898098,0.000000,0.000000}%
\pgfsetfillcolor{currentfill}%
\pgfsetfillopacity{0.800000}%
\pgfsetlinewidth{0.000000pt}%
\definecolor{currentstroke}{rgb}{0.000000,0.000000,0.000000}%
\pgfsetstrokecolor{currentstroke}%
\pgfsetdash{}{0pt}%
\pgfpathmoveto{\pgfqpoint{2.452866in}{2.798590in}}%
\pgfpathlineto{\pgfqpoint{2.382947in}{2.837093in}}%
\pgfpathlineto{\pgfqpoint{2.452866in}{2.796392in}}%
\pgfpathlineto{\pgfqpoint{2.522784in}{2.829342in}}%
\pgfpathlineto{\pgfqpoint{2.452866in}{2.798590in}}%
\pgfpathclose%
\pgfusepath{fill}%
\end{pgfscope}%
\begin{pgfscope}%
\pgfpathrectangle{\pgfqpoint{0.637500in}{0.550000in}}{\pgfqpoint{3.850000in}{3.850000in}}%
\pgfusepath{clip}%
\pgfsetbuttcap%
\pgfsetroundjoin%
\definecolor{currentfill}{rgb}{0.897487,0.000000,0.000000}%
\pgfsetfillcolor{currentfill}%
\pgfsetfillopacity{0.800000}%
\pgfsetlinewidth{0.000000pt}%
\definecolor{currentstroke}{rgb}{0.000000,0.000000,0.000000}%
\pgfsetstrokecolor{currentstroke}%
\pgfsetdash{}{0pt}%
\pgfpathmoveto{\pgfqpoint{2.592703in}{2.870440in}}%
\pgfpathlineto{\pgfqpoint{2.522784in}{2.829342in}}%
\pgfpathlineto{\pgfqpoint{2.592703in}{2.868241in}}%
\pgfpathlineto{\pgfqpoint{2.662622in}{2.909338in}}%
\pgfpathlineto{\pgfqpoint{2.592703in}{2.870440in}}%
\pgfpathclose%
\pgfusepath{fill}%
\end{pgfscope}%
\begin{pgfscope}%
\pgfpathrectangle{\pgfqpoint{0.637500in}{0.550000in}}{\pgfqpoint{3.850000in}{3.850000in}}%
\pgfusepath{clip}%
\pgfsetbuttcap%
\pgfsetroundjoin%
\definecolor{currentfill}{rgb}{0.897487,0.000000,0.000000}%
\pgfsetfillcolor{currentfill}%
\pgfsetfillopacity{0.800000}%
\pgfsetlinewidth{0.000000pt}%
\definecolor{currentstroke}{rgb}{0.000000,0.000000,0.000000}%
\pgfsetstrokecolor{currentstroke}%
\pgfsetdash{}{0pt}%
\pgfpathmoveto{\pgfqpoint{2.313028in}{2.878190in}}%
\pgfpathlineto{\pgfqpoint{2.243110in}{2.917089in}}%
\pgfpathlineto{\pgfqpoint{2.313028in}{2.875992in}}%
\pgfpathlineto{\pgfqpoint{2.382947in}{2.837093in}}%
\pgfpathlineto{\pgfqpoint{2.313028in}{2.878190in}}%
\pgfpathclose%
\pgfusepath{fill}%
\end{pgfscope}%
\begin{pgfscope}%
\pgfpathrectangle{\pgfqpoint{0.637500in}{0.550000in}}{\pgfqpoint{3.850000in}{3.850000in}}%
\pgfusepath{clip}%
\pgfsetbuttcap%
\pgfsetroundjoin%
\definecolor{currentfill}{rgb}{0.909646,0.000000,0.000000}%
\pgfsetfillcolor{currentfill}%
\pgfsetfillopacity{0.800000}%
\pgfsetlinewidth{0.000000pt}%
\definecolor{currentstroke}{rgb}{0.000000,0.000000,0.000000}%
\pgfsetstrokecolor{currentstroke}%
\pgfsetdash{}{0pt}%
\pgfpathmoveto{\pgfqpoint{2.522784in}{2.829342in}}%
\pgfpathlineto{\pgfqpoint{2.452866in}{2.796392in}}%
\pgfpathlineto{\pgfqpoint{2.522784in}{2.827144in}}%
\pgfpathlineto{\pgfqpoint{2.592703in}{2.868241in}}%
\pgfpathlineto{\pgfqpoint{2.522784in}{2.829342in}}%
\pgfpathclose%
\pgfusepath{fill}%
\end{pgfscope}%
\begin{pgfscope}%
\pgfpathrectangle{\pgfqpoint{0.637500in}{0.550000in}}{\pgfqpoint{3.850000in}{3.850000in}}%
\pgfusepath{clip}%
\pgfsetbuttcap%
\pgfsetroundjoin%
\definecolor{currentfill}{rgb}{0.897487,0.000000,0.000000}%
\pgfsetfillcolor{currentfill}%
\pgfsetfillopacity{0.800000}%
\pgfsetlinewidth{0.000000pt}%
\definecolor{currentstroke}{rgb}{0.000000,0.000000,0.000000}%
\pgfsetstrokecolor{currentstroke}%
\pgfsetdash{}{0pt}%
\pgfpathmoveto{\pgfqpoint{2.732540in}{2.950435in}}%
\pgfpathlineto{\pgfqpoint{2.662622in}{2.909338in}}%
\pgfpathlineto{\pgfqpoint{2.732540in}{2.948236in}}%
\pgfpathlineto{\pgfqpoint{2.802459in}{2.989333in}}%
\pgfpathlineto{\pgfqpoint{2.732540in}{2.950435in}}%
\pgfpathclose%
\pgfusepath{fill}%
\end{pgfscope}%
\begin{pgfscope}%
\pgfpathrectangle{\pgfqpoint{0.637500in}{0.550000in}}{\pgfqpoint{3.850000in}{3.850000in}}%
\pgfusepath{clip}%
\pgfsetbuttcap%
\pgfsetroundjoin%
\definecolor{currentfill}{rgb}{0.897487,0.000000,0.000000}%
\pgfsetfillcolor{currentfill}%
\pgfsetfillopacity{0.800000}%
\pgfsetlinewidth{0.000000pt}%
\definecolor{currentstroke}{rgb}{0.000000,0.000000,0.000000}%
\pgfsetstrokecolor{currentstroke}%
\pgfsetdash{}{0pt}%
\pgfpathmoveto{\pgfqpoint{2.382947in}{2.837093in}}%
\pgfpathlineto{\pgfqpoint{2.313028in}{2.875992in}}%
\pgfpathlineto{\pgfqpoint{2.382947in}{2.834895in}}%
\pgfpathlineto{\pgfqpoint{2.452866in}{2.796392in}}%
\pgfpathlineto{\pgfqpoint{2.382947in}{2.837093in}}%
\pgfpathclose%
\pgfusepath{fill}%
\end{pgfscope}%
\begin{pgfscope}%
\pgfpathrectangle{\pgfqpoint{0.637500in}{0.550000in}}{\pgfqpoint{3.850000in}{3.850000in}}%
\pgfusepath{clip}%
\pgfsetbuttcap%
\pgfsetroundjoin%
\definecolor{currentfill}{rgb}{0.897487,0.000000,0.000000}%
\pgfsetfillcolor{currentfill}%
\pgfsetfillopacity{0.800000}%
\pgfsetlinewidth{0.000000pt}%
\definecolor{currentstroke}{rgb}{0.000000,0.000000,0.000000}%
\pgfsetstrokecolor{currentstroke}%
\pgfsetdash{}{0pt}%
\pgfpathmoveto{\pgfqpoint{2.662622in}{2.909338in}}%
\pgfpathlineto{\pgfqpoint{2.592703in}{2.868241in}}%
\pgfpathlineto{\pgfqpoint{2.662622in}{2.907139in}}%
\pgfpathlineto{\pgfqpoint{2.732540in}{2.948236in}}%
\pgfpathlineto{\pgfqpoint{2.662622in}{2.909338in}}%
\pgfpathclose%
\pgfusepath{fill}%
\end{pgfscope}%
\begin{pgfscope}%
\pgfpathrectangle{\pgfqpoint{0.637500in}{0.550000in}}{\pgfqpoint{3.850000in}{3.850000in}}%
\pgfusepath{clip}%
\pgfsetbuttcap%
\pgfsetroundjoin%
\definecolor{currentfill}{rgb}{0.897487,0.000000,0.000000}%
\pgfsetfillcolor{currentfill}%
\pgfsetfillopacity{0.800000}%
\pgfsetlinewidth{0.000000pt}%
\definecolor{currentstroke}{rgb}{0.000000,0.000000,0.000000}%
\pgfsetstrokecolor{currentstroke}%
\pgfsetdash{}{0pt}%
\pgfpathmoveto{\pgfqpoint{2.243110in}{2.917089in}}%
\pgfpathlineto{\pgfqpoint{2.173191in}{2.955987in}}%
\pgfpathlineto{\pgfqpoint{2.243110in}{2.914890in}}%
\pgfpathlineto{\pgfqpoint{2.313028in}{2.875992in}}%
\pgfpathlineto{\pgfqpoint{2.243110in}{2.917089in}}%
\pgfpathclose%
\pgfusepath{fill}%
\end{pgfscope}%
\begin{pgfscope}%
\pgfpathrectangle{\pgfqpoint{0.637500in}{0.550000in}}{\pgfqpoint{3.850000in}{3.850000in}}%
\pgfusepath{clip}%
\pgfsetbuttcap%
\pgfsetroundjoin%
\definecolor{currentfill}{rgb}{0.898098,0.000000,0.000000}%
\pgfsetfillcolor{currentfill}%
\pgfsetfillopacity{0.800000}%
\pgfsetlinewidth{0.000000pt}%
\definecolor{currentstroke}{rgb}{0.000000,0.000000,0.000000}%
\pgfsetstrokecolor{currentstroke}%
\pgfsetdash{}{0pt}%
\pgfpathmoveto{\pgfqpoint{2.452866in}{2.796392in}}%
\pgfpathlineto{\pgfqpoint{2.382947in}{2.834895in}}%
\pgfpathlineto{\pgfqpoint{2.452866in}{2.794193in}}%
\pgfpathlineto{\pgfqpoint{2.522784in}{2.827144in}}%
\pgfpathlineto{\pgfqpoint{2.452866in}{2.796392in}}%
\pgfpathclose%
\pgfusepath{fill}%
\end{pgfscope}%
\begin{pgfscope}%
\pgfpathrectangle{\pgfqpoint{0.637500in}{0.550000in}}{\pgfqpoint{3.850000in}{3.850000in}}%
\pgfusepath{clip}%
\pgfsetbuttcap%
\pgfsetroundjoin%
\definecolor{currentfill}{rgb}{0.897487,0.000000,0.000000}%
\pgfsetfillcolor{currentfill}%
\pgfsetfillopacity{0.800000}%
\pgfsetlinewidth{0.000000pt}%
\definecolor{currentstroke}{rgb}{0.000000,0.000000,0.000000}%
\pgfsetstrokecolor{currentstroke}%
\pgfsetdash{}{0pt}%
\pgfpathmoveto{\pgfqpoint{2.592703in}{2.868241in}}%
\pgfpathlineto{\pgfqpoint{2.522784in}{2.827144in}}%
\pgfpathlineto{\pgfqpoint{2.592703in}{2.866042in}}%
\pgfpathlineto{\pgfqpoint{2.662622in}{2.907139in}}%
\pgfpathlineto{\pgfqpoint{2.592703in}{2.868241in}}%
\pgfpathclose%
\pgfusepath{fill}%
\end{pgfscope}%
\begin{pgfscope}%
\pgfpathrectangle{\pgfqpoint{0.637500in}{0.550000in}}{\pgfqpoint{3.850000in}{3.850000in}}%
\pgfusepath{clip}%
\pgfsetbuttcap%
\pgfsetroundjoin%
\definecolor{currentfill}{rgb}{0.897487,0.000000,0.000000}%
\pgfsetfillcolor{currentfill}%
\pgfsetfillopacity{0.800000}%
\pgfsetlinewidth{0.000000pt}%
\definecolor{currentstroke}{rgb}{0.000000,0.000000,0.000000}%
\pgfsetstrokecolor{currentstroke}%
\pgfsetdash{}{0pt}%
\pgfpathmoveto{\pgfqpoint{2.802459in}{2.989333in}}%
\pgfpathlineto{\pgfqpoint{2.732540in}{2.948236in}}%
\pgfpathlineto{\pgfqpoint{2.802459in}{2.987135in}}%
\pgfpathlineto{\pgfqpoint{2.872378in}{3.028232in}}%
\pgfpathlineto{\pgfqpoint{2.802459in}{2.989333in}}%
\pgfpathclose%
\pgfusepath{fill}%
\end{pgfscope}%
\begin{pgfscope}%
\pgfpathrectangle{\pgfqpoint{0.637500in}{0.550000in}}{\pgfqpoint{3.850000in}{3.850000in}}%
\pgfusepath{clip}%
\pgfsetbuttcap%
\pgfsetroundjoin%
\definecolor{currentfill}{rgb}{0.897487,0.000000,0.000000}%
\pgfsetfillcolor{currentfill}%
\pgfsetfillopacity{0.800000}%
\pgfsetlinewidth{0.000000pt}%
\definecolor{currentstroke}{rgb}{0.000000,0.000000,0.000000}%
\pgfsetstrokecolor{currentstroke}%
\pgfsetdash{}{0pt}%
\pgfpathmoveto{\pgfqpoint{2.313028in}{2.875992in}}%
\pgfpathlineto{\pgfqpoint{2.243110in}{2.914890in}}%
\pgfpathlineto{\pgfqpoint{2.313028in}{2.873793in}}%
\pgfpathlineto{\pgfqpoint{2.382947in}{2.834895in}}%
\pgfpathlineto{\pgfqpoint{2.313028in}{2.875992in}}%
\pgfpathclose%
\pgfusepath{fill}%
\end{pgfscope}%
\begin{pgfscope}%
\pgfpathrectangle{\pgfqpoint{0.637500in}{0.550000in}}{\pgfqpoint{3.850000in}{3.850000in}}%
\pgfusepath{clip}%
\pgfsetbuttcap%
\pgfsetroundjoin%
\definecolor{currentfill}{rgb}{0.909646,0.000000,0.000000}%
\pgfsetfillcolor{currentfill}%
\pgfsetfillopacity{0.800000}%
\pgfsetlinewidth{0.000000pt}%
\definecolor{currentstroke}{rgb}{0.000000,0.000000,0.000000}%
\pgfsetstrokecolor{currentstroke}%
\pgfsetdash{}{0pt}%
\pgfpathmoveto{\pgfqpoint{2.522784in}{2.827144in}}%
\pgfpathlineto{\pgfqpoint{2.452866in}{2.794193in}}%
\pgfpathlineto{\pgfqpoint{2.522784in}{2.824945in}}%
\pgfpathlineto{\pgfqpoint{2.592703in}{2.866042in}}%
\pgfpathlineto{\pgfqpoint{2.522784in}{2.827144in}}%
\pgfpathclose%
\pgfusepath{fill}%
\end{pgfscope}%
\begin{pgfscope}%
\pgfpathrectangle{\pgfqpoint{0.637500in}{0.550000in}}{\pgfqpoint{3.850000in}{3.850000in}}%
\pgfusepath{clip}%
\pgfsetbuttcap%
\pgfsetroundjoin%
\definecolor{currentfill}{rgb}{0.897487,0.000000,0.000000}%
\pgfsetfillcolor{currentfill}%
\pgfsetfillopacity{0.800000}%
\pgfsetlinewidth{0.000000pt}%
\definecolor{currentstroke}{rgb}{0.000000,0.000000,0.000000}%
\pgfsetstrokecolor{currentstroke}%
\pgfsetdash{}{0pt}%
\pgfpathmoveto{\pgfqpoint{2.732540in}{2.948236in}}%
\pgfpathlineto{\pgfqpoint{2.662622in}{2.907139in}}%
\pgfpathlineto{\pgfqpoint{2.732540in}{2.946038in}}%
\pgfpathlineto{\pgfqpoint{2.802459in}{2.987135in}}%
\pgfpathlineto{\pgfqpoint{2.732540in}{2.948236in}}%
\pgfpathclose%
\pgfusepath{fill}%
\end{pgfscope}%
\begin{pgfscope}%
\pgfpathrectangle{\pgfqpoint{0.637500in}{0.550000in}}{\pgfqpoint{3.850000in}{3.850000in}}%
\pgfusepath{clip}%
\pgfsetbuttcap%
\pgfsetroundjoin%
\definecolor{currentfill}{rgb}{0.897487,0.000000,0.000000}%
\pgfsetfillcolor{currentfill}%
\pgfsetfillopacity{0.800000}%
\pgfsetlinewidth{0.000000pt}%
\definecolor{currentstroke}{rgb}{0.000000,0.000000,0.000000}%
\pgfsetstrokecolor{currentstroke}%
\pgfsetdash{}{0pt}%
\pgfpathmoveto{\pgfqpoint{2.382947in}{2.834895in}}%
\pgfpathlineto{\pgfqpoint{2.313028in}{2.873793in}}%
\pgfpathlineto{\pgfqpoint{2.382947in}{2.832696in}}%
\pgfpathlineto{\pgfqpoint{2.452866in}{2.794193in}}%
\pgfpathlineto{\pgfqpoint{2.382947in}{2.834895in}}%
\pgfpathclose%
\pgfusepath{fill}%
\end{pgfscope}%
\begin{pgfscope}%
\pgfpathrectangle{\pgfqpoint{0.637500in}{0.550000in}}{\pgfqpoint{3.850000in}{3.850000in}}%
\pgfusepath{clip}%
\pgfsetbuttcap%
\pgfsetroundjoin%
\definecolor{currentfill}{rgb}{0.897487,0.000000,0.000000}%
\pgfsetfillcolor{currentfill}%
\pgfsetfillopacity{0.800000}%
\pgfsetlinewidth{0.000000pt}%
\definecolor{currentstroke}{rgb}{0.000000,0.000000,0.000000}%
\pgfsetstrokecolor{currentstroke}%
\pgfsetdash{}{0pt}%
\pgfpathmoveto{\pgfqpoint{2.173191in}{2.955987in}}%
\pgfpathlineto{\pgfqpoint{2.103272in}{2.994886in}}%
\pgfpathlineto{\pgfqpoint{2.173191in}{2.953788in}}%
\pgfpathlineto{\pgfqpoint{2.243110in}{2.914890in}}%
\pgfpathlineto{\pgfqpoint{2.173191in}{2.955987in}}%
\pgfpathclose%
\pgfusepath{fill}%
\end{pgfscope}%
\begin{pgfscope}%
\pgfpathrectangle{\pgfqpoint{0.637500in}{0.550000in}}{\pgfqpoint{3.850000in}{3.850000in}}%
\pgfusepath{clip}%
\pgfsetbuttcap%
\pgfsetroundjoin%
\definecolor{currentfill}{rgb}{0.897487,0.000000,0.000000}%
\pgfsetfillcolor{currentfill}%
\pgfsetfillopacity{0.800000}%
\pgfsetlinewidth{0.000000pt}%
\definecolor{currentstroke}{rgb}{0.000000,0.000000,0.000000}%
\pgfsetstrokecolor{currentstroke}%
\pgfsetdash{}{0pt}%
\pgfpathmoveto{\pgfqpoint{2.662622in}{2.907139in}}%
\pgfpathlineto{\pgfqpoint{2.592703in}{2.866042in}}%
\pgfpathlineto{\pgfqpoint{2.662622in}{2.904940in}}%
\pgfpathlineto{\pgfqpoint{2.732540in}{2.946038in}}%
\pgfpathlineto{\pgfqpoint{2.662622in}{2.907139in}}%
\pgfpathclose%
\pgfusepath{fill}%
\end{pgfscope}%
\begin{pgfscope}%
\pgfpathrectangle{\pgfqpoint{0.637500in}{0.550000in}}{\pgfqpoint{3.850000in}{3.850000in}}%
\pgfusepath{clip}%
\pgfsetbuttcap%
\pgfsetroundjoin%
\definecolor{currentfill}{rgb}{0.897487,0.000000,0.000000}%
\pgfsetfillcolor{currentfill}%
\pgfsetfillopacity{0.800000}%
\pgfsetlinewidth{0.000000pt}%
\definecolor{currentstroke}{rgb}{0.000000,0.000000,0.000000}%
\pgfsetstrokecolor{currentstroke}%
\pgfsetdash{}{0pt}%
\pgfpathmoveto{\pgfqpoint{2.872378in}{3.028232in}}%
\pgfpathlineto{\pgfqpoint{2.802459in}{2.987135in}}%
\pgfpathlineto{\pgfqpoint{2.872378in}{3.026033in}}%
\pgfpathlineto{\pgfqpoint{2.942296in}{3.067130in}}%
\pgfpathlineto{\pgfqpoint{2.872378in}{3.028232in}}%
\pgfpathclose%
\pgfusepath{fill}%
\end{pgfscope}%
\begin{pgfscope}%
\pgfpathrectangle{\pgfqpoint{0.637500in}{0.550000in}}{\pgfqpoint{3.850000in}{3.850000in}}%
\pgfusepath{clip}%
\pgfsetbuttcap%
\pgfsetroundjoin%
\definecolor{currentfill}{rgb}{0.897487,0.000000,0.000000}%
\pgfsetfillcolor{currentfill}%
\pgfsetfillopacity{0.800000}%
\pgfsetlinewidth{0.000000pt}%
\definecolor{currentstroke}{rgb}{0.000000,0.000000,0.000000}%
\pgfsetstrokecolor{currentstroke}%
\pgfsetdash{}{0pt}%
\pgfpathmoveto{\pgfqpoint{2.243110in}{2.914890in}}%
\pgfpathlineto{\pgfqpoint{2.173191in}{2.953788in}}%
\pgfpathlineto{\pgfqpoint{2.243110in}{2.912691in}}%
\pgfpathlineto{\pgfqpoint{2.313028in}{2.873793in}}%
\pgfpathlineto{\pgfqpoint{2.243110in}{2.914890in}}%
\pgfpathclose%
\pgfusepath{fill}%
\end{pgfscope}%
\begin{pgfscope}%
\pgfpathrectangle{\pgfqpoint{0.637500in}{0.550000in}}{\pgfqpoint{3.850000in}{3.850000in}}%
\pgfusepath{clip}%
\pgfsetbuttcap%
\pgfsetroundjoin%
\definecolor{currentfill}{rgb}{0.898098,0.000000,0.000000}%
\pgfsetfillcolor{currentfill}%
\pgfsetfillopacity{0.800000}%
\pgfsetlinewidth{0.000000pt}%
\definecolor{currentstroke}{rgb}{0.000000,0.000000,0.000000}%
\pgfsetstrokecolor{currentstroke}%
\pgfsetdash{}{0pt}%
\pgfpathmoveto{\pgfqpoint{2.452866in}{2.794193in}}%
\pgfpathlineto{\pgfqpoint{2.382947in}{2.832696in}}%
\pgfpathlineto{\pgfqpoint{2.452866in}{2.791994in}}%
\pgfpathlineto{\pgfqpoint{2.522784in}{2.824945in}}%
\pgfpathlineto{\pgfqpoint{2.452866in}{2.794193in}}%
\pgfpathclose%
\pgfusepath{fill}%
\end{pgfscope}%
\begin{pgfscope}%
\pgfpathrectangle{\pgfqpoint{0.637500in}{0.550000in}}{\pgfqpoint{3.850000in}{3.850000in}}%
\pgfusepath{clip}%
\pgfsetbuttcap%
\pgfsetroundjoin%
\definecolor{currentfill}{rgb}{0.897487,0.000000,0.000000}%
\pgfsetfillcolor{currentfill}%
\pgfsetfillopacity{0.800000}%
\pgfsetlinewidth{0.000000pt}%
\definecolor{currentstroke}{rgb}{0.000000,0.000000,0.000000}%
\pgfsetstrokecolor{currentstroke}%
\pgfsetdash{}{0pt}%
\pgfpathmoveto{\pgfqpoint{2.592703in}{2.866042in}}%
\pgfpathlineto{\pgfqpoint{2.522784in}{2.824945in}}%
\pgfpathlineto{\pgfqpoint{2.592703in}{2.863843in}}%
\pgfpathlineto{\pgfqpoint{2.662622in}{2.904940in}}%
\pgfpathlineto{\pgfqpoint{2.592703in}{2.866042in}}%
\pgfpathclose%
\pgfusepath{fill}%
\end{pgfscope}%
\begin{pgfscope}%
\pgfpathrectangle{\pgfqpoint{0.637500in}{0.550000in}}{\pgfqpoint{3.850000in}{3.850000in}}%
\pgfusepath{clip}%
\pgfsetbuttcap%
\pgfsetroundjoin%
\definecolor{currentfill}{rgb}{0.897487,0.000000,0.000000}%
\pgfsetfillcolor{currentfill}%
\pgfsetfillopacity{0.800000}%
\pgfsetlinewidth{0.000000pt}%
\definecolor{currentstroke}{rgb}{0.000000,0.000000,0.000000}%
\pgfsetstrokecolor{currentstroke}%
\pgfsetdash{}{0pt}%
\pgfpathmoveto{\pgfqpoint{2.802459in}{2.987135in}}%
\pgfpathlineto{\pgfqpoint{2.732540in}{2.946038in}}%
\pgfpathlineto{\pgfqpoint{2.802459in}{2.984936in}}%
\pgfpathlineto{\pgfqpoint{2.872378in}{3.026033in}}%
\pgfpathlineto{\pgfqpoint{2.802459in}{2.987135in}}%
\pgfpathclose%
\pgfusepath{fill}%
\end{pgfscope}%
\begin{pgfscope}%
\pgfpathrectangle{\pgfqpoint{0.637500in}{0.550000in}}{\pgfqpoint{3.850000in}{3.850000in}}%
\pgfusepath{clip}%
\pgfsetbuttcap%
\pgfsetroundjoin%
\definecolor{currentfill}{rgb}{0.897487,0.000000,0.000000}%
\pgfsetfillcolor{currentfill}%
\pgfsetfillopacity{0.800000}%
\pgfsetlinewidth{0.000000pt}%
\definecolor{currentstroke}{rgb}{0.000000,0.000000,0.000000}%
\pgfsetstrokecolor{currentstroke}%
\pgfsetdash{}{0pt}%
\pgfpathmoveto{\pgfqpoint{2.313028in}{2.873793in}}%
\pgfpathlineto{\pgfqpoint{2.243110in}{2.912691in}}%
\pgfpathlineto{\pgfqpoint{2.313028in}{2.871594in}}%
\pgfpathlineto{\pgfqpoint{2.382947in}{2.832696in}}%
\pgfpathlineto{\pgfqpoint{2.313028in}{2.873793in}}%
\pgfpathclose%
\pgfusepath{fill}%
\end{pgfscope}%
\begin{pgfscope}%
\pgfpathrectangle{\pgfqpoint{0.637500in}{0.550000in}}{\pgfqpoint{3.850000in}{3.850000in}}%
\pgfusepath{clip}%
\pgfsetbuttcap%
\pgfsetroundjoin%
\definecolor{currentfill}{rgb}{0.897487,0.000000,0.000000}%
\pgfsetfillcolor{currentfill}%
\pgfsetfillopacity{0.800000}%
\pgfsetlinewidth{0.000000pt}%
\definecolor{currentstroke}{rgb}{0.000000,0.000000,0.000000}%
\pgfsetstrokecolor{currentstroke}%
\pgfsetdash{}{0pt}%
\pgfpathmoveto{\pgfqpoint{2.103272in}{2.994886in}}%
\pgfpathlineto{\pgfqpoint{2.033354in}{3.033784in}}%
\pgfpathlineto{\pgfqpoint{2.103272in}{2.992687in}}%
\pgfpathlineto{\pgfqpoint{2.173191in}{2.953788in}}%
\pgfpathlineto{\pgfqpoint{2.103272in}{2.994886in}}%
\pgfpathclose%
\pgfusepath{fill}%
\end{pgfscope}%
\begin{pgfscope}%
\pgfpathrectangle{\pgfqpoint{0.637500in}{0.550000in}}{\pgfqpoint{3.850000in}{3.850000in}}%
\pgfusepath{clip}%
\pgfsetbuttcap%
\pgfsetroundjoin%
\definecolor{currentfill}{rgb}{0.909646,0.000000,0.000000}%
\pgfsetfillcolor{currentfill}%
\pgfsetfillopacity{0.800000}%
\pgfsetlinewidth{0.000000pt}%
\definecolor{currentstroke}{rgb}{0.000000,0.000000,0.000000}%
\pgfsetstrokecolor{currentstroke}%
\pgfsetdash{}{0pt}%
\pgfpathmoveto{\pgfqpoint{2.522784in}{2.824945in}}%
\pgfpathlineto{\pgfqpoint{2.452866in}{2.791994in}}%
\pgfpathlineto{\pgfqpoint{2.522784in}{2.822746in}}%
\pgfpathlineto{\pgfqpoint{2.592703in}{2.863843in}}%
\pgfpathlineto{\pgfqpoint{2.522784in}{2.824945in}}%
\pgfpathclose%
\pgfusepath{fill}%
\end{pgfscope}%
\begin{pgfscope}%
\pgfpathrectangle{\pgfqpoint{0.637500in}{0.550000in}}{\pgfqpoint{3.850000in}{3.850000in}}%
\pgfusepath{clip}%
\pgfsetbuttcap%
\pgfsetroundjoin%
\definecolor{currentfill}{rgb}{0.897487,0.000000,0.000000}%
\pgfsetfillcolor{currentfill}%
\pgfsetfillopacity{0.800000}%
\pgfsetlinewidth{0.000000pt}%
\definecolor{currentstroke}{rgb}{0.000000,0.000000,0.000000}%
\pgfsetstrokecolor{currentstroke}%
\pgfsetdash{}{0pt}%
\pgfpathmoveto{\pgfqpoint{2.732540in}{2.946038in}}%
\pgfpathlineto{\pgfqpoint{2.662622in}{2.904940in}}%
\pgfpathlineto{\pgfqpoint{2.732540in}{2.943839in}}%
\pgfpathlineto{\pgfqpoint{2.802459in}{2.984936in}}%
\pgfpathlineto{\pgfqpoint{2.732540in}{2.946038in}}%
\pgfpathclose%
\pgfusepath{fill}%
\end{pgfscope}%
\begin{pgfscope}%
\pgfpathrectangle{\pgfqpoint{0.637500in}{0.550000in}}{\pgfqpoint{3.850000in}{3.850000in}}%
\pgfusepath{clip}%
\pgfsetbuttcap%
\pgfsetroundjoin%
\definecolor{currentfill}{rgb}{0.897487,0.000000,0.000000}%
\pgfsetfillcolor{currentfill}%
\pgfsetfillopacity{0.800000}%
\pgfsetlinewidth{0.000000pt}%
\definecolor{currentstroke}{rgb}{0.000000,0.000000,0.000000}%
\pgfsetstrokecolor{currentstroke}%
\pgfsetdash{}{0pt}%
\pgfpathmoveto{\pgfqpoint{2.942296in}{3.067130in}}%
\pgfpathlineto{\pgfqpoint{2.872378in}{3.026033in}}%
\pgfpathlineto{\pgfqpoint{2.942296in}{3.064931in}}%
\pgfpathlineto{\pgfqpoint{3.012215in}{3.106029in}}%
\pgfpathlineto{\pgfqpoint{2.942296in}{3.067130in}}%
\pgfpathclose%
\pgfusepath{fill}%
\end{pgfscope}%
\begin{pgfscope}%
\pgfpathrectangle{\pgfqpoint{0.637500in}{0.550000in}}{\pgfqpoint{3.850000in}{3.850000in}}%
\pgfusepath{clip}%
\pgfsetbuttcap%
\pgfsetroundjoin%
\definecolor{currentfill}{rgb}{0.897487,0.000000,0.000000}%
\pgfsetfillcolor{currentfill}%
\pgfsetfillopacity{0.800000}%
\pgfsetlinewidth{0.000000pt}%
\definecolor{currentstroke}{rgb}{0.000000,0.000000,0.000000}%
\pgfsetstrokecolor{currentstroke}%
\pgfsetdash{}{0pt}%
\pgfpathmoveto{\pgfqpoint{2.382947in}{2.832696in}}%
\pgfpathlineto{\pgfqpoint{2.313028in}{2.871594in}}%
\pgfpathlineto{\pgfqpoint{2.382947in}{2.830497in}}%
\pgfpathlineto{\pgfqpoint{2.452866in}{2.791994in}}%
\pgfpathlineto{\pgfqpoint{2.382947in}{2.832696in}}%
\pgfpathclose%
\pgfusepath{fill}%
\end{pgfscope}%
\begin{pgfscope}%
\pgfpathrectangle{\pgfqpoint{0.637500in}{0.550000in}}{\pgfqpoint{3.850000in}{3.850000in}}%
\pgfusepath{clip}%
\pgfsetbuttcap%
\pgfsetroundjoin%
\definecolor{currentfill}{rgb}{0.897487,0.000000,0.000000}%
\pgfsetfillcolor{currentfill}%
\pgfsetfillopacity{0.800000}%
\pgfsetlinewidth{0.000000pt}%
\definecolor{currentstroke}{rgb}{0.000000,0.000000,0.000000}%
\pgfsetstrokecolor{currentstroke}%
\pgfsetdash{}{0pt}%
\pgfpathmoveto{\pgfqpoint{2.173191in}{2.953788in}}%
\pgfpathlineto{\pgfqpoint{2.103272in}{2.992687in}}%
\pgfpathlineto{\pgfqpoint{2.173191in}{2.951590in}}%
\pgfpathlineto{\pgfqpoint{2.243110in}{2.912691in}}%
\pgfpathlineto{\pgfqpoint{2.173191in}{2.953788in}}%
\pgfpathclose%
\pgfusepath{fill}%
\end{pgfscope}%
\begin{pgfscope}%
\pgfpathrectangle{\pgfqpoint{0.637500in}{0.550000in}}{\pgfqpoint{3.850000in}{3.850000in}}%
\pgfusepath{clip}%
\pgfsetbuttcap%
\pgfsetroundjoin%
\definecolor{currentfill}{rgb}{0.897487,0.000000,0.000000}%
\pgfsetfillcolor{currentfill}%
\pgfsetfillopacity{0.800000}%
\pgfsetlinewidth{0.000000pt}%
\definecolor{currentstroke}{rgb}{0.000000,0.000000,0.000000}%
\pgfsetstrokecolor{currentstroke}%
\pgfsetdash{}{0pt}%
\pgfpathmoveto{\pgfqpoint{2.662622in}{2.904940in}}%
\pgfpathlineto{\pgfqpoint{2.592703in}{2.863843in}}%
\pgfpathlineto{\pgfqpoint{2.662622in}{2.902742in}}%
\pgfpathlineto{\pgfqpoint{2.732540in}{2.943839in}}%
\pgfpathlineto{\pgfqpoint{2.662622in}{2.904940in}}%
\pgfpathclose%
\pgfusepath{fill}%
\end{pgfscope}%
\begin{pgfscope}%
\pgfpathrectangle{\pgfqpoint{0.637500in}{0.550000in}}{\pgfqpoint{3.850000in}{3.850000in}}%
\pgfusepath{clip}%
\pgfsetbuttcap%
\pgfsetroundjoin%
\definecolor{currentfill}{rgb}{0.897487,0.000000,0.000000}%
\pgfsetfillcolor{currentfill}%
\pgfsetfillopacity{0.800000}%
\pgfsetlinewidth{0.000000pt}%
\definecolor{currentstroke}{rgb}{0.000000,0.000000,0.000000}%
\pgfsetstrokecolor{currentstroke}%
\pgfsetdash{}{0pt}%
\pgfpathmoveto{\pgfqpoint{2.872378in}{3.026033in}}%
\pgfpathlineto{\pgfqpoint{2.802459in}{2.984936in}}%
\pgfpathlineto{\pgfqpoint{2.872378in}{3.023834in}}%
\pgfpathlineto{\pgfqpoint{2.942296in}{3.064931in}}%
\pgfpathlineto{\pgfqpoint{2.872378in}{3.026033in}}%
\pgfpathclose%
\pgfusepath{fill}%
\end{pgfscope}%
\begin{pgfscope}%
\pgfpathrectangle{\pgfqpoint{0.637500in}{0.550000in}}{\pgfqpoint{3.850000in}{3.850000in}}%
\pgfusepath{clip}%
\pgfsetbuttcap%
\pgfsetroundjoin%
\definecolor{currentfill}{rgb}{0.897487,0.000000,0.000000}%
\pgfsetfillcolor{currentfill}%
\pgfsetfillopacity{0.800000}%
\pgfsetlinewidth{0.000000pt}%
\definecolor{currentstroke}{rgb}{0.000000,0.000000,0.000000}%
\pgfsetstrokecolor{currentstroke}%
\pgfsetdash{}{0pt}%
\pgfpathmoveto{\pgfqpoint{2.243110in}{2.912691in}}%
\pgfpathlineto{\pgfqpoint{2.173191in}{2.951590in}}%
\pgfpathlineto{\pgfqpoint{2.243110in}{2.910493in}}%
\pgfpathlineto{\pgfqpoint{2.313028in}{2.871594in}}%
\pgfpathlineto{\pgfqpoint{2.243110in}{2.912691in}}%
\pgfpathclose%
\pgfusepath{fill}%
\end{pgfscope}%
\begin{pgfscope}%
\pgfpathrectangle{\pgfqpoint{0.637500in}{0.550000in}}{\pgfqpoint{3.850000in}{3.850000in}}%
\pgfusepath{clip}%
\pgfsetbuttcap%
\pgfsetroundjoin%
\definecolor{currentfill}{rgb}{0.897487,0.000000,0.000000}%
\pgfsetfillcolor{currentfill}%
\pgfsetfillopacity{0.800000}%
\pgfsetlinewidth{0.000000pt}%
\definecolor{currentstroke}{rgb}{0.000000,0.000000,0.000000}%
\pgfsetstrokecolor{currentstroke}%
\pgfsetdash{}{0pt}%
\pgfpathmoveto{\pgfqpoint{2.033354in}{3.033784in}}%
\pgfpathlineto{\pgfqpoint{1.963435in}{3.072682in}}%
\pgfpathlineto{\pgfqpoint{2.033354in}{3.031585in}}%
\pgfpathlineto{\pgfqpoint{2.103272in}{2.992687in}}%
\pgfpathlineto{\pgfqpoint{2.033354in}{3.033784in}}%
\pgfpathclose%
\pgfusepath{fill}%
\end{pgfscope}%
\begin{pgfscope}%
\pgfpathrectangle{\pgfqpoint{0.637500in}{0.550000in}}{\pgfqpoint{3.850000in}{3.850000in}}%
\pgfusepath{clip}%
\pgfsetbuttcap%
\pgfsetroundjoin%
\definecolor{currentfill}{rgb}{0.898098,0.000000,0.000000}%
\pgfsetfillcolor{currentfill}%
\pgfsetfillopacity{0.800000}%
\pgfsetlinewidth{0.000000pt}%
\definecolor{currentstroke}{rgb}{0.000000,0.000000,0.000000}%
\pgfsetstrokecolor{currentstroke}%
\pgfsetdash{}{0pt}%
\pgfpathmoveto{\pgfqpoint{2.452866in}{2.791994in}}%
\pgfpathlineto{\pgfqpoint{2.382947in}{2.830497in}}%
\pgfpathlineto{\pgfqpoint{2.452866in}{2.789795in}}%
\pgfpathlineto{\pgfqpoint{2.522784in}{2.822746in}}%
\pgfpathlineto{\pgfqpoint{2.452866in}{2.791994in}}%
\pgfpathclose%
\pgfusepath{fill}%
\end{pgfscope}%
\begin{pgfscope}%
\pgfpathrectangle{\pgfqpoint{0.637500in}{0.550000in}}{\pgfqpoint{3.850000in}{3.850000in}}%
\pgfusepath{clip}%
\pgfsetbuttcap%
\pgfsetroundjoin%
\definecolor{currentfill}{rgb}{0.897487,0.000000,0.000000}%
\pgfsetfillcolor{currentfill}%
\pgfsetfillopacity{0.800000}%
\pgfsetlinewidth{0.000000pt}%
\definecolor{currentstroke}{rgb}{0.000000,0.000000,0.000000}%
\pgfsetstrokecolor{currentstroke}%
\pgfsetdash{}{0pt}%
\pgfpathmoveto{\pgfqpoint{2.592703in}{2.863843in}}%
\pgfpathlineto{\pgfqpoint{2.522784in}{2.822746in}}%
\pgfpathlineto{\pgfqpoint{2.592703in}{2.861644in}}%
\pgfpathlineto{\pgfqpoint{2.662622in}{2.902742in}}%
\pgfpathlineto{\pgfqpoint{2.592703in}{2.863843in}}%
\pgfpathclose%
\pgfusepath{fill}%
\end{pgfscope}%
\begin{pgfscope}%
\pgfpathrectangle{\pgfqpoint{0.637500in}{0.550000in}}{\pgfqpoint{3.850000in}{3.850000in}}%
\pgfusepath{clip}%
\pgfsetbuttcap%
\pgfsetroundjoin%
\definecolor{currentfill}{rgb}{0.897487,0.000000,0.000000}%
\pgfsetfillcolor{currentfill}%
\pgfsetfillopacity{0.800000}%
\pgfsetlinewidth{0.000000pt}%
\definecolor{currentstroke}{rgb}{0.000000,0.000000,0.000000}%
\pgfsetstrokecolor{currentstroke}%
\pgfsetdash{}{0pt}%
\pgfpathmoveto{\pgfqpoint{2.802459in}{2.984936in}}%
\pgfpathlineto{\pgfqpoint{2.732540in}{2.943839in}}%
\pgfpathlineto{\pgfqpoint{2.802459in}{2.982737in}}%
\pgfpathlineto{\pgfqpoint{2.872378in}{3.023834in}}%
\pgfpathlineto{\pgfqpoint{2.802459in}{2.984936in}}%
\pgfpathclose%
\pgfusepath{fill}%
\end{pgfscope}%
\begin{pgfscope}%
\pgfpathrectangle{\pgfqpoint{0.637500in}{0.550000in}}{\pgfqpoint{3.850000in}{3.850000in}}%
\pgfusepath{clip}%
\pgfsetbuttcap%
\pgfsetroundjoin%
\definecolor{currentfill}{rgb}{0.897487,0.000000,0.000000}%
\pgfsetfillcolor{currentfill}%
\pgfsetfillopacity{0.800000}%
\pgfsetlinewidth{0.000000pt}%
\definecolor{currentstroke}{rgb}{0.000000,0.000000,0.000000}%
\pgfsetstrokecolor{currentstroke}%
\pgfsetdash{}{0pt}%
\pgfpathmoveto{\pgfqpoint{3.012215in}{3.106029in}}%
\pgfpathlineto{\pgfqpoint{2.942296in}{3.064931in}}%
\pgfpathlineto{\pgfqpoint{3.012215in}{3.103830in}}%
\pgfpathlineto{\pgfqpoint{3.082134in}{3.144927in}}%
\pgfpathlineto{\pgfqpoint{3.012215in}{3.106029in}}%
\pgfpathclose%
\pgfusepath{fill}%
\end{pgfscope}%
\begin{pgfscope}%
\pgfpathrectangle{\pgfqpoint{0.637500in}{0.550000in}}{\pgfqpoint{3.850000in}{3.850000in}}%
\pgfusepath{clip}%
\pgfsetbuttcap%
\pgfsetroundjoin%
\definecolor{currentfill}{rgb}{0.897487,0.000000,0.000000}%
\pgfsetfillcolor{currentfill}%
\pgfsetfillopacity{0.800000}%
\pgfsetlinewidth{0.000000pt}%
\definecolor{currentstroke}{rgb}{0.000000,0.000000,0.000000}%
\pgfsetstrokecolor{currentstroke}%
\pgfsetdash{}{0pt}%
\pgfpathmoveto{\pgfqpoint{2.313028in}{2.871594in}}%
\pgfpathlineto{\pgfqpoint{2.243110in}{2.910493in}}%
\pgfpathlineto{\pgfqpoint{2.313028in}{2.869395in}}%
\pgfpathlineto{\pgfqpoint{2.382947in}{2.830497in}}%
\pgfpathlineto{\pgfqpoint{2.313028in}{2.871594in}}%
\pgfpathclose%
\pgfusepath{fill}%
\end{pgfscope}%
\begin{pgfscope}%
\pgfpathrectangle{\pgfqpoint{0.637500in}{0.550000in}}{\pgfqpoint{3.850000in}{3.850000in}}%
\pgfusepath{clip}%
\pgfsetbuttcap%
\pgfsetroundjoin%
\definecolor{currentfill}{rgb}{0.897487,0.000000,0.000000}%
\pgfsetfillcolor{currentfill}%
\pgfsetfillopacity{0.800000}%
\pgfsetlinewidth{0.000000pt}%
\definecolor{currentstroke}{rgb}{0.000000,0.000000,0.000000}%
\pgfsetstrokecolor{currentstroke}%
\pgfsetdash{}{0pt}%
\pgfpathmoveto{\pgfqpoint{2.103272in}{2.992687in}}%
\pgfpathlineto{\pgfqpoint{2.033354in}{3.031585in}}%
\pgfpathlineto{\pgfqpoint{2.103272in}{2.990488in}}%
\pgfpathlineto{\pgfqpoint{2.173191in}{2.951590in}}%
\pgfpathlineto{\pgfqpoint{2.103272in}{2.992687in}}%
\pgfpathclose%
\pgfusepath{fill}%
\end{pgfscope}%
\begin{pgfscope}%
\pgfpathrectangle{\pgfqpoint{0.637500in}{0.550000in}}{\pgfqpoint{3.850000in}{3.850000in}}%
\pgfusepath{clip}%
\pgfsetbuttcap%
\pgfsetroundjoin%
\definecolor{currentfill}{rgb}{0.909646,0.000000,0.000000}%
\pgfsetfillcolor{currentfill}%
\pgfsetfillopacity{0.800000}%
\pgfsetlinewidth{0.000000pt}%
\definecolor{currentstroke}{rgb}{0.000000,0.000000,0.000000}%
\pgfsetstrokecolor{currentstroke}%
\pgfsetdash{}{0pt}%
\pgfpathmoveto{\pgfqpoint{2.522784in}{2.822746in}}%
\pgfpathlineto{\pgfqpoint{2.452866in}{2.789795in}}%
\pgfpathlineto{\pgfqpoint{2.522784in}{2.820547in}}%
\pgfpathlineto{\pgfqpoint{2.592703in}{2.861644in}}%
\pgfpathlineto{\pgfqpoint{2.522784in}{2.822746in}}%
\pgfpathclose%
\pgfusepath{fill}%
\end{pgfscope}%
\begin{pgfscope}%
\pgfpathrectangle{\pgfqpoint{0.637500in}{0.550000in}}{\pgfqpoint{3.850000in}{3.850000in}}%
\pgfusepath{clip}%
\pgfsetbuttcap%
\pgfsetroundjoin%
\definecolor{currentfill}{rgb}{0.897487,0.000000,0.000000}%
\pgfsetfillcolor{currentfill}%
\pgfsetfillopacity{0.800000}%
\pgfsetlinewidth{0.000000pt}%
\definecolor{currentstroke}{rgb}{0.000000,0.000000,0.000000}%
\pgfsetstrokecolor{currentstroke}%
\pgfsetdash{}{0pt}%
\pgfpathmoveto{\pgfqpoint{2.732540in}{2.943839in}}%
\pgfpathlineto{\pgfqpoint{2.662622in}{2.902742in}}%
\pgfpathlineto{\pgfqpoint{2.732540in}{2.941640in}}%
\pgfpathlineto{\pgfqpoint{2.802459in}{2.982737in}}%
\pgfpathlineto{\pgfqpoint{2.732540in}{2.943839in}}%
\pgfpathclose%
\pgfusepath{fill}%
\end{pgfscope}%
\begin{pgfscope}%
\pgfpathrectangle{\pgfqpoint{0.637500in}{0.550000in}}{\pgfqpoint{3.850000in}{3.850000in}}%
\pgfusepath{clip}%
\pgfsetbuttcap%
\pgfsetroundjoin%
\definecolor{currentfill}{rgb}{0.897487,0.000000,0.000000}%
\pgfsetfillcolor{currentfill}%
\pgfsetfillopacity{0.800000}%
\pgfsetlinewidth{0.000000pt}%
\definecolor{currentstroke}{rgb}{0.000000,0.000000,0.000000}%
\pgfsetstrokecolor{currentstroke}%
\pgfsetdash{}{0pt}%
\pgfpathmoveto{\pgfqpoint{2.942296in}{3.064931in}}%
\pgfpathlineto{\pgfqpoint{2.872378in}{3.023834in}}%
\pgfpathlineto{\pgfqpoint{2.942296in}{3.062733in}}%
\pgfpathlineto{\pgfqpoint{3.012215in}{3.103830in}}%
\pgfpathlineto{\pgfqpoint{2.942296in}{3.064931in}}%
\pgfpathclose%
\pgfusepath{fill}%
\end{pgfscope}%
\begin{pgfscope}%
\pgfpathrectangle{\pgfqpoint{0.637500in}{0.550000in}}{\pgfqpoint{3.850000in}{3.850000in}}%
\pgfusepath{clip}%
\pgfsetbuttcap%
\pgfsetroundjoin%
\definecolor{currentfill}{rgb}{0.897487,0.000000,0.000000}%
\pgfsetfillcolor{currentfill}%
\pgfsetfillopacity{0.800000}%
\pgfsetlinewidth{0.000000pt}%
\definecolor{currentstroke}{rgb}{0.000000,0.000000,0.000000}%
\pgfsetstrokecolor{currentstroke}%
\pgfsetdash{}{0pt}%
\pgfpathmoveto{\pgfqpoint{2.382947in}{2.830497in}}%
\pgfpathlineto{\pgfqpoint{2.313028in}{2.869395in}}%
\pgfpathlineto{\pgfqpoint{2.382947in}{2.828298in}}%
\pgfpathlineto{\pgfqpoint{2.452866in}{2.789795in}}%
\pgfpathlineto{\pgfqpoint{2.382947in}{2.830497in}}%
\pgfpathclose%
\pgfusepath{fill}%
\end{pgfscope}%
\begin{pgfscope}%
\pgfpathrectangle{\pgfqpoint{0.637500in}{0.550000in}}{\pgfqpoint{3.850000in}{3.850000in}}%
\pgfusepath{clip}%
\pgfsetbuttcap%
\pgfsetroundjoin%
\definecolor{currentfill}{rgb}{0.897487,0.000000,0.000000}%
\pgfsetfillcolor{currentfill}%
\pgfsetfillopacity{0.800000}%
\pgfsetlinewidth{0.000000pt}%
\definecolor{currentstroke}{rgb}{0.000000,0.000000,0.000000}%
\pgfsetstrokecolor{currentstroke}%
\pgfsetdash{}{0pt}%
\pgfpathmoveto{\pgfqpoint{2.173191in}{2.951590in}}%
\pgfpathlineto{\pgfqpoint{2.103272in}{2.990488in}}%
\pgfpathlineto{\pgfqpoint{2.173191in}{2.949391in}}%
\pgfpathlineto{\pgfqpoint{2.243110in}{2.910493in}}%
\pgfpathlineto{\pgfqpoint{2.173191in}{2.951590in}}%
\pgfpathclose%
\pgfusepath{fill}%
\end{pgfscope}%
\begin{pgfscope}%
\pgfpathrectangle{\pgfqpoint{0.637500in}{0.550000in}}{\pgfqpoint{3.850000in}{3.850000in}}%
\pgfusepath{clip}%
\pgfsetbuttcap%
\pgfsetroundjoin%
\definecolor{currentfill}{rgb}{0.897487,0.000000,0.000000}%
\pgfsetfillcolor{currentfill}%
\pgfsetfillopacity{0.800000}%
\pgfsetlinewidth{0.000000pt}%
\definecolor{currentstroke}{rgb}{0.000000,0.000000,0.000000}%
\pgfsetstrokecolor{currentstroke}%
\pgfsetdash{}{0pt}%
\pgfpathmoveto{\pgfqpoint{1.963435in}{3.072682in}}%
\pgfpathlineto{\pgfqpoint{1.893516in}{3.111581in}}%
\pgfpathlineto{\pgfqpoint{1.963435in}{3.070484in}}%
\pgfpathlineto{\pgfqpoint{2.033354in}{3.031585in}}%
\pgfpathlineto{\pgfqpoint{1.963435in}{3.072682in}}%
\pgfpathclose%
\pgfusepath{fill}%
\end{pgfscope}%
\begin{pgfscope}%
\pgfpathrectangle{\pgfqpoint{0.637500in}{0.550000in}}{\pgfqpoint{3.850000in}{3.850000in}}%
\pgfusepath{clip}%
\pgfsetbuttcap%
\pgfsetroundjoin%
\definecolor{currentfill}{rgb}{0.897487,0.000000,0.000000}%
\pgfsetfillcolor{currentfill}%
\pgfsetfillopacity{0.800000}%
\pgfsetlinewidth{0.000000pt}%
\definecolor{currentstroke}{rgb}{0.000000,0.000000,0.000000}%
\pgfsetstrokecolor{currentstroke}%
\pgfsetdash{}{0pt}%
\pgfpathmoveto{\pgfqpoint{2.662622in}{2.902742in}}%
\pgfpathlineto{\pgfqpoint{2.592703in}{2.861644in}}%
\pgfpathlineto{\pgfqpoint{2.662622in}{2.900543in}}%
\pgfpathlineto{\pgfqpoint{2.732540in}{2.941640in}}%
\pgfpathlineto{\pgfqpoint{2.662622in}{2.902742in}}%
\pgfpathclose%
\pgfusepath{fill}%
\end{pgfscope}%
\begin{pgfscope}%
\pgfpathrectangle{\pgfqpoint{0.637500in}{0.550000in}}{\pgfqpoint{3.850000in}{3.850000in}}%
\pgfusepath{clip}%
\pgfsetbuttcap%
\pgfsetroundjoin%
\definecolor{currentfill}{rgb}{0.897487,0.000000,0.000000}%
\pgfsetfillcolor{currentfill}%
\pgfsetfillopacity{0.800000}%
\pgfsetlinewidth{0.000000pt}%
\definecolor{currentstroke}{rgb}{0.000000,0.000000,0.000000}%
\pgfsetstrokecolor{currentstroke}%
\pgfsetdash{}{0pt}%
\pgfpathmoveto{\pgfqpoint{2.872378in}{3.023834in}}%
\pgfpathlineto{\pgfqpoint{2.802459in}{2.982737in}}%
\pgfpathlineto{\pgfqpoint{2.872378in}{3.021636in}}%
\pgfpathlineto{\pgfqpoint{2.942296in}{3.062733in}}%
\pgfpathlineto{\pgfqpoint{2.872378in}{3.023834in}}%
\pgfpathclose%
\pgfusepath{fill}%
\end{pgfscope}%
\begin{pgfscope}%
\pgfpathrectangle{\pgfqpoint{0.637500in}{0.550000in}}{\pgfqpoint{3.850000in}{3.850000in}}%
\pgfusepath{clip}%
\pgfsetbuttcap%
\pgfsetroundjoin%
\definecolor{currentfill}{rgb}{0.897487,0.000000,0.000000}%
\pgfsetfillcolor{currentfill}%
\pgfsetfillopacity{0.800000}%
\pgfsetlinewidth{0.000000pt}%
\definecolor{currentstroke}{rgb}{0.000000,0.000000,0.000000}%
\pgfsetstrokecolor{currentstroke}%
\pgfsetdash{}{0pt}%
\pgfpathmoveto{\pgfqpoint{3.082134in}{3.144927in}}%
\pgfpathlineto{\pgfqpoint{3.012215in}{3.103830in}}%
\pgfpathlineto{\pgfqpoint{3.082134in}{3.142728in}}%
\pgfpathlineto{\pgfqpoint{3.152052in}{3.183825in}}%
\pgfpathlineto{\pgfqpoint{3.082134in}{3.144927in}}%
\pgfpathclose%
\pgfusepath{fill}%
\end{pgfscope}%
\begin{pgfscope}%
\pgfpathrectangle{\pgfqpoint{0.637500in}{0.550000in}}{\pgfqpoint{3.850000in}{3.850000in}}%
\pgfusepath{clip}%
\pgfsetbuttcap%
\pgfsetroundjoin%
\definecolor{currentfill}{rgb}{0.897487,0.000000,0.000000}%
\pgfsetfillcolor{currentfill}%
\pgfsetfillopacity{0.800000}%
\pgfsetlinewidth{0.000000pt}%
\definecolor{currentstroke}{rgb}{0.000000,0.000000,0.000000}%
\pgfsetstrokecolor{currentstroke}%
\pgfsetdash{}{0pt}%
\pgfpathmoveto{\pgfqpoint{2.243110in}{2.910493in}}%
\pgfpathlineto{\pgfqpoint{2.173191in}{2.949391in}}%
\pgfpathlineto{\pgfqpoint{2.243110in}{2.908294in}}%
\pgfpathlineto{\pgfqpoint{2.313028in}{2.869395in}}%
\pgfpathlineto{\pgfqpoint{2.243110in}{2.910493in}}%
\pgfpathclose%
\pgfusepath{fill}%
\end{pgfscope}%
\begin{pgfscope}%
\pgfpathrectangle{\pgfqpoint{0.637500in}{0.550000in}}{\pgfqpoint{3.850000in}{3.850000in}}%
\pgfusepath{clip}%
\pgfsetbuttcap%
\pgfsetroundjoin%
\definecolor{currentfill}{rgb}{0.897487,0.000000,0.000000}%
\pgfsetfillcolor{currentfill}%
\pgfsetfillopacity{0.800000}%
\pgfsetlinewidth{0.000000pt}%
\definecolor{currentstroke}{rgb}{0.000000,0.000000,0.000000}%
\pgfsetstrokecolor{currentstroke}%
\pgfsetdash{}{0pt}%
\pgfpathmoveto{\pgfqpoint{2.033354in}{3.031585in}}%
\pgfpathlineto{\pgfqpoint{1.963435in}{3.070484in}}%
\pgfpathlineto{\pgfqpoint{2.033354in}{3.029386in}}%
\pgfpathlineto{\pgfqpoint{2.103272in}{2.990488in}}%
\pgfpathlineto{\pgfqpoint{2.033354in}{3.031585in}}%
\pgfpathclose%
\pgfusepath{fill}%
\end{pgfscope}%
\begin{pgfscope}%
\pgfpathrectangle{\pgfqpoint{0.637500in}{0.550000in}}{\pgfqpoint{3.850000in}{3.850000in}}%
\pgfusepath{clip}%
\pgfsetbuttcap%
\pgfsetroundjoin%
\definecolor{currentfill}{rgb}{0.898098,0.000000,0.000000}%
\pgfsetfillcolor{currentfill}%
\pgfsetfillopacity{0.800000}%
\pgfsetlinewidth{0.000000pt}%
\definecolor{currentstroke}{rgb}{0.000000,0.000000,0.000000}%
\pgfsetstrokecolor{currentstroke}%
\pgfsetdash{}{0pt}%
\pgfpathmoveto{\pgfqpoint{2.452866in}{2.789795in}}%
\pgfpathlineto{\pgfqpoint{2.382947in}{2.828298in}}%
\pgfpathlineto{\pgfqpoint{2.452866in}{2.787596in}}%
\pgfpathlineto{\pgfqpoint{2.522784in}{2.820547in}}%
\pgfpathlineto{\pgfqpoint{2.452866in}{2.789795in}}%
\pgfpathclose%
\pgfusepath{fill}%
\end{pgfscope}%
\begin{pgfscope}%
\pgfpathrectangle{\pgfqpoint{0.637500in}{0.550000in}}{\pgfqpoint{3.850000in}{3.850000in}}%
\pgfusepath{clip}%
\pgfsetbuttcap%
\pgfsetroundjoin%
\definecolor{currentfill}{rgb}{0.897487,0.000000,0.000000}%
\pgfsetfillcolor{currentfill}%
\pgfsetfillopacity{0.800000}%
\pgfsetlinewidth{0.000000pt}%
\definecolor{currentstroke}{rgb}{0.000000,0.000000,0.000000}%
\pgfsetstrokecolor{currentstroke}%
\pgfsetdash{}{0pt}%
\pgfpathmoveto{\pgfqpoint{2.592703in}{2.861644in}}%
\pgfpathlineto{\pgfqpoint{2.522784in}{2.820547in}}%
\pgfpathlineto{\pgfqpoint{2.592703in}{2.859446in}}%
\pgfpathlineto{\pgfqpoint{2.662622in}{2.900543in}}%
\pgfpathlineto{\pgfqpoint{2.592703in}{2.861644in}}%
\pgfpathclose%
\pgfusepath{fill}%
\end{pgfscope}%
\begin{pgfscope}%
\pgfpathrectangle{\pgfqpoint{0.637500in}{0.550000in}}{\pgfqpoint{3.850000in}{3.850000in}}%
\pgfusepath{clip}%
\pgfsetbuttcap%
\pgfsetroundjoin%
\definecolor{currentfill}{rgb}{0.897487,0.000000,0.000000}%
\pgfsetfillcolor{currentfill}%
\pgfsetfillopacity{0.800000}%
\pgfsetlinewidth{0.000000pt}%
\definecolor{currentstroke}{rgb}{0.000000,0.000000,0.000000}%
\pgfsetstrokecolor{currentstroke}%
\pgfsetdash{}{0pt}%
\pgfpathmoveto{\pgfqpoint{2.802459in}{2.982737in}}%
\pgfpathlineto{\pgfqpoint{2.732540in}{2.941640in}}%
\pgfpathlineto{\pgfqpoint{2.802459in}{2.980538in}}%
\pgfpathlineto{\pgfqpoint{2.872378in}{3.021636in}}%
\pgfpathlineto{\pgfqpoint{2.802459in}{2.982737in}}%
\pgfpathclose%
\pgfusepath{fill}%
\end{pgfscope}%
\begin{pgfscope}%
\pgfpathrectangle{\pgfqpoint{0.637500in}{0.550000in}}{\pgfqpoint{3.850000in}{3.850000in}}%
\pgfusepath{clip}%
\pgfsetbuttcap%
\pgfsetroundjoin%
\definecolor{currentfill}{rgb}{0.897487,0.000000,0.000000}%
\pgfsetfillcolor{currentfill}%
\pgfsetfillopacity{0.800000}%
\pgfsetlinewidth{0.000000pt}%
\definecolor{currentstroke}{rgb}{0.000000,0.000000,0.000000}%
\pgfsetstrokecolor{currentstroke}%
\pgfsetdash{}{0pt}%
\pgfpathmoveto{\pgfqpoint{3.012215in}{3.103830in}}%
\pgfpathlineto{\pgfqpoint{2.942296in}{3.062733in}}%
\pgfpathlineto{\pgfqpoint{3.012215in}{3.101631in}}%
\pgfpathlineto{\pgfqpoint{3.082134in}{3.142728in}}%
\pgfpathlineto{\pgfqpoint{3.012215in}{3.103830in}}%
\pgfpathclose%
\pgfusepath{fill}%
\end{pgfscope}%
\begin{pgfscope}%
\pgfpathrectangle{\pgfqpoint{0.637500in}{0.550000in}}{\pgfqpoint{3.850000in}{3.850000in}}%
\pgfusepath{clip}%
\pgfsetbuttcap%
\pgfsetroundjoin%
\definecolor{currentfill}{rgb}{0.897487,0.000000,0.000000}%
\pgfsetfillcolor{currentfill}%
\pgfsetfillopacity{0.800000}%
\pgfsetlinewidth{0.000000pt}%
\definecolor{currentstroke}{rgb}{0.000000,0.000000,0.000000}%
\pgfsetstrokecolor{currentstroke}%
\pgfsetdash{}{0pt}%
\pgfpathmoveto{\pgfqpoint{2.313028in}{2.869395in}}%
\pgfpathlineto{\pgfqpoint{2.243110in}{2.908294in}}%
\pgfpathlineto{\pgfqpoint{2.313028in}{2.867197in}}%
\pgfpathlineto{\pgfqpoint{2.382947in}{2.828298in}}%
\pgfpathlineto{\pgfqpoint{2.313028in}{2.869395in}}%
\pgfpathclose%
\pgfusepath{fill}%
\end{pgfscope}%
\begin{pgfscope}%
\pgfpathrectangle{\pgfqpoint{0.637500in}{0.550000in}}{\pgfqpoint{3.850000in}{3.850000in}}%
\pgfusepath{clip}%
\pgfsetbuttcap%
\pgfsetroundjoin%
\definecolor{currentfill}{rgb}{0.897487,0.000000,0.000000}%
\pgfsetfillcolor{currentfill}%
\pgfsetfillopacity{0.800000}%
\pgfsetlinewidth{0.000000pt}%
\definecolor{currentstroke}{rgb}{0.000000,0.000000,0.000000}%
\pgfsetstrokecolor{currentstroke}%
\pgfsetdash{}{0pt}%
\pgfpathmoveto{\pgfqpoint{2.103272in}{2.990488in}}%
\pgfpathlineto{\pgfqpoint{2.033354in}{3.029386in}}%
\pgfpathlineto{\pgfqpoint{2.103272in}{2.988289in}}%
\pgfpathlineto{\pgfqpoint{2.173191in}{2.949391in}}%
\pgfpathlineto{\pgfqpoint{2.103272in}{2.990488in}}%
\pgfpathclose%
\pgfusepath{fill}%
\end{pgfscope}%
\begin{pgfscope}%
\pgfpathrectangle{\pgfqpoint{0.637500in}{0.550000in}}{\pgfqpoint{3.850000in}{3.850000in}}%
\pgfusepath{clip}%
\pgfsetbuttcap%
\pgfsetroundjoin%
\definecolor{currentfill}{rgb}{0.897487,0.000000,0.000000}%
\pgfsetfillcolor{currentfill}%
\pgfsetfillopacity{0.800000}%
\pgfsetlinewidth{0.000000pt}%
\definecolor{currentstroke}{rgb}{0.000000,0.000000,0.000000}%
\pgfsetstrokecolor{currentstroke}%
\pgfsetdash{}{0pt}%
\pgfpathmoveto{\pgfqpoint{1.893516in}{3.111581in}}%
\pgfpathlineto{\pgfqpoint{1.823598in}{3.150479in}}%
\pgfpathlineto{\pgfqpoint{1.893516in}{3.109382in}}%
\pgfpathlineto{\pgfqpoint{1.963435in}{3.070484in}}%
\pgfpathlineto{\pgfqpoint{1.893516in}{3.111581in}}%
\pgfpathclose%
\pgfusepath{fill}%
\end{pgfscope}%
\begin{pgfscope}%
\pgfpathrectangle{\pgfqpoint{0.637500in}{0.550000in}}{\pgfqpoint{3.850000in}{3.850000in}}%
\pgfusepath{clip}%
\pgfsetbuttcap%
\pgfsetroundjoin%
\definecolor{currentfill}{rgb}{0.909646,0.000000,0.000000}%
\pgfsetfillcolor{currentfill}%
\pgfsetfillopacity{0.800000}%
\pgfsetlinewidth{0.000000pt}%
\definecolor{currentstroke}{rgb}{0.000000,0.000000,0.000000}%
\pgfsetstrokecolor{currentstroke}%
\pgfsetdash{}{0pt}%
\pgfpathmoveto{\pgfqpoint{2.522784in}{2.820547in}}%
\pgfpathlineto{\pgfqpoint{2.452866in}{2.787596in}}%
\pgfpathlineto{\pgfqpoint{2.522784in}{2.818349in}}%
\pgfpathlineto{\pgfqpoint{2.592703in}{2.859446in}}%
\pgfpathlineto{\pgfqpoint{2.522784in}{2.820547in}}%
\pgfpathclose%
\pgfusepath{fill}%
\end{pgfscope}%
\begin{pgfscope}%
\pgfpathrectangle{\pgfqpoint{0.637500in}{0.550000in}}{\pgfqpoint{3.850000in}{3.850000in}}%
\pgfusepath{clip}%
\pgfsetbuttcap%
\pgfsetroundjoin%
\definecolor{currentfill}{rgb}{0.897487,0.000000,0.000000}%
\pgfsetfillcolor{currentfill}%
\pgfsetfillopacity{0.800000}%
\pgfsetlinewidth{0.000000pt}%
\definecolor{currentstroke}{rgb}{0.000000,0.000000,0.000000}%
\pgfsetstrokecolor{currentstroke}%
\pgfsetdash{}{0pt}%
\pgfpathmoveto{\pgfqpoint{2.732540in}{2.941640in}}%
\pgfpathlineto{\pgfqpoint{2.662622in}{2.900543in}}%
\pgfpathlineto{\pgfqpoint{2.732540in}{2.939441in}}%
\pgfpathlineto{\pgfqpoint{2.802459in}{2.980538in}}%
\pgfpathlineto{\pgfqpoint{2.732540in}{2.941640in}}%
\pgfpathclose%
\pgfusepath{fill}%
\end{pgfscope}%
\begin{pgfscope}%
\pgfpathrectangle{\pgfqpoint{0.637500in}{0.550000in}}{\pgfqpoint{3.850000in}{3.850000in}}%
\pgfusepath{clip}%
\pgfsetbuttcap%
\pgfsetroundjoin%
\definecolor{currentfill}{rgb}{0.897487,0.000000,0.000000}%
\pgfsetfillcolor{currentfill}%
\pgfsetfillopacity{0.800000}%
\pgfsetlinewidth{0.000000pt}%
\definecolor{currentstroke}{rgb}{0.000000,0.000000,0.000000}%
\pgfsetstrokecolor{currentstroke}%
\pgfsetdash{}{0pt}%
\pgfpathmoveto{\pgfqpoint{2.942296in}{3.062733in}}%
\pgfpathlineto{\pgfqpoint{2.872378in}{3.021636in}}%
\pgfpathlineto{\pgfqpoint{2.942296in}{3.060534in}}%
\pgfpathlineto{\pgfqpoint{3.012215in}{3.101631in}}%
\pgfpathlineto{\pgfqpoint{2.942296in}{3.062733in}}%
\pgfpathclose%
\pgfusepath{fill}%
\end{pgfscope}%
\begin{pgfscope}%
\pgfpathrectangle{\pgfqpoint{0.637500in}{0.550000in}}{\pgfqpoint{3.850000in}{3.850000in}}%
\pgfusepath{clip}%
\pgfsetbuttcap%
\pgfsetroundjoin%
\definecolor{currentfill}{rgb}{0.897487,0.000000,0.000000}%
\pgfsetfillcolor{currentfill}%
\pgfsetfillopacity{0.800000}%
\pgfsetlinewidth{0.000000pt}%
\definecolor{currentstroke}{rgb}{0.000000,0.000000,0.000000}%
\pgfsetstrokecolor{currentstroke}%
\pgfsetdash{}{0pt}%
\pgfpathmoveto{\pgfqpoint{3.152052in}{3.183825in}}%
\pgfpathlineto{\pgfqpoint{3.082134in}{3.142728in}}%
\pgfpathlineto{\pgfqpoint{3.152052in}{3.181627in}}%
\pgfpathlineto{\pgfqpoint{3.221971in}{3.222724in}}%
\pgfpathlineto{\pgfqpoint{3.152052in}{3.183825in}}%
\pgfpathclose%
\pgfusepath{fill}%
\end{pgfscope}%
\begin{pgfscope}%
\pgfpathrectangle{\pgfqpoint{0.637500in}{0.550000in}}{\pgfqpoint{3.850000in}{3.850000in}}%
\pgfusepath{clip}%
\pgfsetbuttcap%
\pgfsetroundjoin%
\definecolor{currentfill}{rgb}{0.897487,0.000000,0.000000}%
\pgfsetfillcolor{currentfill}%
\pgfsetfillopacity{0.800000}%
\pgfsetlinewidth{0.000000pt}%
\definecolor{currentstroke}{rgb}{0.000000,0.000000,0.000000}%
\pgfsetstrokecolor{currentstroke}%
\pgfsetdash{}{0pt}%
\pgfpathmoveto{\pgfqpoint{2.382947in}{2.828298in}}%
\pgfpathlineto{\pgfqpoint{2.313028in}{2.867197in}}%
\pgfpathlineto{\pgfqpoint{2.382947in}{2.826099in}}%
\pgfpathlineto{\pgfqpoint{2.452866in}{2.787596in}}%
\pgfpathlineto{\pgfqpoint{2.382947in}{2.828298in}}%
\pgfpathclose%
\pgfusepath{fill}%
\end{pgfscope}%
\begin{pgfscope}%
\pgfpathrectangle{\pgfqpoint{0.637500in}{0.550000in}}{\pgfqpoint{3.850000in}{3.850000in}}%
\pgfusepath{clip}%
\pgfsetbuttcap%
\pgfsetroundjoin%
\definecolor{currentfill}{rgb}{0.897487,0.000000,0.000000}%
\pgfsetfillcolor{currentfill}%
\pgfsetfillopacity{0.800000}%
\pgfsetlinewidth{0.000000pt}%
\definecolor{currentstroke}{rgb}{0.000000,0.000000,0.000000}%
\pgfsetstrokecolor{currentstroke}%
\pgfsetdash{}{0pt}%
\pgfpathmoveto{\pgfqpoint{2.173191in}{2.949391in}}%
\pgfpathlineto{\pgfqpoint{2.103272in}{2.988289in}}%
\pgfpathlineto{\pgfqpoint{2.173191in}{2.947192in}}%
\pgfpathlineto{\pgfqpoint{2.243110in}{2.908294in}}%
\pgfpathlineto{\pgfqpoint{2.173191in}{2.949391in}}%
\pgfpathclose%
\pgfusepath{fill}%
\end{pgfscope}%
\begin{pgfscope}%
\pgfpathrectangle{\pgfqpoint{0.637500in}{0.550000in}}{\pgfqpoint{3.850000in}{3.850000in}}%
\pgfusepath{clip}%
\pgfsetbuttcap%
\pgfsetroundjoin%
\definecolor{currentfill}{rgb}{0.897487,0.000000,0.000000}%
\pgfsetfillcolor{currentfill}%
\pgfsetfillopacity{0.800000}%
\pgfsetlinewidth{0.000000pt}%
\definecolor{currentstroke}{rgb}{0.000000,0.000000,0.000000}%
\pgfsetstrokecolor{currentstroke}%
\pgfsetdash{}{0pt}%
\pgfpathmoveto{\pgfqpoint{1.963435in}{3.070484in}}%
\pgfpathlineto{\pgfqpoint{1.893516in}{3.109382in}}%
\pgfpathlineto{\pgfqpoint{1.963435in}{3.068285in}}%
\pgfpathlineto{\pgfqpoint{2.033354in}{3.029386in}}%
\pgfpathlineto{\pgfqpoint{1.963435in}{3.070484in}}%
\pgfpathclose%
\pgfusepath{fill}%
\end{pgfscope}%
\begin{pgfscope}%
\pgfpathrectangle{\pgfqpoint{0.637500in}{0.550000in}}{\pgfqpoint{3.850000in}{3.850000in}}%
\pgfusepath{clip}%
\pgfsetbuttcap%
\pgfsetroundjoin%
\definecolor{currentfill}{rgb}{0.897487,0.000000,0.000000}%
\pgfsetfillcolor{currentfill}%
\pgfsetfillopacity{0.800000}%
\pgfsetlinewidth{0.000000pt}%
\definecolor{currentstroke}{rgb}{0.000000,0.000000,0.000000}%
\pgfsetstrokecolor{currentstroke}%
\pgfsetdash{}{0pt}%
\pgfpathmoveto{\pgfqpoint{2.662622in}{2.900543in}}%
\pgfpathlineto{\pgfqpoint{2.592703in}{2.859446in}}%
\pgfpathlineto{\pgfqpoint{2.662622in}{2.898344in}}%
\pgfpathlineto{\pgfqpoint{2.732540in}{2.939441in}}%
\pgfpathlineto{\pgfqpoint{2.662622in}{2.900543in}}%
\pgfpathclose%
\pgfusepath{fill}%
\end{pgfscope}%
\begin{pgfscope}%
\pgfpathrectangle{\pgfqpoint{0.637500in}{0.550000in}}{\pgfqpoint{3.850000in}{3.850000in}}%
\pgfusepath{clip}%
\pgfsetbuttcap%
\pgfsetroundjoin%
\definecolor{currentfill}{rgb}{0.897487,0.000000,0.000000}%
\pgfsetfillcolor{currentfill}%
\pgfsetfillopacity{0.800000}%
\pgfsetlinewidth{0.000000pt}%
\definecolor{currentstroke}{rgb}{0.000000,0.000000,0.000000}%
\pgfsetstrokecolor{currentstroke}%
\pgfsetdash{}{0pt}%
\pgfpathmoveto{\pgfqpoint{2.872378in}{3.021636in}}%
\pgfpathlineto{\pgfqpoint{2.802459in}{2.980538in}}%
\pgfpathlineto{\pgfqpoint{2.872378in}{3.019437in}}%
\pgfpathlineto{\pgfqpoint{2.942296in}{3.060534in}}%
\pgfpathlineto{\pgfqpoint{2.872378in}{3.021636in}}%
\pgfpathclose%
\pgfusepath{fill}%
\end{pgfscope}%
\begin{pgfscope}%
\pgfpathrectangle{\pgfqpoint{0.637500in}{0.550000in}}{\pgfqpoint{3.850000in}{3.850000in}}%
\pgfusepath{clip}%
\pgfsetbuttcap%
\pgfsetroundjoin%
\definecolor{currentfill}{rgb}{0.897487,0.000000,0.000000}%
\pgfsetfillcolor{currentfill}%
\pgfsetfillopacity{0.800000}%
\pgfsetlinewidth{0.000000pt}%
\definecolor{currentstroke}{rgb}{0.000000,0.000000,0.000000}%
\pgfsetstrokecolor{currentstroke}%
\pgfsetdash{}{0pt}%
\pgfpathmoveto{\pgfqpoint{3.082134in}{3.142728in}}%
\pgfpathlineto{\pgfqpoint{3.012215in}{3.101631in}}%
\pgfpathlineto{\pgfqpoint{3.082134in}{3.140529in}}%
\pgfpathlineto{\pgfqpoint{3.152052in}{3.181627in}}%
\pgfpathlineto{\pgfqpoint{3.082134in}{3.142728in}}%
\pgfpathclose%
\pgfusepath{fill}%
\end{pgfscope}%
\begin{pgfscope}%
\pgfpathrectangle{\pgfqpoint{0.637500in}{0.550000in}}{\pgfqpoint{3.850000in}{3.850000in}}%
\pgfusepath{clip}%
\pgfsetbuttcap%
\pgfsetroundjoin%
\definecolor{currentfill}{rgb}{0.897487,0.000000,0.000000}%
\pgfsetfillcolor{currentfill}%
\pgfsetfillopacity{0.800000}%
\pgfsetlinewidth{0.000000pt}%
\definecolor{currentstroke}{rgb}{0.000000,0.000000,0.000000}%
\pgfsetstrokecolor{currentstroke}%
\pgfsetdash{}{0pt}%
\pgfpathmoveto{\pgfqpoint{2.243110in}{2.908294in}}%
\pgfpathlineto{\pgfqpoint{2.173191in}{2.947192in}}%
\pgfpathlineto{\pgfqpoint{2.243110in}{2.906095in}}%
\pgfpathlineto{\pgfqpoint{2.313028in}{2.867197in}}%
\pgfpathlineto{\pgfqpoint{2.243110in}{2.908294in}}%
\pgfpathclose%
\pgfusepath{fill}%
\end{pgfscope}%
\begin{pgfscope}%
\pgfpathrectangle{\pgfqpoint{0.637500in}{0.550000in}}{\pgfqpoint{3.850000in}{3.850000in}}%
\pgfusepath{clip}%
\pgfsetbuttcap%
\pgfsetroundjoin%
\definecolor{currentfill}{rgb}{0.897487,0.000000,0.000000}%
\pgfsetfillcolor{currentfill}%
\pgfsetfillopacity{0.800000}%
\pgfsetlinewidth{0.000000pt}%
\definecolor{currentstroke}{rgb}{0.000000,0.000000,0.000000}%
\pgfsetstrokecolor{currentstroke}%
\pgfsetdash{}{0pt}%
\pgfpathmoveto{\pgfqpoint{2.033354in}{3.029386in}}%
\pgfpathlineto{\pgfqpoint{1.963435in}{3.068285in}}%
\pgfpathlineto{\pgfqpoint{2.033354in}{3.027188in}}%
\pgfpathlineto{\pgfqpoint{2.103272in}{2.988289in}}%
\pgfpathlineto{\pgfqpoint{2.033354in}{3.029386in}}%
\pgfpathclose%
\pgfusepath{fill}%
\end{pgfscope}%
\begin{pgfscope}%
\pgfpathrectangle{\pgfqpoint{0.637500in}{0.550000in}}{\pgfqpoint{3.850000in}{3.850000in}}%
\pgfusepath{clip}%
\pgfsetbuttcap%
\pgfsetroundjoin%
\definecolor{currentfill}{rgb}{0.897487,0.000000,0.000000}%
\pgfsetfillcolor{currentfill}%
\pgfsetfillopacity{0.800000}%
\pgfsetlinewidth{0.000000pt}%
\definecolor{currentstroke}{rgb}{0.000000,0.000000,0.000000}%
\pgfsetstrokecolor{currentstroke}%
\pgfsetdash{}{0pt}%
\pgfpathmoveto{\pgfqpoint{1.823598in}{3.150479in}}%
\pgfpathlineto{\pgfqpoint{1.753679in}{3.189378in}}%
\pgfpathlineto{\pgfqpoint{1.823598in}{3.148280in}}%
\pgfpathlineto{\pgfqpoint{1.893516in}{3.109382in}}%
\pgfpathlineto{\pgfqpoint{1.823598in}{3.150479in}}%
\pgfpathclose%
\pgfusepath{fill}%
\end{pgfscope}%
\begin{pgfscope}%
\pgfpathrectangle{\pgfqpoint{0.637500in}{0.550000in}}{\pgfqpoint{3.850000in}{3.850000in}}%
\pgfusepath{clip}%
\pgfsetbuttcap%
\pgfsetroundjoin%
\definecolor{currentfill}{rgb}{0.898098,0.000000,0.000000}%
\pgfsetfillcolor{currentfill}%
\pgfsetfillopacity{0.800000}%
\pgfsetlinewidth{0.000000pt}%
\definecolor{currentstroke}{rgb}{0.000000,0.000000,0.000000}%
\pgfsetstrokecolor{currentstroke}%
\pgfsetdash{}{0pt}%
\pgfpathmoveto{\pgfqpoint{2.452866in}{2.787596in}}%
\pgfpathlineto{\pgfqpoint{2.382947in}{2.826099in}}%
\pgfpathlineto{\pgfqpoint{2.452866in}{2.785398in}}%
\pgfpathlineto{\pgfqpoint{2.522784in}{2.818349in}}%
\pgfpathlineto{\pgfqpoint{2.452866in}{2.787596in}}%
\pgfpathclose%
\pgfusepath{fill}%
\end{pgfscope}%
\begin{pgfscope}%
\pgfpathrectangle{\pgfqpoint{0.637500in}{0.550000in}}{\pgfqpoint{3.850000in}{3.850000in}}%
\pgfusepath{clip}%
\pgfsetbuttcap%
\pgfsetroundjoin%
\definecolor{currentfill}{rgb}{0.897487,0.000000,0.000000}%
\pgfsetfillcolor{currentfill}%
\pgfsetfillopacity{0.800000}%
\pgfsetlinewidth{0.000000pt}%
\definecolor{currentstroke}{rgb}{0.000000,0.000000,0.000000}%
\pgfsetstrokecolor{currentstroke}%
\pgfsetdash{}{0pt}%
\pgfpathmoveto{\pgfqpoint{2.592703in}{2.859446in}}%
\pgfpathlineto{\pgfqpoint{2.522784in}{2.818349in}}%
\pgfpathlineto{\pgfqpoint{2.592703in}{2.857247in}}%
\pgfpathlineto{\pgfqpoint{2.662622in}{2.898344in}}%
\pgfpathlineto{\pgfqpoint{2.592703in}{2.859446in}}%
\pgfpathclose%
\pgfusepath{fill}%
\end{pgfscope}%
\begin{pgfscope}%
\pgfpathrectangle{\pgfqpoint{0.637500in}{0.550000in}}{\pgfqpoint{3.850000in}{3.850000in}}%
\pgfusepath{clip}%
\pgfsetbuttcap%
\pgfsetroundjoin%
\definecolor{currentfill}{rgb}{0.897487,0.000000,0.000000}%
\pgfsetfillcolor{currentfill}%
\pgfsetfillopacity{0.800000}%
\pgfsetlinewidth{0.000000pt}%
\definecolor{currentstroke}{rgb}{0.000000,0.000000,0.000000}%
\pgfsetstrokecolor{currentstroke}%
\pgfsetdash{}{0pt}%
\pgfpathmoveto{\pgfqpoint{2.802459in}{2.980538in}}%
\pgfpathlineto{\pgfqpoint{2.732540in}{2.939441in}}%
\pgfpathlineto{\pgfqpoint{2.802459in}{2.978340in}}%
\pgfpathlineto{\pgfqpoint{2.872378in}{3.019437in}}%
\pgfpathlineto{\pgfqpoint{2.802459in}{2.980538in}}%
\pgfpathclose%
\pgfusepath{fill}%
\end{pgfscope}%
\begin{pgfscope}%
\pgfpathrectangle{\pgfqpoint{0.637500in}{0.550000in}}{\pgfqpoint{3.850000in}{3.850000in}}%
\pgfusepath{clip}%
\pgfsetbuttcap%
\pgfsetroundjoin%
\definecolor{currentfill}{rgb}{0.897487,0.000000,0.000000}%
\pgfsetfillcolor{currentfill}%
\pgfsetfillopacity{0.800000}%
\pgfsetlinewidth{0.000000pt}%
\definecolor{currentstroke}{rgb}{0.000000,0.000000,0.000000}%
\pgfsetstrokecolor{currentstroke}%
\pgfsetdash{}{0pt}%
\pgfpathmoveto{\pgfqpoint{3.012215in}{3.101631in}}%
\pgfpathlineto{\pgfqpoint{2.942296in}{3.060534in}}%
\pgfpathlineto{\pgfqpoint{3.012215in}{3.099432in}}%
\pgfpathlineto{\pgfqpoint{3.082134in}{3.140529in}}%
\pgfpathlineto{\pgfqpoint{3.012215in}{3.101631in}}%
\pgfpathclose%
\pgfusepath{fill}%
\end{pgfscope}%
\begin{pgfscope}%
\pgfpathrectangle{\pgfqpoint{0.637500in}{0.550000in}}{\pgfqpoint{3.850000in}{3.850000in}}%
\pgfusepath{clip}%
\pgfsetbuttcap%
\pgfsetroundjoin%
\definecolor{currentfill}{rgb}{0.897487,0.000000,0.000000}%
\pgfsetfillcolor{currentfill}%
\pgfsetfillopacity{0.800000}%
\pgfsetlinewidth{0.000000pt}%
\definecolor{currentstroke}{rgb}{0.000000,0.000000,0.000000}%
\pgfsetstrokecolor{currentstroke}%
\pgfsetdash{}{0pt}%
\pgfpathmoveto{\pgfqpoint{3.221971in}{3.222724in}}%
\pgfpathlineto{\pgfqpoint{3.152052in}{3.181627in}}%
\pgfpathlineto{\pgfqpoint{3.221971in}{3.220525in}}%
\pgfpathlineto{\pgfqpoint{3.291890in}{3.261622in}}%
\pgfpathlineto{\pgfqpoint{3.221971in}{3.222724in}}%
\pgfpathclose%
\pgfusepath{fill}%
\end{pgfscope}%
\begin{pgfscope}%
\pgfpathrectangle{\pgfqpoint{0.637500in}{0.550000in}}{\pgfqpoint{3.850000in}{3.850000in}}%
\pgfusepath{clip}%
\pgfsetbuttcap%
\pgfsetroundjoin%
\definecolor{currentfill}{rgb}{0.897487,0.000000,0.000000}%
\pgfsetfillcolor{currentfill}%
\pgfsetfillopacity{0.800000}%
\pgfsetlinewidth{0.000000pt}%
\definecolor{currentstroke}{rgb}{0.000000,0.000000,0.000000}%
\pgfsetstrokecolor{currentstroke}%
\pgfsetdash{}{0pt}%
\pgfpathmoveto{\pgfqpoint{2.313028in}{2.867197in}}%
\pgfpathlineto{\pgfqpoint{2.243110in}{2.906095in}}%
\pgfpathlineto{\pgfqpoint{2.313028in}{2.864998in}}%
\pgfpathlineto{\pgfqpoint{2.382947in}{2.826099in}}%
\pgfpathlineto{\pgfqpoint{2.313028in}{2.867197in}}%
\pgfpathclose%
\pgfusepath{fill}%
\end{pgfscope}%
\begin{pgfscope}%
\pgfpathrectangle{\pgfqpoint{0.637500in}{0.550000in}}{\pgfqpoint{3.850000in}{3.850000in}}%
\pgfusepath{clip}%
\pgfsetbuttcap%
\pgfsetroundjoin%
\definecolor{currentfill}{rgb}{0.897487,0.000000,0.000000}%
\pgfsetfillcolor{currentfill}%
\pgfsetfillopacity{0.800000}%
\pgfsetlinewidth{0.000000pt}%
\definecolor{currentstroke}{rgb}{0.000000,0.000000,0.000000}%
\pgfsetstrokecolor{currentstroke}%
\pgfsetdash{}{0pt}%
\pgfpathmoveto{\pgfqpoint{2.103272in}{2.988289in}}%
\pgfpathlineto{\pgfqpoint{2.033354in}{3.027188in}}%
\pgfpathlineto{\pgfqpoint{2.103272in}{2.986090in}}%
\pgfpathlineto{\pgfqpoint{2.173191in}{2.947192in}}%
\pgfpathlineto{\pgfqpoint{2.103272in}{2.988289in}}%
\pgfpathclose%
\pgfusepath{fill}%
\end{pgfscope}%
\begin{pgfscope}%
\pgfpathrectangle{\pgfqpoint{0.637500in}{0.550000in}}{\pgfqpoint{3.850000in}{3.850000in}}%
\pgfusepath{clip}%
\pgfsetbuttcap%
\pgfsetroundjoin%
\definecolor{currentfill}{rgb}{0.897487,0.000000,0.000000}%
\pgfsetfillcolor{currentfill}%
\pgfsetfillopacity{0.800000}%
\pgfsetlinewidth{0.000000pt}%
\definecolor{currentstroke}{rgb}{0.000000,0.000000,0.000000}%
\pgfsetstrokecolor{currentstroke}%
\pgfsetdash{}{0pt}%
\pgfpathmoveto{\pgfqpoint{1.893516in}{3.109382in}}%
\pgfpathlineto{\pgfqpoint{1.823598in}{3.148280in}}%
\pgfpathlineto{\pgfqpoint{1.893516in}{3.107183in}}%
\pgfpathlineto{\pgfqpoint{1.963435in}{3.068285in}}%
\pgfpathlineto{\pgfqpoint{1.893516in}{3.109382in}}%
\pgfpathclose%
\pgfusepath{fill}%
\end{pgfscope}%
\begin{pgfscope}%
\pgfpathrectangle{\pgfqpoint{0.637500in}{0.550000in}}{\pgfqpoint{3.850000in}{3.850000in}}%
\pgfusepath{clip}%
\pgfsetbuttcap%
\pgfsetroundjoin%
\definecolor{currentfill}{rgb}{0.909646,0.000000,0.000000}%
\pgfsetfillcolor{currentfill}%
\pgfsetfillopacity{0.800000}%
\pgfsetlinewidth{0.000000pt}%
\definecolor{currentstroke}{rgb}{0.000000,0.000000,0.000000}%
\pgfsetstrokecolor{currentstroke}%
\pgfsetdash{}{0pt}%
\pgfpathmoveto{\pgfqpoint{2.522784in}{2.818349in}}%
\pgfpathlineto{\pgfqpoint{2.452866in}{2.785398in}}%
\pgfpathlineto{\pgfqpoint{2.522784in}{2.816150in}}%
\pgfpathlineto{\pgfqpoint{2.592703in}{2.857247in}}%
\pgfpathlineto{\pgfqpoint{2.522784in}{2.818349in}}%
\pgfpathclose%
\pgfusepath{fill}%
\end{pgfscope}%
\begin{pgfscope}%
\pgfpathrectangle{\pgfqpoint{0.637500in}{0.550000in}}{\pgfqpoint{3.850000in}{3.850000in}}%
\pgfusepath{clip}%
\pgfsetbuttcap%
\pgfsetroundjoin%
\definecolor{currentfill}{rgb}{0.897487,0.000000,0.000000}%
\pgfsetfillcolor{currentfill}%
\pgfsetfillopacity{0.800000}%
\pgfsetlinewidth{0.000000pt}%
\definecolor{currentstroke}{rgb}{0.000000,0.000000,0.000000}%
\pgfsetstrokecolor{currentstroke}%
\pgfsetdash{}{0pt}%
\pgfpathmoveto{\pgfqpoint{2.732540in}{2.939441in}}%
\pgfpathlineto{\pgfqpoint{2.662622in}{2.898344in}}%
\pgfpathlineto{\pgfqpoint{2.732540in}{2.937242in}}%
\pgfpathlineto{\pgfqpoint{2.802459in}{2.978340in}}%
\pgfpathlineto{\pgfqpoint{2.732540in}{2.939441in}}%
\pgfpathclose%
\pgfusepath{fill}%
\end{pgfscope}%
\begin{pgfscope}%
\pgfpathrectangle{\pgfqpoint{0.637500in}{0.550000in}}{\pgfqpoint{3.850000in}{3.850000in}}%
\pgfusepath{clip}%
\pgfsetbuttcap%
\pgfsetroundjoin%
\definecolor{currentfill}{rgb}{0.897487,0.000000,0.000000}%
\pgfsetfillcolor{currentfill}%
\pgfsetfillopacity{0.800000}%
\pgfsetlinewidth{0.000000pt}%
\definecolor{currentstroke}{rgb}{0.000000,0.000000,0.000000}%
\pgfsetstrokecolor{currentstroke}%
\pgfsetdash{}{0pt}%
\pgfpathmoveto{\pgfqpoint{2.942296in}{3.060534in}}%
\pgfpathlineto{\pgfqpoint{2.872378in}{3.019437in}}%
\pgfpathlineto{\pgfqpoint{2.942296in}{3.058335in}}%
\pgfpathlineto{\pgfqpoint{3.012215in}{3.099432in}}%
\pgfpathlineto{\pgfqpoint{2.942296in}{3.060534in}}%
\pgfpathclose%
\pgfusepath{fill}%
\end{pgfscope}%
\begin{pgfscope}%
\pgfpathrectangle{\pgfqpoint{0.637500in}{0.550000in}}{\pgfqpoint{3.850000in}{3.850000in}}%
\pgfusepath{clip}%
\pgfsetbuttcap%
\pgfsetroundjoin%
\definecolor{currentfill}{rgb}{0.897487,0.000000,0.000000}%
\pgfsetfillcolor{currentfill}%
\pgfsetfillopacity{0.800000}%
\pgfsetlinewidth{0.000000pt}%
\definecolor{currentstroke}{rgb}{0.000000,0.000000,0.000000}%
\pgfsetstrokecolor{currentstroke}%
\pgfsetdash{}{0pt}%
\pgfpathmoveto{\pgfqpoint{3.152052in}{3.181627in}}%
\pgfpathlineto{\pgfqpoint{3.082134in}{3.140529in}}%
\pgfpathlineto{\pgfqpoint{3.152052in}{3.179428in}}%
\pgfpathlineto{\pgfqpoint{3.221971in}{3.220525in}}%
\pgfpathlineto{\pgfqpoint{3.152052in}{3.181627in}}%
\pgfpathclose%
\pgfusepath{fill}%
\end{pgfscope}%
\begin{pgfscope}%
\pgfpathrectangle{\pgfqpoint{0.637500in}{0.550000in}}{\pgfqpoint{3.850000in}{3.850000in}}%
\pgfusepath{clip}%
\pgfsetbuttcap%
\pgfsetroundjoin%
\definecolor{currentfill}{rgb}{0.897487,0.000000,0.000000}%
\pgfsetfillcolor{currentfill}%
\pgfsetfillopacity{0.800000}%
\pgfsetlinewidth{0.000000pt}%
\definecolor{currentstroke}{rgb}{0.000000,0.000000,0.000000}%
\pgfsetstrokecolor{currentstroke}%
\pgfsetdash{}{0pt}%
\pgfpathmoveto{\pgfqpoint{2.382947in}{2.826099in}}%
\pgfpathlineto{\pgfqpoint{2.313028in}{2.864998in}}%
\pgfpathlineto{\pgfqpoint{2.382947in}{2.823901in}}%
\pgfpathlineto{\pgfqpoint{2.452866in}{2.785398in}}%
\pgfpathlineto{\pgfqpoint{2.382947in}{2.826099in}}%
\pgfpathclose%
\pgfusepath{fill}%
\end{pgfscope}%
\begin{pgfscope}%
\pgfpathrectangle{\pgfqpoint{0.637500in}{0.550000in}}{\pgfqpoint{3.850000in}{3.850000in}}%
\pgfusepath{clip}%
\pgfsetbuttcap%
\pgfsetroundjoin%
\definecolor{currentfill}{rgb}{0.897487,0.000000,0.000000}%
\pgfsetfillcolor{currentfill}%
\pgfsetfillopacity{0.800000}%
\pgfsetlinewidth{0.000000pt}%
\definecolor{currentstroke}{rgb}{0.000000,0.000000,0.000000}%
\pgfsetstrokecolor{currentstroke}%
\pgfsetdash{}{0pt}%
\pgfpathmoveto{\pgfqpoint{2.173191in}{2.947192in}}%
\pgfpathlineto{\pgfqpoint{2.103272in}{2.986090in}}%
\pgfpathlineto{\pgfqpoint{2.173191in}{2.944993in}}%
\pgfpathlineto{\pgfqpoint{2.243110in}{2.906095in}}%
\pgfpathlineto{\pgfqpoint{2.173191in}{2.947192in}}%
\pgfpathclose%
\pgfusepath{fill}%
\end{pgfscope}%
\begin{pgfscope}%
\pgfpathrectangle{\pgfqpoint{0.637500in}{0.550000in}}{\pgfqpoint{3.850000in}{3.850000in}}%
\pgfusepath{clip}%
\pgfsetbuttcap%
\pgfsetroundjoin%
\definecolor{currentfill}{rgb}{0.897487,0.000000,0.000000}%
\pgfsetfillcolor{currentfill}%
\pgfsetfillopacity{0.800000}%
\pgfsetlinewidth{0.000000pt}%
\definecolor{currentstroke}{rgb}{0.000000,0.000000,0.000000}%
\pgfsetstrokecolor{currentstroke}%
\pgfsetdash{}{0pt}%
\pgfpathmoveto{\pgfqpoint{1.963435in}{3.068285in}}%
\pgfpathlineto{\pgfqpoint{1.893516in}{3.107183in}}%
\pgfpathlineto{\pgfqpoint{1.963435in}{3.066086in}}%
\pgfpathlineto{\pgfqpoint{2.033354in}{3.027188in}}%
\pgfpathlineto{\pgfqpoint{1.963435in}{3.068285in}}%
\pgfpathclose%
\pgfusepath{fill}%
\end{pgfscope}%
\begin{pgfscope}%
\pgfpathrectangle{\pgfqpoint{0.637500in}{0.550000in}}{\pgfqpoint{3.850000in}{3.850000in}}%
\pgfusepath{clip}%
\pgfsetbuttcap%
\pgfsetroundjoin%
\definecolor{currentfill}{rgb}{0.897487,0.000000,0.000000}%
\pgfsetfillcolor{currentfill}%
\pgfsetfillopacity{0.800000}%
\pgfsetlinewidth{0.000000pt}%
\definecolor{currentstroke}{rgb}{0.000000,0.000000,0.000000}%
\pgfsetstrokecolor{currentstroke}%
\pgfsetdash{}{0pt}%
\pgfpathmoveto{\pgfqpoint{1.753679in}{3.189378in}}%
\pgfpathlineto{\pgfqpoint{1.683761in}{3.228276in}}%
\pgfpathlineto{\pgfqpoint{1.753679in}{3.187179in}}%
\pgfpathlineto{\pgfqpoint{1.823598in}{3.148280in}}%
\pgfpathlineto{\pgfqpoint{1.753679in}{3.189378in}}%
\pgfpathclose%
\pgfusepath{fill}%
\end{pgfscope}%
\begin{pgfscope}%
\pgfpathrectangle{\pgfqpoint{0.637500in}{0.550000in}}{\pgfqpoint{3.850000in}{3.850000in}}%
\pgfusepath{clip}%
\pgfsetbuttcap%
\pgfsetroundjoin%
\definecolor{currentfill}{rgb}{0.897487,0.000000,0.000000}%
\pgfsetfillcolor{currentfill}%
\pgfsetfillopacity{0.800000}%
\pgfsetlinewidth{0.000000pt}%
\definecolor{currentstroke}{rgb}{0.000000,0.000000,0.000000}%
\pgfsetstrokecolor{currentstroke}%
\pgfsetdash{}{0pt}%
\pgfpathmoveto{\pgfqpoint{2.662622in}{2.898344in}}%
\pgfpathlineto{\pgfqpoint{2.592703in}{2.857247in}}%
\pgfpathlineto{\pgfqpoint{2.662622in}{2.896145in}}%
\pgfpathlineto{\pgfqpoint{2.732540in}{2.937242in}}%
\pgfpathlineto{\pgfqpoint{2.662622in}{2.898344in}}%
\pgfpathclose%
\pgfusepath{fill}%
\end{pgfscope}%
\begin{pgfscope}%
\pgfpathrectangle{\pgfqpoint{0.637500in}{0.550000in}}{\pgfqpoint{3.850000in}{3.850000in}}%
\pgfusepath{clip}%
\pgfsetbuttcap%
\pgfsetroundjoin%
\definecolor{currentfill}{rgb}{0.897487,0.000000,0.000000}%
\pgfsetfillcolor{currentfill}%
\pgfsetfillopacity{0.800000}%
\pgfsetlinewidth{0.000000pt}%
\definecolor{currentstroke}{rgb}{0.000000,0.000000,0.000000}%
\pgfsetstrokecolor{currentstroke}%
\pgfsetdash{}{0pt}%
\pgfpathmoveto{\pgfqpoint{2.872378in}{3.019437in}}%
\pgfpathlineto{\pgfqpoint{2.802459in}{2.978340in}}%
\pgfpathlineto{\pgfqpoint{2.872378in}{3.017238in}}%
\pgfpathlineto{\pgfqpoint{2.942296in}{3.058335in}}%
\pgfpathlineto{\pgfqpoint{2.872378in}{3.019437in}}%
\pgfpathclose%
\pgfusepath{fill}%
\end{pgfscope}%
\begin{pgfscope}%
\pgfpathrectangle{\pgfqpoint{0.637500in}{0.550000in}}{\pgfqpoint{3.850000in}{3.850000in}}%
\pgfusepath{clip}%
\pgfsetbuttcap%
\pgfsetroundjoin%
\definecolor{currentfill}{rgb}{0.897487,0.000000,0.000000}%
\pgfsetfillcolor{currentfill}%
\pgfsetfillopacity{0.800000}%
\pgfsetlinewidth{0.000000pt}%
\definecolor{currentstroke}{rgb}{0.000000,0.000000,0.000000}%
\pgfsetstrokecolor{currentstroke}%
\pgfsetdash{}{0pt}%
\pgfpathmoveto{\pgfqpoint{3.082134in}{3.140529in}}%
\pgfpathlineto{\pgfqpoint{3.012215in}{3.099432in}}%
\pgfpathlineto{\pgfqpoint{3.082134in}{3.138331in}}%
\pgfpathlineto{\pgfqpoint{3.152052in}{3.179428in}}%
\pgfpathlineto{\pgfqpoint{3.082134in}{3.140529in}}%
\pgfpathclose%
\pgfusepath{fill}%
\end{pgfscope}%
\begin{pgfscope}%
\pgfpathrectangle{\pgfqpoint{0.637500in}{0.550000in}}{\pgfqpoint{3.850000in}{3.850000in}}%
\pgfusepath{clip}%
\pgfsetbuttcap%
\pgfsetroundjoin%
\definecolor{currentfill}{rgb}{0.897487,0.000000,0.000000}%
\pgfsetfillcolor{currentfill}%
\pgfsetfillopacity{0.800000}%
\pgfsetlinewidth{0.000000pt}%
\definecolor{currentstroke}{rgb}{0.000000,0.000000,0.000000}%
\pgfsetstrokecolor{currentstroke}%
\pgfsetdash{}{0pt}%
\pgfpathmoveto{\pgfqpoint{3.291890in}{3.261622in}}%
\pgfpathlineto{\pgfqpoint{3.221971in}{3.220525in}}%
\pgfpathlineto{\pgfqpoint{3.291890in}{3.259423in}}%
\pgfpathlineto{\pgfqpoint{3.361808in}{3.300521in}}%
\pgfpathlineto{\pgfqpoint{3.291890in}{3.261622in}}%
\pgfpathclose%
\pgfusepath{fill}%
\end{pgfscope}%
\begin{pgfscope}%
\pgfpathrectangle{\pgfqpoint{0.637500in}{0.550000in}}{\pgfqpoint{3.850000in}{3.850000in}}%
\pgfusepath{clip}%
\pgfsetbuttcap%
\pgfsetroundjoin%
\definecolor{currentfill}{rgb}{0.897487,0.000000,0.000000}%
\pgfsetfillcolor{currentfill}%
\pgfsetfillopacity{0.800000}%
\pgfsetlinewidth{0.000000pt}%
\definecolor{currentstroke}{rgb}{0.000000,0.000000,0.000000}%
\pgfsetstrokecolor{currentstroke}%
\pgfsetdash{}{0pt}%
\pgfpathmoveto{\pgfqpoint{2.243110in}{2.906095in}}%
\pgfpathlineto{\pgfqpoint{2.173191in}{2.944993in}}%
\pgfpathlineto{\pgfqpoint{2.243110in}{2.903896in}}%
\pgfpathlineto{\pgfqpoint{2.313028in}{2.864998in}}%
\pgfpathlineto{\pgfqpoint{2.243110in}{2.906095in}}%
\pgfpathclose%
\pgfusepath{fill}%
\end{pgfscope}%
\begin{pgfscope}%
\pgfpathrectangle{\pgfqpoint{0.637500in}{0.550000in}}{\pgfqpoint{3.850000in}{3.850000in}}%
\pgfusepath{clip}%
\pgfsetbuttcap%
\pgfsetroundjoin%
\definecolor{currentfill}{rgb}{0.897487,0.000000,0.000000}%
\pgfsetfillcolor{currentfill}%
\pgfsetfillopacity{0.800000}%
\pgfsetlinewidth{0.000000pt}%
\definecolor{currentstroke}{rgb}{0.000000,0.000000,0.000000}%
\pgfsetstrokecolor{currentstroke}%
\pgfsetdash{}{0pt}%
\pgfpathmoveto{\pgfqpoint{2.033354in}{3.027188in}}%
\pgfpathlineto{\pgfqpoint{1.963435in}{3.066086in}}%
\pgfpathlineto{\pgfqpoint{2.033354in}{3.024989in}}%
\pgfpathlineto{\pgfqpoint{2.103272in}{2.986090in}}%
\pgfpathlineto{\pgfqpoint{2.033354in}{3.027188in}}%
\pgfpathclose%
\pgfusepath{fill}%
\end{pgfscope}%
\begin{pgfscope}%
\pgfpathrectangle{\pgfqpoint{0.637500in}{0.550000in}}{\pgfqpoint{3.850000in}{3.850000in}}%
\pgfusepath{clip}%
\pgfsetbuttcap%
\pgfsetroundjoin%
\definecolor{currentfill}{rgb}{0.897487,0.000000,0.000000}%
\pgfsetfillcolor{currentfill}%
\pgfsetfillopacity{0.800000}%
\pgfsetlinewidth{0.000000pt}%
\definecolor{currentstroke}{rgb}{0.000000,0.000000,0.000000}%
\pgfsetstrokecolor{currentstroke}%
\pgfsetdash{}{0pt}%
\pgfpathmoveto{\pgfqpoint{1.823598in}{3.148280in}}%
\pgfpathlineto{\pgfqpoint{1.753679in}{3.187179in}}%
\pgfpathlineto{\pgfqpoint{1.823598in}{3.146082in}}%
\pgfpathlineto{\pgfqpoint{1.893516in}{3.107183in}}%
\pgfpathlineto{\pgfqpoint{1.823598in}{3.148280in}}%
\pgfpathclose%
\pgfusepath{fill}%
\end{pgfscope}%
\begin{pgfscope}%
\pgfpathrectangle{\pgfqpoint{0.637500in}{0.550000in}}{\pgfqpoint{3.850000in}{3.850000in}}%
\pgfusepath{clip}%
\pgfsetbuttcap%
\pgfsetroundjoin%
\definecolor{currentfill}{rgb}{0.898098,0.000000,0.000000}%
\pgfsetfillcolor{currentfill}%
\pgfsetfillopacity{0.800000}%
\pgfsetlinewidth{0.000000pt}%
\definecolor{currentstroke}{rgb}{0.000000,0.000000,0.000000}%
\pgfsetstrokecolor{currentstroke}%
\pgfsetdash{}{0pt}%
\pgfpathmoveto{\pgfqpoint{2.452866in}{2.785398in}}%
\pgfpathlineto{\pgfqpoint{2.382947in}{2.823901in}}%
\pgfpathlineto{\pgfqpoint{2.452866in}{2.783199in}}%
\pgfpathlineto{\pgfqpoint{2.522784in}{2.816150in}}%
\pgfpathlineto{\pgfqpoint{2.452866in}{2.785398in}}%
\pgfpathclose%
\pgfusepath{fill}%
\end{pgfscope}%
\begin{pgfscope}%
\pgfpathrectangle{\pgfqpoint{0.637500in}{0.550000in}}{\pgfqpoint{3.850000in}{3.850000in}}%
\pgfusepath{clip}%
\pgfsetbuttcap%
\pgfsetroundjoin%
\definecolor{currentfill}{rgb}{0.897487,0.000000,0.000000}%
\pgfsetfillcolor{currentfill}%
\pgfsetfillopacity{0.800000}%
\pgfsetlinewidth{0.000000pt}%
\definecolor{currentstroke}{rgb}{0.000000,0.000000,0.000000}%
\pgfsetstrokecolor{currentstroke}%
\pgfsetdash{}{0pt}%
\pgfpathmoveto{\pgfqpoint{2.592703in}{2.857247in}}%
\pgfpathlineto{\pgfqpoint{2.522784in}{2.816150in}}%
\pgfpathlineto{\pgfqpoint{2.592703in}{2.855048in}}%
\pgfpathlineto{\pgfqpoint{2.662622in}{2.896145in}}%
\pgfpathlineto{\pgfqpoint{2.592703in}{2.857247in}}%
\pgfpathclose%
\pgfusepath{fill}%
\end{pgfscope}%
\begin{pgfscope}%
\pgfpathrectangle{\pgfqpoint{0.637500in}{0.550000in}}{\pgfqpoint{3.850000in}{3.850000in}}%
\pgfusepath{clip}%
\pgfsetbuttcap%
\pgfsetroundjoin%
\definecolor{currentfill}{rgb}{0.897487,0.000000,0.000000}%
\pgfsetfillcolor{currentfill}%
\pgfsetfillopacity{0.800000}%
\pgfsetlinewidth{0.000000pt}%
\definecolor{currentstroke}{rgb}{0.000000,0.000000,0.000000}%
\pgfsetstrokecolor{currentstroke}%
\pgfsetdash{}{0pt}%
\pgfpathmoveto{\pgfqpoint{2.802459in}{2.978340in}}%
\pgfpathlineto{\pgfqpoint{2.732540in}{2.937242in}}%
\pgfpathlineto{\pgfqpoint{2.802459in}{2.976141in}}%
\pgfpathlineto{\pgfqpoint{2.872378in}{3.017238in}}%
\pgfpathlineto{\pgfqpoint{2.802459in}{2.978340in}}%
\pgfpathclose%
\pgfusepath{fill}%
\end{pgfscope}%
\begin{pgfscope}%
\pgfpathrectangle{\pgfqpoint{0.637500in}{0.550000in}}{\pgfqpoint{3.850000in}{3.850000in}}%
\pgfusepath{clip}%
\pgfsetbuttcap%
\pgfsetroundjoin%
\definecolor{currentfill}{rgb}{0.897487,0.000000,0.000000}%
\pgfsetfillcolor{currentfill}%
\pgfsetfillopacity{0.800000}%
\pgfsetlinewidth{0.000000pt}%
\definecolor{currentstroke}{rgb}{0.000000,0.000000,0.000000}%
\pgfsetstrokecolor{currentstroke}%
\pgfsetdash{}{0pt}%
\pgfpathmoveto{\pgfqpoint{3.012215in}{3.099432in}}%
\pgfpathlineto{\pgfqpoint{2.942296in}{3.058335in}}%
\pgfpathlineto{\pgfqpoint{3.012215in}{3.097234in}}%
\pgfpathlineto{\pgfqpoint{3.082134in}{3.138331in}}%
\pgfpathlineto{\pgfqpoint{3.012215in}{3.099432in}}%
\pgfpathclose%
\pgfusepath{fill}%
\end{pgfscope}%
\begin{pgfscope}%
\pgfpathrectangle{\pgfqpoint{0.637500in}{0.550000in}}{\pgfqpoint{3.850000in}{3.850000in}}%
\pgfusepath{clip}%
\pgfsetbuttcap%
\pgfsetroundjoin%
\definecolor{currentfill}{rgb}{0.897487,0.000000,0.000000}%
\pgfsetfillcolor{currentfill}%
\pgfsetfillopacity{0.800000}%
\pgfsetlinewidth{0.000000pt}%
\definecolor{currentstroke}{rgb}{0.000000,0.000000,0.000000}%
\pgfsetstrokecolor{currentstroke}%
\pgfsetdash{}{0pt}%
\pgfpathmoveto{\pgfqpoint{3.221971in}{3.220525in}}%
\pgfpathlineto{\pgfqpoint{3.152052in}{3.179428in}}%
\pgfpathlineto{\pgfqpoint{3.221971in}{3.218326in}}%
\pgfpathlineto{\pgfqpoint{3.291890in}{3.259423in}}%
\pgfpathlineto{\pgfqpoint{3.221971in}{3.220525in}}%
\pgfpathclose%
\pgfusepath{fill}%
\end{pgfscope}%
\begin{pgfscope}%
\pgfpathrectangle{\pgfqpoint{0.637500in}{0.550000in}}{\pgfqpoint{3.850000in}{3.850000in}}%
\pgfusepath{clip}%
\pgfsetbuttcap%
\pgfsetroundjoin%
\definecolor{currentfill}{rgb}{0.897487,0.000000,0.000000}%
\pgfsetfillcolor{currentfill}%
\pgfsetfillopacity{0.800000}%
\pgfsetlinewidth{0.000000pt}%
\definecolor{currentstroke}{rgb}{0.000000,0.000000,0.000000}%
\pgfsetstrokecolor{currentstroke}%
\pgfsetdash{}{0pt}%
\pgfpathmoveto{\pgfqpoint{2.313028in}{2.864998in}}%
\pgfpathlineto{\pgfqpoint{2.243110in}{2.903896in}}%
\pgfpathlineto{\pgfqpoint{2.313028in}{2.862799in}}%
\pgfpathlineto{\pgfqpoint{2.382947in}{2.823901in}}%
\pgfpathlineto{\pgfqpoint{2.313028in}{2.864998in}}%
\pgfpathclose%
\pgfusepath{fill}%
\end{pgfscope}%
\begin{pgfscope}%
\pgfpathrectangle{\pgfqpoint{0.637500in}{0.550000in}}{\pgfqpoint{3.850000in}{3.850000in}}%
\pgfusepath{clip}%
\pgfsetbuttcap%
\pgfsetroundjoin%
\definecolor{currentfill}{rgb}{0.897487,0.000000,0.000000}%
\pgfsetfillcolor{currentfill}%
\pgfsetfillopacity{0.800000}%
\pgfsetlinewidth{0.000000pt}%
\definecolor{currentstroke}{rgb}{0.000000,0.000000,0.000000}%
\pgfsetstrokecolor{currentstroke}%
\pgfsetdash{}{0pt}%
\pgfpathmoveto{\pgfqpoint{2.103272in}{2.986090in}}%
\pgfpathlineto{\pgfqpoint{2.033354in}{3.024989in}}%
\pgfpathlineto{\pgfqpoint{2.103272in}{2.983892in}}%
\pgfpathlineto{\pgfqpoint{2.173191in}{2.944993in}}%
\pgfpathlineto{\pgfqpoint{2.103272in}{2.986090in}}%
\pgfpathclose%
\pgfusepath{fill}%
\end{pgfscope}%
\begin{pgfscope}%
\pgfpathrectangle{\pgfqpoint{0.637500in}{0.550000in}}{\pgfqpoint{3.850000in}{3.850000in}}%
\pgfusepath{clip}%
\pgfsetbuttcap%
\pgfsetroundjoin%
\definecolor{currentfill}{rgb}{0.897487,0.000000,0.000000}%
\pgfsetfillcolor{currentfill}%
\pgfsetfillopacity{0.800000}%
\pgfsetlinewidth{0.000000pt}%
\definecolor{currentstroke}{rgb}{0.000000,0.000000,0.000000}%
\pgfsetstrokecolor{currentstroke}%
\pgfsetdash{}{0pt}%
\pgfpathmoveto{\pgfqpoint{1.893516in}{3.107183in}}%
\pgfpathlineto{\pgfqpoint{1.823598in}{3.146082in}}%
\pgfpathlineto{\pgfqpoint{1.893516in}{3.104984in}}%
\pgfpathlineto{\pgfqpoint{1.963435in}{3.066086in}}%
\pgfpathlineto{\pgfqpoint{1.893516in}{3.107183in}}%
\pgfpathclose%
\pgfusepath{fill}%
\end{pgfscope}%
\begin{pgfscope}%
\pgfpathrectangle{\pgfqpoint{0.637500in}{0.550000in}}{\pgfqpoint{3.850000in}{3.850000in}}%
\pgfusepath{clip}%
\pgfsetbuttcap%
\pgfsetroundjoin%
\definecolor{currentfill}{rgb}{0.897487,0.000000,0.000000}%
\pgfsetfillcolor{currentfill}%
\pgfsetfillopacity{0.800000}%
\pgfsetlinewidth{0.000000pt}%
\definecolor{currentstroke}{rgb}{0.000000,0.000000,0.000000}%
\pgfsetstrokecolor{currentstroke}%
\pgfsetdash{}{0pt}%
\pgfpathmoveto{\pgfqpoint{1.683761in}{3.228276in}}%
\pgfpathlineto{\pgfqpoint{1.613842in}{3.267174in}}%
\pgfpathlineto{\pgfqpoint{1.683761in}{3.226077in}}%
\pgfpathlineto{\pgfqpoint{1.753679in}{3.187179in}}%
\pgfpathlineto{\pgfqpoint{1.683761in}{3.228276in}}%
\pgfpathclose%
\pgfusepath{fill}%
\end{pgfscope}%
\begin{pgfscope}%
\pgfpathrectangle{\pgfqpoint{0.637500in}{0.550000in}}{\pgfqpoint{3.850000in}{3.850000in}}%
\pgfusepath{clip}%
\pgfsetbuttcap%
\pgfsetroundjoin%
\definecolor{currentfill}{rgb}{0.909646,0.000000,0.000000}%
\pgfsetfillcolor{currentfill}%
\pgfsetfillopacity{0.800000}%
\pgfsetlinewidth{0.000000pt}%
\definecolor{currentstroke}{rgb}{0.000000,0.000000,0.000000}%
\pgfsetstrokecolor{currentstroke}%
\pgfsetdash{}{0pt}%
\pgfpathmoveto{\pgfqpoint{2.522784in}{2.816150in}}%
\pgfpathlineto{\pgfqpoint{2.452866in}{2.783199in}}%
\pgfpathlineto{\pgfqpoint{2.522784in}{2.813951in}}%
\pgfpathlineto{\pgfqpoint{2.592703in}{2.855048in}}%
\pgfpathlineto{\pgfqpoint{2.522784in}{2.816150in}}%
\pgfpathclose%
\pgfusepath{fill}%
\end{pgfscope}%
\begin{pgfscope}%
\pgfpathrectangle{\pgfqpoint{0.637500in}{0.550000in}}{\pgfqpoint{3.850000in}{3.850000in}}%
\pgfusepath{clip}%
\pgfsetbuttcap%
\pgfsetroundjoin%
\definecolor{currentfill}{rgb}{0.897487,0.000000,0.000000}%
\pgfsetfillcolor{currentfill}%
\pgfsetfillopacity{0.800000}%
\pgfsetlinewidth{0.000000pt}%
\definecolor{currentstroke}{rgb}{0.000000,0.000000,0.000000}%
\pgfsetstrokecolor{currentstroke}%
\pgfsetdash{}{0pt}%
\pgfpathmoveto{\pgfqpoint{2.732540in}{2.937242in}}%
\pgfpathlineto{\pgfqpoint{2.662622in}{2.896145in}}%
\pgfpathlineto{\pgfqpoint{2.732540in}{2.935044in}}%
\pgfpathlineto{\pgfqpoint{2.802459in}{2.976141in}}%
\pgfpathlineto{\pgfqpoint{2.732540in}{2.937242in}}%
\pgfpathclose%
\pgfusepath{fill}%
\end{pgfscope}%
\begin{pgfscope}%
\pgfpathrectangle{\pgfqpoint{0.637500in}{0.550000in}}{\pgfqpoint{3.850000in}{3.850000in}}%
\pgfusepath{clip}%
\pgfsetbuttcap%
\pgfsetroundjoin%
\definecolor{currentfill}{rgb}{0.897487,0.000000,0.000000}%
\pgfsetfillcolor{currentfill}%
\pgfsetfillopacity{0.800000}%
\pgfsetlinewidth{0.000000pt}%
\definecolor{currentstroke}{rgb}{0.000000,0.000000,0.000000}%
\pgfsetstrokecolor{currentstroke}%
\pgfsetdash{}{0pt}%
\pgfpathmoveto{\pgfqpoint{2.942296in}{3.058335in}}%
\pgfpathlineto{\pgfqpoint{2.872378in}{3.017238in}}%
\pgfpathlineto{\pgfqpoint{2.942296in}{3.056136in}}%
\pgfpathlineto{\pgfqpoint{3.012215in}{3.097234in}}%
\pgfpathlineto{\pgfqpoint{2.942296in}{3.058335in}}%
\pgfpathclose%
\pgfusepath{fill}%
\end{pgfscope}%
\begin{pgfscope}%
\pgfpathrectangle{\pgfqpoint{0.637500in}{0.550000in}}{\pgfqpoint{3.850000in}{3.850000in}}%
\pgfusepath{clip}%
\pgfsetbuttcap%
\pgfsetroundjoin%
\definecolor{currentfill}{rgb}{0.897487,0.000000,0.000000}%
\pgfsetfillcolor{currentfill}%
\pgfsetfillopacity{0.800000}%
\pgfsetlinewidth{0.000000pt}%
\definecolor{currentstroke}{rgb}{0.000000,0.000000,0.000000}%
\pgfsetstrokecolor{currentstroke}%
\pgfsetdash{}{0pt}%
\pgfpathmoveto{\pgfqpoint{3.152052in}{3.179428in}}%
\pgfpathlineto{\pgfqpoint{3.082134in}{3.138331in}}%
\pgfpathlineto{\pgfqpoint{3.152052in}{3.177229in}}%
\pgfpathlineto{\pgfqpoint{3.221971in}{3.218326in}}%
\pgfpathlineto{\pgfqpoint{3.152052in}{3.179428in}}%
\pgfpathclose%
\pgfusepath{fill}%
\end{pgfscope}%
\begin{pgfscope}%
\pgfpathrectangle{\pgfqpoint{0.637500in}{0.550000in}}{\pgfqpoint{3.850000in}{3.850000in}}%
\pgfusepath{clip}%
\pgfsetbuttcap%
\pgfsetroundjoin%
\definecolor{currentfill}{rgb}{0.897487,0.000000,0.000000}%
\pgfsetfillcolor{currentfill}%
\pgfsetfillopacity{0.800000}%
\pgfsetlinewidth{0.000000pt}%
\definecolor{currentstroke}{rgb}{0.000000,0.000000,0.000000}%
\pgfsetstrokecolor{currentstroke}%
\pgfsetdash{}{0pt}%
\pgfpathmoveto{\pgfqpoint{2.382947in}{2.823901in}}%
\pgfpathlineto{\pgfqpoint{2.313028in}{2.862799in}}%
\pgfpathlineto{\pgfqpoint{2.382947in}{2.821702in}}%
\pgfpathlineto{\pgfqpoint{2.452866in}{2.783199in}}%
\pgfpathlineto{\pgfqpoint{2.382947in}{2.823901in}}%
\pgfpathclose%
\pgfusepath{fill}%
\end{pgfscope}%
\begin{pgfscope}%
\pgfpathrectangle{\pgfqpoint{0.637500in}{0.550000in}}{\pgfqpoint{3.850000in}{3.850000in}}%
\pgfusepath{clip}%
\pgfsetbuttcap%
\pgfsetroundjoin%
\definecolor{currentfill}{rgb}{0.897487,0.000000,0.000000}%
\pgfsetfillcolor{currentfill}%
\pgfsetfillopacity{0.800000}%
\pgfsetlinewidth{0.000000pt}%
\definecolor{currentstroke}{rgb}{0.000000,0.000000,0.000000}%
\pgfsetstrokecolor{currentstroke}%
\pgfsetdash{}{0pt}%
\pgfpathmoveto{\pgfqpoint{3.361808in}{3.300521in}}%
\pgfpathlineto{\pgfqpoint{3.291890in}{3.259423in}}%
\pgfpathlineto{\pgfqpoint{3.361808in}{3.298322in}}%
\pgfpathlineto{\pgfqpoint{3.431727in}{3.339419in}}%
\pgfpathlineto{\pgfqpoint{3.361808in}{3.300521in}}%
\pgfpathclose%
\pgfusepath{fill}%
\end{pgfscope}%
\begin{pgfscope}%
\pgfpathrectangle{\pgfqpoint{0.637500in}{0.550000in}}{\pgfqpoint{3.850000in}{3.850000in}}%
\pgfusepath{clip}%
\pgfsetbuttcap%
\pgfsetroundjoin%
\definecolor{currentfill}{rgb}{0.897487,0.000000,0.000000}%
\pgfsetfillcolor{currentfill}%
\pgfsetfillopacity{0.800000}%
\pgfsetlinewidth{0.000000pt}%
\definecolor{currentstroke}{rgb}{0.000000,0.000000,0.000000}%
\pgfsetstrokecolor{currentstroke}%
\pgfsetdash{}{0pt}%
\pgfpathmoveto{\pgfqpoint{2.173191in}{2.944993in}}%
\pgfpathlineto{\pgfqpoint{2.103272in}{2.983892in}}%
\pgfpathlineto{\pgfqpoint{2.173191in}{2.942795in}}%
\pgfpathlineto{\pgfqpoint{2.243110in}{2.903896in}}%
\pgfpathlineto{\pgfqpoint{2.173191in}{2.944993in}}%
\pgfpathclose%
\pgfusepath{fill}%
\end{pgfscope}%
\begin{pgfscope}%
\pgfpathrectangle{\pgfqpoint{0.637500in}{0.550000in}}{\pgfqpoint{3.850000in}{3.850000in}}%
\pgfusepath{clip}%
\pgfsetbuttcap%
\pgfsetroundjoin%
\definecolor{currentfill}{rgb}{0.897487,0.000000,0.000000}%
\pgfsetfillcolor{currentfill}%
\pgfsetfillopacity{0.800000}%
\pgfsetlinewidth{0.000000pt}%
\definecolor{currentstroke}{rgb}{0.000000,0.000000,0.000000}%
\pgfsetstrokecolor{currentstroke}%
\pgfsetdash{}{0pt}%
\pgfpathmoveto{\pgfqpoint{1.963435in}{3.066086in}}%
\pgfpathlineto{\pgfqpoint{1.893516in}{3.104984in}}%
\pgfpathlineto{\pgfqpoint{1.963435in}{3.063887in}}%
\pgfpathlineto{\pgfqpoint{2.033354in}{3.024989in}}%
\pgfpathlineto{\pgfqpoint{1.963435in}{3.066086in}}%
\pgfpathclose%
\pgfusepath{fill}%
\end{pgfscope}%
\begin{pgfscope}%
\pgfpathrectangle{\pgfqpoint{0.637500in}{0.550000in}}{\pgfqpoint{3.850000in}{3.850000in}}%
\pgfusepath{clip}%
\pgfsetbuttcap%
\pgfsetroundjoin%
\definecolor{currentfill}{rgb}{0.897487,0.000000,0.000000}%
\pgfsetfillcolor{currentfill}%
\pgfsetfillopacity{0.800000}%
\pgfsetlinewidth{0.000000pt}%
\definecolor{currentstroke}{rgb}{0.000000,0.000000,0.000000}%
\pgfsetstrokecolor{currentstroke}%
\pgfsetdash{}{0pt}%
\pgfpathmoveto{\pgfqpoint{1.753679in}{3.187179in}}%
\pgfpathlineto{\pgfqpoint{1.683761in}{3.226077in}}%
\pgfpathlineto{\pgfqpoint{1.753679in}{3.184980in}}%
\pgfpathlineto{\pgfqpoint{1.823598in}{3.146082in}}%
\pgfpathlineto{\pgfqpoint{1.753679in}{3.187179in}}%
\pgfpathclose%
\pgfusepath{fill}%
\end{pgfscope}%
\begin{pgfscope}%
\pgfpathrectangle{\pgfqpoint{0.637500in}{0.550000in}}{\pgfqpoint{3.850000in}{3.850000in}}%
\pgfusepath{clip}%
\pgfsetbuttcap%
\pgfsetroundjoin%
\definecolor{currentfill}{rgb}{0.897487,0.000000,0.000000}%
\pgfsetfillcolor{currentfill}%
\pgfsetfillopacity{0.800000}%
\pgfsetlinewidth{0.000000pt}%
\definecolor{currentstroke}{rgb}{0.000000,0.000000,0.000000}%
\pgfsetstrokecolor{currentstroke}%
\pgfsetdash{}{0pt}%
\pgfpathmoveto{\pgfqpoint{2.662622in}{2.896145in}}%
\pgfpathlineto{\pgfqpoint{2.592703in}{2.855048in}}%
\pgfpathlineto{\pgfqpoint{2.662622in}{2.893946in}}%
\pgfpathlineto{\pgfqpoint{2.732540in}{2.935044in}}%
\pgfpathlineto{\pgfqpoint{2.662622in}{2.896145in}}%
\pgfpathclose%
\pgfusepath{fill}%
\end{pgfscope}%
\begin{pgfscope}%
\pgfpathrectangle{\pgfqpoint{0.637500in}{0.550000in}}{\pgfqpoint{3.850000in}{3.850000in}}%
\pgfusepath{clip}%
\pgfsetbuttcap%
\pgfsetroundjoin%
\definecolor{currentfill}{rgb}{0.897487,0.000000,0.000000}%
\pgfsetfillcolor{currentfill}%
\pgfsetfillopacity{0.800000}%
\pgfsetlinewidth{0.000000pt}%
\definecolor{currentstroke}{rgb}{0.000000,0.000000,0.000000}%
\pgfsetstrokecolor{currentstroke}%
\pgfsetdash{}{0pt}%
\pgfpathmoveto{\pgfqpoint{2.872378in}{3.017238in}}%
\pgfpathlineto{\pgfqpoint{2.802459in}{2.976141in}}%
\pgfpathlineto{\pgfqpoint{2.872378in}{3.015039in}}%
\pgfpathlineto{\pgfqpoint{2.942296in}{3.056136in}}%
\pgfpathlineto{\pgfqpoint{2.872378in}{3.017238in}}%
\pgfpathclose%
\pgfusepath{fill}%
\end{pgfscope}%
\begin{pgfscope}%
\pgfpathrectangle{\pgfqpoint{0.637500in}{0.550000in}}{\pgfqpoint{3.850000in}{3.850000in}}%
\pgfusepath{clip}%
\pgfsetbuttcap%
\pgfsetroundjoin%
\definecolor{currentfill}{rgb}{0.897487,0.000000,0.000000}%
\pgfsetfillcolor{currentfill}%
\pgfsetfillopacity{0.800000}%
\pgfsetlinewidth{0.000000pt}%
\definecolor{currentstroke}{rgb}{0.000000,0.000000,0.000000}%
\pgfsetstrokecolor{currentstroke}%
\pgfsetdash{}{0pt}%
\pgfpathmoveto{\pgfqpoint{3.082134in}{3.138331in}}%
\pgfpathlineto{\pgfqpoint{3.012215in}{3.097234in}}%
\pgfpathlineto{\pgfqpoint{3.082134in}{3.136132in}}%
\pgfpathlineto{\pgfqpoint{3.152052in}{3.177229in}}%
\pgfpathlineto{\pgfqpoint{3.082134in}{3.138331in}}%
\pgfpathclose%
\pgfusepath{fill}%
\end{pgfscope}%
\begin{pgfscope}%
\pgfpathrectangle{\pgfqpoint{0.637500in}{0.550000in}}{\pgfqpoint{3.850000in}{3.850000in}}%
\pgfusepath{clip}%
\pgfsetbuttcap%
\pgfsetroundjoin%
\definecolor{currentfill}{rgb}{0.897487,0.000000,0.000000}%
\pgfsetfillcolor{currentfill}%
\pgfsetfillopacity{0.800000}%
\pgfsetlinewidth{0.000000pt}%
\definecolor{currentstroke}{rgb}{0.000000,0.000000,0.000000}%
\pgfsetstrokecolor{currentstroke}%
\pgfsetdash{}{0pt}%
\pgfpathmoveto{\pgfqpoint{3.291890in}{3.259423in}}%
\pgfpathlineto{\pgfqpoint{3.221971in}{3.218326in}}%
\pgfpathlineto{\pgfqpoint{3.291890in}{3.257225in}}%
\pgfpathlineto{\pgfqpoint{3.361808in}{3.298322in}}%
\pgfpathlineto{\pgfqpoint{3.291890in}{3.259423in}}%
\pgfpathclose%
\pgfusepath{fill}%
\end{pgfscope}%
\begin{pgfscope}%
\pgfpathrectangle{\pgfqpoint{0.637500in}{0.550000in}}{\pgfqpoint{3.850000in}{3.850000in}}%
\pgfusepath{clip}%
\pgfsetbuttcap%
\pgfsetroundjoin%
\definecolor{currentfill}{rgb}{0.897487,0.000000,0.000000}%
\pgfsetfillcolor{currentfill}%
\pgfsetfillopacity{0.800000}%
\pgfsetlinewidth{0.000000pt}%
\definecolor{currentstroke}{rgb}{0.000000,0.000000,0.000000}%
\pgfsetstrokecolor{currentstroke}%
\pgfsetdash{}{0pt}%
\pgfpathmoveto{\pgfqpoint{2.243110in}{2.903896in}}%
\pgfpathlineto{\pgfqpoint{2.173191in}{2.942795in}}%
\pgfpathlineto{\pgfqpoint{2.243110in}{2.901697in}}%
\pgfpathlineto{\pgfqpoint{2.313028in}{2.862799in}}%
\pgfpathlineto{\pgfqpoint{2.243110in}{2.903896in}}%
\pgfpathclose%
\pgfusepath{fill}%
\end{pgfscope}%
\begin{pgfscope}%
\pgfpathrectangle{\pgfqpoint{0.637500in}{0.550000in}}{\pgfqpoint{3.850000in}{3.850000in}}%
\pgfusepath{clip}%
\pgfsetbuttcap%
\pgfsetroundjoin%
\definecolor{currentfill}{rgb}{0.897487,0.000000,0.000000}%
\pgfsetfillcolor{currentfill}%
\pgfsetfillopacity{0.800000}%
\pgfsetlinewidth{0.000000pt}%
\definecolor{currentstroke}{rgb}{0.000000,0.000000,0.000000}%
\pgfsetstrokecolor{currentstroke}%
\pgfsetdash{}{0pt}%
\pgfpathmoveto{\pgfqpoint{2.033354in}{3.024989in}}%
\pgfpathlineto{\pgfqpoint{1.963435in}{3.063887in}}%
\pgfpathlineto{\pgfqpoint{2.033354in}{3.022790in}}%
\pgfpathlineto{\pgfqpoint{2.103272in}{2.983892in}}%
\pgfpathlineto{\pgfqpoint{2.033354in}{3.024989in}}%
\pgfpathclose%
\pgfusepath{fill}%
\end{pgfscope}%
\begin{pgfscope}%
\pgfpathrectangle{\pgfqpoint{0.637500in}{0.550000in}}{\pgfqpoint{3.850000in}{3.850000in}}%
\pgfusepath{clip}%
\pgfsetbuttcap%
\pgfsetroundjoin%
\definecolor{currentfill}{rgb}{0.897487,0.000000,0.000000}%
\pgfsetfillcolor{currentfill}%
\pgfsetfillopacity{0.800000}%
\pgfsetlinewidth{0.000000pt}%
\definecolor{currentstroke}{rgb}{0.000000,0.000000,0.000000}%
\pgfsetstrokecolor{currentstroke}%
\pgfsetdash{}{0pt}%
\pgfpathmoveto{\pgfqpoint{1.823598in}{3.146082in}}%
\pgfpathlineto{\pgfqpoint{1.753679in}{3.184980in}}%
\pgfpathlineto{\pgfqpoint{1.823598in}{3.143883in}}%
\pgfpathlineto{\pgfqpoint{1.893516in}{3.104984in}}%
\pgfpathlineto{\pgfqpoint{1.823598in}{3.146082in}}%
\pgfpathclose%
\pgfusepath{fill}%
\end{pgfscope}%
\begin{pgfscope}%
\pgfpathrectangle{\pgfqpoint{0.637500in}{0.550000in}}{\pgfqpoint{3.850000in}{3.850000in}}%
\pgfusepath{clip}%
\pgfsetbuttcap%
\pgfsetroundjoin%
\definecolor{currentfill}{rgb}{0.897487,0.000000,0.000000}%
\pgfsetfillcolor{currentfill}%
\pgfsetfillopacity{0.800000}%
\pgfsetlinewidth{0.000000pt}%
\definecolor{currentstroke}{rgb}{0.000000,0.000000,0.000000}%
\pgfsetstrokecolor{currentstroke}%
\pgfsetdash{}{0pt}%
\pgfpathmoveto{\pgfqpoint{1.613842in}{3.267174in}}%
\pgfpathlineto{\pgfqpoint{1.543923in}{3.306073in}}%
\pgfpathlineto{\pgfqpoint{1.613842in}{3.264975in}}%
\pgfpathlineto{\pgfqpoint{1.683761in}{3.226077in}}%
\pgfpathlineto{\pgfqpoint{1.613842in}{3.267174in}}%
\pgfpathclose%
\pgfusepath{fill}%
\end{pgfscope}%
\begin{pgfscope}%
\pgfpathrectangle{\pgfqpoint{0.637500in}{0.550000in}}{\pgfqpoint{3.850000in}{3.850000in}}%
\pgfusepath{clip}%
\pgfsetbuttcap%
\pgfsetroundjoin%
\definecolor{currentfill}{rgb}{0.898098,0.000000,0.000000}%
\pgfsetfillcolor{currentfill}%
\pgfsetfillopacity{0.800000}%
\pgfsetlinewidth{0.000000pt}%
\definecolor{currentstroke}{rgb}{0.000000,0.000000,0.000000}%
\pgfsetstrokecolor{currentstroke}%
\pgfsetdash{}{0pt}%
\pgfpathmoveto{\pgfqpoint{2.452866in}{2.783199in}}%
\pgfpathlineto{\pgfqpoint{2.382947in}{2.821702in}}%
\pgfpathlineto{\pgfqpoint{2.452866in}{2.781000in}}%
\pgfpathlineto{\pgfqpoint{2.522784in}{2.813951in}}%
\pgfpathlineto{\pgfqpoint{2.452866in}{2.783199in}}%
\pgfpathclose%
\pgfusepath{fill}%
\end{pgfscope}%
\begin{pgfscope}%
\pgfpathrectangle{\pgfqpoint{0.637500in}{0.550000in}}{\pgfqpoint{3.850000in}{3.850000in}}%
\pgfusepath{clip}%
\pgfsetbuttcap%
\pgfsetroundjoin%
\definecolor{currentfill}{rgb}{0.897487,0.000000,0.000000}%
\pgfsetfillcolor{currentfill}%
\pgfsetfillopacity{0.800000}%
\pgfsetlinewidth{0.000000pt}%
\definecolor{currentstroke}{rgb}{0.000000,0.000000,0.000000}%
\pgfsetstrokecolor{currentstroke}%
\pgfsetdash{}{0pt}%
\pgfpathmoveto{\pgfqpoint{2.592703in}{2.855048in}}%
\pgfpathlineto{\pgfqpoint{2.522784in}{2.813951in}}%
\pgfpathlineto{\pgfqpoint{2.592703in}{2.852849in}}%
\pgfpathlineto{\pgfqpoint{2.662622in}{2.893946in}}%
\pgfpathlineto{\pgfqpoint{2.592703in}{2.855048in}}%
\pgfpathclose%
\pgfusepath{fill}%
\end{pgfscope}%
\begin{pgfscope}%
\pgfpathrectangle{\pgfqpoint{0.637500in}{0.550000in}}{\pgfqpoint{3.850000in}{3.850000in}}%
\pgfusepath{clip}%
\pgfsetbuttcap%
\pgfsetroundjoin%
\definecolor{currentfill}{rgb}{0.897487,0.000000,0.000000}%
\pgfsetfillcolor{currentfill}%
\pgfsetfillopacity{0.800000}%
\pgfsetlinewidth{0.000000pt}%
\definecolor{currentstroke}{rgb}{0.000000,0.000000,0.000000}%
\pgfsetstrokecolor{currentstroke}%
\pgfsetdash{}{0pt}%
\pgfpathmoveto{\pgfqpoint{2.802459in}{2.976141in}}%
\pgfpathlineto{\pgfqpoint{2.732540in}{2.935044in}}%
\pgfpathlineto{\pgfqpoint{2.802459in}{2.973942in}}%
\pgfpathlineto{\pgfqpoint{2.872378in}{3.015039in}}%
\pgfpathlineto{\pgfqpoint{2.802459in}{2.976141in}}%
\pgfpathclose%
\pgfusepath{fill}%
\end{pgfscope}%
\begin{pgfscope}%
\pgfpathrectangle{\pgfqpoint{0.637500in}{0.550000in}}{\pgfqpoint{3.850000in}{3.850000in}}%
\pgfusepath{clip}%
\pgfsetbuttcap%
\pgfsetroundjoin%
\definecolor{currentfill}{rgb}{0.897487,0.000000,0.000000}%
\pgfsetfillcolor{currentfill}%
\pgfsetfillopacity{0.800000}%
\pgfsetlinewidth{0.000000pt}%
\definecolor{currentstroke}{rgb}{0.000000,0.000000,0.000000}%
\pgfsetstrokecolor{currentstroke}%
\pgfsetdash{}{0pt}%
\pgfpathmoveto{\pgfqpoint{3.012215in}{3.097234in}}%
\pgfpathlineto{\pgfqpoint{2.942296in}{3.056136in}}%
\pgfpathlineto{\pgfqpoint{3.012215in}{3.095035in}}%
\pgfpathlineto{\pgfqpoint{3.082134in}{3.136132in}}%
\pgfpathlineto{\pgfqpoint{3.012215in}{3.097234in}}%
\pgfpathclose%
\pgfusepath{fill}%
\end{pgfscope}%
\begin{pgfscope}%
\pgfpathrectangle{\pgfqpoint{0.637500in}{0.550000in}}{\pgfqpoint{3.850000in}{3.850000in}}%
\pgfusepath{clip}%
\pgfsetbuttcap%
\pgfsetroundjoin%
\definecolor{currentfill}{rgb}{0.897487,0.000000,0.000000}%
\pgfsetfillcolor{currentfill}%
\pgfsetfillopacity{0.800000}%
\pgfsetlinewidth{0.000000pt}%
\definecolor{currentstroke}{rgb}{0.000000,0.000000,0.000000}%
\pgfsetstrokecolor{currentstroke}%
\pgfsetdash{}{0pt}%
\pgfpathmoveto{\pgfqpoint{3.221971in}{3.218326in}}%
\pgfpathlineto{\pgfqpoint{3.152052in}{3.177229in}}%
\pgfpathlineto{\pgfqpoint{3.221971in}{3.216127in}}%
\pgfpathlineto{\pgfqpoint{3.291890in}{3.257225in}}%
\pgfpathlineto{\pgfqpoint{3.221971in}{3.218326in}}%
\pgfpathclose%
\pgfusepath{fill}%
\end{pgfscope}%
\begin{pgfscope}%
\pgfpathrectangle{\pgfqpoint{0.637500in}{0.550000in}}{\pgfqpoint{3.850000in}{3.850000in}}%
\pgfusepath{clip}%
\pgfsetbuttcap%
\pgfsetroundjoin%
\definecolor{currentfill}{rgb}{0.897487,0.000000,0.000000}%
\pgfsetfillcolor{currentfill}%
\pgfsetfillopacity{0.800000}%
\pgfsetlinewidth{0.000000pt}%
\definecolor{currentstroke}{rgb}{0.000000,0.000000,0.000000}%
\pgfsetstrokecolor{currentstroke}%
\pgfsetdash{}{0pt}%
\pgfpathmoveto{\pgfqpoint{2.313028in}{2.862799in}}%
\pgfpathlineto{\pgfqpoint{2.243110in}{2.901697in}}%
\pgfpathlineto{\pgfqpoint{2.313028in}{2.860600in}}%
\pgfpathlineto{\pgfqpoint{2.382947in}{2.821702in}}%
\pgfpathlineto{\pgfqpoint{2.313028in}{2.862799in}}%
\pgfpathclose%
\pgfusepath{fill}%
\end{pgfscope}%
\begin{pgfscope}%
\pgfpathrectangle{\pgfqpoint{0.637500in}{0.550000in}}{\pgfqpoint{3.850000in}{3.850000in}}%
\pgfusepath{clip}%
\pgfsetbuttcap%
\pgfsetroundjoin%
\definecolor{currentfill}{rgb}{0.897487,0.000000,0.000000}%
\pgfsetfillcolor{currentfill}%
\pgfsetfillopacity{0.800000}%
\pgfsetlinewidth{0.000000pt}%
\definecolor{currentstroke}{rgb}{0.000000,0.000000,0.000000}%
\pgfsetstrokecolor{currentstroke}%
\pgfsetdash{}{0pt}%
\pgfpathmoveto{\pgfqpoint{3.431727in}{3.339419in}}%
\pgfpathlineto{\pgfqpoint{3.361808in}{3.298322in}}%
\pgfpathlineto{\pgfqpoint{3.431727in}{3.337220in}}%
\pgfpathlineto{\pgfqpoint{3.501646in}{3.378317in}}%
\pgfpathlineto{\pgfqpoint{3.431727in}{3.339419in}}%
\pgfpathclose%
\pgfusepath{fill}%
\end{pgfscope}%
\begin{pgfscope}%
\pgfpathrectangle{\pgfqpoint{0.637500in}{0.550000in}}{\pgfqpoint{3.850000in}{3.850000in}}%
\pgfusepath{clip}%
\pgfsetbuttcap%
\pgfsetroundjoin%
\definecolor{currentfill}{rgb}{0.897487,0.000000,0.000000}%
\pgfsetfillcolor{currentfill}%
\pgfsetfillopacity{0.800000}%
\pgfsetlinewidth{0.000000pt}%
\definecolor{currentstroke}{rgb}{0.000000,0.000000,0.000000}%
\pgfsetstrokecolor{currentstroke}%
\pgfsetdash{}{0pt}%
\pgfpathmoveto{\pgfqpoint{2.103272in}{2.983892in}}%
\pgfpathlineto{\pgfqpoint{2.033354in}{3.022790in}}%
\pgfpathlineto{\pgfqpoint{2.103272in}{2.981693in}}%
\pgfpathlineto{\pgfqpoint{2.173191in}{2.942795in}}%
\pgfpathlineto{\pgfqpoint{2.103272in}{2.983892in}}%
\pgfpathclose%
\pgfusepath{fill}%
\end{pgfscope}%
\begin{pgfscope}%
\pgfpathrectangle{\pgfqpoint{0.637500in}{0.550000in}}{\pgfqpoint{3.850000in}{3.850000in}}%
\pgfusepath{clip}%
\pgfsetbuttcap%
\pgfsetroundjoin%
\definecolor{currentfill}{rgb}{0.897487,0.000000,0.000000}%
\pgfsetfillcolor{currentfill}%
\pgfsetfillopacity{0.800000}%
\pgfsetlinewidth{0.000000pt}%
\definecolor{currentstroke}{rgb}{0.000000,0.000000,0.000000}%
\pgfsetstrokecolor{currentstroke}%
\pgfsetdash{}{0pt}%
\pgfpathmoveto{\pgfqpoint{1.893516in}{3.104984in}}%
\pgfpathlineto{\pgfqpoint{1.823598in}{3.143883in}}%
\pgfpathlineto{\pgfqpoint{1.893516in}{3.102786in}}%
\pgfpathlineto{\pgfqpoint{1.963435in}{3.063887in}}%
\pgfpathlineto{\pgfqpoint{1.893516in}{3.104984in}}%
\pgfpathclose%
\pgfusepath{fill}%
\end{pgfscope}%
\begin{pgfscope}%
\pgfpathrectangle{\pgfqpoint{0.637500in}{0.550000in}}{\pgfqpoint{3.850000in}{3.850000in}}%
\pgfusepath{clip}%
\pgfsetbuttcap%
\pgfsetroundjoin%
\definecolor{currentfill}{rgb}{0.897487,0.000000,0.000000}%
\pgfsetfillcolor{currentfill}%
\pgfsetfillopacity{0.800000}%
\pgfsetlinewidth{0.000000pt}%
\definecolor{currentstroke}{rgb}{0.000000,0.000000,0.000000}%
\pgfsetstrokecolor{currentstroke}%
\pgfsetdash{}{0pt}%
\pgfpathmoveto{\pgfqpoint{1.683761in}{3.226077in}}%
\pgfpathlineto{\pgfqpoint{1.613842in}{3.264975in}}%
\pgfpathlineto{\pgfqpoint{1.683761in}{3.223878in}}%
\pgfpathlineto{\pgfqpoint{1.753679in}{3.184980in}}%
\pgfpathlineto{\pgfqpoint{1.683761in}{3.226077in}}%
\pgfpathclose%
\pgfusepath{fill}%
\end{pgfscope}%
\begin{pgfscope}%
\pgfpathrectangle{\pgfqpoint{0.637500in}{0.550000in}}{\pgfqpoint{3.850000in}{3.850000in}}%
\pgfusepath{clip}%
\pgfsetbuttcap%
\pgfsetroundjoin%
\definecolor{currentfill}{rgb}{0.909646,0.000000,0.000000}%
\pgfsetfillcolor{currentfill}%
\pgfsetfillopacity{0.800000}%
\pgfsetlinewidth{0.000000pt}%
\definecolor{currentstroke}{rgb}{0.000000,0.000000,0.000000}%
\pgfsetstrokecolor{currentstroke}%
\pgfsetdash{}{0pt}%
\pgfpathmoveto{\pgfqpoint{2.522784in}{2.813951in}}%
\pgfpathlineto{\pgfqpoint{2.452866in}{2.781000in}}%
\pgfpathlineto{\pgfqpoint{2.522784in}{2.811752in}}%
\pgfpathlineto{\pgfqpoint{2.592703in}{2.852849in}}%
\pgfpathlineto{\pgfqpoint{2.522784in}{2.813951in}}%
\pgfpathclose%
\pgfusepath{fill}%
\end{pgfscope}%
\begin{pgfscope}%
\pgfpathrectangle{\pgfqpoint{0.637500in}{0.550000in}}{\pgfqpoint{3.850000in}{3.850000in}}%
\pgfusepath{clip}%
\pgfsetbuttcap%
\pgfsetroundjoin%
\definecolor{currentfill}{rgb}{0.897487,0.000000,0.000000}%
\pgfsetfillcolor{currentfill}%
\pgfsetfillopacity{0.800000}%
\pgfsetlinewidth{0.000000pt}%
\definecolor{currentstroke}{rgb}{0.000000,0.000000,0.000000}%
\pgfsetstrokecolor{currentstroke}%
\pgfsetdash{}{0pt}%
\pgfpathmoveto{\pgfqpoint{2.732540in}{2.935044in}}%
\pgfpathlineto{\pgfqpoint{2.662622in}{2.893946in}}%
\pgfpathlineto{\pgfqpoint{2.732540in}{2.932845in}}%
\pgfpathlineto{\pgfqpoint{2.802459in}{2.973942in}}%
\pgfpathlineto{\pgfqpoint{2.732540in}{2.935044in}}%
\pgfpathclose%
\pgfusepath{fill}%
\end{pgfscope}%
\begin{pgfscope}%
\pgfpathrectangle{\pgfqpoint{0.637500in}{0.550000in}}{\pgfqpoint{3.850000in}{3.850000in}}%
\pgfusepath{clip}%
\pgfsetbuttcap%
\pgfsetroundjoin%
\definecolor{currentfill}{rgb}{0.897487,0.000000,0.000000}%
\pgfsetfillcolor{currentfill}%
\pgfsetfillopacity{0.800000}%
\pgfsetlinewidth{0.000000pt}%
\definecolor{currentstroke}{rgb}{0.000000,0.000000,0.000000}%
\pgfsetstrokecolor{currentstroke}%
\pgfsetdash{}{0pt}%
\pgfpathmoveto{\pgfqpoint{2.942296in}{3.056136in}}%
\pgfpathlineto{\pgfqpoint{2.872378in}{3.015039in}}%
\pgfpathlineto{\pgfqpoint{2.942296in}{3.053938in}}%
\pgfpathlineto{\pgfqpoint{3.012215in}{3.095035in}}%
\pgfpathlineto{\pgfqpoint{2.942296in}{3.056136in}}%
\pgfpathclose%
\pgfusepath{fill}%
\end{pgfscope}%
\begin{pgfscope}%
\pgfpathrectangle{\pgfqpoint{0.637500in}{0.550000in}}{\pgfqpoint{3.850000in}{3.850000in}}%
\pgfusepath{clip}%
\pgfsetbuttcap%
\pgfsetroundjoin%
\definecolor{currentfill}{rgb}{0.897487,0.000000,0.000000}%
\pgfsetfillcolor{currentfill}%
\pgfsetfillopacity{0.800000}%
\pgfsetlinewidth{0.000000pt}%
\definecolor{currentstroke}{rgb}{0.000000,0.000000,0.000000}%
\pgfsetstrokecolor{currentstroke}%
\pgfsetdash{}{0pt}%
\pgfpathmoveto{\pgfqpoint{3.152052in}{3.177229in}}%
\pgfpathlineto{\pgfqpoint{3.082134in}{3.136132in}}%
\pgfpathlineto{\pgfqpoint{3.152052in}{3.175030in}}%
\pgfpathlineto{\pgfqpoint{3.221971in}{3.216127in}}%
\pgfpathlineto{\pgfqpoint{3.152052in}{3.177229in}}%
\pgfpathclose%
\pgfusepath{fill}%
\end{pgfscope}%
\begin{pgfscope}%
\pgfpathrectangle{\pgfqpoint{0.637500in}{0.550000in}}{\pgfqpoint{3.850000in}{3.850000in}}%
\pgfusepath{clip}%
\pgfsetbuttcap%
\pgfsetroundjoin%
\definecolor{currentfill}{rgb}{0.897487,0.000000,0.000000}%
\pgfsetfillcolor{currentfill}%
\pgfsetfillopacity{0.800000}%
\pgfsetlinewidth{0.000000pt}%
\definecolor{currentstroke}{rgb}{0.000000,0.000000,0.000000}%
\pgfsetstrokecolor{currentstroke}%
\pgfsetdash{}{0pt}%
\pgfpathmoveto{\pgfqpoint{2.382947in}{2.821702in}}%
\pgfpathlineto{\pgfqpoint{2.313028in}{2.860600in}}%
\pgfpathlineto{\pgfqpoint{2.382947in}{2.819503in}}%
\pgfpathlineto{\pgfqpoint{2.452866in}{2.781000in}}%
\pgfpathlineto{\pgfqpoint{2.382947in}{2.821702in}}%
\pgfpathclose%
\pgfusepath{fill}%
\end{pgfscope}%
\begin{pgfscope}%
\pgfpathrectangle{\pgfqpoint{0.637500in}{0.550000in}}{\pgfqpoint{3.850000in}{3.850000in}}%
\pgfusepath{clip}%
\pgfsetbuttcap%
\pgfsetroundjoin%
\definecolor{currentfill}{rgb}{0.897487,0.000000,0.000000}%
\pgfsetfillcolor{currentfill}%
\pgfsetfillopacity{0.800000}%
\pgfsetlinewidth{0.000000pt}%
\definecolor{currentstroke}{rgb}{0.000000,0.000000,0.000000}%
\pgfsetstrokecolor{currentstroke}%
\pgfsetdash{}{0pt}%
\pgfpathmoveto{\pgfqpoint{3.361808in}{3.298322in}}%
\pgfpathlineto{\pgfqpoint{3.291890in}{3.257225in}}%
\pgfpathlineto{\pgfqpoint{3.361808in}{3.296123in}}%
\pgfpathlineto{\pgfqpoint{3.431727in}{3.337220in}}%
\pgfpathlineto{\pgfqpoint{3.361808in}{3.298322in}}%
\pgfpathclose%
\pgfusepath{fill}%
\end{pgfscope}%
\begin{pgfscope}%
\pgfpathrectangle{\pgfqpoint{0.637500in}{0.550000in}}{\pgfqpoint{3.850000in}{3.850000in}}%
\pgfusepath{clip}%
\pgfsetbuttcap%
\pgfsetroundjoin%
\definecolor{currentfill}{rgb}{0.897487,0.000000,0.000000}%
\pgfsetfillcolor{currentfill}%
\pgfsetfillopacity{0.800000}%
\pgfsetlinewidth{0.000000pt}%
\definecolor{currentstroke}{rgb}{0.000000,0.000000,0.000000}%
\pgfsetstrokecolor{currentstroke}%
\pgfsetdash{}{0pt}%
\pgfpathmoveto{\pgfqpoint{2.173191in}{2.942795in}}%
\pgfpathlineto{\pgfqpoint{2.103272in}{2.981693in}}%
\pgfpathlineto{\pgfqpoint{2.173191in}{2.940596in}}%
\pgfpathlineto{\pgfqpoint{2.243110in}{2.901697in}}%
\pgfpathlineto{\pgfqpoint{2.173191in}{2.942795in}}%
\pgfpathclose%
\pgfusepath{fill}%
\end{pgfscope}%
\begin{pgfscope}%
\pgfpathrectangle{\pgfqpoint{0.637500in}{0.550000in}}{\pgfqpoint{3.850000in}{3.850000in}}%
\pgfusepath{clip}%
\pgfsetbuttcap%
\pgfsetroundjoin%
\definecolor{currentfill}{rgb}{0.897487,0.000000,0.000000}%
\pgfsetfillcolor{currentfill}%
\pgfsetfillopacity{0.800000}%
\pgfsetlinewidth{0.000000pt}%
\definecolor{currentstroke}{rgb}{0.000000,0.000000,0.000000}%
\pgfsetstrokecolor{currentstroke}%
\pgfsetdash{}{0pt}%
\pgfpathmoveto{\pgfqpoint{1.963435in}{3.063887in}}%
\pgfpathlineto{\pgfqpoint{1.893516in}{3.102786in}}%
\pgfpathlineto{\pgfqpoint{1.963435in}{3.061688in}}%
\pgfpathlineto{\pgfqpoint{2.033354in}{3.022790in}}%
\pgfpathlineto{\pgfqpoint{1.963435in}{3.063887in}}%
\pgfpathclose%
\pgfusepath{fill}%
\end{pgfscope}%
\begin{pgfscope}%
\pgfpathrectangle{\pgfqpoint{0.637500in}{0.550000in}}{\pgfqpoint{3.850000in}{3.850000in}}%
\pgfusepath{clip}%
\pgfsetbuttcap%
\pgfsetroundjoin%
\definecolor{currentfill}{rgb}{0.897487,0.000000,0.000000}%
\pgfsetfillcolor{currentfill}%
\pgfsetfillopacity{0.800000}%
\pgfsetlinewidth{0.000000pt}%
\definecolor{currentstroke}{rgb}{0.000000,0.000000,0.000000}%
\pgfsetstrokecolor{currentstroke}%
\pgfsetdash{}{0pt}%
\pgfpathmoveto{\pgfqpoint{1.753679in}{3.184980in}}%
\pgfpathlineto{\pgfqpoint{1.683761in}{3.223878in}}%
\pgfpathlineto{\pgfqpoint{1.753679in}{3.182781in}}%
\pgfpathlineto{\pgfqpoint{1.823598in}{3.143883in}}%
\pgfpathlineto{\pgfqpoint{1.753679in}{3.184980in}}%
\pgfpathclose%
\pgfusepath{fill}%
\end{pgfscope}%
\begin{pgfscope}%
\pgfpathrectangle{\pgfqpoint{0.637500in}{0.550000in}}{\pgfqpoint{3.850000in}{3.850000in}}%
\pgfusepath{clip}%
\pgfsetbuttcap%
\pgfsetroundjoin%
\definecolor{currentfill}{rgb}{0.897487,0.000000,0.000000}%
\pgfsetfillcolor{currentfill}%
\pgfsetfillopacity{0.800000}%
\pgfsetlinewidth{0.000000pt}%
\definecolor{currentstroke}{rgb}{0.000000,0.000000,0.000000}%
\pgfsetstrokecolor{currentstroke}%
\pgfsetdash{}{0pt}%
\pgfpathmoveto{\pgfqpoint{1.543923in}{3.306073in}}%
\pgfpathlineto{\pgfqpoint{1.474005in}{3.344971in}}%
\pgfpathlineto{\pgfqpoint{1.543923in}{3.303874in}}%
\pgfpathlineto{\pgfqpoint{1.613842in}{3.264975in}}%
\pgfpathlineto{\pgfqpoint{1.543923in}{3.306073in}}%
\pgfpathclose%
\pgfusepath{fill}%
\end{pgfscope}%
\begin{pgfscope}%
\pgfpathrectangle{\pgfqpoint{0.637500in}{0.550000in}}{\pgfqpoint{3.850000in}{3.850000in}}%
\pgfusepath{clip}%
\pgfsetbuttcap%
\pgfsetroundjoin%
\definecolor{currentfill}{rgb}{0.897487,0.000000,0.000000}%
\pgfsetfillcolor{currentfill}%
\pgfsetfillopacity{0.800000}%
\pgfsetlinewidth{0.000000pt}%
\definecolor{currentstroke}{rgb}{0.000000,0.000000,0.000000}%
\pgfsetstrokecolor{currentstroke}%
\pgfsetdash{}{0pt}%
\pgfpathmoveto{\pgfqpoint{2.662622in}{2.893946in}}%
\pgfpathlineto{\pgfqpoint{2.592703in}{2.852849in}}%
\pgfpathlineto{\pgfqpoint{2.662622in}{2.891748in}}%
\pgfpathlineto{\pgfqpoint{2.732540in}{2.932845in}}%
\pgfpathlineto{\pgfqpoint{2.662622in}{2.893946in}}%
\pgfpathclose%
\pgfusepath{fill}%
\end{pgfscope}%
\begin{pgfscope}%
\pgfpathrectangle{\pgfqpoint{0.637500in}{0.550000in}}{\pgfqpoint{3.850000in}{3.850000in}}%
\pgfusepath{clip}%
\pgfsetbuttcap%
\pgfsetroundjoin%
\definecolor{currentfill}{rgb}{0.897487,0.000000,0.000000}%
\pgfsetfillcolor{currentfill}%
\pgfsetfillopacity{0.800000}%
\pgfsetlinewidth{0.000000pt}%
\definecolor{currentstroke}{rgb}{0.000000,0.000000,0.000000}%
\pgfsetstrokecolor{currentstroke}%
\pgfsetdash{}{0pt}%
\pgfpathmoveto{\pgfqpoint{2.872378in}{3.015039in}}%
\pgfpathlineto{\pgfqpoint{2.802459in}{2.973942in}}%
\pgfpathlineto{\pgfqpoint{2.872378in}{3.012840in}}%
\pgfpathlineto{\pgfqpoint{2.942296in}{3.053938in}}%
\pgfpathlineto{\pgfqpoint{2.872378in}{3.015039in}}%
\pgfpathclose%
\pgfusepath{fill}%
\end{pgfscope}%
\begin{pgfscope}%
\pgfpathrectangle{\pgfqpoint{0.637500in}{0.550000in}}{\pgfqpoint{3.850000in}{3.850000in}}%
\pgfusepath{clip}%
\pgfsetbuttcap%
\pgfsetroundjoin%
\definecolor{currentfill}{rgb}{0.897487,0.000000,0.000000}%
\pgfsetfillcolor{currentfill}%
\pgfsetfillopacity{0.800000}%
\pgfsetlinewidth{0.000000pt}%
\definecolor{currentstroke}{rgb}{0.000000,0.000000,0.000000}%
\pgfsetstrokecolor{currentstroke}%
\pgfsetdash{}{0pt}%
\pgfpathmoveto{\pgfqpoint{3.082134in}{3.136132in}}%
\pgfpathlineto{\pgfqpoint{3.012215in}{3.095035in}}%
\pgfpathlineto{\pgfqpoint{3.082134in}{3.133933in}}%
\pgfpathlineto{\pgfqpoint{3.152052in}{3.175030in}}%
\pgfpathlineto{\pgfqpoint{3.082134in}{3.136132in}}%
\pgfpathclose%
\pgfusepath{fill}%
\end{pgfscope}%
\begin{pgfscope}%
\pgfpathrectangle{\pgfqpoint{0.637500in}{0.550000in}}{\pgfqpoint{3.850000in}{3.850000in}}%
\pgfusepath{clip}%
\pgfsetbuttcap%
\pgfsetroundjoin%
\definecolor{currentfill}{rgb}{0.897487,0.000000,0.000000}%
\pgfsetfillcolor{currentfill}%
\pgfsetfillopacity{0.800000}%
\pgfsetlinewidth{0.000000pt}%
\definecolor{currentstroke}{rgb}{0.000000,0.000000,0.000000}%
\pgfsetstrokecolor{currentstroke}%
\pgfsetdash{}{0pt}%
\pgfpathmoveto{\pgfqpoint{3.291890in}{3.257225in}}%
\pgfpathlineto{\pgfqpoint{3.221971in}{3.216127in}}%
\pgfpathlineto{\pgfqpoint{3.291890in}{3.255026in}}%
\pgfpathlineto{\pgfqpoint{3.361808in}{3.296123in}}%
\pgfpathlineto{\pgfqpoint{3.291890in}{3.257225in}}%
\pgfpathclose%
\pgfusepath{fill}%
\end{pgfscope}%
\begin{pgfscope}%
\pgfpathrectangle{\pgfqpoint{0.637500in}{0.550000in}}{\pgfqpoint{3.850000in}{3.850000in}}%
\pgfusepath{clip}%
\pgfsetbuttcap%
\pgfsetroundjoin%
\definecolor{currentfill}{rgb}{0.897487,0.000000,0.000000}%
\pgfsetfillcolor{currentfill}%
\pgfsetfillopacity{0.800000}%
\pgfsetlinewidth{0.000000pt}%
\definecolor{currentstroke}{rgb}{0.000000,0.000000,0.000000}%
\pgfsetstrokecolor{currentstroke}%
\pgfsetdash{}{0pt}%
\pgfpathmoveto{\pgfqpoint{2.243110in}{2.901697in}}%
\pgfpathlineto{\pgfqpoint{2.173191in}{2.940596in}}%
\pgfpathlineto{\pgfqpoint{2.243110in}{2.899499in}}%
\pgfpathlineto{\pgfqpoint{2.313028in}{2.860600in}}%
\pgfpathlineto{\pgfqpoint{2.243110in}{2.901697in}}%
\pgfpathclose%
\pgfusepath{fill}%
\end{pgfscope}%
\begin{pgfscope}%
\pgfpathrectangle{\pgfqpoint{0.637500in}{0.550000in}}{\pgfqpoint{3.850000in}{3.850000in}}%
\pgfusepath{clip}%
\pgfsetbuttcap%
\pgfsetroundjoin%
\definecolor{currentfill}{rgb}{0.897487,0.000000,0.000000}%
\pgfsetfillcolor{currentfill}%
\pgfsetfillopacity{0.800000}%
\pgfsetlinewidth{0.000000pt}%
\definecolor{currentstroke}{rgb}{0.000000,0.000000,0.000000}%
\pgfsetstrokecolor{currentstroke}%
\pgfsetdash{}{0pt}%
\pgfpathmoveto{\pgfqpoint{3.501646in}{3.378317in}}%
\pgfpathlineto{\pgfqpoint{3.431727in}{3.337220in}}%
\pgfpathlineto{\pgfqpoint{3.501646in}{3.376119in}}%
\pgfpathlineto{\pgfqpoint{3.571564in}{3.417216in}}%
\pgfpathlineto{\pgfqpoint{3.501646in}{3.378317in}}%
\pgfpathclose%
\pgfusepath{fill}%
\end{pgfscope}%
\begin{pgfscope}%
\pgfpathrectangle{\pgfqpoint{0.637500in}{0.550000in}}{\pgfqpoint{3.850000in}{3.850000in}}%
\pgfusepath{clip}%
\pgfsetbuttcap%
\pgfsetroundjoin%
\definecolor{currentfill}{rgb}{0.897487,0.000000,0.000000}%
\pgfsetfillcolor{currentfill}%
\pgfsetfillopacity{0.800000}%
\pgfsetlinewidth{0.000000pt}%
\definecolor{currentstroke}{rgb}{0.000000,0.000000,0.000000}%
\pgfsetstrokecolor{currentstroke}%
\pgfsetdash{}{0pt}%
\pgfpathmoveto{\pgfqpoint{2.033354in}{3.022790in}}%
\pgfpathlineto{\pgfqpoint{1.963435in}{3.061688in}}%
\pgfpathlineto{\pgfqpoint{2.033354in}{3.020591in}}%
\pgfpathlineto{\pgfqpoint{2.103272in}{2.981693in}}%
\pgfpathlineto{\pgfqpoint{2.033354in}{3.022790in}}%
\pgfpathclose%
\pgfusepath{fill}%
\end{pgfscope}%
\begin{pgfscope}%
\pgfpathrectangle{\pgfqpoint{0.637500in}{0.550000in}}{\pgfqpoint{3.850000in}{3.850000in}}%
\pgfusepath{clip}%
\pgfsetbuttcap%
\pgfsetroundjoin%
\definecolor{currentfill}{rgb}{0.897487,0.000000,0.000000}%
\pgfsetfillcolor{currentfill}%
\pgfsetfillopacity{0.800000}%
\pgfsetlinewidth{0.000000pt}%
\definecolor{currentstroke}{rgb}{0.000000,0.000000,0.000000}%
\pgfsetstrokecolor{currentstroke}%
\pgfsetdash{}{0pt}%
\pgfpathmoveto{\pgfqpoint{1.823598in}{3.143883in}}%
\pgfpathlineto{\pgfqpoint{1.753679in}{3.182781in}}%
\pgfpathlineto{\pgfqpoint{1.823598in}{3.141684in}}%
\pgfpathlineto{\pgfqpoint{1.893516in}{3.102786in}}%
\pgfpathlineto{\pgfqpoint{1.823598in}{3.143883in}}%
\pgfpathclose%
\pgfusepath{fill}%
\end{pgfscope}%
\begin{pgfscope}%
\pgfpathrectangle{\pgfqpoint{0.637500in}{0.550000in}}{\pgfqpoint{3.850000in}{3.850000in}}%
\pgfusepath{clip}%
\pgfsetbuttcap%
\pgfsetroundjoin%
\definecolor{currentfill}{rgb}{0.897487,0.000000,0.000000}%
\pgfsetfillcolor{currentfill}%
\pgfsetfillopacity{0.800000}%
\pgfsetlinewidth{0.000000pt}%
\definecolor{currentstroke}{rgb}{0.000000,0.000000,0.000000}%
\pgfsetstrokecolor{currentstroke}%
\pgfsetdash{}{0pt}%
\pgfpathmoveto{\pgfqpoint{1.613842in}{3.264975in}}%
\pgfpathlineto{\pgfqpoint{1.543923in}{3.303874in}}%
\pgfpathlineto{\pgfqpoint{1.613842in}{3.262777in}}%
\pgfpathlineto{\pgfqpoint{1.683761in}{3.223878in}}%
\pgfpathlineto{\pgfqpoint{1.613842in}{3.264975in}}%
\pgfpathclose%
\pgfusepath{fill}%
\end{pgfscope}%
\begin{pgfscope}%
\pgfpathrectangle{\pgfqpoint{0.637500in}{0.550000in}}{\pgfqpoint{3.850000in}{3.850000in}}%
\pgfusepath{clip}%
\pgfsetbuttcap%
\pgfsetroundjoin%
\definecolor{currentfill}{rgb}{0.898098,0.000000,0.000000}%
\pgfsetfillcolor{currentfill}%
\pgfsetfillopacity{0.800000}%
\pgfsetlinewidth{0.000000pt}%
\definecolor{currentstroke}{rgb}{0.000000,0.000000,0.000000}%
\pgfsetstrokecolor{currentstroke}%
\pgfsetdash{}{0pt}%
\pgfpathmoveto{\pgfqpoint{2.452866in}{2.781000in}}%
\pgfpathlineto{\pgfqpoint{2.382947in}{2.819503in}}%
\pgfpathlineto{\pgfqpoint{2.452866in}{2.778801in}}%
\pgfpathlineto{\pgfqpoint{2.522784in}{2.811752in}}%
\pgfpathlineto{\pgfqpoint{2.452866in}{2.781000in}}%
\pgfpathclose%
\pgfusepath{fill}%
\end{pgfscope}%
\begin{pgfscope}%
\pgfpathrectangle{\pgfqpoint{0.637500in}{0.550000in}}{\pgfqpoint{3.850000in}{3.850000in}}%
\pgfusepath{clip}%
\pgfsetbuttcap%
\pgfsetroundjoin%
\definecolor{currentfill}{rgb}{0.897487,0.000000,0.000000}%
\pgfsetfillcolor{currentfill}%
\pgfsetfillopacity{0.800000}%
\pgfsetlinewidth{0.000000pt}%
\definecolor{currentstroke}{rgb}{0.000000,0.000000,0.000000}%
\pgfsetstrokecolor{currentstroke}%
\pgfsetdash{}{0pt}%
\pgfpathmoveto{\pgfqpoint{2.592703in}{2.852849in}}%
\pgfpathlineto{\pgfqpoint{2.522784in}{2.811752in}}%
\pgfpathlineto{\pgfqpoint{2.592703in}{2.850651in}}%
\pgfpathlineto{\pgfqpoint{2.662622in}{2.891748in}}%
\pgfpathlineto{\pgfqpoint{2.592703in}{2.852849in}}%
\pgfpathclose%
\pgfusepath{fill}%
\end{pgfscope}%
\begin{pgfscope}%
\pgfpathrectangle{\pgfqpoint{0.637500in}{0.550000in}}{\pgfqpoint{3.850000in}{3.850000in}}%
\pgfusepath{clip}%
\pgfsetbuttcap%
\pgfsetroundjoin%
\definecolor{currentfill}{rgb}{0.897487,0.000000,0.000000}%
\pgfsetfillcolor{currentfill}%
\pgfsetfillopacity{0.800000}%
\pgfsetlinewidth{0.000000pt}%
\definecolor{currentstroke}{rgb}{0.000000,0.000000,0.000000}%
\pgfsetstrokecolor{currentstroke}%
\pgfsetdash{}{0pt}%
\pgfpathmoveto{\pgfqpoint{2.802459in}{2.973942in}}%
\pgfpathlineto{\pgfqpoint{2.732540in}{2.932845in}}%
\pgfpathlineto{\pgfqpoint{2.802459in}{2.971743in}}%
\pgfpathlineto{\pgfqpoint{2.872378in}{3.012840in}}%
\pgfpathlineto{\pgfqpoint{2.802459in}{2.973942in}}%
\pgfpathclose%
\pgfusepath{fill}%
\end{pgfscope}%
\begin{pgfscope}%
\pgfpathrectangle{\pgfqpoint{0.637500in}{0.550000in}}{\pgfqpoint{3.850000in}{3.850000in}}%
\pgfusepath{clip}%
\pgfsetbuttcap%
\pgfsetroundjoin%
\definecolor{currentfill}{rgb}{0.897487,0.000000,0.000000}%
\pgfsetfillcolor{currentfill}%
\pgfsetfillopacity{0.800000}%
\pgfsetlinewidth{0.000000pt}%
\definecolor{currentstroke}{rgb}{0.000000,0.000000,0.000000}%
\pgfsetstrokecolor{currentstroke}%
\pgfsetdash{}{0pt}%
\pgfpathmoveto{\pgfqpoint{3.012215in}{3.095035in}}%
\pgfpathlineto{\pgfqpoint{2.942296in}{3.053938in}}%
\pgfpathlineto{\pgfqpoint{3.012215in}{3.092836in}}%
\pgfpathlineto{\pgfqpoint{3.082134in}{3.133933in}}%
\pgfpathlineto{\pgfqpoint{3.012215in}{3.095035in}}%
\pgfpathclose%
\pgfusepath{fill}%
\end{pgfscope}%
\begin{pgfscope}%
\pgfpathrectangle{\pgfqpoint{0.637500in}{0.550000in}}{\pgfqpoint{3.850000in}{3.850000in}}%
\pgfusepath{clip}%
\pgfsetbuttcap%
\pgfsetroundjoin%
\definecolor{currentfill}{rgb}{0.897487,0.000000,0.000000}%
\pgfsetfillcolor{currentfill}%
\pgfsetfillopacity{0.800000}%
\pgfsetlinewidth{0.000000pt}%
\definecolor{currentstroke}{rgb}{0.000000,0.000000,0.000000}%
\pgfsetstrokecolor{currentstroke}%
\pgfsetdash{}{0pt}%
\pgfpathmoveto{\pgfqpoint{3.221971in}{3.216127in}}%
\pgfpathlineto{\pgfqpoint{3.152052in}{3.175030in}}%
\pgfpathlineto{\pgfqpoint{3.221971in}{3.213929in}}%
\pgfpathlineto{\pgfqpoint{3.291890in}{3.255026in}}%
\pgfpathlineto{\pgfqpoint{3.221971in}{3.216127in}}%
\pgfpathclose%
\pgfusepath{fill}%
\end{pgfscope}%
\begin{pgfscope}%
\pgfpathrectangle{\pgfqpoint{0.637500in}{0.550000in}}{\pgfqpoint{3.850000in}{3.850000in}}%
\pgfusepath{clip}%
\pgfsetbuttcap%
\pgfsetroundjoin%
\definecolor{currentfill}{rgb}{0.897487,0.000000,0.000000}%
\pgfsetfillcolor{currentfill}%
\pgfsetfillopacity{0.800000}%
\pgfsetlinewidth{0.000000pt}%
\definecolor{currentstroke}{rgb}{0.000000,0.000000,0.000000}%
\pgfsetstrokecolor{currentstroke}%
\pgfsetdash{}{0pt}%
\pgfpathmoveto{\pgfqpoint{2.313028in}{2.860600in}}%
\pgfpathlineto{\pgfqpoint{2.243110in}{2.899499in}}%
\pgfpathlineto{\pgfqpoint{2.313028in}{2.858401in}}%
\pgfpathlineto{\pgfqpoint{2.382947in}{2.819503in}}%
\pgfpathlineto{\pgfqpoint{2.313028in}{2.860600in}}%
\pgfpathclose%
\pgfusepath{fill}%
\end{pgfscope}%
\begin{pgfscope}%
\pgfpathrectangle{\pgfqpoint{0.637500in}{0.550000in}}{\pgfqpoint{3.850000in}{3.850000in}}%
\pgfusepath{clip}%
\pgfsetbuttcap%
\pgfsetroundjoin%
\definecolor{currentfill}{rgb}{0.897487,0.000000,0.000000}%
\pgfsetfillcolor{currentfill}%
\pgfsetfillopacity{0.800000}%
\pgfsetlinewidth{0.000000pt}%
\definecolor{currentstroke}{rgb}{0.000000,0.000000,0.000000}%
\pgfsetstrokecolor{currentstroke}%
\pgfsetdash{}{0pt}%
\pgfpathmoveto{\pgfqpoint{3.431727in}{3.337220in}}%
\pgfpathlineto{\pgfqpoint{3.361808in}{3.296123in}}%
\pgfpathlineto{\pgfqpoint{3.431727in}{3.335021in}}%
\pgfpathlineto{\pgfqpoint{3.501646in}{3.376119in}}%
\pgfpathlineto{\pgfqpoint{3.431727in}{3.337220in}}%
\pgfpathclose%
\pgfusepath{fill}%
\end{pgfscope}%
\begin{pgfscope}%
\pgfpathrectangle{\pgfqpoint{0.637500in}{0.550000in}}{\pgfqpoint{3.850000in}{3.850000in}}%
\pgfusepath{clip}%
\pgfsetbuttcap%
\pgfsetroundjoin%
\definecolor{currentfill}{rgb}{0.897487,0.000000,0.000000}%
\pgfsetfillcolor{currentfill}%
\pgfsetfillopacity{0.800000}%
\pgfsetlinewidth{0.000000pt}%
\definecolor{currentstroke}{rgb}{0.000000,0.000000,0.000000}%
\pgfsetstrokecolor{currentstroke}%
\pgfsetdash{}{0pt}%
\pgfpathmoveto{\pgfqpoint{2.103272in}{2.981693in}}%
\pgfpathlineto{\pgfqpoint{2.033354in}{3.020591in}}%
\pgfpathlineto{\pgfqpoint{2.103272in}{2.979494in}}%
\pgfpathlineto{\pgfqpoint{2.173191in}{2.940596in}}%
\pgfpathlineto{\pgfqpoint{2.103272in}{2.981693in}}%
\pgfpathclose%
\pgfusepath{fill}%
\end{pgfscope}%
\begin{pgfscope}%
\pgfpathrectangle{\pgfqpoint{0.637500in}{0.550000in}}{\pgfqpoint{3.850000in}{3.850000in}}%
\pgfusepath{clip}%
\pgfsetbuttcap%
\pgfsetroundjoin%
\definecolor{currentfill}{rgb}{0.897487,0.000000,0.000000}%
\pgfsetfillcolor{currentfill}%
\pgfsetfillopacity{0.800000}%
\pgfsetlinewidth{0.000000pt}%
\definecolor{currentstroke}{rgb}{0.000000,0.000000,0.000000}%
\pgfsetstrokecolor{currentstroke}%
\pgfsetdash{}{0pt}%
\pgfpathmoveto{\pgfqpoint{1.893516in}{3.102786in}}%
\pgfpathlineto{\pgfqpoint{1.823598in}{3.141684in}}%
\pgfpathlineto{\pgfqpoint{1.893516in}{3.100587in}}%
\pgfpathlineto{\pgfqpoint{1.963435in}{3.061688in}}%
\pgfpathlineto{\pgfqpoint{1.893516in}{3.102786in}}%
\pgfpathclose%
\pgfusepath{fill}%
\end{pgfscope}%
\begin{pgfscope}%
\pgfpathrectangle{\pgfqpoint{0.637500in}{0.550000in}}{\pgfqpoint{3.850000in}{3.850000in}}%
\pgfusepath{clip}%
\pgfsetbuttcap%
\pgfsetroundjoin%
\definecolor{currentfill}{rgb}{0.897487,0.000000,0.000000}%
\pgfsetfillcolor{currentfill}%
\pgfsetfillopacity{0.800000}%
\pgfsetlinewidth{0.000000pt}%
\definecolor{currentstroke}{rgb}{0.000000,0.000000,0.000000}%
\pgfsetstrokecolor{currentstroke}%
\pgfsetdash{}{0pt}%
\pgfpathmoveto{\pgfqpoint{1.683761in}{3.223878in}}%
\pgfpathlineto{\pgfqpoint{1.613842in}{3.262777in}}%
\pgfpathlineto{\pgfqpoint{1.683761in}{3.221680in}}%
\pgfpathlineto{\pgfqpoint{1.753679in}{3.182781in}}%
\pgfpathlineto{\pgfqpoint{1.683761in}{3.223878in}}%
\pgfpathclose%
\pgfusepath{fill}%
\end{pgfscope}%
\begin{pgfscope}%
\pgfpathrectangle{\pgfqpoint{0.637500in}{0.550000in}}{\pgfqpoint{3.850000in}{3.850000in}}%
\pgfusepath{clip}%
\pgfsetbuttcap%
\pgfsetroundjoin%
\definecolor{currentfill}{rgb}{0.897487,0.000000,0.000000}%
\pgfsetfillcolor{currentfill}%
\pgfsetfillopacity{0.800000}%
\pgfsetlinewidth{0.000000pt}%
\definecolor{currentstroke}{rgb}{0.000000,0.000000,0.000000}%
\pgfsetstrokecolor{currentstroke}%
\pgfsetdash{}{0pt}%
\pgfpathmoveto{\pgfqpoint{1.474005in}{3.344971in}}%
\pgfpathlineto{\pgfqpoint{1.404086in}{3.383869in}}%
\pgfpathlineto{\pgfqpoint{1.474005in}{3.342772in}}%
\pgfpathlineto{\pgfqpoint{1.543923in}{3.303874in}}%
\pgfpathlineto{\pgfqpoint{1.474005in}{3.344971in}}%
\pgfpathclose%
\pgfusepath{fill}%
\end{pgfscope}%
\begin{pgfscope}%
\pgfpathrectangle{\pgfqpoint{0.637500in}{0.550000in}}{\pgfqpoint{3.850000in}{3.850000in}}%
\pgfusepath{clip}%
\pgfsetbuttcap%
\pgfsetroundjoin%
\definecolor{currentfill}{rgb}{0.909646,0.000000,0.000000}%
\pgfsetfillcolor{currentfill}%
\pgfsetfillopacity{0.800000}%
\pgfsetlinewidth{0.000000pt}%
\definecolor{currentstroke}{rgb}{0.000000,0.000000,0.000000}%
\pgfsetstrokecolor{currentstroke}%
\pgfsetdash{}{0pt}%
\pgfpathmoveto{\pgfqpoint{2.522784in}{2.811752in}}%
\pgfpathlineto{\pgfqpoint{2.452866in}{2.778801in}}%
\pgfpathlineto{\pgfqpoint{2.522784in}{2.809553in}}%
\pgfpathlineto{\pgfqpoint{2.592703in}{2.850651in}}%
\pgfpathlineto{\pgfqpoint{2.522784in}{2.811752in}}%
\pgfpathclose%
\pgfusepath{fill}%
\end{pgfscope}%
\begin{pgfscope}%
\pgfpathrectangle{\pgfqpoint{0.637500in}{0.550000in}}{\pgfqpoint{3.850000in}{3.850000in}}%
\pgfusepath{clip}%
\pgfsetbuttcap%
\pgfsetroundjoin%
\definecolor{currentfill}{rgb}{0.897487,0.000000,0.000000}%
\pgfsetfillcolor{currentfill}%
\pgfsetfillopacity{0.800000}%
\pgfsetlinewidth{0.000000pt}%
\definecolor{currentstroke}{rgb}{0.000000,0.000000,0.000000}%
\pgfsetstrokecolor{currentstroke}%
\pgfsetdash{}{0pt}%
\pgfpathmoveto{\pgfqpoint{2.732540in}{2.932845in}}%
\pgfpathlineto{\pgfqpoint{2.662622in}{2.891748in}}%
\pgfpathlineto{\pgfqpoint{2.732540in}{2.930646in}}%
\pgfpathlineto{\pgfqpoint{2.802459in}{2.971743in}}%
\pgfpathlineto{\pgfqpoint{2.732540in}{2.932845in}}%
\pgfpathclose%
\pgfusepath{fill}%
\end{pgfscope}%
\begin{pgfscope}%
\pgfpathrectangle{\pgfqpoint{0.637500in}{0.550000in}}{\pgfqpoint{3.850000in}{3.850000in}}%
\pgfusepath{clip}%
\pgfsetbuttcap%
\pgfsetroundjoin%
\definecolor{currentfill}{rgb}{0.897487,0.000000,0.000000}%
\pgfsetfillcolor{currentfill}%
\pgfsetfillopacity{0.800000}%
\pgfsetlinewidth{0.000000pt}%
\definecolor{currentstroke}{rgb}{0.000000,0.000000,0.000000}%
\pgfsetstrokecolor{currentstroke}%
\pgfsetdash{}{0pt}%
\pgfpathmoveto{\pgfqpoint{2.942296in}{3.053938in}}%
\pgfpathlineto{\pgfqpoint{2.872378in}{3.012840in}}%
\pgfpathlineto{\pgfqpoint{2.942296in}{3.051739in}}%
\pgfpathlineto{\pgfqpoint{3.012215in}{3.092836in}}%
\pgfpathlineto{\pgfqpoint{2.942296in}{3.053938in}}%
\pgfpathclose%
\pgfusepath{fill}%
\end{pgfscope}%
\begin{pgfscope}%
\pgfpathrectangle{\pgfqpoint{0.637500in}{0.550000in}}{\pgfqpoint{3.850000in}{3.850000in}}%
\pgfusepath{clip}%
\pgfsetbuttcap%
\pgfsetroundjoin%
\definecolor{currentfill}{rgb}{0.897487,0.000000,0.000000}%
\pgfsetfillcolor{currentfill}%
\pgfsetfillopacity{0.800000}%
\pgfsetlinewidth{0.000000pt}%
\definecolor{currentstroke}{rgb}{0.000000,0.000000,0.000000}%
\pgfsetstrokecolor{currentstroke}%
\pgfsetdash{}{0pt}%
\pgfpathmoveto{\pgfqpoint{3.152052in}{3.175030in}}%
\pgfpathlineto{\pgfqpoint{3.082134in}{3.133933in}}%
\pgfpathlineto{\pgfqpoint{3.152052in}{3.172831in}}%
\pgfpathlineto{\pgfqpoint{3.221971in}{3.213929in}}%
\pgfpathlineto{\pgfqpoint{3.152052in}{3.175030in}}%
\pgfpathclose%
\pgfusepath{fill}%
\end{pgfscope}%
\begin{pgfscope}%
\pgfpathrectangle{\pgfqpoint{0.637500in}{0.550000in}}{\pgfqpoint{3.850000in}{3.850000in}}%
\pgfusepath{clip}%
\pgfsetbuttcap%
\pgfsetroundjoin%
\definecolor{currentfill}{rgb}{0.897487,0.000000,0.000000}%
\pgfsetfillcolor{currentfill}%
\pgfsetfillopacity{0.800000}%
\pgfsetlinewidth{0.000000pt}%
\definecolor{currentstroke}{rgb}{0.000000,0.000000,0.000000}%
\pgfsetstrokecolor{currentstroke}%
\pgfsetdash{}{0pt}%
\pgfpathmoveto{\pgfqpoint{2.382947in}{2.819503in}}%
\pgfpathlineto{\pgfqpoint{2.313028in}{2.858401in}}%
\pgfpathlineto{\pgfqpoint{2.382947in}{2.817304in}}%
\pgfpathlineto{\pgfqpoint{2.452866in}{2.778801in}}%
\pgfpathlineto{\pgfqpoint{2.382947in}{2.819503in}}%
\pgfpathclose%
\pgfusepath{fill}%
\end{pgfscope}%
\begin{pgfscope}%
\pgfpathrectangle{\pgfqpoint{0.637500in}{0.550000in}}{\pgfqpoint{3.850000in}{3.850000in}}%
\pgfusepath{clip}%
\pgfsetbuttcap%
\pgfsetroundjoin%
\definecolor{currentfill}{rgb}{0.897487,0.000000,0.000000}%
\pgfsetfillcolor{currentfill}%
\pgfsetfillopacity{0.800000}%
\pgfsetlinewidth{0.000000pt}%
\definecolor{currentstroke}{rgb}{0.000000,0.000000,0.000000}%
\pgfsetstrokecolor{currentstroke}%
\pgfsetdash{}{0pt}%
\pgfpathmoveto{\pgfqpoint{3.361808in}{3.296123in}}%
\pgfpathlineto{\pgfqpoint{3.291890in}{3.255026in}}%
\pgfpathlineto{\pgfqpoint{3.361808in}{3.293924in}}%
\pgfpathlineto{\pgfqpoint{3.431727in}{3.335021in}}%
\pgfpathlineto{\pgfqpoint{3.361808in}{3.296123in}}%
\pgfpathclose%
\pgfusepath{fill}%
\end{pgfscope}%
\begin{pgfscope}%
\pgfpathrectangle{\pgfqpoint{0.637500in}{0.550000in}}{\pgfqpoint{3.850000in}{3.850000in}}%
\pgfusepath{clip}%
\pgfsetbuttcap%
\pgfsetroundjoin%
\definecolor{currentfill}{rgb}{0.897487,0.000000,0.000000}%
\pgfsetfillcolor{currentfill}%
\pgfsetfillopacity{0.800000}%
\pgfsetlinewidth{0.000000pt}%
\definecolor{currentstroke}{rgb}{0.000000,0.000000,0.000000}%
\pgfsetstrokecolor{currentstroke}%
\pgfsetdash{}{0pt}%
\pgfpathmoveto{\pgfqpoint{2.173191in}{2.940596in}}%
\pgfpathlineto{\pgfqpoint{2.103272in}{2.979494in}}%
\pgfpathlineto{\pgfqpoint{2.173191in}{2.938397in}}%
\pgfpathlineto{\pgfqpoint{2.243110in}{2.899499in}}%
\pgfpathlineto{\pgfqpoint{2.173191in}{2.940596in}}%
\pgfpathclose%
\pgfusepath{fill}%
\end{pgfscope}%
\begin{pgfscope}%
\pgfpathrectangle{\pgfqpoint{0.637500in}{0.550000in}}{\pgfqpoint{3.850000in}{3.850000in}}%
\pgfusepath{clip}%
\pgfsetbuttcap%
\pgfsetroundjoin%
\definecolor{currentfill}{rgb}{0.897487,0.000000,0.000000}%
\pgfsetfillcolor{currentfill}%
\pgfsetfillopacity{0.800000}%
\pgfsetlinewidth{0.000000pt}%
\definecolor{currentstroke}{rgb}{0.000000,0.000000,0.000000}%
\pgfsetstrokecolor{currentstroke}%
\pgfsetdash{}{0pt}%
\pgfpathmoveto{\pgfqpoint{3.571564in}{3.417216in}}%
\pgfpathlineto{\pgfqpoint{3.501646in}{3.376119in}}%
\pgfpathlineto{\pgfqpoint{3.571564in}{3.415017in}}%
\pgfpathlineto{\pgfqpoint{3.641483in}{3.456114in}}%
\pgfpathlineto{\pgfqpoint{3.571564in}{3.417216in}}%
\pgfpathclose%
\pgfusepath{fill}%
\end{pgfscope}%
\begin{pgfscope}%
\pgfpathrectangle{\pgfqpoint{0.637500in}{0.550000in}}{\pgfqpoint{3.850000in}{3.850000in}}%
\pgfusepath{clip}%
\pgfsetbuttcap%
\pgfsetroundjoin%
\definecolor{currentfill}{rgb}{0.897487,0.000000,0.000000}%
\pgfsetfillcolor{currentfill}%
\pgfsetfillopacity{0.800000}%
\pgfsetlinewidth{0.000000pt}%
\definecolor{currentstroke}{rgb}{0.000000,0.000000,0.000000}%
\pgfsetstrokecolor{currentstroke}%
\pgfsetdash{}{0pt}%
\pgfpathmoveto{\pgfqpoint{1.963435in}{3.061688in}}%
\pgfpathlineto{\pgfqpoint{1.893516in}{3.100587in}}%
\pgfpathlineto{\pgfqpoint{1.963435in}{3.059490in}}%
\pgfpathlineto{\pgfqpoint{2.033354in}{3.020591in}}%
\pgfpathlineto{\pgfqpoint{1.963435in}{3.061688in}}%
\pgfpathclose%
\pgfusepath{fill}%
\end{pgfscope}%
\begin{pgfscope}%
\pgfpathrectangle{\pgfqpoint{0.637500in}{0.550000in}}{\pgfqpoint{3.850000in}{3.850000in}}%
\pgfusepath{clip}%
\pgfsetbuttcap%
\pgfsetroundjoin%
\definecolor{currentfill}{rgb}{0.897487,0.000000,0.000000}%
\pgfsetfillcolor{currentfill}%
\pgfsetfillopacity{0.800000}%
\pgfsetlinewidth{0.000000pt}%
\definecolor{currentstroke}{rgb}{0.000000,0.000000,0.000000}%
\pgfsetstrokecolor{currentstroke}%
\pgfsetdash{}{0pt}%
\pgfpathmoveto{\pgfqpoint{1.753679in}{3.182781in}}%
\pgfpathlineto{\pgfqpoint{1.683761in}{3.221680in}}%
\pgfpathlineto{\pgfqpoint{1.753679in}{3.180582in}}%
\pgfpathlineto{\pgfqpoint{1.823598in}{3.141684in}}%
\pgfpathlineto{\pgfqpoint{1.753679in}{3.182781in}}%
\pgfpathclose%
\pgfusepath{fill}%
\end{pgfscope}%
\begin{pgfscope}%
\pgfpathrectangle{\pgfqpoint{0.637500in}{0.550000in}}{\pgfqpoint{3.850000in}{3.850000in}}%
\pgfusepath{clip}%
\pgfsetbuttcap%
\pgfsetroundjoin%
\definecolor{currentfill}{rgb}{0.897487,0.000000,0.000000}%
\pgfsetfillcolor{currentfill}%
\pgfsetfillopacity{0.800000}%
\pgfsetlinewidth{0.000000pt}%
\definecolor{currentstroke}{rgb}{0.000000,0.000000,0.000000}%
\pgfsetstrokecolor{currentstroke}%
\pgfsetdash{}{0pt}%
\pgfpathmoveto{\pgfqpoint{1.543923in}{3.303874in}}%
\pgfpathlineto{\pgfqpoint{1.474005in}{3.342772in}}%
\pgfpathlineto{\pgfqpoint{1.543923in}{3.301675in}}%
\pgfpathlineto{\pgfqpoint{1.613842in}{3.262777in}}%
\pgfpathlineto{\pgfqpoint{1.543923in}{3.303874in}}%
\pgfpathclose%
\pgfusepath{fill}%
\end{pgfscope}%
\begin{pgfscope}%
\pgfpathrectangle{\pgfqpoint{0.637500in}{0.550000in}}{\pgfqpoint{3.850000in}{3.850000in}}%
\pgfusepath{clip}%
\pgfsetbuttcap%
\pgfsetroundjoin%
\definecolor{currentfill}{rgb}{0.897487,0.000000,0.000000}%
\pgfsetfillcolor{currentfill}%
\pgfsetfillopacity{0.800000}%
\pgfsetlinewidth{0.000000pt}%
\definecolor{currentstroke}{rgb}{0.000000,0.000000,0.000000}%
\pgfsetstrokecolor{currentstroke}%
\pgfsetdash{}{0pt}%
\pgfpathmoveto{\pgfqpoint{2.662622in}{2.891748in}}%
\pgfpathlineto{\pgfqpoint{2.592703in}{2.850651in}}%
\pgfpathlineto{\pgfqpoint{2.662622in}{2.889549in}}%
\pgfpathlineto{\pgfqpoint{2.732540in}{2.930646in}}%
\pgfpathlineto{\pgfqpoint{2.662622in}{2.891748in}}%
\pgfpathclose%
\pgfusepath{fill}%
\end{pgfscope}%
\begin{pgfscope}%
\pgfpathrectangle{\pgfqpoint{0.637500in}{0.550000in}}{\pgfqpoint{3.850000in}{3.850000in}}%
\pgfusepath{clip}%
\pgfsetbuttcap%
\pgfsetroundjoin%
\definecolor{currentfill}{rgb}{0.897487,0.000000,0.000000}%
\pgfsetfillcolor{currentfill}%
\pgfsetfillopacity{0.800000}%
\pgfsetlinewidth{0.000000pt}%
\definecolor{currentstroke}{rgb}{0.000000,0.000000,0.000000}%
\pgfsetstrokecolor{currentstroke}%
\pgfsetdash{}{0pt}%
\pgfpathmoveto{\pgfqpoint{2.872378in}{3.012840in}}%
\pgfpathlineto{\pgfqpoint{2.802459in}{2.971743in}}%
\pgfpathlineto{\pgfqpoint{2.872378in}{3.010642in}}%
\pgfpathlineto{\pgfqpoint{2.942296in}{3.051739in}}%
\pgfpathlineto{\pgfqpoint{2.872378in}{3.012840in}}%
\pgfpathclose%
\pgfusepath{fill}%
\end{pgfscope}%
\begin{pgfscope}%
\pgfpathrectangle{\pgfqpoint{0.637500in}{0.550000in}}{\pgfqpoint{3.850000in}{3.850000in}}%
\pgfusepath{clip}%
\pgfsetbuttcap%
\pgfsetroundjoin%
\definecolor{currentfill}{rgb}{0.897487,0.000000,0.000000}%
\pgfsetfillcolor{currentfill}%
\pgfsetfillopacity{0.800000}%
\pgfsetlinewidth{0.000000pt}%
\definecolor{currentstroke}{rgb}{0.000000,0.000000,0.000000}%
\pgfsetstrokecolor{currentstroke}%
\pgfsetdash{}{0pt}%
\pgfpathmoveto{\pgfqpoint{3.082134in}{3.133933in}}%
\pgfpathlineto{\pgfqpoint{3.012215in}{3.092836in}}%
\pgfpathlineto{\pgfqpoint{3.082134in}{3.131734in}}%
\pgfpathlineto{\pgfqpoint{3.152052in}{3.172831in}}%
\pgfpathlineto{\pgfqpoint{3.082134in}{3.133933in}}%
\pgfpathclose%
\pgfusepath{fill}%
\end{pgfscope}%
\begin{pgfscope}%
\pgfpathrectangle{\pgfqpoint{0.637500in}{0.550000in}}{\pgfqpoint{3.850000in}{3.850000in}}%
\pgfusepath{clip}%
\pgfsetbuttcap%
\pgfsetroundjoin%
\definecolor{currentfill}{rgb}{0.897487,0.000000,0.000000}%
\pgfsetfillcolor{currentfill}%
\pgfsetfillopacity{0.800000}%
\pgfsetlinewidth{0.000000pt}%
\definecolor{currentstroke}{rgb}{0.000000,0.000000,0.000000}%
\pgfsetstrokecolor{currentstroke}%
\pgfsetdash{}{0pt}%
\pgfpathmoveto{\pgfqpoint{3.291890in}{3.255026in}}%
\pgfpathlineto{\pgfqpoint{3.221971in}{3.213929in}}%
\pgfpathlineto{\pgfqpoint{3.291890in}{3.252827in}}%
\pgfpathlineto{\pgfqpoint{3.361808in}{3.293924in}}%
\pgfpathlineto{\pgfqpoint{3.291890in}{3.255026in}}%
\pgfpathclose%
\pgfusepath{fill}%
\end{pgfscope}%
\begin{pgfscope}%
\pgfpathrectangle{\pgfqpoint{0.637500in}{0.550000in}}{\pgfqpoint{3.850000in}{3.850000in}}%
\pgfusepath{clip}%
\pgfsetbuttcap%
\pgfsetroundjoin%
\definecolor{currentfill}{rgb}{0.897487,0.000000,0.000000}%
\pgfsetfillcolor{currentfill}%
\pgfsetfillopacity{0.800000}%
\pgfsetlinewidth{0.000000pt}%
\definecolor{currentstroke}{rgb}{0.000000,0.000000,0.000000}%
\pgfsetstrokecolor{currentstroke}%
\pgfsetdash{}{0pt}%
\pgfpathmoveto{\pgfqpoint{2.243110in}{2.899499in}}%
\pgfpathlineto{\pgfqpoint{2.173191in}{2.938397in}}%
\pgfpathlineto{\pgfqpoint{2.243110in}{2.897300in}}%
\pgfpathlineto{\pgfqpoint{2.313028in}{2.858401in}}%
\pgfpathlineto{\pgfqpoint{2.243110in}{2.899499in}}%
\pgfpathclose%
\pgfusepath{fill}%
\end{pgfscope}%
\begin{pgfscope}%
\pgfpathrectangle{\pgfqpoint{0.637500in}{0.550000in}}{\pgfqpoint{3.850000in}{3.850000in}}%
\pgfusepath{clip}%
\pgfsetbuttcap%
\pgfsetroundjoin%
\definecolor{currentfill}{rgb}{0.897487,0.000000,0.000000}%
\pgfsetfillcolor{currentfill}%
\pgfsetfillopacity{0.800000}%
\pgfsetlinewidth{0.000000pt}%
\definecolor{currentstroke}{rgb}{0.000000,0.000000,0.000000}%
\pgfsetstrokecolor{currentstroke}%
\pgfsetdash{}{0pt}%
\pgfpathmoveto{\pgfqpoint{3.501646in}{3.376119in}}%
\pgfpathlineto{\pgfqpoint{3.431727in}{3.335021in}}%
\pgfpathlineto{\pgfqpoint{3.501646in}{3.373920in}}%
\pgfpathlineto{\pgfqpoint{3.571564in}{3.415017in}}%
\pgfpathlineto{\pgfqpoint{3.501646in}{3.376119in}}%
\pgfpathclose%
\pgfusepath{fill}%
\end{pgfscope}%
\begin{pgfscope}%
\pgfpathrectangle{\pgfqpoint{0.637500in}{0.550000in}}{\pgfqpoint{3.850000in}{3.850000in}}%
\pgfusepath{clip}%
\pgfsetbuttcap%
\pgfsetroundjoin%
\definecolor{currentfill}{rgb}{0.897487,0.000000,0.000000}%
\pgfsetfillcolor{currentfill}%
\pgfsetfillopacity{0.800000}%
\pgfsetlinewidth{0.000000pt}%
\definecolor{currentstroke}{rgb}{0.000000,0.000000,0.000000}%
\pgfsetstrokecolor{currentstroke}%
\pgfsetdash{}{0pt}%
\pgfpathmoveto{\pgfqpoint{2.033354in}{3.020591in}}%
\pgfpathlineto{\pgfqpoint{1.963435in}{3.059490in}}%
\pgfpathlineto{\pgfqpoint{2.033354in}{3.018393in}}%
\pgfpathlineto{\pgfqpoint{2.103272in}{2.979494in}}%
\pgfpathlineto{\pgfqpoint{2.033354in}{3.020591in}}%
\pgfpathclose%
\pgfusepath{fill}%
\end{pgfscope}%
\begin{pgfscope}%
\pgfpathrectangle{\pgfqpoint{0.637500in}{0.550000in}}{\pgfqpoint{3.850000in}{3.850000in}}%
\pgfusepath{clip}%
\pgfsetbuttcap%
\pgfsetroundjoin%
\definecolor{currentfill}{rgb}{0.897487,0.000000,0.000000}%
\pgfsetfillcolor{currentfill}%
\pgfsetfillopacity{0.800000}%
\pgfsetlinewidth{0.000000pt}%
\definecolor{currentstroke}{rgb}{0.000000,0.000000,0.000000}%
\pgfsetstrokecolor{currentstroke}%
\pgfsetdash{}{0pt}%
\pgfpathmoveto{\pgfqpoint{1.823598in}{3.141684in}}%
\pgfpathlineto{\pgfqpoint{1.753679in}{3.180582in}}%
\pgfpathlineto{\pgfqpoint{1.823598in}{3.139485in}}%
\pgfpathlineto{\pgfqpoint{1.893516in}{3.100587in}}%
\pgfpathlineto{\pgfqpoint{1.823598in}{3.141684in}}%
\pgfpathclose%
\pgfusepath{fill}%
\end{pgfscope}%
\begin{pgfscope}%
\pgfpathrectangle{\pgfqpoint{0.637500in}{0.550000in}}{\pgfqpoint{3.850000in}{3.850000in}}%
\pgfusepath{clip}%
\pgfsetbuttcap%
\pgfsetroundjoin%
\definecolor{currentfill}{rgb}{0.897487,0.000000,0.000000}%
\pgfsetfillcolor{currentfill}%
\pgfsetfillopacity{0.800000}%
\pgfsetlinewidth{0.000000pt}%
\definecolor{currentstroke}{rgb}{0.000000,0.000000,0.000000}%
\pgfsetstrokecolor{currentstroke}%
\pgfsetdash{}{0pt}%
\pgfpathmoveto{\pgfqpoint{1.613842in}{3.262777in}}%
\pgfpathlineto{\pgfqpoint{1.543923in}{3.301675in}}%
\pgfpathlineto{\pgfqpoint{1.613842in}{3.260578in}}%
\pgfpathlineto{\pgfqpoint{1.683761in}{3.221680in}}%
\pgfpathlineto{\pgfqpoint{1.613842in}{3.262777in}}%
\pgfpathclose%
\pgfusepath{fill}%
\end{pgfscope}%
\begin{pgfscope}%
\pgfpathrectangle{\pgfqpoint{0.637500in}{0.550000in}}{\pgfqpoint{3.850000in}{3.850000in}}%
\pgfusepath{clip}%
\pgfsetbuttcap%
\pgfsetroundjoin%
\definecolor{currentfill}{rgb}{0.897487,0.000000,0.000000}%
\pgfsetfillcolor{currentfill}%
\pgfsetfillopacity{0.800000}%
\pgfsetlinewidth{0.000000pt}%
\definecolor{currentstroke}{rgb}{0.000000,0.000000,0.000000}%
\pgfsetstrokecolor{currentstroke}%
\pgfsetdash{}{0pt}%
\pgfpathmoveto{\pgfqpoint{1.404086in}{3.383869in}}%
\pgfpathlineto{\pgfqpoint{1.334167in}{3.422768in}}%
\pgfpathlineto{\pgfqpoint{1.404086in}{3.381671in}}%
\pgfpathlineto{\pgfqpoint{1.474005in}{3.342772in}}%
\pgfpathlineto{\pgfqpoint{1.404086in}{3.383869in}}%
\pgfpathclose%
\pgfusepath{fill}%
\end{pgfscope}%
\begin{pgfscope}%
\pgfpathrectangle{\pgfqpoint{0.637500in}{0.550000in}}{\pgfqpoint{3.850000in}{3.850000in}}%
\pgfusepath{clip}%
\pgfsetbuttcap%
\pgfsetroundjoin%
\definecolor{currentfill}{rgb}{0.898098,0.000000,0.000000}%
\pgfsetfillcolor{currentfill}%
\pgfsetfillopacity{0.800000}%
\pgfsetlinewidth{0.000000pt}%
\definecolor{currentstroke}{rgb}{0.000000,0.000000,0.000000}%
\pgfsetstrokecolor{currentstroke}%
\pgfsetdash{}{0pt}%
\pgfpathmoveto{\pgfqpoint{2.452866in}{2.778801in}}%
\pgfpathlineto{\pgfqpoint{2.382947in}{2.817304in}}%
\pgfpathlineto{\pgfqpoint{2.452866in}{2.776603in}}%
\pgfpathlineto{\pgfqpoint{2.522784in}{2.809553in}}%
\pgfpathlineto{\pgfqpoint{2.452866in}{2.778801in}}%
\pgfpathclose%
\pgfusepath{fill}%
\end{pgfscope}%
\begin{pgfscope}%
\pgfpathrectangle{\pgfqpoint{0.637500in}{0.550000in}}{\pgfqpoint{3.850000in}{3.850000in}}%
\pgfusepath{clip}%
\pgfsetbuttcap%
\pgfsetroundjoin%
\definecolor{currentfill}{rgb}{0.897487,0.000000,0.000000}%
\pgfsetfillcolor{currentfill}%
\pgfsetfillopacity{0.800000}%
\pgfsetlinewidth{0.000000pt}%
\definecolor{currentstroke}{rgb}{0.000000,0.000000,0.000000}%
\pgfsetstrokecolor{currentstroke}%
\pgfsetdash{}{0pt}%
\pgfpathmoveto{\pgfqpoint{2.592703in}{2.850651in}}%
\pgfpathlineto{\pgfqpoint{2.522784in}{2.809553in}}%
\pgfpathlineto{\pgfqpoint{2.592703in}{2.848452in}}%
\pgfpathlineto{\pgfqpoint{2.662622in}{2.889549in}}%
\pgfpathlineto{\pgfqpoint{2.592703in}{2.850651in}}%
\pgfpathclose%
\pgfusepath{fill}%
\end{pgfscope}%
\begin{pgfscope}%
\pgfpathrectangle{\pgfqpoint{0.637500in}{0.550000in}}{\pgfqpoint{3.850000in}{3.850000in}}%
\pgfusepath{clip}%
\pgfsetbuttcap%
\pgfsetroundjoin%
\definecolor{currentfill}{rgb}{0.897487,0.000000,0.000000}%
\pgfsetfillcolor{currentfill}%
\pgfsetfillopacity{0.800000}%
\pgfsetlinewidth{0.000000pt}%
\definecolor{currentstroke}{rgb}{0.000000,0.000000,0.000000}%
\pgfsetstrokecolor{currentstroke}%
\pgfsetdash{}{0pt}%
\pgfpathmoveto{\pgfqpoint{2.802459in}{2.971743in}}%
\pgfpathlineto{\pgfqpoint{2.732540in}{2.930646in}}%
\pgfpathlineto{\pgfqpoint{2.802459in}{2.969544in}}%
\pgfpathlineto{\pgfqpoint{2.872378in}{3.010642in}}%
\pgfpathlineto{\pgfqpoint{2.802459in}{2.971743in}}%
\pgfpathclose%
\pgfusepath{fill}%
\end{pgfscope}%
\begin{pgfscope}%
\pgfpathrectangle{\pgfqpoint{0.637500in}{0.550000in}}{\pgfqpoint{3.850000in}{3.850000in}}%
\pgfusepath{clip}%
\pgfsetbuttcap%
\pgfsetroundjoin%
\definecolor{currentfill}{rgb}{0.897487,0.000000,0.000000}%
\pgfsetfillcolor{currentfill}%
\pgfsetfillopacity{0.800000}%
\pgfsetlinewidth{0.000000pt}%
\definecolor{currentstroke}{rgb}{0.000000,0.000000,0.000000}%
\pgfsetstrokecolor{currentstroke}%
\pgfsetdash{}{0pt}%
\pgfpathmoveto{\pgfqpoint{3.012215in}{3.092836in}}%
\pgfpathlineto{\pgfqpoint{2.942296in}{3.051739in}}%
\pgfpathlineto{\pgfqpoint{3.012215in}{3.090637in}}%
\pgfpathlineto{\pgfqpoint{3.082134in}{3.131734in}}%
\pgfpathlineto{\pgfqpoint{3.012215in}{3.092836in}}%
\pgfpathclose%
\pgfusepath{fill}%
\end{pgfscope}%
\begin{pgfscope}%
\pgfpathrectangle{\pgfqpoint{0.637500in}{0.550000in}}{\pgfqpoint{3.850000in}{3.850000in}}%
\pgfusepath{clip}%
\pgfsetbuttcap%
\pgfsetroundjoin%
\definecolor{currentfill}{rgb}{0.897487,0.000000,0.000000}%
\pgfsetfillcolor{currentfill}%
\pgfsetfillopacity{0.800000}%
\pgfsetlinewidth{0.000000pt}%
\definecolor{currentstroke}{rgb}{0.000000,0.000000,0.000000}%
\pgfsetstrokecolor{currentstroke}%
\pgfsetdash{}{0pt}%
\pgfpathmoveto{\pgfqpoint{3.221971in}{3.213929in}}%
\pgfpathlineto{\pgfqpoint{3.152052in}{3.172831in}}%
\pgfpathlineto{\pgfqpoint{3.221971in}{3.211730in}}%
\pgfpathlineto{\pgfqpoint{3.291890in}{3.252827in}}%
\pgfpathlineto{\pgfqpoint{3.221971in}{3.213929in}}%
\pgfpathclose%
\pgfusepath{fill}%
\end{pgfscope}%
\begin{pgfscope}%
\pgfpathrectangle{\pgfqpoint{0.637500in}{0.550000in}}{\pgfqpoint{3.850000in}{3.850000in}}%
\pgfusepath{clip}%
\pgfsetbuttcap%
\pgfsetroundjoin%
\definecolor{currentfill}{rgb}{0.897487,0.000000,0.000000}%
\pgfsetfillcolor{currentfill}%
\pgfsetfillopacity{0.800000}%
\pgfsetlinewidth{0.000000pt}%
\definecolor{currentstroke}{rgb}{0.000000,0.000000,0.000000}%
\pgfsetstrokecolor{currentstroke}%
\pgfsetdash{}{0pt}%
\pgfpathmoveto{\pgfqpoint{2.313028in}{2.858401in}}%
\pgfpathlineto{\pgfqpoint{2.243110in}{2.897300in}}%
\pgfpathlineto{\pgfqpoint{2.313028in}{2.856203in}}%
\pgfpathlineto{\pgfqpoint{2.382947in}{2.817304in}}%
\pgfpathlineto{\pgfqpoint{2.313028in}{2.858401in}}%
\pgfpathclose%
\pgfusepath{fill}%
\end{pgfscope}%
\begin{pgfscope}%
\pgfpathrectangle{\pgfqpoint{0.637500in}{0.550000in}}{\pgfqpoint{3.850000in}{3.850000in}}%
\pgfusepath{clip}%
\pgfsetbuttcap%
\pgfsetroundjoin%
\definecolor{currentfill}{rgb}{0.897487,0.000000,0.000000}%
\pgfsetfillcolor{currentfill}%
\pgfsetfillopacity{0.800000}%
\pgfsetlinewidth{0.000000pt}%
\definecolor{currentstroke}{rgb}{0.000000,0.000000,0.000000}%
\pgfsetstrokecolor{currentstroke}%
\pgfsetdash{}{0pt}%
\pgfpathmoveto{\pgfqpoint{3.431727in}{3.335021in}}%
\pgfpathlineto{\pgfqpoint{3.361808in}{3.293924in}}%
\pgfpathlineto{\pgfqpoint{3.431727in}{3.332823in}}%
\pgfpathlineto{\pgfqpoint{3.501646in}{3.373920in}}%
\pgfpathlineto{\pgfqpoint{3.431727in}{3.335021in}}%
\pgfpathclose%
\pgfusepath{fill}%
\end{pgfscope}%
\begin{pgfscope}%
\pgfpathrectangle{\pgfqpoint{0.637500in}{0.550000in}}{\pgfqpoint{3.850000in}{3.850000in}}%
\pgfusepath{clip}%
\pgfsetbuttcap%
\pgfsetroundjoin%
\definecolor{currentfill}{rgb}{0.897487,0.000000,0.000000}%
\pgfsetfillcolor{currentfill}%
\pgfsetfillopacity{0.800000}%
\pgfsetlinewidth{0.000000pt}%
\definecolor{currentstroke}{rgb}{0.000000,0.000000,0.000000}%
\pgfsetstrokecolor{currentstroke}%
\pgfsetdash{}{0pt}%
\pgfpathmoveto{\pgfqpoint{2.103272in}{2.979494in}}%
\pgfpathlineto{\pgfqpoint{2.033354in}{3.018393in}}%
\pgfpathlineto{\pgfqpoint{2.103272in}{2.977295in}}%
\pgfpathlineto{\pgfqpoint{2.173191in}{2.938397in}}%
\pgfpathlineto{\pgfqpoint{2.103272in}{2.979494in}}%
\pgfpathclose%
\pgfusepath{fill}%
\end{pgfscope}%
\begin{pgfscope}%
\pgfpathrectangle{\pgfqpoint{0.637500in}{0.550000in}}{\pgfqpoint{3.850000in}{3.850000in}}%
\pgfusepath{clip}%
\pgfsetbuttcap%
\pgfsetroundjoin%
\definecolor{currentfill}{rgb}{0.897487,0.000000,0.000000}%
\pgfsetfillcolor{currentfill}%
\pgfsetfillopacity{0.800000}%
\pgfsetlinewidth{0.000000pt}%
\definecolor{currentstroke}{rgb}{0.000000,0.000000,0.000000}%
\pgfsetstrokecolor{currentstroke}%
\pgfsetdash{}{0pt}%
\pgfpathmoveto{\pgfqpoint{3.641483in}{3.456114in}}%
\pgfpathlineto{\pgfqpoint{3.571564in}{3.415017in}}%
\pgfpathlineto{\pgfqpoint{3.641483in}{3.453915in}}%
\pgfpathlineto{\pgfqpoint{3.711402in}{3.495012in}}%
\pgfpathlineto{\pgfqpoint{3.641483in}{3.456114in}}%
\pgfpathclose%
\pgfusepath{fill}%
\end{pgfscope}%
\begin{pgfscope}%
\pgfpathrectangle{\pgfqpoint{0.637500in}{0.550000in}}{\pgfqpoint{3.850000in}{3.850000in}}%
\pgfusepath{clip}%
\pgfsetbuttcap%
\pgfsetroundjoin%
\definecolor{currentfill}{rgb}{0.897487,0.000000,0.000000}%
\pgfsetfillcolor{currentfill}%
\pgfsetfillopacity{0.800000}%
\pgfsetlinewidth{0.000000pt}%
\definecolor{currentstroke}{rgb}{0.000000,0.000000,0.000000}%
\pgfsetstrokecolor{currentstroke}%
\pgfsetdash{}{0pt}%
\pgfpathmoveto{\pgfqpoint{1.893516in}{3.100587in}}%
\pgfpathlineto{\pgfqpoint{1.823598in}{3.139485in}}%
\pgfpathlineto{\pgfqpoint{1.893516in}{3.098388in}}%
\pgfpathlineto{\pgfqpoint{1.963435in}{3.059490in}}%
\pgfpathlineto{\pgfqpoint{1.893516in}{3.100587in}}%
\pgfpathclose%
\pgfusepath{fill}%
\end{pgfscope}%
\begin{pgfscope}%
\pgfpathrectangle{\pgfqpoint{0.637500in}{0.550000in}}{\pgfqpoint{3.850000in}{3.850000in}}%
\pgfusepath{clip}%
\pgfsetbuttcap%
\pgfsetroundjoin%
\definecolor{currentfill}{rgb}{0.897487,0.000000,0.000000}%
\pgfsetfillcolor{currentfill}%
\pgfsetfillopacity{0.800000}%
\pgfsetlinewidth{0.000000pt}%
\definecolor{currentstroke}{rgb}{0.000000,0.000000,0.000000}%
\pgfsetstrokecolor{currentstroke}%
\pgfsetdash{}{0pt}%
\pgfpathmoveto{\pgfqpoint{1.683761in}{3.221680in}}%
\pgfpathlineto{\pgfqpoint{1.613842in}{3.260578in}}%
\pgfpathlineto{\pgfqpoint{1.683761in}{3.219481in}}%
\pgfpathlineto{\pgfqpoint{1.753679in}{3.180582in}}%
\pgfpathlineto{\pgfqpoint{1.683761in}{3.221680in}}%
\pgfpathclose%
\pgfusepath{fill}%
\end{pgfscope}%
\begin{pgfscope}%
\pgfpathrectangle{\pgfqpoint{0.637500in}{0.550000in}}{\pgfqpoint{3.850000in}{3.850000in}}%
\pgfusepath{clip}%
\pgfsetbuttcap%
\pgfsetroundjoin%
\definecolor{currentfill}{rgb}{0.897487,0.000000,0.000000}%
\pgfsetfillcolor{currentfill}%
\pgfsetfillopacity{0.800000}%
\pgfsetlinewidth{0.000000pt}%
\definecolor{currentstroke}{rgb}{0.000000,0.000000,0.000000}%
\pgfsetstrokecolor{currentstroke}%
\pgfsetdash{}{0pt}%
\pgfpathmoveto{\pgfqpoint{1.474005in}{3.342772in}}%
\pgfpathlineto{\pgfqpoint{1.404086in}{3.381671in}}%
\pgfpathlineto{\pgfqpoint{1.474005in}{3.340573in}}%
\pgfpathlineto{\pgfqpoint{1.543923in}{3.301675in}}%
\pgfpathlineto{\pgfqpoint{1.474005in}{3.342772in}}%
\pgfpathclose%
\pgfusepath{fill}%
\end{pgfscope}%
\begin{pgfscope}%
\pgfpathrectangle{\pgfqpoint{0.637500in}{0.550000in}}{\pgfqpoint{3.850000in}{3.850000in}}%
\pgfusepath{clip}%
\pgfsetbuttcap%
\pgfsetroundjoin%
\definecolor{currentfill}{rgb}{0.909646,0.000000,0.000000}%
\pgfsetfillcolor{currentfill}%
\pgfsetfillopacity{0.800000}%
\pgfsetlinewidth{0.000000pt}%
\definecolor{currentstroke}{rgb}{0.000000,0.000000,0.000000}%
\pgfsetstrokecolor{currentstroke}%
\pgfsetdash{}{0pt}%
\pgfpathmoveto{\pgfqpoint{2.522784in}{2.809553in}}%
\pgfpathlineto{\pgfqpoint{2.452866in}{2.776603in}}%
\pgfpathlineto{\pgfqpoint{2.522784in}{2.807355in}}%
\pgfpathlineto{\pgfqpoint{2.592703in}{2.848452in}}%
\pgfpathlineto{\pgfqpoint{2.522784in}{2.809553in}}%
\pgfpathclose%
\pgfusepath{fill}%
\end{pgfscope}%
\begin{pgfscope}%
\pgfpathrectangle{\pgfqpoint{0.637500in}{0.550000in}}{\pgfqpoint{3.850000in}{3.850000in}}%
\pgfusepath{clip}%
\pgfsetbuttcap%
\pgfsetroundjoin%
\definecolor{currentfill}{rgb}{0.897487,0.000000,0.000000}%
\pgfsetfillcolor{currentfill}%
\pgfsetfillopacity{0.800000}%
\pgfsetlinewidth{0.000000pt}%
\definecolor{currentstroke}{rgb}{0.000000,0.000000,0.000000}%
\pgfsetstrokecolor{currentstroke}%
\pgfsetdash{}{0pt}%
\pgfpathmoveto{\pgfqpoint{2.732540in}{2.930646in}}%
\pgfpathlineto{\pgfqpoint{2.662622in}{2.889549in}}%
\pgfpathlineto{\pgfqpoint{2.732540in}{2.928447in}}%
\pgfpathlineto{\pgfqpoint{2.802459in}{2.969544in}}%
\pgfpathlineto{\pgfqpoint{2.732540in}{2.930646in}}%
\pgfpathclose%
\pgfusepath{fill}%
\end{pgfscope}%
\begin{pgfscope}%
\pgfpathrectangle{\pgfqpoint{0.637500in}{0.550000in}}{\pgfqpoint{3.850000in}{3.850000in}}%
\pgfusepath{clip}%
\pgfsetbuttcap%
\pgfsetroundjoin%
\definecolor{currentfill}{rgb}{0.897487,0.000000,0.000000}%
\pgfsetfillcolor{currentfill}%
\pgfsetfillopacity{0.800000}%
\pgfsetlinewidth{0.000000pt}%
\definecolor{currentstroke}{rgb}{0.000000,0.000000,0.000000}%
\pgfsetstrokecolor{currentstroke}%
\pgfsetdash{}{0pt}%
\pgfpathmoveto{\pgfqpoint{2.942296in}{3.051739in}}%
\pgfpathlineto{\pgfqpoint{2.872378in}{3.010642in}}%
\pgfpathlineto{\pgfqpoint{2.942296in}{3.049540in}}%
\pgfpathlineto{\pgfqpoint{3.012215in}{3.090637in}}%
\pgfpathlineto{\pgfqpoint{2.942296in}{3.051739in}}%
\pgfpathclose%
\pgfusepath{fill}%
\end{pgfscope}%
\begin{pgfscope}%
\pgfpathrectangle{\pgfqpoint{0.637500in}{0.550000in}}{\pgfqpoint{3.850000in}{3.850000in}}%
\pgfusepath{clip}%
\pgfsetbuttcap%
\pgfsetroundjoin%
\definecolor{currentfill}{rgb}{0.897487,0.000000,0.000000}%
\pgfsetfillcolor{currentfill}%
\pgfsetfillopacity{0.800000}%
\pgfsetlinewidth{0.000000pt}%
\definecolor{currentstroke}{rgb}{0.000000,0.000000,0.000000}%
\pgfsetstrokecolor{currentstroke}%
\pgfsetdash{}{0pt}%
\pgfpathmoveto{\pgfqpoint{3.152052in}{3.172831in}}%
\pgfpathlineto{\pgfqpoint{3.082134in}{3.131734in}}%
\pgfpathlineto{\pgfqpoint{3.152052in}{3.170633in}}%
\pgfpathlineto{\pgfqpoint{3.221971in}{3.211730in}}%
\pgfpathlineto{\pgfqpoint{3.152052in}{3.172831in}}%
\pgfpathclose%
\pgfusepath{fill}%
\end{pgfscope}%
\begin{pgfscope}%
\pgfpathrectangle{\pgfqpoint{0.637500in}{0.550000in}}{\pgfqpoint{3.850000in}{3.850000in}}%
\pgfusepath{clip}%
\pgfsetbuttcap%
\pgfsetroundjoin%
\definecolor{currentfill}{rgb}{0.897487,0.000000,0.000000}%
\pgfsetfillcolor{currentfill}%
\pgfsetfillopacity{0.800000}%
\pgfsetlinewidth{0.000000pt}%
\definecolor{currentstroke}{rgb}{0.000000,0.000000,0.000000}%
\pgfsetstrokecolor{currentstroke}%
\pgfsetdash{}{0pt}%
\pgfpathmoveto{\pgfqpoint{2.382947in}{2.817304in}}%
\pgfpathlineto{\pgfqpoint{2.313028in}{2.856203in}}%
\pgfpathlineto{\pgfqpoint{2.382947in}{2.815106in}}%
\pgfpathlineto{\pgfqpoint{2.452866in}{2.776603in}}%
\pgfpathlineto{\pgfqpoint{2.382947in}{2.817304in}}%
\pgfpathclose%
\pgfusepath{fill}%
\end{pgfscope}%
\begin{pgfscope}%
\pgfpathrectangle{\pgfqpoint{0.637500in}{0.550000in}}{\pgfqpoint{3.850000in}{3.850000in}}%
\pgfusepath{clip}%
\pgfsetbuttcap%
\pgfsetroundjoin%
\definecolor{currentfill}{rgb}{0.897487,0.000000,0.000000}%
\pgfsetfillcolor{currentfill}%
\pgfsetfillopacity{0.800000}%
\pgfsetlinewidth{0.000000pt}%
\definecolor{currentstroke}{rgb}{0.000000,0.000000,0.000000}%
\pgfsetstrokecolor{currentstroke}%
\pgfsetdash{}{0pt}%
\pgfpathmoveto{\pgfqpoint{3.361808in}{3.293924in}}%
\pgfpathlineto{\pgfqpoint{3.291890in}{3.252827in}}%
\pgfpathlineto{\pgfqpoint{3.361808in}{3.291725in}}%
\pgfpathlineto{\pgfqpoint{3.431727in}{3.332823in}}%
\pgfpathlineto{\pgfqpoint{3.361808in}{3.293924in}}%
\pgfpathclose%
\pgfusepath{fill}%
\end{pgfscope}%
\begin{pgfscope}%
\pgfpathrectangle{\pgfqpoint{0.637500in}{0.550000in}}{\pgfqpoint{3.850000in}{3.850000in}}%
\pgfusepath{clip}%
\pgfsetbuttcap%
\pgfsetroundjoin%
\definecolor{currentfill}{rgb}{0.897487,0.000000,0.000000}%
\pgfsetfillcolor{currentfill}%
\pgfsetfillopacity{0.800000}%
\pgfsetlinewidth{0.000000pt}%
\definecolor{currentstroke}{rgb}{0.000000,0.000000,0.000000}%
\pgfsetstrokecolor{currentstroke}%
\pgfsetdash{}{0pt}%
\pgfpathmoveto{\pgfqpoint{2.173191in}{2.938397in}}%
\pgfpathlineto{\pgfqpoint{2.103272in}{2.977295in}}%
\pgfpathlineto{\pgfqpoint{2.173191in}{2.936198in}}%
\pgfpathlineto{\pgfqpoint{2.243110in}{2.897300in}}%
\pgfpathlineto{\pgfqpoint{2.173191in}{2.938397in}}%
\pgfpathclose%
\pgfusepath{fill}%
\end{pgfscope}%
\begin{pgfscope}%
\pgfpathrectangle{\pgfqpoint{0.637500in}{0.550000in}}{\pgfqpoint{3.850000in}{3.850000in}}%
\pgfusepath{clip}%
\pgfsetbuttcap%
\pgfsetroundjoin%
\definecolor{currentfill}{rgb}{0.897487,0.000000,0.000000}%
\pgfsetfillcolor{currentfill}%
\pgfsetfillopacity{0.800000}%
\pgfsetlinewidth{0.000000pt}%
\definecolor{currentstroke}{rgb}{0.000000,0.000000,0.000000}%
\pgfsetstrokecolor{currentstroke}%
\pgfsetdash{}{0pt}%
\pgfpathmoveto{\pgfqpoint{3.571564in}{3.415017in}}%
\pgfpathlineto{\pgfqpoint{3.501646in}{3.373920in}}%
\pgfpathlineto{\pgfqpoint{3.571564in}{3.412818in}}%
\pgfpathlineto{\pgfqpoint{3.641483in}{3.453915in}}%
\pgfpathlineto{\pgfqpoint{3.571564in}{3.415017in}}%
\pgfpathclose%
\pgfusepath{fill}%
\end{pgfscope}%
\begin{pgfscope}%
\pgfpathrectangle{\pgfqpoint{0.637500in}{0.550000in}}{\pgfqpoint{3.850000in}{3.850000in}}%
\pgfusepath{clip}%
\pgfsetbuttcap%
\pgfsetroundjoin%
\definecolor{currentfill}{rgb}{0.897487,0.000000,0.000000}%
\pgfsetfillcolor{currentfill}%
\pgfsetfillopacity{0.800000}%
\pgfsetlinewidth{0.000000pt}%
\definecolor{currentstroke}{rgb}{0.000000,0.000000,0.000000}%
\pgfsetstrokecolor{currentstroke}%
\pgfsetdash{}{0pt}%
\pgfpathmoveto{\pgfqpoint{1.963435in}{3.059490in}}%
\pgfpathlineto{\pgfqpoint{1.893516in}{3.098388in}}%
\pgfpathlineto{\pgfqpoint{1.963435in}{3.057291in}}%
\pgfpathlineto{\pgfqpoint{2.033354in}{3.018393in}}%
\pgfpathlineto{\pgfqpoint{1.963435in}{3.059490in}}%
\pgfpathclose%
\pgfusepath{fill}%
\end{pgfscope}%
\begin{pgfscope}%
\pgfpathrectangle{\pgfqpoint{0.637500in}{0.550000in}}{\pgfqpoint{3.850000in}{3.850000in}}%
\pgfusepath{clip}%
\pgfsetbuttcap%
\pgfsetroundjoin%
\definecolor{currentfill}{rgb}{0.897487,0.000000,0.000000}%
\pgfsetfillcolor{currentfill}%
\pgfsetfillopacity{0.800000}%
\pgfsetlinewidth{0.000000pt}%
\definecolor{currentstroke}{rgb}{0.000000,0.000000,0.000000}%
\pgfsetstrokecolor{currentstroke}%
\pgfsetdash{}{0pt}%
\pgfpathmoveto{\pgfqpoint{1.753679in}{3.180582in}}%
\pgfpathlineto{\pgfqpoint{1.683761in}{3.219481in}}%
\pgfpathlineto{\pgfqpoint{1.753679in}{3.178384in}}%
\pgfpathlineto{\pgfqpoint{1.823598in}{3.139485in}}%
\pgfpathlineto{\pgfqpoint{1.753679in}{3.180582in}}%
\pgfpathclose%
\pgfusepath{fill}%
\end{pgfscope}%
\begin{pgfscope}%
\pgfpathrectangle{\pgfqpoint{0.637500in}{0.550000in}}{\pgfqpoint{3.850000in}{3.850000in}}%
\pgfusepath{clip}%
\pgfsetbuttcap%
\pgfsetroundjoin%
\definecolor{currentfill}{rgb}{0.897487,0.000000,0.000000}%
\pgfsetfillcolor{currentfill}%
\pgfsetfillopacity{0.800000}%
\pgfsetlinewidth{0.000000pt}%
\definecolor{currentstroke}{rgb}{0.000000,0.000000,0.000000}%
\pgfsetstrokecolor{currentstroke}%
\pgfsetdash{}{0pt}%
\pgfpathmoveto{\pgfqpoint{1.543923in}{3.301675in}}%
\pgfpathlineto{\pgfqpoint{1.474005in}{3.340573in}}%
\pgfpathlineto{\pgfqpoint{1.543923in}{3.299476in}}%
\pgfpathlineto{\pgfqpoint{1.613842in}{3.260578in}}%
\pgfpathlineto{\pgfqpoint{1.543923in}{3.301675in}}%
\pgfpathclose%
\pgfusepath{fill}%
\end{pgfscope}%
\begin{pgfscope}%
\pgfpathrectangle{\pgfqpoint{0.637500in}{0.550000in}}{\pgfqpoint{3.850000in}{3.850000in}}%
\pgfusepath{clip}%
\pgfsetbuttcap%
\pgfsetroundjoin%
\definecolor{currentfill}{rgb}{0.897487,0.000000,0.000000}%
\pgfsetfillcolor{currentfill}%
\pgfsetfillopacity{0.800000}%
\pgfsetlinewidth{0.000000pt}%
\definecolor{currentstroke}{rgb}{0.000000,0.000000,0.000000}%
\pgfsetstrokecolor{currentstroke}%
\pgfsetdash{}{0pt}%
\pgfpathmoveto{\pgfqpoint{1.334167in}{3.422768in}}%
\pgfpathlineto{\pgfqpoint{1.264249in}{3.461666in}}%
\pgfpathlineto{\pgfqpoint{1.334167in}{3.420569in}}%
\pgfpathlineto{\pgfqpoint{1.404086in}{3.381671in}}%
\pgfpathlineto{\pgfqpoint{1.334167in}{3.422768in}}%
\pgfpathclose%
\pgfusepath{fill}%
\end{pgfscope}%
\begin{pgfscope}%
\pgfpathrectangle{\pgfqpoint{0.637500in}{0.550000in}}{\pgfqpoint{3.850000in}{3.850000in}}%
\pgfusepath{clip}%
\pgfsetbuttcap%
\pgfsetroundjoin%
\definecolor{currentfill}{rgb}{0.897487,0.000000,0.000000}%
\pgfsetfillcolor{currentfill}%
\pgfsetfillopacity{0.800000}%
\pgfsetlinewidth{0.000000pt}%
\definecolor{currentstroke}{rgb}{0.000000,0.000000,0.000000}%
\pgfsetstrokecolor{currentstroke}%
\pgfsetdash{}{0pt}%
\pgfpathmoveto{\pgfqpoint{2.662622in}{2.889549in}}%
\pgfpathlineto{\pgfqpoint{2.592703in}{2.848452in}}%
\pgfpathlineto{\pgfqpoint{2.662622in}{2.887350in}}%
\pgfpathlineto{\pgfqpoint{2.732540in}{2.928447in}}%
\pgfpathlineto{\pgfqpoint{2.662622in}{2.889549in}}%
\pgfpathclose%
\pgfusepath{fill}%
\end{pgfscope}%
\begin{pgfscope}%
\pgfpathrectangle{\pgfqpoint{0.637500in}{0.550000in}}{\pgfqpoint{3.850000in}{3.850000in}}%
\pgfusepath{clip}%
\pgfsetbuttcap%
\pgfsetroundjoin%
\definecolor{currentfill}{rgb}{0.897487,0.000000,0.000000}%
\pgfsetfillcolor{currentfill}%
\pgfsetfillopacity{0.800000}%
\pgfsetlinewidth{0.000000pt}%
\definecolor{currentstroke}{rgb}{0.000000,0.000000,0.000000}%
\pgfsetstrokecolor{currentstroke}%
\pgfsetdash{}{0pt}%
\pgfpathmoveto{\pgfqpoint{2.872378in}{3.010642in}}%
\pgfpathlineto{\pgfqpoint{2.802459in}{2.969544in}}%
\pgfpathlineto{\pgfqpoint{2.872378in}{3.008443in}}%
\pgfpathlineto{\pgfqpoint{2.942296in}{3.049540in}}%
\pgfpathlineto{\pgfqpoint{2.872378in}{3.010642in}}%
\pgfpathclose%
\pgfusepath{fill}%
\end{pgfscope}%
\begin{pgfscope}%
\pgfpathrectangle{\pgfqpoint{0.637500in}{0.550000in}}{\pgfqpoint{3.850000in}{3.850000in}}%
\pgfusepath{clip}%
\pgfsetbuttcap%
\pgfsetroundjoin%
\definecolor{currentfill}{rgb}{0.897487,0.000000,0.000000}%
\pgfsetfillcolor{currentfill}%
\pgfsetfillopacity{0.800000}%
\pgfsetlinewidth{0.000000pt}%
\definecolor{currentstroke}{rgb}{0.000000,0.000000,0.000000}%
\pgfsetstrokecolor{currentstroke}%
\pgfsetdash{}{0pt}%
\pgfpathmoveto{\pgfqpoint{3.082134in}{3.131734in}}%
\pgfpathlineto{\pgfqpoint{3.012215in}{3.090637in}}%
\pgfpathlineto{\pgfqpoint{3.082134in}{3.129536in}}%
\pgfpathlineto{\pgfqpoint{3.152052in}{3.170633in}}%
\pgfpathlineto{\pgfqpoint{3.082134in}{3.131734in}}%
\pgfpathclose%
\pgfusepath{fill}%
\end{pgfscope}%
\begin{pgfscope}%
\pgfpathrectangle{\pgfqpoint{0.637500in}{0.550000in}}{\pgfqpoint{3.850000in}{3.850000in}}%
\pgfusepath{clip}%
\pgfsetbuttcap%
\pgfsetroundjoin%
\definecolor{currentfill}{rgb}{0.897487,0.000000,0.000000}%
\pgfsetfillcolor{currentfill}%
\pgfsetfillopacity{0.800000}%
\pgfsetlinewidth{0.000000pt}%
\definecolor{currentstroke}{rgb}{0.000000,0.000000,0.000000}%
\pgfsetstrokecolor{currentstroke}%
\pgfsetdash{}{0pt}%
\pgfpathmoveto{\pgfqpoint{3.291890in}{3.252827in}}%
\pgfpathlineto{\pgfqpoint{3.221971in}{3.211730in}}%
\pgfpathlineto{\pgfqpoint{3.291890in}{3.250628in}}%
\pgfpathlineto{\pgfqpoint{3.361808in}{3.291725in}}%
\pgfpathlineto{\pgfqpoint{3.291890in}{3.252827in}}%
\pgfpathclose%
\pgfusepath{fill}%
\end{pgfscope}%
\begin{pgfscope}%
\pgfpathrectangle{\pgfqpoint{0.637500in}{0.550000in}}{\pgfqpoint{3.850000in}{3.850000in}}%
\pgfusepath{clip}%
\pgfsetbuttcap%
\pgfsetroundjoin%
\definecolor{currentfill}{rgb}{0.897487,0.000000,0.000000}%
\pgfsetfillcolor{currentfill}%
\pgfsetfillopacity{0.800000}%
\pgfsetlinewidth{0.000000pt}%
\definecolor{currentstroke}{rgb}{0.000000,0.000000,0.000000}%
\pgfsetstrokecolor{currentstroke}%
\pgfsetdash{}{0pt}%
\pgfpathmoveto{\pgfqpoint{2.243110in}{2.897300in}}%
\pgfpathlineto{\pgfqpoint{2.173191in}{2.936198in}}%
\pgfpathlineto{\pgfqpoint{2.243110in}{2.895101in}}%
\pgfpathlineto{\pgfqpoint{2.313028in}{2.856203in}}%
\pgfpathlineto{\pgfqpoint{2.243110in}{2.897300in}}%
\pgfpathclose%
\pgfusepath{fill}%
\end{pgfscope}%
\begin{pgfscope}%
\pgfpathrectangle{\pgfqpoint{0.637500in}{0.550000in}}{\pgfqpoint{3.850000in}{3.850000in}}%
\pgfusepath{clip}%
\pgfsetbuttcap%
\pgfsetroundjoin%
\definecolor{currentfill}{rgb}{0.897487,0.000000,0.000000}%
\pgfsetfillcolor{currentfill}%
\pgfsetfillopacity{0.800000}%
\pgfsetlinewidth{0.000000pt}%
\definecolor{currentstroke}{rgb}{0.000000,0.000000,0.000000}%
\pgfsetstrokecolor{currentstroke}%
\pgfsetdash{}{0pt}%
\pgfpathmoveto{\pgfqpoint{3.501646in}{3.373920in}}%
\pgfpathlineto{\pgfqpoint{3.431727in}{3.332823in}}%
\pgfpathlineto{\pgfqpoint{3.501646in}{3.371721in}}%
\pgfpathlineto{\pgfqpoint{3.571564in}{3.412818in}}%
\pgfpathlineto{\pgfqpoint{3.501646in}{3.373920in}}%
\pgfpathclose%
\pgfusepath{fill}%
\end{pgfscope}%
\begin{pgfscope}%
\pgfpathrectangle{\pgfqpoint{0.637500in}{0.550000in}}{\pgfqpoint{3.850000in}{3.850000in}}%
\pgfusepath{clip}%
\pgfsetbuttcap%
\pgfsetroundjoin%
\definecolor{currentfill}{rgb}{0.897487,0.000000,0.000000}%
\pgfsetfillcolor{currentfill}%
\pgfsetfillopacity{0.800000}%
\pgfsetlinewidth{0.000000pt}%
\definecolor{currentstroke}{rgb}{0.000000,0.000000,0.000000}%
\pgfsetstrokecolor{currentstroke}%
\pgfsetdash{}{0pt}%
\pgfpathmoveto{\pgfqpoint{2.033354in}{3.018393in}}%
\pgfpathlineto{\pgfqpoint{1.963435in}{3.057291in}}%
\pgfpathlineto{\pgfqpoint{2.033354in}{3.016194in}}%
\pgfpathlineto{\pgfqpoint{2.103272in}{2.977295in}}%
\pgfpathlineto{\pgfqpoint{2.033354in}{3.018393in}}%
\pgfpathclose%
\pgfusepath{fill}%
\end{pgfscope}%
\begin{pgfscope}%
\pgfpathrectangle{\pgfqpoint{0.637500in}{0.550000in}}{\pgfqpoint{3.850000in}{3.850000in}}%
\pgfusepath{clip}%
\pgfsetbuttcap%
\pgfsetroundjoin%
\definecolor{currentfill}{rgb}{0.897487,0.000000,0.000000}%
\pgfsetfillcolor{currentfill}%
\pgfsetfillopacity{0.800000}%
\pgfsetlinewidth{0.000000pt}%
\definecolor{currentstroke}{rgb}{0.000000,0.000000,0.000000}%
\pgfsetstrokecolor{currentstroke}%
\pgfsetdash{}{0pt}%
\pgfpathmoveto{\pgfqpoint{3.711402in}{3.495012in}}%
\pgfpathlineto{\pgfqpoint{3.641483in}{3.453915in}}%
\pgfpathlineto{\pgfqpoint{3.711402in}{3.492814in}}%
\pgfpathlineto{\pgfqpoint{3.781320in}{3.533911in}}%
\pgfpathlineto{\pgfqpoint{3.711402in}{3.495012in}}%
\pgfpathclose%
\pgfusepath{fill}%
\end{pgfscope}%
\begin{pgfscope}%
\pgfpathrectangle{\pgfqpoint{0.637500in}{0.550000in}}{\pgfqpoint{3.850000in}{3.850000in}}%
\pgfusepath{clip}%
\pgfsetbuttcap%
\pgfsetroundjoin%
\definecolor{currentfill}{rgb}{0.897487,0.000000,0.000000}%
\pgfsetfillcolor{currentfill}%
\pgfsetfillopacity{0.800000}%
\pgfsetlinewidth{0.000000pt}%
\definecolor{currentstroke}{rgb}{0.000000,0.000000,0.000000}%
\pgfsetstrokecolor{currentstroke}%
\pgfsetdash{}{0pt}%
\pgfpathmoveto{\pgfqpoint{1.823598in}{3.139485in}}%
\pgfpathlineto{\pgfqpoint{1.753679in}{3.178384in}}%
\pgfpathlineto{\pgfqpoint{1.823598in}{3.137286in}}%
\pgfpathlineto{\pgfqpoint{1.893516in}{3.098388in}}%
\pgfpathlineto{\pgfqpoint{1.823598in}{3.139485in}}%
\pgfpathclose%
\pgfusepath{fill}%
\end{pgfscope}%
\begin{pgfscope}%
\pgfpathrectangle{\pgfqpoint{0.637500in}{0.550000in}}{\pgfqpoint{3.850000in}{3.850000in}}%
\pgfusepath{clip}%
\pgfsetbuttcap%
\pgfsetroundjoin%
\definecolor{currentfill}{rgb}{0.897487,0.000000,0.000000}%
\pgfsetfillcolor{currentfill}%
\pgfsetfillopacity{0.800000}%
\pgfsetlinewidth{0.000000pt}%
\definecolor{currentstroke}{rgb}{0.000000,0.000000,0.000000}%
\pgfsetstrokecolor{currentstroke}%
\pgfsetdash{}{0pt}%
\pgfpathmoveto{\pgfqpoint{1.613842in}{3.260578in}}%
\pgfpathlineto{\pgfqpoint{1.543923in}{3.299476in}}%
\pgfpathlineto{\pgfqpoint{1.613842in}{3.258379in}}%
\pgfpathlineto{\pgfqpoint{1.683761in}{3.219481in}}%
\pgfpathlineto{\pgfqpoint{1.613842in}{3.260578in}}%
\pgfpathclose%
\pgfusepath{fill}%
\end{pgfscope}%
\begin{pgfscope}%
\pgfpathrectangle{\pgfqpoint{0.637500in}{0.550000in}}{\pgfqpoint{3.850000in}{3.850000in}}%
\pgfusepath{clip}%
\pgfsetbuttcap%
\pgfsetroundjoin%
\definecolor{currentfill}{rgb}{0.897487,0.000000,0.000000}%
\pgfsetfillcolor{currentfill}%
\pgfsetfillopacity{0.800000}%
\pgfsetlinewidth{0.000000pt}%
\definecolor{currentstroke}{rgb}{0.000000,0.000000,0.000000}%
\pgfsetstrokecolor{currentstroke}%
\pgfsetdash{}{0pt}%
\pgfpathmoveto{\pgfqpoint{1.404086in}{3.381671in}}%
\pgfpathlineto{\pgfqpoint{1.334167in}{3.420569in}}%
\pgfpathlineto{\pgfqpoint{1.404086in}{3.379472in}}%
\pgfpathlineto{\pgfqpoint{1.474005in}{3.340573in}}%
\pgfpathlineto{\pgfqpoint{1.404086in}{3.381671in}}%
\pgfpathclose%
\pgfusepath{fill}%
\end{pgfscope}%
\begin{pgfscope}%
\pgfpathrectangle{\pgfqpoint{0.637500in}{0.550000in}}{\pgfqpoint{3.850000in}{3.850000in}}%
\pgfusepath{clip}%
\pgfsetbuttcap%
\pgfsetroundjoin%
\definecolor{currentfill}{rgb}{0.898098,0.000000,0.000000}%
\pgfsetfillcolor{currentfill}%
\pgfsetfillopacity{0.800000}%
\pgfsetlinewidth{0.000000pt}%
\definecolor{currentstroke}{rgb}{0.000000,0.000000,0.000000}%
\pgfsetstrokecolor{currentstroke}%
\pgfsetdash{}{0pt}%
\pgfpathmoveto{\pgfqpoint{2.452866in}{2.776603in}}%
\pgfpathlineto{\pgfqpoint{2.382947in}{2.815106in}}%
\pgfpathlineto{\pgfqpoint{2.452866in}{2.774404in}}%
\pgfpathlineto{\pgfqpoint{2.522784in}{2.807355in}}%
\pgfpathlineto{\pgfqpoint{2.452866in}{2.776603in}}%
\pgfpathclose%
\pgfusepath{fill}%
\end{pgfscope}%
\begin{pgfscope}%
\pgfpathrectangle{\pgfqpoint{0.637500in}{0.550000in}}{\pgfqpoint{3.850000in}{3.850000in}}%
\pgfusepath{clip}%
\pgfsetbuttcap%
\pgfsetroundjoin%
\definecolor{currentfill}{rgb}{0.897487,0.000000,0.000000}%
\pgfsetfillcolor{currentfill}%
\pgfsetfillopacity{0.800000}%
\pgfsetlinewidth{0.000000pt}%
\definecolor{currentstroke}{rgb}{0.000000,0.000000,0.000000}%
\pgfsetstrokecolor{currentstroke}%
\pgfsetdash{}{0pt}%
\pgfpathmoveto{\pgfqpoint{2.592703in}{2.848452in}}%
\pgfpathlineto{\pgfqpoint{2.522784in}{2.807355in}}%
\pgfpathlineto{\pgfqpoint{2.592703in}{2.846253in}}%
\pgfpathlineto{\pgfqpoint{2.662622in}{2.887350in}}%
\pgfpathlineto{\pgfqpoint{2.592703in}{2.848452in}}%
\pgfpathclose%
\pgfusepath{fill}%
\end{pgfscope}%
\begin{pgfscope}%
\pgfpathrectangle{\pgfqpoint{0.637500in}{0.550000in}}{\pgfqpoint{3.850000in}{3.850000in}}%
\pgfusepath{clip}%
\pgfsetbuttcap%
\pgfsetroundjoin%
\definecolor{currentfill}{rgb}{0.897487,0.000000,0.000000}%
\pgfsetfillcolor{currentfill}%
\pgfsetfillopacity{0.800000}%
\pgfsetlinewidth{0.000000pt}%
\definecolor{currentstroke}{rgb}{0.000000,0.000000,0.000000}%
\pgfsetstrokecolor{currentstroke}%
\pgfsetdash{}{0pt}%
\pgfpathmoveto{\pgfqpoint{2.802459in}{2.969544in}}%
\pgfpathlineto{\pgfqpoint{2.732540in}{2.928447in}}%
\pgfpathlineto{\pgfqpoint{2.802459in}{2.967346in}}%
\pgfpathlineto{\pgfqpoint{2.872378in}{3.008443in}}%
\pgfpathlineto{\pgfqpoint{2.802459in}{2.969544in}}%
\pgfpathclose%
\pgfusepath{fill}%
\end{pgfscope}%
\begin{pgfscope}%
\pgfpathrectangle{\pgfqpoint{0.637500in}{0.550000in}}{\pgfqpoint{3.850000in}{3.850000in}}%
\pgfusepath{clip}%
\pgfsetbuttcap%
\pgfsetroundjoin%
\definecolor{currentfill}{rgb}{0.897487,0.000000,0.000000}%
\pgfsetfillcolor{currentfill}%
\pgfsetfillopacity{0.800000}%
\pgfsetlinewidth{0.000000pt}%
\definecolor{currentstroke}{rgb}{0.000000,0.000000,0.000000}%
\pgfsetstrokecolor{currentstroke}%
\pgfsetdash{}{0pt}%
\pgfpathmoveto{\pgfqpoint{3.012215in}{3.090637in}}%
\pgfpathlineto{\pgfqpoint{2.942296in}{3.049540in}}%
\pgfpathlineto{\pgfqpoint{3.012215in}{3.088438in}}%
\pgfpathlineto{\pgfqpoint{3.082134in}{3.129536in}}%
\pgfpathlineto{\pgfqpoint{3.012215in}{3.090637in}}%
\pgfpathclose%
\pgfusepath{fill}%
\end{pgfscope}%
\begin{pgfscope}%
\pgfpathrectangle{\pgfqpoint{0.637500in}{0.550000in}}{\pgfqpoint{3.850000in}{3.850000in}}%
\pgfusepath{clip}%
\pgfsetbuttcap%
\pgfsetroundjoin%
\definecolor{currentfill}{rgb}{0.897487,0.000000,0.000000}%
\pgfsetfillcolor{currentfill}%
\pgfsetfillopacity{0.800000}%
\pgfsetlinewidth{0.000000pt}%
\definecolor{currentstroke}{rgb}{0.000000,0.000000,0.000000}%
\pgfsetstrokecolor{currentstroke}%
\pgfsetdash{}{0pt}%
\pgfpathmoveto{\pgfqpoint{3.221971in}{3.211730in}}%
\pgfpathlineto{\pgfqpoint{3.152052in}{3.170633in}}%
\pgfpathlineto{\pgfqpoint{3.221971in}{3.209531in}}%
\pgfpathlineto{\pgfqpoint{3.291890in}{3.250628in}}%
\pgfpathlineto{\pgfqpoint{3.221971in}{3.211730in}}%
\pgfpathclose%
\pgfusepath{fill}%
\end{pgfscope}%
\begin{pgfscope}%
\pgfpathrectangle{\pgfqpoint{0.637500in}{0.550000in}}{\pgfqpoint{3.850000in}{3.850000in}}%
\pgfusepath{clip}%
\pgfsetbuttcap%
\pgfsetroundjoin%
\definecolor{currentfill}{rgb}{0.897487,0.000000,0.000000}%
\pgfsetfillcolor{currentfill}%
\pgfsetfillopacity{0.800000}%
\pgfsetlinewidth{0.000000pt}%
\definecolor{currentstroke}{rgb}{0.000000,0.000000,0.000000}%
\pgfsetstrokecolor{currentstroke}%
\pgfsetdash{}{0pt}%
\pgfpathmoveto{\pgfqpoint{2.313028in}{2.856203in}}%
\pgfpathlineto{\pgfqpoint{2.243110in}{2.895101in}}%
\pgfpathlineto{\pgfqpoint{2.313028in}{2.854004in}}%
\pgfpathlineto{\pgfqpoint{2.382947in}{2.815106in}}%
\pgfpathlineto{\pgfqpoint{2.313028in}{2.856203in}}%
\pgfpathclose%
\pgfusepath{fill}%
\end{pgfscope}%
\begin{pgfscope}%
\pgfpathrectangle{\pgfqpoint{0.637500in}{0.550000in}}{\pgfqpoint{3.850000in}{3.850000in}}%
\pgfusepath{clip}%
\pgfsetbuttcap%
\pgfsetroundjoin%
\definecolor{currentfill}{rgb}{0.897487,0.000000,0.000000}%
\pgfsetfillcolor{currentfill}%
\pgfsetfillopacity{0.800000}%
\pgfsetlinewidth{0.000000pt}%
\definecolor{currentstroke}{rgb}{0.000000,0.000000,0.000000}%
\pgfsetstrokecolor{currentstroke}%
\pgfsetdash{}{0pt}%
\pgfpathmoveto{\pgfqpoint{3.431727in}{3.332823in}}%
\pgfpathlineto{\pgfqpoint{3.361808in}{3.291725in}}%
\pgfpathlineto{\pgfqpoint{3.431727in}{3.330624in}}%
\pgfpathlineto{\pgfqpoint{3.501646in}{3.371721in}}%
\pgfpathlineto{\pgfqpoint{3.431727in}{3.332823in}}%
\pgfpathclose%
\pgfusepath{fill}%
\end{pgfscope}%
\begin{pgfscope}%
\pgfpathrectangle{\pgfqpoint{0.637500in}{0.550000in}}{\pgfqpoint{3.850000in}{3.850000in}}%
\pgfusepath{clip}%
\pgfsetbuttcap%
\pgfsetroundjoin%
\definecolor{currentfill}{rgb}{0.897487,0.000000,0.000000}%
\pgfsetfillcolor{currentfill}%
\pgfsetfillopacity{0.800000}%
\pgfsetlinewidth{0.000000pt}%
\definecolor{currentstroke}{rgb}{0.000000,0.000000,0.000000}%
\pgfsetstrokecolor{currentstroke}%
\pgfsetdash{}{0pt}%
\pgfpathmoveto{\pgfqpoint{2.103272in}{2.977295in}}%
\pgfpathlineto{\pgfqpoint{2.033354in}{3.016194in}}%
\pgfpathlineto{\pgfqpoint{2.103272in}{2.975097in}}%
\pgfpathlineto{\pgfqpoint{2.173191in}{2.936198in}}%
\pgfpathlineto{\pgfqpoint{2.103272in}{2.977295in}}%
\pgfpathclose%
\pgfusepath{fill}%
\end{pgfscope}%
\begin{pgfscope}%
\pgfpathrectangle{\pgfqpoint{0.637500in}{0.550000in}}{\pgfqpoint{3.850000in}{3.850000in}}%
\pgfusepath{clip}%
\pgfsetbuttcap%
\pgfsetroundjoin%
\definecolor{currentfill}{rgb}{0.897487,0.000000,0.000000}%
\pgfsetfillcolor{currentfill}%
\pgfsetfillopacity{0.800000}%
\pgfsetlinewidth{0.000000pt}%
\definecolor{currentstroke}{rgb}{0.000000,0.000000,0.000000}%
\pgfsetstrokecolor{currentstroke}%
\pgfsetdash{}{0pt}%
\pgfpathmoveto{\pgfqpoint{3.641483in}{3.453915in}}%
\pgfpathlineto{\pgfqpoint{3.571564in}{3.412818in}}%
\pgfpathlineto{\pgfqpoint{3.641483in}{3.451716in}}%
\pgfpathlineto{\pgfqpoint{3.711402in}{3.492814in}}%
\pgfpathlineto{\pgfqpoint{3.641483in}{3.453915in}}%
\pgfpathclose%
\pgfusepath{fill}%
\end{pgfscope}%
\begin{pgfscope}%
\pgfpathrectangle{\pgfqpoint{0.637500in}{0.550000in}}{\pgfqpoint{3.850000in}{3.850000in}}%
\pgfusepath{clip}%
\pgfsetbuttcap%
\pgfsetroundjoin%
\definecolor{currentfill}{rgb}{0.897487,0.000000,0.000000}%
\pgfsetfillcolor{currentfill}%
\pgfsetfillopacity{0.800000}%
\pgfsetlinewidth{0.000000pt}%
\definecolor{currentstroke}{rgb}{0.000000,0.000000,0.000000}%
\pgfsetstrokecolor{currentstroke}%
\pgfsetdash{}{0pt}%
\pgfpathmoveto{\pgfqpoint{1.893516in}{3.098388in}}%
\pgfpathlineto{\pgfqpoint{1.823598in}{3.137286in}}%
\pgfpathlineto{\pgfqpoint{1.893516in}{3.096189in}}%
\pgfpathlineto{\pgfqpoint{1.963435in}{3.057291in}}%
\pgfpathlineto{\pgfqpoint{1.893516in}{3.098388in}}%
\pgfpathclose%
\pgfusepath{fill}%
\end{pgfscope}%
\begin{pgfscope}%
\pgfpathrectangle{\pgfqpoint{0.637500in}{0.550000in}}{\pgfqpoint{3.850000in}{3.850000in}}%
\pgfusepath{clip}%
\pgfsetbuttcap%
\pgfsetroundjoin%
\definecolor{currentfill}{rgb}{0.897487,0.000000,0.000000}%
\pgfsetfillcolor{currentfill}%
\pgfsetfillopacity{0.800000}%
\pgfsetlinewidth{0.000000pt}%
\definecolor{currentstroke}{rgb}{0.000000,0.000000,0.000000}%
\pgfsetstrokecolor{currentstroke}%
\pgfsetdash{}{0pt}%
\pgfpathmoveto{\pgfqpoint{1.683761in}{3.219481in}}%
\pgfpathlineto{\pgfqpoint{1.613842in}{3.258379in}}%
\pgfpathlineto{\pgfqpoint{1.683761in}{3.217282in}}%
\pgfpathlineto{\pgfqpoint{1.753679in}{3.178384in}}%
\pgfpathlineto{\pgfqpoint{1.683761in}{3.219481in}}%
\pgfpathclose%
\pgfusepath{fill}%
\end{pgfscope}%
\begin{pgfscope}%
\pgfpathrectangle{\pgfqpoint{0.637500in}{0.550000in}}{\pgfqpoint{3.850000in}{3.850000in}}%
\pgfusepath{clip}%
\pgfsetbuttcap%
\pgfsetroundjoin%
\definecolor{currentfill}{rgb}{0.897487,0.000000,0.000000}%
\pgfsetfillcolor{currentfill}%
\pgfsetfillopacity{0.800000}%
\pgfsetlinewidth{0.000000pt}%
\definecolor{currentstroke}{rgb}{0.000000,0.000000,0.000000}%
\pgfsetstrokecolor{currentstroke}%
\pgfsetdash{}{0pt}%
\pgfpathmoveto{\pgfqpoint{1.474005in}{3.340573in}}%
\pgfpathlineto{\pgfqpoint{1.404086in}{3.379472in}}%
\pgfpathlineto{\pgfqpoint{1.474005in}{3.338375in}}%
\pgfpathlineto{\pgfqpoint{1.543923in}{3.299476in}}%
\pgfpathlineto{\pgfqpoint{1.474005in}{3.340573in}}%
\pgfpathclose%
\pgfusepath{fill}%
\end{pgfscope}%
\begin{pgfscope}%
\pgfpathrectangle{\pgfqpoint{0.637500in}{0.550000in}}{\pgfqpoint{3.850000in}{3.850000in}}%
\pgfusepath{clip}%
\pgfsetbuttcap%
\pgfsetroundjoin%
\definecolor{currentfill}{rgb}{0.909646,0.000000,0.000000}%
\pgfsetfillcolor{currentfill}%
\pgfsetfillopacity{0.800000}%
\pgfsetlinewidth{0.000000pt}%
\definecolor{currentstroke}{rgb}{0.000000,0.000000,0.000000}%
\pgfsetstrokecolor{currentstroke}%
\pgfsetdash{}{0pt}%
\pgfpathmoveto{\pgfqpoint{2.522784in}{2.807355in}}%
\pgfpathlineto{\pgfqpoint{2.452866in}{2.774404in}}%
\pgfpathlineto{\pgfqpoint{2.522784in}{2.805156in}}%
\pgfpathlineto{\pgfqpoint{2.592703in}{2.846253in}}%
\pgfpathlineto{\pgfqpoint{2.522784in}{2.807355in}}%
\pgfpathclose%
\pgfusepath{fill}%
\end{pgfscope}%
\begin{pgfscope}%
\pgfpathrectangle{\pgfqpoint{0.637500in}{0.550000in}}{\pgfqpoint{3.850000in}{3.850000in}}%
\pgfusepath{clip}%
\pgfsetbuttcap%
\pgfsetroundjoin%
\definecolor{currentfill}{rgb}{0.897487,0.000000,0.000000}%
\pgfsetfillcolor{currentfill}%
\pgfsetfillopacity{0.800000}%
\pgfsetlinewidth{0.000000pt}%
\definecolor{currentstroke}{rgb}{0.000000,0.000000,0.000000}%
\pgfsetstrokecolor{currentstroke}%
\pgfsetdash{}{0pt}%
\pgfpathmoveto{\pgfqpoint{2.732540in}{2.928447in}}%
\pgfpathlineto{\pgfqpoint{2.662622in}{2.887350in}}%
\pgfpathlineto{\pgfqpoint{2.732540in}{2.926249in}}%
\pgfpathlineto{\pgfqpoint{2.802459in}{2.967346in}}%
\pgfpathlineto{\pgfqpoint{2.732540in}{2.928447in}}%
\pgfpathclose%
\pgfusepath{fill}%
\end{pgfscope}%
\begin{pgfscope}%
\pgfpathrectangle{\pgfqpoint{0.637500in}{0.550000in}}{\pgfqpoint{3.850000in}{3.850000in}}%
\pgfusepath{clip}%
\pgfsetbuttcap%
\pgfsetroundjoin%
\definecolor{currentfill}{rgb}{0.897487,0.000000,0.000000}%
\pgfsetfillcolor{currentfill}%
\pgfsetfillopacity{0.800000}%
\pgfsetlinewidth{0.000000pt}%
\definecolor{currentstroke}{rgb}{0.000000,0.000000,0.000000}%
\pgfsetstrokecolor{currentstroke}%
\pgfsetdash{}{0pt}%
\pgfpathmoveto{\pgfqpoint{2.942296in}{3.049540in}}%
\pgfpathlineto{\pgfqpoint{2.872378in}{3.008443in}}%
\pgfpathlineto{\pgfqpoint{2.942296in}{3.047341in}}%
\pgfpathlineto{\pgfqpoint{3.012215in}{3.088438in}}%
\pgfpathlineto{\pgfqpoint{2.942296in}{3.049540in}}%
\pgfpathclose%
\pgfusepath{fill}%
\end{pgfscope}%
\begin{pgfscope}%
\pgfpathrectangle{\pgfqpoint{0.637500in}{0.550000in}}{\pgfqpoint{3.850000in}{3.850000in}}%
\pgfusepath{clip}%
\pgfsetbuttcap%
\pgfsetroundjoin%
\definecolor{currentfill}{rgb}{0.897487,0.000000,0.000000}%
\pgfsetfillcolor{currentfill}%
\pgfsetfillopacity{0.800000}%
\pgfsetlinewidth{0.000000pt}%
\definecolor{currentstroke}{rgb}{0.000000,0.000000,0.000000}%
\pgfsetstrokecolor{currentstroke}%
\pgfsetdash{}{0pt}%
\pgfpathmoveto{\pgfqpoint{3.152052in}{3.170633in}}%
\pgfpathlineto{\pgfqpoint{3.082134in}{3.129536in}}%
\pgfpathlineto{\pgfqpoint{3.152052in}{3.168434in}}%
\pgfpathlineto{\pgfqpoint{3.221971in}{3.209531in}}%
\pgfpathlineto{\pgfqpoint{3.152052in}{3.170633in}}%
\pgfpathclose%
\pgfusepath{fill}%
\end{pgfscope}%
\begin{pgfscope}%
\pgfpathrectangle{\pgfqpoint{0.637500in}{0.550000in}}{\pgfqpoint{3.850000in}{3.850000in}}%
\pgfusepath{clip}%
\pgfsetbuttcap%
\pgfsetroundjoin%
\definecolor{currentfill}{rgb}{0.897487,0.000000,0.000000}%
\pgfsetfillcolor{currentfill}%
\pgfsetfillopacity{0.800000}%
\pgfsetlinewidth{0.000000pt}%
\definecolor{currentstroke}{rgb}{0.000000,0.000000,0.000000}%
\pgfsetstrokecolor{currentstroke}%
\pgfsetdash{}{0pt}%
\pgfpathmoveto{\pgfqpoint{2.382947in}{2.815106in}}%
\pgfpathlineto{\pgfqpoint{2.313028in}{2.854004in}}%
\pgfpathlineto{\pgfqpoint{2.382947in}{2.812907in}}%
\pgfpathlineto{\pgfqpoint{2.452866in}{2.774404in}}%
\pgfpathlineto{\pgfqpoint{2.382947in}{2.815106in}}%
\pgfpathclose%
\pgfusepath{fill}%
\end{pgfscope}%
\begin{pgfscope}%
\pgfpathrectangle{\pgfqpoint{0.637500in}{0.550000in}}{\pgfqpoint{3.850000in}{3.850000in}}%
\pgfusepath{clip}%
\pgfsetbuttcap%
\pgfsetroundjoin%
\definecolor{currentfill}{rgb}{0.897487,0.000000,0.000000}%
\pgfsetfillcolor{currentfill}%
\pgfsetfillopacity{0.800000}%
\pgfsetlinewidth{0.000000pt}%
\definecolor{currentstroke}{rgb}{0.000000,0.000000,0.000000}%
\pgfsetstrokecolor{currentstroke}%
\pgfsetdash{}{0pt}%
\pgfpathmoveto{\pgfqpoint{3.361808in}{3.291725in}}%
\pgfpathlineto{\pgfqpoint{3.291890in}{3.250628in}}%
\pgfpathlineto{\pgfqpoint{3.361808in}{3.289527in}}%
\pgfpathlineto{\pgfqpoint{3.431727in}{3.330624in}}%
\pgfpathlineto{\pgfqpoint{3.361808in}{3.291725in}}%
\pgfpathclose%
\pgfusepath{fill}%
\end{pgfscope}%
\begin{pgfscope}%
\pgfpathrectangle{\pgfqpoint{0.637500in}{0.550000in}}{\pgfqpoint{3.850000in}{3.850000in}}%
\pgfusepath{clip}%
\pgfsetbuttcap%
\pgfsetroundjoin%
\definecolor{currentfill}{rgb}{0.897487,0.000000,0.000000}%
\pgfsetfillcolor{currentfill}%
\pgfsetfillopacity{0.800000}%
\pgfsetlinewidth{0.000000pt}%
\definecolor{currentstroke}{rgb}{0.000000,0.000000,0.000000}%
\pgfsetstrokecolor{currentstroke}%
\pgfsetdash{}{0pt}%
\pgfpathmoveto{\pgfqpoint{2.173191in}{2.936198in}}%
\pgfpathlineto{\pgfqpoint{2.103272in}{2.975097in}}%
\pgfpathlineto{\pgfqpoint{2.173191in}{2.933999in}}%
\pgfpathlineto{\pgfqpoint{2.243110in}{2.895101in}}%
\pgfpathlineto{\pgfqpoint{2.173191in}{2.936198in}}%
\pgfpathclose%
\pgfusepath{fill}%
\end{pgfscope}%
\begin{pgfscope}%
\pgfpathrectangle{\pgfqpoint{0.637500in}{0.550000in}}{\pgfqpoint{3.850000in}{3.850000in}}%
\pgfusepath{clip}%
\pgfsetbuttcap%
\pgfsetroundjoin%
\definecolor{currentfill}{rgb}{0.897487,0.000000,0.000000}%
\pgfsetfillcolor{currentfill}%
\pgfsetfillopacity{0.800000}%
\pgfsetlinewidth{0.000000pt}%
\definecolor{currentstroke}{rgb}{0.000000,0.000000,0.000000}%
\pgfsetstrokecolor{currentstroke}%
\pgfsetdash{}{0pt}%
\pgfpathmoveto{\pgfqpoint{3.571564in}{3.412818in}}%
\pgfpathlineto{\pgfqpoint{3.501646in}{3.371721in}}%
\pgfpathlineto{\pgfqpoint{3.571564in}{3.410619in}}%
\pgfpathlineto{\pgfqpoint{3.641483in}{3.451716in}}%
\pgfpathlineto{\pgfqpoint{3.571564in}{3.412818in}}%
\pgfpathclose%
\pgfusepath{fill}%
\end{pgfscope}%
\begin{pgfscope}%
\pgfpathrectangle{\pgfqpoint{0.637500in}{0.550000in}}{\pgfqpoint{3.850000in}{3.850000in}}%
\pgfusepath{clip}%
\pgfsetbuttcap%
\pgfsetroundjoin%
\definecolor{currentfill}{rgb}{0.897487,0.000000,0.000000}%
\pgfsetfillcolor{currentfill}%
\pgfsetfillopacity{0.800000}%
\pgfsetlinewidth{0.000000pt}%
\definecolor{currentstroke}{rgb}{0.000000,0.000000,0.000000}%
\pgfsetstrokecolor{currentstroke}%
\pgfsetdash{}{0pt}%
\pgfpathmoveto{\pgfqpoint{1.963435in}{3.057291in}}%
\pgfpathlineto{\pgfqpoint{1.893516in}{3.096189in}}%
\pgfpathlineto{\pgfqpoint{1.963435in}{3.055092in}}%
\pgfpathlineto{\pgfqpoint{2.033354in}{3.016194in}}%
\pgfpathlineto{\pgfqpoint{1.963435in}{3.057291in}}%
\pgfpathclose%
\pgfusepath{fill}%
\end{pgfscope}%
\begin{pgfscope}%
\pgfpathrectangle{\pgfqpoint{0.637500in}{0.550000in}}{\pgfqpoint{3.850000in}{3.850000in}}%
\pgfusepath{clip}%
\pgfsetbuttcap%
\pgfsetroundjoin%
\definecolor{currentfill}{rgb}{0.897487,0.000000,0.000000}%
\pgfsetfillcolor{currentfill}%
\pgfsetfillopacity{0.800000}%
\pgfsetlinewidth{0.000000pt}%
\definecolor{currentstroke}{rgb}{0.000000,0.000000,0.000000}%
\pgfsetstrokecolor{currentstroke}%
\pgfsetdash{}{0pt}%
\pgfpathmoveto{\pgfqpoint{1.753679in}{3.178384in}}%
\pgfpathlineto{\pgfqpoint{1.683761in}{3.217282in}}%
\pgfpathlineto{\pgfqpoint{1.753679in}{3.176185in}}%
\pgfpathlineto{\pgfqpoint{1.823598in}{3.137286in}}%
\pgfpathlineto{\pgfqpoint{1.753679in}{3.178384in}}%
\pgfpathclose%
\pgfusepath{fill}%
\end{pgfscope}%
\begin{pgfscope}%
\pgfpathrectangle{\pgfqpoint{0.637500in}{0.550000in}}{\pgfqpoint{3.850000in}{3.850000in}}%
\pgfusepath{clip}%
\pgfsetbuttcap%
\pgfsetroundjoin%
\definecolor{currentfill}{rgb}{0.897487,0.000000,0.000000}%
\pgfsetfillcolor{currentfill}%
\pgfsetfillopacity{0.800000}%
\pgfsetlinewidth{0.000000pt}%
\definecolor{currentstroke}{rgb}{0.000000,0.000000,0.000000}%
\pgfsetstrokecolor{currentstroke}%
\pgfsetdash{}{0pt}%
\pgfpathmoveto{\pgfqpoint{1.543923in}{3.299476in}}%
\pgfpathlineto{\pgfqpoint{1.474005in}{3.338375in}}%
\pgfpathlineto{\pgfqpoint{1.543923in}{3.297278in}}%
\pgfpathlineto{\pgfqpoint{1.613842in}{3.258379in}}%
\pgfpathlineto{\pgfqpoint{1.543923in}{3.299476in}}%
\pgfpathclose%
\pgfusepath{fill}%
\end{pgfscope}%
\begin{pgfscope}%
\pgfpathrectangle{\pgfqpoint{0.637500in}{0.550000in}}{\pgfqpoint{3.850000in}{3.850000in}}%
\pgfusepath{clip}%
\pgfsetbuttcap%
\pgfsetroundjoin%
\definecolor{currentfill}{rgb}{0.897487,0.000000,0.000000}%
\pgfsetfillcolor{currentfill}%
\pgfsetfillopacity{0.800000}%
\pgfsetlinewidth{0.000000pt}%
\definecolor{currentstroke}{rgb}{0.000000,0.000000,0.000000}%
\pgfsetstrokecolor{currentstroke}%
\pgfsetdash{}{0pt}%
\pgfpathmoveto{\pgfqpoint{2.662622in}{2.887350in}}%
\pgfpathlineto{\pgfqpoint{2.592703in}{2.846253in}}%
\pgfpathlineto{\pgfqpoint{2.662622in}{2.885151in}}%
\pgfpathlineto{\pgfqpoint{2.732540in}{2.926249in}}%
\pgfpathlineto{\pgfqpoint{2.662622in}{2.887350in}}%
\pgfpathclose%
\pgfusepath{fill}%
\end{pgfscope}%
\begin{pgfscope}%
\pgfpathrectangle{\pgfqpoint{0.637500in}{0.550000in}}{\pgfqpoint{3.850000in}{3.850000in}}%
\pgfusepath{clip}%
\pgfsetbuttcap%
\pgfsetroundjoin%
\definecolor{currentfill}{rgb}{0.897487,0.000000,0.000000}%
\pgfsetfillcolor{currentfill}%
\pgfsetfillopacity{0.800000}%
\pgfsetlinewidth{0.000000pt}%
\definecolor{currentstroke}{rgb}{0.000000,0.000000,0.000000}%
\pgfsetstrokecolor{currentstroke}%
\pgfsetdash{}{0pt}%
\pgfpathmoveto{\pgfqpoint{2.872378in}{3.008443in}}%
\pgfpathlineto{\pgfqpoint{2.802459in}{2.967346in}}%
\pgfpathlineto{\pgfqpoint{2.872378in}{3.006244in}}%
\pgfpathlineto{\pgfqpoint{2.942296in}{3.047341in}}%
\pgfpathlineto{\pgfqpoint{2.872378in}{3.008443in}}%
\pgfpathclose%
\pgfusepath{fill}%
\end{pgfscope}%
\begin{pgfscope}%
\pgfpathrectangle{\pgfqpoint{0.637500in}{0.550000in}}{\pgfqpoint{3.850000in}{3.850000in}}%
\pgfusepath{clip}%
\pgfsetbuttcap%
\pgfsetroundjoin%
\definecolor{currentfill}{rgb}{0.897487,0.000000,0.000000}%
\pgfsetfillcolor{currentfill}%
\pgfsetfillopacity{0.800000}%
\pgfsetlinewidth{0.000000pt}%
\definecolor{currentstroke}{rgb}{0.000000,0.000000,0.000000}%
\pgfsetstrokecolor{currentstroke}%
\pgfsetdash{}{0pt}%
\pgfpathmoveto{\pgfqpoint{3.082134in}{3.129536in}}%
\pgfpathlineto{\pgfqpoint{3.012215in}{3.088438in}}%
\pgfpathlineto{\pgfqpoint{3.082134in}{3.127337in}}%
\pgfpathlineto{\pgfqpoint{3.152052in}{3.168434in}}%
\pgfpathlineto{\pgfqpoint{3.082134in}{3.129536in}}%
\pgfpathclose%
\pgfusepath{fill}%
\end{pgfscope}%
\begin{pgfscope}%
\pgfpathrectangle{\pgfqpoint{0.637500in}{0.550000in}}{\pgfqpoint{3.850000in}{3.850000in}}%
\pgfusepath{clip}%
\pgfsetbuttcap%
\pgfsetroundjoin%
\definecolor{currentfill}{rgb}{0.897487,0.000000,0.000000}%
\pgfsetfillcolor{currentfill}%
\pgfsetfillopacity{0.800000}%
\pgfsetlinewidth{0.000000pt}%
\definecolor{currentstroke}{rgb}{0.000000,0.000000,0.000000}%
\pgfsetstrokecolor{currentstroke}%
\pgfsetdash{}{0pt}%
\pgfpathmoveto{\pgfqpoint{3.291890in}{3.250628in}}%
\pgfpathlineto{\pgfqpoint{3.221971in}{3.209531in}}%
\pgfpathlineto{\pgfqpoint{3.291890in}{3.248429in}}%
\pgfpathlineto{\pgfqpoint{3.361808in}{3.289527in}}%
\pgfpathlineto{\pgfqpoint{3.291890in}{3.250628in}}%
\pgfpathclose%
\pgfusepath{fill}%
\end{pgfscope}%
\begin{pgfscope}%
\pgfpathrectangle{\pgfqpoint{0.637500in}{0.550000in}}{\pgfqpoint{3.850000in}{3.850000in}}%
\pgfusepath{clip}%
\pgfsetbuttcap%
\pgfsetroundjoin%
\definecolor{currentfill}{rgb}{0.897487,0.000000,0.000000}%
\pgfsetfillcolor{currentfill}%
\pgfsetfillopacity{0.800000}%
\pgfsetlinewidth{0.000000pt}%
\definecolor{currentstroke}{rgb}{0.000000,0.000000,0.000000}%
\pgfsetstrokecolor{currentstroke}%
\pgfsetdash{}{0pt}%
\pgfpathmoveto{\pgfqpoint{2.243110in}{2.895101in}}%
\pgfpathlineto{\pgfqpoint{2.173191in}{2.933999in}}%
\pgfpathlineto{\pgfqpoint{2.243110in}{2.892902in}}%
\pgfpathlineto{\pgfqpoint{2.313028in}{2.854004in}}%
\pgfpathlineto{\pgfqpoint{2.243110in}{2.895101in}}%
\pgfpathclose%
\pgfusepath{fill}%
\end{pgfscope}%
\begin{pgfscope}%
\pgfpathrectangle{\pgfqpoint{0.637500in}{0.550000in}}{\pgfqpoint{3.850000in}{3.850000in}}%
\pgfusepath{clip}%
\pgfsetbuttcap%
\pgfsetroundjoin%
\definecolor{currentfill}{rgb}{0.897487,0.000000,0.000000}%
\pgfsetfillcolor{currentfill}%
\pgfsetfillopacity{0.800000}%
\pgfsetlinewidth{0.000000pt}%
\definecolor{currentstroke}{rgb}{0.000000,0.000000,0.000000}%
\pgfsetstrokecolor{currentstroke}%
\pgfsetdash{}{0pt}%
\pgfpathmoveto{\pgfqpoint{3.501646in}{3.371721in}}%
\pgfpathlineto{\pgfqpoint{3.431727in}{3.330624in}}%
\pgfpathlineto{\pgfqpoint{3.501646in}{3.369522in}}%
\pgfpathlineto{\pgfqpoint{3.571564in}{3.410619in}}%
\pgfpathlineto{\pgfqpoint{3.501646in}{3.371721in}}%
\pgfpathclose%
\pgfusepath{fill}%
\end{pgfscope}%
\begin{pgfscope}%
\pgfpathrectangle{\pgfqpoint{0.637500in}{0.550000in}}{\pgfqpoint{3.850000in}{3.850000in}}%
\pgfusepath{clip}%
\pgfsetbuttcap%
\pgfsetroundjoin%
\definecolor{currentfill}{rgb}{0.897487,0.000000,0.000000}%
\pgfsetfillcolor{currentfill}%
\pgfsetfillopacity{0.800000}%
\pgfsetlinewidth{0.000000pt}%
\definecolor{currentstroke}{rgb}{0.000000,0.000000,0.000000}%
\pgfsetstrokecolor{currentstroke}%
\pgfsetdash{}{0pt}%
\pgfpathmoveto{\pgfqpoint{2.033354in}{3.016194in}}%
\pgfpathlineto{\pgfqpoint{1.963435in}{3.055092in}}%
\pgfpathlineto{\pgfqpoint{2.033354in}{3.013995in}}%
\pgfpathlineto{\pgfqpoint{2.103272in}{2.975097in}}%
\pgfpathlineto{\pgfqpoint{2.033354in}{3.016194in}}%
\pgfpathclose%
\pgfusepath{fill}%
\end{pgfscope}%
\begin{pgfscope}%
\pgfpathrectangle{\pgfqpoint{0.637500in}{0.550000in}}{\pgfqpoint{3.850000in}{3.850000in}}%
\pgfusepath{clip}%
\pgfsetbuttcap%
\pgfsetroundjoin%
\definecolor{currentfill}{rgb}{0.897487,0.000000,0.000000}%
\pgfsetfillcolor{currentfill}%
\pgfsetfillopacity{0.800000}%
\pgfsetlinewidth{0.000000pt}%
\definecolor{currentstroke}{rgb}{0.000000,0.000000,0.000000}%
\pgfsetstrokecolor{currentstroke}%
\pgfsetdash{}{0pt}%
\pgfpathmoveto{\pgfqpoint{1.823598in}{3.137286in}}%
\pgfpathlineto{\pgfqpoint{1.753679in}{3.176185in}}%
\pgfpathlineto{\pgfqpoint{1.823598in}{3.135088in}}%
\pgfpathlineto{\pgfqpoint{1.893516in}{3.096189in}}%
\pgfpathlineto{\pgfqpoint{1.823598in}{3.137286in}}%
\pgfpathclose%
\pgfusepath{fill}%
\end{pgfscope}%
\begin{pgfscope}%
\pgfpathrectangle{\pgfqpoint{0.637500in}{0.550000in}}{\pgfqpoint{3.850000in}{3.850000in}}%
\pgfusepath{clip}%
\pgfsetbuttcap%
\pgfsetroundjoin%
\definecolor{currentfill}{rgb}{0.897487,0.000000,0.000000}%
\pgfsetfillcolor{currentfill}%
\pgfsetfillopacity{0.800000}%
\pgfsetlinewidth{0.000000pt}%
\definecolor{currentstroke}{rgb}{0.000000,0.000000,0.000000}%
\pgfsetstrokecolor{currentstroke}%
\pgfsetdash{}{0pt}%
\pgfpathmoveto{\pgfqpoint{1.613842in}{3.258379in}}%
\pgfpathlineto{\pgfqpoint{1.543923in}{3.297278in}}%
\pgfpathlineto{\pgfqpoint{1.613842in}{3.256180in}}%
\pgfpathlineto{\pgfqpoint{1.683761in}{3.217282in}}%
\pgfpathlineto{\pgfqpoint{1.613842in}{3.258379in}}%
\pgfpathclose%
\pgfusepath{fill}%
\end{pgfscope}%
\begin{pgfscope}%
\pgfpathrectangle{\pgfqpoint{0.637500in}{0.550000in}}{\pgfqpoint{3.850000in}{3.850000in}}%
\pgfusepath{clip}%
\pgfsetbuttcap%
\pgfsetroundjoin%
\definecolor{currentfill}{rgb}{0.898098,0.000000,0.000000}%
\pgfsetfillcolor{currentfill}%
\pgfsetfillopacity{0.800000}%
\pgfsetlinewidth{0.000000pt}%
\definecolor{currentstroke}{rgb}{0.000000,0.000000,0.000000}%
\pgfsetstrokecolor{currentstroke}%
\pgfsetdash{}{0pt}%
\pgfpathmoveto{\pgfqpoint{2.452866in}{2.774404in}}%
\pgfpathlineto{\pgfqpoint{2.382947in}{2.812907in}}%
\pgfpathlineto{\pgfqpoint{2.452866in}{2.772205in}}%
\pgfpathlineto{\pgfqpoint{2.522784in}{2.805156in}}%
\pgfpathlineto{\pgfqpoint{2.452866in}{2.774404in}}%
\pgfpathclose%
\pgfusepath{fill}%
\end{pgfscope}%
\begin{pgfscope}%
\pgfpathrectangle{\pgfqpoint{0.637500in}{0.550000in}}{\pgfqpoint{3.850000in}{3.850000in}}%
\pgfusepath{clip}%
\pgfsetbuttcap%
\pgfsetroundjoin%
\definecolor{currentfill}{rgb}{0.897487,0.000000,0.000000}%
\pgfsetfillcolor{currentfill}%
\pgfsetfillopacity{0.800000}%
\pgfsetlinewidth{0.000000pt}%
\definecolor{currentstroke}{rgb}{0.000000,0.000000,0.000000}%
\pgfsetstrokecolor{currentstroke}%
\pgfsetdash{}{0pt}%
\pgfpathmoveto{\pgfqpoint{2.592703in}{2.846253in}}%
\pgfpathlineto{\pgfqpoint{2.522784in}{2.805156in}}%
\pgfpathlineto{\pgfqpoint{2.592703in}{2.844054in}}%
\pgfpathlineto{\pgfqpoint{2.662622in}{2.885151in}}%
\pgfpathlineto{\pgfqpoint{2.592703in}{2.846253in}}%
\pgfpathclose%
\pgfusepath{fill}%
\end{pgfscope}%
\begin{pgfscope}%
\pgfpathrectangle{\pgfqpoint{0.637500in}{0.550000in}}{\pgfqpoint{3.850000in}{3.850000in}}%
\pgfusepath{clip}%
\pgfsetbuttcap%
\pgfsetroundjoin%
\definecolor{currentfill}{rgb}{0.897487,0.000000,0.000000}%
\pgfsetfillcolor{currentfill}%
\pgfsetfillopacity{0.800000}%
\pgfsetlinewidth{0.000000pt}%
\definecolor{currentstroke}{rgb}{0.000000,0.000000,0.000000}%
\pgfsetstrokecolor{currentstroke}%
\pgfsetdash{}{0pt}%
\pgfpathmoveto{\pgfqpoint{2.802459in}{2.967346in}}%
\pgfpathlineto{\pgfqpoint{2.732540in}{2.926249in}}%
\pgfpathlineto{\pgfqpoint{2.802459in}{2.965147in}}%
\pgfpathlineto{\pgfqpoint{2.872378in}{3.006244in}}%
\pgfpathlineto{\pgfqpoint{2.802459in}{2.967346in}}%
\pgfpathclose%
\pgfusepath{fill}%
\end{pgfscope}%
\begin{pgfscope}%
\pgfpathrectangle{\pgfqpoint{0.637500in}{0.550000in}}{\pgfqpoint{3.850000in}{3.850000in}}%
\pgfusepath{clip}%
\pgfsetbuttcap%
\pgfsetroundjoin%
\definecolor{currentfill}{rgb}{0.897487,0.000000,0.000000}%
\pgfsetfillcolor{currentfill}%
\pgfsetfillopacity{0.800000}%
\pgfsetlinewidth{0.000000pt}%
\definecolor{currentstroke}{rgb}{0.000000,0.000000,0.000000}%
\pgfsetstrokecolor{currentstroke}%
\pgfsetdash{}{0pt}%
\pgfpathmoveto{\pgfqpoint{3.012215in}{3.088438in}}%
\pgfpathlineto{\pgfqpoint{2.942296in}{3.047341in}}%
\pgfpathlineto{\pgfqpoint{3.012215in}{3.086240in}}%
\pgfpathlineto{\pgfqpoint{3.082134in}{3.127337in}}%
\pgfpathlineto{\pgfqpoint{3.012215in}{3.088438in}}%
\pgfpathclose%
\pgfusepath{fill}%
\end{pgfscope}%
\begin{pgfscope}%
\pgfpathrectangle{\pgfqpoint{0.637500in}{0.550000in}}{\pgfqpoint{3.850000in}{3.850000in}}%
\pgfusepath{clip}%
\pgfsetbuttcap%
\pgfsetroundjoin%
\definecolor{currentfill}{rgb}{0.897487,0.000000,0.000000}%
\pgfsetfillcolor{currentfill}%
\pgfsetfillopacity{0.800000}%
\pgfsetlinewidth{0.000000pt}%
\definecolor{currentstroke}{rgb}{0.000000,0.000000,0.000000}%
\pgfsetstrokecolor{currentstroke}%
\pgfsetdash{}{0pt}%
\pgfpathmoveto{\pgfqpoint{3.221971in}{3.209531in}}%
\pgfpathlineto{\pgfqpoint{3.152052in}{3.168434in}}%
\pgfpathlineto{\pgfqpoint{3.221971in}{3.207332in}}%
\pgfpathlineto{\pgfqpoint{3.291890in}{3.248429in}}%
\pgfpathlineto{\pgfqpoint{3.221971in}{3.209531in}}%
\pgfpathclose%
\pgfusepath{fill}%
\end{pgfscope}%
\begin{pgfscope}%
\pgfpathrectangle{\pgfqpoint{0.637500in}{0.550000in}}{\pgfqpoint{3.850000in}{3.850000in}}%
\pgfusepath{clip}%
\pgfsetbuttcap%
\pgfsetroundjoin%
\definecolor{currentfill}{rgb}{0.897487,0.000000,0.000000}%
\pgfsetfillcolor{currentfill}%
\pgfsetfillopacity{0.800000}%
\pgfsetlinewidth{0.000000pt}%
\definecolor{currentstroke}{rgb}{0.000000,0.000000,0.000000}%
\pgfsetstrokecolor{currentstroke}%
\pgfsetdash{}{0pt}%
\pgfpathmoveto{\pgfqpoint{2.313028in}{2.854004in}}%
\pgfpathlineto{\pgfqpoint{2.243110in}{2.892902in}}%
\pgfpathlineto{\pgfqpoint{2.313028in}{2.851805in}}%
\pgfpathlineto{\pgfqpoint{2.382947in}{2.812907in}}%
\pgfpathlineto{\pgfqpoint{2.313028in}{2.854004in}}%
\pgfpathclose%
\pgfusepath{fill}%
\end{pgfscope}%
\begin{pgfscope}%
\pgfpathrectangle{\pgfqpoint{0.637500in}{0.550000in}}{\pgfqpoint{3.850000in}{3.850000in}}%
\pgfusepath{clip}%
\pgfsetbuttcap%
\pgfsetroundjoin%
\definecolor{currentfill}{rgb}{0.897487,0.000000,0.000000}%
\pgfsetfillcolor{currentfill}%
\pgfsetfillopacity{0.800000}%
\pgfsetlinewidth{0.000000pt}%
\definecolor{currentstroke}{rgb}{0.000000,0.000000,0.000000}%
\pgfsetstrokecolor{currentstroke}%
\pgfsetdash{}{0pt}%
\pgfpathmoveto{\pgfqpoint{3.431727in}{3.330624in}}%
\pgfpathlineto{\pgfqpoint{3.361808in}{3.289527in}}%
\pgfpathlineto{\pgfqpoint{3.431727in}{3.328425in}}%
\pgfpathlineto{\pgfqpoint{3.501646in}{3.369522in}}%
\pgfpathlineto{\pgfqpoint{3.431727in}{3.330624in}}%
\pgfpathclose%
\pgfusepath{fill}%
\end{pgfscope}%
\begin{pgfscope}%
\pgfpathrectangle{\pgfqpoint{0.637500in}{0.550000in}}{\pgfqpoint{3.850000in}{3.850000in}}%
\pgfusepath{clip}%
\pgfsetbuttcap%
\pgfsetroundjoin%
\definecolor{currentfill}{rgb}{0.897487,0.000000,0.000000}%
\pgfsetfillcolor{currentfill}%
\pgfsetfillopacity{0.800000}%
\pgfsetlinewidth{0.000000pt}%
\definecolor{currentstroke}{rgb}{0.000000,0.000000,0.000000}%
\pgfsetstrokecolor{currentstroke}%
\pgfsetdash{}{0pt}%
\pgfpathmoveto{\pgfqpoint{2.103272in}{2.975097in}}%
\pgfpathlineto{\pgfqpoint{2.033354in}{3.013995in}}%
\pgfpathlineto{\pgfqpoint{2.103272in}{2.972898in}}%
\pgfpathlineto{\pgfqpoint{2.173191in}{2.933999in}}%
\pgfpathlineto{\pgfqpoint{2.103272in}{2.975097in}}%
\pgfpathclose%
\pgfusepath{fill}%
\end{pgfscope}%
\begin{pgfscope}%
\pgfpathrectangle{\pgfqpoint{0.637500in}{0.550000in}}{\pgfqpoint{3.850000in}{3.850000in}}%
\pgfusepath{clip}%
\pgfsetbuttcap%
\pgfsetroundjoin%
\definecolor{currentfill}{rgb}{0.897487,0.000000,0.000000}%
\pgfsetfillcolor{currentfill}%
\pgfsetfillopacity{0.800000}%
\pgfsetlinewidth{0.000000pt}%
\definecolor{currentstroke}{rgb}{0.000000,0.000000,0.000000}%
\pgfsetstrokecolor{currentstroke}%
\pgfsetdash{}{0pt}%
\pgfpathmoveto{\pgfqpoint{1.893516in}{3.096189in}}%
\pgfpathlineto{\pgfqpoint{1.823598in}{3.135088in}}%
\pgfpathlineto{\pgfqpoint{1.893516in}{3.093991in}}%
\pgfpathlineto{\pgfqpoint{1.963435in}{3.055092in}}%
\pgfpathlineto{\pgfqpoint{1.893516in}{3.096189in}}%
\pgfpathclose%
\pgfusepath{fill}%
\end{pgfscope}%
\begin{pgfscope}%
\pgfpathrectangle{\pgfqpoint{0.637500in}{0.550000in}}{\pgfqpoint{3.850000in}{3.850000in}}%
\pgfusepath{clip}%
\pgfsetbuttcap%
\pgfsetroundjoin%
\definecolor{currentfill}{rgb}{0.897487,0.000000,0.000000}%
\pgfsetfillcolor{currentfill}%
\pgfsetfillopacity{0.800000}%
\pgfsetlinewidth{0.000000pt}%
\definecolor{currentstroke}{rgb}{0.000000,0.000000,0.000000}%
\pgfsetstrokecolor{currentstroke}%
\pgfsetdash{}{0pt}%
\pgfpathmoveto{\pgfqpoint{1.683761in}{3.217282in}}%
\pgfpathlineto{\pgfqpoint{1.613842in}{3.256180in}}%
\pgfpathlineto{\pgfqpoint{1.683761in}{3.215083in}}%
\pgfpathlineto{\pgfqpoint{1.753679in}{3.176185in}}%
\pgfpathlineto{\pgfqpoint{1.683761in}{3.217282in}}%
\pgfpathclose%
\pgfusepath{fill}%
\end{pgfscope}%
\begin{pgfscope}%
\pgfpathrectangle{\pgfqpoint{0.637500in}{0.550000in}}{\pgfqpoint{3.850000in}{3.850000in}}%
\pgfusepath{clip}%
\pgfsetbuttcap%
\pgfsetroundjoin%
\definecolor{currentfill}{rgb}{0.909646,0.000000,0.000000}%
\pgfsetfillcolor{currentfill}%
\pgfsetfillopacity{0.800000}%
\pgfsetlinewidth{0.000000pt}%
\definecolor{currentstroke}{rgb}{0.000000,0.000000,0.000000}%
\pgfsetstrokecolor{currentstroke}%
\pgfsetdash{}{0pt}%
\pgfpathmoveto{\pgfqpoint{2.522784in}{2.805156in}}%
\pgfpathlineto{\pgfqpoint{2.452866in}{2.772205in}}%
\pgfpathlineto{\pgfqpoint{2.522784in}{2.802957in}}%
\pgfpathlineto{\pgfqpoint{2.592703in}{2.844054in}}%
\pgfpathlineto{\pgfqpoint{2.522784in}{2.805156in}}%
\pgfpathclose%
\pgfusepath{fill}%
\end{pgfscope}%
\begin{pgfscope}%
\pgfpathrectangle{\pgfqpoint{0.637500in}{0.550000in}}{\pgfqpoint{3.850000in}{3.850000in}}%
\pgfusepath{clip}%
\pgfsetbuttcap%
\pgfsetroundjoin%
\definecolor{currentfill}{rgb}{0.897487,0.000000,0.000000}%
\pgfsetfillcolor{currentfill}%
\pgfsetfillopacity{0.800000}%
\pgfsetlinewidth{0.000000pt}%
\definecolor{currentstroke}{rgb}{0.000000,0.000000,0.000000}%
\pgfsetstrokecolor{currentstroke}%
\pgfsetdash{}{0pt}%
\pgfpathmoveto{\pgfqpoint{2.732540in}{2.926249in}}%
\pgfpathlineto{\pgfqpoint{2.662622in}{2.885151in}}%
\pgfpathlineto{\pgfqpoint{2.732540in}{2.924050in}}%
\pgfpathlineto{\pgfqpoint{2.802459in}{2.965147in}}%
\pgfpathlineto{\pgfqpoint{2.732540in}{2.926249in}}%
\pgfpathclose%
\pgfusepath{fill}%
\end{pgfscope}%
\begin{pgfscope}%
\pgfpathrectangle{\pgfqpoint{0.637500in}{0.550000in}}{\pgfqpoint{3.850000in}{3.850000in}}%
\pgfusepath{clip}%
\pgfsetbuttcap%
\pgfsetroundjoin%
\definecolor{currentfill}{rgb}{0.897487,0.000000,0.000000}%
\pgfsetfillcolor{currentfill}%
\pgfsetfillopacity{0.800000}%
\pgfsetlinewidth{0.000000pt}%
\definecolor{currentstroke}{rgb}{0.000000,0.000000,0.000000}%
\pgfsetstrokecolor{currentstroke}%
\pgfsetdash{}{0pt}%
\pgfpathmoveto{\pgfqpoint{2.942296in}{3.047341in}}%
\pgfpathlineto{\pgfqpoint{2.872378in}{3.006244in}}%
\pgfpathlineto{\pgfqpoint{2.942296in}{3.045142in}}%
\pgfpathlineto{\pgfqpoint{3.012215in}{3.086240in}}%
\pgfpathlineto{\pgfqpoint{2.942296in}{3.047341in}}%
\pgfpathclose%
\pgfusepath{fill}%
\end{pgfscope}%
\begin{pgfscope}%
\pgfpathrectangle{\pgfqpoint{0.637500in}{0.550000in}}{\pgfqpoint{3.850000in}{3.850000in}}%
\pgfusepath{clip}%
\pgfsetbuttcap%
\pgfsetroundjoin%
\definecolor{currentfill}{rgb}{0.897487,0.000000,0.000000}%
\pgfsetfillcolor{currentfill}%
\pgfsetfillopacity{0.800000}%
\pgfsetlinewidth{0.000000pt}%
\definecolor{currentstroke}{rgb}{0.000000,0.000000,0.000000}%
\pgfsetstrokecolor{currentstroke}%
\pgfsetdash{}{0pt}%
\pgfpathmoveto{\pgfqpoint{3.152052in}{3.168434in}}%
\pgfpathlineto{\pgfqpoint{3.082134in}{3.127337in}}%
\pgfpathlineto{\pgfqpoint{3.152052in}{3.166235in}}%
\pgfpathlineto{\pgfqpoint{3.221971in}{3.207332in}}%
\pgfpathlineto{\pgfqpoint{3.152052in}{3.168434in}}%
\pgfpathclose%
\pgfusepath{fill}%
\end{pgfscope}%
\begin{pgfscope}%
\pgfpathrectangle{\pgfqpoint{0.637500in}{0.550000in}}{\pgfqpoint{3.850000in}{3.850000in}}%
\pgfusepath{clip}%
\pgfsetbuttcap%
\pgfsetroundjoin%
\definecolor{currentfill}{rgb}{0.897487,0.000000,0.000000}%
\pgfsetfillcolor{currentfill}%
\pgfsetfillopacity{0.800000}%
\pgfsetlinewidth{0.000000pt}%
\definecolor{currentstroke}{rgb}{0.000000,0.000000,0.000000}%
\pgfsetstrokecolor{currentstroke}%
\pgfsetdash{}{0pt}%
\pgfpathmoveto{\pgfqpoint{2.382947in}{2.812907in}}%
\pgfpathlineto{\pgfqpoint{2.313028in}{2.851805in}}%
\pgfpathlineto{\pgfqpoint{2.382947in}{2.810708in}}%
\pgfpathlineto{\pgfqpoint{2.452866in}{2.772205in}}%
\pgfpathlineto{\pgfqpoint{2.382947in}{2.812907in}}%
\pgfpathclose%
\pgfusepath{fill}%
\end{pgfscope}%
\begin{pgfscope}%
\pgfpathrectangle{\pgfqpoint{0.637500in}{0.550000in}}{\pgfqpoint{3.850000in}{3.850000in}}%
\pgfusepath{clip}%
\pgfsetbuttcap%
\pgfsetroundjoin%
\definecolor{currentfill}{rgb}{0.897487,0.000000,0.000000}%
\pgfsetfillcolor{currentfill}%
\pgfsetfillopacity{0.800000}%
\pgfsetlinewidth{0.000000pt}%
\definecolor{currentstroke}{rgb}{0.000000,0.000000,0.000000}%
\pgfsetstrokecolor{currentstroke}%
\pgfsetdash{}{0pt}%
\pgfpathmoveto{\pgfqpoint{3.361808in}{3.289527in}}%
\pgfpathlineto{\pgfqpoint{3.291890in}{3.248429in}}%
\pgfpathlineto{\pgfqpoint{3.361808in}{3.287328in}}%
\pgfpathlineto{\pgfqpoint{3.431727in}{3.328425in}}%
\pgfpathlineto{\pgfqpoint{3.361808in}{3.289527in}}%
\pgfpathclose%
\pgfusepath{fill}%
\end{pgfscope}%
\begin{pgfscope}%
\pgfpathrectangle{\pgfqpoint{0.637500in}{0.550000in}}{\pgfqpoint{3.850000in}{3.850000in}}%
\pgfusepath{clip}%
\pgfsetbuttcap%
\pgfsetroundjoin%
\definecolor{currentfill}{rgb}{0.897487,0.000000,0.000000}%
\pgfsetfillcolor{currentfill}%
\pgfsetfillopacity{0.800000}%
\pgfsetlinewidth{0.000000pt}%
\definecolor{currentstroke}{rgb}{0.000000,0.000000,0.000000}%
\pgfsetstrokecolor{currentstroke}%
\pgfsetdash{}{0pt}%
\pgfpathmoveto{\pgfqpoint{2.173191in}{2.933999in}}%
\pgfpathlineto{\pgfqpoint{2.103272in}{2.972898in}}%
\pgfpathlineto{\pgfqpoint{2.173191in}{2.931801in}}%
\pgfpathlineto{\pgfqpoint{2.243110in}{2.892902in}}%
\pgfpathlineto{\pgfqpoint{2.173191in}{2.933999in}}%
\pgfpathclose%
\pgfusepath{fill}%
\end{pgfscope}%
\begin{pgfscope}%
\pgfpathrectangle{\pgfqpoint{0.637500in}{0.550000in}}{\pgfqpoint{3.850000in}{3.850000in}}%
\pgfusepath{clip}%
\pgfsetbuttcap%
\pgfsetroundjoin%
\definecolor{currentfill}{rgb}{0.897487,0.000000,0.000000}%
\pgfsetfillcolor{currentfill}%
\pgfsetfillopacity{0.800000}%
\pgfsetlinewidth{0.000000pt}%
\definecolor{currentstroke}{rgb}{0.000000,0.000000,0.000000}%
\pgfsetstrokecolor{currentstroke}%
\pgfsetdash{}{0pt}%
\pgfpathmoveto{\pgfqpoint{1.963435in}{3.055092in}}%
\pgfpathlineto{\pgfqpoint{1.893516in}{3.093991in}}%
\pgfpathlineto{\pgfqpoint{1.963435in}{3.052893in}}%
\pgfpathlineto{\pgfqpoint{2.033354in}{3.013995in}}%
\pgfpathlineto{\pgfqpoint{1.963435in}{3.055092in}}%
\pgfpathclose%
\pgfusepath{fill}%
\end{pgfscope}%
\begin{pgfscope}%
\pgfpathrectangle{\pgfqpoint{0.637500in}{0.550000in}}{\pgfqpoint{3.850000in}{3.850000in}}%
\pgfusepath{clip}%
\pgfsetbuttcap%
\pgfsetroundjoin%
\definecolor{currentfill}{rgb}{0.897487,0.000000,0.000000}%
\pgfsetfillcolor{currentfill}%
\pgfsetfillopacity{0.800000}%
\pgfsetlinewidth{0.000000pt}%
\definecolor{currentstroke}{rgb}{0.000000,0.000000,0.000000}%
\pgfsetstrokecolor{currentstroke}%
\pgfsetdash{}{0pt}%
\pgfpathmoveto{\pgfqpoint{1.753679in}{3.176185in}}%
\pgfpathlineto{\pgfqpoint{1.683761in}{3.215083in}}%
\pgfpathlineto{\pgfqpoint{1.753679in}{3.173986in}}%
\pgfpathlineto{\pgfqpoint{1.823598in}{3.135088in}}%
\pgfpathlineto{\pgfqpoint{1.753679in}{3.176185in}}%
\pgfpathclose%
\pgfusepath{fill}%
\end{pgfscope}%
\begin{pgfscope}%
\pgfpathrectangle{\pgfqpoint{0.637500in}{0.550000in}}{\pgfqpoint{3.850000in}{3.850000in}}%
\pgfusepath{clip}%
\pgfsetbuttcap%
\pgfsetroundjoin%
\definecolor{currentfill}{rgb}{0.897487,0.000000,0.000000}%
\pgfsetfillcolor{currentfill}%
\pgfsetfillopacity{0.800000}%
\pgfsetlinewidth{0.000000pt}%
\definecolor{currentstroke}{rgb}{0.000000,0.000000,0.000000}%
\pgfsetstrokecolor{currentstroke}%
\pgfsetdash{}{0pt}%
\pgfpathmoveto{\pgfqpoint{2.662622in}{2.885151in}}%
\pgfpathlineto{\pgfqpoint{2.592703in}{2.844054in}}%
\pgfpathlineto{\pgfqpoint{2.662622in}{2.882953in}}%
\pgfpathlineto{\pgfqpoint{2.732540in}{2.924050in}}%
\pgfpathlineto{\pgfqpoint{2.662622in}{2.885151in}}%
\pgfpathclose%
\pgfusepath{fill}%
\end{pgfscope}%
\begin{pgfscope}%
\pgfpathrectangle{\pgfqpoint{0.637500in}{0.550000in}}{\pgfqpoint{3.850000in}{3.850000in}}%
\pgfusepath{clip}%
\pgfsetbuttcap%
\pgfsetroundjoin%
\definecolor{currentfill}{rgb}{0.897487,0.000000,0.000000}%
\pgfsetfillcolor{currentfill}%
\pgfsetfillopacity{0.800000}%
\pgfsetlinewidth{0.000000pt}%
\definecolor{currentstroke}{rgb}{0.000000,0.000000,0.000000}%
\pgfsetstrokecolor{currentstroke}%
\pgfsetdash{}{0pt}%
\pgfpathmoveto{\pgfqpoint{2.872378in}{3.006244in}}%
\pgfpathlineto{\pgfqpoint{2.802459in}{2.965147in}}%
\pgfpathlineto{\pgfqpoint{2.872378in}{3.004045in}}%
\pgfpathlineto{\pgfqpoint{2.942296in}{3.045142in}}%
\pgfpathlineto{\pgfqpoint{2.872378in}{3.006244in}}%
\pgfpathclose%
\pgfusepath{fill}%
\end{pgfscope}%
\begin{pgfscope}%
\pgfpathrectangle{\pgfqpoint{0.637500in}{0.550000in}}{\pgfqpoint{3.850000in}{3.850000in}}%
\pgfusepath{clip}%
\pgfsetbuttcap%
\pgfsetroundjoin%
\definecolor{currentfill}{rgb}{0.897487,0.000000,0.000000}%
\pgfsetfillcolor{currentfill}%
\pgfsetfillopacity{0.800000}%
\pgfsetlinewidth{0.000000pt}%
\definecolor{currentstroke}{rgb}{0.000000,0.000000,0.000000}%
\pgfsetstrokecolor{currentstroke}%
\pgfsetdash{}{0pt}%
\pgfpathmoveto{\pgfqpoint{3.082134in}{3.127337in}}%
\pgfpathlineto{\pgfqpoint{3.012215in}{3.086240in}}%
\pgfpathlineto{\pgfqpoint{3.082134in}{3.125138in}}%
\pgfpathlineto{\pgfqpoint{3.152052in}{3.166235in}}%
\pgfpathlineto{\pgfqpoint{3.082134in}{3.127337in}}%
\pgfpathclose%
\pgfusepath{fill}%
\end{pgfscope}%
\begin{pgfscope}%
\pgfpathrectangle{\pgfqpoint{0.637500in}{0.550000in}}{\pgfqpoint{3.850000in}{3.850000in}}%
\pgfusepath{clip}%
\pgfsetbuttcap%
\pgfsetroundjoin%
\definecolor{currentfill}{rgb}{0.897487,0.000000,0.000000}%
\pgfsetfillcolor{currentfill}%
\pgfsetfillopacity{0.800000}%
\pgfsetlinewidth{0.000000pt}%
\definecolor{currentstroke}{rgb}{0.000000,0.000000,0.000000}%
\pgfsetstrokecolor{currentstroke}%
\pgfsetdash{}{0pt}%
\pgfpathmoveto{\pgfqpoint{3.291890in}{3.248429in}}%
\pgfpathlineto{\pgfqpoint{3.221971in}{3.207332in}}%
\pgfpathlineto{\pgfqpoint{3.291890in}{3.246231in}}%
\pgfpathlineto{\pgfqpoint{3.361808in}{3.287328in}}%
\pgfpathlineto{\pgfqpoint{3.291890in}{3.248429in}}%
\pgfpathclose%
\pgfusepath{fill}%
\end{pgfscope}%
\begin{pgfscope}%
\pgfpathrectangle{\pgfqpoint{0.637500in}{0.550000in}}{\pgfqpoint{3.850000in}{3.850000in}}%
\pgfusepath{clip}%
\pgfsetbuttcap%
\pgfsetroundjoin%
\definecolor{currentfill}{rgb}{0.897487,0.000000,0.000000}%
\pgfsetfillcolor{currentfill}%
\pgfsetfillopacity{0.800000}%
\pgfsetlinewidth{0.000000pt}%
\definecolor{currentstroke}{rgb}{0.000000,0.000000,0.000000}%
\pgfsetstrokecolor{currentstroke}%
\pgfsetdash{}{0pt}%
\pgfpathmoveto{\pgfqpoint{2.243110in}{2.892902in}}%
\pgfpathlineto{\pgfqpoint{2.173191in}{2.931801in}}%
\pgfpathlineto{\pgfqpoint{2.243110in}{2.890703in}}%
\pgfpathlineto{\pgfqpoint{2.313028in}{2.851805in}}%
\pgfpathlineto{\pgfqpoint{2.243110in}{2.892902in}}%
\pgfpathclose%
\pgfusepath{fill}%
\end{pgfscope}%
\begin{pgfscope}%
\pgfpathrectangle{\pgfqpoint{0.637500in}{0.550000in}}{\pgfqpoint{3.850000in}{3.850000in}}%
\pgfusepath{clip}%
\pgfsetbuttcap%
\pgfsetroundjoin%
\definecolor{currentfill}{rgb}{0.897487,0.000000,0.000000}%
\pgfsetfillcolor{currentfill}%
\pgfsetfillopacity{0.800000}%
\pgfsetlinewidth{0.000000pt}%
\definecolor{currentstroke}{rgb}{0.000000,0.000000,0.000000}%
\pgfsetstrokecolor{currentstroke}%
\pgfsetdash{}{0pt}%
\pgfpathmoveto{\pgfqpoint{2.033354in}{3.013995in}}%
\pgfpathlineto{\pgfqpoint{1.963435in}{3.052893in}}%
\pgfpathlineto{\pgfqpoint{2.033354in}{3.011796in}}%
\pgfpathlineto{\pgfqpoint{2.103272in}{2.972898in}}%
\pgfpathlineto{\pgfqpoint{2.033354in}{3.013995in}}%
\pgfpathclose%
\pgfusepath{fill}%
\end{pgfscope}%
\begin{pgfscope}%
\pgfpathrectangle{\pgfqpoint{0.637500in}{0.550000in}}{\pgfqpoint{3.850000in}{3.850000in}}%
\pgfusepath{clip}%
\pgfsetbuttcap%
\pgfsetroundjoin%
\definecolor{currentfill}{rgb}{0.897487,0.000000,0.000000}%
\pgfsetfillcolor{currentfill}%
\pgfsetfillopacity{0.800000}%
\pgfsetlinewidth{0.000000pt}%
\definecolor{currentstroke}{rgb}{0.000000,0.000000,0.000000}%
\pgfsetstrokecolor{currentstroke}%
\pgfsetdash{}{0pt}%
\pgfpathmoveto{\pgfqpoint{1.823598in}{3.135088in}}%
\pgfpathlineto{\pgfqpoint{1.753679in}{3.173986in}}%
\pgfpathlineto{\pgfqpoint{1.823598in}{3.132889in}}%
\pgfpathlineto{\pgfqpoint{1.893516in}{3.093991in}}%
\pgfpathlineto{\pgfqpoint{1.823598in}{3.135088in}}%
\pgfpathclose%
\pgfusepath{fill}%
\end{pgfscope}%
\begin{pgfscope}%
\pgfpathrectangle{\pgfqpoint{0.637500in}{0.550000in}}{\pgfqpoint{3.850000in}{3.850000in}}%
\pgfusepath{clip}%
\pgfsetbuttcap%
\pgfsetroundjoin%
\definecolor{currentfill}{rgb}{0.898098,0.000000,0.000000}%
\pgfsetfillcolor{currentfill}%
\pgfsetfillopacity{0.800000}%
\pgfsetlinewidth{0.000000pt}%
\definecolor{currentstroke}{rgb}{0.000000,0.000000,0.000000}%
\pgfsetstrokecolor{currentstroke}%
\pgfsetdash{}{0pt}%
\pgfpathmoveto{\pgfqpoint{2.452866in}{2.772205in}}%
\pgfpathlineto{\pgfqpoint{2.382947in}{2.810708in}}%
\pgfpathlineto{\pgfqpoint{2.452866in}{2.770006in}}%
\pgfpathlineto{\pgfqpoint{2.522784in}{2.802957in}}%
\pgfpathlineto{\pgfqpoint{2.452866in}{2.772205in}}%
\pgfpathclose%
\pgfusepath{fill}%
\end{pgfscope}%
\begin{pgfscope}%
\pgfpathrectangle{\pgfqpoint{0.637500in}{0.550000in}}{\pgfqpoint{3.850000in}{3.850000in}}%
\pgfusepath{clip}%
\pgfsetbuttcap%
\pgfsetroundjoin%
\definecolor{currentfill}{rgb}{0.897487,0.000000,0.000000}%
\pgfsetfillcolor{currentfill}%
\pgfsetfillopacity{0.800000}%
\pgfsetlinewidth{0.000000pt}%
\definecolor{currentstroke}{rgb}{0.000000,0.000000,0.000000}%
\pgfsetstrokecolor{currentstroke}%
\pgfsetdash{}{0pt}%
\pgfpathmoveto{\pgfqpoint{2.592703in}{2.844054in}}%
\pgfpathlineto{\pgfqpoint{2.522784in}{2.802957in}}%
\pgfpathlineto{\pgfqpoint{2.592703in}{2.841855in}}%
\pgfpathlineto{\pgfqpoint{2.662622in}{2.882953in}}%
\pgfpathlineto{\pgfqpoint{2.592703in}{2.844054in}}%
\pgfpathclose%
\pgfusepath{fill}%
\end{pgfscope}%
\begin{pgfscope}%
\pgfpathrectangle{\pgfqpoint{0.637500in}{0.550000in}}{\pgfqpoint{3.850000in}{3.850000in}}%
\pgfusepath{clip}%
\pgfsetbuttcap%
\pgfsetroundjoin%
\definecolor{currentfill}{rgb}{0.897487,0.000000,0.000000}%
\pgfsetfillcolor{currentfill}%
\pgfsetfillopacity{0.800000}%
\pgfsetlinewidth{0.000000pt}%
\definecolor{currentstroke}{rgb}{0.000000,0.000000,0.000000}%
\pgfsetstrokecolor{currentstroke}%
\pgfsetdash{}{0pt}%
\pgfpathmoveto{\pgfqpoint{2.802459in}{2.965147in}}%
\pgfpathlineto{\pgfqpoint{2.732540in}{2.924050in}}%
\pgfpathlineto{\pgfqpoint{2.802459in}{2.962948in}}%
\pgfpathlineto{\pgfqpoint{2.872378in}{3.004045in}}%
\pgfpathlineto{\pgfqpoint{2.802459in}{2.965147in}}%
\pgfpathclose%
\pgfusepath{fill}%
\end{pgfscope}%
\begin{pgfscope}%
\pgfpathrectangle{\pgfqpoint{0.637500in}{0.550000in}}{\pgfqpoint{3.850000in}{3.850000in}}%
\pgfusepath{clip}%
\pgfsetbuttcap%
\pgfsetroundjoin%
\definecolor{currentfill}{rgb}{0.897487,0.000000,0.000000}%
\pgfsetfillcolor{currentfill}%
\pgfsetfillopacity{0.800000}%
\pgfsetlinewidth{0.000000pt}%
\definecolor{currentstroke}{rgb}{0.000000,0.000000,0.000000}%
\pgfsetstrokecolor{currentstroke}%
\pgfsetdash{}{0pt}%
\pgfpathmoveto{\pgfqpoint{3.012215in}{3.086240in}}%
\pgfpathlineto{\pgfqpoint{2.942296in}{3.045142in}}%
\pgfpathlineto{\pgfqpoint{3.012215in}{3.084041in}}%
\pgfpathlineto{\pgfqpoint{3.082134in}{3.125138in}}%
\pgfpathlineto{\pgfqpoint{3.012215in}{3.086240in}}%
\pgfpathclose%
\pgfusepath{fill}%
\end{pgfscope}%
\begin{pgfscope}%
\pgfpathrectangle{\pgfqpoint{0.637500in}{0.550000in}}{\pgfqpoint{3.850000in}{3.850000in}}%
\pgfusepath{clip}%
\pgfsetbuttcap%
\pgfsetroundjoin%
\definecolor{currentfill}{rgb}{0.897487,0.000000,0.000000}%
\pgfsetfillcolor{currentfill}%
\pgfsetfillopacity{0.800000}%
\pgfsetlinewidth{0.000000pt}%
\definecolor{currentstroke}{rgb}{0.000000,0.000000,0.000000}%
\pgfsetstrokecolor{currentstroke}%
\pgfsetdash{}{0pt}%
\pgfpathmoveto{\pgfqpoint{3.221971in}{3.207332in}}%
\pgfpathlineto{\pgfqpoint{3.152052in}{3.166235in}}%
\pgfpathlineto{\pgfqpoint{3.221971in}{3.205134in}}%
\pgfpathlineto{\pgfqpoint{3.291890in}{3.246231in}}%
\pgfpathlineto{\pgfqpoint{3.221971in}{3.207332in}}%
\pgfpathclose%
\pgfusepath{fill}%
\end{pgfscope}%
\begin{pgfscope}%
\pgfpathrectangle{\pgfqpoint{0.637500in}{0.550000in}}{\pgfqpoint{3.850000in}{3.850000in}}%
\pgfusepath{clip}%
\pgfsetbuttcap%
\pgfsetroundjoin%
\definecolor{currentfill}{rgb}{0.897487,0.000000,0.000000}%
\pgfsetfillcolor{currentfill}%
\pgfsetfillopacity{0.800000}%
\pgfsetlinewidth{0.000000pt}%
\definecolor{currentstroke}{rgb}{0.000000,0.000000,0.000000}%
\pgfsetstrokecolor{currentstroke}%
\pgfsetdash{}{0pt}%
\pgfpathmoveto{\pgfqpoint{2.313028in}{2.851805in}}%
\pgfpathlineto{\pgfqpoint{2.243110in}{2.890703in}}%
\pgfpathlineto{\pgfqpoint{2.313028in}{2.849606in}}%
\pgfpathlineto{\pgfqpoint{2.382947in}{2.810708in}}%
\pgfpathlineto{\pgfqpoint{2.313028in}{2.851805in}}%
\pgfpathclose%
\pgfusepath{fill}%
\end{pgfscope}%
\begin{pgfscope}%
\pgfpathrectangle{\pgfqpoint{0.637500in}{0.550000in}}{\pgfqpoint{3.850000in}{3.850000in}}%
\pgfusepath{clip}%
\pgfsetbuttcap%
\pgfsetroundjoin%
\definecolor{currentfill}{rgb}{0.897487,0.000000,0.000000}%
\pgfsetfillcolor{currentfill}%
\pgfsetfillopacity{0.800000}%
\pgfsetlinewidth{0.000000pt}%
\definecolor{currentstroke}{rgb}{0.000000,0.000000,0.000000}%
\pgfsetstrokecolor{currentstroke}%
\pgfsetdash{}{0pt}%
\pgfpathmoveto{\pgfqpoint{2.103272in}{2.972898in}}%
\pgfpathlineto{\pgfqpoint{2.033354in}{3.011796in}}%
\pgfpathlineto{\pgfqpoint{2.103272in}{2.970699in}}%
\pgfpathlineto{\pgfqpoint{2.173191in}{2.931801in}}%
\pgfpathlineto{\pgfqpoint{2.103272in}{2.972898in}}%
\pgfpathclose%
\pgfusepath{fill}%
\end{pgfscope}%
\begin{pgfscope}%
\pgfpathrectangle{\pgfqpoint{0.637500in}{0.550000in}}{\pgfqpoint{3.850000in}{3.850000in}}%
\pgfusepath{clip}%
\pgfsetbuttcap%
\pgfsetroundjoin%
\definecolor{currentfill}{rgb}{0.897487,0.000000,0.000000}%
\pgfsetfillcolor{currentfill}%
\pgfsetfillopacity{0.800000}%
\pgfsetlinewidth{0.000000pt}%
\definecolor{currentstroke}{rgb}{0.000000,0.000000,0.000000}%
\pgfsetstrokecolor{currentstroke}%
\pgfsetdash{}{0pt}%
\pgfpathmoveto{\pgfqpoint{1.893516in}{3.093991in}}%
\pgfpathlineto{\pgfqpoint{1.823598in}{3.132889in}}%
\pgfpathlineto{\pgfqpoint{1.893516in}{3.091792in}}%
\pgfpathlineto{\pgfqpoint{1.963435in}{3.052893in}}%
\pgfpathlineto{\pgfqpoint{1.893516in}{3.093991in}}%
\pgfpathclose%
\pgfusepath{fill}%
\end{pgfscope}%
\begin{pgfscope}%
\pgfpathrectangle{\pgfqpoint{0.637500in}{0.550000in}}{\pgfqpoint{3.850000in}{3.850000in}}%
\pgfusepath{clip}%
\pgfsetbuttcap%
\pgfsetroundjoin%
\definecolor{currentfill}{rgb}{0.909646,0.000000,0.000000}%
\pgfsetfillcolor{currentfill}%
\pgfsetfillopacity{0.800000}%
\pgfsetlinewidth{0.000000pt}%
\definecolor{currentstroke}{rgb}{0.000000,0.000000,0.000000}%
\pgfsetstrokecolor{currentstroke}%
\pgfsetdash{}{0pt}%
\pgfpathmoveto{\pgfqpoint{2.522784in}{2.802957in}}%
\pgfpathlineto{\pgfqpoint{2.452866in}{2.770006in}}%
\pgfpathlineto{\pgfqpoint{2.522784in}{2.800758in}}%
\pgfpathlineto{\pgfqpoint{2.592703in}{2.841855in}}%
\pgfpathlineto{\pgfqpoint{2.522784in}{2.802957in}}%
\pgfpathclose%
\pgfusepath{fill}%
\end{pgfscope}%
\begin{pgfscope}%
\pgfpathrectangle{\pgfqpoint{0.637500in}{0.550000in}}{\pgfqpoint{3.850000in}{3.850000in}}%
\pgfusepath{clip}%
\pgfsetbuttcap%
\pgfsetroundjoin%
\definecolor{currentfill}{rgb}{0.897487,0.000000,0.000000}%
\pgfsetfillcolor{currentfill}%
\pgfsetfillopacity{0.800000}%
\pgfsetlinewidth{0.000000pt}%
\definecolor{currentstroke}{rgb}{0.000000,0.000000,0.000000}%
\pgfsetstrokecolor{currentstroke}%
\pgfsetdash{}{0pt}%
\pgfpathmoveto{\pgfqpoint{2.732540in}{2.924050in}}%
\pgfpathlineto{\pgfqpoint{2.662622in}{2.882953in}}%
\pgfpathlineto{\pgfqpoint{2.732540in}{2.921851in}}%
\pgfpathlineto{\pgfqpoint{2.802459in}{2.962948in}}%
\pgfpathlineto{\pgfqpoint{2.732540in}{2.924050in}}%
\pgfpathclose%
\pgfusepath{fill}%
\end{pgfscope}%
\begin{pgfscope}%
\pgfpathrectangle{\pgfqpoint{0.637500in}{0.550000in}}{\pgfqpoint{3.850000in}{3.850000in}}%
\pgfusepath{clip}%
\pgfsetbuttcap%
\pgfsetroundjoin%
\definecolor{currentfill}{rgb}{0.897487,0.000000,0.000000}%
\pgfsetfillcolor{currentfill}%
\pgfsetfillopacity{0.800000}%
\pgfsetlinewidth{0.000000pt}%
\definecolor{currentstroke}{rgb}{0.000000,0.000000,0.000000}%
\pgfsetstrokecolor{currentstroke}%
\pgfsetdash{}{0pt}%
\pgfpathmoveto{\pgfqpoint{2.942296in}{3.045142in}}%
\pgfpathlineto{\pgfqpoint{2.872378in}{3.004045in}}%
\pgfpathlineto{\pgfqpoint{2.942296in}{3.042944in}}%
\pgfpathlineto{\pgfqpoint{3.012215in}{3.084041in}}%
\pgfpathlineto{\pgfqpoint{2.942296in}{3.045142in}}%
\pgfpathclose%
\pgfusepath{fill}%
\end{pgfscope}%
\begin{pgfscope}%
\pgfpathrectangle{\pgfqpoint{0.637500in}{0.550000in}}{\pgfqpoint{3.850000in}{3.850000in}}%
\pgfusepath{clip}%
\pgfsetbuttcap%
\pgfsetroundjoin%
\definecolor{currentfill}{rgb}{0.897487,0.000000,0.000000}%
\pgfsetfillcolor{currentfill}%
\pgfsetfillopacity{0.800000}%
\pgfsetlinewidth{0.000000pt}%
\definecolor{currentstroke}{rgb}{0.000000,0.000000,0.000000}%
\pgfsetstrokecolor{currentstroke}%
\pgfsetdash{}{0pt}%
\pgfpathmoveto{\pgfqpoint{3.152052in}{3.166235in}}%
\pgfpathlineto{\pgfqpoint{3.082134in}{3.125138in}}%
\pgfpathlineto{\pgfqpoint{3.152052in}{3.164036in}}%
\pgfpathlineto{\pgfqpoint{3.221971in}{3.205134in}}%
\pgfpathlineto{\pgfqpoint{3.152052in}{3.166235in}}%
\pgfpathclose%
\pgfusepath{fill}%
\end{pgfscope}%
\begin{pgfscope}%
\pgfpathrectangle{\pgfqpoint{0.637500in}{0.550000in}}{\pgfqpoint{3.850000in}{3.850000in}}%
\pgfusepath{clip}%
\pgfsetbuttcap%
\pgfsetroundjoin%
\definecolor{currentfill}{rgb}{0.897487,0.000000,0.000000}%
\pgfsetfillcolor{currentfill}%
\pgfsetfillopacity{0.800000}%
\pgfsetlinewidth{0.000000pt}%
\definecolor{currentstroke}{rgb}{0.000000,0.000000,0.000000}%
\pgfsetstrokecolor{currentstroke}%
\pgfsetdash{}{0pt}%
\pgfpathmoveto{\pgfqpoint{2.382947in}{2.810708in}}%
\pgfpathlineto{\pgfqpoint{2.313028in}{2.849606in}}%
\pgfpathlineto{\pgfqpoint{2.382947in}{2.808509in}}%
\pgfpathlineto{\pgfqpoint{2.452866in}{2.770006in}}%
\pgfpathlineto{\pgfqpoint{2.382947in}{2.810708in}}%
\pgfpathclose%
\pgfusepath{fill}%
\end{pgfscope}%
\begin{pgfscope}%
\pgfpathrectangle{\pgfqpoint{0.637500in}{0.550000in}}{\pgfqpoint{3.850000in}{3.850000in}}%
\pgfusepath{clip}%
\pgfsetbuttcap%
\pgfsetroundjoin%
\definecolor{currentfill}{rgb}{0.897487,0.000000,0.000000}%
\pgfsetfillcolor{currentfill}%
\pgfsetfillopacity{0.800000}%
\pgfsetlinewidth{0.000000pt}%
\definecolor{currentstroke}{rgb}{0.000000,0.000000,0.000000}%
\pgfsetstrokecolor{currentstroke}%
\pgfsetdash{}{0pt}%
\pgfpathmoveto{\pgfqpoint{2.173191in}{2.931801in}}%
\pgfpathlineto{\pgfqpoint{2.103272in}{2.970699in}}%
\pgfpathlineto{\pgfqpoint{2.173191in}{2.929602in}}%
\pgfpathlineto{\pgfqpoint{2.243110in}{2.890703in}}%
\pgfpathlineto{\pgfqpoint{2.173191in}{2.931801in}}%
\pgfpathclose%
\pgfusepath{fill}%
\end{pgfscope}%
\begin{pgfscope}%
\pgfpathrectangle{\pgfqpoint{0.637500in}{0.550000in}}{\pgfqpoint{3.850000in}{3.850000in}}%
\pgfusepath{clip}%
\pgfsetbuttcap%
\pgfsetroundjoin%
\definecolor{currentfill}{rgb}{0.897487,0.000000,0.000000}%
\pgfsetfillcolor{currentfill}%
\pgfsetfillopacity{0.800000}%
\pgfsetlinewidth{0.000000pt}%
\definecolor{currentstroke}{rgb}{0.000000,0.000000,0.000000}%
\pgfsetstrokecolor{currentstroke}%
\pgfsetdash{}{0pt}%
\pgfpathmoveto{\pgfqpoint{1.963435in}{3.052893in}}%
\pgfpathlineto{\pgfqpoint{1.893516in}{3.091792in}}%
\pgfpathlineto{\pgfqpoint{1.963435in}{3.050695in}}%
\pgfpathlineto{\pgfqpoint{2.033354in}{3.011796in}}%
\pgfpathlineto{\pgfqpoint{1.963435in}{3.052893in}}%
\pgfpathclose%
\pgfusepath{fill}%
\end{pgfscope}%
\begin{pgfscope}%
\pgfpathrectangle{\pgfqpoint{0.637500in}{0.550000in}}{\pgfqpoint{3.850000in}{3.850000in}}%
\pgfusepath{clip}%
\pgfsetbuttcap%
\pgfsetroundjoin%
\definecolor{currentfill}{rgb}{0.897487,0.000000,0.000000}%
\pgfsetfillcolor{currentfill}%
\pgfsetfillopacity{0.800000}%
\pgfsetlinewidth{0.000000pt}%
\definecolor{currentstroke}{rgb}{0.000000,0.000000,0.000000}%
\pgfsetstrokecolor{currentstroke}%
\pgfsetdash{}{0pt}%
\pgfpathmoveto{\pgfqpoint{2.662622in}{2.882953in}}%
\pgfpathlineto{\pgfqpoint{2.592703in}{2.841855in}}%
\pgfpathlineto{\pgfqpoint{2.662622in}{2.880754in}}%
\pgfpathlineto{\pgfqpoint{2.732540in}{2.921851in}}%
\pgfpathlineto{\pgfqpoint{2.662622in}{2.882953in}}%
\pgfpathclose%
\pgfusepath{fill}%
\end{pgfscope}%
\begin{pgfscope}%
\pgfpathrectangle{\pgfqpoint{0.637500in}{0.550000in}}{\pgfqpoint{3.850000in}{3.850000in}}%
\pgfusepath{clip}%
\pgfsetbuttcap%
\pgfsetroundjoin%
\definecolor{currentfill}{rgb}{0.897487,0.000000,0.000000}%
\pgfsetfillcolor{currentfill}%
\pgfsetfillopacity{0.800000}%
\pgfsetlinewidth{0.000000pt}%
\definecolor{currentstroke}{rgb}{0.000000,0.000000,0.000000}%
\pgfsetstrokecolor{currentstroke}%
\pgfsetdash{}{0pt}%
\pgfpathmoveto{\pgfqpoint{2.872378in}{3.004045in}}%
\pgfpathlineto{\pgfqpoint{2.802459in}{2.962948in}}%
\pgfpathlineto{\pgfqpoint{2.872378in}{3.001847in}}%
\pgfpathlineto{\pgfqpoint{2.942296in}{3.042944in}}%
\pgfpathlineto{\pgfqpoint{2.872378in}{3.004045in}}%
\pgfpathclose%
\pgfusepath{fill}%
\end{pgfscope}%
\begin{pgfscope}%
\pgfpathrectangle{\pgfqpoint{0.637500in}{0.550000in}}{\pgfqpoint{3.850000in}{3.850000in}}%
\pgfusepath{clip}%
\pgfsetbuttcap%
\pgfsetroundjoin%
\definecolor{currentfill}{rgb}{0.897487,0.000000,0.000000}%
\pgfsetfillcolor{currentfill}%
\pgfsetfillopacity{0.800000}%
\pgfsetlinewidth{0.000000pt}%
\definecolor{currentstroke}{rgb}{0.000000,0.000000,0.000000}%
\pgfsetstrokecolor{currentstroke}%
\pgfsetdash{}{0pt}%
\pgfpathmoveto{\pgfqpoint{3.082134in}{3.125138in}}%
\pgfpathlineto{\pgfqpoint{3.012215in}{3.084041in}}%
\pgfpathlineto{\pgfqpoint{3.082134in}{3.122939in}}%
\pgfpathlineto{\pgfqpoint{3.152052in}{3.164036in}}%
\pgfpathlineto{\pgfqpoint{3.082134in}{3.125138in}}%
\pgfpathclose%
\pgfusepath{fill}%
\end{pgfscope}%
\begin{pgfscope}%
\pgfpathrectangle{\pgfqpoint{0.637500in}{0.550000in}}{\pgfqpoint{3.850000in}{3.850000in}}%
\pgfusepath{clip}%
\pgfsetbuttcap%
\pgfsetroundjoin%
\definecolor{currentfill}{rgb}{0.897487,0.000000,0.000000}%
\pgfsetfillcolor{currentfill}%
\pgfsetfillopacity{0.800000}%
\pgfsetlinewidth{0.000000pt}%
\definecolor{currentstroke}{rgb}{0.000000,0.000000,0.000000}%
\pgfsetstrokecolor{currentstroke}%
\pgfsetdash{}{0pt}%
\pgfpathmoveto{\pgfqpoint{2.243110in}{2.890703in}}%
\pgfpathlineto{\pgfqpoint{2.173191in}{2.929602in}}%
\pgfpathlineto{\pgfqpoint{2.243110in}{2.888505in}}%
\pgfpathlineto{\pgfqpoint{2.313028in}{2.849606in}}%
\pgfpathlineto{\pgfqpoint{2.243110in}{2.890703in}}%
\pgfpathclose%
\pgfusepath{fill}%
\end{pgfscope}%
\begin{pgfscope}%
\pgfpathrectangle{\pgfqpoint{0.637500in}{0.550000in}}{\pgfqpoint{3.850000in}{3.850000in}}%
\pgfusepath{clip}%
\pgfsetbuttcap%
\pgfsetroundjoin%
\definecolor{currentfill}{rgb}{0.897487,0.000000,0.000000}%
\pgfsetfillcolor{currentfill}%
\pgfsetfillopacity{0.800000}%
\pgfsetlinewidth{0.000000pt}%
\definecolor{currentstroke}{rgb}{0.000000,0.000000,0.000000}%
\pgfsetstrokecolor{currentstroke}%
\pgfsetdash{}{0pt}%
\pgfpathmoveto{\pgfqpoint{2.033354in}{3.011796in}}%
\pgfpathlineto{\pgfqpoint{1.963435in}{3.050695in}}%
\pgfpathlineto{\pgfqpoint{2.033354in}{3.009597in}}%
\pgfpathlineto{\pgfqpoint{2.103272in}{2.970699in}}%
\pgfpathlineto{\pgfqpoint{2.033354in}{3.011796in}}%
\pgfpathclose%
\pgfusepath{fill}%
\end{pgfscope}%
\begin{pgfscope}%
\pgfpathrectangle{\pgfqpoint{0.637500in}{0.550000in}}{\pgfqpoint{3.850000in}{3.850000in}}%
\pgfusepath{clip}%
\pgfsetbuttcap%
\pgfsetroundjoin%
\definecolor{currentfill}{rgb}{0.898098,0.000000,0.000000}%
\pgfsetfillcolor{currentfill}%
\pgfsetfillopacity{0.800000}%
\pgfsetlinewidth{0.000000pt}%
\definecolor{currentstroke}{rgb}{0.000000,0.000000,0.000000}%
\pgfsetstrokecolor{currentstroke}%
\pgfsetdash{}{0pt}%
\pgfpathmoveto{\pgfqpoint{2.452866in}{2.770006in}}%
\pgfpathlineto{\pgfqpoint{2.382947in}{2.808509in}}%
\pgfpathlineto{\pgfqpoint{2.452866in}{2.767807in}}%
\pgfpathlineto{\pgfqpoint{2.522784in}{2.800758in}}%
\pgfpathlineto{\pgfqpoint{2.452866in}{2.770006in}}%
\pgfpathclose%
\pgfusepath{fill}%
\end{pgfscope}%
\begin{pgfscope}%
\pgfpathrectangle{\pgfqpoint{0.637500in}{0.550000in}}{\pgfqpoint{3.850000in}{3.850000in}}%
\pgfusepath{clip}%
\pgfsetbuttcap%
\pgfsetroundjoin%
\definecolor{currentfill}{rgb}{0.897487,0.000000,0.000000}%
\pgfsetfillcolor{currentfill}%
\pgfsetfillopacity{0.800000}%
\pgfsetlinewidth{0.000000pt}%
\definecolor{currentstroke}{rgb}{0.000000,0.000000,0.000000}%
\pgfsetstrokecolor{currentstroke}%
\pgfsetdash{}{0pt}%
\pgfpathmoveto{\pgfqpoint{2.592703in}{2.841855in}}%
\pgfpathlineto{\pgfqpoint{2.522784in}{2.800758in}}%
\pgfpathlineto{\pgfqpoint{2.592703in}{2.839657in}}%
\pgfpathlineto{\pgfqpoint{2.662622in}{2.880754in}}%
\pgfpathlineto{\pgfqpoint{2.592703in}{2.841855in}}%
\pgfpathclose%
\pgfusepath{fill}%
\end{pgfscope}%
\begin{pgfscope}%
\pgfpathrectangle{\pgfqpoint{0.637500in}{0.550000in}}{\pgfqpoint{3.850000in}{3.850000in}}%
\pgfusepath{clip}%
\pgfsetbuttcap%
\pgfsetroundjoin%
\definecolor{currentfill}{rgb}{0.897487,0.000000,0.000000}%
\pgfsetfillcolor{currentfill}%
\pgfsetfillopacity{0.800000}%
\pgfsetlinewidth{0.000000pt}%
\definecolor{currentstroke}{rgb}{0.000000,0.000000,0.000000}%
\pgfsetstrokecolor{currentstroke}%
\pgfsetdash{}{0pt}%
\pgfpathmoveto{\pgfqpoint{2.802459in}{2.962948in}}%
\pgfpathlineto{\pgfqpoint{2.732540in}{2.921851in}}%
\pgfpathlineto{\pgfqpoint{2.802459in}{2.960749in}}%
\pgfpathlineto{\pgfqpoint{2.872378in}{3.001847in}}%
\pgfpathlineto{\pgfqpoint{2.802459in}{2.962948in}}%
\pgfpathclose%
\pgfusepath{fill}%
\end{pgfscope}%
\begin{pgfscope}%
\pgfpathrectangle{\pgfqpoint{0.637500in}{0.550000in}}{\pgfqpoint{3.850000in}{3.850000in}}%
\pgfusepath{clip}%
\pgfsetbuttcap%
\pgfsetroundjoin%
\definecolor{currentfill}{rgb}{0.897487,0.000000,0.000000}%
\pgfsetfillcolor{currentfill}%
\pgfsetfillopacity{0.800000}%
\pgfsetlinewidth{0.000000pt}%
\definecolor{currentstroke}{rgb}{0.000000,0.000000,0.000000}%
\pgfsetstrokecolor{currentstroke}%
\pgfsetdash{}{0pt}%
\pgfpathmoveto{\pgfqpoint{3.012215in}{3.084041in}}%
\pgfpathlineto{\pgfqpoint{2.942296in}{3.042944in}}%
\pgfpathlineto{\pgfqpoint{3.012215in}{3.081842in}}%
\pgfpathlineto{\pgfqpoint{3.082134in}{3.122939in}}%
\pgfpathlineto{\pgfqpoint{3.012215in}{3.084041in}}%
\pgfpathclose%
\pgfusepath{fill}%
\end{pgfscope}%
\begin{pgfscope}%
\pgfpathrectangle{\pgfqpoint{0.637500in}{0.550000in}}{\pgfqpoint{3.850000in}{3.850000in}}%
\pgfusepath{clip}%
\pgfsetbuttcap%
\pgfsetroundjoin%
\definecolor{currentfill}{rgb}{0.897487,0.000000,0.000000}%
\pgfsetfillcolor{currentfill}%
\pgfsetfillopacity{0.800000}%
\pgfsetlinewidth{0.000000pt}%
\definecolor{currentstroke}{rgb}{0.000000,0.000000,0.000000}%
\pgfsetstrokecolor{currentstroke}%
\pgfsetdash{}{0pt}%
\pgfpathmoveto{\pgfqpoint{2.313028in}{2.849606in}}%
\pgfpathlineto{\pgfqpoint{2.243110in}{2.888505in}}%
\pgfpathlineto{\pgfqpoint{2.313028in}{2.847408in}}%
\pgfpathlineto{\pgfqpoint{2.382947in}{2.808509in}}%
\pgfpathlineto{\pgfqpoint{2.313028in}{2.849606in}}%
\pgfpathclose%
\pgfusepath{fill}%
\end{pgfscope}%
\begin{pgfscope}%
\pgfpathrectangle{\pgfqpoint{0.637500in}{0.550000in}}{\pgfqpoint{3.850000in}{3.850000in}}%
\pgfusepath{clip}%
\pgfsetbuttcap%
\pgfsetroundjoin%
\definecolor{currentfill}{rgb}{0.897487,0.000000,0.000000}%
\pgfsetfillcolor{currentfill}%
\pgfsetfillopacity{0.800000}%
\pgfsetlinewidth{0.000000pt}%
\definecolor{currentstroke}{rgb}{0.000000,0.000000,0.000000}%
\pgfsetstrokecolor{currentstroke}%
\pgfsetdash{}{0pt}%
\pgfpathmoveto{\pgfqpoint{2.103272in}{2.970699in}}%
\pgfpathlineto{\pgfqpoint{2.033354in}{3.009597in}}%
\pgfpathlineto{\pgfqpoint{2.103272in}{2.968500in}}%
\pgfpathlineto{\pgfqpoint{2.173191in}{2.929602in}}%
\pgfpathlineto{\pgfqpoint{2.103272in}{2.970699in}}%
\pgfpathclose%
\pgfusepath{fill}%
\end{pgfscope}%
\begin{pgfscope}%
\pgfpathrectangle{\pgfqpoint{0.637500in}{0.550000in}}{\pgfqpoint{3.850000in}{3.850000in}}%
\pgfusepath{clip}%
\pgfsetbuttcap%
\pgfsetroundjoin%
\definecolor{currentfill}{rgb}{0.909646,0.000000,0.000000}%
\pgfsetfillcolor{currentfill}%
\pgfsetfillopacity{0.800000}%
\pgfsetlinewidth{0.000000pt}%
\definecolor{currentstroke}{rgb}{0.000000,0.000000,0.000000}%
\pgfsetstrokecolor{currentstroke}%
\pgfsetdash{}{0pt}%
\pgfpathmoveto{\pgfqpoint{2.522784in}{2.800758in}}%
\pgfpathlineto{\pgfqpoint{2.452866in}{2.767807in}}%
\pgfpathlineto{\pgfqpoint{2.522784in}{2.798559in}}%
\pgfpathlineto{\pgfqpoint{2.592703in}{2.839657in}}%
\pgfpathlineto{\pgfqpoint{2.522784in}{2.800758in}}%
\pgfpathclose%
\pgfusepath{fill}%
\end{pgfscope}%
\begin{pgfscope}%
\pgfpathrectangle{\pgfqpoint{0.637500in}{0.550000in}}{\pgfqpoint{3.850000in}{3.850000in}}%
\pgfusepath{clip}%
\pgfsetbuttcap%
\pgfsetroundjoin%
\definecolor{currentfill}{rgb}{0.897487,0.000000,0.000000}%
\pgfsetfillcolor{currentfill}%
\pgfsetfillopacity{0.800000}%
\pgfsetlinewidth{0.000000pt}%
\definecolor{currentstroke}{rgb}{0.000000,0.000000,0.000000}%
\pgfsetstrokecolor{currentstroke}%
\pgfsetdash{}{0pt}%
\pgfpathmoveto{\pgfqpoint{2.732540in}{2.921851in}}%
\pgfpathlineto{\pgfqpoint{2.662622in}{2.880754in}}%
\pgfpathlineto{\pgfqpoint{2.732540in}{2.919652in}}%
\pgfpathlineto{\pgfqpoint{2.802459in}{2.960749in}}%
\pgfpathlineto{\pgfqpoint{2.732540in}{2.921851in}}%
\pgfpathclose%
\pgfusepath{fill}%
\end{pgfscope}%
\begin{pgfscope}%
\pgfpathrectangle{\pgfqpoint{0.637500in}{0.550000in}}{\pgfqpoint{3.850000in}{3.850000in}}%
\pgfusepath{clip}%
\pgfsetbuttcap%
\pgfsetroundjoin%
\definecolor{currentfill}{rgb}{0.897487,0.000000,0.000000}%
\pgfsetfillcolor{currentfill}%
\pgfsetfillopacity{0.800000}%
\pgfsetlinewidth{0.000000pt}%
\definecolor{currentstroke}{rgb}{0.000000,0.000000,0.000000}%
\pgfsetstrokecolor{currentstroke}%
\pgfsetdash{}{0pt}%
\pgfpathmoveto{\pgfqpoint{2.942296in}{3.042944in}}%
\pgfpathlineto{\pgfqpoint{2.872378in}{3.001847in}}%
\pgfpathlineto{\pgfqpoint{2.942296in}{3.040745in}}%
\pgfpathlineto{\pgfqpoint{3.012215in}{3.081842in}}%
\pgfpathlineto{\pgfqpoint{2.942296in}{3.042944in}}%
\pgfpathclose%
\pgfusepath{fill}%
\end{pgfscope}%
\begin{pgfscope}%
\pgfpathrectangle{\pgfqpoint{0.637500in}{0.550000in}}{\pgfqpoint{3.850000in}{3.850000in}}%
\pgfusepath{clip}%
\pgfsetbuttcap%
\pgfsetroundjoin%
\definecolor{currentfill}{rgb}{0.897487,0.000000,0.000000}%
\pgfsetfillcolor{currentfill}%
\pgfsetfillopacity{0.800000}%
\pgfsetlinewidth{0.000000pt}%
\definecolor{currentstroke}{rgb}{0.000000,0.000000,0.000000}%
\pgfsetstrokecolor{currentstroke}%
\pgfsetdash{}{0pt}%
\pgfpathmoveto{\pgfqpoint{2.382947in}{2.808509in}}%
\pgfpathlineto{\pgfqpoint{2.313028in}{2.847408in}}%
\pgfpathlineto{\pgfqpoint{2.382947in}{2.806310in}}%
\pgfpathlineto{\pgfqpoint{2.452866in}{2.767807in}}%
\pgfpathlineto{\pgfqpoint{2.382947in}{2.808509in}}%
\pgfpathclose%
\pgfusepath{fill}%
\end{pgfscope}%
\begin{pgfscope}%
\pgfpathrectangle{\pgfqpoint{0.637500in}{0.550000in}}{\pgfqpoint{3.850000in}{3.850000in}}%
\pgfusepath{clip}%
\pgfsetbuttcap%
\pgfsetroundjoin%
\definecolor{currentfill}{rgb}{0.897487,0.000000,0.000000}%
\pgfsetfillcolor{currentfill}%
\pgfsetfillopacity{0.800000}%
\pgfsetlinewidth{0.000000pt}%
\definecolor{currentstroke}{rgb}{0.000000,0.000000,0.000000}%
\pgfsetstrokecolor{currentstroke}%
\pgfsetdash{}{0pt}%
\pgfpathmoveto{\pgfqpoint{2.173191in}{2.929602in}}%
\pgfpathlineto{\pgfqpoint{2.103272in}{2.968500in}}%
\pgfpathlineto{\pgfqpoint{2.173191in}{2.927403in}}%
\pgfpathlineto{\pgfqpoint{2.243110in}{2.888505in}}%
\pgfpathlineto{\pgfqpoint{2.173191in}{2.929602in}}%
\pgfpathclose%
\pgfusepath{fill}%
\end{pgfscope}%
\begin{pgfscope}%
\pgfpathrectangle{\pgfqpoint{0.637500in}{0.550000in}}{\pgfqpoint{3.850000in}{3.850000in}}%
\pgfusepath{clip}%
\pgfsetbuttcap%
\pgfsetroundjoin%
\definecolor{currentfill}{rgb}{0.897487,0.000000,0.000000}%
\pgfsetfillcolor{currentfill}%
\pgfsetfillopacity{0.800000}%
\pgfsetlinewidth{0.000000pt}%
\definecolor{currentstroke}{rgb}{0.000000,0.000000,0.000000}%
\pgfsetstrokecolor{currentstroke}%
\pgfsetdash{}{0pt}%
\pgfpathmoveto{\pgfqpoint{2.662622in}{2.880754in}}%
\pgfpathlineto{\pgfqpoint{2.592703in}{2.839657in}}%
\pgfpathlineto{\pgfqpoint{2.662622in}{2.878555in}}%
\pgfpathlineto{\pgfqpoint{2.732540in}{2.919652in}}%
\pgfpathlineto{\pgfqpoint{2.662622in}{2.880754in}}%
\pgfpathclose%
\pgfusepath{fill}%
\end{pgfscope}%
\begin{pgfscope}%
\pgfpathrectangle{\pgfqpoint{0.637500in}{0.550000in}}{\pgfqpoint{3.850000in}{3.850000in}}%
\pgfusepath{clip}%
\pgfsetbuttcap%
\pgfsetroundjoin%
\definecolor{currentfill}{rgb}{0.897487,0.000000,0.000000}%
\pgfsetfillcolor{currentfill}%
\pgfsetfillopacity{0.800000}%
\pgfsetlinewidth{0.000000pt}%
\definecolor{currentstroke}{rgb}{0.000000,0.000000,0.000000}%
\pgfsetstrokecolor{currentstroke}%
\pgfsetdash{}{0pt}%
\pgfpathmoveto{\pgfqpoint{2.872378in}{3.001847in}}%
\pgfpathlineto{\pgfqpoint{2.802459in}{2.960749in}}%
\pgfpathlineto{\pgfqpoint{2.872378in}{2.999648in}}%
\pgfpathlineto{\pgfqpoint{2.942296in}{3.040745in}}%
\pgfpathlineto{\pgfqpoint{2.872378in}{3.001847in}}%
\pgfpathclose%
\pgfusepath{fill}%
\end{pgfscope}%
\begin{pgfscope}%
\pgfpathrectangle{\pgfqpoint{0.637500in}{0.550000in}}{\pgfqpoint{3.850000in}{3.850000in}}%
\pgfusepath{clip}%
\pgfsetbuttcap%
\pgfsetroundjoin%
\definecolor{currentfill}{rgb}{0.897487,0.000000,0.000000}%
\pgfsetfillcolor{currentfill}%
\pgfsetfillopacity{0.800000}%
\pgfsetlinewidth{0.000000pt}%
\definecolor{currentstroke}{rgb}{0.000000,0.000000,0.000000}%
\pgfsetstrokecolor{currentstroke}%
\pgfsetdash{}{0pt}%
\pgfpathmoveto{\pgfqpoint{2.243110in}{2.888505in}}%
\pgfpathlineto{\pgfqpoint{2.173191in}{2.927403in}}%
\pgfpathlineto{\pgfqpoint{2.243110in}{2.886306in}}%
\pgfpathlineto{\pgfqpoint{2.313028in}{2.847408in}}%
\pgfpathlineto{\pgfqpoint{2.243110in}{2.888505in}}%
\pgfpathclose%
\pgfusepath{fill}%
\end{pgfscope}%
\begin{pgfscope}%
\pgfpathrectangle{\pgfqpoint{0.637500in}{0.550000in}}{\pgfqpoint{3.850000in}{3.850000in}}%
\pgfusepath{clip}%
\pgfsetbuttcap%
\pgfsetroundjoin%
\definecolor{currentfill}{rgb}{0.898098,0.000000,0.000000}%
\pgfsetfillcolor{currentfill}%
\pgfsetfillopacity{0.800000}%
\pgfsetlinewidth{0.000000pt}%
\definecolor{currentstroke}{rgb}{0.000000,0.000000,0.000000}%
\pgfsetstrokecolor{currentstroke}%
\pgfsetdash{}{0pt}%
\pgfpathmoveto{\pgfqpoint{2.452866in}{2.767807in}}%
\pgfpathlineto{\pgfqpoint{2.382947in}{2.806310in}}%
\pgfpathlineto{\pgfqpoint{2.452866in}{2.765609in}}%
\pgfpathlineto{\pgfqpoint{2.522784in}{2.798559in}}%
\pgfpathlineto{\pgfqpoint{2.452866in}{2.767807in}}%
\pgfpathclose%
\pgfusepath{fill}%
\end{pgfscope}%
\begin{pgfscope}%
\pgfpathrectangle{\pgfqpoint{0.637500in}{0.550000in}}{\pgfqpoint{3.850000in}{3.850000in}}%
\pgfusepath{clip}%
\pgfsetbuttcap%
\pgfsetroundjoin%
\definecolor{currentfill}{rgb}{0.897487,0.000000,0.000000}%
\pgfsetfillcolor{currentfill}%
\pgfsetfillopacity{0.800000}%
\pgfsetlinewidth{0.000000pt}%
\definecolor{currentstroke}{rgb}{0.000000,0.000000,0.000000}%
\pgfsetstrokecolor{currentstroke}%
\pgfsetdash{}{0pt}%
\pgfpathmoveto{\pgfqpoint{2.592703in}{2.839657in}}%
\pgfpathlineto{\pgfqpoint{2.522784in}{2.798559in}}%
\pgfpathlineto{\pgfqpoint{2.592703in}{2.837458in}}%
\pgfpathlineto{\pgfqpoint{2.662622in}{2.878555in}}%
\pgfpathlineto{\pgfqpoint{2.592703in}{2.839657in}}%
\pgfpathclose%
\pgfusepath{fill}%
\end{pgfscope}%
\begin{pgfscope}%
\pgfpathrectangle{\pgfqpoint{0.637500in}{0.550000in}}{\pgfqpoint{3.850000in}{3.850000in}}%
\pgfusepath{clip}%
\pgfsetbuttcap%
\pgfsetroundjoin%
\definecolor{currentfill}{rgb}{0.897487,0.000000,0.000000}%
\pgfsetfillcolor{currentfill}%
\pgfsetfillopacity{0.800000}%
\pgfsetlinewidth{0.000000pt}%
\definecolor{currentstroke}{rgb}{0.000000,0.000000,0.000000}%
\pgfsetstrokecolor{currentstroke}%
\pgfsetdash{}{0pt}%
\pgfpathmoveto{\pgfqpoint{2.802459in}{2.960749in}}%
\pgfpathlineto{\pgfqpoint{2.732540in}{2.919652in}}%
\pgfpathlineto{\pgfqpoint{2.802459in}{2.958551in}}%
\pgfpathlineto{\pgfqpoint{2.872378in}{2.999648in}}%
\pgfpathlineto{\pgfqpoint{2.802459in}{2.960749in}}%
\pgfpathclose%
\pgfusepath{fill}%
\end{pgfscope}%
\begin{pgfscope}%
\pgfpathrectangle{\pgfqpoint{0.637500in}{0.550000in}}{\pgfqpoint{3.850000in}{3.850000in}}%
\pgfusepath{clip}%
\pgfsetbuttcap%
\pgfsetroundjoin%
\definecolor{currentfill}{rgb}{0.897487,0.000000,0.000000}%
\pgfsetfillcolor{currentfill}%
\pgfsetfillopacity{0.800000}%
\pgfsetlinewidth{0.000000pt}%
\definecolor{currentstroke}{rgb}{0.000000,0.000000,0.000000}%
\pgfsetstrokecolor{currentstroke}%
\pgfsetdash{}{0pt}%
\pgfpathmoveto{\pgfqpoint{2.313028in}{2.847408in}}%
\pgfpathlineto{\pgfqpoint{2.243110in}{2.886306in}}%
\pgfpathlineto{\pgfqpoint{2.313028in}{2.845209in}}%
\pgfpathlineto{\pgfqpoint{2.382947in}{2.806310in}}%
\pgfpathlineto{\pgfqpoint{2.313028in}{2.847408in}}%
\pgfpathclose%
\pgfusepath{fill}%
\end{pgfscope}%
\begin{pgfscope}%
\pgfpathrectangle{\pgfqpoint{0.637500in}{0.550000in}}{\pgfqpoint{3.850000in}{3.850000in}}%
\pgfusepath{clip}%
\pgfsetbuttcap%
\pgfsetroundjoin%
\definecolor{currentfill}{rgb}{0.909646,0.000000,0.000000}%
\pgfsetfillcolor{currentfill}%
\pgfsetfillopacity{0.800000}%
\pgfsetlinewidth{0.000000pt}%
\definecolor{currentstroke}{rgb}{0.000000,0.000000,0.000000}%
\pgfsetstrokecolor{currentstroke}%
\pgfsetdash{}{0pt}%
\pgfpathmoveto{\pgfqpoint{2.522784in}{2.798559in}}%
\pgfpathlineto{\pgfqpoint{2.452866in}{2.765609in}}%
\pgfpathlineto{\pgfqpoint{2.522784in}{2.796361in}}%
\pgfpathlineto{\pgfqpoint{2.592703in}{2.837458in}}%
\pgfpathlineto{\pgfqpoint{2.522784in}{2.798559in}}%
\pgfpathclose%
\pgfusepath{fill}%
\end{pgfscope}%
\begin{pgfscope}%
\pgfpathrectangle{\pgfqpoint{0.637500in}{0.550000in}}{\pgfqpoint{3.850000in}{3.850000in}}%
\pgfusepath{clip}%
\pgfsetbuttcap%
\pgfsetroundjoin%
\definecolor{currentfill}{rgb}{0.897487,0.000000,0.000000}%
\pgfsetfillcolor{currentfill}%
\pgfsetfillopacity{0.800000}%
\pgfsetlinewidth{0.000000pt}%
\definecolor{currentstroke}{rgb}{0.000000,0.000000,0.000000}%
\pgfsetstrokecolor{currentstroke}%
\pgfsetdash{}{0pt}%
\pgfpathmoveto{\pgfqpoint{2.732540in}{2.919652in}}%
\pgfpathlineto{\pgfqpoint{2.662622in}{2.878555in}}%
\pgfpathlineto{\pgfqpoint{2.732540in}{2.917453in}}%
\pgfpathlineto{\pgfqpoint{2.802459in}{2.958551in}}%
\pgfpathlineto{\pgfqpoint{2.732540in}{2.919652in}}%
\pgfpathclose%
\pgfusepath{fill}%
\end{pgfscope}%
\begin{pgfscope}%
\pgfpathrectangle{\pgfqpoint{0.637500in}{0.550000in}}{\pgfqpoint{3.850000in}{3.850000in}}%
\pgfusepath{clip}%
\pgfsetbuttcap%
\pgfsetroundjoin%
\definecolor{currentfill}{rgb}{0.897487,0.000000,0.000000}%
\pgfsetfillcolor{currentfill}%
\pgfsetfillopacity{0.800000}%
\pgfsetlinewidth{0.000000pt}%
\definecolor{currentstroke}{rgb}{0.000000,0.000000,0.000000}%
\pgfsetstrokecolor{currentstroke}%
\pgfsetdash{}{0pt}%
\pgfpathmoveto{\pgfqpoint{2.382947in}{2.806310in}}%
\pgfpathlineto{\pgfqpoint{2.313028in}{2.845209in}}%
\pgfpathlineto{\pgfqpoint{2.382947in}{2.804112in}}%
\pgfpathlineto{\pgfqpoint{2.452866in}{2.765609in}}%
\pgfpathlineto{\pgfqpoint{2.382947in}{2.806310in}}%
\pgfpathclose%
\pgfusepath{fill}%
\end{pgfscope}%
\begin{pgfscope}%
\pgfpathrectangle{\pgfqpoint{0.637500in}{0.550000in}}{\pgfqpoint{3.850000in}{3.850000in}}%
\pgfusepath{clip}%
\pgfsetbuttcap%
\pgfsetroundjoin%
\definecolor{currentfill}{rgb}{0.897487,0.000000,0.000000}%
\pgfsetfillcolor{currentfill}%
\pgfsetfillopacity{0.800000}%
\pgfsetlinewidth{0.000000pt}%
\definecolor{currentstroke}{rgb}{0.000000,0.000000,0.000000}%
\pgfsetstrokecolor{currentstroke}%
\pgfsetdash{}{0pt}%
\pgfpathmoveto{\pgfqpoint{2.662622in}{2.878555in}}%
\pgfpathlineto{\pgfqpoint{2.592703in}{2.837458in}}%
\pgfpathlineto{\pgfqpoint{2.662622in}{2.876356in}}%
\pgfpathlineto{\pgfqpoint{2.732540in}{2.917453in}}%
\pgfpathlineto{\pgfqpoint{2.662622in}{2.878555in}}%
\pgfpathclose%
\pgfusepath{fill}%
\end{pgfscope}%
\begin{pgfscope}%
\pgfpathrectangle{\pgfqpoint{0.637500in}{0.550000in}}{\pgfqpoint{3.850000in}{3.850000in}}%
\pgfusepath{clip}%
\pgfsetbuttcap%
\pgfsetroundjoin%
\definecolor{currentfill}{rgb}{0.898098,0.000000,0.000000}%
\pgfsetfillcolor{currentfill}%
\pgfsetfillopacity{0.800000}%
\pgfsetlinewidth{0.000000pt}%
\definecolor{currentstroke}{rgb}{0.000000,0.000000,0.000000}%
\pgfsetstrokecolor{currentstroke}%
\pgfsetdash{}{0pt}%
\pgfpathmoveto{\pgfqpoint{2.452866in}{2.765609in}}%
\pgfpathlineto{\pgfqpoint{2.382947in}{2.804112in}}%
\pgfpathlineto{\pgfqpoint{2.452866in}{2.763410in}}%
\pgfpathlineto{\pgfqpoint{2.522784in}{2.796361in}}%
\pgfpathlineto{\pgfqpoint{2.452866in}{2.765609in}}%
\pgfpathclose%
\pgfusepath{fill}%
\end{pgfscope}%
\begin{pgfscope}%
\pgfpathrectangle{\pgfqpoint{0.637500in}{0.550000in}}{\pgfqpoint{3.850000in}{3.850000in}}%
\pgfusepath{clip}%
\pgfsetbuttcap%
\pgfsetroundjoin%
\definecolor{currentfill}{rgb}{0.897487,0.000000,0.000000}%
\pgfsetfillcolor{currentfill}%
\pgfsetfillopacity{0.800000}%
\pgfsetlinewidth{0.000000pt}%
\definecolor{currentstroke}{rgb}{0.000000,0.000000,0.000000}%
\pgfsetstrokecolor{currentstroke}%
\pgfsetdash{}{0pt}%
\pgfpathmoveto{\pgfqpoint{2.592703in}{2.837458in}}%
\pgfpathlineto{\pgfqpoint{2.522784in}{2.796361in}}%
\pgfpathlineto{\pgfqpoint{2.592703in}{2.835259in}}%
\pgfpathlineto{\pgfqpoint{2.662622in}{2.876356in}}%
\pgfpathlineto{\pgfqpoint{2.592703in}{2.837458in}}%
\pgfpathclose%
\pgfusepath{fill}%
\end{pgfscope}%
\begin{pgfscope}%
\pgfpathrectangle{\pgfqpoint{0.637500in}{0.550000in}}{\pgfqpoint{3.850000in}{3.850000in}}%
\pgfusepath{clip}%
\pgfsetbuttcap%
\pgfsetroundjoin%
\definecolor{currentfill}{rgb}{0.909646,0.000000,0.000000}%
\pgfsetfillcolor{currentfill}%
\pgfsetfillopacity{0.800000}%
\pgfsetlinewidth{0.000000pt}%
\definecolor{currentstroke}{rgb}{0.000000,0.000000,0.000000}%
\pgfsetstrokecolor{currentstroke}%
\pgfsetdash{}{0pt}%
\pgfpathmoveto{\pgfqpoint{2.522784in}{2.796361in}}%
\pgfpathlineto{\pgfqpoint{2.452866in}{2.763410in}}%
\pgfpathlineto{\pgfqpoint{2.522784in}{2.794162in}}%
\pgfpathlineto{\pgfqpoint{2.592703in}{2.835259in}}%
\pgfpathlineto{\pgfqpoint{2.522784in}{2.796361in}}%
\pgfpathclose%
\pgfusepath{fill}%
\end{pgfscope}%
\begin{pgfscope}%
\pgfpathrectangle{\pgfqpoint{0.637500in}{0.550000in}}{\pgfqpoint{3.850000in}{3.850000in}}%
\pgfusepath{clip}%
\pgfsetrectcap%
\pgfsetroundjoin%
\pgfsetlinewidth{1.204500pt}%
\definecolor{currentstroke}{rgb}{1.000000,0.576471,0.309804}%
\pgfsetstrokecolor{currentstroke}%
\pgfsetdash{}{0pt}%
\pgfpathmoveto{\pgfqpoint{2.282917in}{1.978549in}}%
\pgfusepath{stroke}%
\end{pgfscope}%
\begin{pgfscope}%
\pgfpathrectangle{\pgfqpoint{0.637500in}{0.550000in}}{\pgfqpoint{3.850000in}{3.850000in}}%
\pgfusepath{clip}%
\pgfsetbuttcap%
\pgfsetroundjoin%
\definecolor{currentfill}{rgb}{1.000000,0.576471,0.309804}%
\pgfsetfillcolor{currentfill}%
\pgfsetlinewidth{1.003750pt}%
\definecolor{currentstroke}{rgb}{1.000000,0.576471,0.309804}%
\pgfsetstrokecolor{currentstroke}%
\pgfsetdash{}{0pt}%
\pgfsys@defobject{currentmarker}{\pgfqpoint{-0.033333in}{-0.033333in}}{\pgfqpoint{0.033333in}{0.033333in}}{%
\pgfpathmoveto{\pgfqpoint{0.000000in}{-0.033333in}}%
\pgfpathcurveto{\pgfqpoint{0.008840in}{-0.033333in}}{\pgfqpoint{0.017319in}{-0.029821in}}{\pgfqpoint{0.023570in}{-0.023570in}}%
\pgfpathcurveto{\pgfqpoint{0.029821in}{-0.017319in}}{\pgfqpoint{0.033333in}{-0.008840in}}{\pgfqpoint{0.033333in}{0.000000in}}%
\pgfpathcurveto{\pgfqpoint{0.033333in}{0.008840in}}{\pgfqpoint{0.029821in}{0.017319in}}{\pgfqpoint{0.023570in}{0.023570in}}%
\pgfpathcurveto{\pgfqpoint{0.017319in}{0.029821in}}{\pgfqpoint{0.008840in}{0.033333in}}{\pgfqpoint{0.000000in}{0.033333in}}%
\pgfpathcurveto{\pgfqpoint{-0.008840in}{0.033333in}}{\pgfqpoint{-0.017319in}{0.029821in}}{\pgfqpoint{-0.023570in}{0.023570in}}%
\pgfpathcurveto{\pgfqpoint{-0.029821in}{0.017319in}}{\pgfqpoint{-0.033333in}{0.008840in}}{\pgfqpoint{-0.033333in}{0.000000in}}%
\pgfpathcurveto{\pgfqpoint{-0.033333in}{-0.008840in}}{\pgfqpoint{-0.029821in}{-0.017319in}}{\pgfqpoint{-0.023570in}{-0.023570in}}%
\pgfpathcurveto{\pgfqpoint{-0.017319in}{-0.029821in}}{\pgfqpoint{-0.008840in}{-0.033333in}}{\pgfqpoint{0.000000in}{-0.033333in}}%
\pgfpathlineto{\pgfqpoint{0.000000in}{-0.033333in}}%
\pgfpathclose%
\pgfusepath{stroke,fill}%
}%
\begin{pgfscope}%
\pgfsys@transformshift{2.282917in}{1.978549in}%
\pgfsys@useobject{currentmarker}{}%
\end{pgfscope}%
\end{pgfscope}%
\begin{pgfscope}%
\pgfpathrectangle{\pgfqpoint{0.637500in}{0.550000in}}{\pgfqpoint{3.850000in}{3.850000in}}%
\pgfusepath{clip}%
\pgfsetrectcap%
\pgfsetroundjoin%
\pgfsetlinewidth{1.204500pt}%
\definecolor{currentstroke}{rgb}{1.000000,0.576471,0.309804}%
\pgfsetstrokecolor{currentstroke}%
\pgfsetdash{}{0pt}%
\pgfpathmoveto{\pgfqpoint{3.240072in}{1.510635in}}%
\pgfusepath{stroke}%
\end{pgfscope}%
\begin{pgfscope}%
\pgfpathrectangle{\pgfqpoint{0.637500in}{0.550000in}}{\pgfqpoint{3.850000in}{3.850000in}}%
\pgfusepath{clip}%
\pgfsetbuttcap%
\pgfsetroundjoin%
\definecolor{currentfill}{rgb}{1.000000,0.576471,0.309804}%
\pgfsetfillcolor{currentfill}%
\pgfsetlinewidth{1.003750pt}%
\definecolor{currentstroke}{rgb}{1.000000,0.576471,0.309804}%
\pgfsetstrokecolor{currentstroke}%
\pgfsetdash{}{0pt}%
\pgfsys@defobject{currentmarker}{\pgfqpoint{-0.033333in}{-0.033333in}}{\pgfqpoint{0.033333in}{0.033333in}}{%
\pgfpathmoveto{\pgfqpoint{0.000000in}{-0.033333in}}%
\pgfpathcurveto{\pgfqpoint{0.008840in}{-0.033333in}}{\pgfqpoint{0.017319in}{-0.029821in}}{\pgfqpoint{0.023570in}{-0.023570in}}%
\pgfpathcurveto{\pgfqpoint{0.029821in}{-0.017319in}}{\pgfqpoint{0.033333in}{-0.008840in}}{\pgfqpoint{0.033333in}{0.000000in}}%
\pgfpathcurveto{\pgfqpoint{0.033333in}{0.008840in}}{\pgfqpoint{0.029821in}{0.017319in}}{\pgfqpoint{0.023570in}{0.023570in}}%
\pgfpathcurveto{\pgfqpoint{0.017319in}{0.029821in}}{\pgfqpoint{0.008840in}{0.033333in}}{\pgfqpoint{0.000000in}{0.033333in}}%
\pgfpathcurveto{\pgfqpoint{-0.008840in}{0.033333in}}{\pgfqpoint{-0.017319in}{0.029821in}}{\pgfqpoint{-0.023570in}{0.023570in}}%
\pgfpathcurveto{\pgfqpoint{-0.029821in}{0.017319in}}{\pgfqpoint{-0.033333in}{0.008840in}}{\pgfqpoint{-0.033333in}{0.000000in}}%
\pgfpathcurveto{\pgfqpoint{-0.033333in}{-0.008840in}}{\pgfqpoint{-0.029821in}{-0.017319in}}{\pgfqpoint{-0.023570in}{-0.023570in}}%
\pgfpathcurveto{\pgfqpoint{-0.017319in}{-0.029821in}}{\pgfqpoint{-0.008840in}{-0.033333in}}{\pgfqpoint{0.000000in}{-0.033333in}}%
\pgfpathlineto{\pgfqpoint{0.000000in}{-0.033333in}}%
\pgfpathclose%
\pgfusepath{stroke,fill}%
}%
\begin{pgfscope}%
\pgfsys@transformshift{3.240072in}{1.510635in}%
\pgfsys@useobject{currentmarker}{}%
\end{pgfscope}%
\end{pgfscope}%
\begin{pgfscope}%
\pgfpathrectangle{\pgfqpoint{0.637500in}{0.550000in}}{\pgfqpoint{3.850000in}{3.850000in}}%
\pgfusepath{clip}%
\pgfsetrectcap%
\pgfsetroundjoin%
\pgfsetlinewidth{1.204500pt}%
\definecolor{currentstroke}{rgb}{1.000000,0.576471,0.309804}%
\pgfsetstrokecolor{currentstroke}%
\pgfsetdash{}{0pt}%
\pgfpathmoveto{\pgfqpoint{1.812332in}{2.464219in}}%
\pgfusepath{stroke}%
\end{pgfscope}%
\begin{pgfscope}%
\pgfpathrectangle{\pgfqpoint{0.637500in}{0.550000in}}{\pgfqpoint{3.850000in}{3.850000in}}%
\pgfusepath{clip}%
\pgfsetbuttcap%
\pgfsetroundjoin%
\definecolor{currentfill}{rgb}{1.000000,0.576471,0.309804}%
\pgfsetfillcolor{currentfill}%
\pgfsetlinewidth{1.003750pt}%
\definecolor{currentstroke}{rgb}{1.000000,0.576471,0.309804}%
\pgfsetstrokecolor{currentstroke}%
\pgfsetdash{}{0pt}%
\pgfsys@defobject{currentmarker}{\pgfqpoint{-0.033333in}{-0.033333in}}{\pgfqpoint{0.033333in}{0.033333in}}{%
\pgfpathmoveto{\pgfqpoint{0.000000in}{-0.033333in}}%
\pgfpathcurveto{\pgfqpoint{0.008840in}{-0.033333in}}{\pgfqpoint{0.017319in}{-0.029821in}}{\pgfqpoint{0.023570in}{-0.023570in}}%
\pgfpathcurveto{\pgfqpoint{0.029821in}{-0.017319in}}{\pgfqpoint{0.033333in}{-0.008840in}}{\pgfqpoint{0.033333in}{0.000000in}}%
\pgfpathcurveto{\pgfqpoint{0.033333in}{0.008840in}}{\pgfqpoint{0.029821in}{0.017319in}}{\pgfqpoint{0.023570in}{0.023570in}}%
\pgfpathcurveto{\pgfqpoint{0.017319in}{0.029821in}}{\pgfqpoint{0.008840in}{0.033333in}}{\pgfqpoint{0.000000in}{0.033333in}}%
\pgfpathcurveto{\pgfqpoint{-0.008840in}{0.033333in}}{\pgfqpoint{-0.017319in}{0.029821in}}{\pgfqpoint{-0.023570in}{0.023570in}}%
\pgfpathcurveto{\pgfqpoint{-0.029821in}{0.017319in}}{\pgfqpoint{-0.033333in}{0.008840in}}{\pgfqpoint{-0.033333in}{0.000000in}}%
\pgfpathcurveto{\pgfqpoint{-0.033333in}{-0.008840in}}{\pgfqpoint{-0.029821in}{-0.017319in}}{\pgfqpoint{-0.023570in}{-0.023570in}}%
\pgfpathcurveto{\pgfqpoint{-0.017319in}{-0.029821in}}{\pgfqpoint{-0.008840in}{-0.033333in}}{\pgfqpoint{0.000000in}{-0.033333in}}%
\pgfpathlineto{\pgfqpoint{0.000000in}{-0.033333in}}%
\pgfpathclose%
\pgfusepath{stroke,fill}%
}%
\begin{pgfscope}%
\pgfsys@transformshift{1.812332in}{2.464219in}%
\pgfsys@useobject{currentmarker}{}%
\end{pgfscope}%
\end{pgfscope}%
\begin{pgfscope}%
\pgfpathrectangle{\pgfqpoint{0.637500in}{0.550000in}}{\pgfqpoint{3.850000in}{3.850000in}}%
\pgfusepath{clip}%
\pgfsetrectcap%
\pgfsetroundjoin%
\pgfsetlinewidth{1.204500pt}%
\definecolor{currentstroke}{rgb}{1.000000,0.576471,0.309804}%
\pgfsetstrokecolor{currentstroke}%
\pgfsetdash{}{0pt}%
\pgfpathmoveto{\pgfqpoint{2.894644in}{2.235919in}}%
\pgfusepath{stroke}%
\end{pgfscope}%
\begin{pgfscope}%
\pgfpathrectangle{\pgfqpoint{0.637500in}{0.550000in}}{\pgfqpoint{3.850000in}{3.850000in}}%
\pgfusepath{clip}%
\pgfsetbuttcap%
\pgfsetroundjoin%
\definecolor{currentfill}{rgb}{1.000000,0.576471,0.309804}%
\pgfsetfillcolor{currentfill}%
\pgfsetlinewidth{1.003750pt}%
\definecolor{currentstroke}{rgb}{1.000000,0.576471,0.309804}%
\pgfsetstrokecolor{currentstroke}%
\pgfsetdash{}{0pt}%
\pgfsys@defobject{currentmarker}{\pgfqpoint{-0.033333in}{-0.033333in}}{\pgfqpoint{0.033333in}{0.033333in}}{%
\pgfpathmoveto{\pgfqpoint{0.000000in}{-0.033333in}}%
\pgfpathcurveto{\pgfqpoint{0.008840in}{-0.033333in}}{\pgfqpoint{0.017319in}{-0.029821in}}{\pgfqpoint{0.023570in}{-0.023570in}}%
\pgfpathcurveto{\pgfqpoint{0.029821in}{-0.017319in}}{\pgfqpoint{0.033333in}{-0.008840in}}{\pgfqpoint{0.033333in}{0.000000in}}%
\pgfpathcurveto{\pgfqpoint{0.033333in}{0.008840in}}{\pgfqpoint{0.029821in}{0.017319in}}{\pgfqpoint{0.023570in}{0.023570in}}%
\pgfpathcurveto{\pgfqpoint{0.017319in}{0.029821in}}{\pgfqpoint{0.008840in}{0.033333in}}{\pgfqpoint{0.000000in}{0.033333in}}%
\pgfpathcurveto{\pgfqpoint{-0.008840in}{0.033333in}}{\pgfqpoint{-0.017319in}{0.029821in}}{\pgfqpoint{-0.023570in}{0.023570in}}%
\pgfpathcurveto{\pgfqpoint{-0.029821in}{0.017319in}}{\pgfqpoint{-0.033333in}{0.008840in}}{\pgfqpoint{-0.033333in}{0.000000in}}%
\pgfpathcurveto{\pgfqpoint{-0.033333in}{-0.008840in}}{\pgfqpoint{-0.029821in}{-0.017319in}}{\pgfqpoint{-0.023570in}{-0.023570in}}%
\pgfpathcurveto{\pgfqpoint{-0.017319in}{-0.029821in}}{\pgfqpoint{-0.008840in}{-0.033333in}}{\pgfqpoint{0.000000in}{-0.033333in}}%
\pgfpathlineto{\pgfqpoint{0.000000in}{-0.033333in}}%
\pgfpathclose%
\pgfusepath{stroke,fill}%
}%
\begin{pgfscope}%
\pgfsys@transformshift{2.894644in}{2.235919in}%
\pgfsys@useobject{currentmarker}{}%
\end{pgfscope}%
\end{pgfscope}%
\begin{pgfscope}%
\pgfpathrectangle{\pgfqpoint{0.637500in}{0.550000in}}{\pgfqpoint{3.850000in}{3.850000in}}%
\pgfusepath{clip}%
\pgfsetrectcap%
\pgfsetroundjoin%
\pgfsetlinewidth{1.204500pt}%
\definecolor{currentstroke}{rgb}{1.000000,0.576471,0.309804}%
\pgfsetstrokecolor{currentstroke}%
\pgfsetdash{}{0pt}%
\pgfpathmoveto{\pgfqpoint{2.898492in}{2.308737in}}%
\pgfusepath{stroke}%
\end{pgfscope}%
\begin{pgfscope}%
\pgfpathrectangle{\pgfqpoint{0.637500in}{0.550000in}}{\pgfqpoint{3.850000in}{3.850000in}}%
\pgfusepath{clip}%
\pgfsetbuttcap%
\pgfsetroundjoin%
\definecolor{currentfill}{rgb}{1.000000,0.576471,0.309804}%
\pgfsetfillcolor{currentfill}%
\pgfsetlinewidth{1.003750pt}%
\definecolor{currentstroke}{rgb}{1.000000,0.576471,0.309804}%
\pgfsetstrokecolor{currentstroke}%
\pgfsetdash{}{0pt}%
\pgfsys@defobject{currentmarker}{\pgfqpoint{-0.033333in}{-0.033333in}}{\pgfqpoint{0.033333in}{0.033333in}}{%
\pgfpathmoveto{\pgfqpoint{0.000000in}{-0.033333in}}%
\pgfpathcurveto{\pgfqpoint{0.008840in}{-0.033333in}}{\pgfqpoint{0.017319in}{-0.029821in}}{\pgfqpoint{0.023570in}{-0.023570in}}%
\pgfpathcurveto{\pgfqpoint{0.029821in}{-0.017319in}}{\pgfqpoint{0.033333in}{-0.008840in}}{\pgfqpoint{0.033333in}{0.000000in}}%
\pgfpathcurveto{\pgfqpoint{0.033333in}{0.008840in}}{\pgfqpoint{0.029821in}{0.017319in}}{\pgfqpoint{0.023570in}{0.023570in}}%
\pgfpathcurveto{\pgfqpoint{0.017319in}{0.029821in}}{\pgfqpoint{0.008840in}{0.033333in}}{\pgfqpoint{0.000000in}{0.033333in}}%
\pgfpathcurveto{\pgfqpoint{-0.008840in}{0.033333in}}{\pgfqpoint{-0.017319in}{0.029821in}}{\pgfqpoint{-0.023570in}{0.023570in}}%
\pgfpathcurveto{\pgfqpoint{-0.029821in}{0.017319in}}{\pgfqpoint{-0.033333in}{0.008840in}}{\pgfqpoint{-0.033333in}{0.000000in}}%
\pgfpathcurveto{\pgfqpoint{-0.033333in}{-0.008840in}}{\pgfqpoint{-0.029821in}{-0.017319in}}{\pgfqpoint{-0.023570in}{-0.023570in}}%
\pgfpathcurveto{\pgfqpoint{-0.017319in}{-0.029821in}}{\pgfqpoint{-0.008840in}{-0.033333in}}{\pgfqpoint{0.000000in}{-0.033333in}}%
\pgfpathlineto{\pgfqpoint{0.000000in}{-0.033333in}}%
\pgfpathclose%
\pgfusepath{stroke,fill}%
}%
\begin{pgfscope}%
\pgfsys@transformshift{2.898492in}{2.308737in}%
\pgfsys@useobject{currentmarker}{}%
\end{pgfscope}%
\end{pgfscope}%
\begin{pgfscope}%
\pgfpathrectangle{\pgfqpoint{0.637500in}{0.550000in}}{\pgfqpoint{3.850000in}{3.850000in}}%
\pgfusepath{clip}%
\pgfsetrectcap%
\pgfsetroundjoin%
\pgfsetlinewidth{1.204500pt}%
\definecolor{currentstroke}{rgb}{1.000000,0.576471,0.309804}%
\pgfsetstrokecolor{currentstroke}%
\pgfsetdash{}{0pt}%
\pgfpathmoveto{\pgfqpoint{2.360880in}{1.415639in}}%
\pgfusepath{stroke}%
\end{pgfscope}%
\begin{pgfscope}%
\pgfpathrectangle{\pgfqpoint{0.637500in}{0.550000in}}{\pgfqpoint{3.850000in}{3.850000in}}%
\pgfusepath{clip}%
\pgfsetbuttcap%
\pgfsetroundjoin%
\definecolor{currentfill}{rgb}{1.000000,0.576471,0.309804}%
\pgfsetfillcolor{currentfill}%
\pgfsetlinewidth{1.003750pt}%
\definecolor{currentstroke}{rgb}{1.000000,0.576471,0.309804}%
\pgfsetstrokecolor{currentstroke}%
\pgfsetdash{}{0pt}%
\pgfsys@defobject{currentmarker}{\pgfqpoint{-0.033333in}{-0.033333in}}{\pgfqpoint{0.033333in}{0.033333in}}{%
\pgfpathmoveto{\pgfqpoint{0.000000in}{-0.033333in}}%
\pgfpathcurveto{\pgfqpoint{0.008840in}{-0.033333in}}{\pgfqpoint{0.017319in}{-0.029821in}}{\pgfqpoint{0.023570in}{-0.023570in}}%
\pgfpathcurveto{\pgfqpoint{0.029821in}{-0.017319in}}{\pgfqpoint{0.033333in}{-0.008840in}}{\pgfqpoint{0.033333in}{0.000000in}}%
\pgfpathcurveto{\pgfqpoint{0.033333in}{0.008840in}}{\pgfqpoint{0.029821in}{0.017319in}}{\pgfqpoint{0.023570in}{0.023570in}}%
\pgfpathcurveto{\pgfqpoint{0.017319in}{0.029821in}}{\pgfqpoint{0.008840in}{0.033333in}}{\pgfqpoint{0.000000in}{0.033333in}}%
\pgfpathcurveto{\pgfqpoint{-0.008840in}{0.033333in}}{\pgfqpoint{-0.017319in}{0.029821in}}{\pgfqpoint{-0.023570in}{0.023570in}}%
\pgfpathcurveto{\pgfqpoint{-0.029821in}{0.017319in}}{\pgfqpoint{-0.033333in}{0.008840in}}{\pgfqpoint{-0.033333in}{0.000000in}}%
\pgfpathcurveto{\pgfqpoint{-0.033333in}{-0.008840in}}{\pgfqpoint{-0.029821in}{-0.017319in}}{\pgfqpoint{-0.023570in}{-0.023570in}}%
\pgfpathcurveto{\pgfqpoint{-0.017319in}{-0.029821in}}{\pgfqpoint{-0.008840in}{-0.033333in}}{\pgfqpoint{0.000000in}{-0.033333in}}%
\pgfpathlineto{\pgfqpoint{0.000000in}{-0.033333in}}%
\pgfpathclose%
\pgfusepath{stroke,fill}%
}%
\begin{pgfscope}%
\pgfsys@transformshift{2.360880in}{1.415639in}%
\pgfsys@useobject{currentmarker}{}%
\end{pgfscope}%
\end{pgfscope}%
\begin{pgfscope}%
\pgfpathrectangle{\pgfqpoint{0.637500in}{0.550000in}}{\pgfqpoint{3.850000in}{3.850000in}}%
\pgfusepath{clip}%
\pgfsetrectcap%
\pgfsetroundjoin%
\pgfsetlinewidth{1.204500pt}%
\definecolor{currentstroke}{rgb}{1.000000,0.576471,0.309804}%
\pgfsetstrokecolor{currentstroke}%
\pgfsetdash{}{0pt}%
\pgfpathmoveto{\pgfqpoint{3.050920in}{2.272571in}}%
\pgfusepath{stroke}%
\end{pgfscope}%
\begin{pgfscope}%
\pgfpathrectangle{\pgfqpoint{0.637500in}{0.550000in}}{\pgfqpoint{3.850000in}{3.850000in}}%
\pgfusepath{clip}%
\pgfsetbuttcap%
\pgfsetroundjoin%
\definecolor{currentfill}{rgb}{1.000000,0.576471,0.309804}%
\pgfsetfillcolor{currentfill}%
\pgfsetlinewidth{1.003750pt}%
\definecolor{currentstroke}{rgb}{1.000000,0.576471,0.309804}%
\pgfsetstrokecolor{currentstroke}%
\pgfsetdash{}{0pt}%
\pgfsys@defobject{currentmarker}{\pgfqpoint{-0.033333in}{-0.033333in}}{\pgfqpoint{0.033333in}{0.033333in}}{%
\pgfpathmoveto{\pgfqpoint{0.000000in}{-0.033333in}}%
\pgfpathcurveto{\pgfqpoint{0.008840in}{-0.033333in}}{\pgfqpoint{0.017319in}{-0.029821in}}{\pgfqpoint{0.023570in}{-0.023570in}}%
\pgfpathcurveto{\pgfqpoint{0.029821in}{-0.017319in}}{\pgfqpoint{0.033333in}{-0.008840in}}{\pgfqpoint{0.033333in}{0.000000in}}%
\pgfpathcurveto{\pgfqpoint{0.033333in}{0.008840in}}{\pgfqpoint{0.029821in}{0.017319in}}{\pgfqpoint{0.023570in}{0.023570in}}%
\pgfpathcurveto{\pgfqpoint{0.017319in}{0.029821in}}{\pgfqpoint{0.008840in}{0.033333in}}{\pgfqpoint{0.000000in}{0.033333in}}%
\pgfpathcurveto{\pgfqpoint{-0.008840in}{0.033333in}}{\pgfqpoint{-0.017319in}{0.029821in}}{\pgfqpoint{-0.023570in}{0.023570in}}%
\pgfpathcurveto{\pgfqpoint{-0.029821in}{0.017319in}}{\pgfqpoint{-0.033333in}{0.008840in}}{\pgfqpoint{-0.033333in}{0.000000in}}%
\pgfpathcurveto{\pgfqpoint{-0.033333in}{-0.008840in}}{\pgfqpoint{-0.029821in}{-0.017319in}}{\pgfqpoint{-0.023570in}{-0.023570in}}%
\pgfpathcurveto{\pgfqpoint{-0.017319in}{-0.029821in}}{\pgfqpoint{-0.008840in}{-0.033333in}}{\pgfqpoint{0.000000in}{-0.033333in}}%
\pgfpathlineto{\pgfqpoint{0.000000in}{-0.033333in}}%
\pgfpathclose%
\pgfusepath{stroke,fill}%
}%
\begin{pgfscope}%
\pgfsys@transformshift{3.050920in}{2.272571in}%
\pgfsys@useobject{currentmarker}{}%
\end{pgfscope}%
\end{pgfscope}%
\begin{pgfscope}%
\pgfpathrectangle{\pgfqpoint{0.637500in}{0.550000in}}{\pgfqpoint{3.850000in}{3.850000in}}%
\pgfusepath{clip}%
\pgfsetrectcap%
\pgfsetroundjoin%
\pgfsetlinewidth{1.204500pt}%
\definecolor{currentstroke}{rgb}{1.000000,0.576471,0.309804}%
\pgfsetstrokecolor{currentstroke}%
\pgfsetdash{}{0pt}%
\pgfpathmoveto{\pgfqpoint{1.798883in}{2.462955in}}%
\pgfusepath{stroke}%
\end{pgfscope}%
\begin{pgfscope}%
\pgfpathrectangle{\pgfqpoint{0.637500in}{0.550000in}}{\pgfqpoint{3.850000in}{3.850000in}}%
\pgfusepath{clip}%
\pgfsetbuttcap%
\pgfsetroundjoin%
\definecolor{currentfill}{rgb}{1.000000,0.576471,0.309804}%
\pgfsetfillcolor{currentfill}%
\pgfsetlinewidth{1.003750pt}%
\definecolor{currentstroke}{rgb}{1.000000,0.576471,0.309804}%
\pgfsetstrokecolor{currentstroke}%
\pgfsetdash{}{0pt}%
\pgfsys@defobject{currentmarker}{\pgfqpoint{-0.033333in}{-0.033333in}}{\pgfqpoint{0.033333in}{0.033333in}}{%
\pgfpathmoveto{\pgfqpoint{0.000000in}{-0.033333in}}%
\pgfpathcurveto{\pgfqpoint{0.008840in}{-0.033333in}}{\pgfqpoint{0.017319in}{-0.029821in}}{\pgfqpoint{0.023570in}{-0.023570in}}%
\pgfpathcurveto{\pgfqpoint{0.029821in}{-0.017319in}}{\pgfqpoint{0.033333in}{-0.008840in}}{\pgfqpoint{0.033333in}{0.000000in}}%
\pgfpathcurveto{\pgfqpoint{0.033333in}{0.008840in}}{\pgfqpoint{0.029821in}{0.017319in}}{\pgfqpoint{0.023570in}{0.023570in}}%
\pgfpathcurveto{\pgfqpoint{0.017319in}{0.029821in}}{\pgfqpoint{0.008840in}{0.033333in}}{\pgfqpoint{0.000000in}{0.033333in}}%
\pgfpathcurveto{\pgfqpoint{-0.008840in}{0.033333in}}{\pgfqpoint{-0.017319in}{0.029821in}}{\pgfqpoint{-0.023570in}{0.023570in}}%
\pgfpathcurveto{\pgfqpoint{-0.029821in}{0.017319in}}{\pgfqpoint{-0.033333in}{0.008840in}}{\pgfqpoint{-0.033333in}{0.000000in}}%
\pgfpathcurveto{\pgfqpoint{-0.033333in}{-0.008840in}}{\pgfqpoint{-0.029821in}{-0.017319in}}{\pgfqpoint{-0.023570in}{-0.023570in}}%
\pgfpathcurveto{\pgfqpoint{-0.017319in}{-0.029821in}}{\pgfqpoint{-0.008840in}{-0.033333in}}{\pgfqpoint{0.000000in}{-0.033333in}}%
\pgfpathlineto{\pgfqpoint{0.000000in}{-0.033333in}}%
\pgfpathclose%
\pgfusepath{stroke,fill}%
}%
\begin{pgfscope}%
\pgfsys@transformshift{1.798883in}{2.462955in}%
\pgfsys@useobject{currentmarker}{}%
\end{pgfscope}%
\end{pgfscope}%
\begin{pgfscope}%
\pgfpathrectangle{\pgfqpoint{0.637500in}{0.550000in}}{\pgfqpoint{3.850000in}{3.850000in}}%
\pgfusepath{clip}%
\pgfsetrectcap%
\pgfsetroundjoin%
\pgfsetlinewidth{1.204500pt}%
\definecolor{currentstroke}{rgb}{1.000000,0.576471,0.309804}%
\pgfsetstrokecolor{currentstroke}%
\pgfsetdash{}{0pt}%
\pgfpathmoveto{\pgfqpoint{1.922023in}{2.269303in}}%
\pgfusepath{stroke}%
\end{pgfscope}%
\begin{pgfscope}%
\pgfpathrectangle{\pgfqpoint{0.637500in}{0.550000in}}{\pgfqpoint{3.850000in}{3.850000in}}%
\pgfusepath{clip}%
\pgfsetbuttcap%
\pgfsetroundjoin%
\definecolor{currentfill}{rgb}{1.000000,0.576471,0.309804}%
\pgfsetfillcolor{currentfill}%
\pgfsetlinewidth{1.003750pt}%
\definecolor{currentstroke}{rgb}{1.000000,0.576471,0.309804}%
\pgfsetstrokecolor{currentstroke}%
\pgfsetdash{}{0pt}%
\pgfsys@defobject{currentmarker}{\pgfqpoint{-0.033333in}{-0.033333in}}{\pgfqpoint{0.033333in}{0.033333in}}{%
\pgfpathmoveto{\pgfqpoint{0.000000in}{-0.033333in}}%
\pgfpathcurveto{\pgfqpoint{0.008840in}{-0.033333in}}{\pgfqpoint{0.017319in}{-0.029821in}}{\pgfqpoint{0.023570in}{-0.023570in}}%
\pgfpathcurveto{\pgfqpoint{0.029821in}{-0.017319in}}{\pgfqpoint{0.033333in}{-0.008840in}}{\pgfqpoint{0.033333in}{0.000000in}}%
\pgfpathcurveto{\pgfqpoint{0.033333in}{0.008840in}}{\pgfqpoint{0.029821in}{0.017319in}}{\pgfqpoint{0.023570in}{0.023570in}}%
\pgfpathcurveto{\pgfqpoint{0.017319in}{0.029821in}}{\pgfqpoint{0.008840in}{0.033333in}}{\pgfqpoint{0.000000in}{0.033333in}}%
\pgfpathcurveto{\pgfqpoint{-0.008840in}{0.033333in}}{\pgfqpoint{-0.017319in}{0.029821in}}{\pgfqpoint{-0.023570in}{0.023570in}}%
\pgfpathcurveto{\pgfqpoint{-0.029821in}{0.017319in}}{\pgfqpoint{-0.033333in}{0.008840in}}{\pgfqpoint{-0.033333in}{0.000000in}}%
\pgfpathcurveto{\pgfqpoint{-0.033333in}{-0.008840in}}{\pgfqpoint{-0.029821in}{-0.017319in}}{\pgfqpoint{-0.023570in}{-0.023570in}}%
\pgfpathcurveto{\pgfqpoint{-0.017319in}{-0.029821in}}{\pgfqpoint{-0.008840in}{-0.033333in}}{\pgfqpoint{0.000000in}{-0.033333in}}%
\pgfpathlineto{\pgfqpoint{0.000000in}{-0.033333in}}%
\pgfpathclose%
\pgfusepath{stroke,fill}%
}%
\begin{pgfscope}%
\pgfsys@transformshift{1.922023in}{2.269303in}%
\pgfsys@useobject{currentmarker}{}%
\end{pgfscope}%
\end{pgfscope}%
\begin{pgfscope}%
\pgfpathrectangle{\pgfqpoint{0.637500in}{0.550000in}}{\pgfqpoint{3.850000in}{3.850000in}}%
\pgfusepath{clip}%
\pgfsetrectcap%
\pgfsetroundjoin%
\pgfsetlinewidth{1.204500pt}%
\definecolor{currentstroke}{rgb}{1.000000,0.576471,0.309804}%
\pgfsetstrokecolor{currentstroke}%
\pgfsetdash{}{0pt}%
\pgfpathmoveto{\pgfqpoint{2.974900in}{2.283413in}}%
\pgfusepath{stroke}%
\end{pgfscope}%
\begin{pgfscope}%
\pgfpathrectangle{\pgfqpoint{0.637500in}{0.550000in}}{\pgfqpoint{3.850000in}{3.850000in}}%
\pgfusepath{clip}%
\pgfsetbuttcap%
\pgfsetroundjoin%
\definecolor{currentfill}{rgb}{1.000000,0.576471,0.309804}%
\pgfsetfillcolor{currentfill}%
\pgfsetlinewidth{1.003750pt}%
\definecolor{currentstroke}{rgb}{1.000000,0.576471,0.309804}%
\pgfsetstrokecolor{currentstroke}%
\pgfsetdash{}{0pt}%
\pgfsys@defobject{currentmarker}{\pgfqpoint{-0.033333in}{-0.033333in}}{\pgfqpoint{0.033333in}{0.033333in}}{%
\pgfpathmoveto{\pgfqpoint{0.000000in}{-0.033333in}}%
\pgfpathcurveto{\pgfqpoint{0.008840in}{-0.033333in}}{\pgfqpoint{0.017319in}{-0.029821in}}{\pgfqpoint{0.023570in}{-0.023570in}}%
\pgfpathcurveto{\pgfqpoint{0.029821in}{-0.017319in}}{\pgfqpoint{0.033333in}{-0.008840in}}{\pgfqpoint{0.033333in}{0.000000in}}%
\pgfpathcurveto{\pgfqpoint{0.033333in}{0.008840in}}{\pgfqpoint{0.029821in}{0.017319in}}{\pgfqpoint{0.023570in}{0.023570in}}%
\pgfpathcurveto{\pgfqpoint{0.017319in}{0.029821in}}{\pgfqpoint{0.008840in}{0.033333in}}{\pgfqpoint{0.000000in}{0.033333in}}%
\pgfpathcurveto{\pgfqpoint{-0.008840in}{0.033333in}}{\pgfqpoint{-0.017319in}{0.029821in}}{\pgfqpoint{-0.023570in}{0.023570in}}%
\pgfpathcurveto{\pgfqpoint{-0.029821in}{0.017319in}}{\pgfqpoint{-0.033333in}{0.008840in}}{\pgfqpoint{-0.033333in}{0.000000in}}%
\pgfpathcurveto{\pgfqpoint{-0.033333in}{-0.008840in}}{\pgfqpoint{-0.029821in}{-0.017319in}}{\pgfqpoint{-0.023570in}{-0.023570in}}%
\pgfpathcurveto{\pgfqpoint{-0.017319in}{-0.029821in}}{\pgfqpoint{-0.008840in}{-0.033333in}}{\pgfqpoint{0.000000in}{-0.033333in}}%
\pgfpathlineto{\pgfqpoint{0.000000in}{-0.033333in}}%
\pgfpathclose%
\pgfusepath{stroke,fill}%
}%
\begin{pgfscope}%
\pgfsys@transformshift{2.974900in}{2.283413in}%
\pgfsys@useobject{currentmarker}{}%
\end{pgfscope}%
\end{pgfscope}%
\begin{pgfscope}%
\pgfpathrectangle{\pgfqpoint{0.637500in}{0.550000in}}{\pgfqpoint{3.850000in}{3.850000in}}%
\pgfusepath{clip}%
\pgfsetrectcap%
\pgfsetroundjoin%
\pgfsetlinewidth{1.204500pt}%
\definecolor{currentstroke}{rgb}{1.000000,0.576471,0.309804}%
\pgfsetstrokecolor{currentstroke}%
\pgfsetdash{}{0pt}%
\pgfpathmoveto{\pgfqpoint{2.598381in}{1.942528in}}%
\pgfusepath{stroke}%
\end{pgfscope}%
\begin{pgfscope}%
\pgfpathrectangle{\pgfqpoint{0.637500in}{0.550000in}}{\pgfqpoint{3.850000in}{3.850000in}}%
\pgfusepath{clip}%
\pgfsetbuttcap%
\pgfsetroundjoin%
\definecolor{currentfill}{rgb}{1.000000,0.576471,0.309804}%
\pgfsetfillcolor{currentfill}%
\pgfsetlinewidth{1.003750pt}%
\definecolor{currentstroke}{rgb}{1.000000,0.576471,0.309804}%
\pgfsetstrokecolor{currentstroke}%
\pgfsetdash{}{0pt}%
\pgfsys@defobject{currentmarker}{\pgfqpoint{-0.033333in}{-0.033333in}}{\pgfqpoint{0.033333in}{0.033333in}}{%
\pgfpathmoveto{\pgfqpoint{0.000000in}{-0.033333in}}%
\pgfpathcurveto{\pgfqpoint{0.008840in}{-0.033333in}}{\pgfqpoint{0.017319in}{-0.029821in}}{\pgfqpoint{0.023570in}{-0.023570in}}%
\pgfpathcurveto{\pgfqpoint{0.029821in}{-0.017319in}}{\pgfqpoint{0.033333in}{-0.008840in}}{\pgfqpoint{0.033333in}{0.000000in}}%
\pgfpathcurveto{\pgfqpoint{0.033333in}{0.008840in}}{\pgfqpoint{0.029821in}{0.017319in}}{\pgfqpoint{0.023570in}{0.023570in}}%
\pgfpathcurveto{\pgfqpoint{0.017319in}{0.029821in}}{\pgfqpoint{0.008840in}{0.033333in}}{\pgfqpoint{0.000000in}{0.033333in}}%
\pgfpathcurveto{\pgfqpoint{-0.008840in}{0.033333in}}{\pgfqpoint{-0.017319in}{0.029821in}}{\pgfqpoint{-0.023570in}{0.023570in}}%
\pgfpathcurveto{\pgfqpoint{-0.029821in}{0.017319in}}{\pgfqpoint{-0.033333in}{0.008840in}}{\pgfqpoint{-0.033333in}{0.000000in}}%
\pgfpathcurveto{\pgfqpoint{-0.033333in}{-0.008840in}}{\pgfqpoint{-0.029821in}{-0.017319in}}{\pgfqpoint{-0.023570in}{-0.023570in}}%
\pgfpathcurveto{\pgfqpoint{-0.017319in}{-0.029821in}}{\pgfqpoint{-0.008840in}{-0.033333in}}{\pgfqpoint{0.000000in}{-0.033333in}}%
\pgfpathlineto{\pgfqpoint{0.000000in}{-0.033333in}}%
\pgfpathclose%
\pgfusepath{stroke,fill}%
}%
\begin{pgfscope}%
\pgfsys@transformshift{2.598381in}{1.942528in}%
\pgfsys@useobject{currentmarker}{}%
\end{pgfscope}%
\end{pgfscope}%
\begin{pgfscope}%
\pgfpathrectangle{\pgfqpoint{0.637500in}{0.550000in}}{\pgfqpoint{3.850000in}{3.850000in}}%
\pgfusepath{clip}%
\pgfsetrectcap%
\pgfsetroundjoin%
\pgfsetlinewidth{1.204500pt}%
\definecolor{currentstroke}{rgb}{1.000000,0.576471,0.309804}%
\pgfsetstrokecolor{currentstroke}%
\pgfsetdash{}{0pt}%
\pgfpathmoveto{\pgfqpoint{2.221472in}{0.890346in}}%
\pgfusepath{stroke}%
\end{pgfscope}%
\begin{pgfscope}%
\pgfpathrectangle{\pgfqpoint{0.637500in}{0.550000in}}{\pgfqpoint{3.850000in}{3.850000in}}%
\pgfusepath{clip}%
\pgfsetbuttcap%
\pgfsetroundjoin%
\definecolor{currentfill}{rgb}{1.000000,0.576471,0.309804}%
\pgfsetfillcolor{currentfill}%
\pgfsetlinewidth{1.003750pt}%
\definecolor{currentstroke}{rgb}{1.000000,0.576471,0.309804}%
\pgfsetstrokecolor{currentstroke}%
\pgfsetdash{}{0pt}%
\pgfsys@defobject{currentmarker}{\pgfqpoint{-0.033333in}{-0.033333in}}{\pgfqpoint{0.033333in}{0.033333in}}{%
\pgfpathmoveto{\pgfqpoint{0.000000in}{-0.033333in}}%
\pgfpathcurveto{\pgfqpoint{0.008840in}{-0.033333in}}{\pgfqpoint{0.017319in}{-0.029821in}}{\pgfqpoint{0.023570in}{-0.023570in}}%
\pgfpathcurveto{\pgfqpoint{0.029821in}{-0.017319in}}{\pgfqpoint{0.033333in}{-0.008840in}}{\pgfqpoint{0.033333in}{0.000000in}}%
\pgfpathcurveto{\pgfqpoint{0.033333in}{0.008840in}}{\pgfqpoint{0.029821in}{0.017319in}}{\pgfqpoint{0.023570in}{0.023570in}}%
\pgfpathcurveto{\pgfqpoint{0.017319in}{0.029821in}}{\pgfqpoint{0.008840in}{0.033333in}}{\pgfqpoint{0.000000in}{0.033333in}}%
\pgfpathcurveto{\pgfqpoint{-0.008840in}{0.033333in}}{\pgfqpoint{-0.017319in}{0.029821in}}{\pgfqpoint{-0.023570in}{0.023570in}}%
\pgfpathcurveto{\pgfqpoint{-0.029821in}{0.017319in}}{\pgfqpoint{-0.033333in}{0.008840in}}{\pgfqpoint{-0.033333in}{0.000000in}}%
\pgfpathcurveto{\pgfqpoint{-0.033333in}{-0.008840in}}{\pgfqpoint{-0.029821in}{-0.017319in}}{\pgfqpoint{-0.023570in}{-0.023570in}}%
\pgfpathcurveto{\pgfqpoint{-0.017319in}{-0.029821in}}{\pgfqpoint{-0.008840in}{-0.033333in}}{\pgfqpoint{0.000000in}{-0.033333in}}%
\pgfpathlineto{\pgfqpoint{0.000000in}{-0.033333in}}%
\pgfpathclose%
\pgfusepath{stroke,fill}%
}%
\begin{pgfscope}%
\pgfsys@transformshift{2.221472in}{0.890346in}%
\pgfsys@useobject{currentmarker}{}%
\end{pgfscope}%
\end{pgfscope}%
\begin{pgfscope}%
\pgfpathrectangle{\pgfqpoint{0.637500in}{0.550000in}}{\pgfqpoint{3.850000in}{3.850000in}}%
\pgfusepath{clip}%
\pgfsetrectcap%
\pgfsetroundjoin%
\pgfsetlinewidth{1.204500pt}%
\definecolor{currentstroke}{rgb}{1.000000,0.576471,0.309804}%
\pgfsetstrokecolor{currentstroke}%
\pgfsetdash{}{0pt}%
\pgfpathmoveto{\pgfqpoint{3.540157in}{2.591797in}}%
\pgfusepath{stroke}%
\end{pgfscope}%
\begin{pgfscope}%
\pgfpathrectangle{\pgfqpoint{0.637500in}{0.550000in}}{\pgfqpoint{3.850000in}{3.850000in}}%
\pgfusepath{clip}%
\pgfsetbuttcap%
\pgfsetroundjoin%
\definecolor{currentfill}{rgb}{1.000000,0.576471,0.309804}%
\pgfsetfillcolor{currentfill}%
\pgfsetlinewidth{1.003750pt}%
\definecolor{currentstroke}{rgb}{1.000000,0.576471,0.309804}%
\pgfsetstrokecolor{currentstroke}%
\pgfsetdash{}{0pt}%
\pgfsys@defobject{currentmarker}{\pgfqpoint{-0.033333in}{-0.033333in}}{\pgfqpoint{0.033333in}{0.033333in}}{%
\pgfpathmoveto{\pgfqpoint{0.000000in}{-0.033333in}}%
\pgfpathcurveto{\pgfqpoint{0.008840in}{-0.033333in}}{\pgfqpoint{0.017319in}{-0.029821in}}{\pgfqpoint{0.023570in}{-0.023570in}}%
\pgfpathcurveto{\pgfqpoint{0.029821in}{-0.017319in}}{\pgfqpoint{0.033333in}{-0.008840in}}{\pgfqpoint{0.033333in}{0.000000in}}%
\pgfpathcurveto{\pgfqpoint{0.033333in}{0.008840in}}{\pgfqpoint{0.029821in}{0.017319in}}{\pgfqpoint{0.023570in}{0.023570in}}%
\pgfpathcurveto{\pgfqpoint{0.017319in}{0.029821in}}{\pgfqpoint{0.008840in}{0.033333in}}{\pgfqpoint{0.000000in}{0.033333in}}%
\pgfpathcurveto{\pgfqpoint{-0.008840in}{0.033333in}}{\pgfqpoint{-0.017319in}{0.029821in}}{\pgfqpoint{-0.023570in}{0.023570in}}%
\pgfpathcurveto{\pgfqpoint{-0.029821in}{0.017319in}}{\pgfqpoint{-0.033333in}{0.008840in}}{\pgfqpoint{-0.033333in}{0.000000in}}%
\pgfpathcurveto{\pgfqpoint{-0.033333in}{-0.008840in}}{\pgfqpoint{-0.029821in}{-0.017319in}}{\pgfqpoint{-0.023570in}{-0.023570in}}%
\pgfpathcurveto{\pgfqpoint{-0.017319in}{-0.029821in}}{\pgfqpoint{-0.008840in}{-0.033333in}}{\pgfqpoint{0.000000in}{-0.033333in}}%
\pgfpathlineto{\pgfqpoint{0.000000in}{-0.033333in}}%
\pgfpathclose%
\pgfusepath{stroke,fill}%
}%
\begin{pgfscope}%
\pgfsys@transformshift{3.540157in}{2.591797in}%
\pgfsys@useobject{currentmarker}{}%
\end{pgfscope}%
\end{pgfscope}%
\begin{pgfscope}%
\pgfpathrectangle{\pgfqpoint{0.637500in}{0.550000in}}{\pgfqpoint{3.850000in}{3.850000in}}%
\pgfusepath{clip}%
\pgfsetrectcap%
\pgfsetroundjoin%
\pgfsetlinewidth{1.204500pt}%
\definecolor{currentstroke}{rgb}{1.000000,0.576471,0.309804}%
\pgfsetstrokecolor{currentstroke}%
\pgfsetdash{}{0pt}%
\pgfpathmoveto{\pgfqpoint{1.727861in}{1.763766in}}%
\pgfusepath{stroke}%
\end{pgfscope}%
\begin{pgfscope}%
\pgfpathrectangle{\pgfqpoint{0.637500in}{0.550000in}}{\pgfqpoint{3.850000in}{3.850000in}}%
\pgfusepath{clip}%
\pgfsetbuttcap%
\pgfsetroundjoin%
\definecolor{currentfill}{rgb}{1.000000,0.576471,0.309804}%
\pgfsetfillcolor{currentfill}%
\pgfsetlinewidth{1.003750pt}%
\definecolor{currentstroke}{rgb}{1.000000,0.576471,0.309804}%
\pgfsetstrokecolor{currentstroke}%
\pgfsetdash{}{0pt}%
\pgfsys@defobject{currentmarker}{\pgfqpoint{-0.033333in}{-0.033333in}}{\pgfqpoint{0.033333in}{0.033333in}}{%
\pgfpathmoveto{\pgfqpoint{0.000000in}{-0.033333in}}%
\pgfpathcurveto{\pgfqpoint{0.008840in}{-0.033333in}}{\pgfqpoint{0.017319in}{-0.029821in}}{\pgfqpoint{0.023570in}{-0.023570in}}%
\pgfpathcurveto{\pgfqpoint{0.029821in}{-0.017319in}}{\pgfqpoint{0.033333in}{-0.008840in}}{\pgfqpoint{0.033333in}{0.000000in}}%
\pgfpathcurveto{\pgfqpoint{0.033333in}{0.008840in}}{\pgfqpoint{0.029821in}{0.017319in}}{\pgfqpoint{0.023570in}{0.023570in}}%
\pgfpathcurveto{\pgfqpoint{0.017319in}{0.029821in}}{\pgfqpoint{0.008840in}{0.033333in}}{\pgfqpoint{0.000000in}{0.033333in}}%
\pgfpathcurveto{\pgfqpoint{-0.008840in}{0.033333in}}{\pgfqpoint{-0.017319in}{0.029821in}}{\pgfqpoint{-0.023570in}{0.023570in}}%
\pgfpathcurveto{\pgfqpoint{-0.029821in}{0.017319in}}{\pgfqpoint{-0.033333in}{0.008840in}}{\pgfqpoint{-0.033333in}{0.000000in}}%
\pgfpathcurveto{\pgfqpoint{-0.033333in}{-0.008840in}}{\pgfqpoint{-0.029821in}{-0.017319in}}{\pgfqpoint{-0.023570in}{-0.023570in}}%
\pgfpathcurveto{\pgfqpoint{-0.017319in}{-0.029821in}}{\pgfqpoint{-0.008840in}{-0.033333in}}{\pgfqpoint{0.000000in}{-0.033333in}}%
\pgfpathlineto{\pgfqpoint{0.000000in}{-0.033333in}}%
\pgfpathclose%
\pgfusepath{stroke,fill}%
}%
\begin{pgfscope}%
\pgfsys@transformshift{1.727861in}{1.763766in}%
\pgfsys@useobject{currentmarker}{}%
\end{pgfscope}%
\end{pgfscope}%
\begin{pgfscope}%
\pgfpathrectangle{\pgfqpoint{0.637500in}{0.550000in}}{\pgfqpoint{3.850000in}{3.850000in}}%
\pgfusepath{clip}%
\pgfsetrectcap%
\pgfsetroundjoin%
\pgfsetlinewidth{1.204500pt}%
\definecolor{currentstroke}{rgb}{1.000000,0.576471,0.309804}%
\pgfsetstrokecolor{currentstroke}%
\pgfsetdash{}{0pt}%
\pgfpathmoveto{\pgfqpoint{2.241122in}{1.322285in}}%
\pgfusepath{stroke}%
\end{pgfscope}%
\begin{pgfscope}%
\pgfpathrectangle{\pgfqpoint{0.637500in}{0.550000in}}{\pgfqpoint{3.850000in}{3.850000in}}%
\pgfusepath{clip}%
\pgfsetbuttcap%
\pgfsetroundjoin%
\definecolor{currentfill}{rgb}{1.000000,0.576471,0.309804}%
\pgfsetfillcolor{currentfill}%
\pgfsetlinewidth{1.003750pt}%
\definecolor{currentstroke}{rgb}{1.000000,0.576471,0.309804}%
\pgfsetstrokecolor{currentstroke}%
\pgfsetdash{}{0pt}%
\pgfsys@defobject{currentmarker}{\pgfqpoint{-0.033333in}{-0.033333in}}{\pgfqpoint{0.033333in}{0.033333in}}{%
\pgfpathmoveto{\pgfqpoint{0.000000in}{-0.033333in}}%
\pgfpathcurveto{\pgfqpoint{0.008840in}{-0.033333in}}{\pgfqpoint{0.017319in}{-0.029821in}}{\pgfqpoint{0.023570in}{-0.023570in}}%
\pgfpathcurveto{\pgfqpoint{0.029821in}{-0.017319in}}{\pgfqpoint{0.033333in}{-0.008840in}}{\pgfqpoint{0.033333in}{0.000000in}}%
\pgfpathcurveto{\pgfqpoint{0.033333in}{0.008840in}}{\pgfqpoint{0.029821in}{0.017319in}}{\pgfqpoint{0.023570in}{0.023570in}}%
\pgfpathcurveto{\pgfqpoint{0.017319in}{0.029821in}}{\pgfqpoint{0.008840in}{0.033333in}}{\pgfqpoint{0.000000in}{0.033333in}}%
\pgfpathcurveto{\pgfqpoint{-0.008840in}{0.033333in}}{\pgfqpoint{-0.017319in}{0.029821in}}{\pgfqpoint{-0.023570in}{0.023570in}}%
\pgfpathcurveto{\pgfqpoint{-0.029821in}{0.017319in}}{\pgfqpoint{-0.033333in}{0.008840in}}{\pgfqpoint{-0.033333in}{0.000000in}}%
\pgfpathcurveto{\pgfqpoint{-0.033333in}{-0.008840in}}{\pgfqpoint{-0.029821in}{-0.017319in}}{\pgfqpoint{-0.023570in}{-0.023570in}}%
\pgfpathcurveto{\pgfqpoint{-0.017319in}{-0.029821in}}{\pgfqpoint{-0.008840in}{-0.033333in}}{\pgfqpoint{0.000000in}{-0.033333in}}%
\pgfpathlineto{\pgfqpoint{0.000000in}{-0.033333in}}%
\pgfpathclose%
\pgfusepath{stroke,fill}%
}%
\begin{pgfscope}%
\pgfsys@transformshift{2.241122in}{1.322285in}%
\pgfsys@useobject{currentmarker}{}%
\end{pgfscope}%
\end{pgfscope}%
\begin{pgfscope}%
\pgfpathrectangle{\pgfqpoint{0.637500in}{0.550000in}}{\pgfqpoint{3.850000in}{3.850000in}}%
\pgfusepath{clip}%
\pgfsetrectcap%
\pgfsetroundjoin%
\pgfsetlinewidth{1.204500pt}%
\definecolor{currentstroke}{rgb}{1.000000,0.576471,0.309804}%
\pgfsetstrokecolor{currentstroke}%
\pgfsetdash{}{0pt}%
\pgfpathmoveto{\pgfqpoint{2.429568in}{1.530108in}}%
\pgfusepath{stroke}%
\end{pgfscope}%
\begin{pgfscope}%
\pgfpathrectangle{\pgfqpoint{0.637500in}{0.550000in}}{\pgfqpoint{3.850000in}{3.850000in}}%
\pgfusepath{clip}%
\pgfsetbuttcap%
\pgfsetroundjoin%
\definecolor{currentfill}{rgb}{1.000000,0.576471,0.309804}%
\pgfsetfillcolor{currentfill}%
\pgfsetlinewidth{1.003750pt}%
\definecolor{currentstroke}{rgb}{1.000000,0.576471,0.309804}%
\pgfsetstrokecolor{currentstroke}%
\pgfsetdash{}{0pt}%
\pgfsys@defobject{currentmarker}{\pgfqpoint{-0.033333in}{-0.033333in}}{\pgfqpoint{0.033333in}{0.033333in}}{%
\pgfpathmoveto{\pgfqpoint{0.000000in}{-0.033333in}}%
\pgfpathcurveto{\pgfqpoint{0.008840in}{-0.033333in}}{\pgfqpoint{0.017319in}{-0.029821in}}{\pgfqpoint{0.023570in}{-0.023570in}}%
\pgfpathcurveto{\pgfqpoint{0.029821in}{-0.017319in}}{\pgfqpoint{0.033333in}{-0.008840in}}{\pgfqpoint{0.033333in}{0.000000in}}%
\pgfpathcurveto{\pgfqpoint{0.033333in}{0.008840in}}{\pgfqpoint{0.029821in}{0.017319in}}{\pgfqpoint{0.023570in}{0.023570in}}%
\pgfpathcurveto{\pgfqpoint{0.017319in}{0.029821in}}{\pgfqpoint{0.008840in}{0.033333in}}{\pgfqpoint{0.000000in}{0.033333in}}%
\pgfpathcurveto{\pgfqpoint{-0.008840in}{0.033333in}}{\pgfqpoint{-0.017319in}{0.029821in}}{\pgfqpoint{-0.023570in}{0.023570in}}%
\pgfpathcurveto{\pgfqpoint{-0.029821in}{0.017319in}}{\pgfqpoint{-0.033333in}{0.008840in}}{\pgfqpoint{-0.033333in}{0.000000in}}%
\pgfpathcurveto{\pgfqpoint{-0.033333in}{-0.008840in}}{\pgfqpoint{-0.029821in}{-0.017319in}}{\pgfqpoint{-0.023570in}{-0.023570in}}%
\pgfpathcurveto{\pgfqpoint{-0.017319in}{-0.029821in}}{\pgfqpoint{-0.008840in}{-0.033333in}}{\pgfqpoint{0.000000in}{-0.033333in}}%
\pgfpathlineto{\pgfqpoint{0.000000in}{-0.033333in}}%
\pgfpathclose%
\pgfusepath{stroke,fill}%
}%
\begin{pgfscope}%
\pgfsys@transformshift{2.429568in}{1.530108in}%
\pgfsys@useobject{currentmarker}{}%
\end{pgfscope}%
\end{pgfscope}%
\begin{pgfscope}%
\pgfpathrectangle{\pgfqpoint{0.637500in}{0.550000in}}{\pgfqpoint{3.850000in}{3.850000in}}%
\pgfusepath{clip}%
\pgfsetrectcap%
\pgfsetroundjoin%
\pgfsetlinewidth{1.204500pt}%
\definecolor{currentstroke}{rgb}{1.000000,0.576471,0.309804}%
\pgfsetstrokecolor{currentstroke}%
\pgfsetdash{}{0pt}%
\pgfpathmoveto{\pgfqpoint{2.693237in}{1.992776in}}%
\pgfusepath{stroke}%
\end{pgfscope}%
\begin{pgfscope}%
\pgfpathrectangle{\pgfqpoint{0.637500in}{0.550000in}}{\pgfqpoint{3.850000in}{3.850000in}}%
\pgfusepath{clip}%
\pgfsetbuttcap%
\pgfsetroundjoin%
\definecolor{currentfill}{rgb}{1.000000,0.576471,0.309804}%
\pgfsetfillcolor{currentfill}%
\pgfsetlinewidth{1.003750pt}%
\definecolor{currentstroke}{rgb}{1.000000,0.576471,0.309804}%
\pgfsetstrokecolor{currentstroke}%
\pgfsetdash{}{0pt}%
\pgfsys@defobject{currentmarker}{\pgfqpoint{-0.033333in}{-0.033333in}}{\pgfqpoint{0.033333in}{0.033333in}}{%
\pgfpathmoveto{\pgfqpoint{0.000000in}{-0.033333in}}%
\pgfpathcurveto{\pgfqpoint{0.008840in}{-0.033333in}}{\pgfqpoint{0.017319in}{-0.029821in}}{\pgfqpoint{0.023570in}{-0.023570in}}%
\pgfpathcurveto{\pgfqpoint{0.029821in}{-0.017319in}}{\pgfqpoint{0.033333in}{-0.008840in}}{\pgfqpoint{0.033333in}{0.000000in}}%
\pgfpathcurveto{\pgfqpoint{0.033333in}{0.008840in}}{\pgfqpoint{0.029821in}{0.017319in}}{\pgfqpoint{0.023570in}{0.023570in}}%
\pgfpathcurveto{\pgfqpoint{0.017319in}{0.029821in}}{\pgfqpoint{0.008840in}{0.033333in}}{\pgfqpoint{0.000000in}{0.033333in}}%
\pgfpathcurveto{\pgfqpoint{-0.008840in}{0.033333in}}{\pgfqpoint{-0.017319in}{0.029821in}}{\pgfqpoint{-0.023570in}{0.023570in}}%
\pgfpathcurveto{\pgfqpoint{-0.029821in}{0.017319in}}{\pgfqpoint{-0.033333in}{0.008840in}}{\pgfqpoint{-0.033333in}{0.000000in}}%
\pgfpathcurveto{\pgfqpoint{-0.033333in}{-0.008840in}}{\pgfqpoint{-0.029821in}{-0.017319in}}{\pgfqpoint{-0.023570in}{-0.023570in}}%
\pgfpathcurveto{\pgfqpoint{-0.017319in}{-0.029821in}}{\pgfqpoint{-0.008840in}{-0.033333in}}{\pgfqpoint{0.000000in}{-0.033333in}}%
\pgfpathlineto{\pgfqpoint{0.000000in}{-0.033333in}}%
\pgfpathclose%
\pgfusepath{stroke,fill}%
}%
\begin{pgfscope}%
\pgfsys@transformshift{2.693237in}{1.992776in}%
\pgfsys@useobject{currentmarker}{}%
\end{pgfscope}%
\end{pgfscope}%
\begin{pgfscope}%
\pgfpathrectangle{\pgfqpoint{0.637500in}{0.550000in}}{\pgfqpoint{3.850000in}{3.850000in}}%
\pgfusepath{clip}%
\pgfsetrectcap%
\pgfsetroundjoin%
\pgfsetlinewidth{1.204500pt}%
\definecolor{currentstroke}{rgb}{1.000000,0.576471,0.309804}%
\pgfsetstrokecolor{currentstroke}%
\pgfsetdash{}{0pt}%
\pgfpathmoveto{\pgfqpoint{2.585038in}{2.191805in}}%
\pgfusepath{stroke}%
\end{pgfscope}%
\begin{pgfscope}%
\pgfpathrectangle{\pgfqpoint{0.637500in}{0.550000in}}{\pgfqpoint{3.850000in}{3.850000in}}%
\pgfusepath{clip}%
\pgfsetbuttcap%
\pgfsetroundjoin%
\definecolor{currentfill}{rgb}{1.000000,0.576471,0.309804}%
\pgfsetfillcolor{currentfill}%
\pgfsetlinewidth{1.003750pt}%
\definecolor{currentstroke}{rgb}{1.000000,0.576471,0.309804}%
\pgfsetstrokecolor{currentstroke}%
\pgfsetdash{}{0pt}%
\pgfsys@defobject{currentmarker}{\pgfqpoint{-0.033333in}{-0.033333in}}{\pgfqpoint{0.033333in}{0.033333in}}{%
\pgfpathmoveto{\pgfqpoint{0.000000in}{-0.033333in}}%
\pgfpathcurveto{\pgfqpoint{0.008840in}{-0.033333in}}{\pgfqpoint{0.017319in}{-0.029821in}}{\pgfqpoint{0.023570in}{-0.023570in}}%
\pgfpathcurveto{\pgfqpoint{0.029821in}{-0.017319in}}{\pgfqpoint{0.033333in}{-0.008840in}}{\pgfqpoint{0.033333in}{0.000000in}}%
\pgfpathcurveto{\pgfqpoint{0.033333in}{0.008840in}}{\pgfqpoint{0.029821in}{0.017319in}}{\pgfqpoint{0.023570in}{0.023570in}}%
\pgfpathcurveto{\pgfqpoint{0.017319in}{0.029821in}}{\pgfqpoint{0.008840in}{0.033333in}}{\pgfqpoint{0.000000in}{0.033333in}}%
\pgfpathcurveto{\pgfqpoint{-0.008840in}{0.033333in}}{\pgfqpoint{-0.017319in}{0.029821in}}{\pgfqpoint{-0.023570in}{0.023570in}}%
\pgfpathcurveto{\pgfqpoint{-0.029821in}{0.017319in}}{\pgfqpoint{-0.033333in}{0.008840in}}{\pgfqpoint{-0.033333in}{0.000000in}}%
\pgfpathcurveto{\pgfqpoint{-0.033333in}{-0.008840in}}{\pgfqpoint{-0.029821in}{-0.017319in}}{\pgfqpoint{-0.023570in}{-0.023570in}}%
\pgfpathcurveto{\pgfqpoint{-0.017319in}{-0.029821in}}{\pgfqpoint{-0.008840in}{-0.033333in}}{\pgfqpoint{0.000000in}{-0.033333in}}%
\pgfpathlineto{\pgfqpoint{0.000000in}{-0.033333in}}%
\pgfpathclose%
\pgfusepath{stroke,fill}%
}%
\begin{pgfscope}%
\pgfsys@transformshift{2.585038in}{2.191805in}%
\pgfsys@useobject{currentmarker}{}%
\end{pgfscope}%
\end{pgfscope}%
\begin{pgfscope}%
\pgfpathrectangle{\pgfqpoint{0.637500in}{0.550000in}}{\pgfqpoint{3.850000in}{3.850000in}}%
\pgfusepath{clip}%
\pgfsetrectcap%
\pgfsetroundjoin%
\pgfsetlinewidth{1.204500pt}%
\definecolor{currentstroke}{rgb}{1.000000,0.576471,0.309804}%
\pgfsetstrokecolor{currentstroke}%
\pgfsetdash{}{0pt}%
\pgfpathmoveto{\pgfqpoint{3.310891in}{1.698041in}}%
\pgfusepath{stroke}%
\end{pgfscope}%
\begin{pgfscope}%
\pgfpathrectangle{\pgfqpoint{0.637500in}{0.550000in}}{\pgfqpoint{3.850000in}{3.850000in}}%
\pgfusepath{clip}%
\pgfsetbuttcap%
\pgfsetroundjoin%
\definecolor{currentfill}{rgb}{1.000000,0.576471,0.309804}%
\pgfsetfillcolor{currentfill}%
\pgfsetlinewidth{1.003750pt}%
\definecolor{currentstroke}{rgb}{1.000000,0.576471,0.309804}%
\pgfsetstrokecolor{currentstroke}%
\pgfsetdash{}{0pt}%
\pgfsys@defobject{currentmarker}{\pgfqpoint{-0.033333in}{-0.033333in}}{\pgfqpoint{0.033333in}{0.033333in}}{%
\pgfpathmoveto{\pgfqpoint{0.000000in}{-0.033333in}}%
\pgfpathcurveto{\pgfqpoint{0.008840in}{-0.033333in}}{\pgfqpoint{0.017319in}{-0.029821in}}{\pgfqpoint{0.023570in}{-0.023570in}}%
\pgfpathcurveto{\pgfqpoint{0.029821in}{-0.017319in}}{\pgfqpoint{0.033333in}{-0.008840in}}{\pgfqpoint{0.033333in}{0.000000in}}%
\pgfpathcurveto{\pgfqpoint{0.033333in}{0.008840in}}{\pgfqpoint{0.029821in}{0.017319in}}{\pgfqpoint{0.023570in}{0.023570in}}%
\pgfpathcurveto{\pgfqpoint{0.017319in}{0.029821in}}{\pgfqpoint{0.008840in}{0.033333in}}{\pgfqpoint{0.000000in}{0.033333in}}%
\pgfpathcurveto{\pgfqpoint{-0.008840in}{0.033333in}}{\pgfqpoint{-0.017319in}{0.029821in}}{\pgfqpoint{-0.023570in}{0.023570in}}%
\pgfpathcurveto{\pgfqpoint{-0.029821in}{0.017319in}}{\pgfqpoint{-0.033333in}{0.008840in}}{\pgfqpoint{-0.033333in}{0.000000in}}%
\pgfpathcurveto{\pgfqpoint{-0.033333in}{-0.008840in}}{\pgfqpoint{-0.029821in}{-0.017319in}}{\pgfqpoint{-0.023570in}{-0.023570in}}%
\pgfpathcurveto{\pgfqpoint{-0.017319in}{-0.029821in}}{\pgfqpoint{-0.008840in}{-0.033333in}}{\pgfqpoint{0.000000in}{-0.033333in}}%
\pgfpathlineto{\pgfqpoint{0.000000in}{-0.033333in}}%
\pgfpathclose%
\pgfusepath{stroke,fill}%
}%
\begin{pgfscope}%
\pgfsys@transformshift{3.310891in}{1.698041in}%
\pgfsys@useobject{currentmarker}{}%
\end{pgfscope}%
\end{pgfscope}%
\begin{pgfscope}%
\pgfpathrectangle{\pgfqpoint{0.637500in}{0.550000in}}{\pgfqpoint{3.850000in}{3.850000in}}%
\pgfusepath{clip}%
\pgfsetrectcap%
\pgfsetroundjoin%
\pgfsetlinewidth{1.204500pt}%
\definecolor{currentstroke}{rgb}{1.000000,0.576471,0.309804}%
\pgfsetstrokecolor{currentstroke}%
\pgfsetdash{}{0pt}%
\pgfpathmoveto{\pgfqpoint{3.240359in}{2.562466in}}%
\pgfusepath{stroke}%
\end{pgfscope}%
\begin{pgfscope}%
\pgfpathrectangle{\pgfqpoint{0.637500in}{0.550000in}}{\pgfqpoint{3.850000in}{3.850000in}}%
\pgfusepath{clip}%
\pgfsetbuttcap%
\pgfsetroundjoin%
\definecolor{currentfill}{rgb}{1.000000,0.576471,0.309804}%
\pgfsetfillcolor{currentfill}%
\pgfsetlinewidth{1.003750pt}%
\definecolor{currentstroke}{rgb}{1.000000,0.576471,0.309804}%
\pgfsetstrokecolor{currentstroke}%
\pgfsetdash{}{0pt}%
\pgfsys@defobject{currentmarker}{\pgfqpoint{-0.033333in}{-0.033333in}}{\pgfqpoint{0.033333in}{0.033333in}}{%
\pgfpathmoveto{\pgfqpoint{0.000000in}{-0.033333in}}%
\pgfpathcurveto{\pgfqpoint{0.008840in}{-0.033333in}}{\pgfqpoint{0.017319in}{-0.029821in}}{\pgfqpoint{0.023570in}{-0.023570in}}%
\pgfpathcurveto{\pgfqpoint{0.029821in}{-0.017319in}}{\pgfqpoint{0.033333in}{-0.008840in}}{\pgfqpoint{0.033333in}{0.000000in}}%
\pgfpathcurveto{\pgfqpoint{0.033333in}{0.008840in}}{\pgfqpoint{0.029821in}{0.017319in}}{\pgfqpoint{0.023570in}{0.023570in}}%
\pgfpathcurveto{\pgfqpoint{0.017319in}{0.029821in}}{\pgfqpoint{0.008840in}{0.033333in}}{\pgfqpoint{0.000000in}{0.033333in}}%
\pgfpathcurveto{\pgfqpoint{-0.008840in}{0.033333in}}{\pgfqpoint{-0.017319in}{0.029821in}}{\pgfqpoint{-0.023570in}{0.023570in}}%
\pgfpathcurveto{\pgfqpoint{-0.029821in}{0.017319in}}{\pgfqpoint{-0.033333in}{0.008840in}}{\pgfqpoint{-0.033333in}{0.000000in}}%
\pgfpathcurveto{\pgfqpoint{-0.033333in}{-0.008840in}}{\pgfqpoint{-0.029821in}{-0.017319in}}{\pgfqpoint{-0.023570in}{-0.023570in}}%
\pgfpathcurveto{\pgfqpoint{-0.017319in}{-0.029821in}}{\pgfqpoint{-0.008840in}{-0.033333in}}{\pgfqpoint{0.000000in}{-0.033333in}}%
\pgfpathlineto{\pgfqpoint{0.000000in}{-0.033333in}}%
\pgfpathclose%
\pgfusepath{stroke,fill}%
}%
\begin{pgfscope}%
\pgfsys@transformshift{3.240359in}{2.562466in}%
\pgfsys@useobject{currentmarker}{}%
\end{pgfscope}%
\end{pgfscope}%
\begin{pgfscope}%
\pgfpathrectangle{\pgfqpoint{0.637500in}{0.550000in}}{\pgfqpoint{3.850000in}{3.850000in}}%
\pgfusepath{clip}%
\pgfsetrectcap%
\pgfsetroundjoin%
\pgfsetlinewidth{1.204500pt}%
\definecolor{currentstroke}{rgb}{1.000000,0.576471,0.309804}%
\pgfsetstrokecolor{currentstroke}%
\pgfsetdash{}{0pt}%
\pgfpathmoveto{\pgfqpoint{1.842309in}{1.988908in}}%
\pgfusepath{stroke}%
\end{pgfscope}%
\begin{pgfscope}%
\pgfpathrectangle{\pgfqpoint{0.637500in}{0.550000in}}{\pgfqpoint{3.850000in}{3.850000in}}%
\pgfusepath{clip}%
\pgfsetbuttcap%
\pgfsetroundjoin%
\definecolor{currentfill}{rgb}{1.000000,0.576471,0.309804}%
\pgfsetfillcolor{currentfill}%
\pgfsetlinewidth{1.003750pt}%
\definecolor{currentstroke}{rgb}{1.000000,0.576471,0.309804}%
\pgfsetstrokecolor{currentstroke}%
\pgfsetdash{}{0pt}%
\pgfsys@defobject{currentmarker}{\pgfqpoint{-0.033333in}{-0.033333in}}{\pgfqpoint{0.033333in}{0.033333in}}{%
\pgfpathmoveto{\pgfqpoint{0.000000in}{-0.033333in}}%
\pgfpathcurveto{\pgfqpoint{0.008840in}{-0.033333in}}{\pgfqpoint{0.017319in}{-0.029821in}}{\pgfqpoint{0.023570in}{-0.023570in}}%
\pgfpathcurveto{\pgfqpoint{0.029821in}{-0.017319in}}{\pgfqpoint{0.033333in}{-0.008840in}}{\pgfqpoint{0.033333in}{0.000000in}}%
\pgfpathcurveto{\pgfqpoint{0.033333in}{0.008840in}}{\pgfqpoint{0.029821in}{0.017319in}}{\pgfqpoint{0.023570in}{0.023570in}}%
\pgfpathcurveto{\pgfqpoint{0.017319in}{0.029821in}}{\pgfqpoint{0.008840in}{0.033333in}}{\pgfqpoint{0.000000in}{0.033333in}}%
\pgfpathcurveto{\pgfqpoint{-0.008840in}{0.033333in}}{\pgfqpoint{-0.017319in}{0.029821in}}{\pgfqpoint{-0.023570in}{0.023570in}}%
\pgfpathcurveto{\pgfqpoint{-0.029821in}{0.017319in}}{\pgfqpoint{-0.033333in}{0.008840in}}{\pgfqpoint{-0.033333in}{0.000000in}}%
\pgfpathcurveto{\pgfqpoint{-0.033333in}{-0.008840in}}{\pgfqpoint{-0.029821in}{-0.017319in}}{\pgfqpoint{-0.023570in}{-0.023570in}}%
\pgfpathcurveto{\pgfqpoint{-0.017319in}{-0.029821in}}{\pgfqpoint{-0.008840in}{-0.033333in}}{\pgfqpoint{0.000000in}{-0.033333in}}%
\pgfpathlineto{\pgfqpoint{0.000000in}{-0.033333in}}%
\pgfpathclose%
\pgfusepath{stroke,fill}%
}%
\begin{pgfscope}%
\pgfsys@transformshift{1.842309in}{1.988908in}%
\pgfsys@useobject{currentmarker}{}%
\end{pgfscope}%
\end{pgfscope}%
\begin{pgfscope}%
\pgfpathrectangle{\pgfqpoint{0.637500in}{0.550000in}}{\pgfqpoint{3.850000in}{3.850000in}}%
\pgfusepath{clip}%
\pgfsetrectcap%
\pgfsetroundjoin%
\pgfsetlinewidth{1.204500pt}%
\definecolor{currentstroke}{rgb}{1.000000,0.576471,0.309804}%
\pgfsetstrokecolor{currentstroke}%
\pgfsetdash{}{0pt}%
\pgfpathmoveto{\pgfqpoint{1.992344in}{1.955305in}}%
\pgfusepath{stroke}%
\end{pgfscope}%
\begin{pgfscope}%
\pgfpathrectangle{\pgfqpoint{0.637500in}{0.550000in}}{\pgfqpoint{3.850000in}{3.850000in}}%
\pgfusepath{clip}%
\pgfsetbuttcap%
\pgfsetroundjoin%
\definecolor{currentfill}{rgb}{1.000000,0.576471,0.309804}%
\pgfsetfillcolor{currentfill}%
\pgfsetlinewidth{1.003750pt}%
\definecolor{currentstroke}{rgb}{1.000000,0.576471,0.309804}%
\pgfsetstrokecolor{currentstroke}%
\pgfsetdash{}{0pt}%
\pgfsys@defobject{currentmarker}{\pgfqpoint{-0.033333in}{-0.033333in}}{\pgfqpoint{0.033333in}{0.033333in}}{%
\pgfpathmoveto{\pgfqpoint{0.000000in}{-0.033333in}}%
\pgfpathcurveto{\pgfqpoint{0.008840in}{-0.033333in}}{\pgfqpoint{0.017319in}{-0.029821in}}{\pgfqpoint{0.023570in}{-0.023570in}}%
\pgfpathcurveto{\pgfqpoint{0.029821in}{-0.017319in}}{\pgfqpoint{0.033333in}{-0.008840in}}{\pgfqpoint{0.033333in}{0.000000in}}%
\pgfpathcurveto{\pgfqpoint{0.033333in}{0.008840in}}{\pgfqpoint{0.029821in}{0.017319in}}{\pgfqpoint{0.023570in}{0.023570in}}%
\pgfpathcurveto{\pgfqpoint{0.017319in}{0.029821in}}{\pgfqpoint{0.008840in}{0.033333in}}{\pgfqpoint{0.000000in}{0.033333in}}%
\pgfpathcurveto{\pgfqpoint{-0.008840in}{0.033333in}}{\pgfqpoint{-0.017319in}{0.029821in}}{\pgfqpoint{-0.023570in}{0.023570in}}%
\pgfpathcurveto{\pgfqpoint{-0.029821in}{0.017319in}}{\pgfqpoint{-0.033333in}{0.008840in}}{\pgfqpoint{-0.033333in}{0.000000in}}%
\pgfpathcurveto{\pgfqpoint{-0.033333in}{-0.008840in}}{\pgfqpoint{-0.029821in}{-0.017319in}}{\pgfqpoint{-0.023570in}{-0.023570in}}%
\pgfpathcurveto{\pgfqpoint{-0.017319in}{-0.029821in}}{\pgfqpoint{-0.008840in}{-0.033333in}}{\pgfqpoint{0.000000in}{-0.033333in}}%
\pgfpathlineto{\pgfqpoint{0.000000in}{-0.033333in}}%
\pgfpathclose%
\pgfusepath{stroke,fill}%
}%
\begin{pgfscope}%
\pgfsys@transformshift{1.992344in}{1.955305in}%
\pgfsys@useobject{currentmarker}{}%
\end{pgfscope}%
\end{pgfscope}%
\begin{pgfscope}%
\pgfpathrectangle{\pgfqpoint{0.637500in}{0.550000in}}{\pgfqpoint{3.850000in}{3.850000in}}%
\pgfusepath{clip}%
\pgfsetrectcap%
\pgfsetroundjoin%
\pgfsetlinewidth{1.204500pt}%
\definecolor{currentstroke}{rgb}{1.000000,0.576471,0.309804}%
\pgfsetstrokecolor{currentstroke}%
\pgfsetdash{}{0pt}%
\pgfpathmoveto{\pgfqpoint{2.446692in}{2.003613in}}%
\pgfusepath{stroke}%
\end{pgfscope}%
\begin{pgfscope}%
\pgfpathrectangle{\pgfqpoint{0.637500in}{0.550000in}}{\pgfqpoint{3.850000in}{3.850000in}}%
\pgfusepath{clip}%
\pgfsetbuttcap%
\pgfsetroundjoin%
\definecolor{currentfill}{rgb}{1.000000,0.576471,0.309804}%
\pgfsetfillcolor{currentfill}%
\pgfsetlinewidth{1.003750pt}%
\definecolor{currentstroke}{rgb}{1.000000,0.576471,0.309804}%
\pgfsetstrokecolor{currentstroke}%
\pgfsetdash{}{0pt}%
\pgfsys@defobject{currentmarker}{\pgfqpoint{-0.033333in}{-0.033333in}}{\pgfqpoint{0.033333in}{0.033333in}}{%
\pgfpathmoveto{\pgfqpoint{0.000000in}{-0.033333in}}%
\pgfpathcurveto{\pgfqpoint{0.008840in}{-0.033333in}}{\pgfqpoint{0.017319in}{-0.029821in}}{\pgfqpoint{0.023570in}{-0.023570in}}%
\pgfpathcurveto{\pgfqpoint{0.029821in}{-0.017319in}}{\pgfqpoint{0.033333in}{-0.008840in}}{\pgfqpoint{0.033333in}{0.000000in}}%
\pgfpathcurveto{\pgfqpoint{0.033333in}{0.008840in}}{\pgfqpoint{0.029821in}{0.017319in}}{\pgfqpoint{0.023570in}{0.023570in}}%
\pgfpathcurveto{\pgfqpoint{0.017319in}{0.029821in}}{\pgfqpoint{0.008840in}{0.033333in}}{\pgfqpoint{0.000000in}{0.033333in}}%
\pgfpathcurveto{\pgfqpoint{-0.008840in}{0.033333in}}{\pgfqpoint{-0.017319in}{0.029821in}}{\pgfqpoint{-0.023570in}{0.023570in}}%
\pgfpathcurveto{\pgfqpoint{-0.029821in}{0.017319in}}{\pgfqpoint{-0.033333in}{0.008840in}}{\pgfqpoint{-0.033333in}{0.000000in}}%
\pgfpathcurveto{\pgfqpoint{-0.033333in}{-0.008840in}}{\pgfqpoint{-0.029821in}{-0.017319in}}{\pgfqpoint{-0.023570in}{-0.023570in}}%
\pgfpathcurveto{\pgfqpoint{-0.017319in}{-0.029821in}}{\pgfqpoint{-0.008840in}{-0.033333in}}{\pgfqpoint{0.000000in}{-0.033333in}}%
\pgfpathlineto{\pgfqpoint{0.000000in}{-0.033333in}}%
\pgfpathclose%
\pgfusepath{stroke,fill}%
}%
\begin{pgfscope}%
\pgfsys@transformshift{2.446692in}{2.003613in}%
\pgfsys@useobject{currentmarker}{}%
\end{pgfscope}%
\end{pgfscope}%
\begin{pgfscope}%
\pgfpathrectangle{\pgfqpoint{0.637500in}{0.550000in}}{\pgfqpoint{3.850000in}{3.850000in}}%
\pgfusepath{clip}%
\pgfsetrectcap%
\pgfsetroundjoin%
\pgfsetlinewidth{1.204500pt}%
\definecolor{currentstroke}{rgb}{1.000000,0.576471,0.309804}%
\pgfsetstrokecolor{currentstroke}%
\pgfsetdash{}{0pt}%
\pgfpathmoveto{\pgfqpoint{3.176826in}{1.964222in}}%
\pgfusepath{stroke}%
\end{pgfscope}%
\begin{pgfscope}%
\pgfpathrectangle{\pgfqpoint{0.637500in}{0.550000in}}{\pgfqpoint{3.850000in}{3.850000in}}%
\pgfusepath{clip}%
\pgfsetbuttcap%
\pgfsetroundjoin%
\definecolor{currentfill}{rgb}{1.000000,0.576471,0.309804}%
\pgfsetfillcolor{currentfill}%
\pgfsetlinewidth{1.003750pt}%
\definecolor{currentstroke}{rgb}{1.000000,0.576471,0.309804}%
\pgfsetstrokecolor{currentstroke}%
\pgfsetdash{}{0pt}%
\pgfsys@defobject{currentmarker}{\pgfqpoint{-0.033333in}{-0.033333in}}{\pgfqpoint{0.033333in}{0.033333in}}{%
\pgfpathmoveto{\pgfqpoint{0.000000in}{-0.033333in}}%
\pgfpathcurveto{\pgfqpoint{0.008840in}{-0.033333in}}{\pgfqpoint{0.017319in}{-0.029821in}}{\pgfqpoint{0.023570in}{-0.023570in}}%
\pgfpathcurveto{\pgfqpoint{0.029821in}{-0.017319in}}{\pgfqpoint{0.033333in}{-0.008840in}}{\pgfqpoint{0.033333in}{0.000000in}}%
\pgfpathcurveto{\pgfqpoint{0.033333in}{0.008840in}}{\pgfqpoint{0.029821in}{0.017319in}}{\pgfqpoint{0.023570in}{0.023570in}}%
\pgfpathcurveto{\pgfqpoint{0.017319in}{0.029821in}}{\pgfqpoint{0.008840in}{0.033333in}}{\pgfqpoint{0.000000in}{0.033333in}}%
\pgfpathcurveto{\pgfqpoint{-0.008840in}{0.033333in}}{\pgfqpoint{-0.017319in}{0.029821in}}{\pgfqpoint{-0.023570in}{0.023570in}}%
\pgfpathcurveto{\pgfqpoint{-0.029821in}{0.017319in}}{\pgfqpoint{-0.033333in}{0.008840in}}{\pgfqpoint{-0.033333in}{0.000000in}}%
\pgfpathcurveto{\pgfqpoint{-0.033333in}{-0.008840in}}{\pgfqpoint{-0.029821in}{-0.017319in}}{\pgfqpoint{-0.023570in}{-0.023570in}}%
\pgfpathcurveto{\pgfqpoint{-0.017319in}{-0.029821in}}{\pgfqpoint{-0.008840in}{-0.033333in}}{\pgfqpoint{0.000000in}{-0.033333in}}%
\pgfpathlineto{\pgfqpoint{0.000000in}{-0.033333in}}%
\pgfpathclose%
\pgfusepath{stroke,fill}%
}%
\begin{pgfscope}%
\pgfsys@transformshift{3.176826in}{1.964222in}%
\pgfsys@useobject{currentmarker}{}%
\end{pgfscope}%
\end{pgfscope}%
\begin{pgfscope}%
\pgfpathrectangle{\pgfqpoint{0.637500in}{0.550000in}}{\pgfqpoint{3.850000in}{3.850000in}}%
\pgfusepath{clip}%
\pgfsetrectcap%
\pgfsetroundjoin%
\pgfsetlinewidth{1.204500pt}%
\definecolor{currentstroke}{rgb}{1.000000,0.576471,0.309804}%
\pgfsetstrokecolor{currentstroke}%
\pgfsetdash{}{0pt}%
\pgfpathmoveto{\pgfqpoint{3.354264in}{2.654793in}}%
\pgfusepath{stroke}%
\end{pgfscope}%
\begin{pgfscope}%
\pgfpathrectangle{\pgfqpoint{0.637500in}{0.550000in}}{\pgfqpoint{3.850000in}{3.850000in}}%
\pgfusepath{clip}%
\pgfsetbuttcap%
\pgfsetroundjoin%
\definecolor{currentfill}{rgb}{1.000000,0.576471,0.309804}%
\pgfsetfillcolor{currentfill}%
\pgfsetlinewidth{1.003750pt}%
\definecolor{currentstroke}{rgb}{1.000000,0.576471,0.309804}%
\pgfsetstrokecolor{currentstroke}%
\pgfsetdash{}{0pt}%
\pgfsys@defobject{currentmarker}{\pgfqpoint{-0.033333in}{-0.033333in}}{\pgfqpoint{0.033333in}{0.033333in}}{%
\pgfpathmoveto{\pgfqpoint{0.000000in}{-0.033333in}}%
\pgfpathcurveto{\pgfqpoint{0.008840in}{-0.033333in}}{\pgfqpoint{0.017319in}{-0.029821in}}{\pgfqpoint{0.023570in}{-0.023570in}}%
\pgfpathcurveto{\pgfqpoint{0.029821in}{-0.017319in}}{\pgfqpoint{0.033333in}{-0.008840in}}{\pgfqpoint{0.033333in}{0.000000in}}%
\pgfpathcurveto{\pgfqpoint{0.033333in}{0.008840in}}{\pgfqpoint{0.029821in}{0.017319in}}{\pgfqpoint{0.023570in}{0.023570in}}%
\pgfpathcurveto{\pgfqpoint{0.017319in}{0.029821in}}{\pgfqpoint{0.008840in}{0.033333in}}{\pgfqpoint{0.000000in}{0.033333in}}%
\pgfpathcurveto{\pgfqpoint{-0.008840in}{0.033333in}}{\pgfqpoint{-0.017319in}{0.029821in}}{\pgfqpoint{-0.023570in}{0.023570in}}%
\pgfpathcurveto{\pgfqpoint{-0.029821in}{0.017319in}}{\pgfqpoint{-0.033333in}{0.008840in}}{\pgfqpoint{-0.033333in}{0.000000in}}%
\pgfpathcurveto{\pgfqpoint{-0.033333in}{-0.008840in}}{\pgfqpoint{-0.029821in}{-0.017319in}}{\pgfqpoint{-0.023570in}{-0.023570in}}%
\pgfpathcurveto{\pgfqpoint{-0.017319in}{-0.029821in}}{\pgfqpoint{-0.008840in}{-0.033333in}}{\pgfqpoint{0.000000in}{-0.033333in}}%
\pgfpathlineto{\pgfqpoint{0.000000in}{-0.033333in}}%
\pgfpathclose%
\pgfusepath{stroke,fill}%
}%
\begin{pgfscope}%
\pgfsys@transformshift{3.354264in}{2.654793in}%
\pgfsys@useobject{currentmarker}{}%
\end{pgfscope}%
\end{pgfscope}%
\begin{pgfscope}%
\pgfpathrectangle{\pgfqpoint{0.637500in}{0.550000in}}{\pgfqpoint{3.850000in}{3.850000in}}%
\pgfusepath{clip}%
\pgfsetrectcap%
\pgfsetroundjoin%
\pgfsetlinewidth{1.204500pt}%
\definecolor{currentstroke}{rgb}{1.000000,0.576471,0.309804}%
\pgfsetstrokecolor{currentstroke}%
\pgfsetdash{}{0pt}%
\pgfpathmoveto{\pgfqpoint{2.818255in}{1.434500in}}%
\pgfusepath{stroke}%
\end{pgfscope}%
\begin{pgfscope}%
\pgfpathrectangle{\pgfqpoint{0.637500in}{0.550000in}}{\pgfqpoint{3.850000in}{3.850000in}}%
\pgfusepath{clip}%
\pgfsetbuttcap%
\pgfsetroundjoin%
\definecolor{currentfill}{rgb}{1.000000,0.576471,0.309804}%
\pgfsetfillcolor{currentfill}%
\pgfsetlinewidth{1.003750pt}%
\definecolor{currentstroke}{rgb}{1.000000,0.576471,0.309804}%
\pgfsetstrokecolor{currentstroke}%
\pgfsetdash{}{0pt}%
\pgfsys@defobject{currentmarker}{\pgfqpoint{-0.033333in}{-0.033333in}}{\pgfqpoint{0.033333in}{0.033333in}}{%
\pgfpathmoveto{\pgfqpoint{0.000000in}{-0.033333in}}%
\pgfpathcurveto{\pgfqpoint{0.008840in}{-0.033333in}}{\pgfqpoint{0.017319in}{-0.029821in}}{\pgfqpoint{0.023570in}{-0.023570in}}%
\pgfpathcurveto{\pgfqpoint{0.029821in}{-0.017319in}}{\pgfqpoint{0.033333in}{-0.008840in}}{\pgfqpoint{0.033333in}{0.000000in}}%
\pgfpathcurveto{\pgfqpoint{0.033333in}{0.008840in}}{\pgfqpoint{0.029821in}{0.017319in}}{\pgfqpoint{0.023570in}{0.023570in}}%
\pgfpathcurveto{\pgfqpoint{0.017319in}{0.029821in}}{\pgfqpoint{0.008840in}{0.033333in}}{\pgfqpoint{0.000000in}{0.033333in}}%
\pgfpathcurveto{\pgfqpoint{-0.008840in}{0.033333in}}{\pgfqpoint{-0.017319in}{0.029821in}}{\pgfqpoint{-0.023570in}{0.023570in}}%
\pgfpathcurveto{\pgfqpoint{-0.029821in}{0.017319in}}{\pgfqpoint{-0.033333in}{0.008840in}}{\pgfqpoint{-0.033333in}{0.000000in}}%
\pgfpathcurveto{\pgfqpoint{-0.033333in}{-0.008840in}}{\pgfqpoint{-0.029821in}{-0.017319in}}{\pgfqpoint{-0.023570in}{-0.023570in}}%
\pgfpathcurveto{\pgfqpoint{-0.017319in}{-0.029821in}}{\pgfqpoint{-0.008840in}{-0.033333in}}{\pgfqpoint{0.000000in}{-0.033333in}}%
\pgfpathlineto{\pgfqpoint{0.000000in}{-0.033333in}}%
\pgfpathclose%
\pgfusepath{stroke,fill}%
}%
\begin{pgfscope}%
\pgfsys@transformshift{2.818255in}{1.434500in}%
\pgfsys@useobject{currentmarker}{}%
\end{pgfscope}%
\end{pgfscope}%
\begin{pgfscope}%
\pgfpathrectangle{\pgfqpoint{0.637500in}{0.550000in}}{\pgfqpoint{3.850000in}{3.850000in}}%
\pgfusepath{clip}%
\pgfsetrectcap%
\pgfsetroundjoin%
\pgfsetlinewidth{1.204500pt}%
\definecolor{currentstroke}{rgb}{1.000000,0.576471,0.309804}%
\pgfsetstrokecolor{currentstroke}%
\pgfsetdash{}{0pt}%
\pgfpathmoveto{\pgfqpoint{2.658430in}{0.877500in}}%
\pgfusepath{stroke}%
\end{pgfscope}%
\begin{pgfscope}%
\pgfpathrectangle{\pgfqpoint{0.637500in}{0.550000in}}{\pgfqpoint{3.850000in}{3.850000in}}%
\pgfusepath{clip}%
\pgfsetbuttcap%
\pgfsetroundjoin%
\definecolor{currentfill}{rgb}{1.000000,0.576471,0.309804}%
\pgfsetfillcolor{currentfill}%
\pgfsetlinewidth{1.003750pt}%
\definecolor{currentstroke}{rgb}{1.000000,0.576471,0.309804}%
\pgfsetstrokecolor{currentstroke}%
\pgfsetdash{}{0pt}%
\pgfsys@defobject{currentmarker}{\pgfqpoint{-0.033333in}{-0.033333in}}{\pgfqpoint{0.033333in}{0.033333in}}{%
\pgfpathmoveto{\pgfqpoint{0.000000in}{-0.033333in}}%
\pgfpathcurveto{\pgfqpoint{0.008840in}{-0.033333in}}{\pgfqpoint{0.017319in}{-0.029821in}}{\pgfqpoint{0.023570in}{-0.023570in}}%
\pgfpathcurveto{\pgfqpoint{0.029821in}{-0.017319in}}{\pgfqpoint{0.033333in}{-0.008840in}}{\pgfqpoint{0.033333in}{0.000000in}}%
\pgfpathcurveto{\pgfqpoint{0.033333in}{0.008840in}}{\pgfqpoint{0.029821in}{0.017319in}}{\pgfqpoint{0.023570in}{0.023570in}}%
\pgfpathcurveto{\pgfqpoint{0.017319in}{0.029821in}}{\pgfqpoint{0.008840in}{0.033333in}}{\pgfqpoint{0.000000in}{0.033333in}}%
\pgfpathcurveto{\pgfqpoint{-0.008840in}{0.033333in}}{\pgfqpoint{-0.017319in}{0.029821in}}{\pgfqpoint{-0.023570in}{0.023570in}}%
\pgfpathcurveto{\pgfqpoint{-0.029821in}{0.017319in}}{\pgfqpoint{-0.033333in}{0.008840in}}{\pgfqpoint{-0.033333in}{0.000000in}}%
\pgfpathcurveto{\pgfqpoint{-0.033333in}{-0.008840in}}{\pgfqpoint{-0.029821in}{-0.017319in}}{\pgfqpoint{-0.023570in}{-0.023570in}}%
\pgfpathcurveto{\pgfqpoint{-0.017319in}{-0.029821in}}{\pgfqpoint{-0.008840in}{-0.033333in}}{\pgfqpoint{0.000000in}{-0.033333in}}%
\pgfpathlineto{\pgfqpoint{0.000000in}{-0.033333in}}%
\pgfpathclose%
\pgfusepath{stroke,fill}%
}%
\begin{pgfscope}%
\pgfsys@transformshift{2.658430in}{0.877500in}%
\pgfsys@useobject{currentmarker}{}%
\end{pgfscope}%
\end{pgfscope}%
\begin{pgfscope}%
\pgfpathrectangle{\pgfqpoint{0.637500in}{0.550000in}}{\pgfqpoint{3.850000in}{3.850000in}}%
\pgfusepath{clip}%
\pgfsetrectcap%
\pgfsetroundjoin%
\pgfsetlinewidth{1.204500pt}%
\definecolor{currentstroke}{rgb}{1.000000,0.576471,0.309804}%
\pgfsetstrokecolor{currentstroke}%
\pgfsetdash{}{0pt}%
\pgfpathmoveto{\pgfqpoint{2.181534in}{2.621254in}}%
\pgfusepath{stroke}%
\end{pgfscope}%
\begin{pgfscope}%
\pgfpathrectangle{\pgfqpoint{0.637500in}{0.550000in}}{\pgfqpoint{3.850000in}{3.850000in}}%
\pgfusepath{clip}%
\pgfsetbuttcap%
\pgfsetroundjoin%
\definecolor{currentfill}{rgb}{1.000000,0.576471,0.309804}%
\pgfsetfillcolor{currentfill}%
\pgfsetlinewidth{1.003750pt}%
\definecolor{currentstroke}{rgb}{1.000000,0.576471,0.309804}%
\pgfsetstrokecolor{currentstroke}%
\pgfsetdash{}{0pt}%
\pgfsys@defobject{currentmarker}{\pgfqpoint{-0.033333in}{-0.033333in}}{\pgfqpoint{0.033333in}{0.033333in}}{%
\pgfpathmoveto{\pgfqpoint{0.000000in}{-0.033333in}}%
\pgfpathcurveto{\pgfqpoint{0.008840in}{-0.033333in}}{\pgfqpoint{0.017319in}{-0.029821in}}{\pgfqpoint{0.023570in}{-0.023570in}}%
\pgfpathcurveto{\pgfqpoint{0.029821in}{-0.017319in}}{\pgfqpoint{0.033333in}{-0.008840in}}{\pgfqpoint{0.033333in}{0.000000in}}%
\pgfpathcurveto{\pgfqpoint{0.033333in}{0.008840in}}{\pgfqpoint{0.029821in}{0.017319in}}{\pgfqpoint{0.023570in}{0.023570in}}%
\pgfpathcurveto{\pgfqpoint{0.017319in}{0.029821in}}{\pgfqpoint{0.008840in}{0.033333in}}{\pgfqpoint{0.000000in}{0.033333in}}%
\pgfpathcurveto{\pgfqpoint{-0.008840in}{0.033333in}}{\pgfqpoint{-0.017319in}{0.029821in}}{\pgfqpoint{-0.023570in}{0.023570in}}%
\pgfpathcurveto{\pgfqpoint{-0.029821in}{0.017319in}}{\pgfqpoint{-0.033333in}{0.008840in}}{\pgfqpoint{-0.033333in}{0.000000in}}%
\pgfpathcurveto{\pgfqpoint{-0.033333in}{-0.008840in}}{\pgfqpoint{-0.029821in}{-0.017319in}}{\pgfqpoint{-0.023570in}{-0.023570in}}%
\pgfpathcurveto{\pgfqpoint{-0.017319in}{-0.029821in}}{\pgfqpoint{-0.008840in}{-0.033333in}}{\pgfqpoint{0.000000in}{-0.033333in}}%
\pgfpathlineto{\pgfqpoint{0.000000in}{-0.033333in}}%
\pgfpathclose%
\pgfusepath{stroke,fill}%
}%
\begin{pgfscope}%
\pgfsys@transformshift{2.181534in}{2.621254in}%
\pgfsys@useobject{currentmarker}{}%
\end{pgfscope}%
\end{pgfscope}%
\begin{pgfscope}%
\pgfpathrectangle{\pgfqpoint{0.637500in}{0.550000in}}{\pgfqpoint{3.850000in}{3.850000in}}%
\pgfusepath{clip}%
\pgfsetrectcap%
\pgfsetroundjoin%
\pgfsetlinewidth{1.204500pt}%
\definecolor{currentstroke}{rgb}{1.000000,0.576471,0.309804}%
\pgfsetstrokecolor{currentstroke}%
\pgfsetdash{}{0pt}%
\pgfpathmoveto{\pgfqpoint{2.574894in}{1.086788in}}%
\pgfusepath{stroke}%
\end{pgfscope}%
\begin{pgfscope}%
\pgfpathrectangle{\pgfqpoint{0.637500in}{0.550000in}}{\pgfqpoint{3.850000in}{3.850000in}}%
\pgfusepath{clip}%
\pgfsetbuttcap%
\pgfsetroundjoin%
\definecolor{currentfill}{rgb}{1.000000,0.576471,0.309804}%
\pgfsetfillcolor{currentfill}%
\pgfsetlinewidth{1.003750pt}%
\definecolor{currentstroke}{rgb}{1.000000,0.576471,0.309804}%
\pgfsetstrokecolor{currentstroke}%
\pgfsetdash{}{0pt}%
\pgfsys@defobject{currentmarker}{\pgfqpoint{-0.033333in}{-0.033333in}}{\pgfqpoint{0.033333in}{0.033333in}}{%
\pgfpathmoveto{\pgfqpoint{0.000000in}{-0.033333in}}%
\pgfpathcurveto{\pgfqpoint{0.008840in}{-0.033333in}}{\pgfqpoint{0.017319in}{-0.029821in}}{\pgfqpoint{0.023570in}{-0.023570in}}%
\pgfpathcurveto{\pgfqpoint{0.029821in}{-0.017319in}}{\pgfqpoint{0.033333in}{-0.008840in}}{\pgfqpoint{0.033333in}{0.000000in}}%
\pgfpathcurveto{\pgfqpoint{0.033333in}{0.008840in}}{\pgfqpoint{0.029821in}{0.017319in}}{\pgfqpoint{0.023570in}{0.023570in}}%
\pgfpathcurveto{\pgfqpoint{0.017319in}{0.029821in}}{\pgfqpoint{0.008840in}{0.033333in}}{\pgfqpoint{0.000000in}{0.033333in}}%
\pgfpathcurveto{\pgfqpoint{-0.008840in}{0.033333in}}{\pgfqpoint{-0.017319in}{0.029821in}}{\pgfqpoint{-0.023570in}{0.023570in}}%
\pgfpathcurveto{\pgfqpoint{-0.029821in}{0.017319in}}{\pgfqpoint{-0.033333in}{0.008840in}}{\pgfqpoint{-0.033333in}{0.000000in}}%
\pgfpathcurveto{\pgfqpoint{-0.033333in}{-0.008840in}}{\pgfqpoint{-0.029821in}{-0.017319in}}{\pgfqpoint{-0.023570in}{-0.023570in}}%
\pgfpathcurveto{\pgfqpoint{-0.017319in}{-0.029821in}}{\pgfqpoint{-0.008840in}{-0.033333in}}{\pgfqpoint{0.000000in}{-0.033333in}}%
\pgfpathlineto{\pgfqpoint{0.000000in}{-0.033333in}}%
\pgfpathclose%
\pgfusepath{stroke,fill}%
}%
\begin{pgfscope}%
\pgfsys@transformshift{2.574894in}{1.086788in}%
\pgfsys@useobject{currentmarker}{}%
\end{pgfscope}%
\end{pgfscope}%
\begin{pgfscope}%
\pgfpathrectangle{\pgfqpoint{0.637500in}{0.550000in}}{\pgfqpoint{3.850000in}{3.850000in}}%
\pgfusepath{clip}%
\pgfsetrectcap%
\pgfsetroundjoin%
\pgfsetlinewidth{1.204500pt}%
\definecolor{currentstroke}{rgb}{1.000000,0.576471,0.309804}%
\pgfsetstrokecolor{currentstroke}%
\pgfsetdash{}{0pt}%
\pgfpathmoveto{\pgfqpoint{1.965792in}{1.906647in}}%
\pgfusepath{stroke}%
\end{pgfscope}%
\begin{pgfscope}%
\pgfpathrectangle{\pgfqpoint{0.637500in}{0.550000in}}{\pgfqpoint{3.850000in}{3.850000in}}%
\pgfusepath{clip}%
\pgfsetbuttcap%
\pgfsetroundjoin%
\definecolor{currentfill}{rgb}{1.000000,0.576471,0.309804}%
\pgfsetfillcolor{currentfill}%
\pgfsetlinewidth{1.003750pt}%
\definecolor{currentstroke}{rgb}{1.000000,0.576471,0.309804}%
\pgfsetstrokecolor{currentstroke}%
\pgfsetdash{}{0pt}%
\pgfsys@defobject{currentmarker}{\pgfqpoint{-0.033333in}{-0.033333in}}{\pgfqpoint{0.033333in}{0.033333in}}{%
\pgfpathmoveto{\pgfqpoint{0.000000in}{-0.033333in}}%
\pgfpathcurveto{\pgfqpoint{0.008840in}{-0.033333in}}{\pgfqpoint{0.017319in}{-0.029821in}}{\pgfqpoint{0.023570in}{-0.023570in}}%
\pgfpathcurveto{\pgfqpoint{0.029821in}{-0.017319in}}{\pgfqpoint{0.033333in}{-0.008840in}}{\pgfqpoint{0.033333in}{0.000000in}}%
\pgfpathcurveto{\pgfqpoint{0.033333in}{0.008840in}}{\pgfqpoint{0.029821in}{0.017319in}}{\pgfqpoint{0.023570in}{0.023570in}}%
\pgfpathcurveto{\pgfqpoint{0.017319in}{0.029821in}}{\pgfqpoint{0.008840in}{0.033333in}}{\pgfqpoint{0.000000in}{0.033333in}}%
\pgfpathcurveto{\pgfqpoint{-0.008840in}{0.033333in}}{\pgfqpoint{-0.017319in}{0.029821in}}{\pgfqpoint{-0.023570in}{0.023570in}}%
\pgfpathcurveto{\pgfqpoint{-0.029821in}{0.017319in}}{\pgfqpoint{-0.033333in}{0.008840in}}{\pgfqpoint{-0.033333in}{0.000000in}}%
\pgfpathcurveto{\pgfqpoint{-0.033333in}{-0.008840in}}{\pgfqpoint{-0.029821in}{-0.017319in}}{\pgfqpoint{-0.023570in}{-0.023570in}}%
\pgfpathcurveto{\pgfqpoint{-0.017319in}{-0.029821in}}{\pgfqpoint{-0.008840in}{-0.033333in}}{\pgfqpoint{0.000000in}{-0.033333in}}%
\pgfpathlineto{\pgfqpoint{0.000000in}{-0.033333in}}%
\pgfpathclose%
\pgfusepath{stroke,fill}%
}%
\begin{pgfscope}%
\pgfsys@transformshift{1.965792in}{1.906647in}%
\pgfsys@useobject{currentmarker}{}%
\end{pgfscope}%
\end{pgfscope}%
\begin{pgfscope}%
\pgfpathrectangle{\pgfqpoint{0.637500in}{0.550000in}}{\pgfqpoint{3.850000in}{3.850000in}}%
\pgfusepath{clip}%
\pgfsetrectcap%
\pgfsetroundjoin%
\pgfsetlinewidth{1.204500pt}%
\definecolor{currentstroke}{rgb}{1.000000,0.576471,0.309804}%
\pgfsetstrokecolor{currentstroke}%
\pgfsetdash{}{0pt}%
\pgfpathmoveto{\pgfqpoint{3.010004in}{2.256656in}}%
\pgfusepath{stroke}%
\end{pgfscope}%
\begin{pgfscope}%
\pgfpathrectangle{\pgfqpoint{0.637500in}{0.550000in}}{\pgfqpoint{3.850000in}{3.850000in}}%
\pgfusepath{clip}%
\pgfsetbuttcap%
\pgfsetroundjoin%
\definecolor{currentfill}{rgb}{1.000000,0.576471,0.309804}%
\pgfsetfillcolor{currentfill}%
\pgfsetlinewidth{1.003750pt}%
\definecolor{currentstroke}{rgb}{1.000000,0.576471,0.309804}%
\pgfsetstrokecolor{currentstroke}%
\pgfsetdash{}{0pt}%
\pgfsys@defobject{currentmarker}{\pgfqpoint{-0.033333in}{-0.033333in}}{\pgfqpoint{0.033333in}{0.033333in}}{%
\pgfpathmoveto{\pgfqpoint{0.000000in}{-0.033333in}}%
\pgfpathcurveto{\pgfqpoint{0.008840in}{-0.033333in}}{\pgfqpoint{0.017319in}{-0.029821in}}{\pgfqpoint{0.023570in}{-0.023570in}}%
\pgfpathcurveto{\pgfqpoint{0.029821in}{-0.017319in}}{\pgfqpoint{0.033333in}{-0.008840in}}{\pgfqpoint{0.033333in}{0.000000in}}%
\pgfpathcurveto{\pgfqpoint{0.033333in}{0.008840in}}{\pgfqpoint{0.029821in}{0.017319in}}{\pgfqpoint{0.023570in}{0.023570in}}%
\pgfpathcurveto{\pgfqpoint{0.017319in}{0.029821in}}{\pgfqpoint{0.008840in}{0.033333in}}{\pgfqpoint{0.000000in}{0.033333in}}%
\pgfpathcurveto{\pgfqpoint{-0.008840in}{0.033333in}}{\pgfqpoint{-0.017319in}{0.029821in}}{\pgfqpoint{-0.023570in}{0.023570in}}%
\pgfpathcurveto{\pgfqpoint{-0.029821in}{0.017319in}}{\pgfqpoint{-0.033333in}{0.008840in}}{\pgfqpoint{-0.033333in}{0.000000in}}%
\pgfpathcurveto{\pgfqpoint{-0.033333in}{-0.008840in}}{\pgfqpoint{-0.029821in}{-0.017319in}}{\pgfqpoint{-0.023570in}{-0.023570in}}%
\pgfpathcurveto{\pgfqpoint{-0.017319in}{-0.029821in}}{\pgfqpoint{-0.008840in}{-0.033333in}}{\pgfqpoint{0.000000in}{-0.033333in}}%
\pgfpathlineto{\pgfqpoint{0.000000in}{-0.033333in}}%
\pgfpathclose%
\pgfusepath{stroke,fill}%
}%
\begin{pgfscope}%
\pgfsys@transformshift{3.010004in}{2.256656in}%
\pgfsys@useobject{currentmarker}{}%
\end{pgfscope}%
\end{pgfscope}%
\begin{pgfscope}%
\pgfpathrectangle{\pgfqpoint{0.637500in}{0.550000in}}{\pgfqpoint{3.850000in}{3.850000in}}%
\pgfusepath{clip}%
\pgfsetrectcap%
\pgfsetroundjoin%
\pgfsetlinewidth{1.204500pt}%
\definecolor{currentstroke}{rgb}{1.000000,0.576471,0.309804}%
\pgfsetstrokecolor{currentstroke}%
\pgfsetdash{}{0pt}%
\pgfpathmoveto{\pgfqpoint{1.476177in}{1.757017in}}%
\pgfusepath{stroke}%
\end{pgfscope}%
\begin{pgfscope}%
\pgfpathrectangle{\pgfqpoint{0.637500in}{0.550000in}}{\pgfqpoint{3.850000in}{3.850000in}}%
\pgfusepath{clip}%
\pgfsetbuttcap%
\pgfsetroundjoin%
\definecolor{currentfill}{rgb}{1.000000,0.576471,0.309804}%
\pgfsetfillcolor{currentfill}%
\pgfsetlinewidth{1.003750pt}%
\definecolor{currentstroke}{rgb}{1.000000,0.576471,0.309804}%
\pgfsetstrokecolor{currentstroke}%
\pgfsetdash{}{0pt}%
\pgfsys@defobject{currentmarker}{\pgfqpoint{-0.033333in}{-0.033333in}}{\pgfqpoint{0.033333in}{0.033333in}}{%
\pgfpathmoveto{\pgfqpoint{0.000000in}{-0.033333in}}%
\pgfpathcurveto{\pgfqpoint{0.008840in}{-0.033333in}}{\pgfqpoint{0.017319in}{-0.029821in}}{\pgfqpoint{0.023570in}{-0.023570in}}%
\pgfpathcurveto{\pgfqpoint{0.029821in}{-0.017319in}}{\pgfqpoint{0.033333in}{-0.008840in}}{\pgfqpoint{0.033333in}{0.000000in}}%
\pgfpathcurveto{\pgfqpoint{0.033333in}{0.008840in}}{\pgfqpoint{0.029821in}{0.017319in}}{\pgfqpoint{0.023570in}{0.023570in}}%
\pgfpathcurveto{\pgfqpoint{0.017319in}{0.029821in}}{\pgfqpoint{0.008840in}{0.033333in}}{\pgfqpoint{0.000000in}{0.033333in}}%
\pgfpathcurveto{\pgfqpoint{-0.008840in}{0.033333in}}{\pgfqpoint{-0.017319in}{0.029821in}}{\pgfqpoint{-0.023570in}{0.023570in}}%
\pgfpathcurveto{\pgfqpoint{-0.029821in}{0.017319in}}{\pgfqpoint{-0.033333in}{0.008840in}}{\pgfqpoint{-0.033333in}{0.000000in}}%
\pgfpathcurveto{\pgfqpoint{-0.033333in}{-0.008840in}}{\pgfqpoint{-0.029821in}{-0.017319in}}{\pgfqpoint{-0.023570in}{-0.023570in}}%
\pgfpathcurveto{\pgfqpoint{-0.017319in}{-0.029821in}}{\pgfqpoint{-0.008840in}{-0.033333in}}{\pgfqpoint{0.000000in}{-0.033333in}}%
\pgfpathlineto{\pgfqpoint{0.000000in}{-0.033333in}}%
\pgfpathclose%
\pgfusepath{stroke,fill}%
}%
\begin{pgfscope}%
\pgfsys@transformshift{1.476177in}{1.757017in}%
\pgfsys@useobject{currentmarker}{}%
\end{pgfscope}%
\end{pgfscope}%
\begin{pgfscope}%
\pgfpathrectangle{\pgfqpoint{0.637500in}{0.550000in}}{\pgfqpoint{3.850000in}{3.850000in}}%
\pgfusepath{clip}%
\pgfsetrectcap%
\pgfsetroundjoin%
\pgfsetlinewidth{1.204500pt}%
\definecolor{currentstroke}{rgb}{1.000000,0.576471,0.309804}%
\pgfsetstrokecolor{currentstroke}%
\pgfsetdash{}{0pt}%
\pgfpathmoveto{\pgfqpoint{2.605684in}{2.020418in}}%
\pgfusepath{stroke}%
\end{pgfscope}%
\begin{pgfscope}%
\pgfpathrectangle{\pgfqpoint{0.637500in}{0.550000in}}{\pgfqpoint{3.850000in}{3.850000in}}%
\pgfusepath{clip}%
\pgfsetbuttcap%
\pgfsetroundjoin%
\definecolor{currentfill}{rgb}{1.000000,0.576471,0.309804}%
\pgfsetfillcolor{currentfill}%
\pgfsetlinewidth{1.003750pt}%
\definecolor{currentstroke}{rgb}{1.000000,0.576471,0.309804}%
\pgfsetstrokecolor{currentstroke}%
\pgfsetdash{}{0pt}%
\pgfsys@defobject{currentmarker}{\pgfqpoint{-0.033333in}{-0.033333in}}{\pgfqpoint{0.033333in}{0.033333in}}{%
\pgfpathmoveto{\pgfqpoint{0.000000in}{-0.033333in}}%
\pgfpathcurveto{\pgfqpoint{0.008840in}{-0.033333in}}{\pgfqpoint{0.017319in}{-0.029821in}}{\pgfqpoint{0.023570in}{-0.023570in}}%
\pgfpathcurveto{\pgfqpoint{0.029821in}{-0.017319in}}{\pgfqpoint{0.033333in}{-0.008840in}}{\pgfqpoint{0.033333in}{0.000000in}}%
\pgfpathcurveto{\pgfqpoint{0.033333in}{0.008840in}}{\pgfqpoint{0.029821in}{0.017319in}}{\pgfqpoint{0.023570in}{0.023570in}}%
\pgfpathcurveto{\pgfqpoint{0.017319in}{0.029821in}}{\pgfqpoint{0.008840in}{0.033333in}}{\pgfqpoint{0.000000in}{0.033333in}}%
\pgfpathcurveto{\pgfqpoint{-0.008840in}{0.033333in}}{\pgfqpoint{-0.017319in}{0.029821in}}{\pgfqpoint{-0.023570in}{0.023570in}}%
\pgfpathcurveto{\pgfqpoint{-0.029821in}{0.017319in}}{\pgfqpoint{-0.033333in}{0.008840in}}{\pgfqpoint{-0.033333in}{0.000000in}}%
\pgfpathcurveto{\pgfqpoint{-0.033333in}{-0.008840in}}{\pgfqpoint{-0.029821in}{-0.017319in}}{\pgfqpoint{-0.023570in}{-0.023570in}}%
\pgfpathcurveto{\pgfqpoint{-0.017319in}{-0.029821in}}{\pgfqpoint{-0.008840in}{-0.033333in}}{\pgfqpoint{0.000000in}{-0.033333in}}%
\pgfpathlineto{\pgfqpoint{0.000000in}{-0.033333in}}%
\pgfpathclose%
\pgfusepath{stroke,fill}%
}%
\begin{pgfscope}%
\pgfsys@transformshift{2.605684in}{2.020418in}%
\pgfsys@useobject{currentmarker}{}%
\end{pgfscope}%
\end{pgfscope}%
\begin{pgfscope}%
\pgfpathrectangle{\pgfqpoint{0.637500in}{0.550000in}}{\pgfqpoint{3.850000in}{3.850000in}}%
\pgfusepath{clip}%
\pgfsetrectcap%
\pgfsetroundjoin%
\pgfsetlinewidth{1.204500pt}%
\definecolor{currentstroke}{rgb}{1.000000,0.576471,0.309804}%
\pgfsetstrokecolor{currentstroke}%
\pgfsetdash{}{0pt}%
\pgfpathmoveto{\pgfqpoint{1.757742in}{2.653790in}}%
\pgfusepath{stroke}%
\end{pgfscope}%
\begin{pgfscope}%
\pgfpathrectangle{\pgfqpoint{0.637500in}{0.550000in}}{\pgfqpoint{3.850000in}{3.850000in}}%
\pgfusepath{clip}%
\pgfsetbuttcap%
\pgfsetroundjoin%
\definecolor{currentfill}{rgb}{1.000000,0.576471,0.309804}%
\pgfsetfillcolor{currentfill}%
\pgfsetlinewidth{1.003750pt}%
\definecolor{currentstroke}{rgb}{1.000000,0.576471,0.309804}%
\pgfsetstrokecolor{currentstroke}%
\pgfsetdash{}{0pt}%
\pgfsys@defobject{currentmarker}{\pgfqpoint{-0.033333in}{-0.033333in}}{\pgfqpoint{0.033333in}{0.033333in}}{%
\pgfpathmoveto{\pgfqpoint{0.000000in}{-0.033333in}}%
\pgfpathcurveto{\pgfqpoint{0.008840in}{-0.033333in}}{\pgfqpoint{0.017319in}{-0.029821in}}{\pgfqpoint{0.023570in}{-0.023570in}}%
\pgfpathcurveto{\pgfqpoint{0.029821in}{-0.017319in}}{\pgfqpoint{0.033333in}{-0.008840in}}{\pgfqpoint{0.033333in}{0.000000in}}%
\pgfpathcurveto{\pgfqpoint{0.033333in}{0.008840in}}{\pgfqpoint{0.029821in}{0.017319in}}{\pgfqpoint{0.023570in}{0.023570in}}%
\pgfpathcurveto{\pgfqpoint{0.017319in}{0.029821in}}{\pgfqpoint{0.008840in}{0.033333in}}{\pgfqpoint{0.000000in}{0.033333in}}%
\pgfpathcurveto{\pgfqpoint{-0.008840in}{0.033333in}}{\pgfqpoint{-0.017319in}{0.029821in}}{\pgfqpoint{-0.023570in}{0.023570in}}%
\pgfpathcurveto{\pgfqpoint{-0.029821in}{0.017319in}}{\pgfqpoint{-0.033333in}{0.008840in}}{\pgfqpoint{-0.033333in}{0.000000in}}%
\pgfpathcurveto{\pgfqpoint{-0.033333in}{-0.008840in}}{\pgfqpoint{-0.029821in}{-0.017319in}}{\pgfqpoint{-0.023570in}{-0.023570in}}%
\pgfpathcurveto{\pgfqpoint{-0.017319in}{-0.029821in}}{\pgfqpoint{-0.008840in}{-0.033333in}}{\pgfqpoint{0.000000in}{-0.033333in}}%
\pgfpathlineto{\pgfqpoint{0.000000in}{-0.033333in}}%
\pgfpathclose%
\pgfusepath{stroke,fill}%
}%
\begin{pgfscope}%
\pgfsys@transformshift{1.757742in}{2.653790in}%
\pgfsys@useobject{currentmarker}{}%
\end{pgfscope}%
\end{pgfscope}%
\begin{pgfscope}%
\pgfpathrectangle{\pgfqpoint{0.637500in}{0.550000in}}{\pgfqpoint{3.850000in}{3.850000in}}%
\pgfusepath{clip}%
\pgfsetrectcap%
\pgfsetroundjoin%
\pgfsetlinewidth{1.204500pt}%
\definecolor{currentstroke}{rgb}{1.000000,0.576471,0.309804}%
\pgfsetstrokecolor{currentstroke}%
\pgfsetdash{}{0pt}%
\pgfpathmoveto{\pgfqpoint{1.471262in}{1.861905in}}%
\pgfusepath{stroke}%
\end{pgfscope}%
\begin{pgfscope}%
\pgfpathrectangle{\pgfqpoint{0.637500in}{0.550000in}}{\pgfqpoint{3.850000in}{3.850000in}}%
\pgfusepath{clip}%
\pgfsetbuttcap%
\pgfsetroundjoin%
\definecolor{currentfill}{rgb}{1.000000,0.576471,0.309804}%
\pgfsetfillcolor{currentfill}%
\pgfsetlinewidth{1.003750pt}%
\definecolor{currentstroke}{rgb}{1.000000,0.576471,0.309804}%
\pgfsetstrokecolor{currentstroke}%
\pgfsetdash{}{0pt}%
\pgfsys@defobject{currentmarker}{\pgfqpoint{-0.033333in}{-0.033333in}}{\pgfqpoint{0.033333in}{0.033333in}}{%
\pgfpathmoveto{\pgfqpoint{0.000000in}{-0.033333in}}%
\pgfpathcurveto{\pgfqpoint{0.008840in}{-0.033333in}}{\pgfqpoint{0.017319in}{-0.029821in}}{\pgfqpoint{0.023570in}{-0.023570in}}%
\pgfpathcurveto{\pgfqpoint{0.029821in}{-0.017319in}}{\pgfqpoint{0.033333in}{-0.008840in}}{\pgfqpoint{0.033333in}{0.000000in}}%
\pgfpathcurveto{\pgfqpoint{0.033333in}{0.008840in}}{\pgfqpoint{0.029821in}{0.017319in}}{\pgfqpoint{0.023570in}{0.023570in}}%
\pgfpathcurveto{\pgfqpoint{0.017319in}{0.029821in}}{\pgfqpoint{0.008840in}{0.033333in}}{\pgfqpoint{0.000000in}{0.033333in}}%
\pgfpathcurveto{\pgfqpoint{-0.008840in}{0.033333in}}{\pgfqpoint{-0.017319in}{0.029821in}}{\pgfqpoint{-0.023570in}{0.023570in}}%
\pgfpathcurveto{\pgfqpoint{-0.029821in}{0.017319in}}{\pgfqpoint{-0.033333in}{0.008840in}}{\pgfqpoint{-0.033333in}{0.000000in}}%
\pgfpathcurveto{\pgfqpoint{-0.033333in}{-0.008840in}}{\pgfqpoint{-0.029821in}{-0.017319in}}{\pgfqpoint{-0.023570in}{-0.023570in}}%
\pgfpathcurveto{\pgfqpoint{-0.017319in}{-0.029821in}}{\pgfqpoint{-0.008840in}{-0.033333in}}{\pgfqpoint{0.000000in}{-0.033333in}}%
\pgfpathlineto{\pgfqpoint{0.000000in}{-0.033333in}}%
\pgfpathclose%
\pgfusepath{stroke,fill}%
}%
\begin{pgfscope}%
\pgfsys@transformshift{1.471262in}{1.861905in}%
\pgfsys@useobject{currentmarker}{}%
\end{pgfscope}%
\end{pgfscope}%
\begin{pgfscope}%
\pgfpathrectangle{\pgfqpoint{0.637500in}{0.550000in}}{\pgfqpoint{3.850000in}{3.850000in}}%
\pgfusepath{clip}%
\pgfsetrectcap%
\pgfsetroundjoin%
\pgfsetlinewidth{1.204500pt}%
\definecolor{currentstroke}{rgb}{1.000000,0.576471,0.309804}%
\pgfsetstrokecolor{currentstroke}%
\pgfsetdash{}{0pt}%
\pgfpathmoveto{\pgfqpoint{3.260024in}{1.550070in}}%
\pgfusepath{stroke}%
\end{pgfscope}%
\begin{pgfscope}%
\pgfpathrectangle{\pgfqpoint{0.637500in}{0.550000in}}{\pgfqpoint{3.850000in}{3.850000in}}%
\pgfusepath{clip}%
\pgfsetbuttcap%
\pgfsetroundjoin%
\definecolor{currentfill}{rgb}{1.000000,0.576471,0.309804}%
\pgfsetfillcolor{currentfill}%
\pgfsetlinewidth{1.003750pt}%
\definecolor{currentstroke}{rgb}{1.000000,0.576471,0.309804}%
\pgfsetstrokecolor{currentstroke}%
\pgfsetdash{}{0pt}%
\pgfsys@defobject{currentmarker}{\pgfqpoint{-0.033333in}{-0.033333in}}{\pgfqpoint{0.033333in}{0.033333in}}{%
\pgfpathmoveto{\pgfqpoint{0.000000in}{-0.033333in}}%
\pgfpathcurveto{\pgfqpoint{0.008840in}{-0.033333in}}{\pgfqpoint{0.017319in}{-0.029821in}}{\pgfqpoint{0.023570in}{-0.023570in}}%
\pgfpathcurveto{\pgfqpoint{0.029821in}{-0.017319in}}{\pgfqpoint{0.033333in}{-0.008840in}}{\pgfqpoint{0.033333in}{0.000000in}}%
\pgfpathcurveto{\pgfqpoint{0.033333in}{0.008840in}}{\pgfqpoint{0.029821in}{0.017319in}}{\pgfqpoint{0.023570in}{0.023570in}}%
\pgfpathcurveto{\pgfqpoint{0.017319in}{0.029821in}}{\pgfqpoint{0.008840in}{0.033333in}}{\pgfqpoint{0.000000in}{0.033333in}}%
\pgfpathcurveto{\pgfqpoint{-0.008840in}{0.033333in}}{\pgfqpoint{-0.017319in}{0.029821in}}{\pgfqpoint{-0.023570in}{0.023570in}}%
\pgfpathcurveto{\pgfqpoint{-0.029821in}{0.017319in}}{\pgfqpoint{-0.033333in}{0.008840in}}{\pgfqpoint{-0.033333in}{0.000000in}}%
\pgfpathcurveto{\pgfqpoint{-0.033333in}{-0.008840in}}{\pgfqpoint{-0.029821in}{-0.017319in}}{\pgfqpoint{-0.023570in}{-0.023570in}}%
\pgfpathcurveto{\pgfqpoint{-0.017319in}{-0.029821in}}{\pgfqpoint{-0.008840in}{-0.033333in}}{\pgfqpoint{0.000000in}{-0.033333in}}%
\pgfpathlineto{\pgfqpoint{0.000000in}{-0.033333in}}%
\pgfpathclose%
\pgfusepath{stroke,fill}%
}%
\begin{pgfscope}%
\pgfsys@transformshift{3.260024in}{1.550070in}%
\pgfsys@useobject{currentmarker}{}%
\end{pgfscope}%
\end{pgfscope}%
\begin{pgfscope}%
\pgfpathrectangle{\pgfqpoint{0.637500in}{0.550000in}}{\pgfqpoint{3.850000in}{3.850000in}}%
\pgfusepath{clip}%
\pgfsetrectcap%
\pgfsetroundjoin%
\pgfsetlinewidth{1.204500pt}%
\definecolor{currentstroke}{rgb}{1.000000,0.576471,0.309804}%
\pgfsetstrokecolor{currentstroke}%
\pgfsetdash{}{0pt}%
\pgfpathmoveto{\pgfqpoint{2.337342in}{2.266589in}}%
\pgfusepath{stroke}%
\end{pgfscope}%
\begin{pgfscope}%
\pgfpathrectangle{\pgfqpoint{0.637500in}{0.550000in}}{\pgfqpoint{3.850000in}{3.850000in}}%
\pgfusepath{clip}%
\pgfsetbuttcap%
\pgfsetroundjoin%
\definecolor{currentfill}{rgb}{1.000000,0.576471,0.309804}%
\pgfsetfillcolor{currentfill}%
\pgfsetlinewidth{1.003750pt}%
\definecolor{currentstroke}{rgb}{1.000000,0.576471,0.309804}%
\pgfsetstrokecolor{currentstroke}%
\pgfsetdash{}{0pt}%
\pgfsys@defobject{currentmarker}{\pgfqpoint{-0.033333in}{-0.033333in}}{\pgfqpoint{0.033333in}{0.033333in}}{%
\pgfpathmoveto{\pgfqpoint{0.000000in}{-0.033333in}}%
\pgfpathcurveto{\pgfqpoint{0.008840in}{-0.033333in}}{\pgfqpoint{0.017319in}{-0.029821in}}{\pgfqpoint{0.023570in}{-0.023570in}}%
\pgfpathcurveto{\pgfqpoint{0.029821in}{-0.017319in}}{\pgfqpoint{0.033333in}{-0.008840in}}{\pgfqpoint{0.033333in}{0.000000in}}%
\pgfpathcurveto{\pgfqpoint{0.033333in}{0.008840in}}{\pgfqpoint{0.029821in}{0.017319in}}{\pgfqpoint{0.023570in}{0.023570in}}%
\pgfpathcurveto{\pgfqpoint{0.017319in}{0.029821in}}{\pgfqpoint{0.008840in}{0.033333in}}{\pgfqpoint{0.000000in}{0.033333in}}%
\pgfpathcurveto{\pgfqpoint{-0.008840in}{0.033333in}}{\pgfqpoint{-0.017319in}{0.029821in}}{\pgfqpoint{-0.023570in}{0.023570in}}%
\pgfpathcurveto{\pgfqpoint{-0.029821in}{0.017319in}}{\pgfqpoint{-0.033333in}{0.008840in}}{\pgfqpoint{-0.033333in}{0.000000in}}%
\pgfpathcurveto{\pgfqpoint{-0.033333in}{-0.008840in}}{\pgfqpoint{-0.029821in}{-0.017319in}}{\pgfqpoint{-0.023570in}{-0.023570in}}%
\pgfpathcurveto{\pgfqpoint{-0.017319in}{-0.029821in}}{\pgfqpoint{-0.008840in}{-0.033333in}}{\pgfqpoint{0.000000in}{-0.033333in}}%
\pgfpathlineto{\pgfqpoint{0.000000in}{-0.033333in}}%
\pgfpathclose%
\pgfusepath{stroke,fill}%
}%
\begin{pgfscope}%
\pgfsys@transformshift{2.337342in}{2.266589in}%
\pgfsys@useobject{currentmarker}{}%
\end{pgfscope}%
\end{pgfscope}%
\begin{pgfscope}%
\pgfpathrectangle{\pgfqpoint{0.637500in}{0.550000in}}{\pgfqpoint{3.850000in}{3.850000in}}%
\pgfusepath{clip}%
\pgfsetrectcap%
\pgfsetroundjoin%
\pgfsetlinewidth{1.204500pt}%
\definecolor{currentstroke}{rgb}{1.000000,0.576471,0.309804}%
\pgfsetstrokecolor{currentstroke}%
\pgfsetdash{}{0pt}%
\pgfpathmoveto{\pgfqpoint{2.044267in}{1.636700in}}%
\pgfusepath{stroke}%
\end{pgfscope}%
\begin{pgfscope}%
\pgfpathrectangle{\pgfqpoint{0.637500in}{0.550000in}}{\pgfqpoint{3.850000in}{3.850000in}}%
\pgfusepath{clip}%
\pgfsetbuttcap%
\pgfsetroundjoin%
\definecolor{currentfill}{rgb}{1.000000,0.576471,0.309804}%
\pgfsetfillcolor{currentfill}%
\pgfsetlinewidth{1.003750pt}%
\definecolor{currentstroke}{rgb}{1.000000,0.576471,0.309804}%
\pgfsetstrokecolor{currentstroke}%
\pgfsetdash{}{0pt}%
\pgfsys@defobject{currentmarker}{\pgfqpoint{-0.033333in}{-0.033333in}}{\pgfqpoint{0.033333in}{0.033333in}}{%
\pgfpathmoveto{\pgfqpoint{0.000000in}{-0.033333in}}%
\pgfpathcurveto{\pgfqpoint{0.008840in}{-0.033333in}}{\pgfqpoint{0.017319in}{-0.029821in}}{\pgfqpoint{0.023570in}{-0.023570in}}%
\pgfpathcurveto{\pgfqpoint{0.029821in}{-0.017319in}}{\pgfqpoint{0.033333in}{-0.008840in}}{\pgfqpoint{0.033333in}{0.000000in}}%
\pgfpathcurveto{\pgfqpoint{0.033333in}{0.008840in}}{\pgfqpoint{0.029821in}{0.017319in}}{\pgfqpoint{0.023570in}{0.023570in}}%
\pgfpathcurveto{\pgfqpoint{0.017319in}{0.029821in}}{\pgfqpoint{0.008840in}{0.033333in}}{\pgfqpoint{0.000000in}{0.033333in}}%
\pgfpathcurveto{\pgfqpoint{-0.008840in}{0.033333in}}{\pgfqpoint{-0.017319in}{0.029821in}}{\pgfqpoint{-0.023570in}{0.023570in}}%
\pgfpathcurveto{\pgfqpoint{-0.029821in}{0.017319in}}{\pgfqpoint{-0.033333in}{0.008840in}}{\pgfqpoint{-0.033333in}{0.000000in}}%
\pgfpathcurveto{\pgfqpoint{-0.033333in}{-0.008840in}}{\pgfqpoint{-0.029821in}{-0.017319in}}{\pgfqpoint{-0.023570in}{-0.023570in}}%
\pgfpathcurveto{\pgfqpoint{-0.017319in}{-0.029821in}}{\pgfqpoint{-0.008840in}{-0.033333in}}{\pgfqpoint{0.000000in}{-0.033333in}}%
\pgfpathlineto{\pgfqpoint{0.000000in}{-0.033333in}}%
\pgfpathclose%
\pgfusepath{stroke,fill}%
}%
\begin{pgfscope}%
\pgfsys@transformshift{2.044267in}{1.636700in}%
\pgfsys@useobject{currentmarker}{}%
\end{pgfscope}%
\end{pgfscope}%
\begin{pgfscope}%
\pgfpathrectangle{\pgfqpoint{0.637500in}{0.550000in}}{\pgfqpoint{3.850000in}{3.850000in}}%
\pgfusepath{clip}%
\pgfsetrectcap%
\pgfsetroundjoin%
\pgfsetlinewidth{1.204500pt}%
\definecolor{currentstroke}{rgb}{1.000000,0.576471,0.309804}%
\pgfsetstrokecolor{currentstroke}%
\pgfsetdash{}{0pt}%
\pgfpathmoveto{\pgfqpoint{2.438912in}{1.524087in}}%
\pgfusepath{stroke}%
\end{pgfscope}%
\begin{pgfscope}%
\pgfpathrectangle{\pgfqpoint{0.637500in}{0.550000in}}{\pgfqpoint{3.850000in}{3.850000in}}%
\pgfusepath{clip}%
\pgfsetbuttcap%
\pgfsetroundjoin%
\definecolor{currentfill}{rgb}{1.000000,0.576471,0.309804}%
\pgfsetfillcolor{currentfill}%
\pgfsetlinewidth{1.003750pt}%
\definecolor{currentstroke}{rgb}{1.000000,0.576471,0.309804}%
\pgfsetstrokecolor{currentstroke}%
\pgfsetdash{}{0pt}%
\pgfsys@defobject{currentmarker}{\pgfqpoint{-0.033333in}{-0.033333in}}{\pgfqpoint{0.033333in}{0.033333in}}{%
\pgfpathmoveto{\pgfqpoint{0.000000in}{-0.033333in}}%
\pgfpathcurveto{\pgfqpoint{0.008840in}{-0.033333in}}{\pgfqpoint{0.017319in}{-0.029821in}}{\pgfqpoint{0.023570in}{-0.023570in}}%
\pgfpathcurveto{\pgfqpoint{0.029821in}{-0.017319in}}{\pgfqpoint{0.033333in}{-0.008840in}}{\pgfqpoint{0.033333in}{0.000000in}}%
\pgfpathcurveto{\pgfqpoint{0.033333in}{0.008840in}}{\pgfqpoint{0.029821in}{0.017319in}}{\pgfqpoint{0.023570in}{0.023570in}}%
\pgfpathcurveto{\pgfqpoint{0.017319in}{0.029821in}}{\pgfqpoint{0.008840in}{0.033333in}}{\pgfqpoint{0.000000in}{0.033333in}}%
\pgfpathcurveto{\pgfqpoint{-0.008840in}{0.033333in}}{\pgfqpoint{-0.017319in}{0.029821in}}{\pgfqpoint{-0.023570in}{0.023570in}}%
\pgfpathcurveto{\pgfqpoint{-0.029821in}{0.017319in}}{\pgfqpoint{-0.033333in}{0.008840in}}{\pgfqpoint{-0.033333in}{0.000000in}}%
\pgfpathcurveto{\pgfqpoint{-0.033333in}{-0.008840in}}{\pgfqpoint{-0.029821in}{-0.017319in}}{\pgfqpoint{-0.023570in}{-0.023570in}}%
\pgfpathcurveto{\pgfqpoint{-0.017319in}{-0.029821in}}{\pgfqpoint{-0.008840in}{-0.033333in}}{\pgfqpoint{0.000000in}{-0.033333in}}%
\pgfpathlineto{\pgfqpoint{0.000000in}{-0.033333in}}%
\pgfpathclose%
\pgfusepath{stroke,fill}%
}%
\begin{pgfscope}%
\pgfsys@transformshift{2.438912in}{1.524087in}%
\pgfsys@useobject{currentmarker}{}%
\end{pgfscope}%
\end{pgfscope}%
\begin{pgfscope}%
\pgfpathrectangle{\pgfqpoint{0.637500in}{0.550000in}}{\pgfqpoint{3.850000in}{3.850000in}}%
\pgfusepath{clip}%
\pgfsetrectcap%
\pgfsetroundjoin%
\pgfsetlinewidth{1.204500pt}%
\definecolor{currentstroke}{rgb}{1.000000,0.576471,0.309804}%
\pgfsetstrokecolor{currentstroke}%
\pgfsetdash{}{0pt}%
\pgfpathmoveto{\pgfqpoint{3.656675in}{2.704588in}}%
\pgfusepath{stroke}%
\end{pgfscope}%
\begin{pgfscope}%
\pgfpathrectangle{\pgfqpoint{0.637500in}{0.550000in}}{\pgfqpoint{3.850000in}{3.850000in}}%
\pgfusepath{clip}%
\pgfsetbuttcap%
\pgfsetroundjoin%
\definecolor{currentfill}{rgb}{1.000000,0.576471,0.309804}%
\pgfsetfillcolor{currentfill}%
\pgfsetlinewidth{1.003750pt}%
\definecolor{currentstroke}{rgb}{1.000000,0.576471,0.309804}%
\pgfsetstrokecolor{currentstroke}%
\pgfsetdash{}{0pt}%
\pgfsys@defobject{currentmarker}{\pgfqpoint{-0.033333in}{-0.033333in}}{\pgfqpoint{0.033333in}{0.033333in}}{%
\pgfpathmoveto{\pgfqpoint{0.000000in}{-0.033333in}}%
\pgfpathcurveto{\pgfqpoint{0.008840in}{-0.033333in}}{\pgfqpoint{0.017319in}{-0.029821in}}{\pgfqpoint{0.023570in}{-0.023570in}}%
\pgfpathcurveto{\pgfqpoint{0.029821in}{-0.017319in}}{\pgfqpoint{0.033333in}{-0.008840in}}{\pgfqpoint{0.033333in}{0.000000in}}%
\pgfpathcurveto{\pgfqpoint{0.033333in}{0.008840in}}{\pgfqpoint{0.029821in}{0.017319in}}{\pgfqpoint{0.023570in}{0.023570in}}%
\pgfpathcurveto{\pgfqpoint{0.017319in}{0.029821in}}{\pgfqpoint{0.008840in}{0.033333in}}{\pgfqpoint{0.000000in}{0.033333in}}%
\pgfpathcurveto{\pgfqpoint{-0.008840in}{0.033333in}}{\pgfqpoint{-0.017319in}{0.029821in}}{\pgfqpoint{-0.023570in}{0.023570in}}%
\pgfpathcurveto{\pgfqpoint{-0.029821in}{0.017319in}}{\pgfqpoint{-0.033333in}{0.008840in}}{\pgfqpoint{-0.033333in}{0.000000in}}%
\pgfpathcurveto{\pgfqpoint{-0.033333in}{-0.008840in}}{\pgfqpoint{-0.029821in}{-0.017319in}}{\pgfqpoint{-0.023570in}{-0.023570in}}%
\pgfpathcurveto{\pgfqpoint{-0.017319in}{-0.029821in}}{\pgfqpoint{-0.008840in}{-0.033333in}}{\pgfqpoint{0.000000in}{-0.033333in}}%
\pgfpathlineto{\pgfqpoint{0.000000in}{-0.033333in}}%
\pgfpathclose%
\pgfusepath{stroke,fill}%
}%
\begin{pgfscope}%
\pgfsys@transformshift{3.656675in}{2.704588in}%
\pgfsys@useobject{currentmarker}{}%
\end{pgfscope}%
\end{pgfscope}%
\begin{pgfscope}%
\pgfpathrectangle{\pgfqpoint{0.637500in}{0.550000in}}{\pgfqpoint{3.850000in}{3.850000in}}%
\pgfusepath{clip}%
\pgfsetrectcap%
\pgfsetroundjoin%
\pgfsetlinewidth{1.204500pt}%
\definecolor{currentstroke}{rgb}{1.000000,0.576471,0.309804}%
\pgfsetstrokecolor{currentstroke}%
\pgfsetdash{}{0pt}%
\pgfpathmoveto{\pgfqpoint{1.994083in}{2.158285in}}%
\pgfusepath{stroke}%
\end{pgfscope}%
\begin{pgfscope}%
\pgfpathrectangle{\pgfqpoint{0.637500in}{0.550000in}}{\pgfqpoint{3.850000in}{3.850000in}}%
\pgfusepath{clip}%
\pgfsetbuttcap%
\pgfsetroundjoin%
\definecolor{currentfill}{rgb}{1.000000,0.576471,0.309804}%
\pgfsetfillcolor{currentfill}%
\pgfsetlinewidth{1.003750pt}%
\definecolor{currentstroke}{rgb}{1.000000,0.576471,0.309804}%
\pgfsetstrokecolor{currentstroke}%
\pgfsetdash{}{0pt}%
\pgfsys@defobject{currentmarker}{\pgfqpoint{-0.033333in}{-0.033333in}}{\pgfqpoint{0.033333in}{0.033333in}}{%
\pgfpathmoveto{\pgfqpoint{0.000000in}{-0.033333in}}%
\pgfpathcurveto{\pgfqpoint{0.008840in}{-0.033333in}}{\pgfqpoint{0.017319in}{-0.029821in}}{\pgfqpoint{0.023570in}{-0.023570in}}%
\pgfpathcurveto{\pgfqpoint{0.029821in}{-0.017319in}}{\pgfqpoint{0.033333in}{-0.008840in}}{\pgfqpoint{0.033333in}{0.000000in}}%
\pgfpathcurveto{\pgfqpoint{0.033333in}{0.008840in}}{\pgfqpoint{0.029821in}{0.017319in}}{\pgfqpoint{0.023570in}{0.023570in}}%
\pgfpathcurveto{\pgfqpoint{0.017319in}{0.029821in}}{\pgfqpoint{0.008840in}{0.033333in}}{\pgfqpoint{0.000000in}{0.033333in}}%
\pgfpathcurveto{\pgfqpoint{-0.008840in}{0.033333in}}{\pgfqpoint{-0.017319in}{0.029821in}}{\pgfqpoint{-0.023570in}{0.023570in}}%
\pgfpathcurveto{\pgfqpoint{-0.029821in}{0.017319in}}{\pgfqpoint{-0.033333in}{0.008840in}}{\pgfqpoint{-0.033333in}{0.000000in}}%
\pgfpathcurveto{\pgfqpoint{-0.033333in}{-0.008840in}}{\pgfqpoint{-0.029821in}{-0.017319in}}{\pgfqpoint{-0.023570in}{-0.023570in}}%
\pgfpathcurveto{\pgfqpoint{-0.017319in}{-0.029821in}}{\pgfqpoint{-0.008840in}{-0.033333in}}{\pgfqpoint{0.000000in}{-0.033333in}}%
\pgfpathlineto{\pgfqpoint{0.000000in}{-0.033333in}}%
\pgfpathclose%
\pgfusepath{stroke,fill}%
}%
\begin{pgfscope}%
\pgfsys@transformshift{1.994083in}{2.158285in}%
\pgfsys@useobject{currentmarker}{}%
\end{pgfscope}%
\end{pgfscope}%
\begin{pgfscope}%
\pgfpathrectangle{\pgfqpoint{0.637500in}{0.550000in}}{\pgfqpoint{3.850000in}{3.850000in}}%
\pgfusepath{clip}%
\pgfsetrectcap%
\pgfsetroundjoin%
\pgfsetlinewidth{1.204500pt}%
\definecolor{currentstroke}{rgb}{1.000000,0.576471,0.309804}%
\pgfsetstrokecolor{currentstroke}%
\pgfsetdash{}{0pt}%
\pgfpathmoveto{\pgfqpoint{2.222389in}{1.663766in}}%
\pgfusepath{stroke}%
\end{pgfscope}%
\begin{pgfscope}%
\pgfpathrectangle{\pgfqpoint{0.637500in}{0.550000in}}{\pgfqpoint{3.850000in}{3.850000in}}%
\pgfusepath{clip}%
\pgfsetbuttcap%
\pgfsetroundjoin%
\definecolor{currentfill}{rgb}{1.000000,0.576471,0.309804}%
\pgfsetfillcolor{currentfill}%
\pgfsetlinewidth{1.003750pt}%
\definecolor{currentstroke}{rgb}{1.000000,0.576471,0.309804}%
\pgfsetstrokecolor{currentstroke}%
\pgfsetdash{}{0pt}%
\pgfsys@defobject{currentmarker}{\pgfqpoint{-0.033333in}{-0.033333in}}{\pgfqpoint{0.033333in}{0.033333in}}{%
\pgfpathmoveto{\pgfqpoint{0.000000in}{-0.033333in}}%
\pgfpathcurveto{\pgfqpoint{0.008840in}{-0.033333in}}{\pgfqpoint{0.017319in}{-0.029821in}}{\pgfqpoint{0.023570in}{-0.023570in}}%
\pgfpathcurveto{\pgfqpoint{0.029821in}{-0.017319in}}{\pgfqpoint{0.033333in}{-0.008840in}}{\pgfqpoint{0.033333in}{0.000000in}}%
\pgfpathcurveto{\pgfqpoint{0.033333in}{0.008840in}}{\pgfqpoint{0.029821in}{0.017319in}}{\pgfqpoint{0.023570in}{0.023570in}}%
\pgfpathcurveto{\pgfqpoint{0.017319in}{0.029821in}}{\pgfqpoint{0.008840in}{0.033333in}}{\pgfqpoint{0.000000in}{0.033333in}}%
\pgfpathcurveto{\pgfqpoint{-0.008840in}{0.033333in}}{\pgfqpoint{-0.017319in}{0.029821in}}{\pgfqpoint{-0.023570in}{0.023570in}}%
\pgfpathcurveto{\pgfqpoint{-0.029821in}{0.017319in}}{\pgfqpoint{-0.033333in}{0.008840in}}{\pgfqpoint{-0.033333in}{0.000000in}}%
\pgfpathcurveto{\pgfqpoint{-0.033333in}{-0.008840in}}{\pgfqpoint{-0.029821in}{-0.017319in}}{\pgfqpoint{-0.023570in}{-0.023570in}}%
\pgfpathcurveto{\pgfqpoint{-0.017319in}{-0.029821in}}{\pgfqpoint{-0.008840in}{-0.033333in}}{\pgfqpoint{0.000000in}{-0.033333in}}%
\pgfpathlineto{\pgfqpoint{0.000000in}{-0.033333in}}%
\pgfpathclose%
\pgfusepath{stroke,fill}%
}%
\begin{pgfscope}%
\pgfsys@transformshift{2.222389in}{1.663766in}%
\pgfsys@useobject{currentmarker}{}%
\end{pgfscope}%
\end{pgfscope}%
\begin{pgfscope}%
\pgfpathrectangle{\pgfqpoint{0.637500in}{0.550000in}}{\pgfqpoint{3.850000in}{3.850000in}}%
\pgfusepath{clip}%
\pgfsetrectcap%
\pgfsetroundjoin%
\pgfsetlinewidth{1.204500pt}%
\definecolor{currentstroke}{rgb}{1.000000,0.576471,0.309804}%
\pgfsetstrokecolor{currentstroke}%
\pgfsetdash{}{0pt}%
\pgfpathmoveto{\pgfqpoint{2.215131in}{1.313637in}}%
\pgfusepath{stroke}%
\end{pgfscope}%
\begin{pgfscope}%
\pgfpathrectangle{\pgfqpoint{0.637500in}{0.550000in}}{\pgfqpoint{3.850000in}{3.850000in}}%
\pgfusepath{clip}%
\pgfsetbuttcap%
\pgfsetroundjoin%
\definecolor{currentfill}{rgb}{1.000000,0.576471,0.309804}%
\pgfsetfillcolor{currentfill}%
\pgfsetlinewidth{1.003750pt}%
\definecolor{currentstroke}{rgb}{1.000000,0.576471,0.309804}%
\pgfsetstrokecolor{currentstroke}%
\pgfsetdash{}{0pt}%
\pgfsys@defobject{currentmarker}{\pgfqpoint{-0.033333in}{-0.033333in}}{\pgfqpoint{0.033333in}{0.033333in}}{%
\pgfpathmoveto{\pgfqpoint{0.000000in}{-0.033333in}}%
\pgfpathcurveto{\pgfqpoint{0.008840in}{-0.033333in}}{\pgfqpoint{0.017319in}{-0.029821in}}{\pgfqpoint{0.023570in}{-0.023570in}}%
\pgfpathcurveto{\pgfqpoint{0.029821in}{-0.017319in}}{\pgfqpoint{0.033333in}{-0.008840in}}{\pgfqpoint{0.033333in}{0.000000in}}%
\pgfpathcurveto{\pgfqpoint{0.033333in}{0.008840in}}{\pgfqpoint{0.029821in}{0.017319in}}{\pgfqpoint{0.023570in}{0.023570in}}%
\pgfpathcurveto{\pgfqpoint{0.017319in}{0.029821in}}{\pgfqpoint{0.008840in}{0.033333in}}{\pgfqpoint{0.000000in}{0.033333in}}%
\pgfpathcurveto{\pgfqpoint{-0.008840in}{0.033333in}}{\pgfqpoint{-0.017319in}{0.029821in}}{\pgfqpoint{-0.023570in}{0.023570in}}%
\pgfpathcurveto{\pgfqpoint{-0.029821in}{0.017319in}}{\pgfqpoint{-0.033333in}{0.008840in}}{\pgfqpoint{-0.033333in}{0.000000in}}%
\pgfpathcurveto{\pgfqpoint{-0.033333in}{-0.008840in}}{\pgfqpoint{-0.029821in}{-0.017319in}}{\pgfqpoint{-0.023570in}{-0.023570in}}%
\pgfpathcurveto{\pgfqpoint{-0.017319in}{-0.029821in}}{\pgfqpoint{-0.008840in}{-0.033333in}}{\pgfqpoint{0.000000in}{-0.033333in}}%
\pgfpathlineto{\pgfqpoint{0.000000in}{-0.033333in}}%
\pgfpathclose%
\pgfusepath{stroke,fill}%
}%
\begin{pgfscope}%
\pgfsys@transformshift{2.215131in}{1.313637in}%
\pgfsys@useobject{currentmarker}{}%
\end{pgfscope}%
\end{pgfscope}%
\begin{pgfscope}%
\pgfpathrectangle{\pgfqpoint{0.637500in}{0.550000in}}{\pgfqpoint{3.850000in}{3.850000in}}%
\pgfusepath{clip}%
\pgfsetrectcap%
\pgfsetroundjoin%
\pgfsetlinewidth{1.204500pt}%
\definecolor{currentstroke}{rgb}{1.000000,0.576471,0.309804}%
\pgfsetstrokecolor{currentstroke}%
\pgfsetdash{}{0pt}%
\pgfpathmoveto{\pgfqpoint{2.697796in}{2.168302in}}%
\pgfusepath{stroke}%
\end{pgfscope}%
\begin{pgfscope}%
\pgfpathrectangle{\pgfqpoint{0.637500in}{0.550000in}}{\pgfqpoint{3.850000in}{3.850000in}}%
\pgfusepath{clip}%
\pgfsetbuttcap%
\pgfsetroundjoin%
\definecolor{currentfill}{rgb}{1.000000,0.576471,0.309804}%
\pgfsetfillcolor{currentfill}%
\pgfsetlinewidth{1.003750pt}%
\definecolor{currentstroke}{rgb}{1.000000,0.576471,0.309804}%
\pgfsetstrokecolor{currentstroke}%
\pgfsetdash{}{0pt}%
\pgfsys@defobject{currentmarker}{\pgfqpoint{-0.033333in}{-0.033333in}}{\pgfqpoint{0.033333in}{0.033333in}}{%
\pgfpathmoveto{\pgfqpoint{0.000000in}{-0.033333in}}%
\pgfpathcurveto{\pgfqpoint{0.008840in}{-0.033333in}}{\pgfqpoint{0.017319in}{-0.029821in}}{\pgfqpoint{0.023570in}{-0.023570in}}%
\pgfpathcurveto{\pgfqpoint{0.029821in}{-0.017319in}}{\pgfqpoint{0.033333in}{-0.008840in}}{\pgfqpoint{0.033333in}{0.000000in}}%
\pgfpathcurveto{\pgfqpoint{0.033333in}{0.008840in}}{\pgfqpoint{0.029821in}{0.017319in}}{\pgfqpoint{0.023570in}{0.023570in}}%
\pgfpathcurveto{\pgfqpoint{0.017319in}{0.029821in}}{\pgfqpoint{0.008840in}{0.033333in}}{\pgfqpoint{0.000000in}{0.033333in}}%
\pgfpathcurveto{\pgfqpoint{-0.008840in}{0.033333in}}{\pgfqpoint{-0.017319in}{0.029821in}}{\pgfqpoint{-0.023570in}{0.023570in}}%
\pgfpathcurveto{\pgfqpoint{-0.029821in}{0.017319in}}{\pgfqpoint{-0.033333in}{0.008840in}}{\pgfqpoint{-0.033333in}{0.000000in}}%
\pgfpathcurveto{\pgfqpoint{-0.033333in}{-0.008840in}}{\pgfqpoint{-0.029821in}{-0.017319in}}{\pgfqpoint{-0.023570in}{-0.023570in}}%
\pgfpathcurveto{\pgfqpoint{-0.017319in}{-0.029821in}}{\pgfqpoint{-0.008840in}{-0.033333in}}{\pgfqpoint{0.000000in}{-0.033333in}}%
\pgfpathlineto{\pgfqpoint{0.000000in}{-0.033333in}}%
\pgfpathclose%
\pgfusepath{stroke,fill}%
}%
\begin{pgfscope}%
\pgfsys@transformshift{2.697796in}{2.168302in}%
\pgfsys@useobject{currentmarker}{}%
\end{pgfscope}%
\end{pgfscope}%
\begin{pgfscope}%
\pgfpathrectangle{\pgfqpoint{0.637500in}{0.550000in}}{\pgfqpoint{3.850000in}{3.850000in}}%
\pgfusepath{clip}%
\pgfsetrectcap%
\pgfsetroundjoin%
\pgfsetlinewidth{1.204500pt}%
\definecolor{currentstroke}{rgb}{1.000000,0.576471,0.309804}%
\pgfsetstrokecolor{currentstroke}%
\pgfsetdash{}{0pt}%
\pgfpathmoveto{\pgfqpoint{2.744856in}{1.654620in}}%
\pgfusepath{stroke}%
\end{pgfscope}%
\begin{pgfscope}%
\pgfpathrectangle{\pgfqpoint{0.637500in}{0.550000in}}{\pgfqpoint{3.850000in}{3.850000in}}%
\pgfusepath{clip}%
\pgfsetbuttcap%
\pgfsetroundjoin%
\definecolor{currentfill}{rgb}{1.000000,0.576471,0.309804}%
\pgfsetfillcolor{currentfill}%
\pgfsetlinewidth{1.003750pt}%
\definecolor{currentstroke}{rgb}{1.000000,0.576471,0.309804}%
\pgfsetstrokecolor{currentstroke}%
\pgfsetdash{}{0pt}%
\pgfsys@defobject{currentmarker}{\pgfqpoint{-0.033333in}{-0.033333in}}{\pgfqpoint{0.033333in}{0.033333in}}{%
\pgfpathmoveto{\pgfqpoint{0.000000in}{-0.033333in}}%
\pgfpathcurveto{\pgfqpoint{0.008840in}{-0.033333in}}{\pgfqpoint{0.017319in}{-0.029821in}}{\pgfqpoint{0.023570in}{-0.023570in}}%
\pgfpathcurveto{\pgfqpoint{0.029821in}{-0.017319in}}{\pgfqpoint{0.033333in}{-0.008840in}}{\pgfqpoint{0.033333in}{0.000000in}}%
\pgfpathcurveto{\pgfqpoint{0.033333in}{0.008840in}}{\pgfqpoint{0.029821in}{0.017319in}}{\pgfqpoint{0.023570in}{0.023570in}}%
\pgfpathcurveto{\pgfqpoint{0.017319in}{0.029821in}}{\pgfqpoint{0.008840in}{0.033333in}}{\pgfqpoint{0.000000in}{0.033333in}}%
\pgfpathcurveto{\pgfqpoint{-0.008840in}{0.033333in}}{\pgfqpoint{-0.017319in}{0.029821in}}{\pgfqpoint{-0.023570in}{0.023570in}}%
\pgfpathcurveto{\pgfqpoint{-0.029821in}{0.017319in}}{\pgfqpoint{-0.033333in}{0.008840in}}{\pgfqpoint{-0.033333in}{0.000000in}}%
\pgfpathcurveto{\pgfqpoint{-0.033333in}{-0.008840in}}{\pgfqpoint{-0.029821in}{-0.017319in}}{\pgfqpoint{-0.023570in}{-0.023570in}}%
\pgfpathcurveto{\pgfqpoint{-0.017319in}{-0.029821in}}{\pgfqpoint{-0.008840in}{-0.033333in}}{\pgfqpoint{0.000000in}{-0.033333in}}%
\pgfpathlineto{\pgfqpoint{0.000000in}{-0.033333in}}%
\pgfpathclose%
\pgfusepath{stroke,fill}%
}%
\begin{pgfscope}%
\pgfsys@transformshift{2.744856in}{1.654620in}%
\pgfsys@useobject{currentmarker}{}%
\end{pgfscope}%
\end{pgfscope}%
\begin{pgfscope}%
\pgfpathrectangle{\pgfqpoint{0.637500in}{0.550000in}}{\pgfqpoint{3.850000in}{3.850000in}}%
\pgfusepath{clip}%
\pgfsetrectcap%
\pgfsetroundjoin%
\pgfsetlinewidth{1.204500pt}%
\definecolor{currentstroke}{rgb}{1.000000,0.576471,0.309804}%
\pgfsetstrokecolor{currentstroke}%
\pgfsetdash{}{0pt}%
\pgfpathmoveto{\pgfqpoint{2.369765in}{1.790155in}}%
\pgfusepath{stroke}%
\end{pgfscope}%
\begin{pgfscope}%
\pgfpathrectangle{\pgfqpoint{0.637500in}{0.550000in}}{\pgfqpoint{3.850000in}{3.850000in}}%
\pgfusepath{clip}%
\pgfsetbuttcap%
\pgfsetroundjoin%
\definecolor{currentfill}{rgb}{1.000000,0.576471,0.309804}%
\pgfsetfillcolor{currentfill}%
\pgfsetlinewidth{1.003750pt}%
\definecolor{currentstroke}{rgb}{1.000000,0.576471,0.309804}%
\pgfsetstrokecolor{currentstroke}%
\pgfsetdash{}{0pt}%
\pgfsys@defobject{currentmarker}{\pgfqpoint{-0.033333in}{-0.033333in}}{\pgfqpoint{0.033333in}{0.033333in}}{%
\pgfpathmoveto{\pgfqpoint{0.000000in}{-0.033333in}}%
\pgfpathcurveto{\pgfqpoint{0.008840in}{-0.033333in}}{\pgfqpoint{0.017319in}{-0.029821in}}{\pgfqpoint{0.023570in}{-0.023570in}}%
\pgfpathcurveto{\pgfqpoint{0.029821in}{-0.017319in}}{\pgfqpoint{0.033333in}{-0.008840in}}{\pgfqpoint{0.033333in}{0.000000in}}%
\pgfpathcurveto{\pgfqpoint{0.033333in}{0.008840in}}{\pgfqpoint{0.029821in}{0.017319in}}{\pgfqpoint{0.023570in}{0.023570in}}%
\pgfpathcurveto{\pgfqpoint{0.017319in}{0.029821in}}{\pgfqpoint{0.008840in}{0.033333in}}{\pgfqpoint{0.000000in}{0.033333in}}%
\pgfpathcurveto{\pgfqpoint{-0.008840in}{0.033333in}}{\pgfqpoint{-0.017319in}{0.029821in}}{\pgfqpoint{-0.023570in}{0.023570in}}%
\pgfpathcurveto{\pgfqpoint{-0.029821in}{0.017319in}}{\pgfqpoint{-0.033333in}{0.008840in}}{\pgfqpoint{-0.033333in}{0.000000in}}%
\pgfpathcurveto{\pgfqpoint{-0.033333in}{-0.008840in}}{\pgfqpoint{-0.029821in}{-0.017319in}}{\pgfqpoint{-0.023570in}{-0.023570in}}%
\pgfpathcurveto{\pgfqpoint{-0.017319in}{-0.029821in}}{\pgfqpoint{-0.008840in}{-0.033333in}}{\pgfqpoint{0.000000in}{-0.033333in}}%
\pgfpathlineto{\pgfqpoint{0.000000in}{-0.033333in}}%
\pgfpathclose%
\pgfusepath{stroke,fill}%
}%
\begin{pgfscope}%
\pgfsys@transformshift{2.369765in}{1.790155in}%
\pgfsys@useobject{currentmarker}{}%
\end{pgfscope}%
\end{pgfscope}%
\begin{pgfscope}%
\pgfpathrectangle{\pgfqpoint{0.637500in}{0.550000in}}{\pgfqpoint{3.850000in}{3.850000in}}%
\pgfusepath{clip}%
\pgfsetrectcap%
\pgfsetroundjoin%
\pgfsetlinewidth{1.204500pt}%
\definecolor{currentstroke}{rgb}{1.000000,0.576471,0.309804}%
\pgfsetstrokecolor{currentstroke}%
\pgfsetdash{}{0pt}%
\pgfpathmoveto{\pgfqpoint{1.547748in}{2.399778in}}%
\pgfusepath{stroke}%
\end{pgfscope}%
\begin{pgfscope}%
\pgfpathrectangle{\pgfqpoint{0.637500in}{0.550000in}}{\pgfqpoint{3.850000in}{3.850000in}}%
\pgfusepath{clip}%
\pgfsetbuttcap%
\pgfsetroundjoin%
\definecolor{currentfill}{rgb}{1.000000,0.576471,0.309804}%
\pgfsetfillcolor{currentfill}%
\pgfsetlinewidth{1.003750pt}%
\definecolor{currentstroke}{rgb}{1.000000,0.576471,0.309804}%
\pgfsetstrokecolor{currentstroke}%
\pgfsetdash{}{0pt}%
\pgfsys@defobject{currentmarker}{\pgfqpoint{-0.033333in}{-0.033333in}}{\pgfqpoint{0.033333in}{0.033333in}}{%
\pgfpathmoveto{\pgfqpoint{0.000000in}{-0.033333in}}%
\pgfpathcurveto{\pgfqpoint{0.008840in}{-0.033333in}}{\pgfqpoint{0.017319in}{-0.029821in}}{\pgfqpoint{0.023570in}{-0.023570in}}%
\pgfpathcurveto{\pgfqpoint{0.029821in}{-0.017319in}}{\pgfqpoint{0.033333in}{-0.008840in}}{\pgfqpoint{0.033333in}{0.000000in}}%
\pgfpathcurveto{\pgfqpoint{0.033333in}{0.008840in}}{\pgfqpoint{0.029821in}{0.017319in}}{\pgfqpoint{0.023570in}{0.023570in}}%
\pgfpathcurveto{\pgfqpoint{0.017319in}{0.029821in}}{\pgfqpoint{0.008840in}{0.033333in}}{\pgfqpoint{0.000000in}{0.033333in}}%
\pgfpathcurveto{\pgfqpoint{-0.008840in}{0.033333in}}{\pgfqpoint{-0.017319in}{0.029821in}}{\pgfqpoint{-0.023570in}{0.023570in}}%
\pgfpathcurveto{\pgfqpoint{-0.029821in}{0.017319in}}{\pgfqpoint{-0.033333in}{0.008840in}}{\pgfqpoint{-0.033333in}{0.000000in}}%
\pgfpathcurveto{\pgfqpoint{-0.033333in}{-0.008840in}}{\pgfqpoint{-0.029821in}{-0.017319in}}{\pgfqpoint{-0.023570in}{-0.023570in}}%
\pgfpathcurveto{\pgfqpoint{-0.017319in}{-0.029821in}}{\pgfqpoint{-0.008840in}{-0.033333in}}{\pgfqpoint{0.000000in}{-0.033333in}}%
\pgfpathlineto{\pgfqpoint{0.000000in}{-0.033333in}}%
\pgfpathclose%
\pgfusepath{stroke,fill}%
}%
\begin{pgfscope}%
\pgfsys@transformshift{1.547748in}{2.399778in}%
\pgfsys@useobject{currentmarker}{}%
\end{pgfscope}%
\end{pgfscope}%
\begin{pgfscope}%
\pgfpathrectangle{\pgfqpoint{0.637500in}{0.550000in}}{\pgfqpoint{3.850000in}{3.850000in}}%
\pgfusepath{clip}%
\pgfsetrectcap%
\pgfsetroundjoin%
\pgfsetlinewidth{1.204500pt}%
\definecolor{currentstroke}{rgb}{1.000000,0.576471,0.309804}%
\pgfsetstrokecolor{currentstroke}%
\pgfsetdash{}{0pt}%
\pgfpathmoveto{\pgfqpoint{1.889289in}{2.520427in}}%
\pgfusepath{stroke}%
\end{pgfscope}%
\begin{pgfscope}%
\pgfpathrectangle{\pgfqpoint{0.637500in}{0.550000in}}{\pgfqpoint{3.850000in}{3.850000in}}%
\pgfusepath{clip}%
\pgfsetbuttcap%
\pgfsetroundjoin%
\definecolor{currentfill}{rgb}{1.000000,0.576471,0.309804}%
\pgfsetfillcolor{currentfill}%
\pgfsetlinewidth{1.003750pt}%
\definecolor{currentstroke}{rgb}{1.000000,0.576471,0.309804}%
\pgfsetstrokecolor{currentstroke}%
\pgfsetdash{}{0pt}%
\pgfsys@defobject{currentmarker}{\pgfqpoint{-0.033333in}{-0.033333in}}{\pgfqpoint{0.033333in}{0.033333in}}{%
\pgfpathmoveto{\pgfqpoint{0.000000in}{-0.033333in}}%
\pgfpathcurveto{\pgfqpoint{0.008840in}{-0.033333in}}{\pgfqpoint{0.017319in}{-0.029821in}}{\pgfqpoint{0.023570in}{-0.023570in}}%
\pgfpathcurveto{\pgfqpoint{0.029821in}{-0.017319in}}{\pgfqpoint{0.033333in}{-0.008840in}}{\pgfqpoint{0.033333in}{0.000000in}}%
\pgfpathcurveto{\pgfqpoint{0.033333in}{0.008840in}}{\pgfqpoint{0.029821in}{0.017319in}}{\pgfqpoint{0.023570in}{0.023570in}}%
\pgfpathcurveto{\pgfqpoint{0.017319in}{0.029821in}}{\pgfqpoint{0.008840in}{0.033333in}}{\pgfqpoint{0.000000in}{0.033333in}}%
\pgfpathcurveto{\pgfqpoint{-0.008840in}{0.033333in}}{\pgfqpoint{-0.017319in}{0.029821in}}{\pgfqpoint{-0.023570in}{0.023570in}}%
\pgfpathcurveto{\pgfqpoint{-0.029821in}{0.017319in}}{\pgfqpoint{-0.033333in}{0.008840in}}{\pgfqpoint{-0.033333in}{0.000000in}}%
\pgfpathcurveto{\pgfqpoint{-0.033333in}{-0.008840in}}{\pgfqpoint{-0.029821in}{-0.017319in}}{\pgfqpoint{-0.023570in}{-0.023570in}}%
\pgfpathcurveto{\pgfqpoint{-0.017319in}{-0.029821in}}{\pgfqpoint{-0.008840in}{-0.033333in}}{\pgfqpoint{0.000000in}{-0.033333in}}%
\pgfpathlineto{\pgfqpoint{0.000000in}{-0.033333in}}%
\pgfpathclose%
\pgfusepath{stroke,fill}%
}%
\begin{pgfscope}%
\pgfsys@transformshift{1.889289in}{2.520427in}%
\pgfsys@useobject{currentmarker}{}%
\end{pgfscope}%
\end{pgfscope}%
\begin{pgfscope}%
\pgfpathrectangle{\pgfqpoint{0.637500in}{0.550000in}}{\pgfqpoint{3.850000in}{3.850000in}}%
\pgfusepath{clip}%
\pgfsetrectcap%
\pgfsetroundjoin%
\pgfsetlinewidth{1.204500pt}%
\definecolor{currentstroke}{rgb}{1.000000,0.576471,0.309804}%
\pgfsetstrokecolor{currentstroke}%
\pgfsetdash{}{0pt}%
\pgfpathmoveto{\pgfqpoint{2.144911in}{1.789160in}}%
\pgfusepath{stroke}%
\end{pgfscope}%
\begin{pgfscope}%
\pgfpathrectangle{\pgfqpoint{0.637500in}{0.550000in}}{\pgfqpoint{3.850000in}{3.850000in}}%
\pgfusepath{clip}%
\pgfsetbuttcap%
\pgfsetroundjoin%
\definecolor{currentfill}{rgb}{1.000000,0.576471,0.309804}%
\pgfsetfillcolor{currentfill}%
\pgfsetlinewidth{1.003750pt}%
\definecolor{currentstroke}{rgb}{1.000000,0.576471,0.309804}%
\pgfsetstrokecolor{currentstroke}%
\pgfsetdash{}{0pt}%
\pgfsys@defobject{currentmarker}{\pgfqpoint{-0.033333in}{-0.033333in}}{\pgfqpoint{0.033333in}{0.033333in}}{%
\pgfpathmoveto{\pgfqpoint{0.000000in}{-0.033333in}}%
\pgfpathcurveto{\pgfqpoint{0.008840in}{-0.033333in}}{\pgfqpoint{0.017319in}{-0.029821in}}{\pgfqpoint{0.023570in}{-0.023570in}}%
\pgfpathcurveto{\pgfqpoint{0.029821in}{-0.017319in}}{\pgfqpoint{0.033333in}{-0.008840in}}{\pgfqpoint{0.033333in}{0.000000in}}%
\pgfpathcurveto{\pgfqpoint{0.033333in}{0.008840in}}{\pgfqpoint{0.029821in}{0.017319in}}{\pgfqpoint{0.023570in}{0.023570in}}%
\pgfpathcurveto{\pgfqpoint{0.017319in}{0.029821in}}{\pgfqpoint{0.008840in}{0.033333in}}{\pgfqpoint{0.000000in}{0.033333in}}%
\pgfpathcurveto{\pgfqpoint{-0.008840in}{0.033333in}}{\pgfqpoint{-0.017319in}{0.029821in}}{\pgfqpoint{-0.023570in}{0.023570in}}%
\pgfpathcurveto{\pgfqpoint{-0.029821in}{0.017319in}}{\pgfqpoint{-0.033333in}{0.008840in}}{\pgfqpoint{-0.033333in}{0.000000in}}%
\pgfpathcurveto{\pgfqpoint{-0.033333in}{-0.008840in}}{\pgfqpoint{-0.029821in}{-0.017319in}}{\pgfqpoint{-0.023570in}{-0.023570in}}%
\pgfpathcurveto{\pgfqpoint{-0.017319in}{-0.029821in}}{\pgfqpoint{-0.008840in}{-0.033333in}}{\pgfqpoint{0.000000in}{-0.033333in}}%
\pgfpathlineto{\pgfqpoint{0.000000in}{-0.033333in}}%
\pgfpathclose%
\pgfusepath{stroke,fill}%
}%
\begin{pgfscope}%
\pgfsys@transformshift{2.144911in}{1.789160in}%
\pgfsys@useobject{currentmarker}{}%
\end{pgfscope}%
\end{pgfscope}%
\begin{pgfscope}%
\pgfpathrectangle{\pgfqpoint{0.637500in}{0.550000in}}{\pgfqpoint{3.850000in}{3.850000in}}%
\pgfusepath{clip}%
\pgfsetrectcap%
\pgfsetroundjoin%
\pgfsetlinewidth{1.204500pt}%
\definecolor{currentstroke}{rgb}{1.000000,0.576471,0.309804}%
\pgfsetstrokecolor{currentstroke}%
\pgfsetdash{}{0pt}%
\pgfpathmoveto{\pgfqpoint{2.208155in}{2.425020in}}%
\pgfusepath{stroke}%
\end{pgfscope}%
\begin{pgfscope}%
\pgfpathrectangle{\pgfqpoint{0.637500in}{0.550000in}}{\pgfqpoint{3.850000in}{3.850000in}}%
\pgfusepath{clip}%
\pgfsetbuttcap%
\pgfsetroundjoin%
\definecolor{currentfill}{rgb}{1.000000,0.576471,0.309804}%
\pgfsetfillcolor{currentfill}%
\pgfsetlinewidth{1.003750pt}%
\definecolor{currentstroke}{rgb}{1.000000,0.576471,0.309804}%
\pgfsetstrokecolor{currentstroke}%
\pgfsetdash{}{0pt}%
\pgfsys@defobject{currentmarker}{\pgfqpoint{-0.033333in}{-0.033333in}}{\pgfqpoint{0.033333in}{0.033333in}}{%
\pgfpathmoveto{\pgfqpoint{0.000000in}{-0.033333in}}%
\pgfpathcurveto{\pgfqpoint{0.008840in}{-0.033333in}}{\pgfqpoint{0.017319in}{-0.029821in}}{\pgfqpoint{0.023570in}{-0.023570in}}%
\pgfpathcurveto{\pgfqpoint{0.029821in}{-0.017319in}}{\pgfqpoint{0.033333in}{-0.008840in}}{\pgfqpoint{0.033333in}{0.000000in}}%
\pgfpathcurveto{\pgfqpoint{0.033333in}{0.008840in}}{\pgfqpoint{0.029821in}{0.017319in}}{\pgfqpoint{0.023570in}{0.023570in}}%
\pgfpathcurveto{\pgfqpoint{0.017319in}{0.029821in}}{\pgfqpoint{0.008840in}{0.033333in}}{\pgfqpoint{0.000000in}{0.033333in}}%
\pgfpathcurveto{\pgfqpoint{-0.008840in}{0.033333in}}{\pgfqpoint{-0.017319in}{0.029821in}}{\pgfqpoint{-0.023570in}{0.023570in}}%
\pgfpathcurveto{\pgfqpoint{-0.029821in}{0.017319in}}{\pgfqpoint{-0.033333in}{0.008840in}}{\pgfqpoint{-0.033333in}{0.000000in}}%
\pgfpathcurveto{\pgfqpoint{-0.033333in}{-0.008840in}}{\pgfqpoint{-0.029821in}{-0.017319in}}{\pgfqpoint{-0.023570in}{-0.023570in}}%
\pgfpathcurveto{\pgfqpoint{-0.017319in}{-0.029821in}}{\pgfqpoint{-0.008840in}{-0.033333in}}{\pgfqpoint{0.000000in}{-0.033333in}}%
\pgfpathlineto{\pgfqpoint{0.000000in}{-0.033333in}}%
\pgfpathclose%
\pgfusepath{stroke,fill}%
}%
\begin{pgfscope}%
\pgfsys@transformshift{2.208155in}{2.425020in}%
\pgfsys@useobject{currentmarker}{}%
\end{pgfscope}%
\end{pgfscope}%
\begin{pgfscope}%
\pgfpathrectangle{\pgfqpoint{0.637500in}{0.550000in}}{\pgfqpoint{3.850000in}{3.850000in}}%
\pgfusepath{clip}%
\pgfsetrectcap%
\pgfsetroundjoin%
\pgfsetlinewidth{1.204500pt}%
\definecolor{currentstroke}{rgb}{0.176471,0.192157,0.258824}%
\pgfsetstrokecolor{currentstroke}%
\pgfsetdash{}{0pt}%
\pgfpathmoveto{\pgfqpoint{2.059160in}{3.285259in}}%
\pgfusepath{stroke}%
\end{pgfscope}%
\begin{pgfscope}%
\pgfpathrectangle{\pgfqpoint{0.637500in}{0.550000in}}{\pgfqpoint{3.850000in}{3.850000in}}%
\pgfusepath{clip}%
\pgfsetbuttcap%
\pgfsetmiterjoin%
\definecolor{currentfill}{rgb}{0.176471,0.192157,0.258824}%
\pgfsetfillcolor{currentfill}%
\pgfsetlinewidth{1.003750pt}%
\definecolor{currentstroke}{rgb}{0.176471,0.192157,0.258824}%
\pgfsetstrokecolor{currentstroke}%
\pgfsetdash{}{0pt}%
\pgfsys@defobject{currentmarker}{\pgfqpoint{-0.033333in}{-0.033333in}}{\pgfqpoint{0.033333in}{0.033333in}}{%
\pgfpathmoveto{\pgfqpoint{-0.000000in}{-0.033333in}}%
\pgfpathlineto{\pgfqpoint{0.033333in}{0.033333in}}%
\pgfpathlineto{\pgfqpoint{-0.033333in}{0.033333in}}%
\pgfpathlineto{\pgfqpoint{-0.000000in}{-0.033333in}}%
\pgfpathclose%
\pgfusepath{stroke,fill}%
}%
\begin{pgfscope}%
\pgfsys@transformshift{2.059160in}{3.285259in}%
\pgfsys@useobject{currentmarker}{}%
\end{pgfscope}%
\end{pgfscope}%
\begin{pgfscope}%
\pgfpathrectangle{\pgfqpoint{0.637500in}{0.550000in}}{\pgfqpoint{3.850000in}{3.850000in}}%
\pgfusepath{clip}%
\pgfsetrectcap%
\pgfsetroundjoin%
\pgfsetlinewidth{1.204500pt}%
\definecolor{currentstroke}{rgb}{0.176471,0.192157,0.258824}%
\pgfsetstrokecolor{currentstroke}%
\pgfsetdash{}{0pt}%
\pgfpathmoveto{\pgfqpoint{2.681430in}{3.566624in}}%
\pgfusepath{stroke}%
\end{pgfscope}%
\begin{pgfscope}%
\pgfpathrectangle{\pgfqpoint{0.637500in}{0.550000in}}{\pgfqpoint{3.850000in}{3.850000in}}%
\pgfusepath{clip}%
\pgfsetbuttcap%
\pgfsetmiterjoin%
\definecolor{currentfill}{rgb}{0.176471,0.192157,0.258824}%
\pgfsetfillcolor{currentfill}%
\pgfsetlinewidth{1.003750pt}%
\definecolor{currentstroke}{rgb}{0.176471,0.192157,0.258824}%
\pgfsetstrokecolor{currentstroke}%
\pgfsetdash{}{0pt}%
\pgfsys@defobject{currentmarker}{\pgfqpoint{-0.033333in}{-0.033333in}}{\pgfqpoint{0.033333in}{0.033333in}}{%
\pgfpathmoveto{\pgfqpoint{-0.000000in}{-0.033333in}}%
\pgfpathlineto{\pgfqpoint{0.033333in}{0.033333in}}%
\pgfpathlineto{\pgfqpoint{-0.033333in}{0.033333in}}%
\pgfpathlineto{\pgfqpoint{-0.000000in}{-0.033333in}}%
\pgfpathclose%
\pgfusepath{stroke,fill}%
}%
\begin{pgfscope}%
\pgfsys@transformshift{2.681430in}{3.566624in}%
\pgfsys@useobject{currentmarker}{}%
\end{pgfscope}%
\end{pgfscope}%
\begin{pgfscope}%
\pgfpathrectangle{\pgfqpoint{0.637500in}{0.550000in}}{\pgfqpoint{3.850000in}{3.850000in}}%
\pgfusepath{clip}%
\pgfsetrectcap%
\pgfsetroundjoin%
\pgfsetlinewidth{1.204500pt}%
\definecolor{currentstroke}{rgb}{0.176471,0.192157,0.258824}%
\pgfsetstrokecolor{currentstroke}%
\pgfsetdash{}{0pt}%
\pgfpathmoveto{\pgfqpoint{2.925376in}{3.383378in}}%
\pgfusepath{stroke}%
\end{pgfscope}%
\begin{pgfscope}%
\pgfpathrectangle{\pgfqpoint{0.637500in}{0.550000in}}{\pgfqpoint{3.850000in}{3.850000in}}%
\pgfusepath{clip}%
\pgfsetbuttcap%
\pgfsetmiterjoin%
\definecolor{currentfill}{rgb}{0.176471,0.192157,0.258824}%
\pgfsetfillcolor{currentfill}%
\pgfsetlinewidth{1.003750pt}%
\definecolor{currentstroke}{rgb}{0.176471,0.192157,0.258824}%
\pgfsetstrokecolor{currentstroke}%
\pgfsetdash{}{0pt}%
\pgfsys@defobject{currentmarker}{\pgfqpoint{-0.033333in}{-0.033333in}}{\pgfqpoint{0.033333in}{0.033333in}}{%
\pgfpathmoveto{\pgfqpoint{-0.000000in}{-0.033333in}}%
\pgfpathlineto{\pgfqpoint{0.033333in}{0.033333in}}%
\pgfpathlineto{\pgfqpoint{-0.033333in}{0.033333in}}%
\pgfpathlineto{\pgfqpoint{-0.000000in}{-0.033333in}}%
\pgfpathclose%
\pgfusepath{stroke,fill}%
}%
\begin{pgfscope}%
\pgfsys@transformshift{2.925376in}{3.383378in}%
\pgfsys@useobject{currentmarker}{}%
\end{pgfscope}%
\end{pgfscope}%
\begin{pgfscope}%
\pgfpathrectangle{\pgfqpoint{0.637500in}{0.550000in}}{\pgfqpoint{3.850000in}{3.850000in}}%
\pgfusepath{clip}%
\pgfsetrectcap%
\pgfsetroundjoin%
\pgfsetlinewidth{1.204500pt}%
\definecolor{currentstroke}{rgb}{0.176471,0.192157,0.258824}%
\pgfsetstrokecolor{currentstroke}%
\pgfsetdash{}{0pt}%
\pgfpathmoveto{\pgfqpoint{2.914503in}{3.508776in}}%
\pgfusepath{stroke}%
\end{pgfscope}%
\begin{pgfscope}%
\pgfpathrectangle{\pgfqpoint{0.637500in}{0.550000in}}{\pgfqpoint{3.850000in}{3.850000in}}%
\pgfusepath{clip}%
\pgfsetbuttcap%
\pgfsetmiterjoin%
\definecolor{currentfill}{rgb}{0.176471,0.192157,0.258824}%
\pgfsetfillcolor{currentfill}%
\pgfsetlinewidth{1.003750pt}%
\definecolor{currentstroke}{rgb}{0.176471,0.192157,0.258824}%
\pgfsetstrokecolor{currentstroke}%
\pgfsetdash{}{0pt}%
\pgfsys@defobject{currentmarker}{\pgfqpoint{-0.033333in}{-0.033333in}}{\pgfqpoint{0.033333in}{0.033333in}}{%
\pgfpathmoveto{\pgfqpoint{-0.000000in}{-0.033333in}}%
\pgfpathlineto{\pgfqpoint{0.033333in}{0.033333in}}%
\pgfpathlineto{\pgfqpoint{-0.033333in}{0.033333in}}%
\pgfpathlineto{\pgfqpoint{-0.000000in}{-0.033333in}}%
\pgfpathclose%
\pgfusepath{stroke,fill}%
}%
\begin{pgfscope}%
\pgfsys@transformshift{2.914503in}{3.508776in}%
\pgfsys@useobject{currentmarker}{}%
\end{pgfscope}%
\end{pgfscope}%
\begin{pgfscope}%
\pgfpathrectangle{\pgfqpoint{0.637500in}{0.550000in}}{\pgfqpoint{3.850000in}{3.850000in}}%
\pgfusepath{clip}%
\pgfsetrectcap%
\pgfsetroundjoin%
\pgfsetlinewidth{1.204500pt}%
\definecolor{currentstroke}{rgb}{0.176471,0.192157,0.258824}%
\pgfsetstrokecolor{currentstroke}%
\pgfsetdash{}{0pt}%
\pgfpathmoveto{\pgfqpoint{1.931274in}{3.695576in}}%
\pgfusepath{stroke}%
\end{pgfscope}%
\begin{pgfscope}%
\pgfpathrectangle{\pgfqpoint{0.637500in}{0.550000in}}{\pgfqpoint{3.850000in}{3.850000in}}%
\pgfusepath{clip}%
\pgfsetbuttcap%
\pgfsetmiterjoin%
\definecolor{currentfill}{rgb}{0.176471,0.192157,0.258824}%
\pgfsetfillcolor{currentfill}%
\pgfsetlinewidth{1.003750pt}%
\definecolor{currentstroke}{rgb}{0.176471,0.192157,0.258824}%
\pgfsetstrokecolor{currentstroke}%
\pgfsetdash{}{0pt}%
\pgfsys@defobject{currentmarker}{\pgfqpoint{-0.033333in}{-0.033333in}}{\pgfqpoint{0.033333in}{0.033333in}}{%
\pgfpathmoveto{\pgfqpoint{-0.000000in}{-0.033333in}}%
\pgfpathlineto{\pgfqpoint{0.033333in}{0.033333in}}%
\pgfpathlineto{\pgfqpoint{-0.033333in}{0.033333in}}%
\pgfpathlineto{\pgfqpoint{-0.000000in}{-0.033333in}}%
\pgfpathclose%
\pgfusepath{stroke,fill}%
}%
\begin{pgfscope}%
\pgfsys@transformshift{1.931274in}{3.695576in}%
\pgfsys@useobject{currentmarker}{}%
\end{pgfscope}%
\end{pgfscope}%
\begin{pgfscope}%
\pgfpathrectangle{\pgfqpoint{0.637500in}{0.550000in}}{\pgfqpoint{3.850000in}{3.850000in}}%
\pgfusepath{clip}%
\pgfsetrectcap%
\pgfsetroundjoin%
\pgfsetlinewidth{1.204500pt}%
\definecolor{currentstroke}{rgb}{0.176471,0.192157,0.258824}%
\pgfsetstrokecolor{currentstroke}%
\pgfsetdash{}{0pt}%
\pgfpathmoveto{\pgfqpoint{2.240949in}{3.141221in}}%
\pgfusepath{stroke}%
\end{pgfscope}%
\begin{pgfscope}%
\pgfpathrectangle{\pgfqpoint{0.637500in}{0.550000in}}{\pgfqpoint{3.850000in}{3.850000in}}%
\pgfusepath{clip}%
\pgfsetbuttcap%
\pgfsetmiterjoin%
\definecolor{currentfill}{rgb}{0.176471,0.192157,0.258824}%
\pgfsetfillcolor{currentfill}%
\pgfsetlinewidth{1.003750pt}%
\definecolor{currentstroke}{rgb}{0.176471,0.192157,0.258824}%
\pgfsetstrokecolor{currentstroke}%
\pgfsetdash{}{0pt}%
\pgfsys@defobject{currentmarker}{\pgfqpoint{-0.033333in}{-0.033333in}}{\pgfqpoint{0.033333in}{0.033333in}}{%
\pgfpathmoveto{\pgfqpoint{-0.000000in}{-0.033333in}}%
\pgfpathlineto{\pgfqpoint{0.033333in}{0.033333in}}%
\pgfpathlineto{\pgfqpoint{-0.033333in}{0.033333in}}%
\pgfpathlineto{\pgfqpoint{-0.000000in}{-0.033333in}}%
\pgfpathclose%
\pgfusepath{stroke,fill}%
}%
\begin{pgfscope}%
\pgfsys@transformshift{2.240949in}{3.141221in}%
\pgfsys@useobject{currentmarker}{}%
\end{pgfscope}%
\end{pgfscope}%
\begin{pgfscope}%
\pgfpathrectangle{\pgfqpoint{0.637500in}{0.550000in}}{\pgfqpoint{3.850000in}{3.850000in}}%
\pgfusepath{clip}%
\pgfsetrectcap%
\pgfsetroundjoin%
\pgfsetlinewidth{1.204500pt}%
\definecolor{currentstroke}{rgb}{0.176471,0.192157,0.258824}%
\pgfsetstrokecolor{currentstroke}%
\pgfsetdash{}{0pt}%
\pgfpathmoveto{\pgfqpoint{2.891865in}{3.216084in}}%
\pgfusepath{stroke}%
\end{pgfscope}%
\begin{pgfscope}%
\pgfpathrectangle{\pgfqpoint{0.637500in}{0.550000in}}{\pgfqpoint{3.850000in}{3.850000in}}%
\pgfusepath{clip}%
\pgfsetbuttcap%
\pgfsetmiterjoin%
\definecolor{currentfill}{rgb}{0.176471,0.192157,0.258824}%
\pgfsetfillcolor{currentfill}%
\pgfsetlinewidth{1.003750pt}%
\definecolor{currentstroke}{rgb}{0.176471,0.192157,0.258824}%
\pgfsetstrokecolor{currentstroke}%
\pgfsetdash{}{0pt}%
\pgfsys@defobject{currentmarker}{\pgfqpoint{-0.033333in}{-0.033333in}}{\pgfqpoint{0.033333in}{0.033333in}}{%
\pgfpathmoveto{\pgfqpoint{-0.000000in}{-0.033333in}}%
\pgfpathlineto{\pgfqpoint{0.033333in}{0.033333in}}%
\pgfpathlineto{\pgfqpoint{-0.033333in}{0.033333in}}%
\pgfpathlineto{\pgfqpoint{-0.000000in}{-0.033333in}}%
\pgfpathclose%
\pgfusepath{stroke,fill}%
}%
\begin{pgfscope}%
\pgfsys@transformshift{2.891865in}{3.216084in}%
\pgfsys@useobject{currentmarker}{}%
\end{pgfscope}%
\end{pgfscope}%
\begin{pgfscope}%
\pgfpathrectangle{\pgfqpoint{0.637500in}{0.550000in}}{\pgfqpoint{3.850000in}{3.850000in}}%
\pgfusepath{clip}%
\pgfsetrectcap%
\pgfsetroundjoin%
\pgfsetlinewidth{1.204500pt}%
\definecolor{currentstroke}{rgb}{0.176471,0.192157,0.258824}%
\pgfsetstrokecolor{currentstroke}%
\pgfsetdash{}{0pt}%
\pgfpathmoveto{\pgfqpoint{3.089664in}{3.421373in}}%
\pgfusepath{stroke}%
\end{pgfscope}%
\begin{pgfscope}%
\pgfpathrectangle{\pgfqpoint{0.637500in}{0.550000in}}{\pgfqpoint{3.850000in}{3.850000in}}%
\pgfusepath{clip}%
\pgfsetbuttcap%
\pgfsetmiterjoin%
\definecolor{currentfill}{rgb}{0.176471,0.192157,0.258824}%
\pgfsetfillcolor{currentfill}%
\pgfsetlinewidth{1.003750pt}%
\definecolor{currentstroke}{rgb}{0.176471,0.192157,0.258824}%
\pgfsetstrokecolor{currentstroke}%
\pgfsetdash{}{0pt}%
\pgfsys@defobject{currentmarker}{\pgfqpoint{-0.033333in}{-0.033333in}}{\pgfqpoint{0.033333in}{0.033333in}}{%
\pgfpathmoveto{\pgfqpoint{-0.000000in}{-0.033333in}}%
\pgfpathlineto{\pgfqpoint{0.033333in}{0.033333in}}%
\pgfpathlineto{\pgfqpoint{-0.033333in}{0.033333in}}%
\pgfpathlineto{\pgfqpoint{-0.000000in}{-0.033333in}}%
\pgfpathclose%
\pgfusepath{stroke,fill}%
}%
\begin{pgfscope}%
\pgfsys@transformshift{3.089664in}{3.421373in}%
\pgfsys@useobject{currentmarker}{}%
\end{pgfscope}%
\end{pgfscope}%
\begin{pgfscope}%
\pgfpathrectangle{\pgfqpoint{0.637500in}{0.550000in}}{\pgfqpoint{3.850000in}{3.850000in}}%
\pgfusepath{clip}%
\pgfsetrectcap%
\pgfsetroundjoin%
\pgfsetlinewidth{1.204500pt}%
\definecolor{currentstroke}{rgb}{0.176471,0.192157,0.258824}%
\pgfsetstrokecolor{currentstroke}%
\pgfsetdash{}{0pt}%
\pgfpathmoveto{\pgfqpoint{1.777267in}{3.489783in}}%
\pgfusepath{stroke}%
\end{pgfscope}%
\begin{pgfscope}%
\pgfpathrectangle{\pgfqpoint{0.637500in}{0.550000in}}{\pgfqpoint{3.850000in}{3.850000in}}%
\pgfusepath{clip}%
\pgfsetbuttcap%
\pgfsetmiterjoin%
\definecolor{currentfill}{rgb}{0.176471,0.192157,0.258824}%
\pgfsetfillcolor{currentfill}%
\pgfsetlinewidth{1.003750pt}%
\definecolor{currentstroke}{rgb}{0.176471,0.192157,0.258824}%
\pgfsetstrokecolor{currentstroke}%
\pgfsetdash{}{0pt}%
\pgfsys@defobject{currentmarker}{\pgfqpoint{-0.033333in}{-0.033333in}}{\pgfqpoint{0.033333in}{0.033333in}}{%
\pgfpathmoveto{\pgfqpoint{-0.000000in}{-0.033333in}}%
\pgfpathlineto{\pgfqpoint{0.033333in}{0.033333in}}%
\pgfpathlineto{\pgfqpoint{-0.033333in}{0.033333in}}%
\pgfpathlineto{\pgfqpoint{-0.000000in}{-0.033333in}}%
\pgfpathclose%
\pgfusepath{stroke,fill}%
}%
\begin{pgfscope}%
\pgfsys@transformshift{1.777267in}{3.489783in}%
\pgfsys@useobject{currentmarker}{}%
\end{pgfscope}%
\end{pgfscope}%
\begin{pgfscope}%
\pgfpathrectangle{\pgfqpoint{0.637500in}{0.550000in}}{\pgfqpoint{3.850000in}{3.850000in}}%
\pgfusepath{clip}%
\pgfsetrectcap%
\pgfsetroundjoin%
\pgfsetlinewidth{1.204500pt}%
\definecolor{currentstroke}{rgb}{0.176471,0.192157,0.258824}%
\pgfsetstrokecolor{currentstroke}%
\pgfsetdash{}{0pt}%
\pgfpathmoveto{\pgfqpoint{2.872147in}{3.660854in}}%
\pgfusepath{stroke}%
\end{pgfscope}%
\begin{pgfscope}%
\pgfpathrectangle{\pgfqpoint{0.637500in}{0.550000in}}{\pgfqpoint{3.850000in}{3.850000in}}%
\pgfusepath{clip}%
\pgfsetbuttcap%
\pgfsetmiterjoin%
\definecolor{currentfill}{rgb}{0.176471,0.192157,0.258824}%
\pgfsetfillcolor{currentfill}%
\pgfsetlinewidth{1.003750pt}%
\definecolor{currentstroke}{rgb}{0.176471,0.192157,0.258824}%
\pgfsetstrokecolor{currentstroke}%
\pgfsetdash{}{0pt}%
\pgfsys@defobject{currentmarker}{\pgfqpoint{-0.033333in}{-0.033333in}}{\pgfqpoint{0.033333in}{0.033333in}}{%
\pgfpathmoveto{\pgfqpoint{-0.000000in}{-0.033333in}}%
\pgfpathlineto{\pgfqpoint{0.033333in}{0.033333in}}%
\pgfpathlineto{\pgfqpoint{-0.033333in}{0.033333in}}%
\pgfpathlineto{\pgfqpoint{-0.000000in}{-0.033333in}}%
\pgfpathclose%
\pgfusepath{stroke,fill}%
}%
\begin{pgfscope}%
\pgfsys@transformshift{2.872147in}{3.660854in}%
\pgfsys@useobject{currentmarker}{}%
\end{pgfscope}%
\end{pgfscope}%
\begin{pgfscope}%
\pgfpathrectangle{\pgfqpoint{0.637500in}{0.550000in}}{\pgfqpoint{3.850000in}{3.850000in}}%
\pgfusepath{clip}%
\pgfsetrectcap%
\pgfsetroundjoin%
\pgfsetlinewidth{1.204500pt}%
\definecolor{currentstroke}{rgb}{0.176471,0.192157,0.258824}%
\pgfsetstrokecolor{currentstroke}%
\pgfsetdash{}{0pt}%
\pgfpathmoveto{\pgfqpoint{3.493352in}{3.367837in}}%
\pgfusepath{stroke}%
\end{pgfscope}%
\begin{pgfscope}%
\pgfpathrectangle{\pgfqpoint{0.637500in}{0.550000in}}{\pgfqpoint{3.850000in}{3.850000in}}%
\pgfusepath{clip}%
\pgfsetbuttcap%
\pgfsetmiterjoin%
\definecolor{currentfill}{rgb}{0.176471,0.192157,0.258824}%
\pgfsetfillcolor{currentfill}%
\pgfsetlinewidth{1.003750pt}%
\definecolor{currentstroke}{rgb}{0.176471,0.192157,0.258824}%
\pgfsetstrokecolor{currentstroke}%
\pgfsetdash{}{0pt}%
\pgfsys@defobject{currentmarker}{\pgfqpoint{-0.033333in}{-0.033333in}}{\pgfqpoint{0.033333in}{0.033333in}}{%
\pgfpathmoveto{\pgfqpoint{-0.000000in}{-0.033333in}}%
\pgfpathlineto{\pgfqpoint{0.033333in}{0.033333in}}%
\pgfpathlineto{\pgfqpoint{-0.033333in}{0.033333in}}%
\pgfpathlineto{\pgfqpoint{-0.000000in}{-0.033333in}}%
\pgfpathclose%
\pgfusepath{stroke,fill}%
}%
\begin{pgfscope}%
\pgfsys@transformshift{3.493352in}{3.367837in}%
\pgfsys@useobject{currentmarker}{}%
\end{pgfscope}%
\end{pgfscope}%
\begin{pgfscope}%
\pgfpathrectangle{\pgfqpoint{0.637500in}{0.550000in}}{\pgfqpoint{3.850000in}{3.850000in}}%
\pgfusepath{clip}%
\pgfsetrectcap%
\pgfsetroundjoin%
\pgfsetlinewidth{1.204500pt}%
\definecolor{currentstroke}{rgb}{0.176471,0.192157,0.258824}%
\pgfsetstrokecolor{currentstroke}%
\pgfsetdash{}{0pt}%
\pgfpathmoveto{\pgfqpoint{3.050342in}{3.670001in}}%
\pgfusepath{stroke}%
\end{pgfscope}%
\begin{pgfscope}%
\pgfpathrectangle{\pgfqpoint{0.637500in}{0.550000in}}{\pgfqpoint{3.850000in}{3.850000in}}%
\pgfusepath{clip}%
\pgfsetbuttcap%
\pgfsetmiterjoin%
\definecolor{currentfill}{rgb}{0.176471,0.192157,0.258824}%
\pgfsetfillcolor{currentfill}%
\pgfsetlinewidth{1.003750pt}%
\definecolor{currentstroke}{rgb}{0.176471,0.192157,0.258824}%
\pgfsetstrokecolor{currentstroke}%
\pgfsetdash{}{0pt}%
\pgfsys@defobject{currentmarker}{\pgfqpoint{-0.033333in}{-0.033333in}}{\pgfqpoint{0.033333in}{0.033333in}}{%
\pgfpathmoveto{\pgfqpoint{-0.000000in}{-0.033333in}}%
\pgfpathlineto{\pgfqpoint{0.033333in}{0.033333in}}%
\pgfpathlineto{\pgfqpoint{-0.033333in}{0.033333in}}%
\pgfpathlineto{\pgfqpoint{-0.000000in}{-0.033333in}}%
\pgfpathclose%
\pgfusepath{stroke,fill}%
}%
\begin{pgfscope}%
\pgfsys@transformshift{3.050342in}{3.670001in}%
\pgfsys@useobject{currentmarker}{}%
\end{pgfscope}%
\end{pgfscope}%
\begin{pgfscope}%
\pgfpathrectangle{\pgfqpoint{0.637500in}{0.550000in}}{\pgfqpoint{3.850000in}{3.850000in}}%
\pgfusepath{clip}%
\pgfsetrectcap%
\pgfsetroundjoin%
\pgfsetlinewidth{1.204500pt}%
\definecolor{currentstroke}{rgb}{0.176471,0.192157,0.258824}%
\pgfsetstrokecolor{currentstroke}%
\pgfsetdash{}{0pt}%
\pgfpathmoveto{\pgfqpoint{2.350163in}{3.736001in}}%
\pgfusepath{stroke}%
\end{pgfscope}%
\begin{pgfscope}%
\pgfpathrectangle{\pgfqpoint{0.637500in}{0.550000in}}{\pgfqpoint{3.850000in}{3.850000in}}%
\pgfusepath{clip}%
\pgfsetbuttcap%
\pgfsetmiterjoin%
\definecolor{currentfill}{rgb}{0.176471,0.192157,0.258824}%
\pgfsetfillcolor{currentfill}%
\pgfsetlinewidth{1.003750pt}%
\definecolor{currentstroke}{rgb}{0.176471,0.192157,0.258824}%
\pgfsetstrokecolor{currentstroke}%
\pgfsetdash{}{0pt}%
\pgfsys@defobject{currentmarker}{\pgfqpoint{-0.033333in}{-0.033333in}}{\pgfqpoint{0.033333in}{0.033333in}}{%
\pgfpathmoveto{\pgfqpoint{-0.000000in}{-0.033333in}}%
\pgfpathlineto{\pgfqpoint{0.033333in}{0.033333in}}%
\pgfpathlineto{\pgfqpoint{-0.033333in}{0.033333in}}%
\pgfpathlineto{\pgfqpoint{-0.000000in}{-0.033333in}}%
\pgfpathclose%
\pgfusepath{stroke,fill}%
}%
\begin{pgfscope}%
\pgfsys@transformshift{2.350163in}{3.736001in}%
\pgfsys@useobject{currentmarker}{}%
\end{pgfscope}%
\end{pgfscope}%
\begin{pgfscope}%
\pgfpathrectangle{\pgfqpoint{0.637500in}{0.550000in}}{\pgfqpoint{3.850000in}{3.850000in}}%
\pgfusepath{clip}%
\pgfsetrectcap%
\pgfsetroundjoin%
\pgfsetlinewidth{1.204500pt}%
\definecolor{currentstroke}{rgb}{0.176471,0.192157,0.258824}%
\pgfsetstrokecolor{currentstroke}%
\pgfsetdash{}{0pt}%
\pgfpathmoveto{\pgfqpoint{2.673401in}{3.255046in}}%
\pgfusepath{stroke}%
\end{pgfscope}%
\begin{pgfscope}%
\pgfpathrectangle{\pgfqpoint{0.637500in}{0.550000in}}{\pgfqpoint{3.850000in}{3.850000in}}%
\pgfusepath{clip}%
\pgfsetbuttcap%
\pgfsetmiterjoin%
\definecolor{currentfill}{rgb}{0.176471,0.192157,0.258824}%
\pgfsetfillcolor{currentfill}%
\pgfsetlinewidth{1.003750pt}%
\definecolor{currentstroke}{rgb}{0.176471,0.192157,0.258824}%
\pgfsetstrokecolor{currentstroke}%
\pgfsetdash{}{0pt}%
\pgfsys@defobject{currentmarker}{\pgfqpoint{-0.033333in}{-0.033333in}}{\pgfqpoint{0.033333in}{0.033333in}}{%
\pgfpathmoveto{\pgfqpoint{-0.000000in}{-0.033333in}}%
\pgfpathlineto{\pgfqpoint{0.033333in}{0.033333in}}%
\pgfpathlineto{\pgfqpoint{-0.033333in}{0.033333in}}%
\pgfpathlineto{\pgfqpoint{-0.000000in}{-0.033333in}}%
\pgfpathclose%
\pgfusepath{stroke,fill}%
}%
\begin{pgfscope}%
\pgfsys@transformshift{2.673401in}{3.255046in}%
\pgfsys@useobject{currentmarker}{}%
\end{pgfscope}%
\end{pgfscope}%
\begin{pgfscope}%
\pgfpathrectangle{\pgfqpoint{0.637500in}{0.550000in}}{\pgfqpoint{3.850000in}{3.850000in}}%
\pgfusepath{clip}%
\pgfsetrectcap%
\pgfsetroundjoin%
\pgfsetlinewidth{1.204500pt}%
\definecolor{currentstroke}{rgb}{0.176471,0.192157,0.258824}%
\pgfsetstrokecolor{currentstroke}%
\pgfsetdash{}{0pt}%
\pgfpathmoveto{\pgfqpoint{2.030638in}{3.317633in}}%
\pgfusepath{stroke}%
\end{pgfscope}%
\begin{pgfscope}%
\pgfpathrectangle{\pgfqpoint{0.637500in}{0.550000in}}{\pgfqpoint{3.850000in}{3.850000in}}%
\pgfusepath{clip}%
\pgfsetbuttcap%
\pgfsetmiterjoin%
\definecolor{currentfill}{rgb}{0.176471,0.192157,0.258824}%
\pgfsetfillcolor{currentfill}%
\pgfsetlinewidth{1.003750pt}%
\definecolor{currentstroke}{rgb}{0.176471,0.192157,0.258824}%
\pgfsetstrokecolor{currentstroke}%
\pgfsetdash{}{0pt}%
\pgfsys@defobject{currentmarker}{\pgfqpoint{-0.033333in}{-0.033333in}}{\pgfqpoint{0.033333in}{0.033333in}}{%
\pgfpathmoveto{\pgfqpoint{-0.000000in}{-0.033333in}}%
\pgfpathlineto{\pgfqpoint{0.033333in}{0.033333in}}%
\pgfpathlineto{\pgfqpoint{-0.033333in}{0.033333in}}%
\pgfpathlineto{\pgfqpoint{-0.000000in}{-0.033333in}}%
\pgfpathclose%
\pgfusepath{stroke,fill}%
}%
\begin{pgfscope}%
\pgfsys@transformshift{2.030638in}{3.317633in}%
\pgfsys@useobject{currentmarker}{}%
\end{pgfscope}%
\end{pgfscope}%
\begin{pgfscope}%
\pgfpathrectangle{\pgfqpoint{0.637500in}{0.550000in}}{\pgfqpoint{3.850000in}{3.850000in}}%
\pgfusepath{clip}%
\pgfsetrectcap%
\pgfsetroundjoin%
\pgfsetlinewidth{1.204500pt}%
\definecolor{currentstroke}{rgb}{0.176471,0.192157,0.258824}%
\pgfsetstrokecolor{currentstroke}%
\pgfsetdash{}{0pt}%
\pgfpathmoveto{\pgfqpoint{2.516888in}{3.009824in}}%
\pgfusepath{stroke}%
\end{pgfscope}%
\begin{pgfscope}%
\pgfpathrectangle{\pgfqpoint{0.637500in}{0.550000in}}{\pgfqpoint{3.850000in}{3.850000in}}%
\pgfusepath{clip}%
\pgfsetbuttcap%
\pgfsetmiterjoin%
\definecolor{currentfill}{rgb}{0.176471,0.192157,0.258824}%
\pgfsetfillcolor{currentfill}%
\pgfsetlinewidth{1.003750pt}%
\definecolor{currentstroke}{rgb}{0.176471,0.192157,0.258824}%
\pgfsetstrokecolor{currentstroke}%
\pgfsetdash{}{0pt}%
\pgfsys@defobject{currentmarker}{\pgfqpoint{-0.033333in}{-0.033333in}}{\pgfqpoint{0.033333in}{0.033333in}}{%
\pgfpathmoveto{\pgfqpoint{-0.000000in}{-0.033333in}}%
\pgfpathlineto{\pgfqpoint{0.033333in}{0.033333in}}%
\pgfpathlineto{\pgfqpoint{-0.033333in}{0.033333in}}%
\pgfpathlineto{\pgfqpoint{-0.000000in}{-0.033333in}}%
\pgfpathclose%
\pgfusepath{stroke,fill}%
}%
\begin{pgfscope}%
\pgfsys@transformshift{2.516888in}{3.009824in}%
\pgfsys@useobject{currentmarker}{}%
\end{pgfscope}%
\end{pgfscope}%
\begin{pgfscope}%
\pgfpathrectangle{\pgfqpoint{0.637500in}{0.550000in}}{\pgfqpoint{3.850000in}{3.850000in}}%
\pgfusepath{clip}%
\pgfsetrectcap%
\pgfsetroundjoin%
\pgfsetlinewidth{1.204500pt}%
\definecolor{currentstroke}{rgb}{0.176471,0.192157,0.258824}%
\pgfsetstrokecolor{currentstroke}%
\pgfsetdash{}{0pt}%
\pgfpathmoveto{\pgfqpoint{2.534630in}{3.971354in}}%
\pgfusepath{stroke}%
\end{pgfscope}%
\begin{pgfscope}%
\pgfpathrectangle{\pgfqpoint{0.637500in}{0.550000in}}{\pgfqpoint{3.850000in}{3.850000in}}%
\pgfusepath{clip}%
\pgfsetbuttcap%
\pgfsetmiterjoin%
\definecolor{currentfill}{rgb}{0.176471,0.192157,0.258824}%
\pgfsetfillcolor{currentfill}%
\pgfsetlinewidth{1.003750pt}%
\definecolor{currentstroke}{rgb}{0.176471,0.192157,0.258824}%
\pgfsetstrokecolor{currentstroke}%
\pgfsetdash{}{0pt}%
\pgfsys@defobject{currentmarker}{\pgfqpoint{-0.033333in}{-0.033333in}}{\pgfqpoint{0.033333in}{0.033333in}}{%
\pgfpathmoveto{\pgfqpoint{-0.000000in}{-0.033333in}}%
\pgfpathlineto{\pgfqpoint{0.033333in}{0.033333in}}%
\pgfpathlineto{\pgfqpoint{-0.033333in}{0.033333in}}%
\pgfpathlineto{\pgfqpoint{-0.000000in}{-0.033333in}}%
\pgfpathclose%
\pgfusepath{stroke,fill}%
}%
\begin{pgfscope}%
\pgfsys@transformshift{2.534630in}{3.971354in}%
\pgfsys@useobject{currentmarker}{}%
\end{pgfscope}%
\end{pgfscope}%
\begin{pgfscope}%
\pgfpathrectangle{\pgfqpoint{0.637500in}{0.550000in}}{\pgfqpoint{3.850000in}{3.850000in}}%
\pgfusepath{clip}%
\pgfsetrectcap%
\pgfsetroundjoin%
\pgfsetlinewidth{1.204500pt}%
\definecolor{currentstroke}{rgb}{0.176471,0.192157,0.258824}%
\pgfsetstrokecolor{currentstroke}%
\pgfsetdash{}{0pt}%
\pgfpathmoveto{\pgfqpoint{2.722700in}{3.731212in}}%
\pgfusepath{stroke}%
\end{pgfscope}%
\begin{pgfscope}%
\pgfpathrectangle{\pgfqpoint{0.637500in}{0.550000in}}{\pgfqpoint{3.850000in}{3.850000in}}%
\pgfusepath{clip}%
\pgfsetbuttcap%
\pgfsetmiterjoin%
\definecolor{currentfill}{rgb}{0.176471,0.192157,0.258824}%
\pgfsetfillcolor{currentfill}%
\pgfsetlinewidth{1.003750pt}%
\definecolor{currentstroke}{rgb}{0.176471,0.192157,0.258824}%
\pgfsetstrokecolor{currentstroke}%
\pgfsetdash{}{0pt}%
\pgfsys@defobject{currentmarker}{\pgfqpoint{-0.033333in}{-0.033333in}}{\pgfqpoint{0.033333in}{0.033333in}}{%
\pgfpathmoveto{\pgfqpoint{-0.000000in}{-0.033333in}}%
\pgfpathlineto{\pgfqpoint{0.033333in}{0.033333in}}%
\pgfpathlineto{\pgfqpoint{-0.033333in}{0.033333in}}%
\pgfpathlineto{\pgfqpoint{-0.000000in}{-0.033333in}}%
\pgfpathclose%
\pgfusepath{stroke,fill}%
}%
\begin{pgfscope}%
\pgfsys@transformshift{2.722700in}{3.731212in}%
\pgfsys@useobject{currentmarker}{}%
\end{pgfscope}%
\end{pgfscope}%
\begin{pgfscope}%
\pgfpathrectangle{\pgfqpoint{0.637500in}{0.550000in}}{\pgfqpoint{3.850000in}{3.850000in}}%
\pgfusepath{clip}%
\pgfsetrectcap%
\pgfsetroundjoin%
\pgfsetlinewidth{1.204500pt}%
\definecolor{currentstroke}{rgb}{0.176471,0.192157,0.258824}%
\pgfsetstrokecolor{currentstroke}%
\pgfsetdash{}{0pt}%
\pgfpathmoveto{\pgfqpoint{2.037172in}{3.225989in}}%
\pgfusepath{stroke}%
\end{pgfscope}%
\begin{pgfscope}%
\pgfpathrectangle{\pgfqpoint{0.637500in}{0.550000in}}{\pgfqpoint{3.850000in}{3.850000in}}%
\pgfusepath{clip}%
\pgfsetbuttcap%
\pgfsetmiterjoin%
\definecolor{currentfill}{rgb}{0.176471,0.192157,0.258824}%
\pgfsetfillcolor{currentfill}%
\pgfsetlinewidth{1.003750pt}%
\definecolor{currentstroke}{rgb}{0.176471,0.192157,0.258824}%
\pgfsetstrokecolor{currentstroke}%
\pgfsetdash{}{0pt}%
\pgfsys@defobject{currentmarker}{\pgfqpoint{-0.033333in}{-0.033333in}}{\pgfqpoint{0.033333in}{0.033333in}}{%
\pgfpathmoveto{\pgfqpoint{-0.000000in}{-0.033333in}}%
\pgfpathlineto{\pgfqpoint{0.033333in}{0.033333in}}%
\pgfpathlineto{\pgfqpoint{-0.033333in}{0.033333in}}%
\pgfpathlineto{\pgfqpoint{-0.000000in}{-0.033333in}}%
\pgfpathclose%
\pgfusepath{stroke,fill}%
}%
\begin{pgfscope}%
\pgfsys@transformshift{2.037172in}{3.225989in}%
\pgfsys@useobject{currentmarker}{}%
\end{pgfscope}%
\end{pgfscope}%
\begin{pgfscope}%
\pgfpathrectangle{\pgfqpoint{0.637500in}{0.550000in}}{\pgfqpoint{3.850000in}{3.850000in}}%
\pgfusepath{clip}%
\pgfsetrectcap%
\pgfsetroundjoin%
\pgfsetlinewidth{1.204500pt}%
\definecolor{currentstroke}{rgb}{0.176471,0.192157,0.258824}%
\pgfsetstrokecolor{currentstroke}%
\pgfsetdash{}{0pt}%
\pgfpathmoveto{\pgfqpoint{2.548120in}{3.419912in}}%
\pgfusepath{stroke}%
\end{pgfscope}%
\begin{pgfscope}%
\pgfpathrectangle{\pgfqpoint{0.637500in}{0.550000in}}{\pgfqpoint{3.850000in}{3.850000in}}%
\pgfusepath{clip}%
\pgfsetbuttcap%
\pgfsetmiterjoin%
\definecolor{currentfill}{rgb}{0.176471,0.192157,0.258824}%
\pgfsetfillcolor{currentfill}%
\pgfsetlinewidth{1.003750pt}%
\definecolor{currentstroke}{rgb}{0.176471,0.192157,0.258824}%
\pgfsetstrokecolor{currentstroke}%
\pgfsetdash{}{0pt}%
\pgfsys@defobject{currentmarker}{\pgfqpoint{-0.033333in}{-0.033333in}}{\pgfqpoint{0.033333in}{0.033333in}}{%
\pgfpathmoveto{\pgfqpoint{-0.000000in}{-0.033333in}}%
\pgfpathlineto{\pgfqpoint{0.033333in}{0.033333in}}%
\pgfpathlineto{\pgfqpoint{-0.033333in}{0.033333in}}%
\pgfpathlineto{\pgfqpoint{-0.000000in}{-0.033333in}}%
\pgfpathclose%
\pgfusepath{stroke,fill}%
}%
\begin{pgfscope}%
\pgfsys@transformshift{2.548120in}{3.419912in}%
\pgfsys@useobject{currentmarker}{}%
\end{pgfscope}%
\end{pgfscope}%
\begin{pgfscope}%
\pgfpathrectangle{\pgfqpoint{0.637500in}{0.550000in}}{\pgfqpoint{3.850000in}{3.850000in}}%
\pgfusepath{clip}%
\pgfsetrectcap%
\pgfsetroundjoin%
\pgfsetlinewidth{1.204500pt}%
\definecolor{currentstroke}{rgb}{0.176471,0.192157,0.258824}%
\pgfsetstrokecolor{currentstroke}%
\pgfsetdash{}{0pt}%
\pgfpathmoveto{\pgfqpoint{2.890235in}{3.438305in}}%
\pgfusepath{stroke}%
\end{pgfscope}%
\begin{pgfscope}%
\pgfpathrectangle{\pgfqpoint{0.637500in}{0.550000in}}{\pgfqpoint{3.850000in}{3.850000in}}%
\pgfusepath{clip}%
\pgfsetbuttcap%
\pgfsetmiterjoin%
\definecolor{currentfill}{rgb}{0.176471,0.192157,0.258824}%
\pgfsetfillcolor{currentfill}%
\pgfsetlinewidth{1.003750pt}%
\definecolor{currentstroke}{rgb}{0.176471,0.192157,0.258824}%
\pgfsetstrokecolor{currentstroke}%
\pgfsetdash{}{0pt}%
\pgfsys@defobject{currentmarker}{\pgfqpoint{-0.033333in}{-0.033333in}}{\pgfqpoint{0.033333in}{0.033333in}}{%
\pgfpathmoveto{\pgfqpoint{-0.000000in}{-0.033333in}}%
\pgfpathlineto{\pgfqpoint{0.033333in}{0.033333in}}%
\pgfpathlineto{\pgfqpoint{-0.033333in}{0.033333in}}%
\pgfpathlineto{\pgfqpoint{-0.000000in}{-0.033333in}}%
\pgfpathclose%
\pgfusepath{stroke,fill}%
}%
\begin{pgfscope}%
\pgfsys@transformshift{2.890235in}{3.438305in}%
\pgfsys@useobject{currentmarker}{}%
\end{pgfscope}%
\end{pgfscope}%
\begin{pgfscope}%
\pgfpathrectangle{\pgfqpoint{0.637500in}{0.550000in}}{\pgfqpoint{3.850000in}{3.850000in}}%
\pgfusepath{clip}%
\pgfsetrectcap%
\pgfsetroundjoin%
\pgfsetlinewidth{1.204500pt}%
\definecolor{currentstroke}{rgb}{0.176471,0.192157,0.258824}%
\pgfsetstrokecolor{currentstroke}%
\pgfsetdash{}{0pt}%
\pgfpathmoveto{\pgfqpoint{2.069741in}{3.604731in}}%
\pgfusepath{stroke}%
\end{pgfscope}%
\begin{pgfscope}%
\pgfpathrectangle{\pgfqpoint{0.637500in}{0.550000in}}{\pgfqpoint{3.850000in}{3.850000in}}%
\pgfusepath{clip}%
\pgfsetbuttcap%
\pgfsetmiterjoin%
\definecolor{currentfill}{rgb}{0.176471,0.192157,0.258824}%
\pgfsetfillcolor{currentfill}%
\pgfsetlinewidth{1.003750pt}%
\definecolor{currentstroke}{rgb}{0.176471,0.192157,0.258824}%
\pgfsetstrokecolor{currentstroke}%
\pgfsetdash{}{0pt}%
\pgfsys@defobject{currentmarker}{\pgfqpoint{-0.033333in}{-0.033333in}}{\pgfqpoint{0.033333in}{0.033333in}}{%
\pgfpathmoveto{\pgfqpoint{-0.000000in}{-0.033333in}}%
\pgfpathlineto{\pgfqpoint{0.033333in}{0.033333in}}%
\pgfpathlineto{\pgfqpoint{-0.033333in}{0.033333in}}%
\pgfpathlineto{\pgfqpoint{-0.000000in}{-0.033333in}}%
\pgfpathclose%
\pgfusepath{stroke,fill}%
}%
\begin{pgfscope}%
\pgfsys@transformshift{2.069741in}{3.604731in}%
\pgfsys@useobject{currentmarker}{}%
\end{pgfscope}%
\end{pgfscope}%
\begin{pgfscope}%
\pgfpathrectangle{\pgfqpoint{0.637500in}{0.550000in}}{\pgfqpoint{3.850000in}{3.850000in}}%
\pgfusepath{clip}%
\pgfsetrectcap%
\pgfsetroundjoin%
\pgfsetlinewidth{1.204500pt}%
\definecolor{currentstroke}{rgb}{0.176471,0.192157,0.258824}%
\pgfsetstrokecolor{currentstroke}%
\pgfsetdash{}{0pt}%
\pgfpathmoveto{\pgfqpoint{3.038973in}{3.300418in}}%
\pgfusepath{stroke}%
\end{pgfscope}%
\begin{pgfscope}%
\pgfpathrectangle{\pgfqpoint{0.637500in}{0.550000in}}{\pgfqpoint{3.850000in}{3.850000in}}%
\pgfusepath{clip}%
\pgfsetbuttcap%
\pgfsetmiterjoin%
\definecolor{currentfill}{rgb}{0.176471,0.192157,0.258824}%
\pgfsetfillcolor{currentfill}%
\pgfsetlinewidth{1.003750pt}%
\definecolor{currentstroke}{rgb}{0.176471,0.192157,0.258824}%
\pgfsetstrokecolor{currentstroke}%
\pgfsetdash{}{0pt}%
\pgfsys@defobject{currentmarker}{\pgfqpoint{-0.033333in}{-0.033333in}}{\pgfqpoint{0.033333in}{0.033333in}}{%
\pgfpathmoveto{\pgfqpoint{-0.000000in}{-0.033333in}}%
\pgfpathlineto{\pgfqpoint{0.033333in}{0.033333in}}%
\pgfpathlineto{\pgfqpoint{-0.033333in}{0.033333in}}%
\pgfpathlineto{\pgfqpoint{-0.000000in}{-0.033333in}}%
\pgfpathclose%
\pgfusepath{stroke,fill}%
}%
\begin{pgfscope}%
\pgfsys@transformshift{3.038973in}{3.300418in}%
\pgfsys@useobject{currentmarker}{}%
\end{pgfscope}%
\end{pgfscope}%
\begin{pgfscope}%
\pgfpathrectangle{\pgfqpoint{0.637500in}{0.550000in}}{\pgfqpoint{3.850000in}{3.850000in}}%
\pgfusepath{clip}%
\pgfsetrectcap%
\pgfsetroundjoin%
\pgfsetlinewidth{1.204500pt}%
\definecolor{currentstroke}{rgb}{0.176471,0.192157,0.258824}%
\pgfsetstrokecolor{currentstroke}%
\pgfsetdash{}{0pt}%
\pgfpathmoveto{\pgfqpoint{3.708515in}{3.775995in}}%
\pgfusepath{stroke}%
\end{pgfscope}%
\begin{pgfscope}%
\pgfpathrectangle{\pgfqpoint{0.637500in}{0.550000in}}{\pgfqpoint{3.850000in}{3.850000in}}%
\pgfusepath{clip}%
\pgfsetbuttcap%
\pgfsetmiterjoin%
\definecolor{currentfill}{rgb}{0.176471,0.192157,0.258824}%
\pgfsetfillcolor{currentfill}%
\pgfsetlinewidth{1.003750pt}%
\definecolor{currentstroke}{rgb}{0.176471,0.192157,0.258824}%
\pgfsetstrokecolor{currentstroke}%
\pgfsetdash{}{0pt}%
\pgfsys@defobject{currentmarker}{\pgfqpoint{-0.033333in}{-0.033333in}}{\pgfqpoint{0.033333in}{0.033333in}}{%
\pgfpathmoveto{\pgfqpoint{-0.000000in}{-0.033333in}}%
\pgfpathlineto{\pgfqpoint{0.033333in}{0.033333in}}%
\pgfpathlineto{\pgfqpoint{-0.033333in}{0.033333in}}%
\pgfpathlineto{\pgfqpoint{-0.000000in}{-0.033333in}}%
\pgfpathclose%
\pgfusepath{stroke,fill}%
}%
\begin{pgfscope}%
\pgfsys@transformshift{3.708515in}{3.775995in}%
\pgfsys@useobject{currentmarker}{}%
\end{pgfscope}%
\end{pgfscope}%
\begin{pgfscope}%
\pgfpathrectangle{\pgfqpoint{0.637500in}{0.550000in}}{\pgfqpoint{3.850000in}{3.850000in}}%
\pgfusepath{clip}%
\pgfsetrectcap%
\pgfsetroundjoin%
\pgfsetlinewidth{1.204500pt}%
\definecolor{currentstroke}{rgb}{0.176471,0.192157,0.258824}%
\pgfsetstrokecolor{currentstroke}%
\pgfsetdash{}{0pt}%
\pgfpathmoveto{\pgfqpoint{2.875234in}{3.727639in}}%
\pgfusepath{stroke}%
\end{pgfscope}%
\begin{pgfscope}%
\pgfpathrectangle{\pgfqpoint{0.637500in}{0.550000in}}{\pgfqpoint{3.850000in}{3.850000in}}%
\pgfusepath{clip}%
\pgfsetbuttcap%
\pgfsetmiterjoin%
\definecolor{currentfill}{rgb}{0.176471,0.192157,0.258824}%
\pgfsetfillcolor{currentfill}%
\pgfsetlinewidth{1.003750pt}%
\definecolor{currentstroke}{rgb}{0.176471,0.192157,0.258824}%
\pgfsetstrokecolor{currentstroke}%
\pgfsetdash{}{0pt}%
\pgfsys@defobject{currentmarker}{\pgfqpoint{-0.033333in}{-0.033333in}}{\pgfqpoint{0.033333in}{0.033333in}}{%
\pgfpathmoveto{\pgfqpoint{-0.000000in}{-0.033333in}}%
\pgfpathlineto{\pgfqpoint{0.033333in}{0.033333in}}%
\pgfpathlineto{\pgfqpoint{-0.033333in}{0.033333in}}%
\pgfpathlineto{\pgfqpoint{-0.000000in}{-0.033333in}}%
\pgfpathclose%
\pgfusepath{stroke,fill}%
}%
\begin{pgfscope}%
\pgfsys@transformshift{2.875234in}{3.727639in}%
\pgfsys@useobject{currentmarker}{}%
\end{pgfscope}%
\end{pgfscope}%
\begin{pgfscope}%
\pgfpathrectangle{\pgfqpoint{0.637500in}{0.550000in}}{\pgfqpoint{3.850000in}{3.850000in}}%
\pgfusepath{clip}%
\pgfsetrectcap%
\pgfsetroundjoin%
\pgfsetlinewidth{1.204500pt}%
\definecolor{currentstroke}{rgb}{0.176471,0.192157,0.258824}%
\pgfsetstrokecolor{currentstroke}%
\pgfsetdash{}{0pt}%
\pgfpathmoveto{\pgfqpoint{2.718909in}{3.558149in}}%
\pgfusepath{stroke}%
\end{pgfscope}%
\begin{pgfscope}%
\pgfpathrectangle{\pgfqpoint{0.637500in}{0.550000in}}{\pgfqpoint{3.850000in}{3.850000in}}%
\pgfusepath{clip}%
\pgfsetbuttcap%
\pgfsetmiterjoin%
\definecolor{currentfill}{rgb}{0.176471,0.192157,0.258824}%
\pgfsetfillcolor{currentfill}%
\pgfsetlinewidth{1.003750pt}%
\definecolor{currentstroke}{rgb}{0.176471,0.192157,0.258824}%
\pgfsetstrokecolor{currentstroke}%
\pgfsetdash{}{0pt}%
\pgfsys@defobject{currentmarker}{\pgfqpoint{-0.033333in}{-0.033333in}}{\pgfqpoint{0.033333in}{0.033333in}}{%
\pgfpathmoveto{\pgfqpoint{-0.000000in}{-0.033333in}}%
\pgfpathlineto{\pgfqpoint{0.033333in}{0.033333in}}%
\pgfpathlineto{\pgfqpoint{-0.033333in}{0.033333in}}%
\pgfpathlineto{\pgfqpoint{-0.000000in}{-0.033333in}}%
\pgfpathclose%
\pgfusepath{stroke,fill}%
}%
\begin{pgfscope}%
\pgfsys@transformshift{2.718909in}{3.558149in}%
\pgfsys@useobject{currentmarker}{}%
\end{pgfscope}%
\end{pgfscope}%
\begin{pgfscope}%
\pgfpathrectangle{\pgfqpoint{0.637500in}{0.550000in}}{\pgfqpoint{3.850000in}{3.850000in}}%
\pgfusepath{clip}%
\pgfsetrectcap%
\pgfsetroundjoin%
\pgfsetlinewidth{1.204500pt}%
\definecolor{currentstroke}{rgb}{0.176471,0.192157,0.258824}%
\pgfsetstrokecolor{currentstroke}%
\pgfsetdash{}{0pt}%
\pgfpathmoveto{\pgfqpoint{2.306740in}{3.460811in}}%
\pgfusepath{stroke}%
\end{pgfscope}%
\begin{pgfscope}%
\pgfpathrectangle{\pgfqpoint{0.637500in}{0.550000in}}{\pgfqpoint{3.850000in}{3.850000in}}%
\pgfusepath{clip}%
\pgfsetbuttcap%
\pgfsetmiterjoin%
\definecolor{currentfill}{rgb}{0.176471,0.192157,0.258824}%
\pgfsetfillcolor{currentfill}%
\pgfsetlinewidth{1.003750pt}%
\definecolor{currentstroke}{rgb}{0.176471,0.192157,0.258824}%
\pgfsetstrokecolor{currentstroke}%
\pgfsetdash{}{0pt}%
\pgfsys@defobject{currentmarker}{\pgfqpoint{-0.033333in}{-0.033333in}}{\pgfqpoint{0.033333in}{0.033333in}}{%
\pgfpathmoveto{\pgfqpoint{-0.000000in}{-0.033333in}}%
\pgfpathlineto{\pgfqpoint{0.033333in}{0.033333in}}%
\pgfpathlineto{\pgfqpoint{-0.033333in}{0.033333in}}%
\pgfpathlineto{\pgfqpoint{-0.000000in}{-0.033333in}}%
\pgfpathclose%
\pgfusepath{stroke,fill}%
}%
\begin{pgfscope}%
\pgfsys@transformshift{2.306740in}{3.460811in}%
\pgfsys@useobject{currentmarker}{}%
\end{pgfscope}%
\end{pgfscope}%
\begin{pgfscope}%
\pgfpathrectangle{\pgfqpoint{0.637500in}{0.550000in}}{\pgfqpoint{3.850000in}{3.850000in}}%
\pgfusepath{clip}%
\pgfsetrectcap%
\pgfsetroundjoin%
\pgfsetlinewidth{1.204500pt}%
\definecolor{currentstroke}{rgb}{0.176471,0.192157,0.258824}%
\pgfsetstrokecolor{currentstroke}%
\pgfsetdash{}{0pt}%
\pgfpathmoveto{\pgfqpoint{2.653964in}{3.083663in}}%
\pgfusepath{stroke}%
\end{pgfscope}%
\begin{pgfscope}%
\pgfpathrectangle{\pgfqpoint{0.637500in}{0.550000in}}{\pgfqpoint{3.850000in}{3.850000in}}%
\pgfusepath{clip}%
\pgfsetbuttcap%
\pgfsetmiterjoin%
\definecolor{currentfill}{rgb}{0.176471,0.192157,0.258824}%
\pgfsetfillcolor{currentfill}%
\pgfsetlinewidth{1.003750pt}%
\definecolor{currentstroke}{rgb}{0.176471,0.192157,0.258824}%
\pgfsetstrokecolor{currentstroke}%
\pgfsetdash{}{0pt}%
\pgfsys@defobject{currentmarker}{\pgfqpoint{-0.033333in}{-0.033333in}}{\pgfqpoint{0.033333in}{0.033333in}}{%
\pgfpathmoveto{\pgfqpoint{-0.000000in}{-0.033333in}}%
\pgfpathlineto{\pgfqpoint{0.033333in}{0.033333in}}%
\pgfpathlineto{\pgfqpoint{-0.033333in}{0.033333in}}%
\pgfpathlineto{\pgfqpoint{-0.000000in}{-0.033333in}}%
\pgfpathclose%
\pgfusepath{stroke,fill}%
}%
\begin{pgfscope}%
\pgfsys@transformshift{2.653964in}{3.083663in}%
\pgfsys@useobject{currentmarker}{}%
\end{pgfscope}%
\end{pgfscope}%
\begin{pgfscope}%
\pgfpathrectangle{\pgfqpoint{0.637500in}{0.550000in}}{\pgfqpoint{3.850000in}{3.850000in}}%
\pgfusepath{clip}%
\pgfsetrectcap%
\pgfsetroundjoin%
\pgfsetlinewidth{1.204500pt}%
\definecolor{currentstroke}{rgb}{0.176471,0.192157,0.258824}%
\pgfsetstrokecolor{currentstroke}%
\pgfsetdash{}{0pt}%
\pgfpathmoveto{\pgfqpoint{2.295178in}{3.699818in}}%
\pgfusepath{stroke}%
\end{pgfscope}%
\begin{pgfscope}%
\pgfpathrectangle{\pgfqpoint{0.637500in}{0.550000in}}{\pgfqpoint{3.850000in}{3.850000in}}%
\pgfusepath{clip}%
\pgfsetbuttcap%
\pgfsetmiterjoin%
\definecolor{currentfill}{rgb}{0.176471,0.192157,0.258824}%
\pgfsetfillcolor{currentfill}%
\pgfsetlinewidth{1.003750pt}%
\definecolor{currentstroke}{rgb}{0.176471,0.192157,0.258824}%
\pgfsetstrokecolor{currentstroke}%
\pgfsetdash{}{0pt}%
\pgfsys@defobject{currentmarker}{\pgfqpoint{-0.033333in}{-0.033333in}}{\pgfqpoint{0.033333in}{0.033333in}}{%
\pgfpathmoveto{\pgfqpoint{-0.000000in}{-0.033333in}}%
\pgfpathlineto{\pgfqpoint{0.033333in}{0.033333in}}%
\pgfpathlineto{\pgfqpoint{-0.033333in}{0.033333in}}%
\pgfpathlineto{\pgfqpoint{-0.000000in}{-0.033333in}}%
\pgfpathclose%
\pgfusepath{stroke,fill}%
}%
\begin{pgfscope}%
\pgfsys@transformshift{2.295178in}{3.699818in}%
\pgfsys@useobject{currentmarker}{}%
\end{pgfscope}%
\end{pgfscope}%
\begin{pgfscope}%
\pgfpathrectangle{\pgfqpoint{0.637500in}{0.550000in}}{\pgfqpoint{3.850000in}{3.850000in}}%
\pgfusepath{clip}%
\pgfsetrectcap%
\pgfsetroundjoin%
\pgfsetlinewidth{1.204500pt}%
\definecolor{currentstroke}{rgb}{0.176471,0.192157,0.258824}%
\pgfsetstrokecolor{currentstroke}%
\pgfsetdash{}{0pt}%
\pgfpathmoveto{\pgfqpoint{2.644044in}{3.575002in}}%
\pgfusepath{stroke}%
\end{pgfscope}%
\begin{pgfscope}%
\pgfpathrectangle{\pgfqpoint{0.637500in}{0.550000in}}{\pgfqpoint{3.850000in}{3.850000in}}%
\pgfusepath{clip}%
\pgfsetbuttcap%
\pgfsetmiterjoin%
\definecolor{currentfill}{rgb}{0.176471,0.192157,0.258824}%
\pgfsetfillcolor{currentfill}%
\pgfsetlinewidth{1.003750pt}%
\definecolor{currentstroke}{rgb}{0.176471,0.192157,0.258824}%
\pgfsetstrokecolor{currentstroke}%
\pgfsetdash{}{0pt}%
\pgfsys@defobject{currentmarker}{\pgfqpoint{-0.033333in}{-0.033333in}}{\pgfqpoint{0.033333in}{0.033333in}}{%
\pgfpathmoveto{\pgfqpoint{-0.000000in}{-0.033333in}}%
\pgfpathlineto{\pgfqpoint{0.033333in}{0.033333in}}%
\pgfpathlineto{\pgfqpoint{-0.033333in}{0.033333in}}%
\pgfpathlineto{\pgfqpoint{-0.000000in}{-0.033333in}}%
\pgfpathclose%
\pgfusepath{stroke,fill}%
}%
\begin{pgfscope}%
\pgfsys@transformshift{2.644044in}{3.575002in}%
\pgfsys@useobject{currentmarker}{}%
\end{pgfscope}%
\end{pgfscope}%
\begin{pgfscope}%
\pgfpathrectangle{\pgfqpoint{0.637500in}{0.550000in}}{\pgfqpoint{3.850000in}{3.850000in}}%
\pgfusepath{clip}%
\pgfsetrectcap%
\pgfsetroundjoin%
\pgfsetlinewidth{1.204500pt}%
\definecolor{currentstroke}{rgb}{0.000000,0.000000,0.000000}%
\pgfsetstrokecolor{currentstroke}%
\pgfsetdash{}{0pt}%
\pgfpathmoveto{\pgfqpoint{2.635126in}{2.624046in}}%
\pgfusepath{stroke}%
\end{pgfscope}%
\begin{pgfscope}%
\pgfpathrectangle{\pgfqpoint{0.637500in}{0.550000in}}{\pgfqpoint{3.850000in}{3.850000in}}%
\pgfusepath{clip}%
\pgfsetbuttcap%
\pgfsetroundjoin%
\definecolor{currentfill}{rgb}{0.000000,0.000000,0.000000}%
\pgfsetfillcolor{currentfill}%
\pgfsetlinewidth{1.003750pt}%
\definecolor{currentstroke}{rgb}{0.000000,0.000000,0.000000}%
\pgfsetstrokecolor{currentstroke}%
\pgfsetdash{}{0pt}%
\pgfsys@defobject{currentmarker}{\pgfqpoint{-0.016667in}{-0.016667in}}{\pgfqpoint{0.016667in}{0.016667in}}{%
\pgfpathmoveto{\pgfqpoint{0.000000in}{-0.016667in}}%
\pgfpathcurveto{\pgfqpoint{0.004420in}{-0.016667in}}{\pgfqpoint{0.008660in}{-0.014911in}}{\pgfqpoint{0.011785in}{-0.011785in}}%
\pgfpathcurveto{\pgfqpoint{0.014911in}{-0.008660in}}{\pgfqpoint{0.016667in}{-0.004420in}}{\pgfqpoint{0.016667in}{0.000000in}}%
\pgfpathcurveto{\pgfqpoint{0.016667in}{0.004420in}}{\pgfqpoint{0.014911in}{0.008660in}}{\pgfqpoint{0.011785in}{0.011785in}}%
\pgfpathcurveto{\pgfqpoint{0.008660in}{0.014911in}}{\pgfqpoint{0.004420in}{0.016667in}}{\pgfqpoint{0.000000in}{0.016667in}}%
\pgfpathcurveto{\pgfqpoint{-0.004420in}{0.016667in}}{\pgfqpoint{-0.008660in}{0.014911in}}{\pgfqpoint{-0.011785in}{0.011785in}}%
\pgfpathcurveto{\pgfqpoint{-0.014911in}{0.008660in}}{\pgfqpoint{-0.016667in}{0.004420in}}{\pgfqpoint{-0.016667in}{0.000000in}}%
\pgfpathcurveto{\pgfqpoint{-0.016667in}{-0.004420in}}{\pgfqpoint{-0.014911in}{-0.008660in}}{\pgfqpoint{-0.011785in}{-0.011785in}}%
\pgfpathcurveto{\pgfqpoint{-0.008660in}{-0.014911in}}{\pgfqpoint{-0.004420in}{-0.016667in}}{\pgfqpoint{0.000000in}{-0.016667in}}%
\pgfpathlineto{\pgfqpoint{0.000000in}{-0.016667in}}%
\pgfpathclose%
\pgfusepath{stroke,fill}%
}%
\begin{pgfscope}%
\pgfsys@transformshift{2.635126in}{2.624046in}%
\pgfsys@useobject{currentmarker}{}%
\end{pgfscope}%
\end{pgfscope}%
\begin{pgfscope}%
\pgfpathrectangle{\pgfqpoint{0.637500in}{0.550000in}}{\pgfqpoint{3.850000in}{3.850000in}}%
\pgfusepath{clip}%
\pgfsetbuttcap%
\pgfsetroundjoin%
\pgfsetlinewidth{1.204500pt}%
\definecolor{currentstroke}{rgb}{0.827451,0.827451,0.827451}%
\pgfsetstrokecolor{currentstroke}%
\pgfsetdash{{1.200000pt}{1.980000pt}}{0.000000pt}%
\pgfpathmoveto{\pgfqpoint{2.635126in}{2.624046in}}%
\pgfpathlineto{\pgfqpoint{2.635126in}{0.540000in}}%
\pgfusepath{stroke}%
\end{pgfscope}%
\begin{pgfscope}%
\pgfpathrectangle{\pgfqpoint{0.637500in}{0.550000in}}{\pgfqpoint{3.850000in}{3.850000in}}%
\pgfusepath{clip}%
\pgfsetrectcap%
\pgfsetroundjoin%
\pgfsetlinewidth{1.204500pt}%
\definecolor{currentstroke}{rgb}{0.000000,0.000000,0.000000}%
\pgfsetstrokecolor{currentstroke}%
\pgfsetdash{}{0pt}%
\pgfpathmoveto{\pgfqpoint{2.635126in}{2.624046in}}%
\pgfpathlineto{\pgfqpoint{0.747322in}{3.713880in}}%
\pgfusepath{stroke}%
\end{pgfscope}%
\begin{pgfscope}%
\pgfpathrectangle{\pgfqpoint{0.637500in}{0.550000in}}{\pgfqpoint{3.850000in}{3.850000in}}%
\pgfusepath{clip}%
\pgfsetrectcap%
\pgfsetroundjoin%
\pgfsetlinewidth{1.204500pt}%
\definecolor{currentstroke}{rgb}{0.000000,0.000000,0.000000}%
\pgfsetstrokecolor{currentstroke}%
\pgfsetdash{}{0pt}%
\pgfpathmoveto{\pgfqpoint{2.635126in}{2.624046in}}%
\pgfpathlineto{\pgfqpoint{4.497500in}{3.699199in}}%
\pgfusepath{stroke}%
\end{pgfscope}%
\end{pgfpicture}%
\makeatother%
\endgroup%
}}
    \end{subfigure}

    \caption{Tropicalized hyperplane tends towards a suboptimal tropical hyperplane}
    \label{fig:LogLog}
\end{figure}


\section{Tropical Polynomials as Piecewise Linear Classifiers}

In this section, we extend our previous approach to piecewise linear
fitting, using tropical polynomials. 

A \emph{tropical polynomial} \emph{function} over $\mathbb{R}_{\max}^{d}$
is defined by
\[
f(x)=\bigoplus_{\alpha\in\mathbb{Z}^{d}}f_{\alpha}x^{\alpha}=\max_{\alpha\in\mathbb{Z}^{d}}\left(f_{\alpha}+\langle x,\alpha\rangle\right),
\]
where $f_{\alpha}=-\infty$ except for finitely many values of $\alpha$,
and with $\langle x,\alpha\rangle=\sum_{i}x_{i}\cdot\alpha_{i}$ under
the convention $(-\infty)\cdot0=0$. The \emph{tropical hypersurface} associated with $f$ is the set of points from
$\mathbb{R}_{\max}^{d}$ where the maximum in $f$ is attained twice.
Tropical hypersurfaces are piecewise linear and divide $\mathbb{R}_{\max}^{d}$
between sectors corresponding to the maximal monomial: they can perform
more complex classification.

If $\mathcal{A}\subset\mathbb{Z}^{d}$ is a set of vectors, we define
the \emph{Veronese embedding} of $x$ as:
\[
\text{ver}_{\mathcal{A}}(x)\coloneqq\left(\langle x,\alpha\rangle\right)_{\alpha\in\mathcal{A}}\in\mathbb{R}_{\max}^{\mathcal{A}},
\]
allowing us to map our point space into a larger space, made up of
various integer combinations of features. To take a fairly exhaustive
subset of monomials, we consider the integer combinations defined
by the integer points of the dilated simplex. Noting $s\in\mathbb{N}^{*}$
a scale parameter, we define
\[
\mathcal{A}_{s}\coloneqq\left\{x\in\mathbb{N}^{d},\quad x_{1}+\cdots+x_{d}=s\right\},
\]
which grows polynomially with $s$. Note that we choose homogeneous monomials to preserve the translation invariance property in the base space. The Veronese embedding defined
by $\mathcal{A}_{s}$ is invertible and if $\mathcal{H}_a$ is a tropical
hyperplane of apex $a$ in $\mathbb{R}_{\max}^{\mathcal{A}}$, $\text{ver}_{\mathcal{A}_{s}}^{-1}(\mathcal{H}_a)$
is a tropical polynomial hypersurface in the initial space, corresponding to the polynomial
\[ f_{-a, \mathcal{A}_s}(x)=\bigoplus_{\alpha\in \mathcal{A}_s} (-a_\alpha)x^\alpha \]
Hence, $\text{ver}_{\mathcal{A}_{s}}$ is a feature map
enabling us to perform piecewise linear classification in $\mathbb{R}_{\max}^{d}$. This brings us close to the framework of neural networks with piecewise linear activations, e.g. relaxed linear units \cite{zhang2018tropical}. $\mathcal{A}_1$ corresponds to tropical hyperplanes: we are generalizing the approach previously developed.

The initial metric can be easily linked to the metric in the augmented space: for $x\in\mathbb{R}_{\max}^d$, we have
\[
\left[\frac{d}{s+d}+\frac{sd^{2}}{(s+d)^{2}}\right]^{-1} \le \frac{\lVert\text{ver}(x)\rVert_{H,\mathbb{R}^{\mathcal{A}_s}_{\max}}}{\lVert x\rVert_{H,\mathbb{R}_{\max}^d}} \le s.
\]
Indeed,
\begin{align*}
\lVert\text{ver}(x)\rVert_{H,\mathbb{R}^{\mathcal{A}_s}_{\max}}&=\max_{\alpha\in\mathcal{A}_{s}}\sum_{i}\alpha_{i}x_{i}-\min_{\alpha\in\mathcal{A}^{s}}\sum_{i}\alpha_{i}x_{i}\\&\le s(\max x-\min x)=s\lVert x\rVert_{H,\mathbb{R}_{\max}^d}.
\end{align*}
The rest of the proof is in Appendix \ref{appendix:NormEquivalence}. We deduce the following result:
\begin{prop}
If data is separable in the augmented space, noting $(a, \rho(T))$ the eigenpair of $T$ applied to $\mathbb{R}_{\max}^s$, data is separable in the initial space by the tropical hypersurface of $f_{-a,\mathcal{A}^s}$ with a margin of at least $-\rho(T)/s$.
\end{prop}

In Figures \ref{fig:moon}, \ref{fig:circular} and \ref{fig:toy_reverse}, we visualize the decision boundaries of the piecewise linear classifier using monomials from $\mathcal{A}_3$. Dotted lines in light gray indicate merged sectors after assignment, and margin lower estimation is represented by the Hilbert balls around each point. These methods obviously work in higher dimensions, but we can only visualize them in lower ones.

\subsection*{Complexity}

The learning speed depends essentially on the dimension of the feature space, i.e. the starting dimension and the number of monomials considered. Although our iteration scheme is efficient, given that $$|\mathcal{A}^s| = \binom{s+d-1}{s},$$ the complexity grows very quickly, opening the way to feature selection questions for directly learning \emph{sparse} polynomials.

On the other hand, when the tropical polynomial is trained, since only active monomials are kept, inference is in $\mathcal{O}(rd)$ only, where $r$ is the number of monomials considered.

\subsection*{Feature selection}

A heuristic for adaptive monomial choice might be to sample pairs of points of different classes, and make perpendicular slopes admissible. In many cases, this gives better results, as we can see in Figures \ref{fig:nofeatureeng} and \ref{fig:featureeng}, in which we allow homogeneous positive real monomials.

\begin{figure}[!h]
    \centering
    \begin{subfigure}{0.35\textwidth}
        \centering
        \resizebox{\linewidth}{!}{%
        \centering
            \clipbox{0.35\width{} 0.25\height{} 0.25\width{} 0.35\height{}}{%% Creator: Matplotlib, PGF backend
%%
%% To include the figure in your LaTeX document, write
%%   \input{<filename>.pgf}
%%
%% Make sure the required packages are loaded in your preamble
%%   \usepackage{pgf}
%%
%% Also ensure that all the required font packages are loaded; for instance,
%% the lmodern package is sometimes necessary when using math font.
%%   \usepackage{lmodern}
%%
%% Figures using additional raster images can only be included by \input if
%% they are in the same directory as the main LaTeX file. For loading figures
%% from other directories you can use the `import` package
%%   \usepackage{import}
%%
%% and then include the figures with
%%   \import{<path to file>}{<filename>.pgf}
%%
%% Matplotlib used the following preamble
%%   
%%   \usepackage{fontspec}
%%   \setmainfont{DejaVuSerif.ttf}[Path=\detokenize{/Users/sam/Library/Python/3.9/lib/python/site-packages/matplotlib/mpl-data/fonts/ttf/}]
%%   \setsansfont{Arial.ttf}[Path=\detokenize{/System/Library/Fonts/Supplemental/}]
%%   \setmonofont{DejaVuSansMono.ttf}[Path=\detokenize{/Users/sam/Library/Python/3.9/lib/python/site-packages/matplotlib/mpl-data/fonts/ttf/}]
%%   \makeatletter\@ifpackageloaded{underscore}{}{\usepackage[strings]{underscore}}\makeatother
%%
\begingroup%
\makeatletter%
\begin{pgfpicture}%
\pgfpathrectangle{\pgfpointorigin}{\pgfqpoint{6.000000in}{6.000000in}}%
\pgfusepath{use as bounding box, clip}%
\begin{pgfscope}%
\pgfsetbuttcap%
\pgfsetmiterjoin%
\definecolor{currentfill}{rgb}{1.000000,1.000000,1.000000}%
\pgfsetfillcolor{currentfill}%
\pgfsetlinewidth{0.000000pt}%
\definecolor{currentstroke}{rgb}{1.000000,1.000000,1.000000}%
\pgfsetstrokecolor{currentstroke}%
\pgfsetdash{}{0pt}%
\pgfpathmoveto{\pgfqpoint{0.000000in}{0.000000in}}%
\pgfpathlineto{\pgfqpoint{6.000000in}{0.000000in}}%
\pgfpathlineto{\pgfqpoint{6.000000in}{6.000000in}}%
\pgfpathlineto{\pgfqpoint{0.000000in}{6.000000in}}%
\pgfpathlineto{\pgfqpoint{0.000000in}{0.000000in}}%
\pgfpathclose%
\pgfusepath{fill}%
\end{pgfscope}%
\begin{pgfscope}%
\pgfsetbuttcap%
\pgfsetmiterjoin%
\definecolor{currentfill}{rgb}{1.000000,1.000000,1.000000}%
\pgfsetfillcolor{currentfill}%
\pgfsetlinewidth{0.000000pt}%
\definecolor{currentstroke}{rgb}{0.000000,0.000000,0.000000}%
\pgfsetstrokecolor{currentstroke}%
\pgfsetstrokeopacity{0.000000}%
\pgfsetdash{}{0pt}%
\pgfpathmoveto{\pgfqpoint{0.765000in}{0.660000in}}%
\pgfpathlineto{\pgfqpoint{5.385000in}{0.660000in}}%
\pgfpathlineto{\pgfqpoint{5.385000in}{5.280000in}}%
\pgfpathlineto{\pgfqpoint{0.765000in}{5.280000in}}%
\pgfpathlineto{\pgfqpoint{0.765000in}{0.660000in}}%
\pgfpathclose%
\pgfusepath{fill}%
\end{pgfscope}%
\begin{pgfscope}%
\pgfpathrectangle{\pgfqpoint{0.765000in}{0.660000in}}{\pgfqpoint{4.620000in}{4.620000in}}%
\pgfusepath{clip}%
\pgfsetroundcap%
\pgfsetroundjoin%
\pgfsetlinewidth{1.204500pt}%
\definecolor{currentstroke}{rgb}{1.000000,0.576471,0.309804}%
\pgfsetstrokecolor{currentstroke}%
\pgfsetdash{}{0pt}%
\pgfpathmoveto{\pgfqpoint{2.763911in}{2.541538in}}%
\pgfusepath{stroke}%
\end{pgfscope}%
\begin{pgfscope}%
\pgfpathrectangle{\pgfqpoint{0.765000in}{0.660000in}}{\pgfqpoint{4.620000in}{4.620000in}}%
\pgfusepath{clip}%
\pgfsetbuttcap%
\pgfsetroundjoin%
\definecolor{currentfill}{rgb}{1.000000,0.576471,0.309804}%
\pgfsetfillcolor{currentfill}%
\pgfsetlinewidth{1.003750pt}%
\definecolor{currentstroke}{rgb}{1.000000,0.576471,0.309804}%
\pgfsetstrokecolor{currentstroke}%
\pgfsetdash{}{0pt}%
\pgfsys@defobject{currentmarker}{\pgfqpoint{-0.033333in}{-0.033333in}}{\pgfqpoint{0.033333in}{0.033333in}}{%
\pgfpathmoveto{\pgfqpoint{0.000000in}{-0.033333in}}%
\pgfpathcurveto{\pgfqpoint{0.008840in}{-0.033333in}}{\pgfqpoint{0.017319in}{-0.029821in}}{\pgfqpoint{0.023570in}{-0.023570in}}%
\pgfpathcurveto{\pgfqpoint{0.029821in}{-0.017319in}}{\pgfqpoint{0.033333in}{-0.008840in}}{\pgfqpoint{0.033333in}{0.000000in}}%
\pgfpathcurveto{\pgfqpoint{0.033333in}{0.008840in}}{\pgfqpoint{0.029821in}{0.017319in}}{\pgfqpoint{0.023570in}{0.023570in}}%
\pgfpathcurveto{\pgfqpoint{0.017319in}{0.029821in}}{\pgfqpoint{0.008840in}{0.033333in}}{\pgfqpoint{0.000000in}{0.033333in}}%
\pgfpathcurveto{\pgfqpoint{-0.008840in}{0.033333in}}{\pgfqpoint{-0.017319in}{0.029821in}}{\pgfqpoint{-0.023570in}{0.023570in}}%
\pgfpathcurveto{\pgfqpoint{-0.029821in}{0.017319in}}{\pgfqpoint{-0.033333in}{0.008840in}}{\pgfqpoint{-0.033333in}{0.000000in}}%
\pgfpathcurveto{\pgfqpoint{-0.033333in}{-0.008840in}}{\pgfqpoint{-0.029821in}{-0.017319in}}{\pgfqpoint{-0.023570in}{-0.023570in}}%
\pgfpathcurveto{\pgfqpoint{-0.017319in}{-0.029821in}}{\pgfqpoint{-0.008840in}{-0.033333in}}{\pgfqpoint{0.000000in}{-0.033333in}}%
\pgfpathlineto{\pgfqpoint{0.000000in}{-0.033333in}}%
\pgfpathclose%
\pgfusepath{stroke,fill}%
}%
\begin{pgfscope}%
\pgfsys@transformshift{2.763911in}{2.541538in}%
\pgfsys@useobject{currentmarker}{}%
\end{pgfscope}%
\end{pgfscope}%
\begin{pgfscope}%
\pgfpathrectangle{\pgfqpoint{0.765000in}{0.660000in}}{\pgfqpoint{4.620000in}{4.620000in}}%
\pgfusepath{clip}%
\pgfsetroundcap%
\pgfsetroundjoin%
\pgfsetlinewidth{1.204500pt}%
\definecolor{currentstroke}{rgb}{1.000000,0.576471,0.309804}%
\pgfsetstrokecolor{currentstroke}%
\pgfsetdash{}{0pt}%
\pgfpathmoveto{\pgfqpoint{3.058689in}{2.317980in}}%
\pgfusepath{stroke}%
\end{pgfscope}%
\begin{pgfscope}%
\pgfpathrectangle{\pgfqpoint{0.765000in}{0.660000in}}{\pgfqpoint{4.620000in}{4.620000in}}%
\pgfusepath{clip}%
\pgfsetbuttcap%
\pgfsetroundjoin%
\definecolor{currentfill}{rgb}{1.000000,0.576471,0.309804}%
\pgfsetfillcolor{currentfill}%
\pgfsetlinewidth{1.003750pt}%
\definecolor{currentstroke}{rgb}{1.000000,0.576471,0.309804}%
\pgfsetstrokecolor{currentstroke}%
\pgfsetdash{}{0pt}%
\pgfsys@defobject{currentmarker}{\pgfqpoint{-0.033333in}{-0.033333in}}{\pgfqpoint{0.033333in}{0.033333in}}{%
\pgfpathmoveto{\pgfqpoint{0.000000in}{-0.033333in}}%
\pgfpathcurveto{\pgfqpoint{0.008840in}{-0.033333in}}{\pgfqpoint{0.017319in}{-0.029821in}}{\pgfqpoint{0.023570in}{-0.023570in}}%
\pgfpathcurveto{\pgfqpoint{0.029821in}{-0.017319in}}{\pgfqpoint{0.033333in}{-0.008840in}}{\pgfqpoint{0.033333in}{0.000000in}}%
\pgfpathcurveto{\pgfqpoint{0.033333in}{0.008840in}}{\pgfqpoint{0.029821in}{0.017319in}}{\pgfqpoint{0.023570in}{0.023570in}}%
\pgfpathcurveto{\pgfqpoint{0.017319in}{0.029821in}}{\pgfqpoint{0.008840in}{0.033333in}}{\pgfqpoint{0.000000in}{0.033333in}}%
\pgfpathcurveto{\pgfqpoint{-0.008840in}{0.033333in}}{\pgfqpoint{-0.017319in}{0.029821in}}{\pgfqpoint{-0.023570in}{0.023570in}}%
\pgfpathcurveto{\pgfqpoint{-0.029821in}{0.017319in}}{\pgfqpoint{-0.033333in}{0.008840in}}{\pgfqpoint{-0.033333in}{0.000000in}}%
\pgfpathcurveto{\pgfqpoint{-0.033333in}{-0.008840in}}{\pgfqpoint{-0.029821in}{-0.017319in}}{\pgfqpoint{-0.023570in}{-0.023570in}}%
\pgfpathcurveto{\pgfqpoint{-0.017319in}{-0.029821in}}{\pgfqpoint{-0.008840in}{-0.033333in}}{\pgfqpoint{0.000000in}{-0.033333in}}%
\pgfpathlineto{\pgfqpoint{0.000000in}{-0.033333in}}%
\pgfpathclose%
\pgfusepath{stroke,fill}%
}%
\begin{pgfscope}%
\pgfsys@transformshift{3.058689in}{2.317980in}%
\pgfsys@useobject{currentmarker}{}%
\end{pgfscope}%
\end{pgfscope}%
\begin{pgfscope}%
\pgfpathrectangle{\pgfqpoint{0.765000in}{0.660000in}}{\pgfqpoint{4.620000in}{4.620000in}}%
\pgfusepath{clip}%
\pgfsetroundcap%
\pgfsetroundjoin%
\pgfsetlinewidth{1.204500pt}%
\definecolor{currentstroke}{rgb}{1.000000,0.576471,0.309804}%
\pgfsetstrokecolor{currentstroke}%
\pgfsetdash{}{0pt}%
\pgfpathmoveto{\pgfqpoint{2.639796in}{2.539542in}}%
\pgfusepath{stroke}%
\end{pgfscope}%
\begin{pgfscope}%
\pgfpathrectangle{\pgfqpoint{0.765000in}{0.660000in}}{\pgfqpoint{4.620000in}{4.620000in}}%
\pgfusepath{clip}%
\pgfsetbuttcap%
\pgfsetroundjoin%
\definecolor{currentfill}{rgb}{1.000000,0.576471,0.309804}%
\pgfsetfillcolor{currentfill}%
\pgfsetlinewidth{1.003750pt}%
\definecolor{currentstroke}{rgb}{1.000000,0.576471,0.309804}%
\pgfsetstrokecolor{currentstroke}%
\pgfsetdash{}{0pt}%
\pgfsys@defobject{currentmarker}{\pgfqpoint{-0.033333in}{-0.033333in}}{\pgfqpoint{0.033333in}{0.033333in}}{%
\pgfpathmoveto{\pgfqpoint{0.000000in}{-0.033333in}}%
\pgfpathcurveto{\pgfqpoint{0.008840in}{-0.033333in}}{\pgfqpoint{0.017319in}{-0.029821in}}{\pgfqpoint{0.023570in}{-0.023570in}}%
\pgfpathcurveto{\pgfqpoint{0.029821in}{-0.017319in}}{\pgfqpoint{0.033333in}{-0.008840in}}{\pgfqpoint{0.033333in}{0.000000in}}%
\pgfpathcurveto{\pgfqpoint{0.033333in}{0.008840in}}{\pgfqpoint{0.029821in}{0.017319in}}{\pgfqpoint{0.023570in}{0.023570in}}%
\pgfpathcurveto{\pgfqpoint{0.017319in}{0.029821in}}{\pgfqpoint{0.008840in}{0.033333in}}{\pgfqpoint{0.000000in}{0.033333in}}%
\pgfpathcurveto{\pgfqpoint{-0.008840in}{0.033333in}}{\pgfqpoint{-0.017319in}{0.029821in}}{\pgfqpoint{-0.023570in}{0.023570in}}%
\pgfpathcurveto{\pgfqpoint{-0.029821in}{0.017319in}}{\pgfqpoint{-0.033333in}{0.008840in}}{\pgfqpoint{-0.033333in}{0.000000in}}%
\pgfpathcurveto{\pgfqpoint{-0.033333in}{-0.008840in}}{\pgfqpoint{-0.029821in}{-0.017319in}}{\pgfqpoint{-0.023570in}{-0.023570in}}%
\pgfpathcurveto{\pgfqpoint{-0.017319in}{-0.029821in}}{\pgfqpoint{-0.008840in}{-0.033333in}}{\pgfqpoint{0.000000in}{-0.033333in}}%
\pgfpathlineto{\pgfqpoint{0.000000in}{-0.033333in}}%
\pgfpathclose%
\pgfusepath{stroke,fill}%
}%
\begin{pgfscope}%
\pgfsys@transformshift{2.639796in}{2.539542in}%
\pgfsys@useobject{currentmarker}{}%
\end{pgfscope}%
\end{pgfscope}%
\begin{pgfscope}%
\pgfpathrectangle{\pgfqpoint{0.765000in}{0.660000in}}{\pgfqpoint{4.620000in}{4.620000in}}%
\pgfusepath{clip}%
\pgfsetroundcap%
\pgfsetroundjoin%
\pgfsetlinewidth{1.204500pt}%
\definecolor{currentstroke}{rgb}{1.000000,0.576471,0.309804}%
\pgfsetstrokecolor{currentstroke}%
\pgfsetdash{}{0pt}%
\pgfpathmoveto{\pgfqpoint{3.154567in}{3.566202in}}%
\pgfusepath{stroke}%
\end{pgfscope}%
\begin{pgfscope}%
\pgfpathrectangle{\pgfqpoint{0.765000in}{0.660000in}}{\pgfqpoint{4.620000in}{4.620000in}}%
\pgfusepath{clip}%
\pgfsetbuttcap%
\pgfsetroundjoin%
\definecolor{currentfill}{rgb}{1.000000,0.576471,0.309804}%
\pgfsetfillcolor{currentfill}%
\pgfsetlinewidth{1.003750pt}%
\definecolor{currentstroke}{rgb}{1.000000,0.576471,0.309804}%
\pgfsetstrokecolor{currentstroke}%
\pgfsetdash{}{0pt}%
\pgfsys@defobject{currentmarker}{\pgfqpoint{-0.033333in}{-0.033333in}}{\pgfqpoint{0.033333in}{0.033333in}}{%
\pgfpathmoveto{\pgfqpoint{0.000000in}{-0.033333in}}%
\pgfpathcurveto{\pgfqpoint{0.008840in}{-0.033333in}}{\pgfqpoint{0.017319in}{-0.029821in}}{\pgfqpoint{0.023570in}{-0.023570in}}%
\pgfpathcurveto{\pgfqpoint{0.029821in}{-0.017319in}}{\pgfqpoint{0.033333in}{-0.008840in}}{\pgfqpoint{0.033333in}{0.000000in}}%
\pgfpathcurveto{\pgfqpoint{0.033333in}{0.008840in}}{\pgfqpoint{0.029821in}{0.017319in}}{\pgfqpoint{0.023570in}{0.023570in}}%
\pgfpathcurveto{\pgfqpoint{0.017319in}{0.029821in}}{\pgfqpoint{0.008840in}{0.033333in}}{\pgfqpoint{0.000000in}{0.033333in}}%
\pgfpathcurveto{\pgfqpoint{-0.008840in}{0.033333in}}{\pgfqpoint{-0.017319in}{0.029821in}}{\pgfqpoint{-0.023570in}{0.023570in}}%
\pgfpathcurveto{\pgfqpoint{-0.029821in}{0.017319in}}{\pgfqpoint{-0.033333in}{0.008840in}}{\pgfqpoint{-0.033333in}{0.000000in}}%
\pgfpathcurveto{\pgfqpoint{-0.033333in}{-0.008840in}}{\pgfqpoint{-0.029821in}{-0.017319in}}{\pgfqpoint{-0.023570in}{-0.023570in}}%
\pgfpathcurveto{\pgfqpoint{-0.017319in}{-0.029821in}}{\pgfqpoint{-0.008840in}{-0.033333in}}{\pgfqpoint{0.000000in}{-0.033333in}}%
\pgfpathlineto{\pgfqpoint{0.000000in}{-0.033333in}}%
\pgfpathclose%
\pgfusepath{stroke,fill}%
}%
\begin{pgfscope}%
\pgfsys@transformshift{3.154567in}{3.566202in}%
\pgfsys@useobject{currentmarker}{}%
\end{pgfscope}%
\end{pgfscope}%
\begin{pgfscope}%
\pgfpathrectangle{\pgfqpoint{0.765000in}{0.660000in}}{\pgfqpoint{4.620000in}{4.620000in}}%
\pgfusepath{clip}%
\pgfsetroundcap%
\pgfsetroundjoin%
\pgfsetlinewidth{1.204500pt}%
\definecolor{currentstroke}{rgb}{1.000000,0.576471,0.309804}%
\pgfsetstrokecolor{currentstroke}%
\pgfsetdash{}{0pt}%
\pgfpathmoveto{\pgfqpoint{4.076955in}{2.611573in}}%
\pgfusepath{stroke}%
\end{pgfscope}%
\begin{pgfscope}%
\pgfpathrectangle{\pgfqpoint{0.765000in}{0.660000in}}{\pgfqpoint{4.620000in}{4.620000in}}%
\pgfusepath{clip}%
\pgfsetbuttcap%
\pgfsetroundjoin%
\definecolor{currentfill}{rgb}{1.000000,0.576471,0.309804}%
\pgfsetfillcolor{currentfill}%
\pgfsetlinewidth{1.003750pt}%
\definecolor{currentstroke}{rgb}{1.000000,0.576471,0.309804}%
\pgfsetstrokecolor{currentstroke}%
\pgfsetdash{}{0pt}%
\pgfsys@defobject{currentmarker}{\pgfqpoint{-0.033333in}{-0.033333in}}{\pgfqpoint{0.033333in}{0.033333in}}{%
\pgfpathmoveto{\pgfqpoint{0.000000in}{-0.033333in}}%
\pgfpathcurveto{\pgfqpoint{0.008840in}{-0.033333in}}{\pgfqpoint{0.017319in}{-0.029821in}}{\pgfqpoint{0.023570in}{-0.023570in}}%
\pgfpathcurveto{\pgfqpoint{0.029821in}{-0.017319in}}{\pgfqpoint{0.033333in}{-0.008840in}}{\pgfqpoint{0.033333in}{0.000000in}}%
\pgfpathcurveto{\pgfqpoint{0.033333in}{0.008840in}}{\pgfqpoint{0.029821in}{0.017319in}}{\pgfqpoint{0.023570in}{0.023570in}}%
\pgfpathcurveto{\pgfqpoint{0.017319in}{0.029821in}}{\pgfqpoint{0.008840in}{0.033333in}}{\pgfqpoint{0.000000in}{0.033333in}}%
\pgfpathcurveto{\pgfqpoint{-0.008840in}{0.033333in}}{\pgfqpoint{-0.017319in}{0.029821in}}{\pgfqpoint{-0.023570in}{0.023570in}}%
\pgfpathcurveto{\pgfqpoint{-0.029821in}{0.017319in}}{\pgfqpoint{-0.033333in}{0.008840in}}{\pgfqpoint{-0.033333in}{0.000000in}}%
\pgfpathcurveto{\pgfqpoint{-0.033333in}{-0.008840in}}{\pgfqpoint{-0.029821in}{-0.017319in}}{\pgfqpoint{-0.023570in}{-0.023570in}}%
\pgfpathcurveto{\pgfqpoint{-0.017319in}{-0.029821in}}{\pgfqpoint{-0.008840in}{-0.033333in}}{\pgfqpoint{0.000000in}{-0.033333in}}%
\pgfpathlineto{\pgfqpoint{0.000000in}{-0.033333in}}%
\pgfpathclose%
\pgfusepath{stroke,fill}%
}%
\begin{pgfscope}%
\pgfsys@transformshift{4.076955in}{2.611573in}%
\pgfsys@useobject{currentmarker}{}%
\end{pgfscope}%
\end{pgfscope}%
\begin{pgfscope}%
\pgfpathrectangle{\pgfqpoint{0.765000in}{0.660000in}}{\pgfqpoint{4.620000in}{4.620000in}}%
\pgfusepath{clip}%
\pgfsetroundcap%
\pgfsetroundjoin%
\pgfsetlinewidth{1.204500pt}%
\definecolor{currentstroke}{rgb}{1.000000,0.576471,0.309804}%
\pgfsetstrokecolor{currentstroke}%
\pgfsetdash{}{0pt}%
\pgfpathmoveto{\pgfqpoint{3.898066in}{2.476508in}}%
\pgfusepath{stroke}%
\end{pgfscope}%
\begin{pgfscope}%
\pgfpathrectangle{\pgfqpoint{0.765000in}{0.660000in}}{\pgfqpoint{4.620000in}{4.620000in}}%
\pgfusepath{clip}%
\pgfsetbuttcap%
\pgfsetroundjoin%
\definecolor{currentfill}{rgb}{1.000000,0.576471,0.309804}%
\pgfsetfillcolor{currentfill}%
\pgfsetlinewidth{1.003750pt}%
\definecolor{currentstroke}{rgb}{1.000000,0.576471,0.309804}%
\pgfsetstrokecolor{currentstroke}%
\pgfsetdash{}{0pt}%
\pgfsys@defobject{currentmarker}{\pgfqpoint{-0.033333in}{-0.033333in}}{\pgfqpoint{0.033333in}{0.033333in}}{%
\pgfpathmoveto{\pgfqpoint{0.000000in}{-0.033333in}}%
\pgfpathcurveto{\pgfqpoint{0.008840in}{-0.033333in}}{\pgfqpoint{0.017319in}{-0.029821in}}{\pgfqpoint{0.023570in}{-0.023570in}}%
\pgfpathcurveto{\pgfqpoint{0.029821in}{-0.017319in}}{\pgfqpoint{0.033333in}{-0.008840in}}{\pgfqpoint{0.033333in}{0.000000in}}%
\pgfpathcurveto{\pgfqpoint{0.033333in}{0.008840in}}{\pgfqpoint{0.029821in}{0.017319in}}{\pgfqpoint{0.023570in}{0.023570in}}%
\pgfpathcurveto{\pgfqpoint{0.017319in}{0.029821in}}{\pgfqpoint{0.008840in}{0.033333in}}{\pgfqpoint{0.000000in}{0.033333in}}%
\pgfpathcurveto{\pgfqpoint{-0.008840in}{0.033333in}}{\pgfqpoint{-0.017319in}{0.029821in}}{\pgfqpoint{-0.023570in}{0.023570in}}%
\pgfpathcurveto{\pgfqpoint{-0.029821in}{0.017319in}}{\pgfqpoint{-0.033333in}{0.008840in}}{\pgfqpoint{-0.033333in}{0.000000in}}%
\pgfpathcurveto{\pgfqpoint{-0.033333in}{-0.008840in}}{\pgfqpoint{-0.029821in}{-0.017319in}}{\pgfqpoint{-0.023570in}{-0.023570in}}%
\pgfpathcurveto{\pgfqpoint{-0.017319in}{-0.029821in}}{\pgfqpoint{-0.008840in}{-0.033333in}}{\pgfqpoint{0.000000in}{-0.033333in}}%
\pgfpathlineto{\pgfqpoint{0.000000in}{-0.033333in}}%
\pgfpathclose%
\pgfusepath{stroke,fill}%
}%
\begin{pgfscope}%
\pgfsys@transformshift{3.898066in}{2.476508in}%
\pgfsys@useobject{currentmarker}{}%
\end{pgfscope}%
\end{pgfscope}%
\begin{pgfscope}%
\pgfpathrectangle{\pgfqpoint{0.765000in}{0.660000in}}{\pgfqpoint{4.620000in}{4.620000in}}%
\pgfusepath{clip}%
\pgfsetroundcap%
\pgfsetroundjoin%
\pgfsetlinewidth{1.204500pt}%
\definecolor{currentstroke}{rgb}{1.000000,0.576471,0.309804}%
\pgfsetstrokecolor{currentstroke}%
\pgfsetdash{}{0pt}%
\pgfpathmoveto{\pgfqpoint{2.635652in}{2.667025in}}%
\pgfusepath{stroke}%
\end{pgfscope}%
\begin{pgfscope}%
\pgfpathrectangle{\pgfqpoint{0.765000in}{0.660000in}}{\pgfqpoint{4.620000in}{4.620000in}}%
\pgfusepath{clip}%
\pgfsetbuttcap%
\pgfsetroundjoin%
\definecolor{currentfill}{rgb}{1.000000,0.576471,0.309804}%
\pgfsetfillcolor{currentfill}%
\pgfsetlinewidth{1.003750pt}%
\definecolor{currentstroke}{rgb}{1.000000,0.576471,0.309804}%
\pgfsetstrokecolor{currentstroke}%
\pgfsetdash{}{0pt}%
\pgfsys@defobject{currentmarker}{\pgfqpoint{-0.033333in}{-0.033333in}}{\pgfqpoint{0.033333in}{0.033333in}}{%
\pgfpathmoveto{\pgfqpoint{0.000000in}{-0.033333in}}%
\pgfpathcurveto{\pgfqpoint{0.008840in}{-0.033333in}}{\pgfqpoint{0.017319in}{-0.029821in}}{\pgfqpoint{0.023570in}{-0.023570in}}%
\pgfpathcurveto{\pgfqpoint{0.029821in}{-0.017319in}}{\pgfqpoint{0.033333in}{-0.008840in}}{\pgfqpoint{0.033333in}{0.000000in}}%
\pgfpathcurveto{\pgfqpoint{0.033333in}{0.008840in}}{\pgfqpoint{0.029821in}{0.017319in}}{\pgfqpoint{0.023570in}{0.023570in}}%
\pgfpathcurveto{\pgfqpoint{0.017319in}{0.029821in}}{\pgfqpoint{0.008840in}{0.033333in}}{\pgfqpoint{0.000000in}{0.033333in}}%
\pgfpathcurveto{\pgfqpoint{-0.008840in}{0.033333in}}{\pgfqpoint{-0.017319in}{0.029821in}}{\pgfqpoint{-0.023570in}{0.023570in}}%
\pgfpathcurveto{\pgfqpoint{-0.029821in}{0.017319in}}{\pgfqpoint{-0.033333in}{0.008840in}}{\pgfqpoint{-0.033333in}{0.000000in}}%
\pgfpathcurveto{\pgfqpoint{-0.033333in}{-0.008840in}}{\pgfqpoint{-0.029821in}{-0.017319in}}{\pgfqpoint{-0.023570in}{-0.023570in}}%
\pgfpathcurveto{\pgfqpoint{-0.017319in}{-0.029821in}}{\pgfqpoint{-0.008840in}{-0.033333in}}{\pgfqpoint{0.000000in}{-0.033333in}}%
\pgfpathlineto{\pgfqpoint{0.000000in}{-0.033333in}}%
\pgfpathclose%
\pgfusepath{stroke,fill}%
}%
\begin{pgfscope}%
\pgfsys@transformshift{2.635652in}{2.667025in}%
\pgfsys@useobject{currentmarker}{}%
\end{pgfscope}%
\end{pgfscope}%
\begin{pgfscope}%
\pgfpathrectangle{\pgfqpoint{0.765000in}{0.660000in}}{\pgfqpoint{4.620000in}{4.620000in}}%
\pgfusepath{clip}%
\pgfsetroundcap%
\pgfsetroundjoin%
\pgfsetlinewidth{1.204500pt}%
\definecolor{currentstroke}{rgb}{1.000000,0.576471,0.309804}%
\pgfsetstrokecolor{currentstroke}%
\pgfsetdash{}{0pt}%
\pgfpathmoveto{\pgfqpoint{3.392622in}{3.280058in}}%
\pgfusepath{stroke}%
\end{pgfscope}%
\begin{pgfscope}%
\pgfpathrectangle{\pgfqpoint{0.765000in}{0.660000in}}{\pgfqpoint{4.620000in}{4.620000in}}%
\pgfusepath{clip}%
\pgfsetbuttcap%
\pgfsetroundjoin%
\definecolor{currentfill}{rgb}{1.000000,0.576471,0.309804}%
\pgfsetfillcolor{currentfill}%
\pgfsetlinewidth{1.003750pt}%
\definecolor{currentstroke}{rgb}{1.000000,0.576471,0.309804}%
\pgfsetstrokecolor{currentstroke}%
\pgfsetdash{}{0pt}%
\pgfsys@defobject{currentmarker}{\pgfqpoint{-0.033333in}{-0.033333in}}{\pgfqpoint{0.033333in}{0.033333in}}{%
\pgfpathmoveto{\pgfqpoint{0.000000in}{-0.033333in}}%
\pgfpathcurveto{\pgfqpoint{0.008840in}{-0.033333in}}{\pgfqpoint{0.017319in}{-0.029821in}}{\pgfqpoint{0.023570in}{-0.023570in}}%
\pgfpathcurveto{\pgfqpoint{0.029821in}{-0.017319in}}{\pgfqpoint{0.033333in}{-0.008840in}}{\pgfqpoint{0.033333in}{0.000000in}}%
\pgfpathcurveto{\pgfqpoint{0.033333in}{0.008840in}}{\pgfqpoint{0.029821in}{0.017319in}}{\pgfqpoint{0.023570in}{0.023570in}}%
\pgfpathcurveto{\pgfqpoint{0.017319in}{0.029821in}}{\pgfqpoint{0.008840in}{0.033333in}}{\pgfqpoint{0.000000in}{0.033333in}}%
\pgfpathcurveto{\pgfqpoint{-0.008840in}{0.033333in}}{\pgfqpoint{-0.017319in}{0.029821in}}{\pgfqpoint{-0.023570in}{0.023570in}}%
\pgfpathcurveto{\pgfqpoint{-0.029821in}{0.017319in}}{\pgfqpoint{-0.033333in}{0.008840in}}{\pgfqpoint{-0.033333in}{0.000000in}}%
\pgfpathcurveto{\pgfqpoint{-0.033333in}{-0.008840in}}{\pgfqpoint{-0.029821in}{-0.017319in}}{\pgfqpoint{-0.023570in}{-0.023570in}}%
\pgfpathcurveto{\pgfqpoint{-0.017319in}{-0.029821in}}{\pgfqpoint{-0.008840in}{-0.033333in}}{\pgfqpoint{0.000000in}{-0.033333in}}%
\pgfpathlineto{\pgfqpoint{0.000000in}{-0.033333in}}%
\pgfpathclose%
\pgfusepath{stroke,fill}%
}%
\begin{pgfscope}%
\pgfsys@transformshift{3.392622in}{3.280058in}%
\pgfsys@useobject{currentmarker}{}%
\end{pgfscope}%
\end{pgfscope}%
\begin{pgfscope}%
\pgfpathrectangle{\pgfqpoint{0.765000in}{0.660000in}}{\pgfqpoint{4.620000in}{4.620000in}}%
\pgfusepath{clip}%
\pgfsetroundcap%
\pgfsetroundjoin%
\pgfsetlinewidth{1.204500pt}%
\definecolor{currentstroke}{rgb}{1.000000,0.576471,0.309804}%
\pgfsetstrokecolor{currentstroke}%
\pgfsetdash{}{0pt}%
\pgfpathmoveto{\pgfqpoint{2.654944in}{3.100985in}}%
\pgfusepath{stroke}%
\end{pgfscope}%
\begin{pgfscope}%
\pgfpathrectangle{\pgfqpoint{0.765000in}{0.660000in}}{\pgfqpoint{4.620000in}{4.620000in}}%
\pgfusepath{clip}%
\pgfsetbuttcap%
\pgfsetroundjoin%
\definecolor{currentfill}{rgb}{1.000000,0.576471,0.309804}%
\pgfsetfillcolor{currentfill}%
\pgfsetlinewidth{1.003750pt}%
\definecolor{currentstroke}{rgb}{1.000000,0.576471,0.309804}%
\pgfsetstrokecolor{currentstroke}%
\pgfsetdash{}{0pt}%
\pgfsys@defobject{currentmarker}{\pgfqpoint{-0.033333in}{-0.033333in}}{\pgfqpoint{0.033333in}{0.033333in}}{%
\pgfpathmoveto{\pgfqpoint{0.000000in}{-0.033333in}}%
\pgfpathcurveto{\pgfqpoint{0.008840in}{-0.033333in}}{\pgfqpoint{0.017319in}{-0.029821in}}{\pgfqpoint{0.023570in}{-0.023570in}}%
\pgfpathcurveto{\pgfqpoint{0.029821in}{-0.017319in}}{\pgfqpoint{0.033333in}{-0.008840in}}{\pgfqpoint{0.033333in}{0.000000in}}%
\pgfpathcurveto{\pgfqpoint{0.033333in}{0.008840in}}{\pgfqpoint{0.029821in}{0.017319in}}{\pgfqpoint{0.023570in}{0.023570in}}%
\pgfpathcurveto{\pgfqpoint{0.017319in}{0.029821in}}{\pgfqpoint{0.008840in}{0.033333in}}{\pgfqpoint{0.000000in}{0.033333in}}%
\pgfpathcurveto{\pgfqpoint{-0.008840in}{0.033333in}}{\pgfqpoint{-0.017319in}{0.029821in}}{\pgfqpoint{-0.023570in}{0.023570in}}%
\pgfpathcurveto{\pgfqpoint{-0.029821in}{0.017319in}}{\pgfqpoint{-0.033333in}{0.008840in}}{\pgfqpoint{-0.033333in}{0.000000in}}%
\pgfpathcurveto{\pgfqpoint{-0.033333in}{-0.008840in}}{\pgfqpoint{-0.029821in}{-0.017319in}}{\pgfqpoint{-0.023570in}{-0.023570in}}%
\pgfpathcurveto{\pgfqpoint{-0.017319in}{-0.029821in}}{\pgfqpoint{-0.008840in}{-0.033333in}}{\pgfqpoint{0.000000in}{-0.033333in}}%
\pgfpathlineto{\pgfqpoint{0.000000in}{-0.033333in}}%
\pgfpathclose%
\pgfusepath{stroke,fill}%
}%
\begin{pgfscope}%
\pgfsys@transformshift{2.654944in}{3.100985in}%
\pgfsys@useobject{currentmarker}{}%
\end{pgfscope}%
\end{pgfscope}%
\begin{pgfscope}%
\pgfpathrectangle{\pgfqpoint{0.765000in}{0.660000in}}{\pgfqpoint{4.620000in}{4.620000in}}%
\pgfusepath{clip}%
\pgfsetroundcap%
\pgfsetroundjoin%
\pgfsetlinewidth{1.204500pt}%
\definecolor{currentstroke}{rgb}{1.000000,0.576471,0.309804}%
\pgfsetstrokecolor{currentstroke}%
\pgfsetdash{}{0pt}%
\pgfpathmoveto{\pgfqpoint{3.280709in}{2.266279in}}%
\pgfusepath{stroke}%
\end{pgfscope}%
\begin{pgfscope}%
\pgfpathrectangle{\pgfqpoint{0.765000in}{0.660000in}}{\pgfqpoint{4.620000in}{4.620000in}}%
\pgfusepath{clip}%
\pgfsetbuttcap%
\pgfsetroundjoin%
\definecolor{currentfill}{rgb}{1.000000,0.576471,0.309804}%
\pgfsetfillcolor{currentfill}%
\pgfsetlinewidth{1.003750pt}%
\definecolor{currentstroke}{rgb}{1.000000,0.576471,0.309804}%
\pgfsetstrokecolor{currentstroke}%
\pgfsetdash{}{0pt}%
\pgfsys@defobject{currentmarker}{\pgfqpoint{-0.033333in}{-0.033333in}}{\pgfqpoint{0.033333in}{0.033333in}}{%
\pgfpathmoveto{\pgfqpoint{0.000000in}{-0.033333in}}%
\pgfpathcurveto{\pgfqpoint{0.008840in}{-0.033333in}}{\pgfqpoint{0.017319in}{-0.029821in}}{\pgfqpoint{0.023570in}{-0.023570in}}%
\pgfpathcurveto{\pgfqpoint{0.029821in}{-0.017319in}}{\pgfqpoint{0.033333in}{-0.008840in}}{\pgfqpoint{0.033333in}{0.000000in}}%
\pgfpathcurveto{\pgfqpoint{0.033333in}{0.008840in}}{\pgfqpoint{0.029821in}{0.017319in}}{\pgfqpoint{0.023570in}{0.023570in}}%
\pgfpathcurveto{\pgfqpoint{0.017319in}{0.029821in}}{\pgfqpoint{0.008840in}{0.033333in}}{\pgfqpoint{0.000000in}{0.033333in}}%
\pgfpathcurveto{\pgfqpoint{-0.008840in}{0.033333in}}{\pgfqpoint{-0.017319in}{0.029821in}}{\pgfqpoint{-0.023570in}{0.023570in}}%
\pgfpathcurveto{\pgfqpoint{-0.029821in}{0.017319in}}{\pgfqpoint{-0.033333in}{0.008840in}}{\pgfqpoint{-0.033333in}{0.000000in}}%
\pgfpathcurveto{\pgfqpoint{-0.033333in}{-0.008840in}}{\pgfqpoint{-0.029821in}{-0.017319in}}{\pgfqpoint{-0.023570in}{-0.023570in}}%
\pgfpathcurveto{\pgfqpoint{-0.017319in}{-0.029821in}}{\pgfqpoint{-0.008840in}{-0.033333in}}{\pgfqpoint{0.000000in}{-0.033333in}}%
\pgfpathlineto{\pgfqpoint{0.000000in}{-0.033333in}}%
\pgfpathclose%
\pgfusepath{stroke,fill}%
}%
\begin{pgfscope}%
\pgfsys@transformshift{3.280709in}{2.266279in}%
\pgfsys@useobject{currentmarker}{}%
\end{pgfscope}%
\end{pgfscope}%
\begin{pgfscope}%
\pgfpathrectangle{\pgfqpoint{0.765000in}{0.660000in}}{\pgfqpoint{4.620000in}{4.620000in}}%
\pgfusepath{clip}%
\pgfsetroundcap%
\pgfsetroundjoin%
\pgfsetlinewidth{1.204500pt}%
\definecolor{currentstroke}{rgb}{1.000000,0.576471,0.309804}%
\pgfsetstrokecolor{currentstroke}%
\pgfsetdash{}{0pt}%
\pgfpathmoveto{\pgfqpoint{3.938125in}{3.250765in}}%
\pgfusepath{stroke}%
\end{pgfscope}%
\begin{pgfscope}%
\pgfpathrectangle{\pgfqpoint{0.765000in}{0.660000in}}{\pgfqpoint{4.620000in}{4.620000in}}%
\pgfusepath{clip}%
\pgfsetbuttcap%
\pgfsetroundjoin%
\definecolor{currentfill}{rgb}{1.000000,0.576471,0.309804}%
\pgfsetfillcolor{currentfill}%
\pgfsetlinewidth{1.003750pt}%
\definecolor{currentstroke}{rgb}{1.000000,0.576471,0.309804}%
\pgfsetstrokecolor{currentstroke}%
\pgfsetdash{}{0pt}%
\pgfsys@defobject{currentmarker}{\pgfqpoint{-0.033333in}{-0.033333in}}{\pgfqpoint{0.033333in}{0.033333in}}{%
\pgfpathmoveto{\pgfqpoint{0.000000in}{-0.033333in}}%
\pgfpathcurveto{\pgfqpoint{0.008840in}{-0.033333in}}{\pgfqpoint{0.017319in}{-0.029821in}}{\pgfqpoint{0.023570in}{-0.023570in}}%
\pgfpathcurveto{\pgfqpoint{0.029821in}{-0.017319in}}{\pgfqpoint{0.033333in}{-0.008840in}}{\pgfqpoint{0.033333in}{0.000000in}}%
\pgfpathcurveto{\pgfqpoint{0.033333in}{0.008840in}}{\pgfqpoint{0.029821in}{0.017319in}}{\pgfqpoint{0.023570in}{0.023570in}}%
\pgfpathcurveto{\pgfqpoint{0.017319in}{0.029821in}}{\pgfqpoint{0.008840in}{0.033333in}}{\pgfqpoint{0.000000in}{0.033333in}}%
\pgfpathcurveto{\pgfqpoint{-0.008840in}{0.033333in}}{\pgfqpoint{-0.017319in}{0.029821in}}{\pgfqpoint{-0.023570in}{0.023570in}}%
\pgfpathcurveto{\pgfqpoint{-0.029821in}{0.017319in}}{\pgfqpoint{-0.033333in}{0.008840in}}{\pgfqpoint{-0.033333in}{0.000000in}}%
\pgfpathcurveto{\pgfqpoint{-0.033333in}{-0.008840in}}{\pgfqpoint{-0.029821in}{-0.017319in}}{\pgfqpoint{-0.023570in}{-0.023570in}}%
\pgfpathcurveto{\pgfqpoint{-0.017319in}{-0.029821in}}{\pgfqpoint{-0.008840in}{-0.033333in}}{\pgfqpoint{0.000000in}{-0.033333in}}%
\pgfpathlineto{\pgfqpoint{0.000000in}{-0.033333in}}%
\pgfpathclose%
\pgfusepath{stroke,fill}%
}%
\begin{pgfscope}%
\pgfsys@transformshift{3.938125in}{3.250765in}%
\pgfsys@useobject{currentmarker}{}%
\end{pgfscope}%
\end{pgfscope}%
\begin{pgfscope}%
\pgfpathrectangle{\pgfqpoint{0.765000in}{0.660000in}}{\pgfqpoint{4.620000in}{4.620000in}}%
\pgfusepath{clip}%
\pgfsetroundcap%
\pgfsetroundjoin%
\pgfsetlinewidth{1.204500pt}%
\definecolor{currentstroke}{rgb}{1.000000,0.576471,0.309804}%
\pgfsetstrokecolor{currentstroke}%
\pgfsetdash{}{0pt}%
\pgfpathmoveto{\pgfqpoint{3.625633in}{3.289218in}}%
\pgfusepath{stroke}%
\end{pgfscope}%
\begin{pgfscope}%
\pgfpathrectangle{\pgfqpoint{0.765000in}{0.660000in}}{\pgfqpoint{4.620000in}{4.620000in}}%
\pgfusepath{clip}%
\pgfsetbuttcap%
\pgfsetroundjoin%
\definecolor{currentfill}{rgb}{1.000000,0.576471,0.309804}%
\pgfsetfillcolor{currentfill}%
\pgfsetlinewidth{1.003750pt}%
\definecolor{currentstroke}{rgb}{1.000000,0.576471,0.309804}%
\pgfsetstrokecolor{currentstroke}%
\pgfsetdash{}{0pt}%
\pgfsys@defobject{currentmarker}{\pgfqpoint{-0.033333in}{-0.033333in}}{\pgfqpoint{0.033333in}{0.033333in}}{%
\pgfpathmoveto{\pgfqpoint{0.000000in}{-0.033333in}}%
\pgfpathcurveto{\pgfqpoint{0.008840in}{-0.033333in}}{\pgfqpoint{0.017319in}{-0.029821in}}{\pgfqpoint{0.023570in}{-0.023570in}}%
\pgfpathcurveto{\pgfqpoint{0.029821in}{-0.017319in}}{\pgfqpoint{0.033333in}{-0.008840in}}{\pgfqpoint{0.033333in}{0.000000in}}%
\pgfpathcurveto{\pgfqpoint{0.033333in}{0.008840in}}{\pgfqpoint{0.029821in}{0.017319in}}{\pgfqpoint{0.023570in}{0.023570in}}%
\pgfpathcurveto{\pgfqpoint{0.017319in}{0.029821in}}{\pgfqpoint{0.008840in}{0.033333in}}{\pgfqpoint{0.000000in}{0.033333in}}%
\pgfpathcurveto{\pgfqpoint{-0.008840in}{0.033333in}}{\pgfqpoint{-0.017319in}{0.029821in}}{\pgfqpoint{-0.023570in}{0.023570in}}%
\pgfpathcurveto{\pgfqpoint{-0.029821in}{0.017319in}}{\pgfqpoint{-0.033333in}{0.008840in}}{\pgfqpoint{-0.033333in}{0.000000in}}%
\pgfpathcurveto{\pgfqpoint{-0.033333in}{-0.008840in}}{\pgfqpoint{-0.029821in}{-0.017319in}}{\pgfqpoint{-0.023570in}{-0.023570in}}%
\pgfpathcurveto{\pgfqpoint{-0.017319in}{-0.029821in}}{\pgfqpoint{-0.008840in}{-0.033333in}}{\pgfqpoint{0.000000in}{-0.033333in}}%
\pgfpathlineto{\pgfqpoint{0.000000in}{-0.033333in}}%
\pgfpathclose%
\pgfusepath{stroke,fill}%
}%
\begin{pgfscope}%
\pgfsys@transformshift{3.625633in}{3.289218in}%
\pgfsys@useobject{currentmarker}{}%
\end{pgfscope}%
\end{pgfscope}%
\begin{pgfscope}%
\pgfpathrectangle{\pgfqpoint{0.765000in}{0.660000in}}{\pgfqpoint{4.620000in}{4.620000in}}%
\pgfusepath{clip}%
\pgfsetroundcap%
\pgfsetroundjoin%
\pgfsetlinewidth{1.204500pt}%
\definecolor{currentstroke}{rgb}{1.000000,0.576471,0.309804}%
\pgfsetstrokecolor{currentstroke}%
\pgfsetdash{}{0pt}%
\pgfpathmoveto{\pgfqpoint{3.810829in}{1.809809in}}%
\pgfusepath{stroke}%
\end{pgfscope}%
\begin{pgfscope}%
\pgfpathrectangle{\pgfqpoint{0.765000in}{0.660000in}}{\pgfqpoint{4.620000in}{4.620000in}}%
\pgfusepath{clip}%
\pgfsetbuttcap%
\pgfsetroundjoin%
\definecolor{currentfill}{rgb}{1.000000,0.576471,0.309804}%
\pgfsetfillcolor{currentfill}%
\pgfsetlinewidth{1.003750pt}%
\definecolor{currentstroke}{rgb}{1.000000,0.576471,0.309804}%
\pgfsetstrokecolor{currentstroke}%
\pgfsetdash{}{0pt}%
\pgfsys@defobject{currentmarker}{\pgfqpoint{-0.033333in}{-0.033333in}}{\pgfqpoint{0.033333in}{0.033333in}}{%
\pgfpathmoveto{\pgfqpoint{0.000000in}{-0.033333in}}%
\pgfpathcurveto{\pgfqpoint{0.008840in}{-0.033333in}}{\pgfqpoint{0.017319in}{-0.029821in}}{\pgfqpoint{0.023570in}{-0.023570in}}%
\pgfpathcurveto{\pgfqpoint{0.029821in}{-0.017319in}}{\pgfqpoint{0.033333in}{-0.008840in}}{\pgfqpoint{0.033333in}{0.000000in}}%
\pgfpathcurveto{\pgfqpoint{0.033333in}{0.008840in}}{\pgfqpoint{0.029821in}{0.017319in}}{\pgfqpoint{0.023570in}{0.023570in}}%
\pgfpathcurveto{\pgfqpoint{0.017319in}{0.029821in}}{\pgfqpoint{0.008840in}{0.033333in}}{\pgfqpoint{0.000000in}{0.033333in}}%
\pgfpathcurveto{\pgfqpoint{-0.008840in}{0.033333in}}{\pgfqpoint{-0.017319in}{0.029821in}}{\pgfqpoint{-0.023570in}{0.023570in}}%
\pgfpathcurveto{\pgfqpoint{-0.029821in}{0.017319in}}{\pgfqpoint{-0.033333in}{0.008840in}}{\pgfqpoint{-0.033333in}{0.000000in}}%
\pgfpathcurveto{\pgfqpoint{-0.033333in}{-0.008840in}}{\pgfqpoint{-0.029821in}{-0.017319in}}{\pgfqpoint{-0.023570in}{-0.023570in}}%
\pgfpathcurveto{\pgfqpoint{-0.017319in}{-0.029821in}}{\pgfqpoint{-0.008840in}{-0.033333in}}{\pgfqpoint{0.000000in}{-0.033333in}}%
\pgfpathlineto{\pgfqpoint{0.000000in}{-0.033333in}}%
\pgfpathclose%
\pgfusepath{stroke,fill}%
}%
\begin{pgfscope}%
\pgfsys@transformshift{3.810829in}{1.809809in}%
\pgfsys@useobject{currentmarker}{}%
\end{pgfscope}%
\end{pgfscope}%
\begin{pgfscope}%
\pgfpathrectangle{\pgfqpoint{0.765000in}{0.660000in}}{\pgfqpoint{4.620000in}{4.620000in}}%
\pgfusepath{clip}%
\pgfsetroundcap%
\pgfsetroundjoin%
\pgfsetlinewidth{1.204500pt}%
\definecolor{currentstroke}{rgb}{1.000000,0.576471,0.309804}%
\pgfsetstrokecolor{currentstroke}%
\pgfsetdash{}{0pt}%
\pgfpathmoveto{\pgfqpoint{2.663465in}{3.011940in}}%
\pgfusepath{stroke}%
\end{pgfscope}%
\begin{pgfscope}%
\pgfpathrectangle{\pgfqpoint{0.765000in}{0.660000in}}{\pgfqpoint{4.620000in}{4.620000in}}%
\pgfusepath{clip}%
\pgfsetbuttcap%
\pgfsetroundjoin%
\definecolor{currentfill}{rgb}{1.000000,0.576471,0.309804}%
\pgfsetfillcolor{currentfill}%
\pgfsetlinewidth{1.003750pt}%
\definecolor{currentstroke}{rgb}{1.000000,0.576471,0.309804}%
\pgfsetstrokecolor{currentstroke}%
\pgfsetdash{}{0pt}%
\pgfsys@defobject{currentmarker}{\pgfqpoint{-0.033333in}{-0.033333in}}{\pgfqpoint{0.033333in}{0.033333in}}{%
\pgfpathmoveto{\pgfqpoint{0.000000in}{-0.033333in}}%
\pgfpathcurveto{\pgfqpoint{0.008840in}{-0.033333in}}{\pgfqpoint{0.017319in}{-0.029821in}}{\pgfqpoint{0.023570in}{-0.023570in}}%
\pgfpathcurveto{\pgfqpoint{0.029821in}{-0.017319in}}{\pgfqpoint{0.033333in}{-0.008840in}}{\pgfqpoint{0.033333in}{0.000000in}}%
\pgfpathcurveto{\pgfqpoint{0.033333in}{0.008840in}}{\pgfqpoint{0.029821in}{0.017319in}}{\pgfqpoint{0.023570in}{0.023570in}}%
\pgfpathcurveto{\pgfqpoint{0.017319in}{0.029821in}}{\pgfqpoint{0.008840in}{0.033333in}}{\pgfqpoint{0.000000in}{0.033333in}}%
\pgfpathcurveto{\pgfqpoint{-0.008840in}{0.033333in}}{\pgfqpoint{-0.017319in}{0.029821in}}{\pgfqpoint{-0.023570in}{0.023570in}}%
\pgfpathcurveto{\pgfqpoint{-0.029821in}{0.017319in}}{\pgfqpoint{-0.033333in}{0.008840in}}{\pgfqpoint{-0.033333in}{0.000000in}}%
\pgfpathcurveto{\pgfqpoint{-0.033333in}{-0.008840in}}{\pgfqpoint{-0.029821in}{-0.017319in}}{\pgfqpoint{-0.023570in}{-0.023570in}}%
\pgfpathcurveto{\pgfqpoint{-0.017319in}{-0.029821in}}{\pgfqpoint{-0.008840in}{-0.033333in}}{\pgfqpoint{0.000000in}{-0.033333in}}%
\pgfpathlineto{\pgfqpoint{0.000000in}{-0.033333in}}%
\pgfpathclose%
\pgfusepath{stroke,fill}%
}%
\begin{pgfscope}%
\pgfsys@transformshift{2.663465in}{3.011940in}%
\pgfsys@useobject{currentmarker}{}%
\end{pgfscope}%
\end{pgfscope}%
\begin{pgfscope}%
\pgfpathrectangle{\pgfqpoint{0.765000in}{0.660000in}}{\pgfqpoint{4.620000in}{4.620000in}}%
\pgfusepath{clip}%
\pgfsetroundcap%
\pgfsetroundjoin%
\pgfsetlinewidth{1.204500pt}%
\definecolor{currentstroke}{rgb}{1.000000,0.576471,0.309804}%
\pgfsetstrokecolor{currentstroke}%
\pgfsetdash{}{0pt}%
\pgfpathmoveto{\pgfqpoint{3.661262in}{2.378627in}}%
\pgfusepath{stroke}%
\end{pgfscope}%
\begin{pgfscope}%
\pgfpathrectangle{\pgfqpoint{0.765000in}{0.660000in}}{\pgfqpoint{4.620000in}{4.620000in}}%
\pgfusepath{clip}%
\pgfsetbuttcap%
\pgfsetroundjoin%
\definecolor{currentfill}{rgb}{1.000000,0.576471,0.309804}%
\pgfsetfillcolor{currentfill}%
\pgfsetlinewidth{1.003750pt}%
\definecolor{currentstroke}{rgb}{1.000000,0.576471,0.309804}%
\pgfsetstrokecolor{currentstroke}%
\pgfsetdash{}{0pt}%
\pgfsys@defobject{currentmarker}{\pgfqpoint{-0.033333in}{-0.033333in}}{\pgfqpoint{0.033333in}{0.033333in}}{%
\pgfpathmoveto{\pgfqpoint{0.000000in}{-0.033333in}}%
\pgfpathcurveto{\pgfqpoint{0.008840in}{-0.033333in}}{\pgfqpoint{0.017319in}{-0.029821in}}{\pgfqpoint{0.023570in}{-0.023570in}}%
\pgfpathcurveto{\pgfqpoint{0.029821in}{-0.017319in}}{\pgfqpoint{0.033333in}{-0.008840in}}{\pgfqpoint{0.033333in}{0.000000in}}%
\pgfpathcurveto{\pgfqpoint{0.033333in}{0.008840in}}{\pgfqpoint{0.029821in}{0.017319in}}{\pgfqpoint{0.023570in}{0.023570in}}%
\pgfpathcurveto{\pgfqpoint{0.017319in}{0.029821in}}{\pgfqpoint{0.008840in}{0.033333in}}{\pgfqpoint{0.000000in}{0.033333in}}%
\pgfpathcurveto{\pgfqpoint{-0.008840in}{0.033333in}}{\pgfqpoint{-0.017319in}{0.029821in}}{\pgfqpoint{-0.023570in}{0.023570in}}%
\pgfpathcurveto{\pgfqpoint{-0.029821in}{0.017319in}}{\pgfqpoint{-0.033333in}{0.008840in}}{\pgfqpoint{-0.033333in}{0.000000in}}%
\pgfpathcurveto{\pgfqpoint{-0.033333in}{-0.008840in}}{\pgfqpoint{-0.029821in}{-0.017319in}}{\pgfqpoint{-0.023570in}{-0.023570in}}%
\pgfpathcurveto{\pgfqpoint{-0.017319in}{-0.029821in}}{\pgfqpoint{-0.008840in}{-0.033333in}}{\pgfqpoint{0.000000in}{-0.033333in}}%
\pgfpathlineto{\pgfqpoint{0.000000in}{-0.033333in}}%
\pgfpathclose%
\pgfusepath{stroke,fill}%
}%
\begin{pgfscope}%
\pgfsys@transformshift{3.661262in}{2.378627in}%
\pgfsys@useobject{currentmarker}{}%
\end{pgfscope}%
\end{pgfscope}%
\begin{pgfscope}%
\pgfpathrectangle{\pgfqpoint{0.765000in}{0.660000in}}{\pgfqpoint{4.620000in}{4.620000in}}%
\pgfusepath{clip}%
\pgfsetroundcap%
\pgfsetroundjoin%
\pgfsetlinewidth{1.204500pt}%
\definecolor{currentstroke}{rgb}{1.000000,0.576471,0.309804}%
\pgfsetstrokecolor{currentstroke}%
\pgfsetdash{}{0pt}%
\pgfpathmoveto{\pgfqpoint{3.407442in}{3.257671in}}%
\pgfusepath{stroke}%
\end{pgfscope}%
\begin{pgfscope}%
\pgfpathrectangle{\pgfqpoint{0.765000in}{0.660000in}}{\pgfqpoint{4.620000in}{4.620000in}}%
\pgfusepath{clip}%
\pgfsetbuttcap%
\pgfsetroundjoin%
\definecolor{currentfill}{rgb}{1.000000,0.576471,0.309804}%
\pgfsetfillcolor{currentfill}%
\pgfsetlinewidth{1.003750pt}%
\definecolor{currentstroke}{rgb}{1.000000,0.576471,0.309804}%
\pgfsetstrokecolor{currentstroke}%
\pgfsetdash{}{0pt}%
\pgfsys@defobject{currentmarker}{\pgfqpoint{-0.033333in}{-0.033333in}}{\pgfqpoint{0.033333in}{0.033333in}}{%
\pgfpathmoveto{\pgfqpoint{0.000000in}{-0.033333in}}%
\pgfpathcurveto{\pgfqpoint{0.008840in}{-0.033333in}}{\pgfqpoint{0.017319in}{-0.029821in}}{\pgfqpoint{0.023570in}{-0.023570in}}%
\pgfpathcurveto{\pgfqpoint{0.029821in}{-0.017319in}}{\pgfqpoint{0.033333in}{-0.008840in}}{\pgfqpoint{0.033333in}{0.000000in}}%
\pgfpathcurveto{\pgfqpoint{0.033333in}{0.008840in}}{\pgfqpoint{0.029821in}{0.017319in}}{\pgfqpoint{0.023570in}{0.023570in}}%
\pgfpathcurveto{\pgfqpoint{0.017319in}{0.029821in}}{\pgfqpoint{0.008840in}{0.033333in}}{\pgfqpoint{0.000000in}{0.033333in}}%
\pgfpathcurveto{\pgfqpoint{-0.008840in}{0.033333in}}{\pgfqpoint{-0.017319in}{0.029821in}}{\pgfqpoint{-0.023570in}{0.023570in}}%
\pgfpathcurveto{\pgfqpoint{-0.029821in}{0.017319in}}{\pgfqpoint{-0.033333in}{0.008840in}}{\pgfqpoint{-0.033333in}{0.000000in}}%
\pgfpathcurveto{\pgfqpoint{-0.033333in}{-0.008840in}}{\pgfqpoint{-0.029821in}{-0.017319in}}{\pgfqpoint{-0.023570in}{-0.023570in}}%
\pgfpathcurveto{\pgfqpoint{-0.017319in}{-0.029821in}}{\pgfqpoint{-0.008840in}{-0.033333in}}{\pgfqpoint{0.000000in}{-0.033333in}}%
\pgfpathlineto{\pgfqpoint{0.000000in}{-0.033333in}}%
\pgfpathclose%
\pgfusepath{stroke,fill}%
}%
\begin{pgfscope}%
\pgfsys@transformshift{3.407442in}{3.257671in}%
\pgfsys@useobject{currentmarker}{}%
\end{pgfscope}%
\end{pgfscope}%
\begin{pgfscope}%
\pgfpathrectangle{\pgfqpoint{0.765000in}{0.660000in}}{\pgfqpoint{4.620000in}{4.620000in}}%
\pgfusepath{clip}%
\pgfsetroundcap%
\pgfsetroundjoin%
\pgfsetlinewidth{1.204500pt}%
\definecolor{currentstroke}{rgb}{1.000000,0.576471,0.309804}%
\pgfsetstrokecolor{currentstroke}%
\pgfsetdash{}{0pt}%
\pgfpathmoveto{\pgfqpoint{3.583556in}{2.437570in}}%
\pgfusepath{stroke}%
\end{pgfscope}%
\begin{pgfscope}%
\pgfpathrectangle{\pgfqpoint{0.765000in}{0.660000in}}{\pgfqpoint{4.620000in}{4.620000in}}%
\pgfusepath{clip}%
\pgfsetbuttcap%
\pgfsetroundjoin%
\definecolor{currentfill}{rgb}{1.000000,0.576471,0.309804}%
\pgfsetfillcolor{currentfill}%
\pgfsetlinewidth{1.003750pt}%
\definecolor{currentstroke}{rgb}{1.000000,0.576471,0.309804}%
\pgfsetstrokecolor{currentstroke}%
\pgfsetdash{}{0pt}%
\pgfsys@defobject{currentmarker}{\pgfqpoint{-0.033333in}{-0.033333in}}{\pgfqpoint{0.033333in}{0.033333in}}{%
\pgfpathmoveto{\pgfqpoint{0.000000in}{-0.033333in}}%
\pgfpathcurveto{\pgfqpoint{0.008840in}{-0.033333in}}{\pgfqpoint{0.017319in}{-0.029821in}}{\pgfqpoint{0.023570in}{-0.023570in}}%
\pgfpathcurveto{\pgfqpoint{0.029821in}{-0.017319in}}{\pgfqpoint{0.033333in}{-0.008840in}}{\pgfqpoint{0.033333in}{0.000000in}}%
\pgfpathcurveto{\pgfqpoint{0.033333in}{0.008840in}}{\pgfqpoint{0.029821in}{0.017319in}}{\pgfqpoint{0.023570in}{0.023570in}}%
\pgfpathcurveto{\pgfqpoint{0.017319in}{0.029821in}}{\pgfqpoint{0.008840in}{0.033333in}}{\pgfqpoint{0.000000in}{0.033333in}}%
\pgfpathcurveto{\pgfqpoint{-0.008840in}{0.033333in}}{\pgfqpoint{-0.017319in}{0.029821in}}{\pgfqpoint{-0.023570in}{0.023570in}}%
\pgfpathcurveto{\pgfqpoint{-0.029821in}{0.017319in}}{\pgfqpoint{-0.033333in}{0.008840in}}{\pgfqpoint{-0.033333in}{0.000000in}}%
\pgfpathcurveto{\pgfqpoint{-0.033333in}{-0.008840in}}{\pgfqpoint{-0.029821in}{-0.017319in}}{\pgfqpoint{-0.023570in}{-0.023570in}}%
\pgfpathcurveto{\pgfqpoint{-0.017319in}{-0.029821in}}{\pgfqpoint{-0.008840in}{-0.033333in}}{\pgfqpoint{0.000000in}{-0.033333in}}%
\pgfpathlineto{\pgfqpoint{0.000000in}{-0.033333in}}%
\pgfpathclose%
\pgfusepath{stroke,fill}%
}%
\begin{pgfscope}%
\pgfsys@transformshift{3.583556in}{2.437570in}%
\pgfsys@useobject{currentmarker}{}%
\end{pgfscope}%
\end{pgfscope}%
\begin{pgfscope}%
\pgfpathrectangle{\pgfqpoint{0.765000in}{0.660000in}}{\pgfqpoint{4.620000in}{4.620000in}}%
\pgfusepath{clip}%
\pgfsetroundcap%
\pgfsetroundjoin%
\pgfsetlinewidth{1.204500pt}%
\definecolor{currentstroke}{rgb}{1.000000,0.576471,0.309804}%
\pgfsetstrokecolor{currentstroke}%
\pgfsetdash{}{0pt}%
\pgfpathmoveto{\pgfqpoint{3.773932in}{2.514303in}}%
\pgfusepath{stroke}%
\end{pgfscope}%
\begin{pgfscope}%
\pgfpathrectangle{\pgfqpoint{0.765000in}{0.660000in}}{\pgfqpoint{4.620000in}{4.620000in}}%
\pgfusepath{clip}%
\pgfsetbuttcap%
\pgfsetroundjoin%
\definecolor{currentfill}{rgb}{1.000000,0.576471,0.309804}%
\pgfsetfillcolor{currentfill}%
\pgfsetlinewidth{1.003750pt}%
\definecolor{currentstroke}{rgb}{1.000000,0.576471,0.309804}%
\pgfsetstrokecolor{currentstroke}%
\pgfsetdash{}{0pt}%
\pgfsys@defobject{currentmarker}{\pgfqpoint{-0.033333in}{-0.033333in}}{\pgfqpoint{0.033333in}{0.033333in}}{%
\pgfpathmoveto{\pgfqpoint{0.000000in}{-0.033333in}}%
\pgfpathcurveto{\pgfqpoint{0.008840in}{-0.033333in}}{\pgfqpoint{0.017319in}{-0.029821in}}{\pgfqpoint{0.023570in}{-0.023570in}}%
\pgfpathcurveto{\pgfqpoint{0.029821in}{-0.017319in}}{\pgfqpoint{0.033333in}{-0.008840in}}{\pgfqpoint{0.033333in}{0.000000in}}%
\pgfpathcurveto{\pgfqpoint{0.033333in}{0.008840in}}{\pgfqpoint{0.029821in}{0.017319in}}{\pgfqpoint{0.023570in}{0.023570in}}%
\pgfpathcurveto{\pgfqpoint{0.017319in}{0.029821in}}{\pgfqpoint{0.008840in}{0.033333in}}{\pgfqpoint{0.000000in}{0.033333in}}%
\pgfpathcurveto{\pgfqpoint{-0.008840in}{0.033333in}}{\pgfqpoint{-0.017319in}{0.029821in}}{\pgfqpoint{-0.023570in}{0.023570in}}%
\pgfpathcurveto{\pgfqpoint{-0.029821in}{0.017319in}}{\pgfqpoint{-0.033333in}{0.008840in}}{\pgfqpoint{-0.033333in}{0.000000in}}%
\pgfpathcurveto{\pgfqpoint{-0.033333in}{-0.008840in}}{\pgfqpoint{-0.029821in}{-0.017319in}}{\pgfqpoint{-0.023570in}{-0.023570in}}%
\pgfpathcurveto{\pgfqpoint{-0.017319in}{-0.029821in}}{\pgfqpoint{-0.008840in}{-0.033333in}}{\pgfqpoint{0.000000in}{-0.033333in}}%
\pgfpathlineto{\pgfqpoint{0.000000in}{-0.033333in}}%
\pgfpathclose%
\pgfusepath{stroke,fill}%
}%
\begin{pgfscope}%
\pgfsys@transformshift{3.773932in}{2.514303in}%
\pgfsys@useobject{currentmarker}{}%
\end{pgfscope}%
\end{pgfscope}%
\begin{pgfscope}%
\pgfpathrectangle{\pgfqpoint{0.765000in}{0.660000in}}{\pgfqpoint{4.620000in}{4.620000in}}%
\pgfusepath{clip}%
\pgfsetroundcap%
\pgfsetroundjoin%
\pgfsetlinewidth{1.204500pt}%
\definecolor{currentstroke}{rgb}{1.000000,0.576471,0.309804}%
\pgfsetstrokecolor{currentstroke}%
\pgfsetdash{}{0pt}%
\pgfpathmoveto{\pgfqpoint{3.208285in}{2.211483in}}%
\pgfusepath{stroke}%
\end{pgfscope}%
\begin{pgfscope}%
\pgfpathrectangle{\pgfqpoint{0.765000in}{0.660000in}}{\pgfqpoint{4.620000in}{4.620000in}}%
\pgfusepath{clip}%
\pgfsetbuttcap%
\pgfsetroundjoin%
\definecolor{currentfill}{rgb}{1.000000,0.576471,0.309804}%
\pgfsetfillcolor{currentfill}%
\pgfsetlinewidth{1.003750pt}%
\definecolor{currentstroke}{rgb}{1.000000,0.576471,0.309804}%
\pgfsetstrokecolor{currentstroke}%
\pgfsetdash{}{0pt}%
\pgfsys@defobject{currentmarker}{\pgfqpoint{-0.033333in}{-0.033333in}}{\pgfqpoint{0.033333in}{0.033333in}}{%
\pgfpathmoveto{\pgfqpoint{0.000000in}{-0.033333in}}%
\pgfpathcurveto{\pgfqpoint{0.008840in}{-0.033333in}}{\pgfqpoint{0.017319in}{-0.029821in}}{\pgfqpoint{0.023570in}{-0.023570in}}%
\pgfpathcurveto{\pgfqpoint{0.029821in}{-0.017319in}}{\pgfqpoint{0.033333in}{-0.008840in}}{\pgfqpoint{0.033333in}{0.000000in}}%
\pgfpathcurveto{\pgfqpoint{0.033333in}{0.008840in}}{\pgfqpoint{0.029821in}{0.017319in}}{\pgfqpoint{0.023570in}{0.023570in}}%
\pgfpathcurveto{\pgfqpoint{0.017319in}{0.029821in}}{\pgfqpoint{0.008840in}{0.033333in}}{\pgfqpoint{0.000000in}{0.033333in}}%
\pgfpathcurveto{\pgfqpoint{-0.008840in}{0.033333in}}{\pgfqpoint{-0.017319in}{0.029821in}}{\pgfqpoint{-0.023570in}{0.023570in}}%
\pgfpathcurveto{\pgfqpoint{-0.029821in}{0.017319in}}{\pgfqpoint{-0.033333in}{0.008840in}}{\pgfqpoint{-0.033333in}{0.000000in}}%
\pgfpathcurveto{\pgfqpoint{-0.033333in}{-0.008840in}}{\pgfqpoint{-0.029821in}{-0.017319in}}{\pgfqpoint{-0.023570in}{-0.023570in}}%
\pgfpathcurveto{\pgfqpoint{-0.017319in}{-0.029821in}}{\pgfqpoint{-0.008840in}{-0.033333in}}{\pgfqpoint{0.000000in}{-0.033333in}}%
\pgfpathlineto{\pgfqpoint{0.000000in}{-0.033333in}}%
\pgfpathclose%
\pgfusepath{stroke,fill}%
}%
\begin{pgfscope}%
\pgfsys@transformshift{3.208285in}{2.211483in}%
\pgfsys@useobject{currentmarker}{}%
\end{pgfscope}%
\end{pgfscope}%
\begin{pgfscope}%
\pgfpathrectangle{\pgfqpoint{0.765000in}{0.660000in}}{\pgfqpoint{4.620000in}{4.620000in}}%
\pgfusepath{clip}%
\pgfsetroundcap%
\pgfsetroundjoin%
\pgfsetlinewidth{1.204500pt}%
\definecolor{currentstroke}{rgb}{1.000000,0.576471,0.309804}%
\pgfsetstrokecolor{currentstroke}%
\pgfsetdash{}{0pt}%
\pgfpathmoveto{\pgfqpoint{3.338848in}{3.512914in}}%
\pgfusepath{stroke}%
\end{pgfscope}%
\begin{pgfscope}%
\pgfpathrectangle{\pgfqpoint{0.765000in}{0.660000in}}{\pgfqpoint{4.620000in}{4.620000in}}%
\pgfusepath{clip}%
\pgfsetbuttcap%
\pgfsetroundjoin%
\definecolor{currentfill}{rgb}{1.000000,0.576471,0.309804}%
\pgfsetfillcolor{currentfill}%
\pgfsetlinewidth{1.003750pt}%
\definecolor{currentstroke}{rgb}{1.000000,0.576471,0.309804}%
\pgfsetstrokecolor{currentstroke}%
\pgfsetdash{}{0pt}%
\pgfsys@defobject{currentmarker}{\pgfqpoint{-0.033333in}{-0.033333in}}{\pgfqpoint{0.033333in}{0.033333in}}{%
\pgfpathmoveto{\pgfqpoint{0.000000in}{-0.033333in}}%
\pgfpathcurveto{\pgfqpoint{0.008840in}{-0.033333in}}{\pgfqpoint{0.017319in}{-0.029821in}}{\pgfqpoint{0.023570in}{-0.023570in}}%
\pgfpathcurveto{\pgfqpoint{0.029821in}{-0.017319in}}{\pgfqpoint{0.033333in}{-0.008840in}}{\pgfqpoint{0.033333in}{0.000000in}}%
\pgfpathcurveto{\pgfqpoint{0.033333in}{0.008840in}}{\pgfqpoint{0.029821in}{0.017319in}}{\pgfqpoint{0.023570in}{0.023570in}}%
\pgfpathcurveto{\pgfqpoint{0.017319in}{0.029821in}}{\pgfqpoint{0.008840in}{0.033333in}}{\pgfqpoint{0.000000in}{0.033333in}}%
\pgfpathcurveto{\pgfqpoint{-0.008840in}{0.033333in}}{\pgfqpoint{-0.017319in}{0.029821in}}{\pgfqpoint{-0.023570in}{0.023570in}}%
\pgfpathcurveto{\pgfqpoint{-0.029821in}{0.017319in}}{\pgfqpoint{-0.033333in}{0.008840in}}{\pgfqpoint{-0.033333in}{0.000000in}}%
\pgfpathcurveto{\pgfqpoint{-0.033333in}{-0.008840in}}{\pgfqpoint{-0.029821in}{-0.017319in}}{\pgfqpoint{-0.023570in}{-0.023570in}}%
\pgfpathcurveto{\pgfqpoint{-0.017319in}{-0.029821in}}{\pgfqpoint{-0.008840in}{-0.033333in}}{\pgfqpoint{0.000000in}{-0.033333in}}%
\pgfpathlineto{\pgfqpoint{0.000000in}{-0.033333in}}%
\pgfpathclose%
\pgfusepath{stroke,fill}%
}%
\begin{pgfscope}%
\pgfsys@transformshift{3.338848in}{3.512914in}%
\pgfsys@useobject{currentmarker}{}%
\end{pgfscope}%
\end{pgfscope}%
\begin{pgfscope}%
\pgfpathrectangle{\pgfqpoint{0.765000in}{0.660000in}}{\pgfqpoint{4.620000in}{4.620000in}}%
\pgfusepath{clip}%
\pgfsetroundcap%
\pgfsetroundjoin%
\pgfsetlinewidth{1.204500pt}%
\definecolor{currentstroke}{rgb}{1.000000,0.576471,0.309804}%
\pgfsetstrokecolor{currentstroke}%
\pgfsetdash{}{0pt}%
\pgfpathmoveto{\pgfqpoint{4.032528in}{3.021926in}}%
\pgfusepath{stroke}%
\end{pgfscope}%
\begin{pgfscope}%
\pgfpathrectangle{\pgfqpoint{0.765000in}{0.660000in}}{\pgfqpoint{4.620000in}{4.620000in}}%
\pgfusepath{clip}%
\pgfsetbuttcap%
\pgfsetroundjoin%
\definecolor{currentfill}{rgb}{1.000000,0.576471,0.309804}%
\pgfsetfillcolor{currentfill}%
\pgfsetlinewidth{1.003750pt}%
\definecolor{currentstroke}{rgb}{1.000000,0.576471,0.309804}%
\pgfsetstrokecolor{currentstroke}%
\pgfsetdash{}{0pt}%
\pgfsys@defobject{currentmarker}{\pgfqpoint{-0.033333in}{-0.033333in}}{\pgfqpoint{0.033333in}{0.033333in}}{%
\pgfpathmoveto{\pgfqpoint{0.000000in}{-0.033333in}}%
\pgfpathcurveto{\pgfqpoint{0.008840in}{-0.033333in}}{\pgfqpoint{0.017319in}{-0.029821in}}{\pgfqpoint{0.023570in}{-0.023570in}}%
\pgfpathcurveto{\pgfqpoint{0.029821in}{-0.017319in}}{\pgfqpoint{0.033333in}{-0.008840in}}{\pgfqpoint{0.033333in}{0.000000in}}%
\pgfpathcurveto{\pgfqpoint{0.033333in}{0.008840in}}{\pgfqpoint{0.029821in}{0.017319in}}{\pgfqpoint{0.023570in}{0.023570in}}%
\pgfpathcurveto{\pgfqpoint{0.017319in}{0.029821in}}{\pgfqpoint{0.008840in}{0.033333in}}{\pgfqpoint{0.000000in}{0.033333in}}%
\pgfpathcurveto{\pgfqpoint{-0.008840in}{0.033333in}}{\pgfqpoint{-0.017319in}{0.029821in}}{\pgfqpoint{-0.023570in}{0.023570in}}%
\pgfpathcurveto{\pgfqpoint{-0.029821in}{0.017319in}}{\pgfqpoint{-0.033333in}{0.008840in}}{\pgfqpoint{-0.033333in}{0.000000in}}%
\pgfpathcurveto{\pgfqpoint{-0.033333in}{-0.008840in}}{\pgfqpoint{-0.029821in}{-0.017319in}}{\pgfqpoint{-0.023570in}{-0.023570in}}%
\pgfpathcurveto{\pgfqpoint{-0.017319in}{-0.029821in}}{\pgfqpoint{-0.008840in}{-0.033333in}}{\pgfqpoint{0.000000in}{-0.033333in}}%
\pgfpathlineto{\pgfqpoint{0.000000in}{-0.033333in}}%
\pgfpathclose%
\pgfusepath{stroke,fill}%
}%
\begin{pgfscope}%
\pgfsys@transformshift{4.032528in}{3.021926in}%
\pgfsys@useobject{currentmarker}{}%
\end{pgfscope}%
\end{pgfscope}%
\begin{pgfscope}%
\pgfpathrectangle{\pgfqpoint{0.765000in}{0.660000in}}{\pgfqpoint{4.620000in}{4.620000in}}%
\pgfusepath{clip}%
\pgfsetroundcap%
\pgfsetroundjoin%
\pgfsetlinewidth{1.204500pt}%
\definecolor{currentstroke}{rgb}{1.000000,0.576471,0.309804}%
\pgfsetstrokecolor{currentstroke}%
\pgfsetdash{}{0pt}%
\pgfpathmoveto{\pgfqpoint{2.781610in}{3.225520in}}%
\pgfusepath{stroke}%
\end{pgfscope}%
\begin{pgfscope}%
\pgfpathrectangle{\pgfqpoint{0.765000in}{0.660000in}}{\pgfqpoint{4.620000in}{4.620000in}}%
\pgfusepath{clip}%
\pgfsetbuttcap%
\pgfsetroundjoin%
\definecolor{currentfill}{rgb}{1.000000,0.576471,0.309804}%
\pgfsetfillcolor{currentfill}%
\pgfsetlinewidth{1.003750pt}%
\definecolor{currentstroke}{rgb}{1.000000,0.576471,0.309804}%
\pgfsetstrokecolor{currentstroke}%
\pgfsetdash{}{0pt}%
\pgfsys@defobject{currentmarker}{\pgfqpoint{-0.033333in}{-0.033333in}}{\pgfqpoint{0.033333in}{0.033333in}}{%
\pgfpathmoveto{\pgfqpoint{0.000000in}{-0.033333in}}%
\pgfpathcurveto{\pgfqpoint{0.008840in}{-0.033333in}}{\pgfqpoint{0.017319in}{-0.029821in}}{\pgfqpoint{0.023570in}{-0.023570in}}%
\pgfpathcurveto{\pgfqpoint{0.029821in}{-0.017319in}}{\pgfqpoint{0.033333in}{-0.008840in}}{\pgfqpoint{0.033333in}{0.000000in}}%
\pgfpathcurveto{\pgfqpoint{0.033333in}{0.008840in}}{\pgfqpoint{0.029821in}{0.017319in}}{\pgfqpoint{0.023570in}{0.023570in}}%
\pgfpathcurveto{\pgfqpoint{0.017319in}{0.029821in}}{\pgfqpoint{0.008840in}{0.033333in}}{\pgfqpoint{0.000000in}{0.033333in}}%
\pgfpathcurveto{\pgfqpoint{-0.008840in}{0.033333in}}{\pgfqpoint{-0.017319in}{0.029821in}}{\pgfqpoint{-0.023570in}{0.023570in}}%
\pgfpathcurveto{\pgfqpoint{-0.029821in}{0.017319in}}{\pgfqpoint{-0.033333in}{0.008840in}}{\pgfqpoint{-0.033333in}{0.000000in}}%
\pgfpathcurveto{\pgfqpoint{-0.033333in}{-0.008840in}}{\pgfqpoint{-0.029821in}{-0.017319in}}{\pgfqpoint{-0.023570in}{-0.023570in}}%
\pgfpathcurveto{\pgfqpoint{-0.017319in}{-0.029821in}}{\pgfqpoint{-0.008840in}{-0.033333in}}{\pgfqpoint{0.000000in}{-0.033333in}}%
\pgfpathlineto{\pgfqpoint{0.000000in}{-0.033333in}}%
\pgfpathclose%
\pgfusepath{stroke,fill}%
}%
\begin{pgfscope}%
\pgfsys@transformshift{2.781610in}{3.225520in}%
\pgfsys@useobject{currentmarker}{}%
\end{pgfscope}%
\end{pgfscope}%
\begin{pgfscope}%
\pgfpathrectangle{\pgfqpoint{0.765000in}{0.660000in}}{\pgfqpoint{4.620000in}{4.620000in}}%
\pgfusepath{clip}%
\pgfsetroundcap%
\pgfsetroundjoin%
\pgfsetlinewidth{1.204500pt}%
\definecolor{currentstroke}{rgb}{1.000000,0.576471,0.309804}%
\pgfsetstrokecolor{currentstroke}%
\pgfsetdash{}{0pt}%
\pgfpathmoveto{\pgfqpoint{3.931881in}{3.557495in}}%
\pgfusepath{stroke}%
\end{pgfscope}%
\begin{pgfscope}%
\pgfpathrectangle{\pgfqpoint{0.765000in}{0.660000in}}{\pgfqpoint{4.620000in}{4.620000in}}%
\pgfusepath{clip}%
\pgfsetbuttcap%
\pgfsetroundjoin%
\definecolor{currentfill}{rgb}{1.000000,0.576471,0.309804}%
\pgfsetfillcolor{currentfill}%
\pgfsetlinewidth{1.003750pt}%
\definecolor{currentstroke}{rgb}{1.000000,0.576471,0.309804}%
\pgfsetstrokecolor{currentstroke}%
\pgfsetdash{}{0pt}%
\pgfsys@defobject{currentmarker}{\pgfqpoint{-0.033333in}{-0.033333in}}{\pgfqpoint{0.033333in}{0.033333in}}{%
\pgfpathmoveto{\pgfqpoint{0.000000in}{-0.033333in}}%
\pgfpathcurveto{\pgfqpoint{0.008840in}{-0.033333in}}{\pgfqpoint{0.017319in}{-0.029821in}}{\pgfqpoint{0.023570in}{-0.023570in}}%
\pgfpathcurveto{\pgfqpoint{0.029821in}{-0.017319in}}{\pgfqpoint{0.033333in}{-0.008840in}}{\pgfqpoint{0.033333in}{0.000000in}}%
\pgfpathcurveto{\pgfqpoint{0.033333in}{0.008840in}}{\pgfqpoint{0.029821in}{0.017319in}}{\pgfqpoint{0.023570in}{0.023570in}}%
\pgfpathcurveto{\pgfqpoint{0.017319in}{0.029821in}}{\pgfqpoint{0.008840in}{0.033333in}}{\pgfqpoint{0.000000in}{0.033333in}}%
\pgfpathcurveto{\pgfqpoint{-0.008840in}{0.033333in}}{\pgfqpoint{-0.017319in}{0.029821in}}{\pgfqpoint{-0.023570in}{0.023570in}}%
\pgfpathcurveto{\pgfqpoint{-0.029821in}{0.017319in}}{\pgfqpoint{-0.033333in}{0.008840in}}{\pgfqpoint{-0.033333in}{0.000000in}}%
\pgfpathcurveto{\pgfqpoint{-0.033333in}{-0.008840in}}{\pgfqpoint{-0.029821in}{-0.017319in}}{\pgfqpoint{-0.023570in}{-0.023570in}}%
\pgfpathcurveto{\pgfqpoint{-0.017319in}{-0.029821in}}{\pgfqpoint{-0.008840in}{-0.033333in}}{\pgfqpoint{0.000000in}{-0.033333in}}%
\pgfpathlineto{\pgfqpoint{0.000000in}{-0.033333in}}%
\pgfpathclose%
\pgfusepath{stroke,fill}%
}%
\begin{pgfscope}%
\pgfsys@transformshift{3.931881in}{3.557495in}%
\pgfsys@useobject{currentmarker}{}%
\end{pgfscope}%
\end{pgfscope}%
\begin{pgfscope}%
\pgfpathrectangle{\pgfqpoint{0.765000in}{0.660000in}}{\pgfqpoint{4.620000in}{4.620000in}}%
\pgfusepath{clip}%
\pgfsetroundcap%
\pgfsetroundjoin%
\pgfsetlinewidth{1.204500pt}%
\definecolor{currentstroke}{rgb}{1.000000,0.576471,0.309804}%
\pgfsetstrokecolor{currentstroke}%
\pgfsetdash{}{0pt}%
\pgfpathmoveto{\pgfqpoint{2.985288in}{3.119283in}}%
\pgfusepath{stroke}%
\end{pgfscope}%
\begin{pgfscope}%
\pgfpathrectangle{\pgfqpoint{0.765000in}{0.660000in}}{\pgfqpoint{4.620000in}{4.620000in}}%
\pgfusepath{clip}%
\pgfsetbuttcap%
\pgfsetroundjoin%
\definecolor{currentfill}{rgb}{1.000000,0.576471,0.309804}%
\pgfsetfillcolor{currentfill}%
\pgfsetlinewidth{1.003750pt}%
\definecolor{currentstroke}{rgb}{1.000000,0.576471,0.309804}%
\pgfsetstrokecolor{currentstroke}%
\pgfsetdash{}{0pt}%
\pgfsys@defobject{currentmarker}{\pgfqpoint{-0.033333in}{-0.033333in}}{\pgfqpoint{0.033333in}{0.033333in}}{%
\pgfpathmoveto{\pgfqpoint{0.000000in}{-0.033333in}}%
\pgfpathcurveto{\pgfqpoint{0.008840in}{-0.033333in}}{\pgfqpoint{0.017319in}{-0.029821in}}{\pgfqpoint{0.023570in}{-0.023570in}}%
\pgfpathcurveto{\pgfqpoint{0.029821in}{-0.017319in}}{\pgfqpoint{0.033333in}{-0.008840in}}{\pgfqpoint{0.033333in}{0.000000in}}%
\pgfpathcurveto{\pgfqpoint{0.033333in}{0.008840in}}{\pgfqpoint{0.029821in}{0.017319in}}{\pgfqpoint{0.023570in}{0.023570in}}%
\pgfpathcurveto{\pgfqpoint{0.017319in}{0.029821in}}{\pgfqpoint{0.008840in}{0.033333in}}{\pgfqpoint{0.000000in}{0.033333in}}%
\pgfpathcurveto{\pgfqpoint{-0.008840in}{0.033333in}}{\pgfqpoint{-0.017319in}{0.029821in}}{\pgfqpoint{-0.023570in}{0.023570in}}%
\pgfpathcurveto{\pgfqpoint{-0.029821in}{0.017319in}}{\pgfqpoint{-0.033333in}{0.008840in}}{\pgfqpoint{-0.033333in}{0.000000in}}%
\pgfpathcurveto{\pgfqpoint{-0.033333in}{-0.008840in}}{\pgfqpoint{-0.029821in}{-0.017319in}}{\pgfqpoint{-0.023570in}{-0.023570in}}%
\pgfpathcurveto{\pgfqpoint{-0.017319in}{-0.029821in}}{\pgfqpoint{-0.008840in}{-0.033333in}}{\pgfqpoint{0.000000in}{-0.033333in}}%
\pgfpathlineto{\pgfqpoint{0.000000in}{-0.033333in}}%
\pgfpathclose%
\pgfusepath{stroke,fill}%
}%
\begin{pgfscope}%
\pgfsys@transformshift{2.985288in}{3.119283in}%
\pgfsys@useobject{currentmarker}{}%
\end{pgfscope}%
\end{pgfscope}%
\begin{pgfscope}%
\pgfpathrectangle{\pgfqpoint{0.765000in}{0.660000in}}{\pgfqpoint{4.620000in}{4.620000in}}%
\pgfusepath{clip}%
\pgfsetroundcap%
\pgfsetroundjoin%
\pgfsetlinewidth{1.204500pt}%
\definecolor{currentstroke}{rgb}{1.000000,0.576471,0.309804}%
\pgfsetstrokecolor{currentstroke}%
\pgfsetdash{}{0pt}%
\pgfpathmoveto{\pgfqpoint{3.790694in}{2.616276in}}%
\pgfusepath{stroke}%
\end{pgfscope}%
\begin{pgfscope}%
\pgfpathrectangle{\pgfqpoint{0.765000in}{0.660000in}}{\pgfqpoint{4.620000in}{4.620000in}}%
\pgfusepath{clip}%
\pgfsetbuttcap%
\pgfsetroundjoin%
\definecolor{currentfill}{rgb}{1.000000,0.576471,0.309804}%
\pgfsetfillcolor{currentfill}%
\pgfsetlinewidth{1.003750pt}%
\definecolor{currentstroke}{rgb}{1.000000,0.576471,0.309804}%
\pgfsetstrokecolor{currentstroke}%
\pgfsetdash{}{0pt}%
\pgfsys@defobject{currentmarker}{\pgfqpoint{-0.033333in}{-0.033333in}}{\pgfqpoint{0.033333in}{0.033333in}}{%
\pgfpathmoveto{\pgfqpoint{0.000000in}{-0.033333in}}%
\pgfpathcurveto{\pgfqpoint{0.008840in}{-0.033333in}}{\pgfqpoint{0.017319in}{-0.029821in}}{\pgfqpoint{0.023570in}{-0.023570in}}%
\pgfpathcurveto{\pgfqpoint{0.029821in}{-0.017319in}}{\pgfqpoint{0.033333in}{-0.008840in}}{\pgfqpoint{0.033333in}{0.000000in}}%
\pgfpathcurveto{\pgfqpoint{0.033333in}{0.008840in}}{\pgfqpoint{0.029821in}{0.017319in}}{\pgfqpoint{0.023570in}{0.023570in}}%
\pgfpathcurveto{\pgfqpoint{0.017319in}{0.029821in}}{\pgfqpoint{0.008840in}{0.033333in}}{\pgfqpoint{0.000000in}{0.033333in}}%
\pgfpathcurveto{\pgfqpoint{-0.008840in}{0.033333in}}{\pgfqpoint{-0.017319in}{0.029821in}}{\pgfqpoint{-0.023570in}{0.023570in}}%
\pgfpathcurveto{\pgfqpoint{-0.029821in}{0.017319in}}{\pgfqpoint{-0.033333in}{0.008840in}}{\pgfqpoint{-0.033333in}{0.000000in}}%
\pgfpathcurveto{\pgfqpoint{-0.033333in}{-0.008840in}}{\pgfqpoint{-0.029821in}{-0.017319in}}{\pgfqpoint{-0.023570in}{-0.023570in}}%
\pgfpathcurveto{\pgfqpoint{-0.017319in}{-0.029821in}}{\pgfqpoint{-0.008840in}{-0.033333in}}{\pgfqpoint{0.000000in}{-0.033333in}}%
\pgfpathlineto{\pgfqpoint{0.000000in}{-0.033333in}}%
\pgfpathclose%
\pgfusepath{stroke,fill}%
}%
\begin{pgfscope}%
\pgfsys@transformshift{3.790694in}{2.616276in}%
\pgfsys@useobject{currentmarker}{}%
\end{pgfscope}%
\end{pgfscope}%
\begin{pgfscope}%
\pgfpathrectangle{\pgfqpoint{0.765000in}{0.660000in}}{\pgfqpoint{4.620000in}{4.620000in}}%
\pgfusepath{clip}%
\pgfsetroundcap%
\pgfsetroundjoin%
\pgfsetlinewidth{1.204500pt}%
\definecolor{currentstroke}{rgb}{1.000000,0.576471,0.309804}%
\pgfsetstrokecolor{currentstroke}%
\pgfsetdash{}{0pt}%
\pgfpathmoveto{\pgfqpoint{2.584312in}{2.592303in}}%
\pgfusepath{stroke}%
\end{pgfscope}%
\begin{pgfscope}%
\pgfpathrectangle{\pgfqpoint{0.765000in}{0.660000in}}{\pgfqpoint{4.620000in}{4.620000in}}%
\pgfusepath{clip}%
\pgfsetbuttcap%
\pgfsetroundjoin%
\definecolor{currentfill}{rgb}{1.000000,0.576471,0.309804}%
\pgfsetfillcolor{currentfill}%
\pgfsetlinewidth{1.003750pt}%
\definecolor{currentstroke}{rgb}{1.000000,0.576471,0.309804}%
\pgfsetstrokecolor{currentstroke}%
\pgfsetdash{}{0pt}%
\pgfsys@defobject{currentmarker}{\pgfqpoint{-0.033333in}{-0.033333in}}{\pgfqpoint{0.033333in}{0.033333in}}{%
\pgfpathmoveto{\pgfqpoint{0.000000in}{-0.033333in}}%
\pgfpathcurveto{\pgfqpoint{0.008840in}{-0.033333in}}{\pgfqpoint{0.017319in}{-0.029821in}}{\pgfqpoint{0.023570in}{-0.023570in}}%
\pgfpathcurveto{\pgfqpoint{0.029821in}{-0.017319in}}{\pgfqpoint{0.033333in}{-0.008840in}}{\pgfqpoint{0.033333in}{0.000000in}}%
\pgfpathcurveto{\pgfqpoint{0.033333in}{0.008840in}}{\pgfqpoint{0.029821in}{0.017319in}}{\pgfqpoint{0.023570in}{0.023570in}}%
\pgfpathcurveto{\pgfqpoint{0.017319in}{0.029821in}}{\pgfqpoint{0.008840in}{0.033333in}}{\pgfqpoint{0.000000in}{0.033333in}}%
\pgfpathcurveto{\pgfqpoint{-0.008840in}{0.033333in}}{\pgfqpoint{-0.017319in}{0.029821in}}{\pgfqpoint{-0.023570in}{0.023570in}}%
\pgfpathcurveto{\pgfqpoint{-0.029821in}{0.017319in}}{\pgfqpoint{-0.033333in}{0.008840in}}{\pgfqpoint{-0.033333in}{0.000000in}}%
\pgfpathcurveto{\pgfqpoint{-0.033333in}{-0.008840in}}{\pgfqpoint{-0.029821in}{-0.017319in}}{\pgfqpoint{-0.023570in}{-0.023570in}}%
\pgfpathcurveto{\pgfqpoint{-0.017319in}{-0.029821in}}{\pgfqpoint{-0.008840in}{-0.033333in}}{\pgfqpoint{0.000000in}{-0.033333in}}%
\pgfpathlineto{\pgfqpoint{0.000000in}{-0.033333in}}%
\pgfpathclose%
\pgfusepath{stroke,fill}%
}%
\begin{pgfscope}%
\pgfsys@transformshift{2.584312in}{2.592303in}%
\pgfsys@useobject{currentmarker}{}%
\end{pgfscope}%
\end{pgfscope}%
\begin{pgfscope}%
\pgfpathrectangle{\pgfqpoint{0.765000in}{0.660000in}}{\pgfqpoint{4.620000in}{4.620000in}}%
\pgfusepath{clip}%
\pgfsetroundcap%
\pgfsetroundjoin%
\pgfsetlinewidth{1.204500pt}%
\definecolor{currentstroke}{rgb}{1.000000,0.576471,0.309804}%
\pgfsetstrokecolor{currentstroke}%
\pgfsetdash{}{0pt}%
\pgfpathmoveto{\pgfqpoint{2.674362in}{2.926085in}}%
\pgfusepath{stroke}%
\end{pgfscope}%
\begin{pgfscope}%
\pgfpathrectangle{\pgfqpoint{0.765000in}{0.660000in}}{\pgfqpoint{4.620000in}{4.620000in}}%
\pgfusepath{clip}%
\pgfsetbuttcap%
\pgfsetroundjoin%
\definecolor{currentfill}{rgb}{1.000000,0.576471,0.309804}%
\pgfsetfillcolor{currentfill}%
\pgfsetlinewidth{1.003750pt}%
\definecolor{currentstroke}{rgb}{1.000000,0.576471,0.309804}%
\pgfsetstrokecolor{currentstroke}%
\pgfsetdash{}{0pt}%
\pgfsys@defobject{currentmarker}{\pgfqpoint{-0.033333in}{-0.033333in}}{\pgfqpoint{0.033333in}{0.033333in}}{%
\pgfpathmoveto{\pgfqpoint{0.000000in}{-0.033333in}}%
\pgfpathcurveto{\pgfqpoint{0.008840in}{-0.033333in}}{\pgfqpoint{0.017319in}{-0.029821in}}{\pgfqpoint{0.023570in}{-0.023570in}}%
\pgfpathcurveto{\pgfqpoint{0.029821in}{-0.017319in}}{\pgfqpoint{0.033333in}{-0.008840in}}{\pgfqpoint{0.033333in}{0.000000in}}%
\pgfpathcurveto{\pgfqpoint{0.033333in}{0.008840in}}{\pgfqpoint{0.029821in}{0.017319in}}{\pgfqpoint{0.023570in}{0.023570in}}%
\pgfpathcurveto{\pgfqpoint{0.017319in}{0.029821in}}{\pgfqpoint{0.008840in}{0.033333in}}{\pgfqpoint{0.000000in}{0.033333in}}%
\pgfpathcurveto{\pgfqpoint{-0.008840in}{0.033333in}}{\pgfqpoint{-0.017319in}{0.029821in}}{\pgfqpoint{-0.023570in}{0.023570in}}%
\pgfpathcurveto{\pgfqpoint{-0.029821in}{0.017319in}}{\pgfqpoint{-0.033333in}{0.008840in}}{\pgfqpoint{-0.033333in}{0.000000in}}%
\pgfpathcurveto{\pgfqpoint{-0.033333in}{-0.008840in}}{\pgfqpoint{-0.029821in}{-0.017319in}}{\pgfqpoint{-0.023570in}{-0.023570in}}%
\pgfpathcurveto{\pgfqpoint{-0.017319in}{-0.029821in}}{\pgfqpoint{-0.008840in}{-0.033333in}}{\pgfqpoint{0.000000in}{-0.033333in}}%
\pgfpathlineto{\pgfqpoint{0.000000in}{-0.033333in}}%
\pgfpathclose%
\pgfusepath{stroke,fill}%
}%
\begin{pgfscope}%
\pgfsys@transformshift{2.674362in}{2.926085in}%
\pgfsys@useobject{currentmarker}{}%
\end{pgfscope}%
\end{pgfscope}%
\begin{pgfscope}%
\pgfpathrectangle{\pgfqpoint{0.765000in}{0.660000in}}{\pgfqpoint{4.620000in}{4.620000in}}%
\pgfusepath{clip}%
\pgfsetroundcap%
\pgfsetroundjoin%
\pgfsetlinewidth{1.204500pt}%
\definecolor{currentstroke}{rgb}{1.000000,0.576471,0.309804}%
\pgfsetstrokecolor{currentstroke}%
\pgfsetdash{}{0pt}%
\pgfpathmoveto{\pgfqpoint{2.403020in}{2.896452in}}%
\pgfusepath{stroke}%
\end{pgfscope}%
\begin{pgfscope}%
\pgfpathrectangle{\pgfqpoint{0.765000in}{0.660000in}}{\pgfqpoint{4.620000in}{4.620000in}}%
\pgfusepath{clip}%
\pgfsetbuttcap%
\pgfsetroundjoin%
\definecolor{currentfill}{rgb}{1.000000,0.576471,0.309804}%
\pgfsetfillcolor{currentfill}%
\pgfsetlinewidth{1.003750pt}%
\definecolor{currentstroke}{rgb}{1.000000,0.576471,0.309804}%
\pgfsetstrokecolor{currentstroke}%
\pgfsetdash{}{0pt}%
\pgfsys@defobject{currentmarker}{\pgfqpoint{-0.033333in}{-0.033333in}}{\pgfqpoint{0.033333in}{0.033333in}}{%
\pgfpathmoveto{\pgfqpoint{0.000000in}{-0.033333in}}%
\pgfpathcurveto{\pgfqpoint{0.008840in}{-0.033333in}}{\pgfqpoint{0.017319in}{-0.029821in}}{\pgfqpoint{0.023570in}{-0.023570in}}%
\pgfpathcurveto{\pgfqpoint{0.029821in}{-0.017319in}}{\pgfqpoint{0.033333in}{-0.008840in}}{\pgfqpoint{0.033333in}{0.000000in}}%
\pgfpathcurveto{\pgfqpoint{0.033333in}{0.008840in}}{\pgfqpoint{0.029821in}{0.017319in}}{\pgfqpoint{0.023570in}{0.023570in}}%
\pgfpathcurveto{\pgfqpoint{0.017319in}{0.029821in}}{\pgfqpoint{0.008840in}{0.033333in}}{\pgfqpoint{0.000000in}{0.033333in}}%
\pgfpathcurveto{\pgfqpoint{-0.008840in}{0.033333in}}{\pgfqpoint{-0.017319in}{0.029821in}}{\pgfqpoint{-0.023570in}{0.023570in}}%
\pgfpathcurveto{\pgfqpoint{-0.029821in}{0.017319in}}{\pgfqpoint{-0.033333in}{0.008840in}}{\pgfqpoint{-0.033333in}{0.000000in}}%
\pgfpathcurveto{\pgfqpoint{-0.033333in}{-0.008840in}}{\pgfqpoint{-0.029821in}{-0.017319in}}{\pgfqpoint{-0.023570in}{-0.023570in}}%
\pgfpathcurveto{\pgfqpoint{-0.017319in}{-0.029821in}}{\pgfqpoint{-0.008840in}{-0.033333in}}{\pgfqpoint{0.000000in}{-0.033333in}}%
\pgfpathlineto{\pgfqpoint{0.000000in}{-0.033333in}}%
\pgfpathclose%
\pgfusepath{stroke,fill}%
}%
\begin{pgfscope}%
\pgfsys@transformshift{2.403020in}{2.896452in}%
\pgfsys@useobject{currentmarker}{}%
\end{pgfscope}%
\end{pgfscope}%
\begin{pgfscope}%
\pgfpathrectangle{\pgfqpoint{0.765000in}{0.660000in}}{\pgfqpoint{4.620000in}{4.620000in}}%
\pgfusepath{clip}%
\pgfsetroundcap%
\pgfsetroundjoin%
\pgfsetlinewidth{1.204500pt}%
\definecolor{currentstroke}{rgb}{1.000000,0.576471,0.309804}%
\pgfsetstrokecolor{currentstroke}%
\pgfsetdash{}{0pt}%
\pgfpathmoveto{\pgfqpoint{3.474543in}{2.284894in}}%
\pgfusepath{stroke}%
\end{pgfscope}%
\begin{pgfscope}%
\pgfpathrectangle{\pgfqpoint{0.765000in}{0.660000in}}{\pgfqpoint{4.620000in}{4.620000in}}%
\pgfusepath{clip}%
\pgfsetbuttcap%
\pgfsetroundjoin%
\definecolor{currentfill}{rgb}{1.000000,0.576471,0.309804}%
\pgfsetfillcolor{currentfill}%
\pgfsetlinewidth{1.003750pt}%
\definecolor{currentstroke}{rgb}{1.000000,0.576471,0.309804}%
\pgfsetstrokecolor{currentstroke}%
\pgfsetdash{}{0pt}%
\pgfsys@defobject{currentmarker}{\pgfqpoint{-0.033333in}{-0.033333in}}{\pgfqpoint{0.033333in}{0.033333in}}{%
\pgfpathmoveto{\pgfqpoint{0.000000in}{-0.033333in}}%
\pgfpathcurveto{\pgfqpoint{0.008840in}{-0.033333in}}{\pgfqpoint{0.017319in}{-0.029821in}}{\pgfqpoint{0.023570in}{-0.023570in}}%
\pgfpathcurveto{\pgfqpoint{0.029821in}{-0.017319in}}{\pgfqpoint{0.033333in}{-0.008840in}}{\pgfqpoint{0.033333in}{0.000000in}}%
\pgfpathcurveto{\pgfqpoint{0.033333in}{0.008840in}}{\pgfqpoint{0.029821in}{0.017319in}}{\pgfqpoint{0.023570in}{0.023570in}}%
\pgfpathcurveto{\pgfqpoint{0.017319in}{0.029821in}}{\pgfqpoint{0.008840in}{0.033333in}}{\pgfqpoint{0.000000in}{0.033333in}}%
\pgfpathcurveto{\pgfqpoint{-0.008840in}{0.033333in}}{\pgfqpoint{-0.017319in}{0.029821in}}{\pgfqpoint{-0.023570in}{0.023570in}}%
\pgfpathcurveto{\pgfqpoint{-0.029821in}{0.017319in}}{\pgfqpoint{-0.033333in}{0.008840in}}{\pgfqpoint{-0.033333in}{0.000000in}}%
\pgfpathcurveto{\pgfqpoint{-0.033333in}{-0.008840in}}{\pgfqpoint{-0.029821in}{-0.017319in}}{\pgfqpoint{-0.023570in}{-0.023570in}}%
\pgfpathcurveto{\pgfqpoint{-0.017319in}{-0.029821in}}{\pgfqpoint{-0.008840in}{-0.033333in}}{\pgfqpoint{0.000000in}{-0.033333in}}%
\pgfpathlineto{\pgfqpoint{0.000000in}{-0.033333in}}%
\pgfpathclose%
\pgfusepath{stroke,fill}%
}%
\begin{pgfscope}%
\pgfsys@transformshift{3.474543in}{2.284894in}%
\pgfsys@useobject{currentmarker}{}%
\end{pgfscope}%
\end{pgfscope}%
\begin{pgfscope}%
\pgfpathrectangle{\pgfqpoint{0.765000in}{0.660000in}}{\pgfqpoint{4.620000in}{4.620000in}}%
\pgfusepath{clip}%
\pgfsetroundcap%
\pgfsetroundjoin%
\pgfsetlinewidth{1.204500pt}%
\definecolor{currentstroke}{rgb}{1.000000,0.576471,0.309804}%
\pgfsetstrokecolor{currentstroke}%
\pgfsetdash{}{0pt}%
\pgfpathmoveto{\pgfqpoint{3.886262in}{3.002183in}}%
\pgfusepath{stroke}%
\end{pgfscope}%
\begin{pgfscope}%
\pgfpathrectangle{\pgfqpoint{0.765000in}{0.660000in}}{\pgfqpoint{4.620000in}{4.620000in}}%
\pgfusepath{clip}%
\pgfsetbuttcap%
\pgfsetroundjoin%
\definecolor{currentfill}{rgb}{1.000000,0.576471,0.309804}%
\pgfsetfillcolor{currentfill}%
\pgfsetlinewidth{1.003750pt}%
\definecolor{currentstroke}{rgb}{1.000000,0.576471,0.309804}%
\pgfsetstrokecolor{currentstroke}%
\pgfsetdash{}{0pt}%
\pgfsys@defobject{currentmarker}{\pgfqpoint{-0.033333in}{-0.033333in}}{\pgfqpoint{0.033333in}{0.033333in}}{%
\pgfpathmoveto{\pgfqpoint{0.000000in}{-0.033333in}}%
\pgfpathcurveto{\pgfqpoint{0.008840in}{-0.033333in}}{\pgfqpoint{0.017319in}{-0.029821in}}{\pgfqpoint{0.023570in}{-0.023570in}}%
\pgfpathcurveto{\pgfqpoint{0.029821in}{-0.017319in}}{\pgfqpoint{0.033333in}{-0.008840in}}{\pgfqpoint{0.033333in}{0.000000in}}%
\pgfpathcurveto{\pgfqpoint{0.033333in}{0.008840in}}{\pgfqpoint{0.029821in}{0.017319in}}{\pgfqpoint{0.023570in}{0.023570in}}%
\pgfpathcurveto{\pgfqpoint{0.017319in}{0.029821in}}{\pgfqpoint{0.008840in}{0.033333in}}{\pgfqpoint{0.000000in}{0.033333in}}%
\pgfpathcurveto{\pgfqpoint{-0.008840in}{0.033333in}}{\pgfqpoint{-0.017319in}{0.029821in}}{\pgfqpoint{-0.023570in}{0.023570in}}%
\pgfpathcurveto{\pgfqpoint{-0.029821in}{0.017319in}}{\pgfqpoint{-0.033333in}{0.008840in}}{\pgfqpoint{-0.033333in}{0.000000in}}%
\pgfpathcurveto{\pgfqpoint{-0.033333in}{-0.008840in}}{\pgfqpoint{-0.029821in}{-0.017319in}}{\pgfqpoint{-0.023570in}{-0.023570in}}%
\pgfpathcurveto{\pgfqpoint{-0.017319in}{-0.029821in}}{\pgfqpoint{-0.008840in}{-0.033333in}}{\pgfqpoint{0.000000in}{-0.033333in}}%
\pgfpathlineto{\pgfqpoint{0.000000in}{-0.033333in}}%
\pgfpathclose%
\pgfusepath{stroke,fill}%
}%
\begin{pgfscope}%
\pgfsys@transformshift{3.886262in}{3.002183in}%
\pgfsys@useobject{currentmarker}{}%
\end{pgfscope}%
\end{pgfscope}%
\begin{pgfscope}%
\pgfpathrectangle{\pgfqpoint{0.765000in}{0.660000in}}{\pgfqpoint{4.620000in}{4.620000in}}%
\pgfusepath{clip}%
\pgfsetroundcap%
\pgfsetroundjoin%
\pgfsetlinewidth{1.204500pt}%
\definecolor{currentstroke}{rgb}{1.000000,0.576471,0.309804}%
\pgfsetstrokecolor{currentstroke}%
\pgfsetdash{}{0pt}%
\pgfpathmoveto{\pgfqpoint{2.877793in}{2.108234in}}%
\pgfusepath{stroke}%
\end{pgfscope}%
\begin{pgfscope}%
\pgfpathrectangle{\pgfqpoint{0.765000in}{0.660000in}}{\pgfqpoint{4.620000in}{4.620000in}}%
\pgfusepath{clip}%
\pgfsetbuttcap%
\pgfsetroundjoin%
\definecolor{currentfill}{rgb}{1.000000,0.576471,0.309804}%
\pgfsetfillcolor{currentfill}%
\pgfsetlinewidth{1.003750pt}%
\definecolor{currentstroke}{rgb}{1.000000,0.576471,0.309804}%
\pgfsetstrokecolor{currentstroke}%
\pgfsetdash{}{0pt}%
\pgfsys@defobject{currentmarker}{\pgfqpoint{-0.033333in}{-0.033333in}}{\pgfqpoint{0.033333in}{0.033333in}}{%
\pgfpathmoveto{\pgfqpoint{0.000000in}{-0.033333in}}%
\pgfpathcurveto{\pgfqpoint{0.008840in}{-0.033333in}}{\pgfqpoint{0.017319in}{-0.029821in}}{\pgfqpoint{0.023570in}{-0.023570in}}%
\pgfpathcurveto{\pgfqpoint{0.029821in}{-0.017319in}}{\pgfqpoint{0.033333in}{-0.008840in}}{\pgfqpoint{0.033333in}{0.000000in}}%
\pgfpathcurveto{\pgfqpoint{0.033333in}{0.008840in}}{\pgfqpoint{0.029821in}{0.017319in}}{\pgfqpoint{0.023570in}{0.023570in}}%
\pgfpathcurveto{\pgfqpoint{0.017319in}{0.029821in}}{\pgfqpoint{0.008840in}{0.033333in}}{\pgfqpoint{0.000000in}{0.033333in}}%
\pgfpathcurveto{\pgfqpoint{-0.008840in}{0.033333in}}{\pgfqpoint{-0.017319in}{0.029821in}}{\pgfqpoint{-0.023570in}{0.023570in}}%
\pgfpathcurveto{\pgfqpoint{-0.029821in}{0.017319in}}{\pgfqpoint{-0.033333in}{0.008840in}}{\pgfqpoint{-0.033333in}{0.000000in}}%
\pgfpathcurveto{\pgfqpoint{-0.033333in}{-0.008840in}}{\pgfqpoint{-0.029821in}{-0.017319in}}{\pgfqpoint{-0.023570in}{-0.023570in}}%
\pgfpathcurveto{\pgfqpoint{-0.017319in}{-0.029821in}}{\pgfqpoint{-0.008840in}{-0.033333in}}{\pgfqpoint{0.000000in}{-0.033333in}}%
\pgfpathlineto{\pgfqpoint{0.000000in}{-0.033333in}}%
\pgfpathclose%
\pgfusepath{stroke,fill}%
}%
\begin{pgfscope}%
\pgfsys@transformshift{2.877793in}{2.108234in}%
\pgfsys@useobject{currentmarker}{}%
\end{pgfscope}%
\end{pgfscope}%
\begin{pgfscope}%
\pgfpathrectangle{\pgfqpoint{0.765000in}{0.660000in}}{\pgfqpoint{4.620000in}{4.620000in}}%
\pgfusepath{clip}%
\pgfsetroundcap%
\pgfsetroundjoin%
\pgfsetlinewidth{1.204500pt}%
\definecolor{currentstroke}{rgb}{1.000000,0.576471,0.309804}%
\pgfsetstrokecolor{currentstroke}%
\pgfsetdash{}{0pt}%
\pgfpathmoveto{\pgfqpoint{2.803182in}{3.074974in}}%
\pgfusepath{stroke}%
\end{pgfscope}%
\begin{pgfscope}%
\pgfpathrectangle{\pgfqpoint{0.765000in}{0.660000in}}{\pgfqpoint{4.620000in}{4.620000in}}%
\pgfusepath{clip}%
\pgfsetbuttcap%
\pgfsetroundjoin%
\definecolor{currentfill}{rgb}{1.000000,0.576471,0.309804}%
\pgfsetfillcolor{currentfill}%
\pgfsetlinewidth{1.003750pt}%
\definecolor{currentstroke}{rgb}{1.000000,0.576471,0.309804}%
\pgfsetstrokecolor{currentstroke}%
\pgfsetdash{}{0pt}%
\pgfsys@defobject{currentmarker}{\pgfqpoint{-0.033333in}{-0.033333in}}{\pgfqpoint{0.033333in}{0.033333in}}{%
\pgfpathmoveto{\pgfqpoint{0.000000in}{-0.033333in}}%
\pgfpathcurveto{\pgfqpoint{0.008840in}{-0.033333in}}{\pgfqpoint{0.017319in}{-0.029821in}}{\pgfqpoint{0.023570in}{-0.023570in}}%
\pgfpathcurveto{\pgfqpoint{0.029821in}{-0.017319in}}{\pgfqpoint{0.033333in}{-0.008840in}}{\pgfqpoint{0.033333in}{0.000000in}}%
\pgfpathcurveto{\pgfqpoint{0.033333in}{0.008840in}}{\pgfqpoint{0.029821in}{0.017319in}}{\pgfqpoint{0.023570in}{0.023570in}}%
\pgfpathcurveto{\pgfqpoint{0.017319in}{0.029821in}}{\pgfqpoint{0.008840in}{0.033333in}}{\pgfqpoint{0.000000in}{0.033333in}}%
\pgfpathcurveto{\pgfqpoint{-0.008840in}{0.033333in}}{\pgfqpoint{-0.017319in}{0.029821in}}{\pgfqpoint{-0.023570in}{0.023570in}}%
\pgfpathcurveto{\pgfqpoint{-0.029821in}{0.017319in}}{\pgfqpoint{-0.033333in}{0.008840in}}{\pgfqpoint{-0.033333in}{0.000000in}}%
\pgfpathcurveto{\pgfqpoint{-0.033333in}{-0.008840in}}{\pgfqpoint{-0.029821in}{-0.017319in}}{\pgfqpoint{-0.023570in}{-0.023570in}}%
\pgfpathcurveto{\pgfqpoint{-0.017319in}{-0.029821in}}{\pgfqpoint{-0.008840in}{-0.033333in}}{\pgfqpoint{0.000000in}{-0.033333in}}%
\pgfpathlineto{\pgfqpoint{0.000000in}{-0.033333in}}%
\pgfpathclose%
\pgfusepath{stroke,fill}%
}%
\begin{pgfscope}%
\pgfsys@transformshift{2.803182in}{3.074974in}%
\pgfsys@useobject{currentmarker}{}%
\end{pgfscope}%
\end{pgfscope}%
\begin{pgfscope}%
\pgfpathrectangle{\pgfqpoint{0.765000in}{0.660000in}}{\pgfqpoint{4.620000in}{4.620000in}}%
\pgfusepath{clip}%
\pgfsetroundcap%
\pgfsetroundjoin%
\pgfsetlinewidth{1.204500pt}%
\definecolor{currentstroke}{rgb}{1.000000,0.576471,0.309804}%
\pgfsetstrokecolor{currentstroke}%
\pgfsetdash{}{0pt}%
\pgfpathmoveto{\pgfqpoint{2.694662in}{3.030419in}}%
\pgfusepath{stroke}%
\end{pgfscope}%
\begin{pgfscope}%
\pgfpathrectangle{\pgfqpoint{0.765000in}{0.660000in}}{\pgfqpoint{4.620000in}{4.620000in}}%
\pgfusepath{clip}%
\pgfsetbuttcap%
\pgfsetroundjoin%
\definecolor{currentfill}{rgb}{1.000000,0.576471,0.309804}%
\pgfsetfillcolor{currentfill}%
\pgfsetlinewidth{1.003750pt}%
\definecolor{currentstroke}{rgb}{1.000000,0.576471,0.309804}%
\pgfsetstrokecolor{currentstroke}%
\pgfsetdash{}{0pt}%
\pgfsys@defobject{currentmarker}{\pgfqpoint{-0.033333in}{-0.033333in}}{\pgfqpoint{0.033333in}{0.033333in}}{%
\pgfpathmoveto{\pgfqpoint{0.000000in}{-0.033333in}}%
\pgfpathcurveto{\pgfqpoint{0.008840in}{-0.033333in}}{\pgfqpoint{0.017319in}{-0.029821in}}{\pgfqpoint{0.023570in}{-0.023570in}}%
\pgfpathcurveto{\pgfqpoint{0.029821in}{-0.017319in}}{\pgfqpoint{0.033333in}{-0.008840in}}{\pgfqpoint{0.033333in}{0.000000in}}%
\pgfpathcurveto{\pgfqpoint{0.033333in}{0.008840in}}{\pgfqpoint{0.029821in}{0.017319in}}{\pgfqpoint{0.023570in}{0.023570in}}%
\pgfpathcurveto{\pgfqpoint{0.017319in}{0.029821in}}{\pgfqpoint{0.008840in}{0.033333in}}{\pgfqpoint{0.000000in}{0.033333in}}%
\pgfpathcurveto{\pgfqpoint{-0.008840in}{0.033333in}}{\pgfqpoint{-0.017319in}{0.029821in}}{\pgfqpoint{-0.023570in}{0.023570in}}%
\pgfpathcurveto{\pgfqpoint{-0.029821in}{0.017319in}}{\pgfqpoint{-0.033333in}{0.008840in}}{\pgfqpoint{-0.033333in}{0.000000in}}%
\pgfpathcurveto{\pgfqpoint{-0.033333in}{-0.008840in}}{\pgfqpoint{-0.029821in}{-0.017319in}}{\pgfqpoint{-0.023570in}{-0.023570in}}%
\pgfpathcurveto{\pgfqpoint{-0.017319in}{-0.029821in}}{\pgfqpoint{-0.008840in}{-0.033333in}}{\pgfqpoint{0.000000in}{-0.033333in}}%
\pgfpathlineto{\pgfqpoint{0.000000in}{-0.033333in}}%
\pgfpathclose%
\pgfusepath{stroke,fill}%
}%
\begin{pgfscope}%
\pgfsys@transformshift{2.694662in}{3.030419in}%
\pgfsys@useobject{currentmarker}{}%
\end{pgfscope}%
\end{pgfscope}%
\begin{pgfscope}%
\pgfpathrectangle{\pgfqpoint{0.765000in}{0.660000in}}{\pgfqpoint{4.620000in}{4.620000in}}%
\pgfusepath{clip}%
\pgfsetroundcap%
\pgfsetroundjoin%
\pgfsetlinewidth{1.204500pt}%
\definecolor{currentstroke}{rgb}{1.000000,0.576471,0.309804}%
\pgfsetstrokecolor{currentstroke}%
\pgfsetdash{}{0pt}%
\pgfpathmoveto{\pgfqpoint{2.776069in}{3.418843in}}%
\pgfusepath{stroke}%
\end{pgfscope}%
\begin{pgfscope}%
\pgfpathrectangle{\pgfqpoint{0.765000in}{0.660000in}}{\pgfqpoint{4.620000in}{4.620000in}}%
\pgfusepath{clip}%
\pgfsetbuttcap%
\pgfsetroundjoin%
\definecolor{currentfill}{rgb}{1.000000,0.576471,0.309804}%
\pgfsetfillcolor{currentfill}%
\pgfsetlinewidth{1.003750pt}%
\definecolor{currentstroke}{rgb}{1.000000,0.576471,0.309804}%
\pgfsetstrokecolor{currentstroke}%
\pgfsetdash{}{0pt}%
\pgfsys@defobject{currentmarker}{\pgfqpoint{-0.033333in}{-0.033333in}}{\pgfqpoint{0.033333in}{0.033333in}}{%
\pgfpathmoveto{\pgfqpoint{0.000000in}{-0.033333in}}%
\pgfpathcurveto{\pgfqpoint{0.008840in}{-0.033333in}}{\pgfqpoint{0.017319in}{-0.029821in}}{\pgfqpoint{0.023570in}{-0.023570in}}%
\pgfpathcurveto{\pgfqpoint{0.029821in}{-0.017319in}}{\pgfqpoint{0.033333in}{-0.008840in}}{\pgfqpoint{0.033333in}{0.000000in}}%
\pgfpathcurveto{\pgfqpoint{0.033333in}{0.008840in}}{\pgfqpoint{0.029821in}{0.017319in}}{\pgfqpoint{0.023570in}{0.023570in}}%
\pgfpathcurveto{\pgfqpoint{0.017319in}{0.029821in}}{\pgfqpoint{0.008840in}{0.033333in}}{\pgfqpoint{0.000000in}{0.033333in}}%
\pgfpathcurveto{\pgfqpoint{-0.008840in}{0.033333in}}{\pgfqpoint{-0.017319in}{0.029821in}}{\pgfqpoint{-0.023570in}{0.023570in}}%
\pgfpathcurveto{\pgfqpoint{-0.029821in}{0.017319in}}{\pgfqpoint{-0.033333in}{0.008840in}}{\pgfqpoint{-0.033333in}{0.000000in}}%
\pgfpathcurveto{\pgfqpoint{-0.033333in}{-0.008840in}}{\pgfqpoint{-0.029821in}{-0.017319in}}{\pgfqpoint{-0.023570in}{-0.023570in}}%
\pgfpathcurveto{\pgfqpoint{-0.017319in}{-0.029821in}}{\pgfqpoint{-0.008840in}{-0.033333in}}{\pgfqpoint{0.000000in}{-0.033333in}}%
\pgfpathlineto{\pgfqpoint{0.000000in}{-0.033333in}}%
\pgfpathclose%
\pgfusepath{stroke,fill}%
}%
\begin{pgfscope}%
\pgfsys@transformshift{2.776069in}{3.418843in}%
\pgfsys@useobject{currentmarker}{}%
\end{pgfscope}%
\end{pgfscope}%
\begin{pgfscope}%
\pgfpathrectangle{\pgfqpoint{0.765000in}{0.660000in}}{\pgfqpoint{4.620000in}{4.620000in}}%
\pgfusepath{clip}%
\pgfsetroundcap%
\pgfsetroundjoin%
\pgfsetlinewidth{1.204500pt}%
\definecolor{currentstroke}{rgb}{1.000000,0.576471,0.309804}%
\pgfsetstrokecolor{currentstroke}%
\pgfsetdash{}{0pt}%
\pgfpathmoveto{\pgfqpoint{2.849456in}{2.519756in}}%
\pgfusepath{stroke}%
\end{pgfscope}%
\begin{pgfscope}%
\pgfpathrectangle{\pgfqpoint{0.765000in}{0.660000in}}{\pgfqpoint{4.620000in}{4.620000in}}%
\pgfusepath{clip}%
\pgfsetbuttcap%
\pgfsetroundjoin%
\definecolor{currentfill}{rgb}{1.000000,0.576471,0.309804}%
\pgfsetfillcolor{currentfill}%
\pgfsetlinewidth{1.003750pt}%
\definecolor{currentstroke}{rgb}{1.000000,0.576471,0.309804}%
\pgfsetstrokecolor{currentstroke}%
\pgfsetdash{}{0pt}%
\pgfsys@defobject{currentmarker}{\pgfqpoint{-0.033333in}{-0.033333in}}{\pgfqpoint{0.033333in}{0.033333in}}{%
\pgfpathmoveto{\pgfqpoint{0.000000in}{-0.033333in}}%
\pgfpathcurveto{\pgfqpoint{0.008840in}{-0.033333in}}{\pgfqpoint{0.017319in}{-0.029821in}}{\pgfqpoint{0.023570in}{-0.023570in}}%
\pgfpathcurveto{\pgfqpoint{0.029821in}{-0.017319in}}{\pgfqpoint{0.033333in}{-0.008840in}}{\pgfqpoint{0.033333in}{0.000000in}}%
\pgfpathcurveto{\pgfqpoint{0.033333in}{0.008840in}}{\pgfqpoint{0.029821in}{0.017319in}}{\pgfqpoint{0.023570in}{0.023570in}}%
\pgfpathcurveto{\pgfqpoint{0.017319in}{0.029821in}}{\pgfqpoint{0.008840in}{0.033333in}}{\pgfqpoint{0.000000in}{0.033333in}}%
\pgfpathcurveto{\pgfqpoint{-0.008840in}{0.033333in}}{\pgfqpoint{-0.017319in}{0.029821in}}{\pgfqpoint{-0.023570in}{0.023570in}}%
\pgfpathcurveto{\pgfqpoint{-0.029821in}{0.017319in}}{\pgfqpoint{-0.033333in}{0.008840in}}{\pgfqpoint{-0.033333in}{0.000000in}}%
\pgfpathcurveto{\pgfqpoint{-0.033333in}{-0.008840in}}{\pgfqpoint{-0.029821in}{-0.017319in}}{\pgfqpoint{-0.023570in}{-0.023570in}}%
\pgfpathcurveto{\pgfqpoint{-0.017319in}{-0.029821in}}{\pgfqpoint{-0.008840in}{-0.033333in}}{\pgfqpoint{0.000000in}{-0.033333in}}%
\pgfpathlineto{\pgfqpoint{0.000000in}{-0.033333in}}%
\pgfpathclose%
\pgfusepath{stroke,fill}%
}%
\begin{pgfscope}%
\pgfsys@transformshift{2.849456in}{2.519756in}%
\pgfsys@useobject{currentmarker}{}%
\end{pgfscope}%
\end{pgfscope}%
\begin{pgfscope}%
\pgfpathrectangle{\pgfqpoint{0.765000in}{0.660000in}}{\pgfqpoint{4.620000in}{4.620000in}}%
\pgfusepath{clip}%
\pgfsetroundcap%
\pgfsetroundjoin%
\pgfsetlinewidth{1.204500pt}%
\definecolor{currentstroke}{rgb}{1.000000,0.576471,0.309804}%
\pgfsetstrokecolor{currentstroke}%
\pgfsetdash{}{0pt}%
\pgfpathmoveto{\pgfqpoint{3.644908in}{3.335753in}}%
\pgfusepath{stroke}%
\end{pgfscope}%
\begin{pgfscope}%
\pgfpathrectangle{\pgfqpoint{0.765000in}{0.660000in}}{\pgfqpoint{4.620000in}{4.620000in}}%
\pgfusepath{clip}%
\pgfsetbuttcap%
\pgfsetroundjoin%
\definecolor{currentfill}{rgb}{1.000000,0.576471,0.309804}%
\pgfsetfillcolor{currentfill}%
\pgfsetlinewidth{1.003750pt}%
\definecolor{currentstroke}{rgb}{1.000000,0.576471,0.309804}%
\pgfsetstrokecolor{currentstroke}%
\pgfsetdash{}{0pt}%
\pgfsys@defobject{currentmarker}{\pgfqpoint{-0.033333in}{-0.033333in}}{\pgfqpoint{0.033333in}{0.033333in}}{%
\pgfpathmoveto{\pgfqpoint{0.000000in}{-0.033333in}}%
\pgfpathcurveto{\pgfqpoint{0.008840in}{-0.033333in}}{\pgfqpoint{0.017319in}{-0.029821in}}{\pgfqpoint{0.023570in}{-0.023570in}}%
\pgfpathcurveto{\pgfqpoint{0.029821in}{-0.017319in}}{\pgfqpoint{0.033333in}{-0.008840in}}{\pgfqpoint{0.033333in}{0.000000in}}%
\pgfpathcurveto{\pgfqpoint{0.033333in}{0.008840in}}{\pgfqpoint{0.029821in}{0.017319in}}{\pgfqpoint{0.023570in}{0.023570in}}%
\pgfpathcurveto{\pgfqpoint{0.017319in}{0.029821in}}{\pgfqpoint{0.008840in}{0.033333in}}{\pgfqpoint{0.000000in}{0.033333in}}%
\pgfpathcurveto{\pgfqpoint{-0.008840in}{0.033333in}}{\pgfqpoint{-0.017319in}{0.029821in}}{\pgfqpoint{-0.023570in}{0.023570in}}%
\pgfpathcurveto{\pgfqpoint{-0.029821in}{0.017319in}}{\pgfqpoint{-0.033333in}{0.008840in}}{\pgfqpoint{-0.033333in}{0.000000in}}%
\pgfpathcurveto{\pgfqpoint{-0.033333in}{-0.008840in}}{\pgfqpoint{-0.029821in}{-0.017319in}}{\pgfqpoint{-0.023570in}{-0.023570in}}%
\pgfpathcurveto{\pgfqpoint{-0.017319in}{-0.029821in}}{\pgfqpoint{-0.008840in}{-0.033333in}}{\pgfqpoint{0.000000in}{-0.033333in}}%
\pgfpathlineto{\pgfqpoint{0.000000in}{-0.033333in}}%
\pgfpathclose%
\pgfusepath{stroke,fill}%
}%
\begin{pgfscope}%
\pgfsys@transformshift{3.644908in}{3.335753in}%
\pgfsys@useobject{currentmarker}{}%
\end{pgfscope}%
\end{pgfscope}%
\begin{pgfscope}%
\pgfpathrectangle{\pgfqpoint{0.765000in}{0.660000in}}{\pgfqpoint{4.620000in}{4.620000in}}%
\pgfusepath{clip}%
\pgfsetroundcap%
\pgfsetroundjoin%
\pgfsetlinewidth{1.204500pt}%
\definecolor{currentstroke}{rgb}{1.000000,0.576471,0.309804}%
\pgfsetstrokecolor{currentstroke}%
\pgfsetdash{}{0pt}%
\pgfpathmoveto{\pgfqpoint{2.937273in}{3.061838in}}%
\pgfusepath{stroke}%
\end{pgfscope}%
\begin{pgfscope}%
\pgfpathrectangle{\pgfqpoint{0.765000in}{0.660000in}}{\pgfqpoint{4.620000in}{4.620000in}}%
\pgfusepath{clip}%
\pgfsetbuttcap%
\pgfsetroundjoin%
\definecolor{currentfill}{rgb}{1.000000,0.576471,0.309804}%
\pgfsetfillcolor{currentfill}%
\pgfsetlinewidth{1.003750pt}%
\definecolor{currentstroke}{rgb}{1.000000,0.576471,0.309804}%
\pgfsetstrokecolor{currentstroke}%
\pgfsetdash{}{0pt}%
\pgfsys@defobject{currentmarker}{\pgfqpoint{-0.033333in}{-0.033333in}}{\pgfqpoint{0.033333in}{0.033333in}}{%
\pgfpathmoveto{\pgfqpoint{0.000000in}{-0.033333in}}%
\pgfpathcurveto{\pgfqpoint{0.008840in}{-0.033333in}}{\pgfqpoint{0.017319in}{-0.029821in}}{\pgfqpoint{0.023570in}{-0.023570in}}%
\pgfpathcurveto{\pgfqpoint{0.029821in}{-0.017319in}}{\pgfqpoint{0.033333in}{-0.008840in}}{\pgfqpoint{0.033333in}{0.000000in}}%
\pgfpathcurveto{\pgfqpoint{0.033333in}{0.008840in}}{\pgfqpoint{0.029821in}{0.017319in}}{\pgfqpoint{0.023570in}{0.023570in}}%
\pgfpathcurveto{\pgfqpoint{0.017319in}{0.029821in}}{\pgfqpoint{0.008840in}{0.033333in}}{\pgfqpoint{0.000000in}{0.033333in}}%
\pgfpathcurveto{\pgfqpoint{-0.008840in}{0.033333in}}{\pgfqpoint{-0.017319in}{0.029821in}}{\pgfqpoint{-0.023570in}{0.023570in}}%
\pgfpathcurveto{\pgfqpoint{-0.029821in}{0.017319in}}{\pgfqpoint{-0.033333in}{0.008840in}}{\pgfqpoint{-0.033333in}{0.000000in}}%
\pgfpathcurveto{\pgfqpoint{-0.033333in}{-0.008840in}}{\pgfqpoint{-0.029821in}{-0.017319in}}{\pgfqpoint{-0.023570in}{-0.023570in}}%
\pgfpathcurveto{\pgfqpoint{-0.017319in}{-0.029821in}}{\pgfqpoint{-0.008840in}{-0.033333in}}{\pgfqpoint{0.000000in}{-0.033333in}}%
\pgfpathlineto{\pgfqpoint{0.000000in}{-0.033333in}}%
\pgfpathclose%
\pgfusepath{stroke,fill}%
}%
\begin{pgfscope}%
\pgfsys@transformshift{2.937273in}{3.061838in}%
\pgfsys@useobject{currentmarker}{}%
\end{pgfscope}%
\end{pgfscope}%
\begin{pgfscope}%
\pgfpathrectangle{\pgfqpoint{0.765000in}{0.660000in}}{\pgfqpoint{4.620000in}{4.620000in}}%
\pgfusepath{clip}%
\pgfsetroundcap%
\pgfsetroundjoin%
\pgfsetlinewidth{1.204500pt}%
\definecolor{currentstroke}{rgb}{1.000000,0.576471,0.309804}%
\pgfsetstrokecolor{currentstroke}%
\pgfsetdash{}{0pt}%
\pgfpathmoveto{\pgfqpoint{3.448945in}{3.287912in}}%
\pgfusepath{stroke}%
\end{pgfscope}%
\begin{pgfscope}%
\pgfpathrectangle{\pgfqpoint{0.765000in}{0.660000in}}{\pgfqpoint{4.620000in}{4.620000in}}%
\pgfusepath{clip}%
\pgfsetbuttcap%
\pgfsetroundjoin%
\definecolor{currentfill}{rgb}{1.000000,0.576471,0.309804}%
\pgfsetfillcolor{currentfill}%
\pgfsetlinewidth{1.003750pt}%
\definecolor{currentstroke}{rgb}{1.000000,0.576471,0.309804}%
\pgfsetstrokecolor{currentstroke}%
\pgfsetdash{}{0pt}%
\pgfsys@defobject{currentmarker}{\pgfqpoint{-0.033333in}{-0.033333in}}{\pgfqpoint{0.033333in}{0.033333in}}{%
\pgfpathmoveto{\pgfqpoint{0.000000in}{-0.033333in}}%
\pgfpathcurveto{\pgfqpoint{0.008840in}{-0.033333in}}{\pgfqpoint{0.017319in}{-0.029821in}}{\pgfqpoint{0.023570in}{-0.023570in}}%
\pgfpathcurveto{\pgfqpoint{0.029821in}{-0.017319in}}{\pgfqpoint{0.033333in}{-0.008840in}}{\pgfqpoint{0.033333in}{0.000000in}}%
\pgfpathcurveto{\pgfqpoint{0.033333in}{0.008840in}}{\pgfqpoint{0.029821in}{0.017319in}}{\pgfqpoint{0.023570in}{0.023570in}}%
\pgfpathcurveto{\pgfqpoint{0.017319in}{0.029821in}}{\pgfqpoint{0.008840in}{0.033333in}}{\pgfqpoint{0.000000in}{0.033333in}}%
\pgfpathcurveto{\pgfqpoint{-0.008840in}{0.033333in}}{\pgfqpoint{-0.017319in}{0.029821in}}{\pgfqpoint{-0.023570in}{0.023570in}}%
\pgfpathcurveto{\pgfqpoint{-0.029821in}{0.017319in}}{\pgfqpoint{-0.033333in}{0.008840in}}{\pgfqpoint{-0.033333in}{0.000000in}}%
\pgfpathcurveto{\pgfqpoint{-0.033333in}{-0.008840in}}{\pgfqpoint{-0.029821in}{-0.017319in}}{\pgfqpoint{-0.023570in}{-0.023570in}}%
\pgfpathcurveto{\pgfqpoint{-0.017319in}{-0.029821in}}{\pgfqpoint{-0.008840in}{-0.033333in}}{\pgfqpoint{0.000000in}{-0.033333in}}%
\pgfpathlineto{\pgfqpoint{0.000000in}{-0.033333in}}%
\pgfpathclose%
\pgfusepath{stroke,fill}%
}%
\begin{pgfscope}%
\pgfsys@transformshift{3.448945in}{3.287912in}%
\pgfsys@useobject{currentmarker}{}%
\end{pgfscope}%
\end{pgfscope}%
\begin{pgfscope}%
\pgfpathrectangle{\pgfqpoint{0.765000in}{0.660000in}}{\pgfqpoint{4.620000in}{4.620000in}}%
\pgfusepath{clip}%
\pgfsetroundcap%
\pgfsetroundjoin%
\pgfsetlinewidth{1.204500pt}%
\definecolor{currentstroke}{rgb}{1.000000,0.576471,0.309804}%
\pgfsetstrokecolor{currentstroke}%
\pgfsetdash{}{0pt}%
\pgfpathmoveto{\pgfqpoint{3.312195in}{3.772983in}}%
\pgfusepath{stroke}%
\end{pgfscope}%
\begin{pgfscope}%
\pgfpathrectangle{\pgfqpoint{0.765000in}{0.660000in}}{\pgfqpoint{4.620000in}{4.620000in}}%
\pgfusepath{clip}%
\pgfsetbuttcap%
\pgfsetroundjoin%
\definecolor{currentfill}{rgb}{1.000000,0.576471,0.309804}%
\pgfsetfillcolor{currentfill}%
\pgfsetlinewidth{1.003750pt}%
\definecolor{currentstroke}{rgb}{1.000000,0.576471,0.309804}%
\pgfsetstrokecolor{currentstroke}%
\pgfsetdash{}{0pt}%
\pgfsys@defobject{currentmarker}{\pgfqpoint{-0.033333in}{-0.033333in}}{\pgfqpoint{0.033333in}{0.033333in}}{%
\pgfpathmoveto{\pgfqpoint{0.000000in}{-0.033333in}}%
\pgfpathcurveto{\pgfqpoint{0.008840in}{-0.033333in}}{\pgfqpoint{0.017319in}{-0.029821in}}{\pgfqpoint{0.023570in}{-0.023570in}}%
\pgfpathcurveto{\pgfqpoint{0.029821in}{-0.017319in}}{\pgfqpoint{0.033333in}{-0.008840in}}{\pgfqpoint{0.033333in}{0.000000in}}%
\pgfpathcurveto{\pgfqpoint{0.033333in}{0.008840in}}{\pgfqpoint{0.029821in}{0.017319in}}{\pgfqpoint{0.023570in}{0.023570in}}%
\pgfpathcurveto{\pgfqpoint{0.017319in}{0.029821in}}{\pgfqpoint{0.008840in}{0.033333in}}{\pgfqpoint{0.000000in}{0.033333in}}%
\pgfpathcurveto{\pgfqpoint{-0.008840in}{0.033333in}}{\pgfqpoint{-0.017319in}{0.029821in}}{\pgfqpoint{-0.023570in}{0.023570in}}%
\pgfpathcurveto{\pgfqpoint{-0.029821in}{0.017319in}}{\pgfqpoint{-0.033333in}{0.008840in}}{\pgfqpoint{-0.033333in}{0.000000in}}%
\pgfpathcurveto{\pgfqpoint{-0.033333in}{-0.008840in}}{\pgfqpoint{-0.029821in}{-0.017319in}}{\pgfqpoint{-0.023570in}{-0.023570in}}%
\pgfpathcurveto{\pgfqpoint{-0.017319in}{-0.029821in}}{\pgfqpoint{-0.008840in}{-0.033333in}}{\pgfqpoint{0.000000in}{-0.033333in}}%
\pgfpathlineto{\pgfqpoint{0.000000in}{-0.033333in}}%
\pgfpathclose%
\pgfusepath{stroke,fill}%
}%
\begin{pgfscope}%
\pgfsys@transformshift{3.312195in}{3.772983in}%
\pgfsys@useobject{currentmarker}{}%
\end{pgfscope}%
\end{pgfscope}%
\begin{pgfscope}%
\pgfpathrectangle{\pgfqpoint{0.765000in}{0.660000in}}{\pgfqpoint{4.620000in}{4.620000in}}%
\pgfusepath{clip}%
\pgfsetroundcap%
\pgfsetroundjoin%
\pgfsetlinewidth{1.204500pt}%
\definecolor{currentstroke}{rgb}{1.000000,0.576471,0.309804}%
\pgfsetstrokecolor{currentstroke}%
\pgfsetdash{}{0pt}%
\pgfpathmoveto{\pgfqpoint{4.007985in}{1.902519in}}%
\pgfusepath{stroke}%
\end{pgfscope}%
\begin{pgfscope}%
\pgfpathrectangle{\pgfqpoint{0.765000in}{0.660000in}}{\pgfqpoint{4.620000in}{4.620000in}}%
\pgfusepath{clip}%
\pgfsetbuttcap%
\pgfsetroundjoin%
\definecolor{currentfill}{rgb}{1.000000,0.576471,0.309804}%
\pgfsetfillcolor{currentfill}%
\pgfsetlinewidth{1.003750pt}%
\definecolor{currentstroke}{rgb}{1.000000,0.576471,0.309804}%
\pgfsetstrokecolor{currentstroke}%
\pgfsetdash{}{0pt}%
\pgfsys@defobject{currentmarker}{\pgfqpoint{-0.033333in}{-0.033333in}}{\pgfqpoint{0.033333in}{0.033333in}}{%
\pgfpathmoveto{\pgfqpoint{0.000000in}{-0.033333in}}%
\pgfpathcurveto{\pgfqpoint{0.008840in}{-0.033333in}}{\pgfqpoint{0.017319in}{-0.029821in}}{\pgfqpoint{0.023570in}{-0.023570in}}%
\pgfpathcurveto{\pgfqpoint{0.029821in}{-0.017319in}}{\pgfqpoint{0.033333in}{-0.008840in}}{\pgfqpoint{0.033333in}{0.000000in}}%
\pgfpathcurveto{\pgfqpoint{0.033333in}{0.008840in}}{\pgfqpoint{0.029821in}{0.017319in}}{\pgfqpoint{0.023570in}{0.023570in}}%
\pgfpathcurveto{\pgfqpoint{0.017319in}{0.029821in}}{\pgfqpoint{0.008840in}{0.033333in}}{\pgfqpoint{0.000000in}{0.033333in}}%
\pgfpathcurveto{\pgfqpoint{-0.008840in}{0.033333in}}{\pgfqpoint{-0.017319in}{0.029821in}}{\pgfqpoint{-0.023570in}{0.023570in}}%
\pgfpathcurveto{\pgfqpoint{-0.029821in}{0.017319in}}{\pgfqpoint{-0.033333in}{0.008840in}}{\pgfqpoint{-0.033333in}{0.000000in}}%
\pgfpathcurveto{\pgfqpoint{-0.033333in}{-0.008840in}}{\pgfqpoint{-0.029821in}{-0.017319in}}{\pgfqpoint{-0.023570in}{-0.023570in}}%
\pgfpathcurveto{\pgfqpoint{-0.017319in}{-0.029821in}}{\pgfqpoint{-0.008840in}{-0.033333in}}{\pgfqpoint{0.000000in}{-0.033333in}}%
\pgfpathlineto{\pgfqpoint{0.000000in}{-0.033333in}}%
\pgfpathclose%
\pgfusepath{stroke,fill}%
}%
\begin{pgfscope}%
\pgfsys@transformshift{4.007985in}{1.902519in}%
\pgfsys@useobject{currentmarker}{}%
\end{pgfscope}%
\end{pgfscope}%
\begin{pgfscope}%
\pgfpathrectangle{\pgfqpoint{0.765000in}{0.660000in}}{\pgfqpoint{4.620000in}{4.620000in}}%
\pgfusepath{clip}%
\pgfsetroundcap%
\pgfsetroundjoin%
\pgfsetlinewidth{1.204500pt}%
\definecolor{currentstroke}{rgb}{1.000000,0.576471,0.309804}%
\pgfsetstrokecolor{currentstroke}%
\pgfsetdash{}{0pt}%
\pgfpathmoveto{\pgfqpoint{2.594980in}{2.796384in}}%
\pgfusepath{stroke}%
\end{pgfscope}%
\begin{pgfscope}%
\pgfpathrectangle{\pgfqpoint{0.765000in}{0.660000in}}{\pgfqpoint{4.620000in}{4.620000in}}%
\pgfusepath{clip}%
\pgfsetbuttcap%
\pgfsetroundjoin%
\definecolor{currentfill}{rgb}{1.000000,0.576471,0.309804}%
\pgfsetfillcolor{currentfill}%
\pgfsetlinewidth{1.003750pt}%
\definecolor{currentstroke}{rgb}{1.000000,0.576471,0.309804}%
\pgfsetstrokecolor{currentstroke}%
\pgfsetdash{}{0pt}%
\pgfsys@defobject{currentmarker}{\pgfqpoint{-0.033333in}{-0.033333in}}{\pgfqpoint{0.033333in}{0.033333in}}{%
\pgfpathmoveto{\pgfqpoint{0.000000in}{-0.033333in}}%
\pgfpathcurveto{\pgfqpoint{0.008840in}{-0.033333in}}{\pgfqpoint{0.017319in}{-0.029821in}}{\pgfqpoint{0.023570in}{-0.023570in}}%
\pgfpathcurveto{\pgfqpoint{0.029821in}{-0.017319in}}{\pgfqpoint{0.033333in}{-0.008840in}}{\pgfqpoint{0.033333in}{0.000000in}}%
\pgfpathcurveto{\pgfqpoint{0.033333in}{0.008840in}}{\pgfqpoint{0.029821in}{0.017319in}}{\pgfqpoint{0.023570in}{0.023570in}}%
\pgfpathcurveto{\pgfqpoint{0.017319in}{0.029821in}}{\pgfqpoint{0.008840in}{0.033333in}}{\pgfqpoint{0.000000in}{0.033333in}}%
\pgfpathcurveto{\pgfqpoint{-0.008840in}{0.033333in}}{\pgfqpoint{-0.017319in}{0.029821in}}{\pgfqpoint{-0.023570in}{0.023570in}}%
\pgfpathcurveto{\pgfqpoint{-0.029821in}{0.017319in}}{\pgfqpoint{-0.033333in}{0.008840in}}{\pgfqpoint{-0.033333in}{0.000000in}}%
\pgfpathcurveto{\pgfqpoint{-0.033333in}{-0.008840in}}{\pgfqpoint{-0.029821in}{-0.017319in}}{\pgfqpoint{-0.023570in}{-0.023570in}}%
\pgfpathcurveto{\pgfqpoint{-0.017319in}{-0.029821in}}{\pgfqpoint{-0.008840in}{-0.033333in}}{\pgfqpoint{0.000000in}{-0.033333in}}%
\pgfpathlineto{\pgfqpoint{0.000000in}{-0.033333in}}%
\pgfpathclose%
\pgfusepath{stroke,fill}%
}%
\begin{pgfscope}%
\pgfsys@transformshift{2.594980in}{2.796384in}%
\pgfsys@useobject{currentmarker}{}%
\end{pgfscope}%
\end{pgfscope}%
\begin{pgfscope}%
\pgfpathrectangle{\pgfqpoint{0.765000in}{0.660000in}}{\pgfqpoint{4.620000in}{4.620000in}}%
\pgfusepath{clip}%
\pgfsetroundcap%
\pgfsetroundjoin%
\pgfsetlinewidth{1.204500pt}%
\definecolor{currentstroke}{rgb}{1.000000,0.576471,0.309804}%
\pgfsetstrokecolor{currentstroke}%
\pgfsetdash{}{0pt}%
\pgfpathmoveto{\pgfqpoint{2.992309in}{2.315058in}}%
\pgfusepath{stroke}%
\end{pgfscope}%
\begin{pgfscope}%
\pgfpathrectangle{\pgfqpoint{0.765000in}{0.660000in}}{\pgfqpoint{4.620000in}{4.620000in}}%
\pgfusepath{clip}%
\pgfsetbuttcap%
\pgfsetroundjoin%
\definecolor{currentfill}{rgb}{1.000000,0.576471,0.309804}%
\pgfsetfillcolor{currentfill}%
\pgfsetlinewidth{1.003750pt}%
\definecolor{currentstroke}{rgb}{1.000000,0.576471,0.309804}%
\pgfsetstrokecolor{currentstroke}%
\pgfsetdash{}{0pt}%
\pgfsys@defobject{currentmarker}{\pgfqpoint{-0.033333in}{-0.033333in}}{\pgfqpoint{0.033333in}{0.033333in}}{%
\pgfpathmoveto{\pgfqpoint{0.000000in}{-0.033333in}}%
\pgfpathcurveto{\pgfqpoint{0.008840in}{-0.033333in}}{\pgfqpoint{0.017319in}{-0.029821in}}{\pgfqpoint{0.023570in}{-0.023570in}}%
\pgfpathcurveto{\pgfqpoint{0.029821in}{-0.017319in}}{\pgfqpoint{0.033333in}{-0.008840in}}{\pgfqpoint{0.033333in}{0.000000in}}%
\pgfpathcurveto{\pgfqpoint{0.033333in}{0.008840in}}{\pgfqpoint{0.029821in}{0.017319in}}{\pgfqpoint{0.023570in}{0.023570in}}%
\pgfpathcurveto{\pgfqpoint{0.017319in}{0.029821in}}{\pgfqpoint{0.008840in}{0.033333in}}{\pgfqpoint{0.000000in}{0.033333in}}%
\pgfpathcurveto{\pgfqpoint{-0.008840in}{0.033333in}}{\pgfqpoint{-0.017319in}{0.029821in}}{\pgfqpoint{-0.023570in}{0.023570in}}%
\pgfpathcurveto{\pgfqpoint{-0.029821in}{0.017319in}}{\pgfqpoint{-0.033333in}{0.008840in}}{\pgfqpoint{-0.033333in}{0.000000in}}%
\pgfpathcurveto{\pgfqpoint{-0.033333in}{-0.008840in}}{\pgfqpoint{-0.029821in}{-0.017319in}}{\pgfqpoint{-0.023570in}{-0.023570in}}%
\pgfpathcurveto{\pgfqpoint{-0.017319in}{-0.029821in}}{\pgfqpoint{-0.008840in}{-0.033333in}}{\pgfqpoint{0.000000in}{-0.033333in}}%
\pgfpathlineto{\pgfqpoint{0.000000in}{-0.033333in}}%
\pgfpathclose%
\pgfusepath{stroke,fill}%
}%
\begin{pgfscope}%
\pgfsys@transformshift{2.992309in}{2.315058in}%
\pgfsys@useobject{currentmarker}{}%
\end{pgfscope}%
\end{pgfscope}%
\begin{pgfscope}%
\pgfpathrectangle{\pgfqpoint{0.765000in}{0.660000in}}{\pgfqpoint{4.620000in}{4.620000in}}%
\pgfusepath{clip}%
\pgfsetroundcap%
\pgfsetroundjoin%
\pgfsetlinewidth{1.204500pt}%
\definecolor{currentstroke}{rgb}{1.000000,0.576471,0.309804}%
\pgfsetstrokecolor{currentstroke}%
\pgfsetdash{}{0pt}%
\pgfpathmoveto{\pgfqpoint{2.840975in}{1.800598in}}%
\pgfusepath{stroke}%
\end{pgfscope}%
\begin{pgfscope}%
\pgfpathrectangle{\pgfqpoint{0.765000in}{0.660000in}}{\pgfqpoint{4.620000in}{4.620000in}}%
\pgfusepath{clip}%
\pgfsetbuttcap%
\pgfsetroundjoin%
\definecolor{currentfill}{rgb}{1.000000,0.576471,0.309804}%
\pgfsetfillcolor{currentfill}%
\pgfsetlinewidth{1.003750pt}%
\definecolor{currentstroke}{rgb}{1.000000,0.576471,0.309804}%
\pgfsetstrokecolor{currentstroke}%
\pgfsetdash{}{0pt}%
\pgfsys@defobject{currentmarker}{\pgfqpoint{-0.033333in}{-0.033333in}}{\pgfqpoint{0.033333in}{0.033333in}}{%
\pgfpathmoveto{\pgfqpoint{0.000000in}{-0.033333in}}%
\pgfpathcurveto{\pgfqpoint{0.008840in}{-0.033333in}}{\pgfqpoint{0.017319in}{-0.029821in}}{\pgfqpoint{0.023570in}{-0.023570in}}%
\pgfpathcurveto{\pgfqpoint{0.029821in}{-0.017319in}}{\pgfqpoint{0.033333in}{-0.008840in}}{\pgfqpoint{0.033333in}{0.000000in}}%
\pgfpathcurveto{\pgfqpoint{0.033333in}{0.008840in}}{\pgfqpoint{0.029821in}{0.017319in}}{\pgfqpoint{0.023570in}{0.023570in}}%
\pgfpathcurveto{\pgfqpoint{0.017319in}{0.029821in}}{\pgfqpoint{0.008840in}{0.033333in}}{\pgfqpoint{0.000000in}{0.033333in}}%
\pgfpathcurveto{\pgfqpoint{-0.008840in}{0.033333in}}{\pgfqpoint{-0.017319in}{0.029821in}}{\pgfqpoint{-0.023570in}{0.023570in}}%
\pgfpathcurveto{\pgfqpoint{-0.029821in}{0.017319in}}{\pgfqpoint{-0.033333in}{0.008840in}}{\pgfqpoint{-0.033333in}{0.000000in}}%
\pgfpathcurveto{\pgfqpoint{-0.033333in}{-0.008840in}}{\pgfqpoint{-0.029821in}{-0.017319in}}{\pgfqpoint{-0.023570in}{-0.023570in}}%
\pgfpathcurveto{\pgfqpoint{-0.017319in}{-0.029821in}}{\pgfqpoint{-0.008840in}{-0.033333in}}{\pgfqpoint{0.000000in}{-0.033333in}}%
\pgfpathlineto{\pgfqpoint{0.000000in}{-0.033333in}}%
\pgfpathclose%
\pgfusepath{stroke,fill}%
}%
\begin{pgfscope}%
\pgfsys@transformshift{2.840975in}{1.800598in}%
\pgfsys@useobject{currentmarker}{}%
\end{pgfscope}%
\end{pgfscope}%
\begin{pgfscope}%
\pgfpathrectangle{\pgfqpoint{0.765000in}{0.660000in}}{\pgfqpoint{4.620000in}{4.620000in}}%
\pgfusepath{clip}%
\pgfsetroundcap%
\pgfsetroundjoin%
\pgfsetlinewidth{1.204500pt}%
\definecolor{currentstroke}{rgb}{1.000000,0.576471,0.309804}%
\pgfsetstrokecolor{currentstroke}%
\pgfsetdash{}{0pt}%
\pgfpathmoveto{\pgfqpoint{3.047320in}{3.117067in}}%
\pgfusepath{stroke}%
\end{pgfscope}%
\begin{pgfscope}%
\pgfpathrectangle{\pgfqpoint{0.765000in}{0.660000in}}{\pgfqpoint{4.620000in}{4.620000in}}%
\pgfusepath{clip}%
\pgfsetbuttcap%
\pgfsetroundjoin%
\definecolor{currentfill}{rgb}{1.000000,0.576471,0.309804}%
\pgfsetfillcolor{currentfill}%
\pgfsetlinewidth{1.003750pt}%
\definecolor{currentstroke}{rgb}{1.000000,0.576471,0.309804}%
\pgfsetstrokecolor{currentstroke}%
\pgfsetdash{}{0pt}%
\pgfsys@defobject{currentmarker}{\pgfqpoint{-0.033333in}{-0.033333in}}{\pgfqpoint{0.033333in}{0.033333in}}{%
\pgfpathmoveto{\pgfqpoint{0.000000in}{-0.033333in}}%
\pgfpathcurveto{\pgfqpoint{0.008840in}{-0.033333in}}{\pgfqpoint{0.017319in}{-0.029821in}}{\pgfqpoint{0.023570in}{-0.023570in}}%
\pgfpathcurveto{\pgfqpoint{0.029821in}{-0.017319in}}{\pgfqpoint{0.033333in}{-0.008840in}}{\pgfqpoint{0.033333in}{0.000000in}}%
\pgfpathcurveto{\pgfqpoint{0.033333in}{0.008840in}}{\pgfqpoint{0.029821in}{0.017319in}}{\pgfqpoint{0.023570in}{0.023570in}}%
\pgfpathcurveto{\pgfqpoint{0.017319in}{0.029821in}}{\pgfqpoint{0.008840in}{0.033333in}}{\pgfqpoint{0.000000in}{0.033333in}}%
\pgfpathcurveto{\pgfqpoint{-0.008840in}{0.033333in}}{\pgfqpoint{-0.017319in}{0.029821in}}{\pgfqpoint{-0.023570in}{0.023570in}}%
\pgfpathcurveto{\pgfqpoint{-0.029821in}{0.017319in}}{\pgfqpoint{-0.033333in}{0.008840in}}{\pgfqpoint{-0.033333in}{0.000000in}}%
\pgfpathcurveto{\pgfqpoint{-0.033333in}{-0.008840in}}{\pgfqpoint{-0.029821in}{-0.017319in}}{\pgfqpoint{-0.023570in}{-0.023570in}}%
\pgfpathcurveto{\pgfqpoint{-0.017319in}{-0.029821in}}{\pgfqpoint{-0.008840in}{-0.033333in}}{\pgfqpoint{0.000000in}{-0.033333in}}%
\pgfpathlineto{\pgfqpoint{0.000000in}{-0.033333in}}%
\pgfpathclose%
\pgfusepath{stroke,fill}%
}%
\begin{pgfscope}%
\pgfsys@transformshift{3.047320in}{3.117067in}%
\pgfsys@useobject{currentmarker}{}%
\end{pgfscope}%
\end{pgfscope}%
\begin{pgfscope}%
\pgfpathrectangle{\pgfqpoint{0.765000in}{0.660000in}}{\pgfqpoint{4.620000in}{4.620000in}}%
\pgfusepath{clip}%
\pgfsetroundcap%
\pgfsetroundjoin%
\pgfsetlinewidth{1.204500pt}%
\definecolor{currentstroke}{rgb}{1.000000,0.576471,0.309804}%
\pgfsetstrokecolor{currentstroke}%
\pgfsetdash{}{0pt}%
\pgfpathmoveto{\pgfqpoint{2.667606in}{2.749869in}}%
\pgfusepath{stroke}%
\end{pgfscope}%
\begin{pgfscope}%
\pgfpathrectangle{\pgfqpoint{0.765000in}{0.660000in}}{\pgfqpoint{4.620000in}{4.620000in}}%
\pgfusepath{clip}%
\pgfsetbuttcap%
\pgfsetroundjoin%
\definecolor{currentfill}{rgb}{1.000000,0.576471,0.309804}%
\pgfsetfillcolor{currentfill}%
\pgfsetlinewidth{1.003750pt}%
\definecolor{currentstroke}{rgb}{1.000000,0.576471,0.309804}%
\pgfsetstrokecolor{currentstroke}%
\pgfsetdash{}{0pt}%
\pgfsys@defobject{currentmarker}{\pgfqpoint{-0.033333in}{-0.033333in}}{\pgfqpoint{0.033333in}{0.033333in}}{%
\pgfpathmoveto{\pgfqpoint{0.000000in}{-0.033333in}}%
\pgfpathcurveto{\pgfqpoint{0.008840in}{-0.033333in}}{\pgfqpoint{0.017319in}{-0.029821in}}{\pgfqpoint{0.023570in}{-0.023570in}}%
\pgfpathcurveto{\pgfqpoint{0.029821in}{-0.017319in}}{\pgfqpoint{0.033333in}{-0.008840in}}{\pgfqpoint{0.033333in}{0.000000in}}%
\pgfpathcurveto{\pgfqpoint{0.033333in}{0.008840in}}{\pgfqpoint{0.029821in}{0.017319in}}{\pgfqpoint{0.023570in}{0.023570in}}%
\pgfpathcurveto{\pgfqpoint{0.017319in}{0.029821in}}{\pgfqpoint{0.008840in}{0.033333in}}{\pgfqpoint{0.000000in}{0.033333in}}%
\pgfpathcurveto{\pgfqpoint{-0.008840in}{0.033333in}}{\pgfqpoint{-0.017319in}{0.029821in}}{\pgfqpoint{-0.023570in}{0.023570in}}%
\pgfpathcurveto{\pgfqpoint{-0.029821in}{0.017319in}}{\pgfqpoint{-0.033333in}{0.008840in}}{\pgfqpoint{-0.033333in}{0.000000in}}%
\pgfpathcurveto{\pgfqpoint{-0.033333in}{-0.008840in}}{\pgfqpoint{-0.029821in}{-0.017319in}}{\pgfqpoint{-0.023570in}{-0.023570in}}%
\pgfpathcurveto{\pgfqpoint{-0.017319in}{-0.029821in}}{\pgfqpoint{-0.008840in}{-0.033333in}}{\pgfqpoint{0.000000in}{-0.033333in}}%
\pgfpathlineto{\pgfqpoint{0.000000in}{-0.033333in}}%
\pgfpathclose%
\pgfusepath{stroke,fill}%
}%
\begin{pgfscope}%
\pgfsys@transformshift{2.667606in}{2.749869in}%
\pgfsys@useobject{currentmarker}{}%
\end{pgfscope}%
\end{pgfscope}%
\begin{pgfscope}%
\pgfpathrectangle{\pgfqpoint{0.765000in}{0.660000in}}{\pgfqpoint{4.620000in}{4.620000in}}%
\pgfusepath{clip}%
\pgfsetroundcap%
\pgfsetroundjoin%
\pgfsetlinewidth{1.204500pt}%
\definecolor{currentstroke}{rgb}{1.000000,0.576471,0.309804}%
\pgfsetstrokecolor{currentstroke}%
\pgfsetdash{}{0pt}%
\pgfpathmoveto{\pgfqpoint{3.256181in}{3.479251in}}%
\pgfusepath{stroke}%
\end{pgfscope}%
\begin{pgfscope}%
\pgfpathrectangle{\pgfqpoint{0.765000in}{0.660000in}}{\pgfqpoint{4.620000in}{4.620000in}}%
\pgfusepath{clip}%
\pgfsetbuttcap%
\pgfsetroundjoin%
\definecolor{currentfill}{rgb}{1.000000,0.576471,0.309804}%
\pgfsetfillcolor{currentfill}%
\pgfsetlinewidth{1.003750pt}%
\definecolor{currentstroke}{rgb}{1.000000,0.576471,0.309804}%
\pgfsetstrokecolor{currentstroke}%
\pgfsetdash{}{0pt}%
\pgfsys@defobject{currentmarker}{\pgfqpoint{-0.033333in}{-0.033333in}}{\pgfqpoint{0.033333in}{0.033333in}}{%
\pgfpathmoveto{\pgfqpoint{0.000000in}{-0.033333in}}%
\pgfpathcurveto{\pgfqpoint{0.008840in}{-0.033333in}}{\pgfqpoint{0.017319in}{-0.029821in}}{\pgfqpoint{0.023570in}{-0.023570in}}%
\pgfpathcurveto{\pgfqpoint{0.029821in}{-0.017319in}}{\pgfqpoint{0.033333in}{-0.008840in}}{\pgfqpoint{0.033333in}{0.000000in}}%
\pgfpathcurveto{\pgfqpoint{0.033333in}{0.008840in}}{\pgfqpoint{0.029821in}{0.017319in}}{\pgfqpoint{0.023570in}{0.023570in}}%
\pgfpathcurveto{\pgfqpoint{0.017319in}{0.029821in}}{\pgfqpoint{0.008840in}{0.033333in}}{\pgfqpoint{0.000000in}{0.033333in}}%
\pgfpathcurveto{\pgfqpoint{-0.008840in}{0.033333in}}{\pgfqpoint{-0.017319in}{0.029821in}}{\pgfqpoint{-0.023570in}{0.023570in}}%
\pgfpathcurveto{\pgfqpoint{-0.029821in}{0.017319in}}{\pgfqpoint{-0.033333in}{0.008840in}}{\pgfqpoint{-0.033333in}{0.000000in}}%
\pgfpathcurveto{\pgfqpoint{-0.033333in}{-0.008840in}}{\pgfqpoint{-0.029821in}{-0.017319in}}{\pgfqpoint{-0.023570in}{-0.023570in}}%
\pgfpathcurveto{\pgfqpoint{-0.017319in}{-0.029821in}}{\pgfqpoint{-0.008840in}{-0.033333in}}{\pgfqpoint{0.000000in}{-0.033333in}}%
\pgfpathlineto{\pgfqpoint{0.000000in}{-0.033333in}}%
\pgfpathclose%
\pgfusepath{stroke,fill}%
}%
\begin{pgfscope}%
\pgfsys@transformshift{3.256181in}{3.479251in}%
\pgfsys@useobject{currentmarker}{}%
\end{pgfscope}%
\end{pgfscope}%
\begin{pgfscope}%
\pgfpathrectangle{\pgfqpoint{0.765000in}{0.660000in}}{\pgfqpoint{4.620000in}{4.620000in}}%
\pgfusepath{clip}%
\pgfsetroundcap%
\pgfsetroundjoin%
\pgfsetlinewidth{1.204500pt}%
\definecolor{currentstroke}{rgb}{1.000000,0.576471,0.309804}%
\pgfsetstrokecolor{currentstroke}%
\pgfsetdash{}{0pt}%
\pgfpathmoveto{\pgfqpoint{2.792384in}{3.191881in}}%
\pgfusepath{stroke}%
\end{pgfscope}%
\begin{pgfscope}%
\pgfpathrectangle{\pgfqpoint{0.765000in}{0.660000in}}{\pgfqpoint{4.620000in}{4.620000in}}%
\pgfusepath{clip}%
\pgfsetbuttcap%
\pgfsetroundjoin%
\definecolor{currentfill}{rgb}{1.000000,0.576471,0.309804}%
\pgfsetfillcolor{currentfill}%
\pgfsetlinewidth{1.003750pt}%
\definecolor{currentstroke}{rgb}{1.000000,0.576471,0.309804}%
\pgfsetstrokecolor{currentstroke}%
\pgfsetdash{}{0pt}%
\pgfsys@defobject{currentmarker}{\pgfqpoint{-0.033333in}{-0.033333in}}{\pgfqpoint{0.033333in}{0.033333in}}{%
\pgfpathmoveto{\pgfqpoint{0.000000in}{-0.033333in}}%
\pgfpathcurveto{\pgfqpoint{0.008840in}{-0.033333in}}{\pgfqpoint{0.017319in}{-0.029821in}}{\pgfqpoint{0.023570in}{-0.023570in}}%
\pgfpathcurveto{\pgfqpoint{0.029821in}{-0.017319in}}{\pgfqpoint{0.033333in}{-0.008840in}}{\pgfqpoint{0.033333in}{0.000000in}}%
\pgfpathcurveto{\pgfqpoint{0.033333in}{0.008840in}}{\pgfqpoint{0.029821in}{0.017319in}}{\pgfqpoint{0.023570in}{0.023570in}}%
\pgfpathcurveto{\pgfqpoint{0.017319in}{0.029821in}}{\pgfqpoint{0.008840in}{0.033333in}}{\pgfqpoint{0.000000in}{0.033333in}}%
\pgfpathcurveto{\pgfqpoint{-0.008840in}{0.033333in}}{\pgfqpoint{-0.017319in}{0.029821in}}{\pgfqpoint{-0.023570in}{0.023570in}}%
\pgfpathcurveto{\pgfqpoint{-0.029821in}{0.017319in}}{\pgfqpoint{-0.033333in}{0.008840in}}{\pgfqpoint{-0.033333in}{0.000000in}}%
\pgfpathcurveto{\pgfqpoint{-0.033333in}{-0.008840in}}{\pgfqpoint{-0.029821in}{-0.017319in}}{\pgfqpoint{-0.023570in}{-0.023570in}}%
\pgfpathcurveto{\pgfqpoint{-0.017319in}{-0.029821in}}{\pgfqpoint{-0.008840in}{-0.033333in}}{\pgfqpoint{0.000000in}{-0.033333in}}%
\pgfpathlineto{\pgfqpoint{0.000000in}{-0.033333in}}%
\pgfpathclose%
\pgfusepath{stroke,fill}%
}%
\begin{pgfscope}%
\pgfsys@transformshift{2.792384in}{3.191881in}%
\pgfsys@useobject{currentmarker}{}%
\end{pgfscope}%
\end{pgfscope}%
\begin{pgfscope}%
\pgfpathrectangle{\pgfqpoint{0.765000in}{0.660000in}}{\pgfqpoint{4.620000in}{4.620000in}}%
\pgfusepath{clip}%
\pgfsetroundcap%
\pgfsetroundjoin%
\pgfsetlinewidth{1.204500pt}%
\definecolor{currentstroke}{rgb}{1.000000,0.576471,0.309804}%
\pgfsetstrokecolor{currentstroke}%
\pgfsetdash{}{0pt}%
\pgfpathmoveto{\pgfqpoint{2.972035in}{1.893135in}}%
\pgfusepath{stroke}%
\end{pgfscope}%
\begin{pgfscope}%
\pgfpathrectangle{\pgfqpoint{0.765000in}{0.660000in}}{\pgfqpoint{4.620000in}{4.620000in}}%
\pgfusepath{clip}%
\pgfsetbuttcap%
\pgfsetroundjoin%
\definecolor{currentfill}{rgb}{1.000000,0.576471,0.309804}%
\pgfsetfillcolor{currentfill}%
\pgfsetlinewidth{1.003750pt}%
\definecolor{currentstroke}{rgb}{1.000000,0.576471,0.309804}%
\pgfsetstrokecolor{currentstroke}%
\pgfsetdash{}{0pt}%
\pgfsys@defobject{currentmarker}{\pgfqpoint{-0.033333in}{-0.033333in}}{\pgfqpoint{0.033333in}{0.033333in}}{%
\pgfpathmoveto{\pgfqpoint{0.000000in}{-0.033333in}}%
\pgfpathcurveto{\pgfqpoint{0.008840in}{-0.033333in}}{\pgfqpoint{0.017319in}{-0.029821in}}{\pgfqpoint{0.023570in}{-0.023570in}}%
\pgfpathcurveto{\pgfqpoint{0.029821in}{-0.017319in}}{\pgfqpoint{0.033333in}{-0.008840in}}{\pgfqpoint{0.033333in}{0.000000in}}%
\pgfpathcurveto{\pgfqpoint{0.033333in}{0.008840in}}{\pgfqpoint{0.029821in}{0.017319in}}{\pgfqpoint{0.023570in}{0.023570in}}%
\pgfpathcurveto{\pgfqpoint{0.017319in}{0.029821in}}{\pgfqpoint{0.008840in}{0.033333in}}{\pgfqpoint{0.000000in}{0.033333in}}%
\pgfpathcurveto{\pgfqpoint{-0.008840in}{0.033333in}}{\pgfqpoint{-0.017319in}{0.029821in}}{\pgfqpoint{-0.023570in}{0.023570in}}%
\pgfpathcurveto{\pgfqpoint{-0.029821in}{0.017319in}}{\pgfqpoint{-0.033333in}{0.008840in}}{\pgfqpoint{-0.033333in}{0.000000in}}%
\pgfpathcurveto{\pgfqpoint{-0.033333in}{-0.008840in}}{\pgfqpoint{-0.029821in}{-0.017319in}}{\pgfqpoint{-0.023570in}{-0.023570in}}%
\pgfpathcurveto{\pgfqpoint{-0.017319in}{-0.029821in}}{\pgfqpoint{-0.008840in}{-0.033333in}}{\pgfqpoint{0.000000in}{-0.033333in}}%
\pgfpathlineto{\pgfqpoint{0.000000in}{-0.033333in}}%
\pgfpathclose%
\pgfusepath{stroke,fill}%
}%
\begin{pgfscope}%
\pgfsys@transformshift{2.972035in}{1.893135in}%
\pgfsys@useobject{currentmarker}{}%
\end{pgfscope}%
\end{pgfscope}%
\begin{pgfscope}%
\pgfpathrectangle{\pgfqpoint{0.765000in}{0.660000in}}{\pgfqpoint{4.620000in}{4.620000in}}%
\pgfusepath{clip}%
\pgfsetroundcap%
\pgfsetroundjoin%
\pgfsetlinewidth{1.204500pt}%
\definecolor{currentstroke}{rgb}{1.000000,0.576471,0.309804}%
\pgfsetstrokecolor{currentstroke}%
\pgfsetdash{}{0pt}%
\pgfpathmoveto{\pgfqpoint{3.857809in}{2.477363in}}%
\pgfusepath{stroke}%
\end{pgfscope}%
\begin{pgfscope}%
\pgfpathrectangle{\pgfqpoint{0.765000in}{0.660000in}}{\pgfqpoint{4.620000in}{4.620000in}}%
\pgfusepath{clip}%
\pgfsetbuttcap%
\pgfsetroundjoin%
\definecolor{currentfill}{rgb}{1.000000,0.576471,0.309804}%
\pgfsetfillcolor{currentfill}%
\pgfsetlinewidth{1.003750pt}%
\definecolor{currentstroke}{rgb}{1.000000,0.576471,0.309804}%
\pgfsetstrokecolor{currentstroke}%
\pgfsetdash{}{0pt}%
\pgfsys@defobject{currentmarker}{\pgfqpoint{-0.033333in}{-0.033333in}}{\pgfqpoint{0.033333in}{0.033333in}}{%
\pgfpathmoveto{\pgfqpoint{0.000000in}{-0.033333in}}%
\pgfpathcurveto{\pgfqpoint{0.008840in}{-0.033333in}}{\pgfqpoint{0.017319in}{-0.029821in}}{\pgfqpoint{0.023570in}{-0.023570in}}%
\pgfpathcurveto{\pgfqpoint{0.029821in}{-0.017319in}}{\pgfqpoint{0.033333in}{-0.008840in}}{\pgfqpoint{0.033333in}{0.000000in}}%
\pgfpathcurveto{\pgfqpoint{0.033333in}{0.008840in}}{\pgfqpoint{0.029821in}{0.017319in}}{\pgfqpoint{0.023570in}{0.023570in}}%
\pgfpathcurveto{\pgfqpoint{0.017319in}{0.029821in}}{\pgfqpoint{0.008840in}{0.033333in}}{\pgfqpoint{0.000000in}{0.033333in}}%
\pgfpathcurveto{\pgfqpoint{-0.008840in}{0.033333in}}{\pgfqpoint{-0.017319in}{0.029821in}}{\pgfqpoint{-0.023570in}{0.023570in}}%
\pgfpathcurveto{\pgfqpoint{-0.029821in}{0.017319in}}{\pgfqpoint{-0.033333in}{0.008840in}}{\pgfqpoint{-0.033333in}{0.000000in}}%
\pgfpathcurveto{\pgfqpoint{-0.033333in}{-0.008840in}}{\pgfqpoint{-0.029821in}{-0.017319in}}{\pgfqpoint{-0.023570in}{-0.023570in}}%
\pgfpathcurveto{\pgfqpoint{-0.017319in}{-0.029821in}}{\pgfqpoint{-0.008840in}{-0.033333in}}{\pgfqpoint{0.000000in}{-0.033333in}}%
\pgfpathlineto{\pgfqpoint{0.000000in}{-0.033333in}}%
\pgfpathclose%
\pgfusepath{stroke,fill}%
}%
\begin{pgfscope}%
\pgfsys@transformshift{3.857809in}{2.477363in}%
\pgfsys@useobject{currentmarker}{}%
\end{pgfscope}%
\end{pgfscope}%
\begin{pgfscope}%
\pgfpathrectangle{\pgfqpoint{0.765000in}{0.660000in}}{\pgfqpoint{4.620000in}{4.620000in}}%
\pgfusepath{clip}%
\pgfsetroundcap%
\pgfsetroundjoin%
\pgfsetlinewidth{1.204500pt}%
\definecolor{currentstroke}{rgb}{1.000000,0.576471,0.309804}%
\pgfsetstrokecolor{currentstroke}%
\pgfsetdash{}{0pt}%
\pgfpathmoveto{\pgfqpoint{3.414950in}{2.349548in}}%
\pgfusepath{stroke}%
\end{pgfscope}%
\begin{pgfscope}%
\pgfpathrectangle{\pgfqpoint{0.765000in}{0.660000in}}{\pgfqpoint{4.620000in}{4.620000in}}%
\pgfusepath{clip}%
\pgfsetbuttcap%
\pgfsetroundjoin%
\definecolor{currentfill}{rgb}{1.000000,0.576471,0.309804}%
\pgfsetfillcolor{currentfill}%
\pgfsetlinewidth{1.003750pt}%
\definecolor{currentstroke}{rgb}{1.000000,0.576471,0.309804}%
\pgfsetstrokecolor{currentstroke}%
\pgfsetdash{}{0pt}%
\pgfsys@defobject{currentmarker}{\pgfqpoint{-0.033333in}{-0.033333in}}{\pgfqpoint{0.033333in}{0.033333in}}{%
\pgfpathmoveto{\pgfqpoint{0.000000in}{-0.033333in}}%
\pgfpathcurveto{\pgfqpoint{0.008840in}{-0.033333in}}{\pgfqpoint{0.017319in}{-0.029821in}}{\pgfqpoint{0.023570in}{-0.023570in}}%
\pgfpathcurveto{\pgfqpoint{0.029821in}{-0.017319in}}{\pgfqpoint{0.033333in}{-0.008840in}}{\pgfqpoint{0.033333in}{0.000000in}}%
\pgfpathcurveto{\pgfqpoint{0.033333in}{0.008840in}}{\pgfqpoint{0.029821in}{0.017319in}}{\pgfqpoint{0.023570in}{0.023570in}}%
\pgfpathcurveto{\pgfqpoint{0.017319in}{0.029821in}}{\pgfqpoint{0.008840in}{0.033333in}}{\pgfqpoint{0.000000in}{0.033333in}}%
\pgfpathcurveto{\pgfqpoint{-0.008840in}{0.033333in}}{\pgfqpoint{-0.017319in}{0.029821in}}{\pgfqpoint{-0.023570in}{0.023570in}}%
\pgfpathcurveto{\pgfqpoint{-0.029821in}{0.017319in}}{\pgfqpoint{-0.033333in}{0.008840in}}{\pgfqpoint{-0.033333in}{0.000000in}}%
\pgfpathcurveto{\pgfqpoint{-0.033333in}{-0.008840in}}{\pgfqpoint{-0.029821in}{-0.017319in}}{\pgfqpoint{-0.023570in}{-0.023570in}}%
\pgfpathcurveto{\pgfqpoint{-0.017319in}{-0.029821in}}{\pgfqpoint{-0.008840in}{-0.033333in}}{\pgfqpoint{0.000000in}{-0.033333in}}%
\pgfpathlineto{\pgfqpoint{0.000000in}{-0.033333in}}%
\pgfpathclose%
\pgfusepath{stroke,fill}%
}%
\begin{pgfscope}%
\pgfsys@transformshift{3.414950in}{2.349548in}%
\pgfsys@useobject{currentmarker}{}%
\end{pgfscope}%
\end{pgfscope}%
\begin{pgfscope}%
\pgfpathrectangle{\pgfqpoint{0.765000in}{0.660000in}}{\pgfqpoint{4.620000in}{4.620000in}}%
\pgfusepath{clip}%
\pgfsetroundcap%
\pgfsetroundjoin%
\pgfsetlinewidth{1.204500pt}%
\definecolor{currentstroke}{rgb}{1.000000,0.576471,0.309804}%
\pgfsetstrokecolor{currentstroke}%
\pgfsetdash{}{0pt}%
\pgfpathmoveto{\pgfqpoint{2.319733in}{2.982624in}}%
\pgfusepath{stroke}%
\end{pgfscope}%
\begin{pgfscope}%
\pgfpathrectangle{\pgfqpoint{0.765000in}{0.660000in}}{\pgfqpoint{4.620000in}{4.620000in}}%
\pgfusepath{clip}%
\pgfsetbuttcap%
\pgfsetroundjoin%
\definecolor{currentfill}{rgb}{1.000000,0.576471,0.309804}%
\pgfsetfillcolor{currentfill}%
\pgfsetlinewidth{1.003750pt}%
\definecolor{currentstroke}{rgb}{1.000000,0.576471,0.309804}%
\pgfsetstrokecolor{currentstroke}%
\pgfsetdash{}{0pt}%
\pgfsys@defobject{currentmarker}{\pgfqpoint{-0.033333in}{-0.033333in}}{\pgfqpoint{0.033333in}{0.033333in}}{%
\pgfpathmoveto{\pgfqpoint{0.000000in}{-0.033333in}}%
\pgfpathcurveto{\pgfqpoint{0.008840in}{-0.033333in}}{\pgfqpoint{0.017319in}{-0.029821in}}{\pgfqpoint{0.023570in}{-0.023570in}}%
\pgfpathcurveto{\pgfqpoint{0.029821in}{-0.017319in}}{\pgfqpoint{0.033333in}{-0.008840in}}{\pgfqpoint{0.033333in}{0.000000in}}%
\pgfpathcurveto{\pgfqpoint{0.033333in}{0.008840in}}{\pgfqpoint{0.029821in}{0.017319in}}{\pgfqpoint{0.023570in}{0.023570in}}%
\pgfpathcurveto{\pgfqpoint{0.017319in}{0.029821in}}{\pgfqpoint{0.008840in}{0.033333in}}{\pgfqpoint{0.000000in}{0.033333in}}%
\pgfpathcurveto{\pgfqpoint{-0.008840in}{0.033333in}}{\pgfqpoint{-0.017319in}{0.029821in}}{\pgfqpoint{-0.023570in}{0.023570in}}%
\pgfpathcurveto{\pgfqpoint{-0.029821in}{0.017319in}}{\pgfqpoint{-0.033333in}{0.008840in}}{\pgfqpoint{-0.033333in}{0.000000in}}%
\pgfpathcurveto{\pgfqpoint{-0.033333in}{-0.008840in}}{\pgfqpoint{-0.029821in}{-0.017319in}}{\pgfqpoint{-0.023570in}{-0.023570in}}%
\pgfpathcurveto{\pgfqpoint{-0.017319in}{-0.029821in}}{\pgfqpoint{-0.008840in}{-0.033333in}}{\pgfqpoint{0.000000in}{-0.033333in}}%
\pgfpathlineto{\pgfqpoint{0.000000in}{-0.033333in}}%
\pgfpathclose%
\pgfusepath{stroke,fill}%
}%
\begin{pgfscope}%
\pgfsys@transformshift{2.319733in}{2.982624in}%
\pgfsys@useobject{currentmarker}{}%
\end{pgfscope}%
\end{pgfscope}%
\begin{pgfscope}%
\pgfpathrectangle{\pgfqpoint{0.765000in}{0.660000in}}{\pgfqpoint{4.620000in}{4.620000in}}%
\pgfusepath{clip}%
\pgfsetroundcap%
\pgfsetroundjoin%
\pgfsetlinewidth{1.204500pt}%
\definecolor{currentstroke}{rgb}{1.000000,0.576471,0.309804}%
\pgfsetstrokecolor{currentstroke}%
\pgfsetdash{}{0pt}%
\pgfpathmoveto{\pgfqpoint{3.001493in}{3.494867in}}%
\pgfusepath{stroke}%
\end{pgfscope}%
\begin{pgfscope}%
\pgfpathrectangle{\pgfqpoint{0.765000in}{0.660000in}}{\pgfqpoint{4.620000in}{4.620000in}}%
\pgfusepath{clip}%
\pgfsetbuttcap%
\pgfsetroundjoin%
\definecolor{currentfill}{rgb}{1.000000,0.576471,0.309804}%
\pgfsetfillcolor{currentfill}%
\pgfsetlinewidth{1.003750pt}%
\definecolor{currentstroke}{rgb}{1.000000,0.576471,0.309804}%
\pgfsetstrokecolor{currentstroke}%
\pgfsetdash{}{0pt}%
\pgfsys@defobject{currentmarker}{\pgfqpoint{-0.033333in}{-0.033333in}}{\pgfqpoint{0.033333in}{0.033333in}}{%
\pgfpathmoveto{\pgfqpoint{0.000000in}{-0.033333in}}%
\pgfpathcurveto{\pgfqpoint{0.008840in}{-0.033333in}}{\pgfqpoint{0.017319in}{-0.029821in}}{\pgfqpoint{0.023570in}{-0.023570in}}%
\pgfpathcurveto{\pgfqpoint{0.029821in}{-0.017319in}}{\pgfqpoint{0.033333in}{-0.008840in}}{\pgfqpoint{0.033333in}{0.000000in}}%
\pgfpathcurveto{\pgfqpoint{0.033333in}{0.008840in}}{\pgfqpoint{0.029821in}{0.017319in}}{\pgfqpoint{0.023570in}{0.023570in}}%
\pgfpathcurveto{\pgfqpoint{0.017319in}{0.029821in}}{\pgfqpoint{0.008840in}{0.033333in}}{\pgfqpoint{0.000000in}{0.033333in}}%
\pgfpathcurveto{\pgfqpoint{-0.008840in}{0.033333in}}{\pgfqpoint{-0.017319in}{0.029821in}}{\pgfqpoint{-0.023570in}{0.023570in}}%
\pgfpathcurveto{\pgfqpoint{-0.029821in}{0.017319in}}{\pgfqpoint{-0.033333in}{0.008840in}}{\pgfqpoint{-0.033333in}{0.000000in}}%
\pgfpathcurveto{\pgfqpoint{-0.033333in}{-0.008840in}}{\pgfqpoint{-0.029821in}{-0.017319in}}{\pgfqpoint{-0.023570in}{-0.023570in}}%
\pgfpathcurveto{\pgfqpoint{-0.017319in}{-0.029821in}}{\pgfqpoint{-0.008840in}{-0.033333in}}{\pgfqpoint{0.000000in}{-0.033333in}}%
\pgfpathlineto{\pgfqpoint{0.000000in}{-0.033333in}}%
\pgfpathclose%
\pgfusepath{stroke,fill}%
}%
\begin{pgfscope}%
\pgfsys@transformshift{3.001493in}{3.494867in}%
\pgfsys@useobject{currentmarker}{}%
\end{pgfscope}%
\end{pgfscope}%
\begin{pgfscope}%
\pgfpathrectangle{\pgfqpoint{0.765000in}{0.660000in}}{\pgfqpoint{4.620000in}{4.620000in}}%
\pgfusepath{clip}%
\pgfsetroundcap%
\pgfsetroundjoin%
\pgfsetlinewidth{1.204500pt}%
\definecolor{currentstroke}{rgb}{1.000000,0.576471,0.309804}%
\pgfsetstrokecolor{currentstroke}%
\pgfsetdash{}{0pt}%
\pgfpathmoveto{\pgfqpoint{3.620514in}{3.110511in}}%
\pgfusepath{stroke}%
\end{pgfscope}%
\begin{pgfscope}%
\pgfpathrectangle{\pgfqpoint{0.765000in}{0.660000in}}{\pgfqpoint{4.620000in}{4.620000in}}%
\pgfusepath{clip}%
\pgfsetbuttcap%
\pgfsetroundjoin%
\definecolor{currentfill}{rgb}{1.000000,0.576471,0.309804}%
\pgfsetfillcolor{currentfill}%
\pgfsetlinewidth{1.003750pt}%
\definecolor{currentstroke}{rgb}{1.000000,0.576471,0.309804}%
\pgfsetstrokecolor{currentstroke}%
\pgfsetdash{}{0pt}%
\pgfsys@defobject{currentmarker}{\pgfqpoint{-0.033333in}{-0.033333in}}{\pgfqpoint{0.033333in}{0.033333in}}{%
\pgfpathmoveto{\pgfqpoint{0.000000in}{-0.033333in}}%
\pgfpathcurveto{\pgfqpoint{0.008840in}{-0.033333in}}{\pgfqpoint{0.017319in}{-0.029821in}}{\pgfqpoint{0.023570in}{-0.023570in}}%
\pgfpathcurveto{\pgfqpoint{0.029821in}{-0.017319in}}{\pgfqpoint{0.033333in}{-0.008840in}}{\pgfqpoint{0.033333in}{0.000000in}}%
\pgfpathcurveto{\pgfqpoint{0.033333in}{0.008840in}}{\pgfqpoint{0.029821in}{0.017319in}}{\pgfqpoint{0.023570in}{0.023570in}}%
\pgfpathcurveto{\pgfqpoint{0.017319in}{0.029821in}}{\pgfqpoint{0.008840in}{0.033333in}}{\pgfqpoint{0.000000in}{0.033333in}}%
\pgfpathcurveto{\pgfqpoint{-0.008840in}{0.033333in}}{\pgfqpoint{-0.017319in}{0.029821in}}{\pgfqpoint{-0.023570in}{0.023570in}}%
\pgfpathcurveto{\pgfqpoint{-0.029821in}{0.017319in}}{\pgfqpoint{-0.033333in}{0.008840in}}{\pgfqpoint{-0.033333in}{0.000000in}}%
\pgfpathcurveto{\pgfqpoint{-0.033333in}{-0.008840in}}{\pgfqpoint{-0.029821in}{-0.017319in}}{\pgfqpoint{-0.023570in}{-0.023570in}}%
\pgfpathcurveto{\pgfqpoint{-0.017319in}{-0.029821in}}{\pgfqpoint{-0.008840in}{-0.033333in}}{\pgfqpoint{0.000000in}{-0.033333in}}%
\pgfpathlineto{\pgfqpoint{0.000000in}{-0.033333in}}%
\pgfpathclose%
\pgfusepath{stroke,fill}%
}%
\begin{pgfscope}%
\pgfsys@transformshift{3.620514in}{3.110511in}%
\pgfsys@useobject{currentmarker}{}%
\end{pgfscope}%
\end{pgfscope}%
\begin{pgfscope}%
\pgfpathrectangle{\pgfqpoint{0.765000in}{0.660000in}}{\pgfqpoint{4.620000in}{4.620000in}}%
\pgfusepath{clip}%
\pgfsetroundcap%
\pgfsetroundjoin%
\pgfsetlinewidth{1.204500pt}%
\definecolor{currentstroke}{rgb}{1.000000,0.576471,0.309804}%
\pgfsetstrokecolor{currentstroke}%
\pgfsetdash{}{0pt}%
\pgfpathmoveto{\pgfqpoint{2.414709in}{3.137430in}}%
\pgfusepath{stroke}%
\end{pgfscope}%
\begin{pgfscope}%
\pgfpathrectangle{\pgfqpoint{0.765000in}{0.660000in}}{\pgfqpoint{4.620000in}{4.620000in}}%
\pgfusepath{clip}%
\pgfsetbuttcap%
\pgfsetroundjoin%
\definecolor{currentfill}{rgb}{1.000000,0.576471,0.309804}%
\pgfsetfillcolor{currentfill}%
\pgfsetlinewidth{1.003750pt}%
\definecolor{currentstroke}{rgb}{1.000000,0.576471,0.309804}%
\pgfsetstrokecolor{currentstroke}%
\pgfsetdash{}{0pt}%
\pgfsys@defobject{currentmarker}{\pgfqpoint{-0.033333in}{-0.033333in}}{\pgfqpoint{0.033333in}{0.033333in}}{%
\pgfpathmoveto{\pgfqpoint{0.000000in}{-0.033333in}}%
\pgfpathcurveto{\pgfqpoint{0.008840in}{-0.033333in}}{\pgfqpoint{0.017319in}{-0.029821in}}{\pgfqpoint{0.023570in}{-0.023570in}}%
\pgfpathcurveto{\pgfqpoint{0.029821in}{-0.017319in}}{\pgfqpoint{0.033333in}{-0.008840in}}{\pgfqpoint{0.033333in}{0.000000in}}%
\pgfpathcurveto{\pgfqpoint{0.033333in}{0.008840in}}{\pgfqpoint{0.029821in}{0.017319in}}{\pgfqpoint{0.023570in}{0.023570in}}%
\pgfpathcurveto{\pgfqpoint{0.017319in}{0.029821in}}{\pgfqpoint{0.008840in}{0.033333in}}{\pgfqpoint{0.000000in}{0.033333in}}%
\pgfpathcurveto{\pgfqpoint{-0.008840in}{0.033333in}}{\pgfqpoint{-0.017319in}{0.029821in}}{\pgfqpoint{-0.023570in}{0.023570in}}%
\pgfpathcurveto{\pgfqpoint{-0.029821in}{0.017319in}}{\pgfqpoint{-0.033333in}{0.008840in}}{\pgfqpoint{-0.033333in}{0.000000in}}%
\pgfpathcurveto{\pgfqpoint{-0.033333in}{-0.008840in}}{\pgfqpoint{-0.029821in}{-0.017319in}}{\pgfqpoint{-0.023570in}{-0.023570in}}%
\pgfpathcurveto{\pgfqpoint{-0.017319in}{-0.029821in}}{\pgfqpoint{-0.008840in}{-0.033333in}}{\pgfqpoint{0.000000in}{-0.033333in}}%
\pgfpathlineto{\pgfqpoint{0.000000in}{-0.033333in}}%
\pgfpathclose%
\pgfusepath{stroke,fill}%
}%
\begin{pgfscope}%
\pgfsys@transformshift{2.414709in}{3.137430in}%
\pgfsys@useobject{currentmarker}{}%
\end{pgfscope}%
\end{pgfscope}%
\begin{pgfscope}%
\pgfpathrectangle{\pgfqpoint{0.765000in}{0.660000in}}{\pgfqpoint{4.620000in}{4.620000in}}%
\pgfusepath{clip}%
\pgfsetroundcap%
\pgfsetroundjoin%
\pgfsetlinewidth{1.204500pt}%
\definecolor{currentstroke}{rgb}{1.000000,0.576471,0.309804}%
\pgfsetstrokecolor{currentstroke}%
\pgfsetdash{}{0pt}%
\pgfpathmoveto{\pgfqpoint{2.431329in}{3.370457in}}%
\pgfusepath{stroke}%
\end{pgfscope}%
\begin{pgfscope}%
\pgfpathrectangle{\pgfqpoint{0.765000in}{0.660000in}}{\pgfqpoint{4.620000in}{4.620000in}}%
\pgfusepath{clip}%
\pgfsetbuttcap%
\pgfsetroundjoin%
\definecolor{currentfill}{rgb}{1.000000,0.576471,0.309804}%
\pgfsetfillcolor{currentfill}%
\pgfsetlinewidth{1.003750pt}%
\definecolor{currentstroke}{rgb}{1.000000,0.576471,0.309804}%
\pgfsetstrokecolor{currentstroke}%
\pgfsetdash{}{0pt}%
\pgfsys@defobject{currentmarker}{\pgfqpoint{-0.033333in}{-0.033333in}}{\pgfqpoint{0.033333in}{0.033333in}}{%
\pgfpathmoveto{\pgfqpoint{0.000000in}{-0.033333in}}%
\pgfpathcurveto{\pgfqpoint{0.008840in}{-0.033333in}}{\pgfqpoint{0.017319in}{-0.029821in}}{\pgfqpoint{0.023570in}{-0.023570in}}%
\pgfpathcurveto{\pgfqpoint{0.029821in}{-0.017319in}}{\pgfqpoint{0.033333in}{-0.008840in}}{\pgfqpoint{0.033333in}{0.000000in}}%
\pgfpathcurveto{\pgfqpoint{0.033333in}{0.008840in}}{\pgfqpoint{0.029821in}{0.017319in}}{\pgfqpoint{0.023570in}{0.023570in}}%
\pgfpathcurveto{\pgfqpoint{0.017319in}{0.029821in}}{\pgfqpoint{0.008840in}{0.033333in}}{\pgfqpoint{0.000000in}{0.033333in}}%
\pgfpathcurveto{\pgfqpoint{-0.008840in}{0.033333in}}{\pgfqpoint{-0.017319in}{0.029821in}}{\pgfqpoint{-0.023570in}{0.023570in}}%
\pgfpathcurveto{\pgfqpoint{-0.029821in}{0.017319in}}{\pgfqpoint{-0.033333in}{0.008840in}}{\pgfqpoint{-0.033333in}{0.000000in}}%
\pgfpathcurveto{\pgfqpoint{-0.033333in}{-0.008840in}}{\pgfqpoint{-0.029821in}{-0.017319in}}{\pgfqpoint{-0.023570in}{-0.023570in}}%
\pgfpathcurveto{\pgfqpoint{-0.017319in}{-0.029821in}}{\pgfqpoint{-0.008840in}{-0.033333in}}{\pgfqpoint{0.000000in}{-0.033333in}}%
\pgfpathlineto{\pgfqpoint{0.000000in}{-0.033333in}}%
\pgfpathclose%
\pgfusepath{stroke,fill}%
}%
\begin{pgfscope}%
\pgfsys@transformshift{2.431329in}{3.370457in}%
\pgfsys@useobject{currentmarker}{}%
\end{pgfscope}%
\end{pgfscope}%
\begin{pgfscope}%
\pgfpathrectangle{\pgfqpoint{0.765000in}{0.660000in}}{\pgfqpoint{4.620000in}{4.620000in}}%
\pgfusepath{clip}%
\pgfsetroundcap%
\pgfsetroundjoin%
\pgfsetlinewidth{1.204500pt}%
\definecolor{currentstroke}{rgb}{1.000000,0.576471,0.309804}%
\pgfsetstrokecolor{currentstroke}%
\pgfsetdash{}{0pt}%
\pgfpathmoveto{\pgfqpoint{3.873137in}{3.278333in}}%
\pgfusepath{stroke}%
\end{pgfscope}%
\begin{pgfscope}%
\pgfpathrectangle{\pgfqpoint{0.765000in}{0.660000in}}{\pgfqpoint{4.620000in}{4.620000in}}%
\pgfusepath{clip}%
\pgfsetbuttcap%
\pgfsetroundjoin%
\definecolor{currentfill}{rgb}{1.000000,0.576471,0.309804}%
\pgfsetfillcolor{currentfill}%
\pgfsetlinewidth{1.003750pt}%
\definecolor{currentstroke}{rgb}{1.000000,0.576471,0.309804}%
\pgfsetstrokecolor{currentstroke}%
\pgfsetdash{}{0pt}%
\pgfsys@defobject{currentmarker}{\pgfqpoint{-0.033333in}{-0.033333in}}{\pgfqpoint{0.033333in}{0.033333in}}{%
\pgfpathmoveto{\pgfqpoint{0.000000in}{-0.033333in}}%
\pgfpathcurveto{\pgfqpoint{0.008840in}{-0.033333in}}{\pgfqpoint{0.017319in}{-0.029821in}}{\pgfqpoint{0.023570in}{-0.023570in}}%
\pgfpathcurveto{\pgfqpoint{0.029821in}{-0.017319in}}{\pgfqpoint{0.033333in}{-0.008840in}}{\pgfqpoint{0.033333in}{0.000000in}}%
\pgfpathcurveto{\pgfqpoint{0.033333in}{0.008840in}}{\pgfqpoint{0.029821in}{0.017319in}}{\pgfqpoint{0.023570in}{0.023570in}}%
\pgfpathcurveto{\pgfqpoint{0.017319in}{0.029821in}}{\pgfqpoint{0.008840in}{0.033333in}}{\pgfqpoint{0.000000in}{0.033333in}}%
\pgfpathcurveto{\pgfqpoint{-0.008840in}{0.033333in}}{\pgfqpoint{-0.017319in}{0.029821in}}{\pgfqpoint{-0.023570in}{0.023570in}}%
\pgfpathcurveto{\pgfqpoint{-0.029821in}{0.017319in}}{\pgfqpoint{-0.033333in}{0.008840in}}{\pgfqpoint{-0.033333in}{0.000000in}}%
\pgfpathcurveto{\pgfqpoint{-0.033333in}{-0.008840in}}{\pgfqpoint{-0.029821in}{-0.017319in}}{\pgfqpoint{-0.023570in}{-0.023570in}}%
\pgfpathcurveto{\pgfqpoint{-0.017319in}{-0.029821in}}{\pgfqpoint{-0.008840in}{-0.033333in}}{\pgfqpoint{0.000000in}{-0.033333in}}%
\pgfpathlineto{\pgfqpoint{0.000000in}{-0.033333in}}%
\pgfpathclose%
\pgfusepath{stroke,fill}%
}%
\begin{pgfscope}%
\pgfsys@transformshift{3.873137in}{3.278333in}%
\pgfsys@useobject{currentmarker}{}%
\end{pgfscope}%
\end{pgfscope}%
\begin{pgfscope}%
\pgfpathrectangle{\pgfqpoint{0.765000in}{0.660000in}}{\pgfqpoint{4.620000in}{4.620000in}}%
\pgfusepath{clip}%
\pgfsetroundcap%
\pgfsetroundjoin%
\pgfsetlinewidth{1.204500pt}%
\definecolor{currentstroke}{rgb}{1.000000,0.576471,0.309804}%
\pgfsetstrokecolor{currentstroke}%
\pgfsetdash{}{0pt}%
\pgfpathmoveto{\pgfqpoint{2.984434in}{3.110429in}}%
\pgfusepath{stroke}%
\end{pgfscope}%
\begin{pgfscope}%
\pgfpathrectangle{\pgfqpoint{0.765000in}{0.660000in}}{\pgfqpoint{4.620000in}{4.620000in}}%
\pgfusepath{clip}%
\pgfsetbuttcap%
\pgfsetroundjoin%
\definecolor{currentfill}{rgb}{1.000000,0.576471,0.309804}%
\pgfsetfillcolor{currentfill}%
\pgfsetlinewidth{1.003750pt}%
\definecolor{currentstroke}{rgb}{1.000000,0.576471,0.309804}%
\pgfsetstrokecolor{currentstroke}%
\pgfsetdash{}{0pt}%
\pgfsys@defobject{currentmarker}{\pgfqpoint{-0.033333in}{-0.033333in}}{\pgfqpoint{0.033333in}{0.033333in}}{%
\pgfpathmoveto{\pgfqpoint{0.000000in}{-0.033333in}}%
\pgfpathcurveto{\pgfqpoint{0.008840in}{-0.033333in}}{\pgfqpoint{0.017319in}{-0.029821in}}{\pgfqpoint{0.023570in}{-0.023570in}}%
\pgfpathcurveto{\pgfqpoint{0.029821in}{-0.017319in}}{\pgfqpoint{0.033333in}{-0.008840in}}{\pgfqpoint{0.033333in}{0.000000in}}%
\pgfpathcurveto{\pgfqpoint{0.033333in}{0.008840in}}{\pgfqpoint{0.029821in}{0.017319in}}{\pgfqpoint{0.023570in}{0.023570in}}%
\pgfpathcurveto{\pgfqpoint{0.017319in}{0.029821in}}{\pgfqpoint{0.008840in}{0.033333in}}{\pgfqpoint{0.000000in}{0.033333in}}%
\pgfpathcurveto{\pgfqpoint{-0.008840in}{0.033333in}}{\pgfqpoint{-0.017319in}{0.029821in}}{\pgfqpoint{-0.023570in}{0.023570in}}%
\pgfpathcurveto{\pgfqpoint{-0.029821in}{0.017319in}}{\pgfqpoint{-0.033333in}{0.008840in}}{\pgfqpoint{-0.033333in}{0.000000in}}%
\pgfpathcurveto{\pgfqpoint{-0.033333in}{-0.008840in}}{\pgfqpoint{-0.029821in}{-0.017319in}}{\pgfqpoint{-0.023570in}{-0.023570in}}%
\pgfpathcurveto{\pgfqpoint{-0.017319in}{-0.029821in}}{\pgfqpoint{-0.008840in}{-0.033333in}}{\pgfqpoint{0.000000in}{-0.033333in}}%
\pgfpathlineto{\pgfqpoint{0.000000in}{-0.033333in}}%
\pgfpathclose%
\pgfusepath{stroke,fill}%
}%
\begin{pgfscope}%
\pgfsys@transformshift{2.984434in}{3.110429in}%
\pgfsys@useobject{currentmarker}{}%
\end{pgfscope}%
\end{pgfscope}%
\begin{pgfscope}%
\pgfpathrectangle{\pgfqpoint{0.765000in}{0.660000in}}{\pgfqpoint{4.620000in}{4.620000in}}%
\pgfusepath{clip}%
\pgfsetroundcap%
\pgfsetroundjoin%
\pgfsetlinewidth{1.204500pt}%
\definecolor{currentstroke}{rgb}{0.176471,0.192157,0.258824}%
\pgfsetstrokecolor{currentstroke}%
\pgfsetdash{}{0pt}%
\pgfpathmoveto{\pgfqpoint{3.245806in}{2.844126in}}%
\pgfusepath{stroke}%
\end{pgfscope}%
\begin{pgfscope}%
\pgfpathrectangle{\pgfqpoint{0.765000in}{0.660000in}}{\pgfqpoint{4.620000in}{4.620000in}}%
\pgfusepath{clip}%
\pgfsetbuttcap%
\pgfsetmiterjoin%
\definecolor{currentfill}{rgb}{0.176471,0.192157,0.258824}%
\pgfsetfillcolor{currentfill}%
\pgfsetlinewidth{1.003750pt}%
\definecolor{currentstroke}{rgb}{0.176471,0.192157,0.258824}%
\pgfsetstrokecolor{currentstroke}%
\pgfsetdash{}{0pt}%
\pgfsys@defobject{currentmarker}{\pgfqpoint{-0.033333in}{-0.033333in}}{\pgfqpoint{0.033333in}{0.033333in}}{%
\pgfpathmoveto{\pgfqpoint{-0.000000in}{-0.033333in}}%
\pgfpathlineto{\pgfqpoint{0.033333in}{0.033333in}}%
\pgfpathlineto{\pgfqpoint{-0.033333in}{0.033333in}}%
\pgfpathlineto{\pgfqpoint{-0.000000in}{-0.033333in}}%
\pgfpathclose%
\pgfusepath{stroke,fill}%
}%
\begin{pgfscope}%
\pgfsys@transformshift{3.245806in}{2.844126in}%
\pgfsys@useobject{currentmarker}{}%
\end{pgfscope}%
\end{pgfscope}%
\begin{pgfscope}%
\pgfpathrectangle{\pgfqpoint{0.765000in}{0.660000in}}{\pgfqpoint{4.620000in}{4.620000in}}%
\pgfusepath{clip}%
\pgfsetroundcap%
\pgfsetroundjoin%
\pgfsetlinewidth{1.204500pt}%
\definecolor{currentstroke}{rgb}{0.176471,0.192157,0.258824}%
\pgfsetstrokecolor{currentstroke}%
\pgfsetdash{}{0pt}%
\pgfpathmoveto{\pgfqpoint{3.207247in}{2.749381in}}%
\pgfusepath{stroke}%
\end{pgfscope}%
\begin{pgfscope}%
\pgfpathrectangle{\pgfqpoint{0.765000in}{0.660000in}}{\pgfqpoint{4.620000in}{4.620000in}}%
\pgfusepath{clip}%
\pgfsetbuttcap%
\pgfsetmiterjoin%
\definecolor{currentfill}{rgb}{0.176471,0.192157,0.258824}%
\pgfsetfillcolor{currentfill}%
\pgfsetlinewidth{1.003750pt}%
\definecolor{currentstroke}{rgb}{0.176471,0.192157,0.258824}%
\pgfsetstrokecolor{currentstroke}%
\pgfsetdash{}{0pt}%
\pgfsys@defobject{currentmarker}{\pgfqpoint{-0.033333in}{-0.033333in}}{\pgfqpoint{0.033333in}{0.033333in}}{%
\pgfpathmoveto{\pgfqpoint{-0.000000in}{-0.033333in}}%
\pgfpathlineto{\pgfqpoint{0.033333in}{0.033333in}}%
\pgfpathlineto{\pgfqpoint{-0.033333in}{0.033333in}}%
\pgfpathlineto{\pgfqpoint{-0.000000in}{-0.033333in}}%
\pgfpathclose%
\pgfusepath{stroke,fill}%
}%
\begin{pgfscope}%
\pgfsys@transformshift{3.207247in}{2.749381in}%
\pgfsys@useobject{currentmarker}{}%
\end{pgfscope}%
\end{pgfscope}%
\begin{pgfscope}%
\pgfpathrectangle{\pgfqpoint{0.765000in}{0.660000in}}{\pgfqpoint{4.620000in}{4.620000in}}%
\pgfusepath{clip}%
\pgfsetroundcap%
\pgfsetroundjoin%
\pgfsetlinewidth{1.204500pt}%
\definecolor{currentstroke}{rgb}{0.176471,0.192157,0.258824}%
\pgfsetstrokecolor{currentstroke}%
\pgfsetdash{}{0pt}%
\pgfpathmoveto{\pgfqpoint{3.264975in}{2.906017in}}%
\pgfusepath{stroke}%
\end{pgfscope}%
\begin{pgfscope}%
\pgfpathrectangle{\pgfqpoint{0.765000in}{0.660000in}}{\pgfqpoint{4.620000in}{4.620000in}}%
\pgfusepath{clip}%
\pgfsetbuttcap%
\pgfsetmiterjoin%
\definecolor{currentfill}{rgb}{0.176471,0.192157,0.258824}%
\pgfsetfillcolor{currentfill}%
\pgfsetlinewidth{1.003750pt}%
\definecolor{currentstroke}{rgb}{0.176471,0.192157,0.258824}%
\pgfsetstrokecolor{currentstroke}%
\pgfsetdash{}{0pt}%
\pgfsys@defobject{currentmarker}{\pgfqpoint{-0.033333in}{-0.033333in}}{\pgfqpoint{0.033333in}{0.033333in}}{%
\pgfpathmoveto{\pgfqpoint{-0.000000in}{-0.033333in}}%
\pgfpathlineto{\pgfqpoint{0.033333in}{0.033333in}}%
\pgfpathlineto{\pgfqpoint{-0.033333in}{0.033333in}}%
\pgfpathlineto{\pgfqpoint{-0.000000in}{-0.033333in}}%
\pgfpathclose%
\pgfusepath{stroke,fill}%
}%
\begin{pgfscope}%
\pgfsys@transformshift{3.264975in}{2.906017in}%
\pgfsys@useobject{currentmarker}{}%
\end{pgfscope}%
\end{pgfscope}%
\begin{pgfscope}%
\pgfpathrectangle{\pgfqpoint{0.765000in}{0.660000in}}{\pgfqpoint{4.620000in}{4.620000in}}%
\pgfusepath{clip}%
\pgfsetroundcap%
\pgfsetroundjoin%
\pgfsetlinewidth{1.204500pt}%
\definecolor{currentstroke}{rgb}{0.176471,0.192157,0.258824}%
\pgfsetstrokecolor{currentstroke}%
\pgfsetdash{}{0pt}%
\pgfpathmoveto{\pgfqpoint{3.342286in}{2.667619in}}%
\pgfusepath{stroke}%
\end{pgfscope}%
\begin{pgfscope}%
\pgfpathrectangle{\pgfqpoint{0.765000in}{0.660000in}}{\pgfqpoint{4.620000in}{4.620000in}}%
\pgfusepath{clip}%
\pgfsetbuttcap%
\pgfsetmiterjoin%
\definecolor{currentfill}{rgb}{0.176471,0.192157,0.258824}%
\pgfsetfillcolor{currentfill}%
\pgfsetlinewidth{1.003750pt}%
\definecolor{currentstroke}{rgb}{0.176471,0.192157,0.258824}%
\pgfsetstrokecolor{currentstroke}%
\pgfsetdash{}{0pt}%
\pgfsys@defobject{currentmarker}{\pgfqpoint{-0.033333in}{-0.033333in}}{\pgfqpoint{0.033333in}{0.033333in}}{%
\pgfpathmoveto{\pgfqpoint{-0.000000in}{-0.033333in}}%
\pgfpathlineto{\pgfqpoint{0.033333in}{0.033333in}}%
\pgfpathlineto{\pgfqpoint{-0.033333in}{0.033333in}}%
\pgfpathlineto{\pgfqpoint{-0.000000in}{-0.033333in}}%
\pgfpathclose%
\pgfusepath{stroke,fill}%
}%
\begin{pgfscope}%
\pgfsys@transformshift{3.342286in}{2.667619in}%
\pgfsys@useobject{currentmarker}{}%
\end{pgfscope}%
\end{pgfscope}%
\begin{pgfscope}%
\pgfpathrectangle{\pgfqpoint{0.765000in}{0.660000in}}{\pgfqpoint{4.620000in}{4.620000in}}%
\pgfusepath{clip}%
\pgfsetroundcap%
\pgfsetroundjoin%
\pgfsetlinewidth{1.204500pt}%
\definecolor{currentstroke}{rgb}{0.176471,0.192157,0.258824}%
\pgfsetstrokecolor{currentstroke}%
\pgfsetdash{}{0pt}%
\pgfpathmoveto{\pgfqpoint{3.379910in}{2.707693in}}%
\pgfusepath{stroke}%
\end{pgfscope}%
\begin{pgfscope}%
\pgfpathrectangle{\pgfqpoint{0.765000in}{0.660000in}}{\pgfqpoint{4.620000in}{4.620000in}}%
\pgfusepath{clip}%
\pgfsetbuttcap%
\pgfsetmiterjoin%
\definecolor{currentfill}{rgb}{0.176471,0.192157,0.258824}%
\pgfsetfillcolor{currentfill}%
\pgfsetlinewidth{1.003750pt}%
\definecolor{currentstroke}{rgb}{0.176471,0.192157,0.258824}%
\pgfsetstrokecolor{currentstroke}%
\pgfsetdash{}{0pt}%
\pgfsys@defobject{currentmarker}{\pgfqpoint{-0.033333in}{-0.033333in}}{\pgfqpoint{0.033333in}{0.033333in}}{%
\pgfpathmoveto{\pgfqpoint{-0.000000in}{-0.033333in}}%
\pgfpathlineto{\pgfqpoint{0.033333in}{0.033333in}}%
\pgfpathlineto{\pgfqpoint{-0.033333in}{0.033333in}}%
\pgfpathlineto{\pgfqpoint{-0.000000in}{-0.033333in}}%
\pgfpathclose%
\pgfusepath{stroke,fill}%
}%
\begin{pgfscope}%
\pgfsys@transformshift{3.379910in}{2.707693in}%
\pgfsys@useobject{currentmarker}{}%
\end{pgfscope}%
\end{pgfscope}%
\begin{pgfscope}%
\pgfpathrectangle{\pgfqpoint{0.765000in}{0.660000in}}{\pgfqpoint{4.620000in}{4.620000in}}%
\pgfusepath{clip}%
\pgfsetroundcap%
\pgfsetroundjoin%
\pgfsetlinewidth{1.204500pt}%
\definecolor{currentstroke}{rgb}{0.176471,0.192157,0.258824}%
\pgfsetstrokecolor{currentstroke}%
\pgfsetdash{}{0pt}%
\pgfpathmoveto{\pgfqpoint{3.210327in}{2.781158in}}%
\pgfusepath{stroke}%
\end{pgfscope}%
\begin{pgfscope}%
\pgfpathrectangle{\pgfqpoint{0.765000in}{0.660000in}}{\pgfqpoint{4.620000in}{4.620000in}}%
\pgfusepath{clip}%
\pgfsetbuttcap%
\pgfsetmiterjoin%
\definecolor{currentfill}{rgb}{0.176471,0.192157,0.258824}%
\pgfsetfillcolor{currentfill}%
\pgfsetlinewidth{1.003750pt}%
\definecolor{currentstroke}{rgb}{0.176471,0.192157,0.258824}%
\pgfsetstrokecolor{currentstroke}%
\pgfsetdash{}{0pt}%
\pgfsys@defobject{currentmarker}{\pgfqpoint{-0.033333in}{-0.033333in}}{\pgfqpoint{0.033333in}{0.033333in}}{%
\pgfpathmoveto{\pgfqpoint{-0.000000in}{-0.033333in}}%
\pgfpathlineto{\pgfqpoint{0.033333in}{0.033333in}}%
\pgfpathlineto{\pgfqpoint{-0.033333in}{0.033333in}}%
\pgfpathlineto{\pgfqpoint{-0.000000in}{-0.033333in}}%
\pgfpathclose%
\pgfusepath{stroke,fill}%
}%
\begin{pgfscope}%
\pgfsys@transformshift{3.210327in}{2.781158in}%
\pgfsys@useobject{currentmarker}{}%
\end{pgfscope}%
\end{pgfscope}%
\begin{pgfscope}%
\pgfpathrectangle{\pgfqpoint{0.765000in}{0.660000in}}{\pgfqpoint{4.620000in}{4.620000in}}%
\pgfusepath{clip}%
\pgfsetroundcap%
\pgfsetroundjoin%
\pgfsetlinewidth{1.204500pt}%
\definecolor{currentstroke}{rgb}{0.176471,0.192157,0.258824}%
\pgfsetstrokecolor{currentstroke}%
\pgfsetdash{}{0pt}%
\pgfpathmoveto{\pgfqpoint{3.368092in}{2.769616in}}%
\pgfusepath{stroke}%
\end{pgfscope}%
\begin{pgfscope}%
\pgfpathrectangle{\pgfqpoint{0.765000in}{0.660000in}}{\pgfqpoint{4.620000in}{4.620000in}}%
\pgfusepath{clip}%
\pgfsetbuttcap%
\pgfsetmiterjoin%
\definecolor{currentfill}{rgb}{0.176471,0.192157,0.258824}%
\pgfsetfillcolor{currentfill}%
\pgfsetlinewidth{1.003750pt}%
\definecolor{currentstroke}{rgb}{0.176471,0.192157,0.258824}%
\pgfsetstrokecolor{currentstroke}%
\pgfsetdash{}{0pt}%
\pgfsys@defobject{currentmarker}{\pgfqpoint{-0.033333in}{-0.033333in}}{\pgfqpoint{0.033333in}{0.033333in}}{%
\pgfpathmoveto{\pgfqpoint{-0.000000in}{-0.033333in}}%
\pgfpathlineto{\pgfqpoint{0.033333in}{0.033333in}}%
\pgfpathlineto{\pgfqpoint{-0.033333in}{0.033333in}}%
\pgfpathlineto{\pgfqpoint{-0.000000in}{-0.033333in}}%
\pgfpathclose%
\pgfusepath{stroke,fill}%
}%
\begin{pgfscope}%
\pgfsys@transformshift{3.368092in}{2.769616in}%
\pgfsys@useobject{currentmarker}{}%
\end{pgfscope}%
\end{pgfscope}%
\begin{pgfscope}%
\pgfpathrectangle{\pgfqpoint{0.765000in}{0.660000in}}{\pgfqpoint{4.620000in}{4.620000in}}%
\pgfusepath{clip}%
\pgfsetroundcap%
\pgfsetroundjoin%
\pgfsetlinewidth{1.204500pt}%
\definecolor{currentstroke}{rgb}{0.176471,0.192157,0.258824}%
\pgfsetstrokecolor{currentstroke}%
\pgfsetdash{}{0pt}%
\pgfpathmoveto{\pgfqpoint{3.292526in}{2.674998in}}%
\pgfusepath{stroke}%
\end{pgfscope}%
\begin{pgfscope}%
\pgfpathrectangle{\pgfqpoint{0.765000in}{0.660000in}}{\pgfqpoint{4.620000in}{4.620000in}}%
\pgfusepath{clip}%
\pgfsetbuttcap%
\pgfsetmiterjoin%
\definecolor{currentfill}{rgb}{0.176471,0.192157,0.258824}%
\pgfsetfillcolor{currentfill}%
\pgfsetlinewidth{1.003750pt}%
\definecolor{currentstroke}{rgb}{0.176471,0.192157,0.258824}%
\pgfsetstrokecolor{currentstroke}%
\pgfsetdash{}{0pt}%
\pgfsys@defobject{currentmarker}{\pgfqpoint{-0.033333in}{-0.033333in}}{\pgfqpoint{0.033333in}{0.033333in}}{%
\pgfpathmoveto{\pgfqpoint{-0.000000in}{-0.033333in}}%
\pgfpathlineto{\pgfqpoint{0.033333in}{0.033333in}}%
\pgfpathlineto{\pgfqpoint{-0.033333in}{0.033333in}}%
\pgfpathlineto{\pgfqpoint{-0.000000in}{-0.033333in}}%
\pgfpathclose%
\pgfusepath{stroke,fill}%
}%
\begin{pgfscope}%
\pgfsys@transformshift{3.292526in}{2.674998in}%
\pgfsys@useobject{currentmarker}{}%
\end{pgfscope}%
\end{pgfscope}%
\begin{pgfscope}%
\pgfpathrectangle{\pgfqpoint{0.765000in}{0.660000in}}{\pgfqpoint{4.620000in}{4.620000in}}%
\pgfusepath{clip}%
\pgfsetroundcap%
\pgfsetroundjoin%
\pgfsetlinewidth{1.204500pt}%
\definecolor{currentstroke}{rgb}{0.176471,0.192157,0.258824}%
\pgfsetstrokecolor{currentstroke}%
\pgfsetdash{}{0pt}%
\pgfpathmoveto{\pgfqpoint{3.281469in}{2.909931in}}%
\pgfusepath{stroke}%
\end{pgfscope}%
\begin{pgfscope}%
\pgfpathrectangle{\pgfqpoint{0.765000in}{0.660000in}}{\pgfqpoint{4.620000in}{4.620000in}}%
\pgfusepath{clip}%
\pgfsetbuttcap%
\pgfsetmiterjoin%
\definecolor{currentfill}{rgb}{0.176471,0.192157,0.258824}%
\pgfsetfillcolor{currentfill}%
\pgfsetlinewidth{1.003750pt}%
\definecolor{currentstroke}{rgb}{0.176471,0.192157,0.258824}%
\pgfsetstrokecolor{currentstroke}%
\pgfsetdash{}{0pt}%
\pgfsys@defobject{currentmarker}{\pgfqpoint{-0.033333in}{-0.033333in}}{\pgfqpoint{0.033333in}{0.033333in}}{%
\pgfpathmoveto{\pgfqpoint{-0.000000in}{-0.033333in}}%
\pgfpathlineto{\pgfqpoint{0.033333in}{0.033333in}}%
\pgfpathlineto{\pgfqpoint{-0.033333in}{0.033333in}}%
\pgfpathlineto{\pgfqpoint{-0.000000in}{-0.033333in}}%
\pgfpathclose%
\pgfusepath{stroke,fill}%
}%
\begin{pgfscope}%
\pgfsys@transformshift{3.281469in}{2.909931in}%
\pgfsys@useobject{currentmarker}{}%
\end{pgfscope}%
\end{pgfscope}%
\begin{pgfscope}%
\pgfpathrectangle{\pgfqpoint{0.765000in}{0.660000in}}{\pgfqpoint{4.620000in}{4.620000in}}%
\pgfusepath{clip}%
\pgfsetroundcap%
\pgfsetroundjoin%
\pgfsetlinewidth{1.204500pt}%
\definecolor{currentstroke}{rgb}{0.176471,0.192157,0.258824}%
\pgfsetstrokecolor{currentstroke}%
\pgfsetdash{}{0pt}%
\pgfpathmoveto{\pgfqpoint{3.378368in}{2.892452in}}%
\pgfusepath{stroke}%
\end{pgfscope}%
\begin{pgfscope}%
\pgfpathrectangle{\pgfqpoint{0.765000in}{0.660000in}}{\pgfqpoint{4.620000in}{4.620000in}}%
\pgfusepath{clip}%
\pgfsetbuttcap%
\pgfsetmiterjoin%
\definecolor{currentfill}{rgb}{0.176471,0.192157,0.258824}%
\pgfsetfillcolor{currentfill}%
\pgfsetlinewidth{1.003750pt}%
\definecolor{currentstroke}{rgb}{0.176471,0.192157,0.258824}%
\pgfsetstrokecolor{currentstroke}%
\pgfsetdash{}{0pt}%
\pgfsys@defobject{currentmarker}{\pgfqpoint{-0.033333in}{-0.033333in}}{\pgfqpoint{0.033333in}{0.033333in}}{%
\pgfpathmoveto{\pgfqpoint{-0.000000in}{-0.033333in}}%
\pgfpathlineto{\pgfqpoint{0.033333in}{0.033333in}}%
\pgfpathlineto{\pgfqpoint{-0.033333in}{0.033333in}}%
\pgfpathlineto{\pgfqpoint{-0.000000in}{-0.033333in}}%
\pgfpathclose%
\pgfusepath{stroke,fill}%
}%
\begin{pgfscope}%
\pgfsys@transformshift{3.378368in}{2.892452in}%
\pgfsys@useobject{currentmarker}{}%
\end{pgfscope}%
\end{pgfscope}%
\begin{pgfscope}%
\pgfpathrectangle{\pgfqpoint{0.765000in}{0.660000in}}{\pgfqpoint{4.620000in}{4.620000in}}%
\pgfusepath{clip}%
\pgfsetroundcap%
\pgfsetroundjoin%
\pgfsetlinewidth{1.204500pt}%
\definecolor{currentstroke}{rgb}{0.176471,0.192157,0.258824}%
\pgfsetstrokecolor{currentstroke}%
\pgfsetdash{}{0pt}%
\pgfpathmoveto{\pgfqpoint{3.342307in}{2.723329in}}%
\pgfusepath{stroke}%
\end{pgfscope}%
\begin{pgfscope}%
\pgfpathrectangle{\pgfqpoint{0.765000in}{0.660000in}}{\pgfqpoint{4.620000in}{4.620000in}}%
\pgfusepath{clip}%
\pgfsetbuttcap%
\pgfsetmiterjoin%
\definecolor{currentfill}{rgb}{0.176471,0.192157,0.258824}%
\pgfsetfillcolor{currentfill}%
\pgfsetlinewidth{1.003750pt}%
\definecolor{currentstroke}{rgb}{0.176471,0.192157,0.258824}%
\pgfsetstrokecolor{currentstroke}%
\pgfsetdash{}{0pt}%
\pgfsys@defobject{currentmarker}{\pgfqpoint{-0.033333in}{-0.033333in}}{\pgfqpoint{0.033333in}{0.033333in}}{%
\pgfpathmoveto{\pgfqpoint{-0.000000in}{-0.033333in}}%
\pgfpathlineto{\pgfqpoint{0.033333in}{0.033333in}}%
\pgfpathlineto{\pgfqpoint{-0.033333in}{0.033333in}}%
\pgfpathlineto{\pgfqpoint{-0.000000in}{-0.033333in}}%
\pgfpathclose%
\pgfusepath{stroke,fill}%
}%
\begin{pgfscope}%
\pgfsys@transformshift{3.342307in}{2.723329in}%
\pgfsys@useobject{currentmarker}{}%
\end{pgfscope}%
\end{pgfscope}%
\begin{pgfscope}%
\pgfpathrectangle{\pgfqpoint{0.765000in}{0.660000in}}{\pgfqpoint{4.620000in}{4.620000in}}%
\pgfusepath{clip}%
\pgfsetroundcap%
\pgfsetroundjoin%
\pgfsetlinewidth{1.204500pt}%
\definecolor{currentstroke}{rgb}{0.176471,0.192157,0.258824}%
\pgfsetstrokecolor{currentstroke}%
\pgfsetdash{}{0pt}%
\pgfpathmoveto{\pgfqpoint{3.423586in}{2.788321in}}%
\pgfusepath{stroke}%
\end{pgfscope}%
\begin{pgfscope}%
\pgfpathrectangle{\pgfqpoint{0.765000in}{0.660000in}}{\pgfqpoint{4.620000in}{4.620000in}}%
\pgfusepath{clip}%
\pgfsetbuttcap%
\pgfsetmiterjoin%
\definecolor{currentfill}{rgb}{0.176471,0.192157,0.258824}%
\pgfsetfillcolor{currentfill}%
\pgfsetlinewidth{1.003750pt}%
\definecolor{currentstroke}{rgb}{0.176471,0.192157,0.258824}%
\pgfsetstrokecolor{currentstroke}%
\pgfsetdash{}{0pt}%
\pgfsys@defobject{currentmarker}{\pgfqpoint{-0.033333in}{-0.033333in}}{\pgfqpoint{0.033333in}{0.033333in}}{%
\pgfpathmoveto{\pgfqpoint{-0.000000in}{-0.033333in}}%
\pgfpathlineto{\pgfqpoint{0.033333in}{0.033333in}}%
\pgfpathlineto{\pgfqpoint{-0.033333in}{0.033333in}}%
\pgfpathlineto{\pgfqpoint{-0.000000in}{-0.033333in}}%
\pgfpathclose%
\pgfusepath{stroke,fill}%
}%
\begin{pgfscope}%
\pgfsys@transformshift{3.423586in}{2.788321in}%
\pgfsys@useobject{currentmarker}{}%
\end{pgfscope}%
\end{pgfscope}%
\begin{pgfscope}%
\pgfpathrectangle{\pgfqpoint{0.765000in}{0.660000in}}{\pgfqpoint{4.620000in}{4.620000in}}%
\pgfusepath{clip}%
\pgfsetroundcap%
\pgfsetroundjoin%
\pgfsetlinewidth{1.204500pt}%
\definecolor{currentstroke}{rgb}{0.176471,0.192157,0.258824}%
\pgfsetstrokecolor{currentstroke}%
\pgfsetdash{}{0pt}%
\pgfpathmoveto{\pgfqpoint{3.235425in}{2.725119in}}%
\pgfusepath{stroke}%
\end{pgfscope}%
\begin{pgfscope}%
\pgfpathrectangle{\pgfqpoint{0.765000in}{0.660000in}}{\pgfqpoint{4.620000in}{4.620000in}}%
\pgfusepath{clip}%
\pgfsetbuttcap%
\pgfsetmiterjoin%
\definecolor{currentfill}{rgb}{0.176471,0.192157,0.258824}%
\pgfsetfillcolor{currentfill}%
\pgfsetlinewidth{1.003750pt}%
\definecolor{currentstroke}{rgb}{0.176471,0.192157,0.258824}%
\pgfsetstrokecolor{currentstroke}%
\pgfsetdash{}{0pt}%
\pgfsys@defobject{currentmarker}{\pgfqpoint{-0.033333in}{-0.033333in}}{\pgfqpoint{0.033333in}{0.033333in}}{%
\pgfpathmoveto{\pgfqpoint{-0.000000in}{-0.033333in}}%
\pgfpathlineto{\pgfqpoint{0.033333in}{0.033333in}}%
\pgfpathlineto{\pgfqpoint{-0.033333in}{0.033333in}}%
\pgfpathlineto{\pgfqpoint{-0.000000in}{-0.033333in}}%
\pgfpathclose%
\pgfusepath{stroke,fill}%
}%
\begin{pgfscope}%
\pgfsys@transformshift{3.235425in}{2.725119in}%
\pgfsys@useobject{currentmarker}{}%
\end{pgfscope}%
\end{pgfscope}%
\begin{pgfscope}%
\pgfpathrectangle{\pgfqpoint{0.765000in}{0.660000in}}{\pgfqpoint{4.620000in}{4.620000in}}%
\pgfusepath{clip}%
\pgfsetroundcap%
\pgfsetroundjoin%
\pgfsetlinewidth{1.204500pt}%
\definecolor{currentstroke}{rgb}{0.176471,0.192157,0.258824}%
\pgfsetstrokecolor{currentstroke}%
\pgfsetdash{}{0pt}%
\pgfpathmoveto{\pgfqpoint{3.222573in}{2.858341in}}%
\pgfusepath{stroke}%
\end{pgfscope}%
\begin{pgfscope}%
\pgfpathrectangle{\pgfqpoint{0.765000in}{0.660000in}}{\pgfqpoint{4.620000in}{4.620000in}}%
\pgfusepath{clip}%
\pgfsetbuttcap%
\pgfsetmiterjoin%
\definecolor{currentfill}{rgb}{0.176471,0.192157,0.258824}%
\pgfsetfillcolor{currentfill}%
\pgfsetlinewidth{1.003750pt}%
\definecolor{currentstroke}{rgb}{0.176471,0.192157,0.258824}%
\pgfsetstrokecolor{currentstroke}%
\pgfsetdash{}{0pt}%
\pgfsys@defobject{currentmarker}{\pgfqpoint{-0.033333in}{-0.033333in}}{\pgfqpoint{0.033333in}{0.033333in}}{%
\pgfpathmoveto{\pgfqpoint{-0.000000in}{-0.033333in}}%
\pgfpathlineto{\pgfqpoint{0.033333in}{0.033333in}}%
\pgfpathlineto{\pgfqpoint{-0.033333in}{0.033333in}}%
\pgfpathlineto{\pgfqpoint{-0.000000in}{-0.033333in}}%
\pgfpathclose%
\pgfusepath{stroke,fill}%
}%
\begin{pgfscope}%
\pgfsys@transformshift{3.222573in}{2.858341in}%
\pgfsys@useobject{currentmarker}{}%
\end{pgfscope}%
\end{pgfscope}%
\begin{pgfscope}%
\pgfpathrectangle{\pgfqpoint{0.765000in}{0.660000in}}{\pgfqpoint{4.620000in}{4.620000in}}%
\pgfusepath{clip}%
\pgfsetroundcap%
\pgfsetroundjoin%
\pgfsetlinewidth{1.204500pt}%
\definecolor{currentstroke}{rgb}{0.176471,0.192157,0.258824}%
\pgfsetstrokecolor{currentstroke}%
\pgfsetdash{}{0pt}%
\pgfpathmoveto{\pgfqpoint{3.305103in}{2.691244in}}%
\pgfusepath{stroke}%
\end{pgfscope}%
\begin{pgfscope}%
\pgfpathrectangle{\pgfqpoint{0.765000in}{0.660000in}}{\pgfqpoint{4.620000in}{4.620000in}}%
\pgfusepath{clip}%
\pgfsetbuttcap%
\pgfsetmiterjoin%
\definecolor{currentfill}{rgb}{0.176471,0.192157,0.258824}%
\pgfsetfillcolor{currentfill}%
\pgfsetlinewidth{1.003750pt}%
\definecolor{currentstroke}{rgb}{0.176471,0.192157,0.258824}%
\pgfsetstrokecolor{currentstroke}%
\pgfsetdash{}{0pt}%
\pgfsys@defobject{currentmarker}{\pgfqpoint{-0.033333in}{-0.033333in}}{\pgfqpoint{0.033333in}{0.033333in}}{%
\pgfpathmoveto{\pgfqpoint{-0.000000in}{-0.033333in}}%
\pgfpathlineto{\pgfqpoint{0.033333in}{0.033333in}}%
\pgfpathlineto{\pgfqpoint{-0.033333in}{0.033333in}}%
\pgfpathlineto{\pgfqpoint{-0.000000in}{-0.033333in}}%
\pgfpathclose%
\pgfusepath{stroke,fill}%
}%
\begin{pgfscope}%
\pgfsys@transformshift{3.305103in}{2.691244in}%
\pgfsys@useobject{currentmarker}{}%
\end{pgfscope}%
\end{pgfscope}%
\begin{pgfscope}%
\pgfpathrectangle{\pgfqpoint{0.765000in}{0.660000in}}{\pgfqpoint{4.620000in}{4.620000in}}%
\pgfusepath{clip}%
\pgfsetroundcap%
\pgfsetroundjoin%
\pgfsetlinewidth{1.204500pt}%
\definecolor{currentstroke}{rgb}{0.176471,0.192157,0.258824}%
\pgfsetstrokecolor{currentstroke}%
\pgfsetdash{}{0pt}%
\pgfpathmoveto{\pgfqpoint{3.308157in}{2.771885in}}%
\pgfusepath{stroke}%
\end{pgfscope}%
\begin{pgfscope}%
\pgfpathrectangle{\pgfqpoint{0.765000in}{0.660000in}}{\pgfqpoint{4.620000in}{4.620000in}}%
\pgfusepath{clip}%
\pgfsetbuttcap%
\pgfsetmiterjoin%
\definecolor{currentfill}{rgb}{0.176471,0.192157,0.258824}%
\pgfsetfillcolor{currentfill}%
\pgfsetlinewidth{1.003750pt}%
\definecolor{currentstroke}{rgb}{0.176471,0.192157,0.258824}%
\pgfsetstrokecolor{currentstroke}%
\pgfsetdash{}{0pt}%
\pgfsys@defobject{currentmarker}{\pgfqpoint{-0.033333in}{-0.033333in}}{\pgfqpoint{0.033333in}{0.033333in}}{%
\pgfpathmoveto{\pgfqpoint{-0.000000in}{-0.033333in}}%
\pgfpathlineto{\pgfqpoint{0.033333in}{0.033333in}}%
\pgfpathlineto{\pgfqpoint{-0.033333in}{0.033333in}}%
\pgfpathlineto{\pgfqpoint{-0.000000in}{-0.033333in}}%
\pgfpathclose%
\pgfusepath{stroke,fill}%
}%
\begin{pgfscope}%
\pgfsys@transformshift{3.308157in}{2.771885in}%
\pgfsys@useobject{currentmarker}{}%
\end{pgfscope}%
\end{pgfscope}%
\begin{pgfscope}%
\pgfpathrectangle{\pgfqpoint{0.765000in}{0.660000in}}{\pgfqpoint{4.620000in}{4.620000in}}%
\pgfusepath{clip}%
\pgfsetroundcap%
\pgfsetroundjoin%
\pgfsetlinewidth{1.204500pt}%
\definecolor{currentstroke}{rgb}{0.176471,0.192157,0.258824}%
\pgfsetstrokecolor{currentstroke}%
\pgfsetdash{}{0pt}%
\pgfpathmoveto{\pgfqpoint{3.189078in}{2.838139in}}%
\pgfusepath{stroke}%
\end{pgfscope}%
\begin{pgfscope}%
\pgfpathrectangle{\pgfqpoint{0.765000in}{0.660000in}}{\pgfqpoint{4.620000in}{4.620000in}}%
\pgfusepath{clip}%
\pgfsetbuttcap%
\pgfsetmiterjoin%
\definecolor{currentfill}{rgb}{0.176471,0.192157,0.258824}%
\pgfsetfillcolor{currentfill}%
\pgfsetlinewidth{1.003750pt}%
\definecolor{currentstroke}{rgb}{0.176471,0.192157,0.258824}%
\pgfsetstrokecolor{currentstroke}%
\pgfsetdash{}{0pt}%
\pgfsys@defobject{currentmarker}{\pgfqpoint{-0.033333in}{-0.033333in}}{\pgfqpoint{0.033333in}{0.033333in}}{%
\pgfpathmoveto{\pgfqpoint{-0.000000in}{-0.033333in}}%
\pgfpathlineto{\pgfqpoint{0.033333in}{0.033333in}}%
\pgfpathlineto{\pgfqpoint{-0.033333in}{0.033333in}}%
\pgfpathlineto{\pgfqpoint{-0.000000in}{-0.033333in}}%
\pgfpathclose%
\pgfusepath{stroke,fill}%
}%
\begin{pgfscope}%
\pgfsys@transformshift{3.189078in}{2.838139in}%
\pgfsys@useobject{currentmarker}{}%
\end{pgfscope}%
\end{pgfscope}%
\begin{pgfscope}%
\pgfpathrectangle{\pgfqpoint{0.765000in}{0.660000in}}{\pgfqpoint{4.620000in}{4.620000in}}%
\pgfusepath{clip}%
\pgfsetroundcap%
\pgfsetroundjoin%
\pgfsetlinewidth{1.204500pt}%
\definecolor{currentstroke}{rgb}{0.176471,0.192157,0.258824}%
\pgfsetstrokecolor{currentstroke}%
\pgfsetdash{}{0pt}%
\pgfpathmoveto{\pgfqpoint{3.264037in}{2.838744in}}%
\pgfusepath{stroke}%
\end{pgfscope}%
\begin{pgfscope}%
\pgfpathrectangle{\pgfqpoint{0.765000in}{0.660000in}}{\pgfqpoint{4.620000in}{4.620000in}}%
\pgfusepath{clip}%
\pgfsetbuttcap%
\pgfsetmiterjoin%
\definecolor{currentfill}{rgb}{0.176471,0.192157,0.258824}%
\pgfsetfillcolor{currentfill}%
\pgfsetlinewidth{1.003750pt}%
\definecolor{currentstroke}{rgb}{0.176471,0.192157,0.258824}%
\pgfsetstrokecolor{currentstroke}%
\pgfsetdash{}{0pt}%
\pgfsys@defobject{currentmarker}{\pgfqpoint{-0.033333in}{-0.033333in}}{\pgfqpoint{0.033333in}{0.033333in}}{%
\pgfpathmoveto{\pgfqpoint{-0.000000in}{-0.033333in}}%
\pgfpathlineto{\pgfqpoint{0.033333in}{0.033333in}}%
\pgfpathlineto{\pgfqpoint{-0.033333in}{0.033333in}}%
\pgfpathlineto{\pgfqpoint{-0.000000in}{-0.033333in}}%
\pgfpathclose%
\pgfusepath{stroke,fill}%
}%
\begin{pgfscope}%
\pgfsys@transformshift{3.264037in}{2.838744in}%
\pgfsys@useobject{currentmarker}{}%
\end{pgfscope}%
\end{pgfscope}%
\begin{pgfscope}%
\pgfpathrectangle{\pgfqpoint{0.765000in}{0.660000in}}{\pgfqpoint{4.620000in}{4.620000in}}%
\pgfusepath{clip}%
\pgfsetroundcap%
\pgfsetroundjoin%
\pgfsetlinewidth{1.204500pt}%
\definecolor{currentstroke}{rgb}{0.176471,0.192157,0.258824}%
\pgfsetstrokecolor{currentstroke}%
\pgfsetdash{}{0pt}%
\pgfpathmoveto{\pgfqpoint{3.363135in}{2.789868in}}%
\pgfusepath{stroke}%
\end{pgfscope}%
\begin{pgfscope}%
\pgfpathrectangle{\pgfqpoint{0.765000in}{0.660000in}}{\pgfqpoint{4.620000in}{4.620000in}}%
\pgfusepath{clip}%
\pgfsetbuttcap%
\pgfsetmiterjoin%
\definecolor{currentfill}{rgb}{0.176471,0.192157,0.258824}%
\pgfsetfillcolor{currentfill}%
\pgfsetlinewidth{1.003750pt}%
\definecolor{currentstroke}{rgb}{0.176471,0.192157,0.258824}%
\pgfsetstrokecolor{currentstroke}%
\pgfsetdash{}{0pt}%
\pgfsys@defobject{currentmarker}{\pgfqpoint{-0.033333in}{-0.033333in}}{\pgfqpoint{0.033333in}{0.033333in}}{%
\pgfpathmoveto{\pgfqpoint{-0.000000in}{-0.033333in}}%
\pgfpathlineto{\pgfqpoint{0.033333in}{0.033333in}}%
\pgfpathlineto{\pgfqpoint{-0.033333in}{0.033333in}}%
\pgfpathlineto{\pgfqpoint{-0.000000in}{-0.033333in}}%
\pgfpathclose%
\pgfusepath{stroke,fill}%
}%
\begin{pgfscope}%
\pgfsys@transformshift{3.363135in}{2.789868in}%
\pgfsys@useobject{currentmarker}{}%
\end{pgfscope}%
\end{pgfscope}%
\begin{pgfscope}%
\pgfpathrectangle{\pgfqpoint{0.765000in}{0.660000in}}{\pgfqpoint{4.620000in}{4.620000in}}%
\pgfusepath{clip}%
\pgfsetroundcap%
\pgfsetroundjoin%
\pgfsetlinewidth{1.204500pt}%
\definecolor{currentstroke}{rgb}{0.176471,0.192157,0.258824}%
\pgfsetstrokecolor{currentstroke}%
\pgfsetdash{}{0pt}%
\pgfpathmoveto{\pgfqpoint{3.355894in}{2.884238in}}%
\pgfusepath{stroke}%
\end{pgfscope}%
\begin{pgfscope}%
\pgfpathrectangle{\pgfqpoint{0.765000in}{0.660000in}}{\pgfqpoint{4.620000in}{4.620000in}}%
\pgfusepath{clip}%
\pgfsetbuttcap%
\pgfsetmiterjoin%
\definecolor{currentfill}{rgb}{0.176471,0.192157,0.258824}%
\pgfsetfillcolor{currentfill}%
\pgfsetlinewidth{1.003750pt}%
\definecolor{currentstroke}{rgb}{0.176471,0.192157,0.258824}%
\pgfsetstrokecolor{currentstroke}%
\pgfsetdash{}{0pt}%
\pgfsys@defobject{currentmarker}{\pgfqpoint{-0.033333in}{-0.033333in}}{\pgfqpoint{0.033333in}{0.033333in}}{%
\pgfpathmoveto{\pgfqpoint{-0.000000in}{-0.033333in}}%
\pgfpathlineto{\pgfqpoint{0.033333in}{0.033333in}}%
\pgfpathlineto{\pgfqpoint{-0.033333in}{0.033333in}}%
\pgfpathlineto{\pgfqpoint{-0.000000in}{-0.033333in}}%
\pgfpathclose%
\pgfusepath{stroke,fill}%
}%
\begin{pgfscope}%
\pgfsys@transformshift{3.355894in}{2.884238in}%
\pgfsys@useobject{currentmarker}{}%
\end{pgfscope}%
\end{pgfscope}%
\begin{pgfscope}%
\pgfpathrectangle{\pgfqpoint{0.765000in}{0.660000in}}{\pgfqpoint{4.620000in}{4.620000in}}%
\pgfusepath{clip}%
\pgfsetroundcap%
\pgfsetroundjoin%
\pgfsetlinewidth{1.204500pt}%
\definecolor{currentstroke}{rgb}{0.176471,0.192157,0.258824}%
\pgfsetstrokecolor{currentstroke}%
\pgfsetdash{}{0pt}%
\pgfpathmoveto{\pgfqpoint{3.321774in}{2.703257in}}%
\pgfusepath{stroke}%
\end{pgfscope}%
\begin{pgfscope}%
\pgfpathrectangle{\pgfqpoint{0.765000in}{0.660000in}}{\pgfqpoint{4.620000in}{4.620000in}}%
\pgfusepath{clip}%
\pgfsetbuttcap%
\pgfsetmiterjoin%
\definecolor{currentfill}{rgb}{0.176471,0.192157,0.258824}%
\pgfsetfillcolor{currentfill}%
\pgfsetlinewidth{1.003750pt}%
\definecolor{currentstroke}{rgb}{0.176471,0.192157,0.258824}%
\pgfsetstrokecolor{currentstroke}%
\pgfsetdash{}{0pt}%
\pgfsys@defobject{currentmarker}{\pgfqpoint{-0.033333in}{-0.033333in}}{\pgfqpoint{0.033333in}{0.033333in}}{%
\pgfpathmoveto{\pgfqpoint{-0.000000in}{-0.033333in}}%
\pgfpathlineto{\pgfqpoint{0.033333in}{0.033333in}}%
\pgfpathlineto{\pgfqpoint{-0.033333in}{0.033333in}}%
\pgfpathlineto{\pgfqpoint{-0.000000in}{-0.033333in}}%
\pgfpathclose%
\pgfusepath{stroke,fill}%
}%
\begin{pgfscope}%
\pgfsys@transformshift{3.321774in}{2.703257in}%
\pgfsys@useobject{currentmarker}{}%
\end{pgfscope}%
\end{pgfscope}%
\begin{pgfscope}%
\pgfpathrectangle{\pgfqpoint{0.765000in}{0.660000in}}{\pgfqpoint{4.620000in}{4.620000in}}%
\pgfusepath{clip}%
\pgfsetroundcap%
\pgfsetroundjoin%
\pgfsetlinewidth{1.204500pt}%
\definecolor{currentstroke}{rgb}{0.176471,0.192157,0.258824}%
\pgfsetstrokecolor{currentstroke}%
\pgfsetdash{}{0pt}%
\pgfpathmoveto{\pgfqpoint{3.318122in}{2.703411in}}%
\pgfusepath{stroke}%
\end{pgfscope}%
\begin{pgfscope}%
\pgfpathrectangle{\pgfqpoint{0.765000in}{0.660000in}}{\pgfqpoint{4.620000in}{4.620000in}}%
\pgfusepath{clip}%
\pgfsetbuttcap%
\pgfsetmiterjoin%
\definecolor{currentfill}{rgb}{0.176471,0.192157,0.258824}%
\pgfsetfillcolor{currentfill}%
\pgfsetlinewidth{1.003750pt}%
\definecolor{currentstroke}{rgb}{0.176471,0.192157,0.258824}%
\pgfsetstrokecolor{currentstroke}%
\pgfsetdash{}{0pt}%
\pgfsys@defobject{currentmarker}{\pgfqpoint{-0.033333in}{-0.033333in}}{\pgfqpoint{0.033333in}{0.033333in}}{%
\pgfpathmoveto{\pgfqpoint{-0.000000in}{-0.033333in}}%
\pgfpathlineto{\pgfqpoint{0.033333in}{0.033333in}}%
\pgfpathlineto{\pgfqpoint{-0.033333in}{0.033333in}}%
\pgfpathlineto{\pgfqpoint{-0.000000in}{-0.033333in}}%
\pgfpathclose%
\pgfusepath{stroke,fill}%
}%
\begin{pgfscope}%
\pgfsys@transformshift{3.318122in}{2.703411in}%
\pgfsys@useobject{currentmarker}{}%
\end{pgfscope}%
\end{pgfscope}%
\begin{pgfscope}%
\pgfpathrectangle{\pgfqpoint{0.765000in}{0.660000in}}{\pgfqpoint{4.620000in}{4.620000in}}%
\pgfusepath{clip}%
\pgfsetroundcap%
\pgfsetroundjoin%
\pgfsetlinewidth{1.204500pt}%
\definecolor{currentstroke}{rgb}{0.176471,0.192157,0.258824}%
\pgfsetstrokecolor{currentstroke}%
\pgfsetdash{}{0pt}%
\pgfpathmoveto{\pgfqpoint{3.354339in}{2.922945in}}%
\pgfusepath{stroke}%
\end{pgfscope}%
\begin{pgfscope}%
\pgfpathrectangle{\pgfqpoint{0.765000in}{0.660000in}}{\pgfqpoint{4.620000in}{4.620000in}}%
\pgfusepath{clip}%
\pgfsetbuttcap%
\pgfsetmiterjoin%
\definecolor{currentfill}{rgb}{0.176471,0.192157,0.258824}%
\pgfsetfillcolor{currentfill}%
\pgfsetlinewidth{1.003750pt}%
\definecolor{currentstroke}{rgb}{0.176471,0.192157,0.258824}%
\pgfsetstrokecolor{currentstroke}%
\pgfsetdash{}{0pt}%
\pgfsys@defobject{currentmarker}{\pgfqpoint{-0.033333in}{-0.033333in}}{\pgfqpoint{0.033333in}{0.033333in}}{%
\pgfpathmoveto{\pgfqpoint{-0.000000in}{-0.033333in}}%
\pgfpathlineto{\pgfqpoint{0.033333in}{0.033333in}}%
\pgfpathlineto{\pgfqpoint{-0.033333in}{0.033333in}}%
\pgfpathlineto{\pgfqpoint{-0.000000in}{-0.033333in}}%
\pgfpathclose%
\pgfusepath{stroke,fill}%
}%
\begin{pgfscope}%
\pgfsys@transformshift{3.354339in}{2.922945in}%
\pgfsys@useobject{currentmarker}{}%
\end{pgfscope}%
\end{pgfscope}%
\begin{pgfscope}%
\pgfpathrectangle{\pgfqpoint{0.765000in}{0.660000in}}{\pgfqpoint{4.620000in}{4.620000in}}%
\pgfusepath{clip}%
\pgfsetroundcap%
\pgfsetroundjoin%
\pgfsetlinewidth{1.204500pt}%
\definecolor{currentstroke}{rgb}{0.176471,0.192157,0.258824}%
\pgfsetstrokecolor{currentstroke}%
\pgfsetdash{}{0pt}%
\pgfpathmoveto{\pgfqpoint{3.258967in}{2.760745in}}%
\pgfusepath{stroke}%
\end{pgfscope}%
\begin{pgfscope}%
\pgfpathrectangle{\pgfqpoint{0.765000in}{0.660000in}}{\pgfqpoint{4.620000in}{4.620000in}}%
\pgfusepath{clip}%
\pgfsetbuttcap%
\pgfsetmiterjoin%
\definecolor{currentfill}{rgb}{0.176471,0.192157,0.258824}%
\pgfsetfillcolor{currentfill}%
\pgfsetlinewidth{1.003750pt}%
\definecolor{currentstroke}{rgb}{0.176471,0.192157,0.258824}%
\pgfsetstrokecolor{currentstroke}%
\pgfsetdash{}{0pt}%
\pgfsys@defobject{currentmarker}{\pgfqpoint{-0.033333in}{-0.033333in}}{\pgfqpoint{0.033333in}{0.033333in}}{%
\pgfpathmoveto{\pgfqpoint{-0.000000in}{-0.033333in}}%
\pgfpathlineto{\pgfqpoint{0.033333in}{0.033333in}}%
\pgfpathlineto{\pgfqpoint{-0.033333in}{0.033333in}}%
\pgfpathlineto{\pgfqpoint{-0.000000in}{-0.033333in}}%
\pgfpathclose%
\pgfusepath{stroke,fill}%
}%
\begin{pgfscope}%
\pgfsys@transformshift{3.258967in}{2.760745in}%
\pgfsys@useobject{currentmarker}{}%
\end{pgfscope}%
\end{pgfscope}%
\begin{pgfscope}%
\pgfpathrectangle{\pgfqpoint{0.765000in}{0.660000in}}{\pgfqpoint{4.620000in}{4.620000in}}%
\pgfusepath{clip}%
\pgfsetroundcap%
\pgfsetroundjoin%
\pgfsetlinewidth{1.204500pt}%
\definecolor{currentstroke}{rgb}{0.176471,0.192157,0.258824}%
\pgfsetstrokecolor{currentstroke}%
\pgfsetdash{}{0pt}%
\pgfpathmoveto{\pgfqpoint{3.388578in}{2.875349in}}%
\pgfusepath{stroke}%
\end{pgfscope}%
\begin{pgfscope}%
\pgfpathrectangle{\pgfqpoint{0.765000in}{0.660000in}}{\pgfqpoint{4.620000in}{4.620000in}}%
\pgfusepath{clip}%
\pgfsetbuttcap%
\pgfsetmiterjoin%
\definecolor{currentfill}{rgb}{0.176471,0.192157,0.258824}%
\pgfsetfillcolor{currentfill}%
\pgfsetlinewidth{1.003750pt}%
\definecolor{currentstroke}{rgb}{0.176471,0.192157,0.258824}%
\pgfsetstrokecolor{currentstroke}%
\pgfsetdash{}{0pt}%
\pgfsys@defobject{currentmarker}{\pgfqpoint{-0.033333in}{-0.033333in}}{\pgfqpoint{0.033333in}{0.033333in}}{%
\pgfpathmoveto{\pgfqpoint{-0.000000in}{-0.033333in}}%
\pgfpathlineto{\pgfqpoint{0.033333in}{0.033333in}}%
\pgfpathlineto{\pgfqpoint{-0.033333in}{0.033333in}}%
\pgfpathlineto{\pgfqpoint{-0.000000in}{-0.033333in}}%
\pgfpathclose%
\pgfusepath{stroke,fill}%
}%
\begin{pgfscope}%
\pgfsys@transformshift{3.388578in}{2.875349in}%
\pgfsys@useobject{currentmarker}{}%
\end{pgfscope}%
\end{pgfscope}%
\begin{pgfscope}%
\pgfpathrectangle{\pgfqpoint{0.765000in}{0.660000in}}{\pgfqpoint{4.620000in}{4.620000in}}%
\pgfusepath{clip}%
\pgfsetroundcap%
\pgfsetroundjoin%
\pgfsetlinewidth{1.204500pt}%
\definecolor{currentstroke}{rgb}{0.176471,0.192157,0.258824}%
\pgfsetstrokecolor{currentstroke}%
\pgfsetdash{}{0pt}%
\pgfpathmoveto{\pgfqpoint{3.369016in}{2.745978in}}%
\pgfusepath{stroke}%
\end{pgfscope}%
\begin{pgfscope}%
\pgfpathrectangle{\pgfqpoint{0.765000in}{0.660000in}}{\pgfqpoint{4.620000in}{4.620000in}}%
\pgfusepath{clip}%
\pgfsetbuttcap%
\pgfsetmiterjoin%
\definecolor{currentfill}{rgb}{0.176471,0.192157,0.258824}%
\pgfsetfillcolor{currentfill}%
\pgfsetlinewidth{1.003750pt}%
\definecolor{currentstroke}{rgb}{0.176471,0.192157,0.258824}%
\pgfsetstrokecolor{currentstroke}%
\pgfsetdash{}{0pt}%
\pgfsys@defobject{currentmarker}{\pgfqpoint{-0.033333in}{-0.033333in}}{\pgfqpoint{0.033333in}{0.033333in}}{%
\pgfpathmoveto{\pgfqpoint{-0.000000in}{-0.033333in}}%
\pgfpathlineto{\pgfqpoint{0.033333in}{0.033333in}}%
\pgfpathlineto{\pgfqpoint{-0.033333in}{0.033333in}}%
\pgfpathlineto{\pgfqpoint{-0.000000in}{-0.033333in}}%
\pgfpathclose%
\pgfusepath{stroke,fill}%
}%
\begin{pgfscope}%
\pgfsys@transformshift{3.369016in}{2.745978in}%
\pgfsys@useobject{currentmarker}{}%
\end{pgfscope}%
\end{pgfscope}%
\begin{pgfscope}%
\pgfpathrectangle{\pgfqpoint{0.765000in}{0.660000in}}{\pgfqpoint{4.620000in}{4.620000in}}%
\pgfusepath{clip}%
\pgfsetroundcap%
\pgfsetroundjoin%
\pgfsetlinewidth{1.204500pt}%
\definecolor{currentstroke}{rgb}{0.176471,0.192157,0.258824}%
\pgfsetstrokecolor{currentstroke}%
\pgfsetdash{}{0pt}%
\pgfpathmoveto{\pgfqpoint{3.238431in}{2.855433in}}%
\pgfusepath{stroke}%
\end{pgfscope}%
\begin{pgfscope}%
\pgfpathrectangle{\pgfqpoint{0.765000in}{0.660000in}}{\pgfqpoint{4.620000in}{4.620000in}}%
\pgfusepath{clip}%
\pgfsetbuttcap%
\pgfsetmiterjoin%
\definecolor{currentfill}{rgb}{0.176471,0.192157,0.258824}%
\pgfsetfillcolor{currentfill}%
\pgfsetlinewidth{1.003750pt}%
\definecolor{currentstroke}{rgb}{0.176471,0.192157,0.258824}%
\pgfsetstrokecolor{currentstroke}%
\pgfsetdash{}{0pt}%
\pgfsys@defobject{currentmarker}{\pgfqpoint{-0.033333in}{-0.033333in}}{\pgfqpoint{0.033333in}{0.033333in}}{%
\pgfpathmoveto{\pgfqpoint{-0.000000in}{-0.033333in}}%
\pgfpathlineto{\pgfqpoint{0.033333in}{0.033333in}}%
\pgfpathlineto{\pgfqpoint{-0.033333in}{0.033333in}}%
\pgfpathlineto{\pgfqpoint{-0.000000in}{-0.033333in}}%
\pgfpathclose%
\pgfusepath{stroke,fill}%
}%
\begin{pgfscope}%
\pgfsys@transformshift{3.238431in}{2.855433in}%
\pgfsys@useobject{currentmarker}{}%
\end{pgfscope}%
\end{pgfscope}%
\begin{pgfscope}%
\pgfpathrectangle{\pgfqpoint{0.765000in}{0.660000in}}{\pgfqpoint{4.620000in}{4.620000in}}%
\pgfusepath{clip}%
\pgfsetroundcap%
\pgfsetroundjoin%
\pgfsetlinewidth{1.204500pt}%
\definecolor{currentstroke}{rgb}{0.176471,0.192157,0.258824}%
\pgfsetstrokecolor{currentstroke}%
\pgfsetdash{}{0pt}%
\pgfpathmoveto{\pgfqpoint{3.255047in}{2.696615in}}%
\pgfusepath{stroke}%
\end{pgfscope}%
\begin{pgfscope}%
\pgfpathrectangle{\pgfqpoint{0.765000in}{0.660000in}}{\pgfqpoint{4.620000in}{4.620000in}}%
\pgfusepath{clip}%
\pgfsetbuttcap%
\pgfsetmiterjoin%
\definecolor{currentfill}{rgb}{0.176471,0.192157,0.258824}%
\pgfsetfillcolor{currentfill}%
\pgfsetlinewidth{1.003750pt}%
\definecolor{currentstroke}{rgb}{0.176471,0.192157,0.258824}%
\pgfsetstrokecolor{currentstroke}%
\pgfsetdash{}{0pt}%
\pgfsys@defobject{currentmarker}{\pgfqpoint{-0.033333in}{-0.033333in}}{\pgfqpoint{0.033333in}{0.033333in}}{%
\pgfpathmoveto{\pgfqpoint{-0.000000in}{-0.033333in}}%
\pgfpathlineto{\pgfqpoint{0.033333in}{0.033333in}}%
\pgfpathlineto{\pgfqpoint{-0.033333in}{0.033333in}}%
\pgfpathlineto{\pgfqpoint{-0.000000in}{-0.033333in}}%
\pgfpathclose%
\pgfusepath{stroke,fill}%
}%
\begin{pgfscope}%
\pgfsys@transformshift{3.255047in}{2.696615in}%
\pgfsys@useobject{currentmarker}{}%
\end{pgfscope}%
\end{pgfscope}%
\begin{pgfscope}%
\pgfpathrectangle{\pgfqpoint{0.765000in}{0.660000in}}{\pgfqpoint{4.620000in}{4.620000in}}%
\pgfusepath{clip}%
\pgfsetroundcap%
\pgfsetroundjoin%
\pgfsetlinewidth{1.204500pt}%
\definecolor{currentstroke}{rgb}{0.176471,0.192157,0.258824}%
\pgfsetstrokecolor{currentstroke}%
\pgfsetdash{}{0pt}%
\pgfpathmoveto{\pgfqpoint{3.160076in}{2.865769in}}%
\pgfusepath{stroke}%
\end{pgfscope}%
\begin{pgfscope}%
\pgfpathrectangle{\pgfqpoint{0.765000in}{0.660000in}}{\pgfqpoint{4.620000in}{4.620000in}}%
\pgfusepath{clip}%
\pgfsetbuttcap%
\pgfsetmiterjoin%
\definecolor{currentfill}{rgb}{0.176471,0.192157,0.258824}%
\pgfsetfillcolor{currentfill}%
\pgfsetlinewidth{1.003750pt}%
\definecolor{currentstroke}{rgb}{0.176471,0.192157,0.258824}%
\pgfsetstrokecolor{currentstroke}%
\pgfsetdash{}{0pt}%
\pgfsys@defobject{currentmarker}{\pgfqpoint{-0.033333in}{-0.033333in}}{\pgfqpoint{0.033333in}{0.033333in}}{%
\pgfpathmoveto{\pgfqpoint{-0.000000in}{-0.033333in}}%
\pgfpathlineto{\pgfqpoint{0.033333in}{0.033333in}}%
\pgfpathlineto{\pgfqpoint{-0.033333in}{0.033333in}}%
\pgfpathlineto{\pgfqpoint{-0.000000in}{-0.033333in}}%
\pgfpathclose%
\pgfusepath{stroke,fill}%
}%
\begin{pgfscope}%
\pgfsys@transformshift{3.160076in}{2.865769in}%
\pgfsys@useobject{currentmarker}{}%
\end{pgfscope}%
\end{pgfscope}%
\begin{pgfscope}%
\pgfpathrectangle{\pgfqpoint{0.765000in}{0.660000in}}{\pgfqpoint{4.620000in}{4.620000in}}%
\pgfusepath{clip}%
\pgfsetroundcap%
\pgfsetroundjoin%
\pgfsetlinewidth{1.204500pt}%
\definecolor{currentstroke}{rgb}{0.176471,0.192157,0.258824}%
\pgfsetstrokecolor{currentstroke}%
\pgfsetdash{}{0pt}%
\pgfpathmoveto{\pgfqpoint{3.179892in}{2.837863in}}%
\pgfusepath{stroke}%
\end{pgfscope}%
\begin{pgfscope}%
\pgfpathrectangle{\pgfqpoint{0.765000in}{0.660000in}}{\pgfqpoint{4.620000in}{4.620000in}}%
\pgfusepath{clip}%
\pgfsetbuttcap%
\pgfsetmiterjoin%
\definecolor{currentfill}{rgb}{0.176471,0.192157,0.258824}%
\pgfsetfillcolor{currentfill}%
\pgfsetlinewidth{1.003750pt}%
\definecolor{currentstroke}{rgb}{0.176471,0.192157,0.258824}%
\pgfsetstrokecolor{currentstroke}%
\pgfsetdash{}{0pt}%
\pgfsys@defobject{currentmarker}{\pgfqpoint{-0.033333in}{-0.033333in}}{\pgfqpoint{0.033333in}{0.033333in}}{%
\pgfpathmoveto{\pgfqpoint{-0.000000in}{-0.033333in}}%
\pgfpathlineto{\pgfqpoint{0.033333in}{0.033333in}}%
\pgfpathlineto{\pgfqpoint{-0.033333in}{0.033333in}}%
\pgfpathlineto{\pgfqpoint{-0.000000in}{-0.033333in}}%
\pgfpathclose%
\pgfusepath{stroke,fill}%
}%
\begin{pgfscope}%
\pgfsys@transformshift{3.179892in}{2.837863in}%
\pgfsys@useobject{currentmarker}{}%
\end{pgfscope}%
\end{pgfscope}%
\begin{pgfscope}%
\pgfpathrectangle{\pgfqpoint{0.765000in}{0.660000in}}{\pgfqpoint{4.620000in}{4.620000in}}%
\pgfusepath{clip}%
\pgfsetroundcap%
\pgfsetroundjoin%
\pgfsetlinewidth{1.204500pt}%
\definecolor{currentstroke}{rgb}{0.176471,0.192157,0.258824}%
\pgfsetstrokecolor{currentstroke}%
\pgfsetdash{}{0pt}%
\pgfpathmoveto{\pgfqpoint{3.366082in}{2.820936in}}%
\pgfusepath{stroke}%
\end{pgfscope}%
\begin{pgfscope}%
\pgfpathrectangle{\pgfqpoint{0.765000in}{0.660000in}}{\pgfqpoint{4.620000in}{4.620000in}}%
\pgfusepath{clip}%
\pgfsetbuttcap%
\pgfsetmiterjoin%
\definecolor{currentfill}{rgb}{0.176471,0.192157,0.258824}%
\pgfsetfillcolor{currentfill}%
\pgfsetlinewidth{1.003750pt}%
\definecolor{currentstroke}{rgb}{0.176471,0.192157,0.258824}%
\pgfsetstrokecolor{currentstroke}%
\pgfsetdash{}{0pt}%
\pgfsys@defobject{currentmarker}{\pgfqpoint{-0.033333in}{-0.033333in}}{\pgfqpoint{0.033333in}{0.033333in}}{%
\pgfpathmoveto{\pgfqpoint{-0.000000in}{-0.033333in}}%
\pgfpathlineto{\pgfqpoint{0.033333in}{0.033333in}}%
\pgfpathlineto{\pgfqpoint{-0.033333in}{0.033333in}}%
\pgfpathlineto{\pgfqpoint{-0.000000in}{-0.033333in}}%
\pgfpathclose%
\pgfusepath{stroke,fill}%
}%
\begin{pgfscope}%
\pgfsys@transformshift{3.366082in}{2.820936in}%
\pgfsys@useobject{currentmarker}{}%
\end{pgfscope}%
\end{pgfscope}%
\begin{pgfscope}%
\pgfpathrectangle{\pgfqpoint{0.765000in}{0.660000in}}{\pgfqpoint{4.620000in}{4.620000in}}%
\pgfusepath{clip}%
\pgfsetroundcap%
\pgfsetroundjoin%
\pgfsetlinewidth{1.204500pt}%
\definecolor{currentstroke}{rgb}{0.176471,0.192157,0.258824}%
\pgfsetstrokecolor{currentstroke}%
\pgfsetdash{}{0pt}%
\pgfpathmoveto{\pgfqpoint{3.370490in}{2.808167in}}%
\pgfusepath{stroke}%
\end{pgfscope}%
\begin{pgfscope}%
\pgfpathrectangle{\pgfqpoint{0.765000in}{0.660000in}}{\pgfqpoint{4.620000in}{4.620000in}}%
\pgfusepath{clip}%
\pgfsetbuttcap%
\pgfsetmiterjoin%
\definecolor{currentfill}{rgb}{0.176471,0.192157,0.258824}%
\pgfsetfillcolor{currentfill}%
\pgfsetlinewidth{1.003750pt}%
\definecolor{currentstroke}{rgb}{0.176471,0.192157,0.258824}%
\pgfsetstrokecolor{currentstroke}%
\pgfsetdash{}{0pt}%
\pgfsys@defobject{currentmarker}{\pgfqpoint{-0.033333in}{-0.033333in}}{\pgfqpoint{0.033333in}{0.033333in}}{%
\pgfpathmoveto{\pgfqpoint{-0.000000in}{-0.033333in}}%
\pgfpathlineto{\pgfqpoint{0.033333in}{0.033333in}}%
\pgfpathlineto{\pgfqpoint{-0.033333in}{0.033333in}}%
\pgfpathlineto{\pgfqpoint{-0.000000in}{-0.033333in}}%
\pgfpathclose%
\pgfusepath{stroke,fill}%
}%
\begin{pgfscope}%
\pgfsys@transformshift{3.370490in}{2.808167in}%
\pgfsys@useobject{currentmarker}{}%
\end{pgfscope}%
\end{pgfscope}%
\begin{pgfscope}%
\pgfpathrectangle{\pgfqpoint{0.765000in}{0.660000in}}{\pgfqpoint{4.620000in}{4.620000in}}%
\pgfusepath{clip}%
\pgfsetroundcap%
\pgfsetroundjoin%
\pgfsetlinewidth{1.204500pt}%
\definecolor{currentstroke}{rgb}{0.176471,0.192157,0.258824}%
\pgfsetstrokecolor{currentstroke}%
\pgfsetdash{}{0pt}%
\pgfpathmoveto{\pgfqpoint{3.211333in}{2.856047in}}%
\pgfusepath{stroke}%
\end{pgfscope}%
\begin{pgfscope}%
\pgfpathrectangle{\pgfqpoint{0.765000in}{0.660000in}}{\pgfqpoint{4.620000in}{4.620000in}}%
\pgfusepath{clip}%
\pgfsetbuttcap%
\pgfsetmiterjoin%
\definecolor{currentfill}{rgb}{0.176471,0.192157,0.258824}%
\pgfsetfillcolor{currentfill}%
\pgfsetlinewidth{1.003750pt}%
\definecolor{currentstroke}{rgb}{0.176471,0.192157,0.258824}%
\pgfsetstrokecolor{currentstroke}%
\pgfsetdash{}{0pt}%
\pgfsys@defobject{currentmarker}{\pgfqpoint{-0.033333in}{-0.033333in}}{\pgfqpoint{0.033333in}{0.033333in}}{%
\pgfpathmoveto{\pgfqpoint{-0.000000in}{-0.033333in}}%
\pgfpathlineto{\pgfqpoint{0.033333in}{0.033333in}}%
\pgfpathlineto{\pgfqpoint{-0.033333in}{0.033333in}}%
\pgfpathlineto{\pgfqpoint{-0.000000in}{-0.033333in}}%
\pgfpathclose%
\pgfusepath{stroke,fill}%
}%
\begin{pgfscope}%
\pgfsys@transformshift{3.211333in}{2.856047in}%
\pgfsys@useobject{currentmarker}{}%
\end{pgfscope}%
\end{pgfscope}%
\begin{pgfscope}%
\pgfpathrectangle{\pgfqpoint{0.765000in}{0.660000in}}{\pgfqpoint{4.620000in}{4.620000in}}%
\pgfusepath{clip}%
\pgfsetroundcap%
\pgfsetroundjoin%
\pgfsetlinewidth{1.204500pt}%
\definecolor{currentstroke}{rgb}{0.176471,0.192157,0.258824}%
\pgfsetstrokecolor{currentstroke}%
\pgfsetdash{}{0pt}%
\pgfpathmoveto{\pgfqpoint{3.367510in}{2.800596in}}%
\pgfusepath{stroke}%
\end{pgfscope}%
\begin{pgfscope}%
\pgfpathrectangle{\pgfqpoint{0.765000in}{0.660000in}}{\pgfqpoint{4.620000in}{4.620000in}}%
\pgfusepath{clip}%
\pgfsetbuttcap%
\pgfsetmiterjoin%
\definecolor{currentfill}{rgb}{0.176471,0.192157,0.258824}%
\pgfsetfillcolor{currentfill}%
\pgfsetlinewidth{1.003750pt}%
\definecolor{currentstroke}{rgb}{0.176471,0.192157,0.258824}%
\pgfsetstrokecolor{currentstroke}%
\pgfsetdash{}{0pt}%
\pgfsys@defobject{currentmarker}{\pgfqpoint{-0.033333in}{-0.033333in}}{\pgfqpoint{0.033333in}{0.033333in}}{%
\pgfpathmoveto{\pgfqpoint{-0.000000in}{-0.033333in}}%
\pgfpathlineto{\pgfqpoint{0.033333in}{0.033333in}}%
\pgfpathlineto{\pgfqpoint{-0.033333in}{0.033333in}}%
\pgfpathlineto{\pgfqpoint{-0.000000in}{-0.033333in}}%
\pgfpathclose%
\pgfusepath{stroke,fill}%
}%
\begin{pgfscope}%
\pgfsys@transformshift{3.367510in}{2.800596in}%
\pgfsys@useobject{currentmarker}{}%
\end{pgfscope}%
\end{pgfscope}%
\begin{pgfscope}%
\pgfpathrectangle{\pgfqpoint{0.765000in}{0.660000in}}{\pgfqpoint{4.620000in}{4.620000in}}%
\pgfusepath{clip}%
\pgfsetroundcap%
\pgfsetroundjoin%
\pgfsetlinewidth{1.204500pt}%
\definecolor{currentstroke}{rgb}{0.176471,0.192157,0.258824}%
\pgfsetstrokecolor{currentstroke}%
\pgfsetdash{}{0pt}%
\pgfpathmoveto{\pgfqpoint{3.456778in}{2.729227in}}%
\pgfusepath{stroke}%
\end{pgfscope}%
\begin{pgfscope}%
\pgfpathrectangle{\pgfqpoint{0.765000in}{0.660000in}}{\pgfqpoint{4.620000in}{4.620000in}}%
\pgfusepath{clip}%
\pgfsetbuttcap%
\pgfsetmiterjoin%
\definecolor{currentfill}{rgb}{0.176471,0.192157,0.258824}%
\pgfsetfillcolor{currentfill}%
\pgfsetlinewidth{1.003750pt}%
\definecolor{currentstroke}{rgb}{0.176471,0.192157,0.258824}%
\pgfsetstrokecolor{currentstroke}%
\pgfsetdash{}{0pt}%
\pgfsys@defobject{currentmarker}{\pgfqpoint{-0.033333in}{-0.033333in}}{\pgfqpoint{0.033333in}{0.033333in}}{%
\pgfpathmoveto{\pgfqpoint{-0.000000in}{-0.033333in}}%
\pgfpathlineto{\pgfqpoint{0.033333in}{0.033333in}}%
\pgfpathlineto{\pgfqpoint{-0.033333in}{0.033333in}}%
\pgfpathlineto{\pgfqpoint{-0.000000in}{-0.033333in}}%
\pgfpathclose%
\pgfusepath{stroke,fill}%
}%
\begin{pgfscope}%
\pgfsys@transformshift{3.456778in}{2.729227in}%
\pgfsys@useobject{currentmarker}{}%
\end{pgfscope}%
\end{pgfscope}%
\begin{pgfscope}%
\pgfpathrectangle{\pgfqpoint{0.765000in}{0.660000in}}{\pgfqpoint{4.620000in}{4.620000in}}%
\pgfusepath{clip}%
\pgfsetroundcap%
\pgfsetroundjoin%
\pgfsetlinewidth{1.204500pt}%
\definecolor{currentstroke}{rgb}{0.176471,0.192157,0.258824}%
\pgfsetstrokecolor{currentstroke}%
\pgfsetdash{}{0pt}%
\pgfpathmoveto{\pgfqpoint{3.294041in}{2.722366in}}%
\pgfusepath{stroke}%
\end{pgfscope}%
\begin{pgfscope}%
\pgfpathrectangle{\pgfqpoint{0.765000in}{0.660000in}}{\pgfqpoint{4.620000in}{4.620000in}}%
\pgfusepath{clip}%
\pgfsetbuttcap%
\pgfsetmiterjoin%
\definecolor{currentfill}{rgb}{0.176471,0.192157,0.258824}%
\pgfsetfillcolor{currentfill}%
\pgfsetlinewidth{1.003750pt}%
\definecolor{currentstroke}{rgb}{0.176471,0.192157,0.258824}%
\pgfsetstrokecolor{currentstroke}%
\pgfsetdash{}{0pt}%
\pgfsys@defobject{currentmarker}{\pgfqpoint{-0.033333in}{-0.033333in}}{\pgfqpoint{0.033333in}{0.033333in}}{%
\pgfpathmoveto{\pgfqpoint{-0.000000in}{-0.033333in}}%
\pgfpathlineto{\pgfqpoint{0.033333in}{0.033333in}}%
\pgfpathlineto{\pgfqpoint{-0.033333in}{0.033333in}}%
\pgfpathlineto{\pgfqpoint{-0.000000in}{-0.033333in}}%
\pgfpathclose%
\pgfusepath{stroke,fill}%
}%
\begin{pgfscope}%
\pgfsys@transformshift{3.294041in}{2.722366in}%
\pgfsys@useobject{currentmarker}{}%
\end{pgfscope}%
\end{pgfscope}%
\begin{pgfscope}%
\pgfpathrectangle{\pgfqpoint{0.765000in}{0.660000in}}{\pgfqpoint{4.620000in}{4.620000in}}%
\pgfusepath{clip}%
\pgfsetroundcap%
\pgfsetroundjoin%
\pgfsetlinewidth{1.204500pt}%
\definecolor{currentstroke}{rgb}{0.176471,0.192157,0.258824}%
\pgfsetstrokecolor{currentstroke}%
\pgfsetdash{}{0pt}%
\pgfpathmoveto{\pgfqpoint{3.382246in}{2.891995in}}%
\pgfusepath{stroke}%
\end{pgfscope}%
\begin{pgfscope}%
\pgfpathrectangle{\pgfqpoint{0.765000in}{0.660000in}}{\pgfqpoint{4.620000in}{4.620000in}}%
\pgfusepath{clip}%
\pgfsetbuttcap%
\pgfsetmiterjoin%
\definecolor{currentfill}{rgb}{0.176471,0.192157,0.258824}%
\pgfsetfillcolor{currentfill}%
\pgfsetlinewidth{1.003750pt}%
\definecolor{currentstroke}{rgb}{0.176471,0.192157,0.258824}%
\pgfsetstrokecolor{currentstroke}%
\pgfsetdash{}{0pt}%
\pgfsys@defobject{currentmarker}{\pgfqpoint{-0.033333in}{-0.033333in}}{\pgfqpoint{0.033333in}{0.033333in}}{%
\pgfpathmoveto{\pgfqpoint{-0.000000in}{-0.033333in}}%
\pgfpathlineto{\pgfqpoint{0.033333in}{0.033333in}}%
\pgfpathlineto{\pgfqpoint{-0.033333in}{0.033333in}}%
\pgfpathlineto{\pgfqpoint{-0.000000in}{-0.033333in}}%
\pgfpathclose%
\pgfusepath{stroke,fill}%
}%
\begin{pgfscope}%
\pgfsys@transformshift{3.382246in}{2.891995in}%
\pgfsys@useobject{currentmarker}{}%
\end{pgfscope}%
\end{pgfscope}%
\begin{pgfscope}%
\pgfpathrectangle{\pgfqpoint{0.765000in}{0.660000in}}{\pgfqpoint{4.620000in}{4.620000in}}%
\pgfusepath{clip}%
\pgfsetroundcap%
\pgfsetroundjoin%
\pgfsetlinewidth{1.204500pt}%
\definecolor{currentstroke}{rgb}{0.176471,0.192157,0.258824}%
\pgfsetstrokecolor{currentstroke}%
\pgfsetdash{}{0pt}%
\pgfpathmoveto{\pgfqpoint{3.208726in}{2.890945in}}%
\pgfusepath{stroke}%
\end{pgfscope}%
\begin{pgfscope}%
\pgfpathrectangle{\pgfqpoint{0.765000in}{0.660000in}}{\pgfqpoint{4.620000in}{4.620000in}}%
\pgfusepath{clip}%
\pgfsetbuttcap%
\pgfsetmiterjoin%
\definecolor{currentfill}{rgb}{0.176471,0.192157,0.258824}%
\pgfsetfillcolor{currentfill}%
\pgfsetlinewidth{1.003750pt}%
\definecolor{currentstroke}{rgb}{0.176471,0.192157,0.258824}%
\pgfsetstrokecolor{currentstroke}%
\pgfsetdash{}{0pt}%
\pgfsys@defobject{currentmarker}{\pgfqpoint{-0.033333in}{-0.033333in}}{\pgfqpoint{0.033333in}{0.033333in}}{%
\pgfpathmoveto{\pgfqpoint{-0.000000in}{-0.033333in}}%
\pgfpathlineto{\pgfqpoint{0.033333in}{0.033333in}}%
\pgfpathlineto{\pgfqpoint{-0.033333in}{0.033333in}}%
\pgfpathlineto{\pgfqpoint{-0.000000in}{-0.033333in}}%
\pgfpathclose%
\pgfusepath{stroke,fill}%
}%
\begin{pgfscope}%
\pgfsys@transformshift{3.208726in}{2.890945in}%
\pgfsys@useobject{currentmarker}{}%
\end{pgfscope}%
\end{pgfscope}%
\begin{pgfscope}%
\pgfpathrectangle{\pgfqpoint{0.765000in}{0.660000in}}{\pgfqpoint{4.620000in}{4.620000in}}%
\pgfusepath{clip}%
\pgfsetroundcap%
\pgfsetroundjoin%
\pgfsetlinewidth{1.204500pt}%
\definecolor{currentstroke}{rgb}{0.176471,0.192157,0.258824}%
\pgfsetstrokecolor{currentstroke}%
\pgfsetdash{}{0pt}%
\pgfpathmoveto{\pgfqpoint{3.228730in}{2.824942in}}%
\pgfusepath{stroke}%
\end{pgfscope}%
\begin{pgfscope}%
\pgfpathrectangle{\pgfqpoint{0.765000in}{0.660000in}}{\pgfqpoint{4.620000in}{4.620000in}}%
\pgfusepath{clip}%
\pgfsetbuttcap%
\pgfsetmiterjoin%
\definecolor{currentfill}{rgb}{0.176471,0.192157,0.258824}%
\pgfsetfillcolor{currentfill}%
\pgfsetlinewidth{1.003750pt}%
\definecolor{currentstroke}{rgb}{0.176471,0.192157,0.258824}%
\pgfsetstrokecolor{currentstroke}%
\pgfsetdash{}{0pt}%
\pgfsys@defobject{currentmarker}{\pgfqpoint{-0.033333in}{-0.033333in}}{\pgfqpoint{0.033333in}{0.033333in}}{%
\pgfpathmoveto{\pgfqpoint{-0.000000in}{-0.033333in}}%
\pgfpathlineto{\pgfqpoint{0.033333in}{0.033333in}}%
\pgfpathlineto{\pgfqpoint{-0.033333in}{0.033333in}}%
\pgfpathlineto{\pgfqpoint{-0.000000in}{-0.033333in}}%
\pgfpathclose%
\pgfusepath{stroke,fill}%
}%
\begin{pgfscope}%
\pgfsys@transformshift{3.228730in}{2.824942in}%
\pgfsys@useobject{currentmarker}{}%
\end{pgfscope}%
\end{pgfscope}%
\begin{pgfscope}%
\pgfpathrectangle{\pgfqpoint{0.765000in}{0.660000in}}{\pgfqpoint{4.620000in}{4.620000in}}%
\pgfusepath{clip}%
\pgfsetroundcap%
\pgfsetroundjoin%
\pgfsetlinewidth{1.204500pt}%
\definecolor{currentstroke}{rgb}{0.176471,0.192157,0.258824}%
\pgfsetstrokecolor{currentstroke}%
\pgfsetdash{}{0pt}%
\pgfpathmoveto{\pgfqpoint{3.363647in}{2.876284in}}%
\pgfusepath{stroke}%
\end{pgfscope}%
\begin{pgfscope}%
\pgfpathrectangle{\pgfqpoint{0.765000in}{0.660000in}}{\pgfqpoint{4.620000in}{4.620000in}}%
\pgfusepath{clip}%
\pgfsetbuttcap%
\pgfsetmiterjoin%
\definecolor{currentfill}{rgb}{0.176471,0.192157,0.258824}%
\pgfsetfillcolor{currentfill}%
\pgfsetlinewidth{1.003750pt}%
\definecolor{currentstroke}{rgb}{0.176471,0.192157,0.258824}%
\pgfsetstrokecolor{currentstroke}%
\pgfsetdash{}{0pt}%
\pgfsys@defobject{currentmarker}{\pgfqpoint{-0.033333in}{-0.033333in}}{\pgfqpoint{0.033333in}{0.033333in}}{%
\pgfpathmoveto{\pgfqpoint{-0.000000in}{-0.033333in}}%
\pgfpathlineto{\pgfqpoint{0.033333in}{0.033333in}}%
\pgfpathlineto{\pgfqpoint{-0.033333in}{0.033333in}}%
\pgfpathlineto{\pgfqpoint{-0.000000in}{-0.033333in}}%
\pgfpathclose%
\pgfusepath{stroke,fill}%
}%
\begin{pgfscope}%
\pgfsys@transformshift{3.363647in}{2.876284in}%
\pgfsys@useobject{currentmarker}{}%
\end{pgfscope}%
\end{pgfscope}%
\begin{pgfscope}%
\pgfpathrectangle{\pgfqpoint{0.765000in}{0.660000in}}{\pgfqpoint{4.620000in}{4.620000in}}%
\pgfusepath{clip}%
\pgfsetroundcap%
\pgfsetroundjoin%
\pgfsetlinewidth{1.204500pt}%
\definecolor{currentstroke}{rgb}{0.176471,0.192157,0.258824}%
\pgfsetstrokecolor{currentstroke}%
\pgfsetdash{}{0pt}%
\pgfpathmoveto{\pgfqpoint{3.217878in}{2.794990in}}%
\pgfusepath{stroke}%
\end{pgfscope}%
\begin{pgfscope}%
\pgfpathrectangle{\pgfqpoint{0.765000in}{0.660000in}}{\pgfqpoint{4.620000in}{4.620000in}}%
\pgfusepath{clip}%
\pgfsetbuttcap%
\pgfsetmiterjoin%
\definecolor{currentfill}{rgb}{0.176471,0.192157,0.258824}%
\pgfsetfillcolor{currentfill}%
\pgfsetlinewidth{1.003750pt}%
\definecolor{currentstroke}{rgb}{0.176471,0.192157,0.258824}%
\pgfsetstrokecolor{currentstroke}%
\pgfsetdash{}{0pt}%
\pgfsys@defobject{currentmarker}{\pgfqpoint{-0.033333in}{-0.033333in}}{\pgfqpoint{0.033333in}{0.033333in}}{%
\pgfpathmoveto{\pgfqpoint{-0.000000in}{-0.033333in}}%
\pgfpathlineto{\pgfqpoint{0.033333in}{0.033333in}}%
\pgfpathlineto{\pgfqpoint{-0.033333in}{0.033333in}}%
\pgfpathlineto{\pgfqpoint{-0.000000in}{-0.033333in}}%
\pgfpathclose%
\pgfusepath{stroke,fill}%
}%
\begin{pgfscope}%
\pgfsys@transformshift{3.217878in}{2.794990in}%
\pgfsys@useobject{currentmarker}{}%
\end{pgfscope}%
\end{pgfscope}%
\begin{pgfscope}%
\pgfpathrectangle{\pgfqpoint{0.765000in}{0.660000in}}{\pgfqpoint{4.620000in}{4.620000in}}%
\pgfusepath{clip}%
\pgfsetroundcap%
\pgfsetroundjoin%
\pgfsetlinewidth{1.204500pt}%
\definecolor{currentstroke}{rgb}{0.176471,0.192157,0.258824}%
\pgfsetstrokecolor{currentstroke}%
\pgfsetdash{}{0pt}%
\pgfpathmoveto{\pgfqpoint{3.301786in}{2.934083in}}%
\pgfusepath{stroke}%
\end{pgfscope}%
\begin{pgfscope}%
\pgfpathrectangle{\pgfqpoint{0.765000in}{0.660000in}}{\pgfqpoint{4.620000in}{4.620000in}}%
\pgfusepath{clip}%
\pgfsetbuttcap%
\pgfsetmiterjoin%
\definecolor{currentfill}{rgb}{0.176471,0.192157,0.258824}%
\pgfsetfillcolor{currentfill}%
\pgfsetlinewidth{1.003750pt}%
\definecolor{currentstroke}{rgb}{0.176471,0.192157,0.258824}%
\pgfsetstrokecolor{currentstroke}%
\pgfsetdash{}{0pt}%
\pgfsys@defobject{currentmarker}{\pgfqpoint{-0.033333in}{-0.033333in}}{\pgfqpoint{0.033333in}{0.033333in}}{%
\pgfpathmoveto{\pgfqpoint{-0.000000in}{-0.033333in}}%
\pgfpathlineto{\pgfqpoint{0.033333in}{0.033333in}}%
\pgfpathlineto{\pgfqpoint{-0.033333in}{0.033333in}}%
\pgfpathlineto{\pgfqpoint{-0.000000in}{-0.033333in}}%
\pgfpathclose%
\pgfusepath{stroke,fill}%
}%
\begin{pgfscope}%
\pgfsys@transformshift{3.301786in}{2.934083in}%
\pgfsys@useobject{currentmarker}{}%
\end{pgfscope}%
\end{pgfscope}%
\begin{pgfscope}%
\pgfpathrectangle{\pgfqpoint{0.765000in}{0.660000in}}{\pgfqpoint{4.620000in}{4.620000in}}%
\pgfusepath{clip}%
\pgfsetroundcap%
\pgfsetroundjoin%
\pgfsetlinewidth{1.204500pt}%
\definecolor{currentstroke}{rgb}{0.176471,0.192157,0.258824}%
\pgfsetstrokecolor{currentstroke}%
\pgfsetdash{}{0pt}%
\pgfpathmoveto{\pgfqpoint{3.364346in}{2.805202in}}%
\pgfusepath{stroke}%
\end{pgfscope}%
\begin{pgfscope}%
\pgfpathrectangle{\pgfqpoint{0.765000in}{0.660000in}}{\pgfqpoint{4.620000in}{4.620000in}}%
\pgfusepath{clip}%
\pgfsetbuttcap%
\pgfsetmiterjoin%
\definecolor{currentfill}{rgb}{0.176471,0.192157,0.258824}%
\pgfsetfillcolor{currentfill}%
\pgfsetlinewidth{1.003750pt}%
\definecolor{currentstroke}{rgb}{0.176471,0.192157,0.258824}%
\pgfsetstrokecolor{currentstroke}%
\pgfsetdash{}{0pt}%
\pgfsys@defobject{currentmarker}{\pgfqpoint{-0.033333in}{-0.033333in}}{\pgfqpoint{0.033333in}{0.033333in}}{%
\pgfpathmoveto{\pgfqpoint{-0.000000in}{-0.033333in}}%
\pgfpathlineto{\pgfqpoint{0.033333in}{0.033333in}}%
\pgfpathlineto{\pgfqpoint{-0.033333in}{0.033333in}}%
\pgfpathlineto{\pgfqpoint{-0.000000in}{-0.033333in}}%
\pgfpathclose%
\pgfusepath{stroke,fill}%
}%
\begin{pgfscope}%
\pgfsys@transformshift{3.364346in}{2.805202in}%
\pgfsys@useobject{currentmarker}{}%
\end{pgfscope}%
\end{pgfscope}%
\begin{pgfscope}%
\pgfpathrectangle{\pgfqpoint{0.765000in}{0.660000in}}{\pgfqpoint{4.620000in}{4.620000in}}%
\pgfusepath{clip}%
\pgfsetroundcap%
\pgfsetroundjoin%
\pgfsetlinewidth{1.204500pt}%
\definecolor{currentstroke}{rgb}{0.000000,0.000000,0.000000}%
\pgfsetstrokecolor{currentstroke}%
\pgfsetdash{}{0pt}%
\pgfpathmoveto{\pgfqpoint{3.325249in}{2.510517in}}%
\pgfusepath{stroke}%
\end{pgfscope}%
\begin{pgfscope}%
\pgfpathrectangle{\pgfqpoint{0.765000in}{0.660000in}}{\pgfqpoint{4.620000in}{4.620000in}}%
\pgfusepath{clip}%
\pgfsetbuttcap%
\pgfsetroundjoin%
\definecolor{currentfill}{rgb}{0.000000,0.000000,0.000000}%
\pgfsetfillcolor{currentfill}%
\pgfsetlinewidth{1.003750pt}%
\definecolor{currentstroke}{rgb}{0.000000,0.000000,0.000000}%
\pgfsetstrokecolor{currentstroke}%
\pgfsetdash{}{0pt}%
\pgfsys@defobject{currentmarker}{\pgfqpoint{-0.016667in}{-0.016667in}}{\pgfqpoint{0.016667in}{0.016667in}}{%
\pgfpathmoveto{\pgfqpoint{0.000000in}{-0.016667in}}%
\pgfpathcurveto{\pgfqpoint{0.004420in}{-0.016667in}}{\pgfqpoint{0.008660in}{-0.014911in}}{\pgfqpoint{0.011785in}{-0.011785in}}%
\pgfpathcurveto{\pgfqpoint{0.014911in}{-0.008660in}}{\pgfqpoint{0.016667in}{-0.004420in}}{\pgfqpoint{0.016667in}{0.000000in}}%
\pgfpathcurveto{\pgfqpoint{0.016667in}{0.004420in}}{\pgfqpoint{0.014911in}{0.008660in}}{\pgfqpoint{0.011785in}{0.011785in}}%
\pgfpathcurveto{\pgfqpoint{0.008660in}{0.014911in}}{\pgfqpoint{0.004420in}{0.016667in}}{\pgfqpoint{0.000000in}{0.016667in}}%
\pgfpathcurveto{\pgfqpoint{-0.004420in}{0.016667in}}{\pgfqpoint{-0.008660in}{0.014911in}}{\pgfqpoint{-0.011785in}{0.011785in}}%
\pgfpathcurveto{\pgfqpoint{-0.014911in}{0.008660in}}{\pgfqpoint{-0.016667in}{0.004420in}}{\pgfqpoint{-0.016667in}{0.000000in}}%
\pgfpathcurveto{\pgfqpoint{-0.016667in}{-0.004420in}}{\pgfqpoint{-0.014911in}{-0.008660in}}{\pgfqpoint{-0.011785in}{-0.011785in}}%
\pgfpathcurveto{\pgfqpoint{-0.008660in}{-0.014911in}}{\pgfqpoint{-0.004420in}{-0.016667in}}{\pgfqpoint{0.000000in}{-0.016667in}}%
\pgfpathlineto{\pgfqpoint{0.000000in}{-0.016667in}}%
\pgfpathclose%
\pgfusepath{stroke,fill}%
}%
\begin{pgfscope}%
\pgfsys@transformshift{3.325249in}{2.510517in}%
\pgfsys@useobject{currentmarker}{}%
\end{pgfscope}%
\end{pgfscope}%
\begin{pgfscope}%
\pgfpathrectangle{\pgfqpoint{0.765000in}{0.660000in}}{\pgfqpoint{4.620000in}{4.620000in}}%
\pgfusepath{clip}%
\pgfsetroundcap%
\pgfsetroundjoin%
\pgfsetlinewidth{1.204500pt}%
\definecolor{currentstroke}{rgb}{0.000000,0.000000,0.000000}%
\pgfsetstrokecolor{currentstroke}%
\pgfsetdash{}{0pt}%
\pgfpathmoveto{\pgfqpoint{3.325249in}{2.510517in}}%
\pgfpathlineto{\pgfqpoint{3.244100in}{2.548548in}}%
\pgfusepath{stroke}%
\end{pgfscope}%
\begin{pgfscope}%
\pgfpathrectangle{\pgfqpoint{0.765000in}{0.660000in}}{\pgfqpoint{4.620000in}{4.620000in}}%
\pgfusepath{clip}%
\pgfsetroundcap%
\pgfsetroundjoin%
\pgfsetlinewidth{1.204500pt}%
\definecolor{currentstroke}{rgb}{0.000000,0.000000,0.000000}%
\pgfsetstrokecolor{currentstroke}%
\pgfsetdash{}{0pt}%
\pgfpathmoveto{\pgfqpoint{3.325249in}{2.510517in}}%
\pgfpathlineto{\pgfqpoint{3.448221in}{2.568149in}}%
\pgfusepath{stroke}%
\end{pgfscope}%
\begin{pgfscope}%
\pgfpathrectangle{\pgfqpoint{0.765000in}{0.660000in}}{\pgfqpoint{4.620000in}{4.620000in}}%
\pgfusepath{clip}%
\pgfsetbuttcap%
\pgfsetroundjoin%
\pgfsetlinewidth{1.204500pt}%
\definecolor{currentstroke}{rgb}{0.827451,0.827451,0.827451}%
\pgfsetstrokecolor{currentstroke}%
\pgfsetdash{{1.200000pt}{1.980000pt}}{0.000000pt}%
\pgfpathmoveto{\pgfqpoint{3.325249in}{2.510517in}}%
\pgfpathlineto{\pgfqpoint{3.325249in}{0.650000in}}%
\pgfusepath{stroke}%
\end{pgfscope}%
\begin{pgfscope}%
\pgfpathrectangle{\pgfqpoint{0.765000in}{0.660000in}}{\pgfqpoint{4.620000in}{4.620000in}}%
\pgfusepath{clip}%
\pgfsetroundcap%
\pgfsetroundjoin%
\pgfsetlinewidth{1.204500pt}%
\definecolor{currentstroke}{rgb}{0.000000,0.000000,0.000000}%
\pgfsetstrokecolor{currentstroke}%
\pgfsetdash{}{0pt}%
\pgfpathmoveto{\pgfqpoint{3.244100in}{2.548548in}}%
\pgfusepath{stroke}%
\end{pgfscope}%
\begin{pgfscope}%
\pgfpathrectangle{\pgfqpoint{0.765000in}{0.660000in}}{\pgfqpoint{4.620000in}{4.620000in}}%
\pgfusepath{clip}%
\pgfsetbuttcap%
\pgfsetroundjoin%
\definecolor{currentfill}{rgb}{0.000000,0.000000,0.000000}%
\pgfsetfillcolor{currentfill}%
\pgfsetlinewidth{1.003750pt}%
\definecolor{currentstroke}{rgb}{0.000000,0.000000,0.000000}%
\pgfsetstrokecolor{currentstroke}%
\pgfsetdash{}{0pt}%
\pgfsys@defobject{currentmarker}{\pgfqpoint{-0.016667in}{-0.016667in}}{\pgfqpoint{0.016667in}{0.016667in}}{%
\pgfpathmoveto{\pgfqpoint{0.000000in}{-0.016667in}}%
\pgfpathcurveto{\pgfqpoint{0.004420in}{-0.016667in}}{\pgfqpoint{0.008660in}{-0.014911in}}{\pgfqpoint{0.011785in}{-0.011785in}}%
\pgfpathcurveto{\pgfqpoint{0.014911in}{-0.008660in}}{\pgfqpoint{0.016667in}{-0.004420in}}{\pgfqpoint{0.016667in}{0.000000in}}%
\pgfpathcurveto{\pgfqpoint{0.016667in}{0.004420in}}{\pgfqpoint{0.014911in}{0.008660in}}{\pgfqpoint{0.011785in}{0.011785in}}%
\pgfpathcurveto{\pgfqpoint{0.008660in}{0.014911in}}{\pgfqpoint{0.004420in}{0.016667in}}{\pgfqpoint{0.000000in}{0.016667in}}%
\pgfpathcurveto{\pgfqpoint{-0.004420in}{0.016667in}}{\pgfqpoint{-0.008660in}{0.014911in}}{\pgfqpoint{-0.011785in}{0.011785in}}%
\pgfpathcurveto{\pgfqpoint{-0.014911in}{0.008660in}}{\pgfqpoint{-0.016667in}{0.004420in}}{\pgfqpoint{-0.016667in}{0.000000in}}%
\pgfpathcurveto{\pgfqpoint{-0.016667in}{-0.004420in}}{\pgfqpoint{-0.014911in}{-0.008660in}}{\pgfqpoint{-0.011785in}{-0.011785in}}%
\pgfpathcurveto{\pgfqpoint{-0.008660in}{-0.014911in}}{\pgfqpoint{-0.004420in}{-0.016667in}}{\pgfqpoint{0.000000in}{-0.016667in}}%
\pgfpathlineto{\pgfqpoint{0.000000in}{-0.016667in}}%
\pgfpathclose%
\pgfusepath{stroke,fill}%
}%
\begin{pgfscope}%
\pgfsys@transformshift{3.244100in}{2.548548in}%
\pgfsys@useobject{currentmarker}{}%
\end{pgfscope}%
\end{pgfscope}%
\begin{pgfscope}%
\pgfpathrectangle{\pgfqpoint{0.765000in}{0.660000in}}{\pgfqpoint{4.620000in}{4.620000in}}%
\pgfusepath{clip}%
\pgfsetroundcap%
\pgfsetroundjoin%
\pgfsetlinewidth{1.204500pt}%
\definecolor{currentstroke}{rgb}{0.000000,0.000000,0.000000}%
\pgfsetstrokecolor{currentstroke}%
\pgfsetdash{}{0pt}%
\pgfpathmoveto{\pgfqpoint{3.244100in}{2.548548in}}%
\pgfpathlineto{\pgfqpoint{3.325249in}{2.510517in}}%
\pgfusepath{stroke}%
\end{pgfscope}%
\begin{pgfscope}%
\pgfpathrectangle{\pgfqpoint{0.765000in}{0.660000in}}{\pgfqpoint{4.620000in}{4.620000in}}%
\pgfusepath{clip}%
\pgfsetroundcap%
\pgfsetroundjoin%
\pgfsetlinewidth{1.204500pt}%
\definecolor{currentstroke}{rgb}{0.000000,0.000000,0.000000}%
\pgfsetstrokecolor{currentstroke}%
\pgfsetdash{}{0pt}%
\pgfpathmoveto{\pgfqpoint{3.244100in}{2.548548in}}%
\pgfpathlineto{\pgfqpoint{3.001127in}{2.890162in}}%
\pgfusepath{stroke}%
\end{pgfscope}%
\begin{pgfscope}%
\pgfpathrectangle{\pgfqpoint{0.765000in}{0.660000in}}{\pgfqpoint{4.620000in}{4.620000in}}%
\pgfusepath{clip}%
\pgfsetbuttcap%
\pgfsetroundjoin%
\pgfsetlinewidth{1.204500pt}%
\definecolor{currentstroke}{rgb}{0.827451,0.827451,0.827451}%
\pgfsetstrokecolor{currentstroke}%
\pgfsetdash{{1.200000pt}{1.980000pt}}{0.000000pt}%
\pgfpathmoveto{\pgfqpoint{3.244100in}{2.548548in}}%
\pgfpathlineto{\pgfqpoint{3.244100in}{0.650000in}}%
\pgfusepath{stroke}%
\end{pgfscope}%
\begin{pgfscope}%
\pgfpathrectangle{\pgfqpoint{0.765000in}{0.660000in}}{\pgfqpoint{4.620000in}{4.620000in}}%
\pgfusepath{clip}%
\pgfsetroundcap%
\pgfsetroundjoin%
\pgfsetlinewidth{1.204500pt}%
\definecolor{currentstroke}{rgb}{0.000000,0.000000,0.000000}%
\pgfsetstrokecolor{currentstroke}%
\pgfsetdash{}{0pt}%
\pgfpathmoveto{\pgfqpoint{3.001127in}{2.942594in}}%
\pgfusepath{stroke}%
\end{pgfscope}%
\begin{pgfscope}%
\pgfpathrectangle{\pgfqpoint{0.765000in}{0.660000in}}{\pgfqpoint{4.620000in}{4.620000in}}%
\pgfusepath{clip}%
\pgfsetbuttcap%
\pgfsetroundjoin%
\definecolor{currentfill}{rgb}{0.000000,0.000000,0.000000}%
\pgfsetfillcolor{currentfill}%
\pgfsetlinewidth{1.003750pt}%
\definecolor{currentstroke}{rgb}{0.000000,0.000000,0.000000}%
\pgfsetstrokecolor{currentstroke}%
\pgfsetdash{}{0pt}%
\pgfsys@defobject{currentmarker}{\pgfqpoint{-0.016667in}{-0.016667in}}{\pgfqpoint{0.016667in}{0.016667in}}{%
\pgfpathmoveto{\pgfqpoint{0.000000in}{-0.016667in}}%
\pgfpathcurveto{\pgfqpoint{0.004420in}{-0.016667in}}{\pgfqpoint{0.008660in}{-0.014911in}}{\pgfqpoint{0.011785in}{-0.011785in}}%
\pgfpathcurveto{\pgfqpoint{0.014911in}{-0.008660in}}{\pgfqpoint{0.016667in}{-0.004420in}}{\pgfqpoint{0.016667in}{0.000000in}}%
\pgfpathcurveto{\pgfqpoint{0.016667in}{0.004420in}}{\pgfqpoint{0.014911in}{0.008660in}}{\pgfqpoint{0.011785in}{0.011785in}}%
\pgfpathcurveto{\pgfqpoint{0.008660in}{0.014911in}}{\pgfqpoint{0.004420in}{0.016667in}}{\pgfqpoint{0.000000in}{0.016667in}}%
\pgfpathcurveto{\pgfqpoint{-0.004420in}{0.016667in}}{\pgfqpoint{-0.008660in}{0.014911in}}{\pgfqpoint{-0.011785in}{0.011785in}}%
\pgfpathcurveto{\pgfqpoint{-0.014911in}{0.008660in}}{\pgfqpoint{-0.016667in}{0.004420in}}{\pgfqpoint{-0.016667in}{0.000000in}}%
\pgfpathcurveto{\pgfqpoint{-0.016667in}{-0.004420in}}{\pgfqpoint{-0.014911in}{-0.008660in}}{\pgfqpoint{-0.011785in}{-0.011785in}}%
\pgfpathcurveto{\pgfqpoint{-0.008660in}{-0.014911in}}{\pgfqpoint{-0.004420in}{-0.016667in}}{\pgfqpoint{0.000000in}{-0.016667in}}%
\pgfpathlineto{\pgfqpoint{0.000000in}{-0.016667in}}%
\pgfpathclose%
\pgfusepath{stroke,fill}%
}%
\begin{pgfscope}%
\pgfsys@transformshift{3.001127in}{2.942594in}%
\pgfsys@useobject{currentmarker}{}%
\end{pgfscope}%
\end{pgfscope}%
\begin{pgfscope}%
\pgfpathrectangle{\pgfqpoint{0.765000in}{0.660000in}}{\pgfqpoint{4.620000in}{4.620000in}}%
\pgfusepath{clip}%
\pgfsetroundcap%
\pgfsetroundjoin%
\pgfsetlinewidth{1.204500pt}%
\definecolor{currentstroke}{rgb}{0.000000,0.000000,0.000000}%
\pgfsetstrokecolor{currentstroke}%
\pgfsetdash{}{0pt}%
\pgfpathmoveto{\pgfqpoint{3.001127in}{2.942594in}}%
\pgfpathlineto{\pgfqpoint{3.001127in}{2.890162in}}%
\pgfusepath{stroke}%
\end{pgfscope}%
\begin{pgfscope}%
\pgfpathrectangle{\pgfqpoint{0.765000in}{0.660000in}}{\pgfqpoint{4.620000in}{4.620000in}}%
\pgfusepath{clip}%
\pgfsetroundcap%
\pgfsetroundjoin%
\pgfsetlinewidth{1.204500pt}%
\definecolor{currentstroke}{rgb}{0.000000,0.000000,0.000000}%
\pgfsetstrokecolor{currentstroke}%
\pgfsetdash{}{0pt}%
\pgfpathmoveto{\pgfqpoint{3.001127in}{2.942594in}}%
\pgfpathlineto{\pgfqpoint{3.141868in}{3.008554in}}%
\pgfusepath{stroke}%
\end{pgfscope}%
\begin{pgfscope}%
\pgfpathrectangle{\pgfqpoint{0.765000in}{0.660000in}}{\pgfqpoint{4.620000in}{4.620000in}}%
\pgfusepath{clip}%
\pgfsetbuttcap%
\pgfsetroundjoin%
\pgfsetlinewidth{1.204500pt}%
\definecolor{currentstroke}{rgb}{0.827451,0.827451,0.827451}%
\pgfsetstrokecolor{currentstroke}%
\pgfsetdash{{1.200000pt}{1.980000pt}}{0.000000pt}%
\pgfpathmoveto{\pgfqpoint{3.001127in}{2.942594in}}%
\pgfpathlineto{\pgfqpoint{0.755000in}{3.995264in}}%
\pgfusepath{stroke}%
\end{pgfscope}%
\begin{pgfscope}%
\pgfpathrectangle{\pgfqpoint{0.765000in}{0.660000in}}{\pgfqpoint{4.620000in}{4.620000in}}%
\pgfusepath{clip}%
\pgfsetroundcap%
\pgfsetroundjoin%
\pgfsetlinewidth{1.204500pt}%
\definecolor{currentstroke}{rgb}{0.000000,0.000000,0.000000}%
\pgfsetstrokecolor{currentstroke}%
\pgfsetdash{}{0pt}%
\pgfpathmoveto{\pgfqpoint{3.001127in}{2.890162in}}%
\pgfusepath{stroke}%
\end{pgfscope}%
\begin{pgfscope}%
\pgfpathrectangle{\pgfqpoint{0.765000in}{0.660000in}}{\pgfqpoint{4.620000in}{4.620000in}}%
\pgfusepath{clip}%
\pgfsetbuttcap%
\pgfsetroundjoin%
\definecolor{currentfill}{rgb}{0.000000,0.000000,0.000000}%
\pgfsetfillcolor{currentfill}%
\pgfsetlinewidth{1.003750pt}%
\definecolor{currentstroke}{rgb}{0.000000,0.000000,0.000000}%
\pgfsetstrokecolor{currentstroke}%
\pgfsetdash{}{0pt}%
\pgfsys@defobject{currentmarker}{\pgfqpoint{-0.016667in}{-0.016667in}}{\pgfqpoint{0.016667in}{0.016667in}}{%
\pgfpathmoveto{\pgfqpoint{0.000000in}{-0.016667in}}%
\pgfpathcurveto{\pgfqpoint{0.004420in}{-0.016667in}}{\pgfqpoint{0.008660in}{-0.014911in}}{\pgfqpoint{0.011785in}{-0.011785in}}%
\pgfpathcurveto{\pgfqpoint{0.014911in}{-0.008660in}}{\pgfqpoint{0.016667in}{-0.004420in}}{\pgfqpoint{0.016667in}{0.000000in}}%
\pgfpathcurveto{\pgfqpoint{0.016667in}{0.004420in}}{\pgfqpoint{0.014911in}{0.008660in}}{\pgfqpoint{0.011785in}{0.011785in}}%
\pgfpathcurveto{\pgfqpoint{0.008660in}{0.014911in}}{\pgfqpoint{0.004420in}{0.016667in}}{\pgfqpoint{0.000000in}{0.016667in}}%
\pgfpathcurveto{\pgfqpoint{-0.004420in}{0.016667in}}{\pgfqpoint{-0.008660in}{0.014911in}}{\pgfqpoint{-0.011785in}{0.011785in}}%
\pgfpathcurveto{\pgfqpoint{-0.014911in}{0.008660in}}{\pgfqpoint{-0.016667in}{0.004420in}}{\pgfqpoint{-0.016667in}{0.000000in}}%
\pgfpathcurveto{\pgfqpoint{-0.016667in}{-0.004420in}}{\pgfqpoint{-0.014911in}{-0.008660in}}{\pgfqpoint{-0.011785in}{-0.011785in}}%
\pgfpathcurveto{\pgfqpoint{-0.008660in}{-0.014911in}}{\pgfqpoint{-0.004420in}{-0.016667in}}{\pgfqpoint{0.000000in}{-0.016667in}}%
\pgfpathlineto{\pgfqpoint{0.000000in}{-0.016667in}}%
\pgfpathclose%
\pgfusepath{stroke,fill}%
}%
\begin{pgfscope}%
\pgfsys@transformshift{3.001127in}{2.890162in}%
\pgfsys@useobject{currentmarker}{}%
\end{pgfscope}%
\end{pgfscope}%
\begin{pgfscope}%
\pgfpathrectangle{\pgfqpoint{0.765000in}{0.660000in}}{\pgfqpoint{4.620000in}{4.620000in}}%
\pgfusepath{clip}%
\pgfsetroundcap%
\pgfsetroundjoin%
\pgfsetlinewidth{1.204500pt}%
\definecolor{currentstroke}{rgb}{0.000000,0.000000,0.000000}%
\pgfsetstrokecolor{currentstroke}%
\pgfsetdash{}{0pt}%
\pgfpathmoveto{\pgfqpoint{3.001127in}{2.890162in}}%
\pgfpathlineto{\pgfqpoint{3.244100in}{2.548548in}}%
\pgfusepath{stroke}%
\end{pgfscope}%
\begin{pgfscope}%
\pgfpathrectangle{\pgfqpoint{0.765000in}{0.660000in}}{\pgfqpoint{4.620000in}{4.620000in}}%
\pgfusepath{clip}%
\pgfsetroundcap%
\pgfsetroundjoin%
\pgfsetlinewidth{1.204500pt}%
\definecolor{currentstroke}{rgb}{0.000000,0.000000,0.000000}%
\pgfsetstrokecolor{currentstroke}%
\pgfsetdash{}{0pt}%
\pgfpathmoveto{\pgfqpoint{3.001127in}{2.890162in}}%
\pgfpathlineto{\pgfqpoint{3.001127in}{2.942594in}}%
\pgfusepath{stroke}%
\end{pgfscope}%
\begin{pgfscope}%
\pgfpathrectangle{\pgfqpoint{0.765000in}{0.660000in}}{\pgfqpoint{4.620000in}{4.620000in}}%
\pgfusepath{clip}%
\pgfsetbuttcap%
\pgfsetroundjoin%
\pgfsetlinewidth{1.204500pt}%
\definecolor{currentstroke}{rgb}{0.827451,0.827451,0.827451}%
\pgfsetstrokecolor{currentstroke}%
\pgfsetdash{{1.200000pt}{1.980000pt}}{0.000000pt}%
\pgfpathmoveto{\pgfqpoint{3.001127in}{2.890162in}}%
\pgfpathlineto{\pgfqpoint{0.755000in}{3.942832in}}%
\pgfusepath{stroke}%
\end{pgfscope}%
\begin{pgfscope}%
\pgfpathrectangle{\pgfqpoint{0.765000in}{0.660000in}}{\pgfqpoint{4.620000in}{4.620000in}}%
\pgfusepath{clip}%
\pgfsetroundcap%
\pgfsetroundjoin%
\pgfsetlinewidth{1.204500pt}%
\definecolor{currentstroke}{rgb}{0.000000,0.000000,0.000000}%
\pgfsetstrokecolor{currentstroke}%
\pgfsetdash{}{0pt}%
\pgfpathmoveto{\pgfqpoint{3.615726in}{2.803656in}}%
\pgfusepath{stroke}%
\end{pgfscope}%
\begin{pgfscope}%
\pgfpathrectangle{\pgfqpoint{0.765000in}{0.660000in}}{\pgfqpoint{4.620000in}{4.620000in}}%
\pgfusepath{clip}%
\pgfsetbuttcap%
\pgfsetroundjoin%
\definecolor{currentfill}{rgb}{0.000000,0.000000,0.000000}%
\pgfsetfillcolor{currentfill}%
\pgfsetlinewidth{1.003750pt}%
\definecolor{currentstroke}{rgb}{0.000000,0.000000,0.000000}%
\pgfsetstrokecolor{currentstroke}%
\pgfsetdash{}{0pt}%
\pgfsys@defobject{currentmarker}{\pgfqpoint{-0.016667in}{-0.016667in}}{\pgfqpoint{0.016667in}{0.016667in}}{%
\pgfpathmoveto{\pgfqpoint{0.000000in}{-0.016667in}}%
\pgfpathcurveto{\pgfqpoint{0.004420in}{-0.016667in}}{\pgfqpoint{0.008660in}{-0.014911in}}{\pgfqpoint{0.011785in}{-0.011785in}}%
\pgfpathcurveto{\pgfqpoint{0.014911in}{-0.008660in}}{\pgfqpoint{0.016667in}{-0.004420in}}{\pgfqpoint{0.016667in}{0.000000in}}%
\pgfpathcurveto{\pgfqpoint{0.016667in}{0.004420in}}{\pgfqpoint{0.014911in}{0.008660in}}{\pgfqpoint{0.011785in}{0.011785in}}%
\pgfpathcurveto{\pgfqpoint{0.008660in}{0.014911in}}{\pgfqpoint{0.004420in}{0.016667in}}{\pgfqpoint{0.000000in}{0.016667in}}%
\pgfpathcurveto{\pgfqpoint{-0.004420in}{0.016667in}}{\pgfqpoint{-0.008660in}{0.014911in}}{\pgfqpoint{-0.011785in}{0.011785in}}%
\pgfpathcurveto{\pgfqpoint{-0.014911in}{0.008660in}}{\pgfqpoint{-0.016667in}{0.004420in}}{\pgfqpoint{-0.016667in}{0.000000in}}%
\pgfpathcurveto{\pgfqpoint{-0.016667in}{-0.004420in}}{\pgfqpoint{-0.014911in}{-0.008660in}}{\pgfqpoint{-0.011785in}{-0.011785in}}%
\pgfpathcurveto{\pgfqpoint{-0.008660in}{-0.014911in}}{\pgfqpoint{-0.004420in}{-0.016667in}}{\pgfqpoint{0.000000in}{-0.016667in}}%
\pgfpathlineto{\pgfqpoint{0.000000in}{-0.016667in}}%
\pgfpathclose%
\pgfusepath{stroke,fill}%
}%
\begin{pgfscope}%
\pgfsys@transformshift{3.615726in}{2.803656in}%
\pgfsys@useobject{currentmarker}{}%
\end{pgfscope}%
\end{pgfscope}%
\begin{pgfscope}%
\pgfpathrectangle{\pgfqpoint{0.765000in}{0.660000in}}{\pgfqpoint{4.620000in}{4.620000in}}%
\pgfusepath{clip}%
\pgfsetroundcap%
\pgfsetroundjoin%
\pgfsetlinewidth{1.204500pt}%
\definecolor{currentstroke}{rgb}{0.000000,0.000000,0.000000}%
\pgfsetstrokecolor{currentstroke}%
\pgfsetdash{}{0pt}%
\pgfpathmoveto{\pgfqpoint{3.615726in}{2.803656in}}%
\pgfpathlineto{\pgfqpoint{3.615726in}{3.008554in}}%
\pgfusepath{stroke}%
\end{pgfscope}%
\begin{pgfscope}%
\pgfpathrectangle{\pgfqpoint{0.765000in}{0.660000in}}{\pgfqpoint{4.620000in}{4.620000in}}%
\pgfusepath{clip}%
\pgfsetroundcap%
\pgfsetroundjoin%
\pgfsetlinewidth{1.204500pt}%
\definecolor{currentstroke}{rgb}{0.000000,0.000000,0.000000}%
\pgfsetstrokecolor{currentstroke}%
\pgfsetdash{}{0pt}%
\pgfpathmoveto{\pgfqpoint{3.615726in}{2.803656in}}%
\pgfpathlineto{\pgfqpoint{3.448221in}{2.568149in}}%
\pgfusepath{stroke}%
\end{pgfscope}%
\begin{pgfscope}%
\pgfpathrectangle{\pgfqpoint{0.765000in}{0.660000in}}{\pgfqpoint{4.620000in}{4.620000in}}%
\pgfusepath{clip}%
\pgfsetbuttcap%
\pgfsetroundjoin%
\pgfsetlinewidth{1.204500pt}%
\definecolor{currentstroke}{rgb}{0.827451,0.827451,0.827451}%
\pgfsetstrokecolor{currentstroke}%
\pgfsetdash{{1.200000pt}{1.980000pt}}{0.000000pt}%
\pgfpathmoveto{\pgfqpoint{3.615726in}{2.803656in}}%
\pgfpathlineto{\pgfqpoint{5.395000in}{3.637530in}}%
\pgfusepath{stroke}%
\end{pgfscope}%
\begin{pgfscope}%
\pgfpathrectangle{\pgfqpoint{0.765000in}{0.660000in}}{\pgfqpoint{4.620000in}{4.620000in}}%
\pgfusepath{clip}%
\pgfsetroundcap%
\pgfsetroundjoin%
\pgfsetlinewidth{1.204500pt}%
\definecolor{currentstroke}{rgb}{0.000000,0.000000,0.000000}%
\pgfsetstrokecolor{currentstroke}%
\pgfsetdash{}{0pt}%
\pgfpathmoveto{\pgfqpoint{3.615726in}{3.008554in}}%
\pgfusepath{stroke}%
\end{pgfscope}%
\begin{pgfscope}%
\pgfpathrectangle{\pgfqpoint{0.765000in}{0.660000in}}{\pgfqpoint{4.620000in}{4.620000in}}%
\pgfusepath{clip}%
\pgfsetbuttcap%
\pgfsetroundjoin%
\definecolor{currentfill}{rgb}{0.000000,0.000000,0.000000}%
\pgfsetfillcolor{currentfill}%
\pgfsetlinewidth{1.003750pt}%
\definecolor{currentstroke}{rgb}{0.000000,0.000000,0.000000}%
\pgfsetstrokecolor{currentstroke}%
\pgfsetdash{}{0pt}%
\pgfsys@defobject{currentmarker}{\pgfqpoint{-0.016667in}{-0.016667in}}{\pgfqpoint{0.016667in}{0.016667in}}{%
\pgfpathmoveto{\pgfqpoint{0.000000in}{-0.016667in}}%
\pgfpathcurveto{\pgfqpoint{0.004420in}{-0.016667in}}{\pgfqpoint{0.008660in}{-0.014911in}}{\pgfqpoint{0.011785in}{-0.011785in}}%
\pgfpathcurveto{\pgfqpoint{0.014911in}{-0.008660in}}{\pgfqpoint{0.016667in}{-0.004420in}}{\pgfqpoint{0.016667in}{0.000000in}}%
\pgfpathcurveto{\pgfqpoint{0.016667in}{0.004420in}}{\pgfqpoint{0.014911in}{0.008660in}}{\pgfqpoint{0.011785in}{0.011785in}}%
\pgfpathcurveto{\pgfqpoint{0.008660in}{0.014911in}}{\pgfqpoint{0.004420in}{0.016667in}}{\pgfqpoint{0.000000in}{0.016667in}}%
\pgfpathcurveto{\pgfqpoint{-0.004420in}{0.016667in}}{\pgfqpoint{-0.008660in}{0.014911in}}{\pgfqpoint{-0.011785in}{0.011785in}}%
\pgfpathcurveto{\pgfqpoint{-0.014911in}{0.008660in}}{\pgfqpoint{-0.016667in}{0.004420in}}{\pgfqpoint{-0.016667in}{0.000000in}}%
\pgfpathcurveto{\pgfqpoint{-0.016667in}{-0.004420in}}{\pgfqpoint{-0.014911in}{-0.008660in}}{\pgfqpoint{-0.011785in}{-0.011785in}}%
\pgfpathcurveto{\pgfqpoint{-0.008660in}{-0.014911in}}{\pgfqpoint{-0.004420in}{-0.016667in}}{\pgfqpoint{0.000000in}{-0.016667in}}%
\pgfpathlineto{\pgfqpoint{0.000000in}{-0.016667in}}%
\pgfpathclose%
\pgfusepath{stroke,fill}%
}%
\begin{pgfscope}%
\pgfsys@transformshift{3.615726in}{3.008554in}%
\pgfsys@useobject{currentmarker}{}%
\end{pgfscope}%
\end{pgfscope}%
\begin{pgfscope}%
\pgfpathrectangle{\pgfqpoint{0.765000in}{0.660000in}}{\pgfqpoint{4.620000in}{4.620000in}}%
\pgfusepath{clip}%
\pgfsetroundcap%
\pgfsetroundjoin%
\pgfsetlinewidth{1.204500pt}%
\definecolor{currentstroke}{rgb}{0.000000,0.000000,0.000000}%
\pgfsetstrokecolor{currentstroke}%
\pgfsetdash{}{0pt}%
\pgfpathmoveto{\pgfqpoint{3.615726in}{3.008554in}}%
\pgfpathlineto{\pgfqpoint{3.615726in}{2.803656in}}%
\pgfusepath{stroke}%
\end{pgfscope}%
\begin{pgfscope}%
\pgfpathrectangle{\pgfqpoint{0.765000in}{0.660000in}}{\pgfqpoint{4.620000in}{4.620000in}}%
\pgfusepath{clip}%
\pgfsetroundcap%
\pgfsetroundjoin%
\pgfsetlinewidth{1.204500pt}%
\definecolor{currentstroke}{rgb}{0.000000,0.000000,0.000000}%
\pgfsetstrokecolor{currentstroke}%
\pgfsetdash{}{0pt}%
\pgfpathmoveto{\pgfqpoint{3.615726in}{3.008554in}}%
\pgfpathlineto{\pgfqpoint{3.141868in}{3.008554in}}%
\pgfusepath{stroke}%
\end{pgfscope}%
\begin{pgfscope}%
\pgfpathrectangle{\pgfqpoint{0.765000in}{0.660000in}}{\pgfqpoint{4.620000in}{4.620000in}}%
\pgfusepath{clip}%
\pgfsetbuttcap%
\pgfsetroundjoin%
\pgfsetlinewidth{1.204500pt}%
\definecolor{currentstroke}{rgb}{0.827451,0.827451,0.827451}%
\pgfsetstrokecolor{currentstroke}%
\pgfsetdash{{1.200000pt}{1.980000pt}}{0.000000pt}%
\pgfpathmoveto{\pgfqpoint{3.615726in}{3.008554in}}%
\pgfpathlineto{\pgfqpoint{5.395000in}{3.842428in}}%
\pgfusepath{stroke}%
\end{pgfscope}%
\begin{pgfscope}%
\pgfpathrectangle{\pgfqpoint{0.765000in}{0.660000in}}{\pgfqpoint{4.620000in}{4.620000in}}%
\pgfusepath{clip}%
\pgfsetroundcap%
\pgfsetroundjoin%
\pgfsetlinewidth{1.204500pt}%
\definecolor{currentstroke}{rgb}{0.000000,0.000000,0.000000}%
\pgfsetstrokecolor{currentstroke}%
\pgfsetdash{}{0pt}%
\pgfpathmoveto{\pgfqpoint{3.448221in}{2.568149in}}%
\pgfusepath{stroke}%
\end{pgfscope}%
\begin{pgfscope}%
\pgfpathrectangle{\pgfqpoint{0.765000in}{0.660000in}}{\pgfqpoint{4.620000in}{4.620000in}}%
\pgfusepath{clip}%
\pgfsetbuttcap%
\pgfsetroundjoin%
\definecolor{currentfill}{rgb}{0.000000,0.000000,0.000000}%
\pgfsetfillcolor{currentfill}%
\pgfsetlinewidth{1.003750pt}%
\definecolor{currentstroke}{rgb}{0.000000,0.000000,0.000000}%
\pgfsetstrokecolor{currentstroke}%
\pgfsetdash{}{0pt}%
\pgfsys@defobject{currentmarker}{\pgfqpoint{-0.016667in}{-0.016667in}}{\pgfqpoint{0.016667in}{0.016667in}}{%
\pgfpathmoveto{\pgfqpoint{0.000000in}{-0.016667in}}%
\pgfpathcurveto{\pgfqpoint{0.004420in}{-0.016667in}}{\pgfqpoint{0.008660in}{-0.014911in}}{\pgfqpoint{0.011785in}{-0.011785in}}%
\pgfpathcurveto{\pgfqpoint{0.014911in}{-0.008660in}}{\pgfqpoint{0.016667in}{-0.004420in}}{\pgfqpoint{0.016667in}{0.000000in}}%
\pgfpathcurveto{\pgfqpoint{0.016667in}{0.004420in}}{\pgfqpoint{0.014911in}{0.008660in}}{\pgfqpoint{0.011785in}{0.011785in}}%
\pgfpathcurveto{\pgfqpoint{0.008660in}{0.014911in}}{\pgfqpoint{0.004420in}{0.016667in}}{\pgfqpoint{0.000000in}{0.016667in}}%
\pgfpathcurveto{\pgfqpoint{-0.004420in}{0.016667in}}{\pgfqpoint{-0.008660in}{0.014911in}}{\pgfqpoint{-0.011785in}{0.011785in}}%
\pgfpathcurveto{\pgfqpoint{-0.014911in}{0.008660in}}{\pgfqpoint{-0.016667in}{0.004420in}}{\pgfqpoint{-0.016667in}{0.000000in}}%
\pgfpathcurveto{\pgfqpoint{-0.016667in}{-0.004420in}}{\pgfqpoint{-0.014911in}{-0.008660in}}{\pgfqpoint{-0.011785in}{-0.011785in}}%
\pgfpathcurveto{\pgfqpoint{-0.008660in}{-0.014911in}}{\pgfqpoint{-0.004420in}{-0.016667in}}{\pgfqpoint{0.000000in}{-0.016667in}}%
\pgfpathlineto{\pgfqpoint{0.000000in}{-0.016667in}}%
\pgfpathclose%
\pgfusepath{stroke,fill}%
}%
\begin{pgfscope}%
\pgfsys@transformshift{3.448221in}{2.568149in}%
\pgfsys@useobject{currentmarker}{}%
\end{pgfscope}%
\end{pgfscope}%
\begin{pgfscope}%
\pgfpathrectangle{\pgfqpoint{0.765000in}{0.660000in}}{\pgfqpoint{4.620000in}{4.620000in}}%
\pgfusepath{clip}%
\pgfsetroundcap%
\pgfsetroundjoin%
\pgfsetlinewidth{1.204500pt}%
\definecolor{currentstroke}{rgb}{0.000000,0.000000,0.000000}%
\pgfsetstrokecolor{currentstroke}%
\pgfsetdash{}{0pt}%
\pgfpathmoveto{\pgfqpoint{3.448221in}{2.568149in}}%
\pgfpathlineto{\pgfqpoint{3.325249in}{2.510517in}}%
\pgfusepath{stroke}%
\end{pgfscope}%
\begin{pgfscope}%
\pgfpathrectangle{\pgfqpoint{0.765000in}{0.660000in}}{\pgfqpoint{4.620000in}{4.620000in}}%
\pgfusepath{clip}%
\pgfsetroundcap%
\pgfsetroundjoin%
\pgfsetlinewidth{1.204500pt}%
\definecolor{currentstroke}{rgb}{0.000000,0.000000,0.000000}%
\pgfsetstrokecolor{currentstroke}%
\pgfsetdash{}{0pt}%
\pgfpathmoveto{\pgfqpoint{3.448221in}{2.568149in}}%
\pgfpathlineto{\pgfqpoint{3.615726in}{2.803656in}}%
\pgfusepath{stroke}%
\end{pgfscope}%
\begin{pgfscope}%
\pgfpathrectangle{\pgfqpoint{0.765000in}{0.660000in}}{\pgfqpoint{4.620000in}{4.620000in}}%
\pgfusepath{clip}%
\pgfsetbuttcap%
\pgfsetroundjoin%
\pgfsetlinewidth{1.204500pt}%
\definecolor{currentstroke}{rgb}{0.827451,0.827451,0.827451}%
\pgfsetstrokecolor{currentstroke}%
\pgfsetdash{{1.200000pt}{1.980000pt}}{0.000000pt}%
\pgfpathmoveto{\pgfqpoint{3.448221in}{2.568149in}}%
\pgfpathlineto{\pgfqpoint{3.448221in}{0.650000in}}%
\pgfusepath{stroke}%
\end{pgfscope}%
\begin{pgfscope}%
\pgfpathrectangle{\pgfqpoint{0.765000in}{0.660000in}}{\pgfqpoint{4.620000in}{4.620000in}}%
\pgfusepath{clip}%
\pgfsetroundcap%
\pgfsetroundjoin%
\pgfsetlinewidth{1.204500pt}%
\definecolor{currentstroke}{rgb}{0.000000,0.000000,0.000000}%
\pgfsetstrokecolor{currentstroke}%
\pgfsetdash{}{0pt}%
\pgfpathmoveto{\pgfqpoint{3.141868in}{3.008554in}}%
\pgfusepath{stroke}%
\end{pgfscope}%
\begin{pgfscope}%
\pgfpathrectangle{\pgfqpoint{0.765000in}{0.660000in}}{\pgfqpoint{4.620000in}{4.620000in}}%
\pgfusepath{clip}%
\pgfsetbuttcap%
\pgfsetroundjoin%
\definecolor{currentfill}{rgb}{0.000000,0.000000,0.000000}%
\pgfsetfillcolor{currentfill}%
\pgfsetlinewidth{1.003750pt}%
\definecolor{currentstroke}{rgb}{0.000000,0.000000,0.000000}%
\pgfsetstrokecolor{currentstroke}%
\pgfsetdash{}{0pt}%
\pgfsys@defobject{currentmarker}{\pgfqpoint{-0.016667in}{-0.016667in}}{\pgfqpoint{0.016667in}{0.016667in}}{%
\pgfpathmoveto{\pgfqpoint{0.000000in}{-0.016667in}}%
\pgfpathcurveto{\pgfqpoint{0.004420in}{-0.016667in}}{\pgfqpoint{0.008660in}{-0.014911in}}{\pgfqpoint{0.011785in}{-0.011785in}}%
\pgfpathcurveto{\pgfqpoint{0.014911in}{-0.008660in}}{\pgfqpoint{0.016667in}{-0.004420in}}{\pgfqpoint{0.016667in}{0.000000in}}%
\pgfpathcurveto{\pgfqpoint{0.016667in}{0.004420in}}{\pgfqpoint{0.014911in}{0.008660in}}{\pgfqpoint{0.011785in}{0.011785in}}%
\pgfpathcurveto{\pgfqpoint{0.008660in}{0.014911in}}{\pgfqpoint{0.004420in}{0.016667in}}{\pgfqpoint{0.000000in}{0.016667in}}%
\pgfpathcurveto{\pgfqpoint{-0.004420in}{0.016667in}}{\pgfqpoint{-0.008660in}{0.014911in}}{\pgfqpoint{-0.011785in}{0.011785in}}%
\pgfpathcurveto{\pgfqpoint{-0.014911in}{0.008660in}}{\pgfqpoint{-0.016667in}{0.004420in}}{\pgfqpoint{-0.016667in}{0.000000in}}%
\pgfpathcurveto{\pgfqpoint{-0.016667in}{-0.004420in}}{\pgfqpoint{-0.014911in}{-0.008660in}}{\pgfqpoint{-0.011785in}{-0.011785in}}%
\pgfpathcurveto{\pgfqpoint{-0.008660in}{-0.014911in}}{\pgfqpoint{-0.004420in}{-0.016667in}}{\pgfqpoint{0.000000in}{-0.016667in}}%
\pgfpathlineto{\pgfqpoint{0.000000in}{-0.016667in}}%
\pgfpathclose%
\pgfusepath{stroke,fill}%
}%
\begin{pgfscope}%
\pgfsys@transformshift{3.141868in}{3.008554in}%
\pgfsys@useobject{currentmarker}{}%
\end{pgfscope}%
\end{pgfscope}%
\begin{pgfscope}%
\pgfpathrectangle{\pgfqpoint{0.765000in}{0.660000in}}{\pgfqpoint{4.620000in}{4.620000in}}%
\pgfusepath{clip}%
\pgfsetroundcap%
\pgfsetroundjoin%
\pgfsetlinewidth{1.204500pt}%
\definecolor{currentstroke}{rgb}{0.000000,0.000000,0.000000}%
\pgfsetstrokecolor{currentstroke}%
\pgfsetdash{}{0pt}%
\pgfpathmoveto{\pgfqpoint{3.141868in}{3.008554in}}%
\pgfpathlineto{\pgfqpoint{3.001127in}{2.942594in}}%
\pgfusepath{stroke}%
\end{pgfscope}%
\begin{pgfscope}%
\pgfpathrectangle{\pgfqpoint{0.765000in}{0.660000in}}{\pgfqpoint{4.620000in}{4.620000in}}%
\pgfusepath{clip}%
\pgfsetroundcap%
\pgfsetroundjoin%
\pgfsetlinewidth{1.204500pt}%
\definecolor{currentstroke}{rgb}{0.000000,0.000000,0.000000}%
\pgfsetstrokecolor{currentstroke}%
\pgfsetdash{}{0pt}%
\pgfpathmoveto{\pgfqpoint{3.141868in}{3.008554in}}%
\pgfpathlineto{\pgfqpoint{3.615726in}{3.008554in}}%
\pgfusepath{stroke}%
\end{pgfscope}%
\begin{pgfscope}%
\pgfpathrectangle{\pgfqpoint{0.765000in}{0.660000in}}{\pgfqpoint{4.620000in}{4.620000in}}%
\pgfusepath{clip}%
\pgfsetbuttcap%
\pgfsetroundjoin%
\pgfsetlinewidth{1.204500pt}%
\definecolor{currentstroke}{rgb}{0.827451,0.827451,0.827451}%
\pgfsetstrokecolor{currentstroke}%
\pgfsetdash{{1.200000pt}{1.980000pt}}{0.000000pt}%
\pgfpathmoveto{\pgfqpoint{3.141868in}{3.008554in}}%
\pgfpathlineto{\pgfqpoint{0.755000in}{4.127182in}}%
\pgfusepath{stroke}%
\end{pgfscope}%
\end{pgfpicture}%
\makeatother%
\endgroup%
}}
        \caption{Using $\mathcal{A}_3$}
        \label{fig:nofeatureeng}
    \end{subfigure}
    \hspace{2cm}
    \centering
    \begin{subfigure}{0.35\textwidth}
        \centering
        \resizebox{\linewidth}{!}{%
        \centering
            \clipbox{0.35\width{} 0.25\height{} 0.27\width{} 0.35\height{}}{%% Creator: Matplotlib, PGF backend
%%
%% To include the figure in your LaTeX document, write
%%   \input{<filename>.pgf}
%%
%% Make sure the required packages are loaded in your preamble
%%   \usepackage{pgf}
%%
%% Also ensure that all the required font packages are loaded; for instance,
%% the lmodern package is sometimes necessary when using math font.
%%   \usepackage{lmodern}
%%
%% Figures using additional raster images can only be included by \input if
%% they are in the same directory as the main LaTeX file. For loading figures
%% from other directories you can use the `import` package
%%   \usepackage{import}
%%
%% and then include the figures with
%%   \import{<path to file>}{<filename>.pgf}
%%
%% Matplotlib used the following preamble
%%   
%%   \usepackage{fontspec}
%%   \setmainfont{DejaVuSerif.ttf}[Path=\detokenize{/Users/sam/Library/Python/3.9/lib/python/site-packages/matplotlib/mpl-data/fonts/ttf/}]
%%   \setsansfont{DejaVuSans.ttf}[Path=\detokenize{/Users/sam/Library/Python/3.9/lib/python/site-packages/matplotlib/mpl-data/fonts/ttf/}]
%%   \setmonofont{DejaVuSansMono.ttf}[Path=\detokenize{/Users/sam/Library/Python/3.9/lib/python/site-packages/matplotlib/mpl-data/fonts/ttf/}]
%%   \makeatletter\@ifpackageloaded{underscore}{}{\usepackage[strings]{underscore}}\makeatother
%%
\begingroup%
\makeatletter%
\begin{pgfpicture}%
\pgfpathrectangle{\pgfpointorigin}{\pgfqpoint{6.000000in}{6.000000in}}%
\pgfusepath{use as bounding box, clip}%
\begin{pgfscope}%
\pgfsetbuttcap%
\pgfsetmiterjoin%
\definecolor{currentfill}{rgb}{1.000000,1.000000,1.000000}%
\pgfsetfillcolor{currentfill}%
\pgfsetlinewidth{0.000000pt}%
\definecolor{currentstroke}{rgb}{1.000000,1.000000,1.000000}%
\pgfsetstrokecolor{currentstroke}%
\pgfsetdash{}{0pt}%
\pgfpathmoveto{\pgfqpoint{0.000000in}{0.000000in}}%
\pgfpathlineto{\pgfqpoint{6.000000in}{0.000000in}}%
\pgfpathlineto{\pgfqpoint{6.000000in}{6.000000in}}%
\pgfpathlineto{\pgfqpoint{0.000000in}{6.000000in}}%
\pgfpathlineto{\pgfqpoint{0.000000in}{0.000000in}}%
\pgfpathclose%
\pgfusepath{fill}%
\end{pgfscope}%
\begin{pgfscope}%
\pgfsetbuttcap%
\pgfsetmiterjoin%
\definecolor{currentfill}{rgb}{1.000000,1.000000,1.000000}%
\pgfsetfillcolor{currentfill}%
\pgfsetlinewidth{0.000000pt}%
\definecolor{currentstroke}{rgb}{0.000000,0.000000,0.000000}%
\pgfsetstrokecolor{currentstroke}%
\pgfsetstrokeopacity{0.000000}%
\pgfsetdash{}{0pt}%
\pgfpathmoveto{\pgfqpoint{0.765000in}{0.660000in}}%
\pgfpathlineto{\pgfqpoint{5.385000in}{0.660000in}}%
\pgfpathlineto{\pgfqpoint{5.385000in}{5.280000in}}%
\pgfpathlineto{\pgfqpoint{0.765000in}{5.280000in}}%
\pgfpathlineto{\pgfqpoint{0.765000in}{0.660000in}}%
\pgfpathclose%
\pgfusepath{fill}%
\end{pgfscope}%
\begin{pgfscope}%
\pgfpathrectangle{\pgfqpoint{0.765000in}{0.660000in}}{\pgfqpoint{4.620000in}{4.620000in}}%
\pgfusepath{clip}%
\pgfsetbuttcap%
\pgfsetroundjoin%
\definecolor{currentfill}{rgb}{1.000000,0.894118,0.788235}%
\pgfsetfillcolor{currentfill}%
\pgfsetlinewidth{0.000000pt}%
\definecolor{currentstroke}{rgb}{1.000000,0.894118,0.788235}%
\pgfsetstrokecolor{currentstroke}%
\pgfsetdash{}{0pt}%
\pgfpathmoveto{\pgfqpoint{3.121227in}{3.527081in}}%
\pgfpathlineto{\pgfqpoint{3.010331in}{3.463055in}}%
\pgfpathlineto{\pgfqpoint{3.121227in}{3.399029in}}%
\pgfpathlineto{\pgfqpoint{3.232124in}{3.463055in}}%
\pgfpathlineto{\pgfqpoint{3.121227in}{3.527081in}}%
\pgfpathclose%
\pgfusepath{fill}%
\end{pgfscope}%
\begin{pgfscope}%
\pgfpathrectangle{\pgfqpoint{0.765000in}{0.660000in}}{\pgfqpoint{4.620000in}{4.620000in}}%
\pgfusepath{clip}%
\pgfsetbuttcap%
\pgfsetroundjoin%
\definecolor{currentfill}{rgb}{1.000000,0.894118,0.788235}%
\pgfsetfillcolor{currentfill}%
\pgfsetlinewidth{0.000000pt}%
\definecolor{currentstroke}{rgb}{1.000000,0.894118,0.788235}%
\pgfsetstrokecolor{currentstroke}%
\pgfsetdash{}{0pt}%
\pgfpathmoveto{\pgfqpoint{3.121227in}{3.527081in}}%
\pgfpathlineto{\pgfqpoint{3.010331in}{3.463055in}}%
\pgfpathlineto{\pgfqpoint{3.010331in}{3.591108in}}%
\pgfpathlineto{\pgfqpoint{3.121227in}{3.655134in}}%
\pgfpathlineto{\pgfqpoint{3.121227in}{3.527081in}}%
\pgfpathclose%
\pgfusepath{fill}%
\end{pgfscope}%
\begin{pgfscope}%
\pgfpathrectangle{\pgfqpoint{0.765000in}{0.660000in}}{\pgfqpoint{4.620000in}{4.620000in}}%
\pgfusepath{clip}%
\pgfsetbuttcap%
\pgfsetroundjoin%
\definecolor{currentfill}{rgb}{1.000000,0.894118,0.788235}%
\pgfsetfillcolor{currentfill}%
\pgfsetlinewidth{0.000000pt}%
\definecolor{currentstroke}{rgb}{1.000000,0.894118,0.788235}%
\pgfsetstrokecolor{currentstroke}%
\pgfsetdash{}{0pt}%
\pgfpathmoveto{\pgfqpoint{3.121227in}{3.527081in}}%
\pgfpathlineto{\pgfqpoint{3.232124in}{3.463055in}}%
\pgfpathlineto{\pgfqpoint{3.232124in}{3.591108in}}%
\pgfpathlineto{\pgfqpoint{3.121227in}{3.655134in}}%
\pgfpathlineto{\pgfqpoint{3.121227in}{3.527081in}}%
\pgfpathclose%
\pgfusepath{fill}%
\end{pgfscope}%
\begin{pgfscope}%
\pgfpathrectangle{\pgfqpoint{0.765000in}{0.660000in}}{\pgfqpoint{4.620000in}{4.620000in}}%
\pgfusepath{clip}%
\pgfsetbuttcap%
\pgfsetroundjoin%
\definecolor{currentfill}{rgb}{1.000000,0.894118,0.788235}%
\pgfsetfillcolor{currentfill}%
\pgfsetlinewidth{0.000000pt}%
\definecolor{currentstroke}{rgb}{1.000000,0.894118,0.788235}%
\pgfsetstrokecolor{currentstroke}%
\pgfsetdash{}{0pt}%
\pgfpathmoveto{\pgfqpoint{3.145667in}{3.522403in}}%
\pgfpathlineto{\pgfqpoint{3.034770in}{3.458376in}}%
\pgfpathlineto{\pgfqpoint{3.145667in}{3.394350in}}%
\pgfpathlineto{\pgfqpoint{3.256563in}{3.458376in}}%
\pgfpathlineto{\pgfqpoint{3.145667in}{3.522403in}}%
\pgfpathclose%
\pgfusepath{fill}%
\end{pgfscope}%
\begin{pgfscope}%
\pgfpathrectangle{\pgfqpoint{0.765000in}{0.660000in}}{\pgfqpoint{4.620000in}{4.620000in}}%
\pgfusepath{clip}%
\pgfsetbuttcap%
\pgfsetroundjoin%
\definecolor{currentfill}{rgb}{1.000000,0.894118,0.788235}%
\pgfsetfillcolor{currentfill}%
\pgfsetlinewidth{0.000000pt}%
\definecolor{currentstroke}{rgb}{1.000000,0.894118,0.788235}%
\pgfsetstrokecolor{currentstroke}%
\pgfsetdash{}{0pt}%
\pgfpathmoveto{\pgfqpoint{3.145667in}{3.522403in}}%
\pgfpathlineto{\pgfqpoint{3.034770in}{3.458376in}}%
\pgfpathlineto{\pgfqpoint{3.034770in}{3.586429in}}%
\pgfpathlineto{\pgfqpoint{3.145667in}{3.650455in}}%
\pgfpathlineto{\pgfqpoint{3.145667in}{3.522403in}}%
\pgfpathclose%
\pgfusepath{fill}%
\end{pgfscope}%
\begin{pgfscope}%
\pgfpathrectangle{\pgfqpoint{0.765000in}{0.660000in}}{\pgfqpoint{4.620000in}{4.620000in}}%
\pgfusepath{clip}%
\pgfsetbuttcap%
\pgfsetroundjoin%
\definecolor{currentfill}{rgb}{1.000000,0.894118,0.788235}%
\pgfsetfillcolor{currentfill}%
\pgfsetlinewidth{0.000000pt}%
\definecolor{currentstroke}{rgb}{1.000000,0.894118,0.788235}%
\pgfsetstrokecolor{currentstroke}%
\pgfsetdash{}{0pt}%
\pgfpathmoveto{\pgfqpoint{3.145667in}{3.522403in}}%
\pgfpathlineto{\pgfqpoint{3.256563in}{3.458376in}}%
\pgfpathlineto{\pgfqpoint{3.256563in}{3.586429in}}%
\pgfpathlineto{\pgfqpoint{3.145667in}{3.650455in}}%
\pgfpathlineto{\pgfqpoint{3.145667in}{3.522403in}}%
\pgfpathclose%
\pgfusepath{fill}%
\end{pgfscope}%
\begin{pgfscope}%
\pgfpathrectangle{\pgfqpoint{0.765000in}{0.660000in}}{\pgfqpoint{4.620000in}{4.620000in}}%
\pgfusepath{clip}%
\pgfsetbuttcap%
\pgfsetroundjoin%
\definecolor{currentfill}{rgb}{1.000000,0.894118,0.788235}%
\pgfsetfillcolor{currentfill}%
\pgfsetlinewidth{0.000000pt}%
\definecolor{currentstroke}{rgb}{1.000000,0.894118,0.788235}%
\pgfsetstrokecolor{currentstroke}%
\pgfsetdash{}{0pt}%
\pgfpathmoveto{\pgfqpoint{3.121227in}{3.655134in}}%
\pgfpathlineto{\pgfqpoint{3.010331in}{3.591108in}}%
\pgfpathlineto{\pgfqpoint{3.121227in}{3.527081in}}%
\pgfpathlineto{\pgfqpoint{3.232124in}{3.591108in}}%
\pgfpathlineto{\pgfqpoint{3.121227in}{3.655134in}}%
\pgfpathclose%
\pgfusepath{fill}%
\end{pgfscope}%
\begin{pgfscope}%
\pgfpathrectangle{\pgfqpoint{0.765000in}{0.660000in}}{\pgfqpoint{4.620000in}{4.620000in}}%
\pgfusepath{clip}%
\pgfsetbuttcap%
\pgfsetroundjoin%
\definecolor{currentfill}{rgb}{1.000000,0.894118,0.788235}%
\pgfsetfillcolor{currentfill}%
\pgfsetlinewidth{0.000000pt}%
\definecolor{currentstroke}{rgb}{1.000000,0.894118,0.788235}%
\pgfsetstrokecolor{currentstroke}%
\pgfsetdash{}{0pt}%
\pgfpathmoveto{\pgfqpoint{3.121227in}{3.399029in}}%
\pgfpathlineto{\pgfqpoint{3.232124in}{3.463055in}}%
\pgfpathlineto{\pgfqpoint{3.232124in}{3.591108in}}%
\pgfpathlineto{\pgfqpoint{3.121227in}{3.527081in}}%
\pgfpathlineto{\pgfqpoint{3.121227in}{3.399029in}}%
\pgfpathclose%
\pgfusepath{fill}%
\end{pgfscope}%
\begin{pgfscope}%
\pgfpathrectangle{\pgfqpoint{0.765000in}{0.660000in}}{\pgfqpoint{4.620000in}{4.620000in}}%
\pgfusepath{clip}%
\pgfsetbuttcap%
\pgfsetroundjoin%
\definecolor{currentfill}{rgb}{1.000000,0.894118,0.788235}%
\pgfsetfillcolor{currentfill}%
\pgfsetlinewidth{0.000000pt}%
\definecolor{currentstroke}{rgb}{1.000000,0.894118,0.788235}%
\pgfsetstrokecolor{currentstroke}%
\pgfsetdash{}{0pt}%
\pgfpathmoveto{\pgfqpoint{3.010331in}{3.463055in}}%
\pgfpathlineto{\pgfqpoint{3.121227in}{3.399029in}}%
\pgfpathlineto{\pgfqpoint{3.121227in}{3.527081in}}%
\pgfpathlineto{\pgfqpoint{3.010331in}{3.591108in}}%
\pgfpathlineto{\pgfqpoint{3.010331in}{3.463055in}}%
\pgfpathclose%
\pgfusepath{fill}%
\end{pgfscope}%
\begin{pgfscope}%
\pgfpathrectangle{\pgfqpoint{0.765000in}{0.660000in}}{\pgfqpoint{4.620000in}{4.620000in}}%
\pgfusepath{clip}%
\pgfsetbuttcap%
\pgfsetroundjoin%
\definecolor{currentfill}{rgb}{1.000000,0.894118,0.788235}%
\pgfsetfillcolor{currentfill}%
\pgfsetlinewidth{0.000000pt}%
\definecolor{currentstroke}{rgb}{1.000000,0.894118,0.788235}%
\pgfsetstrokecolor{currentstroke}%
\pgfsetdash{}{0pt}%
\pgfpathmoveto{\pgfqpoint{3.145667in}{3.650455in}}%
\pgfpathlineto{\pgfqpoint{3.034770in}{3.586429in}}%
\pgfpathlineto{\pgfqpoint{3.145667in}{3.522403in}}%
\pgfpathlineto{\pgfqpoint{3.256563in}{3.586429in}}%
\pgfpathlineto{\pgfqpoint{3.145667in}{3.650455in}}%
\pgfpathclose%
\pgfusepath{fill}%
\end{pgfscope}%
\begin{pgfscope}%
\pgfpathrectangle{\pgfqpoint{0.765000in}{0.660000in}}{\pgfqpoint{4.620000in}{4.620000in}}%
\pgfusepath{clip}%
\pgfsetbuttcap%
\pgfsetroundjoin%
\definecolor{currentfill}{rgb}{1.000000,0.894118,0.788235}%
\pgfsetfillcolor{currentfill}%
\pgfsetlinewidth{0.000000pt}%
\definecolor{currentstroke}{rgb}{1.000000,0.894118,0.788235}%
\pgfsetstrokecolor{currentstroke}%
\pgfsetdash{}{0pt}%
\pgfpathmoveto{\pgfqpoint{3.145667in}{3.394350in}}%
\pgfpathlineto{\pgfqpoint{3.256563in}{3.458376in}}%
\pgfpathlineto{\pgfqpoint{3.256563in}{3.586429in}}%
\pgfpathlineto{\pgfqpoint{3.145667in}{3.522403in}}%
\pgfpathlineto{\pgfqpoint{3.145667in}{3.394350in}}%
\pgfpathclose%
\pgfusepath{fill}%
\end{pgfscope}%
\begin{pgfscope}%
\pgfpathrectangle{\pgfqpoint{0.765000in}{0.660000in}}{\pgfqpoint{4.620000in}{4.620000in}}%
\pgfusepath{clip}%
\pgfsetbuttcap%
\pgfsetroundjoin%
\definecolor{currentfill}{rgb}{1.000000,0.894118,0.788235}%
\pgfsetfillcolor{currentfill}%
\pgfsetlinewidth{0.000000pt}%
\definecolor{currentstroke}{rgb}{1.000000,0.894118,0.788235}%
\pgfsetstrokecolor{currentstroke}%
\pgfsetdash{}{0pt}%
\pgfpathmoveto{\pgfqpoint{3.034770in}{3.458376in}}%
\pgfpathlineto{\pgfqpoint{3.145667in}{3.394350in}}%
\pgfpathlineto{\pgfqpoint{3.145667in}{3.522403in}}%
\pgfpathlineto{\pgfqpoint{3.034770in}{3.586429in}}%
\pgfpathlineto{\pgfqpoint{3.034770in}{3.458376in}}%
\pgfpathclose%
\pgfusepath{fill}%
\end{pgfscope}%
\begin{pgfscope}%
\pgfpathrectangle{\pgfqpoint{0.765000in}{0.660000in}}{\pgfqpoint{4.620000in}{4.620000in}}%
\pgfusepath{clip}%
\pgfsetbuttcap%
\pgfsetroundjoin%
\definecolor{currentfill}{rgb}{1.000000,0.894118,0.788235}%
\pgfsetfillcolor{currentfill}%
\pgfsetlinewidth{0.000000pt}%
\definecolor{currentstroke}{rgb}{1.000000,0.894118,0.788235}%
\pgfsetstrokecolor{currentstroke}%
\pgfsetdash{}{0pt}%
\pgfpathmoveto{\pgfqpoint{3.121227in}{3.527081in}}%
\pgfpathlineto{\pgfqpoint{3.010331in}{3.463055in}}%
\pgfpathlineto{\pgfqpoint{3.034770in}{3.458376in}}%
\pgfpathlineto{\pgfqpoint{3.145667in}{3.522403in}}%
\pgfpathlineto{\pgfqpoint{3.121227in}{3.527081in}}%
\pgfpathclose%
\pgfusepath{fill}%
\end{pgfscope}%
\begin{pgfscope}%
\pgfpathrectangle{\pgfqpoint{0.765000in}{0.660000in}}{\pgfqpoint{4.620000in}{4.620000in}}%
\pgfusepath{clip}%
\pgfsetbuttcap%
\pgfsetroundjoin%
\definecolor{currentfill}{rgb}{1.000000,0.894118,0.788235}%
\pgfsetfillcolor{currentfill}%
\pgfsetlinewidth{0.000000pt}%
\definecolor{currentstroke}{rgb}{1.000000,0.894118,0.788235}%
\pgfsetstrokecolor{currentstroke}%
\pgfsetdash{}{0pt}%
\pgfpathmoveto{\pgfqpoint{3.232124in}{3.463055in}}%
\pgfpathlineto{\pgfqpoint{3.121227in}{3.527081in}}%
\pgfpathlineto{\pgfqpoint{3.145667in}{3.522403in}}%
\pgfpathlineto{\pgfqpoint{3.256563in}{3.458376in}}%
\pgfpathlineto{\pgfqpoint{3.232124in}{3.463055in}}%
\pgfpathclose%
\pgfusepath{fill}%
\end{pgfscope}%
\begin{pgfscope}%
\pgfpathrectangle{\pgfqpoint{0.765000in}{0.660000in}}{\pgfqpoint{4.620000in}{4.620000in}}%
\pgfusepath{clip}%
\pgfsetbuttcap%
\pgfsetroundjoin%
\definecolor{currentfill}{rgb}{1.000000,0.894118,0.788235}%
\pgfsetfillcolor{currentfill}%
\pgfsetlinewidth{0.000000pt}%
\definecolor{currentstroke}{rgb}{1.000000,0.894118,0.788235}%
\pgfsetstrokecolor{currentstroke}%
\pgfsetdash{}{0pt}%
\pgfpathmoveto{\pgfqpoint{3.121227in}{3.527081in}}%
\pgfpathlineto{\pgfqpoint{3.121227in}{3.655134in}}%
\pgfpathlineto{\pgfqpoint{3.145667in}{3.650455in}}%
\pgfpathlineto{\pgfqpoint{3.256563in}{3.458376in}}%
\pgfpathlineto{\pgfqpoint{3.121227in}{3.527081in}}%
\pgfpathclose%
\pgfusepath{fill}%
\end{pgfscope}%
\begin{pgfscope}%
\pgfpathrectangle{\pgfqpoint{0.765000in}{0.660000in}}{\pgfqpoint{4.620000in}{4.620000in}}%
\pgfusepath{clip}%
\pgfsetbuttcap%
\pgfsetroundjoin%
\definecolor{currentfill}{rgb}{1.000000,0.894118,0.788235}%
\pgfsetfillcolor{currentfill}%
\pgfsetlinewidth{0.000000pt}%
\definecolor{currentstroke}{rgb}{1.000000,0.894118,0.788235}%
\pgfsetstrokecolor{currentstroke}%
\pgfsetdash{}{0pt}%
\pgfpathmoveto{\pgfqpoint{3.232124in}{3.463055in}}%
\pgfpathlineto{\pgfqpoint{3.232124in}{3.591108in}}%
\pgfpathlineto{\pgfqpoint{3.256563in}{3.586429in}}%
\pgfpathlineto{\pgfqpoint{3.145667in}{3.522403in}}%
\pgfpathlineto{\pgfqpoint{3.232124in}{3.463055in}}%
\pgfpathclose%
\pgfusepath{fill}%
\end{pgfscope}%
\begin{pgfscope}%
\pgfpathrectangle{\pgfqpoint{0.765000in}{0.660000in}}{\pgfqpoint{4.620000in}{4.620000in}}%
\pgfusepath{clip}%
\pgfsetbuttcap%
\pgfsetroundjoin%
\definecolor{currentfill}{rgb}{1.000000,0.894118,0.788235}%
\pgfsetfillcolor{currentfill}%
\pgfsetlinewidth{0.000000pt}%
\definecolor{currentstroke}{rgb}{1.000000,0.894118,0.788235}%
\pgfsetstrokecolor{currentstroke}%
\pgfsetdash{}{0pt}%
\pgfpathmoveto{\pgfqpoint{3.010331in}{3.463055in}}%
\pgfpathlineto{\pgfqpoint{3.121227in}{3.399029in}}%
\pgfpathlineto{\pgfqpoint{3.145667in}{3.394350in}}%
\pgfpathlineto{\pgfqpoint{3.034770in}{3.458376in}}%
\pgfpathlineto{\pgfqpoint{3.010331in}{3.463055in}}%
\pgfpathclose%
\pgfusepath{fill}%
\end{pgfscope}%
\begin{pgfscope}%
\pgfpathrectangle{\pgfqpoint{0.765000in}{0.660000in}}{\pgfqpoint{4.620000in}{4.620000in}}%
\pgfusepath{clip}%
\pgfsetbuttcap%
\pgfsetroundjoin%
\definecolor{currentfill}{rgb}{1.000000,0.894118,0.788235}%
\pgfsetfillcolor{currentfill}%
\pgfsetlinewidth{0.000000pt}%
\definecolor{currentstroke}{rgb}{1.000000,0.894118,0.788235}%
\pgfsetstrokecolor{currentstroke}%
\pgfsetdash{}{0pt}%
\pgfpathmoveto{\pgfqpoint{3.121227in}{3.399029in}}%
\pgfpathlineto{\pgfqpoint{3.232124in}{3.463055in}}%
\pgfpathlineto{\pgfqpoint{3.256563in}{3.458376in}}%
\pgfpathlineto{\pgfqpoint{3.145667in}{3.394350in}}%
\pgfpathlineto{\pgfqpoint{3.121227in}{3.399029in}}%
\pgfpathclose%
\pgfusepath{fill}%
\end{pgfscope}%
\begin{pgfscope}%
\pgfpathrectangle{\pgfqpoint{0.765000in}{0.660000in}}{\pgfqpoint{4.620000in}{4.620000in}}%
\pgfusepath{clip}%
\pgfsetbuttcap%
\pgfsetroundjoin%
\definecolor{currentfill}{rgb}{1.000000,0.894118,0.788235}%
\pgfsetfillcolor{currentfill}%
\pgfsetlinewidth{0.000000pt}%
\definecolor{currentstroke}{rgb}{1.000000,0.894118,0.788235}%
\pgfsetstrokecolor{currentstroke}%
\pgfsetdash{}{0pt}%
\pgfpathmoveto{\pgfqpoint{3.121227in}{3.655134in}}%
\pgfpathlineto{\pgfqpoint{3.010331in}{3.591108in}}%
\pgfpathlineto{\pgfqpoint{3.034770in}{3.586429in}}%
\pgfpathlineto{\pgfqpoint{3.145667in}{3.650455in}}%
\pgfpathlineto{\pgfqpoint{3.121227in}{3.655134in}}%
\pgfpathclose%
\pgfusepath{fill}%
\end{pgfscope}%
\begin{pgfscope}%
\pgfpathrectangle{\pgfqpoint{0.765000in}{0.660000in}}{\pgfqpoint{4.620000in}{4.620000in}}%
\pgfusepath{clip}%
\pgfsetbuttcap%
\pgfsetroundjoin%
\definecolor{currentfill}{rgb}{1.000000,0.894118,0.788235}%
\pgfsetfillcolor{currentfill}%
\pgfsetlinewidth{0.000000pt}%
\definecolor{currentstroke}{rgb}{1.000000,0.894118,0.788235}%
\pgfsetstrokecolor{currentstroke}%
\pgfsetdash{}{0pt}%
\pgfpathmoveto{\pgfqpoint{3.232124in}{3.591108in}}%
\pgfpathlineto{\pgfqpoint{3.121227in}{3.655134in}}%
\pgfpathlineto{\pgfqpoint{3.145667in}{3.650455in}}%
\pgfpathlineto{\pgfqpoint{3.256563in}{3.586429in}}%
\pgfpathlineto{\pgfqpoint{3.232124in}{3.591108in}}%
\pgfpathclose%
\pgfusepath{fill}%
\end{pgfscope}%
\begin{pgfscope}%
\pgfpathrectangle{\pgfqpoint{0.765000in}{0.660000in}}{\pgfqpoint{4.620000in}{4.620000in}}%
\pgfusepath{clip}%
\pgfsetbuttcap%
\pgfsetroundjoin%
\definecolor{currentfill}{rgb}{1.000000,0.894118,0.788235}%
\pgfsetfillcolor{currentfill}%
\pgfsetlinewidth{0.000000pt}%
\definecolor{currentstroke}{rgb}{1.000000,0.894118,0.788235}%
\pgfsetstrokecolor{currentstroke}%
\pgfsetdash{}{0pt}%
\pgfpathmoveto{\pgfqpoint{3.010331in}{3.463055in}}%
\pgfpathlineto{\pgfqpoint{3.010331in}{3.591108in}}%
\pgfpathlineto{\pgfqpoint{3.034770in}{3.586429in}}%
\pgfpathlineto{\pgfqpoint{3.145667in}{3.394350in}}%
\pgfpathlineto{\pgfqpoint{3.010331in}{3.463055in}}%
\pgfpathclose%
\pgfusepath{fill}%
\end{pgfscope}%
\begin{pgfscope}%
\pgfpathrectangle{\pgfqpoint{0.765000in}{0.660000in}}{\pgfqpoint{4.620000in}{4.620000in}}%
\pgfusepath{clip}%
\pgfsetbuttcap%
\pgfsetroundjoin%
\definecolor{currentfill}{rgb}{1.000000,0.894118,0.788235}%
\pgfsetfillcolor{currentfill}%
\pgfsetlinewidth{0.000000pt}%
\definecolor{currentstroke}{rgb}{1.000000,0.894118,0.788235}%
\pgfsetstrokecolor{currentstroke}%
\pgfsetdash{}{0pt}%
\pgfpathmoveto{\pgfqpoint{3.121227in}{3.399029in}}%
\pgfpathlineto{\pgfqpoint{3.121227in}{3.527081in}}%
\pgfpathlineto{\pgfqpoint{3.145667in}{3.522403in}}%
\pgfpathlineto{\pgfqpoint{3.034770in}{3.458376in}}%
\pgfpathlineto{\pgfqpoint{3.121227in}{3.399029in}}%
\pgfpathclose%
\pgfusepath{fill}%
\end{pgfscope}%
\begin{pgfscope}%
\pgfpathrectangle{\pgfqpoint{0.765000in}{0.660000in}}{\pgfqpoint{4.620000in}{4.620000in}}%
\pgfusepath{clip}%
\pgfsetbuttcap%
\pgfsetroundjoin%
\definecolor{currentfill}{rgb}{1.000000,0.894118,0.788235}%
\pgfsetfillcolor{currentfill}%
\pgfsetlinewidth{0.000000pt}%
\definecolor{currentstroke}{rgb}{1.000000,0.894118,0.788235}%
\pgfsetstrokecolor{currentstroke}%
\pgfsetdash{}{0pt}%
\pgfpathmoveto{\pgfqpoint{3.121227in}{3.527081in}}%
\pgfpathlineto{\pgfqpoint{3.232124in}{3.591108in}}%
\pgfpathlineto{\pgfqpoint{3.256563in}{3.586429in}}%
\pgfpathlineto{\pgfqpoint{3.145667in}{3.522403in}}%
\pgfpathlineto{\pgfqpoint{3.121227in}{3.527081in}}%
\pgfpathclose%
\pgfusepath{fill}%
\end{pgfscope}%
\begin{pgfscope}%
\pgfpathrectangle{\pgfqpoint{0.765000in}{0.660000in}}{\pgfqpoint{4.620000in}{4.620000in}}%
\pgfusepath{clip}%
\pgfsetbuttcap%
\pgfsetroundjoin%
\definecolor{currentfill}{rgb}{1.000000,0.894118,0.788235}%
\pgfsetfillcolor{currentfill}%
\pgfsetlinewidth{0.000000pt}%
\definecolor{currentstroke}{rgb}{1.000000,0.894118,0.788235}%
\pgfsetstrokecolor{currentstroke}%
\pgfsetdash{}{0pt}%
\pgfpathmoveto{\pgfqpoint{3.010331in}{3.591108in}}%
\pgfpathlineto{\pgfqpoint{3.121227in}{3.527081in}}%
\pgfpathlineto{\pgfqpoint{3.145667in}{3.522403in}}%
\pgfpathlineto{\pgfqpoint{3.034770in}{3.586429in}}%
\pgfpathlineto{\pgfqpoint{3.010331in}{3.591108in}}%
\pgfpathclose%
\pgfusepath{fill}%
\end{pgfscope}%
\begin{pgfscope}%
\pgfpathrectangle{\pgfqpoint{0.765000in}{0.660000in}}{\pgfqpoint{4.620000in}{4.620000in}}%
\pgfusepath{clip}%
\pgfsetbuttcap%
\pgfsetroundjoin%
\definecolor{currentfill}{rgb}{1.000000,0.894118,0.788235}%
\pgfsetfillcolor{currentfill}%
\pgfsetlinewidth{0.000000pt}%
\definecolor{currentstroke}{rgb}{1.000000,0.894118,0.788235}%
\pgfsetstrokecolor{currentstroke}%
\pgfsetdash{}{0pt}%
\pgfpathmoveto{\pgfqpoint{3.121227in}{3.527081in}}%
\pgfpathlineto{\pgfqpoint{3.010331in}{3.463055in}}%
\pgfpathlineto{\pgfqpoint{3.121227in}{3.399029in}}%
\pgfpathlineto{\pgfqpoint{3.232124in}{3.463055in}}%
\pgfpathlineto{\pgfqpoint{3.121227in}{3.527081in}}%
\pgfpathclose%
\pgfusepath{fill}%
\end{pgfscope}%
\begin{pgfscope}%
\pgfpathrectangle{\pgfqpoint{0.765000in}{0.660000in}}{\pgfqpoint{4.620000in}{4.620000in}}%
\pgfusepath{clip}%
\pgfsetbuttcap%
\pgfsetroundjoin%
\definecolor{currentfill}{rgb}{1.000000,0.894118,0.788235}%
\pgfsetfillcolor{currentfill}%
\pgfsetlinewidth{0.000000pt}%
\definecolor{currentstroke}{rgb}{1.000000,0.894118,0.788235}%
\pgfsetstrokecolor{currentstroke}%
\pgfsetdash{}{0pt}%
\pgfpathmoveto{\pgfqpoint{3.121227in}{3.527081in}}%
\pgfpathlineto{\pgfqpoint{3.010331in}{3.463055in}}%
\pgfpathlineto{\pgfqpoint{3.010331in}{3.591108in}}%
\pgfpathlineto{\pgfqpoint{3.121227in}{3.655134in}}%
\pgfpathlineto{\pgfqpoint{3.121227in}{3.527081in}}%
\pgfpathclose%
\pgfusepath{fill}%
\end{pgfscope}%
\begin{pgfscope}%
\pgfpathrectangle{\pgfqpoint{0.765000in}{0.660000in}}{\pgfqpoint{4.620000in}{4.620000in}}%
\pgfusepath{clip}%
\pgfsetbuttcap%
\pgfsetroundjoin%
\definecolor{currentfill}{rgb}{1.000000,0.894118,0.788235}%
\pgfsetfillcolor{currentfill}%
\pgfsetlinewidth{0.000000pt}%
\definecolor{currentstroke}{rgb}{1.000000,0.894118,0.788235}%
\pgfsetstrokecolor{currentstroke}%
\pgfsetdash{}{0pt}%
\pgfpathmoveto{\pgfqpoint{3.121227in}{3.527081in}}%
\pgfpathlineto{\pgfqpoint{3.232124in}{3.463055in}}%
\pgfpathlineto{\pgfqpoint{3.232124in}{3.591108in}}%
\pgfpathlineto{\pgfqpoint{3.121227in}{3.655134in}}%
\pgfpathlineto{\pgfqpoint{3.121227in}{3.527081in}}%
\pgfpathclose%
\pgfusepath{fill}%
\end{pgfscope}%
\begin{pgfscope}%
\pgfpathrectangle{\pgfqpoint{0.765000in}{0.660000in}}{\pgfqpoint{4.620000in}{4.620000in}}%
\pgfusepath{clip}%
\pgfsetbuttcap%
\pgfsetroundjoin%
\definecolor{currentfill}{rgb}{1.000000,0.894118,0.788235}%
\pgfsetfillcolor{currentfill}%
\pgfsetlinewidth{0.000000pt}%
\definecolor{currentstroke}{rgb}{1.000000,0.894118,0.788235}%
\pgfsetstrokecolor{currentstroke}%
\pgfsetdash{}{0pt}%
\pgfpathmoveto{\pgfqpoint{3.107502in}{3.529134in}}%
\pgfpathlineto{\pgfqpoint{2.996605in}{3.465108in}}%
\pgfpathlineto{\pgfqpoint{3.107502in}{3.401081in}}%
\pgfpathlineto{\pgfqpoint{3.218398in}{3.465108in}}%
\pgfpathlineto{\pgfqpoint{3.107502in}{3.529134in}}%
\pgfpathclose%
\pgfusepath{fill}%
\end{pgfscope}%
\begin{pgfscope}%
\pgfpathrectangle{\pgfqpoint{0.765000in}{0.660000in}}{\pgfqpoint{4.620000in}{4.620000in}}%
\pgfusepath{clip}%
\pgfsetbuttcap%
\pgfsetroundjoin%
\definecolor{currentfill}{rgb}{1.000000,0.894118,0.788235}%
\pgfsetfillcolor{currentfill}%
\pgfsetlinewidth{0.000000pt}%
\definecolor{currentstroke}{rgb}{1.000000,0.894118,0.788235}%
\pgfsetstrokecolor{currentstroke}%
\pgfsetdash{}{0pt}%
\pgfpathmoveto{\pgfqpoint{3.107502in}{3.529134in}}%
\pgfpathlineto{\pgfqpoint{2.996605in}{3.465108in}}%
\pgfpathlineto{\pgfqpoint{2.996605in}{3.593160in}}%
\pgfpathlineto{\pgfqpoint{3.107502in}{3.657186in}}%
\pgfpathlineto{\pgfqpoint{3.107502in}{3.529134in}}%
\pgfpathclose%
\pgfusepath{fill}%
\end{pgfscope}%
\begin{pgfscope}%
\pgfpathrectangle{\pgfqpoint{0.765000in}{0.660000in}}{\pgfqpoint{4.620000in}{4.620000in}}%
\pgfusepath{clip}%
\pgfsetbuttcap%
\pgfsetroundjoin%
\definecolor{currentfill}{rgb}{1.000000,0.894118,0.788235}%
\pgfsetfillcolor{currentfill}%
\pgfsetlinewidth{0.000000pt}%
\definecolor{currentstroke}{rgb}{1.000000,0.894118,0.788235}%
\pgfsetstrokecolor{currentstroke}%
\pgfsetdash{}{0pt}%
\pgfpathmoveto{\pgfqpoint{3.107502in}{3.529134in}}%
\pgfpathlineto{\pgfqpoint{3.218398in}{3.465108in}}%
\pgfpathlineto{\pgfqpoint{3.218398in}{3.593160in}}%
\pgfpathlineto{\pgfqpoint{3.107502in}{3.657186in}}%
\pgfpathlineto{\pgfqpoint{3.107502in}{3.529134in}}%
\pgfpathclose%
\pgfusepath{fill}%
\end{pgfscope}%
\begin{pgfscope}%
\pgfpathrectangle{\pgfqpoint{0.765000in}{0.660000in}}{\pgfqpoint{4.620000in}{4.620000in}}%
\pgfusepath{clip}%
\pgfsetbuttcap%
\pgfsetroundjoin%
\definecolor{currentfill}{rgb}{1.000000,0.894118,0.788235}%
\pgfsetfillcolor{currentfill}%
\pgfsetlinewidth{0.000000pt}%
\definecolor{currentstroke}{rgb}{1.000000,0.894118,0.788235}%
\pgfsetstrokecolor{currentstroke}%
\pgfsetdash{}{0pt}%
\pgfpathmoveto{\pgfqpoint{3.121227in}{3.655134in}}%
\pgfpathlineto{\pgfqpoint{3.010331in}{3.591108in}}%
\pgfpathlineto{\pgfqpoint{3.121227in}{3.527081in}}%
\pgfpathlineto{\pgfqpoint{3.232124in}{3.591108in}}%
\pgfpathlineto{\pgfqpoint{3.121227in}{3.655134in}}%
\pgfpathclose%
\pgfusepath{fill}%
\end{pgfscope}%
\begin{pgfscope}%
\pgfpathrectangle{\pgfqpoint{0.765000in}{0.660000in}}{\pgfqpoint{4.620000in}{4.620000in}}%
\pgfusepath{clip}%
\pgfsetbuttcap%
\pgfsetroundjoin%
\definecolor{currentfill}{rgb}{1.000000,0.894118,0.788235}%
\pgfsetfillcolor{currentfill}%
\pgfsetlinewidth{0.000000pt}%
\definecolor{currentstroke}{rgb}{1.000000,0.894118,0.788235}%
\pgfsetstrokecolor{currentstroke}%
\pgfsetdash{}{0pt}%
\pgfpathmoveto{\pgfqpoint{3.121227in}{3.399029in}}%
\pgfpathlineto{\pgfqpoint{3.232124in}{3.463055in}}%
\pgfpathlineto{\pgfqpoint{3.232124in}{3.591108in}}%
\pgfpathlineto{\pgfqpoint{3.121227in}{3.527081in}}%
\pgfpathlineto{\pgfqpoint{3.121227in}{3.399029in}}%
\pgfpathclose%
\pgfusepath{fill}%
\end{pgfscope}%
\begin{pgfscope}%
\pgfpathrectangle{\pgfqpoint{0.765000in}{0.660000in}}{\pgfqpoint{4.620000in}{4.620000in}}%
\pgfusepath{clip}%
\pgfsetbuttcap%
\pgfsetroundjoin%
\definecolor{currentfill}{rgb}{1.000000,0.894118,0.788235}%
\pgfsetfillcolor{currentfill}%
\pgfsetlinewidth{0.000000pt}%
\definecolor{currentstroke}{rgb}{1.000000,0.894118,0.788235}%
\pgfsetstrokecolor{currentstroke}%
\pgfsetdash{}{0pt}%
\pgfpathmoveto{\pgfqpoint{3.010331in}{3.463055in}}%
\pgfpathlineto{\pgfqpoint{3.121227in}{3.399029in}}%
\pgfpathlineto{\pgfqpoint{3.121227in}{3.527081in}}%
\pgfpathlineto{\pgfqpoint{3.010331in}{3.591108in}}%
\pgfpathlineto{\pgfqpoint{3.010331in}{3.463055in}}%
\pgfpathclose%
\pgfusepath{fill}%
\end{pgfscope}%
\begin{pgfscope}%
\pgfpathrectangle{\pgfqpoint{0.765000in}{0.660000in}}{\pgfqpoint{4.620000in}{4.620000in}}%
\pgfusepath{clip}%
\pgfsetbuttcap%
\pgfsetroundjoin%
\definecolor{currentfill}{rgb}{1.000000,0.894118,0.788235}%
\pgfsetfillcolor{currentfill}%
\pgfsetlinewidth{0.000000pt}%
\definecolor{currentstroke}{rgb}{1.000000,0.894118,0.788235}%
\pgfsetstrokecolor{currentstroke}%
\pgfsetdash{}{0pt}%
\pgfpathmoveto{\pgfqpoint{3.107502in}{3.657186in}}%
\pgfpathlineto{\pgfqpoint{2.996605in}{3.593160in}}%
\pgfpathlineto{\pgfqpoint{3.107502in}{3.529134in}}%
\pgfpathlineto{\pgfqpoint{3.218398in}{3.593160in}}%
\pgfpathlineto{\pgfqpoint{3.107502in}{3.657186in}}%
\pgfpathclose%
\pgfusepath{fill}%
\end{pgfscope}%
\begin{pgfscope}%
\pgfpathrectangle{\pgfqpoint{0.765000in}{0.660000in}}{\pgfqpoint{4.620000in}{4.620000in}}%
\pgfusepath{clip}%
\pgfsetbuttcap%
\pgfsetroundjoin%
\definecolor{currentfill}{rgb}{1.000000,0.894118,0.788235}%
\pgfsetfillcolor{currentfill}%
\pgfsetlinewidth{0.000000pt}%
\definecolor{currentstroke}{rgb}{1.000000,0.894118,0.788235}%
\pgfsetstrokecolor{currentstroke}%
\pgfsetdash{}{0pt}%
\pgfpathmoveto{\pgfqpoint{3.107502in}{3.401081in}}%
\pgfpathlineto{\pgfqpoint{3.218398in}{3.465108in}}%
\pgfpathlineto{\pgfqpoint{3.218398in}{3.593160in}}%
\pgfpathlineto{\pgfqpoint{3.107502in}{3.529134in}}%
\pgfpathlineto{\pgfqpoint{3.107502in}{3.401081in}}%
\pgfpathclose%
\pgfusepath{fill}%
\end{pgfscope}%
\begin{pgfscope}%
\pgfpathrectangle{\pgfqpoint{0.765000in}{0.660000in}}{\pgfqpoint{4.620000in}{4.620000in}}%
\pgfusepath{clip}%
\pgfsetbuttcap%
\pgfsetroundjoin%
\definecolor{currentfill}{rgb}{1.000000,0.894118,0.788235}%
\pgfsetfillcolor{currentfill}%
\pgfsetlinewidth{0.000000pt}%
\definecolor{currentstroke}{rgb}{1.000000,0.894118,0.788235}%
\pgfsetstrokecolor{currentstroke}%
\pgfsetdash{}{0pt}%
\pgfpathmoveto{\pgfqpoint{2.996605in}{3.465108in}}%
\pgfpathlineto{\pgfqpoint{3.107502in}{3.401081in}}%
\pgfpathlineto{\pgfqpoint{3.107502in}{3.529134in}}%
\pgfpathlineto{\pgfqpoint{2.996605in}{3.593160in}}%
\pgfpathlineto{\pgfqpoint{2.996605in}{3.465108in}}%
\pgfpathclose%
\pgfusepath{fill}%
\end{pgfscope}%
\begin{pgfscope}%
\pgfpathrectangle{\pgfqpoint{0.765000in}{0.660000in}}{\pgfqpoint{4.620000in}{4.620000in}}%
\pgfusepath{clip}%
\pgfsetbuttcap%
\pgfsetroundjoin%
\definecolor{currentfill}{rgb}{1.000000,0.894118,0.788235}%
\pgfsetfillcolor{currentfill}%
\pgfsetlinewidth{0.000000pt}%
\definecolor{currentstroke}{rgb}{1.000000,0.894118,0.788235}%
\pgfsetstrokecolor{currentstroke}%
\pgfsetdash{}{0pt}%
\pgfpathmoveto{\pgfqpoint{3.121227in}{3.527081in}}%
\pgfpathlineto{\pgfqpoint{3.010331in}{3.463055in}}%
\pgfpathlineto{\pgfqpoint{2.996605in}{3.465108in}}%
\pgfpathlineto{\pgfqpoint{3.107502in}{3.529134in}}%
\pgfpathlineto{\pgfqpoint{3.121227in}{3.527081in}}%
\pgfpathclose%
\pgfusepath{fill}%
\end{pgfscope}%
\begin{pgfscope}%
\pgfpathrectangle{\pgfqpoint{0.765000in}{0.660000in}}{\pgfqpoint{4.620000in}{4.620000in}}%
\pgfusepath{clip}%
\pgfsetbuttcap%
\pgfsetroundjoin%
\definecolor{currentfill}{rgb}{1.000000,0.894118,0.788235}%
\pgfsetfillcolor{currentfill}%
\pgfsetlinewidth{0.000000pt}%
\definecolor{currentstroke}{rgb}{1.000000,0.894118,0.788235}%
\pgfsetstrokecolor{currentstroke}%
\pgfsetdash{}{0pt}%
\pgfpathmoveto{\pgfqpoint{3.232124in}{3.463055in}}%
\pgfpathlineto{\pgfqpoint{3.121227in}{3.527081in}}%
\pgfpathlineto{\pgfqpoint{3.107502in}{3.529134in}}%
\pgfpathlineto{\pgfqpoint{3.218398in}{3.465108in}}%
\pgfpathlineto{\pgfqpoint{3.232124in}{3.463055in}}%
\pgfpathclose%
\pgfusepath{fill}%
\end{pgfscope}%
\begin{pgfscope}%
\pgfpathrectangle{\pgfqpoint{0.765000in}{0.660000in}}{\pgfqpoint{4.620000in}{4.620000in}}%
\pgfusepath{clip}%
\pgfsetbuttcap%
\pgfsetroundjoin%
\definecolor{currentfill}{rgb}{1.000000,0.894118,0.788235}%
\pgfsetfillcolor{currentfill}%
\pgfsetlinewidth{0.000000pt}%
\definecolor{currentstroke}{rgb}{1.000000,0.894118,0.788235}%
\pgfsetstrokecolor{currentstroke}%
\pgfsetdash{}{0pt}%
\pgfpathmoveto{\pgfqpoint{3.121227in}{3.527081in}}%
\pgfpathlineto{\pgfqpoint{3.121227in}{3.655134in}}%
\pgfpathlineto{\pgfqpoint{3.107502in}{3.657186in}}%
\pgfpathlineto{\pgfqpoint{3.218398in}{3.465108in}}%
\pgfpathlineto{\pgfqpoint{3.121227in}{3.527081in}}%
\pgfpathclose%
\pgfusepath{fill}%
\end{pgfscope}%
\begin{pgfscope}%
\pgfpathrectangle{\pgfqpoint{0.765000in}{0.660000in}}{\pgfqpoint{4.620000in}{4.620000in}}%
\pgfusepath{clip}%
\pgfsetbuttcap%
\pgfsetroundjoin%
\definecolor{currentfill}{rgb}{1.000000,0.894118,0.788235}%
\pgfsetfillcolor{currentfill}%
\pgfsetlinewidth{0.000000pt}%
\definecolor{currentstroke}{rgb}{1.000000,0.894118,0.788235}%
\pgfsetstrokecolor{currentstroke}%
\pgfsetdash{}{0pt}%
\pgfpathmoveto{\pgfqpoint{3.232124in}{3.463055in}}%
\pgfpathlineto{\pgfqpoint{3.232124in}{3.591108in}}%
\pgfpathlineto{\pgfqpoint{3.218398in}{3.593160in}}%
\pgfpathlineto{\pgfqpoint{3.107502in}{3.529134in}}%
\pgfpathlineto{\pgfqpoint{3.232124in}{3.463055in}}%
\pgfpathclose%
\pgfusepath{fill}%
\end{pgfscope}%
\begin{pgfscope}%
\pgfpathrectangle{\pgfqpoint{0.765000in}{0.660000in}}{\pgfqpoint{4.620000in}{4.620000in}}%
\pgfusepath{clip}%
\pgfsetbuttcap%
\pgfsetroundjoin%
\definecolor{currentfill}{rgb}{1.000000,0.894118,0.788235}%
\pgfsetfillcolor{currentfill}%
\pgfsetlinewidth{0.000000pt}%
\definecolor{currentstroke}{rgb}{1.000000,0.894118,0.788235}%
\pgfsetstrokecolor{currentstroke}%
\pgfsetdash{}{0pt}%
\pgfpathmoveto{\pgfqpoint{3.010331in}{3.463055in}}%
\pgfpathlineto{\pgfqpoint{3.121227in}{3.399029in}}%
\pgfpathlineto{\pgfqpoint{3.107502in}{3.401081in}}%
\pgfpathlineto{\pgfqpoint{2.996605in}{3.465108in}}%
\pgfpathlineto{\pgfqpoint{3.010331in}{3.463055in}}%
\pgfpathclose%
\pgfusepath{fill}%
\end{pgfscope}%
\begin{pgfscope}%
\pgfpathrectangle{\pgfqpoint{0.765000in}{0.660000in}}{\pgfqpoint{4.620000in}{4.620000in}}%
\pgfusepath{clip}%
\pgfsetbuttcap%
\pgfsetroundjoin%
\definecolor{currentfill}{rgb}{1.000000,0.894118,0.788235}%
\pgfsetfillcolor{currentfill}%
\pgfsetlinewidth{0.000000pt}%
\definecolor{currentstroke}{rgb}{1.000000,0.894118,0.788235}%
\pgfsetstrokecolor{currentstroke}%
\pgfsetdash{}{0pt}%
\pgfpathmoveto{\pgfqpoint{3.121227in}{3.399029in}}%
\pgfpathlineto{\pgfqpoint{3.232124in}{3.463055in}}%
\pgfpathlineto{\pgfqpoint{3.218398in}{3.465108in}}%
\pgfpathlineto{\pgfqpoint{3.107502in}{3.401081in}}%
\pgfpathlineto{\pgfqpoint{3.121227in}{3.399029in}}%
\pgfpathclose%
\pgfusepath{fill}%
\end{pgfscope}%
\begin{pgfscope}%
\pgfpathrectangle{\pgfqpoint{0.765000in}{0.660000in}}{\pgfqpoint{4.620000in}{4.620000in}}%
\pgfusepath{clip}%
\pgfsetbuttcap%
\pgfsetroundjoin%
\definecolor{currentfill}{rgb}{1.000000,0.894118,0.788235}%
\pgfsetfillcolor{currentfill}%
\pgfsetlinewidth{0.000000pt}%
\definecolor{currentstroke}{rgb}{1.000000,0.894118,0.788235}%
\pgfsetstrokecolor{currentstroke}%
\pgfsetdash{}{0pt}%
\pgfpathmoveto{\pgfqpoint{3.121227in}{3.655134in}}%
\pgfpathlineto{\pgfqpoint{3.010331in}{3.591108in}}%
\pgfpathlineto{\pgfqpoint{2.996605in}{3.593160in}}%
\pgfpathlineto{\pgfqpoint{3.107502in}{3.657186in}}%
\pgfpathlineto{\pgfqpoint{3.121227in}{3.655134in}}%
\pgfpathclose%
\pgfusepath{fill}%
\end{pgfscope}%
\begin{pgfscope}%
\pgfpathrectangle{\pgfqpoint{0.765000in}{0.660000in}}{\pgfqpoint{4.620000in}{4.620000in}}%
\pgfusepath{clip}%
\pgfsetbuttcap%
\pgfsetroundjoin%
\definecolor{currentfill}{rgb}{1.000000,0.894118,0.788235}%
\pgfsetfillcolor{currentfill}%
\pgfsetlinewidth{0.000000pt}%
\definecolor{currentstroke}{rgb}{1.000000,0.894118,0.788235}%
\pgfsetstrokecolor{currentstroke}%
\pgfsetdash{}{0pt}%
\pgfpathmoveto{\pgfqpoint{3.232124in}{3.591108in}}%
\pgfpathlineto{\pgfqpoint{3.121227in}{3.655134in}}%
\pgfpathlineto{\pgfqpoint{3.107502in}{3.657186in}}%
\pgfpathlineto{\pgfqpoint{3.218398in}{3.593160in}}%
\pgfpathlineto{\pgfqpoint{3.232124in}{3.591108in}}%
\pgfpathclose%
\pgfusepath{fill}%
\end{pgfscope}%
\begin{pgfscope}%
\pgfpathrectangle{\pgfqpoint{0.765000in}{0.660000in}}{\pgfqpoint{4.620000in}{4.620000in}}%
\pgfusepath{clip}%
\pgfsetbuttcap%
\pgfsetroundjoin%
\definecolor{currentfill}{rgb}{1.000000,0.894118,0.788235}%
\pgfsetfillcolor{currentfill}%
\pgfsetlinewidth{0.000000pt}%
\definecolor{currentstroke}{rgb}{1.000000,0.894118,0.788235}%
\pgfsetstrokecolor{currentstroke}%
\pgfsetdash{}{0pt}%
\pgfpathmoveto{\pgfqpoint{3.010331in}{3.463055in}}%
\pgfpathlineto{\pgfqpoint{3.010331in}{3.591108in}}%
\pgfpathlineto{\pgfqpoint{2.996605in}{3.593160in}}%
\pgfpathlineto{\pgfqpoint{3.107502in}{3.401081in}}%
\pgfpathlineto{\pgfqpoint{3.010331in}{3.463055in}}%
\pgfpathclose%
\pgfusepath{fill}%
\end{pgfscope}%
\begin{pgfscope}%
\pgfpathrectangle{\pgfqpoint{0.765000in}{0.660000in}}{\pgfqpoint{4.620000in}{4.620000in}}%
\pgfusepath{clip}%
\pgfsetbuttcap%
\pgfsetroundjoin%
\definecolor{currentfill}{rgb}{1.000000,0.894118,0.788235}%
\pgfsetfillcolor{currentfill}%
\pgfsetlinewidth{0.000000pt}%
\definecolor{currentstroke}{rgb}{1.000000,0.894118,0.788235}%
\pgfsetstrokecolor{currentstroke}%
\pgfsetdash{}{0pt}%
\pgfpathmoveto{\pgfqpoint{3.121227in}{3.399029in}}%
\pgfpathlineto{\pgfqpoint{3.121227in}{3.527081in}}%
\pgfpathlineto{\pgfqpoint{3.107502in}{3.529134in}}%
\pgfpathlineto{\pgfqpoint{2.996605in}{3.465108in}}%
\pgfpathlineto{\pgfqpoint{3.121227in}{3.399029in}}%
\pgfpathclose%
\pgfusepath{fill}%
\end{pgfscope}%
\begin{pgfscope}%
\pgfpathrectangle{\pgfqpoint{0.765000in}{0.660000in}}{\pgfqpoint{4.620000in}{4.620000in}}%
\pgfusepath{clip}%
\pgfsetbuttcap%
\pgfsetroundjoin%
\definecolor{currentfill}{rgb}{1.000000,0.894118,0.788235}%
\pgfsetfillcolor{currentfill}%
\pgfsetlinewidth{0.000000pt}%
\definecolor{currentstroke}{rgb}{1.000000,0.894118,0.788235}%
\pgfsetstrokecolor{currentstroke}%
\pgfsetdash{}{0pt}%
\pgfpathmoveto{\pgfqpoint{3.121227in}{3.527081in}}%
\pgfpathlineto{\pgfqpoint{3.232124in}{3.591108in}}%
\pgfpathlineto{\pgfqpoint{3.218398in}{3.593160in}}%
\pgfpathlineto{\pgfqpoint{3.107502in}{3.529134in}}%
\pgfpathlineto{\pgfqpoint{3.121227in}{3.527081in}}%
\pgfpathclose%
\pgfusepath{fill}%
\end{pgfscope}%
\begin{pgfscope}%
\pgfpathrectangle{\pgfqpoint{0.765000in}{0.660000in}}{\pgfqpoint{4.620000in}{4.620000in}}%
\pgfusepath{clip}%
\pgfsetbuttcap%
\pgfsetroundjoin%
\definecolor{currentfill}{rgb}{1.000000,0.894118,0.788235}%
\pgfsetfillcolor{currentfill}%
\pgfsetlinewidth{0.000000pt}%
\definecolor{currentstroke}{rgb}{1.000000,0.894118,0.788235}%
\pgfsetstrokecolor{currentstroke}%
\pgfsetdash{}{0pt}%
\pgfpathmoveto{\pgfqpoint{3.010331in}{3.591108in}}%
\pgfpathlineto{\pgfqpoint{3.121227in}{3.527081in}}%
\pgfpathlineto{\pgfqpoint{3.107502in}{3.529134in}}%
\pgfpathlineto{\pgfqpoint{2.996605in}{3.593160in}}%
\pgfpathlineto{\pgfqpoint{3.010331in}{3.591108in}}%
\pgfpathclose%
\pgfusepath{fill}%
\end{pgfscope}%
\begin{pgfscope}%
\pgfpathrectangle{\pgfqpoint{0.765000in}{0.660000in}}{\pgfqpoint{4.620000in}{4.620000in}}%
\pgfusepath{clip}%
\pgfsetbuttcap%
\pgfsetroundjoin%
\definecolor{currentfill}{rgb}{1.000000,0.894118,0.788235}%
\pgfsetfillcolor{currentfill}%
\pgfsetlinewidth{0.000000pt}%
\definecolor{currentstroke}{rgb}{1.000000,0.894118,0.788235}%
\pgfsetstrokecolor{currentstroke}%
\pgfsetdash{}{0pt}%
\pgfpathmoveto{\pgfqpoint{3.145667in}{3.522403in}}%
\pgfpathlineto{\pgfqpoint{3.034770in}{3.458376in}}%
\pgfpathlineto{\pgfqpoint{3.145667in}{3.394350in}}%
\pgfpathlineto{\pgfqpoint{3.256563in}{3.458376in}}%
\pgfpathlineto{\pgfqpoint{3.145667in}{3.522403in}}%
\pgfpathclose%
\pgfusepath{fill}%
\end{pgfscope}%
\begin{pgfscope}%
\pgfpathrectangle{\pgfqpoint{0.765000in}{0.660000in}}{\pgfqpoint{4.620000in}{4.620000in}}%
\pgfusepath{clip}%
\pgfsetbuttcap%
\pgfsetroundjoin%
\definecolor{currentfill}{rgb}{1.000000,0.894118,0.788235}%
\pgfsetfillcolor{currentfill}%
\pgfsetlinewidth{0.000000pt}%
\definecolor{currentstroke}{rgb}{1.000000,0.894118,0.788235}%
\pgfsetstrokecolor{currentstroke}%
\pgfsetdash{}{0pt}%
\pgfpathmoveto{\pgfqpoint{3.145667in}{3.522403in}}%
\pgfpathlineto{\pgfqpoint{3.034770in}{3.458376in}}%
\pgfpathlineto{\pgfqpoint{3.034770in}{3.586429in}}%
\pgfpathlineto{\pgfqpoint{3.145667in}{3.650455in}}%
\pgfpathlineto{\pgfqpoint{3.145667in}{3.522403in}}%
\pgfpathclose%
\pgfusepath{fill}%
\end{pgfscope}%
\begin{pgfscope}%
\pgfpathrectangle{\pgfqpoint{0.765000in}{0.660000in}}{\pgfqpoint{4.620000in}{4.620000in}}%
\pgfusepath{clip}%
\pgfsetbuttcap%
\pgfsetroundjoin%
\definecolor{currentfill}{rgb}{1.000000,0.894118,0.788235}%
\pgfsetfillcolor{currentfill}%
\pgfsetlinewidth{0.000000pt}%
\definecolor{currentstroke}{rgb}{1.000000,0.894118,0.788235}%
\pgfsetstrokecolor{currentstroke}%
\pgfsetdash{}{0pt}%
\pgfpathmoveto{\pgfqpoint{3.145667in}{3.522403in}}%
\pgfpathlineto{\pgfqpoint{3.256563in}{3.458376in}}%
\pgfpathlineto{\pgfqpoint{3.256563in}{3.586429in}}%
\pgfpathlineto{\pgfqpoint{3.145667in}{3.650455in}}%
\pgfpathlineto{\pgfqpoint{3.145667in}{3.522403in}}%
\pgfpathclose%
\pgfusepath{fill}%
\end{pgfscope}%
\begin{pgfscope}%
\pgfpathrectangle{\pgfqpoint{0.765000in}{0.660000in}}{\pgfqpoint{4.620000in}{4.620000in}}%
\pgfusepath{clip}%
\pgfsetbuttcap%
\pgfsetroundjoin%
\definecolor{currentfill}{rgb}{1.000000,0.894118,0.788235}%
\pgfsetfillcolor{currentfill}%
\pgfsetlinewidth{0.000000pt}%
\definecolor{currentstroke}{rgb}{1.000000,0.894118,0.788235}%
\pgfsetstrokecolor{currentstroke}%
\pgfsetdash{}{0pt}%
\pgfpathmoveto{\pgfqpoint{3.121227in}{3.527081in}}%
\pgfpathlineto{\pgfqpoint{3.010331in}{3.463055in}}%
\pgfpathlineto{\pgfqpoint{3.121227in}{3.399029in}}%
\pgfpathlineto{\pgfqpoint{3.232124in}{3.463055in}}%
\pgfpathlineto{\pgfqpoint{3.121227in}{3.527081in}}%
\pgfpathclose%
\pgfusepath{fill}%
\end{pgfscope}%
\begin{pgfscope}%
\pgfpathrectangle{\pgfqpoint{0.765000in}{0.660000in}}{\pgfqpoint{4.620000in}{4.620000in}}%
\pgfusepath{clip}%
\pgfsetbuttcap%
\pgfsetroundjoin%
\definecolor{currentfill}{rgb}{1.000000,0.894118,0.788235}%
\pgfsetfillcolor{currentfill}%
\pgfsetlinewidth{0.000000pt}%
\definecolor{currentstroke}{rgb}{1.000000,0.894118,0.788235}%
\pgfsetstrokecolor{currentstroke}%
\pgfsetdash{}{0pt}%
\pgfpathmoveto{\pgfqpoint{3.121227in}{3.527081in}}%
\pgfpathlineto{\pgfqpoint{3.010331in}{3.463055in}}%
\pgfpathlineto{\pgfqpoint{3.010331in}{3.591108in}}%
\pgfpathlineto{\pgfqpoint{3.121227in}{3.655134in}}%
\pgfpathlineto{\pgfqpoint{3.121227in}{3.527081in}}%
\pgfpathclose%
\pgfusepath{fill}%
\end{pgfscope}%
\begin{pgfscope}%
\pgfpathrectangle{\pgfqpoint{0.765000in}{0.660000in}}{\pgfqpoint{4.620000in}{4.620000in}}%
\pgfusepath{clip}%
\pgfsetbuttcap%
\pgfsetroundjoin%
\definecolor{currentfill}{rgb}{1.000000,0.894118,0.788235}%
\pgfsetfillcolor{currentfill}%
\pgfsetlinewidth{0.000000pt}%
\definecolor{currentstroke}{rgb}{1.000000,0.894118,0.788235}%
\pgfsetstrokecolor{currentstroke}%
\pgfsetdash{}{0pt}%
\pgfpathmoveto{\pgfqpoint{3.121227in}{3.527081in}}%
\pgfpathlineto{\pgfqpoint{3.232124in}{3.463055in}}%
\pgfpathlineto{\pgfqpoint{3.232124in}{3.591108in}}%
\pgfpathlineto{\pgfqpoint{3.121227in}{3.655134in}}%
\pgfpathlineto{\pgfqpoint{3.121227in}{3.527081in}}%
\pgfpathclose%
\pgfusepath{fill}%
\end{pgfscope}%
\begin{pgfscope}%
\pgfpathrectangle{\pgfqpoint{0.765000in}{0.660000in}}{\pgfqpoint{4.620000in}{4.620000in}}%
\pgfusepath{clip}%
\pgfsetbuttcap%
\pgfsetroundjoin%
\definecolor{currentfill}{rgb}{1.000000,0.894118,0.788235}%
\pgfsetfillcolor{currentfill}%
\pgfsetlinewidth{0.000000pt}%
\definecolor{currentstroke}{rgb}{1.000000,0.894118,0.788235}%
\pgfsetstrokecolor{currentstroke}%
\pgfsetdash{}{0pt}%
\pgfpathmoveto{\pgfqpoint{3.145667in}{3.650455in}}%
\pgfpathlineto{\pgfqpoint{3.034770in}{3.586429in}}%
\pgfpathlineto{\pgfqpoint{3.145667in}{3.522403in}}%
\pgfpathlineto{\pgfqpoint{3.256563in}{3.586429in}}%
\pgfpathlineto{\pgfqpoint{3.145667in}{3.650455in}}%
\pgfpathclose%
\pgfusepath{fill}%
\end{pgfscope}%
\begin{pgfscope}%
\pgfpathrectangle{\pgfqpoint{0.765000in}{0.660000in}}{\pgfqpoint{4.620000in}{4.620000in}}%
\pgfusepath{clip}%
\pgfsetbuttcap%
\pgfsetroundjoin%
\definecolor{currentfill}{rgb}{1.000000,0.894118,0.788235}%
\pgfsetfillcolor{currentfill}%
\pgfsetlinewidth{0.000000pt}%
\definecolor{currentstroke}{rgb}{1.000000,0.894118,0.788235}%
\pgfsetstrokecolor{currentstroke}%
\pgfsetdash{}{0pt}%
\pgfpathmoveto{\pgfqpoint{3.145667in}{3.394350in}}%
\pgfpathlineto{\pgfqpoint{3.256563in}{3.458376in}}%
\pgfpathlineto{\pgfqpoint{3.256563in}{3.586429in}}%
\pgfpathlineto{\pgfqpoint{3.145667in}{3.522403in}}%
\pgfpathlineto{\pgfqpoint{3.145667in}{3.394350in}}%
\pgfpathclose%
\pgfusepath{fill}%
\end{pgfscope}%
\begin{pgfscope}%
\pgfpathrectangle{\pgfqpoint{0.765000in}{0.660000in}}{\pgfqpoint{4.620000in}{4.620000in}}%
\pgfusepath{clip}%
\pgfsetbuttcap%
\pgfsetroundjoin%
\definecolor{currentfill}{rgb}{1.000000,0.894118,0.788235}%
\pgfsetfillcolor{currentfill}%
\pgfsetlinewidth{0.000000pt}%
\definecolor{currentstroke}{rgb}{1.000000,0.894118,0.788235}%
\pgfsetstrokecolor{currentstroke}%
\pgfsetdash{}{0pt}%
\pgfpathmoveto{\pgfqpoint{3.034770in}{3.458376in}}%
\pgfpathlineto{\pgfqpoint{3.145667in}{3.394350in}}%
\pgfpathlineto{\pgfqpoint{3.145667in}{3.522403in}}%
\pgfpathlineto{\pgfqpoint{3.034770in}{3.586429in}}%
\pgfpathlineto{\pgfqpoint{3.034770in}{3.458376in}}%
\pgfpathclose%
\pgfusepath{fill}%
\end{pgfscope}%
\begin{pgfscope}%
\pgfpathrectangle{\pgfqpoint{0.765000in}{0.660000in}}{\pgfqpoint{4.620000in}{4.620000in}}%
\pgfusepath{clip}%
\pgfsetbuttcap%
\pgfsetroundjoin%
\definecolor{currentfill}{rgb}{1.000000,0.894118,0.788235}%
\pgfsetfillcolor{currentfill}%
\pgfsetlinewidth{0.000000pt}%
\definecolor{currentstroke}{rgb}{1.000000,0.894118,0.788235}%
\pgfsetstrokecolor{currentstroke}%
\pgfsetdash{}{0pt}%
\pgfpathmoveto{\pgfqpoint{3.121227in}{3.655134in}}%
\pgfpathlineto{\pgfqpoint{3.010331in}{3.591108in}}%
\pgfpathlineto{\pgfqpoint{3.121227in}{3.527081in}}%
\pgfpathlineto{\pgfqpoint{3.232124in}{3.591108in}}%
\pgfpathlineto{\pgfqpoint{3.121227in}{3.655134in}}%
\pgfpathclose%
\pgfusepath{fill}%
\end{pgfscope}%
\begin{pgfscope}%
\pgfpathrectangle{\pgfqpoint{0.765000in}{0.660000in}}{\pgfqpoint{4.620000in}{4.620000in}}%
\pgfusepath{clip}%
\pgfsetbuttcap%
\pgfsetroundjoin%
\definecolor{currentfill}{rgb}{1.000000,0.894118,0.788235}%
\pgfsetfillcolor{currentfill}%
\pgfsetlinewidth{0.000000pt}%
\definecolor{currentstroke}{rgb}{1.000000,0.894118,0.788235}%
\pgfsetstrokecolor{currentstroke}%
\pgfsetdash{}{0pt}%
\pgfpathmoveto{\pgfqpoint{3.121227in}{3.399029in}}%
\pgfpathlineto{\pgfqpoint{3.232124in}{3.463055in}}%
\pgfpathlineto{\pgfqpoint{3.232124in}{3.591108in}}%
\pgfpathlineto{\pgfqpoint{3.121227in}{3.527081in}}%
\pgfpathlineto{\pgfqpoint{3.121227in}{3.399029in}}%
\pgfpathclose%
\pgfusepath{fill}%
\end{pgfscope}%
\begin{pgfscope}%
\pgfpathrectangle{\pgfqpoint{0.765000in}{0.660000in}}{\pgfqpoint{4.620000in}{4.620000in}}%
\pgfusepath{clip}%
\pgfsetbuttcap%
\pgfsetroundjoin%
\definecolor{currentfill}{rgb}{1.000000,0.894118,0.788235}%
\pgfsetfillcolor{currentfill}%
\pgfsetlinewidth{0.000000pt}%
\definecolor{currentstroke}{rgb}{1.000000,0.894118,0.788235}%
\pgfsetstrokecolor{currentstroke}%
\pgfsetdash{}{0pt}%
\pgfpathmoveto{\pgfqpoint{3.010331in}{3.463055in}}%
\pgfpathlineto{\pgfqpoint{3.121227in}{3.399029in}}%
\pgfpathlineto{\pgfqpoint{3.121227in}{3.527081in}}%
\pgfpathlineto{\pgfqpoint{3.010331in}{3.591108in}}%
\pgfpathlineto{\pgfqpoint{3.010331in}{3.463055in}}%
\pgfpathclose%
\pgfusepath{fill}%
\end{pgfscope}%
\begin{pgfscope}%
\pgfpathrectangle{\pgfqpoint{0.765000in}{0.660000in}}{\pgfqpoint{4.620000in}{4.620000in}}%
\pgfusepath{clip}%
\pgfsetbuttcap%
\pgfsetroundjoin%
\definecolor{currentfill}{rgb}{1.000000,0.894118,0.788235}%
\pgfsetfillcolor{currentfill}%
\pgfsetlinewidth{0.000000pt}%
\definecolor{currentstroke}{rgb}{1.000000,0.894118,0.788235}%
\pgfsetstrokecolor{currentstroke}%
\pgfsetdash{}{0pt}%
\pgfpathmoveto{\pgfqpoint{3.145667in}{3.522403in}}%
\pgfpathlineto{\pgfqpoint{3.034770in}{3.458376in}}%
\pgfpathlineto{\pgfqpoint{3.010331in}{3.463055in}}%
\pgfpathlineto{\pgfqpoint{3.121227in}{3.527081in}}%
\pgfpathlineto{\pgfqpoint{3.145667in}{3.522403in}}%
\pgfpathclose%
\pgfusepath{fill}%
\end{pgfscope}%
\begin{pgfscope}%
\pgfpathrectangle{\pgfqpoint{0.765000in}{0.660000in}}{\pgfqpoint{4.620000in}{4.620000in}}%
\pgfusepath{clip}%
\pgfsetbuttcap%
\pgfsetroundjoin%
\definecolor{currentfill}{rgb}{1.000000,0.894118,0.788235}%
\pgfsetfillcolor{currentfill}%
\pgfsetlinewidth{0.000000pt}%
\definecolor{currentstroke}{rgb}{1.000000,0.894118,0.788235}%
\pgfsetstrokecolor{currentstroke}%
\pgfsetdash{}{0pt}%
\pgfpathmoveto{\pgfqpoint{3.256563in}{3.458376in}}%
\pgfpathlineto{\pgfqpoint{3.145667in}{3.522403in}}%
\pgfpathlineto{\pgfqpoint{3.121227in}{3.527081in}}%
\pgfpathlineto{\pgfqpoint{3.232124in}{3.463055in}}%
\pgfpathlineto{\pgfqpoint{3.256563in}{3.458376in}}%
\pgfpathclose%
\pgfusepath{fill}%
\end{pgfscope}%
\begin{pgfscope}%
\pgfpathrectangle{\pgfqpoint{0.765000in}{0.660000in}}{\pgfqpoint{4.620000in}{4.620000in}}%
\pgfusepath{clip}%
\pgfsetbuttcap%
\pgfsetroundjoin%
\definecolor{currentfill}{rgb}{1.000000,0.894118,0.788235}%
\pgfsetfillcolor{currentfill}%
\pgfsetlinewidth{0.000000pt}%
\definecolor{currentstroke}{rgb}{1.000000,0.894118,0.788235}%
\pgfsetstrokecolor{currentstroke}%
\pgfsetdash{}{0pt}%
\pgfpathmoveto{\pgfqpoint{3.145667in}{3.522403in}}%
\pgfpathlineto{\pgfqpoint{3.145667in}{3.650455in}}%
\pgfpathlineto{\pgfqpoint{3.121227in}{3.655134in}}%
\pgfpathlineto{\pgfqpoint{3.232124in}{3.463055in}}%
\pgfpathlineto{\pgfqpoint{3.145667in}{3.522403in}}%
\pgfpathclose%
\pgfusepath{fill}%
\end{pgfscope}%
\begin{pgfscope}%
\pgfpathrectangle{\pgfqpoint{0.765000in}{0.660000in}}{\pgfqpoint{4.620000in}{4.620000in}}%
\pgfusepath{clip}%
\pgfsetbuttcap%
\pgfsetroundjoin%
\definecolor{currentfill}{rgb}{1.000000,0.894118,0.788235}%
\pgfsetfillcolor{currentfill}%
\pgfsetlinewidth{0.000000pt}%
\definecolor{currentstroke}{rgb}{1.000000,0.894118,0.788235}%
\pgfsetstrokecolor{currentstroke}%
\pgfsetdash{}{0pt}%
\pgfpathmoveto{\pgfqpoint{3.256563in}{3.458376in}}%
\pgfpathlineto{\pgfqpoint{3.256563in}{3.586429in}}%
\pgfpathlineto{\pgfqpoint{3.232124in}{3.591108in}}%
\pgfpathlineto{\pgfqpoint{3.121227in}{3.527081in}}%
\pgfpathlineto{\pgfqpoint{3.256563in}{3.458376in}}%
\pgfpathclose%
\pgfusepath{fill}%
\end{pgfscope}%
\begin{pgfscope}%
\pgfpathrectangle{\pgfqpoint{0.765000in}{0.660000in}}{\pgfqpoint{4.620000in}{4.620000in}}%
\pgfusepath{clip}%
\pgfsetbuttcap%
\pgfsetroundjoin%
\definecolor{currentfill}{rgb}{1.000000,0.894118,0.788235}%
\pgfsetfillcolor{currentfill}%
\pgfsetlinewidth{0.000000pt}%
\definecolor{currentstroke}{rgb}{1.000000,0.894118,0.788235}%
\pgfsetstrokecolor{currentstroke}%
\pgfsetdash{}{0pt}%
\pgfpathmoveto{\pgfqpoint{3.034770in}{3.458376in}}%
\pgfpathlineto{\pgfqpoint{3.145667in}{3.394350in}}%
\pgfpathlineto{\pgfqpoint{3.121227in}{3.399029in}}%
\pgfpathlineto{\pgfqpoint{3.010331in}{3.463055in}}%
\pgfpathlineto{\pgfqpoint{3.034770in}{3.458376in}}%
\pgfpathclose%
\pgfusepath{fill}%
\end{pgfscope}%
\begin{pgfscope}%
\pgfpathrectangle{\pgfqpoint{0.765000in}{0.660000in}}{\pgfqpoint{4.620000in}{4.620000in}}%
\pgfusepath{clip}%
\pgfsetbuttcap%
\pgfsetroundjoin%
\definecolor{currentfill}{rgb}{1.000000,0.894118,0.788235}%
\pgfsetfillcolor{currentfill}%
\pgfsetlinewidth{0.000000pt}%
\definecolor{currentstroke}{rgb}{1.000000,0.894118,0.788235}%
\pgfsetstrokecolor{currentstroke}%
\pgfsetdash{}{0pt}%
\pgfpathmoveto{\pgfqpoint{3.145667in}{3.394350in}}%
\pgfpathlineto{\pgfqpoint{3.256563in}{3.458376in}}%
\pgfpathlineto{\pgfqpoint{3.232124in}{3.463055in}}%
\pgfpathlineto{\pgfqpoint{3.121227in}{3.399029in}}%
\pgfpathlineto{\pgfqpoint{3.145667in}{3.394350in}}%
\pgfpathclose%
\pgfusepath{fill}%
\end{pgfscope}%
\begin{pgfscope}%
\pgfpathrectangle{\pgfqpoint{0.765000in}{0.660000in}}{\pgfqpoint{4.620000in}{4.620000in}}%
\pgfusepath{clip}%
\pgfsetbuttcap%
\pgfsetroundjoin%
\definecolor{currentfill}{rgb}{1.000000,0.894118,0.788235}%
\pgfsetfillcolor{currentfill}%
\pgfsetlinewidth{0.000000pt}%
\definecolor{currentstroke}{rgb}{1.000000,0.894118,0.788235}%
\pgfsetstrokecolor{currentstroke}%
\pgfsetdash{}{0pt}%
\pgfpathmoveto{\pgfqpoint{3.145667in}{3.650455in}}%
\pgfpathlineto{\pgfqpoint{3.034770in}{3.586429in}}%
\pgfpathlineto{\pgfqpoint{3.010331in}{3.591108in}}%
\pgfpathlineto{\pgfqpoint{3.121227in}{3.655134in}}%
\pgfpathlineto{\pgfqpoint{3.145667in}{3.650455in}}%
\pgfpathclose%
\pgfusepath{fill}%
\end{pgfscope}%
\begin{pgfscope}%
\pgfpathrectangle{\pgfqpoint{0.765000in}{0.660000in}}{\pgfqpoint{4.620000in}{4.620000in}}%
\pgfusepath{clip}%
\pgfsetbuttcap%
\pgfsetroundjoin%
\definecolor{currentfill}{rgb}{1.000000,0.894118,0.788235}%
\pgfsetfillcolor{currentfill}%
\pgfsetlinewidth{0.000000pt}%
\definecolor{currentstroke}{rgb}{1.000000,0.894118,0.788235}%
\pgfsetstrokecolor{currentstroke}%
\pgfsetdash{}{0pt}%
\pgfpathmoveto{\pgfqpoint{3.256563in}{3.586429in}}%
\pgfpathlineto{\pgfqpoint{3.145667in}{3.650455in}}%
\pgfpathlineto{\pgfqpoint{3.121227in}{3.655134in}}%
\pgfpathlineto{\pgfqpoint{3.232124in}{3.591108in}}%
\pgfpathlineto{\pgfqpoint{3.256563in}{3.586429in}}%
\pgfpathclose%
\pgfusepath{fill}%
\end{pgfscope}%
\begin{pgfscope}%
\pgfpathrectangle{\pgfqpoint{0.765000in}{0.660000in}}{\pgfqpoint{4.620000in}{4.620000in}}%
\pgfusepath{clip}%
\pgfsetbuttcap%
\pgfsetroundjoin%
\definecolor{currentfill}{rgb}{1.000000,0.894118,0.788235}%
\pgfsetfillcolor{currentfill}%
\pgfsetlinewidth{0.000000pt}%
\definecolor{currentstroke}{rgb}{1.000000,0.894118,0.788235}%
\pgfsetstrokecolor{currentstroke}%
\pgfsetdash{}{0pt}%
\pgfpathmoveto{\pgfqpoint{3.034770in}{3.458376in}}%
\pgfpathlineto{\pgfqpoint{3.034770in}{3.586429in}}%
\pgfpathlineto{\pgfqpoint{3.010331in}{3.591108in}}%
\pgfpathlineto{\pgfqpoint{3.121227in}{3.399029in}}%
\pgfpathlineto{\pgfqpoint{3.034770in}{3.458376in}}%
\pgfpathclose%
\pgfusepath{fill}%
\end{pgfscope}%
\begin{pgfscope}%
\pgfpathrectangle{\pgfqpoint{0.765000in}{0.660000in}}{\pgfqpoint{4.620000in}{4.620000in}}%
\pgfusepath{clip}%
\pgfsetbuttcap%
\pgfsetroundjoin%
\definecolor{currentfill}{rgb}{1.000000,0.894118,0.788235}%
\pgfsetfillcolor{currentfill}%
\pgfsetlinewidth{0.000000pt}%
\definecolor{currentstroke}{rgb}{1.000000,0.894118,0.788235}%
\pgfsetstrokecolor{currentstroke}%
\pgfsetdash{}{0pt}%
\pgfpathmoveto{\pgfqpoint{3.145667in}{3.394350in}}%
\pgfpathlineto{\pgfqpoint{3.145667in}{3.522403in}}%
\pgfpathlineto{\pgfqpoint{3.121227in}{3.527081in}}%
\pgfpathlineto{\pgfqpoint{3.010331in}{3.463055in}}%
\pgfpathlineto{\pgfqpoint{3.145667in}{3.394350in}}%
\pgfpathclose%
\pgfusepath{fill}%
\end{pgfscope}%
\begin{pgfscope}%
\pgfpathrectangle{\pgfqpoint{0.765000in}{0.660000in}}{\pgfqpoint{4.620000in}{4.620000in}}%
\pgfusepath{clip}%
\pgfsetbuttcap%
\pgfsetroundjoin%
\definecolor{currentfill}{rgb}{1.000000,0.894118,0.788235}%
\pgfsetfillcolor{currentfill}%
\pgfsetlinewidth{0.000000pt}%
\definecolor{currentstroke}{rgb}{1.000000,0.894118,0.788235}%
\pgfsetstrokecolor{currentstroke}%
\pgfsetdash{}{0pt}%
\pgfpathmoveto{\pgfqpoint{3.145667in}{3.522403in}}%
\pgfpathlineto{\pgfqpoint{3.256563in}{3.586429in}}%
\pgfpathlineto{\pgfqpoint{3.232124in}{3.591108in}}%
\pgfpathlineto{\pgfqpoint{3.121227in}{3.527081in}}%
\pgfpathlineto{\pgfqpoint{3.145667in}{3.522403in}}%
\pgfpathclose%
\pgfusepath{fill}%
\end{pgfscope}%
\begin{pgfscope}%
\pgfpathrectangle{\pgfqpoint{0.765000in}{0.660000in}}{\pgfqpoint{4.620000in}{4.620000in}}%
\pgfusepath{clip}%
\pgfsetbuttcap%
\pgfsetroundjoin%
\definecolor{currentfill}{rgb}{1.000000,0.894118,0.788235}%
\pgfsetfillcolor{currentfill}%
\pgfsetlinewidth{0.000000pt}%
\definecolor{currentstroke}{rgb}{1.000000,0.894118,0.788235}%
\pgfsetstrokecolor{currentstroke}%
\pgfsetdash{}{0pt}%
\pgfpathmoveto{\pgfqpoint{3.034770in}{3.586429in}}%
\pgfpathlineto{\pgfqpoint{3.145667in}{3.522403in}}%
\pgfpathlineto{\pgfqpoint{3.121227in}{3.527081in}}%
\pgfpathlineto{\pgfqpoint{3.010331in}{3.591108in}}%
\pgfpathlineto{\pgfqpoint{3.034770in}{3.586429in}}%
\pgfpathclose%
\pgfusepath{fill}%
\end{pgfscope}%
\begin{pgfscope}%
\pgfpathrectangle{\pgfqpoint{0.765000in}{0.660000in}}{\pgfqpoint{4.620000in}{4.620000in}}%
\pgfusepath{clip}%
\pgfsetbuttcap%
\pgfsetroundjoin%
\definecolor{currentfill}{rgb}{1.000000,0.894118,0.788235}%
\pgfsetfillcolor{currentfill}%
\pgfsetlinewidth{0.000000pt}%
\definecolor{currentstroke}{rgb}{1.000000,0.894118,0.788235}%
\pgfsetstrokecolor{currentstroke}%
\pgfsetdash{}{0pt}%
\pgfpathmoveto{\pgfqpoint{3.145667in}{3.522403in}}%
\pgfpathlineto{\pgfqpoint{3.034770in}{3.458376in}}%
\pgfpathlineto{\pgfqpoint{3.145667in}{3.394350in}}%
\pgfpathlineto{\pgfqpoint{3.256563in}{3.458376in}}%
\pgfpathlineto{\pgfqpoint{3.145667in}{3.522403in}}%
\pgfpathclose%
\pgfusepath{fill}%
\end{pgfscope}%
\begin{pgfscope}%
\pgfpathrectangle{\pgfqpoint{0.765000in}{0.660000in}}{\pgfqpoint{4.620000in}{4.620000in}}%
\pgfusepath{clip}%
\pgfsetbuttcap%
\pgfsetroundjoin%
\definecolor{currentfill}{rgb}{1.000000,0.894118,0.788235}%
\pgfsetfillcolor{currentfill}%
\pgfsetlinewidth{0.000000pt}%
\definecolor{currentstroke}{rgb}{1.000000,0.894118,0.788235}%
\pgfsetstrokecolor{currentstroke}%
\pgfsetdash{}{0pt}%
\pgfpathmoveto{\pgfqpoint{3.145667in}{3.522403in}}%
\pgfpathlineto{\pgfqpoint{3.034770in}{3.458376in}}%
\pgfpathlineto{\pgfqpoint{3.034770in}{3.586429in}}%
\pgfpathlineto{\pgfqpoint{3.145667in}{3.650455in}}%
\pgfpathlineto{\pgfqpoint{3.145667in}{3.522403in}}%
\pgfpathclose%
\pgfusepath{fill}%
\end{pgfscope}%
\begin{pgfscope}%
\pgfpathrectangle{\pgfqpoint{0.765000in}{0.660000in}}{\pgfqpoint{4.620000in}{4.620000in}}%
\pgfusepath{clip}%
\pgfsetbuttcap%
\pgfsetroundjoin%
\definecolor{currentfill}{rgb}{1.000000,0.894118,0.788235}%
\pgfsetfillcolor{currentfill}%
\pgfsetlinewidth{0.000000pt}%
\definecolor{currentstroke}{rgb}{1.000000,0.894118,0.788235}%
\pgfsetstrokecolor{currentstroke}%
\pgfsetdash{}{0pt}%
\pgfpathmoveto{\pgfqpoint{3.145667in}{3.522403in}}%
\pgfpathlineto{\pgfqpoint{3.256563in}{3.458376in}}%
\pgfpathlineto{\pgfqpoint{3.256563in}{3.586429in}}%
\pgfpathlineto{\pgfqpoint{3.145667in}{3.650455in}}%
\pgfpathlineto{\pgfqpoint{3.145667in}{3.522403in}}%
\pgfpathclose%
\pgfusepath{fill}%
\end{pgfscope}%
\begin{pgfscope}%
\pgfpathrectangle{\pgfqpoint{0.765000in}{0.660000in}}{\pgfqpoint{4.620000in}{4.620000in}}%
\pgfusepath{clip}%
\pgfsetbuttcap%
\pgfsetroundjoin%
\definecolor{currentfill}{rgb}{1.000000,0.894118,0.788235}%
\pgfsetfillcolor{currentfill}%
\pgfsetlinewidth{0.000000pt}%
\definecolor{currentstroke}{rgb}{1.000000,0.894118,0.788235}%
\pgfsetstrokecolor{currentstroke}%
\pgfsetdash{}{0pt}%
\pgfpathmoveto{\pgfqpoint{3.230111in}{3.500132in}}%
\pgfpathlineto{\pgfqpoint{3.119215in}{3.436106in}}%
\pgfpathlineto{\pgfqpoint{3.230111in}{3.372080in}}%
\pgfpathlineto{\pgfqpoint{3.341008in}{3.436106in}}%
\pgfpathlineto{\pgfqpoint{3.230111in}{3.500132in}}%
\pgfpathclose%
\pgfusepath{fill}%
\end{pgfscope}%
\begin{pgfscope}%
\pgfpathrectangle{\pgfqpoint{0.765000in}{0.660000in}}{\pgfqpoint{4.620000in}{4.620000in}}%
\pgfusepath{clip}%
\pgfsetbuttcap%
\pgfsetroundjoin%
\definecolor{currentfill}{rgb}{1.000000,0.894118,0.788235}%
\pgfsetfillcolor{currentfill}%
\pgfsetlinewidth{0.000000pt}%
\definecolor{currentstroke}{rgb}{1.000000,0.894118,0.788235}%
\pgfsetstrokecolor{currentstroke}%
\pgfsetdash{}{0pt}%
\pgfpathmoveto{\pgfqpoint{3.230111in}{3.500132in}}%
\pgfpathlineto{\pgfqpoint{3.119215in}{3.436106in}}%
\pgfpathlineto{\pgfqpoint{3.119215in}{3.564159in}}%
\pgfpathlineto{\pgfqpoint{3.230111in}{3.628185in}}%
\pgfpathlineto{\pgfqpoint{3.230111in}{3.500132in}}%
\pgfpathclose%
\pgfusepath{fill}%
\end{pgfscope}%
\begin{pgfscope}%
\pgfpathrectangle{\pgfqpoint{0.765000in}{0.660000in}}{\pgfqpoint{4.620000in}{4.620000in}}%
\pgfusepath{clip}%
\pgfsetbuttcap%
\pgfsetroundjoin%
\definecolor{currentfill}{rgb}{1.000000,0.894118,0.788235}%
\pgfsetfillcolor{currentfill}%
\pgfsetlinewidth{0.000000pt}%
\definecolor{currentstroke}{rgb}{1.000000,0.894118,0.788235}%
\pgfsetstrokecolor{currentstroke}%
\pgfsetdash{}{0pt}%
\pgfpathmoveto{\pgfqpoint{3.230111in}{3.500132in}}%
\pgfpathlineto{\pgfqpoint{3.341008in}{3.436106in}}%
\pgfpathlineto{\pgfqpoint{3.341008in}{3.564159in}}%
\pgfpathlineto{\pgfqpoint{3.230111in}{3.628185in}}%
\pgfpathlineto{\pgfqpoint{3.230111in}{3.500132in}}%
\pgfpathclose%
\pgfusepath{fill}%
\end{pgfscope}%
\begin{pgfscope}%
\pgfpathrectangle{\pgfqpoint{0.765000in}{0.660000in}}{\pgfqpoint{4.620000in}{4.620000in}}%
\pgfusepath{clip}%
\pgfsetbuttcap%
\pgfsetroundjoin%
\definecolor{currentfill}{rgb}{1.000000,0.894118,0.788235}%
\pgfsetfillcolor{currentfill}%
\pgfsetlinewidth{0.000000pt}%
\definecolor{currentstroke}{rgb}{1.000000,0.894118,0.788235}%
\pgfsetstrokecolor{currentstroke}%
\pgfsetdash{}{0pt}%
\pgfpathmoveto{\pgfqpoint{3.145667in}{3.650455in}}%
\pgfpathlineto{\pgfqpoint{3.034770in}{3.586429in}}%
\pgfpathlineto{\pgfqpoint{3.145667in}{3.522403in}}%
\pgfpathlineto{\pgfqpoint{3.256563in}{3.586429in}}%
\pgfpathlineto{\pgfqpoint{3.145667in}{3.650455in}}%
\pgfpathclose%
\pgfusepath{fill}%
\end{pgfscope}%
\begin{pgfscope}%
\pgfpathrectangle{\pgfqpoint{0.765000in}{0.660000in}}{\pgfqpoint{4.620000in}{4.620000in}}%
\pgfusepath{clip}%
\pgfsetbuttcap%
\pgfsetroundjoin%
\definecolor{currentfill}{rgb}{1.000000,0.894118,0.788235}%
\pgfsetfillcolor{currentfill}%
\pgfsetlinewidth{0.000000pt}%
\definecolor{currentstroke}{rgb}{1.000000,0.894118,0.788235}%
\pgfsetstrokecolor{currentstroke}%
\pgfsetdash{}{0pt}%
\pgfpathmoveto{\pgfqpoint{3.145667in}{3.394350in}}%
\pgfpathlineto{\pgfqpoint{3.256563in}{3.458376in}}%
\pgfpathlineto{\pgfqpoint{3.256563in}{3.586429in}}%
\pgfpathlineto{\pgfqpoint{3.145667in}{3.522403in}}%
\pgfpathlineto{\pgfqpoint{3.145667in}{3.394350in}}%
\pgfpathclose%
\pgfusepath{fill}%
\end{pgfscope}%
\begin{pgfscope}%
\pgfpathrectangle{\pgfqpoint{0.765000in}{0.660000in}}{\pgfqpoint{4.620000in}{4.620000in}}%
\pgfusepath{clip}%
\pgfsetbuttcap%
\pgfsetroundjoin%
\definecolor{currentfill}{rgb}{1.000000,0.894118,0.788235}%
\pgfsetfillcolor{currentfill}%
\pgfsetlinewidth{0.000000pt}%
\definecolor{currentstroke}{rgb}{1.000000,0.894118,0.788235}%
\pgfsetstrokecolor{currentstroke}%
\pgfsetdash{}{0pt}%
\pgfpathmoveto{\pgfqpoint{3.034770in}{3.458376in}}%
\pgfpathlineto{\pgfqpoint{3.145667in}{3.394350in}}%
\pgfpathlineto{\pgfqpoint{3.145667in}{3.522403in}}%
\pgfpathlineto{\pgfqpoint{3.034770in}{3.586429in}}%
\pgfpathlineto{\pgfqpoint{3.034770in}{3.458376in}}%
\pgfpathclose%
\pgfusepath{fill}%
\end{pgfscope}%
\begin{pgfscope}%
\pgfpathrectangle{\pgfqpoint{0.765000in}{0.660000in}}{\pgfqpoint{4.620000in}{4.620000in}}%
\pgfusepath{clip}%
\pgfsetbuttcap%
\pgfsetroundjoin%
\definecolor{currentfill}{rgb}{1.000000,0.894118,0.788235}%
\pgfsetfillcolor{currentfill}%
\pgfsetlinewidth{0.000000pt}%
\definecolor{currentstroke}{rgb}{1.000000,0.894118,0.788235}%
\pgfsetstrokecolor{currentstroke}%
\pgfsetdash{}{0pt}%
\pgfpathmoveto{\pgfqpoint{3.230111in}{3.628185in}}%
\pgfpathlineto{\pgfqpoint{3.119215in}{3.564159in}}%
\pgfpathlineto{\pgfqpoint{3.230111in}{3.500132in}}%
\pgfpathlineto{\pgfqpoint{3.341008in}{3.564159in}}%
\pgfpathlineto{\pgfqpoint{3.230111in}{3.628185in}}%
\pgfpathclose%
\pgfusepath{fill}%
\end{pgfscope}%
\begin{pgfscope}%
\pgfpathrectangle{\pgfqpoint{0.765000in}{0.660000in}}{\pgfqpoint{4.620000in}{4.620000in}}%
\pgfusepath{clip}%
\pgfsetbuttcap%
\pgfsetroundjoin%
\definecolor{currentfill}{rgb}{1.000000,0.894118,0.788235}%
\pgfsetfillcolor{currentfill}%
\pgfsetlinewidth{0.000000pt}%
\definecolor{currentstroke}{rgb}{1.000000,0.894118,0.788235}%
\pgfsetstrokecolor{currentstroke}%
\pgfsetdash{}{0pt}%
\pgfpathmoveto{\pgfqpoint{3.230111in}{3.372080in}}%
\pgfpathlineto{\pgfqpoint{3.341008in}{3.436106in}}%
\pgfpathlineto{\pgfqpoint{3.341008in}{3.564159in}}%
\pgfpathlineto{\pgfqpoint{3.230111in}{3.500132in}}%
\pgfpathlineto{\pgfqpoint{3.230111in}{3.372080in}}%
\pgfpathclose%
\pgfusepath{fill}%
\end{pgfscope}%
\begin{pgfscope}%
\pgfpathrectangle{\pgfqpoint{0.765000in}{0.660000in}}{\pgfqpoint{4.620000in}{4.620000in}}%
\pgfusepath{clip}%
\pgfsetbuttcap%
\pgfsetroundjoin%
\definecolor{currentfill}{rgb}{1.000000,0.894118,0.788235}%
\pgfsetfillcolor{currentfill}%
\pgfsetlinewidth{0.000000pt}%
\definecolor{currentstroke}{rgb}{1.000000,0.894118,0.788235}%
\pgfsetstrokecolor{currentstroke}%
\pgfsetdash{}{0pt}%
\pgfpathmoveto{\pgfqpoint{3.119215in}{3.436106in}}%
\pgfpathlineto{\pgfqpoint{3.230111in}{3.372080in}}%
\pgfpathlineto{\pgfqpoint{3.230111in}{3.500132in}}%
\pgfpathlineto{\pgfqpoint{3.119215in}{3.564159in}}%
\pgfpathlineto{\pgfqpoint{3.119215in}{3.436106in}}%
\pgfpathclose%
\pgfusepath{fill}%
\end{pgfscope}%
\begin{pgfscope}%
\pgfpathrectangle{\pgfqpoint{0.765000in}{0.660000in}}{\pgfqpoint{4.620000in}{4.620000in}}%
\pgfusepath{clip}%
\pgfsetbuttcap%
\pgfsetroundjoin%
\definecolor{currentfill}{rgb}{1.000000,0.894118,0.788235}%
\pgfsetfillcolor{currentfill}%
\pgfsetlinewidth{0.000000pt}%
\definecolor{currentstroke}{rgb}{1.000000,0.894118,0.788235}%
\pgfsetstrokecolor{currentstroke}%
\pgfsetdash{}{0pt}%
\pgfpathmoveto{\pgfqpoint{3.145667in}{3.522403in}}%
\pgfpathlineto{\pgfqpoint{3.034770in}{3.458376in}}%
\pgfpathlineto{\pgfqpoint{3.119215in}{3.436106in}}%
\pgfpathlineto{\pgfqpoint{3.230111in}{3.500132in}}%
\pgfpathlineto{\pgfqpoint{3.145667in}{3.522403in}}%
\pgfpathclose%
\pgfusepath{fill}%
\end{pgfscope}%
\begin{pgfscope}%
\pgfpathrectangle{\pgfqpoint{0.765000in}{0.660000in}}{\pgfqpoint{4.620000in}{4.620000in}}%
\pgfusepath{clip}%
\pgfsetbuttcap%
\pgfsetroundjoin%
\definecolor{currentfill}{rgb}{1.000000,0.894118,0.788235}%
\pgfsetfillcolor{currentfill}%
\pgfsetlinewidth{0.000000pt}%
\definecolor{currentstroke}{rgb}{1.000000,0.894118,0.788235}%
\pgfsetstrokecolor{currentstroke}%
\pgfsetdash{}{0pt}%
\pgfpathmoveto{\pgfqpoint{3.256563in}{3.458376in}}%
\pgfpathlineto{\pgfqpoint{3.145667in}{3.522403in}}%
\pgfpathlineto{\pgfqpoint{3.230111in}{3.500132in}}%
\pgfpathlineto{\pgfqpoint{3.341008in}{3.436106in}}%
\pgfpathlineto{\pgfqpoint{3.256563in}{3.458376in}}%
\pgfpathclose%
\pgfusepath{fill}%
\end{pgfscope}%
\begin{pgfscope}%
\pgfpathrectangle{\pgfqpoint{0.765000in}{0.660000in}}{\pgfqpoint{4.620000in}{4.620000in}}%
\pgfusepath{clip}%
\pgfsetbuttcap%
\pgfsetroundjoin%
\definecolor{currentfill}{rgb}{1.000000,0.894118,0.788235}%
\pgfsetfillcolor{currentfill}%
\pgfsetlinewidth{0.000000pt}%
\definecolor{currentstroke}{rgb}{1.000000,0.894118,0.788235}%
\pgfsetstrokecolor{currentstroke}%
\pgfsetdash{}{0pt}%
\pgfpathmoveto{\pgfqpoint{3.145667in}{3.522403in}}%
\pgfpathlineto{\pgfqpoint{3.145667in}{3.650455in}}%
\pgfpathlineto{\pgfqpoint{3.230111in}{3.628185in}}%
\pgfpathlineto{\pgfqpoint{3.341008in}{3.436106in}}%
\pgfpathlineto{\pgfqpoint{3.145667in}{3.522403in}}%
\pgfpathclose%
\pgfusepath{fill}%
\end{pgfscope}%
\begin{pgfscope}%
\pgfpathrectangle{\pgfqpoint{0.765000in}{0.660000in}}{\pgfqpoint{4.620000in}{4.620000in}}%
\pgfusepath{clip}%
\pgfsetbuttcap%
\pgfsetroundjoin%
\definecolor{currentfill}{rgb}{1.000000,0.894118,0.788235}%
\pgfsetfillcolor{currentfill}%
\pgfsetlinewidth{0.000000pt}%
\definecolor{currentstroke}{rgb}{1.000000,0.894118,0.788235}%
\pgfsetstrokecolor{currentstroke}%
\pgfsetdash{}{0pt}%
\pgfpathmoveto{\pgfqpoint{3.256563in}{3.458376in}}%
\pgfpathlineto{\pgfqpoint{3.256563in}{3.586429in}}%
\pgfpathlineto{\pgfqpoint{3.341008in}{3.564159in}}%
\pgfpathlineto{\pgfqpoint{3.230111in}{3.500132in}}%
\pgfpathlineto{\pgfqpoint{3.256563in}{3.458376in}}%
\pgfpathclose%
\pgfusepath{fill}%
\end{pgfscope}%
\begin{pgfscope}%
\pgfpathrectangle{\pgfqpoint{0.765000in}{0.660000in}}{\pgfqpoint{4.620000in}{4.620000in}}%
\pgfusepath{clip}%
\pgfsetbuttcap%
\pgfsetroundjoin%
\definecolor{currentfill}{rgb}{1.000000,0.894118,0.788235}%
\pgfsetfillcolor{currentfill}%
\pgfsetlinewidth{0.000000pt}%
\definecolor{currentstroke}{rgb}{1.000000,0.894118,0.788235}%
\pgfsetstrokecolor{currentstroke}%
\pgfsetdash{}{0pt}%
\pgfpathmoveto{\pgfqpoint{3.034770in}{3.458376in}}%
\pgfpathlineto{\pgfqpoint{3.145667in}{3.394350in}}%
\pgfpathlineto{\pgfqpoint{3.230111in}{3.372080in}}%
\pgfpathlineto{\pgfqpoint{3.119215in}{3.436106in}}%
\pgfpathlineto{\pgfqpoint{3.034770in}{3.458376in}}%
\pgfpathclose%
\pgfusepath{fill}%
\end{pgfscope}%
\begin{pgfscope}%
\pgfpathrectangle{\pgfqpoint{0.765000in}{0.660000in}}{\pgfqpoint{4.620000in}{4.620000in}}%
\pgfusepath{clip}%
\pgfsetbuttcap%
\pgfsetroundjoin%
\definecolor{currentfill}{rgb}{1.000000,0.894118,0.788235}%
\pgfsetfillcolor{currentfill}%
\pgfsetlinewidth{0.000000pt}%
\definecolor{currentstroke}{rgb}{1.000000,0.894118,0.788235}%
\pgfsetstrokecolor{currentstroke}%
\pgfsetdash{}{0pt}%
\pgfpathmoveto{\pgfqpoint{3.145667in}{3.394350in}}%
\pgfpathlineto{\pgfqpoint{3.256563in}{3.458376in}}%
\pgfpathlineto{\pgfqpoint{3.341008in}{3.436106in}}%
\pgfpathlineto{\pgfqpoint{3.230111in}{3.372080in}}%
\pgfpathlineto{\pgfqpoint{3.145667in}{3.394350in}}%
\pgfpathclose%
\pgfusepath{fill}%
\end{pgfscope}%
\begin{pgfscope}%
\pgfpathrectangle{\pgfqpoint{0.765000in}{0.660000in}}{\pgfqpoint{4.620000in}{4.620000in}}%
\pgfusepath{clip}%
\pgfsetbuttcap%
\pgfsetroundjoin%
\definecolor{currentfill}{rgb}{1.000000,0.894118,0.788235}%
\pgfsetfillcolor{currentfill}%
\pgfsetlinewidth{0.000000pt}%
\definecolor{currentstroke}{rgb}{1.000000,0.894118,0.788235}%
\pgfsetstrokecolor{currentstroke}%
\pgfsetdash{}{0pt}%
\pgfpathmoveto{\pgfqpoint{3.145667in}{3.650455in}}%
\pgfpathlineto{\pgfqpoint{3.034770in}{3.586429in}}%
\pgfpathlineto{\pgfqpoint{3.119215in}{3.564159in}}%
\pgfpathlineto{\pgfqpoint{3.230111in}{3.628185in}}%
\pgfpathlineto{\pgfqpoint{3.145667in}{3.650455in}}%
\pgfpathclose%
\pgfusepath{fill}%
\end{pgfscope}%
\begin{pgfscope}%
\pgfpathrectangle{\pgfqpoint{0.765000in}{0.660000in}}{\pgfqpoint{4.620000in}{4.620000in}}%
\pgfusepath{clip}%
\pgfsetbuttcap%
\pgfsetroundjoin%
\definecolor{currentfill}{rgb}{1.000000,0.894118,0.788235}%
\pgfsetfillcolor{currentfill}%
\pgfsetlinewidth{0.000000pt}%
\definecolor{currentstroke}{rgb}{1.000000,0.894118,0.788235}%
\pgfsetstrokecolor{currentstroke}%
\pgfsetdash{}{0pt}%
\pgfpathmoveto{\pgfqpoint{3.256563in}{3.586429in}}%
\pgfpathlineto{\pgfqpoint{3.145667in}{3.650455in}}%
\pgfpathlineto{\pgfqpoint{3.230111in}{3.628185in}}%
\pgfpathlineto{\pgfqpoint{3.341008in}{3.564159in}}%
\pgfpathlineto{\pgfqpoint{3.256563in}{3.586429in}}%
\pgfpathclose%
\pgfusepath{fill}%
\end{pgfscope}%
\begin{pgfscope}%
\pgfpathrectangle{\pgfqpoint{0.765000in}{0.660000in}}{\pgfqpoint{4.620000in}{4.620000in}}%
\pgfusepath{clip}%
\pgfsetbuttcap%
\pgfsetroundjoin%
\definecolor{currentfill}{rgb}{1.000000,0.894118,0.788235}%
\pgfsetfillcolor{currentfill}%
\pgfsetlinewidth{0.000000pt}%
\definecolor{currentstroke}{rgb}{1.000000,0.894118,0.788235}%
\pgfsetstrokecolor{currentstroke}%
\pgfsetdash{}{0pt}%
\pgfpathmoveto{\pgfqpoint{3.034770in}{3.458376in}}%
\pgfpathlineto{\pgfqpoint{3.034770in}{3.586429in}}%
\pgfpathlineto{\pgfqpoint{3.119215in}{3.564159in}}%
\pgfpathlineto{\pgfqpoint{3.230111in}{3.372080in}}%
\pgfpathlineto{\pgfqpoint{3.034770in}{3.458376in}}%
\pgfpathclose%
\pgfusepath{fill}%
\end{pgfscope}%
\begin{pgfscope}%
\pgfpathrectangle{\pgfqpoint{0.765000in}{0.660000in}}{\pgfqpoint{4.620000in}{4.620000in}}%
\pgfusepath{clip}%
\pgfsetbuttcap%
\pgfsetroundjoin%
\definecolor{currentfill}{rgb}{1.000000,0.894118,0.788235}%
\pgfsetfillcolor{currentfill}%
\pgfsetlinewidth{0.000000pt}%
\definecolor{currentstroke}{rgb}{1.000000,0.894118,0.788235}%
\pgfsetstrokecolor{currentstroke}%
\pgfsetdash{}{0pt}%
\pgfpathmoveto{\pgfqpoint{3.145667in}{3.394350in}}%
\pgfpathlineto{\pgfqpoint{3.145667in}{3.522403in}}%
\pgfpathlineto{\pgfqpoint{3.230111in}{3.500132in}}%
\pgfpathlineto{\pgfqpoint{3.119215in}{3.436106in}}%
\pgfpathlineto{\pgfqpoint{3.145667in}{3.394350in}}%
\pgfpathclose%
\pgfusepath{fill}%
\end{pgfscope}%
\begin{pgfscope}%
\pgfpathrectangle{\pgfqpoint{0.765000in}{0.660000in}}{\pgfqpoint{4.620000in}{4.620000in}}%
\pgfusepath{clip}%
\pgfsetbuttcap%
\pgfsetroundjoin%
\definecolor{currentfill}{rgb}{1.000000,0.894118,0.788235}%
\pgfsetfillcolor{currentfill}%
\pgfsetlinewidth{0.000000pt}%
\definecolor{currentstroke}{rgb}{1.000000,0.894118,0.788235}%
\pgfsetstrokecolor{currentstroke}%
\pgfsetdash{}{0pt}%
\pgfpathmoveto{\pgfqpoint{3.145667in}{3.522403in}}%
\pgfpathlineto{\pgfqpoint{3.256563in}{3.586429in}}%
\pgfpathlineto{\pgfqpoint{3.341008in}{3.564159in}}%
\pgfpathlineto{\pgfqpoint{3.230111in}{3.500132in}}%
\pgfpathlineto{\pgfqpoint{3.145667in}{3.522403in}}%
\pgfpathclose%
\pgfusepath{fill}%
\end{pgfscope}%
\begin{pgfscope}%
\pgfpathrectangle{\pgfqpoint{0.765000in}{0.660000in}}{\pgfqpoint{4.620000in}{4.620000in}}%
\pgfusepath{clip}%
\pgfsetbuttcap%
\pgfsetroundjoin%
\definecolor{currentfill}{rgb}{1.000000,0.894118,0.788235}%
\pgfsetfillcolor{currentfill}%
\pgfsetlinewidth{0.000000pt}%
\definecolor{currentstroke}{rgb}{1.000000,0.894118,0.788235}%
\pgfsetstrokecolor{currentstroke}%
\pgfsetdash{}{0pt}%
\pgfpathmoveto{\pgfqpoint{3.034770in}{3.586429in}}%
\pgfpathlineto{\pgfqpoint{3.145667in}{3.522403in}}%
\pgfpathlineto{\pgfqpoint{3.230111in}{3.500132in}}%
\pgfpathlineto{\pgfqpoint{3.119215in}{3.564159in}}%
\pgfpathlineto{\pgfqpoint{3.034770in}{3.586429in}}%
\pgfpathclose%
\pgfusepath{fill}%
\end{pgfscope}%
\begin{pgfscope}%
\pgfpathrectangle{\pgfqpoint{0.765000in}{0.660000in}}{\pgfqpoint{4.620000in}{4.620000in}}%
\pgfusepath{clip}%
\pgfsetbuttcap%
\pgfsetroundjoin%
\definecolor{currentfill}{rgb}{1.000000,0.894118,0.788235}%
\pgfsetfillcolor{currentfill}%
\pgfsetlinewidth{0.000000pt}%
\definecolor{currentstroke}{rgb}{1.000000,0.894118,0.788235}%
\pgfsetstrokecolor{currentstroke}%
\pgfsetdash{}{0pt}%
\pgfpathmoveto{\pgfqpoint{2.979627in}{2.512096in}}%
\pgfpathlineto{\pgfqpoint{2.868730in}{2.448069in}}%
\pgfpathlineto{\pgfqpoint{2.979627in}{2.384043in}}%
\pgfpathlineto{\pgfqpoint{3.090524in}{2.448069in}}%
\pgfpathlineto{\pgfqpoint{2.979627in}{2.512096in}}%
\pgfpathclose%
\pgfusepath{fill}%
\end{pgfscope}%
\begin{pgfscope}%
\pgfpathrectangle{\pgfqpoint{0.765000in}{0.660000in}}{\pgfqpoint{4.620000in}{4.620000in}}%
\pgfusepath{clip}%
\pgfsetbuttcap%
\pgfsetroundjoin%
\definecolor{currentfill}{rgb}{1.000000,0.894118,0.788235}%
\pgfsetfillcolor{currentfill}%
\pgfsetlinewidth{0.000000pt}%
\definecolor{currentstroke}{rgb}{1.000000,0.894118,0.788235}%
\pgfsetstrokecolor{currentstroke}%
\pgfsetdash{}{0pt}%
\pgfpathmoveto{\pgfqpoint{2.979627in}{2.512096in}}%
\pgfpathlineto{\pgfqpoint{2.868730in}{2.448069in}}%
\pgfpathlineto{\pgfqpoint{2.868730in}{2.576122in}}%
\pgfpathlineto{\pgfqpoint{2.979627in}{2.640148in}}%
\pgfpathlineto{\pgfqpoint{2.979627in}{2.512096in}}%
\pgfpathclose%
\pgfusepath{fill}%
\end{pgfscope}%
\begin{pgfscope}%
\pgfpathrectangle{\pgfqpoint{0.765000in}{0.660000in}}{\pgfqpoint{4.620000in}{4.620000in}}%
\pgfusepath{clip}%
\pgfsetbuttcap%
\pgfsetroundjoin%
\definecolor{currentfill}{rgb}{1.000000,0.894118,0.788235}%
\pgfsetfillcolor{currentfill}%
\pgfsetlinewidth{0.000000pt}%
\definecolor{currentstroke}{rgb}{1.000000,0.894118,0.788235}%
\pgfsetstrokecolor{currentstroke}%
\pgfsetdash{}{0pt}%
\pgfpathmoveto{\pgfqpoint{2.979627in}{2.512096in}}%
\pgfpathlineto{\pgfqpoint{3.090524in}{2.448069in}}%
\pgfpathlineto{\pgfqpoint{3.090524in}{2.576122in}}%
\pgfpathlineto{\pgfqpoint{2.979627in}{2.640148in}}%
\pgfpathlineto{\pgfqpoint{2.979627in}{2.512096in}}%
\pgfpathclose%
\pgfusepath{fill}%
\end{pgfscope}%
\begin{pgfscope}%
\pgfpathrectangle{\pgfqpoint{0.765000in}{0.660000in}}{\pgfqpoint{4.620000in}{4.620000in}}%
\pgfusepath{clip}%
\pgfsetbuttcap%
\pgfsetroundjoin%
\definecolor{currentfill}{rgb}{1.000000,0.894118,0.788235}%
\pgfsetfillcolor{currentfill}%
\pgfsetlinewidth{0.000000pt}%
\definecolor{currentstroke}{rgb}{1.000000,0.894118,0.788235}%
\pgfsetstrokecolor{currentstroke}%
\pgfsetdash{}{0pt}%
\pgfpathmoveto{\pgfqpoint{2.952980in}{2.520265in}}%
\pgfpathlineto{\pgfqpoint{2.842083in}{2.456239in}}%
\pgfpathlineto{\pgfqpoint{2.952980in}{2.392213in}}%
\pgfpathlineto{\pgfqpoint{3.063876in}{2.456239in}}%
\pgfpathlineto{\pgfqpoint{2.952980in}{2.520265in}}%
\pgfpathclose%
\pgfusepath{fill}%
\end{pgfscope}%
\begin{pgfscope}%
\pgfpathrectangle{\pgfqpoint{0.765000in}{0.660000in}}{\pgfqpoint{4.620000in}{4.620000in}}%
\pgfusepath{clip}%
\pgfsetbuttcap%
\pgfsetroundjoin%
\definecolor{currentfill}{rgb}{1.000000,0.894118,0.788235}%
\pgfsetfillcolor{currentfill}%
\pgfsetlinewidth{0.000000pt}%
\definecolor{currentstroke}{rgb}{1.000000,0.894118,0.788235}%
\pgfsetstrokecolor{currentstroke}%
\pgfsetdash{}{0pt}%
\pgfpathmoveto{\pgfqpoint{2.952980in}{2.520265in}}%
\pgfpathlineto{\pgfqpoint{2.842083in}{2.456239in}}%
\pgfpathlineto{\pgfqpoint{2.842083in}{2.584292in}}%
\pgfpathlineto{\pgfqpoint{2.952980in}{2.648318in}}%
\pgfpathlineto{\pgfqpoint{2.952980in}{2.520265in}}%
\pgfpathclose%
\pgfusepath{fill}%
\end{pgfscope}%
\begin{pgfscope}%
\pgfpathrectangle{\pgfqpoint{0.765000in}{0.660000in}}{\pgfqpoint{4.620000in}{4.620000in}}%
\pgfusepath{clip}%
\pgfsetbuttcap%
\pgfsetroundjoin%
\definecolor{currentfill}{rgb}{1.000000,0.894118,0.788235}%
\pgfsetfillcolor{currentfill}%
\pgfsetlinewidth{0.000000pt}%
\definecolor{currentstroke}{rgb}{1.000000,0.894118,0.788235}%
\pgfsetstrokecolor{currentstroke}%
\pgfsetdash{}{0pt}%
\pgfpathmoveto{\pgfqpoint{2.952980in}{2.520265in}}%
\pgfpathlineto{\pgfqpoint{3.063876in}{2.456239in}}%
\pgfpathlineto{\pgfqpoint{3.063876in}{2.584292in}}%
\pgfpathlineto{\pgfqpoint{2.952980in}{2.648318in}}%
\pgfpathlineto{\pgfqpoint{2.952980in}{2.520265in}}%
\pgfpathclose%
\pgfusepath{fill}%
\end{pgfscope}%
\begin{pgfscope}%
\pgfpathrectangle{\pgfqpoint{0.765000in}{0.660000in}}{\pgfqpoint{4.620000in}{4.620000in}}%
\pgfusepath{clip}%
\pgfsetbuttcap%
\pgfsetroundjoin%
\definecolor{currentfill}{rgb}{1.000000,0.894118,0.788235}%
\pgfsetfillcolor{currentfill}%
\pgfsetlinewidth{0.000000pt}%
\definecolor{currentstroke}{rgb}{1.000000,0.894118,0.788235}%
\pgfsetstrokecolor{currentstroke}%
\pgfsetdash{}{0pt}%
\pgfpathmoveto{\pgfqpoint{2.979627in}{2.640148in}}%
\pgfpathlineto{\pgfqpoint{2.868730in}{2.576122in}}%
\pgfpathlineto{\pgfqpoint{2.979627in}{2.512096in}}%
\pgfpathlineto{\pgfqpoint{3.090524in}{2.576122in}}%
\pgfpathlineto{\pgfqpoint{2.979627in}{2.640148in}}%
\pgfpathclose%
\pgfusepath{fill}%
\end{pgfscope}%
\begin{pgfscope}%
\pgfpathrectangle{\pgfqpoint{0.765000in}{0.660000in}}{\pgfqpoint{4.620000in}{4.620000in}}%
\pgfusepath{clip}%
\pgfsetbuttcap%
\pgfsetroundjoin%
\definecolor{currentfill}{rgb}{1.000000,0.894118,0.788235}%
\pgfsetfillcolor{currentfill}%
\pgfsetlinewidth{0.000000pt}%
\definecolor{currentstroke}{rgb}{1.000000,0.894118,0.788235}%
\pgfsetstrokecolor{currentstroke}%
\pgfsetdash{}{0pt}%
\pgfpathmoveto{\pgfqpoint{2.979627in}{2.384043in}}%
\pgfpathlineto{\pgfqpoint{3.090524in}{2.448069in}}%
\pgfpathlineto{\pgfqpoint{3.090524in}{2.576122in}}%
\pgfpathlineto{\pgfqpoint{2.979627in}{2.512096in}}%
\pgfpathlineto{\pgfqpoint{2.979627in}{2.384043in}}%
\pgfpathclose%
\pgfusepath{fill}%
\end{pgfscope}%
\begin{pgfscope}%
\pgfpathrectangle{\pgfqpoint{0.765000in}{0.660000in}}{\pgfqpoint{4.620000in}{4.620000in}}%
\pgfusepath{clip}%
\pgfsetbuttcap%
\pgfsetroundjoin%
\definecolor{currentfill}{rgb}{1.000000,0.894118,0.788235}%
\pgfsetfillcolor{currentfill}%
\pgfsetlinewidth{0.000000pt}%
\definecolor{currentstroke}{rgb}{1.000000,0.894118,0.788235}%
\pgfsetstrokecolor{currentstroke}%
\pgfsetdash{}{0pt}%
\pgfpathmoveto{\pgfqpoint{2.868730in}{2.448069in}}%
\pgfpathlineto{\pgfqpoint{2.979627in}{2.384043in}}%
\pgfpathlineto{\pgfqpoint{2.979627in}{2.512096in}}%
\pgfpathlineto{\pgfqpoint{2.868730in}{2.576122in}}%
\pgfpathlineto{\pgfqpoint{2.868730in}{2.448069in}}%
\pgfpathclose%
\pgfusepath{fill}%
\end{pgfscope}%
\begin{pgfscope}%
\pgfpathrectangle{\pgfqpoint{0.765000in}{0.660000in}}{\pgfqpoint{4.620000in}{4.620000in}}%
\pgfusepath{clip}%
\pgfsetbuttcap%
\pgfsetroundjoin%
\definecolor{currentfill}{rgb}{1.000000,0.894118,0.788235}%
\pgfsetfillcolor{currentfill}%
\pgfsetlinewidth{0.000000pt}%
\definecolor{currentstroke}{rgb}{1.000000,0.894118,0.788235}%
\pgfsetstrokecolor{currentstroke}%
\pgfsetdash{}{0pt}%
\pgfpathmoveto{\pgfqpoint{2.952980in}{2.648318in}}%
\pgfpathlineto{\pgfqpoint{2.842083in}{2.584292in}}%
\pgfpathlineto{\pgfqpoint{2.952980in}{2.520265in}}%
\pgfpathlineto{\pgfqpoint{3.063876in}{2.584292in}}%
\pgfpathlineto{\pgfqpoint{2.952980in}{2.648318in}}%
\pgfpathclose%
\pgfusepath{fill}%
\end{pgfscope}%
\begin{pgfscope}%
\pgfpathrectangle{\pgfqpoint{0.765000in}{0.660000in}}{\pgfqpoint{4.620000in}{4.620000in}}%
\pgfusepath{clip}%
\pgfsetbuttcap%
\pgfsetroundjoin%
\definecolor{currentfill}{rgb}{1.000000,0.894118,0.788235}%
\pgfsetfillcolor{currentfill}%
\pgfsetlinewidth{0.000000pt}%
\definecolor{currentstroke}{rgb}{1.000000,0.894118,0.788235}%
\pgfsetstrokecolor{currentstroke}%
\pgfsetdash{}{0pt}%
\pgfpathmoveto{\pgfqpoint{2.952980in}{2.392213in}}%
\pgfpathlineto{\pgfqpoint{3.063876in}{2.456239in}}%
\pgfpathlineto{\pgfqpoint{3.063876in}{2.584292in}}%
\pgfpathlineto{\pgfqpoint{2.952980in}{2.520265in}}%
\pgfpathlineto{\pgfqpoint{2.952980in}{2.392213in}}%
\pgfpathclose%
\pgfusepath{fill}%
\end{pgfscope}%
\begin{pgfscope}%
\pgfpathrectangle{\pgfqpoint{0.765000in}{0.660000in}}{\pgfqpoint{4.620000in}{4.620000in}}%
\pgfusepath{clip}%
\pgfsetbuttcap%
\pgfsetroundjoin%
\definecolor{currentfill}{rgb}{1.000000,0.894118,0.788235}%
\pgfsetfillcolor{currentfill}%
\pgfsetlinewidth{0.000000pt}%
\definecolor{currentstroke}{rgb}{1.000000,0.894118,0.788235}%
\pgfsetstrokecolor{currentstroke}%
\pgfsetdash{}{0pt}%
\pgfpathmoveto{\pgfqpoint{2.842083in}{2.456239in}}%
\pgfpathlineto{\pgfqpoint{2.952980in}{2.392213in}}%
\pgfpathlineto{\pgfqpoint{2.952980in}{2.520265in}}%
\pgfpathlineto{\pgfqpoint{2.842083in}{2.584292in}}%
\pgfpathlineto{\pgfqpoint{2.842083in}{2.456239in}}%
\pgfpathclose%
\pgfusepath{fill}%
\end{pgfscope}%
\begin{pgfscope}%
\pgfpathrectangle{\pgfqpoint{0.765000in}{0.660000in}}{\pgfqpoint{4.620000in}{4.620000in}}%
\pgfusepath{clip}%
\pgfsetbuttcap%
\pgfsetroundjoin%
\definecolor{currentfill}{rgb}{1.000000,0.894118,0.788235}%
\pgfsetfillcolor{currentfill}%
\pgfsetlinewidth{0.000000pt}%
\definecolor{currentstroke}{rgb}{1.000000,0.894118,0.788235}%
\pgfsetstrokecolor{currentstroke}%
\pgfsetdash{}{0pt}%
\pgfpathmoveto{\pgfqpoint{2.979627in}{2.512096in}}%
\pgfpathlineto{\pgfqpoint{2.868730in}{2.448069in}}%
\pgfpathlineto{\pgfqpoint{2.842083in}{2.456239in}}%
\pgfpathlineto{\pgfqpoint{2.952980in}{2.520265in}}%
\pgfpathlineto{\pgfqpoint{2.979627in}{2.512096in}}%
\pgfpathclose%
\pgfusepath{fill}%
\end{pgfscope}%
\begin{pgfscope}%
\pgfpathrectangle{\pgfqpoint{0.765000in}{0.660000in}}{\pgfqpoint{4.620000in}{4.620000in}}%
\pgfusepath{clip}%
\pgfsetbuttcap%
\pgfsetroundjoin%
\definecolor{currentfill}{rgb}{1.000000,0.894118,0.788235}%
\pgfsetfillcolor{currentfill}%
\pgfsetlinewidth{0.000000pt}%
\definecolor{currentstroke}{rgb}{1.000000,0.894118,0.788235}%
\pgfsetstrokecolor{currentstroke}%
\pgfsetdash{}{0pt}%
\pgfpathmoveto{\pgfqpoint{3.090524in}{2.448069in}}%
\pgfpathlineto{\pgfqpoint{2.979627in}{2.512096in}}%
\pgfpathlineto{\pgfqpoint{2.952980in}{2.520265in}}%
\pgfpathlineto{\pgfqpoint{3.063876in}{2.456239in}}%
\pgfpathlineto{\pgfqpoint{3.090524in}{2.448069in}}%
\pgfpathclose%
\pgfusepath{fill}%
\end{pgfscope}%
\begin{pgfscope}%
\pgfpathrectangle{\pgfqpoint{0.765000in}{0.660000in}}{\pgfqpoint{4.620000in}{4.620000in}}%
\pgfusepath{clip}%
\pgfsetbuttcap%
\pgfsetroundjoin%
\definecolor{currentfill}{rgb}{1.000000,0.894118,0.788235}%
\pgfsetfillcolor{currentfill}%
\pgfsetlinewidth{0.000000pt}%
\definecolor{currentstroke}{rgb}{1.000000,0.894118,0.788235}%
\pgfsetstrokecolor{currentstroke}%
\pgfsetdash{}{0pt}%
\pgfpathmoveto{\pgfqpoint{2.979627in}{2.512096in}}%
\pgfpathlineto{\pgfqpoint{2.979627in}{2.640148in}}%
\pgfpathlineto{\pgfqpoint{2.952980in}{2.648318in}}%
\pgfpathlineto{\pgfqpoint{3.063876in}{2.456239in}}%
\pgfpathlineto{\pgfqpoint{2.979627in}{2.512096in}}%
\pgfpathclose%
\pgfusepath{fill}%
\end{pgfscope}%
\begin{pgfscope}%
\pgfpathrectangle{\pgfqpoint{0.765000in}{0.660000in}}{\pgfqpoint{4.620000in}{4.620000in}}%
\pgfusepath{clip}%
\pgfsetbuttcap%
\pgfsetroundjoin%
\definecolor{currentfill}{rgb}{1.000000,0.894118,0.788235}%
\pgfsetfillcolor{currentfill}%
\pgfsetlinewidth{0.000000pt}%
\definecolor{currentstroke}{rgb}{1.000000,0.894118,0.788235}%
\pgfsetstrokecolor{currentstroke}%
\pgfsetdash{}{0pt}%
\pgfpathmoveto{\pgfqpoint{3.090524in}{2.448069in}}%
\pgfpathlineto{\pgfqpoint{3.090524in}{2.576122in}}%
\pgfpathlineto{\pgfqpoint{3.063876in}{2.584292in}}%
\pgfpathlineto{\pgfqpoint{2.952980in}{2.520265in}}%
\pgfpathlineto{\pgfqpoint{3.090524in}{2.448069in}}%
\pgfpathclose%
\pgfusepath{fill}%
\end{pgfscope}%
\begin{pgfscope}%
\pgfpathrectangle{\pgfqpoint{0.765000in}{0.660000in}}{\pgfqpoint{4.620000in}{4.620000in}}%
\pgfusepath{clip}%
\pgfsetbuttcap%
\pgfsetroundjoin%
\definecolor{currentfill}{rgb}{1.000000,0.894118,0.788235}%
\pgfsetfillcolor{currentfill}%
\pgfsetlinewidth{0.000000pt}%
\definecolor{currentstroke}{rgb}{1.000000,0.894118,0.788235}%
\pgfsetstrokecolor{currentstroke}%
\pgfsetdash{}{0pt}%
\pgfpathmoveto{\pgfqpoint{2.868730in}{2.448069in}}%
\pgfpathlineto{\pgfqpoint{2.979627in}{2.384043in}}%
\pgfpathlineto{\pgfqpoint{2.952980in}{2.392213in}}%
\pgfpathlineto{\pgfqpoint{2.842083in}{2.456239in}}%
\pgfpathlineto{\pgfqpoint{2.868730in}{2.448069in}}%
\pgfpathclose%
\pgfusepath{fill}%
\end{pgfscope}%
\begin{pgfscope}%
\pgfpathrectangle{\pgfqpoint{0.765000in}{0.660000in}}{\pgfqpoint{4.620000in}{4.620000in}}%
\pgfusepath{clip}%
\pgfsetbuttcap%
\pgfsetroundjoin%
\definecolor{currentfill}{rgb}{1.000000,0.894118,0.788235}%
\pgfsetfillcolor{currentfill}%
\pgfsetlinewidth{0.000000pt}%
\definecolor{currentstroke}{rgb}{1.000000,0.894118,0.788235}%
\pgfsetstrokecolor{currentstroke}%
\pgfsetdash{}{0pt}%
\pgfpathmoveto{\pgfqpoint{2.979627in}{2.384043in}}%
\pgfpathlineto{\pgfqpoint{3.090524in}{2.448069in}}%
\pgfpathlineto{\pgfqpoint{3.063876in}{2.456239in}}%
\pgfpathlineto{\pgfqpoint{2.952980in}{2.392213in}}%
\pgfpathlineto{\pgfqpoint{2.979627in}{2.384043in}}%
\pgfpathclose%
\pgfusepath{fill}%
\end{pgfscope}%
\begin{pgfscope}%
\pgfpathrectangle{\pgfqpoint{0.765000in}{0.660000in}}{\pgfqpoint{4.620000in}{4.620000in}}%
\pgfusepath{clip}%
\pgfsetbuttcap%
\pgfsetroundjoin%
\definecolor{currentfill}{rgb}{1.000000,0.894118,0.788235}%
\pgfsetfillcolor{currentfill}%
\pgfsetlinewidth{0.000000pt}%
\definecolor{currentstroke}{rgb}{1.000000,0.894118,0.788235}%
\pgfsetstrokecolor{currentstroke}%
\pgfsetdash{}{0pt}%
\pgfpathmoveto{\pgfqpoint{2.979627in}{2.640148in}}%
\pgfpathlineto{\pgfqpoint{2.868730in}{2.576122in}}%
\pgfpathlineto{\pgfqpoint{2.842083in}{2.584292in}}%
\pgfpathlineto{\pgfqpoint{2.952980in}{2.648318in}}%
\pgfpathlineto{\pgfqpoint{2.979627in}{2.640148in}}%
\pgfpathclose%
\pgfusepath{fill}%
\end{pgfscope}%
\begin{pgfscope}%
\pgfpathrectangle{\pgfqpoint{0.765000in}{0.660000in}}{\pgfqpoint{4.620000in}{4.620000in}}%
\pgfusepath{clip}%
\pgfsetbuttcap%
\pgfsetroundjoin%
\definecolor{currentfill}{rgb}{1.000000,0.894118,0.788235}%
\pgfsetfillcolor{currentfill}%
\pgfsetlinewidth{0.000000pt}%
\definecolor{currentstroke}{rgb}{1.000000,0.894118,0.788235}%
\pgfsetstrokecolor{currentstroke}%
\pgfsetdash{}{0pt}%
\pgfpathmoveto{\pgfqpoint{3.090524in}{2.576122in}}%
\pgfpathlineto{\pgfqpoint{2.979627in}{2.640148in}}%
\pgfpathlineto{\pgfqpoint{2.952980in}{2.648318in}}%
\pgfpathlineto{\pgfqpoint{3.063876in}{2.584292in}}%
\pgfpathlineto{\pgfqpoint{3.090524in}{2.576122in}}%
\pgfpathclose%
\pgfusepath{fill}%
\end{pgfscope}%
\begin{pgfscope}%
\pgfpathrectangle{\pgfqpoint{0.765000in}{0.660000in}}{\pgfqpoint{4.620000in}{4.620000in}}%
\pgfusepath{clip}%
\pgfsetbuttcap%
\pgfsetroundjoin%
\definecolor{currentfill}{rgb}{1.000000,0.894118,0.788235}%
\pgfsetfillcolor{currentfill}%
\pgfsetlinewidth{0.000000pt}%
\definecolor{currentstroke}{rgb}{1.000000,0.894118,0.788235}%
\pgfsetstrokecolor{currentstroke}%
\pgfsetdash{}{0pt}%
\pgfpathmoveto{\pgfqpoint{2.868730in}{2.448069in}}%
\pgfpathlineto{\pgfqpoint{2.868730in}{2.576122in}}%
\pgfpathlineto{\pgfqpoint{2.842083in}{2.584292in}}%
\pgfpathlineto{\pgfqpoint{2.952980in}{2.392213in}}%
\pgfpathlineto{\pgfqpoint{2.868730in}{2.448069in}}%
\pgfpathclose%
\pgfusepath{fill}%
\end{pgfscope}%
\begin{pgfscope}%
\pgfpathrectangle{\pgfqpoint{0.765000in}{0.660000in}}{\pgfqpoint{4.620000in}{4.620000in}}%
\pgfusepath{clip}%
\pgfsetbuttcap%
\pgfsetroundjoin%
\definecolor{currentfill}{rgb}{1.000000,0.894118,0.788235}%
\pgfsetfillcolor{currentfill}%
\pgfsetlinewidth{0.000000pt}%
\definecolor{currentstroke}{rgb}{1.000000,0.894118,0.788235}%
\pgfsetstrokecolor{currentstroke}%
\pgfsetdash{}{0pt}%
\pgfpathmoveto{\pgfqpoint{2.979627in}{2.384043in}}%
\pgfpathlineto{\pgfqpoint{2.979627in}{2.512096in}}%
\pgfpathlineto{\pgfqpoint{2.952980in}{2.520265in}}%
\pgfpathlineto{\pgfqpoint{2.842083in}{2.456239in}}%
\pgfpathlineto{\pgfqpoint{2.979627in}{2.384043in}}%
\pgfpathclose%
\pgfusepath{fill}%
\end{pgfscope}%
\begin{pgfscope}%
\pgfpathrectangle{\pgfqpoint{0.765000in}{0.660000in}}{\pgfqpoint{4.620000in}{4.620000in}}%
\pgfusepath{clip}%
\pgfsetbuttcap%
\pgfsetroundjoin%
\definecolor{currentfill}{rgb}{1.000000,0.894118,0.788235}%
\pgfsetfillcolor{currentfill}%
\pgfsetlinewidth{0.000000pt}%
\definecolor{currentstroke}{rgb}{1.000000,0.894118,0.788235}%
\pgfsetstrokecolor{currentstroke}%
\pgfsetdash{}{0pt}%
\pgfpathmoveto{\pgfqpoint{2.979627in}{2.512096in}}%
\pgfpathlineto{\pgfqpoint{3.090524in}{2.576122in}}%
\pgfpathlineto{\pgfqpoint{3.063876in}{2.584292in}}%
\pgfpathlineto{\pgfqpoint{2.952980in}{2.520265in}}%
\pgfpathlineto{\pgfqpoint{2.979627in}{2.512096in}}%
\pgfpathclose%
\pgfusepath{fill}%
\end{pgfscope}%
\begin{pgfscope}%
\pgfpathrectangle{\pgfqpoint{0.765000in}{0.660000in}}{\pgfqpoint{4.620000in}{4.620000in}}%
\pgfusepath{clip}%
\pgfsetbuttcap%
\pgfsetroundjoin%
\definecolor{currentfill}{rgb}{1.000000,0.894118,0.788235}%
\pgfsetfillcolor{currentfill}%
\pgfsetlinewidth{0.000000pt}%
\definecolor{currentstroke}{rgb}{1.000000,0.894118,0.788235}%
\pgfsetstrokecolor{currentstroke}%
\pgfsetdash{}{0pt}%
\pgfpathmoveto{\pgfqpoint{2.868730in}{2.576122in}}%
\pgfpathlineto{\pgfqpoint{2.979627in}{2.512096in}}%
\pgfpathlineto{\pgfqpoint{2.952980in}{2.520265in}}%
\pgfpathlineto{\pgfqpoint{2.842083in}{2.584292in}}%
\pgfpathlineto{\pgfqpoint{2.868730in}{2.576122in}}%
\pgfpathclose%
\pgfusepath{fill}%
\end{pgfscope}%
\begin{pgfscope}%
\pgfpathrectangle{\pgfqpoint{0.765000in}{0.660000in}}{\pgfqpoint{4.620000in}{4.620000in}}%
\pgfusepath{clip}%
\pgfsetbuttcap%
\pgfsetroundjoin%
\definecolor{currentfill}{rgb}{1.000000,0.894118,0.788235}%
\pgfsetfillcolor{currentfill}%
\pgfsetlinewidth{0.000000pt}%
\definecolor{currentstroke}{rgb}{1.000000,0.894118,0.788235}%
\pgfsetstrokecolor{currentstroke}%
\pgfsetdash{}{0pt}%
\pgfpathmoveto{\pgfqpoint{2.979627in}{2.512096in}}%
\pgfpathlineto{\pgfqpoint{2.868730in}{2.448069in}}%
\pgfpathlineto{\pgfqpoint{2.979627in}{2.384043in}}%
\pgfpathlineto{\pgfqpoint{3.090524in}{2.448069in}}%
\pgfpathlineto{\pgfqpoint{2.979627in}{2.512096in}}%
\pgfpathclose%
\pgfusepath{fill}%
\end{pgfscope}%
\begin{pgfscope}%
\pgfpathrectangle{\pgfqpoint{0.765000in}{0.660000in}}{\pgfqpoint{4.620000in}{4.620000in}}%
\pgfusepath{clip}%
\pgfsetbuttcap%
\pgfsetroundjoin%
\definecolor{currentfill}{rgb}{1.000000,0.894118,0.788235}%
\pgfsetfillcolor{currentfill}%
\pgfsetlinewidth{0.000000pt}%
\definecolor{currentstroke}{rgb}{1.000000,0.894118,0.788235}%
\pgfsetstrokecolor{currentstroke}%
\pgfsetdash{}{0pt}%
\pgfpathmoveto{\pgfqpoint{2.979627in}{2.512096in}}%
\pgfpathlineto{\pgfqpoint{2.868730in}{2.448069in}}%
\pgfpathlineto{\pgfqpoint{2.868730in}{2.576122in}}%
\pgfpathlineto{\pgfqpoint{2.979627in}{2.640148in}}%
\pgfpathlineto{\pgfqpoint{2.979627in}{2.512096in}}%
\pgfpathclose%
\pgfusepath{fill}%
\end{pgfscope}%
\begin{pgfscope}%
\pgfpathrectangle{\pgfqpoint{0.765000in}{0.660000in}}{\pgfqpoint{4.620000in}{4.620000in}}%
\pgfusepath{clip}%
\pgfsetbuttcap%
\pgfsetroundjoin%
\definecolor{currentfill}{rgb}{1.000000,0.894118,0.788235}%
\pgfsetfillcolor{currentfill}%
\pgfsetlinewidth{0.000000pt}%
\definecolor{currentstroke}{rgb}{1.000000,0.894118,0.788235}%
\pgfsetstrokecolor{currentstroke}%
\pgfsetdash{}{0pt}%
\pgfpathmoveto{\pgfqpoint{2.979627in}{2.512096in}}%
\pgfpathlineto{\pgfqpoint{3.090524in}{2.448069in}}%
\pgfpathlineto{\pgfqpoint{3.090524in}{2.576122in}}%
\pgfpathlineto{\pgfqpoint{2.979627in}{2.640148in}}%
\pgfpathlineto{\pgfqpoint{2.979627in}{2.512096in}}%
\pgfpathclose%
\pgfusepath{fill}%
\end{pgfscope}%
\begin{pgfscope}%
\pgfpathrectangle{\pgfqpoint{0.765000in}{0.660000in}}{\pgfqpoint{4.620000in}{4.620000in}}%
\pgfusepath{clip}%
\pgfsetbuttcap%
\pgfsetroundjoin%
\definecolor{currentfill}{rgb}{1.000000,0.894118,0.788235}%
\pgfsetfillcolor{currentfill}%
\pgfsetlinewidth{0.000000pt}%
\definecolor{currentstroke}{rgb}{1.000000,0.894118,0.788235}%
\pgfsetstrokecolor{currentstroke}%
\pgfsetdash{}{0pt}%
\pgfpathmoveto{\pgfqpoint{3.071936in}{2.492937in}}%
\pgfpathlineto{\pgfqpoint{2.961039in}{2.428911in}}%
\pgfpathlineto{\pgfqpoint{3.071936in}{2.364884in}}%
\pgfpathlineto{\pgfqpoint{3.182833in}{2.428911in}}%
\pgfpathlineto{\pgfqpoint{3.071936in}{2.492937in}}%
\pgfpathclose%
\pgfusepath{fill}%
\end{pgfscope}%
\begin{pgfscope}%
\pgfpathrectangle{\pgfqpoint{0.765000in}{0.660000in}}{\pgfqpoint{4.620000in}{4.620000in}}%
\pgfusepath{clip}%
\pgfsetbuttcap%
\pgfsetroundjoin%
\definecolor{currentfill}{rgb}{1.000000,0.894118,0.788235}%
\pgfsetfillcolor{currentfill}%
\pgfsetlinewidth{0.000000pt}%
\definecolor{currentstroke}{rgb}{1.000000,0.894118,0.788235}%
\pgfsetstrokecolor{currentstroke}%
\pgfsetdash{}{0pt}%
\pgfpathmoveto{\pgfqpoint{3.071936in}{2.492937in}}%
\pgfpathlineto{\pgfqpoint{2.961039in}{2.428911in}}%
\pgfpathlineto{\pgfqpoint{2.961039in}{2.556963in}}%
\pgfpathlineto{\pgfqpoint{3.071936in}{2.620989in}}%
\pgfpathlineto{\pgfqpoint{3.071936in}{2.492937in}}%
\pgfpathclose%
\pgfusepath{fill}%
\end{pgfscope}%
\begin{pgfscope}%
\pgfpathrectangle{\pgfqpoint{0.765000in}{0.660000in}}{\pgfqpoint{4.620000in}{4.620000in}}%
\pgfusepath{clip}%
\pgfsetbuttcap%
\pgfsetroundjoin%
\definecolor{currentfill}{rgb}{1.000000,0.894118,0.788235}%
\pgfsetfillcolor{currentfill}%
\pgfsetlinewidth{0.000000pt}%
\definecolor{currentstroke}{rgb}{1.000000,0.894118,0.788235}%
\pgfsetstrokecolor{currentstroke}%
\pgfsetdash{}{0pt}%
\pgfpathmoveto{\pgfqpoint{3.071936in}{2.492937in}}%
\pgfpathlineto{\pgfqpoint{3.182833in}{2.428911in}}%
\pgfpathlineto{\pgfqpoint{3.182833in}{2.556963in}}%
\pgfpathlineto{\pgfqpoint{3.071936in}{2.620989in}}%
\pgfpathlineto{\pgfqpoint{3.071936in}{2.492937in}}%
\pgfpathclose%
\pgfusepath{fill}%
\end{pgfscope}%
\begin{pgfscope}%
\pgfpathrectangle{\pgfqpoint{0.765000in}{0.660000in}}{\pgfqpoint{4.620000in}{4.620000in}}%
\pgfusepath{clip}%
\pgfsetbuttcap%
\pgfsetroundjoin%
\definecolor{currentfill}{rgb}{1.000000,0.894118,0.788235}%
\pgfsetfillcolor{currentfill}%
\pgfsetlinewidth{0.000000pt}%
\definecolor{currentstroke}{rgb}{1.000000,0.894118,0.788235}%
\pgfsetstrokecolor{currentstroke}%
\pgfsetdash{}{0pt}%
\pgfpathmoveto{\pgfqpoint{2.979627in}{2.640148in}}%
\pgfpathlineto{\pgfqpoint{2.868730in}{2.576122in}}%
\pgfpathlineto{\pgfqpoint{2.979627in}{2.512096in}}%
\pgfpathlineto{\pgfqpoint{3.090524in}{2.576122in}}%
\pgfpathlineto{\pgfqpoint{2.979627in}{2.640148in}}%
\pgfpathclose%
\pgfusepath{fill}%
\end{pgfscope}%
\begin{pgfscope}%
\pgfpathrectangle{\pgfqpoint{0.765000in}{0.660000in}}{\pgfqpoint{4.620000in}{4.620000in}}%
\pgfusepath{clip}%
\pgfsetbuttcap%
\pgfsetroundjoin%
\definecolor{currentfill}{rgb}{1.000000,0.894118,0.788235}%
\pgfsetfillcolor{currentfill}%
\pgfsetlinewidth{0.000000pt}%
\definecolor{currentstroke}{rgb}{1.000000,0.894118,0.788235}%
\pgfsetstrokecolor{currentstroke}%
\pgfsetdash{}{0pt}%
\pgfpathmoveto{\pgfqpoint{2.979627in}{2.384043in}}%
\pgfpathlineto{\pgfqpoint{3.090524in}{2.448069in}}%
\pgfpathlineto{\pgfqpoint{3.090524in}{2.576122in}}%
\pgfpathlineto{\pgfqpoint{2.979627in}{2.512096in}}%
\pgfpathlineto{\pgfqpoint{2.979627in}{2.384043in}}%
\pgfpathclose%
\pgfusepath{fill}%
\end{pgfscope}%
\begin{pgfscope}%
\pgfpathrectangle{\pgfqpoint{0.765000in}{0.660000in}}{\pgfqpoint{4.620000in}{4.620000in}}%
\pgfusepath{clip}%
\pgfsetbuttcap%
\pgfsetroundjoin%
\definecolor{currentfill}{rgb}{1.000000,0.894118,0.788235}%
\pgfsetfillcolor{currentfill}%
\pgfsetlinewidth{0.000000pt}%
\definecolor{currentstroke}{rgb}{1.000000,0.894118,0.788235}%
\pgfsetstrokecolor{currentstroke}%
\pgfsetdash{}{0pt}%
\pgfpathmoveto{\pgfqpoint{2.868730in}{2.448069in}}%
\pgfpathlineto{\pgfqpoint{2.979627in}{2.384043in}}%
\pgfpathlineto{\pgfqpoint{2.979627in}{2.512096in}}%
\pgfpathlineto{\pgfqpoint{2.868730in}{2.576122in}}%
\pgfpathlineto{\pgfqpoint{2.868730in}{2.448069in}}%
\pgfpathclose%
\pgfusepath{fill}%
\end{pgfscope}%
\begin{pgfscope}%
\pgfpathrectangle{\pgfqpoint{0.765000in}{0.660000in}}{\pgfqpoint{4.620000in}{4.620000in}}%
\pgfusepath{clip}%
\pgfsetbuttcap%
\pgfsetroundjoin%
\definecolor{currentfill}{rgb}{1.000000,0.894118,0.788235}%
\pgfsetfillcolor{currentfill}%
\pgfsetlinewidth{0.000000pt}%
\definecolor{currentstroke}{rgb}{1.000000,0.894118,0.788235}%
\pgfsetstrokecolor{currentstroke}%
\pgfsetdash{}{0pt}%
\pgfpathmoveto{\pgfqpoint{3.071936in}{2.620989in}}%
\pgfpathlineto{\pgfqpoint{2.961039in}{2.556963in}}%
\pgfpathlineto{\pgfqpoint{3.071936in}{2.492937in}}%
\pgfpathlineto{\pgfqpoint{3.182833in}{2.556963in}}%
\pgfpathlineto{\pgfqpoint{3.071936in}{2.620989in}}%
\pgfpathclose%
\pgfusepath{fill}%
\end{pgfscope}%
\begin{pgfscope}%
\pgfpathrectangle{\pgfqpoint{0.765000in}{0.660000in}}{\pgfqpoint{4.620000in}{4.620000in}}%
\pgfusepath{clip}%
\pgfsetbuttcap%
\pgfsetroundjoin%
\definecolor{currentfill}{rgb}{1.000000,0.894118,0.788235}%
\pgfsetfillcolor{currentfill}%
\pgfsetlinewidth{0.000000pt}%
\definecolor{currentstroke}{rgb}{1.000000,0.894118,0.788235}%
\pgfsetstrokecolor{currentstroke}%
\pgfsetdash{}{0pt}%
\pgfpathmoveto{\pgfqpoint{3.071936in}{2.364884in}}%
\pgfpathlineto{\pgfqpoint{3.182833in}{2.428911in}}%
\pgfpathlineto{\pgfqpoint{3.182833in}{2.556963in}}%
\pgfpathlineto{\pgfqpoint{3.071936in}{2.492937in}}%
\pgfpathlineto{\pgfqpoint{3.071936in}{2.364884in}}%
\pgfpathclose%
\pgfusepath{fill}%
\end{pgfscope}%
\begin{pgfscope}%
\pgfpathrectangle{\pgfqpoint{0.765000in}{0.660000in}}{\pgfqpoint{4.620000in}{4.620000in}}%
\pgfusepath{clip}%
\pgfsetbuttcap%
\pgfsetroundjoin%
\definecolor{currentfill}{rgb}{1.000000,0.894118,0.788235}%
\pgfsetfillcolor{currentfill}%
\pgfsetlinewidth{0.000000pt}%
\definecolor{currentstroke}{rgb}{1.000000,0.894118,0.788235}%
\pgfsetstrokecolor{currentstroke}%
\pgfsetdash{}{0pt}%
\pgfpathmoveto{\pgfqpoint{2.961039in}{2.428911in}}%
\pgfpathlineto{\pgfqpoint{3.071936in}{2.364884in}}%
\pgfpathlineto{\pgfqpoint{3.071936in}{2.492937in}}%
\pgfpathlineto{\pgfqpoint{2.961039in}{2.556963in}}%
\pgfpathlineto{\pgfqpoint{2.961039in}{2.428911in}}%
\pgfpathclose%
\pgfusepath{fill}%
\end{pgfscope}%
\begin{pgfscope}%
\pgfpathrectangle{\pgfqpoint{0.765000in}{0.660000in}}{\pgfqpoint{4.620000in}{4.620000in}}%
\pgfusepath{clip}%
\pgfsetbuttcap%
\pgfsetroundjoin%
\definecolor{currentfill}{rgb}{1.000000,0.894118,0.788235}%
\pgfsetfillcolor{currentfill}%
\pgfsetlinewidth{0.000000pt}%
\definecolor{currentstroke}{rgb}{1.000000,0.894118,0.788235}%
\pgfsetstrokecolor{currentstroke}%
\pgfsetdash{}{0pt}%
\pgfpathmoveto{\pgfqpoint{2.979627in}{2.512096in}}%
\pgfpathlineto{\pgfqpoint{2.868730in}{2.448069in}}%
\pgfpathlineto{\pgfqpoint{2.961039in}{2.428911in}}%
\pgfpathlineto{\pgfqpoint{3.071936in}{2.492937in}}%
\pgfpathlineto{\pgfqpoint{2.979627in}{2.512096in}}%
\pgfpathclose%
\pgfusepath{fill}%
\end{pgfscope}%
\begin{pgfscope}%
\pgfpathrectangle{\pgfqpoint{0.765000in}{0.660000in}}{\pgfqpoint{4.620000in}{4.620000in}}%
\pgfusepath{clip}%
\pgfsetbuttcap%
\pgfsetroundjoin%
\definecolor{currentfill}{rgb}{1.000000,0.894118,0.788235}%
\pgfsetfillcolor{currentfill}%
\pgfsetlinewidth{0.000000pt}%
\definecolor{currentstroke}{rgb}{1.000000,0.894118,0.788235}%
\pgfsetstrokecolor{currentstroke}%
\pgfsetdash{}{0pt}%
\pgfpathmoveto{\pgfqpoint{3.090524in}{2.448069in}}%
\pgfpathlineto{\pgfqpoint{2.979627in}{2.512096in}}%
\pgfpathlineto{\pgfqpoint{3.071936in}{2.492937in}}%
\pgfpathlineto{\pgfqpoint{3.182833in}{2.428911in}}%
\pgfpathlineto{\pgfqpoint{3.090524in}{2.448069in}}%
\pgfpathclose%
\pgfusepath{fill}%
\end{pgfscope}%
\begin{pgfscope}%
\pgfpathrectangle{\pgfqpoint{0.765000in}{0.660000in}}{\pgfqpoint{4.620000in}{4.620000in}}%
\pgfusepath{clip}%
\pgfsetbuttcap%
\pgfsetroundjoin%
\definecolor{currentfill}{rgb}{1.000000,0.894118,0.788235}%
\pgfsetfillcolor{currentfill}%
\pgfsetlinewidth{0.000000pt}%
\definecolor{currentstroke}{rgb}{1.000000,0.894118,0.788235}%
\pgfsetstrokecolor{currentstroke}%
\pgfsetdash{}{0pt}%
\pgfpathmoveto{\pgfqpoint{2.979627in}{2.512096in}}%
\pgfpathlineto{\pgfqpoint{2.979627in}{2.640148in}}%
\pgfpathlineto{\pgfqpoint{3.071936in}{2.620989in}}%
\pgfpathlineto{\pgfqpoint{3.182833in}{2.428911in}}%
\pgfpathlineto{\pgfqpoint{2.979627in}{2.512096in}}%
\pgfpathclose%
\pgfusepath{fill}%
\end{pgfscope}%
\begin{pgfscope}%
\pgfpathrectangle{\pgfqpoint{0.765000in}{0.660000in}}{\pgfqpoint{4.620000in}{4.620000in}}%
\pgfusepath{clip}%
\pgfsetbuttcap%
\pgfsetroundjoin%
\definecolor{currentfill}{rgb}{1.000000,0.894118,0.788235}%
\pgfsetfillcolor{currentfill}%
\pgfsetlinewidth{0.000000pt}%
\definecolor{currentstroke}{rgb}{1.000000,0.894118,0.788235}%
\pgfsetstrokecolor{currentstroke}%
\pgfsetdash{}{0pt}%
\pgfpathmoveto{\pgfqpoint{3.090524in}{2.448069in}}%
\pgfpathlineto{\pgfqpoint{3.090524in}{2.576122in}}%
\pgfpathlineto{\pgfqpoint{3.182833in}{2.556963in}}%
\pgfpathlineto{\pgfqpoint{3.071936in}{2.492937in}}%
\pgfpathlineto{\pgfqpoint{3.090524in}{2.448069in}}%
\pgfpathclose%
\pgfusepath{fill}%
\end{pgfscope}%
\begin{pgfscope}%
\pgfpathrectangle{\pgfqpoint{0.765000in}{0.660000in}}{\pgfqpoint{4.620000in}{4.620000in}}%
\pgfusepath{clip}%
\pgfsetbuttcap%
\pgfsetroundjoin%
\definecolor{currentfill}{rgb}{1.000000,0.894118,0.788235}%
\pgfsetfillcolor{currentfill}%
\pgfsetlinewidth{0.000000pt}%
\definecolor{currentstroke}{rgb}{1.000000,0.894118,0.788235}%
\pgfsetstrokecolor{currentstroke}%
\pgfsetdash{}{0pt}%
\pgfpathmoveto{\pgfqpoint{2.868730in}{2.448069in}}%
\pgfpathlineto{\pgfqpoint{2.979627in}{2.384043in}}%
\pgfpathlineto{\pgfqpoint{3.071936in}{2.364884in}}%
\pgfpathlineto{\pgfqpoint{2.961039in}{2.428911in}}%
\pgfpathlineto{\pgfqpoint{2.868730in}{2.448069in}}%
\pgfpathclose%
\pgfusepath{fill}%
\end{pgfscope}%
\begin{pgfscope}%
\pgfpathrectangle{\pgfqpoint{0.765000in}{0.660000in}}{\pgfqpoint{4.620000in}{4.620000in}}%
\pgfusepath{clip}%
\pgfsetbuttcap%
\pgfsetroundjoin%
\definecolor{currentfill}{rgb}{1.000000,0.894118,0.788235}%
\pgfsetfillcolor{currentfill}%
\pgfsetlinewidth{0.000000pt}%
\definecolor{currentstroke}{rgb}{1.000000,0.894118,0.788235}%
\pgfsetstrokecolor{currentstroke}%
\pgfsetdash{}{0pt}%
\pgfpathmoveto{\pgfqpoint{2.979627in}{2.384043in}}%
\pgfpathlineto{\pgfqpoint{3.090524in}{2.448069in}}%
\pgfpathlineto{\pgfqpoint{3.182833in}{2.428911in}}%
\pgfpathlineto{\pgfqpoint{3.071936in}{2.364884in}}%
\pgfpathlineto{\pgfqpoint{2.979627in}{2.384043in}}%
\pgfpathclose%
\pgfusepath{fill}%
\end{pgfscope}%
\begin{pgfscope}%
\pgfpathrectangle{\pgfqpoint{0.765000in}{0.660000in}}{\pgfqpoint{4.620000in}{4.620000in}}%
\pgfusepath{clip}%
\pgfsetbuttcap%
\pgfsetroundjoin%
\definecolor{currentfill}{rgb}{1.000000,0.894118,0.788235}%
\pgfsetfillcolor{currentfill}%
\pgfsetlinewidth{0.000000pt}%
\definecolor{currentstroke}{rgb}{1.000000,0.894118,0.788235}%
\pgfsetstrokecolor{currentstroke}%
\pgfsetdash{}{0pt}%
\pgfpathmoveto{\pgfqpoint{2.979627in}{2.640148in}}%
\pgfpathlineto{\pgfqpoint{2.868730in}{2.576122in}}%
\pgfpathlineto{\pgfqpoint{2.961039in}{2.556963in}}%
\pgfpathlineto{\pgfqpoint{3.071936in}{2.620989in}}%
\pgfpathlineto{\pgfqpoint{2.979627in}{2.640148in}}%
\pgfpathclose%
\pgfusepath{fill}%
\end{pgfscope}%
\begin{pgfscope}%
\pgfpathrectangle{\pgfqpoint{0.765000in}{0.660000in}}{\pgfqpoint{4.620000in}{4.620000in}}%
\pgfusepath{clip}%
\pgfsetbuttcap%
\pgfsetroundjoin%
\definecolor{currentfill}{rgb}{1.000000,0.894118,0.788235}%
\pgfsetfillcolor{currentfill}%
\pgfsetlinewidth{0.000000pt}%
\definecolor{currentstroke}{rgb}{1.000000,0.894118,0.788235}%
\pgfsetstrokecolor{currentstroke}%
\pgfsetdash{}{0pt}%
\pgfpathmoveto{\pgfqpoint{3.090524in}{2.576122in}}%
\pgfpathlineto{\pgfqpoint{2.979627in}{2.640148in}}%
\pgfpathlineto{\pgfqpoint{3.071936in}{2.620989in}}%
\pgfpathlineto{\pgfqpoint{3.182833in}{2.556963in}}%
\pgfpathlineto{\pgfqpoint{3.090524in}{2.576122in}}%
\pgfpathclose%
\pgfusepath{fill}%
\end{pgfscope}%
\begin{pgfscope}%
\pgfpathrectangle{\pgfqpoint{0.765000in}{0.660000in}}{\pgfqpoint{4.620000in}{4.620000in}}%
\pgfusepath{clip}%
\pgfsetbuttcap%
\pgfsetroundjoin%
\definecolor{currentfill}{rgb}{1.000000,0.894118,0.788235}%
\pgfsetfillcolor{currentfill}%
\pgfsetlinewidth{0.000000pt}%
\definecolor{currentstroke}{rgb}{1.000000,0.894118,0.788235}%
\pgfsetstrokecolor{currentstroke}%
\pgfsetdash{}{0pt}%
\pgfpathmoveto{\pgfqpoint{2.868730in}{2.448069in}}%
\pgfpathlineto{\pgfqpoint{2.868730in}{2.576122in}}%
\pgfpathlineto{\pgfqpoint{2.961039in}{2.556963in}}%
\pgfpathlineto{\pgfqpoint{3.071936in}{2.364884in}}%
\pgfpathlineto{\pgfqpoint{2.868730in}{2.448069in}}%
\pgfpathclose%
\pgfusepath{fill}%
\end{pgfscope}%
\begin{pgfscope}%
\pgfpathrectangle{\pgfqpoint{0.765000in}{0.660000in}}{\pgfqpoint{4.620000in}{4.620000in}}%
\pgfusepath{clip}%
\pgfsetbuttcap%
\pgfsetroundjoin%
\definecolor{currentfill}{rgb}{1.000000,0.894118,0.788235}%
\pgfsetfillcolor{currentfill}%
\pgfsetlinewidth{0.000000pt}%
\definecolor{currentstroke}{rgb}{1.000000,0.894118,0.788235}%
\pgfsetstrokecolor{currentstroke}%
\pgfsetdash{}{0pt}%
\pgfpathmoveto{\pgfqpoint{2.979627in}{2.384043in}}%
\pgfpathlineto{\pgfqpoint{2.979627in}{2.512096in}}%
\pgfpathlineto{\pgfqpoint{3.071936in}{2.492937in}}%
\pgfpathlineto{\pgfqpoint{2.961039in}{2.428911in}}%
\pgfpathlineto{\pgfqpoint{2.979627in}{2.384043in}}%
\pgfpathclose%
\pgfusepath{fill}%
\end{pgfscope}%
\begin{pgfscope}%
\pgfpathrectangle{\pgfqpoint{0.765000in}{0.660000in}}{\pgfqpoint{4.620000in}{4.620000in}}%
\pgfusepath{clip}%
\pgfsetbuttcap%
\pgfsetroundjoin%
\definecolor{currentfill}{rgb}{1.000000,0.894118,0.788235}%
\pgfsetfillcolor{currentfill}%
\pgfsetlinewidth{0.000000pt}%
\definecolor{currentstroke}{rgb}{1.000000,0.894118,0.788235}%
\pgfsetstrokecolor{currentstroke}%
\pgfsetdash{}{0pt}%
\pgfpathmoveto{\pgfqpoint{2.979627in}{2.512096in}}%
\pgfpathlineto{\pgfqpoint{3.090524in}{2.576122in}}%
\pgfpathlineto{\pgfqpoint{3.182833in}{2.556963in}}%
\pgfpathlineto{\pgfqpoint{3.071936in}{2.492937in}}%
\pgfpathlineto{\pgfqpoint{2.979627in}{2.512096in}}%
\pgfpathclose%
\pgfusepath{fill}%
\end{pgfscope}%
\begin{pgfscope}%
\pgfpathrectangle{\pgfqpoint{0.765000in}{0.660000in}}{\pgfqpoint{4.620000in}{4.620000in}}%
\pgfusepath{clip}%
\pgfsetbuttcap%
\pgfsetroundjoin%
\definecolor{currentfill}{rgb}{1.000000,0.894118,0.788235}%
\pgfsetfillcolor{currentfill}%
\pgfsetlinewidth{0.000000pt}%
\definecolor{currentstroke}{rgb}{1.000000,0.894118,0.788235}%
\pgfsetstrokecolor{currentstroke}%
\pgfsetdash{}{0pt}%
\pgfpathmoveto{\pgfqpoint{2.868730in}{2.576122in}}%
\pgfpathlineto{\pgfqpoint{2.979627in}{2.512096in}}%
\pgfpathlineto{\pgfqpoint{3.071936in}{2.492937in}}%
\pgfpathlineto{\pgfqpoint{2.961039in}{2.556963in}}%
\pgfpathlineto{\pgfqpoint{2.868730in}{2.576122in}}%
\pgfpathclose%
\pgfusepath{fill}%
\end{pgfscope}%
\begin{pgfscope}%
\pgfpathrectangle{\pgfqpoint{0.765000in}{0.660000in}}{\pgfqpoint{4.620000in}{4.620000in}}%
\pgfusepath{clip}%
\pgfsetbuttcap%
\pgfsetroundjoin%
\definecolor{currentfill}{rgb}{1.000000,0.894118,0.788235}%
\pgfsetfillcolor{currentfill}%
\pgfsetlinewidth{0.000000pt}%
\definecolor{currentstroke}{rgb}{1.000000,0.894118,0.788235}%
\pgfsetstrokecolor{currentstroke}%
\pgfsetdash{}{0pt}%
\pgfpathmoveto{\pgfqpoint{2.952980in}{2.520265in}}%
\pgfpathlineto{\pgfqpoint{2.842083in}{2.456239in}}%
\pgfpathlineto{\pgfqpoint{2.952980in}{2.392213in}}%
\pgfpathlineto{\pgfqpoint{3.063876in}{2.456239in}}%
\pgfpathlineto{\pgfqpoint{2.952980in}{2.520265in}}%
\pgfpathclose%
\pgfusepath{fill}%
\end{pgfscope}%
\begin{pgfscope}%
\pgfpathrectangle{\pgfqpoint{0.765000in}{0.660000in}}{\pgfqpoint{4.620000in}{4.620000in}}%
\pgfusepath{clip}%
\pgfsetbuttcap%
\pgfsetroundjoin%
\definecolor{currentfill}{rgb}{1.000000,0.894118,0.788235}%
\pgfsetfillcolor{currentfill}%
\pgfsetlinewidth{0.000000pt}%
\definecolor{currentstroke}{rgb}{1.000000,0.894118,0.788235}%
\pgfsetstrokecolor{currentstroke}%
\pgfsetdash{}{0pt}%
\pgfpathmoveto{\pgfqpoint{2.952980in}{2.520265in}}%
\pgfpathlineto{\pgfqpoint{2.842083in}{2.456239in}}%
\pgfpathlineto{\pgfqpoint{2.842083in}{2.584292in}}%
\pgfpathlineto{\pgfqpoint{2.952980in}{2.648318in}}%
\pgfpathlineto{\pgfqpoint{2.952980in}{2.520265in}}%
\pgfpathclose%
\pgfusepath{fill}%
\end{pgfscope}%
\begin{pgfscope}%
\pgfpathrectangle{\pgfqpoint{0.765000in}{0.660000in}}{\pgfqpoint{4.620000in}{4.620000in}}%
\pgfusepath{clip}%
\pgfsetbuttcap%
\pgfsetroundjoin%
\definecolor{currentfill}{rgb}{1.000000,0.894118,0.788235}%
\pgfsetfillcolor{currentfill}%
\pgfsetlinewidth{0.000000pt}%
\definecolor{currentstroke}{rgb}{1.000000,0.894118,0.788235}%
\pgfsetstrokecolor{currentstroke}%
\pgfsetdash{}{0pt}%
\pgfpathmoveto{\pgfqpoint{2.952980in}{2.520265in}}%
\pgfpathlineto{\pgfqpoint{3.063876in}{2.456239in}}%
\pgfpathlineto{\pgfqpoint{3.063876in}{2.584292in}}%
\pgfpathlineto{\pgfqpoint{2.952980in}{2.648318in}}%
\pgfpathlineto{\pgfqpoint{2.952980in}{2.520265in}}%
\pgfpathclose%
\pgfusepath{fill}%
\end{pgfscope}%
\begin{pgfscope}%
\pgfpathrectangle{\pgfqpoint{0.765000in}{0.660000in}}{\pgfqpoint{4.620000in}{4.620000in}}%
\pgfusepath{clip}%
\pgfsetbuttcap%
\pgfsetroundjoin%
\definecolor{currentfill}{rgb}{1.000000,0.894118,0.788235}%
\pgfsetfillcolor{currentfill}%
\pgfsetlinewidth{0.000000pt}%
\definecolor{currentstroke}{rgb}{1.000000,0.894118,0.788235}%
\pgfsetstrokecolor{currentstroke}%
\pgfsetdash{}{0pt}%
\pgfpathmoveto{\pgfqpoint{2.979627in}{2.512096in}}%
\pgfpathlineto{\pgfqpoint{2.868730in}{2.448069in}}%
\pgfpathlineto{\pgfqpoint{2.979627in}{2.384043in}}%
\pgfpathlineto{\pgfqpoint{3.090524in}{2.448069in}}%
\pgfpathlineto{\pgfqpoint{2.979627in}{2.512096in}}%
\pgfpathclose%
\pgfusepath{fill}%
\end{pgfscope}%
\begin{pgfscope}%
\pgfpathrectangle{\pgfqpoint{0.765000in}{0.660000in}}{\pgfqpoint{4.620000in}{4.620000in}}%
\pgfusepath{clip}%
\pgfsetbuttcap%
\pgfsetroundjoin%
\definecolor{currentfill}{rgb}{1.000000,0.894118,0.788235}%
\pgfsetfillcolor{currentfill}%
\pgfsetlinewidth{0.000000pt}%
\definecolor{currentstroke}{rgb}{1.000000,0.894118,0.788235}%
\pgfsetstrokecolor{currentstroke}%
\pgfsetdash{}{0pt}%
\pgfpathmoveto{\pgfqpoint{2.979627in}{2.512096in}}%
\pgfpathlineto{\pgfqpoint{2.868730in}{2.448069in}}%
\pgfpathlineto{\pgfqpoint{2.868730in}{2.576122in}}%
\pgfpathlineto{\pgfqpoint{2.979627in}{2.640148in}}%
\pgfpathlineto{\pgfqpoint{2.979627in}{2.512096in}}%
\pgfpathclose%
\pgfusepath{fill}%
\end{pgfscope}%
\begin{pgfscope}%
\pgfpathrectangle{\pgfqpoint{0.765000in}{0.660000in}}{\pgfqpoint{4.620000in}{4.620000in}}%
\pgfusepath{clip}%
\pgfsetbuttcap%
\pgfsetroundjoin%
\definecolor{currentfill}{rgb}{1.000000,0.894118,0.788235}%
\pgfsetfillcolor{currentfill}%
\pgfsetlinewidth{0.000000pt}%
\definecolor{currentstroke}{rgb}{1.000000,0.894118,0.788235}%
\pgfsetstrokecolor{currentstroke}%
\pgfsetdash{}{0pt}%
\pgfpathmoveto{\pgfqpoint{2.979627in}{2.512096in}}%
\pgfpathlineto{\pgfqpoint{3.090524in}{2.448069in}}%
\pgfpathlineto{\pgfqpoint{3.090524in}{2.576122in}}%
\pgfpathlineto{\pgfqpoint{2.979627in}{2.640148in}}%
\pgfpathlineto{\pgfqpoint{2.979627in}{2.512096in}}%
\pgfpathclose%
\pgfusepath{fill}%
\end{pgfscope}%
\begin{pgfscope}%
\pgfpathrectangle{\pgfqpoint{0.765000in}{0.660000in}}{\pgfqpoint{4.620000in}{4.620000in}}%
\pgfusepath{clip}%
\pgfsetbuttcap%
\pgfsetroundjoin%
\definecolor{currentfill}{rgb}{1.000000,0.894118,0.788235}%
\pgfsetfillcolor{currentfill}%
\pgfsetlinewidth{0.000000pt}%
\definecolor{currentstroke}{rgb}{1.000000,0.894118,0.788235}%
\pgfsetstrokecolor{currentstroke}%
\pgfsetdash{}{0pt}%
\pgfpathmoveto{\pgfqpoint{2.952980in}{2.648318in}}%
\pgfpathlineto{\pgfqpoint{2.842083in}{2.584292in}}%
\pgfpathlineto{\pgfqpoint{2.952980in}{2.520265in}}%
\pgfpathlineto{\pgfqpoint{3.063876in}{2.584292in}}%
\pgfpathlineto{\pgfqpoint{2.952980in}{2.648318in}}%
\pgfpathclose%
\pgfusepath{fill}%
\end{pgfscope}%
\begin{pgfscope}%
\pgfpathrectangle{\pgfqpoint{0.765000in}{0.660000in}}{\pgfqpoint{4.620000in}{4.620000in}}%
\pgfusepath{clip}%
\pgfsetbuttcap%
\pgfsetroundjoin%
\definecolor{currentfill}{rgb}{1.000000,0.894118,0.788235}%
\pgfsetfillcolor{currentfill}%
\pgfsetlinewidth{0.000000pt}%
\definecolor{currentstroke}{rgb}{1.000000,0.894118,0.788235}%
\pgfsetstrokecolor{currentstroke}%
\pgfsetdash{}{0pt}%
\pgfpathmoveto{\pgfqpoint{2.952980in}{2.392213in}}%
\pgfpathlineto{\pgfqpoint{3.063876in}{2.456239in}}%
\pgfpathlineto{\pgfqpoint{3.063876in}{2.584292in}}%
\pgfpathlineto{\pgfqpoint{2.952980in}{2.520265in}}%
\pgfpathlineto{\pgfqpoint{2.952980in}{2.392213in}}%
\pgfpathclose%
\pgfusepath{fill}%
\end{pgfscope}%
\begin{pgfscope}%
\pgfpathrectangle{\pgfqpoint{0.765000in}{0.660000in}}{\pgfqpoint{4.620000in}{4.620000in}}%
\pgfusepath{clip}%
\pgfsetbuttcap%
\pgfsetroundjoin%
\definecolor{currentfill}{rgb}{1.000000,0.894118,0.788235}%
\pgfsetfillcolor{currentfill}%
\pgfsetlinewidth{0.000000pt}%
\definecolor{currentstroke}{rgb}{1.000000,0.894118,0.788235}%
\pgfsetstrokecolor{currentstroke}%
\pgfsetdash{}{0pt}%
\pgfpathmoveto{\pgfqpoint{2.842083in}{2.456239in}}%
\pgfpathlineto{\pgfqpoint{2.952980in}{2.392213in}}%
\pgfpathlineto{\pgfqpoint{2.952980in}{2.520265in}}%
\pgfpathlineto{\pgfqpoint{2.842083in}{2.584292in}}%
\pgfpathlineto{\pgfqpoint{2.842083in}{2.456239in}}%
\pgfpathclose%
\pgfusepath{fill}%
\end{pgfscope}%
\begin{pgfscope}%
\pgfpathrectangle{\pgfqpoint{0.765000in}{0.660000in}}{\pgfqpoint{4.620000in}{4.620000in}}%
\pgfusepath{clip}%
\pgfsetbuttcap%
\pgfsetroundjoin%
\definecolor{currentfill}{rgb}{1.000000,0.894118,0.788235}%
\pgfsetfillcolor{currentfill}%
\pgfsetlinewidth{0.000000pt}%
\definecolor{currentstroke}{rgb}{1.000000,0.894118,0.788235}%
\pgfsetstrokecolor{currentstroke}%
\pgfsetdash{}{0pt}%
\pgfpathmoveto{\pgfqpoint{2.979627in}{2.640148in}}%
\pgfpathlineto{\pgfqpoint{2.868730in}{2.576122in}}%
\pgfpathlineto{\pgfqpoint{2.979627in}{2.512096in}}%
\pgfpathlineto{\pgfqpoint{3.090524in}{2.576122in}}%
\pgfpathlineto{\pgfqpoint{2.979627in}{2.640148in}}%
\pgfpathclose%
\pgfusepath{fill}%
\end{pgfscope}%
\begin{pgfscope}%
\pgfpathrectangle{\pgfqpoint{0.765000in}{0.660000in}}{\pgfqpoint{4.620000in}{4.620000in}}%
\pgfusepath{clip}%
\pgfsetbuttcap%
\pgfsetroundjoin%
\definecolor{currentfill}{rgb}{1.000000,0.894118,0.788235}%
\pgfsetfillcolor{currentfill}%
\pgfsetlinewidth{0.000000pt}%
\definecolor{currentstroke}{rgb}{1.000000,0.894118,0.788235}%
\pgfsetstrokecolor{currentstroke}%
\pgfsetdash{}{0pt}%
\pgfpathmoveto{\pgfqpoint{2.979627in}{2.384043in}}%
\pgfpathlineto{\pgfqpoint{3.090524in}{2.448069in}}%
\pgfpathlineto{\pgfqpoint{3.090524in}{2.576122in}}%
\pgfpathlineto{\pgfqpoint{2.979627in}{2.512096in}}%
\pgfpathlineto{\pgfqpoint{2.979627in}{2.384043in}}%
\pgfpathclose%
\pgfusepath{fill}%
\end{pgfscope}%
\begin{pgfscope}%
\pgfpathrectangle{\pgfqpoint{0.765000in}{0.660000in}}{\pgfqpoint{4.620000in}{4.620000in}}%
\pgfusepath{clip}%
\pgfsetbuttcap%
\pgfsetroundjoin%
\definecolor{currentfill}{rgb}{1.000000,0.894118,0.788235}%
\pgfsetfillcolor{currentfill}%
\pgfsetlinewidth{0.000000pt}%
\definecolor{currentstroke}{rgb}{1.000000,0.894118,0.788235}%
\pgfsetstrokecolor{currentstroke}%
\pgfsetdash{}{0pt}%
\pgfpathmoveto{\pgfqpoint{2.868730in}{2.448069in}}%
\pgfpathlineto{\pgfqpoint{2.979627in}{2.384043in}}%
\pgfpathlineto{\pgfqpoint{2.979627in}{2.512096in}}%
\pgfpathlineto{\pgfqpoint{2.868730in}{2.576122in}}%
\pgfpathlineto{\pgfqpoint{2.868730in}{2.448069in}}%
\pgfpathclose%
\pgfusepath{fill}%
\end{pgfscope}%
\begin{pgfscope}%
\pgfpathrectangle{\pgfqpoint{0.765000in}{0.660000in}}{\pgfqpoint{4.620000in}{4.620000in}}%
\pgfusepath{clip}%
\pgfsetbuttcap%
\pgfsetroundjoin%
\definecolor{currentfill}{rgb}{1.000000,0.894118,0.788235}%
\pgfsetfillcolor{currentfill}%
\pgfsetlinewidth{0.000000pt}%
\definecolor{currentstroke}{rgb}{1.000000,0.894118,0.788235}%
\pgfsetstrokecolor{currentstroke}%
\pgfsetdash{}{0pt}%
\pgfpathmoveto{\pgfqpoint{2.952980in}{2.520265in}}%
\pgfpathlineto{\pgfqpoint{2.842083in}{2.456239in}}%
\pgfpathlineto{\pgfqpoint{2.868730in}{2.448069in}}%
\pgfpathlineto{\pgfqpoint{2.979627in}{2.512096in}}%
\pgfpathlineto{\pgfqpoint{2.952980in}{2.520265in}}%
\pgfpathclose%
\pgfusepath{fill}%
\end{pgfscope}%
\begin{pgfscope}%
\pgfpathrectangle{\pgfqpoint{0.765000in}{0.660000in}}{\pgfqpoint{4.620000in}{4.620000in}}%
\pgfusepath{clip}%
\pgfsetbuttcap%
\pgfsetroundjoin%
\definecolor{currentfill}{rgb}{1.000000,0.894118,0.788235}%
\pgfsetfillcolor{currentfill}%
\pgfsetlinewidth{0.000000pt}%
\definecolor{currentstroke}{rgb}{1.000000,0.894118,0.788235}%
\pgfsetstrokecolor{currentstroke}%
\pgfsetdash{}{0pt}%
\pgfpathmoveto{\pgfqpoint{3.063876in}{2.456239in}}%
\pgfpathlineto{\pgfqpoint{2.952980in}{2.520265in}}%
\pgfpathlineto{\pgfqpoint{2.979627in}{2.512096in}}%
\pgfpathlineto{\pgfqpoint{3.090524in}{2.448069in}}%
\pgfpathlineto{\pgfqpoint{3.063876in}{2.456239in}}%
\pgfpathclose%
\pgfusepath{fill}%
\end{pgfscope}%
\begin{pgfscope}%
\pgfpathrectangle{\pgfqpoint{0.765000in}{0.660000in}}{\pgfqpoint{4.620000in}{4.620000in}}%
\pgfusepath{clip}%
\pgfsetbuttcap%
\pgfsetroundjoin%
\definecolor{currentfill}{rgb}{1.000000,0.894118,0.788235}%
\pgfsetfillcolor{currentfill}%
\pgfsetlinewidth{0.000000pt}%
\definecolor{currentstroke}{rgb}{1.000000,0.894118,0.788235}%
\pgfsetstrokecolor{currentstroke}%
\pgfsetdash{}{0pt}%
\pgfpathmoveto{\pgfqpoint{2.952980in}{2.520265in}}%
\pgfpathlineto{\pgfqpoint{2.952980in}{2.648318in}}%
\pgfpathlineto{\pgfqpoint{2.979627in}{2.640148in}}%
\pgfpathlineto{\pgfqpoint{3.090524in}{2.448069in}}%
\pgfpathlineto{\pgfqpoint{2.952980in}{2.520265in}}%
\pgfpathclose%
\pgfusepath{fill}%
\end{pgfscope}%
\begin{pgfscope}%
\pgfpathrectangle{\pgfqpoint{0.765000in}{0.660000in}}{\pgfqpoint{4.620000in}{4.620000in}}%
\pgfusepath{clip}%
\pgfsetbuttcap%
\pgfsetroundjoin%
\definecolor{currentfill}{rgb}{1.000000,0.894118,0.788235}%
\pgfsetfillcolor{currentfill}%
\pgfsetlinewidth{0.000000pt}%
\definecolor{currentstroke}{rgb}{1.000000,0.894118,0.788235}%
\pgfsetstrokecolor{currentstroke}%
\pgfsetdash{}{0pt}%
\pgfpathmoveto{\pgfqpoint{3.063876in}{2.456239in}}%
\pgfpathlineto{\pgfqpoint{3.063876in}{2.584292in}}%
\pgfpathlineto{\pgfqpoint{3.090524in}{2.576122in}}%
\pgfpathlineto{\pgfqpoint{2.979627in}{2.512096in}}%
\pgfpathlineto{\pgfqpoint{3.063876in}{2.456239in}}%
\pgfpathclose%
\pgfusepath{fill}%
\end{pgfscope}%
\begin{pgfscope}%
\pgfpathrectangle{\pgfqpoint{0.765000in}{0.660000in}}{\pgfqpoint{4.620000in}{4.620000in}}%
\pgfusepath{clip}%
\pgfsetbuttcap%
\pgfsetroundjoin%
\definecolor{currentfill}{rgb}{1.000000,0.894118,0.788235}%
\pgfsetfillcolor{currentfill}%
\pgfsetlinewidth{0.000000pt}%
\definecolor{currentstroke}{rgb}{1.000000,0.894118,0.788235}%
\pgfsetstrokecolor{currentstroke}%
\pgfsetdash{}{0pt}%
\pgfpathmoveto{\pgfqpoint{2.842083in}{2.456239in}}%
\pgfpathlineto{\pgfqpoint{2.952980in}{2.392213in}}%
\pgfpathlineto{\pgfqpoint{2.979627in}{2.384043in}}%
\pgfpathlineto{\pgfqpoint{2.868730in}{2.448069in}}%
\pgfpathlineto{\pgfqpoint{2.842083in}{2.456239in}}%
\pgfpathclose%
\pgfusepath{fill}%
\end{pgfscope}%
\begin{pgfscope}%
\pgfpathrectangle{\pgfqpoint{0.765000in}{0.660000in}}{\pgfqpoint{4.620000in}{4.620000in}}%
\pgfusepath{clip}%
\pgfsetbuttcap%
\pgfsetroundjoin%
\definecolor{currentfill}{rgb}{1.000000,0.894118,0.788235}%
\pgfsetfillcolor{currentfill}%
\pgfsetlinewidth{0.000000pt}%
\definecolor{currentstroke}{rgb}{1.000000,0.894118,0.788235}%
\pgfsetstrokecolor{currentstroke}%
\pgfsetdash{}{0pt}%
\pgfpathmoveto{\pgfqpoint{2.952980in}{2.392213in}}%
\pgfpathlineto{\pgfqpoint{3.063876in}{2.456239in}}%
\pgfpathlineto{\pgfqpoint{3.090524in}{2.448069in}}%
\pgfpathlineto{\pgfqpoint{2.979627in}{2.384043in}}%
\pgfpathlineto{\pgfqpoint{2.952980in}{2.392213in}}%
\pgfpathclose%
\pgfusepath{fill}%
\end{pgfscope}%
\begin{pgfscope}%
\pgfpathrectangle{\pgfqpoint{0.765000in}{0.660000in}}{\pgfqpoint{4.620000in}{4.620000in}}%
\pgfusepath{clip}%
\pgfsetbuttcap%
\pgfsetroundjoin%
\definecolor{currentfill}{rgb}{1.000000,0.894118,0.788235}%
\pgfsetfillcolor{currentfill}%
\pgfsetlinewidth{0.000000pt}%
\definecolor{currentstroke}{rgb}{1.000000,0.894118,0.788235}%
\pgfsetstrokecolor{currentstroke}%
\pgfsetdash{}{0pt}%
\pgfpathmoveto{\pgfqpoint{2.952980in}{2.648318in}}%
\pgfpathlineto{\pgfqpoint{2.842083in}{2.584292in}}%
\pgfpathlineto{\pgfqpoint{2.868730in}{2.576122in}}%
\pgfpathlineto{\pgfqpoint{2.979627in}{2.640148in}}%
\pgfpathlineto{\pgfqpoint{2.952980in}{2.648318in}}%
\pgfpathclose%
\pgfusepath{fill}%
\end{pgfscope}%
\begin{pgfscope}%
\pgfpathrectangle{\pgfqpoint{0.765000in}{0.660000in}}{\pgfqpoint{4.620000in}{4.620000in}}%
\pgfusepath{clip}%
\pgfsetbuttcap%
\pgfsetroundjoin%
\definecolor{currentfill}{rgb}{1.000000,0.894118,0.788235}%
\pgfsetfillcolor{currentfill}%
\pgfsetlinewidth{0.000000pt}%
\definecolor{currentstroke}{rgb}{1.000000,0.894118,0.788235}%
\pgfsetstrokecolor{currentstroke}%
\pgfsetdash{}{0pt}%
\pgfpathmoveto{\pgfqpoint{3.063876in}{2.584292in}}%
\pgfpathlineto{\pgfqpoint{2.952980in}{2.648318in}}%
\pgfpathlineto{\pgfqpoint{2.979627in}{2.640148in}}%
\pgfpathlineto{\pgfqpoint{3.090524in}{2.576122in}}%
\pgfpathlineto{\pgfqpoint{3.063876in}{2.584292in}}%
\pgfpathclose%
\pgfusepath{fill}%
\end{pgfscope}%
\begin{pgfscope}%
\pgfpathrectangle{\pgfqpoint{0.765000in}{0.660000in}}{\pgfqpoint{4.620000in}{4.620000in}}%
\pgfusepath{clip}%
\pgfsetbuttcap%
\pgfsetroundjoin%
\definecolor{currentfill}{rgb}{1.000000,0.894118,0.788235}%
\pgfsetfillcolor{currentfill}%
\pgfsetlinewidth{0.000000pt}%
\definecolor{currentstroke}{rgb}{1.000000,0.894118,0.788235}%
\pgfsetstrokecolor{currentstroke}%
\pgfsetdash{}{0pt}%
\pgfpathmoveto{\pgfqpoint{2.842083in}{2.456239in}}%
\pgfpathlineto{\pgfqpoint{2.842083in}{2.584292in}}%
\pgfpathlineto{\pgfqpoint{2.868730in}{2.576122in}}%
\pgfpathlineto{\pgfqpoint{2.979627in}{2.384043in}}%
\pgfpathlineto{\pgfqpoint{2.842083in}{2.456239in}}%
\pgfpathclose%
\pgfusepath{fill}%
\end{pgfscope}%
\begin{pgfscope}%
\pgfpathrectangle{\pgfqpoint{0.765000in}{0.660000in}}{\pgfqpoint{4.620000in}{4.620000in}}%
\pgfusepath{clip}%
\pgfsetbuttcap%
\pgfsetroundjoin%
\definecolor{currentfill}{rgb}{1.000000,0.894118,0.788235}%
\pgfsetfillcolor{currentfill}%
\pgfsetlinewidth{0.000000pt}%
\definecolor{currentstroke}{rgb}{1.000000,0.894118,0.788235}%
\pgfsetstrokecolor{currentstroke}%
\pgfsetdash{}{0pt}%
\pgfpathmoveto{\pgfqpoint{2.952980in}{2.392213in}}%
\pgfpathlineto{\pgfqpoint{2.952980in}{2.520265in}}%
\pgfpathlineto{\pgfqpoint{2.979627in}{2.512096in}}%
\pgfpathlineto{\pgfqpoint{2.868730in}{2.448069in}}%
\pgfpathlineto{\pgfqpoint{2.952980in}{2.392213in}}%
\pgfpathclose%
\pgfusepath{fill}%
\end{pgfscope}%
\begin{pgfscope}%
\pgfpathrectangle{\pgfqpoint{0.765000in}{0.660000in}}{\pgfqpoint{4.620000in}{4.620000in}}%
\pgfusepath{clip}%
\pgfsetbuttcap%
\pgfsetroundjoin%
\definecolor{currentfill}{rgb}{1.000000,0.894118,0.788235}%
\pgfsetfillcolor{currentfill}%
\pgfsetlinewidth{0.000000pt}%
\definecolor{currentstroke}{rgb}{1.000000,0.894118,0.788235}%
\pgfsetstrokecolor{currentstroke}%
\pgfsetdash{}{0pt}%
\pgfpathmoveto{\pgfqpoint{2.952980in}{2.520265in}}%
\pgfpathlineto{\pgfqpoint{3.063876in}{2.584292in}}%
\pgfpathlineto{\pgfqpoint{3.090524in}{2.576122in}}%
\pgfpathlineto{\pgfqpoint{2.979627in}{2.512096in}}%
\pgfpathlineto{\pgfqpoint{2.952980in}{2.520265in}}%
\pgfpathclose%
\pgfusepath{fill}%
\end{pgfscope}%
\begin{pgfscope}%
\pgfpathrectangle{\pgfqpoint{0.765000in}{0.660000in}}{\pgfqpoint{4.620000in}{4.620000in}}%
\pgfusepath{clip}%
\pgfsetbuttcap%
\pgfsetroundjoin%
\definecolor{currentfill}{rgb}{1.000000,0.894118,0.788235}%
\pgfsetfillcolor{currentfill}%
\pgfsetlinewidth{0.000000pt}%
\definecolor{currentstroke}{rgb}{1.000000,0.894118,0.788235}%
\pgfsetstrokecolor{currentstroke}%
\pgfsetdash{}{0pt}%
\pgfpathmoveto{\pgfqpoint{2.842083in}{2.584292in}}%
\pgfpathlineto{\pgfqpoint{2.952980in}{2.520265in}}%
\pgfpathlineto{\pgfqpoint{2.979627in}{2.512096in}}%
\pgfpathlineto{\pgfqpoint{2.868730in}{2.576122in}}%
\pgfpathlineto{\pgfqpoint{2.842083in}{2.584292in}}%
\pgfpathclose%
\pgfusepath{fill}%
\end{pgfscope}%
\begin{pgfscope}%
\pgfpathrectangle{\pgfqpoint{0.765000in}{0.660000in}}{\pgfqpoint{4.620000in}{4.620000in}}%
\pgfusepath{clip}%
\pgfsetbuttcap%
\pgfsetroundjoin%
\definecolor{currentfill}{rgb}{1.000000,0.894118,0.788235}%
\pgfsetfillcolor{currentfill}%
\pgfsetlinewidth{0.000000pt}%
\definecolor{currentstroke}{rgb}{1.000000,0.894118,0.788235}%
\pgfsetstrokecolor{currentstroke}%
\pgfsetdash{}{0pt}%
\pgfpathmoveto{\pgfqpoint{2.952980in}{2.520265in}}%
\pgfpathlineto{\pgfqpoint{2.842083in}{2.456239in}}%
\pgfpathlineto{\pgfqpoint{2.952980in}{2.392213in}}%
\pgfpathlineto{\pgfqpoint{3.063876in}{2.456239in}}%
\pgfpathlineto{\pgfqpoint{2.952980in}{2.520265in}}%
\pgfpathclose%
\pgfusepath{fill}%
\end{pgfscope}%
\begin{pgfscope}%
\pgfpathrectangle{\pgfqpoint{0.765000in}{0.660000in}}{\pgfqpoint{4.620000in}{4.620000in}}%
\pgfusepath{clip}%
\pgfsetbuttcap%
\pgfsetroundjoin%
\definecolor{currentfill}{rgb}{1.000000,0.894118,0.788235}%
\pgfsetfillcolor{currentfill}%
\pgfsetlinewidth{0.000000pt}%
\definecolor{currentstroke}{rgb}{1.000000,0.894118,0.788235}%
\pgfsetstrokecolor{currentstroke}%
\pgfsetdash{}{0pt}%
\pgfpathmoveto{\pgfqpoint{2.952980in}{2.520265in}}%
\pgfpathlineto{\pgfqpoint{2.842083in}{2.456239in}}%
\pgfpathlineto{\pgfqpoint{2.842083in}{2.584292in}}%
\pgfpathlineto{\pgfqpoint{2.952980in}{2.648318in}}%
\pgfpathlineto{\pgfqpoint{2.952980in}{2.520265in}}%
\pgfpathclose%
\pgfusepath{fill}%
\end{pgfscope}%
\begin{pgfscope}%
\pgfpathrectangle{\pgfqpoint{0.765000in}{0.660000in}}{\pgfqpoint{4.620000in}{4.620000in}}%
\pgfusepath{clip}%
\pgfsetbuttcap%
\pgfsetroundjoin%
\definecolor{currentfill}{rgb}{1.000000,0.894118,0.788235}%
\pgfsetfillcolor{currentfill}%
\pgfsetlinewidth{0.000000pt}%
\definecolor{currentstroke}{rgb}{1.000000,0.894118,0.788235}%
\pgfsetstrokecolor{currentstroke}%
\pgfsetdash{}{0pt}%
\pgfpathmoveto{\pgfqpoint{2.952980in}{2.520265in}}%
\pgfpathlineto{\pgfqpoint{3.063876in}{2.456239in}}%
\pgfpathlineto{\pgfqpoint{3.063876in}{2.584292in}}%
\pgfpathlineto{\pgfqpoint{2.952980in}{2.648318in}}%
\pgfpathlineto{\pgfqpoint{2.952980in}{2.520265in}}%
\pgfpathclose%
\pgfusepath{fill}%
\end{pgfscope}%
\begin{pgfscope}%
\pgfpathrectangle{\pgfqpoint{0.765000in}{0.660000in}}{\pgfqpoint{4.620000in}{4.620000in}}%
\pgfusepath{clip}%
\pgfsetbuttcap%
\pgfsetroundjoin%
\definecolor{currentfill}{rgb}{1.000000,0.894118,0.788235}%
\pgfsetfillcolor{currentfill}%
\pgfsetlinewidth{0.000000pt}%
\definecolor{currentstroke}{rgb}{1.000000,0.894118,0.788235}%
\pgfsetstrokecolor{currentstroke}%
\pgfsetdash{}{0pt}%
\pgfpathmoveto{\pgfqpoint{2.929883in}{2.530541in}}%
\pgfpathlineto{\pgfqpoint{2.818986in}{2.466515in}}%
\pgfpathlineto{\pgfqpoint{2.929883in}{2.402489in}}%
\pgfpathlineto{\pgfqpoint{3.040780in}{2.466515in}}%
\pgfpathlineto{\pgfqpoint{2.929883in}{2.530541in}}%
\pgfpathclose%
\pgfusepath{fill}%
\end{pgfscope}%
\begin{pgfscope}%
\pgfpathrectangle{\pgfqpoint{0.765000in}{0.660000in}}{\pgfqpoint{4.620000in}{4.620000in}}%
\pgfusepath{clip}%
\pgfsetbuttcap%
\pgfsetroundjoin%
\definecolor{currentfill}{rgb}{1.000000,0.894118,0.788235}%
\pgfsetfillcolor{currentfill}%
\pgfsetlinewidth{0.000000pt}%
\definecolor{currentstroke}{rgb}{1.000000,0.894118,0.788235}%
\pgfsetstrokecolor{currentstroke}%
\pgfsetdash{}{0pt}%
\pgfpathmoveto{\pgfqpoint{2.929883in}{2.530541in}}%
\pgfpathlineto{\pgfqpoint{2.818986in}{2.466515in}}%
\pgfpathlineto{\pgfqpoint{2.818986in}{2.594568in}}%
\pgfpathlineto{\pgfqpoint{2.929883in}{2.658594in}}%
\pgfpathlineto{\pgfqpoint{2.929883in}{2.530541in}}%
\pgfpathclose%
\pgfusepath{fill}%
\end{pgfscope}%
\begin{pgfscope}%
\pgfpathrectangle{\pgfqpoint{0.765000in}{0.660000in}}{\pgfqpoint{4.620000in}{4.620000in}}%
\pgfusepath{clip}%
\pgfsetbuttcap%
\pgfsetroundjoin%
\definecolor{currentfill}{rgb}{1.000000,0.894118,0.788235}%
\pgfsetfillcolor{currentfill}%
\pgfsetlinewidth{0.000000pt}%
\definecolor{currentstroke}{rgb}{1.000000,0.894118,0.788235}%
\pgfsetstrokecolor{currentstroke}%
\pgfsetdash{}{0pt}%
\pgfpathmoveto{\pgfqpoint{2.929883in}{2.530541in}}%
\pgfpathlineto{\pgfqpoint{3.040780in}{2.466515in}}%
\pgfpathlineto{\pgfqpoint{3.040780in}{2.594568in}}%
\pgfpathlineto{\pgfqpoint{2.929883in}{2.658594in}}%
\pgfpathlineto{\pgfqpoint{2.929883in}{2.530541in}}%
\pgfpathclose%
\pgfusepath{fill}%
\end{pgfscope}%
\begin{pgfscope}%
\pgfpathrectangle{\pgfqpoint{0.765000in}{0.660000in}}{\pgfqpoint{4.620000in}{4.620000in}}%
\pgfusepath{clip}%
\pgfsetbuttcap%
\pgfsetroundjoin%
\definecolor{currentfill}{rgb}{1.000000,0.894118,0.788235}%
\pgfsetfillcolor{currentfill}%
\pgfsetlinewidth{0.000000pt}%
\definecolor{currentstroke}{rgb}{1.000000,0.894118,0.788235}%
\pgfsetstrokecolor{currentstroke}%
\pgfsetdash{}{0pt}%
\pgfpathmoveto{\pgfqpoint{2.952980in}{2.648318in}}%
\pgfpathlineto{\pgfqpoint{2.842083in}{2.584292in}}%
\pgfpathlineto{\pgfqpoint{2.952980in}{2.520265in}}%
\pgfpathlineto{\pgfqpoint{3.063876in}{2.584292in}}%
\pgfpathlineto{\pgfqpoint{2.952980in}{2.648318in}}%
\pgfpathclose%
\pgfusepath{fill}%
\end{pgfscope}%
\begin{pgfscope}%
\pgfpathrectangle{\pgfqpoint{0.765000in}{0.660000in}}{\pgfqpoint{4.620000in}{4.620000in}}%
\pgfusepath{clip}%
\pgfsetbuttcap%
\pgfsetroundjoin%
\definecolor{currentfill}{rgb}{1.000000,0.894118,0.788235}%
\pgfsetfillcolor{currentfill}%
\pgfsetlinewidth{0.000000pt}%
\definecolor{currentstroke}{rgb}{1.000000,0.894118,0.788235}%
\pgfsetstrokecolor{currentstroke}%
\pgfsetdash{}{0pt}%
\pgfpathmoveto{\pgfqpoint{2.952980in}{2.392213in}}%
\pgfpathlineto{\pgfqpoint{3.063876in}{2.456239in}}%
\pgfpathlineto{\pgfqpoint{3.063876in}{2.584292in}}%
\pgfpathlineto{\pgfqpoint{2.952980in}{2.520265in}}%
\pgfpathlineto{\pgfqpoint{2.952980in}{2.392213in}}%
\pgfpathclose%
\pgfusepath{fill}%
\end{pgfscope}%
\begin{pgfscope}%
\pgfpathrectangle{\pgfqpoint{0.765000in}{0.660000in}}{\pgfqpoint{4.620000in}{4.620000in}}%
\pgfusepath{clip}%
\pgfsetbuttcap%
\pgfsetroundjoin%
\definecolor{currentfill}{rgb}{1.000000,0.894118,0.788235}%
\pgfsetfillcolor{currentfill}%
\pgfsetlinewidth{0.000000pt}%
\definecolor{currentstroke}{rgb}{1.000000,0.894118,0.788235}%
\pgfsetstrokecolor{currentstroke}%
\pgfsetdash{}{0pt}%
\pgfpathmoveto{\pgfqpoint{2.842083in}{2.456239in}}%
\pgfpathlineto{\pgfqpoint{2.952980in}{2.392213in}}%
\pgfpathlineto{\pgfqpoint{2.952980in}{2.520265in}}%
\pgfpathlineto{\pgfqpoint{2.842083in}{2.584292in}}%
\pgfpathlineto{\pgfqpoint{2.842083in}{2.456239in}}%
\pgfpathclose%
\pgfusepath{fill}%
\end{pgfscope}%
\begin{pgfscope}%
\pgfpathrectangle{\pgfqpoint{0.765000in}{0.660000in}}{\pgfqpoint{4.620000in}{4.620000in}}%
\pgfusepath{clip}%
\pgfsetbuttcap%
\pgfsetroundjoin%
\definecolor{currentfill}{rgb}{1.000000,0.894118,0.788235}%
\pgfsetfillcolor{currentfill}%
\pgfsetlinewidth{0.000000pt}%
\definecolor{currentstroke}{rgb}{1.000000,0.894118,0.788235}%
\pgfsetstrokecolor{currentstroke}%
\pgfsetdash{}{0pt}%
\pgfpathmoveto{\pgfqpoint{2.929883in}{2.658594in}}%
\pgfpathlineto{\pgfqpoint{2.818986in}{2.594568in}}%
\pgfpathlineto{\pgfqpoint{2.929883in}{2.530541in}}%
\pgfpathlineto{\pgfqpoint{3.040780in}{2.594568in}}%
\pgfpathlineto{\pgfqpoint{2.929883in}{2.658594in}}%
\pgfpathclose%
\pgfusepath{fill}%
\end{pgfscope}%
\begin{pgfscope}%
\pgfpathrectangle{\pgfqpoint{0.765000in}{0.660000in}}{\pgfqpoint{4.620000in}{4.620000in}}%
\pgfusepath{clip}%
\pgfsetbuttcap%
\pgfsetroundjoin%
\definecolor{currentfill}{rgb}{1.000000,0.894118,0.788235}%
\pgfsetfillcolor{currentfill}%
\pgfsetlinewidth{0.000000pt}%
\definecolor{currentstroke}{rgb}{1.000000,0.894118,0.788235}%
\pgfsetstrokecolor{currentstroke}%
\pgfsetdash{}{0pt}%
\pgfpathmoveto{\pgfqpoint{2.929883in}{2.402489in}}%
\pgfpathlineto{\pgfqpoint{3.040780in}{2.466515in}}%
\pgfpathlineto{\pgfqpoint{3.040780in}{2.594568in}}%
\pgfpathlineto{\pgfqpoint{2.929883in}{2.530541in}}%
\pgfpathlineto{\pgfqpoint{2.929883in}{2.402489in}}%
\pgfpathclose%
\pgfusepath{fill}%
\end{pgfscope}%
\begin{pgfscope}%
\pgfpathrectangle{\pgfqpoint{0.765000in}{0.660000in}}{\pgfqpoint{4.620000in}{4.620000in}}%
\pgfusepath{clip}%
\pgfsetbuttcap%
\pgfsetroundjoin%
\definecolor{currentfill}{rgb}{1.000000,0.894118,0.788235}%
\pgfsetfillcolor{currentfill}%
\pgfsetlinewidth{0.000000pt}%
\definecolor{currentstroke}{rgb}{1.000000,0.894118,0.788235}%
\pgfsetstrokecolor{currentstroke}%
\pgfsetdash{}{0pt}%
\pgfpathmoveto{\pgfqpoint{2.818986in}{2.466515in}}%
\pgfpathlineto{\pgfqpoint{2.929883in}{2.402489in}}%
\pgfpathlineto{\pgfqpoint{2.929883in}{2.530541in}}%
\pgfpathlineto{\pgfqpoint{2.818986in}{2.594568in}}%
\pgfpathlineto{\pgfqpoint{2.818986in}{2.466515in}}%
\pgfpathclose%
\pgfusepath{fill}%
\end{pgfscope}%
\begin{pgfscope}%
\pgfpathrectangle{\pgfqpoint{0.765000in}{0.660000in}}{\pgfqpoint{4.620000in}{4.620000in}}%
\pgfusepath{clip}%
\pgfsetbuttcap%
\pgfsetroundjoin%
\definecolor{currentfill}{rgb}{1.000000,0.894118,0.788235}%
\pgfsetfillcolor{currentfill}%
\pgfsetlinewidth{0.000000pt}%
\definecolor{currentstroke}{rgb}{1.000000,0.894118,0.788235}%
\pgfsetstrokecolor{currentstroke}%
\pgfsetdash{}{0pt}%
\pgfpathmoveto{\pgfqpoint{2.952980in}{2.520265in}}%
\pgfpathlineto{\pgfqpoint{2.842083in}{2.456239in}}%
\pgfpathlineto{\pgfqpoint{2.818986in}{2.466515in}}%
\pgfpathlineto{\pgfqpoint{2.929883in}{2.530541in}}%
\pgfpathlineto{\pgfqpoint{2.952980in}{2.520265in}}%
\pgfpathclose%
\pgfusepath{fill}%
\end{pgfscope}%
\begin{pgfscope}%
\pgfpathrectangle{\pgfqpoint{0.765000in}{0.660000in}}{\pgfqpoint{4.620000in}{4.620000in}}%
\pgfusepath{clip}%
\pgfsetbuttcap%
\pgfsetroundjoin%
\definecolor{currentfill}{rgb}{1.000000,0.894118,0.788235}%
\pgfsetfillcolor{currentfill}%
\pgfsetlinewidth{0.000000pt}%
\definecolor{currentstroke}{rgb}{1.000000,0.894118,0.788235}%
\pgfsetstrokecolor{currentstroke}%
\pgfsetdash{}{0pt}%
\pgfpathmoveto{\pgfqpoint{3.063876in}{2.456239in}}%
\pgfpathlineto{\pgfqpoint{2.952980in}{2.520265in}}%
\pgfpathlineto{\pgfqpoint{2.929883in}{2.530541in}}%
\pgfpathlineto{\pgfqpoint{3.040780in}{2.466515in}}%
\pgfpathlineto{\pgfqpoint{3.063876in}{2.456239in}}%
\pgfpathclose%
\pgfusepath{fill}%
\end{pgfscope}%
\begin{pgfscope}%
\pgfpathrectangle{\pgfqpoint{0.765000in}{0.660000in}}{\pgfqpoint{4.620000in}{4.620000in}}%
\pgfusepath{clip}%
\pgfsetbuttcap%
\pgfsetroundjoin%
\definecolor{currentfill}{rgb}{1.000000,0.894118,0.788235}%
\pgfsetfillcolor{currentfill}%
\pgfsetlinewidth{0.000000pt}%
\definecolor{currentstroke}{rgb}{1.000000,0.894118,0.788235}%
\pgfsetstrokecolor{currentstroke}%
\pgfsetdash{}{0pt}%
\pgfpathmoveto{\pgfqpoint{2.952980in}{2.520265in}}%
\pgfpathlineto{\pgfqpoint{2.952980in}{2.648318in}}%
\pgfpathlineto{\pgfqpoint{2.929883in}{2.658594in}}%
\pgfpathlineto{\pgfqpoint{3.040780in}{2.466515in}}%
\pgfpathlineto{\pgfqpoint{2.952980in}{2.520265in}}%
\pgfpathclose%
\pgfusepath{fill}%
\end{pgfscope}%
\begin{pgfscope}%
\pgfpathrectangle{\pgfqpoint{0.765000in}{0.660000in}}{\pgfqpoint{4.620000in}{4.620000in}}%
\pgfusepath{clip}%
\pgfsetbuttcap%
\pgfsetroundjoin%
\definecolor{currentfill}{rgb}{1.000000,0.894118,0.788235}%
\pgfsetfillcolor{currentfill}%
\pgfsetlinewidth{0.000000pt}%
\definecolor{currentstroke}{rgb}{1.000000,0.894118,0.788235}%
\pgfsetstrokecolor{currentstroke}%
\pgfsetdash{}{0pt}%
\pgfpathmoveto{\pgfqpoint{3.063876in}{2.456239in}}%
\pgfpathlineto{\pgfqpoint{3.063876in}{2.584292in}}%
\pgfpathlineto{\pgfqpoint{3.040780in}{2.594568in}}%
\pgfpathlineto{\pgfqpoint{2.929883in}{2.530541in}}%
\pgfpathlineto{\pgfqpoint{3.063876in}{2.456239in}}%
\pgfpathclose%
\pgfusepath{fill}%
\end{pgfscope}%
\begin{pgfscope}%
\pgfpathrectangle{\pgfqpoint{0.765000in}{0.660000in}}{\pgfqpoint{4.620000in}{4.620000in}}%
\pgfusepath{clip}%
\pgfsetbuttcap%
\pgfsetroundjoin%
\definecolor{currentfill}{rgb}{1.000000,0.894118,0.788235}%
\pgfsetfillcolor{currentfill}%
\pgfsetlinewidth{0.000000pt}%
\definecolor{currentstroke}{rgb}{1.000000,0.894118,0.788235}%
\pgfsetstrokecolor{currentstroke}%
\pgfsetdash{}{0pt}%
\pgfpathmoveto{\pgfqpoint{2.842083in}{2.456239in}}%
\pgfpathlineto{\pgfqpoint{2.952980in}{2.392213in}}%
\pgfpathlineto{\pgfqpoint{2.929883in}{2.402489in}}%
\pgfpathlineto{\pgfqpoint{2.818986in}{2.466515in}}%
\pgfpathlineto{\pgfqpoint{2.842083in}{2.456239in}}%
\pgfpathclose%
\pgfusepath{fill}%
\end{pgfscope}%
\begin{pgfscope}%
\pgfpathrectangle{\pgfqpoint{0.765000in}{0.660000in}}{\pgfqpoint{4.620000in}{4.620000in}}%
\pgfusepath{clip}%
\pgfsetbuttcap%
\pgfsetroundjoin%
\definecolor{currentfill}{rgb}{1.000000,0.894118,0.788235}%
\pgfsetfillcolor{currentfill}%
\pgfsetlinewidth{0.000000pt}%
\definecolor{currentstroke}{rgb}{1.000000,0.894118,0.788235}%
\pgfsetstrokecolor{currentstroke}%
\pgfsetdash{}{0pt}%
\pgfpathmoveto{\pgfqpoint{2.952980in}{2.392213in}}%
\pgfpathlineto{\pgfqpoint{3.063876in}{2.456239in}}%
\pgfpathlineto{\pgfqpoint{3.040780in}{2.466515in}}%
\pgfpathlineto{\pgfqpoint{2.929883in}{2.402489in}}%
\pgfpathlineto{\pgfqpoint{2.952980in}{2.392213in}}%
\pgfpathclose%
\pgfusepath{fill}%
\end{pgfscope}%
\begin{pgfscope}%
\pgfpathrectangle{\pgfqpoint{0.765000in}{0.660000in}}{\pgfqpoint{4.620000in}{4.620000in}}%
\pgfusepath{clip}%
\pgfsetbuttcap%
\pgfsetroundjoin%
\definecolor{currentfill}{rgb}{1.000000,0.894118,0.788235}%
\pgfsetfillcolor{currentfill}%
\pgfsetlinewidth{0.000000pt}%
\definecolor{currentstroke}{rgb}{1.000000,0.894118,0.788235}%
\pgfsetstrokecolor{currentstroke}%
\pgfsetdash{}{0pt}%
\pgfpathmoveto{\pgfqpoint{2.952980in}{2.648318in}}%
\pgfpathlineto{\pgfqpoint{2.842083in}{2.584292in}}%
\pgfpathlineto{\pgfqpoint{2.818986in}{2.594568in}}%
\pgfpathlineto{\pgfqpoint{2.929883in}{2.658594in}}%
\pgfpathlineto{\pgfqpoint{2.952980in}{2.648318in}}%
\pgfpathclose%
\pgfusepath{fill}%
\end{pgfscope}%
\begin{pgfscope}%
\pgfpathrectangle{\pgfqpoint{0.765000in}{0.660000in}}{\pgfqpoint{4.620000in}{4.620000in}}%
\pgfusepath{clip}%
\pgfsetbuttcap%
\pgfsetroundjoin%
\definecolor{currentfill}{rgb}{1.000000,0.894118,0.788235}%
\pgfsetfillcolor{currentfill}%
\pgfsetlinewidth{0.000000pt}%
\definecolor{currentstroke}{rgb}{1.000000,0.894118,0.788235}%
\pgfsetstrokecolor{currentstroke}%
\pgfsetdash{}{0pt}%
\pgfpathmoveto{\pgfqpoint{3.063876in}{2.584292in}}%
\pgfpathlineto{\pgfqpoint{2.952980in}{2.648318in}}%
\pgfpathlineto{\pgfqpoint{2.929883in}{2.658594in}}%
\pgfpathlineto{\pgfqpoint{3.040780in}{2.594568in}}%
\pgfpathlineto{\pgfqpoint{3.063876in}{2.584292in}}%
\pgfpathclose%
\pgfusepath{fill}%
\end{pgfscope}%
\begin{pgfscope}%
\pgfpathrectangle{\pgfqpoint{0.765000in}{0.660000in}}{\pgfqpoint{4.620000in}{4.620000in}}%
\pgfusepath{clip}%
\pgfsetbuttcap%
\pgfsetroundjoin%
\definecolor{currentfill}{rgb}{1.000000,0.894118,0.788235}%
\pgfsetfillcolor{currentfill}%
\pgfsetlinewidth{0.000000pt}%
\definecolor{currentstroke}{rgb}{1.000000,0.894118,0.788235}%
\pgfsetstrokecolor{currentstroke}%
\pgfsetdash{}{0pt}%
\pgfpathmoveto{\pgfqpoint{2.842083in}{2.456239in}}%
\pgfpathlineto{\pgfqpoint{2.842083in}{2.584292in}}%
\pgfpathlineto{\pgfqpoint{2.818986in}{2.594568in}}%
\pgfpathlineto{\pgfqpoint{2.929883in}{2.402489in}}%
\pgfpathlineto{\pgfqpoint{2.842083in}{2.456239in}}%
\pgfpathclose%
\pgfusepath{fill}%
\end{pgfscope}%
\begin{pgfscope}%
\pgfpathrectangle{\pgfqpoint{0.765000in}{0.660000in}}{\pgfqpoint{4.620000in}{4.620000in}}%
\pgfusepath{clip}%
\pgfsetbuttcap%
\pgfsetroundjoin%
\definecolor{currentfill}{rgb}{1.000000,0.894118,0.788235}%
\pgfsetfillcolor{currentfill}%
\pgfsetlinewidth{0.000000pt}%
\definecolor{currentstroke}{rgb}{1.000000,0.894118,0.788235}%
\pgfsetstrokecolor{currentstroke}%
\pgfsetdash{}{0pt}%
\pgfpathmoveto{\pgfqpoint{2.952980in}{2.392213in}}%
\pgfpathlineto{\pgfqpoint{2.952980in}{2.520265in}}%
\pgfpathlineto{\pgfqpoint{2.929883in}{2.530541in}}%
\pgfpathlineto{\pgfqpoint{2.818986in}{2.466515in}}%
\pgfpathlineto{\pgfqpoint{2.952980in}{2.392213in}}%
\pgfpathclose%
\pgfusepath{fill}%
\end{pgfscope}%
\begin{pgfscope}%
\pgfpathrectangle{\pgfqpoint{0.765000in}{0.660000in}}{\pgfqpoint{4.620000in}{4.620000in}}%
\pgfusepath{clip}%
\pgfsetbuttcap%
\pgfsetroundjoin%
\definecolor{currentfill}{rgb}{1.000000,0.894118,0.788235}%
\pgfsetfillcolor{currentfill}%
\pgfsetlinewidth{0.000000pt}%
\definecolor{currentstroke}{rgb}{1.000000,0.894118,0.788235}%
\pgfsetstrokecolor{currentstroke}%
\pgfsetdash{}{0pt}%
\pgfpathmoveto{\pgfqpoint{2.952980in}{2.520265in}}%
\pgfpathlineto{\pgfqpoint{3.063876in}{2.584292in}}%
\pgfpathlineto{\pgfqpoint{3.040780in}{2.594568in}}%
\pgfpathlineto{\pgfqpoint{2.929883in}{2.530541in}}%
\pgfpathlineto{\pgfqpoint{2.952980in}{2.520265in}}%
\pgfpathclose%
\pgfusepath{fill}%
\end{pgfscope}%
\begin{pgfscope}%
\pgfpathrectangle{\pgfqpoint{0.765000in}{0.660000in}}{\pgfqpoint{4.620000in}{4.620000in}}%
\pgfusepath{clip}%
\pgfsetbuttcap%
\pgfsetroundjoin%
\definecolor{currentfill}{rgb}{1.000000,0.894118,0.788235}%
\pgfsetfillcolor{currentfill}%
\pgfsetlinewidth{0.000000pt}%
\definecolor{currentstroke}{rgb}{1.000000,0.894118,0.788235}%
\pgfsetstrokecolor{currentstroke}%
\pgfsetdash{}{0pt}%
\pgfpathmoveto{\pgfqpoint{2.842083in}{2.584292in}}%
\pgfpathlineto{\pgfqpoint{2.952980in}{2.520265in}}%
\pgfpathlineto{\pgfqpoint{2.929883in}{2.530541in}}%
\pgfpathlineto{\pgfqpoint{2.818986in}{2.594568in}}%
\pgfpathlineto{\pgfqpoint{2.842083in}{2.584292in}}%
\pgfpathclose%
\pgfusepath{fill}%
\end{pgfscope}%
\begin{pgfscope}%
\pgfpathrectangle{\pgfqpoint{0.765000in}{0.660000in}}{\pgfqpoint{4.620000in}{4.620000in}}%
\pgfusepath{clip}%
\pgfsetbuttcap%
\pgfsetroundjoin%
\definecolor{currentfill}{rgb}{1.000000,0.894118,0.788235}%
\pgfsetfillcolor{currentfill}%
\pgfsetlinewidth{0.000000pt}%
\definecolor{currentstroke}{rgb}{1.000000,0.894118,0.788235}%
\pgfsetstrokecolor{currentstroke}%
\pgfsetdash{}{0pt}%
\pgfpathmoveto{\pgfqpoint{3.519640in}{2.712796in}}%
\pgfpathlineto{\pgfqpoint{3.408744in}{2.648770in}}%
\pgfpathlineto{\pgfqpoint{3.519640in}{2.584743in}}%
\pgfpathlineto{\pgfqpoint{3.630537in}{2.648770in}}%
\pgfpathlineto{\pgfqpoint{3.519640in}{2.712796in}}%
\pgfpathclose%
\pgfusepath{fill}%
\end{pgfscope}%
\begin{pgfscope}%
\pgfpathrectangle{\pgfqpoint{0.765000in}{0.660000in}}{\pgfqpoint{4.620000in}{4.620000in}}%
\pgfusepath{clip}%
\pgfsetbuttcap%
\pgfsetroundjoin%
\definecolor{currentfill}{rgb}{1.000000,0.894118,0.788235}%
\pgfsetfillcolor{currentfill}%
\pgfsetlinewidth{0.000000pt}%
\definecolor{currentstroke}{rgb}{1.000000,0.894118,0.788235}%
\pgfsetstrokecolor{currentstroke}%
\pgfsetdash{}{0pt}%
\pgfpathmoveto{\pgfqpoint{3.519640in}{2.712796in}}%
\pgfpathlineto{\pgfqpoint{3.408744in}{2.648770in}}%
\pgfpathlineto{\pgfqpoint{3.408744in}{2.776822in}}%
\pgfpathlineto{\pgfqpoint{3.519640in}{2.840848in}}%
\pgfpathlineto{\pgfqpoint{3.519640in}{2.712796in}}%
\pgfpathclose%
\pgfusepath{fill}%
\end{pgfscope}%
\begin{pgfscope}%
\pgfpathrectangle{\pgfqpoint{0.765000in}{0.660000in}}{\pgfqpoint{4.620000in}{4.620000in}}%
\pgfusepath{clip}%
\pgfsetbuttcap%
\pgfsetroundjoin%
\definecolor{currentfill}{rgb}{1.000000,0.894118,0.788235}%
\pgfsetfillcolor{currentfill}%
\pgfsetlinewidth{0.000000pt}%
\definecolor{currentstroke}{rgb}{1.000000,0.894118,0.788235}%
\pgfsetstrokecolor{currentstroke}%
\pgfsetdash{}{0pt}%
\pgfpathmoveto{\pgfqpoint{3.519640in}{2.712796in}}%
\pgfpathlineto{\pgfqpoint{3.630537in}{2.648770in}}%
\pgfpathlineto{\pgfqpoint{3.630537in}{2.776822in}}%
\pgfpathlineto{\pgfqpoint{3.519640in}{2.840848in}}%
\pgfpathlineto{\pgfqpoint{3.519640in}{2.712796in}}%
\pgfpathclose%
\pgfusepath{fill}%
\end{pgfscope}%
\begin{pgfscope}%
\pgfpathrectangle{\pgfqpoint{0.765000in}{0.660000in}}{\pgfqpoint{4.620000in}{4.620000in}}%
\pgfusepath{clip}%
\pgfsetbuttcap%
\pgfsetroundjoin%
\definecolor{currentfill}{rgb}{1.000000,0.894118,0.788235}%
\pgfsetfillcolor{currentfill}%
\pgfsetlinewidth{0.000000pt}%
\definecolor{currentstroke}{rgb}{1.000000,0.894118,0.788235}%
\pgfsetstrokecolor{currentstroke}%
\pgfsetdash{}{0pt}%
\pgfpathmoveto{\pgfqpoint{3.546263in}{2.765531in}}%
\pgfpathlineto{\pgfqpoint{3.435366in}{2.701504in}}%
\pgfpathlineto{\pgfqpoint{3.546263in}{2.637478in}}%
\pgfpathlineto{\pgfqpoint{3.657159in}{2.701504in}}%
\pgfpathlineto{\pgfqpoint{3.546263in}{2.765531in}}%
\pgfpathclose%
\pgfusepath{fill}%
\end{pgfscope}%
\begin{pgfscope}%
\pgfpathrectangle{\pgfqpoint{0.765000in}{0.660000in}}{\pgfqpoint{4.620000in}{4.620000in}}%
\pgfusepath{clip}%
\pgfsetbuttcap%
\pgfsetroundjoin%
\definecolor{currentfill}{rgb}{1.000000,0.894118,0.788235}%
\pgfsetfillcolor{currentfill}%
\pgfsetlinewidth{0.000000pt}%
\definecolor{currentstroke}{rgb}{1.000000,0.894118,0.788235}%
\pgfsetstrokecolor{currentstroke}%
\pgfsetdash{}{0pt}%
\pgfpathmoveto{\pgfqpoint{3.546263in}{2.765531in}}%
\pgfpathlineto{\pgfqpoint{3.435366in}{2.701504in}}%
\pgfpathlineto{\pgfqpoint{3.435366in}{2.829557in}}%
\pgfpathlineto{\pgfqpoint{3.546263in}{2.893583in}}%
\pgfpathlineto{\pgfqpoint{3.546263in}{2.765531in}}%
\pgfpathclose%
\pgfusepath{fill}%
\end{pgfscope}%
\begin{pgfscope}%
\pgfpathrectangle{\pgfqpoint{0.765000in}{0.660000in}}{\pgfqpoint{4.620000in}{4.620000in}}%
\pgfusepath{clip}%
\pgfsetbuttcap%
\pgfsetroundjoin%
\definecolor{currentfill}{rgb}{1.000000,0.894118,0.788235}%
\pgfsetfillcolor{currentfill}%
\pgfsetlinewidth{0.000000pt}%
\definecolor{currentstroke}{rgb}{1.000000,0.894118,0.788235}%
\pgfsetstrokecolor{currentstroke}%
\pgfsetdash{}{0pt}%
\pgfpathmoveto{\pgfqpoint{3.546263in}{2.765531in}}%
\pgfpathlineto{\pgfqpoint{3.657159in}{2.701504in}}%
\pgfpathlineto{\pgfqpoint{3.657159in}{2.829557in}}%
\pgfpathlineto{\pgfqpoint{3.546263in}{2.893583in}}%
\pgfpathlineto{\pgfqpoint{3.546263in}{2.765531in}}%
\pgfpathclose%
\pgfusepath{fill}%
\end{pgfscope}%
\begin{pgfscope}%
\pgfpathrectangle{\pgfqpoint{0.765000in}{0.660000in}}{\pgfqpoint{4.620000in}{4.620000in}}%
\pgfusepath{clip}%
\pgfsetbuttcap%
\pgfsetroundjoin%
\definecolor{currentfill}{rgb}{1.000000,0.894118,0.788235}%
\pgfsetfillcolor{currentfill}%
\pgfsetlinewidth{0.000000pt}%
\definecolor{currentstroke}{rgb}{1.000000,0.894118,0.788235}%
\pgfsetstrokecolor{currentstroke}%
\pgfsetdash{}{0pt}%
\pgfpathmoveto{\pgfqpoint{3.519640in}{2.840848in}}%
\pgfpathlineto{\pgfqpoint{3.408744in}{2.776822in}}%
\pgfpathlineto{\pgfqpoint{3.519640in}{2.712796in}}%
\pgfpathlineto{\pgfqpoint{3.630537in}{2.776822in}}%
\pgfpathlineto{\pgfqpoint{3.519640in}{2.840848in}}%
\pgfpathclose%
\pgfusepath{fill}%
\end{pgfscope}%
\begin{pgfscope}%
\pgfpathrectangle{\pgfqpoint{0.765000in}{0.660000in}}{\pgfqpoint{4.620000in}{4.620000in}}%
\pgfusepath{clip}%
\pgfsetbuttcap%
\pgfsetroundjoin%
\definecolor{currentfill}{rgb}{1.000000,0.894118,0.788235}%
\pgfsetfillcolor{currentfill}%
\pgfsetlinewidth{0.000000pt}%
\definecolor{currentstroke}{rgb}{1.000000,0.894118,0.788235}%
\pgfsetstrokecolor{currentstroke}%
\pgfsetdash{}{0pt}%
\pgfpathmoveto{\pgfqpoint{3.519640in}{2.584743in}}%
\pgfpathlineto{\pgfqpoint{3.630537in}{2.648770in}}%
\pgfpathlineto{\pgfqpoint{3.630537in}{2.776822in}}%
\pgfpathlineto{\pgfqpoint{3.519640in}{2.712796in}}%
\pgfpathlineto{\pgfqpoint{3.519640in}{2.584743in}}%
\pgfpathclose%
\pgfusepath{fill}%
\end{pgfscope}%
\begin{pgfscope}%
\pgfpathrectangle{\pgfqpoint{0.765000in}{0.660000in}}{\pgfqpoint{4.620000in}{4.620000in}}%
\pgfusepath{clip}%
\pgfsetbuttcap%
\pgfsetroundjoin%
\definecolor{currentfill}{rgb}{1.000000,0.894118,0.788235}%
\pgfsetfillcolor{currentfill}%
\pgfsetlinewidth{0.000000pt}%
\definecolor{currentstroke}{rgb}{1.000000,0.894118,0.788235}%
\pgfsetstrokecolor{currentstroke}%
\pgfsetdash{}{0pt}%
\pgfpathmoveto{\pgfqpoint{3.408744in}{2.648770in}}%
\pgfpathlineto{\pgfqpoint{3.519640in}{2.584743in}}%
\pgfpathlineto{\pgfqpoint{3.519640in}{2.712796in}}%
\pgfpathlineto{\pgfqpoint{3.408744in}{2.776822in}}%
\pgfpathlineto{\pgfqpoint{3.408744in}{2.648770in}}%
\pgfpathclose%
\pgfusepath{fill}%
\end{pgfscope}%
\begin{pgfscope}%
\pgfpathrectangle{\pgfqpoint{0.765000in}{0.660000in}}{\pgfqpoint{4.620000in}{4.620000in}}%
\pgfusepath{clip}%
\pgfsetbuttcap%
\pgfsetroundjoin%
\definecolor{currentfill}{rgb}{1.000000,0.894118,0.788235}%
\pgfsetfillcolor{currentfill}%
\pgfsetlinewidth{0.000000pt}%
\definecolor{currentstroke}{rgb}{1.000000,0.894118,0.788235}%
\pgfsetstrokecolor{currentstroke}%
\pgfsetdash{}{0pt}%
\pgfpathmoveto{\pgfqpoint{3.546263in}{2.893583in}}%
\pgfpathlineto{\pgfqpoint{3.435366in}{2.829557in}}%
\pgfpathlineto{\pgfqpoint{3.546263in}{2.765531in}}%
\pgfpathlineto{\pgfqpoint{3.657159in}{2.829557in}}%
\pgfpathlineto{\pgfqpoint{3.546263in}{2.893583in}}%
\pgfpathclose%
\pgfusepath{fill}%
\end{pgfscope}%
\begin{pgfscope}%
\pgfpathrectangle{\pgfqpoint{0.765000in}{0.660000in}}{\pgfqpoint{4.620000in}{4.620000in}}%
\pgfusepath{clip}%
\pgfsetbuttcap%
\pgfsetroundjoin%
\definecolor{currentfill}{rgb}{1.000000,0.894118,0.788235}%
\pgfsetfillcolor{currentfill}%
\pgfsetlinewidth{0.000000pt}%
\definecolor{currentstroke}{rgb}{1.000000,0.894118,0.788235}%
\pgfsetstrokecolor{currentstroke}%
\pgfsetdash{}{0pt}%
\pgfpathmoveto{\pgfqpoint{3.546263in}{2.637478in}}%
\pgfpathlineto{\pgfqpoint{3.657159in}{2.701504in}}%
\pgfpathlineto{\pgfqpoint{3.657159in}{2.829557in}}%
\pgfpathlineto{\pgfqpoint{3.546263in}{2.765531in}}%
\pgfpathlineto{\pgfqpoint{3.546263in}{2.637478in}}%
\pgfpathclose%
\pgfusepath{fill}%
\end{pgfscope}%
\begin{pgfscope}%
\pgfpathrectangle{\pgfqpoint{0.765000in}{0.660000in}}{\pgfqpoint{4.620000in}{4.620000in}}%
\pgfusepath{clip}%
\pgfsetbuttcap%
\pgfsetroundjoin%
\definecolor{currentfill}{rgb}{1.000000,0.894118,0.788235}%
\pgfsetfillcolor{currentfill}%
\pgfsetlinewidth{0.000000pt}%
\definecolor{currentstroke}{rgb}{1.000000,0.894118,0.788235}%
\pgfsetstrokecolor{currentstroke}%
\pgfsetdash{}{0pt}%
\pgfpathmoveto{\pgfqpoint{3.435366in}{2.701504in}}%
\pgfpathlineto{\pgfqpoint{3.546263in}{2.637478in}}%
\pgfpathlineto{\pgfqpoint{3.546263in}{2.765531in}}%
\pgfpathlineto{\pgfqpoint{3.435366in}{2.829557in}}%
\pgfpathlineto{\pgfqpoint{3.435366in}{2.701504in}}%
\pgfpathclose%
\pgfusepath{fill}%
\end{pgfscope}%
\begin{pgfscope}%
\pgfpathrectangle{\pgfqpoint{0.765000in}{0.660000in}}{\pgfqpoint{4.620000in}{4.620000in}}%
\pgfusepath{clip}%
\pgfsetbuttcap%
\pgfsetroundjoin%
\definecolor{currentfill}{rgb}{1.000000,0.894118,0.788235}%
\pgfsetfillcolor{currentfill}%
\pgfsetlinewidth{0.000000pt}%
\definecolor{currentstroke}{rgb}{1.000000,0.894118,0.788235}%
\pgfsetstrokecolor{currentstroke}%
\pgfsetdash{}{0pt}%
\pgfpathmoveto{\pgfqpoint{3.519640in}{2.712796in}}%
\pgfpathlineto{\pgfqpoint{3.408744in}{2.648770in}}%
\pgfpathlineto{\pgfqpoint{3.435366in}{2.701504in}}%
\pgfpathlineto{\pgfqpoint{3.546263in}{2.765531in}}%
\pgfpathlineto{\pgfqpoint{3.519640in}{2.712796in}}%
\pgfpathclose%
\pgfusepath{fill}%
\end{pgfscope}%
\begin{pgfscope}%
\pgfpathrectangle{\pgfqpoint{0.765000in}{0.660000in}}{\pgfqpoint{4.620000in}{4.620000in}}%
\pgfusepath{clip}%
\pgfsetbuttcap%
\pgfsetroundjoin%
\definecolor{currentfill}{rgb}{1.000000,0.894118,0.788235}%
\pgfsetfillcolor{currentfill}%
\pgfsetlinewidth{0.000000pt}%
\definecolor{currentstroke}{rgb}{1.000000,0.894118,0.788235}%
\pgfsetstrokecolor{currentstroke}%
\pgfsetdash{}{0pt}%
\pgfpathmoveto{\pgfqpoint{3.630537in}{2.648770in}}%
\pgfpathlineto{\pgfqpoint{3.519640in}{2.712796in}}%
\pgfpathlineto{\pgfqpoint{3.546263in}{2.765531in}}%
\pgfpathlineto{\pgfqpoint{3.657159in}{2.701504in}}%
\pgfpathlineto{\pgfqpoint{3.630537in}{2.648770in}}%
\pgfpathclose%
\pgfusepath{fill}%
\end{pgfscope}%
\begin{pgfscope}%
\pgfpathrectangle{\pgfqpoint{0.765000in}{0.660000in}}{\pgfqpoint{4.620000in}{4.620000in}}%
\pgfusepath{clip}%
\pgfsetbuttcap%
\pgfsetroundjoin%
\definecolor{currentfill}{rgb}{1.000000,0.894118,0.788235}%
\pgfsetfillcolor{currentfill}%
\pgfsetlinewidth{0.000000pt}%
\definecolor{currentstroke}{rgb}{1.000000,0.894118,0.788235}%
\pgfsetstrokecolor{currentstroke}%
\pgfsetdash{}{0pt}%
\pgfpathmoveto{\pgfqpoint{3.519640in}{2.712796in}}%
\pgfpathlineto{\pgfqpoint{3.519640in}{2.840848in}}%
\pgfpathlineto{\pgfqpoint{3.546263in}{2.893583in}}%
\pgfpathlineto{\pgfqpoint{3.657159in}{2.701504in}}%
\pgfpathlineto{\pgfqpoint{3.519640in}{2.712796in}}%
\pgfpathclose%
\pgfusepath{fill}%
\end{pgfscope}%
\begin{pgfscope}%
\pgfpathrectangle{\pgfqpoint{0.765000in}{0.660000in}}{\pgfqpoint{4.620000in}{4.620000in}}%
\pgfusepath{clip}%
\pgfsetbuttcap%
\pgfsetroundjoin%
\definecolor{currentfill}{rgb}{1.000000,0.894118,0.788235}%
\pgfsetfillcolor{currentfill}%
\pgfsetlinewidth{0.000000pt}%
\definecolor{currentstroke}{rgb}{1.000000,0.894118,0.788235}%
\pgfsetstrokecolor{currentstroke}%
\pgfsetdash{}{0pt}%
\pgfpathmoveto{\pgfqpoint{3.630537in}{2.648770in}}%
\pgfpathlineto{\pgfqpoint{3.630537in}{2.776822in}}%
\pgfpathlineto{\pgfqpoint{3.657159in}{2.829557in}}%
\pgfpathlineto{\pgfqpoint{3.546263in}{2.765531in}}%
\pgfpathlineto{\pgfqpoint{3.630537in}{2.648770in}}%
\pgfpathclose%
\pgfusepath{fill}%
\end{pgfscope}%
\begin{pgfscope}%
\pgfpathrectangle{\pgfqpoint{0.765000in}{0.660000in}}{\pgfqpoint{4.620000in}{4.620000in}}%
\pgfusepath{clip}%
\pgfsetbuttcap%
\pgfsetroundjoin%
\definecolor{currentfill}{rgb}{1.000000,0.894118,0.788235}%
\pgfsetfillcolor{currentfill}%
\pgfsetlinewidth{0.000000pt}%
\definecolor{currentstroke}{rgb}{1.000000,0.894118,0.788235}%
\pgfsetstrokecolor{currentstroke}%
\pgfsetdash{}{0pt}%
\pgfpathmoveto{\pgfqpoint{3.408744in}{2.648770in}}%
\pgfpathlineto{\pgfqpoint{3.519640in}{2.584743in}}%
\pgfpathlineto{\pgfqpoint{3.546263in}{2.637478in}}%
\pgfpathlineto{\pgfqpoint{3.435366in}{2.701504in}}%
\pgfpathlineto{\pgfqpoint{3.408744in}{2.648770in}}%
\pgfpathclose%
\pgfusepath{fill}%
\end{pgfscope}%
\begin{pgfscope}%
\pgfpathrectangle{\pgfqpoint{0.765000in}{0.660000in}}{\pgfqpoint{4.620000in}{4.620000in}}%
\pgfusepath{clip}%
\pgfsetbuttcap%
\pgfsetroundjoin%
\definecolor{currentfill}{rgb}{1.000000,0.894118,0.788235}%
\pgfsetfillcolor{currentfill}%
\pgfsetlinewidth{0.000000pt}%
\definecolor{currentstroke}{rgb}{1.000000,0.894118,0.788235}%
\pgfsetstrokecolor{currentstroke}%
\pgfsetdash{}{0pt}%
\pgfpathmoveto{\pgfqpoint{3.519640in}{2.584743in}}%
\pgfpathlineto{\pgfqpoint{3.630537in}{2.648770in}}%
\pgfpathlineto{\pgfqpoint{3.657159in}{2.701504in}}%
\pgfpathlineto{\pgfqpoint{3.546263in}{2.637478in}}%
\pgfpathlineto{\pgfqpoint{3.519640in}{2.584743in}}%
\pgfpathclose%
\pgfusepath{fill}%
\end{pgfscope}%
\begin{pgfscope}%
\pgfpathrectangle{\pgfqpoint{0.765000in}{0.660000in}}{\pgfqpoint{4.620000in}{4.620000in}}%
\pgfusepath{clip}%
\pgfsetbuttcap%
\pgfsetroundjoin%
\definecolor{currentfill}{rgb}{1.000000,0.894118,0.788235}%
\pgfsetfillcolor{currentfill}%
\pgfsetlinewidth{0.000000pt}%
\definecolor{currentstroke}{rgb}{1.000000,0.894118,0.788235}%
\pgfsetstrokecolor{currentstroke}%
\pgfsetdash{}{0pt}%
\pgfpathmoveto{\pgfqpoint{3.519640in}{2.840848in}}%
\pgfpathlineto{\pgfqpoint{3.408744in}{2.776822in}}%
\pgfpathlineto{\pgfqpoint{3.435366in}{2.829557in}}%
\pgfpathlineto{\pgfqpoint{3.546263in}{2.893583in}}%
\pgfpathlineto{\pgfqpoint{3.519640in}{2.840848in}}%
\pgfpathclose%
\pgfusepath{fill}%
\end{pgfscope}%
\begin{pgfscope}%
\pgfpathrectangle{\pgfqpoint{0.765000in}{0.660000in}}{\pgfqpoint{4.620000in}{4.620000in}}%
\pgfusepath{clip}%
\pgfsetbuttcap%
\pgfsetroundjoin%
\definecolor{currentfill}{rgb}{1.000000,0.894118,0.788235}%
\pgfsetfillcolor{currentfill}%
\pgfsetlinewidth{0.000000pt}%
\definecolor{currentstroke}{rgb}{1.000000,0.894118,0.788235}%
\pgfsetstrokecolor{currentstroke}%
\pgfsetdash{}{0pt}%
\pgfpathmoveto{\pgfqpoint{3.630537in}{2.776822in}}%
\pgfpathlineto{\pgfqpoint{3.519640in}{2.840848in}}%
\pgfpathlineto{\pgfqpoint{3.546263in}{2.893583in}}%
\pgfpathlineto{\pgfqpoint{3.657159in}{2.829557in}}%
\pgfpathlineto{\pgfqpoint{3.630537in}{2.776822in}}%
\pgfpathclose%
\pgfusepath{fill}%
\end{pgfscope}%
\begin{pgfscope}%
\pgfpathrectangle{\pgfqpoint{0.765000in}{0.660000in}}{\pgfqpoint{4.620000in}{4.620000in}}%
\pgfusepath{clip}%
\pgfsetbuttcap%
\pgfsetroundjoin%
\definecolor{currentfill}{rgb}{1.000000,0.894118,0.788235}%
\pgfsetfillcolor{currentfill}%
\pgfsetlinewidth{0.000000pt}%
\definecolor{currentstroke}{rgb}{1.000000,0.894118,0.788235}%
\pgfsetstrokecolor{currentstroke}%
\pgfsetdash{}{0pt}%
\pgfpathmoveto{\pgfqpoint{3.408744in}{2.648770in}}%
\pgfpathlineto{\pgfqpoint{3.408744in}{2.776822in}}%
\pgfpathlineto{\pgfqpoint{3.435366in}{2.829557in}}%
\pgfpathlineto{\pgfqpoint{3.546263in}{2.637478in}}%
\pgfpathlineto{\pgfqpoint{3.408744in}{2.648770in}}%
\pgfpathclose%
\pgfusepath{fill}%
\end{pgfscope}%
\begin{pgfscope}%
\pgfpathrectangle{\pgfqpoint{0.765000in}{0.660000in}}{\pgfqpoint{4.620000in}{4.620000in}}%
\pgfusepath{clip}%
\pgfsetbuttcap%
\pgfsetroundjoin%
\definecolor{currentfill}{rgb}{1.000000,0.894118,0.788235}%
\pgfsetfillcolor{currentfill}%
\pgfsetlinewidth{0.000000pt}%
\definecolor{currentstroke}{rgb}{1.000000,0.894118,0.788235}%
\pgfsetstrokecolor{currentstroke}%
\pgfsetdash{}{0pt}%
\pgfpathmoveto{\pgfqpoint{3.519640in}{2.584743in}}%
\pgfpathlineto{\pgfqpoint{3.519640in}{2.712796in}}%
\pgfpathlineto{\pgfqpoint{3.546263in}{2.765531in}}%
\pgfpathlineto{\pgfqpoint{3.435366in}{2.701504in}}%
\pgfpathlineto{\pgfqpoint{3.519640in}{2.584743in}}%
\pgfpathclose%
\pgfusepath{fill}%
\end{pgfscope}%
\begin{pgfscope}%
\pgfpathrectangle{\pgfqpoint{0.765000in}{0.660000in}}{\pgfqpoint{4.620000in}{4.620000in}}%
\pgfusepath{clip}%
\pgfsetbuttcap%
\pgfsetroundjoin%
\definecolor{currentfill}{rgb}{1.000000,0.894118,0.788235}%
\pgfsetfillcolor{currentfill}%
\pgfsetlinewidth{0.000000pt}%
\definecolor{currentstroke}{rgb}{1.000000,0.894118,0.788235}%
\pgfsetstrokecolor{currentstroke}%
\pgfsetdash{}{0pt}%
\pgfpathmoveto{\pgfqpoint{3.519640in}{2.712796in}}%
\pgfpathlineto{\pgfqpoint{3.630537in}{2.776822in}}%
\pgfpathlineto{\pgfqpoint{3.657159in}{2.829557in}}%
\pgfpathlineto{\pgfqpoint{3.546263in}{2.765531in}}%
\pgfpathlineto{\pgfqpoint{3.519640in}{2.712796in}}%
\pgfpathclose%
\pgfusepath{fill}%
\end{pgfscope}%
\begin{pgfscope}%
\pgfpathrectangle{\pgfqpoint{0.765000in}{0.660000in}}{\pgfqpoint{4.620000in}{4.620000in}}%
\pgfusepath{clip}%
\pgfsetbuttcap%
\pgfsetroundjoin%
\definecolor{currentfill}{rgb}{1.000000,0.894118,0.788235}%
\pgfsetfillcolor{currentfill}%
\pgfsetlinewidth{0.000000pt}%
\definecolor{currentstroke}{rgb}{1.000000,0.894118,0.788235}%
\pgfsetstrokecolor{currentstroke}%
\pgfsetdash{}{0pt}%
\pgfpathmoveto{\pgfqpoint{3.408744in}{2.776822in}}%
\pgfpathlineto{\pgfqpoint{3.519640in}{2.712796in}}%
\pgfpathlineto{\pgfqpoint{3.546263in}{2.765531in}}%
\pgfpathlineto{\pgfqpoint{3.435366in}{2.829557in}}%
\pgfpathlineto{\pgfqpoint{3.408744in}{2.776822in}}%
\pgfpathclose%
\pgfusepath{fill}%
\end{pgfscope}%
\begin{pgfscope}%
\pgfpathrectangle{\pgfqpoint{0.765000in}{0.660000in}}{\pgfqpoint{4.620000in}{4.620000in}}%
\pgfusepath{clip}%
\pgfsetbuttcap%
\pgfsetroundjoin%
\definecolor{currentfill}{rgb}{1.000000,0.894118,0.788235}%
\pgfsetfillcolor{currentfill}%
\pgfsetlinewidth{0.000000pt}%
\definecolor{currentstroke}{rgb}{1.000000,0.894118,0.788235}%
\pgfsetstrokecolor{currentstroke}%
\pgfsetdash{}{0pt}%
\pgfpathmoveto{\pgfqpoint{3.519640in}{2.712796in}}%
\pgfpathlineto{\pgfqpoint{3.408744in}{2.648770in}}%
\pgfpathlineto{\pgfqpoint{3.519640in}{2.584743in}}%
\pgfpathlineto{\pgfqpoint{3.630537in}{2.648770in}}%
\pgfpathlineto{\pgfqpoint{3.519640in}{2.712796in}}%
\pgfpathclose%
\pgfusepath{fill}%
\end{pgfscope}%
\begin{pgfscope}%
\pgfpathrectangle{\pgfqpoint{0.765000in}{0.660000in}}{\pgfqpoint{4.620000in}{4.620000in}}%
\pgfusepath{clip}%
\pgfsetbuttcap%
\pgfsetroundjoin%
\definecolor{currentfill}{rgb}{1.000000,0.894118,0.788235}%
\pgfsetfillcolor{currentfill}%
\pgfsetlinewidth{0.000000pt}%
\definecolor{currentstroke}{rgb}{1.000000,0.894118,0.788235}%
\pgfsetstrokecolor{currentstroke}%
\pgfsetdash{}{0pt}%
\pgfpathmoveto{\pgfqpoint{3.519640in}{2.712796in}}%
\pgfpathlineto{\pgfqpoint{3.408744in}{2.648770in}}%
\pgfpathlineto{\pgfqpoint{3.408744in}{2.776822in}}%
\pgfpathlineto{\pgfqpoint{3.519640in}{2.840848in}}%
\pgfpathlineto{\pgfqpoint{3.519640in}{2.712796in}}%
\pgfpathclose%
\pgfusepath{fill}%
\end{pgfscope}%
\begin{pgfscope}%
\pgfpathrectangle{\pgfqpoint{0.765000in}{0.660000in}}{\pgfqpoint{4.620000in}{4.620000in}}%
\pgfusepath{clip}%
\pgfsetbuttcap%
\pgfsetroundjoin%
\definecolor{currentfill}{rgb}{1.000000,0.894118,0.788235}%
\pgfsetfillcolor{currentfill}%
\pgfsetlinewidth{0.000000pt}%
\definecolor{currentstroke}{rgb}{1.000000,0.894118,0.788235}%
\pgfsetstrokecolor{currentstroke}%
\pgfsetdash{}{0pt}%
\pgfpathmoveto{\pgfqpoint{3.519640in}{2.712796in}}%
\pgfpathlineto{\pgfqpoint{3.630537in}{2.648770in}}%
\pgfpathlineto{\pgfqpoint{3.630537in}{2.776822in}}%
\pgfpathlineto{\pgfqpoint{3.519640in}{2.840848in}}%
\pgfpathlineto{\pgfqpoint{3.519640in}{2.712796in}}%
\pgfpathclose%
\pgfusepath{fill}%
\end{pgfscope}%
\begin{pgfscope}%
\pgfpathrectangle{\pgfqpoint{0.765000in}{0.660000in}}{\pgfqpoint{4.620000in}{4.620000in}}%
\pgfusepath{clip}%
\pgfsetbuttcap%
\pgfsetroundjoin%
\definecolor{currentfill}{rgb}{1.000000,0.894118,0.788235}%
\pgfsetfillcolor{currentfill}%
\pgfsetlinewidth{0.000000pt}%
\definecolor{currentstroke}{rgb}{1.000000,0.894118,0.788235}%
\pgfsetstrokecolor{currentstroke}%
\pgfsetdash{}{0pt}%
\pgfpathmoveto{\pgfqpoint{3.477043in}{2.674663in}}%
\pgfpathlineto{\pgfqpoint{3.366147in}{2.610637in}}%
\pgfpathlineto{\pgfqpoint{3.477043in}{2.546611in}}%
\pgfpathlineto{\pgfqpoint{3.587940in}{2.610637in}}%
\pgfpathlineto{\pgfqpoint{3.477043in}{2.674663in}}%
\pgfpathclose%
\pgfusepath{fill}%
\end{pgfscope}%
\begin{pgfscope}%
\pgfpathrectangle{\pgfqpoint{0.765000in}{0.660000in}}{\pgfqpoint{4.620000in}{4.620000in}}%
\pgfusepath{clip}%
\pgfsetbuttcap%
\pgfsetroundjoin%
\definecolor{currentfill}{rgb}{1.000000,0.894118,0.788235}%
\pgfsetfillcolor{currentfill}%
\pgfsetlinewidth{0.000000pt}%
\definecolor{currentstroke}{rgb}{1.000000,0.894118,0.788235}%
\pgfsetstrokecolor{currentstroke}%
\pgfsetdash{}{0pt}%
\pgfpathmoveto{\pgfqpoint{3.477043in}{2.674663in}}%
\pgfpathlineto{\pgfqpoint{3.366147in}{2.610637in}}%
\pgfpathlineto{\pgfqpoint{3.366147in}{2.738689in}}%
\pgfpathlineto{\pgfqpoint{3.477043in}{2.802716in}}%
\pgfpathlineto{\pgfqpoint{3.477043in}{2.674663in}}%
\pgfpathclose%
\pgfusepath{fill}%
\end{pgfscope}%
\begin{pgfscope}%
\pgfpathrectangle{\pgfqpoint{0.765000in}{0.660000in}}{\pgfqpoint{4.620000in}{4.620000in}}%
\pgfusepath{clip}%
\pgfsetbuttcap%
\pgfsetroundjoin%
\definecolor{currentfill}{rgb}{1.000000,0.894118,0.788235}%
\pgfsetfillcolor{currentfill}%
\pgfsetlinewidth{0.000000pt}%
\definecolor{currentstroke}{rgb}{1.000000,0.894118,0.788235}%
\pgfsetstrokecolor{currentstroke}%
\pgfsetdash{}{0pt}%
\pgfpathmoveto{\pgfqpoint{3.477043in}{2.674663in}}%
\pgfpathlineto{\pgfqpoint{3.587940in}{2.610637in}}%
\pgfpathlineto{\pgfqpoint{3.587940in}{2.738689in}}%
\pgfpathlineto{\pgfqpoint{3.477043in}{2.802716in}}%
\pgfpathlineto{\pgfqpoint{3.477043in}{2.674663in}}%
\pgfpathclose%
\pgfusepath{fill}%
\end{pgfscope}%
\begin{pgfscope}%
\pgfpathrectangle{\pgfqpoint{0.765000in}{0.660000in}}{\pgfqpoint{4.620000in}{4.620000in}}%
\pgfusepath{clip}%
\pgfsetbuttcap%
\pgfsetroundjoin%
\definecolor{currentfill}{rgb}{1.000000,0.894118,0.788235}%
\pgfsetfillcolor{currentfill}%
\pgfsetlinewidth{0.000000pt}%
\definecolor{currentstroke}{rgb}{1.000000,0.894118,0.788235}%
\pgfsetstrokecolor{currentstroke}%
\pgfsetdash{}{0pt}%
\pgfpathmoveto{\pgfqpoint{3.519640in}{2.840848in}}%
\pgfpathlineto{\pgfqpoint{3.408744in}{2.776822in}}%
\pgfpathlineto{\pgfqpoint{3.519640in}{2.712796in}}%
\pgfpathlineto{\pgfqpoint{3.630537in}{2.776822in}}%
\pgfpathlineto{\pgfqpoint{3.519640in}{2.840848in}}%
\pgfpathclose%
\pgfusepath{fill}%
\end{pgfscope}%
\begin{pgfscope}%
\pgfpathrectangle{\pgfqpoint{0.765000in}{0.660000in}}{\pgfqpoint{4.620000in}{4.620000in}}%
\pgfusepath{clip}%
\pgfsetbuttcap%
\pgfsetroundjoin%
\definecolor{currentfill}{rgb}{1.000000,0.894118,0.788235}%
\pgfsetfillcolor{currentfill}%
\pgfsetlinewidth{0.000000pt}%
\definecolor{currentstroke}{rgb}{1.000000,0.894118,0.788235}%
\pgfsetstrokecolor{currentstroke}%
\pgfsetdash{}{0pt}%
\pgfpathmoveto{\pgfqpoint{3.519640in}{2.584743in}}%
\pgfpathlineto{\pgfqpoint{3.630537in}{2.648770in}}%
\pgfpathlineto{\pgfqpoint{3.630537in}{2.776822in}}%
\pgfpathlineto{\pgfqpoint{3.519640in}{2.712796in}}%
\pgfpathlineto{\pgfqpoint{3.519640in}{2.584743in}}%
\pgfpathclose%
\pgfusepath{fill}%
\end{pgfscope}%
\begin{pgfscope}%
\pgfpathrectangle{\pgfqpoint{0.765000in}{0.660000in}}{\pgfqpoint{4.620000in}{4.620000in}}%
\pgfusepath{clip}%
\pgfsetbuttcap%
\pgfsetroundjoin%
\definecolor{currentfill}{rgb}{1.000000,0.894118,0.788235}%
\pgfsetfillcolor{currentfill}%
\pgfsetlinewidth{0.000000pt}%
\definecolor{currentstroke}{rgb}{1.000000,0.894118,0.788235}%
\pgfsetstrokecolor{currentstroke}%
\pgfsetdash{}{0pt}%
\pgfpathmoveto{\pgfqpoint{3.408744in}{2.648770in}}%
\pgfpathlineto{\pgfqpoint{3.519640in}{2.584743in}}%
\pgfpathlineto{\pgfqpoint{3.519640in}{2.712796in}}%
\pgfpathlineto{\pgfqpoint{3.408744in}{2.776822in}}%
\pgfpathlineto{\pgfqpoint{3.408744in}{2.648770in}}%
\pgfpathclose%
\pgfusepath{fill}%
\end{pgfscope}%
\begin{pgfscope}%
\pgfpathrectangle{\pgfqpoint{0.765000in}{0.660000in}}{\pgfqpoint{4.620000in}{4.620000in}}%
\pgfusepath{clip}%
\pgfsetbuttcap%
\pgfsetroundjoin%
\definecolor{currentfill}{rgb}{1.000000,0.894118,0.788235}%
\pgfsetfillcolor{currentfill}%
\pgfsetlinewidth{0.000000pt}%
\definecolor{currentstroke}{rgb}{1.000000,0.894118,0.788235}%
\pgfsetstrokecolor{currentstroke}%
\pgfsetdash{}{0pt}%
\pgfpathmoveto{\pgfqpoint{3.477043in}{2.802716in}}%
\pgfpathlineto{\pgfqpoint{3.366147in}{2.738689in}}%
\pgfpathlineto{\pgfqpoint{3.477043in}{2.674663in}}%
\pgfpathlineto{\pgfqpoint{3.587940in}{2.738689in}}%
\pgfpathlineto{\pgfqpoint{3.477043in}{2.802716in}}%
\pgfpathclose%
\pgfusepath{fill}%
\end{pgfscope}%
\begin{pgfscope}%
\pgfpathrectangle{\pgfqpoint{0.765000in}{0.660000in}}{\pgfqpoint{4.620000in}{4.620000in}}%
\pgfusepath{clip}%
\pgfsetbuttcap%
\pgfsetroundjoin%
\definecolor{currentfill}{rgb}{1.000000,0.894118,0.788235}%
\pgfsetfillcolor{currentfill}%
\pgfsetlinewidth{0.000000pt}%
\definecolor{currentstroke}{rgb}{1.000000,0.894118,0.788235}%
\pgfsetstrokecolor{currentstroke}%
\pgfsetdash{}{0pt}%
\pgfpathmoveto{\pgfqpoint{3.477043in}{2.546611in}}%
\pgfpathlineto{\pgfqpoint{3.587940in}{2.610637in}}%
\pgfpathlineto{\pgfqpoint{3.587940in}{2.738689in}}%
\pgfpathlineto{\pgfqpoint{3.477043in}{2.674663in}}%
\pgfpathlineto{\pgfqpoint{3.477043in}{2.546611in}}%
\pgfpathclose%
\pgfusepath{fill}%
\end{pgfscope}%
\begin{pgfscope}%
\pgfpathrectangle{\pgfqpoint{0.765000in}{0.660000in}}{\pgfqpoint{4.620000in}{4.620000in}}%
\pgfusepath{clip}%
\pgfsetbuttcap%
\pgfsetroundjoin%
\definecolor{currentfill}{rgb}{1.000000,0.894118,0.788235}%
\pgfsetfillcolor{currentfill}%
\pgfsetlinewidth{0.000000pt}%
\definecolor{currentstroke}{rgb}{1.000000,0.894118,0.788235}%
\pgfsetstrokecolor{currentstroke}%
\pgfsetdash{}{0pt}%
\pgfpathmoveto{\pgfqpoint{3.366147in}{2.610637in}}%
\pgfpathlineto{\pgfqpoint{3.477043in}{2.546611in}}%
\pgfpathlineto{\pgfqpoint{3.477043in}{2.674663in}}%
\pgfpathlineto{\pgfqpoint{3.366147in}{2.738689in}}%
\pgfpathlineto{\pgfqpoint{3.366147in}{2.610637in}}%
\pgfpathclose%
\pgfusepath{fill}%
\end{pgfscope}%
\begin{pgfscope}%
\pgfpathrectangle{\pgfqpoint{0.765000in}{0.660000in}}{\pgfqpoint{4.620000in}{4.620000in}}%
\pgfusepath{clip}%
\pgfsetbuttcap%
\pgfsetroundjoin%
\definecolor{currentfill}{rgb}{1.000000,0.894118,0.788235}%
\pgfsetfillcolor{currentfill}%
\pgfsetlinewidth{0.000000pt}%
\definecolor{currentstroke}{rgb}{1.000000,0.894118,0.788235}%
\pgfsetstrokecolor{currentstroke}%
\pgfsetdash{}{0pt}%
\pgfpathmoveto{\pgfqpoint{3.519640in}{2.712796in}}%
\pgfpathlineto{\pgfqpoint{3.408744in}{2.648770in}}%
\pgfpathlineto{\pgfqpoint{3.366147in}{2.610637in}}%
\pgfpathlineto{\pgfqpoint{3.477043in}{2.674663in}}%
\pgfpathlineto{\pgfqpoint{3.519640in}{2.712796in}}%
\pgfpathclose%
\pgfusepath{fill}%
\end{pgfscope}%
\begin{pgfscope}%
\pgfpathrectangle{\pgfqpoint{0.765000in}{0.660000in}}{\pgfqpoint{4.620000in}{4.620000in}}%
\pgfusepath{clip}%
\pgfsetbuttcap%
\pgfsetroundjoin%
\definecolor{currentfill}{rgb}{1.000000,0.894118,0.788235}%
\pgfsetfillcolor{currentfill}%
\pgfsetlinewidth{0.000000pt}%
\definecolor{currentstroke}{rgb}{1.000000,0.894118,0.788235}%
\pgfsetstrokecolor{currentstroke}%
\pgfsetdash{}{0pt}%
\pgfpathmoveto{\pgfqpoint{3.630537in}{2.648770in}}%
\pgfpathlineto{\pgfqpoint{3.519640in}{2.712796in}}%
\pgfpathlineto{\pgfqpoint{3.477043in}{2.674663in}}%
\pgfpathlineto{\pgfqpoint{3.587940in}{2.610637in}}%
\pgfpathlineto{\pgfqpoint{3.630537in}{2.648770in}}%
\pgfpathclose%
\pgfusepath{fill}%
\end{pgfscope}%
\begin{pgfscope}%
\pgfpathrectangle{\pgfqpoint{0.765000in}{0.660000in}}{\pgfqpoint{4.620000in}{4.620000in}}%
\pgfusepath{clip}%
\pgfsetbuttcap%
\pgfsetroundjoin%
\definecolor{currentfill}{rgb}{1.000000,0.894118,0.788235}%
\pgfsetfillcolor{currentfill}%
\pgfsetlinewidth{0.000000pt}%
\definecolor{currentstroke}{rgb}{1.000000,0.894118,0.788235}%
\pgfsetstrokecolor{currentstroke}%
\pgfsetdash{}{0pt}%
\pgfpathmoveto{\pgfqpoint{3.519640in}{2.712796in}}%
\pgfpathlineto{\pgfqpoint{3.519640in}{2.840848in}}%
\pgfpathlineto{\pgfqpoint{3.477043in}{2.802716in}}%
\pgfpathlineto{\pgfqpoint{3.587940in}{2.610637in}}%
\pgfpathlineto{\pgfqpoint{3.519640in}{2.712796in}}%
\pgfpathclose%
\pgfusepath{fill}%
\end{pgfscope}%
\begin{pgfscope}%
\pgfpathrectangle{\pgfqpoint{0.765000in}{0.660000in}}{\pgfqpoint{4.620000in}{4.620000in}}%
\pgfusepath{clip}%
\pgfsetbuttcap%
\pgfsetroundjoin%
\definecolor{currentfill}{rgb}{1.000000,0.894118,0.788235}%
\pgfsetfillcolor{currentfill}%
\pgfsetlinewidth{0.000000pt}%
\definecolor{currentstroke}{rgb}{1.000000,0.894118,0.788235}%
\pgfsetstrokecolor{currentstroke}%
\pgfsetdash{}{0pt}%
\pgfpathmoveto{\pgfqpoint{3.630537in}{2.648770in}}%
\pgfpathlineto{\pgfqpoint{3.630537in}{2.776822in}}%
\pgfpathlineto{\pgfqpoint{3.587940in}{2.738689in}}%
\pgfpathlineto{\pgfqpoint{3.477043in}{2.674663in}}%
\pgfpathlineto{\pgfqpoint{3.630537in}{2.648770in}}%
\pgfpathclose%
\pgfusepath{fill}%
\end{pgfscope}%
\begin{pgfscope}%
\pgfpathrectangle{\pgfqpoint{0.765000in}{0.660000in}}{\pgfqpoint{4.620000in}{4.620000in}}%
\pgfusepath{clip}%
\pgfsetbuttcap%
\pgfsetroundjoin%
\definecolor{currentfill}{rgb}{1.000000,0.894118,0.788235}%
\pgfsetfillcolor{currentfill}%
\pgfsetlinewidth{0.000000pt}%
\definecolor{currentstroke}{rgb}{1.000000,0.894118,0.788235}%
\pgfsetstrokecolor{currentstroke}%
\pgfsetdash{}{0pt}%
\pgfpathmoveto{\pgfqpoint{3.408744in}{2.648770in}}%
\pgfpathlineto{\pgfqpoint{3.519640in}{2.584743in}}%
\pgfpathlineto{\pgfqpoint{3.477043in}{2.546611in}}%
\pgfpathlineto{\pgfqpoint{3.366147in}{2.610637in}}%
\pgfpathlineto{\pgfqpoint{3.408744in}{2.648770in}}%
\pgfpathclose%
\pgfusepath{fill}%
\end{pgfscope}%
\begin{pgfscope}%
\pgfpathrectangle{\pgfqpoint{0.765000in}{0.660000in}}{\pgfqpoint{4.620000in}{4.620000in}}%
\pgfusepath{clip}%
\pgfsetbuttcap%
\pgfsetroundjoin%
\definecolor{currentfill}{rgb}{1.000000,0.894118,0.788235}%
\pgfsetfillcolor{currentfill}%
\pgfsetlinewidth{0.000000pt}%
\definecolor{currentstroke}{rgb}{1.000000,0.894118,0.788235}%
\pgfsetstrokecolor{currentstroke}%
\pgfsetdash{}{0pt}%
\pgfpathmoveto{\pgfqpoint{3.519640in}{2.584743in}}%
\pgfpathlineto{\pgfqpoint{3.630537in}{2.648770in}}%
\pgfpathlineto{\pgfqpoint{3.587940in}{2.610637in}}%
\pgfpathlineto{\pgfqpoint{3.477043in}{2.546611in}}%
\pgfpathlineto{\pgfqpoint{3.519640in}{2.584743in}}%
\pgfpathclose%
\pgfusepath{fill}%
\end{pgfscope}%
\begin{pgfscope}%
\pgfpathrectangle{\pgfqpoint{0.765000in}{0.660000in}}{\pgfqpoint{4.620000in}{4.620000in}}%
\pgfusepath{clip}%
\pgfsetbuttcap%
\pgfsetroundjoin%
\definecolor{currentfill}{rgb}{1.000000,0.894118,0.788235}%
\pgfsetfillcolor{currentfill}%
\pgfsetlinewidth{0.000000pt}%
\definecolor{currentstroke}{rgb}{1.000000,0.894118,0.788235}%
\pgfsetstrokecolor{currentstroke}%
\pgfsetdash{}{0pt}%
\pgfpathmoveto{\pgfqpoint{3.519640in}{2.840848in}}%
\pgfpathlineto{\pgfqpoint{3.408744in}{2.776822in}}%
\pgfpathlineto{\pgfqpoint{3.366147in}{2.738689in}}%
\pgfpathlineto{\pgfqpoint{3.477043in}{2.802716in}}%
\pgfpathlineto{\pgfqpoint{3.519640in}{2.840848in}}%
\pgfpathclose%
\pgfusepath{fill}%
\end{pgfscope}%
\begin{pgfscope}%
\pgfpathrectangle{\pgfqpoint{0.765000in}{0.660000in}}{\pgfqpoint{4.620000in}{4.620000in}}%
\pgfusepath{clip}%
\pgfsetbuttcap%
\pgfsetroundjoin%
\definecolor{currentfill}{rgb}{1.000000,0.894118,0.788235}%
\pgfsetfillcolor{currentfill}%
\pgfsetlinewidth{0.000000pt}%
\definecolor{currentstroke}{rgb}{1.000000,0.894118,0.788235}%
\pgfsetstrokecolor{currentstroke}%
\pgfsetdash{}{0pt}%
\pgfpathmoveto{\pgfqpoint{3.630537in}{2.776822in}}%
\pgfpathlineto{\pgfqpoint{3.519640in}{2.840848in}}%
\pgfpathlineto{\pgfqpoint{3.477043in}{2.802716in}}%
\pgfpathlineto{\pgfqpoint{3.587940in}{2.738689in}}%
\pgfpathlineto{\pgfqpoint{3.630537in}{2.776822in}}%
\pgfpathclose%
\pgfusepath{fill}%
\end{pgfscope}%
\begin{pgfscope}%
\pgfpathrectangle{\pgfqpoint{0.765000in}{0.660000in}}{\pgfqpoint{4.620000in}{4.620000in}}%
\pgfusepath{clip}%
\pgfsetbuttcap%
\pgfsetroundjoin%
\definecolor{currentfill}{rgb}{1.000000,0.894118,0.788235}%
\pgfsetfillcolor{currentfill}%
\pgfsetlinewidth{0.000000pt}%
\definecolor{currentstroke}{rgb}{1.000000,0.894118,0.788235}%
\pgfsetstrokecolor{currentstroke}%
\pgfsetdash{}{0pt}%
\pgfpathmoveto{\pgfqpoint{3.408744in}{2.648770in}}%
\pgfpathlineto{\pgfqpoint{3.408744in}{2.776822in}}%
\pgfpathlineto{\pgfqpoint{3.366147in}{2.738689in}}%
\pgfpathlineto{\pgfqpoint{3.477043in}{2.546611in}}%
\pgfpathlineto{\pgfqpoint{3.408744in}{2.648770in}}%
\pgfpathclose%
\pgfusepath{fill}%
\end{pgfscope}%
\begin{pgfscope}%
\pgfpathrectangle{\pgfqpoint{0.765000in}{0.660000in}}{\pgfqpoint{4.620000in}{4.620000in}}%
\pgfusepath{clip}%
\pgfsetbuttcap%
\pgfsetroundjoin%
\definecolor{currentfill}{rgb}{1.000000,0.894118,0.788235}%
\pgfsetfillcolor{currentfill}%
\pgfsetlinewidth{0.000000pt}%
\definecolor{currentstroke}{rgb}{1.000000,0.894118,0.788235}%
\pgfsetstrokecolor{currentstroke}%
\pgfsetdash{}{0pt}%
\pgfpathmoveto{\pgfqpoint{3.519640in}{2.584743in}}%
\pgfpathlineto{\pgfqpoint{3.519640in}{2.712796in}}%
\pgfpathlineto{\pgfqpoint{3.477043in}{2.674663in}}%
\pgfpathlineto{\pgfqpoint{3.366147in}{2.610637in}}%
\pgfpathlineto{\pgfqpoint{3.519640in}{2.584743in}}%
\pgfpathclose%
\pgfusepath{fill}%
\end{pgfscope}%
\begin{pgfscope}%
\pgfpathrectangle{\pgfqpoint{0.765000in}{0.660000in}}{\pgfqpoint{4.620000in}{4.620000in}}%
\pgfusepath{clip}%
\pgfsetbuttcap%
\pgfsetroundjoin%
\definecolor{currentfill}{rgb}{1.000000,0.894118,0.788235}%
\pgfsetfillcolor{currentfill}%
\pgfsetlinewidth{0.000000pt}%
\definecolor{currentstroke}{rgb}{1.000000,0.894118,0.788235}%
\pgfsetstrokecolor{currentstroke}%
\pgfsetdash{}{0pt}%
\pgfpathmoveto{\pgfqpoint{3.519640in}{2.712796in}}%
\pgfpathlineto{\pgfqpoint{3.630537in}{2.776822in}}%
\pgfpathlineto{\pgfqpoint{3.587940in}{2.738689in}}%
\pgfpathlineto{\pgfqpoint{3.477043in}{2.674663in}}%
\pgfpathlineto{\pgfqpoint{3.519640in}{2.712796in}}%
\pgfpathclose%
\pgfusepath{fill}%
\end{pgfscope}%
\begin{pgfscope}%
\pgfpathrectangle{\pgfqpoint{0.765000in}{0.660000in}}{\pgfqpoint{4.620000in}{4.620000in}}%
\pgfusepath{clip}%
\pgfsetbuttcap%
\pgfsetroundjoin%
\definecolor{currentfill}{rgb}{1.000000,0.894118,0.788235}%
\pgfsetfillcolor{currentfill}%
\pgfsetlinewidth{0.000000pt}%
\definecolor{currentstroke}{rgb}{1.000000,0.894118,0.788235}%
\pgfsetstrokecolor{currentstroke}%
\pgfsetdash{}{0pt}%
\pgfpathmoveto{\pgfqpoint{3.408744in}{2.776822in}}%
\pgfpathlineto{\pgfqpoint{3.519640in}{2.712796in}}%
\pgfpathlineto{\pgfqpoint{3.477043in}{2.674663in}}%
\pgfpathlineto{\pgfqpoint{3.366147in}{2.738689in}}%
\pgfpathlineto{\pgfqpoint{3.408744in}{2.776822in}}%
\pgfpathclose%
\pgfusepath{fill}%
\end{pgfscope}%
\begin{pgfscope}%
\pgfpathrectangle{\pgfqpoint{0.765000in}{0.660000in}}{\pgfqpoint{4.620000in}{4.620000in}}%
\pgfusepath{clip}%
\pgfsetbuttcap%
\pgfsetroundjoin%
\definecolor{currentfill}{rgb}{1.000000,0.894118,0.788235}%
\pgfsetfillcolor{currentfill}%
\pgfsetlinewidth{0.000000pt}%
\definecolor{currentstroke}{rgb}{1.000000,0.894118,0.788235}%
\pgfsetstrokecolor{currentstroke}%
\pgfsetdash{}{0pt}%
\pgfpathmoveto{\pgfqpoint{3.546263in}{2.765531in}}%
\pgfpathlineto{\pgfqpoint{3.435366in}{2.701504in}}%
\pgfpathlineto{\pgfqpoint{3.546263in}{2.637478in}}%
\pgfpathlineto{\pgfqpoint{3.657159in}{2.701504in}}%
\pgfpathlineto{\pgfqpoint{3.546263in}{2.765531in}}%
\pgfpathclose%
\pgfusepath{fill}%
\end{pgfscope}%
\begin{pgfscope}%
\pgfpathrectangle{\pgfqpoint{0.765000in}{0.660000in}}{\pgfqpoint{4.620000in}{4.620000in}}%
\pgfusepath{clip}%
\pgfsetbuttcap%
\pgfsetroundjoin%
\definecolor{currentfill}{rgb}{1.000000,0.894118,0.788235}%
\pgfsetfillcolor{currentfill}%
\pgfsetlinewidth{0.000000pt}%
\definecolor{currentstroke}{rgb}{1.000000,0.894118,0.788235}%
\pgfsetstrokecolor{currentstroke}%
\pgfsetdash{}{0pt}%
\pgfpathmoveto{\pgfqpoint{3.546263in}{2.765531in}}%
\pgfpathlineto{\pgfqpoint{3.435366in}{2.701504in}}%
\pgfpathlineto{\pgfqpoint{3.435366in}{2.829557in}}%
\pgfpathlineto{\pgfqpoint{3.546263in}{2.893583in}}%
\pgfpathlineto{\pgfqpoint{3.546263in}{2.765531in}}%
\pgfpathclose%
\pgfusepath{fill}%
\end{pgfscope}%
\begin{pgfscope}%
\pgfpathrectangle{\pgfqpoint{0.765000in}{0.660000in}}{\pgfqpoint{4.620000in}{4.620000in}}%
\pgfusepath{clip}%
\pgfsetbuttcap%
\pgfsetroundjoin%
\definecolor{currentfill}{rgb}{1.000000,0.894118,0.788235}%
\pgfsetfillcolor{currentfill}%
\pgfsetlinewidth{0.000000pt}%
\definecolor{currentstroke}{rgb}{1.000000,0.894118,0.788235}%
\pgfsetstrokecolor{currentstroke}%
\pgfsetdash{}{0pt}%
\pgfpathmoveto{\pgfqpoint{3.546263in}{2.765531in}}%
\pgfpathlineto{\pgfqpoint{3.657159in}{2.701504in}}%
\pgfpathlineto{\pgfqpoint{3.657159in}{2.829557in}}%
\pgfpathlineto{\pgfqpoint{3.546263in}{2.893583in}}%
\pgfpathlineto{\pgfqpoint{3.546263in}{2.765531in}}%
\pgfpathclose%
\pgfusepath{fill}%
\end{pgfscope}%
\begin{pgfscope}%
\pgfpathrectangle{\pgfqpoint{0.765000in}{0.660000in}}{\pgfqpoint{4.620000in}{4.620000in}}%
\pgfusepath{clip}%
\pgfsetbuttcap%
\pgfsetroundjoin%
\definecolor{currentfill}{rgb}{1.000000,0.894118,0.788235}%
\pgfsetfillcolor{currentfill}%
\pgfsetlinewidth{0.000000pt}%
\definecolor{currentstroke}{rgb}{1.000000,0.894118,0.788235}%
\pgfsetstrokecolor{currentstroke}%
\pgfsetdash{}{0pt}%
\pgfpathmoveto{\pgfqpoint{3.519640in}{2.712796in}}%
\pgfpathlineto{\pgfqpoint{3.408744in}{2.648770in}}%
\pgfpathlineto{\pgfqpoint{3.519640in}{2.584743in}}%
\pgfpathlineto{\pgfqpoint{3.630537in}{2.648770in}}%
\pgfpathlineto{\pgfqpoint{3.519640in}{2.712796in}}%
\pgfpathclose%
\pgfusepath{fill}%
\end{pgfscope}%
\begin{pgfscope}%
\pgfpathrectangle{\pgfqpoint{0.765000in}{0.660000in}}{\pgfqpoint{4.620000in}{4.620000in}}%
\pgfusepath{clip}%
\pgfsetbuttcap%
\pgfsetroundjoin%
\definecolor{currentfill}{rgb}{1.000000,0.894118,0.788235}%
\pgfsetfillcolor{currentfill}%
\pgfsetlinewidth{0.000000pt}%
\definecolor{currentstroke}{rgb}{1.000000,0.894118,0.788235}%
\pgfsetstrokecolor{currentstroke}%
\pgfsetdash{}{0pt}%
\pgfpathmoveto{\pgfqpoint{3.519640in}{2.712796in}}%
\pgfpathlineto{\pgfqpoint{3.408744in}{2.648770in}}%
\pgfpathlineto{\pgfqpoint{3.408744in}{2.776822in}}%
\pgfpathlineto{\pgfqpoint{3.519640in}{2.840848in}}%
\pgfpathlineto{\pgfqpoint{3.519640in}{2.712796in}}%
\pgfpathclose%
\pgfusepath{fill}%
\end{pgfscope}%
\begin{pgfscope}%
\pgfpathrectangle{\pgfqpoint{0.765000in}{0.660000in}}{\pgfqpoint{4.620000in}{4.620000in}}%
\pgfusepath{clip}%
\pgfsetbuttcap%
\pgfsetroundjoin%
\definecolor{currentfill}{rgb}{1.000000,0.894118,0.788235}%
\pgfsetfillcolor{currentfill}%
\pgfsetlinewidth{0.000000pt}%
\definecolor{currentstroke}{rgb}{1.000000,0.894118,0.788235}%
\pgfsetstrokecolor{currentstroke}%
\pgfsetdash{}{0pt}%
\pgfpathmoveto{\pgfqpoint{3.519640in}{2.712796in}}%
\pgfpathlineto{\pgfqpoint{3.630537in}{2.648770in}}%
\pgfpathlineto{\pgfqpoint{3.630537in}{2.776822in}}%
\pgfpathlineto{\pgfqpoint{3.519640in}{2.840848in}}%
\pgfpathlineto{\pgfqpoint{3.519640in}{2.712796in}}%
\pgfpathclose%
\pgfusepath{fill}%
\end{pgfscope}%
\begin{pgfscope}%
\pgfpathrectangle{\pgfqpoint{0.765000in}{0.660000in}}{\pgfqpoint{4.620000in}{4.620000in}}%
\pgfusepath{clip}%
\pgfsetbuttcap%
\pgfsetroundjoin%
\definecolor{currentfill}{rgb}{1.000000,0.894118,0.788235}%
\pgfsetfillcolor{currentfill}%
\pgfsetlinewidth{0.000000pt}%
\definecolor{currentstroke}{rgb}{1.000000,0.894118,0.788235}%
\pgfsetstrokecolor{currentstroke}%
\pgfsetdash{}{0pt}%
\pgfpathmoveto{\pgfqpoint{3.546263in}{2.893583in}}%
\pgfpathlineto{\pgfqpoint{3.435366in}{2.829557in}}%
\pgfpathlineto{\pgfqpoint{3.546263in}{2.765531in}}%
\pgfpathlineto{\pgfqpoint{3.657159in}{2.829557in}}%
\pgfpathlineto{\pgfqpoint{3.546263in}{2.893583in}}%
\pgfpathclose%
\pgfusepath{fill}%
\end{pgfscope}%
\begin{pgfscope}%
\pgfpathrectangle{\pgfqpoint{0.765000in}{0.660000in}}{\pgfqpoint{4.620000in}{4.620000in}}%
\pgfusepath{clip}%
\pgfsetbuttcap%
\pgfsetroundjoin%
\definecolor{currentfill}{rgb}{1.000000,0.894118,0.788235}%
\pgfsetfillcolor{currentfill}%
\pgfsetlinewidth{0.000000pt}%
\definecolor{currentstroke}{rgb}{1.000000,0.894118,0.788235}%
\pgfsetstrokecolor{currentstroke}%
\pgfsetdash{}{0pt}%
\pgfpathmoveto{\pgfqpoint{3.546263in}{2.637478in}}%
\pgfpathlineto{\pgfqpoint{3.657159in}{2.701504in}}%
\pgfpathlineto{\pgfqpoint{3.657159in}{2.829557in}}%
\pgfpathlineto{\pgfqpoint{3.546263in}{2.765531in}}%
\pgfpathlineto{\pgfqpoint{3.546263in}{2.637478in}}%
\pgfpathclose%
\pgfusepath{fill}%
\end{pgfscope}%
\begin{pgfscope}%
\pgfpathrectangle{\pgfqpoint{0.765000in}{0.660000in}}{\pgfqpoint{4.620000in}{4.620000in}}%
\pgfusepath{clip}%
\pgfsetbuttcap%
\pgfsetroundjoin%
\definecolor{currentfill}{rgb}{1.000000,0.894118,0.788235}%
\pgfsetfillcolor{currentfill}%
\pgfsetlinewidth{0.000000pt}%
\definecolor{currentstroke}{rgb}{1.000000,0.894118,0.788235}%
\pgfsetstrokecolor{currentstroke}%
\pgfsetdash{}{0pt}%
\pgfpathmoveto{\pgfqpoint{3.435366in}{2.701504in}}%
\pgfpathlineto{\pgfqpoint{3.546263in}{2.637478in}}%
\pgfpathlineto{\pgfqpoint{3.546263in}{2.765531in}}%
\pgfpathlineto{\pgfqpoint{3.435366in}{2.829557in}}%
\pgfpathlineto{\pgfqpoint{3.435366in}{2.701504in}}%
\pgfpathclose%
\pgfusepath{fill}%
\end{pgfscope}%
\begin{pgfscope}%
\pgfpathrectangle{\pgfqpoint{0.765000in}{0.660000in}}{\pgfqpoint{4.620000in}{4.620000in}}%
\pgfusepath{clip}%
\pgfsetbuttcap%
\pgfsetroundjoin%
\definecolor{currentfill}{rgb}{1.000000,0.894118,0.788235}%
\pgfsetfillcolor{currentfill}%
\pgfsetlinewidth{0.000000pt}%
\definecolor{currentstroke}{rgb}{1.000000,0.894118,0.788235}%
\pgfsetstrokecolor{currentstroke}%
\pgfsetdash{}{0pt}%
\pgfpathmoveto{\pgfqpoint{3.519640in}{2.840848in}}%
\pgfpathlineto{\pgfqpoint{3.408744in}{2.776822in}}%
\pgfpathlineto{\pgfqpoint{3.519640in}{2.712796in}}%
\pgfpathlineto{\pgfqpoint{3.630537in}{2.776822in}}%
\pgfpathlineto{\pgfqpoint{3.519640in}{2.840848in}}%
\pgfpathclose%
\pgfusepath{fill}%
\end{pgfscope}%
\begin{pgfscope}%
\pgfpathrectangle{\pgfqpoint{0.765000in}{0.660000in}}{\pgfqpoint{4.620000in}{4.620000in}}%
\pgfusepath{clip}%
\pgfsetbuttcap%
\pgfsetroundjoin%
\definecolor{currentfill}{rgb}{1.000000,0.894118,0.788235}%
\pgfsetfillcolor{currentfill}%
\pgfsetlinewidth{0.000000pt}%
\definecolor{currentstroke}{rgb}{1.000000,0.894118,0.788235}%
\pgfsetstrokecolor{currentstroke}%
\pgfsetdash{}{0pt}%
\pgfpathmoveto{\pgfqpoint{3.519640in}{2.584743in}}%
\pgfpathlineto{\pgfqpoint{3.630537in}{2.648770in}}%
\pgfpathlineto{\pgfqpoint{3.630537in}{2.776822in}}%
\pgfpathlineto{\pgfqpoint{3.519640in}{2.712796in}}%
\pgfpathlineto{\pgfqpoint{3.519640in}{2.584743in}}%
\pgfpathclose%
\pgfusepath{fill}%
\end{pgfscope}%
\begin{pgfscope}%
\pgfpathrectangle{\pgfqpoint{0.765000in}{0.660000in}}{\pgfqpoint{4.620000in}{4.620000in}}%
\pgfusepath{clip}%
\pgfsetbuttcap%
\pgfsetroundjoin%
\definecolor{currentfill}{rgb}{1.000000,0.894118,0.788235}%
\pgfsetfillcolor{currentfill}%
\pgfsetlinewidth{0.000000pt}%
\definecolor{currentstroke}{rgb}{1.000000,0.894118,0.788235}%
\pgfsetstrokecolor{currentstroke}%
\pgfsetdash{}{0pt}%
\pgfpathmoveto{\pgfqpoint{3.408744in}{2.648770in}}%
\pgfpathlineto{\pgfqpoint{3.519640in}{2.584743in}}%
\pgfpathlineto{\pgfqpoint{3.519640in}{2.712796in}}%
\pgfpathlineto{\pgfqpoint{3.408744in}{2.776822in}}%
\pgfpathlineto{\pgfqpoint{3.408744in}{2.648770in}}%
\pgfpathclose%
\pgfusepath{fill}%
\end{pgfscope}%
\begin{pgfscope}%
\pgfpathrectangle{\pgfqpoint{0.765000in}{0.660000in}}{\pgfqpoint{4.620000in}{4.620000in}}%
\pgfusepath{clip}%
\pgfsetbuttcap%
\pgfsetroundjoin%
\definecolor{currentfill}{rgb}{1.000000,0.894118,0.788235}%
\pgfsetfillcolor{currentfill}%
\pgfsetlinewidth{0.000000pt}%
\definecolor{currentstroke}{rgb}{1.000000,0.894118,0.788235}%
\pgfsetstrokecolor{currentstroke}%
\pgfsetdash{}{0pt}%
\pgfpathmoveto{\pgfqpoint{3.546263in}{2.765531in}}%
\pgfpathlineto{\pgfqpoint{3.435366in}{2.701504in}}%
\pgfpathlineto{\pgfqpoint{3.408744in}{2.648770in}}%
\pgfpathlineto{\pgfqpoint{3.519640in}{2.712796in}}%
\pgfpathlineto{\pgfqpoint{3.546263in}{2.765531in}}%
\pgfpathclose%
\pgfusepath{fill}%
\end{pgfscope}%
\begin{pgfscope}%
\pgfpathrectangle{\pgfqpoint{0.765000in}{0.660000in}}{\pgfqpoint{4.620000in}{4.620000in}}%
\pgfusepath{clip}%
\pgfsetbuttcap%
\pgfsetroundjoin%
\definecolor{currentfill}{rgb}{1.000000,0.894118,0.788235}%
\pgfsetfillcolor{currentfill}%
\pgfsetlinewidth{0.000000pt}%
\definecolor{currentstroke}{rgb}{1.000000,0.894118,0.788235}%
\pgfsetstrokecolor{currentstroke}%
\pgfsetdash{}{0pt}%
\pgfpathmoveto{\pgfqpoint{3.657159in}{2.701504in}}%
\pgfpathlineto{\pgfqpoint{3.546263in}{2.765531in}}%
\pgfpathlineto{\pgfqpoint{3.519640in}{2.712796in}}%
\pgfpathlineto{\pgfqpoint{3.630537in}{2.648770in}}%
\pgfpathlineto{\pgfqpoint{3.657159in}{2.701504in}}%
\pgfpathclose%
\pgfusepath{fill}%
\end{pgfscope}%
\begin{pgfscope}%
\pgfpathrectangle{\pgfqpoint{0.765000in}{0.660000in}}{\pgfqpoint{4.620000in}{4.620000in}}%
\pgfusepath{clip}%
\pgfsetbuttcap%
\pgfsetroundjoin%
\definecolor{currentfill}{rgb}{1.000000,0.894118,0.788235}%
\pgfsetfillcolor{currentfill}%
\pgfsetlinewidth{0.000000pt}%
\definecolor{currentstroke}{rgb}{1.000000,0.894118,0.788235}%
\pgfsetstrokecolor{currentstroke}%
\pgfsetdash{}{0pt}%
\pgfpathmoveto{\pgfqpoint{3.546263in}{2.765531in}}%
\pgfpathlineto{\pgfqpoint{3.546263in}{2.893583in}}%
\pgfpathlineto{\pgfqpoint{3.519640in}{2.840848in}}%
\pgfpathlineto{\pgfqpoint{3.630537in}{2.648770in}}%
\pgfpathlineto{\pgfqpoint{3.546263in}{2.765531in}}%
\pgfpathclose%
\pgfusepath{fill}%
\end{pgfscope}%
\begin{pgfscope}%
\pgfpathrectangle{\pgfqpoint{0.765000in}{0.660000in}}{\pgfqpoint{4.620000in}{4.620000in}}%
\pgfusepath{clip}%
\pgfsetbuttcap%
\pgfsetroundjoin%
\definecolor{currentfill}{rgb}{1.000000,0.894118,0.788235}%
\pgfsetfillcolor{currentfill}%
\pgfsetlinewidth{0.000000pt}%
\definecolor{currentstroke}{rgb}{1.000000,0.894118,0.788235}%
\pgfsetstrokecolor{currentstroke}%
\pgfsetdash{}{0pt}%
\pgfpathmoveto{\pgfqpoint{3.657159in}{2.701504in}}%
\pgfpathlineto{\pgfqpoint{3.657159in}{2.829557in}}%
\pgfpathlineto{\pgfqpoint{3.630537in}{2.776822in}}%
\pgfpathlineto{\pgfqpoint{3.519640in}{2.712796in}}%
\pgfpathlineto{\pgfqpoint{3.657159in}{2.701504in}}%
\pgfpathclose%
\pgfusepath{fill}%
\end{pgfscope}%
\begin{pgfscope}%
\pgfpathrectangle{\pgfqpoint{0.765000in}{0.660000in}}{\pgfqpoint{4.620000in}{4.620000in}}%
\pgfusepath{clip}%
\pgfsetbuttcap%
\pgfsetroundjoin%
\definecolor{currentfill}{rgb}{1.000000,0.894118,0.788235}%
\pgfsetfillcolor{currentfill}%
\pgfsetlinewidth{0.000000pt}%
\definecolor{currentstroke}{rgb}{1.000000,0.894118,0.788235}%
\pgfsetstrokecolor{currentstroke}%
\pgfsetdash{}{0pt}%
\pgfpathmoveto{\pgfqpoint{3.435366in}{2.701504in}}%
\pgfpathlineto{\pgfqpoint{3.546263in}{2.637478in}}%
\pgfpathlineto{\pgfqpoint{3.519640in}{2.584743in}}%
\pgfpathlineto{\pgfqpoint{3.408744in}{2.648770in}}%
\pgfpathlineto{\pgfqpoint{3.435366in}{2.701504in}}%
\pgfpathclose%
\pgfusepath{fill}%
\end{pgfscope}%
\begin{pgfscope}%
\pgfpathrectangle{\pgfqpoint{0.765000in}{0.660000in}}{\pgfqpoint{4.620000in}{4.620000in}}%
\pgfusepath{clip}%
\pgfsetbuttcap%
\pgfsetroundjoin%
\definecolor{currentfill}{rgb}{1.000000,0.894118,0.788235}%
\pgfsetfillcolor{currentfill}%
\pgfsetlinewidth{0.000000pt}%
\definecolor{currentstroke}{rgb}{1.000000,0.894118,0.788235}%
\pgfsetstrokecolor{currentstroke}%
\pgfsetdash{}{0pt}%
\pgfpathmoveto{\pgfqpoint{3.546263in}{2.637478in}}%
\pgfpathlineto{\pgfqpoint{3.657159in}{2.701504in}}%
\pgfpathlineto{\pgfqpoint{3.630537in}{2.648770in}}%
\pgfpathlineto{\pgfqpoint{3.519640in}{2.584743in}}%
\pgfpathlineto{\pgfqpoint{3.546263in}{2.637478in}}%
\pgfpathclose%
\pgfusepath{fill}%
\end{pgfscope}%
\begin{pgfscope}%
\pgfpathrectangle{\pgfqpoint{0.765000in}{0.660000in}}{\pgfqpoint{4.620000in}{4.620000in}}%
\pgfusepath{clip}%
\pgfsetbuttcap%
\pgfsetroundjoin%
\definecolor{currentfill}{rgb}{1.000000,0.894118,0.788235}%
\pgfsetfillcolor{currentfill}%
\pgfsetlinewidth{0.000000pt}%
\definecolor{currentstroke}{rgb}{1.000000,0.894118,0.788235}%
\pgfsetstrokecolor{currentstroke}%
\pgfsetdash{}{0pt}%
\pgfpathmoveto{\pgfqpoint{3.546263in}{2.893583in}}%
\pgfpathlineto{\pgfqpoint{3.435366in}{2.829557in}}%
\pgfpathlineto{\pgfqpoint{3.408744in}{2.776822in}}%
\pgfpathlineto{\pgfqpoint{3.519640in}{2.840848in}}%
\pgfpathlineto{\pgfqpoint{3.546263in}{2.893583in}}%
\pgfpathclose%
\pgfusepath{fill}%
\end{pgfscope}%
\begin{pgfscope}%
\pgfpathrectangle{\pgfqpoint{0.765000in}{0.660000in}}{\pgfqpoint{4.620000in}{4.620000in}}%
\pgfusepath{clip}%
\pgfsetbuttcap%
\pgfsetroundjoin%
\definecolor{currentfill}{rgb}{1.000000,0.894118,0.788235}%
\pgfsetfillcolor{currentfill}%
\pgfsetlinewidth{0.000000pt}%
\definecolor{currentstroke}{rgb}{1.000000,0.894118,0.788235}%
\pgfsetstrokecolor{currentstroke}%
\pgfsetdash{}{0pt}%
\pgfpathmoveto{\pgfqpoint{3.657159in}{2.829557in}}%
\pgfpathlineto{\pgfqpoint{3.546263in}{2.893583in}}%
\pgfpathlineto{\pgfqpoint{3.519640in}{2.840848in}}%
\pgfpathlineto{\pgfqpoint{3.630537in}{2.776822in}}%
\pgfpathlineto{\pgfqpoint{3.657159in}{2.829557in}}%
\pgfpathclose%
\pgfusepath{fill}%
\end{pgfscope}%
\begin{pgfscope}%
\pgfpathrectangle{\pgfqpoint{0.765000in}{0.660000in}}{\pgfqpoint{4.620000in}{4.620000in}}%
\pgfusepath{clip}%
\pgfsetbuttcap%
\pgfsetroundjoin%
\definecolor{currentfill}{rgb}{1.000000,0.894118,0.788235}%
\pgfsetfillcolor{currentfill}%
\pgfsetlinewidth{0.000000pt}%
\definecolor{currentstroke}{rgb}{1.000000,0.894118,0.788235}%
\pgfsetstrokecolor{currentstroke}%
\pgfsetdash{}{0pt}%
\pgfpathmoveto{\pgfqpoint{3.435366in}{2.701504in}}%
\pgfpathlineto{\pgfqpoint{3.435366in}{2.829557in}}%
\pgfpathlineto{\pgfqpoint{3.408744in}{2.776822in}}%
\pgfpathlineto{\pgfqpoint{3.519640in}{2.584743in}}%
\pgfpathlineto{\pgfqpoint{3.435366in}{2.701504in}}%
\pgfpathclose%
\pgfusepath{fill}%
\end{pgfscope}%
\begin{pgfscope}%
\pgfpathrectangle{\pgfqpoint{0.765000in}{0.660000in}}{\pgfqpoint{4.620000in}{4.620000in}}%
\pgfusepath{clip}%
\pgfsetbuttcap%
\pgfsetroundjoin%
\definecolor{currentfill}{rgb}{1.000000,0.894118,0.788235}%
\pgfsetfillcolor{currentfill}%
\pgfsetlinewidth{0.000000pt}%
\definecolor{currentstroke}{rgb}{1.000000,0.894118,0.788235}%
\pgfsetstrokecolor{currentstroke}%
\pgfsetdash{}{0pt}%
\pgfpathmoveto{\pgfqpoint{3.546263in}{2.637478in}}%
\pgfpathlineto{\pgfqpoint{3.546263in}{2.765531in}}%
\pgfpathlineto{\pgfqpoint{3.519640in}{2.712796in}}%
\pgfpathlineto{\pgfqpoint{3.408744in}{2.648770in}}%
\pgfpathlineto{\pgfqpoint{3.546263in}{2.637478in}}%
\pgfpathclose%
\pgfusepath{fill}%
\end{pgfscope}%
\begin{pgfscope}%
\pgfpathrectangle{\pgfqpoint{0.765000in}{0.660000in}}{\pgfqpoint{4.620000in}{4.620000in}}%
\pgfusepath{clip}%
\pgfsetbuttcap%
\pgfsetroundjoin%
\definecolor{currentfill}{rgb}{1.000000,0.894118,0.788235}%
\pgfsetfillcolor{currentfill}%
\pgfsetlinewidth{0.000000pt}%
\definecolor{currentstroke}{rgb}{1.000000,0.894118,0.788235}%
\pgfsetstrokecolor{currentstroke}%
\pgfsetdash{}{0pt}%
\pgfpathmoveto{\pgfqpoint{3.546263in}{2.765531in}}%
\pgfpathlineto{\pgfqpoint{3.657159in}{2.829557in}}%
\pgfpathlineto{\pgfqpoint{3.630537in}{2.776822in}}%
\pgfpathlineto{\pgfqpoint{3.519640in}{2.712796in}}%
\pgfpathlineto{\pgfqpoint{3.546263in}{2.765531in}}%
\pgfpathclose%
\pgfusepath{fill}%
\end{pgfscope}%
\begin{pgfscope}%
\pgfpathrectangle{\pgfqpoint{0.765000in}{0.660000in}}{\pgfqpoint{4.620000in}{4.620000in}}%
\pgfusepath{clip}%
\pgfsetbuttcap%
\pgfsetroundjoin%
\definecolor{currentfill}{rgb}{1.000000,0.894118,0.788235}%
\pgfsetfillcolor{currentfill}%
\pgfsetlinewidth{0.000000pt}%
\definecolor{currentstroke}{rgb}{1.000000,0.894118,0.788235}%
\pgfsetstrokecolor{currentstroke}%
\pgfsetdash{}{0pt}%
\pgfpathmoveto{\pgfqpoint{3.435366in}{2.829557in}}%
\pgfpathlineto{\pgfqpoint{3.546263in}{2.765531in}}%
\pgfpathlineto{\pgfqpoint{3.519640in}{2.712796in}}%
\pgfpathlineto{\pgfqpoint{3.408744in}{2.776822in}}%
\pgfpathlineto{\pgfqpoint{3.435366in}{2.829557in}}%
\pgfpathclose%
\pgfusepath{fill}%
\end{pgfscope}%
\begin{pgfscope}%
\pgfpathrectangle{\pgfqpoint{0.765000in}{0.660000in}}{\pgfqpoint{4.620000in}{4.620000in}}%
\pgfusepath{clip}%
\pgfsetbuttcap%
\pgfsetroundjoin%
\definecolor{currentfill}{rgb}{1.000000,0.894118,0.788235}%
\pgfsetfillcolor{currentfill}%
\pgfsetlinewidth{0.000000pt}%
\definecolor{currentstroke}{rgb}{1.000000,0.894118,0.788235}%
\pgfsetstrokecolor{currentstroke}%
\pgfsetdash{}{0pt}%
\pgfpathmoveto{\pgfqpoint{3.546263in}{2.765531in}}%
\pgfpathlineto{\pgfqpoint{3.435366in}{2.701504in}}%
\pgfpathlineto{\pgfqpoint{3.546263in}{2.637478in}}%
\pgfpathlineto{\pgfqpoint{3.657159in}{2.701504in}}%
\pgfpathlineto{\pgfqpoint{3.546263in}{2.765531in}}%
\pgfpathclose%
\pgfusepath{fill}%
\end{pgfscope}%
\begin{pgfscope}%
\pgfpathrectangle{\pgfqpoint{0.765000in}{0.660000in}}{\pgfqpoint{4.620000in}{4.620000in}}%
\pgfusepath{clip}%
\pgfsetbuttcap%
\pgfsetroundjoin%
\definecolor{currentfill}{rgb}{1.000000,0.894118,0.788235}%
\pgfsetfillcolor{currentfill}%
\pgfsetlinewidth{0.000000pt}%
\definecolor{currentstroke}{rgb}{1.000000,0.894118,0.788235}%
\pgfsetstrokecolor{currentstroke}%
\pgfsetdash{}{0pt}%
\pgfpathmoveto{\pgfqpoint{3.546263in}{2.765531in}}%
\pgfpathlineto{\pgfqpoint{3.435366in}{2.701504in}}%
\pgfpathlineto{\pgfqpoint{3.435366in}{2.829557in}}%
\pgfpathlineto{\pgfqpoint{3.546263in}{2.893583in}}%
\pgfpathlineto{\pgfqpoint{3.546263in}{2.765531in}}%
\pgfpathclose%
\pgfusepath{fill}%
\end{pgfscope}%
\begin{pgfscope}%
\pgfpathrectangle{\pgfqpoint{0.765000in}{0.660000in}}{\pgfqpoint{4.620000in}{4.620000in}}%
\pgfusepath{clip}%
\pgfsetbuttcap%
\pgfsetroundjoin%
\definecolor{currentfill}{rgb}{1.000000,0.894118,0.788235}%
\pgfsetfillcolor{currentfill}%
\pgfsetlinewidth{0.000000pt}%
\definecolor{currentstroke}{rgb}{1.000000,0.894118,0.788235}%
\pgfsetstrokecolor{currentstroke}%
\pgfsetdash{}{0pt}%
\pgfpathmoveto{\pgfqpoint{3.546263in}{2.765531in}}%
\pgfpathlineto{\pgfqpoint{3.657159in}{2.701504in}}%
\pgfpathlineto{\pgfqpoint{3.657159in}{2.829557in}}%
\pgfpathlineto{\pgfqpoint{3.546263in}{2.893583in}}%
\pgfpathlineto{\pgfqpoint{3.546263in}{2.765531in}}%
\pgfpathclose%
\pgfusepath{fill}%
\end{pgfscope}%
\begin{pgfscope}%
\pgfpathrectangle{\pgfqpoint{0.765000in}{0.660000in}}{\pgfqpoint{4.620000in}{4.620000in}}%
\pgfusepath{clip}%
\pgfsetbuttcap%
\pgfsetroundjoin%
\definecolor{currentfill}{rgb}{1.000000,0.894118,0.788235}%
\pgfsetfillcolor{currentfill}%
\pgfsetlinewidth{0.000000pt}%
\definecolor{currentstroke}{rgb}{1.000000,0.894118,0.788235}%
\pgfsetstrokecolor{currentstroke}%
\pgfsetdash{}{0pt}%
\pgfpathmoveto{\pgfqpoint{3.551407in}{2.776901in}}%
\pgfpathlineto{\pgfqpoint{3.440511in}{2.712874in}}%
\pgfpathlineto{\pgfqpoint{3.551407in}{2.648848in}}%
\pgfpathlineto{\pgfqpoint{3.662304in}{2.712874in}}%
\pgfpathlineto{\pgfqpoint{3.551407in}{2.776901in}}%
\pgfpathclose%
\pgfusepath{fill}%
\end{pgfscope}%
\begin{pgfscope}%
\pgfpathrectangle{\pgfqpoint{0.765000in}{0.660000in}}{\pgfqpoint{4.620000in}{4.620000in}}%
\pgfusepath{clip}%
\pgfsetbuttcap%
\pgfsetroundjoin%
\definecolor{currentfill}{rgb}{1.000000,0.894118,0.788235}%
\pgfsetfillcolor{currentfill}%
\pgfsetlinewidth{0.000000pt}%
\definecolor{currentstroke}{rgb}{1.000000,0.894118,0.788235}%
\pgfsetstrokecolor{currentstroke}%
\pgfsetdash{}{0pt}%
\pgfpathmoveto{\pgfqpoint{3.551407in}{2.776901in}}%
\pgfpathlineto{\pgfqpoint{3.440511in}{2.712874in}}%
\pgfpathlineto{\pgfqpoint{3.440511in}{2.840927in}}%
\pgfpathlineto{\pgfqpoint{3.551407in}{2.904953in}}%
\pgfpathlineto{\pgfqpoint{3.551407in}{2.776901in}}%
\pgfpathclose%
\pgfusepath{fill}%
\end{pgfscope}%
\begin{pgfscope}%
\pgfpathrectangle{\pgfqpoint{0.765000in}{0.660000in}}{\pgfqpoint{4.620000in}{4.620000in}}%
\pgfusepath{clip}%
\pgfsetbuttcap%
\pgfsetroundjoin%
\definecolor{currentfill}{rgb}{1.000000,0.894118,0.788235}%
\pgfsetfillcolor{currentfill}%
\pgfsetlinewidth{0.000000pt}%
\definecolor{currentstroke}{rgb}{1.000000,0.894118,0.788235}%
\pgfsetstrokecolor{currentstroke}%
\pgfsetdash{}{0pt}%
\pgfpathmoveto{\pgfqpoint{3.551407in}{2.776901in}}%
\pgfpathlineto{\pgfqpoint{3.662304in}{2.712874in}}%
\pgfpathlineto{\pgfqpoint{3.662304in}{2.840927in}}%
\pgfpathlineto{\pgfqpoint{3.551407in}{2.904953in}}%
\pgfpathlineto{\pgfqpoint{3.551407in}{2.776901in}}%
\pgfpathclose%
\pgfusepath{fill}%
\end{pgfscope}%
\begin{pgfscope}%
\pgfpathrectangle{\pgfqpoint{0.765000in}{0.660000in}}{\pgfqpoint{4.620000in}{4.620000in}}%
\pgfusepath{clip}%
\pgfsetbuttcap%
\pgfsetroundjoin%
\definecolor{currentfill}{rgb}{1.000000,0.894118,0.788235}%
\pgfsetfillcolor{currentfill}%
\pgfsetlinewidth{0.000000pt}%
\definecolor{currentstroke}{rgb}{1.000000,0.894118,0.788235}%
\pgfsetstrokecolor{currentstroke}%
\pgfsetdash{}{0pt}%
\pgfpathmoveto{\pgfqpoint{3.546263in}{2.893583in}}%
\pgfpathlineto{\pgfqpoint{3.435366in}{2.829557in}}%
\pgfpathlineto{\pgfqpoint{3.546263in}{2.765531in}}%
\pgfpathlineto{\pgfqpoint{3.657159in}{2.829557in}}%
\pgfpathlineto{\pgfqpoint{3.546263in}{2.893583in}}%
\pgfpathclose%
\pgfusepath{fill}%
\end{pgfscope}%
\begin{pgfscope}%
\pgfpathrectangle{\pgfqpoint{0.765000in}{0.660000in}}{\pgfqpoint{4.620000in}{4.620000in}}%
\pgfusepath{clip}%
\pgfsetbuttcap%
\pgfsetroundjoin%
\definecolor{currentfill}{rgb}{1.000000,0.894118,0.788235}%
\pgfsetfillcolor{currentfill}%
\pgfsetlinewidth{0.000000pt}%
\definecolor{currentstroke}{rgb}{1.000000,0.894118,0.788235}%
\pgfsetstrokecolor{currentstroke}%
\pgfsetdash{}{0pt}%
\pgfpathmoveto{\pgfqpoint{3.546263in}{2.637478in}}%
\pgfpathlineto{\pgfqpoint{3.657159in}{2.701504in}}%
\pgfpathlineto{\pgfqpoint{3.657159in}{2.829557in}}%
\pgfpathlineto{\pgfqpoint{3.546263in}{2.765531in}}%
\pgfpathlineto{\pgfqpoint{3.546263in}{2.637478in}}%
\pgfpathclose%
\pgfusepath{fill}%
\end{pgfscope}%
\begin{pgfscope}%
\pgfpathrectangle{\pgfqpoint{0.765000in}{0.660000in}}{\pgfqpoint{4.620000in}{4.620000in}}%
\pgfusepath{clip}%
\pgfsetbuttcap%
\pgfsetroundjoin%
\definecolor{currentfill}{rgb}{1.000000,0.894118,0.788235}%
\pgfsetfillcolor{currentfill}%
\pgfsetlinewidth{0.000000pt}%
\definecolor{currentstroke}{rgb}{1.000000,0.894118,0.788235}%
\pgfsetstrokecolor{currentstroke}%
\pgfsetdash{}{0pt}%
\pgfpathmoveto{\pgfqpoint{3.435366in}{2.701504in}}%
\pgfpathlineto{\pgfqpoint{3.546263in}{2.637478in}}%
\pgfpathlineto{\pgfqpoint{3.546263in}{2.765531in}}%
\pgfpathlineto{\pgfqpoint{3.435366in}{2.829557in}}%
\pgfpathlineto{\pgfqpoint{3.435366in}{2.701504in}}%
\pgfpathclose%
\pgfusepath{fill}%
\end{pgfscope}%
\begin{pgfscope}%
\pgfpathrectangle{\pgfqpoint{0.765000in}{0.660000in}}{\pgfqpoint{4.620000in}{4.620000in}}%
\pgfusepath{clip}%
\pgfsetbuttcap%
\pgfsetroundjoin%
\definecolor{currentfill}{rgb}{1.000000,0.894118,0.788235}%
\pgfsetfillcolor{currentfill}%
\pgfsetlinewidth{0.000000pt}%
\definecolor{currentstroke}{rgb}{1.000000,0.894118,0.788235}%
\pgfsetstrokecolor{currentstroke}%
\pgfsetdash{}{0pt}%
\pgfpathmoveto{\pgfqpoint{3.551407in}{2.904953in}}%
\pgfpathlineto{\pgfqpoint{3.440511in}{2.840927in}}%
\pgfpathlineto{\pgfqpoint{3.551407in}{2.776901in}}%
\pgfpathlineto{\pgfqpoint{3.662304in}{2.840927in}}%
\pgfpathlineto{\pgfqpoint{3.551407in}{2.904953in}}%
\pgfpathclose%
\pgfusepath{fill}%
\end{pgfscope}%
\begin{pgfscope}%
\pgfpathrectangle{\pgfqpoint{0.765000in}{0.660000in}}{\pgfqpoint{4.620000in}{4.620000in}}%
\pgfusepath{clip}%
\pgfsetbuttcap%
\pgfsetroundjoin%
\definecolor{currentfill}{rgb}{1.000000,0.894118,0.788235}%
\pgfsetfillcolor{currentfill}%
\pgfsetlinewidth{0.000000pt}%
\definecolor{currentstroke}{rgb}{1.000000,0.894118,0.788235}%
\pgfsetstrokecolor{currentstroke}%
\pgfsetdash{}{0pt}%
\pgfpathmoveto{\pgfqpoint{3.551407in}{2.648848in}}%
\pgfpathlineto{\pgfqpoint{3.662304in}{2.712874in}}%
\pgfpathlineto{\pgfqpoint{3.662304in}{2.840927in}}%
\pgfpathlineto{\pgfqpoint{3.551407in}{2.776901in}}%
\pgfpathlineto{\pgfqpoint{3.551407in}{2.648848in}}%
\pgfpathclose%
\pgfusepath{fill}%
\end{pgfscope}%
\begin{pgfscope}%
\pgfpathrectangle{\pgfqpoint{0.765000in}{0.660000in}}{\pgfqpoint{4.620000in}{4.620000in}}%
\pgfusepath{clip}%
\pgfsetbuttcap%
\pgfsetroundjoin%
\definecolor{currentfill}{rgb}{1.000000,0.894118,0.788235}%
\pgfsetfillcolor{currentfill}%
\pgfsetlinewidth{0.000000pt}%
\definecolor{currentstroke}{rgb}{1.000000,0.894118,0.788235}%
\pgfsetstrokecolor{currentstroke}%
\pgfsetdash{}{0pt}%
\pgfpathmoveto{\pgfqpoint{3.440511in}{2.712874in}}%
\pgfpathlineto{\pgfqpoint{3.551407in}{2.648848in}}%
\pgfpathlineto{\pgfqpoint{3.551407in}{2.776901in}}%
\pgfpathlineto{\pgfqpoint{3.440511in}{2.840927in}}%
\pgfpathlineto{\pgfqpoint{3.440511in}{2.712874in}}%
\pgfpathclose%
\pgfusepath{fill}%
\end{pgfscope}%
\begin{pgfscope}%
\pgfpathrectangle{\pgfqpoint{0.765000in}{0.660000in}}{\pgfqpoint{4.620000in}{4.620000in}}%
\pgfusepath{clip}%
\pgfsetbuttcap%
\pgfsetroundjoin%
\definecolor{currentfill}{rgb}{1.000000,0.894118,0.788235}%
\pgfsetfillcolor{currentfill}%
\pgfsetlinewidth{0.000000pt}%
\definecolor{currentstroke}{rgb}{1.000000,0.894118,0.788235}%
\pgfsetstrokecolor{currentstroke}%
\pgfsetdash{}{0pt}%
\pgfpathmoveto{\pgfqpoint{3.546263in}{2.765531in}}%
\pgfpathlineto{\pgfqpoint{3.435366in}{2.701504in}}%
\pgfpathlineto{\pgfqpoint{3.440511in}{2.712874in}}%
\pgfpathlineto{\pgfqpoint{3.551407in}{2.776901in}}%
\pgfpathlineto{\pgfqpoint{3.546263in}{2.765531in}}%
\pgfpathclose%
\pgfusepath{fill}%
\end{pgfscope}%
\begin{pgfscope}%
\pgfpathrectangle{\pgfqpoint{0.765000in}{0.660000in}}{\pgfqpoint{4.620000in}{4.620000in}}%
\pgfusepath{clip}%
\pgfsetbuttcap%
\pgfsetroundjoin%
\definecolor{currentfill}{rgb}{1.000000,0.894118,0.788235}%
\pgfsetfillcolor{currentfill}%
\pgfsetlinewidth{0.000000pt}%
\definecolor{currentstroke}{rgb}{1.000000,0.894118,0.788235}%
\pgfsetstrokecolor{currentstroke}%
\pgfsetdash{}{0pt}%
\pgfpathmoveto{\pgfqpoint{3.657159in}{2.701504in}}%
\pgfpathlineto{\pgfqpoint{3.546263in}{2.765531in}}%
\pgfpathlineto{\pgfqpoint{3.551407in}{2.776901in}}%
\pgfpathlineto{\pgfqpoint{3.662304in}{2.712874in}}%
\pgfpathlineto{\pgfqpoint{3.657159in}{2.701504in}}%
\pgfpathclose%
\pgfusepath{fill}%
\end{pgfscope}%
\begin{pgfscope}%
\pgfpathrectangle{\pgfqpoint{0.765000in}{0.660000in}}{\pgfqpoint{4.620000in}{4.620000in}}%
\pgfusepath{clip}%
\pgfsetbuttcap%
\pgfsetroundjoin%
\definecolor{currentfill}{rgb}{1.000000,0.894118,0.788235}%
\pgfsetfillcolor{currentfill}%
\pgfsetlinewidth{0.000000pt}%
\definecolor{currentstroke}{rgb}{1.000000,0.894118,0.788235}%
\pgfsetstrokecolor{currentstroke}%
\pgfsetdash{}{0pt}%
\pgfpathmoveto{\pgfqpoint{3.546263in}{2.765531in}}%
\pgfpathlineto{\pgfqpoint{3.546263in}{2.893583in}}%
\pgfpathlineto{\pgfqpoint{3.551407in}{2.904953in}}%
\pgfpathlineto{\pgfqpoint{3.662304in}{2.712874in}}%
\pgfpathlineto{\pgfqpoint{3.546263in}{2.765531in}}%
\pgfpathclose%
\pgfusepath{fill}%
\end{pgfscope}%
\begin{pgfscope}%
\pgfpathrectangle{\pgfqpoint{0.765000in}{0.660000in}}{\pgfqpoint{4.620000in}{4.620000in}}%
\pgfusepath{clip}%
\pgfsetbuttcap%
\pgfsetroundjoin%
\definecolor{currentfill}{rgb}{1.000000,0.894118,0.788235}%
\pgfsetfillcolor{currentfill}%
\pgfsetlinewidth{0.000000pt}%
\definecolor{currentstroke}{rgb}{1.000000,0.894118,0.788235}%
\pgfsetstrokecolor{currentstroke}%
\pgfsetdash{}{0pt}%
\pgfpathmoveto{\pgfqpoint{3.657159in}{2.701504in}}%
\pgfpathlineto{\pgfqpoint{3.657159in}{2.829557in}}%
\pgfpathlineto{\pgfqpoint{3.662304in}{2.840927in}}%
\pgfpathlineto{\pgfqpoint{3.551407in}{2.776901in}}%
\pgfpathlineto{\pgfqpoint{3.657159in}{2.701504in}}%
\pgfpathclose%
\pgfusepath{fill}%
\end{pgfscope}%
\begin{pgfscope}%
\pgfpathrectangle{\pgfqpoint{0.765000in}{0.660000in}}{\pgfqpoint{4.620000in}{4.620000in}}%
\pgfusepath{clip}%
\pgfsetbuttcap%
\pgfsetroundjoin%
\definecolor{currentfill}{rgb}{1.000000,0.894118,0.788235}%
\pgfsetfillcolor{currentfill}%
\pgfsetlinewidth{0.000000pt}%
\definecolor{currentstroke}{rgb}{1.000000,0.894118,0.788235}%
\pgfsetstrokecolor{currentstroke}%
\pgfsetdash{}{0pt}%
\pgfpathmoveto{\pgfqpoint{3.435366in}{2.701504in}}%
\pgfpathlineto{\pgfqpoint{3.546263in}{2.637478in}}%
\pgfpathlineto{\pgfqpoint{3.551407in}{2.648848in}}%
\pgfpathlineto{\pgfqpoint{3.440511in}{2.712874in}}%
\pgfpathlineto{\pgfqpoint{3.435366in}{2.701504in}}%
\pgfpathclose%
\pgfusepath{fill}%
\end{pgfscope}%
\begin{pgfscope}%
\pgfpathrectangle{\pgfqpoint{0.765000in}{0.660000in}}{\pgfqpoint{4.620000in}{4.620000in}}%
\pgfusepath{clip}%
\pgfsetbuttcap%
\pgfsetroundjoin%
\definecolor{currentfill}{rgb}{1.000000,0.894118,0.788235}%
\pgfsetfillcolor{currentfill}%
\pgfsetlinewidth{0.000000pt}%
\definecolor{currentstroke}{rgb}{1.000000,0.894118,0.788235}%
\pgfsetstrokecolor{currentstroke}%
\pgfsetdash{}{0pt}%
\pgfpathmoveto{\pgfqpoint{3.546263in}{2.637478in}}%
\pgfpathlineto{\pgfqpoint{3.657159in}{2.701504in}}%
\pgfpathlineto{\pgfqpoint{3.662304in}{2.712874in}}%
\pgfpathlineto{\pgfqpoint{3.551407in}{2.648848in}}%
\pgfpathlineto{\pgfqpoint{3.546263in}{2.637478in}}%
\pgfpathclose%
\pgfusepath{fill}%
\end{pgfscope}%
\begin{pgfscope}%
\pgfpathrectangle{\pgfqpoint{0.765000in}{0.660000in}}{\pgfqpoint{4.620000in}{4.620000in}}%
\pgfusepath{clip}%
\pgfsetbuttcap%
\pgfsetroundjoin%
\definecolor{currentfill}{rgb}{1.000000,0.894118,0.788235}%
\pgfsetfillcolor{currentfill}%
\pgfsetlinewidth{0.000000pt}%
\definecolor{currentstroke}{rgb}{1.000000,0.894118,0.788235}%
\pgfsetstrokecolor{currentstroke}%
\pgfsetdash{}{0pt}%
\pgfpathmoveto{\pgfqpoint{3.546263in}{2.893583in}}%
\pgfpathlineto{\pgfqpoint{3.435366in}{2.829557in}}%
\pgfpathlineto{\pgfqpoint{3.440511in}{2.840927in}}%
\pgfpathlineto{\pgfqpoint{3.551407in}{2.904953in}}%
\pgfpathlineto{\pgfqpoint{3.546263in}{2.893583in}}%
\pgfpathclose%
\pgfusepath{fill}%
\end{pgfscope}%
\begin{pgfscope}%
\pgfpathrectangle{\pgfqpoint{0.765000in}{0.660000in}}{\pgfqpoint{4.620000in}{4.620000in}}%
\pgfusepath{clip}%
\pgfsetbuttcap%
\pgfsetroundjoin%
\definecolor{currentfill}{rgb}{1.000000,0.894118,0.788235}%
\pgfsetfillcolor{currentfill}%
\pgfsetlinewidth{0.000000pt}%
\definecolor{currentstroke}{rgb}{1.000000,0.894118,0.788235}%
\pgfsetstrokecolor{currentstroke}%
\pgfsetdash{}{0pt}%
\pgfpathmoveto{\pgfqpoint{3.657159in}{2.829557in}}%
\pgfpathlineto{\pgfqpoint{3.546263in}{2.893583in}}%
\pgfpathlineto{\pgfqpoint{3.551407in}{2.904953in}}%
\pgfpathlineto{\pgfqpoint{3.662304in}{2.840927in}}%
\pgfpathlineto{\pgfqpoint{3.657159in}{2.829557in}}%
\pgfpathclose%
\pgfusepath{fill}%
\end{pgfscope}%
\begin{pgfscope}%
\pgfpathrectangle{\pgfqpoint{0.765000in}{0.660000in}}{\pgfqpoint{4.620000in}{4.620000in}}%
\pgfusepath{clip}%
\pgfsetbuttcap%
\pgfsetroundjoin%
\definecolor{currentfill}{rgb}{1.000000,0.894118,0.788235}%
\pgfsetfillcolor{currentfill}%
\pgfsetlinewidth{0.000000pt}%
\definecolor{currentstroke}{rgb}{1.000000,0.894118,0.788235}%
\pgfsetstrokecolor{currentstroke}%
\pgfsetdash{}{0pt}%
\pgfpathmoveto{\pgfqpoint{3.435366in}{2.701504in}}%
\pgfpathlineto{\pgfqpoint{3.435366in}{2.829557in}}%
\pgfpathlineto{\pgfqpoint{3.440511in}{2.840927in}}%
\pgfpathlineto{\pgfqpoint{3.551407in}{2.648848in}}%
\pgfpathlineto{\pgfqpoint{3.435366in}{2.701504in}}%
\pgfpathclose%
\pgfusepath{fill}%
\end{pgfscope}%
\begin{pgfscope}%
\pgfpathrectangle{\pgfqpoint{0.765000in}{0.660000in}}{\pgfqpoint{4.620000in}{4.620000in}}%
\pgfusepath{clip}%
\pgfsetbuttcap%
\pgfsetroundjoin%
\definecolor{currentfill}{rgb}{1.000000,0.894118,0.788235}%
\pgfsetfillcolor{currentfill}%
\pgfsetlinewidth{0.000000pt}%
\definecolor{currentstroke}{rgb}{1.000000,0.894118,0.788235}%
\pgfsetstrokecolor{currentstroke}%
\pgfsetdash{}{0pt}%
\pgfpathmoveto{\pgfqpoint{3.546263in}{2.637478in}}%
\pgfpathlineto{\pgfqpoint{3.546263in}{2.765531in}}%
\pgfpathlineto{\pgfqpoint{3.551407in}{2.776901in}}%
\pgfpathlineto{\pgfqpoint{3.440511in}{2.712874in}}%
\pgfpathlineto{\pgfqpoint{3.546263in}{2.637478in}}%
\pgfpathclose%
\pgfusepath{fill}%
\end{pgfscope}%
\begin{pgfscope}%
\pgfpathrectangle{\pgfqpoint{0.765000in}{0.660000in}}{\pgfqpoint{4.620000in}{4.620000in}}%
\pgfusepath{clip}%
\pgfsetbuttcap%
\pgfsetroundjoin%
\definecolor{currentfill}{rgb}{1.000000,0.894118,0.788235}%
\pgfsetfillcolor{currentfill}%
\pgfsetlinewidth{0.000000pt}%
\definecolor{currentstroke}{rgb}{1.000000,0.894118,0.788235}%
\pgfsetstrokecolor{currentstroke}%
\pgfsetdash{}{0pt}%
\pgfpathmoveto{\pgfqpoint{3.546263in}{2.765531in}}%
\pgfpathlineto{\pgfqpoint{3.657159in}{2.829557in}}%
\pgfpathlineto{\pgfqpoint{3.662304in}{2.840927in}}%
\pgfpathlineto{\pgfqpoint{3.551407in}{2.776901in}}%
\pgfpathlineto{\pgfqpoint{3.546263in}{2.765531in}}%
\pgfpathclose%
\pgfusepath{fill}%
\end{pgfscope}%
\begin{pgfscope}%
\pgfpathrectangle{\pgfqpoint{0.765000in}{0.660000in}}{\pgfqpoint{4.620000in}{4.620000in}}%
\pgfusepath{clip}%
\pgfsetbuttcap%
\pgfsetroundjoin%
\definecolor{currentfill}{rgb}{1.000000,0.894118,0.788235}%
\pgfsetfillcolor{currentfill}%
\pgfsetlinewidth{0.000000pt}%
\definecolor{currentstroke}{rgb}{1.000000,0.894118,0.788235}%
\pgfsetstrokecolor{currentstroke}%
\pgfsetdash{}{0pt}%
\pgfpathmoveto{\pgfqpoint{3.435366in}{2.829557in}}%
\pgfpathlineto{\pgfqpoint{3.546263in}{2.765531in}}%
\pgfpathlineto{\pgfqpoint{3.551407in}{2.776901in}}%
\pgfpathlineto{\pgfqpoint{3.440511in}{2.840927in}}%
\pgfpathlineto{\pgfqpoint{3.435366in}{2.829557in}}%
\pgfpathclose%
\pgfusepath{fill}%
\end{pgfscope}%
\begin{pgfscope}%
\pgfpathrectangle{\pgfqpoint{0.765000in}{0.660000in}}{\pgfqpoint{4.620000in}{4.620000in}}%
\pgfusepath{clip}%
\pgfsetbuttcap%
\pgfsetroundjoin%
\definecolor{currentfill}{rgb}{1.000000,0.894118,0.788235}%
\pgfsetfillcolor{currentfill}%
\pgfsetlinewidth{0.000000pt}%
\definecolor{currentstroke}{rgb}{1.000000,0.894118,0.788235}%
\pgfsetstrokecolor{currentstroke}%
\pgfsetdash{}{0pt}%
\pgfpathmoveto{\pgfqpoint{2.708745in}{3.356240in}}%
\pgfpathlineto{\pgfqpoint{2.597849in}{3.292213in}}%
\pgfpathlineto{\pgfqpoint{2.708745in}{3.228187in}}%
\pgfpathlineto{\pgfqpoint{2.819642in}{3.292213in}}%
\pgfpathlineto{\pgfqpoint{2.708745in}{3.356240in}}%
\pgfpathclose%
\pgfusepath{fill}%
\end{pgfscope}%
\begin{pgfscope}%
\pgfpathrectangle{\pgfqpoint{0.765000in}{0.660000in}}{\pgfqpoint{4.620000in}{4.620000in}}%
\pgfusepath{clip}%
\pgfsetbuttcap%
\pgfsetroundjoin%
\definecolor{currentfill}{rgb}{1.000000,0.894118,0.788235}%
\pgfsetfillcolor{currentfill}%
\pgfsetlinewidth{0.000000pt}%
\definecolor{currentstroke}{rgb}{1.000000,0.894118,0.788235}%
\pgfsetstrokecolor{currentstroke}%
\pgfsetdash{}{0pt}%
\pgfpathmoveto{\pgfqpoint{2.708745in}{3.356240in}}%
\pgfpathlineto{\pgfqpoint{2.597849in}{3.292213in}}%
\pgfpathlineto{\pgfqpoint{2.597849in}{3.420266in}}%
\pgfpathlineto{\pgfqpoint{2.708745in}{3.484292in}}%
\pgfpathlineto{\pgfqpoint{2.708745in}{3.356240in}}%
\pgfpathclose%
\pgfusepath{fill}%
\end{pgfscope}%
\begin{pgfscope}%
\pgfpathrectangle{\pgfqpoint{0.765000in}{0.660000in}}{\pgfqpoint{4.620000in}{4.620000in}}%
\pgfusepath{clip}%
\pgfsetbuttcap%
\pgfsetroundjoin%
\definecolor{currentfill}{rgb}{1.000000,0.894118,0.788235}%
\pgfsetfillcolor{currentfill}%
\pgfsetlinewidth{0.000000pt}%
\definecolor{currentstroke}{rgb}{1.000000,0.894118,0.788235}%
\pgfsetstrokecolor{currentstroke}%
\pgfsetdash{}{0pt}%
\pgfpathmoveto{\pgfqpoint{2.708745in}{3.356240in}}%
\pgfpathlineto{\pgfqpoint{2.819642in}{3.292213in}}%
\pgfpathlineto{\pgfqpoint{2.819642in}{3.420266in}}%
\pgfpathlineto{\pgfqpoint{2.708745in}{3.484292in}}%
\pgfpathlineto{\pgfqpoint{2.708745in}{3.356240in}}%
\pgfpathclose%
\pgfusepath{fill}%
\end{pgfscope}%
\begin{pgfscope}%
\pgfpathrectangle{\pgfqpoint{0.765000in}{0.660000in}}{\pgfqpoint{4.620000in}{4.620000in}}%
\pgfusepath{clip}%
\pgfsetbuttcap%
\pgfsetroundjoin%
\definecolor{currentfill}{rgb}{1.000000,0.894118,0.788235}%
\pgfsetfillcolor{currentfill}%
\pgfsetlinewidth{0.000000pt}%
\definecolor{currentstroke}{rgb}{1.000000,0.894118,0.788235}%
\pgfsetstrokecolor{currentstroke}%
\pgfsetdash{}{0pt}%
\pgfpathmoveto{\pgfqpoint{2.692300in}{3.344662in}}%
\pgfpathlineto{\pgfqpoint{2.581404in}{3.280636in}}%
\pgfpathlineto{\pgfqpoint{2.692300in}{3.216609in}}%
\pgfpathlineto{\pgfqpoint{2.803197in}{3.280636in}}%
\pgfpathlineto{\pgfqpoint{2.692300in}{3.344662in}}%
\pgfpathclose%
\pgfusepath{fill}%
\end{pgfscope}%
\begin{pgfscope}%
\pgfpathrectangle{\pgfqpoint{0.765000in}{0.660000in}}{\pgfqpoint{4.620000in}{4.620000in}}%
\pgfusepath{clip}%
\pgfsetbuttcap%
\pgfsetroundjoin%
\definecolor{currentfill}{rgb}{1.000000,0.894118,0.788235}%
\pgfsetfillcolor{currentfill}%
\pgfsetlinewidth{0.000000pt}%
\definecolor{currentstroke}{rgb}{1.000000,0.894118,0.788235}%
\pgfsetstrokecolor{currentstroke}%
\pgfsetdash{}{0pt}%
\pgfpathmoveto{\pgfqpoint{2.692300in}{3.344662in}}%
\pgfpathlineto{\pgfqpoint{2.581404in}{3.280636in}}%
\pgfpathlineto{\pgfqpoint{2.581404in}{3.408688in}}%
\pgfpathlineto{\pgfqpoint{2.692300in}{3.472714in}}%
\pgfpathlineto{\pgfqpoint{2.692300in}{3.344662in}}%
\pgfpathclose%
\pgfusepath{fill}%
\end{pgfscope}%
\begin{pgfscope}%
\pgfpathrectangle{\pgfqpoint{0.765000in}{0.660000in}}{\pgfqpoint{4.620000in}{4.620000in}}%
\pgfusepath{clip}%
\pgfsetbuttcap%
\pgfsetroundjoin%
\definecolor{currentfill}{rgb}{1.000000,0.894118,0.788235}%
\pgfsetfillcolor{currentfill}%
\pgfsetlinewidth{0.000000pt}%
\definecolor{currentstroke}{rgb}{1.000000,0.894118,0.788235}%
\pgfsetstrokecolor{currentstroke}%
\pgfsetdash{}{0pt}%
\pgfpathmoveto{\pgfqpoint{2.692300in}{3.344662in}}%
\pgfpathlineto{\pgfqpoint{2.803197in}{3.280636in}}%
\pgfpathlineto{\pgfqpoint{2.803197in}{3.408688in}}%
\pgfpathlineto{\pgfqpoint{2.692300in}{3.472714in}}%
\pgfpathlineto{\pgfqpoint{2.692300in}{3.344662in}}%
\pgfpathclose%
\pgfusepath{fill}%
\end{pgfscope}%
\begin{pgfscope}%
\pgfpathrectangle{\pgfqpoint{0.765000in}{0.660000in}}{\pgfqpoint{4.620000in}{4.620000in}}%
\pgfusepath{clip}%
\pgfsetbuttcap%
\pgfsetroundjoin%
\definecolor{currentfill}{rgb}{1.000000,0.894118,0.788235}%
\pgfsetfillcolor{currentfill}%
\pgfsetlinewidth{0.000000pt}%
\definecolor{currentstroke}{rgb}{1.000000,0.894118,0.788235}%
\pgfsetstrokecolor{currentstroke}%
\pgfsetdash{}{0pt}%
\pgfpathmoveto{\pgfqpoint{2.708745in}{3.484292in}}%
\pgfpathlineto{\pgfqpoint{2.597849in}{3.420266in}}%
\pgfpathlineto{\pgfqpoint{2.708745in}{3.356240in}}%
\pgfpathlineto{\pgfqpoint{2.819642in}{3.420266in}}%
\pgfpathlineto{\pgfqpoint{2.708745in}{3.484292in}}%
\pgfpathclose%
\pgfusepath{fill}%
\end{pgfscope}%
\begin{pgfscope}%
\pgfpathrectangle{\pgfqpoint{0.765000in}{0.660000in}}{\pgfqpoint{4.620000in}{4.620000in}}%
\pgfusepath{clip}%
\pgfsetbuttcap%
\pgfsetroundjoin%
\definecolor{currentfill}{rgb}{1.000000,0.894118,0.788235}%
\pgfsetfillcolor{currentfill}%
\pgfsetlinewidth{0.000000pt}%
\definecolor{currentstroke}{rgb}{1.000000,0.894118,0.788235}%
\pgfsetstrokecolor{currentstroke}%
\pgfsetdash{}{0pt}%
\pgfpathmoveto{\pgfqpoint{2.708745in}{3.228187in}}%
\pgfpathlineto{\pgfqpoint{2.819642in}{3.292213in}}%
\pgfpathlineto{\pgfqpoint{2.819642in}{3.420266in}}%
\pgfpathlineto{\pgfqpoint{2.708745in}{3.356240in}}%
\pgfpathlineto{\pgfqpoint{2.708745in}{3.228187in}}%
\pgfpathclose%
\pgfusepath{fill}%
\end{pgfscope}%
\begin{pgfscope}%
\pgfpathrectangle{\pgfqpoint{0.765000in}{0.660000in}}{\pgfqpoint{4.620000in}{4.620000in}}%
\pgfusepath{clip}%
\pgfsetbuttcap%
\pgfsetroundjoin%
\definecolor{currentfill}{rgb}{1.000000,0.894118,0.788235}%
\pgfsetfillcolor{currentfill}%
\pgfsetlinewidth{0.000000pt}%
\definecolor{currentstroke}{rgb}{1.000000,0.894118,0.788235}%
\pgfsetstrokecolor{currentstroke}%
\pgfsetdash{}{0pt}%
\pgfpathmoveto{\pgfqpoint{2.597849in}{3.292213in}}%
\pgfpathlineto{\pgfqpoint{2.708745in}{3.228187in}}%
\pgfpathlineto{\pgfqpoint{2.708745in}{3.356240in}}%
\pgfpathlineto{\pgfqpoint{2.597849in}{3.420266in}}%
\pgfpathlineto{\pgfqpoint{2.597849in}{3.292213in}}%
\pgfpathclose%
\pgfusepath{fill}%
\end{pgfscope}%
\begin{pgfscope}%
\pgfpathrectangle{\pgfqpoint{0.765000in}{0.660000in}}{\pgfqpoint{4.620000in}{4.620000in}}%
\pgfusepath{clip}%
\pgfsetbuttcap%
\pgfsetroundjoin%
\definecolor{currentfill}{rgb}{1.000000,0.894118,0.788235}%
\pgfsetfillcolor{currentfill}%
\pgfsetlinewidth{0.000000pt}%
\definecolor{currentstroke}{rgb}{1.000000,0.894118,0.788235}%
\pgfsetstrokecolor{currentstroke}%
\pgfsetdash{}{0pt}%
\pgfpathmoveto{\pgfqpoint{2.692300in}{3.472714in}}%
\pgfpathlineto{\pgfqpoint{2.581404in}{3.408688in}}%
\pgfpathlineto{\pgfqpoint{2.692300in}{3.344662in}}%
\pgfpathlineto{\pgfqpoint{2.803197in}{3.408688in}}%
\pgfpathlineto{\pgfqpoint{2.692300in}{3.472714in}}%
\pgfpathclose%
\pgfusepath{fill}%
\end{pgfscope}%
\begin{pgfscope}%
\pgfpathrectangle{\pgfqpoint{0.765000in}{0.660000in}}{\pgfqpoint{4.620000in}{4.620000in}}%
\pgfusepath{clip}%
\pgfsetbuttcap%
\pgfsetroundjoin%
\definecolor{currentfill}{rgb}{1.000000,0.894118,0.788235}%
\pgfsetfillcolor{currentfill}%
\pgfsetlinewidth{0.000000pt}%
\definecolor{currentstroke}{rgb}{1.000000,0.894118,0.788235}%
\pgfsetstrokecolor{currentstroke}%
\pgfsetdash{}{0pt}%
\pgfpathmoveto{\pgfqpoint{2.692300in}{3.216609in}}%
\pgfpathlineto{\pgfqpoint{2.803197in}{3.280636in}}%
\pgfpathlineto{\pgfqpoint{2.803197in}{3.408688in}}%
\pgfpathlineto{\pgfqpoint{2.692300in}{3.344662in}}%
\pgfpathlineto{\pgfqpoint{2.692300in}{3.216609in}}%
\pgfpathclose%
\pgfusepath{fill}%
\end{pgfscope}%
\begin{pgfscope}%
\pgfpathrectangle{\pgfqpoint{0.765000in}{0.660000in}}{\pgfqpoint{4.620000in}{4.620000in}}%
\pgfusepath{clip}%
\pgfsetbuttcap%
\pgfsetroundjoin%
\definecolor{currentfill}{rgb}{1.000000,0.894118,0.788235}%
\pgfsetfillcolor{currentfill}%
\pgfsetlinewidth{0.000000pt}%
\definecolor{currentstroke}{rgb}{1.000000,0.894118,0.788235}%
\pgfsetstrokecolor{currentstroke}%
\pgfsetdash{}{0pt}%
\pgfpathmoveto{\pgfqpoint{2.581404in}{3.280636in}}%
\pgfpathlineto{\pgfqpoint{2.692300in}{3.216609in}}%
\pgfpathlineto{\pgfqpoint{2.692300in}{3.344662in}}%
\pgfpathlineto{\pgfqpoint{2.581404in}{3.408688in}}%
\pgfpathlineto{\pgfqpoint{2.581404in}{3.280636in}}%
\pgfpathclose%
\pgfusepath{fill}%
\end{pgfscope}%
\begin{pgfscope}%
\pgfpathrectangle{\pgfqpoint{0.765000in}{0.660000in}}{\pgfqpoint{4.620000in}{4.620000in}}%
\pgfusepath{clip}%
\pgfsetbuttcap%
\pgfsetroundjoin%
\definecolor{currentfill}{rgb}{1.000000,0.894118,0.788235}%
\pgfsetfillcolor{currentfill}%
\pgfsetlinewidth{0.000000pt}%
\definecolor{currentstroke}{rgb}{1.000000,0.894118,0.788235}%
\pgfsetstrokecolor{currentstroke}%
\pgfsetdash{}{0pt}%
\pgfpathmoveto{\pgfqpoint{2.708745in}{3.356240in}}%
\pgfpathlineto{\pgfqpoint{2.597849in}{3.292213in}}%
\pgfpathlineto{\pgfqpoint{2.581404in}{3.280636in}}%
\pgfpathlineto{\pgfqpoint{2.692300in}{3.344662in}}%
\pgfpathlineto{\pgfqpoint{2.708745in}{3.356240in}}%
\pgfpathclose%
\pgfusepath{fill}%
\end{pgfscope}%
\begin{pgfscope}%
\pgfpathrectangle{\pgfqpoint{0.765000in}{0.660000in}}{\pgfqpoint{4.620000in}{4.620000in}}%
\pgfusepath{clip}%
\pgfsetbuttcap%
\pgfsetroundjoin%
\definecolor{currentfill}{rgb}{1.000000,0.894118,0.788235}%
\pgfsetfillcolor{currentfill}%
\pgfsetlinewidth{0.000000pt}%
\definecolor{currentstroke}{rgb}{1.000000,0.894118,0.788235}%
\pgfsetstrokecolor{currentstroke}%
\pgfsetdash{}{0pt}%
\pgfpathmoveto{\pgfqpoint{2.819642in}{3.292213in}}%
\pgfpathlineto{\pgfqpoint{2.708745in}{3.356240in}}%
\pgfpathlineto{\pgfqpoint{2.692300in}{3.344662in}}%
\pgfpathlineto{\pgfqpoint{2.803197in}{3.280636in}}%
\pgfpathlineto{\pgfqpoint{2.819642in}{3.292213in}}%
\pgfpathclose%
\pgfusepath{fill}%
\end{pgfscope}%
\begin{pgfscope}%
\pgfpathrectangle{\pgfqpoint{0.765000in}{0.660000in}}{\pgfqpoint{4.620000in}{4.620000in}}%
\pgfusepath{clip}%
\pgfsetbuttcap%
\pgfsetroundjoin%
\definecolor{currentfill}{rgb}{1.000000,0.894118,0.788235}%
\pgfsetfillcolor{currentfill}%
\pgfsetlinewidth{0.000000pt}%
\definecolor{currentstroke}{rgb}{1.000000,0.894118,0.788235}%
\pgfsetstrokecolor{currentstroke}%
\pgfsetdash{}{0pt}%
\pgfpathmoveto{\pgfqpoint{2.708745in}{3.356240in}}%
\pgfpathlineto{\pgfqpoint{2.708745in}{3.484292in}}%
\pgfpathlineto{\pgfqpoint{2.692300in}{3.472714in}}%
\pgfpathlineto{\pgfqpoint{2.803197in}{3.280636in}}%
\pgfpathlineto{\pgfqpoint{2.708745in}{3.356240in}}%
\pgfpathclose%
\pgfusepath{fill}%
\end{pgfscope}%
\begin{pgfscope}%
\pgfpathrectangle{\pgfqpoint{0.765000in}{0.660000in}}{\pgfqpoint{4.620000in}{4.620000in}}%
\pgfusepath{clip}%
\pgfsetbuttcap%
\pgfsetroundjoin%
\definecolor{currentfill}{rgb}{1.000000,0.894118,0.788235}%
\pgfsetfillcolor{currentfill}%
\pgfsetlinewidth{0.000000pt}%
\definecolor{currentstroke}{rgb}{1.000000,0.894118,0.788235}%
\pgfsetstrokecolor{currentstroke}%
\pgfsetdash{}{0pt}%
\pgfpathmoveto{\pgfqpoint{2.819642in}{3.292213in}}%
\pgfpathlineto{\pgfqpoint{2.819642in}{3.420266in}}%
\pgfpathlineto{\pgfqpoint{2.803197in}{3.408688in}}%
\pgfpathlineto{\pgfqpoint{2.692300in}{3.344662in}}%
\pgfpathlineto{\pgfqpoint{2.819642in}{3.292213in}}%
\pgfpathclose%
\pgfusepath{fill}%
\end{pgfscope}%
\begin{pgfscope}%
\pgfpathrectangle{\pgfqpoint{0.765000in}{0.660000in}}{\pgfqpoint{4.620000in}{4.620000in}}%
\pgfusepath{clip}%
\pgfsetbuttcap%
\pgfsetroundjoin%
\definecolor{currentfill}{rgb}{1.000000,0.894118,0.788235}%
\pgfsetfillcolor{currentfill}%
\pgfsetlinewidth{0.000000pt}%
\definecolor{currentstroke}{rgb}{1.000000,0.894118,0.788235}%
\pgfsetstrokecolor{currentstroke}%
\pgfsetdash{}{0pt}%
\pgfpathmoveto{\pgfqpoint{2.597849in}{3.292213in}}%
\pgfpathlineto{\pgfqpoint{2.708745in}{3.228187in}}%
\pgfpathlineto{\pgfqpoint{2.692300in}{3.216609in}}%
\pgfpathlineto{\pgfqpoint{2.581404in}{3.280636in}}%
\pgfpathlineto{\pgfqpoint{2.597849in}{3.292213in}}%
\pgfpathclose%
\pgfusepath{fill}%
\end{pgfscope}%
\begin{pgfscope}%
\pgfpathrectangle{\pgfqpoint{0.765000in}{0.660000in}}{\pgfqpoint{4.620000in}{4.620000in}}%
\pgfusepath{clip}%
\pgfsetbuttcap%
\pgfsetroundjoin%
\definecolor{currentfill}{rgb}{1.000000,0.894118,0.788235}%
\pgfsetfillcolor{currentfill}%
\pgfsetlinewidth{0.000000pt}%
\definecolor{currentstroke}{rgb}{1.000000,0.894118,0.788235}%
\pgfsetstrokecolor{currentstroke}%
\pgfsetdash{}{0pt}%
\pgfpathmoveto{\pgfqpoint{2.708745in}{3.228187in}}%
\pgfpathlineto{\pgfqpoint{2.819642in}{3.292213in}}%
\pgfpathlineto{\pgfqpoint{2.803197in}{3.280636in}}%
\pgfpathlineto{\pgfqpoint{2.692300in}{3.216609in}}%
\pgfpathlineto{\pgfqpoint{2.708745in}{3.228187in}}%
\pgfpathclose%
\pgfusepath{fill}%
\end{pgfscope}%
\begin{pgfscope}%
\pgfpathrectangle{\pgfqpoint{0.765000in}{0.660000in}}{\pgfqpoint{4.620000in}{4.620000in}}%
\pgfusepath{clip}%
\pgfsetbuttcap%
\pgfsetroundjoin%
\definecolor{currentfill}{rgb}{1.000000,0.894118,0.788235}%
\pgfsetfillcolor{currentfill}%
\pgfsetlinewidth{0.000000pt}%
\definecolor{currentstroke}{rgb}{1.000000,0.894118,0.788235}%
\pgfsetstrokecolor{currentstroke}%
\pgfsetdash{}{0pt}%
\pgfpathmoveto{\pgfqpoint{2.708745in}{3.484292in}}%
\pgfpathlineto{\pgfqpoint{2.597849in}{3.420266in}}%
\pgfpathlineto{\pgfqpoint{2.581404in}{3.408688in}}%
\pgfpathlineto{\pgfqpoint{2.692300in}{3.472714in}}%
\pgfpathlineto{\pgfqpoint{2.708745in}{3.484292in}}%
\pgfpathclose%
\pgfusepath{fill}%
\end{pgfscope}%
\begin{pgfscope}%
\pgfpathrectangle{\pgfqpoint{0.765000in}{0.660000in}}{\pgfqpoint{4.620000in}{4.620000in}}%
\pgfusepath{clip}%
\pgfsetbuttcap%
\pgfsetroundjoin%
\definecolor{currentfill}{rgb}{1.000000,0.894118,0.788235}%
\pgfsetfillcolor{currentfill}%
\pgfsetlinewidth{0.000000pt}%
\definecolor{currentstroke}{rgb}{1.000000,0.894118,0.788235}%
\pgfsetstrokecolor{currentstroke}%
\pgfsetdash{}{0pt}%
\pgfpathmoveto{\pgfqpoint{2.819642in}{3.420266in}}%
\pgfpathlineto{\pgfqpoint{2.708745in}{3.484292in}}%
\pgfpathlineto{\pgfqpoint{2.692300in}{3.472714in}}%
\pgfpathlineto{\pgfqpoint{2.803197in}{3.408688in}}%
\pgfpathlineto{\pgfqpoint{2.819642in}{3.420266in}}%
\pgfpathclose%
\pgfusepath{fill}%
\end{pgfscope}%
\begin{pgfscope}%
\pgfpathrectangle{\pgfqpoint{0.765000in}{0.660000in}}{\pgfqpoint{4.620000in}{4.620000in}}%
\pgfusepath{clip}%
\pgfsetbuttcap%
\pgfsetroundjoin%
\definecolor{currentfill}{rgb}{1.000000,0.894118,0.788235}%
\pgfsetfillcolor{currentfill}%
\pgfsetlinewidth{0.000000pt}%
\definecolor{currentstroke}{rgb}{1.000000,0.894118,0.788235}%
\pgfsetstrokecolor{currentstroke}%
\pgfsetdash{}{0pt}%
\pgfpathmoveto{\pgfqpoint{2.597849in}{3.292213in}}%
\pgfpathlineto{\pgfqpoint{2.597849in}{3.420266in}}%
\pgfpathlineto{\pgfqpoint{2.581404in}{3.408688in}}%
\pgfpathlineto{\pgfqpoint{2.692300in}{3.216609in}}%
\pgfpathlineto{\pgfqpoint{2.597849in}{3.292213in}}%
\pgfpathclose%
\pgfusepath{fill}%
\end{pgfscope}%
\begin{pgfscope}%
\pgfpathrectangle{\pgfqpoint{0.765000in}{0.660000in}}{\pgfqpoint{4.620000in}{4.620000in}}%
\pgfusepath{clip}%
\pgfsetbuttcap%
\pgfsetroundjoin%
\definecolor{currentfill}{rgb}{1.000000,0.894118,0.788235}%
\pgfsetfillcolor{currentfill}%
\pgfsetlinewidth{0.000000pt}%
\definecolor{currentstroke}{rgb}{1.000000,0.894118,0.788235}%
\pgfsetstrokecolor{currentstroke}%
\pgfsetdash{}{0pt}%
\pgfpathmoveto{\pgfqpoint{2.708745in}{3.228187in}}%
\pgfpathlineto{\pgfqpoint{2.708745in}{3.356240in}}%
\pgfpathlineto{\pgfqpoint{2.692300in}{3.344662in}}%
\pgfpathlineto{\pgfqpoint{2.581404in}{3.280636in}}%
\pgfpathlineto{\pgfqpoint{2.708745in}{3.228187in}}%
\pgfpathclose%
\pgfusepath{fill}%
\end{pgfscope}%
\begin{pgfscope}%
\pgfpathrectangle{\pgfqpoint{0.765000in}{0.660000in}}{\pgfqpoint{4.620000in}{4.620000in}}%
\pgfusepath{clip}%
\pgfsetbuttcap%
\pgfsetroundjoin%
\definecolor{currentfill}{rgb}{1.000000,0.894118,0.788235}%
\pgfsetfillcolor{currentfill}%
\pgfsetlinewidth{0.000000pt}%
\definecolor{currentstroke}{rgb}{1.000000,0.894118,0.788235}%
\pgfsetstrokecolor{currentstroke}%
\pgfsetdash{}{0pt}%
\pgfpathmoveto{\pgfqpoint{2.708745in}{3.356240in}}%
\pgfpathlineto{\pgfqpoint{2.819642in}{3.420266in}}%
\pgfpathlineto{\pgfqpoint{2.803197in}{3.408688in}}%
\pgfpathlineto{\pgfqpoint{2.692300in}{3.344662in}}%
\pgfpathlineto{\pgfqpoint{2.708745in}{3.356240in}}%
\pgfpathclose%
\pgfusepath{fill}%
\end{pgfscope}%
\begin{pgfscope}%
\pgfpathrectangle{\pgfqpoint{0.765000in}{0.660000in}}{\pgfqpoint{4.620000in}{4.620000in}}%
\pgfusepath{clip}%
\pgfsetbuttcap%
\pgfsetroundjoin%
\definecolor{currentfill}{rgb}{1.000000,0.894118,0.788235}%
\pgfsetfillcolor{currentfill}%
\pgfsetlinewidth{0.000000pt}%
\definecolor{currentstroke}{rgb}{1.000000,0.894118,0.788235}%
\pgfsetstrokecolor{currentstroke}%
\pgfsetdash{}{0pt}%
\pgfpathmoveto{\pgfqpoint{2.597849in}{3.420266in}}%
\pgfpathlineto{\pgfqpoint{2.708745in}{3.356240in}}%
\pgfpathlineto{\pgfqpoint{2.692300in}{3.344662in}}%
\pgfpathlineto{\pgfqpoint{2.581404in}{3.408688in}}%
\pgfpathlineto{\pgfqpoint{2.597849in}{3.420266in}}%
\pgfpathclose%
\pgfusepath{fill}%
\end{pgfscope}%
\begin{pgfscope}%
\pgfpathrectangle{\pgfqpoint{0.765000in}{0.660000in}}{\pgfqpoint{4.620000in}{4.620000in}}%
\pgfusepath{clip}%
\pgfsetbuttcap%
\pgfsetroundjoin%
\definecolor{currentfill}{rgb}{1.000000,0.894118,0.788235}%
\pgfsetfillcolor{currentfill}%
\pgfsetlinewidth{0.000000pt}%
\definecolor{currentstroke}{rgb}{1.000000,0.894118,0.788235}%
\pgfsetstrokecolor{currentstroke}%
\pgfsetdash{}{0pt}%
\pgfpathmoveto{\pgfqpoint{2.708745in}{3.356240in}}%
\pgfpathlineto{\pgfqpoint{2.597849in}{3.292213in}}%
\pgfpathlineto{\pgfqpoint{2.708745in}{3.228187in}}%
\pgfpathlineto{\pgfqpoint{2.819642in}{3.292213in}}%
\pgfpathlineto{\pgfqpoint{2.708745in}{3.356240in}}%
\pgfpathclose%
\pgfusepath{fill}%
\end{pgfscope}%
\begin{pgfscope}%
\pgfpathrectangle{\pgfqpoint{0.765000in}{0.660000in}}{\pgfqpoint{4.620000in}{4.620000in}}%
\pgfusepath{clip}%
\pgfsetbuttcap%
\pgfsetroundjoin%
\definecolor{currentfill}{rgb}{1.000000,0.894118,0.788235}%
\pgfsetfillcolor{currentfill}%
\pgfsetlinewidth{0.000000pt}%
\definecolor{currentstroke}{rgb}{1.000000,0.894118,0.788235}%
\pgfsetstrokecolor{currentstroke}%
\pgfsetdash{}{0pt}%
\pgfpathmoveto{\pgfqpoint{2.708745in}{3.356240in}}%
\pgfpathlineto{\pgfqpoint{2.597849in}{3.292213in}}%
\pgfpathlineto{\pgfqpoint{2.597849in}{3.420266in}}%
\pgfpathlineto{\pgfqpoint{2.708745in}{3.484292in}}%
\pgfpathlineto{\pgfqpoint{2.708745in}{3.356240in}}%
\pgfpathclose%
\pgfusepath{fill}%
\end{pgfscope}%
\begin{pgfscope}%
\pgfpathrectangle{\pgfqpoint{0.765000in}{0.660000in}}{\pgfqpoint{4.620000in}{4.620000in}}%
\pgfusepath{clip}%
\pgfsetbuttcap%
\pgfsetroundjoin%
\definecolor{currentfill}{rgb}{1.000000,0.894118,0.788235}%
\pgfsetfillcolor{currentfill}%
\pgfsetlinewidth{0.000000pt}%
\definecolor{currentstroke}{rgb}{1.000000,0.894118,0.788235}%
\pgfsetstrokecolor{currentstroke}%
\pgfsetdash{}{0pt}%
\pgfpathmoveto{\pgfqpoint{2.708745in}{3.356240in}}%
\pgfpathlineto{\pgfqpoint{2.819642in}{3.292213in}}%
\pgfpathlineto{\pgfqpoint{2.819642in}{3.420266in}}%
\pgfpathlineto{\pgfqpoint{2.708745in}{3.484292in}}%
\pgfpathlineto{\pgfqpoint{2.708745in}{3.356240in}}%
\pgfpathclose%
\pgfusepath{fill}%
\end{pgfscope}%
\begin{pgfscope}%
\pgfpathrectangle{\pgfqpoint{0.765000in}{0.660000in}}{\pgfqpoint{4.620000in}{4.620000in}}%
\pgfusepath{clip}%
\pgfsetbuttcap%
\pgfsetroundjoin%
\definecolor{currentfill}{rgb}{1.000000,0.894118,0.788235}%
\pgfsetfillcolor{currentfill}%
\pgfsetlinewidth{0.000000pt}%
\definecolor{currentstroke}{rgb}{1.000000,0.894118,0.788235}%
\pgfsetstrokecolor{currentstroke}%
\pgfsetdash{}{0pt}%
\pgfpathmoveto{\pgfqpoint{2.788056in}{3.406767in}}%
\pgfpathlineto{\pgfqpoint{2.677159in}{3.342741in}}%
\pgfpathlineto{\pgfqpoint{2.788056in}{3.278715in}}%
\pgfpathlineto{\pgfqpoint{2.898952in}{3.342741in}}%
\pgfpathlineto{\pgfqpoint{2.788056in}{3.406767in}}%
\pgfpathclose%
\pgfusepath{fill}%
\end{pgfscope}%
\begin{pgfscope}%
\pgfpathrectangle{\pgfqpoint{0.765000in}{0.660000in}}{\pgfqpoint{4.620000in}{4.620000in}}%
\pgfusepath{clip}%
\pgfsetbuttcap%
\pgfsetroundjoin%
\definecolor{currentfill}{rgb}{1.000000,0.894118,0.788235}%
\pgfsetfillcolor{currentfill}%
\pgfsetlinewidth{0.000000pt}%
\definecolor{currentstroke}{rgb}{1.000000,0.894118,0.788235}%
\pgfsetstrokecolor{currentstroke}%
\pgfsetdash{}{0pt}%
\pgfpathmoveto{\pgfqpoint{2.788056in}{3.406767in}}%
\pgfpathlineto{\pgfqpoint{2.677159in}{3.342741in}}%
\pgfpathlineto{\pgfqpoint{2.677159in}{3.470793in}}%
\pgfpathlineto{\pgfqpoint{2.788056in}{3.534820in}}%
\pgfpathlineto{\pgfqpoint{2.788056in}{3.406767in}}%
\pgfpathclose%
\pgfusepath{fill}%
\end{pgfscope}%
\begin{pgfscope}%
\pgfpathrectangle{\pgfqpoint{0.765000in}{0.660000in}}{\pgfqpoint{4.620000in}{4.620000in}}%
\pgfusepath{clip}%
\pgfsetbuttcap%
\pgfsetroundjoin%
\definecolor{currentfill}{rgb}{1.000000,0.894118,0.788235}%
\pgfsetfillcolor{currentfill}%
\pgfsetlinewidth{0.000000pt}%
\definecolor{currentstroke}{rgb}{1.000000,0.894118,0.788235}%
\pgfsetstrokecolor{currentstroke}%
\pgfsetdash{}{0pt}%
\pgfpathmoveto{\pgfqpoint{2.788056in}{3.406767in}}%
\pgfpathlineto{\pgfqpoint{2.898952in}{3.342741in}}%
\pgfpathlineto{\pgfqpoint{2.898952in}{3.470793in}}%
\pgfpathlineto{\pgfqpoint{2.788056in}{3.534820in}}%
\pgfpathlineto{\pgfqpoint{2.788056in}{3.406767in}}%
\pgfpathclose%
\pgfusepath{fill}%
\end{pgfscope}%
\begin{pgfscope}%
\pgfpathrectangle{\pgfqpoint{0.765000in}{0.660000in}}{\pgfqpoint{4.620000in}{4.620000in}}%
\pgfusepath{clip}%
\pgfsetbuttcap%
\pgfsetroundjoin%
\definecolor{currentfill}{rgb}{1.000000,0.894118,0.788235}%
\pgfsetfillcolor{currentfill}%
\pgfsetlinewidth{0.000000pt}%
\definecolor{currentstroke}{rgb}{1.000000,0.894118,0.788235}%
\pgfsetstrokecolor{currentstroke}%
\pgfsetdash{}{0pt}%
\pgfpathmoveto{\pgfqpoint{2.708745in}{3.484292in}}%
\pgfpathlineto{\pgfqpoint{2.597849in}{3.420266in}}%
\pgfpathlineto{\pgfqpoint{2.708745in}{3.356240in}}%
\pgfpathlineto{\pgfqpoint{2.819642in}{3.420266in}}%
\pgfpathlineto{\pgfqpoint{2.708745in}{3.484292in}}%
\pgfpathclose%
\pgfusepath{fill}%
\end{pgfscope}%
\begin{pgfscope}%
\pgfpathrectangle{\pgfqpoint{0.765000in}{0.660000in}}{\pgfqpoint{4.620000in}{4.620000in}}%
\pgfusepath{clip}%
\pgfsetbuttcap%
\pgfsetroundjoin%
\definecolor{currentfill}{rgb}{1.000000,0.894118,0.788235}%
\pgfsetfillcolor{currentfill}%
\pgfsetlinewidth{0.000000pt}%
\definecolor{currentstroke}{rgb}{1.000000,0.894118,0.788235}%
\pgfsetstrokecolor{currentstroke}%
\pgfsetdash{}{0pt}%
\pgfpathmoveto{\pgfqpoint{2.708745in}{3.228187in}}%
\pgfpathlineto{\pgfqpoint{2.819642in}{3.292213in}}%
\pgfpathlineto{\pgfqpoint{2.819642in}{3.420266in}}%
\pgfpathlineto{\pgfqpoint{2.708745in}{3.356240in}}%
\pgfpathlineto{\pgfqpoint{2.708745in}{3.228187in}}%
\pgfpathclose%
\pgfusepath{fill}%
\end{pgfscope}%
\begin{pgfscope}%
\pgfpathrectangle{\pgfqpoint{0.765000in}{0.660000in}}{\pgfqpoint{4.620000in}{4.620000in}}%
\pgfusepath{clip}%
\pgfsetbuttcap%
\pgfsetroundjoin%
\definecolor{currentfill}{rgb}{1.000000,0.894118,0.788235}%
\pgfsetfillcolor{currentfill}%
\pgfsetlinewidth{0.000000pt}%
\definecolor{currentstroke}{rgb}{1.000000,0.894118,0.788235}%
\pgfsetstrokecolor{currentstroke}%
\pgfsetdash{}{0pt}%
\pgfpathmoveto{\pgfqpoint{2.597849in}{3.292213in}}%
\pgfpathlineto{\pgfqpoint{2.708745in}{3.228187in}}%
\pgfpathlineto{\pgfqpoint{2.708745in}{3.356240in}}%
\pgfpathlineto{\pgfqpoint{2.597849in}{3.420266in}}%
\pgfpathlineto{\pgfqpoint{2.597849in}{3.292213in}}%
\pgfpathclose%
\pgfusepath{fill}%
\end{pgfscope}%
\begin{pgfscope}%
\pgfpathrectangle{\pgfqpoint{0.765000in}{0.660000in}}{\pgfqpoint{4.620000in}{4.620000in}}%
\pgfusepath{clip}%
\pgfsetbuttcap%
\pgfsetroundjoin%
\definecolor{currentfill}{rgb}{1.000000,0.894118,0.788235}%
\pgfsetfillcolor{currentfill}%
\pgfsetlinewidth{0.000000pt}%
\definecolor{currentstroke}{rgb}{1.000000,0.894118,0.788235}%
\pgfsetstrokecolor{currentstroke}%
\pgfsetdash{}{0pt}%
\pgfpathmoveto{\pgfqpoint{2.788056in}{3.534820in}}%
\pgfpathlineto{\pgfqpoint{2.677159in}{3.470793in}}%
\pgfpathlineto{\pgfqpoint{2.788056in}{3.406767in}}%
\pgfpathlineto{\pgfqpoint{2.898952in}{3.470793in}}%
\pgfpathlineto{\pgfqpoint{2.788056in}{3.534820in}}%
\pgfpathclose%
\pgfusepath{fill}%
\end{pgfscope}%
\begin{pgfscope}%
\pgfpathrectangle{\pgfqpoint{0.765000in}{0.660000in}}{\pgfqpoint{4.620000in}{4.620000in}}%
\pgfusepath{clip}%
\pgfsetbuttcap%
\pgfsetroundjoin%
\definecolor{currentfill}{rgb}{1.000000,0.894118,0.788235}%
\pgfsetfillcolor{currentfill}%
\pgfsetlinewidth{0.000000pt}%
\definecolor{currentstroke}{rgb}{1.000000,0.894118,0.788235}%
\pgfsetstrokecolor{currentstroke}%
\pgfsetdash{}{0pt}%
\pgfpathmoveto{\pgfqpoint{2.788056in}{3.278715in}}%
\pgfpathlineto{\pgfqpoint{2.898952in}{3.342741in}}%
\pgfpathlineto{\pgfqpoint{2.898952in}{3.470793in}}%
\pgfpathlineto{\pgfqpoint{2.788056in}{3.406767in}}%
\pgfpathlineto{\pgfqpoint{2.788056in}{3.278715in}}%
\pgfpathclose%
\pgfusepath{fill}%
\end{pgfscope}%
\begin{pgfscope}%
\pgfpathrectangle{\pgfqpoint{0.765000in}{0.660000in}}{\pgfqpoint{4.620000in}{4.620000in}}%
\pgfusepath{clip}%
\pgfsetbuttcap%
\pgfsetroundjoin%
\definecolor{currentfill}{rgb}{1.000000,0.894118,0.788235}%
\pgfsetfillcolor{currentfill}%
\pgfsetlinewidth{0.000000pt}%
\definecolor{currentstroke}{rgb}{1.000000,0.894118,0.788235}%
\pgfsetstrokecolor{currentstroke}%
\pgfsetdash{}{0pt}%
\pgfpathmoveto{\pgfqpoint{2.677159in}{3.342741in}}%
\pgfpathlineto{\pgfqpoint{2.788056in}{3.278715in}}%
\pgfpathlineto{\pgfqpoint{2.788056in}{3.406767in}}%
\pgfpathlineto{\pgfqpoint{2.677159in}{3.470793in}}%
\pgfpathlineto{\pgfqpoint{2.677159in}{3.342741in}}%
\pgfpathclose%
\pgfusepath{fill}%
\end{pgfscope}%
\begin{pgfscope}%
\pgfpathrectangle{\pgfqpoint{0.765000in}{0.660000in}}{\pgfqpoint{4.620000in}{4.620000in}}%
\pgfusepath{clip}%
\pgfsetbuttcap%
\pgfsetroundjoin%
\definecolor{currentfill}{rgb}{1.000000,0.894118,0.788235}%
\pgfsetfillcolor{currentfill}%
\pgfsetlinewidth{0.000000pt}%
\definecolor{currentstroke}{rgb}{1.000000,0.894118,0.788235}%
\pgfsetstrokecolor{currentstroke}%
\pgfsetdash{}{0pt}%
\pgfpathmoveto{\pgfqpoint{2.708745in}{3.356240in}}%
\pgfpathlineto{\pgfqpoint{2.597849in}{3.292213in}}%
\pgfpathlineto{\pgfqpoint{2.677159in}{3.342741in}}%
\pgfpathlineto{\pgfqpoint{2.788056in}{3.406767in}}%
\pgfpathlineto{\pgfqpoint{2.708745in}{3.356240in}}%
\pgfpathclose%
\pgfusepath{fill}%
\end{pgfscope}%
\begin{pgfscope}%
\pgfpathrectangle{\pgfqpoint{0.765000in}{0.660000in}}{\pgfqpoint{4.620000in}{4.620000in}}%
\pgfusepath{clip}%
\pgfsetbuttcap%
\pgfsetroundjoin%
\definecolor{currentfill}{rgb}{1.000000,0.894118,0.788235}%
\pgfsetfillcolor{currentfill}%
\pgfsetlinewidth{0.000000pt}%
\definecolor{currentstroke}{rgb}{1.000000,0.894118,0.788235}%
\pgfsetstrokecolor{currentstroke}%
\pgfsetdash{}{0pt}%
\pgfpathmoveto{\pgfqpoint{2.819642in}{3.292213in}}%
\pgfpathlineto{\pgfqpoint{2.708745in}{3.356240in}}%
\pgfpathlineto{\pgfqpoint{2.788056in}{3.406767in}}%
\pgfpathlineto{\pgfqpoint{2.898952in}{3.342741in}}%
\pgfpathlineto{\pgfqpoint{2.819642in}{3.292213in}}%
\pgfpathclose%
\pgfusepath{fill}%
\end{pgfscope}%
\begin{pgfscope}%
\pgfpathrectangle{\pgfqpoint{0.765000in}{0.660000in}}{\pgfqpoint{4.620000in}{4.620000in}}%
\pgfusepath{clip}%
\pgfsetbuttcap%
\pgfsetroundjoin%
\definecolor{currentfill}{rgb}{1.000000,0.894118,0.788235}%
\pgfsetfillcolor{currentfill}%
\pgfsetlinewidth{0.000000pt}%
\definecolor{currentstroke}{rgb}{1.000000,0.894118,0.788235}%
\pgfsetstrokecolor{currentstroke}%
\pgfsetdash{}{0pt}%
\pgfpathmoveto{\pgfqpoint{2.708745in}{3.356240in}}%
\pgfpathlineto{\pgfqpoint{2.708745in}{3.484292in}}%
\pgfpathlineto{\pgfqpoint{2.788056in}{3.534820in}}%
\pgfpathlineto{\pgfqpoint{2.898952in}{3.342741in}}%
\pgfpathlineto{\pgfqpoint{2.708745in}{3.356240in}}%
\pgfpathclose%
\pgfusepath{fill}%
\end{pgfscope}%
\begin{pgfscope}%
\pgfpathrectangle{\pgfqpoint{0.765000in}{0.660000in}}{\pgfqpoint{4.620000in}{4.620000in}}%
\pgfusepath{clip}%
\pgfsetbuttcap%
\pgfsetroundjoin%
\definecolor{currentfill}{rgb}{1.000000,0.894118,0.788235}%
\pgfsetfillcolor{currentfill}%
\pgfsetlinewidth{0.000000pt}%
\definecolor{currentstroke}{rgb}{1.000000,0.894118,0.788235}%
\pgfsetstrokecolor{currentstroke}%
\pgfsetdash{}{0pt}%
\pgfpathmoveto{\pgfqpoint{2.819642in}{3.292213in}}%
\pgfpathlineto{\pgfqpoint{2.819642in}{3.420266in}}%
\pgfpathlineto{\pgfqpoint{2.898952in}{3.470793in}}%
\pgfpathlineto{\pgfqpoint{2.788056in}{3.406767in}}%
\pgfpathlineto{\pgfqpoint{2.819642in}{3.292213in}}%
\pgfpathclose%
\pgfusepath{fill}%
\end{pgfscope}%
\begin{pgfscope}%
\pgfpathrectangle{\pgfqpoint{0.765000in}{0.660000in}}{\pgfqpoint{4.620000in}{4.620000in}}%
\pgfusepath{clip}%
\pgfsetbuttcap%
\pgfsetroundjoin%
\definecolor{currentfill}{rgb}{1.000000,0.894118,0.788235}%
\pgfsetfillcolor{currentfill}%
\pgfsetlinewidth{0.000000pt}%
\definecolor{currentstroke}{rgb}{1.000000,0.894118,0.788235}%
\pgfsetstrokecolor{currentstroke}%
\pgfsetdash{}{0pt}%
\pgfpathmoveto{\pgfqpoint{2.597849in}{3.292213in}}%
\pgfpathlineto{\pgfqpoint{2.708745in}{3.228187in}}%
\pgfpathlineto{\pgfqpoint{2.788056in}{3.278715in}}%
\pgfpathlineto{\pgfqpoint{2.677159in}{3.342741in}}%
\pgfpathlineto{\pgfqpoint{2.597849in}{3.292213in}}%
\pgfpathclose%
\pgfusepath{fill}%
\end{pgfscope}%
\begin{pgfscope}%
\pgfpathrectangle{\pgfqpoint{0.765000in}{0.660000in}}{\pgfqpoint{4.620000in}{4.620000in}}%
\pgfusepath{clip}%
\pgfsetbuttcap%
\pgfsetroundjoin%
\definecolor{currentfill}{rgb}{1.000000,0.894118,0.788235}%
\pgfsetfillcolor{currentfill}%
\pgfsetlinewidth{0.000000pt}%
\definecolor{currentstroke}{rgb}{1.000000,0.894118,0.788235}%
\pgfsetstrokecolor{currentstroke}%
\pgfsetdash{}{0pt}%
\pgfpathmoveto{\pgfqpoint{2.708745in}{3.228187in}}%
\pgfpathlineto{\pgfqpoint{2.819642in}{3.292213in}}%
\pgfpathlineto{\pgfqpoint{2.898952in}{3.342741in}}%
\pgfpathlineto{\pgfqpoint{2.788056in}{3.278715in}}%
\pgfpathlineto{\pgfqpoint{2.708745in}{3.228187in}}%
\pgfpathclose%
\pgfusepath{fill}%
\end{pgfscope}%
\begin{pgfscope}%
\pgfpathrectangle{\pgfqpoint{0.765000in}{0.660000in}}{\pgfqpoint{4.620000in}{4.620000in}}%
\pgfusepath{clip}%
\pgfsetbuttcap%
\pgfsetroundjoin%
\definecolor{currentfill}{rgb}{1.000000,0.894118,0.788235}%
\pgfsetfillcolor{currentfill}%
\pgfsetlinewidth{0.000000pt}%
\definecolor{currentstroke}{rgb}{1.000000,0.894118,0.788235}%
\pgfsetstrokecolor{currentstroke}%
\pgfsetdash{}{0pt}%
\pgfpathmoveto{\pgfqpoint{2.708745in}{3.484292in}}%
\pgfpathlineto{\pgfqpoint{2.597849in}{3.420266in}}%
\pgfpathlineto{\pgfqpoint{2.677159in}{3.470793in}}%
\pgfpathlineto{\pgfqpoint{2.788056in}{3.534820in}}%
\pgfpathlineto{\pgfqpoint{2.708745in}{3.484292in}}%
\pgfpathclose%
\pgfusepath{fill}%
\end{pgfscope}%
\begin{pgfscope}%
\pgfpathrectangle{\pgfqpoint{0.765000in}{0.660000in}}{\pgfqpoint{4.620000in}{4.620000in}}%
\pgfusepath{clip}%
\pgfsetbuttcap%
\pgfsetroundjoin%
\definecolor{currentfill}{rgb}{1.000000,0.894118,0.788235}%
\pgfsetfillcolor{currentfill}%
\pgfsetlinewidth{0.000000pt}%
\definecolor{currentstroke}{rgb}{1.000000,0.894118,0.788235}%
\pgfsetstrokecolor{currentstroke}%
\pgfsetdash{}{0pt}%
\pgfpathmoveto{\pgfqpoint{2.819642in}{3.420266in}}%
\pgfpathlineto{\pgfqpoint{2.708745in}{3.484292in}}%
\pgfpathlineto{\pgfqpoint{2.788056in}{3.534820in}}%
\pgfpathlineto{\pgfqpoint{2.898952in}{3.470793in}}%
\pgfpathlineto{\pgfqpoint{2.819642in}{3.420266in}}%
\pgfpathclose%
\pgfusepath{fill}%
\end{pgfscope}%
\begin{pgfscope}%
\pgfpathrectangle{\pgfqpoint{0.765000in}{0.660000in}}{\pgfqpoint{4.620000in}{4.620000in}}%
\pgfusepath{clip}%
\pgfsetbuttcap%
\pgfsetroundjoin%
\definecolor{currentfill}{rgb}{1.000000,0.894118,0.788235}%
\pgfsetfillcolor{currentfill}%
\pgfsetlinewidth{0.000000pt}%
\definecolor{currentstroke}{rgb}{1.000000,0.894118,0.788235}%
\pgfsetstrokecolor{currentstroke}%
\pgfsetdash{}{0pt}%
\pgfpathmoveto{\pgfqpoint{2.597849in}{3.292213in}}%
\pgfpathlineto{\pgfqpoint{2.597849in}{3.420266in}}%
\pgfpathlineto{\pgfqpoint{2.677159in}{3.470793in}}%
\pgfpathlineto{\pgfqpoint{2.788056in}{3.278715in}}%
\pgfpathlineto{\pgfqpoint{2.597849in}{3.292213in}}%
\pgfpathclose%
\pgfusepath{fill}%
\end{pgfscope}%
\begin{pgfscope}%
\pgfpathrectangle{\pgfqpoint{0.765000in}{0.660000in}}{\pgfqpoint{4.620000in}{4.620000in}}%
\pgfusepath{clip}%
\pgfsetbuttcap%
\pgfsetroundjoin%
\definecolor{currentfill}{rgb}{1.000000,0.894118,0.788235}%
\pgfsetfillcolor{currentfill}%
\pgfsetlinewidth{0.000000pt}%
\definecolor{currentstroke}{rgb}{1.000000,0.894118,0.788235}%
\pgfsetstrokecolor{currentstroke}%
\pgfsetdash{}{0pt}%
\pgfpathmoveto{\pgfqpoint{2.708745in}{3.228187in}}%
\pgfpathlineto{\pgfqpoint{2.708745in}{3.356240in}}%
\pgfpathlineto{\pgfqpoint{2.788056in}{3.406767in}}%
\pgfpathlineto{\pgfqpoint{2.677159in}{3.342741in}}%
\pgfpathlineto{\pgfqpoint{2.708745in}{3.228187in}}%
\pgfpathclose%
\pgfusepath{fill}%
\end{pgfscope}%
\begin{pgfscope}%
\pgfpathrectangle{\pgfqpoint{0.765000in}{0.660000in}}{\pgfqpoint{4.620000in}{4.620000in}}%
\pgfusepath{clip}%
\pgfsetbuttcap%
\pgfsetroundjoin%
\definecolor{currentfill}{rgb}{1.000000,0.894118,0.788235}%
\pgfsetfillcolor{currentfill}%
\pgfsetlinewidth{0.000000pt}%
\definecolor{currentstroke}{rgb}{1.000000,0.894118,0.788235}%
\pgfsetstrokecolor{currentstroke}%
\pgfsetdash{}{0pt}%
\pgfpathmoveto{\pgfqpoint{2.708745in}{3.356240in}}%
\pgfpathlineto{\pgfqpoint{2.819642in}{3.420266in}}%
\pgfpathlineto{\pgfqpoint{2.898952in}{3.470793in}}%
\pgfpathlineto{\pgfqpoint{2.788056in}{3.406767in}}%
\pgfpathlineto{\pgfqpoint{2.708745in}{3.356240in}}%
\pgfpathclose%
\pgfusepath{fill}%
\end{pgfscope}%
\begin{pgfscope}%
\pgfpathrectangle{\pgfqpoint{0.765000in}{0.660000in}}{\pgfqpoint{4.620000in}{4.620000in}}%
\pgfusepath{clip}%
\pgfsetbuttcap%
\pgfsetroundjoin%
\definecolor{currentfill}{rgb}{1.000000,0.894118,0.788235}%
\pgfsetfillcolor{currentfill}%
\pgfsetlinewidth{0.000000pt}%
\definecolor{currentstroke}{rgb}{1.000000,0.894118,0.788235}%
\pgfsetstrokecolor{currentstroke}%
\pgfsetdash{}{0pt}%
\pgfpathmoveto{\pgfqpoint{2.597849in}{3.420266in}}%
\pgfpathlineto{\pgfqpoint{2.708745in}{3.356240in}}%
\pgfpathlineto{\pgfqpoint{2.788056in}{3.406767in}}%
\pgfpathlineto{\pgfqpoint{2.677159in}{3.470793in}}%
\pgfpathlineto{\pgfqpoint{2.597849in}{3.420266in}}%
\pgfpathclose%
\pgfusepath{fill}%
\end{pgfscope}%
\begin{pgfscope}%
\pgfpathrectangle{\pgfqpoint{0.765000in}{0.660000in}}{\pgfqpoint{4.620000in}{4.620000in}}%
\pgfusepath{clip}%
\pgfsetbuttcap%
\pgfsetroundjoin%
\definecolor{currentfill}{rgb}{1.000000,0.894118,0.788235}%
\pgfsetfillcolor{currentfill}%
\pgfsetlinewidth{0.000000pt}%
\definecolor{currentstroke}{rgb}{1.000000,0.894118,0.788235}%
\pgfsetstrokecolor{currentstroke}%
\pgfsetdash{}{0pt}%
\pgfpathmoveto{\pgfqpoint{2.692300in}{3.344662in}}%
\pgfpathlineto{\pgfqpoint{2.581404in}{3.280636in}}%
\pgfpathlineto{\pgfqpoint{2.692300in}{3.216609in}}%
\pgfpathlineto{\pgfqpoint{2.803197in}{3.280636in}}%
\pgfpathlineto{\pgfqpoint{2.692300in}{3.344662in}}%
\pgfpathclose%
\pgfusepath{fill}%
\end{pgfscope}%
\begin{pgfscope}%
\pgfpathrectangle{\pgfqpoint{0.765000in}{0.660000in}}{\pgfqpoint{4.620000in}{4.620000in}}%
\pgfusepath{clip}%
\pgfsetbuttcap%
\pgfsetroundjoin%
\definecolor{currentfill}{rgb}{1.000000,0.894118,0.788235}%
\pgfsetfillcolor{currentfill}%
\pgfsetlinewidth{0.000000pt}%
\definecolor{currentstroke}{rgb}{1.000000,0.894118,0.788235}%
\pgfsetstrokecolor{currentstroke}%
\pgfsetdash{}{0pt}%
\pgfpathmoveto{\pgfqpoint{2.692300in}{3.344662in}}%
\pgfpathlineto{\pgfqpoint{2.581404in}{3.280636in}}%
\pgfpathlineto{\pgfqpoint{2.581404in}{3.408688in}}%
\pgfpathlineto{\pgfqpoint{2.692300in}{3.472714in}}%
\pgfpathlineto{\pgfqpoint{2.692300in}{3.344662in}}%
\pgfpathclose%
\pgfusepath{fill}%
\end{pgfscope}%
\begin{pgfscope}%
\pgfpathrectangle{\pgfqpoint{0.765000in}{0.660000in}}{\pgfqpoint{4.620000in}{4.620000in}}%
\pgfusepath{clip}%
\pgfsetbuttcap%
\pgfsetroundjoin%
\definecolor{currentfill}{rgb}{1.000000,0.894118,0.788235}%
\pgfsetfillcolor{currentfill}%
\pgfsetlinewidth{0.000000pt}%
\definecolor{currentstroke}{rgb}{1.000000,0.894118,0.788235}%
\pgfsetstrokecolor{currentstroke}%
\pgfsetdash{}{0pt}%
\pgfpathmoveto{\pgfqpoint{2.692300in}{3.344662in}}%
\pgfpathlineto{\pgfqpoint{2.803197in}{3.280636in}}%
\pgfpathlineto{\pgfqpoint{2.803197in}{3.408688in}}%
\pgfpathlineto{\pgfqpoint{2.692300in}{3.472714in}}%
\pgfpathlineto{\pgfqpoint{2.692300in}{3.344662in}}%
\pgfpathclose%
\pgfusepath{fill}%
\end{pgfscope}%
\begin{pgfscope}%
\pgfpathrectangle{\pgfqpoint{0.765000in}{0.660000in}}{\pgfqpoint{4.620000in}{4.620000in}}%
\pgfusepath{clip}%
\pgfsetbuttcap%
\pgfsetroundjoin%
\definecolor{currentfill}{rgb}{1.000000,0.894118,0.788235}%
\pgfsetfillcolor{currentfill}%
\pgfsetlinewidth{0.000000pt}%
\definecolor{currentstroke}{rgb}{1.000000,0.894118,0.788235}%
\pgfsetstrokecolor{currentstroke}%
\pgfsetdash{}{0pt}%
\pgfpathmoveto{\pgfqpoint{2.708745in}{3.356240in}}%
\pgfpathlineto{\pgfqpoint{2.597849in}{3.292213in}}%
\pgfpathlineto{\pgfqpoint{2.708745in}{3.228187in}}%
\pgfpathlineto{\pgfqpoint{2.819642in}{3.292213in}}%
\pgfpathlineto{\pgfqpoint{2.708745in}{3.356240in}}%
\pgfpathclose%
\pgfusepath{fill}%
\end{pgfscope}%
\begin{pgfscope}%
\pgfpathrectangle{\pgfqpoint{0.765000in}{0.660000in}}{\pgfqpoint{4.620000in}{4.620000in}}%
\pgfusepath{clip}%
\pgfsetbuttcap%
\pgfsetroundjoin%
\definecolor{currentfill}{rgb}{1.000000,0.894118,0.788235}%
\pgfsetfillcolor{currentfill}%
\pgfsetlinewidth{0.000000pt}%
\definecolor{currentstroke}{rgb}{1.000000,0.894118,0.788235}%
\pgfsetstrokecolor{currentstroke}%
\pgfsetdash{}{0pt}%
\pgfpathmoveto{\pgfqpoint{2.708745in}{3.356240in}}%
\pgfpathlineto{\pgfqpoint{2.597849in}{3.292213in}}%
\pgfpathlineto{\pgfqpoint{2.597849in}{3.420266in}}%
\pgfpathlineto{\pgfqpoint{2.708745in}{3.484292in}}%
\pgfpathlineto{\pgfqpoint{2.708745in}{3.356240in}}%
\pgfpathclose%
\pgfusepath{fill}%
\end{pgfscope}%
\begin{pgfscope}%
\pgfpathrectangle{\pgfqpoint{0.765000in}{0.660000in}}{\pgfqpoint{4.620000in}{4.620000in}}%
\pgfusepath{clip}%
\pgfsetbuttcap%
\pgfsetroundjoin%
\definecolor{currentfill}{rgb}{1.000000,0.894118,0.788235}%
\pgfsetfillcolor{currentfill}%
\pgfsetlinewidth{0.000000pt}%
\definecolor{currentstroke}{rgb}{1.000000,0.894118,0.788235}%
\pgfsetstrokecolor{currentstroke}%
\pgfsetdash{}{0pt}%
\pgfpathmoveto{\pgfqpoint{2.708745in}{3.356240in}}%
\pgfpathlineto{\pgfqpoint{2.819642in}{3.292213in}}%
\pgfpathlineto{\pgfqpoint{2.819642in}{3.420266in}}%
\pgfpathlineto{\pgfqpoint{2.708745in}{3.484292in}}%
\pgfpathlineto{\pgfqpoint{2.708745in}{3.356240in}}%
\pgfpathclose%
\pgfusepath{fill}%
\end{pgfscope}%
\begin{pgfscope}%
\pgfpathrectangle{\pgfqpoint{0.765000in}{0.660000in}}{\pgfqpoint{4.620000in}{4.620000in}}%
\pgfusepath{clip}%
\pgfsetbuttcap%
\pgfsetroundjoin%
\definecolor{currentfill}{rgb}{1.000000,0.894118,0.788235}%
\pgfsetfillcolor{currentfill}%
\pgfsetlinewidth{0.000000pt}%
\definecolor{currentstroke}{rgb}{1.000000,0.894118,0.788235}%
\pgfsetstrokecolor{currentstroke}%
\pgfsetdash{}{0pt}%
\pgfpathmoveto{\pgfqpoint{2.692300in}{3.472714in}}%
\pgfpathlineto{\pgfqpoint{2.581404in}{3.408688in}}%
\pgfpathlineto{\pgfqpoint{2.692300in}{3.344662in}}%
\pgfpathlineto{\pgfqpoint{2.803197in}{3.408688in}}%
\pgfpathlineto{\pgfqpoint{2.692300in}{3.472714in}}%
\pgfpathclose%
\pgfusepath{fill}%
\end{pgfscope}%
\begin{pgfscope}%
\pgfpathrectangle{\pgfqpoint{0.765000in}{0.660000in}}{\pgfqpoint{4.620000in}{4.620000in}}%
\pgfusepath{clip}%
\pgfsetbuttcap%
\pgfsetroundjoin%
\definecolor{currentfill}{rgb}{1.000000,0.894118,0.788235}%
\pgfsetfillcolor{currentfill}%
\pgfsetlinewidth{0.000000pt}%
\definecolor{currentstroke}{rgb}{1.000000,0.894118,0.788235}%
\pgfsetstrokecolor{currentstroke}%
\pgfsetdash{}{0pt}%
\pgfpathmoveto{\pgfqpoint{2.692300in}{3.216609in}}%
\pgfpathlineto{\pgfqpoint{2.803197in}{3.280636in}}%
\pgfpathlineto{\pgfqpoint{2.803197in}{3.408688in}}%
\pgfpathlineto{\pgfqpoint{2.692300in}{3.344662in}}%
\pgfpathlineto{\pgfqpoint{2.692300in}{3.216609in}}%
\pgfpathclose%
\pgfusepath{fill}%
\end{pgfscope}%
\begin{pgfscope}%
\pgfpathrectangle{\pgfqpoint{0.765000in}{0.660000in}}{\pgfqpoint{4.620000in}{4.620000in}}%
\pgfusepath{clip}%
\pgfsetbuttcap%
\pgfsetroundjoin%
\definecolor{currentfill}{rgb}{1.000000,0.894118,0.788235}%
\pgfsetfillcolor{currentfill}%
\pgfsetlinewidth{0.000000pt}%
\definecolor{currentstroke}{rgb}{1.000000,0.894118,0.788235}%
\pgfsetstrokecolor{currentstroke}%
\pgfsetdash{}{0pt}%
\pgfpathmoveto{\pgfqpoint{2.581404in}{3.280636in}}%
\pgfpathlineto{\pgfqpoint{2.692300in}{3.216609in}}%
\pgfpathlineto{\pgfqpoint{2.692300in}{3.344662in}}%
\pgfpathlineto{\pgfqpoint{2.581404in}{3.408688in}}%
\pgfpathlineto{\pgfqpoint{2.581404in}{3.280636in}}%
\pgfpathclose%
\pgfusepath{fill}%
\end{pgfscope}%
\begin{pgfscope}%
\pgfpathrectangle{\pgfqpoint{0.765000in}{0.660000in}}{\pgfqpoint{4.620000in}{4.620000in}}%
\pgfusepath{clip}%
\pgfsetbuttcap%
\pgfsetroundjoin%
\definecolor{currentfill}{rgb}{1.000000,0.894118,0.788235}%
\pgfsetfillcolor{currentfill}%
\pgfsetlinewidth{0.000000pt}%
\definecolor{currentstroke}{rgb}{1.000000,0.894118,0.788235}%
\pgfsetstrokecolor{currentstroke}%
\pgfsetdash{}{0pt}%
\pgfpathmoveto{\pgfqpoint{2.708745in}{3.484292in}}%
\pgfpathlineto{\pgfqpoint{2.597849in}{3.420266in}}%
\pgfpathlineto{\pgfqpoint{2.708745in}{3.356240in}}%
\pgfpathlineto{\pgfqpoint{2.819642in}{3.420266in}}%
\pgfpathlineto{\pgfqpoint{2.708745in}{3.484292in}}%
\pgfpathclose%
\pgfusepath{fill}%
\end{pgfscope}%
\begin{pgfscope}%
\pgfpathrectangle{\pgfqpoint{0.765000in}{0.660000in}}{\pgfqpoint{4.620000in}{4.620000in}}%
\pgfusepath{clip}%
\pgfsetbuttcap%
\pgfsetroundjoin%
\definecolor{currentfill}{rgb}{1.000000,0.894118,0.788235}%
\pgfsetfillcolor{currentfill}%
\pgfsetlinewidth{0.000000pt}%
\definecolor{currentstroke}{rgb}{1.000000,0.894118,0.788235}%
\pgfsetstrokecolor{currentstroke}%
\pgfsetdash{}{0pt}%
\pgfpathmoveto{\pgfqpoint{2.708745in}{3.228187in}}%
\pgfpathlineto{\pgfqpoint{2.819642in}{3.292213in}}%
\pgfpathlineto{\pgfqpoint{2.819642in}{3.420266in}}%
\pgfpathlineto{\pgfqpoint{2.708745in}{3.356240in}}%
\pgfpathlineto{\pgfqpoint{2.708745in}{3.228187in}}%
\pgfpathclose%
\pgfusepath{fill}%
\end{pgfscope}%
\begin{pgfscope}%
\pgfpathrectangle{\pgfqpoint{0.765000in}{0.660000in}}{\pgfqpoint{4.620000in}{4.620000in}}%
\pgfusepath{clip}%
\pgfsetbuttcap%
\pgfsetroundjoin%
\definecolor{currentfill}{rgb}{1.000000,0.894118,0.788235}%
\pgfsetfillcolor{currentfill}%
\pgfsetlinewidth{0.000000pt}%
\definecolor{currentstroke}{rgb}{1.000000,0.894118,0.788235}%
\pgfsetstrokecolor{currentstroke}%
\pgfsetdash{}{0pt}%
\pgfpathmoveto{\pgfqpoint{2.597849in}{3.292213in}}%
\pgfpathlineto{\pgfqpoint{2.708745in}{3.228187in}}%
\pgfpathlineto{\pgfqpoint{2.708745in}{3.356240in}}%
\pgfpathlineto{\pgfqpoint{2.597849in}{3.420266in}}%
\pgfpathlineto{\pgfqpoint{2.597849in}{3.292213in}}%
\pgfpathclose%
\pgfusepath{fill}%
\end{pgfscope}%
\begin{pgfscope}%
\pgfpathrectangle{\pgfqpoint{0.765000in}{0.660000in}}{\pgfqpoint{4.620000in}{4.620000in}}%
\pgfusepath{clip}%
\pgfsetbuttcap%
\pgfsetroundjoin%
\definecolor{currentfill}{rgb}{1.000000,0.894118,0.788235}%
\pgfsetfillcolor{currentfill}%
\pgfsetlinewidth{0.000000pt}%
\definecolor{currentstroke}{rgb}{1.000000,0.894118,0.788235}%
\pgfsetstrokecolor{currentstroke}%
\pgfsetdash{}{0pt}%
\pgfpathmoveto{\pgfqpoint{2.692300in}{3.344662in}}%
\pgfpathlineto{\pgfqpoint{2.581404in}{3.280636in}}%
\pgfpathlineto{\pgfqpoint{2.597849in}{3.292213in}}%
\pgfpathlineto{\pgfqpoint{2.708745in}{3.356240in}}%
\pgfpathlineto{\pgfqpoint{2.692300in}{3.344662in}}%
\pgfpathclose%
\pgfusepath{fill}%
\end{pgfscope}%
\begin{pgfscope}%
\pgfpathrectangle{\pgfqpoint{0.765000in}{0.660000in}}{\pgfqpoint{4.620000in}{4.620000in}}%
\pgfusepath{clip}%
\pgfsetbuttcap%
\pgfsetroundjoin%
\definecolor{currentfill}{rgb}{1.000000,0.894118,0.788235}%
\pgfsetfillcolor{currentfill}%
\pgfsetlinewidth{0.000000pt}%
\definecolor{currentstroke}{rgb}{1.000000,0.894118,0.788235}%
\pgfsetstrokecolor{currentstroke}%
\pgfsetdash{}{0pt}%
\pgfpathmoveto{\pgfqpoint{2.803197in}{3.280636in}}%
\pgfpathlineto{\pgfqpoint{2.692300in}{3.344662in}}%
\pgfpathlineto{\pgfqpoint{2.708745in}{3.356240in}}%
\pgfpathlineto{\pgfqpoint{2.819642in}{3.292213in}}%
\pgfpathlineto{\pgfqpoint{2.803197in}{3.280636in}}%
\pgfpathclose%
\pgfusepath{fill}%
\end{pgfscope}%
\begin{pgfscope}%
\pgfpathrectangle{\pgfqpoint{0.765000in}{0.660000in}}{\pgfqpoint{4.620000in}{4.620000in}}%
\pgfusepath{clip}%
\pgfsetbuttcap%
\pgfsetroundjoin%
\definecolor{currentfill}{rgb}{1.000000,0.894118,0.788235}%
\pgfsetfillcolor{currentfill}%
\pgfsetlinewidth{0.000000pt}%
\definecolor{currentstroke}{rgb}{1.000000,0.894118,0.788235}%
\pgfsetstrokecolor{currentstroke}%
\pgfsetdash{}{0pt}%
\pgfpathmoveto{\pgfqpoint{2.692300in}{3.344662in}}%
\pgfpathlineto{\pgfqpoint{2.692300in}{3.472714in}}%
\pgfpathlineto{\pgfqpoint{2.708745in}{3.484292in}}%
\pgfpathlineto{\pgfqpoint{2.819642in}{3.292213in}}%
\pgfpathlineto{\pgfqpoint{2.692300in}{3.344662in}}%
\pgfpathclose%
\pgfusepath{fill}%
\end{pgfscope}%
\begin{pgfscope}%
\pgfpathrectangle{\pgfqpoint{0.765000in}{0.660000in}}{\pgfqpoint{4.620000in}{4.620000in}}%
\pgfusepath{clip}%
\pgfsetbuttcap%
\pgfsetroundjoin%
\definecolor{currentfill}{rgb}{1.000000,0.894118,0.788235}%
\pgfsetfillcolor{currentfill}%
\pgfsetlinewidth{0.000000pt}%
\definecolor{currentstroke}{rgb}{1.000000,0.894118,0.788235}%
\pgfsetstrokecolor{currentstroke}%
\pgfsetdash{}{0pt}%
\pgfpathmoveto{\pgfqpoint{2.803197in}{3.280636in}}%
\pgfpathlineto{\pgfqpoint{2.803197in}{3.408688in}}%
\pgfpathlineto{\pgfqpoint{2.819642in}{3.420266in}}%
\pgfpathlineto{\pgfqpoint{2.708745in}{3.356240in}}%
\pgfpathlineto{\pgfqpoint{2.803197in}{3.280636in}}%
\pgfpathclose%
\pgfusepath{fill}%
\end{pgfscope}%
\begin{pgfscope}%
\pgfpathrectangle{\pgfqpoint{0.765000in}{0.660000in}}{\pgfqpoint{4.620000in}{4.620000in}}%
\pgfusepath{clip}%
\pgfsetbuttcap%
\pgfsetroundjoin%
\definecolor{currentfill}{rgb}{1.000000,0.894118,0.788235}%
\pgfsetfillcolor{currentfill}%
\pgfsetlinewidth{0.000000pt}%
\definecolor{currentstroke}{rgb}{1.000000,0.894118,0.788235}%
\pgfsetstrokecolor{currentstroke}%
\pgfsetdash{}{0pt}%
\pgfpathmoveto{\pgfqpoint{2.581404in}{3.280636in}}%
\pgfpathlineto{\pgfqpoint{2.692300in}{3.216609in}}%
\pgfpathlineto{\pgfqpoint{2.708745in}{3.228187in}}%
\pgfpathlineto{\pgfqpoint{2.597849in}{3.292213in}}%
\pgfpathlineto{\pgfqpoint{2.581404in}{3.280636in}}%
\pgfpathclose%
\pgfusepath{fill}%
\end{pgfscope}%
\begin{pgfscope}%
\pgfpathrectangle{\pgfqpoint{0.765000in}{0.660000in}}{\pgfqpoint{4.620000in}{4.620000in}}%
\pgfusepath{clip}%
\pgfsetbuttcap%
\pgfsetroundjoin%
\definecolor{currentfill}{rgb}{1.000000,0.894118,0.788235}%
\pgfsetfillcolor{currentfill}%
\pgfsetlinewidth{0.000000pt}%
\definecolor{currentstroke}{rgb}{1.000000,0.894118,0.788235}%
\pgfsetstrokecolor{currentstroke}%
\pgfsetdash{}{0pt}%
\pgfpathmoveto{\pgfqpoint{2.692300in}{3.216609in}}%
\pgfpathlineto{\pgfqpoint{2.803197in}{3.280636in}}%
\pgfpathlineto{\pgfqpoint{2.819642in}{3.292213in}}%
\pgfpathlineto{\pgfqpoint{2.708745in}{3.228187in}}%
\pgfpathlineto{\pgfqpoint{2.692300in}{3.216609in}}%
\pgfpathclose%
\pgfusepath{fill}%
\end{pgfscope}%
\begin{pgfscope}%
\pgfpathrectangle{\pgfqpoint{0.765000in}{0.660000in}}{\pgfqpoint{4.620000in}{4.620000in}}%
\pgfusepath{clip}%
\pgfsetbuttcap%
\pgfsetroundjoin%
\definecolor{currentfill}{rgb}{1.000000,0.894118,0.788235}%
\pgfsetfillcolor{currentfill}%
\pgfsetlinewidth{0.000000pt}%
\definecolor{currentstroke}{rgb}{1.000000,0.894118,0.788235}%
\pgfsetstrokecolor{currentstroke}%
\pgfsetdash{}{0pt}%
\pgfpathmoveto{\pgfqpoint{2.692300in}{3.472714in}}%
\pgfpathlineto{\pgfqpoint{2.581404in}{3.408688in}}%
\pgfpathlineto{\pgfqpoint{2.597849in}{3.420266in}}%
\pgfpathlineto{\pgfqpoint{2.708745in}{3.484292in}}%
\pgfpathlineto{\pgfqpoint{2.692300in}{3.472714in}}%
\pgfpathclose%
\pgfusepath{fill}%
\end{pgfscope}%
\begin{pgfscope}%
\pgfpathrectangle{\pgfqpoint{0.765000in}{0.660000in}}{\pgfqpoint{4.620000in}{4.620000in}}%
\pgfusepath{clip}%
\pgfsetbuttcap%
\pgfsetroundjoin%
\definecolor{currentfill}{rgb}{1.000000,0.894118,0.788235}%
\pgfsetfillcolor{currentfill}%
\pgfsetlinewidth{0.000000pt}%
\definecolor{currentstroke}{rgb}{1.000000,0.894118,0.788235}%
\pgfsetstrokecolor{currentstroke}%
\pgfsetdash{}{0pt}%
\pgfpathmoveto{\pgfqpoint{2.803197in}{3.408688in}}%
\pgfpathlineto{\pgfqpoint{2.692300in}{3.472714in}}%
\pgfpathlineto{\pgfqpoint{2.708745in}{3.484292in}}%
\pgfpathlineto{\pgfqpoint{2.819642in}{3.420266in}}%
\pgfpathlineto{\pgfqpoint{2.803197in}{3.408688in}}%
\pgfpathclose%
\pgfusepath{fill}%
\end{pgfscope}%
\begin{pgfscope}%
\pgfpathrectangle{\pgfqpoint{0.765000in}{0.660000in}}{\pgfqpoint{4.620000in}{4.620000in}}%
\pgfusepath{clip}%
\pgfsetbuttcap%
\pgfsetroundjoin%
\definecolor{currentfill}{rgb}{1.000000,0.894118,0.788235}%
\pgfsetfillcolor{currentfill}%
\pgfsetlinewidth{0.000000pt}%
\definecolor{currentstroke}{rgb}{1.000000,0.894118,0.788235}%
\pgfsetstrokecolor{currentstroke}%
\pgfsetdash{}{0pt}%
\pgfpathmoveto{\pgfqpoint{2.581404in}{3.280636in}}%
\pgfpathlineto{\pgfqpoint{2.581404in}{3.408688in}}%
\pgfpathlineto{\pgfqpoint{2.597849in}{3.420266in}}%
\pgfpathlineto{\pgfqpoint{2.708745in}{3.228187in}}%
\pgfpathlineto{\pgfqpoint{2.581404in}{3.280636in}}%
\pgfpathclose%
\pgfusepath{fill}%
\end{pgfscope}%
\begin{pgfscope}%
\pgfpathrectangle{\pgfqpoint{0.765000in}{0.660000in}}{\pgfqpoint{4.620000in}{4.620000in}}%
\pgfusepath{clip}%
\pgfsetbuttcap%
\pgfsetroundjoin%
\definecolor{currentfill}{rgb}{1.000000,0.894118,0.788235}%
\pgfsetfillcolor{currentfill}%
\pgfsetlinewidth{0.000000pt}%
\definecolor{currentstroke}{rgb}{1.000000,0.894118,0.788235}%
\pgfsetstrokecolor{currentstroke}%
\pgfsetdash{}{0pt}%
\pgfpathmoveto{\pgfqpoint{2.692300in}{3.216609in}}%
\pgfpathlineto{\pgfqpoint{2.692300in}{3.344662in}}%
\pgfpathlineto{\pgfqpoint{2.708745in}{3.356240in}}%
\pgfpathlineto{\pgfqpoint{2.597849in}{3.292213in}}%
\pgfpathlineto{\pgfqpoint{2.692300in}{3.216609in}}%
\pgfpathclose%
\pgfusepath{fill}%
\end{pgfscope}%
\begin{pgfscope}%
\pgfpathrectangle{\pgfqpoint{0.765000in}{0.660000in}}{\pgfqpoint{4.620000in}{4.620000in}}%
\pgfusepath{clip}%
\pgfsetbuttcap%
\pgfsetroundjoin%
\definecolor{currentfill}{rgb}{1.000000,0.894118,0.788235}%
\pgfsetfillcolor{currentfill}%
\pgfsetlinewidth{0.000000pt}%
\definecolor{currentstroke}{rgb}{1.000000,0.894118,0.788235}%
\pgfsetstrokecolor{currentstroke}%
\pgfsetdash{}{0pt}%
\pgfpathmoveto{\pgfqpoint{2.692300in}{3.344662in}}%
\pgfpathlineto{\pgfqpoint{2.803197in}{3.408688in}}%
\pgfpathlineto{\pgfqpoint{2.819642in}{3.420266in}}%
\pgfpathlineto{\pgfqpoint{2.708745in}{3.356240in}}%
\pgfpathlineto{\pgfqpoint{2.692300in}{3.344662in}}%
\pgfpathclose%
\pgfusepath{fill}%
\end{pgfscope}%
\begin{pgfscope}%
\pgfpathrectangle{\pgfqpoint{0.765000in}{0.660000in}}{\pgfqpoint{4.620000in}{4.620000in}}%
\pgfusepath{clip}%
\pgfsetbuttcap%
\pgfsetroundjoin%
\definecolor{currentfill}{rgb}{1.000000,0.894118,0.788235}%
\pgfsetfillcolor{currentfill}%
\pgfsetlinewidth{0.000000pt}%
\definecolor{currentstroke}{rgb}{1.000000,0.894118,0.788235}%
\pgfsetstrokecolor{currentstroke}%
\pgfsetdash{}{0pt}%
\pgfpathmoveto{\pgfqpoint{2.581404in}{3.408688in}}%
\pgfpathlineto{\pgfqpoint{2.692300in}{3.344662in}}%
\pgfpathlineto{\pgfqpoint{2.708745in}{3.356240in}}%
\pgfpathlineto{\pgfqpoint{2.597849in}{3.420266in}}%
\pgfpathlineto{\pgfqpoint{2.581404in}{3.408688in}}%
\pgfpathclose%
\pgfusepath{fill}%
\end{pgfscope}%
\begin{pgfscope}%
\pgfpathrectangle{\pgfqpoint{0.765000in}{0.660000in}}{\pgfqpoint{4.620000in}{4.620000in}}%
\pgfusepath{clip}%
\pgfsetbuttcap%
\pgfsetroundjoin%
\definecolor{currentfill}{rgb}{1.000000,0.894118,0.788235}%
\pgfsetfillcolor{currentfill}%
\pgfsetlinewidth{0.000000pt}%
\definecolor{currentstroke}{rgb}{1.000000,0.894118,0.788235}%
\pgfsetstrokecolor{currentstroke}%
\pgfsetdash{}{0pt}%
\pgfpathmoveto{\pgfqpoint{2.692300in}{3.344662in}}%
\pgfpathlineto{\pgfqpoint{2.581404in}{3.280636in}}%
\pgfpathlineto{\pgfqpoint{2.692300in}{3.216609in}}%
\pgfpathlineto{\pgfqpoint{2.803197in}{3.280636in}}%
\pgfpathlineto{\pgfqpoint{2.692300in}{3.344662in}}%
\pgfpathclose%
\pgfusepath{fill}%
\end{pgfscope}%
\begin{pgfscope}%
\pgfpathrectangle{\pgfqpoint{0.765000in}{0.660000in}}{\pgfqpoint{4.620000in}{4.620000in}}%
\pgfusepath{clip}%
\pgfsetbuttcap%
\pgfsetroundjoin%
\definecolor{currentfill}{rgb}{1.000000,0.894118,0.788235}%
\pgfsetfillcolor{currentfill}%
\pgfsetlinewidth{0.000000pt}%
\definecolor{currentstroke}{rgb}{1.000000,0.894118,0.788235}%
\pgfsetstrokecolor{currentstroke}%
\pgfsetdash{}{0pt}%
\pgfpathmoveto{\pgfqpoint{2.692300in}{3.344662in}}%
\pgfpathlineto{\pgfqpoint{2.581404in}{3.280636in}}%
\pgfpathlineto{\pgfqpoint{2.581404in}{3.408688in}}%
\pgfpathlineto{\pgfqpoint{2.692300in}{3.472714in}}%
\pgfpathlineto{\pgfqpoint{2.692300in}{3.344662in}}%
\pgfpathclose%
\pgfusepath{fill}%
\end{pgfscope}%
\begin{pgfscope}%
\pgfpathrectangle{\pgfqpoint{0.765000in}{0.660000in}}{\pgfqpoint{4.620000in}{4.620000in}}%
\pgfusepath{clip}%
\pgfsetbuttcap%
\pgfsetroundjoin%
\definecolor{currentfill}{rgb}{1.000000,0.894118,0.788235}%
\pgfsetfillcolor{currentfill}%
\pgfsetlinewidth{0.000000pt}%
\definecolor{currentstroke}{rgb}{1.000000,0.894118,0.788235}%
\pgfsetstrokecolor{currentstroke}%
\pgfsetdash{}{0pt}%
\pgfpathmoveto{\pgfqpoint{2.692300in}{3.344662in}}%
\pgfpathlineto{\pgfqpoint{2.803197in}{3.280636in}}%
\pgfpathlineto{\pgfqpoint{2.803197in}{3.408688in}}%
\pgfpathlineto{\pgfqpoint{2.692300in}{3.472714in}}%
\pgfpathlineto{\pgfqpoint{2.692300in}{3.344662in}}%
\pgfpathclose%
\pgfusepath{fill}%
\end{pgfscope}%
\begin{pgfscope}%
\pgfpathrectangle{\pgfqpoint{0.765000in}{0.660000in}}{\pgfqpoint{4.620000in}{4.620000in}}%
\pgfusepath{clip}%
\pgfsetbuttcap%
\pgfsetroundjoin%
\definecolor{currentfill}{rgb}{1.000000,0.894118,0.788235}%
\pgfsetfillcolor{currentfill}%
\pgfsetlinewidth{0.000000pt}%
\definecolor{currentstroke}{rgb}{1.000000,0.894118,0.788235}%
\pgfsetstrokecolor{currentstroke}%
\pgfsetdash{}{0pt}%
\pgfpathmoveto{\pgfqpoint{2.671300in}{3.327008in}}%
\pgfpathlineto{\pgfqpoint{2.560403in}{3.262981in}}%
\pgfpathlineto{\pgfqpoint{2.671300in}{3.198955in}}%
\pgfpathlineto{\pgfqpoint{2.782196in}{3.262981in}}%
\pgfpathlineto{\pgfqpoint{2.671300in}{3.327008in}}%
\pgfpathclose%
\pgfusepath{fill}%
\end{pgfscope}%
\begin{pgfscope}%
\pgfpathrectangle{\pgfqpoint{0.765000in}{0.660000in}}{\pgfqpoint{4.620000in}{4.620000in}}%
\pgfusepath{clip}%
\pgfsetbuttcap%
\pgfsetroundjoin%
\definecolor{currentfill}{rgb}{1.000000,0.894118,0.788235}%
\pgfsetfillcolor{currentfill}%
\pgfsetlinewidth{0.000000pt}%
\definecolor{currentstroke}{rgb}{1.000000,0.894118,0.788235}%
\pgfsetstrokecolor{currentstroke}%
\pgfsetdash{}{0pt}%
\pgfpathmoveto{\pgfqpoint{2.671300in}{3.327008in}}%
\pgfpathlineto{\pgfqpoint{2.560403in}{3.262981in}}%
\pgfpathlineto{\pgfqpoint{2.560403in}{3.391034in}}%
\pgfpathlineto{\pgfqpoint{2.671300in}{3.455060in}}%
\pgfpathlineto{\pgfqpoint{2.671300in}{3.327008in}}%
\pgfpathclose%
\pgfusepath{fill}%
\end{pgfscope}%
\begin{pgfscope}%
\pgfpathrectangle{\pgfqpoint{0.765000in}{0.660000in}}{\pgfqpoint{4.620000in}{4.620000in}}%
\pgfusepath{clip}%
\pgfsetbuttcap%
\pgfsetroundjoin%
\definecolor{currentfill}{rgb}{1.000000,0.894118,0.788235}%
\pgfsetfillcolor{currentfill}%
\pgfsetlinewidth{0.000000pt}%
\definecolor{currentstroke}{rgb}{1.000000,0.894118,0.788235}%
\pgfsetstrokecolor{currentstroke}%
\pgfsetdash{}{0pt}%
\pgfpathmoveto{\pgfqpoint{2.671300in}{3.327008in}}%
\pgfpathlineto{\pgfqpoint{2.782196in}{3.262981in}}%
\pgfpathlineto{\pgfqpoint{2.782196in}{3.391034in}}%
\pgfpathlineto{\pgfqpoint{2.671300in}{3.455060in}}%
\pgfpathlineto{\pgfqpoint{2.671300in}{3.327008in}}%
\pgfpathclose%
\pgfusepath{fill}%
\end{pgfscope}%
\begin{pgfscope}%
\pgfpathrectangle{\pgfqpoint{0.765000in}{0.660000in}}{\pgfqpoint{4.620000in}{4.620000in}}%
\pgfusepath{clip}%
\pgfsetbuttcap%
\pgfsetroundjoin%
\definecolor{currentfill}{rgb}{1.000000,0.894118,0.788235}%
\pgfsetfillcolor{currentfill}%
\pgfsetlinewidth{0.000000pt}%
\definecolor{currentstroke}{rgb}{1.000000,0.894118,0.788235}%
\pgfsetstrokecolor{currentstroke}%
\pgfsetdash{}{0pt}%
\pgfpathmoveto{\pgfqpoint{2.692300in}{3.472714in}}%
\pgfpathlineto{\pgfqpoint{2.581404in}{3.408688in}}%
\pgfpathlineto{\pgfqpoint{2.692300in}{3.344662in}}%
\pgfpathlineto{\pgfqpoint{2.803197in}{3.408688in}}%
\pgfpathlineto{\pgfqpoint{2.692300in}{3.472714in}}%
\pgfpathclose%
\pgfusepath{fill}%
\end{pgfscope}%
\begin{pgfscope}%
\pgfpathrectangle{\pgfqpoint{0.765000in}{0.660000in}}{\pgfqpoint{4.620000in}{4.620000in}}%
\pgfusepath{clip}%
\pgfsetbuttcap%
\pgfsetroundjoin%
\definecolor{currentfill}{rgb}{1.000000,0.894118,0.788235}%
\pgfsetfillcolor{currentfill}%
\pgfsetlinewidth{0.000000pt}%
\definecolor{currentstroke}{rgb}{1.000000,0.894118,0.788235}%
\pgfsetstrokecolor{currentstroke}%
\pgfsetdash{}{0pt}%
\pgfpathmoveto{\pgfqpoint{2.692300in}{3.216609in}}%
\pgfpathlineto{\pgfqpoint{2.803197in}{3.280636in}}%
\pgfpathlineto{\pgfqpoint{2.803197in}{3.408688in}}%
\pgfpathlineto{\pgfqpoint{2.692300in}{3.344662in}}%
\pgfpathlineto{\pgfqpoint{2.692300in}{3.216609in}}%
\pgfpathclose%
\pgfusepath{fill}%
\end{pgfscope}%
\begin{pgfscope}%
\pgfpathrectangle{\pgfqpoint{0.765000in}{0.660000in}}{\pgfqpoint{4.620000in}{4.620000in}}%
\pgfusepath{clip}%
\pgfsetbuttcap%
\pgfsetroundjoin%
\definecolor{currentfill}{rgb}{1.000000,0.894118,0.788235}%
\pgfsetfillcolor{currentfill}%
\pgfsetlinewidth{0.000000pt}%
\definecolor{currentstroke}{rgb}{1.000000,0.894118,0.788235}%
\pgfsetstrokecolor{currentstroke}%
\pgfsetdash{}{0pt}%
\pgfpathmoveto{\pgfqpoint{2.581404in}{3.280636in}}%
\pgfpathlineto{\pgfqpoint{2.692300in}{3.216609in}}%
\pgfpathlineto{\pgfqpoint{2.692300in}{3.344662in}}%
\pgfpathlineto{\pgfqpoint{2.581404in}{3.408688in}}%
\pgfpathlineto{\pgfqpoint{2.581404in}{3.280636in}}%
\pgfpathclose%
\pgfusepath{fill}%
\end{pgfscope}%
\begin{pgfscope}%
\pgfpathrectangle{\pgfqpoint{0.765000in}{0.660000in}}{\pgfqpoint{4.620000in}{4.620000in}}%
\pgfusepath{clip}%
\pgfsetbuttcap%
\pgfsetroundjoin%
\definecolor{currentfill}{rgb}{1.000000,0.894118,0.788235}%
\pgfsetfillcolor{currentfill}%
\pgfsetlinewidth{0.000000pt}%
\definecolor{currentstroke}{rgb}{1.000000,0.894118,0.788235}%
\pgfsetstrokecolor{currentstroke}%
\pgfsetdash{}{0pt}%
\pgfpathmoveto{\pgfqpoint{2.671300in}{3.455060in}}%
\pgfpathlineto{\pgfqpoint{2.560403in}{3.391034in}}%
\pgfpathlineto{\pgfqpoint{2.671300in}{3.327008in}}%
\pgfpathlineto{\pgfqpoint{2.782196in}{3.391034in}}%
\pgfpathlineto{\pgfqpoint{2.671300in}{3.455060in}}%
\pgfpathclose%
\pgfusepath{fill}%
\end{pgfscope}%
\begin{pgfscope}%
\pgfpathrectangle{\pgfqpoint{0.765000in}{0.660000in}}{\pgfqpoint{4.620000in}{4.620000in}}%
\pgfusepath{clip}%
\pgfsetbuttcap%
\pgfsetroundjoin%
\definecolor{currentfill}{rgb}{1.000000,0.894118,0.788235}%
\pgfsetfillcolor{currentfill}%
\pgfsetlinewidth{0.000000pt}%
\definecolor{currentstroke}{rgb}{1.000000,0.894118,0.788235}%
\pgfsetstrokecolor{currentstroke}%
\pgfsetdash{}{0pt}%
\pgfpathmoveto{\pgfqpoint{2.671300in}{3.198955in}}%
\pgfpathlineto{\pgfqpoint{2.782196in}{3.262981in}}%
\pgfpathlineto{\pgfqpoint{2.782196in}{3.391034in}}%
\pgfpathlineto{\pgfqpoint{2.671300in}{3.327008in}}%
\pgfpathlineto{\pgfqpoint{2.671300in}{3.198955in}}%
\pgfpathclose%
\pgfusepath{fill}%
\end{pgfscope}%
\begin{pgfscope}%
\pgfpathrectangle{\pgfqpoint{0.765000in}{0.660000in}}{\pgfqpoint{4.620000in}{4.620000in}}%
\pgfusepath{clip}%
\pgfsetbuttcap%
\pgfsetroundjoin%
\definecolor{currentfill}{rgb}{1.000000,0.894118,0.788235}%
\pgfsetfillcolor{currentfill}%
\pgfsetlinewidth{0.000000pt}%
\definecolor{currentstroke}{rgb}{1.000000,0.894118,0.788235}%
\pgfsetstrokecolor{currentstroke}%
\pgfsetdash{}{0pt}%
\pgfpathmoveto{\pgfqpoint{2.560403in}{3.262981in}}%
\pgfpathlineto{\pgfqpoint{2.671300in}{3.198955in}}%
\pgfpathlineto{\pgfqpoint{2.671300in}{3.327008in}}%
\pgfpathlineto{\pgfqpoint{2.560403in}{3.391034in}}%
\pgfpathlineto{\pgfqpoint{2.560403in}{3.262981in}}%
\pgfpathclose%
\pgfusepath{fill}%
\end{pgfscope}%
\begin{pgfscope}%
\pgfpathrectangle{\pgfqpoint{0.765000in}{0.660000in}}{\pgfqpoint{4.620000in}{4.620000in}}%
\pgfusepath{clip}%
\pgfsetbuttcap%
\pgfsetroundjoin%
\definecolor{currentfill}{rgb}{1.000000,0.894118,0.788235}%
\pgfsetfillcolor{currentfill}%
\pgfsetlinewidth{0.000000pt}%
\definecolor{currentstroke}{rgb}{1.000000,0.894118,0.788235}%
\pgfsetstrokecolor{currentstroke}%
\pgfsetdash{}{0pt}%
\pgfpathmoveto{\pgfqpoint{2.692300in}{3.344662in}}%
\pgfpathlineto{\pgfqpoint{2.581404in}{3.280636in}}%
\pgfpathlineto{\pgfqpoint{2.560403in}{3.262981in}}%
\pgfpathlineto{\pgfqpoint{2.671300in}{3.327008in}}%
\pgfpathlineto{\pgfqpoint{2.692300in}{3.344662in}}%
\pgfpathclose%
\pgfusepath{fill}%
\end{pgfscope}%
\begin{pgfscope}%
\pgfpathrectangle{\pgfqpoint{0.765000in}{0.660000in}}{\pgfqpoint{4.620000in}{4.620000in}}%
\pgfusepath{clip}%
\pgfsetbuttcap%
\pgfsetroundjoin%
\definecolor{currentfill}{rgb}{1.000000,0.894118,0.788235}%
\pgfsetfillcolor{currentfill}%
\pgfsetlinewidth{0.000000pt}%
\definecolor{currentstroke}{rgb}{1.000000,0.894118,0.788235}%
\pgfsetstrokecolor{currentstroke}%
\pgfsetdash{}{0pt}%
\pgfpathmoveto{\pgfqpoint{2.803197in}{3.280636in}}%
\pgfpathlineto{\pgfqpoint{2.692300in}{3.344662in}}%
\pgfpathlineto{\pgfqpoint{2.671300in}{3.327008in}}%
\pgfpathlineto{\pgfqpoint{2.782196in}{3.262981in}}%
\pgfpathlineto{\pgfqpoint{2.803197in}{3.280636in}}%
\pgfpathclose%
\pgfusepath{fill}%
\end{pgfscope}%
\begin{pgfscope}%
\pgfpathrectangle{\pgfqpoint{0.765000in}{0.660000in}}{\pgfqpoint{4.620000in}{4.620000in}}%
\pgfusepath{clip}%
\pgfsetbuttcap%
\pgfsetroundjoin%
\definecolor{currentfill}{rgb}{1.000000,0.894118,0.788235}%
\pgfsetfillcolor{currentfill}%
\pgfsetlinewidth{0.000000pt}%
\definecolor{currentstroke}{rgb}{1.000000,0.894118,0.788235}%
\pgfsetstrokecolor{currentstroke}%
\pgfsetdash{}{0pt}%
\pgfpathmoveto{\pgfqpoint{2.692300in}{3.344662in}}%
\pgfpathlineto{\pgfqpoint{2.692300in}{3.472714in}}%
\pgfpathlineto{\pgfqpoint{2.671300in}{3.455060in}}%
\pgfpathlineto{\pgfqpoint{2.782196in}{3.262981in}}%
\pgfpathlineto{\pgfqpoint{2.692300in}{3.344662in}}%
\pgfpathclose%
\pgfusepath{fill}%
\end{pgfscope}%
\begin{pgfscope}%
\pgfpathrectangle{\pgfqpoint{0.765000in}{0.660000in}}{\pgfqpoint{4.620000in}{4.620000in}}%
\pgfusepath{clip}%
\pgfsetbuttcap%
\pgfsetroundjoin%
\definecolor{currentfill}{rgb}{1.000000,0.894118,0.788235}%
\pgfsetfillcolor{currentfill}%
\pgfsetlinewidth{0.000000pt}%
\definecolor{currentstroke}{rgb}{1.000000,0.894118,0.788235}%
\pgfsetstrokecolor{currentstroke}%
\pgfsetdash{}{0pt}%
\pgfpathmoveto{\pgfqpoint{2.803197in}{3.280636in}}%
\pgfpathlineto{\pgfqpoint{2.803197in}{3.408688in}}%
\pgfpathlineto{\pgfqpoint{2.782196in}{3.391034in}}%
\pgfpathlineto{\pgfqpoint{2.671300in}{3.327008in}}%
\pgfpathlineto{\pgfqpoint{2.803197in}{3.280636in}}%
\pgfpathclose%
\pgfusepath{fill}%
\end{pgfscope}%
\begin{pgfscope}%
\pgfpathrectangle{\pgfqpoint{0.765000in}{0.660000in}}{\pgfqpoint{4.620000in}{4.620000in}}%
\pgfusepath{clip}%
\pgfsetbuttcap%
\pgfsetroundjoin%
\definecolor{currentfill}{rgb}{1.000000,0.894118,0.788235}%
\pgfsetfillcolor{currentfill}%
\pgfsetlinewidth{0.000000pt}%
\definecolor{currentstroke}{rgb}{1.000000,0.894118,0.788235}%
\pgfsetstrokecolor{currentstroke}%
\pgfsetdash{}{0pt}%
\pgfpathmoveto{\pgfqpoint{2.581404in}{3.280636in}}%
\pgfpathlineto{\pgfqpoint{2.692300in}{3.216609in}}%
\pgfpathlineto{\pgfqpoint{2.671300in}{3.198955in}}%
\pgfpathlineto{\pgfqpoint{2.560403in}{3.262981in}}%
\pgfpathlineto{\pgfqpoint{2.581404in}{3.280636in}}%
\pgfpathclose%
\pgfusepath{fill}%
\end{pgfscope}%
\begin{pgfscope}%
\pgfpathrectangle{\pgfqpoint{0.765000in}{0.660000in}}{\pgfqpoint{4.620000in}{4.620000in}}%
\pgfusepath{clip}%
\pgfsetbuttcap%
\pgfsetroundjoin%
\definecolor{currentfill}{rgb}{1.000000,0.894118,0.788235}%
\pgfsetfillcolor{currentfill}%
\pgfsetlinewidth{0.000000pt}%
\definecolor{currentstroke}{rgb}{1.000000,0.894118,0.788235}%
\pgfsetstrokecolor{currentstroke}%
\pgfsetdash{}{0pt}%
\pgfpathmoveto{\pgfqpoint{2.692300in}{3.216609in}}%
\pgfpathlineto{\pgfqpoint{2.803197in}{3.280636in}}%
\pgfpathlineto{\pgfqpoint{2.782196in}{3.262981in}}%
\pgfpathlineto{\pgfqpoint{2.671300in}{3.198955in}}%
\pgfpathlineto{\pgfqpoint{2.692300in}{3.216609in}}%
\pgfpathclose%
\pgfusepath{fill}%
\end{pgfscope}%
\begin{pgfscope}%
\pgfpathrectangle{\pgfqpoint{0.765000in}{0.660000in}}{\pgfqpoint{4.620000in}{4.620000in}}%
\pgfusepath{clip}%
\pgfsetbuttcap%
\pgfsetroundjoin%
\definecolor{currentfill}{rgb}{1.000000,0.894118,0.788235}%
\pgfsetfillcolor{currentfill}%
\pgfsetlinewidth{0.000000pt}%
\definecolor{currentstroke}{rgb}{1.000000,0.894118,0.788235}%
\pgfsetstrokecolor{currentstroke}%
\pgfsetdash{}{0pt}%
\pgfpathmoveto{\pgfqpoint{2.692300in}{3.472714in}}%
\pgfpathlineto{\pgfqpoint{2.581404in}{3.408688in}}%
\pgfpathlineto{\pgfqpoint{2.560403in}{3.391034in}}%
\pgfpathlineto{\pgfqpoint{2.671300in}{3.455060in}}%
\pgfpathlineto{\pgfqpoint{2.692300in}{3.472714in}}%
\pgfpathclose%
\pgfusepath{fill}%
\end{pgfscope}%
\begin{pgfscope}%
\pgfpathrectangle{\pgfqpoint{0.765000in}{0.660000in}}{\pgfqpoint{4.620000in}{4.620000in}}%
\pgfusepath{clip}%
\pgfsetbuttcap%
\pgfsetroundjoin%
\definecolor{currentfill}{rgb}{1.000000,0.894118,0.788235}%
\pgfsetfillcolor{currentfill}%
\pgfsetlinewidth{0.000000pt}%
\definecolor{currentstroke}{rgb}{1.000000,0.894118,0.788235}%
\pgfsetstrokecolor{currentstroke}%
\pgfsetdash{}{0pt}%
\pgfpathmoveto{\pgfqpoint{2.803197in}{3.408688in}}%
\pgfpathlineto{\pgfqpoint{2.692300in}{3.472714in}}%
\pgfpathlineto{\pgfqpoint{2.671300in}{3.455060in}}%
\pgfpathlineto{\pgfqpoint{2.782196in}{3.391034in}}%
\pgfpathlineto{\pgfqpoint{2.803197in}{3.408688in}}%
\pgfpathclose%
\pgfusepath{fill}%
\end{pgfscope}%
\begin{pgfscope}%
\pgfpathrectangle{\pgfqpoint{0.765000in}{0.660000in}}{\pgfqpoint{4.620000in}{4.620000in}}%
\pgfusepath{clip}%
\pgfsetbuttcap%
\pgfsetroundjoin%
\definecolor{currentfill}{rgb}{1.000000,0.894118,0.788235}%
\pgfsetfillcolor{currentfill}%
\pgfsetlinewidth{0.000000pt}%
\definecolor{currentstroke}{rgb}{1.000000,0.894118,0.788235}%
\pgfsetstrokecolor{currentstroke}%
\pgfsetdash{}{0pt}%
\pgfpathmoveto{\pgfqpoint{2.581404in}{3.280636in}}%
\pgfpathlineto{\pgfqpoint{2.581404in}{3.408688in}}%
\pgfpathlineto{\pgfqpoint{2.560403in}{3.391034in}}%
\pgfpathlineto{\pgfqpoint{2.671300in}{3.198955in}}%
\pgfpathlineto{\pgfqpoint{2.581404in}{3.280636in}}%
\pgfpathclose%
\pgfusepath{fill}%
\end{pgfscope}%
\begin{pgfscope}%
\pgfpathrectangle{\pgfqpoint{0.765000in}{0.660000in}}{\pgfqpoint{4.620000in}{4.620000in}}%
\pgfusepath{clip}%
\pgfsetbuttcap%
\pgfsetroundjoin%
\definecolor{currentfill}{rgb}{1.000000,0.894118,0.788235}%
\pgfsetfillcolor{currentfill}%
\pgfsetlinewidth{0.000000pt}%
\definecolor{currentstroke}{rgb}{1.000000,0.894118,0.788235}%
\pgfsetstrokecolor{currentstroke}%
\pgfsetdash{}{0pt}%
\pgfpathmoveto{\pgfqpoint{2.692300in}{3.216609in}}%
\pgfpathlineto{\pgfqpoint{2.692300in}{3.344662in}}%
\pgfpathlineto{\pgfqpoint{2.671300in}{3.327008in}}%
\pgfpathlineto{\pgfqpoint{2.560403in}{3.262981in}}%
\pgfpathlineto{\pgfqpoint{2.692300in}{3.216609in}}%
\pgfpathclose%
\pgfusepath{fill}%
\end{pgfscope}%
\begin{pgfscope}%
\pgfpathrectangle{\pgfqpoint{0.765000in}{0.660000in}}{\pgfqpoint{4.620000in}{4.620000in}}%
\pgfusepath{clip}%
\pgfsetbuttcap%
\pgfsetroundjoin%
\definecolor{currentfill}{rgb}{1.000000,0.894118,0.788235}%
\pgfsetfillcolor{currentfill}%
\pgfsetlinewidth{0.000000pt}%
\definecolor{currentstroke}{rgb}{1.000000,0.894118,0.788235}%
\pgfsetstrokecolor{currentstroke}%
\pgfsetdash{}{0pt}%
\pgfpathmoveto{\pgfqpoint{2.692300in}{3.344662in}}%
\pgfpathlineto{\pgfqpoint{2.803197in}{3.408688in}}%
\pgfpathlineto{\pgfqpoint{2.782196in}{3.391034in}}%
\pgfpathlineto{\pgfqpoint{2.671300in}{3.327008in}}%
\pgfpathlineto{\pgfqpoint{2.692300in}{3.344662in}}%
\pgfpathclose%
\pgfusepath{fill}%
\end{pgfscope}%
\begin{pgfscope}%
\pgfpathrectangle{\pgfqpoint{0.765000in}{0.660000in}}{\pgfqpoint{4.620000in}{4.620000in}}%
\pgfusepath{clip}%
\pgfsetbuttcap%
\pgfsetroundjoin%
\definecolor{currentfill}{rgb}{1.000000,0.894118,0.788235}%
\pgfsetfillcolor{currentfill}%
\pgfsetlinewidth{0.000000pt}%
\definecolor{currentstroke}{rgb}{1.000000,0.894118,0.788235}%
\pgfsetstrokecolor{currentstroke}%
\pgfsetdash{}{0pt}%
\pgfpathmoveto{\pgfqpoint{2.581404in}{3.408688in}}%
\pgfpathlineto{\pgfqpoint{2.692300in}{3.344662in}}%
\pgfpathlineto{\pgfqpoint{2.671300in}{3.327008in}}%
\pgfpathlineto{\pgfqpoint{2.560403in}{3.391034in}}%
\pgfpathlineto{\pgfqpoint{2.581404in}{3.408688in}}%
\pgfpathclose%
\pgfusepath{fill}%
\end{pgfscope}%
\begin{pgfscope}%
\pgfpathrectangle{\pgfqpoint{0.765000in}{0.660000in}}{\pgfqpoint{4.620000in}{4.620000in}}%
\pgfusepath{clip}%
\pgfsetbuttcap%
\pgfsetroundjoin%
\definecolor{currentfill}{rgb}{1.000000,0.894118,0.788235}%
\pgfsetfillcolor{currentfill}%
\pgfsetlinewidth{0.000000pt}%
\definecolor{currentstroke}{rgb}{1.000000,0.894118,0.788235}%
\pgfsetstrokecolor{currentstroke}%
\pgfsetdash{}{0pt}%
\pgfpathmoveto{\pgfqpoint{3.425935in}{3.326669in}}%
\pgfpathlineto{\pgfqpoint{3.315038in}{3.262643in}}%
\pgfpathlineto{\pgfqpoint{3.425935in}{3.198617in}}%
\pgfpathlineto{\pgfqpoint{3.536832in}{3.262643in}}%
\pgfpathlineto{\pgfqpoint{3.425935in}{3.326669in}}%
\pgfpathclose%
\pgfusepath{fill}%
\end{pgfscope}%
\begin{pgfscope}%
\pgfpathrectangle{\pgfqpoint{0.765000in}{0.660000in}}{\pgfqpoint{4.620000in}{4.620000in}}%
\pgfusepath{clip}%
\pgfsetbuttcap%
\pgfsetroundjoin%
\definecolor{currentfill}{rgb}{1.000000,0.894118,0.788235}%
\pgfsetfillcolor{currentfill}%
\pgfsetlinewidth{0.000000pt}%
\definecolor{currentstroke}{rgb}{1.000000,0.894118,0.788235}%
\pgfsetstrokecolor{currentstroke}%
\pgfsetdash{}{0pt}%
\pgfpathmoveto{\pgfqpoint{3.425935in}{3.326669in}}%
\pgfpathlineto{\pgfqpoint{3.315038in}{3.262643in}}%
\pgfpathlineto{\pgfqpoint{3.315038in}{3.390696in}}%
\pgfpathlineto{\pgfqpoint{3.425935in}{3.454722in}}%
\pgfpathlineto{\pgfqpoint{3.425935in}{3.326669in}}%
\pgfpathclose%
\pgfusepath{fill}%
\end{pgfscope}%
\begin{pgfscope}%
\pgfpathrectangle{\pgfqpoint{0.765000in}{0.660000in}}{\pgfqpoint{4.620000in}{4.620000in}}%
\pgfusepath{clip}%
\pgfsetbuttcap%
\pgfsetroundjoin%
\definecolor{currentfill}{rgb}{1.000000,0.894118,0.788235}%
\pgfsetfillcolor{currentfill}%
\pgfsetlinewidth{0.000000pt}%
\definecolor{currentstroke}{rgb}{1.000000,0.894118,0.788235}%
\pgfsetstrokecolor{currentstroke}%
\pgfsetdash{}{0pt}%
\pgfpathmoveto{\pgfqpoint{3.425935in}{3.326669in}}%
\pgfpathlineto{\pgfqpoint{3.536832in}{3.262643in}}%
\pgfpathlineto{\pgfqpoint{3.536832in}{3.390696in}}%
\pgfpathlineto{\pgfqpoint{3.425935in}{3.454722in}}%
\pgfpathlineto{\pgfqpoint{3.425935in}{3.326669in}}%
\pgfpathclose%
\pgfusepath{fill}%
\end{pgfscope}%
\begin{pgfscope}%
\pgfpathrectangle{\pgfqpoint{0.765000in}{0.660000in}}{\pgfqpoint{4.620000in}{4.620000in}}%
\pgfusepath{clip}%
\pgfsetbuttcap%
\pgfsetroundjoin%
\definecolor{currentfill}{rgb}{1.000000,0.894118,0.788235}%
\pgfsetfillcolor{currentfill}%
\pgfsetlinewidth{0.000000pt}%
\definecolor{currentstroke}{rgb}{1.000000,0.894118,0.788235}%
\pgfsetstrokecolor{currentstroke}%
\pgfsetdash{}{0pt}%
\pgfpathmoveto{\pgfqpoint{3.357740in}{3.416522in}}%
\pgfpathlineto{\pgfqpoint{3.246843in}{3.352496in}}%
\pgfpathlineto{\pgfqpoint{3.357740in}{3.288470in}}%
\pgfpathlineto{\pgfqpoint{3.468636in}{3.352496in}}%
\pgfpathlineto{\pgfqpoint{3.357740in}{3.416522in}}%
\pgfpathclose%
\pgfusepath{fill}%
\end{pgfscope}%
\begin{pgfscope}%
\pgfpathrectangle{\pgfqpoint{0.765000in}{0.660000in}}{\pgfqpoint{4.620000in}{4.620000in}}%
\pgfusepath{clip}%
\pgfsetbuttcap%
\pgfsetroundjoin%
\definecolor{currentfill}{rgb}{1.000000,0.894118,0.788235}%
\pgfsetfillcolor{currentfill}%
\pgfsetlinewidth{0.000000pt}%
\definecolor{currentstroke}{rgb}{1.000000,0.894118,0.788235}%
\pgfsetstrokecolor{currentstroke}%
\pgfsetdash{}{0pt}%
\pgfpathmoveto{\pgfqpoint{3.357740in}{3.416522in}}%
\pgfpathlineto{\pgfqpoint{3.246843in}{3.352496in}}%
\pgfpathlineto{\pgfqpoint{3.246843in}{3.480548in}}%
\pgfpathlineto{\pgfqpoint{3.357740in}{3.544575in}}%
\pgfpathlineto{\pgfqpoint{3.357740in}{3.416522in}}%
\pgfpathclose%
\pgfusepath{fill}%
\end{pgfscope}%
\begin{pgfscope}%
\pgfpathrectangle{\pgfqpoint{0.765000in}{0.660000in}}{\pgfqpoint{4.620000in}{4.620000in}}%
\pgfusepath{clip}%
\pgfsetbuttcap%
\pgfsetroundjoin%
\definecolor{currentfill}{rgb}{1.000000,0.894118,0.788235}%
\pgfsetfillcolor{currentfill}%
\pgfsetlinewidth{0.000000pt}%
\definecolor{currentstroke}{rgb}{1.000000,0.894118,0.788235}%
\pgfsetstrokecolor{currentstroke}%
\pgfsetdash{}{0pt}%
\pgfpathmoveto{\pgfqpoint{3.357740in}{3.416522in}}%
\pgfpathlineto{\pgfqpoint{3.468636in}{3.352496in}}%
\pgfpathlineto{\pgfqpoint{3.468636in}{3.480548in}}%
\pgfpathlineto{\pgfqpoint{3.357740in}{3.544575in}}%
\pgfpathlineto{\pgfqpoint{3.357740in}{3.416522in}}%
\pgfpathclose%
\pgfusepath{fill}%
\end{pgfscope}%
\begin{pgfscope}%
\pgfpathrectangle{\pgfqpoint{0.765000in}{0.660000in}}{\pgfqpoint{4.620000in}{4.620000in}}%
\pgfusepath{clip}%
\pgfsetbuttcap%
\pgfsetroundjoin%
\definecolor{currentfill}{rgb}{1.000000,0.894118,0.788235}%
\pgfsetfillcolor{currentfill}%
\pgfsetlinewidth{0.000000pt}%
\definecolor{currentstroke}{rgb}{1.000000,0.894118,0.788235}%
\pgfsetstrokecolor{currentstroke}%
\pgfsetdash{}{0pt}%
\pgfpathmoveto{\pgfqpoint{3.425935in}{3.454722in}}%
\pgfpathlineto{\pgfqpoint{3.315038in}{3.390696in}}%
\pgfpathlineto{\pgfqpoint{3.425935in}{3.326669in}}%
\pgfpathlineto{\pgfqpoint{3.536832in}{3.390696in}}%
\pgfpathlineto{\pgfqpoint{3.425935in}{3.454722in}}%
\pgfpathclose%
\pgfusepath{fill}%
\end{pgfscope}%
\begin{pgfscope}%
\pgfpathrectangle{\pgfqpoint{0.765000in}{0.660000in}}{\pgfqpoint{4.620000in}{4.620000in}}%
\pgfusepath{clip}%
\pgfsetbuttcap%
\pgfsetroundjoin%
\definecolor{currentfill}{rgb}{1.000000,0.894118,0.788235}%
\pgfsetfillcolor{currentfill}%
\pgfsetlinewidth{0.000000pt}%
\definecolor{currentstroke}{rgb}{1.000000,0.894118,0.788235}%
\pgfsetstrokecolor{currentstroke}%
\pgfsetdash{}{0pt}%
\pgfpathmoveto{\pgfqpoint{3.425935in}{3.198617in}}%
\pgfpathlineto{\pgfqpoint{3.536832in}{3.262643in}}%
\pgfpathlineto{\pgfqpoint{3.536832in}{3.390696in}}%
\pgfpathlineto{\pgfqpoint{3.425935in}{3.326669in}}%
\pgfpathlineto{\pgfqpoint{3.425935in}{3.198617in}}%
\pgfpathclose%
\pgfusepath{fill}%
\end{pgfscope}%
\begin{pgfscope}%
\pgfpathrectangle{\pgfqpoint{0.765000in}{0.660000in}}{\pgfqpoint{4.620000in}{4.620000in}}%
\pgfusepath{clip}%
\pgfsetbuttcap%
\pgfsetroundjoin%
\definecolor{currentfill}{rgb}{1.000000,0.894118,0.788235}%
\pgfsetfillcolor{currentfill}%
\pgfsetlinewidth{0.000000pt}%
\definecolor{currentstroke}{rgb}{1.000000,0.894118,0.788235}%
\pgfsetstrokecolor{currentstroke}%
\pgfsetdash{}{0pt}%
\pgfpathmoveto{\pgfqpoint{3.315038in}{3.262643in}}%
\pgfpathlineto{\pgfqpoint{3.425935in}{3.198617in}}%
\pgfpathlineto{\pgfqpoint{3.425935in}{3.326669in}}%
\pgfpathlineto{\pgfqpoint{3.315038in}{3.390696in}}%
\pgfpathlineto{\pgfqpoint{3.315038in}{3.262643in}}%
\pgfpathclose%
\pgfusepath{fill}%
\end{pgfscope}%
\begin{pgfscope}%
\pgfpathrectangle{\pgfqpoint{0.765000in}{0.660000in}}{\pgfqpoint{4.620000in}{4.620000in}}%
\pgfusepath{clip}%
\pgfsetbuttcap%
\pgfsetroundjoin%
\definecolor{currentfill}{rgb}{1.000000,0.894118,0.788235}%
\pgfsetfillcolor{currentfill}%
\pgfsetlinewidth{0.000000pt}%
\definecolor{currentstroke}{rgb}{1.000000,0.894118,0.788235}%
\pgfsetstrokecolor{currentstroke}%
\pgfsetdash{}{0pt}%
\pgfpathmoveto{\pgfqpoint{3.357740in}{3.544575in}}%
\pgfpathlineto{\pgfqpoint{3.246843in}{3.480548in}}%
\pgfpathlineto{\pgfqpoint{3.357740in}{3.416522in}}%
\pgfpathlineto{\pgfqpoint{3.468636in}{3.480548in}}%
\pgfpathlineto{\pgfqpoint{3.357740in}{3.544575in}}%
\pgfpathclose%
\pgfusepath{fill}%
\end{pgfscope}%
\begin{pgfscope}%
\pgfpathrectangle{\pgfqpoint{0.765000in}{0.660000in}}{\pgfqpoint{4.620000in}{4.620000in}}%
\pgfusepath{clip}%
\pgfsetbuttcap%
\pgfsetroundjoin%
\definecolor{currentfill}{rgb}{1.000000,0.894118,0.788235}%
\pgfsetfillcolor{currentfill}%
\pgfsetlinewidth{0.000000pt}%
\definecolor{currentstroke}{rgb}{1.000000,0.894118,0.788235}%
\pgfsetstrokecolor{currentstroke}%
\pgfsetdash{}{0pt}%
\pgfpathmoveto{\pgfqpoint{3.357740in}{3.288470in}}%
\pgfpathlineto{\pgfqpoint{3.468636in}{3.352496in}}%
\pgfpathlineto{\pgfqpoint{3.468636in}{3.480548in}}%
\pgfpathlineto{\pgfqpoint{3.357740in}{3.416522in}}%
\pgfpathlineto{\pgfqpoint{3.357740in}{3.288470in}}%
\pgfpathclose%
\pgfusepath{fill}%
\end{pgfscope}%
\begin{pgfscope}%
\pgfpathrectangle{\pgfqpoint{0.765000in}{0.660000in}}{\pgfqpoint{4.620000in}{4.620000in}}%
\pgfusepath{clip}%
\pgfsetbuttcap%
\pgfsetroundjoin%
\definecolor{currentfill}{rgb}{1.000000,0.894118,0.788235}%
\pgfsetfillcolor{currentfill}%
\pgfsetlinewidth{0.000000pt}%
\definecolor{currentstroke}{rgb}{1.000000,0.894118,0.788235}%
\pgfsetstrokecolor{currentstroke}%
\pgfsetdash{}{0pt}%
\pgfpathmoveto{\pgfqpoint{3.246843in}{3.352496in}}%
\pgfpathlineto{\pgfqpoint{3.357740in}{3.288470in}}%
\pgfpathlineto{\pgfqpoint{3.357740in}{3.416522in}}%
\pgfpathlineto{\pgfqpoint{3.246843in}{3.480548in}}%
\pgfpathlineto{\pgfqpoint{3.246843in}{3.352496in}}%
\pgfpathclose%
\pgfusepath{fill}%
\end{pgfscope}%
\begin{pgfscope}%
\pgfpathrectangle{\pgfqpoint{0.765000in}{0.660000in}}{\pgfqpoint{4.620000in}{4.620000in}}%
\pgfusepath{clip}%
\pgfsetbuttcap%
\pgfsetroundjoin%
\definecolor{currentfill}{rgb}{1.000000,0.894118,0.788235}%
\pgfsetfillcolor{currentfill}%
\pgfsetlinewidth{0.000000pt}%
\definecolor{currentstroke}{rgb}{1.000000,0.894118,0.788235}%
\pgfsetstrokecolor{currentstroke}%
\pgfsetdash{}{0pt}%
\pgfpathmoveto{\pgfqpoint{3.425935in}{3.326669in}}%
\pgfpathlineto{\pgfqpoint{3.315038in}{3.262643in}}%
\pgfpathlineto{\pgfqpoint{3.246843in}{3.352496in}}%
\pgfpathlineto{\pgfqpoint{3.357740in}{3.416522in}}%
\pgfpathlineto{\pgfqpoint{3.425935in}{3.326669in}}%
\pgfpathclose%
\pgfusepath{fill}%
\end{pgfscope}%
\begin{pgfscope}%
\pgfpathrectangle{\pgfqpoint{0.765000in}{0.660000in}}{\pgfqpoint{4.620000in}{4.620000in}}%
\pgfusepath{clip}%
\pgfsetbuttcap%
\pgfsetroundjoin%
\definecolor{currentfill}{rgb}{1.000000,0.894118,0.788235}%
\pgfsetfillcolor{currentfill}%
\pgfsetlinewidth{0.000000pt}%
\definecolor{currentstroke}{rgb}{1.000000,0.894118,0.788235}%
\pgfsetstrokecolor{currentstroke}%
\pgfsetdash{}{0pt}%
\pgfpathmoveto{\pgfqpoint{3.536832in}{3.262643in}}%
\pgfpathlineto{\pgfqpoint{3.425935in}{3.326669in}}%
\pgfpathlineto{\pgfqpoint{3.357740in}{3.416522in}}%
\pgfpathlineto{\pgfqpoint{3.468636in}{3.352496in}}%
\pgfpathlineto{\pgfqpoint{3.536832in}{3.262643in}}%
\pgfpathclose%
\pgfusepath{fill}%
\end{pgfscope}%
\begin{pgfscope}%
\pgfpathrectangle{\pgfqpoint{0.765000in}{0.660000in}}{\pgfqpoint{4.620000in}{4.620000in}}%
\pgfusepath{clip}%
\pgfsetbuttcap%
\pgfsetroundjoin%
\definecolor{currentfill}{rgb}{1.000000,0.894118,0.788235}%
\pgfsetfillcolor{currentfill}%
\pgfsetlinewidth{0.000000pt}%
\definecolor{currentstroke}{rgb}{1.000000,0.894118,0.788235}%
\pgfsetstrokecolor{currentstroke}%
\pgfsetdash{}{0pt}%
\pgfpathmoveto{\pgfqpoint{3.425935in}{3.326669in}}%
\pgfpathlineto{\pgfqpoint{3.425935in}{3.454722in}}%
\pgfpathlineto{\pgfqpoint{3.357740in}{3.544575in}}%
\pgfpathlineto{\pgfqpoint{3.468636in}{3.352496in}}%
\pgfpathlineto{\pgfqpoint{3.425935in}{3.326669in}}%
\pgfpathclose%
\pgfusepath{fill}%
\end{pgfscope}%
\begin{pgfscope}%
\pgfpathrectangle{\pgfqpoint{0.765000in}{0.660000in}}{\pgfqpoint{4.620000in}{4.620000in}}%
\pgfusepath{clip}%
\pgfsetbuttcap%
\pgfsetroundjoin%
\definecolor{currentfill}{rgb}{1.000000,0.894118,0.788235}%
\pgfsetfillcolor{currentfill}%
\pgfsetlinewidth{0.000000pt}%
\definecolor{currentstroke}{rgb}{1.000000,0.894118,0.788235}%
\pgfsetstrokecolor{currentstroke}%
\pgfsetdash{}{0pt}%
\pgfpathmoveto{\pgfqpoint{3.536832in}{3.262643in}}%
\pgfpathlineto{\pgfqpoint{3.536832in}{3.390696in}}%
\pgfpathlineto{\pgfqpoint{3.468636in}{3.480548in}}%
\pgfpathlineto{\pgfqpoint{3.357740in}{3.416522in}}%
\pgfpathlineto{\pgfqpoint{3.536832in}{3.262643in}}%
\pgfpathclose%
\pgfusepath{fill}%
\end{pgfscope}%
\begin{pgfscope}%
\pgfpathrectangle{\pgfqpoint{0.765000in}{0.660000in}}{\pgfqpoint{4.620000in}{4.620000in}}%
\pgfusepath{clip}%
\pgfsetbuttcap%
\pgfsetroundjoin%
\definecolor{currentfill}{rgb}{1.000000,0.894118,0.788235}%
\pgfsetfillcolor{currentfill}%
\pgfsetlinewidth{0.000000pt}%
\definecolor{currentstroke}{rgb}{1.000000,0.894118,0.788235}%
\pgfsetstrokecolor{currentstroke}%
\pgfsetdash{}{0pt}%
\pgfpathmoveto{\pgfqpoint{3.315038in}{3.262643in}}%
\pgfpathlineto{\pgfqpoint{3.425935in}{3.198617in}}%
\pgfpathlineto{\pgfqpoint{3.357740in}{3.288470in}}%
\pgfpathlineto{\pgfqpoint{3.246843in}{3.352496in}}%
\pgfpathlineto{\pgfqpoint{3.315038in}{3.262643in}}%
\pgfpathclose%
\pgfusepath{fill}%
\end{pgfscope}%
\begin{pgfscope}%
\pgfpathrectangle{\pgfqpoint{0.765000in}{0.660000in}}{\pgfqpoint{4.620000in}{4.620000in}}%
\pgfusepath{clip}%
\pgfsetbuttcap%
\pgfsetroundjoin%
\definecolor{currentfill}{rgb}{1.000000,0.894118,0.788235}%
\pgfsetfillcolor{currentfill}%
\pgfsetlinewidth{0.000000pt}%
\definecolor{currentstroke}{rgb}{1.000000,0.894118,0.788235}%
\pgfsetstrokecolor{currentstroke}%
\pgfsetdash{}{0pt}%
\pgfpathmoveto{\pgfqpoint{3.425935in}{3.198617in}}%
\pgfpathlineto{\pgfqpoint{3.536832in}{3.262643in}}%
\pgfpathlineto{\pgfqpoint{3.468636in}{3.352496in}}%
\pgfpathlineto{\pgfqpoint{3.357740in}{3.288470in}}%
\pgfpathlineto{\pgfqpoint{3.425935in}{3.198617in}}%
\pgfpathclose%
\pgfusepath{fill}%
\end{pgfscope}%
\begin{pgfscope}%
\pgfpathrectangle{\pgfqpoint{0.765000in}{0.660000in}}{\pgfqpoint{4.620000in}{4.620000in}}%
\pgfusepath{clip}%
\pgfsetbuttcap%
\pgfsetroundjoin%
\definecolor{currentfill}{rgb}{1.000000,0.894118,0.788235}%
\pgfsetfillcolor{currentfill}%
\pgfsetlinewidth{0.000000pt}%
\definecolor{currentstroke}{rgb}{1.000000,0.894118,0.788235}%
\pgfsetstrokecolor{currentstroke}%
\pgfsetdash{}{0pt}%
\pgfpathmoveto{\pgfqpoint{3.425935in}{3.454722in}}%
\pgfpathlineto{\pgfqpoint{3.315038in}{3.390696in}}%
\pgfpathlineto{\pgfqpoint{3.246843in}{3.480548in}}%
\pgfpathlineto{\pgfqpoint{3.357740in}{3.544575in}}%
\pgfpathlineto{\pgfqpoint{3.425935in}{3.454722in}}%
\pgfpathclose%
\pgfusepath{fill}%
\end{pgfscope}%
\begin{pgfscope}%
\pgfpathrectangle{\pgfqpoint{0.765000in}{0.660000in}}{\pgfqpoint{4.620000in}{4.620000in}}%
\pgfusepath{clip}%
\pgfsetbuttcap%
\pgfsetroundjoin%
\definecolor{currentfill}{rgb}{1.000000,0.894118,0.788235}%
\pgfsetfillcolor{currentfill}%
\pgfsetlinewidth{0.000000pt}%
\definecolor{currentstroke}{rgb}{1.000000,0.894118,0.788235}%
\pgfsetstrokecolor{currentstroke}%
\pgfsetdash{}{0pt}%
\pgfpathmoveto{\pgfqpoint{3.536832in}{3.390696in}}%
\pgfpathlineto{\pgfqpoint{3.425935in}{3.454722in}}%
\pgfpathlineto{\pgfqpoint{3.357740in}{3.544575in}}%
\pgfpathlineto{\pgfqpoint{3.468636in}{3.480548in}}%
\pgfpathlineto{\pgfqpoint{3.536832in}{3.390696in}}%
\pgfpathclose%
\pgfusepath{fill}%
\end{pgfscope}%
\begin{pgfscope}%
\pgfpathrectangle{\pgfqpoint{0.765000in}{0.660000in}}{\pgfqpoint{4.620000in}{4.620000in}}%
\pgfusepath{clip}%
\pgfsetbuttcap%
\pgfsetroundjoin%
\definecolor{currentfill}{rgb}{1.000000,0.894118,0.788235}%
\pgfsetfillcolor{currentfill}%
\pgfsetlinewidth{0.000000pt}%
\definecolor{currentstroke}{rgb}{1.000000,0.894118,0.788235}%
\pgfsetstrokecolor{currentstroke}%
\pgfsetdash{}{0pt}%
\pgfpathmoveto{\pgfqpoint{3.315038in}{3.262643in}}%
\pgfpathlineto{\pgfqpoint{3.315038in}{3.390696in}}%
\pgfpathlineto{\pgfqpoint{3.246843in}{3.480548in}}%
\pgfpathlineto{\pgfqpoint{3.357740in}{3.288470in}}%
\pgfpathlineto{\pgfqpoint{3.315038in}{3.262643in}}%
\pgfpathclose%
\pgfusepath{fill}%
\end{pgfscope}%
\begin{pgfscope}%
\pgfpathrectangle{\pgfqpoint{0.765000in}{0.660000in}}{\pgfqpoint{4.620000in}{4.620000in}}%
\pgfusepath{clip}%
\pgfsetbuttcap%
\pgfsetroundjoin%
\definecolor{currentfill}{rgb}{1.000000,0.894118,0.788235}%
\pgfsetfillcolor{currentfill}%
\pgfsetlinewidth{0.000000pt}%
\definecolor{currentstroke}{rgb}{1.000000,0.894118,0.788235}%
\pgfsetstrokecolor{currentstroke}%
\pgfsetdash{}{0pt}%
\pgfpathmoveto{\pgfqpoint{3.425935in}{3.198617in}}%
\pgfpathlineto{\pgfqpoint{3.425935in}{3.326669in}}%
\pgfpathlineto{\pgfqpoint{3.357740in}{3.416522in}}%
\pgfpathlineto{\pgfqpoint{3.246843in}{3.352496in}}%
\pgfpathlineto{\pgfqpoint{3.425935in}{3.198617in}}%
\pgfpathclose%
\pgfusepath{fill}%
\end{pgfscope}%
\begin{pgfscope}%
\pgfpathrectangle{\pgfqpoint{0.765000in}{0.660000in}}{\pgfqpoint{4.620000in}{4.620000in}}%
\pgfusepath{clip}%
\pgfsetbuttcap%
\pgfsetroundjoin%
\definecolor{currentfill}{rgb}{1.000000,0.894118,0.788235}%
\pgfsetfillcolor{currentfill}%
\pgfsetlinewidth{0.000000pt}%
\definecolor{currentstroke}{rgb}{1.000000,0.894118,0.788235}%
\pgfsetstrokecolor{currentstroke}%
\pgfsetdash{}{0pt}%
\pgfpathmoveto{\pgfqpoint{3.425935in}{3.326669in}}%
\pgfpathlineto{\pgfqpoint{3.536832in}{3.390696in}}%
\pgfpathlineto{\pgfqpoint{3.468636in}{3.480548in}}%
\pgfpathlineto{\pgfqpoint{3.357740in}{3.416522in}}%
\pgfpathlineto{\pgfqpoint{3.425935in}{3.326669in}}%
\pgfpathclose%
\pgfusepath{fill}%
\end{pgfscope}%
\begin{pgfscope}%
\pgfpathrectangle{\pgfqpoint{0.765000in}{0.660000in}}{\pgfqpoint{4.620000in}{4.620000in}}%
\pgfusepath{clip}%
\pgfsetbuttcap%
\pgfsetroundjoin%
\definecolor{currentfill}{rgb}{1.000000,0.894118,0.788235}%
\pgfsetfillcolor{currentfill}%
\pgfsetlinewidth{0.000000pt}%
\definecolor{currentstroke}{rgb}{1.000000,0.894118,0.788235}%
\pgfsetstrokecolor{currentstroke}%
\pgfsetdash{}{0pt}%
\pgfpathmoveto{\pgfqpoint{3.315038in}{3.390696in}}%
\pgfpathlineto{\pgfqpoint{3.425935in}{3.326669in}}%
\pgfpathlineto{\pgfqpoint{3.357740in}{3.416522in}}%
\pgfpathlineto{\pgfqpoint{3.246843in}{3.480548in}}%
\pgfpathlineto{\pgfqpoint{3.315038in}{3.390696in}}%
\pgfpathclose%
\pgfusepath{fill}%
\end{pgfscope}%
\begin{pgfscope}%
\pgfpathrectangle{\pgfqpoint{0.765000in}{0.660000in}}{\pgfqpoint{4.620000in}{4.620000in}}%
\pgfusepath{clip}%
\pgfsetbuttcap%
\pgfsetroundjoin%
\definecolor{currentfill}{rgb}{1.000000,0.894118,0.788235}%
\pgfsetfillcolor{currentfill}%
\pgfsetlinewidth{0.000000pt}%
\definecolor{currentstroke}{rgb}{1.000000,0.894118,0.788235}%
\pgfsetstrokecolor{currentstroke}%
\pgfsetdash{}{0pt}%
\pgfpathmoveto{\pgfqpoint{3.425935in}{3.326669in}}%
\pgfpathlineto{\pgfqpoint{3.315038in}{3.262643in}}%
\pgfpathlineto{\pgfqpoint{3.425935in}{3.198617in}}%
\pgfpathlineto{\pgfqpoint{3.536832in}{3.262643in}}%
\pgfpathlineto{\pgfqpoint{3.425935in}{3.326669in}}%
\pgfpathclose%
\pgfusepath{fill}%
\end{pgfscope}%
\begin{pgfscope}%
\pgfpathrectangle{\pgfqpoint{0.765000in}{0.660000in}}{\pgfqpoint{4.620000in}{4.620000in}}%
\pgfusepath{clip}%
\pgfsetbuttcap%
\pgfsetroundjoin%
\definecolor{currentfill}{rgb}{1.000000,0.894118,0.788235}%
\pgfsetfillcolor{currentfill}%
\pgfsetlinewidth{0.000000pt}%
\definecolor{currentstroke}{rgb}{1.000000,0.894118,0.788235}%
\pgfsetstrokecolor{currentstroke}%
\pgfsetdash{}{0pt}%
\pgfpathmoveto{\pgfqpoint{3.425935in}{3.326669in}}%
\pgfpathlineto{\pgfqpoint{3.315038in}{3.262643in}}%
\pgfpathlineto{\pgfqpoint{3.315038in}{3.390696in}}%
\pgfpathlineto{\pgfqpoint{3.425935in}{3.454722in}}%
\pgfpathlineto{\pgfqpoint{3.425935in}{3.326669in}}%
\pgfpathclose%
\pgfusepath{fill}%
\end{pgfscope}%
\begin{pgfscope}%
\pgfpathrectangle{\pgfqpoint{0.765000in}{0.660000in}}{\pgfqpoint{4.620000in}{4.620000in}}%
\pgfusepath{clip}%
\pgfsetbuttcap%
\pgfsetroundjoin%
\definecolor{currentfill}{rgb}{1.000000,0.894118,0.788235}%
\pgfsetfillcolor{currentfill}%
\pgfsetlinewidth{0.000000pt}%
\definecolor{currentstroke}{rgb}{1.000000,0.894118,0.788235}%
\pgfsetstrokecolor{currentstroke}%
\pgfsetdash{}{0pt}%
\pgfpathmoveto{\pgfqpoint{3.425935in}{3.326669in}}%
\pgfpathlineto{\pgfqpoint{3.536832in}{3.262643in}}%
\pgfpathlineto{\pgfqpoint{3.536832in}{3.390696in}}%
\pgfpathlineto{\pgfqpoint{3.425935in}{3.454722in}}%
\pgfpathlineto{\pgfqpoint{3.425935in}{3.326669in}}%
\pgfpathclose%
\pgfusepath{fill}%
\end{pgfscope}%
\begin{pgfscope}%
\pgfpathrectangle{\pgfqpoint{0.765000in}{0.660000in}}{\pgfqpoint{4.620000in}{4.620000in}}%
\pgfusepath{clip}%
\pgfsetbuttcap%
\pgfsetroundjoin%
\definecolor{currentfill}{rgb}{1.000000,0.894118,0.788235}%
\pgfsetfillcolor{currentfill}%
\pgfsetlinewidth{0.000000pt}%
\definecolor{currentstroke}{rgb}{1.000000,0.894118,0.788235}%
\pgfsetstrokecolor{currentstroke}%
\pgfsetdash{}{0pt}%
\pgfpathmoveto{\pgfqpoint{3.575586in}{2.945282in}}%
\pgfpathlineto{\pgfqpoint{3.464690in}{2.881255in}}%
\pgfpathlineto{\pgfqpoint{3.575586in}{2.817229in}}%
\pgfpathlineto{\pgfqpoint{3.686483in}{2.881255in}}%
\pgfpathlineto{\pgfqpoint{3.575586in}{2.945282in}}%
\pgfpathclose%
\pgfusepath{fill}%
\end{pgfscope}%
\begin{pgfscope}%
\pgfpathrectangle{\pgfqpoint{0.765000in}{0.660000in}}{\pgfqpoint{4.620000in}{4.620000in}}%
\pgfusepath{clip}%
\pgfsetbuttcap%
\pgfsetroundjoin%
\definecolor{currentfill}{rgb}{1.000000,0.894118,0.788235}%
\pgfsetfillcolor{currentfill}%
\pgfsetlinewidth{0.000000pt}%
\definecolor{currentstroke}{rgb}{1.000000,0.894118,0.788235}%
\pgfsetstrokecolor{currentstroke}%
\pgfsetdash{}{0pt}%
\pgfpathmoveto{\pgfqpoint{3.575586in}{2.945282in}}%
\pgfpathlineto{\pgfqpoint{3.464690in}{2.881255in}}%
\pgfpathlineto{\pgfqpoint{3.464690in}{3.009308in}}%
\pgfpathlineto{\pgfqpoint{3.575586in}{3.073334in}}%
\pgfpathlineto{\pgfqpoint{3.575586in}{2.945282in}}%
\pgfpathclose%
\pgfusepath{fill}%
\end{pgfscope}%
\begin{pgfscope}%
\pgfpathrectangle{\pgfqpoint{0.765000in}{0.660000in}}{\pgfqpoint{4.620000in}{4.620000in}}%
\pgfusepath{clip}%
\pgfsetbuttcap%
\pgfsetroundjoin%
\definecolor{currentfill}{rgb}{1.000000,0.894118,0.788235}%
\pgfsetfillcolor{currentfill}%
\pgfsetlinewidth{0.000000pt}%
\definecolor{currentstroke}{rgb}{1.000000,0.894118,0.788235}%
\pgfsetstrokecolor{currentstroke}%
\pgfsetdash{}{0pt}%
\pgfpathmoveto{\pgfqpoint{3.575586in}{2.945282in}}%
\pgfpathlineto{\pgfqpoint{3.686483in}{2.881255in}}%
\pgfpathlineto{\pgfqpoint{3.686483in}{3.009308in}}%
\pgfpathlineto{\pgfqpoint{3.575586in}{3.073334in}}%
\pgfpathlineto{\pgfqpoint{3.575586in}{2.945282in}}%
\pgfpathclose%
\pgfusepath{fill}%
\end{pgfscope}%
\begin{pgfscope}%
\pgfpathrectangle{\pgfqpoint{0.765000in}{0.660000in}}{\pgfqpoint{4.620000in}{4.620000in}}%
\pgfusepath{clip}%
\pgfsetbuttcap%
\pgfsetroundjoin%
\definecolor{currentfill}{rgb}{1.000000,0.894118,0.788235}%
\pgfsetfillcolor{currentfill}%
\pgfsetlinewidth{0.000000pt}%
\definecolor{currentstroke}{rgb}{1.000000,0.894118,0.788235}%
\pgfsetstrokecolor{currentstroke}%
\pgfsetdash{}{0pt}%
\pgfpathmoveto{\pgfqpoint{3.425935in}{3.454722in}}%
\pgfpathlineto{\pgfqpoint{3.315038in}{3.390696in}}%
\pgfpathlineto{\pgfqpoint{3.425935in}{3.326669in}}%
\pgfpathlineto{\pgfqpoint{3.536832in}{3.390696in}}%
\pgfpathlineto{\pgfqpoint{3.425935in}{3.454722in}}%
\pgfpathclose%
\pgfusepath{fill}%
\end{pgfscope}%
\begin{pgfscope}%
\pgfpathrectangle{\pgfqpoint{0.765000in}{0.660000in}}{\pgfqpoint{4.620000in}{4.620000in}}%
\pgfusepath{clip}%
\pgfsetbuttcap%
\pgfsetroundjoin%
\definecolor{currentfill}{rgb}{1.000000,0.894118,0.788235}%
\pgfsetfillcolor{currentfill}%
\pgfsetlinewidth{0.000000pt}%
\definecolor{currentstroke}{rgb}{1.000000,0.894118,0.788235}%
\pgfsetstrokecolor{currentstroke}%
\pgfsetdash{}{0pt}%
\pgfpathmoveto{\pgfqpoint{3.425935in}{3.198617in}}%
\pgfpathlineto{\pgfqpoint{3.536832in}{3.262643in}}%
\pgfpathlineto{\pgfqpoint{3.536832in}{3.390696in}}%
\pgfpathlineto{\pgfqpoint{3.425935in}{3.326669in}}%
\pgfpathlineto{\pgfqpoint{3.425935in}{3.198617in}}%
\pgfpathclose%
\pgfusepath{fill}%
\end{pgfscope}%
\begin{pgfscope}%
\pgfpathrectangle{\pgfqpoint{0.765000in}{0.660000in}}{\pgfqpoint{4.620000in}{4.620000in}}%
\pgfusepath{clip}%
\pgfsetbuttcap%
\pgfsetroundjoin%
\definecolor{currentfill}{rgb}{1.000000,0.894118,0.788235}%
\pgfsetfillcolor{currentfill}%
\pgfsetlinewidth{0.000000pt}%
\definecolor{currentstroke}{rgb}{1.000000,0.894118,0.788235}%
\pgfsetstrokecolor{currentstroke}%
\pgfsetdash{}{0pt}%
\pgfpathmoveto{\pgfqpoint{3.315038in}{3.262643in}}%
\pgfpathlineto{\pgfqpoint{3.425935in}{3.198617in}}%
\pgfpathlineto{\pgfqpoint{3.425935in}{3.326669in}}%
\pgfpathlineto{\pgfqpoint{3.315038in}{3.390696in}}%
\pgfpathlineto{\pgfqpoint{3.315038in}{3.262643in}}%
\pgfpathclose%
\pgfusepath{fill}%
\end{pgfscope}%
\begin{pgfscope}%
\pgfpathrectangle{\pgfqpoint{0.765000in}{0.660000in}}{\pgfqpoint{4.620000in}{4.620000in}}%
\pgfusepath{clip}%
\pgfsetbuttcap%
\pgfsetroundjoin%
\definecolor{currentfill}{rgb}{1.000000,0.894118,0.788235}%
\pgfsetfillcolor{currentfill}%
\pgfsetlinewidth{0.000000pt}%
\definecolor{currentstroke}{rgb}{1.000000,0.894118,0.788235}%
\pgfsetstrokecolor{currentstroke}%
\pgfsetdash{}{0pt}%
\pgfpathmoveto{\pgfqpoint{3.575586in}{3.073334in}}%
\pgfpathlineto{\pgfqpoint{3.464690in}{3.009308in}}%
\pgfpathlineto{\pgfqpoint{3.575586in}{2.945282in}}%
\pgfpathlineto{\pgfqpoint{3.686483in}{3.009308in}}%
\pgfpathlineto{\pgfqpoint{3.575586in}{3.073334in}}%
\pgfpathclose%
\pgfusepath{fill}%
\end{pgfscope}%
\begin{pgfscope}%
\pgfpathrectangle{\pgfqpoint{0.765000in}{0.660000in}}{\pgfqpoint{4.620000in}{4.620000in}}%
\pgfusepath{clip}%
\pgfsetbuttcap%
\pgfsetroundjoin%
\definecolor{currentfill}{rgb}{1.000000,0.894118,0.788235}%
\pgfsetfillcolor{currentfill}%
\pgfsetlinewidth{0.000000pt}%
\definecolor{currentstroke}{rgb}{1.000000,0.894118,0.788235}%
\pgfsetstrokecolor{currentstroke}%
\pgfsetdash{}{0pt}%
\pgfpathmoveto{\pgfqpoint{3.575586in}{2.817229in}}%
\pgfpathlineto{\pgfqpoint{3.686483in}{2.881255in}}%
\pgfpathlineto{\pgfqpoint{3.686483in}{3.009308in}}%
\pgfpathlineto{\pgfqpoint{3.575586in}{2.945282in}}%
\pgfpathlineto{\pgfqpoint{3.575586in}{2.817229in}}%
\pgfpathclose%
\pgfusepath{fill}%
\end{pgfscope}%
\begin{pgfscope}%
\pgfpathrectangle{\pgfqpoint{0.765000in}{0.660000in}}{\pgfqpoint{4.620000in}{4.620000in}}%
\pgfusepath{clip}%
\pgfsetbuttcap%
\pgfsetroundjoin%
\definecolor{currentfill}{rgb}{1.000000,0.894118,0.788235}%
\pgfsetfillcolor{currentfill}%
\pgfsetlinewidth{0.000000pt}%
\definecolor{currentstroke}{rgb}{1.000000,0.894118,0.788235}%
\pgfsetstrokecolor{currentstroke}%
\pgfsetdash{}{0pt}%
\pgfpathmoveto{\pgfqpoint{3.464690in}{2.881255in}}%
\pgfpathlineto{\pgfqpoint{3.575586in}{2.817229in}}%
\pgfpathlineto{\pgfqpoint{3.575586in}{2.945282in}}%
\pgfpathlineto{\pgfqpoint{3.464690in}{3.009308in}}%
\pgfpathlineto{\pgfqpoint{3.464690in}{2.881255in}}%
\pgfpathclose%
\pgfusepath{fill}%
\end{pgfscope}%
\begin{pgfscope}%
\pgfpathrectangle{\pgfqpoint{0.765000in}{0.660000in}}{\pgfqpoint{4.620000in}{4.620000in}}%
\pgfusepath{clip}%
\pgfsetbuttcap%
\pgfsetroundjoin%
\definecolor{currentfill}{rgb}{1.000000,0.894118,0.788235}%
\pgfsetfillcolor{currentfill}%
\pgfsetlinewidth{0.000000pt}%
\definecolor{currentstroke}{rgb}{1.000000,0.894118,0.788235}%
\pgfsetstrokecolor{currentstroke}%
\pgfsetdash{}{0pt}%
\pgfpathmoveto{\pgfqpoint{3.425935in}{3.326669in}}%
\pgfpathlineto{\pgfqpoint{3.315038in}{3.262643in}}%
\pgfpathlineto{\pgfqpoint{3.464690in}{2.881255in}}%
\pgfpathlineto{\pgfqpoint{3.575586in}{2.945282in}}%
\pgfpathlineto{\pgfqpoint{3.425935in}{3.326669in}}%
\pgfpathclose%
\pgfusepath{fill}%
\end{pgfscope}%
\begin{pgfscope}%
\pgfpathrectangle{\pgfqpoint{0.765000in}{0.660000in}}{\pgfqpoint{4.620000in}{4.620000in}}%
\pgfusepath{clip}%
\pgfsetbuttcap%
\pgfsetroundjoin%
\definecolor{currentfill}{rgb}{1.000000,0.894118,0.788235}%
\pgfsetfillcolor{currentfill}%
\pgfsetlinewidth{0.000000pt}%
\definecolor{currentstroke}{rgb}{1.000000,0.894118,0.788235}%
\pgfsetstrokecolor{currentstroke}%
\pgfsetdash{}{0pt}%
\pgfpathmoveto{\pgfqpoint{3.536832in}{3.262643in}}%
\pgfpathlineto{\pgfqpoint{3.425935in}{3.326669in}}%
\pgfpathlineto{\pgfqpoint{3.575586in}{2.945282in}}%
\pgfpathlineto{\pgfqpoint{3.686483in}{2.881255in}}%
\pgfpathlineto{\pgfqpoint{3.536832in}{3.262643in}}%
\pgfpathclose%
\pgfusepath{fill}%
\end{pgfscope}%
\begin{pgfscope}%
\pgfpathrectangle{\pgfqpoint{0.765000in}{0.660000in}}{\pgfqpoint{4.620000in}{4.620000in}}%
\pgfusepath{clip}%
\pgfsetbuttcap%
\pgfsetroundjoin%
\definecolor{currentfill}{rgb}{1.000000,0.894118,0.788235}%
\pgfsetfillcolor{currentfill}%
\pgfsetlinewidth{0.000000pt}%
\definecolor{currentstroke}{rgb}{1.000000,0.894118,0.788235}%
\pgfsetstrokecolor{currentstroke}%
\pgfsetdash{}{0pt}%
\pgfpathmoveto{\pgfqpoint{3.425935in}{3.326669in}}%
\pgfpathlineto{\pgfqpoint{3.425935in}{3.454722in}}%
\pgfpathlineto{\pgfqpoint{3.575586in}{3.073334in}}%
\pgfpathlineto{\pgfqpoint{3.686483in}{2.881255in}}%
\pgfpathlineto{\pgfqpoint{3.425935in}{3.326669in}}%
\pgfpathclose%
\pgfusepath{fill}%
\end{pgfscope}%
\begin{pgfscope}%
\pgfpathrectangle{\pgfqpoint{0.765000in}{0.660000in}}{\pgfqpoint{4.620000in}{4.620000in}}%
\pgfusepath{clip}%
\pgfsetbuttcap%
\pgfsetroundjoin%
\definecolor{currentfill}{rgb}{1.000000,0.894118,0.788235}%
\pgfsetfillcolor{currentfill}%
\pgfsetlinewidth{0.000000pt}%
\definecolor{currentstroke}{rgb}{1.000000,0.894118,0.788235}%
\pgfsetstrokecolor{currentstroke}%
\pgfsetdash{}{0pt}%
\pgfpathmoveto{\pgfqpoint{3.536832in}{3.262643in}}%
\pgfpathlineto{\pgfqpoint{3.536832in}{3.390696in}}%
\pgfpathlineto{\pgfqpoint{3.686483in}{3.009308in}}%
\pgfpathlineto{\pgfqpoint{3.575586in}{2.945282in}}%
\pgfpathlineto{\pgfqpoint{3.536832in}{3.262643in}}%
\pgfpathclose%
\pgfusepath{fill}%
\end{pgfscope}%
\begin{pgfscope}%
\pgfpathrectangle{\pgfqpoint{0.765000in}{0.660000in}}{\pgfqpoint{4.620000in}{4.620000in}}%
\pgfusepath{clip}%
\pgfsetbuttcap%
\pgfsetroundjoin%
\definecolor{currentfill}{rgb}{1.000000,0.894118,0.788235}%
\pgfsetfillcolor{currentfill}%
\pgfsetlinewidth{0.000000pt}%
\definecolor{currentstroke}{rgb}{1.000000,0.894118,0.788235}%
\pgfsetstrokecolor{currentstroke}%
\pgfsetdash{}{0pt}%
\pgfpathmoveto{\pgfqpoint{3.315038in}{3.262643in}}%
\pgfpathlineto{\pgfqpoint{3.425935in}{3.198617in}}%
\pgfpathlineto{\pgfqpoint{3.575586in}{2.817229in}}%
\pgfpathlineto{\pgfqpoint{3.464690in}{2.881255in}}%
\pgfpathlineto{\pgfqpoint{3.315038in}{3.262643in}}%
\pgfpathclose%
\pgfusepath{fill}%
\end{pgfscope}%
\begin{pgfscope}%
\pgfpathrectangle{\pgfqpoint{0.765000in}{0.660000in}}{\pgfqpoint{4.620000in}{4.620000in}}%
\pgfusepath{clip}%
\pgfsetbuttcap%
\pgfsetroundjoin%
\definecolor{currentfill}{rgb}{1.000000,0.894118,0.788235}%
\pgfsetfillcolor{currentfill}%
\pgfsetlinewidth{0.000000pt}%
\definecolor{currentstroke}{rgb}{1.000000,0.894118,0.788235}%
\pgfsetstrokecolor{currentstroke}%
\pgfsetdash{}{0pt}%
\pgfpathmoveto{\pgfqpoint{3.425935in}{3.198617in}}%
\pgfpathlineto{\pgfqpoint{3.536832in}{3.262643in}}%
\pgfpathlineto{\pgfqpoint{3.686483in}{2.881255in}}%
\pgfpathlineto{\pgfqpoint{3.575586in}{2.817229in}}%
\pgfpathlineto{\pgfqpoint{3.425935in}{3.198617in}}%
\pgfpathclose%
\pgfusepath{fill}%
\end{pgfscope}%
\begin{pgfscope}%
\pgfpathrectangle{\pgfqpoint{0.765000in}{0.660000in}}{\pgfqpoint{4.620000in}{4.620000in}}%
\pgfusepath{clip}%
\pgfsetbuttcap%
\pgfsetroundjoin%
\definecolor{currentfill}{rgb}{1.000000,0.894118,0.788235}%
\pgfsetfillcolor{currentfill}%
\pgfsetlinewidth{0.000000pt}%
\definecolor{currentstroke}{rgb}{1.000000,0.894118,0.788235}%
\pgfsetstrokecolor{currentstroke}%
\pgfsetdash{}{0pt}%
\pgfpathmoveto{\pgfqpoint{3.425935in}{3.454722in}}%
\pgfpathlineto{\pgfqpoint{3.315038in}{3.390696in}}%
\pgfpathlineto{\pgfqpoint{3.464690in}{3.009308in}}%
\pgfpathlineto{\pgfqpoint{3.575586in}{3.073334in}}%
\pgfpathlineto{\pgfqpoint{3.425935in}{3.454722in}}%
\pgfpathclose%
\pgfusepath{fill}%
\end{pgfscope}%
\begin{pgfscope}%
\pgfpathrectangle{\pgfqpoint{0.765000in}{0.660000in}}{\pgfqpoint{4.620000in}{4.620000in}}%
\pgfusepath{clip}%
\pgfsetbuttcap%
\pgfsetroundjoin%
\definecolor{currentfill}{rgb}{1.000000,0.894118,0.788235}%
\pgfsetfillcolor{currentfill}%
\pgfsetlinewidth{0.000000pt}%
\definecolor{currentstroke}{rgb}{1.000000,0.894118,0.788235}%
\pgfsetstrokecolor{currentstroke}%
\pgfsetdash{}{0pt}%
\pgfpathmoveto{\pgfqpoint{3.536832in}{3.390696in}}%
\pgfpathlineto{\pgfqpoint{3.425935in}{3.454722in}}%
\pgfpathlineto{\pgfqpoint{3.575586in}{3.073334in}}%
\pgfpathlineto{\pgfqpoint{3.686483in}{3.009308in}}%
\pgfpathlineto{\pgfqpoint{3.536832in}{3.390696in}}%
\pgfpathclose%
\pgfusepath{fill}%
\end{pgfscope}%
\begin{pgfscope}%
\pgfpathrectangle{\pgfqpoint{0.765000in}{0.660000in}}{\pgfqpoint{4.620000in}{4.620000in}}%
\pgfusepath{clip}%
\pgfsetbuttcap%
\pgfsetroundjoin%
\definecolor{currentfill}{rgb}{1.000000,0.894118,0.788235}%
\pgfsetfillcolor{currentfill}%
\pgfsetlinewidth{0.000000pt}%
\definecolor{currentstroke}{rgb}{1.000000,0.894118,0.788235}%
\pgfsetstrokecolor{currentstroke}%
\pgfsetdash{}{0pt}%
\pgfpathmoveto{\pgfqpoint{3.315038in}{3.262643in}}%
\pgfpathlineto{\pgfqpoint{3.315038in}{3.390696in}}%
\pgfpathlineto{\pgfqpoint{3.464690in}{3.009308in}}%
\pgfpathlineto{\pgfqpoint{3.575586in}{2.817229in}}%
\pgfpathlineto{\pgfqpoint{3.315038in}{3.262643in}}%
\pgfpathclose%
\pgfusepath{fill}%
\end{pgfscope}%
\begin{pgfscope}%
\pgfpathrectangle{\pgfqpoint{0.765000in}{0.660000in}}{\pgfqpoint{4.620000in}{4.620000in}}%
\pgfusepath{clip}%
\pgfsetbuttcap%
\pgfsetroundjoin%
\definecolor{currentfill}{rgb}{1.000000,0.894118,0.788235}%
\pgfsetfillcolor{currentfill}%
\pgfsetlinewidth{0.000000pt}%
\definecolor{currentstroke}{rgb}{1.000000,0.894118,0.788235}%
\pgfsetstrokecolor{currentstroke}%
\pgfsetdash{}{0pt}%
\pgfpathmoveto{\pgfqpoint{3.425935in}{3.198617in}}%
\pgfpathlineto{\pgfqpoint{3.425935in}{3.326669in}}%
\pgfpathlineto{\pgfqpoint{3.575586in}{2.945282in}}%
\pgfpathlineto{\pgfqpoint{3.464690in}{2.881255in}}%
\pgfpathlineto{\pgfqpoint{3.425935in}{3.198617in}}%
\pgfpathclose%
\pgfusepath{fill}%
\end{pgfscope}%
\begin{pgfscope}%
\pgfpathrectangle{\pgfqpoint{0.765000in}{0.660000in}}{\pgfqpoint{4.620000in}{4.620000in}}%
\pgfusepath{clip}%
\pgfsetbuttcap%
\pgfsetroundjoin%
\definecolor{currentfill}{rgb}{1.000000,0.894118,0.788235}%
\pgfsetfillcolor{currentfill}%
\pgfsetlinewidth{0.000000pt}%
\definecolor{currentstroke}{rgb}{1.000000,0.894118,0.788235}%
\pgfsetstrokecolor{currentstroke}%
\pgfsetdash{}{0pt}%
\pgfpathmoveto{\pgfqpoint{3.425935in}{3.326669in}}%
\pgfpathlineto{\pgfqpoint{3.536832in}{3.390696in}}%
\pgfpathlineto{\pgfqpoint{3.686483in}{3.009308in}}%
\pgfpathlineto{\pgfqpoint{3.575586in}{2.945282in}}%
\pgfpathlineto{\pgfqpoint{3.425935in}{3.326669in}}%
\pgfpathclose%
\pgfusepath{fill}%
\end{pgfscope}%
\begin{pgfscope}%
\pgfpathrectangle{\pgfqpoint{0.765000in}{0.660000in}}{\pgfqpoint{4.620000in}{4.620000in}}%
\pgfusepath{clip}%
\pgfsetbuttcap%
\pgfsetroundjoin%
\definecolor{currentfill}{rgb}{1.000000,0.894118,0.788235}%
\pgfsetfillcolor{currentfill}%
\pgfsetlinewidth{0.000000pt}%
\definecolor{currentstroke}{rgb}{1.000000,0.894118,0.788235}%
\pgfsetstrokecolor{currentstroke}%
\pgfsetdash{}{0pt}%
\pgfpathmoveto{\pgfqpoint{3.315038in}{3.390696in}}%
\pgfpathlineto{\pgfqpoint{3.425935in}{3.326669in}}%
\pgfpathlineto{\pgfqpoint{3.575586in}{2.945282in}}%
\pgfpathlineto{\pgfqpoint{3.464690in}{3.009308in}}%
\pgfpathlineto{\pgfqpoint{3.315038in}{3.390696in}}%
\pgfpathclose%
\pgfusepath{fill}%
\end{pgfscope}%
\begin{pgfscope}%
\pgfpathrectangle{\pgfqpoint{0.765000in}{0.660000in}}{\pgfqpoint{4.620000in}{4.620000in}}%
\pgfusepath{clip}%
\pgfsetbuttcap%
\pgfsetroundjoin%
\definecolor{currentfill}{rgb}{1.000000,0.894118,0.788235}%
\pgfsetfillcolor{currentfill}%
\pgfsetlinewidth{0.000000pt}%
\definecolor{currentstroke}{rgb}{1.000000,0.894118,0.788235}%
\pgfsetstrokecolor{currentstroke}%
\pgfsetdash{}{0pt}%
\pgfpathmoveto{\pgfqpoint{3.357740in}{3.416522in}}%
\pgfpathlineto{\pgfqpoint{3.246843in}{3.352496in}}%
\pgfpathlineto{\pgfqpoint{3.357740in}{3.288470in}}%
\pgfpathlineto{\pgfqpoint{3.468636in}{3.352496in}}%
\pgfpathlineto{\pgfqpoint{3.357740in}{3.416522in}}%
\pgfpathclose%
\pgfusepath{fill}%
\end{pgfscope}%
\begin{pgfscope}%
\pgfpathrectangle{\pgfqpoint{0.765000in}{0.660000in}}{\pgfqpoint{4.620000in}{4.620000in}}%
\pgfusepath{clip}%
\pgfsetbuttcap%
\pgfsetroundjoin%
\definecolor{currentfill}{rgb}{1.000000,0.894118,0.788235}%
\pgfsetfillcolor{currentfill}%
\pgfsetlinewidth{0.000000pt}%
\definecolor{currentstroke}{rgb}{1.000000,0.894118,0.788235}%
\pgfsetstrokecolor{currentstroke}%
\pgfsetdash{}{0pt}%
\pgfpathmoveto{\pgfqpoint{3.357740in}{3.416522in}}%
\pgfpathlineto{\pgfqpoint{3.246843in}{3.352496in}}%
\pgfpathlineto{\pgfqpoint{3.246843in}{3.480548in}}%
\pgfpathlineto{\pgfqpoint{3.357740in}{3.544575in}}%
\pgfpathlineto{\pgfqpoint{3.357740in}{3.416522in}}%
\pgfpathclose%
\pgfusepath{fill}%
\end{pgfscope}%
\begin{pgfscope}%
\pgfpathrectangle{\pgfqpoint{0.765000in}{0.660000in}}{\pgfqpoint{4.620000in}{4.620000in}}%
\pgfusepath{clip}%
\pgfsetbuttcap%
\pgfsetroundjoin%
\definecolor{currentfill}{rgb}{1.000000,0.894118,0.788235}%
\pgfsetfillcolor{currentfill}%
\pgfsetlinewidth{0.000000pt}%
\definecolor{currentstroke}{rgb}{1.000000,0.894118,0.788235}%
\pgfsetstrokecolor{currentstroke}%
\pgfsetdash{}{0pt}%
\pgfpathmoveto{\pgfqpoint{3.357740in}{3.416522in}}%
\pgfpathlineto{\pgfqpoint{3.468636in}{3.352496in}}%
\pgfpathlineto{\pgfqpoint{3.468636in}{3.480548in}}%
\pgfpathlineto{\pgfqpoint{3.357740in}{3.544575in}}%
\pgfpathlineto{\pgfqpoint{3.357740in}{3.416522in}}%
\pgfpathclose%
\pgfusepath{fill}%
\end{pgfscope}%
\begin{pgfscope}%
\pgfpathrectangle{\pgfqpoint{0.765000in}{0.660000in}}{\pgfqpoint{4.620000in}{4.620000in}}%
\pgfusepath{clip}%
\pgfsetbuttcap%
\pgfsetroundjoin%
\definecolor{currentfill}{rgb}{1.000000,0.894118,0.788235}%
\pgfsetfillcolor{currentfill}%
\pgfsetlinewidth{0.000000pt}%
\definecolor{currentstroke}{rgb}{1.000000,0.894118,0.788235}%
\pgfsetstrokecolor{currentstroke}%
\pgfsetdash{}{0pt}%
\pgfpathmoveto{\pgfqpoint{3.425935in}{3.326669in}}%
\pgfpathlineto{\pgfqpoint{3.315038in}{3.262643in}}%
\pgfpathlineto{\pgfqpoint{3.425935in}{3.198617in}}%
\pgfpathlineto{\pgfqpoint{3.536832in}{3.262643in}}%
\pgfpathlineto{\pgfqpoint{3.425935in}{3.326669in}}%
\pgfpathclose%
\pgfusepath{fill}%
\end{pgfscope}%
\begin{pgfscope}%
\pgfpathrectangle{\pgfqpoint{0.765000in}{0.660000in}}{\pgfqpoint{4.620000in}{4.620000in}}%
\pgfusepath{clip}%
\pgfsetbuttcap%
\pgfsetroundjoin%
\definecolor{currentfill}{rgb}{1.000000,0.894118,0.788235}%
\pgfsetfillcolor{currentfill}%
\pgfsetlinewidth{0.000000pt}%
\definecolor{currentstroke}{rgb}{1.000000,0.894118,0.788235}%
\pgfsetstrokecolor{currentstroke}%
\pgfsetdash{}{0pt}%
\pgfpathmoveto{\pgfqpoint{3.425935in}{3.326669in}}%
\pgfpathlineto{\pgfqpoint{3.315038in}{3.262643in}}%
\pgfpathlineto{\pgfqpoint{3.315038in}{3.390696in}}%
\pgfpathlineto{\pgfqpoint{3.425935in}{3.454722in}}%
\pgfpathlineto{\pgfqpoint{3.425935in}{3.326669in}}%
\pgfpathclose%
\pgfusepath{fill}%
\end{pgfscope}%
\begin{pgfscope}%
\pgfpathrectangle{\pgfqpoint{0.765000in}{0.660000in}}{\pgfqpoint{4.620000in}{4.620000in}}%
\pgfusepath{clip}%
\pgfsetbuttcap%
\pgfsetroundjoin%
\definecolor{currentfill}{rgb}{1.000000,0.894118,0.788235}%
\pgfsetfillcolor{currentfill}%
\pgfsetlinewidth{0.000000pt}%
\definecolor{currentstroke}{rgb}{1.000000,0.894118,0.788235}%
\pgfsetstrokecolor{currentstroke}%
\pgfsetdash{}{0pt}%
\pgfpathmoveto{\pgfqpoint{3.425935in}{3.326669in}}%
\pgfpathlineto{\pgfqpoint{3.536832in}{3.262643in}}%
\pgfpathlineto{\pgfqpoint{3.536832in}{3.390696in}}%
\pgfpathlineto{\pgfqpoint{3.425935in}{3.454722in}}%
\pgfpathlineto{\pgfqpoint{3.425935in}{3.326669in}}%
\pgfpathclose%
\pgfusepath{fill}%
\end{pgfscope}%
\begin{pgfscope}%
\pgfpathrectangle{\pgfqpoint{0.765000in}{0.660000in}}{\pgfqpoint{4.620000in}{4.620000in}}%
\pgfusepath{clip}%
\pgfsetbuttcap%
\pgfsetroundjoin%
\definecolor{currentfill}{rgb}{1.000000,0.894118,0.788235}%
\pgfsetfillcolor{currentfill}%
\pgfsetlinewidth{0.000000pt}%
\definecolor{currentstroke}{rgb}{1.000000,0.894118,0.788235}%
\pgfsetstrokecolor{currentstroke}%
\pgfsetdash{}{0pt}%
\pgfpathmoveto{\pgfqpoint{3.357740in}{3.544575in}}%
\pgfpathlineto{\pgfqpoint{3.246843in}{3.480548in}}%
\pgfpathlineto{\pgfqpoint{3.357740in}{3.416522in}}%
\pgfpathlineto{\pgfqpoint{3.468636in}{3.480548in}}%
\pgfpathlineto{\pgfqpoint{3.357740in}{3.544575in}}%
\pgfpathclose%
\pgfusepath{fill}%
\end{pgfscope}%
\begin{pgfscope}%
\pgfpathrectangle{\pgfqpoint{0.765000in}{0.660000in}}{\pgfqpoint{4.620000in}{4.620000in}}%
\pgfusepath{clip}%
\pgfsetbuttcap%
\pgfsetroundjoin%
\definecolor{currentfill}{rgb}{1.000000,0.894118,0.788235}%
\pgfsetfillcolor{currentfill}%
\pgfsetlinewidth{0.000000pt}%
\definecolor{currentstroke}{rgb}{1.000000,0.894118,0.788235}%
\pgfsetstrokecolor{currentstroke}%
\pgfsetdash{}{0pt}%
\pgfpathmoveto{\pgfqpoint{3.357740in}{3.288470in}}%
\pgfpathlineto{\pgfqpoint{3.468636in}{3.352496in}}%
\pgfpathlineto{\pgfqpoint{3.468636in}{3.480548in}}%
\pgfpathlineto{\pgfqpoint{3.357740in}{3.416522in}}%
\pgfpathlineto{\pgfqpoint{3.357740in}{3.288470in}}%
\pgfpathclose%
\pgfusepath{fill}%
\end{pgfscope}%
\begin{pgfscope}%
\pgfpathrectangle{\pgfqpoint{0.765000in}{0.660000in}}{\pgfqpoint{4.620000in}{4.620000in}}%
\pgfusepath{clip}%
\pgfsetbuttcap%
\pgfsetroundjoin%
\definecolor{currentfill}{rgb}{1.000000,0.894118,0.788235}%
\pgfsetfillcolor{currentfill}%
\pgfsetlinewidth{0.000000pt}%
\definecolor{currentstroke}{rgb}{1.000000,0.894118,0.788235}%
\pgfsetstrokecolor{currentstroke}%
\pgfsetdash{}{0pt}%
\pgfpathmoveto{\pgfqpoint{3.246843in}{3.352496in}}%
\pgfpathlineto{\pgfqpoint{3.357740in}{3.288470in}}%
\pgfpathlineto{\pgfqpoint{3.357740in}{3.416522in}}%
\pgfpathlineto{\pgfqpoint{3.246843in}{3.480548in}}%
\pgfpathlineto{\pgfqpoint{3.246843in}{3.352496in}}%
\pgfpathclose%
\pgfusepath{fill}%
\end{pgfscope}%
\begin{pgfscope}%
\pgfpathrectangle{\pgfqpoint{0.765000in}{0.660000in}}{\pgfqpoint{4.620000in}{4.620000in}}%
\pgfusepath{clip}%
\pgfsetbuttcap%
\pgfsetroundjoin%
\definecolor{currentfill}{rgb}{1.000000,0.894118,0.788235}%
\pgfsetfillcolor{currentfill}%
\pgfsetlinewidth{0.000000pt}%
\definecolor{currentstroke}{rgb}{1.000000,0.894118,0.788235}%
\pgfsetstrokecolor{currentstroke}%
\pgfsetdash{}{0pt}%
\pgfpathmoveto{\pgfqpoint{3.425935in}{3.454722in}}%
\pgfpathlineto{\pgfqpoint{3.315038in}{3.390696in}}%
\pgfpathlineto{\pgfqpoint{3.425935in}{3.326669in}}%
\pgfpathlineto{\pgfqpoint{3.536832in}{3.390696in}}%
\pgfpathlineto{\pgfqpoint{3.425935in}{3.454722in}}%
\pgfpathclose%
\pgfusepath{fill}%
\end{pgfscope}%
\begin{pgfscope}%
\pgfpathrectangle{\pgfqpoint{0.765000in}{0.660000in}}{\pgfqpoint{4.620000in}{4.620000in}}%
\pgfusepath{clip}%
\pgfsetbuttcap%
\pgfsetroundjoin%
\definecolor{currentfill}{rgb}{1.000000,0.894118,0.788235}%
\pgfsetfillcolor{currentfill}%
\pgfsetlinewidth{0.000000pt}%
\definecolor{currentstroke}{rgb}{1.000000,0.894118,0.788235}%
\pgfsetstrokecolor{currentstroke}%
\pgfsetdash{}{0pt}%
\pgfpathmoveto{\pgfqpoint{3.425935in}{3.198617in}}%
\pgfpathlineto{\pgfqpoint{3.536832in}{3.262643in}}%
\pgfpathlineto{\pgfqpoint{3.536832in}{3.390696in}}%
\pgfpathlineto{\pgfqpoint{3.425935in}{3.326669in}}%
\pgfpathlineto{\pgfqpoint{3.425935in}{3.198617in}}%
\pgfpathclose%
\pgfusepath{fill}%
\end{pgfscope}%
\begin{pgfscope}%
\pgfpathrectangle{\pgfqpoint{0.765000in}{0.660000in}}{\pgfqpoint{4.620000in}{4.620000in}}%
\pgfusepath{clip}%
\pgfsetbuttcap%
\pgfsetroundjoin%
\definecolor{currentfill}{rgb}{1.000000,0.894118,0.788235}%
\pgfsetfillcolor{currentfill}%
\pgfsetlinewidth{0.000000pt}%
\definecolor{currentstroke}{rgb}{1.000000,0.894118,0.788235}%
\pgfsetstrokecolor{currentstroke}%
\pgfsetdash{}{0pt}%
\pgfpathmoveto{\pgfqpoint{3.315038in}{3.262643in}}%
\pgfpathlineto{\pgfqpoint{3.425935in}{3.198617in}}%
\pgfpathlineto{\pgfqpoint{3.425935in}{3.326669in}}%
\pgfpathlineto{\pgfqpoint{3.315038in}{3.390696in}}%
\pgfpathlineto{\pgfqpoint{3.315038in}{3.262643in}}%
\pgfpathclose%
\pgfusepath{fill}%
\end{pgfscope}%
\begin{pgfscope}%
\pgfpathrectangle{\pgfqpoint{0.765000in}{0.660000in}}{\pgfqpoint{4.620000in}{4.620000in}}%
\pgfusepath{clip}%
\pgfsetbuttcap%
\pgfsetroundjoin%
\definecolor{currentfill}{rgb}{1.000000,0.894118,0.788235}%
\pgfsetfillcolor{currentfill}%
\pgfsetlinewidth{0.000000pt}%
\definecolor{currentstroke}{rgb}{1.000000,0.894118,0.788235}%
\pgfsetstrokecolor{currentstroke}%
\pgfsetdash{}{0pt}%
\pgfpathmoveto{\pgfqpoint{3.357740in}{3.416522in}}%
\pgfpathlineto{\pgfqpoint{3.246843in}{3.352496in}}%
\pgfpathlineto{\pgfqpoint{3.315038in}{3.262643in}}%
\pgfpathlineto{\pgfqpoint{3.425935in}{3.326669in}}%
\pgfpathlineto{\pgfqpoint{3.357740in}{3.416522in}}%
\pgfpathclose%
\pgfusepath{fill}%
\end{pgfscope}%
\begin{pgfscope}%
\pgfpathrectangle{\pgfqpoint{0.765000in}{0.660000in}}{\pgfqpoint{4.620000in}{4.620000in}}%
\pgfusepath{clip}%
\pgfsetbuttcap%
\pgfsetroundjoin%
\definecolor{currentfill}{rgb}{1.000000,0.894118,0.788235}%
\pgfsetfillcolor{currentfill}%
\pgfsetlinewidth{0.000000pt}%
\definecolor{currentstroke}{rgb}{1.000000,0.894118,0.788235}%
\pgfsetstrokecolor{currentstroke}%
\pgfsetdash{}{0pt}%
\pgfpathmoveto{\pgfqpoint{3.468636in}{3.352496in}}%
\pgfpathlineto{\pgfqpoint{3.357740in}{3.416522in}}%
\pgfpathlineto{\pgfqpoint{3.425935in}{3.326669in}}%
\pgfpathlineto{\pgfqpoint{3.536832in}{3.262643in}}%
\pgfpathlineto{\pgfqpoint{3.468636in}{3.352496in}}%
\pgfpathclose%
\pgfusepath{fill}%
\end{pgfscope}%
\begin{pgfscope}%
\pgfpathrectangle{\pgfqpoint{0.765000in}{0.660000in}}{\pgfqpoint{4.620000in}{4.620000in}}%
\pgfusepath{clip}%
\pgfsetbuttcap%
\pgfsetroundjoin%
\definecolor{currentfill}{rgb}{1.000000,0.894118,0.788235}%
\pgfsetfillcolor{currentfill}%
\pgfsetlinewidth{0.000000pt}%
\definecolor{currentstroke}{rgb}{1.000000,0.894118,0.788235}%
\pgfsetstrokecolor{currentstroke}%
\pgfsetdash{}{0pt}%
\pgfpathmoveto{\pgfqpoint{3.357740in}{3.416522in}}%
\pgfpathlineto{\pgfqpoint{3.357740in}{3.544575in}}%
\pgfpathlineto{\pgfqpoint{3.425935in}{3.454722in}}%
\pgfpathlineto{\pgfqpoint{3.536832in}{3.262643in}}%
\pgfpathlineto{\pgfqpoint{3.357740in}{3.416522in}}%
\pgfpathclose%
\pgfusepath{fill}%
\end{pgfscope}%
\begin{pgfscope}%
\pgfpathrectangle{\pgfqpoint{0.765000in}{0.660000in}}{\pgfqpoint{4.620000in}{4.620000in}}%
\pgfusepath{clip}%
\pgfsetbuttcap%
\pgfsetroundjoin%
\definecolor{currentfill}{rgb}{1.000000,0.894118,0.788235}%
\pgfsetfillcolor{currentfill}%
\pgfsetlinewidth{0.000000pt}%
\definecolor{currentstroke}{rgb}{1.000000,0.894118,0.788235}%
\pgfsetstrokecolor{currentstroke}%
\pgfsetdash{}{0pt}%
\pgfpathmoveto{\pgfqpoint{3.468636in}{3.352496in}}%
\pgfpathlineto{\pgfqpoint{3.468636in}{3.480548in}}%
\pgfpathlineto{\pgfqpoint{3.536832in}{3.390696in}}%
\pgfpathlineto{\pgfqpoint{3.425935in}{3.326669in}}%
\pgfpathlineto{\pgfqpoint{3.468636in}{3.352496in}}%
\pgfpathclose%
\pgfusepath{fill}%
\end{pgfscope}%
\begin{pgfscope}%
\pgfpathrectangle{\pgfqpoint{0.765000in}{0.660000in}}{\pgfqpoint{4.620000in}{4.620000in}}%
\pgfusepath{clip}%
\pgfsetbuttcap%
\pgfsetroundjoin%
\definecolor{currentfill}{rgb}{1.000000,0.894118,0.788235}%
\pgfsetfillcolor{currentfill}%
\pgfsetlinewidth{0.000000pt}%
\definecolor{currentstroke}{rgb}{1.000000,0.894118,0.788235}%
\pgfsetstrokecolor{currentstroke}%
\pgfsetdash{}{0pt}%
\pgfpathmoveto{\pgfqpoint{3.246843in}{3.352496in}}%
\pgfpathlineto{\pgfqpoint{3.357740in}{3.288470in}}%
\pgfpathlineto{\pgfqpoint{3.425935in}{3.198617in}}%
\pgfpathlineto{\pgfqpoint{3.315038in}{3.262643in}}%
\pgfpathlineto{\pgfqpoint{3.246843in}{3.352496in}}%
\pgfpathclose%
\pgfusepath{fill}%
\end{pgfscope}%
\begin{pgfscope}%
\pgfpathrectangle{\pgfqpoint{0.765000in}{0.660000in}}{\pgfqpoint{4.620000in}{4.620000in}}%
\pgfusepath{clip}%
\pgfsetbuttcap%
\pgfsetroundjoin%
\definecolor{currentfill}{rgb}{1.000000,0.894118,0.788235}%
\pgfsetfillcolor{currentfill}%
\pgfsetlinewidth{0.000000pt}%
\definecolor{currentstroke}{rgb}{1.000000,0.894118,0.788235}%
\pgfsetstrokecolor{currentstroke}%
\pgfsetdash{}{0pt}%
\pgfpathmoveto{\pgfqpoint{3.357740in}{3.288470in}}%
\pgfpathlineto{\pgfqpoint{3.468636in}{3.352496in}}%
\pgfpathlineto{\pgfqpoint{3.536832in}{3.262643in}}%
\pgfpathlineto{\pgfqpoint{3.425935in}{3.198617in}}%
\pgfpathlineto{\pgfqpoint{3.357740in}{3.288470in}}%
\pgfpathclose%
\pgfusepath{fill}%
\end{pgfscope}%
\begin{pgfscope}%
\pgfpathrectangle{\pgfqpoint{0.765000in}{0.660000in}}{\pgfqpoint{4.620000in}{4.620000in}}%
\pgfusepath{clip}%
\pgfsetbuttcap%
\pgfsetroundjoin%
\definecolor{currentfill}{rgb}{1.000000,0.894118,0.788235}%
\pgfsetfillcolor{currentfill}%
\pgfsetlinewidth{0.000000pt}%
\definecolor{currentstroke}{rgb}{1.000000,0.894118,0.788235}%
\pgfsetstrokecolor{currentstroke}%
\pgfsetdash{}{0pt}%
\pgfpathmoveto{\pgfqpoint{3.357740in}{3.544575in}}%
\pgfpathlineto{\pgfqpoint{3.246843in}{3.480548in}}%
\pgfpathlineto{\pgfqpoint{3.315038in}{3.390696in}}%
\pgfpathlineto{\pgfqpoint{3.425935in}{3.454722in}}%
\pgfpathlineto{\pgfqpoint{3.357740in}{3.544575in}}%
\pgfpathclose%
\pgfusepath{fill}%
\end{pgfscope}%
\begin{pgfscope}%
\pgfpathrectangle{\pgfqpoint{0.765000in}{0.660000in}}{\pgfqpoint{4.620000in}{4.620000in}}%
\pgfusepath{clip}%
\pgfsetbuttcap%
\pgfsetroundjoin%
\definecolor{currentfill}{rgb}{1.000000,0.894118,0.788235}%
\pgfsetfillcolor{currentfill}%
\pgfsetlinewidth{0.000000pt}%
\definecolor{currentstroke}{rgb}{1.000000,0.894118,0.788235}%
\pgfsetstrokecolor{currentstroke}%
\pgfsetdash{}{0pt}%
\pgfpathmoveto{\pgfqpoint{3.468636in}{3.480548in}}%
\pgfpathlineto{\pgfqpoint{3.357740in}{3.544575in}}%
\pgfpathlineto{\pgfqpoint{3.425935in}{3.454722in}}%
\pgfpathlineto{\pgfqpoint{3.536832in}{3.390696in}}%
\pgfpathlineto{\pgfqpoint{3.468636in}{3.480548in}}%
\pgfpathclose%
\pgfusepath{fill}%
\end{pgfscope}%
\begin{pgfscope}%
\pgfpathrectangle{\pgfqpoint{0.765000in}{0.660000in}}{\pgfqpoint{4.620000in}{4.620000in}}%
\pgfusepath{clip}%
\pgfsetbuttcap%
\pgfsetroundjoin%
\definecolor{currentfill}{rgb}{1.000000,0.894118,0.788235}%
\pgfsetfillcolor{currentfill}%
\pgfsetlinewidth{0.000000pt}%
\definecolor{currentstroke}{rgb}{1.000000,0.894118,0.788235}%
\pgfsetstrokecolor{currentstroke}%
\pgfsetdash{}{0pt}%
\pgfpathmoveto{\pgfqpoint{3.246843in}{3.352496in}}%
\pgfpathlineto{\pgfqpoint{3.246843in}{3.480548in}}%
\pgfpathlineto{\pgfqpoint{3.315038in}{3.390696in}}%
\pgfpathlineto{\pgfqpoint{3.425935in}{3.198617in}}%
\pgfpathlineto{\pgfqpoint{3.246843in}{3.352496in}}%
\pgfpathclose%
\pgfusepath{fill}%
\end{pgfscope}%
\begin{pgfscope}%
\pgfpathrectangle{\pgfqpoint{0.765000in}{0.660000in}}{\pgfqpoint{4.620000in}{4.620000in}}%
\pgfusepath{clip}%
\pgfsetbuttcap%
\pgfsetroundjoin%
\definecolor{currentfill}{rgb}{1.000000,0.894118,0.788235}%
\pgfsetfillcolor{currentfill}%
\pgfsetlinewidth{0.000000pt}%
\definecolor{currentstroke}{rgb}{1.000000,0.894118,0.788235}%
\pgfsetstrokecolor{currentstroke}%
\pgfsetdash{}{0pt}%
\pgfpathmoveto{\pgfqpoint{3.357740in}{3.288470in}}%
\pgfpathlineto{\pgfqpoint{3.357740in}{3.416522in}}%
\pgfpathlineto{\pgfqpoint{3.425935in}{3.326669in}}%
\pgfpathlineto{\pgfqpoint{3.315038in}{3.262643in}}%
\pgfpathlineto{\pgfqpoint{3.357740in}{3.288470in}}%
\pgfpathclose%
\pgfusepath{fill}%
\end{pgfscope}%
\begin{pgfscope}%
\pgfpathrectangle{\pgfqpoint{0.765000in}{0.660000in}}{\pgfqpoint{4.620000in}{4.620000in}}%
\pgfusepath{clip}%
\pgfsetbuttcap%
\pgfsetroundjoin%
\definecolor{currentfill}{rgb}{1.000000,0.894118,0.788235}%
\pgfsetfillcolor{currentfill}%
\pgfsetlinewidth{0.000000pt}%
\definecolor{currentstroke}{rgb}{1.000000,0.894118,0.788235}%
\pgfsetstrokecolor{currentstroke}%
\pgfsetdash{}{0pt}%
\pgfpathmoveto{\pgfqpoint{3.357740in}{3.416522in}}%
\pgfpathlineto{\pgfqpoint{3.468636in}{3.480548in}}%
\pgfpathlineto{\pgfqpoint{3.536832in}{3.390696in}}%
\pgfpathlineto{\pgfqpoint{3.425935in}{3.326669in}}%
\pgfpathlineto{\pgfqpoint{3.357740in}{3.416522in}}%
\pgfpathclose%
\pgfusepath{fill}%
\end{pgfscope}%
\begin{pgfscope}%
\pgfpathrectangle{\pgfqpoint{0.765000in}{0.660000in}}{\pgfqpoint{4.620000in}{4.620000in}}%
\pgfusepath{clip}%
\pgfsetbuttcap%
\pgfsetroundjoin%
\definecolor{currentfill}{rgb}{1.000000,0.894118,0.788235}%
\pgfsetfillcolor{currentfill}%
\pgfsetlinewidth{0.000000pt}%
\definecolor{currentstroke}{rgb}{1.000000,0.894118,0.788235}%
\pgfsetstrokecolor{currentstroke}%
\pgfsetdash{}{0pt}%
\pgfpathmoveto{\pgfqpoint{3.246843in}{3.480548in}}%
\pgfpathlineto{\pgfqpoint{3.357740in}{3.416522in}}%
\pgfpathlineto{\pgfqpoint{3.425935in}{3.326669in}}%
\pgfpathlineto{\pgfqpoint{3.315038in}{3.390696in}}%
\pgfpathlineto{\pgfqpoint{3.246843in}{3.480548in}}%
\pgfpathclose%
\pgfusepath{fill}%
\end{pgfscope}%
\begin{pgfscope}%
\pgfpathrectangle{\pgfqpoint{0.765000in}{0.660000in}}{\pgfqpoint{4.620000in}{4.620000in}}%
\pgfusepath{clip}%
\pgfsetbuttcap%
\pgfsetroundjoin%
\definecolor{currentfill}{rgb}{1.000000,0.894118,0.788235}%
\pgfsetfillcolor{currentfill}%
\pgfsetlinewidth{0.000000pt}%
\definecolor{currentstroke}{rgb}{1.000000,0.894118,0.788235}%
\pgfsetstrokecolor{currentstroke}%
\pgfsetdash{}{0pt}%
\pgfpathmoveto{\pgfqpoint{3.357740in}{3.416522in}}%
\pgfpathlineto{\pgfqpoint{3.246843in}{3.352496in}}%
\pgfpathlineto{\pgfqpoint{3.357740in}{3.288470in}}%
\pgfpathlineto{\pgfqpoint{3.468636in}{3.352496in}}%
\pgfpathlineto{\pgfqpoint{3.357740in}{3.416522in}}%
\pgfpathclose%
\pgfusepath{fill}%
\end{pgfscope}%
\begin{pgfscope}%
\pgfpathrectangle{\pgfqpoint{0.765000in}{0.660000in}}{\pgfqpoint{4.620000in}{4.620000in}}%
\pgfusepath{clip}%
\pgfsetbuttcap%
\pgfsetroundjoin%
\definecolor{currentfill}{rgb}{1.000000,0.894118,0.788235}%
\pgfsetfillcolor{currentfill}%
\pgfsetlinewidth{0.000000pt}%
\definecolor{currentstroke}{rgb}{1.000000,0.894118,0.788235}%
\pgfsetstrokecolor{currentstroke}%
\pgfsetdash{}{0pt}%
\pgfpathmoveto{\pgfqpoint{3.357740in}{3.416522in}}%
\pgfpathlineto{\pgfqpoint{3.246843in}{3.352496in}}%
\pgfpathlineto{\pgfqpoint{3.246843in}{3.480548in}}%
\pgfpathlineto{\pgfqpoint{3.357740in}{3.544575in}}%
\pgfpathlineto{\pgfqpoint{3.357740in}{3.416522in}}%
\pgfpathclose%
\pgfusepath{fill}%
\end{pgfscope}%
\begin{pgfscope}%
\pgfpathrectangle{\pgfqpoint{0.765000in}{0.660000in}}{\pgfqpoint{4.620000in}{4.620000in}}%
\pgfusepath{clip}%
\pgfsetbuttcap%
\pgfsetroundjoin%
\definecolor{currentfill}{rgb}{1.000000,0.894118,0.788235}%
\pgfsetfillcolor{currentfill}%
\pgfsetlinewidth{0.000000pt}%
\definecolor{currentstroke}{rgb}{1.000000,0.894118,0.788235}%
\pgfsetstrokecolor{currentstroke}%
\pgfsetdash{}{0pt}%
\pgfpathmoveto{\pgfqpoint{3.357740in}{3.416522in}}%
\pgfpathlineto{\pgfqpoint{3.468636in}{3.352496in}}%
\pgfpathlineto{\pgfqpoint{3.468636in}{3.480548in}}%
\pgfpathlineto{\pgfqpoint{3.357740in}{3.544575in}}%
\pgfpathlineto{\pgfqpoint{3.357740in}{3.416522in}}%
\pgfpathclose%
\pgfusepath{fill}%
\end{pgfscope}%
\begin{pgfscope}%
\pgfpathrectangle{\pgfqpoint{0.765000in}{0.660000in}}{\pgfqpoint{4.620000in}{4.620000in}}%
\pgfusepath{clip}%
\pgfsetbuttcap%
\pgfsetroundjoin%
\definecolor{currentfill}{rgb}{1.000000,0.894118,0.788235}%
\pgfsetfillcolor{currentfill}%
\pgfsetlinewidth{0.000000pt}%
\definecolor{currentstroke}{rgb}{1.000000,0.894118,0.788235}%
\pgfsetstrokecolor{currentstroke}%
\pgfsetdash{}{0pt}%
\pgfpathmoveto{\pgfqpoint{3.350280in}{3.424899in}}%
\pgfpathlineto{\pgfqpoint{3.239384in}{3.360872in}}%
\pgfpathlineto{\pgfqpoint{3.350280in}{3.296846in}}%
\pgfpathlineto{\pgfqpoint{3.461177in}{3.360872in}}%
\pgfpathlineto{\pgfqpoint{3.350280in}{3.424899in}}%
\pgfpathclose%
\pgfusepath{fill}%
\end{pgfscope}%
\begin{pgfscope}%
\pgfpathrectangle{\pgfqpoint{0.765000in}{0.660000in}}{\pgfqpoint{4.620000in}{4.620000in}}%
\pgfusepath{clip}%
\pgfsetbuttcap%
\pgfsetroundjoin%
\definecolor{currentfill}{rgb}{1.000000,0.894118,0.788235}%
\pgfsetfillcolor{currentfill}%
\pgfsetlinewidth{0.000000pt}%
\definecolor{currentstroke}{rgb}{1.000000,0.894118,0.788235}%
\pgfsetstrokecolor{currentstroke}%
\pgfsetdash{}{0pt}%
\pgfpathmoveto{\pgfqpoint{3.350280in}{3.424899in}}%
\pgfpathlineto{\pgfqpoint{3.239384in}{3.360872in}}%
\pgfpathlineto{\pgfqpoint{3.239384in}{3.488925in}}%
\pgfpathlineto{\pgfqpoint{3.350280in}{3.552951in}}%
\pgfpathlineto{\pgfqpoint{3.350280in}{3.424899in}}%
\pgfpathclose%
\pgfusepath{fill}%
\end{pgfscope}%
\begin{pgfscope}%
\pgfpathrectangle{\pgfqpoint{0.765000in}{0.660000in}}{\pgfqpoint{4.620000in}{4.620000in}}%
\pgfusepath{clip}%
\pgfsetbuttcap%
\pgfsetroundjoin%
\definecolor{currentfill}{rgb}{1.000000,0.894118,0.788235}%
\pgfsetfillcolor{currentfill}%
\pgfsetlinewidth{0.000000pt}%
\definecolor{currentstroke}{rgb}{1.000000,0.894118,0.788235}%
\pgfsetstrokecolor{currentstroke}%
\pgfsetdash{}{0pt}%
\pgfpathmoveto{\pgfqpoint{3.350280in}{3.424899in}}%
\pgfpathlineto{\pgfqpoint{3.461177in}{3.360872in}}%
\pgfpathlineto{\pgfqpoint{3.461177in}{3.488925in}}%
\pgfpathlineto{\pgfqpoint{3.350280in}{3.552951in}}%
\pgfpathlineto{\pgfqpoint{3.350280in}{3.424899in}}%
\pgfpathclose%
\pgfusepath{fill}%
\end{pgfscope}%
\begin{pgfscope}%
\pgfpathrectangle{\pgfqpoint{0.765000in}{0.660000in}}{\pgfqpoint{4.620000in}{4.620000in}}%
\pgfusepath{clip}%
\pgfsetbuttcap%
\pgfsetroundjoin%
\definecolor{currentfill}{rgb}{1.000000,0.894118,0.788235}%
\pgfsetfillcolor{currentfill}%
\pgfsetlinewidth{0.000000pt}%
\definecolor{currentstroke}{rgb}{1.000000,0.894118,0.788235}%
\pgfsetstrokecolor{currentstroke}%
\pgfsetdash{}{0pt}%
\pgfpathmoveto{\pgfqpoint{3.357740in}{3.544575in}}%
\pgfpathlineto{\pgfqpoint{3.246843in}{3.480548in}}%
\pgfpathlineto{\pgfqpoint{3.357740in}{3.416522in}}%
\pgfpathlineto{\pgfqpoint{3.468636in}{3.480548in}}%
\pgfpathlineto{\pgfqpoint{3.357740in}{3.544575in}}%
\pgfpathclose%
\pgfusepath{fill}%
\end{pgfscope}%
\begin{pgfscope}%
\pgfpathrectangle{\pgfqpoint{0.765000in}{0.660000in}}{\pgfqpoint{4.620000in}{4.620000in}}%
\pgfusepath{clip}%
\pgfsetbuttcap%
\pgfsetroundjoin%
\definecolor{currentfill}{rgb}{1.000000,0.894118,0.788235}%
\pgfsetfillcolor{currentfill}%
\pgfsetlinewidth{0.000000pt}%
\definecolor{currentstroke}{rgb}{1.000000,0.894118,0.788235}%
\pgfsetstrokecolor{currentstroke}%
\pgfsetdash{}{0pt}%
\pgfpathmoveto{\pgfqpoint{3.357740in}{3.288470in}}%
\pgfpathlineto{\pgfqpoint{3.468636in}{3.352496in}}%
\pgfpathlineto{\pgfqpoint{3.468636in}{3.480548in}}%
\pgfpathlineto{\pgfqpoint{3.357740in}{3.416522in}}%
\pgfpathlineto{\pgfqpoint{3.357740in}{3.288470in}}%
\pgfpathclose%
\pgfusepath{fill}%
\end{pgfscope}%
\begin{pgfscope}%
\pgfpathrectangle{\pgfqpoint{0.765000in}{0.660000in}}{\pgfqpoint{4.620000in}{4.620000in}}%
\pgfusepath{clip}%
\pgfsetbuttcap%
\pgfsetroundjoin%
\definecolor{currentfill}{rgb}{1.000000,0.894118,0.788235}%
\pgfsetfillcolor{currentfill}%
\pgfsetlinewidth{0.000000pt}%
\definecolor{currentstroke}{rgb}{1.000000,0.894118,0.788235}%
\pgfsetstrokecolor{currentstroke}%
\pgfsetdash{}{0pt}%
\pgfpathmoveto{\pgfqpoint{3.246843in}{3.352496in}}%
\pgfpathlineto{\pgfqpoint{3.357740in}{3.288470in}}%
\pgfpathlineto{\pgfqpoint{3.357740in}{3.416522in}}%
\pgfpathlineto{\pgfqpoint{3.246843in}{3.480548in}}%
\pgfpathlineto{\pgfqpoint{3.246843in}{3.352496in}}%
\pgfpathclose%
\pgfusepath{fill}%
\end{pgfscope}%
\begin{pgfscope}%
\pgfpathrectangle{\pgfqpoint{0.765000in}{0.660000in}}{\pgfqpoint{4.620000in}{4.620000in}}%
\pgfusepath{clip}%
\pgfsetbuttcap%
\pgfsetroundjoin%
\definecolor{currentfill}{rgb}{1.000000,0.894118,0.788235}%
\pgfsetfillcolor{currentfill}%
\pgfsetlinewidth{0.000000pt}%
\definecolor{currentstroke}{rgb}{1.000000,0.894118,0.788235}%
\pgfsetstrokecolor{currentstroke}%
\pgfsetdash{}{0pt}%
\pgfpathmoveto{\pgfqpoint{3.350280in}{3.552951in}}%
\pgfpathlineto{\pgfqpoint{3.239384in}{3.488925in}}%
\pgfpathlineto{\pgfqpoint{3.350280in}{3.424899in}}%
\pgfpathlineto{\pgfqpoint{3.461177in}{3.488925in}}%
\pgfpathlineto{\pgfqpoint{3.350280in}{3.552951in}}%
\pgfpathclose%
\pgfusepath{fill}%
\end{pgfscope}%
\begin{pgfscope}%
\pgfpathrectangle{\pgfqpoint{0.765000in}{0.660000in}}{\pgfqpoint{4.620000in}{4.620000in}}%
\pgfusepath{clip}%
\pgfsetbuttcap%
\pgfsetroundjoin%
\definecolor{currentfill}{rgb}{1.000000,0.894118,0.788235}%
\pgfsetfillcolor{currentfill}%
\pgfsetlinewidth{0.000000pt}%
\definecolor{currentstroke}{rgb}{1.000000,0.894118,0.788235}%
\pgfsetstrokecolor{currentstroke}%
\pgfsetdash{}{0pt}%
\pgfpathmoveto{\pgfqpoint{3.350280in}{3.296846in}}%
\pgfpathlineto{\pgfqpoint{3.461177in}{3.360872in}}%
\pgfpathlineto{\pgfqpoint{3.461177in}{3.488925in}}%
\pgfpathlineto{\pgfqpoint{3.350280in}{3.424899in}}%
\pgfpathlineto{\pgfqpoint{3.350280in}{3.296846in}}%
\pgfpathclose%
\pgfusepath{fill}%
\end{pgfscope}%
\begin{pgfscope}%
\pgfpathrectangle{\pgfqpoint{0.765000in}{0.660000in}}{\pgfqpoint{4.620000in}{4.620000in}}%
\pgfusepath{clip}%
\pgfsetbuttcap%
\pgfsetroundjoin%
\definecolor{currentfill}{rgb}{1.000000,0.894118,0.788235}%
\pgfsetfillcolor{currentfill}%
\pgfsetlinewidth{0.000000pt}%
\definecolor{currentstroke}{rgb}{1.000000,0.894118,0.788235}%
\pgfsetstrokecolor{currentstroke}%
\pgfsetdash{}{0pt}%
\pgfpathmoveto{\pgfqpoint{3.239384in}{3.360872in}}%
\pgfpathlineto{\pgfqpoint{3.350280in}{3.296846in}}%
\pgfpathlineto{\pgfqpoint{3.350280in}{3.424899in}}%
\pgfpathlineto{\pgfqpoint{3.239384in}{3.488925in}}%
\pgfpathlineto{\pgfqpoint{3.239384in}{3.360872in}}%
\pgfpathclose%
\pgfusepath{fill}%
\end{pgfscope}%
\begin{pgfscope}%
\pgfpathrectangle{\pgfqpoint{0.765000in}{0.660000in}}{\pgfqpoint{4.620000in}{4.620000in}}%
\pgfusepath{clip}%
\pgfsetbuttcap%
\pgfsetroundjoin%
\definecolor{currentfill}{rgb}{1.000000,0.894118,0.788235}%
\pgfsetfillcolor{currentfill}%
\pgfsetlinewidth{0.000000pt}%
\definecolor{currentstroke}{rgb}{1.000000,0.894118,0.788235}%
\pgfsetstrokecolor{currentstroke}%
\pgfsetdash{}{0pt}%
\pgfpathmoveto{\pgfqpoint{3.357740in}{3.416522in}}%
\pgfpathlineto{\pgfqpoint{3.246843in}{3.352496in}}%
\pgfpathlineto{\pgfqpoint{3.239384in}{3.360872in}}%
\pgfpathlineto{\pgfqpoint{3.350280in}{3.424899in}}%
\pgfpathlineto{\pgfqpoint{3.357740in}{3.416522in}}%
\pgfpathclose%
\pgfusepath{fill}%
\end{pgfscope}%
\begin{pgfscope}%
\pgfpathrectangle{\pgfqpoint{0.765000in}{0.660000in}}{\pgfqpoint{4.620000in}{4.620000in}}%
\pgfusepath{clip}%
\pgfsetbuttcap%
\pgfsetroundjoin%
\definecolor{currentfill}{rgb}{1.000000,0.894118,0.788235}%
\pgfsetfillcolor{currentfill}%
\pgfsetlinewidth{0.000000pt}%
\definecolor{currentstroke}{rgb}{1.000000,0.894118,0.788235}%
\pgfsetstrokecolor{currentstroke}%
\pgfsetdash{}{0pt}%
\pgfpathmoveto{\pgfqpoint{3.468636in}{3.352496in}}%
\pgfpathlineto{\pgfqpoint{3.357740in}{3.416522in}}%
\pgfpathlineto{\pgfqpoint{3.350280in}{3.424899in}}%
\pgfpathlineto{\pgfqpoint{3.461177in}{3.360872in}}%
\pgfpathlineto{\pgfqpoint{3.468636in}{3.352496in}}%
\pgfpathclose%
\pgfusepath{fill}%
\end{pgfscope}%
\begin{pgfscope}%
\pgfpathrectangle{\pgfqpoint{0.765000in}{0.660000in}}{\pgfqpoint{4.620000in}{4.620000in}}%
\pgfusepath{clip}%
\pgfsetbuttcap%
\pgfsetroundjoin%
\definecolor{currentfill}{rgb}{1.000000,0.894118,0.788235}%
\pgfsetfillcolor{currentfill}%
\pgfsetlinewidth{0.000000pt}%
\definecolor{currentstroke}{rgb}{1.000000,0.894118,0.788235}%
\pgfsetstrokecolor{currentstroke}%
\pgfsetdash{}{0pt}%
\pgfpathmoveto{\pgfqpoint{3.357740in}{3.416522in}}%
\pgfpathlineto{\pgfqpoint{3.357740in}{3.544575in}}%
\pgfpathlineto{\pgfqpoint{3.350280in}{3.552951in}}%
\pgfpathlineto{\pgfqpoint{3.461177in}{3.360872in}}%
\pgfpathlineto{\pgfqpoint{3.357740in}{3.416522in}}%
\pgfpathclose%
\pgfusepath{fill}%
\end{pgfscope}%
\begin{pgfscope}%
\pgfpathrectangle{\pgfqpoint{0.765000in}{0.660000in}}{\pgfqpoint{4.620000in}{4.620000in}}%
\pgfusepath{clip}%
\pgfsetbuttcap%
\pgfsetroundjoin%
\definecolor{currentfill}{rgb}{1.000000,0.894118,0.788235}%
\pgfsetfillcolor{currentfill}%
\pgfsetlinewidth{0.000000pt}%
\definecolor{currentstroke}{rgb}{1.000000,0.894118,0.788235}%
\pgfsetstrokecolor{currentstroke}%
\pgfsetdash{}{0pt}%
\pgfpathmoveto{\pgfqpoint{3.468636in}{3.352496in}}%
\pgfpathlineto{\pgfqpoint{3.468636in}{3.480548in}}%
\pgfpathlineto{\pgfqpoint{3.461177in}{3.488925in}}%
\pgfpathlineto{\pgfqpoint{3.350280in}{3.424899in}}%
\pgfpathlineto{\pgfqpoint{3.468636in}{3.352496in}}%
\pgfpathclose%
\pgfusepath{fill}%
\end{pgfscope}%
\begin{pgfscope}%
\pgfpathrectangle{\pgfqpoint{0.765000in}{0.660000in}}{\pgfqpoint{4.620000in}{4.620000in}}%
\pgfusepath{clip}%
\pgfsetbuttcap%
\pgfsetroundjoin%
\definecolor{currentfill}{rgb}{1.000000,0.894118,0.788235}%
\pgfsetfillcolor{currentfill}%
\pgfsetlinewidth{0.000000pt}%
\definecolor{currentstroke}{rgb}{1.000000,0.894118,0.788235}%
\pgfsetstrokecolor{currentstroke}%
\pgfsetdash{}{0pt}%
\pgfpathmoveto{\pgfqpoint{3.246843in}{3.352496in}}%
\pgfpathlineto{\pgfqpoint{3.357740in}{3.288470in}}%
\pgfpathlineto{\pgfqpoint{3.350280in}{3.296846in}}%
\pgfpathlineto{\pgfqpoint{3.239384in}{3.360872in}}%
\pgfpathlineto{\pgfqpoint{3.246843in}{3.352496in}}%
\pgfpathclose%
\pgfusepath{fill}%
\end{pgfscope}%
\begin{pgfscope}%
\pgfpathrectangle{\pgfqpoint{0.765000in}{0.660000in}}{\pgfqpoint{4.620000in}{4.620000in}}%
\pgfusepath{clip}%
\pgfsetbuttcap%
\pgfsetroundjoin%
\definecolor{currentfill}{rgb}{1.000000,0.894118,0.788235}%
\pgfsetfillcolor{currentfill}%
\pgfsetlinewidth{0.000000pt}%
\definecolor{currentstroke}{rgb}{1.000000,0.894118,0.788235}%
\pgfsetstrokecolor{currentstroke}%
\pgfsetdash{}{0pt}%
\pgfpathmoveto{\pgfqpoint{3.357740in}{3.288470in}}%
\pgfpathlineto{\pgfqpoint{3.468636in}{3.352496in}}%
\pgfpathlineto{\pgfqpoint{3.461177in}{3.360872in}}%
\pgfpathlineto{\pgfqpoint{3.350280in}{3.296846in}}%
\pgfpathlineto{\pgfqpoint{3.357740in}{3.288470in}}%
\pgfpathclose%
\pgfusepath{fill}%
\end{pgfscope}%
\begin{pgfscope}%
\pgfpathrectangle{\pgfqpoint{0.765000in}{0.660000in}}{\pgfqpoint{4.620000in}{4.620000in}}%
\pgfusepath{clip}%
\pgfsetbuttcap%
\pgfsetroundjoin%
\definecolor{currentfill}{rgb}{1.000000,0.894118,0.788235}%
\pgfsetfillcolor{currentfill}%
\pgfsetlinewidth{0.000000pt}%
\definecolor{currentstroke}{rgb}{1.000000,0.894118,0.788235}%
\pgfsetstrokecolor{currentstroke}%
\pgfsetdash{}{0pt}%
\pgfpathmoveto{\pgfqpoint{3.357740in}{3.544575in}}%
\pgfpathlineto{\pgfqpoint{3.246843in}{3.480548in}}%
\pgfpathlineto{\pgfqpoint{3.239384in}{3.488925in}}%
\pgfpathlineto{\pgfqpoint{3.350280in}{3.552951in}}%
\pgfpathlineto{\pgfqpoint{3.357740in}{3.544575in}}%
\pgfpathclose%
\pgfusepath{fill}%
\end{pgfscope}%
\begin{pgfscope}%
\pgfpathrectangle{\pgfqpoint{0.765000in}{0.660000in}}{\pgfqpoint{4.620000in}{4.620000in}}%
\pgfusepath{clip}%
\pgfsetbuttcap%
\pgfsetroundjoin%
\definecolor{currentfill}{rgb}{1.000000,0.894118,0.788235}%
\pgfsetfillcolor{currentfill}%
\pgfsetlinewidth{0.000000pt}%
\definecolor{currentstroke}{rgb}{1.000000,0.894118,0.788235}%
\pgfsetstrokecolor{currentstroke}%
\pgfsetdash{}{0pt}%
\pgfpathmoveto{\pgfqpoint{3.468636in}{3.480548in}}%
\pgfpathlineto{\pgfqpoint{3.357740in}{3.544575in}}%
\pgfpathlineto{\pgfqpoint{3.350280in}{3.552951in}}%
\pgfpathlineto{\pgfqpoint{3.461177in}{3.488925in}}%
\pgfpathlineto{\pgfqpoint{3.468636in}{3.480548in}}%
\pgfpathclose%
\pgfusepath{fill}%
\end{pgfscope}%
\begin{pgfscope}%
\pgfpathrectangle{\pgfqpoint{0.765000in}{0.660000in}}{\pgfqpoint{4.620000in}{4.620000in}}%
\pgfusepath{clip}%
\pgfsetbuttcap%
\pgfsetroundjoin%
\definecolor{currentfill}{rgb}{1.000000,0.894118,0.788235}%
\pgfsetfillcolor{currentfill}%
\pgfsetlinewidth{0.000000pt}%
\definecolor{currentstroke}{rgb}{1.000000,0.894118,0.788235}%
\pgfsetstrokecolor{currentstroke}%
\pgfsetdash{}{0pt}%
\pgfpathmoveto{\pgfqpoint{3.246843in}{3.352496in}}%
\pgfpathlineto{\pgfqpoint{3.246843in}{3.480548in}}%
\pgfpathlineto{\pgfqpoint{3.239384in}{3.488925in}}%
\pgfpathlineto{\pgfqpoint{3.350280in}{3.296846in}}%
\pgfpathlineto{\pgfqpoint{3.246843in}{3.352496in}}%
\pgfpathclose%
\pgfusepath{fill}%
\end{pgfscope}%
\begin{pgfscope}%
\pgfpathrectangle{\pgfqpoint{0.765000in}{0.660000in}}{\pgfqpoint{4.620000in}{4.620000in}}%
\pgfusepath{clip}%
\pgfsetbuttcap%
\pgfsetroundjoin%
\definecolor{currentfill}{rgb}{1.000000,0.894118,0.788235}%
\pgfsetfillcolor{currentfill}%
\pgfsetlinewidth{0.000000pt}%
\definecolor{currentstroke}{rgb}{1.000000,0.894118,0.788235}%
\pgfsetstrokecolor{currentstroke}%
\pgfsetdash{}{0pt}%
\pgfpathmoveto{\pgfqpoint{3.357740in}{3.288470in}}%
\pgfpathlineto{\pgfqpoint{3.357740in}{3.416522in}}%
\pgfpathlineto{\pgfqpoint{3.350280in}{3.424899in}}%
\pgfpathlineto{\pgfqpoint{3.239384in}{3.360872in}}%
\pgfpathlineto{\pgfqpoint{3.357740in}{3.288470in}}%
\pgfpathclose%
\pgfusepath{fill}%
\end{pgfscope}%
\begin{pgfscope}%
\pgfpathrectangle{\pgfqpoint{0.765000in}{0.660000in}}{\pgfqpoint{4.620000in}{4.620000in}}%
\pgfusepath{clip}%
\pgfsetbuttcap%
\pgfsetroundjoin%
\definecolor{currentfill}{rgb}{1.000000,0.894118,0.788235}%
\pgfsetfillcolor{currentfill}%
\pgfsetlinewidth{0.000000pt}%
\definecolor{currentstroke}{rgb}{1.000000,0.894118,0.788235}%
\pgfsetstrokecolor{currentstroke}%
\pgfsetdash{}{0pt}%
\pgfpathmoveto{\pgfqpoint{3.357740in}{3.416522in}}%
\pgfpathlineto{\pgfqpoint{3.468636in}{3.480548in}}%
\pgfpathlineto{\pgfqpoint{3.461177in}{3.488925in}}%
\pgfpathlineto{\pgfqpoint{3.350280in}{3.424899in}}%
\pgfpathlineto{\pgfqpoint{3.357740in}{3.416522in}}%
\pgfpathclose%
\pgfusepath{fill}%
\end{pgfscope}%
\begin{pgfscope}%
\pgfpathrectangle{\pgfqpoint{0.765000in}{0.660000in}}{\pgfqpoint{4.620000in}{4.620000in}}%
\pgfusepath{clip}%
\pgfsetbuttcap%
\pgfsetroundjoin%
\definecolor{currentfill}{rgb}{1.000000,0.894118,0.788235}%
\pgfsetfillcolor{currentfill}%
\pgfsetlinewidth{0.000000pt}%
\definecolor{currentstroke}{rgb}{1.000000,0.894118,0.788235}%
\pgfsetstrokecolor{currentstroke}%
\pgfsetdash{}{0pt}%
\pgfpathmoveto{\pgfqpoint{3.246843in}{3.480548in}}%
\pgfpathlineto{\pgfqpoint{3.357740in}{3.416522in}}%
\pgfpathlineto{\pgfqpoint{3.350280in}{3.424899in}}%
\pgfpathlineto{\pgfqpoint{3.239384in}{3.488925in}}%
\pgfpathlineto{\pgfqpoint{3.246843in}{3.480548in}}%
\pgfpathclose%
\pgfusepath{fill}%
\end{pgfscope}%
\begin{pgfscope}%
\pgfpathrectangle{\pgfqpoint{0.765000in}{0.660000in}}{\pgfqpoint{4.620000in}{4.620000in}}%
\pgfusepath{clip}%
\pgfsetbuttcap%
\pgfsetroundjoin%
\definecolor{currentfill}{rgb}{1.000000,0.894118,0.788235}%
\pgfsetfillcolor{currentfill}%
\pgfsetlinewidth{0.000000pt}%
\definecolor{currentstroke}{rgb}{1.000000,0.894118,0.788235}%
\pgfsetstrokecolor{currentstroke}%
\pgfsetdash{}{0pt}%
\pgfpathmoveto{\pgfqpoint{2.604377in}{3.013317in}}%
\pgfpathlineto{\pgfqpoint{2.493481in}{2.949291in}}%
\pgfpathlineto{\pgfqpoint{2.604377in}{2.885265in}}%
\pgfpathlineto{\pgfqpoint{2.715274in}{2.949291in}}%
\pgfpathlineto{\pgfqpoint{2.604377in}{3.013317in}}%
\pgfpathclose%
\pgfusepath{fill}%
\end{pgfscope}%
\begin{pgfscope}%
\pgfpathrectangle{\pgfqpoint{0.765000in}{0.660000in}}{\pgfqpoint{4.620000in}{4.620000in}}%
\pgfusepath{clip}%
\pgfsetbuttcap%
\pgfsetroundjoin%
\definecolor{currentfill}{rgb}{1.000000,0.894118,0.788235}%
\pgfsetfillcolor{currentfill}%
\pgfsetlinewidth{0.000000pt}%
\definecolor{currentstroke}{rgb}{1.000000,0.894118,0.788235}%
\pgfsetstrokecolor{currentstroke}%
\pgfsetdash{}{0pt}%
\pgfpathmoveto{\pgfqpoint{2.604377in}{3.013317in}}%
\pgfpathlineto{\pgfqpoint{2.493481in}{2.949291in}}%
\pgfpathlineto{\pgfqpoint{2.493481in}{3.077344in}}%
\pgfpathlineto{\pgfqpoint{2.604377in}{3.141370in}}%
\pgfpathlineto{\pgfqpoint{2.604377in}{3.013317in}}%
\pgfpathclose%
\pgfusepath{fill}%
\end{pgfscope}%
\begin{pgfscope}%
\pgfpathrectangle{\pgfqpoint{0.765000in}{0.660000in}}{\pgfqpoint{4.620000in}{4.620000in}}%
\pgfusepath{clip}%
\pgfsetbuttcap%
\pgfsetroundjoin%
\definecolor{currentfill}{rgb}{1.000000,0.894118,0.788235}%
\pgfsetfillcolor{currentfill}%
\pgfsetlinewidth{0.000000pt}%
\definecolor{currentstroke}{rgb}{1.000000,0.894118,0.788235}%
\pgfsetstrokecolor{currentstroke}%
\pgfsetdash{}{0pt}%
\pgfpathmoveto{\pgfqpoint{2.604377in}{3.013317in}}%
\pgfpathlineto{\pgfqpoint{2.715274in}{2.949291in}}%
\pgfpathlineto{\pgfqpoint{2.715274in}{3.077344in}}%
\pgfpathlineto{\pgfqpoint{2.604377in}{3.141370in}}%
\pgfpathlineto{\pgfqpoint{2.604377in}{3.013317in}}%
\pgfpathclose%
\pgfusepath{fill}%
\end{pgfscope}%
\begin{pgfscope}%
\pgfpathrectangle{\pgfqpoint{0.765000in}{0.660000in}}{\pgfqpoint{4.620000in}{4.620000in}}%
\pgfusepath{clip}%
\pgfsetbuttcap%
\pgfsetroundjoin%
\definecolor{currentfill}{rgb}{1.000000,0.894118,0.788235}%
\pgfsetfillcolor{currentfill}%
\pgfsetlinewidth{0.000000pt}%
\definecolor{currentstroke}{rgb}{1.000000,0.894118,0.788235}%
\pgfsetstrokecolor{currentstroke}%
\pgfsetdash{}{0pt}%
\pgfpathmoveto{\pgfqpoint{2.672370in}{2.810270in}}%
\pgfpathlineto{\pgfqpoint{2.561474in}{2.746243in}}%
\pgfpathlineto{\pgfqpoint{2.672370in}{2.682217in}}%
\pgfpathlineto{\pgfqpoint{2.783267in}{2.746243in}}%
\pgfpathlineto{\pgfqpoint{2.672370in}{2.810270in}}%
\pgfpathclose%
\pgfusepath{fill}%
\end{pgfscope}%
\begin{pgfscope}%
\pgfpathrectangle{\pgfqpoint{0.765000in}{0.660000in}}{\pgfqpoint{4.620000in}{4.620000in}}%
\pgfusepath{clip}%
\pgfsetbuttcap%
\pgfsetroundjoin%
\definecolor{currentfill}{rgb}{1.000000,0.894118,0.788235}%
\pgfsetfillcolor{currentfill}%
\pgfsetlinewidth{0.000000pt}%
\definecolor{currentstroke}{rgb}{1.000000,0.894118,0.788235}%
\pgfsetstrokecolor{currentstroke}%
\pgfsetdash{}{0pt}%
\pgfpathmoveto{\pgfqpoint{2.672370in}{2.810270in}}%
\pgfpathlineto{\pgfqpoint{2.561474in}{2.746243in}}%
\pgfpathlineto{\pgfqpoint{2.561474in}{2.874296in}}%
\pgfpathlineto{\pgfqpoint{2.672370in}{2.938322in}}%
\pgfpathlineto{\pgfqpoint{2.672370in}{2.810270in}}%
\pgfpathclose%
\pgfusepath{fill}%
\end{pgfscope}%
\begin{pgfscope}%
\pgfpathrectangle{\pgfqpoint{0.765000in}{0.660000in}}{\pgfqpoint{4.620000in}{4.620000in}}%
\pgfusepath{clip}%
\pgfsetbuttcap%
\pgfsetroundjoin%
\definecolor{currentfill}{rgb}{1.000000,0.894118,0.788235}%
\pgfsetfillcolor{currentfill}%
\pgfsetlinewidth{0.000000pt}%
\definecolor{currentstroke}{rgb}{1.000000,0.894118,0.788235}%
\pgfsetstrokecolor{currentstroke}%
\pgfsetdash{}{0pt}%
\pgfpathmoveto{\pgfqpoint{2.672370in}{2.810270in}}%
\pgfpathlineto{\pgfqpoint{2.783267in}{2.746243in}}%
\pgfpathlineto{\pgfqpoint{2.783267in}{2.874296in}}%
\pgfpathlineto{\pgfqpoint{2.672370in}{2.938322in}}%
\pgfpathlineto{\pgfqpoint{2.672370in}{2.810270in}}%
\pgfpathclose%
\pgfusepath{fill}%
\end{pgfscope}%
\begin{pgfscope}%
\pgfpathrectangle{\pgfqpoint{0.765000in}{0.660000in}}{\pgfqpoint{4.620000in}{4.620000in}}%
\pgfusepath{clip}%
\pgfsetbuttcap%
\pgfsetroundjoin%
\definecolor{currentfill}{rgb}{1.000000,0.894118,0.788235}%
\pgfsetfillcolor{currentfill}%
\pgfsetlinewidth{0.000000pt}%
\definecolor{currentstroke}{rgb}{1.000000,0.894118,0.788235}%
\pgfsetstrokecolor{currentstroke}%
\pgfsetdash{}{0pt}%
\pgfpathmoveto{\pgfqpoint{2.604377in}{3.141370in}}%
\pgfpathlineto{\pgfqpoint{2.493481in}{3.077344in}}%
\pgfpathlineto{\pgfqpoint{2.604377in}{3.013317in}}%
\pgfpathlineto{\pgfqpoint{2.715274in}{3.077344in}}%
\pgfpathlineto{\pgfqpoint{2.604377in}{3.141370in}}%
\pgfpathclose%
\pgfusepath{fill}%
\end{pgfscope}%
\begin{pgfscope}%
\pgfpathrectangle{\pgfqpoint{0.765000in}{0.660000in}}{\pgfqpoint{4.620000in}{4.620000in}}%
\pgfusepath{clip}%
\pgfsetbuttcap%
\pgfsetroundjoin%
\definecolor{currentfill}{rgb}{1.000000,0.894118,0.788235}%
\pgfsetfillcolor{currentfill}%
\pgfsetlinewidth{0.000000pt}%
\definecolor{currentstroke}{rgb}{1.000000,0.894118,0.788235}%
\pgfsetstrokecolor{currentstroke}%
\pgfsetdash{}{0pt}%
\pgfpathmoveto{\pgfqpoint{2.604377in}{2.885265in}}%
\pgfpathlineto{\pgfqpoint{2.715274in}{2.949291in}}%
\pgfpathlineto{\pgfqpoint{2.715274in}{3.077344in}}%
\pgfpathlineto{\pgfqpoint{2.604377in}{3.013317in}}%
\pgfpathlineto{\pgfqpoint{2.604377in}{2.885265in}}%
\pgfpathclose%
\pgfusepath{fill}%
\end{pgfscope}%
\begin{pgfscope}%
\pgfpathrectangle{\pgfqpoint{0.765000in}{0.660000in}}{\pgfqpoint{4.620000in}{4.620000in}}%
\pgfusepath{clip}%
\pgfsetbuttcap%
\pgfsetroundjoin%
\definecolor{currentfill}{rgb}{1.000000,0.894118,0.788235}%
\pgfsetfillcolor{currentfill}%
\pgfsetlinewidth{0.000000pt}%
\definecolor{currentstroke}{rgb}{1.000000,0.894118,0.788235}%
\pgfsetstrokecolor{currentstroke}%
\pgfsetdash{}{0pt}%
\pgfpathmoveto{\pgfqpoint{2.493481in}{2.949291in}}%
\pgfpathlineto{\pgfqpoint{2.604377in}{2.885265in}}%
\pgfpathlineto{\pgfqpoint{2.604377in}{3.013317in}}%
\pgfpathlineto{\pgfqpoint{2.493481in}{3.077344in}}%
\pgfpathlineto{\pgfqpoint{2.493481in}{2.949291in}}%
\pgfpathclose%
\pgfusepath{fill}%
\end{pgfscope}%
\begin{pgfscope}%
\pgfpathrectangle{\pgfqpoint{0.765000in}{0.660000in}}{\pgfqpoint{4.620000in}{4.620000in}}%
\pgfusepath{clip}%
\pgfsetbuttcap%
\pgfsetroundjoin%
\definecolor{currentfill}{rgb}{1.000000,0.894118,0.788235}%
\pgfsetfillcolor{currentfill}%
\pgfsetlinewidth{0.000000pt}%
\definecolor{currentstroke}{rgb}{1.000000,0.894118,0.788235}%
\pgfsetstrokecolor{currentstroke}%
\pgfsetdash{}{0pt}%
\pgfpathmoveto{\pgfqpoint{2.672370in}{2.938322in}}%
\pgfpathlineto{\pgfqpoint{2.561474in}{2.874296in}}%
\pgfpathlineto{\pgfqpoint{2.672370in}{2.810270in}}%
\pgfpathlineto{\pgfqpoint{2.783267in}{2.874296in}}%
\pgfpathlineto{\pgfqpoint{2.672370in}{2.938322in}}%
\pgfpathclose%
\pgfusepath{fill}%
\end{pgfscope}%
\begin{pgfscope}%
\pgfpathrectangle{\pgfqpoint{0.765000in}{0.660000in}}{\pgfqpoint{4.620000in}{4.620000in}}%
\pgfusepath{clip}%
\pgfsetbuttcap%
\pgfsetroundjoin%
\definecolor{currentfill}{rgb}{1.000000,0.894118,0.788235}%
\pgfsetfillcolor{currentfill}%
\pgfsetlinewidth{0.000000pt}%
\definecolor{currentstroke}{rgb}{1.000000,0.894118,0.788235}%
\pgfsetstrokecolor{currentstroke}%
\pgfsetdash{}{0pt}%
\pgfpathmoveto{\pgfqpoint{2.672370in}{2.682217in}}%
\pgfpathlineto{\pgfqpoint{2.783267in}{2.746243in}}%
\pgfpathlineto{\pgfqpoint{2.783267in}{2.874296in}}%
\pgfpathlineto{\pgfqpoint{2.672370in}{2.810270in}}%
\pgfpathlineto{\pgfqpoint{2.672370in}{2.682217in}}%
\pgfpathclose%
\pgfusepath{fill}%
\end{pgfscope}%
\begin{pgfscope}%
\pgfpathrectangle{\pgfqpoint{0.765000in}{0.660000in}}{\pgfqpoint{4.620000in}{4.620000in}}%
\pgfusepath{clip}%
\pgfsetbuttcap%
\pgfsetroundjoin%
\definecolor{currentfill}{rgb}{1.000000,0.894118,0.788235}%
\pgfsetfillcolor{currentfill}%
\pgfsetlinewidth{0.000000pt}%
\definecolor{currentstroke}{rgb}{1.000000,0.894118,0.788235}%
\pgfsetstrokecolor{currentstroke}%
\pgfsetdash{}{0pt}%
\pgfpathmoveto{\pgfqpoint{2.561474in}{2.746243in}}%
\pgfpathlineto{\pgfqpoint{2.672370in}{2.682217in}}%
\pgfpathlineto{\pgfqpoint{2.672370in}{2.810270in}}%
\pgfpathlineto{\pgfqpoint{2.561474in}{2.874296in}}%
\pgfpathlineto{\pgfqpoint{2.561474in}{2.746243in}}%
\pgfpathclose%
\pgfusepath{fill}%
\end{pgfscope}%
\begin{pgfscope}%
\pgfpathrectangle{\pgfqpoint{0.765000in}{0.660000in}}{\pgfqpoint{4.620000in}{4.620000in}}%
\pgfusepath{clip}%
\pgfsetbuttcap%
\pgfsetroundjoin%
\definecolor{currentfill}{rgb}{1.000000,0.894118,0.788235}%
\pgfsetfillcolor{currentfill}%
\pgfsetlinewidth{0.000000pt}%
\definecolor{currentstroke}{rgb}{1.000000,0.894118,0.788235}%
\pgfsetstrokecolor{currentstroke}%
\pgfsetdash{}{0pt}%
\pgfpathmoveto{\pgfqpoint{2.604377in}{3.013317in}}%
\pgfpathlineto{\pgfqpoint{2.493481in}{2.949291in}}%
\pgfpathlineto{\pgfqpoint{2.561474in}{2.746243in}}%
\pgfpathlineto{\pgfqpoint{2.672370in}{2.810270in}}%
\pgfpathlineto{\pgfqpoint{2.604377in}{3.013317in}}%
\pgfpathclose%
\pgfusepath{fill}%
\end{pgfscope}%
\begin{pgfscope}%
\pgfpathrectangle{\pgfqpoint{0.765000in}{0.660000in}}{\pgfqpoint{4.620000in}{4.620000in}}%
\pgfusepath{clip}%
\pgfsetbuttcap%
\pgfsetroundjoin%
\definecolor{currentfill}{rgb}{1.000000,0.894118,0.788235}%
\pgfsetfillcolor{currentfill}%
\pgfsetlinewidth{0.000000pt}%
\definecolor{currentstroke}{rgb}{1.000000,0.894118,0.788235}%
\pgfsetstrokecolor{currentstroke}%
\pgfsetdash{}{0pt}%
\pgfpathmoveto{\pgfqpoint{2.715274in}{2.949291in}}%
\pgfpathlineto{\pgfqpoint{2.604377in}{3.013317in}}%
\pgfpathlineto{\pgfqpoint{2.672370in}{2.810270in}}%
\pgfpathlineto{\pgfqpoint{2.783267in}{2.746243in}}%
\pgfpathlineto{\pgfqpoint{2.715274in}{2.949291in}}%
\pgfpathclose%
\pgfusepath{fill}%
\end{pgfscope}%
\begin{pgfscope}%
\pgfpathrectangle{\pgfqpoint{0.765000in}{0.660000in}}{\pgfqpoint{4.620000in}{4.620000in}}%
\pgfusepath{clip}%
\pgfsetbuttcap%
\pgfsetroundjoin%
\definecolor{currentfill}{rgb}{1.000000,0.894118,0.788235}%
\pgfsetfillcolor{currentfill}%
\pgfsetlinewidth{0.000000pt}%
\definecolor{currentstroke}{rgb}{1.000000,0.894118,0.788235}%
\pgfsetstrokecolor{currentstroke}%
\pgfsetdash{}{0pt}%
\pgfpathmoveto{\pgfqpoint{2.604377in}{3.013317in}}%
\pgfpathlineto{\pgfqpoint{2.604377in}{3.141370in}}%
\pgfpathlineto{\pgfqpoint{2.672370in}{2.938322in}}%
\pgfpathlineto{\pgfqpoint{2.783267in}{2.746243in}}%
\pgfpathlineto{\pgfqpoint{2.604377in}{3.013317in}}%
\pgfpathclose%
\pgfusepath{fill}%
\end{pgfscope}%
\begin{pgfscope}%
\pgfpathrectangle{\pgfqpoint{0.765000in}{0.660000in}}{\pgfqpoint{4.620000in}{4.620000in}}%
\pgfusepath{clip}%
\pgfsetbuttcap%
\pgfsetroundjoin%
\definecolor{currentfill}{rgb}{1.000000,0.894118,0.788235}%
\pgfsetfillcolor{currentfill}%
\pgfsetlinewidth{0.000000pt}%
\definecolor{currentstroke}{rgb}{1.000000,0.894118,0.788235}%
\pgfsetstrokecolor{currentstroke}%
\pgfsetdash{}{0pt}%
\pgfpathmoveto{\pgfqpoint{2.715274in}{2.949291in}}%
\pgfpathlineto{\pgfqpoint{2.715274in}{3.077344in}}%
\pgfpathlineto{\pgfqpoint{2.783267in}{2.874296in}}%
\pgfpathlineto{\pgfqpoint{2.672370in}{2.810270in}}%
\pgfpathlineto{\pgfqpoint{2.715274in}{2.949291in}}%
\pgfpathclose%
\pgfusepath{fill}%
\end{pgfscope}%
\begin{pgfscope}%
\pgfpathrectangle{\pgfqpoint{0.765000in}{0.660000in}}{\pgfqpoint{4.620000in}{4.620000in}}%
\pgfusepath{clip}%
\pgfsetbuttcap%
\pgfsetroundjoin%
\definecolor{currentfill}{rgb}{1.000000,0.894118,0.788235}%
\pgfsetfillcolor{currentfill}%
\pgfsetlinewidth{0.000000pt}%
\definecolor{currentstroke}{rgb}{1.000000,0.894118,0.788235}%
\pgfsetstrokecolor{currentstroke}%
\pgfsetdash{}{0pt}%
\pgfpathmoveto{\pgfqpoint{2.493481in}{2.949291in}}%
\pgfpathlineto{\pgfqpoint{2.604377in}{2.885265in}}%
\pgfpathlineto{\pgfqpoint{2.672370in}{2.682217in}}%
\pgfpathlineto{\pgfqpoint{2.561474in}{2.746243in}}%
\pgfpathlineto{\pgfqpoint{2.493481in}{2.949291in}}%
\pgfpathclose%
\pgfusepath{fill}%
\end{pgfscope}%
\begin{pgfscope}%
\pgfpathrectangle{\pgfqpoint{0.765000in}{0.660000in}}{\pgfqpoint{4.620000in}{4.620000in}}%
\pgfusepath{clip}%
\pgfsetbuttcap%
\pgfsetroundjoin%
\definecolor{currentfill}{rgb}{1.000000,0.894118,0.788235}%
\pgfsetfillcolor{currentfill}%
\pgfsetlinewidth{0.000000pt}%
\definecolor{currentstroke}{rgb}{1.000000,0.894118,0.788235}%
\pgfsetstrokecolor{currentstroke}%
\pgfsetdash{}{0pt}%
\pgfpathmoveto{\pgfqpoint{2.604377in}{2.885265in}}%
\pgfpathlineto{\pgfqpoint{2.715274in}{2.949291in}}%
\pgfpathlineto{\pgfqpoint{2.783267in}{2.746243in}}%
\pgfpathlineto{\pgfqpoint{2.672370in}{2.682217in}}%
\pgfpathlineto{\pgfqpoint{2.604377in}{2.885265in}}%
\pgfpathclose%
\pgfusepath{fill}%
\end{pgfscope}%
\begin{pgfscope}%
\pgfpathrectangle{\pgfqpoint{0.765000in}{0.660000in}}{\pgfqpoint{4.620000in}{4.620000in}}%
\pgfusepath{clip}%
\pgfsetbuttcap%
\pgfsetroundjoin%
\definecolor{currentfill}{rgb}{1.000000,0.894118,0.788235}%
\pgfsetfillcolor{currentfill}%
\pgfsetlinewidth{0.000000pt}%
\definecolor{currentstroke}{rgb}{1.000000,0.894118,0.788235}%
\pgfsetstrokecolor{currentstroke}%
\pgfsetdash{}{0pt}%
\pgfpathmoveto{\pgfqpoint{2.604377in}{3.141370in}}%
\pgfpathlineto{\pgfqpoint{2.493481in}{3.077344in}}%
\pgfpathlineto{\pgfqpoint{2.561474in}{2.874296in}}%
\pgfpathlineto{\pgfqpoint{2.672370in}{2.938322in}}%
\pgfpathlineto{\pgfqpoint{2.604377in}{3.141370in}}%
\pgfpathclose%
\pgfusepath{fill}%
\end{pgfscope}%
\begin{pgfscope}%
\pgfpathrectangle{\pgfqpoint{0.765000in}{0.660000in}}{\pgfqpoint{4.620000in}{4.620000in}}%
\pgfusepath{clip}%
\pgfsetbuttcap%
\pgfsetroundjoin%
\definecolor{currentfill}{rgb}{1.000000,0.894118,0.788235}%
\pgfsetfillcolor{currentfill}%
\pgfsetlinewidth{0.000000pt}%
\definecolor{currentstroke}{rgb}{1.000000,0.894118,0.788235}%
\pgfsetstrokecolor{currentstroke}%
\pgfsetdash{}{0pt}%
\pgfpathmoveto{\pgfqpoint{2.715274in}{3.077344in}}%
\pgfpathlineto{\pgfqpoint{2.604377in}{3.141370in}}%
\pgfpathlineto{\pgfqpoint{2.672370in}{2.938322in}}%
\pgfpathlineto{\pgfqpoint{2.783267in}{2.874296in}}%
\pgfpathlineto{\pgfqpoint{2.715274in}{3.077344in}}%
\pgfpathclose%
\pgfusepath{fill}%
\end{pgfscope}%
\begin{pgfscope}%
\pgfpathrectangle{\pgfqpoint{0.765000in}{0.660000in}}{\pgfqpoint{4.620000in}{4.620000in}}%
\pgfusepath{clip}%
\pgfsetbuttcap%
\pgfsetroundjoin%
\definecolor{currentfill}{rgb}{1.000000,0.894118,0.788235}%
\pgfsetfillcolor{currentfill}%
\pgfsetlinewidth{0.000000pt}%
\definecolor{currentstroke}{rgb}{1.000000,0.894118,0.788235}%
\pgfsetstrokecolor{currentstroke}%
\pgfsetdash{}{0pt}%
\pgfpathmoveto{\pgfqpoint{2.493481in}{2.949291in}}%
\pgfpathlineto{\pgfqpoint{2.493481in}{3.077344in}}%
\pgfpathlineto{\pgfqpoint{2.561474in}{2.874296in}}%
\pgfpathlineto{\pgfqpoint{2.672370in}{2.682217in}}%
\pgfpathlineto{\pgfqpoint{2.493481in}{2.949291in}}%
\pgfpathclose%
\pgfusepath{fill}%
\end{pgfscope}%
\begin{pgfscope}%
\pgfpathrectangle{\pgfqpoint{0.765000in}{0.660000in}}{\pgfqpoint{4.620000in}{4.620000in}}%
\pgfusepath{clip}%
\pgfsetbuttcap%
\pgfsetroundjoin%
\definecolor{currentfill}{rgb}{1.000000,0.894118,0.788235}%
\pgfsetfillcolor{currentfill}%
\pgfsetlinewidth{0.000000pt}%
\definecolor{currentstroke}{rgb}{1.000000,0.894118,0.788235}%
\pgfsetstrokecolor{currentstroke}%
\pgfsetdash{}{0pt}%
\pgfpathmoveto{\pgfqpoint{2.604377in}{2.885265in}}%
\pgfpathlineto{\pgfqpoint{2.604377in}{3.013317in}}%
\pgfpathlineto{\pgfqpoint{2.672370in}{2.810270in}}%
\pgfpathlineto{\pgfqpoint{2.561474in}{2.746243in}}%
\pgfpathlineto{\pgfqpoint{2.604377in}{2.885265in}}%
\pgfpathclose%
\pgfusepath{fill}%
\end{pgfscope}%
\begin{pgfscope}%
\pgfpathrectangle{\pgfqpoint{0.765000in}{0.660000in}}{\pgfqpoint{4.620000in}{4.620000in}}%
\pgfusepath{clip}%
\pgfsetbuttcap%
\pgfsetroundjoin%
\definecolor{currentfill}{rgb}{1.000000,0.894118,0.788235}%
\pgfsetfillcolor{currentfill}%
\pgfsetlinewidth{0.000000pt}%
\definecolor{currentstroke}{rgb}{1.000000,0.894118,0.788235}%
\pgfsetstrokecolor{currentstroke}%
\pgfsetdash{}{0pt}%
\pgfpathmoveto{\pgfqpoint{2.604377in}{3.013317in}}%
\pgfpathlineto{\pgfqpoint{2.715274in}{3.077344in}}%
\pgfpathlineto{\pgfqpoint{2.783267in}{2.874296in}}%
\pgfpathlineto{\pgfqpoint{2.672370in}{2.810270in}}%
\pgfpathlineto{\pgfqpoint{2.604377in}{3.013317in}}%
\pgfpathclose%
\pgfusepath{fill}%
\end{pgfscope}%
\begin{pgfscope}%
\pgfpathrectangle{\pgfqpoint{0.765000in}{0.660000in}}{\pgfqpoint{4.620000in}{4.620000in}}%
\pgfusepath{clip}%
\pgfsetbuttcap%
\pgfsetroundjoin%
\definecolor{currentfill}{rgb}{1.000000,0.894118,0.788235}%
\pgfsetfillcolor{currentfill}%
\pgfsetlinewidth{0.000000pt}%
\definecolor{currentstroke}{rgb}{1.000000,0.894118,0.788235}%
\pgfsetstrokecolor{currentstroke}%
\pgfsetdash{}{0pt}%
\pgfpathmoveto{\pgfqpoint{2.493481in}{3.077344in}}%
\pgfpathlineto{\pgfqpoint{2.604377in}{3.013317in}}%
\pgfpathlineto{\pgfqpoint{2.672370in}{2.810270in}}%
\pgfpathlineto{\pgfqpoint{2.561474in}{2.874296in}}%
\pgfpathlineto{\pgfqpoint{2.493481in}{3.077344in}}%
\pgfpathclose%
\pgfusepath{fill}%
\end{pgfscope}%
\begin{pgfscope}%
\pgfpathrectangle{\pgfqpoint{0.765000in}{0.660000in}}{\pgfqpoint{4.620000in}{4.620000in}}%
\pgfusepath{clip}%
\pgfsetbuttcap%
\pgfsetroundjoin%
\definecolor{currentfill}{rgb}{1.000000,0.894118,0.788235}%
\pgfsetfillcolor{currentfill}%
\pgfsetlinewidth{0.000000pt}%
\definecolor{currentstroke}{rgb}{1.000000,0.894118,0.788235}%
\pgfsetstrokecolor{currentstroke}%
\pgfsetdash{}{0pt}%
\pgfpathmoveto{\pgfqpoint{2.604377in}{3.013317in}}%
\pgfpathlineto{\pgfqpoint{2.493481in}{2.949291in}}%
\pgfpathlineto{\pgfqpoint{2.604377in}{2.885265in}}%
\pgfpathlineto{\pgfqpoint{2.715274in}{2.949291in}}%
\pgfpathlineto{\pgfqpoint{2.604377in}{3.013317in}}%
\pgfpathclose%
\pgfusepath{fill}%
\end{pgfscope}%
\begin{pgfscope}%
\pgfpathrectangle{\pgfqpoint{0.765000in}{0.660000in}}{\pgfqpoint{4.620000in}{4.620000in}}%
\pgfusepath{clip}%
\pgfsetbuttcap%
\pgfsetroundjoin%
\definecolor{currentfill}{rgb}{1.000000,0.894118,0.788235}%
\pgfsetfillcolor{currentfill}%
\pgfsetlinewidth{0.000000pt}%
\definecolor{currentstroke}{rgb}{1.000000,0.894118,0.788235}%
\pgfsetstrokecolor{currentstroke}%
\pgfsetdash{}{0pt}%
\pgfpathmoveto{\pgfqpoint{2.604377in}{3.013317in}}%
\pgfpathlineto{\pgfqpoint{2.493481in}{2.949291in}}%
\pgfpathlineto{\pgfqpoint{2.493481in}{3.077344in}}%
\pgfpathlineto{\pgfqpoint{2.604377in}{3.141370in}}%
\pgfpathlineto{\pgfqpoint{2.604377in}{3.013317in}}%
\pgfpathclose%
\pgfusepath{fill}%
\end{pgfscope}%
\begin{pgfscope}%
\pgfpathrectangle{\pgfqpoint{0.765000in}{0.660000in}}{\pgfqpoint{4.620000in}{4.620000in}}%
\pgfusepath{clip}%
\pgfsetbuttcap%
\pgfsetroundjoin%
\definecolor{currentfill}{rgb}{1.000000,0.894118,0.788235}%
\pgfsetfillcolor{currentfill}%
\pgfsetlinewidth{0.000000pt}%
\definecolor{currentstroke}{rgb}{1.000000,0.894118,0.788235}%
\pgfsetstrokecolor{currentstroke}%
\pgfsetdash{}{0pt}%
\pgfpathmoveto{\pgfqpoint{2.604377in}{3.013317in}}%
\pgfpathlineto{\pgfqpoint{2.715274in}{2.949291in}}%
\pgfpathlineto{\pgfqpoint{2.715274in}{3.077344in}}%
\pgfpathlineto{\pgfqpoint{2.604377in}{3.141370in}}%
\pgfpathlineto{\pgfqpoint{2.604377in}{3.013317in}}%
\pgfpathclose%
\pgfusepath{fill}%
\end{pgfscope}%
\begin{pgfscope}%
\pgfpathrectangle{\pgfqpoint{0.765000in}{0.660000in}}{\pgfqpoint{4.620000in}{4.620000in}}%
\pgfusepath{clip}%
\pgfsetbuttcap%
\pgfsetroundjoin%
\definecolor{currentfill}{rgb}{1.000000,0.894118,0.788235}%
\pgfsetfillcolor{currentfill}%
\pgfsetlinewidth{0.000000pt}%
\definecolor{currentstroke}{rgb}{1.000000,0.894118,0.788235}%
\pgfsetstrokecolor{currentstroke}%
\pgfsetdash{}{0pt}%
\pgfpathmoveto{\pgfqpoint{2.586744in}{3.073269in}}%
\pgfpathlineto{\pgfqpoint{2.475847in}{3.009242in}}%
\pgfpathlineto{\pgfqpoint{2.586744in}{2.945216in}}%
\pgfpathlineto{\pgfqpoint{2.697640in}{3.009242in}}%
\pgfpathlineto{\pgfqpoint{2.586744in}{3.073269in}}%
\pgfpathclose%
\pgfusepath{fill}%
\end{pgfscope}%
\begin{pgfscope}%
\pgfpathrectangle{\pgfqpoint{0.765000in}{0.660000in}}{\pgfqpoint{4.620000in}{4.620000in}}%
\pgfusepath{clip}%
\pgfsetbuttcap%
\pgfsetroundjoin%
\definecolor{currentfill}{rgb}{1.000000,0.894118,0.788235}%
\pgfsetfillcolor{currentfill}%
\pgfsetlinewidth{0.000000pt}%
\definecolor{currentstroke}{rgb}{1.000000,0.894118,0.788235}%
\pgfsetstrokecolor{currentstroke}%
\pgfsetdash{}{0pt}%
\pgfpathmoveto{\pgfqpoint{2.586744in}{3.073269in}}%
\pgfpathlineto{\pgfqpoint{2.475847in}{3.009242in}}%
\pgfpathlineto{\pgfqpoint{2.475847in}{3.137295in}}%
\pgfpathlineto{\pgfqpoint{2.586744in}{3.201321in}}%
\pgfpathlineto{\pgfqpoint{2.586744in}{3.073269in}}%
\pgfpathclose%
\pgfusepath{fill}%
\end{pgfscope}%
\begin{pgfscope}%
\pgfpathrectangle{\pgfqpoint{0.765000in}{0.660000in}}{\pgfqpoint{4.620000in}{4.620000in}}%
\pgfusepath{clip}%
\pgfsetbuttcap%
\pgfsetroundjoin%
\definecolor{currentfill}{rgb}{1.000000,0.894118,0.788235}%
\pgfsetfillcolor{currentfill}%
\pgfsetlinewidth{0.000000pt}%
\definecolor{currentstroke}{rgb}{1.000000,0.894118,0.788235}%
\pgfsetstrokecolor{currentstroke}%
\pgfsetdash{}{0pt}%
\pgfpathmoveto{\pgfqpoint{2.586744in}{3.073269in}}%
\pgfpathlineto{\pgfqpoint{2.697640in}{3.009242in}}%
\pgfpathlineto{\pgfqpoint{2.697640in}{3.137295in}}%
\pgfpathlineto{\pgfqpoint{2.586744in}{3.201321in}}%
\pgfpathlineto{\pgfqpoint{2.586744in}{3.073269in}}%
\pgfpathclose%
\pgfusepath{fill}%
\end{pgfscope}%
\begin{pgfscope}%
\pgfpathrectangle{\pgfqpoint{0.765000in}{0.660000in}}{\pgfqpoint{4.620000in}{4.620000in}}%
\pgfusepath{clip}%
\pgfsetbuttcap%
\pgfsetroundjoin%
\definecolor{currentfill}{rgb}{1.000000,0.894118,0.788235}%
\pgfsetfillcolor{currentfill}%
\pgfsetlinewidth{0.000000pt}%
\definecolor{currentstroke}{rgb}{1.000000,0.894118,0.788235}%
\pgfsetstrokecolor{currentstroke}%
\pgfsetdash{}{0pt}%
\pgfpathmoveto{\pgfqpoint{2.604377in}{3.141370in}}%
\pgfpathlineto{\pgfqpoint{2.493481in}{3.077344in}}%
\pgfpathlineto{\pgfqpoint{2.604377in}{3.013317in}}%
\pgfpathlineto{\pgfqpoint{2.715274in}{3.077344in}}%
\pgfpathlineto{\pgfqpoint{2.604377in}{3.141370in}}%
\pgfpathclose%
\pgfusepath{fill}%
\end{pgfscope}%
\begin{pgfscope}%
\pgfpathrectangle{\pgfqpoint{0.765000in}{0.660000in}}{\pgfqpoint{4.620000in}{4.620000in}}%
\pgfusepath{clip}%
\pgfsetbuttcap%
\pgfsetroundjoin%
\definecolor{currentfill}{rgb}{1.000000,0.894118,0.788235}%
\pgfsetfillcolor{currentfill}%
\pgfsetlinewidth{0.000000pt}%
\definecolor{currentstroke}{rgb}{1.000000,0.894118,0.788235}%
\pgfsetstrokecolor{currentstroke}%
\pgfsetdash{}{0pt}%
\pgfpathmoveto{\pgfqpoint{2.604377in}{2.885265in}}%
\pgfpathlineto{\pgfqpoint{2.715274in}{2.949291in}}%
\pgfpathlineto{\pgfqpoint{2.715274in}{3.077344in}}%
\pgfpathlineto{\pgfqpoint{2.604377in}{3.013317in}}%
\pgfpathlineto{\pgfqpoint{2.604377in}{2.885265in}}%
\pgfpathclose%
\pgfusepath{fill}%
\end{pgfscope}%
\begin{pgfscope}%
\pgfpathrectangle{\pgfqpoint{0.765000in}{0.660000in}}{\pgfqpoint{4.620000in}{4.620000in}}%
\pgfusepath{clip}%
\pgfsetbuttcap%
\pgfsetroundjoin%
\definecolor{currentfill}{rgb}{1.000000,0.894118,0.788235}%
\pgfsetfillcolor{currentfill}%
\pgfsetlinewidth{0.000000pt}%
\definecolor{currentstroke}{rgb}{1.000000,0.894118,0.788235}%
\pgfsetstrokecolor{currentstroke}%
\pgfsetdash{}{0pt}%
\pgfpathmoveto{\pgfqpoint{2.493481in}{2.949291in}}%
\pgfpathlineto{\pgfqpoint{2.604377in}{2.885265in}}%
\pgfpathlineto{\pgfqpoint{2.604377in}{3.013317in}}%
\pgfpathlineto{\pgfqpoint{2.493481in}{3.077344in}}%
\pgfpathlineto{\pgfqpoint{2.493481in}{2.949291in}}%
\pgfpathclose%
\pgfusepath{fill}%
\end{pgfscope}%
\begin{pgfscope}%
\pgfpathrectangle{\pgfqpoint{0.765000in}{0.660000in}}{\pgfqpoint{4.620000in}{4.620000in}}%
\pgfusepath{clip}%
\pgfsetbuttcap%
\pgfsetroundjoin%
\definecolor{currentfill}{rgb}{1.000000,0.894118,0.788235}%
\pgfsetfillcolor{currentfill}%
\pgfsetlinewidth{0.000000pt}%
\definecolor{currentstroke}{rgb}{1.000000,0.894118,0.788235}%
\pgfsetstrokecolor{currentstroke}%
\pgfsetdash{}{0pt}%
\pgfpathmoveto{\pgfqpoint{2.586744in}{3.201321in}}%
\pgfpathlineto{\pgfqpoint{2.475847in}{3.137295in}}%
\pgfpathlineto{\pgfqpoint{2.586744in}{3.073269in}}%
\pgfpathlineto{\pgfqpoint{2.697640in}{3.137295in}}%
\pgfpathlineto{\pgfqpoint{2.586744in}{3.201321in}}%
\pgfpathclose%
\pgfusepath{fill}%
\end{pgfscope}%
\begin{pgfscope}%
\pgfpathrectangle{\pgfqpoint{0.765000in}{0.660000in}}{\pgfqpoint{4.620000in}{4.620000in}}%
\pgfusepath{clip}%
\pgfsetbuttcap%
\pgfsetroundjoin%
\definecolor{currentfill}{rgb}{1.000000,0.894118,0.788235}%
\pgfsetfillcolor{currentfill}%
\pgfsetlinewidth{0.000000pt}%
\definecolor{currentstroke}{rgb}{1.000000,0.894118,0.788235}%
\pgfsetstrokecolor{currentstroke}%
\pgfsetdash{}{0pt}%
\pgfpathmoveto{\pgfqpoint{2.586744in}{2.945216in}}%
\pgfpathlineto{\pgfqpoint{2.697640in}{3.009242in}}%
\pgfpathlineto{\pgfqpoint{2.697640in}{3.137295in}}%
\pgfpathlineto{\pgfqpoint{2.586744in}{3.073269in}}%
\pgfpathlineto{\pgfqpoint{2.586744in}{2.945216in}}%
\pgfpathclose%
\pgfusepath{fill}%
\end{pgfscope}%
\begin{pgfscope}%
\pgfpathrectangle{\pgfqpoint{0.765000in}{0.660000in}}{\pgfqpoint{4.620000in}{4.620000in}}%
\pgfusepath{clip}%
\pgfsetbuttcap%
\pgfsetroundjoin%
\definecolor{currentfill}{rgb}{1.000000,0.894118,0.788235}%
\pgfsetfillcolor{currentfill}%
\pgfsetlinewidth{0.000000pt}%
\definecolor{currentstroke}{rgb}{1.000000,0.894118,0.788235}%
\pgfsetstrokecolor{currentstroke}%
\pgfsetdash{}{0pt}%
\pgfpathmoveto{\pgfqpoint{2.475847in}{3.009242in}}%
\pgfpathlineto{\pgfqpoint{2.586744in}{2.945216in}}%
\pgfpathlineto{\pgfqpoint{2.586744in}{3.073269in}}%
\pgfpathlineto{\pgfqpoint{2.475847in}{3.137295in}}%
\pgfpathlineto{\pgfqpoint{2.475847in}{3.009242in}}%
\pgfpathclose%
\pgfusepath{fill}%
\end{pgfscope}%
\begin{pgfscope}%
\pgfpathrectangle{\pgfqpoint{0.765000in}{0.660000in}}{\pgfqpoint{4.620000in}{4.620000in}}%
\pgfusepath{clip}%
\pgfsetbuttcap%
\pgfsetroundjoin%
\definecolor{currentfill}{rgb}{1.000000,0.894118,0.788235}%
\pgfsetfillcolor{currentfill}%
\pgfsetlinewidth{0.000000pt}%
\definecolor{currentstroke}{rgb}{1.000000,0.894118,0.788235}%
\pgfsetstrokecolor{currentstroke}%
\pgfsetdash{}{0pt}%
\pgfpathmoveto{\pgfqpoint{2.604377in}{3.013317in}}%
\pgfpathlineto{\pgfqpoint{2.493481in}{2.949291in}}%
\pgfpathlineto{\pgfqpoint{2.475847in}{3.009242in}}%
\pgfpathlineto{\pgfqpoint{2.586744in}{3.073269in}}%
\pgfpathlineto{\pgfqpoint{2.604377in}{3.013317in}}%
\pgfpathclose%
\pgfusepath{fill}%
\end{pgfscope}%
\begin{pgfscope}%
\pgfpathrectangle{\pgfqpoint{0.765000in}{0.660000in}}{\pgfqpoint{4.620000in}{4.620000in}}%
\pgfusepath{clip}%
\pgfsetbuttcap%
\pgfsetroundjoin%
\definecolor{currentfill}{rgb}{1.000000,0.894118,0.788235}%
\pgfsetfillcolor{currentfill}%
\pgfsetlinewidth{0.000000pt}%
\definecolor{currentstroke}{rgb}{1.000000,0.894118,0.788235}%
\pgfsetstrokecolor{currentstroke}%
\pgfsetdash{}{0pt}%
\pgfpathmoveto{\pgfqpoint{2.715274in}{2.949291in}}%
\pgfpathlineto{\pgfqpoint{2.604377in}{3.013317in}}%
\pgfpathlineto{\pgfqpoint{2.586744in}{3.073269in}}%
\pgfpathlineto{\pgfqpoint{2.697640in}{3.009242in}}%
\pgfpathlineto{\pgfqpoint{2.715274in}{2.949291in}}%
\pgfpathclose%
\pgfusepath{fill}%
\end{pgfscope}%
\begin{pgfscope}%
\pgfpathrectangle{\pgfqpoint{0.765000in}{0.660000in}}{\pgfqpoint{4.620000in}{4.620000in}}%
\pgfusepath{clip}%
\pgfsetbuttcap%
\pgfsetroundjoin%
\definecolor{currentfill}{rgb}{1.000000,0.894118,0.788235}%
\pgfsetfillcolor{currentfill}%
\pgfsetlinewidth{0.000000pt}%
\definecolor{currentstroke}{rgb}{1.000000,0.894118,0.788235}%
\pgfsetstrokecolor{currentstroke}%
\pgfsetdash{}{0pt}%
\pgfpathmoveto{\pgfqpoint{2.604377in}{3.013317in}}%
\pgfpathlineto{\pgfqpoint{2.604377in}{3.141370in}}%
\pgfpathlineto{\pgfqpoint{2.586744in}{3.201321in}}%
\pgfpathlineto{\pgfqpoint{2.697640in}{3.009242in}}%
\pgfpathlineto{\pgfqpoint{2.604377in}{3.013317in}}%
\pgfpathclose%
\pgfusepath{fill}%
\end{pgfscope}%
\begin{pgfscope}%
\pgfpathrectangle{\pgfqpoint{0.765000in}{0.660000in}}{\pgfqpoint{4.620000in}{4.620000in}}%
\pgfusepath{clip}%
\pgfsetbuttcap%
\pgfsetroundjoin%
\definecolor{currentfill}{rgb}{1.000000,0.894118,0.788235}%
\pgfsetfillcolor{currentfill}%
\pgfsetlinewidth{0.000000pt}%
\definecolor{currentstroke}{rgb}{1.000000,0.894118,0.788235}%
\pgfsetstrokecolor{currentstroke}%
\pgfsetdash{}{0pt}%
\pgfpathmoveto{\pgfqpoint{2.715274in}{2.949291in}}%
\pgfpathlineto{\pgfqpoint{2.715274in}{3.077344in}}%
\pgfpathlineto{\pgfqpoint{2.697640in}{3.137295in}}%
\pgfpathlineto{\pgfqpoint{2.586744in}{3.073269in}}%
\pgfpathlineto{\pgfqpoint{2.715274in}{2.949291in}}%
\pgfpathclose%
\pgfusepath{fill}%
\end{pgfscope}%
\begin{pgfscope}%
\pgfpathrectangle{\pgfqpoint{0.765000in}{0.660000in}}{\pgfqpoint{4.620000in}{4.620000in}}%
\pgfusepath{clip}%
\pgfsetbuttcap%
\pgfsetroundjoin%
\definecolor{currentfill}{rgb}{1.000000,0.894118,0.788235}%
\pgfsetfillcolor{currentfill}%
\pgfsetlinewidth{0.000000pt}%
\definecolor{currentstroke}{rgb}{1.000000,0.894118,0.788235}%
\pgfsetstrokecolor{currentstroke}%
\pgfsetdash{}{0pt}%
\pgfpathmoveto{\pgfqpoint{2.493481in}{2.949291in}}%
\pgfpathlineto{\pgfqpoint{2.604377in}{2.885265in}}%
\pgfpathlineto{\pgfqpoint{2.586744in}{2.945216in}}%
\pgfpathlineto{\pgfqpoint{2.475847in}{3.009242in}}%
\pgfpathlineto{\pgfqpoint{2.493481in}{2.949291in}}%
\pgfpathclose%
\pgfusepath{fill}%
\end{pgfscope}%
\begin{pgfscope}%
\pgfpathrectangle{\pgfqpoint{0.765000in}{0.660000in}}{\pgfqpoint{4.620000in}{4.620000in}}%
\pgfusepath{clip}%
\pgfsetbuttcap%
\pgfsetroundjoin%
\definecolor{currentfill}{rgb}{1.000000,0.894118,0.788235}%
\pgfsetfillcolor{currentfill}%
\pgfsetlinewidth{0.000000pt}%
\definecolor{currentstroke}{rgb}{1.000000,0.894118,0.788235}%
\pgfsetstrokecolor{currentstroke}%
\pgfsetdash{}{0pt}%
\pgfpathmoveto{\pgfqpoint{2.604377in}{2.885265in}}%
\pgfpathlineto{\pgfqpoint{2.715274in}{2.949291in}}%
\pgfpathlineto{\pgfqpoint{2.697640in}{3.009242in}}%
\pgfpathlineto{\pgfqpoint{2.586744in}{2.945216in}}%
\pgfpathlineto{\pgfqpoint{2.604377in}{2.885265in}}%
\pgfpathclose%
\pgfusepath{fill}%
\end{pgfscope}%
\begin{pgfscope}%
\pgfpathrectangle{\pgfqpoint{0.765000in}{0.660000in}}{\pgfqpoint{4.620000in}{4.620000in}}%
\pgfusepath{clip}%
\pgfsetbuttcap%
\pgfsetroundjoin%
\definecolor{currentfill}{rgb}{1.000000,0.894118,0.788235}%
\pgfsetfillcolor{currentfill}%
\pgfsetlinewidth{0.000000pt}%
\definecolor{currentstroke}{rgb}{1.000000,0.894118,0.788235}%
\pgfsetstrokecolor{currentstroke}%
\pgfsetdash{}{0pt}%
\pgfpathmoveto{\pgfqpoint{2.604377in}{3.141370in}}%
\pgfpathlineto{\pgfqpoint{2.493481in}{3.077344in}}%
\pgfpathlineto{\pgfqpoint{2.475847in}{3.137295in}}%
\pgfpathlineto{\pgfqpoint{2.586744in}{3.201321in}}%
\pgfpathlineto{\pgfqpoint{2.604377in}{3.141370in}}%
\pgfpathclose%
\pgfusepath{fill}%
\end{pgfscope}%
\begin{pgfscope}%
\pgfpathrectangle{\pgfqpoint{0.765000in}{0.660000in}}{\pgfqpoint{4.620000in}{4.620000in}}%
\pgfusepath{clip}%
\pgfsetbuttcap%
\pgfsetroundjoin%
\definecolor{currentfill}{rgb}{1.000000,0.894118,0.788235}%
\pgfsetfillcolor{currentfill}%
\pgfsetlinewidth{0.000000pt}%
\definecolor{currentstroke}{rgb}{1.000000,0.894118,0.788235}%
\pgfsetstrokecolor{currentstroke}%
\pgfsetdash{}{0pt}%
\pgfpathmoveto{\pgfqpoint{2.715274in}{3.077344in}}%
\pgfpathlineto{\pgfqpoint{2.604377in}{3.141370in}}%
\pgfpathlineto{\pgfqpoint{2.586744in}{3.201321in}}%
\pgfpathlineto{\pgfqpoint{2.697640in}{3.137295in}}%
\pgfpathlineto{\pgfqpoint{2.715274in}{3.077344in}}%
\pgfpathclose%
\pgfusepath{fill}%
\end{pgfscope}%
\begin{pgfscope}%
\pgfpathrectangle{\pgfqpoint{0.765000in}{0.660000in}}{\pgfqpoint{4.620000in}{4.620000in}}%
\pgfusepath{clip}%
\pgfsetbuttcap%
\pgfsetroundjoin%
\definecolor{currentfill}{rgb}{1.000000,0.894118,0.788235}%
\pgfsetfillcolor{currentfill}%
\pgfsetlinewidth{0.000000pt}%
\definecolor{currentstroke}{rgb}{1.000000,0.894118,0.788235}%
\pgfsetstrokecolor{currentstroke}%
\pgfsetdash{}{0pt}%
\pgfpathmoveto{\pgfqpoint{2.493481in}{2.949291in}}%
\pgfpathlineto{\pgfqpoint{2.493481in}{3.077344in}}%
\pgfpathlineto{\pgfqpoint{2.475847in}{3.137295in}}%
\pgfpathlineto{\pgfqpoint{2.586744in}{2.945216in}}%
\pgfpathlineto{\pgfqpoint{2.493481in}{2.949291in}}%
\pgfpathclose%
\pgfusepath{fill}%
\end{pgfscope}%
\begin{pgfscope}%
\pgfpathrectangle{\pgfqpoint{0.765000in}{0.660000in}}{\pgfqpoint{4.620000in}{4.620000in}}%
\pgfusepath{clip}%
\pgfsetbuttcap%
\pgfsetroundjoin%
\definecolor{currentfill}{rgb}{1.000000,0.894118,0.788235}%
\pgfsetfillcolor{currentfill}%
\pgfsetlinewidth{0.000000pt}%
\definecolor{currentstroke}{rgb}{1.000000,0.894118,0.788235}%
\pgfsetstrokecolor{currentstroke}%
\pgfsetdash{}{0pt}%
\pgfpathmoveto{\pgfqpoint{2.604377in}{2.885265in}}%
\pgfpathlineto{\pgfqpoint{2.604377in}{3.013317in}}%
\pgfpathlineto{\pgfqpoint{2.586744in}{3.073269in}}%
\pgfpathlineto{\pgfqpoint{2.475847in}{3.009242in}}%
\pgfpathlineto{\pgfqpoint{2.604377in}{2.885265in}}%
\pgfpathclose%
\pgfusepath{fill}%
\end{pgfscope}%
\begin{pgfscope}%
\pgfpathrectangle{\pgfqpoint{0.765000in}{0.660000in}}{\pgfqpoint{4.620000in}{4.620000in}}%
\pgfusepath{clip}%
\pgfsetbuttcap%
\pgfsetroundjoin%
\definecolor{currentfill}{rgb}{1.000000,0.894118,0.788235}%
\pgfsetfillcolor{currentfill}%
\pgfsetlinewidth{0.000000pt}%
\definecolor{currentstroke}{rgb}{1.000000,0.894118,0.788235}%
\pgfsetstrokecolor{currentstroke}%
\pgfsetdash{}{0pt}%
\pgfpathmoveto{\pgfqpoint{2.604377in}{3.013317in}}%
\pgfpathlineto{\pgfqpoint{2.715274in}{3.077344in}}%
\pgfpathlineto{\pgfqpoint{2.697640in}{3.137295in}}%
\pgfpathlineto{\pgfqpoint{2.586744in}{3.073269in}}%
\pgfpathlineto{\pgfqpoint{2.604377in}{3.013317in}}%
\pgfpathclose%
\pgfusepath{fill}%
\end{pgfscope}%
\begin{pgfscope}%
\pgfpathrectangle{\pgfqpoint{0.765000in}{0.660000in}}{\pgfqpoint{4.620000in}{4.620000in}}%
\pgfusepath{clip}%
\pgfsetbuttcap%
\pgfsetroundjoin%
\definecolor{currentfill}{rgb}{1.000000,0.894118,0.788235}%
\pgfsetfillcolor{currentfill}%
\pgfsetlinewidth{0.000000pt}%
\definecolor{currentstroke}{rgb}{1.000000,0.894118,0.788235}%
\pgfsetstrokecolor{currentstroke}%
\pgfsetdash{}{0pt}%
\pgfpathmoveto{\pgfqpoint{2.493481in}{3.077344in}}%
\pgfpathlineto{\pgfqpoint{2.604377in}{3.013317in}}%
\pgfpathlineto{\pgfqpoint{2.586744in}{3.073269in}}%
\pgfpathlineto{\pgfqpoint{2.475847in}{3.137295in}}%
\pgfpathlineto{\pgfqpoint{2.493481in}{3.077344in}}%
\pgfpathclose%
\pgfusepath{fill}%
\end{pgfscope}%
\begin{pgfscope}%
\pgfpathrectangle{\pgfqpoint{0.765000in}{0.660000in}}{\pgfqpoint{4.620000in}{4.620000in}}%
\pgfusepath{clip}%
\pgfsetbuttcap%
\pgfsetroundjoin%
\definecolor{currentfill}{rgb}{1.000000,0.894118,0.788235}%
\pgfsetfillcolor{currentfill}%
\pgfsetlinewidth{0.000000pt}%
\definecolor{currentstroke}{rgb}{1.000000,0.894118,0.788235}%
\pgfsetstrokecolor{currentstroke}%
\pgfsetdash{}{0pt}%
\pgfpathmoveto{\pgfqpoint{2.672370in}{2.810270in}}%
\pgfpathlineto{\pgfqpoint{2.561474in}{2.746243in}}%
\pgfpathlineto{\pgfqpoint{2.672370in}{2.682217in}}%
\pgfpathlineto{\pgfqpoint{2.783267in}{2.746243in}}%
\pgfpathlineto{\pgfqpoint{2.672370in}{2.810270in}}%
\pgfpathclose%
\pgfusepath{fill}%
\end{pgfscope}%
\begin{pgfscope}%
\pgfpathrectangle{\pgfqpoint{0.765000in}{0.660000in}}{\pgfqpoint{4.620000in}{4.620000in}}%
\pgfusepath{clip}%
\pgfsetbuttcap%
\pgfsetroundjoin%
\definecolor{currentfill}{rgb}{1.000000,0.894118,0.788235}%
\pgfsetfillcolor{currentfill}%
\pgfsetlinewidth{0.000000pt}%
\definecolor{currentstroke}{rgb}{1.000000,0.894118,0.788235}%
\pgfsetstrokecolor{currentstroke}%
\pgfsetdash{}{0pt}%
\pgfpathmoveto{\pgfqpoint{2.672370in}{2.810270in}}%
\pgfpathlineto{\pgfqpoint{2.561474in}{2.746243in}}%
\pgfpathlineto{\pgfqpoint{2.561474in}{2.874296in}}%
\pgfpathlineto{\pgfqpoint{2.672370in}{2.938322in}}%
\pgfpathlineto{\pgfqpoint{2.672370in}{2.810270in}}%
\pgfpathclose%
\pgfusepath{fill}%
\end{pgfscope}%
\begin{pgfscope}%
\pgfpathrectangle{\pgfqpoint{0.765000in}{0.660000in}}{\pgfqpoint{4.620000in}{4.620000in}}%
\pgfusepath{clip}%
\pgfsetbuttcap%
\pgfsetroundjoin%
\definecolor{currentfill}{rgb}{1.000000,0.894118,0.788235}%
\pgfsetfillcolor{currentfill}%
\pgfsetlinewidth{0.000000pt}%
\definecolor{currentstroke}{rgb}{1.000000,0.894118,0.788235}%
\pgfsetstrokecolor{currentstroke}%
\pgfsetdash{}{0pt}%
\pgfpathmoveto{\pgfqpoint{2.672370in}{2.810270in}}%
\pgfpathlineto{\pgfqpoint{2.783267in}{2.746243in}}%
\pgfpathlineto{\pgfqpoint{2.783267in}{2.874296in}}%
\pgfpathlineto{\pgfqpoint{2.672370in}{2.938322in}}%
\pgfpathlineto{\pgfqpoint{2.672370in}{2.810270in}}%
\pgfpathclose%
\pgfusepath{fill}%
\end{pgfscope}%
\begin{pgfscope}%
\pgfpathrectangle{\pgfqpoint{0.765000in}{0.660000in}}{\pgfqpoint{4.620000in}{4.620000in}}%
\pgfusepath{clip}%
\pgfsetbuttcap%
\pgfsetroundjoin%
\definecolor{currentfill}{rgb}{1.000000,0.894118,0.788235}%
\pgfsetfillcolor{currentfill}%
\pgfsetlinewidth{0.000000pt}%
\definecolor{currentstroke}{rgb}{1.000000,0.894118,0.788235}%
\pgfsetstrokecolor{currentstroke}%
\pgfsetdash{}{0pt}%
\pgfpathmoveto{\pgfqpoint{2.604377in}{3.013317in}}%
\pgfpathlineto{\pgfqpoint{2.493481in}{2.949291in}}%
\pgfpathlineto{\pgfqpoint{2.604377in}{2.885265in}}%
\pgfpathlineto{\pgfqpoint{2.715274in}{2.949291in}}%
\pgfpathlineto{\pgfqpoint{2.604377in}{3.013317in}}%
\pgfpathclose%
\pgfusepath{fill}%
\end{pgfscope}%
\begin{pgfscope}%
\pgfpathrectangle{\pgfqpoint{0.765000in}{0.660000in}}{\pgfqpoint{4.620000in}{4.620000in}}%
\pgfusepath{clip}%
\pgfsetbuttcap%
\pgfsetroundjoin%
\definecolor{currentfill}{rgb}{1.000000,0.894118,0.788235}%
\pgfsetfillcolor{currentfill}%
\pgfsetlinewidth{0.000000pt}%
\definecolor{currentstroke}{rgb}{1.000000,0.894118,0.788235}%
\pgfsetstrokecolor{currentstroke}%
\pgfsetdash{}{0pt}%
\pgfpathmoveto{\pgfqpoint{2.604377in}{3.013317in}}%
\pgfpathlineto{\pgfqpoint{2.493481in}{2.949291in}}%
\pgfpathlineto{\pgfqpoint{2.493481in}{3.077344in}}%
\pgfpathlineto{\pgfqpoint{2.604377in}{3.141370in}}%
\pgfpathlineto{\pgfqpoint{2.604377in}{3.013317in}}%
\pgfpathclose%
\pgfusepath{fill}%
\end{pgfscope}%
\begin{pgfscope}%
\pgfpathrectangle{\pgfqpoint{0.765000in}{0.660000in}}{\pgfqpoint{4.620000in}{4.620000in}}%
\pgfusepath{clip}%
\pgfsetbuttcap%
\pgfsetroundjoin%
\definecolor{currentfill}{rgb}{1.000000,0.894118,0.788235}%
\pgfsetfillcolor{currentfill}%
\pgfsetlinewidth{0.000000pt}%
\definecolor{currentstroke}{rgb}{1.000000,0.894118,0.788235}%
\pgfsetstrokecolor{currentstroke}%
\pgfsetdash{}{0pt}%
\pgfpathmoveto{\pgfqpoint{2.604377in}{3.013317in}}%
\pgfpathlineto{\pgfqpoint{2.715274in}{2.949291in}}%
\pgfpathlineto{\pgfqpoint{2.715274in}{3.077344in}}%
\pgfpathlineto{\pgfqpoint{2.604377in}{3.141370in}}%
\pgfpathlineto{\pgfqpoint{2.604377in}{3.013317in}}%
\pgfpathclose%
\pgfusepath{fill}%
\end{pgfscope}%
\begin{pgfscope}%
\pgfpathrectangle{\pgfqpoint{0.765000in}{0.660000in}}{\pgfqpoint{4.620000in}{4.620000in}}%
\pgfusepath{clip}%
\pgfsetbuttcap%
\pgfsetroundjoin%
\definecolor{currentfill}{rgb}{1.000000,0.894118,0.788235}%
\pgfsetfillcolor{currentfill}%
\pgfsetlinewidth{0.000000pt}%
\definecolor{currentstroke}{rgb}{1.000000,0.894118,0.788235}%
\pgfsetstrokecolor{currentstroke}%
\pgfsetdash{}{0pt}%
\pgfpathmoveto{\pgfqpoint{2.672370in}{2.938322in}}%
\pgfpathlineto{\pgfqpoint{2.561474in}{2.874296in}}%
\pgfpathlineto{\pgfqpoint{2.672370in}{2.810270in}}%
\pgfpathlineto{\pgfqpoint{2.783267in}{2.874296in}}%
\pgfpathlineto{\pgfqpoint{2.672370in}{2.938322in}}%
\pgfpathclose%
\pgfusepath{fill}%
\end{pgfscope}%
\begin{pgfscope}%
\pgfpathrectangle{\pgfqpoint{0.765000in}{0.660000in}}{\pgfqpoint{4.620000in}{4.620000in}}%
\pgfusepath{clip}%
\pgfsetbuttcap%
\pgfsetroundjoin%
\definecolor{currentfill}{rgb}{1.000000,0.894118,0.788235}%
\pgfsetfillcolor{currentfill}%
\pgfsetlinewidth{0.000000pt}%
\definecolor{currentstroke}{rgb}{1.000000,0.894118,0.788235}%
\pgfsetstrokecolor{currentstroke}%
\pgfsetdash{}{0pt}%
\pgfpathmoveto{\pgfqpoint{2.672370in}{2.682217in}}%
\pgfpathlineto{\pgfqpoint{2.783267in}{2.746243in}}%
\pgfpathlineto{\pgfqpoint{2.783267in}{2.874296in}}%
\pgfpathlineto{\pgfqpoint{2.672370in}{2.810270in}}%
\pgfpathlineto{\pgfqpoint{2.672370in}{2.682217in}}%
\pgfpathclose%
\pgfusepath{fill}%
\end{pgfscope}%
\begin{pgfscope}%
\pgfpathrectangle{\pgfqpoint{0.765000in}{0.660000in}}{\pgfqpoint{4.620000in}{4.620000in}}%
\pgfusepath{clip}%
\pgfsetbuttcap%
\pgfsetroundjoin%
\definecolor{currentfill}{rgb}{1.000000,0.894118,0.788235}%
\pgfsetfillcolor{currentfill}%
\pgfsetlinewidth{0.000000pt}%
\definecolor{currentstroke}{rgb}{1.000000,0.894118,0.788235}%
\pgfsetstrokecolor{currentstroke}%
\pgfsetdash{}{0pt}%
\pgfpathmoveto{\pgfqpoint{2.561474in}{2.746243in}}%
\pgfpathlineto{\pgfqpoint{2.672370in}{2.682217in}}%
\pgfpathlineto{\pgfqpoint{2.672370in}{2.810270in}}%
\pgfpathlineto{\pgfqpoint{2.561474in}{2.874296in}}%
\pgfpathlineto{\pgfqpoint{2.561474in}{2.746243in}}%
\pgfpathclose%
\pgfusepath{fill}%
\end{pgfscope}%
\begin{pgfscope}%
\pgfpathrectangle{\pgfqpoint{0.765000in}{0.660000in}}{\pgfqpoint{4.620000in}{4.620000in}}%
\pgfusepath{clip}%
\pgfsetbuttcap%
\pgfsetroundjoin%
\definecolor{currentfill}{rgb}{1.000000,0.894118,0.788235}%
\pgfsetfillcolor{currentfill}%
\pgfsetlinewidth{0.000000pt}%
\definecolor{currentstroke}{rgb}{1.000000,0.894118,0.788235}%
\pgfsetstrokecolor{currentstroke}%
\pgfsetdash{}{0pt}%
\pgfpathmoveto{\pgfqpoint{2.604377in}{3.141370in}}%
\pgfpathlineto{\pgfqpoint{2.493481in}{3.077344in}}%
\pgfpathlineto{\pgfqpoint{2.604377in}{3.013317in}}%
\pgfpathlineto{\pgfqpoint{2.715274in}{3.077344in}}%
\pgfpathlineto{\pgfqpoint{2.604377in}{3.141370in}}%
\pgfpathclose%
\pgfusepath{fill}%
\end{pgfscope}%
\begin{pgfscope}%
\pgfpathrectangle{\pgfqpoint{0.765000in}{0.660000in}}{\pgfqpoint{4.620000in}{4.620000in}}%
\pgfusepath{clip}%
\pgfsetbuttcap%
\pgfsetroundjoin%
\definecolor{currentfill}{rgb}{1.000000,0.894118,0.788235}%
\pgfsetfillcolor{currentfill}%
\pgfsetlinewidth{0.000000pt}%
\definecolor{currentstroke}{rgb}{1.000000,0.894118,0.788235}%
\pgfsetstrokecolor{currentstroke}%
\pgfsetdash{}{0pt}%
\pgfpathmoveto{\pgfqpoint{2.604377in}{2.885265in}}%
\pgfpathlineto{\pgfqpoint{2.715274in}{2.949291in}}%
\pgfpathlineto{\pgfqpoint{2.715274in}{3.077344in}}%
\pgfpathlineto{\pgfqpoint{2.604377in}{3.013317in}}%
\pgfpathlineto{\pgfqpoint{2.604377in}{2.885265in}}%
\pgfpathclose%
\pgfusepath{fill}%
\end{pgfscope}%
\begin{pgfscope}%
\pgfpathrectangle{\pgfqpoint{0.765000in}{0.660000in}}{\pgfqpoint{4.620000in}{4.620000in}}%
\pgfusepath{clip}%
\pgfsetbuttcap%
\pgfsetroundjoin%
\definecolor{currentfill}{rgb}{1.000000,0.894118,0.788235}%
\pgfsetfillcolor{currentfill}%
\pgfsetlinewidth{0.000000pt}%
\definecolor{currentstroke}{rgb}{1.000000,0.894118,0.788235}%
\pgfsetstrokecolor{currentstroke}%
\pgfsetdash{}{0pt}%
\pgfpathmoveto{\pgfqpoint{2.493481in}{2.949291in}}%
\pgfpathlineto{\pgfqpoint{2.604377in}{2.885265in}}%
\pgfpathlineto{\pgfqpoint{2.604377in}{3.013317in}}%
\pgfpathlineto{\pgfqpoint{2.493481in}{3.077344in}}%
\pgfpathlineto{\pgfqpoint{2.493481in}{2.949291in}}%
\pgfpathclose%
\pgfusepath{fill}%
\end{pgfscope}%
\begin{pgfscope}%
\pgfpathrectangle{\pgfqpoint{0.765000in}{0.660000in}}{\pgfqpoint{4.620000in}{4.620000in}}%
\pgfusepath{clip}%
\pgfsetbuttcap%
\pgfsetroundjoin%
\definecolor{currentfill}{rgb}{1.000000,0.894118,0.788235}%
\pgfsetfillcolor{currentfill}%
\pgfsetlinewidth{0.000000pt}%
\definecolor{currentstroke}{rgb}{1.000000,0.894118,0.788235}%
\pgfsetstrokecolor{currentstroke}%
\pgfsetdash{}{0pt}%
\pgfpathmoveto{\pgfqpoint{2.672370in}{2.810270in}}%
\pgfpathlineto{\pgfqpoint{2.561474in}{2.746243in}}%
\pgfpathlineto{\pgfqpoint{2.493481in}{2.949291in}}%
\pgfpathlineto{\pgfqpoint{2.604377in}{3.013317in}}%
\pgfpathlineto{\pgfqpoint{2.672370in}{2.810270in}}%
\pgfpathclose%
\pgfusepath{fill}%
\end{pgfscope}%
\begin{pgfscope}%
\pgfpathrectangle{\pgfqpoint{0.765000in}{0.660000in}}{\pgfqpoint{4.620000in}{4.620000in}}%
\pgfusepath{clip}%
\pgfsetbuttcap%
\pgfsetroundjoin%
\definecolor{currentfill}{rgb}{1.000000,0.894118,0.788235}%
\pgfsetfillcolor{currentfill}%
\pgfsetlinewidth{0.000000pt}%
\definecolor{currentstroke}{rgb}{1.000000,0.894118,0.788235}%
\pgfsetstrokecolor{currentstroke}%
\pgfsetdash{}{0pt}%
\pgfpathmoveto{\pgfqpoint{2.783267in}{2.746243in}}%
\pgfpathlineto{\pgfqpoint{2.672370in}{2.810270in}}%
\pgfpathlineto{\pgfqpoint{2.604377in}{3.013317in}}%
\pgfpathlineto{\pgfqpoint{2.715274in}{2.949291in}}%
\pgfpathlineto{\pgfqpoint{2.783267in}{2.746243in}}%
\pgfpathclose%
\pgfusepath{fill}%
\end{pgfscope}%
\begin{pgfscope}%
\pgfpathrectangle{\pgfqpoint{0.765000in}{0.660000in}}{\pgfqpoint{4.620000in}{4.620000in}}%
\pgfusepath{clip}%
\pgfsetbuttcap%
\pgfsetroundjoin%
\definecolor{currentfill}{rgb}{1.000000,0.894118,0.788235}%
\pgfsetfillcolor{currentfill}%
\pgfsetlinewidth{0.000000pt}%
\definecolor{currentstroke}{rgb}{1.000000,0.894118,0.788235}%
\pgfsetstrokecolor{currentstroke}%
\pgfsetdash{}{0pt}%
\pgfpathmoveto{\pgfqpoint{2.672370in}{2.810270in}}%
\pgfpathlineto{\pgfqpoint{2.672370in}{2.938322in}}%
\pgfpathlineto{\pgfqpoint{2.604377in}{3.141370in}}%
\pgfpathlineto{\pgfqpoint{2.715274in}{2.949291in}}%
\pgfpathlineto{\pgfqpoint{2.672370in}{2.810270in}}%
\pgfpathclose%
\pgfusepath{fill}%
\end{pgfscope}%
\begin{pgfscope}%
\pgfpathrectangle{\pgfqpoint{0.765000in}{0.660000in}}{\pgfqpoint{4.620000in}{4.620000in}}%
\pgfusepath{clip}%
\pgfsetbuttcap%
\pgfsetroundjoin%
\definecolor{currentfill}{rgb}{1.000000,0.894118,0.788235}%
\pgfsetfillcolor{currentfill}%
\pgfsetlinewidth{0.000000pt}%
\definecolor{currentstroke}{rgb}{1.000000,0.894118,0.788235}%
\pgfsetstrokecolor{currentstroke}%
\pgfsetdash{}{0pt}%
\pgfpathmoveto{\pgfqpoint{2.783267in}{2.746243in}}%
\pgfpathlineto{\pgfqpoint{2.783267in}{2.874296in}}%
\pgfpathlineto{\pgfqpoint{2.715274in}{3.077344in}}%
\pgfpathlineto{\pgfqpoint{2.604377in}{3.013317in}}%
\pgfpathlineto{\pgfqpoint{2.783267in}{2.746243in}}%
\pgfpathclose%
\pgfusepath{fill}%
\end{pgfscope}%
\begin{pgfscope}%
\pgfpathrectangle{\pgfqpoint{0.765000in}{0.660000in}}{\pgfqpoint{4.620000in}{4.620000in}}%
\pgfusepath{clip}%
\pgfsetbuttcap%
\pgfsetroundjoin%
\definecolor{currentfill}{rgb}{1.000000,0.894118,0.788235}%
\pgfsetfillcolor{currentfill}%
\pgfsetlinewidth{0.000000pt}%
\definecolor{currentstroke}{rgb}{1.000000,0.894118,0.788235}%
\pgfsetstrokecolor{currentstroke}%
\pgfsetdash{}{0pt}%
\pgfpathmoveto{\pgfqpoint{2.561474in}{2.746243in}}%
\pgfpathlineto{\pgfqpoint{2.672370in}{2.682217in}}%
\pgfpathlineto{\pgfqpoint{2.604377in}{2.885265in}}%
\pgfpathlineto{\pgfqpoint{2.493481in}{2.949291in}}%
\pgfpathlineto{\pgfqpoint{2.561474in}{2.746243in}}%
\pgfpathclose%
\pgfusepath{fill}%
\end{pgfscope}%
\begin{pgfscope}%
\pgfpathrectangle{\pgfqpoint{0.765000in}{0.660000in}}{\pgfqpoint{4.620000in}{4.620000in}}%
\pgfusepath{clip}%
\pgfsetbuttcap%
\pgfsetroundjoin%
\definecolor{currentfill}{rgb}{1.000000,0.894118,0.788235}%
\pgfsetfillcolor{currentfill}%
\pgfsetlinewidth{0.000000pt}%
\definecolor{currentstroke}{rgb}{1.000000,0.894118,0.788235}%
\pgfsetstrokecolor{currentstroke}%
\pgfsetdash{}{0pt}%
\pgfpathmoveto{\pgfqpoint{2.672370in}{2.682217in}}%
\pgfpathlineto{\pgfqpoint{2.783267in}{2.746243in}}%
\pgfpathlineto{\pgfqpoint{2.715274in}{2.949291in}}%
\pgfpathlineto{\pgfqpoint{2.604377in}{2.885265in}}%
\pgfpathlineto{\pgfqpoint{2.672370in}{2.682217in}}%
\pgfpathclose%
\pgfusepath{fill}%
\end{pgfscope}%
\begin{pgfscope}%
\pgfpathrectangle{\pgfqpoint{0.765000in}{0.660000in}}{\pgfqpoint{4.620000in}{4.620000in}}%
\pgfusepath{clip}%
\pgfsetbuttcap%
\pgfsetroundjoin%
\definecolor{currentfill}{rgb}{1.000000,0.894118,0.788235}%
\pgfsetfillcolor{currentfill}%
\pgfsetlinewidth{0.000000pt}%
\definecolor{currentstroke}{rgb}{1.000000,0.894118,0.788235}%
\pgfsetstrokecolor{currentstroke}%
\pgfsetdash{}{0pt}%
\pgfpathmoveto{\pgfqpoint{2.672370in}{2.938322in}}%
\pgfpathlineto{\pgfqpoint{2.561474in}{2.874296in}}%
\pgfpathlineto{\pgfqpoint{2.493481in}{3.077344in}}%
\pgfpathlineto{\pgfqpoint{2.604377in}{3.141370in}}%
\pgfpathlineto{\pgfqpoint{2.672370in}{2.938322in}}%
\pgfpathclose%
\pgfusepath{fill}%
\end{pgfscope}%
\begin{pgfscope}%
\pgfpathrectangle{\pgfqpoint{0.765000in}{0.660000in}}{\pgfqpoint{4.620000in}{4.620000in}}%
\pgfusepath{clip}%
\pgfsetbuttcap%
\pgfsetroundjoin%
\definecolor{currentfill}{rgb}{1.000000,0.894118,0.788235}%
\pgfsetfillcolor{currentfill}%
\pgfsetlinewidth{0.000000pt}%
\definecolor{currentstroke}{rgb}{1.000000,0.894118,0.788235}%
\pgfsetstrokecolor{currentstroke}%
\pgfsetdash{}{0pt}%
\pgfpathmoveto{\pgfqpoint{2.783267in}{2.874296in}}%
\pgfpathlineto{\pgfqpoint{2.672370in}{2.938322in}}%
\pgfpathlineto{\pgfqpoint{2.604377in}{3.141370in}}%
\pgfpathlineto{\pgfqpoint{2.715274in}{3.077344in}}%
\pgfpathlineto{\pgfqpoint{2.783267in}{2.874296in}}%
\pgfpathclose%
\pgfusepath{fill}%
\end{pgfscope}%
\begin{pgfscope}%
\pgfpathrectangle{\pgfqpoint{0.765000in}{0.660000in}}{\pgfqpoint{4.620000in}{4.620000in}}%
\pgfusepath{clip}%
\pgfsetbuttcap%
\pgfsetroundjoin%
\definecolor{currentfill}{rgb}{1.000000,0.894118,0.788235}%
\pgfsetfillcolor{currentfill}%
\pgfsetlinewidth{0.000000pt}%
\definecolor{currentstroke}{rgb}{1.000000,0.894118,0.788235}%
\pgfsetstrokecolor{currentstroke}%
\pgfsetdash{}{0pt}%
\pgfpathmoveto{\pgfqpoint{2.561474in}{2.746243in}}%
\pgfpathlineto{\pgfqpoint{2.561474in}{2.874296in}}%
\pgfpathlineto{\pgfqpoint{2.493481in}{3.077344in}}%
\pgfpathlineto{\pgfqpoint{2.604377in}{2.885265in}}%
\pgfpathlineto{\pgfqpoint{2.561474in}{2.746243in}}%
\pgfpathclose%
\pgfusepath{fill}%
\end{pgfscope}%
\begin{pgfscope}%
\pgfpathrectangle{\pgfqpoint{0.765000in}{0.660000in}}{\pgfqpoint{4.620000in}{4.620000in}}%
\pgfusepath{clip}%
\pgfsetbuttcap%
\pgfsetroundjoin%
\definecolor{currentfill}{rgb}{1.000000,0.894118,0.788235}%
\pgfsetfillcolor{currentfill}%
\pgfsetlinewidth{0.000000pt}%
\definecolor{currentstroke}{rgb}{1.000000,0.894118,0.788235}%
\pgfsetstrokecolor{currentstroke}%
\pgfsetdash{}{0pt}%
\pgfpathmoveto{\pgfqpoint{2.672370in}{2.682217in}}%
\pgfpathlineto{\pgfqpoint{2.672370in}{2.810270in}}%
\pgfpathlineto{\pgfqpoint{2.604377in}{3.013317in}}%
\pgfpathlineto{\pgfqpoint{2.493481in}{2.949291in}}%
\pgfpathlineto{\pgfqpoint{2.672370in}{2.682217in}}%
\pgfpathclose%
\pgfusepath{fill}%
\end{pgfscope}%
\begin{pgfscope}%
\pgfpathrectangle{\pgfqpoint{0.765000in}{0.660000in}}{\pgfqpoint{4.620000in}{4.620000in}}%
\pgfusepath{clip}%
\pgfsetbuttcap%
\pgfsetroundjoin%
\definecolor{currentfill}{rgb}{1.000000,0.894118,0.788235}%
\pgfsetfillcolor{currentfill}%
\pgfsetlinewidth{0.000000pt}%
\definecolor{currentstroke}{rgb}{1.000000,0.894118,0.788235}%
\pgfsetstrokecolor{currentstroke}%
\pgfsetdash{}{0pt}%
\pgfpathmoveto{\pgfqpoint{2.672370in}{2.810270in}}%
\pgfpathlineto{\pgfqpoint{2.783267in}{2.874296in}}%
\pgfpathlineto{\pgfqpoint{2.715274in}{3.077344in}}%
\pgfpathlineto{\pgfqpoint{2.604377in}{3.013317in}}%
\pgfpathlineto{\pgfqpoint{2.672370in}{2.810270in}}%
\pgfpathclose%
\pgfusepath{fill}%
\end{pgfscope}%
\begin{pgfscope}%
\pgfpathrectangle{\pgfqpoint{0.765000in}{0.660000in}}{\pgfqpoint{4.620000in}{4.620000in}}%
\pgfusepath{clip}%
\pgfsetbuttcap%
\pgfsetroundjoin%
\definecolor{currentfill}{rgb}{1.000000,0.894118,0.788235}%
\pgfsetfillcolor{currentfill}%
\pgfsetlinewidth{0.000000pt}%
\definecolor{currentstroke}{rgb}{1.000000,0.894118,0.788235}%
\pgfsetstrokecolor{currentstroke}%
\pgfsetdash{}{0pt}%
\pgfpathmoveto{\pgfqpoint{2.561474in}{2.874296in}}%
\pgfpathlineto{\pgfqpoint{2.672370in}{2.810270in}}%
\pgfpathlineto{\pgfqpoint{2.604377in}{3.013317in}}%
\pgfpathlineto{\pgfqpoint{2.493481in}{3.077344in}}%
\pgfpathlineto{\pgfqpoint{2.561474in}{2.874296in}}%
\pgfpathclose%
\pgfusepath{fill}%
\end{pgfscope}%
\begin{pgfscope}%
\pgfpathrectangle{\pgfqpoint{0.765000in}{0.660000in}}{\pgfqpoint{4.620000in}{4.620000in}}%
\pgfusepath{clip}%
\pgfsetbuttcap%
\pgfsetroundjoin%
\definecolor{currentfill}{rgb}{1.000000,0.894118,0.788235}%
\pgfsetfillcolor{currentfill}%
\pgfsetlinewidth{0.000000pt}%
\definecolor{currentstroke}{rgb}{1.000000,0.894118,0.788235}%
\pgfsetstrokecolor{currentstroke}%
\pgfsetdash{}{0pt}%
\pgfpathmoveto{\pgfqpoint{2.672370in}{2.810270in}}%
\pgfpathlineto{\pgfqpoint{2.561474in}{2.746243in}}%
\pgfpathlineto{\pgfqpoint{2.672370in}{2.682217in}}%
\pgfpathlineto{\pgfqpoint{2.783267in}{2.746243in}}%
\pgfpathlineto{\pgfqpoint{2.672370in}{2.810270in}}%
\pgfpathclose%
\pgfusepath{fill}%
\end{pgfscope}%
\begin{pgfscope}%
\pgfpathrectangle{\pgfqpoint{0.765000in}{0.660000in}}{\pgfqpoint{4.620000in}{4.620000in}}%
\pgfusepath{clip}%
\pgfsetbuttcap%
\pgfsetroundjoin%
\definecolor{currentfill}{rgb}{1.000000,0.894118,0.788235}%
\pgfsetfillcolor{currentfill}%
\pgfsetlinewidth{0.000000pt}%
\definecolor{currentstroke}{rgb}{1.000000,0.894118,0.788235}%
\pgfsetstrokecolor{currentstroke}%
\pgfsetdash{}{0pt}%
\pgfpathmoveto{\pgfqpoint{2.672370in}{2.810270in}}%
\pgfpathlineto{\pgfqpoint{2.561474in}{2.746243in}}%
\pgfpathlineto{\pgfqpoint{2.561474in}{2.874296in}}%
\pgfpathlineto{\pgfqpoint{2.672370in}{2.938322in}}%
\pgfpathlineto{\pgfqpoint{2.672370in}{2.810270in}}%
\pgfpathclose%
\pgfusepath{fill}%
\end{pgfscope}%
\begin{pgfscope}%
\pgfpathrectangle{\pgfqpoint{0.765000in}{0.660000in}}{\pgfqpoint{4.620000in}{4.620000in}}%
\pgfusepath{clip}%
\pgfsetbuttcap%
\pgfsetroundjoin%
\definecolor{currentfill}{rgb}{1.000000,0.894118,0.788235}%
\pgfsetfillcolor{currentfill}%
\pgfsetlinewidth{0.000000pt}%
\definecolor{currentstroke}{rgb}{1.000000,0.894118,0.788235}%
\pgfsetstrokecolor{currentstroke}%
\pgfsetdash{}{0pt}%
\pgfpathmoveto{\pgfqpoint{2.672370in}{2.810270in}}%
\pgfpathlineto{\pgfqpoint{2.783267in}{2.746243in}}%
\pgfpathlineto{\pgfqpoint{2.783267in}{2.874296in}}%
\pgfpathlineto{\pgfqpoint{2.672370in}{2.938322in}}%
\pgfpathlineto{\pgfqpoint{2.672370in}{2.810270in}}%
\pgfpathclose%
\pgfusepath{fill}%
\end{pgfscope}%
\begin{pgfscope}%
\pgfpathrectangle{\pgfqpoint{0.765000in}{0.660000in}}{\pgfqpoint{4.620000in}{4.620000in}}%
\pgfusepath{clip}%
\pgfsetbuttcap%
\pgfsetroundjoin%
\definecolor{currentfill}{rgb}{1.000000,0.894118,0.788235}%
\pgfsetfillcolor{currentfill}%
\pgfsetlinewidth{0.000000pt}%
\definecolor{currentstroke}{rgb}{1.000000,0.894118,0.788235}%
\pgfsetstrokecolor{currentstroke}%
\pgfsetdash{}{0pt}%
\pgfpathmoveto{\pgfqpoint{2.688236in}{2.773437in}}%
\pgfpathlineto{\pgfqpoint{2.577340in}{2.709411in}}%
\pgfpathlineto{\pgfqpoint{2.688236in}{2.645385in}}%
\pgfpathlineto{\pgfqpoint{2.799133in}{2.709411in}}%
\pgfpathlineto{\pgfqpoint{2.688236in}{2.773437in}}%
\pgfpathclose%
\pgfusepath{fill}%
\end{pgfscope}%
\begin{pgfscope}%
\pgfpathrectangle{\pgfqpoint{0.765000in}{0.660000in}}{\pgfqpoint{4.620000in}{4.620000in}}%
\pgfusepath{clip}%
\pgfsetbuttcap%
\pgfsetroundjoin%
\definecolor{currentfill}{rgb}{1.000000,0.894118,0.788235}%
\pgfsetfillcolor{currentfill}%
\pgfsetlinewidth{0.000000pt}%
\definecolor{currentstroke}{rgb}{1.000000,0.894118,0.788235}%
\pgfsetstrokecolor{currentstroke}%
\pgfsetdash{}{0pt}%
\pgfpathmoveto{\pgfqpoint{2.688236in}{2.773437in}}%
\pgfpathlineto{\pgfqpoint{2.577340in}{2.709411in}}%
\pgfpathlineto{\pgfqpoint{2.577340in}{2.837463in}}%
\pgfpathlineto{\pgfqpoint{2.688236in}{2.901490in}}%
\pgfpathlineto{\pgfqpoint{2.688236in}{2.773437in}}%
\pgfpathclose%
\pgfusepath{fill}%
\end{pgfscope}%
\begin{pgfscope}%
\pgfpathrectangle{\pgfqpoint{0.765000in}{0.660000in}}{\pgfqpoint{4.620000in}{4.620000in}}%
\pgfusepath{clip}%
\pgfsetbuttcap%
\pgfsetroundjoin%
\definecolor{currentfill}{rgb}{1.000000,0.894118,0.788235}%
\pgfsetfillcolor{currentfill}%
\pgfsetlinewidth{0.000000pt}%
\definecolor{currentstroke}{rgb}{1.000000,0.894118,0.788235}%
\pgfsetstrokecolor{currentstroke}%
\pgfsetdash{}{0pt}%
\pgfpathmoveto{\pgfqpoint{2.688236in}{2.773437in}}%
\pgfpathlineto{\pgfqpoint{2.799133in}{2.709411in}}%
\pgfpathlineto{\pgfqpoint{2.799133in}{2.837463in}}%
\pgfpathlineto{\pgfqpoint{2.688236in}{2.901490in}}%
\pgfpathlineto{\pgfqpoint{2.688236in}{2.773437in}}%
\pgfpathclose%
\pgfusepath{fill}%
\end{pgfscope}%
\begin{pgfscope}%
\pgfpathrectangle{\pgfqpoint{0.765000in}{0.660000in}}{\pgfqpoint{4.620000in}{4.620000in}}%
\pgfusepath{clip}%
\pgfsetbuttcap%
\pgfsetroundjoin%
\definecolor{currentfill}{rgb}{1.000000,0.894118,0.788235}%
\pgfsetfillcolor{currentfill}%
\pgfsetlinewidth{0.000000pt}%
\definecolor{currentstroke}{rgb}{1.000000,0.894118,0.788235}%
\pgfsetstrokecolor{currentstroke}%
\pgfsetdash{}{0pt}%
\pgfpathmoveto{\pgfqpoint{2.672370in}{2.938322in}}%
\pgfpathlineto{\pgfqpoint{2.561474in}{2.874296in}}%
\pgfpathlineto{\pgfqpoint{2.672370in}{2.810270in}}%
\pgfpathlineto{\pgfqpoint{2.783267in}{2.874296in}}%
\pgfpathlineto{\pgfqpoint{2.672370in}{2.938322in}}%
\pgfpathclose%
\pgfusepath{fill}%
\end{pgfscope}%
\begin{pgfscope}%
\pgfpathrectangle{\pgfqpoint{0.765000in}{0.660000in}}{\pgfqpoint{4.620000in}{4.620000in}}%
\pgfusepath{clip}%
\pgfsetbuttcap%
\pgfsetroundjoin%
\definecolor{currentfill}{rgb}{1.000000,0.894118,0.788235}%
\pgfsetfillcolor{currentfill}%
\pgfsetlinewidth{0.000000pt}%
\definecolor{currentstroke}{rgb}{1.000000,0.894118,0.788235}%
\pgfsetstrokecolor{currentstroke}%
\pgfsetdash{}{0pt}%
\pgfpathmoveto{\pgfqpoint{2.672370in}{2.682217in}}%
\pgfpathlineto{\pgfqpoint{2.783267in}{2.746243in}}%
\pgfpathlineto{\pgfqpoint{2.783267in}{2.874296in}}%
\pgfpathlineto{\pgfqpoint{2.672370in}{2.810270in}}%
\pgfpathlineto{\pgfqpoint{2.672370in}{2.682217in}}%
\pgfpathclose%
\pgfusepath{fill}%
\end{pgfscope}%
\begin{pgfscope}%
\pgfpathrectangle{\pgfqpoint{0.765000in}{0.660000in}}{\pgfqpoint{4.620000in}{4.620000in}}%
\pgfusepath{clip}%
\pgfsetbuttcap%
\pgfsetroundjoin%
\definecolor{currentfill}{rgb}{1.000000,0.894118,0.788235}%
\pgfsetfillcolor{currentfill}%
\pgfsetlinewidth{0.000000pt}%
\definecolor{currentstroke}{rgb}{1.000000,0.894118,0.788235}%
\pgfsetstrokecolor{currentstroke}%
\pgfsetdash{}{0pt}%
\pgfpathmoveto{\pgfqpoint{2.561474in}{2.746243in}}%
\pgfpathlineto{\pgfqpoint{2.672370in}{2.682217in}}%
\pgfpathlineto{\pgfqpoint{2.672370in}{2.810270in}}%
\pgfpathlineto{\pgfqpoint{2.561474in}{2.874296in}}%
\pgfpathlineto{\pgfqpoint{2.561474in}{2.746243in}}%
\pgfpathclose%
\pgfusepath{fill}%
\end{pgfscope}%
\begin{pgfscope}%
\pgfpathrectangle{\pgfqpoint{0.765000in}{0.660000in}}{\pgfqpoint{4.620000in}{4.620000in}}%
\pgfusepath{clip}%
\pgfsetbuttcap%
\pgfsetroundjoin%
\definecolor{currentfill}{rgb}{1.000000,0.894118,0.788235}%
\pgfsetfillcolor{currentfill}%
\pgfsetlinewidth{0.000000pt}%
\definecolor{currentstroke}{rgb}{1.000000,0.894118,0.788235}%
\pgfsetstrokecolor{currentstroke}%
\pgfsetdash{}{0pt}%
\pgfpathmoveto{\pgfqpoint{2.688236in}{2.901490in}}%
\pgfpathlineto{\pgfqpoint{2.577340in}{2.837463in}}%
\pgfpathlineto{\pgfqpoint{2.688236in}{2.773437in}}%
\pgfpathlineto{\pgfqpoint{2.799133in}{2.837463in}}%
\pgfpathlineto{\pgfqpoint{2.688236in}{2.901490in}}%
\pgfpathclose%
\pgfusepath{fill}%
\end{pgfscope}%
\begin{pgfscope}%
\pgfpathrectangle{\pgfqpoint{0.765000in}{0.660000in}}{\pgfqpoint{4.620000in}{4.620000in}}%
\pgfusepath{clip}%
\pgfsetbuttcap%
\pgfsetroundjoin%
\definecolor{currentfill}{rgb}{1.000000,0.894118,0.788235}%
\pgfsetfillcolor{currentfill}%
\pgfsetlinewidth{0.000000pt}%
\definecolor{currentstroke}{rgb}{1.000000,0.894118,0.788235}%
\pgfsetstrokecolor{currentstroke}%
\pgfsetdash{}{0pt}%
\pgfpathmoveto{\pgfqpoint{2.688236in}{2.645385in}}%
\pgfpathlineto{\pgfqpoint{2.799133in}{2.709411in}}%
\pgfpathlineto{\pgfqpoint{2.799133in}{2.837463in}}%
\pgfpathlineto{\pgfqpoint{2.688236in}{2.773437in}}%
\pgfpathlineto{\pgfqpoint{2.688236in}{2.645385in}}%
\pgfpathclose%
\pgfusepath{fill}%
\end{pgfscope}%
\begin{pgfscope}%
\pgfpathrectangle{\pgfqpoint{0.765000in}{0.660000in}}{\pgfqpoint{4.620000in}{4.620000in}}%
\pgfusepath{clip}%
\pgfsetbuttcap%
\pgfsetroundjoin%
\definecolor{currentfill}{rgb}{1.000000,0.894118,0.788235}%
\pgfsetfillcolor{currentfill}%
\pgfsetlinewidth{0.000000pt}%
\definecolor{currentstroke}{rgb}{1.000000,0.894118,0.788235}%
\pgfsetstrokecolor{currentstroke}%
\pgfsetdash{}{0pt}%
\pgfpathmoveto{\pgfqpoint{2.577340in}{2.709411in}}%
\pgfpathlineto{\pgfqpoint{2.688236in}{2.645385in}}%
\pgfpathlineto{\pgfqpoint{2.688236in}{2.773437in}}%
\pgfpathlineto{\pgfqpoint{2.577340in}{2.837463in}}%
\pgfpathlineto{\pgfqpoint{2.577340in}{2.709411in}}%
\pgfpathclose%
\pgfusepath{fill}%
\end{pgfscope}%
\begin{pgfscope}%
\pgfpathrectangle{\pgfqpoint{0.765000in}{0.660000in}}{\pgfqpoint{4.620000in}{4.620000in}}%
\pgfusepath{clip}%
\pgfsetbuttcap%
\pgfsetroundjoin%
\definecolor{currentfill}{rgb}{1.000000,0.894118,0.788235}%
\pgfsetfillcolor{currentfill}%
\pgfsetlinewidth{0.000000pt}%
\definecolor{currentstroke}{rgb}{1.000000,0.894118,0.788235}%
\pgfsetstrokecolor{currentstroke}%
\pgfsetdash{}{0pt}%
\pgfpathmoveto{\pgfqpoint{2.672370in}{2.810270in}}%
\pgfpathlineto{\pgfqpoint{2.561474in}{2.746243in}}%
\pgfpathlineto{\pgfqpoint{2.577340in}{2.709411in}}%
\pgfpathlineto{\pgfqpoint{2.688236in}{2.773437in}}%
\pgfpathlineto{\pgfqpoint{2.672370in}{2.810270in}}%
\pgfpathclose%
\pgfusepath{fill}%
\end{pgfscope}%
\begin{pgfscope}%
\pgfpathrectangle{\pgfqpoint{0.765000in}{0.660000in}}{\pgfqpoint{4.620000in}{4.620000in}}%
\pgfusepath{clip}%
\pgfsetbuttcap%
\pgfsetroundjoin%
\definecolor{currentfill}{rgb}{1.000000,0.894118,0.788235}%
\pgfsetfillcolor{currentfill}%
\pgfsetlinewidth{0.000000pt}%
\definecolor{currentstroke}{rgb}{1.000000,0.894118,0.788235}%
\pgfsetstrokecolor{currentstroke}%
\pgfsetdash{}{0pt}%
\pgfpathmoveto{\pgfqpoint{2.783267in}{2.746243in}}%
\pgfpathlineto{\pgfqpoint{2.672370in}{2.810270in}}%
\pgfpathlineto{\pgfqpoint{2.688236in}{2.773437in}}%
\pgfpathlineto{\pgfqpoint{2.799133in}{2.709411in}}%
\pgfpathlineto{\pgfqpoint{2.783267in}{2.746243in}}%
\pgfpathclose%
\pgfusepath{fill}%
\end{pgfscope}%
\begin{pgfscope}%
\pgfpathrectangle{\pgfqpoint{0.765000in}{0.660000in}}{\pgfqpoint{4.620000in}{4.620000in}}%
\pgfusepath{clip}%
\pgfsetbuttcap%
\pgfsetroundjoin%
\definecolor{currentfill}{rgb}{1.000000,0.894118,0.788235}%
\pgfsetfillcolor{currentfill}%
\pgfsetlinewidth{0.000000pt}%
\definecolor{currentstroke}{rgb}{1.000000,0.894118,0.788235}%
\pgfsetstrokecolor{currentstroke}%
\pgfsetdash{}{0pt}%
\pgfpathmoveto{\pgfqpoint{2.672370in}{2.810270in}}%
\pgfpathlineto{\pgfqpoint{2.672370in}{2.938322in}}%
\pgfpathlineto{\pgfqpoint{2.688236in}{2.901490in}}%
\pgfpathlineto{\pgfqpoint{2.799133in}{2.709411in}}%
\pgfpathlineto{\pgfqpoint{2.672370in}{2.810270in}}%
\pgfpathclose%
\pgfusepath{fill}%
\end{pgfscope}%
\begin{pgfscope}%
\pgfpathrectangle{\pgfqpoint{0.765000in}{0.660000in}}{\pgfqpoint{4.620000in}{4.620000in}}%
\pgfusepath{clip}%
\pgfsetbuttcap%
\pgfsetroundjoin%
\definecolor{currentfill}{rgb}{1.000000,0.894118,0.788235}%
\pgfsetfillcolor{currentfill}%
\pgfsetlinewidth{0.000000pt}%
\definecolor{currentstroke}{rgb}{1.000000,0.894118,0.788235}%
\pgfsetstrokecolor{currentstroke}%
\pgfsetdash{}{0pt}%
\pgfpathmoveto{\pgfqpoint{2.783267in}{2.746243in}}%
\pgfpathlineto{\pgfqpoint{2.783267in}{2.874296in}}%
\pgfpathlineto{\pgfqpoint{2.799133in}{2.837463in}}%
\pgfpathlineto{\pgfqpoint{2.688236in}{2.773437in}}%
\pgfpathlineto{\pgfqpoint{2.783267in}{2.746243in}}%
\pgfpathclose%
\pgfusepath{fill}%
\end{pgfscope}%
\begin{pgfscope}%
\pgfpathrectangle{\pgfqpoint{0.765000in}{0.660000in}}{\pgfqpoint{4.620000in}{4.620000in}}%
\pgfusepath{clip}%
\pgfsetbuttcap%
\pgfsetroundjoin%
\definecolor{currentfill}{rgb}{1.000000,0.894118,0.788235}%
\pgfsetfillcolor{currentfill}%
\pgfsetlinewidth{0.000000pt}%
\definecolor{currentstroke}{rgb}{1.000000,0.894118,0.788235}%
\pgfsetstrokecolor{currentstroke}%
\pgfsetdash{}{0pt}%
\pgfpathmoveto{\pgfqpoint{2.561474in}{2.746243in}}%
\pgfpathlineto{\pgfqpoint{2.672370in}{2.682217in}}%
\pgfpathlineto{\pgfqpoint{2.688236in}{2.645385in}}%
\pgfpathlineto{\pgfqpoint{2.577340in}{2.709411in}}%
\pgfpathlineto{\pgfqpoint{2.561474in}{2.746243in}}%
\pgfpathclose%
\pgfusepath{fill}%
\end{pgfscope}%
\begin{pgfscope}%
\pgfpathrectangle{\pgfqpoint{0.765000in}{0.660000in}}{\pgfqpoint{4.620000in}{4.620000in}}%
\pgfusepath{clip}%
\pgfsetbuttcap%
\pgfsetroundjoin%
\definecolor{currentfill}{rgb}{1.000000,0.894118,0.788235}%
\pgfsetfillcolor{currentfill}%
\pgfsetlinewidth{0.000000pt}%
\definecolor{currentstroke}{rgb}{1.000000,0.894118,0.788235}%
\pgfsetstrokecolor{currentstroke}%
\pgfsetdash{}{0pt}%
\pgfpathmoveto{\pgfqpoint{2.672370in}{2.682217in}}%
\pgfpathlineto{\pgfqpoint{2.783267in}{2.746243in}}%
\pgfpathlineto{\pgfqpoint{2.799133in}{2.709411in}}%
\pgfpathlineto{\pgfqpoint{2.688236in}{2.645385in}}%
\pgfpathlineto{\pgfqpoint{2.672370in}{2.682217in}}%
\pgfpathclose%
\pgfusepath{fill}%
\end{pgfscope}%
\begin{pgfscope}%
\pgfpathrectangle{\pgfqpoint{0.765000in}{0.660000in}}{\pgfqpoint{4.620000in}{4.620000in}}%
\pgfusepath{clip}%
\pgfsetbuttcap%
\pgfsetroundjoin%
\definecolor{currentfill}{rgb}{1.000000,0.894118,0.788235}%
\pgfsetfillcolor{currentfill}%
\pgfsetlinewidth{0.000000pt}%
\definecolor{currentstroke}{rgb}{1.000000,0.894118,0.788235}%
\pgfsetstrokecolor{currentstroke}%
\pgfsetdash{}{0pt}%
\pgfpathmoveto{\pgfqpoint{2.672370in}{2.938322in}}%
\pgfpathlineto{\pgfqpoint{2.561474in}{2.874296in}}%
\pgfpathlineto{\pgfqpoint{2.577340in}{2.837463in}}%
\pgfpathlineto{\pgfqpoint{2.688236in}{2.901490in}}%
\pgfpathlineto{\pgfqpoint{2.672370in}{2.938322in}}%
\pgfpathclose%
\pgfusepath{fill}%
\end{pgfscope}%
\begin{pgfscope}%
\pgfpathrectangle{\pgfqpoint{0.765000in}{0.660000in}}{\pgfqpoint{4.620000in}{4.620000in}}%
\pgfusepath{clip}%
\pgfsetbuttcap%
\pgfsetroundjoin%
\definecolor{currentfill}{rgb}{1.000000,0.894118,0.788235}%
\pgfsetfillcolor{currentfill}%
\pgfsetlinewidth{0.000000pt}%
\definecolor{currentstroke}{rgb}{1.000000,0.894118,0.788235}%
\pgfsetstrokecolor{currentstroke}%
\pgfsetdash{}{0pt}%
\pgfpathmoveto{\pgfqpoint{2.783267in}{2.874296in}}%
\pgfpathlineto{\pgfqpoint{2.672370in}{2.938322in}}%
\pgfpathlineto{\pgfqpoint{2.688236in}{2.901490in}}%
\pgfpathlineto{\pgfqpoint{2.799133in}{2.837463in}}%
\pgfpathlineto{\pgfqpoint{2.783267in}{2.874296in}}%
\pgfpathclose%
\pgfusepath{fill}%
\end{pgfscope}%
\begin{pgfscope}%
\pgfpathrectangle{\pgfqpoint{0.765000in}{0.660000in}}{\pgfqpoint{4.620000in}{4.620000in}}%
\pgfusepath{clip}%
\pgfsetbuttcap%
\pgfsetroundjoin%
\definecolor{currentfill}{rgb}{1.000000,0.894118,0.788235}%
\pgfsetfillcolor{currentfill}%
\pgfsetlinewidth{0.000000pt}%
\definecolor{currentstroke}{rgb}{1.000000,0.894118,0.788235}%
\pgfsetstrokecolor{currentstroke}%
\pgfsetdash{}{0pt}%
\pgfpathmoveto{\pgfqpoint{2.561474in}{2.746243in}}%
\pgfpathlineto{\pgfqpoint{2.561474in}{2.874296in}}%
\pgfpathlineto{\pgfqpoint{2.577340in}{2.837463in}}%
\pgfpathlineto{\pgfqpoint{2.688236in}{2.645385in}}%
\pgfpathlineto{\pgfqpoint{2.561474in}{2.746243in}}%
\pgfpathclose%
\pgfusepath{fill}%
\end{pgfscope}%
\begin{pgfscope}%
\pgfpathrectangle{\pgfqpoint{0.765000in}{0.660000in}}{\pgfqpoint{4.620000in}{4.620000in}}%
\pgfusepath{clip}%
\pgfsetbuttcap%
\pgfsetroundjoin%
\definecolor{currentfill}{rgb}{1.000000,0.894118,0.788235}%
\pgfsetfillcolor{currentfill}%
\pgfsetlinewidth{0.000000pt}%
\definecolor{currentstroke}{rgb}{1.000000,0.894118,0.788235}%
\pgfsetstrokecolor{currentstroke}%
\pgfsetdash{}{0pt}%
\pgfpathmoveto{\pgfqpoint{2.672370in}{2.682217in}}%
\pgfpathlineto{\pgfqpoint{2.672370in}{2.810270in}}%
\pgfpathlineto{\pgfqpoint{2.688236in}{2.773437in}}%
\pgfpathlineto{\pgfqpoint{2.577340in}{2.709411in}}%
\pgfpathlineto{\pgfqpoint{2.672370in}{2.682217in}}%
\pgfpathclose%
\pgfusepath{fill}%
\end{pgfscope}%
\begin{pgfscope}%
\pgfpathrectangle{\pgfqpoint{0.765000in}{0.660000in}}{\pgfqpoint{4.620000in}{4.620000in}}%
\pgfusepath{clip}%
\pgfsetbuttcap%
\pgfsetroundjoin%
\definecolor{currentfill}{rgb}{1.000000,0.894118,0.788235}%
\pgfsetfillcolor{currentfill}%
\pgfsetlinewidth{0.000000pt}%
\definecolor{currentstroke}{rgb}{1.000000,0.894118,0.788235}%
\pgfsetstrokecolor{currentstroke}%
\pgfsetdash{}{0pt}%
\pgfpathmoveto{\pgfqpoint{2.672370in}{2.810270in}}%
\pgfpathlineto{\pgfqpoint{2.783267in}{2.874296in}}%
\pgfpathlineto{\pgfqpoint{2.799133in}{2.837463in}}%
\pgfpathlineto{\pgfqpoint{2.688236in}{2.773437in}}%
\pgfpathlineto{\pgfqpoint{2.672370in}{2.810270in}}%
\pgfpathclose%
\pgfusepath{fill}%
\end{pgfscope}%
\begin{pgfscope}%
\pgfpathrectangle{\pgfqpoint{0.765000in}{0.660000in}}{\pgfqpoint{4.620000in}{4.620000in}}%
\pgfusepath{clip}%
\pgfsetbuttcap%
\pgfsetroundjoin%
\definecolor{currentfill}{rgb}{1.000000,0.894118,0.788235}%
\pgfsetfillcolor{currentfill}%
\pgfsetlinewidth{0.000000pt}%
\definecolor{currentstroke}{rgb}{1.000000,0.894118,0.788235}%
\pgfsetstrokecolor{currentstroke}%
\pgfsetdash{}{0pt}%
\pgfpathmoveto{\pgfqpoint{2.561474in}{2.874296in}}%
\pgfpathlineto{\pgfqpoint{2.672370in}{2.810270in}}%
\pgfpathlineto{\pgfqpoint{2.688236in}{2.773437in}}%
\pgfpathlineto{\pgfqpoint{2.577340in}{2.837463in}}%
\pgfpathlineto{\pgfqpoint{2.561474in}{2.874296in}}%
\pgfpathclose%
\pgfusepath{fill}%
\end{pgfscope}%
\begin{pgfscope}%
\pgfpathrectangle{\pgfqpoint{0.765000in}{0.660000in}}{\pgfqpoint{4.620000in}{4.620000in}}%
\pgfusepath{clip}%
\pgfsetbuttcap%
\pgfsetroundjoin%
\definecolor{currentfill}{rgb}{1.000000,0.894118,0.788235}%
\pgfsetfillcolor{currentfill}%
\pgfsetlinewidth{0.000000pt}%
\definecolor{currentstroke}{rgb}{1.000000,0.894118,0.788235}%
\pgfsetstrokecolor{currentstroke}%
\pgfsetdash{}{0pt}%
\pgfpathmoveto{\pgfqpoint{3.575586in}{2.945282in}}%
\pgfpathlineto{\pgfqpoint{3.464690in}{2.881255in}}%
\pgfpathlineto{\pgfqpoint{3.575586in}{2.817229in}}%
\pgfpathlineto{\pgfqpoint{3.686483in}{2.881255in}}%
\pgfpathlineto{\pgfqpoint{3.575586in}{2.945282in}}%
\pgfpathclose%
\pgfusepath{fill}%
\end{pgfscope}%
\begin{pgfscope}%
\pgfpathrectangle{\pgfqpoint{0.765000in}{0.660000in}}{\pgfqpoint{4.620000in}{4.620000in}}%
\pgfusepath{clip}%
\pgfsetbuttcap%
\pgfsetroundjoin%
\definecolor{currentfill}{rgb}{1.000000,0.894118,0.788235}%
\pgfsetfillcolor{currentfill}%
\pgfsetlinewidth{0.000000pt}%
\definecolor{currentstroke}{rgb}{1.000000,0.894118,0.788235}%
\pgfsetstrokecolor{currentstroke}%
\pgfsetdash{}{0pt}%
\pgfpathmoveto{\pgfqpoint{3.575586in}{2.945282in}}%
\pgfpathlineto{\pgfqpoint{3.464690in}{2.881255in}}%
\pgfpathlineto{\pgfqpoint{3.464690in}{3.009308in}}%
\pgfpathlineto{\pgfqpoint{3.575586in}{3.073334in}}%
\pgfpathlineto{\pgfqpoint{3.575586in}{2.945282in}}%
\pgfpathclose%
\pgfusepath{fill}%
\end{pgfscope}%
\begin{pgfscope}%
\pgfpathrectangle{\pgfqpoint{0.765000in}{0.660000in}}{\pgfqpoint{4.620000in}{4.620000in}}%
\pgfusepath{clip}%
\pgfsetbuttcap%
\pgfsetroundjoin%
\definecolor{currentfill}{rgb}{1.000000,0.894118,0.788235}%
\pgfsetfillcolor{currentfill}%
\pgfsetlinewidth{0.000000pt}%
\definecolor{currentstroke}{rgb}{1.000000,0.894118,0.788235}%
\pgfsetstrokecolor{currentstroke}%
\pgfsetdash{}{0pt}%
\pgfpathmoveto{\pgfqpoint{3.575586in}{2.945282in}}%
\pgfpathlineto{\pgfqpoint{3.686483in}{2.881255in}}%
\pgfpathlineto{\pgfqpoint{3.686483in}{3.009308in}}%
\pgfpathlineto{\pgfqpoint{3.575586in}{3.073334in}}%
\pgfpathlineto{\pgfqpoint{3.575586in}{2.945282in}}%
\pgfpathclose%
\pgfusepath{fill}%
\end{pgfscope}%
\begin{pgfscope}%
\pgfpathrectangle{\pgfqpoint{0.765000in}{0.660000in}}{\pgfqpoint{4.620000in}{4.620000in}}%
\pgfusepath{clip}%
\pgfsetbuttcap%
\pgfsetroundjoin%
\definecolor{currentfill}{rgb}{1.000000,0.894118,0.788235}%
\pgfsetfillcolor{currentfill}%
\pgfsetlinewidth{0.000000pt}%
\definecolor{currentstroke}{rgb}{1.000000,0.894118,0.788235}%
\pgfsetstrokecolor{currentstroke}%
\pgfsetdash{}{0pt}%
\pgfpathmoveto{\pgfqpoint{3.425935in}{3.326669in}}%
\pgfpathlineto{\pgfqpoint{3.315038in}{3.262643in}}%
\pgfpathlineto{\pgfqpoint{3.425935in}{3.198617in}}%
\pgfpathlineto{\pgfqpoint{3.536832in}{3.262643in}}%
\pgfpathlineto{\pgfqpoint{3.425935in}{3.326669in}}%
\pgfpathclose%
\pgfusepath{fill}%
\end{pgfscope}%
\begin{pgfscope}%
\pgfpathrectangle{\pgfqpoint{0.765000in}{0.660000in}}{\pgfqpoint{4.620000in}{4.620000in}}%
\pgfusepath{clip}%
\pgfsetbuttcap%
\pgfsetroundjoin%
\definecolor{currentfill}{rgb}{1.000000,0.894118,0.788235}%
\pgfsetfillcolor{currentfill}%
\pgfsetlinewidth{0.000000pt}%
\definecolor{currentstroke}{rgb}{1.000000,0.894118,0.788235}%
\pgfsetstrokecolor{currentstroke}%
\pgfsetdash{}{0pt}%
\pgfpathmoveto{\pgfqpoint{3.425935in}{3.326669in}}%
\pgfpathlineto{\pgfqpoint{3.315038in}{3.262643in}}%
\pgfpathlineto{\pgfqpoint{3.315038in}{3.390696in}}%
\pgfpathlineto{\pgfqpoint{3.425935in}{3.454722in}}%
\pgfpathlineto{\pgfqpoint{3.425935in}{3.326669in}}%
\pgfpathclose%
\pgfusepath{fill}%
\end{pgfscope}%
\begin{pgfscope}%
\pgfpathrectangle{\pgfqpoint{0.765000in}{0.660000in}}{\pgfqpoint{4.620000in}{4.620000in}}%
\pgfusepath{clip}%
\pgfsetbuttcap%
\pgfsetroundjoin%
\definecolor{currentfill}{rgb}{1.000000,0.894118,0.788235}%
\pgfsetfillcolor{currentfill}%
\pgfsetlinewidth{0.000000pt}%
\definecolor{currentstroke}{rgb}{1.000000,0.894118,0.788235}%
\pgfsetstrokecolor{currentstroke}%
\pgfsetdash{}{0pt}%
\pgfpathmoveto{\pgfqpoint{3.425935in}{3.326669in}}%
\pgfpathlineto{\pgfqpoint{3.536832in}{3.262643in}}%
\pgfpathlineto{\pgfqpoint{3.536832in}{3.390696in}}%
\pgfpathlineto{\pgfqpoint{3.425935in}{3.454722in}}%
\pgfpathlineto{\pgfqpoint{3.425935in}{3.326669in}}%
\pgfpathclose%
\pgfusepath{fill}%
\end{pgfscope}%
\begin{pgfscope}%
\pgfpathrectangle{\pgfqpoint{0.765000in}{0.660000in}}{\pgfqpoint{4.620000in}{4.620000in}}%
\pgfusepath{clip}%
\pgfsetbuttcap%
\pgfsetroundjoin%
\definecolor{currentfill}{rgb}{1.000000,0.894118,0.788235}%
\pgfsetfillcolor{currentfill}%
\pgfsetlinewidth{0.000000pt}%
\definecolor{currentstroke}{rgb}{1.000000,0.894118,0.788235}%
\pgfsetstrokecolor{currentstroke}%
\pgfsetdash{}{0pt}%
\pgfpathmoveto{\pgfqpoint{3.575586in}{3.073334in}}%
\pgfpathlineto{\pgfqpoint{3.464690in}{3.009308in}}%
\pgfpathlineto{\pgfqpoint{3.575586in}{2.945282in}}%
\pgfpathlineto{\pgfqpoint{3.686483in}{3.009308in}}%
\pgfpathlineto{\pgfqpoint{3.575586in}{3.073334in}}%
\pgfpathclose%
\pgfusepath{fill}%
\end{pgfscope}%
\begin{pgfscope}%
\pgfpathrectangle{\pgfqpoint{0.765000in}{0.660000in}}{\pgfqpoint{4.620000in}{4.620000in}}%
\pgfusepath{clip}%
\pgfsetbuttcap%
\pgfsetroundjoin%
\definecolor{currentfill}{rgb}{1.000000,0.894118,0.788235}%
\pgfsetfillcolor{currentfill}%
\pgfsetlinewidth{0.000000pt}%
\definecolor{currentstroke}{rgb}{1.000000,0.894118,0.788235}%
\pgfsetstrokecolor{currentstroke}%
\pgfsetdash{}{0pt}%
\pgfpathmoveto{\pgfqpoint{3.575586in}{2.817229in}}%
\pgfpathlineto{\pgfqpoint{3.686483in}{2.881255in}}%
\pgfpathlineto{\pgfqpoint{3.686483in}{3.009308in}}%
\pgfpathlineto{\pgfqpoint{3.575586in}{2.945282in}}%
\pgfpathlineto{\pgfqpoint{3.575586in}{2.817229in}}%
\pgfpathclose%
\pgfusepath{fill}%
\end{pgfscope}%
\begin{pgfscope}%
\pgfpathrectangle{\pgfqpoint{0.765000in}{0.660000in}}{\pgfqpoint{4.620000in}{4.620000in}}%
\pgfusepath{clip}%
\pgfsetbuttcap%
\pgfsetroundjoin%
\definecolor{currentfill}{rgb}{1.000000,0.894118,0.788235}%
\pgfsetfillcolor{currentfill}%
\pgfsetlinewidth{0.000000pt}%
\definecolor{currentstroke}{rgb}{1.000000,0.894118,0.788235}%
\pgfsetstrokecolor{currentstroke}%
\pgfsetdash{}{0pt}%
\pgfpathmoveto{\pgfqpoint{3.464690in}{2.881255in}}%
\pgfpathlineto{\pgfqpoint{3.575586in}{2.817229in}}%
\pgfpathlineto{\pgfqpoint{3.575586in}{2.945282in}}%
\pgfpathlineto{\pgfqpoint{3.464690in}{3.009308in}}%
\pgfpathlineto{\pgfqpoint{3.464690in}{2.881255in}}%
\pgfpathclose%
\pgfusepath{fill}%
\end{pgfscope}%
\begin{pgfscope}%
\pgfpathrectangle{\pgfqpoint{0.765000in}{0.660000in}}{\pgfqpoint{4.620000in}{4.620000in}}%
\pgfusepath{clip}%
\pgfsetbuttcap%
\pgfsetroundjoin%
\definecolor{currentfill}{rgb}{1.000000,0.894118,0.788235}%
\pgfsetfillcolor{currentfill}%
\pgfsetlinewidth{0.000000pt}%
\definecolor{currentstroke}{rgb}{1.000000,0.894118,0.788235}%
\pgfsetstrokecolor{currentstroke}%
\pgfsetdash{}{0pt}%
\pgfpathmoveto{\pgfqpoint{3.425935in}{3.454722in}}%
\pgfpathlineto{\pgfqpoint{3.315038in}{3.390696in}}%
\pgfpathlineto{\pgfqpoint{3.425935in}{3.326669in}}%
\pgfpathlineto{\pgfqpoint{3.536832in}{3.390696in}}%
\pgfpathlineto{\pgfqpoint{3.425935in}{3.454722in}}%
\pgfpathclose%
\pgfusepath{fill}%
\end{pgfscope}%
\begin{pgfscope}%
\pgfpathrectangle{\pgfqpoint{0.765000in}{0.660000in}}{\pgfqpoint{4.620000in}{4.620000in}}%
\pgfusepath{clip}%
\pgfsetbuttcap%
\pgfsetroundjoin%
\definecolor{currentfill}{rgb}{1.000000,0.894118,0.788235}%
\pgfsetfillcolor{currentfill}%
\pgfsetlinewidth{0.000000pt}%
\definecolor{currentstroke}{rgb}{1.000000,0.894118,0.788235}%
\pgfsetstrokecolor{currentstroke}%
\pgfsetdash{}{0pt}%
\pgfpathmoveto{\pgfqpoint{3.425935in}{3.198617in}}%
\pgfpathlineto{\pgfqpoint{3.536832in}{3.262643in}}%
\pgfpathlineto{\pgfqpoint{3.536832in}{3.390696in}}%
\pgfpathlineto{\pgfqpoint{3.425935in}{3.326669in}}%
\pgfpathlineto{\pgfqpoint{3.425935in}{3.198617in}}%
\pgfpathclose%
\pgfusepath{fill}%
\end{pgfscope}%
\begin{pgfscope}%
\pgfpathrectangle{\pgfqpoint{0.765000in}{0.660000in}}{\pgfqpoint{4.620000in}{4.620000in}}%
\pgfusepath{clip}%
\pgfsetbuttcap%
\pgfsetroundjoin%
\definecolor{currentfill}{rgb}{1.000000,0.894118,0.788235}%
\pgfsetfillcolor{currentfill}%
\pgfsetlinewidth{0.000000pt}%
\definecolor{currentstroke}{rgb}{1.000000,0.894118,0.788235}%
\pgfsetstrokecolor{currentstroke}%
\pgfsetdash{}{0pt}%
\pgfpathmoveto{\pgfqpoint{3.315038in}{3.262643in}}%
\pgfpathlineto{\pgfqpoint{3.425935in}{3.198617in}}%
\pgfpathlineto{\pgfqpoint{3.425935in}{3.326669in}}%
\pgfpathlineto{\pgfqpoint{3.315038in}{3.390696in}}%
\pgfpathlineto{\pgfqpoint{3.315038in}{3.262643in}}%
\pgfpathclose%
\pgfusepath{fill}%
\end{pgfscope}%
\begin{pgfscope}%
\pgfpathrectangle{\pgfqpoint{0.765000in}{0.660000in}}{\pgfqpoint{4.620000in}{4.620000in}}%
\pgfusepath{clip}%
\pgfsetbuttcap%
\pgfsetroundjoin%
\definecolor{currentfill}{rgb}{1.000000,0.894118,0.788235}%
\pgfsetfillcolor{currentfill}%
\pgfsetlinewidth{0.000000pt}%
\definecolor{currentstroke}{rgb}{1.000000,0.894118,0.788235}%
\pgfsetstrokecolor{currentstroke}%
\pgfsetdash{}{0pt}%
\pgfpathmoveto{\pgfqpoint{3.575586in}{2.945282in}}%
\pgfpathlineto{\pgfqpoint{3.464690in}{2.881255in}}%
\pgfpathlineto{\pgfqpoint{3.315038in}{3.262643in}}%
\pgfpathlineto{\pgfqpoint{3.425935in}{3.326669in}}%
\pgfpathlineto{\pgfqpoint{3.575586in}{2.945282in}}%
\pgfpathclose%
\pgfusepath{fill}%
\end{pgfscope}%
\begin{pgfscope}%
\pgfpathrectangle{\pgfqpoint{0.765000in}{0.660000in}}{\pgfqpoint{4.620000in}{4.620000in}}%
\pgfusepath{clip}%
\pgfsetbuttcap%
\pgfsetroundjoin%
\definecolor{currentfill}{rgb}{1.000000,0.894118,0.788235}%
\pgfsetfillcolor{currentfill}%
\pgfsetlinewidth{0.000000pt}%
\definecolor{currentstroke}{rgb}{1.000000,0.894118,0.788235}%
\pgfsetstrokecolor{currentstroke}%
\pgfsetdash{}{0pt}%
\pgfpathmoveto{\pgfqpoint{3.686483in}{2.881255in}}%
\pgfpathlineto{\pgfqpoint{3.575586in}{2.945282in}}%
\pgfpathlineto{\pgfqpoint{3.425935in}{3.326669in}}%
\pgfpathlineto{\pgfqpoint{3.536832in}{3.262643in}}%
\pgfpathlineto{\pgfqpoint{3.686483in}{2.881255in}}%
\pgfpathclose%
\pgfusepath{fill}%
\end{pgfscope}%
\begin{pgfscope}%
\pgfpathrectangle{\pgfqpoint{0.765000in}{0.660000in}}{\pgfqpoint{4.620000in}{4.620000in}}%
\pgfusepath{clip}%
\pgfsetbuttcap%
\pgfsetroundjoin%
\definecolor{currentfill}{rgb}{1.000000,0.894118,0.788235}%
\pgfsetfillcolor{currentfill}%
\pgfsetlinewidth{0.000000pt}%
\definecolor{currentstroke}{rgb}{1.000000,0.894118,0.788235}%
\pgfsetstrokecolor{currentstroke}%
\pgfsetdash{}{0pt}%
\pgfpathmoveto{\pgfqpoint{3.575586in}{2.945282in}}%
\pgfpathlineto{\pgfqpoint{3.575586in}{3.073334in}}%
\pgfpathlineto{\pgfqpoint{3.425935in}{3.454722in}}%
\pgfpathlineto{\pgfqpoint{3.536832in}{3.262643in}}%
\pgfpathlineto{\pgfqpoint{3.575586in}{2.945282in}}%
\pgfpathclose%
\pgfusepath{fill}%
\end{pgfscope}%
\begin{pgfscope}%
\pgfpathrectangle{\pgfqpoint{0.765000in}{0.660000in}}{\pgfqpoint{4.620000in}{4.620000in}}%
\pgfusepath{clip}%
\pgfsetbuttcap%
\pgfsetroundjoin%
\definecolor{currentfill}{rgb}{1.000000,0.894118,0.788235}%
\pgfsetfillcolor{currentfill}%
\pgfsetlinewidth{0.000000pt}%
\definecolor{currentstroke}{rgb}{1.000000,0.894118,0.788235}%
\pgfsetstrokecolor{currentstroke}%
\pgfsetdash{}{0pt}%
\pgfpathmoveto{\pgfqpoint{3.686483in}{2.881255in}}%
\pgfpathlineto{\pgfqpoint{3.686483in}{3.009308in}}%
\pgfpathlineto{\pgfqpoint{3.536832in}{3.390696in}}%
\pgfpathlineto{\pgfqpoint{3.425935in}{3.326669in}}%
\pgfpathlineto{\pgfqpoint{3.686483in}{2.881255in}}%
\pgfpathclose%
\pgfusepath{fill}%
\end{pgfscope}%
\begin{pgfscope}%
\pgfpathrectangle{\pgfqpoint{0.765000in}{0.660000in}}{\pgfqpoint{4.620000in}{4.620000in}}%
\pgfusepath{clip}%
\pgfsetbuttcap%
\pgfsetroundjoin%
\definecolor{currentfill}{rgb}{1.000000,0.894118,0.788235}%
\pgfsetfillcolor{currentfill}%
\pgfsetlinewidth{0.000000pt}%
\definecolor{currentstroke}{rgb}{1.000000,0.894118,0.788235}%
\pgfsetstrokecolor{currentstroke}%
\pgfsetdash{}{0pt}%
\pgfpathmoveto{\pgfqpoint{3.464690in}{2.881255in}}%
\pgfpathlineto{\pgfqpoint{3.575586in}{2.817229in}}%
\pgfpathlineto{\pgfqpoint{3.425935in}{3.198617in}}%
\pgfpathlineto{\pgfqpoint{3.315038in}{3.262643in}}%
\pgfpathlineto{\pgfqpoint{3.464690in}{2.881255in}}%
\pgfpathclose%
\pgfusepath{fill}%
\end{pgfscope}%
\begin{pgfscope}%
\pgfpathrectangle{\pgfqpoint{0.765000in}{0.660000in}}{\pgfqpoint{4.620000in}{4.620000in}}%
\pgfusepath{clip}%
\pgfsetbuttcap%
\pgfsetroundjoin%
\definecolor{currentfill}{rgb}{1.000000,0.894118,0.788235}%
\pgfsetfillcolor{currentfill}%
\pgfsetlinewidth{0.000000pt}%
\definecolor{currentstroke}{rgb}{1.000000,0.894118,0.788235}%
\pgfsetstrokecolor{currentstroke}%
\pgfsetdash{}{0pt}%
\pgfpathmoveto{\pgfqpoint{3.575586in}{2.817229in}}%
\pgfpathlineto{\pgfqpoint{3.686483in}{2.881255in}}%
\pgfpathlineto{\pgfqpoint{3.536832in}{3.262643in}}%
\pgfpathlineto{\pgfqpoint{3.425935in}{3.198617in}}%
\pgfpathlineto{\pgfqpoint{3.575586in}{2.817229in}}%
\pgfpathclose%
\pgfusepath{fill}%
\end{pgfscope}%
\begin{pgfscope}%
\pgfpathrectangle{\pgfqpoint{0.765000in}{0.660000in}}{\pgfqpoint{4.620000in}{4.620000in}}%
\pgfusepath{clip}%
\pgfsetbuttcap%
\pgfsetroundjoin%
\definecolor{currentfill}{rgb}{1.000000,0.894118,0.788235}%
\pgfsetfillcolor{currentfill}%
\pgfsetlinewidth{0.000000pt}%
\definecolor{currentstroke}{rgb}{1.000000,0.894118,0.788235}%
\pgfsetstrokecolor{currentstroke}%
\pgfsetdash{}{0pt}%
\pgfpathmoveto{\pgfqpoint{3.575586in}{3.073334in}}%
\pgfpathlineto{\pgfqpoint{3.464690in}{3.009308in}}%
\pgfpathlineto{\pgfqpoint{3.315038in}{3.390696in}}%
\pgfpathlineto{\pgfqpoint{3.425935in}{3.454722in}}%
\pgfpathlineto{\pgfqpoint{3.575586in}{3.073334in}}%
\pgfpathclose%
\pgfusepath{fill}%
\end{pgfscope}%
\begin{pgfscope}%
\pgfpathrectangle{\pgfqpoint{0.765000in}{0.660000in}}{\pgfqpoint{4.620000in}{4.620000in}}%
\pgfusepath{clip}%
\pgfsetbuttcap%
\pgfsetroundjoin%
\definecolor{currentfill}{rgb}{1.000000,0.894118,0.788235}%
\pgfsetfillcolor{currentfill}%
\pgfsetlinewidth{0.000000pt}%
\definecolor{currentstroke}{rgb}{1.000000,0.894118,0.788235}%
\pgfsetstrokecolor{currentstroke}%
\pgfsetdash{}{0pt}%
\pgfpathmoveto{\pgfqpoint{3.686483in}{3.009308in}}%
\pgfpathlineto{\pgfqpoint{3.575586in}{3.073334in}}%
\pgfpathlineto{\pgfqpoint{3.425935in}{3.454722in}}%
\pgfpathlineto{\pgfqpoint{3.536832in}{3.390696in}}%
\pgfpathlineto{\pgfqpoint{3.686483in}{3.009308in}}%
\pgfpathclose%
\pgfusepath{fill}%
\end{pgfscope}%
\begin{pgfscope}%
\pgfpathrectangle{\pgfqpoint{0.765000in}{0.660000in}}{\pgfqpoint{4.620000in}{4.620000in}}%
\pgfusepath{clip}%
\pgfsetbuttcap%
\pgfsetroundjoin%
\definecolor{currentfill}{rgb}{1.000000,0.894118,0.788235}%
\pgfsetfillcolor{currentfill}%
\pgfsetlinewidth{0.000000pt}%
\definecolor{currentstroke}{rgb}{1.000000,0.894118,0.788235}%
\pgfsetstrokecolor{currentstroke}%
\pgfsetdash{}{0pt}%
\pgfpathmoveto{\pgfqpoint{3.464690in}{2.881255in}}%
\pgfpathlineto{\pgfqpoint{3.464690in}{3.009308in}}%
\pgfpathlineto{\pgfqpoint{3.315038in}{3.390696in}}%
\pgfpathlineto{\pgfqpoint{3.425935in}{3.198617in}}%
\pgfpathlineto{\pgfqpoint{3.464690in}{2.881255in}}%
\pgfpathclose%
\pgfusepath{fill}%
\end{pgfscope}%
\begin{pgfscope}%
\pgfpathrectangle{\pgfqpoint{0.765000in}{0.660000in}}{\pgfqpoint{4.620000in}{4.620000in}}%
\pgfusepath{clip}%
\pgfsetbuttcap%
\pgfsetroundjoin%
\definecolor{currentfill}{rgb}{1.000000,0.894118,0.788235}%
\pgfsetfillcolor{currentfill}%
\pgfsetlinewidth{0.000000pt}%
\definecolor{currentstroke}{rgb}{1.000000,0.894118,0.788235}%
\pgfsetstrokecolor{currentstroke}%
\pgfsetdash{}{0pt}%
\pgfpathmoveto{\pgfqpoint{3.575586in}{2.817229in}}%
\pgfpathlineto{\pgfqpoint{3.575586in}{2.945282in}}%
\pgfpathlineto{\pgfqpoint{3.425935in}{3.326669in}}%
\pgfpathlineto{\pgfqpoint{3.315038in}{3.262643in}}%
\pgfpathlineto{\pgfqpoint{3.575586in}{2.817229in}}%
\pgfpathclose%
\pgfusepath{fill}%
\end{pgfscope}%
\begin{pgfscope}%
\pgfpathrectangle{\pgfqpoint{0.765000in}{0.660000in}}{\pgfqpoint{4.620000in}{4.620000in}}%
\pgfusepath{clip}%
\pgfsetbuttcap%
\pgfsetroundjoin%
\definecolor{currentfill}{rgb}{1.000000,0.894118,0.788235}%
\pgfsetfillcolor{currentfill}%
\pgfsetlinewidth{0.000000pt}%
\definecolor{currentstroke}{rgb}{1.000000,0.894118,0.788235}%
\pgfsetstrokecolor{currentstroke}%
\pgfsetdash{}{0pt}%
\pgfpathmoveto{\pgfqpoint{3.575586in}{2.945282in}}%
\pgfpathlineto{\pgfqpoint{3.686483in}{3.009308in}}%
\pgfpathlineto{\pgfqpoint{3.536832in}{3.390696in}}%
\pgfpathlineto{\pgfqpoint{3.425935in}{3.326669in}}%
\pgfpathlineto{\pgfqpoint{3.575586in}{2.945282in}}%
\pgfpathclose%
\pgfusepath{fill}%
\end{pgfscope}%
\begin{pgfscope}%
\pgfpathrectangle{\pgfqpoint{0.765000in}{0.660000in}}{\pgfqpoint{4.620000in}{4.620000in}}%
\pgfusepath{clip}%
\pgfsetbuttcap%
\pgfsetroundjoin%
\definecolor{currentfill}{rgb}{1.000000,0.894118,0.788235}%
\pgfsetfillcolor{currentfill}%
\pgfsetlinewidth{0.000000pt}%
\definecolor{currentstroke}{rgb}{1.000000,0.894118,0.788235}%
\pgfsetstrokecolor{currentstroke}%
\pgfsetdash{}{0pt}%
\pgfpathmoveto{\pgfqpoint{3.464690in}{3.009308in}}%
\pgfpathlineto{\pgfqpoint{3.575586in}{2.945282in}}%
\pgfpathlineto{\pgfqpoint{3.425935in}{3.326669in}}%
\pgfpathlineto{\pgfqpoint{3.315038in}{3.390696in}}%
\pgfpathlineto{\pgfqpoint{3.464690in}{3.009308in}}%
\pgfpathclose%
\pgfusepath{fill}%
\end{pgfscope}%
\begin{pgfscope}%
\pgfpathrectangle{\pgfqpoint{0.765000in}{0.660000in}}{\pgfqpoint{4.620000in}{4.620000in}}%
\pgfusepath{clip}%
\pgfsetbuttcap%
\pgfsetroundjoin%
\definecolor{currentfill}{rgb}{1.000000,0.894118,0.788235}%
\pgfsetfillcolor{currentfill}%
\pgfsetlinewidth{0.000000pt}%
\definecolor{currentstroke}{rgb}{1.000000,0.894118,0.788235}%
\pgfsetstrokecolor{currentstroke}%
\pgfsetdash{}{0pt}%
\pgfpathmoveto{\pgfqpoint{3.575586in}{2.945282in}}%
\pgfpathlineto{\pgfqpoint{3.464690in}{2.881255in}}%
\pgfpathlineto{\pgfqpoint{3.575586in}{2.817229in}}%
\pgfpathlineto{\pgfqpoint{3.686483in}{2.881255in}}%
\pgfpathlineto{\pgfqpoint{3.575586in}{2.945282in}}%
\pgfpathclose%
\pgfusepath{fill}%
\end{pgfscope}%
\begin{pgfscope}%
\pgfpathrectangle{\pgfqpoint{0.765000in}{0.660000in}}{\pgfqpoint{4.620000in}{4.620000in}}%
\pgfusepath{clip}%
\pgfsetbuttcap%
\pgfsetroundjoin%
\definecolor{currentfill}{rgb}{1.000000,0.894118,0.788235}%
\pgfsetfillcolor{currentfill}%
\pgfsetlinewidth{0.000000pt}%
\definecolor{currentstroke}{rgb}{1.000000,0.894118,0.788235}%
\pgfsetstrokecolor{currentstroke}%
\pgfsetdash{}{0pt}%
\pgfpathmoveto{\pgfqpoint{3.575586in}{2.945282in}}%
\pgfpathlineto{\pgfqpoint{3.464690in}{2.881255in}}%
\pgfpathlineto{\pgfqpoint{3.464690in}{3.009308in}}%
\pgfpathlineto{\pgfqpoint{3.575586in}{3.073334in}}%
\pgfpathlineto{\pgfqpoint{3.575586in}{2.945282in}}%
\pgfpathclose%
\pgfusepath{fill}%
\end{pgfscope}%
\begin{pgfscope}%
\pgfpathrectangle{\pgfqpoint{0.765000in}{0.660000in}}{\pgfqpoint{4.620000in}{4.620000in}}%
\pgfusepath{clip}%
\pgfsetbuttcap%
\pgfsetroundjoin%
\definecolor{currentfill}{rgb}{1.000000,0.894118,0.788235}%
\pgfsetfillcolor{currentfill}%
\pgfsetlinewidth{0.000000pt}%
\definecolor{currentstroke}{rgb}{1.000000,0.894118,0.788235}%
\pgfsetstrokecolor{currentstroke}%
\pgfsetdash{}{0pt}%
\pgfpathmoveto{\pgfqpoint{3.575586in}{2.945282in}}%
\pgfpathlineto{\pgfqpoint{3.686483in}{2.881255in}}%
\pgfpathlineto{\pgfqpoint{3.686483in}{3.009308in}}%
\pgfpathlineto{\pgfqpoint{3.575586in}{3.073334in}}%
\pgfpathlineto{\pgfqpoint{3.575586in}{2.945282in}}%
\pgfpathclose%
\pgfusepath{fill}%
\end{pgfscope}%
\begin{pgfscope}%
\pgfpathrectangle{\pgfqpoint{0.765000in}{0.660000in}}{\pgfqpoint{4.620000in}{4.620000in}}%
\pgfusepath{clip}%
\pgfsetbuttcap%
\pgfsetroundjoin%
\definecolor{currentfill}{rgb}{1.000000,0.894118,0.788235}%
\pgfsetfillcolor{currentfill}%
\pgfsetlinewidth{0.000000pt}%
\definecolor{currentstroke}{rgb}{1.000000,0.894118,0.788235}%
\pgfsetstrokecolor{currentstroke}%
\pgfsetdash{}{0pt}%
\pgfpathmoveto{\pgfqpoint{3.572472in}{2.854798in}}%
\pgfpathlineto{\pgfqpoint{3.461575in}{2.790772in}}%
\pgfpathlineto{\pgfqpoint{3.572472in}{2.726746in}}%
\pgfpathlineto{\pgfqpoint{3.683368in}{2.790772in}}%
\pgfpathlineto{\pgfqpoint{3.572472in}{2.854798in}}%
\pgfpathclose%
\pgfusepath{fill}%
\end{pgfscope}%
\begin{pgfscope}%
\pgfpathrectangle{\pgfqpoint{0.765000in}{0.660000in}}{\pgfqpoint{4.620000in}{4.620000in}}%
\pgfusepath{clip}%
\pgfsetbuttcap%
\pgfsetroundjoin%
\definecolor{currentfill}{rgb}{1.000000,0.894118,0.788235}%
\pgfsetfillcolor{currentfill}%
\pgfsetlinewidth{0.000000pt}%
\definecolor{currentstroke}{rgb}{1.000000,0.894118,0.788235}%
\pgfsetstrokecolor{currentstroke}%
\pgfsetdash{}{0pt}%
\pgfpathmoveto{\pgfqpoint{3.572472in}{2.854798in}}%
\pgfpathlineto{\pgfqpoint{3.461575in}{2.790772in}}%
\pgfpathlineto{\pgfqpoint{3.461575in}{2.918824in}}%
\pgfpathlineto{\pgfqpoint{3.572472in}{2.982851in}}%
\pgfpathlineto{\pgfqpoint{3.572472in}{2.854798in}}%
\pgfpathclose%
\pgfusepath{fill}%
\end{pgfscope}%
\begin{pgfscope}%
\pgfpathrectangle{\pgfqpoint{0.765000in}{0.660000in}}{\pgfqpoint{4.620000in}{4.620000in}}%
\pgfusepath{clip}%
\pgfsetbuttcap%
\pgfsetroundjoin%
\definecolor{currentfill}{rgb}{1.000000,0.894118,0.788235}%
\pgfsetfillcolor{currentfill}%
\pgfsetlinewidth{0.000000pt}%
\definecolor{currentstroke}{rgb}{1.000000,0.894118,0.788235}%
\pgfsetstrokecolor{currentstroke}%
\pgfsetdash{}{0pt}%
\pgfpathmoveto{\pgfqpoint{3.572472in}{2.854798in}}%
\pgfpathlineto{\pgfqpoint{3.683368in}{2.790772in}}%
\pgfpathlineto{\pgfqpoint{3.683368in}{2.918824in}}%
\pgfpathlineto{\pgfqpoint{3.572472in}{2.982851in}}%
\pgfpathlineto{\pgfqpoint{3.572472in}{2.854798in}}%
\pgfpathclose%
\pgfusepath{fill}%
\end{pgfscope}%
\begin{pgfscope}%
\pgfpathrectangle{\pgfqpoint{0.765000in}{0.660000in}}{\pgfqpoint{4.620000in}{4.620000in}}%
\pgfusepath{clip}%
\pgfsetbuttcap%
\pgfsetroundjoin%
\definecolor{currentfill}{rgb}{1.000000,0.894118,0.788235}%
\pgfsetfillcolor{currentfill}%
\pgfsetlinewidth{0.000000pt}%
\definecolor{currentstroke}{rgb}{1.000000,0.894118,0.788235}%
\pgfsetstrokecolor{currentstroke}%
\pgfsetdash{}{0pt}%
\pgfpathmoveto{\pgfqpoint{3.575586in}{3.073334in}}%
\pgfpathlineto{\pgfqpoint{3.464690in}{3.009308in}}%
\pgfpathlineto{\pgfqpoint{3.575586in}{2.945282in}}%
\pgfpathlineto{\pgfqpoint{3.686483in}{3.009308in}}%
\pgfpathlineto{\pgfqpoint{3.575586in}{3.073334in}}%
\pgfpathclose%
\pgfusepath{fill}%
\end{pgfscope}%
\begin{pgfscope}%
\pgfpathrectangle{\pgfqpoint{0.765000in}{0.660000in}}{\pgfqpoint{4.620000in}{4.620000in}}%
\pgfusepath{clip}%
\pgfsetbuttcap%
\pgfsetroundjoin%
\definecolor{currentfill}{rgb}{1.000000,0.894118,0.788235}%
\pgfsetfillcolor{currentfill}%
\pgfsetlinewidth{0.000000pt}%
\definecolor{currentstroke}{rgb}{1.000000,0.894118,0.788235}%
\pgfsetstrokecolor{currentstroke}%
\pgfsetdash{}{0pt}%
\pgfpathmoveto{\pgfqpoint{3.575586in}{2.817229in}}%
\pgfpathlineto{\pgfqpoint{3.686483in}{2.881255in}}%
\pgfpathlineto{\pgfqpoint{3.686483in}{3.009308in}}%
\pgfpathlineto{\pgfqpoint{3.575586in}{2.945282in}}%
\pgfpathlineto{\pgfqpoint{3.575586in}{2.817229in}}%
\pgfpathclose%
\pgfusepath{fill}%
\end{pgfscope}%
\begin{pgfscope}%
\pgfpathrectangle{\pgfqpoint{0.765000in}{0.660000in}}{\pgfqpoint{4.620000in}{4.620000in}}%
\pgfusepath{clip}%
\pgfsetbuttcap%
\pgfsetroundjoin%
\definecolor{currentfill}{rgb}{1.000000,0.894118,0.788235}%
\pgfsetfillcolor{currentfill}%
\pgfsetlinewidth{0.000000pt}%
\definecolor{currentstroke}{rgb}{1.000000,0.894118,0.788235}%
\pgfsetstrokecolor{currentstroke}%
\pgfsetdash{}{0pt}%
\pgfpathmoveto{\pgfqpoint{3.464690in}{2.881255in}}%
\pgfpathlineto{\pgfqpoint{3.575586in}{2.817229in}}%
\pgfpathlineto{\pgfqpoint{3.575586in}{2.945282in}}%
\pgfpathlineto{\pgfqpoint{3.464690in}{3.009308in}}%
\pgfpathlineto{\pgfqpoint{3.464690in}{2.881255in}}%
\pgfpathclose%
\pgfusepath{fill}%
\end{pgfscope}%
\begin{pgfscope}%
\pgfpathrectangle{\pgfqpoint{0.765000in}{0.660000in}}{\pgfqpoint{4.620000in}{4.620000in}}%
\pgfusepath{clip}%
\pgfsetbuttcap%
\pgfsetroundjoin%
\definecolor{currentfill}{rgb}{1.000000,0.894118,0.788235}%
\pgfsetfillcolor{currentfill}%
\pgfsetlinewidth{0.000000pt}%
\definecolor{currentstroke}{rgb}{1.000000,0.894118,0.788235}%
\pgfsetstrokecolor{currentstroke}%
\pgfsetdash{}{0pt}%
\pgfpathmoveto{\pgfqpoint{3.572472in}{2.982851in}}%
\pgfpathlineto{\pgfqpoint{3.461575in}{2.918824in}}%
\pgfpathlineto{\pgfqpoint{3.572472in}{2.854798in}}%
\pgfpathlineto{\pgfqpoint{3.683368in}{2.918824in}}%
\pgfpathlineto{\pgfqpoint{3.572472in}{2.982851in}}%
\pgfpathclose%
\pgfusepath{fill}%
\end{pgfscope}%
\begin{pgfscope}%
\pgfpathrectangle{\pgfqpoint{0.765000in}{0.660000in}}{\pgfqpoint{4.620000in}{4.620000in}}%
\pgfusepath{clip}%
\pgfsetbuttcap%
\pgfsetroundjoin%
\definecolor{currentfill}{rgb}{1.000000,0.894118,0.788235}%
\pgfsetfillcolor{currentfill}%
\pgfsetlinewidth{0.000000pt}%
\definecolor{currentstroke}{rgb}{1.000000,0.894118,0.788235}%
\pgfsetstrokecolor{currentstroke}%
\pgfsetdash{}{0pt}%
\pgfpathmoveto{\pgfqpoint{3.572472in}{2.726746in}}%
\pgfpathlineto{\pgfqpoint{3.683368in}{2.790772in}}%
\pgfpathlineto{\pgfqpoint{3.683368in}{2.918824in}}%
\pgfpathlineto{\pgfqpoint{3.572472in}{2.854798in}}%
\pgfpathlineto{\pgfqpoint{3.572472in}{2.726746in}}%
\pgfpathclose%
\pgfusepath{fill}%
\end{pgfscope}%
\begin{pgfscope}%
\pgfpathrectangle{\pgfqpoint{0.765000in}{0.660000in}}{\pgfqpoint{4.620000in}{4.620000in}}%
\pgfusepath{clip}%
\pgfsetbuttcap%
\pgfsetroundjoin%
\definecolor{currentfill}{rgb}{1.000000,0.894118,0.788235}%
\pgfsetfillcolor{currentfill}%
\pgfsetlinewidth{0.000000pt}%
\definecolor{currentstroke}{rgb}{1.000000,0.894118,0.788235}%
\pgfsetstrokecolor{currentstroke}%
\pgfsetdash{}{0pt}%
\pgfpathmoveto{\pgfqpoint{3.461575in}{2.790772in}}%
\pgfpathlineto{\pgfqpoint{3.572472in}{2.726746in}}%
\pgfpathlineto{\pgfqpoint{3.572472in}{2.854798in}}%
\pgfpathlineto{\pgfqpoint{3.461575in}{2.918824in}}%
\pgfpathlineto{\pgfqpoint{3.461575in}{2.790772in}}%
\pgfpathclose%
\pgfusepath{fill}%
\end{pgfscope}%
\begin{pgfscope}%
\pgfpathrectangle{\pgfqpoint{0.765000in}{0.660000in}}{\pgfqpoint{4.620000in}{4.620000in}}%
\pgfusepath{clip}%
\pgfsetbuttcap%
\pgfsetroundjoin%
\definecolor{currentfill}{rgb}{1.000000,0.894118,0.788235}%
\pgfsetfillcolor{currentfill}%
\pgfsetlinewidth{0.000000pt}%
\definecolor{currentstroke}{rgb}{1.000000,0.894118,0.788235}%
\pgfsetstrokecolor{currentstroke}%
\pgfsetdash{}{0pt}%
\pgfpathmoveto{\pgfqpoint{3.575586in}{2.945282in}}%
\pgfpathlineto{\pgfqpoint{3.464690in}{2.881255in}}%
\pgfpathlineto{\pgfqpoint{3.461575in}{2.790772in}}%
\pgfpathlineto{\pgfqpoint{3.572472in}{2.854798in}}%
\pgfpathlineto{\pgfqpoint{3.575586in}{2.945282in}}%
\pgfpathclose%
\pgfusepath{fill}%
\end{pgfscope}%
\begin{pgfscope}%
\pgfpathrectangle{\pgfqpoint{0.765000in}{0.660000in}}{\pgfqpoint{4.620000in}{4.620000in}}%
\pgfusepath{clip}%
\pgfsetbuttcap%
\pgfsetroundjoin%
\definecolor{currentfill}{rgb}{1.000000,0.894118,0.788235}%
\pgfsetfillcolor{currentfill}%
\pgfsetlinewidth{0.000000pt}%
\definecolor{currentstroke}{rgb}{1.000000,0.894118,0.788235}%
\pgfsetstrokecolor{currentstroke}%
\pgfsetdash{}{0pt}%
\pgfpathmoveto{\pgfqpoint{3.686483in}{2.881255in}}%
\pgfpathlineto{\pgfqpoint{3.575586in}{2.945282in}}%
\pgfpathlineto{\pgfqpoint{3.572472in}{2.854798in}}%
\pgfpathlineto{\pgfqpoint{3.683368in}{2.790772in}}%
\pgfpathlineto{\pgfqpoint{3.686483in}{2.881255in}}%
\pgfpathclose%
\pgfusepath{fill}%
\end{pgfscope}%
\begin{pgfscope}%
\pgfpathrectangle{\pgfqpoint{0.765000in}{0.660000in}}{\pgfqpoint{4.620000in}{4.620000in}}%
\pgfusepath{clip}%
\pgfsetbuttcap%
\pgfsetroundjoin%
\definecolor{currentfill}{rgb}{1.000000,0.894118,0.788235}%
\pgfsetfillcolor{currentfill}%
\pgfsetlinewidth{0.000000pt}%
\definecolor{currentstroke}{rgb}{1.000000,0.894118,0.788235}%
\pgfsetstrokecolor{currentstroke}%
\pgfsetdash{}{0pt}%
\pgfpathmoveto{\pgfqpoint{3.575586in}{2.945282in}}%
\pgfpathlineto{\pgfqpoint{3.575586in}{3.073334in}}%
\pgfpathlineto{\pgfqpoint{3.572472in}{2.982851in}}%
\pgfpathlineto{\pgfqpoint{3.683368in}{2.790772in}}%
\pgfpathlineto{\pgfqpoint{3.575586in}{2.945282in}}%
\pgfpathclose%
\pgfusepath{fill}%
\end{pgfscope}%
\begin{pgfscope}%
\pgfpathrectangle{\pgfqpoint{0.765000in}{0.660000in}}{\pgfqpoint{4.620000in}{4.620000in}}%
\pgfusepath{clip}%
\pgfsetbuttcap%
\pgfsetroundjoin%
\definecolor{currentfill}{rgb}{1.000000,0.894118,0.788235}%
\pgfsetfillcolor{currentfill}%
\pgfsetlinewidth{0.000000pt}%
\definecolor{currentstroke}{rgb}{1.000000,0.894118,0.788235}%
\pgfsetstrokecolor{currentstroke}%
\pgfsetdash{}{0pt}%
\pgfpathmoveto{\pgfqpoint{3.686483in}{2.881255in}}%
\pgfpathlineto{\pgfqpoint{3.686483in}{3.009308in}}%
\pgfpathlineto{\pgfqpoint{3.683368in}{2.918824in}}%
\pgfpathlineto{\pgfqpoint{3.572472in}{2.854798in}}%
\pgfpathlineto{\pgfqpoint{3.686483in}{2.881255in}}%
\pgfpathclose%
\pgfusepath{fill}%
\end{pgfscope}%
\begin{pgfscope}%
\pgfpathrectangle{\pgfqpoint{0.765000in}{0.660000in}}{\pgfqpoint{4.620000in}{4.620000in}}%
\pgfusepath{clip}%
\pgfsetbuttcap%
\pgfsetroundjoin%
\definecolor{currentfill}{rgb}{1.000000,0.894118,0.788235}%
\pgfsetfillcolor{currentfill}%
\pgfsetlinewidth{0.000000pt}%
\definecolor{currentstroke}{rgb}{1.000000,0.894118,0.788235}%
\pgfsetstrokecolor{currentstroke}%
\pgfsetdash{}{0pt}%
\pgfpathmoveto{\pgfqpoint{3.464690in}{2.881255in}}%
\pgfpathlineto{\pgfqpoint{3.575586in}{2.817229in}}%
\pgfpathlineto{\pgfqpoint{3.572472in}{2.726746in}}%
\pgfpathlineto{\pgfqpoint{3.461575in}{2.790772in}}%
\pgfpathlineto{\pgfqpoint{3.464690in}{2.881255in}}%
\pgfpathclose%
\pgfusepath{fill}%
\end{pgfscope}%
\begin{pgfscope}%
\pgfpathrectangle{\pgfqpoint{0.765000in}{0.660000in}}{\pgfqpoint{4.620000in}{4.620000in}}%
\pgfusepath{clip}%
\pgfsetbuttcap%
\pgfsetroundjoin%
\definecolor{currentfill}{rgb}{1.000000,0.894118,0.788235}%
\pgfsetfillcolor{currentfill}%
\pgfsetlinewidth{0.000000pt}%
\definecolor{currentstroke}{rgb}{1.000000,0.894118,0.788235}%
\pgfsetstrokecolor{currentstroke}%
\pgfsetdash{}{0pt}%
\pgfpathmoveto{\pgfqpoint{3.575586in}{2.817229in}}%
\pgfpathlineto{\pgfqpoint{3.686483in}{2.881255in}}%
\pgfpathlineto{\pgfqpoint{3.683368in}{2.790772in}}%
\pgfpathlineto{\pgfqpoint{3.572472in}{2.726746in}}%
\pgfpathlineto{\pgfqpoint{3.575586in}{2.817229in}}%
\pgfpathclose%
\pgfusepath{fill}%
\end{pgfscope}%
\begin{pgfscope}%
\pgfpathrectangle{\pgfqpoint{0.765000in}{0.660000in}}{\pgfqpoint{4.620000in}{4.620000in}}%
\pgfusepath{clip}%
\pgfsetbuttcap%
\pgfsetroundjoin%
\definecolor{currentfill}{rgb}{1.000000,0.894118,0.788235}%
\pgfsetfillcolor{currentfill}%
\pgfsetlinewidth{0.000000pt}%
\definecolor{currentstroke}{rgb}{1.000000,0.894118,0.788235}%
\pgfsetstrokecolor{currentstroke}%
\pgfsetdash{}{0pt}%
\pgfpathmoveto{\pgfqpoint{3.575586in}{3.073334in}}%
\pgfpathlineto{\pgfqpoint{3.464690in}{3.009308in}}%
\pgfpathlineto{\pgfqpoint{3.461575in}{2.918824in}}%
\pgfpathlineto{\pgfqpoint{3.572472in}{2.982851in}}%
\pgfpathlineto{\pgfqpoint{3.575586in}{3.073334in}}%
\pgfpathclose%
\pgfusepath{fill}%
\end{pgfscope}%
\begin{pgfscope}%
\pgfpathrectangle{\pgfqpoint{0.765000in}{0.660000in}}{\pgfqpoint{4.620000in}{4.620000in}}%
\pgfusepath{clip}%
\pgfsetbuttcap%
\pgfsetroundjoin%
\definecolor{currentfill}{rgb}{1.000000,0.894118,0.788235}%
\pgfsetfillcolor{currentfill}%
\pgfsetlinewidth{0.000000pt}%
\definecolor{currentstroke}{rgb}{1.000000,0.894118,0.788235}%
\pgfsetstrokecolor{currentstroke}%
\pgfsetdash{}{0pt}%
\pgfpathmoveto{\pgfqpoint{3.686483in}{3.009308in}}%
\pgfpathlineto{\pgfqpoint{3.575586in}{3.073334in}}%
\pgfpathlineto{\pgfqpoint{3.572472in}{2.982851in}}%
\pgfpathlineto{\pgfqpoint{3.683368in}{2.918824in}}%
\pgfpathlineto{\pgfqpoint{3.686483in}{3.009308in}}%
\pgfpathclose%
\pgfusepath{fill}%
\end{pgfscope}%
\begin{pgfscope}%
\pgfpathrectangle{\pgfqpoint{0.765000in}{0.660000in}}{\pgfqpoint{4.620000in}{4.620000in}}%
\pgfusepath{clip}%
\pgfsetbuttcap%
\pgfsetroundjoin%
\definecolor{currentfill}{rgb}{1.000000,0.894118,0.788235}%
\pgfsetfillcolor{currentfill}%
\pgfsetlinewidth{0.000000pt}%
\definecolor{currentstroke}{rgb}{1.000000,0.894118,0.788235}%
\pgfsetstrokecolor{currentstroke}%
\pgfsetdash{}{0pt}%
\pgfpathmoveto{\pgfqpoint{3.464690in}{2.881255in}}%
\pgfpathlineto{\pgfqpoint{3.464690in}{3.009308in}}%
\pgfpathlineto{\pgfqpoint{3.461575in}{2.918824in}}%
\pgfpathlineto{\pgfqpoint{3.572472in}{2.726746in}}%
\pgfpathlineto{\pgfqpoint{3.464690in}{2.881255in}}%
\pgfpathclose%
\pgfusepath{fill}%
\end{pgfscope}%
\begin{pgfscope}%
\pgfpathrectangle{\pgfqpoint{0.765000in}{0.660000in}}{\pgfqpoint{4.620000in}{4.620000in}}%
\pgfusepath{clip}%
\pgfsetbuttcap%
\pgfsetroundjoin%
\definecolor{currentfill}{rgb}{1.000000,0.894118,0.788235}%
\pgfsetfillcolor{currentfill}%
\pgfsetlinewidth{0.000000pt}%
\definecolor{currentstroke}{rgb}{1.000000,0.894118,0.788235}%
\pgfsetstrokecolor{currentstroke}%
\pgfsetdash{}{0pt}%
\pgfpathmoveto{\pgfqpoint{3.575586in}{2.817229in}}%
\pgfpathlineto{\pgfqpoint{3.575586in}{2.945282in}}%
\pgfpathlineto{\pgfqpoint{3.572472in}{2.854798in}}%
\pgfpathlineto{\pgfqpoint{3.461575in}{2.790772in}}%
\pgfpathlineto{\pgfqpoint{3.575586in}{2.817229in}}%
\pgfpathclose%
\pgfusepath{fill}%
\end{pgfscope}%
\begin{pgfscope}%
\pgfpathrectangle{\pgfqpoint{0.765000in}{0.660000in}}{\pgfqpoint{4.620000in}{4.620000in}}%
\pgfusepath{clip}%
\pgfsetbuttcap%
\pgfsetroundjoin%
\definecolor{currentfill}{rgb}{1.000000,0.894118,0.788235}%
\pgfsetfillcolor{currentfill}%
\pgfsetlinewidth{0.000000pt}%
\definecolor{currentstroke}{rgb}{1.000000,0.894118,0.788235}%
\pgfsetstrokecolor{currentstroke}%
\pgfsetdash{}{0pt}%
\pgfpathmoveto{\pgfqpoint{3.575586in}{2.945282in}}%
\pgfpathlineto{\pgfqpoint{3.686483in}{3.009308in}}%
\pgfpathlineto{\pgfqpoint{3.683368in}{2.918824in}}%
\pgfpathlineto{\pgfqpoint{3.572472in}{2.854798in}}%
\pgfpathlineto{\pgfqpoint{3.575586in}{2.945282in}}%
\pgfpathclose%
\pgfusepath{fill}%
\end{pgfscope}%
\begin{pgfscope}%
\pgfpathrectangle{\pgfqpoint{0.765000in}{0.660000in}}{\pgfqpoint{4.620000in}{4.620000in}}%
\pgfusepath{clip}%
\pgfsetbuttcap%
\pgfsetroundjoin%
\definecolor{currentfill}{rgb}{1.000000,0.894118,0.788235}%
\pgfsetfillcolor{currentfill}%
\pgfsetlinewidth{0.000000pt}%
\definecolor{currentstroke}{rgb}{1.000000,0.894118,0.788235}%
\pgfsetstrokecolor{currentstroke}%
\pgfsetdash{}{0pt}%
\pgfpathmoveto{\pgfqpoint{3.464690in}{3.009308in}}%
\pgfpathlineto{\pgfqpoint{3.575586in}{2.945282in}}%
\pgfpathlineto{\pgfqpoint{3.572472in}{2.854798in}}%
\pgfpathlineto{\pgfqpoint{3.461575in}{2.918824in}}%
\pgfpathlineto{\pgfqpoint{3.464690in}{3.009308in}}%
\pgfpathclose%
\pgfusepath{fill}%
\end{pgfscope}%
\begin{pgfscope}%
\pgfpathrectangle{\pgfqpoint{0.765000in}{0.660000in}}{\pgfqpoint{4.620000in}{4.620000in}}%
\pgfusepath{clip}%
\pgfsetbuttcap%
\pgfsetroundjoin%
\definecolor{currentfill}{rgb}{1.000000,0.894118,0.788235}%
\pgfsetfillcolor{currentfill}%
\pgfsetlinewidth{0.000000pt}%
\definecolor{currentstroke}{rgb}{1.000000,0.894118,0.788235}%
\pgfsetstrokecolor{currentstroke}%
\pgfsetdash{}{0pt}%
\pgfpathmoveto{\pgfqpoint{2.586744in}{3.073269in}}%
\pgfpathlineto{\pgfqpoint{2.475847in}{3.009242in}}%
\pgfpathlineto{\pgfqpoint{2.586744in}{2.945216in}}%
\pgfpathlineto{\pgfqpoint{2.697640in}{3.009242in}}%
\pgfpathlineto{\pgfqpoint{2.586744in}{3.073269in}}%
\pgfpathclose%
\pgfusepath{fill}%
\end{pgfscope}%
\begin{pgfscope}%
\pgfpathrectangle{\pgfqpoint{0.765000in}{0.660000in}}{\pgfqpoint{4.620000in}{4.620000in}}%
\pgfusepath{clip}%
\pgfsetbuttcap%
\pgfsetroundjoin%
\definecolor{currentfill}{rgb}{1.000000,0.894118,0.788235}%
\pgfsetfillcolor{currentfill}%
\pgfsetlinewidth{0.000000pt}%
\definecolor{currentstroke}{rgb}{1.000000,0.894118,0.788235}%
\pgfsetstrokecolor{currentstroke}%
\pgfsetdash{}{0pt}%
\pgfpathmoveto{\pgfqpoint{2.586744in}{3.073269in}}%
\pgfpathlineto{\pgfqpoint{2.475847in}{3.009242in}}%
\pgfpathlineto{\pgfqpoint{2.475847in}{3.137295in}}%
\pgfpathlineto{\pgfqpoint{2.586744in}{3.201321in}}%
\pgfpathlineto{\pgfqpoint{2.586744in}{3.073269in}}%
\pgfpathclose%
\pgfusepath{fill}%
\end{pgfscope}%
\begin{pgfscope}%
\pgfpathrectangle{\pgfqpoint{0.765000in}{0.660000in}}{\pgfqpoint{4.620000in}{4.620000in}}%
\pgfusepath{clip}%
\pgfsetbuttcap%
\pgfsetroundjoin%
\definecolor{currentfill}{rgb}{1.000000,0.894118,0.788235}%
\pgfsetfillcolor{currentfill}%
\pgfsetlinewidth{0.000000pt}%
\definecolor{currentstroke}{rgb}{1.000000,0.894118,0.788235}%
\pgfsetstrokecolor{currentstroke}%
\pgfsetdash{}{0pt}%
\pgfpathmoveto{\pgfqpoint{2.586744in}{3.073269in}}%
\pgfpathlineto{\pgfqpoint{2.697640in}{3.009242in}}%
\pgfpathlineto{\pgfqpoint{2.697640in}{3.137295in}}%
\pgfpathlineto{\pgfqpoint{2.586744in}{3.201321in}}%
\pgfpathlineto{\pgfqpoint{2.586744in}{3.073269in}}%
\pgfpathclose%
\pgfusepath{fill}%
\end{pgfscope}%
\begin{pgfscope}%
\pgfpathrectangle{\pgfqpoint{0.765000in}{0.660000in}}{\pgfqpoint{4.620000in}{4.620000in}}%
\pgfusepath{clip}%
\pgfsetbuttcap%
\pgfsetroundjoin%
\definecolor{currentfill}{rgb}{1.000000,0.894118,0.788235}%
\pgfsetfillcolor{currentfill}%
\pgfsetlinewidth{0.000000pt}%
\definecolor{currentstroke}{rgb}{1.000000,0.894118,0.788235}%
\pgfsetstrokecolor{currentstroke}%
\pgfsetdash{}{0pt}%
\pgfpathmoveto{\pgfqpoint{2.604377in}{3.013317in}}%
\pgfpathlineto{\pgfqpoint{2.493481in}{2.949291in}}%
\pgfpathlineto{\pgfqpoint{2.604377in}{2.885265in}}%
\pgfpathlineto{\pgfqpoint{2.715274in}{2.949291in}}%
\pgfpathlineto{\pgfqpoint{2.604377in}{3.013317in}}%
\pgfpathclose%
\pgfusepath{fill}%
\end{pgfscope}%
\begin{pgfscope}%
\pgfpathrectangle{\pgfqpoint{0.765000in}{0.660000in}}{\pgfqpoint{4.620000in}{4.620000in}}%
\pgfusepath{clip}%
\pgfsetbuttcap%
\pgfsetroundjoin%
\definecolor{currentfill}{rgb}{1.000000,0.894118,0.788235}%
\pgfsetfillcolor{currentfill}%
\pgfsetlinewidth{0.000000pt}%
\definecolor{currentstroke}{rgb}{1.000000,0.894118,0.788235}%
\pgfsetstrokecolor{currentstroke}%
\pgfsetdash{}{0pt}%
\pgfpathmoveto{\pgfqpoint{2.604377in}{3.013317in}}%
\pgfpathlineto{\pgfqpoint{2.493481in}{2.949291in}}%
\pgfpathlineto{\pgfqpoint{2.493481in}{3.077344in}}%
\pgfpathlineto{\pgfqpoint{2.604377in}{3.141370in}}%
\pgfpathlineto{\pgfqpoint{2.604377in}{3.013317in}}%
\pgfpathclose%
\pgfusepath{fill}%
\end{pgfscope}%
\begin{pgfscope}%
\pgfpathrectangle{\pgfqpoint{0.765000in}{0.660000in}}{\pgfqpoint{4.620000in}{4.620000in}}%
\pgfusepath{clip}%
\pgfsetbuttcap%
\pgfsetroundjoin%
\definecolor{currentfill}{rgb}{1.000000,0.894118,0.788235}%
\pgfsetfillcolor{currentfill}%
\pgfsetlinewidth{0.000000pt}%
\definecolor{currentstroke}{rgb}{1.000000,0.894118,0.788235}%
\pgfsetstrokecolor{currentstroke}%
\pgfsetdash{}{0pt}%
\pgfpathmoveto{\pgfqpoint{2.604377in}{3.013317in}}%
\pgfpathlineto{\pgfqpoint{2.715274in}{2.949291in}}%
\pgfpathlineto{\pgfqpoint{2.715274in}{3.077344in}}%
\pgfpathlineto{\pgfqpoint{2.604377in}{3.141370in}}%
\pgfpathlineto{\pgfqpoint{2.604377in}{3.013317in}}%
\pgfpathclose%
\pgfusepath{fill}%
\end{pgfscope}%
\begin{pgfscope}%
\pgfpathrectangle{\pgfqpoint{0.765000in}{0.660000in}}{\pgfqpoint{4.620000in}{4.620000in}}%
\pgfusepath{clip}%
\pgfsetbuttcap%
\pgfsetroundjoin%
\definecolor{currentfill}{rgb}{1.000000,0.894118,0.788235}%
\pgfsetfillcolor{currentfill}%
\pgfsetlinewidth{0.000000pt}%
\definecolor{currentstroke}{rgb}{1.000000,0.894118,0.788235}%
\pgfsetstrokecolor{currentstroke}%
\pgfsetdash{}{0pt}%
\pgfpathmoveto{\pgfqpoint{2.586744in}{3.201321in}}%
\pgfpathlineto{\pgfqpoint{2.475847in}{3.137295in}}%
\pgfpathlineto{\pgfqpoint{2.586744in}{3.073269in}}%
\pgfpathlineto{\pgfqpoint{2.697640in}{3.137295in}}%
\pgfpathlineto{\pgfqpoint{2.586744in}{3.201321in}}%
\pgfpathclose%
\pgfusepath{fill}%
\end{pgfscope}%
\begin{pgfscope}%
\pgfpathrectangle{\pgfqpoint{0.765000in}{0.660000in}}{\pgfqpoint{4.620000in}{4.620000in}}%
\pgfusepath{clip}%
\pgfsetbuttcap%
\pgfsetroundjoin%
\definecolor{currentfill}{rgb}{1.000000,0.894118,0.788235}%
\pgfsetfillcolor{currentfill}%
\pgfsetlinewidth{0.000000pt}%
\definecolor{currentstroke}{rgb}{1.000000,0.894118,0.788235}%
\pgfsetstrokecolor{currentstroke}%
\pgfsetdash{}{0pt}%
\pgfpathmoveto{\pgfqpoint{2.586744in}{2.945216in}}%
\pgfpathlineto{\pgfqpoint{2.697640in}{3.009242in}}%
\pgfpathlineto{\pgfqpoint{2.697640in}{3.137295in}}%
\pgfpathlineto{\pgfqpoint{2.586744in}{3.073269in}}%
\pgfpathlineto{\pgfqpoint{2.586744in}{2.945216in}}%
\pgfpathclose%
\pgfusepath{fill}%
\end{pgfscope}%
\begin{pgfscope}%
\pgfpathrectangle{\pgfqpoint{0.765000in}{0.660000in}}{\pgfqpoint{4.620000in}{4.620000in}}%
\pgfusepath{clip}%
\pgfsetbuttcap%
\pgfsetroundjoin%
\definecolor{currentfill}{rgb}{1.000000,0.894118,0.788235}%
\pgfsetfillcolor{currentfill}%
\pgfsetlinewidth{0.000000pt}%
\definecolor{currentstroke}{rgb}{1.000000,0.894118,0.788235}%
\pgfsetstrokecolor{currentstroke}%
\pgfsetdash{}{0pt}%
\pgfpathmoveto{\pgfqpoint{2.475847in}{3.009242in}}%
\pgfpathlineto{\pgfqpoint{2.586744in}{2.945216in}}%
\pgfpathlineto{\pgfqpoint{2.586744in}{3.073269in}}%
\pgfpathlineto{\pgfqpoint{2.475847in}{3.137295in}}%
\pgfpathlineto{\pgfqpoint{2.475847in}{3.009242in}}%
\pgfpathclose%
\pgfusepath{fill}%
\end{pgfscope}%
\begin{pgfscope}%
\pgfpathrectangle{\pgfqpoint{0.765000in}{0.660000in}}{\pgfqpoint{4.620000in}{4.620000in}}%
\pgfusepath{clip}%
\pgfsetbuttcap%
\pgfsetroundjoin%
\definecolor{currentfill}{rgb}{1.000000,0.894118,0.788235}%
\pgfsetfillcolor{currentfill}%
\pgfsetlinewidth{0.000000pt}%
\definecolor{currentstroke}{rgb}{1.000000,0.894118,0.788235}%
\pgfsetstrokecolor{currentstroke}%
\pgfsetdash{}{0pt}%
\pgfpathmoveto{\pgfqpoint{2.604377in}{3.141370in}}%
\pgfpathlineto{\pgfqpoint{2.493481in}{3.077344in}}%
\pgfpathlineto{\pgfqpoint{2.604377in}{3.013317in}}%
\pgfpathlineto{\pgfqpoint{2.715274in}{3.077344in}}%
\pgfpathlineto{\pgfqpoint{2.604377in}{3.141370in}}%
\pgfpathclose%
\pgfusepath{fill}%
\end{pgfscope}%
\begin{pgfscope}%
\pgfpathrectangle{\pgfqpoint{0.765000in}{0.660000in}}{\pgfqpoint{4.620000in}{4.620000in}}%
\pgfusepath{clip}%
\pgfsetbuttcap%
\pgfsetroundjoin%
\definecolor{currentfill}{rgb}{1.000000,0.894118,0.788235}%
\pgfsetfillcolor{currentfill}%
\pgfsetlinewidth{0.000000pt}%
\definecolor{currentstroke}{rgb}{1.000000,0.894118,0.788235}%
\pgfsetstrokecolor{currentstroke}%
\pgfsetdash{}{0pt}%
\pgfpathmoveto{\pgfqpoint{2.604377in}{2.885265in}}%
\pgfpathlineto{\pgfqpoint{2.715274in}{2.949291in}}%
\pgfpathlineto{\pgfqpoint{2.715274in}{3.077344in}}%
\pgfpathlineto{\pgfqpoint{2.604377in}{3.013317in}}%
\pgfpathlineto{\pgfqpoint{2.604377in}{2.885265in}}%
\pgfpathclose%
\pgfusepath{fill}%
\end{pgfscope}%
\begin{pgfscope}%
\pgfpathrectangle{\pgfqpoint{0.765000in}{0.660000in}}{\pgfqpoint{4.620000in}{4.620000in}}%
\pgfusepath{clip}%
\pgfsetbuttcap%
\pgfsetroundjoin%
\definecolor{currentfill}{rgb}{1.000000,0.894118,0.788235}%
\pgfsetfillcolor{currentfill}%
\pgfsetlinewidth{0.000000pt}%
\definecolor{currentstroke}{rgb}{1.000000,0.894118,0.788235}%
\pgfsetstrokecolor{currentstroke}%
\pgfsetdash{}{0pt}%
\pgfpathmoveto{\pgfqpoint{2.493481in}{2.949291in}}%
\pgfpathlineto{\pgfqpoint{2.604377in}{2.885265in}}%
\pgfpathlineto{\pgfqpoint{2.604377in}{3.013317in}}%
\pgfpathlineto{\pgfqpoint{2.493481in}{3.077344in}}%
\pgfpathlineto{\pgfqpoint{2.493481in}{2.949291in}}%
\pgfpathclose%
\pgfusepath{fill}%
\end{pgfscope}%
\begin{pgfscope}%
\pgfpathrectangle{\pgfqpoint{0.765000in}{0.660000in}}{\pgfqpoint{4.620000in}{4.620000in}}%
\pgfusepath{clip}%
\pgfsetbuttcap%
\pgfsetroundjoin%
\definecolor{currentfill}{rgb}{1.000000,0.894118,0.788235}%
\pgfsetfillcolor{currentfill}%
\pgfsetlinewidth{0.000000pt}%
\definecolor{currentstroke}{rgb}{1.000000,0.894118,0.788235}%
\pgfsetstrokecolor{currentstroke}%
\pgfsetdash{}{0pt}%
\pgfpathmoveto{\pgfqpoint{2.586744in}{3.073269in}}%
\pgfpathlineto{\pgfqpoint{2.475847in}{3.009242in}}%
\pgfpathlineto{\pgfqpoint{2.493481in}{2.949291in}}%
\pgfpathlineto{\pgfqpoint{2.604377in}{3.013317in}}%
\pgfpathlineto{\pgfqpoint{2.586744in}{3.073269in}}%
\pgfpathclose%
\pgfusepath{fill}%
\end{pgfscope}%
\begin{pgfscope}%
\pgfpathrectangle{\pgfqpoint{0.765000in}{0.660000in}}{\pgfqpoint{4.620000in}{4.620000in}}%
\pgfusepath{clip}%
\pgfsetbuttcap%
\pgfsetroundjoin%
\definecolor{currentfill}{rgb}{1.000000,0.894118,0.788235}%
\pgfsetfillcolor{currentfill}%
\pgfsetlinewidth{0.000000pt}%
\definecolor{currentstroke}{rgb}{1.000000,0.894118,0.788235}%
\pgfsetstrokecolor{currentstroke}%
\pgfsetdash{}{0pt}%
\pgfpathmoveto{\pgfqpoint{2.697640in}{3.009242in}}%
\pgfpathlineto{\pgfqpoint{2.586744in}{3.073269in}}%
\pgfpathlineto{\pgfqpoint{2.604377in}{3.013317in}}%
\pgfpathlineto{\pgfqpoint{2.715274in}{2.949291in}}%
\pgfpathlineto{\pgfqpoint{2.697640in}{3.009242in}}%
\pgfpathclose%
\pgfusepath{fill}%
\end{pgfscope}%
\begin{pgfscope}%
\pgfpathrectangle{\pgfqpoint{0.765000in}{0.660000in}}{\pgfqpoint{4.620000in}{4.620000in}}%
\pgfusepath{clip}%
\pgfsetbuttcap%
\pgfsetroundjoin%
\definecolor{currentfill}{rgb}{1.000000,0.894118,0.788235}%
\pgfsetfillcolor{currentfill}%
\pgfsetlinewidth{0.000000pt}%
\definecolor{currentstroke}{rgb}{1.000000,0.894118,0.788235}%
\pgfsetstrokecolor{currentstroke}%
\pgfsetdash{}{0pt}%
\pgfpathmoveto{\pgfqpoint{2.586744in}{3.073269in}}%
\pgfpathlineto{\pgfqpoint{2.586744in}{3.201321in}}%
\pgfpathlineto{\pgfqpoint{2.604377in}{3.141370in}}%
\pgfpathlineto{\pgfqpoint{2.715274in}{2.949291in}}%
\pgfpathlineto{\pgfqpoint{2.586744in}{3.073269in}}%
\pgfpathclose%
\pgfusepath{fill}%
\end{pgfscope}%
\begin{pgfscope}%
\pgfpathrectangle{\pgfqpoint{0.765000in}{0.660000in}}{\pgfqpoint{4.620000in}{4.620000in}}%
\pgfusepath{clip}%
\pgfsetbuttcap%
\pgfsetroundjoin%
\definecolor{currentfill}{rgb}{1.000000,0.894118,0.788235}%
\pgfsetfillcolor{currentfill}%
\pgfsetlinewidth{0.000000pt}%
\definecolor{currentstroke}{rgb}{1.000000,0.894118,0.788235}%
\pgfsetstrokecolor{currentstroke}%
\pgfsetdash{}{0pt}%
\pgfpathmoveto{\pgfqpoint{2.697640in}{3.009242in}}%
\pgfpathlineto{\pgfqpoint{2.697640in}{3.137295in}}%
\pgfpathlineto{\pgfqpoint{2.715274in}{3.077344in}}%
\pgfpathlineto{\pgfqpoint{2.604377in}{3.013317in}}%
\pgfpathlineto{\pgfqpoint{2.697640in}{3.009242in}}%
\pgfpathclose%
\pgfusepath{fill}%
\end{pgfscope}%
\begin{pgfscope}%
\pgfpathrectangle{\pgfqpoint{0.765000in}{0.660000in}}{\pgfqpoint{4.620000in}{4.620000in}}%
\pgfusepath{clip}%
\pgfsetbuttcap%
\pgfsetroundjoin%
\definecolor{currentfill}{rgb}{1.000000,0.894118,0.788235}%
\pgfsetfillcolor{currentfill}%
\pgfsetlinewidth{0.000000pt}%
\definecolor{currentstroke}{rgb}{1.000000,0.894118,0.788235}%
\pgfsetstrokecolor{currentstroke}%
\pgfsetdash{}{0pt}%
\pgfpathmoveto{\pgfqpoint{2.475847in}{3.009242in}}%
\pgfpathlineto{\pgfqpoint{2.586744in}{2.945216in}}%
\pgfpathlineto{\pgfqpoint{2.604377in}{2.885265in}}%
\pgfpathlineto{\pgfqpoint{2.493481in}{2.949291in}}%
\pgfpathlineto{\pgfqpoint{2.475847in}{3.009242in}}%
\pgfpathclose%
\pgfusepath{fill}%
\end{pgfscope}%
\begin{pgfscope}%
\pgfpathrectangle{\pgfqpoint{0.765000in}{0.660000in}}{\pgfqpoint{4.620000in}{4.620000in}}%
\pgfusepath{clip}%
\pgfsetbuttcap%
\pgfsetroundjoin%
\definecolor{currentfill}{rgb}{1.000000,0.894118,0.788235}%
\pgfsetfillcolor{currentfill}%
\pgfsetlinewidth{0.000000pt}%
\definecolor{currentstroke}{rgb}{1.000000,0.894118,0.788235}%
\pgfsetstrokecolor{currentstroke}%
\pgfsetdash{}{0pt}%
\pgfpathmoveto{\pgfqpoint{2.586744in}{2.945216in}}%
\pgfpathlineto{\pgfqpoint{2.697640in}{3.009242in}}%
\pgfpathlineto{\pgfqpoint{2.715274in}{2.949291in}}%
\pgfpathlineto{\pgfqpoint{2.604377in}{2.885265in}}%
\pgfpathlineto{\pgfqpoint{2.586744in}{2.945216in}}%
\pgfpathclose%
\pgfusepath{fill}%
\end{pgfscope}%
\begin{pgfscope}%
\pgfpathrectangle{\pgfqpoint{0.765000in}{0.660000in}}{\pgfqpoint{4.620000in}{4.620000in}}%
\pgfusepath{clip}%
\pgfsetbuttcap%
\pgfsetroundjoin%
\definecolor{currentfill}{rgb}{1.000000,0.894118,0.788235}%
\pgfsetfillcolor{currentfill}%
\pgfsetlinewidth{0.000000pt}%
\definecolor{currentstroke}{rgb}{1.000000,0.894118,0.788235}%
\pgfsetstrokecolor{currentstroke}%
\pgfsetdash{}{0pt}%
\pgfpathmoveto{\pgfqpoint{2.586744in}{3.201321in}}%
\pgfpathlineto{\pgfqpoint{2.475847in}{3.137295in}}%
\pgfpathlineto{\pgfqpoint{2.493481in}{3.077344in}}%
\pgfpathlineto{\pgfqpoint{2.604377in}{3.141370in}}%
\pgfpathlineto{\pgfqpoint{2.586744in}{3.201321in}}%
\pgfpathclose%
\pgfusepath{fill}%
\end{pgfscope}%
\begin{pgfscope}%
\pgfpathrectangle{\pgfqpoint{0.765000in}{0.660000in}}{\pgfqpoint{4.620000in}{4.620000in}}%
\pgfusepath{clip}%
\pgfsetbuttcap%
\pgfsetroundjoin%
\definecolor{currentfill}{rgb}{1.000000,0.894118,0.788235}%
\pgfsetfillcolor{currentfill}%
\pgfsetlinewidth{0.000000pt}%
\definecolor{currentstroke}{rgb}{1.000000,0.894118,0.788235}%
\pgfsetstrokecolor{currentstroke}%
\pgfsetdash{}{0pt}%
\pgfpathmoveto{\pgfqpoint{2.697640in}{3.137295in}}%
\pgfpathlineto{\pgfqpoint{2.586744in}{3.201321in}}%
\pgfpathlineto{\pgfqpoint{2.604377in}{3.141370in}}%
\pgfpathlineto{\pgfqpoint{2.715274in}{3.077344in}}%
\pgfpathlineto{\pgfqpoint{2.697640in}{3.137295in}}%
\pgfpathclose%
\pgfusepath{fill}%
\end{pgfscope}%
\begin{pgfscope}%
\pgfpathrectangle{\pgfqpoint{0.765000in}{0.660000in}}{\pgfqpoint{4.620000in}{4.620000in}}%
\pgfusepath{clip}%
\pgfsetbuttcap%
\pgfsetroundjoin%
\definecolor{currentfill}{rgb}{1.000000,0.894118,0.788235}%
\pgfsetfillcolor{currentfill}%
\pgfsetlinewidth{0.000000pt}%
\definecolor{currentstroke}{rgb}{1.000000,0.894118,0.788235}%
\pgfsetstrokecolor{currentstroke}%
\pgfsetdash{}{0pt}%
\pgfpathmoveto{\pgfqpoint{2.475847in}{3.009242in}}%
\pgfpathlineto{\pgfqpoint{2.475847in}{3.137295in}}%
\pgfpathlineto{\pgfqpoint{2.493481in}{3.077344in}}%
\pgfpathlineto{\pgfqpoint{2.604377in}{2.885265in}}%
\pgfpathlineto{\pgfqpoint{2.475847in}{3.009242in}}%
\pgfpathclose%
\pgfusepath{fill}%
\end{pgfscope}%
\begin{pgfscope}%
\pgfpathrectangle{\pgfqpoint{0.765000in}{0.660000in}}{\pgfqpoint{4.620000in}{4.620000in}}%
\pgfusepath{clip}%
\pgfsetbuttcap%
\pgfsetroundjoin%
\definecolor{currentfill}{rgb}{1.000000,0.894118,0.788235}%
\pgfsetfillcolor{currentfill}%
\pgfsetlinewidth{0.000000pt}%
\definecolor{currentstroke}{rgb}{1.000000,0.894118,0.788235}%
\pgfsetstrokecolor{currentstroke}%
\pgfsetdash{}{0pt}%
\pgfpathmoveto{\pgfqpoint{2.586744in}{2.945216in}}%
\pgfpathlineto{\pgfqpoint{2.586744in}{3.073269in}}%
\pgfpathlineto{\pgfqpoint{2.604377in}{3.013317in}}%
\pgfpathlineto{\pgfqpoint{2.493481in}{2.949291in}}%
\pgfpathlineto{\pgfqpoint{2.586744in}{2.945216in}}%
\pgfpathclose%
\pgfusepath{fill}%
\end{pgfscope}%
\begin{pgfscope}%
\pgfpathrectangle{\pgfqpoint{0.765000in}{0.660000in}}{\pgfqpoint{4.620000in}{4.620000in}}%
\pgfusepath{clip}%
\pgfsetbuttcap%
\pgfsetroundjoin%
\definecolor{currentfill}{rgb}{1.000000,0.894118,0.788235}%
\pgfsetfillcolor{currentfill}%
\pgfsetlinewidth{0.000000pt}%
\definecolor{currentstroke}{rgb}{1.000000,0.894118,0.788235}%
\pgfsetstrokecolor{currentstroke}%
\pgfsetdash{}{0pt}%
\pgfpathmoveto{\pgfqpoint{2.586744in}{3.073269in}}%
\pgfpathlineto{\pgfqpoint{2.697640in}{3.137295in}}%
\pgfpathlineto{\pgfqpoint{2.715274in}{3.077344in}}%
\pgfpathlineto{\pgfqpoint{2.604377in}{3.013317in}}%
\pgfpathlineto{\pgfqpoint{2.586744in}{3.073269in}}%
\pgfpathclose%
\pgfusepath{fill}%
\end{pgfscope}%
\begin{pgfscope}%
\pgfpathrectangle{\pgfqpoint{0.765000in}{0.660000in}}{\pgfqpoint{4.620000in}{4.620000in}}%
\pgfusepath{clip}%
\pgfsetbuttcap%
\pgfsetroundjoin%
\definecolor{currentfill}{rgb}{1.000000,0.894118,0.788235}%
\pgfsetfillcolor{currentfill}%
\pgfsetlinewidth{0.000000pt}%
\definecolor{currentstroke}{rgb}{1.000000,0.894118,0.788235}%
\pgfsetstrokecolor{currentstroke}%
\pgfsetdash{}{0pt}%
\pgfpathmoveto{\pgfqpoint{2.475847in}{3.137295in}}%
\pgfpathlineto{\pgfqpoint{2.586744in}{3.073269in}}%
\pgfpathlineto{\pgfqpoint{2.604377in}{3.013317in}}%
\pgfpathlineto{\pgfqpoint{2.493481in}{3.077344in}}%
\pgfpathlineto{\pgfqpoint{2.475847in}{3.137295in}}%
\pgfpathclose%
\pgfusepath{fill}%
\end{pgfscope}%
\begin{pgfscope}%
\pgfpathrectangle{\pgfqpoint{0.765000in}{0.660000in}}{\pgfqpoint{4.620000in}{4.620000in}}%
\pgfusepath{clip}%
\pgfsetbuttcap%
\pgfsetroundjoin%
\definecolor{currentfill}{rgb}{1.000000,0.894118,0.788235}%
\pgfsetfillcolor{currentfill}%
\pgfsetlinewidth{0.000000pt}%
\definecolor{currentstroke}{rgb}{1.000000,0.894118,0.788235}%
\pgfsetstrokecolor{currentstroke}%
\pgfsetdash{}{0pt}%
\pgfpathmoveto{\pgfqpoint{2.586744in}{3.073269in}}%
\pgfpathlineto{\pgfqpoint{2.475847in}{3.009242in}}%
\pgfpathlineto{\pgfqpoint{2.586744in}{2.945216in}}%
\pgfpathlineto{\pgfqpoint{2.697640in}{3.009242in}}%
\pgfpathlineto{\pgfqpoint{2.586744in}{3.073269in}}%
\pgfpathclose%
\pgfusepath{fill}%
\end{pgfscope}%
\begin{pgfscope}%
\pgfpathrectangle{\pgfqpoint{0.765000in}{0.660000in}}{\pgfqpoint{4.620000in}{4.620000in}}%
\pgfusepath{clip}%
\pgfsetbuttcap%
\pgfsetroundjoin%
\definecolor{currentfill}{rgb}{1.000000,0.894118,0.788235}%
\pgfsetfillcolor{currentfill}%
\pgfsetlinewidth{0.000000pt}%
\definecolor{currentstroke}{rgb}{1.000000,0.894118,0.788235}%
\pgfsetstrokecolor{currentstroke}%
\pgfsetdash{}{0pt}%
\pgfpathmoveto{\pgfqpoint{2.586744in}{3.073269in}}%
\pgfpathlineto{\pgfqpoint{2.475847in}{3.009242in}}%
\pgfpathlineto{\pgfqpoint{2.475847in}{3.137295in}}%
\pgfpathlineto{\pgfqpoint{2.586744in}{3.201321in}}%
\pgfpathlineto{\pgfqpoint{2.586744in}{3.073269in}}%
\pgfpathclose%
\pgfusepath{fill}%
\end{pgfscope}%
\begin{pgfscope}%
\pgfpathrectangle{\pgfqpoint{0.765000in}{0.660000in}}{\pgfqpoint{4.620000in}{4.620000in}}%
\pgfusepath{clip}%
\pgfsetbuttcap%
\pgfsetroundjoin%
\definecolor{currentfill}{rgb}{1.000000,0.894118,0.788235}%
\pgfsetfillcolor{currentfill}%
\pgfsetlinewidth{0.000000pt}%
\definecolor{currentstroke}{rgb}{1.000000,0.894118,0.788235}%
\pgfsetstrokecolor{currentstroke}%
\pgfsetdash{}{0pt}%
\pgfpathmoveto{\pgfqpoint{2.586744in}{3.073269in}}%
\pgfpathlineto{\pgfqpoint{2.697640in}{3.009242in}}%
\pgfpathlineto{\pgfqpoint{2.697640in}{3.137295in}}%
\pgfpathlineto{\pgfqpoint{2.586744in}{3.201321in}}%
\pgfpathlineto{\pgfqpoint{2.586744in}{3.073269in}}%
\pgfpathclose%
\pgfusepath{fill}%
\end{pgfscope}%
\begin{pgfscope}%
\pgfpathrectangle{\pgfqpoint{0.765000in}{0.660000in}}{\pgfqpoint{4.620000in}{4.620000in}}%
\pgfusepath{clip}%
\pgfsetbuttcap%
\pgfsetroundjoin%
\definecolor{currentfill}{rgb}{1.000000,0.894118,0.788235}%
\pgfsetfillcolor{currentfill}%
\pgfsetlinewidth{0.000000pt}%
\definecolor{currentstroke}{rgb}{1.000000,0.894118,0.788235}%
\pgfsetstrokecolor{currentstroke}%
\pgfsetdash{}{0pt}%
\pgfpathmoveto{\pgfqpoint{2.583171in}{3.110796in}}%
\pgfpathlineto{\pgfqpoint{2.472275in}{3.046769in}}%
\pgfpathlineto{\pgfqpoint{2.583171in}{2.982743in}}%
\pgfpathlineto{\pgfqpoint{2.694068in}{3.046769in}}%
\pgfpathlineto{\pgfqpoint{2.583171in}{3.110796in}}%
\pgfpathclose%
\pgfusepath{fill}%
\end{pgfscope}%
\begin{pgfscope}%
\pgfpathrectangle{\pgfqpoint{0.765000in}{0.660000in}}{\pgfqpoint{4.620000in}{4.620000in}}%
\pgfusepath{clip}%
\pgfsetbuttcap%
\pgfsetroundjoin%
\definecolor{currentfill}{rgb}{1.000000,0.894118,0.788235}%
\pgfsetfillcolor{currentfill}%
\pgfsetlinewidth{0.000000pt}%
\definecolor{currentstroke}{rgb}{1.000000,0.894118,0.788235}%
\pgfsetstrokecolor{currentstroke}%
\pgfsetdash{}{0pt}%
\pgfpathmoveto{\pgfqpoint{2.583171in}{3.110796in}}%
\pgfpathlineto{\pgfqpoint{2.472275in}{3.046769in}}%
\pgfpathlineto{\pgfqpoint{2.472275in}{3.174822in}}%
\pgfpathlineto{\pgfqpoint{2.583171in}{3.238848in}}%
\pgfpathlineto{\pgfqpoint{2.583171in}{3.110796in}}%
\pgfpathclose%
\pgfusepath{fill}%
\end{pgfscope}%
\begin{pgfscope}%
\pgfpathrectangle{\pgfqpoint{0.765000in}{0.660000in}}{\pgfqpoint{4.620000in}{4.620000in}}%
\pgfusepath{clip}%
\pgfsetbuttcap%
\pgfsetroundjoin%
\definecolor{currentfill}{rgb}{1.000000,0.894118,0.788235}%
\pgfsetfillcolor{currentfill}%
\pgfsetlinewidth{0.000000pt}%
\definecolor{currentstroke}{rgb}{1.000000,0.894118,0.788235}%
\pgfsetstrokecolor{currentstroke}%
\pgfsetdash{}{0pt}%
\pgfpathmoveto{\pgfqpoint{2.583171in}{3.110796in}}%
\pgfpathlineto{\pgfqpoint{2.694068in}{3.046769in}}%
\pgfpathlineto{\pgfqpoint{2.694068in}{3.174822in}}%
\pgfpathlineto{\pgfqpoint{2.583171in}{3.238848in}}%
\pgfpathlineto{\pgfqpoint{2.583171in}{3.110796in}}%
\pgfpathclose%
\pgfusepath{fill}%
\end{pgfscope}%
\begin{pgfscope}%
\pgfpathrectangle{\pgfqpoint{0.765000in}{0.660000in}}{\pgfqpoint{4.620000in}{4.620000in}}%
\pgfusepath{clip}%
\pgfsetbuttcap%
\pgfsetroundjoin%
\definecolor{currentfill}{rgb}{1.000000,0.894118,0.788235}%
\pgfsetfillcolor{currentfill}%
\pgfsetlinewidth{0.000000pt}%
\definecolor{currentstroke}{rgb}{1.000000,0.894118,0.788235}%
\pgfsetstrokecolor{currentstroke}%
\pgfsetdash{}{0pt}%
\pgfpathmoveto{\pgfqpoint{2.586744in}{3.201321in}}%
\pgfpathlineto{\pgfqpoint{2.475847in}{3.137295in}}%
\pgfpathlineto{\pgfqpoint{2.586744in}{3.073269in}}%
\pgfpathlineto{\pgfqpoint{2.697640in}{3.137295in}}%
\pgfpathlineto{\pgfqpoint{2.586744in}{3.201321in}}%
\pgfpathclose%
\pgfusepath{fill}%
\end{pgfscope}%
\begin{pgfscope}%
\pgfpathrectangle{\pgfqpoint{0.765000in}{0.660000in}}{\pgfqpoint{4.620000in}{4.620000in}}%
\pgfusepath{clip}%
\pgfsetbuttcap%
\pgfsetroundjoin%
\definecolor{currentfill}{rgb}{1.000000,0.894118,0.788235}%
\pgfsetfillcolor{currentfill}%
\pgfsetlinewidth{0.000000pt}%
\definecolor{currentstroke}{rgb}{1.000000,0.894118,0.788235}%
\pgfsetstrokecolor{currentstroke}%
\pgfsetdash{}{0pt}%
\pgfpathmoveto{\pgfqpoint{2.586744in}{2.945216in}}%
\pgfpathlineto{\pgfqpoint{2.697640in}{3.009242in}}%
\pgfpathlineto{\pgfqpoint{2.697640in}{3.137295in}}%
\pgfpathlineto{\pgfqpoint{2.586744in}{3.073269in}}%
\pgfpathlineto{\pgfqpoint{2.586744in}{2.945216in}}%
\pgfpathclose%
\pgfusepath{fill}%
\end{pgfscope}%
\begin{pgfscope}%
\pgfpathrectangle{\pgfqpoint{0.765000in}{0.660000in}}{\pgfqpoint{4.620000in}{4.620000in}}%
\pgfusepath{clip}%
\pgfsetbuttcap%
\pgfsetroundjoin%
\definecolor{currentfill}{rgb}{1.000000,0.894118,0.788235}%
\pgfsetfillcolor{currentfill}%
\pgfsetlinewidth{0.000000pt}%
\definecolor{currentstroke}{rgb}{1.000000,0.894118,0.788235}%
\pgfsetstrokecolor{currentstroke}%
\pgfsetdash{}{0pt}%
\pgfpathmoveto{\pgfqpoint{2.475847in}{3.009242in}}%
\pgfpathlineto{\pgfqpoint{2.586744in}{2.945216in}}%
\pgfpathlineto{\pgfqpoint{2.586744in}{3.073269in}}%
\pgfpathlineto{\pgfqpoint{2.475847in}{3.137295in}}%
\pgfpathlineto{\pgfqpoint{2.475847in}{3.009242in}}%
\pgfpathclose%
\pgfusepath{fill}%
\end{pgfscope}%
\begin{pgfscope}%
\pgfpathrectangle{\pgfqpoint{0.765000in}{0.660000in}}{\pgfqpoint{4.620000in}{4.620000in}}%
\pgfusepath{clip}%
\pgfsetbuttcap%
\pgfsetroundjoin%
\definecolor{currentfill}{rgb}{1.000000,0.894118,0.788235}%
\pgfsetfillcolor{currentfill}%
\pgfsetlinewidth{0.000000pt}%
\definecolor{currentstroke}{rgb}{1.000000,0.894118,0.788235}%
\pgfsetstrokecolor{currentstroke}%
\pgfsetdash{}{0pt}%
\pgfpathmoveto{\pgfqpoint{2.583171in}{3.238848in}}%
\pgfpathlineto{\pgfqpoint{2.472275in}{3.174822in}}%
\pgfpathlineto{\pgfqpoint{2.583171in}{3.110796in}}%
\pgfpathlineto{\pgfqpoint{2.694068in}{3.174822in}}%
\pgfpathlineto{\pgfqpoint{2.583171in}{3.238848in}}%
\pgfpathclose%
\pgfusepath{fill}%
\end{pgfscope}%
\begin{pgfscope}%
\pgfpathrectangle{\pgfqpoint{0.765000in}{0.660000in}}{\pgfqpoint{4.620000in}{4.620000in}}%
\pgfusepath{clip}%
\pgfsetbuttcap%
\pgfsetroundjoin%
\definecolor{currentfill}{rgb}{1.000000,0.894118,0.788235}%
\pgfsetfillcolor{currentfill}%
\pgfsetlinewidth{0.000000pt}%
\definecolor{currentstroke}{rgb}{1.000000,0.894118,0.788235}%
\pgfsetstrokecolor{currentstroke}%
\pgfsetdash{}{0pt}%
\pgfpathmoveto{\pgfqpoint{2.583171in}{2.982743in}}%
\pgfpathlineto{\pgfqpoint{2.694068in}{3.046769in}}%
\pgfpathlineto{\pgfqpoint{2.694068in}{3.174822in}}%
\pgfpathlineto{\pgfqpoint{2.583171in}{3.110796in}}%
\pgfpathlineto{\pgfqpoint{2.583171in}{2.982743in}}%
\pgfpathclose%
\pgfusepath{fill}%
\end{pgfscope}%
\begin{pgfscope}%
\pgfpathrectangle{\pgfqpoint{0.765000in}{0.660000in}}{\pgfqpoint{4.620000in}{4.620000in}}%
\pgfusepath{clip}%
\pgfsetbuttcap%
\pgfsetroundjoin%
\definecolor{currentfill}{rgb}{1.000000,0.894118,0.788235}%
\pgfsetfillcolor{currentfill}%
\pgfsetlinewidth{0.000000pt}%
\definecolor{currentstroke}{rgb}{1.000000,0.894118,0.788235}%
\pgfsetstrokecolor{currentstroke}%
\pgfsetdash{}{0pt}%
\pgfpathmoveto{\pgfqpoint{2.472275in}{3.046769in}}%
\pgfpathlineto{\pgfqpoint{2.583171in}{2.982743in}}%
\pgfpathlineto{\pgfqpoint{2.583171in}{3.110796in}}%
\pgfpathlineto{\pgfqpoint{2.472275in}{3.174822in}}%
\pgfpathlineto{\pgfqpoint{2.472275in}{3.046769in}}%
\pgfpathclose%
\pgfusepath{fill}%
\end{pgfscope}%
\begin{pgfscope}%
\pgfpathrectangle{\pgfqpoint{0.765000in}{0.660000in}}{\pgfqpoint{4.620000in}{4.620000in}}%
\pgfusepath{clip}%
\pgfsetbuttcap%
\pgfsetroundjoin%
\definecolor{currentfill}{rgb}{1.000000,0.894118,0.788235}%
\pgfsetfillcolor{currentfill}%
\pgfsetlinewidth{0.000000pt}%
\definecolor{currentstroke}{rgb}{1.000000,0.894118,0.788235}%
\pgfsetstrokecolor{currentstroke}%
\pgfsetdash{}{0pt}%
\pgfpathmoveto{\pgfqpoint{2.586744in}{3.073269in}}%
\pgfpathlineto{\pgfqpoint{2.475847in}{3.009242in}}%
\pgfpathlineto{\pgfqpoint{2.472275in}{3.046769in}}%
\pgfpathlineto{\pgfqpoint{2.583171in}{3.110796in}}%
\pgfpathlineto{\pgfqpoint{2.586744in}{3.073269in}}%
\pgfpathclose%
\pgfusepath{fill}%
\end{pgfscope}%
\begin{pgfscope}%
\pgfpathrectangle{\pgfqpoint{0.765000in}{0.660000in}}{\pgfqpoint{4.620000in}{4.620000in}}%
\pgfusepath{clip}%
\pgfsetbuttcap%
\pgfsetroundjoin%
\definecolor{currentfill}{rgb}{1.000000,0.894118,0.788235}%
\pgfsetfillcolor{currentfill}%
\pgfsetlinewidth{0.000000pt}%
\definecolor{currentstroke}{rgb}{1.000000,0.894118,0.788235}%
\pgfsetstrokecolor{currentstroke}%
\pgfsetdash{}{0pt}%
\pgfpathmoveto{\pgfqpoint{2.697640in}{3.009242in}}%
\pgfpathlineto{\pgfqpoint{2.586744in}{3.073269in}}%
\pgfpathlineto{\pgfqpoint{2.583171in}{3.110796in}}%
\pgfpathlineto{\pgfqpoint{2.694068in}{3.046769in}}%
\pgfpathlineto{\pgfqpoint{2.697640in}{3.009242in}}%
\pgfpathclose%
\pgfusepath{fill}%
\end{pgfscope}%
\begin{pgfscope}%
\pgfpathrectangle{\pgfqpoint{0.765000in}{0.660000in}}{\pgfqpoint{4.620000in}{4.620000in}}%
\pgfusepath{clip}%
\pgfsetbuttcap%
\pgfsetroundjoin%
\definecolor{currentfill}{rgb}{1.000000,0.894118,0.788235}%
\pgfsetfillcolor{currentfill}%
\pgfsetlinewidth{0.000000pt}%
\definecolor{currentstroke}{rgb}{1.000000,0.894118,0.788235}%
\pgfsetstrokecolor{currentstroke}%
\pgfsetdash{}{0pt}%
\pgfpathmoveto{\pgfqpoint{2.586744in}{3.073269in}}%
\pgfpathlineto{\pgfqpoint{2.586744in}{3.201321in}}%
\pgfpathlineto{\pgfqpoint{2.583171in}{3.238848in}}%
\pgfpathlineto{\pgfqpoint{2.694068in}{3.046769in}}%
\pgfpathlineto{\pgfqpoint{2.586744in}{3.073269in}}%
\pgfpathclose%
\pgfusepath{fill}%
\end{pgfscope}%
\begin{pgfscope}%
\pgfpathrectangle{\pgfqpoint{0.765000in}{0.660000in}}{\pgfqpoint{4.620000in}{4.620000in}}%
\pgfusepath{clip}%
\pgfsetbuttcap%
\pgfsetroundjoin%
\definecolor{currentfill}{rgb}{1.000000,0.894118,0.788235}%
\pgfsetfillcolor{currentfill}%
\pgfsetlinewidth{0.000000pt}%
\definecolor{currentstroke}{rgb}{1.000000,0.894118,0.788235}%
\pgfsetstrokecolor{currentstroke}%
\pgfsetdash{}{0pt}%
\pgfpathmoveto{\pgfqpoint{2.697640in}{3.009242in}}%
\pgfpathlineto{\pgfqpoint{2.697640in}{3.137295in}}%
\pgfpathlineto{\pgfqpoint{2.694068in}{3.174822in}}%
\pgfpathlineto{\pgfqpoint{2.583171in}{3.110796in}}%
\pgfpathlineto{\pgfqpoint{2.697640in}{3.009242in}}%
\pgfpathclose%
\pgfusepath{fill}%
\end{pgfscope}%
\begin{pgfscope}%
\pgfpathrectangle{\pgfqpoint{0.765000in}{0.660000in}}{\pgfqpoint{4.620000in}{4.620000in}}%
\pgfusepath{clip}%
\pgfsetbuttcap%
\pgfsetroundjoin%
\definecolor{currentfill}{rgb}{1.000000,0.894118,0.788235}%
\pgfsetfillcolor{currentfill}%
\pgfsetlinewidth{0.000000pt}%
\definecolor{currentstroke}{rgb}{1.000000,0.894118,0.788235}%
\pgfsetstrokecolor{currentstroke}%
\pgfsetdash{}{0pt}%
\pgfpathmoveto{\pgfqpoint{2.475847in}{3.009242in}}%
\pgfpathlineto{\pgfqpoint{2.586744in}{2.945216in}}%
\pgfpathlineto{\pgfqpoint{2.583171in}{2.982743in}}%
\pgfpathlineto{\pgfqpoint{2.472275in}{3.046769in}}%
\pgfpathlineto{\pgfqpoint{2.475847in}{3.009242in}}%
\pgfpathclose%
\pgfusepath{fill}%
\end{pgfscope}%
\begin{pgfscope}%
\pgfpathrectangle{\pgfqpoint{0.765000in}{0.660000in}}{\pgfqpoint{4.620000in}{4.620000in}}%
\pgfusepath{clip}%
\pgfsetbuttcap%
\pgfsetroundjoin%
\definecolor{currentfill}{rgb}{1.000000,0.894118,0.788235}%
\pgfsetfillcolor{currentfill}%
\pgfsetlinewidth{0.000000pt}%
\definecolor{currentstroke}{rgb}{1.000000,0.894118,0.788235}%
\pgfsetstrokecolor{currentstroke}%
\pgfsetdash{}{0pt}%
\pgfpathmoveto{\pgfqpoint{2.586744in}{2.945216in}}%
\pgfpathlineto{\pgfqpoint{2.697640in}{3.009242in}}%
\pgfpathlineto{\pgfqpoint{2.694068in}{3.046769in}}%
\pgfpathlineto{\pgfqpoint{2.583171in}{2.982743in}}%
\pgfpathlineto{\pgfqpoint{2.586744in}{2.945216in}}%
\pgfpathclose%
\pgfusepath{fill}%
\end{pgfscope}%
\begin{pgfscope}%
\pgfpathrectangle{\pgfqpoint{0.765000in}{0.660000in}}{\pgfqpoint{4.620000in}{4.620000in}}%
\pgfusepath{clip}%
\pgfsetbuttcap%
\pgfsetroundjoin%
\definecolor{currentfill}{rgb}{1.000000,0.894118,0.788235}%
\pgfsetfillcolor{currentfill}%
\pgfsetlinewidth{0.000000pt}%
\definecolor{currentstroke}{rgb}{1.000000,0.894118,0.788235}%
\pgfsetstrokecolor{currentstroke}%
\pgfsetdash{}{0pt}%
\pgfpathmoveto{\pgfqpoint{2.586744in}{3.201321in}}%
\pgfpathlineto{\pgfqpoint{2.475847in}{3.137295in}}%
\pgfpathlineto{\pgfqpoint{2.472275in}{3.174822in}}%
\pgfpathlineto{\pgfqpoint{2.583171in}{3.238848in}}%
\pgfpathlineto{\pgfqpoint{2.586744in}{3.201321in}}%
\pgfpathclose%
\pgfusepath{fill}%
\end{pgfscope}%
\begin{pgfscope}%
\pgfpathrectangle{\pgfqpoint{0.765000in}{0.660000in}}{\pgfqpoint{4.620000in}{4.620000in}}%
\pgfusepath{clip}%
\pgfsetbuttcap%
\pgfsetroundjoin%
\definecolor{currentfill}{rgb}{1.000000,0.894118,0.788235}%
\pgfsetfillcolor{currentfill}%
\pgfsetlinewidth{0.000000pt}%
\definecolor{currentstroke}{rgb}{1.000000,0.894118,0.788235}%
\pgfsetstrokecolor{currentstroke}%
\pgfsetdash{}{0pt}%
\pgfpathmoveto{\pgfqpoint{2.697640in}{3.137295in}}%
\pgfpathlineto{\pgfqpoint{2.586744in}{3.201321in}}%
\pgfpathlineto{\pgfqpoint{2.583171in}{3.238848in}}%
\pgfpathlineto{\pgfqpoint{2.694068in}{3.174822in}}%
\pgfpathlineto{\pgfqpoint{2.697640in}{3.137295in}}%
\pgfpathclose%
\pgfusepath{fill}%
\end{pgfscope}%
\begin{pgfscope}%
\pgfpathrectangle{\pgfqpoint{0.765000in}{0.660000in}}{\pgfqpoint{4.620000in}{4.620000in}}%
\pgfusepath{clip}%
\pgfsetbuttcap%
\pgfsetroundjoin%
\definecolor{currentfill}{rgb}{1.000000,0.894118,0.788235}%
\pgfsetfillcolor{currentfill}%
\pgfsetlinewidth{0.000000pt}%
\definecolor{currentstroke}{rgb}{1.000000,0.894118,0.788235}%
\pgfsetstrokecolor{currentstroke}%
\pgfsetdash{}{0pt}%
\pgfpathmoveto{\pgfqpoint{2.475847in}{3.009242in}}%
\pgfpathlineto{\pgfqpoint{2.475847in}{3.137295in}}%
\pgfpathlineto{\pgfqpoint{2.472275in}{3.174822in}}%
\pgfpathlineto{\pgfqpoint{2.583171in}{2.982743in}}%
\pgfpathlineto{\pgfqpoint{2.475847in}{3.009242in}}%
\pgfpathclose%
\pgfusepath{fill}%
\end{pgfscope}%
\begin{pgfscope}%
\pgfpathrectangle{\pgfqpoint{0.765000in}{0.660000in}}{\pgfqpoint{4.620000in}{4.620000in}}%
\pgfusepath{clip}%
\pgfsetbuttcap%
\pgfsetroundjoin%
\definecolor{currentfill}{rgb}{1.000000,0.894118,0.788235}%
\pgfsetfillcolor{currentfill}%
\pgfsetlinewidth{0.000000pt}%
\definecolor{currentstroke}{rgb}{1.000000,0.894118,0.788235}%
\pgfsetstrokecolor{currentstroke}%
\pgfsetdash{}{0pt}%
\pgfpathmoveto{\pgfqpoint{2.586744in}{2.945216in}}%
\pgfpathlineto{\pgfqpoint{2.586744in}{3.073269in}}%
\pgfpathlineto{\pgfqpoint{2.583171in}{3.110796in}}%
\pgfpathlineto{\pgfqpoint{2.472275in}{3.046769in}}%
\pgfpathlineto{\pgfqpoint{2.586744in}{2.945216in}}%
\pgfpathclose%
\pgfusepath{fill}%
\end{pgfscope}%
\begin{pgfscope}%
\pgfpathrectangle{\pgfqpoint{0.765000in}{0.660000in}}{\pgfqpoint{4.620000in}{4.620000in}}%
\pgfusepath{clip}%
\pgfsetbuttcap%
\pgfsetroundjoin%
\definecolor{currentfill}{rgb}{1.000000,0.894118,0.788235}%
\pgfsetfillcolor{currentfill}%
\pgfsetlinewidth{0.000000pt}%
\definecolor{currentstroke}{rgb}{1.000000,0.894118,0.788235}%
\pgfsetstrokecolor{currentstroke}%
\pgfsetdash{}{0pt}%
\pgfpathmoveto{\pgfqpoint{2.586744in}{3.073269in}}%
\pgfpathlineto{\pgfqpoint{2.697640in}{3.137295in}}%
\pgfpathlineto{\pgfqpoint{2.694068in}{3.174822in}}%
\pgfpathlineto{\pgfqpoint{2.583171in}{3.110796in}}%
\pgfpathlineto{\pgfqpoint{2.586744in}{3.073269in}}%
\pgfpathclose%
\pgfusepath{fill}%
\end{pgfscope}%
\begin{pgfscope}%
\pgfpathrectangle{\pgfqpoint{0.765000in}{0.660000in}}{\pgfqpoint{4.620000in}{4.620000in}}%
\pgfusepath{clip}%
\pgfsetbuttcap%
\pgfsetroundjoin%
\definecolor{currentfill}{rgb}{1.000000,0.894118,0.788235}%
\pgfsetfillcolor{currentfill}%
\pgfsetlinewidth{0.000000pt}%
\definecolor{currentstroke}{rgb}{1.000000,0.894118,0.788235}%
\pgfsetstrokecolor{currentstroke}%
\pgfsetdash{}{0pt}%
\pgfpathmoveto{\pgfqpoint{2.475847in}{3.137295in}}%
\pgfpathlineto{\pgfqpoint{2.586744in}{3.073269in}}%
\pgfpathlineto{\pgfqpoint{2.583171in}{3.110796in}}%
\pgfpathlineto{\pgfqpoint{2.472275in}{3.174822in}}%
\pgfpathlineto{\pgfqpoint{2.475847in}{3.137295in}}%
\pgfpathclose%
\pgfusepath{fill}%
\end{pgfscope}%
\begin{pgfscope}%
\pgfpathrectangle{\pgfqpoint{0.765000in}{0.660000in}}{\pgfqpoint{4.620000in}{4.620000in}}%
\pgfusepath{clip}%
\pgfsetbuttcap%
\pgfsetroundjoin%
\definecolor{currentfill}{rgb}{1.000000,0.894118,0.788235}%
\pgfsetfillcolor{currentfill}%
\pgfsetlinewidth{0.000000pt}%
\definecolor{currentstroke}{rgb}{1.000000,0.894118,0.788235}%
\pgfsetstrokecolor{currentstroke}%
\pgfsetdash{}{0pt}%
\pgfpathmoveto{\pgfqpoint{2.635438in}{3.285868in}}%
\pgfpathlineto{\pgfqpoint{2.524541in}{3.221842in}}%
\pgfpathlineto{\pgfqpoint{2.635438in}{3.157816in}}%
\pgfpathlineto{\pgfqpoint{2.746335in}{3.221842in}}%
\pgfpathlineto{\pgfqpoint{2.635438in}{3.285868in}}%
\pgfpathclose%
\pgfusepath{fill}%
\end{pgfscope}%
\begin{pgfscope}%
\pgfpathrectangle{\pgfqpoint{0.765000in}{0.660000in}}{\pgfqpoint{4.620000in}{4.620000in}}%
\pgfusepath{clip}%
\pgfsetbuttcap%
\pgfsetroundjoin%
\definecolor{currentfill}{rgb}{1.000000,0.894118,0.788235}%
\pgfsetfillcolor{currentfill}%
\pgfsetlinewidth{0.000000pt}%
\definecolor{currentstroke}{rgb}{1.000000,0.894118,0.788235}%
\pgfsetstrokecolor{currentstroke}%
\pgfsetdash{}{0pt}%
\pgfpathmoveto{\pgfqpoint{2.635438in}{3.285868in}}%
\pgfpathlineto{\pgfqpoint{2.524541in}{3.221842in}}%
\pgfpathlineto{\pgfqpoint{2.524541in}{3.349894in}}%
\pgfpathlineto{\pgfqpoint{2.635438in}{3.413921in}}%
\pgfpathlineto{\pgfqpoint{2.635438in}{3.285868in}}%
\pgfpathclose%
\pgfusepath{fill}%
\end{pgfscope}%
\begin{pgfscope}%
\pgfpathrectangle{\pgfqpoint{0.765000in}{0.660000in}}{\pgfqpoint{4.620000in}{4.620000in}}%
\pgfusepath{clip}%
\pgfsetbuttcap%
\pgfsetroundjoin%
\definecolor{currentfill}{rgb}{1.000000,0.894118,0.788235}%
\pgfsetfillcolor{currentfill}%
\pgfsetlinewidth{0.000000pt}%
\definecolor{currentstroke}{rgb}{1.000000,0.894118,0.788235}%
\pgfsetstrokecolor{currentstroke}%
\pgfsetdash{}{0pt}%
\pgfpathmoveto{\pgfqpoint{2.635438in}{3.285868in}}%
\pgfpathlineto{\pgfqpoint{2.746335in}{3.221842in}}%
\pgfpathlineto{\pgfqpoint{2.746335in}{3.349894in}}%
\pgfpathlineto{\pgfqpoint{2.635438in}{3.413921in}}%
\pgfpathlineto{\pgfqpoint{2.635438in}{3.285868in}}%
\pgfpathclose%
\pgfusepath{fill}%
\end{pgfscope}%
\begin{pgfscope}%
\pgfpathrectangle{\pgfqpoint{0.765000in}{0.660000in}}{\pgfqpoint{4.620000in}{4.620000in}}%
\pgfusepath{clip}%
\pgfsetbuttcap%
\pgfsetroundjoin%
\definecolor{currentfill}{rgb}{1.000000,0.894118,0.788235}%
\pgfsetfillcolor{currentfill}%
\pgfsetlinewidth{0.000000pt}%
\definecolor{currentstroke}{rgb}{1.000000,0.894118,0.788235}%
\pgfsetstrokecolor{currentstroke}%
\pgfsetdash{}{0pt}%
\pgfpathmoveto{\pgfqpoint{2.613910in}{3.253240in}}%
\pgfpathlineto{\pgfqpoint{2.503014in}{3.189213in}}%
\pgfpathlineto{\pgfqpoint{2.613910in}{3.125187in}}%
\pgfpathlineto{\pgfqpoint{2.724807in}{3.189213in}}%
\pgfpathlineto{\pgfqpoint{2.613910in}{3.253240in}}%
\pgfpathclose%
\pgfusepath{fill}%
\end{pgfscope}%
\begin{pgfscope}%
\pgfpathrectangle{\pgfqpoint{0.765000in}{0.660000in}}{\pgfqpoint{4.620000in}{4.620000in}}%
\pgfusepath{clip}%
\pgfsetbuttcap%
\pgfsetroundjoin%
\definecolor{currentfill}{rgb}{1.000000,0.894118,0.788235}%
\pgfsetfillcolor{currentfill}%
\pgfsetlinewidth{0.000000pt}%
\definecolor{currentstroke}{rgb}{1.000000,0.894118,0.788235}%
\pgfsetstrokecolor{currentstroke}%
\pgfsetdash{}{0pt}%
\pgfpathmoveto{\pgfqpoint{2.613910in}{3.253240in}}%
\pgfpathlineto{\pgfqpoint{2.503014in}{3.189213in}}%
\pgfpathlineto{\pgfqpoint{2.503014in}{3.317266in}}%
\pgfpathlineto{\pgfqpoint{2.613910in}{3.381292in}}%
\pgfpathlineto{\pgfqpoint{2.613910in}{3.253240in}}%
\pgfpathclose%
\pgfusepath{fill}%
\end{pgfscope}%
\begin{pgfscope}%
\pgfpathrectangle{\pgfqpoint{0.765000in}{0.660000in}}{\pgfqpoint{4.620000in}{4.620000in}}%
\pgfusepath{clip}%
\pgfsetbuttcap%
\pgfsetroundjoin%
\definecolor{currentfill}{rgb}{1.000000,0.894118,0.788235}%
\pgfsetfillcolor{currentfill}%
\pgfsetlinewidth{0.000000pt}%
\definecolor{currentstroke}{rgb}{1.000000,0.894118,0.788235}%
\pgfsetstrokecolor{currentstroke}%
\pgfsetdash{}{0pt}%
\pgfpathmoveto{\pgfqpoint{2.613910in}{3.253240in}}%
\pgfpathlineto{\pgfqpoint{2.724807in}{3.189213in}}%
\pgfpathlineto{\pgfqpoint{2.724807in}{3.317266in}}%
\pgfpathlineto{\pgfqpoint{2.613910in}{3.381292in}}%
\pgfpathlineto{\pgfqpoint{2.613910in}{3.253240in}}%
\pgfpathclose%
\pgfusepath{fill}%
\end{pgfscope}%
\begin{pgfscope}%
\pgfpathrectangle{\pgfqpoint{0.765000in}{0.660000in}}{\pgfqpoint{4.620000in}{4.620000in}}%
\pgfusepath{clip}%
\pgfsetbuttcap%
\pgfsetroundjoin%
\definecolor{currentfill}{rgb}{1.000000,0.894118,0.788235}%
\pgfsetfillcolor{currentfill}%
\pgfsetlinewidth{0.000000pt}%
\definecolor{currentstroke}{rgb}{1.000000,0.894118,0.788235}%
\pgfsetstrokecolor{currentstroke}%
\pgfsetdash{}{0pt}%
\pgfpathmoveto{\pgfqpoint{2.635438in}{3.413921in}}%
\pgfpathlineto{\pgfqpoint{2.524541in}{3.349894in}}%
\pgfpathlineto{\pgfqpoint{2.635438in}{3.285868in}}%
\pgfpathlineto{\pgfqpoint{2.746335in}{3.349894in}}%
\pgfpathlineto{\pgfqpoint{2.635438in}{3.413921in}}%
\pgfpathclose%
\pgfusepath{fill}%
\end{pgfscope}%
\begin{pgfscope}%
\pgfpathrectangle{\pgfqpoint{0.765000in}{0.660000in}}{\pgfqpoint{4.620000in}{4.620000in}}%
\pgfusepath{clip}%
\pgfsetbuttcap%
\pgfsetroundjoin%
\definecolor{currentfill}{rgb}{1.000000,0.894118,0.788235}%
\pgfsetfillcolor{currentfill}%
\pgfsetlinewidth{0.000000pt}%
\definecolor{currentstroke}{rgb}{1.000000,0.894118,0.788235}%
\pgfsetstrokecolor{currentstroke}%
\pgfsetdash{}{0pt}%
\pgfpathmoveto{\pgfqpoint{2.635438in}{3.157816in}}%
\pgfpathlineto{\pgfqpoint{2.746335in}{3.221842in}}%
\pgfpathlineto{\pgfqpoint{2.746335in}{3.349894in}}%
\pgfpathlineto{\pgfqpoint{2.635438in}{3.285868in}}%
\pgfpathlineto{\pgfqpoint{2.635438in}{3.157816in}}%
\pgfpathclose%
\pgfusepath{fill}%
\end{pgfscope}%
\begin{pgfscope}%
\pgfpathrectangle{\pgfqpoint{0.765000in}{0.660000in}}{\pgfqpoint{4.620000in}{4.620000in}}%
\pgfusepath{clip}%
\pgfsetbuttcap%
\pgfsetroundjoin%
\definecolor{currentfill}{rgb}{1.000000,0.894118,0.788235}%
\pgfsetfillcolor{currentfill}%
\pgfsetlinewidth{0.000000pt}%
\definecolor{currentstroke}{rgb}{1.000000,0.894118,0.788235}%
\pgfsetstrokecolor{currentstroke}%
\pgfsetdash{}{0pt}%
\pgfpathmoveto{\pgfqpoint{2.524541in}{3.221842in}}%
\pgfpathlineto{\pgfqpoint{2.635438in}{3.157816in}}%
\pgfpathlineto{\pgfqpoint{2.635438in}{3.285868in}}%
\pgfpathlineto{\pgfqpoint{2.524541in}{3.349894in}}%
\pgfpathlineto{\pgfqpoint{2.524541in}{3.221842in}}%
\pgfpathclose%
\pgfusepath{fill}%
\end{pgfscope}%
\begin{pgfscope}%
\pgfpathrectangle{\pgfqpoint{0.765000in}{0.660000in}}{\pgfqpoint{4.620000in}{4.620000in}}%
\pgfusepath{clip}%
\pgfsetbuttcap%
\pgfsetroundjoin%
\definecolor{currentfill}{rgb}{1.000000,0.894118,0.788235}%
\pgfsetfillcolor{currentfill}%
\pgfsetlinewidth{0.000000pt}%
\definecolor{currentstroke}{rgb}{1.000000,0.894118,0.788235}%
\pgfsetstrokecolor{currentstroke}%
\pgfsetdash{}{0pt}%
\pgfpathmoveto{\pgfqpoint{2.613910in}{3.381292in}}%
\pgfpathlineto{\pgfqpoint{2.503014in}{3.317266in}}%
\pgfpathlineto{\pgfqpoint{2.613910in}{3.253240in}}%
\pgfpathlineto{\pgfqpoint{2.724807in}{3.317266in}}%
\pgfpathlineto{\pgfqpoint{2.613910in}{3.381292in}}%
\pgfpathclose%
\pgfusepath{fill}%
\end{pgfscope}%
\begin{pgfscope}%
\pgfpathrectangle{\pgfqpoint{0.765000in}{0.660000in}}{\pgfqpoint{4.620000in}{4.620000in}}%
\pgfusepath{clip}%
\pgfsetbuttcap%
\pgfsetroundjoin%
\definecolor{currentfill}{rgb}{1.000000,0.894118,0.788235}%
\pgfsetfillcolor{currentfill}%
\pgfsetlinewidth{0.000000pt}%
\definecolor{currentstroke}{rgb}{1.000000,0.894118,0.788235}%
\pgfsetstrokecolor{currentstroke}%
\pgfsetdash{}{0pt}%
\pgfpathmoveto{\pgfqpoint{2.613910in}{3.125187in}}%
\pgfpathlineto{\pgfqpoint{2.724807in}{3.189213in}}%
\pgfpathlineto{\pgfqpoint{2.724807in}{3.317266in}}%
\pgfpathlineto{\pgfqpoint{2.613910in}{3.253240in}}%
\pgfpathlineto{\pgfqpoint{2.613910in}{3.125187in}}%
\pgfpathclose%
\pgfusepath{fill}%
\end{pgfscope}%
\begin{pgfscope}%
\pgfpathrectangle{\pgfqpoint{0.765000in}{0.660000in}}{\pgfqpoint{4.620000in}{4.620000in}}%
\pgfusepath{clip}%
\pgfsetbuttcap%
\pgfsetroundjoin%
\definecolor{currentfill}{rgb}{1.000000,0.894118,0.788235}%
\pgfsetfillcolor{currentfill}%
\pgfsetlinewidth{0.000000pt}%
\definecolor{currentstroke}{rgb}{1.000000,0.894118,0.788235}%
\pgfsetstrokecolor{currentstroke}%
\pgfsetdash{}{0pt}%
\pgfpathmoveto{\pgfqpoint{2.503014in}{3.189213in}}%
\pgfpathlineto{\pgfqpoint{2.613910in}{3.125187in}}%
\pgfpathlineto{\pgfqpoint{2.613910in}{3.253240in}}%
\pgfpathlineto{\pgfqpoint{2.503014in}{3.317266in}}%
\pgfpathlineto{\pgfqpoint{2.503014in}{3.189213in}}%
\pgfpathclose%
\pgfusepath{fill}%
\end{pgfscope}%
\begin{pgfscope}%
\pgfpathrectangle{\pgfqpoint{0.765000in}{0.660000in}}{\pgfqpoint{4.620000in}{4.620000in}}%
\pgfusepath{clip}%
\pgfsetbuttcap%
\pgfsetroundjoin%
\definecolor{currentfill}{rgb}{1.000000,0.894118,0.788235}%
\pgfsetfillcolor{currentfill}%
\pgfsetlinewidth{0.000000pt}%
\definecolor{currentstroke}{rgb}{1.000000,0.894118,0.788235}%
\pgfsetstrokecolor{currentstroke}%
\pgfsetdash{}{0pt}%
\pgfpathmoveto{\pgfqpoint{2.635438in}{3.285868in}}%
\pgfpathlineto{\pgfqpoint{2.524541in}{3.221842in}}%
\pgfpathlineto{\pgfqpoint{2.503014in}{3.189213in}}%
\pgfpathlineto{\pgfqpoint{2.613910in}{3.253240in}}%
\pgfpathlineto{\pgfqpoint{2.635438in}{3.285868in}}%
\pgfpathclose%
\pgfusepath{fill}%
\end{pgfscope}%
\begin{pgfscope}%
\pgfpathrectangle{\pgfqpoint{0.765000in}{0.660000in}}{\pgfqpoint{4.620000in}{4.620000in}}%
\pgfusepath{clip}%
\pgfsetbuttcap%
\pgfsetroundjoin%
\definecolor{currentfill}{rgb}{1.000000,0.894118,0.788235}%
\pgfsetfillcolor{currentfill}%
\pgfsetlinewidth{0.000000pt}%
\definecolor{currentstroke}{rgb}{1.000000,0.894118,0.788235}%
\pgfsetstrokecolor{currentstroke}%
\pgfsetdash{}{0pt}%
\pgfpathmoveto{\pgfqpoint{2.746335in}{3.221842in}}%
\pgfpathlineto{\pgfqpoint{2.635438in}{3.285868in}}%
\pgfpathlineto{\pgfqpoint{2.613910in}{3.253240in}}%
\pgfpathlineto{\pgfqpoint{2.724807in}{3.189213in}}%
\pgfpathlineto{\pgfqpoint{2.746335in}{3.221842in}}%
\pgfpathclose%
\pgfusepath{fill}%
\end{pgfscope}%
\begin{pgfscope}%
\pgfpathrectangle{\pgfqpoint{0.765000in}{0.660000in}}{\pgfqpoint{4.620000in}{4.620000in}}%
\pgfusepath{clip}%
\pgfsetbuttcap%
\pgfsetroundjoin%
\definecolor{currentfill}{rgb}{1.000000,0.894118,0.788235}%
\pgfsetfillcolor{currentfill}%
\pgfsetlinewidth{0.000000pt}%
\definecolor{currentstroke}{rgb}{1.000000,0.894118,0.788235}%
\pgfsetstrokecolor{currentstroke}%
\pgfsetdash{}{0pt}%
\pgfpathmoveto{\pgfqpoint{2.635438in}{3.285868in}}%
\pgfpathlineto{\pgfqpoint{2.635438in}{3.413921in}}%
\pgfpathlineto{\pgfqpoint{2.613910in}{3.381292in}}%
\pgfpathlineto{\pgfqpoint{2.724807in}{3.189213in}}%
\pgfpathlineto{\pgfqpoint{2.635438in}{3.285868in}}%
\pgfpathclose%
\pgfusepath{fill}%
\end{pgfscope}%
\begin{pgfscope}%
\pgfpathrectangle{\pgfqpoint{0.765000in}{0.660000in}}{\pgfqpoint{4.620000in}{4.620000in}}%
\pgfusepath{clip}%
\pgfsetbuttcap%
\pgfsetroundjoin%
\definecolor{currentfill}{rgb}{1.000000,0.894118,0.788235}%
\pgfsetfillcolor{currentfill}%
\pgfsetlinewidth{0.000000pt}%
\definecolor{currentstroke}{rgb}{1.000000,0.894118,0.788235}%
\pgfsetstrokecolor{currentstroke}%
\pgfsetdash{}{0pt}%
\pgfpathmoveto{\pgfqpoint{2.746335in}{3.221842in}}%
\pgfpathlineto{\pgfqpoint{2.746335in}{3.349894in}}%
\pgfpathlineto{\pgfqpoint{2.724807in}{3.317266in}}%
\pgfpathlineto{\pgfqpoint{2.613910in}{3.253240in}}%
\pgfpathlineto{\pgfqpoint{2.746335in}{3.221842in}}%
\pgfpathclose%
\pgfusepath{fill}%
\end{pgfscope}%
\begin{pgfscope}%
\pgfpathrectangle{\pgfqpoint{0.765000in}{0.660000in}}{\pgfqpoint{4.620000in}{4.620000in}}%
\pgfusepath{clip}%
\pgfsetbuttcap%
\pgfsetroundjoin%
\definecolor{currentfill}{rgb}{1.000000,0.894118,0.788235}%
\pgfsetfillcolor{currentfill}%
\pgfsetlinewidth{0.000000pt}%
\definecolor{currentstroke}{rgb}{1.000000,0.894118,0.788235}%
\pgfsetstrokecolor{currentstroke}%
\pgfsetdash{}{0pt}%
\pgfpathmoveto{\pgfqpoint{2.524541in}{3.221842in}}%
\pgfpathlineto{\pgfqpoint{2.635438in}{3.157816in}}%
\pgfpathlineto{\pgfqpoint{2.613910in}{3.125187in}}%
\pgfpathlineto{\pgfqpoint{2.503014in}{3.189213in}}%
\pgfpathlineto{\pgfqpoint{2.524541in}{3.221842in}}%
\pgfpathclose%
\pgfusepath{fill}%
\end{pgfscope}%
\begin{pgfscope}%
\pgfpathrectangle{\pgfqpoint{0.765000in}{0.660000in}}{\pgfqpoint{4.620000in}{4.620000in}}%
\pgfusepath{clip}%
\pgfsetbuttcap%
\pgfsetroundjoin%
\definecolor{currentfill}{rgb}{1.000000,0.894118,0.788235}%
\pgfsetfillcolor{currentfill}%
\pgfsetlinewidth{0.000000pt}%
\definecolor{currentstroke}{rgb}{1.000000,0.894118,0.788235}%
\pgfsetstrokecolor{currentstroke}%
\pgfsetdash{}{0pt}%
\pgfpathmoveto{\pgfqpoint{2.635438in}{3.157816in}}%
\pgfpathlineto{\pgfqpoint{2.746335in}{3.221842in}}%
\pgfpathlineto{\pgfqpoint{2.724807in}{3.189213in}}%
\pgfpathlineto{\pgfqpoint{2.613910in}{3.125187in}}%
\pgfpathlineto{\pgfqpoint{2.635438in}{3.157816in}}%
\pgfpathclose%
\pgfusepath{fill}%
\end{pgfscope}%
\begin{pgfscope}%
\pgfpathrectangle{\pgfqpoint{0.765000in}{0.660000in}}{\pgfqpoint{4.620000in}{4.620000in}}%
\pgfusepath{clip}%
\pgfsetbuttcap%
\pgfsetroundjoin%
\definecolor{currentfill}{rgb}{1.000000,0.894118,0.788235}%
\pgfsetfillcolor{currentfill}%
\pgfsetlinewidth{0.000000pt}%
\definecolor{currentstroke}{rgb}{1.000000,0.894118,0.788235}%
\pgfsetstrokecolor{currentstroke}%
\pgfsetdash{}{0pt}%
\pgfpathmoveto{\pgfqpoint{2.635438in}{3.413921in}}%
\pgfpathlineto{\pgfqpoint{2.524541in}{3.349894in}}%
\pgfpathlineto{\pgfqpoint{2.503014in}{3.317266in}}%
\pgfpathlineto{\pgfqpoint{2.613910in}{3.381292in}}%
\pgfpathlineto{\pgfqpoint{2.635438in}{3.413921in}}%
\pgfpathclose%
\pgfusepath{fill}%
\end{pgfscope}%
\begin{pgfscope}%
\pgfpathrectangle{\pgfqpoint{0.765000in}{0.660000in}}{\pgfqpoint{4.620000in}{4.620000in}}%
\pgfusepath{clip}%
\pgfsetbuttcap%
\pgfsetroundjoin%
\definecolor{currentfill}{rgb}{1.000000,0.894118,0.788235}%
\pgfsetfillcolor{currentfill}%
\pgfsetlinewidth{0.000000pt}%
\definecolor{currentstroke}{rgb}{1.000000,0.894118,0.788235}%
\pgfsetstrokecolor{currentstroke}%
\pgfsetdash{}{0pt}%
\pgfpathmoveto{\pgfqpoint{2.746335in}{3.349894in}}%
\pgfpathlineto{\pgfqpoint{2.635438in}{3.413921in}}%
\pgfpathlineto{\pgfqpoint{2.613910in}{3.381292in}}%
\pgfpathlineto{\pgfqpoint{2.724807in}{3.317266in}}%
\pgfpathlineto{\pgfqpoint{2.746335in}{3.349894in}}%
\pgfpathclose%
\pgfusepath{fill}%
\end{pgfscope}%
\begin{pgfscope}%
\pgfpathrectangle{\pgfqpoint{0.765000in}{0.660000in}}{\pgfqpoint{4.620000in}{4.620000in}}%
\pgfusepath{clip}%
\pgfsetbuttcap%
\pgfsetroundjoin%
\definecolor{currentfill}{rgb}{1.000000,0.894118,0.788235}%
\pgfsetfillcolor{currentfill}%
\pgfsetlinewidth{0.000000pt}%
\definecolor{currentstroke}{rgb}{1.000000,0.894118,0.788235}%
\pgfsetstrokecolor{currentstroke}%
\pgfsetdash{}{0pt}%
\pgfpathmoveto{\pgfqpoint{2.524541in}{3.221842in}}%
\pgfpathlineto{\pgfqpoint{2.524541in}{3.349894in}}%
\pgfpathlineto{\pgfqpoint{2.503014in}{3.317266in}}%
\pgfpathlineto{\pgfqpoint{2.613910in}{3.125187in}}%
\pgfpathlineto{\pgfqpoint{2.524541in}{3.221842in}}%
\pgfpathclose%
\pgfusepath{fill}%
\end{pgfscope}%
\begin{pgfscope}%
\pgfpathrectangle{\pgfqpoint{0.765000in}{0.660000in}}{\pgfqpoint{4.620000in}{4.620000in}}%
\pgfusepath{clip}%
\pgfsetbuttcap%
\pgfsetroundjoin%
\definecolor{currentfill}{rgb}{1.000000,0.894118,0.788235}%
\pgfsetfillcolor{currentfill}%
\pgfsetlinewidth{0.000000pt}%
\definecolor{currentstroke}{rgb}{1.000000,0.894118,0.788235}%
\pgfsetstrokecolor{currentstroke}%
\pgfsetdash{}{0pt}%
\pgfpathmoveto{\pgfqpoint{2.635438in}{3.157816in}}%
\pgfpathlineto{\pgfqpoint{2.635438in}{3.285868in}}%
\pgfpathlineto{\pgfqpoint{2.613910in}{3.253240in}}%
\pgfpathlineto{\pgfqpoint{2.503014in}{3.189213in}}%
\pgfpathlineto{\pgfqpoint{2.635438in}{3.157816in}}%
\pgfpathclose%
\pgfusepath{fill}%
\end{pgfscope}%
\begin{pgfscope}%
\pgfpathrectangle{\pgfqpoint{0.765000in}{0.660000in}}{\pgfqpoint{4.620000in}{4.620000in}}%
\pgfusepath{clip}%
\pgfsetbuttcap%
\pgfsetroundjoin%
\definecolor{currentfill}{rgb}{1.000000,0.894118,0.788235}%
\pgfsetfillcolor{currentfill}%
\pgfsetlinewidth{0.000000pt}%
\definecolor{currentstroke}{rgb}{1.000000,0.894118,0.788235}%
\pgfsetstrokecolor{currentstroke}%
\pgfsetdash{}{0pt}%
\pgfpathmoveto{\pgfqpoint{2.635438in}{3.285868in}}%
\pgfpathlineto{\pgfqpoint{2.746335in}{3.349894in}}%
\pgfpathlineto{\pgfqpoint{2.724807in}{3.317266in}}%
\pgfpathlineto{\pgfqpoint{2.613910in}{3.253240in}}%
\pgfpathlineto{\pgfqpoint{2.635438in}{3.285868in}}%
\pgfpathclose%
\pgfusepath{fill}%
\end{pgfscope}%
\begin{pgfscope}%
\pgfpathrectangle{\pgfqpoint{0.765000in}{0.660000in}}{\pgfqpoint{4.620000in}{4.620000in}}%
\pgfusepath{clip}%
\pgfsetbuttcap%
\pgfsetroundjoin%
\definecolor{currentfill}{rgb}{1.000000,0.894118,0.788235}%
\pgfsetfillcolor{currentfill}%
\pgfsetlinewidth{0.000000pt}%
\definecolor{currentstroke}{rgb}{1.000000,0.894118,0.788235}%
\pgfsetstrokecolor{currentstroke}%
\pgfsetdash{}{0pt}%
\pgfpathmoveto{\pgfqpoint{2.524541in}{3.349894in}}%
\pgfpathlineto{\pgfqpoint{2.635438in}{3.285868in}}%
\pgfpathlineto{\pgfqpoint{2.613910in}{3.253240in}}%
\pgfpathlineto{\pgfqpoint{2.503014in}{3.317266in}}%
\pgfpathlineto{\pgfqpoint{2.524541in}{3.349894in}}%
\pgfpathclose%
\pgfusepath{fill}%
\end{pgfscope}%
\begin{pgfscope}%
\pgfpathrectangle{\pgfqpoint{0.765000in}{0.660000in}}{\pgfqpoint{4.620000in}{4.620000in}}%
\pgfusepath{clip}%
\pgfsetbuttcap%
\pgfsetroundjoin%
\definecolor{currentfill}{rgb}{1.000000,0.894118,0.788235}%
\pgfsetfillcolor{currentfill}%
\pgfsetlinewidth{0.000000pt}%
\definecolor{currentstroke}{rgb}{1.000000,0.894118,0.788235}%
\pgfsetstrokecolor{currentstroke}%
\pgfsetdash{}{0pt}%
\pgfpathmoveto{\pgfqpoint{2.635438in}{3.285868in}}%
\pgfpathlineto{\pgfqpoint{2.524541in}{3.221842in}}%
\pgfpathlineto{\pgfqpoint{2.635438in}{3.157816in}}%
\pgfpathlineto{\pgfqpoint{2.746335in}{3.221842in}}%
\pgfpathlineto{\pgfqpoint{2.635438in}{3.285868in}}%
\pgfpathclose%
\pgfusepath{fill}%
\end{pgfscope}%
\begin{pgfscope}%
\pgfpathrectangle{\pgfqpoint{0.765000in}{0.660000in}}{\pgfqpoint{4.620000in}{4.620000in}}%
\pgfusepath{clip}%
\pgfsetbuttcap%
\pgfsetroundjoin%
\definecolor{currentfill}{rgb}{1.000000,0.894118,0.788235}%
\pgfsetfillcolor{currentfill}%
\pgfsetlinewidth{0.000000pt}%
\definecolor{currentstroke}{rgb}{1.000000,0.894118,0.788235}%
\pgfsetstrokecolor{currentstroke}%
\pgfsetdash{}{0pt}%
\pgfpathmoveto{\pgfqpoint{2.635438in}{3.285868in}}%
\pgfpathlineto{\pgfqpoint{2.524541in}{3.221842in}}%
\pgfpathlineto{\pgfqpoint{2.524541in}{3.349894in}}%
\pgfpathlineto{\pgfqpoint{2.635438in}{3.413921in}}%
\pgfpathlineto{\pgfqpoint{2.635438in}{3.285868in}}%
\pgfpathclose%
\pgfusepath{fill}%
\end{pgfscope}%
\begin{pgfscope}%
\pgfpathrectangle{\pgfqpoint{0.765000in}{0.660000in}}{\pgfqpoint{4.620000in}{4.620000in}}%
\pgfusepath{clip}%
\pgfsetbuttcap%
\pgfsetroundjoin%
\definecolor{currentfill}{rgb}{1.000000,0.894118,0.788235}%
\pgfsetfillcolor{currentfill}%
\pgfsetlinewidth{0.000000pt}%
\definecolor{currentstroke}{rgb}{1.000000,0.894118,0.788235}%
\pgfsetstrokecolor{currentstroke}%
\pgfsetdash{}{0pt}%
\pgfpathmoveto{\pgfqpoint{2.635438in}{3.285868in}}%
\pgfpathlineto{\pgfqpoint{2.746335in}{3.221842in}}%
\pgfpathlineto{\pgfqpoint{2.746335in}{3.349894in}}%
\pgfpathlineto{\pgfqpoint{2.635438in}{3.413921in}}%
\pgfpathlineto{\pgfqpoint{2.635438in}{3.285868in}}%
\pgfpathclose%
\pgfusepath{fill}%
\end{pgfscope}%
\begin{pgfscope}%
\pgfpathrectangle{\pgfqpoint{0.765000in}{0.660000in}}{\pgfqpoint{4.620000in}{4.620000in}}%
\pgfusepath{clip}%
\pgfsetbuttcap%
\pgfsetroundjoin%
\definecolor{currentfill}{rgb}{1.000000,0.894118,0.788235}%
\pgfsetfillcolor{currentfill}%
\pgfsetlinewidth{0.000000pt}%
\definecolor{currentstroke}{rgb}{1.000000,0.894118,0.788235}%
\pgfsetstrokecolor{currentstroke}%
\pgfsetdash{}{0pt}%
\pgfpathmoveto{\pgfqpoint{2.642560in}{3.294969in}}%
\pgfpathlineto{\pgfqpoint{2.531663in}{3.230943in}}%
\pgfpathlineto{\pgfqpoint{2.642560in}{3.166917in}}%
\pgfpathlineto{\pgfqpoint{2.753457in}{3.230943in}}%
\pgfpathlineto{\pgfqpoint{2.642560in}{3.294969in}}%
\pgfpathclose%
\pgfusepath{fill}%
\end{pgfscope}%
\begin{pgfscope}%
\pgfpathrectangle{\pgfqpoint{0.765000in}{0.660000in}}{\pgfqpoint{4.620000in}{4.620000in}}%
\pgfusepath{clip}%
\pgfsetbuttcap%
\pgfsetroundjoin%
\definecolor{currentfill}{rgb}{1.000000,0.894118,0.788235}%
\pgfsetfillcolor{currentfill}%
\pgfsetlinewidth{0.000000pt}%
\definecolor{currentstroke}{rgb}{1.000000,0.894118,0.788235}%
\pgfsetstrokecolor{currentstroke}%
\pgfsetdash{}{0pt}%
\pgfpathmoveto{\pgfqpoint{2.642560in}{3.294969in}}%
\pgfpathlineto{\pgfqpoint{2.531663in}{3.230943in}}%
\pgfpathlineto{\pgfqpoint{2.531663in}{3.358995in}}%
\pgfpathlineto{\pgfqpoint{2.642560in}{3.423022in}}%
\pgfpathlineto{\pgfqpoint{2.642560in}{3.294969in}}%
\pgfpathclose%
\pgfusepath{fill}%
\end{pgfscope}%
\begin{pgfscope}%
\pgfpathrectangle{\pgfqpoint{0.765000in}{0.660000in}}{\pgfqpoint{4.620000in}{4.620000in}}%
\pgfusepath{clip}%
\pgfsetbuttcap%
\pgfsetroundjoin%
\definecolor{currentfill}{rgb}{1.000000,0.894118,0.788235}%
\pgfsetfillcolor{currentfill}%
\pgfsetlinewidth{0.000000pt}%
\definecolor{currentstroke}{rgb}{1.000000,0.894118,0.788235}%
\pgfsetstrokecolor{currentstroke}%
\pgfsetdash{}{0pt}%
\pgfpathmoveto{\pgfqpoint{2.642560in}{3.294969in}}%
\pgfpathlineto{\pgfqpoint{2.753457in}{3.230943in}}%
\pgfpathlineto{\pgfqpoint{2.753457in}{3.358995in}}%
\pgfpathlineto{\pgfqpoint{2.642560in}{3.423022in}}%
\pgfpathlineto{\pgfqpoint{2.642560in}{3.294969in}}%
\pgfpathclose%
\pgfusepath{fill}%
\end{pgfscope}%
\begin{pgfscope}%
\pgfpathrectangle{\pgfqpoint{0.765000in}{0.660000in}}{\pgfqpoint{4.620000in}{4.620000in}}%
\pgfusepath{clip}%
\pgfsetbuttcap%
\pgfsetroundjoin%
\definecolor{currentfill}{rgb}{1.000000,0.894118,0.788235}%
\pgfsetfillcolor{currentfill}%
\pgfsetlinewidth{0.000000pt}%
\definecolor{currentstroke}{rgb}{1.000000,0.894118,0.788235}%
\pgfsetstrokecolor{currentstroke}%
\pgfsetdash{}{0pt}%
\pgfpathmoveto{\pgfqpoint{2.635438in}{3.413921in}}%
\pgfpathlineto{\pgfqpoint{2.524541in}{3.349894in}}%
\pgfpathlineto{\pgfqpoint{2.635438in}{3.285868in}}%
\pgfpathlineto{\pgfqpoint{2.746335in}{3.349894in}}%
\pgfpathlineto{\pgfqpoint{2.635438in}{3.413921in}}%
\pgfpathclose%
\pgfusepath{fill}%
\end{pgfscope}%
\begin{pgfscope}%
\pgfpathrectangle{\pgfqpoint{0.765000in}{0.660000in}}{\pgfqpoint{4.620000in}{4.620000in}}%
\pgfusepath{clip}%
\pgfsetbuttcap%
\pgfsetroundjoin%
\definecolor{currentfill}{rgb}{1.000000,0.894118,0.788235}%
\pgfsetfillcolor{currentfill}%
\pgfsetlinewidth{0.000000pt}%
\definecolor{currentstroke}{rgb}{1.000000,0.894118,0.788235}%
\pgfsetstrokecolor{currentstroke}%
\pgfsetdash{}{0pt}%
\pgfpathmoveto{\pgfqpoint{2.635438in}{3.157816in}}%
\pgfpathlineto{\pgfqpoint{2.746335in}{3.221842in}}%
\pgfpathlineto{\pgfqpoint{2.746335in}{3.349894in}}%
\pgfpathlineto{\pgfqpoint{2.635438in}{3.285868in}}%
\pgfpathlineto{\pgfqpoint{2.635438in}{3.157816in}}%
\pgfpathclose%
\pgfusepath{fill}%
\end{pgfscope}%
\begin{pgfscope}%
\pgfpathrectangle{\pgfqpoint{0.765000in}{0.660000in}}{\pgfqpoint{4.620000in}{4.620000in}}%
\pgfusepath{clip}%
\pgfsetbuttcap%
\pgfsetroundjoin%
\definecolor{currentfill}{rgb}{1.000000,0.894118,0.788235}%
\pgfsetfillcolor{currentfill}%
\pgfsetlinewidth{0.000000pt}%
\definecolor{currentstroke}{rgb}{1.000000,0.894118,0.788235}%
\pgfsetstrokecolor{currentstroke}%
\pgfsetdash{}{0pt}%
\pgfpathmoveto{\pgfqpoint{2.524541in}{3.221842in}}%
\pgfpathlineto{\pgfqpoint{2.635438in}{3.157816in}}%
\pgfpathlineto{\pgfqpoint{2.635438in}{3.285868in}}%
\pgfpathlineto{\pgfqpoint{2.524541in}{3.349894in}}%
\pgfpathlineto{\pgfqpoint{2.524541in}{3.221842in}}%
\pgfpathclose%
\pgfusepath{fill}%
\end{pgfscope}%
\begin{pgfscope}%
\pgfpathrectangle{\pgfqpoint{0.765000in}{0.660000in}}{\pgfqpoint{4.620000in}{4.620000in}}%
\pgfusepath{clip}%
\pgfsetbuttcap%
\pgfsetroundjoin%
\definecolor{currentfill}{rgb}{1.000000,0.894118,0.788235}%
\pgfsetfillcolor{currentfill}%
\pgfsetlinewidth{0.000000pt}%
\definecolor{currentstroke}{rgb}{1.000000,0.894118,0.788235}%
\pgfsetstrokecolor{currentstroke}%
\pgfsetdash{}{0pt}%
\pgfpathmoveto{\pgfqpoint{2.642560in}{3.423022in}}%
\pgfpathlineto{\pgfqpoint{2.531663in}{3.358995in}}%
\pgfpathlineto{\pgfqpoint{2.642560in}{3.294969in}}%
\pgfpathlineto{\pgfqpoint{2.753457in}{3.358995in}}%
\pgfpathlineto{\pgfqpoint{2.642560in}{3.423022in}}%
\pgfpathclose%
\pgfusepath{fill}%
\end{pgfscope}%
\begin{pgfscope}%
\pgfpathrectangle{\pgfqpoint{0.765000in}{0.660000in}}{\pgfqpoint{4.620000in}{4.620000in}}%
\pgfusepath{clip}%
\pgfsetbuttcap%
\pgfsetroundjoin%
\definecolor{currentfill}{rgb}{1.000000,0.894118,0.788235}%
\pgfsetfillcolor{currentfill}%
\pgfsetlinewidth{0.000000pt}%
\definecolor{currentstroke}{rgb}{1.000000,0.894118,0.788235}%
\pgfsetstrokecolor{currentstroke}%
\pgfsetdash{}{0pt}%
\pgfpathmoveto{\pgfqpoint{2.642560in}{3.166917in}}%
\pgfpathlineto{\pgfqpoint{2.753457in}{3.230943in}}%
\pgfpathlineto{\pgfqpoint{2.753457in}{3.358995in}}%
\pgfpathlineto{\pgfqpoint{2.642560in}{3.294969in}}%
\pgfpathlineto{\pgfqpoint{2.642560in}{3.166917in}}%
\pgfpathclose%
\pgfusepath{fill}%
\end{pgfscope}%
\begin{pgfscope}%
\pgfpathrectangle{\pgfqpoint{0.765000in}{0.660000in}}{\pgfqpoint{4.620000in}{4.620000in}}%
\pgfusepath{clip}%
\pgfsetbuttcap%
\pgfsetroundjoin%
\definecolor{currentfill}{rgb}{1.000000,0.894118,0.788235}%
\pgfsetfillcolor{currentfill}%
\pgfsetlinewidth{0.000000pt}%
\definecolor{currentstroke}{rgb}{1.000000,0.894118,0.788235}%
\pgfsetstrokecolor{currentstroke}%
\pgfsetdash{}{0pt}%
\pgfpathmoveto{\pgfqpoint{2.531663in}{3.230943in}}%
\pgfpathlineto{\pgfqpoint{2.642560in}{3.166917in}}%
\pgfpathlineto{\pgfqpoint{2.642560in}{3.294969in}}%
\pgfpathlineto{\pgfqpoint{2.531663in}{3.358995in}}%
\pgfpathlineto{\pgfqpoint{2.531663in}{3.230943in}}%
\pgfpathclose%
\pgfusepath{fill}%
\end{pgfscope}%
\begin{pgfscope}%
\pgfpathrectangle{\pgfqpoint{0.765000in}{0.660000in}}{\pgfqpoint{4.620000in}{4.620000in}}%
\pgfusepath{clip}%
\pgfsetbuttcap%
\pgfsetroundjoin%
\definecolor{currentfill}{rgb}{1.000000,0.894118,0.788235}%
\pgfsetfillcolor{currentfill}%
\pgfsetlinewidth{0.000000pt}%
\definecolor{currentstroke}{rgb}{1.000000,0.894118,0.788235}%
\pgfsetstrokecolor{currentstroke}%
\pgfsetdash{}{0pt}%
\pgfpathmoveto{\pgfqpoint{2.635438in}{3.285868in}}%
\pgfpathlineto{\pgfqpoint{2.524541in}{3.221842in}}%
\pgfpathlineto{\pgfqpoint{2.531663in}{3.230943in}}%
\pgfpathlineto{\pgfqpoint{2.642560in}{3.294969in}}%
\pgfpathlineto{\pgfqpoint{2.635438in}{3.285868in}}%
\pgfpathclose%
\pgfusepath{fill}%
\end{pgfscope}%
\begin{pgfscope}%
\pgfpathrectangle{\pgfqpoint{0.765000in}{0.660000in}}{\pgfqpoint{4.620000in}{4.620000in}}%
\pgfusepath{clip}%
\pgfsetbuttcap%
\pgfsetroundjoin%
\definecolor{currentfill}{rgb}{1.000000,0.894118,0.788235}%
\pgfsetfillcolor{currentfill}%
\pgfsetlinewidth{0.000000pt}%
\definecolor{currentstroke}{rgb}{1.000000,0.894118,0.788235}%
\pgfsetstrokecolor{currentstroke}%
\pgfsetdash{}{0pt}%
\pgfpathmoveto{\pgfqpoint{2.746335in}{3.221842in}}%
\pgfpathlineto{\pgfqpoint{2.635438in}{3.285868in}}%
\pgfpathlineto{\pgfqpoint{2.642560in}{3.294969in}}%
\pgfpathlineto{\pgfqpoint{2.753457in}{3.230943in}}%
\pgfpathlineto{\pgfqpoint{2.746335in}{3.221842in}}%
\pgfpathclose%
\pgfusepath{fill}%
\end{pgfscope}%
\begin{pgfscope}%
\pgfpathrectangle{\pgfqpoint{0.765000in}{0.660000in}}{\pgfqpoint{4.620000in}{4.620000in}}%
\pgfusepath{clip}%
\pgfsetbuttcap%
\pgfsetroundjoin%
\definecolor{currentfill}{rgb}{1.000000,0.894118,0.788235}%
\pgfsetfillcolor{currentfill}%
\pgfsetlinewidth{0.000000pt}%
\definecolor{currentstroke}{rgb}{1.000000,0.894118,0.788235}%
\pgfsetstrokecolor{currentstroke}%
\pgfsetdash{}{0pt}%
\pgfpathmoveto{\pgfqpoint{2.635438in}{3.285868in}}%
\pgfpathlineto{\pgfqpoint{2.635438in}{3.413921in}}%
\pgfpathlineto{\pgfqpoint{2.642560in}{3.423022in}}%
\pgfpathlineto{\pgfqpoint{2.753457in}{3.230943in}}%
\pgfpathlineto{\pgfqpoint{2.635438in}{3.285868in}}%
\pgfpathclose%
\pgfusepath{fill}%
\end{pgfscope}%
\begin{pgfscope}%
\pgfpathrectangle{\pgfqpoint{0.765000in}{0.660000in}}{\pgfqpoint{4.620000in}{4.620000in}}%
\pgfusepath{clip}%
\pgfsetbuttcap%
\pgfsetroundjoin%
\definecolor{currentfill}{rgb}{1.000000,0.894118,0.788235}%
\pgfsetfillcolor{currentfill}%
\pgfsetlinewidth{0.000000pt}%
\definecolor{currentstroke}{rgb}{1.000000,0.894118,0.788235}%
\pgfsetstrokecolor{currentstroke}%
\pgfsetdash{}{0pt}%
\pgfpathmoveto{\pgfqpoint{2.746335in}{3.221842in}}%
\pgfpathlineto{\pgfqpoint{2.746335in}{3.349894in}}%
\pgfpathlineto{\pgfqpoint{2.753457in}{3.358995in}}%
\pgfpathlineto{\pgfqpoint{2.642560in}{3.294969in}}%
\pgfpathlineto{\pgfqpoint{2.746335in}{3.221842in}}%
\pgfpathclose%
\pgfusepath{fill}%
\end{pgfscope}%
\begin{pgfscope}%
\pgfpathrectangle{\pgfqpoint{0.765000in}{0.660000in}}{\pgfqpoint{4.620000in}{4.620000in}}%
\pgfusepath{clip}%
\pgfsetbuttcap%
\pgfsetroundjoin%
\definecolor{currentfill}{rgb}{1.000000,0.894118,0.788235}%
\pgfsetfillcolor{currentfill}%
\pgfsetlinewidth{0.000000pt}%
\definecolor{currentstroke}{rgb}{1.000000,0.894118,0.788235}%
\pgfsetstrokecolor{currentstroke}%
\pgfsetdash{}{0pt}%
\pgfpathmoveto{\pgfqpoint{2.524541in}{3.221842in}}%
\pgfpathlineto{\pgfqpoint{2.635438in}{3.157816in}}%
\pgfpathlineto{\pgfqpoint{2.642560in}{3.166917in}}%
\pgfpathlineto{\pgfqpoint{2.531663in}{3.230943in}}%
\pgfpathlineto{\pgfqpoint{2.524541in}{3.221842in}}%
\pgfpathclose%
\pgfusepath{fill}%
\end{pgfscope}%
\begin{pgfscope}%
\pgfpathrectangle{\pgfqpoint{0.765000in}{0.660000in}}{\pgfqpoint{4.620000in}{4.620000in}}%
\pgfusepath{clip}%
\pgfsetbuttcap%
\pgfsetroundjoin%
\definecolor{currentfill}{rgb}{1.000000,0.894118,0.788235}%
\pgfsetfillcolor{currentfill}%
\pgfsetlinewidth{0.000000pt}%
\definecolor{currentstroke}{rgb}{1.000000,0.894118,0.788235}%
\pgfsetstrokecolor{currentstroke}%
\pgfsetdash{}{0pt}%
\pgfpathmoveto{\pgfqpoint{2.635438in}{3.157816in}}%
\pgfpathlineto{\pgfqpoint{2.746335in}{3.221842in}}%
\pgfpathlineto{\pgfqpoint{2.753457in}{3.230943in}}%
\pgfpathlineto{\pgfqpoint{2.642560in}{3.166917in}}%
\pgfpathlineto{\pgfqpoint{2.635438in}{3.157816in}}%
\pgfpathclose%
\pgfusepath{fill}%
\end{pgfscope}%
\begin{pgfscope}%
\pgfpathrectangle{\pgfqpoint{0.765000in}{0.660000in}}{\pgfqpoint{4.620000in}{4.620000in}}%
\pgfusepath{clip}%
\pgfsetbuttcap%
\pgfsetroundjoin%
\definecolor{currentfill}{rgb}{1.000000,0.894118,0.788235}%
\pgfsetfillcolor{currentfill}%
\pgfsetlinewidth{0.000000pt}%
\definecolor{currentstroke}{rgb}{1.000000,0.894118,0.788235}%
\pgfsetstrokecolor{currentstroke}%
\pgfsetdash{}{0pt}%
\pgfpathmoveto{\pgfqpoint{2.635438in}{3.413921in}}%
\pgfpathlineto{\pgfqpoint{2.524541in}{3.349894in}}%
\pgfpathlineto{\pgfqpoint{2.531663in}{3.358995in}}%
\pgfpathlineto{\pgfqpoint{2.642560in}{3.423022in}}%
\pgfpathlineto{\pgfqpoint{2.635438in}{3.413921in}}%
\pgfpathclose%
\pgfusepath{fill}%
\end{pgfscope}%
\begin{pgfscope}%
\pgfpathrectangle{\pgfqpoint{0.765000in}{0.660000in}}{\pgfqpoint{4.620000in}{4.620000in}}%
\pgfusepath{clip}%
\pgfsetbuttcap%
\pgfsetroundjoin%
\definecolor{currentfill}{rgb}{1.000000,0.894118,0.788235}%
\pgfsetfillcolor{currentfill}%
\pgfsetlinewidth{0.000000pt}%
\definecolor{currentstroke}{rgb}{1.000000,0.894118,0.788235}%
\pgfsetstrokecolor{currentstroke}%
\pgfsetdash{}{0pt}%
\pgfpathmoveto{\pgfqpoint{2.746335in}{3.349894in}}%
\pgfpathlineto{\pgfqpoint{2.635438in}{3.413921in}}%
\pgfpathlineto{\pgfqpoint{2.642560in}{3.423022in}}%
\pgfpathlineto{\pgfqpoint{2.753457in}{3.358995in}}%
\pgfpathlineto{\pgfqpoint{2.746335in}{3.349894in}}%
\pgfpathclose%
\pgfusepath{fill}%
\end{pgfscope}%
\begin{pgfscope}%
\pgfpathrectangle{\pgfqpoint{0.765000in}{0.660000in}}{\pgfqpoint{4.620000in}{4.620000in}}%
\pgfusepath{clip}%
\pgfsetbuttcap%
\pgfsetroundjoin%
\definecolor{currentfill}{rgb}{1.000000,0.894118,0.788235}%
\pgfsetfillcolor{currentfill}%
\pgfsetlinewidth{0.000000pt}%
\definecolor{currentstroke}{rgb}{1.000000,0.894118,0.788235}%
\pgfsetstrokecolor{currentstroke}%
\pgfsetdash{}{0pt}%
\pgfpathmoveto{\pgfqpoint{2.524541in}{3.221842in}}%
\pgfpathlineto{\pgfqpoint{2.524541in}{3.349894in}}%
\pgfpathlineto{\pgfqpoint{2.531663in}{3.358995in}}%
\pgfpathlineto{\pgfqpoint{2.642560in}{3.166917in}}%
\pgfpathlineto{\pgfqpoint{2.524541in}{3.221842in}}%
\pgfpathclose%
\pgfusepath{fill}%
\end{pgfscope}%
\begin{pgfscope}%
\pgfpathrectangle{\pgfqpoint{0.765000in}{0.660000in}}{\pgfqpoint{4.620000in}{4.620000in}}%
\pgfusepath{clip}%
\pgfsetbuttcap%
\pgfsetroundjoin%
\definecolor{currentfill}{rgb}{1.000000,0.894118,0.788235}%
\pgfsetfillcolor{currentfill}%
\pgfsetlinewidth{0.000000pt}%
\definecolor{currentstroke}{rgb}{1.000000,0.894118,0.788235}%
\pgfsetstrokecolor{currentstroke}%
\pgfsetdash{}{0pt}%
\pgfpathmoveto{\pgfqpoint{2.635438in}{3.157816in}}%
\pgfpathlineto{\pgfqpoint{2.635438in}{3.285868in}}%
\pgfpathlineto{\pgfqpoint{2.642560in}{3.294969in}}%
\pgfpathlineto{\pgfqpoint{2.531663in}{3.230943in}}%
\pgfpathlineto{\pgfqpoint{2.635438in}{3.157816in}}%
\pgfpathclose%
\pgfusepath{fill}%
\end{pgfscope}%
\begin{pgfscope}%
\pgfpathrectangle{\pgfqpoint{0.765000in}{0.660000in}}{\pgfqpoint{4.620000in}{4.620000in}}%
\pgfusepath{clip}%
\pgfsetbuttcap%
\pgfsetroundjoin%
\definecolor{currentfill}{rgb}{1.000000,0.894118,0.788235}%
\pgfsetfillcolor{currentfill}%
\pgfsetlinewidth{0.000000pt}%
\definecolor{currentstroke}{rgb}{1.000000,0.894118,0.788235}%
\pgfsetstrokecolor{currentstroke}%
\pgfsetdash{}{0pt}%
\pgfpathmoveto{\pgfqpoint{2.635438in}{3.285868in}}%
\pgfpathlineto{\pgfqpoint{2.746335in}{3.349894in}}%
\pgfpathlineto{\pgfqpoint{2.753457in}{3.358995in}}%
\pgfpathlineto{\pgfqpoint{2.642560in}{3.294969in}}%
\pgfpathlineto{\pgfqpoint{2.635438in}{3.285868in}}%
\pgfpathclose%
\pgfusepath{fill}%
\end{pgfscope}%
\begin{pgfscope}%
\pgfpathrectangle{\pgfqpoint{0.765000in}{0.660000in}}{\pgfqpoint{4.620000in}{4.620000in}}%
\pgfusepath{clip}%
\pgfsetbuttcap%
\pgfsetroundjoin%
\definecolor{currentfill}{rgb}{1.000000,0.894118,0.788235}%
\pgfsetfillcolor{currentfill}%
\pgfsetlinewidth{0.000000pt}%
\definecolor{currentstroke}{rgb}{1.000000,0.894118,0.788235}%
\pgfsetstrokecolor{currentstroke}%
\pgfsetdash{}{0pt}%
\pgfpathmoveto{\pgfqpoint{2.524541in}{3.349894in}}%
\pgfpathlineto{\pgfqpoint{2.635438in}{3.285868in}}%
\pgfpathlineto{\pgfqpoint{2.642560in}{3.294969in}}%
\pgfpathlineto{\pgfqpoint{2.531663in}{3.358995in}}%
\pgfpathlineto{\pgfqpoint{2.524541in}{3.349894in}}%
\pgfpathclose%
\pgfusepath{fill}%
\end{pgfscope}%
\begin{pgfscope}%
\pgfpathrectangle{\pgfqpoint{0.765000in}{0.660000in}}{\pgfqpoint{4.620000in}{4.620000in}}%
\pgfusepath{clip}%
\pgfsetbuttcap%
\pgfsetroundjoin%
\definecolor{currentfill}{rgb}{1.000000,0.894118,0.788235}%
\pgfsetfillcolor{currentfill}%
\pgfsetlinewidth{0.000000pt}%
\definecolor{currentstroke}{rgb}{1.000000,0.894118,0.788235}%
\pgfsetstrokecolor{currentstroke}%
\pgfsetdash{}{0pt}%
\pgfpathmoveto{\pgfqpoint{2.613910in}{3.253240in}}%
\pgfpathlineto{\pgfqpoint{2.503014in}{3.189213in}}%
\pgfpathlineto{\pgfqpoint{2.613910in}{3.125187in}}%
\pgfpathlineto{\pgfqpoint{2.724807in}{3.189213in}}%
\pgfpathlineto{\pgfqpoint{2.613910in}{3.253240in}}%
\pgfpathclose%
\pgfusepath{fill}%
\end{pgfscope}%
\begin{pgfscope}%
\pgfpathrectangle{\pgfqpoint{0.765000in}{0.660000in}}{\pgfqpoint{4.620000in}{4.620000in}}%
\pgfusepath{clip}%
\pgfsetbuttcap%
\pgfsetroundjoin%
\definecolor{currentfill}{rgb}{1.000000,0.894118,0.788235}%
\pgfsetfillcolor{currentfill}%
\pgfsetlinewidth{0.000000pt}%
\definecolor{currentstroke}{rgb}{1.000000,0.894118,0.788235}%
\pgfsetstrokecolor{currentstroke}%
\pgfsetdash{}{0pt}%
\pgfpathmoveto{\pgfqpoint{2.613910in}{3.253240in}}%
\pgfpathlineto{\pgfqpoint{2.503014in}{3.189213in}}%
\pgfpathlineto{\pgfqpoint{2.503014in}{3.317266in}}%
\pgfpathlineto{\pgfqpoint{2.613910in}{3.381292in}}%
\pgfpathlineto{\pgfqpoint{2.613910in}{3.253240in}}%
\pgfpathclose%
\pgfusepath{fill}%
\end{pgfscope}%
\begin{pgfscope}%
\pgfpathrectangle{\pgfqpoint{0.765000in}{0.660000in}}{\pgfqpoint{4.620000in}{4.620000in}}%
\pgfusepath{clip}%
\pgfsetbuttcap%
\pgfsetroundjoin%
\definecolor{currentfill}{rgb}{1.000000,0.894118,0.788235}%
\pgfsetfillcolor{currentfill}%
\pgfsetlinewidth{0.000000pt}%
\definecolor{currentstroke}{rgb}{1.000000,0.894118,0.788235}%
\pgfsetstrokecolor{currentstroke}%
\pgfsetdash{}{0pt}%
\pgfpathmoveto{\pgfqpoint{2.613910in}{3.253240in}}%
\pgfpathlineto{\pgfqpoint{2.724807in}{3.189213in}}%
\pgfpathlineto{\pgfqpoint{2.724807in}{3.317266in}}%
\pgfpathlineto{\pgfqpoint{2.613910in}{3.381292in}}%
\pgfpathlineto{\pgfqpoint{2.613910in}{3.253240in}}%
\pgfpathclose%
\pgfusepath{fill}%
\end{pgfscope}%
\begin{pgfscope}%
\pgfpathrectangle{\pgfqpoint{0.765000in}{0.660000in}}{\pgfqpoint{4.620000in}{4.620000in}}%
\pgfusepath{clip}%
\pgfsetbuttcap%
\pgfsetroundjoin%
\definecolor{currentfill}{rgb}{1.000000,0.894118,0.788235}%
\pgfsetfillcolor{currentfill}%
\pgfsetlinewidth{0.000000pt}%
\definecolor{currentstroke}{rgb}{1.000000,0.894118,0.788235}%
\pgfsetstrokecolor{currentstroke}%
\pgfsetdash{}{0pt}%
\pgfpathmoveto{\pgfqpoint{2.635438in}{3.285868in}}%
\pgfpathlineto{\pgfqpoint{2.524541in}{3.221842in}}%
\pgfpathlineto{\pgfqpoint{2.635438in}{3.157816in}}%
\pgfpathlineto{\pgfqpoint{2.746335in}{3.221842in}}%
\pgfpathlineto{\pgfqpoint{2.635438in}{3.285868in}}%
\pgfpathclose%
\pgfusepath{fill}%
\end{pgfscope}%
\begin{pgfscope}%
\pgfpathrectangle{\pgfqpoint{0.765000in}{0.660000in}}{\pgfqpoint{4.620000in}{4.620000in}}%
\pgfusepath{clip}%
\pgfsetbuttcap%
\pgfsetroundjoin%
\definecolor{currentfill}{rgb}{1.000000,0.894118,0.788235}%
\pgfsetfillcolor{currentfill}%
\pgfsetlinewidth{0.000000pt}%
\definecolor{currentstroke}{rgb}{1.000000,0.894118,0.788235}%
\pgfsetstrokecolor{currentstroke}%
\pgfsetdash{}{0pt}%
\pgfpathmoveto{\pgfqpoint{2.635438in}{3.285868in}}%
\pgfpathlineto{\pgfqpoint{2.524541in}{3.221842in}}%
\pgfpathlineto{\pgfqpoint{2.524541in}{3.349894in}}%
\pgfpathlineto{\pgfqpoint{2.635438in}{3.413921in}}%
\pgfpathlineto{\pgfqpoint{2.635438in}{3.285868in}}%
\pgfpathclose%
\pgfusepath{fill}%
\end{pgfscope}%
\begin{pgfscope}%
\pgfpathrectangle{\pgfqpoint{0.765000in}{0.660000in}}{\pgfqpoint{4.620000in}{4.620000in}}%
\pgfusepath{clip}%
\pgfsetbuttcap%
\pgfsetroundjoin%
\definecolor{currentfill}{rgb}{1.000000,0.894118,0.788235}%
\pgfsetfillcolor{currentfill}%
\pgfsetlinewidth{0.000000pt}%
\definecolor{currentstroke}{rgb}{1.000000,0.894118,0.788235}%
\pgfsetstrokecolor{currentstroke}%
\pgfsetdash{}{0pt}%
\pgfpathmoveto{\pgfqpoint{2.635438in}{3.285868in}}%
\pgfpathlineto{\pgfqpoint{2.746335in}{3.221842in}}%
\pgfpathlineto{\pgfqpoint{2.746335in}{3.349894in}}%
\pgfpathlineto{\pgfqpoint{2.635438in}{3.413921in}}%
\pgfpathlineto{\pgfqpoint{2.635438in}{3.285868in}}%
\pgfpathclose%
\pgfusepath{fill}%
\end{pgfscope}%
\begin{pgfscope}%
\pgfpathrectangle{\pgfqpoint{0.765000in}{0.660000in}}{\pgfqpoint{4.620000in}{4.620000in}}%
\pgfusepath{clip}%
\pgfsetbuttcap%
\pgfsetroundjoin%
\definecolor{currentfill}{rgb}{1.000000,0.894118,0.788235}%
\pgfsetfillcolor{currentfill}%
\pgfsetlinewidth{0.000000pt}%
\definecolor{currentstroke}{rgb}{1.000000,0.894118,0.788235}%
\pgfsetstrokecolor{currentstroke}%
\pgfsetdash{}{0pt}%
\pgfpathmoveto{\pgfqpoint{2.613910in}{3.381292in}}%
\pgfpathlineto{\pgfqpoint{2.503014in}{3.317266in}}%
\pgfpathlineto{\pgfqpoint{2.613910in}{3.253240in}}%
\pgfpathlineto{\pgfqpoint{2.724807in}{3.317266in}}%
\pgfpathlineto{\pgfqpoint{2.613910in}{3.381292in}}%
\pgfpathclose%
\pgfusepath{fill}%
\end{pgfscope}%
\begin{pgfscope}%
\pgfpathrectangle{\pgfqpoint{0.765000in}{0.660000in}}{\pgfqpoint{4.620000in}{4.620000in}}%
\pgfusepath{clip}%
\pgfsetbuttcap%
\pgfsetroundjoin%
\definecolor{currentfill}{rgb}{1.000000,0.894118,0.788235}%
\pgfsetfillcolor{currentfill}%
\pgfsetlinewidth{0.000000pt}%
\definecolor{currentstroke}{rgb}{1.000000,0.894118,0.788235}%
\pgfsetstrokecolor{currentstroke}%
\pgfsetdash{}{0pt}%
\pgfpathmoveto{\pgfqpoint{2.613910in}{3.125187in}}%
\pgfpathlineto{\pgfqpoint{2.724807in}{3.189213in}}%
\pgfpathlineto{\pgfqpoint{2.724807in}{3.317266in}}%
\pgfpathlineto{\pgfqpoint{2.613910in}{3.253240in}}%
\pgfpathlineto{\pgfqpoint{2.613910in}{3.125187in}}%
\pgfpathclose%
\pgfusepath{fill}%
\end{pgfscope}%
\begin{pgfscope}%
\pgfpathrectangle{\pgfqpoint{0.765000in}{0.660000in}}{\pgfqpoint{4.620000in}{4.620000in}}%
\pgfusepath{clip}%
\pgfsetbuttcap%
\pgfsetroundjoin%
\definecolor{currentfill}{rgb}{1.000000,0.894118,0.788235}%
\pgfsetfillcolor{currentfill}%
\pgfsetlinewidth{0.000000pt}%
\definecolor{currentstroke}{rgb}{1.000000,0.894118,0.788235}%
\pgfsetstrokecolor{currentstroke}%
\pgfsetdash{}{0pt}%
\pgfpathmoveto{\pgfqpoint{2.503014in}{3.189213in}}%
\pgfpathlineto{\pgfqpoint{2.613910in}{3.125187in}}%
\pgfpathlineto{\pgfqpoint{2.613910in}{3.253240in}}%
\pgfpathlineto{\pgfqpoint{2.503014in}{3.317266in}}%
\pgfpathlineto{\pgfqpoint{2.503014in}{3.189213in}}%
\pgfpathclose%
\pgfusepath{fill}%
\end{pgfscope}%
\begin{pgfscope}%
\pgfpathrectangle{\pgfqpoint{0.765000in}{0.660000in}}{\pgfqpoint{4.620000in}{4.620000in}}%
\pgfusepath{clip}%
\pgfsetbuttcap%
\pgfsetroundjoin%
\definecolor{currentfill}{rgb}{1.000000,0.894118,0.788235}%
\pgfsetfillcolor{currentfill}%
\pgfsetlinewidth{0.000000pt}%
\definecolor{currentstroke}{rgb}{1.000000,0.894118,0.788235}%
\pgfsetstrokecolor{currentstroke}%
\pgfsetdash{}{0pt}%
\pgfpathmoveto{\pgfqpoint{2.635438in}{3.413921in}}%
\pgfpathlineto{\pgfqpoint{2.524541in}{3.349894in}}%
\pgfpathlineto{\pgfqpoint{2.635438in}{3.285868in}}%
\pgfpathlineto{\pgfqpoint{2.746335in}{3.349894in}}%
\pgfpathlineto{\pgfqpoint{2.635438in}{3.413921in}}%
\pgfpathclose%
\pgfusepath{fill}%
\end{pgfscope}%
\begin{pgfscope}%
\pgfpathrectangle{\pgfqpoint{0.765000in}{0.660000in}}{\pgfqpoint{4.620000in}{4.620000in}}%
\pgfusepath{clip}%
\pgfsetbuttcap%
\pgfsetroundjoin%
\definecolor{currentfill}{rgb}{1.000000,0.894118,0.788235}%
\pgfsetfillcolor{currentfill}%
\pgfsetlinewidth{0.000000pt}%
\definecolor{currentstroke}{rgb}{1.000000,0.894118,0.788235}%
\pgfsetstrokecolor{currentstroke}%
\pgfsetdash{}{0pt}%
\pgfpathmoveto{\pgfqpoint{2.635438in}{3.157816in}}%
\pgfpathlineto{\pgfqpoint{2.746335in}{3.221842in}}%
\pgfpathlineto{\pgfqpoint{2.746335in}{3.349894in}}%
\pgfpathlineto{\pgfqpoint{2.635438in}{3.285868in}}%
\pgfpathlineto{\pgfqpoint{2.635438in}{3.157816in}}%
\pgfpathclose%
\pgfusepath{fill}%
\end{pgfscope}%
\begin{pgfscope}%
\pgfpathrectangle{\pgfqpoint{0.765000in}{0.660000in}}{\pgfqpoint{4.620000in}{4.620000in}}%
\pgfusepath{clip}%
\pgfsetbuttcap%
\pgfsetroundjoin%
\definecolor{currentfill}{rgb}{1.000000,0.894118,0.788235}%
\pgfsetfillcolor{currentfill}%
\pgfsetlinewidth{0.000000pt}%
\definecolor{currentstroke}{rgb}{1.000000,0.894118,0.788235}%
\pgfsetstrokecolor{currentstroke}%
\pgfsetdash{}{0pt}%
\pgfpathmoveto{\pgfqpoint{2.524541in}{3.221842in}}%
\pgfpathlineto{\pgfqpoint{2.635438in}{3.157816in}}%
\pgfpathlineto{\pgfqpoint{2.635438in}{3.285868in}}%
\pgfpathlineto{\pgfqpoint{2.524541in}{3.349894in}}%
\pgfpathlineto{\pgfqpoint{2.524541in}{3.221842in}}%
\pgfpathclose%
\pgfusepath{fill}%
\end{pgfscope}%
\begin{pgfscope}%
\pgfpathrectangle{\pgfqpoint{0.765000in}{0.660000in}}{\pgfqpoint{4.620000in}{4.620000in}}%
\pgfusepath{clip}%
\pgfsetbuttcap%
\pgfsetroundjoin%
\definecolor{currentfill}{rgb}{1.000000,0.894118,0.788235}%
\pgfsetfillcolor{currentfill}%
\pgfsetlinewidth{0.000000pt}%
\definecolor{currentstroke}{rgb}{1.000000,0.894118,0.788235}%
\pgfsetstrokecolor{currentstroke}%
\pgfsetdash{}{0pt}%
\pgfpathmoveto{\pgfqpoint{2.613910in}{3.253240in}}%
\pgfpathlineto{\pgfqpoint{2.503014in}{3.189213in}}%
\pgfpathlineto{\pgfqpoint{2.524541in}{3.221842in}}%
\pgfpathlineto{\pgfqpoint{2.635438in}{3.285868in}}%
\pgfpathlineto{\pgfqpoint{2.613910in}{3.253240in}}%
\pgfpathclose%
\pgfusepath{fill}%
\end{pgfscope}%
\begin{pgfscope}%
\pgfpathrectangle{\pgfqpoint{0.765000in}{0.660000in}}{\pgfqpoint{4.620000in}{4.620000in}}%
\pgfusepath{clip}%
\pgfsetbuttcap%
\pgfsetroundjoin%
\definecolor{currentfill}{rgb}{1.000000,0.894118,0.788235}%
\pgfsetfillcolor{currentfill}%
\pgfsetlinewidth{0.000000pt}%
\definecolor{currentstroke}{rgb}{1.000000,0.894118,0.788235}%
\pgfsetstrokecolor{currentstroke}%
\pgfsetdash{}{0pt}%
\pgfpathmoveto{\pgfqpoint{2.724807in}{3.189213in}}%
\pgfpathlineto{\pgfqpoint{2.613910in}{3.253240in}}%
\pgfpathlineto{\pgfqpoint{2.635438in}{3.285868in}}%
\pgfpathlineto{\pgfqpoint{2.746335in}{3.221842in}}%
\pgfpathlineto{\pgfqpoint{2.724807in}{3.189213in}}%
\pgfpathclose%
\pgfusepath{fill}%
\end{pgfscope}%
\begin{pgfscope}%
\pgfpathrectangle{\pgfqpoint{0.765000in}{0.660000in}}{\pgfqpoint{4.620000in}{4.620000in}}%
\pgfusepath{clip}%
\pgfsetbuttcap%
\pgfsetroundjoin%
\definecolor{currentfill}{rgb}{1.000000,0.894118,0.788235}%
\pgfsetfillcolor{currentfill}%
\pgfsetlinewidth{0.000000pt}%
\definecolor{currentstroke}{rgb}{1.000000,0.894118,0.788235}%
\pgfsetstrokecolor{currentstroke}%
\pgfsetdash{}{0pt}%
\pgfpathmoveto{\pgfqpoint{2.613910in}{3.253240in}}%
\pgfpathlineto{\pgfqpoint{2.613910in}{3.381292in}}%
\pgfpathlineto{\pgfqpoint{2.635438in}{3.413921in}}%
\pgfpathlineto{\pgfqpoint{2.746335in}{3.221842in}}%
\pgfpathlineto{\pgfqpoint{2.613910in}{3.253240in}}%
\pgfpathclose%
\pgfusepath{fill}%
\end{pgfscope}%
\begin{pgfscope}%
\pgfpathrectangle{\pgfqpoint{0.765000in}{0.660000in}}{\pgfqpoint{4.620000in}{4.620000in}}%
\pgfusepath{clip}%
\pgfsetbuttcap%
\pgfsetroundjoin%
\definecolor{currentfill}{rgb}{1.000000,0.894118,0.788235}%
\pgfsetfillcolor{currentfill}%
\pgfsetlinewidth{0.000000pt}%
\definecolor{currentstroke}{rgb}{1.000000,0.894118,0.788235}%
\pgfsetstrokecolor{currentstroke}%
\pgfsetdash{}{0pt}%
\pgfpathmoveto{\pgfqpoint{2.724807in}{3.189213in}}%
\pgfpathlineto{\pgfqpoint{2.724807in}{3.317266in}}%
\pgfpathlineto{\pgfqpoint{2.746335in}{3.349894in}}%
\pgfpathlineto{\pgfqpoint{2.635438in}{3.285868in}}%
\pgfpathlineto{\pgfqpoint{2.724807in}{3.189213in}}%
\pgfpathclose%
\pgfusepath{fill}%
\end{pgfscope}%
\begin{pgfscope}%
\pgfpathrectangle{\pgfqpoint{0.765000in}{0.660000in}}{\pgfqpoint{4.620000in}{4.620000in}}%
\pgfusepath{clip}%
\pgfsetbuttcap%
\pgfsetroundjoin%
\definecolor{currentfill}{rgb}{1.000000,0.894118,0.788235}%
\pgfsetfillcolor{currentfill}%
\pgfsetlinewidth{0.000000pt}%
\definecolor{currentstroke}{rgb}{1.000000,0.894118,0.788235}%
\pgfsetstrokecolor{currentstroke}%
\pgfsetdash{}{0pt}%
\pgfpathmoveto{\pgfqpoint{2.503014in}{3.189213in}}%
\pgfpathlineto{\pgfqpoint{2.613910in}{3.125187in}}%
\pgfpathlineto{\pgfqpoint{2.635438in}{3.157816in}}%
\pgfpathlineto{\pgfqpoint{2.524541in}{3.221842in}}%
\pgfpathlineto{\pgfqpoint{2.503014in}{3.189213in}}%
\pgfpathclose%
\pgfusepath{fill}%
\end{pgfscope}%
\begin{pgfscope}%
\pgfpathrectangle{\pgfqpoint{0.765000in}{0.660000in}}{\pgfqpoint{4.620000in}{4.620000in}}%
\pgfusepath{clip}%
\pgfsetbuttcap%
\pgfsetroundjoin%
\definecolor{currentfill}{rgb}{1.000000,0.894118,0.788235}%
\pgfsetfillcolor{currentfill}%
\pgfsetlinewidth{0.000000pt}%
\definecolor{currentstroke}{rgb}{1.000000,0.894118,0.788235}%
\pgfsetstrokecolor{currentstroke}%
\pgfsetdash{}{0pt}%
\pgfpathmoveto{\pgfqpoint{2.613910in}{3.125187in}}%
\pgfpathlineto{\pgfqpoint{2.724807in}{3.189213in}}%
\pgfpathlineto{\pgfqpoint{2.746335in}{3.221842in}}%
\pgfpathlineto{\pgfqpoint{2.635438in}{3.157816in}}%
\pgfpathlineto{\pgfqpoint{2.613910in}{3.125187in}}%
\pgfpathclose%
\pgfusepath{fill}%
\end{pgfscope}%
\begin{pgfscope}%
\pgfpathrectangle{\pgfqpoint{0.765000in}{0.660000in}}{\pgfqpoint{4.620000in}{4.620000in}}%
\pgfusepath{clip}%
\pgfsetbuttcap%
\pgfsetroundjoin%
\definecolor{currentfill}{rgb}{1.000000,0.894118,0.788235}%
\pgfsetfillcolor{currentfill}%
\pgfsetlinewidth{0.000000pt}%
\definecolor{currentstroke}{rgb}{1.000000,0.894118,0.788235}%
\pgfsetstrokecolor{currentstroke}%
\pgfsetdash{}{0pt}%
\pgfpathmoveto{\pgfqpoint{2.613910in}{3.381292in}}%
\pgfpathlineto{\pgfqpoint{2.503014in}{3.317266in}}%
\pgfpathlineto{\pgfqpoint{2.524541in}{3.349894in}}%
\pgfpathlineto{\pgfqpoint{2.635438in}{3.413921in}}%
\pgfpathlineto{\pgfqpoint{2.613910in}{3.381292in}}%
\pgfpathclose%
\pgfusepath{fill}%
\end{pgfscope}%
\begin{pgfscope}%
\pgfpathrectangle{\pgfqpoint{0.765000in}{0.660000in}}{\pgfqpoint{4.620000in}{4.620000in}}%
\pgfusepath{clip}%
\pgfsetbuttcap%
\pgfsetroundjoin%
\definecolor{currentfill}{rgb}{1.000000,0.894118,0.788235}%
\pgfsetfillcolor{currentfill}%
\pgfsetlinewidth{0.000000pt}%
\definecolor{currentstroke}{rgb}{1.000000,0.894118,0.788235}%
\pgfsetstrokecolor{currentstroke}%
\pgfsetdash{}{0pt}%
\pgfpathmoveto{\pgfqpoint{2.724807in}{3.317266in}}%
\pgfpathlineto{\pgfqpoint{2.613910in}{3.381292in}}%
\pgfpathlineto{\pgfqpoint{2.635438in}{3.413921in}}%
\pgfpathlineto{\pgfqpoint{2.746335in}{3.349894in}}%
\pgfpathlineto{\pgfqpoint{2.724807in}{3.317266in}}%
\pgfpathclose%
\pgfusepath{fill}%
\end{pgfscope}%
\begin{pgfscope}%
\pgfpathrectangle{\pgfqpoint{0.765000in}{0.660000in}}{\pgfqpoint{4.620000in}{4.620000in}}%
\pgfusepath{clip}%
\pgfsetbuttcap%
\pgfsetroundjoin%
\definecolor{currentfill}{rgb}{1.000000,0.894118,0.788235}%
\pgfsetfillcolor{currentfill}%
\pgfsetlinewidth{0.000000pt}%
\definecolor{currentstroke}{rgb}{1.000000,0.894118,0.788235}%
\pgfsetstrokecolor{currentstroke}%
\pgfsetdash{}{0pt}%
\pgfpathmoveto{\pgfqpoint{2.503014in}{3.189213in}}%
\pgfpathlineto{\pgfqpoint{2.503014in}{3.317266in}}%
\pgfpathlineto{\pgfqpoint{2.524541in}{3.349894in}}%
\pgfpathlineto{\pgfqpoint{2.635438in}{3.157816in}}%
\pgfpathlineto{\pgfqpoint{2.503014in}{3.189213in}}%
\pgfpathclose%
\pgfusepath{fill}%
\end{pgfscope}%
\begin{pgfscope}%
\pgfpathrectangle{\pgfqpoint{0.765000in}{0.660000in}}{\pgfqpoint{4.620000in}{4.620000in}}%
\pgfusepath{clip}%
\pgfsetbuttcap%
\pgfsetroundjoin%
\definecolor{currentfill}{rgb}{1.000000,0.894118,0.788235}%
\pgfsetfillcolor{currentfill}%
\pgfsetlinewidth{0.000000pt}%
\definecolor{currentstroke}{rgb}{1.000000,0.894118,0.788235}%
\pgfsetstrokecolor{currentstroke}%
\pgfsetdash{}{0pt}%
\pgfpathmoveto{\pgfqpoint{2.613910in}{3.125187in}}%
\pgfpathlineto{\pgfqpoint{2.613910in}{3.253240in}}%
\pgfpathlineto{\pgfqpoint{2.635438in}{3.285868in}}%
\pgfpathlineto{\pgfqpoint{2.524541in}{3.221842in}}%
\pgfpathlineto{\pgfqpoint{2.613910in}{3.125187in}}%
\pgfpathclose%
\pgfusepath{fill}%
\end{pgfscope}%
\begin{pgfscope}%
\pgfpathrectangle{\pgfqpoint{0.765000in}{0.660000in}}{\pgfqpoint{4.620000in}{4.620000in}}%
\pgfusepath{clip}%
\pgfsetbuttcap%
\pgfsetroundjoin%
\definecolor{currentfill}{rgb}{1.000000,0.894118,0.788235}%
\pgfsetfillcolor{currentfill}%
\pgfsetlinewidth{0.000000pt}%
\definecolor{currentstroke}{rgb}{1.000000,0.894118,0.788235}%
\pgfsetstrokecolor{currentstroke}%
\pgfsetdash{}{0pt}%
\pgfpathmoveto{\pgfqpoint{2.613910in}{3.253240in}}%
\pgfpathlineto{\pgfqpoint{2.724807in}{3.317266in}}%
\pgfpathlineto{\pgfqpoint{2.746335in}{3.349894in}}%
\pgfpathlineto{\pgfqpoint{2.635438in}{3.285868in}}%
\pgfpathlineto{\pgfqpoint{2.613910in}{3.253240in}}%
\pgfpathclose%
\pgfusepath{fill}%
\end{pgfscope}%
\begin{pgfscope}%
\pgfpathrectangle{\pgfqpoint{0.765000in}{0.660000in}}{\pgfqpoint{4.620000in}{4.620000in}}%
\pgfusepath{clip}%
\pgfsetbuttcap%
\pgfsetroundjoin%
\definecolor{currentfill}{rgb}{1.000000,0.894118,0.788235}%
\pgfsetfillcolor{currentfill}%
\pgfsetlinewidth{0.000000pt}%
\definecolor{currentstroke}{rgb}{1.000000,0.894118,0.788235}%
\pgfsetstrokecolor{currentstroke}%
\pgfsetdash{}{0pt}%
\pgfpathmoveto{\pgfqpoint{2.503014in}{3.317266in}}%
\pgfpathlineto{\pgfqpoint{2.613910in}{3.253240in}}%
\pgfpathlineto{\pgfqpoint{2.635438in}{3.285868in}}%
\pgfpathlineto{\pgfqpoint{2.524541in}{3.349894in}}%
\pgfpathlineto{\pgfqpoint{2.503014in}{3.317266in}}%
\pgfpathclose%
\pgfusepath{fill}%
\end{pgfscope}%
\begin{pgfscope}%
\pgfpathrectangle{\pgfqpoint{0.765000in}{0.660000in}}{\pgfqpoint{4.620000in}{4.620000in}}%
\pgfusepath{clip}%
\pgfsetbuttcap%
\pgfsetroundjoin%
\definecolor{currentfill}{rgb}{1.000000,0.894118,0.788235}%
\pgfsetfillcolor{currentfill}%
\pgfsetlinewidth{0.000000pt}%
\definecolor{currentstroke}{rgb}{1.000000,0.894118,0.788235}%
\pgfsetstrokecolor{currentstroke}%
\pgfsetdash{}{0pt}%
\pgfpathmoveto{\pgfqpoint{2.613910in}{3.253240in}}%
\pgfpathlineto{\pgfqpoint{2.503014in}{3.189213in}}%
\pgfpathlineto{\pgfqpoint{2.613910in}{3.125187in}}%
\pgfpathlineto{\pgfqpoint{2.724807in}{3.189213in}}%
\pgfpathlineto{\pgfqpoint{2.613910in}{3.253240in}}%
\pgfpathclose%
\pgfusepath{fill}%
\end{pgfscope}%
\begin{pgfscope}%
\pgfpathrectangle{\pgfqpoint{0.765000in}{0.660000in}}{\pgfqpoint{4.620000in}{4.620000in}}%
\pgfusepath{clip}%
\pgfsetbuttcap%
\pgfsetroundjoin%
\definecolor{currentfill}{rgb}{1.000000,0.894118,0.788235}%
\pgfsetfillcolor{currentfill}%
\pgfsetlinewidth{0.000000pt}%
\definecolor{currentstroke}{rgb}{1.000000,0.894118,0.788235}%
\pgfsetstrokecolor{currentstroke}%
\pgfsetdash{}{0pt}%
\pgfpathmoveto{\pgfqpoint{2.613910in}{3.253240in}}%
\pgfpathlineto{\pgfqpoint{2.503014in}{3.189213in}}%
\pgfpathlineto{\pgfqpoint{2.503014in}{3.317266in}}%
\pgfpathlineto{\pgfqpoint{2.613910in}{3.381292in}}%
\pgfpathlineto{\pgfqpoint{2.613910in}{3.253240in}}%
\pgfpathclose%
\pgfusepath{fill}%
\end{pgfscope}%
\begin{pgfscope}%
\pgfpathrectangle{\pgfqpoint{0.765000in}{0.660000in}}{\pgfqpoint{4.620000in}{4.620000in}}%
\pgfusepath{clip}%
\pgfsetbuttcap%
\pgfsetroundjoin%
\definecolor{currentfill}{rgb}{1.000000,0.894118,0.788235}%
\pgfsetfillcolor{currentfill}%
\pgfsetlinewidth{0.000000pt}%
\definecolor{currentstroke}{rgb}{1.000000,0.894118,0.788235}%
\pgfsetstrokecolor{currentstroke}%
\pgfsetdash{}{0pt}%
\pgfpathmoveto{\pgfqpoint{2.613910in}{3.253240in}}%
\pgfpathlineto{\pgfqpoint{2.724807in}{3.189213in}}%
\pgfpathlineto{\pgfqpoint{2.724807in}{3.317266in}}%
\pgfpathlineto{\pgfqpoint{2.613910in}{3.381292in}}%
\pgfpathlineto{\pgfqpoint{2.613910in}{3.253240in}}%
\pgfpathclose%
\pgfusepath{fill}%
\end{pgfscope}%
\begin{pgfscope}%
\pgfpathrectangle{\pgfqpoint{0.765000in}{0.660000in}}{\pgfqpoint{4.620000in}{4.620000in}}%
\pgfusepath{clip}%
\pgfsetbuttcap%
\pgfsetroundjoin%
\definecolor{currentfill}{rgb}{1.000000,0.894118,0.788235}%
\pgfsetfillcolor{currentfill}%
\pgfsetlinewidth{0.000000pt}%
\definecolor{currentstroke}{rgb}{1.000000,0.894118,0.788235}%
\pgfsetstrokecolor{currentstroke}%
\pgfsetdash{}{0pt}%
\pgfpathmoveto{\pgfqpoint{2.597727in}{3.213477in}}%
\pgfpathlineto{\pgfqpoint{2.486831in}{3.149451in}}%
\pgfpathlineto{\pgfqpoint{2.597727in}{3.085425in}}%
\pgfpathlineto{\pgfqpoint{2.708624in}{3.149451in}}%
\pgfpathlineto{\pgfqpoint{2.597727in}{3.213477in}}%
\pgfpathclose%
\pgfusepath{fill}%
\end{pgfscope}%
\begin{pgfscope}%
\pgfpathrectangle{\pgfqpoint{0.765000in}{0.660000in}}{\pgfqpoint{4.620000in}{4.620000in}}%
\pgfusepath{clip}%
\pgfsetbuttcap%
\pgfsetroundjoin%
\definecolor{currentfill}{rgb}{1.000000,0.894118,0.788235}%
\pgfsetfillcolor{currentfill}%
\pgfsetlinewidth{0.000000pt}%
\definecolor{currentstroke}{rgb}{1.000000,0.894118,0.788235}%
\pgfsetstrokecolor{currentstroke}%
\pgfsetdash{}{0pt}%
\pgfpathmoveto{\pgfqpoint{2.597727in}{3.213477in}}%
\pgfpathlineto{\pgfqpoint{2.486831in}{3.149451in}}%
\pgfpathlineto{\pgfqpoint{2.486831in}{3.277504in}}%
\pgfpathlineto{\pgfqpoint{2.597727in}{3.341530in}}%
\pgfpathlineto{\pgfqpoint{2.597727in}{3.213477in}}%
\pgfpathclose%
\pgfusepath{fill}%
\end{pgfscope}%
\begin{pgfscope}%
\pgfpathrectangle{\pgfqpoint{0.765000in}{0.660000in}}{\pgfqpoint{4.620000in}{4.620000in}}%
\pgfusepath{clip}%
\pgfsetbuttcap%
\pgfsetroundjoin%
\definecolor{currentfill}{rgb}{1.000000,0.894118,0.788235}%
\pgfsetfillcolor{currentfill}%
\pgfsetlinewidth{0.000000pt}%
\definecolor{currentstroke}{rgb}{1.000000,0.894118,0.788235}%
\pgfsetstrokecolor{currentstroke}%
\pgfsetdash{}{0pt}%
\pgfpathmoveto{\pgfqpoint{2.597727in}{3.213477in}}%
\pgfpathlineto{\pgfqpoint{2.708624in}{3.149451in}}%
\pgfpathlineto{\pgfqpoint{2.708624in}{3.277504in}}%
\pgfpathlineto{\pgfqpoint{2.597727in}{3.341530in}}%
\pgfpathlineto{\pgfqpoint{2.597727in}{3.213477in}}%
\pgfpathclose%
\pgfusepath{fill}%
\end{pgfscope}%
\begin{pgfscope}%
\pgfpathrectangle{\pgfqpoint{0.765000in}{0.660000in}}{\pgfqpoint{4.620000in}{4.620000in}}%
\pgfusepath{clip}%
\pgfsetbuttcap%
\pgfsetroundjoin%
\definecolor{currentfill}{rgb}{1.000000,0.894118,0.788235}%
\pgfsetfillcolor{currentfill}%
\pgfsetlinewidth{0.000000pt}%
\definecolor{currentstroke}{rgb}{1.000000,0.894118,0.788235}%
\pgfsetstrokecolor{currentstroke}%
\pgfsetdash{}{0pt}%
\pgfpathmoveto{\pgfqpoint{2.613910in}{3.381292in}}%
\pgfpathlineto{\pgfqpoint{2.503014in}{3.317266in}}%
\pgfpathlineto{\pgfqpoint{2.613910in}{3.253240in}}%
\pgfpathlineto{\pgfqpoint{2.724807in}{3.317266in}}%
\pgfpathlineto{\pgfqpoint{2.613910in}{3.381292in}}%
\pgfpathclose%
\pgfusepath{fill}%
\end{pgfscope}%
\begin{pgfscope}%
\pgfpathrectangle{\pgfqpoint{0.765000in}{0.660000in}}{\pgfqpoint{4.620000in}{4.620000in}}%
\pgfusepath{clip}%
\pgfsetbuttcap%
\pgfsetroundjoin%
\definecolor{currentfill}{rgb}{1.000000,0.894118,0.788235}%
\pgfsetfillcolor{currentfill}%
\pgfsetlinewidth{0.000000pt}%
\definecolor{currentstroke}{rgb}{1.000000,0.894118,0.788235}%
\pgfsetstrokecolor{currentstroke}%
\pgfsetdash{}{0pt}%
\pgfpathmoveto{\pgfqpoint{2.613910in}{3.125187in}}%
\pgfpathlineto{\pgfqpoint{2.724807in}{3.189213in}}%
\pgfpathlineto{\pgfqpoint{2.724807in}{3.317266in}}%
\pgfpathlineto{\pgfqpoint{2.613910in}{3.253240in}}%
\pgfpathlineto{\pgfqpoint{2.613910in}{3.125187in}}%
\pgfpathclose%
\pgfusepath{fill}%
\end{pgfscope}%
\begin{pgfscope}%
\pgfpathrectangle{\pgfqpoint{0.765000in}{0.660000in}}{\pgfqpoint{4.620000in}{4.620000in}}%
\pgfusepath{clip}%
\pgfsetbuttcap%
\pgfsetroundjoin%
\definecolor{currentfill}{rgb}{1.000000,0.894118,0.788235}%
\pgfsetfillcolor{currentfill}%
\pgfsetlinewidth{0.000000pt}%
\definecolor{currentstroke}{rgb}{1.000000,0.894118,0.788235}%
\pgfsetstrokecolor{currentstroke}%
\pgfsetdash{}{0pt}%
\pgfpathmoveto{\pgfqpoint{2.503014in}{3.189213in}}%
\pgfpathlineto{\pgfqpoint{2.613910in}{3.125187in}}%
\pgfpathlineto{\pgfqpoint{2.613910in}{3.253240in}}%
\pgfpathlineto{\pgfqpoint{2.503014in}{3.317266in}}%
\pgfpathlineto{\pgfqpoint{2.503014in}{3.189213in}}%
\pgfpathclose%
\pgfusepath{fill}%
\end{pgfscope}%
\begin{pgfscope}%
\pgfpathrectangle{\pgfqpoint{0.765000in}{0.660000in}}{\pgfqpoint{4.620000in}{4.620000in}}%
\pgfusepath{clip}%
\pgfsetbuttcap%
\pgfsetroundjoin%
\definecolor{currentfill}{rgb}{1.000000,0.894118,0.788235}%
\pgfsetfillcolor{currentfill}%
\pgfsetlinewidth{0.000000pt}%
\definecolor{currentstroke}{rgb}{1.000000,0.894118,0.788235}%
\pgfsetstrokecolor{currentstroke}%
\pgfsetdash{}{0pt}%
\pgfpathmoveto{\pgfqpoint{2.597727in}{3.341530in}}%
\pgfpathlineto{\pgfqpoint{2.486831in}{3.277504in}}%
\pgfpathlineto{\pgfqpoint{2.597727in}{3.213477in}}%
\pgfpathlineto{\pgfqpoint{2.708624in}{3.277504in}}%
\pgfpathlineto{\pgfqpoint{2.597727in}{3.341530in}}%
\pgfpathclose%
\pgfusepath{fill}%
\end{pgfscope}%
\begin{pgfscope}%
\pgfpathrectangle{\pgfqpoint{0.765000in}{0.660000in}}{\pgfqpoint{4.620000in}{4.620000in}}%
\pgfusepath{clip}%
\pgfsetbuttcap%
\pgfsetroundjoin%
\definecolor{currentfill}{rgb}{1.000000,0.894118,0.788235}%
\pgfsetfillcolor{currentfill}%
\pgfsetlinewidth{0.000000pt}%
\definecolor{currentstroke}{rgb}{1.000000,0.894118,0.788235}%
\pgfsetstrokecolor{currentstroke}%
\pgfsetdash{}{0pt}%
\pgfpathmoveto{\pgfqpoint{2.597727in}{3.085425in}}%
\pgfpathlineto{\pgfqpoint{2.708624in}{3.149451in}}%
\pgfpathlineto{\pgfqpoint{2.708624in}{3.277504in}}%
\pgfpathlineto{\pgfqpoint{2.597727in}{3.213477in}}%
\pgfpathlineto{\pgfqpoint{2.597727in}{3.085425in}}%
\pgfpathclose%
\pgfusepath{fill}%
\end{pgfscope}%
\begin{pgfscope}%
\pgfpathrectangle{\pgfqpoint{0.765000in}{0.660000in}}{\pgfqpoint{4.620000in}{4.620000in}}%
\pgfusepath{clip}%
\pgfsetbuttcap%
\pgfsetroundjoin%
\definecolor{currentfill}{rgb}{1.000000,0.894118,0.788235}%
\pgfsetfillcolor{currentfill}%
\pgfsetlinewidth{0.000000pt}%
\definecolor{currentstroke}{rgb}{1.000000,0.894118,0.788235}%
\pgfsetstrokecolor{currentstroke}%
\pgfsetdash{}{0pt}%
\pgfpathmoveto{\pgfqpoint{2.486831in}{3.149451in}}%
\pgfpathlineto{\pgfqpoint{2.597727in}{3.085425in}}%
\pgfpathlineto{\pgfqpoint{2.597727in}{3.213477in}}%
\pgfpathlineto{\pgfqpoint{2.486831in}{3.277504in}}%
\pgfpathlineto{\pgfqpoint{2.486831in}{3.149451in}}%
\pgfpathclose%
\pgfusepath{fill}%
\end{pgfscope}%
\begin{pgfscope}%
\pgfpathrectangle{\pgfqpoint{0.765000in}{0.660000in}}{\pgfqpoint{4.620000in}{4.620000in}}%
\pgfusepath{clip}%
\pgfsetbuttcap%
\pgfsetroundjoin%
\definecolor{currentfill}{rgb}{1.000000,0.894118,0.788235}%
\pgfsetfillcolor{currentfill}%
\pgfsetlinewidth{0.000000pt}%
\definecolor{currentstroke}{rgb}{1.000000,0.894118,0.788235}%
\pgfsetstrokecolor{currentstroke}%
\pgfsetdash{}{0pt}%
\pgfpathmoveto{\pgfqpoint{2.613910in}{3.253240in}}%
\pgfpathlineto{\pgfqpoint{2.503014in}{3.189213in}}%
\pgfpathlineto{\pgfqpoint{2.486831in}{3.149451in}}%
\pgfpathlineto{\pgfqpoint{2.597727in}{3.213477in}}%
\pgfpathlineto{\pgfqpoint{2.613910in}{3.253240in}}%
\pgfpathclose%
\pgfusepath{fill}%
\end{pgfscope}%
\begin{pgfscope}%
\pgfpathrectangle{\pgfqpoint{0.765000in}{0.660000in}}{\pgfqpoint{4.620000in}{4.620000in}}%
\pgfusepath{clip}%
\pgfsetbuttcap%
\pgfsetroundjoin%
\definecolor{currentfill}{rgb}{1.000000,0.894118,0.788235}%
\pgfsetfillcolor{currentfill}%
\pgfsetlinewidth{0.000000pt}%
\definecolor{currentstroke}{rgb}{1.000000,0.894118,0.788235}%
\pgfsetstrokecolor{currentstroke}%
\pgfsetdash{}{0pt}%
\pgfpathmoveto{\pgfqpoint{2.724807in}{3.189213in}}%
\pgfpathlineto{\pgfqpoint{2.613910in}{3.253240in}}%
\pgfpathlineto{\pgfqpoint{2.597727in}{3.213477in}}%
\pgfpathlineto{\pgfqpoint{2.708624in}{3.149451in}}%
\pgfpathlineto{\pgfqpoint{2.724807in}{3.189213in}}%
\pgfpathclose%
\pgfusepath{fill}%
\end{pgfscope}%
\begin{pgfscope}%
\pgfpathrectangle{\pgfqpoint{0.765000in}{0.660000in}}{\pgfqpoint{4.620000in}{4.620000in}}%
\pgfusepath{clip}%
\pgfsetbuttcap%
\pgfsetroundjoin%
\definecolor{currentfill}{rgb}{1.000000,0.894118,0.788235}%
\pgfsetfillcolor{currentfill}%
\pgfsetlinewidth{0.000000pt}%
\definecolor{currentstroke}{rgb}{1.000000,0.894118,0.788235}%
\pgfsetstrokecolor{currentstroke}%
\pgfsetdash{}{0pt}%
\pgfpathmoveto{\pgfqpoint{2.613910in}{3.253240in}}%
\pgfpathlineto{\pgfqpoint{2.613910in}{3.381292in}}%
\pgfpathlineto{\pgfqpoint{2.597727in}{3.341530in}}%
\pgfpathlineto{\pgfqpoint{2.708624in}{3.149451in}}%
\pgfpathlineto{\pgfqpoint{2.613910in}{3.253240in}}%
\pgfpathclose%
\pgfusepath{fill}%
\end{pgfscope}%
\begin{pgfscope}%
\pgfpathrectangle{\pgfqpoint{0.765000in}{0.660000in}}{\pgfqpoint{4.620000in}{4.620000in}}%
\pgfusepath{clip}%
\pgfsetbuttcap%
\pgfsetroundjoin%
\definecolor{currentfill}{rgb}{1.000000,0.894118,0.788235}%
\pgfsetfillcolor{currentfill}%
\pgfsetlinewidth{0.000000pt}%
\definecolor{currentstroke}{rgb}{1.000000,0.894118,0.788235}%
\pgfsetstrokecolor{currentstroke}%
\pgfsetdash{}{0pt}%
\pgfpathmoveto{\pgfqpoint{2.724807in}{3.189213in}}%
\pgfpathlineto{\pgfqpoint{2.724807in}{3.317266in}}%
\pgfpathlineto{\pgfqpoint{2.708624in}{3.277504in}}%
\pgfpathlineto{\pgfqpoint{2.597727in}{3.213477in}}%
\pgfpathlineto{\pgfqpoint{2.724807in}{3.189213in}}%
\pgfpathclose%
\pgfusepath{fill}%
\end{pgfscope}%
\begin{pgfscope}%
\pgfpathrectangle{\pgfqpoint{0.765000in}{0.660000in}}{\pgfqpoint{4.620000in}{4.620000in}}%
\pgfusepath{clip}%
\pgfsetbuttcap%
\pgfsetroundjoin%
\definecolor{currentfill}{rgb}{1.000000,0.894118,0.788235}%
\pgfsetfillcolor{currentfill}%
\pgfsetlinewidth{0.000000pt}%
\definecolor{currentstroke}{rgb}{1.000000,0.894118,0.788235}%
\pgfsetstrokecolor{currentstroke}%
\pgfsetdash{}{0pt}%
\pgfpathmoveto{\pgfqpoint{2.503014in}{3.189213in}}%
\pgfpathlineto{\pgfqpoint{2.613910in}{3.125187in}}%
\pgfpathlineto{\pgfqpoint{2.597727in}{3.085425in}}%
\pgfpathlineto{\pgfqpoint{2.486831in}{3.149451in}}%
\pgfpathlineto{\pgfqpoint{2.503014in}{3.189213in}}%
\pgfpathclose%
\pgfusepath{fill}%
\end{pgfscope}%
\begin{pgfscope}%
\pgfpathrectangle{\pgfqpoint{0.765000in}{0.660000in}}{\pgfqpoint{4.620000in}{4.620000in}}%
\pgfusepath{clip}%
\pgfsetbuttcap%
\pgfsetroundjoin%
\definecolor{currentfill}{rgb}{1.000000,0.894118,0.788235}%
\pgfsetfillcolor{currentfill}%
\pgfsetlinewidth{0.000000pt}%
\definecolor{currentstroke}{rgb}{1.000000,0.894118,0.788235}%
\pgfsetstrokecolor{currentstroke}%
\pgfsetdash{}{0pt}%
\pgfpathmoveto{\pgfqpoint{2.613910in}{3.125187in}}%
\pgfpathlineto{\pgfqpoint{2.724807in}{3.189213in}}%
\pgfpathlineto{\pgfqpoint{2.708624in}{3.149451in}}%
\pgfpathlineto{\pgfqpoint{2.597727in}{3.085425in}}%
\pgfpathlineto{\pgfqpoint{2.613910in}{3.125187in}}%
\pgfpathclose%
\pgfusepath{fill}%
\end{pgfscope}%
\begin{pgfscope}%
\pgfpathrectangle{\pgfqpoint{0.765000in}{0.660000in}}{\pgfqpoint{4.620000in}{4.620000in}}%
\pgfusepath{clip}%
\pgfsetbuttcap%
\pgfsetroundjoin%
\definecolor{currentfill}{rgb}{1.000000,0.894118,0.788235}%
\pgfsetfillcolor{currentfill}%
\pgfsetlinewidth{0.000000pt}%
\definecolor{currentstroke}{rgb}{1.000000,0.894118,0.788235}%
\pgfsetstrokecolor{currentstroke}%
\pgfsetdash{}{0pt}%
\pgfpathmoveto{\pgfqpoint{2.613910in}{3.381292in}}%
\pgfpathlineto{\pgfqpoint{2.503014in}{3.317266in}}%
\pgfpathlineto{\pgfqpoint{2.486831in}{3.277504in}}%
\pgfpathlineto{\pgfqpoint{2.597727in}{3.341530in}}%
\pgfpathlineto{\pgfqpoint{2.613910in}{3.381292in}}%
\pgfpathclose%
\pgfusepath{fill}%
\end{pgfscope}%
\begin{pgfscope}%
\pgfpathrectangle{\pgfqpoint{0.765000in}{0.660000in}}{\pgfqpoint{4.620000in}{4.620000in}}%
\pgfusepath{clip}%
\pgfsetbuttcap%
\pgfsetroundjoin%
\definecolor{currentfill}{rgb}{1.000000,0.894118,0.788235}%
\pgfsetfillcolor{currentfill}%
\pgfsetlinewidth{0.000000pt}%
\definecolor{currentstroke}{rgb}{1.000000,0.894118,0.788235}%
\pgfsetstrokecolor{currentstroke}%
\pgfsetdash{}{0pt}%
\pgfpathmoveto{\pgfqpoint{2.724807in}{3.317266in}}%
\pgfpathlineto{\pgfqpoint{2.613910in}{3.381292in}}%
\pgfpathlineto{\pgfqpoint{2.597727in}{3.341530in}}%
\pgfpathlineto{\pgfqpoint{2.708624in}{3.277504in}}%
\pgfpathlineto{\pgfqpoint{2.724807in}{3.317266in}}%
\pgfpathclose%
\pgfusepath{fill}%
\end{pgfscope}%
\begin{pgfscope}%
\pgfpathrectangle{\pgfqpoint{0.765000in}{0.660000in}}{\pgfqpoint{4.620000in}{4.620000in}}%
\pgfusepath{clip}%
\pgfsetbuttcap%
\pgfsetroundjoin%
\definecolor{currentfill}{rgb}{1.000000,0.894118,0.788235}%
\pgfsetfillcolor{currentfill}%
\pgfsetlinewidth{0.000000pt}%
\definecolor{currentstroke}{rgb}{1.000000,0.894118,0.788235}%
\pgfsetstrokecolor{currentstroke}%
\pgfsetdash{}{0pt}%
\pgfpathmoveto{\pgfqpoint{2.503014in}{3.189213in}}%
\pgfpathlineto{\pgfqpoint{2.503014in}{3.317266in}}%
\pgfpathlineto{\pgfqpoint{2.486831in}{3.277504in}}%
\pgfpathlineto{\pgfqpoint{2.597727in}{3.085425in}}%
\pgfpathlineto{\pgfqpoint{2.503014in}{3.189213in}}%
\pgfpathclose%
\pgfusepath{fill}%
\end{pgfscope}%
\begin{pgfscope}%
\pgfpathrectangle{\pgfqpoint{0.765000in}{0.660000in}}{\pgfqpoint{4.620000in}{4.620000in}}%
\pgfusepath{clip}%
\pgfsetbuttcap%
\pgfsetroundjoin%
\definecolor{currentfill}{rgb}{1.000000,0.894118,0.788235}%
\pgfsetfillcolor{currentfill}%
\pgfsetlinewidth{0.000000pt}%
\definecolor{currentstroke}{rgb}{1.000000,0.894118,0.788235}%
\pgfsetstrokecolor{currentstroke}%
\pgfsetdash{}{0pt}%
\pgfpathmoveto{\pgfqpoint{2.613910in}{3.125187in}}%
\pgfpathlineto{\pgfqpoint{2.613910in}{3.253240in}}%
\pgfpathlineto{\pgfqpoint{2.597727in}{3.213477in}}%
\pgfpathlineto{\pgfqpoint{2.486831in}{3.149451in}}%
\pgfpathlineto{\pgfqpoint{2.613910in}{3.125187in}}%
\pgfpathclose%
\pgfusepath{fill}%
\end{pgfscope}%
\begin{pgfscope}%
\pgfpathrectangle{\pgfqpoint{0.765000in}{0.660000in}}{\pgfqpoint{4.620000in}{4.620000in}}%
\pgfusepath{clip}%
\pgfsetbuttcap%
\pgfsetroundjoin%
\definecolor{currentfill}{rgb}{1.000000,0.894118,0.788235}%
\pgfsetfillcolor{currentfill}%
\pgfsetlinewidth{0.000000pt}%
\definecolor{currentstroke}{rgb}{1.000000,0.894118,0.788235}%
\pgfsetstrokecolor{currentstroke}%
\pgfsetdash{}{0pt}%
\pgfpathmoveto{\pgfqpoint{2.613910in}{3.253240in}}%
\pgfpathlineto{\pgfqpoint{2.724807in}{3.317266in}}%
\pgfpathlineto{\pgfqpoint{2.708624in}{3.277504in}}%
\pgfpathlineto{\pgfqpoint{2.597727in}{3.213477in}}%
\pgfpathlineto{\pgfqpoint{2.613910in}{3.253240in}}%
\pgfpathclose%
\pgfusepath{fill}%
\end{pgfscope}%
\begin{pgfscope}%
\pgfpathrectangle{\pgfqpoint{0.765000in}{0.660000in}}{\pgfqpoint{4.620000in}{4.620000in}}%
\pgfusepath{clip}%
\pgfsetbuttcap%
\pgfsetroundjoin%
\definecolor{currentfill}{rgb}{1.000000,0.894118,0.788235}%
\pgfsetfillcolor{currentfill}%
\pgfsetlinewidth{0.000000pt}%
\definecolor{currentstroke}{rgb}{1.000000,0.894118,0.788235}%
\pgfsetstrokecolor{currentstroke}%
\pgfsetdash{}{0pt}%
\pgfpathmoveto{\pgfqpoint{2.503014in}{3.317266in}}%
\pgfpathlineto{\pgfqpoint{2.613910in}{3.253240in}}%
\pgfpathlineto{\pgfqpoint{2.597727in}{3.213477in}}%
\pgfpathlineto{\pgfqpoint{2.486831in}{3.277504in}}%
\pgfpathlineto{\pgfqpoint{2.503014in}{3.317266in}}%
\pgfpathclose%
\pgfusepath{fill}%
\end{pgfscope}%
\begin{pgfscope}%
\pgfpathrectangle{\pgfqpoint{0.765000in}{0.660000in}}{\pgfqpoint{4.620000in}{4.620000in}}%
\pgfusepath{clip}%
\pgfsetbuttcap%
\pgfsetroundjoin%
\definecolor{currentfill}{rgb}{1.000000,0.894118,0.788235}%
\pgfsetfillcolor{currentfill}%
\pgfsetlinewidth{0.000000pt}%
\definecolor{currentstroke}{rgb}{1.000000,0.894118,0.788235}%
\pgfsetstrokecolor{currentstroke}%
\pgfsetdash{}{0pt}%
\pgfpathmoveto{\pgfqpoint{3.477043in}{2.674663in}}%
\pgfpathlineto{\pgfqpoint{3.366147in}{2.610637in}}%
\pgfpathlineto{\pgfqpoint{3.477043in}{2.546611in}}%
\pgfpathlineto{\pgfqpoint{3.587940in}{2.610637in}}%
\pgfpathlineto{\pgfqpoint{3.477043in}{2.674663in}}%
\pgfpathclose%
\pgfusepath{fill}%
\end{pgfscope}%
\begin{pgfscope}%
\pgfpathrectangle{\pgfqpoint{0.765000in}{0.660000in}}{\pgfqpoint{4.620000in}{4.620000in}}%
\pgfusepath{clip}%
\pgfsetbuttcap%
\pgfsetroundjoin%
\definecolor{currentfill}{rgb}{1.000000,0.894118,0.788235}%
\pgfsetfillcolor{currentfill}%
\pgfsetlinewidth{0.000000pt}%
\definecolor{currentstroke}{rgb}{1.000000,0.894118,0.788235}%
\pgfsetstrokecolor{currentstroke}%
\pgfsetdash{}{0pt}%
\pgfpathmoveto{\pgfqpoint{3.477043in}{2.674663in}}%
\pgfpathlineto{\pgfqpoint{3.366147in}{2.610637in}}%
\pgfpathlineto{\pgfqpoint{3.366147in}{2.738689in}}%
\pgfpathlineto{\pgfqpoint{3.477043in}{2.802716in}}%
\pgfpathlineto{\pgfqpoint{3.477043in}{2.674663in}}%
\pgfpathclose%
\pgfusepath{fill}%
\end{pgfscope}%
\begin{pgfscope}%
\pgfpathrectangle{\pgfqpoint{0.765000in}{0.660000in}}{\pgfqpoint{4.620000in}{4.620000in}}%
\pgfusepath{clip}%
\pgfsetbuttcap%
\pgfsetroundjoin%
\definecolor{currentfill}{rgb}{1.000000,0.894118,0.788235}%
\pgfsetfillcolor{currentfill}%
\pgfsetlinewidth{0.000000pt}%
\definecolor{currentstroke}{rgb}{1.000000,0.894118,0.788235}%
\pgfsetstrokecolor{currentstroke}%
\pgfsetdash{}{0pt}%
\pgfpathmoveto{\pgfqpoint{3.477043in}{2.674663in}}%
\pgfpathlineto{\pgfqpoint{3.587940in}{2.610637in}}%
\pgfpathlineto{\pgfqpoint{3.587940in}{2.738689in}}%
\pgfpathlineto{\pgfqpoint{3.477043in}{2.802716in}}%
\pgfpathlineto{\pgfqpoint{3.477043in}{2.674663in}}%
\pgfpathclose%
\pgfusepath{fill}%
\end{pgfscope}%
\begin{pgfscope}%
\pgfpathrectangle{\pgfqpoint{0.765000in}{0.660000in}}{\pgfqpoint{4.620000in}{4.620000in}}%
\pgfusepath{clip}%
\pgfsetbuttcap%
\pgfsetroundjoin%
\definecolor{currentfill}{rgb}{1.000000,0.894118,0.788235}%
\pgfsetfillcolor{currentfill}%
\pgfsetlinewidth{0.000000pt}%
\definecolor{currentstroke}{rgb}{1.000000,0.894118,0.788235}%
\pgfsetstrokecolor{currentstroke}%
\pgfsetdash{}{0pt}%
\pgfpathmoveto{\pgfqpoint{3.519640in}{2.712796in}}%
\pgfpathlineto{\pgfqpoint{3.408744in}{2.648770in}}%
\pgfpathlineto{\pgfqpoint{3.519640in}{2.584743in}}%
\pgfpathlineto{\pgfqpoint{3.630537in}{2.648770in}}%
\pgfpathlineto{\pgfqpoint{3.519640in}{2.712796in}}%
\pgfpathclose%
\pgfusepath{fill}%
\end{pgfscope}%
\begin{pgfscope}%
\pgfpathrectangle{\pgfqpoint{0.765000in}{0.660000in}}{\pgfqpoint{4.620000in}{4.620000in}}%
\pgfusepath{clip}%
\pgfsetbuttcap%
\pgfsetroundjoin%
\definecolor{currentfill}{rgb}{1.000000,0.894118,0.788235}%
\pgfsetfillcolor{currentfill}%
\pgfsetlinewidth{0.000000pt}%
\definecolor{currentstroke}{rgb}{1.000000,0.894118,0.788235}%
\pgfsetstrokecolor{currentstroke}%
\pgfsetdash{}{0pt}%
\pgfpathmoveto{\pgfqpoint{3.519640in}{2.712796in}}%
\pgfpathlineto{\pgfqpoint{3.408744in}{2.648770in}}%
\pgfpathlineto{\pgfqpoint{3.408744in}{2.776822in}}%
\pgfpathlineto{\pgfqpoint{3.519640in}{2.840848in}}%
\pgfpathlineto{\pgfqpoint{3.519640in}{2.712796in}}%
\pgfpathclose%
\pgfusepath{fill}%
\end{pgfscope}%
\begin{pgfscope}%
\pgfpathrectangle{\pgfqpoint{0.765000in}{0.660000in}}{\pgfqpoint{4.620000in}{4.620000in}}%
\pgfusepath{clip}%
\pgfsetbuttcap%
\pgfsetroundjoin%
\definecolor{currentfill}{rgb}{1.000000,0.894118,0.788235}%
\pgfsetfillcolor{currentfill}%
\pgfsetlinewidth{0.000000pt}%
\definecolor{currentstroke}{rgb}{1.000000,0.894118,0.788235}%
\pgfsetstrokecolor{currentstroke}%
\pgfsetdash{}{0pt}%
\pgfpathmoveto{\pgfqpoint{3.519640in}{2.712796in}}%
\pgfpathlineto{\pgfqpoint{3.630537in}{2.648770in}}%
\pgfpathlineto{\pgfqpoint{3.630537in}{2.776822in}}%
\pgfpathlineto{\pgfqpoint{3.519640in}{2.840848in}}%
\pgfpathlineto{\pgfqpoint{3.519640in}{2.712796in}}%
\pgfpathclose%
\pgfusepath{fill}%
\end{pgfscope}%
\begin{pgfscope}%
\pgfpathrectangle{\pgfqpoint{0.765000in}{0.660000in}}{\pgfqpoint{4.620000in}{4.620000in}}%
\pgfusepath{clip}%
\pgfsetbuttcap%
\pgfsetroundjoin%
\definecolor{currentfill}{rgb}{1.000000,0.894118,0.788235}%
\pgfsetfillcolor{currentfill}%
\pgfsetlinewidth{0.000000pt}%
\definecolor{currentstroke}{rgb}{1.000000,0.894118,0.788235}%
\pgfsetstrokecolor{currentstroke}%
\pgfsetdash{}{0pt}%
\pgfpathmoveto{\pgfqpoint{3.477043in}{2.802716in}}%
\pgfpathlineto{\pgfqpoint{3.366147in}{2.738689in}}%
\pgfpathlineto{\pgfqpoint{3.477043in}{2.674663in}}%
\pgfpathlineto{\pgfqpoint{3.587940in}{2.738689in}}%
\pgfpathlineto{\pgfqpoint{3.477043in}{2.802716in}}%
\pgfpathclose%
\pgfusepath{fill}%
\end{pgfscope}%
\begin{pgfscope}%
\pgfpathrectangle{\pgfqpoint{0.765000in}{0.660000in}}{\pgfqpoint{4.620000in}{4.620000in}}%
\pgfusepath{clip}%
\pgfsetbuttcap%
\pgfsetroundjoin%
\definecolor{currentfill}{rgb}{1.000000,0.894118,0.788235}%
\pgfsetfillcolor{currentfill}%
\pgfsetlinewidth{0.000000pt}%
\definecolor{currentstroke}{rgb}{1.000000,0.894118,0.788235}%
\pgfsetstrokecolor{currentstroke}%
\pgfsetdash{}{0pt}%
\pgfpathmoveto{\pgfqpoint{3.477043in}{2.546611in}}%
\pgfpathlineto{\pgfqpoint{3.587940in}{2.610637in}}%
\pgfpathlineto{\pgfqpoint{3.587940in}{2.738689in}}%
\pgfpathlineto{\pgfqpoint{3.477043in}{2.674663in}}%
\pgfpathlineto{\pgfqpoint{3.477043in}{2.546611in}}%
\pgfpathclose%
\pgfusepath{fill}%
\end{pgfscope}%
\begin{pgfscope}%
\pgfpathrectangle{\pgfqpoint{0.765000in}{0.660000in}}{\pgfqpoint{4.620000in}{4.620000in}}%
\pgfusepath{clip}%
\pgfsetbuttcap%
\pgfsetroundjoin%
\definecolor{currentfill}{rgb}{1.000000,0.894118,0.788235}%
\pgfsetfillcolor{currentfill}%
\pgfsetlinewidth{0.000000pt}%
\definecolor{currentstroke}{rgb}{1.000000,0.894118,0.788235}%
\pgfsetstrokecolor{currentstroke}%
\pgfsetdash{}{0pt}%
\pgfpathmoveto{\pgfqpoint{3.366147in}{2.610637in}}%
\pgfpathlineto{\pgfqpoint{3.477043in}{2.546611in}}%
\pgfpathlineto{\pgfqpoint{3.477043in}{2.674663in}}%
\pgfpathlineto{\pgfqpoint{3.366147in}{2.738689in}}%
\pgfpathlineto{\pgfqpoint{3.366147in}{2.610637in}}%
\pgfpathclose%
\pgfusepath{fill}%
\end{pgfscope}%
\begin{pgfscope}%
\pgfpathrectangle{\pgfqpoint{0.765000in}{0.660000in}}{\pgfqpoint{4.620000in}{4.620000in}}%
\pgfusepath{clip}%
\pgfsetbuttcap%
\pgfsetroundjoin%
\definecolor{currentfill}{rgb}{1.000000,0.894118,0.788235}%
\pgfsetfillcolor{currentfill}%
\pgfsetlinewidth{0.000000pt}%
\definecolor{currentstroke}{rgb}{1.000000,0.894118,0.788235}%
\pgfsetstrokecolor{currentstroke}%
\pgfsetdash{}{0pt}%
\pgfpathmoveto{\pgfqpoint{3.519640in}{2.840848in}}%
\pgfpathlineto{\pgfqpoint{3.408744in}{2.776822in}}%
\pgfpathlineto{\pgfqpoint{3.519640in}{2.712796in}}%
\pgfpathlineto{\pgfqpoint{3.630537in}{2.776822in}}%
\pgfpathlineto{\pgfqpoint{3.519640in}{2.840848in}}%
\pgfpathclose%
\pgfusepath{fill}%
\end{pgfscope}%
\begin{pgfscope}%
\pgfpathrectangle{\pgfqpoint{0.765000in}{0.660000in}}{\pgfqpoint{4.620000in}{4.620000in}}%
\pgfusepath{clip}%
\pgfsetbuttcap%
\pgfsetroundjoin%
\definecolor{currentfill}{rgb}{1.000000,0.894118,0.788235}%
\pgfsetfillcolor{currentfill}%
\pgfsetlinewidth{0.000000pt}%
\definecolor{currentstroke}{rgb}{1.000000,0.894118,0.788235}%
\pgfsetstrokecolor{currentstroke}%
\pgfsetdash{}{0pt}%
\pgfpathmoveto{\pgfqpoint{3.519640in}{2.584743in}}%
\pgfpathlineto{\pgfqpoint{3.630537in}{2.648770in}}%
\pgfpathlineto{\pgfqpoint{3.630537in}{2.776822in}}%
\pgfpathlineto{\pgfqpoint{3.519640in}{2.712796in}}%
\pgfpathlineto{\pgfqpoint{3.519640in}{2.584743in}}%
\pgfpathclose%
\pgfusepath{fill}%
\end{pgfscope}%
\begin{pgfscope}%
\pgfpathrectangle{\pgfqpoint{0.765000in}{0.660000in}}{\pgfqpoint{4.620000in}{4.620000in}}%
\pgfusepath{clip}%
\pgfsetbuttcap%
\pgfsetroundjoin%
\definecolor{currentfill}{rgb}{1.000000,0.894118,0.788235}%
\pgfsetfillcolor{currentfill}%
\pgfsetlinewidth{0.000000pt}%
\definecolor{currentstroke}{rgb}{1.000000,0.894118,0.788235}%
\pgfsetstrokecolor{currentstroke}%
\pgfsetdash{}{0pt}%
\pgfpathmoveto{\pgfqpoint{3.408744in}{2.648770in}}%
\pgfpathlineto{\pgfqpoint{3.519640in}{2.584743in}}%
\pgfpathlineto{\pgfqpoint{3.519640in}{2.712796in}}%
\pgfpathlineto{\pgfqpoint{3.408744in}{2.776822in}}%
\pgfpathlineto{\pgfqpoint{3.408744in}{2.648770in}}%
\pgfpathclose%
\pgfusepath{fill}%
\end{pgfscope}%
\begin{pgfscope}%
\pgfpathrectangle{\pgfqpoint{0.765000in}{0.660000in}}{\pgfqpoint{4.620000in}{4.620000in}}%
\pgfusepath{clip}%
\pgfsetbuttcap%
\pgfsetroundjoin%
\definecolor{currentfill}{rgb}{1.000000,0.894118,0.788235}%
\pgfsetfillcolor{currentfill}%
\pgfsetlinewidth{0.000000pt}%
\definecolor{currentstroke}{rgb}{1.000000,0.894118,0.788235}%
\pgfsetstrokecolor{currentstroke}%
\pgfsetdash{}{0pt}%
\pgfpathmoveto{\pgfqpoint{3.477043in}{2.674663in}}%
\pgfpathlineto{\pgfqpoint{3.366147in}{2.610637in}}%
\pgfpathlineto{\pgfqpoint{3.408744in}{2.648770in}}%
\pgfpathlineto{\pgfqpoint{3.519640in}{2.712796in}}%
\pgfpathlineto{\pgfqpoint{3.477043in}{2.674663in}}%
\pgfpathclose%
\pgfusepath{fill}%
\end{pgfscope}%
\begin{pgfscope}%
\pgfpathrectangle{\pgfqpoint{0.765000in}{0.660000in}}{\pgfqpoint{4.620000in}{4.620000in}}%
\pgfusepath{clip}%
\pgfsetbuttcap%
\pgfsetroundjoin%
\definecolor{currentfill}{rgb}{1.000000,0.894118,0.788235}%
\pgfsetfillcolor{currentfill}%
\pgfsetlinewidth{0.000000pt}%
\definecolor{currentstroke}{rgb}{1.000000,0.894118,0.788235}%
\pgfsetstrokecolor{currentstroke}%
\pgfsetdash{}{0pt}%
\pgfpathmoveto{\pgfqpoint{3.587940in}{2.610637in}}%
\pgfpathlineto{\pgfqpoint{3.477043in}{2.674663in}}%
\pgfpathlineto{\pgfqpoint{3.519640in}{2.712796in}}%
\pgfpathlineto{\pgfqpoint{3.630537in}{2.648770in}}%
\pgfpathlineto{\pgfqpoint{3.587940in}{2.610637in}}%
\pgfpathclose%
\pgfusepath{fill}%
\end{pgfscope}%
\begin{pgfscope}%
\pgfpathrectangle{\pgfqpoint{0.765000in}{0.660000in}}{\pgfqpoint{4.620000in}{4.620000in}}%
\pgfusepath{clip}%
\pgfsetbuttcap%
\pgfsetroundjoin%
\definecolor{currentfill}{rgb}{1.000000,0.894118,0.788235}%
\pgfsetfillcolor{currentfill}%
\pgfsetlinewidth{0.000000pt}%
\definecolor{currentstroke}{rgb}{1.000000,0.894118,0.788235}%
\pgfsetstrokecolor{currentstroke}%
\pgfsetdash{}{0pt}%
\pgfpathmoveto{\pgfqpoint{3.477043in}{2.674663in}}%
\pgfpathlineto{\pgfqpoint{3.477043in}{2.802716in}}%
\pgfpathlineto{\pgfqpoint{3.519640in}{2.840848in}}%
\pgfpathlineto{\pgfqpoint{3.630537in}{2.648770in}}%
\pgfpathlineto{\pgfqpoint{3.477043in}{2.674663in}}%
\pgfpathclose%
\pgfusepath{fill}%
\end{pgfscope}%
\begin{pgfscope}%
\pgfpathrectangle{\pgfqpoint{0.765000in}{0.660000in}}{\pgfqpoint{4.620000in}{4.620000in}}%
\pgfusepath{clip}%
\pgfsetbuttcap%
\pgfsetroundjoin%
\definecolor{currentfill}{rgb}{1.000000,0.894118,0.788235}%
\pgfsetfillcolor{currentfill}%
\pgfsetlinewidth{0.000000pt}%
\definecolor{currentstroke}{rgb}{1.000000,0.894118,0.788235}%
\pgfsetstrokecolor{currentstroke}%
\pgfsetdash{}{0pt}%
\pgfpathmoveto{\pgfqpoint{3.587940in}{2.610637in}}%
\pgfpathlineto{\pgfqpoint{3.587940in}{2.738689in}}%
\pgfpathlineto{\pgfqpoint{3.630537in}{2.776822in}}%
\pgfpathlineto{\pgfqpoint{3.519640in}{2.712796in}}%
\pgfpathlineto{\pgfqpoint{3.587940in}{2.610637in}}%
\pgfpathclose%
\pgfusepath{fill}%
\end{pgfscope}%
\begin{pgfscope}%
\pgfpathrectangle{\pgfqpoint{0.765000in}{0.660000in}}{\pgfqpoint{4.620000in}{4.620000in}}%
\pgfusepath{clip}%
\pgfsetbuttcap%
\pgfsetroundjoin%
\definecolor{currentfill}{rgb}{1.000000,0.894118,0.788235}%
\pgfsetfillcolor{currentfill}%
\pgfsetlinewidth{0.000000pt}%
\definecolor{currentstroke}{rgb}{1.000000,0.894118,0.788235}%
\pgfsetstrokecolor{currentstroke}%
\pgfsetdash{}{0pt}%
\pgfpathmoveto{\pgfqpoint{3.366147in}{2.610637in}}%
\pgfpathlineto{\pgfqpoint{3.477043in}{2.546611in}}%
\pgfpathlineto{\pgfqpoint{3.519640in}{2.584743in}}%
\pgfpathlineto{\pgfqpoint{3.408744in}{2.648770in}}%
\pgfpathlineto{\pgfqpoint{3.366147in}{2.610637in}}%
\pgfpathclose%
\pgfusepath{fill}%
\end{pgfscope}%
\begin{pgfscope}%
\pgfpathrectangle{\pgfqpoint{0.765000in}{0.660000in}}{\pgfqpoint{4.620000in}{4.620000in}}%
\pgfusepath{clip}%
\pgfsetbuttcap%
\pgfsetroundjoin%
\definecolor{currentfill}{rgb}{1.000000,0.894118,0.788235}%
\pgfsetfillcolor{currentfill}%
\pgfsetlinewidth{0.000000pt}%
\definecolor{currentstroke}{rgb}{1.000000,0.894118,0.788235}%
\pgfsetstrokecolor{currentstroke}%
\pgfsetdash{}{0pt}%
\pgfpathmoveto{\pgfqpoint{3.477043in}{2.546611in}}%
\pgfpathlineto{\pgfqpoint{3.587940in}{2.610637in}}%
\pgfpathlineto{\pgfqpoint{3.630537in}{2.648770in}}%
\pgfpathlineto{\pgfqpoint{3.519640in}{2.584743in}}%
\pgfpathlineto{\pgfqpoint{3.477043in}{2.546611in}}%
\pgfpathclose%
\pgfusepath{fill}%
\end{pgfscope}%
\begin{pgfscope}%
\pgfpathrectangle{\pgfqpoint{0.765000in}{0.660000in}}{\pgfqpoint{4.620000in}{4.620000in}}%
\pgfusepath{clip}%
\pgfsetbuttcap%
\pgfsetroundjoin%
\definecolor{currentfill}{rgb}{1.000000,0.894118,0.788235}%
\pgfsetfillcolor{currentfill}%
\pgfsetlinewidth{0.000000pt}%
\definecolor{currentstroke}{rgb}{1.000000,0.894118,0.788235}%
\pgfsetstrokecolor{currentstroke}%
\pgfsetdash{}{0pt}%
\pgfpathmoveto{\pgfqpoint{3.477043in}{2.802716in}}%
\pgfpathlineto{\pgfqpoint{3.366147in}{2.738689in}}%
\pgfpathlineto{\pgfqpoint{3.408744in}{2.776822in}}%
\pgfpathlineto{\pgfqpoint{3.519640in}{2.840848in}}%
\pgfpathlineto{\pgfqpoint{3.477043in}{2.802716in}}%
\pgfpathclose%
\pgfusepath{fill}%
\end{pgfscope}%
\begin{pgfscope}%
\pgfpathrectangle{\pgfqpoint{0.765000in}{0.660000in}}{\pgfqpoint{4.620000in}{4.620000in}}%
\pgfusepath{clip}%
\pgfsetbuttcap%
\pgfsetroundjoin%
\definecolor{currentfill}{rgb}{1.000000,0.894118,0.788235}%
\pgfsetfillcolor{currentfill}%
\pgfsetlinewidth{0.000000pt}%
\definecolor{currentstroke}{rgb}{1.000000,0.894118,0.788235}%
\pgfsetstrokecolor{currentstroke}%
\pgfsetdash{}{0pt}%
\pgfpathmoveto{\pgfqpoint{3.587940in}{2.738689in}}%
\pgfpathlineto{\pgfqpoint{3.477043in}{2.802716in}}%
\pgfpathlineto{\pgfqpoint{3.519640in}{2.840848in}}%
\pgfpathlineto{\pgfqpoint{3.630537in}{2.776822in}}%
\pgfpathlineto{\pgfqpoint{3.587940in}{2.738689in}}%
\pgfpathclose%
\pgfusepath{fill}%
\end{pgfscope}%
\begin{pgfscope}%
\pgfpathrectangle{\pgfqpoint{0.765000in}{0.660000in}}{\pgfqpoint{4.620000in}{4.620000in}}%
\pgfusepath{clip}%
\pgfsetbuttcap%
\pgfsetroundjoin%
\definecolor{currentfill}{rgb}{1.000000,0.894118,0.788235}%
\pgfsetfillcolor{currentfill}%
\pgfsetlinewidth{0.000000pt}%
\definecolor{currentstroke}{rgb}{1.000000,0.894118,0.788235}%
\pgfsetstrokecolor{currentstroke}%
\pgfsetdash{}{0pt}%
\pgfpathmoveto{\pgfqpoint{3.366147in}{2.610637in}}%
\pgfpathlineto{\pgfqpoint{3.366147in}{2.738689in}}%
\pgfpathlineto{\pgfqpoint{3.408744in}{2.776822in}}%
\pgfpathlineto{\pgfqpoint{3.519640in}{2.584743in}}%
\pgfpathlineto{\pgfqpoint{3.366147in}{2.610637in}}%
\pgfpathclose%
\pgfusepath{fill}%
\end{pgfscope}%
\begin{pgfscope}%
\pgfpathrectangle{\pgfqpoint{0.765000in}{0.660000in}}{\pgfqpoint{4.620000in}{4.620000in}}%
\pgfusepath{clip}%
\pgfsetbuttcap%
\pgfsetroundjoin%
\definecolor{currentfill}{rgb}{1.000000,0.894118,0.788235}%
\pgfsetfillcolor{currentfill}%
\pgfsetlinewidth{0.000000pt}%
\definecolor{currentstroke}{rgb}{1.000000,0.894118,0.788235}%
\pgfsetstrokecolor{currentstroke}%
\pgfsetdash{}{0pt}%
\pgfpathmoveto{\pgfqpoint{3.477043in}{2.546611in}}%
\pgfpathlineto{\pgfqpoint{3.477043in}{2.674663in}}%
\pgfpathlineto{\pgfqpoint{3.519640in}{2.712796in}}%
\pgfpathlineto{\pgfqpoint{3.408744in}{2.648770in}}%
\pgfpathlineto{\pgfqpoint{3.477043in}{2.546611in}}%
\pgfpathclose%
\pgfusepath{fill}%
\end{pgfscope}%
\begin{pgfscope}%
\pgfpathrectangle{\pgfqpoint{0.765000in}{0.660000in}}{\pgfqpoint{4.620000in}{4.620000in}}%
\pgfusepath{clip}%
\pgfsetbuttcap%
\pgfsetroundjoin%
\definecolor{currentfill}{rgb}{1.000000,0.894118,0.788235}%
\pgfsetfillcolor{currentfill}%
\pgfsetlinewidth{0.000000pt}%
\definecolor{currentstroke}{rgb}{1.000000,0.894118,0.788235}%
\pgfsetstrokecolor{currentstroke}%
\pgfsetdash{}{0pt}%
\pgfpathmoveto{\pgfqpoint{3.477043in}{2.674663in}}%
\pgfpathlineto{\pgfqpoint{3.587940in}{2.738689in}}%
\pgfpathlineto{\pgfqpoint{3.630537in}{2.776822in}}%
\pgfpathlineto{\pgfqpoint{3.519640in}{2.712796in}}%
\pgfpathlineto{\pgfqpoint{3.477043in}{2.674663in}}%
\pgfpathclose%
\pgfusepath{fill}%
\end{pgfscope}%
\begin{pgfscope}%
\pgfpathrectangle{\pgfqpoint{0.765000in}{0.660000in}}{\pgfqpoint{4.620000in}{4.620000in}}%
\pgfusepath{clip}%
\pgfsetbuttcap%
\pgfsetroundjoin%
\definecolor{currentfill}{rgb}{1.000000,0.894118,0.788235}%
\pgfsetfillcolor{currentfill}%
\pgfsetlinewidth{0.000000pt}%
\definecolor{currentstroke}{rgb}{1.000000,0.894118,0.788235}%
\pgfsetstrokecolor{currentstroke}%
\pgfsetdash{}{0pt}%
\pgfpathmoveto{\pgfqpoint{3.366147in}{2.738689in}}%
\pgfpathlineto{\pgfqpoint{3.477043in}{2.674663in}}%
\pgfpathlineto{\pgfqpoint{3.519640in}{2.712796in}}%
\pgfpathlineto{\pgfqpoint{3.408744in}{2.776822in}}%
\pgfpathlineto{\pgfqpoint{3.366147in}{2.738689in}}%
\pgfpathclose%
\pgfusepath{fill}%
\end{pgfscope}%
\begin{pgfscope}%
\pgfpathrectangle{\pgfqpoint{0.765000in}{0.660000in}}{\pgfqpoint{4.620000in}{4.620000in}}%
\pgfusepath{clip}%
\pgfsetbuttcap%
\pgfsetroundjoin%
\definecolor{currentfill}{rgb}{1.000000,0.894118,0.788235}%
\pgfsetfillcolor{currentfill}%
\pgfsetlinewidth{0.000000pt}%
\definecolor{currentstroke}{rgb}{1.000000,0.894118,0.788235}%
\pgfsetstrokecolor{currentstroke}%
\pgfsetdash{}{0pt}%
\pgfpathmoveto{\pgfqpoint{3.477043in}{2.674663in}}%
\pgfpathlineto{\pgfqpoint{3.366147in}{2.610637in}}%
\pgfpathlineto{\pgfqpoint{3.477043in}{2.546611in}}%
\pgfpathlineto{\pgfqpoint{3.587940in}{2.610637in}}%
\pgfpathlineto{\pgfqpoint{3.477043in}{2.674663in}}%
\pgfpathclose%
\pgfusepath{fill}%
\end{pgfscope}%
\begin{pgfscope}%
\pgfpathrectangle{\pgfqpoint{0.765000in}{0.660000in}}{\pgfqpoint{4.620000in}{4.620000in}}%
\pgfusepath{clip}%
\pgfsetbuttcap%
\pgfsetroundjoin%
\definecolor{currentfill}{rgb}{1.000000,0.894118,0.788235}%
\pgfsetfillcolor{currentfill}%
\pgfsetlinewidth{0.000000pt}%
\definecolor{currentstroke}{rgb}{1.000000,0.894118,0.788235}%
\pgfsetstrokecolor{currentstroke}%
\pgfsetdash{}{0pt}%
\pgfpathmoveto{\pgfqpoint{3.477043in}{2.674663in}}%
\pgfpathlineto{\pgfqpoint{3.366147in}{2.610637in}}%
\pgfpathlineto{\pgfqpoint{3.366147in}{2.738689in}}%
\pgfpathlineto{\pgfqpoint{3.477043in}{2.802716in}}%
\pgfpathlineto{\pgfqpoint{3.477043in}{2.674663in}}%
\pgfpathclose%
\pgfusepath{fill}%
\end{pgfscope}%
\begin{pgfscope}%
\pgfpathrectangle{\pgfqpoint{0.765000in}{0.660000in}}{\pgfqpoint{4.620000in}{4.620000in}}%
\pgfusepath{clip}%
\pgfsetbuttcap%
\pgfsetroundjoin%
\definecolor{currentfill}{rgb}{1.000000,0.894118,0.788235}%
\pgfsetfillcolor{currentfill}%
\pgfsetlinewidth{0.000000pt}%
\definecolor{currentstroke}{rgb}{1.000000,0.894118,0.788235}%
\pgfsetstrokecolor{currentstroke}%
\pgfsetdash{}{0pt}%
\pgfpathmoveto{\pgfqpoint{3.477043in}{2.674663in}}%
\pgfpathlineto{\pgfqpoint{3.587940in}{2.610637in}}%
\pgfpathlineto{\pgfqpoint{3.587940in}{2.738689in}}%
\pgfpathlineto{\pgfqpoint{3.477043in}{2.802716in}}%
\pgfpathlineto{\pgfqpoint{3.477043in}{2.674663in}}%
\pgfpathclose%
\pgfusepath{fill}%
\end{pgfscope}%
\begin{pgfscope}%
\pgfpathrectangle{\pgfqpoint{0.765000in}{0.660000in}}{\pgfqpoint{4.620000in}{4.620000in}}%
\pgfusepath{clip}%
\pgfsetbuttcap%
\pgfsetroundjoin%
\definecolor{currentfill}{rgb}{1.000000,0.894118,0.788235}%
\pgfsetfillcolor{currentfill}%
\pgfsetlinewidth{0.000000pt}%
\definecolor{currentstroke}{rgb}{1.000000,0.894118,0.788235}%
\pgfsetstrokecolor{currentstroke}%
\pgfsetdash{}{0pt}%
\pgfpathmoveto{\pgfqpoint{3.463186in}{2.662942in}}%
\pgfpathlineto{\pgfqpoint{3.352289in}{2.598916in}}%
\pgfpathlineto{\pgfqpoint{3.463186in}{2.534890in}}%
\pgfpathlineto{\pgfqpoint{3.574082in}{2.598916in}}%
\pgfpathlineto{\pgfqpoint{3.463186in}{2.662942in}}%
\pgfpathclose%
\pgfusepath{fill}%
\end{pgfscope}%
\begin{pgfscope}%
\pgfpathrectangle{\pgfqpoint{0.765000in}{0.660000in}}{\pgfqpoint{4.620000in}{4.620000in}}%
\pgfusepath{clip}%
\pgfsetbuttcap%
\pgfsetroundjoin%
\definecolor{currentfill}{rgb}{1.000000,0.894118,0.788235}%
\pgfsetfillcolor{currentfill}%
\pgfsetlinewidth{0.000000pt}%
\definecolor{currentstroke}{rgb}{1.000000,0.894118,0.788235}%
\pgfsetstrokecolor{currentstroke}%
\pgfsetdash{}{0pt}%
\pgfpathmoveto{\pgfqpoint{3.463186in}{2.662942in}}%
\pgfpathlineto{\pgfqpoint{3.352289in}{2.598916in}}%
\pgfpathlineto{\pgfqpoint{3.352289in}{2.726968in}}%
\pgfpathlineto{\pgfqpoint{3.463186in}{2.790995in}}%
\pgfpathlineto{\pgfqpoint{3.463186in}{2.662942in}}%
\pgfpathclose%
\pgfusepath{fill}%
\end{pgfscope}%
\begin{pgfscope}%
\pgfpathrectangle{\pgfqpoint{0.765000in}{0.660000in}}{\pgfqpoint{4.620000in}{4.620000in}}%
\pgfusepath{clip}%
\pgfsetbuttcap%
\pgfsetroundjoin%
\definecolor{currentfill}{rgb}{1.000000,0.894118,0.788235}%
\pgfsetfillcolor{currentfill}%
\pgfsetlinewidth{0.000000pt}%
\definecolor{currentstroke}{rgb}{1.000000,0.894118,0.788235}%
\pgfsetstrokecolor{currentstroke}%
\pgfsetdash{}{0pt}%
\pgfpathmoveto{\pgfqpoint{3.463186in}{2.662942in}}%
\pgfpathlineto{\pgfqpoint{3.574082in}{2.598916in}}%
\pgfpathlineto{\pgfqpoint{3.574082in}{2.726968in}}%
\pgfpathlineto{\pgfqpoint{3.463186in}{2.790995in}}%
\pgfpathlineto{\pgfqpoint{3.463186in}{2.662942in}}%
\pgfpathclose%
\pgfusepath{fill}%
\end{pgfscope}%
\begin{pgfscope}%
\pgfpathrectangle{\pgfqpoint{0.765000in}{0.660000in}}{\pgfqpoint{4.620000in}{4.620000in}}%
\pgfusepath{clip}%
\pgfsetbuttcap%
\pgfsetroundjoin%
\definecolor{currentfill}{rgb}{1.000000,0.894118,0.788235}%
\pgfsetfillcolor{currentfill}%
\pgfsetlinewidth{0.000000pt}%
\definecolor{currentstroke}{rgb}{1.000000,0.894118,0.788235}%
\pgfsetstrokecolor{currentstroke}%
\pgfsetdash{}{0pt}%
\pgfpathmoveto{\pgfqpoint{3.477043in}{2.802716in}}%
\pgfpathlineto{\pgfqpoint{3.366147in}{2.738689in}}%
\pgfpathlineto{\pgfqpoint{3.477043in}{2.674663in}}%
\pgfpathlineto{\pgfqpoint{3.587940in}{2.738689in}}%
\pgfpathlineto{\pgfqpoint{3.477043in}{2.802716in}}%
\pgfpathclose%
\pgfusepath{fill}%
\end{pgfscope}%
\begin{pgfscope}%
\pgfpathrectangle{\pgfqpoint{0.765000in}{0.660000in}}{\pgfqpoint{4.620000in}{4.620000in}}%
\pgfusepath{clip}%
\pgfsetbuttcap%
\pgfsetroundjoin%
\definecolor{currentfill}{rgb}{1.000000,0.894118,0.788235}%
\pgfsetfillcolor{currentfill}%
\pgfsetlinewidth{0.000000pt}%
\definecolor{currentstroke}{rgb}{1.000000,0.894118,0.788235}%
\pgfsetstrokecolor{currentstroke}%
\pgfsetdash{}{0pt}%
\pgfpathmoveto{\pgfqpoint{3.477043in}{2.546611in}}%
\pgfpathlineto{\pgfqpoint{3.587940in}{2.610637in}}%
\pgfpathlineto{\pgfqpoint{3.587940in}{2.738689in}}%
\pgfpathlineto{\pgfqpoint{3.477043in}{2.674663in}}%
\pgfpathlineto{\pgfqpoint{3.477043in}{2.546611in}}%
\pgfpathclose%
\pgfusepath{fill}%
\end{pgfscope}%
\begin{pgfscope}%
\pgfpathrectangle{\pgfqpoint{0.765000in}{0.660000in}}{\pgfqpoint{4.620000in}{4.620000in}}%
\pgfusepath{clip}%
\pgfsetbuttcap%
\pgfsetroundjoin%
\definecolor{currentfill}{rgb}{1.000000,0.894118,0.788235}%
\pgfsetfillcolor{currentfill}%
\pgfsetlinewidth{0.000000pt}%
\definecolor{currentstroke}{rgb}{1.000000,0.894118,0.788235}%
\pgfsetstrokecolor{currentstroke}%
\pgfsetdash{}{0pt}%
\pgfpathmoveto{\pgfqpoint{3.366147in}{2.610637in}}%
\pgfpathlineto{\pgfqpoint{3.477043in}{2.546611in}}%
\pgfpathlineto{\pgfqpoint{3.477043in}{2.674663in}}%
\pgfpathlineto{\pgfqpoint{3.366147in}{2.738689in}}%
\pgfpathlineto{\pgfqpoint{3.366147in}{2.610637in}}%
\pgfpathclose%
\pgfusepath{fill}%
\end{pgfscope}%
\begin{pgfscope}%
\pgfpathrectangle{\pgfqpoint{0.765000in}{0.660000in}}{\pgfqpoint{4.620000in}{4.620000in}}%
\pgfusepath{clip}%
\pgfsetbuttcap%
\pgfsetroundjoin%
\definecolor{currentfill}{rgb}{1.000000,0.894118,0.788235}%
\pgfsetfillcolor{currentfill}%
\pgfsetlinewidth{0.000000pt}%
\definecolor{currentstroke}{rgb}{1.000000,0.894118,0.788235}%
\pgfsetstrokecolor{currentstroke}%
\pgfsetdash{}{0pt}%
\pgfpathmoveto{\pgfqpoint{3.463186in}{2.790995in}}%
\pgfpathlineto{\pgfqpoint{3.352289in}{2.726968in}}%
\pgfpathlineto{\pgfqpoint{3.463186in}{2.662942in}}%
\pgfpathlineto{\pgfqpoint{3.574082in}{2.726968in}}%
\pgfpathlineto{\pgfqpoint{3.463186in}{2.790995in}}%
\pgfpathclose%
\pgfusepath{fill}%
\end{pgfscope}%
\begin{pgfscope}%
\pgfpathrectangle{\pgfqpoint{0.765000in}{0.660000in}}{\pgfqpoint{4.620000in}{4.620000in}}%
\pgfusepath{clip}%
\pgfsetbuttcap%
\pgfsetroundjoin%
\definecolor{currentfill}{rgb}{1.000000,0.894118,0.788235}%
\pgfsetfillcolor{currentfill}%
\pgfsetlinewidth{0.000000pt}%
\definecolor{currentstroke}{rgb}{1.000000,0.894118,0.788235}%
\pgfsetstrokecolor{currentstroke}%
\pgfsetdash{}{0pt}%
\pgfpathmoveto{\pgfqpoint{3.463186in}{2.534890in}}%
\pgfpathlineto{\pgfqpoint{3.574082in}{2.598916in}}%
\pgfpathlineto{\pgfqpoint{3.574082in}{2.726968in}}%
\pgfpathlineto{\pgfqpoint{3.463186in}{2.662942in}}%
\pgfpathlineto{\pgfqpoint{3.463186in}{2.534890in}}%
\pgfpathclose%
\pgfusepath{fill}%
\end{pgfscope}%
\begin{pgfscope}%
\pgfpathrectangle{\pgfqpoint{0.765000in}{0.660000in}}{\pgfqpoint{4.620000in}{4.620000in}}%
\pgfusepath{clip}%
\pgfsetbuttcap%
\pgfsetroundjoin%
\definecolor{currentfill}{rgb}{1.000000,0.894118,0.788235}%
\pgfsetfillcolor{currentfill}%
\pgfsetlinewidth{0.000000pt}%
\definecolor{currentstroke}{rgb}{1.000000,0.894118,0.788235}%
\pgfsetstrokecolor{currentstroke}%
\pgfsetdash{}{0pt}%
\pgfpathmoveto{\pgfqpoint{3.352289in}{2.598916in}}%
\pgfpathlineto{\pgfqpoint{3.463186in}{2.534890in}}%
\pgfpathlineto{\pgfqpoint{3.463186in}{2.662942in}}%
\pgfpathlineto{\pgfqpoint{3.352289in}{2.726968in}}%
\pgfpathlineto{\pgfqpoint{3.352289in}{2.598916in}}%
\pgfpathclose%
\pgfusepath{fill}%
\end{pgfscope}%
\begin{pgfscope}%
\pgfpathrectangle{\pgfqpoint{0.765000in}{0.660000in}}{\pgfqpoint{4.620000in}{4.620000in}}%
\pgfusepath{clip}%
\pgfsetbuttcap%
\pgfsetroundjoin%
\definecolor{currentfill}{rgb}{1.000000,0.894118,0.788235}%
\pgfsetfillcolor{currentfill}%
\pgfsetlinewidth{0.000000pt}%
\definecolor{currentstroke}{rgb}{1.000000,0.894118,0.788235}%
\pgfsetstrokecolor{currentstroke}%
\pgfsetdash{}{0pt}%
\pgfpathmoveto{\pgfqpoint{3.477043in}{2.674663in}}%
\pgfpathlineto{\pgfqpoint{3.366147in}{2.610637in}}%
\pgfpathlineto{\pgfqpoint{3.352289in}{2.598916in}}%
\pgfpathlineto{\pgfqpoint{3.463186in}{2.662942in}}%
\pgfpathlineto{\pgfqpoint{3.477043in}{2.674663in}}%
\pgfpathclose%
\pgfusepath{fill}%
\end{pgfscope}%
\begin{pgfscope}%
\pgfpathrectangle{\pgfqpoint{0.765000in}{0.660000in}}{\pgfqpoint{4.620000in}{4.620000in}}%
\pgfusepath{clip}%
\pgfsetbuttcap%
\pgfsetroundjoin%
\definecolor{currentfill}{rgb}{1.000000,0.894118,0.788235}%
\pgfsetfillcolor{currentfill}%
\pgfsetlinewidth{0.000000pt}%
\definecolor{currentstroke}{rgb}{1.000000,0.894118,0.788235}%
\pgfsetstrokecolor{currentstroke}%
\pgfsetdash{}{0pt}%
\pgfpathmoveto{\pgfqpoint{3.587940in}{2.610637in}}%
\pgfpathlineto{\pgfqpoint{3.477043in}{2.674663in}}%
\pgfpathlineto{\pgfqpoint{3.463186in}{2.662942in}}%
\pgfpathlineto{\pgfqpoint{3.574082in}{2.598916in}}%
\pgfpathlineto{\pgfqpoint{3.587940in}{2.610637in}}%
\pgfpathclose%
\pgfusepath{fill}%
\end{pgfscope}%
\begin{pgfscope}%
\pgfpathrectangle{\pgfqpoint{0.765000in}{0.660000in}}{\pgfqpoint{4.620000in}{4.620000in}}%
\pgfusepath{clip}%
\pgfsetbuttcap%
\pgfsetroundjoin%
\definecolor{currentfill}{rgb}{1.000000,0.894118,0.788235}%
\pgfsetfillcolor{currentfill}%
\pgfsetlinewidth{0.000000pt}%
\definecolor{currentstroke}{rgb}{1.000000,0.894118,0.788235}%
\pgfsetstrokecolor{currentstroke}%
\pgfsetdash{}{0pt}%
\pgfpathmoveto{\pgfqpoint{3.477043in}{2.674663in}}%
\pgfpathlineto{\pgfqpoint{3.477043in}{2.802716in}}%
\pgfpathlineto{\pgfqpoint{3.463186in}{2.790995in}}%
\pgfpathlineto{\pgfqpoint{3.574082in}{2.598916in}}%
\pgfpathlineto{\pgfqpoint{3.477043in}{2.674663in}}%
\pgfpathclose%
\pgfusepath{fill}%
\end{pgfscope}%
\begin{pgfscope}%
\pgfpathrectangle{\pgfqpoint{0.765000in}{0.660000in}}{\pgfqpoint{4.620000in}{4.620000in}}%
\pgfusepath{clip}%
\pgfsetbuttcap%
\pgfsetroundjoin%
\definecolor{currentfill}{rgb}{1.000000,0.894118,0.788235}%
\pgfsetfillcolor{currentfill}%
\pgfsetlinewidth{0.000000pt}%
\definecolor{currentstroke}{rgb}{1.000000,0.894118,0.788235}%
\pgfsetstrokecolor{currentstroke}%
\pgfsetdash{}{0pt}%
\pgfpathmoveto{\pgfqpoint{3.587940in}{2.610637in}}%
\pgfpathlineto{\pgfqpoint{3.587940in}{2.738689in}}%
\pgfpathlineto{\pgfqpoint{3.574082in}{2.726968in}}%
\pgfpathlineto{\pgfqpoint{3.463186in}{2.662942in}}%
\pgfpathlineto{\pgfqpoint{3.587940in}{2.610637in}}%
\pgfpathclose%
\pgfusepath{fill}%
\end{pgfscope}%
\begin{pgfscope}%
\pgfpathrectangle{\pgfqpoint{0.765000in}{0.660000in}}{\pgfqpoint{4.620000in}{4.620000in}}%
\pgfusepath{clip}%
\pgfsetbuttcap%
\pgfsetroundjoin%
\definecolor{currentfill}{rgb}{1.000000,0.894118,0.788235}%
\pgfsetfillcolor{currentfill}%
\pgfsetlinewidth{0.000000pt}%
\definecolor{currentstroke}{rgb}{1.000000,0.894118,0.788235}%
\pgfsetstrokecolor{currentstroke}%
\pgfsetdash{}{0pt}%
\pgfpathmoveto{\pgfqpoint{3.366147in}{2.610637in}}%
\pgfpathlineto{\pgfqpoint{3.477043in}{2.546611in}}%
\pgfpathlineto{\pgfqpoint{3.463186in}{2.534890in}}%
\pgfpathlineto{\pgfqpoint{3.352289in}{2.598916in}}%
\pgfpathlineto{\pgfqpoint{3.366147in}{2.610637in}}%
\pgfpathclose%
\pgfusepath{fill}%
\end{pgfscope}%
\begin{pgfscope}%
\pgfpathrectangle{\pgfqpoint{0.765000in}{0.660000in}}{\pgfqpoint{4.620000in}{4.620000in}}%
\pgfusepath{clip}%
\pgfsetbuttcap%
\pgfsetroundjoin%
\definecolor{currentfill}{rgb}{1.000000,0.894118,0.788235}%
\pgfsetfillcolor{currentfill}%
\pgfsetlinewidth{0.000000pt}%
\definecolor{currentstroke}{rgb}{1.000000,0.894118,0.788235}%
\pgfsetstrokecolor{currentstroke}%
\pgfsetdash{}{0pt}%
\pgfpathmoveto{\pgfqpoint{3.477043in}{2.546611in}}%
\pgfpathlineto{\pgfqpoint{3.587940in}{2.610637in}}%
\pgfpathlineto{\pgfqpoint{3.574082in}{2.598916in}}%
\pgfpathlineto{\pgfqpoint{3.463186in}{2.534890in}}%
\pgfpathlineto{\pgfqpoint{3.477043in}{2.546611in}}%
\pgfpathclose%
\pgfusepath{fill}%
\end{pgfscope}%
\begin{pgfscope}%
\pgfpathrectangle{\pgfqpoint{0.765000in}{0.660000in}}{\pgfqpoint{4.620000in}{4.620000in}}%
\pgfusepath{clip}%
\pgfsetbuttcap%
\pgfsetroundjoin%
\definecolor{currentfill}{rgb}{1.000000,0.894118,0.788235}%
\pgfsetfillcolor{currentfill}%
\pgfsetlinewidth{0.000000pt}%
\definecolor{currentstroke}{rgb}{1.000000,0.894118,0.788235}%
\pgfsetstrokecolor{currentstroke}%
\pgfsetdash{}{0pt}%
\pgfpathmoveto{\pgfqpoint{3.477043in}{2.802716in}}%
\pgfpathlineto{\pgfqpoint{3.366147in}{2.738689in}}%
\pgfpathlineto{\pgfqpoint{3.352289in}{2.726968in}}%
\pgfpathlineto{\pgfqpoint{3.463186in}{2.790995in}}%
\pgfpathlineto{\pgfqpoint{3.477043in}{2.802716in}}%
\pgfpathclose%
\pgfusepath{fill}%
\end{pgfscope}%
\begin{pgfscope}%
\pgfpathrectangle{\pgfqpoint{0.765000in}{0.660000in}}{\pgfqpoint{4.620000in}{4.620000in}}%
\pgfusepath{clip}%
\pgfsetbuttcap%
\pgfsetroundjoin%
\definecolor{currentfill}{rgb}{1.000000,0.894118,0.788235}%
\pgfsetfillcolor{currentfill}%
\pgfsetlinewidth{0.000000pt}%
\definecolor{currentstroke}{rgb}{1.000000,0.894118,0.788235}%
\pgfsetstrokecolor{currentstroke}%
\pgfsetdash{}{0pt}%
\pgfpathmoveto{\pgfqpoint{3.587940in}{2.738689in}}%
\pgfpathlineto{\pgfqpoint{3.477043in}{2.802716in}}%
\pgfpathlineto{\pgfqpoint{3.463186in}{2.790995in}}%
\pgfpathlineto{\pgfqpoint{3.574082in}{2.726968in}}%
\pgfpathlineto{\pgfqpoint{3.587940in}{2.738689in}}%
\pgfpathclose%
\pgfusepath{fill}%
\end{pgfscope}%
\begin{pgfscope}%
\pgfpathrectangle{\pgfqpoint{0.765000in}{0.660000in}}{\pgfqpoint{4.620000in}{4.620000in}}%
\pgfusepath{clip}%
\pgfsetbuttcap%
\pgfsetroundjoin%
\definecolor{currentfill}{rgb}{1.000000,0.894118,0.788235}%
\pgfsetfillcolor{currentfill}%
\pgfsetlinewidth{0.000000pt}%
\definecolor{currentstroke}{rgb}{1.000000,0.894118,0.788235}%
\pgfsetstrokecolor{currentstroke}%
\pgfsetdash{}{0pt}%
\pgfpathmoveto{\pgfqpoint{3.366147in}{2.610637in}}%
\pgfpathlineto{\pgfqpoint{3.366147in}{2.738689in}}%
\pgfpathlineto{\pgfqpoint{3.352289in}{2.726968in}}%
\pgfpathlineto{\pgfqpoint{3.463186in}{2.534890in}}%
\pgfpathlineto{\pgfqpoint{3.366147in}{2.610637in}}%
\pgfpathclose%
\pgfusepath{fill}%
\end{pgfscope}%
\begin{pgfscope}%
\pgfpathrectangle{\pgfqpoint{0.765000in}{0.660000in}}{\pgfqpoint{4.620000in}{4.620000in}}%
\pgfusepath{clip}%
\pgfsetbuttcap%
\pgfsetroundjoin%
\definecolor{currentfill}{rgb}{1.000000,0.894118,0.788235}%
\pgfsetfillcolor{currentfill}%
\pgfsetlinewidth{0.000000pt}%
\definecolor{currentstroke}{rgb}{1.000000,0.894118,0.788235}%
\pgfsetstrokecolor{currentstroke}%
\pgfsetdash{}{0pt}%
\pgfpathmoveto{\pgfqpoint{3.477043in}{2.546611in}}%
\pgfpathlineto{\pgfqpoint{3.477043in}{2.674663in}}%
\pgfpathlineto{\pgfqpoint{3.463186in}{2.662942in}}%
\pgfpathlineto{\pgfqpoint{3.352289in}{2.598916in}}%
\pgfpathlineto{\pgfqpoint{3.477043in}{2.546611in}}%
\pgfpathclose%
\pgfusepath{fill}%
\end{pgfscope}%
\begin{pgfscope}%
\pgfpathrectangle{\pgfqpoint{0.765000in}{0.660000in}}{\pgfqpoint{4.620000in}{4.620000in}}%
\pgfusepath{clip}%
\pgfsetbuttcap%
\pgfsetroundjoin%
\definecolor{currentfill}{rgb}{1.000000,0.894118,0.788235}%
\pgfsetfillcolor{currentfill}%
\pgfsetlinewidth{0.000000pt}%
\definecolor{currentstroke}{rgb}{1.000000,0.894118,0.788235}%
\pgfsetstrokecolor{currentstroke}%
\pgfsetdash{}{0pt}%
\pgfpathmoveto{\pgfqpoint{3.477043in}{2.674663in}}%
\pgfpathlineto{\pgfqpoint{3.587940in}{2.738689in}}%
\pgfpathlineto{\pgfqpoint{3.574082in}{2.726968in}}%
\pgfpathlineto{\pgfqpoint{3.463186in}{2.662942in}}%
\pgfpathlineto{\pgfqpoint{3.477043in}{2.674663in}}%
\pgfpathclose%
\pgfusepath{fill}%
\end{pgfscope}%
\begin{pgfscope}%
\pgfpathrectangle{\pgfqpoint{0.765000in}{0.660000in}}{\pgfqpoint{4.620000in}{4.620000in}}%
\pgfusepath{clip}%
\pgfsetbuttcap%
\pgfsetroundjoin%
\definecolor{currentfill}{rgb}{1.000000,0.894118,0.788235}%
\pgfsetfillcolor{currentfill}%
\pgfsetlinewidth{0.000000pt}%
\definecolor{currentstroke}{rgb}{1.000000,0.894118,0.788235}%
\pgfsetstrokecolor{currentstroke}%
\pgfsetdash{}{0pt}%
\pgfpathmoveto{\pgfqpoint{3.366147in}{2.738689in}}%
\pgfpathlineto{\pgfqpoint{3.477043in}{2.674663in}}%
\pgfpathlineto{\pgfqpoint{3.463186in}{2.662942in}}%
\pgfpathlineto{\pgfqpoint{3.352289in}{2.726968in}}%
\pgfpathlineto{\pgfqpoint{3.366147in}{2.738689in}}%
\pgfpathclose%
\pgfusepath{fill}%
\end{pgfscope}%
\begin{pgfscope}%
\pgfpathrectangle{\pgfqpoint{0.765000in}{0.660000in}}{\pgfqpoint{4.620000in}{4.620000in}}%
\pgfusepath{clip}%
\pgfsetbuttcap%
\pgfsetroundjoin%
\definecolor{currentfill}{rgb}{1.000000,0.894118,0.788235}%
\pgfsetfillcolor{currentfill}%
\pgfsetlinewidth{0.000000pt}%
\definecolor{currentstroke}{rgb}{1.000000,0.894118,0.788235}%
\pgfsetstrokecolor{currentstroke}%
\pgfsetdash{}{0pt}%
\pgfpathmoveto{\pgfqpoint{2.793000in}{3.409870in}}%
\pgfpathlineto{\pgfqpoint{2.682103in}{3.345844in}}%
\pgfpathlineto{\pgfqpoint{2.793000in}{3.281817in}}%
\pgfpathlineto{\pgfqpoint{2.903897in}{3.345844in}}%
\pgfpathlineto{\pgfqpoint{2.793000in}{3.409870in}}%
\pgfpathclose%
\pgfusepath{fill}%
\end{pgfscope}%
\begin{pgfscope}%
\pgfpathrectangle{\pgfqpoint{0.765000in}{0.660000in}}{\pgfqpoint{4.620000in}{4.620000in}}%
\pgfusepath{clip}%
\pgfsetbuttcap%
\pgfsetroundjoin%
\definecolor{currentfill}{rgb}{1.000000,0.894118,0.788235}%
\pgfsetfillcolor{currentfill}%
\pgfsetlinewidth{0.000000pt}%
\definecolor{currentstroke}{rgb}{1.000000,0.894118,0.788235}%
\pgfsetstrokecolor{currentstroke}%
\pgfsetdash{}{0pt}%
\pgfpathmoveto{\pgfqpoint{2.793000in}{3.409870in}}%
\pgfpathlineto{\pgfqpoint{2.682103in}{3.345844in}}%
\pgfpathlineto{\pgfqpoint{2.682103in}{3.473896in}}%
\pgfpathlineto{\pgfqpoint{2.793000in}{3.537922in}}%
\pgfpathlineto{\pgfqpoint{2.793000in}{3.409870in}}%
\pgfpathclose%
\pgfusepath{fill}%
\end{pgfscope}%
\begin{pgfscope}%
\pgfpathrectangle{\pgfqpoint{0.765000in}{0.660000in}}{\pgfqpoint{4.620000in}{4.620000in}}%
\pgfusepath{clip}%
\pgfsetbuttcap%
\pgfsetroundjoin%
\definecolor{currentfill}{rgb}{1.000000,0.894118,0.788235}%
\pgfsetfillcolor{currentfill}%
\pgfsetlinewidth{0.000000pt}%
\definecolor{currentstroke}{rgb}{1.000000,0.894118,0.788235}%
\pgfsetstrokecolor{currentstroke}%
\pgfsetdash{}{0pt}%
\pgfpathmoveto{\pgfqpoint{2.793000in}{3.409870in}}%
\pgfpathlineto{\pgfqpoint{2.903897in}{3.345844in}}%
\pgfpathlineto{\pgfqpoint{2.903897in}{3.473896in}}%
\pgfpathlineto{\pgfqpoint{2.793000in}{3.537922in}}%
\pgfpathlineto{\pgfqpoint{2.793000in}{3.409870in}}%
\pgfpathclose%
\pgfusepath{fill}%
\end{pgfscope}%
\begin{pgfscope}%
\pgfpathrectangle{\pgfqpoint{0.765000in}{0.660000in}}{\pgfqpoint{4.620000in}{4.620000in}}%
\pgfusepath{clip}%
\pgfsetbuttcap%
\pgfsetroundjoin%
\definecolor{currentfill}{rgb}{1.000000,0.894118,0.788235}%
\pgfsetfillcolor{currentfill}%
\pgfsetlinewidth{0.000000pt}%
\definecolor{currentstroke}{rgb}{1.000000,0.894118,0.788235}%
\pgfsetstrokecolor{currentstroke}%
\pgfsetdash{}{0pt}%
\pgfpathmoveto{\pgfqpoint{2.788056in}{3.406767in}}%
\pgfpathlineto{\pgfqpoint{2.677159in}{3.342741in}}%
\pgfpathlineto{\pgfqpoint{2.788056in}{3.278715in}}%
\pgfpathlineto{\pgfqpoint{2.898952in}{3.342741in}}%
\pgfpathlineto{\pgfqpoint{2.788056in}{3.406767in}}%
\pgfpathclose%
\pgfusepath{fill}%
\end{pgfscope}%
\begin{pgfscope}%
\pgfpathrectangle{\pgfqpoint{0.765000in}{0.660000in}}{\pgfqpoint{4.620000in}{4.620000in}}%
\pgfusepath{clip}%
\pgfsetbuttcap%
\pgfsetroundjoin%
\definecolor{currentfill}{rgb}{1.000000,0.894118,0.788235}%
\pgfsetfillcolor{currentfill}%
\pgfsetlinewidth{0.000000pt}%
\definecolor{currentstroke}{rgb}{1.000000,0.894118,0.788235}%
\pgfsetstrokecolor{currentstroke}%
\pgfsetdash{}{0pt}%
\pgfpathmoveto{\pgfqpoint{2.788056in}{3.406767in}}%
\pgfpathlineto{\pgfqpoint{2.677159in}{3.342741in}}%
\pgfpathlineto{\pgfqpoint{2.677159in}{3.470793in}}%
\pgfpathlineto{\pgfqpoint{2.788056in}{3.534820in}}%
\pgfpathlineto{\pgfqpoint{2.788056in}{3.406767in}}%
\pgfpathclose%
\pgfusepath{fill}%
\end{pgfscope}%
\begin{pgfscope}%
\pgfpathrectangle{\pgfqpoint{0.765000in}{0.660000in}}{\pgfqpoint{4.620000in}{4.620000in}}%
\pgfusepath{clip}%
\pgfsetbuttcap%
\pgfsetroundjoin%
\definecolor{currentfill}{rgb}{1.000000,0.894118,0.788235}%
\pgfsetfillcolor{currentfill}%
\pgfsetlinewidth{0.000000pt}%
\definecolor{currentstroke}{rgb}{1.000000,0.894118,0.788235}%
\pgfsetstrokecolor{currentstroke}%
\pgfsetdash{}{0pt}%
\pgfpathmoveto{\pgfqpoint{2.788056in}{3.406767in}}%
\pgfpathlineto{\pgfqpoint{2.898952in}{3.342741in}}%
\pgfpathlineto{\pgfqpoint{2.898952in}{3.470793in}}%
\pgfpathlineto{\pgfqpoint{2.788056in}{3.534820in}}%
\pgfpathlineto{\pgfqpoint{2.788056in}{3.406767in}}%
\pgfpathclose%
\pgfusepath{fill}%
\end{pgfscope}%
\begin{pgfscope}%
\pgfpathrectangle{\pgfqpoint{0.765000in}{0.660000in}}{\pgfqpoint{4.620000in}{4.620000in}}%
\pgfusepath{clip}%
\pgfsetbuttcap%
\pgfsetroundjoin%
\definecolor{currentfill}{rgb}{1.000000,0.894118,0.788235}%
\pgfsetfillcolor{currentfill}%
\pgfsetlinewidth{0.000000pt}%
\definecolor{currentstroke}{rgb}{1.000000,0.894118,0.788235}%
\pgfsetstrokecolor{currentstroke}%
\pgfsetdash{}{0pt}%
\pgfpathmoveto{\pgfqpoint{2.793000in}{3.537922in}}%
\pgfpathlineto{\pgfqpoint{2.682103in}{3.473896in}}%
\pgfpathlineto{\pgfqpoint{2.793000in}{3.409870in}}%
\pgfpathlineto{\pgfqpoint{2.903897in}{3.473896in}}%
\pgfpathlineto{\pgfqpoint{2.793000in}{3.537922in}}%
\pgfpathclose%
\pgfusepath{fill}%
\end{pgfscope}%
\begin{pgfscope}%
\pgfpathrectangle{\pgfqpoint{0.765000in}{0.660000in}}{\pgfqpoint{4.620000in}{4.620000in}}%
\pgfusepath{clip}%
\pgfsetbuttcap%
\pgfsetroundjoin%
\definecolor{currentfill}{rgb}{1.000000,0.894118,0.788235}%
\pgfsetfillcolor{currentfill}%
\pgfsetlinewidth{0.000000pt}%
\definecolor{currentstroke}{rgb}{1.000000,0.894118,0.788235}%
\pgfsetstrokecolor{currentstroke}%
\pgfsetdash{}{0pt}%
\pgfpathmoveto{\pgfqpoint{2.793000in}{3.281817in}}%
\pgfpathlineto{\pgfqpoint{2.903897in}{3.345844in}}%
\pgfpathlineto{\pgfqpoint{2.903897in}{3.473896in}}%
\pgfpathlineto{\pgfqpoint{2.793000in}{3.409870in}}%
\pgfpathlineto{\pgfqpoint{2.793000in}{3.281817in}}%
\pgfpathclose%
\pgfusepath{fill}%
\end{pgfscope}%
\begin{pgfscope}%
\pgfpathrectangle{\pgfqpoint{0.765000in}{0.660000in}}{\pgfqpoint{4.620000in}{4.620000in}}%
\pgfusepath{clip}%
\pgfsetbuttcap%
\pgfsetroundjoin%
\definecolor{currentfill}{rgb}{1.000000,0.894118,0.788235}%
\pgfsetfillcolor{currentfill}%
\pgfsetlinewidth{0.000000pt}%
\definecolor{currentstroke}{rgb}{1.000000,0.894118,0.788235}%
\pgfsetstrokecolor{currentstroke}%
\pgfsetdash{}{0pt}%
\pgfpathmoveto{\pgfqpoint{2.682103in}{3.345844in}}%
\pgfpathlineto{\pgfqpoint{2.793000in}{3.281817in}}%
\pgfpathlineto{\pgfqpoint{2.793000in}{3.409870in}}%
\pgfpathlineto{\pgfqpoint{2.682103in}{3.473896in}}%
\pgfpathlineto{\pgfqpoint{2.682103in}{3.345844in}}%
\pgfpathclose%
\pgfusepath{fill}%
\end{pgfscope}%
\begin{pgfscope}%
\pgfpathrectangle{\pgfqpoint{0.765000in}{0.660000in}}{\pgfqpoint{4.620000in}{4.620000in}}%
\pgfusepath{clip}%
\pgfsetbuttcap%
\pgfsetroundjoin%
\definecolor{currentfill}{rgb}{1.000000,0.894118,0.788235}%
\pgfsetfillcolor{currentfill}%
\pgfsetlinewidth{0.000000pt}%
\definecolor{currentstroke}{rgb}{1.000000,0.894118,0.788235}%
\pgfsetstrokecolor{currentstroke}%
\pgfsetdash{}{0pt}%
\pgfpathmoveto{\pgfqpoint{2.788056in}{3.534820in}}%
\pgfpathlineto{\pgfqpoint{2.677159in}{3.470793in}}%
\pgfpathlineto{\pgfqpoint{2.788056in}{3.406767in}}%
\pgfpathlineto{\pgfqpoint{2.898952in}{3.470793in}}%
\pgfpathlineto{\pgfqpoint{2.788056in}{3.534820in}}%
\pgfpathclose%
\pgfusepath{fill}%
\end{pgfscope}%
\begin{pgfscope}%
\pgfpathrectangle{\pgfqpoint{0.765000in}{0.660000in}}{\pgfqpoint{4.620000in}{4.620000in}}%
\pgfusepath{clip}%
\pgfsetbuttcap%
\pgfsetroundjoin%
\definecolor{currentfill}{rgb}{1.000000,0.894118,0.788235}%
\pgfsetfillcolor{currentfill}%
\pgfsetlinewidth{0.000000pt}%
\definecolor{currentstroke}{rgb}{1.000000,0.894118,0.788235}%
\pgfsetstrokecolor{currentstroke}%
\pgfsetdash{}{0pt}%
\pgfpathmoveto{\pgfqpoint{2.788056in}{3.278715in}}%
\pgfpathlineto{\pgfqpoint{2.898952in}{3.342741in}}%
\pgfpathlineto{\pgfqpoint{2.898952in}{3.470793in}}%
\pgfpathlineto{\pgfqpoint{2.788056in}{3.406767in}}%
\pgfpathlineto{\pgfqpoint{2.788056in}{3.278715in}}%
\pgfpathclose%
\pgfusepath{fill}%
\end{pgfscope}%
\begin{pgfscope}%
\pgfpathrectangle{\pgfqpoint{0.765000in}{0.660000in}}{\pgfqpoint{4.620000in}{4.620000in}}%
\pgfusepath{clip}%
\pgfsetbuttcap%
\pgfsetroundjoin%
\definecolor{currentfill}{rgb}{1.000000,0.894118,0.788235}%
\pgfsetfillcolor{currentfill}%
\pgfsetlinewidth{0.000000pt}%
\definecolor{currentstroke}{rgb}{1.000000,0.894118,0.788235}%
\pgfsetstrokecolor{currentstroke}%
\pgfsetdash{}{0pt}%
\pgfpathmoveto{\pgfqpoint{2.677159in}{3.342741in}}%
\pgfpathlineto{\pgfqpoint{2.788056in}{3.278715in}}%
\pgfpathlineto{\pgfqpoint{2.788056in}{3.406767in}}%
\pgfpathlineto{\pgfqpoint{2.677159in}{3.470793in}}%
\pgfpathlineto{\pgfqpoint{2.677159in}{3.342741in}}%
\pgfpathclose%
\pgfusepath{fill}%
\end{pgfscope}%
\begin{pgfscope}%
\pgfpathrectangle{\pgfqpoint{0.765000in}{0.660000in}}{\pgfqpoint{4.620000in}{4.620000in}}%
\pgfusepath{clip}%
\pgfsetbuttcap%
\pgfsetroundjoin%
\definecolor{currentfill}{rgb}{1.000000,0.894118,0.788235}%
\pgfsetfillcolor{currentfill}%
\pgfsetlinewidth{0.000000pt}%
\definecolor{currentstroke}{rgb}{1.000000,0.894118,0.788235}%
\pgfsetstrokecolor{currentstroke}%
\pgfsetdash{}{0pt}%
\pgfpathmoveto{\pgfqpoint{2.793000in}{3.409870in}}%
\pgfpathlineto{\pgfqpoint{2.682103in}{3.345844in}}%
\pgfpathlineto{\pgfqpoint{2.677159in}{3.342741in}}%
\pgfpathlineto{\pgfqpoint{2.788056in}{3.406767in}}%
\pgfpathlineto{\pgfqpoint{2.793000in}{3.409870in}}%
\pgfpathclose%
\pgfusepath{fill}%
\end{pgfscope}%
\begin{pgfscope}%
\pgfpathrectangle{\pgfqpoint{0.765000in}{0.660000in}}{\pgfqpoint{4.620000in}{4.620000in}}%
\pgfusepath{clip}%
\pgfsetbuttcap%
\pgfsetroundjoin%
\definecolor{currentfill}{rgb}{1.000000,0.894118,0.788235}%
\pgfsetfillcolor{currentfill}%
\pgfsetlinewidth{0.000000pt}%
\definecolor{currentstroke}{rgb}{1.000000,0.894118,0.788235}%
\pgfsetstrokecolor{currentstroke}%
\pgfsetdash{}{0pt}%
\pgfpathmoveto{\pgfqpoint{2.903897in}{3.345844in}}%
\pgfpathlineto{\pgfqpoint{2.793000in}{3.409870in}}%
\pgfpathlineto{\pgfqpoint{2.788056in}{3.406767in}}%
\pgfpathlineto{\pgfqpoint{2.898952in}{3.342741in}}%
\pgfpathlineto{\pgfqpoint{2.903897in}{3.345844in}}%
\pgfpathclose%
\pgfusepath{fill}%
\end{pgfscope}%
\begin{pgfscope}%
\pgfpathrectangle{\pgfqpoint{0.765000in}{0.660000in}}{\pgfqpoint{4.620000in}{4.620000in}}%
\pgfusepath{clip}%
\pgfsetbuttcap%
\pgfsetroundjoin%
\definecolor{currentfill}{rgb}{1.000000,0.894118,0.788235}%
\pgfsetfillcolor{currentfill}%
\pgfsetlinewidth{0.000000pt}%
\definecolor{currentstroke}{rgb}{1.000000,0.894118,0.788235}%
\pgfsetstrokecolor{currentstroke}%
\pgfsetdash{}{0pt}%
\pgfpathmoveto{\pgfqpoint{2.793000in}{3.409870in}}%
\pgfpathlineto{\pgfqpoint{2.793000in}{3.537922in}}%
\pgfpathlineto{\pgfqpoint{2.788056in}{3.534820in}}%
\pgfpathlineto{\pgfqpoint{2.898952in}{3.342741in}}%
\pgfpathlineto{\pgfqpoint{2.793000in}{3.409870in}}%
\pgfpathclose%
\pgfusepath{fill}%
\end{pgfscope}%
\begin{pgfscope}%
\pgfpathrectangle{\pgfqpoint{0.765000in}{0.660000in}}{\pgfqpoint{4.620000in}{4.620000in}}%
\pgfusepath{clip}%
\pgfsetbuttcap%
\pgfsetroundjoin%
\definecolor{currentfill}{rgb}{1.000000,0.894118,0.788235}%
\pgfsetfillcolor{currentfill}%
\pgfsetlinewidth{0.000000pt}%
\definecolor{currentstroke}{rgb}{1.000000,0.894118,0.788235}%
\pgfsetstrokecolor{currentstroke}%
\pgfsetdash{}{0pt}%
\pgfpathmoveto{\pgfqpoint{2.903897in}{3.345844in}}%
\pgfpathlineto{\pgfqpoint{2.903897in}{3.473896in}}%
\pgfpathlineto{\pgfqpoint{2.898952in}{3.470793in}}%
\pgfpathlineto{\pgfqpoint{2.788056in}{3.406767in}}%
\pgfpathlineto{\pgfqpoint{2.903897in}{3.345844in}}%
\pgfpathclose%
\pgfusepath{fill}%
\end{pgfscope}%
\begin{pgfscope}%
\pgfpathrectangle{\pgfqpoint{0.765000in}{0.660000in}}{\pgfqpoint{4.620000in}{4.620000in}}%
\pgfusepath{clip}%
\pgfsetbuttcap%
\pgfsetroundjoin%
\definecolor{currentfill}{rgb}{1.000000,0.894118,0.788235}%
\pgfsetfillcolor{currentfill}%
\pgfsetlinewidth{0.000000pt}%
\definecolor{currentstroke}{rgb}{1.000000,0.894118,0.788235}%
\pgfsetstrokecolor{currentstroke}%
\pgfsetdash{}{0pt}%
\pgfpathmoveto{\pgfqpoint{2.682103in}{3.345844in}}%
\pgfpathlineto{\pgfqpoint{2.793000in}{3.281817in}}%
\pgfpathlineto{\pgfqpoint{2.788056in}{3.278715in}}%
\pgfpathlineto{\pgfqpoint{2.677159in}{3.342741in}}%
\pgfpathlineto{\pgfqpoint{2.682103in}{3.345844in}}%
\pgfpathclose%
\pgfusepath{fill}%
\end{pgfscope}%
\begin{pgfscope}%
\pgfpathrectangle{\pgfqpoint{0.765000in}{0.660000in}}{\pgfqpoint{4.620000in}{4.620000in}}%
\pgfusepath{clip}%
\pgfsetbuttcap%
\pgfsetroundjoin%
\definecolor{currentfill}{rgb}{1.000000,0.894118,0.788235}%
\pgfsetfillcolor{currentfill}%
\pgfsetlinewidth{0.000000pt}%
\definecolor{currentstroke}{rgb}{1.000000,0.894118,0.788235}%
\pgfsetstrokecolor{currentstroke}%
\pgfsetdash{}{0pt}%
\pgfpathmoveto{\pgfqpoint{2.793000in}{3.281817in}}%
\pgfpathlineto{\pgfqpoint{2.903897in}{3.345844in}}%
\pgfpathlineto{\pgfqpoint{2.898952in}{3.342741in}}%
\pgfpathlineto{\pgfqpoint{2.788056in}{3.278715in}}%
\pgfpathlineto{\pgfqpoint{2.793000in}{3.281817in}}%
\pgfpathclose%
\pgfusepath{fill}%
\end{pgfscope}%
\begin{pgfscope}%
\pgfpathrectangle{\pgfqpoint{0.765000in}{0.660000in}}{\pgfqpoint{4.620000in}{4.620000in}}%
\pgfusepath{clip}%
\pgfsetbuttcap%
\pgfsetroundjoin%
\definecolor{currentfill}{rgb}{1.000000,0.894118,0.788235}%
\pgfsetfillcolor{currentfill}%
\pgfsetlinewidth{0.000000pt}%
\definecolor{currentstroke}{rgb}{1.000000,0.894118,0.788235}%
\pgfsetstrokecolor{currentstroke}%
\pgfsetdash{}{0pt}%
\pgfpathmoveto{\pgfqpoint{2.793000in}{3.537922in}}%
\pgfpathlineto{\pgfqpoint{2.682103in}{3.473896in}}%
\pgfpathlineto{\pgfqpoint{2.677159in}{3.470793in}}%
\pgfpathlineto{\pgfqpoint{2.788056in}{3.534820in}}%
\pgfpathlineto{\pgfqpoint{2.793000in}{3.537922in}}%
\pgfpathclose%
\pgfusepath{fill}%
\end{pgfscope}%
\begin{pgfscope}%
\pgfpathrectangle{\pgfqpoint{0.765000in}{0.660000in}}{\pgfqpoint{4.620000in}{4.620000in}}%
\pgfusepath{clip}%
\pgfsetbuttcap%
\pgfsetroundjoin%
\definecolor{currentfill}{rgb}{1.000000,0.894118,0.788235}%
\pgfsetfillcolor{currentfill}%
\pgfsetlinewidth{0.000000pt}%
\definecolor{currentstroke}{rgb}{1.000000,0.894118,0.788235}%
\pgfsetstrokecolor{currentstroke}%
\pgfsetdash{}{0pt}%
\pgfpathmoveto{\pgfqpoint{2.903897in}{3.473896in}}%
\pgfpathlineto{\pgfqpoint{2.793000in}{3.537922in}}%
\pgfpathlineto{\pgfqpoint{2.788056in}{3.534820in}}%
\pgfpathlineto{\pgfqpoint{2.898952in}{3.470793in}}%
\pgfpathlineto{\pgfqpoint{2.903897in}{3.473896in}}%
\pgfpathclose%
\pgfusepath{fill}%
\end{pgfscope}%
\begin{pgfscope}%
\pgfpathrectangle{\pgfqpoint{0.765000in}{0.660000in}}{\pgfqpoint{4.620000in}{4.620000in}}%
\pgfusepath{clip}%
\pgfsetbuttcap%
\pgfsetroundjoin%
\definecolor{currentfill}{rgb}{1.000000,0.894118,0.788235}%
\pgfsetfillcolor{currentfill}%
\pgfsetlinewidth{0.000000pt}%
\definecolor{currentstroke}{rgb}{1.000000,0.894118,0.788235}%
\pgfsetstrokecolor{currentstroke}%
\pgfsetdash{}{0pt}%
\pgfpathmoveto{\pgfqpoint{2.682103in}{3.345844in}}%
\pgfpathlineto{\pgfqpoint{2.682103in}{3.473896in}}%
\pgfpathlineto{\pgfqpoint{2.677159in}{3.470793in}}%
\pgfpathlineto{\pgfqpoint{2.788056in}{3.278715in}}%
\pgfpathlineto{\pgfqpoint{2.682103in}{3.345844in}}%
\pgfpathclose%
\pgfusepath{fill}%
\end{pgfscope}%
\begin{pgfscope}%
\pgfpathrectangle{\pgfqpoint{0.765000in}{0.660000in}}{\pgfqpoint{4.620000in}{4.620000in}}%
\pgfusepath{clip}%
\pgfsetbuttcap%
\pgfsetroundjoin%
\definecolor{currentfill}{rgb}{1.000000,0.894118,0.788235}%
\pgfsetfillcolor{currentfill}%
\pgfsetlinewidth{0.000000pt}%
\definecolor{currentstroke}{rgb}{1.000000,0.894118,0.788235}%
\pgfsetstrokecolor{currentstroke}%
\pgfsetdash{}{0pt}%
\pgfpathmoveto{\pgfqpoint{2.793000in}{3.281817in}}%
\pgfpathlineto{\pgfqpoint{2.793000in}{3.409870in}}%
\pgfpathlineto{\pgfqpoint{2.788056in}{3.406767in}}%
\pgfpathlineto{\pgfqpoint{2.677159in}{3.342741in}}%
\pgfpathlineto{\pgfqpoint{2.793000in}{3.281817in}}%
\pgfpathclose%
\pgfusepath{fill}%
\end{pgfscope}%
\begin{pgfscope}%
\pgfpathrectangle{\pgfqpoint{0.765000in}{0.660000in}}{\pgfqpoint{4.620000in}{4.620000in}}%
\pgfusepath{clip}%
\pgfsetbuttcap%
\pgfsetroundjoin%
\definecolor{currentfill}{rgb}{1.000000,0.894118,0.788235}%
\pgfsetfillcolor{currentfill}%
\pgfsetlinewidth{0.000000pt}%
\definecolor{currentstroke}{rgb}{1.000000,0.894118,0.788235}%
\pgfsetstrokecolor{currentstroke}%
\pgfsetdash{}{0pt}%
\pgfpathmoveto{\pgfqpoint{2.793000in}{3.409870in}}%
\pgfpathlineto{\pgfqpoint{2.903897in}{3.473896in}}%
\pgfpathlineto{\pgfqpoint{2.898952in}{3.470793in}}%
\pgfpathlineto{\pgfqpoint{2.788056in}{3.406767in}}%
\pgfpathlineto{\pgfqpoint{2.793000in}{3.409870in}}%
\pgfpathclose%
\pgfusepath{fill}%
\end{pgfscope}%
\begin{pgfscope}%
\pgfpathrectangle{\pgfqpoint{0.765000in}{0.660000in}}{\pgfqpoint{4.620000in}{4.620000in}}%
\pgfusepath{clip}%
\pgfsetbuttcap%
\pgfsetroundjoin%
\definecolor{currentfill}{rgb}{1.000000,0.894118,0.788235}%
\pgfsetfillcolor{currentfill}%
\pgfsetlinewidth{0.000000pt}%
\definecolor{currentstroke}{rgb}{1.000000,0.894118,0.788235}%
\pgfsetstrokecolor{currentstroke}%
\pgfsetdash{}{0pt}%
\pgfpathmoveto{\pgfqpoint{2.682103in}{3.473896in}}%
\pgfpathlineto{\pgfqpoint{2.793000in}{3.409870in}}%
\pgfpathlineto{\pgfqpoint{2.788056in}{3.406767in}}%
\pgfpathlineto{\pgfqpoint{2.677159in}{3.470793in}}%
\pgfpathlineto{\pgfqpoint{2.682103in}{3.473896in}}%
\pgfpathclose%
\pgfusepath{fill}%
\end{pgfscope}%
\begin{pgfscope}%
\pgfpathrectangle{\pgfqpoint{0.765000in}{0.660000in}}{\pgfqpoint{4.620000in}{4.620000in}}%
\pgfusepath{clip}%
\pgfsetbuttcap%
\pgfsetroundjoin%
\definecolor{currentfill}{rgb}{1.000000,0.894118,0.788235}%
\pgfsetfillcolor{currentfill}%
\pgfsetlinewidth{0.000000pt}%
\definecolor{currentstroke}{rgb}{1.000000,0.894118,0.788235}%
\pgfsetstrokecolor{currentstroke}%
\pgfsetdash{}{0pt}%
\pgfpathmoveto{\pgfqpoint{2.793000in}{3.409870in}}%
\pgfpathlineto{\pgfqpoint{2.682103in}{3.345844in}}%
\pgfpathlineto{\pgfqpoint{2.793000in}{3.281817in}}%
\pgfpathlineto{\pgfqpoint{2.903897in}{3.345844in}}%
\pgfpathlineto{\pgfqpoint{2.793000in}{3.409870in}}%
\pgfpathclose%
\pgfusepath{fill}%
\end{pgfscope}%
\begin{pgfscope}%
\pgfpathrectangle{\pgfqpoint{0.765000in}{0.660000in}}{\pgfqpoint{4.620000in}{4.620000in}}%
\pgfusepath{clip}%
\pgfsetbuttcap%
\pgfsetroundjoin%
\definecolor{currentfill}{rgb}{1.000000,0.894118,0.788235}%
\pgfsetfillcolor{currentfill}%
\pgfsetlinewidth{0.000000pt}%
\definecolor{currentstroke}{rgb}{1.000000,0.894118,0.788235}%
\pgfsetstrokecolor{currentstroke}%
\pgfsetdash{}{0pt}%
\pgfpathmoveto{\pgfqpoint{2.793000in}{3.409870in}}%
\pgfpathlineto{\pgfqpoint{2.682103in}{3.345844in}}%
\pgfpathlineto{\pgfqpoint{2.682103in}{3.473896in}}%
\pgfpathlineto{\pgfqpoint{2.793000in}{3.537922in}}%
\pgfpathlineto{\pgfqpoint{2.793000in}{3.409870in}}%
\pgfpathclose%
\pgfusepath{fill}%
\end{pgfscope}%
\begin{pgfscope}%
\pgfpathrectangle{\pgfqpoint{0.765000in}{0.660000in}}{\pgfqpoint{4.620000in}{4.620000in}}%
\pgfusepath{clip}%
\pgfsetbuttcap%
\pgfsetroundjoin%
\definecolor{currentfill}{rgb}{1.000000,0.894118,0.788235}%
\pgfsetfillcolor{currentfill}%
\pgfsetlinewidth{0.000000pt}%
\definecolor{currentstroke}{rgb}{1.000000,0.894118,0.788235}%
\pgfsetstrokecolor{currentstroke}%
\pgfsetdash{}{0pt}%
\pgfpathmoveto{\pgfqpoint{2.793000in}{3.409870in}}%
\pgfpathlineto{\pgfqpoint{2.903897in}{3.345844in}}%
\pgfpathlineto{\pgfqpoint{2.903897in}{3.473896in}}%
\pgfpathlineto{\pgfqpoint{2.793000in}{3.537922in}}%
\pgfpathlineto{\pgfqpoint{2.793000in}{3.409870in}}%
\pgfpathclose%
\pgfusepath{fill}%
\end{pgfscope}%
\begin{pgfscope}%
\pgfpathrectangle{\pgfqpoint{0.765000in}{0.660000in}}{\pgfqpoint{4.620000in}{4.620000in}}%
\pgfusepath{clip}%
\pgfsetbuttcap%
\pgfsetroundjoin%
\definecolor{currentfill}{rgb}{1.000000,0.894118,0.788235}%
\pgfsetfillcolor{currentfill}%
\pgfsetlinewidth{0.000000pt}%
\definecolor{currentstroke}{rgb}{1.000000,0.894118,0.788235}%
\pgfsetstrokecolor{currentstroke}%
\pgfsetdash{}{0pt}%
\pgfpathmoveto{\pgfqpoint{2.913379in}{3.479681in}}%
\pgfpathlineto{\pgfqpoint{2.802482in}{3.415655in}}%
\pgfpathlineto{\pgfqpoint{2.913379in}{3.351628in}}%
\pgfpathlineto{\pgfqpoint{3.024275in}{3.415655in}}%
\pgfpathlineto{\pgfqpoint{2.913379in}{3.479681in}}%
\pgfpathclose%
\pgfusepath{fill}%
\end{pgfscope}%
\begin{pgfscope}%
\pgfpathrectangle{\pgfqpoint{0.765000in}{0.660000in}}{\pgfqpoint{4.620000in}{4.620000in}}%
\pgfusepath{clip}%
\pgfsetbuttcap%
\pgfsetroundjoin%
\definecolor{currentfill}{rgb}{1.000000,0.894118,0.788235}%
\pgfsetfillcolor{currentfill}%
\pgfsetlinewidth{0.000000pt}%
\definecolor{currentstroke}{rgb}{1.000000,0.894118,0.788235}%
\pgfsetstrokecolor{currentstroke}%
\pgfsetdash{}{0pt}%
\pgfpathmoveto{\pgfqpoint{2.913379in}{3.479681in}}%
\pgfpathlineto{\pgfqpoint{2.802482in}{3.415655in}}%
\pgfpathlineto{\pgfqpoint{2.802482in}{3.543707in}}%
\pgfpathlineto{\pgfqpoint{2.913379in}{3.607733in}}%
\pgfpathlineto{\pgfqpoint{2.913379in}{3.479681in}}%
\pgfpathclose%
\pgfusepath{fill}%
\end{pgfscope}%
\begin{pgfscope}%
\pgfpathrectangle{\pgfqpoint{0.765000in}{0.660000in}}{\pgfqpoint{4.620000in}{4.620000in}}%
\pgfusepath{clip}%
\pgfsetbuttcap%
\pgfsetroundjoin%
\definecolor{currentfill}{rgb}{1.000000,0.894118,0.788235}%
\pgfsetfillcolor{currentfill}%
\pgfsetlinewidth{0.000000pt}%
\definecolor{currentstroke}{rgb}{1.000000,0.894118,0.788235}%
\pgfsetstrokecolor{currentstroke}%
\pgfsetdash{}{0pt}%
\pgfpathmoveto{\pgfqpoint{2.913379in}{3.479681in}}%
\pgfpathlineto{\pgfqpoint{3.024275in}{3.415655in}}%
\pgfpathlineto{\pgfqpoint{3.024275in}{3.543707in}}%
\pgfpathlineto{\pgfqpoint{2.913379in}{3.607733in}}%
\pgfpathlineto{\pgfqpoint{2.913379in}{3.479681in}}%
\pgfpathclose%
\pgfusepath{fill}%
\end{pgfscope}%
\begin{pgfscope}%
\pgfpathrectangle{\pgfqpoint{0.765000in}{0.660000in}}{\pgfqpoint{4.620000in}{4.620000in}}%
\pgfusepath{clip}%
\pgfsetbuttcap%
\pgfsetroundjoin%
\definecolor{currentfill}{rgb}{1.000000,0.894118,0.788235}%
\pgfsetfillcolor{currentfill}%
\pgfsetlinewidth{0.000000pt}%
\definecolor{currentstroke}{rgb}{1.000000,0.894118,0.788235}%
\pgfsetstrokecolor{currentstroke}%
\pgfsetdash{}{0pt}%
\pgfpathmoveto{\pgfqpoint{2.793000in}{3.537922in}}%
\pgfpathlineto{\pgfqpoint{2.682103in}{3.473896in}}%
\pgfpathlineto{\pgfqpoint{2.793000in}{3.409870in}}%
\pgfpathlineto{\pgfqpoint{2.903897in}{3.473896in}}%
\pgfpathlineto{\pgfqpoint{2.793000in}{3.537922in}}%
\pgfpathclose%
\pgfusepath{fill}%
\end{pgfscope}%
\begin{pgfscope}%
\pgfpathrectangle{\pgfqpoint{0.765000in}{0.660000in}}{\pgfqpoint{4.620000in}{4.620000in}}%
\pgfusepath{clip}%
\pgfsetbuttcap%
\pgfsetroundjoin%
\definecolor{currentfill}{rgb}{1.000000,0.894118,0.788235}%
\pgfsetfillcolor{currentfill}%
\pgfsetlinewidth{0.000000pt}%
\definecolor{currentstroke}{rgb}{1.000000,0.894118,0.788235}%
\pgfsetstrokecolor{currentstroke}%
\pgfsetdash{}{0pt}%
\pgfpathmoveto{\pgfqpoint{2.793000in}{3.281817in}}%
\pgfpathlineto{\pgfqpoint{2.903897in}{3.345844in}}%
\pgfpathlineto{\pgfqpoint{2.903897in}{3.473896in}}%
\pgfpathlineto{\pgfqpoint{2.793000in}{3.409870in}}%
\pgfpathlineto{\pgfqpoint{2.793000in}{3.281817in}}%
\pgfpathclose%
\pgfusepath{fill}%
\end{pgfscope}%
\begin{pgfscope}%
\pgfpathrectangle{\pgfqpoint{0.765000in}{0.660000in}}{\pgfqpoint{4.620000in}{4.620000in}}%
\pgfusepath{clip}%
\pgfsetbuttcap%
\pgfsetroundjoin%
\definecolor{currentfill}{rgb}{1.000000,0.894118,0.788235}%
\pgfsetfillcolor{currentfill}%
\pgfsetlinewidth{0.000000pt}%
\definecolor{currentstroke}{rgb}{1.000000,0.894118,0.788235}%
\pgfsetstrokecolor{currentstroke}%
\pgfsetdash{}{0pt}%
\pgfpathmoveto{\pgfqpoint{2.682103in}{3.345844in}}%
\pgfpathlineto{\pgfqpoint{2.793000in}{3.281817in}}%
\pgfpathlineto{\pgfqpoint{2.793000in}{3.409870in}}%
\pgfpathlineto{\pgfqpoint{2.682103in}{3.473896in}}%
\pgfpathlineto{\pgfqpoint{2.682103in}{3.345844in}}%
\pgfpathclose%
\pgfusepath{fill}%
\end{pgfscope}%
\begin{pgfscope}%
\pgfpathrectangle{\pgfqpoint{0.765000in}{0.660000in}}{\pgfqpoint{4.620000in}{4.620000in}}%
\pgfusepath{clip}%
\pgfsetbuttcap%
\pgfsetroundjoin%
\definecolor{currentfill}{rgb}{1.000000,0.894118,0.788235}%
\pgfsetfillcolor{currentfill}%
\pgfsetlinewidth{0.000000pt}%
\definecolor{currentstroke}{rgb}{1.000000,0.894118,0.788235}%
\pgfsetstrokecolor{currentstroke}%
\pgfsetdash{}{0pt}%
\pgfpathmoveto{\pgfqpoint{2.913379in}{3.607733in}}%
\pgfpathlineto{\pgfqpoint{2.802482in}{3.543707in}}%
\pgfpathlineto{\pgfqpoint{2.913379in}{3.479681in}}%
\pgfpathlineto{\pgfqpoint{3.024275in}{3.543707in}}%
\pgfpathlineto{\pgfqpoint{2.913379in}{3.607733in}}%
\pgfpathclose%
\pgfusepath{fill}%
\end{pgfscope}%
\begin{pgfscope}%
\pgfpathrectangle{\pgfqpoint{0.765000in}{0.660000in}}{\pgfqpoint{4.620000in}{4.620000in}}%
\pgfusepath{clip}%
\pgfsetbuttcap%
\pgfsetroundjoin%
\definecolor{currentfill}{rgb}{1.000000,0.894118,0.788235}%
\pgfsetfillcolor{currentfill}%
\pgfsetlinewidth{0.000000pt}%
\definecolor{currentstroke}{rgb}{1.000000,0.894118,0.788235}%
\pgfsetstrokecolor{currentstroke}%
\pgfsetdash{}{0pt}%
\pgfpathmoveto{\pgfqpoint{2.913379in}{3.351628in}}%
\pgfpathlineto{\pgfqpoint{3.024275in}{3.415655in}}%
\pgfpathlineto{\pgfqpoint{3.024275in}{3.543707in}}%
\pgfpathlineto{\pgfqpoint{2.913379in}{3.479681in}}%
\pgfpathlineto{\pgfqpoint{2.913379in}{3.351628in}}%
\pgfpathclose%
\pgfusepath{fill}%
\end{pgfscope}%
\begin{pgfscope}%
\pgfpathrectangle{\pgfqpoint{0.765000in}{0.660000in}}{\pgfqpoint{4.620000in}{4.620000in}}%
\pgfusepath{clip}%
\pgfsetbuttcap%
\pgfsetroundjoin%
\definecolor{currentfill}{rgb}{1.000000,0.894118,0.788235}%
\pgfsetfillcolor{currentfill}%
\pgfsetlinewidth{0.000000pt}%
\definecolor{currentstroke}{rgb}{1.000000,0.894118,0.788235}%
\pgfsetstrokecolor{currentstroke}%
\pgfsetdash{}{0pt}%
\pgfpathmoveto{\pgfqpoint{2.802482in}{3.415655in}}%
\pgfpathlineto{\pgfqpoint{2.913379in}{3.351628in}}%
\pgfpathlineto{\pgfqpoint{2.913379in}{3.479681in}}%
\pgfpathlineto{\pgfqpoint{2.802482in}{3.543707in}}%
\pgfpathlineto{\pgfqpoint{2.802482in}{3.415655in}}%
\pgfpathclose%
\pgfusepath{fill}%
\end{pgfscope}%
\begin{pgfscope}%
\pgfpathrectangle{\pgfqpoint{0.765000in}{0.660000in}}{\pgfqpoint{4.620000in}{4.620000in}}%
\pgfusepath{clip}%
\pgfsetbuttcap%
\pgfsetroundjoin%
\definecolor{currentfill}{rgb}{1.000000,0.894118,0.788235}%
\pgfsetfillcolor{currentfill}%
\pgfsetlinewidth{0.000000pt}%
\definecolor{currentstroke}{rgb}{1.000000,0.894118,0.788235}%
\pgfsetstrokecolor{currentstroke}%
\pgfsetdash{}{0pt}%
\pgfpathmoveto{\pgfqpoint{2.793000in}{3.409870in}}%
\pgfpathlineto{\pgfqpoint{2.682103in}{3.345844in}}%
\pgfpathlineto{\pgfqpoint{2.802482in}{3.415655in}}%
\pgfpathlineto{\pgfqpoint{2.913379in}{3.479681in}}%
\pgfpathlineto{\pgfqpoint{2.793000in}{3.409870in}}%
\pgfpathclose%
\pgfusepath{fill}%
\end{pgfscope}%
\begin{pgfscope}%
\pgfpathrectangle{\pgfqpoint{0.765000in}{0.660000in}}{\pgfqpoint{4.620000in}{4.620000in}}%
\pgfusepath{clip}%
\pgfsetbuttcap%
\pgfsetroundjoin%
\definecolor{currentfill}{rgb}{1.000000,0.894118,0.788235}%
\pgfsetfillcolor{currentfill}%
\pgfsetlinewidth{0.000000pt}%
\definecolor{currentstroke}{rgb}{1.000000,0.894118,0.788235}%
\pgfsetstrokecolor{currentstroke}%
\pgfsetdash{}{0pt}%
\pgfpathmoveto{\pgfqpoint{2.903897in}{3.345844in}}%
\pgfpathlineto{\pgfqpoint{2.793000in}{3.409870in}}%
\pgfpathlineto{\pgfqpoint{2.913379in}{3.479681in}}%
\pgfpathlineto{\pgfqpoint{3.024275in}{3.415655in}}%
\pgfpathlineto{\pgfqpoint{2.903897in}{3.345844in}}%
\pgfpathclose%
\pgfusepath{fill}%
\end{pgfscope}%
\begin{pgfscope}%
\pgfpathrectangle{\pgfqpoint{0.765000in}{0.660000in}}{\pgfqpoint{4.620000in}{4.620000in}}%
\pgfusepath{clip}%
\pgfsetbuttcap%
\pgfsetroundjoin%
\definecolor{currentfill}{rgb}{1.000000,0.894118,0.788235}%
\pgfsetfillcolor{currentfill}%
\pgfsetlinewidth{0.000000pt}%
\definecolor{currentstroke}{rgb}{1.000000,0.894118,0.788235}%
\pgfsetstrokecolor{currentstroke}%
\pgfsetdash{}{0pt}%
\pgfpathmoveto{\pgfqpoint{2.793000in}{3.409870in}}%
\pgfpathlineto{\pgfqpoint{2.793000in}{3.537922in}}%
\pgfpathlineto{\pgfqpoint{2.913379in}{3.607733in}}%
\pgfpathlineto{\pgfqpoint{3.024275in}{3.415655in}}%
\pgfpathlineto{\pgfqpoint{2.793000in}{3.409870in}}%
\pgfpathclose%
\pgfusepath{fill}%
\end{pgfscope}%
\begin{pgfscope}%
\pgfpathrectangle{\pgfqpoint{0.765000in}{0.660000in}}{\pgfqpoint{4.620000in}{4.620000in}}%
\pgfusepath{clip}%
\pgfsetbuttcap%
\pgfsetroundjoin%
\definecolor{currentfill}{rgb}{1.000000,0.894118,0.788235}%
\pgfsetfillcolor{currentfill}%
\pgfsetlinewidth{0.000000pt}%
\definecolor{currentstroke}{rgb}{1.000000,0.894118,0.788235}%
\pgfsetstrokecolor{currentstroke}%
\pgfsetdash{}{0pt}%
\pgfpathmoveto{\pgfqpoint{2.903897in}{3.345844in}}%
\pgfpathlineto{\pgfqpoint{2.903897in}{3.473896in}}%
\pgfpathlineto{\pgfqpoint{3.024275in}{3.543707in}}%
\pgfpathlineto{\pgfqpoint{2.913379in}{3.479681in}}%
\pgfpathlineto{\pgfqpoint{2.903897in}{3.345844in}}%
\pgfpathclose%
\pgfusepath{fill}%
\end{pgfscope}%
\begin{pgfscope}%
\pgfpathrectangle{\pgfqpoint{0.765000in}{0.660000in}}{\pgfqpoint{4.620000in}{4.620000in}}%
\pgfusepath{clip}%
\pgfsetbuttcap%
\pgfsetroundjoin%
\definecolor{currentfill}{rgb}{1.000000,0.894118,0.788235}%
\pgfsetfillcolor{currentfill}%
\pgfsetlinewidth{0.000000pt}%
\definecolor{currentstroke}{rgb}{1.000000,0.894118,0.788235}%
\pgfsetstrokecolor{currentstroke}%
\pgfsetdash{}{0pt}%
\pgfpathmoveto{\pgfqpoint{2.682103in}{3.345844in}}%
\pgfpathlineto{\pgfqpoint{2.793000in}{3.281817in}}%
\pgfpathlineto{\pgfqpoint{2.913379in}{3.351628in}}%
\pgfpathlineto{\pgfqpoint{2.802482in}{3.415655in}}%
\pgfpathlineto{\pgfqpoint{2.682103in}{3.345844in}}%
\pgfpathclose%
\pgfusepath{fill}%
\end{pgfscope}%
\begin{pgfscope}%
\pgfpathrectangle{\pgfqpoint{0.765000in}{0.660000in}}{\pgfqpoint{4.620000in}{4.620000in}}%
\pgfusepath{clip}%
\pgfsetbuttcap%
\pgfsetroundjoin%
\definecolor{currentfill}{rgb}{1.000000,0.894118,0.788235}%
\pgfsetfillcolor{currentfill}%
\pgfsetlinewidth{0.000000pt}%
\definecolor{currentstroke}{rgb}{1.000000,0.894118,0.788235}%
\pgfsetstrokecolor{currentstroke}%
\pgfsetdash{}{0pt}%
\pgfpathmoveto{\pgfqpoint{2.793000in}{3.281817in}}%
\pgfpathlineto{\pgfqpoint{2.903897in}{3.345844in}}%
\pgfpathlineto{\pgfqpoint{3.024275in}{3.415655in}}%
\pgfpathlineto{\pgfqpoint{2.913379in}{3.351628in}}%
\pgfpathlineto{\pgfqpoint{2.793000in}{3.281817in}}%
\pgfpathclose%
\pgfusepath{fill}%
\end{pgfscope}%
\begin{pgfscope}%
\pgfpathrectangle{\pgfqpoint{0.765000in}{0.660000in}}{\pgfqpoint{4.620000in}{4.620000in}}%
\pgfusepath{clip}%
\pgfsetbuttcap%
\pgfsetroundjoin%
\definecolor{currentfill}{rgb}{1.000000,0.894118,0.788235}%
\pgfsetfillcolor{currentfill}%
\pgfsetlinewidth{0.000000pt}%
\definecolor{currentstroke}{rgb}{1.000000,0.894118,0.788235}%
\pgfsetstrokecolor{currentstroke}%
\pgfsetdash{}{0pt}%
\pgfpathmoveto{\pgfqpoint{2.793000in}{3.537922in}}%
\pgfpathlineto{\pgfqpoint{2.682103in}{3.473896in}}%
\pgfpathlineto{\pgfqpoint{2.802482in}{3.543707in}}%
\pgfpathlineto{\pgfqpoint{2.913379in}{3.607733in}}%
\pgfpathlineto{\pgfqpoint{2.793000in}{3.537922in}}%
\pgfpathclose%
\pgfusepath{fill}%
\end{pgfscope}%
\begin{pgfscope}%
\pgfpathrectangle{\pgfqpoint{0.765000in}{0.660000in}}{\pgfqpoint{4.620000in}{4.620000in}}%
\pgfusepath{clip}%
\pgfsetbuttcap%
\pgfsetroundjoin%
\definecolor{currentfill}{rgb}{1.000000,0.894118,0.788235}%
\pgfsetfillcolor{currentfill}%
\pgfsetlinewidth{0.000000pt}%
\definecolor{currentstroke}{rgb}{1.000000,0.894118,0.788235}%
\pgfsetstrokecolor{currentstroke}%
\pgfsetdash{}{0pt}%
\pgfpathmoveto{\pgfqpoint{2.903897in}{3.473896in}}%
\pgfpathlineto{\pgfqpoint{2.793000in}{3.537922in}}%
\pgfpathlineto{\pgfqpoint{2.913379in}{3.607733in}}%
\pgfpathlineto{\pgfqpoint{3.024275in}{3.543707in}}%
\pgfpathlineto{\pgfqpoint{2.903897in}{3.473896in}}%
\pgfpathclose%
\pgfusepath{fill}%
\end{pgfscope}%
\begin{pgfscope}%
\pgfpathrectangle{\pgfqpoint{0.765000in}{0.660000in}}{\pgfqpoint{4.620000in}{4.620000in}}%
\pgfusepath{clip}%
\pgfsetbuttcap%
\pgfsetroundjoin%
\definecolor{currentfill}{rgb}{1.000000,0.894118,0.788235}%
\pgfsetfillcolor{currentfill}%
\pgfsetlinewidth{0.000000pt}%
\definecolor{currentstroke}{rgb}{1.000000,0.894118,0.788235}%
\pgfsetstrokecolor{currentstroke}%
\pgfsetdash{}{0pt}%
\pgfpathmoveto{\pgfqpoint{2.682103in}{3.345844in}}%
\pgfpathlineto{\pgfqpoint{2.682103in}{3.473896in}}%
\pgfpathlineto{\pgfqpoint{2.802482in}{3.543707in}}%
\pgfpathlineto{\pgfqpoint{2.913379in}{3.351628in}}%
\pgfpathlineto{\pgfqpoint{2.682103in}{3.345844in}}%
\pgfpathclose%
\pgfusepath{fill}%
\end{pgfscope}%
\begin{pgfscope}%
\pgfpathrectangle{\pgfqpoint{0.765000in}{0.660000in}}{\pgfqpoint{4.620000in}{4.620000in}}%
\pgfusepath{clip}%
\pgfsetbuttcap%
\pgfsetroundjoin%
\definecolor{currentfill}{rgb}{1.000000,0.894118,0.788235}%
\pgfsetfillcolor{currentfill}%
\pgfsetlinewidth{0.000000pt}%
\definecolor{currentstroke}{rgb}{1.000000,0.894118,0.788235}%
\pgfsetstrokecolor{currentstroke}%
\pgfsetdash{}{0pt}%
\pgfpathmoveto{\pgfqpoint{2.793000in}{3.281817in}}%
\pgfpathlineto{\pgfqpoint{2.793000in}{3.409870in}}%
\pgfpathlineto{\pgfqpoint{2.913379in}{3.479681in}}%
\pgfpathlineto{\pgfqpoint{2.802482in}{3.415655in}}%
\pgfpathlineto{\pgfqpoint{2.793000in}{3.281817in}}%
\pgfpathclose%
\pgfusepath{fill}%
\end{pgfscope}%
\begin{pgfscope}%
\pgfpathrectangle{\pgfqpoint{0.765000in}{0.660000in}}{\pgfqpoint{4.620000in}{4.620000in}}%
\pgfusepath{clip}%
\pgfsetbuttcap%
\pgfsetroundjoin%
\definecolor{currentfill}{rgb}{1.000000,0.894118,0.788235}%
\pgfsetfillcolor{currentfill}%
\pgfsetlinewidth{0.000000pt}%
\definecolor{currentstroke}{rgb}{1.000000,0.894118,0.788235}%
\pgfsetstrokecolor{currentstroke}%
\pgfsetdash{}{0pt}%
\pgfpathmoveto{\pgfqpoint{2.793000in}{3.409870in}}%
\pgfpathlineto{\pgfqpoint{2.903897in}{3.473896in}}%
\pgfpathlineto{\pgfqpoint{3.024275in}{3.543707in}}%
\pgfpathlineto{\pgfqpoint{2.913379in}{3.479681in}}%
\pgfpathlineto{\pgfqpoint{2.793000in}{3.409870in}}%
\pgfpathclose%
\pgfusepath{fill}%
\end{pgfscope}%
\begin{pgfscope}%
\pgfpathrectangle{\pgfqpoint{0.765000in}{0.660000in}}{\pgfqpoint{4.620000in}{4.620000in}}%
\pgfusepath{clip}%
\pgfsetbuttcap%
\pgfsetroundjoin%
\definecolor{currentfill}{rgb}{1.000000,0.894118,0.788235}%
\pgfsetfillcolor{currentfill}%
\pgfsetlinewidth{0.000000pt}%
\definecolor{currentstroke}{rgb}{1.000000,0.894118,0.788235}%
\pgfsetstrokecolor{currentstroke}%
\pgfsetdash{}{0pt}%
\pgfpathmoveto{\pgfqpoint{2.682103in}{3.473896in}}%
\pgfpathlineto{\pgfqpoint{2.793000in}{3.409870in}}%
\pgfpathlineto{\pgfqpoint{2.913379in}{3.479681in}}%
\pgfpathlineto{\pgfqpoint{2.802482in}{3.543707in}}%
\pgfpathlineto{\pgfqpoint{2.682103in}{3.473896in}}%
\pgfpathclose%
\pgfusepath{fill}%
\end{pgfscope}%
\begin{pgfscope}%
\pgfpathrectangle{\pgfqpoint{0.765000in}{0.660000in}}{\pgfqpoint{4.620000in}{4.620000in}}%
\pgfusepath{clip}%
\pgfsetbuttcap%
\pgfsetroundjoin%
\definecolor{currentfill}{rgb}{1.000000,0.894118,0.788235}%
\pgfsetfillcolor{currentfill}%
\pgfsetlinewidth{0.000000pt}%
\definecolor{currentstroke}{rgb}{1.000000,0.894118,0.788235}%
\pgfsetstrokecolor{currentstroke}%
\pgfsetdash{}{0pt}%
\pgfpathmoveto{\pgfqpoint{2.788056in}{3.406767in}}%
\pgfpathlineto{\pgfqpoint{2.677159in}{3.342741in}}%
\pgfpathlineto{\pgfqpoint{2.788056in}{3.278715in}}%
\pgfpathlineto{\pgfqpoint{2.898952in}{3.342741in}}%
\pgfpathlineto{\pgfqpoint{2.788056in}{3.406767in}}%
\pgfpathclose%
\pgfusepath{fill}%
\end{pgfscope}%
\begin{pgfscope}%
\pgfpathrectangle{\pgfqpoint{0.765000in}{0.660000in}}{\pgfqpoint{4.620000in}{4.620000in}}%
\pgfusepath{clip}%
\pgfsetbuttcap%
\pgfsetroundjoin%
\definecolor{currentfill}{rgb}{1.000000,0.894118,0.788235}%
\pgfsetfillcolor{currentfill}%
\pgfsetlinewidth{0.000000pt}%
\definecolor{currentstroke}{rgb}{1.000000,0.894118,0.788235}%
\pgfsetstrokecolor{currentstroke}%
\pgfsetdash{}{0pt}%
\pgfpathmoveto{\pgfqpoint{2.788056in}{3.406767in}}%
\pgfpathlineto{\pgfqpoint{2.677159in}{3.342741in}}%
\pgfpathlineto{\pgfqpoint{2.677159in}{3.470793in}}%
\pgfpathlineto{\pgfqpoint{2.788056in}{3.534820in}}%
\pgfpathlineto{\pgfqpoint{2.788056in}{3.406767in}}%
\pgfpathclose%
\pgfusepath{fill}%
\end{pgfscope}%
\begin{pgfscope}%
\pgfpathrectangle{\pgfqpoint{0.765000in}{0.660000in}}{\pgfqpoint{4.620000in}{4.620000in}}%
\pgfusepath{clip}%
\pgfsetbuttcap%
\pgfsetroundjoin%
\definecolor{currentfill}{rgb}{1.000000,0.894118,0.788235}%
\pgfsetfillcolor{currentfill}%
\pgfsetlinewidth{0.000000pt}%
\definecolor{currentstroke}{rgb}{1.000000,0.894118,0.788235}%
\pgfsetstrokecolor{currentstroke}%
\pgfsetdash{}{0pt}%
\pgfpathmoveto{\pgfqpoint{2.788056in}{3.406767in}}%
\pgfpathlineto{\pgfqpoint{2.898952in}{3.342741in}}%
\pgfpathlineto{\pgfqpoint{2.898952in}{3.470793in}}%
\pgfpathlineto{\pgfqpoint{2.788056in}{3.534820in}}%
\pgfpathlineto{\pgfqpoint{2.788056in}{3.406767in}}%
\pgfpathclose%
\pgfusepath{fill}%
\end{pgfscope}%
\begin{pgfscope}%
\pgfpathrectangle{\pgfqpoint{0.765000in}{0.660000in}}{\pgfqpoint{4.620000in}{4.620000in}}%
\pgfusepath{clip}%
\pgfsetbuttcap%
\pgfsetroundjoin%
\definecolor{currentfill}{rgb}{1.000000,0.894118,0.788235}%
\pgfsetfillcolor{currentfill}%
\pgfsetlinewidth{0.000000pt}%
\definecolor{currentstroke}{rgb}{1.000000,0.894118,0.788235}%
\pgfsetstrokecolor{currentstroke}%
\pgfsetdash{}{0pt}%
\pgfpathmoveto{\pgfqpoint{2.708745in}{3.356240in}}%
\pgfpathlineto{\pgfqpoint{2.597849in}{3.292213in}}%
\pgfpathlineto{\pgfqpoint{2.708745in}{3.228187in}}%
\pgfpathlineto{\pgfqpoint{2.819642in}{3.292213in}}%
\pgfpathlineto{\pgfqpoint{2.708745in}{3.356240in}}%
\pgfpathclose%
\pgfusepath{fill}%
\end{pgfscope}%
\begin{pgfscope}%
\pgfpathrectangle{\pgfqpoint{0.765000in}{0.660000in}}{\pgfqpoint{4.620000in}{4.620000in}}%
\pgfusepath{clip}%
\pgfsetbuttcap%
\pgfsetroundjoin%
\definecolor{currentfill}{rgb}{1.000000,0.894118,0.788235}%
\pgfsetfillcolor{currentfill}%
\pgfsetlinewidth{0.000000pt}%
\definecolor{currentstroke}{rgb}{1.000000,0.894118,0.788235}%
\pgfsetstrokecolor{currentstroke}%
\pgfsetdash{}{0pt}%
\pgfpathmoveto{\pgfqpoint{2.708745in}{3.356240in}}%
\pgfpathlineto{\pgfqpoint{2.597849in}{3.292213in}}%
\pgfpathlineto{\pgfqpoint{2.597849in}{3.420266in}}%
\pgfpathlineto{\pgfqpoint{2.708745in}{3.484292in}}%
\pgfpathlineto{\pgfqpoint{2.708745in}{3.356240in}}%
\pgfpathclose%
\pgfusepath{fill}%
\end{pgfscope}%
\begin{pgfscope}%
\pgfpathrectangle{\pgfqpoint{0.765000in}{0.660000in}}{\pgfqpoint{4.620000in}{4.620000in}}%
\pgfusepath{clip}%
\pgfsetbuttcap%
\pgfsetroundjoin%
\definecolor{currentfill}{rgb}{1.000000,0.894118,0.788235}%
\pgfsetfillcolor{currentfill}%
\pgfsetlinewidth{0.000000pt}%
\definecolor{currentstroke}{rgb}{1.000000,0.894118,0.788235}%
\pgfsetstrokecolor{currentstroke}%
\pgfsetdash{}{0pt}%
\pgfpathmoveto{\pgfqpoint{2.708745in}{3.356240in}}%
\pgfpathlineto{\pgfqpoint{2.819642in}{3.292213in}}%
\pgfpathlineto{\pgfqpoint{2.819642in}{3.420266in}}%
\pgfpathlineto{\pgfqpoint{2.708745in}{3.484292in}}%
\pgfpathlineto{\pgfqpoint{2.708745in}{3.356240in}}%
\pgfpathclose%
\pgfusepath{fill}%
\end{pgfscope}%
\begin{pgfscope}%
\pgfpathrectangle{\pgfqpoint{0.765000in}{0.660000in}}{\pgfqpoint{4.620000in}{4.620000in}}%
\pgfusepath{clip}%
\pgfsetbuttcap%
\pgfsetroundjoin%
\definecolor{currentfill}{rgb}{1.000000,0.894118,0.788235}%
\pgfsetfillcolor{currentfill}%
\pgfsetlinewidth{0.000000pt}%
\definecolor{currentstroke}{rgb}{1.000000,0.894118,0.788235}%
\pgfsetstrokecolor{currentstroke}%
\pgfsetdash{}{0pt}%
\pgfpathmoveto{\pgfqpoint{2.788056in}{3.534820in}}%
\pgfpathlineto{\pgfqpoint{2.677159in}{3.470793in}}%
\pgfpathlineto{\pgfqpoint{2.788056in}{3.406767in}}%
\pgfpathlineto{\pgfqpoint{2.898952in}{3.470793in}}%
\pgfpathlineto{\pgfqpoint{2.788056in}{3.534820in}}%
\pgfpathclose%
\pgfusepath{fill}%
\end{pgfscope}%
\begin{pgfscope}%
\pgfpathrectangle{\pgfqpoint{0.765000in}{0.660000in}}{\pgfqpoint{4.620000in}{4.620000in}}%
\pgfusepath{clip}%
\pgfsetbuttcap%
\pgfsetroundjoin%
\definecolor{currentfill}{rgb}{1.000000,0.894118,0.788235}%
\pgfsetfillcolor{currentfill}%
\pgfsetlinewidth{0.000000pt}%
\definecolor{currentstroke}{rgb}{1.000000,0.894118,0.788235}%
\pgfsetstrokecolor{currentstroke}%
\pgfsetdash{}{0pt}%
\pgfpathmoveto{\pgfqpoint{2.788056in}{3.278715in}}%
\pgfpathlineto{\pgfqpoint{2.898952in}{3.342741in}}%
\pgfpathlineto{\pgfqpoint{2.898952in}{3.470793in}}%
\pgfpathlineto{\pgfqpoint{2.788056in}{3.406767in}}%
\pgfpathlineto{\pgfqpoint{2.788056in}{3.278715in}}%
\pgfpathclose%
\pgfusepath{fill}%
\end{pgfscope}%
\begin{pgfscope}%
\pgfpathrectangle{\pgfqpoint{0.765000in}{0.660000in}}{\pgfqpoint{4.620000in}{4.620000in}}%
\pgfusepath{clip}%
\pgfsetbuttcap%
\pgfsetroundjoin%
\definecolor{currentfill}{rgb}{1.000000,0.894118,0.788235}%
\pgfsetfillcolor{currentfill}%
\pgfsetlinewidth{0.000000pt}%
\definecolor{currentstroke}{rgb}{1.000000,0.894118,0.788235}%
\pgfsetstrokecolor{currentstroke}%
\pgfsetdash{}{0pt}%
\pgfpathmoveto{\pgfqpoint{2.677159in}{3.342741in}}%
\pgfpathlineto{\pgfqpoint{2.788056in}{3.278715in}}%
\pgfpathlineto{\pgfqpoint{2.788056in}{3.406767in}}%
\pgfpathlineto{\pgfqpoint{2.677159in}{3.470793in}}%
\pgfpathlineto{\pgfqpoint{2.677159in}{3.342741in}}%
\pgfpathclose%
\pgfusepath{fill}%
\end{pgfscope}%
\begin{pgfscope}%
\pgfpathrectangle{\pgfqpoint{0.765000in}{0.660000in}}{\pgfqpoint{4.620000in}{4.620000in}}%
\pgfusepath{clip}%
\pgfsetbuttcap%
\pgfsetroundjoin%
\definecolor{currentfill}{rgb}{1.000000,0.894118,0.788235}%
\pgfsetfillcolor{currentfill}%
\pgfsetlinewidth{0.000000pt}%
\definecolor{currentstroke}{rgb}{1.000000,0.894118,0.788235}%
\pgfsetstrokecolor{currentstroke}%
\pgfsetdash{}{0pt}%
\pgfpathmoveto{\pgfqpoint{2.708745in}{3.484292in}}%
\pgfpathlineto{\pgfqpoint{2.597849in}{3.420266in}}%
\pgfpathlineto{\pgfqpoint{2.708745in}{3.356240in}}%
\pgfpathlineto{\pgfqpoint{2.819642in}{3.420266in}}%
\pgfpathlineto{\pgfqpoint{2.708745in}{3.484292in}}%
\pgfpathclose%
\pgfusepath{fill}%
\end{pgfscope}%
\begin{pgfscope}%
\pgfpathrectangle{\pgfqpoint{0.765000in}{0.660000in}}{\pgfqpoint{4.620000in}{4.620000in}}%
\pgfusepath{clip}%
\pgfsetbuttcap%
\pgfsetroundjoin%
\definecolor{currentfill}{rgb}{1.000000,0.894118,0.788235}%
\pgfsetfillcolor{currentfill}%
\pgfsetlinewidth{0.000000pt}%
\definecolor{currentstroke}{rgb}{1.000000,0.894118,0.788235}%
\pgfsetstrokecolor{currentstroke}%
\pgfsetdash{}{0pt}%
\pgfpathmoveto{\pgfqpoint{2.708745in}{3.228187in}}%
\pgfpathlineto{\pgfqpoint{2.819642in}{3.292213in}}%
\pgfpathlineto{\pgfqpoint{2.819642in}{3.420266in}}%
\pgfpathlineto{\pgfqpoint{2.708745in}{3.356240in}}%
\pgfpathlineto{\pgfqpoint{2.708745in}{3.228187in}}%
\pgfpathclose%
\pgfusepath{fill}%
\end{pgfscope}%
\begin{pgfscope}%
\pgfpathrectangle{\pgfqpoint{0.765000in}{0.660000in}}{\pgfqpoint{4.620000in}{4.620000in}}%
\pgfusepath{clip}%
\pgfsetbuttcap%
\pgfsetroundjoin%
\definecolor{currentfill}{rgb}{1.000000,0.894118,0.788235}%
\pgfsetfillcolor{currentfill}%
\pgfsetlinewidth{0.000000pt}%
\definecolor{currentstroke}{rgb}{1.000000,0.894118,0.788235}%
\pgfsetstrokecolor{currentstroke}%
\pgfsetdash{}{0pt}%
\pgfpathmoveto{\pgfqpoint{2.597849in}{3.292213in}}%
\pgfpathlineto{\pgfqpoint{2.708745in}{3.228187in}}%
\pgfpathlineto{\pgfqpoint{2.708745in}{3.356240in}}%
\pgfpathlineto{\pgfqpoint{2.597849in}{3.420266in}}%
\pgfpathlineto{\pgfqpoint{2.597849in}{3.292213in}}%
\pgfpathclose%
\pgfusepath{fill}%
\end{pgfscope}%
\begin{pgfscope}%
\pgfpathrectangle{\pgfqpoint{0.765000in}{0.660000in}}{\pgfqpoint{4.620000in}{4.620000in}}%
\pgfusepath{clip}%
\pgfsetbuttcap%
\pgfsetroundjoin%
\definecolor{currentfill}{rgb}{1.000000,0.894118,0.788235}%
\pgfsetfillcolor{currentfill}%
\pgfsetlinewidth{0.000000pt}%
\definecolor{currentstroke}{rgb}{1.000000,0.894118,0.788235}%
\pgfsetstrokecolor{currentstroke}%
\pgfsetdash{}{0pt}%
\pgfpathmoveto{\pgfqpoint{2.788056in}{3.406767in}}%
\pgfpathlineto{\pgfqpoint{2.677159in}{3.342741in}}%
\pgfpathlineto{\pgfqpoint{2.597849in}{3.292213in}}%
\pgfpathlineto{\pgfqpoint{2.708745in}{3.356240in}}%
\pgfpathlineto{\pgfqpoint{2.788056in}{3.406767in}}%
\pgfpathclose%
\pgfusepath{fill}%
\end{pgfscope}%
\begin{pgfscope}%
\pgfpathrectangle{\pgfqpoint{0.765000in}{0.660000in}}{\pgfqpoint{4.620000in}{4.620000in}}%
\pgfusepath{clip}%
\pgfsetbuttcap%
\pgfsetroundjoin%
\definecolor{currentfill}{rgb}{1.000000,0.894118,0.788235}%
\pgfsetfillcolor{currentfill}%
\pgfsetlinewidth{0.000000pt}%
\definecolor{currentstroke}{rgb}{1.000000,0.894118,0.788235}%
\pgfsetstrokecolor{currentstroke}%
\pgfsetdash{}{0pt}%
\pgfpathmoveto{\pgfqpoint{2.898952in}{3.342741in}}%
\pgfpathlineto{\pgfqpoint{2.788056in}{3.406767in}}%
\pgfpathlineto{\pgfqpoint{2.708745in}{3.356240in}}%
\pgfpathlineto{\pgfqpoint{2.819642in}{3.292213in}}%
\pgfpathlineto{\pgfqpoint{2.898952in}{3.342741in}}%
\pgfpathclose%
\pgfusepath{fill}%
\end{pgfscope}%
\begin{pgfscope}%
\pgfpathrectangle{\pgfqpoint{0.765000in}{0.660000in}}{\pgfqpoint{4.620000in}{4.620000in}}%
\pgfusepath{clip}%
\pgfsetbuttcap%
\pgfsetroundjoin%
\definecolor{currentfill}{rgb}{1.000000,0.894118,0.788235}%
\pgfsetfillcolor{currentfill}%
\pgfsetlinewidth{0.000000pt}%
\definecolor{currentstroke}{rgb}{1.000000,0.894118,0.788235}%
\pgfsetstrokecolor{currentstroke}%
\pgfsetdash{}{0pt}%
\pgfpathmoveto{\pgfqpoint{2.788056in}{3.406767in}}%
\pgfpathlineto{\pgfqpoint{2.788056in}{3.534820in}}%
\pgfpathlineto{\pgfqpoint{2.708745in}{3.484292in}}%
\pgfpathlineto{\pgfqpoint{2.819642in}{3.292213in}}%
\pgfpathlineto{\pgfqpoint{2.788056in}{3.406767in}}%
\pgfpathclose%
\pgfusepath{fill}%
\end{pgfscope}%
\begin{pgfscope}%
\pgfpathrectangle{\pgfqpoint{0.765000in}{0.660000in}}{\pgfqpoint{4.620000in}{4.620000in}}%
\pgfusepath{clip}%
\pgfsetbuttcap%
\pgfsetroundjoin%
\definecolor{currentfill}{rgb}{1.000000,0.894118,0.788235}%
\pgfsetfillcolor{currentfill}%
\pgfsetlinewidth{0.000000pt}%
\definecolor{currentstroke}{rgb}{1.000000,0.894118,0.788235}%
\pgfsetstrokecolor{currentstroke}%
\pgfsetdash{}{0pt}%
\pgfpathmoveto{\pgfqpoint{2.898952in}{3.342741in}}%
\pgfpathlineto{\pgfqpoint{2.898952in}{3.470793in}}%
\pgfpathlineto{\pgfqpoint{2.819642in}{3.420266in}}%
\pgfpathlineto{\pgfqpoint{2.708745in}{3.356240in}}%
\pgfpathlineto{\pgfqpoint{2.898952in}{3.342741in}}%
\pgfpathclose%
\pgfusepath{fill}%
\end{pgfscope}%
\begin{pgfscope}%
\pgfpathrectangle{\pgfqpoint{0.765000in}{0.660000in}}{\pgfqpoint{4.620000in}{4.620000in}}%
\pgfusepath{clip}%
\pgfsetbuttcap%
\pgfsetroundjoin%
\definecolor{currentfill}{rgb}{1.000000,0.894118,0.788235}%
\pgfsetfillcolor{currentfill}%
\pgfsetlinewidth{0.000000pt}%
\definecolor{currentstroke}{rgb}{1.000000,0.894118,0.788235}%
\pgfsetstrokecolor{currentstroke}%
\pgfsetdash{}{0pt}%
\pgfpathmoveto{\pgfqpoint{2.677159in}{3.342741in}}%
\pgfpathlineto{\pgfqpoint{2.788056in}{3.278715in}}%
\pgfpathlineto{\pgfqpoint{2.708745in}{3.228187in}}%
\pgfpathlineto{\pgfqpoint{2.597849in}{3.292213in}}%
\pgfpathlineto{\pgfqpoint{2.677159in}{3.342741in}}%
\pgfpathclose%
\pgfusepath{fill}%
\end{pgfscope}%
\begin{pgfscope}%
\pgfpathrectangle{\pgfqpoint{0.765000in}{0.660000in}}{\pgfqpoint{4.620000in}{4.620000in}}%
\pgfusepath{clip}%
\pgfsetbuttcap%
\pgfsetroundjoin%
\definecolor{currentfill}{rgb}{1.000000,0.894118,0.788235}%
\pgfsetfillcolor{currentfill}%
\pgfsetlinewidth{0.000000pt}%
\definecolor{currentstroke}{rgb}{1.000000,0.894118,0.788235}%
\pgfsetstrokecolor{currentstroke}%
\pgfsetdash{}{0pt}%
\pgfpathmoveto{\pgfqpoint{2.788056in}{3.278715in}}%
\pgfpathlineto{\pgfqpoint{2.898952in}{3.342741in}}%
\pgfpathlineto{\pgfqpoint{2.819642in}{3.292213in}}%
\pgfpathlineto{\pgfqpoint{2.708745in}{3.228187in}}%
\pgfpathlineto{\pgfqpoint{2.788056in}{3.278715in}}%
\pgfpathclose%
\pgfusepath{fill}%
\end{pgfscope}%
\begin{pgfscope}%
\pgfpathrectangle{\pgfqpoint{0.765000in}{0.660000in}}{\pgfqpoint{4.620000in}{4.620000in}}%
\pgfusepath{clip}%
\pgfsetbuttcap%
\pgfsetroundjoin%
\definecolor{currentfill}{rgb}{1.000000,0.894118,0.788235}%
\pgfsetfillcolor{currentfill}%
\pgfsetlinewidth{0.000000pt}%
\definecolor{currentstroke}{rgb}{1.000000,0.894118,0.788235}%
\pgfsetstrokecolor{currentstroke}%
\pgfsetdash{}{0pt}%
\pgfpathmoveto{\pgfqpoint{2.788056in}{3.534820in}}%
\pgfpathlineto{\pgfqpoint{2.677159in}{3.470793in}}%
\pgfpathlineto{\pgfqpoint{2.597849in}{3.420266in}}%
\pgfpathlineto{\pgfqpoint{2.708745in}{3.484292in}}%
\pgfpathlineto{\pgfqpoint{2.788056in}{3.534820in}}%
\pgfpathclose%
\pgfusepath{fill}%
\end{pgfscope}%
\begin{pgfscope}%
\pgfpathrectangle{\pgfqpoint{0.765000in}{0.660000in}}{\pgfqpoint{4.620000in}{4.620000in}}%
\pgfusepath{clip}%
\pgfsetbuttcap%
\pgfsetroundjoin%
\definecolor{currentfill}{rgb}{1.000000,0.894118,0.788235}%
\pgfsetfillcolor{currentfill}%
\pgfsetlinewidth{0.000000pt}%
\definecolor{currentstroke}{rgb}{1.000000,0.894118,0.788235}%
\pgfsetstrokecolor{currentstroke}%
\pgfsetdash{}{0pt}%
\pgfpathmoveto{\pgfqpoint{2.898952in}{3.470793in}}%
\pgfpathlineto{\pgfqpoint{2.788056in}{3.534820in}}%
\pgfpathlineto{\pgfqpoint{2.708745in}{3.484292in}}%
\pgfpathlineto{\pgfqpoint{2.819642in}{3.420266in}}%
\pgfpathlineto{\pgfqpoint{2.898952in}{3.470793in}}%
\pgfpathclose%
\pgfusepath{fill}%
\end{pgfscope}%
\begin{pgfscope}%
\pgfpathrectangle{\pgfqpoint{0.765000in}{0.660000in}}{\pgfqpoint{4.620000in}{4.620000in}}%
\pgfusepath{clip}%
\pgfsetbuttcap%
\pgfsetroundjoin%
\definecolor{currentfill}{rgb}{1.000000,0.894118,0.788235}%
\pgfsetfillcolor{currentfill}%
\pgfsetlinewidth{0.000000pt}%
\definecolor{currentstroke}{rgb}{1.000000,0.894118,0.788235}%
\pgfsetstrokecolor{currentstroke}%
\pgfsetdash{}{0pt}%
\pgfpathmoveto{\pgfqpoint{2.677159in}{3.342741in}}%
\pgfpathlineto{\pgfqpoint{2.677159in}{3.470793in}}%
\pgfpathlineto{\pgfqpoint{2.597849in}{3.420266in}}%
\pgfpathlineto{\pgfqpoint{2.708745in}{3.228187in}}%
\pgfpathlineto{\pgfqpoint{2.677159in}{3.342741in}}%
\pgfpathclose%
\pgfusepath{fill}%
\end{pgfscope}%
\begin{pgfscope}%
\pgfpathrectangle{\pgfqpoint{0.765000in}{0.660000in}}{\pgfqpoint{4.620000in}{4.620000in}}%
\pgfusepath{clip}%
\pgfsetbuttcap%
\pgfsetroundjoin%
\definecolor{currentfill}{rgb}{1.000000,0.894118,0.788235}%
\pgfsetfillcolor{currentfill}%
\pgfsetlinewidth{0.000000pt}%
\definecolor{currentstroke}{rgb}{1.000000,0.894118,0.788235}%
\pgfsetstrokecolor{currentstroke}%
\pgfsetdash{}{0pt}%
\pgfpathmoveto{\pgfqpoint{2.788056in}{3.278715in}}%
\pgfpathlineto{\pgfqpoint{2.788056in}{3.406767in}}%
\pgfpathlineto{\pgfqpoint{2.708745in}{3.356240in}}%
\pgfpathlineto{\pgfqpoint{2.597849in}{3.292213in}}%
\pgfpathlineto{\pgfqpoint{2.788056in}{3.278715in}}%
\pgfpathclose%
\pgfusepath{fill}%
\end{pgfscope}%
\begin{pgfscope}%
\pgfpathrectangle{\pgfqpoint{0.765000in}{0.660000in}}{\pgfqpoint{4.620000in}{4.620000in}}%
\pgfusepath{clip}%
\pgfsetbuttcap%
\pgfsetroundjoin%
\definecolor{currentfill}{rgb}{1.000000,0.894118,0.788235}%
\pgfsetfillcolor{currentfill}%
\pgfsetlinewidth{0.000000pt}%
\definecolor{currentstroke}{rgb}{1.000000,0.894118,0.788235}%
\pgfsetstrokecolor{currentstroke}%
\pgfsetdash{}{0pt}%
\pgfpathmoveto{\pgfqpoint{2.788056in}{3.406767in}}%
\pgfpathlineto{\pgfqpoint{2.898952in}{3.470793in}}%
\pgfpathlineto{\pgfqpoint{2.819642in}{3.420266in}}%
\pgfpathlineto{\pgfqpoint{2.708745in}{3.356240in}}%
\pgfpathlineto{\pgfqpoint{2.788056in}{3.406767in}}%
\pgfpathclose%
\pgfusepath{fill}%
\end{pgfscope}%
\begin{pgfscope}%
\pgfpathrectangle{\pgfqpoint{0.765000in}{0.660000in}}{\pgfqpoint{4.620000in}{4.620000in}}%
\pgfusepath{clip}%
\pgfsetbuttcap%
\pgfsetroundjoin%
\definecolor{currentfill}{rgb}{1.000000,0.894118,0.788235}%
\pgfsetfillcolor{currentfill}%
\pgfsetlinewidth{0.000000pt}%
\definecolor{currentstroke}{rgb}{1.000000,0.894118,0.788235}%
\pgfsetstrokecolor{currentstroke}%
\pgfsetdash{}{0pt}%
\pgfpathmoveto{\pgfqpoint{2.677159in}{3.470793in}}%
\pgfpathlineto{\pgfqpoint{2.788056in}{3.406767in}}%
\pgfpathlineto{\pgfqpoint{2.708745in}{3.356240in}}%
\pgfpathlineto{\pgfqpoint{2.597849in}{3.420266in}}%
\pgfpathlineto{\pgfqpoint{2.677159in}{3.470793in}}%
\pgfpathclose%
\pgfusepath{fill}%
\end{pgfscope}%
\begin{pgfscope}%
\pgfpathrectangle{\pgfqpoint{0.765000in}{0.660000in}}{\pgfqpoint{4.620000in}{4.620000in}}%
\pgfusepath{clip}%
\pgfsetbuttcap%
\pgfsetroundjoin%
\definecolor{currentfill}{rgb}{1.000000,0.894118,0.788235}%
\pgfsetfillcolor{currentfill}%
\pgfsetlinewidth{0.000000pt}%
\definecolor{currentstroke}{rgb}{1.000000,0.894118,0.788235}%
\pgfsetstrokecolor{currentstroke}%
\pgfsetdash{}{0pt}%
\pgfpathmoveto{\pgfqpoint{2.788056in}{3.406767in}}%
\pgfpathlineto{\pgfqpoint{2.677159in}{3.342741in}}%
\pgfpathlineto{\pgfqpoint{2.788056in}{3.278715in}}%
\pgfpathlineto{\pgfqpoint{2.898952in}{3.342741in}}%
\pgfpathlineto{\pgfqpoint{2.788056in}{3.406767in}}%
\pgfpathclose%
\pgfusepath{fill}%
\end{pgfscope}%
\begin{pgfscope}%
\pgfpathrectangle{\pgfqpoint{0.765000in}{0.660000in}}{\pgfqpoint{4.620000in}{4.620000in}}%
\pgfusepath{clip}%
\pgfsetbuttcap%
\pgfsetroundjoin%
\definecolor{currentfill}{rgb}{1.000000,0.894118,0.788235}%
\pgfsetfillcolor{currentfill}%
\pgfsetlinewidth{0.000000pt}%
\definecolor{currentstroke}{rgb}{1.000000,0.894118,0.788235}%
\pgfsetstrokecolor{currentstroke}%
\pgfsetdash{}{0pt}%
\pgfpathmoveto{\pgfqpoint{2.788056in}{3.406767in}}%
\pgfpathlineto{\pgfqpoint{2.677159in}{3.342741in}}%
\pgfpathlineto{\pgfqpoint{2.677159in}{3.470793in}}%
\pgfpathlineto{\pgfqpoint{2.788056in}{3.534820in}}%
\pgfpathlineto{\pgfqpoint{2.788056in}{3.406767in}}%
\pgfpathclose%
\pgfusepath{fill}%
\end{pgfscope}%
\begin{pgfscope}%
\pgfpathrectangle{\pgfqpoint{0.765000in}{0.660000in}}{\pgfqpoint{4.620000in}{4.620000in}}%
\pgfusepath{clip}%
\pgfsetbuttcap%
\pgfsetroundjoin%
\definecolor{currentfill}{rgb}{1.000000,0.894118,0.788235}%
\pgfsetfillcolor{currentfill}%
\pgfsetlinewidth{0.000000pt}%
\definecolor{currentstroke}{rgb}{1.000000,0.894118,0.788235}%
\pgfsetstrokecolor{currentstroke}%
\pgfsetdash{}{0pt}%
\pgfpathmoveto{\pgfqpoint{2.788056in}{3.406767in}}%
\pgfpathlineto{\pgfqpoint{2.898952in}{3.342741in}}%
\pgfpathlineto{\pgfqpoint{2.898952in}{3.470793in}}%
\pgfpathlineto{\pgfqpoint{2.788056in}{3.534820in}}%
\pgfpathlineto{\pgfqpoint{2.788056in}{3.406767in}}%
\pgfpathclose%
\pgfusepath{fill}%
\end{pgfscope}%
\begin{pgfscope}%
\pgfpathrectangle{\pgfqpoint{0.765000in}{0.660000in}}{\pgfqpoint{4.620000in}{4.620000in}}%
\pgfusepath{clip}%
\pgfsetbuttcap%
\pgfsetroundjoin%
\definecolor{currentfill}{rgb}{1.000000,0.894118,0.788235}%
\pgfsetfillcolor{currentfill}%
\pgfsetlinewidth{0.000000pt}%
\definecolor{currentstroke}{rgb}{1.000000,0.894118,0.788235}%
\pgfsetstrokecolor{currentstroke}%
\pgfsetdash{}{0pt}%
\pgfpathmoveto{\pgfqpoint{2.793000in}{3.409870in}}%
\pgfpathlineto{\pgfqpoint{2.682103in}{3.345844in}}%
\pgfpathlineto{\pgfqpoint{2.793000in}{3.281817in}}%
\pgfpathlineto{\pgfqpoint{2.903897in}{3.345844in}}%
\pgfpathlineto{\pgfqpoint{2.793000in}{3.409870in}}%
\pgfpathclose%
\pgfusepath{fill}%
\end{pgfscope}%
\begin{pgfscope}%
\pgfpathrectangle{\pgfqpoint{0.765000in}{0.660000in}}{\pgfqpoint{4.620000in}{4.620000in}}%
\pgfusepath{clip}%
\pgfsetbuttcap%
\pgfsetroundjoin%
\definecolor{currentfill}{rgb}{1.000000,0.894118,0.788235}%
\pgfsetfillcolor{currentfill}%
\pgfsetlinewidth{0.000000pt}%
\definecolor{currentstroke}{rgb}{1.000000,0.894118,0.788235}%
\pgfsetstrokecolor{currentstroke}%
\pgfsetdash{}{0pt}%
\pgfpathmoveto{\pgfqpoint{2.793000in}{3.409870in}}%
\pgfpathlineto{\pgfqpoint{2.682103in}{3.345844in}}%
\pgfpathlineto{\pgfqpoint{2.682103in}{3.473896in}}%
\pgfpathlineto{\pgfqpoint{2.793000in}{3.537922in}}%
\pgfpathlineto{\pgfqpoint{2.793000in}{3.409870in}}%
\pgfpathclose%
\pgfusepath{fill}%
\end{pgfscope}%
\begin{pgfscope}%
\pgfpathrectangle{\pgfqpoint{0.765000in}{0.660000in}}{\pgfqpoint{4.620000in}{4.620000in}}%
\pgfusepath{clip}%
\pgfsetbuttcap%
\pgfsetroundjoin%
\definecolor{currentfill}{rgb}{1.000000,0.894118,0.788235}%
\pgfsetfillcolor{currentfill}%
\pgfsetlinewidth{0.000000pt}%
\definecolor{currentstroke}{rgb}{1.000000,0.894118,0.788235}%
\pgfsetstrokecolor{currentstroke}%
\pgfsetdash{}{0pt}%
\pgfpathmoveto{\pgfqpoint{2.793000in}{3.409870in}}%
\pgfpathlineto{\pgfqpoint{2.903897in}{3.345844in}}%
\pgfpathlineto{\pgfqpoint{2.903897in}{3.473896in}}%
\pgfpathlineto{\pgfqpoint{2.793000in}{3.537922in}}%
\pgfpathlineto{\pgfqpoint{2.793000in}{3.409870in}}%
\pgfpathclose%
\pgfusepath{fill}%
\end{pgfscope}%
\begin{pgfscope}%
\pgfpathrectangle{\pgfqpoint{0.765000in}{0.660000in}}{\pgfqpoint{4.620000in}{4.620000in}}%
\pgfusepath{clip}%
\pgfsetbuttcap%
\pgfsetroundjoin%
\definecolor{currentfill}{rgb}{1.000000,0.894118,0.788235}%
\pgfsetfillcolor{currentfill}%
\pgfsetlinewidth{0.000000pt}%
\definecolor{currentstroke}{rgb}{1.000000,0.894118,0.788235}%
\pgfsetstrokecolor{currentstroke}%
\pgfsetdash{}{0pt}%
\pgfpathmoveto{\pgfqpoint{2.788056in}{3.534820in}}%
\pgfpathlineto{\pgfqpoint{2.677159in}{3.470793in}}%
\pgfpathlineto{\pgfqpoint{2.788056in}{3.406767in}}%
\pgfpathlineto{\pgfqpoint{2.898952in}{3.470793in}}%
\pgfpathlineto{\pgfqpoint{2.788056in}{3.534820in}}%
\pgfpathclose%
\pgfusepath{fill}%
\end{pgfscope}%
\begin{pgfscope}%
\pgfpathrectangle{\pgfqpoint{0.765000in}{0.660000in}}{\pgfqpoint{4.620000in}{4.620000in}}%
\pgfusepath{clip}%
\pgfsetbuttcap%
\pgfsetroundjoin%
\definecolor{currentfill}{rgb}{1.000000,0.894118,0.788235}%
\pgfsetfillcolor{currentfill}%
\pgfsetlinewidth{0.000000pt}%
\definecolor{currentstroke}{rgb}{1.000000,0.894118,0.788235}%
\pgfsetstrokecolor{currentstroke}%
\pgfsetdash{}{0pt}%
\pgfpathmoveto{\pgfqpoint{2.788056in}{3.278715in}}%
\pgfpathlineto{\pgfqpoint{2.898952in}{3.342741in}}%
\pgfpathlineto{\pgfqpoint{2.898952in}{3.470793in}}%
\pgfpathlineto{\pgfqpoint{2.788056in}{3.406767in}}%
\pgfpathlineto{\pgfqpoint{2.788056in}{3.278715in}}%
\pgfpathclose%
\pgfusepath{fill}%
\end{pgfscope}%
\begin{pgfscope}%
\pgfpathrectangle{\pgfqpoint{0.765000in}{0.660000in}}{\pgfqpoint{4.620000in}{4.620000in}}%
\pgfusepath{clip}%
\pgfsetbuttcap%
\pgfsetroundjoin%
\definecolor{currentfill}{rgb}{1.000000,0.894118,0.788235}%
\pgfsetfillcolor{currentfill}%
\pgfsetlinewidth{0.000000pt}%
\definecolor{currentstroke}{rgb}{1.000000,0.894118,0.788235}%
\pgfsetstrokecolor{currentstroke}%
\pgfsetdash{}{0pt}%
\pgfpathmoveto{\pgfqpoint{2.677159in}{3.342741in}}%
\pgfpathlineto{\pgfqpoint{2.788056in}{3.278715in}}%
\pgfpathlineto{\pgfqpoint{2.788056in}{3.406767in}}%
\pgfpathlineto{\pgfqpoint{2.677159in}{3.470793in}}%
\pgfpathlineto{\pgfqpoint{2.677159in}{3.342741in}}%
\pgfpathclose%
\pgfusepath{fill}%
\end{pgfscope}%
\begin{pgfscope}%
\pgfpathrectangle{\pgfqpoint{0.765000in}{0.660000in}}{\pgfqpoint{4.620000in}{4.620000in}}%
\pgfusepath{clip}%
\pgfsetbuttcap%
\pgfsetroundjoin%
\definecolor{currentfill}{rgb}{1.000000,0.894118,0.788235}%
\pgfsetfillcolor{currentfill}%
\pgfsetlinewidth{0.000000pt}%
\definecolor{currentstroke}{rgb}{1.000000,0.894118,0.788235}%
\pgfsetstrokecolor{currentstroke}%
\pgfsetdash{}{0pt}%
\pgfpathmoveto{\pgfqpoint{2.793000in}{3.537922in}}%
\pgfpathlineto{\pgfqpoint{2.682103in}{3.473896in}}%
\pgfpathlineto{\pgfqpoint{2.793000in}{3.409870in}}%
\pgfpathlineto{\pgfqpoint{2.903897in}{3.473896in}}%
\pgfpathlineto{\pgfqpoint{2.793000in}{3.537922in}}%
\pgfpathclose%
\pgfusepath{fill}%
\end{pgfscope}%
\begin{pgfscope}%
\pgfpathrectangle{\pgfqpoint{0.765000in}{0.660000in}}{\pgfqpoint{4.620000in}{4.620000in}}%
\pgfusepath{clip}%
\pgfsetbuttcap%
\pgfsetroundjoin%
\definecolor{currentfill}{rgb}{1.000000,0.894118,0.788235}%
\pgfsetfillcolor{currentfill}%
\pgfsetlinewidth{0.000000pt}%
\definecolor{currentstroke}{rgb}{1.000000,0.894118,0.788235}%
\pgfsetstrokecolor{currentstroke}%
\pgfsetdash{}{0pt}%
\pgfpathmoveto{\pgfqpoint{2.793000in}{3.281817in}}%
\pgfpathlineto{\pgfqpoint{2.903897in}{3.345844in}}%
\pgfpathlineto{\pgfqpoint{2.903897in}{3.473896in}}%
\pgfpathlineto{\pgfqpoint{2.793000in}{3.409870in}}%
\pgfpathlineto{\pgfqpoint{2.793000in}{3.281817in}}%
\pgfpathclose%
\pgfusepath{fill}%
\end{pgfscope}%
\begin{pgfscope}%
\pgfpathrectangle{\pgfqpoint{0.765000in}{0.660000in}}{\pgfqpoint{4.620000in}{4.620000in}}%
\pgfusepath{clip}%
\pgfsetbuttcap%
\pgfsetroundjoin%
\definecolor{currentfill}{rgb}{1.000000,0.894118,0.788235}%
\pgfsetfillcolor{currentfill}%
\pgfsetlinewidth{0.000000pt}%
\definecolor{currentstroke}{rgb}{1.000000,0.894118,0.788235}%
\pgfsetstrokecolor{currentstroke}%
\pgfsetdash{}{0pt}%
\pgfpathmoveto{\pgfqpoint{2.682103in}{3.345844in}}%
\pgfpathlineto{\pgfqpoint{2.793000in}{3.281817in}}%
\pgfpathlineto{\pgfqpoint{2.793000in}{3.409870in}}%
\pgfpathlineto{\pgfqpoint{2.682103in}{3.473896in}}%
\pgfpathlineto{\pgfqpoint{2.682103in}{3.345844in}}%
\pgfpathclose%
\pgfusepath{fill}%
\end{pgfscope}%
\begin{pgfscope}%
\pgfpathrectangle{\pgfqpoint{0.765000in}{0.660000in}}{\pgfqpoint{4.620000in}{4.620000in}}%
\pgfusepath{clip}%
\pgfsetbuttcap%
\pgfsetroundjoin%
\definecolor{currentfill}{rgb}{1.000000,0.894118,0.788235}%
\pgfsetfillcolor{currentfill}%
\pgfsetlinewidth{0.000000pt}%
\definecolor{currentstroke}{rgb}{1.000000,0.894118,0.788235}%
\pgfsetstrokecolor{currentstroke}%
\pgfsetdash{}{0pt}%
\pgfpathmoveto{\pgfqpoint{2.788056in}{3.406767in}}%
\pgfpathlineto{\pgfqpoint{2.677159in}{3.342741in}}%
\pgfpathlineto{\pgfqpoint{2.682103in}{3.345844in}}%
\pgfpathlineto{\pgfqpoint{2.793000in}{3.409870in}}%
\pgfpathlineto{\pgfqpoint{2.788056in}{3.406767in}}%
\pgfpathclose%
\pgfusepath{fill}%
\end{pgfscope}%
\begin{pgfscope}%
\pgfpathrectangle{\pgfqpoint{0.765000in}{0.660000in}}{\pgfqpoint{4.620000in}{4.620000in}}%
\pgfusepath{clip}%
\pgfsetbuttcap%
\pgfsetroundjoin%
\definecolor{currentfill}{rgb}{1.000000,0.894118,0.788235}%
\pgfsetfillcolor{currentfill}%
\pgfsetlinewidth{0.000000pt}%
\definecolor{currentstroke}{rgb}{1.000000,0.894118,0.788235}%
\pgfsetstrokecolor{currentstroke}%
\pgfsetdash{}{0pt}%
\pgfpathmoveto{\pgfqpoint{2.898952in}{3.342741in}}%
\pgfpathlineto{\pgfqpoint{2.788056in}{3.406767in}}%
\pgfpathlineto{\pgfqpoint{2.793000in}{3.409870in}}%
\pgfpathlineto{\pgfqpoint{2.903897in}{3.345844in}}%
\pgfpathlineto{\pgfqpoint{2.898952in}{3.342741in}}%
\pgfpathclose%
\pgfusepath{fill}%
\end{pgfscope}%
\begin{pgfscope}%
\pgfpathrectangle{\pgfqpoint{0.765000in}{0.660000in}}{\pgfqpoint{4.620000in}{4.620000in}}%
\pgfusepath{clip}%
\pgfsetbuttcap%
\pgfsetroundjoin%
\definecolor{currentfill}{rgb}{1.000000,0.894118,0.788235}%
\pgfsetfillcolor{currentfill}%
\pgfsetlinewidth{0.000000pt}%
\definecolor{currentstroke}{rgb}{1.000000,0.894118,0.788235}%
\pgfsetstrokecolor{currentstroke}%
\pgfsetdash{}{0pt}%
\pgfpathmoveto{\pgfqpoint{2.788056in}{3.406767in}}%
\pgfpathlineto{\pgfqpoint{2.788056in}{3.534820in}}%
\pgfpathlineto{\pgfqpoint{2.793000in}{3.537922in}}%
\pgfpathlineto{\pgfqpoint{2.903897in}{3.345844in}}%
\pgfpathlineto{\pgfqpoint{2.788056in}{3.406767in}}%
\pgfpathclose%
\pgfusepath{fill}%
\end{pgfscope}%
\begin{pgfscope}%
\pgfpathrectangle{\pgfqpoint{0.765000in}{0.660000in}}{\pgfqpoint{4.620000in}{4.620000in}}%
\pgfusepath{clip}%
\pgfsetbuttcap%
\pgfsetroundjoin%
\definecolor{currentfill}{rgb}{1.000000,0.894118,0.788235}%
\pgfsetfillcolor{currentfill}%
\pgfsetlinewidth{0.000000pt}%
\definecolor{currentstroke}{rgb}{1.000000,0.894118,0.788235}%
\pgfsetstrokecolor{currentstroke}%
\pgfsetdash{}{0pt}%
\pgfpathmoveto{\pgfqpoint{2.898952in}{3.342741in}}%
\pgfpathlineto{\pgfqpoint{2.898952in}{3.470793in}}%
\pgfpathlineto{\pgfqpoint{2.903897in}{3.473896in}}%
\pgfpathlineto{\pgfqpoint{2.793000in}{3.409870in}}%
\pgfpathlineto{\pgfqpoint{2.898952in}{3.342741in}}%
\pgfpathclose%
\pgfusepath{fill}%
\end{pgfscope}%
\begin{pgfscope}%
\pgfpathrectangle{\pgfqpoint{0.765000in}{0.660000in}}{\pgfqpoint{4.620000in}{4.620000in}}%
\pgfusepath{clip}%
\pgfsetbuttcap%
\pgfsetroundjoin%
\definecolor{currentfill}{rgb}{1.000000,0.894118,0.788235}%
\pgfsetfillcolor{currentfill}%
\pgfsetlinewidth{0.000000pt}%
\definecolor{currentstroke}{rgb}{1.000000,0.894118,0.788235}%
\pgfsetstrokecolor{currentstroke}%
\pgfsetdash{}{0pt}%
\pgfpathmoveto{\pgfqpoint{2.677159in}{3.342741in}}%
\pgfpathlineto{\pgfqpoint{2.788056in}{3.278715in}}%
\pgfpathlineto{\pgfqpoint{2.793000in}{3.281817in}}%
\pgfpathlineto{\pgfqpoint{2.682103in}{3.345844in}}%
\pgfpathlineto{\pgfqpoint{2.677159in}{3.342741in}}%
\pgfpathclose%
\pgfusepath{fill}%
\end{pgfscope}%
\begin{pgfscope}%
\pgfpathrectangle{\pgfqpoint{0.765000in}{0.660000in}}{\pgfqpoint{4.620000in}{4.620000in}}%
\pgfusepath{clip}%
\pgfsetbuttcap%
\pgfsetroundjoin%
\definecolor{currentfill}{rgb}{1.000000,0.894118,0.788235}%
\pgfsetfillcolor{currentfill}%
\pgfsetlinewidth{0.000000pt}%
\definecolor{currentstroke}{rgb}{1.000000,0.894118,0.788235}%
\pgfsetstrokecolor{currentstroke}%
\pgfsetdash{}{0pt}%
\pgfpathmoveto{\pgfqpoint{2.788056in}{3.278715in}}%
\pgfpathlineto{\pgfqpoint{2.898952in}{3.342741in}}%
\pgfpathlineto{\pgfqpoint{2.903897in}{3.345844in}}%
\pgfpathlineto{\pgfqpoint{2.793000in}{3.281817in}}%
\pgfpathlineto{\pgfqpoint{2.788056in}{3.278715in}}%
\pgfpathclose%
\pgfusepath{fill}%
\end{pgfscope}%
\begin{pgfscope}%
\pgfpathrectangle{\pgfqpoint{0.765000in}{0.660000in}}{\pgfqpoint{4.620000in}{4.620000in}}%
\pgfusepath{clip}%
\pgfsetbuttcap%
\pgfsetroundjoin%
\definecolor{currentfill}{rgb}{1.000000,0.894118,0.788235}%
\pgfsetfillcolor{currentfill}%
\pgfsetlinewidth{0.000000pt}%
\definecolor{currentstroke}{rgb}{1.000000,0.894118,0.788235}%
\pgfsetstrokecolor{currentstroke}%
\pgfsetdash{}{0pt}%
\pgfpathmoveto{\pgfqpoint{2.788056in}{3.534820in}}%
\pgfpathlineto{\pgfqpoint{2.677159in}{3.470793in}}%
\pgfpathlineto{\pgfqpoint{2.682103in}{3.473896in}}%
\pgfpathlineto{\pgfqpoint{2.793000in}{3.537922in}}%
\pgfpathlineto{\pgfqpoint{2.788056in}{3.534820in}}%
\pgfpathclose%
\pgfusepath{fill}%
\end{pgfscope}%
\begin{pgfscope}%
\pgfpathrectangle{\pgfqpoint{0.765000in}{0.660000in}}{\pgfqpoint{4.620000in}{4.620000in}}%
\pgfusepath{clip}%
\pgfsetbuttcap%
\pgfsetroundjoin%
\definecolor{currentfill}{rgb}{1.000000,0.894118,0.788235}%
\pgfsetfillcolor{currentfill}%
\pgfsetlinewidth{0.000000pt}%
\definecolor{currentstroke}{rgb}{1.000000,0.894118,0.788235}%
\pgfsetstrokecolor{currentstroke}%
\pgfsetdash{}{0pt}%
\pgfpathmoveto{\pgfqpoint{2.898952in}{3.470793in}}%
\pgfpathlineto{\pgfqpoint{2.788056in}{3.534820in}}%
\pgfpathlineto{\pgfqpoint{2.793000in}{3.537922in}}%
\pgfpathlineto{\pgfqpoint{2.903897in}{3.473896in}}%
\pgfpathlineto{\pgfqpoint{2.898952in}{3.470793in}}%
\pgfpathclose%
\pgfusepath{fill}%
\end{pgfscope}%
\begin{pgfscope}%
\pgfpathrectangle{\pgfqpoint{0.765000in}{0.660000in}}{\pgfqpoint{4.620000in}{4.620000in}}%
\pgfusepath{clip}%
\pgfsetbuttcap%
\pgfsetroundjoin%
\definecolor{currentfill}{rgb}{1.000000,0.894118,0.788235}%
\pgfsetfillcolor{currentfill}%
\pgfsetlinewidth{0.000000pt}%
\definecolor{currentstroke}{rgb}{1.000000,0.894118,0.788235}%
\pgfsetstrokecolor{currentstroke}%
\pgfsetdash{}{0pt}%
\pgfpathmoveto{\pgfqpoint{2.677159in}{3.342741in}}%
\pgfpathlineto{\pgfqpoint{2.677159in}{3.470793in}}%
\pgfpathlineto{\pgfqpoint{2.682103in}{3.473896in}}%
\pgfpathlineto{\pgfqpoint{2.793000in}{3.281817in}}%
\pgfpathlineto{\pgfqpoint{2.677159in}{3.342741in}}%
\pgfpathclose%
\pgfusepath{fill}%
\end{pgfscope}%
\begin{pgfscope}%
\pgfpathrectangle{\pgfqpoint{0.765000in}{0.660000in}}{\pgfqpoint{4.620000in}{4.620000in}}%
\pgfusepath{clip}%
\pgfsetbuttcap%
\pgfsetroundjoin%
\definecolor{currentfill}{rgb}{1.000000,0.894118,0.788235}%
\pgfsetfillcolor{currentfill}%
\pgfsetlinewidth{0.000000pt}%
\definecolor{currentstroke}{rgb}{1.000000,0.894118,0.788235}%
\pgfsetstrokecolor{currentstroke}%
\pgfsetdash{}{0pt}%
\pgfpathmoveto{\pgfqpoint{2.788056in}{3.278715in}}%
\pgfpathlineto{\pgfqpoint{2.788056in}{3.406767in}}%
\pgfpathlineto{\pgfqpoint{2.793000in}{3.409870in}}%
\pgfpathlineto{\pgfqpoint{2.682103in}{3.345844in}}%
\pgfpathlineto{\pgfqpoint{2.788056in}{3.278715in}}%
\pgfpathclose%
\pgfusepath{fill}%
\end{pgfscope}%
\begin{pgfscope}%
\pgfpathrectangle{\pgfqpoint{0.765000in}{0.660000in}}{\pgfqpoint{4.620000in}{4.620000in}}%
\pgfusepath{clip}%
\pgfsetbuttcap%
\pgfsetroundjoin%
\definecolor{currentfill}{rgb}{1.000000,0.894118,0.788235}%
\pgfsetfillcolor{currentfill}%
\pgfsetlinewidth{0.000000pt}%
\definecolor{currentstroke}{rgb}{1.000000,0.894118,0.788235}%
\pgfsetstrokecolor{currentstroke}%
\pgfsetdash{}{0pt}%
\pgfpathmoveto{\pgfqpoint{2.788056in}{3.406767in}}%
\pgfpathlineto{\pgfqpoint{2.898952in}{3.470793in}}%
\pgfpathlineto{\pgfqpoint{2.903897in}{3.473896in}}%
\pgfpathlineto{\pgfqpoint{2.793000in}{3.409870in}}%
\pgfpathlineto{\pgfqpoint{2.788056in}{3.406767in}}%
\pgfpathclose%
\pgfusepath{fill}%
\end{pgfscope}%
\begin{pgfscope}%
\pgfpathrectangle{\pgfqpoint{0.765000in}{0.660000in}}{\pgfqpoint{4.620000in}{4.620000in}}%
\pgfusepath{clip}%
\pgfsetbuttcap%
\pgfsetroundjoin%
\definecolor{currentfill}{rgb}{1.000000,0.894118,0.788235}%
\pgfsetfillcolor{currentfill}%
\pgfsetlinewidth{0.000000pt}%
\definecolor{currentstroke}{rgb}{1.000000,0.894118,0.788235}%
\pgfsetstrokecolor{currentstroke}%
\pgfsetdash{}{0pt}%
\pgfpathmoveto{\pgfqpoint{2.677159in}{3.470793in}}%
\pgfpathlineto{\pgfqpoint{2.788056in}{3.406767in}}%
\pgfpathlineto{\pgfqpoint{2.793000in}{3.409870in}}%
\pgfpathlineto{\pgfqpoint{2.682103in}{3.473896in}}%
\pgfpathlineto{\pgfqpoint{2.677159in}{3.470793in}}%
\pgfpathclose%
\pgfusepath{fill}%
\end{pgfscope}%
\begin{pgfscope}%
\pgfpathrectangle{\pgfqpoint{0.765000in}{0.660000in}}{\pgfqpoint{4.620000in}{4.620000in}}%
\pgfusepath{clip}%
\pgfsetbuttcap%
\pgfsetroundjoin%
\definecolor{currentfill}{rgb}{1.000000,0.894118,0.788235}%
\pgfsetfillcolor{currentfill}%
\pgfsetlinewidth{0.000000pt}%
\definecolor{currentstroke}{rgb}{1.000000,0.894118,0.788235}%
\pgfsetstrokecolor{currentstroke}%
\pgfsetdash{}{0pt}%
\pgfpathmoveto{\pgfqpoint{3.563846in}{2.808203in}}%
\pgfpathlineto{\pgfqpoint{3.452949in}{2.744176in}}%
\pgfpathlineto{\pgfqpoint{3.563846in}{2.680150in}}%
\pgfpathlineto{\pgfqpoint{3.674742in}{2.744176in}}%
\pgfpathlineto{\pgfqpoint{3.563846in}{2.808203in}}%
\pgfpathclose%
\pgfusepath{fill}%
\end{pgfscope}%
\begin{pgfscope}%
\pgfpathrectangle{\pgfqpoint{0.765000in}{0.660000in}}{\pgfqpoint{4.620000in}{4.620000in}}%
\pgfusepath{clip}%
\pgfsetbuttcap%
\pgfsetroundjoin%
\definecolor{currentfill}{rgb}{1.000000,0.894118,0.788235}%
\pgfsetfillcolor{currentfill}%
\pgfsetlinewidth{0.000000pt}%
\definecolor{currentstroke}{rgb}{1.000000,0.894118,0.788235}%
\pgfsetstrokecolor{currentstroke}%
\pgfsetdash{}{0pt}%
\pgfpathmoveto{\pgfqpoint{3.563846in}{2.808203in}}%
\pgfpathlineto{\pgfqpoint{3.452949in}{2.744176in}}%
\pgfpathlineto{\pgfqpoint{3.452949in}{2.872229in}}%
\pgfpathlineto{\pgfqpoint{3.563846in}{2.936255in}}%
\pgfpathlineto{\pgfqpoint{3.563846in}{2.808203in}}%
\pgfpathclose%
\pgfusepath{fill}%
\end{pgfscope}%
\begin{pgfscope}%
\pgfpathrectangle{\pgfqpoint{0.765000in}{0.660000in}}{\pgfqpoint{4.620000in}{4.620000in}}%
\pgfusepath{clip}%
\pgfsetbuttcap%
\pgfsetroundjoin%
\definecolor{currentfill}{rgb}{1.000000,0.894118,0.788235}%
\pgfsetfillcolor{currentfill}%
\pgfsetlinewidth{0.000000pt}%
\definecolor{currentstroke}{rgb}{1.000000,0.894118,0.788235}%
\pgfsetstrokecolor{currentstroke}%
\pgfsetdash{}{0pt}%
\pgfpathmoveto{\pgfqpoint{3.563846in}{2.808203in}}%
\pgfpathlineto{\pgfqpoint{3.674742in}{2.744176in}}%
\pgfpathlineto{\pgfqpoint{3.674742in}{2.872229in}}%
\pgfpathlineto{\pgfqpoint{3.563846in}{2.936255in}}%
\pgfpathlineto{\pgfqpoint{3.563846in}{2.808203in}}%
\pgfpathclose%
\pgfusepath{fill}%
\end{pgfscope}%
\begin{pgfscope}%
\pgfpathrectangle{\pgfqpoint{0.765000in}{0.660000in}}{\pgfqpoint{4.620000in}{4.620000in}}%
\pgfusepath{clip}%
\pgfsetbuttcap%
\pgfsetroundjoin%
\definecolor{currentfill}{rgb}{1.000000,0.894118,0.788235}%
\pgfsetfillcolor{currentfill}%
\pgfsetlinewidth{0.000000pt}%
\definecolor{currentstroke}{rgb}{1.000000,0.894118,0.788235}%
\pgfsetstrokecolor{currentstroke}%
\pgfsetdash{}{0pt}%
\pgfpathmoveto{\pgfqpoint{3.551407in}{2.776901in}}%
\pgfpathlineto{\pgfqpoint{3.440511in}{2.712874in}}%
\pgfpathlineto{\pgfqpoint{3.551407in}{2.648848in}}%
\pgfpathlineto{\pgfqpoint{3.662304in}{2.712874in}}%
\pgfpathlineto{\pgfqpoint{3.551407in}{2.776901in}}%
\pgfpathclose%
\pgfusepath{fill}%
\end{pgfscope}%
\begin{pgfscope}%
\pgfpathrectangle{\pgfqpoint{0.765000in}{0.660000in}}{\pgfqpoint{4.620000in}{4.620000in}}%
\pgfusepath{clip}%
\pgfsetbuttcap%
\pgfsetroundjoin%
\definecolor{currentfill}{rgb}{1.000000,0.894118,0.788235}%
\pgfsetfillcolor{currentfill}%
\pgfsetlinewidth{0.000000pt}%
\definecolor{currentstroke}{rgb}{1.000000,0.894118,0.788235}%
\pgfsetstrokecolor{currentstroke}%
\pgfsetdash{}{0pt}%
\pgfpathmoveto{\pgfqpoint{3.551407in}{2.776901in}}%
\pgfpathlineto{\pgfqpoint{3.440511in}{2.712874in}}%
\pgfpathlineto{\pgfqpoint{3.440511in}{2.840927in}}%
\pgfpathlineto{\pgfqpoint{3.551407in}{2.904953in}}%
\pgfpathlineto{\pgfqpoint{3.551407in}{2.776901in}}%
\pgfpathclose%
\pgfusepath{fill}%
\end{pgfscope}%
\begin{pgfscope}%
\pgfpathrectangle{\pgfqpoint{0.765000in}{0.660000in}}{\pgfqpoint{4.620000in}{4.620000in}}%
\pgfusepath{clip}%
\pgfsetbuttcap%
\pgfsetroundjoin%
\definecolor{currentfill}{rgb}{1.000000,0.894118,0.788235}%
\pgfsetfillcolor{currentfill}%
\pgfsetlinewidth{0.000000pt}%
\definecolor{currentstroke}{rgb}{1.000000,0.894118,0.788235}%
\pgfsetstrokecolor{currentstroke}%
\pgfsetdash{}{0pt}%
\pgfpathmoveto{\pgfqpoint{3.551407in}{2.776901in}}%
\pgfpathlineto{\pgfqpoint{3.662304in}{2.712874in}}%
\pgfpathlineto{\pgfqpoint{3.662304in}{2.840927in}}%
\pgfpathlineto{\pgfqpoint{3.551407in}{2.904953in}}%
\pgfpathlineto{\pgfqpoint{3.551407in}{2.776901in}}%
\pgfpathclose%
\pgfusepath{fill}%
\end{pgfscope}%
\begin{pgfscope}%
\pgfpathrectangle{\pgfqpoint{0.765000in}{0.660000in}}{\pgfqpoint{4.620000in}{4.620000in}}%
\pgfusepath{clip}%
\pgfsetbuttcap%
\pgfsetroundjoin%
\definecolor{currentfill}{rgb}{1.000000,0.894118,0.788235}%
\pgfsetfillcolor{currentfill}%
\pgfsetlinewidth{0.000000pt}%
\definecolor{currentstroke}{rgb}{1.000000,0.894118,0.788235}%
\pgfsetstrokecolor{currentstroke}%
\pgfsetdash{}{0pt}%
\pgfpathmoveto{\pgfqpoint{3.563846in}{2.936255in}}%
\pgfpathlineto{\pgfqpoint{3.452949in}{2.872229in}}%
\pgfpathlineto{\pgfqpoint{3.563846in}{2.808203in}}%
\pgfpathlineto{\pgfqpoint{3.674742in}{2.872229in}}%
\pgfpathlineto{\pgfqpoint{3.563846in}{2.936255in}}%
\pgfpathclose%
\pgfusepath{fill}%
\end{pgfscope}%
\begin{pgfscope}%
\pgfpathrectangle{\pgfqpoint{0.765000in}{0.660000in}}{\pgfqpoint{4.620000in}{4.620000in}}%
\pgfusepath{clip}%
\pgfsetbuttcap%
\pgfsetroundjoin%
\definecolor{currentfill}{rgb}{1.000000,0.894118,0.788235}%
\pgfsetfillcolor{currentfill}%
\pgfsetlinewidth{0.000000pt}%
\definecolor{currentstroke}{rgb}{1.000000,0.894118,0.788235}%
\pgfsetstrokecolor{currentstroke}%
\pgfsetdash{}{0pt}%
\pgfpathmoveto{\pgfqpoint{3.563846in}{2.680150in}}%
\pgfpathlineto{\pgfqpoint{3.674742in}{2.744176in}}%
\pgfpathlineto{\pgfqpoint{3.674742in}{2.872229in}}%
\pgfpathlineto{\pgfqpoint{3.563846in}{2.808203in}}%
\pgfpathlineto{\pgfqpoint{3.563846in}{2.680150in}}%
\pgfpathclose%
\pgfusepath{fill}%
\end{pgfscope}%
\begin{pgfscope}%
\pgfpathrectangle{\pgfqpoint{0.765000in}{0.660000in}}{\pgfqpoint{4.620000in}{4.620000in}}%
\pgfusepath{clip}%
\pgfsetbuttcap%
\pgfsetroundjoin%
\definecolor{currentfill}{rgb}{1.000000,0.894118,0.788235}%
\pgfsetfillcolor{currentfill}%
\pgfsetlinewidth{0.000000pt}%
\definecolor{currentstroke}{rgb}{1.000000,0.894118,0.788235}%
\pgfsetstrokecolor{currentstroke}%
\pgfsetdash{}{0pt}%
\pgfpathmoveto{\pgfqpoint{3.452949in}{2.744176in}}%
\pgfpathlineto{\pgfqpoint{3.563846in}{2.680150in}}%
\pgfpathlineto{\pgfqpoint{3.563846in}{2.808203in}}%
\pgfpathlineto{\pgfqpoint{3.452949in}{2.872229in}}%
\pgfpathlineto{\pgfqpoint{3.452949in}{2.744176in}}%
\pgfpathclose%
\pgfusepath{fill}%
\end{pgfscope}%
\begin{pgfscope}%
\pgfpathrectangle{\pgfqpoint{0.765000in}{0.660000in}}{\pgfqpoint{4.620000in}{4.620000in}}%
\pgfusepath{clip}%
\pgfsetbuttcap%
\pgfsetroundjoin%
\definecolor{currentfill}{rgb}{1.000000,0.894118,0.788235}%
\pgfsetfillcolor{currentfill}%
\pgfsetlinewidth{0.000000pt}%
\definecolor{currentstroke}{rgb}{1.000000,0.894118,0.788235}%
\pgfsetstrokecolor{currentstroke}%
\pgfsetdash{}{0pt}%
\pgfpathmoveto{\pgfqpoint{3.551407in}{2.904953in}}%
\pgfpathlineto{\pgfqpoint{3.440511in}{2.840927in}}%
\pgfpathlineto{\pgfqpoint{3.551407in}{2.776901in}}%
\pgfpathlineto{\pgfqpoint{3.662304in}{2.840927in}}%
\pgfpathlineto{\pgfqpoint{3.551407in}{2.904953in}}%
\pgfpathclose%
\pgfusepath{fill}%
\end{pgfscope}%
\begin{pgfscope}%
\pgfpathrectangle{\pgfqpoint{0.765000in}{0.660000in}}{\pgfqpoint{4.620000in}{4.620000in}}%
\pgfusepath{clip}%
\pgfsetbuttcap%
\pgfsetroundjoin%
\definecolor{currentfill}{rgb}{1.000000,0.894118,0.788235}%
\pgfsetfillcolor{currentfill}%
\pgfsetlinewidth{0.000000pt}%
\definecolor{currentstroke}{rgb}{1.000000,0.894118,0.788235}%
\pgfsetstrokecolor{currentstroke}%
\pgfsetdash{}{0pt}%
\pgfpathmoveto{\pgfqpoint{3.551407in}{2.648848in}}%
\pgfpathlineto{\pgfqpoint{3.662304in}{2.712874in}}%
\pgfpathlineto{\pgfqpoint{3.662304in}{2.840927in}}%
\pgfpathlineto{\pgfqpoint{3.551407in}{2.776901in}}%
\pgfpathlineto{\pgfqpoint{3.551407in}{2.648848in}}%
\pgfpathclose%
\pgfusepath{fill}%
\end{pgfscope}%
\begin{pgfscope}%
\pgfpathrectangle{\pgfqpoint{0.765000in}{0.660000in}}{\pgfqpoint{4.620000in}{4.620000in}}%
\pgfusepath{clip}%
\pgfsetbuttcap%
\pgfsetroundjoin%
\definecolor{currentfill}{rgb}{1.000000,0.894118,0.788235}%
\pgfsetfillcolor{currentfill}%
\pgfsetlinewidth{0.000000pt}%
\definecolor{currentstroke}{rgb}{1.000000,0.894118,0.788235}%
\pgfsetstrokecolor{currentstroke}%
\pgfsetdash{}{0pt}%
\pgfpathmoveto{\pgfqpoint{3.440511in}{2.712874in}}%
\pgfpathlineto{\pgfqpoint{3.551407in}{2.648848in}}%
\pgfpathlineto{\pgfqpoint{3.551407in}{2.776901in}}%
\pgfpathlineto{\pgfqpoint{3.440511in}{2.840927in}}%
\pgfpathlineto{\pgfqpoint{3.440511in}{2.712874in}}%
\pgfpathclose%
\pgfusepath{fill}%
\end{pgfscope}%
\begin{pgfscope}%
\pgfpathrectangle{\pgfqpoint{0.765000in}{0.660000in}}{\pgfqpoint{4.620000in}{4.620000in}}%
\pgfusepath{clip}%
\pgfsetbuttcap%
\pgfsetroundjoin%
\definecolor{currentfill}{rgb}{1.000000,0.894118,0.788235}%
\pgfsetfillcolor{currentfill}%
\pgfsetlinewidth{0.000000pt}%
\definecolor{currentstroke}{rgb}{1.000000,0.894118,0.788235}%
\pgfsetstrokecolor{currentstroke}%
\pgfsetdash{}{0pt}%
\pgfpathmoveto{\pgfqpoint{3.563846in}{2.808203in}}%
\pgfpathlineto{\pgfqpoint{3.452949in}{2.744176in}}%
\pgfpathlineto{\pgfqpoint{3.440511in}{2.712874in}}%
\pgfpathlineto{\pgfqpoint{3.551407in}{2.776901in}}%
\pgfpathlineto{\pgfqpoint{3.563846in}{2.808203in}}%
\pgfpathclose%
\pgfusepath{fill}%
\end{pgfscope}%
\begin{pgfscope}%
\pgfpathrectangle{\pgfqpoint{0.765000in}{0.660000in}}{\pgfqpoint{4.620000in}{4.620000in}}%
\pgfusepath{clip}%
\pgfsetbuttcap%
\pgfsetroundjoin%
\definecolor{currentfill}{rgb}{1.000000,0.894118,0.788235}%
\pgfsetfillcolor{currentfill}%
\pgfsetlinewidth{0.000000pt}%
\definecolor{currentstroke}{rgb}{1.000000,0.894118,0.788235}%
\pgfsetstrokecolor{currentstroke}%
\pgfsetdash{}{0pt}%
\pgfpathmoveto{\pgfqpoint{3.674742in}{2.744176in}}%
\pgfpathlineto{\pgfqpoint{3.563846in}{2.808203in}}%
\pgfpathlineto{\pgfqpoint{3.551407in}{2.776901in}}%
\pgfpathlineto{\pgfqpoint{3.662304in}{2.712874in}}%
\pgfpathlineto{\pgfqpoint{3.674742in}{2.744176in}}%
\pgfpathclose%
\pgfusepath{fill}%
\end{pgfscope}%
\begin{pgfscope}%
\pgfpathrectangle{\pgfqpoint{0.765000in}{0.660000in}}{\pgfqpoint{4.620000in}{4.620000in}}%
\pgfusepath{clip}%
\pgfsetbuttcap%
\pgfsetroundjoin%
\definecolor{currentfill}{rgb}{1.000000,0.894118,0.788235}%
\pgfsetfillcolor{currentfill}%
\pgfsetlinewidth{0.000000pt}%
\definecolor{currentstroke}{rgb}{1.000000,0.894118,0.788235}%
\pgfsetstrokecolor{currentstroke}%
\pgfsetdash{}{0pt}%
\pgfpathmoveto{\pgfqpoint{3.563846in}{2.808203in}}%
\pgfpathlineto{\pgfqpoint{3.563846in}{2.936255in}}%
\pgfpathlineto{\pgfqpoint{3.551407in}{2.904953in}}%
\pgfpathlineto{\pgfqpoint{3.662304in}{2.712874in}}%
\pgfpathlineto{\pgfqpoint{3.563846in}{2.808203in}}%
\pgfpathclose%
\pgfusepath{fill}%
\end{pgfscope}%
\begin{pgfscope}%
\pgfpathrectangle{\pgfqpoint{0.765000in}{0.660000in}}{\pgfqpoint{4.620000in}{4.620000in}}%
\pgfusepath{clip}%
\pgfsetbuttcap%
\pgfsetroundjoin%
\definecolor{currentfill}{rgb}{1.000000,0.894118,0.788235}%
\pgfsetfillcolor{currentfill}%
\pgfsetlinewidth{0.000000pt}%
\definecolor{currentstroke}{rgb}{1.000000,0.894118,0.788235}%
\pgfsetstrokecolor{currentstroke}%
\pgfsetdash{}{0pt}%
\pgfpathmoveto{\pgfqpoint{3.674742in}{2.744176in}}%
\pgfpathlineto{\pgfqpoint{3.674742in}{2.872229in}}%
\pgfpathlineto{\pgfqpoint{3.662304in}{2.840927in}}%
\pgfpathlineto{\pgfqpoint{3.551407in}{2.776901in}}%
\pgfpathlineto{\pgfqpoint{3.674742in}{2.744176in}}%
\pgfpathclose%
\pgfusepath{fill}%
\end{pgfscope}%
\begin{pgfscope}%
\pgfpathrectangle{\pgfqpoint{0.765000in}{0.660000in}}{\pgfqpoint{4.620000in}{4.620000in}}%
\pgfusepath{clip}%
\pgfsetbuttcap%
\pgfsetroundjoin%
\definecolor{currentfill}{rgb}{1.000000,0.894118,0.788235}%
\pgfsetfillcolor{currentfill}%
\pgfsetlinewidth{0.000000pt}%
\definecolor{currentstroke}{rgb}{1.000000,0.894118,0.788235}%
\pgfsetstrokecolor{currentstroke}%
\pgfsetdash{}{0pt}%
\pgfpathmoveto{\pgfqpoint{3.452949in}{2.744176in}}%
\pgfpathlineto{\pgfqpoint{3.563846in}{2.680150in}}%
\pgfpathlineto{\pgfqpoint{3.551407in}{2.648848in}}%
\pgfpathlineto{\pgfqpoint{3.440511in}{2.712874in}}%
\pgfpathlineto{\pgfqpoint{3.452949in}{2.744176in}}%
\pgfpathclose%
\pgfusepath{fill}%
\end{pgfscope}%
\begin{pgfscope}%
\pgfpathrectangle{\pgfqpoint{0.765000in}{0.660000in}}{\pgfqpoint{4.620000in}{4.620000in}}%
\pgfusepath{clip}%
\pgfsetbuttcap%
\pgfsetroundjoin%
\definecolor{currentfill}{rgb}{1.000000,0.894118,0.788235}%
\pgfsetfillcolor{currentfill}%
\pgfsetlinewidth{0.000000pt}%
\definecolor{currentstroke}{rgb}{1.000000,0.894118,0.788235}%
\pgfsetstrokecolor{currentstroke}%
\pgfsetdash{}{0pt}%
\pgfpathmoveto{\pgfqpoint{3.563846in}{2.680150in}}%
\pgfpathlineto{\pgfqpoint{3.674742in}{2.744176in}}%
\pgfpathlineto{\pgfqpoint{3.662304in}{2.712874in}}%
\pgfpathlineto{\pgfqpoint{3.551407in}{2.648848in}}%
\pgfpathlineto{\pgfqpoint{3.563846in}{2.680150in}}%
\pgfpathclose%
\pgfusepath{fill}%
\end{pgfscope}%
\begin{pgfscope}%
\pgfpathrectangle{\pgfqpoint{0.765000in}{0.660000in}}{\pgfqpoint{4.620000in}{4.620000in}}%
\pgfusepath{clip}%
\pgfsetbuttcap%
\pgfsetroundjoin%
\definecolor{currentfill}{rgb}{1.000000,0.894118,0.788235}%
\pgfsetfillcolor{currentfill}%
\pgfsetlinewidth{0.000000pt}%
\definecolor{currentstroke}{rgb}{1.000000,0.894118,0.788235}%
\pgfsetstrokecolor{currentstroke}%
\pgfsetdash{}{0pt}%
\pgfpathmoveto{\pgfqpoint{3.563846in}{2.936255in}}%
\pgfpathlineto{\pgfqpoint{3.452949in}{2.872229in}}%
\pgfpathlineto{\pgfqpoint{3.440511in}{2.840927in}}%
\pgfpathlineto{\pgfqpoint{3.551407in}{2.904953in}}%
\pgfpathlineto{\pgfqpoint{3.563846in}{2.936255in}}%
\pgfpathclose%
\pgfusepath{fill}%
\end{pgfscope}%
\begin{pgfscope}%
\pgfpathrectangle{\pgfqpoint{0.765000in}{0.660000in}}{\pgfqpoint{4.620000in}{4.620000in}}%
\pgfusepath{clip}%
\pgfsetbuttcap%
\pgfsetroundjoin%
\definecolor{currentfill}{rgb}{1.000000,0.894118,0.788235}%
\pgfsetfillcolor{currentfill}%
\pgfsetlinewidth{0.000000pt}%
\definecolor{currentstroke}{rgb}{1.000000,0.894118,0.788235}%
\pgfsetstrokecolor{currentstroke}%
\pgfsetdash{}{0pt}%
\pgfpathmoveto{\pgfqpoint{3.674742in}{2.872229in}}%
\pgfpathlineto{\pgfqpoint{3.563846in}{2.936255in}}%
\pgfpathlineto{\pgfqpoint{3.551407in}{2.904953in}}%
\pgfpathlineto{\pgfqpoint{3.662304in}{2.840927in}}%
\pgfpathlineto{\pgfqpoint{3.674742in}{2.872229in}}%
\pgfpathclose%
\pgfusepath{fill}%
\end{pgfscope}%
\begin{pgfscope}%
\pgfpathrectangle{\pgfqpoint{0.765000in}{0.660000in}}{\pgfqpoint{4.620000in}{4.620000in}}%
\pgfusepath{clip}%
\pgfsetbuttcap%
\pgfsetroundjoin%
\definecolor{currentfill}{rgb}{1.000000,0.894118,0.788235}%
\pgfsetfillcolor{currentfill}%
\pgfsetlinewidth{0.000000pt}%
\definecolor{currentstroke}{rgb}{1.000000,0.894118,0.788235}%
\pgfsetstrokecolor{currentstroke}%
\pgfsetdash{}{0pt}%
\pgfpathmoveto{\pgfqpoint{3.452949in}{2.744176in}}%
\pgfpathlineto{\pgfqpoint{3.452949in}{2.872229in}}%
\pgfpathlineto{\pgfqpoint{3.440511in}{2.840927in}}%
\pgfpathlineto{\pgfqpoint{3.551407in}{2.648848in}}%
\pgfpathlineto{\pgfqpoint{3.452949in}{2.744176in}}%
\pgfpathclose%
\pgfusepath{fill}%
\end{pgfscope}%
\begin{pgfscope}%
\pgfpathrectangle{\pgfqpoint{0.765000in}{0.660000in}}{\pgfqpoint{4.620000in}{4.620000in}}%
\pgfusepath{clip}%
\pgfsetbuttcap%
\pgfsetroundjoin%
\definecolor{currentfill}{rgb}{1.000000,0.894118,0.788235}%
\pgfsetfillcolor{currentfill}%
\pgfsetlinewidth{0.000000pt}%
\definecolor{currentstroke}{rgb}{1.000000,0.894118,0.788235}%
\pgfsetstrokecolor{currentstroke}%
\pgfsetdash{}{0pt}%
\pgfpathmoveto{\pgfqpoint{3.563846in}{2.680150in}}%
\pgfpathlineto{\pgfqpoint{3.563846in}{2.808203in}}%
\pgfpathlineto{\pgfqpoint{3.551407in}{2.776901in}}%
\pgfpathlineto{\pgfqpoint{3.440511in}{2.712874in}}%
\pgfpathlineto{\pgfqpoint{3.563846in}{2.680150in}}%
\pgfpathclose%
\pgfusepath{fill}%
\end{pgfscope}%
\begin{pgfscope}%
\pgfpathrectangle{\pgfqpoint{0.765000in}{0.660000in}}{\pgfqpoint{4.620000in}{4.620000in}}%
\pgfusepath{clip}%
\pgfsetbuttcap%
\pgfsetroundjoin%
\definecolor{currentfill}{rgb}{1.000000,0.894118,0.788235}%
\pgfsetfillcolor{currentfill}%
\pgfsetlinewidth{0.000000pt}%
\definecolor{currentstroke}{rgb}{1.000000,0.894118,0.788235}%
\pgfsetstrokecolor{currentstroke}%
\pgfsetdash{}{0pt}%
\pgfpathmoveto{\pgfqpoint{3.563846in}{2.808203in}}%
\pgfpathlineto{\pgfqpoint{3.674742in}{2.872229in}}%
\pgfpathlineto{\pgfqpoint{3.662304in}{2.840927in}}%
\pgfpathlineto{\pgfqpoint{3.551407in}{2.776901in}}%
\pgfpathlineto{\pgfqpoint{3.563846in}{2.808203in}}%
\pgfpathclose%
\pgfusepath{fill}%
\end{pgfscope}%
\begin{pgfscope}%
\pgfpathrectangle{\pgfqpoint{0.765000in}{0.660000in}}{\pgfqpoint{4.620000in}{4.620000in}}%
\pgfusepath{clip}%
\pgfsetbuttcap%
\pgfsetroundjoin%
\definecolor{currentfill}{rgb}{1.000000,0.894118,0.788235}%
\pgfsetfillcolor{currentfill}%
\pgfsetlinewidth{0.000000pt}%
\definecolor{currentstroke}{rgb}{1.000000,0.894118,0.788235}%
\pgfsetstrokecolor{currentstroke}%
\pgfsetdash{}{0pt}%
\pgfpathmoveto{\pgfqpoint{3.452949in}{2.872229in}}%
\pgfpathlineto{\pgfqpoint{3.563846in}{2.808203in}}%
\pgfpathlineto{\pgfqpoint{3.551407in}{2.776901in}}%
\pgfpathlineto{\pgfqpoint{3.440511in}{2.840927in}}%
\pgfpathlineto{\pgfqpoint{3.452949in}{2.872229in}}%
\pgfpathclose%
\pgfusepath{fill}%
\end{pgfscope}%
\begin{pgfscope}%
\pgfpathrectangle{\pgfqpoint{0.765000in}{0.660000in}}{\pgfqpoint{4.620000in}{4.620000in}}%
\pgfusepath{clip}%
\pgfsetbuttcap%
\pgfsetroundjoin%
\definecolor{currentfill}{rgb}{1.000000,0.894118,0.788235}%
\pgfsetfillcolor{currentfill}%
\pgfsetlinewidth{0.000000pt}%
\definecolor{currentstroke}{rgb}{1.000000,0.894118,0.788235}%
\pgfsetstrokecolor{currentstroke}%
\pgfsetdash{}{0pt}%
\pgfpathmoveto{\pgfqpoint{3.563846in}{2.808203in}}%
\pgfpathlineto{\pgfqpoint{3.452949in}{2.744176in}}%
\pgfpathlineto{\pgfqpoint{3.563846in}{2.680150in}}%
\pgfpathlineto{\pgfqpoint{3.674742in}{2.744176in}}%
\pgfpathlineto{\pgfqpoint{3.563846in}{2.808203in}}%
\pgfpathclose%
\pgfusepath{fill}%
\end{pgfscope}%
\begin{pgfscope}%
\pgfpathrectangle{\pgfqpoint{0.765000in}{0.660000in}}{\pgfqpoint{4.620000in}{4.620000in}}%
\pgfusepath{clip}%
\pgfsetbuttcap%
\pgfsetroundjoin%
\definecolor{currentfill}{rgb}{1.000000,0.894118,0.788235}%
\pgfsetfillcolor{currentfill}%
\pgfsetlinewidth{0.000000pt}%
\definecolor{currentstroke}{rgb}{1.000000,0.894118,0.788235}%
\pgfsetstrokecolor{currentstroke}%
\pgfsetdash{}{0pt}%
\pgfpathmoveto{\pgfqpoint{3.563846in}{2.808203in}}%
\pgfpathlineto{\pgfqpoint{3.452949in}{2.744176in}}%
\pgfpathlineto{\pgfqpoint{3.452949in}{2.872229in}}%
\pgfpathlineto{\pgfqpoint{3.563846in}{2.936255in}}%
\pgfpathlineto{\pgfqpoint{3.563846in}{2.808203in}}%
\pgfpathclose%
\pgfusepath{fill}%
\end{pgfscope}%
\begin{pgfscope}%
\pgfpathrectangle{\pgfqpoint{0.765000in}{0.660000in}}{\pgfqpoint{4.620000in}{4.620000in}}%
\pgfusepath{clip}%
\pgfsetbuttcap%
\pgfsetroundjoin%
\definecolor{currentfill}{rgb}{1.000000,0.894118,0.788235}%
\pgfsetfillcolor{currentfill}%
\pgfsetlinewidth{0.000000pt}%
\definecolor{currentstroke}{rgb}{1.000000,0.894118,0.788235}%
\pgfsetstrokecolor{currentstroke}%
\pgfsetdash{}{0pt}%
\pgfpathmoveto{\pgfqpoint{3.563846in}{2.808203in}}%
\pgfpathlineto{\pgfqpoint{3.674742in}{2.744176in}}%
\pgfpathlineto{\pgfqpoint{3.674742in}{2.872229in}}%
\pgfpathlineto{\pgfqpoint{3.563846in}{2.936255in}}%
\pgfpathlineto{\pgfqpoint{3.563846in}{2.808203in}}%
\pgfpathclose%
\pgfusepath{fill}%
\end{pgfscope}%
\begin{pgfscope}%
\pgfpathrectangle{\pgfqpoint{0.765000in}{0.660000in}}{\pgfqpoint{4.620000in}{4.620000in}}%
\pgfusepath{clip}%
\pgfsetbuttcap%
\pgfsetroundjoin%
\definecolor{currentfill}{rgb}{1.000000,0.894118,0.788235}%
\pgfsetfillcolor{currentfill}%
\pgfsetlinewidth{0.000000pt}%
\definecolor{currentstroke}{rgb}{1.000000,0.894118,0.788235}%
\pgfsetstrokecolor{currentstroke}%
\pgfsetdash{}{0pt}%
\pgfpathmoveto{\pgfqpoint{3.572472in}{2.854798in}}%
\pgfpathlineto{\pgfqpoint{3.461575in}{2.790772in}}%
\pgfpathlineto{\pgfqpoint{3.572472in}{2.726746in}}%
\pgfpathlineto{\pgfqpoint{3.683368in}{2.790772in}}%
\pgfpathlineto{\pgfqpoint{3.572472in}{2.854798in}}%
\pgfpathclose%
\pgfusepath{fill}%
\end{pgfscope}%
\begin{pgfscope}%
\pgfpathrectangle{\pgfqpoint{0.765000in}{0.660000in}}{\pgfqpoint{4.620000in}{4.620000in}}%
\pgfusepath{clip}%
\pgfsetbuttcap%
\pgfsetroundjoin%
\definecolor{currentfill}{rgb}{1.000000,0.894118,0.788235}%
\pgfsetfillcolor{currentfill}%
\pgfsetlinewidth{0.000000pt}%
\definecolor{currentstroke}{rgb}{1.000000,0.894118,0.788235}%
\pgfsetstrokecolor{currentstroke}%
\pgfsetdash{}{0pt}%
\pgfpathmoveto{\pgfqpoint{3.572472in}{2.854798in}}%
\pgfpathlineto{\pgfqpoint{3.461575in}{2.790772in}}%
\pgfpathlineto{\pgfqpoint{3.461575in}{2.918824in}}%
\pgfpathlineto{\pgfqpoint{3.572472in}{2.982851in}}%
\pgfpathlineto{\pgfqpoint{3.572472in}{2.854798in}}%
\pgfpathclose%
\pgfusepath{fill}%
\end{pgfscope}%
\begin{pgfscope}%
\pgfpathrectangle{\pgfqpoint{0.765000in}{0.660000in}}{\pgfqpoint{4.620000in}{4.620000in}}%
\pgfusepath{clip}%
\pgfsetbuttcap%
\pgfsetroundjoin%
\definecolor{currentfill}{rgb}{1.000000,0.894118,0.788235}%
\pgfsetfillcolor{currentfill}%
\pgfsetlinewidth{0.000000pt}%
\definecolor{currentstroke}{rgb}{1.000000,0.894118,0.788235}%
\pgfsetstrokecolor{currentstroke}%
\pgfsetdash{}{0pt}%
\pgfpathmoveto{\pgfqpoint{3.572472in}{2.854798in}}%
\pgfpathlineto{\pgfqpoint{3.683368in}{2.790772in}}%
\pgfpathlineto{\pgfqpoint{3.683368in}{2.918824in}}%
\pgfpathlineto{\pgfqpoint{3.572472in}{2.982851in}}%
\pgfpathlineto{\pgfqpoint{3.572472in}{2.854798in}}%
\pgfpathclose%
\pgfusepath{fill}%
\end{pgfscope}%
\begin{pgfscope}%
\pgfpathrectangle{\pgfqpoint{0.765000in}{0.660000in}}{\pgfqpoint{4.620000in}{4.620000in}}%
\pgfusepath{clip}%
\pgfsetbuttcap%
\pgfsetroundjoin%
\definecolor{currentfill}{rgb}{1.000000,0.894118,0.788235}%
\pgfsetfillcolor{currentfill}%
\pgfsetlinewidth{0.000000pt}%
\definecolor{currentstroke}{rgb}{1.000000,0.894118,0.788235}%
\pgfsetstrokecolor{currentstroke}%
\pgfsetdash{}{0pt}%
\pgfpathmoveto{\pgfqpoint{3.563846in}{2.936255in}}%
\pgfpathlineto{\pgfqpoint{3.452949in}{2.872229in}}%
\pgfpathlineto{\pgfqpoint{3.563846in}{2.808203in}}%
\pgfpathlineto{\pgfqpoint{3.674742in}{2.872229in}}%
\pgfpathlineto{\pgfqpoint{3.563846in}{2.936255in}}%
\pgfpathclose%
\pgfusepath{fill}%
\end{pgfscope}%
\begin{pgfscope}%
\pgfpathrectangle{\pgfqpoint{0.765000in}{0.660000in}}{\pgfqpoint{4.620000in}{4.620000in}}%
\pgfusepath{clip}%
\pgfsetbuttcap%
\pgfsetroundjoin%
\definecolor{currentfill}{rgb}{1.000000,0.894118,0.788235}%
\pgfsetfillcolor{currentfill}%
\pgfsetlinewidth{0.000000pt}%
\definecolor{currentstroke}{rgb}{1.000000,0.894118,0.788235}%
\pgfsetstrokecolor{currentstroke}%
\pgfsetdash{}{0pt}%
\pgfpathmoveto{\pgfqpoint{3.563846in}{2.680150in}}%
\pgfpathlineto{\pgfqpoint{3.674742in}{2.744176in}}%
\pgfpathlineto{\pgfqpoint{3.674742in}{2.872229in}}%
\pgfpathlineto{\pgfqpoint{3.563846in}{2.808203in}}%
\pgfpathlineto{\pgfqpoint{3.563846in}{2.680150in}}%
\pgfpathclose%
\pgfusepath{fill}%
\end{pgfscope}%
\begin{pgfscope}%
\pgfpathrectangle{\pgfqpoint{0.765000in}{0.660000in}}{\pgfqpoint{4.620000in}{4.620000in}}%
\pgfusepath{clip}%
\pgfsetbuttcap%
\pgfsetroundjoin%
\definecolor{currentfill}{rgb}{1.000000,0.894118,0.788235}%
\pgfsetfillcolor{currentfill}%
\pgfsetlinewidth{0.000000pt}%
\definecolor{currentstroke}{rgb}{1.000000,0.894118,0.788235}%
\pgfsetstrokecolor{currentstroke}%
\pgfsetdash{}{0pt}%
\pgfpathmoveto{\pgfqpoint{3.452949in}{2.744176in}}%
\pgfpathlineto{\pgfqpoint{3.563846in}{2.680150in}}%
\pgfpathlineto{\pgfqpoint{3.563846in}{2.808203in}}%
\pgfpathlineto{\pgfqpoint{3.452949in}{2.872229in}}%
\pgfpathlineto{\pgfqpoint{3.452949in}{2.744176in}}%
\pgfpathclose%
\pgfusepath{fill}%
\end{pgfscope}%
\begin{pgfscope}%
\pgfpathrectangle{\pgfqpoint{0.765000in}{0.660000in}}{\pgfqpoint{4.620000in}{4.620000in}}%
\pgfusepath{clip}%
\pgfsetbuttcap%
\pgfsetroundjoin%
\definecolor{currentfill}{rgb}{1.000000,0.894118,0.788235}%
\pgfsetfillcolor{currentfill}%
\pgfsetlinewidth{0.000000pt}%
\definecolor{currentstroke}{rgb}{1.000000,0.894118,0.788235}%
\pgfsetstrokecolor{currentstroke}%
\pgfsetdash{}{0pt}%
\pgfpathmoveto{\pgfqpoint{3.572472in}{2.982851in}}%
\pgfpathlineto{\pgfqpoint{3.461575in}{2.918824in}}%
\pgfpathlineto{\pgfqpoint{3.572472in}{2.854798in}}%
\pgfpathlineto{\pgfqpoint{3.683368in}{2.918824in}}%
\pgfpathlineto{\pgfqpoint{3.572472in}{2.982851in}}%
\pgfpathclose%
\pgfusepath{fill}%
\end{pgfscope}%
\begin{pgfscope}%
\pgfpathrectangle{\pgfqpoint{0.765000in}{0.660000in}}{\pgfqpoint{4.620000in}{4.620000in}}%
\pgfusepath{clip}%
\pgfsetbuttcap%
\pgfsetroundjoin%
\definecolor{currentfill}{rgb}{1.000000,0.894118,0.788235}%
\pgfsetfillcolor{currentfill}%
\pgfsetlinewidth{0.000000pt}%
\definecolor{currentstroke}{rgb}{1.000000,0.894118,0.788235}%
\pgfsetstrokecolor{currentstroke}%
\pgfsetdash{}{0pt}%
\pgfpathmoveto{\pgfqpoint{3.572472in}{2.726746in}}%
\pgfpathlineto{\pgfqpoint{3.683368in}{2.790772in}}%
\pgfpathlineto{\pgfqpoint{3.683368in}{2.918824in}}%
\pgfpathlineto{\pgfqpoint{3.572472in}{2.854798in}}%
\pgfpathlineto{\pgfqpoint{3.572472in}{2.726746in}}%
\pgfpathclose%
\pgfusepath{fill}%
\end{pgfscope}%
\begin{pgfscope}%
\pgfpathrectangle{\pgfqpoint{0.765000in}{0.660000in}}{\pgfqpoint{4.620000in}{4.620000in}}%
\pgfusepath{clip}%
\pgfsetbuttcap%
\pgfsetroundjoin%
\definecolor{currentfill}{rgb}{1.000000,0.894118,0.788235}%
\pgfsetfillcolor{currentfill}%
\pgfsetlinewidth{0.000000pt}%
\definecolor{currentstroke}{rgb}{1.000000,0.894118,0.788235}%
\pgfsetstrokecolor{currentstroke}%
\pgfsetdash{}{0pt}%
\pgfpathmoveto{\pgfqpoint{3.461575in}{2.790772in}}%
\pgfpathlineto{\pgfqpoint{3.572472in}{2.726746in}}%
\pgfpathlineto{\pgfqpoint{3.572472in}{2.854798in}}%
\pgfpathlineto{\pgfqpoint{3.461575in}{2.918824in}}%
\pgfpathlineto{\pgfqpoint{3.461575in}{2.790772in}}%
\pgfpathclose%
\pgfusepath{fill}%
\end{pgfscope}%
\begin{pgfscope}%
\pgfpathrectangle{\pgfqpoint{0.765000in}{0.660000in}}{\pgfqpoint{4.620000in}{4.620000in}}%
\pgfusepath{clip}%
\pgfsetbuttcap%
\pgfsetroundjoin%
\definecolor{currentfill}{rgb}{1.000000,0.894118,0.788235}%
\pgfsetfillcolor{currentfill}%
\pgfsetlinewidth{0.000000pt}%
\definecolor{currentstroke}{rgb}{1.000000,0.894118,0.788235}%
\pgfsetstrokecolor{currentstroke}%
\pgfsetdash{}{0pt}%
\pgfpathmoveto{\pgfqpoint{3.563846in}{2.808203in}}%
\pgfpathlineto{\pgfqpoint{3.452949in}{2.744176in}}%
\pgfpathlineto{\pgfqpoint{3.461575in}{2.790772in}}%
\pgfpathlineto{\pgfqpoint{3.572472in}{2.854798in}}%
\pgfpathlineto{\pgfqpoint{3.563846in}{2.808203in}}%
\pgfpathclose%
\pgfusepath{fill}%
\end{pgfscope}%
\begin{pgfscope}%
\pgfpathrectangle{\pgfqpoint{0.765000in}{0.660000in}}{\pgfqpoint{4.620000in}{4.620000in}}%
\pgfusepath{clip}%
\pgfsetbuttcap%
\pgfsetroundjoin%
\definecolor{currentfill}{rgb}{1.000000,0.894118,0.788235}%
\pgfsetfillcolor{currentfill}%
\pgfsetlinewidth{0.000000pt}%
\definecolor{currentstroke}{rgb}{1.000000,0.894118,0.788235}%
\pgfsetstrokecolor{currentstroke}%
\pgfsetdash{}{0pt}%
\pgfpathmoveto{\pgfqpoint{3.674742in}{2.744176in}}%
\pgfpathlineto{\pgfqpoint{3.563846in}{2.808203in}}%
\pgfpathlineto{\pgfqpoint{3.572472in}{2.854798in}}%
\pgfpathlineto{\pgfqpoint{3.683368in}{2.790772in}}%
\pgfpathlineto{\pgfqpoint{3.674742in}{2.744176in}}%
\pgfpathclose%
\pgfusepath{fill}%
\end{pgfscope}%
\begin{pgfscope}%
\pgfpathrectangle{\pgfqpoint{0.765000in}{0.660000in}}{\pgfqpoint{4.620000in}{4.620000in}}%
\pgfusepath{clip}%
\pgfsetbuttcap%
\pgfsetroundjoin%
\definecolor{currentfill}{rgb}{1.000000,0.894118,0.788235}%
\pgfsetfillcolor{currentfill}%
\pgfsetlinewidth{0.000000pt}%
\definecolor{currentstroke}{rgb}{1.000000,0.894118,0.788235}%
\pgfsetstrokecolor{currentstroke}%
\pgfsetdash{}{0pt}%
\pgfpathmoveto{\pgfqpoint{3.563846in}{2.808203in}}%
\pgfpathlineto{\pgfqpoint{3.563846in}{2.936255in}}%
\pgfpathlineto{\pgfqpoint{3.572472in}{2.982851in}}%
\pgfpathlineto{\pgfqpoint{3.683368in}{2.790772in}}%
\pgfpathlineto{\pgfqpoint{3.563846in}{2.808203in}}%
\pgfpathclose%
\pgfusepath{fill}%
\end{pgfscope}%
\begin{pgfscope}%
\pgfpathrectangle{\pgfqpoint{0.765000in}{0.660000in}}{\pgfqpoint{4.620000in}{4.620000in}}%
\pgfusepath{clip}%
\pgfsetbuttcap%
\pgfsetroundjoin%
\definecolor{currentfill}{rgb}{1.000000,0.894118,0.788235}%
\pgfsetfillcolor{currentfill}%
\pgfsetlinewidth{0.000000pt}%
\definecolor{currentstroke}{rgb}{1.000000,0.894118,0.788235}%
\pgfsetstrokecolor{currentstroke}%
\pgfsetdash{}{0pt}%
\pgfpathmoveto{\pgfqpoint{3.674742in}{2.744176in}}%
\pgfpathlineto{\pgfqpoint{3.674742in}{2.872229in}}%
\pgfpathlineto{\pgfqpoint{3.683368in}{2.918824in}}%
\pgfpathlineto{\pgfqpoint{3.572472in}{2.854798in}}%
\pgfpathlineto{\pgfqpoint{3.674742in}{2.744176in}}%
\pgfpathclose%
\pgfusepath{fill}%
\end{pgfscope}%
\begin{pgfscope}%
\pgfpathrectangle{\pgfqpoint{0.765000in}{0.660000in}}{\pgfqpoint{4.620000in}{4.620000in}}%
\pgfusepath{clip}%
\pgfsetbuttcap%
\pgfsetroundjoin%
\definecolor{currentfill}{rgb}{1.000000,0.894118,0.788235}%
\pgfsetfillcolor{currentfill}%
\pgfsetlinewidth{0.000000pt}%
\definecolor{currentstroke}{rgb}{1.000000,0.894118,0.788235}%
\pgfsetstrokecolor{currentstroke}%
\pgfsetdash{}{0pt}%
\pgfpathmoveto{\pgfqpoint{3.452949in}{2.744176in}}%
\pgfpathlineto{\pgfqpoint{3.563846in}{2.680150in}}%
\pgfpathlineto{\pgfqpoint{3.572472in}{2.726746in}}%
\pgfpathlineto{\pgfqpoint{3.461575in}{2.790772in}}%
\pgfpathlineto{\pgfqpoint{3.452949in}{2.744176in}}%
\pgfpathclose%
\pgfusepath{fill}%
\end{pgfscope}%
\begin{pgfscope}%
\pgfpathrectangle{\pgfqpoint{0.765000in}{0.660000in}}{\pgfqpoint{4.620000in}{4.620000in}}%
\pgfusepath{clip}%
\pgfsetbuttcap%
\pgfsetroundjoin%
\definecolor{currentfill}{rgb}{1.000000,0.894118,0.788235}%
\pgfsetfillcolor{currentfill}%
\pgfsetlinewidth{0.000000pt}%
\definecolor{currentstroke}{rgb}{1.000000,0.894118,0.788235}%
\pgfsetstrokecolor{currentstroke}%
\pgfsetdash{}{0pt}%
\pgfpathmoveto{\pgfqpoint{3.563846in}{2.680150in}}%
\pgfpathlineto{\pgfqpoint{3.674742in}{2.744176in}}%
\pgfpathlineto{\pgfqpoint{3.683368in}{2.790772in}}%
\pgfpathlineto{\pgfqpoint{3.572472in}{2.726746in}}%
\pgfpathlineto{\pgfqpoint{3.563846in}{2.680150in}}%
\pgfpathclose%
\pgfusepath{fill}%
\end{pgfscope}%
\begin{pgfscope}%
\pgfpathrectangle{\pgfqpoint{0.765000in}{0.660000in}}{\pgfqpoint{4.620000in}{4.620000in}}%
\pgfusepath{clip}%
\pgfsetbuttcap%
\pgfsetroundjoin%
\definecolor{currentfill}{rgb}{1.000000,0.894118,0.788235}%
\pgfsetfillcolor{currentfill}%
\pgfsetlinewidth{0.000000pt}%
\definecolor{currentstroke}{rgb}{1.000000,0.894118,0.788235}%
\pgfsetstrokecolor{currentstroke}%
\pgfsetdash{}{0pt}%
\pgfpathmoveto{\pgfqpoint{3.563846in}{2.936255in}}%
\pgfpathlineto{\pgfqpoint{3.452949in}{2.872229in}}%
\pgfpathlineto{\pgfqpoint{3.461575in}{2.918824in}}%
\pgfpathlineto{\pgfqpoint{3.572472in}{2.982851in}}%
\pgfpathlineto{\pgfqpoint{3.563846in}{2.936255in}}%
\pgfpathclose%
\pgfusepath{fill}%
\end{pgfscope}%
\begin{pgfscope}%
\pgfpathrectangle{\pgfqpoint{0.765000in}{0.660000in}}{\pgfqpoint{4.620000in}{4.620000in}}%
\pgfusepath{clip}%
\pgfsetbuttcap%
\pgfsetroundjoin%
\definecolor{currentfill}{rgb}{1.000000,0.894118,0.788235}%
\pgfsetfillcolor{currentfill}%
\pgfsetlinewidth{0.000000pt}%
\definecolor{currentstroke}{rgb}{1.000000,0.894118,0.788235}%
\pgfsetstrokecolor{currentstroke}%
\pgfsetdash{}{0pt}%
\pgfpathmoveto{\pgfqpoint{3.674742in}{2.872229in}}%
\pgfpathlineto{\pgfqpoint{3.563846in}{2.936255in}}%
\pgfpathlineto{\pgfqpoint{3.572472in}{2.982851in}}%
\pgfpathlineto{\pgfqpoint{3.683368in}{2.918824in}}%
\pgfpathlineto{\pgfqpoint{3.674742in}{2.872229in}}%
\pgfpathclose%
\pgfusepath{fill}%
\end{pgfscope}%
\begin{pgfscope}%
\pgfpathrectangle{\pgfqpoint{0.765000in}{0.660000in}}{\pgfqpoint{4.620000in}{4.620000in}}%
\pgfusepath{clip}%
\pgfsetbuttcap%
\pgfsetroundjoin%
\definecolor{currentfill}{rgb}{1.000000,0.894118,0.788235}%
\pgfsetfillcolor{currentfill}%
\pgfsetlinewidth{0.000000pt}%
\definecolor{currentstroke}{rgb}{1.000000,0.894118,0.788235}%
\pgfsetstrokecolor{currentstroke}%
\pgfsetdash{}{0pt}%
\pgfpathmoveto{\pgfqpoint{3.452949in}{2.744176in}}%
\pgfpathlineto{\pgfqpoint{3.452949in}{2.872229in}}%
\pgfpathlineto{\pgfqpoint{3.461575in}{2.918824in}}%
\pgfpathlineto{\pgfqpoint{3.572472in}{2.726746in}}%
\pgfpathlineto{\pgfqpoint{3.452949in}{2.744176in}}%
\pgfpathclose%
\pgfusepath{fill}%
\end{pgfscope}%
\begin{pgfscope}%
\pgfpathrectangle{\pgfqpoint{0.765000in}{0.660000in}}{\pgfqpoint{4.620000in}{4.620000in}}%
\pgfusepath{clip}%
\pgfsetbuttcap%
\pgfsetroundjoin%
\definecolor{currentfill}{rgb}{1.000000,0.894118,0.788235}%
\pgfsetfillcolor{currentfill}%
\pgfsetlinewidth{0.000000pt}%
\definecolor{currentstroke}{rgb}{1.000000,0.894118,0.788235}%
\pgfsetstrokecolor{currentstroke}%
\pgfsetdash{}{0pt}%
\pgfpathmoveto{\pgfqpoint{3.563846in}{2.680150in}}%
\pgfpathlineto{\pgfqpoint{3.563846in}{2.808203in}}%
\pgfpathlineto{\pgfqpoint{3.572472in}{2.854798in}}%
\pgfpathlineto{\pgfqpoint{3.461575in}{2.790772in}}%
\pgfpathlineto{\pgfqpoint{3.563846in}{2.680150in}}%
\pgfpathclose%
\pgfusepath{fill}%
\end{pgfscope}%
\begin{pgfscope}%
\pgfpathrectangle{\pgfqpoint{0.765000in}{0.660000in}}{\pgfqpoint{4.620000in}{4.620000in}}%
\pgfusepath{clip}%
\pgfsetbuttcap%
\pgfsetroundjoin%
\definecolor{currentfill}{rgb}{1.000000,0.894118,0.788235}%
\pgfsetfillcolor{currentfill}%
\pgfsetlinewidth{0.000000pt}%
\definecolor{currentstroke}{rgb}{1.000000,0.894118,0.788235}%
\pgfsetstrokecolor{currentstroke}%
\pgfsetdash{}{0pt}%
\pgfpathmoveto{\pgfqpoint{3.563846in}{2.808203in}}%
\pgfpathlineto{\pgfqpoint{3.674742in}{2.872229in}}%
\pgfpathlineto{\pgfqpoint{3.683368in}{2.918824in}}%
\pgfpathlineto{\pgfqpoint{3.572472in}{2.854798in}}%
\pgfpathlineto{\pgfqpoint{3.563846in}{2.808203in}}%
\pgfpathclose%
\pgfusepath{fill}%
\end{pgfscope}%
\begin{pgfscope}%
\pgfpathrectangle{\pgfqpoint{0.765000in}{0.660000in}}{\pgfqpoint{4.620000in}{4.620000in}}%
\pgfusepath{clip}%
\pgfsetbuttcap%
\pgfsetroundjoin%
\definecolor{currentfill}{rgb}{1.000000,0.894118,0.788235}%
\pgfsetfillcolor{currentfill}%
\pgfsetlinewidth{0.000000pt}%
\definecolor{currentstroke}{rgb}{1.000000,0.894118,0.788235}%
\pgfsetstrokecolor{currentstroke}%
\pgfsetdash{}{0pt}%
\pgfpathmoveto{\pgfqpoint{3.452949in}{2.872229in}}%
\pgfpathlineto{\pgfqpoint{3.563846in}{2.808203in}}%
\pgfpathlineto{\pgfqpoint{3.572472in}{2.854798in}}%
\pgfpathlineto{\pgfqpoint{3.461575in}{2.918824in}}%
\pgfpathlineto{\pgfqpoint{3.452949in}{2.872229in}}%
\pgfpathclose%
\pgfusepath{fill}%
\end{pgfscope}%
\begin{pgfscope}%
\pgfpathrectangle{\pgfqpoint{0.765000in}{0.660000in}}{\pgfqpoint{4.620000in}{4.620000in}}%
\pgfusepath{clip}%
\pgfsetbuttcap%
\pgfsetroundjoin%
\definecolor{currentfill}{rgb}{1.000000,0.894118,0.788235}%
\pgfsetfillcolor{currentfill}%
\pgfsetlinewidth{0.000000pt}%
\definecolor{currentstroke}{rgb}{1.000000,0.894118,0.788235}%
\pgfsetstrokecolor{currentstroke}%
\pgfsetdash{}{0pt}%
\pgfpathmoveto{\pgfqpoint{3.551407in}{2.776901in}}%
\pgfpathlineto{\pgfqpoint{3.440511in}{2.712874in}}%
\pgfpathlineto{\pgfqpoint{3.551407in}{2.648848in}}%
\pgfpathlineto{\pgfqpoint{3.662304in}{2.712874in}}%
\pgfpathlineto{\pgfqpoint{3.551407in}{2.776901in}}%
\pgfpathclose%
\pgfusepath{fill}%
\end{pgfscope}%
\begin{pgfscope}%
\pgfpathrectangle{\pgfqpoint{0.765000in}{0.660000in}}{\pgfqpoint{4.620000in}{4.620000in}}%
\pgfusepath{clip}%
\pgfsetbuttcap%
\pgfsetroundjoin%
\definecolor{currentfill}{rgb}{1.000000,0.894118,0.788235}%
\pgfsetfillcolor{currentfill}%
\pgfsetlinewidth{0.000000pt}%
\definecolor{currentstroke}{rgb}{1.000000,0.894118,0.788235}%
\pgfsetstrokecolor{currentstroke}%
\pgfsetdash{}{0pt}%
\pgfpathmoveto{\pgfqpoint{3.551407in}{2.776901in}}%
\pgfpathlineto{\pgfqpoint{3.440511in}{2.712874in}}%
\pgfpathlineto{\pgfqpoint{3.440511in}{2.840927in}}%
\pgfpathlineto{\pgfqpoint{3.551407in}{2.904953in}}%
\pgfpathlineto{\pgfqpoint{3.551407in}{2.776901in}}%
\pgfpathclose%
\pgfusepath{fill}%
\end{pgfscope}%
\begin{pgfscope}%
\pgfpathrectangle{\pgfqpoint{0.765000in}{0.660000in}}{\pgfqpoint{4.620000in}{4.620000in}}%
\pgfusepath{clip}%
\pgfsetbuttcap%
\pgfsetroundjoin%
\definecolor{currentfill}{rgb}{1.000000,0.894118,0.788235}%
\pgfsetfillcolor{currentfill}%
\pgfsetlinewidth{0.000000pt}%
\definecolor{currentstroke}{rgb}{1.000000,0.894118,0.788235}%
\pgfsetstrokecolor{currentstroke}%
\pgfsetdash{}{0pt}%
\pgfpathmoveto{\pgfqpoint{3.551407in}{2.776901in}}%
\pgfpathlineto{\pgfqpoint{3.662304in}{2.712874in}}%
\pgfpathlineto{\pgfqpoint{3.662304in}{2.840927in}}%
\pgfpathlineto{\pgfqpoint{3.551407in}{2.904953in}}%
\pgfpathlineto{\pgfqpoint{3.551407in}{2.776901in}}%
\pgfpathclose%
\pgfusepath{fill}%
\end{pgfscope}%
\begin{pgfscope}%
\pgfpathrectangle{\pgfqpoint{0.765000in}{0.660000in}}{\pgfqpoint{4.620000in}{4.620000in}}%
\pgfusepath{clip}%
\pgfsetbuttcap%
\pgfsetroundjoin%
\definecolor{currentfill}{rgb}{1.000000,0.894118,0.788235}%
\pgfsetfillcolor{currentfill}%
\pgfsetlinewidth{0.000000pt}%
\definecolor{currentstroke}{rgb}{1.000000,0.894118,0.788235}%
\pgfsetstrokecolor{currentstroke}%
\pgfsetdash{}{0pt}%
\pgfpathmoveto{\pgfqpoint{3.546263in}{2.765531in}}%
\pgfpathlineto{\pgfqpoint{3.435366in}{2.701504in}}%
\pgfpathlineto{\pgfqpoint{3.546263in}{2.637478in}}%
\pgfpathlineto{\pgfqpoint{3.657159in}{2.701504in}}%
\pgfpathlineto{\pgfqpoint{3.546263in}{2.765531in}}%
\pgfpathclose%
\pgfusepath{fill}%
\end{pgfscope}%
\begin{pgfscope}%
\pgfpathrectangle{\pgfqpoint{0.765000in}{0.660000in}}{\pgfqpoint{4.620000in}{4.620000in}}%
\pgfusepath{clip}%
\pgfsetbuttcap%
\pgfsetroundjoin%
\definecolor{currentfill}{rgb}{1.000000,0.894118,0.788235}%
\pgfsetfillcolor{currentfill}%
\pgfsetlinewidth{0.000000pt}%
\definecolor{currentstroke}{rgb}{1.000000,0.894118,0.788235}%
\pgfsetstrokecolor{currentstroke}%
\pgfsetdash{}{0pt}%
\pgfpathmoveto{\pgfqpoint{3.546263in}{2.765531in}}%
\pgfpathlineto{\pgfqpoint{3.435366in}{2.701504in}}%
\pgfpathlineto{\pgfqpoint{3.435366in}{2.829557in}}%
\pgfpathlineto{\pgfqpoint{3.546263in}{2.893583in}}%
\pgfpathlineto{\pgfqpoint{3.546263in}{2.765531in}}%
\pgfpathclose%
\pgfusepath{fill}%
\end{pgfscope}%
\begin{pgfscope}%
\pgfpathrectangle{\pgfqpoint{0.765000in}{0.660000in}}{\pgfqpoint{4.620000in}{4.620000in}}%
\pgfusepath{clip}%
\pgfsetbuttcap%
\pgfsetroundjoin%
\definecolor{currentfill}{rgb}{1.000000,0.894118,0.788235}%
\pgfsetfillcolor{currentfill}%
\pgfsetlinewidth{0.000000pt}%
\definecolor{currentstroke}{rgb}{1.000000,0.894118,0.788235}%
\pgfsetstrokecolor{currentstroke}%
\pgfsetdash{}{0pt}%
\pgfpathmoveto{\pgfqpoint{3.546263in}{2.765531in}}%
\pgfpathlineto{\pgfqpoint{3.657159in}{2.701504in}}%
\pgfpathlineto{\pgfqpoint{3.657159in}{2.829557in}}%
\pgfpathlineto{\pgfqpoint{3.546263in}{2.893583in}}%
\pgfpathlineto{\pgfqpoint{3.546263in}{2.765531in}}%
\pgfpathclose%
\pgfusepath{fill}%
\end{pgfscope}%
\begin{pgfscope}%
\pgfpathrectangle{\pgfqpoint{0.765000in}{0.660000in}}{\pgfqpoint{4.620000in}{4.620000in}}%
\pgfusepath{clip}%
\pgfsetbuttcap%
\pgfsetroundjoin%
\definecolor{currentfill}{rgb}{1.000000,0.894118,0.788235}%
\pgfsetfillcolor{currentfill}%
\pgfsetlinewidth{0.000000pt}%
\definecolor{currentstroke}{rgb}{1.000000,0.894118,0.788235}%
\pgfsetstrokecolor{currentstroke}%
\pgfsetdash{}{0pt}%
\pgfpathmoveto{\pgfqpoint{3.551407in}{2.904953in}}%
\pgfpathlineto{\pgfqpoint{3.440511in}{2.840927in}}%
\pgfpathlineto{\pgfqpoint{3.551407in}{2.776901in}}%
\pgfpathlineto{\pgfqpoint{3.662304in}{2.840927in}}%
\pgfpathlineto{\pgfqpoint{3.551407in}{2.904953in}}%
\pgfpathclose%
\pgfusepath{fill}%
\end{pgfscope}%
\begin{pgfscope}%
\pgfpathrectangle{\pgfqpoint{0.765000in}{0.660000in}}{\pgfqpoint{4.620000in}{4.620000in}}%
\pgfusepath{clip}%
\pgfsetbuttcap%
\pgfsetroundjoin%
\definecolor{currentfill}{rgb}{1.000000,0.894118,0.788235}%
\pgfsetfillcolor{currentfill}%
\pgfsetlinewidth{0.000000pt}%
\definecolor{currentstroke}{rgb}{1.000000,0.894118,0.788235}%
\pgfsetstrokecolor{currentstroke}%
\pgfsetdash{}{0pt}%
\pgfpathmoveto{\pgfqpoint{3.551407in}{2.648848in}}%
\pgfpathlineto{\pgfqpoint{3.662304in}{2.712874in}}%
\pgfpathlineto{\pgfqpoint{3.662304in}{2.840927in}}%
\pgfpathlineto{\pgfqpoint{3.551407in}{2.776901in}}%
\pgfpathlineto{\pgfqpoint{3.551407in}{2.648848in}}%
\pgfpathclose%
\pgfusepath{fill}%
\end{pgfscope}%
\begin{pgfscope}%
\pgfpathrectangle{\pgfqpoint{0.765000in}{0.660000in}}{\pgfqpoint{4.620000in}{4.620000in}}%
\pgfusepath{clip}%
\pgfsetbuttcap%
\pgfsetroundjoin%
\definecolor{currentfill}{rgb}{1.000000,0.894118,0.788235}%
\pgfsetfillcolor{currentfill}%
\pgfsetlinewidth{0.000000pt}%
\definecolor{currentstroke}{rgb}{1.000000,0.894118,0.788235}%
\pgfsetstrokecolor{currentstroke}%
\pgfsetdash{}{0pt}%
\pgfpathmoveto{\pgfqpoint{3.440511in}{2.712874in}}%
\pgfpathlineto{\pgfqpoint{3.551407in}{2.648848in}}%
\pgfpathlineto{\pgfqpoint{3.551407in}{2.776901in}}%
\pgfpathlineto{\pgfqpoint{3.440511in}{2.840927in}}%
\pgfpathlineto{\pgfqpoint{3.440511in}{2.712874in}}%
\pgfpathclose%
\pgfusepath{fill}%
\end{pgfscope}%
\begin{pgfscope}%
\pgfpathrectangle{\pgfqpoint{0.765000in}{0.660000in}}{\pgfqpoint{4.620000in}{4.620000in}}%
\pgfusepath{clip}%
\pgfsetbuttcap%
\pgfsetroundjoin%
\definecolor{currentfill}{rgb}{1.000000,0.894118,0.788235}%
\pgfsetfillcolor{currentfill}%
\pgfsetlinewidth{0.000000pt}%
\definecolor{currentstroke}{rgb}{1.000000,0.894118,0.788235}%
\pgfsetstrokecolor{currentstroke}%
\pgfsetdash{}{0pt}%
\pgfpathmoveto{\pgfqpoint{3.546263in}{2.893583in}}%
\pgfpathlineto{\pgfqpoint{3.435366in}{2.829557in}}%
\pgfpathlineto{\pgfqpoint{3.546263in}{2.765531in}}%
\pgfpathlineto{\pgfqpoint{3.657159in}{2.829557in}}%
\pgfpathlineto{\pgfqpoint{3.546263in}{2.893583in}}%
\pgfpathclose%
\pgfusepath{fill}%
\end{pgfscope}%
\begin{pgfscope}%
\pgfpathrectangle{\pgfqpoint{0.765000in}{0.660000in}}{\pgfqpoint{4.620000in}{4.620000in}}%
\pgfusepath{clip}%
\pgfsetbuttcap%
\pgfsetroundjoin%
\definecolor{currentfill}{rgb}{1.000000,0.894118,0.788235}%
\pgfsetfillcolor{currentfill}%
\pgfsetlinewidth{0.000000pt}%
\definecolor{currentstroke}{rgb}{1.000000,0.894118,0.788235}%
\pgfsetstrokecolor{currentstroke}%
\pgfsetdash{}{0pt}%
\pgfpathmoveto{\pgfqpoint{3.546263in}{2.637478in}}%
\pgfpathlineto{\pgfqpoint{3.657159in}{2.701504in}}%
\pgfpathlineto{\pgfqpoint{3.657159in}{2.829557in}}%
\pgfpathlineto{\pgfqpoint{3.546263in}{2.765531in}}%
\pgfpathlineto{\pgfqpoint{3.546263in}{2.637478in}}%
\pgfpathclose%
\pgfusepath{fill}%
\end{pgfscope}%
\begin{pgfscope}%
\pgfpathrectangle{\pgfqpoint{0.765000in}{0.660000in}}{\pgfqpoint{4.620000in}{4.620000in}}%
\pgfusepath{clip}%
\pgfsetbuttcap%
\pgfsetroundjoin%
\definecolor{currentfill}{rgb}{1.000000,0.894118,0.788235}%
\pgfsetfillcolor{currentfill}%
\pgfsetlinewidth{0.000000pt}%
\definecolor{currentstroke}{rgb}{1.000000,0.894118,0.788235}%
\pgfsetstrokecolor{currentstroke}%
\pgfsetdash{}{0pt}%
\pgfpathmoveto{\pgfqpoint{3.435366in}{2.701504in}}%
\pgfpathlineto{\pgfqpoint{3.546263in}{2.637478in}}%
\pgfpathlineto{\pgfqpoint{3.546263in}{2.765531in}}%
\pgfpathlineto{\pgfqpoint{3.435366in}{2.829557in}}%
\pgfpathlineto{\pgfqpoint{3.435366in}{2.701504in}}%
\pgfpathclose%
\pgfusepath{fill}%
\end{pgfscope}%
\begin{pgfscope}%
\pgfpathrectangle{\pgfqpoint{0.765000in}{0.660000in}}{\pgfqpoint{4.620000in}{4.620000in}}%
\pgfusepath{clip}%
\pgfsetbuttcap%
\pgfsetroundjoin%
\definecolor{currentfill}{rgb}{1.000000,0.894118,0.788235}%
\pgfsetfillcolor{currentfill}%
\pgfsetlinewidth{0.000000pt}%
\definecolor{currentstroke}{rgb}{1.000000,0.894118,0.788235}%
\pgfsetstrokecolor{currentstroke}%
\pgfsetdash{}{0pt}%
\pgfpathmoveto{\pgfqpoint{3.551407in}{2.776901in}}%
\pgfpathlineto{\pgfqpoint{3.440511in}{2.712874in}}%
\pgfpathlineto{\pgfqpoint{3.435366in}{2.701504in}}%
\pgfpathlineto{\pgfqpoint{3.546263in}{2.765531in}}%
\pgfpathlineto{\pgfqpoint{3.551407in}{2.776901in}}%
\pgfpathclose%
\pgfusepath{fill}%
\end{pgfscope}%
\begin{pgfscope}%
\pgfpathrectangle{\pgfqpoint{0.765000in}{0.660000in}}{\pgfqpoint{4.620000in}{4.620000in}}%
\pgfusepath{clip}%
\pgfsetbuttcap%
\pgfsetroundjoin%
\definecolor{currentfill}{rgb}{1.000000,0.894118,0.788235}%
\pgfsetfillcolor{currentfill}%
\pgfsetlinewidth{0.000000pt}%
\definecolor{currentstroke}{rgb}{1.000000,0.894118,0.788235}%
\pgfsetstrokecolor{currentstroke}%
\pgfsetdash{}{0pt}%
\pgfpathmoveto{\pgfqpoint{3.662304in}{2.712874in}}%
\pgfpathlineto{\pgfqpoint{3.551407in}{2.776901in}}%
\pgfpathlineto{\pgfqpoint{3.546263in}{2.765531in}}%
\pgfpathlineto{\pgfqpoint{3.657159in}{2.701504in}}%
\pgfpathlineto{\pgfqpoint{3.662304in}{2.712874in}}%
\pgfpathclose%
\pgfusepath{fill}%
\end{pgfscope}%
\begin{pgfscope}%
\pgfpathrectangle{\pgfqpoint{0.765000in}{0.660000in}}{\pgfqpoint{4.620000in}{4.620000in}}%
\pgfusepath{clip}%
\pgfsetbuttcap%
\pgfsetroundjoin%
\definecolor{currentfill}{rgb}{1.000000,0.894118,0.788235}%
\pgfsetfillcolor{currentfill}%
\pgfsetlinewidth{0.000000pt}%
\definecolor{currentstroke}{rgb}{1.000000,0.894118,0.788235}%
\pgfsetstrokecolor{currentstroke}%
\pgfsetdash{}{0pt}%
\pgfpathmoveto{\pgfqpoint{3.551407in}{2.776901in}}%
\pgfpathlineto{\pgfqpoint{3.551407in}{2.904953in}}%
\pgfpathlineto{\pgfqpoint{3.546263in}{2.893583in}}%
\pgfpathlineto{\pgfqpoint{3.657159in}{2.701504in}}%
\pgfpathlineto{\pgfqpoint{3.551407in}{2.776901in}}%
\pgfpathclose%
\pgfusepath{fill}%
\end{pgfscope}%
\begin{pgfscope}%
\pgfpathrectangle{\pgfqpoint{0.765000in}{0.660000in}}{\pgfqpoint{4.620000in}{4.620000in}}%
\pgfusepath{clip}%
\pgfsetbuttcap%
\pgfsetroundjoin%
\definecolor{currentfill}{rgb}{1.000000,0.894118,0.788235}%
\pgfsetfillcolor{currentfill}%
\pgfsetlinewidth{0.000000pt}%
\definecolor{currentstroke}{rgb}{1.000000,0.894118,0.788235}%
\pgfsetstrokecolor{currentstroke}%
\pgfsetdash{}{0pt}%
\pgfpathmoveto{\pgfqpoint{3.662304in}{2.712874in}}%
\pgfpathlineto{\pgfqpoint{3.662304in}{2.840927in}}%
\pgfpathlineto{\pgfqpoint{3.657159in}{2.829557in}}%
\pgfpathlineto{\pgfqpoint{3.546263in}{2.765531in}}%
\pgfpathlineto{\pgfqpoint{3.662304in}{2.712874in}}%
\pgfpathclose%
\pgfusepath{fill}%
\end{pgfscope}%
\begin{pgfscope}%
\pgfpathrectangle{\pgfqpoint{0.765000in}{0.660000in}}{\pgfqpoint{4.620000in}{4.620000in}}%
\pgfusepath{clip}%
\pgfsetbuttcap%
\pgfsetroundjoin%
\definecolor{currentfill}{rgb}{1.000000,0.894118,0.788235}%
\pgfsetfillcolor{currentfill}%
\pgfsetlinewidth{0.000000pt}%
\definecolor{currentstroke}{rgb}{1.000000,0.894118,0.788235}%
\pgfsetstrokecolor{currentstroke}%
\pgfsetdash{}{0pt}%
\pgfpathmoveto{\pgfqpoint{3.440511in}{2.712874in}}%
\pgfpathlineto{\pgfqpoint{3.551407in}{2.648848in}}%
\pgfpathlineto{\pgfqpoint{3.546263in}{2.637478in}}%
\pgfpathlineto{\pgfqpoint{3.435366in}{2.701504in}}%
\pgfpathlineto{\pgfqpoint{3.440511in}{2.712874in}}%
\pgfpathclose%
\pgfusepath{fill}%
\end{pgfscope}%
\begin{pgfscope}%
\pgfpathrectangle{\pgfqpoint{0.765000in}{0.660000in}}{\pgfqpoint{4.620000in}{4.620000in}}%
\pgfusepath{clip}%
\pgfsetbuttcap%
\pgfsetroundjoin%
\definecolor{currentfill}{rgb}{1.000000,0.894118,0.788235}%
\pgfsetfillcolor{currentfill}%
\pgfsetlinewidth{0.000000pt}%
\definecolor{currentstroke}{rgb}{1.000000,0.894118,0.788235}%
\pgfsetstrokecolor{currentstroke}%
\pgfsetdash{}{0pt}%
\pgfpathmoveto{\pgfqpoint{3.551407in}{2.648848in}}%
\pgfpathlineto{\pgfqpoint{3.662304in}{2.712874in}}%
\pgfpathlineto{\pgfqpoint{3.657159in}{2.701504in}}%
\pgfpathlineto{\pgfqpoint{3.546263in}{2.637478in}}%
\pgfpathlineto{\pgfqpoint{3.551407in}{2.648848in}}%
\pgfpathclose%
\pgfusepath{fill}%
\end{pgfscope}%
\begin{pgfscope}%
\pgfpathrectangle{\pgfqpoint{0.765000in}{0.660000in}}{\pgfqpoint{4.620000in}{4.620000in}}%
\pgfusepath{clip}%
\pgfsetbuttcap%
\pgfsetroundjoin%
\definecolor{currentfill}{rgb}{1.000000,0.894118,0.788235}%
\pgfsetfillcolor{currentfill}%
\pgfsetlinewidth{0.000000pt}%
\definecolor{currentstroke}{rgb}{1.000000,0.894118,0.788235}%
\pgfsetstrokecolor{currentstroke}%
\pgfsetdash{}{0pt}%
\pgfpathmoveto{\pgfqpoint{3.551407in}{2.904953in}}%
\pgfpathlineto{\pgfqpoint{3.440511in}{2.840927in}}%
\pgfpathlineto{\pgfqpoint{3.435366in}{2.829557in}}%
\pgfpathlineto{\pgfqpoint{3.546263in}{2.893583in}}%
\pgfpathlineto{\pgfqpoint{3.551407in}{2.904953in}}%
\pgfpathclose%
\pgfusepath{fill}%
\end{pgfscope}%
\begin{pgfscope}%
\pgfpathrectangle{\pgfqpoint{0.765000in}{0.660000in}}{\pgfqpoint{4.620000in}{4.620000in}}%
\pgfusepath{clip}%
\pgfsetbuttcap%
\pgfsetroundjoin%
\definecolor{currentfill}{rgb}{1.000000,0.894118,0.788235}%
\pgfsetfillcolor{currentfill}%
\pgfsetlinewidth{0.000000pt}%
\definecolor{currentstroke}{rgb}{1.000000,0.894118,0.788235}%
\pgfsetstrokecolor{currentstroke}%
\pgfsetdash{}{0pt}%
\pgfpathmoveto{\pgfqpoint{3.662304in}{2.840927in}}%
\pgfpathlineto{\pgfqpoint{3.551407in}{2.904953in}}%
\pgfpathlineto{\pgfqpoint{3.546263in}{2.893583in}}%
\pgfpathlineto{\pgfqpoint{3.657159in}{2.829557in}}%
\pgfpathlineto{\pgfqpoint{3.662304in}{2.840927in}}%
\pgfpathclose%
\pgfusepath{fill}%
\end{pgfscope}%
\begin{pgfscope}%
\pgfpathrectangle{\pgfqpoint{0.765000in}{0.660000in}}{\pgfqpoint{4.620000in}{4.620000in}}%
\pgfusepath{clip}%
\pgfsetbuttcap%
\pgfsetroundjoin%
\definecolor{currentfill}{rgb}{1.000000,0.894118,0.788235}%
\pgfsetfillcolor{currentfill}%
\pgfsetlinewidth{0.000000pt}%
\definecolor{currentstroke}{rgb}{1.000000,0.894118,0.788235}%
\pgfsetstrokecolor{currentstroke}%
\pgfsetdash{}{0pt}%
\pgfpathmoveto{\pgfqpoint{3.440511in}{2.712874in}}%
\pgfpathlineto{\pgfqpoint{3.440511in}{2.840927in}}%
\pgfpathlineto{\pgfqpoint{3.435366in}{2.829557in}}%
\pgfpathlineto{\pgfqpoint{3.546263in}{2.637478in}}%
\pgfpathlineto{\pgfqpoint{3.440511in}{2.712874in}}%
\pgfpathclose%
\pgfusepath{fill}%
\end{pgfscope}%
\begin{pgfscope}%
\pgfpathrectangle{\pgfqpoint{0.765000in}{0.660000in}}{\pgfqpoint{4.620000in}{4.620000in}}%
\pgfusepath{clip}%
\pgfsetbuttcap%
\pgfsetroundjoin%
\definecolor{currentfill}{rgb}{1.000000,0.894118,0.788235}%
\pgfsetfillcolor{currentfill}%
\pgfsetlinewidth{0.000000pt}%
\definecolor{currentstroke}{rgb}{1.000000,0.894118,0.788235}%
\pgfsetstrokecolor{currentstroke}%
\pgfsetdash{}{0pt}%
\pgfpathmoveto{\pgfqpoint{3.551407in}{2.648848in}}%
\pgfpathlineto{\pgfqpoint{3.551407in}{2.776901in}}%
\pgfpathlineto{\pgfqpoint{3.546263in}{2.765531in}}%
\pgfpathlineto{\pgfqpoint{3.435366in}{2.701504in}}%
\pgfpathlineto{\pgfqpoint{3.551407in}{2.648848in}}%
\pgfpathclose%
\pgfusepath{fill}%
\end{pgfscope}%
\begin{pgfscope}%
\pgfpathrectangle{\pgfqpoint{0.765000in}{0.660000in}}{\pgfqpoint{4.620000in}{4.620000in}}%
\pgfusepath{clip}%
\pgfsetbuttcap%
\pgfsetroundjoin%
\definecolor{currentfill}{rgb}{1.000000,0.894118,0.788235}%
\pgfsetfillcolor{currentfill}%
\pgfsetlinewidth{0.000000pt}%
\definecolor{currentstroke}{rgb}{1.000000,0.894118,0.788235}%
\pgfsetstrokecolor{currentstroke}%
\pgfsetdash{}{0pt}%
\pgfpathmoveto{\pgfqpoint{3.551407in}{2.776901in}}%
\pgfpathlineto{\pgfqpoint{3.662304in}{2.840927in}}%
\pgfpathlineto{\pgfqpoint{3.657159in}{2.829557in}}%
\pgfpathlineto{\pgfqpoint{3.546263in}{2.765531in}}%
\pgfpathlineto{\pgfqpoint{3.551407in}{2.776901in}}%
\pgfpathclose%
\pgfusepath{fill}%
\end{pgfscope}%
\begin{pgfscope}%
\pgfpathrectangle{\pgfqpoint{0.765000in}{0.660000in}}{\pgfqpoint{4.620000in}{4.620000in}}%
\pgfusepath{clip}%
\pgfsetbuttcap%
\pgfsetroundjoin%
\definecolor{currentfill}{rgb}{1.000000,0.894118,0.788235}%
\pgfsetfillcolor{currentfill}%
\pgfsetlinewidth{0.000000pt}%
\definecolor{currentstroke}{rgb}{1.000000,0.894118,0.788235}%
\pgfsetstrokecolor{currentstroke}%
\pgfsetdash{}{0pt}%
\pgfpathmoveto{\pgfqpoint{3.440511in}{2.840927in}}%
\pgfpathlineto{\pgfqpoint{3.551407in}{2.776901in}}%
\pgfpathlineto{\pgfqpoint{3.546263in}{2.765531in}}%
\pgfpathlineto{\pgfqpoint{3.435366in}{2.829557in}}%
\pgfpathlineto{\pgfqpoint{3.440511in}{2.840927in}}%
\pgfpathclose%
\pgfusepath{fill}%
\end{pgfscope}%
\begin{pgfscope}%
\pgfpathrectangle{\pgfqpoint{0.765000in}{0.660000in}}{\pgfqpoint{4.620000in}{4.620000in}}%
\pgfusepath{clip}%
\pgfsetbuttcap%
\pgfsetroundjoin%
\definecolor{currentfill}{rgb}{1.000000,0.894118,0.788235}%
\pgfsetfillcolor{currentfill}%
\pgfsetlinewidth{0.000000pt}%
\definecolor{currentstroke}{rgb}{1.000000,0.894118,0.788235}%
\pgfsetstrokecolor{currentstroke}%
\pgfsetdash{}{0pt}%
\pgfpathmoveto{\pgfqpoint{3.551407in}{2.776901in}}%
\pgfpathlineto{\pgfqpoint{3.440511in}{2.712874in}}%
\pgfpathlineto{\pgfqpoint{3.551407in}{2.648848in}}%
\pgfpathlineto{\pgfqpoint{3.662304in}{2.712874in}}%
\pgfpathlineto{\pgfqpoint{3.551407in}{2.776901in}}%
\pgfpathclose%
\pgfusepath{fill}%
\end{pgfscope}%
\begin{pgfscope}%
\pgfpathrectangle{\pgfqpoint{0.765000in}{0.660000in}}{\pgfqpoint{4.620000in}{4.620000in}}%
\pgfusepath{clip}%
\pgfsetbuttcap%
\pgfsetroundjoin%
\definecolor{currentfill}{rgb}{1.000000,0.894118,0.788235}%
\pgfsetfillcolor{currentfill}%
\pgfsetlinewidth{0.000000pt}%
\definecolor{currentstroke}{rgb}{1.000000,0.894118,0.788235}%
\pgfsetstrokecolor{currentstroke}%
\pgfsetdash{}{0pt}%
\pgfpathmoveto{\pgfqpoint{3.551407in}{2.776901in}}%
\pgfpathlineto{\pgfqpoint{3.440511in}{2.712874in}}%
\pgfpathlineto{\pgfqpoint{3.440511in}{2.840927in}}%
\pgfpathlineto{\pgfqpoint{3.551407in}{2.904953in}}%
\pgfpathlineto{\pgfqpoint{3.551407in}{2.776901in}}%
\pgfpathclose%
\pgfusepath{fill}%
\end{pgfscope}%
\begin{pgfscope}%
\pgfpathrectangle{\pgfqpoint{0.765000in}{0.660000in}}{\pgfqpoint{4.620000in}{4.620000in}}%
\pgfusepath{clip}%
\pgfsetbuttcap%
\pgfsetroundjoin%
\definecolor{currentfill}{rgb}{1.000000,0.894118,0.788235}%
\pgfsetfillcolor{currentfill}%
\pgfsetlinewidth{0.000000pt}%
\definecolor{currentstroke}{rgb}{1.000000,0.894118,0.788235}%
\pgfsetstrokecolor{currentstroke}%
\pgfsetdash{}{0pt}%
\pgfpathmoveto{\pgfqpoint{3.551407in}{2.776901in}}%
\pgfpathlineto{\pgfqpoint{3.662304in}{2.712874in}}%
\pgfpathlineto{\pgfqpoint{3.662304in}{2.840927in}}%
\pgfpathlineto{\pgfqpoint{3.551407in}{2.904953in}}%
\pgfpathlineto{\pgfqpoint{3.551407in}{2.776901in}}%
\pgfpathclose%
\pgfusepath{fill}%
\end{pgfscope}%
\begin{pgfscope}%
\pgfpathrectangle{\pgfqpoint{0.765000in}{0.660000in}}{\pgfqpoint{4.620000in}{4.620000in}}%
\pgfusepath{clip}%
\pgfsetbuttcap%
\pgfsetroundjoin%
\definecolor{currentfill}{rgb}{1.000000,0.894118,0.788235}%
\pgfsetfillcolor{currentfill}%
\pgfsetlinewidth{0.000000pt}%
\definecolor{currentstroke}{rgb}{1.000000,0.894118,0.788235}%
\pgfsetstrokecolor{currentstroke}%
\pgfsetdash{}{0pt}%
\pgfpathmoveto{\pgfqpoint{3.563846in}{2.808203in}}%
\pgfpathlineto{\pgfqpoint{3.452949in}{2.744176in}}%
\pgfpathlineto{\pgfqpoint{3.563846in}{2.680150in}}%
\pgfpathlineto{\pgfqpoint{3.674742in}{2.744176in}}%
\pgfpathlineto{\pgfqpoint{3.563846in}{2.808203in}}%
\pgfpathclose%
\pgfusepath{fill}%
\end{pgfscope}%
\begin{pgfscope}%
\pgfpathrectangle{\pgfqpoint{0.765000in}{0.660000in}}{\pgfqpoint{4.620000in}{4.620000in}}%
\pgfusepath{clip}%
\pgfsetbuttcap%
\pgfsetroundjoin%
\definecolor{currentfill}{rgb}{1.000000,0.894118,0.788235}%
\pgfsetfillcolor{currentfill}%
\pgfsetlinewidth{0.000000pt}%
\definecolor{currentstroke}{rgb}{1.000000,0.894118,0.788235}%
\pgfsetstrokecolor{currentstroke}%
\pgfsetdash{}{0pt}%
\pgfpathmoveto{\pgfqpoint{3.563846in}{2.808203in}}%
\pgfpathlineto{\pgfqpoint{3.452949in}{2.744176in}}%
\pgfpathlineto{\pgfqpoint{3.452949in}{2.872229in}}%
\pgfpathlineto{\pgfqpoint{3.563846in}{2.936255in}}%
\pgfpathlineto{\pgfqpoint{3.563846in}{2.808203in}}%
\pgfpathclose%
\pgfusepath{fill}%
\end{pgfscope}%
\begin{pgfscope}%
\pgfpathrectangle{\pgfqpoint{0.765000in}{0.660000in}}{\pgfqpoint{4.620000in}{4.620000in}}%
\pgfusepath{clip}%
\pgfsetbuttcap%
\pgfsetroundjoin%
\definecolor{currentfill}{rgb}{1.000000,0.894118,0.788235}%
\pgfsetfillcolor{currentfill}%
\pgfsetlinewidth{0.000000pt}%
\definecolor{currentstroke}{rgb}{1.000000,0.894118,0.788235}%
\pgfsetstrokecolor{currentstroke}%
\pgfsetdash{}{0pt}%
\pgfpathmoveto{\pgfqpoint{3.563846in}{2.808203in}}%
\pgfpathlineto{\pgfqpoint{3.674742in}{2.744176in}}%
\pgfpathlineto{\pgfqpoint{3.674742in}{2.872229in}}%
\pgfpathlineto{\pgfqpoint{3.563846in}{2.936255in}}%
\pgfpathlineto{\pgfqpoint{3.563846in}{2.808203in}}%
\pgfpathclose%
\pgfusepath{fill}%
\end{pgfscope}%
\begin{pgfscope}%
\pgfpathrectangle{\pgfqpoint{0.765000in}{0.660000in}}{\pgfqpoint{4.620000in}{4.620000in}}%
\pgfusepath{clip}%
\pgfsetbuttcap%
\pgfsetroundjoin%
\definecolor{currentfill}{rgb}{1.000000,0.894118,0.788235}%
\pgfsetfillcolor{currentfill}%
\pgfsetlinewidth{0.000000pt}%
\definecolor{currentstroke}{rgb}{1.000000,0.894118,0.788235}%
\pgfsetstrokecolor{currentstroke}%
\pgfsetdash{}{0pt}%
\pgfpathmoveto{\pgfqpoint{3.551407in}{2.904953in}}%
\pgfpathlineto{\pgfqpoint{3.440511in}{2.840927in}}%
\pgfpathlineto{\pgfqpoint{3.551407in}{2.776901in}}%
\pgfpathlineto{\pgfqpoint{3.662304in}{2.840927in}}%
\pgfpathlineto{\pgfqpoint{3.551407in}{2.904953in}}%
\pgfpathclose%
\pgfusepath{fill}%
\end{pgfscope}%
\begin{pgfscope}%
\pgfpathrectangle{\pgfqpoint{0.765000in}{0.660000in}}{\pgfqpoint{4.620000in}{4.620000in}}%
\pgfusepath{clip}%
\pgfsetbuttcap%
\pgfsetroundjoin%
\definecolor{currentfill}{rgb}{1.000000,0.894118,0.788235}%
\pgfsetfillcolor{currentfill}%
\pgfsetlinewidth{0.000000pt}%
\definecolor{currentstroke}{rgb}{1.000000,0.894118,0.788235}%
\pgfsetstrokecolor{currentstroke}%
\pgfsetdash{}{0pt}%
\pgfpathmoveto{\pgfqpoint{3.551407in}{2.648848in}}%
\pgfpathlineto{\pgfqpoint{3.662304in}{2.712874in}}%
\pgfpathlineto{\pgfqpoint{3.662304in}{2.840927in}}%
\pgfpathlineto{\pgfqpoint{3.551407in}{2.776901in}}%
\pgfpathlineto{\pgfqpoint{3.551407in}{2.648848in}}%
\pgfpathclose%
\pgfusepath{fill}%
\end{pgfscope}%
\begin{pgfscope}%
\pgfpathrectangle{\pgfqpoint{0.765000in}{0.660000in}}{\pgfqpoint{4.620000in}{4.620000in}}%
\pgfusepath{clip}%
\pgfsetbuttcap%
\pgfsetroundjoin%
\definecolor{currentfill}{rgb}{1.000000,0.894118,0.788235}%
\pgfsetfillcolor{currentfill}%
\pgfsetlinewidth{0.000000pt}%
\definecolor{currentstroke}{rgb}{1.000000,0.894118,0.788235}%
\pgfsetstrokecolor{currentstroke}%
\pgfsetdash{}{0pt}%
\pgfpathmoveto{\pgfqpoint{3.440511in}{2.712874in}}%
\pgfpathlineto{\pgfqpoint{3.551407in}{2.648848in}}%
\pgfpathlineto{\pgfqpoint{3.551407in}{2.776901in}}%
\pgfpathlineto{\pgfqpoint{3.440511in}{2.840927in}}%
\pgfpathlineto{\pgfqpoint{3.440511in}{2.712874in}}%
\pgfpathclose%
\pgfusepath{fill}%
\end{pgfscope}%
\begin{pgfscope}%
\pgfpathrectangle{\pgfqpoint{0.765000in}{0.660000in}}{\pgfqpoint{4.620000in}{4.620000in}}%
\pgfusepath{clip}%
\pgfsetbuttcap%
\pgfsetroundjoin%
\definecolor{currentfill}{rgb}{1.000000,0.894118,0.788235}%
\pgfsetfillcolor{currentfill}%
\pgfsetlinewidth{0.000000pt}%
\definecolor{currentstroke}{rgb}{1.000000,0.894118,0.788235}%
\pgfsetstrokecolor{currentstroke}%
\pgfsetdash{}{0pt}%
\pgfpathmoveto{\pgfqpoint{3.563846in}{2.936255in}}%
\pgfpathlineto{\pgfqpoint{3.452949in}{2.872229in}}%
\pgfpathlineto{\pgfqpoint{3.563846in}{2.808203in}}%
\pgfpathlineto{\pgfqpoint{3.674742in}{2.872229in}}%
\pgfpathlineto{\pgfqpoint{3.563846in}{2.936255in}}%
\pgfpathclose%
\pgfusepath{fill}%
\end{pgfscope}%
\begin{pgfscope}%
\pgfpathrectangle{\pgfqpoint{0.765000in}{0.660000in}}{\pgfqpoint{4.620000in}{4.620000in}}%
\pgfusepath{clip}%
\pgfsetbuttcap%
\pgfsetroundjoin%
\definecolor{currentfill}{rgb}{1.000000,0.894118,0.788235}%
\pgfsetfillcolor{currentfill}%
\pgfsetlinewidth{0.000000pt}%
\definecolor{currentstroke}{rgb}{1.000000,0.894118,0.788235}%
\pgfsetstrokecolor{currentstroke}%
\pgfsetdash{}{0pt}%
\pgfpathmoveto{\pgfqpoint{3.563846in}{2.680150in}}%
\pgfpathlineto{\pgfqpoint{3.674742in}{2.744176in}}%
\pgfpathlineto{\pgfqpoint{3.674742in}{2.872229in}}%
\pgfpathlineto{\pgfqpoint{3.563846in}{2.808203in}}%
\pgfpathlineto{\pgfqpoint{3.563846in}{2.680150in}}%
\pgfpathclose%
\pgfusepath{fill}%
\end{pgfscope}%
\begin{pgfscope}%
\pgfpathrectangle{\pgfqpoint{0.765000in}{0.660000in}}{\pgfqpoint{4.620000in}{4.620000in}}%
\pgfusepath{clip}%
\pgfsetbuttcap%
\pgfsetroundjoin%
\definecolor{currentfill}{rgb}{1.000000,0.894118,0.788235}%
\pgfsetfillcolor{currentfill}%
\pgfsetlinewidth{0.000000pt}%
\definecolor{currentstroke}{rgb}{1.000000,0.894118,0.788235}%
\pgfsetstrokecolor{currentstroke}%
\pgfsetdash{}{0pt}%
\pgfpathmoveto{\pgfqpoint{3.452949in}{2.744176in}}%
\pgfpathlineto{\pgfqpoint{3.563846in}{2.680150in}}%
\pgfpathlineto{\pgfqpoint{3.563846in}{2.808203in}}%
\pgfpathlineto{\pgfqpoint{3.452949in}{2.872229in}}%
\pgfpathlineto{\pgfqpoint{3.452949in}{2.744176in}}%
\pgfpathclose%
\pgfusepath{fill}%
\end{pgfscope}%
\begin{pgfscope}%
\pgfpathrectangle{\pgfqpoint{0.765000in}{0.660000in}}{\pgfqpoint{4.620000in}{4.620000in}}%
\pgfusepath{clip}%
\pgfsetbuttcap%
\pgfsetroundjoin%
\definecolor{currentfill}{rgb}{1.000000,0.894118,0.788235}%
\pgfsetfillcolor{currentfill}%
\pgfsetlinewidth{0.000000pt}%
\definecolor{currentstroke}{rgb}{1.000000,0.894118,0.788235}%
\pgfsetstrokecolor{currentstroke}%
\pgfsetdash{}{0pt}%
\pgfpathmoveto{\pgfqpoint{3.551407in}{2.776901in}}%
\pgfpathlineto{\pgfqpoint{3.440511in}{2.712874in}}%
\pgfpathlineto{\pgfqpoint{3.452949in}{2.744176in}}%
\pgfpathlineto{\pgfqpoint{3.563846in}{2.808203in}}%
\pgfpathlineto{\pgfqpoint{3.551407in}{2.776901in}}%
\pgfpathclose%
\pgfusepath{fill}%
\end{pgfscope}%
\begin{pgfscope}%
\pgfpathrectangle{\pgfqpoint{0.765000in}{0.660000in}}{\pgfqpoint{4.620000in}{4.620000in}}%
\pgfusepath{clip}%
\pgfsetbuttcap%
\pgfsetroundjoin%
\definecolor{currentfill}{rgb}{1.000000,0.894118,0.788235}%
\pgfsetfillcolor{currentfill}%
\pgfsetlinewidth{0.000000pt}%
\definecolor{currentstroke}{rgb}{1.000000,0.894118,0.788235}%
\pgfsetstrokecolor{currentstroke}%
\pgfsetdash{}{0pt}%
\pgfpathmoveto{\pgfqpoint{3.662304in}{2.712874in}}%
\pgfpathlineto{\pgfqpoint{3.551407in}{2.776901in}}%
\pgfpathlineto{\pgfqpoint{3.563846in}{2.808203in}}%
\pgfpathlineto{\pgfqpoint{3.674742in}{2.744176in}}%
\pgfpathlineto{\pgfqpoint{3.662304in}{2.712874in}}%
\pgfpathclose%
\pgfusepath{fill}%
\end{pgfscope}%
\begin{pgfscope}%
\pgfpathrectangle{\pgfqpoint{0.765000in}{0.660000in}}{\pgfqpoint{4.620000in}{4.620000in}}%
\pgfusepath{clip}%
\pgfsetbuttcap%
\pgfsetroundjoin%
\definecolor{currentfill}{rgb}{1.000000,0.894118,0.788235}%
\pgfsetfillcolor{currentfill}%
\pgfsetlinewidth{0.000000pt}%
\definecolor{currentstroke}{rgb}{1.000000,0.894118,0.788235}%
\pgfsetstrokecolor{currentstroke}%
\pgfsetdash{}{0pt}%
\pgfpathmoveto{\pgfqpoint{3.551407in}{2.776901in}}%
\pgfpathlineto{\pgfqpoint{3.551407in}{2.904953in}}%
\pgfpathlineto{\pgfqpoint{3.563846in}{2.936255in}}%
\pgfpathlineto{\pgfqpoint{3.674742in}{2.744176in}}%
\pgfpathlineto{\pgfqpoint{3.551407in}{2.776901in}}%
\pgfpathclose%
\pgfusepath{fill}%
\end{pgfscope}%
\begin{pgfscope}%
\pgfpathrectangle{\pgfqpoint{0.765000in}{0.660000in}}{\pgfqpoint{4.620000in}{4.620000in}}%
\pgfusepath{clip}%
\pgfsetbuttcap%
\pgfsetroundjoin%
\definecolor{currentfill}{rgb}{1.000000,0.894118,0.788235}%
\pgfsetfillcolor{currentfill}%
\pgfsetlinewidth{0.000000pt}%
\definecolor{currentstroke}{rgb}{1.000000,0.894118,0.788235}%
\pgfsetstrokecolor{currentstroke}%
\pgfsetdash{}{0pt}%
\pgfpathmoveto{\pgfqpoint{3.662304in}{2.712874in}}%
\pgfpathlineto{\pgfqpoint{3.662304in}{2.840927in}}%
\pgfpathlineto{\pgfqpoint{3.674742in}{2.872229in}}%
\pgfpathlineto{\pgfqpoint{3.563846in}{2.808203in}}%
\pgfpathlineto{\pgfqpoint{3.662304in}{2.712874in}}%
\pgfpathclose%
\pgfusepath{fill}%
\end{pgfscope}%
\begin{pgfscope}%
\pgfpathrectangle{\pgfqpoint{0.765000in}{0.660000in}}{\pgfqpoint{4.620000in}{4.620000in}}%
\pgfusepath{clip}%
\pgfsetbuttcap%
\pgfsetroundjoin%
\definecolor{currentfill}{rgb}{1.000000,0.894118,0.788235}%
\pgfsetfillcolor{currentfill}%
\pgfsetlinewidth{0.000000pt}%
\definecolor{currentstroke}{rgb}{1.000000,0.894118,0.788235}%
\pgfsetstrokecolor{currentstroke}%
\pgfsetdash{}{0pt}%
\pgfpathmoveto{\pgfqpoint{3.440511in}{2.712874in}}%
\pgfpathlineto{\pgfqpoint{3.551407in}{2.648848in}}%
\pgfpathlineto{\pgfqpoint{3.563846in}{2.680150in}}%
\pgfpathlineto{\pgfqpoint{3.452949in}{2.744176in}}%
\pgfpathlineto{\pgfqpoint{3.440511in}{2.712874in}}%
\pgfpathclose%
\pgfusepath{fill}%
\end{pgfscope}%
\begin{pgfscope}%
\pgfpathrectangle{\pgfqpoint{0.765000in}{0.660000in}}{\pgfqpoint{4.620000in}{4.620000in}}%
\pgfusepath{clip}%
\pgfsetbuttcap%
\pgfsetroundjoin%
\definecolor{currentfill}{rgb}{1.000000,0.894118,0.788235}%
\pgfsetfillcolor{currentfill}%
\pgfsetlinewidth{0.000000pt}%
\definecolor{currentstroke}{rgb}{1.000000,0.894118,0.788235}%
\pgfsetstrokecolor{currentstroke}%
\pgfsetdash{}{0pt}%
\pgfpathmoveto{\pgfqpoint{3.551407in}{2.648848in}}%
\pgfpathlineto{\pgfqpoint{3.662304in}{2.712874in}}%
\pgfpathlineto{\pgfqpoint{3.674742in}{2.744176in}}%
\pgfpathlineto{\pgfqpoint{3.563846in}{2.680150in}}%
\pgfpathlineto{\pgfqpoint{3.551407in}{2.648848in}}%
\pgfpathclose%
\pgfusepath{fill}%
\end{pgfscope}%
\begin{pgfscope}%
\pgfpathrectangle{\pgfqpoint{0.765000in}{0.660000in}}{\pgfqpoint{4.620000in}{4.620000in}}%
\pgfusepath{clip}%
\pgfsetbuttcap%
\pgfsetroundjoin%
\definecolor{currentfill}{rgb}{1.000000,0.894118,0.788235}%
\pgfsetfillcolor{currentfill}%
\pgfsetlinewidth{0.000000pt}%
\definecolor{currentstroke}{rgb}{1.000000,0.894118,0.788235}%
\pgfsetstrokecolor{currentstroke}%
\pgfsetdash{}{0pt}%
\pgfpathmoveto{\pgfqpoint{3.551407in}{2.904953in}}%
\pgfpathlineto{\pgfqpoint{3.440511in}{2.840927in}}%
\pgfpathlineto{\pgfqpoint{3.452949in}{2.872229in}}%
\pgfpathlineto{\pgfqpoint{3.563846in}{2.936255in}}%
\pgfpathlineto{\pgfqpoint{3.551407in}{2.904953in}}%
\pgfpathclose%
\pgfusepath{fill}%
\end{pgfscope}%
\begin{pgfscope}%
\pgfpathrectangle{\pgfqpoint{0.765000in}{0.660000in}}{\pgfqpoint{4.620000in}{4.620000in}}%
\pgfusepath{clip}%
\pgfsetbuttcap%
\pgfsetroundjoin%
\definecolor{currentfill}{rgb}{1.000000,0.894118,0.788235}%
\pgfsetfillcolor{currentfill}%
\pgfsetlinewidth{0.000000pt}%
\definecolor{currentstroke}{rgb}{1.000000,0.894118,0.788235}%
\pgfsetstrokecolor{currentstroke}%
\pgfsetdash{}{0pt}%
\pgfpathmoveto{\pgfqpoint{3.662304in}{2.840927in}}%
\pgfpathlineto{\pgfqpoint{3.551407in}{2.904953in}}%
\pgfpathlineto{\pgfqpoint{3.563846in}{2.936255in}}%
\pgfpathlineto{\pgfqpoint{3.674742in}{2.872229in}}%
\pgfpathlineto{\pgfqpoint{3.662304in}{2.840927in}}%
\pgfpathclose%
\pgfusepath{fill}%
\end{pgfscope}%
\begin{pgfscope}%
\pgfpathrectangle{\pgfqpoint{0.765000in}{0.660000in}}{\pgfqpoint{4.620000in}{4.620000in}}%
\pgfusepath{clip}%
\pgfsetbuttcap%
\pgfsetroundjoin%
\definecolor{currentfill}{rgb}{1.000000,0.894118,0.788235}%
\pgfsetfillcolor{currentfill}%
\pgfsetlinewidth{0.000000pt}%
\definecolor{currentstroke}{rgb}{1.000000,0.894118,0.788235}%
\pgfsetstrokecolor{currentstroke}%
\pgfsetdash{}{0pt}%
\pgfpathmoveto{\pgfqpoint{3.440511in}{2.712874in}}%
\pgfpathlineto{\pgfqpoint{3.440511in}{2.840927in}}%
\pgfpathlineto{\pgfqpoint{3.452949in}{2.872229in}}%
\pgfpathlineto{\pgfqpoint{3.563846in}{2.680150in}}%
\pgfpathlineto{\pgfqpoint{3.440511in}{2.712874in}}%
\pgfpathclose%
\pgfusepath{fill}%
\end{pgfscope}%
\begin{pgfscope}%
\pgfpathrectangle{\pgfqpoint{0.765000in}{0.660000in}}{\pgfqpoint{4.620000in}{4.620000in}}%
\pgfusepath{clip}%
\pgfsetbuttcap%
\pgfsetroundjoin%
\definecolor{currentfill}{rgb}{1.000000,0.894118,0.788235}%
\pgfsetfillcolor{currentfill}%
\pgfsetlinewidth{0.000000pt}%
\definecolor{currentstroke}{rgb}{1.000000,0.894118,0.788235}%
\pgfsetstrokecolor{currentstroke}%
\pgfsetdash{}{0pt}%
\pgfpathmoveto{\pgfqpoint{3.551407in}{2.648848in}}%
\pgfpathlineto{\pgfqpoint{3.551407in}{2.776901in}}%
\pgfpathlineto{\pgfqpoint{3.563846in}{2.808203in}}%
\pgfpathlineto{\pgfqpoint{3.452949in}{2.744176in}}%
\pgfpathlineto{\pgfqpoint{3.551407in}{2.648848in}}%
\pgfpathclose%
\pgfusepath{fill}%
\end{pgfscope}%
\begin{pgfscope}%
\pgfpathrectangle{\pgfqpoint{0.765000in}{0.660000in}}{\pgfqpoint{4.620000in}{4.620000in}}%
\pgfusepath{clip}%
\pgfsetbuttcap%
\pgfsetroundjoin%
\definecolor{currentfill}{rgb}{1.000000,0.894118,0.788235}%
\pgfsetfillcolor{currentfill}%
\pgfsetlinewidth{0.000000pt}%
\definecolor{currentstroke}{rgb}{1.000000,0.894118,0.788235}%
\pgfsetstrokecolor{currentstroke}%
\pgfsetdash{}{0pt}%
\pgfpathmoveto{\pgfqpoint{3.551407in}{2.776901in}}%
\pgfpathlineto{\pgfqpoint{3.662304in}{2.840927in}}%
\pgfpathlineto{\pgfqpoint{3.674742in}{2.872229in}}%
\pgfpathlineto{\pgfqpoint{3.563846in}{2.808203in}}%
\pgfpathlineto{\pgfqpoint{3.551407in}{2.776901in}}%
\pgfpathclose%
\pgfusepath{fill}%
\end{pgfscope}%
\begin{pgfscope}%
\pgfpathrectangle{\pgfqpoint{0.765000in}{0.660000in}}{\pgfqpoint{4.620000in}{4.620000in}}%
\pgfusepath{clip}%
\pgfsetbuttcap%
\pgfsetroundjoin%
\definecolor{currentfill}{rgb}{1.000000,0.894118,0.788235}%
\pgfsetfillcolor{currentfill}%
\pgfsetlinewidth{0.000000pt}%
\definecolor{currentstroke}{rgb}{1.000000,0.894118,0.788235}%
\pgfsetstrokecolor{currentstroke}%
\pgfsetdash{}{0pt}%
\pgfpathmoveto{\pgfqpoint{3.440511in}{2.840927in}}%
\pgfpathlineto{\pgfqpoint{3.551407in}{2.776901in}}%
\pgfpathlineto{\pgfqpoint{3.563846in}{2.808203in}}%
\pgfpathlineto{\pgfqpoint{3.452949in}{2.872229in}}%
\pgfpathlineto{\pgfqpoint{3.440511in}{2.840927in}}%
\pgfpathclose%
\pgfusepath{fill}%
\end{pgfscope}%
\begin{pgfscope}%
\pgfpathrectangle{\pgfqpoint{0.765000in}{0.660000in}}{\pgfqpoint{4.620000in}{4.620000in}}%
\pgfusepath{clip}%
\pgfsetbuttcap%
\pgfsetroundjoin%
\definecolor{currentfill}{rgb}{1.000000,0.894118,0.788235}%
\pgfsetfillcolor{currentfill}%
\pgfsetlinewidth{0.000000pt}%
\definecolor{currentstroke}{rgb}{1.000000,0.894118,0.788235}%
\pgfsetstrokecolor{currentstroke}%
\pgfsetdash{}{0pt}%
\pgfpathmoveto{\pgfqpoint{3.085349in}{3.530998in}}%
\pgfpathlineto{\pgfqpoint{2.974452in}{3.466972in}}%
\pgfpathlineto{\pgfqpoint{3.085349in}{3.402945in}}%
\pgfpathlineto{\pgfqpoint{3.196246in}{3.466972in}}%
\pgfpathlineto{\pgfqpoint{3.085349in}{3.530998in}}%
\pgfpathclose%
\pgfusepath{fill}%
\end{pgfscope}%
\begin{pgfscope}%
\pgfpathrectangle{\pgfqpoint{0.765000in}{0.660000in}}{\pgfqpoint{4.620000in}{4.620000in}}%
\pgfusepath{clip}%
\pgfsetbuttcap%
\pgfsetroundjoin%
\definecolor{currentfill}{rgb}{1.000000,0.894118,0.788235}%
\pgfsetfillcolor{currentfill}%
\pgfsetlinewidth{0.000000pt}%
\definecolor{currentstroke}{rgb}{1.000000,0.894118,0.788235}%
\pgfsetstrokecolor{currentstroke}%
\pgfsetdash{}{0pt}%
\pgfpathmoveto{\pgfqpoint{3.085349in}{3.530998in}}%
\pgfpathlineto{\pgfqpoint{2.974452in}{3.466972in}}%
\pgfpathlineto{\pgfqpoint{2.974452in}{3.595024in}}%
\pgfpathlineto{\pgfqpoint{3.085349in}{3.659050in}}%
\pgfpathlineto{\pgfqpoint{3.085349in}{3.530998in}}%
\pgfpathclose%
\pgfusepath{fill}%
\end{pgfscope}%
\begin{pgfscope}%
\pgfpathrectangle{\pgfqpoint{0.765000in}{0.660000in}}{\pgfqpoint{4.620000in}{4.620000in}}%
\pgfusepath{clip}%
\pgfsetbuttcap%
\pgfsetroundjoin%
\definecolor{currentfill}{rgb}{1.000000,0.894118,0.788235}%
\pgfsetfillcolor{currentfill}%
\pgfsetlinewidth{0.000000pt}%
\definecolor{currentstroke}{rgb}{1.000000,0.894118,0.788235}%
\pgfsetstrokecolor{currentstroke}%
\pgfsetdash{}{0pt}%
\pgfpathmoveto{\pgfqpoint{3.085349in}{3.530998in}}%
\pgfpathlineto{\pgfqpoint{3.196246in}{3.466972in}}%
\pgfpathlineto{\pgfqpoint{3.196246in}{3.595024in}}%
\pgfpathlineto{\pgfqpoint{3.085349in}{3.659050in}}%
\pgfpathlineto{\pgfqpoint{3.085349in}{3.530998in}}%
\pgfpathclose%
\pgfusepath{fill}%
\end{pgfscope}%
\begin{pgfscope}%
\pgfpathrectangle{\pgfqpoint{0.765000in}{0.660000in}}{\pgfqpoint{4.620000in}{4.620000in}}%
\pgfusepath{clip}%
\pgfsetbuttcap%
\pgfsetroundjoin%
\definecolor{currentfill}{rgb}{1.000000,0.894118,0.788235}%
\pgfsetfillcolor{currentfill}%
\pgfsetlinewidth{0.000000pt}%
\definecolor{currentstroke}{rgb}{1.000000,0.894118,0.788235}%
\pgfsetstrokecolor{currentstroke}%
\pgfsetdash{}{0pt}%
\pgfpathmoveto{\pgfqpoint{3.046316in}{3.530257in}}%
\pgfpathlineto{\pgfqpoint{2.935419in}{3.466231in}}%
\pgfpathlineto{\pgfqpoint{3.046316in}{3.402204in}}%
\pgfpathlineto{\pgfqpoint{3.157212in}{3.466231in}}%
\pgfpathlineto{\pgfqpoint{3.046316in}{3.530257in}}%
\pgfpathclose%
\pgfusepath{fill}%
\end{pgfscope}%
\begin{pgfscope}%
\pgfpathrectangle{\pgfqpoint{0.765000in}{0.660000in}}{\pgfqpoint{4.620000in}{4.620000in}}%
\pgfusepath{clip}%
\pgfsetbuttcap%
\pgfsetroundjoin%
\definecolor{currentfill}{rgb}{1.000000,0.894118,0.788235}%
\pgfsetfillcolor{currentfill}%
\pgfsetlinewidth{0.000000pt}%
\definecolor{currentstroke}{rgb}{1.000000,0.894118,0.788235}%
\pgfsetstrokecolor{currentstroke}%
\pgfsetdash{}{0pt}%
\pgfpathmoveto{\pgfqpoint{3.046316in}{3.530257in}}%
\pgfpathlineto{\pgfqpoint{2.935419in}{3.466231in}}%
\pgfpathlineto{\pgfqpoint{2.935419in}{3.594283in}}%
\pgfpathlineto{\pgfqpoint{3.046316in}{3.658309in}}%
\pgfpathlineto{\pgfqpoint{3.046316in}{3.530257in}}%
\pgfpathclose%
\pgfusepath{fill}%
\end{pgfscope}%
\begin{pgfscope}%
\pgfpathrectangle{\pgfqpoint{0.765000in}{0.660000in}}{\pgfqpoint{4.620000in}{4.620000in}}%
\pgfusepath{clip}%
\pgfsetbuttcap%
\pgfsetroundjoin%
\definecolor{currentfill}{rgb}{1.000000,0.894118,0.788235}%
\pgfsetfillcolor{currentfill}%
\pgfsetlinewidth{0.000000pt}%
\definecolor{currentstroke}{rgb}{1.000000,0.894118,0.788235}%
\pgfsetstrokecolor{currentstroke}%
\pgfsetdash{}{0pt}%
\pgfpathmoveto{\pgfqpoint{3.046316in}{3.530257in}}%
\pgfpathlineto{\pgfqpoint{3.157212in}{3.466231in}}%
\pgfpathlineto{\pgfqpoint{3.157212in}{3.594283in}}%
\pgfpathlineto{\pgfqpoint{3.046316in}{3.658309in}}%
\pgfpathlineto{\pgfqpoint{3.046316in}{3.530257in}}%
\pgfpathclose%
\pgfusepath{fill}%
\end{pgfscope}%
\begin{pgfscope}%
\pgfpathrectangle{\pgfqpoint{0.765000in}{0.660000in}}{\pgfqpoint{4.620000in}{4.620000in}}%
\pgfusepath{clip}%
\pgfsetbuttcap%
\pgfsetroundjoin%
\definecolor{currentfill}{rgb}{1.000000,0.894118,0.788235}%
\pgfsetfillcolor{currentfill}%
\pgfsetlinewidth{0.000000pt}%
\definecolor{currentstroke}{rgb}{1.000000,0.894118,0.788235}%
\pgfsetstrokecolor{currentstroke}%
\pgfsetdash{}{0pt}%
\pgfpathmoveto{\pgfqpoint{3.085349in}{3.659050in}}%
\pgfpathlineto{\pgfqpoint{2.974452in}{3.595024in}}%
\pgfpathlineto{\pgfqpoint{3.085349in}{3.530998in}}%
\pgfpathlineto{\pgfqpoint{3.196246in}{3.595024in}}%
\pgfpathlineto{\pgfqpoint{3.085349in}{3.659050in}}%
\pgfpathclose%
\pgfusepath{fill}%
\end{pgfscope}%
\begin{pgfscope}%
\pgfpathrectangle{\pgfqpoint{0.765000in}{0.660000in}}{\pgfqpoint{4.620000in}{4.620000in}}%
\pgfusepath{clip}%
\pgfsetbuttcap%
\pgfsetroundjoin%
\definecolor{currentfill}{rgb}{1.000000,0.894118,0.788235}%
\pgfsetfillcolor{currentfill}%
\pgfsetlinewidth{0.000000pt}%
\definecolor{currentstroke}{rgb}{1.000000,0.894118,0.788235}%
\pgfsetstrokecolor{currentstroke}%
\pgfsetdash{}{0pt}%
\pgfpathmoveto{\pgfqpoint{3.085349in}{3.402945in}}%
\pgfpathlineto{\pgfqpoint{3.196246in}{3.466972in}}%
\pgfpathlineto{\pgfqpoint{3.196246in}{3.595024in}}%
\pgfpathlineto{\pgfqpoint{3.085349in}{3.530998in}}%
\pgfpathlineto{\pgfqpoint{3.085349in}{3.402945in}}%
\pgfpathclose%
\pgfusepath{fill}%
\end{pgfscope}%
\begin{pgfscope}%
\pgfpathrectangle{\pgfqpoint{0.765000in}{0.660000in}}{\pgfqpoint{4.620000in}{4.620000in}}%
\pgfusepath{clip}%
\pgfsetbuttcap%
\pgfsetroundjoin%
\definecolor{currentfill}{rgb}{1.000000,0.894118,0.788235}%
\pgfsetfillcolor{currentfill}%
\pgfsetlinewidth{0.000000pt}%
\definecolor{currentstroke}{rgb}{1.000000,0.894118,0.788235}%
\pgfsetstrokecolor{currentstroke}%
\pgfsetdash{}{0pt}%
\pgfpathmoveto{\pgfqpoint{2.974452in}{3.466972in}}%
\pgfpathlineto{\pgfqpoint{3.085349in}{3.402945in}}%
\pgfpathlineto{\pgfqpoint{3.085349in}{3.530998in}}%
\pgfpathlineto{\pgfqpoint{2.974452in}{3.595024in}}%
\pgfpathlineto{\pgfqpoint{2.974452in}{3.466972in}}%
\pgfpathclose%
\pgfusepath{fill}%
\end{pgfscope}%
\begin{pgfscope}%
\pgfpathrectangle{\pgfqpoint{0.765000in}{0.660000in}}{\pgfqpoint{4.620000in}{4.620000in}}%
\pgfusepath{clip}%
\pgfsetbuttcap%
\pgfsetroundjoin%
\definecolor{currentfill}{rgb}{1.000000,0.894118,0.788235}%
\pgfsetfillcolor{currentfill}%
\pgfsetlinewidth{0.000000pt}%
\definecolor{currentstroke}{rgb}{1.000000,0.894118,0.788235}%
\pgfsetstrokecolor{currentstroke}%
\pgfsetdash{}{0pt}%
\pgfpathmoveto{\pgfqpoint{3.046316in}{3.658309in}}%
\pgfpathlineto{\pgfqpoint{2.935419in}{3.594283in}}%
\pgfpathlineto{\pgfqpoint{3.046316in}{3.530257in}}%
\pgfpathlineto{\pgfqpoint{3.157212in}{3.594283in}}%
\pgfpathlineto{\pgfqpoint{3.046316in}{3.658309in}}%
\pgfpathclose%
\pgfusepath{fill}%
\end{pgfscope}%
\begin{pgfscope}%
\pgfpathrectangle{\pgfqpoint{0.765000in}{0.660000in}}{\pgfqpoint{4.620000in}{4.620000in}}%
\pgfusepath{clip}%
\pgfsetbuttcap%
\pgfsetroundjoin%
\definecolor{currentfill}{rgb}{1.000000,0.894118,0.788235}%
\pgfsetfillcolor{currentfill}%
\pgfsetlinewidth{0.000000pt}%
\definecolor{currentstroke}{rgb}{1.000000,0.894118,0.788235}%
\pgfsetstrokecolor{currentstroke}%
\pgfsetdash{}{0pt}%
\pgfpathmoveto{\pgfqpoint{3.046316in}{3.402204in}}%
\pgfpathlineto{\pgfqpoint{3.157212in}{3.466231in}}%
\pgfpathlineto{\pgfqpoint{3.157212in}{3.594283in}}%
\pgfpathlineto{\pgfqpoint{3.046316in}{3.530257in}}%
\pgfpathlineto{\pgfqpoint{3.046316in}{3.402204in}}%
\pgfpathclose%
\pgfusepath{fill}%
\end{pgfscope}%
\begin{pgfscope}%
\pgfpathrectangle{\pgfqpoint{0.765000in}{0.660000in}}{\pgfqpoint{4.620000in}{4.620000in}}%
\pgfusepath{clip}%
\pgfsetbuttcap%
\pgfsetroundjoin%
\definecolor{currentfill}{rgb}{1.000000,0.894118,0.788235}%
\pgfsetfillcolor{currentfill}%
\pgfsetlinewidth{0.000000pt}%
\definecolor{currentstroke}{rgb}{1.000000,0.894118,0.788235}%
\pgfsetstrokecolor{currentstroke}%
\pgfsetdash{}{0pt}%
\pgfpathmoveto{\pgfqpoint{2.935419in}{3.466231in}}%
\pgfpathlineto{\pgfqpoint{3.046316in}{3.402204in}}%
\pgfpathlineto{\pgfqpoint{3.046316in}{3.530257in}}%
\pgfpathlineto{\pgfqpoint{2.935419in}{3.594283in}}%
\pgfpathlineto{\pgfqpoint{2.935419in}{3.466231in}}%
\pgfpathclose%
\pgfusepath{fill}%
\end{pgfscope}%
\begin{pgfscope}%
\pgfpathrectangle{\pgfqpoint{0.765000in}{0.660000in}}{\pgfqpoint{4.620000in}{4.620000in}}%
\pgfusepath{clip}%
\pgfsetbuttcap%
\pgfsetroundjoin%
\definecolor{currentfill}{rgb}{1.000000,0.894118,0.788235}%
\pgfsetfillcolor{currentfill}%
\pgfsetlinewidth{0.000000pt}%
\definecolor{currentstroke}{rgb}{1.000000,0.894118,0.788235}%
\pgfsetstrokecolor{currentstroke}%
\pgfsetdash{}{0pt}%
\pgfpathmoveto{\pgfqpoint{3.085349in}{3.530998in}}%
\pgfpathlineto{\pgfqpoint{2.974452in}{3.466972in}}%
\pgfpathlineto{\pgfqpoint{2.935419in}{3.466231in}}%
\pgfpathlineto{\pgfqpoint{3.046316in}{3.530257in}}%
\pgfpathlineto{\pgfqpoint{3.085349in}{3.530998in}}%
\pgfpathclose%
\pgfusepath{fill}%
\end{pgfscope}%
\begin{pgfscope}%
\pgfpathrectangle{\pgfqpoint{0.765000in}{0.660000in}}{\pgfqpoint{4.620000in}{4.620000in}}%
\pgfusepath{clip}%
\pgfsetbuttcap%
\pgfsetroundjoin%
\definecolor{currentfill}{rgb}{1.000000,0.894118,0.788235}%
\pgfsetfillcolor{currentfill}%
\pgfsetlinewidth{0.000000pt}%
\definecolor{currentstroke}{rgb}{1.000000,0.894118,0.788235}%
\pgfsetstrokecolor{currentstroke}%
\pgfsetdash{}{0pt}%
\pgfpathmoveto{\pgfqpoint{3.196246in}{3.466972in}}%
\pgfpathlineto{\pgfqpoint{3.085349in}{3.530998in}}%
\pgfpathlineto{\pgfqpoint{3.046316in}{3.530257in}}%
\pgfpathlineto{\pgfqpoint{3.157212in}{3.466231in}}%
\pgfpathlineto{\pgfqpoint{3.196246in}{3.466972in}}%
\pgfpathclose%
\pgfusepath{fill}%
\end{pgfscope}%
\begin{pgfscope}%
\pgfpathrectangle{\pgfqpoint{0.765000in}{0.660000in}}{\pgfqpoint{4.620000in}{4.620000in}}%
\pgfusepath{clip}%
\pgfsetbuttcap%
\pgfsetroundjoin%
\definecolor{currentfill}{rgb}{1.000000,0.894118,0.788235}%
\pgfsetfillcolor{currentfill}%
\pgfsetlinewidth{0.000000pt}%
\definecolor{currentstroke}{rgb}{1.000000,0.894118,0.788235}%
\pgfsetstrokecolor{currentstroke}%
\pgfsetdash{}{0pt}%
\pgfpathmoveto{\pgfqpoint{3.085349in}{3.530998in}}%
\pgfpathlineto{\pgfqpoint{3.085349in}{3.659050in}}%
\pgfpathlineto{\pgfqpoint{3.046316in}{3.658309in}}%
\pgfpathlineto{\pgfqpoint{3.157212in}{3.466231in}}%
\pgfpathlineto{\pgfqpoint{3.085349in}{3.530998in}}%
\pgfpathclose%
\pgfusepath{fill}%
\end{pgfscope}%
\begin{pgfscope}%
\pgfpathrectangle{\pgfqpoint{0.765000in}{0.660000in}}{\pgfqpoint{4.620000in}{4.620000in}}%
\pgfusepath{clip}%
\pgfsetbuttcap%
\pgfsetroundjoin%
\definecolor{currentfill}{rgb}{1.000000,0.894118,0.788235}%
\pgfsetfillcolor{currentfill}%
\pgfsetlinewidth{0.000000pt}%
\definecolor{currentstroke}{rgb}{1.000000,0.894118,0.788235}%
\pgfsetstrokecolor{currentstroke}%
\pgfsetdash{}{0pt}%
\pgfpathmoveto{\pgfqpoint{3.196246in}{3.466972in}}%
\pgfpathlineto{\pgfqpoint{3.196246in}{3.595024in}}%
\pgfpathlineto{\pgfqpoint{3.157212in}{3.594283in}}%
\pgfpathlineto{\pgfqpoint{3.046316in}{3.530257in}}%
\pgfpathlineto{\pgfqpoint{3.196246in}{3.466972in}}%
\pgfpathclose%
\pgfusepath{fill}%
\end{pgfscope}%
\begin{pgfscope}%
\pgfpathrectangle{\pgfqpoint{0.765000in}{0.660000in}}{\pgfqpoint{4.620000in}{4.620000in}}%
\pgfusepath{clip}%
\pgfsetbuttcap%
\pgfsetroundjoin%
\definecolor{currentfill}{rgb}{1.000000,0.894118,0.788235}%
\pgfsetfillcolor{currentfill}%
\pgfsetlinewidth{0.000000pt}%
\definecolor{currentstroke}{rgb}{1.000000,0.894118,0.788235}%
\pgfsetstrokecolor{currentstroke}%
\pgfsetdash{}{0pt}%
\pgfpathmoveto{\pgfqpoint{2.974452in}{3.466972in}}%
\pgfpathlineto{\pgfqpoint{3.085349in}{3.402945in}}%
\pgfpathlineto{\pgfqpoint{3.046316in}{3.402204in}}%
\pgfpathlineto{\pgfqpoint{2.935419in}{3.466231in}}%
\pgfpathlineto{\pgfqpoint{2.974452in}{3.466972in}}%
\pgfpathclose%
\pgfusepath{fill}%
\end{pgfscope}%
\begin{pgfscope}%
\pgfpathrectangle{\pgfqpoint{0.765000in}{0.660000in}}{\pgfqpoint{4.620000in}{4.620000in}}%
\pgfusepath{clip}%
\pgfsetbuttcap%
\pgfsetroundjoin%
\definecolor{currentfill}{rgb}{1.000000,0.894118,0.788235}%
\pgfsetfillcolor{currentfill}%
\pgfsetlinewidth{0.000000pt}%
\definecolor{currentstroke}{rgb}{1.000000,0.894118,0.788235}%
\pgfsetstrokecolor{currentstroke}%
\pgfsetdash{}{0pt}%
\pgfpathmoveto{\pgfqpoint{3.085349in}{3.402945in}}%
\pgfpathlineto{\pgfqpoint{3.196246in}{3.466972in}}%
\pgfpathlineto{\pgfqpoint{3.157212in}{3.466231in}}%
\pgfpathlineto{\pgfqpoint{3.046316in}{3.402204in}}%
\pgfpathlineto{\pgfqpoint{3.085349in}{3.402945in}}%
\pgfpathclose%
\pgfusepath{fill}%
\end{pgfscope}%
\begin{pgfscope}%
\pgfpathrectangle{\pgfqpoint{0.765000in}{0.660000in}}{\pgfqpoint{4.620000in}{4.620000in}}%
\pgfusepath{clip}%
\pgfsetbuttcap%
\pgfsetroundjoin%
\definecolor{currentfill}{rgb}{1.000000,0.894118,0.788235}%
\pgfsetfillcolor{currentfill}%
\pgfsetlinewidth{0.000000pt}%
\definecolor{currentstroke}{rgb}{1.000000,0.894118,0.788235}%
\pgfsetstrokecolor{currentstroke}%
\pgfsetdash{}{0pt}%
\pgfpathmoveto{\pgfqpoint{3.085349in}{3.659050in}}%
\pgfpathlineto{\pgfqpoint{2.974452in}{3.595024in}}%
\pgfpathlineto{\pgfqpoint{2.935419in}{3.594283in}}%
\pgfpathlineto{\pgfqpoint{3.046316in}{3.658309in}}%
\pgfpathlineto{\pgfqpoint{3.085349in}{3.659050in}}%
\pgfpathclose%
\pgfusepath{fill}%
\end{pgfscope}%
\begin{pgfscope}%
\pgfpathrectangle{\pgfqpoint{0.765000in}{0.660000in}}{\pgfqpoint{4.620000in}{4.620000in}}%
\pgfusepath{clip}%
\pgfsetbuttcap%
\pgfsetroundjoin%
\definecolor{currentfill}{rgb}{1.000000,0.894118,0.788235}%
\pgfsetfillcolor{currentfill}%
\pgfsetlinewidth{0.000000pt}%
\definecolor{currentstroke}{rgb}{1.000000,0.894118,0.788235}%
\pgfsetstrokecolor{currentstroke}%
\pgfsetdash{}{0pt}%
\pgfpathmoveto{\pgfqpoint{3.196246in}{3.595024in}}%
\pgfpathlineto{\pgfqpoint{3.085349in}{3.659050in}}%
\pgfpathlineto{\pgfqpoint{3.046316in}{3.658309in}}%
\pgfpathlineto{\pgfqpoint{3.157212in}{3.594283in}}%
\pgfpathlineto{\pgfqpoint{3.196246in}{3.595024in}}%
\pgfpathclose%
\pgfusepath{fill}%
\end{pgfscope}%
\begin{pgfscope}%
\pgfpathrectangle{\pgfqpoint{0.765000in}{0.660000in}}{\pgfqpoint{4.620000in}{4.620000in}}%
\pgfusepath{clip}%
\pgfsetbuttcap%
\pgfsetroundjoin%
\definecolor{currentfill}{rgb}{1.000000,0.894118,0.788235}%
\pgfsetfillcolor{currentfill}%
\pgfsetlinewidth{0.000000pt}%
\definecolor{currentstroke}{rgb}{1.000000,0.894118,0.788235}%
\pgfsetstrokecolor{currentstroke}%
\pgfsetdash{}{0pt}%
\pgfpathmoveto{\pgfqpoint{2.974452in}{3.466972in}}%
\pgfpathlineto{\pgfqpoint{2.974452in}{3.595024in}}%
\pgfpathlineto{\pgfqpoint{2.935419in}{3.594283in}}%
\pgfpathlineto{\pgfqpoint{3.046316in}{3.402204in}}%
\pgfpathlineto{\pgfqpoint{2.974452in}{3.466972in}}%
\pgfpathclose%
\pgfusepath{fill}%
\end{pgfscope}%
\begin{pgfscope}%
\pgfpathrectangle{\pgfqpoint{0.765000in}{0.660000in}}{\pgfqpoint{4.620000in}{4.620000in}}%
\pgfusepath{clip}%
\pgfsetbuttcap%
\pgfsetroundjoin%
\definecolor{currentfill}{rgb}{1.000000,0.894118,0.788235}%
\pgfsetfillcolor{currentfill}%
\pgfsetlinewidth{0.000000pt}%
\definecolor{currentstroke}{rgb}{1.000000,0.894118,0.788235}%
\pgfsetstrokecolor{currentstroke}%
\pgfsetdash{}{0pt}%
\pgfpathmoveto{\pgfqpoint{3.085349in}{3.402945in}}%
\pgfpathlineto{\pgfqpoint{3.085349in}{3.530998in}}%
\pgfpathlineto{\pgfqpoint{3.046316in}{3.530257in}}%
\pgfpathlineto{\pgfqpoint{2.935419in}{3.466231in}}%
\pgfpathlineto{\pgfqpoint{3.085349in}{3.402945in}}%
\pgfpathclose%
\pgfusepath{fill}%
\end{pgfscope}%
\begin{pgfscope}%
\pgfpathrectangle{\pgfqpoint{0.765000in}{0.660000in}}{\pgfqpoint{4.620000in}{4.620000in}}%
\pgfusepath{clip}%
\pgfsetbuttcap%
\pgfsetroundjoin%
\definecolor{currentfill}{rgb}{1.000000,0.894118,0.788235}%
\pgfsetfillcolor{currentfill}%
\pgfsetlinewidth{0.000000pt}%
\definecolor{currentstroke}{rgb}{1.000000,0.894118,0.788235}%
\pgfsetstrokecolor{currentstroke}%
\pgfsetdash{}{0pt}%
\pgfpathmoveto{\pgfqpoint{3.085349in}{3.530998in}}%
\pgfpathlineto{\pgfqpoint{3.196246in}{3.595024in}}%
\pgfpathlineto{\pgfqpoint{3.157212in}{3.594283in}}%
\pgfpathlineto{\pgfqpoint{3.046316in}{3.530257in}}%
\pgfpathlineto{\pgfqpoint{3.085349in}{3.530998in}}%
\pgfpathclose%
\pgfusepath{fill}%
\end{pgfscope}%
\begin{pgfscope}%
\pgfpathrectangle{\pgfqpoint{0.765000in}{0.660000in}}{\pgfqpoint{4.620000in}{4.620000in}}%
\pgfusepath{clip}%
\pgfsetbuttcap%
\pgfsetroundjoin%
\definecolor{currentfill}{rgb}{1.000000,0.894118,0.788235}%
\pgfsetfillcolor{currentfill}%
\pgfsetlinewidth{0.000000pt}%
\definecolor{currentstroke}{rgb}{1.000000,0.894118,0.788235}%
\pgfsetstrokecolor{currentstroke}%
\pgfsetdash{}{0pt}%
\pgfpathmoveto{\pgfqpoint{2.974452in}{3.595024in}}%
\pgfpathlineto{\pgfqpoint{3.085349in}{3.530998in}}%
\pgfpathlineto{\pgfqpoint{3.046316in}{3.530257in}}%
\pgfpathlineto{\pgfqpoint{2.935419in}{3.594283in}}%
\pgfpathlineto{\pgfqpoint{2.974452in}{3.595024in}}%
\pgfpathclose%
\pgfusepath{fill}%
\end{pgfscope}%
\begin{pgfscope}%
\pgfpathrectangle{\pgfqpoint{0.765000in}{0.660000in}}{\pgfqpoint{4.620000in}{4.620000in}}%
\pgfusepath{clip}%
\pgfsetbuttcap%
\pgfsetroundjoin%
\definecolor{currentfill}{rgb}{1.000000,0.894118,0.788235}%
\pgfsetfillcolor{currentfill}%
\pgfsetlinewidth{0.000000pt}%
\definecolor{currentstroke}{rgb}{1.000000,0.894118,0.788235}%
\pgfsetstrokecolor{currentstroke}%
\pgfsetdash{}{0pt}%
\pgfpathmoveto{\pgfqpoint{3.085349in}{3.530998in}}%
\pgfpathlineto{\pgfqpoint{2.974452in}{3.466972in}}%
\pgfpathlineto{\pgfqpoint{3.085349in}{3.402945in}}%
\pgfpathlineto{\pgfqpoint{3.196246in}{3.466972in}}%
\pgfpathlineto{\pgfqpoint{3.085349in}{3.530998in}}%
\pgfpathclose%
\pgfusepath{fill}%
\end{pgfscope}%
\begin{pgfscope}%
\pgfpathrectangle{\pgfqpoint{0.765000in}{0.660000in}}{\pgfqpoint{4.620000in}{4.620000in}}%
\pgfusepath{clip}%
\pgfsetbuttcap%
\pgfsetroundjoin%
\definecolor{currentfill}{rgb}{1.000000,0.894118,0.788235}%
\pgfsetfillcolor{currentfill}%
\pgfsetlinewidth{0.000000pt}%
\definecolor{currentstroke}{rgb}{1.000000,0.894118,0.788235}%
\pgfsetstrokecolor{currentstroke}%
\pgfsetdash{}{0pt}%
\pgfpathmoveto{\pgfqpoint{3.085349in}{3.530998in}}%
\pgfpathlineto{\pgfqpoint{2.974452in}{3.466972in}}%
\pgfpathlineto{\pgfqpoint{2.974452in}{3.595024in}}%
\pgfpathlineto{\pgfqpoint{3.085349in}{3.659050in}}%
\pgfpathlineto{\pgfqpoint{3.085349in}{3.530998in}}%
\pgfpathclose%
\pgfusepath{fill}%
\end{pgfscope}%
\begin{pgfscope}%
\pgfpathrectangle{\pgfqpoint{0.765000in}{0.660000in}}{\pgfqpoint{4.620000in}{4.620000in}}%
\pgfusepath{clip}%
\pgfsetbuttcap%
\pgfsetroundjoin%
\definecolor{currentfill}{rgb}{1.000000,0.894118,0.788235}%
\pgfsetfillcolor{currentfill}%
\pgfsetlinewidth{0.000000pt}%
\definecolor{currentstroke}{rgb}{1.000000,0.894118,0.788235}%
\pgfsetstrokecolor{currentstroke}%
\pgfsetdash{}{0pt}%
\pgfpathmoveto{\pgfqpoint{3.085349in}{3.530998in}}%
\pgfpathlineto{\pgfqpoint{3.196246in}{3.466972in}}%
\pgfpathlineto{\pgfqpoint{3.196246in}{3.595024in}}%
\pgfpathlineto{\pgfqpoint{3.085349in}{3.659050in}}%
\pgfpathlineto{\pgfqpoint{3.085349in}{3.530998in}}%
\pgfpathclose%
\pgfusepath{fill}%
\end{pgfscope}%
\begin{pgfscope}%
\pgfpathrectangle{\pgfqpoint{0.765000in}{0.660000in}}{\pgfqpoint{4.620000in}{4.620000in}}%
\pgfusepath{clip}%
\pgfsetbuttcap%
\pgfsetroundjoin%
\definecolor{currentfill}{rgb}{1.000000,0.894118,0.788235}%
\pgfsetfillcolor{currentfill}%
\pgfsetlinewidth{0.000000pt}%
\definecolor{currentstroke}{rgb}{1.000000,0.894118,0.788235}%
\pgfsetstrokecolor{currentstroke}%
\pgfsetdash{}{0pt}%
\pgfpathmoveto{\pgfqpoint{3.107502in}{3.529134in}}%
\pgfpathlineto{\pgfqpoint{2.996605in}{3.465108in}}%
\pgfpathlineto{\pgfqpoint{3.107502in}{3.401081in}}%
\pgfpathlineto{\pgfqpoint{3.218398in}{3.465108in}}%
\pgfpathlineto{\pgfqpoint{3.107502in}{3.529134in}}%
\pgfpathclose%
\pgfusepath{fill}%
\end{pgfscope}%
\begin{pgfscope}%
\pgfpathrectangle{\pgfqpoint{0.765000in}{0.660000in}}{\pgfqpoint{4.620000in}{4.620000in}}%
\pgfusepath{clip}%
\pgfsetbuttcap%
\pgfsetroundjoin%
\definecolor{currentfill}{rgb}{1.000000,0.894118,0.788235}%
\pgfsetfillcolor{currentfill}%
\pgfsetlinewidth{0.000000pt}%
\definecolor{currentstroke}{rgb}{1.000000,0.894118,0.788235}%
\pgfsetstrokecolor{currentstroke}%
\pgfsetdash{}{0pt}%
\pgfpathmoveto{\pgfqpoint{3.107502in}{3.529134in}}%
\pgfpathlineto{\pgfqpoint{2.996605in}{3.465108in}}%
\pgfpathlineto{\pgfqpoint{2.996605in}{3.593160in}}%
\pgfpathlineto{\pgfqpoint{3.107502in}{3.657186in}}%
\pgfpathlineto{\pgfqpoint{3.107502in}{3.529134in}}%
\pgfpathclose%
\pgfusepath{fill}%
\end{pgfscope}%
\begin{pgfscope}%
\pgfpathrectangle{\pgfqpoint{0.765000in}{0.660000in}}{\pgfqpoint{4.620000in}{4.620000in}}%
\pgfusepath{clip}%
\pgfsetbuttcap%
\pgfsetroundjoin%
\definecolor{currentfill}{rgb}{1.000000,0.894118,0.788235}%
\pgfsetfillcolor{currentfill}%
\pgfsetlinewidth{0.000000pt}%
\definecolor{currentstroke}{rgb}{1.000000,0.894118,0.788235}%
\pgfsetstrokecolor{currentstroke}%
\pgfsetdash{}{0pt}%
\pgfpathmoveto{\pgfqpoint{3.107502in}{3.529134in}}%
\pgfpathlineto{\pgfqpoint{3.218398in}{3.465108in}}%
\pgfpathlineto{\pgfqpoint{3.218398in}{3.593160in}}%
\pgfpathlineto{\pgfqpoint{3.107502in}{3.657186in}}%
\pgfpathlineto{\pgfqpoint{3.107502in}{3.529134in}}%
\pgfpathclose%
\pgfusepath{fill}%
\end{pgfscope}%
\begin{pgfscope}%
\pgfpathrectangle{\pgfqpoint{0.765000in}{0.660000in}}{\pgfqpoint{4.620000in}{4.620000in}}%
\pgfusepath{clip}%
\pgfsetbuttcap%
\pgfsetroundjoin%
\definecolor{currentfill}{rgb}{1.000000,0.894118,0.788235}%
\pgfsetfillcolor{currentfill}%
\pgfsetlinewidth{0.000000pt}%
\definecolor{currentstroke}{rgb}{1.000000,0.894118,0.788235}%
\pgfsetstrokecolor{currentstroke}%
\pgfsetdash{}{0pt}%
\pgfpathmoveto{\pgfqpoint{3.085349in}{3.659050in}}%
\pgfpathlineto{\pgfqpoint{2.974452in}{3.595024in}}%
\pgfpathlineto{\pgfqpoint{3.085349in}{3.530998in}}%
\pgfpathlineto{\pgfqpoint{3.196246in}{3.595024in}}%
\pgfpathlineto{\pgfqpoint{3.085349in}{3.659050in}}%
\pgfpathclose%
\pgfusepath{fill}%
\end{pgfscope}%
\begin{pgfscope}%
\pgfpathrectangle{\pgfqpoint{0.765000in}{0.660000in}}{\pgfqpoint{4.620000in}{4.620000in}}%
\pgfusepath{clip}%
\pgfsetbuttcap%
\pgfsetroundjoin%
\definecolor{currentfill}{rgb}{1.000000,0.894118,0.788235}%
\pgfsetfillcolor{currentfill}%
\pgfsetlinewidth{0.000000pt}%
\definecolor{currentstroke}{rgb}{1.000000,0.894118,0.788235}%
\pgfsetstrokecolor{currentstroke}%
\pgfsetdash{}{0pt}%
\pgfpathmoveto{\pgfqpoint{3.085349in}{3.402945in}}%
\pgfpathlineto{\pgfqpoint{3.196246in}{3.466972in}}%
\pgfpathlineto{\pgfqpoint{3.196246in}{3.595024in}}%
\pgfpathlineto{\pgfqpoint{3.085349in}{3.530998in}}%
\pgfpathlineto{\pgfqpoint{3.085349in}{3.402945in}}%
\pgfpathclose%
\pgfusepath{fill}%
\end{pgfscope}%
\begin{pgfscope}%
\pgfpathrectangle{\pgfqpoint{0.765000in}{0.660000in}}{\pgfqpoint{4.620000in}{4.620000in}}%
\pgfusepath{clip}%
\pgfsetbuttcap%
\pgfsetroundjoin%
\definecolor{currentfill}{rgb}{1.000000,0.894118,0.788235}%
\pgfsetfillcolor{currentfill}%
\pgfsetlinewidth{0.000000pt}%
\definecolor{currentstroke}{rgb}{1.000000,0.894118,0.788235}%
\pgfsetstrokecolor{currentstroke}%
\pgfsetdash{}{0pt}%
\pgfpathmoveto{\pgfqpoint{2.974452in}{3.466972in}}%
\pgfpathlineto{\pgfqpoint{3.085349in}{3.402945in}}%
\pgfpathlineto{\pgfqpoint{3.085349in}{3.530998in}}%
\pgfpathlineto{\pgfqpoint{2.974452in}{3.595024in}}%
\pgfpathlineto{\pgfqpoint{2.974452in}{3.466972in}}%
\pgfpathclose%
\pgfusepath{fill}%
\end{pgfscope}%
\begin{pgfscope}%
\pgfpathrectangle{\pgfqpoint{0.765000in}{0.660000in}}{\pgfqpoint{4.620000in}{4.620000in}}%
\pgfusepath{clip}%
\pgfsetbuttcap%
\pgfsetroundjoin%
\definecolor{currentfill}{rgb}{1.000000,0.894118,0.788235}%
\pgfsetfillcolor{currentfill}%
\pgfsetlinewidth{0.000000pt}%
\definecolor{currentstroke}{rgb}{1.000000,0.894118,0.788235}%
\pgfsetstrokecolor{currentstroke}%
\pgfsetdash{}{0pt}%
\pgfpathmoveto{\pgfqpoint{3.107502in}{3.657186in}}%
\pgfpathlineto{\pgfqpoint{2.996605in}{3.593160in}}%
\pgfpathlineto{\pgfqpoint{3.107502in}{3.529134in}}%
\pgfpathlineto{\pgfqpoint{3.218398in}{3.593160in}}%
\pgfpathlineto{\pgfqpoint{3.107502in}{3.657186in}}%
\pgfpathclose%
\pgfusepath{fill}%
\end{pgfscope}%
\begin{pgfscope}%
\pgfpathrectangle{\pgfqpoint{0.765000in}{0.660000in}}{\pgfqpoint{4.620000in}{4.620000in}}%
\pgfusepath{clip}%
\pgfsetbuttcap%
\pgfsetroundjoin%
\definecolor{currentfill}{rgb}{1.000000,0.894118,0.788235}%
\pgfsetfillcolor{currentfill}%
\pgfsetlinewidth{0.000000pt}%
\definecolor{currentstroke}{rgb}{1.000000,0.894118,0.788235}%
\pgfsetstrokecolor{currentstroke}%
\pgfsetdash{}{0pt}%
\pgfpathmoveto{\pgfqpoint{3.107502in}{3.401081in}}%
\pgfpathlineto{\pgfqpoint{3.218398in}{3.465108in}}%
\pgfpathlineto{\pgfqpoint{3.218398in}{3.593160in}}%
\pgfpathlineto{\pgfqpoint{3.107502in}{3.529134in}}%
\pgfpathlineto{\pgfqpoint{3.107502in}{3.401081in}}%
\pgfpathclose%
\pgfusepath{fill}%
\end{pgfscope}%
\begin{pgfscope}%
\pgfpathrectangle{\pgfqpoint{0.765000in}{0.660000in}}{\pgfqpoint{4.620000in}{4.620000in}}%
\pgfusepath{clip}%
\pgfsetbuttcap%
\pgfsetroundjoin%
\definecolor{currentfill}{rgb}{1.000000,0.894118,0.788235}%
\pgfsetfillcolor{currentfill}%
\pgfsetlinewidth{0.000000pt}%
\definecolor{currentstroke}{rgb}{1.000000,0.894118,0.788235}%
\pgfsetstrokecolor{currentstroke}%
\pgfsetdash{}{0pt}%
\pgfpathmoveto{\pgfqpoint{2.996605in}{3.465108in}}%
\pgfpathlineto{\pgfqpoint{3.107502in}{3.401081in}}%
\pgfpathlineto{\pgfqpoint{3.107502in}{3.529134in}}%
\pgfpathlineto{\pgfqpoint{2.996605in}{3.593160in}}%
\pgfpathlineto{\pgfqpoint{2.996605in}{3.465108in}}%
\pgfpathclose%
\pgfusepath{fill}%
\end{pgfscope}%
\begin{pgfscope}%
\pgfpathrectangle{\pgfqpoint{0.765000in}{0.660000in}}{\pgfqpoint{4.620000in}{4.620000in}}%
\pgfusepath{clip}%
\pgfsetbuttcap%
\pgfsetroundjoin%
\definecolor{currentfill}{rgb}{1.000000,0.894118,0.788235}%
\pgfsetfillcolor{currentfill}%
\pgfsetlinewidth{0.000000pt}%
\definecolor{currentstroke}{rgb}{1.000000,0.894118,0.788235}%
\pgfsetstrokecolor{currentstroke}%
\pgfsetdash{}{0pt}%
\pgfpathmoveto{\pgfqpoint{3.085349in}{3.530998in}}%
\pgfpathlineto{\pgfqpoint{2.974452in}{3.466972in}}%
\pgfpathlineto{\pgfqpoint{2.996605in}{3.465108in}}%
\pgfpathlineto{\pgfqpoint{3.107502in}{3.529134in}}%
\pgfpathlineto{\pgfqpoint{3.085349in}{3.530998in}}%
\pgfpathclose%
\pgfusepath{fill}%
\end{pgfscope}%
\begin{pgfscope}%
\pgfpathrectangle{\pgfqpoint{0.765000in}{0.660000in}}{\pgfqpoint{4.620000in}{4.620000in}}%
\pgfusepath{clip}%
\pgfsetbuttcap%
\pgfsetroundjoin%
\definecolor{currentfill}{rgb}{1.000000,0.894118,0.788235}%
\pgfsetfillcolor{currentfill}%
\pgfsetlinewidth{0.000000pt}%
\definecolor{currentstroke}{rgb}{1.000000,0.894118,0.788235}%
\pgfsetstrokecolor{currentstroke}%
\pgfsetdash{}{0pt}%
\pgfpathmoveto{\pgfqpoint{3.196246in}{3.466972in}}%
\pgfpathlineto{\pgfqpoint{3.085349in}{3.530998in}}%
\pgfpathlineto{\pgfqpoint{3.107502in}{3.529134in}}%
\pgfpathlineto{\pgfqpoint{3.218398in}{3.465108in}}%
\pgfpathlineto{\pgfqpoint{3.196246in}{3.466972in}}%
\pgfpathclose%
\pgfusepath{fill}%
\end{pgfscope}%
\begin{pgfscope}%
\pgfpathrectangle{\pgfqpoint{0.765000in}{0.660000in}}{\pgfqpoint{4.620000in}{4.620000in}}%
\pgfusepath{clip}%
\pgfsetbuttcap%
\pgfsetroundjoin%
\definecolor{currentfill}{rgb}{1.000000,0.894118,0.788235}%
\pgfsetfillcolor{currentfill}%
\pgfsetlinewidth{0.000000pt}%
\definecolor{currentstroke}{rgb}{1.000000,0.894118,0.788235}%
\pgfsetstrokecolor{currentstroke}%
\pgfsetdash{}{0pt}%
\pgfpathmoveto{\pgfqpoint{3.085349in}{3.530998in}}%
\pgfpathlineto{\pgfqpoint{3.085349in}{3.659050in}}%
\pgfpathlineto{\pgfqpoint{3.107502in}{3.657186in}}%
\pgfpathlineto{\pgfqpoint{3.218398in}{3.465108in}}%
\pgfpathlineto{\pgfqpoint{3.085349in}{3.530998in}}%
\pgfpathclose%
\pgfusepath{fill}%
\end{pgfscope}%
\begin{pgfscope}%
\pgfpathrectangle{\pgfqpoint{0.765000in}{0.660000in}}{\pgfqpoint{4.620000in}{4.620000in}}%
\pgfusepath{clip}%
\pgfsetbuttcap%
\pgfsetroundjoin%
\definecolor{currentfill}{rgb}{1.000000,0.894118,0.788235}%
\pgfsetfillcolor{currentfill}%
\pgfsetlinewidth{0.000000pt}%
\definecolor{currentstroke}{rgb}{1.000000,0.894118,0.788235}%
\pgfsetstrokecolor{currentstroke}%
\pgfsetdash{}{0pt}%
\pgfpathmoveto{\pgfqpoint{3.196246in}{3.466972in}}%
\pgfpathlineto{\pgfqpoint{3.196246in}{3.595024in}}%
\pgfpathlineto{\pgfqpoint{3.218398in}{3.593160in}}%
\pgfpathlineto{\pgfqpoint{3.107502in}{3.529134in}}%
\pgfpathlineto{\pgfqpoint{3.196246in}{3.466972in}}%
\pgfpathclose%
\pgfusepath{fill}%
\end{pgfscope}%
\begin{pgfscope}%
\pgfpathrectangle{\pgfqpoint{0.765000in}{0.660000in}}{\pgfqpoint{4.620000in}{4.620000in}}%
\pgfusepath{clip}%
\pgfsetbuttcap%
\pgfsetroundjoin%
\definecolor{currentfill}{rgb}{1.000000,0.894118,0.788235}%
\pgfsetfillcolor{currentfill}%
\pgfsetlinewidth{0.000000pt}%
\definecolor{currentstroke}{rgb}{1.000000,0.894118,0.788235}%
\pgfsetstrokecolor{currentstroke}%
\pgfsetdash{}{0pt}%
\pgfpathmoveto{\pgfqpoint{2.974452in}{3.466972in}}%
\pgfpathlineto{\pgfqpoint{3.085349in}{3.402945in}}%
\pgfpathlineto{\pgfqpoint{3.107502in}{3.401081in}}%
\pgfpathlineto{\pgfqpoint{2.996605in}{3.465108in}}%
\pgfpathlineto{\pgfqpoint{2.974452in}{3.466972in}}%
\pgfpathclose%
\pgfusepath{fill}%
\end{pgfscope}%
\begin{pgfscope}%
\pgfpathrectangle{\pgfqpoint{0.765000in}{0.660000in}}{\pgfqpoint{4.620000in}{4.620000in}}%
\pgfusepath{clip}%
\pgfsetbuttcap%
\pgfsetroundjoin%
\definecolor{currentfill}{rgb}{1.000000,0.894118,0.788235}%
\pgfsetfillcolor{currentfill}%
\pgfsetlinewidth{0.000000pt}%
\definecolor{currentstroke}{rgb}{1.000000,0.894118,0.788235}%
\pgfsetstrokecolor{currentstroke}%
\pgfsetdash{}{0pt}%
\pgfpathmoveto{\pgfqpoint{3.085349in}{3.402945in}}%
\pgfpathlineto{\pgfqpoint{3.196246in}{3.466972in}}%
\pgfpathlineto{\pgfqpoint{3.218398in}{3.465108in}}%
\pgfpathlineto{\pgfqpoint{3.107502in}{3.401081in}}%
\pgfpathlineto{\pgfqpoint{3.085349in}{3.402945in}}%
\pgfpathclose%
\pgfusepath{fill}%
\end{pgfscope}%
\begin{pgfscope}%
\pgfpathrectangle{\pgfqpoint{0.765000in}{0.660000in}}{\pgfqpoint{4.620000in}{4.620000in}}%
\pgfusepath{clip}%
\pgfsetbuttcap%
\pgfsetroundjoin%
\definecolor{currentfill}{rgb}{1.000000,0.894118,0.788235}%
\pgfsetfillcolor{currentfill}%
\pgfsetlinewidth{0.000000pt}%
\definecolor{currentstroke}{rgb}{1.000000,0.894118,0.788235}%
\pgfsetstrokecolor{currentstroke}%
\pgfsetdash{}{0pt}%
\pgfpathmoveto{\pgfqpoint{3.085349in}{3.659050in}}%
\pgfpathlineto{\pgfqpoint{2.974452in}{3.595024in}}%
\pgfpathlineto{\pgfqpoint{2.996605in}{3.593160in}}%
\pgfpathlineto{\pgfqpoint{3.107502in}{3.657186in}}%
\pgfpathlineto{\pgfqpoint{3.085349in}{3.659050in}}%
\pgfpathclose%
\pgfusepath{fill}%
\end{pgfscope}%
\begin{pgfscope}%
\pgfpathrectangle{\pgfqpoint{0.765000in}{0.660000in}}{\pgfqpoint{4.620000in}{4.620000in}}%
\pgfusepath{clip}%
\pgfsetbuttcap%
\pgfsetroundjoin%
\definecolor{currentfill}{rgb}{1.000000,0.894118,0.788235}%
\pgfsetfillcolor{currentfill}%
\pgfsetlinewidth{0.000000pt}%
\definecolor{currentstroke}{rgb}{1.000000,0.894118,0.788235}%
\pgfsetstrokecolor{currentstroke}%
\pgfsetdash{}{0pt}%
\pgfpathmoveto{\pgfqpoint{3.196246in}{3.595024in}}%
\pgfpathlineto{\pgfqpoint{3.085349in}{3.659050in}}%
\pgfpathlineto{\pgfqpoint{3.107502in}{3.657186in}}%
\pgfpathlineto{\pgfqpoint{3.218398in}{3.593160in}}%
\pgfpathlineto{\pgfqpoint{3.196246in}{3.595024in}}%
\pgfpathclose%
\pgfusepath{fill}%
\end{pgfscope}%
\begin{pgfscope}%
\pgfpathrectangle{\pgfqpoint{0.765000in}{0.660000in}}{\pgfqpoint{4.620000in}{4.620000in}}%
\pgfusepath{clip}%
\pgfsetbuttcap%
\pgfsetroundjoin%
\definecolor{currentfill}{rgb}{1.000000,0.894118,0.788235}%
\pgfsetfillcolor{currentfill}%
\pgfsetlinewidth{0.000000pt}%
\definecolor{currentstroke}{rgb}{1.000000,0.894118,0.788235}%
\pgfsetstrokecolor{currentstroke}%
\pgfsetdash{}{0pt}%
\pgfpathmoveto{\pgfqpoint{2.974452in}{3.466972in}}%
\pgfpathlineto{\pgfqpoint{2.974452in}{3.595024in}}%
\pgfpathlineto{\pgfqpoint{2.996605in}{3.593160in}}%
\pgfpathlineto{\pgfqpoint{3.107502in}{3.401081in}}%
\pgfpathlineto{\pgfqpoint{2.974452in}{3.466972in}}%
\pgfpathclose%
\pgfusepath{fill}%
\end{pgfscope}%
\begin{pgfscope}%
\pgfpathrectangle{\pgfqpoint{0.765000in}{0.660000in}}{\pgfqpoint{4.620000in}{4.620000in}}%
\pgfusepath{clip}%
\pgfsetbuttcap%
\pgfsetroundjoin%
\definecolor{currentfill}{rgb}{1.000000,0.894118,0.788235}%
\pgfsetfillcolor{currentfill}%
\pgfsetlinewidth{0.000000pt}%
\definecolor{currentstroke}{rgb}{1.000000,0.894118,0.788235}%
\pgfsetstrokecolor{currentstroke}%
\pgfsetdash{}{0pt}%
\pgfpathmoveto{\pgfqpoint{3.085349in}{3.402945in}}%
\pgfpathlineto{\pgfqpoint{3.085349in}{3.530998in}}%
\pgfpathlineto{\pgfqpoint{3.107502in}{3.529134in}}%
\pgfpathlineto{\pgfqpoint{2.996605in}{3.465108in}}%
\pgfpathlineto{\pgfqpoint{3.085349in}{3.402945in}}%
\pgfpathclose%
\pgfusepath{fill}%
\end{pgfscope}%
\begin{pgfscope}%
\pgfpathrectangle{\pgfqpoint{0.765000in}{0.660000in}}{\pgfqpoint{4.620000in}{4.620000in}}%
\pgfusepath{clip}%
\pgfsetbuttcap%
\pgfsetroundjoin%
\definecolor{currentfill}{rgb}{1.000000,0.894118,0.788235}%
\pgfsetfillcolor{currentfill}%
\pgfsetlinewidth{0.000000pt}%
\definecolor{currentstroke}{rgb}{1.000000,0.894118,0.788235}%
\pgfsetstrokecolor{currentstroke}%
\pgfsetdash{}{0pt}%
\pgfpathmoveto{\pgfqpoint{3.085349in}{3.530998in}}%
\pgfpathlineto{\pgfqpoint{3.196246in}{3.595024in}}%
\pgfpathlineto{\pgfqpoint{3.218398in}{3.593160in}}%
\pgfpathlineto{\pgfqpoint{3.107502in}{3.529134in}}%
\pgfpathlineto{\pgfqpoint{3.085349in}{3.530998in}}%
\pgfpathclose%
\pgfusepath{fill}%
\end{pgfscope}%
\begin{pgfscope}%
\pgfpathrectangle{\pgfqpoint{0.765000in}{0.660000in}}{\pgfqpoint{4.620000in}{4.620000in}}%
\pgfusepath{clip}%
\pgfsetbuttcap%
\pgfsetroundjoin%
\definecolor{currentfill}{rgb}{1.000000,0.894118,0.788235}%
\pgfsetfillcolor{currentfill}%
\pgfsetlinewidth{0.000000pt}%
\definecolor{currentstroke}{rgb}{1.000000,0.894118,0.788235}%
\pgfsetstrokecolor{currentstroke}%
\pgfsetdash{}{0pt}%
\pgfpathmoveto{\pgfqpoint{2.974452in}{3.595024in}}%
\pgfpathlineto{\pgfqpoint{3.085349in}{3.530998in}}%
\pgfpathlineto{\pgfqpoint{3.107502in}{3.529134in}}%
\pgfpathlineto{\pgfqpoint{2.996605in}{3.593160in}}%
\pgfpathlineto{\pgfqpoint{2.974452in}{3.595024in}}%
\pgfpathclose%
\pgfusepath{fill}%
\end{pgfscope}%
\begin{pgfscope}%
\pgfpathrectangle{\pgfqpoint{0.765000in}{0.660000in}}{\pgfqpoint{4.620000in}{4.620000in}}%
\pgfusepath{clip}%
\pgfsetbuttcap%
\pgfsetroundjoin%
\definecolor{currentfill}{rgb}{1.000000,0.894118,0.788235}%
\pgfsetfillcolor{currentfill}%
\pgfsetlinewidth{0.000000pt}%
\definecolor{currentstroke}{rgb}{1.000000,0.894118,0.788235}%
\pgfsetstrokecolor{currentstroke}%
\pgfsetdash{}{0pt}%
\pgfpathmoveto{\pgfqpoint{3.046316in}{3.530257in}}%
\pgfpathlineto{\pgfqpoint{2.935419in}{3.466231in}}%
\pgfpathlineto{\pgfqpoint{3.046316in}{3.402204in}}%
\pgfpathlineto{\pgfqpoint{3.157212in}{3.466231in}}%
\pgfpathlineto{\pgfqpoint{3.046316in}{3.530257in}}%
\pgfpathclose%
\pgfusepath{fill}%
\end{pgfscope}%
\begin{pgfscope}%
\pgfpathrectangle{\pgfqpoint{0.765000in}{0.660000in}}{\pgfqpoint{4.620000in}{4.620000in}}%
\pgfusepath{clip}%
\pgfsetbuttcap%
\pgfsetroundjoin%
\definecolor{currentfill}{rgb}{1.000000,0.894118,0.788235}%
\pgfsetfillcolor{currentfill}%
\pgfsetlinewidth{0.000000pt}%
\definecolor{currentstroke}{rgb}{1.000000,0.894118,0.788235}%
\pgfsetstrokecolor{currentstroke}%
\pgfsetdash{}{0pt}%
\pgfpathmoveto{\pgfqpoint{3.046316in}{3.530257in}}%
\pgfpathlineto{\pgfqpoint{2.935419in}{3.466231in}}%
\pgfpathlineto{\pgfqpoint{2.935419in}{3.594283in}}%
\pgfpathlineto{\pgfqpoint{3.046316in}{3.658309in}}%
\pgfpathlineto{\pgfqpoint{3.046316in}{3.530257in}}%
\pgfpathclose%
\pgfusepath{fill}%
\end{pgfscope}%
\begin{pgfscope}%
\pgfpathrectangle{\pgfqpoint{0.765000in}{0.660000in}}{\pgfqpoint{4.620000in}{4.620000in}}%
\pgfusepath{clip}%
\pgfsetbuttcap%
\pgfsetroundjoin%
\definecolor{currentfill}{rgb}{1.000000,0.894118,0.788235}%
\pgfsetfillcolor{currentfill}%
\pgfsetlinewidth{0.000000pt}%
\definecolor{currentstroke}{rgb}{1.000000,0.894118,0.788235}%
\pgfsetstrokecolor{currentstroke}%
\pgfsetdash{}{0pt}%
\pgfpathmoveto{\pgfqpoint{3.046316in}{3.530257in}}%
\pgfpathlineto{\pgfqpoint{3.157212in}{3.466231in}}%
\pgfpathlineto{\pgfqpoint{3.157212in}{3.594283in}}%
\pgfpathlineto{\pgfqpoint{3.046316in}{3.658309in}}%
\pgfpathlineto{\pgfqpoint{3.046316in}{3.530257in}}%
\pgfpathclose%
\pgfusepath{fill}%
\end{pgfscope}%
\begin{pgfscope}%
\pgfpathrectangle{\pgfqpoint{0.765000in}{0.660000in}}{\pgfqpoint{4.620000in}{4.620000in}}%
\pgfusepath{clip}%
\pgfsetbuttcap%
\pgfsetroundjoin%
\definecolor{currentfill}{rgb}{1.000000,0.894118,0.788235}%
\pgfsetfillcolor{currentfill}%
\pgfsetlinewidth{0.000000pt}%
\definecolor{currentstroke}{rgb}{1.000000,0.894118,0.788235}%
\pgfsetstrokecolor{currentstroke}%
\pgfsetdash{}{0pt}%
\pgfpathmoveto{\pgfqpoint{3.085349in}{3.530998in}}%
\pgfpathlineto{\pgfqpoint{2.974452in}{3.466972in}}%
\pgfpathlineto{\pgfqpoint{3.085349in}{3.402945in}}%
\pgfpathlineto{\pgfqpoint{3.196246in}{3.466972in}}%
\pgfpathlineto{\pgfqpoint{3.085349in}{3.530998in}}%
\pgfpathclose%
\pgfusepath{fill}%
\end{pgfscope}%
\begin{pgfscope}%
\pgfpathrectangle{\pgfqpoint{0.765000in}{0.660000in}}{\pgfqpoint{4.620000in}{4.620000in}}%
\pgfusepath{clip}%
\pgfsetbuttcap%
\pgfsetroundjoin%
\definecolor{currentfill}{rgb}{1.000000,0.894118,0.788235}%
\pgfsetfillcolor{currentfill}%
\pgfsetlinewidth{0.000000pt}%
\definecolor{currentstroke}{rgb}{1.000000,0.894118,0.788235}%
\pgfsetstrokecolor{currentstroke}%
\pgfsetdash{}{0pt}%
\pgfpathmoveto{\pgfqpoint{3.085349in}{3.530998in}}%
\pgfpathlineto{\pgfqpoint{2.974452in}{3.466972in}}%
\pgfpathlineto{\pgfqpoint{2.974452in}{3.595024in}}%
\pgfpathlineto{\pgfqpoint{3.085349in}{3.659050in}}%
\pgfpathlineto{\pgfqpoint{3.085349in}{3.530998in}}%
\pgfpathclose%
\pgfusepath{fill}%
\end{pgfscope}%
\begin{pgfscope}%
\pgfpathrectangle{\pgfqpoint{0.765000in}{0.660000in}}{\pgfqpoint{4.620000in}{4.620000in}}%
\pgfusepath{clip}%
\pgfsetbuttcap%
\pgfsetroundjoin%
\definecolor{currentfill}{rgb}{1.000000,0.894118,0.788235}%
\pgfsetfillcolor{currentfill}%
\pgfsetlinewidth{0.000000pt}%
\definecolor{currentstroke}{rgb}{1.000000,0.894118,0.788235}%
\pgfsetstrokecolor{currentstroke}%
\pgfsetdash{}{0pt}%
\pgfpathmoveto{\pgfqpoint{3.085349in}{3.530998in}}%
\pgfpathlineto{\pgfqpoint{3.196246in}{3.466972in}}%
\pgfpathlineto{\pgfqpoint{3.196246in}{3.595024in}}%
\pgfpathlineto{\pgfqpoint{3.085349in}{3.659050in}}%
\pgfpathlineto{\pgfqpoint{3.085349in}{3.530998in}}%
\pgfpathclose%
\pgfusepath{fill}%
\end{pgfscope}%
\begin{pgfscope}%
\pgfpathrectangle{\pgfqpoint{0.765000in}{0.660000in}}{\pgfqpoint{4.620000in}{4.620000in}}%
\pgfusepath{clip}%
\pgfsetbuttcap%
\pgfsetroundjoin%
\definecolor{currentfill}{rgb}{1.000000,0.894118,0.788235}%
\pgfsetfillcolor{currentfill}%
\pgfsetlinewidth{0.000000pt}%
\definecolor{currentstroke}{rgb}{1.000000,0.894118,0.788235}%
\pgfsetstrokecolor{currentstroke}%
\pgfsetdash{}{0pt}%
\pgfpathmoveto{\pgfqpoint{3.046316in}{3.658309in}}%
\pgfpathlineto{\pgfqpoint{2.935419in}{3.594283in}}%
\pgfpathlineto{\pgfqpoint{3.046316in}{3.530257in}}%
\pgfpathlineto{\pgfqpoint{3.157212in}{3.594283in}}%
\pgfpathlineto{\pgfqpoint{3.046316in}{3.658309in}}%
\pgfpathclose%
\pgfusepath{fill}%
\end{pgfscope}%
\begin{pgfscope}%
\pgfpathrectangle{\pgfqpoint{0.765000in}{0.660000in}}{\pgfqpoint{4.620000in}{4.620000in}}%
\pgfusepath{clip}%
\pgfsetbuttcap%
\pgfsetroundjoin%
\definecolor{currentfill}{rgb}{1.000000,0.894118,0.788235}%
\pgfsetfillcolor{currentfill}%
\pgfsetlinewidth{0.000000pt}%
\definecolor{currentstroke}{rgb}{1.000000,0.894118,0.788235}%
\pgfsetstrokecolor{currentstroke}%
\pgfsetdash{}{0pt}%
\pgfpathmoveto{\pgfqpoint{3.046316in}{3.402204in}}%
\pgfpathlineto{\pgfqpoint{3.157212in}{3.466231in}}%
\pgfpathlineto{\pgfqpoint{3.157212in}{3.594283in}}%
\pgfpathlineto{\pgfqpoint{3.046316in}{3.530257in}}%
\pgfpathlineto{\pgfqpoint{3.046316in}{3.402204in}}%
\pgfpathclose%
\pgfusepath{fill}%
\end{pgfscope}%
\begin{pgfscope}%
\pgfpathrectangle{\pgfqpoint{0.765000in}{0.660000in}}{\pgfqpoint{4.620000in}{4.620000in}}%
\pgfusepath{clip}%
\pgfsetbuttcap%
\pgfsetroundjoin%
\definecolor{currentfill}{rgb}{1.000000,0.894118,0.788235}%
\pgfsetfillcolor{currentfill}%
\pgfsetlinewidth{0.000000pt}%
\definecolor{currentstroke}{rgb}{1.000000,0.894118,0.788235}%
\pgfsetstrokecolor{currentstroke}%
\pgfsetdash{}{0pt}%
\pgfpathmoveto{\pgfqpoint{2.935419in}{3.466231in}}%
\pgfpathlineto{\pgfqpoint{3.046316in}{3.402204in}}%
\pgfpathlineto{\pgfqpoint{3.046316in}{3.530257in}}%
\pgfpathlineto{\pgfqpoint{2.935419in}{3.594283in}}%
\pgfpathlineto{\pgfqpoint{2.935419in}{3.466231in}}%
\pgfpathclose%
\pgfusepath{fill}%
\end{pgfscope}%
\begin{pgfscope}%
\pgfpathrectangle{\pgfqpoint{0.765000in}{0.660000in}}{\pgfqpoint{4.620000in}{4.620000in}}%
\pgfusepath{clip}%
\pgfsetbuttcap%
\pgfsetroundjoin%
\definecolor{currentfill}{rgb}{1.000000,0.894118,0.788235}%
\pgfsetfillcolor{currentfill}%
\pgfsetlinewidth{0.000000pt}%
\definecolor{currentstroke}{rgb}{1.000000,0.894118,0.788235}%
\pgfsetstrokecolor{currentstroke}%
\pgfsetdash{}{0pt}%
\pgfpathmoveto{\pgfqpoint{3.085349in}{3.659050in}}%
\pgfpathlineto{\pgfqpoint{2.974452in}{3.595024in}}%
\pgfpathlineto{\pgfqpoint{3.085349in}{3.530998in}}%
\pgfpathlineto{\pgfqpoint{3.196246in}{3.595024in}}%
\pgfpathlineto{\pgfqpoint{3.085349in}{3.659050in}}%
\pgfpathclose%
\pgfusepath{fill}%
\end{pgfscope}%
\begin{pgfscope}%
\pgfpathrectangle{\pgfqpoint{0.765000in}{0.660000in}}{\pgfqpoint{4.620000in}{4.620000in}}%
\pgfusepath{clip}%
\pgfsetbuttcap%
\pgfsetroundjoin%
\definecolor{currentfill}{rgb}{1.000000,0.894118,0.788235}%
\pgfsetfillcolor{currentfill}%
\pgfsetlinewidth{0.000000pt}%
\definecolor{currentstroke}{rgb}{1.000000,0.894118,0.788235}%
\pgfsetstrokecolor{currentstroke}%
\pgfsetdash{}{0pt}%
\pgfpathmoveto{\pgfqpoint{3.085349in}{3.402945in}}%
\pgfpathlineto{\pgfqpoint{3.196246in}{3.466972in}}%
\pgfpathlineto{\pgfqpoint{3.196246in}{3.595024in}}%
\pgfpathlineto{\pgfqpoint{3.085349in}{3.530998in}}%
\pgfpathlineto{\pgfqpoint{3.085349in}{3.402945in}}%
\pgfpathclose%
\pgfusepath{fill}%
\end{pgfscope}%
\begin{pgfscope}%
\pgfpathrectangle{\pgfqpoint{0.765000in}{0.660000in}}{\pgfqpoint{4.620000in}{4.620000in}}%
\pgfusepath{clip}%
\pgfsetbuttcap%
\pgfsetroundjoin%
\definecolor{currentfill}{rgb}{1.000000,0.894118,0.788235}%
\pgfsetfillcolor{currentfill}%
\pgfsetlinewidth{0.000000pt}%
\definecolor{currentstroke}{rgb}{1.000000,0.894118,0.788235}%
\pgfsetstrokecolor{currentstroke}%
\pgfsetdash{}{0pt}%
\pgfpathmoveto{\pgfqpoint{2.974452in}{3.466972in}}%
\pgfpathlineto{\pgfqpoint{3.085349in}{3.402945in}}%
\pgfpathlineto{\pgfqpoint{3.085349in}{3.530998in}}%
\pgfpathlineto{\pgfqpoint{2.974452in}{3.595024in}}%
\pgfpathlineto{\pgfqpoint{2.974452in}{3.466972in}}%
\pgfpathclose%
\pgfusepath{fill}%
\end{pgfscope}%
\begin{pgfscope}%
\pgfpathrectangle{\pgfqpoint{0.765000in}{0.660000in}}{\pgfqpoint{4.620000in}{4.620000in}}%
\pgfusepath{clip}%
\pgfsetbuttcap%
\pgfsetroundjoin%
\definecolor{currentfill}{rgb}{1.000000,0.894118,0.788235}%
\pgfsetfillcolor{currentfill}%
\pgfsetlinewidth{0.000000pt}%
\definecolor{currentstroke}{rgb}{1.000000,0.894118,0.788235}%
\pgfsetstrokecolor{currentstroke}%
\pgfsetdash{}{0pt}%
\pgfpathmoveto{\pgfqpoint{3.046316in}{3.530257in}}%
\pgfpathlineto{\pgfqpoint{2.935419in}{3.466231in}}%
\pgfpathlineto{\pgfqpoint{2.974452in}{3.466972in}}%
\pgfpathlineto{\pgfqpoint{3.085349in}{3.530998in}}%
\pgfpathlineto{\pgfqpoint{3.046316in}{3.530257in}}%
\pgfpathclose%
\pgfusepath{fill}%
\end{pgfscope}%
\begin{pgfscope}%
\pgfpathrectangle{\pgfqpoint{0.765000in}{0.660000in}}{\pgfqpoint{4.620000in}{4.620000in}}%
\pgfusepath{clip}%
\pgfsetbuttcap%
\pgfsetroundjoin%
\definecolor{currentfill}{rgb}{1.000000,0.894118,0.788235}%
\pgfsetfillcolor{currentfill}%
\pgfsetlinewidth{0.000000pt}%
\definecolor{currentstroke}{rgb}{1.000000,0.894118,0.788235}%
\pgfsetstrokecolor{currentstroke}%
\pgfsetdash{}{0pt}%
\pgfpathmoveto{\pgfqpoint{3.157212in}{3.466231in}}%
\pgfpathlineto{\pgfqpoint{3.046316in}{3.530257in}}%
\pgfpathlineto{\pgfqpoint{3.085349in}{3.530998in}}%
\pgfpathlineto{\pgfqpoint{3.196246in}{3.466972in}}%
\pgfpathlineto{\pgfqpoint{3.157212in}{3.466231in}}%
\pgfpathclose%
\pgfusepath{fill}%
\end{pgfscope}%
\begin{pgfscope}%
\pgfpathrectangle{\pgfqpoint{0.765000in}{0.660000in}}{\pgfqpoint{4.620000in}{4.620000in}}%
\pgfusepath{clip}%
\pgfsetbuttcap%
\pgfsetroundjoin%
\definecolor{currentfill}{rgb}{1.000000,0.894118,0.788235}%
\pgfsetfillcolor{currentfill}%
\pgfsetlinewidth{0.000000pt}%
\definecolor{currentstroke}{rgb}{1.000000,0.894118,0.788235}%
\pgfsetstrokecolor{currentstroke}%
\pgfsetdash{}{0pt}%
\pgfpathmoveto{\pgfqpoint{3.046316in}{3.530257in}}%
\pgfpathlineto{\pgfqpoint{3.046316in}{3.658309in}}%
\pgfpathlineto{\pgfqpoint{3.085349in}{3.659050in}}%
\pgfpathlineto{\pgfqpoint{3.196246in}{3.466972in}}%
\pgfpathlineto{\pgfqpoint{3.046316in}{3.530257in}}%
\pgfpathclose%
\pgfusepath{fill}%
\end{pgfscope}%
\begin{pgfscope}%
\pgfpathrectangle{\pgfqpoint{0.765000in}{0.660000in}}{\pgfqpoint{4.620000in}{4.620000in}}%
\pgfusepath{clip}%
\pgfsetbuttcap%
\pgfsetroundjoin%
\definecolor{currentfill}{rgb}{1.000000,0.894118,0.788235}%
\pgfsetfillcolor{currentfill}%
\pgfsetlinewidth{0.000000pt}%
\definecolor{currentstroke}{rgb}{1.000000,0.894118,0.788235}%
\pgfsetstrokecolor{currentstroke}%
\pgfsetdash{}{0pt}%
\pgfpathmoveto{\pgfqpoint{3.157212in}{3.466231in}}%
\pgfpathlineto{\pgfqpoint{3.157212in}{3.594283in}}%
\pgfpathlineto{\pgfqpoint{3.196246in}{3.595024in}}%
\pgfpathlineto{\pgfqpoint{3.085349in}{3.530998in}}%
\pgfpathlineto{\pgfqpoint{3.157212in}{3.466231in}}%
\pgfpathclose%
\pgfusepath{fill}%
\end{pgfscope}%
\begin{pgfscope}%
\pgfpathrectangle{\pgfqpoint{0.765000in}{0.660000in}}{\pgfqpoint{4.620000in}{4.620000in}}%
\pgfusepath{clip}%
\pgfsetbuttcap%
\pgfsetroundjoin%
\definecolor{currentfill}{rgb}{1.000000,0.894118,0.788235}%
\pgfsetfillcolor{currentfill}%
\pgfsetlinewidth{0.000000pt}%
\definecolor{currentstroke}{rgb}{1.000000,0.894118,0.788235}%
\pgfsetstrokecolor{currentstroke}%
\pgfsetdash{}{0pt}%
\pgfpathmoveto{\pgfqpoint{2.935419in}{3.466231in}}%
\pgfpathlineto{\pgfqpoint{3.046316in}{3.402204in}}%
\pgfpathlineto{\pgfqpoint{3.085349in}{3.402945in}}%
\pgfpathlineto{\pgfqpoint{2.974452in}{3.466972in}}%
\pgfpathlineto{\pgfqpoint{2.935419in}{3.466231in}}%
\pgfpathclose%
\pgfusepath{fill}%
\end{pgfscope}%
\begin{pgfscope}%
\pgfpathrectangle{\pgfqpoint{0.765000in}{0.660000in}}{\pgfqpoint{4.620000in}{4.620000in}}%
\pgfusepath{clip}%
\pgfsetbuttcap%
\pgfsetroundjoin%
\definecolor{currentfill}{rgb}{1.000000,0.894118,0.788235}%
\pgfsetfillcolor{currentfill}%
\pgfsetlinewidth{0.000000pt}%
\definecolor{currentstroke}{rgb}{1.000000,0.894118,0.788235}%
\pgfsetstrokecolor{currentstroke}%
\pgfsetdash{}{0pt}%
\pgfpathmoveto{\pgfqpoint{3.046316in}{3.402204in}}%
\pgfpathlineto{\pgfqpoint{3.157212in}{3.466231in}}%
\pgfpathlineto{\pgfqpoint{3.196246in}{3.466972in}}%
\pgfpathlineto{\pgfqpoint{3.085349in}{3.402945in}}%
\pgfpathlineto{\pgfqpoint{3.046316in}{3.402204in}}%
\pgfpathclose%
\pgfusepath{fill}%
\end{pgfscope}%
\begin{pgfscope}%
\pgfpathrectangle{\pgfqpoint{0.765000in}{0.660000in}}{\pgfqpoint{4.620000in}{4.620000in}}%
\pgfusepath{clip}%
\pgfsetbuttcap%
\pgfsetroundjoin%
\definecolor{currentfill}{rgb}{1.000000,0.894118,0.788235}%
\pgfsetfillcolor{currentfill}%
\pgfsetlinewidth{0.000000pt}%
\definecolor{currentstroke}{rgb}{1.000000,0.894118,0.788235}%
\pgfsetstrokecolor{currentstroke}%
\pgfsetdash{}{0pt}%
\pgfpathmoveto{\pgfqpoint{3.046316in}{3.658309in}}%
\pgfpathlineto{\pgfqpoint{2.935419in}{3.594283in}}%
\pgfpathlineto{\pgfqpoint{2.974452in}{3.595024in}}%
\pgfpathlineto{\pgfqpoint{3.085349in}{3.659050in}}%
\pgfpathlineto{\pgfqpoint{3.046316in}{3.658309in}}%
\pgfpathclose%
\pgfusepath{fill}%
\end{pgfscope}%
\begin{pgfscope}%
\pgfpathrectangle{\pgfqpoint{0.765000in}{0.660000in}}{\pgfqpoint{4.620000in}{4.620000in}}%
\pgfusepath{clip}%
\pgfsetbuttcap%
\pgfsetroundjoin%
\definecolor{currentfill}{rgb}{1.000000,0.894118,0.788235}%
\pgfsetfillcolor{currentfill}%
\pgfsetlinewidth{0.000000pt}%
\definecolor{currentstroke}{rgb}{1.000000,0.894118,0.788235}%
\pgfsetstrokecolor{currentstroke}%
\pgfsetdash{}{0pt}%
\pgfpathmoveto{\pgfqpoint{3.157212in}{3.594283in}}%
\pgfpathlineto{\pgfqpoint{3.046316in}{3.658309in}}%
\pgfpathlineto{\pgfqpoint{3.085349in}{3.659050in}}%
\pgfpathlineto{\pgfqpoint{3.196246in}{3.595024in}}%
\pgfpathlineto{\pgfqpoint{3.157212in}{3.594283in}}%
\pgfpathclose%
\pgfusepath{fill}%
\end{pgfscope}%
\begin{pgfscope}%
\pgfpathrectangle{\pgfqpoint{0.765000in}{0.660000in}}{\pgfqpoint{4.620000in}{4.620000in}}%
\pgfusepath{clip}%
\pgfsetbuttcap%
\pgfsetroundjoin%
\definecolor{currentfill}{rgb}{1.000000,0.894118,0.788235}%
\pgfsetfillcolor{currentfill}%
\pgfsetlinewidth{0.000000pt}%
\definecolor{currentstroke}{rgb}{1.000000,0.894118,0.788235}%
\pgfsetstrokecolor{currentstroke}%
\pgfsetdash{}{0pt}%
\pgfpathmoveto{\pgfqpoint{2.935419in}{3.466231in}}%
\pgfpathlineto{\pgfqpoint{2.935419in}{3.594283in}}%
\pgfpathlineto{\pgfqpoint{2.974452in}{3.595024in}}%
\pgfpathlineto{\pgfqpoint{3.085349in}{3.402945in}}%
\pgfpathlineto{\pgfqpoint{2.935419in}{3.466231in}}%
\pgfpathclose%
\pgfusepath{fill}%
\end{pgfscope}%
\begin{pgfscope}%
\pgfpathrectangle{\pgfqpoint{0.765000in}{0.660000in}}{\pgfqpoint{4.620000in}{4.620000in}}%
\pgfusepath{clip}%
\pgfsetbuttcap%
\pgfsetroundjoin%
\definecolor{currentfill}{rgb}{1.000000,0.894118,0.788235}%
\pgfsetfillcolor{currentfill}%
\pgfsetlinewidth{0.000000pt}%
\definecolor{currentstroke}{rgb}{1.000000,0.894118,0.788235}%
\pgfsetstrokecolor{currentstroke}%
\pgfsetdash{}{0pt}%
\pgfpathmoveto{\pgfqpoint{3.046316in}{3.402204in}}%
\pgfpathlineto{\pgfqpoint{3.046316in}{3.530257in}}%
\pgfpathlineto{\pgfqpoint{3.085349in}{3.530998in}}%
\pgfpathlineto{\pgfqpoint{2.974452in}{3.466972in}}%
\pgfpathlineto{\pgfqpoint{3.046316in}{3.402204in}}%
\pgfpathclose%
\pgfusepath{fill}%
\end{pgfscope}%
\begin{pgfscope}%
\pgfpathrectangle{\pgfqpoint{0.765000in}{0.660000in}}{\pgfqpoint{4.620000in}{4.620000in}}%
\pgfusepath{clip}%
\pgfsetbuttcap%
\pgfsetroundjoin%
\definecolor{currentfill}{rgb}{1.000000,0.894118,0.788235}%
\pgfsetfillcolor{currentfill}%
\pgfsetlinewidth{0.000000pt}%
\definecolor{currentstroke}{rgb}{1.000000,0.894118,0.788235}%
\pgfsetstrokecolor{currentstroke}%
\pgfsetdash{}{0pt}%
\pgfpathmoveto{\pgfqpoint{3.046316in}{3.530257in}}%
\pgfpathlineto{\pgfqpoint{3.157212in}{3.594283in}}%
\pgfpathlineto{\pgfqpoint{3.196246in}{3.595024in}}%
\pgfpathlineto{\pgfqpoint{3.085349in}{3.530998in}}%
\pgfpathlineto{\pgfqpoint{3.046316in}{3.530257in}}%
\pgfpathclose%
\pgfusepath{fill}%
\end{pgfscope}%
\begin{pgfscope}%
\pgfpathrectangle{\pgfqpoint{0.765000in}{0.660000in}}{\pgfqpoint{4.620000in}{4.620000in}}%
\pgfusepath{clip}%
\pgfsetbuttcap%
\pgfsetroundjoin%
\definecolor{currentfill}{rgb}{1.000000,0.894118,0.788235}%
\pgfsetfillcolor{currentfill}%
\pgfsetlinewidth{0.000000pt}%
\definecolor{currentstroke}{rgb}{1.000000,0.894118,0.788235}%
\pgfsetstrokecolor{currentstroke}%
\pgfsetdash{}{0pt}%
\pgfpathmoveto{\pgfqpoint{2.935419in}{3.594283in}}%
\pgfpathlineto{\pgfqpoint{3.046316in}{3.530257in}}%
\pgfpathlineto{\pgfqpoint{3.085349in}{3.530998in}}%
\pgfpathlineto{\pgfqpoint{2.974452in}{3.595024in}}%
\pgfpathlineto{\pgfqpoint{2.935419in}{3.594283in}}%
\pgfpathclose%
\pgfusepath{fill}%
\end{pgfscope}%
\begin{pgfscope}%
\pgfpathrectangle{\pgfqpoint{0.765000in}{0.660000in}}{\pgfqpoint{4.620000in}{4.620000in}}%
\pgfusepath{clip}%
\pgfsetbuttcap%
\pgfsetroundjoin%
\definecolor{currentfill}{rgb}{1.000000,0.894118,0.788235}%
\pgfsetfillcolor{currentfill}%
\pgfsetlinewidth{0.000000pt}%
\definecolor{currentstroke}{rgb}{1.000000,0.894118,0.788235}%
\pgfsetstrokecolor{currentstroke}%
\pgfsetdash{}{0pt}%
\pgfpathmoveto{\pgfqpoint{3.046316in}{3.530257in}}%
\pgfpathlineto{\pgfqpoint{2.935419in}{3.466231in}}%
\pgfpathlineto{\pgfqpoint{3.046316in}{3.402204in}}%
\pgfpathlineto{\pgfqpoint{3.157212in}{3.466231in}}%
\pgfpathlineto{\pgfqpoint{3.046316in}{3.530257in}}%
\pgfpathclose%
\pgfusepath{fill}%
\end{pgfscope}%
\begin{pgfscope}%
\pgfpathrectangle{\pgfqpoint{0.765000in}{0.660000in}}{\pgfqpoint{4.620000in}{4.620000in}}%
\pgfusepath{clip}%
\pgfsetbuttcap%
\pgfsetroundjoin%
\definecolor{currentfill}{rgb}{1.000000,0.894118,0.788235}%
\pgfsetfillcolor{currentfill}%
\pgfsetlinewidth{0.000000pt}%
\definecolor{currentstroke}{rgb}{1.000000,0.894118,0.788235}%
\pgfsetstrokecolor{currentstroke}%
\pgfsetdash{}{0pt}%
\pgfpathmoveto{\pgfqpoint{3.046316in}{3.530257in}}%
\pgfpathlineto{\pgfqpoint{2.935419in}{3.466231in}}%
\pgfpathlineto{\pgfqpoint{2.935419in}{3.594283in}}%
\pgfpathlineto{\pgfqpoint{3.046316in}{3.658309in}}%
\pgfpathlineto{\pgfqpoint{3.046316in}{3.530257in}}%
\pgfpathclose%
\pgfusepath{fill}%
\end{pgfscope}%
\begin{pgfscope}%
\pgfpathrectangle{\pgfqpoint{0.765000in}{0.660000in}}{\pgfqpoint{4.620000in}{4.620000in}}%
\pgfusepath{clip}%
\pgfsetbuttcap%
\pgfsetroundjoin%
\definecolor{currentfill}{rgb}{1.000000,0.894118,0.788235}%
\pgfsetfillcolor{currentfill}%
\pgfsetlinewidth{0.000000pt}%
\definecolor{currentstroke}{rgb}{1.000000,0.894118,0.788235}%
\pgfsetstrokecolor{currentstroke}%
\pgfsetdash{}{0pt}%
\pgfpathmoveto{\pgfqpoint{3.046316in}{3.530257in}}%
\pgfpathlineto{\pgfqpoint{3.157212in}{3.466231in}}%
\pgfpathlineto{\pgfqpoint{3.157212in}{3.594283in}}%
\pgfpathlineto{\pgfqpoint{3.046316in}{3.658309in}}%
\pgfpathlineto{\pgfqpoint{3.046316in}{3.530257in}}%
\pgfpathclose%
\pgfusepath{fill}%
\end{pgfscope}%
\begin{pgfscope}%
\pgfpathrectangle{\pgfqpoint{0.765000in}{0.660000in}}{\pgfqpoint{4.620000in}{4.620000in}}%
\pgfusepath{clip}%
\pgfsetbuttcap%
\pgfsetroundjoin%
\definecolor{currentfill}{rgb}{1.000000,0.894118,0.788235}%
\pgfsetfillcolor{currentfill}%
\pgfsetlinewidth{0.000000pt}%
\definecolor{currentstroke}{rgb}{1.000000,0.894118,0.788235}%
\pgfsetstrokecolor{currentstroke}%
\pgfsetdash{}{0pt}%
\pgfpathmoveto{\pgfqpoint{2.982382in}{3.516877in}}%
\pgfpathlineto{\pgfqpoint{2.871485in}{3.452851in}}%
\pgfpathlineto{\pgfqpoint{2.982382in}{3.388824in}}%
\pgfpathlineto{\pgfqpoint{3.093279in}{3.452851in}}%
\pgfpathlineto{\pgfqpoint{2.982382in}{3.516877in}}%
\pgfpathclose%
\pgfusepath{fill}%
\end{pgfscope}%
\begin{pgfscope}%
\pgfpathrectangle{\pgfqpoint{0.765000in}{0.660000in}}{\pgfqpoint{4.620000in}{4.620000in}}%
\pgfusepath{clip}%
\pgfsetbuttcap%
\pgfsetroundjoin%
\definecolor{currentfill}{rgb}{1.000000,0.894118,0.788235}%
\pgfsetfillcolor{currentfill}%
\pgfsetlinewidth{0.000000pt}%
\definecolor{currentstroke}{rgb}{1.000000,0.894118,0.788235}%
\pgfsetstrokecolor{currentstroke}%
\pgfsetdash{}{0pt}%
\pgfpathmoveto{\pgfqpoint{2.982382in}{3.516877in}}%
\pgfpathlineto{\pgfqpoint{2.871485in}{3.452851in}}%
\pgfpathlineto{\pgfqpoint{2.871485in}{3.580903in}}%
\pgfpathlineto{\pgfqpoint{2.982382in}{3.644929in}}%
\pgfpathlineto{\pgfqpoint{2.982382in}{3.516877in}}%
\pgfpathclose%
\pgfusepath{fill}%
\end{pgfscope}%
\begin{pgfscope}%
\pgfpathrectangle{\pgfqpoint{0.765000in}{0.660000in}}{\pgfqpoint{4.620000in}{4.620000in}}%
\pgfusepath{clip}%
\pgfsetbuttcap%
\pgfsetroundjoin%
\definecolor{currentfill}{rgb}{1.000000,0.894118,0.788235}%
\pgfsetfillcolor{currentfill}%
\pgfsetlinewidth{0.000000pt}%
\definecolor{currentstroke}{rgb}{1.000000,0.894118,0.788235}%
\pgfsetstrokecolor{currentstroke}%
\pgfsetdash{}{0pt}%
\pgfpathmoveto{\pgfqpoint{2.982382in}{3.516877in}}%
\pgfpathlineto{\pgfqpoint{3.093279in}{3.452851in}}%
\pgfpathlineto{\pgfqpoint{3.093279in}{3.580903in}}%
\pgfpathlineto{\pgfqpoint{2.982382in}{3.644929in}}%
\pgfpathlineto{\pgfqpoint{2.982382in}{3.516877in}}%
\pgfpathclose%
\pgfusepath{fill}%
\end{pgfscope}%
\begin{pgfscope}%
\pgfpathrectangle{\pgfqpoint{0.765000in}{0.660000in}}{\pgfqpoint{4.620000in}{4.620000in}}%
\pgfusepath{clip}%
\pgfsetbuttcap%
\pgfsetroundjoin%
\definecolor{currentfill}{rgb}{1.000000,0.894118,0.788235}%
\pgfsetfillcolor{currentfill}%
\pgfsetlinewidth{0.000000pt}%
\definecolor{currentstroke}{rgb}{1.000000,0.894118,0.788235}%
\pgfsetstrokecolor{currentstroke}%
\pgfsetdash{}{0pt}%
\pgfpathmoveto{\pgfqpoint{3.046316in}{3.658309in}}%
\pgfpathlineto{\pgfqpoint{2.935419in}{3.594283in}}%
\pgfpathlineto{\pgfqpoint{3.046316in}{3.530257in}}%
\pgfpathlineto{\pgfqpoint{3.157212in}{3.594283in}}%
\pgfpathlineto{\pgfqpoint{3.046316in}{3.658309in}}%
\pgfpathclose%
\pgfusepath{fill}%
\end{pgfscope}%
\begin{pgfscope}%
\pgfpathrectangle{\pgfqpoint{0.765000in}{0.660000in}}{\pgfqpoint{4.620000in}{4.620000in}}%
\pgfusepath{clip}%
\pgfsetbuttcap%
\pgfsetroundjoin%
\definecolor{currentfill}{rgb}{1.000000,0.894118,0.788235}%
\pgfsetfillcolor{currentfill}%
\pgfsetlinewidth{0.000000pt}%
\definecolor{currentstroke}{rgb}{1.000000,0.894118,0.788235}%
\pgfsetstrokecolor{currentstroke}%
\pgfsetdash{}{0pt}%
\pgfpathmoveto{\pgfqpoint{3.046316in}{3.402204in}}%
\pgfpathlineto{\pgfqpoint{3.157212in}{3.466231in}}%
\pgfpathlineto{\pgfqpoint{3.157212in}{3.594283in}}%
\pgfpathlineto{\pgfqpoint{3.046316in}{3.530257in}}%
\pgfpathlineto{\pgfqpoint{3.046316in}{3.402204in}}%
\pgfpathclose%
\pgfusepath{fill}%
\end{pgfscope}%
\begin{pgfscope}%
\pgfpathrectangle{\pgfqpoint{0.765000in}{0.660000in}}{\pgfqpoint{4.620000in}{4.620000in}}%
\pgfusepath{clip}%
\pgfsetbuttcap%
\pgfsetroundjoin%
\definecolor{currentfill}{rgb}{1.000000,0.894118,0.788235}%
\pgfsetfillcolor{currentfill}%
\pgfsetlinewidth{0.000000pt}%
\definecolor{currentstroke}{rgb}{1.000000,0.894118,0.788235}%
\pgfsetstrokecolor{currentstroke}%
\pgfsetdash{}{0pt}%
\pgfpathmoveto{\pgfqpoint{2.935419in}{3.466231in}}%
\pgfpathlineto{\pgfqpoint{3.046316in}{3.402204in}}%
\pgfpathlineto{\pgfqpoint{3.046316in}{3.530257in}}%
\pgfpathlineto{\pgfqpoint{2.935419in}{3.594283in}}%
\pgfpathlineto{\pgfqpoint{2.935419in}{3.466231in}}%
\pgfpathclose%
\pgfusepath{fill}%
\end{pgfscope}%
\begin{pgfscope}%
\pgfpathrectangle{\pgfqpoint{0.765000in}{0.660000in}}{\pgfqpoint{4.620000in}{4.620000in}}%
\pgfusepath{clip}%
\pgfsetbuttcap%
\pgfsetroundjoin%
\definecolor{currentfill}{rgb}{1.000000,0.894118,0.788235}%
\pgfsetfillcolor{currentfill}%
\pgfsetlinewidth{0.000000pt}%
\definecolor{currentstroke}{rgb}{1.000000,0.894118,0.788235}%
\pgfsetstrokecolor{currentstroke}%
\pgfsetdash{}{0pt}%
\pgfpathmoveto{\pgfqpoint{2.982382in}{3.644929in}}%
\pgfpathlineto{\pgfqpoint{2.871485in}{3.580903in}}%
\pgfpathlineto{\pgfqpoint{2.982382in}{3.516877in}}%
\pgfpathlineto{\pgfqpoint{3.093279in}{3.580903in}}%
\pgfpathlineto{\pgfqpoint{2.982382in}{3.644929in}}%
\pgfpathclose%
\pgfusepath{fill}%
\end{pgfscope}%
\begin{pgfscope}%
\pgfpathrectangle{\pgfqpoint{0.765000in}{0.660000in}}{\pgfqpoint{4.620000in}{4.620000in}}%
\pgfusepath{clip}%
\pgfsetbuttcap%
\pgfsetroundjoin%
\definecolor{currentfill}{rgb}{1.000000,0.894118,0.788235}%
\pgfsetfillcolor{currentfill}%
\pgfsetlinewidth{0.000000pt}%
\definecolor{currentstroke}{rgb}{1.000000,0.894118,0.788235}%
\pgfsetstrokecolor{currentstroke}%
\pgfsetdash{}{0pt}%
\pgfpathmoveto{\pgfqpoint{2.982382in}{3.388824in}}%
\pgfpathlineto{\pgfqpoint{3.093279in}{3.452851in}}%
\pgfpathlineto{\pgfqpoint{3.093279in}{3.580903in}}%
\pgfpathlineto{\pgfqpoint{2.982382in}{3.516877in}}%
\pgfpathlineto{\pgfqpoint{2.982382in}{3.388824in}}%
\pgfpathclose%
\pgfusepath{fill}%
\end{pgfscope}%
\begin{pgfscope}%
\pgfpathrectangle{\pgfqpoint{0.765000in}{0.660000in}}{\pgfqpoint{4.620000in}{4.620000in}}%
\pgfusepath{clip}%
\pgfsetbuttcap%
\pgfsetroundjoin%
\definecolor{currentfill}{rgb}{1.000000,0.894118,0.788235}%
\pgfsetfillcolor{currentfill}%
\pgfsetlinewidth{0.000000pt}%
\definecolor{currentstroke}{rgb}{1.000000,0.894118,0.788235}%
\pgfsetstrokecolor{currentstroke}%
\pgfsetdash{}{0pt}%
\pgfpathmoveto{\pgfqpoint{2.871485in}{3.452851in}}%
\pgfpathlineto{\pgfqpoint{2.982382in}{3.388824in}}%
\pgfpathlineto{\pgfqpoint{2.982382in}{3.516877in}}%
\pgfpathlineto{\pgfqpoint{2.871485in}{3.580903in}}%
\pgfpathlineto{\pgfqpoint{2.871485in}{3.452851in}}%
\pgfpathclose%
\pgfusepath{fill}%
\end{pgfscope}%
\begin{pgfscope}%
\pgfpathrectangle{\pgfqpoint{0.765000in}{0.660000in}}{\pgfqpoint{4.620000in}{4.620000in}}%
\pgfusepath{clip}%
\pgfsetbuttcap%
\pgfsetroundjoin%
\definecolor{currentfill}{rgb}{1.000000,0.894118,0.788235}%
\pgfsetfillcolor{currentfill}%
\pgfsetlinewidth{0.000000pt}%
\definecolor{currentstroke}{rgb}{1.000000,0.894118,0.788235}%
\pgfsetstrokecolor{currentstroke}%
\pgfsetdash{}{0pt}%
\pgfpathmoveto{\pgfqpoint{3.046316in}{3.530257in}}%
\pgfpathlineto{\pgfqpoint{2.935419in}{3.466231in}}%
\pgfpathlineto{\pgfqpoint{2.871485in}{3.452851in}}%
\pgfpathlineto{\pgfqpoint{2.982382in}{3.516877in}}%
\pgfpathlineto{\pgfqpoint{3.046316in}{3.530257in}}%
\pgfpathclose%
\pgfusepath{fill}%
\end{pgfscope}%
\begin{pgfscope}%
\pgfpathrectangle{\pgfqpoint{0.765000in}{0.660000in}}{\pgfqpoint{4.620000in}{4.620000in}}%
\pgfusepath{clip}%
\pgfsetbuttcap%
\pgfsetroundjoin%
\definecolor{currentfill}{rgb}{1.000000,0.894118,0.788235}%
\pgfsetfillcolor{currentfill}%
\pgfsetlinewidth{0.000000pt}%
\definecolor{currentstroke}{rgb}{1.000000,0.894118,0.788235}%
\pgfsetstrokecolor{currentstroke}%
\pgfsetdash{}{0pt}%
\pgfpathmoveto{\pgfqpoint{3.157212in}{3.466231in}}%
\pgfpathlineto{\pgfqpoint{3.046316in}{3.530257in}}%
\pgfpathlineto{\pgfqpoint{2.982382in}{3.516877in}}%
\pgfpathlineto{\pgfqpoint{3.093279in}{3.452851in}}%
\pgfpathlineto{\pgfqpoint{3.157212in}{3.466231in}}%
\pgfpathclose%
\pgfusepath{fill}%
\end{pgfscope}%
\begin{pgfscope}%
\pgfpathrectangle{\pgfqpoint{0.765000in}{0.660000in}}{\pgfqpoint{4.620000in}{4.620000in}}%
\pgfusepath{clip}%
\pgfsetbuttcap%
\pgfsetroundjoin%
\definecolor{currentfill}{rgb}{1.000000,0.894118,0.788235}%
\pgfsetfillcolor{currentfill}%
\pgfsetlinewidth{0.000000pt}%
\definecolor{currentstroke}{rgb}{1.000000,0.894118,0.788235}%
\pgfsetstrokecolor{currentstroke}%
\pgfsetdash{}{0pt}%
\pgfpathmoveto{\pgfqpoint{3.046316in}{3.530257in}}%
\pgfpathlineto{\pgfqpoint{3.046316in}{3.658309in}}%
\pgfpathlineto{\pgfqpoint{2.982382in}{3.644929in}}%
\pgfpathlineto{\pgfqpoint{3.093279in}{3.452851in}}%
\pgfpathlineto{\pgfqpoint{3.046316in}{3.530257in}}%
\pgfpathclose%
\pgfusepath{fill}%
\end{pgfscope}%
\begin{pgfscope}%
\pgfpathrectangle{\pgfqpoint{0.765000in}{0.660000in}}{\pgfqpoint{4.620000in}{4.620000in}}%
\pgfusepath{clip}%
\pgfsetbuttcap%
\pgfsetroundjoin%
\definecolor{currentfill}{rgb}{1.000000,0.894118,0.788235}%
\pgfsetfillcolor{currentfill}%
\pgfsetlinewidth{0.000000pt}%
\definecolor{currentstroke}{rgb}{1.000000,0.894118,0.788235}%
\pgfsetstrokecolor{currentstroke}%
\pgfsetdash{}{0pt}%
\pgfpathmoveto{\pgfqpoint{3.157212in}{3.466231in}}%
\pgfpathlineto{\pgfqpoint{3.157212in}{3.594283in}}%
\pgfpathlineto{\pgfqpoint{3.093279in}{3.580903in}}%
\pgfpathlineto{\pgfqpoint{2.982382in}{3.516877in}}%
\pgfpathlineto{\pgfqpoint{3.157212in}{3.466231in}}%
\pgfpathclose%
\pgfusepath{fill}%
\end{pgfscope}%
\begin{pgfscope}%
\pgfpathrectangle{\pgfqpoint{0.765000in}{0.660000in}}{\pgfqpoint{4.620000in}{4.620000in}}%
\pgfusepath{clip}%
\pgfsetbuttcap%
\pgfsetroundjoin%
\definecolor{currentfill}{rgb}{1.000000,0.894118,0.788235}%
\pgfsetfillcolor{currentfill}%
\pgfsetlinewidth{0.000000pt}%
\definecolor{currentstroke}{rgb}{1.000000,0.894118,0.788235}%
\pgfsetstrokecolor{currentstroke}%
\pgfsetdash{}{0pt}%
\pgfpathmoveto{\pgfqpoint{2.935419in}{3.466231in}}%
\pgfpathlineto{\pgfqpoint{3.046316in}{3.402204in}}%
\pgfpathlineto{\pgfqpoint{2.982382in}{3.388824in}}%
\pgfpathlineto{\pgfqpoint{2.871485in}{3.452851in}}%
\pgfpathlineto{\pgfqpoint{2.935419in}{3.466231in}}%
\pgfpathclose%
\pgfusepath{fill}%
\end{pgfscope}%
\begin{pgfscope}%
\pgfpathrectangle{\pgfqpoint{0.765000in}{0.660000in}}{\pgfqpoint{4.620000in}{4.620000in}}%
\pgfusepath{clip}%
\pgfsetbuttcap%
\pgfsetroundjoin%
\definecolor{currentfill}{rgb}{1.000000,0.894118,0.788235}%
\pgfsetfillcolor{currentfill}%
\pgfsetlinewidth{0.000000pt}%
\definecolor{currentstroke}{rgb}{1.000000,0.894118,0.788235}%
\pgfsetstrokecolor{currentstroke}%
\pgfsetdash{}{0pt}%
\pgfpathmoveto{\pgfqpoint{3.046316in}{3.402204in}}%
\pgfpathlineto{\pgfqpoint{3.157212in}{3.466231in}}%
\pgfpathlineto{\pgfqpoint{3.093279in}{3.452851in}}%
\pgfpathlineto{\pgfqpoint{2.982382in}{3.388824in}}%
\pgfpathlineto{\pgfqpoint{3.046316in}{3.402204in}}%
\pgfpathclose%
\pgfusepath{fill}%
\end{pgfscope}%
\begin{pgfscope}%
\pgfpathrectangle{\pgfqpoint{0.765000in}{0.660000in}}{\pgfqpoint{4.620000in}{4.620000in}}%
\pgfusepath{clip}%
\pgfsetbuttcap%
\pgfsetroundjoin%
\definecolor{currentfill}{rgb}{1.000000,0.894118,0.788235}%
\pgfsetfillcolor{currentfill}%
\pgfsetlinewidth{0.000000pt}%
\definecolor{currentstroke}{rgb}{1.000000,0.894118,0.788235}%
\pgfsetstrokecolor{currentstroke}%
\pgfsetdash{}{0pt}%
\pgfpathmoveto{\pgfqpoint{3.046316in}{3.658309in}}%
\pgfpathlineto{\pgfqpoint{2.935419in}{3.594283in}}%
\pgfpathlineto{\pgfqpoint{2.871485in}{3.580903in}}%
\pgfpathlineto{\pgfqpoint{2.982382in}{3.644929in}}%
\pgfpathlineto{\pgfqpoint{3.046316in}{3.658309in}}%
\pgfpathclose%
\pgfusepath{fill}%
\end{pgfscope}%
\begin{pgfscope}%
\pgfpathrectangle{\pgfqpoint{0.765000in}{0.660000in}}{\pgfqpoint{4.620000in}{4.620000in}}%
\pgfusepath{clip}%
\pgfsetbuttcap%
\pgfsetroundjoin%
\definecolor{currentfill}{rgb}{1.000000,0.894118,0.788235}%
\pgfsetfillcolor{currentfill}%
\pgfsetlinewidth{0.000000pt}%
\definecolor{currentstroke}{rgb}{1.000000,0.894118,0.788235}%
\pgfsetstrokecolor{currentstroke}%
\pgfsetdash{}{0pt}%
\pgfpathmoveto{\pgfqpoint{3.157212in}{3.594283in}}%
\pgfpathlineto{\pgfqpoint{3.046316in}{3.658309in}}%
\pgfpathlineto{\pgfqpoint{2.982382in}{3.644929in}}%
\pgfpathlineto{\pgfqpoint{3.093279in}{3.580903in}}%
\pgfpathlineto{\pgfqpoint{3.157212in}{3.594283in}}%
\pgfpathclose%
\pgfusepath{fill}%
\end{pgfscope}%
\begin{pgfscope}%
\pgfpathrectangle{\pgfqpoint{0.765000in}{0.660000in}}{\pgfqpoint{4.620000in}{4.620000in}}%
\pgfusepath{clip}%
\pgfsetbuttcap%
\pgfsetroundjoin%
\definecolor{currentfill}{rgb}{1.000000,0.894118,0.788235}%
\pgfsetfillcolor{currentfill}%
\pgfsetlinewidth{0.000000pt}%
\definecolor{currentstroke}{rgb}{1.000000,0.894118,0.788235}%
\pgfsetstrokecolor{currentstroke}%
\pgfsetdash{}{0pt}%
\pgfpathmoveto{\pgfqpoint{2.935419in}{3.466231in}}%
\pgfpathlineto{\pgfqpoint{2.935419in}{3.594283in}}%
\pgfpathlineto{\pgfqpoint{2.871485in}{3.580903in}}%
\pgfpathlineto{\pgfqpoint{2.982382in}{3.388824in}}%
\pgfpathlineto{\pgfqpoint{2.935419in}{3.466231in}}%
\pgfpathclose%
\pgfusepath{fill}%
\end{pgfscope}%
\begin{pgfscope}%
\pgfpathrectangle{\pgfqpoint{0.765000in}{0.660000in}}{\pgfqpoint{4.620000in}{4.620000in}}%
\pgfusepath{clip}%
\pgfsetbuttcap%
\pgfsetroundjoin%
\definecolor{currentfill}{rgb}{1.000000,0.894118,0.788235}%
\pgfsetfillcolor{currentfill}%
\pgfsetlinewidth{0.000000pt}%
\definecolor{currentstroke}{rgb}{1.000000,0.894118,0.788235}%
\pgfsetstrokecolor{currentstroke}%
\pgfsetdash{}{0pt}%
\pgfpathmoveto{\pgfqpoint{3.046316in}{3.402204in}}%
\pgfpathlineto{\pgfqpoint{3.046316in}{3.530257in}}%
\pgfpathlineto{\pgfqpoint{2.982382in}{3.516877in}}%
\pgfpathlineto{\pgfqpoint{2.871485in}{3.452851in}}%
\pgfpathlineto{\pgfqpoint{3.046316in}{3.402204in}}%
\pgfpathclose%
\pgfusepath{fill}%
\end{pgfscope}%
\begin{pgfscope}%
\pgfpathrectangle{\pgfqpoint{0.765000in}{0.660000in}}{\pgfqpoint{4.620000in}{4.620000in}}%
\pgfusepath{clip}%
\pgfsetbuttcap%
\pgfsetroundjoin%
\definecolor{currentfill}{rgb}{1.000000,0.894118,0.788235}%
\pgfsetfillcolor{currentfill}%
\pgfsetlinewidth{0.000000pt}%
\definecolor{currentstroke}{rgb}{1.000000,0.894118,0.788235}%
\pgfsetstrokecolor{currentstroke}%
\pgfsetdash{}{0pt}%
\pgfpathmoveto{\pgfqpoint{3.046316in}{3.530257in}}%
\pgfpathlineto{\pgfqpoint{3.157212in}{3.594283in}}%
\pgfpathlineto{\pgfqpoint{3.093279in}{3.580903in}}%
\pgfpathlineto{\pgfqpoint{2.982382in}{3.516877in}}%
\pgfpathlineto{\pgfqpoint{3.046316in}{3.530257in}}%
\pgfpathclose%
\pgfusepath{fill}%
\end{pgfscope}%
\begin{pgfscope}%
\pgfpathrectangle{\pgfqpoint{0.765000in}{0.660000in}}{\pgfqpoint{4.620000in}{4.620000in}}%
\pgfusepath{clip}%
\pgfsetbuttcap%
\pgfsetroundjoin%
\definecolor{currentfill}{rgb}{1.000000,0.894118,0.788235}%
\pgfsetfillcolor{currentfill}%
\pgfsetlinewidth{0.000000pt}%
\definecolor{currentstroke}{rgb}{1.000000,0.894118,0.788235}%
\pgfsetstrokecolor{currentstroke}%
\pgfsetdash{}{0pt}%
\pgfpathmoveto{\pgfqpoint{2.935419in}{3.594283in}}%
\pgfpathlineto{\pgfqpoint{3.046316in}{3.530257in}}%
\pgfpathlineto{\pgfqpoint{2.982382in}{3.516877in}}%
\pgfpathlineto{\pgfqpoint{2.871485in}{3.580903in}}%
\pgfpathlineto{\pgfqpoint{2.935419in}{3.594283in}}%
\pgfpathclose%
\pgfusepath{fill}%
\end{pgfscope}%
\begin{pgfscope}%
\pgfpathrectangle{\pgfqpoint{0.765000in}{0.660000in}}{\pgfqpoint{4.620000in}{4.620000in}}%
\pgfusepath{clip}%
\pgfsetbuttcap%
\pgfsetroundjoin%
\definecolor{currentfill}{rgb}{1.000000,0.894118,0.788235}%
\pgfsetfillcolor{currentfill}%
\pgfsetlinewidth{0.000000pt}%
\definecolor{currentstroke}{rgb}{1.000000,0.894118,0.788235}%
\pgfsetstrokecolor{currentstroke}%
\pgfsetdash{}{0pt}%
\pgfpathmoveto{\pgfqpoint{3.242912in}{2.518793in}}%
\pgfpathlineto{\pgfqpoint{3.132016in}{2.454767in}}%
\pgfpathlineto{\pgfqpoint{3.242912in}{2.390740in}}%
\pgfpathlineto{\pgfqpoint{3.353809in}{2.454767in}}%
\pgfpathlineto{\pgfqpoint{3.242912in}{2.518793in}}%
\pgfpathclose%
\pgfusepath{fill}%
\end{pgfscope}%
\begin{pgfscope}%
\pgfpathrectangle{\pgfqpoint{0.765000in}{0.660000in}}{\pgfqpoint{4.620000in}{4.620000in}}%
\pgfusepath{clip}%
\pgfsetbuttcap%
\pgfsetroundjoin%
\definecolor{currentfill}{rgb}{1.000000,0.894118,0.788235}%
\pgfsetfillcolor{currentfill}%
\pgfsetlinewidth{0.000000pt}%
\definecolor{currentstroke}{rgb}{1.000000,0.894118,0.788235}%
\pgfsetstrokecolor{currentstroke}%
\pgfsetdash{}{0pt}%
\pgfpathmoveto{\pgfqpoint{3.242912in}{2.518793in}}%
\pgfpathlineto{\pgfqpoint{3.132016in}{2.454767in}}%
\pgfpathlineto{\pgfqpoint{3.132016in}{2.582819in}}%
\pgfpathlineto{\pgfqpoint{3.242912in}{2.646845in}}%
\pgfpathlineto{\pgfqpoint{3.242912in}{2.518793in}}%
\pgfpathclose%
\pgfusepath{fill}%
\end{pgfscope}%
\begin{pgfscope}%
\pgfpathrectangle{\pgfqpoint{0.765000in}{0.660000in}}{\pgfqpoint{4.620000in}{4.620000in}}%
\pgfusepath{clip}%
\pgfsetbuttcap%
\pgfsetroundjoin%
\definecolor{currentfill}{rgb}{1.000000,0.894118,0.788235}%
\pgfsetfillcolor{currentfill}%
\pgfsetlinewidth{0.000000pt}%
\definecolor{currentstroke}{rgb}{1.000000,0.894118,0.788235}%
\pgfsetstrokecolor{currentstroke}%
\pgfsetdash{}{0pt}%
\pgfpathmoveto{\pgfqpoint{3.242912in}{2.518793in}}%
\pgfpathlineto{\pgfqpoint{3.353809in}{2.454767in}}%
\pgfpathlineto{\pgfqpoint{3.353809in}{2.582819in}}%
\pgfpathlineto{\pgfqpoint{3.242912in}{2.646845in}}%
\pgfpathlineto{\pgfqpoint{3.242912in}{2.518793in}}%
\pgfpathclose%
\pgfusepath{fill}%
\end{pgfscope}%
\begin{pgfscope}%
\pgfpathrectangle{\pgfqpoint{0.765000in}{0.660000in}}{\pgfqpoint{4.620000in}{4.620000in}}%
\pgfusepath{clip}%
\pgfsetbuttcap%
\pgfsetroundjoin%
\definecolor{currentfill}{rgb}{1.000000,0.894118,0.788235}%
\pgfsetfillcolor{currentfill}%
\pgfsetlinewidth{0.000000pt}%
\definecolor{currentstroke}{rgb}{1.000000,0.894118,0.788235}%
\pgfsetstrokecolor{currentstroke}%
\pgfsetdash{}{0pt}%
\pgfpathmoveto{\pgfqpoint{3.186392in}{2.493495in}}%
\pgfpathlineto{\pgfqpoint{3.075496in}{2.429469in}}%
\pgfpathlineto{\pgfqpoint{3.186392in}{2.365442in}}%
\pgfpathlineto{\pgfqpoint{3.297289in}{2.429469in}}%
\pgfpathlineto{\pgfqpoint{3.186392in}{2.493495in}}%
\pgfpathclose%
\pgfusepath{fill}%
\end{pgfscope}%
\begin{pgfscope}%
\pgfpathrectangle{\pgfqpoint{0.765000in}{0.660000in}}{\pgfqpoint{4.620000in}{4.620000in}}%
\pgfusepath{clip}%
\pgfsetbuttcap%
\pgfsetroundjoin%
\definecolor{currentfill}{rgb}{1.000000,0.894118,0.788235}%
\pgfsetfillcolor{currentfill}%
\pgfsetlinewidth{0.000000pt}%
\definecolor{currentstroke}{rgb}{1.000000,0.894118,0.788235}%
\pgfsetstrokecolor{currentstroke}%
\pgfsetdash{}{0pt}%
\pgfpathmoveto{\pgfqpoint{3.186392in}{2.493495in}}%
\pgfpathlineto{\pgfqpoint{3.075496in}{2.429469in}}%
\pgfpathlineto{\pgfqpoint{3.075496in}{2.557521in}}%
\pgfpathlineto{\pgfqpoint{3.186392in}{2.621547in}}%
\pgfpathlineto{\pgfqpoint{3.186392in}{2.493495in}}%
\pgfpathclose%
\pgfusepath{fill}%
\end{pgfscope}%
\begin{pgfscope}%
\pgfpathrectangle{\pgfqpoint{0.765000in}{0.660000in}}{\pgfqpoint{4.620000in}{4.620000in}}%
\pgfusepath{clip}%
\pgfsetbuttcap%
\pgfsetroundjoin%
\definecolor{currentfill}{rgb}{1.000000,0.894118,0.788235}%
\pgfsetfillcolor{currentfill}%
\pgfsetlinewidth{0.000000pt}%
\definecolor{currentstroke}{rgb}{1.000000,0.894118,0.788235}%
\pgfsetstrokecolor{currentstroke}%
\pgfsetdash{}{0pt}%
\pgfpathmoveto{\pgfqpoint{3.186392in}{2.493495in}}%
\pgfpathlineto{\pgfqpoint{3.297289in}{2.429469in}}%
\pgfpathlineto{\pgfqpoint{3.297289in}{2.557521in}}%
\pgfpathlineto{\pgfqpoint{3.186392in}{2.621547in}}%
\pgfpathlineto{\pgfqpoint{3.186392in}{2.493495in}}%
\pgfpathclose%
\pgfusepath{fill}%
\end{pgfscope}%
\begin{pgfscope}%
\pgfpathrectangle{\pgfqpoint{0.765000in}{0.660000in}}{\pgfqpoint{4.620000in}{4.620000in}}%
\pgfusepath{clip}%
\pgfsetbuttcap%
\pgfsetroundjoin%
\definecolor{currentfill}{rgb}{1.000000,0.894118,0.788235}%
\pgfsetfillcolor{currentfill}%
\pgfsetlinewidth{0.000000pt}%
\definecolor{currentstroke}{rgb}{1.000000,0.894118,0.788235}%
\pgfsetstrokecolor{currentstroke}%
\pgfsetdash{}{0pt}%
\pgfpathmoveto{\pgfqpoint{3.242912in}{2.646845in}}%
\pgfpathlineto{\pgfqpoint{3.132016in}{2.582819in}}%
\pgfpathlineto{\pgfqpoint{3.242912in}{2.518793in}}%
\pgfpathlineto{\pgfqpoint{3.353809in}{2.582819in}}%
\pgfpathlineto{\pgfqpoint{3.242912in}{2.646845in}}%
\pgfpathclose%
\pgfusepath{fill}%
\end{pgfscope}%
\begin{pgfscope}%
\pgfpathrectangle{\pgfqpoint{0.765000in}{0.660000in}}{\pgfqpoint{4.620000in}{4.620000in}}%
\pgfusepath{clip}%
\pgfsetbuttcap%
\pgfsetroundjoin%
\definecolor{currentfill}{rgb}{1.000000,0.894118,0.788235}%
\pgfsetfillcolor{currentfill}%
\pgfsetlinewidth{0.000000pt}%
\definecolor{currentstroke}{rgb}{1.000000,0.894118,0.788235}%
\pgfsetstrokecolor{currentstroke}%
\pgfsetdash{}{0pt}%
\pgfpathmoveto{\pgfqpoint{3.242912in}{2.390740in}}%
\pgfpathlineto{\pgfqpoint{3.353809in}{2.454767in}}%
\pgfpathlineto{\pgfqpoint{3.353809in}{2.582819in}}%
\pgfpathlineto{\pgfqpoint{3.242912in}{2.518793in}}%
\pgfpathlineto{\pgfqpoint{3.242912in}{2.390740in}}%
\pgfpathclose%
\pgfusepath{fill}%
\end{pgfscope}%
\begin{pgfscope}%
\pgfpathrectangle{\pgfqpoint{0.765000in}{0.660000in}}{\pgfqpoint{4.620000in}{4.620000in}}%
\pgfusepath{clip}%
\pgfsetbuttcap%
\pgfsetroundjoin%
\definecolor{currentfill}{rgb}{1.000000,0.894118,0.788235}%
\pgfsetfillcolor{currentfill}%
\pgfsetlinewidth{0.000000pt}%
\definecolor{currentstroke}{rgb}{1.000000,0.894118,0.788235}%
\pgfsetstrokecolor{currentstroke}%
\pgfsetdash{}{0pt}%
\pgfpathmoveto{\pgfqpoint{3.132016in}{2.454767in}}%
\pgfpathlineto{\pgfqpoint{3.242912in}{2.390740in}}%
\pgfpathlineto{\pgfqpoint{3.242912in}{2.518793in}}%
\pgfpathlineto{\pgfqpoint{3.132016in}{2.582819in}}%
\pgfpathlineto{\pgfqpoint{3.132016in}{2.454767in}}%
\pgfpathclose%
\pgfusepath{fill}%
\end{pgfscope}%
\begin{pgfscope}%
\pgfpathrectangle{\pgfqpoint{0.765000in}{0.660000in}}{\pgfqpoint{4.620000in}{4.620000in}}%
\pgfusepath{clip}%
\pgfsetbuttcap%
\pgfsetroundjoin%
\definecolor{currentfill}{rgb}{1.000000,0.894118,0.788235}%
\pgfsetfillcolor{currentfill}%
\pgfsetlinewidth{0.000000pt}%
\definecolor{currentstroke}{rgb}{1.000000,0.894118,0.788235}%
\pgfsetstrokecolor{currentstroke}%
\pgfsetdash{}{0pt}%
\pgfpathmoveto{\pgfqpoint{3.186392in}{2.621547in}}%
\pgfpathlineto{\pgfqpoint{3.075496in}{2.557521in}}%
\pgfpathlineto{\pgfqpoint{3.186392in}{2.493495in}}%
\pgfpathlineto{\pgfqpoint{3.297289in}{2.557521in}}%
\pgfpathlineto{\pgfqpoint{3.186392in}{2.621547in}}%
\pgfpathclose%
\pgfusepath{fill}%
\end{pgfscope}%
\begin{pgfscope}%
\pgfpathrectangle{\pgfqpoint{0.765000in}{0.660000in}}{\pgfqpoint{4.620000in}{4.620000in}}%
\pgfusepath{clip}%
\pgfsetbuttcap%
\pgfsetroundjoin%
\definecolor{currentfill}{rgb}{1.000000,0.894118,0.788235}%
\pgfsetfillcolor{currentfill}%
\pgfsetlinewidth{0.000000pt}%
\definecolor{currentstroke}{rgb}{1.000000,0.894118,0.788235}%
\pgfsetstrokecolor{currentstroke}%
\pgfsetdash{}{0pt}%
\pgfpathmoveto{\pgfqpoint{3.186392in}{2.365442in}}%
\pgfpathlineto{\pgfqpoint{3.297289in}{2.429469in}}%
\pgfpathlineto{\pgfqpoint{3.297289in}{2.557521in}}%
\pgfpathlineto{\pgfqpoint{3.186392in}{2.493495in}}%
\pgfpathlineto{\pgfqpoint{3.186392in}{2.365442in}}%
\pgfpathclose%
\pgfusepath{fill}%
\end{pgfscope}%
\begin{pgfscope}%
\pgfpathrectangle{\pgfqpoint{0.765000in}{0.660000in}}{\pgfqpoint{4.620000in}{4.620000in}}%
\pgfusepath{clip}%
\pgfsetbuttcap%
\pgfsetroundjoin%
\definecolor{currentfill}{rgb}{1.000000,0.894118,0.788235}%
\pgfsetfillcolor{currentfill}%
\pgfsetlinewidth{0.000000pt}%
\definecolor{currentstroke}{rgb}{1.000000,0.894118,0.788235}%
\pgfsetstrokecolor{currentstroke}%
\pgfsetdash{}{0pt}%
\pgfpathmoveto{\pgfqpoint{3.075496in}{2.429469in}}%
\pgfpathlineto{\pgfqpoint{3.186392in}{2.365442in}}%
\pgfpathlineto{\pgfqpoint{3.186392in}{2.493495in}}%
\pgfpathlineto{\pgfqpoint{3.075496in}{2.557521in}}%
\pgfpathlineto{\pgfqpoint{3.075496in}{2.429469in}}%
\pgfpathclose%
\pgfusepath{fill}%
\end{pgfscope}%
\begin{pgfscope}%
\pgfpathrectangle{\pgfqpoint{0.765000in}{0.660000in}}{\pgfqpoint{4.620000in}{4.620000in}}%
\pgfusepath{clip}%
\pgfsetbuttcap%
\pgfsetroundjoin%
\definecolor{currentfill}{rgb}{1.000000,0.894118,0.788235}%
\pgfsetfillcolor{currentfill}%
\pgfsetlinewidth{0.000000pt}%
\definecolor{currentstroke}{rgb}{1.000000,0.894118,0.788235}%
\pgfsetstrokecolor{currentstroke}%
\pgfsetdash{}{0pt}%
\pgfpathmoveto{\pgfqpoint{3.242912in}{2.518793in}}%
\pgfpathlineto{\pgfqpoint{3.132016in}{2.454767in}}%
\pgfpathlineto{\pgfqpoint{3.075496in}{2.429469in}}%
\pgfpathlineto{\pgfqpoint{3.186392in}{2.493495in}}%
\pgfpathlineto{\pgfqpoint{3.242912in}{2.518793in}}%
\pgfpathclose%
\pgfusepath{fill}%
\end{pgfscope}%
\begin{pgfscope}%
\pgfpathrectangle{\pgfqpoint{0.765000in}{0.660000in}}{\pgfqpoint{4.620000in}{4.620000in}}%
\pgfusepath{clip}%
\pgfsetbuttcap%
\pgfsetroundjoin%
\definecolor{currentfill}{rgb}{1.000000,0.894118,0.788235}%
\pgfsetfillcolor{currentfill}%
\pgfsetlinewidth{0.000000pt}%
\definecolor{currentstroke}{rgb}{1.000000,0.894118,0.788235}%
\pgfsetstrokecolor{currentstroke}%
\pgfsetdash{}{0pt}%
\pgfpathmoveto{\pgfqpoint{3.353809in}{2.454767in}}%
\pgfpathlineto{\pgfqpoint{3.242912in}{2.518793in}}%
\pgfpathlineto{\pgfqpoint{3.186392in}{2.493495in}}%
\pgfpathlineto{\pgfqpoint{3.297289in}{2.429469in}}%
\pgfpathlineto{\pgfqpoint{3.353809in}{2.454767in}}%
\pgfpathclose%
\pgfusepath{fill}%
\end{pgfscope}%
\begin{pgfscope}%
\pgfpathrectangle{\pgfqpoint{0.765000in}{0.660000in}}{\pgfqpoint{4.620000in}{4.620000in}}%
\pgfusepath{clip}%
\pgfsetbuttcap%
\pgfsetroundjoin%
\definecolor{currentfill}{rgb}{1.000000,0.894118,0.788235}%
\pgfsetfillcolor{currentfill}%
\pgfsetlinewidth{0.000000pt}%
\definecolor{currentstroke}{rgb}{1.000000,0.894118,0.788235}%
\pgfsetstrokecolor{currentstroke}%
\pgfsetdash{}{0pt}%
\pgfpathmoveto{\pgfqpoint{3.242912in}{2.518793in}}%
\pgfpathlineto{\pgfqpoint{3.242912in}{2.646845in}}%
\pgfpathlineto{\pgfqpoint{3.186392in}{2.621547in}}%
\pgfpathlineto{\pgfqpoint{3.297289in}{2.429469in}}%
\pgfpathlineto{\pgfqpoint{3.242912in}{2.518793in}}%
\pgfpathclose%
\pgfusepath{fill}%
\end{pgfscope}%
\begin{pgfscope}%
\pgfpathrectangle{\pgfqpoint{0.765000in}{0.660000in}}{\pgfqpoint{4.620000in}{4.620000in}}%
\pgfusepath{clip}%
\pgfsetbuttcap%
\pgfsetroundjoin%
\definecolor{currentfill}{rgb}{1.000000,0.894118,0.788235}%
\pgfsetfillcolor{currentfill}%
\pgfsetlinewidth{0.000000pt}%
\definecolor{currentstroke}{rgb}{1.000000,0.894118,0.788235}%
\pgfsetstrokecolor{currentstroke}%
\pgfsetdash{}{0pt}%
\pgfpathmoveto{\pgfqpoint{3.353809in}{2.454767in}}%
\pgfpathlineto{\pgfqpoint{3.353809in}{2.582819in}}%
\pgfpathlineto{\pgfqpoint{3.297289in}{2.557521in}}%
\pgfpathlineto{\pgfqpoint{3.186392in}{2.493495in}}%
\pgfpathlineto{\pgfqpoint{3.353809in}{2.454767in}}%
\pgfpathclose%
\pgfusepath{fill}%
\end{pgfscope}%
\begin{pgfscope}%
\pgfpathrectangle{\pgfqpoint{0.765000in}{0.660000in}}{\pgfqpoint{4.620000in}{4.620000in}}%
\pgfusepath{clip}%
\pgfsetbuttcap%
\pgfsetroundjoin%
\definecolor{currentfill}{rgb}{1.000000,0.894118,0.788235}%
\pgfsetfillcolor{currentfill}%
\pgfsetlinewidth{0.000000pt}%
\definecolor{currentstroke}{rgb}{1.000000,0.894118,0.788235}%
\pgfsetstrokecolor{currentstroke}%
\pgfsetdash{}{0pt}%
\pgfpathmoveto{\pgfqpoint{3.132016in}{2.454767in}}%
\pgfpathlineto{\pgfqpoint{3.242912in}{2.390740in}}%
\pgfpathlineto{\pgfqpoint{3.186392in}{2.365442in}}%
\pgfpathlineto{\pgfqpoint{3.075496in}{2.429469in}}%
\pgfpathlineto{\pgfqpoint{3.132016in}{2.454767in}}%
\pgfpathclose%
\pgfusepath{fill}%
\end{pgfscope}%
\begin{pgfscope}%
\pgfpathrectangle{\pgfqpoint{0.765000in}{0.660000in}}{\pgfqpoint{4.620000in}{4.620000in}}%
\pgfusepath{clip}%
\pgfsetbuttcap%
\pgfsetroundjoin%
\definecolor{currentfill}{rgb}{1.000000,0.894118,0.788235}%
\pgfsetfillcolor{currentfill}%
\pgfsetlinewidth{0.000000pt}%
\definecolor{currentstroke}{rgb}{1.000000,0.894118,0.788235}%
\pgfsetstrokecolor{currentstroke}%
\pgfsetdash{}{0pt}%
\pgfpathmoveto{\pgfqpoint{3.242912in}{2.390740in}}%
\pgfpathlineto{\pgfqpoint{3.353809in}{2.454767in}}%
\pgfpathlineto{\pgfqpoint{3.297289in}{2.429469in}}%
\pgfpathlineto{\pgfqpoint{3.186392in}{2.365442in}}%
\pgfpathlineto{\pgfqpoint{3.242912in}{2.390740in}}%
\pgfpathclose%
\pgfusepath{fill}%
\end{pgfscope}%
\begin{pgfscope}%
\pgfpathrectangle{\pgfqpoint{0.765000in}{0.660000in}}{\pgfqpoint{4.620000in}{4.620000in}}%
\pgfusepath{clip}%
\pgfsetbuttcap%
\pgfsetroundjoin%
\definecolor{currentfill}{rgb}{1.000000,0.894118,0.788235}%
\pgfsetfillcolor{currentfill}%
\pgfsetlinewidth{0.000000pt}%
\definecolor{currentstroke}{rgb}{1.000000,0.894118,0.788235}%
\pgfsetstrokecolor{currentstroke}%
\pgfsetdash{}{0pt}%
\pgfpathmoveto{\pgfqpoint{3.242912in}{2.646845in}}%
\pgfpathlineto{\pgfqpoint{3.132016in}{2.582819in}}%
\pgfpathlineto{\pgfqpoint{3.075496in}{2.557521in}}%
\pgfpathlineto{\pgfqpoint{3.186392in}{2.621547in}}%
\pgfpathlineto{\pgfqpoint{3.242912in}{2.646845in}}%
\pgfpathclose%
\pgfusepath{fill}%
\end{pgfscope}%
\begin{pgfscope}%
\pgfpathrectangle{\pgfqpoint{0.765000in}{0.660000in}}{\pgfqpoint{4.620000in}{4.620000in}}%
\pgfusepath{clip}%
\pgfsetbuttcap%
\pgfsetroundjoin%
\definecolor{currentfill}{rgb}{1.000000,0.894118,0.788235}%
\pgfsetfillcolor{currentfill}%
\pgfsetlinewidth{0.000000pt}%
\definecolor{currentstroke}{rgb}{1.000000,0.894118,0.788235}%
\pgfsetstrokecolor{currentstroke}%
\pgfsetdash{}{0pt}%
\pgfpathmoveto{\pgfqpoint{3.353809in}{2.582819in}}%
\pgfpathlineto{\pgfqpoint{3.242912in}{2.646845in}}%
\pgfpathlineto{\pgfqpoint{3.186392in}{2.621547in}}%
\pgfpathlineto{\pgfqpoint{3.297289in}{2.557521in}}%
\pgfpathlineto{\pgfqpoint{3.353809in}{2.582819in}}%
\pgfpathclose%
\pgfusepath{fill}%
\end{pgfscope}%
\begin{pgfscope}%
\pgfpathrectangle{\pgfqpoint{0.765000in}{0.660000in}}{\pgfqpoint{4.620000in}{4.620000in}}%
\pgfusepath{clip}%
\pgfsetbuttcap%
\pgfsetroundjoin%
\definecolor{currentfill}{rgb}{1.000000,0.894118,0.788235}%
\pgfsetfillcolor{currentfill}%
\pgfsetlinewidth{0.000000pt}%
\definecolor{currentstroke}{rgb}{1.000000,0.894118,0.788235}%
\pgfsetstrokecolor{currentstroke}%
\pgfsetdash{}{0pt}%
\pgfpathmoveto{\pgfqpoint{3.132016in}{2.454767in}}%
\pgfpathlineto{\pgfqpoint{3.132016in}{2.582819in}}%
\pgfpathlineto{\pgfqpoint{3.075496in}{2.557521in}}%
\pgfpathlineto{\pgfqpoint{3.186392in}{2.365442in}}%
\pgfpathlineto{\pgfqpoint{3.132016in}{2.454767in}}%
\pgfpathclose%
\pgfusepath{fill}%
\end{pgfscope}%
\begin{pgfscope}%
\pgfpathrectangle{\pgfqpoint{0.765000in}{0.660000in}}{\pgfqpoint{4.620000in}{4.620000in}}%
\pgfusepath{clip}%
\pgfsetbuttcap%
\pgfsetroundjoin%
\definecolor{currentfill}{rgb}{1.000000,0.894118,0.788235}%
\pgfsetfillcolor{currentfill}%
\pgfsetlinewidth{0.000000pt}%
\definecolor{currentstroke}{rgb}{1.000000,0.894118,0.788235}%
\pgfsetstrokecolor{currentstroke}%
\pgfsetdash{}{0pt}%
\pgfpathmoveto{\pgfqpoint{3.242912in}{2.390740in}}%
\pgfpathlineto{\pgfqpoint{3.242912in}{2.518793in}}%
\pgfpathlineto{\pgfqpoint{3.186392in}{2.493495in}}%
\pgfpathlineto{\pgfqpoint{3.075496in}{2.429469in}}%
\pgfpathlineto{\pgfqpoint{3.242912in}{2.390740in}}%
\pgfpathclose%
\pgfusepath{fill}%
\end{pgfscope}%
\begin{pgfscope}%
\pgfpathrectangle{\pgfqpoint{0.765000in}{0.660000in}}{\pgfqpoint{4.620000in}{4.620000in}}%
\pgfusepath{clip}%
\pgfsetbuttcap%
\pgfsetroundjoin%
\definecolor{currentfill}{rgb}{1.000000,0.894118,0.788235}%
\pgfsetfillcolor{currentfill}%
\pgfsetlinewidth{0.000000pt}%
\definecolor{currentstroke}{rgb}{1.000000,0.894118,0.788235}%
\pgfsetstrokecolor{currentstroke}%
\pgfsetdash{}{0pt}%
\pgfpathmoveto{\pgfqpoint{3.242912in}{2.518793in}}%
\pgfpathlineto{\pgfqpoint{3.353809in}{2.582819in}}%
\pgfpathlineto{\pgfqpoint{3.297289in}{2.557521in}}%
\pgfpathlineto{\pgfqpoint{3.186392in}{2.493495in}}%
\pgfpathlineto{\pgfqpoint{3.242912in}{2.518793in}}%
\pgfpathclose%
\pgfusepath{fill}%
\end{pgfscope}%
\begin{pgfscope}%
\pgfpathrectangle{\pgfqpoint{0.765000in}{0.660000in}}{\pgfqpoint{4.620000in}{4.620000in}}%
\pgfusepath{clip}%
\pgfsetbuttcap%
\pgfsetroundjoin%
\definecolor{currentfill}{rgb}{1.000000,0.894118,0.788235}%
\pgfsetfillcolor{currentfill}%
\pgfsetlinewidth{0.000000pt}%
\definecolor{currentstroke}{rgb}{1.000000,0.894118,0.788235}%
\pgfsetstrokecolor{currentstroke}%
\pgfsetdash{}{0pt}%
\pgfpathmoveto{\pgfqpoint{3.132016in}{2.582819in}}%
\pgfpathlineto{\pgfqpoint{3.242912in}{2.518793in}}%
\pgfpathlineto{\pgfqpoint{3.186392in}{2.493495in}}%
\pgfpathlineto{\pgfqpoint{3.075496in}{2.557521in}}%
\pgfpathlineto{\pgfqpoint{3.132016in}{2.582819in}}%
\pgfpathclose%
\pgfusepath{fill}%
\end{pgfscope}%
\begin{pgfscope}%
\pgfpathrectangle{\pgfqpoint{0.765000in}{0.660000in}}{\pgfqpoint{4.620000in}{4.620000in}}%
\pgfusepath{clip}%
\pgfsetbuttcap%
\pgfsetroundjoin%
\definecolor{currentfill}{rgb}{1.000000,0.894118,0.788235}%
\pgfsetfillcolor{currentfill}%
\pgfsetlinewidth{0.000000pt}%
\definecolor{currentstroke}{rgb}{1.000000,0.894118,0.788235}%
\pgfsetstrokecolor{currentstroke}%
\pgfsetdash{}{0pt}%
\pgfpathmoveto{\pgfqpoint{3.242912in}{2.518793in}}%
\pgfpathlineto{\pgfqpoint{3.132016in}{2.454767in}}%
\pgfpathlineto{\pgfqpoint{3.242912in}{2.390740in}}%
\pgfpathlineto{\pgfqpoint{3.353809in}{2.454767in}}%
\pgfpathlineto{\pgfqpoint{3.242912in}{2.518793in}}%
\pgfpathclose%
\pgfusepath{fill}%
\end{pgfscope}%
\begin{pgfscope}%
\pgfpathrectangle{\pgfqpoint{0.765000in}{0.660000in}}{\pgfqpoint{4.620000in}{4.620000in}}%
\pgfusepath{clip}%
\pgfsetbuttcap%
\pgfsetroundjoin%
\definecolor{currentfill}{rgb}{1.000000,0.894118,0.788235}%
\pgfsetfillcolor{currentfill}%
\pgfsetlinewidth{0.000000pt}%
\definecolor{currentstroke}{rgb}{1.000000,0.894118,0.788235}%
\pgfsetstrokecolor{currentstroke}%
\pgfsetdash{}{0pt}%
\pgfpathmoveto{\pgfqpoint{3.242912in}{2.518793in}}%
\pgfpathlineto{\pgfqpoint{3.132016in}{2.454767in}}%
\pgfpathlineto{\pgfqpoint{3.132016in}{2.582819in}}%
\pgfpathlineto{\pgfqpoint{3.242912in}{2.646845in}}%
\pgfpathlineto{\pgfqpoint{3.242912in}{2.518793in}}%
\pgfpathclose%
\pgfusepath{fill}%
\end{pgfscope}%
\begin{pgfscope}%
\pgfpathrectangle{\pgfqpoint{0.765000in}{0.660000in}}{\pgfqpoint{4.620000in}{4.620000in}}%
\pgfusepath{clip}%
\pgfsetbuttcap%
\pgfsetroundjoin%
\definecolor{currentfill}{rgb}{1.000000,0.894118,0.788235}%
\pgfsetfillcolor{currentfill}%
\pgfsetlinewidth{0.000000pt}%
\definecolor{currentstroke}{rgb}{1.000000,0.894118,0.788235}%
\pgfsetstrokecolor{currentstroke}%
\pgfsetdash{}{0pt}%
\pgfpathmoveto{\pgfqpoint{3.242912in}{2.518793in}}%
\pgfpathlineto{\pgfqpoint{3.353809in}{2.454767in}}%
\pgfpathlineto{\pgfqpoint{3.353809in}{2.582819in}}%
\pgfpathlineto{\pgfqpoint{3.242912in}{2.646845in}}%
\pgfpathlineto{\pgfqpoint{3.242912in}{2.518793in}}%
\pgfpathclose%
\pgfusepath{fill}%
\end{pgfscope}%
\begin{pgfscope}%
\pgfpathrectangle{\pgfqpoint{0.765000in}{0.660000in}}{\pgfqpoint{4.620000in}{4.620000in}}%
\pgfusepath{clip}%
\pgfsetbuttcap%
\pgfsetroundjoin%
\definecolor{currentfill}{rgb}{1.000000,0.894118,0.788235}%
\pgfsetfillcolor{currentfill}%
\pgfsetlinewidth{0.000000pt}%
\definecolor{currentstroke}{rgb}{1.000000,0.894118,0.788235}%
\pgfsetstrokecolor{currentstroke}%
\pgfsetdash{}{0pt}%
\pgfpathmoveto{\pgfqpoint{3.293363in}{2.546452in}}%
\pgfpathlineto{\pgfqpoint{3.182466in}{2.482426in}}%
\pgfpathlineto{\pgfqpoint{3.293363in}{2.418400in}}%
\pgfpathlineto{\pgfqpoint{3.404259in}{2.482426in}}%
\pgfpathlineto{\pgfqpoint{3.293363in}{2.546452in}}%
\pgfpathclose%
\pgfusepath{fill}%
\end{pgfscope}%
\begin{pgfscope}%
\pgfpathrectangle{\pgfqpoint{0.765000in}{0.660000in}}{\pgfqpoint{4.620000in}{4.620000in}}%
\pgfusepath{clip}%
\pgfsetbuttcap%
\pgfsetroundjoin%
\definecolor{currentfill}{rgb}{1.000000,0.894118,0.788235}%
\pgfsetfillcolor{currentfill}%
\pgfsetlinewidth{0.000000pt}%
\definecolor{currentstroke}{rgb}{1.000000,0.894118,0.788235}%
\pgfsetstrokecolor{currentstroke}%
\pgfsetdash{}{0pt}%
\pgfpathmoveto{\pgfqpoint{3.293363in}{2.546452in}}%
\pgfpathlineto{\pgfqpoint{3.182466in}{2.482426in}}%
\pgfpathlineto{\pgfqpoint{3.182466in}{2.610478in}}%
\pgfpathlineto{\pgfqpoint{3.293363in}{2.674505in}}%
\pgfpathlineto{\pgfqpoint{3.293363in}{2.546452in}}%
\pgfpathclose%
\pgfusepath{fill}%
\end{pgfscope}%
\begin{pgfscope}%
\pgfpathrectangle{\pgfqpoint{0.765000in}{0.660000in}}{\pgfqpoint{4.620000in}{4.620000in}}%
\pgfusepath{clip}%
\pgfsetbuttcap%
\pgfsetroundjoin%
\definecolor{currentfill}{rgb}{1.000000,0.894118,0.788235}%
\pgfsetfillcolor{currentfill}%
\pgfsetlinewidth{0.000000pt}%
\definecolor{currentstroke}{rgb}{1.000000,0.894118,0.788235}%
\pgfsetstrokecolor{currentstroke}%
\pgfsetdash{}{0pt}%
\pgfpathmoveto{\pgfqpoint{3.293363in}{2.546452in}}%
\pgfpathlineto{\pgfqpoint{3.404259in}{2.482426in}}%
\pgfpathlineto{\pgfqpoint{3.404259in}{2.610478in}}%
\pgfpathlineto{\pgfqpoint{3.293363in}{2.674505in}}%
\pgfpathlineto{\pgfqpoint{3.293363in}{2.546452in}}%
\pgfpathclose%
\pgfusepath{fill}%
\end{pgfscope}%
\begin{pgfscope}%
\pgfpathrectangle{\pgfqpoint{0.765000in}{0.660000in}}{\pgfqpoint{4.620000in}{4.620000in}}%
\pgfusepath{clip}%
\pgfsetbuttcap%
\pgfsetroundjoin%
\definecolor{currentfill}{rgb}{1.000000,0.894118,0.788235}%
\pgfsetfillcolor{currentfill}%
\pgfsetlinewidth{0.000000pt}%
\definecolor{currentstroke}{rgb}{1.000000,0.894118,0.788235}%
\pgfsetstrokecolor{currentstroke}%
\pgfsetdash{}{0pt}%
\pgfpathmoveto{\pgfqpoint{3.242912in}{2.646845in}}%
\pgfpathlineto{\pgfqpoint{3.132016in}{2.582819in}}%
\pgfpathlineto{\pgfqpoint{3.242912in}{2.518793in}}%
\pgfpathlineto{\pgfqpoint{3.353809in}{2.582819in}}%
\pgfpathlineto{\pgfqpoint{3.242912in}{2.646845in}}%
\pgfpathclose%
\pgfusepath{fill}%
\end{pgfscope}%
\begin{pgfscope}%
\pgfpathrectangle{\pgfqpoint{0.765000in}{0.660000in}}{\pgfqpoint{4.620000in}{4.620000in}}%
\pgfusepath{clip}%
\pgfsetbuttcap%
\pgfsetroundjoin%
\definecolor{currentfill}{rgb}{1.000000,0.894118,0.788235}%
\pgfsetfillcolor{currentfill}%
\pgfsetlinewidth{0.000000pt}%
\definecolor{currentstroke}{rgb}{1.000000,0.894118,0.788235}%
\pgfsetstrokecolor{currentstroke}%
\pgfsetdash{}{0pt}%
\pgfpathmoveto{\pgfqpoint{3.242912in}{2.390740in}}%
\pgfpathlineto{\pgfqpoint{3.353809in}{2.454767in}}%
\pgfpathlineto{\pgfqpoint{3.353809in}{2.582819in}}%
\pgfpathlineto{\pgfqpoint{3.242912in}{2.518793in}}%
\pgfpathlineto{\pgfqpoint{3.242912in}{2.390740in}}%
\pgfpathclose%
\pgfusepath{fill}%
\end{pgfscope}%
\begin{pgfscope}%
\pgfpathrectangle{\pgfqpoint{0.765000in}{0.660000in}}{\pgfqpoint{4.620000in}{4.620000in}}%
\pgfusepath{clip}%
\pgfsetbuttcap%
\pgfsetroundjoin%
\definecolor{currentfill}{rgb}{1.000000,0.894118,0.788235}%
\pgfsetfillcolor{currentfill}%
\pgfsetlinewidth{0.000000pt}%
\definecolor{currentstroke}{rgb}{1.000000,0.894118,0.788235}%
\pgfsetstrokecolor{currentstroke}%
\pgfsetdash{}{0pt}%
\pgfpathmoveto{\pgfqpoint{3.132016in}{2.454767in}}%
\pgfpathlineto{\pgfqpoint{3.242912in}{2.390740in}}%
\pgfpathlineto{\pgfqpoint{3.242912in}{2.518793in}}%
\pgfpathlineto{\pgfqpoint{3.132016in}{2.582819in}}%
\pgfpathlineto{\pgfqpoint{3.132016in}{2.454767in}}%
\pgfpathclose%
\pgfusepath{fill}%
\end{pgfscope}%
\begin{pgfscope}%
\pgfpathrectangle{\pgfqpoint{0.765000in}{0.660000in}}{\pgfqpoint{4.620000in}{4.620000in}}%
\pgfusepath{clip}%
\pgfsetbuttcap%
\pgfsetroundjoin%
\definecolor{currentfill}{rgb}{1.000000,0.894118,0.788235}%
\pgfsetfillcolor{currentfill}%
\pgfsetlinewidth{0.000000pt}%
\definecolor{currentstroke}{rgb}{1.000000,0.894118,0.788235}%
\pgfsetstrokecolor{currentstroke}%
\pgfsetdash{}{0pt}%
\pgfpathmoveto{\pgfqpoint{3.293363in}{2.674505in}}%
\pgfpathlineto{\pgfqpoint{3.182466in}{2.610478in}}%
\pgfpathlineto{\pgfqpoint{3.293363in}{2.546452in}}%
\pgfpathlineto{\pgfqpoint{3.404259in}{2.610478in}}%
\pgfpathlineto{\pgfqpoint{3.293363in}{2.674505in}}%
\pgfpathclose%
\pgfusepath{fill}%
\end{pgfscope}%
\begin{pgfscope}%
\pgfpathrectangle{\pgfqpoint{0.765000in}{0.660000in}}{\pgfqpoint{4.620000in}{4.620000in}}%
\pgfusepath{clip}%
\pgfsetbuttcap%
\pgfsetroundjoin%
\definecolor{currentfill}{rgb}{1.000000,0.894118,0.788235}%
\pgfsetfillcolor{currentfill}%
\pgfsetlinewidth{0.000000pt}%
\definecolor{currentstroke}{rgb}{1.000000,0.894118,0.788235}%
\pgfsetstrokecolor{currentstroke}%
\pgfsetdash{}{0pt}%
\pgfpathmoveto{\pgfqpoint{3.293363in}{2.418400in}}%
\pgfpathlineto{\pgfqpoint{3.404259in}{2.482426in}}%
\pgfpathlineto{\pgfqpoint{3.404259in}{2.610478in}}%
\pgfpathlineto{\pgfqpoint{3.293363in}{2.546452in}}%
\pgfpathlineto{\pgfqpoint{3.293363in}{2.418400in}}%
\pgfpathclose%
\pgfusepath{fill}%
\end{pgfscope}%
\begin{pgfscope}%
\pgfpathrectangle{\pgfqpoint{0.765000in}{0.660000in}}{\pgfqpoint{4.620000in}{4.620000in}}%
\pgfusepath{clip}%
\pgfsetbuttcap%
\pgfsetroundjoin%
\definecolor{currentfill}{rgb}{1.000000,0.894118,0.788235}%
\pgfsetfillcolor{currentfill}%
\pgfsetlinewidth{0.000000pt}%
\definecolor{currentstroke}{rgb}{1.000000,0.894118,0.788235}%
\pgfsetstrokecolor{currentstroke}%
\pgfsetdash{}{0pt}%
\pgfpathmoveto{\pgfqpoint{3.182466in}{2.482426in}}%
\pgfpathlineto{\pgfqpoint{3.293363in}{2.418400in}}%
\pgfpathlineto{\pgfqpoint{3.293363in}{2.546452in}}%
\pgfpathlineto{\pgfqpoint{3.182466in}{2.610478in}}%
\pgfpathlineto{\pgfqpoint{3.182466in}{2.482426in}}%
\pgfpathclose%
\pgfusepath{fill}%
\end{pgfscope}%
\begin{pgfscope}%
\pgfpathrectangle{\pgfqpoint{0.765000in}{0.660000in}}{\pgfqpoint{4.620000in}{4.620000in}}%
\pgfusepath{clip}%
\pgfsetbuttcap%
\pgfsetroundjoin%
\definecolor{currentfill}{rgb}{1.000000,0.894118,0.788235}%
\pgfsetfillcolor{currentfill}%
\pgfsetlinewidth{0.000000pt}%
\definecolor{currentstroke}{rgb}{1.000000,0.894118,0.788235}%
\pgfsetstrokecolor{currentstroke}%
\pgfsetdash{}{0pt}%
\pgfpathmoveto{\pgfqpoint{3.242912in}{2.518793in}}%
\pgfpathlineto{\pgfqpoint{3.132016in}{2.454767in}}%
\pgfpathlineto{\pgfqpoint{3.182466in}{2.482426in}}%
\pgfpathlineto{\pgfqpoint{3.293363in}{2.546452in}}%
\pgfpathlineto{\pgfqpoint{3.242912in}{2.518793in}}%
\pgfpathclose%
\pgfusepath{fill}%
\end{pgfscope}%
\begin{pgfscope}%
\pgfpathrectangle{\pgfqpoint{0.765000in}{0.660000in}}{\pgfqpoint{4.620000in}{4.620000in}}%
\pgfusepath{clip}%
\pgfsetbuttcap%
\pgfsetroundjoin%
\definecolor{currentfill}{rgb}{1.000000,0.894118,0.788235}%
\pgfsetfillcolor{currentfill}%
\pgfsetlinewidth{0.000000pt}%
\definecolor{currentstroke}{rgb}{1.000000,0.894118,0.788235}%
\pgfsetstrokecolor{currentstroke}%
\pgfsetdash{}{0pt}%
\pgfpathmoveto{\pgfqpoint{3.353809in}{2.454767in}}%
\pgfpathlineto{\pgfqpoint{3.242912in}{2.518793in}}%
\pgfpathlineto{\pgfqpoint{3.293363in}{2.546452in}}%
\pgfpathlineto{\pgfqpoint{3.404259in}{2.482426in}}%
\pgfpathlineto{\pgfqpoint{3.353809in}{2.454767in}}%
\pgfpathclose%
\pgfusepath{fill}%
\end{pgfscope}%
\begin{pgfscope}%
\pgfpathrectangle{\pgfqpoint{0.765000in}{0.660000in}}{\pgfqpoint{4.620000in}{4.620000in}}%
\pgfusepath{clip}%
\pgfsetbuttcap%
\pgfsetroundjoin%
\definecolor{currentfill}{rgb}{1.000000,0.894118,0.788235}%
\pgfsetfillcolor{currentfill}%
\pgfsetlinewidth{0.000000pt}%
\definecolor{currentstroke}{rgb}{1.000000,0.894118,0.788235}%
\pgfsetstrokecolor{currentstroke}%
\pgfsetdash{}{0pt}%
\pgfpathmoveto{\pgfqpoint{3.242912in}{2.518793in}}%
\pgfpathlineto{\pgfqpoint{3.242912in}{2.646845in}}%
\pgfpathlineto{\pgfqpoint{3.293363in}{2.674505in}}%
\pgfpathlineto{\pgfqpoint{3.404259in}{2.482426in}}%
\pgfpathlineto{\pgfqpoint{3.242912in}{2.518793in}}%
\pgfpathclose%
\pgfusepath{fill}%
\end{pgfscope}%
\begin{pgfscope}%
\pgfpathrectangle{\pgfqpoint{0.765000in}{0.660000in}}{\pgfqpoint{4.620000in}{4.620000in}}%
\pgfusepath{clip}%
\pgfsetbuttcap%
\pgfsetroundjoin%
\definecolor{currentfill}{rgb}{1.000000,0.894118,0.788235}%
\pgfsetfillcolor{currentfill}%
\pgfsetlinewidth{0.000000pt}%
\definecolor{currentstroke}{rgb}{1.000000,0.894118,0.788235}%
\pgfsetstrokecolor{currentstroke}%
\pgfsetdash{}{0pt}%
\pgfpathmoveto{\pgfqpoint{3.353809in}{2.454767in}}%
\pgfpathlineto{\pgfqpoint{3.353809in}{2.582819in}}%
\pgfpathlineto{\pgfqpoint{3.404259in}{2.610478in}}%
\pgfpathlineto{\pgfqpoint{3.293363in}{2.546452in}}%
\pgfpathlineto{\pgfqpoint{3.353809in}{2.454767in}}%
\pgfpathclose%
\pgfusepath{fill}%
\end{pgfscope}%
\begin{pgfscope}%
\pgfpathrectangle{\pgfqpoint{0.765000in}{0.660000in}}{\pgfqpoint{4.620000in}{4.620000in}}%
\pgfusepath{clip}%
\pgfsetbuttcap%
\pgfsetroundjoin%
\definecolor{currentfill}{rgb}{1.000000,0.894118,0.788235}%
\pgfsetfillcolor{currentfill}%
\pgfsetlinewidth{0.000000pt}%
\definecolor{currentstroke}{rgb}{1.000000,0.894118,0.788235}%
\pgfsetstrokecolor{currentstroke}%
\pgfsetdash{}{0pt}%
\pgfpathmoveto{\pgfqpoint{3.132016in}{2.454767in}}%
\pgfpathlineto{\pgfqpoint{3.242912in}{2.390740in}}%
\pgfpathlineto{\pgfqpoint{3.293363in}{2.418400in}}%
\pgfpathlineto{\pgfqpoint{3.182466in}{2.482426in}}%
\pgfpathlineto{\pgfqpoint{3.132016in}{2.454767in}}%
\pgfpathclose%
\pgfusepath{fill}%
\end{pgfscope}%
\begin{pgfscope}%
\pgfpathrectangle{\pgfqpoint{0.765000in}{0.660000in}}{\pgfqpoint{4.620000in}{4.620000in}}%
\pgfusepath{clip}%
\pgfsetbuttcap%
\pgfsetroundjoin%
\definecolor{currentfill}{rgb}{1.000000,0.894118,0.788235}%
\pgfsetfillcolor{currentfill}%
\pgfsetlinewidth{0.000000pt}%
\definecolor{currentstroke}{rgb}{1.000000,0.894118,0.788235}%
\pgfsetstrokecolor{currentstroke}%
\pgfsetdash{}{0pt}%
\pgfpathmoveto{\pgfqpoint{3.242912in}{2.390740in}}%
\pgfpathlineto{\pgfqpoint{3.353809in}{2.454767in}}%
\pgfpathlineto{\pgfqpoint{3.404259in}{2.482426in}}%
\pgfpathlineto{\pgfqpoint{3.293363in}{2.418400in}}%
\pgfpathlineto{\pgfqpoint{3.242912in}{2.390740in}}%
\pgfpathclose%
\pgfusepath{fill}%
\end{pgfscope}%
\begin{pgfscope}%
\pgfpathrectangle{\pgfqpoint{0.765000in}{0.660000in}}{\pgfqpoint{4.620000in}{4.620000in}}%
\pgfusepath{clip}%
\pgfsetbuttcap%
\pgfsetroundjoin%
\definecolor{currentfill}{rgb}{1.000000,0.894118,0.788235}%
\pgfsetfillcolor{currentfill}%
\pgfsetlinewidth{0.000000pt}%
\definecolor{currentstroke}{rgb}{1.000000,0.894118,0.788235}%
\pgfsetstrokecolor{currentstroke}%
\pgfsetdash{}{0pt}%
\pgfpathmoveto{\pgfqpoint{3.242912in}{2.646845in}}%
\pgfpathlineto{\pgfqpoint{3.132016in}{2.582819in}}%
\pgfpathlineto{\pgfqpoint{3.182466in}{2.610478in}}%
\pgfpathlineto{\pgfqpoint{3.293363in}{2.674505in}}%
\pgfpathlineto{\pgfqpoint{3.242912in}{2.646845in}}%
\pgfpathclose%
\pgfusepath{fill}%
\end{pgfscope}%
\begin{pgfscope}%
\pgfpathrectangle{\pgfqpoint{0.765000in}{0.660000in}}{\pgfqpoint{4.620000in}{4.620000in}}%
\pgfusepath{clip}%
\pgfsetbuttcap%
\pgfsetroundjoin%
\definecolor{currentfill}{rgb}{1.000000,0.894118,0.788235}%
\pgfsetfillcolor{currentfill}%
\pgfsetlinewidth{0.000000pt}%
\definecolor{currentstroke}{rgb}{1.000000,0.894118,0.788235}%
\pgfsetstrokecolor{currentstroke}%
\pgfsetdash{}{0pt}%
\pgfpathmoveto{\pgfqpoint{3.353809in}{2.582819in}}%
\pgfpathlineto{\pgfqpoint{3.242912in}{2.646845in}}%
\pgfpathlineto{\pgfqpoint{3.293363in}{2.674505in}}%
\pgfpathlineto{\pgfqpoint{3.404259in}{2.610478in}}%
\pgfpathlineto{\pgfqpoint{3.353809in}{2.582819in}}%
\pgfpathclose%
\pgfusepath{fill}%
\end{pgfscope}%
\begin{pgfscope}%
\pgfpathrectangle{\pgfqpoint{0.765000in}{0.660000in}}{\pgfqpoint{4.620000in}{4.620000in}}%
\pgfusepath{clip}%
\pgfsetbuttcap%
\pgfsetroundjoin%
\definecolor{currentfill}{rgb}{1.000000,0.894118,0.788235}%
\pgfsetfillcolor{currentfill}%
\pgfsetlinewidth{0.000000pt}%
\definecolor{currentstroke}{rgb}{1.000000,0.894118,0.788235}%
\pgfsetstrokecolor{currentstroke}%
\pgfsetdash{}{0pt}%
\pgfpathmoveto{\pgfqpoint{3.132016in}{2.454767in}}%
\pgfpathlineto{\pgfqpoint{3.132016in}{2.582819in}}%
\pgfpathlineto{\pgfqpoint{3.182466in}{2.610478in}}%
\pgfpathlineto{\pgfqpoint{3.293363in}{2.418400in}}%
\pgfpathlineto{\pgfqpoint{3.132016in}{2.454767in}}%
\pgfpathclose%
\pgfusepath{fill}%
\end{pgfscope}%
\begin{pgfscope}%
\pgfpathrectangle{\pgfqpoint{0.765000in}{0.660000in}}{\pgfqpoint{4.620000in}{4.620000in}}%
\pgfusepath{clip}%
\pgfsetbuttcap%
\pgfsetroundjoin%
\definecolor{currentfill}{rgb}{1.000000,0.894118,0.788235}%
\pgfsetfillcolor{currentfill}%
\pgfsetlinewidth{0.000000pt}%
\definecolor{currentstroke}{rgb}{1.000000,0.894118,0.788235}%
\pgfsetstrokecolor{currentstroke}%
\pgfsetdash{}{0pt}%
\pgfpathmoveto{\pgfqpoint{3.242912in}{2.390740in}}%
\pgfpathlineto{\pgfqpoint{3.242912in}{2.518793in}}%
\pgfpathlineto{\pgfqpoint{3.293363in}{2.546452in}}%
\pgfpathlineto{\pgfqpoint{3.182466in}{2.482426in}}%
\pgfpathlineto{\pgfqpoint{3.242912in}{2.390740in}}%
\pgfpathclose%
\pgfusepath{fill}%
\end{pgfscope}%
\begin{pgfscope}%
\pgfpathrectangle{\pgfqpoint{0.765000in}{0.660000in}}{\pgfqpoint{4.620000in}{4.620000in}}%
\pgfusepath{clip}%
\pgfsetbuttcap%
\pgfsetroundjoin%
\definecolor{currentfill}{rgb}{1.000000,0.894118,0.788235}%
\pgfsetfillcolor{currentfill}%
\pgfsetlinewidth{0.000000pt}%
\definecolor{currentstroke}{rgb}{1.000000,0.894118,0.788235}%
\pgfsetstrokecolor{currentstroke}%
\pgfsetdash{}{0pt}%
\pgfpathmoveto{\pgfqpoint{3.242912in}{2.518793in}}%
\pgfpathlineto{\pgfqpoint{3.353809in}{2.582819in}}%
\pgfpathlineto{\pgfqpoint{3.404259in}{2.610478in}}%
\pgfpathlineto{\pgfqpoint{3.293363in}{2.546452in}}%
\pgfpathlineto{\pgfqpoint{3.242912in}{2.518793in}}%
\pgfpathclose%
\pgfusepath{fill}%
\end{pgfscope}%
\begin{pgfscope}%
\pgfpathrectangle{\pgfqpoint{0.765000in}{0.660000in}}{\pgfqpoint{4.620000in}{4.620000in}}%
\pgfusepath{clip}%
\pgfsetbuttcap%
\pgfsetroundjoin%
\definecolor{currentfill}{rgb}{1.000000,0.894118,0.788235}%
\pgfsetfillcolor{currentfill}%
\pgfsetlinewidth{0.000000pt}%
\definecolor{currentstroke}{rgb}{1.000000,0.894118,0.788235}%
\pgfsetstrokecolor{currentstroke}%
\pgfsetdash{}{0pt}%
\pgfpathmoveto{\pgfqpoint{3.132016in}{2.582819in}}%
\pgfpathlineto{\pgfqpoint{3.242912in}{2.518793in}}%
\pgfpathlineto{\pgfqpoint{3.293363in}{2.546452in}}%
\pgfpathlineto{\pgfqpoint{3.182466in}{2.610478in}}%
\pgfpathlineto{\pgfqpoint{3.132016in}{2.582819in}}%
\pgfpathclose%
\pgfusepath{fill}%
\end{pgfscope}%
\begin{pgfscope}%
\pgfpathrectangle{\pgfqpoint{0.765000in}{0.660000in}}{\pgfqpoint{4.620000in}{4.620000in}}%
\pgfusepath{clip}%
\pgfsetbuttcap%
\pgfsetroundjoin%
\definecolor{currentfill}{rgb}{1.000000,0.894118,0.788235}%
\pgfsetfillcolor{currentfill}%
\pgfsetlinewidth{0.000000pt}%
\definecolor{currentstroke}{rgb}{1.000000,0.894118,0.788235}%
\pgfsetstrokecolor{currentstroke}%
\pgfsetdash{}{0pt}%
\pgfpathmoveto{\pgfqpoint{3.186392in}{2.493495in}}%
\pgfpathlineto{\pgfqpoint{3.075496in}{2.429469in}}%
\pgfpathlineto{\pgfqpoint{3.186392in}{2.365442in}}%
\pgfpathlineto{\pgfqpoint{3.297289in}{2.429469in}}%
\pgfpathlineto{\pgfqpoint{3.186392in}{2.493495in}}%
\pgfpathclose%
\pgfusepath{fill}%
\end{pgfscope}%
\begin{pgfscope}%
\pgfpathrectangle{\pgfqpoint{0.765000in}{0.660000in}}{\pgfqpoint{4.620000in}{4.620000in}}%
\pgfusepath{clip}%
\pgfsetbuttcap%
\pgfsetroundjoin%
\definecolor{currentfill}{rgb}{1.000000,0.894118,0.788235}%
\pgfsetfillcolor{currentfill}%
\pgfsetlinewidth{0.000000pt}%
\definecolor{currentstroke}{rgb}{1.000000,0.894118,0.788235}%
\pgfsetstrokecolor{currentstroke}%
\pgfsetdash{}{0pt}%
\pgfpathmoveto{\pgfqpoint{3.186392in}{2.493495in}}%
\pgfpathlineto{\pgfqpoint{3.075496in}{2.429469in}}%
\pgfpathlineto{\pgfqpoint{3.075496in}{2.557521in}}%
\pgfpathlineto{\pgfqpoint{3.186392in}{2.621547in}}%
\pgfpathlineto{\pgfqpoint{3.186392in}{2.493495in}}%
\pgfpathclose%
\pgfusepath{fill}%
\end{pgfscope}%
\begin{pgfscope}%
\pgfpathrectangle{\pgfqpoint{0.765000in}{0.660000in}}{\pgfqpoint{4.620000in}{4.620000in}}%
\pgfusepath{clip}%
\pgfsetbuttcap%
\pgfsetroundjoin%
\definecolor{currentfill}{rgb}{1.000000,0.894118,0.788235}%
\pgfsetfillcolor{currentfill}%
\pgfsetlinewidth{0.000000pt}%
\definecolor{currentstroke}{rgb}{1.000000,0.894118,0.788235}%
\pgfsetstrokecolor{currentstroke}%
\pgfsetdash{}{0pt}%
\pgfpathmoveto{\pgfqpoint{3.186392in}{2.493495in}}%
\pgfpathlineto{\pgfqpoint{3.297289in}{2.429469in}}%
\pgfpathlineto{\pgfqpoint{3.297289in}{2.557521in}}%
\pgfpathlineto{\pgfqpoint{3.186392in}{2.621547in}}%
\pgfpathlineto{\pgfqpoint{3.186392in}{2.493495in}}%
\pgfpathclose%
\pgfusepath{fill}%
\end{pgfscope}%
\begin{pgfscope}%
\pgfpathrectangle{\pgfqpoint{0.765000in}{0.660000in}}{\pgfqpoint{4.620000in}{4.620000in}}%
\pgfusepath{clip}%
\pgfsetbuttcap%
\pgfsetroundjoin%
\definecolor{currentfill}{rgb}{1.000000,0.894118,0.788235}%
\pgfsetfillcolor{currentfill}%
\pgfsetlinewidth{0.000000pt}%
\definecolor{currentstroke}{rgb}{1.000000,0.894118,0.788235}%
\pgfsetstrokecolor{currentstroke}%
\pgfsetdash{}{0pt}%
\pgfpathmoveto{\pgfqpoint{3.242912in}{2.518793in}}%
\pgfpathlineto{\pgfqpoint{3.132016in}{2.454767in}}%
\pgfpathlineto{\pgfqpoint{3.242912in}{2.390740in}}%
\pgfpathlineto{\pgfqpoint{3.353809in}{2.454767in}}%
\pgfpathlineto{\pgfqpoint{3.242912in}{2.518793in}}%
\pgfpathclose%
\pgfusepath{fill}%
\end{pgfscope}%
\begin{pgfscope}%
\pgfpathrectangle{\pgfqpoint{0.765000in}{0.660000in}}{\pgfqpoint{4.620000in}{4.620000in}}%
\pgfusepath{clip}%
\pgfsetbuttcap%
\pgfsetroundjoin%
\definecolor{currentfill}{rgb}{1.000000,0.894118,0.788235}%
\pgfsetfillcolor{currentfill}%
\pgfsetlinewidth{0.000000pt}%
\definecolor{currentstroke}{rgb}{1.000000,0.894118,0.788235}%
\pgfsetstrokecolor{currentstroke}%
\pgfsetdash{}{0pt}%
\pgfpathmoveto{\pgfqpoint{3.242912in}{2.518793in}}%
\pgfpathlineto{\pgfqpoint{3.132016in}{2.454767in}}%
\pgfpathlineto{\pgfqpoint{3.132016in}{2.582819in}}%
\pgfpathlineto{\pgfqpoint{3.242912in}{2.646845in}}%
\pgfpathlineto{\pgfqpoint{3.242912in}{2.518793in}}%
\pgfpathclose%
\pgfusepath{fill}%
\end{pgfscope}%
\begin{pgfscope}%
\pgfpathrectangle{\pgfqpoint{0.765000in}{0.660000in}}{\pgfqpoint{4.620000in}{4.620000in}}%
\pgfusepath{clip}%
\pgfsetbuttcap%
\pgfsetroundjoin%
\definecolor{currentfill}{rgb}{1.000000,0.894118,0.788235}%
\pgfsetfillcolor{currentfill}%
\pgfsetlinewidth{0.000000pt}%
\definecolor{currentstroke}{rgb}{1.000000,0.894118,0.788235}%
\pgfsetstrokecolor{currentstroke}%
\pgfsetdash{}{0pt}%
\pgfpathmoveto{\pgfqpoint{3.242912in}{2.518793in}}%
\pgfpathlineto{\pgfqpoint{3.353809in}{2.454767in}}%
\pgfpathlineto{\pgfqpoint{3.353809in}{2.582819in}}%
\pgfpathlineto{\pgfqpoint{3.242912in}{2.646845in}}%
\pgfpathlineto{\pgfqpoint{3.242912in}{2.518793in}}%
\pgfpathclose%
\pgfusepath{fill}%
\end{pgfscope}%
\begin{pgfscope}%
\pgfpathrectangle{\pgfqpoint{0.765000in}{0.660000in}}{\pgfqpoint{4.620000in}{4.620000in}}%
\pgfusepath{clip}%
\pgfsetbuttcap%
\pgfsetroundjoin%
\definecolor{currentfill}{rgb}{1.000000,0.894118,0.788235}%
\pgfsetfillcolor{currentfill}%
\pgfsetlinewidth{0.000000pt}%
\definecolor{currentstroke}{rgb}{1.000000,0.894118,0.788235}%
\pgfsetstrokecolor{currentstroke}%
\pgfsetdash{}{0pt}%
\pgfpathmoveto{\pgfqpoint{3.186392in}{2.621547in}}%
\pgfpathlineto{\pgfqpoint{3.075496in}{2.557521in}}%
\pgfpathlineto{\pgfqpoint{3.186392in}{2.493495in}}%
\pgfpathlineto{\pgfqpoint{3.297289in}{2.557521in}}%
\pgfpathlineto{\pgfqpoint{3.186392in}{2.621547in}}%
\pgfpathclose%
\pgfusepath{fill}%
\end{pgfscope}%
\begin{pgfscope}%
\pgfpathrectangle{\pgfqpoint{0.765000in}{0.660000in}}{\pgfqpoint{4.620000in}{4.620000in}}%
\pgfusepath{clip}%
\pgfsetbuttcap%
\pgfsetroundjoin%
\definecolor{currentfill}{rgb}{1.000000,0.894118,0.788235}%
\pgfsetfillcolor{currentfill}%
\pgfsetlinewidth{0.000000pt}%
\definecolor{currentstroke}{rgb}{1.000000,0.894118,0.788235}%
\pgfsetstrokecolor{currentstroke}%
\pgfsetdash{}{0pt}%
\pgfpathmoveto{\pgfqpoint{3.186392in}{2.365442in}}%
\pgfpathlineto{\pgfqpoint{3.297289in}{2.429469in}}%
\pgfpathlineto{\pgfqpoint{3.297289in}{2.557521in}}%
\pgfpathlineto{\pgfqpoint{3.186392in}{2.493495in}}%
\pgfpathlineto{\pgfqpoint{3.186392in}{2.365442in}}%
\pgfpathclose%
\pgfusepath{fill}%
\end{pgfscope}%
\begin{pgfscope}%
\pgfpathrectangle{\pgfqpoint{0.765000in}{0.660000in}}{\pgfqpoint{4.620000in}{4.620000in}}%
\pgfusepath{clip}%
\pgfsetbuttcap%
\pgfsetroundjoin%
\definecolor{currentfill}{rgb}{1.000000,0.894118,0.788235}%
\pgfsetfillcolor{currentfill}%
\pgfsetlinewidth{0.000000pt}%
\definecolor{currentstroke}{rgb}{1.000000,0.894118,0.788235}%
\pgfsetstrokecolor{currentstroke}%
\pgfsetdash{}{0pt}%
\pgfpathmoveto{\pgfqpoint{3.075496in}{2.429469in}}%
\pgfpathlineto{\pgfqpoint{3.186392in}{2.365442in}}%
\pgfpathlineto{\pgfqpoint{3.186392in}{2.493495in}}%
\pgfpathlineto{\pgfqpoint{3.075496in}{2.557521in}}%
\pgfpathlineto{\pgfqpoint{3.075496in}{2.429469in}}%
\pgfpathclose%
\pgfusepath{fill}%
\end{pgfscope}%
\begin{pgfscope}%
\pgfpathrectangle{\pgfqpoint{0.765000in}{0.660000in}}{\pgfqpoint{4.620000in}{4.620000in}}%
\pgfusepath{clip}%
\pgfsetbuttcap%
\pgfsetroundjoin%
\definecolor{currentfill}{rgb}{1.000000,0.894118,0.788235}%
\pgfsetfillcolor{currentfill}%
\pgfsetlinewidth{0.000000pt}%
\definecolor{currentstroke}{rgb}{1.000000,0.894118,0.788235}%
\pgfsetstrokecolor{currentstroke}%
\pgfsetdash{}{0pt}%
\pgfpathmoveto{\pgfqpoint{3.242912in}{2.646845in}}%
\pgfpathlineto{\pgfqpoint{3.132016in}{2.582819in}}%
\pgfpathlineto{\pgfqpoint{3.242912in}{2.518793in}}%
\pgfpathlineto{\pgfqpoint{3.353809in}{2.582819in}}%
\pgfpathlineto{\pgfqpoint{3.242912in}{2.646845in}}%
\pgfpathclose%
\pgfusepath{fill}%
\end{pgfscope}%
\begin{pgfscope}%
\pgfpathrectangle{\pgfqpoint{0.765000in}{0.660000in}}{\pgfqpoint{4.620000in}{4.620000in}}%
\pgfusepath{clip}%
\pgfsetbuttcap%
\pgfsetroundjoin%
\definecolor{currentfill}{rgb}{1.000000,0.894118,0.788235}%
\pgfsetfillcolor{currentfill}%
\pgfsetlinewidth{0.000000pt}%
\definecolor{currentstroke}{rgb}{1.000000,0.894118,0.788235}%
\pgfsetstrokecolor{currentstroke}%
\pgfsetdash{}{0pt}%
\pgfpathmoveto{\pgfqpoint{3.242912in}{2.390740in}}%
\pgfpathlineto{\pgfqpoint{3.353809in}{2.454767in}}%
\pgfpathlineto{\pgfqpoint{3.353809in}{2.582819in}}%
\pgfpathlineto{\pgfqpoint{3.242912in}{2.518793in}}%
\pgfpathlineto{\pgfqpoint{3.242912in}{2.390740in}}%
\pgfpathclose%
\pgfusepath{fill}%
\end{pgfscope}%
\begin{pgfscope}%
\pgfpathrectangle{\pgfqpoint{0.765000in}{0.660000in}}{\pgfqpoint{4.620000in}{4.620000in}}%
\pgfusepath{clip}%
\pgfsetbuttcap%
\pgfsetroundjoin%
\definecolor{currentfill}{rgb}{1.000000,0.894118,0.788235}%
\pgfsetfillcolor{currentfill}%
\pgfsetlinewidth{0.000000pt}%
\definecolor{currentstroke}{rgb}{1.000000,0.894118,0.788235}%
\pgfsetstrokecolor{currentstroke}%
\pgfsetdash{}{0pt}%
\pgfpathmoveto{\pgfqpoint{3.132016in}{2.454767in}}%
\pgfpathlineto{\pgfqpoint{3.242912in}{2.390740in}}%
\pgfpathlineto{\pgfqpoint{3.242912in}{2.518793in}}%
\pgfpathlineto{\pgfqpoint{3.132016in}{2.582819in}}%
\pgfpathlineto{\pgfqpoint{3.132016in}{2.454767in}}%
\pgfpathclose%
\pgfusepath{fill}%
\end{pgfscope}%
\begin{pgfscope}%
\pgfpathrectangle{\pgfqpoint{0.765000in}{0.660000in}}{\pgfqpoint{4.620000in}{4.620000in}}%
\pgfusepath{clip}%
\pgfsetbuttcap%
\pgfsetroundjoin%
\definecolor{currentfill}{rgb}{1.000000,0.894118,0.788235}%
\pgfsetfillcolor{currentfill}%
\pgfsetlinewidth{0.000000pt}%
\definecolor{currentstroke}{rgb}{1.000000,0.894118,0.788235}%
\pgfsetstrokecolor{currentstroke}%
\pgfsetdash{}{0pt}%
\pgfpathmoveto{\pgfqpoint{3.186392in}{2.493495in}}%
\pgfpathlineto{\pgfqpoint{3.075496in}{2.429469in}}%
\pgfpathlineto{\pgfqpoint{3.132016in}{2.454767in}}%
\pgfpathlineto{\pgfqpoint{3.242912in}{2.518793in}}%
\pgfpathlineto{\pgfqpoint{3.186392in}{2.493495in}}%
\pgfpathclose%
\pgfusepath{fill}%
\end{pgfscope}%
\begin{pgfscope}%
\pgfpathrectangle{\pgfqpoint{0.765000in}{0.660000in}}{\pgfqpoint{4.620000in}{4.620000in}}%
\pgfusepath{clip}%
\pgfsetbuttcap%
\pgfsetroundjoin%
\definecolor{currentfill}{rgb}{1.000000,0.894118,0.788235}%
\pgfsetfillcolor{currentfill}%
\pgfsetlinewidth{0.000000pt}%
\definecolor{currentstroke}{rgb}{1.000000,0.894118,0.788235}%
\pgfsetstrokecolor{currentstroke}%
\pgfsetdash{}{0pt}%
\pgfpathmoveto{\pgfqpoint{3.297289in}{2.429469in}}%
\pgfpathlineto{\pgfqpoint{3.186392in}{2.493495in}}%
\pgfpathlineto{\pgfqpoint{3.242912in}{2.518793in}}%
\pgfpathlineto{\pgfqpoint{3.353809in}{2.454767in}}%
\pgfpathlineto{\pgfqpoint{3.297289in}{2.429469in}}%
\pgfpathclose%
\pgfusepath{fill}%
\end{pgfscope}%
\begin{pgfscope}%
\pgfpathrectangle{\pgfqpoint{0.765000in}{0.660000in}}{\pgfqpoint{4.620000in}{4.620000in}}%
\pgfusepath{clip}%
\pgfsetbuttcap%
\pgfsetroundjoin%
\definecolor{currentfill}{rgb}{1.000000,0.894118,0.788235}%
\pgfsetfillcolor{currentfill}%
\pgfsetlinewidth{0.000000pt}%
\definecolor{currentstroke}{rgb}{1.000000,0.894118,0.788235}%
\pgfsetstrokecolor{currentstroke}%
\pgfsetdash{}{0pt}%
\pgfpathmoveto{\pgfqpoint{3.186392in}{2.493495in}}%
\pgfpathlineto{\pgfqpoint{3.186392in}{2.621547in}}%
\pgfpathlineto{\pgfqpoint{3.242912in}{2.646845in}}%
\pgfpathlineto{\pgfqpoint{3.353809in}{2.454767in}}%
\pgfpathlineto{\pgfqpoint{3.186392in}{2.493495in}}%
\pgfpathclose%
\pgfusepath{fill}%
\end{pgfscope}%
\begin{pgfscope}%
\pgfpathrectangle{\pgfqpoint{0.765000in}{0.660000in}}{\pgfqpoint{4.620000in}{4.620000in}}%
\pgfusepath{clip}%
\pgfsetbuttcap%
\pgfsetroundjoin%
\definecolor{currentfill}{rgb}{1.000000,0.894118,0.788235}%
\pgfsetfillcolor{currentfill}%
\pgfsetlinewidth{0.000000pt}%
\definecolor{currentstroke}{rgb}{1.000000,0.894118,0.788235}%
\pgfsetstrokecolor{currentstroke}%
\pgfsetdash{}{0pt}%
\pgfpathmoveto{\pgfqpoint{3.297289in}{2.429469in}}%
\pgfpathlineto{\pgfqpoint{3.297289in}{2.557521in}}%
\pgfpathlineto{\pgfqpoint{3.353809in}{2.582819in}}%
\pgfpathlineto{\pgfqpoint{3.242912in}{2.518793in}}%
\pgfpathlineto{\pgfqpoint{3.297289in}{2.429469in}}%
\pgfpathclose%
\pgfusepath{fill}%
\end{pgfscope}%
\begin{pgfscope}%
\pgfpathrectangle{\pgfqpoint{0.765000in}{0.660000in}}{\pgfqpoint{4.620000in}{4.620000in}}%
\pgfusepath{clip}%
\pgfsetbuttcap%
\pgfsetroundjoin%
\definecolor{currentfill}{rgb}{1.000000,0.894118,0.788235}%
\pgfsetfillcolor{currentfill}%
\pgfsetlinewidth{0.000000pt}%
\definecolor{currentstroke}{rgb}{1.000000,0.894118,0.788235}%
\pgfsetstrokecolor{currentstroke}%
\pgfsetdash{}{0pt}%
\pgfpathmoveto{\pgfqpoint{3.075496in}{2.429469in}}%
\pgfpathlineto{\pgfqpoint{3.186392in}{2.365442in}}%
\pgfpathlineto{\pgfqpoint{3.242912in}{2.390740in}}%
\pgfpathlineto{\pgfqpoint{3.132016in}{2.454767in}}%
\pgfpathlineto{\pgfqpoint{3.075496in}{2.429469in}}%
\pgfpathclose%
\pgfusepath{fill}%
\end{pgfscope}%
\begin{pgfscope}%
\pgfpathrectangle{\pgfqpoint{0.765000in}{0.660000in}}{\pgfqpoint{4.620000in}{4.620000in}}%
\pgfusepath{clip}%
\pgfsetbuttcap%
\pgfsetroundjoin%
\definecolor{currentfill}{rgb}{1.000000,0.894118,0.788235}%
\pgfsetfillcolor{currentfill}%
\pgfsetlinewidth{0.000000pt}%
\definecolor{currentstroke}{rgb}{1.000000,0.894118,0.788235}%
\pgfsetstrokecolor{currentstroke}%
\pgfsetdash{}{0pt}%
\pgfpathmoveto{\pgfqpoint{3.186392in}{2.365442in}}%
\pgfpathlineto{\pgfqpoint{3.297289in}{2.429469in}}%
\pgfpathlineto{\pgfqpoint{3.353809in}{2.454767in}}%
\pgfpathlineto{\pgfqpoint{3.242912in}{2.390740in}}%
\pgfpathlineto{\pgfqpoint{3.186392in}{2.365442in}}%
\pgfpathclose%
\pgfusepath{fill}%
\end{pgfscope}%
\begin{pgfscope}%
\pgfpathrectangle{\pgfqpoint{0.765000in}{0.660000in}}{\pgfqpoint{4.620000in}{4.620000in}}%
\pgfusepath{clip}%
\pgfsetbuttcap%
\pgfsetroundjoin%
\definecolor{currentfill}{rgb}{1.000000,0.894118,0.788235}%
\pgfsetfillcolor{currentfill}%
\pgfsetlinewidth{0.000000pt}%
\definecolor{currentstroke}{rgb}{1.000000,0.894118,0.788235}%
\pgfsetstrokecolor{currentstroke}%
\pgfsetdash{}{0pt}%
\pgfpathmoveto{\pgfqpoint{3.186392in}{2.621547in}}%
\pgfpathlineto{\pgfqpoint{3.075496in}{2.557521in}}%
\pgfpathlineto{\pgfqpoint{3.132016in}{2.582819in}}%
\pgfpathlineto{\pgfqpoint{3.242912in}{2.646845in}}%
\pgfpathlineto{\pgfqpoint{3.186392in}{2.621547in}}%
\pgfpathclose%
\pgfusepath{fill}%
\end{pgfscope}%
\begin{pgfscope}%
\pgfpathrectangle{\pgfqpoint{0.765000in}{0.660000in}}{\pgfqpoint{4.620000in}{4.620000in}}%
\pgfusepath{clip}%
\pgfsetbuttcap%
\pgfsetroundjoin%
\definecolor{currentfill}{rgb}{1.000000,0.894118,0.788235}%
\pgfsetfillcolor{currentfill}%
\pgfsetlinewidth{0.000000pt}%
\definecolor{currentstroke}{rgb}{1.000000,0.894118,0.788235}%
\pgfsetstrokecolor{currentstroke}%
\pgfsetdash{}{0pt}%
\pgfpathmoveto{\pgfqpoint{3.297289in}{2.557521in}}%
\pgfpathlineto{\pgfqpoint{3.186392in}{2.621547in}}%
\pgfpathlineto{\pgfqpoint{3.242912in}{2.646845in}}%
\pgfpathlineto{\pgfqpoint{3.353809in}{2.582819in}}%
\pgfpathlineto{\pgfqpoint{3.297289in}{2.557521in}}%
\pgfpathclose%
\pgfusepath{fill}%
\end{pgfscope}%
\begin{pgfscope}%
\pgfpathrectangle{\pgfqpoint{0.765000in}{0.660000in}}{\pgfqpoint{4.620000in}{4.620000in}}%
\pgfusepath{clip}%
\pgfsetbuttcap%
\pgfsetroundjoin%
\definecolor{currentfill}{rgb}{1.000000,0.894118,0.788235}%
\pgfsetfillcolor{currentfill}%
\pgfsetlinewidth{0.000000pt}%
\definecolor{currentstroke}{rgb}{1.000000,0.894118,0.788235}%
\pgfsetstrokecolor{currentstroke}%
\pgfsetdash{}{0pt}%
\pgfpathmoveto{\pgfqpoint{3.075496in}{2.429469in}}%
\pgfpathlineto{\pgfqpoint{3.075496in}{2.557521in}}%
\pgfpathlineto{\pgfqpoint{3.132016in}{2.582819in}}%
\pgfpathlineto{\pgfqpoint{3.242912in}{2.390740in}}%
\pgfpathlineto{\pgfqpoint{3.075496in}{2.429469in}}%
\pgfpathclose%
\pgfusepath{fill}%
\end{pgfscope}%
\begin{pgfscope}%
\pgfpathrectangle{\pgfqpoint{0.765000in}{0.660000in}}{\pgfqpoint{4.620000in}{4.620000in}}%
\pgfusepath{clip}%
\pgfsetbuttcap%
\pgfsetroundjoin%
\definecolor{currentfill}{rgb}{1.000000,0.894118,0.788235}%
\pgfsetfillcolor{currentfill}%
\pgfsetlinewidth{0.000000pt}%
\definecolor{currentstroke}{rgb}{1.000000,0.894118,0.788235}%
\pgfsetstrokecolor{currentstroke}%
\pgfsetdash{}{0pt}%
\pgfpathmoveto{\pgfqpoint{3.186392in}{2.365442in}}%
\pgfpathlineto{\pgfqpoint{3.186392in}{2.493495in}}%
\pgfpathlineto{\pgfqpoint{3.242912in}{2.518793in}}%
\pgfpathlineto{\pgfqpoint{3.132016in}{2.454767in}}%
\pgfpathlineto{\pgfqpoint{3.186392in}{2.365442in}}%
\pgfpathclose%
\pgfusepath{fill}%
\end{pgfscope}%
\begin{pgfscope}%
\pgfpathrectangle{\pgfqpoint{0.765000in}{0.660000in}}{\pgfqpoint{4.620000in}{4.620000in}}%
\pgfusepath{clip}%
\pgfsetbuttcap%
\pgfsetroundjoin%
\definecolor{currentfill}{rgb}{1.000000,0.894118,0.788235}%
\pgfsetfillcolor{currentfill}%
\pgfsetlinewidth{0.000000pt}%
\definecolor{currentstroke}{rgb}{1.000000,0.894118,0.788235}%
\pgfsetstrokecolor{currentstroke}%
\pgfsetdash{}{0pt}%
\pgfpathmoveto{\pgfqpoint{3.186392in}{2.493495in}}%
\pgfpathlineto{\pgfqpoint{3.297289in}{2.557521in}}%
\pgfpathlineto{\pgfqpoint{3.353809in}{2.582819in}}%
\pgfpathlineto{\pgfqpoint{3.242912in}{2.518793in}}%
\pgfpathlineto{\pgfqpoint{3.186392in}{2.493495in}}%
\pgfpathclose%
\pgfusepath{fill}%
\end{pgfscope}%
\begin{pgfscope}%
\pgfpathrectangle{\pgfqpoint{0.765000in}{0.660000in}}{\pgfqpoint{4.620000in}{4.620000in}}%
\pgfusepath{clip}%
\pgfsetbuttcap%
\pgfsetroundjoin%
\definecolor{currentfill}{rgb}{1.000000,0.894118,0.788235}%
\pgfsetfillcolor{currentfill}%
\pgfsetlinewidth{0.000000pt}%
\definecolor{currentstroke}{rgb}{1.000000,0.894118,0.788235}%
\pgfsetstrokecolor{currentstroke}%
\pgfsetdash{}{0pt}%
\pgfpathmoveto{\pgfqpoint{3.075496in}{2.557521in}}%
\pgfpathlineto{\pgfqpoint{3.186392in}{2.493495in}}%
\pgfpathlineto{\pgfqpoint{3.242912in}{2.518793in}}%
\pgfpathlineto{\pgfqpoint{3.132016in}{2.582819in}}%
\pgfpathlineto{\pgfqpoint{3.075496in}{2.557521in}}%
\pgfpathclose%
\pgfusepath{fill}%
\end{pgfscope}%
\begin{pgfscope}%
\pgfpathrectangle{\pgfqpoint{0.765000in}{0.660000in}}{\pgfqpoint{4.620000in}{4.620000in}}%
\pgfusepath{clip}%
\pgfsetbuttcap%
\pgfsetroundjoin%
\definecolor{currentfill}{rgb}{1.000000,0.894118,0.788235}%
\pgfsetfillcolor{currentfill}%
\pgfsetlinewidth{0.000000pt}%
\definecolor{currentstroke}{rgb}{1.000000,0.894118,0.788235}%
\pgfsetstrokecolor{currentstroke}%
\pgfsetdash{}{0pt}%
\pgfpathmoveto{\pgfqpoint{3.186392in}{2.493495in}}%
\pgfpathlineto{\pgfqpoint{3.075496in}{2.429469in}}%
\pgfpathlineto{\pgfqpoint{3.186392in}{2.365442in}}%
\pgfpathlineto{\pgfqpoint{3.297289in}{2.429469in}}%
\pgfpathlineto{\pgfqpoint{3.186392in}{2.493495in}}%
\pgfpathclose%
\pgfusepath{fill}%
\end{pgfscope}%
\begin{pgfscope}%
\pgfpathrectangle{\pgfqpoint{0.765000in}{0.660000in}}{\pgfqpoint{4.620000in}{4.620000in}}%
\pgfusepath{clip}%
\pgfsetbuttcap%
\pgfsetroundjoin%
\definecolor{currentfill}{rgb}{1.000000,0.894118,0.788235}%
\pgfsetfillcolor{currentfill}%
\pgfsetlinewidth{0.000000pt}%
\definecolor{currentstroke}{rgb}{1.000000,0.894118,0.788235}%
\pgfsetstrokecolor{currentstroke}%
\pgfsetdash{}{0pt}%
\pgfpathmoveto{\pgfqpoint{3.186392in}{2.493495in}}%
\pgfpathlineto{\pgfqpoint{3.075496in}{2.429469in}}%
\pgfpathlineto{\pgfqpoint{3.075496in}{2.557521in}}%
\pgfpathlineto{\pgfqpoint{3.186392in}{2.621547in}}%
\pgfpathlineto{\pgfqpoint{3.186392in}{2.493495in}}%
\pgfpathclose%
\pgfusepath{fill}%
\end{pgfscope}%
\begin{pgfscope}%
\pgfpathrectangle{\pgfqpoint{0.765000in}{0.660000in}}{\pgfqpoint{4.620000in}{4.620000in}}%
\pgfusepath{clip}%
\pgfsetbuttcap%
\pgfsetroundjoin%
\definecolor{currentfill}{rgb}{1.000000,0.894118,0.788235}%
\pgfsetfillcolor{currentfill}%
\pgfsetlinewidth{0.000000pt}%
\definecolor{currentstroke}{rgb}{1.000000,0.894118,0.788235}%
\pgfsetstrokecolor{currentstroke}%
\pgfsetdash{}{0pt}%
\pgfpathmoveto{\pgfqpoint{3.186392in}{2.493495in}}%
\pgfpathlineto{\pgfqpoint{3.297289in}{2.429469in}}%
\pgfpathlineto{\pgfqpoint{3.297289in}{2.557521in}}%
\pgfpathlineto{\pgfqpoint{3.186392in}{2.621547in}}%
\pgfpathlineto{\pgfqpoint{3.186392in}{2.493495in}}%
\pgfpathclose%
\pgfusepath{fill}%
\end{pgfscope}%
\begin{pgfscope}%
\pgfpathrectangle{\pgfqpoint{0.765000in}{0.660000in}}{\pgfqpoint{4.620000in}{4.620000in}}%
\pgfusepath{clip}%
\pgfsetbuttcap%
\pgfsetroundjoin%
\definecolor{currentfill}{rgb}{1.000000,0.894118,0.788235}%
\pgfsetfillcolor{currentfill}%
\pgfsetlinewidth{0.000000pt}%
\definecolor{currentstroke}{rgb}{1.000000,0.894118,0.788235}%
\pgfsetstrokecolor{currentstroke}%
\pgfsetdash{}{0pt}%
\pgfpathmoveto{\pgfqpoint{3.120119in}{2.489790in}}%
\pgfpathlineto{\pgfqpoint{3.009222in}{2.425764in}}%
\pgfpathlineto{\pgfqpoint{3.120119in}{2.361737in}}%
\pgfpathlineto{\pgfqpoint{3.231016in}{2.425764in}}%
\pgfpathlineto{\pgfqpoint{3.120119in}{2.489790in}}%
\pgfpathclose%
\pgfusepath{fill}%
\end{pgfscope}%
\begin{pgfscope}%
\pgfpathrectangle{\pgfqpoint{0.765000in}{0.660000in}}{\pgfqpoint{4.620000in}{4.620000in}}%
\pgfusepath{clip}%
\pgfsetbuttcap%
\pgfsetroundjoin%
\definecolor{currentfill}{rgb}{1.000000,0.894118,0.788235}%
\pgfsetfillcolor{currentfill}%
\pgfsetlinewidth{0.000000pt}%
\definecolor{currentstroke}{rgb}{1.000000,0.894118,0.788235}%
\pgfsetstrokecolor{currentstroke}%
\pgfsetdash{}{0pt}%
\pgfpathmoveto{\pgfqpoint{3.120119in}{2.489790in}}%
\pgfpathlineto{\pgfqpoint{3.009222in}{2.425764in}}%
\pgfpathlineto{\pgfqpoint{3.009222in}{2.553816in}}%
\pgfpathlineto{\pgfqpoint{3.120119in}{2.617842in}}%
\pgfpathlineto{\pgfqpoint{3.120119in}{2.489790in}}%
\pgfpathclose%
\pgfusepath{fill}%
\end{pgfscope}%
\begin{pgfscope}%
\pgfpathrectangle{\pgfqpoint{0.765000in}{0.660000in}}{\pgfqpoint{4.620000in}{4.620000in}}%
\pgfusepath{clip}%
\pgfsetbuttcap%
\pgfsetroundjoin%
\definecolor{currentfill}{rgb}{1.000000,0.894118,0.788235}%
\pgfsetfillcolor{currentfill}%
\pgfsetlinewidth{0.000000pt}%
\definecolor{currentstroke}{rgb}{1.000000,0.894118,0.788235}%
\pgfsetstrokecolor{currentstroke}%
\pgfsetdash{}{0pt}%
\pgfpathmoveto{\pgfqpoint{3.120119in}{2.489790in}}%
\pgfpathlineto{\pgfqpoint{3.231016in}{2.425764in}}%
\pgfpathlineto{\pgfqpoint{3.231016in}{2.553816in}}%
\pgfpathlineto{\pgfqpoint{3.120119in}{2.617842in}}%
\pgfpathlineto{\pgfqpoint{3.120119in}{2.489790in}}%
\pgfpathclose%
\pgfusepath{fill}%
\end{pgfscope}%
\begin{pgfscope}%
\pgfpathrectangle{\pgfqpoint{0.765000in}{0.660000in}}{\pgfqpoint{4.620000in}{4.620000in}}%
\pgfusepath{clip}%
\pgfsetbuttcap%
\pgfsetroundjoin%
\definecolor{currentfill}{rgb}{1.000000,0.894118,0.788235}%
\pgfsetfillcolor{currentfill}%
\pgfsetlinewidth{0.000000pt}%
\definecolor{currentstroke}{rgb}{1.000000,0.894118,0.788235}%
\pgfsetstrokecolor{currentstroke}%
\pgfsetdash{}{0pt}%
\pgfpathmoveto{\pgfqpoint{3.186392in}{2.621547in}}%
\pgfpathlineto{\pgfqpoint{3.075496in}{2.557521in}}%
\pgfpathlineto{\pgfqpoint{3.186392in}{2.493495in}}%
\pgfpathlineto{\pgfqpoint{3.297289in}{2.557521in}}%
\pgfpathlineto{\pgfqpoint{3.186392in}{2.621547in}}%
\pgfpathclose%
\pgfusepath{fill}%
\end{pgfscope}%
\begin{pgfscope}%
\pgfpathrectangle{\pgfqpoint{0.765000in}{0.660000in}}{\pgfqpoint{4.620000in}{4.620000in}}%
\pgfusepath{clip}%
\pgfsetbuttcap%
\pgfsetroundjoin%
\definecolor{currentfill}{rgb}{1.000000,0.894118,0.788235}%
\pgfsetfillcolor{currentfill}%
\pgfsetlinewidth{0.000000pt}%
\definecolor{currentstroke}{rgb}{1.000000,0.894118,0.788235}%
\pgfsetstrokecolor{currentstroke}%
\pgfsetdash{}{0pt}%
\pgfpathmoveto{\pgfqpoint{3.186392in}{2.365442in}}%
\pgfpathlineto{\pgfqpoint{3.297289in}{2.429469in}}%
\pgfpathlineto{\pgfqpoint{3.297289in}{2.557521in}}%
\pgfpathlineto{\pgfqpoint{3.186392in}{2.493495in}}%
\pgfpathlineto{\pgfqpoint{3.186392in}{2.365442in}}%
\pgfpathclose%
\pgfusepath{fill}%
\end{pgfscope}%
\begin{pgfscope}%
\pgfpathrectangle{\pgfqpoint{0.765000in}{0.660000in}}{\pgfqpoint{4.620000in}{4.620000in}}%
\pgfusepath{clip}%
\pgfsetbuttcap%
\pgfsetroundjoin%
\definecolor{currentfill}{rgb}{1.000000,0.894118,0.788235}%
\pgfsetfillcolor{currentfill}%
\pgfsetlinewidth{0.000000pt}%
\definecolor{currentstroke}{rgb}{1.000000,0.894118,0.788235}%
\pgfsetstrokecolor{currentstroke}%
\pgfsetdash{}{0pt}%
\pgfpathmoveto{\pgfqpoint{3.075496in}{2.429469in}}%
\pgfpathlineto{\pgfqpoint{3.186392in}{2.365442in}}%
\pgfpathlineto{\pgfqpoint{3.186392in}{2.493495in}}%
\pgfpathlineto{\pgfqpoint{3.075496in}{2.557521in}}%
\pgfpathlineto{\pgfqpoint{3.075496in}{2.429469in}}%
\pgfpathclose%
\pgfusepath{fill}%
\end{pgfscope}%
\begin{pgfscope}%
\pgfpathrectangle{\pgfqpoint{0.765000in}{0.660000in}}{\pgfqpoint{4.620000in}{4.620000in}}%
\pgfusepath{clip}%
\pgfsetbuttcap%
\pgfsetroundjoin%
\definecolor{currentfill}{rgb}{1.000000,0.894118,0.788235}%
\pgfsetfillcolor{currentfill}%
\pgfsetlinewidth{0.000000pt}%
\definecolor{currentstroke}{rgb}{1.000000,0.894118,0.788235}%
\pgfsetstrokecolor{currentstroke}%
\pgfsetdash{}{0pt}%
\pgfpathmoveto{\pgfqpoint{3.120119in}{2.617842in}}%
\pgfpathlineto{\pgfqpoint{3.009222in}{2.553816in}}%
\pgfpathlineto{\pgfqpoint{3.120119in}{2.489790in}}%
\pgfpathlineto{\pgfqpoint{3.231016in}{2.553816in}}%
\pgfpathlineto{\pgfqpoint{3.120119in}{2.617842in}}%
\pgfpathclose%
\pgfusepath{fill}%
\end{pgfscope}%
\begin{pgfscope}%
\pgfpathrectangle{\pgfqpoint{0.765000in}{0.660000in}}{\pgfqpoint{4.620000in}{4.620000in}}%
\pgfusepath{clip}%
\pgfsetbuttcap%
\pgfsetroundjoin%
\definecolor{currentfill}{rgb}{1.000000,0.894118,0.788235}%
\pgfsetfillcolor{currentfill}%
\pgfsetlinewidth{0.000000pt}%
\definecolor{currentstroke}{rgb}{1.000000,0.894118,0.788235}%
\pgfsetstrokecolor{currentstroke}%
\pgfsetdash{}{0pt}%
\pgfpathmoveto{\pgfqpoint{3.120119in}{2.361737in}}%
\pgfpathlineto{\pgfqpoint{3.231016in}{2.425764in}}%
\pgfpathlineto{\pgfqpoint{3.231016in}{2.553816in}}%
\pgfpathlineto{\pgfqpoint{3.120119in}{2.489790in}}%
\pgfpathlineto{\pgfqpoint{3.120119in}{2.361737in}}%
\pgfpathclose%
\pgfusepath{fill}%
\end{pgfscope}%
\begin{pgfscope}%
\pgfpathrectangle{\pgfqpoint{0.765000in}{0.660000in}}{\pgfqpoint{4.620000in}{4.620000in}}%
\pgfusepath{clip}%
\pgfsetbuttcap%
\pgfsetroundjoin%
\definecolor{currentfill}{rgb}{1.000000,0.894118,0.788235}%
\pgfsetfillcolor{currentfill}%
\pgfsetlinewidth{0.000000pt}%
\definecolor{currentstroke}{rgb}{1.000000,0.894118,0.788235}%
\pgfsetstrokecolor{currentstroke}%
\pgfsetdash{}{0pt}%
\pgfpathmoveto{\pgfqpoint{3.009222in}{2.425764in}}%
\pgfpathlineto{\pgfqpoint{3.120119in}{2.361737in}}%
\pgfpathlineto{\pgfqpoint{3.120119in}{2.489790in}}%
\pgfpathlineto{\pgfqpoint{3.009222in}{2.553816in}}%
\pgfpathlineto{\pgfqpoint{3.009222in}{2.425764in}}%
\pgfpathclose%
\pgfusepath{fill}%
\end{pgfscope}%
\begin{pgfscope}%
\pgfpathrectangle{\pgfqpoint{0.765000in}{0.660000in}}{\pgfqpoint{4.620000in}{4.620000in}}%
\pgfusepath{clip}%
\pgfsetbuttcap%
\pgfsetroundjoin%
\definecolor{currentfill}{rgb}{1.000000,0.894118,0.788235}%
\pgfsetfillcolor{currentfill}%
\pgfsetlinewidth{0.000000pt}%
\definecolor{currentstroke}{rgb}{1.000000,0.894118,0.788235}%
\pgfsetstrokecolor{currentstroke}%
\pgfsetdash{}{0pt}%
\pgfpathmoveto{\pgfqpoint{3.186392in}{2.493495in}}%
\pgfpathlineto{\pgfqpoint{3.075496in}{2.429469in}}%
\pgfpathlineto{\pgfqpoint{3.009222in}{2.425764in}}%
\pgfpathlineto{\pgfqpoint{3.120119in}{2.489790in}}%
\pgfpathlineto{\pgfqpoint{3.186392in}{2.493495in}}%
\pgfpathclose%
\pgfusepath{fill}%
\end{pgfscope}%
\begin{pgfscope}%
\pgfpathrectangle{\pgfqpoint{0.765000in}{0.660000in}}{\pgfqpoint{4.620000in}{4.620000in}}%
\pgfusepath{clip}%
\pgfsetbuttcap%
\pgfsetroundjoin%
\definecolor{currentfill}{rgb}{1.000000,0.894118,0.788235}%
\pgfsetfillcolor{currentfill}%
\pgfsetlinewidth{0.000000pt}%
\definecolor{currentstroke}{rgb}{1.000000,0.894118,0.788235}%
\pgfsetstrokecolor{currentstroke}%
\pgfsetdash{}{0pt}%
\pgfpathmoveto{\pgfqpoint{3.297289in}{2.429469in}}%
\pgfpathlineto{\pgfqpoint{3.186392in}{2.493495in}}%
\pgfpathlineto{\pgfqpoint{3.120119in}{2.489790in}}%
\pgfpathlineto{\pgfqpoint{3.231016in}{2.425764in}}%
\pgfpathlineto{\pgfqpoint{3.297289in}{2.429469in}}%
\pgfpathclose%
\pgfusepath{fill}%
\end{pgfscope}%
\begin{pgfscope}%
\pgfpathrectangle{\pgfqpoint{0.765000in}{0.660000in}}{\pgfqpoint{4.620000in}{4.620000in}}%
\pgfusepath{clip}%
\pgfsetbuttcap%
\pgfsetroundjoin%
\definecolor{currentfill}{rgb}{1.000000,0.894118,0.788235}%
\pgfsetfillcolor{currentfill}%
\pgfsetlinewidth{0.000000pt}%
\definecolor{currentstroke}{rgb}{1.000000,0.894118,0.788235}%
\pgfsetstrokecolor{currentstroke}%
\pgfsetdash{}{0pt}%
\pgfpathmoveto{\pgfqpoint{3.186392in}{2.493495in}}%
\pgfpathlineto{\pgfqpoint{3.186392in}{2.621547in}}%
\pgfpathlineto{\pgfqpoint{3.120119in}{2.617842in}}%
\pgfpathlineto{\pgfqpoint{3.231016in}{2.425764in}}%
\pgfpathlineto{\pgfqpoint{3.186392in}{2.493495in}}%
\pgfpathclose%
\pgfusepath{fill}%
\end{pgfscope}%
\begin{pgfscope}%
\pgfpathrectangle{\pgfqpoint{0.765000in}{0.660000in}}{\pgfqpoint{4.620000in}{4.620000in}}%
\pgfusepath{clip}%
\pgfsetbuttcap%
\pgfsetroundjoin%
\definecolor{currentfill}{rgb}{1.000000,0.894118,0.788235}%
\pgfsetfillcolor{currentfill}%
\pgfsetlinewidth{0.000000pt}%
\definecolor{currentstroke}{rgb}{1.000000,0.894118,0.788235}%
\pgfsetstrokecolor{currentstroke}%
\pgfsetdash{}{0pt}%
\pgfpathmoveto{\pgfqpoint{3.297289in}{2.429469in}}%
\pgfpathlineto{\pgfqpoint{3.297289in}{2.557521in}}%
\pgfpathlineto{\pgfqpoint{3.231016in}{2.553816in}}%
\pgfpathlineto{\pgfqpoint{3.120119in}{2.489790in}}%
\pgfpathlineto{\pgfqpoint{3.297289in}{2.429469in}}%
\pgfpathclose%
\pgfusepath{fill}%
\end{pgfscope}%
\begin{pgfscope}%
\pgfpathrectangle{\pgfqpoint{0.765000in}{0.660000in}}{\pgfqpoint{4.620000in}{4.620000in}}%
\pgfusepath{clip}%
\pgfsetbuttcap%
\pgfsetroundjoin%
\definecolor{currentfill}{rgb}{1.000000,0.894118,0.788235}%
\pgfsetfillcolor{currentfill}%
\pgfsetlinewidth{0.000000pt}%
\definecolor{currentstroke}{rgb}{1.000000,0.894118,0.788235}%
\pgfsetstrokecolor{currentstroke}%
\pgfsetdash{}{0pt}%
\pgfpathmoveto{\pgfqpoint{3.075496in}{2.429469in}}%
\pgfpathlineto{\pgfqpoint{3.186392in}{2.365442in}}%
\pgfpathlineto{\pgfqpoint{3.120119in}{2.361737in}}%
\pgfpathlineto{\pgfqpoint{3.009222in}{2.425764in}}%
\pgfpathlineto{\pgfqpoint{3.075496in}{2.429469in}}%
\pgfpathclose%
\pgfusepath{fill}%
\end{pgfscope}%
\begin{pgfscope}%
\pgfpathrectangle{\pgfqpoint{0.765000in}{0.660000in}}{\pgfqpoint{4.620000in}{4.620000in}}%
\pgfusepath{clip}%
\pgfsetbuttcap%
\pgfsetroundjoin%
\definecolor{currentfill}{rgb}{1.000000,0.894118,0.788235}%
\pgfsetfillcolor{currentfill}%
\pgfsetlinewidth{0.000000pt}%
\definecolor{currentstroke}{rgb}{1.000000,0.894118,0.788235}%
\pgfsetstrokecolor{currentstroke}%
\pgfsetdash{}{0pt}%
\pgfpathmoveto{\pgfqpoint{3.186392in}{2.365442in}}%
\pgfpathlineto{\pgfqpoint{3.297289in}{2.429469in}}%
\pgfpathlineto{\pgfqpoint{3.231016in}{2.425764in}}%
\pgfpathlineto{\pgfqpoint{3.120119in}{2.361737in}}%
\pgfpathlineto{\pgfqpoint{3.186392in}{2.365442in}}%
\pgfpathclose%
\pgfusepath{fill}%
\end{pgfscope}%
\begin{pgfscope}%
\pgfpathrectangle{\pgfqpoint{0.765000in}{0.660000in}}{\pgfqpoint{4.620000in}{4.620000in}}%
\pgfusepath{clip}%
\pgfsetbuttcap%
\pgfsetroundjoin%
\definecolor{currentfill}{rgb}{1.000000,0.894118,0.788235}%
\pgfsetfillcolor{currentfill}%
\pgfsetlinewidth{0.000000pt}%
\definecolor{currentstroke}{rgb}{1.000000,0.894118,0.788235}%
\pgfsetstrokecolor{currentstroke}%
\pgfsetdash{}{0pt}%
\pgfpathmoveto{\pgfqpoint{3.186392in}{2.621547in}}%
\pgfpathlineto{\pgfqpoint{3.075496in}{2.557521in}}%
\pgfpathlineto{\pgfqpoint{3.009222in}{2.553816in}}%
\pgfpathlineto{\pgfqpoint{3.120119in}{2.617842in}}%
\pgfpathlineto{\pgfqpoint{3.186392in}{2.621547in}}%
\pgfpathclose%
\pgfusepath{fill}%
\end{pgfscope}%
\begin{pgfscope}%
\pgfpathrectangle{\pgfqpoint{0.765000in}{0.660000in}}{\pgfqpoint{4.620000in}{4.620000in}}%
\pgfusepath{clip}%
\pgfsetbuttcap%
\pgfsetroundjoin%
\definecolor{currentfill}{rgb}{1.000000,0.894118,0.788235}%
\pgfsetfillcolor{currentfill}%
\pgfsetlinewidth{0.000000pt}%
\definecolor{currentstroke}{rgb}{1.000000,0.894118,0.788235}%
\pgfsetstrokecolor{currentstroke}%
\pgfsetdash{}{0pt}%
\pgfpathmoveto{\pgfqpoint{3.297289in}{2.557521in}}%
\pgfpathlineto{\pgfqpoint{3.186392in}{2.621547in}}%
\pgfpathlineto{\pgfqpoint{3.120119in}{2.617842in}}%
\pgfpathlineto{\pgfqpoint{3.231016in}{2.553816in}}%
\pgfpathlineto{\pgfqpoint{3.297289in}{2.557521in}}%
\pgfpathclose%
\pgfusepath{fill}%
\end{pgfscope}%
\begin{pgfscope}%
\pgfpathrectangle{\pgfqpoint{0.765000in}{0.660000in}}{\pgfqpoint{4.620000in}{4.620000in}}%
\pgfusepath{clip}%
\pgfsetbuttcap%
\pgfsetroundjoin%
\definecolor{currentfill}{rgb}{1.000000,0.894118,0.788235}%
\pgfsetfillcolor{currentfill}%
\pgfsetlinewidth{0.000000pt}%
\definecolor{currentstroke}{rgb}{1.000000,0.894118,0.788235}%
\pgfsetstrokecolor{currentstroke}%
\pgfsetdash{}{0pt}%
\pgfpathmoveto{\pgfqpoint{3.075496in}{2.429469in}}%
\pgfpathlineto{\pgfqpoint{3.075496in}{2.557521in}}%
\pgfpathlineto{\pgfqpoint{3.009222in}{2.553816in}}%
\pgfpathlineto{\pgfqpoint{3.120119in}{2.361737in}}%
\pgfpathlineto{\pgfqpoint{3.075496in}{2.429469in}}%
\pgfpathclose%
\pgfusepath{fill}%
\end{pgfscope}%
\begin{pgfscope}%
\pgfpathrectangle{\pgfqpoint{0.765000in}{0.660000in}}{\pgfqpoint{4.620000in}{4.620000in}}%
\pgfusepath{clip}%
\pgfsetbuttcap%
\pgfsetroundjoin%
\definecolor{currentfill}{rgb}{1.000000,0.894118,0.788235}%
\pgfsetfillcolor{currentfill}%
\pgfsetlinewidth{0.000000pt}%
\definecolor{currentstroke}{rgb}{1.000000,0.894118,0.788235}%
\pgfsetstrokecolor{currentstroke}%
\pgfsetdash{}{0pt}%
\pgfpathmoveto{\pgfqpoint{3.186392in}{2.365442in}}%
\pgfpathlineto{\pgfqpoint{3.186392in}{2.493495in}}%
\pgfpathlineto{\pgfqpoint{3.120119in}{2.489790in}}%
\pgfpathlineto{\pgfqpoint{3.009222in}{2.425764in}}%
\pgfpathlineto{\pgfqpoint{3.186392in}{2.365442in}}%
\pgfpathclose%
\pgfusepath{fill}%
\end{pgfscope}%
\begin{pgfscope}%
\pgfpathrectangle{\pgfqpoint{0.765000in}{0.660000in}}{\pgfqpoint{4.620000in}{4.620000in}}%
\pgfusepath{clip}%
\pgfsetbuttcap%
\pgfsetroundjoin%
\definecolor{currentfill}{rgb}{1.000000,0.894118,0.788235}%
\pgfsetfillcolor{currentfill}%
\pgfsetlinewidth{0.000000pt}%
\definecolor{currentstroke}{rgb}{1.000000,0.894118,0.788235}%
\pgfsetstrokecolor{currentstroke}%
\pgfsetdash{}{0pt}%
\pgfpathmoveto{\pgfqpoint{3.186392in}{2.493495in}}%
\pgfpathlineto{\pgfqpoint{3.297289in}{2.557521in}}%
\pgfpathlineto{\pgfqpoint{3.231016in}{2.553816in}}%
\pgfpathlineto{\pgfqpoint{3.120119in}{2.489790in}}%
\pgfpathlineto{\pgfqpoint{3.186392in}{2.493495in}}%
\pgfpathclose%
\pgfusepath{fill}%
\end{pgfscope}%
\begin{pgfscope}%
\pgfpathrectangle{\pgfqpoint{0.765000in}{0.660000in}}{\pgfqpoint{4.620000in}{4.620000in}}%
\pgfusepath{clip}%
\pgfsetbuttcap%
\pgfsetroundjoin%
\definecolor{currentfill}{rgb}{1.000000,0.894118,0.788235}%
\pgfsetfillcolor{currentfill}%
\pgfsetlinewidth{0.000000pt}%
\definecolor{currentstroke}{rgb}{1.000000,0.894118,0.788235}%
\pgfsetstrokecolor{currentstroke}%
\pgfsetdash{}{0pt}%
\pgfpathmoveto{\pgfqpoint{3.075496in}{2.557521in}}%
\pgfpathlineto{\pgfqpoint{3.186392in}{2.493495in}}%
\pgfpathlineto{\pgfqpoint{3.120119in}{2.489790in}}%
\pgfpathlineto{\pgfqpoint{3.009222in}{2.553816in}}%
\pgfpathlineto{\pgfqpoint{3.075496in}{2.557521in}}%
\pgfpathclose%
\pgfusepath{fill}%
\end{pgfscope}%
\begin{pgfscope}%
\pgfpathrectangle{\pgfqpoint{0.765000in}{0.660000in}}{\pgfqpoint{4.620000in}{4.620000in}}%
\pgfusepath{clip}%
\pgfsetbuttcap%
\pgfsetroundjoin%
\definecolor{currentfill}{rgb}{1.000000,0.894118,0.788235}%
\pgfsetfillcolor{currentfill}%
\pgfsetlinewidth{0.000000pt}%
\definecolor{currentstroke}{rgb}{1.000000,0.894118,0.788235}%
\pgfsetstrokecolor{currentstroke}%
\pgfsetdash{}{0pt}%
\pgfpathmoveto{\pgfqpoint{3.071936in}{2.492937in}}%
\pgfpathlineto{\pgfqpoint{2.961039in}{2.428911in}}%
\pgfpathlineto{\pgfqpoint{3.071936in}{2.364884in}}%
\pgfpathlineto{\pgfqpoint{3.182833in}{2.428911in}}%
\pgfpathlineto{\pgfqpoint{3.071936in}{2.492937in}}%
\pgfpathclose%
\pgfusepath{fill}%
\end{pgfscope}%
\begin{pgfscope}%
\pgfpathrectangle{\pgfqpoint{0.765000in}{0.660000in}}{\pgfqpoint{4.620000in}{4.620000in}}%
\pgfusepath{clip}%
\pgfsetbuttcap%
\pgfsetroundjoin%
\definecolor{currentfill}{rgb}{1.000000,0.894118,0.788235}%
\pgfsetfillcolor{currentfill}%
\pgfsetlinewidth{0.000000pt}%
\definecolor{currentstroke}{rgb}{1.000000,0.894118,0.788235}%
\pgfsetstrokecolor{currentstroke}%
\pgfsetdash{}{0pt}%
\pgfpathmoveto{\pgfqpoint{3.071936in}{2.492937in}}%
\pgfpathlineto{\pgfqpoint{2.961039in}{2.428911in}}%
\pgfpathlineto{\pgfqpoint{2.961039in}{2.556963in}}%
\pgfpathlineto{\pgfqpoint{3.071936in}{2.620989in}}%
\pgfpathlineto{\pgfqpoint{3.071936in}{2.492937in}}%
\pgfpathclose%
\pgfusepath{fill}%
\end{pgfscope}%
\begin{pgfscope}%
\pgfpathrectangle{\pgfqpoint{0.765000in}{0.660000in}}{\pgfqpoint{4.620000in}{4.620000in}}%
\pgfusepath{clip}%
\pgfsetbuttcap%
\pgfsetroundjoin%
\definecolor{currentfill}{rgb}{1.000000,0.894118,0.788235}%
\pgfsetfillcolor{currentfill}%
\pgfsetlinewidth{0.000000pt}%
\definecolor{currentstroke}{rgb}{1.000000,0.894118,0.788235}%
\pgfsetstrokecolor{currentstroke}%
\pgfsetdash{}{0pt}%
\pgfpathmoveto{\pgfqpoint{3.071936in}{2.492937in}}%
\pgfpathlineto{\pgfqpoint{3.182833in}{2.428911in}}%
\pgfpathlineto{\pgfqpoint{3.182833in}{2.556963in}}%
\pgfpathlineto{\pgfqpoint{3.071936in}{2.620989in}}%
\pgfpathlineto{\pgfqpoint{3.071936in}{2.492937in}}%
\pgfpathclose%
\pgfusepath{fill}%
\end{pgfscope}%
\begin{pgfscope}%
\pgfpathrectangle{\pgfqpoint{0.765000in}{0.660000in}}{\pgfqpoint{4.620000in}{4.620000in}}%
\pgfusepath{clip}%
\pgfsetbuttcap%
\pgfsetroundjoin%
\definecolor{currentfill}{rgb}{1.000000,0.894118,0.788235}%
\pgfsetfillcolor{currentfill}%
\pgfsetlinewidth{0.000000pt}%
\definecolor{currentstroke}{rgb}{1.000000,0.894118,0.788235}%
\pgfsetstrokecolor{currentstroke}%
\pgfsetdash{}{0pt}%
\pgfpathmoveto{\pgfqpoint{2.979627in}{2.512096in}}%
\pgfpathlineto{\pgfqpoint{2.868730in}{2.448069in}}%
\pgfpathlineto{\pgfqpoint{2.979627in}{2.384043in}}%
\pgfpathlineto{\pgfqpoint{3.090524in}{2.448069in}}%
\pgfpathlineto{\pgfqpoint{2.979627in}{2.512096in}}%
\pgfpathclose%
\pgfusepath{fill}%
\end{pgfscope}%
\begin{pgfscope}%
\pgfpathrectangle{\pgfqpoint{0.765000in}{0.660000in}}{\pgfqpoint{4.620000in}{4.620000in}}%
\pgfusepath{clip}%
\pgfsetbuttcap%
\pgfsetroundjoin%
\definecolor{currentfill}{rgb}{1.000000,0.894118,0.788235}%
\pgfsetfillcolor{currentfill}%
\pgfsetlinewidth{0.000000pt}%
\definecolor{currentstroke}{rgb}{1.000000,0.894118,0.788235}%
\pgfsetstrokecolor{currentstroke}%
\pgfsetdash{}{0pt}%
\pgfpathmoveto{\pgfqpoint{2.979627in}{2.512096in}}%
\pgfpathlineto{\pgfqpoint{2.868730in}{2.448069in}}%
\pgfpathlineto{\pgfqpoint{2.868730in}{2.576122in}}%
\pgfpathlineto{\pgfqpoint{2.979627in}{2.640148in}}%
\pgfpathlineto{\pgfqpoint{2.979627in}{2.512096in}}%
\pgfpathclose%
\pgfusepath{fill}%
\end{pgfscope}%
\begin{pgfscope}%
\pgfpathrectangle{\pgfqpoint{0.765000in}{0.660000in}}{\pgfqpoint{4.620000in}{4.620000in}}%
\pgfusepath{clip}%
\pgfsetbuttcap%
\pgfsetroundjoin%
\definecolor{currentfill}{rgb}{1.000000,0.894118,0.788235}%
\pgfsetfillcolor{currentfill}%
\pgfsetlinewidth{0.000000pt}%
\definecolor{currentstroke}{rgb}{1.000000,0.894118,0.788235}%
\pgfsetstrokecolor{currentstroke}%
\pgfsetdash{}{0pt}%
\pgfpathmoveto{\pgfqpoint{2.979627in}{2.512096in}}%
\pgfpathlineto{\pgfqpoint{3.090524in}{2.448069in}}%
\pgfpathlineto{\pgfqpoint{3.090524in}{2.576122in}}%
\pgfpathlineto{\pgfqpoint{2.979627in}{2.640148in}}%
\pgfpathlineto{\pgfqpoint{2.979627in}{2.512096in}}%
\pgfpathclose%
\pgfusepath{fill}%
\end{pgfscope}%
\begin{pgfscope}%
\pgfpathrectangle{\pgfqpoint{0.765000in}{0.660000in}}{\pgfqpoint{4.620000in}{4.620000in}}%
\pgfusepath{clip}%
\pgfsetbuttcap%
\pgfsetroundjoin%
\definecolor{currentfill}{rgb}{1.000000,0.894118,0.788235}%
\pgfsetfillcolor{currentfill}%
\pgfsetlinewidth{0.000000pt}%
\definecolor{currentstroke}{rgb}{1.000000,0.894118,0.788235}%
\pgfsetstrokecolor{currentstroke}%
\pgfsetdash{}{0pt}%
\pgfpathmoveto{\pgfqpoint{3.071936in}{2.620989in}}%
\pgfpathlineto{\pgfqpoint{2.961039in}{2.556963in}}%
\pgfpathlineto{\pgfqpoint{3.071936in}{2.492937in}}%
\pgfpathlineto{\pgfqpoint{3.182833in}{2.556963in}}%
\pgfpathlineto{\pgfqpoint{3.071936in}{2.620989in}}%
\pgfpathclose%
\pgfusepath{fill}%
\end{pgfscope}%
\begin{pgfscope}%
\pgfpathrectangle{\pgfqpoint{0.765000in}{0.660000in}}{\pgfqpoint{4.620000in}{4.620000in}}%
\pgfusepath{clip}%
\pgfsetbuttcap%
\pgfsetroundjoin%
\definecolor{currentfill}{rgb}{1.000000,0.894118,0.788235}%
\pgfsetfillcolor{currentfill}%
\pgfsetlinewidth{0.000000pt}%
\definecolor{currentstroke}{rgb}{1.000000,0.894118,0.788235}%
\pgfsetstrokecolor{currentstroke}%
\pgfsetdash{}{0pt}%
\pgfpathmoveto{\pgfqpoint{3.071936in}{2.364884in}}%
\pgfpathlineto{\pgfqpoint{3.182833in}{2.428911in}}%
\pgfpathlineto{\pgfqpoint{3.182833in}{2.556963in}}%
\pgfpathlineto{\pgfqpoint{3.071936in}{2.492937in}}%
\pgfpathlineto{\pgfqpoint{3.071936in}{2.364884in}}%
\pgfpathclose%
\pgfusepath{fill}%
\end{pgfscope}%
\begin{pgfscope}%
\pgfpathrectangle{\pgfqpoint{0.765000in}{0.660000in}}{\pgfqpoint{4.620000in}{4.620000in}}%
\pgfusepath{clip}%
\pgfsetbuttcap%
\pgfsetroundjoin%
\definecolor{currentfill}{rgb}{1.000000,0.894118,0.788235}%
\pgfsetfillcolor{currentfill}%
\pgfsetlinewidth{0.000000pt}%
\definecolor{currentstroke}{rgb}{1.000000,0.894118,0.788235}%
\pgfsetstrokecolor{currentstroke}%
\pgfsetdash{}{0pt}%
\pgfpathmoveto{\pgfqpoint{2.961039in}{2.428911in}}%
\pgfpathlineto{\pgfqpoint{3.071936in}{2.364884in}}%
\pgfpathlineto{\pgfqpoint{3.071936in}{2.492937in}}%
\pgfpathlineto{\pgfqpoint{2.961039in}{2.556963in}}%
\pgfpathlineto{\pgfqpoint{2.961039in}{2.428911in}}%
\pgfpathclose%
\pgfusepath{fill}%
\end{pgfscope}%
\begin{pgfscope}%
\pgfpathrectangle{\pgfqpoint{0.765000in}{0.660000in}}{\pgfqpoint{4.620000in}{4.620000in}}%
\pgfusepath{clip}%
\pgfsetbuttcap%
\pgfsetroundjoin%
\definecolor{currentfill}{rgb}{1.000000,0.894118,0.788235}%
\pgfsetfillcolor{currentfill}%
\pgfsetlinewidth{0.000000pt}%
\definecolor{currentstroke}{rgb}{1.000000,0.894118,0.788235}%
\pgfsetstrokecolor{currentstroke}%
\pgfsetdash{}{0pt}%
\pgfpathmoveto{\pgfqpoint{2.979627in}{2.640148in}}%
\pgfpathlineto{\pgfqpoint{2.868730in}{2.576122in}}%
\pgfpathlineto{\pgfqpoint{2.979627in}{2.512096in}}%
\pgfpathlineto{\pgfqpoint{3.090524in}{2.576122in}}%
\pgfpathlineto{\pgfqpoint{2.979627in}{2.640148in}}%
\pgfpathclose%
\pgfusepath{fill}%
\end{pgfscope}%
\begin{pgfscope}%
\pgfpathrectangle{\pgfqpoint{0.765000in}{0.660000in}}{\pgfqpoint{4.620000in}{4.620000in}}%
\pgfusepath{clip}%
\pgfsetbuttcap%
\pgfsetroundjoin%
\definecolor{currentfill}{rgb}{1.000000,0.894118,0.788235}%
\pgfsetfillcolor{currentfill}%
\pgfsetlinewidth{0.000000pt}%
\definecolor{currentstroke}{rgb}{1.000000,0.894118,0.788235}%
\pgfsetstrokecolor{currentstroke}%
\pgfsetdash{}{0pt}%
\pgfpathmoveto{\pgfqpoint{2.979627in}{2.384043in}}%
\pgfpathlineto{\pgfqpoint{3.090524in}{2.448069in}}%
\pgfpathlineto{\pgfqpoint{3.090524in}{2.576122in}}%
\pgfpathlineto{\pgfqpoint{2.979627in}{2.512096in}}%
\pgfpathlineto{\pgfqpoint{2.979627in}{2.384043in}}%
\pgfpathclose%
\pgfusepath{fill}%
\end{pgfscope}%
\begin{pgfscope}%
\pgfpathrectangle{\pgfqpoint{0.765000in}{0.660000in}}{\pgfqpoint{4.620000in}{4.620000in}}%
\pgfusepath{clip}%
\pgfsetbuttcap%
\pgfsetroundjoin%
\definecolor{currentfill}{rgb}{1.000000,0.894118,0.788235}%
\pgfsetfillcolor{currentfill}%
\pgfsetlinewidth{0.000000pt}%
\definecolor{currentstroke}{rgb}{1.000000,0.894118,0.788235}%
\pgfsetstrokecolor{currentstroke}%
\pgfsetdash{}{0pt}%
\pgfpathmoveto{\pgfqpoint{2.868730in}{2.448069in}}%
\pgfpathlineto{\pgfqpoint{2.979627in}{2.384043in}}%
\pgfpathlineto{\pgfqpoint{2.979627in}{2.512096in}}%
\pgfpathlineto{\pgfqpoint{2.868730in}{2.576122in}}%
\pgfpathlineto{\pgfqpoint{2.868730in}{2.448069in}}%
\pgfpathclose%
\pgfusepath{fill}%
\end{pgfscope}%
\begin{pgfscope}%
\pgfpathrectangle{\pgfqpoint{0.765000in}{0.660000in}}{\pgfqpoint{4.620000in}{4.620000in}}%
\pgfusepath{clip}%
\pgfsetbuttcap%
\pgfsetroundjoin%
\definecolor{currentfill}{rgb}{1.000000,0.894118,0.788235}%
\pgfsetfillcolor{currentfill}%
\pgfsetlinewidth{0.000000pt}%
\definecolor{currentstroke}{rgb}{1.000000,0.894118,0.788235}%
\pgfsetstrokecolor{currentstroke}%
\pgfsetdash{}{0pt}%
\pgfpathmoveto{\pgfqpoint{3.071936in}{2.492937in}}%
\pgfpathlineto{\pgfqpoint{2.961039in}{2.428911in}}%
\pgfpathlineto{\pgfqpoint{2.868730in}{2.448069in}}%
\pgfpathlineto{\pgfqpoint{2.979627in}{2.512096in}}%
\pgfpathlineto{\pgfqpoint{3.071936in}{2.492937in}}%
\pgfpathclose%
\pgfusepath{fill}%
\end{pgfscope}%
\begin{pgfscope}%
\pgfpathrectangle{\pgfqpoint{0.765000in}{0.660000in}}{\pgfqpoint{4.620000in}{4.620000in}}%
\pgfusepath{clip}%
\pgfsetbuttcap%
\pgfsetroundjoin%
\definecolor{currentfill}{rgb}{1.000000,0.894118,0.788235}%
\pgfsetfillcolor{currentfill}%
\pgfsetlinewidth{0.000000pt}%
\definecolor{currentstroke}{rgb}{1.000000,0.894118,0.788235}%
\pgfsetstrokecolor{currentstroke}%
\pgfsetdash{}{0pt}%
\pgfpathmoveto{\pgfqpoint{3.182833in}{2.428911in}}%
\pgfpathlineto{\pgfqpoint{3.071936in}{2.492937in}}%
\pgfpathlineto{\pgfqpoint{2.979627in}{2.512096in}}%
\pgfpathlineto{\pgfqpoint{3.090524in}{2.448069in}}%
\pgfpathlineto{\pgfqpoint{3.182833in}{2.428911in}}%
\pgfpathclose%
\pgfusepath{fill}%
\end{pgfscope}%
\begin{pgfscope}%
\pgfpathrectangle{\pgfqpoint{0.765000in}{0.660000in}}{\pgfqpoint{4.620000in}{4.620000in}}%
\pgfusepath{clip}%
\pgfsetbuttcap%
\pgfsetroundjoin%
\definecolor{currentfill}{rgb}{1.000000,0.894118,0.788235}%
\pgfsetfillcolor{currentfill}%
\pgfsetlinewidth{0.000000pt}%
\definecolor{currentstroke}{rgb}{1.000000,0.894118,0.788235}%
\pgfsetstrokecolor{currentstroke}%
\pgfsetdash{}{0pt}%
\pgfpathmoveto{\pgfqpoint{3.071936in}{2.492937in}}%
\pgfpathlineto{\pgfqpoint{3.071936in}{2.620989in}}%
\pgfpathlineto{\pgfqpoint{2.979627in}{2.640148in}}%
\pgfpathlineto{\pgfqpoint{3.090524in}{2.448069in}}%
\pgfpathlineto{\pgfqpoint{3.071936in}{2.492937in}}%
\pgfpathclose%
\pgfusepath{fill}%
\end{pgfscope}%
\begin{pgfscope}%
\pgfpathrectangle{\pgfqpoint{0.765000in}{0.660000in}}{\pgfqpoint{4.620000in}{4.620000in}}%
\pgfusepath{clip}%
\pgfsetbuttcap%
\pgfsetroundjoin%
\definecolor{currentfill}{rgb}{1.000000,0.894118,0.788235}%
\pgfsetfillcolor{currentfill}%
\pgfsetlinewidth{0.000000pt}%
\definecolor{currentstroke}{rgb}{1.000000,0.894118,0.788235}%
\pgfsetstrokecolor{currentstroke}%
\pgfsetdash{}{0pt}%
\pgfpathmoveto{\pgfqpoint{3.182833in}{2.428911in}}%
\pgfpathlineto{\pgfqpoint{3.182833in}{2.556963in}}%
\pgfpathlineto{\pgfqpoint{3.090524in}{2.576122in}}%
\pgfpathlineto{\pgfqpoint{2.979627in}{2.512096in}}%
\pgfpathlineto{\pgfqpoint{3.182833in}{2.428911in}}%
\pgfpathclose%
\pgfusepath{fill}%
\end{pgfscope}%
\begin{pgfscope}%
\pgfpathrectangle{\pgfqpoint{0.765000in}{0.660000in}}{\pgfqpoint{4.620000in}{4.620000in}}%
\pgfusepath{clip}%
\pgfsetbuttcap%
\pgfsetroundjoin%
\definecolor{currentfill}{rgb}{1.000000,0.894118,0.788235}%
\pgfsetfillcolor{currentfill}%
\pgfsetlinewidth{0.000000pt}%
\definecolor{currentstroke}{rgb}{1.000000,0.894118,0.788235}%
\pgfsetstrokecolor{currentstroke}%
\pgfsetdash{}{0pt}%
\pgfpathmoveto{\pgfqpoint{2.961039in}{2.428911in}}%
\pgfpathlineto{\pgfqpoint{3.071936in}{2.364884in}}%
\pgfpathlineto{\pgfqpoint{2.979627in}{2.384043in}}%
\pgfpathlineto{\pgfqpoint{2.868730in}{2.448069in}}%
\pgfpathlineto{\pgfqpoint{2.961039in}{2.428911in}}%
\pgfpathclose%
\pgfusepath{fill}%
\end{pgfscope}%
\begin{pgfscope}%
\pgfpathrectangle{\pgfqpoint{0.765000in}{0.660000in}}{\pgfqpoint{4.620000in}{4.620000in}}%
\pgfusepath{clip}%
\pgfsetbuttcap%
\pgfsetroundjoin%
\definecolor{currentfill}{rgb}{1.000000,0.894118,0.788235}%
\pgfsetfillcolor{currentfill}%
\pgfsetlinewidth{0.000000pt}%
\definecolor{currentstroke}{rgb}{1.000000,0.894118,0.788235}%
\pgfsetstrokecolor{currentstroke}%
\pgfsetdash{}{0pt}%
\pgfpathmoveto{\pgfqpoint{3.071936in}{2.364884in}}%
\pgfpathlineto{\pgfqpoint{3.182833in}{2.428911in}}%
\pgfpathlineto{\pgfqpoint{3.090524in}{2.448069in}}%
\pgfpathlineto{\pgfqpoint{2.979627in}{2.384043in}}%
\pgfpathlineto{\pgfqpoint{3.071936in}{2.364884in}}%
\pgfpathclose%
\pgfusepath{fill}%
\end{pgfscope}%
\begin{pgfscope}%
\pgfpathrectangle{\pgfqpoint{0.765000in}{0.660000in}}{\pgfqpoint{4.620000in}{4.620000in}}%
\pgfusepath{clip}%
\pgfsetbuttcap%
\pgfsetroundjoin%
\definecolor{currentfill}{rgb}{1.000000,0.894118,0.788235}%
\pgfsetfillcolor{currentfill}%
\pgfsetlinewidth{0.000000pt}%
\definecolor{currentstroke}{rgb}{1.000000,0.894118,0.788235}%
\pgfsetstrokecolor{currentstroke}%
\pgfsetdash{}{0pt}%
\pgfpathmoveto{\pgfqpoint{3.071936in}{2.620989in}}%
\pgfpathlineto{\pgfqpoint{2.961039in}{2.556963in}}%
\pgfpathlineto{\pgfqpoint{2.868730in}{2.576122in}}%
\pgfpathlineto{\pgfqpoint{2.979627in}{2.640148in}}%
\pgfpathlineto{\pgfqpoint{3.071936in}{2.620989in}}%
\pgfpathclose%
\pgfusepath{fill}%
\end{pgfscope}%
\begin{pgfscope}%
\pgfpathrectangle{\pgfqpoint{0.765000in}{0.660000in}}{\pgfqpoint{4.620000in}{4.620000in}}%
\pgfusepath{clip}%
\pgfsetbuttcap%
\pgfsetroundjoin%
\definecolor{currentfill}{rgb}{1.000000,0.894118,0.788235}%
\pgfsetfillcolor{currentfill}%
\pgfsetlinewidth{0.000000pt}%
\definecolor{currentstroke}{rgb}{1.000000,0.894118,0.788235}%
\pgfsetstrokecolor{currentstroke}%
\pgfsetdash{}{0pt}%
\pgfpathmoveto{\pgfqpoint{3.182833in}{2.556963in}}%
\pgfpathlineto{\pgfqpoint{3.071936in}{2.620989in}}%
\pgfpathlineto{\pgfqpoint{2.979627in}{2.640148in}}%
\pgfpathlineto{\pgfqpoint{3.090524in}{2.576122in}}%
\pgfpathlineto{\pgfqpoint{3.182833in}{2.556963in}}%
\pgfpathclose%
\pgfusepath{fill}%
\end{pgfscope}%
\begin{pgfscope}%
\pgfpathrectangle{\pgfqpoint{0.765000in}{0.660000in}}{\pgfqpoint{4.620000in}{4.620000in}}%
\pgfusepath{clip}%
\pgfsetbuttcap%
\pgfsetroundjoin%
\definecolor{currentfill}{rgb}{1.000000,0.894118,0.788235}%
\pgfsetfillcolor{currentfill}%
\pgfsetlinewidth{0.000000pt}%
\definecolor{currentstroke}{rgb}{1.000000,0.894118,0.788235}%
\pgfsetstrokecolor{currentstroke}%
\pgfsetdash{}{0pt}%
\pgfpathmoveto{\pgfqpoint{2.961039in}{2.428911in}}%
\pgfpathlineto{\pgfqpoint{2.961039in}{2.556963in}}%
\pgfpathlineto{\pgfqpoint{2.868730in}{2.576122in}}%
\pgfpathlineto{\pgfqpoint{2.979627in}{2.384043in}}%
\pgfpathlineto{\pgfqpoint{2.961039in}{2.428911in}}%
\pgfpathclose%
\pgfusepath{fill}%
\end{pgfscope}%
\begin{pgfscope}%
\pgfpathrectangle{\pgfqpoint{0.765000in}{0.660000in}}{\pgfqpoint{4.620000in}{4.620000in}}%
\pgfusepath{clip}%
\pgfsetbuttcap%
\pgfsetroundjoin%
\definecolor{currentfill}{rgb}{1.000000,0.894118,0.788235}%
\pgfsetfillcolor{currentfill}%
\pgfsetlinewidth{0.000000pt}%
\definecolor{currentstroke}{rgb}{1.000000,0.894118,0.788235}%
\pgfsetstrokecolor{currentstroke}%
\pgfsetdash{}{0pt}%
\pgfpathmoveto{\pgfqpoint{3.071936in}{2.364884in}}%
\pgfpathlineto{\pgfqpoint{3.071936in}{2.492937in}}%
\pgfpathlineto{\pgfqpoint{2.979627in}{2.512096in}}%
\pgfpathlineto{\pgfqpoint{2.868730in}{2.448069in}}%
\pgfpathlineto{\pgfqpoint{3.071936in}{2.364884in}}%
\pgfpathclose%
\pgfusepath{fill}%
\end{pgfscope}%
\begin{pgfscope}%
\pgfpathrectangle{\pgfqpoint{0.765000in}{0.660000in}}{\pgfqpoint{4.620000in}{4.620000in}}%
\pgfusepath{clip}%
\pgfsetbuttcap%
\pgfsetroundjoin%
\definecolor{currentfill}{rgb}{1.000000,0.894118,0.788235}%
\pgfsetfillcolor{currentfill}%
\pgfsetlinewidth{0.000000pt}%
\definecolor{currentstroke}{rgb}{1.000000,0.894118,0.788235}%
\pgfsetstrokecolor{currentstroke}%
\pgfsetdash{}{0pt}%
\pgfpathmoveto{\pgfqpoint{3.071936in}{2.492937in}}%
\pgfpathlineto{\pgfqpoint{3.182833in}{2.556963in}}%
\pgfpathlineto{\pgfqpoint{3.090524in}{2.576122in}}%
\pgfpathlineto{\pgfqpoint{2.979627in}{2.512096in}}%
\pgfpathlineto{\pgfqpoint{3.071936in}{2.492937in}}%
\pgfpathclose%
\pgfusepath{fill}%
\end{pgfscope}%
\begin{pgfscope}%
\pgfpathrectangle{\pgfqpoint{0.765000in}{0.660000in}}{\pgfqpoint{4.620000in}{4.620000in}}%
\pgfusepath{clip}%
\pgfsetbuttcap%
\pgfsetroundjoin%
\definecolor{currentfill}{rgb}{1.000000,0.894118,0.788235}%
\pgfsetfillcolor{currentfill}%
\pgfsetlinewidth{0.000000pt}%
\definecolor{currentstroke}{rgb}{1.000000,0.894118,0.788235}%
\pgfsetstrokecolor{currentstroke}%
\pgfsetdash{}{0pt}%
\pgfpathmoveto{\pgfqpoint{2.961039in}{2.556963in}}%
\pgfpathlineto{\pgfqpoint{3.071936in}{2.492937in}}%
\pgfpathlineto{\pgfqpoint{2.979627in}{2.512096in}}%
\pgfpathlineto{\pgfqpoint{2.868730in}{2.576122in}}%
\pgfpathlineto{\pgfqpoint{2.961039in}{2.556963in}}%
\pgfpathclose%
\pgfusepath{fill}%
\end{pgfscope}%
\begin{pgfscope}%
\pgfpathrectangle{\pgfqpoint{0.765000in}{0.660000in}}{\pgfqpoint{4.620000in}{4.620000in}}%
\pgfusepath{clip}%
\pgfsetbuttcap%
\pgfsetroundjoin%
\definecolor{currentfill}{rgb}{1.000000,0.894118,0.788235}%
\pgfsetfillcolor{currentfill}%
\pgfsetlinewidth{0.000000pt}%
\definecolor{currentstroke}{rgb}{1.000000,0.894118,0.788235}%
\pgfsetstrokecolor{currentstroke}%
\pgfsetdash{}{0pt}%
\pgfpathmoveto{\pgfqpoint{3.071936in}{2.492937in}}%
\pgfpathlineto{\pgfqpoint{2.961039in}{2.428911in}}%
\pgfpathlineto{\pgfqpoint{3.071936in}{2.364884in}}%
\pgfpathlineto{\pgfqpoint{3.182833in}{2.428911in}}%
\pgfpathlineto{\pgfqpoint{3.071936in}{2.492937in}}%
\pgfpathclose%
\pgfusepath{fill}%
\end{pgfscope}%
\begin{pgfscope}%
\pgfpathrectangle{\pgfqpoint{0.765000in}{0.660000in}}{\pgfqpoint{4.620000in}{4.620000in}}%
\pgfusepath{clip}%
\pgfsetbuttcap%
\pgfsetroundjoin%
\definecolor{currentfill}{rgb}{1.000000,0.894118,0.788235}%
\pgfsetfillcolor{currentfill}%
\pgfsetlinewidth{0.000000pt}%
\definecolor{currentstroke}{rgb}{1.000000,0.894118,0.788235}%
\pgfsetstrokecolor{currentstroke}%
\pgfsetdash{}{0pt}%
\pgfpathmoveto{\pgfqpoint{3.071936in}{2.492937in}}%
\pgfpathlineto{\pgfqpoint{2.961039in}{2.428911in}}%
\pgfpathlineto{\pgfqpoint{2.961039in}{2.556963in}}%
\pgfpathlineto{\pgfqpoint{3.071936in}{2.620989in}}%
\pgfpathlineto{\pgfqpoint{3.071936in}{2.492937in}}%
\pgfpathclose%
\pgfusepath{fill}%
\end{pgfscope}%
\begin{pgfscope}%
\pgfpathrectangle{\pgfqpoint{0.765000in}{0.660000in}}{\pgfqpoint{4.620000in}{4.620000in}}%
\pgfusepath{clip}%
\pgfsetbuttcap%
\pgfsetroundjoin%
\definecolor{currentfill}{rgb}{1.000000,0.894118,0.788235}%
\pgfsetfillcolor{currentfill}%
\pgfsetlinewidth{0.000000pt}%
\definecolor{currentstroke}{rgb}{1.000000,0.894118,0.788235}%
\pgfsetstrokecolor{currentstroke}%
\pgfsetdash{}{0pt}%
\pgfpathmoveto{\pgfqpoint{3.071936in}{2.492937in}}%
\pgfpathlineto{\pgfqpoint{3.182833in}{2.428911in}}%
\pgfpathlineto{\pgfqpoint{3.182833in}{2.556963in}}%
\pgfpathlineto{\pgfqpoint{3.071936in}{2.620989in}}%
\pgfpathlineto{\pgfqpoint{3.071936in}{2.492937in}}%
\pgfpathclose%
\pgfusepath{fill}%
\end{pgfscope}%
\begin{pgfscope}%
\pgfpathrectangle{\pgfqpoint{0.765000in}{0.660000in}}{\pgfqpoint{4.620000in}{4.620000in}}%
\pgfusepath{clip}%
\pgfsetbuttcap%
\pgfsetroundjoin%
\definecolor{currentfill}{rgb}{1.000000,0.894118,0.788235}%
\pgfsetfillcolor{currentfill}%
\pgfsetlinewidth{0.000000pt}%
\definecolor{currentstroke}{rgb}{1.000000,0.894118,0.788235}%
\pgfsetstrokecolor{currentstroke}%
\pgfsetdash{}{0pt}%
\pgfpathmoveto{\pgfqpoint{3.120119in}{2.489790in}}%
\pgfpathlineto{\pgfqpoint{3.009222in}{2.425764in}}%
\pgfpathlineto{\pgfqpoint{3.120119in}{2.361737in}}%
\pgfpathlineto{\pgfqpoint{3.231016in}{2.425764in}}%
\pgfpathlineto{\pgfqpoint{3.120119in}{2.489790in}}%
\pgfpathclose%
\pgfusepath{fill}%
\end{pgfscope}%
\begin{pgfscope}%
\pgfpathrectangle{\pgfqpoint{0.765000in}{0.660000in}}{\pgfqpoint{4.620000in}{4.620000in}}%
\pgfusepath{clip}%
\pgfsetbuttcap%
\pgfsetroundjoin%
\definecolor{currentfill}{rgb}{1.000000,0.894118,0.788235}%
\pgfsetfillcolor{currentfill}%
\pgfsetlinewidth{0.000000pt}%
\definecolor{currentstroke}{rgb}{1.000000,0.894118,0.788235}%
\pgfsetstrokecolor{currentstroke}%
\pgfsetdash{}{0pt}%
\pgfpathmoveto{\pgfqpoint{3.120119in}{2.489790in}}%
\pgfpathlineto{\pgfqpoint{3.009222in}{2.425764in}}%
\pgfpathlineto{\pgfqpoint{3.009222in}{2.553816in}}%
\pgfpathlineto{\pgfqpoint{3.120119in}{2.617842in}}%
\pgfpathlineto{\pgfqpoint{3.120119in}{2.489790in}}%
\pgfpathclose%
\pgfusepath{fill}%
\end{pgfscope}%
\begin{pgfscope}%
\pgfpathrectangle{\pgfqpoint{0.765000in}{0.660000in}}{\pgfqpoint{4.620000in}{4.620000in}}%
\pgfusepath{clip}%
\pgfsetbuttcap%
\pgfsetroundjoin%
\definecolor{currentfill}{rgb}{1.000000,0.894118,0.788235}%
\pgfsetfillcolor{currentfill}%
\pgfsetlinewidth{0.000000pt}%
\definecolor{currentstroke}{rgb}{1.000000,0.894118,0.788235}%
\pgfsetstrokecolor{currentstroke}%
\pgfsetdash{}{0pt}%
\pgfpathmoveto{\pgfqpoint{3.120119in}{2.489790in}}%
\pgfpathlineto{\pgfqpoint{3.231016in}{2.425764in}}%
\pgfpathlineto{\pgfqpoint{3.231016in}{2.553816in}}%
\pgfpathlineto{\pgfqpoint{3.120119in}{2.617842in}}%
\pgfpathlineto{\pgfqpoint{3.120119in}{2.489790in}}%
\pgfpathclose%
\pgfusepath{fill}%
\end{pgfscope}%
\begin{pgfscope}%
\pgfpathrectangle{\pgfqpoint{0.765000in}{0.660000in}}{\pgfqpoint{4.620000in}{4.620000in}}%
\pgfusepath{clip}%
\pgfsetbuttcap%
\pgfsetroundjoin%
\definecolor{currentfill}{rgb}{1.000000,0.894118,0.788235}%
\pgfsetfillcolor{currentfill}%
\pgfsetlinewidth{0.000000pt}%
\definecolor{currentstroke}{rgb}{1.000000,0.894118,0.788235}%
\pgfsetstrokecolor{currentstroke}%
\pgfsetdash{}{0pt}%
\pgfpathmoveto{\pgfqpoint{3.071936in}{2.620989in}}%
\pgfpathlineto{\pgfqpoint{2.961039in}{2.556963in}}%
\pgfpathlineto{\pgfqpoint{3.071936in}{2.492937in}}%
\pgfpathlineto{\pgfqpoint{3.182833in}{2.556963in}}%
\pgfpathlineto{\pgfqpoint{3.071936in}{2.620989in}}%
\pgfpathclose%
\pgfusepath{fill}%
\end{pgfscope}%
\begin{pgfscope}%
\pgfpathrectangle{\pgfqpoint{0.765000in}{0.660000in}}{\pgfqpoint{4.620000in}{4.620000in}}%
\pgfusepath{clip}%
\pgfsetbuttcap%
\pgfsetroundjoin%
\definecolor{currentfill}{rgb}{1.000000,0.894118,0.788235}%
\pgfsetfillcolor{currentfill}%
\pgfsetlinewidth{0.000000pt}%
\definecolor{currentstroke}{rgb}{1.000000,0.894118,0.788235}%
\pgfsetstrokecolor{currentstroke}%
\pgfsetdash{}{0pt}%
\pgfpathmoveto{\pgfqpoint{3.071936in}{2.364884in}}%
\pgfpathlineto{\pgfqpoint{3.182833in}{2.428911in}}%
\pgfpathlineto{\pgfqpoint{3.182833in}{2.556963in}}%
\pgfpathlineto{\pgfqpoint{3.071936in}{2.492937in}}%
\pgfpathlineto{\pgfqpoint{3.071936in}{2.364884in}}%
\pgfpathclose%
\pgfusepath{fill}%
\end{pgfscope}%
\begin{pgfscope}%
\pgfpathrectangle{\pgfqpoint{0.765000in}{0.660000in}}{\pgfqpoint{4.620000in}{4.620000in}}%
\pgfusepath{clip}%
\pgfsetbuttcap%
\pgfsetroundjoin%
\definecolor{currentfill}{rgb}{1.000000,0.894118,0.788235}%
\pgfsetfillcolor{currentfill}%
\pgfsetlinewidth{0.000000pt}%
\definecolor{currentstroke}{rgb}{1.000000,0.894118,0.788235}%
\pgfsetstrokecolor{currentstroke}%
\pgfsetdash{}{0pt}%
\pgfpathmoveto{\pgfqpoint{2.961039in}{2.428911in}}%
\pgfpathlineto{\pgfqpoint{3.071936in}{2.364884in}}%
\pgfpathlineto{\pgfqpoint{3.071936in}{2.492937in}}%
\pgfpathlineto{\pgfqpoint{2.961039in}{2.556963in}}%
\pgfpathlineto{\pgfqpoint{2.961039in}{2.428911in}}%
\pgfpathclose%
\pgfusepath{fill}%
\end{pgfscope}%
\begin{pgfscope}%
\pgfpathrectangle{\pgfqpoint{0.765000in}{0.660000in}}{\pgfqpoint{4.620000in}{4.620000in}}%
\pgfusepath{clip}%
\pgfsetbuttcap%
\pgfsetroundjoin%
\definecolor{currentfill}{rgb}{1.000000,0.894118,0.788235}%
\pgfsetfillcolor{currentfill}%
\pgfsetlinewidth{0.000000pt}%
\definecolor{currentstroke}{rgb}{1.000000,0.894118,0.788235}%
\pgfsetstrokecolor{currentstroke}%
\pgfsetdash{}{0pt}%
\pgfpathmoveto{\pgfqpoint{3.120119in}{2.617842in}}%
\pgfpathlineto{\pgfqpoint{3.009222in}{2.553816in}}%
\pgfpathlineto{\pgfqpoint{3.120119in}{2.489790in}}%
\pgfpathlineto{\pgfqpoint{3.231016in}{2.553816in}}%
\pgfpathlineto{\pgfqpoint{3.120119in}{2.617842in}}%
\pgfpathclose%
\pgfusepath{fill}%
\end{pgfscope}%
\begin{pgfscope}%
\pgfpathrectangle{\pgfqpoint{0.765000in}{0.660000in}}{\pgfqpoint{4.620000in}{4.620000in}}%
\pgfusepath{clip}%
\pgfsetbuttcap%
\pgfsetroundjoin%
\definecolor{currentfill}{rgb}{1.000000,0.894118,0.788235}%
\pgfsetfillcolor{currentfill}%
\pgfsetlinewidth{0.000000pt}%
\definecolor{currentstroke}{rgb}{1.000000,0.894118,0.788235}%
\pgfsetstrokecolor{currentstroke}%
\pgfsetdash{}{0pt}%
\pgfpathmoveto{\pgfqpoint{3.120119in}{2.361737in}}%
\pgfpathlineto{\pgfqpoint{3.231016in}{2.425764in}}%
\pgfpathlineto{\pgfqpoint{3.231016in}{2.553816in}}%
\pgfpathlineto{\pgfqpoint{3.120119in}{2.489790in}}%
\pgfpathlineto{\pgfqpoint{3.120119in}{2.361737in}}%
\pgfpathclose%
\pgfusepath{fill}%
\end{pgfscope}%
\begin{pgfscope}%
\pgfpathrectangle{\pgfqpoint{0.765000in}{0.660000in}}{\pgfqpoint{4.620000in}{4.620000in}}%
\pgfusepath{clip}%
\pgfsetbuttcap%
\pgfsetroundjoin%
\definecolor{currentfill}{rgb}{1.000000,0.894118,0.788235}%
\pgfsetfillcolor{currentfill}%
\pgfsetlinewidth{0.000000pt}%
\definecolor{currentstroke}{rgb}{1.000000,0.894118,0.788235}%
\pgfsetstrokecolor{currentstroke}%
\pgfsetdash{}{0pt}%
\pgfpathmoveto{\pgfqpoint{3.009222in}{2.425764in}}%
\pgfpathlineto{\pgfqpoint{3.120119in}{2.361737in}}%
\pgfpathlineto{\pgfqpoint{3.120119in}{2.489790in}}%
\pgfpathlineto{\pgfqpoint{3.009222in}{2.553816in}}%
\pgfpathlineto{\pgfqpoint{3.009222in}{2.425764in}}%
\pgfpathclose%
\pgfusepath{fill}%
\end{pgfscope}%
\begin{pgfscope}%
\pgfpathrectangle{\pgfqpoint{0.765000in}{0.660000in}}{\pgfqpoint{4.620000in}{4.620000in}}%
\pgfusepath{clip}%
\pgfsetbuttcap%
\pgfsetroundjoin%
\definecolor{currentfill}{rgb}{1.000000,0.894118,0.788235}%
\pgfsetfillcolor{currentfill}%
\pgfsetlinewidth{0.000000pt}%
\definecolor{currentstroke}{rgb}{1.000000,0.894118,0.788235}%
\pgfsetstrokecolor{currentstroke}%
\pgfsetdash{}{0pt}%
\pgfpathmoveto{\pgfqpoint{3.071936in}{2.492937in}}%
\pgfpathlineto{\pgfqpoint{2.961039in}{2.428911in}}%
\pgfpathlineto{\pgfqpoint{3.009222in}{2.425764in}}%
\pgfpathlineto{\pgfqpoint{3.120119in}{2.489790in}}%
\pgfpathlineto{\pgfqpoint{3.071936in}{2.492937in}}%
\pgfpathclose%
\pgfusepath{fill}%
\end{pgfscope}%
\begin{pgfscope}%
\pgfpathrectangle{\pgfqpoint{0.765000in}{0.660000in}}{\pgfqpoint{4.620000in}{4.620000in}}%
\pgfusepath{clip}%
\pgfsetbuttcap%
\pgfsetroundjoin%
\definecolor{currentfill}{rgb}{1.000000,0.894118,0.788235}%
\pgfsetfillcolor{currentfill}%
\pgfsetlinewidth{0.000000pt}%
\definecolor{currentstroke}{rgb}{1.000000,0.894118,0.788235}%
\pgfsetstrokecolor{currentstroke}%
\pgfsetdash{}{0pt}%
\pgfpathmoveto{\pgfqpoint{3.182833in}{2.428911in}}%
\pgfpathlineto{\pgfqpoint{3.071936in}{2.492937in}}%
\pgfpathlineto{\pgfqpoint{3.120119in}{2.489790in}}%
\pgfpathlineto{\pgfqpoint{3.231016in}{2.425764in}}%
\pgfpathlineto{\pgfqpoint{3.182833in}{2.428911in}}%
\pgfpathclose%
\pgfusepath{fill}%
\end{pgfscope}%
\begin{pgfscope}%
\pgfpathrectangle{\pgfqpoint{0.765000in}{0.660000in}}{\pgfqpoint{4.620000in}{4.620000in}}%
\pgfusepath{clip}%
\pgfsetbuttcap%
\pgfsetroundjoin%
\definecolor{currentfill}{rgb}{1.000000,0.894118,0.788235}%
\pgfsetfillcolor{currentfill}%
\pgfsetlinewidth{0.000000pt}%
\definecolor{currentstroke}{rgb}{1.000000,0.894118,0.788235}%
\pgfsetstrokecolor{currentstroke}%
\pgfsetdash{}{0pt}%
\pgfpathmoveto{\pgfqpoint{3.071936in}{2.492937in}}%
\pgfpathlineto{\pgfqpoint{3.071936in}{2.620989in}}%
\pgfpathlineto{\pgfqpoint{3.120119in}{2.617842in}}%
\pgfpathlineto{\pgfqpoint{3.231016in}{2.425764in}}%
\pgfpathlineto{\pgfqpoint{3.071936in}{2.492937in}}%
\pgfpathclose%
\pgfusepath{fill}%
\end{pgfscope}%
\begin{pgfscope}%
\pgfpathrectangle{\pgfqpoint{0.765000in}{0.660000in}}{\pgfqpoint{4.620000in}{4.620000in}}%
\pgfusepath{clip}%
\pgfsetbuttcap%
\pgfsetroundjoin%
\definecolor{currentfill}{rgb}{1.000000,0.894118,0.788235}%
\pgfsetfillcolor{currentfill}%
\pgfsetlinewidth{0.000000pt}%
\definecolor{currentstroke}{rgb}{1.000000,0.894118,0.788235}%
\pgfsetstrokecolor{currentstroke}%
\pgfsetdash{}{0pt}%
\pgfpathmoveto{\pgfqpoint{3.182833in}{2.428911in}}%
\pgfpathlineto{\pgfqpoint{3.182833in}{2.556963in}}%
\pgfpathlineto{\pgfqpoint{3.231016in}{2.553816in}}%
\pgfpathlineto{\pgfqpoint{3.120119in}{2.489790in}}%
\pgfpathlineto{\pgfqpoint{3.182833in}{2.428911in}}%
\pgfpathclose%
\pgfusepath{fill}%
\end{pgfscope}%
\begin{pgfscope}%
\pgfpathrectangle{\pgfqpoint{0.765000in}{0.660000in}}{\pgfqpoint{4.620000in}{4.620000in}}%
\pgfusepath{clip}%
\pgfsetbuttcap%
\pgfsetroundjoin%
\definecolor{currentfill}{rgb}{1.000000,0.894118,0.788235}%
\pgfsetfillcolor{currentfill}%
\pgfsetlinewidth{0.000000pt}%
\definecolor{currentstroke}{rgb}{1.000000,0.894118,0.788235}%
\pgfsetstrokecolor{currentstroke}%
\pgfsetdash{}{0pt}%
\pgfpathmoveto{\pgfqpoint{2.961039in}{2.428911in}}%
\pgfpathlineto{\pgfqpoint{3.071936in}{2.364884in}}%
\pgfpathlineto{\pgfqpoint{3.120119in}{2.361737in}}%
\pgfpathlineto{\pgfqpoint{3.009222in}{2.425764in}}%
\pgfpathlineto{\pgfqpoint{2.961039in}{2.428911in}}%
\pgfpathclose%
\pgfusepath{fill}%
\end{pgfscope}%
\begin{pgfscope}%
\pgfpathrectangle{\pgfqpoint{0.765000in}{0.660000in}}{\pgfqpoint{4.620000in}{4.620000in}}%
\pgfusepath{clip}%
\pgfsetbuttcap%
\pgfsetroundjoin%
\definecolor{currentfill}{rgb}{1.000000,0.894118,0.788235}%
\pgfsetfillcolor{currentfill}%
\pgfsetlinewidth{0.000000pt}%
\definecolor{currentstroke}{rgb}{1.000000,0.894118,0.788235}%
\pgfsetstrokecolor{currentstroke}%
\pgfsetdash{}{0pt}%
\pgfpathmoveto{\pgfqpoint{3.071936in}{2.364884in}}%
\pgfpathlineto{\pgfqpoint{3.182833in}{2.428911in}}%
\pgfpathlineto{\pgfqpoint{3.231016in}{2.425764in}}%
\pgfpathlineto{\pgfqpoint{3.120119in}{2.361737in}}%
\pgfpathlineto{\pgfqpoint{3.071936in}{2.364884in}}%
\pgfpathclose%
\pgfusepath{fill}%
\end{pgfscope}%
\begin{pgfscope}%
\pgfpathrectangle{\pgfqpoint{0.765000in}{0.660000in}}{\pgfqpoint{4.620000in}{4.620000in}}%
\pgfusepath{clip}%
\pgfsetbuttcap%
\pgfsetroundjoin%
\definecolor{currentfill}{rgb}{1.000000,0.894118,0.788235}%
\pgfsetfillcolor{currentfill}%
\pgfsetlinewidth{0.000000pt}%
\definecolor{currentstroke}{rgb}{1.000000,0.894118,0.788235}%
\pgfsetstrokecolor{currentstroke}%
\pgfsetdash{}{0pt}%
\pgfpathmoveto{\pgfqpoint{3.071936in}{2.620989in}}%
\pgfpathlineto{\pgfqpoint{2.961039in}{2.556963in}}%
\pgfpathlineto{\pgfqpoint{3.009222in}{2.553816in}}%
\pgfpathlineto{\pgfqpoint{3.120119in}{2.617842in}}%
\pgfpathlineto{\pgfqpoint{3.071936in}{2.620989in}}%
\pgfpathclose%
\pgfusepath{fill}%
\end{pgfscope}%
\begin{pgfscope}%
\pgfpathrectangle{\pgfqpoint{0.765000in}{0.660000in}}{\pgfqpoint{4.620000in}{4.620000in}}%
\pgfusepath{clip}%
\pgfsetbuttcap%
\pgfsetroundjoin%
\definecolor{currentfill}{rgb}{1.000000,0.894118,0.788235}%
\pgfsetfillcolor{currentfill}%
\pgfsetlinewidth{0.000000pt}%
\definecolor{currentstroke}{rgb}{1.000000,0.894118,0.788235}%
\pgfsetstrokecolor{currentstroke}%
\pgfsetdash{}{0pt}%
\pgfpathmoveto{\pgfqpoint{3.182833in}{2.556963in}}%
\pgfpathlineto{\pgfqpoint{3.071936in}{2.620989in}}%
\pgfpathlineto{\pgfqpoint{3.120119in}{2.617842in}}%
\pgfpathlineto{\pgfqpoint{3.231016in}{2.553816in}}%
\pgfpathlineto{\pgfqpoint{3.182833in}{2.556963in}}%
\pgfpathclose%
\pgfusepath{fill}%
\end{pgfscope}%
\begin{pgfscope}%
\pgfpathrectangle{\pgfqpoint{0.765000in}{0.660000in}}{\pgfqpoint{4.620000in}{4.620000in}}%
\pgfusepath{clip}%
\pgfsetbuttcap%
\pgfsetroundjoin%
\definecolor{currentfill}{rgb}{1.000000,0.894118,0.788235}%
\pgfsetfillcolor{currentfill}%
\pgfsetlinewidth{0.000000pt}%
\definecolor{currentstroke}{rgb}{1.000000,0.894118,0.788235}%
\pgfsetstrokecolor{currentstroke}%
\pgfsetdash{}{0pt}%
\pgfpathmoveto{\pgfqpoint{2.961039in}{2.428911in}}%
\pgfpathlineto{\pgfqpoint{2.961039in}{2.556963in}}%
\pgfpathlineto{\pgfqpoint{3.009222in}{2.553816in}}%
\pgfpathlineto{\pgfqpoint{3.120119in}{2.361737in}}%
\pgfpathlineto{\pgfqpoint{2.961039in}{2.428911in}}%
\pgfpathclose%
\pgfusepath{fill}%
\end{pgfscope}%
\begin{pgfscope}%
\pgfpathrectangle{\pgfqpoint{0.765000in}{0.660000in}}{\pgfqpoint{4.620000in}{4.620000in}}%
\pgfusepath{clip}%
\pgfsetbuttcap%
\pgfsetroundjoin%
\definecolor{currentfill}{rgb}{1.000000,0.894118,0.788235}%
\pgfsetfillcolor{currentfill}%
\pgfsetlinewidth{0.000000pt}%
\definecolor{currentstroke}{rgb}{1.000000,0.894118,0.788235}%
\pgfsetstrokecolor{currentstroke}%
\pgfsetdash{}{0pt}%
\pgfpathmoveto{\pgfqpoint{3.071936in}{2.364884in}}%
\pgfpathlineto{\pgfqpoint{3.071936in}{2.492937in}}%
\pgfpathlineto{\pgfqpoint{3.120119in}{2.489790in}}%
\pgfpathlineto{\pgfqpoint{3.009222in}{2.425764in}}%
\pgfpathlineto{\pgfqpoint{3.071936in}{2.364884in}}%
\pgfpathclose%
\pgfusepath{fill}%
\end{pgfscope}%
\begin{pgfscope}%
\pgfpathrectangle{\pgfqpoint{0.765000in}{0.660000in}}{\pgfqpoint{4.620000in}{4.620000in}}%
\pgfusepath{clip}%
\pgfsetbuttcap%
\pgfsetroundjoin%
\definecolor{currentfill}{rgb}{1.000000,0.894118,0.788235}%
\pgfsetfillcolor{currentfill}%
\pgfsetlinewidth{0.000000pt}%
\definecolor{currentstroke}{rgb}{1.000000,0.894118,0.788235}%
\pgfsetstrokecolor{currentstroke}%
\pgfsetdash{}{0pt}%
\pgfpathmoveto{\pgfqpoint{3.071936in}{2.492937in}}%
\pgfpathlineto{\pgfqpoint{3.182833in}{2.556963in}}%
\pgfpathlineto{\pgfqpoint{3.231016in}{2.553816in}}%
\pgfpathlineto{\pgfqpoint{3.120119in}{2.489790in}}%
\pgfpathlineto{\pgfqpoint{3.071936in}{2.492937in}}%
\pgfpathclose%
\pgfusepath{fill}%
\end{pgfscope}%
\begin{pgfscope}%
\pgfpathrectangle{\pgfqpoint{0.765000in}{0.660000in}}{\pgfqpoint{4.620000in}{4.620000in}}%
\pgfusepath{clip}%
\pgfsetbuttcap%
\pgfsetroundjoin%
\definecolor{currentfill}{rgb}{1.000000,0.894118,0.788235}%
\pgfsetfillcolor{currentfill}%
\pgfsetlinewidth{0.000000pt}%
\definecolor{currentstroke}{rgb}{1.000000,0.894118,0.788235}%
\pgfsetstrokecolor{currentstroke}%
\pgfsetdash{}{0pt}%
\pgfpathmoveto{\pgfqpoint{2.961039in}{2.556963in}}%
\pgfpathlineto{\pgfqpoint{3.071936in}{2.492937in}}%
\pgfpathlineto{\pgfqpoint{3.120119in}{2.489790in}}%
\pgfpathlineto{\pgfqpoint{3.009222in}{2.553816in}}%
\pgfpathlineto{\pgfqpoint{2.961039in}{2.556963in}}%
\pgfpathclose%
\pgfusepath{fill}%
\end{pgfscope}%
\begin{pgfscope}%
\pgfpathrectangle{\pgfqpoint{0.765000in}{0.660000in}}{\pgfqpoint{4.620000in}{4.620000in}}%
\pgfusepath{clip}%
\pgfsetbuttcap%
\pgfsetroundjoin%
\definecolor{currentfill}{rgb}{1.000000,0.894118,0.788235}%
\pgfsetfillcolor{currentfill}%
\pgfsetlinewidth{0.000000pt}%
\definecolor{currentstroke}{rgb}{1.000000,0.894118,0.788235}%
\pgfsetstrokecolor{currentstroke}%
\pgfsetdash{}{0pt}%
\pgfpathmoveto{\pgfqpoint{3.120119in}{2.489790in}}%
\pgfpathlineto{\pgfqpoint{3.009222in}{2.425764in}}%
\pgfpathlineto{\pgfqpoint{3.120119in}{2.361737in}}%
\pgfpathlineto{\pgfqpoint{3.231016in}{2.425764in}}%
\pgfpathlineto{\pgfqpoint{3.120119in}{2.489790in}}%
\pgfpathclose%
\pgfusepath{fill}%
\end{pgfscope}%
\begin{pgfscope}%
\pgfpathrectangle{\pgfqpoint{0.765000in}{0.660000in}}{\pgfqpoint{4.620000in}{4.620000in}}%
\pgfusepath{clip}%
\pgfsetbuttcap%
\pgfsetroundjoin%
\definecolor{currentfill}{rgb}{1.000000,0.894118,0.788235}%
\pgfsetfillcolor{currentfill}%
\pgfsetlinewidth{0.000000pt}%
\definecolor{currentstroke}{rgb}{1.000000,0.894118,0.788235}%
\pgfsetstrokecolor{currentstroke}%
\pgfsetdash{}{0pt}%
\pgfpathmoveto{\pgfqpoint{3.120119in}{2.489790in}}%
\pgfpathlineto{\pgfqpoint{3.009222in}{2.425764in}}%
\pgfpathlineto{\pgfqpoint{3.009222in}{2.553816in}}%
\pgfpathlineto{\pgfqpoint{3.120119in}{2.617842in}}%
\pgfpathlineto{\pgfqpoint{3.120119in}{2.489790in}}%
\pgfpathclose%
\pgfusepath{fill}%
\end{pgfscope}%
\begin{pgfscope}%
\pgfpathrectangle{\pgfqpoint{0.765000in}{0.660000in}}{\pgfqpoint{4.620000in}{4.620000in}}%
\pgfusepath{clip}%
\pgfsetbuttcap%
\pgfsetroundjoin%
\definecolor{currentfill}{rgb}{1.000000,0.894118,0.788235}%
\pgfsetfillcolor{currentfill}%
\pgfsetlinewidth{0.000000pt}%
\definecolor{currentstroke}{rgb}{1.000000,0.894118,0.788235}%
\pgfsetstrokecolor{currentstroke}%
\pgfsetdash{}{0pt}%
\pgfpathmoveto{\pgfqpoint{3.120119in}{2.489790in}}%
\pgfpathlineto{\pgfqpoint{3.231016in}{2.425764in}}%
\pgfpathlineto{\pgfqpoint{3.231016in}{2.553816in}}%
\pgfpathlineto{\pgfqpoint{3.120119in}{2.617842in}}%
\pgfpathlineto{\pgfqpoint{3.120119in}{2.489790in}}%
\pgfpathclose%
\pgfusepath{fill}%
\end{pgfscope}%
\begin{pgfscope}%
\pgfpathrectangle{\pgfqpoint{0.765000in}{0.660000in}}{\pgfqpoint{4.620000in}{4.620000in}}%
\pgfusepath{clip}%
\pgfsetbuttcap%
\pgfsetroundjoin%
\definecolor{currentfill}{rgb}{1.000000,0.894118,0.788235}%
\pgfsetfillcolor{currentfill}%
\pgfsetlinewidth{0.000000pt}%
\definecolor{currentstroke}{rgb}{1.000000,0.894118,0.788235}%
\pgfsetstrokecolor{currentstroke}%
\pgfsetdash{}{0pt}%
\pgfpathmoveto{\pgfqpoint{3.186392in}{2.493495in}}%
\pgfpathlineto{\pgfqpoint{3.075496in}{2.429469in}}%
\pgfpathlineto{\pgfqpoint{3.186392in}{2.365442in}}%
\pgfpathlineto{\pgfqpoint{3.297289in}{2.429469in}}%
\pgfpathlineto{\pgfqpoint{3.186392in}{2.493495in}}%
\pgfpathclose%
\pgfusepath{fill}%
\end{pgfscope}%
\begin{pgfscope}%
\pgfpathrectangle{\pgfqpoint{0.765000in}{0.660000in}}{\pgfqpoint{4.620000in}{4.620000in}}%
\pgfusepath{clip}%
\pgfsetbuttcap%
\pgfsetroundjoin%
\definecolor{currentfill}{rgb}{1.000000,0.894118,0.788235}%
\pgfsetfillcolor{currentfill}%
\pgfsetlinewidth{0.000000pt}%
\definecolor{currentstroke}{rgb}{1.000000,0.894118,0.788235}%
\pgfsetstrokecolor{currentstroke}%
\pgfsetdash{}{0pt}%
\pgfpathmoveto{\pgfqpoint{3.186392in}{2.493495in}}%
\pgfpathlineto{\pgfqpoint{3.075496in}{2.429469in}}%
\pgfpathlineto{\pgfqpoint{3.075496in}{2.557521in}}%
\pgfpathlineto{\pgfqpoint{3.186392in}{2.621547in}}%
\pgfpathlineto{\pgfqpoint{3.186392in}{2.493495in}}%
\pgfpathclose%
\pgfusepath{fill}%
\end{pgfscope}%
\begin{pgfscope}%
\pgfpathrectangle{\pgfqpoint{0.765000in}{0.660000in}}{\pgfqpoint{4.620000in}{4.620000in}}%
\pgfusepath{clip}%
\pgfsetbuttcap%
\pgfsetroundjoin%
\definecolor{currentfill}{rgb}{1.000000,0.894118,0.788235}%
\pgfsetfillcolor{currentfill}%
\pgfsetlinewidth{0.000000pt}%
\definecolor{currentstroke}{rgb}{1.000000,0.894118,0.788235}%
\pgfsetstrokecolor{currentstroke}%
\pgfsetdash{}{0pt}%
\pgfpathmoveto{\pgfqpoint{3.186392in}{2.493495in}}%
\pgfpathlineto{\pgfqpoint{3.297289in}{2.429469in}}%
\pgfpathlineto{\pgfqpoint{3.297289in}{2.557521in}}%
\pgfpathlineto{\pgfqpoint{3.186392in}{2.621547in}}%
\pgfpathlineto{\pgfqpoint{3.186392in}{2.493495in}}%
\pgfpathclose%
\pgfusepath{fill}%
\end{pgfscope}%
\begin{pgfscope}%
\pgfpathrectangle{\pgfqpoint{0.765000in}{0.660000in}}{\pgfqpoint{4.620000in}{4.620000in}}%
\pgfusepath{clip}%
\pgfsetbuttcap%
\pgfsetroundjoin%
\definecolor{currentfill}{rgb}{1.000000,0.894118,0.788235}%
\pgfsetfillcolor{currentfill}%
\pgfsetlinewidth{0.000000pt}%
\definecolor{currentstroke}{rgb}{1.000000,0.894118,0.788235}%
\pgfsetstrokecolor{currentstroke}%
\pgfsetdash{}{0pt}%
\pgfpathmoveto{\pgfqpoint{3.120119in}{2.617842in}}%
\pgfpathlineto{\pgfqpoint{3.009222in}{2.553816in}}%
\pgfpathlineto{\pgfqpoint{3.120119in}{2.489790in}}%
\pgfpathlineto{\pgfqpoint{3.231016in}{2.553816in}}%
\pgfpathlineto{\pgfqpoint{3.120119in}{2.617842in}}%
\pgfpathclose%
\pgfusepath{fill}%
\end{pgfscope}%
\begin{pgfscope}%
\pgfpathrectangle{\pgfqpoint{0.765000in}{0.660000in}}{\pgfqpoint{4.620000in}{4.620000in}}%
\pgfusepath{clip}%
\pgfsetbuttcap%
\pgfsetroundjoin%
\definecolor{currentfill}{rgb}{1.000000,0.894118,0.788235}%
\pgfsetfillcolor{currentfill}%
\pgfsetlinewidth{0.000000pt}%
\definecolor{currentstroke}{rgb}{1.000000,0.894118,0.788235}%
\pgfsetstrokecolor{currentstroke}%
\pgfsetdash{}{0pt}%
\pgfpathmoveto{\pgfqpoint{3.120119in}{2.361737in}}%
\pgfpathlineto{\pgfqpoint{3.231016in}{2.425764in}}%
\pgfpathlineto{\pgfqpoint{3.231016in}{2.553816in}}%
\pgfpathlineto{\pgfqpoint{3.120119in}{2.489790in}}%
\pgfpathlineto{\pgfqpoint{3.120119in}{2.361737in}}%
\pgfpathclose%
\pgfusepath{fill}%
\end{pgfscope}%
\begin{pgfscope}%
\pgfpathrectangle{\pgfqpoint{0.765000in}{0.660000in}}{\pgfqpoint{4.620000in}{4.620000in}}%
\pgfusepath{clip}%
\pgfsetbuttcap%
\pgfsetroundjoin%
\definecolor{currentfill}{rgb}{1.000000,0.894118,0.788235}%
\pgfsetfillcolor{currentfill}%
\pgfsetlinewidth{0.000000pt}%
\definecolor{currentstroke}{rgb}{1.000000,0.894118,0.788235}%
\pgfsetstrokecolor{currentstroke}%
\pgfsetdash{}{0pt}%
\pgfpathmoveto{\pgfqpoint{3.009222in}{2.425764in}}%
\pgfpathlineto{\pgfqpoint{3.120119in}{2.361737in}}%
\pgfpathlineto{\pgfqpoint{3.120119in}{2.489790in}}%
\pgfpathlineto{\pgfqpoint{3.009222in}{2.553816in}}%
\pgfpathlineto{\pgfqpoint{3.009222in}{2.425764in}}%
\pgfpathclose%
\pgfusepath{fill}%
\end{pgfscope}%
\begin{pgfscope}%
\pgfpathrectangle{\pgfqpoint{0.765000in}{0.660000in}}{\pgfqpoint{4.620000in}{4.620000in}}%
\pgfusepath{clip}%
\pgfsetbuttcap%
\pgfsetroundjoin%
\definecolor{currentfill}{rgb}{1.000000,0.894118,0.788235}%
\pgfsetfillcolor{currentfill}%
\pgfsetlinewidth{0.000000pt}%
\definecolor{currentstroke}{rgb}{1.000000,0.894118,0.788235}%
\pgfsetstrokecolor{currentstroke}%
\pgfsetdash{}{0pt}%
\pgfpathmoveto{\pgfqpoint{3.186392in}{2.621547in}}%
\pgfpathlineto{\pgfqpoint{3.075496in}{2.557521in}}%
\pgfpathlineto{\pgfqpoint{3.186392in}{2.493495in}}%
\pgfpathlineto{\pgfqpoint{3.297289in}{2.557521in}}%
\pgfpathlineto{\pgfqpoint{3.186392in}{2.621547in}}%
\pgfpathclose%
\pgfusepath{fill}%
\end{pgfscope}%
\begin{pgfscope}%
\pgfpathrectangle{\pgfqpoint{0.765000in}{0.660000in}}{\pgfqpoint{4.620000in}{4.620000in}}%
\pgfusepath{clip}%
\pgfsetbuttcap%
\pgfsetroundjoin%
\definecolor{currentfill}{rgb}{1.000000,0.894118,0.788235}%
\pgfsetfillcolor{currentfill}%
\pgfsetlinewidth{0.000000pt}%
\definecolor{currentstroke}{rgb}{1.000000,0.894118,0.788235}%
\pgfsetstrokecolor{currentstroke}%
\pgfsetdash{}{0pt}%
\pgfpathmoveto{\pgfqpoint{3.186392in}{2.365442in}}%
\pgfpathlineto{\pgfqpoint{3.297289in}{2.429469in}}%
\pgfpathlineto{\pgfqpoint{3.297289in}{2.557521in}}%
\pgfpathlineto{\pgfqpoint{3.186392in}{2.493495in}}%
\pgfpathlineto{\pgfqpoint{3.186392in}{2.365442in}}%
\pgfpathclose%
\pgfusepath{fill}%
\end{pgfscope}%
\begin{pgfscope}%
\pgfpathrectangle{\pgfqpoint{0.765000in}{0.660000in}}{\pgfqpoint{4.620000in}{4.620000in}}%
\pgfusepath{clip}%
\pgfsetbuttcap%
\pgfsetroundjoin%
\definecolor{currentfill}{rgb}{1.000000,0.894118,0.788235}%
\pgfsetfillcolor{currentfill}%
\pgfsetlinewidth{0.000000pt}%
\definecolor{currentstroke}{rgb}{1.000000,0.894118,0.788235}%
\pgfsetstrokecolor{currentstroke}%
\pgfsetdash{}{0pt}%
\pgfpathmoveto{\pgfqpoint{3.075496in}{2.429469in}}%
\pgfpathlineto{\pgfqpoint{3.186392in}{2.365442in}}%
\pgfpathlineto{\pgfqpoint{3.186392in}{2.493495in}}%
\pgfpathlineto{\pgfqpoint{3.075496in}{2.557521in}}%
\pgfpathlineto{\pgfqpoint{3.075496in}{2.429469in}}%
\pgfpathclose%
\pgfusepath{fill}%
\end{pgfscope}%
\begin{pgfscope}%
\pgfpathrectangle{\pgfqpoint{0.765000in}{0.660000in}}{\pgfqpoint{4.620000in}{4.620000in}}%
\pgfusepath{clip}%
\pgfsetbuttcap%
\pgfsetroundjoin%
\definecolor{currentfill}{rgb}{1.000000,0.894118,0.788235}%
\pgfsetfillcolor{currentfill}%
\pgfsetlinewidth{0.000000pt}%
\definecolor{currentstroke}{rgb}{1.000000,0.894118,0.788235}%
\pgfsetstrokecolor{currentstroke}%
\pgfsetdash{}{0pt}%
\pgfpathmoveto{\pgfqpoint{3.120119in}{2.489790in}}%
\pgfpathlineto{\pgfqpoint{3.009222in}{2.425764in}}%
\pgfpathlineto{\pgfqpoint{3.075496in}{2.429469in}}%
\pgfpathlineto{\pgfqpoint{3.186392in}{2.493495in}}%
\pgfpathlineto{\pgfqpoint{3.120119in}{2.489790in}}%
\pgfpathclose%
\pgfusepath{fill}%
\end{pgfscope}%
\begin{pgfscope}%
\pgfpathrectangle{\pgfqpoint{0.765000in}{0.660000in}}{\pgfqpoint{4.620000in}{4.620000in}}%
\pgfusepath{clip}%
\pgfsetbuttcap%
\pgfsetroundjoin%
\definecolor{currentfill}{rgb}{1.000000,0.894118,0.788235}%
\pgfsetfillcolor{currentfill}%
\pgfsetlinewidth{0.000000pt}%
\definecolor{currentstroke}{rgb}{1.000000,0.894118,0.788235}%
\pgfsetstrokecolor{currentstroke}%
\pgfsetdash{}{0pt}%
\pgfpathmoveto{\pgfqpoint{3.231016in}{2.425764in}}%
\pgfpathlineto{\pgfqpoint{3.120119in}{2.489790in}}%
\pgfpathlineto{\pgfqpoint{3.186392in}{2.493495in}}%
\pgfpathlineto{\pgfqpoint{3.297289in}{2.429469in}}%
\pgfpathlineto{\pgfqpoint{3.231016in}{2.425764in}}%
\pgfpathclose%
\pgfusepath{fill}%
\end{pgfscope}%
\begin{pgfscope}%
\pgfpathrectangle{\pgfqpoint{0.765000in}{0.660000in}}{\pgfqpoint{4.620000in}{4.620000in}}%
\pgfusepath{clip}%
\pgfsetbuttcap%
\pgfsetroundjoin%
\definecolor{currentfill}{rgb}{1.000000,0.894118,0.788235}%
\pgfsetfillcolor{currentfill}%
\pgfsetlinewidth{0.000000pt}%
\definecolor{currentstroke}{rgb}{1.000000,0.894118,0.788235}%
\pgfsetstrokecolor{currentstroke}%
\pgfsetdash{}{0pt}%
\pgfpathmoveto{\pgfqpoint{3.120119in}{2.489790in}}%
\pgfpathlineto{\pgfqpoint{3.120119in}{2.617842in}}%
\pgfpathlineto{\pgfqpoint{3.186392in}{2.621547in}}%
\pgfpathlineto{\pgfqpoint{3.297289in}{2.429469in}}%
\pgfpathlineto{\pgfqpoint{3.120119in}{2.489790in}}%
\pgfpathclose%
\pgfusepath{fill}%
\end{pgfscope}%
\begin{pgfscope}%
\pgfpathrectangle{\pgfqpoint{0.765000in}{0.660000in}}{\pgfqpoint{4.620000in}{4.620000in}}%
\pgfusepath{clip}%
\pgfsetbuttcap%
\pgfsetroundjoin%
\definecolor{currentfill}{rgb}{1.000000,0.894118,0.788235}%
\pgfsetfillcolor{currentfill}%
\pgfsetlinewidth{0.000000pt}%
\definecolor{currentstroke}{rgb}{1.000000,0.894118,0.788235}%
\pgfsetstrokecolor{currentstroke}%
\pgfsetdash{}{0pt}%
\pgfpathmoveto{\pgfqpoint{3.231016in}{2.425764in}}%
\pgfpathlineto{\pgfqpoint{3.231016in}{2.553816in}}%
\pgfpathlineto{\pgfqpoint{3.297289in}{2.557521in}}%
\pgfpathlineto{\pgfqpoint{3.186392in}{2.493495in}}%
\pgfpathlineto{\pgfqpoint{3.231016in}{2.425764in}}%
\pgfpathclose%
\pgfusepath{fill}%
\end{pgfscope}%
\begin{pgfscope}%
\pgfpathrectangle{\pgfqpoint{0.765000in}{0.660000in}}{\pgfqpoint{4.620000in}{4.620000in}}%
\pgfusepath{clip}%
\pgfsetbuttcap%
\pgfsetroundjoin%
\definecolor{currentfill}{rgb}{1.000000,0.894118,0.788235}%
\pgfsetfillcolor{currentfill}%
\pgfsetlinewidth{0.000000pt}%
\definecolor{currentstroke}{rgb}{1.000000,0.894118,0.788235}%
\pgfsetstrokecolor{currentstroke}%
\pgfsetdash{}{0pt}%
\pgfpathmoveto{\pgfqpoint{3.009222in}{2.425764in}}%
\pgfpathlineto{\pgfqpoint{3.120119in}{2.361737in}}%
\pgfpathlineto{\pgfqpoint{3.186392in}{2.365442in}}%
\pgfpathlineto{\pgfqpoint{3.075496in}{2.429469in}}%
\pgfpathlineto{\pgfqpoint{3.009222in}{2.425764in}}%
\pgfpathclose%
\pgfusepath{fill}%
\end{pgfscope}%
\begin{pgfscope}%
\pgfpathrectangle{\pgfqpoint{0.765000in}{0.660000in}}{\pgfqpoint{4.620000in}{4.620000in}}%
\pgfusepath{clip}%
\pgfsetbuttcap%
\pgfsetroundjoin%
\definecolor{currentfill}{rgb}{1.000000,0.894118,0.788235}%
\pgfsetfillcolor{currentfill}%
\pgfsetlinewidth{0.000000pt}%
\definecolor{currentstroke}{rgb}{1.000000,0.894118,0.788235}%
\pgfsetstrokecolor{currentstroke}%
\pgfsetdash{}{0pt}%
\pgfpathmoveto{\pgfqpoint{3.120119in}{2.361737in}}%
\pgfpathlineto{\pgfqpoint{3.231016in}{2.425764in}}%
\pgfpathlineto{\pgfqpoint{3.297289in}{2.429469in}}%
\pgfpathlineto{\pgfqpoint{3.186392in}{2.365442in}}%
\pgfpathlineto{\pgfqpoint{3.120119in}{2.361737in}}%
\pgfpathclose%
\pgfusepath{fill}%
\end{pgfscope}%
\begin{pgfscope}%
\pgfpathrectangle{\pgfqpoint{0.765000in}{0.660000in}}{\pgfqpoint{4.620000in}{4.620000in}}%
\pgfusepath{clip}%
\pgfsetbuttcap%
\pgfsetroundjoin%
\definecolor{currentfill}{rgb}{1.000000,0.894118,0.788235}%
\pgfsetfillcolor{currentfill}%
\pgfsetlinewidth{0.000000pt}%
\definecolor{currentstroke}{rgb}{1.000000,0.894118,0.788235}%
\pgfsetstrokecolor{currentstroke}%
\pgfsetdash{}{0pt}%
\pgfpathmoveto{\pgfqpoint{3.120119in}{2.617842in}}%
\pgfpathlineto{\pgfqpoint{3.009222in}{2.553816in}}%
\pgfpathlineto{\pgfqpoint{3.075496in}{2.557521in}}%
\pgfpathlineto{\pgfqpoint{3.186392in}{2.621547in}}%
\pgfpathlineto{\pgfqpoint{3.120119in}{2.617842in}}%
\pgfpathclose%
\pgfusepath{fill}%
\end{pgfscope}%
\begin{pgfscope}%
\pgfpathrectangle{\pgfqpoint{0.765000in}{0.660000in}}{\pgfqpoint{4.620000in}{4.620000in}}%
\pgfusepath{clip}%
\pgfsetbuttcap%
\pgfsetroundjoin%
\definecolor{currentfill}{rgb}{1.000000,0.894118,0.788235}%
\pgfsetfillcolor{currentfill}%
\pgfsetlinewidth{0.000000pt}%
\definecolor{currentstroke}{rgb}{1.000000,0.894118,0.788235}%
\pgfsetstrokecolor{currentstroke}%
\pgfsetdash{}{0pt}%
\pgfpathmoveto{\pgfqpoint{3.231016in}{2.553816in}}%
\pgfpathlineto{\pgfqpoint{3.120119in}{2.617842in}}%
\pgfpathlineto{\pgfqpoint{3.186392in}{2.621547in}}%
\pgfpathlineto{\pgfqpoint{3.297289in}{2.557521in}}%
\pgfpathlineto{\pgfqpoint{3.231016in}{2.553816in}}%
\pgfpathclose%
\pgfusepath{fill}%
\end{pgfscope}%
\begin{pgfscope}%
\pgfpathrectangle{\pgfqpoint{0.765000in}{0.660000in}}{\pgfqpoint{4.620000in}{4.620000in}}%
\pgfusepath{clip}%
\pgfsetbuttcap%
\pgfsetroundjoin%
\definecolor{currentfill}{rgb}{1.000000,0.894118,0.788235}%
\pgfsetfillcolor{currentfill}%
\pgfsetlinewidth{0.000000pt}%
\definecolor{currentstroke}{rgb}{1.000000,0.894118,0.788235}%
\pgfsetstrokecolor{currentstroke}%
\pgfsetdash{}{0pt}%
\pgfpathmoveto{\pgfqpoint{3.009222in}{2.425764in}}%
\pgfpathlineto{\pgfqpoint{3.009222in}{2.553816in}}%
\pgfpathlineto{\pgfqpoint{3.075496in}{2.557521in}}%
\pgfpathlineto{\pgfqpoint{3.186392in}{2.365442in}}%
\pgfpathlineto{\pgfqpoint{3.009222in}{2.425764in}}%
\pgfpathclose%
\pgfusepath{fill}%
\end{pgfscope}%
\begin{pgfscope}%
\pgfpathrectangle{\pgfqpoint{0.765000in}{0.660000in}}{\pgfqpoint{4.620000in}{4.620000in}}%
\pgfusepath{clip}%
\pgfsetbuttcap%
\pgfsetroundjoin%
\definecolor{currentfill}{rgb}{1.000000,0.894118,0.788235}%
\pgfsetfillcolor{currentfill}%
\pgfsetlinewidth{0.000000pt}%
\definecolor{currentstroke}{rgb}{1.000000,0.894118,0.788235}%
\pgfsetstrokecolor{currentstroke}%
\pgfsetdash{}{0pt}%
\pgfpathmoveto{\pgfqpoint{3.120119in}{2.361737in}}%
\pgfpathlineto{\pgfqpoint{3.120119in}{2.489790in}}%
\pgfpathlineto{\pgfqpoint{3.186392in}{2.493495in}}%
\pgfpathlineto{\pgfqpoint{3.075496in}{2.429469in}}%
\pgfpathlineto{\pgfqpoint{3.120119in}{2.361737in}}%
\pgfpathclose%
\pgfusepath{fill}%
\end{pgfscope}%
\begin{pgfscope}%
\pgfpathrectangle{\pgfqpoint{0.765000in}{0.660000in}}{\pgfqpoint{4.620000in}{4.620000in}}%
\pgfusepath{clip}%
\pgfsetbuttcap%
\pgfsetroundjoin%
\definecolor{currentfill}{rgb}{1.000000,0.894118,0.788235}%
\pgfsetfillcolor{currentfill}%
\pgfsetlinewidth{0.000000pt}%
\definecolor{currentstroke}{rgb}{1.000000,0.894118,0.788235}%
\pgfsetstrokecolor{currentstroke}%
\pgfsetdash{}{0pt}%
\pgfpathmoveto{\pgfqpoint{3.120119in}{2.489790in}}%
\pgfpathlineto{\pgfqpoint{3.231016in}{2.553816in}}%
\pgfpathlineto{\pgfqpoint{3.297289in}{2.557521in}}%
\pgfpathlineto{\pgfqpoint{3.186392in}{2.493495in}}%
\pgfpathlineto{\pgfqpoint{3.120119in}{2.489790in}}%
\pgfpathclose%
\pgfusepath{fill}%
\end{pgfscope}%
\begin{pgfscope}%
\pgfpathrectangle{\pgfqpoint{0.765000in}{0.660000in}}{\pgfqpoint{4.620000in}{4.620000in}}%
\pgfusepath{clip}%
\pgfsetbuttcap%
\pgfsetroundjoin%
\definecolor{currentfill}{rgb}{1.000000,0.894118,0.788235}%
\pgfsetfillcolor{currentfill}%
\pgfsetlinewidth{0.000000pt}%
\definecolor{currentstroke}{rgb}{1.000000,0.894118,0.788235}%
\pgfsetstrokecolor{currentstroke}%
\pgfsetdash{}{0pt}%
\pgfpathmoveto{\pgfqpoint{3.009222in}{2.553816in}}%
\pgfpathlineto{\pgfqpoint{3.120119in}{2.489790in}}%
\pgfpathlineto{\pgfqpoint{3.186392in}{2.493495in}}%
\pgfpathlineto{\pgfqpoint{3.075496in}{2.557521in}}%
\pgfpathlineto{\pgfqpoint{3.009222in}{2.553816in}}%
\pgfpathclose%
\pgfusepath{fill}%
\end{pgfscope}%
\begin{pgfscope}%
\pgfpathrectangle{\pgfqpoint{0.765000in}{0.660000in}}{\pgfqpoint{4.620000in}{4.620000in}}%
\pgfusepath{clip}%
\pgfsetbuttcap%
\pgfsetroundjoin%
\definecolor{currentfill}{rgb}{1.000000,0.894118,0.788235}%
\pgfsetfillcolor{currentfill}%
\pgfsetlinewidth{0.000000pt}%
\definecolor{currentstroke}{rgb}{1.000000,0.894118,0.788235}%
\pgfsetstrokecolor{currentstroke}%
\pgfsetdash{}{0pt}%
\pgfpathmoveto{\pgfqpoint{3.120119in}{2.489790in}}%
\pgfpathlineto{\pgfqpoint{3.009222in}{2.425764in}}%
\pgfpathlineto{\pgfqpoint{3.120119in}{2.361737in}}%
\pgfpathlineto{\pgfqpoint{3.231016in}{2.425764in}}%
\pgfpathlineto{\pgfqpoint{3.120119in}{2.489790in}}%
\pgfpathclose%
\pgfusepath{fill}%
\end{pgfscope}%
\begin{pgfscope}%
\pgfpathrectangle{\pgfqpoint{0.765000in}{0.660000in}}{\pgfqpoint{4.620000in}{4.620000in}}%
\pgfusepath{clip}%
\pgfsetbuttcap%
\pgfsetroundjoin%
\definecolor{currentfill}{rgb}{1.000000,0.894118,0.788235}%
\pgfsetfillcolor{currentfill}%
\pgfsetlinewidth{0.000000pt}%
\definecolor{currentstroke}{rgb}{1.000000,0.894118,0.788235}%
\pgfsetstrokecolor{currentstroke}%
\pgfsetdash{}{0pt}%
\pgfpathmoveto{\pgfqpoint{3.120119in}{2.489790in}}%
\pgfpathlineto{\pgfqpoint{3.009222in}{2.425764in}}%
\pgfpathlineto{\pgfqpoint{3.009222in}{2.553816in}}%
\pgfpathlineto{\pgfqpoint{3.120119in}{2.617842in}}%
\pgfpathlineto{\pgfqpoint{3.120119in}{2.489790in}}%
\pgfpathclose%
\pgfusepath{fill}%
\end{pgfscope}%
\begin{pgfscope}%
\pgfpathrectangle{\pgfqpoint{0.765000in}{0.660000in}}{\pgfqpoint{4.620000in}{4.620000in}}%
\pgfusepath{clip}%
\pgfsetbuttcap%
\pgfsetroundjoin%
\definecolor{currentfill}{rgb}{1.000000,0.894118,0.788235}%
\pgfsetfillcolor{currentfill}%
\pgfsetlinewidth{0.000000pt}%
\definecolor{currentstroke}{rgb}{1.000000,0.894118,0.788235}%
\pgfsetstrokecolor{currentstroke}%
\pgfsetdash{}{0pt}%
\pgfpathmoveto{\pgfqpoint{3.120119in}{2.489790in}}%
\pgfpathlineto{\pgfqpoint{3.231016in}{2.425764in}}%
\pgfpathlineto{\pgfqpoint{3.231016in}{2.553816in}}%
\pgfpathlineto{\pgfqpoint{3.120119in}{2.617842in}}%
\pgfpathlineto{\pgfqpoint{3.120119in}{2.489790in}}%
\pgfpathclose%
\pgfusepath{fill}%
\end{pgfscope}%
\begin{pgfscope}%
\pgfpathrectangle{\pgfqpoint{0.765000in}{0.660000in}}{\pgfqpoint{4.620000in}{4.620000in}}%
\pgfusepath{clip}%
\pgfsetbuttcap%
\pgfsetroundjoin%
\definecolor{currentfill}{rgb}{1.000000,0.894118,0.788235}%
\pgfsetfillcolor{currentfill}%
\pgfsetlinewidth{0.000000pt}%
\definecolor{currentstroke}{rgb}{1.000000,0.894118,0.788235}%
\pgfsetstrokecolor{currentstroke}%
\pgfsetdash{}{0pt}%
\pgfpathmoveto{\pgfqpoint{3.071936in}{2.492937in}}%
\pgfpathlineto{\pgfqpoint{2.961039in}{2.428911in}}%
\pgfpathlineto{\pgfqpoint{3.071936in}{2.364884in}}%
\pgfpathlineto{\pgfqpoint{3.182833in}{2.428911in}}%
\pgfpathlineto{\pgfqpoint{3.071936in}{2.492937in}}%
\pgfpathclose%
\pgfusepath{fill}%
\end{pgfscope}%
\begin{pgfscope}%
\pgfpathrectangle{\pgfqpoint{0.765000in}{0.660000in}}{\pgfqpoint{4.620000in}{4.620000in}}%
\pgfusepath{clip}%
\pgfsetbuttcap%
\pgfsetroundjoin%
\definecolor{currentfill}{rgb}{1.000000,0.894118,0.788235}%
\pgfsetfillcolor{currentfill}%
\pgfsetlinewidth{0.000000pt}%
\definecolor{currentstroke}{rgb}{1.000000,0.894118,0.788235}%
\pgfsetstrokecolor{currentstroke}%
\pgfsetdash{}{0pt}%
\pgfpathmoveto{\pgfqpoint{3.071936in}{2.492937in}}%
\pgfpathlineto{\pgfqpoint{2.961039in}{2.428911in}}%
\pgfpathlineto{\pgfqpoint{2.961039in}{2.556963in}}%
\pgfpathlineto{\pgfqpoint{3.071936in}{2.620989in}}%
\pgfpathlineto{\pgfqpoint{3.071936in}{2.492937in}}%
\pgfpathclose%
\pgfusepath{fill}%
\end{pgfscope}%
\begin{pgfscope}%
\pgfpathrectangle{\pgfqpoint{0.765000in}{0.660000in}}{\pgfqpoint{4.620000in}{4.620000in}}%
\pgfusepath{clip}%
\pgfsetbuttcap%
\pgfsetroundjoin%
\definecolor{currentfill}{rgb}{1.000000,0.894118,0.788235}%
\pgfsetfillcolor{currentfill}%
\pgfsetlinewidth{0.000000pt}%
\definecolor{currentstroke}{rgb}{1.000000,0.894118,0.788235}%
\pgfsetstrokecolor{currentstroke}%
\pgfsetdash{}{0pt}%
\pgfpathmoveto{\pgfqpoint{3.071936in}{2.492937in}}%
\pgfpathlineto{\pgfqpoint{3.182833in}{2.428911in}}%
\pgfpathlineto{\pgfqpoint{3.182833in}{2.556963in}}%
\pgfpathlineto{\pgfqpoint{3.071936in}{2.620989in}}%
\pgfpathlineto{\pgfqpoint{3.071936in}{2.492937in}}%
\pgfpathclose%
\pgfusepath{fill}%
\end{pgfscope}%
\begin{pgfscope}%
\pgfpathrectangle{\pgfqpoint{0.765000in}{0.660000in}}{\pgfqpoint{4.620000in}{4.620000in}}%
\pgfusepath{clip}%
\pgfsetbuttcap%
\pgfsetroundjoin%
\definecolor{currentfill}{rgb}{1.000000,0.894118,0.788235}%
\pgfsetfillcolor{currentfill}%
\pgfsetlinewidth{0.000000pt}%
\definecolor{currentstroke}{rgb}{1.000000,0.894118,0.788235}%
\pgfsetstrokecolor{currentstroke}%
\pgfsetdash{}{0pt}%
\pgfpathmoveto{\pgfqpoint{3.120119in}{2.617842in}}%
\pgfpathlineto{\pgfqpoint{3.009222in}{2.553816in}}%
\pgfpathlineto{\pgfqpoint{3.120119in}{2.489790in}}%
\pgfpathlineto{\pgfqpoint{3.231016in}{2.553816in}}%
\pgfpathlineto{\pgfqpoint{3.120119in}{2.617842in}}%
\pgfpathclose%
\pgfusepath{fill}%
\end{pgfscope}%
\begin{pgfscope}%
\pgfpathrectangle{\pgfqpoint{0.765000in}{0.660000in}}{\pgfqpoint{4.620000in}{4.620000in}}%
\pgfusepath{clip}%
\pgfsetbuttcap%
\pgfsetroundjoin%
\definecolor{currentfill}{rgb}{1.000000,0.894118,0.788235}%
\pgfsetfillcolor{currentfill}%
\pgfsetlinewidth{0.000000pt}%
\definecolor{currentstroke}{rgb}{1.000000,0.894118,0.788235}%
\pgfsetstrokecolor{currentstroke}%
\pgfsetdash{}{0pt}%
\pgfpathmoveto{\pgfqpoint{3.120119in}{2.361737in}}%
\pgfpathlineto{\pgfqpoint{3.231016in}{2.425764in}}%
\pgfpathlineto{\pgfqpoint{3.231016in}{2.553816in}}%
\pgfpathlineto{\pgfqpoint{3.120119in}{2.489790in}}%
\pgfpathlineto{\pgfqpoint{3.120119in}{2.361737in}}%
\pgfpathclose%
\pgfusepath{fill}%
\end{pgfscope}%
\begin{pgfscope}%
\pgfpathrectangle{\pgfqpoint{0.765000in}{0.660000in}}{\pgfqpoint{4.620000in}{4.620000in}}%
\pgfusepath{clip}%
\pgfsetbuttcap%
\pgfsetroundjoin%
\definecolor{currentfill}{rgb}{1.000000,0.894118,0.788235}%
\pgfsetfillcolor{currentfill}%
\pgfsetlinewidth{0.000000pt}%
\definecolor{currentstroke}{rgb}{1.000000,0.894118,0.788235}%
\pgfsetstrokecolor{currentstroke}%
\pgfsetdash{}{0pt}%
\pgfpathmoveto{\pgfqpoint{3.009222in}{2.425764in}}%
\pgfpathlineto{\pgfqpoint{3.120119in}{2.361737in}}%
\pgfpathlineto{\pgfqpoint{3.120119in}{2.489790in}}%
\pgfpathlineto{\pgfqpoint{3.009222in}{2.553816in}}%
\pgfpathlineto{\pgfqpoint{3.009222in}{2.425764in}}%
\pgfpathclose%
\pgfusepath{fill}%
\end{pgfscope}%
\begin{pgfscope}%
\pgfpathrectangle{\pgfqpoint{0.765000in}{0.660000in}}{\pgfqpoint{4.620000in}{4.620000in}}%
\pgfusepath{clip}%
\pgfsetbuttcap%
\pgfsetroundjoin%
\definecolor{currentfill}{rgb}{1.000000,0.894118,0.788235}%
\pgfsetfillcolor{currentfill}%
\pgfsetlinewidth{0.000000pt}%
\definecolor{currentstroke}{rgb}{1.000000,0.894118,0.788235}%
\pgfsetstrokecolor{currentstroke}%
\pgfsetdash{}{0pt}%
\pgfpathmoveto{\pgfqpoint{3.071936in}{2.620989in}}%
\pgfpathlineto{\pgfqpoint{2.961039in}{2.556963in}}%
\pgfpathlineto{\pgfqpoint{3.071936in}{2.492937in}}%
\pgfpathlineto{\pgfqpoint{3.182833in}{2.556963in}}%
\pgfpathlineto{\pgfqpoint{3.071936in}{2.620989in}}%
\pgfpathclose%
\pgfusepath{fill}%
\end{pgfscope}%
\begin{pgfscope}%
\pgfpathrectangle{\pgfqpoint{0.765000in}{0.660000in}}{\pgfqpoint{4.620000in}{4.620000in}}%
\pgfusepath{clip}%
\pgfsetbuttcap%
\pgfsetroundjoin%
\definecolor{currentfill}{rgb}{1.000000,0.894118,0.788235}%
\pgfsetfillcolor{currentfill}%
\pgfsetlinewidth{0.000000pt}%
\definecolor{currentstroke}{rgb}{1.000000,0.894118,0.788235}%
\pgfsetstrokecolor{currentstroke}%
\pgfsetdash{}{0pt}%
\pgfpathmoveto{\pgfqpoint{3.071936in}{2.364884in}}%
\pgfpathlineto{\pgfqpoint{3.182833in}{2.428911in}}%
\pgfpathlineto{\pgfqpoint{3.182833in}{2.556963in}}%
\pgfpathlineto{\pgfqpoint{3.071936in}{2.492937in}}%
\pgfpathlineto{\pgfqpoint{3.071936in}{2.364884in}}%
\pgfpathclose%
\pgfusepath{fill}%
\end{pgfscope}%
\begin{pgfscope}%
\pgfpathrectangle{\pgfqpoint{0.765000in}{0.660000in}}{\pgfqpoint{4.620000in}{4.620000in}}%
\pgfusepath{clip}%
\pgfsetbuttcap%
\pgfsetroundjoin%
\definecolor{currentfill}{rgb}{1.000000,0.894118,0.788235}%
\pgfsetfillcolor{currentfill}%
\pgfsetlinewidth{0.000000pt}%
\definecolor{currentstroke}{rgb}{1.000000,0.894118,0.788235}%
\pgfsetstrokecolor{currentstroke}%
\pgfsetdash{}{0pt}%
\pgfpathmoveto{\pgfqpoint{2.961039in}{2.428911in}}%
\pgfpathlineto{\pgfqpoint{3.071936in}{2.364884in}}%
\pgfpathlineto{\pgfqpoint{3.071936in}{2.492937in}}%
\pgfpathlineto{\pgfqpoint{2.961039in}{2.556963in}}%
\pgfpathlineto{\pgfqpoint{2.961039in}{2.428911in}}%
\pgfpathclose%
\pgfusepath{fill}%
\end{pgfscope}%
\begin{pgfscope}%
\pgfpathrectangle{\pgfqpoint{0.765000in}{0.660000in}}{\pgfqpoint{4.620000in}{4.620000in}}%
\pgfusepath{clip}%
\pgfsetbuttcap%
\pgfsetroundjoin%
\definecolor{currentfill}{rgb}{1.000000,0.894118,0.788235}%
\pgfsetfillcolor{currentfill}%
\pgfsetlinewidth{0.000000pt}%
\definecolor{currentstroke}{rgb}{1.000000,0.894118,0.788235}%
\pgfsetstrokecolor{currentstroke}%
\pgfsetdash{}{0pt}%
\pgfpathmoveto{\pgfqpoint{3.120119in}{2.489790in}}%
\pgfpathlineto{\pgfqpoint{3.009222in}{2.425764in}}%
\pgfpathlineto{\pgfqpoint{2.961039in}{2.428911in}}%
\pgfpathlineto{\pgfqpoint{3.071936in}{2.492937in}}%
\pgfpathlineto{\pgfqpoint{3.120119in}{2.489790in}}%
\pgfpathclose%
\pgfusepath{fill}%
\end{pgfscope}%
\begin{pgfscope}%
\pgfpathrectangle{\pgfqpoint{0.765000in}{0.660000in}}{\pgfqpoint{4.620000in}{4.620000in}}%
\pgfusepath{clip}%
\pgfsetbuttcap%
\pgfsetroundjoin%
\definecolor{currentfill}{rgb}{1.000000,0.894118,0.788235}%
\pgfsetfillcolor{currentfill}%
\pgfsetlinewidth{0.000000pt}%
\definecolor{currentstroke}{rgb}{1.000000,0.894118,0.788235}%
\pgfsetstrokecolor{currentstroke}%
\pgfsetdash{}{0pt}%
\pgfpathmoveto{\pgfqpoint{3.231016in}{2.425764in}}%
\pgfpathlineto{\pgfqpoint{3.120119in}{2.489790in}}%
\pgfpathlineto{\pgfqpoint{3.071936in}{2.492937in}}%
\pgfpathlineto{\pgfqpoint{3.182833in}{2.428911in}}%
\pgfpathlineto{\pgfqpoint{3.231016in}{2.425764in}}%
\pgfpathclose%
\pgfusepath{fill}%
\end{pgfscope}%
\begin{pgfscope}%
\pgfpathrectangle{\pgfqpoint{0.765000in}{0.660000in}}{\pgfqpoint{4.620000in}{4.620000in}}%
\pgfusepath{clip}%
\pgfsetbuttcap%
\pgfsetroundjoin%
\definecolor{currentfill}{rgb}{1.000000,0.894118,0.788235}%
\pgfsetfillcolor{currentfill}%
\pgfsetlinewidth{0.000000pt}%
\definecolor{currentstroke}{rgb}{1.000000,0.894118,0.788235}%
\pgfsetstrokecolor{currentstroke}%
\pgfsetdash{}{0pt}%
\pgfpathmoveto{\pgfqpoint{3.120119in}{2.489790in}}%
\pgfpathlineto{\pgfqpoint{3.120119in}{2.617842in}}%
\pgfpathlineto{\pgfqpoint{3.071936in}{2.620989in}}%
\pgfpathlineto{\pgfqpoint{3.182833in}{2.428911in}}%
\pgfpathlineto{\pgfqpoint{3.120119in}{2.489790in}}%
\pgfpathclose%
\pgfusepath{fill}%
\end{pgfscope}%
\begin{pgfscope}%
\pgfpathrectangle{\pgfqpoint{0.765000in}{0.660000in}}{\pgfqpoint{4.620000in}{4.620000in}}%
\pgfusepath{clip}%
\pgfsetbuttcap%
\pgfsetroundjoin%
\definecolor{currentfill}{rgb}{1.000000,0.894118,0.788235}%
\pgfsetfillcolor{currentfill}%
\pgfsetlinewidth{0.000000pt}%
\definecolor{currentstroke}{rgb}{1.000000,0.894118,0.788235}%
\pgfsetstrokecolor{currentstroke}%
\pgfsetdash{}{0pt}%
\pgfpathmoveto{\pgfqpoint{3.231016in}{2.425764in}}%
\pgfpathlineto{\pgfqpoint{3.231016in}{2.553816in}}%
\pgfpathlineto{\pgfqpoint{3.182833in}{2.556963in}}%
\pgfpathlineto{\pgfqpoint{3.071936in}{2.492937in}}%
\pgfpathlineto{\pgfqpoint{3.231016in}{2.425764in}}%
\pgfpathclose%
\pgfusepath{fill}%
\end{pgfscope}%
\begin{pgfscope}%
\pgfpathrectangle{\pgfqpoint{0.765000in}{0.660000in}}{\pgfqpoint{4.620000in}{4.620000in}}%
\pgfusepath{clip}%
\pgfsetbuttcap%
\pgfsetroundjoin%
\definecolor{currentfill}{rgb}{1.000000,0.894118,0.788235}%
\pgfsetfillcolor{currentfill}%
\pgfsetlinewidth{0.000000pt}%
\definecolor{currentstroke}{rgb}{1.000000,0.894118,0.788235}%
\pgfsetstrokecolor{currentstroke}%
\pgfsetdash{}{0pt}%
\pgfpathmoveto{\pgfqpoint{3.009222in}{2.425764in}}%
\pgfpathlineto{\pgfqpoint{3.120119in}{2.361737in}}%
\pgfpathlineto{\pgfqpoint{3.071936in}{2.364884in}}%
\pgfpathlineto{\pgfqpoint{2.961039in}{2.428911in}}%
\pgfpathlineto{\pgfqpoint{3.009222in}{2.425764in}}%
\pgfpathclose%
\pgfusepath{fill}%
\end{pgfscope}%
\begin{pgfscope}%
\pgfpathrectangle{\pgfqpoint{0.765000in}{0.660000in}}{\pgfqpoint{4.620000in}{4.620000in}}%
\pgfusepath{clip}%
\pgfsetbuttcap%
\pgfsetroundjoin%
\definecolor{currentfill}{rgb}{1.000000,0.894118,0.788235}%
\pgfsetfillcolor{currentfill}%
\pgfsetlinewidth{0.000000pt}%
\definecolor{currentstroke}{rgb}{1.000000,0.894118,0.788235}%
\pgfsetstrokecolor{currentstroke}%
\pgfsetdash{}{0pt}%
\pgfpathmoveto{\pgfqpoint{3.120119in}{2.361737in}}%
\pgfpathlineto{\pgfqpoint{3.231016in}{2.425764in}}%
\pgfpathlineto{\pgfqpoint{3.182833in}{2.428911in}}%
\pgfpathlineto{\pgfqpoint{3.071936in}{2.364884in}}%
\pgfpathlineto{\pgfqpoint{3.120119in}{2.361737in}}%
\pgfpathclose%
\pgfusepath{fill}%
\end{pgfscope}%
\begin{pgfscope}%
\pgfpathrectangle{\pgfqpoint{0.765000in}{0.660000in}}{\pgfqpoint{4.620000in}{4.620000in}}%
\pgfusepath{clip}%
\pgfsetbuttcap%
\pgfsetroundjoin%
\definecolor{currentfill}{rgb}{1.000000,0.894118,0.788235}%
\pgfsetfillcolor{currentfill}%
\pgfsetlinewidth{0.000000pt}%
\definecolor{currentstroke}{rgb}{1.000000,0.894118,0.788235}%
\pgfsetstrokecolor{currentstroke}%
\pgfsetdash{}{0pt}%
\pgfpathmoveto{\pgfqpoint{3.120119in}{2.617842in}}%
\pgfpathlineto{\pgfqpoint{3.009222in}{2.553816in}}%
\pgfpathlineto{\pgfqpoint{2.961039in}{2.556963in}}%
\pgfpathlineto{\pgfqpoint{3.071936in}{2.620989in}}%
\pgfpathlineto{\pgfqpoint{3.120119in}{2.617842in}}%
\pgfpathclose%
\pgfusepath{fill}%
\end{pgfscope}%
\begin{pgfscope}%
\pgfpathrectangle{\pgfqpoint{0.765000in}{0.660000in}}{\pgfqpoint{4.620000in}{4.620000in}}%
\pgfusepath{clip}%
\pgfsetbuttcap%
\pgfsetroundjoin%
\definecolor{currentfill}{rgb}{1.000000,0.894118,0.788235}%
\pgfsetfillcolor{currentfill}%
\pgfsetlinewidth{0.000000pt}%
\definecolor{currentstroke}{rgb}{1.000000,0.894118,0.788235}%
\pgfsetstrokecolor{currentstroke}%
\pgfsetdash{}{0pt}%
\pgfpathmoveto{\pgfqpoint{3.231016in}{2.553816in}}%
\pgfpathlineto{\pgfqpoint{3.120119in}{2.617842in}}%
\pgfpathlineto{\pgfqpoint{3.071936in}{2.620989in}}%
\pgfpathlineto{\pgfqpoint{3.182833in}{2.556963in}}%
\pgfpathlineto{\pgfqpoint{3.231016in}{2.553816in}}%
\pgfpathclose%
\pgfusepath{fill}%
\end{pgfscope}%
\begin{pgfscope}%
\pgfpathrectangle{\pgfqpoint{0.765000in}{0.660000in}}{\pgfqpoint{4.620000in}{4.620000in}}%
\pgfusepath{clip}%
\pgfsetbuttcap%
\pgfsetroundjoin%
\definecolor{currentfill}{rgb}{1.000000,0.894118,0.788235}%
\pgfsetfillcolor{currentfill}%
\pgfsetlinewidth{0.000000pt}%
\definecolor{currentstroke}{rgb}{1.000000,0.894118,0.788235}%
\pgfsetstrokecolor{currentstroke}%
\pgfsetdash{}{0pt}%
\pgfpathmoveto{\pgfqpoint{3.009222in}{2.425764in}}%
\pgfpathlineto{\pgfqpoint{3.009222in}{2.553816in}}%
\pgfpathlineto{\pgfqpoint{2.961039in}{2.556963in}}%
\pgfpathlineto{\pgfqpoint{3.071936in}{2.364884in}}%
\pgfpathlineto{\pgfqpoint{3.009222in}{2.425764in}}%
\pgfpathclose%
\pgfusepath{fill}%
\end{pgfscope}%
\begin{pgfscope}%
\pgfpathrectangle{\pgfqpoint{0.765000in}{0.660000in}}{\pgfqpoint{4.620000in}{4.620000in}}%
\pgfusepath{clip}%
\pgfsetbuttcap%
\pgfsetroundjoin%
\definecolor{currentfill}{rgb}{1.000000,0.894118,0.788235}%
\pgfsetfillcolor{currentfill}%
\pgfsetlinewidth{0.000000pt}%
\definecolor{currentstroke}{rgb}{1.000000,0.894118,0.788235}%
\pgfsetstrokecolor{currentstroke}%
\pgfsetdash{}{0pt}%
\pgfpathmoveto{\pgfqpoint{3.120119in}{2.361737in}}%
\pgfpathlineto{\pgfqpoint{3.120119in}{2.489790in}}%
\pgfpathlineto{\pgfqpoint{3.071936in}{2.492937in}}%
\pgfpathlineto{\pgfqpoint{2.961039in}{2.428911in}}%
\pgfpathlineto{\pgfqpoint{3.120119in}{2.361737in}}%
\pgfpathclose%
\pgfusepath{fill}%
\end{pgfscope}%
\begin{pgfscope}%
\pgfpathrectangle{\pgfqpoint{0.765000in}{0.660000in}}{\pgfqpoint{4.620000in}{4.620000in}}%
\pgfusepath{clip}%
\pgfsetbuttcap%
\pgfsetroundjoin%
\definecolor{currentfill}{rgb}{1.000000,0.894118,0.788235}%
\pgfsetfillcolor{currentfill}%
\pgfsetlinewidth{0.000000pt}%
\definecolor{currentstroke}{rgb}{1.000000,0.894118,0.788235}%
\pgfsetstrokecolor{currentstroke}%
\pgfsetdash{}{0pt}%
\pgfpathmoveto{\pgfqpoint{3.120119in}{2.489790in}}%
\pgfpathlineto{\pgfqpoint{3.231016in}{2.553816in}}%
\pgfpathlineto{\pgfqpoint{3.182833in}{2.556963in}}%
\pgfpathlineto{\pgfqpoint{3.071936in}{2.492937in}}%
\pgfpathlineto{\pgfqpoint{3.120119in}{2.489790in}}%
\pgfpathclose%
\pgfusepath{fill}%
\end{pgfscope}%
\begin{pgfscope}%
\pgfpathrectangle{\pgfqpoint{0.765000in}{0.660000in}}{\pgfqpoint{4.620000in}{4.620000in}}%
\pgfusepath{clip}%
\pgfsetbuttcap%
\pgfsetroundjoin%
\definecolor{currentfill}{rgb}{1.000000,0.894118,0.788235}%
\pgfsetfillcolor{currentfill}%
\pgfsetlinewidth{0.000000pt}%
\definecolor{currentstroke}{rgb}{1.000000,0.894118,0.788235}%
\pgfsetstrokecolor{currentstroke}%
\pgfsetdash{}{0pt}%
\pgfpathmoveto{\pgfqpoint{3.009222in}{2.553816in}}%
\pgfpathlineto{\pgfqpoint{3.120119in}{2.489790in}}%
\pgfpathlineto{\pgfqpoint{3.071936in}{2.492937in}}%
\pgfpathlineto{\pgfqpoint{2.961039in}{2.556963in}}%
\pgfpathlineto{\pgfqpoint{3.009222in}{2.553816in}}%
\pgfpathclose%
\pgfusepath{fill}%
\end{pgfscope}%
\begin{pgfscope}%
\pgfpathrectangle{\pgfqpoint{0.765000in}{0.660000in}}{\pgfqpoint{4.620000in}{4.620000in}}%
\pgfusepath{clip}%
\pgfsetbuttcap%
\pgfsetroundjoin%
\definecolor{currentfill}{rgb}{1.000000,0.894118,0.788235}%
\pgfsetfillcolor{currentfill}%
\pgfsetlinewidth{0.000000pt}%
\definecolor{currentstroke}{rgb}{1.000000,0.894118,0.788235}%
\pgfsetstrokecolor{currentstroke}%
\pgfsetdash{}{0pt}%
\pgfpathmoveto{\pgfqpoint{3.296975in}{3.470058in}}%
\pgfpathlineto{\pgfqpoint{3.186079in}{3.406032in}}%
\pgfpathlineto{\pgfqpoint{3.296975in}{3.342005in}}%
\pgfpathlineto{\pgfqpoint{3.407872in}{3.406032in}}%
\pgfpathlineto{\pgfqpoint{3.296975in}{3.470058in}}%
\pgfpathclose%
\pgfusepath{fill}%
\end{pgfscope}%
\begin{pgfscope}%
\pgfpathrectangle{\pgfqpoint{0.765000in}{0.660000in}}{\pgfqpoint{4.620000in}{4.620000in}}%
\pgfusepath{clip}%
\pgfsetbuttcap%
\pgfsetroundjoin%
\definecolor{currentfill}{rgb}{1.000000,0.894118,0.788235}%
\pgfsetfillcolor{currentfill}%
\pgfsetlinewidth{0.000000pt}%
\definecolor{currentstroke}{rgb}{1.000000,0.894118,0.788235}%
\pgfsetstrokecolor{currentstroke}%
\pgfsetdash{}{0pt}%
\pgfpathmoveto{\pgfqpoint{3.296975in}{3.470058in}}%
\pgfpathlineto{\pgfqpoint{3.186079in}{3.406032in}}%
\pgfpathlineto{\pgfqpoint{3.186079in}{3.534084in}}%
\pgfpathlineto{\pgfqpoint{3.296975in}{3.598110in}}%
\pgfpathlineto{\pgfqpoint{3.296975in}{3.470058in}}%
\pgfpathclose%
\pgfusepath{fill}%
\end{pgfscope}%
\begin{pgfscope}%
\pgfpathrectangle{\pgfqpoint{0.765000in}{0.660000in}}{\pgfqpoint{4.620000in}{4.620000in}}%
\pgfusepath{clip}%
\pgfsetbuttcap%
\pgfsetroundjoin%
\definecolor{currentfill}{rgb}{1.000000,0.894118,0.788235}%
\pgfsetfillcolor{currentfill}%
\pgfsetlinewidth{0.000000pt}%
\definecolor{currentstroke}{rgb}{1.000000,0.894118,0.788235}%
\pgfsetstrokecolor{currentstroke}%
\pgfsetdash{}{0pt}%
\pgfpathmoveto{\pgfqpoint{3.296975in}{3.470058in}}%
\pgfpathlineto{\pgfqpoint{3.407872in}{3.406032in}}%
\pgfpathlineto{\pgfqpoint{3.407872in}{3.534084in}}%
\pgfpathlineto{\pgfqpoint{3.296975in}{3.598110in}}%
\pgfpathlineto{\pgfqpoint{3.296975in}{3.470058in}}%
\pgfpathclose%
\pgfusepath{fill}%
\end{pgfscope}%
\begin{pgfscope}%
\pgfpathrectangle{\pgfqpoint{0.765000in}{0.660000in}}{\pgfqpoint{4.620000in}{4.620000in}}%
\pgfusepath{clip}%
\pgfsetbuttcap%
\pgfsetroundjoin%
\definecolor{currentfill}{rgb}{1.000000,0.894118,0.788235}%
\pgfsetfillcolor{currentfill}%
\pgfsetlinewidth{0.000000pt}%
\definecolor{currentstroke}{rgb}{1.000000,0.894118,0.788235}%
\pgfsetstrokecolor{currentstroke}%
\pgfsetdash{}{0pt}%
\pgfpathmoveto{\pgfqpoint{3.265487in}{3.489049in}}%
\pgfpathlineto{\pgfqpoint{3.154590in}{3.425023in}}%
\pgfpathlineto{\pgfqpoint{3.265487in}{3.360996in}}%
\pgfpathlineto{\pgfqpoint{3.376384in}{3.425023in}}%
\pgfpathlineto{\pgfqpoint{3.265487in}{3.489049in}}%
\pgfpathclose%
\pgfusepath{fill}%
\end{pgfscope}%
\begin{pgfscope}%
\pgfpathrectangle{\pgfqpoint{0.765000in}{0.660000in}}{\pgfqpoint{4.620000in}{4.620000in}}%
\pgfusepath{clip}%
\pgfsetbuttcap%
\pgfsetroundjoin%
\definecolor{currentfill}{rgb}{1.000000,0.894118,0.788235}%
\pgfsetfillcolor{currentfill}%
\pgfsetlinewidth{0.000000pt}%
\definecolor{currentstroke}{rgb}{1.000000,0.894118,0.788235}%
\pgfsetstrokecolor{currentstroke}%
\pgfsetdash{}{0pt}%
\pgfpathmoveto{\pgfqpoint{3.265487in}{3.489049in}}%
\pgfpathlineto{\pgfqpoint{3.154590in}{3.425023in}}%
\pgfpathlineto{\pgfqpoint{3.154590in}{3.553075in}}%
\pgfpathlineto{\pgfqpoint{3.265487in}{3.617101in}}%
\pgfpathlineto{\pgfqpoint{3.265487in}{3.489049in}}%
\pgfpathclose%
\pgfusepath{fill}%
\end{pgfscope}%
\begin{pgfscope}%
\pgfpathrectangle{\pgfqpoint{0.765000in}{0.660000in}}{\pgfqpoint{4.620000in}{4.620000in}}%
\pgfusepath{clip}%
\pgfsetbuttcap%
\pgfsetroundjoin%
\definecolor{currentfill}{rgb}{1.000000,0.894118,0.788235}%
\pgfsetfillcolor{currentfill}%
\pgfsetlinewidth{0.000000pt}%
\definecolor{currentstroke}{rgb}{1.000000,0.894118,0.788235}%
\pgfsetstrokecolor{currentstroke}%
\pgfsetdash{}{0pt}%
\pgfpathmoveto{\pgfqpoint{3.265487in}{3.489049in}}%
\pgfpathlineto{\pgfqpoint{3.376384in}{3.425023in}}%
\pgfpathlineto{\pgfqpoint{3.376384in}{3.553075in}}%
\pgfpathlineto{\pgfqpoint{3.265487in}{3.617101in}}%
\pgfpathlineto{\pgfqpoint{3.265487in}{3.489049in}}%
\pgfpathclose%
\pgfusepath{fill}%
\end{pgfscope}%
\begin{pgfscope}%
\pgfpathrectangle{\pgfqpoint{0.765000in}{0.660000in}}{\pgfqpoint{4.620000in}{4.620000in}}%
\pgfusepath{clip}%
\pgfsetbuttcap%
\pgfsetroundjoin%
\definecolor{currentfill}{rgb}{1.000000,0.894118,0.788235}%
\pgfsetfillcolor{currentfill}%
\pgfsetlinewidth{0.000000pt}%
\definecolor{currentstroke}{rgb}{1.000000,0.894118,0.788235}%
\pgfsetstrokecolor{currentstroke}%
\pgfsetdash{}{0pt}%
\pgfpathmoveto{\pgfqpoint{3.296975in}{3.598110in}}%
\pgfpathlineto{\pgfqpoint{3.186079in}{3.534084in}}%
\pgfpathlineto{\pgfqpoint{3.296975in}{3.470058in}}%
\pgfpathlineto{\pgfqpoint{3.407872in}{3.534084in}}%
\pgfpathlineto{\pgfqpoint{3.296975in}{3.598110in}}%
\pgfpathclose%
\pgfusepath{fill}%
\end{pgfscope}%
\begin{pgfscope}%
\pgfpathrectangle{\pgfqpoint{0.765000in}{0.660000in}}{\pgfqpoint{4.620000in}{4.620000in}}%
\pgfusepath{clip}%
\pgfsetbuttcap%
\pgfsetroundjoin%
\definecolor{currentfill}{rgb}{1.000000,0.894118,0.788235}%
\pgfsetfillcolor{currentfill}%
\pgfsetlinewidth{0.000000pt}%
\definecolor{currentstroke}{rgb}{1.000000,0.894118,0.788235}%
\pgfsetstrokecolor{currentstroke}%
\pgfsetdash{}{0pt}%
\pgfpathmoveto{\pgfqpoint{3.296975in}{3.342005in}}%
\pgfpathlineto{\pgfqpoint{3.407872in}{3.406032in}}%
\pgfpathlineto{\pgfqpoint{3.407872in}{3.534084in}}%
\pgfpathlineto{\pgfqpoint{3.296975in}{3.470058in}}%
\pgfpathlineto{\pgfqpoint{3.296975in}{3.342005in}}%
\pgfpathclose%
\pgfusepath{fill}%
\end{pgfscope}%
\begin{pgfscope}%
\pgfpathrectangle{\pgfqpoint{0.765000in}{0.660000in}}{\pgfqpoint{4.620000in}{4.620000in}}%
\pgfusepath{clip}%
\pgfsetbuttcap%
\pgfsetroundjoin%
\definecolor{currentfill}{rgb}{1.000000,0.894118,0.788235}%
\pgfsetfillcolor{currentfill}%
\pgfsetlinewidth{0.000000pt}%
\definecolor{currentstroke}{rgb}{1.000000,0.894118,0.788235}%
\pgfsetstrokecolor{currentstroke}%
\pgfsetdash{}{0pt}%
\pgfpathmoveto{\pgfqpoint{3.186079in}{3.406032in}}%
\pgfpathlineto{\pgfqpoint{3.296975in}{3.342005in}}%
\pgfpathlineto{\pgfqpoint{3.296975in}{3.470058in}}%
\pgfpathlineto{\pgfqpoint{3.186079in}{3.534084in}}%
\pgfpathlineto{\pgfqpoint{3.186079in}{3.406032in}}%
\pgfpathclose%
\pgfusepath{fill}%
\end{pgfscope}%
\begin{pgfscope}%
\pgfpathrectangle{\pgfqpoint{0.765000in}{0.660000in}}{\pgfqpoint{4.620000in}{4.620000in}}%
\pgfusepath{clip}%
\pgfsetbuttcap%
\pgfsetroundjoin%
\definecolor{currentfill}{rgb}{1.000000,0.894118,0.788235}%
\pgfsetfillcolor{currentfill}%
\pgfsetlinewidth{0.000000pt}%
\definecolor{currentstroke}{rgb}{1.000000,0.894118,0.788235}%
\pgfsetstrokecolor{currentstroke}%
\pgfsetdash{}{0pt}%
\pgfpathmoveto{\pgfqpoint{3.265487in}{3.617101in}}%
\pgfpathlineto{\pgfqpoint{3.154590in}{3.553075in}}%
\pgfpathlineto{\pgfqpoint{3.265487in}{3.489049in}}%
\pgfpathlineto{\pgfqpoint{3.376384in}{3.553075in}}%
\pgfpathlineto{\pgfqpoint{3.265487in}{3.617101in}}%
\pgfpathclose%
\pgfusepath{fill}%
\end{pgfscope}%
\begin{pgfscope}%
\pgfpathrectangle{\pgfqpoint{0.765000in}{0.660000in}}{\pgfqpoint{4.620000in}{4.620000in}}%
\pgfusepath{clip}%
\pgfsetbuttcap%
\pgfsetroundjoin%
\definecolor{currentfill}{rgb}{1.000000,0.894118,0.788235}%
\pgfsetfillcolor{currentfill}%
\pgfsetlinewidth{0.000000pt}%
\definecolor{currentstroke}{rgb}{1.000000,0.894118,0.788235}%
\pgfsetstrokecolor{currentstroke}%
\pgfsetdash{}{0pt}%
\pgfpathmoveto{\pgfqpoint{3.265487in}{3.360996in}}%
\pgfpathlineto{\pgfqpoint{3.376384in}{3.425023in}}%
\pgfpathlineto{\pgfqpoint{3.376384in}{3.553075in}}%
\pgfpathlineto{\pgfqpoint{3.265487in}{3.489049in}}%
\pgfpathlineto{\pgfqpoint{3.265487in}{3.360996in}}%
\pgfpathclose%
\pgfusepath{fill}%
\end{pgfscope}%
\begin{pgfscope}%
\pgfpathrectangle{\pgfqpoint{0.765000in}{0.660000in}}{\pgfqpoint{4.620000in}{4.620000in}}%
\pgfusepath{clip}%
\pgfsetbuttcap%
\pgfsetroundjoin%
\definecolor{currentfill}{rgb}{1.000000,0.894118,0.788235}%
\pgfsetfillcolor{currentfill}%
\pgfsetlinewidth{0.000000pt}%
\definecolor{currentstroke}{rgb}{1.000000,0.894118,0.788235}%
\pgfsetstrokecolor{currentstroke}%
\pgfsetdash{}{0pt}%
\pgfpathmoveto{\pgfqpoint{3.154590in}{3.425023in}}%
\pgfpathlineto{\pgfqpoint{3.265487in}{3.360996in}}%
\pgfpathlineto{\pgfqpoint{3.265487in}{3.489049in}}%
\pgfpathlineto{\pgfqpoint{3.154590in}{3.553075in}}%
\pgfpathlineto{\pgfqpoint{3.154590in}{3.425023in}}%
\pgfpathclose%
\pgfusepath{fill}%
\end{pgfscope}%
\begin{pgfscope}%
\pgfpathrectangle{\pgfqpoint{0.765000in}{0.660000in}}{\pgfqpoint{4.620000in}{4.620000in}}%
\pgfusepath{clip}%
\pgfsetbuttcap%
\pgfsetroundjoin%
\definecolor{currentfill}{rgb}{1.000000,0.894118,0.788235}%
\pgfsetfillcolor{currentfill}%
\pgfsetlinewidth{0.000000pt}%
\definecolor{currentstroke}{rgb}{1.000000,0.894118,0.788235}%
\pgfsetstrokecolor{currentstroke}%
\pgfsetdash{}{0pt}%
\pgfpathmoveto{\pgfqpoint{3.296975in}{3.470058in}}%
\pgfpathlineto{\pgfqpoint{3.186079in}{3.406032in}}%
\pgfpathlineto{\pgfqpoint{3.154590in}{3.425023in}}%
\pgfpathlineto{\pgfqpoint{3.265487in}{3.489049in}}%
\pgfpathlineto{\pgfqpoint{3.296975in}{3.470058in}}%
\pgfpathclose%
\pgfusepath{fill}%
\end{pgfscope}%
\begin{pgfscope}%
\pgfpathrectangle{\pgfqpoint{0.765000in}{0.660000in}}{\pgfqpoint{4.620000in}{4.620000in}}%
\pgfusepath{clip}%
\pgfsetbuttcap%
\pgfsetroundjoin%
\definecolor{currentfill}{rgb}{1.000000,0.894118,0.788235}%
\pgfsetfillcolor{currentfill}%
\pgfsetlinewidth{0.000000pt}%
\definecolor{currentstroke}{rgb}{1.000000,0.894118,0.788235}%
\pgfsetstrokecolor{currentstroke}%
\pgfsetdash{}{0pt}%
\pgfpathmoveto{\pgfqpoint{3.407872in}{3.406032in}}%
\pgfpathlineto{\pgfqpoint{3.296975in}{3.470058in}}%
\pgfpathlineto{\pgfqpoint{3.265487in}{3.489049in}}%
\pgfpathlineto{\pgfqpoint{3.376384in}{3.425023in}}%
\pgfpathlineto{\pgfqpoint{3.407872in}{3.406032in}}%
\pgfpathclose%
\pgfusepath{fill}%
\end{pgfscope}%
\begin{pgfscope}%
\pgfpathrectangle{\pgfqpoint{0.765000in}{0.660000in}}{\pgfqpoint{4.620000in}{4.620000in}}%
\pgfusepath{clip}%
\pgfsetbuttcap%
\pgfsetroundjoin%
\definecolor{currentfill}{rgb}{1.000000,0.894118,0.788235}%
\pgfsetfillcolor{currentfill}%
\pgfsetlinewidth{0.000000pt}%
\definecolor{currentstroke}{rgb}{1.000000,0.894118,0.788235}%
\pgfsetstrokecolor{currentstroke}%
\pgfsetdash{}{0pt}%
\pgfpathmoveto{\pgfqpoint{3.296975in}{3.470058in}}%
\pgfpathlineto{\pgfqpoint{3.296975in}{3.598110in}}%
\pgfpathlineto{\pgfqpoint{3.265487in}{3.617101in}}%
\pgfpathlineto{\pgfqpoint{3.376384in}{3.425023in}}%
\pgfpathlineto{\pgfqpoint{3.296975in}{3.470058in}}%
\pgfpathclose%
\pgfusepath{fill}%
\end{pgfscope}%
\begin{pgfscope}%
\pgfpathrectangle{\pgfqpoint{0.765000in}{0.660000in}}{\pgfqpoint{4.620000in}{4.620000in}}%
\pgfusepath{clip}%
\pgfsetbuttcap%
\pgfsetroundjoin%
\definecolor{currentfill}{rgb}{1.000000,0.894118,0.788235}%
\pgfsetfillcolor{currentfill}%
\pgfsetlinewidth{0.000000pt}%
\definecolor{currentstroke}{rgb}{1.000000,0.894118,0.788235}%
\pgfsetstrokecolor{currentstroke}%
\pgfsetdash{}{0pt}%
\pgfpathmoveto{\pgfqpoint{3.407872in}{3.406032in}}%
\pgfpathlineto{\pgfqpoint{3.407872in}{3.534084in}}%
\pgfpathlineto{\pgfqpoint{3.376384in}{3.553075in}}%
\pgfpathlineto{\pgfqpoint{3.265487in}{3.489049in}}%
\pgfpathlineto{\pgfqpoint{3.407872in}{3.406032in}}%
\pgfpathclose%
\pgfusepath{fill}%
\end{pgfscope}%
\begin{pgfscope}%
\pgfpathrectangle{\pgfqpoint{0.765000in}{0.660000in}}{\pgfqpoint{4.620000in}{4.620000in}}%
\pgfusepath{clip}%
\pgfsetbuttcap%
\pgfsetroundjoin%
\definecolor{currentfill}{rgb}{1.000000,0.894118,0.788235}%
\pgfsetfillcolor{currentfill}%
\pgfsetlinewidth{0.000000pt}%
\definecolor{currentstroke}{rgb}{1.000000,0.894118,0.788235}%
\pgfsetstrokecolor{currentstroke}%
\pgfsetdash{}{0pt}%
\pgfpathmoveto{\pgfqpoint{3.186079in}{3.406032in}}%
\pgfpathlineto{\pgfqpoint{3.296975in}{3.342005in}}%
\pgfpathlineto{\pgfqpoint{3.265487in}{3.360996in}}%
\pgfpathlineto{\pgfqpoint{3.154590in}{3.425023in}}%
\pgfpathlineto{\pgfqpoint{3.186079in}{3.406032in}}%
\pgfpathclose%
\pgfusepath{fill}%
\end{pgfscope}%
\begin{pgfscope}%
\pgfpathrectangle{\pgfqpoint{0.765000in}{0.660000in}}{\pgfqpoint{4.620000in}{4.620000in}}%
\pgfusepath{clip}%
\pgfsetbuttcap%
\pgfsetroundjoin%
\definecolor{currentfill}{rgb}{1.000000,0.894118,0.788235}%
\pgfsetfillcolor{currentfill}%
\pgfsetlinewidth{0.000000pt}%
\definecolor{currentstroke}{rgb}{1.000000,0.894118,0.788235}%
\pgfsetstrokecolor{currentstroke}%
\pgfsetdash{}{0pt}%
\pgfpathmoveto{\pgfqpoint{3.296975in}{3.342005in}}%
\pgfpathlineto{\pgfqpoint{3.407872in}{3.406032in}}%
\pgfpathlineto{\pgfqpoint{3.376384in}{3.425023in}}%
\pgfpathlineto{\pgfqpoint{3.265487in}{3.360996in}}%
\pgfpathlineto{\pgfqpoint{3.296975in}{3.342005in}}%
\pgfpathclose%
\pgfusepath{fill}%
\end{pgfscope}%
\begin{pgfscope}%
\pgfpathrectangle{\pgfqpoint{0.765000in}{0.660000in}}{\pgfqpoint{4.620000in}{4.620000in}}%
\pgfusepath{clip}%
\pgfsetbuttcap%
\pgfsetroundjoin%
\definecolor{currentfill}{rgb}{1.000000,0.894118,0.788235}%
\pgfsetfillcolor{currentfill}%
\pgfsetlinewidth{0.000000pt}%
\definecolor{currentstroke}{rgb}{1.000000,0.894118,0.788235}%
\pgfsetstrokecolor{currentstroke}%
\pgfsetdash{}{0pt}%
\pgfpathmoveto{\pgfqpoint{3.296975in}{3.598110in}}%
\pgfpathlineto{\pgfqpoint{3.186079in}{3.534084in}}%
\pgfpathlineto{\pgfqpoint{3.154590in}{3.553075in}}%
\pgfpathlineto{\pgfqpoint{3.265487in}{3.617101in}}%
\pgfpathlineto{\pgfqpoint{3.296975in}{3.598110in}}%
\pgfpathclose%
\pgfusepath{fill}%
\end{pgfscope}%
\begin{pgfscope}%
\pgfpathrectangle{\pgfqpoint{0.765000in}{0.660000in}}{\pgfqpoint{4.620000in}{4.620000in}}%
\pgfusepath{clip}%
\pgfsetbuttcap%
\pgfsetroundjoin%
\definecolor{currentfill}{rgb}{1.000000,0.894118,0.788235}%
\pgfsetfillcolor{currentfill}%
\pgfsetlinewidth{0.000000pt}%
\definecolor{currentstroke}{rgb}{1.000000,0.894118,0.788235}%
\pgfsetstrokecolor{currentstroke}%
\pgfsetdash{}{0pt}%
\pgfpathmoveto{\pgfqpoint{3.407872in}{3.534084in}}%
\pgfpathlineto{\pgfqpoint{3.296975in}{3.598110in}}%
\pgfpathlineto{\pgfqpoint{3.265487in}{3.617101in}}%
\pgfpathlineto{\pgfqpoint{3.376384in}{3.553075in}}%
\pgfpathlineto{\pgfqpoint{3.407872in}{3.534084in}}%
\pgfpathclose%
\pgfusepath{fill}%
\end{pgfscope}%
\begin{pgfscope}%
\pgfpathrectangle{\pgfqpoint{0.765000in}{0.660000in}}{\pgfqpoint{4.620000in}{4.620000in}}%
\pgfusepath{clip}%
\pgfsetbuttcap%
\pgfsetroundjoin%
\definecolor{currentfill}{rgb}{1.000000,0.894118,0.788235}%
\pgfsetfillcolor{currentfill}%
\pgfsetlinewidth{0.000000pt}%
\definecolor{currentstroke}{rgb}{1.000000,0.894118,0.788235}%
\pgfsetstrokecolor{currentstroke}%
\pgfsetdash{}{0pt}%
\pgfpathmoveto{\pgfqpoint{3.186079in}{3.406032in}}%
\pgfpathlineto{\pgfqpoint{3.186079in}{3.534084in}}%
\pgfpathlineto{\pgfqpoint{3.154590in}{3.553075in}}%
\pgfpathlineto{\pgfqpoint{3.265487in}{3.360996in}}%
\pgfpathlineto{\pgfqpoint{3.186079in}{3.406032in}}%
\pgfpathclose%
\pgfusepath{fill}%
\end{pgfscope}%
\begin{pgfscope}%
\pgfpathrectangle{\pgfqpoint{0.765000in}{0.660000in}}{\pgfqpoint{4.620000in}{4.620000in}}%
\pgfusepath{clip}%
\pgfsetbuttcap%
\pgfsetroundjoin%
\definecolor{currentfill}{rgb}{1.000000,0.894118,0.788235}%
\pgfsetfillcolor{currentfill}%
\pgfsetlinewidth{0.000000pt}%
\definecolor{currentstroke}{rgb}{1.000000,0.894118,0.788235}%
\pgfsetstrokecolor{currentstroke}%
\pgfsetdash{}{0pt}%
\pgfpathmoveto{\pgfqpoint{3.296975in}{3.342005in}}%
\pgfpathlineto{\pgfqpoint{3.296975in}{3.470058in}}%
\pgfpathlineto{\pgfqpoint{3.265487in}{3.489049in}}%
\pgfpathlineto{\pgfqpoint{3.154590in}{3.425023in}}%
\pgfpathlineto{\pgfqpoint{3.296975in}{3.342005in}}%
\pgfpathclose%
\pgfusepath{fill}%
\end{pgfscope}%
\begin{pgfscope}%
\pgfpathrectangle{\pgfqpoint{0.765000in}{0.660000in}}{\pgfqpoint{4.620000in}{4.620000in}}%
\pgfusepath{clip}%
\pgfsetbuttcap%
\pgfsetroundjoin%
\definecolor{currentfill}{rgb}{1.000000,0.894118,0.788235}%
\pgfsetfillcolor{currentfill}%
\pgfsetlinewidth{0.000000pt}%
\definecolor{currentstroke}{rgb}{1.000000,0.894118,0.788235}%
\pgfsetstrokecolor{currentstroke}%
\pgfsetdash{}{0pt}%
\pgfpathmoveto{\pgfqpoint{3.296975in}{3.470058in}}%
\pgfpathlineto{\pgfqpoint{3.407872in}{3.534084in}}%
\pgfpathlineto{\pgfqpoint{3.376384in}{3.553075in}}%
\pgfpathlineto{\pgfqpoint{3.265487in}{3.489049in}}%
\pgfpathlineto{\pgfqpoint{3.296975in}{3.470058in}}%
\pgfpathclose%
\pgfusepath{fill}%
\end{pgfscope}%
\begin{pgfscope}%
\pgfpathrectangle{\pgfqpoint{0.765000in}{0.660000in}}{\pgfqpoint{4.620000in}{4.620000in}}%
\pgfusepath{clip}%
\pgfsetbuttcap%
\pgfsetroundjoin%
\definecolor{currentfill}{rgb}{1.000000,0.894118,0.788235}%
\pgfsetfillcolor{currentfill}%
\pgfsetlinewidth{0.000000pt}%
\definecolor{currentstroke}{rgb}{1.000000,0.894118,0.788235}%
\pgfsetstrokecolor{currentstroke}%
\pgfsetdash{}{0pt}%
\pgfpathmoveto{\pgfqpoint{3.186079in}{3.534084in}}%
\pgfpathlineto{\pgfqpoint{3.296975in}{3.470058in}}%
\pgfpathlineto{\pgfqpoint{3.265487in}{3.489049in}}%
\pgfpathlineto{\pgfqpoint{3.154590in}{3.553075in}}%
\pgfpathlineto{\pgfqpoint{3.186079in}{3.534084in}}%
\pgfpathclose%
\pgfusepath{fill}%
\end{pgfscope}%
\begin{pgfscope}%
\pgfpathrectangle{\pgfqpoint{0.765000in}{0.660000in}}{\pgfqpoint{4.620000in}{4.620000in}}%
\pgfusepath{clip}%
\pgfsetbuttcap%
\pgfsetroundjoin%
\definecolor{currentfill}{rgb}{1.000000,0.894118,0.788235}%
\pgfsetfillcolor{currentfill}%
\pgfsetlinewidth{0.000000pt}%
\definecolor{currentstroke}{rgb}{1.000000,0.894118,0.788235}%
\pgfsetstrokecolor{currentstroke}%
\pgfsetdash{}{0pt}%
\pgfpathmoveto{\pgfqpoint{3.296975in}{3.470058in}}%
\pgfpathlineto{\pgfqpoint{3.186079in}{3.406032in}}%
\pgfpathlineto{\pgfqpoint{3.296975in}{3.342005in}}%
\pgfpathlineto{\pgfqpoint{3.407872in}{3.406032in}}%
\pgfpathlineto{\pgfqpoint{3.296975in}{3.470058in}}%
\pgfpathclose%
\pgfusepath{fill}%
\end{pgfscope}%
\begin{pgfscope}%
\pgfpathrectangle{\pgfqpoint{0.765000in}{0.660000in}}{\pgfqpoint{4.620000in}{4.620000in}}%
\pgfusepath{clip}%
\pgfsetbuttcap%
\pgfsetroundjoin%
\definecolor{currentfill}{rgb}{1.000000,0.894118,0.788235}%
\pgfsetfillcolor{currentfill}%
\pgfsetlinewidth{0.000000pt}%
\definecolor{currentstroke}{rgb}{1.000000,0.894118,0.788235}%
\pgfsetstrokecolor{currentstroke}%
\pgfsetdash{}{0pt}%
\pgfpathmoveto{\pgfqpoint{3.296975in}{3.470058in}}%
\pgfpathlineto{\pgfqpoint{3.186079in}{3.406032in}}%
\pgfpathlineto{\pgfqpoint{3.186079in}{3.534084in}}%
\pgfpathlineto{\pgfqpoint{3.296975in}{3.598110in}}%
\pgfpathlineto{\pgfqpoint{3.296975in}{3.470058in}}%
\pgfpathclose%
\pgfusepath{fill}%
\end{pgfscope}%
\begin{pgfscope}%
\pgfpathrectangle{\pgfqpoint{0.765000in}{0.660000in}}{\pgfqpoint{4.620000in}{4.620000in}}%
\pgfusepath{clip}%
\pgfsetbuttcap%
\pgfsetroundjoin%
\definecolor{currentfill}{rgb}{1.000000,0.894118,0.788235}%
\pgfsetfillcolor{currentfill}%
\pgfsetlinewidth{0.000000pt}%
\definecolor{currentstroke}{rgb}{1.000000,0.894118,0.788235}%
\pgfsetstrokecolor{currentstroke}%
\pgfsetdash{}{0pt}%
\pgfpathmoveto{\pgfqpoint{3.296975in}{3.470058in}}%
\pgfpathlineto{\pgfqpoint{3.407872in}{3.406032in}}%
\pgfpathlineto{\pgfqpoint{3.407872in}{3.534084in}}%
\pgfpathlineto{\pgfqpoint{3.296975in}{3.598110in}}%
\pgfpathlineto{\pgfqpoint{3.296975in}{3.470058in}}%
\pgfpathclose%
\pgfusepath{fill}%
\end{pgfscope}%
\begin{pgfscope}%
\pgfpathrectangle{\pgfqpoint{0.765000in}{0.660000in}}{\pgfqpoint{4.620000in}{4.620000in}}%
\pgfusepath{clip}%
\pgfsetbuttcap%
\pgfsetroundjoin%
\definecolor{currentfill}{rgb}{1.000000,0.894118,0.788235}%
\pgfsetfillcolor{currentfill}%
\pgfsetlinewidth{0.000000pt}%
\definecolor{currentstroke}{rgb}{1.000000,0.894118,0.788235}%
\pgfsetstrokecolor{currentstroke}%
\pgfsetdash{}{0pt}%
\pgfpathmoveto{\pgfqpoint{3.310017in}{3.461366in}}%
\pgfpathlineto{\pgfqpoint{3.199120in}{3.397340in}}%
\pgfpathlineto{\pgfqpoint{3.310017in}{3.333313in}}%
\pgfpathlineto{\pgfqpoint{3.420914in}{3.397340in}}%
\pgfpathlineto{\pgfqpoint{3.310017in}{3.461366in}}%
\pgfpathclose%
\pgfusepath{fill}%
\end{pgfscope}%
\begin{pgfscope}%
\pgfpathrectangle{\pgfqpoint{0.765000in}{0.660000in}}{\pgfqpoint{4.620000in}{4.620000in}}%
\pgfusepath{clip}%
\pgfsetbuttcap%
\pgfsetroundjoin%
\definecolor{currentfill}{rgb}{1.000000,0.894118,0.788235}%
\pgfsetfillcolor{currentfill}%
\pgfsetlinewidth{0.000000pt}%
\definecolor{currentstroke}{rgb}{1.000000,0.894118,0.788235}%
\pgfsetstrokecolor{currentstroke}%
\pgfsetdash{}{0pt}%
\pgfpathmoveto{\pgfqpoint{3.310017in}{3.461366in}}%
\pgfpathlineto{\pgfqpoint{3.199120in}{3.397340in}}%
\pgfpathlineto{\pgfqpoint{3.199120in}{3.525392in}}%
\pgfpathlineto{\pgfqpoint{3.310017in}{3.589418in}}%
\pgfpathlineto{\pgfqpoint{3.310017in}{3.461366in}}%
\pgfpathclose%
\pgfusepath{fill}%
\end{pgfscope}%
\begin{pgfscope}%
\pgfpathrectangle{\pgfqpoint{0.765000in}{0.660000in}}{\pgfqpoint{4.620000in}{4.620000in}}%
\pgfusepath{clip}%
\pgfsetbuttcap%
\pgfsetroundjoin%
\definecolor{currentfill}{rgb}{1.000000,0.894118,0.788235}%
\pgfsetfillcolor{currentfill}%
\pgfsetlinewidth{0.000000pt}%
\definecolor{currentstroke}{rgb}{1.000000,0.894118,0.788235}%
\pgfsetstrokecolor{currentstroke}%
\pgfsetdash{}{0pt}%
\pgfpathmoveto{\pgfqpoint{3.310017in}{3.461366in}}%
\pgfpathlineto{\pgfqpoint{3.420914in}{3.397340in}}%
\pgfpathlineto{\pgfqpoint{3.420914in}{3.525392in}}%
\pgfpathlineto{\pgfqpoint{3.310017in}{3.589418in}}%
\pgfpathlineto{\pgfqpoint{3.310017in}{3.461366in}}%
\pgfpathclose%
\pgfusepath{fill}%
\end{pgfscope}%
\begin{pgfscope}%
\pgfpathrectangle{\pgfqpoint{0.765000in}{0.660000in}}{\pgfqpoint{4.620000in}{4.620000in}}%
\pgfusepath{clip}%
\pgfsetbuttcap%
\pgfsetroundjoin%
\definecolor{currentfill}{rgb}{1.000000,0.894118,0.788235}%
\pgfsetfillcolor{currentfill}%
\pgfsetlinewidth{0.000000pt}%
\definecolor{currentstroke}{rgb}{1.000000,0.894118,0.788235}%
\pgfsetstrokecolor{currentstroke}%
\pgfsetdash{}{0pt}%
\pgfpathmoveto{\pgfqpoint{3.296975in}{3.598110in}}%
\pgfpathlineto{\pgfqpoint{3.186079in}{3.534084in}}%
\pgfpathlineto{\pgfqpoint{3.296975in}{3.470058in}}%
\pgfpathlineto{\pgfqpoint{3.407872in}{3.534084in}}%
\pgfpathlineto{\pgfqpoint{3.296975in}{3.598110in}}%
\pgfpathclose%
\pgfusepath{fill}%
\end{pgfscope}%
\begin{pgfscope}%
\pgfpathrectangle{\pgfqpoint{0.765000in}{0.660000in}}{\pgfqpoint{4.620000in}{4.620000in}}%
\pgfusepath{clip}%
\pgfsetbuttcap%
\pgfsetroundjoin%
\definecolor{currentfill}{rgb}{1.000000,0.894118,0.788235}%
\pgfsetfillcolor{currentfill}%
\pgfsetlinewidth{0.000000pt}%
\definecolor{currentstroke}{rgb}{1.000000,0.894118,0.788235}%
\pgfsetstrokecolor{currentstroke}%
\pgfsetdash{}{0pt}%
\pgfpathmoveto{\pgfqpoint{3.296975in}{3.342005in}}%
\pgfpathlineto{\pgfqpoint{3.407872in}{3.406032in}}%
\pgfpathlineto{\pgfqpoint{3.407872in}{3.534084in}}%
\pgfpathlineto{\pgfqpoint{3.296975in}{3.470058in}}%
\pgfpathlineto{\pgfqpoint{3.296975in}{3.342005in}}%
\pgfpathclose%
\pgfusepath{fill}%
\end{pgfscope}%
\begin{pgfscope}%
\pgfpathrectangle{\pgfqpoint{0.765000in}{0.660000in}}{\pgfqpoint{4.620000in}{4.620000in}}%
\pgfusepath{clip}%
\pgfsetbuttcap%
\pgfsetroundjoin%
\definecolor{currentfill}{rgb}{1.000000,0.894118,0.788235}%
\pgfsetfillcolor{currentfill}%
\pgfsetlinewidth{0.000000pt}%
\definecolor{currentstroke}{rgb}{1.000000,0.894118,0.788235}%
\pgfsetstrokecolor{currentstroke}%
\pgfsetdash{}{0pt}%
\pgfpathmoveto{\pgfqpoint{3.186079in}{3.406032in}}%
\pgfpathlineto{\pgfqpoint{3.296975in}{3.342005in}}%
\pgfpathlineto{\pgfqpoint{3.296975in}{3.470058in}}%
\pgfpathlineto{\pgfqpoint{3.186079in}{3.534084in}}%
\pgfpathlineto{\pgfqpoint{3.186079in}{3.406032in}}%
\pgfpathclose%
\pgfusepath{fill}%
\end{pgfscope}%
\begin{pgfscope}%
\pgfpathrectangle{\pgfqpoint{0.765000in}{0.660000in}}{\pgfqpoint{4.620000in}{4.620000in}}%
\pgfusepath{clip}%
\pgfsetbuttcap%
\pgfsetroundjoin%
\definecolor{currentfill}{rgb}{1.000000,0.894118,0.788235}%
\pgfsetfillcolor{currentfill}%
\pgfsetlinewidth{0.000000pt}%
\definecolor{currentstroke}{rgb}{1.000000,0.894118,0.788235}%
\pgfsetstrokecolor{currentstroke}%
\pgfsetdash{}{0pt}%
\pgfpathmoveto{\pgfqpoint{3.310017in}{3.589418in}}%
\pgfpathlineto{\pgfqpoint{3.199120in}{3.525392in}}%
\pgfpathlineto{\pgfqpoint{3.310017in}{3.461366in}}%
\pgfpathlineto{\pgfqpoint{3.420914in}{3.525392in}}%
\pgfpathlineto{\pgfqpoint{3.310017in}{3.589418in}}%
\pgfpathclose%
\pgfusepath{fill}%
\end{pgfscope}%
\begin{pgfscope}%
\pgfpathrectangle{\pgfqpoint{0.765000in}{0.660000in}}{\pgfqpoint{4.620000in}{4.620000in}}%
\pgfusepath{clip}%
\pgfsetbuttcap%
\pgfsetroundjoin%
\definecolor{currentfill}{rgb}{1.000000,0.894118,0.788235}%
\pgfsetfillcolor{currentfill}%
\pgfsetlinewidth{0.000000pt}%
\definecolor{currentstroke}{rgb}{1.000000,0.894118,0.788235}%
\pgfsetstrokecolor{currentstroke}%
\pgfsetdash{}{0pt}%
\pgfpathmoveto{\pgfqpoint{3.310017in}{3.333313in}}%
\pgfpathlineto{\pgfqpoint{3.420914in}{3.397340in}}%
\pgfpathlineto{\pgfqpoint{3.420914in}{3.525392in}}%
\pgfpathlineto{\pgfqpoint{3.310017in}{3.461366in}}%
\pgfpathlineto{\pgfqpoint{3.310017in}{3.333313in}}%
\pgfpathclose%
\pgfusepath{fill}%
\end{pgfscope}%
\begin{pgfscope}%
\pgfpathrectangle{\pgfqpoint{0.765000in}{0.660000in}}{\pgfqpoint{4.620000in}{4.620000in}}%
\pgfusepath{clip}%
\pgfsetbuttcap%
\pgfsetroundjoin%
\definecolor{currentfill}{rgb}{1.000000,0.894118,0.788235}%
\pgfsetfillcolor{currentfill}%
\pgfsetlinewidth{0.000000pt}%
\definecolor{currentstroke}{rgb}{1.000000,0.894118,0.788235}%
\pgfsetstrokecolor{currentstroke}%
\pgfsetdash{}{0pt}%
\pgfpathmoveto{\pgfqpoint{3.199120in}{3.397340in}}%
\pgfpathlineto{\pgfqpoint{3.310017in}{3.333313in}}%
\pgfpathlineto{\pgfqpoint{3.310017in}{3.461366in}}%
\pgfpathlineto{\pgfqpoint{3.199120in}{3.525392in}}%
\pgfpathlineto{\pgfqpoint{3.199120in}{3.397340in}}%
\pgfpathclose%
\pgfusepath{fill}%
\end{pgfscope}%
\begin{pgfscope}%
\pgfpathrectangle{\pgfqpoint{0.765000in}{0.660000in}}{\pgfqpoint{4.620000in}{4.620000in}}%
\pgfusepath{clip}%
\pgfsetbuttcap%
\pgfsetroundjoin%
\definecolor{currentfill}{rgb}{1.000000,0.894118,0.788235}%
\pgfsetfillcolor{currentfill}%
\pgfsetlinewidth{0.000000pt}%
\definecolor{currentstroke}{rgb}{1.000000,0.894118,0.788235}%
\pgfsetstrokecolor{currentstroke}%
\pgfsetdash{}{0pt}%
\pgfpathmoveto{\pgfqpoint{3.296975in}{3.470058in}}%
\pgfpathlineto{\pgfqpoint{3.186079in}{3.406032in}}%
\pgfpathlineto{\pgfqpoint{3.199120in}{3.397340in}}%
\pgfpathlineto{\pgfqpoint{3.310017in}{3.461366in}}%
\pgfpathlineto{\pgfqpoint{3.296975in}{3.470058in}}%
\pgfpathclose%
\pgfusepath{fill}%
\end{pgfscope}%
\begin{pgfscope}%
\pgfpathrectangle{\pgfqpoint{0.765000in}{0.660000in}}{\pgfqpoint{4.620000in}{4.620000in}}%
\pgfusepath{clip}%
\pgfsetbuttcap%
\pgfsetroundjoin%
\definecolor{currentfill}{rgb}{1.000000,0.894118,0.788235}%
\pgfsetfillcolor{currentfill}%
\pgfsetlinewidth{0.000000pt}%
\definecolor{currentstroke}{rgb}{1.000000,0.894118,0.788235}%
\pgfsetstrokecolor{currentstroke}%
\pgfsetdash{}{0pt}%
\pgfpathmoveto{\pgfqpoint{3.407872in}{3.406032in}}%
\pgfpathlineto{\pgfqpoint{3.296975in}{3.470058in}}%
\pgfpathlineto{\pgfqpoint{3.310017in}{3.461366in}}%
\pgfpathlineto{\pgfqpoint{3.420914in}{3.397340in}}%
\pgfpathlineto{\pgfqpoint{3.407872in}{3.406032in}}%
\pgfpathclose%
\pgfusepath{fill}%
\end{pgfscope}%
\begin{pgfscope}%
\pgfpathrectangle{\pgfqpoint{0.765000in}{0.660000in}}{\pgfqpoint{4.620000in}{4.620000in}}%
\pgfusepath{clip}%
\pgfsetbuttcap%
\pgfsetroundjoin%
\definecolor{currentfill}{rgb}{1.000000,0.894118,0.788235}%
\pgfsetfillcolor{currentfill}%
\pgfsetlinewidth{0.000000pt}%
\definecolor{currentstroke}{rgb}{1.000000,0.894118,0.788235}%
\pgfsetstrokecolor{currentstroke}%
\pgfsetdash{}{0pt}%
\pgfpathmoveto{\pgfqpoint{3.296975in}{3.470058in}}%
\pgfpathlineto{\pgfqpoint{3.296975in}{3.598110in}}%
\pgfpathlineto{\pgfqpoint{3.310017in}{3.589418in}}%
\pgfpathlineto{\pgfqpoint{3.420914in}{3.397340in}}%
\pgfpathlineto{\pgfqpoint{3.296975in}{3.470058in}}%
\pgfpathclose%
\pgfusepath{fill}%
\end{pgfscope}%
\begin{pgfscope}%
\pgfpathrectangle{\pgfqpoint{0.765000in}{0.660000in}}{\pgfqpoint{4.620000in}{4.620000in}}%
\pgfusepath{clip}%
\pgfsetbuttcap%
\pgfsetroundjoin%
\definecolor{currentfill}{rgb}{1.000000,0.894118,0.788235}%
\pgfsetfillcolor{currentfill}%
\pgfsetlinewidth{0.000000pt}%
\definecolor{currentstroke}{rgb}{1.000000,0.894118,0.788235}%
\pgfsetstrokecolor{currentstroke}%
\pgfsetdash{}{0pt}%
\pgfpathmoveto{\pgfqpoint{3.407872in}{3.406032in}}%
\pgfpathlineto{\pgfqpoint{3.407872in}{3.534084in}}%
\pgfpathlineto{\pgfqpoint{3.420914in}{3.525392in}}%
\pgfpathlineto{\pgfqpoint{3.310017in}{3.461366in}}%
\pgfpathlineto{\pgfqpoint{3.407872in}{3.406032in}}%
\pgfpathclose%
\pgfusepath{fill}%
\end{pgfscope}%
\begin{pgfscope}%
\pgfpathrectangle{\pgfqpoint{0.765000in}{0.660000in}}{\pgfqpoint{4.620000in}{4.620000in}}%
\pgfusepath{clip}%
\pgfsetbuttcap%
\pgfsetroundjoin%
\definecolor{currentfill}{rgb}{1.000000,0.894118,0.788235}%
\pgfsetfillcolor{currentfill}%
\pgfsetlinewidth{0.000000pt}%
\definecolor{currentstroke}{rgb}{1.000000,0.894118,0.788235}%
\pgfsetstrokecolor{currentstroke}%
\pgfsetdash{}{0pt}%
\pgfpathmoveto{\pgfqpoint{3.186079in}{3.406032in}}%
\pgfpathlineto{\pgfqpoint{3.296975in}{3.342005in}}%
\pgfpathlineto{\pgfqpoint{3.310017in}{3.333313in}}%
\pgfpathlineto{\pgfqpoint{3.199120in}{3.397340in}}%
\pgfpathlineto{\pgfqpoint{3.186079in}{3.406032in}}%
\pgfpathclose%
\pgfusepath{fill}%
\end{pgfscope}%
\begin{pgfscope}%
\pgfpathrectangle{\pgfqpoint{0.765000in}{0.660000in}}{\pgfqpoint{4.620000in}{4.620000in}}%
\pgfusepath{clip}%
\pgfsetbuttcap%
\pgfsetroundjoin%
\definecolor{currentfill}{rgb}{1.000000,0.894118,0.788235}%
\pgfsetfillcolor{currentfill}%
\pgfsetlinewidth{0.000000pt}%
\definecolor{currentstroke}{rgb}{1.000000,0.894118,0.788235}%
\pgfsetstrokecolor{currentstroke}%
\pgfsetdash{}{0pt}%
\pgfpathmoveto{\pgfqpoint{3.296975in}{3.342005in}}%
\pgfpathlineto{\pgfqpoint{3.407872in}{3.406032in}}%
\pgfpathlineto{\pgfqpoint{3.420914in}{3.397340in}}%
\pgfpathlineto{\pgfqpoint{3.310017in}{3.333313in}}%
\pgfpathlineto{\pgfqpoint{3.296975in}{3.342005in}}%
\pgfpathclose%
\pgfusepath{fill}%
\end{pgfscope}%
\begin{pgfscope}%
\pgfpathrectangle{\pgfqpoint{0.765000in}{0.660000in}}{\pgfqpoint{4.620000in}{4.620000in}}%
\pgfusepath{clip}%
\pgfsetbuttcap%
\pgfsetroundjoin%
\definecolor{currentfill}{rgb}{1.000000,0.894118,0.788235}%
\pgfsetfillcolor{currentfill}%
\pgfsetlinewidth{0.000000pt}%
\definecolor{currentstroke}{rgb}{1.000000,0.894118,0.788235}%
\pgfsetstrokecolor{currentstroke}%
\pgfsetdash{}{0pt}%
\pgfpathmoveto{\pgfqpoint{3.296975in}{3.598110in}}%
\pgfpathlineto{\pgfqpoint{3.186079in}{3.534084in}}%
\pgfpathlineto{\pgfqpoint{3.199120in}{3.525392in}}%
\pgfpathlineto{\pgfqpoint{3.310017in}{3.589418in}}%
\pgfpathlineto{\pgfqpoint{3.296975in}{3.598110in}}%
\pgfpathclose%
\pgfusepath{fill}%
\end{pgfscope}%
\begin{pgfscope}%
\pgfpathrectangle{\pgfqpoint{0.765000in}{0.660000in}}{\pgfqpoint{4.620000in}{4.620000in}}%
\pgfusepath{clip}%
\pgfsetbuttcap%
\pgfsetroundjoin%
\definecolor{currentfill}{rgb}{1.000000,0.894118,0.788235}%
\pgfsetfillcolor{currentfill}%
\pgfsetlinewidth{0.000000pt}%
\definecolor{currentstroke}{rgb}{1.000000,0.894118,0.788235}%
\pgfsetstrokecolor{currentstroke}%
\pgfsetdash{}{0pt}%
\pgfpathmoveto{\pgfqpoint{3.407872in}{3.534084in}}%
\pgfpathlineto{\pgfqpoint{3.296975in}{3.598110in}}%
\pgfpathlineto{\pgfqpoint{3.310017in}{3.589418in}}%
\pgfpathlineto{\pgfqpoint{3.420914in}{3.525392in}}%
\pgfpathlineto{\pgfqpoint{3.407872in}{3.534084in}}%
\pgfpathclose%
\pgfusepath{fill}%
\end{pgfscope}%
\begin{pgfscope}%
\pgfpathrectangle{\pgfqpoint{0.765000in}{0.660000in}}{\pgfqpoint{4.620000in}{4.620000in}}%
\pgfusepath{clip}%
\pgfsetbuttcap%
\pgfsetroundjoin%
\definecolor{currentfill}{rgb}{1.000000,0.894118,0.788235}%
\pgfsetfillcolor{currentfill}%
\pgfsetlinewidth{0.000000pt}%
\definecolor{currentstroke}{rgb}{1.000000,0.894118,0.788235}%
\pgfsetstrokecolor{currentstroke}%
\pgfsetdash{}{0pt}%
\pgfpathmoveto{\pgfqpoint{3.186079in}{3.406032in}}%
\pgfpathlineto{\pgfqpoint{3.186079in}{3.534084in}}%
\pgfpathlineto{\pgfqpoint{3.199120in}{3.525392in}}%
\pgfpathlineto{\pgfqpoint{3.310017in}{3.333313in}}%
\pgfpathlineto{\pgfqpoint{3.186079in}{3.406032in}}%
\pgfpathclose%
\pgfusepath{fill}%
\end{pgfscope}%
\begin{pgfscope}%
\pgfpathrectangle{\pgfqpoint{0.765000in}{0.660000in}}{\pgfqpoint{4.620000in}{4.620000in}}%
\pgfusepath{clip}%
\pgfsetbuttcap%
\pgfsetroundjoin%
\definecolor{currentfill}{rgb}{1.000000,0.894118,0.788235}%
\pgfsetfillcolor{currentfill}%
\pgfsetlinewidth{0.000000pt}%
\definecolor{currentstroke}{rgb}{1.000000,0.894118,0.788235}%
\pgfsetstrokecolor{currentstroke}%
\pgfsetdash{}{0pt}%
\pgfpathmoveto{\pgfqpoint{3.296975in}{3.342005in}}%
\pgfpathlineto{\pgfqpoint{3.296975in}{3.470058in}}%
\pgfpathlineto{\pgfqpoint{3.310017in}{3.461366in}}%
\pgfpathlineto{\pgfqpoint{3.199120in}{3.397340in}}%
\pgfpathlineto{\pgfqpoint{3.296975in}{3.342005in}}%
\pgfpathclose%
\pgfusepath{fill}%
\end{pgfscope}%
\begin{pgfscope}%
\pgfpathrectangle{\pgfqpoint{0.765000in}{0.660000in}}{\pgfqpoint{4.620000in}{4.620000in}}%
\pgfusepath{clip}%
\pgfsetbuttcap%
\pgfsetroundjoin%
\definecolor{currentfill}{rgb}{1.000000,0.894118,0.788235}%
\pgfsetfillcolor{currentfill}%
\pgfsetlinewidth{0.000000pt}%
\definecolor{currentstroke}{rgb}{1.000000,0.894118,0.788235}%
\pgfsetstrokecolor{currentstroke}%
\pgfsetdash{}{0pt}%
\pgfpathmoveto{\pgfqpoint{3.296975in}{3.470058in}}%
\pgfpathlineto{\pgfqpoint{3.407872in}{3.534084in}}%
\pgfpathlineto{\pgfqpoint{3.420914in}{3.525392in}}%
\pgfpathlineto{\pgfqpoint{3.310017in}{3.461366in}}%
\pgfpathlineto{\pgfqpoint{3.296975in}{3.470058in}}%
\pgfpathclose%
\pgfusepath{fill}%
\end{pgfscope}%
\begin{pgfscope}%
\pgfpathrectangle{\pgfqpoint{0.765000in}{0.660000in}}{\pgfqpoint{4.620000in}{4.620000in}}%
\pgfusepath{clip}%
\pgfsetbuttcap%
\pgfsetroundjoin%
\definecolor{currentfill}{rgb}{1.000000,0.894118,0.788235}%
\pgfsetfillcolor{currentfill}%
\pgfsetlinewidth{0.000000pt}%
\definecolor{currentstroke}{rgb}{1.000000,0.894118,0.788235}%
\pgfsetstrokecolor{currentstroke}%
\pgfsetdash{}{0pt}%
\pgfpathmoveto{\pgfqpoint{3.186079in}{3.534084in}}%
\pgfpathlineto{\pgfqpoint{3.296975in}{3.470058in}}%
\pgfpathlineto{\pgfqpoint{3.310017in}{3.461366in}}%
\pgfpathlineto{\pgfqpoint{3.199120in}{3.525392in}}%
\pgfpathlineto{\pgfqpoint{3.186079in}{3.534084in}}%
\pgfpathclose%
\pgfusepath{fill}%
\end{pgfscope}%
\begin{pgfscope}%
\pgfpathrectangle{\pgfqpoint{0.765000in}{0.660000in}}{\pgfqpoint{4.620000in}{4.620000in}}%
\pgfusepath{clip}%
\pgfsetbuttcap%
\pgfsetroundjoin%
\definecolor{currentfill}{rgb}{1.000000,0.894118,0.788235}%
\pgfsetfillcolor{currentfill}%
\pgfsetlinewidth{0.000000pt}%
\definecolor{currentstroke}{rgb}{1.000000,0.894118,0.788235}%
\pgfsetstrokecolor{currentstroke}%
\pgfsetdash{}{0pt}%
\pgfpathmoveto{\pgfqpoint{3.265487in}{3.489049in}}%
\pgfpathlineto{\pgfqpoint{3.154590in}{3.425023in}}%
\pgfpathlineto{\pgfqpoint{3.265487in}{3.360996in}}%
\pgfpathlineto{\pgfqpoint{3.376384in}{3.425023in}}%
\pgfpathlineto{\pgfqpoint{3.265487in}{3.489049in}}%
\pgfpathclose%
\pgfusepath{fill}%
\end{pgfscope}%
\begin{pgfscope}%
\pgfpathrectangle{\pgfqpoint{0.765000in}{0.660000in}}{\pgfqpoint{4.620000in}{4.620000in}}%
\pgfusepath{clip}%
\pgfsetbuttcap%
\pgfsetroundjoin%
\definecolor{currentfill}{rgb}{1.000000,0.894118,0.788235}%
\pgfsetfillcolor{currentfill}%
\pgfsetlinewidth{0.000000pt}%
\definecolor{currentstroke}{rgb}{1.000000,0.894118,0.788235}%
\pgfsetstrokecolor{currentstroke}%
\pgfsetdash{}{0pt}%
\pgfpathmoveto{\pgfqpoint{3.265487in}{3.489049in}}%
\pgfpathlineto{\pgfqpoint{3.154590in}{3.425023in}}%
\pgfpathlineto{\pgfqpoint{3.154590in}{3.553075in}}%
\pgfpathlineto{\pgfqpoint{3.265487in}{3.617101in}}%
\pgfpathlineto{\pgfqpoint{3.265487in}{3.489049in}}%
\pgfpathclose%
\pgfusepath{fill}%
\end{pgfscope}%
\begin{pgfscope}%
\pgfpathrectangle{\pgfqpoint{0.765000in}{0.660000in}}{\pgfqpoint{4.620000in}{4.620000in}}%
\pgfusepath{clip}%
\pgfsetbuttcap%
\pgfsetroundjoin%
\definecolor{currentfill}{rgb}{1.000000,0.894118,0.788235}%
\pgfsetfillcolor{currentfill}%
\pgfsetlinewidth{0.000000pt}%
\definecolor{currentstroke}{rgb}{1.000000,0.894118,0.788235}%
\pgfsetstrokecolor{currentstroke}%
\pgfsetdash{}{0pt}%
\pgfpathmoveto{\pgfqpoint{3.265487in}{3.489049in}}%
\pgfpathlineto{\pgfqpoint{3.376384in}{3.425023in}}%
\pgfpathlineto{\pgfqpoint{3.376384in}{3.553075in}}%
\pgfpathlineto{\pgfqpoint{3.265487in}{3.617101in}}%
\pgfpathlineto{\pgfqpoint{3.265487in}{3.489049in}}%
\pgfpathclose%
\pgfusepath{fill}%
\end{pgfscope}%
\begin{pgfscope}%
\pgfpathrectangle{\pgfqpoint{0.765000in}{0.660000in}}{\pgfqpoint{4.620000in}{4.620000in}}%
\pgfusepath{clip}%
\pgfsetbuttcap%
\pgfsetroundjoin%
\definecolor{currentfill}{rgb}{1.000000,0.894118,0.788235}%
\pgfsetfillcolor{currentfill}%
\pgfsetlinewidth{0.000000pt}%
\definecolor{currentstroke}{rgb}{1.000000,0.894118,0.788235}%
\pgfsetstrokecolor{currentstroke}%
\pgfsetdash{}{0pt}%
\pgfpathmoveto{\pgfqpoint{3.296975in}{3.470058in}}%
\pgfpathlineto{\pgfqpoint{3.186079in}{3.406032in}}%
\pgfpathlineto{\pgfqpoint{3.296975in}{3.342005in}}%
\pgfpathlineto{\pgfqpoint{3.407872in}{3.406032in}}%
\pgfpathlineto{\pgfqpoint{3.296975in}{3.470058in}}%
\pgfpathclose%
\pgfusepath{fill}%
\end{pgfscope}%
\begin{pgfscope}%
\pgfpathrectangle{\pgfqpoint{0.765000in}{0.660000in}}{\pgfqpoint{4.620000in}{4.620000in}}%
\pgfusepath{clip}%
\pgfsetbuttcap%
\pgfsetroundjoin%
\definecolor{currentfill}{rgb}{1.000000,0.894118,0.788235}%
\pgfsetfillcolor{currentfill}%
\pgfsetlinewidth{0.000000pt}%
\definecolor{currentstroke}{rgb}{1.000000,0.894118,0.788235}%
\pgfsetstrokecolor{currentstroke}%
\pgfsetdash{}{0pt}%
\pgfpathmoveto{\pgfqpoint{3.296975in}{3.470058in}}%
\pgfpathlineto{\pgfqpoint{3.186079in}{3.406032in}}%
\pgfpathlineto{\pgfqpoint{3.186079in}{3.534084in}}%
\pgfpathlineto{\pgfqpoint{3.296975in}{3.598110in}}%
\pgfpathlineto{\pgfqpoint{3.296975in}{3.470058in}}%
\pgfpathclose%
\pgfusepath{fill}%
\end{pgfscope}%
\begin{pgfscope}%
\pgfpathrectangle{\pgfqpoint{0.765000in}{0.660000in}}{\pgfqpoint{4.620000in}{4.620000in}}%
\pgfusepath{clip}%
\pgfsetbuttcap%
\pgfsetroundjoin%
\definecolor{currentfill}{rgb}{1.000000,0.894118,0.788235}%
\pgfsetfillcolor{currentfill}%
\pgfsetlinewidth{0.000000pt}%
\definecolor{currentstroke}{rgb}{1.000000,0.894118,0.788235}%
\pgfsetstrokecolor{currentstroke}%
\pgfsetdash{}{0pt}%
\pgfpathmoveto{\pgfqpoint{3.296975in}{3.470058in}}%
\pgfpathlineto{\pgfqpoint{3.407872in}{3.406032in}}%
\pgfpathlineto{\pgfqpoint{3.407872in}{3.534084in}}%
\pgfpathlineto{\pgfqpoint{3.296975in}{3.598110in}}%
\pgfpathlineto{\pgfqpoint{3.296975in}{3.470058in}}%
\pgfpathclose%
\pgfusepath{fill}%
\end{pgfscope}%
\begin{pgfscope}%
\pgfpathrectangle{\pgfqpoint{0.765000in}{0.660000in}}{\pgfqpoint{4.620000in}{4.620000in}}%
\pgfusepath{clip}%
\pgfsetbuttcap%
\pgfsetroundjoin%
\definecolor{currentfill}{rgb}{1.000000,0.894118,0.788235}%
\pgfsetfillcolor{currentfill}%
\pgfsetlinewidth{0.000000pt}%
\definecolor{currentstroke}{rgb}{1.000000,0.894118,0.788235}%
\pgfsetstrokecolor{currentstroke}%
\pgfsetdash{}{0pt}%
\pgfpathmoveto{\pgfqpoint{3.265487in}{3.617101in}}%
\pgfpathlineto{\pgfqpoint{3.154590in}{3.553075in}}%
\pgfpathlineto{\pgfqpoint{3.265487in}{3.489049in}}%
\pgfpathlineto{\pgfqpoint{3.376384in}{3.553075in}}%
\pgfpathlineto{\pgfqpoint{3.265487in}{3.617101in}}%
\pgfpathclose%
\pgfusepath{fill}%
\end{pgfscope}%
\begin{pgfscope}%
\pgfpathrectangle{\pgfqpoint{0.765000in}{0.660000in}}{\pgfqpoint{4.620000in}{4.620000in}}%
\pgfusepath{clip}%
\pgfsetbuttcap%
\pgfsetroundjoin%
\definecolor{currentfill}{rgb}{1.000000,0.894118,0.788235}%
\pgfsetfillcolor{currentfill}%
\pgfsetlinewidth{0.000000pt}%
\definecolor{currentstroke}{rgb}{1.000000,0.894118,0.788235}%
\pgfsetstrokecolor{currentstroke}%
\pgfsetdash{}{0pt}%
\pgfpathmoveto{\pgfqpoint{3.265487in}{3.360996in}}%
\pgfpathlineto{\pgfqpoint{3.376384in}{3.425023in}}%
\pgfpathlineto{\pgfqpoint{3.376384in}{3.553075in}}%
\pgfpathlineto{\pgfqpoint{3.265487in}{3.489049in}}%
\pgfpathlineto{\pgfqpoint{3.265487in}{3.360996in}}%
\pgfpathclose%
\pgfusepath{fill}%
\end{pgfscope}%
\begin{pgfscope}%
\pgfpathrectangle{\pgfqpoint{0.765000in}{0.660000in}}{\pgfqpoint{4.620000in}{4.620000in}}%
\pgfusepath{clip}%
\pgfsetbuttcap%
\pgfsetroundjoin%
\definecolor{currentfill}{rgb}{1.000000,0.894118,0.788235}%
\pgfsetfillcolor{currentfill}%
\pgfsetlinewidth{0.000000pt}%
\definecolor{currentstroke}{rgb}{1.000000,0.894118,0.788235}%
\pgfsetstrokecolor{currentstroke}%
\pgfsetdash{}{0pt}%
\pgfpathmoveto{\pgfqpoint{3.154590in}{3.425023in}}%
\pgfpathlineto{\pgfqpoint{3.265487in}{3.360996in}}%
\pgfpathlineto{\pgfqpoint{3.265487in}{3.489049in}}%
\pgfpathlineto{\pgfqpoint{3.154590in}{3.553075in}}%
\pgfpathlineto{\pgfqpoint{3.154590in}{3.425023in}}%
\pgfpathclose%
\pgfusepath{fill}%
\end{pgfscope}%
\begin{pgfscope}%
\pgfpathrectangle{\pgfqpoint{0.765000in}{0.660000in}}{\pgfqpoint{4.620000in}{4.620000in}}%
\pgfusepath{clip}%
\pgfsetbuttcap%
\pgfsetroundjoin%
\definecolor{currentfill}{rgb}{1.000000,0.894118,0.788235}%
\pgfsetfillcolor{currentfill}%
\pgfsetlinewidth{0.000000pt}%
\definecolor{currentstroke}{rgb}{1.000000,0.894118,0.788235}%
\pgfsetstrokecolor{currentstroke}%
\pgfsetdash{}{0pt}%
\pgfpathmoveto{\pgfqpoint{3.296975in}{3.598110in}}%
\pgfpathlineto{\pgfqpoint{3.186079in}{3.534084in}}%
\pgfpathlineto{\pgfqpoint{3.296975in}{3.470058in}}%
\pgfpathlineto{\pgfqpoint{3.407872in}{3.534084in}}%
\pgfpathlineto{\pgfqpoint{3.296975in}{3.598110in}}%
\pgfpathclose%
\pgfusepath{fill}%
\end{pgfscope}%
\begin{pgfscope}%
\pgfpathrectangle{\pgfqpoint{0.765000in}{0.660000in}}{\pgfqpoint{4.620000in}{4.620000in}}%
\pgfusepath{clip}%
\pgfsetbuttcap%
\pgfsetroundjoin%
\definecolor{currentfill}{rgb}{1.000000,0.894118,0.788235}%
\pgfsetfillcolor{currentfill}%
\pgfsetlinewidth{0.000000pt}%
\definecolor{currentstroke}{rgb}{1.000000,0.894118,0.788235}%
\pgfsetstrokecolor{currentstroke}%
\pgfsetdash{}{0pt}%
\pgfpathmoveto{\pgfqpoint{3.296975in}{3.342005in}}%
\pgfpathlineto{\pgfqpoint{3.407872in}{3.406032in}}%
\pgfpathlineto{\pgfqpoint{3.407872in}{3.534084in}}%
\pgfpathlineto{\pgfqpoint{3.296975in}{3.470058in}}%
\pgfpathlineto{\pgfqpoint{3.296975in}{3.342005in}}%
\pgfpathclose%
\pgfusepath{fill}%
\end{pgfscope}%
\begin{pgfscope}%
\pgfpathrectangle{\pgfqpoint{0.765000in}{0.660000in}}{\pgfqpoint{4.620000in}{4.620000in}}%
\pgfusepath{clip}%
\pgfsetbuttcap%
\pgfsetroundjoin%
\definecolor{currentfill}{rgb}{1.000000,0.894118,0.788235}%
\pgfsetfillcolor{currentfill}%
\pgfsetlinewidth{0.000000pt}%
\definecolor{currentstroke}{rgb}{1.000000,0.894118,0.788235}%
\pgfsetstrokecolor{currentstroke}%
\pgfsetdash{}{0pt}%
\pgfpathmoveto{\pgfqpoint{3.186079in}{3.406032in}}%
\pgfpathlineto{\pgfqpoint{3.296975in}{3.342005in}}%
\pgfpathlineto{\pgfqpoint{3.296975in}{3.470058in}}%
\pgfpathlineto{\pgfqpoint{3.186079in}{3.534084in}}%
\pgfpathlineto{\pgfqpoint{3.186079in}{3.406032in}}%
\pgfpathclose%
\pgfusepath{fill}%
\end{pgfscope}%
\begin{pgfscope}%
\pgfpathrectangle{\pgfqpoint{0.765000in}{0.660000in}}{\pgfqpoint{4.620000in}{4.620000in}}%
\pgfusepath{clip}%
\pgfsetbuttcap%
\pgfsetroundjoin%
\definecolor{currentfill}{rgb}{1.000000,0.894118,0.788235}%
\pgfsetfillcolor{currentfill}%
\pgfsetlinewidth{0.000000pt}%
\definecolor{currentstroke}{rgb}{1.000000,0.894118,0.788235}%
\pgfsetstrokecolor{currentstroke}%
\pgfsetdash{}{0pt}%
\pgfpathmoveto{\pgfqpoint{3.265487in}{3.489049in}}%
\pgfpathlineto{\pgfqpoint{3.154590in}{3.425023in}}%
\pgfpathlineto{\pgfqpoint{3.186079in}{3.406032in}}%
\pgfpathlineto{\pgfqpoint{3.296975in}{3.470058in}}%
\pgfpathlineto{\pgfqpoint{3.265487in}{3.489049in}}%
\pgfpathclose%
\pgfusepath{fill}%
\end{pgfscope}%
\begin{pgfscope}%
\pgfpathrectangle{\pgfqpoint{0.765000in}{0.660000in}}{\pgfqpoint{4.620000in}{4.620000in}}%
\pgfusepath{clip}%
\pgfsetbuttcap%
\pgfsetroundjoin%
\definecolor{currentfill}{rgb}{1.000000,0.894118,0.788235}%
\pgfsetfillcolor{currentfill}%
\pgfsetlinewidth{0.000000pt}%
\definecolor{currentstroke}{rgb}{1.000000,0.894118,0.788235}%
\pgfsetstrokecolor{currentstroke}%
\pgfsetdash{}{0pt}%
\pgfpathmoveto{\pgfqpoint{3.376384in}{3.425023in}}%
\pgfpathlineto{\pgfqpoint{3.265487in}{3.489049in}}%
\pgfpathlineto{\pgfqpoint{3.296975in}{3.470058in}}%
\pgfpathlineto{\pgfqpoint{3.407872in}{3.406032in}}%
\pgfpathlineto{\pgfqpoint{3.376384in}{3.425023in}}%
\pgfpathclose%
\pgfusepath{fill}%
\end{pgfscope}%
\begin{pgfscope}%
\pgfpathrectangle{\pgfqpoint{0.765000in}{0.660000in}}{\pgfqpoint{4.620000in}{4.620000in}}%
\pgfusepath{clip}%
\pgfsetbuttcap%
\pgfsetroundjoin%
\definecolor{currentfill}{rgb}{1.000000,0.894118,0.788235}%
\pgfsetfillcolor{currentfill}%
\pgfsetlinewidth{0.000000pt}%
\definecolor{currentstroke}{rgb}{1.000000,0.894118,0.788235}%
\pgfsetstrokecolor{currentstroke}%
\pgfsetdash{}{0pt}%
\pgfpathmoveto{\pgfqpoint{3.265487in}{3.489049in}}%
\pgfpathlineto{\pgfqpoint{3.265487in}{3.617101in}}%
\pgfpathlineto{\pgfqpoint{3.296975in}{3.598110in}}%
\pgfpathlineto{\pgfqpoint{3.407872in}{3.406032in}}%
\pgfpathlineto{\pgfqpoint{3.265487in}{3.489049in}}%
\pgfpathclose%
\pgfusepath{fill}%
\end{pgfscope}%
\begin{pgfscope}%
\pgfpathrectangle{\pgfqpoint{0.765000in}{0.660000in}}{\pgfqpoint{4.620000in}{4.620000in}}%
\pgfusepath{clip}%
\pgfsetbuttcap%
\pgfsetroundjoin%
\definecolor{currentfill}{rgb}{1.000000,0.894118,0.788235}%
\pgfsetfillcolor{currentfill}%
\pgfsetlinewidth{0.000000pt}%
\definecolor{currentstroke}{rgb}{1.000000,0.894118,0.788235}%
\pgfsetstrokecolor{currentstroke}%
\pgfsetdash{}{0pt}%
\pgfpathmoveto{\pgfqpoint{3.376384in}{3.425023in}}%
\pgfpathlineto{\pgfqpoint{3.376384in}{3.553075in}}%
\pgfpathlineto{\pgfqpoint{3.407872in}{3.534084in}}%
\pgfpathlineto{\pgfqpoint{3.296975in}{3.470058in}}%
\pgfpathlineto{\pgfqpoint{3.376384in}{3.425023in}}%
\pgfpathclose%
\pgfusepath{fill}%
\end{pgfscope}%
\begin{pgfscope}%
\pgfpathrectangle{\pgfqpoint{0.765000in}{0.660000in}}{\pgfqpoint{4.620000in}{4.620000in}}%
\pgfusepath{clip}%
\pgfsetbuttcap%
\pgfsetroundjoin%
\definecolor{currentfill}{rgb}{1.000000,0.894118,0.788235}%
\pgfsetfillcolor{currentfill}%
\pgfsetlinewidth{0.000000pt}%
\definecolor{currentstroke}{rgb}{1.000000,0.894118,0.788235}%
\pgfsetstrokecolor{currentstroke}%
\pgfsetdash{}{0pt}%
\pgfpathmoveto{\pgfqpoint{3.154590in}{3.425023in}}%
\pgfpathlineto{\pgfqpoint{3.265487in}{3.360996in}}%
\pgfpathlineto{\pgfqpoint{3.296975in}{3.342005in}}%
\pgfpathlineto{\pgfqpoint{3.186079in}{3.406032in}}%
\pgfpathlineto{\pgfqpoint{3.154590in}{3.425023in}}%
\pgfpathclose%
\pgfusepath{fill}%
\end{pgfscope}%
\begin{pgfscope}%
\pgfpathrectangle{\pgfqpoint{0.765000in}{0.660000in}}{\pgfqpoint{4.620000in}{4.620000in}}%
\pgfusepath{clip}%
\pgfsetbuttcap%
\pgfsetroundjoin%
\definecolor{currentfill}{rgb}{1.000000,0.894118,0.788235}%
\pgfsetfillcolor{currentfill}%
\pgfsetlinewidth{0.000000pt}%
\definecolor{currentstroke}{rgb}{1.000000,0.894118,0.788235}%
\pgfsetstrokecolor{currentstroke}%
\pgfsetdash{}{0pt}%
\pgfpathmoveto{\pgfqpoint{3.265487in}{3.360996in}}%
\pgfpathlineto{\pgfqpoint{3.376384in}{3.425023in}}%
\pgfpathlineto{\pgfqpoint{3.407872in}{3.406032in}}%
\pgfpathlineto{\pgfqpoint{3.296975in}{3.342005in}}%
\pgfpathlineto{\pgfqpoint{3.265487in}{3.360996in}}%
\pgfpathclose%
\pgfusepath{fill}%
\end{pgfscope}%
\begin{pgfscope}%
\pgfpathrectangle{\pgfqpoint{0.765000in}{0.660000in}}{\pgfqpoint{4.620000in}{4.620000in}}%
\pgfusepath{clip}%
\pgfsetbuttcap%
\pgfsetroundjoin%
\definecolor{currentfill}{rgb}{1.000000,0.894118,0.788235}%
\pgfsetfillcolor{currentfill}%
\pgfsetlinewidth{0.000000pt}%
\definecolor{currentstroke}{rgb}{1.000000,0.894118,0.788235}%
\pgfsetstrokecolor{currentstroke}%
\pgfsetdash{}{0pt}%
\pgfpathmoveto{\pgfqpoint{3.265487in}{3.617101in}}%
\pgfpathlineto{\pgfqpoint{3.154590in}{3.553075in}}%
\pgfpathlineto{\pgfqpoint{3.186079in}{3.534084in}}%
\pgfpathlineto{\pgfqpoint{3.296975in}{3.598110in}}%
\pgfpathlineto{\pgfqpoint{3.265487in}{3.617101in}}%
\pgfpathclose%
\pgfusepath{fill}%
\end{pgfscope}%
\begin{pgfscope}%
\pgfpathrectangle{\pgfqpoint{0.765000in}{0.660000in}}{\pgfqpoint{4.620000in}{4.620000in}}%
\pgfusepath{clip}%
\pgfsetbuttcap%
\pgfsetroundjoin%
\definecolor{currentfill}{rgb}{1.000000,0.894118,0.788235}%
\pgfsetfillcolor{currentfill}%
\pgfsetlinewidth{0.000000pt}%
\definecolor{currentstroke}{rgb}{1.000000,0.894118,0.788235}%
\pgfsetstrokecolor{currentstroke}%
\pgfsetdash{}{0pt}%
\pgfpathmoveto{\pgfqpoint{3.376384in}{3.553075in}}%
\pgfpathlineto{\pgfqpoint{3.265487in}{3.617101in}}%
\pgfpathlineto{\pgfqpoint{3.296975in}{3.598110in}}%
\pgfpathlineto{\pgfqpoint{3.407872in}{3.534084in}}%
\pgfpathlineto{\pgfqpoint{3.376384in}{3.553075in}}%
\pgfpathclose%
\pgfusepath{fill}%
\end{pgfscope}%
\begin{pgfscope}%
\pgfpathrectangle{\pgfqpoint{0.765000in}{0.660000in}}{\pgfqpoint{4.620000in}{4.620000in}}%
\pgfusepath{clip}%
\pgfsetbuttcap%
\pgfsetroundjoin%
\definecolor{currentfill}{rgb}{1.000000,0.894118,0.788235}%
\pgfsetfillcolor{currentfill}%
\pgfsetlinewidth{0.000000pt}%
\definecolor{currentstroke}{rgb}{1.000000,0.894118,0.788235}%
\pgfsetstrokecolor{currentstroke}%
\pgfsetdash{}{0pt}%
\pgfpathmoveto{\pgfqpoint{3.154590in}{3.425023in}}%
\pgfpathlineto{\pgfqpoint{3.154590in}{3.553075in}}%
\pgfpathlineto{\pgfqpoint{3.186079in}{3.534084in}}%
\pgfpathlineto{\pgfqpoint{3.296975in}{3.342005in}}%
\pgfpathlineto{\pgfqpoint{3.154590in}{3.425023in}}%
\pgfpathclose%
\pgfusepath{fill}%
\end{pgfscope}%
\begin{pgfscope}%
\pgfpathrectangle{\pgfqpoint{0.765000in}{0.660000in}}{\pgfqpoint{4.620000in}{4.620000in}}%
\pgfusepath{clip}%
\pgfsetbuttcap%
\pgfsetroundjoin%
\definecolor{currentfill}{rgb}{1.000000,0.894118,0.788235}%
\pgfsetfillcolor{currentfill}%
\pgfsetlinewidth{0.000000pt}%
\definecolor{currentstroke}{rgb}{1.000000,0.894118,0.788235}%
\pgfsetstrokecolor{currentstroke}%
\pgfsetdash{}{0pt}%
\pgfpathmoveto{\pgfqpoint{3.265487in}{3.360996in}}%
\pgfpathlineto{\pgfqpoint{3.265487in}{3.489049in}}%
\pgfpathlineto{\pgfqpoint{3.296975in}{3.470058in}}%
\pgfpathlineto{\pgfqpoint{3.186079in}{3.406032in}}%
\pgfpathlineto{\pgfqpoint{3.265487in}{3.360996in}}%
\pgfpathclose%
\pgfusepath{fill}%
\end{pgfscope}%
\begin{pgfscope}%
\pgfpathrectangle{\pgfqpoint{0.765000in}{0.660000in}}{\pgfqpoint{4.620000in}{4.620000in}}%
\pgfusepath{clip}%
\pgfsetbuttcap%
\pgfsetroundjoin%
\definecolor{currentfill}{rgb}{1.000000,0.894118,0.788235}%
\pgfsetfillcolor{currentfill}%
\pgfsetlinewidth{0.000000pt}%
\definecolor{currentstroke}{rgb}{1.000000,0.894118,0.788235}%
\pgfsetstrokecolor{currentstroke}%
\pgfsetdash{}{0pt}%
\pgfpathmoveto{\pgfqpoint{3.265487in}{3.489049in}}%
\pgfpathlineto{\pgfqpoint{3.376384in}{3.553075in}}%
\pgfpathlineto{\pgfqpoint{3.407872in}{3.534084in}}%
\pgfpathlineto{\pgfqpoint{3.296975in}{3.470058in}}%
\pgfpathlineto{\pgfqpoint{3.265487in}{3.489049in}}%
\pgfpathclose%
\pgfusepath{fill}%
\end{pgfscope}%
\begin{pgfscope}%
\pgfpathrectangle{\pgfqpoint{0.765000in}{0.660000in}}{\pgfqpoint{4.620000in}{4.620000in}}%
\pgfusepath{clip}%
\pgfsetbuttcap%
\pgfsetroundjoin%
\definecolor{currentfill}{rgb}{1.000000,0.894118,0.788235}%
\pgfsetfillcolor{currentfill}%
\pgfsetlinewidth{0.000000pt}%
\definecolor{currentstroke}{rgb}{1.000000,0.894118,0.788235}%
\pgfsetstrokecolor{currentstroke}%
\pgfsetdash{}{0pt}%
\pgfpathmoveto{\pgfqpoint{3.154590in}{3.553075in}}%
\pgfpathlineto{\pgfqpoint{3.265487in}{3.489049in}}%
\pgfpathlineto{\pgfqpoint{3.296975in}{3.470058in}}%
\pgfpathlineto{\pgfqpoint{3.186079in}{3.534084in}}%
\pgfpathlineto{\pgfqpoint{3.154590in}{3.553075in}}%
\pgfpathclose%
\pgfusepath{fill}%
\end{pgfscope}%
\begin{pgfscope}%
\pgfpathrectangle{\pgfqpoint{0.765000in}{0.660000in}}{\pgfqpoint{4.620000in}{4.620000in}}%
\pgfusepath{clip}%
\pgfsetbuttcap%
\pgfsetroundjoin%
\definecolor{currentfill}{rgb}{1.000000,0.894118,0.788235}%
\pgfsetfillcolor{currentfill}%
\pgfsetlinewidth{0.000000pt}%
\definecolor{currentstroke}{rgb}{1.000000,0.894118,0.788235}%
\pgfsetstrokecolor{currentstroke}%
\pgfsetdash{}{0pt}%
\pgfpathmoveto{\pgfqpoint{3.265487in}{3.489049in}}%
\pgfpathlineto{\pgfqpoint{3.154590in}{3.425023in}}%
\pgfpathlineto{\pgfqpoint{3.265487in}{3.360996in}}%
\pgfpathlineto{\pgfqpoint{3.376384in}{3.425023in}}%
\pgfpathlineto{\pgfqpoint{3.265487in}{3.489049in}}%
\pgfpathclose%
\pgfusepath{fill}%
\end{pgfscope}%
\begin{pgfscope}%
\pgfpathrectangle{\pgfqpoint{0.765000in}{0.660000in}}{\pgfqpoint{4.620000in}{4.620000in}}%
\pgfusepath{clip}%
\pgfsetbuttcap%
\pgfsetroundjoin%
\definecolor{currentfill}{rgb}{1.000000,0.894118,0.788235}%
\pgfsetfillcolor{currentfill}%
\pgfsetlinewidth{0.000000pt}%
\definecolor{currentstroke}{rgb}{1.000000,0.894118,0.788235}%
\pgfsetstrokecolor{currentstroke}%
\pgfsetdash{}{0pt}%
\pgfpathmoveto{\pgfqpoint{3.265487in}{3.489049in}}%
\pgfpathlineto{\pgfqpoint{3.154590in}{3.425023in}}%
\pgfpathlineto{\pgfqpoint{3.154590in}{3.553075in}}%
\pgfpathlineto{\pgfqpoint{3.265487in}{3.617101in}}%
\pgfpathlineto{\pgfqpoint{3.265487in}{3.489049in}}%
\pgfpathclose%
\pgfusepath{fill}%
\end{pgfscope}%
\begin{pgfscope}%
\pgfpathrectangle{\pgfqpoint{0.765000in}{0.660000in}}{\pgfqpoint{4.620000in}{4.620000in}}%
\pgfusepath{clip}%
\pgfsetbuttcap%
\pgfsetroundjoin%
\definecolor{currentfill}{rgb}{1.000000,0.894118,0.788235}%
\pgfsetfillcolor{currentfill}%
\pgfsetlinewidth{0.000000pt}%
\definecolor{currentstroke}{rgb}{1.000000,0.894118,0.788235}%
\pgfsetstrokecolor{currentstroke}%
\pgfsetdash{}{0pt}%
\pgfpathmoveto{\pgfqpoint{3.265487in}{3.489049in}}%
\pgfpathlineto{\pgfqpoint{3.376384in}{3.425023in}}%
\pgfpathlineto{\pgfqpoint{3.376384in}{3.553075in}}%
\pgfpathlineto{\pgfqpoint{3.265487in}{3.617101in}}%
\pgfpathlineto{\pgfqpoint{3.265487in}{3.489049in}}%
\pgfpathclose%
\pgfusepath{fill}%
\end{pgfscope}%
\begin{pgfscope}%
\pgfpathrectangle{\pgfqpoint{0.765000in}{0.660000in}}{\pgfqpoint{4.620000in}{4.620000in}}%
\pgfusepath{clip}%
\pgfsetbuttcap%
\pgfsetroundjoin%
\definecolor{currentfill}{rgb}{1.000000,0.894118,0.788235}%
\pgfsetfillcolor{currentfill}%
\pgfsetlinewidth{0.000000pt}%
\definecolor{currentstroke}{rgb}{1.000000,0.894118,0.788235}%
\pgfsetstrokecolor{currentstroke}%
\pgfsetdash{}{0pt}%
\pgfpathmoveto{\pgfqpoint{3.230111in}{3.500132in}}%
\pgfpathlineto{\pgfqpoint{3.119215in}{3.436106in}}%
\pgfpathlineto{\pgfqpoint{3.230111in}{3.372080in}}%
\pgfpathlineto{\pgfqpoint{3.341008in}{3.436106in}}%
\pgfpathlineto{\pgfqpoint{3.230111in}{3.500132in}}%
\pgfpathclose%
\pgfusepath{fill}%
\end{pgfscope}%
\begin{pgfscope}%
\pgfpathrectangle{\pgfqpoint{0.765000in}{0.660000in}}{\pgfqpoint{4.620000in}{4.620000in}}%
\pgfusepath{clip}%
\pgfsetbuttcap%
\pgfsetroundjoin%
\definecolor{currentfill}{rgb}{1.000000,0.894118,0.788235}%
\pgfsetfillcolor{currentfill}%
\pgfsetlinewidth{0.000000pt}%
\definecolor{currentstroke}{rgb}{1.000000,0.894118,0.788235}%
\pgfsetstrokecolor{currentstroke}%
\pgfsetdash{}{0pt}%
\pgfpathmoveto{\pgfqpoint{3.230111in}{3.500132in}}%
\pgfpathlineto{\pgfqpoint{3.119215in}{3.436106in}}%
\pgfpathlineto{\pgfqpoint{3.119215in}{3.564159in}}%
\pgfpathlineto{\pgfqpoint{3.230111in}{3.628185in}}%
\pgfpathlineto{\pgfqpoint{3.230111in}{3.500132in}}%
\pgfpathclose%
\pgfusepath{fill}%
\end{pgfscope}%
\begin{pgfscope}%
\pgfpathrectangle{\pgfqpoint{0.765000in}{0.660000in}}{\pgfqpoint{4.620000in}{4.620000in}}%
\pgfusepath{clip}%
\pgfsetbuttcap%
\pgfsetroundjoin%
\definecolor{currentfill}{rgb}{1.000000,0.894118,0.788235}%
\pgfsetfillcolor{currentfill}%
\pgfsetlinewidth{0.000000pt}%
\definecolor{currentstroke}{rgb}{1.000000,0.894118,0.788235}%
\pgfsetstrokecolor{currentstroke}%
\pgfsetdash{}{0pt}%
\pgfpathmoveto{\pgfqpoint{3.230111in}{3.500132in}}%
\pgfpathlineto{\pgfqpoint{3.341008in}{3.436106in}}%
\pgfpathlineto{\pgfqpoint{3.341008in}{3.564159in}}%
\pgfpathlineto{\pgfqpoint{3.230111in}{3.628185in}}%
\pgfpathlineto{\pgfqpoint{3.230111in}{3.500132in}}%
\pgfpathclose%
\pgfusepath{fill}%
\end{pgfscope}%
\begin{pgfscope}%
\pgfpathrectangle{\pgfqpoint{0.765000in}{0.660000in}}{\pgfqpoint{4.620000in}{4.620000in}}%
\pgfusepath{clip}%
\pgfsetbuttcap%
\pgfsetroundjoin%
\definecolor{currentfill}{rgb}{1.000000,0.894118,0.788235}%
\pgfsetfillcolor{currentfill}%
\pgfsetlinewidth{0.000000pt}%
\definecolor{currentstroke}{rgb}{1.000000,0.894118,0.788235}%
\pgfsetstrokecolor{currentstroke}%
\pgfsetdash{}{0pt}%
\pgfpathmoveto{\pgfqpoint{3.265487in}{3.617101in}}%
\pgfpathlineto{\pgfqpoint{3.154590in}{3.553075in}}%
\pgfpathlineto{\pgfqpoint{3.265487in}{3.489049in}}%
\pgfpathlineto{\pgfqpoint{3.376384in}{3.553075in}}%
\pgfpathlineto{\pgfqpoint{3.265487in}{3.617101in}}%
\pgfpathclose%
\pgfusepath{fill}%
\end{pgfscope}%
\begin{pgfscope}%
\pgfpathrectangle{\pgfqpoint{0.765000in}{0.660000in}}{\pgfqpoint{4.620000in}{4.620000in}}%
\pgfusepath{clip}%
\pgfsetbuttcap%
\pgfsetroundjoin%
\definecolor{currentfill}{rgb}{1.000000,0.894118,0.788235}%
\pgfsetfillcolor{currentfill}%
\pgfsetlinewidth{0.000000pt}%
\definecolor{currentstroke}{rgb}{1.000000,0.894118,0.788235}%
\pgfsetstrokecolor{currentstroke}%
\pgfsetdash{}{0pt}%
\pgfpathmoveto{\pgfqpoint{3.265487in}{3.360996in}}%
\pgfpathlineto{\pgfqpoint{3.376384in}{3.425023in}}%
\pgfpathlineto{\pgfqpoint{3.376384in}{3.553075in}}%
\pgfpathlineto{\pgfqpoint{3.265487in}{3.489049in}}%
\pgfpathlineto{\pgfqpoint{3.265487in}{3.360996in}}%
\pgfpathclose%
\pgfusepath{fill}%
\end{pgfscope}%
\begin{pgfscope}%
\pgfpathrectangle{\pgfqpoint{0.765000in}{0.660000in}}{\pgfqpoint{4.620000in}{4.620000in}}%
\pgfusepath{clip}%
\pgfsetbuttcap%
\pgfsetroundjoin%
\definecolor{currentfill}{rgb}{1.000000,0.894118,0.788235}%
\pgfsetfillcolor{currentfill}%
\pgfsetlinewidth{0.000000pt}%
\definecolor{currentstroke}{rgb}{1.000000,0.894118,0.788235}%
\pgfsetstrokecolor{currentstroke}%
\pgfsetdash{}{0pt}%
\pgfpathmoveto{\pgfqpoint{3.154590in}{3.425023in}}%
\pgfpathlineto{\pgfqpoint{3.265487in}{3.360996in}}%
\pgfpathlineto{\pgfqpoint{3.265487in}{3.489049in}}%
\pgfpathlineto{\pgfqpoint{3.154590in}{3.553075in}}%
\pgfpathlineto{\pgfqpoint{3.154590in}{3.425023in}}%
\pgfpathclose%
\pgfusepath{fill}%
\end{pgfscope}%
\begin{pgfscope}%
\pgfpathrectangle{\pgfqpoint{0.765000in}{0.660000in}}{\pgfqpoint{4.620000in}{4.620000in}}%
\pgfusepath{clip}%
\pgfsetbuttcap%
\pgfsetroundjoin%
\definecolor{currentfill}{rgb}{1.000000,0.894118,0.788235}%
\pgfsetfillcolor{currentfill}%
\pgfsetlinewidth{0.000000pt}%
\definecolor{currentstroke}{rgb}{1.000000,0.894118,0.788235}%
\pgfsetstrokecolor{currentstroke}%
\pgfsetdash{}{0pt}%
\pgfpathmoveto{\pgfqpoint{3.230111in}{3.628185in}}%
\pgfpathlineto{\pgfqpoint{3.119215in}{3.564159in}}%
\pgfpathlineto{\pgfqpoint{3.230111in}{3.500132in}}%
\pgfpathlineto{\pgfqpoint{3.341008in}{3.564159in}}%
\pgfpathlineto{\pgfqpoint{3.230111in}{3.628185in}}%
\pgfpathclose%
\pgfusepath{fill}%
\end{pgfscope}%
\begin{pgfscope}%
\pgfpathrectangle{\pgfqpoint{0.765000in}{0.660000in}}{\pgfqpoint{4.620000in}{4.620000in}}%
\pgfusepath{clip}%
\pgfsetbuttcap%
\pgfsetroundjoin%
\definecolor{currentfill}{rgb}{1.000000,0.894118,0.788235}%
\pgfsetfillcolor{currentfill}%
\pgfsetlinewidth{0.000000pt}%
\definecolor{currentstroke}{rgb}{1.000000,0.894118,0.788235}%
\pgfsetstrokecolor{currentstroke}%
\pgfsetdash{}{0pt}%
\pgfpathmoveto{\pgfqpoint{3.230111in}{3.372080in}}%
\pgfpathlineto{\pgfqpoint{3.341008in}{3.436106in}}%
\pgfpathlineto{\pgfqpoint{3.341008in}{3.564159in}}%
\pgfpathlineto{\pgfqpoint{3.230111in}{3.500132in}}%
\pgfpathlineto{\pgfqpoint{3.230111in}{3.372080in}}%
\pgfpathclose%
\pgfusepath{fill}%
\end{pgfscope}%
\begin{pgfscope}%
\pgfpathrectangle{\pgfqpoint{0.765000in}{0.660000in}}{\pgfqpoint{4.620000in}{4.620000in}}%
\pgfusepath{clip}%
\pgfsetbuttcap%
\pgfsetroundjoin%
\definecolor{currentfill}{rgb}{1.000000,0.894118,0.788235}%
\pgfsetfillcolor{currentfill}%
\pgfsetlinewidth{0.000000pt}%
\definecolor{currentstroke}{rgb}{1.000000,0.894118,0.788235}%
\pgfsetstrokecolor{currentstroke}%
\pgfsetdash{}{0pt}%
\pgfpathmoveto{\pgfqpoint{3.119215in}{3.436106in}}%
\pgfpathlineto{\pgfqpoint{3.230111in}{3.372080in}}%
\pgfpathlineto{\pgfqpoint{3.230111in}{3.500132in}}%
\pgfpathlineto{\pgfqpoint{3.119215in}{3.564159in}}%
\pgfpathlineto{\pgfqpoint{3.119215in}{3.436106in}}%
\pgfpathclose%
\pgfusepath{fill}%
\end{pgfscope}%
\begin{pgfscope}%
\pgfpathrectangle{\pgfqpoint{0.765000in}{0.660000in}}{\pgfqpoint{4.620000in}{4.620000in}}%
\pgfusepath{clip}%
\pgfsetbuttcap%
\pgfsetroundjoin%
\definecolor{currentfill}{rgb}{1.000000,0.894118,0.788235}%
\pgfsetfillcolor{currentfill}%
\pgfsetlinewidth{0.000000pt}%
\definecolor{currentstroke}{rgb}{1.000000,0.894118,0.788235}%
\pgfsetstrokecolor{currentstroke}%
\pgfsetdash{}{0pt}%
\pgfpathmoveto{\pgfqpoint{3.265487in}{3.489049in}}%
\pgfpathlineto{\pgfqpoint{3.154590in}{3.425023in}}%
\pgfpathlineto{\pgfqpoint{3.119215in}{3.436106in}}%
\pgfpathlineto{\pgfqpoint{3.230111in}{3.500132in}}%
\pgfpathlineto{\pgfqpoint{3.265487in}{3.489049in}}%
\pgfpathclose%
\pgfusepath{fill}%
\end{pgfscope}%
\begin{pgfscope}%
\pgfpathrectangle{\pgfqpoint{0.765000in}{0.660000in}}{\pgfqpoint{4.620000in}{4.620000in}}%
\pgfusepath{clip}%
\pgfsetbuttcap%
\pgfsetroundjoin%
\definecolor{currentfill}{rgb}{1.000000,0.894118,0.788235}%
\pgfsetfillcolor{currentfill}%
\pgfsetlinewidth{0.000000pt}%
\definecolor{currentstroke}{rgb}{1.000000,0.894118,0.788235}%
\pgfsetstrokecolor{currentstroke}%
\pgfsetdash{}{0pt}%
\pgfpathmoveto{\pgfqpoint{3.376384in}{3.425023in}}%
\pgfpathlineto{\pgfqpoint{3.265487in}{3.489049in}}%
\pgfpathlineto{\pgfqpoint{3.230111in}{3.500132in}}%
\pgfpathlineto{\pgfqpoint{3.341008in}{3.436106in}}%
\pgfpathlineto{\pgfqpoint{3.376384in}{3.425023in}}%
\pgfpathclose%
\pgfusepath{fill}%
\end{pgfscope}%
\begin{pgfscope}%
\pgfpathrectangle{\pgfqpoint{0.765000in}{0.660000in}}{\pgfqpoint{4.620000in}{4.620000in}}%
\pgfusepath{clip}%
\pgfsetbuttcap%
\pgfsetroundjoin%
\definecolor{currentfill}{rgb}{1.000000,0.894118,0.788235}%
\pgfsetfillcolor{currentfill}%
\pgfsetlinewidth{0.000000pt}%
\definecolor{currentstroke}{rgb}{1.000000,0.894118,0.788235}%
\pgfsetstrokecolor{currentstroke}%
\pgfsetdash{}{0pt}%
\pgfpathmoveto{\pgfqpoint{3.265487in}{3.489049in}}%
\pgfpathlineto{\pgfqpoint{3.265487in}{3.617101in}}%
\pgfpathlineto{\pgfqpoint{3.230111in}{3.628185in}}%
\pgfpathlineto{\pgfqpoint{3.341008in}{3.436106in}}%
\pgfpathlineto{\pgfqpoint{3.265487in}{3.489049in}}%
\pgfpathclose%
\pgfusepath{fill}%
\end{pgfscope}%
\begin{pgfscope}%
\pgfpathrectangle{\pgfqpoint{0.765000in}{0.660000in}}{\pgfqpoint{4.620000in}{4.620000in}}%
\pgfusepath{clip}%
\pgfsetbuttcap%
\pgfsetroundjoin%
\definecolor{currentfill}{rgb}{1.000000,0.894118,0.788235}%
\pgfsetfillcolor{currentfill}%
\pgfsetlinewidth{0.000000pt}%
\definecolor{currentstroke}{rgb}{1.000000,0.894118,0.788235}%
\pgfsetstrokecolor{currentstroke}%
\pgfsetdash{}{0pt}%
\pgfpathmoveto{\pgfqpoint{3.376384in}{3.425023in}}%
\pgfpathlineto{\pgfqpoint{3.376384in}{3.553075in}}%
\pgfpathlineto{\pgfqpoint{3.341008in}{3.564159in}}%
\pgfpathlineto{\pgfqpoint{3.230111in}{3.500132in}}%
\pgfpathlineto{\pgfqpoint{3.376384in}{3.425023in}}%
\pgfpathclose%
\pgfusepath{fill}%
\end{pgfscope}%
\begin{pgfscope}%
\pgfpathrectangle{\pgfqpoint{0.765000in}{0.660000in}}{\pgfqpoint{4.620000in}{4.620000in}}%
\pgfusepath{clip}%
\pgfsetbuttcap%
\pgfsetroundjoin%
\definecolor{currentfill}{rgb}{1.000000,0.894118,0.788235}%
\pgfsetfillcolor{currentfill}%
\pgfsetlinewidth{0.000000pt}%
\definecolor{currentstroke}{rgb}{1.000000,0.894118,0.788235}%
\pgfsetstrokecolor{currentstroke}%
\pgfsetdash{}{0pt}%
\pgfpathmoveto{\pgfqpoint{3.154590in}{3.425023in}}%
\pgfpathlineto{\pgfqpoint{3.265487in}{3.360996in}}%
\pgfpathlineto{\pgfqpoint{3.230111in}{3.372080in}}%
\pgfpathlineto{\pgfqpoint{3.119215in}{3.436106in}}%
\pgfpathlineto{\pgfqpoint{3.154590in}{3.425023in}}%
\pgfpathclose%
\pgfusepath{fill}%
\end{pgfscope}%
\begin{pgfscope}%
\pgfpathrectangle{\pgfqpoint{0.765000in}{0.660000in}}{\pgfqpoint{4.620000in}{4.620000in}}%
\pgfusepath{clip}%
\pgfsetbuttcap%
\pgfsetroundjoin%
\definecolor{currentfill}{rgb}{1.000000,0.894118,0.788235}%
\pgfsetfillcolor{currentfill}%
\pgfsetlinewidth{0.000000pt}%
\definecolor{currentstroke}{rgb}{1.000000,0.894118,0.788235}%
\pgfsetstrokecolor{currentstroke}%
\pgfsetdash{}{0pt}%
\pgfpathmoveto{\pgfqpoint{3.265487in}{3.360996in}}%
\pgfpathlineto{\pgfqpoint{3.376384in}{3.425023in}}%
\pgfpathlineto{\pgfqpoint{3.341008in}{3.436106in}}%
\pgfpathlineto{\pgfqpoint{3.230111in}{3.372080in}}%
\pgfpathlineto{\pgfqpoint{3.265487in}{3.360996in}}%
\pgfpathclose%
\pgfusepath{fill}%
\end{pgfscope}%
\begin{pgfscope}%
\pgfpathrectangle{\pgfqpoint{0.765000in}{0.660000in}}{\pgfqpoint{4.620000in}{4.620000in}}%
\pgfusepath{clip}%
\pgfsetbuttcap%
\pgfsetroundjoin%
\definecolor{currentfill}{rgb}{1.000000,0.894118,0.788235}%
\pgfsetfillcolor{currentfill}%
\pgfsetlinewidth{0.000000pt}%
\definecolor{currentstroke}{rgb}{1.000000,0.894118,0.788235}%
\pgfsetstrokecolor{currentstroke}%
\pgfsetdash{}{0pt}%
\pgfpathmoveto{\pgfqpoint{3.265487in}{3.617101in}}%
\pgfpathlineto{\pgfqpoint{3.154590in}{3.553075in}}%
\pgfpathlineto{\pgfqpoint{3.119215in}{3.564159in}}%
\pgfpathlineto{\pgfqpoint{3.230111in}{3.628185in}}%
\pgfpathlineto{\pgfqpoint{3.265487in}{3.617101in}}%
\pgfpathclose%
\pgfusepath{fill}%
\end{pgfscope}%
\begin{pgfscope}%
\pgfpathrectangle{\pgfqpoint{0.765000in}{0.660000in}}{\pgfqpoint{4.620000in}{4.620000in}}%
\pgfusepath{clip}%
\pgfsetbuttcap%
\pgfsetroundjoin%
\definecolor{currentfill}{rgb}{1.000000,0.894118,0.788235}%
\pgfsetfillcolor{currentfill}%
\pgfsetlinewidth{0.000000pt}%
\definecolor{currentstroke}{rgb}{1.000000,0.894118,0.788235}%
\pgfsetstrokecolor{currentstroke}%
\pgfsetdash{}{0pt}%
\pgfpathmoveto{\pgfqpoint{3.376384in}{3.553075in}}%
\pgfpathlineto{\pgfqpoint{3.265487in}{3.617101in}}%
\pgfpathlineto{\pgfqpoint{3.230111in}{3.628185in}}%
\pgfpathlineto{\pgfqpoint{3.341008in}{3.564159in}}%
\pgfpathlineto{\pgfqpoint{3.376384in}{3.553075in}}%
\pgfpathclose%
\pgfusepath{fill}%
\end{pgfscope}%
\begin{pgfscope}%
\pgfpathrectangle{\pgfqpoint{0.765000in}{0.660000in}}{\pgfqpoint{4.620000in}{4.620000in}}%
\pgfusepath{clip}%
\pgfsetbuttcap%
\pgfsetroundjoin%
\definecolor{currentfill}{rgb}{1.000000,0.894118,0.788235}%
\pgfsetfillcolor{currentfill}%
\pgfsetlinewidth{0.000000pt}%
\definecolor{currentstroke}{rgb}{1.000000,0.894118,0.788235}%
\pgfsetstrokecolor{currentstroke}%
\pgfsetdash{}{0pt}%
\pgfpathmoveto{\pgfqpoint{3.154590in}{3.425023in}}%
\pgfpathlineto{\pgfqpoint{3.154590in}{3.553075in}}%
\pgfpathlineto{\pgfqpoint{3.119215in}{3.564159in}}%
\pgfpathlineto{\pgfqpoint{3.230111in}{3.372080in}}%
\pgfpathlineto{\pgfqpoint{3.154590in}{3.425023in}}%
\pgfpathclose%
\pgfusepath{fill}%
\end{pgfscope}%
\begin{pgfscope}%
\pgfpathrectangle{\pgfqpoint{0.765000in}{0.660000in}}{\pgfqpoint{4.620000in}{4.620000in}}%
\pgfusepath{clip}%
\pgfsetbuttcap%
\pgfsetroundjoin%
\definecolor{currentfill}{rgb}{1.000000,0.894118,0.788235}%
\pgfsetfillcolor{currentfill}%
\pgfsetlinewidth{0.000000pt}%
\definecolor{currentstroke}{rgb}{1.000000,0.894118,0.788235}%
\pgfsetstrokecolor{currentstroke}%
\pgfsetdash{}{0pt}%
\pgfpathmoveto{\pgfqpoint{3.265487in}{3.360996in}}%
\pgfpathlineto{\pgfqpoint{3.265487in}{3.489049in}}%
\pgfpathlineto{\pgfqpoint{3.230111in}{3.500132in}}%
\pgfpathlineto{\pgfqpoint{3.119215in}{3.436106in}}%
\pgfpathlineto{\pgfqpoint{3.265487in}{3.360996in}}%
\pgfpathclose%
\pgfusepath{fill}%
\end{pgfscope}%
\begin{pgfscope}%
\pgfpathrectangle{\pgfqpoint{0.765000in}{0.660000in}}{\pgfqpoint{4.620000in}{4.620000in}}%
\pgfusepath{clip}%
\pgfsetbuttcap%
\pgfsetroundjoin%
\definecolor{currentfill}{rgb}{1.000000,0.894118,0.788235}%
\pgfsetfillcolor{currentfill}%
\pgfsetlinewidth{0.000000pt}%
\definecolor{currentstroke}{rgb}{1.000000,0.894118,0.788235}%
\pgfsetstrokecolor{currentstroke}%
\pgfsetdash{}{0pt}%
\pgfpathmoveto{\pgfqpoint{3.265487in}{3.489049in}}%
\pgfpathlineto{\pgfqpoint{3.376384in}{3.553075in}}%
\pgfpathlineto{\pgfqpoint{3.341008in}{3.564159in}}%
\pgfpathlineto{\pgfqpoint{3.230111in}{3.500132in}}%
\pgfpathlineto{\pgfqpoint{3.265487in}{3.489049in}}%
\pgfpathclose%
\pgfusepath{fill}%
\end{pgfscope}%
\begin{pgfscope}%
\pgfpathrectangle{\pgfqpoint{0.765000in}{0.660000in}}{\pgfqpoint{4.620000in}{4.620000in}}%
\pgfusepath{clip}%
\pgfsetbuttcap%
\pgfsetroundjoin%
\definecolor{currentfill}{rgb}{1.000000,0.894118,0.788235}%
\pgfsetfillcolor{currentfill}%
\pgfsetlinewidth{0.000000pt}%
\definecolor{currentstroke}{rgb}{1.000000,0.894118,0.788235}%
\pgfsetstrokecolor{currentstroke}%
\pgfsetdash{}{0pt}%
\pgfpathmoveto{\pgfqpoint{3.154590in}{3.553075in}}%
\pgfpathlineto{\pgfqpoint{3.265487in}{3.489049in}}%
\pgfpathlineto{\pgfqpoint{3.230111in}{3.500132in}}%
\pgfpathlineto{\pgfqpoint{3.119215in}{3.564159in}}%
\pgfpathlineto{\pgfqpoint{3.154590in}{3.553075in}}%
\pgfpathclose%
\pgfusepath{fill}%
\end{pgfscope}%
\begin{pgfscope}%
\pgfpathrectangle{\pgfqpoint{0.765000in}{0.660000in}}{\pgfqpoint{4.620000in}{4.620000in}}%
\pgfusepath{clip}%
\pgfsetbuttcap%
\pgfsetroundjoin%
\definecolor{currentfill}{rgb}{1.000000,0.894118,0.788235}%
\pgfsetfillcolor{currentfill}%
\pgfsetlinewidth{0.000000pt}%
\definecolor{currentstroke}{rgb}{1.000000,0.894118,0.788235}%
\pgfsetstrokecolor{currentstroke}%
\pgfsetdash{}{0pt}%
\pgfpathmoveto{\pgfqpoint{2.929883in}{2.530541in}}%
\pgfpathlineto{\pgfqpoint{2.818986in}{2.466515in}}%
\pgfpathlineto{\pgfqpoint{2.929883in}{2.402489in}}%
\pgfpathlineto{\pgfqpoint{3.040780in}{2.466515in}}%
\pgfpathlineto{\pgfqpoint{2.929883in}{2.530541in}}%
\pgfpathclose%
\pgfusepath{fill}%
\end{pgfscope}%
\begin{pgfscope}%
\pgfpathrectangle{\pgfqpoint{0.765000in}{0.660000in}}{\pgfqpoint{4.620000in}{4.620000in}}%
\pgfusepath{clip}%
\pgfsetbuttcap%
\pgfsetroundjoin%
\definecolor{currentfill}{rgb}{1.000000,0.894118,0.788235}%
\pgfsetfillcolor{currentfill}%
\pgfsetlinewidth{0.000000pt}%
\definecolor{currentstroke}{rgb}{1.000000,0.894118,0.788235}%
\pgfsetstrokecolor{currentstroke}%
\pgfsetdash{}{0pt}%
\pgfpathmoveto{\pgfqpoint{2.929883in}{2.530541in}}%
\pgfpathlineto{\pgfqpoint{2.818986in}{2.466515in}}%
\pgfpathlineto{\pgfqpoint{2.818986in}{2.594568in}}%
\pgfpathlineto{\pgfqpoint{2.929883in}{2.658594in}}%
\pgfpathlineto{\pgfqpoint{2.929883in}{2.530541in}}%
\pgfpathclose%
\pgfusepath{fill}%
\end{pgfscope}%
\begin{pgfscope}%
\pgfpathrectangle{\pgfqpoint{0.765000in}{0.660000in}}{\pgfqpoint{4.620000in}{4.620000in}}%
\pgfusepath{clip}%
\pgfsetbuttcap%
\pgfsetroundjoin%
\definecolor{currentfill}{rgb}{1.000000,0.894118,0.788235}%
\pgfsetfillcolor{currentfill}%
\pgfsetlinewidth{0.000000pt}%
\definecolor{currentstroke}{rgb}{1.000000,0.894118,0.788235}%
\pgfsetstrokecolor{currentstroke}%
\pgfsetdash{}{0pt}%
\pgfpathmoveto{\pgfqpoint{2.929883in}{2.530541in}}%
\pgfpathlineto{\pgfqpoint{3.040780in}{2.466515in}}%
\pgfpathlineto{\pgfqpoint{3.040780in}{2.594568in}}%
\pgfpathlineto{\pgfqpoint{2.929883in}{2.658594in}}%
\pgfpathlineto{\pgfqpoint{2.929883in}{2.530541in}}%
\pgfpathclose%
\pgfusepath{fill}%
\end{pgfscope}%
\begin{pgfscope}%
\pgfpathrectangle{\pgfqpoint{0.765000in}{0.660000in}}{\pgfqpoint{4.620000in}{4.620000in}}%
\pgfusepath{clip}%
\pgfsetbuttcap%
\pgfsetroundjoin%
\definecolor{currentfill}{rgb}{1.000000,0.894118,0.788235}%
\pgfsetfillcolor{currentfill}%
\pgfsetlinewidth{0.000000pt}%
\definecolor{currentstroke}{rgb}{1.000000,0.894118,0.788235}%
\pgfsetstrokecolor{currentstroke}%
\pgfsetdash{}{0pt}%
\pgfpathmoveto{\pgfqpoint{2.952980in}{2.520265in}}%
\pgfpathlineto{\pgfqpoint{2.842083in}{2.456239in}}%
\pgfpathlineto{\pgfqpoint{2.952980in}{2.392213in}}%
\pgfpathlineto{\pgfqpoint{3.063876in}{2.456239in}}%
\pgfpathlineto{\pgfqpoint{2.952980in}{2.520265in}}%
\pgfpathclose%
\pgfusepath{fill}%
\end{pgfscope}%
\begin{pgfscope}%
\pgfpathrectangle{\pgfqpoint{0.765000in}{0.660000in}}{\pgfqpoint{4.620000in}{4.620000in}}%
\pgfusepath{clip}%
\pgfsetbuttcap%
\pgfsetroundjoin%
\definecolor{currentfill}{rgb}{1.000000,0.894118,0.788235}%
\pgfsetfillcolor{currentfill}%
\pgfsetlinewidth{0.000000pt}%
\definecolor{currentstroke}{rgb}{1.000000,0.894118,0.788235}%
\pgfsetstrokecolor{currentstroke}%
\pgfsetdash{}{0pt}%
\pgfpathmoveto{\pgfqpoint{2.952980in}{2.520265in}}%
\pgfpathlineto{\pgfqpoint{2.842083in}{2.456239in}}%
\pgfpathlineto{\pgfqpoint{2.842083in}{2.584292in}}%
\pgfpathlineto{\pgfqpoint{2.952980in}{2.648318in}}%
\pgfpathlineto{\pgfqpoint{2.952980in}{2.520265in}}%
\pgfpathclose%
\pgfusepath{fill}%
\end{pgfscope}%
\begin{pgfscope}%
\pgfpathrectangle{\pgfqpoint{0.765000in}{0.660000in}}{\pgfqpoint{4.620000in}{4.620000in}}%
\pgfusepath{clip}%
\pgfsetbuttcap%
\pgfsetroundjoin%
\definecolor{currentfill}{rgb}{1.000000,0.894118,0.788235}%
\pgfsetfillcolor{currentfill}%
\pgfsetlinewidth{0.000000pt}%
\definecolor{currentstroke}{rgb}{1.000000,0.894118,0.788235}%
\pgfsetstrokecolor{currentstroke}%
\pgfsetdash{}{0pt}%
\pgfpathmoveto{\pgfqpoint{2.952980in}{2.520265in}}%
\pgfpathlineto{\pgfqpoint{3.063876in}{2.456239in}}%
\pgfpathlineto{\pgfqpoint{3.063876in}{2.584292in}}%
\pgfpathlineto{\pgfqpoint{2.952980in}{2.648318in}}%
\pgfpathlineto{\pgfqpoint{2.952980in}{2.520265in}}%
\pgfpathclose%
\pgfusepath{fill}%
\end{pgfscope}%
\begin{pgfscope}%
\pgfpathrectangle{\pgfqpoint{0.765000in}{0.660000in}}{\pgfqpoint{4.620000in}{4.620000in}}%
\pgfusepath{clip}%
\pgfsetbuttcap%
\pgfsetroundjoin%
\definecolor{currentfill}{rgb}{1.000000,0.894118,0.788235}%
\pgfsetfillcolor{currentfill}%
\pgfsetlinewidth{0.000000pt}%
\definecolor{currentstroke}{rgb}{1.000000,0.894118,0.788235}%
\pgfsetstrokecolor{currentstroke}%
\pgfsetdash{}{0pt}%
\pgfpathmoveto{\pgfqpoint{2.929883in}{2.658594in}}%
\pgfpathlineto{\pgfqpoint{2.818986in}{2.594568in}}%
\pgfpathlineto{\pgfqpoint{2.929883in}{2.530541in}}%
\pgfpathlineto{\pgfqpoint{3.040780in}{2.594568in}}%
\pgfpathlineto{\pgfqpoint{2.929883in}{2.658594in}}%
\pgfpathclose%
\pgfusepath{fill}%
\end{pgfscope}%
\begin{pgfscope}%
\pgfpathrectangle{\pgfqpoint{0.765000in}{0.660000in}}{\pgfqpoint{4.620000in}{4.620000in}}%
\pgfusepath{clip}%
\pgfsetbuttcap%
\pgfsetroundjoin%
\definecolor{currentfill}{rgb}{1.000000,0.894118,0.788235}%
\pgfsetfillcolor{currentfill}%
\pgfsetlinewidth{0.000000pt}%
\definecolor{currentstroke}{rgb}{1.000000,0.894118,0.788235}%
\pgfsetstrokecolor{currentstroke}%
\pgfsetdash{}{0pt}%
\pgfpathmoveto{\pgfqpoint{2.929883in}{2.402489in}}%
\pgfpathlineto{\pgfqpoint{3.040780in}{2.466515in}}%
\pgfpathlineto{\pgfqpoint{3.040780in}{2.594568in}}%
\pgfpathlineto{\pgfqpoint{2.929883in}{2.530541in}}%
\pgfpathlineto{\pgfqpoint{2.929883in}{2.402489in}}%
\pgfpathclose%
\pgfusepath{fill}%
\end{pgfscope}%
\begin{pgfscope}%
\pgfpathrectangle{\pgfqpoint{0.765000in}{0.660000in}}{\pgfqpoint{4.620000in}{4.620000in}}%
\pgfusepath{clip}%
\pgfsetbuttcap%
\pgfsetroundjoin%
\definecolor{currentfill}{rgb}{1.000000,0.894118,0.788235}%
\pgfsetfillcolor{currentfill}%
\pgfsetlinewidth{0.000000pt}%
\definecolor{currentstroke}{rgb}{1.000000,0.894118,0.788235}%
\pgfsetstrokecolor{currentstroke}%
\pgfsetdash{}{0pt}%
\pgfpathmoveto{\pgfqpoint{2.818986in}{2.466515in}}%
\pgfpathlineto{\pgfqpoint{2.929883in}{2.402489in}}%
\pgfpathlineto{\pgfqpoint{2.929883in}{2.530541in}}%
\pgfpathlineto{\pgfqpoint{2.818986in}{2.594568in}}%
\pgfpathlineto{\pgfqpoint{2.818986in}{2.466515in}}%
\pgfpathclose%
\pgfusepath{fill}%
\end{pgfscope}%
\begin{pgfscope}%
\pgfpathrectangle{\pgfqpoint{0.765000in}{0.660000in}}{\pgfqpoint{4.620000in}{4.620000in}}%
\pgfusepath{clip}%
\pgfsetbuttcap%
\pgfsetroundjoin%
\definecolor{currentfill}{rgb}{1.000000,0.894118,0.788235}%
\pgfsetfillcolor{currentfill}%
\pgfsetlinewidth{0.000000pt}%
\definecolor{currentstroke}{rgb}{1.000000,0.894118,0.788235}%
\pgfsetstrokecolor{currentstroke}%
\pgfsetdash{}{0pt}%
\pgfpathmoveto{\pgfqpoint{2.952980in}{2.648318in}}%
\pgfpathlineto{\pgfqpoint{2.842083in}{2.584292in}}%
\pgfpathlineto{\pgfqpoint{2.952980in}{2.520265in}}%
\pgfpathlineto{\pgfqpoint{3.063876in}{2.584292in}}%
\pgfpathlineto{\pgfqpoint{2.952980in}{2.648318in}}%
\pgfpathclose%
\pgfusepath{fill}%
\end{pgfscope}%
\begin{pgfscope}%
\pgfpathrectangle{\pgfqpoint{0.765000in}{0.660000in}}{\pgfqpoint{4.620000in}{4.620000in}}%
\pgfusepath{clip}%
\pgfsetbuttcap%
\pgfsetroundjoin%
\definecolor{currentfill}{rgb}{1.000000,0.894118,0.788235}%
\pgfsetfillcolor{currentfill}%
\pgfsetlinewidth{0.000000pt}%
\definecolor{currentstroke}{rgb}{1.000000,0.894118,0.788235}%
\pgfsetstrokecolor{currentstroke}%
\pgfsetdash{}{0pt}%
\pgfpathmoveto{\pgfqpoint{2.952980in}{2.392213in}}%
\pgfpathlineto{\pgfqpoint{3.063876in}{2.456239in}}%
\pgfpathlineto{\pgfqpoint{3.063876in}{2.584292in}}%
\pgfpathlineto{\pgfqpoint{2.952980in}{2.520265in}}%
\pgfpathlineto{\pgfqpoint{2.952980in}{2.392213in}}%
\pgfpathclose%
\pgfusepath{fill}%
\end{pgfscope}%
\begin{pgfscope}%
\pgfpathrectangle{\pgfqpoint{0.765000in}{0.660000in}}{\pgfqpoint{4.620000in}{4.620000in}}%
\pgfusepath{clip}%
\pgfsetbuttcap%
\pgfsetroundjoin%
\definecolor{currentfill}{rgb}{1.000000,0.894118,0.788235}%
\pgfsetfillcolor{currentfill}%
\pgfsetlinewidth{0.000000pt}%
\definecolor{currentstroke}{rgb}{1.000000,0.894118,0.788235}%
\pgfsetstrokecolor{currentstroke}%
\pgfsetdash{}{0pt}%
\pgfpathmoveto{\pgfqpoint{2.842083in}{2.456239in}}%
\pgfpathlineto{\pgfqpoint{2.952980in}{2.392213in}}%
\pgfpathlineto{\pgfqpoint{2.952980in}{2.520265in}}%
\pgfpathlineto{\pgfqpoint{2.842083in}{2.584292in}}%
\pgfpathlineto{\pgfqpoint{2.842083in}{2.456239in}}%
\pgfpathclose%
\pgfusepath{fill}%
\end{pgfscope}%
\begin{pgfscope}%
\pgfpathrectangle{\pgfqpoint{0.765000in}{0.660000in}}{\pgfqpoint{4.620000in}{4.620000in}}%
\pgfusepath{clip}%
\pgfsetbuttcap%
\pgfsetroundjoin%
\definecolor{currentfill}{rgb}{1.000000,0.894118,0.788235}%
\pgfsetfillcolor{currentfill}%
\pgfsetlinewidth{0.000000pt}%
\definecolor{currentstroke}{rgb}{1.000000,0.894118,0.788235}%
\pgfsetstrokecolor{currentstroke}%
\pgfsetdash{}{0pt}%
\pgfpathmoveto{\pgfqpoint{2.929883in}{2.530541in}}%
\pgfpathlineto{\pgfqpoint{2.818986in}{2.466515in}}%
\pgfpathlineto{\pgfqpoint{2.842083in}{2.456239in}}%
\pgfpathlineto{\pgfqpoint{2.952980in}{2.520265in}}%
\pgfpathlineto{\pgfqpoint{2.929883in}{2.530541in}}%
\pgfpathclose%
\pgfusepath{fill}%
\end{pgfscope}%
\begin{pgfscope}%
\pgfpathrectangle{\pgfqpoint{0.765000in}{0.660000in}}{\pgfqpoint{4.620000in}{4.620000in}}%
\pgfusepath{clip}%
\pgfsetbuttcap%
\pgfsetroundjoin%
\definecolor{currentfill}{rgb}{1.000000,0.894118,0.788235}%
\pgfsetfillcolor{currentfill}%
\pgfsetlinewidth{0.000000pt}%
\definecolor{currentstroke}{rgb}{1.000000,0.894118,0.788235}%
\pgfsetstrokecolor{currentstroke}%
\pgfsetdash{}{0pt}%
\pgfpathmoveto{\pgfqpoint{3.040780in}{2.466515in}}%
\pgfpathlineto{\pgfqpoint{2.929883in}{2.530541in}}%
\pgfpathlineto{\pgfqpoint{2.952980in}{2.520265in}}%
\pgfpathlineto{\pgfqpoint{3.063876in}{2.456239in}}%
\pgfpathlineto{\pgfqpoint{3.040780in}{2.466515in}}%
\pgfpathclose%
\pgfusepath{fill}%
\end{pgfscope}%
\begin{pgfscope}%
\pgfpathrectangle{\pgfqpoint{0.765000in}{0.660000in}}{\pgfqpoint{4.620000in}{4.620000in}}%
\pgfusepath{clip}%
\pgfsetbuttcap%
\pgfsetroundjoin%
\definecolor{currentfill}{rgb}{1.000000,0.894118,0.788235}%
\pgfsetfillcolor{currentfill}%
\pgfsetlinewidth{0.000000pt}%
\definecolor{currentstroke}{rgb}{1.000000,0.894118,0.788235}%
\pgfsetstrokecolor{currentstroke}%
\pgfsetdash{}{0pt}%
\pgfpathmoveto{\pgfqpoint{2.929883in}{2.530541in}}%
\pgfpathlineto{\pgfqpoint{2.929883in}{2.658594in}}%
\pgfpathlineto{\pgfqpoint{2.952980in}{2.648318in}}%
\pgfpathlineto{\pgfqpoint{3.063876in}{2.456239in}}%
\pgfpathlineto{\pgfqpoint{2.929883in}{2.530541in}}%
\pgfpathclose%
\pgfusepath{fill}%
\end{pgfscope}%
\begin{pgfscope}%
\pgfpathrectangle{\pgfqpoint{0.765000in}{0.660000in}}{\pgfqpoint{4.620000in}{4.620000in}}%
\pgfusepath{clip}%
\pgfsetbuttcap%
\pgfsetroundjoin%
\definecolor{currentfill}{rgb}{1.000000,0.894118,0.788235}%
\pgfsetfillcolor{currentfill}%
\pgfsetlinewidth{0.000000pt}%
\definecolor{currentstroke}{rgb}{1.000000,0.894118,0.788235}%
\pgfsetstrokecolor{currentstroke}%
\pgfsetdash{}{0pt}%
\pgfpathmoveto{\pgfqpoint{3.040780in}{2.466515in}}%
\pgfpathlineto{\pgfqpoint{3.040780in}{2.594568in}}%
\pgfpathlineto{\pgfqpoint{3.063876in}{2.584292in}}%
\pgfpathlineto{\pgfqpoint{2.952980in}{2.520265in}}%
\pgfpathlineto{\pgfqpoint{3.040780in}{2.466515in}}%
\pgfpathclose%
\pgfusepath{fill}%
\end{pgfscope}%
\begin{pgfscope}%
\pgfpathrectangle{\pgfqpoint{0.765000in}{0.660000in}}{\pgfqpoint{4.620000in}{4.620000in}}%
\pgfusepath{clip}%
\pgfsetbuttcap%
\pgfsetroundjoin%
\definecolor{currentfill}{rgb}{1.000000,0.894118,0.788235}%
\pgfsetfillcolor{currentfill}%
\pgfsetlinewidth{0.000000pt}%
\definecolor{currentstroke}{rgb}{1.000000,0.894118,0.788235}%
\pgfsetstrokecolor{currentstroke}%
\pgfsetdash{}{0pt}%
\pgfpathmoveto{\pgfqpoint{2.818986in}{2.466515in}}%
\pgfpathlineto{\pgfqpoint{2.929883in}{2.402489in}}%
\pgfpathlineto{\pgfqpoint{2.952980in}{2.392213in}}%
\pgfpathlineto{\pgfqpoint{2.842083in}{2.456239in}}%
\pgfpathlineto{\pgfqpoint{2.818986in}{2.466515in}}%
\pgfpathclose%
\pgfusepath{fill}%
\end{pgfscope}%
\begin{pgfscope}%
\pgfpathrectangle{\pgfqpoint{0.765000in}{0.660000in}}{\pgfqpoint{4.620000in}{4.620000in}}%
\pgfusepath{clip}%
\pgfsetbuttcap%
\pgfsetroundjoin%
\definecolor{currentfill}{rgb}{1.000000,0.894118,0.788235}%
\pgfsetfillcolor{currentfill}%
\pgfsetlinewidth{0.000000pt}%
\definecolor{currentstroke}{rgb}{1.000000,0.894118,0.788235}%
\pgfsetstrokecolor{currentstroke}%
\pgfsetdash{}{0pt}%
\pgfpathmoveto{\pgfqpoint{2.929883in}{2.402489in}}%
\pgfpathlineto{\pgfqpoint{3.040780in}{2.466515in}}%
\pgfpathlineto{\pgfqpoint{3.063876in}{2.456239in}}%
\pgfpathlineto{\pgfqpoint{2.952980in}{2.392213in}}%
\pgfpathlineto{\pgfqpoint{2.929883in}{2.402489in}}%
\pgfpathclose%
\pgfusepath{fill}%
\end{pgfscope}%
\begin{pgfscope}%
\pgfpathrectangle{\pgfqpoint{0.765000in}{0.660000in}}{\pgfqpoint{4.620000in}{4.620000in}}%
\pgfusepath{clip}%
\pgfsetbuttcap%
\pgfsetroundjoin%
\definecolor{currentfill}{rgb}{1.000000,0.894118,0.788235}%
\pgfsetfillcolor{currentfill}%
\pgfsetlinewidth{0.000000pt}%
\definecolor{currentstroke}{rgb}{1.000000,0.894118,0.788235}%
\pgfsetstrokecolor{currentstroke}%
\pgfsetdash{}{0pt}%
\pgfpathmoveto{\pgfqpoint{2.929883in}{2.658594in}}%
\pgfpathlineto{\pgfqpoint{2.818986in}{2.594568in}}%
\pgfpathlineto{\pgfqpoint{2.842083in}{2.584292in}}%
\pgfpathlineto{\pgfqpoint{2.952980in}{2.648318in}}%
\pgfpathlineto{\pgfqpoint{2.929883in}{2.658594in}}%
\pgfpathclose%
\pgfusepath{fill}%
\end{pgfscope}%
\begin{pgfscope}%
\pgfpathrectangle{\pgfqpoint{0.765000in}{0.660000in}}{\pgfqpoint{4.620000in}{4.620000in}}%
\pgfusepath{clip}%
\pgfsetbuttcap%
\pgfsetroundjoin%
\definecolor{currentfill}{rgb}{1.000000,0.894118,0.788235}%
\pgfsetfillcolor{currentfill}%
\pgfsetlinewidth{0.000000pt}%
\definecolor{currentstroke}{rgb}{1.000000,0.894118,0.788235}%
\pgfsetstrokecolor{currentstroke}%
\pgfsetdash{}{0pt}%
\pgfpathmoveto{\pgfqpoint{3.040780in}{2.594568in}}%
\pgfpathlineto{\pgfqpoint{2.929883in}{2.658594in}}%
\pgfpathlineto{\pgfqpoint{2.952980in}{2.648318in}}%
\pgfpathlineto{\pgfqpoint{3.063876in}{2.584292in}}%
\pgfpathlineto{\pgfqpoint{3.040780in}{2.594568in}}%
\pgfpathclose%
\pgfusepath{fill}%
\end{pgfscope}%
\begin{pgfscope}%
\pgfpathrectangle{\pgfqpoint{0.765000in}{0.660000in}}{\pgfqpoint{4.620000in}{4.620000in}}%
\pgfusepath{clip}%
\pgfsetbuttcap%
\pgfsetroundjoin%
\definecolor{currentfill}{rgb}{1.000000,0.894118,0.788235}%
\pgfsetfillcolor{currentfill}%
\pgfsetlinewidth{0.000000pt}%
\definecolor{currentstroke}{rgb}{1.000000,0.894118,0.788235}%
\pgfsetstrokecolor{currentstroke}%
\pgfsetdash{}{0pt}%
\pgfpathmoveto{\pgfqpoint{2.818986in}{2.466515in}}%
\pgfpathlineto{\pgfqpoint{2.818986in}{2.594568in}}%
\pgfpathlineto{\pgfqpoint{2.842083in}{2.584292in}}%
\pgfpathlineto{\pgfqpoint{2.952980in}{2.392213in}}%
\pgfpathlineto{\pgfqpoint{2.818986in}{2.466515in}}%
\pgfpathclose%
\pgfusepath{fill}%
\end{pgfscope}%
\begin{pgfscope}%
\pgfpathrectangle{\pgfqpoint{0.765000in}{0.660000in}}{\pgfqpoint{4.620000in}{4.620000in}}%
\pgfusepath{clip}%
\pgfsetbuttcap%
\pgfsetroundjoin%
\definecolor{currentfill}{rgb}{1.000000,0.894118,0.788235}%
\pgfsetfillcolor{currentfill}%
\pgfsetlinewidth{0.000000pt}%
\definecolor{currentstroke}{rgb}{1.000000,0.894118,0.788235}%
\pgfsetstrokecolor{currentstroke}%
\pgfsetdash{}{0pt}%
\pgfpathmoveto{\pgfqpoint{2.929883in}{2.402489in}}%
\pgfpathlineto{\pgfqpoint{2.929883in}{2.530541in}}%
\pgfpathlineto{\pgfqpoint{2.952980in}{2.520265in}}%
\pgfpathlineto{\pgfqpoint{2.842083in}{2.456239in}}%
\pgfpathlineto{\pgfqpoint{2.929883in}{2.402489in}}%
\pgfpathclose%
\pgfusepath{fill}%
\end{pgfscope}%
\begin{pgfscope}%
\pgfpathrectangle{\pgfqpoint{0.765000in}{0.660000in}}{\pgfqpoint{4.620000in}{4.620000in}}%
\pgfusepath{clip}%
\pgfsetbuttcap%
\pgfsetroundjoin%
\definecolor{currentfill}{rgb}{1.000000,0.894118,0.788235}%
\pgfsetfillcolor{currentfill}%
\pgfsetlinewidth{0.000000pt}%
\definecolor{currentstroke}{rgb}{1.000000,0.894118,0.788235}%
\pgfsetstrokecolor{currentstroke}%
\pgfsetdash{}{0pt}%
\pgfpathmoveto{\pgfqpoint{2.929883in}{2.530541in}}%
\pgfpathlineto{\pgfqpoint{3.040780in}{2.594568in}}%
\pgfpathlineto{\pgfqpoint{3.063876in}{2.584292in}}%
\pgfpathlineto{\pgfqpoint{2.952980in}{2.520265in}}%
\pgfpathlineto{\pgfqpoint{2.929883in}{2.530541in}}%
\pgfpathclose%
\pgfusepath{fill}%
\end{pgfscope}%
\begin{pgfscope}%
\pgfpathrectangle{\pgfqpoint{0.765000in}{0.660000in}}{\pgfqpoint{4.620000in}{4.620000in}}%
\pgfusepath{clip}%
\pgfsetbuttcap%
\pgfsetroundjoin%
\definecolor{currentfill}{rgb}{1.000000,0.894118,0.788235}%
\pgfsetfillcolor{currentfill}%
\pgfsetlinewidth{0.000000pt}%
\definecolor{currentstroke}{rgb}{1.000000,0.894118,0.788235}%
\pgfsetstrokecolor{currentstroke}%
\pgfsetdash{}{0pt}%
\pgfpathmoveto{\pgfqpoint{2.818986in}{2.594568in}}%
\pgfpathlineto{\pgfqpoint{2.929883in}{2.530541in}}%
\pgfpathlineto{\pgfqpoint{2.952980in}{2.520265in}}%
\pgfpathlineto{\pgfqpoint{2.842083in}{2.584292in}}%
\pgfpathlineto{\pgfqpoint{2.818986in}{2.594568in}}%
\pgfpathclose%
\pgfusepath{fill}%
\end{pgfscope}%
\begin{pgfscope}%
\pgfpathrectangle{\pgfqpoint{0.765000in}{0.660000in}}{\pgfqpoint{4.620000in}{4.620000in}}%
\pgfusepath{clip}%
\pgfsetbuttcap%
\pgfsetroundjoin%
\definecolor{currentfill}{rgb}{1.000000,0.894118,0.788235}%
\pgfsetfillcolor{currentfill}%
\pgfsetlinewidth{0.000000pt}%
\definecolor{currentstroke}{rgb}{1.000000,0.894118,0.788235}%
\pgfsetstrokecolor{currentstroke}%
\pgfsetdash{}{0pt}%
\pgfpathmoveto{\pgfqpoint{2.929883in}{2.530541in}}%
\pgfpathlineto{\pgfqpoint{2.818986in}{2.466515in}}%
\pgfpathlineto{\pgfqpoint{2.929883in}{2.402489in}}%
\pgfpathlineto{\pgfqpoint{3.040780in}{2.466515in}}%
\pgfpathlineto{\pgfqpoint{2.929883in}{2.530541in}}%
\pgfpathclose%
\pgfusepath{fill}%
\end{pgfscope}%
\begin{pgfscope}%
\pgfpathrectangle{\pgfqpoint{0.765000in}{0.660000in}}{\pgfqpoint{4.620000in}{4.620000in}}%
\pgfusepath{clip}%
\pgfsetbuttcap%
\pgfsetroundjoin%
\definecolor{currentfill}{rgb}{1.000000,0.894118,0.788235}%
\pgfsetfillcolor{currentfill}%
\pgfsetlinewidth{0.000000pt}%
\definecolor{currentstroke}{rgb}{1.000000,0.894118,0.788235}%
\pgfsetstrokecolor{currentstroke}%
\pgfsetdash{}{0pt}%
\pgfpathmoveto{\pgfqpoint{2.929883in}{2.530541in}}%
\pgfpathlineto{\pgfqpoint{2.818986in}{2.466515in}}%
\pgfpathlineto{\pgfqpoint{2.818986in}{2.594568in}}%
\pgfpathlineto{\pgfqpoint{2.929883in}{2.658594in}}%
\pgfpathlineto{\pgfqpoint{2.929883in}{2.530541in}}%
\pgfpathclose%
\pgfusepath{fill}%
\end{pgfscope}%
\begin{pgfscope}%
\pgfpathrectangle{\pgfqpoint{0.765000in}{0.660000in}}{\pgfqpoint{4.620000in}{4.620000in}}%
\pgfusepath{clip}%
\pgfsetbuttcap%
\pgfsetroundjoin%
\definecolor{currentfill}{rgb}{1.000000,0.894118,0.788235}%
\pgfsetfillcolor{currentfill}%
\pgfsetlinewidth{0.000000pt}%
\definecolor{currentstroke}{rgb}{1.000000,0.894118,0.788235}%
\pgfsetstrokecolor{currentstroke}%
\pgfsetdash{}{0pt}%
\pgfpathmoveto{\pgfqpoint{2.929883in}{2.530541in}}%
\pgfpathlineto{\pgfqpoint{3.040780in}{2.466515in}}%
\pgfpathlineto{\pgfqpoint{3.040780in}{2.594568in}}%
\pgfpathlineto{\pgfqpoint{2.929883in}{2.658594in}}%
\pgfpathlineto{\pgfqpoint{2.929883in}{2.530541in}}%
\pgfpathclose%
\pgfusepath{fill}%
\end{pgfscope}%
\begin{pgfscope}%
\pgfpathrectangle{\pgfqpoint{0.765000in}{0.660000in}}{\pgfqpoint{4.620000in}{4.620000in}}%
\pgfusepath{clip}%
\pgfsetbuttcap%
\pgfsetroundjoin%
\definecolor{currentfill}{rgb}{1.000000,0.894118,0.788235}%
\pgfsetfillcolor{currentfill}%
\pgfsetlinewidth{0.000000pt}%
\definecolor{currentstroke}{rgb}{1.000000,0.894118,0.788235}%
\pgfsetstrokecolor{currentstroke}%
\pgfsetdash{}{0pt}%
\pgfpathmoveto{\pgfqpoint{2.915238in}{2.538405in}}%
\pgfpathlineto{\pgfqpoint{2.804341in}{2.474379in}}%
\pgfpathlineto{\pgfqpoint{2.915238in}{2.410353in}}%
\pgfpathlineto{\pgfqpoint{3.026135in}{2.474379in}}%
\pgfpathlineto{\pgfqpoint{2.915238in}{2.538405in}}%
\pgfpathclose%
\pgfusepath{fill}%
\end{pgfscope}%
\begin{pgfscope}%
\pgfpathrectangle{\pgfqpoint{0.765000in}{0.660000in}}{\pgfqpoint{4.620000in}{4.620000in}}%
\pgfusepath{clip}%
\pgfsetbuttcap%
\pgfsetroundjoin%
\definecolor{currentfill}{rgb}{1.000000,0.894118,0.788235}%
\pgfsetfillcolor{currentfill}%
\pgfsetlinewidth{0.000000pt}%
\definecolor{currentstroke}{rgb}{1.000000,0.894118,0.788235}%
\pgfsetstrokecolor{currentstroke}%
\pgfsetdash{}{0pt}%
\pgfpathmoveto{\pgfqpoint{2.915238in}{2.538405in}}%
\pgfpathlineto{\pgfqpoint{2.804341in}{2.474379in}}%
\pgfpathlineto{\pgfqpoint{2.804341in}{2.602432in}}%
\pgfpathlineto{\pgfqpoint{2.915238in}{2.666458in}}%
\pgfpathlineto{\pgfqpoint{2.915238in}{2.538405in}}%
\pgfpathclose%
\pgfusepath{fill}%
\end{pgfscope}%
\begin{pgfscope}%
\pgfpathrectangle{\pgfqpoint{0.765000in}{0.660000in}}{\pgfqpoint{4.620000in}{4.620000in}}%
\pgfusepath{clip}%
\pgfsetbuttcap%
\pgfsetroundjoin%
\definecolor{currentfill}{rgb}{1.000000,0.894118,0.788235}%
\pgfsetfillcolor{currentfill}%
\pgfsetlinewidth{0.000000pt}%
\definecolor{currentstroke}{rgb}{1.000000,0.894118,0.788235}%
\pgfsetstrokecolor{currentstroke}%
\pgfsetdash{}{0pt}%
\pgfpathmoveto{\pgfqpoint{2.915238in}{2.538405in}}%
\pgfpathlineto{\pgfqpoint{3.026135in}{2.474379in}}%
\pgfpathlineto{\pgfqpoint{3.026135in}{2.602432in}}%
\pgfpathlineto{\pgfqpoint{2.915238in}{2.666458in}}%
\pgfpathlineto{\pgfqpoint{2.915238in}{2.538405in}}%
\pgfpathclose%
\pgfusepath{fill}%
\end{pgfscope}%
\begin{pgfscope}%
\pgfpathrectangle{\pgfqpoint{0.765000in}{0.660000in}}{\pgfqpoint{4.620000in}{4.620000in}}%
\pgfusepath{clip}%
\pgfsetbuttcap%
\pgfsetroundjoin%
\definecolor{currentfill}{rgb}{1.000000,0.894118,0.788235}%
\pgfsetfillcolor{currentfill}%
\pgfsetlinewidth{0.000000pt}%
\definecolor{currentstroke}{rgb}{1.000000,0.894118,0.788235}%
\pgfsetstrokecolor{currentstroke}%
\pgfsetdash{}{0pt}%
\pgfpathmoveto{\pgfqpoint{2.929883in}{2.658594in}}%
\pgfpathlineto{\pgfqpoint{2.818986in}{2.594568in}}%
\pgfpathlineto{\pgfqpoint{2.929883in}{2.530541in}}%
\pgfpathlineto{\pgfqpoint{3.040780in}{2.594568in}}%
\pgfpathlineto{\pgfqpoint{2.929883in}{2.658594in}}%
\pgfpathclose%
\pgfusepath{fill}%
\end{pgfscope}%
\begin{pgfscope}%
\pgfpathrectangle{\pgfqpoint{0.765000in}{0.660000in}}{\pgfqpoint{4.620000in}{4.620000in}}%
\pgfusepath{clip}%
\pgfsetbuttcap%
\pgfsetroundjoin%
\definecolor{currentfill}{rgb}{1.000000,0.894118,0.788235}%
\pgfsetfillcolor{currentfill}%
\pgfsetlinewidth{0.000000pt}%
\definecolor{currentstroke}{rgb}{1.000000,0.894118,0.788235}%
\pgfsetstrokecolor{currentstroke}%
\pgfsetdash{}{0pt}%
\pgfpathmoveto{\pgfqpoint{2.929883in}{2.402489in}}%
\pgfpathlineto{\pgfqpoint{3.040780in}{2.466515in}}%
\pgfpathlineto{\pgfqpoint{3.040780in}{2.594568in}}%
\pgfpathlineto{\pgfqpoint{2.929883in}{2.530541in}}%
\pgfpathlineto{\pgfqpoint{2.929883in}{2.402489in}}%
\pgfpathclose%
\pgfusepath{fill}%
\end{pgfscope}%
\begin{pgfscope}%
\pgfpathrectangle{\pgfqpoint{0.765000in}{0.660000in}}{\pgfqpoint{4.620000in}{4.620000in}}%
\pgfusepath{clip}%
\pgfsetbuttcap%
\pgfsetroundjoin%
\definecolor{currentfill}{rgb}{1.000000,0.894118,0.788235}%
\pgfsetfillcolor{currentfill}%
\pgfsetlinewidth{0.000000pt}%
\definecolor{currentstroke}{rgb}{1.000000,0.894118,0.788235}%
\pgfsetstrokecolor{currentstroke}%
\pgfsetdash{}{0pt}%
\pgfpathmoveto{\pgfqpoint{2.818986in}{2.466515in}}%
\pgfpathlineto{\pgfqpoint{2.929883in}{2.402489in}}%
\pgfpathlineto{\pgfqpoint{2.929883in}{2.530541in}}%
\pgfpathlineto{\pgfqpoint{2.818986in}{2.594568in}}%
\pgfpathlineto{\pgfqpoint{2.818986in}{2.466515in}}%
\pgfpathclose%
\pgfusepath{fill}%
\end{pgfscope}%
\begin{pgfscope}%
\pgfpathrectangle{\pgfqpoint{0.765000in}{0.660000in}}{\pgfqpoint{4.620000in}{4.620000in}}%
\pgfusepath{clip}%
\pgfsetbuttcap%
\pgfsetroundjoin%
\definecolor{currentfill}{rgb}{1.000000,0.894118,0.788235}%
\pgfsetfillcolor{currentfill}%
\pgfsetlinewidth{0.000000pt}%
\definecolor{currentstroke}{rgb}{1.000000,0.894118,0.788235}%
\pgfsetstrokecolor{currentstroke}%
\pgfsetdash{}{0pt}%
\pgfpathmoveto{\pgfqpoint{2.915238in}{2.666458in}}%
\pgfpathlineto{\pgfqpoint{2.804341in}{2.602432in}}%
\pgfpathlineto{\pgfqpoint{2.915238in}{2.538405in}}%
\pgfpathlineto{\pgfqpoint{3.026135in}{2.602432in}}%
\pgfpathlineto{\pgfqpoint{2.915238in}{2.666458in}}%
\pgfpathclose%
\pgfusepath{fill}%
\end{pgfscope}%
\begin{pgfscope}%
\pgfpathrectangle{\pgfqpoint{0.765000in}{0.660000in}}{\pgfqpoint{4.620000in}{4.620000in}}%
\pgfusepath{clip}%
\pgfsetbuttcap%
\pgfsetroundjoin%
\definecolor{currentfill}{rgb}{1.000000,0.894118,0.788235}%
\pgfsetfillcolor{currentfill}%
\pgfsetlinewidth{0.000000pt}%
\definecolor{currentstroke}{rgb}{1.000000,0.894118,0.788235}%
\pgfsetstrokecolor{currentstroke}%
\pgfsetdash{}{0pt}%
\pgfpathmoveto{\pgfqpoint{2.915238in}{2.410353in}}%
\pgfpathlineto{\pgfqpoint{3.026135in}{2.474379in}}%
\pgfpathlineto{\pgfqpoint{3.026135in}{2.602432in}}%
\pgfpathlineto{\pgfqpoint{2.915238in}{2.538405in}}%
\pgfpathlineto{\pgfqpoint{2.915238in}{2.410353in}}%
\pgfpathclose%
\pgfusepath{fill}%
\end{pgfscope}%
\begin{pgfscope}%
\pgfpathrectangle{\pgfqpoint{0.765000in}{0.660000in}}{\pgfqpoint{4.620000in}{4.620000in}}%
\pgfusepath{clip}%
\pgfsetbuttcap%
\pgfsetroundjoin%
\definecolor{currentfill}{rgb}{1.000000,0.894118,0.788235}%
\pgfsetfillcolor{currentfill}%
\pgfsetlinewidth{0.000000pt}%
\definecolor{currentstroke}{rgb}{1.000000,0.894118,0.788235}%
\pgfsetstrokecolor{currentstroke}%
\pgfsetdash{}{0pt}%
\pgfpathmoveto{\pgfqpoint{2.804341in}{2.474379in}}%
\pgfpathlineto{\pgfqpoint{2.915238in}{2.410353in}}%
\pgfpathlineto{\pgfqpoint{2.915238in}{2.538405in}}%
\pgfpathlineto{\pgfqpoint{2.804341in}{2.602432in}}%
\pgfpathlineto{\pgfqpoint{2.804341in}{2.474379in}}%
\pgfpathclose%
\pgfusepath{fill}%
\end{pgfscope}%
\begin{pgfscope}%
\pgfpathrectangle{\pgfqpoint{0.765000in}{0.660000in}}{\pgfqpoint{4.620000in}{4.620000in}}%
\pgfusepath{clip}%
\pgfsetbuttcap%
\pgfsetroundjoin%
\definecolor{currentfill}{rgb}{1.000000,0.894118,0.788235}%
\pgfsetfillcolor{currentfill}%
\pgfsetlinewidth{0.000000pt}%
\definecolor{currentstroke}{rgb}{1.000000,0.894118,0.788235}%
\pgfsetstrokecolor{currentstroke}%
\pgfsetdash{}{0pt}%
\pgfpathmoveto{\pgfqpoint{2.929883in}{2.530541in}}%
\pgfpathlineto{\pgfqpoint{2.818986in}{2.466515in}}%
\pgfpathlineto{\pgfqpoint{2.804341in}{2.474379in}}%
\pgfpathlineto{\pgfqpoint{2.915238in}{2.538405in}}%
\pgfpathlineto{\pgfqpoint{2.929883in}{2.530541in}}%
\pgfpathclose%
\pgfusepath{fill}%
\end{pgfscope}%
\begin{pgfscope}%
\pgfpathrectangle{\pgfqpoint{0.765000in}{0.660000in}}{\pgfqpoint{4.620000in}{4.620000in}}%
\pgfusepath{clip}%
\pgfsetbuttcap%
\pgfsetroundjoin%
\definecolor{currentfill}{rgb}{1.000000,0.894118,0.788235}%
\pgfsetfillcolor{currentfill}%
\pgfsetlinewidth{0.000000pt}%
\definecolor{currentstroke}{rgb}{1.000000,0.894118,0.788235}%
\pgfsetstrokecolor{currentstroke}%
\pgfsetdash{}{0pt}%
\pgfpathmoveto{\pgfqpoint{3.040780in}{2.466515in}}%
\pgfpathlineto{\pgfqpoint{2.929883in}{2.530541in}}%
\pgfpathlineto{\pgfqpoint{2.915238in}{2.538405in}}%
\pgfpathlineto{\pgfqpoint{3.026135in}{2.474379in}}%
\pgfpathlineto{\pgfqpoint{3.040780in}{2.466515in}}%
\pgfpathclose%
\pgfusepath{fill}%
\end{pgfscope}%
\begin{pgfscope}%
\pgfpathrectangle{\pgfqpoint{0.765000in}{0.660000in}}{\pgfqpoint{4.620000in}{4.620000in}}%
\pgfusepath{clip}%
\pgfsetbuttcap%
\pgfsetroundjoin%
\definecolor{currentfill}{rgb}{1.000000,0.894118,0.788235}%
\pgfsetfillcolor{currentfill}%
\pgfsetlinewidth{0.000000pt}%
\definecolor{currentstroke}{rgb}{1.000000,0.894118,0.788235}%
\pgfsetstrokecolor{currentstroke}%
\pgfsetdash{}{0pt}%
\pgfpathmoveto{\pgfqpoint{2.929883in}{2.530541in}}%
\pgfpathlineto{\pgfqpoint{2.929883in}{2.658594in}}%
\pgfpathlineto{\pgfqpoint{2.915238in}{2.666458in}}%
\pgfpathlineto{\pgfqpoint{3.026135in}{2.474379in}}%
\pgfpathlineto{\pgfqpoint{2.929883in}{2.530541in}}%
\pgfpathclose%
\pgfusepath{fill}%
\end{pgfscope}%
\begin{pgfscope}%
\pgfpathrectangle{\pgfqpoint{0.765000in}{0.660000in}}{\pgfqpoint{4.620000in}{4.620000in}}%
\pgfusepath{clip}%
\pgfsetbuttcap%
\pgfsetroundjoin%
\definecolor{currentfill}{rgb}{1.000000,0.894118,0.788235}%
\pgfsetfillcolor{currentfill}%
\pgfsetlinewidth{0.000000pt}%
\definecolor{currentstroke}{rgb}{1.000000,0.894118,0.788235}%
\pgfsetstrokecolor{currentstroke}%
\pgfsetdash{}{0pt}%
\pgfpathmoveto{\pgfqpoint{3.040780in}{2.466515in}}%
\pgfpathlineto{\pgfqpoint{3.040780in}{2.594568in}}%
\pgfpathlineto{\pgfqpoint{3.026135in}{2.602432in}}%
\pgfpathlineto{\pgfqpoint{2.915238in}{2.538405in}}%
\pgfpathlineto{\pgfqpoint{3.040780in}{2.466515in}}%
\pgfpathclose%
\pgfusepath{fill}%
\end{pgfscope}%
\begin{pgfscope}%
\pgfpathrectangle{\pgfqpoint{0.765000in}{0.660000in}}{\pgfqpoint{4.620000in}{4.620000in}}%
\pgfusepath{clip}%
\pgfsetbuttcap%
\pgfsetroundjoin%
\definecolor{currentfill}{rgb}{1.000000,0.894118,0.788235}%
\pgfsetfillcolor{currentfill}%
\pgfsetlinewidth{0.000000pt}%
\definecolor{currentstroke}{rgb}{1.000000,0.894118,0.788235}%
\pgfsetstrokecolor{currentstroke}%
\pgfsetdash{}{0pt}%
\pgfpathmoveto{\pgfqpoint{2.818986in}{2.466515in}}%
\pgfpathlineto{\pgfqpoint{2.929883in}{2.402489in}}%
\pgfpathlineto{\pgfqpoint{2.915238in}{2.410353in}}%
\pgfpathlineto{\pgfqpoint{2.804341in}{2.474379in}}%
\pgfpathlineto{\pgfqpoint{2.818986in}{2.466515in}}%
\pgfpathclose%
\pgfusepath{fill}%
\end{pgfscope}%
\begin{pgfscope}%
\pgfpathrectangle{\pgfqpoint{0.765000in}{0.660000in}}{\pgfqpoint{4.620000in}{4.620000in}}%
\pgfusepath{clip}%
\pgfsetbuttcap%
\pgfsetroundjoin%
\definecolor{currentfill}{rgb}{1.000000,0.894118,0.788235}%
\pgfsetfillcolor{currentfill}%
\pgfsetlinewidth{0.000000pt}%
\definecolor{currentstroke}{rgb}{1.000000,0.894118,0.788235}%
\pgfsetstrokecolor{currentstroke}%
\pgfsetdash{}{0pt}%
\pgfpathmoveto{\pgfqpoint{2.929883in}{2.402489in}}%
\pgfpathlineto{\pgfqpoint{3.040780in}{2.466515in}}%
\pgfpathlineto{\pgfqpoint{3.026135in}{2.474379in}}%
\pgfpathlineto{\pgfqpoint{2.915238in}{2.410353in}}%
\pgfpathlineto{\pgfqpoint{2.929883in}{2.402489in}}%
\pgfpathclose%
\pgfusepath{fill}%
\end{pgfscope}%
\begin{pgfscope}%
\pgfpathrectangle{\pgfqpoint{0.765000in}{0.660000in}}{\pgfqpoint{4.620000in}{4.620000in}}%
\pgfusepath{clip}%
\pgfsetbuttcap%
\pgfsetroundjoin%
\definecolor{currentfill}{rgb}{1.000000,0.894118,0.788235}%
\pgfsetfillcolor{currentfill}%
\pgfsetlinewidth{0.000000pt}%
\definecolor{currentstroke}{rgb}{1.000000,0.894118,0.788235}%
\pgfsetstrokecolor{currentstroke}%
\pgfsetdash{}{0pt}%
\pgfpathmoveto{\pgfqpoint{2.929883in}{2.658594in}}%
\pgfpathlineto{\pgfqpoint{2.818986in}{2.594568in}}%
\pgfpathlineto{\pgfqpoint{2.804341in}{2.602432in}}%
\pgfpathlineto{\pgfqpoint{2.915238in}{2.666458in}}%
\pgfpathlineto{\pgfqpoint{2.929883in}{2.658594in}}%
\pgfpathclose%
\pgfusepath{fill}%
\end{pgfscope}%
\begin{pgfscope}%
\pgfpathrectangle{\pgfqpoint{0.765000in}{0.660000in}}{\pgfqpoint{4.620000in}{4.620000in}}%
\pgfusepath{clip}%
\pgfsetbuttcap%
\pgfsetroundjoin%
\definecolor{currentfill}{rgb}{1.000000,0.894118,0.788235}%
\pgfsetfillcolor{currentfill}%
\pgfsetlinewidth{0.000000pt}%
\definecolor{currentstroke}{rgb}{1.000000,0.894118,0.788235}%
\pgfsetstrokecolor{currentstroke}%
\pgfsetdash{}{0pt}%
\pgfpathmoveto{\pgfqpoint{3.040780in}{2.594568in}}%
\pgfpathlineto{\pgfqpoint{2.929883in}{2.658594in}}%
\pgfpathlineto{\pgfqpoint{2.915238in}{2.666458in}}%
\pgfpathlineto{\pgfqpoint{3.026135in}{2.602432in}}%
\pgfpathlineto{\pgfqpoint{3.040780in}{2.594568in}}%
\pgfpathclose%
\pgfusepath{fill}%
\end{pgfscope}%
\begin{pgfscope}%
\pgfpathrectangle{\pgfqpoint{0.765000in}{0.660000in}}{\pgfqpoint{4.620000in}{4.620000in}}%
\pgfusepath{clip}%
\pgfsetbuttcap%
\pgfsetroundjoin%
\definecolor{currentfill}{rgb}{1.000000,0.894118,0.788235}%
\pgfsetfillcolor{currentfill}%
\pgfsetlinewidth{0.000000pt}%
\definecolor{currentstroke}{rgb}{1.000000,0.894118,0.788235}%
\pgfsetstrokecolor{currentstroke}%
\pgfsetdash{}{0pt}%
\pgfpathmoveto{\pgfqpoint{2.818986in}{2.466515in}}%
\pgfpathlineto{\pgfqpoint{2.818986in}{2.594568in}}%
\pgfpathlineto{\pgfqpoint{2.804341in}{2.602432in}}%
\pgfpathlineto{\pgfqpoint{2.915238in}{2.410353in}}%
\pgfpathlineto{\pgfqpoint{2.818986in}{2.466515in}}%
\pgfpathclose%
\pgfusepath{fill}%
\end{pgfscope}%
\begin{pgfscope}%
\pgfpathrectangle{\pgfqpoint{0.765000in}{0.660000in}}{\pgfqpoint{4.620000in}{4.620000in}}%
\pgfusepath{clip}%
\pgfsetbuttcap%
\pgfsetroundjoin%
\definecolor{currentfill}{rgb}{1.000000,0.894118,0.788235}%
\pgfsetfillcolor{currentfill}%
\pgfsetlinewidth{0.000000pt}%
\definecolor{currentstroke}{rgb}{1.000000,0.894118,0.788235}%
\pgfsetstrokecolor{currentstroke}%
\pgfsetdash{}{0pt}%
\pgfpathmoveto{\pgfqpoint{2.929883in}{2.402489in}}%
\pgfpathlineto{\pgfqpoint{2.929883in}{2.530541in}}%
\pgfpathlineto{\pgfqpoint{2.915238in}{2.538405in}}%
\pgfpathlineto{\pgfqpoint{2.804341in}{2.474379in}}%
\pgfpathlineto{\pgfqpoint{2.929883in}{2.402489in}}%
\pgfpathclose%
\pgfusepath{fill}%
\end{pgfscope}%
\begin{pgfscope}%
\pgfpathrectangle{\pgfqpoint{0.765000in}{0.660000in}}{\pgfqpoint{4.620000in}{4.620000in}}%
\pgfusepath{clip}%
\pgfsetbuttcap%
\pgfsetroundjoin%
\definecolor{currentfill}{rgb}{1.000000,0.894118,0.788235}%
\pgfsetfillcolor{currentfill}%
\pgfsetlinewidth{0.000000pt}%
\definecolor{currentstroke}{rgb}{1.000000,0.894118,0.788235}%
\pgfsetstrokecolor{currentstroke}%
\pgfsetdash{}{0pt}%
\pgfpathmoveto{\pgfqpoint{2.929883in}{2.530541in}}%
\pgfpathlineto{\pgfqpoint{3.040780in}{2.594568in}}%
\pgfpathlineto{\pgfqpoint{3.026135in}{2.602432in}}%
\pgfpathlineto{\pgfqpoint{2.915238in}{2.538405in}}%
\pgfpathlineto{\pgfqpoint{2.929883in}{2.530541in}}%
\pgfpathclose%
\pgfusepath{fill}%
\end{pgfscope}%
\begin{pgfscope}%
\pgfpathrectangle{\pgfqpoint{0.765000in}{0.660000in}}{\pgfqpoint{4.620000in}{4.620000in}}%
\pgfusepath{clip}%
\pgfsetbuttcap%
\pgfsetroundjoin%
\definecolor{currentfill}{rgb}{1.000000,0.894118,0.788235}%
\pgfsetfillcolor{currentfill}%
\pgfsetlinewidth{0.000000pt}%
\definecolor{currentstroke}{rgb}{1.000000,0.894118,0.788235}%
\pgfsetstrokecolor{currentstroke}%
\pgfsetdash{}{0pt}%
\pgfpathmoveto{\pgfqpoint{2.818986in}{2.594568in}}%
\pgfpathlineto{\pgfqpoint{2.929883in}{2.530541in}}%
\pgfpathlineto{\pgfqpoint{2.915238in}{2.538405in}}%
\pgfpathlineto{\pgfqpoint{2.804341in}{2.602432in}}%
\pgfpathlineto{\pgfqpoint{2.818986in}{2.594568in}}%
\pgfpathclose%
\pgfusepath{fill}%
\end{pgfscope}%
\begin{pgfscope}%
\pgfpathrectangle{\pgfqpoint{0.765000in}{0.660000in}}{\pgfqpoint{4.620000in}{4.620000in}}%
\pgfusepath{clip}%
\pgfsetbuttcap%
\pgfsetroundjoin%
\definecolor{currentfill}{rgb}{1.000000,0.894118,0.788235}%
\pgfsetfillcolor{currentfill}%
\pgfsetlinewidth{0.000000pt}%
\definecolor{currentstroke}{rgb}{1.000000,0.894118,0.788235}%
\pgfsetstrokecolor{currentstroke}%
\pgfsetdash{}{0pt}%
\pgfpathmoveto{\pgfqpoint{3.334893in}{3.441865in}}%
\pgfpathlineto{\pgfqpoint{3.223996in}{3.377839in}}%
\pgfpathlineto{\pgfqpoint{3.334893in}{3.313813in}}%
\pgfpathlineto{\pgfqpoint{3.445790in}{3.377839in}}%
\pgfpathlineto{\pgfqpoint{3.334893in}{3.441865in}}%
\pgfpathclose%
\pgfusepath{fill}%
\end{pgfscope}%
\begin{pgfscope}%
\pgfpathrectangle{\pgfqpoint{0.765000in}{0.660000in}}{\pgfqpoint{4.620000in}{4.620000in}}%
\pgfusepath{clip}%
\pgfsetbuttcap%
\pgfsetroundjoin%
\definecolor{currentfill}{rgb}{1.000000,0.894118,0.788235}%
\pgfsetfillcolor{currentfill}%
\pgfsetlinewidth{0.000000pt}%
\definecolor{currentstroke}{rgb}{1.000000,0.894118,0.788235}%
\pgfsetstrokecolor{currentstroke}%
\pgfsetdash{}{0pt}%
\pgfpathmoveto{\pgfqpoint{3.334893in}{3.441865in}}%
\pgfpathlineto{\pgfqpoint{3.223996in}{3.377839in}}%
\pgfpathlineto{\pgfqpoint{3.223996in}{3.505892in}}%
\pgfpathlineto{\pgfqpoint{3.334893in}{3.569918in}}%
\pgfpathlineto{\pgfqpoint{3.334893in}{3.441865in}}%
\pgfpathclose%
\pgfusepath{fill}%
\end{pgfscope}%
\begin{pgfscope}%
\pgfpathrectangle{\pgfqpoint{0.765000in}{0.660000in}}{\pgfqpoint{4.620000in}{4.620000in}}%
\pgfusepath{clip}%
\pgfsetbuttcap%
\pgfsetroundjoin%
\definecolor{currentfill}{rgb}{1.000000,0.894118,0.788235}%
\pgfsetfillcolor{currentfill}%
\pgfsetlinewidth{0.000000pt}%
\definecolor{currentstroke}{rgb}{1.000000,0.894118,0.788235}%
\pgfsetstrokecolor{currentstroke}%
\pgfsetdash{}{0pt}%
\pgfpathmoveto{\pgfqpoint{3.334893in}{3.441865in}}%
\pgfpathlineto{\pgfqpoint{3.445790in}{3.377839in}}%
\pgfpathlineto{\pgfqpoint{3.445790in}{3.505892in}}%
\pgfpathlineto{\pgfqpoint{3.334893in}{3.569918in}}%
\pgfpathlineto{\pgfqpoint{3.334893in}{3.441865in}}%
\pgfpathclose%
\pgfusepath{fill}%
\end{pgfscope}%
\begin{pgfscope}%
\pgfpathrectangle{\pgfqpoint{0.765000in}{0.660000in}}{\pgfqpoint{4.620000in}{4.620000in}}%
\pgfusepath{clip}%
\pgfsetbuttcap%
\pgfsetroundjoin%
\definecolor{currentfill}{rgb}{1.000000,0.894118,0.788235}%
\pgfsetfillcolor{currentfill}%
\pgfsetlinewidth{0.000000pt}%
\definecolor{currentstroke}{rgb}{1.000000,0.894118,0.788235}%
\pgfsetstrokecolor{currentstroke}%
\pgfsetdash{}{0pt}%
\pgfpathmoveto{\pgfqpoint{3.310017in}{3.461366in}}%
\pgfpathlineto{\pgfqpoint{3.199120in}{3.397340in}}%
\pgfpathlineto{\pgfqpoint{3.310017in}{3.333313in}}%
\pgfpathlineto{\pgfqpoint{3.420914in}{3.397340in}}%
\pgfpathlineto{\pgfqpoint{3.310017in}{3.461366in}}%
\pgfpathclose%
\pgfusepath{fill}%
\end{pgfscope}%
\begin{pgfscope}%
\pgfpathrectangle{\pgfqpoint{0.765000in}{0.660000in}}{\pgfqpoint{4.620000in}{4.620000in}}%
\pgfusepath{clip}%
\pgfsetbuttcap%
\pgfsetroundjoin%
\definecolor{currentfill}{rgb}{1.000000,0.894118,0.788235}%
\pgfsetfillcolor{currentfill}%
\pgfsetlinewidth{0.000000pt}%
\definecolor{currentstroke}{rgb}{1.000000,0.894118,0.788235}%
\pgfsetstrokecolor{currentstroke}%
\pgfsetdash{}{0pt}%
\pgfpathmoveto{\pgfqpoint{3.310017in}{3.461366in}}%
\pgfpathlineto{\pgfqpoint{3.199120in}{3.397340in}}%
\pgfpathlineto{\pgfqpoint{3.199120in}{3.525392in}}%
\pgfpathlineto{\pgfqpoint{3.310017in}{3.589418in}}%
\pgfpathlineto{\pgfqpoint{3.310017in}{3.461366in}}%
\pgfpathclose%
\pgfusepath{fill}%
\end{pgfscope}%
\begin{pgfscope}%
\pgfpathrectangle{\pgfqpoint{0.765000in}{0.660000in}}{\pgfqpoint{4.620000in}{4.620000in}}%
\pgfusepath{clip}%
\pgfsetbuttcap%
\pgfsetroundjoin%
\definecolor{currentfill}{rgb}{1.000000,0.894118,0.788235}%
\pgfsetfillcolor{currentfill}%
\pgfsetlinewidth{0.000000pt}%
\definecolor{currentstroke}{rgb}{1.000000,0.894118,0.788235}%
\pgfsetstrokecolor{currentstroke}%
\pgfsetdash{}{0pt}%
\pgfpathmoveto{\pgfqpoint{3.310017in}{3.461366in}}%
\pgfpathlineto{\pgfqpoint{3.420914in}{3.397340in}}%
\pgfpathlineto{\pgfqpoint{3.420914in}{3.525392in}}%
\pgfpathlineto{\pgfqpoint{3.310017in}{3.589418in}}%
\pgfpathlineto{\pgfqpoint{3.310017in}{3.461366in}}%
\pgfpathclose%
\pgfusepath{fill}%
\end{pgfscope}%
\begin{pgfscope}%
\pgfpathrectangle{\pgfqpoint{0.765000in}{0.660000in}}{\pgfqpoint{4.620000in}{4.620000in}}%
\pgfusepath{clip}%
\pgfsetbuttcap%
\pgfsetroundjoin%
\definecolor{currentfill}{rgb}{1.000000,0.894118,0.788235}%
\pgfsetfillcolor{currentfill}%
\pgfsetlinewidth{0.000000pt}%
\definecolor{currentstroke}{rgb}{1.000000,0.894118,0.788235}%
\pgfsetstrokecolor{currentstroke}%
\pgfsetdash{}{0pt}%
\pgfpathmoveto{\pgfqpoint{3.334893in}{3.569918in}}%
\pgfpathlineto{\pgfqpoint{3.223996in}{3.505892in}}%
\pgfpathlineto{\pgfqpoint{3.334893in}{3.441865in}}%
\pgfpathlineto{\pgfqpoint{3.445790in}{3.505892in}}%
\pgfpathlineto{\pgfqpoint{3.334893in}{3.569918in}}%
\pgfpathclose%
\pgfusepath{fill}%
\end{pgfscope}%
\begin{pgfscope}%
\pgfpathrectangle{\pgfqpoint{0.765000in}{0.660000in}}{\pgfqpoint{4.620000in}{4.620000in}}%
\pgfusepath{clip}%
\pgfsetbuttcap%
\pgfsetroundjoin%
\definecolor{currentfill}{rgb}{1.000000,0.894118,0.788235}%
\pgfsetfillcolor{currentfill}%
\pgfsetlinewidth{0.000000pt}%
\definecolor{currentstroke}{rgb}{1.000000,0.894118,0.788235}%
\pgfsetstrokecolor{currentstroke}%
\pgfsetdash{}{0pt}%
\pgfpathmoveto{\pgfqpoint{3.334893in}{3.313813in}}%
\pgfpathlineto{\pgfqpoint{3.445790in}{3.377839in}}%
\pgfpathlineto{\pgfqpoint{3.445790in}{3.505892in}}%
\pgfpathlineto{\pgfqpoint{3.334893in}{3.441865in}}%
\pgfpathlineto{\pgfqpoint{3.334893in}{3.313813in}}%
\pgfpathclose%
\pgfusepath{fill}%
\end{pgfscope}%
\begin{pgfscope}%
\pgfpathrectangle{\pgfqpoint{0.765000in}{0.660000in}}{\pgfqpoint{4.620000in}{4.620000in}}%
\pgfusepath{clip}%
\pgfsetbuttcap%
\pgfsetroundjoin%
\definecolor{currentfill}{rgb}{1.000000,0.894118,0.788235}%
\pgfsetfillcolor{currentfill}%
\pgfsetlinewidth{0.000000pt}%
\definecolor{currentstroke}{rgb}{1.000000,0.894118,0.788235}%
\pgfsetstrokecolor{currentstroke}%
\pgfsetdash{}{0pt}%
\pgfpathmoveto{\pgfqpoint{3.223996in}{3.377839in}}%
\pgfpathlineto{\pgfqpoint{3.334893in}{3.313813in}}%
\pgfpathlineto{\pgfqpoint{3.334893in}{3.441865in}}%
\pgfpathlineto{\pgfqpoint{3.223996in}{3.505892in}}%
\pgfpathlineto{\pgfqpoint{3.223996in}{3.377839in}}%
\pgfpathclose%
\pgfusepath{fill}%
\end{pgfscope}%
\begin{pgfscope}%
\pgfpathrectangle{\pgfqpoint{0.765000in}{0.660000in}}{\pgfqpoint{4.620000in}{4.620000in}}%
\pgfusepath{clip}%
\pgfsetbuttcap%
\pgfsetroundjoin%
\definecolor{currentfill}{rgb}{1.000000,0.894118,0.788235}%
\pgfsetfillcolor{currentfill}%
\pgfsetlinewidth{0.000000pt}%
\definecolor{currentstroke}{rgb}{1.000000,0.894118,0.788235}%
\pgfsetstrokecolor{currentstroke}%
\pgfsetdash{}{0pt}%
\pgfpathmoveto{\pgfqpoint{3.310017in}{3.589418in}}%
\pgfpathlineto{\pgfqpoint{3.199120in}{3.525392in}}%
\pgfpathlineto{\pgfqpoint{3.310017in}{3.461366in}}%
\pgfpathlineto{\pgfqpoint{3.420914in}{3.525392in}}%
\pgfpathlineto{\pgfqpoint{3.310017in}{3.589418in}}%
\pgfpathclose%
\pgfusepath{fill}%
\end{pgfscope}%
\begin{pgfscope}%
\pgfpathrectangle{\pgfqpoint{0.765000in}{0.660000in}}{\pgfqpoint{4.620000in}{4.620000in}}%
\pgfusepath{clip}%
\pgfsetbuttcap%
\pgfsetroundjoin%
\definecolor{currentfill}{rgb}{1.000000,0.894118,0.788235}%
\pgfsetfillcolor{currentfill}%
\pgfsetlinewidth{0.000000pt}%
\definecolor{currentstroke}{rgb}{1.000000,0.894118,0.788235}%
\pgfsetstrokecolor{currentstroke}%
\pgfsetdash{}{0pt}%
\pgfpathmoveto{\pgfqpoint{3.310017in}{3.333313in}}%
\pgfpathlineto{\pgfqpoint{3.420914in}{3.397340in}}%
\pgfpathlineto{\pgfqpoint{3.420914in}{3.525392in}}%
\pgfpathlineto{\pgfqpoint{3.310017in}{3.461366in}}%
\pgfpathlineto{\pgfqpoint{3.310017in}{3.333313in}}%
\pgfpathclose%
\pgfusepath{fill}%
\end{pgfscope}%
\begin{pgfscope}%
\pgfpathrectangle{\pgfqpoint{0.765000in}{0.660000in}}{\pgfqpoint{4.620000in}{4.620000in}}%
\pgfusepath{clip}%
\pgfsetbuttcap%
\pgfsetroundjoin%
\definecolor{currentfill}{rgb}{1.000000,0.894118,0.788235}%
\pgfsetfillcolor{currentfill}%
\pgfsetlinewidth{0.000000pt}%
\definecolor{currentstroke}{rgb}{1.000000,0.894118,0.788235}%
\pgfsetstrokecolor{currentstroke}%
\pgfsetdash{}{0pt}%
\pgfpathmoveto{\pgfqpoint{3.199120in}{3.397340in}}%
\pgfpathlineto{\pgfqpoint{3.310017in}{3.333313in}}%
\pgfpathlineto{\pgfqpoint{3.310017in}{3.461366in}}%
\pgfpathlineto{\pgfqpoint{3.199120in}{3.525392in}}%
\pgfpathlineto{\pgfqpoint{3.199120in}{3.397340in}}%
\pgfpathclose%
\pgfusepath{fill}%
\end{pgfscope}%
\begin{pgfscope}%
\pgfpathrectangle{\pgfqpoint{0.765000in}{0.660000in}}{\pgfqpoint{4.620000in}{4.620000in}}%
\pgfusepath{clip}%
\pgfsetbuttcap%
\pgfsetroundjoin%
\definecolor{currentfill}{rgb}{1.000000,0.894118,0.788235}%
\pgfsetfillcolor{currentfill}%
\pgfsetlinewidth{0.000000pt}%
\definecolor{currentstroke}{rgb}{1.000000,0.894118,0.788235}%
\pgfsetstrokecolor{currentstroke}%
\pgfsetdash{}{0pt}%
\pgfpathmoveto{\pgfqpoint{3.334893in}{3.441865in}}%
\pgfpathlineto{\pgfqpoint{3.223996in}{3.377839in}}%
\pgfpathlineto{\pgfqpoint{3.199120in}{3.397340in}}%
\pgfpathlineto{\pgfqpoint{3.310017in}{3.461366in}}%
\pgfpathlineto{\pgfqpoint{3.334893in}{3.441865in}}%
\pgfpathclose%
\pgfusepath{fill}%
\end{pgfscope}%
\begin{pgfscope}%
\pgfpathrectangle{\pgfqpoint{0.765000in}{0.660000in}}{\pgfqpoint{4.620000in}{4.620000in}}%
\pgfusepath{clip}%
\pgfsetbuttcap%
\pgfsetroundjoin%
\definecolor{currentfill}{rgb}{1.000000,0.894118,0.788235}%
\pgfsetfillcolor{currentfill}%
\pgfsetlinewidth{0.000000pt}%
\definecolor{currentstroke}{rgb}{1.000000,0.894118,0.788235}%
\pgfsetstrokecolor{currentstroke}%
\pgfsetdash{}{0pt}%
\pgfpathmoveto{\pgfqpoint{3.445790in}{3.377839in}}%
\pgfpathlineto{\pgfqpoint{3.334893in}{3.441865in}}%
\pgfpathlineto{\pgfqpoint{3.310017in}{3.461366in}}%
\pgfpathlineto{\pgfqpoint{3.420914in}{3.397340in}}%
\pgfpathlineto{\pgfqpoint{3.445790in}{3.377839in}}%
\pgfpathclose%
\pgfusepath{fill}%
\end{pgfscope}%
\begin{pgfscope}%
\pgfpathrectangle{\pgfqpoint{0.765000in}{0.660000in}}{\pgfqpoint{4.620000in}{4.620000in}}%
\pgfusepath{clip}%
\pgfsetbuttcap%
\pgfsetroundjoin%
\definecolor{currentfill}{rgb}{1.000000,0.894118,0.788235}%
\pgfsetfillcolor{currentfill}%
\pgfsetlinewidth{0.000000pt}%
\definecolor{currentstroke}{rgb}{1.000000,0.894118,0.788235}%
\pgfsetstrokecolor{currentstroke}%
\pgfsetdash{}{0pt}%
\pgfpathmoveto{\pgfqpoint{3.334893in}{3.441865in}}%
\pgfpathlineto{\pgfqpoint{3.334893in}{3.569918in}}%
\pgfpathlineto{\pgfqpoint{3.310017in}{3.589418in}}%
\pgfpathlineto{\pgfqpoint{3.420914in}{3.397340in}}%
\pgfpathlineto{\pgfqpoint{3.334893in}{3.441865in}}%
\pgfpathclose%
\pgfusepath{fill}%
\end{pgfscope}%
\begin{pgfscope}%
\pgfpathrectangle{\pgfqpoint{0.765000in}{0.660000in}}{\pgfqpoint{4.620000in}{4.620000in}}%
\pgfusepath{clip}%
\pgfsetbuttcap%
\pgfsetroundjoin%
\definecolor{currentfill}{rgb}{1.000000,0.894118,0.788235}%
\pgfsetfillcolor{currentfill}%
\pgfsetlinewidth{0.000000pt}%
\definecolor{currentstroke}{rgb}{1.000000,0.894118,0.788235}%
\pgfsetstrokecolor{currentstroke}%
\pgfsetdash{}{0pt}%
\pgfpathmoveto{\pgfqpoint{3.445790in}{3.377839in}}%
\pgfpathlineto{\pgfqpoint{3.445790in}{3.505892in}}%
\pgfpathlineto{\pgfqpoint{3.420914in}{3.525392in}}%
\pgfpathlineto{\pgfqpoint{3.310017in}{3.461366in}}%
\pgfpathlineto{\pgfqpoint{3.445790in}{3.377839in}}%
\pgfpathclose%
\pgfusepath{fill}%
\end{pgfscope}%
\begin{pgfscope}%
\pgfpathrectangle{\pgfqpoint{0.765000in}{0.660000in}}{\pgfqpoint{4.620000in}{4.620000in}}%
\pgfusepath{clip}%
\pgfsetbuttcap%
\pgfsetroundjoin%
\definecolor{currentfill}{rgb}{1.000000,0.894118,0.788235}%
\pgfsetfillcolor{currentfill}%
\pgfsetlinewidth{0.000000pt}%
\definecolor{currentstroke}{rgb}{1.000000,0.894118,0.788235}%
\pgfsetstrokecolor{currentstroke}%
\pgfsetdash{}{0pt}%
\pgfpathmoveto{\pgfqpoint{3.223996in}{3.377839in}}%
\pgfpathlineto{\pgfqpoint{3.334893in}{3.313813in}}%
\pgfpathlineto{\pgfqpoint{3.310017in}{3.333313in}}%
\pgfpathlineto{\pgfqpoint{3.199120in}{3.397340in}}%
\pgfpathlineto{\pgfqpoint{3.223996in}{3.377839in}}%
\pgfpathclose%
\pgfusepath{fill}%
\end{pgfscope}%
\begin{pgfscope}%
\pgfpathrectangle{\pgfqpoint{0.765000in}{0.660000in}}{\pgfqpoint{4.620000in}{4.620000in}}%
\pgfusepath{clip}%
\pgfsetbuttcap%
\pgfsetroundjoin%
\definecolor{currentfill}{rgb}{1.000000,0.894118,0.788235}%
\pgfsetfillcolor{currentfill}%
\pgfsetlinewidth{0.000000pt}%
\definecolor{currentstroke}{rgb}{1.000000,0.894118,0.788235}%
\pgfsetstrokecolor{currentstroke}%
\pgfsetdash{}{0pt}%
\pgfpathmoveto{\pgfqpoint{3.334893in}{3.313813in}}%
\pgfpathlineto{\pgfqpoint{3.445790in}{3.377839in}}%
\pgfpathlineto{\pgfqpoint{3.420914in}{3.397340in}}%
\pgfpathlineto{\pgfqpoint{3.310017in}{3.333313in}}%
\pgfpathlineto{\pgfqpoint{3.334893in}{3.313813in}}%
\pgfpathclose%
\pgfusepath{fill}%
\end{pgfscope}%
\begin{pgfscope}%
\pgfpathrectangle{\pgfqpoint{0.765000in}{0.660000in}}{\pgfqpoint{4.620000in}{4.620000in}}%
\pgfusepath{clip}%
\pgfsetbuttcap%
\pgfsetroundjoin%
\definecolor{currentfill}{rgb}{1.000000,0.894118,0.788235}%
\pgfsetfillcolor{currentfill}%
\pgfsetlinewidth{0.000000pt}%
\definecolor{currentstroke}{rgb}{1.000000,0.894118,0.788235}%
\pgfsetstrokecolor{currentstroke}%
\pgfsetdash{}{0pt}%
\pgfpathmoveto{\pgfqpoint{3.334893in}{3.569918in}}%
\pgfpathlineto{\pgfqpoint{3.223996in}{3.505892in}}%
\pgfpathlineto{\pgfqpoint{3.199120in}{3.525392in}}%
\pgfpathlineto{\pgfqpoint{3.310017in}{3.589418in}}%
\pgfpathlineto{\pgfqpoint{3.334893in}{3.569918in}}%
\pgfpathclose%
\pgfusepath{fill}%
\end{pgfscope}%
\begin{pgfscope}%
\pgfpathrectangle{\pgfqpoint{0.765000in}{0.660000in}}{\pgfqpoint{4.620000in}{4.620000in}}%
\pgfusepath{clip}%
\pgfsetbuttcap%
\pgfsetroundjoin%
\definecolor{currentfill}{rgb}{1.000000,0.894118,0.788235}%
\pgfsetfillcolor{currentfill}%
\pgfsetlinewidth{0.000000pt}%
\definecolor{currentstroke}{rgb}{1.000000,0.894118,0.788235}%
\pgfsetstrokecolor{currentstroke}%
\pgfsetdash{}{0pt}%
\pgfpathmoveto{\pgfqpoint{3.445790in}{3.505892in}}%
\pgfpathlineto{\pgfqpoint{3.334893in}{3.569918in}}%
\pgfpathlineto{\pgfqpoint{3.310017in}{3.589418in}}%
\pgfpathlineto{\pgfqpoint{3.420914in}{3.525392in}}%
\pgfpathlineto{\pgfqpoint{3.445790in}{3.505892in}}%
\pgfpathclose%
\pgfusepath{fill}%
\end{pgfscope}%
\begin{pgfscope}%
\pgfpathrectangle{\pgfqpoint{0.765000in}{0.660000in}}{\pgfqpoint{4.620000in}{4.620000in}}%
\pgfusepath{clip}%
\pgfsetbuttcap%
\pgfsetroundjoin%
\definecolor{currentfill}{rgb}{1.000000,0.894118,0.788235}%
\pgfsetfillcolor{currentfill}%
\pgfsetlinewidth{0.000000pt}%
\definecolor{currentstroke}{rgb}{1.000000,0.894118,0.788235}%
\pgfsetstrokecolor{currentstroke}%
\pgfsetdash{}{0pt}%
\pgfpathmoveto{\pgfqpoint{3.223996in}{3.377839in}}%
\pgfpathlineto{\pgfqpoint{3.223996in}{3.505892in}}%
\pgfpathlineto{\pgfqpoint{3.199120in}{3.525392in}}%
\pgfpathlineto{\pgfqpoint{3.310017in}{3.333313in}}%
\pgfpathlineto{\pgfqpoint{3.223996in}{3.377839in}}%
\pgfpathclose%
\pgfusepath{fill}%
\end{pgfscope}%
\begin{pgfscope}%
\pgfpathrectangle{\pgfqpoint{0.765000in}{0.660000in}}{\pgfqpoint{4.620000in}{4.620000in}}%
\pgfusepath{clip}%
\pgfsetbuttcap%
\pgfsetroundjoin%
\definecolor{currentfill}{rgb}{1.000000,0.894118,0.788235}%
\pgfsetfillcolor{currentfill}%
\pgfsetlinewidth{0.000000pt}%
\definecolor{currentstroke}{rgb}{1.000000,0.894118,0.788235}%
\pgfsetstrokecolor{currentstroke}%
\pgfsetdash{}{0pt}%
\pgfpathmoveto{\pgfqpoint{3.334893in}{3.313813in}}%
\pgfpathlineto{\pgfqpoint{3.334893in}{3.441865in}}%
\pgfpathlineto{\pgfqpoint{3.310017in}{3.461366in}}%
\pgfpathlineto{\pgfqpoint{3.199120in}{3.397340in}}%
\pgfpathlineto{\pgfqpoint{3.334893in}{3.313813in}}%
\pgfpathclose%
\pgfusepath{fill}%
\end{pgfscope}%
\begin{pgfscope}%
\pgfpathrectangle{\pgfqpoint{0.765000in}{0.660000in}}{\pgfqpoint{4.620000in}{4.620000in}}%
\pgfusepath{clip}%
\pgfsetbuttcap%
\pgfsetroundjoin%
\definecolor{currentfill}{rgb}{1.000000,0.894118,0.788235}%
\pgfsetfillcolor{currentfill}%
\pgfsetlinewidth{0.000000pt}%
\definecolor{currentstroke}{rgb}{1.000000,0.894118,0.788235}%
\pgfsetstrokecolor{currentstroke}%
\pgfsetdash{}{0pt}%
\pgfpathmoveto{\pgfqpoint{3.334893in}{3.441865in}}%
\pgfpathlineto{\pgfqpoint{3.445790in}{3.505892in}}%
\pgfpathlineto{\pgfqpoint{3.420914in}{3.525392in}}%
\pgfpathlineto{\pgfqpoint{3.310017in}{3.461366in}}%
\pgfpathlineto{\pgfqpoint{3.334893in}{3.441865in}}%
\pgfpathclose%
\pgfusepath{fill}%
\end{pgfscope}%
\begin{pgfscope}%
\pgfpathrectangle{\pgfqpoint{0.765000in}{0.660000in}}{\pgfqpoint{4.620000in}{4.620000in}}%
\pgfusepath{clip}%
\pgfsetbuttcap%
\pgfsetroundjoin%
\definecolor{currentfill}{rgb}{1.000000,0.894118,0.788235}%
\pgfsetfillcolor{currentfill}%
\pgfsetlinewidth{0.000000pt}%
\definecolor{currentstroke}{rgb}{1.000000,0.894118,0.788235}%
\pgfsetstrokecolor{currentstroke}%
\pgfsetdash{}{0pt}%
\pgfpathmoveto{\pgfqpoint{3.223996in}{3.505892in}}%
\pgfpathlineto{\pgfqpoint{3.334893in}{3.441865in}}%
\pgfpathlineto{\pgfqpoint{3.310017in}{3.461366in}}%
\pgfpathlineto{\pgfqpoint{3.199120in}{3.525392in}}%
\pgfpathlineto{\pgfqpoint{3.223996in}{3.505892in}}%
\pgfpathclose%
\pgfusepath{fill}%
\end{pgfscope}%
\begin{pgfscope}%
\pgfpathrectangle{\pgfqpoint{0.765000in}{0.660000in}}{\pgfqpoint{4.620000in}{4.620000in}}%
\pgfusepath{clip}%
\pgfsetbuttcap%
\pgfsetroundjoin%
\definecolor{currentfill}{rgb}{1.000000,0.894118,0.788235}%
\pgfsetfillcolor{currentfill}%
\pgfsetlinewidth{0.000000pt}%
\definecolor{currentstroke}{rgb}{1.000000,0.894118,0.788235}%
\pgfsetstrokecolor{currentstroke}%
\pgfsetdash{}{0pt}%
\pgfpathmoveto{\pgfqpoint{3.334893in}{3.441865in}}%
\pgfpathlineto{\pgfqpoint{3.223996in}{3.377839in}}%
\pgfpathlineto{\pgfqpoint{3.334893in}{3.313813in}}%
\pgfpathlineto{\pgfqpoint{3.445790in}{3.377839in}}%
\pgfpathlineto{\pgfqpoint{3.334893in}{3.441865in}}%
\pgfpathclose%
\pgfusepath{fill}%
\end{pgfscope}%
\begin{pgfscope}%
\pgfpathrectangle{\pgfqpoint{0.765000in}{0.660000in}}{\pgfqpoint{4.620000in}{4.620000in}}%
\pgfusepath{clip}%
\pgfsetbuttcap%
\pgfsetroundjoin%
\definecolor{currentfill}{rgb}{1.000000,0.894118,0.788235}%
\pgfsetfillcolor{currentfill}%
\pgfsetlinewidth{0.000000pt}%
\definecolor{currentstroke}{rgb}{1.000000,0.894118,0.788235}%
\pgfsetstrokecolor{currentstroke}%
\pgfsetdash{}{0pt}%
\pgfpathmoveto{\pgfqpoint{3.334893in}{3.441865in}}%
\pgfpathlineto{\pgfqpoint{3.223996in}{3.377839in}}%
\pgfpathlineto{\pgfqpoint{3.223996in}{3.505892in}}%
\pgfpathlineto{\pgfqpoint{3.334893in}{3.569918in}}%
\pgfpathlineto{\pgfqpoint{3.334893in}{3.441865in}}%
\pgfpathclose%
\pgfusepath{fill}%
\end{pgfscope}%
\begin{pgfscope}%
\pgfpathrectangle{\pgfqpoint{0.765000in}{0.660000in}}{\pgfqpoint{4.620000in}{4.620000in}}%
\pgfusepath{clip}%
\pgfsetbuttcap%
\pgfsetroundjoin%
\definecolor{currentfill}{rgb}{1.000000,0.894118,0.788235}%
\pgfsetfillcolor{currentfill}%
\pgfsetlinewidth{0.000000pt}%
\definecolor{currentstroke}{rgb}{1.000000,0.894118,0.788235}%
\pgfsetstrokecolor{currentstroke}%
\pgfsetdash{}{0pt}%
\pgfpathmoveto{\pgfqpoint{3.334893in}{3.441865in}}%
\pgfpathlineto{\pgfqpoint{3.445790in}{3.377839in}}%
\pgfpathlineto{\pgfqpoint{3.445790in}{3.505892in}}%
\pgfpathlineto{\pgfqpoint{3.334893in}{3.569918in}}%
\pgfpathlineto{\pgfqpoint{3.334893in}{3.441865in}}%
\pgfpathclose%
\pgfusepath{fill}%
\end{pgfscope}%
\begin{pgfscope}%
\pgfpathrectangle{\pgfqpoint{0.765000in}{0.660000in}}{\pgfqpoint{4.620000in}{4.620000in}}%
\pgfusepath{clip}%
\pgfsetbuttcap%
\pgfsetroundjoin%
\definecolor{currentfill}{rgb}{1.000000,0.894118,0.788235}%
\pgfsetfillcolor{currentfill}%
\pgfsetlinewidth{0.000000pt}%
\definecolor{currentstroke}{rgb}{1.000000,0.894118,0.788235}%
\pgfsetstrokecolor{currentstroke}%
\pgfsetdash{}{0pt}%
\pgfpathmoveto{\pgfqpoint{3.350280in}{3.424899in}}%
\pgfpathlineto{\pgfqpoint{3.239384in}{3.360872in}}%
\pgfpathlineto{\pgfqpoint{3.350280in}{3.296846in}}%
\pgfpathlineto{\pgfqpoint{3.461177in}{3.360872in}}%
\pgfpathlineto{\pgfqpoint{3.350280in}{3.424899in}}%
\pgfpathclose%
\pgfusepath{fill}%
\end{pgfscope}%
\begin{pgfscope}%
\pgfpathrectangle{\pgfqpoint{0.765000in}{0.660000in}}{\pgfqpoint{4.620000in}{4.620000in}}%
\pgfusepath{clip}%
\pgfsetbuttcap%
\pgfsetroundjoin%
\definecolor{currentfill}{rgb}{1.000000,0.894118,0.788235}%
\pgfsetfillcolor{currentfill}%
\pgfsetlinewidth{0.000000pt}%
\definecolor{currentstroke}{rgb}{1.000000,0.894118,0.788235}%
\pgfsetstrokecolor{currentstroke}%
\pgfsetdash{}{0pt}%
\pgfpathmoveto{\pgfqpoint{3.350280in}{3.424899in}}%
\pgfpathlineto{\pgfqpoint{3.239384in}{3.360872in}}%
\pgfpathlineto{\pgfqpoint{3.239384in}{3.488925in}}%
\pgfpathlineto{\pgfqpoint{3.350280in}{3.552951in}}%
\pgfpathlineto{\pgfqpoint{3.350280in}{3.424899in}}%
\pgfpathclose%
\pgfusepath{fill}%
\end{pgfscope}%
\begin{pgfscope}%
\pgfpathrectangle{\pgfqpoint{0.765000in}{0.660000in}}{\pgfqpoint{4.620000in}{4.620000in}}%
\pgfusepath{clip}%
\pgfsetbuttcap%
\pgfsetroundjoin%
\definecolor{currentfill}{rgb}{1.000000,0.894118,0.788235}%
\pgfsetfillcolor{currentfill}%
\pgfsetlinewidth{0.000000pt}%
\definecolor{currentstroke}{rgb}{1.000000,0.894118,0.788235}%
\pgfsetstrokecolor{currentstroke}%
\pgfsetdash{}{0pt}%
\pgfpathmoveto{\pgfqpoint{3.350280in}{3.424899in}}%
\pgfpathlineto{\pgfqpoint{3.461177in}{3.360872in}}%
\pgfpathlineto{\pgfqpoint{3.461177in}{3.488925in}}%
\pgfpathlineto{\pgfqpoint{3.350280in}{3.552951in}}%
\pgfpathlineto{\pgfqpoint{3.350280in}{3.424899in}}%
\pgfpathclose%
\pgfusepath{fill}%
\end{pgfscope}%
\begin{pgfscope}%
\pgfpathrectangle{\pgfqpoint{0.765000in}{0.660000in}}{\pgfqpoint{4.620000in}{4.620000in}}%
\pgfusepath{clip}%
\pgfsetbuttcap%
\pgfsetroundjoin%
\definecolor{currentfill}{rgb}{1.000000,0.894118,0.788235}%
\pgfsetfillcolor{currentfill}%
\pgfsetlinewidth{0.000000pt}%
\definecolor{currentstroke}{rgb}{1.000000,0.894118,0.788235}%
\pgfsetstrokecolor{currentstroke}%
\pgfsetdash{}{0pt}%
\pgfpathmoveto{\pgfqpoint{3.334893in}{3.569918in}}%
\pgfpathlineto{\pgfqpoint{3.223996in}{3.505892in}}%
\pgfpathlineto{\pgfqpoint{3.334893in}{3.441865in}}%
\pgfpathlineto{\pgfqpoint{3.445790in}{3.505892in}}%
\pgfpathlineto{\pgfqpoint{3.334893in}{3.569918in}}%
\pgfpathclose%
\pgfusepath{fill}%
\end{pgfscope}%
\begin{pgfscope}%
\pgfpathrectangle{\pgfqpoint{0.765000in}{0.660000in}}{\pgfqpoint{4.620000in}{4.620000in}}%
\pgfusepath{clip}%
\pgfsetbuttcap%
\pgfsetroundjoin%
\definecolor{currentfill}{rgb}{1.000000,0.894118,0.788235}%
\pgfsetfillcolor{currentfill}%
\pgfsetlinewidth{0.000000pt}%
\definecolor{currentstroke}{rgb}{1.000000,0.894118,0.788235}%
\pgfsetstrokecolor{currentstroke}%
\pgfsetdash{}{0pt}%
\pgfpathmoveto{\pgfqpoint{3.334893in}{3.313813in}}%
\pgfpathlineto{\pgfqpoint{3.445790in}{3.377839in}}%
\pgfpathlineto{\pgfqpoint{3.445790in}{3.505892in}}%
\pgfpathlineto{\pgfqpoint{3.334893in}{3.441865in}}%
\pgfpathlineto{\pgfqpoint{3.334893in}{3.313813in}}%
\pgfpathclose%
\pgfusepath{fill}%
\end{pgfscope}%
\begin{pgfscope}%
\pgfpathrectangle{\pgfqpoint{0.765000in}{0.660000in}}{\pgfqpoint{4.620000in}{4.620000in}}%
\pgfusepath{clip}%
\pgfsetbuttcap%
\pgfsetroundjoin%
\definecolor{currentfill}{rgb}{1.000000,0.894118,0.788235}%
\pgfsetfillcolor{currentfill}%
\pgfsetlinewidth{0.000000pt}%
\definecolor{currentstroke}{rgb}{1.000000,0.894118,0.788235}%
\pgfsetstrokecolor{currentstroke}%
\pgfsetdash{}{0pt}%
\pgfpathmoveto{\pgfqpoint{3.223996in}{3.377839in}}%
\pgfpathlineto{\pgfqpoint{3.334893in}{3.313813in}}%
\pgfpathlineto{\pgfqpoint{3.334893in}{3.441865in}}%
\pgfpathlineto{\pgfqpoint{3.223996in}{3.505892in}}%
\pgfpathlineto{\pgfqpoint{3.223996in}{3.377839in}}%
\pgfpathclose%
\pgfusepath{fill}%
\end{pgfscope}%
\begin{pgfscope}%
\pgfpathrectangle{\pgfqpoint{0.765000in}{0.660000in}}{\pgfqpoint{4.620000in}{4.620000in}}%
\pgfusepath{clip}%
\pgfsetbuttcap%
\pgfsetroundjoin%
\definecolor{currentfill}{rgb}{1.000000,0.894118,0.788235}%
\pgfsetfillcolor{currentfill}%
\pgfsetlinewidth{0.000000pt}%
\definecolor{currentstroke}{rgb}{1.000000,0.894118,0.788235}%
\pgfsetstrokecolor{currentstroke}%
\pgfsetdash{}{0pt}%
\pgfpathmoveto{\pgfqpoint{3.350280in}{3.552951in}}%
\pgfpathlineto{\pgfqpoint{3.239384in}{3.488925in}}%
\pgfpathlineto{\pgfqpoint{3.350280in}{3.424899in}}%
\pgfpathlineto{\pgfqpoint{3.461177in}{3.488925in}}%
\pgfpathlineto{\pgfqpoint{3.350280in}{3.552951in}}%
\pgfpathclose%
\pgfusepath{fill}%
\end{pgfscope}%
\begin{pgfscope}%
\pgfpathrectangle{\pgfqpoint{0.765000in}{0.660000in}}{\pgfqpoint{4.620000in}{4.620000in}}%
\pgfusepath{clip}%
\pgfsetbuttcap%
\pgfsetroundjoin%
\definecolor{currentfill}{rgb}{1.000000,0.894118,0.788235}%
\pgfsetfillcolor{currentfill}%
\pgfsetlinewidth{0.000000pt}%
\definecolor{currentstroke}{rgb}{1.000000,0.894118,0.788235}%
\pgfsetstrokecolor{currentstroke}%
\pgfsetdash{}{0pt}%
\pgfpathmoveto{\pgfqpoint{3.350280in}{3.296846in}}%
\pgfpathlineto{\pgfqpoint{3.461177in}{3.360872in}}%
\pgfpathlineto{\pgfqpoint{3.461177in}{3.488925in}}%
\pgfpathlineto{\pgfqpoint{3.350280in}{3.424899in}}%
\pgfpathlineto{\pgfqpoint{3.350280in}{3.296846in}}%
\pgfpathclose%
\pgfusepath{fill}%
\end{pgfscope}%
\begin{pgfscope}%
\pgfpathrectangle{\pgfqpoint{0.765000in}{0.660000in}}{\pgfqpoint{4.620000in}{4.620000in}}%
\pgfusepath{clip}%
\pgfsetbuttcap%
\pgfsetroundjoin%
\definecolor{currentfill}{rgb}{1.000000,0.894118,0.788235}%
\pgfsetfillcolor{currentfill}%
\pgfsetlinewidth{0.000000pt}%
\definecolor{currentstroke}{rgb}{1.000000,0.894118,0.788235}%
\pgfsetstrokecolor{currentstroke}%
\pgfsetdash{}{0pt}%
\pgfpathmoveto{\pgfqpoint{3.239384in}{3.360872in}}%
\pgfpathlineto{\pgfqpoint{3.350280in}{3.296846in}}%
\pgfpathlineto{\pgfqpoint{3.350280in}{3.424899in}}%
\pgfpathlineto{\pgfqpoint{3.239384in}{3.488925in}}%
\pgfpathlineto{\pgfqpoint{3.239384in}{3.360872in}}%
\pgfpathclose%
\pgfusepath{fill}%
\end{pgfscope}%
\begin{pgfscope}%
\pgfpathrectangle{\pgfqpoint{0.765000in}{0.660000in}}{\pgfqpoint{4.620000in}{4.620000in}}%
\pgfusepath{clip}%
\pgfsetbuttcap%
\pgfsetroundjoin%
\definecolor{currentfill}{rgb}{1.000000,0.894118,0.788235}%
\pgfsetfillcolor{currentfill}%
\pgfsetlinewidth{0.000000pt}%
\definecolor{currentstroke}{rgb}{1.000000,0.894118,0.788235}%
\pgfsetstrokecolor{currentstroke}%
\pgfsetdash{}{0pt}%
\pgfpathmoveto{\pgfqpoint{3.334893in}{3.441865in}}%
\pgfpathlineto{\pgfqpoint{3.223996in}{3.377839in}}%
\pgfpathlineto{\pgfqpoint{3.239384in}{3.360872in}}%
\pgfpathlineto{\pgfqpoint{3.350280in}{3.424899in}}%
\pgfpathlineto{\pgfqpoint{3.334893in}{3.441865in}}%
\pgfpathclose%
\pgfusepath{fill}%
\end{pgfscope}%
\begin{pgfscope}%
\pgfpathrectangle{\pgfqpoint{0.765000in}{0.660000in}}{\pgfqpoint{4.620000in}{4.620000in}}%
\pgfusepath{clip}%
\pgfsetbuttcap%
\pgfsetroundjoin%
\definecolor{currentfill}{rgb}{1.000000,0.894118,0.788235}%
\pgfsetfillcolor{currentfill}%
\pgfsetlinewidth{0.000000pt}%
\definecolor{currentstroke}{rgb}{1.000000,0.894118,0.788235}%
\pgfsetstrokecolor{currentstroke}%
\pgfsetdash{}{0pt}%
\pgfpathmoveto{\pgfqpoint{3.445790in}{3.377839in}}%
\pgfpathlineto{\pgfqpoint{3.334893in}{3.441865in}}%
\pgfpathlineto{\pgfqpoint{3.350280in}{3.424899in}}%
\pgfpathlineto{\pgfqpoint{3.461177in}{3.360872in}}%
\pgfpathlineto{\pgfqpoint{3.445790in}{3.377839in}}%
\pgfpathclose%
\pgfusepath{fill}%
\end{pgfscope}%
\begin{pgfscope}%
\pgfpathrectangle{\pgfqpoint{0.765000in}{0.660000in}}{\pgfqpoint{4.620000in}{4.620000in}}%
\pgfusepath{clip}%
\pgfsetbuttcap%
\pgfsetroundjoin%
\definecolor{currentfill}{rgb}{1.000000,0.894118,0.788235}%
\pgfsetfillcolor{currentfill}%
\pgfsetlinewidth{0.000000pt}%
\definecolor{currentstroke}{rgb}{1.000000,0.894118,0.788235}%
\pgfsetstrokecolor{currentstroke}%
\pgfsetdash{}{0pt}%
\pgfpathmoveto{\pgfqpoint{3.334893in}{3.441865in}}%
\pgfpathlineto{\pgfqpoint{3.334893in}{3.569918in}}%
\pgfpathlineto{\pgfqpoint{3.350280in}{3.552951in}}%
\pgfpathlineto{\pgfqpoint{3.461177in}{3.360872in}}%
\pgfpathlineto{\pgfqpoint{3.334893in}{3.441865in}}%
\pgfpathclose%
\pgfusepath{fill}%
\end{pgfscope}%
\begin{pgfscope}%
\pgfpathrectangle{\pgfqpoint{0.765000in}{0.660000in}}{\pgfqpoint{4.620000in}{4.620000in}}%
\pgfusepath{clip}%
\pgfsetbuttcap%
\pgfsetroundjoin%
\definecolor{currentfill}{rgb}{1.000000,0.894118,0.788235}%
\pgfsetfillcolor{currentfill}%
\pgfsetlinewidth{0.000000pt}%
\definecolor{currentstroke}{rgb}{1.000000,0.894118,0.788235}%
\pgfsetstrokecolor{currentstroke}%
\pgfsetdash{}{0pt}%
\pgfpathmoveto{\pgfqpoint{3.445790in}{3.377839in}}%
\pgfpathlineto{\pgfqpoint{3.445790in}{3.505892in}}%
\pgfpathlineto{\pgfqpoint{3.461177in}{3.488925in}}%
\pgfpathlineto{\pgfqpoint{3.350280in}{3.424899in}}%
\pgfpathlineto{\pgfqpoint{3.445790in}{3.377839in}}%
\pgfpathclose%
\pgfusepath{fill}%
\end{pgfscope}%
\begin{pgfscope}%
\pgfpathrectangle{\pgfqpoint{0.765000in}{0.660000in}}{\pgfqpoint{4.620000in}{4.620000in}}%
\pgfusepath{clip}%
\pgfsetbuttcap%
\pgfsetroundjoin%
\definecolor{currentfill}{rgb}{1.000000,0.894118,0.788235}%
\pgfsetfillcolor{currentfill}%
\pgfsetlinewidth{0.000000pt}%
\definecolor{currentstroke}{rgb}{1.000000,0.894118,0.788235}%
\pgfsetstrokecolor{currentstroke}%
\pgfsetdash{}{0pt}%
\pgfpathmoveto{\pgfqpoint{3.223996in}{3.377839in}}%
\pgfpathlineto{\pgfqpoint{3.334893in}{3.313813in}}%
\pgfpathlineto{\pgfqpoint{3.350280in}{3.296846in}}%
\pgfpathlineto{\pgfqpoint{3.239384in}{3.360872in}}%
\pgfpathlineto{\pgfqpoint{3.223996in}{3.377839in}}%
\pgfpathclose%
\pgfusepath{fill}%
\end{pgfscope}%
\begin{pgfscope}%
\pgfpathrectangle{\pgfqpoint{0.765000in}{0.660000in}}{\pgfqpoint{4.620000in}{4.620000in}}%
\pgfusepath{clip}%
\pgfsetbuttcap%
\pgfsetroundjoin%
\definecolor{currentfill}{rgb}{1.000000,0.894118,0.788235}%
\pgfsetfillcolor{currentfill}%
\pgfsetlinewidth{0.000000pt}%
\definecolor{currentstroke}{rgb}{1.000000,0.894118,0.788235}%
\pgfsetstrokecolor{currentstroke}%
\pgfsetdash{}{0pt}%
\pgfpathmoveto{\pgfqpoint{3.334893in}{3.313813in}}%
\pgfpathlineto{\pgfqpoint{3.445790in}{3.377839in}}%
\pgfpathlineto{\pgfqpoint{3.461177in}{3.360872in}}%
\pgfpathlineto{\pgfqpoint{3.350280in}{3.296846in}}%
\pgfpathlineto{\pgfqpoint{3.334893in}{3.313813in}}%
\pgfpathclose%
\pgfusepath{fill}%
\end{pgfscope}%
\begin{pgfscope}%
\pgfpathrectangle{\pgfqpoint{0.765000in}{0.660000in}}{\pgfqpoint{4.620000in}{4.620000in}}%
\pgfusepath{clip}%
\pgfsetbuttcap%
\pgfsetroundjoin%
\definecolor{currentfill}{rgb}{1.000000,0.894118,0.788235}%
\pgfsetfillcolor{currentfill}%
\pgfsetlinewidth{0.000000pt}%
\definecolor{currentstroke}{rgb}{1.000000,0.894118,0.788235}%
\pgfsetstrokecolor{currentstroke}%
\pgfsetdash{}{0pt}%
\pgfpathmoveto{\pgfqpoint{3.334893in}{3.569918in}}%
\pgfpathlineto{\pgfqpoint{3.223996in}{3.505892in}}%
\pgfpathlineto{\pgfqpoint{3.239384in}{3.488925in}}%
\pgfpathlineto{\pgfqpoint{3.350280in}{3.552951in}}%
\pgfpathlineto{\pgfqpoint{3.334893in}{3.569918in}}%
\pgfpathclose%
\pgfusepath{fill}%
\end{pgfscope}%
\begin{pgfscope}%
\pgfpathrectangle{\pgfqpoint{0.765000in}{0.660000in}}{\pgfqpoint{4.620000in}{4.620000in}}%
\pgfusepath{clip}%
\pgfsetbuttcap%
\pgfsetroundjoin%
\definecolor{currentfill}{rgb}{1.000000,0.894118,0.788235}%
\pgfsetfillcolor{currentfill}%
\pgfsetlinewidth{0.000000pt}%
\definecolor{currentstroke}{rgb}{1.000000,0.894118,0.788235}%
\pgfsetstrokecolor{currentstroke}%
\pgfsetdash{}{0pt}%
\pgfpathmoveto{\pgfqpoint{3.445790in}{3.505892in}}%
\pgfpathlineto{\pgfqpoint{3.334893in}{3.569918in}}%
\pgfpathlineto{\pgfqpoint{3.350280in}{3.552951in}}%
\pgfpathlineto{\pgfqpoint{3.461177in}{3.488925in}}%
\pgfpathlineto{\pgfqpoint{3.445790in}{3.505892in}}%
\pgfpathclose%
\pgfusepath{fill}%
\end{pgfscope}%
\begin{pgfscope}%
\pgfpathrectangle{\pgfqpoint{0.765000in}{0.660000in}}{\pgfqpoint{4.620000in}{4.620000in}}%
\pgfusepath{clip}%
\pgfsetbuttcap%
\pgfsetroundjoin%
\definecolor{currentfill}{rgb}{1.000000,0.894118,0.788235}%
\pgfsetfillcolor{currentfill}%
\pgfsetlinewidth{0.000000pt}%
\definecolor{currentstroke}{rgb}{1.000000,0.894118,0.788235}%
\pgfsetstrokecolor{currentstroke}%
\pgfsetdash{}{0pt}%
\pgfpathmoveto{\pgfqpoint{3.223996in}{3.377839in}}%
\pgfpathlineto{\pgfqpoint{3.223996in}{3.505892in}}%
\pgfpathlineto{\pgfqpoint{3.239384in}{3.488925in}}%
\pgfpathlineto{\pgfqpoint{3.350280in}{3.296846in}}%
\pgfpathlineto{\pgfqpoint{3.223996in}{3.377839in}}%
\pgfpathclose%
\pgfusepath{fill}%
\end{pgfscope}%
\begin{pgfscope}%
\pgfpathrectangle{\pgfqpoint{0.765000in}{0.660000in}}{\pgfqpoint{4.620000in}{4.620000in}}%
\pgfusepath{clip}%
\pgfsetbuttcap%
\pgfsetroundjoin%
\definecolor{currentfill}{rgb}{1.000000,0.894118,0.788235}%
\pgfsetfillcolor{currentfill}%
\pgfsetlinewidth{0.000000pt}%
\definecolor{currentstroke}{rgb}{1.000000,0.894118,0.788235}%
\pgfsetstrokecolor{currentstroke}%
\pgfsetdash{}{0pt}%
\pgfpathmoveto{\pgfqpoint{3.334893in}{3.313813in}}%
\pgfpathlineto{\pgfqpoint{3.334893in}{3.441865in}}%
\pgfpathlineto{\pgfqpoint{3.350280in}{3.424899in}}%
\pgfpathlineto{\pgfqpoint{3.239384in}{3.360872in}}%
\pgfpathlineto{\pgfqpoint{3.334893in}{3.313813in}}%
\pgfpathclose%
\pgfusepath{fill}%
\end{pgfscope}%
\begin{pgfscope}%
\pgfpathrectangle{\pgfqpoint{0.765000in}{0.660000in}}{\pgfqpoint{4.620000in}{4.620000in}}%
\pgfusepath{clip}%
\pgfsetbuttcap%
\pgfsetroundjoin%
\definecolor{currentfill}{rgb}{1.000000,0.894118,0.788235}%
\pgfsetfillcolor{currentfill}%
\pgfsetlinewidth{0.000000pt}%
\definecolor{currentstroke}{rgb}{1.000000,0.894118,0.788235}%
\pgfsetstrokecolor{currentstroke}%
\pgfsetdash{}{0pt}%
\pgfpathmoveto{\pgfqpoint{3.334893in}{3.441865in}}%
\pgfpathlineto{\pgfqpoint{3.445790in}{3.505892in}}%
\pgfpathlineto{\pgfqpoint{3.461177in}{3.488925in}}%
\pgfpathlineto{\pgfqpoint{3.350280in}{3.424899in}}%
\pgfpathlineto{\pgfqpoint{3.334893in}{3.441865in}}%
\pgfpathclose%
\pgfusepath{fill}%
\end{pgfscope}%
\begin{pgfscope}%
\pgfpathrectangle{\pgfqpoint{0.765000in}{0.660000in}}{\pgfqpoint{4.620000in}{4.620000in}}%
\pgfusepath{clip}%
\pgfsetbuttcap%
\pgfsetroundjoin%
\definecolor{currentfill}{rgb}{1.000000,0.894118,0.788235}%
\pgfsetfillcolor{currentfill}%
\pgfsetlinewidth{0.000000pt}%
\definecolor{currentstroke}{rgb}{1.000000,0.894118,0.788235}%
\pgfsetstrokecolor{currentstroke}%
\pgfsetdash{}{0pt}%
\pgfpathmoveto{\pgfqpoint{3.223996in}{3.505892in}}%
\pgfpathlineto{\pgfqpoint{3.334893in}{3.441865in}}%
\pgfpathlineto{\pgfqpoint{3.350280in}{3.424899in}}%
\pgfpathlineto{\pgfqpoint{3.239384in}{3.488925in}}%
\pgfpathlineto{\pgfqpoint{3.223996in}{3.505892in}}%
\pgfpathclose%
\pgfusepath{fill}%
\end{pgfscope}%
\begin{pgfscope}%
\pgfpathrectangle{\pgfqpoint{0.765000in}{0.660000in}}{\pgfqpoint{4.620000in}{4.620000in}}%
\pgfusepath{clip}%
\pgfsetbuttcap%
\pgfsetroundjoin%
\definecolor{currentfill}{rgb}{1.000000,0.894118,0.788235}%
\pgfsetfillcolor{currentfill}%
\pgfsetlinewidth{0.000000pt}%
\definecolor{currentstroke}{rgb}{1.000000,0.894118,0.788235}%
\pgfsetstrokecolor{currentstroke}%
\pgfsetdash{}{0pt}%
\pgfpathmoveto{\pgfqpoint{3.310017in}{3.461366in}}%
\pgfpathlineto{\pgfqpoint{3.199120in}{3.397340in}}%
\pgfpathlineto{\pgfqpoint{3.310017in}{3.333313in}}%
\pgfpathlineto{\pgfqpoint{3.420914in}{3.397340in}}%
\pgfpathlineto{\pgfqpoint{3.310017in}{3.461366in}}%
\pgfpathclose%
\pgfusepath{fill}%
\end{pgfscope}%
\begin{pgfscope}%
\pgfpathrectangle{\pgfqpoint{0.765000in}{0.660000in}}{\pgfqpoint{4.620000in}{4.620000in}}%
\pgfusepath{clip}%
\pgfsetbuttcap%
\pgfsetroundjoin%
\definecolor{currentfill}{rgb}{1.000000,0.894118,0.788235}%
\pgfsetfillcolor{currentfill}%
\pgfsetlinewidth{0.000000pt}%
\definecolor{currentstroke}{rgb}{1.000000,0.894118,0.788235}%
\pgfsetstrokecolor{currentstroke}%
\pgfsetdash{}{0pt}%
\pgfpathmoveto{\pgfqpoint{3.310017in}{3.461366in}}%
\pgfpathlineto{\pgfqpoint{3.199120in}{3.397340in}}%
\pgfpathlineto{\pgfqpoint{3.199120in}{3.525392in}}%
\pgfpathlineto{\pgfqpoint{3.310017in}{3.589418in}}%
\pgfpathlineto{\pgfqpoint{3.310017in}{3.461366in}}%
\pgfpathclose%
\pgfusepath{fill}%
\end{pgfscope}%
\begin{pgfscope}%
\pgfpathrectangle{\pgfqpoint{0.765000in}{0.660000in}}{\pgfqpoint{4.620000in}{4.620000in}}%
\pgfusepath{clip}%
\pgfsetbuttcap%
\pgfsetroundjoin%
\definecolor{currentfill}{rgb}{1.000000,0.894118,0.788235}%
\pgfsetfillcolor{currentfill}%
\pgfsetlinewidth{0.000000pt}%
\definecolor{currentstroke}{rgb}{1.000000,0.894118,0.788235}%
\pgfsetstrokecolor{currentstroke}%
\pgfsetdash{}{0pt}%
\pgfpathmoveto{\pgfqpoint{3.310017in}{3.461366in}}%
\pgfpathlineto{\pgfqpoint{3.420914in}{3.397340in}}%
\pgfpathlineto{\pgfqpoint{3.420914in}{3.525392in}}%
\pgfpathlineto{\pgfqpoint{3.310017in}{3.589418in}}%
\pgfpathlineto{\pgfqpoint{3.310017in}{3.461366in}}%
\pgfpathclose%
\pgfusepath{fill}%
\end{pgfscope}%
\begin{pgfscope}%
\pgfpathrectangle{\pgfqpoint{0.765000in}{0.660000in}}{\pgfqpoint{4.620000in}{4.620000in}}%
\pgfusepath{clip}%
\pgfsetbuttcap%
\pgfsetroundjoin%
\definecolor{currentfill}{rgb}{1.000000,0.894118,0.788235}%
\pgfsetfillcolor{currentfill}%
\pgfsetlinewidth{0.000000pt}%
\definecolor{currentstroke}{rgb}{1.000000,0.894118,0.788235}%
\pgfsetstrokecolor{currentstroke}%
\pgfsetdash{}{0pt}%
\pgfpathmoveto{\pgfqpoint{3.296975in}{3.470058in}}%
\pgfpathlineto{\pgfqpoint{3.186079in}{3.406032in}}%
\pgfpathlineto{\pgfqpoint{3.296975in}{3.342005in}}%
\pgfpathlineto{\pgfqpoint{3.407872in}{3.406032in}}%
\pgfpathlineto{\pgfqpoint{3.296975in}{3.470058in}}%
\pgfpathclose%
\pgfusepath{fill}%
\end{pgfscope}%
\begin{pgfscope}%
\pgfpathrectangle{\pgfqpoint{0.765000in}{0.660000in}}{\pgfqpoint{4.620000in}{4.620000in}}%
\pgfusepath{clip}%
\pgfsetbuttcap%
\pgfsetroundjoin%
\definecolor{currentfill}{rgb}{1.000000,0.894118,0.788235}%
\pgfsetfillcolor{currentfill}%
\pgfsetlinewidth{0.000000pt}%
\definecolor{currentstroke}{rgb}{1.000000,0.894118,0.788235}%
\pgfsetstrokecolor{currentstroke}%
\pgfsetdash{}{0pt}%
\pgfpathmoveto{\pgfqpoint{3.296975in}{3.470058in}}%
\pgfpathlineto{\pgfqpoint{3.186079in}{3.406032in}}%
\pgfpathlineto{\pgfqpoint{3.186079in}{3.534084in}}%
\pgfpathlineto{\pgfqpoint{3.296975in}{3.598110in}}%
\pgfpathlineto{\pgfqpoint{3.296975in}{3.470058in}}%
\pgfpathclose%
\pgfusepath{fill}%
\end{pgfscope}%
\begin{pgfscope}%
\pgfpathrectangle{\pgfqpoint{0.765000in}{0.660000in}}{\pgfqpoint{4.620000in}{4.620000in}}%
\pgfusepath{clip}%
\pgfsetbuttcap%
\pgfsetroundjoin%
\definecolor{currentfill}{rgb}{1.000000,0.894118,0.788235}%
\pgfsetfillcolor{currentfill}%
\pgfsetlinewidth{0.000000pt}%
\definecolor{currentstroke}{rgb}{1.000000,0.894118,0.788235}%
\pgfsetstrokecolor{currentstroke}%
\pgfsetdash{}{0pt}%
\pgfpathmoveto{\pgfqpoint{3.296975in}{3.470058in}}%
\pgfpathlineto{\pgfqpoint{3.407872in}{3.406032in}}%
\pgfpathlineto{\pgfqpoint{3.407872in}{3.534084in}}%
\pgfpathlineto{\pgfqpoint{3.296975in}{3.598110in}}%
\pgfpathlineto{\pgfqpoint{3.296975in}{3.470058in}}%
\pgfpathclose%
\pgfusepath{fill}%
\end{pgfscope}%
\begin{pgfscope}%
\pgfpathrectangle{\pgfqpoint{0.765000in}{0.660000in}}{\pgfqpoint{4.620000in}{4.620000in}}%
\pgfusepath{clip}%
\pgfsetbuttcap%
\pgfsetroundjoin%
\definecolor{currentfill}{rgb}{1.000000,0.894118,0.788235}%
\pgfsetfillcolor{currentfill}%
\pgfsetlinewidth{0.000000pt}%
\definecolor{currentstroke}{rgb}{1.000000,0.894118,0.788235}%
\pgfsetstrokecolor{currentstroke}%
\pgfsetdash{}{0pt}%
\pgfpathmoveto{\pgfqpoint{3.310017in}{3.589418in}}%
\pgfpathlineto{\pgfqpoint{3.199120in}{3.525392in}}%
\pgfpathlineto{\pgfqpoint{3.310017in}{3.461366in}}%
\pgfpathlineto{\pgfqpoint{3.420914in}{3.525392in}}%
\pgfpathlineto{\pgfqpoint{3.310017in}{3.589418in}}%
\pgfpathclose%
\pgfusepath{fill}%
\end{pgfscope}%
\begin{pgfscope}%
\pgfpathrectangle{\pgfqpoint{0.765000in}{0.660000in}}{\pgfqpoint{4.620000in}{4.620000in}}%
\pgfusepath{clip}%
\pgfsetbuttcap%
\pgfsetroundjoin%
\definecolor{currentfill}{rgb}{1.000000,0.894118,0.788235}%
\pgfsetfillcolor{currentfill}%
\pgfsetlinewidth{0.000000pt}%
\definecolor{currentstroke}{rgb}{1.000000,0.894118,0.788235}%
\pgfsetstrokecolor{currentstroke}%
\pgfsetdash{}{0pt}%
\pgfpathmoveto{\pgfqpoint{3.310017in}{3.333313in}}%
\pgfpathlineto{\pgfqpoint{3.420914in}{3.397340in}}%
\pgfpathlineto{\pgfqpoint{3.420914in}{3.525392in}}%
\pgfpathlineto{\pgfqpoint{3.310017in}{3.461366in}}%
\pgfpathlineto{\pgfqpoint{3.310017in}{3.333313in}}%
\pgfpathclose%
\pgfusepath{fill}%
\end{pgfscope}%
\begin{pgfscope}%
\pgfpathrectangle{\pgfqpoint{0.765000in}{0.660000in}}{\pgfqpoint{4.620000in}{4.620000in}}%
\pgfusepath{clip}%
\pgfsetbuttcap%
\pgfsetroundjoin%
\definecolor{currentfill}{rgb}{1.000000,0.894118,0.788235}%
\pgfsetfillcolor{currentfill}%
\pgfsetlinewidth{0.000000pt}%
\definecolor{currentstroke}{rgb}{1.000000,0.894118,0.788235}%
\pgfsetstrokecolor{currentstroke}%
\pgfsetdash{}{0pt}%
\pgfpathmoveto{\pgfqpoint{3.199120in}{3.397340in}}%
\pgfpathlineto{\pgfqpoint{3.310017in}{3.333313in}}%
\pgfpathlineto{\pgfqpoint{3.310017in}{3.461366in}}%
\pgfpathlineto{\pgfqpoint{3.199120in}{3.525392in}}%
\pgfpathlineto{\pgfqpoint{3.199120in}{3.397340in}}%
\pgfpathclose%
\pgfusepath{fill}%
\end{pgfscope}%
\begin{pgfscope}%
\pgfpathrectangle{\pgfqpoint{0.765000in}{0.660000in}}{\pgfqpoint{4.620000in}{4.620000in}}%
\pgfusepath{clip}%
\pgfsetbuttcap%
\pgfsetroundjoin%
\definecolor{currentfill}{rgb}{1.000000,0.894118,0.788235}%
\pgfsetfillcolor{currentfill}%
\pgfsetlinewidth{0.000000pt}%
\definecolor{currentstroke}{rgb}{1.000000,0.894118,0.788235}%
\pgfsetstrokecolor{currentstroke}%
\pgfsetdash{}{0pt}%
\pgfpathmoveto{\pgfqpoint{3.296975in}{3.598110in}}%
\pgfpathlineto{\pgfqpoint{3.186079in}{3.534084in}}%
\pgfpathlineto{\pgfqpoint{3.296975in}{3.470058in}}%
\pgfpathlineto{\pgfqpoint{3.407872in}{3.534084in}}%
\pgfpathlineto{\pgfqpoint{3.296975in}{3.598110in}}%
\pgfpathclose%
\pgfusepath{fill}%
\end{pgfscope}%
\begin{pgfscope}%
\pgfpathrectangle{\pgfqpoint{0.765000in}{0.660000in}}{\pgfqpoint{4.620000in}{4.620000in}}%
\pgfusepath{clip}%
\pgfsetbuttcap%
\pgfsetroundjoin%
\definecolor{currentfill}{rgb}{1.000000,0.894118,0.788235}%
\pgfsetfillcolor{currentfill}%
\pgfsetlinewidth{0.000000pt}%
\definecolor{currentstroke}{rgb}{1.000000,0.894118,0.788235}%
\pgfsetstrokecolor{currentstroke}%
\pgfsetdash{}{0pt}%
\pgfpathmoveto{\pgfqpoint{3.296975in}{3.342005in}}%
\pgfpathlineto{\pgfqpoint{3.407872in}{3.406032in}}%
\pgfpathlineto{\pgfqpoint{3.407872in}{3.534084in}}%
\pgfpathlineto{\pgfqpoint{3.296975in}{3.470058in}}%
\pgfpathlineto{\pgfqpoint{3.296975in}{3.342005in}}%
\pgfpathclose%
\pgfusepath{fill}%
\end{pgfscope}%
\begin{pgfscope}%
\pgfpathrectangle{\pgfqpoint{0.765000in}{0.660000in}}{\pgfqpoint{4.620000in}{4.620000in}}%
\pgfusepath{clip}%
\pgfsetbuttcap%
\pgfsetroundjoin%
\definecolor{currentfill}{rgb}{1.000000,0.894118,0.788235}%
\pgfsetfillcolor{currentfill}%
\pgfsetlinewidth{0.000000pt}%
\definecolor{currentstroke}{rgb}{1.000000,0.894118,0.788235}%
\pgfsetstrokecolor{currentstroke}%
\pgfsetdash{}{0pt}%
\pgfpathmoveto{\pgfqpoint{3.186079in}{3.406032in}}%
\pgfpathlineto{\pgfqpoint{3.296975in}{3.342005in}}%
\pgfpathlineto{\pgfqpoint{3.296975in}{3.470058in}}%
\pgfpathlineto{\pgfqpoint{3.186079in}{3.534084in}}%
\pgfpathlineto{\pgfqpoint{3.186079in}{3.406032in}}%
\pgfpathclose%
\pgfusepath{fill}%
\end{pgfscope}%
\begin{pgfscope}%
\pgfpathrectangle{\pgfqpoint{0.765000in}{0.660000in}}{\pgfqpoint{4.620000in}{4.620000in}}%
\pgfusepath{clip}%
\pgfsetbuttcap%
\pgfsetroundjoin%
\definecolor{currentfill}{rgb}{1.000000,0.894118,0.788235}%
\pgfsetfillcolor{currentfill}%
\pgfsetlinewidth{0.000000pt}%
\definecolor{currentstroke}{rgb}{1.000000,0.894118,0.788235}%
\pgfsetstrokecolor{currentstroke}%
\pgfsetdash{}{0pt}%
\pgfpathmoveto{\pgfqpoint{3.310017in}{3.461366in}}%
\pgfpathlineto{\pgfqpoint{3.199120in}{3.397340in}}%
\pgfpathlineto{\pgfqpoint{3.186079in}{3.406032in}}%
\pgfpathlineto{\pgfqpoint{3.296975in}{3.470058in}}%
\pgfpathlineto{\pgfqpoint{3.310017in}{3.461366in}}%
\pgfpathclose%
\pgfusepath{fill}%
\end{pgfscope}%
\begin{pgfscope}%
\pgfpathrectangle{\pgfqpoint{0.765000in}{0.660000in}}{\pgfqpoint{4.620000in}{4.620000in}}%
\pgfusepath{clip}%
\pgfsetbuttcap%
\pgfsetroundjoin%
\definecolor{currentfill}{rgb}{1.000000,0.894118,0.788235}%
\pgfsetfillcolor{currentfill}%
\pgfsetlinewidth{0.000000pt}%
\definecolor{currentstroke}{rgb}{1.000000,0.894118,0.788235}%
\pgfsetstrokecolor{currentstroke}%
\pgfsetdash{}{0pt}%
\pgfpathmoveto{\pgfqpoint{3.420914in}{3.397340in}}%
\pgfpathlineto{\pgfqpoint{3.310017in}{3.461366in}}%
\pgfpathlineto{\pgfqpoint{3.296975in}{3.470058in}}%
\pgfpathlineto{\pgfqpoint{3.407872in}{3.406032in}}%
\pgfpathlineto{\pgfqpoint{3.420914in}{3.397340in}}%
\pgfpathclose%
\pgfusepath{fill}%
\end{pgfscope}%
\begin{pgfscope}%
\pgfpathrectangle{\pgfqpoint{0.765000in}{0.660000in}}{\pgfqpoint{4.620000in}{4.620000in}}%
\pgfusepath{clip}%
\pgfsetbuttcap%
\pgfsetroundjoin%
\definecolor{currentfill}{rgb}{1.000000,0.894118,0.788235}%
\pgfsetfillcolor{currentfill}%
\pgfsetlinewidth{0.000000pt}%
\definecolor{currentstroke}{rgb}{1.000000,0.894118,0.788235}%
\pgfsetstrokecolor{currentstroke}%
\pgfsetdash{}{0pt}%
\pgfpathmoveto{\pgfqpoint{3.310017in}{3.461366in}}%
\pgfpathlineto{\pgfqpoint{3.310017in}{3.589418in}}%
\pgfpathlineto{\pgfqpoint{3.296975in}{3.598110in}}%
\pgfpathlineto{\pgfqpoint{3.407872in}{3.406032in}}%
\pgfpathlineto{\pgfqpoint{3.310017in}{3.461366in}}%
\pgfpathclose%
\pgfusepath{fill}%
\end{pgfscope}%
\begin{pgfscope}%
\pgfpathrectangle{\pgfqpoint{0.765000in}{0.660000in}}{\pgfqpoint{4.620000in}{4.620000in}}%
\pgfusepath{clip}%
\pgfsetbuttcap%
\pgfsetroundjoin%
\definecolor{currentfill}{rgb}{1.000000,0.894118,0.788235}%
\pgfsetfillcolor{currentfill}%
\pgfsetlinewidth{0.000000pt}%
\definecolor{currentstroke}{rgb}{1.000000,0.894118,0.788235}%
\pgfsetstrokecolor{currentstroke}%
\pgfsetdash{}{0pt}%
\pgfpathmoveto{\pgfqpoint{3.420914in}{3.397340in}}%
\pgfpathlineto{\pgfqpoint{3.420914in}{3.525392in}}%
\pgfpathlineto{\pgfqpoint{3.407872in}{3.534084in}}%
\pgfpathlineto{\pgfqpoint{3.296975in}{3.470058in}}%
\pgfpathlineto{\pgfqpoint{3.420914in}{3.397340in}}%
\pgfpathclose%
\pgfusepath{fill}%
\end{pgfscope}%
\begin{pgfscope}%
\pgfpathrectangle{\pgfqpoint{0.765000in}{0.660000in}}{\pgfqpoint{4.620000in}{4.620000in}}%
\pgfusepath{clip}%
\pgfsetbuttcap%
\pgfsetroundjoin%
\definecolor{currentfill}{rgb}{1.000000,0.894118,0.788235}%
\pgfsetfillcolor{currentfill}%
\pgfsetlinewidth{0.000000pt}%
\definecolor{currentstroke}{rgb}{1.000000,0.894118,0.788235}%
\pgfsetstrokecolor{currentstroke}%
\pgfsetdash{}{0pt}%
\pgfpathmoveto{\pgfqpoint{3.199120in}{3.397340in}}%
\pgfpathlineto{\pgfqpoint{3.310017in}{3.333313in}}%
\pgfpathlineto{\pgfqpoint{3.296975in}{3.342005in}}%
\pgfpathlineto{\pgfqpoint{3.186079in}{3.406032in}}%
\pgfpathlineto{\pgfqpoint{3.199120in}{3.397340in}}%
\pgfpathclose%
\pgfusepath{fill}%
\end{pgfscope}%
\begin{pgfscope}%
\pgfpathrectangle{\pgfqpoint{0.765000in}{0.660000in}}{\pgfqpoint{4.620000in}{4.620000in}}%
\pgfusepath{clip}%
\pgfsetbuttcap%
\pgfsetroundjoin%
\definecolor{currentfill}{rgb}{1.000000,0.894118,0.788235}%
\pgfsetfillcolor{currentfill}%
\pgfsetlinewidth{0.000000pt}%
\definecolor{currentstroke}{rgb}{1.000000,0.894118,0.788235}%
\pgfsetstrokecolor{currentstroke}%
\pgfsetdash{}{0pt}%
\pgfpathmoveto{\pgfqpoint{3.310017in}{3.333313in}}%
\pgfpathlineto{\pgfqpoint{3.420914in}{3.397340in}}%
\pgfpathlineto{\pgfqpoint{3.407872in}{3.406032in}}%
\pgfpathlineto{\pgfqpoint{3.296975in}{3.342005in}}%
\pgfpathlineto{\pgfqpoint{3.310017in}{3.333313in}}%
\pgfpathclose%
\pgfusepath{fill}%
\end{pgfscope}%
\begin{pgfscope}%
\pgfpathrectangle{\pgfqpoint{0.765000in}{0.660000in}}{\pgfqpoint{4.620000in}{4.620000in}}%
\pgfusepath{clip}%
\pgfsetbuttcap%
\pgfsetroundjoin%
\definecolor{currentfill}{rgb}{1.000000,0.894118,0.788235}%
\pgfsetfillcolor{currentfill}%
\pgfsetlinewidth{0.000000pt}%
\definecolor{currentstroke}{rgb}{1.000000,0.894118,0.788235}%
\pgfsetstrokecolor{currentstroke}%
\pgfsetdash{}{0pt}%
\pgfpathmoveto{\pgfqpoint{3.310017in}{3.589418in}}%
\pgfpathlineto{\pgfqpoint{3.199120in}{3.525392in}}%
\pgfpathlineto{\pgfqpoint{3.186079in}{3.534084in}}%
\pgfpathlineto{\pgfqpoint{3.296975in}{3.598110in}}%
\pgfpathlineto{\pgfqpoint{3.310017in}{3.589418in}}%
\pgfpathclose%
\pgfusepath{fill}%
\end{pgfscope}%
\begin{pgfscope}%
\pgfpathrectangle{\pgfqpoint{0.765000in}{0.660000in}}{\pgfqpoint{4.620000in}{4.620000in}}%
\pgfusepath{clip}%
\pgfsetbuttcap%
\pgfsetroundjoin%
\definecolor{currentfill}{rgb}{1.000000,0.894118,0.788235}%
\pgfsetfillcolor{currentfill}%
\pgfsetlinewidth{0.000000pt}%
\definecolor{currentstroke}{rgb}{1.000000,0.894118,0.788235}%
\pgfsetstrokecolor{currentstroke}%
\pgfsetdash{}{0pt}%
\pgfpathmoveto{\pgfqpoint{3.420914in}{3.525392in}}%
\pgfpathlineto{\pgfqpoint{3.310017in}{3.589418in}}%
\pgfpathlineto{\pgfqpoint{3.296975in}{3.598110in}}%
\pgfpathlineto{\pgfqpoint{3.407872in}{3.534084in}}%
\pgfpathlineto{\pgfqpoint{3.420914in}{3.525392in}}%
\pgfpathclose%
\pgfusepath{fill}%
\end{pgfscope}%
\begin{pgfscope}%
\pgfpathrectangle{\pgfqpoint{0.765000in}{0.660000in}}{\pgfqpoint{4.620000in}{4.620000in}}%
\pgfusepath{clip}%
\pgfsetbuttcap%
\pgfsetroundjoin%
\definecolor{currentfill}{rgb}{1.000000,0.894118,0.788235}%
\pgfsetfillcolor{currentfill}%
\pgfsetlinewidth{0.000000pt}%
\definecolor{currentstroke}{rgb}{1.000000,0.894118,0.788235}%
\pgfsetstrokecolor{currentstroke}%
\pgfsetdash{}{0pt}%
\pgfpathmoveto{\pgfqpoint{3.199120in}{3.397340in}}%
\pgfpathlineto{\pgfqpoint{3.199120in}{3.525392in}}%
\pgfpathlineto{\pgfqpoint{3.186079in}{3.534084in}}%
\pgfpathlineto{\pgfqpoint{3.296975in}{3.342005in}}%
\pgfpathlineto{\pgfqpoint{3.199120in}{3.397340in}}%
\pgfpathclose%
\pgfusepath{fill}%
\end{pgfscope}%
\begin{pgfscope}%
\pgfpathrectangle{\pgfqpoint{0.765000in}{0.660000in}}{\pgfqpoint{4.620000in}{4.620000in}}%
\pgfusepath{clip}%
\pgfsetbuttcap%
\pgfsetroundjoin%
\definecolor{currentfill}{rgb}{1.000000,0.894118,0.788235}%
\pgfsetfillcolor{currentfill}%
\pgfsetlinewidth{0.000000pt}%
\definecolor{currentstroke}{rgb}{1.000000,0.894118,0.788235}%
\pgfsetstrokecolor{currentstroke}%
\pgfsetdash{}{0pt}%
\pgfpathmoveto{\pgfqpoint{3.310017in}{3.333313in}}%
\pgfpathlineto{\pgfqpoint{3.310017in}{3.461366in}}%
\pgfpathlineto{\pgfqpoint{3.296975in}{3.470058in}}%
\pgfpathlineto{\pgfqpoint{3.186079in}{3.406032in}}%
\pgfpathlineto{\pgfqpoint{3.310017in}{3.333313in}}%
\pgfpathclose%
\pgfusepath{fill}%
\end{pgfscope}%
\begin{pgfscope}%
\pgfpathrectangle{\pgfqpoint{0.765000in}{0.660000in}}{\pgfqpoint{4.620000in}{4.620000in}}%
\pgfusepath{clip}%
\pgfsetbuttcap%
\pgfsetroundjoin%
\definecolor{currentfill}{rgb}{1.000000,0.894118,0.788235}%
\pgfsetfillcolor{currentfill}%
\pgfsetlinewidth{0.000000pt}%
\definecolor{currentstroke}{rgb}{1.000000,0.894118,0.788235}%
\pgfsetstrokecolor{currentstroke}%
\pgfsetdash{}{0pt}%
\pgfpathmoveto{\pgfqpoint{3.310017in}{3.461366in}}%
\pgfpathlineto{\pgfqpoint{3.420914in}{3.525392in}}%
\pgfpathlineto{\pgfqpoint{3.407872in}{3.534084in}}%
\pgfpathlineto{\pgfqpoint{3.296975in}{3.470058in}}%
\pgfpathlineto{\pgfqpoint{3.310017in}{3.461366in}}%
\pgfpathclose%
\pgfusepath{fill}%
\end{pgfscope}%
\begin{pgfscope}%
\pgfpathrectangle{\pgfqpoint{0.765000in}{0.660000in}}{\pgfqpoint{4.620000in}{4.620000in}}%
\pgfusepath{clip}%
\pgfsetbuttcap%
\pgfsetroundjoin%
\definecolor{currentfill}{rgb}{1.000000,0.894118,0.788235}%
\pgfsetfillcolor{currentfill}%
\pgfsetlinewidth{0.000000pt}%
\definecolor{currentstroke}{rgb}{1.000000,0.894118,0.788235}%
\pgfsetstrokecolor{currentstroke}%
\pgfsetdash{}{0pt}%
\pgfpathmoveto{\pgfqpoint{3.199120in}{3.525392in}}%
\pgfpathlineto{\pgfqpoint{3.310017in}{3.461366in}}%
\pgfpathlineto{\pgfqpoint{3.296975in}{3.470058in}}%
\pgfpathlineto{\pgfqpoint{3.186079in}{3.534084in}}%
\pgfpathlineto{\pgfqpoint{3.199120in}{3.525392in}}%
\pgfpathclose%
\pgfusepath{fill}%
\end{pgfscope}%
\begin{pgfscope}%
\pgfpathrectangle{\pgfqpoint{0.765000in}{0.660000in}}{\pgfqpoint{4.620000in}{4.620000in}}%
\pgfusepath{clip}%
\pgfsetbuttcap%
\pgfsetroundjoin%
\definecolor{currentfill}{rgb}{1.000000,0.894118,0.788235}%
\pgfsetfillcolor{currentfill}%
\pgfsetlinewidth{0.000000pt}%
\definecolor{currentstroke}{rgb}{1.000000,0.894118,0.788235}%
\pgfsetstrokecolor{currentstroke}%
\pgfsetdash{}{0pt}%
\pgfpathmoveto{\pgfqpoint{3.310017in}{3.461366in}}%
\pgfpathlineto{\pgfqpoint{3.199120in}{3.397340in}}%
\pgfpathlineto{\pgfqpoint{3.310017in}{3.333313in}}%
\pgfpathlineto{\pgfqpoint{3.420914in}{3.397340in}}%
\pgfpathlineto{\pgfqpoint{3.310017in}{3.461366in}}%
\pgfpathclose%
\pgfusepath{fill}%
\end{pgfscope}%
\begin{pgfscope}%
\pgfpathrectangle{\pgfqpoint{0.765000in}{0.660000in}}{\pgfqpoint{4.620000in}{4.620000in}}%
\pgfusepath{clip}%
\pgfsetbuttcap%
\pgfsetroundjoin%
\definecolor{currentfill}{rgb}{1.000000,0.894118,0.788235}%
\pgfsetfillcolor{currentfill}%
\pgfsetlinewidth{0.000000pt}%
\definecolor{currentstroke}{rgb}{1.000000,0.894118,0.788235}%
\pgfsetstrokecolor{currentstroke}%
\pgfsetdash{}{0pt}%
\pgfpathmoveto{\pgfqpoint{3.310017in}{3.461366in}}%
\pgfpathlineto{\pgfqpoint{3.199120in}{3.397340in}}%
\pgfpathlineto{\pgfqpoint{3.199120in}{3.525392in}}%
\pgfpathlineto{\pgfqpoint{3.310017in}{3.589418in}}%
\pgfpathlineto{\pgfqpoint{3.310017in}{3.461366in}}%
\pgfpathclose%
\pgfusepath{fill}%
\end{pgfscope}%
\begin{pgfscope}%
\pgfpathrectangle{\pgfqpoint{0.765000in}{0.660000in}}{\pgfqpoint{4.620000in}{4.620000in}}%
\pgfusepath{clip}%
\pgfsetbuttcap%
\pgfsetroundjoin%
\definecolor{currentfill}{rgb}{1.000000,0.894118,0.788235}%
\pgfsetfillcolor{currentfill}%
\pgfsetlinewidth{0.000000pt}%
\definecolor{currentstroke}{rgb}{1.000000,0.894118,0.788235}%
\pgfsetstrokecolor{currentstroke}%
\pgfsetdash{}{0pt}%
\pgfpathmoveto{\pgfqpoint{3.310017in}{3.461366in}}%
\pgfpathlineto{\pgfqpoint{3.420914in}{3.397340in}}%
\pgfpathlineto{\pgfqpoint{3.420914in}{3.525392in}}%
\pgfpathlineto{\pgfqpoint{3.310017in}{3.589418in}}%
\pgfpathlineto{\pgfqpoint{3.310017in}{3.461366in}}%
\pgfpathclose%
\pgfusepath{fill}%
\end{pgfscope}%
\begin{pgfscope}%
\pgfpathrectangle{\pgfqpoint{0.765000in}{0.660000in}}{\pgfqpoint{4.620000in}{4.620000in}}%
\pgfusepath{clip}%
\pgfsetbuttcap%
\pgfsetroundjoin%
\definecolor{currentfill}{rgb}{1.000000,0.894118,0.788235}%
\pgfsetfillcolor{currentfill}%
\pgfsetlinewidth{0.000000pt}%
\definecolor{currentstroke}{rgb}{1.000000,0.894118,0.788235}%
\pgfsetstrokecolor{currentstroke}%
\pgfsetdash{}{0pt}%
\pgfpathmoveto{\pgfqpoint{3.334893in}{3.441865in}}%
\pgfpathlineto{\pgfqpoint{3.223996in}{3.377839in}}%
\pgfpathlineto{\pgfqpoint{3.334893in}{3.313813in}}%
\pgfpathlineto{\pgfqpoint{3.445790in}{3.377839in}}%
\pgfpathlineto{\pgfqpoint{3.334893in}{3.441865in}}%
\pgfpathclose%
\pgfusepath{fill}%
\end{pgfscope}%
\begin{pgfscope}%
\pgfpathrectangle{\pgfqpoint{0.765000in}{0.660000in}}{\pgfqpoint{4.620000in}{4.620000in}}%
\pgfusepath{clip}%
\pgfsetbuttcap%
\pgfsetroundjoin%
\definecolor{currentfill}{rgb}{1.000000,0.894118,0.788235}%
\pgfsetfillcolor{currentfill}%
\pgfsetlinewidth{0.000000pt}%
\definecolor{currentstroke}{rgb}{1.000000,0.894118,0.788235}%
\pgfsetstrokecolor{currentstroke}%
\pgfsetdash{}{0pt}%
\pgfpathmoveto{\pgfqpoint{3.334893in}{3.441865in}}%
\pgfpathlineto{\pgfqpoint{3.223996in}{3.377839in}}%
\pgfpathlineto{\pgfqpoint{3.223996in}{3.505892in}}%
\pgfpathlineto{\pgfqpoint{3.334893in}{3.569918in}}%
\pgfpathlineto{\pgfqpoint{3.334893in}{3.441865in}}%
\pgfpathclose%
\pgfusepath{fill}%
\end{pgfscope}%
\begin{pgfscope}%
\pgfpathrectangle{\pgfqpoint{0.765000in}{0.660000in}}{\pgfqpoint{4.620000in}{4.620000in}}%
\pgfusepath{clip}%
\pgfsetbuttcap%
\pgfsetroundjoin%
\definecolor{currentfill}{rgb}{1.000000,0.894118,0.788235}%
\pgfsetfillcolor{currentfill}%
\pgfsetlinewidth{0.000000pt}%
\definecolor{currentstroke}{rgb}{1.000000,0.894118,0.788235}%
\pgfsetstrokecolor{currentstroke}%
\pgfsetdash{}{0pt}%
\pgfpathmoveto{\pgfqpoint{3.334893in}{3.441865in}}%
\pgfpathlineto{\pgfqpoint{3.445790in}{3.377839in}}%
\pgfpathlineto{\pgfqpoint{3.445790in}{3.505892in}}%
\pgfpathlineto{\pgfqpoint{3.334893in}{3.569918in}}%
\pgfpathlineto{\pgfqpoint{3.334893in}{3.441865in}}%
\pgfpathclose%
\pgfusepath{fill}%
\end{pgfscope}%
\begin{pgfscope}%
\pgfpathrectangle{\pgfqpoint{0.765000in}{0.660000in}}{\pgfqpoint{4.620000in}{4.620000in}}%
\pgfusepath{clip}%
\pgfsetbuttcap%
\pgfsetroundjoin%
\definecolor{currentfill}{rgb}{1.000000,0.894118,0.788235}%
\pgfsetfillcolor{currentfill}%
\pgfsetlinewidth{0.000000pt}%
\definecolor{currentstroke}{rgb}{1.000000,0.894118,0.788235}%
\pgfsetstrokecolor{currentstroke}%
\pgfsetdash{}{0pt}%
\pgfpathmoveto{\pgfqpoint{3.310017in}{3.589418in}}%
\pgfpathlineto{\pgfqpoint{3.199120in}{3.525392in}}%
\pgfpathlineto{\pgfqpoint{3.310017in}{3.461366in}}%
\pgfpathlineto{\pgfqpoint{3.420914in}{3.525392in}}%
\pgfpathlineto{\pgfqpoint{3.310017in}{3.589418in}}%
\pgfpathclose%
\pgfusepath{fill}%
\end{pgfscope}%
\begin{pgfscope}%
\pgfpathrectangle{\pgfqpoint{0.765000in}{0.660000in}}{\pgfqpoint{4.620000in}{4.620000in}}%
\pgfusepath{clip}%
\pgfsetbuttcap%
\pgfsetroundjoin%
\definecolor{currentfill}{rgb}{1.000000,0.894118,0.788235}%
\pgfsetfillcolor{currentfill}%
\pgfsetlinewidth{0.000000pt}%
\definecolor{currentstroke}{rgb}{1.000000,0.894118,0.788235}%
\pgfsetstrokecolor{currentstroke}%
\pgfsetdash{}{0pt}%
\pgfpathmoveto{\pgfqpoint{3.310017in}{3.333313in}}%
\pgfpathlineto{\pgfqpoint{3.420914in}{3.397340in}}%
\pgfpathlineto{\pgfqpoint{3.420914in}{3.525392in}}%
\pgfpathlineto{\pgfqpoint{3.310017in}{3.461366in}}%
\pgfpathlineto{\pgfqpoint{3.310017in}{3.333313in}}%
\pgfpathclose%
\pgfusepath{fill}%
\end{pgfscope}%
\begin{pgfscope}%
\pgfpathrectangle{\pgfqpoint{0.765000in}{0.660000in}}{\pgfqpoint{4.620000in}{4.620000in}}%
\pgfusepath{clip}%
\pgfsetbuttcap%
\pgfsetroundjoin%
\definecolor{currentfill}{rgb}{1.000000,0.894118,0.788235}%
\pgfsetfillcolor{currentfill}%
\pgfsetlinewidth{0.000000pt}%
\definecolor{currentstroke}{rgb}{1.000000,0.894118,0.788235}%
\pgfsetstrokecolor{currentstroke}%
\pgfsetdash{}{0pt}%
\pgfpathmoveto{\pgfqpoint{3.199120in}{3.397340in}}%
\pgfpathlineto{\pgfqpoint{3.310017in}{3.333313in}}%
\pgfpathlineto{\pgfqpoint{3.310017in}{3.461366in}}%
\pgfpathlineto{\pgfqpoint{3.199120in}{3.525392in}}%
\pgfpathlineto{\pgfqpoint{3.199120in}{3.397340in}}%
\pgfpathclose%
\pgfusepath{fill}%
\end{pgfscope}%
\begin{pgfscope}%
\pgfpathrectangle{\pgfqpoint{0.765000in}{0.660000in}}{\pgfqpoint{4.620000in}{4.620000in}}%
\pgfusepath{clip}%
\pgfsetbuttcap%
\pgfsetroundjoin%
\definecolor{currentfill}{rgb}{1.000000,0.894118,0.788235}%
\pgfsetfillcolor{currentfill}%
\pgfsetlinewidth{0.000000pt}%
\definecolor{currentstroke}{rgb}{1.000000,0.894118,0.788235}%
\pgfsetstrokecolor{currentstroke}%
\pgfsetdash{}{0pt}%
\pgfpathmoveto{\pgfqpoint{3.334893in}{3.569918in}}%
\pgfpathlineto{\pgfqpoint{3.223996in}{3.505892in}}%
\pgfpathlineto{\pgfqpoint{3.334893in}{3.441865in}}%
\pgfpathlineto{\pgfqpoint{3.445790in}{3.505892in}}%
\pgfpathlineto{\pgfqpoint{3.334893in}{3.569918in}}%
\pgfpathclose%
\pgfusepath{fill}%
\end{pgfscope}%
\begin{pgfscope}%
\pgfpathrectangle{\pgfqpoint{0.765000in}{0.660000in}}{\pgfqpoint{4.620000in}{4.620000in}}%
\pgfusepath{clip}%
\pgfsetbuttcap%
\pgfsetroundjoin%
\definecolor{currentfill}{rgb}{1.000000,0.894118,0.788235}%
\pgfsetfillcolor{currentfill}%
\pgfsetlinewidth{0.000000pt}%
\definecolor{currentstroke}{rgb}{1.000000,0.894118,0.788235}%
\pgfsetstrokecolor{currentstroke}%
\pgfsetdash{}{0pt}%
\pgfpathmoveto{\pgfqpoint{3.334893in}{3.313813in}}%
\pgfpathlineto{\pgfqpoint{3.445790in}{3.377839in}}%
\pgfpathlineto{\pgfqpoint{3.445790in}{3.505892in}}%
\pgfpathlineto{\pgfqpoint{3.334893in}{3.441865in}}%
\pgfpathlineto{\pgfqpoint{3.334893in}{3.313813in}}%
\pgfpathclose%
\pgfusepath{fill}%
\end{pgfscope}%
\begin{pgfscope}%
\pgfpathrectangle{\pgfqpoint{0.765000in}{0.660000in}}{\pgfqpoint{4.620000in}{4.620000in}}%
\pgfusepath{clip}%
\pgfsetbuttcap%
\pgfsetroundjoin%
\definecolor{currentfill}{rgb}{1.000000,0.894118,0.788235}%
\pgfsetfillcolor{currentfill}%
\pgfsetlinewidth{0.000000pt}%
\definecolor{currentstroke}{rgb}{1.000000,0.894118,0.788235}%
\pgfsetstrokecolor{currentstroke}%
\pgfsetdash{}{0pt}%
\pgfpathmoveto{\pgfqpoint{3.223996in}{3.377839in}}%
\pgfpathlineto{\pgfqpoint{3.334893in}{3.313813in}}%
\pgfpathlineto{\pgfqpoint{3.334893in}{3.441865in}}%
\pgfpathlineto{\pgfqpoint{3.223996in}{3.505892in}}%
\pgfpathlineto{\pgfqpoint{3.223996in}{3.377839in}}%
\pgfpathclose%
\pgfusepath{fill}%
\end{pgfscope}%
\begin{pgfscope}%
\pgfpathrectangle{\pgfqpoint{0.765000in}{0.660000in}}{\pgfqpoint{4.620000in}{4.620000in}}%
\pgfusepath{clip}%
\pgfsetbuttcap%
\pgfsetroundjoin%
\definecolor{currentfill}{rgb}{1.000000,0.894118,0.788235}%
\pgfsetfillcolor{currentfill}%
\pgfsetlinewidth{0.000000pt}%
\definecolor{currentstroke}{rgb}{1.000000,0.894118,0.788235}%
\pgfsetstrokecolor{currentstroke}%
\pgfsetdash{}{0pt}%
\pgfpathmoveto{\pgfqpoint{3.310017in}{3.461366in}}%
\pgfpathlineto{\pgfqpoint{3.199120in}{3.397340in}}%
\pgfpathlineto{\pgfqpoint{3.223996in}{3.377839in}}%
\pgfpathlineto{\pgfqpoint{3.334893in}{3.441865in}}%
\pgfpathlineto{\pgfqpoint{3.310017in}{3.461366in}}%
\pgfpathclose%
\pgfusepath{fill}%
\end{pgfscope}%
\begin{pgfscope}%
\pgfpathrectangle{\pgfqpoint{0.765000in}{0.660000in}}{\pgfqpoint{4.620000in}{4.620000in}}%
\pgfusepath{clip}%
\pgfsetbuttcap%
\pgfsetroundjoin%
\definecolor{currentfill}{rgb}{1.000000,0.894118,0.788235}%
\pgfsetfillcolor{currentfill}%
\pgfsetlinewidth{0.000000pt}%
\definecolor{currentstroke}{rgb}{1.000000,0.894118,0.788235}%
\pgfsetstrokecolor{currentstroke}%
\pgfsetdash{}{0pt}%
\pgfpathmoveto{\pgfqpoint{3.420914in}{3.397340in}}%
\pgfpathlineto{\pgfqpoint{3.310017in}{3.461366in}}%
\pgfpathlineto{\pgfqpoint{3.334893in}{3.441865in}}%
\pgfpathlineto{\pgfqpoint{3.445790in}{3.377839in}}%
\pgfpathlineto{\pgfqpoint{3.420914in}{3.397340in}}%
\pgfpathclose%
\pgfusepath{fill}%
\end{pgfscope}%
\begin{pgfscope}%
\pgfpathrectangle{\pgfqpoint{0.765000in}{0.660000in}}{\pgfqpoint{4.620000in}{4.620000in}}%
\pgfusepath{clip}%
\pgfsetbuttcap%
\pgfsetroundjoin%
\definecolor{currentfill}{rgb}{1.000000,0.894118,0.788235}%
\pgfsetfillcolor{currentfill}%
\pgfsetlinewidth{0.000000pt}%
\definecolor{currentstroke}{rgb}{1.000000,0.894118,0.788235}%
\pgfsetstrokecolor{currentstroke}%
\pgfsetdash{}{0pt}%
\pgfpathmoveto{\pgfqpoint{3.310017in}{3.461366in}}%
\pgfpathlineto{\pgfqpoint{3.310017in}{3.589418in}}%
\pgfpathlineto{\pgfqpoint{3.334893in}{3.569918in}}%
\pgfpathlineto{\pgfqpoint{3.445790in}{3.377839in}}%
\pgfpathlineto{\pgfqpoint{3.310017in}{3.461366in}}%
\pgfpathclose%
\pgfusepath{fill}%
\end{pgfscope}%
\begin{pgfscope}%
\pgfpathrectangle{\pgfqpoint{0.765000in}{0.660000in}}{\pgfqpoint{4.620000in}{4.620000in}}%
\pgfusepath{clip}%
\pgfsetbuttcap%
\pgfsetroundjoin%
\definecolor{currentfill}{rgb}{1.000000,0.894118,0.788235}%
\pgfsetfillcolor{currentfill}%
\pgfsetlinewidth{0.000000pt}%
\definecolor{currentstroke}{rgb}{1.000000,0.894118,0.788235}%
\pgfsetstrokecolor{currentstroke}%
\pgfsetdash{}{0pt}%
\pgfpathmoveto{\pgfqpoint{3.420914in}{3.397340in}}%
\pgfpathlineto{\pgfqpoint{3.420914in}{3.525392in}}%
\pgfpathlineto{\pgfqpoint{3.445790in}{3.505892in}}%
\pgfpathlineto{\pgfqpoint{3.334893in}{3.441865in}}%
\pgfpathlineto{\pgfqpoint{3.420914in}{3.397340in}}%
\pgfpathclose%
\pgfusepath{fill}%
\end{pgfscope}%
\begin{pgfscope}%
\pgfpathrectangle{\pgfqpoint{0.765000in}{0.660000in}}{\pgfqpoint{4.620000in}{4.620000in}}%
\pgfusepath{clip}%
\pgfsetbuttcap%
\pgfsetroundjoin%
\definecolor{currentfill}{rgb}{1.000000,0.894118,0.788235}%
\pgfsetfillcolor{currentfill}%
\pgfsetlinewidth{0.000000pt}%
\definecolor{currentstroke}{rgb}{1.000000,0.894118,0.788235}%
\pgfsetstrokecolor{currentstroke}%
\pgfsetdash{}{0pt}%
\pgfpathmoveto{\pgfqpoint{3.199120in}{3.397340in}}%
\pgfpathlineto{\pgfqpoint{3.310017in}{3.333313in}}%
\pgfpathlineto{\pgfqpoint{3.334893in}{3.313813in}}%
\pgfpathlineto{\pgfqpoint{3.223996in}{3.377839in}}%
\pgfpathlineto{\pgfqpoint{3.199120in}{3.397340in}}%
\pgfpathclose%
\pgfusepath{fill}%
\end{pgfscope}%
\begin{pgfscope}%
\pgfpathrectangle{\pgfqpoint{0.765000in}{0.660000in}}{\pgfqpoint{4.620000in}{4.620000in}}%
\pgfusepath{clip}%
\pgfsetbuttcap%
\pgfsetroundjoin%
\definecolor{currentfill}{rgb}{1.000000,0.894118,0.788235}%
\pgfsetfillcolor{currentfill}%
\pgfsetlinewidth{0.000000pt}%
\definecolor{currentstroke}{rgb}{1.000000,0.894118,0.788235}%
\pgfsetstrokecolor{currentstroke}%
\pgfsetdash{}{0pt}%
\pgfpathmoveto{\pgfqpoint{3.310017in}{3.333313in}}%
\pgfpathlineto{\pgfqpoint{3.420914in}{3.397340in}}%
\pgfpathlineto{\pgfqpoint{3.445790in}{3.377839in}}%
\pgfpathlineto{\pgfqpoint{3.334893in}{3.313813in}}%
\pgfpathlineto{\pgfqpoint{3.310017in}{3.333313in}}%
\pgfpathclose%
\pgfusepath{fill}%
\end{pgfscope}%
\begin{pgfscope}%
\pgfpathrectangle{\pgfqpoint{0.765000in}{0.660000in}}{\pgfqpoint{4.620000in}{4.620000in}}%
\pgfusepath{clip}%
\pgfsetbuttcap%
\pgfsetroundjoin%
\definecolor{currentfill}{rgb}{1.000000,0.894118,0.788235}%
\pgfsetfillcolor{currentfill}%
\pgfsetlinewidth{0.000000pt}%
\definecolor{currentstroke}{rgb}{1.000000,0.894118,0.788235}%
\pgfsetstrokecolor{currentstroke}%
\pgfsetdash{}{0pt}%
\pgfpathmoveto{\pgfqpoint{3.310017in}{3.589418in}}%
\pgfpathlineto{\pgfqpoint{3.199120in}{3.525392in}}%
\pgfpathlineto{\pgfqpoint{3.223996in}{3.505892in}}%
\pgfpathlineto{\pgfqpoint{3.334893in}{3.569918in}}%
\pgfpathlineto{\pgfqpoint{3.310017in}{3.589418in}}%
\pgfpathclose%
\pgfusepath{fill}%
\end{pgfscope}%
\begin{pgfscope}%
\pgfpathrectangle{\pgfqpoint{0.765000in}{0.660000in}}{\pgfqpoint{4.620000in}{4.620000in}}%
\pgfusepath{clip}%
\pgfsetbuttcap%
\pgfsetroundjoin%
\definecolor{currentfill}{rgb}{1.000000,0.894118,0.788235}%
\pgfsetfillcolor{currentfill}%
\pgfsetlinewidth{0.000000pt}%
\definecolor{currentstroke}{rgb}{1.000000,0.894118,0.788235}%
\pgfsetstrokecolor{currentstroke}%
\pgfsetdash{}{0pt}%
\pgfpathmoveto{\pgfqpoint{3.420914in}{3.525392in}}%
\pgfpathlineto{\pgfqpoint{3.310017in}{3.589418in}}%
\pgfpathlineto{\pgfqpoint{3.334893in}{3.569918in}}%
\pgfpathlineto{\pgfqpoint{3.445790in}{3.505892in}}%
\pgfpathlineto{\pgfqpoint{3.420914in}{3.525392in}}%
\pgfpathclose%
\pgfusepath{fill}%
\end{pgfscope}%
\begin{pgfscope}%
\pgfpathrectangle{\pgfqpoint{0.765000in}{0.660000in}}{\pgfqpoint{4.620000in}{4.620000in}}%
\pgfusepath{clip}%
\pgfsetbuttcap%
\pgfsetroundjoin%
\definecolor{currentfill}{rgb}{1.000000,0.894118,0.788235}%
\pgfsetfillcolor{currentfill}%
\pgfsetlinewidth{0.000000pt}%
\definecolor{currentstroke}{rgb}{1.000000,0.894118,0.788235}%
\pgfsetstrokecolor{currentstroke}%
\pgfsetdash{}{0pt}%
\pgfpathmoveto{\pgfqpoint{3.199120in}{3.397340in}}%
\pgfpathlineto{\pgfqpoint{3.199120in}{3.525392in}}%
\pgfpathlineto{\pgfqpoint{3.223996in}{3.505892in}}%
\pgfpathlineto{\pgfqpoint{3.334893in}{3.313813in}}%
\pgfpathlineto{\pgfqpoint{3.199120in}{3.397340in}}%
\pgfpathclose%
\pgfusepath{fill}%
\end{pgfscope}%
\begin{pgfscope}%
\pgfpathrectangle{\pgfqpoint{0.765000in}{0.660000in}}{\pgfqpoint{4.620000in}{4.620000in}}%
\pgfusepath{clip}%
\pgfsetbuttcap%
\pgfsetroundjoin%
\definecolor{currentfill}{rgb}{1.000000,0.894118,0.788235}%
\pgfsetfillcolor{currentfill}%
\pgfsetlinewidth{0.000000pt}%
\definecolor{currentstroke}{rgb}{1.000000,0.894118,0.788235}%
\pgfsetstrokecolor{currentstroke}%
\pgfsetdash{}{0pt}%
\pgfpathmoveto{\pgfqpoint{3.310017in}{3.333313in}}%
\pgfpathlineto{\pgfqpoint{3.310017in}{3.461366in}}%
\pgfpathlineto{\pgfqpoint{3.334893in}{3.441865in}}%
\pgfpathlineto{\pgfqpoint{3.223996in}{3.377839in}}%
\pgfpathlineto{\pgfqpoint{3.310017in}{3.333313in}}%
\pgfpathclose%
\pgfusepath{fill}%
\end{pgfscope}%
\begin{pgfscope}%
\pgfpathrectangle{\pgfqpoint{0.765000in}{0.660000in}}{\pgfqpoint{4.620000in}{4.620000in}}%
\pgfusepath{clip}%
\pgfsetbuttcap%
\pgfsetroundjoin%
\definecolor{currentfill}{rgb}{1.000000,0.894118,0.788235}%
\pgfsetfillcolor{currentfill}%
\pgfsetlinewidth{0.000000pt}%
\definecolor{currentstroke}{rgb}{1.000000,0.894118,0.788235}%
\pgfsetstrokecolor{currentstroke}%
\pgfsetdash{}{0pt}%
\pgfpathmoveto{\pgfqpoint{3.310017in}{3.461366in}}%
\pgfpathlineto{\pgfqpoint{3.420914in}{3.525392in}}%
\pgfpathlineto{\pgfqpoint{3.445790in}{3.505892in}}%
\pgfpathlineto{\pgfqpoint{3.334893in}{3.441865in}}%
\pgfpathlineto{\pgfqpoint{3.310017in}{3.461366in}}%
\pgfpathclose%
\pgfusepath{fill}%
\end{pgfscope}%
\begin{pgfscope}%
\pgfpathrectangle{\pgfqpoint{0.765000in}{0.660000in}}{\pgfqpoint{4.620000in}{4.620000in}}%
\pgfusepath{clip}%
\pgfsetbuttcap%
\pgfsetroundjoin%
\definecolor{currentfill}{rgb}{1.000000,0.894118,0.788235}%
\pgfsetfillcolor{currentfill}%
\pgfsetlinewidth{0.000000pt}%
\definecolor{currentstroke}{rgb}{1.000000,0.894118,0.788235}%
\pgfsetstrokecolor{currentstroke}%
\pgfsetdash{}{0pt}%
\pgfpathmoveto{\pgfqpoint{3.199120in}{3.525392in}}%
\pgfpathlineto{\pgfqpoint{3.310017in}{3.461366in}}%
\pgfpathlineto{\pgfqpoint{3.334893in}{3.441865in}}%
\pgfpathlineto{\pgfqpoint{3.223996in}{3.505892in}}%
\pgfpathlineto{\pgfqpoint{3.199120in}{3.525392in}}%
\pgfpathclose%
\pgfusepath{fill}%
\end{pgfscope}%
\begin{pgfscope}%
\pgfpathrectangle{\pgfqpoint{0.765000in}{0.660000in}}{\pgfqpoint{4.620000in}{4.620000in}}%
\pgfusepath{clip}%
\pgfsetbuttcap%
\pgfsetroundjoin%
\definecolor{currentfill}{rgb}{1.000000,0.894118,0.788235}%
\pgfsetfillcolor{currentfill}%
\pgfsetlinewidth{0.000000pt}%
\definecolor{currentstroke}{rgb}{1.000000,0.894118,0.788235}%
\pgfsetstrokecolor{currentstroke}%
\pgfsetdash{}{0pt}%
\pgfpathmoveto{\pgfqpoint{2.815582in}{2.599184in}}%
\pgfpathlineto{\pgfqpoint{2.704685in}{2.535158in}}%
\pgfpathlineto{\pgfqpoint{2.815582in}{2.471132in}}%
\pgfpathlineto{\pgfqpoint{2.926478in}{2.535158in}}%
\pgfpathlineto{\pgfqpoint{2.815582in}{2.599184in}}%
\pgfpathclose%
\pgfusepath{fill}%
\end{pgfscope}%
\begin{pgfscope}%
\pgfpathrectangle{\pgfqpoint{0.765000in}{0.660000in}}{\pgfqpoint{4.620000in}{4.620000in}}%
\pgfusepath{clip}%
\pgfsetbuttcap%
\pgfsetroundjoin%
\definecolor{currentfill}{rgb}{1.000000,0.894118,0.788235}%
\pgfsetfillcolor{currentfill}%
\pgfsetlinewidth{0.000000pt}%
\definecolor{currentstroke}{rgb}{1.000000,0.894118,0.788235}%
\pgfsetstrokecolor{currentstroke}%
\pgfsetdash{}{0pt}%
\pgfpathmoveto{\pgfqpoint{2.815582in}{2.599184in}}%
\pgfpathlineto{\pgfqpoint{2.704685in}{2.535158in}}%
\pgfpathlineto{\pgfqpoint{2.704685in}{2.663210in}}%
\pgfpathlineto{\pgfqpoint{2.815582in}{2.727237in}}%
\pgfpathlineto{\pgfqpoint{2.815582in}{2.599184in}}%
\pgfpathclose%
\pgfusepath{fill}%
\end{pgfscope}%
\begin{pgfscope}%
\pgfpathrectangle{\pgfqpoint{0.765000in}{0.660000in}}{\pgfqpoint{4.620000in}{4.620000in}}%
\pgfusepath{clip}%
\pgfsetbuttcap%
\pgfsetroundjoin%
\definecolor{currentfill}{rgb}{1.000000,0.894118,0.788235}%
\pgfsetfillcolor{currentfill}%
\pgfsetlinewidth{0.000000pt}%
\definecolor{currentstroke}{rgb}{1.000000,0.894118,0.788235}%
\pgfsetstrokecolor{currentstroke}%
\pgfsetdash{}{0pt}%
\pgfpathmoveto{\pgfqpoint{2.815582in}{2.599184in}}%
\pgfpathlineto{\pgfqpoint{2.926478in}{2.535158in}}%
\pgfpathlineto{\pgfqpoint{2.926478in}{2.663210in}}%
\pgfpathlineto{\pgfqpoint{2.815582in}{2.727237in}}%
\pgfpathlineto{\pgfqpoint{2.815582in}{2.599184in}}%
\pgfpathclose%
\pgfusepath{fill}%
\end{pgfscope}%
\begin{pgfscope}%
\pgfpathrectangle{\pgfqpoint{0.765000in}{0.660000in}}{\pgfqpoint{4.620000in}{4.620000in}}%
\pgfusepath{clip}%
\pgfsetbuttcap%
\pgfsetroundjoin%
\definecolor{currentfill}{rgb}{1.000000,0.894118,0.788235}%
\pgfsetfillcolor{currentfill}%
\pgfsetlinewidth{0.000000pt}%
\definecolor{currentstroke}{rgb}{1.000000,0.894118,0.788235}%
\pgfsetstrokecolor{currentstroke}%
\pgfsetdash{}{0pt}%
\pgfpathmoveto{\pgfqpoint{2.915238in}{2.538405in}}%
\pgfpathlineto{\pgfqpoint{2.804341in}{2.474379in}}%
\pgfpathlineto{\pgfqpoint{2.915238in}{2.410353in}}%
\pgfpathlineto{\pgfqpoint{3.026135in}{2.474379in}}%
\pgfpathlineto{\pgfqpoint{2.915238in}{2.538405in}}%
\pgfpathclose%
\pgfusepath{fill}%
\end{pgfscope}%
\begin{pgfscope}%
\pgfpathrectangle{\pgfqpoint{0.765000in}{0.660000in}}{\pgfqpoint{4.620000in}{4.620000in}}%
\pgfusepath{clip}%
\pgfsetbuttcap%
\pgfsetroundjoin%
\definecolor{currentfill}{rgb}{1.000000,0.894118,0.788235}%
\pgfsetfillcolor{currentfill}%
\pgfsetlinewidth{0.000000pt}%
\definecolor{currentstroke}{rgb}{1.000000,0.894118,0.788235}%
\pgfsetstrokecolor{currentstroke}%
\pgfsetdash{}{0pt}%
\pgfpathmoveto{\pgfqpoint{2.915238in}{2.538405in}}%
\pgfpathlineto{\pgfqpoint{2.804341in}{2.474379in}}%
\pgfpathlineto{\pgfqpoint{2.804341in}{2.602432in}}%
\pgfpathlineto{\pgfqpoint{2.915238in}{2.666458in}}%
\pgfpathlineto{\pgfqpoint{2.915238in}{2.538405in}}%
\pgfpathclose%
\pgfusepath{fill}%
\end{pgfscope}%
\begin{pgfscope}%
\pgfpathrectangle{\pgfqpoint{0.765000in}{0.660000in}}{\pgfqpoint{4.620000in}{4.620000in}}%
\pgfusepath{clip}%
\pgfsetbuttcap%
\pgfsetroundjoin%
\definecolor{currentfill}{rgb}{1.000000,0.894118,0.788235}%
\pgfsetfillcolor{currentfill}%
\pgfsetlinewidth{0.000000pt}%
\definecolor{currentstroke}{rgb}{1.000000,0.894118,0.788235}%
\pgfsetstrokecolor{currentstroke}%
\pgfsetdash{}{0pt}%
\pgfpathmoveto{\pgfqpoint{2.915238in}{2.538405in}}%
\pgfpathlineto{\pgfqpoint{3.026135in}{2.474379in}}%
\pgfpathlineto{\pgfqpoint{3.026135in}{2.602432in}}%
\pgfpathlineto{\pgfqpoint{2.915238in}{2.666458in}}%
\pgfpathlineto{\pgfqpoint{2.915238in}{2.538405in}}%
\pgfpathclose%
\pgfusepath{fill}%
\end{pgfscope}%
\begin{pgfscope}%
\pgfpathrectangle{\pgfqpoint{0.765000in}{0.660000in}}{\pgfqpoint{4.620000in}{4.620000in}}%
\pgfusepath{clip}%
\pgfsetbuttcap%
\pgfsetroundjoin%
\definecolor{currentfill}{rgb}{1.000000,0.894118,0.788235}%
\pgfsetfillcolor{currentfill}%
\pgfsetlinewidth{0.000000pt}%
\definecolor{currentstroke}{rgb}{1.000000,0.894118,0.788235}%
\pgfsetstrokecolor{currentstroke}%
\pgfsetdash{}{0pt}%
\pgfpathmoveto{\pgfqpoint{2.815582in}{2.727237in}}%
\pgfpathlineto{\pgfqpoint{2.704685in}{2.663210in}}%
\pgfpathlineto{\pgfqpoint{2.815582in}{2.599184in}}%
\pgfpathlineto{\pgfqpoint{2.926478in}{2.663210in}}%
\pgfpathlineto{\pgfqpoint{2.815582in}{2.727237in}}%
\pgfpathclose%
\pgfusepath{fill}%
\end{pgfscope}%
\begin{pgfscope}%
\pgfpathrectangle{\pgfqpoint{0.765000in}{0.660000in}}{\pgfqpoint{4.620000in}{4.620000in}}%
\pgfusepath{clip}%
\pgfsetbuttcap%
\pgfsetroundjoin%
\definecolor{currentfill}{rgb}{1.000000,0.894118,0.788235}%
\pgfsetfillcolor{currentfill}%
\pgfsetlinewidth{0.000000pt}%
\definecolor{currentstroke}{rgb}{1.000000,0.894118,0.788235}%
\pgfsetstrokecolor{currentstroke}%
\pgfsetdash{}{0pt}%
\pgfpathmoveto{\pgfqpoint{2.815582in}{2.471132in}}%
\pgfpathlineto{\pgfqpoint{2.926478in}{2.535158in}}%
\pgfpathlineto{\pgfqpoint{2.926478in}{2.663210in}}%
\pgfpathlineto{\pgfqpoint{2.815582in}{2.599184in}}%
\pgfpathlineto{\pgfqpoint{2.815582in}{2.471132in}}%
\pgfpathclose%
\pgfusepath{fill}%
\end{pgfscope}%
\begin{pgfscope}%
\pgfpathrectangle{\pgfqpoint{0.765000in}{0.660000in}}{\pgfqpoint{4.620000in}{4.620000in}}%
\pgfusepath{clip}%
\pgfsetbuttcap%
\pgfsetroundjoin%
\definecolor{currentfill}{rgb}{1.000000,0.894118,0.788235}%
\pgfsetfillcolor{currentfill}%
\pgfsetlinewidth{0.000000pt}%
\definecolor{currentstroke}{rgb}{1.000000,0.894118,0.788235}%
\pgfsetstrokecolor{currentstroke}%
\pgfsetdash{}{0pt}%
\pgfpathmoveto{\pgfqpoint{2.704685in}{2.535158in}}%
\pgfpathlineto{\pgfqpoint{2.815582in}{2.471132in}}%
\pgfpathlineto{\pgfqpoint{2.815582in}{2.599184in}}%
\pgfpathlineto{\pgfqpoint{2.704685in}{2.663210in}}%
\pgfpathlineto{\pgfqpoint{2.704685in}{2.535158in}}%
\pgfpathclose%
\pgfusepath{fill}%
\end{pgfscope}%
\begin{pgfscope}%
\pgfpathrectangle{\pgfqpoint{0.765000in}{0.660000in}}{\pgfqpoint{4.620000in}{4.620000in}}%
\pgfusepath{clip}%
\pgfsetbuttcap%
\pgfsetroundjoin%
\definecolor{currentfill}{rgb}{1.000000,0.894118,0.788235}%
\pgfsetfillcolor{currentfill}%
\pgfsetlinewidth{0.000000pt}%
\definecolor{currentstroke}{rgb}{1.000000,0.894118,0.788235}%
\pgfsetstrokecolor{currentstroke}%
\pgfsetdash{}{0pt}%
\pgfpathmoveto{\pgfqpoint{2.915238in}{2.666458in}}%
\pgfpathlineto{\pgfqpoint{2.804341in}{2.602432in}}%
\pgfpathlineto{\pgfqpoint{2.915238in}{2.538405in}}%
\pgfpathlineto{\pgfqpoint{3.026135in}{2.602432in}}%
\pgfpathlineto{\pgfqpoint{2.915238in}{2.666458in}}%
\pgfpathclose%
\pgfusepath{fill}%
\end{pgfscope}%
\begin{pgfscope}%
\pgfpathrectangle{\pgfqpoint{0.765000in}{0.660000in}}{\pgfqpoint{4.620000in}{4.620000in}}%
\pgfusepath{clip}%
\pgfsetbuttcap%
\pgfsetroundjoin%
\definecolor{currentfill}{rgb}{1.000000,0.894118,0.788235}%
\pgfsetfillcolor{currentfill}%
\pgfsetlinewidth{0.000000pt}%
\definecolor{currentstroke}{rgb}{1.000000,0.894118,0.788235}%
\pgfsetstrokecolor{currentstroke}%
\pgfsetdash{}{0pt}%
\pgfpathmoveto{\pgfqpoint{2.915238in}{2.410353in}}%
\pgfpathlineto{\pgfqpoint{3.026135in}{2.474379in}}%
\pgfpathlineto{\pgfqpoint{3.026135in}{2.602432in}}%
\pgfpathlineto{\pgfqpoint{2.915238in}{2.538405in}}%
\pgfpathlineto{\pgfqpoint{2.915238in}{2.410353in}}%
\pgfpathclose%
\pgfusepath{fill}%
\end{pgfscope}%
\begin{pgfscope}%
\pgfpathrectangle{\pgfqpoint{0.765000in}{0.660000in}}{\pgfqpoint{4.620000in}{4.620000in}}%
\pgfusepath{clip}%
\pgfsetbuttcap%
\pgfsetroundjoin%
\definecolor{currentfill}{rgb}{1.000000,0.894118,0.788235}%
\pgfsetfillcolor{currentfill}%
\pgfsetlinewidth{0.000000pt}%
\definecolor{currentstroke}{rgb}{1.000000,0.894118,0.788235}%
\pgfsetstrokecolor{currentstroke}%
\pgfsetdash{}{0pt}%
\pgfpathmoveto{\pgfqpoint{2.804341in}{2.474379in}}%
\pgfpathlineto{\pgfqpoint{2.915238in}{2.410353in}}%
\pgfpathlineto{\pgfqpoint{2.915238in}{2.538405in}}%
\pgfpathlineto{\pgfqpoint{2.804341in}{2.602432in}}%
\pgfpathlineto{\pgfqpoint{2.804341in}{2.474379in}}%
\pgfpathclose%
\pgfusepath{fill}%
\end{pgfscope}%
\begin{pgfscope}%
\pgfpathrectangle{\pgfqpoint{0.765000in}{0.660000in}}{\pgfqpoint{4.620000in}{4.620000in}}%
\pgfusepath{clip}%
\pgfsetbuttcap%
\pgfsetroundjoin%
\definecolor{currentfill}{rgb}{1.000000,0.894118,0.788235}%
\pgfsetfillcolor{currentfill}%
\pgfsetlinewidth{0.000000pt}%
\definecolor{currentstroke}{rgb}{1.000000,0.894118,0.788235}%
\pgfsetstrokecolor{currentstroke}%
\pgfsetdash{}{0pt}%
\pgfpathmoveto{\pgfqpoint{2.815582in}{2.599184in}}%
\pgfpathlineto{\pgfqpoint{2.704685in}{2.535158in}}%
\pgfpathlineto{\pgfqpoint{2.804341in}{2.474379in}}%
\pgfpathlineto{\pgfqpoint{2.915238in}{2.538405in}}%
\pgfpathlineto{\pgfqpoint{2.815582in}{2.599184in}}%
\pgfpathclose%
\pgfusepath{fill}%
\end{pgfscope}%
\begin{pgfscope}%
\pgfpathrectangle{\pgfqpoint{0.765000in}{0.660000in}}{\pgfqpoint{4.620000in}{4.620000in}}%
\pgfusepath{clip}%
\pgfsetbuttcap%
\pgfsetroundjoin%
\definecolor{currentfill}{rgb}{1.000000,0.894118,0.788235}%
\pgfsetfillcolor{currentfill}%
\pgfsetlinewidth{0.000000pt}%
\definecolor{currentstroke}{rgb}{1.000000,0.894118,0.788235}%
\pgfsetstrokecolor{currentstroke}%
\pgfsetdash{}{0pt}%
\pgfpathmoveto{\pgfqpoint{2.926478in}{2.535158in}}%
\pgfpathlineto{\pgfqpoint{2.815582in}{2.599184in}}%
\pgfpathlineto{\pgfqpoint{2.915238in}{2.538405in}}%
\pgfpathlineto{\pgfqpoint{3.026135in}{2.474379in}}%
\pgfpathlineto{\pgfqpoint{2.926478in}{2.535158in}}%
\pgfpathclose%
\pgfusepath{fill}%
\end{pgfscope}%
\begin{pgfscope}%
\pgfpathrectangle{\pgfqpoint{0.765000in}{0.660000in}}{\pgfqpoint{4.620000in}{4.620000in}}%
\pgfusepath{clip}%
\pgfsetbuttcap%
\pgfsetroundjoin%
\definecolor{currentfill}{rgb}{1.000000,0.894118,0.788235}%
\pgfsetfillcolor{currentfill}%
\pgfsetlinewidth{0.000000pt}%
\definecolor{currentstroke}{rgb}{1.000000,0.894118,0.788235}%
\pgfsetstrokecolor{currentstroke}%
\pgfsetdash{}{0pt}%
\pgfpathmoveto{\pgfqpoint{2.815582in}{2.599184in}}%
\pgfpathlineto{\pgfqpoint{2.815582in}{2.727237in}}%
\pgfpathlineto{\pgfqpoint{2.915238in}{2.666458in}}%
\pgfpathlineto{\pgfqpoint{3.026135in}{2.474379in}}%
\pgfpathlineto{\pgfqpoint{2.815582in}{2.599184in}}%
\pgfpathclose%
\pgfusepath{fill}%
\end{pgfscope}%
\begin{pgfscope}%
\pgfpathrectangle{\pgfqpoint{0.765000in}{0.660000in}}{\pgfqpoint{4.620000in}{4.620000in}}%
\pgfusepath{clip}%
\pgfsetbuttcap%
\pgfsetroundjoin%
\definecolor{currentfill}{rgb}{1.000000,0.894118,0.788235}%
\pgfsetfillcolor{currentfill}%
\pgfsetlinewidth{0.000000pt}%
\definecolor{currentstroke}{rgb}{1.000000,0.894118,0.788235}%
\pgfsetstrokecolor{currentstroke}%
\pgfsetdash{}{0pt}%
\pgfpathmoveto{\pgfqpoint{2.926478in}{2.535158in}}%
\pgfpathlineto{\pgfqpoint{2.926478in}{2.663210in}}%
\pgfpathlineto{\pgfqpoint{3.026135in}{2.602432in}}%
\pgfpathlineto{\pgfqpoint{2.915238in}{2.538405in}}%
\pgfpathlineto{\pgfqpoint{2.926478in}{2.535158in}}%
\pgfpathclose%
\pgfusepath{fill}%
\end{pgfscope}%
\begin{pgfscope}%
\pgfpathrectangle{\pgfqpoint{0.765000in}{0.660000in}}{\pgfqpoint{4.620000in}{4.620000in}}%
\pgfusepath{clip}%
\pgfsetbuttcap%
\pgfsetroundjoin%
\definecolor{currentfill}{rgb}{1.000000,0.894118,0.788235}%
\pgfsetfillcolor{currentfill}%
\pgfsetlinewidth{0.000000pt}%
\definecolor{currentstroke}{rgb}{1.000000,0.894118,0.788235}%
\pgfsetstrokecolor{currentstroke}%
\pgfsetdash{}{0pt}%
\pgfpathmoveto{\pgfqpoint{2.704685in}{2.535158in}}%
\pgfpathlineto{\pgfqpoint{2.815582in}{2.471132in}}%
\pgfpathlineto{\pgfqpoint{2.915238in}{2.410353in}}%
\pgfpathlineto{\pgfqpoint{2.804341in}{2.474379in}}%
\pgfpathlineto{\pgfqpoint{2.704685in}{2.535158in}}%
\pgfpathclose%
\pgfusepath{fill}%
\end{pgfscope}%
\begin{pgfscope}%
\pgfpathrectangle{\pgfqpoint{0.765000in}{0.660000in}}{\pgfqpoint{4.620000in}{4.620000in}}%
\pgfusepath{clip}%
\pgfsetbuttcap%
\pgfsetroundjoin%
\definecolor{currentfill}{rgb}{1.000000,0.894118,0.788235}%
\pgfsetfillcolor{currentfill}%
\pgfsetlinewidth{0.000000pt}%
\definecolor{currentstroke}{rgb}{1.000000,0.894118,0.788235}%
\pgfsetstrokecolor{currentstroke}%
\pgfsetdash{}{0pt}%
\pgfpathmoveto{\pgfqpoint{2.815582in}{2.471132in}}%
\pgfpathlineto{\pgfqpoint{2.926478in}{2.535158in}}%
\pgfpathlineto{\pgfqpoint{3.026135in}{2.474379in}}%
\pgfpathlineto{\pgfqpoint{2.915238in}{2.410353in}}%
\pgfpathlineto{\pgfqpoint{2.815582in}{2.471132in}}%
\pgfpathclose%
\pgfusepath{fill}%
\end{pgfscope}%
\begin{pgfscope}%
\pgfpathrectangle{\pgfqpoint{0.765000in}{0.660000in}}{\pgfqpoint{4.620000in}{4.620000in}}%
\pgfusepath{clip}%
\pgfsetbuttcap%
\pgfsetroundjoin%
\definecolor{currentfill}{rgb}{1.000000,0.894118,0.788235}%
\pgfsetfillcolor{currentfill}%
\pgfsetlinewidth{0.000000pt}%
\definecolor{currentstroke}{rgb}{1.000000,0.894118,0.788235}%
\pgfsetstrokecolor{currentstroke}%
\pgfsetdash{}{0pt}%
\pgfpathmoveto{\pgfqpoint{2.815582in}{2.727237in}}%
\pgfpathlineto{\pgfqpoint{2.704685in}{2.663210in}}%
\pgfpathlineto{\pgfqpoint{2.804341in}{2.602432in}}%
\pgfpathlineto{\pgfqpoint{2.915238in}{2.666458in}}%
\pgfpathlineto{\pgfqpoint{2.815582in}{2.727237in}}%
\pgfpathclose%
\pgfusepath{fill}%
\end{pgfscope}%
\begin{pgfscope}%
\pgfpathrectangle{\pgfqpoint{0.765000in}{0.660000in}}{\pgfqpoint{4.620000in}{4.620000in}}%
\pgfusepath{clip}%
\pgfsetbuttcap%
\pgfsetroundjoin%
\definecolor{currentfill}{rgb}{1.000000,0.894118,0.788235}%
\pgfsetfillcolor{currentfill}%
\pgfsetlinewidth{0.000000pt}%
\definecolor{currentstroke}{rgb}{1.000000,0.894118,0.788235}%
\pgfsetstrokecolor{currentstroke}%
\pgfsetdash{}{0pt}%
\pgfpathmoveto{\pgfqpoint{2.926478in}{2.663210in}}%
\pgfpathlineto{\pgfqpoint{2.815582in}{2.727237in}}%
\pgfpathlineto{\pgfqpoint{2.915238in}{2.666458in}}%
\pgfpathlineto{\pgfqpoint{3.026135in}{2.602432in}}%
\pgfpathlineto{\pgfqpoint{2.926478in}{2.663210in}}%
\pgfpathclose%
\pgfusepath{fill}%
\end{pgfscope}%
\begin{pgfscope}%
\pgfpathrectangle{\pgfqpoint{0.765000in}{0.660000in}}{\pgfqpoint{4.620000in}{4.620000in}}%
\pgfusepath{clip}%
\pgfsetbuttcap%
\pgfsetroundjoin%
\definecolor{currentfill}{rgb}{1.000000,0.894118,0.788235}%
\pgfsetfillcolor{currentfill}%
\pgfsetlinewidth{0.000000pt}%
\definecolor{currentstroke}{rgb}{1.000000,0.894118,0.788235}%
\pgfsetstrokecolor{currentstroke}%
\pgfsetdash{}{0pt}%
\pgfpathmoveto{\pgfqpoint{2.704685in}{2.535158in}}%
\pgfpathlineto{\pgfqpoint{2.704685in}{2.663210in}}%
\pgfpathlineto{\pgfqpoint{2.804341in}{2.602432in}}%
\pgfpathlineto{\pgfqpoint{2.915238in}{2.410353in}}%
\pgfpathlineto{\pgfqpoint{2.704685in}{2.535158in}}%
\pgfpathclose%
\pgfusepath{fill}%
\end{pgfscope}%
\begin{pgfscope}%
\pgfpathrectangle{\pgfqpoint{0.765000in}{0.660000in}}{\pgfqpoint{4.620000in}{4.620000in}}%
\pgfusepath{clip}%
\pgfsetbuttcap%
\pgfsetroundjoin%
\definecolor{currentfill}{rgb}{1.000000,0.894118,0.788235}%
\pgfsetfillcolor{currentfill}%
\pgfsetlinewidth{0.000000pt}%
\definecolor{currentstroke}{rgb}{1.000000,0.894118,0.788235}%
\pgfsetstrokecolor{currentstroke}%
\pgfsetdash{}{0pt}%
\pgfpathmoveto{\pgfqpoint{2.815582in}{2.471132in}}%
\pgfpathlineto{\pgfqpoint{2.815582in}{2.599184in}}%
\pgfpathlineto{\pgfqpoint{2.915238in}{2.538405in}}%
\pgfpathlineto{\pgfqpoint{2.804341in}{2.474379in}}%
\pgfpathlineto{\pgfqpoint{2.815582in}{2.471132in}}%
\pgfpathclose%
\pgfusepath{fill}%
\end{pgfscope}%
\begin{pgfscope}%
\pgfpathrectangle{\pgfqpoint{0.765000in}{0.660000in}}{\pgfqpoint{4.620000in}{4.620000in}}%
\pgfusepath{clip}%
\pgfsetbuttcap%
\pgfsetroundjoin%
\definecolor{currentfill}{rgb}{1.000000,0.894118,0.788235}%
\pgfsetfillcolor{currentfill}%
\pgfsetlinewidth{0.000000pt}%
\definecolor{currentstroke}{rgb}{1.000000,0.894118,0.788235}%
\pgfsetstrokecolor{currentstroke}%
\pgfsetdash{}{0pt}%
\pgfpathmoveto{\pgfqpoint{2.815582in}{2.599184in}}%
\pgfpathlineto{\pgfqpoint{2.926478in}{2.663210in}}%
\pgfpathlineto{\pgfqpoint{3.026135in}{2.602432in}}%
\pgfpathlineto{\pgfqpoint{2.915238in}{2.538405in}}%
\pgfpathlineto{\pgfqpoint{2.815582in}{2.599184in}}%
\pgfpathclose%
\pgfusepath{fill}%
\end{pgfscope}%
\begin{pgfscope}%
\pgfpathrectangle{\pgfqpoint{0.765000in}{0.660000in}}{\pgfqpoint{4.620000in}{4.620000in}}%
\pgfusepath{clip}%
\pgfsetbuttcap%
\pgfsetroundjoin%
\definecolor{currentfill}{rgb}{1.000000,0.894118,0.788235}%
\pgfsetfillcolor{currentfill}%
\pgfsetlinewidth{0.000000pt}%
\definecolor{currentstroke}{rgb}{1.000000,0.894118,0.788235}%
\pgfsetstrokecolor{currentstroke}%
\pgfsetdash{}{0pt}%
\pgfpathmoveto{\pgfqpoint{2.704685in}{2.663210in}}%
\pgfpathlineto{\pgfqpoint{2.815582in}{2.599184in}}%
\pgfpathlineto{\pgfqpoint{2.915238in}{2.538405in}}%
\pgfpathlineto{\pgfqpoint{2.804341in}{2.602432in}}%
\pgfpathlineto{\pgfqpoint{2.704685in}{2.663210in}}%
\pgfpathclose%
\pgfusepath{fill}%
\end{pgfscope}%
\begin{pgfscope}%
\pgfpathrectangle{\pgfqpoint{0.765000in}{0.660000in}}{\pgfqpoint{4.620000in}{4.620000in}}%
\pgfusepath{clip}%
\pgfsetbuttcap%
\pgfsetroundjoin%
\definecolor{currentfill}{rgb}{1.000000,0.894118,0.788235}%
\pgfsetfillcolor{currentfill}%
\pgfsetlinewidth{0.000000pt}%
\definecolor{currentstroke}{rgb}{1.000000,0.894118,0.788235}%
\pgfsetstrokecolor{currentstroke}%
\pgfsetdash{}{0pt}%
\pgfpathmoveto{\pgfqpoint{2.815582in}{2.599184in}}%
\pgfpathlineto{\pgfqpoint{2.704685in}{2.535158in}}%
\pgfpathlineto{\pgfqpoint{2.815582in}{2.471132in}}%
\pgfpathlineto{\pgfqpoint{2.926478in}{2.535158in}}%
\pgfpathlineto{\pgfqpoint{2.815582in}{2.599184in}}%
\pgfpathclose%
\pgfusepath{fill}%
\end{pgfscope}%
\begin{pgfscope}%
\pgfpathrectangle{\pgfqpoint{0.765000in}{0.660000in}}{\pgfqpoint{4.620000in}{4.620000in}}%
\pgfusepath{clip}%
\pgfsetbuttcap%
\pgfsetroundjoin%
\definecolor{currentfill}{rgb}{1.000000,0.894118,0.788235}%
\pgfsetfillcolor{currentfill}%
\pgfsetlinewidth{0.000000pt}%
\definecolor{currentstroke}{rgb}{1.000000,0.894118,0.788235}%
\pgfsetstrokecolor{currentstroke}%
\pgfsetdash{}{0pt}%
\pgfpathmoveto{\pgfqpoint{2.815582in}{2.599184in}}%
\pgfpathlineto{\pgfqpoint{2.704685in}{2.535158in}}%
\pgfpathlineto{\pgfqpoint{2.704685in}{2.663210in}}%
\pgfpathlineto{\pgfqpoint{2.815582in}{2.727237in}}%
\pgfpathlineto{\pgfqpoint{2.815582in}{2.599184in}}%
\pgfpathclose%
\pgfusepath{fill}%
\end{pgfscope}%
\begin{pgfscope}%
\pgfpathrectangle{\pgfqpoint{0.765000in}{0.660000in}}{\pgfqpoint{4.620000in}{4.620000in}}%
\pgfusepath{clip}%
\pgfsetbuttcap%
\pgfsetroundjoin%
\definecolor{currentfill}{rgb}{1.000000,0.894118,0.788235}%
\pgfsetfillcolor{currentfill}%
\pgfsetlinewidth{0.000000pt}%
\definecolor{currentstroke}{rgb}{1.000000,0.894118,0.788235}%
\pgfsetstrokecolor{currentstroke}%
\pgfsetdash{}{0pt}%
\pgfpathmoveto{\pgfqpoint{2.815582in}{2.599184in}}%
\pgfpathlineto{\pgfqpoint{2.926478in}{2.535158in}}%
\pgfpathlineto{\pgfqpoint{2.926478in}{2.663210in}}%
\pgfpathlineto{\pgfqpoint{2.815582in}{2.727237in}}%
\pgfpathlineto{\pgfqpoint{2.815582in}{2.599184in}}%
\pgfpathclose%
\pgfusepath{fill}%
\end{pgfscope}%
\begin{pgfscope}%
\pgfpathrectangle{\pgfqpoint{0.765000in}{0.660000in}}{\pgfqpoint{4.620000in}{4.620000in}}%
\pgfusepath{clip}%
\pgfsetbuttcap%
\pgfsetroundjoin%
\definecolor{currentfill}{rgb}{1.000000,0.894118,0.788235}%
\pgfsetfillcolor{currentfill}%
\pgfsetlinewidth{0.000000pt}%
\definecolor{currentstroke}{rgb}{1.000000,0.894118,0.788235}%
\pgfsetstrokecolor{currentstroke}%
\pgfsetdash{}{0pt}%
\pgfpathmoveto{\pgfqpoint{2.688236in}{2.773437in}}%
\pgfpathlineto{\pgfqpoint{2.577340in}{2.709411in}}%
\pgfpathlineto{\pgfqpoint{2.688236in}{2.645385in}}%
\pgfpathlineto{\pgfqpoint{2.799133in}{2.709411in}}%
\pgfpathlineto{\pgfqpoint{2.688236in}{2.773437in}}%
\pgfpathclose%
\pgfusepath{fill}%
\end{pgfscope}%
\begin{pgfscope}%
\pgfpathrectangle{\pgfqpoint{0.765000in}{0.660000in}}{\pgfqpoint{4.620000in}{4.620000in}}%
\pgfusepath{clip}%
\pgfsetbuttcap%
\pgfsetroundjoin%
\definecolor{currentfill}{rgb}{1.000000,0.894118,0.788235}%
\pgfsetfillcolor{currentfill}%
\pgfsetlinewidth{0.000000pt}%
\definecolor{currentstroke}{rgb}{1.000000,0.894118,0.788235}%
\pgfsetstrokecolor{currentstroke}%
\pgfsetdash{}{0pt}%
\pgfpathmoveto{\pgfqpoint{2.688236in}{2.773437in}}%
\pgfpathlineto{\pgfqpoint{2.577340in}{2.709411in}}%
\pgfpathlineto{\pgfqpoint{2.577340in}{2.837463in}}%
\pgfpathlineto{\pgfqpoint{2.688236in}{2.901490in}}%
\pgfpathlineto{\pgfqpoint{2.688236in}{2.773437in}}%
\pgfpathclose%
\pgfusepath{fill}%
\end{pgfscope}%
\begin{pgfscope}%
\pgfpathrectangle{\pgfqpoint{0.765000in}{0.660000in}}{\pgfqpoint{4.620000in}{4.620000in}}%
\pgfusepath{clip}%
\pgfsetbuttcap%
\pgfsetroundjoin%
\definecolor{currentfill}{rgb}{1.000000,0.894118,0.788235}%
\pgfsetfillcolor{currentfill}%
\pgfsetlinewidth{0.000000pt}%
\definecolor{currentstroke}{rgb}{1.000000,0.894118,0.788235}%
\pgfsetstrokecolor{currentstroke}%
\pgfsetdash{}{0pt}%
\pgfpathmoveto{\pgfqpoint{2.688236in}{2.773437in}}%
\pgfpathlineto{\pgfqpoint{2.799133in}{2.709411in}}%
\pgfpathlineto{\pgfqpoint{2.799133in}{2.837463in}}%
\pgfpathlineto{\pgfqpoint{2.688236in}{2.901490in}}%
\pgfpathlineto{\pgfqpoint{2.688236in}{2.773437in}}%
\pgfpathclose%
\pgfusepath{fill}%
\end{pgfscope}%
\begin{pgfscope}%
\pgfpathrectangle{\pgfqpoint{0.765000in}{0.660000in}}{\pgfqpoint{4.620000in}{4.620000in}}%
\pgfusepath{clip}%
\pgfsetbuttcap%
\pgfsetroundjoin%
\definecolor{currentfill}{rgb}{1.000000,0.894118,0.788235}%
\pgfsetfillcolor{currentfill}%
\pgfsetlinewidth{0.000000pt}%
\definecolor{currentstroke}{rgb}{1.000000,0.894118,0.788235}%
\pgfsetstrokecolor{currentstroke}%
\pgfsetdash{}{0pt}%
\pgfpathmoveto{\pgfqpoint{2.815582in}{2.727237in}}%
\pgfpathlineto{\pgfqpoint{2.704685in}{2.663210in}}%
\pgfpathlineto{\pgfqpoint{2.815582in}{2.599184in}}%
\pgfpathlineto{\pgfqpoint{2.926478in}{2.663210in}}%
\pgfpathlineto{\pgfqpoint{2.815582in}{2.727237in}}%
\pgfpathclose%
\pgfusepath{fill}%
\end{pgfscope}%
\begin{pgfscope}%
\pgfpathrectangle{\pgfqpoint{0.765000in}{0.660000in}}{\pgfqpoint{4.620000in}{4.620000in}}%
\pgfusepath{clip}%
\pgfsetbuttcap%
\pgfsetroundjoin%
\definecolor{currentfill}{rgb}{1.000000,0.894118,0.788235}%
\pgfsetfillcolor{currentfill}%
\pgfsetlinewidth{0.000000pt}%
\definecolor{currentstroke}{rgb}{1.000000,0.894118,0.788235}%
\pgfsetstrokecolor{currentstroke}%
\pgfsetdash{}{0pt}%
\pgfpathmoveto{\pgfqpoint{2.815582in}{2.471132in}}%
\pgfpathlineto{\pgfqpoint{2.926478in}{2.535158in}}%
\pgfpathlineto{\pgfqpoint{2.926478in}{2.663210in}}%
\pgfpathlineto{\pgfqpoint{2.815582in}{2.599184in}}%
\pgfpathlineto{\pgfqpoint{2.815582in}{2.471132in}}%
\pgfpathclose%
\pgfusepath{fill}%
\end{pgfscope}%
\begin{pgfscope}%
\pgfpathrectangle{\pgfqpoint{0.765000in}{0.660000in}}{\pgfqpoint{4.620000in}{4.620000in}}%
\pgfusepath{clip}%
\pgfsetbuttcap%
\pgfsetroundjoin%
\definecolor{currentfill}{rgb}{1.000000,0.894118,0.788235}%
\pgfsetfillcolor{currentfill}%
\pgfsetlinewidth{0.000000pt}%
\definecolor{currentstroke}{rgb}{1.000000,0.894118,0.788235}%
\pgfsetstrokecolor{currentstroke}%
\pgfsetdash{}{0pt}%
\pgfpathmoveto{\pgfqpoint{2.704685in}{2.535158in}}%
\pgfpathlineto{\pgfqpoint{2.815582in}{2.471132in}}%
\pgfpathlineto{\pgfqpoint{2.815582in}{2.599184in}}%
\pgfpathlineto{\pgfqpoint{2.704685in}{2.663210in}}%
\pgfpathlineto{\pgfqpoint{2.704685in}{2.535158in}}%
\pgfpathclose%
\pgfusepath{fill}%
\end{pgfscope}%
\begin{pgfscope}%
\pgfpathrectangle{\pgfqpoint{0.765000in}{0.660000in}}{\pgfqpoint{4.620000in}{4.620000in}}%
\pgfusepath{clip}%
\pgfsetbuttcap%
\pgfsetroundjoin%
\definecolor{currentfill}{rgb}{1.000000,0.894118,0.788235}%
\pgfsetfillcolor{currentfill}%
\pgfsetlinewidth{0.000000pt}%
\definecolor{currentstroke}{rgb}{1.000000,0.894118,0.788235}%
\pgfsetstrokecolor{currentstroke}%
\pgfsetdash{}{0pt}%
\pgfpathmoveto{\pgfqpoint{2.688236in}{2.901490in}}%
\pgfpathlineto{\pgfqpoint{2.577340in}{2.837463in}}%
\pgfpathlineto{\pgfqpoint{2.688236in}{2.773437in}}%
\pgfpathlineto{\pgfqpoint{2.799133in}{2.837463in}}%
\pgfpathlineto{\pgfqpoint{2.688236in}{2.901490in}}%
\pgfpathclose%
\pgfusepath{fill}%
\end{pgfscope}%
\begin{pgfscope}%
\pgfpathrectangle{\pgfqpoint{0.765000in}{0.660000in}}{\pgfqpoint{4.620000in}{4.620000in}}%
\pgfusepath{clip}%
\pgfsetbuttcap%
\pgfsetroundjoin%
\definecolor{currentfill}{rgb}{1.000000,0.894118,0.788235}%
\pgfsetfillcolor{currentfill}%
\pgfsetlinewidth{0.000000pt}%
\definecolor{currentstroke}{rgb}{1.000000,0.894118,0.788235}%
\pgfsetstrokecolor{currentstroke}%
\pgfsetdash{}{0pt}%
\pgfpathmoveto{\pgfqpoint{2.688236in}{2.645385in}}%
\pgfpathlineto{\pgfqpoint{2.799133in}{2.709411in}}%
\pgfpathlineto{\pgfqpoint{2.799133in}{2.837463in}}%
\pgfpathlineto{\pgfqpoint{2.688236in}{2.773437in}}%
\pgfpathlineto{\pgfqpoint{2.688236in}{2.645385in}}%
\pgfpathclose%
\pgfusepath{fill}%
\end{pgfscope}%
\begin{pgfscope}%
\pgfpathrectangle{\pgfqpoint{0.765000in}{0.660000in}}{\pgfqpoint{4.620000in}{4.620000in}}%
\pgfusepath{clip}%
\pgfsetbuttcap%
\pgfsetroundjoin%
\definecolor{currentfill}{rgb}{1.000000,0.894118,0.788235}%
\pgfsetfillcolor{currentfill}%
\pgfsetlinewidth{0.000000pt}%
\definecolor{currentstroke}{rgb}{1.000000,0.894118,0.788235}%
\pgfsetstrokecolor{currentstroke}%
\pgfsetdash{}{0pt}%
\pgfpathmoveto{\pgfqpoint{2.577340in}{2.709411in}}%
\pgfpathlineto{\pgfqpoint{2.688236in}{2.645385in}}%
\pgfpathlineto{\pgfqpoint{2.688236in}{2.773437in}}%
\pgfpathlineto{\pgfqpoint{2.577340in}{2.837463in}}%
\pgfpathlineto{\pgfqpoint{2.577340in}{2.709411in}}%
\pgfpathclose%
\pgfusepath{fill}%
\end{pgfscope}%
\begin{pgfscope}%
\pgfpathrectangle{\pgfqpoint{0.765000in}{0.660000in}}{\pgfqpoint{4.620000in}{4.620000in}}%
\pgfusepath{clip}%
\pgfsetbuttcap%
\pgfsetroundjoin%
\definecolor{currentfill}{rgb}{1.000000,0.894118,0.788235}%
\pgfsetfillcolor{currentfill}%
\pgfsetlinewidth{0.000000pt}%
\definecolor{currentstroke}{rgb}{1.000000,0.894118,0.788235}%
\pgfsetstrokecolor{currentstroke}%
\pgfsetdash{}{0pt}%
\pgfpathmoveto{\pgfqpoint{2.815582in}{2.599184in}}%
\pgfpathlineto{\pgfqpoint{2.704685in}{2.535158in}}%
\pgfpathlineto{\pgfqpoint{2.577340in}{2.709411in}}%
\pgfpathlineto{\pgfqpoint{2.688236in}{2.773437in}}%
\pgfpathlineto{\pgfqpoint{2.815582in}{2.599184in}}%
\pgfpathclose%
\pgfusepath{fill}%
\end{pgfscope}%
\begin{pgfscope}%
\pgfpathrectangle{\pgfqpoint{0.765000in}{0.660000in}}{\pgfqpoint{4.620000in}{4.620000in}}%
\pgfusepath{clip}%
\pgfsetbuttcap%
\pgfsetroundjoin%
\definecolor{currentfill}{rgb}{1.000000,0.894118,0.788235}%
\pgfsetfillcolor{currentfill}%
\pgfsetlinewidth{0.000000pt}%
\definecolor{currentstroke}{rgb}{1.000000,0.894118,0.788235}%
\pgfsetstrokecolor{currentstroke}%
\pgfsetdash{}{0pt}%
\pgfpathmoveto{\pgfqpoint{2.926478in}{2.535158in}}%
\pgfpathlineto{\pgfqpoint{2.815582in}{2.599184in}}%
\pgfpathlineto{\pgfqpoint{2.688236in}{2.773437in}}%
\pgfpathlineto{\pgfqpoint{2.799133in}{2.709411in}}%
\pgfpathlineto{\pgfqpoint{2.926478in}{2.535158in}}%
\pgfpathclose%
\pgfusepath{fill}%
\end{pgfscope}%
\begin{pgfscope}%
\pgfpathrectangle{\pgfqpoint{0.765000in}{0.660000in}}{\pgfqpoint{4.620000in}{4.620000in}}%
\pgfusepath{clip}%
\pgfsetbuttcap%
\pgfsetroundjoin%
\definecolor{currentfill}{rgb}{1.000000,0.894118,0.788235}%
\pgfsetfillcolor{currentfill}%
\pgfsetlinewidth{0.000000pt}%
\definecolor{currentstroke}{rgb}{1.000000,0.894118,0.788235}%
\pgfsetstrokecolor{currentstroke}%
\pgfsetdash{}{0pt}%
\pgfpathmoveto{\pgfqpoint{2.815582in}{2.599184in}}%
\pgfpathlineto{\pgfqpoint{2.815582in}{2.727237in}}%
\pgfpathlineto{\pgfqpoint{2.688236in}{2.901490in}}%
\pgfpathlineto{\pgfqpoint{2.799133in}{2.709411in}}%
\pgfpathlineto{\pgfqpoint{2.815582in}{2.599184in}}%
\pgfpathclose%
\pgfusepath{fill}%
\end{pgfscope}%
\begin{pgfscope}%
\pgfpathrectangle{\pgfqpoint{0.765000in}{0.660000in}}{\pgfqpoint{4.620000in}{4.620000in}}%
\pgfusepath{clip}%
\pgfsetbuttcap%
\pgfsetroundjoin%
\definecolor{currentfill}{rgb}{1.000000,0.894118,0.788235}%
\pgfsetfillcolor{currentfill}%
\pgfsetlinewidth{0.000000pt}%
\definecolor{currentstroke}{rgb}{1.000000,0.894118,0.788235}%
\pgfsetstrokecolor{currentstroke}%
\pgfsetdash{}{0pt}%
\pgfpathmoveto{\pgfqpoint{2.926478in}{2.535158in}}%
\pgfpathlineto{\pgfqpoint{2.926478in}{2.663210in}}%
\pgfpathlineto{\pgfqpoint{2.799133in}{2.837463in}}%
\pgfpathlineto{\pgfqpoint{2.688236in}{2.773437in}}%
\pgfpathlineto{\pgfqpoint{2.926478in}{2.535158in}}%
\pgfpathclose%
\pgfusepath{fill}%
\end{pgfscope}%
\begin{pgfscope}%
\pgfpathrectangle{\pgfqpoint{0.765000in}{0.660000in}}{\pgfqpoint{4.620000in}{4.620000in}}%
\pgfusepath{clip}%
\pgfsetbuttcap%
\pgfsetroundjoin%
\definecolor{currentfill}{rgb}{1.000000,0.894118,0.788235}%
\pgfsetfillcolor{currentfill}%
\pgfsetlinewidth{0.000000pt}%
\definecolor{currentstroke}{rgb}{1.000000,0.894118,0.788235}%
\pgfsetstrokecolor{currentstroke}%
\pgfsetdash{}{0pt}%
\pgfpathmoveto{\pgfqpoint{2.704685in}{2.535158in}}%
\pgfpathlineto{\pgfqpoint{2.815582in}{2.471132in}}%
\pgfpathlineto{\pgfqpoint{2.688236in}{2.645385in}}%
\pgfpathlineto{\pgfqpoint{2.577340in}{2.709411in}}%
\pgfpathlineto{\pgfqpoint{2.704685in}{2.535158in}}%
\pgfpathclose%
\pgfusepath{fill}%
\end{pgfscope}%
\begin{pgfscope}%
\pgfpathrectangle{\pgfqpoint{0.765000in}{0.660000in}}{\pgfqpoint{4.620000in}{4.620000in}}%
\pgfusepath{clip}%
\pgfsetbuttcap%
\pgfsetroundjoin%
\definecolor{currentfill}{rgb}{1.000000,0.894118,0.788235}%
\pgfsetfillcolor{currentfill}%
\pgfsetlinewidth{0.000000pt}%
\definecolor{currentstroke}{rgb}{1.000000,0.894118,0.788235}%
\pgfsetstrokecolor{currentstroke}%
\pgfsetdash{}{0pt}%
\pgfpathmoveto{\pgfqpoint{2.815582in}{2.471132in}}%
\pgfpathlineto{\pgfqpoint{2.926478in}{2.535158in}}%
\pgfpathlineto{\pgfqpoint{2.799133in}{2.709411in}}%
\pgfpathlineto{\pgfqpoint{2.688236in}{2.645385in}}%
\pgfpathlineto{\pgfqpoint{2.815582in}{2.471132in}}%
\pgfpathclose%
\pgfusepath{fill}%
\end{pgfscope}%
\begin{pgfscope}%
\pgfpathrectangle{\pgfqpoint{0.765000in}{0.660000in}}{\pgfqpoint{4.620000in}{4.620000in}}%
\pgfusepath{clip}%
\pgfsetbuttcap%
\pgfsetroundjoin%
\definecolor{currentfill}{rgb}{1.000000,0.894118,0.788235}%
\pgfsetfillcolor{currentfill}%
\pgfsetlinewidth{0.000000pt}%
\definecolor{currentstroke}{rgb}{1.000000,0.894118,0.788235}%
\pgfsetstrokecolor{currentstroke}%
\pgfsetdash{}{0pt}%
\pgfpathmoveto{\pgfqpoint{2.815582in}{2.727237in}}%
\pgfpathlineto{\pgfqpoint{2.704685in}{2.663210in}}%
\pgfpathlineto{\pgfqpoint{2.577340in}{2.837463in}}%
\pgfpathlineto{\pgfqpoint{2.688236in}{2.901490in}}%
\pgfpathlineto{\pgfqpoint{2.815582in}{2.727237in}}%
\pgfpathclose%
\pgfusepath{fill}%
\end{pgfscope}%
\begin{pgfscope}%
\pgfpathrectangle{\pgfqpoint{0.765000in}{0.660000in}}{\pgfqpoint{4.620000in}{4.620000in}}%
\pgfusepath{clip}%
\pgfsetbuttcap%
\pgfsetroundjoin%
\definecolor{currentfill}{rgb}{1.000000,0.894118,0.788235}%
\pgfsetfillcolor{currentfill}%
\pgfsetlinewidth{0.000000pt}%
\definecolor{currentstroke}{rgb}{1.000000,0.894118,0.788235}%
\pgfsetstrokecolor{currentstroke}%
\pgfsetdash{}{0pt}%
\pgfpathmoveto{\pgfqpoint{2.926478in}{2.663210in}}%
\pgfpathlineto{\pgfqpoint{2.815582in}{2.727237in}}%
\pgfpathlineto{\pgfqpoint{2.688236in}{2.901490in}}%
\pgfpathlineto{\pgfqpoint{2.799133in}{2.837463in}}%
\pgfpathlineto{\pgfqpoint{2.926478in}{2.663210in}}%
\pgfpathclose%
\pgfusepath{fill}%
\end{pgfscope}%
\begin{pgfscope}%
\pgfpathrectangle{\pgfqpoint{0.765000in}{0.660000in}}{\pgfqpoint{4.620000in}{4.620000in}}%
\pgfusepath{clip}%
\pgfsetbuttcap%
\pgfsetroundjoin%
\definecolor{currentfill}{rgb}{1.000000,0.894118,0.788235}%
\pgfsetfillcolor{currentfill}%
\pgfsetlinewidth{0.000000pt}%
\definecolor{currentstroke}{rgb}{1.000000,0.894118,0.788235}%
\pgfsetstrokecolor{currentstroke}%
\pgfsetdash{}{0pt}%
\pgfpathmoveto{\pgfqpoint{2.704685in}{2.535158in}}%
\pgfpathlineto{\pgfqpoint{2.704685in}{2.663210in}}%
\pgfpathlineto{\pgfqpoint{2.577340in}{2.837463in}}%
\pgfpathlineto{\pgfqpoint{2.688236in}{2.645385in}}%
\pgfpathlineto{\pgfqpoint{2.704685in}{2.535158in}}%
\pgfpathclose%
\pgfusepath{fill}%
\end{pgfscope}%
\begin{pgfscope}%
\pgfpathrectangle{\pgfqpoint{0.765000in}{0.660000in}}{\pgfqpoint{4.620000in}{4.620000in}}%
\pgfusepath{clip}%
\pgfsetbuttcap%
\pgfsetroundjoin%
\definecolor{currentfill}{rgb}{1.000000,0.894118,0.788235}%
\pgfsetfillcolor{currentfill}%
\pgfsetlinewidth{0.000000pt}%
\definecolor{currentstroke}{rgb}{1.000000,0.894118,0.788235}%
\pgfsetstrokecolor{currentstroke}%
\pgfsetdash{}{0pt}%
\pgfpathmoveto{\pgfqpoint{2.815582in}{2.471132in}}%
\pgfpathlineto{\pgfqpoint{2.815582in}{2.599184in}}%
\pgfpathlineto{\pgfqpoint{2.688236in}{2.773437in}}%
\pgfpathlineto{\pgfqpoint{2.577340in}{2.709411in}}%
\pgfpathlineto{\pgfqpoint{2.815582in}{2.471132in}}%
\pgfpathclose%
\pgfusepath{fill}%
\end{pgfscope}%
\begin{pgfscope}%
\pgfpathrectangle{\pgfqpoint{0.765000in}{0.660000in}}{\pgfqpoint{4.620000in}{4.620000in}}%
\pgfusepath{clip}%
\pgfsetbuttcap%
\pgfsetroundjoin%
\definecolor{currentfill}{rgb}{1.000000,0.894118,0.788235}%
\pgfsetfillcolor{currentfill}%
\pgfsetlinewidth{0.000000pt}%
\definecolor{currentstroke}{rgb}{1.000000,0.894118,0.788235}%
\pgfsetstrokecolor{currentstroke}%
\pgfsetdash{}{0pt}%
\pgfpathmoveto{\pgfqpoint{2.815582in}{2.599184in}}%
\pgfpathlineto{\pgfqpoint{2.926478in}{2.663210in}}%
\pgfpathlineto{\pgfqpoint{2.799133in}{2.837463in}}%
\pgfpathlineto{\pgfqpoint{2.688236in}{2.773437in}}%
\pgfpathlineto{\pgfqpoint{2.815582in}{2.599184in}}%
\pgfpathclose%
\pgfusepath{fill}%
\end{pgfscope}%
\begin{pgfscope}%
\pgfpathrectangle{\pgfqpoint{0.765000in}{0.660000in}}{\pgfqpoint{4.620000in}{4.620000in}}%
\pgfusepath{clip}%
\pgfsetbuttcap%
\pgfsetroundjoin%
\definecolor{currentfill}{rgb}{1.000000,0.894118,0.788235}%
\pgfsetfillcolor{currentfill}%
\pgfsetlinewidth{0.000000pt}%
\definecolor{currentstroke}{rgb}{1.000000,0.894118,0.788235}%
\pgfsetstrokecolor{currentstroke}%
\pgfsetdash{}{0pt}%
\pgfpathmoveto{\pgfqpoint{2.704685in}{2.663210in}}%
\pgfpathlineto{\pgfqpoint{2.815582in}{2.599184in}}%
\pgfpathlineto{\pgfqpoint{2.688236in}{2.773437in}}%
\pgfpathlineto{\pgfqpoint{2.577340in}{2.837463in}}%
\pgfpathlineto{\pgfqpoint{2.704685in}{2.663210in}}%
\pgfpathclose%
\pgfusepath{fill}%
\end{pgfscope}%
\begin{pgfscope}%
\pgfpathrectangle{\pgfqpoint{0.765000in}{0.660000in}}{\pgfqpoint{4.620000in}{4.620000in}}%
\pgfusepath{clip}%
\pgfsetbuttcap%
\pgfsetroundjoin%
\definecolor{currentfill}{rgb}{1.000000,0.894118,0.788235}%
\pgfsetfillcolor{currentfill}%
\pgfsetlinewidth{0.000000pt}%
\definecolor{currentstroke}{rgb}{1.000000,0.894118,0.788235}%
\pgfsetstrokecolor{currentstroke}%
\pgfsetdash{}{0pt}%
\pgfpathmoveto{\pgfqpoint{2.915238in}{2.538405in}}%
\pgfpathlineto{\pgfqpoint{2.804341in}{2.474379in}}%
\pgfpathlineto{\pgfqpoint{2.915238in}{2.410353in}}%
\pgfpathlineto{\pgfqpoint{3.026135in}{2.474379in}}%
\pgfpathlineto{\pgfqpoint{2.915238in}{2.538405in}}%
\pgfpathclose%
\pgfusepath{fill}%
\end{pgfscope}%
\begin{pgfscope}%
\pgfpathrectangle{\pgfqpoint{0.765000in}{0.660000in}}{\pgfqpoint{4.620000in}{4.620000in}}%
\pgfusepath{clip}%
\pgfsetbuttcap%
\pgfsetroundjoin%
\definecolor{currentfill}{rgb}{1.000000,0.894118,0.788235}%
\pgfsetfillcolor{currentfill}%
\pgfsetlinewidth{0.000000pt}%
\definecolor{currentstroke}{rgb}{1.000000,0.894118,0.788235}%
\pgfsetstrokecolor{currentstroke}%
\pgfsetdash{}{0pt}%
\pgfpathmoveto{\pgfqpoint{2.915238in}{2.538405in}}%
\pgfpathlineto{\pgfqpoint{2.804341in}{2.474379in}}%
\pgfpathlineto{\pgfqpoint{2.804341in}{2.602432in}}%
\pgfpathlineto{\pgfqpoint{2.915238in}{2.666458in}}%
\pgfpathlineto{\pgfqpoint{2.915238in}{2.538405in}}%
\pgfpathclose%
\pgfusepath{fill}%
\end{pgfscope}%
\begin{pgfscope}%
\pgfpathrectangle{\pgfqpoint{0.765000in}{0.660000in}}{\pgfqpoint{4.620000in}{4.620000in}}%
\pgfusepath{clip}%
\pgfsetbuttcap%
\pgfsetroundjoin%
\definecolor{currentfill}{rgb}{1.000000,0.894118,0.788235}%
\pgfsetfillcolor{currentfill}%
\pgfsetlinewidth{0.000000pt}%
\definecolor{currentstroke}{rgb}{1.000000,0.894118,0.788235}%
\pgfsetstrokecolor{currentstroke}%
\pgfsetdash{}{0pt}%
\pgfpathmoveto{\pgfqpoint{2.915238in}{2.538405in}}%
\pgfpathlineto{\pgfqpoint{3.026135in}{2.474379in}}%
\pgfpathlineto{\pgfqpoint{3.026135in}{2.602432in}}%
\pgfpathlineto{\pgfqpoint{2.915238in}{2.666458in}}%
\pgfpathlineto{\pgfqpoint{2.915238in}{2.538405in}}%
\pgfpathclose%
\pgfusepath{fill}%
\end{pgfscope}%
\begin{pgfscope}%
\pgfpathrectangle{\pgfqpoint{0.765000in}{0.660000in}}{\pgfqpoint{4.620000in}{4.620000in}}%
\pgfusepath{clip}%
\pgfsetbuttcap%
\pgfsetroundjoin%
\definecolor{currentfill}{rgb}{1.000000,0.894118,0.788235}%
\pgfsetfillcolor{currentfill}%
\pgfsetlinewidth{0.000000pt}%
\definecolor{currentstroke}{rgb}{1.000000,0.894118,0.788235}%
\pgfsetstrokecolor{currentstroke}%
\pgfsetdash{}{0pt}%
\pgfpathmoveto{\pgfqpoint{2.929883in}{2.530541in}}%
\pgfpathlineto{\pgfqpoint{2.818986in}{2.466515in}}%
\pgfpathlineto{\pgfqpoint{2.929883in}{2.402489in}}%
\pgfpathlineto{\pgfqpoint{3.040780in}{2.466515in}}%
\pgfpathlineto{\pgfqpoint{2.929883in}{2.530541in}}%
\pgfpathclose%
\pgfusepath{fill}%
\end{pgfscope}%
\begin{pgfscope}%
\pgfpathrectangle{\pgfqpoint{0.765000in}{0.660000in}}{\pgfqpoint{4.620000in}{4.620000in}}%
\pgfusepath{clip}%
\pgfsetbuttcap%
\pgfsetroundjoin%
\definecolor{currentfill}{rgb}{1.000000,0.894118,0.788235}%
\pgfsetfillcolor{currentfill}%
\pgfsetlinewidth{0.000000pt}%
\definecolor{currentstroke}{rgb}{1.000000,0.894118,0.788235}%
\pgfsetstrokecolor{currentstroke}%
\pgfsetdash{}{0pt}%
\pgfpathmoveto{\pgfqpoint{2.929883in}{2.530541in}}%
\pgfpathlineto{\pgfqpoint{2.818986in}{2.466515in}}%
\pgfpathlineto{\pgfqpoint{2.818986in}{2.594568in}}%
\pgfpathlineto{\pgfqpoint{2.929883in}{2.658594in}}%
\pgfpathlineto{\pgfqpoint{2.929883in}{2.530541in}}%
\pgfpathclose%
\pgfusepath{fill}%
\end{pgfscope}%
\begin{pgfscope}%
\pgfpathrectangle{\pgfqpoint{0.765000in}{0.660000in}}{\pgfqpoint{4.620000in}{4.620000in}}%
\pgfusepath{clip}%
\pgfsetbuttcap%
\pgfsetroundjoin%
\definecolor{currentfill}{rgb}{1.000000,0.894118,0.788235}%
\pgfsetfillcolor{currentfill}%
\pgfsetlinewidth{0.000000pt}%
\definecolor{currentstroke}{rgb}{1.000000,0.894118,0.788235}%
\pgfsetstrokecolor{currentstroke}%
\pgfsetdash{}{0pt}%
\pgfpathmoveto{\pgfqpoint{2.929883in}{2.530541in}}%
\pgfpathlineto{\pgfqpoint{3.040780in}{2.466515in}}%
\pgfpathlineto{\pgfqpoint{3.040780in}{2.594568in}}%
\pgfpathlineto{\pgfqpoint{2.929883in}{2.658594in}}%
\pgfpathlineto{\pgfqpoint{2.929883in}{2.530541in}}%
\pgfpathclose%
\pgfusepath{fill}%
\end{pgfscope}%
\begin{pgfscope}%
\pgfpathrectangle{\pgfqpoint{0.765000in}{0.660000in}}{\pgfqpoint{4.620000in}{4.620000in}}%
\pgfusepath{clip}%
\pgfsetbuttcap%
\pgfsetroundjoin%
\definecolor{currentfill}{rgb}{1.000000,0.894118,0.788235}%
\pgfsetfillcolor{currentfill}%
\pgfsetlinewidth{0.000000pt}%
\definecolor{currentstroke}{rgb}{1.000000,0.894118,0.788235}%
\pgfsetstrokecolor{currentstroke}%
\pgfsetdash{}{0pt}%
\pgfpathmoveto{\pgfqpoint{2.915238in}{2.666458in}}%
\pgfpathlineto{\pgfqpoint{2.804341in}{2.602432in}}%
\pgfpathlineto{\pgfqpoint{2.915238in}{2.538405in}}%
\pgfpathlineto{\pgfqpoint{3.026135in}{2.602432in}}%
\pgfpathlineto{\pgfqpoint{2.915238in}{2.666458in}}%
\pgfpathclose%
\pgfusepath{fill}%
\end{pgfscope}%
\begin{pgfscope}%
\pgfpathrectangle{\pgfqpoint{0.765000in}{0.660000in}}{\pgfqpoint{4.620000in}{4.620000in}}%
\pgfusepath{clip}%
\pgfsetbuttcap%
\pgfsetroundjoin%
\definecolor{currentfill}{rgb}{1.000000,0.894118,0.788235}%
\pgfsetfillcolor{currentfill}%
\pgfsetlinewidth{0.000000pt}%
\definecolor{currentstroke}{rgb}{1.000000,0.894118,0.788235}%
\pgfsetstrokecolor{currentstroke}%
\pgfsetdash{}{0pt}%
\pgfpathmoveto{\pgfqpoint{2.915238in}{2.410353in}}%
\pgfpathlineto{\pgfqpoint{3.026135in}{2.474379in}}%
\pgfpathlineto{\pgfqpoint{3.026135in}{2.602432in}}%
\pgfpathlineto{\pgfqpoint{2.915238in}{2.538405in}}%
\pgfpathlineto{\pgfqpoint{2.915238in}{2.410353in}}%
\pgfpathclose%
\pgfusepath{fill}%
\end{pgfscope}%
\begin{pgfscope}%
\pgfpathrectangle{\pgfqpoint{0.765000in}{0.660000in}}{\pgfqpoint{4.620000in}{4.620000in}}%
\pgfusepath{clip}%
\pgfsetbuttcap%
\pgfsetroundjoin%
\definecolor{currentfill}{rgb}{1.000000,0.894118,0.788235}%
\pgfsetfillcolor{currentfill}%
\pgfsetlinewidth{0.000000pt}%
\definecolor{currentstroke}{rgb}{1.000000,0.894118,0.788235}%
\pgfsetstrokecolor{currentstroke}%
\pgfsetdash{}{0pt}%
\pgfpathmoveto{\pgfqpoint{2.804341in}{2.474379in}}%
\pgfpathlineto{\pgfqpoint{2.915238in}{2.410353in}}%
\pgfpathlineto{\pgfqpoint{2.915238in}{2.538405in}}%
\pgfpathlineto{\pgfqpoint{2.804341in}{2.602432in}}%
\pgfpathlineto{\pgfqpoint{2.804341in}{2.474379in}}%
\pgfpathclose%
\pgfusepath{fill}%
\end{pgfscope}%
\begin{pgfscope}%
\pgfpathrectangle{\pgfqpoint{0.765000in}{0.660000in}}{\pgfqpoint{4.620000in}{4.620000in}}%
\pgfusepath{clip}%
\pgfsetbuttcap%
\pgfsetroundjoin%
\definecolor{currentfill}{rgb}{1.000000,0.894118,0.788235}%
\pgfsetfillcolor{currentfill}%
\pgfsetlinewidth{0.000000pt}%
\definecolor{currentstroke}{rgb}{1.000000,0.894118,0.788235}%
\pgfsetstrokecolor{currentstroke}%
\pgfsetdash{}{0pt}%
\pgfpathmoveto{\pgfqpoint{2.929883in}{2.658594in}}%
\pgfpathlineto{\pgfqpoint{2.818986in}{2.594568in}}%
\pgfpathlineto{\pgfqpoint{2.929883in}{2.530541in}}%
\pgfpathlineto{\pgfqpoint{3.040780in}{2.594568in}}%
\pgfpathlineto{\pgfqpoint{2.929883in}{2.658594in}}%
\pgfpathclose%
\pgfusepath{fill}%
\end{pgfscope}%
\begin{pgfscope}%
\pgfpathrectangle{\pgfqpoint{0.765000in}{0.660000in}}{\pgfqpoint{4.620000in}{4.620000in}}%
\pgfusepath{clip}%
\pgfsetbuttcap%
\pgfsetroundjoin%
\definecolor{currentfill}{rgb}{1.000000,0.894118,0.788235}%
\pgfsetfillcolor{currentfill}%
\pgfsetlinewidth{0.000000pt}%
\definecolor{currentstroke}{rgb}{1.000000,0.894118,0.788235}%
\pgfsetstrokecolor{currentstroke}%
\pgfsetdash{}{0pt}%
\pgfpathmoveto{\pgfqpoint{2.929883in}{2.402489in}}%
\pgfpathlineto{\pgfqpoint{3.040780in}{2.466515in}}%
\pgfpathlineto{\pgfqpoint{3.040780in}{2.594568in}}%
\pgfpathlineto{\pgfqpoint{2.929883in}{2.530541in}}%
\pgfpathlineto{\pgfqpoint{2.929883in}{2.402489in}}%
\pgfpathclose%
\pgfusepath{fill}%
\end{pgfscope}%
\begin{pgfscope}%
\pgfpathrectangle{\pgfqpoint{0.765000in}{0.660000in}}{\pgfqpoint{4.620000in}{4.620000in}}%
\pgfusepath{clip}%
\pgfsetbuttcap%
\pgfsetroundjoin%
\definecolor{currentfill}{rgb}{1.000000,0.894118,0.788235}%
\pgfsetfillcolor{currentfill}%
\pgfsetlinewidth{0.000000pt}%
\definecolor{currentstroke}{rgb}{1.000000,0.894118,0.788235}%
\pgfsetstrokecolor{currentstroke}%
\pgfsetdash{}{0pt}%
\pgfpathmoveto{\pgfqpoint{2.818986in}{2.466515in}}%
\pgfpathlineto{\pgfqpoint{2.929883in}{2.402489in}}%
\pgfpathlineto{\pgfqpoint{2.929883in}{2.530541in}}%
\pgfpathlineto{\pgfqpoint{2.818986in}{2.594568in}}%
\pgfpathlineto{\pgfqpoint{2.818986in}{2.466515in}}%
\pgfpathclose%
\pgfusepath{fill}%
\end{pgfscope}%
\begin{pgfscope}%
\pgfpathrectangle{\pgfqpoint{0.765000in}{0.660000in}}{\pgfqpoint{4.620000in}{4.620000in}}%
\pgfusepath{clip}%
\pgfsetbuttcap%
\pgfsetroundjoin%
\definecolor{currentfill}{rgb}{1.000000,0.894118,0.788235}%
\pgfsetfillcolor{currentfill}%
\pgfsetlinewidth{0.000000pt}%
\definecolor{currentstroke}{rgb}{1.000000,0.894118,0.788235}%
\pgfsetstrokecolor{currentstroke}%
\pgfsetdash{}{0pt}%
\pgfpathmoveto{\pgfqpoint{2.915238in}{2.538405in}}%
\pgfpathlineto{\pgfqpoint{2.804341in}{2.474379in}}%
\pgfpathlineto{\pgfqpoint{2.818986in}{2.466515in}}%
\pgfpathlineto{\pgfqpoint{2.929883in}{2.530541in}}%
\pgfpathlineto{\pgfqpoint{2.915238in}{2.538405in}}%
\pgfpathclose%
\pgfusepath{fill}%
\end{pgfscope}%
\begin{pgfscope}%
\pgfpathrectangle{\pgfqpoint{0.765000in}{0.660000in}}{\pgfqpoint{4.620000in}{4.620000in}}%
\pgfusepath{clip}%
\pgfsetbuttcap%
\pgfsetroundjoin%
\definecolor{currentfill}{rgb}{1.000000,0.894118,0.788235}%
\pgfsetfillcolor{currentfill}%
\pgfsetlinewidth{0.000000pt}%
\definecolor{currentstroke}{rgb}{1.000000,0.894118,0.788235}%
\pgfsetstrokecolor{currentstroke}%
\pgfsetdash{}{0pt}%
\pgfpathmoveto{\pgfqpoint{3.026135in}{2.474379in}}%
\pgfpathlineto{\pgfqpoint{2.915238in}{2.538405in}}%
\pgfpathlineto{\pgfqpoint{2.929883in}{2.530541in}}%
\pgfpathlineto{\pgfqpoint{3.040780in}{2.466515in}}%
\pgfpathlineto{\pgfqpoint{3.026135in}{2.474379in}}%
\pgfpathclose%
\pgfusepath{fill}%
\end{pgfscope}%
\begin{pgfscope}%
\pgfpathrectangle{\pgfqpoint{0.765000in}{0.660000in}}{\pgfqpoint{4.620000in}{4.620000in}}%
\pgfusepath{clip}%
\pgfsetbuttcap%
\pgfsetroundjoin%
\definecolor{currentfill}{rgb}{1.000000,0.894118,0.788235}%
\pgfsetfillcolor{currentfill}%
\pgfsetlinewidth{0.000000pt}%
\definecolor{currentstroke}{rgb}{1.000000,0.894118,0.788235}%
\pgfsetstrokecolor{currentstroke}%
\pgfsetdash{}{0pt}%
\pgfpathmoveto{\pgfqpoint{2.915238in}{2.538405in}}%
\pgfpathlineto{\pgfqpoint{2.915238in}{2.666458in}}%
\pgfpathlineto{\pgfqpoint{2.929883in}{2.658594in}}%
\pgfpathlineto{\pgfqpoint{3.040780in}{2.466515in}}%
\pgfpathlineto{\pgfqpoint{2.915238in}{2.538405in}}%
\pgfpathclose%
\pgfusepath{fill}%
\end{pgfscope}%
\begin{pgfscope}%
\pgfpathrectangle{\pgfqpoint{0.765000in}{0.660000in}}{\pgfqpoint{4.620000in}{4.620000in}}%
\pgfusepath{clip}%
\pgfsetbuttcap%
\pgfsetroundjoin%
\definecolor{currentfill}{rgb}{1.000000,0.894118,0.788235}%
\pgfsetfillcolor{currentfill}%
\pgfsetlinewidth{0.000000pt}%
\definecolor{currentstroke}{rgb}{1.000000,0.894118,0.788235}%
\pgfsetstrokecolor{currentstroke}%
\pgfsetdash{}{0pt}%
\pgfpathmoveto{\pgfqpoint{3.026135in}{2.474379in}}%
\pgfpathlineto{\pgfqpoint{3.026135in}{2.602432in}}%
\pgfpathlineto{\pgfqpoint{3.040780in}{2.594568in}}%
\pgfpathlineto{\pgfqpoint{2.929883in}{2.530541in}}%
\pgfpathlineto{\pgfqpoint{3.026135in}{2.474379in}}%
\pgfpathclose%
\pgfusepath{fill}%
\end{pgfscope}%
\begin{pgfscope}%
\pgfpathrectangle{\pgfqpoint{0.765000in}{0.660000in}}{\pgfqpoint{4.620000in}{4.620000in}}%
\pgfusepath{clip}%
\pgfsetbuttcap%
\pgfsetroundjoin%
\definecolor{currentfill}{rgb}{1.000000,0.894118,0.788235}%
\pgfsetfillcolor{currentfill}%
\pgfsetlinewidth{0.000000pt}%
\definecolor{currentstroke}{rgb}{1.000000,0.894118,0.788235}%
\pgfsetstrokecolor{currentstroke}%
\pgfsetdash{}{0pt}%
\pgfpathmoveto{\pgfqpoint{2.804341in}{2.474379in}}%
\pgfpathlineto{\pgfqpoint{2.915238in}{2.410353in}}%
\pgfpathlineto{\pgfqpoint{2.929883in}{2.402489in}}%
\pgfpathlineto{\pgfqpoint{2.818986in}{2.466515in}}%
\pgfpathlineto{\pgfqpoint{2.804341in}{2.474379in}}%
\pgfpathclose%
\pgfusepath{fill}%
\end{pgfscope}%
\begin{pgfscope}%
\pgfpathrectangle{\pgfqpoint{0.765000in}{0.660000in}}{\pgfqpoint{4.620000in}{4.620000in}}%
\pgfusepath{clip}%
\pgfsetbuttcap%
\pgfsetroundjoin%
\definecolor{currentfill}{rgb}{1.000000,0.894118,0.788235}%
\pgfsetfillcolor{currentfill}%
\pgfsetlinewidth{0.000000pt}%
\definecolor{currentstroke}{rgb}{1.000000,0.894118,0.788235}%
\pgfsetstrokecolor{currentstroke}%
\pgfsetdash{}{0pt}%
\pgfpathmoveto{\pgfqpoint{2.915238in}{2.410353in}}%
\pgfpathlineto{\pgfqpoint{3.026135in}{2.474379in}}%
\pgfpathlineto{\pgfqpoint{3.040780in}{2.466515in}}%
\pgfpathlineto{\pgfqpoint{2.929883in}{2.402489in}}%
\pgfpathlineto{\pgfqpoint{2.915238in}{2.410353in}}%
\pgfpathclose%
\pgfusepath{fill}%
\end{pgfscope}%
\begin{pgfscope}%
\pgfpathrectangle{\pgfqpoint{0.765000in}{0.660000in}}{\pgfqpoint{4.620000in}{4.620000in}}%
\pgfusepath{clip}%
\pgfsetbuttcap%
\pgfsetroundjoin%
\definecolor{currentfill}{rgb}{1.000000,0.894118,0.788235}%
\pgfsetfillcolor{currentfill}%
\pgfsetlinewidth{0.000000pt}%
\definecolor{currentstroke}{rgb}{1.000000,0.894118,0.788235}%
\pgfsetstrokecolor{currentstroke}%
\pgfsetdash{}{0pt}%
\pgfpathmoveto{\pgfqpoint{2.915238in}{2.666458in}}%
\pgfpathlineto{\pgfqpoint{2.804341in}{2.602432in}}%
\pgfpathlineto{\pgfqpoint{2.818986in}{2.594568in}}%
\pgfpathlineto{\pgfqpoint{2.929883in}{2.658594in}}%
\pgfpathlineto{\pgfqpoint{2.915238in}{2.666458in}}%
\pgfpathclose%
\pgfusepath{fill}%
\end{pgfscope}%
\begin{pgfscope}%
\pgfpathrectangle{\pgfqpoint{0.765000in}{0.660000in}}{\pgfqpoint{4.620000in}{4.620000in}}%
\pgfusepath{clip}%
\pgfsetbuttcap%
\pgfsetroundjoin%
\definecolor{currentfill}{rgb}{1.000000,0.894118,0.788235}%
\pgfsetfillcolor{currentfill}%
\pgfsetlinewidth{0.000000pt}%
\definecolor{currentstroke}{rgb}{1.000000,0.894118,0.788235}%
\pgfsetstrokecolor{currentstroke}%
\pgfsetdash{}{0pt}%
\pgfpathmoveto{\pgfqpoint{3.026135in}{2.602432in}}%
\pgfpathlineto{\pgfqpoint{2.915238in}{2.666458in}}%
\pgfpathlineto{\pgfqpoint{2.929883in}{2.658594in}}%
\pgfpathlineto{\pgfqpoint{3.040780in}{2.594568in}}%
\pgfpathlineto{\pgfqpoint{3.026135in}{2.602432in}}%
\pgfpathclose%
\pgfusepath{fill}%
\end{pgfscope}%
\begin{pgfscope}%
\pgfpathrectangle{\pgfqpoint{0.765000in}{0.660000in}}{\pgfqpoint{4.620000in}{4.620000in}}%
\pgfusepath{clip}%
\pgfsetbuttcap%
\pgfsetroundjoin%
\definecolor{currentfill}{rgb}{1.000000,0.894118,0.788235}%
\pgfsetfillcolor{currentfill}%
\pgfsetlinewidth{0.000000pt}%
\definecolor{currentstroke}{rgb}{1.000000,0.894118,0.788235}%
\pgfsetstrokecolor{currentstroke}%
\pgfsetdash{}{0pt}%
\pgfpathmoveto{\pgfqpoint{2.804341in}{2.474379in}}%
\pgfpathlineto{\pgfqpoint{2.804341in}{2.602432in}}%
\pgfpathlineto{\pgfqpoint{2.818986in}{2.594568in}}%
\pgfpathlineto{\pgfqpoint{2.929883in}{2.402489in}}%
\pgfpathlineto{\pgfqpoint{2.804341in}{2.474379in}}%
\pgfpathclose%
\pgfusepath{fill}%
\end{pgfscope}%
\begin{pgfscope}%
\pgfpathrectangle{\pgfqpoint{0.765000in}{0.660000in}}{\pgfqpoint{4.620000in}{4.620000in}}%
\pgfusepath{clip}%
\pgfsetbuttcap%
\pgfsetroundjoin%
\definecolor{currentfill}{rgb}{1.000000,0.894118,0.788235}%
\pgfsetfillcolor{currentfill}%
\pgfsetlinewidth{0.000000pt}%
\definecolor{currentstroke}{rgb}{1.000000,0.894118,0.788235}%
\pgfsetstrokecolor{currentstroke}%
\pgfsetdash{}{0pt}%
\pgfpathmoveto{\pgfqpoint{2.915238in}{2.410353in}}%
\pgfpathlineto{\pgfqpoint{2.915238in}{2.538405in}}%
\pgfpathlineto{\pgfqpoint{2.929883in}{2.530541in}}%
\pgfpathlineto{\pgfqpoint{2.818986in}{2.466515in}}%
\pgfpathlineto{\pgfqpoint{2.915238in}{2.410353in}}%
\pgfpathclose%
\pgfusepath{fill}%
\end{pgfscope}%
\begin{pgfscope}%
\pgfpathrectangle{\pgfqpoint{0.765000in}{0.660000in}}{\pgfqpoint{4.620000in}{4.620000in}}%
\pgfusepath{clip}%
\pgfsetbuttcap%
\pgfsetroundjoin%
\definecolor{currentfill}{rgb}{1.000000,0.894118,0.788235}%
\pgfsetfillcolor{currentfill}%
\pgfsetlinewidth{0.000000pt}%
\definecolor{currentstroke}{rgb}{1.000000,0.894118,0.788235}%
\pgfsetstrokecolor{currentstroke}%
\pgfsetdash{}{0pt}%
\pgfpathmoveto{\pgfqpoint{2.915238in}{2.538405in}}%
\pgfpathlineto{\pgfqpoint{3.026135in}{2.602432in}}%
\pgfpathlineto{\pgfqpoint{3.040780in}{2.594568in}}%
\pgfpathlineto{\pgfqpoint{2.929883in}{2.530541in}}%
\pgfpathlineto{\pgfqpoint{2.915238in}{2.538405in}}%
\pgfpathclose%
\pgfusepath{fill}%
\end{pgfscope}%
\begin{pgfscope}%
\pgfpathrectangle{\pgfqpoint{0.765000in}{0.660000in}}{\pgfqpoint{4.620000in}{4.620000in}}%
\pgfusepath{clip}%
\pgfsetbuttcap%
\pgfsetroundjoin%
\definecolor{currentfill}{rgb}{1.000000,0.894118,0.788235}%
\pgfsetfillcolor{currentfill}%
\pgfsetlinewidth{0.000000pt}%
\definecolor{currentstroke}{rgb}{1.000000,0.894118,0.788235}%
\pgfsetstrokecolor{currentstroke}%
\pgfsetdash{}{0pt}%
\pgfpathmoveto{\pgfqpoint{2.804341in}{2.602432in}}%
\pgfpathlineto{\pgfqpoint{2.915238in}{2.538405in}}%
\pgfpathlineto{\pgfqpoint{2.929883in}{2.530541in}}%
\pgfpathlineto{\pgfqpoint{2.818986in}{2.594568in}}%
\pgfpathlineto{\pgfqpoint{2.804341in}{2.602432in}}%
\pgfpathclose%
\pgfusepath{fill}%
\end{pgfscope}%
\begin{pgfscope}%
\pgfpathrectangle{\pgfqpoint{0.765000in}{0.660000in}}{\pgfqpoint{4.620000in}{4.620000in}}%
\pgfusepath{clip}%
\pgfsetbuttcap%
\pgfsetroundjoin%
\definecolor{currentfill}{rgb}{1.000000,0.894118,0.788235}%
\pgfsetfillcolor{currentfill}%
\pgfsetlinewidth{0.000000pt}%
\definecolor{currentstroke}{rgb}{1.000000,0.894118,0.788235}%
\pgfsetstrokecolor{currentstroke}%
\pgfsetdash{}{0pt}%
\pgfpathmoveto{\pgfqpoint{2.915238in}{2.538405in}}%
\pgfpathlineto{\pgfqpoint{2.804341in}{2.474379in}}%
\pgfpathlineto{\pgfqpoint{2.915238in}{2.410353in}}%
\pgfpathlineto{\pgfqpoint{3.026135in}{2.474379in}}%
\pgfpathlineto{\pgfqpoint{2.915238in}{2.538405in}}%
\pgfpathclose%
\pgfusepath{fill}%
\end{pgfscope}%
\begin{pgfscope}%
\pgfpathrectangle{\pgfqpoint{0.765000in}{0.660000in}}{\pgfqpoint{4.620000in}{4.620000in}}%
\pgfusepath{clip}%
\pgfsetbuttcap%
\pgfsetroundjoin%
\definecolor{currentfill}{rgb}{1.000000,0.894118,0.788235}%
\pgfsetfillcolor{currentfill}%
\pgfsetlinewidth{0.000000pt}%
\definecolor{currentstroke}{rgb}{1.000000,0.894118,0.788235}%
\pgfsetstrokecolor{currentstroke}%
\pgfsetdash{}{0pt}%
\pgfpathmoveto{\pgfqpoint{2.915238in}{2.538405in}}%
\pgfpathlineto{\pgfqpoint{2.804341in}{2.474379in}}%
\pgfpathlineto{\pgfqpoint{2.804341in}{2.602432in}}%
\pgfpathlineto{\pgfqpoint{2.915238in}{2.666458in}}%
\pgfpathlineto{\pgfqpoint{2.915238in}{2.538405in}}%
\pgfpathclose%
\pgfusepath{fill}%
\end{pgfscope}%
\begin{pgfscope}%
\pgfpathrectangle{\pgfqpoint{0.765000in}{0.660000in}}{\pgfqpoint{4.620000in}{4.620000in}}%
\pgfusepath{clip}%
\pgfsetbuttcap%
\pgfsetroundjoin%
\definecolor{currentfill}{rgb}{1.000000,0.894118,0.788235}%
\pgfsetfillcolor{currentfill}%
\pgfsetlinewidth{0.000000pt}%
\definecolor{currentstroke}{rgb}{1.000000,0.894118,0.788235}%
\pgfsetstrokecolor{currentstroke}%
\pgfsetdash{}{0pt}%
\pgfpathmoveto{\pgfqpoint{2.915238in}{2.538405in}}%
\pgfpathlineto{\pgfqpoint{3.026135in}{2.474379in}}%
\pgfpathlineto{\pgfqpoint{3.026135in}{2.602432in}}%
\pgfpathlineto{\pgfqpoint{2.915238in}{2.666458in}}%
\pgfpathlineto{\pgfqpoint{2.915238in}{2.538405in}}%
\pgfpathclose%
\pgfusepath{fill}%
\end{pgfscope}%
\begin{pgfscope}%
\pgfpathrectangle{\pgfqpoint{0.765000in}{0.660000in}}{\pgfqpoint{4.620000in}{4.620000in}}%
\pgfusepath{clip}%
\pgfsetbuttcap%
\pgfsetroundjoin%
\definecolor{currentfill}{rgb}{1.000000,0.894118,0.788235}%
\pgfsetfillcolor{currentfill}%
\pgfsetlinewidth{0.000000pt}%
\definecolor{currentstroke}{rgb}{1.000000,0.894118,0.788235}%
\pgfsetstrokecolor{currentstroke}%
\pgfsetdash{}{0pt}%
\pgfpathmoveto{\pgfqpoint{2.815582in}{2.599184in}}%
\pgfpathlineto{\pgfqpoint{2.704685in}{2.535158in}}%
\pgfpathlineto{\pgfqpoint{2.815582in}{2.471132in}}%
\pgfpathlineto{\pgfqpoint{2.926478in}{2.535158in}}%
\pgfpathlineto{\pgfqpoint{2.815582in}{2.599184in}}%
\pgfpathclose%
\pgfusepath{fill}%
\end{pgfscope}%
\begin{pgfscope}%
\pgfpathrectangle{\pgfqpoint{0.765000in}{0.660000in}}{\pgfqpoint{4.620000in}{4.620000in}}%
\pgfusepath{clip}%
\pgfsetbuttcap%
\pgfsetroundjoin%
\definecolor{currentfill}{rgb}{1.000000,0.894118,0.788235}%
\pgfsetfillcolor{currentfill}%
\pgfsetlinewidth{0.000000pt}%
\definecolor{currentstroke}{rgb}{1.000000,0.894118,0.788235}%
\pgfsetstrokecolor{currentstroke}%
\pgfsetdash{}{0pt}%
\pgfpathmoveto{\pgfqpoint{2.815582in}{2.599184in}}%
\pgfpathlineto{\pgfqpoint{2.704685in}{2.535158in}}%
\pgfpathlineto{\pgfqpoint{2.704685in}{2.663210in}}%
\pgfpathlineto{\pgfqpoint{2.815582in}{2.727237in}}%
\pgfpathlineto{\pgfqpoint{2.815582in}{2.599184in}}%
\pgfpathclose%
\pgfusepath{fill}%
\end{pgfscope}%
\begin{pgfscope}%
\pgfpathrectangle{\pgfqpoint{0.765000in}{0.660000in}}{\pgfqpoint{4.620000in}{4.620000in}}%
\pgfusepath{clip}%
\pgfsetbuttcap%
\pgfsetroundjoin%
\definecolor{currentfill}{rgb}{1.000000,0.894118,0.788235}%
\pgfsetfillcolor{currentfill}%
\pgfsetlinewidth{0.000000pt}%
\definecolor{currentstroke}{rgb}{1.000000,0.894118,0.788235}%
\pgfsetstrokecolor{currentstroke}%
\pgfsetdash{}{0pt}%
\pgfpathmoveto{\pgfqpoint{2.815582in}{2.599184in}}%
\pgfpathlineto{\pgfqpoint{2.926478in}{2.535158in}}%
\pgfpathlineto{\pgfqpoint{2.926478in}{2.663210in}}%
\pgfpathlineto{\pgfqpoint{2.815582in}{2.727237in}}%
\pgfpathlineto{\pgfqpoint{2.815582in}{2.599184in}}%
\pgfpathclose%
\pgfusepath{fill}%
\end{pgfscope}%
\begin{pgfscope}%
\pgfpathrectangle{\pgfqpoint{0.765000in}{0.660000in}}{\pgfqpoint{4.620000in}{4.620000in}}%
\pgfusepath{clip}%
\pgfsetbuttcap%
\pgfsetroundjoin%
\definecolor{currentfill}{rgb}{1.000000,0.894118,0.788235}%
\pgfsetfillcolor{currentfill}%
\pgfsetlinewidth{0.000000pt}%
\definecolor{currentstroke}{rgb}{1.000000,0.894118,0.788235}%
\pgfsetstrokecolor{currentstroke}%
\pgfsetdash{}{0pt}%
\pgfpathmoveto{\pgfqpoint{2.915238in}{2.666458in}}%
\pgfpathlineto{\pgfqpoint{2.804341in}{2.602432in}}%
\pgfpathlineto{\pgfqpoint{2.915238in}{2.538405in}}%
\pgfpathlineto{\pgfqpoint{3.026135in}{2.602432in}}%
\pgfpathlineto{\pgfqpoint{2.915238in}{2.666458in}}%
\pgfpathclose%
\pgfusepath{fill}%
\end{pgfscope}%
\begin{pgfscope}%
\pgfpathrectangle{\pgfqpoint{0.765000in}{0.660000in}}{\pgfqpoint{4.620000in}{4.620000in}}%
\pgfusepath{clip}%
\pgfsetbuttcap%
\pgfsetroundjoin%
\definecolor{currentfill}{rgb}{1.000000,0.894118,0.788235}%
\pgfsetfillcolor{currentfill}%
\pgfsetlinewidth{0.000000pt}%
\definecolor{currentstroke}{rgb}{1.000000,0.894118,0.788235}%
\pgfsetstrokecolor{currentstroke}%
\pgfsetdash{}{0pt}%
\pgfpathmoveto{\pgfqpoint{2.915238in}{2.410353in}}%
\pgfpathlineto{\pgfqpoint{3.026135in}{2.474379in}}%
\pgfpathlineto{\pgfqpoint{3.026135in}{2.602432in}}%
\pgfpathlineto{\pgfqpoint{2.915238in}{2.538405in}}%
\pgfpathlineto{\pgfqpoint{2.915238in}{2.410353in}}%
\pgfpathclose%
\pgfusepath{fill}%
\end{pgfscope}%
\begin{pgfscope}%
\pgfpathrectangle{\pgfqpoint{0.765000in}{0.660000in}}{\pgfqpoint{4.620000in}{4.620000in}}%
\pgfusepath{clip}%
\pgfsetbuttcap%
\pgfsetroundjoin%
\definecolor{currentfill}{rgb}{1.000000,0.894118,0.788235}%
\pgfsetfillcolor{currentfill}%
\pgfsetlinewidth{0.000000pt}%
\definecolor{currentstroke}{rgb}{1.000000,0.894118,0.788235}%
\pgfsetstrokecolor{currentstroke}%
\pgfsetdash{}{0pt}%
\pgfpathmoveto{\pgfqpoint{2.804341in}{2.474379in}}%
\pgfpathlineto{\pgfqpoint{2.915238in}{2.410353in}}%
\pgfpathlineto{\pgfqpoint{2.915238in}{2.538405in}}%
\pgfpathlineto{\pgfqpoint{2.804341in}{2.602432in}}%
\pgfpathlineto{\pgfqpoint{2.804341in}{2.474379in}}%
\pgfpathclose%
\pgfusepath{fill}%
\end{pgfscope}%
\begin{pgfscope}%
\pgfpathrectangle{\pgfqpoint{0.765000in}{0.660000in}}{\pgfqpoint{4.620000in}{4.620000in}}%
\pgfusepath{clip}%
\pgfsetbuttcap%
\pgfsetroundjoin%
\definecolor{currentfill}{rgb}{1.000000,0.894118,0.788235}%
\pgfsetfillcolor{currentfill}%
\pgfsetlinewidth{0.000000pt}%
\definecolor{currentstroke}{rgb}{1.000000,0.894118,0.788235}%
\pgfsetstrokecolor{currentstroke}%
\pgfsetdash{}{0pt}%
\pgfpathmoveto{\pgfqpoint{2.815582in}{2.727237in}}%
\pgfpathlineto{\pgfqpoint{2.704685in}{2.663210in}}%
\pgfpathlineto{\pgfqpoint{2.815582in}{2.599184in}}%
\pgfpathlineto{\pgfqpoint{2.926478in}{2.663210in}}%
\pgfpathlineto{\pgfqpoint{2.815582in}{2.727237in}}%
\pgfpathclose%
\pgfusepath{fill}%
\end{pgfscope}%
\begin{pgfscope}%
\pgfpathrectangle{\pgfqpoint{0.765000in}{0.660000in}}{\pgfqpoint{4.620000in}{4.620000in}}%
\pgfusepath{clip}%
\pgfsetbuttcap%
\pgfsetroundjoin%
\definecolor{currentfill}{rgb}{1.000000,0.894118,0.788235}%
\pgfsetfillcolor{currentfill}%
\pgfsetlinewidth{0.000000pt}%
\definecolor{currentstroke}{rgb}{1.000000,0.894118,0.788235}%
\pgfsetstrokecolor{currentstroke}%
\pgfsetdash{}{0pt}%
\pgfpathmoveto{\pgfqpoint{2.815582in}{2.471132in}}%
\pgfpathlineto{\pgfqpoint{2.926478in}{2.535158in}}%
\pgfpathlineto{\pgfqpoint{2.926478in}{2.663210in}}%
\pgfpathlineto{\pgfqpoint{2.815582in}{2.599184in}}%
\pgfpathlineto{\pgfqpoint{2.815582in}{2.471132in}}%
\pgfpathclose%
\pgfusepath{fill}%
\end{pgfscope}%
\begin{pgfscope}%
\pgfpathrectangle{\pgfqpoint{0.765000in}{0.660000in}}{\pgfqpoint{4.620000in}{4.620000in}}%
\pgfusepath{clip}%
\pgfsetbuttcap%
\pgfsetroundjoin%
\definecolor{currentfill}{rgb}{1.000000,0.894118,0.788235}%
\pgfsetfillcolor{currentfill}%
\pgfsetlinewidth{0.000000pt}%
\definecolor{currentstroke}{rgb}{1.000000,0.894118,0.788235}%
\pgfsetstrokecolor{currentstroke}%
\pgfsetdash{}{0pt}%
\pgfpathmoveto{\pgfqpoint{2.704685in}{2.535158in}}%
\pgfpathlineto{\pgfqpoint{2.815582in}{2.471132in}}%
\pgfpathlineto{\pgfqpoint{2.815582in}{2.599184in}}%
\pgfpathlineto{\pgfqpoint{2.704685in}{2.663210in}}%
\pgfpathlineto{\pgfqpoint{2.704685in}{2.535158in}}%
\pgfpathclose%
\pgfusepath{fill}%
\end{pgfscope}%
\begin{pgfscope}%
\pgfpathrectangle{\pgfqpoint{0.765000in}{0.660000in}}{\pgfqpoint{4.620000in}{4.620000in}}%
\pgfusepath{clip}%
\pgfsetbuttcap%
\pgfsetroundjoin%
\definecolor{currentfill}{rgb}{1.000000,0.894118,0.788235}%
\pgfsetfillcolor{currentfill}%
\pgfsetlinewidth{0.000000pt}%
\definecolor{currentstroke}{rgb}{1.000000,0.894118,0.788235}%
\pgfsetstrokecolor{currentstroke}%
\pgfsetdash{}{0pt}%
\pgfpathmoveto{\pgfqpoint{2.915238in}{2.538405in}}%
\pgfpathlineto{\pgfqpoint{2.804341in}{2.474379in}}%
\pgfpathlineto{\pgfqpoint{2.704685in}{2.535158in}}%
\pgfpathlineto{\pgfqpoint{2.815582in}{2.599184in}}%
\pgfpathlineto{\pgfqpoint{2.915238in}{2.538405in}}%
\pgfpathclose%
\pgfusepath{fill}%
\end{pgfscope}%
\begin{pgfscope}%
\pgfpathrectangle{\pgfqpoint{0.765000in}{0.660000in}}{\pgfqpoint{4.620000in}{4.620000in}}%
\pgfusepath{clip}%
\pgfsetbuttcap%
\pgfsetroundjoin%
\definecolor{currentfill}{rgb}{1.000000,0.894118,0.788235}%
\pgfsetfillcolor{currentfill}%
\pgfsetlinewidth{0.000000pt}%
\definecolor{currentstroke}{rgb}{1.000000,0.894118,0.788235}%
\pgfsetstrokecolor{currentstroke}%
\pgfsetdash{}{0pt}%
\pgfpathmoveto{\pgfqpoint{3.026135in}{2.474379in}}%
\pgfpathlineto{\pgfqpoint{2.915238in}{2.538405in}}%
\pgfpathlineto{\pgfqpoint{2.815582in}{2.599184in}}%
\pgfpathlineto{\pgfqpoint{2.926478in}{2.535158in}}%
\pgfpathlineto{\pgfqpoint{3.026135in}{2.474379in}}%
\pgfpathclose%
\pgfusepath{fill}%
\end{pgfscope}%
\begin{pgfscope}%
\pgfpathrectangle{\pgfqpoint{0.765000in}{0.660000in}}{\pgfqpoint{4.620000in}{4.620000in}}%
\pgfusepath{clip}%
\pgfsetbuttcap%
\pgfsetroundjoin%
\definecolor{currentfill}{rgb}{1.000000,0.894118,0.788235}%
\pgfsetfillcolor{currentfill}%
\pgfsetlinewidth{0.000000pt}%
\definecolor{currentstroke}{rgb}{1.000000,0.894118,0.788235}%
\pgfsetstrokecolor{currentstroke}%
\pgfsetdash{}{0pt}%
\pgfpathmoveto{\pgfqpoint{2.915238in}{2.538405in}}%
\pgfpathlineto{\pgfqpoint{2.915238in}{2.666458in}}%
\pgfpathlineto{\pgfqpoint{2.815582in}{2.727237in}}%
\pgfpathlineto{\pgfqpoint{2.926478in}{2.535158in}}%
\pgfpathlineto{\pgfqpoint{2.915238in}{2.538405in}}%
\pgfpathclose%
\pgfusepath{fill}%
\end{pgfscope}%
\begin{pgfscope}%
\pgfpathrectangle{\pgfqpoint{0.765000in}{0.660000in}}{\pgfqpoint{4.620000in}{4.620000in}}%
\pgfusepath{clip}%
\pgfsetbuttcap%
\pgfsetroundjoin%
\definecolor{currentfill}{rgb}{1.000000,0.894118,0.788235}%
\pgfsetfillcolor{currentfill}%
\pgfsetlinewidth{0.000000pt}%
\definecolor{currentstroke}{rgb}{1.000000,0.894118,0.788235}%
\pgfsetstrokecolor{currentstroke}%
\pgfsetdash{}{0pt}%
\pgfpathmoveto{\pgfqpoint{3.026135in}{2.474379in}}%
\pgfpathlineto{\pgfqpoint{3.026135in}{2.602432in}}%
\pgfpathlineto{\pgfqpoint{2.926478in}{2.663210in}}%
\pgfpathlineto{\pgfqpoint{2.815582in}{2.599184in}}%
\pgfpathlineto{\pgfqpoint{3.026135in}{2.474379in}}%
\pgfpathclose%
\pgfusepath{fill}%
\end{pgfscope}%
\begin{pgfscope}%
\pgfpathrectangle{\pgfqpoint{0.765000in}{0.660000in}}{\pgfqpoint{4.620000in}{4.620000in}}%
\pgfusepath{clip}%
\pgfsetbuttcap%
\pgfsetroundjoin%
\definecolor{currentfill}{rgb}{1.000000,0.894118,0.788235}%
\pgfsetfillcolor{currentfill}%
\pgfsetlinewidth{0.000000pt}%
\definecolor{currentstroke}{rgb}{1.000000,0.894118,0.788235}%
\pgfsetstrokecolor{currentstroke}%
\pgfsetdash{}{0pt}%
\pgfpathmoveto{\pgfqpoint{2.804341in}{2.474379in}}%
\pgfpathlineto{\pgfqpoint{2.915238in}{2.410353in}}%
\pgfpathlineto{\pgfqpoint{2.815582in}{2.471132in}}%
\pgfpathlineto{\pgfqpoint{2.704685in}{2.535158in}}%
\pgfpathlineto{\pgfqpoint{2.804341in}{2.474379in}}%
\pgfpathclose%
\pgfusepath{fill}%
\end{pgfscope}%
\begin{pgfscope}%
\pgfpathrectangle{\pgfqpoint{0.765000in}{0.660000in}}{\pgfqpoint{4.620000in}{4.620000in}}%
\pgfusepath{clip}%
\pgfsetbuttcap%
\pgfsetroundjoin%
\definecolor{currentfill}{rgb}{1.000000,0.894118,0.788235}%
\pgfsetfillcolor{currentfill}%
\pgfsetlinewidth{0.000000pt}%
\definecolor{currentstroke}{rgb}{1.000000,0.894118,0.788235}%
\pgfsetstrokecolor{currentstroke}%
\pgfsetdash{}{0pt}%
\pgfpathmoveto{\pgfqpoint{2.915238in}{2.410353in}}%
\pgfpathlineto{\pgfqpoint{3.026135in}{2.474379in}}%
\pgfpathlineto{\pgfqpoint{2.926478in}{2.535158in}}%
\pgfpathlineto{\pgfqpoint{2.815582in}{2.471132in}}%
\pgfpathlineto{\pgfqpoint{2.915238in}{2.410353in}}%
\pgfpathclose%
\pgfusepath{fill}%
\end{pgfscope}%
\begin{pgfscope}%
\pgfpathrectangle{\pgfqpoint{0.765000in}{0.660000in}}{\pgfqpoint{4.620000in}{4.620000in}}%
\pgfusepath{clip}%
\pgfsetbuttcap%
\pgfsetroundjoin%
\definecolor{currentfill}{rgb}{1.000000,0.894118,0.788235}%
\pgfsetfillcolor{currentfill}%
\pgfsetlinewidth{0.000000pt}%
\definecolor{currentstroke}{rgb}{1.000000,0.894118,0.788235}%
\pgfsetstrokecolor{currentstroke}%
\pgfsetdash{}{0pt}%
\pgfpathmoveto{\pgfqpoint{2.915238in}{2.666458in}}%
\pgfpathlineto{\pgfqpoint{2.804341in}{2.602432in}}%
\pgfpathlineto{\pgfqpoint{2.704685in}{2.663210in}}%
\pgfpathlineto{\pgfqpoint{2.815582in}{2.727237in}}%
\pgfpathlineto{\pgfqpoint{2.915238in}{2.666458in}}%
\pgfpathclose%
\pgfusepath{fill}%
\end{pgfscope}%
\begin{pgfscope}%
\pgfpathrectangle{\pgfqpoint{0.765000in}{0.660000in}}{\pgfqpoint{4.620000in}{4.620000in}}%
\pgfusepath{clip}%
\pgfsetbuttcap%
\pgfsetroundjoin%
\definecolor{currentfill}{rgb}{1.000000,0.894118,0.788235}%
\pgfsetfillcolor{currentfill}%
\pgfsetlinewidth{0.000000pt}%
\definecolor{currentstroke}{rgb}{1.000000,0.894118,0.788235}%
\pgfsetstrokecolor{currentstroke}%
\pgfsetdash{}{0pt}%
\pgfpathmoveto{\pgfqpoint{3.026135in}{2.602432in}}%
\pgfpathlineto{\pgfqpoint{2.915238in}{2.666458in}}%
\pgfpathlineto{\pgfqpoint{2.815582in}{2.727237in}}%
\pgfpathlineto{\pgfqpoint{2.926478in}{2.663210in}}%
\pgfpathlineto{\pgfqpoint{3.026135in}{2.602432in}}%
\pgfpathclose%
\pgfusepath{fill}%
\end{pgfscope}%
\begin{pgfscope}%
\pgfpathrectangle{\pgfqpoint{0.765000in}{0.660000in}}{\pgfqpoint{4.620000in}{4.620000in}}%
\pgfusepath{clip}%
\pgfsetbuttcap%
\pgfsetroundjoin%
\definecolor{currentfill}{rgb}{1.000000,0.894118,0.788235}%
\pgfsetfillcolor{currentfill}%
\pgfsetlinewidth{0.000000pt}%
\definecolor{currentstroke}{rgb}{1.000000,0.894118,0.788235}%
\pgfsetstrokecolor{currentstroke}%
\pgfsetdash{}{0pt}%
\pgfpathmoveto{\pgfqpoint{2.804341in}{2.474379in}}%
\pgfpathlineto{\pgfqpoint{2.804341in}{2.602432in}}%
\pgfpathlineto{\pgfqpoint{2.704685in}{2.663210in}}%
\pgfpathlineto{\pgfqpoint{2.815582in}{2.471132in}}%
\pgfpathlineto{\pgfqpoint{2.804341in}{2.474379in}}%
\pgfpathclose%
\pgfusepath{fill}%
\end{pgfscope}%
\begin{pgfscope}%
\pgfpathrectangle{\pgfqpoint{0.765000in}{0.660000in}}{\pgfqpoint{4.620000in}{4.620000in}}%
\pgfusepath{clip}%
\pgfsetbuttcap%
\pgfsetroundjoin%
\definecolor{currentfill}{rgb}{1.000000,0.894118,0.788235}%
\pgfsetfillcolor{currentfill}%
\pgfsetlinewidth{0.000000pt}%
\definecolor{currentstroke}{rgb}{1.000000,0.894118,0.788235}%
\pgfsetstrokecolor{currentstroke}%
\pgfsetdash{}{0pt}%
\pgfpathmoveto{\pgfqpoint{2.915238in}{2.410353in}}%
\pgfpathlineto{\pgfqpoint{2.915238in}{2.538405in}}%
\pgfpathlineto{\pgfqpoint{2.815582in}{2.599184in}}%
\pgfpathlineto{\pgfqpoint{2.704685in}{2.535158in}}%
\pgfpathlineto{\pgfqpoint{2.915238in}{2.410353in}}%
\pgfpathclose%
\pgfusepath{fill}%
\end{pgfscope}%
\begin{pgfscope}%
\pgfpathrectangle{\pgfqpoint{0.765000in}{0.660000in}}{\pgfqpoint{4.620000in}{4.620000in}}%
\pgfusepath{clip}%
\pgfsetbuttcap%
\pgfsetroundjoin%
\definecolor{currentfill}{rgb}{1.000000,0.894118,0.788235}%
\pgfsetfillcolor{currentfill}%
\pgfsetlinewidth{0.000000pt}%
\definecolor{currentstroke}{rgb}{1.000000,0.894118,0.788235}%
\pgfsetstrokecolor{currentstroke}%
\pgfsetdash{}{0pt}%
\pgfpathmoveto{\pgfqpoint{2.915238in}{2.538405in}}%
\pgfpathlineto{\pgfqpoint{3.026135in}{2.602432in}}%
\pgfpathlineto{\pgfqpoint{2.926478in}{2.663210in}}%
\pgfpathlineto{\pgfqpoint{2.815582in}{2.599184in}}%
\pgfpathlineto{\pgfqpoint{2.915238in}{2.538405in}}%
\pgfpathclose%
\pgfusepath{fill}%
\end{pgfscope}%
\begin{pgfscope}%
\pgfpathrectangle{\pgfqpoint{0.765000in}{0.660000in}}{\pgfqpoint{4.620000in}{4.620000in}}%
\pgfusepath{clip}%
\pgfsetbuttcap%
\pgfsetroundjoin%
\definecolor{currentfill}{rgb}{1.000000,0.894118,0.788235}%
\pgfsetfillcolor{currentfill}%
\pgfsetlinewidth{0.000000pt}%
\definecolor{currentstroke}{rgb}{1.000000,0.894118,0.788235}%
\pgfsetstrokecolor{currentstroke}%
\pgfsetdash{}{0pt}%
\pgfpathmoveto{\pgfqpoint{2.804341in}{2.602432in}}%
\pgfpathlineto{\pgfqpoint{2.915238in}{2.538405in}}%
\pgfpathlineto{\pgfqpoint{2.815582in}{2.599184in}}%
\pgfpathlineto{\pgfqpoint{2.704685in}{2.663210in}}%
\pgfpathlineto{\pgfqpoint{2.804341in}{2.602432in}}%
\pgfpathclose%
\pgfusepath{fill}%
\end{pgfscope}%
\begin{pgfscope}%
\pgfpathrectangle{\pgfqpoint{0.765000in}{0.660000in}}{\pgfqpoint{4.620000in}{4.620000in}}%
\pgfusepath{clip}%
\pgfsetbuttcap%
\pgfsetroundjoin%
\definecolor{currentfill}{rgb}{1.000000,0.894118,0.788235}%
\pgfsetfillcolor{currentfill}%
\pgfsetlinewidth{0.000000pt}%
\definecolor{currentstroke}{rgb}{1.000000,0.894118,0.788235}%
\pgfsetstrokecolor{currentstroke}%
\pgfsetdash{}{0pt}%
\pgfpathmoveto{\pgfqpoint{2.583171in}{3.110796in}}%
\pgfpathlineto{\pgfqpoint{2.472275in}{3.046769in}}%
\pgfpathlineto{\pgfqpoint{2.583171in}{2.982743in}}%
\pgfpathlineto{\pgfqpoint{2.694068in}{3.046769in}}%
\pgfpathlineto{\pgfqpoint{2.583171in}{3.110796in}}%
\pgfpathclose%
\pgfusepath{fill}%
\end{pgfscope}%
\begin{pgfscope}%
\pgfpathrectangle{\pgfqpoint{0.765000in}{0.660000in}}{\pgfqpoint{4.620000in}{4.620000in}}%
\pgfusepath{clip}%
\pgfsetbuttcap%
\pgfsetroundjoin%
\definecolor{currentfill}{rgb}{1.000000,0.894118,0.788235}%
\pgfsetfillcolor{currentfill}%
\pgfsetlinewidth{0.000000pt}%
\definecolor{currentstroke}{rgb}{1.000000,0.894118,0.788235}%
\pgfsetstrokecolor{currentstroke}%
\pgfsetdash{}{0pt}%
\pgfpathmoveto{\pgfqpoint{2.583171in}{3.110796in}}%
\pgfpathlineto{\pgfqpoint{2.472275in}{3.046769in}}%
\pgfpathlineto{\pgfqpoint{2.472275in}{3.174822in}}%
\pgfpathlineto{\pgfqpoint{2.583171in}{3.238848in}}%
\pgfpathlineto{\pgfqpoint{2.583171in}{3.110796in}}%
\pgfpathclose%
\pgfusepath{fill}%
\end{pgfscope}%
\begin{pgfscope}%
\pgfpathrectangle{\pgfqpoint{0.765000in}{0.660000in}}{\pgfqpoint{4.620000in}{4.620000in}}%
\pgfusepath{clip}%
\pgfsetbuttcap%
\pgfsetroundjoin%
\definecolor{currentfill}{rgb}{1.000000,0.894118,0.788235}%
\pgfsetfillcolor{currentfill}%
\pgfsetlinewidth{0.000000pt}%
\definecolor{currentstroke}{rgb}{1.000000,0.894118,0.788235}%
\pgfsetstrokecolor{currentstroke}%
\pgfsetdash{}{0pt}%
\pgfpathmoveto{\pgfqpoint{2.583171in}{3.110796in}}%
\pgfpathlineto{\pgfqpoint{2.694068in}{3.046769in}}%
\pgfpathlineto{\pgfqpoint{2.694068in}{3.174822in}}%
\pgfpathlineto{\pgfqpoint{2.583171in}{3.238848in}}%
\pgfpathlineto{\pgfqpoint{2.583171in}{3.110796in}}%
\pgfpathclose%
\pgfusepath{fill}%
\end{pgfscope}%
\begin{pgfscope}%
\pgfpathrectangle{\pgfqpoint{0.765000in}{0.660000in}}{\pgfqpoint{4.620000in}{4.620000in}}%
\pgfusepath{clip}%
\pgfsetbuttcap%
\pgfsetroundjoin%
\definecolor{currentfill}{rgb}{1.000000,0.894118,0.788235}%
\pgfsetfillcolor{currentfill}%
\pgfsetlinewidth{0.000000pt}%
\definecolor{currentstroke}{rgb}{1.000000,0.894118,0.788235}%
\pgfsetstrokecolor{currentstroke}%
\pgfsetdash{}{0pt}%
\pgfpathmoveto{\pgfqpoint{2.586744in}{3.073269in}}%
\pgfpathlineto{\pgfqpoint{2.475847in}{3.009242in}}%
\pgfpathlineto{\pgfqpoint{2.586744in}{2.945216in}}%
\pgfpathlineto{\pgfqpoint{2.697640in}{3.009242in}}%
\pgfpathlineto{\pgfqpoint{2.586744in}{3.073269in}}%
\pgfpathclose%
\pgfusepath{fill}%
\end{pgfscope}%
\begin{pgfscope}%
\pgfpathrectangle{\pgfqpoint{0.765000in}{0.660000in}}{\pgfqpoint{4.620000in}{4.620000in}}%
\pgfusepath{clip}%
\pgfsetbuttcap%
\pgfsetroundjoin%
\definecolor{currentfill}{rgb}{1.000000,0.894118,0.788235}%
\pgfsetfillcolor{currentfill}%
\pgfsetlinewidth{0.000000pt}%
\definecolor{currentstroke}{rgb}{1.000000,0.894118,0.788235}%
\pgfsetstrokecolor{currentstroke}%
\pgfsetdash{}{0pt}%
\pgfpathmoveto{\pgfqpoint{2.586744in}{3.073269in}}%
\pgfpathlineto{\pgfqpoint{2.475847in}{3.009242in}}%
\pgfpathlineto{\pgfqpoint{2.475847in}{3.137295in}}%
\pgfpathlineto{\pgfqpoint{2.586744in}{3.201321in}}%
\pgfpathlineto{\pgfqpoint{2.586744in}{3.073269in}}%
\pgfpathclose%
\pgfusepath{fill}%
\end{pgfscope}%
\begin{pgfscope}%
\pgfpathrectangle{\pgfqpoint{0.765000in}{0.660000in}}{\pgfqpoint{4.620000in}{4.620000in}}%
\pgfusepath{clip}%
\pgfsetbuttcap%
\pgfsetroundjoin%
\definecolor{currentfill}{rgb}{1.000000,0.894118,0.788235}%
\pgfsetfillcolor{currentfill}%
\pgfsetlinewidth{0.000000pt}%
\definecolor{currentstroke}{rgb}{1.000000,0.894118,0.788235}%
\pgfsetstrokecolor{currentstroke}%
\pgfsetdash{}{0pt}%
\pgfpathmoveto{\pgfqpoint{2.586744in}{3.073269in}}%
\pgfpathlineto{\pgfqpoint{2.697640in}{3.009242in}}%
\pgfpathlineto{\pgfqpoint{2.697640in}{3.137295in}}%
\pgfpathlineto{\pgfqpoint{2.586744in}{3.201321in}}%
\pgfpathlineto{\pgfqpoint{2.586744in}{3.073269in}}%
\pgfpathclose%
\pgfusepath{fill}%
\end{pgfscope}%
\begin{pgfscope}%
\pgfpathrectangle{\pgfqpoint{0.765000in}{0.660000in}}{\pgfqpoint{4.620000in}{4.620000in}}%
\pgfusepath{clip}%
\pgfsetbuttcap%
\pgfsetroundjoin%
\definecolor{currentfill}{rgb}{1.000000,0.894118,0.788235}%
\pgfsetfillcolor{currentfill}%
\pgfsetlinewidth{0.000000pt}%
\definecolor{currentstroke}{rgb}{1.000000,0.894118,0.788235}%
\pgfsetstrokecolor{currentstroke}%
\pgfsetdash{}{0pt}%
\pgfpathmoveto{\pgfqpoint{2.583171in}{3.238848in}}%
\pgfpathlineto{\pgfqpoint{2.472275in}{3.174822in}}%
\pgfpathlineto{\pgfqpoint{2.583171in}{3.110796in}}%
\pgfpathlineto{\pgfqpoint{2.694068in}{3.174822in}}%
\pgfpathlineto{\pgfqpoint{2.583171in}{3.238848in}}%
\pgfpathclose%
\pgfusepath{fill}%
\end{pgfscope}%
\begin{pgfscope}%
\pgfpathrectangle{\pgfqpoint{0.765000in}{0.660000in}}{\pgfqpoint{4.620000in}{4.620000in}}%
\pgfusepath{clip}%
\pgfsetbuttcap%
\pgfsetroundjoin%
\definecolor{currentfill}{rgb}{1.000000,0.894118,0.788235}%
\pgfsetfillcolor{currentfill}%
\pgfsetlinewidth{0.000000pt}%
\definecolor{currentstroke}{rgb}{1.000000,0.894118,0.788235}%
\pgfsetstrokecolor{currentstroke}%
\pgfsetdash{}{0pt}%
\pgfpathmoveto{\pgfqpoint{2.583171in}{2.982743in}}%
\pgfpathlineto{\pgfqpoint{2.694068in}{3.046769in}}%
\pgfpathlineto{\pgfqpoint{2.694068in}{3.174822in}}%
\pgfpathlineto{\pgfqpoint{2.583171in}{3.110796in}}%
\pgfpathlineto{\pgfqpoint{2.583171in}{2.982743in}}%
\pgfpathclose%
\pgfusepath{fill}%
\end{pgfscope}%
\begin{pgfscope}%
\pgfpathrectangle{\pgfqpoint{0.765000in}{0.660000in}}{\pgfqpoint{4.620000in}{4.620000in}}%
\pgfusepath{clip}%
\pgfsetbuttcap%
\pgfsetroundjoin%
\definecolor{currentfill}{rgb}{1.000000,0.894118,0.788235}%
\pgfsetfillcolor{currentfill}%
\pgfsetlinewidth{0.000000pt}%
\definecolor{currentstroke}{rgb}{1.000000,0.894118,0.788235}%
\pgfsetstrokecolor{currentstroke}%
\pgfsetdash{}{0pt}%
\pgfpathmoveto{\pgfqpoint{2.472275in}{3.046769in}}%
\pgfpathlineto{\pgfqpoint{2.583171in}{2.982743in}}%
\pgfpathlineto{\pgfqpoint{2.583171in}{3.110796in}}%
\pgfpathlineto{\pgfqpoint{2.472275in}{3.174822in}}%
\pgfpathlineto{\pgfqpoint{2.472275in}{3.046769in}}%
\pgfpathclose%
\pgfusepath{fill}%
\end{pgfscope}%
\begin{pgfscope}%
\pgfpathrectangle{\pgfqpoint{0.765000in}{0.660000in}}{\pgfqpoint{4.620000in}{4.620000in}}%
\pgfusepath{clip}%
\pgfsetbuttcap%
\pgfsetroundjoin%
\definecolor{currentfill}{rgb}{1.000000,0.894118,0.788235}%
\pgfsetfillcolor{currentfill}%
\pgfsetlinewidth{0.000000pt}%
\definecolor{currentstroke}{rgb}{1.000000,0.894118,0.788235}%
\pgfsetstrokecolor{currentstroke}%
\pgfsetdash{}{0pt}%
\pgfpathmoveto{\pgfqpoint{2.586744in}{3.201321in}}%
\pgfpathlineto{\pgfqpoint{2.475847in}{3.137295in}}%
\pgfpathlineto{\pgfqpoint{2.586744in}{3.073269in}}%
\pgfpathlineto{\pgfqpoint{2.697640in}{3.137295in}}%
\pgfpathlineto{\pgfqpoint{2.586744in}{3.201321in}}%
\pgfpathclose%
\pgfusepath{fill}%
\end{pgfscope}%
\begin{pgfscope}%
\pgfpathrectangle{\pgfqpoint{0.765000in}{0.660000in}}{\pgfqpoint{4.620000in}{4.620000in}}%
\pgfusepath{clip}%
\pgfsetbuttcap%
\pgfsetroundjoin%
\definecolor{currentfill}{rgb}{1.000000,0.894118,0.788235}%
\pgfsetfillcolor{currentfill}%
\pgfsetlinewidth{0.000000pt}%
\definecolor{currentstroke}{rgb}{1.000000,0.894118,0.788235}%
\pgfsetstrokecolor{currentstroke}%
\pgfsetdash{}{0pt}%
\pgfpathmoveto{\pgfqpoint{2.586744in}{2.945216in}}%
\pgfpathlineto{\pgfqpoint{2.697640in}{3.009242in}}%
\pgfpathlineto{\pgfqpoint{2.697640in}{3.137295in}}%
\pgfpathlineto{\pgfqpoint{2.586744in}{3.073269in}}%
\pgfpathlineto{\pgfqpoint{2.586744in}{2.945216in}}%
\pgfpathclose%
\pgfusepath{fill}%
\end{pgfscope}%
\begin{pgfscope}%
\pgfpathrectangle{\pgfqpoint{0.765000in}{0.660000in}}{\pgfqpoint{4.620000in}{4.620000in}}%
\pgfusepath{clip}%
\pgfsetbuttcap%
\pgfsetroundjoin%
\definecolor{currentfill}{rgb}{1.000000,0.894118,0.788235}%
\pgfsetfillcolor{currentfill}%
\pgfsetlinewidth{0.000000pt}%
\definecolor{currentstroke}{rgb}{1.000000,0.894118,0.788235}%
\pgfsetstrokecolor{currentstroke}%
\pgfsetdash{}{0pt}%
\pgfpathmoveto{\pgfqpoint{2.475847in}{3.009242in}}%
\pgfpathlineto{\pgfqpoint{2.586744in}{2.945216in}}%
\pgfpathlineto{\pgfqpoint{2.586744in}{3.073269in}}%
\pgfpathlineto{\pgfqpoint{2.475847in}{3.137295in}}%
\pgfpathlineto{\pgfqpoint{2.475847in}{3.009242in}}%
\pgfpathclose%
\pgfusepath{fill}%
\end{pgfscope}%
\begin{pgfscope}%
\pgfpathrectangle{\pgfqpoint{0.765000in}{0.660000in}}{\pgfqpoint{4.620000in}{4.620000in}}%
\pgfusepath{clip}%
\pgfsetbuttcap%
\pgfsetroundjoin%
\definecolor{currentfill}{rgb}{1.000000,0.894118,0.788235}%
\pgfsetfillcolor{currentfill}%
\pgfsetlinewidth{0.000000pt}%
\definecolor{currentstroke}{rgb}{1.000000,0.894118,0.788235}%
\pgfsetstrokecolor{currentstroke}%
\pgfsetdash{}{0pt}%
\pgfpathmoveto{\pgfqpoint{2.583171in}{3.110796in}}%
\pgfpathlineto{\pgfqpoint{2.472275in}{3.046769in}}%
\pgfpathlineto{\pgfqpoint{2.475847in}{3.009242in}}%
\pgfpathlineto{\pgfqpoint{2.586744in}{3.073269in}}%
\pgfpathlineto{\pgfqpoint{2.583171in}{3.110796in}}%
\pgfpathclose%
\pgfusepath{fill}%
\end{pgfscope}%
\begin{pgfscope}%
\pgfpathrectangle{\pgfqpoint{0.765000in}{0.660000in}}{\pgfqpoint{4.620000in}{4.620000in}}%
\pgfusepath{clip}%
\pgfsetbuttcap%
\pgfsetroundjoin%
\definecolor{currentfill}{rgb}{1.000000,0.894118,0.788235}%
\pgfsetfillcolor{currentfill}%
\pgfsetlinewidth{0.000000pt}%
\definecolor{currentstroke}{rgb}{1.000000,0.894118,0.788235}%
\pgfsetstrokecolor{currentstroke}%
\pgfsetdash{}{0pt}%
\pgfpathmoveto{\pgfqpoint{2.694068in}{3.046769in}}%
\pgfpathlineto{\pgfqpoint{2.583171in}{3.110796in}}%
\pgfpathlineto{\pgfqpoint{2.586744in}{3.073269in}}%
\pgfpathlineto{\pgfqpoint{2.697640in}{3.009242in}}%
\pgfpathlineto{\pgfqpoint{2.694068in}{3.046769in}}%
\pgfpathclose%
\pgfusepath{fill}%
\end{pgfscope}%
\begin{pgfscope}%
\pgfpathrectangle{\pgfqpoint{0.765000in}{0.660000in}}{\pgfqpoint{4.620000in}{4.620000in}}%
\pgfusepath{clip}%
\pgfsetbuttcap%
\pgfsetroundjoin%
\definecolor{currentfill}{rgb}{1.000000,0.894118,0.788235}%
\pgfsetfillcolor{currentfill}%
\pgfsetlinewidth{0.000000pt}%
\definecolor{currentstroke}{rgb}{1.000000,0.894118,0.788235}%
\pgfsetstrokecolor{currentstroke}%
\pgfsetdash{}{0pt}%
\pgfpathmoveto{\pgfqpoint{2.583171in}{3.110796in}}%
\pgfpathlineto{\pgfqpoint{2.583171in}{3.238848in}}%
\pgfpathlineto{\pgfqpoint{2.586744in}{3.201321in}}%
\pgfpathlineto{\pgfqpoint{2.697640in}{3.009242in}}%
\pgfpathlineto{\pgfqpoint{2.583171in}{3.110796in}}%
\pgfpathclose%
\pgfusepath{fill}%
\end{pgfscope}%
\begin{pgfscope}%
\pgfpathrectangle{\pgfqpoint{0.765000in}{0.660000in}}{\pgfqpoint{4.620000in}{4.620000in}}%
\pgfusepath{clip}%
\pgfsetbuttcap%
\pgfsetroundjoin%
\definecolor{currentfill}{rgb}{1.000000,0.894118,0.788235}%
\pgfsetfillcolor{currentfill}%
\pgfsetlinewidth{0.000000pt}%
\definecolor{currentstroke}{rgb}{1.000000,0.894118,0.788235}%
\pgfsetstrokecolor{currentstroke}%
\pgfsetdash{}{0pt}%
\pgfpathmoveto{\pgfqpoint{2.694068in}{3.046769in}}%
\pgfpathlineto{\pgfqpoint{2.694068in}{3.174822in}}%
\pgfpathlineto{\pgfqpoint{2.697640in}{3.137295in}}%
\pgfpathlineto{\pgfqpoint{2.586744in}{3.073269in}}%
\pgfpathlineto{\pgfqpoint{2.694068in}{3.046769in}}%
\pgfpathclose%
\pgfusepath{fill}%
\end{pgfscope}%
\begin{pgfscope}%
\pgfpathrectangle{\pgfqpoint{0.765000in}{0.660000in}}{\pgfqpoint{4.620000in}{4.620000in}}%
\pgfusepath{clip}%
\pgfsetbuttcap%
\pgfsetroundjoin%
\definecolor{currentfill}{rgb}{1.000000,0.894118,0.788235}%
\pgfsetfillcolor{currentfill}%
\pgfsetlinewidth{0.000000pt}%
\definecolor{currentstroke}{rgb}{1.000000,0.894118,0.788235}%
\pgfsetstrokecolor{currentstroke}%
\pgfsetdash{}{0pt}%
\pgfpathmoveto{\pgfqpoint{2.472275in}{3.046769in}}%
\pgfpathlineto{\pgfqpoint{2.583171in}{2.982743in}}%
\pgfpathlineto{\pgfqpoint{2.586744in}{2.945216in}}%
\pgfpathlineto{\pgfqpoint{2.475847in}{3.009242in}}%
\pgfpathlineto{\pgfqpoint{2.472275in}{3.046769in}}%
\pgfpathclose%
\pgfusepath{fill}%
\end{pgfscope}%
\begin{pgfscope}%
\pgfpathrectangle{\pgfqpoint{0.765000in}{0.660000in}}{\pgfqpoint{4.620000in}{4.620000in}}%
\pgfusepath{clip}%
\pgfsetbuttcap%
\pgfsetroundjoin%
\definecolor{currentfill}{rgb}{1.000000,0.894118,0.788235}%
\pgfsetfillcolor{currentfill}%
\pgfsetlinewidth{0.000000pt}%
\definecolor{currentstroke}{rgb}{1.000000,0.894118,0.788235}%
\pgfsetstrokecolor{currentstroke}%
\pgfsetdash{}{0pt}%
\pgfpathmoveto{\pgfqpoint{2.583171in}{2.982743in}}%
\pgfpathlineto{\pgfqpoint{2.694068in}{3.046769in}}%
\pgfpathlineto{\pgfqpoint{2.697640in}{3.009242in}}%
\pgfpathlineto{\pgfqpoint{2.586744in}{2.945216in}}%
\pgfpathlineto{\pgfqpoint{2.583171in}{2.982743in}}%
\pgfpathclose%
\pgfusepath{fill}%
\end{pgfscope}%
\begin{pgfscope}%
\pgfpathrectangle{\pgfqpoint{0.765000in}{0.660000in}}{\pgfqpoint{4.620000in}{4.620000in}}%
\pgfusepath{clip}%
\pgfsetbuttcap%
\pgfsetroundjoin%
\definecolor{currentfill}{rgb}{1.000000,0.894118,0.788235}%
\pgfsetfillcolor{currentfill}%
\pgfsetlinewidth{0.000000pt}%
\definecolor{currentstroke}{rgb}{1.000000,0.894118,0.788235}%
\pgfsetstrokecolor{currentstroke}%
\pgfsetdash{}{0pt}%
\pgfpathmoveto{\pgfqpoint{2.583171in}{3.238848in}}%
\pgfpathlineto{\pgfqpoint{2.472275in}{3.174822in}}%
\pgfpathlineto{\pgfqpoint{2.475847in}{3.137295in}}%
\pgfpathlineto{\pgfqpoint{2.586744in}{3.201321in}}%
\pgfpathlineto{\pgfqpoint{2.583171in}{3.238848in}}%
\pgfpathclose%
\pgfusepath{fill}%
\end{pgfscope}%
\begin{pgfscope}%
\pgfpathrectangle{\pgfqpoint{0.765000in}{0.660000in}}{\pgfqpoint{4.620000in}{4.620000in}}%
\pgfusepath{clip}%
\pgfsetbuttcap%
\pgfsetroundjoin%
\definecolor{currentfill}{rgb}{1.000000,0.894118,0.788235}%
\pgfsetfillcolor{currentfill}%
\pgfsetlinewidth{0.000000pt}%
\definecolor{currentstroke}{rgb}{1.000000,0.894118,0.788235}%
\pgfsetstrokecolor{currentstroke}%
\pgfsetdash{}{0pt}%
\pgfpathmoveto{\pgfqpoint{2.694068in}{3.174822in}}%
\pgfpathlineto{\pgfqpoint{2.583171in}{3.238848in}}%
\pgfpathlineto{\pgfqpoint{2.586744in}{3.201321in}}%
\pgfpathlineto{\pgfqpoint{2.697640in}{3.137295in}}%
\pgfpathlineto{\pgfqpoint{2.694068in}{3.174822in}}%
\pgfpathclose%
\pgfusepath{fill}%
\end{pgfscope}%
\begin{pgfscope}%
\pgfpathrectangle{\pgfqpoint{0.765000in}{0.660000in}}{\pgfqpoint{4.620000in}{4.620000in}}%
\pgfusepath{clip}%
\pgfsetbuttcap%
\pgfsetroundjoin%
\definecolor{currentfill}{rgb}{1.000000,0.894118,0.788235}%
\pgfsetfillcolor{currentfill}%
\pgfsetlinewidth{0.000000pt}%
\definecolor{currentstroke}{rgb}{1.000000,0.894118,0.788235}%
\pgfsetstrokecolor{currentstroke}%
\pgfsetdash{}{0pt}%
\pgfpathmoveto{\pgfqpoint{2.472275in}{3.046769in}}%
\pgfpathlineto{\pgfqpoint{2.472275in}{3.174822in}}%
\pgfpathlineto{\pgfqpoint{2.475847in}{3.137295in}}%
\pgfpathlineto{\pgfqpoint{2.586744in}{2.945216in}}%
\pgfpathlineto{\pgfqpoint{2.472275in}{3.046769in}}%
\pgfpathclose%
\pgfusepath{fill}%
\end{pgfscope}%
\begin{pgfscope}%
\pgfpathrectangle{\pgfqpoint{0.765000in}{0.660000in}}{\pgfqpoint{4.620000in}{4.620000in}}%
\pgfusepath{clip}%
\pgfsetbuttcap%
\pgfsetroundjoin%
\definecolor{currentfill}{rgb}{1.000000,0.894118,0.788235}%
\pgfsetfillcolor{currentfill}%
\pgfsetlinewidth{0.000000pt}%
\definecolor{currentstroke}{rgb}{1.000000,0.894118,0.788235}%
\pgfsetstrokecolor{currentstroke}%
\pgfsetdash{}{0pt}%
\pgfpathmoveto{\pgfqpoint{2.583171in}{2.982743in}}%
\pgfpathlineto{\pgfqpoint{2.583171in}{3.110796in}}%
\pgfpathlineto{\pgfqpoint{2.586744in}{3.073269in}}%
\pgfpathlineto{\pgfqpoint{2.475847in}{3.009242in}}%
\pgfpathlineto{\pgfqpoint{2.583171in}{2.982743in}}%
\pgfpathclose%
\pgfusepath{fill}%
\end{pgfscope}%
\begin{pgfscope}%
\pgfpathrectangle{\pgfqpoint{0.765000in}{0.660000in}}{\pgfqpoint{4.620000in}{4.620000in}}%
\pgfusepath{clip}%
\pgfsetbuttcap%
\pgfsetroundjoin%
\definecolor{currentfill}{rgb}{1.000000,0.894118,0.788235}%
\pgfsetfillcolor{currentfill}%
\pgfsetlinewidth{0.000000pt}%
\definecolor{currentstroke}{rgb}{1.000000,0.894118,0.788235}%
\pgfsetstrokecolor{currentstroke}%
\pgfsetdash{}{0pt}%
\pgfpathmoveto{\pgfqpoint{2.583171in}{3.110796in}}%
\pgfpathlineto{\pgfqpoint{2.694068in}{3.174822in}}%
\pgfpathlineto{\pgfqpoint{2.697640in}{3.137295in}}%
\pgfpathlineto{\pgfqpoint{2.586744in}{3.073269in}}%
\pgfpathlineto{\pgfqpoint{2.583171in}{3.110796in}}%
\pgfpathclose%
\pgfusepath{fill}%
\end{pgfscope}%
\begin{pgfscope}%
\pgfpathrectangle{\pgfqpoint{0.765000in}{0.660000in}}{\pgfqpoint{4.620000in}{4.620000in}}%
\pgfusepath{clip}%
\pgfsetbuttcap%
\pgfsetroundjoin%
\definecolor{currentfill}{rgb}{1.000000,0.894118,0.788235}%
\pgfsetfillcolor{currentfill}%
\pgfsetlinewidth{0.000000pt}%
\definecolor{currentstroke}{rgb}{1.000000,0.894118,0.788235}%
\pgfsetstrokecolor{currentstroke}%
\pgfsetdash{}{0pt}%
\pgfpathmoveto{\pgfqpoint{2.472275in}{3.174822in}}%
\pgfpathlineto{\pgfqpoint{2.583171in}{3.110796in}}%
\pgfpathlineto{\pgfqpoint{2.586744in}{3.073269in}}%
\pgfpathlineto{\pgfqpoint{2.475847in}{3.137295in}}%
\pgfpathlineto{\pgfqpoint{2.472275in}{3.174822in}}%
\pgfpathclose%
\pgfusepath{fill}%
\end{pgfscope}%
\begin{pgfscope}%
\pgfpathrectangle{\pgfqpoint{0.765000in}{0.660000in}}{\pgfqpoint{4.620000in}{4.620000in}}%
\pgfusepath{clip}%
\pgfsetbuttcap%
\pgfsetroundjoin%
\definecolor{currentfill}{rgb}{1.000000,0.894118,0.788235}%
\pgfsetfillcolor{currentfill}%
\pgfsetlinewidth{0.000000pt}%
\definecolor{currentstroke}{rgb}{1.000000,0.894118,0.788235}%
\pgfsetstrokecolor{currentstroke}%
\pgfsetdash{}{0pt}%
\pgfpathmoveto{\pgfqpoint{2.583171in}{3.110796in}}%
\pgfpathlineto{\pgfqpoint{2.472275in}{3.046769in}}%
\pgfpathlineto{\pgfqpoint{2.583171in}{2.982743in}}%
\pgfpathlineto{\pgfqpoint{2.694068in}{3.046769in}}%
\pgfpathlineto{\pgfqpoint{2.583171in}{3.110796in}}%
\pgfpathclose%
\pgfusepath{fill}%
\end{pgfscope}%
\begin{pgfscope}%
\pgfpathrectangle{\pgfqpoint{0.765000in}{0.660000in}}{\pgfqpoint{4.620000in}{4.620000in}}%
\pgfusepath{clip}%
\pgfsetbuttcap%
\pgfsetroundjoin%
\definecolor{currentfill}{rgb}{1.000000,0.894118,0.788235}%
\pgfsetfillcolor{currentfill}%
\pgfsetlinewidth{0.000000pt}%
\definecolor{currentstroke}{rgb}{1.000000,0.894118,0.788235}%
\pgfsetstrokecolor{currentstroke}%
\pgfsetdash{}{0pt}%
\pgfpathmoveto{\pgfqpoint{2.583171in}{3.110796in}}%
\pgfpathlineto{\pgfqpoint{2.472275in}{3.046769in}}%
\pgfpathlineto{\pgfqpoint{2.472275in}{3.174822in}}%
\pgfpathlineto{\pgfqpoint{2.583171in}{3.238848in}}%
\pgfpathlineto{\pgfqpoint{2.583171in}{3.110796in}}%
\pgfpathclose%
\pgfusepath{fill}%
\end{pgfscope}%
\begin{pgfscope}%
\pgfpathrectangle{\pgfqpoint{0.765000in}{0.660000in}}{\pgfqpoint{4.620000in}{4.620000in}}%
\pgfusepath{clip}%
\pgfsetbuttcap%
\pgfsetroundjoin%
\definecolor{currentfill}{rgb}{1.000000,0.894118,0.788235}%
\pgfsetfillcolor{currentfill}%
\pgfsetlinewidth{0.000000pt}%
\definecolor{currentstroke}{rgb}{1.000000,0.894118,0.788235}%
\pgfsetstrokecolor{currentstroke}%
\pgfsetdash{}{0pt}%
\pgfpathmoveto{\pgfqpoint{2.583171in}{3.110796in}}%
\pgfpathlineto{\pgfqpoint{2.694068in}{3.046769in}}%
\pgfpathlineto{\pgfqpoint{2.694068in}{3.174822in}}%
\pgfpathlineto{\pgfqpoint{2.583171in}{3.238848in}}%
\pgfpathlineto{\pgfqpoint{2.583171in}{3.110796in}}%
\pgfpathclose%
\pgfusepath{fill}%
\end{pgfscope}%
\begin{pgfscope}%
\pgfpathrectangle{\pgfqpoint{0.765000in}{0.660000in}}{\pgfqpoint{4.620000in}{4.620000in}}%
\pgfusepath{clip}%
\pgfsetbuttcap%
\pgfsetroundjoin%
\definecolor{currentfill}{rgb}{1.000000,0.894118,0.788235}%
\pgfsetfillcolor{currentfill}%
\pgfsetlinewidth{0.000000pt}%
\definecolor{currentstroke}{rgb}{1.000000,0.894118,0.788235}%
\pgfsetstrokecolor{currentstroke}%
\pgfsetdash{}{0pt}%
\pgfpathmoveto{\pgfqpoint{2.582969in}{3.161487in}}%
\pgfpathlineto{\pgfqpoint{2.472072in}{3.097461in}}%
\pgfpathlineto{\pgfqpoint{2.582969in}{3.033435in}}%
\pgfpathlineto{\pgfqpoint{2.693865in}{3.097461in}}%
\pgfpathlineto{\pgfqpoint{2.582969in}{3.161487in}}%
\pgfpathclose%
\pgfusepath{fill}%
\end{pgfscope}%
\begin{pgfscope}%
\pgfpathrectangle{\pgfqpoint{0.765000in}{0.660000in}}{\pgfqpoint{4.620000in}{4.620000in}}%
\pgfusepath{clip}%
\pgfsetbuttcap%
\pgfsetroundjoin%
\definecolor{currentfill}{rgb}{1.000000,0.894118,0.788235}%
\pgfsetfillcolor{currentfill}%
\pgfsetlinewidth{0.000000pt}%
\definecolor{currentstroke}{rgb}{1.000000,0.894118,0.788235}%
\pgfsetstrokecolor{currentstroke}%
\pgfsetdash{}{0pt}%
\pgfpathmoveto{\pgfqpoint{2.582969in}{3.161487in}}%
\pgfpathlineto{\pgfqpoint{2.472072in}{3.097461in}}%
\pgfpathlineto{\pgfqpoint{2.472072in}{3.225514in}}%
\pgfpathlineto{\pgfqpoint{2.582969in}{3.289540in}}%
\pgfpathlineto{\pgfqpoint{2.582969in}{3.161487in}}%
\pgfpathclose%
\pgfusepath{fill}%
\end{pgfscope}%
\begin{pgfscope}%
\pgfpathrectangle{\pgfqpoint{0.765000in}{0.660000in}}{\pgfqpoint{4.620000in}{4.620000in}}%
\pgfusepath{clip}%
\pgfsetbuttcap%
\pgfsetroundjoin%
\definecolor{currentfill}{rgb}{1.000000,0.894118,0.788235}%
\pgfsetfillcolor{currentfill}%
\pgfsetlinewidth{0.000000pt}%
\definecolor{currentstroke}{rgb}{1.000000,0.894118,0.788235}%
\pgfsetstrokecolor{currentstroke}%
\pgfsetdash{}{0pt}%
\pgfpathmoveto{\pgfqpoint{2.582969in}{3.161487in}}%
\pgfpathlineto{\pgfqpoint{2.693865in}{3.097461in}}%
\pgfpathlineto{\pgfqpoint{2.693865in}{3.225514in}}%
\pgfpathlineto{\pgfqpoint{2.582969in}{3.289540in}}%
\pgfpathlineto{\pgfqpoint{2.582969in}{3.161487in}}%
\pgfpathclose%
\pgfusepath{fill}%
\end{pgfscope}%
\begin{pgfscope}%
\pgfpathrectangle{\pgfqpoint{0.765000in}{0.660000in}}{\pgfqpoint{4.620000in}{4.620000in}}%
\pgfusepath{clip}%
\pgfsetbuttcap%
\pgfsetroundjoin%
\definecolor{currentfill}{rgb}{1.000000,0.894118,0.788235}%
\pgfsetfillcolor{currentfill}%
\pgfsetlinewidth{0.000000pt}%
\definecolor{currentstroke}{rgb}{1.000000,0.894118,0.788235}%
\pgfsetstrokecolor{currentstroke}%
\pgfsetdash{}{0pt}%
\pgfpathmoveto{\pgfqpoint{2.583171in}{3.238848in}}%
\pgfpathlineto{\pgfqpoint{2.472275in}{3.174822in}}%
\pgfpathlineto{\pgfqpoint{2.583171in}{3.110796in}}%
\pgfpathlineto{\pgfqpoint{2.694068in}{3.174822in}}%
\pgfpathlineto{\pgfqpoint{2.583171in}{3.238848in}}%
\pgfpathclose%
\pgfusepath{fill}%
\end{pgfscope}%
\begin{pgfscope}%
\pgfpathrectangle{\pgfqpoint{0.765000in}{0.660000in}}{\pgfqpoint{4.620000in}{4.620000in}}%
\pgfusepath{clip}%
\pgfsetbuttcap%
\pgfsetroundjoin%
\definecolor{currentfill}{rgb}{1.000000,0.894118,0.788235}%
\pgfsetfillcolor{currentfill}%
\pgfsetlinewidth{0.000000pt}%
\definecolor{currentstroke}{rgb}{1.000000,0.894118,0.788235}%
\pgfsetstrokecolor{currentstroke}%
\pgfsetdash{}{0pt}%
\pgfpathmoveto{\pgfqpoint{2.583171in}{2.982743in}}%
\pgfpathlineto{\pgfqpoint{2.694068in}{3.046769in}}%
\pgfpathlineto{\pgfqpoint{2.694068in}{3.174822in}}%
\pgfpathlineto{\pgfqpoint{2.583171in}{3.110796in}}%
\pgfpathlineto{\pgfqpoint{2.583171in}{2.982743in}}%
\pgfpathclose%
\pgfusepath{fill}%
\end{pgfscope}%
\begin{pgfscope}%
\pgfpathrectangle{\pgfqpoint{0.765000in}{0.660000in}}{\pgfqpoint{4.620000in}{4.620000in}}%
\pgfusepath{clip}%
\pgfsetbuttcap%
\pgfsetroundjoin%
\definecolor{currentfill}{rgb}{1.000000,0.894118,0.788235}%
\pgfsetfillcolor{currentfill}%
\pgfsetlinewidth{0.000000pt}%
\definecolor{currentstroke}{rgb}{1.000000,0.894118,0.788235}%
\pgfsetstrokecolor{currentstroke}%
\pgfsetdash{}{0pt}%
\pgfpathmoveto{\pgfqpoint{2.472275in}{3.046769in}}%
\pgfpathlineto{\pgfqpoint{2.583171in}{2.982743in}}%
\pgfpathlineto{\pgfqpoint{2.583171in}{3.110796in}}%
\pgfpathlineto{\pgfqpoint{2.472275in}{3.174822in}}%
\pgfpathlineto{\pgfqpoint{2.472275in}{3.046769in}}%
\pgfpathclose%
\pgfusepath{fill}%
\end{pgfscope}%
\begin{pgfscope}%
\pgfpathrectangle{\pgfqpoint{0.765000in}{0.660000in}}{\pgfqpoint{4.620000in}{4.620000in}}%
\pgfusepath{clip}%
\pgfsetbuttcap%
\pgfsetroundjoin%
\definecolor{currentfill}{rgb}{1.000000,0.894118,0.788235}%
\pgfsetfillcolor{currentfill}%
\pgfsetlinewidth{0.000000pt}%
\definecolor{currentstroke}{rgb}{1.000000,0.894118,0.788235}%
\pgfsetstrokecolor{currentstroke}%
\pgfsetdash{}{0pt}%
\pgfpathmoveto{\pgfqpoint{2.582969in}{3.289540in}}%
\pgfpathlineto{\pgfqpoint{2.472072in}{3.225514in}}%
\pgfpathlineto{\pgfqpoint{2.582969in}{3.161487in}}%
\pgfpathlineto{\pgfqpoint{2.693865in}{3.225514in}}%
\pgfpathlineto{\pgfqpoint{2.582969in}{3.289540in}}%
\pgfpathclose%
\pgfusepath{fill}%
\end{pgfscope}%
\begin{pgfscope}%
\pgfpathrectangle{\pgfqpoint{0.765000in}{0.660000in}}{\pgfqpoint{4.620000in}{4.620000in}}%
\pgfusepath{clip}%
\pgfsetbuttcap%
\pgfsetroundjoin%
\definecolor{currentfill}{rgb}{1.000000,0.894118,0.788235}%
\pgfsetfillcolor{currentfill}%
\pgfsetlinewidth{0.000000pt}%
\definecolor{currentstroke}{rgb}{1.000000,0.894118,0.788235}%
\pgfsetstrokecolor{currentstroke}%
\pgfsetdash{}{0pt}%
\pgfpathmoveto{\pgfqpoint{2.582969in}{3.033435in}}%
\pgfpathlineto{\pgfqpoint{2.693865in}{3.097461in}}%
\pgfpathlineto{\pgfqpoint{2.693865in}{3.225514in}}%
\pgfpathlineto{\pgfqpoint{2.582969in}{3.161487in}}%
\pgfpathlineto{\pgfqpoint{2.582969in}{3.033435in}}%
\pgfpathclose%
\pgfusepath{fill}%
\end{pgfscope}%
\begin{pgfscope}%
\pgfpathrectangle{\pgfqpoint{0.765000in}{0.660000in}}{\pgfqpoint{4.620000in}{4.620000in}}%
\pgfusepath{clip}%
\pgfsetbuttcap%
\pgfsetroundjoin%
\definecolor{currentfill}{rgb}{1.000000,0.894118,0.788235}%
\pgfsetfillcolor{currentfill}%
\pgfsetlinewidth{0.000000pt}%
\definecolor{currentstroke}{rgb}{1.000000,0.894118,0.788235}%
\pgfsetstrokecolor{currentstroke}%
\pgfsetdash{}{0pt}%
\pgfpathmoveto{\pgfqpoint{2.472072in}{3.097461in}}%
\pgfpathlineto{\pgfqpoint{2.582969in}{3.033435in}}%
\pgfpathlineto{\pgfqpoint{2.582969in}{3.161487in}}%
\pgfpathlineto{\pgfqpoint{2.472072in}{3.225514in}}%
\pgfpathlineto{\pgfqpoint{2.472072in}{3.097461in}}%
\pgfpathclose%
\pgfusepath{fill}%
\end{pgfscope}%
\begin{pgfscope}%
\pgfpathrectangle{\pgfqpoint{0.765000in}{0.660000in}}{\pgfqpoint{4.620000in}{4.620000in}}%
\pgfusepath{clip}%
\pgfsetbuttcap%
\pgfsetroundjoin%
\definecolor{currentfill}{rgb}{1.000000,0.894118,0.788235}%
\pgfsetfillcolor{currentfill}%
\pgfsetlinewidth{0.000000pt}%
\definecolor{currentstroke}{rgb}{1.000000,0.894118,0.788235}%
\pgfsetstrokecolor{currentstroke}%
\pgfsetdash{}{0pt}%
\pgfpathmoveto{\pgfqpoint{2.583171in}{3.110796in}}%
\pgfpathlineto{\pgfqpoint{2.472275in}{3.046769in}}%
\pgfpathlineto{\pgfqpoint{2.472072in}{3.097461in}}%
\pgfpathlineto{\pgfqpoint{2.582969in}{3.161487in}}%
\pgfpathlineto{\pgfqpoint{2.583171in}{3.110796in}}%
\pgfpathclose%
\pgfusepath{fill}%
\end{pgfscope}%
\begin{pgfscope}%
\pgfpathrectangle{\pgfqpoint{0.765000in}{0.660000in}}{\pgfqpoint{4.620000in}{4.620000in}}%
\pgfusepath{clip}%
\pgfsetbuttcap%
\pgfsetroundjoin%
\definecolor{currentfill}{rgb}{1.000000,0.894118,0.788235}%
\pgfsetfillcolor{currentfill}%
\pgfsetlinewidth{0.000000pt}%
\definecolor{currentstroke}{rgb}{1.000000,0.894118,0.788235}%
\pgfsetstrokecolor{currentstroke}%
\pgfsetdash{}{0pt}%
\pgfpathmoveto{\pgfqpoint{2.694068in}{3.046769in}}%
\pgfpathlineto{\pgfqpoint{2.583171in}{3.110796in}}%
\pgfpathlineto{\pgfqpoint{2.582969in}{3.161487in}}%
\pgfpathlineto{\pgfqpoint{2.693865in}{3.097461in}}%
\pgfpathlineto{\pgfqpoint{2.694068in}{3.046769in}}%
\pgfpathclose%
\pgfusepath{fill}%
\end{pgfscope}%
\begin{pgfscope}%
\pgfpathrectangle{\pgfqpoint{0.765000in}{0.660000in}}{\pgfqpoint{4.620000in}{4.620000in}}%
\pgfusepath{clip}%
\pgfsetbuttcap%
\pgfsetroundjoin%
\definecolor{currentfill}{rgb}{1.000000,0.894118,0.788235}%
\pgfsetfillcolor{currentfill}%
\pgfsetlinewidth{0.000000pt}%
\definecolor{currentstroke}{rgb}{1.000000,0.894118,0.788235}%
\pgfsetstrokecolor{currentstroke}%
\pgfsetdash{}{0pt}%
\pgfpathmoveto{\pgfqpoint{2.583171in}{3.110796in}}%
\pgfpathlineto{\pgfqpoint{2.583171in}{3.238848in}}%
\pgfpathlineto{\pgfqpoint{2.582969in}{3.289540in}}%
\pgfpathlineto{\pgfqpoint{2.693865in}{3.097461in}}%
\pgfpathlineto{\pgfqpoint{2.583171in}{3.110796in}}%
\pgfpathclose%
\pgfusepath{fill}%
\end{pgfscope}%
\begin{pgfscope}%
\pgfpathrectangle{\pgfqpoint{0.765000in}{0.660000in}}{\pgfqpoint{4.620000in}{4.620000in}}%
\pgfusepath{clip}%
\pgfsetbuttcap%
\pgfsetroundjoin%
\definecolor{currentfill}{rgb}{1.000000,0.894118,0.788235}%
\pgfsetfillcolor{currentfill}%
\pgfsetlinewidth{0.000000pt}%
\definecolor{currentstroke}{rgb}{1.000000,0.894118,0.788235}%
\pgfsetstrokecolor{currentstroke}%
\pgfsetdash{}{0pt}%
\pgfpathmoveto{\pgfqpoint{2.694068in}{3.046769in}}%
\pgfpathlineto{\pgfqpoint{2.694068in}{3.174822in}}%
\pgfpathlineto{\pgfqpoint{2.693865in}{3.225514in}}%
\pgfpathlineto{\pgfqpoint{2.582969in}{3.161487in}}%
\pgfpathlineto{\pgfqpoint{2.694068in}{3.046769in}}%
\pgfpathclose%
\pgfusepath{fill}%
\end{pgfscope}%
\begin{pgfscope}%
\pgfpathrectangle{\pgfqpoint{0.765000in}{0.660000in}}{\pgfqpoint{4.620000in}{4.620000in}}%
\pgfusepath{clip}%
\pgfsetbuttcap%
\pgfsetroundjoin%
\definecolor{currentfill}{rgb}{1.000000,0.894118,0.788235}%
\pgfsetfillcolor{currentfill}%
\pgfsetlinewidth{0.000000pt}%
\definecolor{currentstroke}{rgb}{1.000000,0.894118,0.788235}%
\pgfsetstrokecolor{currentstroke}%
\pgfsetdash{}{0pt}%
\pgfpathmoveto{\pgfqpoint{2.472275in}{3.046769in}}%
\pgfpathlineto{\pgfqpoint{2.583171in}{2.982743in}}%
\pgfpathlineto{\pgfqpoint{2.582969in}{3.033435in}}%
\pgfpathlineto{\pgfqpoint{2.472072in}{3.097461in}}%
\pgfpathlineto{\pgfqpoint{2.472275in}{3.046769in}}%
\pgfpathclose%
\pgfusepath{fill}%
\end{pgfscope}%
\begin{pgfscope}%
\pgfpathrectangle{\pgfqpoint{0.765000in}{0.660000in}}{\pgfqpoint{4.620000in}{4.620000in}}%
\pgfusepath{clip}%
\pgfsetbuttcap%
\pgfsetroundjoin%
\definecolor{currentfill}{rgb}{1.000000,0.894118,0.788235}%
\pgfsetfillcolor{currentfill}%
\pgfsetlinewidth{0.000000pt}%
\definecolor{currentstroke}{rgb}{1.000000,0.894118,0.788235}%
\pgfsetstrokecolor{currentstroke}%
\pgfsetdash{}{0pt}%
\pgfpathmoveto{\pgfqpoint{2.583171in}{2.982743in}}%
\pgfpathlineto{\pgfqpoint{2.694068in}{3.046769in}}%
\pgfpathlineto{\pgfqpoint{2.693865in}{3.097461in}}%
\pgfpathlineto{\pgfqpoint{2.582969in}{3.033435in}}%
\pgfpathlineto{\pgfqpoint{2.583171in}{2.982743in}}%
\pgfpathclose%
\pgfusepath{fill}%
\end{pgfscope}%
\begin{pgfscope}%
\pgfpathrectangle{\pgfqpoint{0.765000in}{0.660000in}}{\pgfqpoint{4.620000in}{4.620000in}}%
\pgfusepath{clip}%
\pgfsetbuttcap%
\pgfsetroundjoin%
\definecolor{currentfill}{rgb}{1.000000,0.894118,0.788235}%
\pgfsetfillcolor{currentfill}%
\pgfsetlinewidth{0.000000pt}%
\definecolor{currentstroke}{rgb}{1.000000,0.894118,0.788235}%
\pgfsetstrokecolor{currentstroke}%
\pgfsetdash{}{0pt}%
\pgfpathmoveto{\pgfqpoint{2.583171in}{3.238848in}}%
\pgfpathlineto{\pgfqpoint{2.472275in}{3.174822in}}%
\pgfpathlineto{\pgfqpoint{2.472072in}{3.225514in}}%
\pgfpathlineto{\pgfqpoint{2.582969in}{3.289540in}}%
\pgfpathlineto{\pgfqpoint{2.583171in}{3.238848in}}%
\pgfpathclose%
\pgfusepath{fill}%
\end{pgfscope}%
\begin{pgfscope}%
\pgfpathrectangle{\pgfqpoint{0.765000in}{0.660000in}}{\pgfqpoint{4.620000in}{4.620000in}}%
\pgfusepath{clip}%
\pgfsetbuttcap%
\pgfsetroundjoin%
\definecolor{currentfill}{rgb}{1.000000,0.894118,0.788235}%
\pgfsetfillcolor{currentfill}%
\pgfsetlinewidth{0.000000pt}%
\definecolor{currentstroke}{rgb}{1.000000,0.894118,0.788235}%
\pgfsetstrokecolor{currentstroke}%
\pgfsetdash{}{0pt}%
\pgfpathmoveto{\pgfqpoint{2.694068in}{3.174822in}}%
\pgfpathlineto{\pgfqpoint{2.583171in}{3.238848in}}%
\pgfpathlineto{\pgfqpoint{2.582969in}{3.289540in}}%
\pgfpathlineto{\pgfqpoint{2.693865in}{3.225514in}}%
\pgfpathlineto{\pgfqpoint{2.694068in}{3.174822in}}%
\pgfpathclose%
\pgfusepath{fill}%
\end{pgfscope}%
\begin{pgfscope}%
\pgfpathrectangle{\pgfqpoint{0.765000in}{0.660000in}}{\pgfqpoint{4.620000in}{4.620000in}}%
\pgfusepath{clip}%
\pgfsetbuttcap%
\pgfsetroundjoin%
\definecolor{currentfill}{rgb}{1.000000,0.894118,0.788235}%
\pgfsetfillcolor{currentfill}%
\pgfsetlinewidth{0.000000pt}%
\definecolor{currentstroke}{rgb}{1.000000,0.894118,0.788235}%
\pgfsetstrokecolor{currentstroke}%
\pgfsetdash{}{0pt}%
\pgfpathmoveto{\pgfqpoint{2.472275in}{3.046769in}}%
\pgfpathlineto{\pgfqpoint{2.472275in}{3.174822in}}%
\pgfpathlineto{\pgfqpoint{2.472072in}{3.225514in}}%
\pgfpathlineto{\pgfqpoint{2.582969in}{3.033435in}}%
\pgfpathlineto{\pgfqpoint{2.472275in}{3.046769in}}%
\pgfpathclose%
\pgfusepath{fill}%
\end{pgfscope}%
\begin{pgfscope}%
\pgfpathrectangle{\pgfqpoint{0.765000in}{0.660000in}}{\pgfqpoint{4.620000in}{4.620000in}}%
\pgfusepath{clip}%
\pgfsetbuttcap%
\pgfsetroundjoin%
\definecolor{currentfill}{rgb}{1.000000,0.894118,0.788235}%
\pgfsetfillcolor{currentfill}%
\pgfsetlinewidth{0.000000pt}%
\definecolor{currentstroke}{rgb}{1.000000,0.894118,0.788235}%
\pgfsetstrokecolor{currentstroke}%
\pgfsetdash{}{0pt}%
\pgfpathmoveto{\pgfqpoint{2.583171in}{2.982743in}}%
\pgfpathlineto{\pgfqpoint{2.583171in}{3.110796in}}%
\pgfpathlineto{\pgfqpoint{2.582969in}{3.161487in}}%
\pgfpathlineto{\pgfqpoint{2.472072in}{3.097461in}}%
\pgfpathlineto{\pgfqpoint{2.583171in}{2.982743in}}%
\pgfpathclose%
\pgfusepath{fill}%
\end{pgfscope}%
\begin{pgfscope}%
\pgfpathrectangle{\pgfqpoint{0.765000in}{0.660000in}}{\pgfqpoint{4.620000in}{4.620000in}}%
\pgfusepath{clip}%
\pgfsetbuttcap%
\pgfsetroundjoin%
\definecolor{currentfill}{rgb}{1.000000,0.894118,0.788235}%
\pgfsetfillcolor{currentfill}%
\pgfsetlinewidth{0.000000pt}%
\definecolor{currentstroke}{rgb}{1.000000,0.894118,0.788235}%
\pgfsetstrokecolor{currentstroke}%
\pgfsetdash{}{0pt}%
\pgfpathmoveto{\pgfqpoint{2.583171in}{3.110796in}}%
\pgfpathlineto{\pgfqpoint{2.694068in}{3.174822in}}%
\pgfpathlineto{\pgfqpoint{2.693865in}{3.225514in}}%
\pgfpathlineto{\pgfqpoint{2.582969in}{3.161487in}}%
\pgfpathlineto{\pgfqpoint{2.583171in}{3.110796in}}%
\pgfpathclose%
\pgfusepath{fill}%
\end{pgfscope}%
\begin{pgfscope}%
\pgfpathrectangle{\pgfqpoint{0.765000in}{0.660000in}}{\pgfqpoint{4.620000in}{4.620000in}}%
\pgfusepath{clip}%
\pgfsetbuttcap%
\pgfsetroundjoin%
\definecolor{currentfill}{rgb}{1.000000,0.894118,0.788235}%
\pgfsetfillcolor{currentfill}%
\pgfsetlinewidth{0.000000pt}%
\definecolor{currentstroke}{rgb}{1.000000,0.894118,0.788235}%
\pgfsetstrokecolor{currentstroke}%
\pgfsetdash{}{0pt}%
\pgfpathmoveto{\pgfqpoint{2.472275in}{3.174822in}}%
\pgfpathlineto{\pgfqpoint{2.583171in}{3.110796in}}%
\pgfpathlineto{\pgfqpoint{2.582969in}{3.161487in}}%
\pgfpathlineto{\pgfqpoint{2.472072in}{3.225514in}}%
\pgfpathlineto{\pgfqpoint{2.472275in}{3.174822in}}%
\pgfpathclose%
\pgfusepath{fill}%
\end{pgfscope}%
\begin{pgfscope}%
\pgfpathrectangle{\pgfqpoint{0.765000in}{0.660000in}}{\pgfqpoint{4.620000in}{4.620000in}}%
\pgfusepath{clip}%
\pgfsetbuttcap%
\pgfsetroundjoin%
\definecolor{currentfill}{rgb}{1.000000,0.894118,0.788235}%
\pgfsetfillcolor{currentfill}%
\pgfsetlinewidth{0.000000pt}%
\definecolor{currentstroke}{rgb}{1.000000,0.894118,0.788235}%
\pgfsetstrokecolor{currentstroke}%
\pgfsetdash{}{0pt}%
\pgfpathmoveto{\pgfqpoint{2.688236in}{2.773437in}}%
\pgfpathlineto{\pgfqpoint{2.577340in}{2.709411in}}%
\pgfpathlineto{\pgfqpoint{2.688236in}{2.645385in}}%
\pgfpathlineto{\pgfqpoint{2.799133in}{2.709411in}}%
\pgfpathlineto{\pgfqpoint{2.688236in}{2.773437in}}%
\pgfpathclose%
\pgfusepath{fill}%
\end{pgfscope}%
\begin{pgfscope}%
\pgfpathrectangle{\pgfqpoint{0.765000in}{0.660000in}}{\pgfqpoint{4.620000in}{4.620000in}}%
\pgfusepath{clip}%
\pgfsetbuttcap%
\pgfsetroundjoin%
\definecolor{currentfill}{rgb}{1.000000,0.894118,0.788235}%
\pgfsetfillcolor{currentfill}%
\pgfsetlinewidth{0.000000pt}%
\definecolor{currentstroke}{rgb}{1.000000,0.894118,0.788235}%
\pgfsetstrokecolor{currentstroke}%
\pgfsetdash{}{0pt}%
\pgfpathmoveto{\pgfqpoint{2.688236in}{2.773437in}}%
\pgfpathlineto{\pgfqpoint{2.577340in}{2.709411in}}%
\pgfpathlineto{\pgfqpoint{2.577340in}{2.837463in}}%
\pgfpathlineto{\pgfqpoint{2.688236in}{2.901490in}}%
\pgfpathlineto{\pgfqpoint{2.688236in}{2.773437in}}%
\pgfpathclose%
\pgfusepath{fill}%
\end{pgfscope}%
\begin{pgfscope}%
\pgfpathrectangle{\pgfqpoint{0.765000in}{0.660000in}}{\pgfqpoint{4.620000in}{4.620000in}}%
\pgfusepath{clip}%
\pgfsetbuttcap%
\pgfsetroundjoin%
\definecolor{currentfill}{rgb}{1.000000,0.894118,0.788235}%
\pgfsetfillcolor{currentfill}%
\pgfsetlinewidth{0.000000pt}%
\definecolor{currentstroke}{rgb}{1.000000,0.894118,0.788235}%
\pgfsetstrokecolor{currentstroke}%
\pgfsetdash{}{0pt}%
\pgfpathmoveto{\pgfqpoint{2.688236in}{2.773437in}}%
\pgfpathlineto{\pgfqpoint{2.799133in}{2.709411in}}%
\pgfpathlineto{\pgfqpoint{2.799133in}{2.837463in}}%
\pgfpathlineto{\pgfqpoint{2.688236in}{2.901490in}}%
\pgfpathlineto{\pgfqpoint{2.688236in}{2.773437in}}%
\pgfpathclose%
\pgfusepath{fill}%
\end{pgfscope}%
\begin{pgfscope}%
\pgfpathrectangle{\pgfqpoint{0.765000in}{0.660000in}}{\pgfqpoint{4.620000in}{4.620000in}}%
\pgfusepath{clip}%
\pgfsetbuttcap%
\pgfsetroundjoin%
\definecolor{currentfill}{rgb}{1.000000,0.894118,0.788235}%
\pgfsetfillcolor{currentfill}%
\pgfsetlinewidth{0.000000pt}%
\definecolor{currentstroke}{rgb}{1.000000,0.894118,0.788235}%
\pgfsetstrokecolor{currentstroke}%
\pgfsetdash{}{0pt}%
\pgfpathmoveto{\pgfqpoint{2.672370in}{2.810270in}}%
\pgfpathlineto{\pgfqpoint{2.561474in}{2.746243in}}%
\pgfpathlineto{\pgfqpoint{2.672370in}{2.682217in}}%
\pgfpathlineto{\pgfqpoint{2.783267in}{2.746243in}}%
\pgfpathlineto{\pgfqpoint{2.672370in}{2.810270in}}%
\pgfpathclose%
\pgfusepath{fill}%
\end{pgfscope}%
\begin{pgfscope}%
\pgfpathrectangle{\pgfqpoint{0.765000in}{0.660000in}}{\pgfqpoint{4.620000in}{4.620000in}}%
\pgfusepath{clip}%
\pgfsetbuttcap%
\pgfsetroundjoin%
\definecolor{currentfill}{rgb}{1.000000,0.894118,0.788235}%
\pgfsetfillcolor{currentfill}%
\pgfsetlinewidth{0.000000pt}%
\definecolor{currentstroke}{rgb}{1.000000,0.894118,0.788235}%
\pgfsetstrokecolor{currentstroke}%
\pgfsetdash{}{0pt}%
\pgfpathmoveto{\pgfqpoint{2.672370in}{2.810270in}}%
\pgfpathlineto{\pgfqpoint{2.561474in}{2.746243in}}%
\pgfpathlineto{\pgfqpoint{2.561474in}{2.874296in}}%
\pgfpathlineto{\pgfqpoint{2.672370in}{2.938322in}}%
\pgfpathlineto{\pgfqpoint{2.672370in}{2.810270in}}%
\pgfpathclose%
\pgfusepath{fill}%
\end{pgfscope}%
\begin{pgfscope}%
\pgfpathrectangle{\pgfqpoint{0.765000in}{0.660000in}}{\pgfqpoint{4.620000in}{4.620000in}}%
\pgfusepath{clip}%
\pgfsetbuttcap%
\pgfsetroundjoin%
\definecolor{currentfill}{rgb}{1.000000,0.894118,0.788235}%
\pgfsetfillcolor{currentfill}%
\pgfsetlinewidth{0.000000pt}%
\definecolor{currentstroke}{rgb}{1.000000,0.894118,0.788235}%
\pgfsetstrokecolor{currentstroke}%
\pgfsetdash{}{0pt}%
\pgfpathmoveto{\pgfqpoint{2.672370in}{2.810270in}}%
\pgfpathlineto{\pgfqpoint{2.783267in}{2.746243in}}%
\pgfpathlineto{\pgfqpoint{2.783267in}{2.874296in}}%
\pgfpathlineto{\pgfqpoint{2.672370in}{2.938322in}}%
\pgfpathlineto{\pgfqpoint{2.672370in}{2.810270in}}%
\pgfpathclose%
\pgfusepath{fill}%
\end{pgfscope}%
\begin{pgfscope}%
\pgfpathrectangle{\pgfqpoint{0.765000in}{0.660000in}}{\pgfqpoint{4.620000in}{4.620000in}}%
\pgfusepath{clip}%
\pgfsetbuttcap%
\pgfsetroundjoin%
\definecolor{currentfill}{rgb}{1.000000,0.894118,0.788235}%
\pgfsetfillcolor{currentfill}%
\pgfsetlinewidth{0.000000pt}%
\definecolor{currentstroke}{rgb}{1.000000,0.894118,0.788235}%
\pgfsetstrokecolor{currentstroke}%
\pgfsetdash{}{0pt}%
\pgfpathmoveto{\pgfqpoint{2.688236in}{2.901490in}}%
\pgfpathlineto{\pgfqpoint{2.577340in}{2.837463in}}%
\pgfpathlineto{\pgfqpoint{2.688236in}{2.773437in}}%
\pgfpathlineto{\pgfqpoint{2.799133in}{2.837463in}}%
\pgfpathlineto{\pgfqpoint{2.688236in}{2.901490in}}%
\pgfpathclose%
\pgfusepath{fill}%
\end{pgfscope}%
\begin{pgfscope}%
\pgfpathrectangle{\pgfqpoint{0.765000in}{0.660000in}}{\pgfqpoint{4.620000in}{4.620000in}}%
\pgfusepath{clip}%
\pgfsetbuttcap%
\pgfsetroundjoin%
\definecolor{currentfill}{rgb}{1.000000,0.894118,0.788235}%
\pgfsetfillcolor{currentfill}%
\pgfsetlinewidth{0.000000pt}%
\definecolor{currentstroke}{rgb}{1.000000,0.894118,0.788235}%
\pgfsetstrokecolor{currentstroke}%
\pgfsetdash{}{0pt}%
\pgfpathmoveto{\pgfqpoint{2.688236in}{2.645385in}}%
\pgfpathlineto{\pgfqpoint{2.799133in}{2.709411in}}%
\pgfpathlineto{\pgfqpoint{2.799133in}{2.837463in}}%
\pgfpathlineto{\pgfqpoint{2.688236in}{2.773437in}}%
\pgfpathlineto{\pgfqpoint{2.688236in}{2.645385in}}%
\pgfpathclose%
\pgfusepath{fill}%
\end{pgfscope}%
\begin{pgfscope}%
\pgfpathrectangle{\pgfqpoint{0.765000in}{0.660000in}}{\pgfqpoint{4.620000in}{4.620000in}}%
\pgfusepath{clip}%
\pgfsetbuttcap%
\pgfsetroundjoin%
\definecolor{currentfill}{rgb}{1.000000,0.894118,0.788235}%
\pgfsetfillcolor{currentfill}%
\pgfsetlinewidth{0.000000pt}%
\definecolor{currentstroke}{rgb}{1.000000,0.894118,0.788235}%
\pgfsetstrokecolor{currentstroke}%
\pgfsetdash{}{0pt}%
\pgfpathmoveto{\pgfqpoint{2.577340in}{2.709411in}}%
\pgfpathlineto{\pgfqpoint{2.688236in}{2.645385in}}%
\pgfpathlineto{\pgfqpoint{2.688236in}{2.773437in}}%
\pgfpathlineto{\pgfqpoint{2.577340in}{2.837463in}}%
\pgfpathlineto{\pgfqpoint{2.577340in}{2.709411in}}%
\pgfpathclose%
\pgfusepath{fill}%
\end{pgfscope}%
\begin{pgfscope}%
\pgfpathrectangle{\pgfqpoint{0.765000in}{0.660000in}}{\pgfqpoint{4.620000in}{4.620000in}}%
\pgfusepath{clip}%
\pgfsetbuttcap%
\pgfsetroundjoin%
\definecolor{currentfill}{rgb}{1.000000,0.894118,0.788235}%
\pgfsetfillcolor{currentfill}%
\pgfsetlinewidth{0.000000pt}%
\definecolor{currentstroke}{rgb}{1.000000,0.894118,0.788235}%
\pgfsetstrokecolor{currentstroke}%
\pgfsetdash{}{0pt}%
\pgfpathmoveto{\pgfqpoint{2.672370in}{2.938322in}}%
\pgfpathlineto{\pgfqpoint{2.561474in}{2.874296in}}%
\pgfpathlineto{\pgfqpoint{2.672370in}{2.810270in}}%
\pgfpathlineto{\pgfqpoint{2.783267in}{2.874296in}}%
\pgfpathlineto{\pgfqpoint{2.672370in}{2.938322in}}%
\pgfpathclose%
\pgfusepath{fill}%
\end{pgfscope}%
\begin{pgfscope}%
\pgfpathrectangle{\pgfqpoint{0.765000in}{0.660000in}}{\pgfqpoint{4.620000in}{4.620000in}}%
\pgfusepath{clip}%
\pgfsetbuttcap%
\pgfsetroundjoin%
\definecolor{currentfill}{rgb}{1.000000,0.894118,0.788235}%
\pgfsetfillcolor{currentfill}%
\pgfsetlinewidth{0.000000pt}%
\definecolor{currentstroke}{rgb}{1.000000,0.894118,0.788235}%
\pgfsetstrokecolor{currentstroke}%
\pgfsetdash{}{0pt}%
\pgfpathmoveto{\pgfqpoint{2.672370in}{2.682217in}}%
\pgfpathlineto{\pgfqpoint{2.783267in}{2.746243in}}%
\pgfpathlineto{\pgfqpoint{2.783267in}{2.874296in}}%
\pgfpathlineto{\pgfqpoint{2.672370in}{2.810270in}}%
\pgfpathlineto{\pgfqpoint{2.672370in}{2.682217in}}%
\pgfpathclose%
\pgfusepath{fill}%
\end{pgfscope}%
\begin{pgfscope}%
\pgfpathrectangle{\pgfqpoint{0.765000in}{0.660000in}}{\pgfqpoint{4.620000in}{4.620000in}}%
\pgfusepath{clip}%
\pgfsetbuttcap%
\pgfsetroundjoin%
\definecolor{currentfill}{rgb}{1.000000,0.894118,0.788235}%
\pgfsetfillcolor{currentfill}%
\pgfsetlinewidth{0.000000pt}%
\definecolor{currentstroke}{rgb}{1.000000,0.894118,0.788235}%
\pgfsetstrokecolor{currentstroke}%
\pgfsetdash{}{0pt}%
\pgfpathmoveto{\pgfqpoint{2.561474in}{2.746243in}}%
\pgfpathlineto{\pgfqpoint{2.672370in}{2.682217in}}%
\pgfpathlineto{\pgfqpoint{2.672370in}{2.810270in}}%
\pgfpathlineto{\pgfqpoint{2.561474in}{2.874296in}}%
\pgfpathlineto{\pgfqpoint{2.561474in}{2.746243in}}%
\pgfpathclose%
\pgfusepath{fill}%
\end{pgfscope}%
\begin{pgfscope}%
\pgfpathrectangle{\pgfqpoint{0.765000in}{0.660000in}}{\pgfqpoint{4.620000in}{4.620000in}}%
\pgfusepath{clip}%
\pgfsetbuttcap%
\pgfsetroundjoin%
\definecolor{currentfill}{rgb}{1.000000,0.894118,0.788235}%
\pgfsetfillcolor{currentfill}%
\pgfsetlinewidth{0.000000pt}%
\definecolor{currentstroke}{rgb}{1.000000,0.894118,0.788235}%
\pgfsetstrokecolor{currentstroke}%
\pgfsetdash{}{0pt}%
\pgfpathmoveto{\pgfqpoint{2.688236in}{2.773437in}}%
\pgfpathlineto{\pgfqpoint{2.577340in}{2.709411in}}%
\pgfpathlineto{\pgfqpoint{2.561474in}{2.746243in}}%
\pgfpathlineto{\pgfqpoint{2.672370in}{2.810270in}}%
\pgfpathlineto{\pgfqpoint{2.688236in}{2.773437in}}%
\pgfpathclose%
\pgfusepath{fill}%
\end{pgfscope}%
\begin{pgfscope}%
\pgfpathrectangle{\pgfqpoint{0.765000in}{0.660000in}}{\pgfqpoint{4.620000in}{4.620000in}}%
\pgfusepath{clip}%
\pgfsetbuttcap%
\pgfsetroundjoin%
\definecolor{currentfill}{rgb}{1.000000,0.894118,0.788235}%
\pgfsetfillcolor{currentfill}%
\pgfsetlinewidth{0.000000pt}%
\definecolor{currentstroke}{rgb}{1.000000,0.894118,0.788235}%
\pgfsetstrokecolor{currentstroke}%
\pgfsetdash{}{0pt}%
\pgfpathmoveto{\pgfqpoint{2.799133in}{2.709411in}}%
\pgfpathlineto{\pgfqpoint{2.688236in}{2.773437in}}%
\pgfpathlineto{\pgfqpoint{2.672370in}{2.810270in}}%
\pgfpathlineto{\pgfqpoint{2.783267in}{2.746243in}}%
\pgfpathlineto{\pgfqpoint{2.799133in}{2.709411in}}%
\pgfpathclose%
\pgfusepath{fill}%
\end{pgfscope}%
\begin{pgfscope}%
\pgfpathrectangle{\pgfqpoint{0.765000in}{0.660000in}}{\pgfqpoint{4.620000in}{4.620000in}}%
\pgfusepath{clip}%
\pgfsetbuttcap%
\pgfsetroundjoin%
\definecolor{currentfill}{rgb}{1.000000,0.894118,0.788235}%
\pgfsetfillcolor{currentfill}%
\pgfsetlinewidth{0.000000pt}%
\definecolor{currentstroke}{rgb}{1.000000,0.894118,0.788235}%
\pgfsetstrokecolor{currentstroke}%
\pgfsetdash{}{0pt}%
\pgfpathmoveto{\pgfqpoint{2.688236in}{2.773437in}}%
\pgfpathlineto{\pgfqpoint{2.688236in}{2.901490in}}%
\pgfpathlineto{\pgfqpoint{2.672370in}{2.938322in}}%
\pgfpathlineto{\pgfqpoint{2.783267in}{2.746243in}}%
\pgfpathlineto{\pgfqpoint{2.688236in}{2.773437in}}%
\pgfpathclose%
\pgfusepath{fill}%
\end{pgfscope}%
\begin{pgfscope}%
\pgfpathrectangle{\pgfqpoint{0.765000in}{0.660000in}}{\pgfqpoint{4.620000in}{4.620000in}}%
\pgfusepath{clip}%
\pgfsetbuttcap%
\pgfsetroundjoin%
\definecolor{currentfill}{rgb}{1.000000,0.894118,0.788235}%
\pgfsetfillcolor{currentfill}%
\pgfsetlinewidth{0.000000pt}%
\definecolor{currentstroke}{rgb}{1.000000,0.894118,0.788235}%
\pgfsetstrokecolor{currentstroke}%
\pgfsetdash{}{0pt}%
\pgfpathmoveto{\pgfqpoint{2.799133in}{2.709411in}}%
\pgfpathlineto{\pgfqpoint{2.799133in}{2.837463in}}%
\pgfpathlineto{\pgfqpoint{2.783267in}{2.874296in}}%
\pgfpathlineto{\pgfqpoint{2.672370in}{2.810270in}}%
\pgfpathlineto{\pgfqpoint{2.799133in}{2.709411in}}%
\pgfpathclose%
\pgfusepath{fill}%
\end{pgfscope}%
\begin{pgfscope}%
\pgfpathrectangle{\pgfqpoint{0.765000in}{0.660000in}}{\pgfqpoint{4.620000in}{4.620000in}}%
\pgfusepath{clip}%
\pgfsetbuttcap%
\pgfsetroundjoin%
\definecolor{currentfill}{rgb}{1.000000,0.894118,0.788235}%
\pgfsetfillcolor{currentfill}%
\pgfsetlinewidth{0.000000pt}%
\definecolor{currentstroke}{rgb}{1.000000,0.894118,0.788235}%
\pgfsetstrokecolor{currentstroke}%
\pgfsetdash{}{0pt}%
\pgfpathmoveto{\pgfqpoint{2.577340in}{2.709411in}}%
\pgfpathlineto{\pgfqpoint{2.688236in}{2.645385in}}%
\pgfpathlineto{\pgfqpoint{2.672370in}{2.682217in}}%
\pgfpathlineto{\pgfqpoint{2.561474in}{2.746243in}}%
\pgfpathlineto{\pgfqpoint{2.577340in}{2.709411in}}%
\pgfpathclose%
\pgfusepath{fill}%
\end{pgfscope}%
\begin{pgfscope}%
\pgfpathrectangle{\pgfqpoint{0.765000in}{0.660000in}}{\pgfqpoint{4.620000in}{4.620000in}}%
\pgfusepath{clip}%
\pgfsetbuttcap%
\pgfsetroundjoin%
\definecolor{currentfill}{rgb}{1.000000,0.894118,0.788235}%
\pgfsetfillcolor{currentfill}%
\pgfsetlinewidth{0.000000pt}%
\definecolor{currentstroke}{rgb}{1.000000,0.894118,0.788235}%
\pgfsetstrokecolor{currentstroke}%
\pgfsetdash{}{0pt}%
\pgfpathmoveto{\pgfqpoint{2.688236in}{2.645385in}}%
\pgfpathlineto{\pgfqpoint{2.799133in}{2.709411in}}%
\pgfpathlineto{\pgfqpoint{2.783267in}{2.746243in}}%
\pgfpathlineto{\pgfqpoint{2.672370in}{2.682217in}}%
\pgfpathlineto{\pgfqpoint{2.688236in}{2.645385in}}%
\pgfpathclose%
\pgfusepath{fill}%
\end{pgfscope}%
\begin{pgfscope}%
\pgfpathrectangle{\pgfqpoint{0.765000in}{0.660000in}}{\pgfqpoint{4.620000in}{4.620000in}}%
\pgfusepath{clip}%
\pgfsetbuttcap%
\pgfsetroundjoin%
\definecolor{currentfill}{rgb}{1.000000,0.894118,0.788235}%
\pgfsetfillcolor{currentfill}%
\pgfsetlinewidth{0.000000pt}%
\definecolor{currentstroke}{rgb}{1.000000,0.894118,0.788235}%
\pgfsetstrokecolor{currentstroke}%
\pgfsetdash{}{0pt}%
\pgfpathmoveto{\pgfqpoint{2.688236in}{2.901490in}}%
\pgfpathlineto{\pgfqpoint{2.577340in}{2.837463in}}%
\pgfpathlineto{\pgfqpoint{2.561474in}{2.874296in}}%
\pgfpathlineto{\pgfqpoint{2.672370in}{2.938322in}}%
\pgfpathlineto{\pgfqpoint{2.688236in}{2.901490in}}%
\pgfpathclose%
\pgfusepath{fill}%
\end{pgfscope}%
\begin{pgfscope}%
\pgfpathrectangle{\pgfqpoint{0.765000in}{0.660000in}}{\pgfqpoint{4.620000in}{4.620000in}}%
\pgfusepath{clip}%
\pgfsetbuttcap%
\pgfsetroundjoin%
\definecolor{currentfill}{rgb}{1.000000,0.894118,0.788235}%
\pgfsetfillcolor{currentfill}%
\pgfsetlinewidth{0.000000pt}%
\definecolor{currentstroke}{rgb}{1.000000,0.894118,0.788235}%
\pgfsetstrokecolor{currentstroke}%
\pgfsetdash{}{0pt}%
\pgfpathmoveto{\pgfqpoint{2.799133in}{2.837463in}}%
\pgfpathlineto{\pgfqpoint{2.688236in}{2.901490in}}%
\pgfpathlineto{\pgfqpoint{2.672370in}{2.938322in}}%
\pgfpathlineto{\pgfqpoint{2.783267in}{2.874296in}}%
\pgfpathlineto{\pgfqpoint{2.799133in}{2.837463in}}%
\pgfpathclose%
\pgfusepath{fill}%
\end{pgfscope}%
\begin{pgfscope}%
\pgfpathrectangle{\pgfqpoint{0.765000in}{0.660000in}}{\pgfqpoint{4.620000in}{4.620000in}}%
\pgfusepath{clip}%
\pgfsetbuttcap%
\pgfsetroundjoin%
\definecolor{currentfill}{rgb}{1.000000,0.894118,0.788235}%
\pgfsetfillcolor{currentfill}%
\pgfsetlinewidth{0.000000pt}%
\definecolor{currentstroke}{rgb}{1.000000,0.894118,0.788235}%
\pgfsetstrokecolor{currentstroke}%
\pgfsetdash{}{0pt}%
\pgfpathmoveto{\pgfqpoint{2.577340in}{2.709411in}}%
\pgfpathlineto{\pgfqpoint{2.577340in}{2.837463in}}%
\pgfpathlineto{\pgfqpoint{2.561474in}{2.874296in}}%
\pgfpathlineto{\pgfqpoint{2.672370in}{2.682217in}}%
\pgfpathlineto{\pgfqpoint{2.577340in}{2.709411in}}%
\pgfpathclose%
\pgfusepath{fill}%
\end{pgfscope}%
\begin{pgfscope}%
\pgfpathrectangle{\pgfqpoint{0.765000in}{0.660000in}}{\pgfqpoint{4.620000in}{4.620000in}}%
\pgfusepath{clip}%
\pgfsetbuttcap%
\pgfsetroundjoin%
\definecolor{currentfill}{rgb}{1.000000,0.894118,0.788235}%
\pgfsetfillcolor{currentfill}%
\pgfsetlinewidth{0.000000pt}%
\definecolor{currentstroke}{rgb}{1.000000,0.894118,0.788235}%
\pgfsetstrokecolor{currentstroke}%
\pgfsetdash{}{0pt}%
\pgfpathmoveto{\pgfqpoint{2.688236in}{2.645385in}}%
\pgfpathlineto{\pgfqpoint{2.688236in}{2.773437in}}%
\pgfpathlineto{\pgfqpoint{2.672370in}{2.810270in}}%
\pgfpathlineto{\pgfqpoint{2.561474in}{2.746243in}}%
\pgfpathlineto{\pgfqpoint{2.688236in}{2.645385in}}%
\pgfpathclose%
\pgfusepath{fill}%
\end{pgfscope}%
\begin{pgfscope}%
\pgfpathrectangle{\pgfqpoint{0.765000in}{0.660000in}}{\pgfqpoint{4.620000in}{4.620000in}}%
\pgfusepath{clip}%
\pgfsetbuttcap%
\pgfsetroundjoin%
\definecolor{currentfill}{rgb}{1.000000,0.894118,0.788235}%
\pgfsetfillcolor{currentfill}%
\pgfsetlinewidth{0.000000pt}%
\definecolor{currentstroke}{rgb}{1.000000,0.894118,0.788235}%
\pgfsetstrokecolor{currentstroke}%
\pgfsetdash{}{0pt}%
\pgfpathmoveto{\pgfqpoint{2.688236in}{2.773437in}}%
\pgfpathlineto{\pgfqpoint{2.799133in}{2.837463in}}%
\pgfpathlineto{\pgfqpoint{2.783267in}{2.874296in}}%
\pgfpathlineto{\pgfqpoint{2.672370in}{2.810270in}}%
\pgfpathlineto{\pgfqpoint{2.688236in}{2.773437in}}%
\pgfpathclose%
\pgfusepath{fill}%
\end{pgfscope}%
\begin{pgfscope}%
\pgfpathrectangle{\pgfqpoint{0.765000in}{0.660000in}}{\pgfqpoint{4.620000in}{4.620000in}}%
\pgfusepath{clip}%
\pgfsetbuttcap%
\pgfsetroundjoin%
\definecolor{currentfill}{rgb}{1.000000,0.894118,0.788235}%
\pgfsetfillcolor{currentfill}%
\pgfsetlinewidth{0.000000pt}%
\definecolor{currentstroke}{rgb}{1.000000,0.894118,0.788235}%
\pgfsetstrokecolor{currentstroke}%
\pgfsetdash{}{0pt}%
\pgfpathmoveto{\pgfqpoint{2.577340in}{2.837463in}}%
\pgfpathlineto{\pgfqpoint{2.688236in}{2.773437in}}%
\pgfpathlineto{\pgfqpoint{2.672370in}{2.810270in}}%
\pgfpathlineto{\pgfqpoint{2.561474in}{2.874296in}}%
\pgfpathlineto{\pgfqpoint{2.577340in}{2.837463in}}%
\pgfpathclose%
\pgfusepath{fill}%
\end{pgfscope}%
\begin{pgfscope}%
\pgfpathrectangle{\pgfqpoint{0.765000in}{0.660000in}}{\pgfqpoint{4.620000in}{4.620000in}}%
\pgfusepath{clip}%
\pgfsetbuttcap%
\pgfsetroundjoin%
\definecolor{currentfill}{rgb}{1.000000,0.894118,0.788235}%
\pgfsetfillcolor{currentfill}%
\pgfsetlinewidth{0.000000pt}%
\definecolor{currentstroke}{rgb}{1.000000,0.894118,0.788235}%
\pgfsetstrokecolor{currentstroke}%
\pgfsetdash{}{0pt}%
\pgfpathmoveto{\pgfqpoint{2.688236in}{2.773437in}}%
\pgfpathlineto{\pgfqpoint{2.577340in}{2.709411in}}%
\pgfpathlineto{\pgfqpoint{2.688236in}{2.645385in}}%
\pgfpathlineto{\pgfqpoint{2.799133in}{2.709411in}}%
\pgfpathlineto{\pgfqpoint{2.688236in}{2.773437in}}%
\pgfpathclose%
\pgfusepath{fill}%
\end{pgfscope}%
\begin{pgfscope}%
\pgfpathrectangle{\pgfqpoint{0.765000in}{0.660000in}}{\pgfqpoint{4.620000in}{4.620000in}}%
\pgfusepath{clip}%
\pgfsetbuttcap%
\pgfsetroundjoin%
\definecolor{currentfill}{rgb}{1.000000,0.894118,0.788235}%
\pgfsetfillcolor{currentfill}%
\pgfsetlinewidth{0.000000pt}%
\definecolor{currentstroke}{rgb}{1.000000,0.894118,0.788235}%
\pgfsetstrokecolor{currentstroke}%
\pgfsetdash{}{0pt}%
\pgfpathmoveto{\pgfqpoint{2.688236in}{2.773437in}}%
\pgfpathlineto{\pgfqpoint{2.577340in}{2.709411in}}%
\pgfpathlineto{\pgfqpoint{2.577340in}{2.837463in}}%
\pgfpathlineto{\pgfqpoint{2.688236in}{2.901490in}}%
\pgfpathlineto{\pgfqpoint{2.688236in}{2.773437in}}%
\pgfpathclose%
\pgfusepath{fill}%
\end{pgfscope}%
\begin{pgfscope}%
\pgfpathrectangle{\pgfqpoint{0.765000in}{0.660000in}}{\pgfqpoint{4.620000in}{4.620000in}}%
\pgfusepath{clip}%
\pgfsetbuttcap%
\pgfsetroundjoin%
\definecolor{currentfill}{rgb}{1.000000,0.894118,0.788235}%
\pgfsetfillcolor{currentfill}%
\pgfsetlinewidth{0.000000pt}%
\definecolor{currentstroke}{rgb}{1.000000,0.894118,0.788235}%
\pgfsetstrokecolor{currentstroke}%
\pgfsetdash{}{0pt}%
\pgfpathmoveto{\pgfqpoint{2.688236in}{2.773437in}}%
\pgfpathlineto{\pgfqpoint{2.799133in}{2.709411in}}%
\pgfpathlineto{\pgfqpoint{2.799133in}{2.837463in}}%
\pgfpathlineto{\pgfqpoint{2.688236in}{2.901490in}}%
\pgfpathlineto{\pgfqpoint{2.688236in}{2.773437in}}%
\pgfpathclose%
\pgfusepath{fill}%
\end{pgfscope}%
\begin{pgfscope}%
\pgfpathrectangle{\pgfqpoint{0.765000in}{0.660000in}}{\pgfqpoint{4.620000in}{4.620000in}}%
\pgfusepath{clip}%
\pgfsetbuttcap%
\pgfsetroundjoin%
\definecolor{currentfill}{rgb}{1.000000,0.894118,0.788235}%
\pgfsetfillcolor{currentfill}%
\pgfsetlinewidth{0.000000pt}%
\definecolor{currentstroke}{rgb}{1.000000,0.894118,0.788235}%
\pgfsetstrokecolor{currentstroke}%
\pgfsetdash{}{0pt}%
\pgfpathmoveto{\pgfqpoint{2.815582in}{2.599184in}}%
\pgfpathlineto{\pgfqpoint{2.704685in}{2.535158in}}%
\pgfpathlineto{\pgfqpoint{2.815582in}{2.471132in}}%
\pgfpathlineto{\pgfqpoint{2.926478in}{2.535158in}}%
\pgfpathlineto{\pgfqpoint{2.815582in}{2.599184in}}%
\pgfpathclose%
\pgfusepath{fill}%
\end{pgfscope}%
\begin{pgfscope}%
\pgfpathrectangle{\pgfqpoint{0.765000in}{0.660000in}}{\pgfqpoint{4.620000in}{4.620000in}}%
\pgfusepath{clip}%
\pgfsetbuttcap%
\pgfsetroundjoin%
\definecolor{currentfill}{rgb}{1.000000,0.894118,0.788235}%
\pgfsetfillcolor{currentfill}%
\pgfsetlinewidth{0.000000pt}%
\definecolor{currentstroke}{rgb}{1.000000,0.894118,0.788235}%
\pgfsetstrokecolor{currentstroke}%
\pgfsetdash{}{0pt}%
\pgfpathmoveto{\pgfqpoint{2.815582in}{2.599184in}}%
\pgfpathlineto{\pgfqpoint{2.704685in}{2.535158in}}%
\pgfpathlineto{\pgfqpoint{2.704685in}{2.663210in}}%
\pgfpathlineto{\pgfqpoint{2.815582in}{2.727237in}}%
\pgfpathlineto{\pgfqpoint{2.815582in}{2.599184in}}%
\pgfpathclose%
\pgfusepath{fill}%
\end{pgfscope}%
\begin{pgfscope}%
\pgfpathrectangle{\pgfqpoint{0.765000in}{0.660000in}}{\pgfqpoint{4.620000in}{4.620000in}}%
\pgfusepath{clip}%
\pgfsetbuttcap%
\pgfsetroundjoin%
\definecolor{currentfill}{rgb}{1.000000,0.894118,0.788235}%
\pgfsetfillcolor{currentfill}%
\pgfsetlinewidth{0.000000pt}%
\definecolor{currentstroke}{rgb}{1.000000,0.894118,0.788235}%
\pgfsetstrokecolor{currentstroke}%
\pgfsetdash{}{0pt}%
\pgfpathmoveto{\pgfqpoint{2.815582in}{2.599184in}}%
\pgfpathlineto{\pgfqpoint{2.926478in}{2.535158in}}%
\pgfpathlineto{\pgfqpoint{2.926478in}{2.663210in}}%
\pgfpathlineto{\pgfqpoint{2.815582in}{2.727237in}}%
\pgfpathlineto{\pgfqpoint{2.815582in}{2.599184in}}%
\pgfpathclose%
\pgfusepath{fill}%
\end{pgfscope}%
\begin{pgfscope}%
\pgfpathrectangle{\pgfqpoint{0.765000in}{0.660000in}}{\pgfqpoint{4.620000in}{4.620000in}}%
\pgfusepath{clip}%
\pgfsetbuttcap%
\pgfsetroundjoin%
\definecolor{currentfill}{rgb}{1.000000,0.894118,0.788235}%
\pgfsetfillcolor{currentfill}%
\pgfsetlinewidth{0.000000pt}%
\definecolor{currentstroke}{rgb}{1.000000,0.894118,0.788235}%
\pgfsetstrokecolor{currentstroke}%
\pgfsetdash{}{0pt}%
\pgfpathmoveto{\pgfqpoint{2.688236in}{2.901490in}}%
\pgfpathlineto{\pgfqpoint{2.577340in}{2.837463in}}%
\pgfpathlineto{\pgfqpoint{2.688236in}{2.773437in}}%
\pgfpathlineto{\pgfqpoint{2.799133in}{2.837463in}}%
\pgfpathlineto{\pgfqpoint{2.688236in}{2.901490in}}%
\pgfpathclose%
\pgfusepath{fill}%
\end{pgfscope}%
\begin{pgfscope}%
\pgfpathrectangle{\pgfqpoint{0.765000in}{0.660000in}}{\pgfqpoint{4.620000in}{4.620000in}}%
\pgfusepath{clip}%
\pgfsetbuttcap%
\pgfsetroundjoin%
\definecolor{currentfill}{rgb}{1.000000,0.894118,0.788235}%
\pgfsetfillcolor{currentfill}%
\pgfsetlinewidth{0.000000pt}%
\definecolor{currentstroke}{rgb}{1.000000,0.894118,0.788235}%
\pgfsetstrokecolor{currentstroke}%
\pgfsetdash{}{0pt}%
\pgfpathmoveto{\pgfqpoint{2.688236in}{2.645385in}}%
\pgfpathlineto{\pgfqpoint{2.799133in}{2.709411in}}%
\pgfpathlineto{\pgfqpoint{2.799133in}{2.837463in}}%
\pgfpathlineto{\pgfqpoint{2.688236in}{2.773437in}}%
\pgfpathlineto{\pgfqpoint{2.688236in}{2.645385in}}%
\pgfpathclose%
\pgfusepath{fill}%
\end{pgfscope}%
\begin{pgfscope}%
\pgfpathrectangle{\pgfqpoint{0.765000in}{0.660000in}}{\pgfqpoint{4.620000in}{4.620000in}}%
\pgfusepath{clip}%
\pgfsetbuttcap%
\pgfsetroundjoin%
\definecolor{currentfill}{rgb}{1.000000,0.894118,0.788235}%
\pgfsetfillcolor{currentfill}%
\pgfsetlinewidth{0.000000pt}%
\definecolor{currentstroke}{rgb}{1.000000,0.894118,0.788235}%
\pgfsetstrokecolor{currentstroke}%
\pgfsetdash{}{0pt}%
\pgfpathmoveto{\pgfqpoint{2.577340in}{2.709411in}}%
\pgfpathlineto{\pgfqpoint{2.688236in}{2.645385in}}%
\pgfpathlineto{\pgfqpoint{2.688236in}{2.773437in}}%
\pgfpathlineto{\pgfqpoint{2.577340in}{2.837463in}}%
\pgfpathlineto{\pgfqpoint{2.577340in}{2.709411in}}%
\pgfpathclose%
\pgfusepath{fill}%
\end{pgfscope}%
\begin{pgfscope}%
\pgfpathrectangle{\pgfqpoint{0.765000in}{0.660000in}}{\pgfqpoint{4.620000in}{4.620000in}}%
\pgfusepath{clip}%
\pgfsetbuttcap%
\pgfsetroundjoin%
\definecolor{currentfill}{rgb}{1.000000,0.894118,0.788235}%
\pgfsetfillcolor{currentfill}%
\pgfsetlinewidth{0.000000pt}%
\definecolor{currentstroke}{rgb}{1.000000,0.894118,0.788235}%
\pgfsetstrokecolor{currentstroke}%
\pgfsetdash{}{0pt}%
\pgfpathmoveto{\pgfqpoint{2.815582in}{2.727237in}}%
\pgfpathlineto{\pgfqpoint{2.704685in}{2.663210in}}%
\pgfpathlineto{\pgfqpoint{2.815582in}{2.599184in}}%
\pgfpathlineto{\pgfqpoint{2.926478in}{2.663210in}}%
\pgfpathlineto{\pgfqpoint{2.815582in}{2.727237in}}%
\pgfpathclose%
\pgfusepath{fill}%
\end{pgfscope}%
\begin{pgfscope}%
\pgfpathrectangle{\pgfqpoint{0.765000in}{0.660000in}}{\pgfqpoint{4.620000in}{4.620000in}}%
\pgfusepath{clip}%
\pgfsetbuttcap%
\pgfsetroundjoin%
\definecolor{currentfill}{rgb}{1.000000,0.894118,0.788235}%
\pgfsetfillcolor{currentfill}%
\pgfsetlinewidth{0.000000pt}%
\definecolor{currentstroke}{rgb}{1.000000,0.894118,0.788235}%
\pgfsetstrokecolor{currentstroke}%
\pgfsetdash{}{0pt}%
\pgfpathmoveto{\pgfqpoint{2.815582in}{2.471132in}}%
\pgfpathlineto{\pgfqpoint{2.926478in}{2.535158in}}%
\pgfpathlineto{\pgfqpoint{2.926478in}{2.663210in}}%
\pgfpathlineto{\pgfqpoint{2.815582in}{2.599184in}}%
\pgfpathlineto{\pgfqpoint{2.815582in}{2.471132in}}%
\pgfpathclose%
\pgfusepath{fill}%
\end{pgfscope}%
\begin{pgfscope}%
\pgfpathrectangle{\pgfqpoint{0.765000in}{0.660000in}}{\pgfqpoint{4.620000in}{4.620000in}}%
\pgfusepath{clip}%
\pgfsetbuttcap%
\pgfsetroundjoin%
\definecolor{currentfill}{rgb}{1.000000,0.894118,0.788235}%
\pgfsetfillcolor{currentfill}%
\pgfsetlinewidth{0.000000pt}%
\definecolor{currentstroke}{rgb}{1.000000,0.894118,0.788235}%
\pgfsetstrokecolor{currentstroke}%
\pgfsetdash{}{0pt}%
\pgfpathmoveto{\pgfqpoint{2.704685in}{2.535158in}}%
\pgfpathlineto{\pgfqpoint{2.815582in}{2.471132in}}%
\pgfpathlineto{\pgfqpoint{2.815582in}{2.599184in}}%
\pgfpathlineto{\pgfqpoint{2.704685in}{2.663210in}}%
\pgfpathlineto{\pgfqpoint{2.704685in}{2.535158in}}%
\pgfpathclose%
\pgfusepath{fill}%
\end{pgfscope}%
\begin{pgfscope}%
\pgfpathrectangle{\pgfqpoint{0.765000in}{0.660000in}}{\pgfqpoint{4.620000in}{4.620000in}}%
\pgfusepath{clip}%
\pgfsetbuttcap%
\pgfsetroundjoin%
\definecolor{currentfill}{rgb}{1.000000,0.894118,0.788235}%
\pgfsetfillcolor{currentfill}%
\pgfsetlinewidth{0.000000pt}%
\definecolor{currentstroke}{rgb}{1.000000,0.894118,0.788235}%
\pgfsetstrokecolor{currentstroke}%
\pgfsetdash{}{0pt}%
\pgfpathmoveto{\pgfqpoint{2.688236in}{2.773437in}}%
\pgfpathlineto{\pgfqpoint{2.577340in}{2.709411in}}%
\pgfpathlineto{\pgfqpoint{2.704685in}{2.535158in}}%
\pgfpathlineto{\pgfqpoint{2.815582in}{2.599184in}}%
\pgfpathlineto{\pgfqpoint{2.688236in}{2.773437in}}%
\pgfpathclose%
\pgfusepath{fill}%
\end{pgfscope}%
\begin{pgfscope}%
\pgfpathrectangle{\pgfqpoint{0.765000in}{0.660000in}}{\pgfqpoint{4.620000in}{4.620000in}}%
\pgfusepath{clip}%
\pgfsetbuttcap%
\pgfsetroundjoin%
\definecolor{currentfill}{rgb}{1.000000,0.894118,0.788235}%
\pgfsetfillcolor{currentfill}%
\pgfsetlinewidth{0.000000pt}%
\definecolor{currentstroke}{rgb}{1.000000,0.894118,0.788235}%
\pgfsetstrokecolor{currentstroke}%
\pgfsetdash{}{0pt}%
\pgfpathmoveto{\pgfqpoint{2.799133in}{2.709411in}}%
\pgfpathlineto{\pgfqpoint{2.688236in}{2.773437in}}%
\pgfpathlineto{\pgfqpoint{2.815582in}{2.599184in}}%
\pgfpathlineto{\pgfqpoint{2.926478in}{2.535158in}}%
\pgfpathlineto{\pgfqpoint{2.799133in}{2.709411in}}%
\pgfpathclose%
\pgfusepath{fill}%
\end{pgfscope}%
\begin{pgfscope}%
\pgfpathrectangle{\pgfqpoint{0.765000in}{0.660000in}}{\pgfqpoint{4.620000in}{4.620000in}}%
\pgfusepath{clip}%
\pgfsetbuttcap%
\pgfsetroundjoin%
\definecolor{currentfill}{rgb}{1.000000,0.894118,0.788235}%
\pgfsetfillcolor{currentfill}%
\pgfsetlinewidth{0.000000pt}%
\definecolor{currentstroke}{rgb}{1.000000,0.894118,0.788235}%
\pgfsetstrokecolor{currentstroke}%
\pgfsetdash{}{0pt}%
\pgfpathmoveto{\pgfqpoint{2.688236in}{2.773437in}}%
\pgfpathlineto{\pgfqpoint{2.688236in}{2.901490in}}%
\pgfpathlineto{\pgfqpoint{2.815582in}{2.727237in}}%
\pgfpathlineto{\pgfqpoint{2.926478in}{2.535158in}}%
\pgfpathlineto{\pgfqpoint{2.688236in}{2.773437in}}%
\pgfpathclose%
\pgfusepath{fill}%
\end{pgfscope}%
\begin{pgfscope}%
\pgfpathrectangle{\pgfqpoint{0.765000in}{0.660000in}}{\pgfqpoint{4.620000in}{4.620000in}}%
\pgfusepath{clip}%
\pgfsetbuttcap%
\pgfsetroundjoin%
\definecolor{currentfill}{rgb}{1.000000,0.894118,0.788235}%
\pgfsetfillcolor{currentfill}%
\pgfsetlinewidth{0.000000pt}%
\definecolor{currentstroke}{rgb}{1.000000,0.894118,0.788235}%
\pgfsetstrokecolor{currentstroke}%
\pgfsetdash{}{0pt}%
\pgfpathmoveto{\pgfqpoint{2.799133in}{2.709411in}}%
\pgfpathlineto{\pgfqpoint{2.799133in}{2.837463in}}%
\pgfpathlineto{\pgfqpoint{2.926478in}{2.663210in}}%
\pgfpathlineto{\pgfqpoint{2.815582in}{2.599184in}}%
\pgfpathlineto{\pgfqpoint{2.799133in}{2.709411in}}%
\pgfpathclose%
\pgfusepath{fill}%
\end{pgfscope}%
\begin{pgfscope}%
\pgfpathrectangle{\pgfqpoint{0.765000in}{0.660000in}}{\pgfqpoint{4.620000in}{4.620000in}}%
\pgfusepath{clip}%
\pgfsetbuttcap%
\pgfsetroundjoin%
\definecolor{currentfill}{rgb}{1.000000,0.894118,0.788235}%
\pgfsetfillcolor{currentfill}%
\pgfsetlinewidth{0.000000pt}%
\definecolor{currentstroke}{rgb}{1.000000,0.894118,0.788235}%
\pgfsetstrokecolor{currentstroke}%
\pgfsetdash{}{0pt}%
\pgfpathmoveto{\pgfqpoint{2.577340in}{2.709411in}}%
\pgfpathlineto{\pgfqpoint{2.688236in}{2.645385in}}%
\pgfpathlineto{\pgfqpoint{2.815582in}{2.471132in}}%
\pgfpathlineto{\pgfqpoint{2.704685in}{2.535158in}}%
\pgfpathlineto{\pgfqpoint{2.577340in}{2.709411in}}%
\pgfpathclose%
\pgfusepath{fill}%
\end{pgfscope}%
\begin{pgfscope}%
\pgfpathrectangle{\pgfqpoint{0.765000in}{0.660000in}}{\pgfqpoint{4.620000in}{4.620000in}}%
\pgfusepath{clip}%
\pgfsetbuttcap%
\pgfsetroundjoin%
\definecolor{currentfill}{rgb}{1.000000,0.894118,0.788235}%
\pgfsetfillcolor{currentfill}%
\pgfsetlinewidth{0.000000pt}%
\definecolor{currentstroke}{rgb}{1.000000,0.894118,0.788235}%
\pgfsetstrokecolor{currentstroke}%
\pgfsetdash{}{0pt}%
\pgfpathmoveto{\pgfqpoint{2.688236in}{2.645385in}}%
\pgfpathlineto{\pgfqpoint{2.799133in}{2.709411in}}%
\pgfpathlineto{\pgfqpoint{2.926478in}{2.535158in}}%
\pgfpathlineto{\pgfqpoint{2.815582in}{2.471132in}}%
\pgfpathlineto{\pgfqpoint{2.688236in}{2.645385in}}%
\pgfpathclose%
\pgfusepath{fill}%
\end{pgfscope}%
\begin{pgfscope}%
\pgfpathrectangle{\pgfqpoint{0.765000in}{0.660000in}}{\pgfqpoint{4.620000in}{4.620000in}}%
\pgfusepath{clip}%
\pgfsetbuttcap%
\pgfsetroundjoin%
\definecolor{currentfill}{rgb}{1.000000,0.894118,0.788235}%
\pgfsetfillcolor{currentfill}%
\pgfsetlinewidth{0.000000pt}%
\definecolor{currentstroke}{rgb}{1.000000,0.894118,0.788235}%
\pgfsetstrokecolor{currentstroke}%
\pgfsetdash{}{0pt}%
\pgfpathmoveto{\pgfqpoint{2.688236in}{2.901490in}}%
\pgfpathlineto{\pgfqpoint{2.577340in}{2.837463in}}%
\pgfpathlineto{\pgfqpoint{2.704685in}{2.663210in}}%
\pgfpathlineto{\pgfqpoint{2.815582in}{2.727237in}}%
\pgfpathlineto{\pgfqpoint{2.688236in}{2.901490in}}%
\pgfpathclose%
\pgfusepath{fill}%
\end{pgfscope}%
\begin{pgfscope}%
\pgfpathrectangle{\pgfqpoint{0.765000in}{0.660000in}}{\pgfqpoint{4.620000in}{4.620000in}}%
\pgfusepath{clip}%
\pgfsetbuttcap%
\pgfsetroundjoin%
\definecolor{currentfill}{rgb}{1.000000,0.894118,0.788235}%
\pgfsetfillcolor{currentfill}%
\pgfsetlinewidth{0.000000pt}%
\definecolor{currentstroke}{rgb}{1.000000,0.894118,0.788235}%
\pgfsetstrokecolor{currentstroke}%
\pgfsetdash{}{0pt}%
\pgfpathmoveto{\pgfqpoint{2.799133in}{2.837463in}}%
\pgfpathlineto{\pgfqpoint{2.688236in}{2.901490in}}%
\pgfpathlineto{\pgfqpoint{2.815582in}{2.727237in}}%
\pgfpathlineto{\pgfqpoint{2.926478in}{2.663210in}}%
\pgfpathlineto{\pgfqpoint{2.799133in}{2.837463in}}%
\pgfpathclose%
\pgfusepath{fill}%
\end{pgfscope}%
\begin{pgfscope}%
\pgfpathrectangle{\pgfqpoint{0.765000in}{0.660000in}}{\pgfqpoint{4.620000in}{4.620000in}}%
\pgfusepath{clip}%
\pgfsetbuttcap%
\pgfsetroundjoin%
\definecolor{currentfill}{rgb}{1.000000,0.894118,0.788235}%
\pgfsetfillcolor{currentfill}%
\pgfsetlinewidth{0.000000pt}%
\definecolor{currentstroke}{rgb}{1.000000,0.894118,0.788235}%
\pgfsetstrokecolor{currentstroke}%
\pgfsetdash{}{0pt}%
\pgfpathmoveto{\pgfqpoint{2.577340in}{2.709411in}}%
\pgfpathlineto{\pgfqpoint{2.577340in}{2.837463in}}%
\pgfpathlineto{\pgfqpoint{2.704685in}{2.663210in}}%
\pgfpathlineto{\pgfqpoint{2.815582in}{2.471132in}}%
\pgfpathlineto{\pgfqpoint{2.577340in}{2.709411in}}%
\pgfpathclose%
\pgfusepath{fill}%
\end{pgfscope}%
\begin{pgfscope}%
\pgfpathrectangle{\pgfqpoint{0.765000in}{0.660000in}}{\pgfqpoint{4.620000in}{4.620000in}}%
\pgfusepath{clip}%
\pgfsetbuttcap%
\pgfsetroundjoin%
\definecolor{currentfill}{rgb}{1.000000,0.894118,0.788235}%
\pgfsetfillcolor{currentfill}%
\pgfsetlinewidth{0.000000pt}%
\definecolor{currentstroke}{rgb}{1.000000,0.894118,0.788235}%
\pgfsetstrokecolor{currentstroke}%
\pgfsetdash{}{0pt}%
\pgfpathmoveto{\pgfqpoint{2.688236in}{2.645385in}}%
\pgfpathlineto{\pgfqpoint{2.688236in}{2.773437in}}%
\pgfpathlineto{\pgfqpoint{2.815582in}{2.599184in}}%
\pgfpathlineto{\pgfqpoint{2.704685in}{2.535158in}}%
\pgfpathlineto{\pgfqpoint{2.688236in}{2.645385in}}%
\pgfpathclose%
\pgfusepath{fill}%
\end{pgfscope}%
\begin{pgfscope}%
\pgfpathrectangle{\pgfqpoint{0.765000in}{0.660000in}}{\pgfqpoint{4.620000in}{4.620000in}}%
\pgfusepath{clip}%
\pgfsetbuttcap%
\pgfsetroundjoin%
\definecolor{currentfill}{rgb}{1.000000,0.894118,0.788235}%
\pgfsetfillcolor{currentfill}%
\pgfsetlinewidth{0.000000pt}%
\definecolor{currentstroke}{rgb}{1.000000,0.894118,0.788235}%
\pgfsetstrokecolor{currentstroke}%
\pgfsetdash{}{0pt}%
\pgfpathmoveto{\pgfqpoint{2.688236in}{2.773437in}}%
\pgfpathlineto{\pgfqpoint{2.799133in}{2.837463in}}%
\pgfpathlineto{\pgfqpoint{2.926478in}{2.663210in}}%
\pgfpathlineto{\pgfqpoint{2.815582in}{2.599184in}}%
\pgfpathlineto{\pgfqpoint{2.688236in}{2.773437in}}%
\pgfpathclose%
\pgfusepath{fill}%
\end{pgfscope}%
\begin{pgfscope}%
\pgfpathrectangle{\pgfqpoint{0.765000in}{0.660000in}}{\pgfqpoint{4.620000in}{4.620000in}}%
\pgfusepath{clip}%
\pgfsetbuttcap%
\pgfsetroundjoin%
\definecolor{currentfill}{rgb}{1.000000,0.894118,0.788235}%
\pgfsetfillcolor{currentfill}%
\pgfsetlinewidth{0.000000pt}%
\definecolor{currentstroke}{rgb}{1.000000,0.894118,0.788235}%
\pgfsetstrokecolor{currentstroke}%
\pgfsetdash{}{0pt}%
\pgfpathmoveto{\pgfqpoint{2.577340in}{2.837463in}}%
\pgfpathlineto{\pgfqpoint{2.688236in}{2.773437in}}%
\pgfpathlineto{\pgfqpoint{2.815582in}{2.599184in}}%
\pgfpathlineto{\pgfqpoint{2.704685in}{2.663210in}}%
\pgfpathlineto{\pgfqpoint{2.577340in}{2.837463in}}%
\pgfpathclose%
\pgfusepath{fill}%
\end{pgfscope}%
\begin{pgfscope}%
\pgfpathrectangle{\pgfqpoint{0.765000in}{0.660000in}}{\pgfqpoint{4.620000in}{4.620000in}}%
\pgfusepath{clip}%
\pgfsetbuttcap%
\pgfsetroundjoin%
\definecolor{currentfill}{rgb}{1.000000,0.894118,0.788235}%
\pgfsetfillcolor{currentfill}%
\pgfsetlinewidth{0.000000pt}%
\definecolor{currentstroke}{rgb}{1.000000,0.894118,0.788235}%
\pgfsetstrokecolor{currentstroke}%
\pgfsetdash{}{0pt}%
\pgfpathmoveto{\pgfqpoint{3.572472in}{2.854798in}}%
\pgfpathlineto{\pgfqpoint{3.461575in}{2.790772in}}%
\pgfpathlineto{\pgfqpoint{3.572472in}{2.726746in}}%
\pgfpathlineto{\pgfqpoint{3.683368in}{2.790772in}}%
\pgfpathlineto{\pgfqpoint{3.572472in}{2.854798in}}%
\pgfpathclose%
\pgfusepath{fill}%
\end{pgfscope}%
\begin{pgfscope}%
\pgfpathrectangle{\pgfqpoint{0.765000in}{0.660000in}}{\pgfqpoint{4.620000in}{4.620000in}}%
\pgfusepath{clip}%
\pgfsetbuttcap%
\pgfsetroundjoin%
\definecolor{currentfill}{rgb}{1.000000,0.894118,0.788235}%
\pgfsetfillcolor{currentfill}%
\pgfsetlinewidth{0.000000pt}%
\definecolor{currentstroke}{rgb}{1.000000,0.894118,0.788235}%
\pgfsetstrokecolor{currentstroke}%
\pgfsetdash{}{0pt}%
\pgfpathmoveto{\pgfqpoint{3.572472in}{2.854798in}}%
\pgfpathlineto{\pgfqpoint{3.461575in}{2.790772in}}%
\pgfpathlineto{\pgfqpoint{3.461575in}{2.918824in}}%
\pgfpathlineto{\pgfqpoint{3.572472in}{2.982851in}}%
\pgfpathlineto{\pgfqpoint{3.572472in}{2.854798in}}%
\pgfpathclose%
\pgfusepath{fill}%
\end{pgfscope}%
\begin{pgfscope}%
\pgfpathrectangle{\pgfqpoint{0.765000in}{0.660000in}}{\pgfqpoint{4.620000in}{4.620000in}}%
\pgfusepath{clip}%
\pgfsetbuttcap%
\pgfsetroundjoin%
\definecolor{currentfill}{rgb}{1.000000,0.894118,0.788235}%
\pgfsetfillcolor{currentfill}%
\pgfsetlinewidth{0.000000pt}%
\definecolor{currentstroke}{rgb}{1.000000,0.894118,0.788235}%
\pgfsetstrokecolor{currentstroke}%
\pgfsetdash{}{0pt}%
\pgfpathmoveto{\pgfqpoint{3.572472in}{2.854798in}}%
\pgfpathlineto{\pgfqpoint{3.683368in}{2.790772in}}%
\pgfpathlineto{\pgfqpoint{3.683368in}{2.918824in}}%
\pgfpathlineto{\pgfqpoint{3.572472in}{2.982851in}}%
\pgfpathlineto{\pgfqpoint{3.572472in}{2.854798in}}%
\pgfpathclose%
\pgfusepath{fill}%
\end{pgfscope}%
\begin{pgfscope}%
\pgfpathrectangle{\pgfqpoint{0.765000in}{0.660000in}}{\pgfqpoint{4.620000in}{4.620000in}}%
\pgfusepath{clip}%
\pgfsetbuttcap%
\pgfsetroundjoin%
\definecolor{currentfill}{rgb}{1.000000,0.894118,0.788235}%
\pgfsetfillcolor{currentfill}%
\pgfsetlinewidth{0.000000pt}%
\definecolor{currentstroke}{rgb}{1.000000,0.894118,0.788235}%
\pgfsetstrokecolor{currentstroke}%
\pgfsetdash{}{0pt}%
\pgfpathmoveto{\pgfqpoint{3.575586in}{2.945282in}}%
\pgfpathlineto{\pgfqpoint{3.464690in}{2.881255in}}%
\pgfpathlineto{\pgfqpoint{3.575586in}{2.817229in}}%
\pgfpathlineto{\pgfqpoint{3.686483in}{2.881255in}}%
\pgfpathlineto{\pgfqpoint{3.575586in}{2.945282in}}%
\pgfpathclose%
\pgfusepath{fill}%
\end{pgfscope}%
\begin{pgfscope}%
\pgfpathrectangle{\pgfqpoint{0.765000in}{0.660000in}}{\pgfqpoint{4.620000in}{4.620000in}}%
\pgfusepath{clip}%
\pgfsetbuttcap%
\pgfsetroundjoin%
\definecolor{currentfill}{rgb}{1.000000,0.894118,0.788235}%
\pgfsetfillcolor{currentfill}%
\pgfsetlinewidth{0.000000pt}%
\definecolor{currentstroke}{rgb}{1.000000,0.894118,0.788235}%
\pgfsetstrokecolor{currentstroke}%
\pgfsetdash{}{0pt}%
\pgfpathmoveto{\pgfqpoint{3.575586in}{2.945282in}}%
\pgfpathlineto{\pgfqpoint{3.464690in}{2.881255in}}%
\pgfpathlineto{\pgfqpoint{3.464690in}{3.009308in}}%
\pgfpathlineto{\pgfqpoint{3.575586in}{3.073334in}}%
\pgfpathlineto{\pgfqpoint{3.575586in}{2.945282in}}%
\pgfpathclose%
\pgfusepath{fill}%
\end{pgfscope}%
\begin{pgfscope}%
\pgfpathrectangle{\pgfqpoint{0.765000in}{0.660000in}}{\pgfqpoint{4.620000in}{4.620000in}}%
\pgfusepath{clip}%
\pgfsetbuttcap%
\pgfsetroundjoin%
\definecolor{currentfill}{rgb}{1.000000,0.894118,0.788235}%
\pgfsetfillcolor{currentfill}%
\pgfsetlinewidth{0.000000pt}%
\definecolor{currentstroke}{rgb}{1.000000,0.894118,0.788235}%
\pgfsetstrokecolor{currentstroke}%
\pgfsetdash{}{0pt}%
\pgfpathmoveto{\pgfqpoint{3.575586in}{2.945282in}}%
\pgfpathlineto{\pgfqpoint{3.686483in}{2.881255in}}%
\pgfpathlineto{\pgfqpoint{3.686483in}{3.009308in}}%
\pgfpathlineto{\pgfqpoint{3.575586in}{3.073334in}}%
\pgfpathlineto{\pgfqpoint{3.575586in}{2.945282in}}%
\pgfpathclose%
\pgfusepath{fill}%
\end{pgfscope}%
\begin{pgfscope}%
\pgfpathrectangle{\pgfqpoint{0.765000in}{0.660000in}}{\pgfqpoint{4.620000in}{4.620000in}}%
\pgfusepath{clip}%
\pgfsetbuttcap%
\pgfsetroundjoin%
\definecolor{currentfill}{rgb}{1.000000,0.894118,0.788235}%
\pgfsetfillcolor{currentfill}%
\pgfsetlinewidth{0.000000pt}%
\definecolor{currentstroke}{rgb}{1.000000,0.894118,0.788235}%
\pgfsetstrokecolor{currentstroke}%
\pgfsetdash{}{0pt}%
\pgfpathmoveto{\pgfqpoint{3.572472in}{2.982851in}}%
\pgfpathlineto{\pgfqpoint{3.461575in}{2.918824in}}%
\pgfpathlineto{\pgfqpoint{3.572472in}{2.854798in}}%
\pgfpathlineto{\pgfqpoint{3.683368in}{2.918824in}}%
\pgfpathlineto{\pgfqpoint{3.572472in}{2.982851in}}%
\pgfpathclose%
\pgfusepath{fill}%
\end{pgfscope}%
\begin{pgfscope}%
\pgfpathrectangle{\pgfqpoint{0.765000in}{0.660000in}}{\pgfqpoint{4.620000in}{4.620000in}}%
\pgfusepath{clip}%
\pgfsetbuttcap%
\pgfsetroundjoin%
\definecolor{currentfill}{rgb}{1.000000,0.894118,0.788235}%
\pgfsetfillcolor{currentfill}%
\pgfsetlinewidth{0.000000pt}%
\definecolor{currentstroke}{rgb}{1.000000,0.894118,0.788235}%
\pgfsetstrokecolor{currentstroke}%
\pgfsetdash{}{0pt}%
\pgfpathmoveto{\pgfqpoint{3.572472in}{2.726746in}}%
\pgfpathlineto{\pgfqpoint{3.683368in}{2.790772in}}%
\pgfpathlineto{\pgfqpoint{3.683368in}{2.918824in}}%
\pgfpathlineto{\pgfqpoint{3.572472in}{2.854798in}}%
\pgfpathlineto{\pgfqpoint{3.572472in}{2.726746in}}%
\pgfpathclose%
\pgfusepath{fill}%
\end{pgfscope}%
\begin{pgfscope}%
\pgfpathrectangle{\pgfqpoint{0.765000in}{0.660000in}}{\pgfqpoint{4.620000in}{4.620000in}}%
\pgfusepath{clip}%
\pgfsetbuttcap%
\pgfsetroundjoin%
\definecolor{currentfill}{rgb}{1.000000,0.894118,0.788235}%
\pgfsetfillcolor{currentfill}%
\pgfsetlinewidth{0.000000pt}%
\definecolor{currentstroke}{rgb}{1.000000,0.894118,0.788235}%
\pgfsetstrokecolor{currentstroke}%
\pgfsetdash{}{0pt}%
\pgfpathmoveto{\pgfqpoint{3.461575in}{2.790772in}}%
\pgfpathlineto{\pgfqpoint{3.572472in}{2.726746in}}%
\pgfpathlineto{\pgfqpoint{3.572472in}{2.854798in}}%
\pgfpathlineto{\pgfqpoint{3.461575in}{2.918824in}}%
\pgfpathlineto{\pgfqpoint{3.461575in}{2.790772in}}%
\pgfpathclose%
\pgfusepath{fill}%
\end{pgfscope}%
\begin{pgfscope}%
\pgfpathrectangle{\pgfqpoint{0.765000in}{0.660000in}}{\pgfqpoint{4.620000in}{4.620000in}}%
\pgfusepath{clip}%
\pgfsetbuttcap%
\pgfsetroundjoin%
\definecolor{currentfill}{rgb}{1.000000,0.894118,0.788235}%
\pgfsetfillcolor{currentfill}%
\pgfsetlinewidth{0.000000pt}%
\definecolor{currentstroke}{rgb}{1.000000,0.894118,0.788235}%
\pgfsetstrokecolor{currentstroke}%
\pgfsetdash{}{0pt}%
\pgfpathmoveto{\pgfqpoint{3.575586in}{3.073334in}}%
\pgfpathlineto{\pgfqpoint{3.464690in}{3.009308in}}%
\pgfpathlineto{\pgfqpoint{3.575586in}{2.945282in}}%
\pgfpathlineto{\pgfqpoint{3.686483in}{3.009308in}}%
\pgfpathlineto{\pgfqpoint{3.575586in}{3.073334in}}%
\pgfpathclose%
\pgfusepath{fill}%
\end{pgfscope}%
\begin{pgfscope}%
\pgfpathrectangle{\pgfqpoint{0.765000in}{0.660000in}}{\pgfqpoint{4.620000in}{4.620000in}}%
\pgfusepath{clip}%
\pgfsetbuttcap%
\pgfsetroundjoin%
\definecolor{currentfill}{rgb}{1.000000,0.894118,0.788235}%
\pgfsetfillcolor{currentfill}%
\pgfsetlinewidth{0.000000pt}%
\definecolor{currentstroke}{rgb}{1.000000,0.894118,0.788235}%
\pgfsetstrokecolor{currentstroke}%
\pgfsetdash{}{0pt}%
\pgfpathmoveto{\pgfqpoint{3.575586in}{2.817229in}}%
\pgfpathlineto{\pgfqpoint{3.686483in}{2.881255in}}%
\pgfpathlineto{\pgfqpoint{3.686483in}{3.009308in}}%
\pgfpathlineto{\pgfqpoint{3.575586in}{2.945282in}}%
\pgfpathlineto{\pgfqpoint{3.575586in}{2.817229in}}%
\pgfpathclose%
\pgfusepath{fill}%
\end{pgfscope}%
\begin{pgfscope}%
\pgfpathrectangle{\pgfqpoint{0.765000in}{0.660000in}}{\pgfqpoint{4.620000in}{4.620000in}}%
\pgfusepath{clip}%
\pgfsetbuttcap%
\pgfsetroundjoin%
\definecolor{currentfill}{rgb}{1.000000,0.894118,0.788235}%
\pgfsetfillcolor{currentfill}%
\pgfsetlinewidth{0.000000pt}%
\definecolor{currentstroke}{rgb}{1.000000,0.894118,0.788235}%
\pgfsetstrokecolor{currentstroke}%
\pgfsetdash{}{0pt}%
\pgfpathmoveto{\pgfqpoint{3.464690in}{2.881255in}}%
\pgfpathlineto{\pgfqpoint{3.575586in}{2.817229in}}%
\pgfpathlineto{\pgfqpoint{3.575586in}{2.945282in}}%
\pgfpathlineto{\pgfqpoint{3.464690in}{3.009308in}}%
\pgfpathlineto{\pgfqpoint{3.464690in}{2.881255in}}%
\pgfpathclose%
\pgfusepath{fill}%
\end{pgfscope}%
\begin{pgfscope}%
\pgfpathrectangle{\pgfqpoint{0.765000in}{0.660000in}}{\pgfqpoint{4.620000in}{4.620000in}}%
\pgfusepath{clip}%
\pgfsetbuttcap%
\pgfsetroundjoin%
\definecolor{currentfill}{rgb}{1.000000,0.894118,0.788235}%
\pgfsetfillcolor{currentfill}%
\pgfsetlinewidth{0.000000pt}%
\definecolor{currentstroke}{rgb}{1.000000,0.894118,0.788235}%
\pgfsetstrokecolor{currentstroke}%
\pgfsetdash{}{0pt}%
\pgfpathmoveto{\pgfqpoint{3.572472in}{2.854798in}}%
\pgfpathlineto{\pgfqpoint{3.461575in}{2.790772in}}%
\pgfpathlineto{\pgfqpoint{3.464690in}{2.881255in}}%
\pgfpathlineto{\pgfqpoint{3.575586in}{2.945282in}}%
\pgfpathlineto{\pgfqpoint{3.572472in}{2.854798in}}%
\pgfpathclose%
\pgfusepath{fill}%
\end{pgfscope}%
\begin{pgfscope}%
\pgfpathrectangle{\pgfqpoint{0.765000in}{0.660000in}}{\pgfqpoint{4.620000in}{4.620000in}}%
\pgfusepath{clip}%
\pgfsetbuttcap%
\pgfsetroundjoin%
\definecolor{currentfill}{rgb}{1.000000,0.894118,0.788235}%
\pgfsetfillcolor{currentfill}%
\pgfsetlinewidth{0.000000pt}%
\definecolor{currentstroke}{rgb}{1.000000,0.894118,0.788235}%
\pgfsetstrokecolor{currentstroke}%
\pgfsetdash{}{0pt}%
\pgfpathmoveto{\pgfqpoint{3.683368in}{2.790772in}}%
\pgfpathlineto{\pgfqpoint{3.572472in}{2.854798in}}%
\pgfpathlineto{\pgfqpoint{3.575586in}{2.945282in}}%
\pgfpathlineto{\pgfqpoint{3.686483in}{2.881255in}}%
\pgfpathlineto{\pgfqpoint{3.683368in}{2.790772in}}%
\pgfpathclose%
\pgfusepath{fill}%
\end{pgfscope}%
\begin{pgfscope}%
\pgfpathrectangle{\pgfqpoint{0.765000in}{0.660000in}}{\pgfqpoint{4.620000in}{4.620000in}}%
\pgfusepath{clip}%
\pgfsetbuttcap%
\pgfsetroundjoin%
\definecolor{currentfill}{rgb}{1.000000,0.894118,0.788235}%
\pgfsetfillcolor{currentfill}%
\pgfsetlinewidth{0.000000pt}%
\definecolor{currentstroke}{rgb}{1.000000,0.894118,0.788235}%
\pgfsetstrokecolor{currentstroke}%
\pgfsetdash{}{0pt}%
\pgfpathmoveto{\pgfqpoint{3.572472in}{2.854798in}}%
\pgfpathlineto{\pgfqpoint{3.572472in}{2.982851in}}%
\pgfpathlineto{\pgfqpoint{3.575586in}{3.073334in}}%
\pgfpathlineto{\pgfqpoint{3.686483in}{2.881255in}}%
\pgfpathlineto{\pgfqpoint{3.572472in}{2.854798in}}%
\pgfpathclose%
\pgfusepath{fill}%
\end{pgfscope}%
\begin{pgfscope}%
\pgfpathrectangle{\pgfqpoint{0.765000in}{0.660000in}}{\pgfqpoint{4.620000in}{4.620000in}}%
\pgfusepath{clip}%
\pgfsetbuttcap%
\pgfsetroundjoin%
\definecolor{currentfill}{rgb}{1.000000,0.894118,0.788235}%
\pgfsetfillcolor{currentfill}%
\pgfsetlinewidth{0.000000pt}%
\definecolor{currentstroke}{rgb}{1.000000,0.894118,0.788235}%
\pgfsetstrokecolor{currentstroke}%
\pgfsetdash{}{0pt}%
\pgfpathmoveto{\pgfqpoint{3.683368in}{2.790772in}}%
\pgfpathlineto{\pgfqpoint{3.683368in}{2.918824in}}%
\pgfpathlineto{\pgfqpoint{3.686483in}{3.009308in}}%
\pgfpathlineto{\pgfqpoint{3.575586in}{2.945282in}}%
\pgfpathlineto{\pgfqpoint{3.683368in}{2.790772in}}%
\pgfpathclose%
\pgfusepath{fill}%
\end{pgfscope}%
\begin{pgfscope}%
\pgfpathrectangle{\pgfqpoint{0.765000in}{0.660000in}}{\pgfqpoint{4.620000in}{4.620000in}}%
\pgfusepath{clip}%
\pgfsetbuttcap%
\pgfsetroundjoin%
\definecolor{currentfill}{rgb}{1.000000,0.894118,0.788235}%
\pgfsetfillcolor{currentfill}%
\pgfsetlinewidth{0.000000pt}%
\definecolor{currentstroke}{rgb}{1.000000,0.894118,0.788235}%
\pgfsetstrokecolor{currentstroke}%
\pgfsetdash{}{0pt}%
\pgfpathmoveto{\pgfqpoint{3.461575in}{2.790772in}}%
\pgfpathlineto{\pgfqpoint{3.572472in}{2.726746in}}%
\pgfpathlineto{\pgfqpoint{3.575586in}{2.817229in}}%
\pgfpathlineto{\pgfqpoint{3.464690in}{2.881255in}}%
\pgfpathlineto{\pgfqpoint{3.461575in}{2.790772in}}%
\pgfpathclose%
\pgfusepath{fill}%
\end{pgfscope}%
\begin{pgfscope}%
\pgfpathrectangle{\pgfqpoint{0.765000in}{0.660000in}}{\pgfqpoint{4.620000in}{4.620000in}}%
\pgfusepath{clip}%
\pgfsetbuttcap%
\pgfsetroundjoin%
\definecolor{currentfill}{rgb}{1.000000,0.894118,0.788235}%
\pgfsetfillcolor{currentfill}%
\pgfsetlinewidth{0.000000pt}%
\definecolor{currentstroke}{rgb}{1.000000,0.894118,0.788235}%
\pgfsetstrokecolor{currentstroke}%
\pgfsetdash{}{0pt}%
\pgfpathmoveto{\pgfqpoint{3.572472in}{2.726746in}}%
\pgfpathlineto{\pgfqpoint{3.683368in}{2.790772in}}%
\pgfpathlineto{\pgfqpoint{3.686483in}{2.881255in}}%
\pgfpathlineto{\pgfqpoint{3.575586in}{2.817229in}}%
\pgfpathlineto{\pgfqpoint{3.572472in}{2.726746in}}%
\pgfpathclose%
\pgfusepath{fill}%
\end{pgfscope}%
\begin{pgfscope}%
\pgfpathrectangle{\pgfqpoint{0.765000in}{0.660000in}}{\pgfqpoint{4.620000in}{4.620000in}}%
\pgfusepath{clip}%
\pgfsetbuttcap%
\pgfsetroundjoin%
\definecolor{currentfill}{rgb}{1.000000,0.894118,0.788235}%
\pgfsetfillcolor{currentfill}%
\pgfsetlinewidth{0.000000pt}%
\definecolor{currentstroke}{rgb}{1.000000,0.894118,0.788235}%
\pgfsetstrokecolor{currentstroke}%
\pgfsetdash{}{0pt}%
\pgfpathmoveto{\pgfqpoint{3.572472in}{2.982851in}}%
\pgfpathlineto{\pgfqpoint{3.461575in}{2.918824in}}%
\pgfpathlineto{\pgfqpoint{3.464690in}{3.009308in}}%
\pgfpathlineto{\pgfqpoint{3.575586in}{3.073334in}}%
\pgfpathlineto{\pgfqpoint{3.572472in}{2.982851in}}%
\pgfpathclose%
\pgfusepath{fill}%
\end{pgfscope}%
\begin{pgfscope}%
\pgfpathrectangle{\pgfqpoint{0.765000in}{0.660000in}}{\pgfqpoint{4.620000in}{4.620000in}}%
\pgfusepath{clip}%
\pgfsetbuttcap%
\pgfsetroundjoin%
\definecolor{currentfill}{rgb}{1.000000,0.894118,0.788235}%
\pgfsetfillcolor{currentfill}%
\pgfsetlinewidth{0.000000pt}%
\definecolor{currentstroke}{rgb}{1.000000,0.894118,0.788235}%
\pgfsetstrokecolor{currentstroke}%
\pgfsetdash{}{0pt}%
\pgfpathmoveto{\pgfqpoint{3.683368in}{2.918824in}}%
\pgfpathlineto{\pgfqpoint{3.572472in}{2.982851in}}%
\pgfpathlineto{\pgfqpoint{3.575586in}{3.073334in}}%
\pgfpathlineto{\pgfqpoint{3.686483in}{3.009308in}}%
\pgfpathlineto{\pgfqpoint{3.683368in}{2.918824in}}%
\pgfpathclose%
\pgfusepath{fill}%
\end{pgfscope}%
\begin{pgfscope}%
\pgfpathrectangle{\pgfqpoint{0.765000in}{0.660000in}}{\pgfqpoint{4.620000in}{4.620000in}}%
\pgfusepath{clip}%
\pgfsetbuttcap%
\pgfsetroundjoin%
\definecolor{currentfill}{rgb}{1.000000,0.894118,0.788235}%
\pgfsetfillcolor{currentfill}%
\pgfsetlinewidth{0.000000pt}%
\definecolor{currentstroke}{rgb}{1.000000,0.894118,0.788235}%
\pgfsetstrokecolor{currentstroke}%
\pgfsetdash{}{0pt}%
\pgfpathmoveto{\pgfqpoint{3.461575in}{2.790772in}}%
\pgfpathlineto{\pgfqpoint{3.461575in}{2.918824in}}%
\pgfpathlineto{\pgfqpoint{3.464690in}{3.009308in}}%
\pgfpathlineto{\pgfqpoint{3.575586in}{2.817229in}}%
\pgfpathlineto{\pgfqpoint{3.461575in}{2.790772in}}%
\pgfpathclose%
\pgfusepath{fill}%
\end{pgfscope}%
\begin{pgfscope}%
\pgfpathrectangle{\pgfqpoint{0.765000in}{0.660000in}}{\pgfqpoint{4.620000in}{4.620000in}}%
\pgfusepath{clip}%
\pgfsetbuttcap%
\pgfsetroundjoin%
\definecolor{currentfill}{rgb}{1.000000,0.894118,0.788235}%
\pgfsetfillcolor{currentfill}%
\pgfsetlinewidth{0.000000pt}%
\definecolor{currentstroke}{rgb}{1.000000,0.894118,0.788235}%
\pgfsetstrokecolor{currentstroke}%
\pgfsetdash{}{0pt}%
\pgfpathmoveto{\pgfqpoint{3.572472in}{2.726746in}}%
\pgfpathlineto{\pgfqpoint{3.572472in}{2.854798in}}%
\pgfpathlineto{\pgfqpoint{3.575586in}{2.945282in}}%
\pgfpathlineto{\pgfqpoint{3.464690in}{2.881255in}}%
\pgfpathlineto{\pgfqpoint{3.572472in}{2.726746in}}%
\pgfpathclose%
\pgfusepath{fill}%
\end{pgfscope}%
\begin{pgfscope}%
\pgfpathrectangle{\pgfqpoint{0.765000in}{0.660000in}}{\pgfqpoint{4.620000in}{4.620000in}}%
\pgfusepath{clip}%
\pgfsetbuttcap%
\pgfsetroundjoin%
\definecolor{currentfill}{rgb}{1.000000,0.894118,0.788235}%
\pgfsetfillcolor{currentfill}%
\pgfsetlinewidth{0.000000pt}%
\definecolor{currentstroke}{rgb}{1.000000,0.894118,0.788235}%
\pgfsetstrokecolor{currentstroke}%
\pgfsetdash{}{0pt}%
\pgfpathmoveto{\pgfqpoint{3.572472in}{2.854798in}}%
\pgfpathlineto{\pgfqpoint{3.683368in}{2.918824in}}%
\pgfpathlineto{\pgfqpoint{3.686483in}{3.009308in}}%
\pgfpathlineto{\pgfqpoint{3.575586in}{2.945282in}}%
\pgfpathlineto{\pgfqpoint{3.572472in}{2.854798in}}%
\pgfpathclose%
\pgfusepath{fill}%
\end{pgfscope}%
\begin{pgfscope}%
\pgfpathrectangle{\pgfqpoint{0.765000in}{0.660000in}}{\pgfqpoint{4.620000in}{4.620000in}}%
\pgfusepath{clip}%
\pgfsetbuttcap%
\pgfsetroundjoin%
\definecolor{currentfill}{rgb}{1.000000,0.894118,0.788235}%
\pgfsetfillcolor{currentfill}%
\pgfsetlinewidth{0.000000pt}%
\definecolor{currentstroke}{rgb}{1.000000,0.894118,0.788235}%
\pgfsetstrokecolor{currentstroke}%
\pgfsetdash{}{0pt}%
\pgfpathmoveto{\pgfqpoint{3.461575in}{2.918824in}}%
\pgfpathlineto{\pgfqpoint{3.572472in}{2.854798in}}%
\pgfpathlineto{\pgfqpoint{3.575586in}{2.945282in}}%
\pgfpathlineto{\pgfqpoint{3.464690in}{3.009308in}}%
\pgfpathlineto{\pgfqpoint{3.461575in}{2.918824in}}%
\pgfpathclose%
\pgfusepath{fill}%
\end{pgfscope}%
\begin{pgfscope}%
\pgfpathrectangle{\pgfqpoint{0.765000in}{0.660000in}}{\pgfqpoint{4.620000in}{4.620000in}}%
\pgfusepath{clip}%
\pgfsetbuttcap%
\pgfsetroundjoin%
\definecolor{currentfill}{rgb}{1.000000,0.894118,0.788235}%
\pgfsetfillcolor{currentfill}%
\pgfsetlinewidth{0.000000pt}%
\definecolor{currentstroke}{rgb}{1.000000,0.894118,0.788235}%
\pgfsetstrokecolor{currentstroke}%
\pgfsetdash{}{0pt}%
\pgfpathmoveto{\pgfqpoint{3.572472in}{2.854798in}}%
\pgfpathlineto{\pgfqpoint{3.461575in}{2.790772in}}%
\pgfpathlineto{\pgfqpoint{3.572472in}{2.726746in}}%
\pgfpathlineto{\pgfqpoint{3.683368in}{2.790772in}}%
\pgfpathlineto{\pgfqpoint{3.572472in}{2.854798in}}%
\pgfpathclose%
\pgfusepath{fill}%
\end{pgfscope}%
\begin{pgfscope}%
\pgfpathrectangle{\pgfqpoint{0.765000in}{0.660000in}}{\pgfqpoint{4.620000in}{4.620000in}}%
\pgfusepath{clip}%
\pgfsetbuttcap%
\pgfsetroundjoin%
\definecolor{currentfill}{rgb}{1.000000,0.894118,0.788235}%
\pgfsetfillcolor{currentfill}%
\pgfsetlinewidth{0.000000pt}%
\definecolor{currentstroke}{rgb}{1.000000,0.894118,0.788235}%
\pgfsetstrokecolor{currentstroke}%
\pgfsetdash{}{0pt}%
\pgfpathmoveto{\pgfqpoint{3.572472in}{2.854798in}}%
\pgfpathlineto{\pgfqpoint{3.461575in}{2.790772in}}%
\pgfpathlineto{\pgfqpoint{3.461575in}{2.918824in}}%
\pgfpathlineto{\pgfqpoint{3.572472in}{2.982851in}}%
\pgfpathlineto{\pgfqpoint{3.572472in}{2.854798in}}%
\pgfpathclose%
\pgfusepath{fill}%
\end{pgfscope}%
\begin{pgfscope}%
\pgfpathrectangle{\pgfqpoint{0.765000in}{0.660000in}}{\pgfqpoint{4.620000in}{4.620000in}}%
\pgfusepath{clip}%
\pgfsetbuttcap%
\pgfsetroundjoin%
\definecolor{currentfill}{rgb}{1.000000,0.894118,0.788235}%
\pgfsetfillcolor{currentfill}%
\pgfsetlinewidth{0.000000pt}%
\definecolor{currentstroke}{rgb}{1.000000,0.894118,0.788235}%
\pgfsetstrokecolor{currentstroke}%
\pgfsetdash{}{0pt}%
\pgfpathmoveto{\pgfqpoint{3.572472in}{2.854798in}}%
\pgfpathlineto{\pgfqpoint{3.683368in}{2.790772in}}%
\pgfpathlineto{\pgfqpoint{3.683368in}{2.918824in}}%
\pgfpathlineto{\pgfqpoint{3.572472in}{2.982851in}}%
\pgfpathlineto{\pgfqpoint{3.572472in}{2.854798in}}%
\pgfpathclose%
\pgfusepath{fill}%
\end{pgfscope}%
\begin{pgfscope}%
\pgfpathrectangle{\pgfqpoint{0.765000in}{0.660000in}}{\pgfqpoint{4.620000in}{4.620000in}}%
\pgfusepath{clip}%
\pgfsetbuttcap%
\pgfsetroundjoin%
\definecolor{currentfill}{rgb}{1.000000,0.894118,0.788235}%
\pgfsetfillcolor{currentfill}%
\pgfsetlinewidth{0.000000pt}%
\definecolor{currentstroke}{rgb}{1.000000,0.894118,0.788235}%
\pgfsetstrokecolor{currentstroke}%
\pgfsetdash{}{0pt}%
\pgfpathmoveto{\pgfqpoint{3.563846in}{2.808203in}}%
\pgfpathlineto{\pgfqpoint{3.452949in}{2.744176in}}%
\pgfpathlineto{\pgfqpoint{3.563846in}{2.680150in}}%
\pgfpathlineto{\pgfqpoint{3.674742in}{2.744176in}}%
\pgfpathlineto{\pgfqpoint{3.563846in}{2.808203in}}%
\pgfpathclose%
\pgfusepath{fill}%
\end{pgfscope}%
\begin{pgfscope}%
\pgfpathrectangle{\pgfqpoint{0.765000in}{0.660000in}}{\pgfqpoint{4.620000in}{4.620000in}}%
\pgfusepath{clip}%
\pgfsetbuttcap%
\pgfsetroundjoin%
\definecolor{currentfill}{rgb}{1.000000,0.894118,0.788235}%
\pgfsetfillcolor{currentfill}%
\pgfsetlinewidth{0.000000pt}%
\definecolor{currentstroke}{rgb}{1.000000,0.894118,0.788235}%
\pgfsetstrokecolor{currentstroke}%
\pgfsetdash{}{0pt}%
\pgfpathmoveto{\pgfqpoint{3.563846in}{2.808203in}}%
\pgfpathlineto{\pgfqpoint{3.452949in}{2.744176in}}%
\pgfpathlineto{\pgfqpoint{3.452949in}{2.872229in}}%
\pgfpathlineto{\pgfqpoint{3.563846in}{2.936255in}}%
\pgfpathlineto{\pgfqpoint{3.563846in}{2.808203in}}%
\pgfpathclose%
\pgfusepath{fill}%
\end{pgfscope}%
\begin{pgfscope}%
\pgfpathrectangle{\pgfqpoint{0.765000in}{0.660000in}}{\pgfqpoint{4.620000in}{4.620000in}}%
\pgfusepath{clip}%
\pgfsetbuttcap%
\pgfsetroundjoin%
\definecolor{currentfill}{rgb}{1.000000,0.894118,0.788235}%
\pgfsetfillcolor{currentfill}%
\pgfsetlinewidth{0.000000pt}%
\definecolor{currentstroke}{rgb}{1.000000,0.894118,0.788235}%
\pgfsetstrokecolor{currentstroke}%
\pgfsetdash{}{0pt}%
\pgfpathmoveto{\pgfqpoint{3.563846in}{2.808203in}}%
\pgfpathlineto{\pgfqpoint{3.674742in}{2.744176in}}%
\pgfpathlineto{\pgfqpoint{3.674742in}{2.872229in}}%
\pgfpathlineto{\pgfqpoint{3.563846in}{2.936255in}}%
\pgfpathlineto{\pgfqpoint{3.563846in}{2.808203in}}%
\pgfpathclose%
\pgfusepath{fill}%
\end{pgfscope}%
\begin{pgfscope}%
\pgfpathrectangle{\pgfqpoint{0.765000in}{0.660000in}}{\pgfqpoint{4.620000in}{4.620000in}}%
\pgfusepath{clip}%
\pgfsetbuttcap%
\pgfsetroundjoin%
\definecolor{currentfill}{rgb}{1.000000,0.894118,0.788235}%
\pgfsetfillcolor{currentfill}%
\pgfsetlinewidth{0.000000pt}%
\definecolor{currentstroke}{rgb}{1.000000,0.894118,0.788235}%
\pgfsetstrokecolor{currentstroke}%
\pgfsetdash{}{0pt}%
\pgfpathmoveto{\pgfqpoint{3.572472in}{2.982851in}}%
\pgfpathlineto{\pgfqpoint{3.461575in}{2.918824in}}%
\pgfpathlineto{\pgfqpoint{3.572472in}{2.854798in}}%
\pgfpathlineto{\pgfqpoint{3.683368in}{2.918824in}}%
\pgfpathlineto{\pgfqpoint{3.572472in}{2.982851in}}%
\pgfpathclose%
\pgfusepath{fill}%
\end{pgfscope}%
\begin{pgfscope}%
\pgfpathrectangle{\pgfqpoint{0.765000in}{0.660000in}}{\pgfqpoint{4.620000in}{4.620000in}}%
\pgfusepath{clip}%
\pgfsetbuttcap%
\pgfsetroundjoin%
\definecolor{currentfill}{rgb}{1.000000,0.894118,0.788235}%
\pgfsetfillcolor{currentfill}%
\pgfsetlinewidth{0.000000pt}%
\definecolor{currentstroke}{rgb}{1.000000,0.894118,0.788235}%
\pgfsetstrokecolor{currentstroke}%
\pgfsetdash{}{0pt}%
\pgfpathmoveto{\pgfqpoint{3.572472in}{2.726746in}}%
\pgfpathlineto{\pgfqpoint{3.683368in}{2.790772in}}%
\pgfpathlineto{\pgfqpoint{3.683368in}{2.918824in}}%
\pgfpathlineto{\pgfqpoint{3.572472in}{2.854798in}}%
\pgfpathlineto{\pgfqpoint{3.572472in}{2.726746in}}%
\pgfpathclose%
\pgfusepath{fill}%
\end{pgfscope}%
\begin{pgfscope}%
\pgfpathrectangle{\pgfqpoint{0.765000in}{0.660000in}}{\pgfqpoint{4.620000in}{4.620000in}}%
\pgfusepath{clip}%
\pgfsetbuttcap%
\pgfsetroundjoin%
\definecolor{currentfill}{rgb}{1.000000,0.894118,0.788235}%
\pgfsetfillcolor{currentfill}%
\pgfsetlinewidth{0.000000pt}%
\definecolor{currentstroke}{rgb}{1.000000,0.894118,0.788235}%
\pgfsetstrokecolor{currentstroke}%
\pgfsetdash{}{0pt}%
\pgfpathmoveto{\pgfqpoint{3.461575in}{2.790772in}}%
\pgfpathlineto{\pgfqpoint{3.572472in}{2.726746in}}%
\pgfpathlineto{\pgfqpoint{3.572472in}{2.854798in}}%
\pgfpathlineto{\pgfqpoint{3.461575in}{2.918824in}}%
\pgfpathlineto{\pgfqpoint{3.461575in}{2.790772in}}%
\pgfpathclose%
\pgfusepath{fill}%
\end{pgfscope}%
\begin{pgfscope}%
\pgfpathrectangle{\pgfqpoint{0.765000in}{0.660000in}}{\pgfqpoint{4.620000in}{4.620000in}}%
\pgfusepath{clip}%
\pgfsetbuttcap%
\pgfsetroundjoin%
\definecolor{currentfill}{rgb}{1.000000,0.894118,0.788235}%
\pgfsetfillcolor{currentfill}%
\pgfsetlinewidth{0.000000pt}%
\definecolor{currentstroke}{rgb}{1.000000,0.894118,0.788235}%
\pgfsetstrokecolor{currentstroke}%
\pgfsetdash{}{0pt}%
\pgfpathmoveto{\pgfqpoint{3.563846in}{2.936255in}}%
\pgfpathlineto{\pgfqpoint{3.452949in}{2.872229in}}%
\pgfpathlineto{\pgfqpoint{3.563846in}{2.808203in}}%
\pgfpathlineto{\pgfqpoint{3.674742in}{2.872229in}}%
\pgfpathlineto{\pgfqpoint{3.563846in}{2.936255in}}%
\pgfpathclose%
\pgfusepath{fill}%
\end{pgfscope}%
\begin{pgfscope}%
\pgfpathrectangle{\pgfqpoint{0.765000in}{0.660000in}}{\pgfqpoint{4.620000in}{4.620000in}}%
\pgfusepath{clip}%
\pgfsetbuttcap%
\pgfsetroundjoin%
\definecolor{currentfill}{rgb}{1.000000,0.894118,0.788235}%
\pgfsetfillcolor{currentfill}%
\pgfsetlinewidth{0.000000pt}%
\definecolor{currentstroke}{rgb}{1.000000,0.894118,0.788235}%
\pgfsetstrokecolor{currentstroke}%
\pgfsetdash{}{0pt}%
\pgfpathmoveto{\pgfqpoint{3.563846in}{2.680150in}}%
\pgfpathlineto{\pgfqpoint{3.674742in}{2.744176in}}%
\pgfpathlineto{\pgfqpoint{3.674742in}{2.872229in}}%
\pgfpathlineto{\pgfqpoint{3.563846in}{2.808203in}}%
\pgfpathlineto{\pgfqpoint{3.563846in}{2.680150in}}%
\pgfpathclose%
\pgfusepath{fill}%
\end{pgfscope}%
\begin{pgfscope}%
\pgfpathrectangle{\pgfqpoint{0.765000in}{0.660000in}}{\pgfqpoint{4.620000in}{4.620000in}}%
\pgfusepath{clip}%
\pgfsetbuttcap%
\pgfsetroundjoin%
\definecolor{currentfill}{rgb}{1.000000,0.894118,0.788235}%
\pgfsetfillcolor{currentfill}%
\pgfsetlinewidth{0.000000pt}%
\definecolor{currentstroke}{rgb}{1.000000,0.894118,0.788235}%
\pgfsetstrokecolor{currentstroke}%
\pgfsetdash{}{0pt}%
\pgfpathmoveto{\pgfqpoint{3.452949in}{2.744176in}}%
\pgfpathlineto{\pgfqpoint{3.563846in}{2.680150in}}%
\pgfpathlineto{\pgfqpoint{3.563846in}{2.808203in}}%
\pgfpathlineto{\pgfqpoint{3.452949in}{2.872229in}}%
\pgfpathlineto{\pgfqpoint{3.452949in}{2.744176in}}%
\pgfpathclose%
\pgfusepath{fill}%
\end{pgfscope}%
\begin{pgfscope}%
\pgfpathrectangle{\pgfqpoint{0.765000in}{0.660000in}}{\pgfqpoint{4.620000in}{4.620000in}}%
\pgfusepath{clip}%
\pgfsetbuttcap%
\pgfsetroundjoin%
\definecolor{currentfill}{rgb}{1.000000,0.894118,0.788235}%
\pgfsetfillcolor{currentfill}%
\pgfsetlinewidth{0.000000pt}%
\definecolor{currentstroke}{rgb}{1.000000,0.894118,0.788235}%
\pgfsetstrokecolor{currentstroke}%
\pgfsetdash{}{0pt}%
\pgfpathmoveto{\pgfqpoint{3.572472in}{2.854798in}}%
\pgfpathlineto{\pgfqpoint{3.461575in}{2.790772in}}%
\pgfpathlineto{\pgfqpoint{3.452949in}{2.744176in}}%
\pgfpathlineto{\pgfqpoint{3.563846in}{2.808203in}}%
\pgfpathlineto{\pgfqpoint{3.572472in}{2.854798in}}%
\pgfpathclose%
\pgfusepath{fill}%
\end{pgfscope}%
\begin{pgfscope}%
\pgfpathrectangle{\pgfqpoint{0.765000in}{0.660000in}}{\pgfqpoint{4.620000in}{4.620000in}}%
\pgfusepath{clip}%
\pgfsetbuttcap%
\pgfsetroundjoin%
\definecolor{currentfill}{rgb}{1.000000,0.894118,0.788235}%
\pgfsetfillcolor{currentfill}%
\pgfsetlinewidth{0.000000pt}%
\definecolor{currentstroke}{rgb}{1.000000,0.894118,0.788235}%
\pgfsetstrokecolor{currentstroke}%
\pgfsetdash{}{0pt}%
\pgfpathmoveto{\pgfqpoint{3.683368in}{2.790772in}}%
\pgfpathlineto{\pgfqpoint{3.572472in}{2.854798in}}%
\pgfpathlineto{\pgfqpoint{3.563846in}{2.808203in}}%
\pgfpathlineto{\pgfqpoint{3.674742in}{2.744176in}}%
\pgfpathlineto{\pgfqpoint{3.683368in}{2.790772in}}%
\pgfpathclose%
\pgfusepath{fill}%
\end{pgfscope}%
\begin{pgfscope}%
\pgfpathrectangle{\pgfqpoint{0.765000in}{0.660000in}}{\pgfqpoint{4.620000in}{4.620000in}}%
\pgfusepath{clip}%
\pgfsetbuttcap%
\pgfsetroundjoin%
\definecolor{currentfill}{rgb}{1.000000,0.894118,0.788235}%
\pgfsetfillcolor{currentfill}%
\pgfsetlinewidth{0.000000pt}%
\definecolor{currentstroke}{rgb}{1.000000,0.894118,0.788235}%
\pgfsetstrokecolor{currentstroke}%
\pgfsetdash{}{0pt}%
\pgfpathmoveto{\pgfqpoint{3.572472in}{2.854798in}}%
\pgfpathlineto{\pgfqpoint{3.572472in}{2.982851in}}%
\pgfpathlineto{\pgfqpoint{3.563846in}{2.936255in}}%
\pgfpathlineto{\pgfqpoint{3.674742in}{2.744176in}}%
\pgfpathlineto{\pgfqpoint{3.572472in}{2.854798in}}%
\pgfpathclose%
\pgfusepath{fill}%
\end{pgfscope}%
\begin{pgfscope}%
\pgfpathrectangle{\pgfqpoint{0.765000in}{0.660000in}}{\pgfqpoint{4.620000in}{4.620000in}}%
\pgfusepath{clip}%
\pgfsetbuttcap%
\pgfsetroundjoin%
\definecolor{currentfill}{rgb}{1.000000,0.894118,0.788235}%
\pgfsetfillcolor{currentfill}%
\pgfsetlinewidth{0.000000pt}%
\definecolor{currentstroke}{rgb}{1.000000,0.894118,0.788235}%
\pgfsetstrokecolor{currentstroke}%
\pgfsetdash{}{0pt}%
\pgfpathmoveto{\pgfqpoint{3.683368in}{2.790772in}}%
\pgfpathlineto{\pgfqpoint{3.683368in}{2.918824in}}%
\pgfpathlineto{\pgfqpoint{3.674742in}{2.872229in}}%
\pgfpathlineto{\pgfqpoint{3.563846in}{2.808203in}}%
\pgfpathlineto{\pgfqpoint{3.683368in}{2.790772in}}%
\pgfpathclose%
\pgfusepath{fill}%
\end{pgfscope}%
\begin{pgfscope}%
\pgfpathrectangle{\pgfqpoint{0.765000in}{0.660000in}}{\pgfqpoint{4.620000in}{4.620000in}}%
\pgfusepath{clip}%
\pgfsetbuttcap%
\pgfsetroundjoin%
\definecolor{currentfill}{rgb}{1.000000,0.894118,0.788235}%
\pgfsetfillcolor{currentfill}%
\pgfsetlinewidth{0.000000pt}%
\definecolor{currentstroke}{rgb}{1.000000,0.894118,0.788235}%
\pgfsetstrokecolor{currentstroke}%
\pgfsetdash{}{0pt}%
\pgfpathmoveto{\pgfqpoint{3.461575in}{2.790772in}}%
\pgfpathlineto{\pgfqpoint{3.572472in}{2.726746in}}%
\pgfpathlineto{\pgfqpoint{3.563846in}{2.680150in}}%
\pgfpathlineto{\pgfqpoint{3.452949in}{2.744176in}}%
\pgfpathlineto{\pgfqpoint{3.461575in}{2.790772in}}%
\pgfpathclose%
\pgfusepath{fill}%
\end{pgfscope}%
\begin{pgfscope}%
\pgfpathrectangle{\pgfqpoint{0.765000in}{0.660000in}}{\pgfqpoint{4.620000in}{4.620000in}}%
\pgfusepath{clip}%
\pgfsetbuttcap%
\pgfsetroundjoin%
\definecolor{currentfill}{rgb}{1.000000,0.894118,0.788235}%
\pgfsetfillcolor{currentfill}%
\pgfsetlinewidth{0.000000pt}%
\definecolor{currentstroke}{rgb}{1.000000,0.894118,0.788235}%
\pgfsetstrokecolor{currentstroke}%
\pgfsetdash{}{0pt}%
\pgfpathmoveto{\pgfqpoint{3.572472in}{2.726746in}}%
\pgfpathlineto{\pgfqpoint{3.683368in}{2.790772in}}%
\pgfpathlineto{\pgfqpoint{3.674742in}{2.744176in}}%
\pgfpathlineto{\pgfqpoint{3.563846in}{2.680150in}}%
\pgfpathlineto{\pgfqpoint{3.572472in}{2.726746in}}%
\pgfpathclose%
\pgfusepath{fill}%
\end{pgfscope}%
\begin{pgfscope}%
\pgfpathrectangle{\pgfqpoint{0.765000in}{0.660000in}}{\pgfqpoint{4.620000in}{4.620000in}}%
\pgfusepath{clip}%
\pgfsetbuttcap%
\pgfsetroundjoin%
\definecolor{currentfill}{rgb}{1.000000,0.894118,0.788235}%
\pgfsetfillcolor{currentfill}%
\pgfsetlinewidth{0.000000pt}%
\definecolor{currentstroke}{rgb}{1.000000,0.894118,0.788235}%
\pgfsetstrokecolor{currentstroke}%
\pgfsetdash{}{0pt}%
\pgfpathmoveto{\pgfqpoint{3.572472in}{2.982851in}}%
\pgfpathlineto{\pgfqpoint{3.461575in}{2.918824in}}%
\pgfpathlineto{\pgfqpoint{3.452949in}{2.872229in}}%
\pgfpathlineto{\pgfqpoint{3.563846in}{2.936255in}}%
\pgfpathlineto{\pgfqpoint{3.572472in}{2.982851in}}%
\pgfpathclose%
\pgfusepath{fill}%
\end{pgfscope}%
\begin{pgfscope}%
\pgfpathrectangle{\pgfqpoint{0.765000in}{0.660000in}}{\pgfqpoint{4.620000in}{4.620000in}}%
\pgfusepath{clip}%
\pgfsetbuttcap%
\pgfsetroundjoin%
\definecolor{currentfill}{rgb}{1.000000,0.894118,0.788235}%
\pgfsetfillcolor{currentfill}%
\pgfsetlinewidth{0.000000pt}%
\definecolor{currentstroke}{rgb}{1.000000,0.894118,0.788235}%
\pgfsetstrokecolor{currentstroke}%
\pgfsetdash{}{0pt}%
\pgfpathmoveto{\pgfqpoint{3.683368in}{2.918824in}}%
\pgfpathlineto{\pgfqpoint{3.572472in}{2.982851in}}%
\pgfpathlineto{\pgfqpoint{3.563846in}{2.936255in}}%
\pgfpathlineto{\pgfqpoint{3.674742in}{2.872229in}}%
\pgfpathlineto{\pgfqpoint{3.683368in}{2.918824in}}%
\pgfpathclose%
\pgfusepath{fill}%
\end{pgfscope}%
\begin{pgfscope}%
\pgfpathrectangle{\pgfqpoint{0.765000in}{0.660000in}}{\pgfqpoint{4.620000in}{4.620000in}}%
\pgfusepath{clip}%
\pgfsetbuttcap%
\pgfsetroundjoin%
\definecolor{currentfill}{rgb}{1.000000,0.894118,0.788235}%
\pgfsetfillcolor{currentfill}%
\pgfsetlinewidth{0.000000pt}%
\definecolor{currentstroke}{rgb}{1.000000,0.894118,0.788235}%
\pgfsetstrokecolor{currentstroke}%
\pgfsetdash{}{0pt}%
\pgfpathmoveto{\pgfqpoint{3.461575in}{2.790772in}}%
\pgfpathlineto{\pgfqpoint{3.461575in}{2.918824in}}%
\pgfpathlineto{\pgfqpoint{3.452949in}{2.872229in}}%
\pgfpathlineto{\pgfqpoint{3.563846in}{2.680150in}}%
\pgfpathlineto{\pgfqpoint{3.461575in}{2.790772in}}%
\pgfpathclose%
\pgfusepath{fill}%
\end{pgfscope}%
\begin{pgfscope}%
\pgfpathrectangle{\pgfqpoint{0.765000in}{0.660000in}}{\pgfqpoint{4.620000in}{4.620000in}}%
\pgfusepath{clip}%
\pgfsetbuttcap%
\pgfsetroundjoin%
\definecolor{currentfill}{rgb}{1.000000,0.894118,0.788235}%
\pgfsetfillcolor{currentfill}%
\pgfsetlinewidth{0.000000pt}%
\definecolor{currentstroke}{rgb}{1.000000,0.894118,0.788235}%
\pgfsetstrokecolor{currentstroke}%
\pgfsetdash{}{0pt}%
\pgfpathmoveto{\pgfqpoint{3.572472in}{2.726746in}}%
\pgfpathlineto{\pgfqpoint{3.572472in}{2.854798in}}%
\pgfpathlineto{\pgfqpoint{3.563846in}{2.808203in}}%
\pgfpathlineto{\pgfqpoint{3.452949in}{2.744176in}}%
\pgfpathlineto{\pgfqpoint{3.572472in}{2.726746in}}%
\pgfpathclose%
\pgfusepath{fill}%
\end{pgfscope}%
\begin{pgfscope}%
\pgfpathrectangle{\pgfqpoint{0.765000in}{0.660000in}}{\pgfqpoint{4.620000in}{4.620000in}}%
\pgfusepath{clip}%
\pgfsetbuttcap%
\pgfsetroundjoin%
\definecolor{currentfill}{rgb}{1.000000,0.894118,0.788235}%
\pgfsetfillcolor{currentfill}%
\pgfsetlinewidth{0.000000pt}%
\definecolor{currentstroke}{rgb}{1.000000,0.894118,0.788235}%
\pgfsetstrokecolor{currentstroke}%
\pgfsetdash{}{0pt}%
\pgfpathmoveto{\pgfqpoint{3.572472in}{2.854798in}}%
\pgfpathlineto{\pgfqpoint{3.683368in}{2.918824in}}%
\pgfpathlineto{\pgfqpoint{3.674742in}{2.872229in}}%
\pgfpathlineto{\pgfqpoint{3.563846in}{2.808203in}}%
\pgfpathlineto{\pgfqpoint{3.572472in}{2.854798in}}%
\pgfpathclose%
\pgfusepath{fill}%
\end{pgfscope}%
\begin{pgfscope}%
\pgfpathrectangle{\pgfqpoint{0.765000in}{0.660000in}}{\pgfqpoint{4.620000in}{4.620000in}}%
\pgfusepath{clip}%
\pgfsetbuttcap%
\pgfsetroundjoin%
\definecolor{currentfill}{rgb}{1.000000,0.894118,0.788235}%
\pgfsetfillcolor{currentfill}%
\pgfsetlinewidth{0.000000pt}%
\definecolor{currentstroke}{rgb}{1.000000,0.894118,0.788235}%
\pgfsetstrokecolor{currentstroke}%
\pgfsetdash{}{0pt}%
\pgfpathmoveto{\pgfqpoint{3.461575in}{2.918824in}}%
\pgfpathlineto{\pgfqpoint{3.572472in}{2.854798in}}%
\pgfpathlineto{\pgfqpoint{3.563846in}{2.808203in}}%
\pgfpathlineto{\pgfqpoint{3.452949in}{2.872229in}}%
\pgfpathlineto{\pgfqpoint{3.461575in}{2.918824in}}%
\pgfpathclose%
\pgfusepath{fill}%
\end{pgfscope}%
\begin{pgfscope}%
\pgfpathrectangle{\pgfqpoint{0.765000in}{0.660000in}}{\pgfqpoint{4.620000in}{4.620000in}}%
\pgfusepath{clip}%
\pgfsetbuttcap%
\pgfsetroundjoin%
\definecolor{currentfill}{rgb}{1.000000,0.894118,0.788235}%
\pgfsetfillcolor{currentfill}%
\pgfsetlinewidth{0.000000pt}%
\definecolor{currentstroke}{rgb}{1.000000,0.894118,0.788235}%
\pgfsetstrokecolor{currentstroke}%
\pgfsetdash{}{0pt}%
\pgfpathmoveto{\pgfqpoint{2.651156in}{3.305927in}}%
\pgfpathlineto{\pgfqpoint{2.540259in}{3.241901in}}%
\pgfpathlineto{\pgfqpoint{2.651156in}{3.177874in}}%
\pgfpathlineto{\pgfqpoint{2.762052in}{3.241901in}}%
\pgfpathlineto{\pgfqpoint{2.651156in}{3.305927in}}%
\pgfpathclose%
\pgfusepath{fill}%
\end{pgfscope}%
\begin{pgfscope}%
\pgfpathrectangle{\pgfqpoint{0.765000in}{0.660000in}}{\pgfqpoint{4.620000in}{4.620000in}}%
\pgfusepath{clip}%
\pgfsetbuttcap%
\pgfsetroundjoin%
\definecolor{currentfill}{rgb}{1.000000,0.894118,0.788235}%
\pgfsetfillcolor{currentfill}%
\pgfsetlinewidth{0.000000pt}%
\definecolor{currentstroke}{rgb}{1.000000,0.894118,0.788235}%
\pgfsetstrokecolor{currentstroke}%
\pgfsetdash{}{0pt}%
\pgfpathmoveto{\pgfqpoint{2.651156in}{3.305927in}}%
\pgfpathlineto{\pgfqpoint{2.540259in}{3.241901in}}%
\pgfpathlineto{\pgfqpoint{2.540259in}{3.369953in}}%
\pgfpathlineto{\pgfqpoint{2.651156in}{3.433979in}}%
\pgfpathlineto{\pgfqpoint{2.651156in}{3.305927in}}%
\pgfpathclose%
\pgfusepath{fill}%
\end{pgfscope}%
\begin{pgfscope}%
\pgfpathrectangle{\pgfqpoint{0.765000in}{0.660000in}}{\pgfqpoint{4.620000in}{4.620000in}}%
\pgfusepath{clip}%
\pgfsetbuttcap%
\pgfsetroundjoin%
\definecolor{currentfill}{rgb}{1.000000,0.894118,0.788235}%
\pgfsetfillcolor{currentfill}%
\pgfsetlinewidth{0.000000pt}%
\definecolor{currentstroke}{rgb}{1.000000,0.894118,0.788235}%
\pgfsetstrokecolor{currentstroke}%
\pgfsetdash{}{0pt}%
\pgfpathmoveto{\pgfqpoint{2.651156in}{3.305927in}}%
\pgfpathlineto{\pgfqpoint{2.762052in}{3.241901in}}%
\pgfpathlineto{\pgfqpoint{2.762052in}{3.369953in}}%
\pgfpathlineto{\pgfqpoint{2.651156in}{3.433979in}}%
\pgfpathlineto{\pgfqpoint{2.651156in}{3.305927in}}%
\pgfpathclose%
\pgfusepath{fill}%
\end{pgfscope}%
\begin{pgfscope}%
\pgfpathrectangle{\pgfqpoint{0.765000in}{0.660000in}}{\pgfqpoint{4.620000in}{4.620000in}}%
\pgfusepath{clip}%
\pgfsetbuttcap%
\pgfsetroundjoin%
\definecolor{currentfill}{rgb}{1.000000,0.894118,0.788235}%
\pgfsetfillcolor{currentfill}%
\pgfsetlinewidth{0.000000pt}%
\definecolor{currentstroke}{rgb}{1.000000,0.894118,0.788235}%
\pgfsetstrokecolor{currentstroke}%
\pgfsetdash{}{0pt}%
\pgfpathmoveto{\pgfqpoint{2.642560in}{3.294969in}}%
\pgfpathlineto{\pgfqpoint{2.531663in}{3.230943in}}%
\pgfpathlineto{\pgfqpoint{2.642560in}{3.166917in}}%
\pgfpathlineto{\pgfqpoint{2.753457in}{3.230943in}}%
\pgfpathlineto{\pgfqpoint{2.642560in}{3.294969in}}%
\pgfpathclose%
\pgfusepath{fill}%
\end{pgfscope}%
\begin{pgfscope}%
\pgfpathrectangle{\pgfqpoint{0.765000in}{0.660000in}}{\pgfqpoint{4.620000in}{4.620000in}}%
\pgfusepath{clip}%
\pgfsetbuttcap%
\pgfsetroundjoin%
\definecolor{currentfill}{rgb}{1.000000,0.894118,0.788235}%
\pgfsetfillcolor{currentfill}%
\pgfsetlinewidth{0.000000pt}%
\definecolor{currentstroke}{rgb}{1.000000,0.894118,0.788235}%
\pgfsetstrokecolor{currentstroke}%
\pgfsetdash{}{0pt}%
\pgfpathmoveto{\pgfqpoint{2.642560in}{3.294969in}}%
\pgfpathlineto{\pgfqpoint{2.531663in}{3.230943in}}%
\pgfpathlineto{\pgfqpoint{2.531663in}{3.358995in}}%
\pgfpathlineto{\pgfqpoint{2.642560in}{3.423022in}}%
\pgfpathlineto{\pgfqpoint{2.642560in}{3.294969in}}%
\pgfpathclose%
\pgfusepath{fill}%
\end{pgfscope}%
\begin{pgfscope}%
\pgfpathrectangle{\pgfqpoint{0.765000in}{0.660000in}}{\pgfqpoint{4.620000in}{4.620000in}}%
\pgfusepath{clip}%
\pgfsetbuttcap%
\pgfsetroundjoin%
\definecolor{currentfill}{rgb}{1.000000,0.894118,0.788235}%
\pgfsetfillcolor{currentfill}%
\pgfsetlinewidth{0.000000pt}%
\definecolor{currentstroke}{rgb}{1.000000,0.894118,0.788235}%
\pgfsetstrokecolor{currentstroke}%
\pgfsetdash{}{0pt}%
\pgfpathmoveto{\pgfqpoint{2.642560in}{3.294969in}}%
\pgfpathlineto{\pgfqpoint{2.753457in}{3.230943in}}%
\pgfpathlineto{\pgfqpoint{2.753457in}{3.358995in}}%
\pgfpathlineto{\pgfqpoint{2.642560in}{3.423022in}}%
\pgfpathlineto{\pgfqpoint{2.642560in}{3.294969in}}%
\pgfpathclose%
\pgfusepath{fill}%
\end{pgfscope}%
\begin{pgfscope}%
\pgfpathrectangle{\pgfqpoint{0.765000in}{0.660000in}}{\pgfqpoint{4.620000in}{4.620000in}}%
\pgfusepath{clip}%
\pgfsetbuttcap%
\pgfsetroundjoin%
\definecolor{currentfill}{rgb}{1.000000,0.894118,0.788235}%
\pgfsetfillcolor{currentfill}%
\pgfsetlinewidth{0.000000pt}%
\definecolor{currentstroke}{rgb}{1.000000,0.894118,0.788235}%
\pgfsetstrokecolor{currentstroke}%
\pgfsetdash{}{0pt}%
\pgfpathmoveto{\pgfqpoint{2.651156in}{3.433979in}}%
\pgfpathlineto{\pgfqpoint{2.540259in}{3.369953in}}%
\pgfpathlineto{\pgfqpoint{2.651156in}{3.305927in}}%
\pgfpathlineto{\pgfqpoint{2.762052in}{3.369953in}}%
\pgfpathlineto{\pgfqpoint{2.651156in}{3.433979in}}%
\pgfpathclose%
\pgfusepath{fill}%
\end{pgfscope}%
\begin{pgfscope}%
\pgfpathrectangle{\pgfqpoint{0.765000in}{0.660000in}}{\pgfqpoint{4.620000in}{4.620000in}}%
\pgfusepath{clip}%
\pgfsetbuttcap%
\pgfsetroundjoin%
\definecolor{currentfill}{rgb}{1.000000,0.894118,0.788235}%
\pgfsetfillcolor{currentfill}%
\pgfsetlinewidth{0.000000pt}%
\definecolor{currentstroke}{rgb}{1.000000,0.894118,0.788235}%
\pgfsetstrokecolor{currentstroke}%
\pgfsetdash{}{0pt}%
\pgfpathmoveto{\pgfqpoint{2.651156in}{3.177874in}}%
\pgfpathlineto{\pgfqpoint{2.762052in}{3.241901in}}%
\pgfpathlineto{\pgfqpoint{2.762052in}{3.369953in}}%
\pgfpathlineto{\pgfqpoint{2.651156in}{3.305927in}}%
\pgfpathlineto{\pgfqpoint{2.651156in}{3.177874in}}%
\pgfpathclose%
\pgfusepath{fill}%
\end{pgfscope}%
\begin{pgfscope}%
\pgfpathrectangle{\pgfqpoint{0.765000in}{0.660000in}}{\pgfqpoint{4.620000in}{4.620000in}}%
\pgfusepath{clip}%
\pgfsetbuttcap%
\pgfsetroundjoin%
\definecolor{currentfill}{rgb}{1.000000,0.894118,0.788235}%
\pgfsetfillcolor{currentfill}%
\pgfsetlinewidth{0.000000pt}%
\definecolor{currentstroke}{rgb}{1.000000,0.894118,0.788235}%
\pgfsetstrokecolor{currentstroke}%
\pgfsetdash{}{0pt}%
\pgfpathmoveto{\pgfqpoint{2.540259in}{3.241901in}}%
\pgfpathlineto{\pgfqpoint{2.651156in}{3.177874in}}%
\pgfpathlineto{\pgfqpoint{2.651156in}{3.305927in}}%
\pgfpathlineto{\pgfqpoint{2.540259in}{3.369953in}}%
\pgfpathlineto{\pgfqpoint{2.540259in}{3.241901in}}%
\pgfpathclose%
\pgfusepath{fill}%
\end{pgfscope}%
\begin{pgfscope}%
\pgfpathrectangle{\pgfqpoint{0.765000in}{0.660000in}}{\pgfqpoint{4.620000in}{4.620000in}}%
\pgfusepath{clip}%
\pgfsetbuttcap%
\pgfsetroundjoin%
\definecolor{currentfill}{rgb}{1.000000,0.894118,0.788235}%
\pgfsetfillcolor{currentfill}%
\pgfsetlinewidth{0.000000pt}%
\definecolor{currentstroke}{rgb}{1.000000,0.894118,0.788235}%
\pgfsetstrokecolor{currentstroke}%
\pgfsetdash{}{0pt}%
\pgfpathmoveto{\pgfqpoint{2.642560in}{3.423022in}}%
\pgfpathlineto{\pgfqpoint{2.531663in}{3.358995in}}%
\pgfpathlineto{\pgfqpoint{2.642560in}{3.294969in}}%
\pgfpathlineto{\pgfqpoint{2.753457in}{3.358995in}}%
\pgfpathlineto{\pgfqpoint{2.642560in}{3.423022in}}%
\pgfpathclose%
\pgfusepath{fill}%
\end{pgfscope}%
\begin{pgfscope}%
\pgfpathrectangle{\pgfqpoint{0.765000in}{0.660000in}}{\pgfqpoint{4.620000in}{4.620000in}}%
\pgfusepath{clip}%
\pgfsetbuttcap%
\pgfsetroundjoin%
\definecolor{currentfill}{rgb}{1.000000,0.894118,0.788235}%
\pgfsetfillcolor{currentfill}%
\pgfsetlinewidth{0.000000pt}%
\definecolor{currentstroke}{rgb}{1.000000,0.894118,0.788235}%
\pgfsetstrokecolor{currentstroke}%
\pgfsetdash{}{0pt}%
\pgfpathmoveto{\pgfqpoint{2.642560in}{3.166917in}}%
\pgfpathlineto{\pgfqpoint{2.753457in}{3.230943in}}%
\pgfpathlineto{\pgfqpoint{2.753457in}{3.358995in}}%
\pgfpathlineto{\pgfqpoint{2.642560in}{3.294969in}}%
\pgfpathlineto{\pgfqpoint{2.642560in}{3.166917in}}%
\pgfpathclose%
\pgfusepath{fill}%
\end{pgfscope}%
\begin{pgfscope}%
\pgfpathrectangle{\pgfqpoint{0.765000in}{0.660000in}}{\pgfqpoint{4.620000in}{4.620000in}}%
\pgfusepath{clip}%
\pgfsetbuttcap%
\pgfsetroundjoin%
\definecolor{currentfill}{rgb}{1.000000,0.894118,0.788235}%
\pgfsetfillcolor{currentfill}%
\pgfsetlinewidth{0.000000pt}%
\definecolor{currentstroke}{rgb}{1.000000,0.894118,0.788235}%
\pgfsetstrokecolor{currentstroke}%
\pgfsetdash{}{0pt}%
\pgfpathmoveto{\pgfqpoint{2.531663in}{3.230943in}}%
\pgfpathlineto{\pgfqpoint{2.642560in}{3.166917in}}%
\pgfpathlineto{\pgfqpoint{2.642560in}{3.294969in}}%
\pgfpathlineto{\pgfqpoint{2.531663in}{3.358995in}}%
\pgfpathlineto{\pgfqpoint{2.531663in}{3.230943in}}%
\pgfpathclose%
\pgfusepath{fill}%
\end{pgfscope}%
\begin{pgfscope}%
\pgfpathrectangle{\pgfqpoint{0.765000in}{0.660000in}}{\pgfqpoint{4.620000in}{4.620000in}}%
\pgfusepath{clip}%
\pgfsetbuttcap%
\pgfsetroundjoin%
\definecolor{currentfill}{rgb}{1.000000,0.894118,0.788235}%
\pgfsetfillcolor{currentfill}%
\pgfsetlinewidth{0.000000pt}%
\definecolor{currentstroke}{rgb}{1.000000,0.894118,0.788235}%
\pgfsetstrokecolor{currentstroke}%
\pgfsetdash{}{0pt}%
\pgfpathmoveto{\pgfqpoint{2.651156in}{3.305927in}}%
\pgfpathlineto{\pgfqpoint{2.540259in}{3.241901in}}%
\pgfpathlineto{\pgfqpoint{2.531663in}{3.230943in}}%
\pgfpathlineto{\pgfqpoint{2.642560in}{3.294969in}}%
\pgfpathlineto{\pgfqpoint{2.651156in}{3.305927in}}%
\pgfpathclose%
\pgfusepath{fill}%
\end{pgfscope}%
\begin{pgfscope}%
\pgfpathrectangle{\pgfqpoint{0.765000in}{0.660000in}}{\pgfqpoint{4.620000in}{4.620000in}}%
\pgfusepath{clip}%
\pgfsetbuttcap%
\pgfsetroundjoin%
\definecolor{currentfill}{rgb}{1.000000,0.894118,0.788235}%
\pgfsetfillcolor{currentfill}%
\pgfsetlinewidth{0.000000pt}%
\definecolor{currentstroke}{rgb}{1.000000,0.894118,0.788235}%
\pgfsetstrokecolor{currentstroke}%
\pgfsetdash{}{0pt}%
\pgfpathmoveto{\pgfqpoint{2.762052in}{3.241901in}}%
\pgfpathlineto{\pgfqpoint{2.651156in}{3.305927in}}%
\pgfpathlineto{\pgfqpoint{2.642560in}{3.294969in}}%
\pgfpathlineto{\pgfqpoint{2.753457in}{3.230943in}}%
\pgfpathlineto{\pgfqpoint{2.762052in}{3.241901in}}%
\pgfpathclose%
\pgfusepath{fill}%
\end{pgfscope}%
\begin{pgfscope}%
\pgfpathrectangle{\pgfqpoint{0.765000in}{0.660000in}}{\pgfqpoint{4.620000in}{4.620000in}}%
\pgfusepath{clip}%
\pgfsetbuttcap%
\pgfsetroundjoin%
\definecolor{currentfill}{rgb}{1.000000,0.894118,0.788235}%
\pgfsetfillcolor{currentfill}%
\pgfsetlinewidth{0.000000pt}%
\definecolor{currentstroke}{rgb}{1.000000,0.894118,0.788235}%
\pgfsetstrokecolor{currentstroke}%
\pgfsetdash{}{0pt}%
\pgfpathmoveto{\pgfqpoint{2.651156in}{3.305927in}}%
\pgfpathlineto{\pgfqpoint{2.651156in}{3.433979in}}%
\pgfpathlineto{\pgfqpoint{2.642560in}{3.423022in}}%
\pgfpathlineto{\pgfqpoint{2.753457in}{3.230943in}}%
\pgfpathlineto{\pgfqpoint{2.651156in}{3.305927in}}%
\pgfpathclose%
\pgfusepath{fill}%
\end{pgfscope}%
\begin{pgfscope}%
\pgfpathrectangle{\pgfqpoint{0.765000in}{0.660000in}}{\pgfqpoint{4.620000in}{4.620000in}}%
\pgfusepath{clip}%
\pgfsetbuttcap%
\pgfsetroundjoin%
\definecolor{currentfill}{rgb}{1.000000,0.894118,0.788235}%
\pgfsetfillcolor{currentfill}%
\pgfsetlinewidth{0.000000pt}%
\definecolor{currentstroke}{rgb}{1.000000,0.894118,0.788235}%
\pgfsetstrokecolor{currentstroke}%
\pgfsetdash{}{0pt}%
\pgfpathmoveto{\pgfqpoint{2.762052in}{3.241901in}}%
\pgfpathlineto{\pgfqpoint{2.762052in}{3.369953in}}%
\pgfpathlineto{\pgfqpoint{2.753457in}{3.358995in}}%
\pgfpathlineto{\pgfqpoint{2.642560in}{3.294969in}}%
\pgfpathlineto{\pgfqpoint{2.762052in}{3.241901in}}%
\pgfpathclose%
\pgfusepath{fill}%
\end{pgfscope}%
\begin{pgfscope}%
\pgfpathrectangle{\pgfqpoint{0.765000in}{0.660000in}}{\pgfqpoint{4.620000in}{4.620000in}}%
\pgfusepath{clip}%
\pgfsetbuttcap%
\pgfsetroundjoin%
\definecolor{currentfill}{rgb}{1.000000,0.894118,0.788235}%
\pgfsetfillcolor{currentfill}%
\pgfsetlinewidth{0.000000pt}%
\definecolor{currentstroke}{rgb}{1.000000,0.894118,0.788235}%
\pgfsetstrokecolor{currentstroke}%
\pgfsetdash{}{0pt}%
\pgfpathmoveto{\pgfqpoint{2.540259in}{3.241901in}}%
\pgfpathlineto{\pgfqpoint{2.651156in}{3.177874in}}%
\pgfpathlineto{\pgfqpoint{2.642560in}{3.166917in}}%
\pgfpathlineto{\pgfqpoint{2.531663in}{3.230943in}}%
\pgfpathlineto{\pgfqpoint{2.540259in}{3.241901in}}%
\pgfpathclose%
\pgfusepath{fill}%
\end{pgfscope}%
\begin{pgfscope}%
\pgfpathrectangle{\pgfqpoint{0.765000in}{0.660000in}}{\pgfqpoint{4.620000in}{4.620000in}}%
\pgfusepath{clip}%
\pgfsetbuttcap%
\pgfsetroundjoin%
\definecolor{currentfill}{rgb}{1.000000,0.894118,0.788235}%
\pgfsetfillcolor{currentfill}%
\pgfsetlinewidth{0.000000pt}%
\definecolor{currentstroke}{rgb}{1.000000,0.894118,0.788235}%
\pgfsetstrokecolor{currentstroke}%
\pgfsetdash{}{0pt}%
\pgfpathmoveto{\pgfqpoint{2.651156in}{3.177874in}}%
\pgfpathlineto{\pgfqpoint{2.762052in}{3.241901in}}%
\pgfpathlineto{\pgfqpoint{2.753457in}{3.230943in}}%
\pgfpathlineto{\pgfqpoint{2.642560in}{3.166917in}}%
\pgfpathlineto{\pgfqpoint{2.651156in}{3.177874in}}%
\pgfpathclose%
\pgfusepath{fill}%
\end{pgfscope}%
\begin{pgfscope}%
\pgfpathrectangle{\pgfqpoint{0.765000in}{0.660000in}}{\pgfqpoint{4.620000in}{4.620000in}}%
\pgfusepath{clip}%
\pgfsetbuttcap%
\pgfsetroundjoin%
\definecolor{currentfill}{rgb}{1.000000,0.894118,0.788235}%
\pgfsetfillcolor{currentfill}%
\pgfsetlinewidth{0.000000pt}%
\definecolor{currentstroke}{rgb}{1.000000,0.894118,0.788235}%
\pgfsetstrokecolor{currentstroke}%
\pgfsetdash{}{0pt}%
\pgfpathmoveto{\pgfqpoint{2.651156in}{3.433979in}}%
\pgfpathlineto{\pgfqpoint{2.540259in}{3.369953in}}%
\pgfpathlineto{\pgfqpoint{2.531663in}{3.358995in}}%
\pgfpathlineto{\pgfqpoint{2.642560in}{3.423022in}}%
\pgfpathlineto{\pgfqpoint{2.651156in}{3.433979in}}%
\pgfpathclose%
\pgfusepath{fill}%
\end{pgfscope}%
\begin{pgfscope}%
\pgfpathrectangle{\pgfqpoint{0.765000in}{0.660000in}}{\pgfqpoint{4.620000in}{4.620000in}}%
\pgfusepath{clip}%
\pgfsetbuttcap%
\pgfsetroundjoin%
\definecolor{currentfill}{rgb}{1.000000,0.894118,0.788235}%
\pgfsetfillcolor{currentfill}%
\pgfsetlinewidth{0.000000pt}%
\definecolor{currentstroke}{rgb}{1.000000,0.894118,0.788235}%
\pgfsetstrokecolor{currentstroke}%
\pgfsetdash{}{0pt}%
\pgfpathmoveto{\pgfqpoint{2.762052in}{3.369953in}}%
\pgfpathlineto{\pgfqpoint{2.651156in}{3.433979in}}%
\pgfpathlineto{\pgfqpoint{2.642560in}{3.423022in}}%
\pgfpathlineto{\pgfqpoint{2.753457in}{3.358995in}}%
\pgfpathlineto{\pgfqpoint{2.762052in}{3.369953in}}%
\pgfpathclose%
\pgfusepath{fill}%
\end{pgfscope}%
\begin{pgfscope}%
\pgfpathrectangle{\pgfqpoint{0.765000in}{0.660000in}}{\pgfqpoint{4.620000in}{4.620000in}}%
\pgfusepath{clip}%
\pgfsetbuttcap%
\pgfsetroundjoin%
\definecolor{currentfill}{rgb}{1.000000,0.894118,0.788235}%
\pgfsetfillcolor{currentfill}%
\pgfsetlinewidth{0.000000pt}%
\definecolor{currentstroke}{rgb}{1.000000,0.894118,0.788235}%
\pgfsetstrokecolor{currentstroke}%
\pgfsetdash{}{0pt}%
\pgfpathmoveto{\pgfqpoint{2.540259in}{3.241901in}}%
\pgfpathlineto{\pgfqpoint{2.540259in}{3.369953in}}%
\pgfpathlineto{\pgfqpoint{2.531663in}{3.358995in}}%
\pgfpathlineto{\pgfqpoint{2.642560in}{3.166917in}}%
\pgfpathlineto{\pgfqpoint{2.540259in}{3.241901in}}%
\pgfpathclose%
\pgfusepath{fill}%
\end{pgfscope}%
\begin{pgfscope}%
\pgfpathrectangle{\pgfqpoint{0.765000in}{0.660000in}}{\pgfqpoint{4.620000in}{4.620000in}}%
\pgfusepath{clip}%
\pgfsetbuttcap%
\pgfsetroundjoin%
\definecolor{currentfill}{rgb}{1.000000,0.894118,0.788235}%
\pgfsetfillcolor{currentfill}%
\pgfsetlinewidth{0.000000pt}%
\definecolor{currentstroke}{rgb}{1.000000,0.894118,0.788235}%
\pgfsetstrokecolor{currentstroke}%
\pgfsetdash{}{0pt}%
\pgfpathmoveto{\pgfqpoint{2.651156in}{3.177874in}}%
\pgfpathlineto{\pgfqpoint{2.651156in}{3.305927in}}%
\pgfpathlineto{\pgfqpoint{2.642560in}{3.294969in}}%
\pgfpathlineto{\pgfqpoint{2.531663in}{3.230943in}}%
\pgfpathlineto{\pgfqpoint{2.651156in}{3.177874in}}%
\pgfpathclose%
\pgfusepath{fill}%
\end{pgfscope}%
\begin{pgfscope}%
\pgfpathrectangle{\pgfqpoint{0.765000in}{0.660000in}}{\pgfqpoint{4.620000in}{4.620000in}}%
\pgfusepath{clip}%
\pgfsetbuttcap%
\pgfsetroundjoin%
\definecolor{currentfill}{rgb}{1.000000,0.894118,0.788235}%
\pgfsetfillcolor{currentfill}%
\pgfsetlinewidth{0.000000pt}%
\definecolor{currentstroke}{rgb}{1.000000,0.894118,0.788235}%
\pgfsetstrokecolor{currentstroke}%
\pgfsetdash{}{0pt}%
\pgfpathmoveto{\pgfqpoint{2.651156in}{3.305927in}}%
\pgfpathlineto{\pgfqpoint{2.762052in}{3.369953in}}%
\pgfpathlineto{\pgfqpoint{2.753457in}{3.358995in}}%
\pgfpathlineto{\pgfqpoint{2.642560in}{3.294969in}}%
\pgfpathlineto{\pgfqpoint{2.651156in}{3.305927in}}%
\pgfpathclose%
\pgfusepath{fill}%
\end{pgfscope}%
\begin{pgfscope}%
\pgfpathrectangle{\pgfqpoint{0.765000in}{0.660000in}}{\pgfqpoint{4.620000in}{4.620000in}}%
\pgfusepath{clip}%
\pgfsetbuttcap%
\pgfsetroundjoin%
\definecolor{currentfill}{rgb}{1.000000,0.894118,0.788235}%
\pgfsetfillcolor{currentfill}%
\pgfsetlinewidth{0.000000pt}%
\definecolor{currentstroke}{rgb}{1.000000,0.894118,0.788235}%
\pgfsetstrokecolor{currentstroke}%
\pgfsetdash{}{0pt}%
\pgfpathmoveto{\pgfqpoint{2.540259in}{3.369953in}}%
\pgfpathlineto{\pgfqpoint{2.651156in}{3.305927in}}%
\pgfpathlineto{\pgfqpoint{2.642560in}{3.294969in}}%
\pgfpathlineto{\pgfqpoint{2.531663in}{3.358995in}}%
\pgfpathlineto{\pgfqpoint{2.540259in}{3.369953in}}%
\pgfpathclose%
\pgfusepath{fill}%
\end{pgfscope}%
\begin{pgfscope}%
\pgfpathrectangle{\pgfqpoint{0.765000in}{0.660000in}}{\pgfqpoint{4.620000in}{4.620000in}}%
\pgfusepath{clip}%
\pgfsetbuttcap%
\pgfsetroundjoin%
\definecolor{currentfill}{rgb}{1.000000,0.894118,0.788235}%
\pgfsetfillcolor{currentfill}%
\pgfsetlinewidth{0.000000pt}%
\definecolor{currentstroke}{rgb}{1.000000,0.894118,0.788235}%
\pgfsetstrokecolor{currentstroke}%
\pgfsetdash{}{0pt}%
\pgfpathmoveto{\pgfqpoint{2.651156in}{3.305927in}}%
\pgfpathlineto{\pgfqpoint{2.540259in}{3.241901in}}%
\pgfpathlineto{\pgfqpoint{2.651156in}{3.177874in}}%
\pgfpathlineto{\pgfqpoint{2.762052in}{3.241901in}}%
\pgfpathlineto{\pgfqpoint{2.651156in}{3.305927in}}%
\pgfpathclose%
\pgfusepath{fill}%
\end{pgfscope}%
\begin{pgfscope}%
\pgfpathrectangle{\pgfqpoint{0.765000in}{0.660000in}}{\pgfqpoint{4.620000in}{4.620000in}}%
\pgfusepath{clip}%
\pgfsetbuttcap%
\pgfsetroundjoin%
\definecolor{currentfill}{rgb}{1.000000,0.894118,0.788235}%
\pgfsetfillcolor{currentfill}%
\pgfsetlinewidth{0.000000pt}%
\definecolor{currentstroke}{rgb}{1.000000,0.894118,0.788235}%
\pgfsetstrokecolor{currentstroke}%
\pgfsetdash{}{0pt}%
\pgfpathmoveto{\pgfqpoint{2.651156in}{3.305927in}}%
\pgfpathlineto{\pgfqpoint{2.540259in}{3.241901in}}%
\pgfpathlineto{\pgfqpoint{2.540259in}{3.369953in}}%
\pgfpathlineto{\pgfqpoint{2.651156in}{3.433979in}}%
\pgfpathlineto{\pgfqpoint{2.651156in}{3.305927in}}%
\pgfpathclose%
\pgfusepath{fill}%
\end{pgfscope}%
\begin{pgfscope}%
\pgfpathrectangle{\pgfqpoint{0.765000in}{0.660000in}}{\pgfqpoint{4.620000in}{4.620000in}}%
\pgfusepath{clip}%
\pgfsetbuttcap%
\pgfsetroundjoin%
\definecolor{currentfill}{rgb}{1.000000,0.894118,0.788235}%
\pgfsetfillcolor{currentfill}%
\pgfsetlinewidth{0.000000pt}%
\definecolor{currentstroke}{rgb}{1.000000,0.894118,0.788235}%
\pgfsetstrokecolor{currentstroke}%
\pgfsetdash{}{0pt}%
\pgfpathmoveto{\pgfqpoint{2.651156in}{3.305927in}}%
\pgfpathlineto{\pgfqpoint{2.762052in}{3.241901in}}%
\pgfpathlineto{\pgfqpoint{2.762052in}{3.369953in}}%
\pgfpathlineto{\pgfqpoint{2.651156in}{3.433979in}}%
\pgfpathlineto{\pgfqpoint{2.651156in}{3.305927in}}%
\pgfpathclose%
\pgfusepath{fill}%
\end{pgfscope}%
\begin{pgfscope}%
\pgfpathrectangle{\pgfqpoint{0.765000in}{0.660000in}}{\pgfqpoint{4.620000in}{4.620000in}}%
\pgfusepath{clip}%
\pgfsetbuttcap%
\pgfsetroundjoin%
\definecolor{currentfill}{rgb}{1.000000,0.894118,0.788235}%
\pgfsetfillcolor{currentfill}%
\pgfsetlinewidth{0.000000pt}%
\definecolor{currentstroke}{rgb}{1.000000,0.894118,0.788235}%
\pgfsetstrokecolor{currentstroke}%
\pgfsetdash{}{0pt}%
\pgfpathmoveto{\pgfqpoint{2.671300in}{3.327008in}}%
\pgfpathlineto{\pgfqpoint{2.560403in}{3.262981in}}%
\pgfpathlineto{\pgfqpoint{2.671300in}{3.198955in}}%
\pgfpathlineto{\pgfqpoint{2.782196in}{3.262981in}}%
\pgfpathlineto{\pgfqpoint{2.671300in}{3.327008in}}%
\pgfpathclose%
\pgfusepath{fill}%
\end{pgfscope}%
\begin{pgfscope}%
\pgfpathrectangle{\pgfqpoint{0.765000in}{0.660000in}}{\pgfqpoint{4.620000in}{4.620000in}}%
\pgfusepath{clip}%
\pgfsetbuttcap%
\pgfsetroundjoin%
\definecolor{currentfill}{rgb}{1.000000,0.894118,0.788235}%
\pgfsetfillcolor{currentfill}%
\pgfsetlinewidth{0.000000pt}%
\definecolor{currentstroke}{rgb}{1.000000,0.894118,0.788235}%
\pgfsetstrokecolor{currentstroke}%
\pgfsetdash{}{0pt}%
\pgfpathmoveto{\pgfqpoint{2.671300in}{3.327008in}}%
\pgfpathlineto{\pgfqpoint{2.560403in}{3.262981in}}%
\pgfpathlineto{\pgfqpoint{2.560403in}{3.391034in}}%
\pgfpathlineto{\pgfqpoint{2.671300in}{3.455060in}}%
\pgfpathlineto{\pgfqpoint{2.671300in}{3.327008in}}%
\pgfpathclose%
\pgfusepath{fill}%
\end{pgfscope}%
\begin{pgfscope}%
\pgfpathrectangle{\pgfqpoint{0.765000in}{0.660000in}}{\pgfqpoint{4.620000in}{4.620000in}}%
\pgfusepath{clip}%
\pgfsetbuttcap%
\pgfsetroundjoin%
\definecolor{currentfill}{rgb}{1.000000,0.894118,0.788235}%
\pgfsetfillcolor{currentfill}%
\pgfsetlinewidth{0.000000pt}%
\definecolor{currentstroke}{rgb}{1.000000,0.894118,0.788235}%
\pgfsetstrokecolor{currentstroke}%
\pgfsetdash{}{0pt}%
\pgfpathmoveto{\pgfqpoint{2.671300in}{3.327008in}}%
\pgfpathlineto{\pgfqpoint{2.782196in}{3.262981in}}%
\pgfpathlineto{\pgfqpoint{2.782196in}{3.391034in}}%
\pgfpathlineto{\pgfqpoint{2.671300in}{3.455060in}}%
\pgfpathlineto{\pgfqpoint{2.671300in}{3.327008in}}%
\pgfpathclose%
\pgfusepath{fill}%
\end{pgfscope}%
\begin{pgfscope}%
\pgfpathrectangle{\pgfqpoint{0.765000in}{0.660000in}}{\pgfqpoint{4.620000in}{4.620000in}}%
\pgfusepath{clip}%
\pgfsetbuttcap%
\pgfsetroundjoin%
\definecolor{currentfill}{rgb}{1.000000,0.894118,0.788235}%
\pgfsetfillcolor{currentfill}%
\pgfsetlinewidth{0.000000pt}%
\definecolor{currentstroke}{rgb}{1.000000,0.894118,0.788235}%
\pgfsetstrokecolor{currentstroke}%
\pgfsetdash{}{0pt}%
\pgfpathmoveto{\pgfqpoint{2.651156in}{3.433979in}}%
\pgfpathlineto{\pgfqpoint{2.540259in}{3.369953in}}%
\pgfpathlineto{\pgfqpoint{2.651156in}{3.305927in}}%
\pgfpathlineto{\pgfqpoint{2.762052in}{3.369953in}}%
\pgfpathlineto{\pgfqpoint{2.651156in}{3.433979in}}%
\pgfpathclose%
\pgfusepath{fill}%
\end{pgfscope}%
\begin{pgfscope}%
\pgfpathrectangle{\pgfqpoint{0.765000in}{0.660000in}}{\pgfqpoint{4.620000in}{4.620000in}}%
\pgfusepath{clip}%
\pgfsetbuttcap%
\pgfsetroundjoin%
\definecolor{currentfill}{rgb}{1.000000,0.894118,0.788235}%
\pgfsetfillcolor{currentfill}%
\pgfsetlinewidth{0.000000pt}%
\definecolor{currentstroke}{rgb}{1.000000,0.894118,0.788235}%
\pgfsetstrokecolor{currentstroke}%
\pgfsetdash{}{0pt}%
\pgfpathmoveto{\pgfqpoint{2.651156in}{3.177874in}}%
\pgfpathlineto{\pgfqpoint{2.762052in}{3.241901in}}%
\pgfpathlineto{\pgfqpoint{2.762052in}{3.369953in}}%
\pgfpathlineto{\pgfqpoint{2.651156in}{3.305927in}}%
\pgfpathlineto{\pgfqpoint{2.651156in}{3.177874in}}%
\pgfpathclose%
\pgfusepath{fill}%
\end{pgfscope}%
\begin{pgfscope}%
\pgfpathrectangle{\pgfqpoint{0.765000in}{0.660000in}}{\pgfqpoint{4.620000in}{4.620000in}}%
\pgfusepath{clip}%
\pgfsetbuttcap%
\pgfsetroundjoin%
\definecolor{currentfill}{rgb}{1.000000,0.894118,0.788235}%
\pgfsetfillcolor{currentfill}%
\pgfsetlinewidth{0.000000pt}%
\definecolor{currentstroke}{rgb}{1.000000,0.894118,0.788235}%
\pgfsetstrokecolor{currentstroke}%
\pgfsetdash{}{0pt}%
\pgfpathmoveto{\pgfqpoint{2.540259in}{3.241901in}}%
\pgfpathlineto{\pgfqpoint{2.651156in}{3.177874in}}%
\pgfpathlineto{\pgfqpoint{2.651156in}{3.305927in}}%
\pgfpathlineto{\pgfqpoint{2.540259in}{3.369953in}}%
\pgfpathlineto{\pgfqpoint{2.540259in}{3.241901in}}%
\pgfpathclose%
\pgfusepath{fill}%
\end{pgfscope}%
\begin{pgfscope}%
\pgfpathrectangle{\pgfqpoint{0.765000in}{0.660000in}}{\pgfqpoint{4.620000in}{4.620000in}}%
\pgfusepath{clip}%
\pgfsetbuttcap%
\pgfsetroundjoin%
\definecolor{currentfill}{rgb}{1.000000,0.894118,0.788235}%
\pgfsetfillcolor{currentfill}%
\pgfsetlinewidth{0.000000pt}%
\definecolor{currentstroke}{rgb}{1.000000,0.894118,0.788235}%
\pgfsetstrokecolor{currentstroke}%
\pgfsetdash{}{0pt}%
\pgfpathmoveto{\pgfqpoint{2.671300in}{3.455060in}}%
\pgfpathlineto{\pgfqpoint{2.560403in}{3.391034in}}%
\pgfpathlineto{\pgfqpoint{2.671300in}{3.327008in}}%
\pgfpathlineto{\pgfqpoint{2.782196in}{3.391034in}}%
\pgfpathlineto{\pgfqpoint{2.671300in}{3.455060in}}%
\pgfpathclose%
\pgfusepath{fill}%
\end{pgfscope}%
\begin{pgfscope}%
\pgfpathrectangle{\pgfqpoint{0.765000in}{0.660000in}}{\pgfqpoint{4.620000in}{4.620000in}}%
\pgfusepath{clip}%
\pgfsetbuttcap%
\pgfsetroundjoin%
\definecolor{currentfill}{rgb}{1.000000,0.894118,0.788235}%
\pgfsetfillcolor{currentfill}%
\pgfsetlinewidth{0.000000pt}%
\definecolor{currentstroke}{rgb}{1.000000,0.894118,0.788235}%
\pgfsetstrokecolor{currentstroke}%
\pgfsetdash{}{0pt}%
\pgfpathmoveto{\pgfqpoint{2.671300in}{3.198955in}}%
\pgfpathlineto{\pgfqpoint{2.782196in}{3.262981in}}%
\pgfpathlineto{\pgfqpoint{2.782196in}{3.391034in}}%
\pgfpathlineto{\pgfqpoint{2.671300in}{3.327008in}}%
\pgfpathlineto{\pgfqpoint{2.671300in}{3.198955in}}%
\pgfpathclose%
\pgfusepath{fill}%
\end{pgfscope}%
\begin{pgfscope}%
\pgfpathrectangle{\pgfqpoint{0.765000in}{0.660000in}}{\pgfqpoint{4.620000in}{4.620000in}}%
\pgfusepath{clip}%
\pgfsetbuttcap%
\pgfsetroundjoin%
\definecolor{currentfill}{rgb}{1.000000,0.894118,0.788235}%
\pgfsetfillcolor{currentfill}%
\pgfsetlinewidth{0.000000pt}%
\definecolor{currentstroke}{rgb}{1.000000,0.894118,0.788235}%
\pgfsetstrokecolor{currentstroke}%
\pgfsetdash{}{0pt}%
\pgfpathmoveto{\pgfqpoint{2.560403in}{3.262981in}}%
\pgfpathlineto{\pgfqpoint{2.671300in}{3.198955in}}%
\pgfpathlineto{\pgfqpoint{2.671300in}{3.327008in}}%
\pgfpathlineto{\pgfqpoint{2.560403in}{3.391034in}}%
\pgfpathlineto{\pgfqpoint{2.560403in}{3.262981in}}%
\pgfpathclose%
\pgfusepath{fill}%
\end{pgfscope}%
\begin{pgfscope}%
\pgfpathrectangle{\pgfqpoint{0.765000in}{0.660000in}}{\pgfqpoint{4.620000in}{4.620000in}}%
\pgfusepath{clip}%
\pgfsetbuttcap%
\pgfsetroundjoin%
\definecolor{currentfill}{rgb}{1.000000,0.894118,0.788235}%
\pgfsetfillcolor{currentfill}%
\pgfsetlinewidth{0.000000pt}%
\definecolor{currentstroke}{rgb}{1.000000,0.894118,0.788235}%
\pgfsetstrokecolor{currentstroke}%
\pgfsetdash{}{0pt}%
\pgfpathmoveto{\pgfqpoint{2.651156in}{3.305927in}}%
\pgfpathlineto{\pgfqpoint{2.540259in}{3.241901in}}%
\pgfpathlineto{\pgfqpoint{2.560403in}{3.262981in}}%
\pgfpathlineto{\pgfqpoint{2.671300in}{3.327008in}}%
\pgfpathlineto{\pgfqpoint{2.651156in}{3.305927in}}%
\pgfpathclose%
\pgfusepath{fill}%
\end{pgfscope}%
\begin{pgfscope}%
\pgfpathrectangle{\pgfqpoint{0.765000in}{0.660000in}}{\pgfqpoint{4.620000in}{4.620000in}}%
\pgfusepath{clip}%
\pgfsetbuttcap%
\pgfsetroundjoin%
\definecolor{currentfill}{rgb}{1.000000,0.894118,0.788235}%
\pgfsetfillcolor{currentfill}%
\pgfsetlinewidth{0.000000pt}%
\definecolor{currentstroke}{rgb}{1.000000,0.894118,0.788235}%
\pgfsetstrokecolor{currentstroke}%
\pgfsetdash{}{0pt}%
\pgfpathmoveto{\pgfqpoint{2.762052in}{3.241901in}}%
\pgfpathlineto{\pgfqpoint{2.651156in}{3.305927in}}%
\pgfpathlineto{\pgfqpoint{2.671300in}{3.327008in}}%
\pgfpathlineto{\pgfqpoint{2.782196in}{3.262981in}}%
\pgfpathlineto{\pgfqpoint{2.762052in}{3.241901in}}%
\pgfpathclose%
\pgfusepath{fill}%
\end{pgfscope}%
\begin{pgfscope}%
\pgfpathrectangle{\pgfqpoint{0.765000in}{0.660000in}}{\pgfqpoint{4.620000in}{4.620000in}}%
\pgfusepath{clip}%
\pgfsetbuttcap%
\pgfsetroundjoin%
\definecolor{currentfill}{rgb}{1.000000,0.894118,0.788235}%
\pgfsetfillcolor{currentfill}%
\pgfsetlinewidth{0.000000pt}%
\definecolor{currentstroke}{rgb}{1.000000,0.894118,0.788235}%
\pgfsetstrokecolor{currentstroke}%
\pgfsetdash{}{0pt}%
\pgfpathmoveto{\pgfqpoint{2.651156in}{3.305927in}}%
\pgfpathlineto{\pgfqpoint{2.651156in}{3.433979in}}%
\pgfpathlineto{\pgfqpoint{2.671300in}{3.455060in}}%
\pgfpathlineto{\pgfqpoint{2.782196in}{3.262981in}}%
\pgfpathlineto{\pgfqpoint{2.651156in}{3.305927in}}%
\pgfpathclose%
\pgfusepath{fill}%
\end{pgfscope}%
\begin{pgfscope}%
\pgfpathrectangle{\pgfqpoint{0.765000in}{0.660000in}}{\pgfqpoint{4.620000in}{4.620000in}}%
\pgfusepath{clip}%
\pgfsetbuttcap%
\pgfsetroundjoin%
\definecolor{currentfill}{rgb}{1.000000,0.894118,0.788235}%
\pgfsetfillcolor{currentfill}%
\pgfsetlinewidth{0.000000pt}%
\definecolor{currentstroke}{rgb}{1.000000,0.894118,0.788235}%
\pgfsetstrokecolor{currentstroke}%
\pgfsetdash{}{0pt}%
\pgfpathmoveto{\pgfqpoint{2.762052in}{3.241901in}}%
\pgfpathlineto{\pgfqpoint{2.762052in}{3.369953in}}%
\pgfpathlineto{\pgfqpoint{2.782196in}{3.391034in}}%
\pgfpathlineto{\pgfqpoint{2.671300in}{3.327008in}}%
\pgfpathlineto{\pgfqpoint{2.762052in}{3.241901in}}%
\pgfpathclose%
\pgfusepath{fill}%
\end{pgfscope}%
\begin{pgfscope}%
\pgfpathrectangle{\pgfqpoint{0.765000in}{0.660000in}}{\pgfqpoint{4.620000in}{4.620000in}}%
\pgfusepath{clip}%
\pgfsetbuttcap%
\pgfsetroundjoin%
\definecolor{currentfill}{rgb}{1.000000,0.894118,0.788235}%
\pgfsetfillcolor{currentfill}%
\pgfsetlinewidth{0.000000pt}%
\definecolor{currentstroke}{rgb}{1.000000,0.894118,0.788235}%
\pgfsetstrokecolor{currentstroke}%
\pgfsetdash{}{0pt}%
\pgfpathmoveto{\pgfqpoint{2.540259in}{3.241901in}}%
\pgfpathlineto{\pgfqpoint{2.651156in}{3.177874in}}%
\pgfpathlineto{\pgfqpoint{2.671300in}{3.198955in}}%
\pgfpathlineto{\pgfqpoint{2.560403in}{3.262981in}}%
\pgfpathlineto{\pgfqpoint{2.540259in}{3.241901in}}%
\pgfpathclose%
\pgfusepath{fill}%
\end{pgfscope}%
\begin{pgfscope}%
\pgfpathrectangle{\pgfqpoint{0.765000in}{0.660000in}}{\pgfqpoint{4.620000in}{4.620000in}}%
\pgfusepath{clip}%
\pgfsetbuttcap%
\pgfsetroundjoin%
\definecolor{currentfill}{rgb}{1.000000,0.894118,0.788235}%
\pgfsetfillcolor{currentfill}%
\pgfsetlinewidth{0.000000pt}%
\definecolor{currentstroke}{rgb}{1.000000,0.894118,0.788235}%
\pgfsetstrokecolor{currentstroke}%
\pgfsetdash{}{0pt}%
\pgfpathmoveto{\pgfqpoint{2.651156in}{3.177874in}}%
\pgfpathlineto{\pgfqpoint{2.762052in}{3.241901in}}%
\pgfpathlineto{\pgfqpoint{2.782196in}{3.262981in}}%
\pgfpathlineto{\pgfqpoint{2.671300in}{3.198955in}}%
\pgfpathlineto{\pgfqpoint{2.651156in}{3.177874in}}%
\pgfpathclose%
\pgfusepath{fill}%
\end{pgfscope}%
\begin{pgfscope}%
\pgfpathrectangle{\pgfqpoint{0.765000in}{0.660000in}}{\pgfqpoint{4.620000in}{4.620000in}}%
\pgfusepath{clip}%
\pgfsetbuttcap%
\pgfsetroundjoin%
\definecolor{currentfill}{rgb}{1.000000,0.894118,0.788235}%
\pgfsetfillcolor{currentfill}%
\pgfsetlinewidth{0.000000pt}%
\definecolor{currentstroke}{rgb}{1.000000,0.894118,0.788235}%
\pgfsetstrokecolor{currentstroke}%
\pgfsetdash{}{0pt}%
\pgfpathmoveto{\pgfqpoint{2.651156in}{3.433979in}}%
\pgfpathlineto{\pgfqpoint{2.540259in}{3.369953in}}%
\pgfpathlineto{\pgfqpoint{2.560403in}{3.391034in}}%
\pgfpathlineto{\pgfqpoint{2.671300in}{3.455060in}}%
\pgfpathlineto{\pgfqpoint{2.651156in}{3.433979in}}%
\pgfpathclose%
\pgfusepath{fill}%
\end{pgfscope}%
\begin{pgfscope}%
\pgfpathrectangle{\pgfqpoint{0.765000in}{0.660000in}}{\pgfqpoint{4.620000in}{4.620000in}}%
\pgfusepath{clip}%
\pgfsetbuttcap%
\pgfsetroundjoin%
\definecolor{currentfill}{rgb}{1.000000,0.894118,0.788235}%
\pgfsetfillcolor{currentfill}%
\pgfsetlinewidth{0.000000pt}%
\definecolor{currentstroke}{rgb}{1.000000,0.894118,0.788235}%
\pgfsetstrokecolor{currentstroke}%
\pgfsetdash{}{0pt}%
\pgfpathmoveto{\pgfqpoint{2.762052in}{3.369953in}}%
\pgfpathlineto{\pgfqpoint{2.651156in}{3.433979in}}%
\pgfpathlineto{\pgfqpoint{2.671300in}{3.455060in}}%
\pgfpathlineto{\pgfqpoint{2.782196in}{3.391034in}}%
\pgfpathlineto{\pgfqpoint{2.762052in}{3.369953in}}%
\pgfpathclose%
\pgfusepath{fill}%
\end{pgfscope}%
\begin{pgfscope}%
\pgfpathrectangle{\pgfqpoint{0.765000in}{0.660000in}}{\pgfqpoint{4.620000in}{4.620000in}}%
\pgfusepath{clip}%
\pgfsetbuttcap%
\pgfsetroundjoin%
\definecolor{currentfill}{rgb}{1.000000,0.894118,0.788235}%
\pgfsetfillcolor{currentfill}%
\pgfsetlinewidth{0.000000pt}%
\definecolor{currentstroke}{rgb}{1.000000,0.894118,0.788235}%
\pgfsetstrokecolor{currentstroke}%
\pgfsetdash{}{0pt}%
\pgfpathmoveto{\pgfqpoint{2.540259in}{3.241901in}}%
\pgfpathlineto{\pgfqpoint{2.540259in}{3.369953in}}%
\pgfpathlineto{\pgfqpoint{2.560403in}{3.391034in}}%
\pgfpathlineto{\pgfqpoint{2.671300in}{3.198955in}}%
\pgfpathlineto{\pgfqpoint{2.540259in}{3.241901in}}%
\pgfpathclose%
\pgfusepath{fill}%
\end{pgfscope}%
\begin{pgfscope}%
\pgfpathrectangle{\pgfqpoint{0.765000in}{0.660000in}}{\pgfqpoint{4.620000in}{4.620000in}}%
\pgfusepath{clip}%
\pgfsetbuttcap%
\pgfsetroundjoin%
\definecolor{currentfill}{rgb}{1.000000,0.894118,0.788235}%
\pgfsetfillcolor{currentfill}%
\pgfsetlinewidth{0.000000pt}%
\definecolor{currentstroke}{rgb}{1.000000,0.894118,0.788235}%
\pgfsetstrokecolor{currentstroke}%
\pgfsetdash{}{0pt}%
\pgfpathmoveto{\pgfqpoint{2.651156in}{3.177874in}}%
\pgfpathlineto{\pgfqpoint{2.651156in}{3.305927in}}%
\pgfpathlineto{\pgfqpoint{2.671300in}{3.327008in}}%
\pgfpathlineto{\pgfqpoint{2.560403in}{3.262981in}}%
\pgfpathlineto{\pgfqpoint{2.651156in}{3.177874in}}%
\pgfpathclose%
\pgfusepath{fill}%
\end{pgfscope}%
\begin{pgfscope}%
\pgfpathrectangle{\pgfqpoint{0.765000in}{0.660000in}}{\pgfqpoint{4.620000in}{4.620000in}}%
\pgfusepath{clip}%
\pgfsetbuttcap%
\pgfsetroundjoin%
\definecolor{currentfill}{rgb}{1.000000,0.894118,0.788235}%
\pgfsetfillcolor{currentfill}%
\pgfsetlinewidth{0.000000pt}%
\definecolor{currentstroke}{rgb}{1.000000,0.894118,0.788235}%
\pgfsetstrokecolor{currentstroke}%
\pgfsetdash{}{0pt}%
\pgfpathmoveto{\pgfqpoint{2.651156in}{3.305927in}}%
\pgfpathlineto{\pgfqpoint{2.762052in}{3.369953in}}%
\pgfpathlineto{\pgfqpoint{2.782196in}{3.391034in}}%
\pgfpathlineto{\pgfqpoint{2.671300in}{3.327008in}}%
\pgfpathlineto{\pgfqpoint{2.651156in}{3.305927in}}%
\pgfpathclose%
\pgfusepath{fill}%
\end{pgfscope}%
\begin{pgfscope}%
\pgfpathrectangle{\pgfqpoint{0.765000in}{0.660000in}}{\pgfqpoint{4.620000in}{4.620000in}}%
\pgfusepath{clip}%
\pgfsetbuttcap%
\pgfsetroundjoin%
\definecolor{currentfill}{rgb}{1.000000,0.894118,0.788235}%
\pgfsetfillcolor{currentfill}%
\pgfsetlinewidth{0.000000pt}%
\definecolor{currentstroke}{rgb}{1.000000,0.894118,0.788235}%
\pgfsetstrokecolor{currentstroke}%
\pgfsetdash{}{0pt}%
\pgfpathmoveto{\pgfqpoint{2.540259in}{3.369953in}}%
\pgfpathlineto{\pgfqpoint{2.651156in}{3.305927in}}%
\pgfpathlineto{\pgfqpoint{2.671300in}{3.327008in}}%
\pgfpathlineto{\pgfqpoint{2.560403in}{3.391034in}}%
\pgfpathlineto{\pgfqpoint{2.540259in}{3.369953in}}%
\pgfpathclose%
\pgfusepath{fill}%
\end{pgfscope}%
\begin{pgfscope}%
\pgfpathrectangle{\pgfqpoint{0.765000in}{0.660000in}}{\pgfqpoint{4.620000in}{4.620000in}}%
\pgfusepath{clip}%
\pgfsetbuttcap%
\pgfsetroundjoin%
\definecolor{currentfill}{rgb}{1.000000,0.894118,0.788235}%
\pgfsetfillcolor{currentfill}%
\pgfsetlinewidth{0.000000pt}%
\definecolor{currentstroke}{rgb}{1.000000,0.894118,0.788235}%
\pgfsetstrokecolor{currentstroke}%
\pgfsetdash{}{0pt}%
\pgfpathmoveto{\pgfqpoint{2.642560in}{3.294969in}}%
\pgfpathlineto{\pgfqpoint{2.531663in}{3.230943in}}%
\pgfpathlineto{\pgfqpoint{2.642560in}{3.166917in}}%
\pgfpathlineto{\pgfqpoint{2.753457in}{3.230943in}}%
\pgfpathlineto{\pgfqpoint{2.642560in}{3.294969in}}%
\pgfpathclose%
\pgfusepath{fill}%
\end{pgfscope}%
\begin{pgfscope}%
\pgfpathrectangle{\pgfqpoint{0.765000in}{0.660000in}}{\pgfqpoint{4.620000in}{4.620000in}}%
\pgfusepath{clip}%
\pgfsetbuttcap%
\pgfsetroundjoin%
\definecolor{currentfill}{rgb}{1.000000,0.894118,0.788235}%
\pgfsetfillcolor{currentfill}%
\pgfsetlinewidth{0.000000pt}%
\definecolor{currentstroke}{rgb}{1.000000,0.894118,0.788235}%
\pgfsetstrokecolor{currentstroke}%
\pgfsetdash{}{0pt}%
\pgfpathmoveto{\pgfqpoint{2.642560in}{3.294969in}}%
\pgfpathlineto{\pgfqpoint{2.531663in}{3.230943in}}%
\pgfpathlineto{\pgfqpoint{2.531663in}{3.358995in}}%
\pgfpathlineto{\pgfqpoint{2.642560in}{3.423022in}}%
\pgfpathlineto{\pgfqpoint{2.642560in}{3.294969in}}%
\pgfpathclose%
\pgfusepath{fill}%
\end{pgfscope}%
\begin{pgfscope}%
\pgfpathrectangle{\pgfqpoint{0.765000in}{0.660000in}}{\pgfqpoint{4.620000in}{4.620000in}}%
\pgfusepath{clip}%
\pgfsetbuttcap%
\pgfsetroundjoin%
\definecolor{currentfill}{rgb}{1.000000,0.894118,0.788235}%
\pgfsetfillcolor{currentfill}%
\pgfsetlinewidth{0.000000pt}%
\definecolor{currentstroke}{rgb}{1.000000,0.894118,0.788235}%
\pgfsetstrokecolor{currentstroke}%
\pgfsetdash{}{0pt}%
\pgfpathmoveto{\pgfqpoint{2.642560in}{3.294969in}}%
\pgfpathlineto{\pgfqpoint{2.753457in}{3.230943in}}%
\pgfpathlineto{\pgfqpoint{2.753457in}{3.358995in}}%
\pgfpathlineto{\pgfqpoint{2.642560in}{3.423022in}}%
\pgfpathlineto{\pgfqpoint{2.642560in}{3.294969in}}%
\pgfpathclose%
\pgfusepath{fill}%
\end{pgfscope}%
\begin{pgfscope}%
\pgfpathrectangle{\pgfqpoint{0.765000in}{0.660000in}}{\pgfqpoint{4.620000in}{4.620000in}}%
\pgfusepath{clip}%
\pgfsetbuttcap%
\pgfsetroundjoin%
\definecolor{currentfill}{rgb}{1.000000,0.894118,0.788235}%
\pgfsetfillcolor{currentfill}%
\pgfsetlinewidth{0.000000pt}%
\definecolor{currentstroke}{rgb}{1.000000,0.894118,0.788235}%
\pgfsetstrokecolor{currentstroke}%
\pgfsetdash{}{0pt}%
\pgfpathmoveto{\pgfqpoint{2.635438in}{3.285868in}}%
\pgfpathlineto{\pgfqpoint{2.524541in}{3.221842in}}%
\pgfpathlineto{\pgfqpoint{2.635438in}{3.157816in}}%
\pgfpathlineto{\pgfqpoint{2.746335in}{3.221842in}}%
\pgfpathlineto{\pgfqpoint{2.635438in}{3.285868in}}%
\pgfpathclose%
\pgfusepath{fill}%
\end{pgfscope}%
\begin{pgfscope}%
\pgfpathrectangle{\pgfqpoint{0.765000in}{0.660000in}}{\pgfqpoint{4.620000in}{4.620000in}}%
\pgfusepath{clip}%
\pgfsetbuttcap%
\pgfsetroundjoin%
\definecolor{currentfill}{rgb}{1.000000,0.894118,0.788235}%
\pgfsetfillcolor{currentfill}%
\pgfsetlinewidth{0.000000pt}%
\definecolor{currentstroke}{rgb}{1.000000,0.894118,0.788235}%
\pgfsetstrokecolor{currentstroke}%
\pgfsetdash{}{0pt}%
\pgfpathmoveto{\pgfqpoint{2.635438in}{3.285868in}}%
\pgfpathlineto{\pgfqpoint{2.524541in}{3.221842in}}%
\pgfpathlineto{\pgfqpoint{2.524541in}{3.349894in}}%
\pgfpathlineto{\pgfqpoint{2.635438in}{3.413921in}}%
\pgfpathlineto{\pgfqpoint{2.635438in}{3.285868in}}%
\pgfpathclose%
\pgfusepath{fill}%
\end{pgfscope}%
\begin{pgfscope}%
\pgfpathrectangle{\pgfqpoint{0.765000in}{0.660000in}}{\pgfqpoint{4.620000in}{4.620000in}}%
\pgfusepath{clip}%
\pgfsetbuttcap%
\pgfsetroundjoin%
\definecolor{currentfill}{rgb}{1.000000,0.894118,0.788235}%
\pgfsetfillcolor{currentfill}%
\pgfsetlinewidth{0.000000pt}%
\definecolor{currentstroke}{rgb}{1.000000,0.894118,0.788235}%
\pgfsetstrokecolor{currentstroke}%
\pgfsetdash{}{0pt}%
\pgfpathmoveto{\pgfqpoint{2.635438in}{3.285868in}}%
\pgfpathlineto{\pgfqpoint{2.746335in}{3.221842in}}%
\pgfpathlineto{\pgfqpoint{2.746335in}{3.349894in}}%
\pgfpathlineto{\pgfqpoint{2.635438in}{3.413921in}}%
\pgfpathlineto{\pgfqpoint{2.635438in}{3.285868in}}%
\pgfpathclose%
\pgfusepath{fill}%
\end{pgfscope}%
\begin{pgfscope}%
\pgfpathrectangle{\pgfqpoint{0.765000in}{0.660000in}}{\pgfqpoint{4.620000in}{4.620000in}}%
\pgfusepath{clip}%
\pgfsetbuttcap%
\pgfsetroundjoin%
\definecolor{currentfill}{rgb}{1.000000,0.894118,0.788235}%
\pgfsetfillcolor{currentfill}%
\pgfsetlinewidth{0.000000pt}%
\definecolor{currentstroke}{rgb}{1.000000,0.894118,0.788235}%
\pgfsetstrokecolor{currentstroke}%
\pgfsetdash{}{0pt}%
\pgfpathmoveto{\pgfqpoint{2.642560in}{3.423022in}}%
\pgfpathlineto{\pgfqpoint{2.531663in}{3.358995in}}%
\pgfpathlineto{\pgfqpoint{2.642560in}{3.294969in}}%
\pgfpathlineto{\pgfqpoint{2.753457in}{3.358995in}}%
\pgfpathlineto{\pgfqpoint{2.642560in}{3.423022in}}%
\pgfpathclose%
\pgfusepath{fill}%
\end{pgfscope}%
\begin{pgfscope}%
\pgfpathrectangle{\pgfqpoint{0.765000in}{0.660000in}}{\pgfqpoint{4.620000in}{4.620000in}}%
\pgfusepath{clip}%
\pgfsetbuttcap%
\pgfsetroundjoin%
\definecolor{currentfill}{rgb}{1.000000,0.894118,0.788235}%
\pgfsetfillcolor{currentfill}%
\pgfsetlinewidth{0.000000pt}%
\definecolor{currentstroke}{rgb}{1.000000,0.894118,0.788235}%
\pgfsetstrokecolor{currentstroke}%
\pgfsetdash{}{0pt}%
\pgfpathmoveto{\pgfqpoint{2.642560in}{3.166917in}}%
\pgfpathlineto{\pgfqpoint{2.753457in}{3.230943in}}%
\pgfpathlineto{\pgfqpoint{2.753457in}{3.358995in}}%
\pgfpathlineto{\pgfqpoint{2.642560in}{3.294969in}}%
\pgfpathlineto{\pgfqpoint{2.642560in}{3.166917in}}%
\pgfpathclose%
\pgfusepath{fill}%
\end{pgfscope}%
\begin{pgfscope}%
\pgfpathrectangle{\pgfqpoint{0.765000in}{0.660000in}}{\pgfqpoint{4.620000in}{4.620000in}}%
\pgfusepath{clip}%
\pgfsetbuttcap%
\pgfsetroundjoin%
\definecolor{currentfill}{rgb}{1.000000,0.894118,0.788235}%
\pgfsetfillcolor{currentfill}%
\pgfsetlinewidth{0.000000pt}%
\definecolor{currentstroke}{rgb}{1.000000,0.894118,0.788235}%
\pgfsetstrokecolor{currentstroke}%
\pgfsetdash{}{0pt}%
\pgfpathmoveto{\pgfqpoint{2.531663in}{3.230943in}}%
\pgfpathlineto{\pgfqpoint{2.642560in}{3.166917in}}%
\pgfpathlineto{\pgfqpoint{2.642560in}{3.294969in}}%
\pgfpathlineto{\pgfqpoint{2.531663in}{3.358995in}}%
\pgfpathlineto{\pgfqpoint{2.531663in}{3.230943in}}%
\pgfpathclose%
\pgfusepath{fill}%
\end{pgfscope}%
\begin{pgfscope}%
\pgfpathrectangle{\pgfqpoint{0.765000in}{0.660000in}}{\pgfqpoint{4.620000in}{4.620000in}}%
\pgfusepath{clip}%
\pgfsetbuttcap%
\pgfsetroundjoin%
\definecolor{currentfill}{rgb}{1.000000,0.894118,0.788235}%
\pgfsetfillcolor{currentfill}%
\pgfsetlinewidth{0.000000pt}%
\definecolor{currentstroke}{rgb}{1.000000,0.894118,0.788235}%
\pgfsetstrokecolor{currentstroke}%
\pgfsetdash{}{0pt}%
\pgfpathmoveto{\pgfqpoint{2.635438in}{3.413921in}}%
\pgfpathlineto{\pgfqpoint{2.524541in}{3.349894in}}%
\pgfpathlineto{\pgfqpoint{2.635438in}{3.285868in}}%
\pgfpathlineto{\pgfqpoint{2.746335in}{3.349894in}}%
\pgfpathlineto{\pgfqpoint{2.635438in}{3.413921in}}%
\pgfpathclose%
\pgfusepath{fill}%
\end{pgfscope}%
\begin{pgfscope}%
\pgfpathrectangle{\pgfqpoint{0.765000in}{0.660000in}}{\pgfqpoint{4.620000in}{4.620000in}}%
\pgfusepath{clip}%
\pgfsetbuttcap%
\pgfsetroundjoin%
\definecolor{currentfill}{rgb}{1.000000,0.894118,0.788235}%
\pgfsetfillcolor{currentfill}%
\pgfsetlinewidth{0.000000pt}%
\definecolor{currentstroke}{rgb}{1.000000,0.894118,0.788235}%
\pgfsetstrokecolor{currentstroke}%
\pgfsetdash{}{0pt}%
\pgfpathmoveto{\pgfqpoint{2.635438in}{3.157816in}}%
\pgfpathlineto{\pgfqpoint{2.746335in}{3.221842in}}%
\pgfpathlineto{\pgfqpoint{2.746335in}{3.349894in}}%
\pgfpathlineto{\pgfqpoint{2.635438in}{3.285868in}}%
\pgfpathlineto{\pgfqpoint{2.635438in}{3.157816in}}%
\pgfpathclose%
\pgfusepath{fill}%
\end{pgfscope}%
\begin{pgfscope}%
\pgfpathrectangle{\pgfqpoint{0.765000in}{0.660000in}}{\pgfqpoint{4.620000in}{4.620000in}}%
\pgfusepath{clip}%
\pgfsetbuttcap%
\pgfsetroundjoin%
\definecolor{currentfill}{rgb}{1.000000,0.894118,0.788235}%
\pgfsetfillcolor{currentfill}%
\pgfsetlinewidth{0.000000pt}%
\definecolor{currentstroke}{rgb}{1.000000,0.894118,0.788235}%
\pgfsetstrokecolor{currentstroke}%
\pgfsetdash{}{0pt}%
\pgfpathmoveto{\pgfqpoint{2.524541in}{3.221842in}}%
\pgfpathlineto{\pgfqpoint{2.635438in}{3.157816in}}%
\pgfpathlineto{\pgfqpoint{2.635438in}{3.285868in}}%
\pgfpathlineto{\pgfqpoint{2.524541in}{3.349894in}}%
\pgfpathlineto{\pgfqpoint{2.524541in}{3.221842in}}%
\pgfpathclose%
\pgfusepath{fill}%
\end{pgfscope}%
\begin{pgfscope}%
\pgfpathrectangle{\pgfqpoint{0.765000in}{0.660000in}}{\pgfqpoint{4.620000in}{4.620000in}}%
\pgfusepath{clip}%
\pgfsetbuttcap%
\pgfsetroundjoin%
\definecolor{currentfill}{rgb}{1.000000,0.894118,0.788235}%
\pgfsetfillcolor{currentfill}%
\pgfsetlinewidth{0.000000pt}%
\definecolor{currentstroke}{rgb}{1.000000,0.894118,0.788235}%
\pgfsetstrokecolor{currentstroke}%
\pgfsetdash{}{0pt}%
\pgfpathmoveto{\pgfqpoint{2.642560in}{3.294969in}}%
\pgfpathlineto{\pgfqpoint{2.531663in}{3.230943in}}%
\pgfpathlineto{\pgfqpoint{2.524541in}{3.221842in}}%
\pgfpathlineto{\pgfqpoint{2.635438in}{3.285868in}}%
\pgfpathlineto{\pgfqpoint{2.642560in}{3.294969in}}%
\pgfpathclose%
\pgfusepath{fill}%
\end{pgfscope}%
\begin{pgfscope}%
\pgfpathrectangle{\pgfqpoint{0.765000in}{0.660000in}}{\pgfqpoint{4.620000in}{4.620000in}}%
\pgfusepath{clip}%
\pgfsetbuttcap%
\pgfsetroundjoin%
\definecolor{currentfill}{rgb}{1.000000,0.894118,0.788235}%
\pgfsetfillcolor{currentfill}%
\pgfsetlinewidth{0.000000pt}%
\definecolor{currentstroke}{rgb}{1.000000,0.894118,0.788235}%
\pgfsetstrokecolor{currentstroke}%
\pgfsetdash{}{0pt}%
\pgfpathmoveto{\pgfqpoint{2.753457in}{3.230943in}}%
\pgfpathlineto{\pgfqpoint{2.642560in}{3.294969in}}%
\pgfpathlineto{\pgfqpoint{2.635438in}{3.285868in}}%
\pgfpathlineto{\pgfqpoint{2.746335in}{3.221842in}}%
\pgfpathlineto{\pgfqpoint{2.753457in}{3.230943in}}%
\pgfpathclose%
\pgfusepath{fill}%
\end{pgfscope}%
\begin{pgfscope}%
\pgfpathrectangle{\pgfqpoint{0.765000in}{0.660000in}}{\pgfqpoint{4.620000in}{4.620000in}}%
\pgfusepath{clip}%
\pgfsetbuttcap%
\pgfsetroundjoin%
\definecolor{currentfill}{rgb}{1.000000,0.894118,0.788235}%
\pgfsetfillcolor{currentfill}%
\pgfsetlinewidth{0.000000pt}%
\definecolor{currentstroke}{rgb}{1.000000,0.894118,0.788235}%
\pgfsetstrokecolor{currentstroke}%
\pgfsetdash{}{0pt}%
\pgfpathmoveto{\pgfqpoint{2.642560in}{3.294969in}}%
\pgfpathlineto{\pgfqpoint{2.642560in}{3.423022in}}%
\pgfpathlineto{\pgfqpoint{2.635438in}{3.413921in}}%
\pgfpathlineto{\pgfqpoint{2.746335in}{3.221842in}}%
\pgfpathlineto{\pgfqpoint{2.642560in}{3.294969in}}%
\pgfpathclose%
\pgfusepath{fill}%
\end{pgfscope}%
\begin{pgfscope}%
\pgfpathrectangle{\pgfqpoint{0.765000in}{0.660000in}}{\pgfqpoint{4.620000in}{4.620000in}}%
\pgfusepath{clip}%
\pgfsetbuttcap%
\pgfsetroundjoin%
\definecolor{currentfill}{rgb}{1.000000,0.894118,0.788235}%
\pgfsetfillcolor{currentfill}%
\pgfsetlinewidth{0.000000pt}%
\definecolor{currentstroke}{rgb}{1.000000,0.894118,0.788235}%
\pgfsetstrokecolor{currentstroke}%
\pgfsetdash{}{0pt}%
\pgfpathmoveto{\pgfqpoint{2.753457in}{3.230943in}}%
\pgfpathlineto{\pgfqpoint{2.753457in}{3.358995in}}%
\pgfpathlineto{\pgfqpoint{2.746335in}{3.349894in}}%
\pgfpathlineto{\pgfqpoint{2.635438in}{3.285868in}}%
\pgfpathlineto{\pgfqpoint{2.753457in}{3.230943in}}%
\pgfpathclose%
\pgfusepath{fill}%
\end{pgfscope}%
\begin{pgfscope}%
\pgfpathrectangle{\pgfqpoint{0.765000in}{0.660000in}}{\pgfqpoint{4.620000in}{4.620000in}}%
\pgfusepath{clip}%
\pgfsetbuttcap%
\pgfsetroundjoin%
\definecolor{currentfill}{rgb}{1.000000,0.894118,0.788235}%
\pgfsetfillcolor{currentfill}%
\pgfsetlinewidth{0.000000pt}%
\definecolor{currentstroke}{rgb}{1.000000,0.894118,0.788235}%
\pgfsetstrokecolor{currentstroke}%
\pgfsetdash{}{0pt}%
\pgfpathmoveto{\pgfqpoint{2.531663in}{3.230943in}}%
\pgfpathlineto{\pgfqpoint{2.642560in}{3.166917in}}%
\pgfpathlineto{\pgfqpoint{2.635438in}{3.157816in}}%
\pgfpathlineto{\pgfqpoint{2.524541in}{3.221842in}}%
\pgfpathlineto{\pgfqpoint{2.531663in}{3.230943in}}%
\pgfpathclose%
\pgfusepath{fill}%
\end{pgfscope}%
\begin{pgfscope}%
\pgfpathrectangle{\pgfqpoint{0.765000in}{0.660000in}}{\pgfqpoint{4.620000in}{4.620000in}}%
\pgfusepath{clip}%
\pgfsetbuttcap%
\pgfsetroundjoin%
\definecolor{currentfill}{rgb}{1.000000,0.894118,0.788235}%
\pgfsetfillcolor{currentfill}%
\pgfsetlinewidth{0.000000pt}%
\definecolor{currentstroke}{rgb}{1.000000,0.894118,0.788235}%
\pgfsetstrokecolor{currentstroke}%
\pgfsetdash{}{0pt}%
\pgfpathmoveto{\pgfqpoint{2.642560in}{3.166917in}}%
\pgfpathlineto{\pgfqpoint{2.753457in}{3.230943in}}%
\pgfpathlineto{\pgfqpoint{2.746335in}{3.221842in}}%
\pgfpathlineto{\pgfqpoint{2.635438in}{3.157816in}}%
\pgfpathlineto{\pgfqpoint{2.642560in}{3.166917in}}%
\pgfpathclose%
\pgfusepath{fill}%
\end{pgfscope}%
\begin{pgfscope}%
\pgfpathrectangle{\pgfqpoint{0.765000in}{0.660000in}}{\pgfqpoint{4.620000in}{4.620000in}}%
\pgfusepath{clip}%
\pgfsetbuttcap%
\pgfsetroundjoin%
\definecolor{currentfill}{rgb}{1.000000,0.894118,0.788235}%
\pgfsetfillcolor{currentfill}%
\pgfsetlinewidth{0.000000pt}%
\definecolor{currentstroke}{rgb}{1.000000,0.894118,0.788235}%
\pgfsetstrokecolor{currentstroke}%
\pgfsetdash{}{0pt}%
\pgfpathmoveto{\pgfqpoint{2.642560in}{3.423022in}}%
\pgfpathlineto{\pgfqpoint{2.531663in}{3.358995in}}%
\pgfpathlineto{\pgfqpoint{2.524541in}{3.349894in}}%
\pgfpathlineto{\pgfqpoint{2.635438in}{3.413921in}}%
\pgfpathlineto{\pgfqpoint{2.642560in}{3.423022in}}%
\pgfpathclose%
\pgfusepath{fill}%
\end{pgfscope}%
\begin{pgfscope}%
\pgfpathrectangle{\pgfqpoint{0.765000in}{0.660000in}}{\pgfqpoint{4.620000in}{4.620000in}}%
\pgfusepath{clip}%
\pgfsetbuttcap%
\pgfsetroundjoin%
\definecolor{currentfill}{rgb}{1.000000,0.894118,0.788235}%
\pgfsetfillcolor{currentfill}%
\pgfsetlinewidth{0.000000pt}%
\definecolor{currentstroke}{rgb}{1.000000,0.894118,0.788235}%
\pgfsetstrokecolor{currentstroke}%
\pgfsetdash{}{0pt}%
\pgfpathmoveto{\pgfqpoint{2.753457in}{3.358995in}}%
\pgfpathlineto{\pgfqpoint{2.642560in}{3.423022in}}%
\pgfpathlineto{\pgfqpoint{2.635438in}{3.413921in}}%
\pgfpathlineto{\pgfqpoint{2.746335in}{3.349894in}}%
\pgfpathlineto{\pgfqpoint{2.753457in}{3.358995in}}%
\pgfpathclose%
\pgfusepath{fill}%
\end{pgfscope}%
\begin{pgfscope}%
\pgfpathrectangle{\pgfqpoint{0.765000in}{0.660000in}}{\pgfqpoint{4.620000in}{4.620000in}}%
\pgfusepath{clip}%
\pgfsetbuttcap%
\pgfsetroundjoin%
\definecolor{currentfill}{rgb}{1.000000,0.894118,0.788235}%
\pgfsetfillcolor{currentfill}%
\pgfsetlinewidth{0.000000pt}%
\definecolor{currentstroke}{rgb}{1.000000,0.894118,0.788235}%
\pgfsetstrokecolor{currentstroke}%
\pgfsetdash{}{0pt}%
\pgfpathmoveto{\pgfqpoint{2.531663in}{3.230943in}}%
\pgfpathlineto{\pgfqpoint{2.531663in}{3.358995in}}%
\pgfpathlineto{\pgfqpoint{2.524541in}{3.349894in}}%
\pgfpathlineto{\pgfqpoint{2.635438in}{3.157816in}}%
\pgfpathlineto{\pgfqpoint{2.531663in}{3.230943in}}%
\pgfpathclose%
\pgfusepath{fill}%
\end{pgfscope}%
\begin{pgfscope}%
\pgfpathrectangle{\pgfqpoint{0.765000in}{0.660000in}}{\pgfqpoint{4.620000in}{4.620000in}}%
\pgfusepath{clip}%
\pgfsetbuttcap%
\pgfsetroundjoin%
\definecolor{currentfill}{rgb}{1.000000,0.894118,0.788235}%
\pgfsetfillcolor{currentfill}%
\pgfsetlinewidth{0.000000pt}%
\definecolor{currentstroke}{rgb}{1.000000,0.894118,0.788235}%
\pgfsetstrokecolor{currentstroke}%
\pgfsetdash{}{0pt}%
\pgfpathmoveto{\pgfqpoint{2.642560in}{3.166917in}}%
\pgfpathlineto{\pgfqpoint{2.642560in}{3.294969in}}%
\pgfpathlineto{\pgfqpoint{2.635438in}{3.285868in}}%
\pgfpathlineto{\pgfqpoint{2.524541in}{3.221842in}}%
\pgfpathlineto{\pgfqpoint{2.642560in}{3.166917in}}%
\pgfpathclose%
\pgfusepath{fill}%
\end{pgfscope}%
\begin{pgfscope}%
\pgfpathrectangle{\pgfqpoint{0.765000in}{0.660000in}}{\pgfqpoint{4.620000in}{4.620000in}}%
\pgfusepath{clip}%
\pgfsetbuttcap%
\pgfsetroundjoin%
\definecolor{currentfill}{rgb}{1.000000,0.894118,0.788235}%
\pgfsetfillcolor{currentfill}%
\pgfsetlinewidth{0.000000pt}%
\definecolor{currentstroke}{rgb}{1.000000,0.894118,0.788235}%
\pgfsetstrokecolor{currentstroke}%
\pgfsetdash{}{0pt}%
\pgfpathmoveto{\pgfqpoint{2.642560in}{3.294969in}}%
\pgfpathlineto{\pgfqpoint{2.753457in}{3.358995in}}%
\pgfpathlineto{\pgfqpoint{2.746335in}{3.349894in}}%
\pgfpathlineto{\pgfqpoint{2.635438in}{3.285868in}}%
\pgfpathlineto{\pgfqpoint{2.642560in}{3.294969in}}%
\pgfpathclose%
\pgfusepath{fill}%
\end{pgfscope}%
\begin{pgfscope}%
\pgfpathrectangle{\pgfqpoint{0.765000in}{0.660000in}}{\pgfqpoint{4.620000in}{4.620000in}}%
\pgfusepath{clip}%
\pgfsetbuttcap%
\pgfsetroundjoin%
\definecolor{currentfill}{rgb}{1.000000,0.894118,0.788235}%
\pgfsetfillcolor{currentfill}%
\pgfsetlinewidth{0.000000pt}%
\definecolor{currentstroke}{rgb}{1.000000,0.894118,0.788235}%
\pgfsetstrokecolor{currentstroke}%
\pgfsetdash{}{0pt}%
\pgfpathmoveto{\pgfqpoint{2.531663in}{3.358995in}}%
\pgfpathlineto{\pgfqpoint{2.642560in}{3.294969in}}%
\pgfpathlineto{\pgfqpoint{2.635438in}{3.285868in}}%
\pgfpathlineto{\pgfqpoint{2.524541in}{3.349894in}}%
\pgfpathlineto{\pgfqpoint{2.531663in}{3.358995in}}%
\pgfpathclose%
\pgfusepath{fill}%
\end{pgfscope}%
\begin{pgfscope}%
\pgfpathrectangle{\pgfqpoint{0.765000in}{0.660000in}}{\pgfqpoint{4.620000in}{4.620000in}}%
\pgfusepath{clip}%
\pgfsetbuttcap%
\pgfsetroundjoin%
\definecolor{currentfill}{rgb}{1.000000,0.894118,0.788235}%
\pgfsetfillcolor{currentfill}%
\pgfsetlinewidth{0.000000pt}%
\definecolor{currentstroke}{rgb}{1.000000,0.894118,0.788235}%
\pgfsetstrokecolor{currentstroke}%
\pgfsetdash{}{0pt}%
\pgfpathmoveto{\pgfqpoint{2.642560in}{3.294969in}}%
\pgfpathlineto{\pgfqpoint{2.531663in}{3.230943in}}%
\pgfpathlineto{\pgfqpoint{2.642560in}{3.166917in}}%
\pgfpathlineto{\pgfqpoint{2.753457in}{3.230943in}}%
\pgfpathlineto{\pgfqpoint{2.642560in}{3.294969in}}%
\pgfpathclose%
\pgfusepath{fill}%
\end{pgfscope}%
\begin{pgfscope}%
\pgfpathrectangle{\pgfqpoint{0.765000in}{0.660000in}}{\pgfqpoint{4.620000in}{4.620000in}}%
\pgfusepath{clip}%
\pgfsetbuttcap%
\pgfsetroundjoin%
\definecolor{currentfill}{rgb}{1.000000,0.894118,0.788235}%
\pgfsetfillcolor{currentfill}%
\pgfsetlinewidth{0.000000pt}%
\definecolor{currentstroke}{rgb}{1.000000,0.894118,0.788235}%
\pgfsetstrokecolor{currentstroke}%
\pgfsetdash{}{0pt}%
\pgfpathmoveto{\pgfqpoint{2.642560in}{3.294969in}}%
\pgfpathlineto{\pgfqpoint{2.531663in}{3.230943in}}%
\pgfpathlineto{\pgfqpoint{2.531663in}{3.358995in}}%
\pgfpathlineto{\pgfqpoint{2.642560in}{3.423022in}}%
\pgfpathlineto{\pgfqpoint{2.642560in}{3.294969in}}%
\pgfpathclose%
\pgfusepath{fill}%
\end{pgfscope}%
\begin{pgfscope}%
\pgfpathrectangle{\pgfqpoint{0.765000in}{0.660000in}}{\pgfqpoint{4.620000in}{4.620000in}}%
\pgfusepath{clip}%
\pgfsetbuttcap%
\pgfsetroundjoin%
\definecolor{currentfill}{rgb}{1.000000,0.894118,0.788235}%
\pgfsetfillcolor{currentfill}%
\pgfsetlinewidth{0.000000pt}%
\definecolor{currentstroke}{rgb}{1.000000,0.894118,0.788235}%
\pgfsetstrokecolor{currentstroke}%
\pgfsetdash{}{0pt}%
\pgfpathmoveto{\pgfqpoint{2.642560in}{3.294969in}}%
\pgfpathlineto{\pgfqpoint{2.753457in}{3.230943in}}%
\pgfpathlineto{\pgfqpoint{2.753457in}{3.358995in}}%
\pgfpathlineto{\pgfqpoint{2.642560in}{3.423022in}}%
\pgfpathlineto{\pgfqpoint{2.642560in}{3.294969in}}%
\pgfpathclose%
\pgfusepath{fill}%
\end{pgfscope}%
\begin{pgfscope}%
\pgfpathrectangle{\pgfqpoint{0.765000in}{0.660000in}}{\pgfqpoint{4.620000in}{4.620000in}}%
\pgfusepath{clip}%
\pgfsetbuttcap%
\pgfsetroundjoin%
\definecolor{currentfill}{rgb}{1.000000,0.894118,0.788235}%
\pgfsetfillcolor{currentfill}%
\pgfsetlinewidth{0.000000pt}%
\definecolor{currentstroke}{rgb}{1.000000,0.894118,0.788235}%
\pgfsetstrokecolor{currentstroke}%
\pgfsetdash{}{0pt}%
\pgfpathmoveto{\pgfqpoint{2.651156in}{3.305927in}}%
\pgfpathlineto{\pgfqpoint{2.540259in}{3.241901in}}%
\pgfpathlineto{\pgfqpoint{2.651156in}{3.177874in}}%
\pgfpathlineto{\pgfqpoint{2.762052in}{3.241901in}}%
\pgfpathlineto{\pgfqpoint{2.651156in}{3.305927in}}%
\pgfpathclose%
\pgfusepath{fill}%
\end{pgfscope}%
\begin{pgfscope}%
\pgfpathrectangle{\pgfqpoint{0.765000in}{0.660000in}}{\pgfqpoint{4.620000in}{4.620000in}}%
\pgfusepath{clip}%
\pgfsetbuttcap%
\pgfsetroundjoin%
\definecolor{currentfill}{rgb}{1.000000,0.894118,0.788235}%
\pgfsetfillcolor{currentfill}%
\pgfsetlinewidth{0.000000pt}%
\definecolor{currentstroke}{rgb}{1.000000,0.894118,0.788235}%
\pgfsetstrokecolor{currentstroke}%
\pgfsetdash{}{0pt}%
\pgfpathmoveto{\pgfqpoint{2.651156in}{3.305927in}}%
\pgfpathlineto{\pgfqpoint{2.540259in}{3.241901in}}%
\pgfpathlineto{\pgfqpoint{2.540259in}{3.369953in}}%
\pgfpathlineto{\pgfqpoint{2.651156in}{3.433979in}}%
\pgfpathlineto{\pgfqpoint{2.651156in}{3.305927in}}%
\pgfpathclose%
\pgfusepath{fill}%
\end{pgfscope}%
\begin{pgfscope}%
\pgfpathrectangle{\pgfqpoint{0.765000in}{0.660000in}}{\pgfqpoint{4.620000in}{4.620000in}}%
\pgfusepath{clip}%
\pgfsetbuttcap%
\pgfsetroundjoin%
\definecolor{currentfill}{rgb}{1.000000,0.894118,0.788235}%
\pgfsetfillcolor{currentfill}%
\pgfsetlinewidth{0.000000pt}%
\definecolor{currentstroke}{rgb}{1.000000,0.894118,0.788235}%
\pgfsetstrokecolor{currentstroke}%
\pgfsetdash{}{0pt}%
\pgfpathmoveto{\pgfqpoint{2.651156in}{3.305927in}}%
\pgfpathlineto{\pgfqpoint{2.762052in}{3.241901in}}%
\pgfpathlineto{\pgfqpoint{2.762052in}{3.369953in}}%
\pgfpathlineto{\pgfqpoint{2.651156in}{3.433979in}}%
\pgfpathlineto{\pgfqpoint{2.651156in}{3.305927in}}%
\pgfpathclose%
\pgfusepath{fill}%
\end{pgfscope}%
\begin{pgfscope}%
\pgfpathrectangle{\pgfqpoint{0.765000in}{0.660000in}}{\pgfqpoint{4.620000in}{4.620000in}}%
\pgfusepath{clip}%
\pgfsetbuttcap%
\pgfsetroundjoin%
\definecolor{currentfill}{rgb}{1.000000,0.894118,0.788235}%
\pgfsetfillcolor{currentfill}%
\pgfsetlinewidth{0.000000pt}%
\definecolor{currentstroke}{rgb}{1.000000,0.894118,0.788235}%
\pgfsetstrokecolor{currentstroke}%
\pgfsetdash{}{0pt}%
\pgfpathmoveto{\pgfqpoint{2.642560in}{3.423022in}}%
\pgfpathlineto{\pgfqpoint{2.531663in}{3.358995in}}%
\pgfpathlineto{\pgfqpoint{2.642560in}{3.294969in}}%
\pgfpathlineto{\pgfqpoint{2.753457in}{3.358995in}}%
\pgfpathlineto{\pgfqpoint{2.642560in}{3.423022in}}%
\pgfpathclose%
\pgfusepath{fill}%
\end{pgfscope}%
\begin{pgfscope}%
\pgfpathrectangle{\pgfqpoint{0.765000in}{0.660000in}}{\pgfqpoint{4.620000in}{4.620000in}}%
\pgfusepath{clip}%
\pgfsetbuttcap%
\pgfsetroundjoin%
\definecolor{currentfill}{rgb}{1.000000,0.894118,0.788235}%
\pgfsetfillcolor{currentfill}%
\pgfsetlinewidth{0.000000pt}%
\definecolor{currentstroke}{rgb}{1.000000,0.894118,0.788235}%
\pgfsetstrokecolor{currentstroke}%
\pgfsetdash{}{0pt}%
\pgfpathmoveto{\pgfqpoint{2.642560in}{3.166917in}}%
\pgfpathlineto{\pgfqpoint{2.753457in}{3.230943in}}%
\pgfpathlineto{\pgfqpoint{2.753457in}{3.358995in}}%
\pgfpathlineto{\pgfqpoint{2.642560in}{3.294969in}}%
\pgfpathlineto{\pgfqpoint{2.642560in}{3.166917in}}%
\pgfpathclose%
\pgfusepath{fill}%
\end{pgfscope}%
\begin{pgfscope}%
\pgfpathrectangle{\pgfqpoint{0.765000in}{0.660000in}}{\pgfqpoint{4.620000in}{4.620000in}}%
\pgfusepath{clip}%
\pgfsetbuttcap%
\pgfsetroundjoin%
\definecolor{currentfill}{rgb}{1.000000,0.894118,0.788235}%
\pgfsetfillcolor{currentfill}%
\pgfsetlinewidth{0.000000pt}%
\definecolor{currentstroke}{rgb}{1.000000,0.894118,0.788235}%
\pgfsetstrokecolor{currentstroke}%
\pgfsetdash{}{0pt}%
\pgfpathmoveto{\pgfqpoint{2.531663in}{3.230943in}}%
\pgfpathlineto{\pgfqpoint{2.642560in}{3.166917in}}%
\pgfpathlineto{\pgfqpoint{2.642560in}{3.294969in}}%
\pgfpathlineto{\pgfqpoint{2.531663in}{3.358995in}}%
\pgfpathlineto{\pgfqpoint{2.531663in}{3.230943in}}%
\pgfpathclose%
\pgfusepath{fill}%
\end{pgfscope}%
\begin{pgfscope}%
\pgfpathrectangle{\pgfqpoint{0.765000in}{0.660000in}}{\pgfqpoint{4.620000in}{4.620000in}}%
\pgfusepath{clip}%
\pgfsetbuttcap%
\pgfsetroundjoin%
\definecolor{currentfill}{rgb}{1.000000,0.894118,0.788235}%
\pgfsetfillcolor{currentfill}%
\pgfsetlinewidth{0.000000pt}%
\definecolor{currentstroke}{rgb}{1.000000,0.894118,0.788235}%
\pgfsetstrokecolor{currentstroke}%
\pgfsetdash{}{0pt}%
\pgfpathmoveto{\pgfqpoint{2.651156in}{3.433979in}}%
\pgfpathlineto{\pgfqpoint{2.540259in}{3.369953in}}%
\pgfpathlineto{\pgfqpoint{2.651156in}{3.305927in}}%
\pgfpathlineto{\pgfqpoint{2.762052in}{3.369953in}}%
\pgfpathlineto{\pgfqpoint{2.651156in}{3.433979in}}%
\pgfpathclose%
\pgfusepath{fill}%
\end{pgfscope}%
\begin{pgfscope}%
\pgfpathrectangle{\pgfqpoint{0.765000in}{0.660000in}}{\pgfqpoint{4.620000in}{4.620000in}}%
\pgfusepath{clip}%
\pgfsetbuttcap%
\pgfsetroundjoin%
\definecolor{currentfill}{rgb}{1.000000,0.894118,0.788235}%
\pgfsetfillcolor{currentfill}%
\pgfsetlinewidth{0.000000pt}%
\definecolor{currentstroke}{rgb}{1.000000,0.894118,0.788235}%
\pgfsetstrokecolor{currentstroke}%
\pgfsetdash{}{0pt}%
\pgfpathmoveto{\pgfqpoint{2.651156in}{3.177874in}}%
\pgfpathlineto{\pgfqpoint{2.762052in}{3.241901in}}%
\pgfpathlineto{\pgfqpoint{2.762052in}{3.369953in}}%
\pgfpathlineto{\pgfqpoint{2.651156in}{3.305927in}}%
\pgfpathlineto{\pgfqpoint{2.651156in}{3.177874in}}%
\pgfpathclose%
\pgfusepath{fill}%
\end{pgfscope}%
\begin{pgfscope}%
\pgfpathrectangle{\pgfqpoint{0.765000in}{0.660000in}}{\pgfqpoint{4.620000in}{4.620000in}}%
\pgfusepath{clip}%
\pgfsetbuttcap%
\pgfsetroundjoin%
\definecolor{currentfill}{rgb}{1.000000,0.894118,0.788235}%
\pgfsetfillcolor{currentfill}%
\pgfsetlinewidth{0.000000pt}%
\definecolor{currentstroke}{rgb}{1.000000,0.894118,0.788235}%
\pgfsetstrokecolor{currentstroke}%
\pgfsetdash{}{0pt}%
\pgfpathmoveto{\pgfqpoint{2.540259in}{3.241901in}}%
\pgfpathlineto{\pgfqpoint{2.651156in}{3.177874in}}%
\pgfpathlineto{\pgfqpoint{2.651156in}{3.305927in}}%
\pgfpathlineto{\pgfqpoint{2.540259in}{3.369953in}}%
\pgfpathlineto{\pgfqpoint{2.540259in}{3.241901in}}%
\pgfpathclose%
\pgfusepath{fill}%
\end{pgfscope}%
\begin{pgfscope}%
\pgfpathrectangle{\pgfqpoint{0.765000in}{0.660000in}}{\pgfqpoint{4.620000in}{4.620000in}}%
\pgfusepath{clip}%
\pgfsetbuttcap%
\pgfsetroundjoin%
\definecolor{currentfill}{rgb}{1.000000,0.894118,0.788235}%
\pgfsetfillcolor{currentfill}%
\pgfsetlinewidth{0.000000pt}%
\definecolor{currentstroke}{rgb}{1.000000,0.894118,0.788235}%
\pgfsetstrokecolor{currentstroke}%
\pgfsetdash{}{0pt}%
\pgfpathmoveto{\pgfqpoint{2.642560in}{3.294969in}}%
\pgfpathlineto{\pgfqpoint{2.531663in}{3.230943in}}%
\pgfpathlineto{\pgfqpoint{2.540259in}{3.241901in}}%
\pgfpathlineto{\pgfqpoint{2.651156in}{3.305927in}}%
\pgfpathlineto{\pgfqpoint{2.642560in}{3.294969in}}%
\pgfpathclose%
\pgfusepath{fill}%
\end{pgfscope}%
\begin{pgfscope}%
\pgfpathrectangle{\pgfqpoint{0.765000in}{0.660000in}}{\pgfqpoint{4.620000in}{4.620000in}}%
\pgfusepath{clip}%
\pgfsetbuttcap%
\pgfsetroundjoin%
\definecolor{currentfill}{rgb}{1.000000,0.894118,0.788235}%
\pgfsetfillcolor{currentfill}%
\pgfsetlinewidth{0.000000pt}%
\definecolor{currentstroke}{rgb}{1.000000,0.894118,0.788235}%
\pgfsetstrokecolor{currentstroke}%
\pgfsetdash{}{0pt}%
\pgfpathmoveto{\pgfqpoint{2.753457in}{3.230943in}}%
\pgfpathlineto{\pgfqpoint{2.642560in}{3.294969in}}%
\pgfpathlineto{\pgfqpoint{2.651156in}{3.305927in}}%
\pgfpathlineto{\pgfqpoint{2.762052in}{3.241901in}}%
\pgfpathlineto{\pgfqpoint{2.753457in}{3.230943in}}%
\pgfpathclose%
\pgfusepath{fill}%
\end{pgfscope}%
\begin{pgfscope}%
\pgfpathrectangle{\pgfqpoint{0.765000in}{0.660000in}}{\pgfqpoint{4.620000in}{4.620000in}}%
\pgfusepath{clip}%
\pgfsetbuttcap%
\pgfsetroundjoin%
\definecolor{currentfill}{rgb}{1.000000,0.894118,0.788235}%
\pgfsetfillcolor{currentfill}%
\pgfsetlinewidth{0.000000pt}%
\definecolor{currentstroke}{rgb}{1.000000,0.894118,0.788235}%
\pgfsetstrokecolor{currentstroke}%
\pgfsetdash{}{0pt}%
\pgfpathmoveto{\pgfqpoint{2.642560in}{3.294969in}}%
\pgfpathlineto{\pgfqpoint{2.642560in}{3.423022in}}%
\pgfpathlineto{\pgfqpoint{2.651156in}{3.433979in}}%
\pgfpathlineto{\pgfqpoint{2.762052in}{3.241901in}}%
\pgfpathlineto{\pgfqpoint{2.642560in}{3.294969in}}%
\pgfpathclose%
\pgfusepath{fill}%
\end{pgfscope}%
\begin{pgfscope}%
\pgfpathrectangle{\pgfqpoint{0.765000in}{0.660000in}}{\pgfqpoint{4.620000in}{4.620000in}}%
\pgfusepath{clip}%
\pgfsetbuttcap%
\pgfsetroundjoin%
\definecolor{currentfill}{rgb}{1.000000,0.894118,0.788235}%
\pgfsetfillcolor{currentfill}%
\pgfsetlinewidth{0.000000pt}%
\definecolor{currentstroke}{rgb}{1.000000,0.894118,0.788235}%
\pgfsetstrokecolor{currentstroke}%
\pgfsetdash{}{0pt}%
\pgfpathmoveto{\pgfqpoint{2.753457in}{3.230943in}}%
\pgfpathlineto{\pgfqpoint{2.753457in}{3.358995in}}%
\pgfpathlineto{\pgfqpoint{2.762052in}{3.369953in}}%
\pgfpathlineto{\pgfqpoint{2.651156in}{3.305927in}}%
\pgfpathlineto{\pgfqpoint{2.753457in}{3.230943in}}%
\pgfpathclose%
\pgfusepath{fill}%
\end{pgfscope}%
\begin{pgfscope}%
\pgfpathrectangle{\pgfqpoint{0.765000in}{0.660000in}}{\pgfqpoint{4.620000in}{4.620000in}}%
\pgfusepath{clip}%
\pgfsetbuttcap%
\pgfsetroundjoin%
\definecolor{currentfill}{rgb}{1.000000,0.894118,0.788235}%
\pgfsetfillcolor{currentfill}%
\pgfsetlinewidth{0.000000pt}%
\definecolor{currentstroke}{rgb}{1.000000,0.894118,0.788235}%
\pgfsetstrokecolor{currentstroke}%
\pgfsetdash{}{0pt}%
\pgfpathmoveto{\pgfqpoint{2.531663in}{3.230943in}}%
\pgfpathlineto{\pgfqpoint{2.642560in}{3.166917in}}%
\pgfpathlineto{\pgfqpoint{2.651156in}{3.177874in}}%
\pgfpathlineto{\pgfqpoint{2.540259in}{3.241901in}}%
\pgfpathlineto{\pgfqpoint{2.531663in}{3.230943in}}%
\pgfpathclose%
\pgfusepath{fill}%
\end{pgfscope}%
\begin{pgfscope}%
\pgfpathrectangle{\pgfqpoint{0.765000in}{0.660000in}}{\pgfqpoint{4.620000in}{4.620000in}}%
\pgfusepath{clip}%
\pgfsetbuttcap%
\pgfsetroundjoin%
\definecolor{currentfill}{rgb}{1.000000,0.894118,0.788235}%
\pgfsetfillcolor{currentfill}%
\pgfsetlinewidth{0.000000pt}%
\definecolor{currentstroke}{rgb}{1.000000,0.894118,0.788235}%
\pgfsetstrokecolor{currentstroke}%
\pgfsetdash{}{0pt}%
\pgfpathmoveto{\pgfqpoint{2.642560in}{3.166917in}}%
\pgfpathlineto{\pgfqpoint{2.753457in}{3.230943in}}%
\pgfpathlineto{\pgfqpoint{2.762052in}{3.241901in}}%
\pgfpathlineto{\pgfqpoint{2.651156in}{3.177874in}}%
\pgfpathlineto{\pgfqpoint{2.642560in}{3.166917in}}%
\pgfpathclose%
\pgfusepath{fill}%
\end{pgfscope}%
\begin{pgfscope}%
\pgfpathrectangle{\pgfqpoint{0.765000in}{0.660000in}}{\pgfqpoint{4.620000in}{4.620000in}}%
\pgfusepath{clip}%
\pgfsetbuttcap%
\pgfsetroundjoin%
\definecolor{currentfill}{rgb}{1.000000,0.894118,0.788235}%
\pgfsetfillcolor{currentfill}%
\pgfsetlinewidth{0.000000pt}%
\definecolor{currentstroke}{rgb}{1.000000,0.894118,0.788235}%
\pgfsetstrokecolor{currentstroke}%
\pgfsetdash{}{0pt}%
\pgfpathmoveto{\pgfqpoint{2.642560in}{3.423022in}}%
\pgfpathlineto{\pgfqpoint{2.531663in}{3.358995in}}%
\pgfpathlineto{\pgfqpoint{2.540259in}{3.369953in}}%
\pgfpathlineto{\pgfqpoint{2.651156in}{3.433979in}}%
\pgfpathlineto{\pgfqpoint{2.642560in}{3.423022in}}%
\pgfpathclose%
\pgfusepath{fill}%
\end{pgfscope}%
\begin{pgfscope}%
\pgfpathrectangle{\pgfqpoint{0.765000in}{0.660000in}}{\pgfqpoint{4.620000in}{4.620000in}}%
\pgfusepath{clip}%
\pgfsetbuttcap%
\pgfsetroundjoin%
\definecolor{currentfill}{rgb}{1.000000,0.894118,0.788235}%
\pgfsetfillcolor{currentfill}%
\pgfsetlinewidth{0.000000pt}%
\definecolor{currentstroke}{rgb}{1.000000,0.894118,0.788235}%
\pgfsetstrokecolor{currentstroke}%
\pgfsetdash{}{0pt}%
\pgfpathmoveto{\pgfqpoint{2.753457in}{3.358995in}}%
\pgfpathlineto{\pgfqpoint{2.642560in}{3.423022in}}%
\pgfpathlineto{\pgfqpoint{2.651156in}{3.433979in}}%
\pgfpathlineto{\pgfqpoint{2.762052in}{3.369953in}}%
\pgfpathlineto{\pgfqpoint{2.753457in}{3.358995in}}%
\pgfpathclose%
\pgfusepath{fill}%
\end{pgfscope}%
\begin{pgfscope}%
\pgfpathrectangle{\pgfqpoint{0.765000in}{0.660000in}}{\pgfqpoint{4.620000in}{4.620000in}}%
\pgfusepath{clip}%
\pgfsetbuttcap%
\pgfsetroundjoin%
\definecolor{currentfill}{rgb}{1.000000,0.894118,0.788235}%
\pgfsetfillcolor{currentfill}%
\pgfsetlinewidth{0.000000pt}%
\definecolor{currentstroke}{rgb}{1.000000,0.894118,0.788235}%
\pgfsetstrokecolor{currentstroke}%
\pgfsetdash{}{0pt}%
\pgfpathmoveto{\pgfqpoint{2.531663in}{3.230943in}}%
\pgfpathlineto{\pgfqpoint{2.531663in}{3.358995in}}%
\pgfpathlineto{\pgfqpoint{2.540259in}{3.369953in}}%
\pgfpathlineto{\pgfqpoint{2.651156in}{3.177874in}}%
\pgfpathlineto{\pgfqpoint{2.531663in}{3.230943in}}%
\pgfpathclose%
\pgfusepath{fill}%
\end{pgfscope}%
\begin{pgfscope}%
\pgfpathrectangle{\pgfqpoint{0.765000in}{0.660000in}}{\pgfqpoint{4.620000in}{4.620000in}}%
\pgfusepath{clip}%
\pgfsetbuttcap%
\pgfsetroundjoin%
\definecolor{currentfill}{rgb}{1.000000,0.894118,0.788235}%
\pgfsetfillcolor{currentfill}%
\pgfsetlinewidth{0.000000pt}%
\definecolor{currentstroke}{rgb}{1.000000,0.894118,0.788235}%
\pgfsetstrokecolor{currentstroke}%
\pgfsetdash{}{0pt}%
\pgfpathmoveto{\pgfqpoint{2.642560in}{3.166917in}}%
\pgfpathlineto{\pgfqpoint{2.642560in}{3.294969in}}%
\pgfpathlineto{\pgfqpoint{2.651156in}{3.305927in}}%
\pgfpathlineto{\pgfqpoint{2.540259in}{3.241901in}}%
\pgfpathlineto{\pgfqpoint{2.642560in}{3.166917in}}%
\pgfpathclose%
\pgfusepath{fill}%
\end{pgfscope}%
\begin{pgfscope}%
\pgfpathrectangle{\pgfqpoint{0.765000in}{0.660000in}}{\pgfqpoint{4.620000in}{4.620000in}}%
\pgfusepath{clip}%
\pgfsetbuttcap%
\pgfsetroundjoin%
\definecolor{currentfill}{rgb}{1.000000,0.894118,0.788235}%
\pgfsetfillcolor{currentfill}%
\pgfsetlinewidth{0.000000pt}%
\definecolor{currentstroke}{rgb}{1.000000,0.894118,0.788235}%
\pgfsetstrokecolor{currentstroke}%
\pgfsetdash{}{0pt}%
\pgfpathmoveto{\pgfqpoint{2.642560in}{3.294969in}}%
\pgfpathlineto{\pgfqpoint{2.753457in}{3.358995in}}%
\pgfpathlineto{\pgfqpoint{2.762052in}{3.369953in}}%
\pgfpathlineto{\pgfqpoint{2.651156in}{3.305927in}}%
\pgfpathlineto{\pgfqpoint{2.642560in}{3.294969in}}%
\pgfpathclose%
\pgfusepath{fill}%
\end{pgfscope}%
\begin{pgfscope}%
\pgfpathrectangle{\pgfqpoint{0.765000in}{0.660000in}}{\pgfqpoint{4.620000in}{4.620000in}}%
\pgfusepath{clip}%
\pgfsetbuttcap%
\pgfsetroundjoin%
\definecolor{currentfill}{rgb}{1.000000,0.894118,0.788235}%
\pgfsetfillcolor{currentfill}%
\pgfsetlinewidth{0.000000pt}%
\definecolor{currentstroke}{rgb}{1.000000,0.894118,0.788235}%
\pgfsetstrokecolor{currentstroke}%
\pgfsetdash{}{0pt}%
\pgfpathmoveto{\pgfqpoint{2.531663in}{3.358995in}}%
\pgfpathlineto{\pgfqpoint{2.642560in}{3.294969in}}%
\pgfpathlineto{\pgfqpoint{2.651156in}{3.305927in}}%
\pgfpathlineto{\pgfqpoint{2.540259in}{3.369953in}}%
\pgfpathlineto{\pgfqpoint{2.531663in}{3.358995in}}%
\pgfpathclose%
\pgfusepath{fill}%
\end{pgfscope}%
\begin{pgfscope}%
\pgfpathrectangle{\pgfqpoint{0.765000in}{0.660000in}}{\pgfqpoint{4.620000in}{4.620000in}}%
\pgfusepath{clip}%
\pgfsetbuttcap%
\pgfsetroundjoin%
\definecolor{currentfill}{rgb}{1.000000,0.894118,0.788235}%
\pgfsetfillcolor{currentfill}%
\pgfsetlinewidth{0.000000pt}%
\definecolor{currentstroke}{rgb}{1.000000,0.894118,0.788235}%
\pgfsetstrokecolor{currentstroke}%
\pgfsetdash{}{0pt}%
\pgfpathmoveto{\pgfqpoint{2.913379in}{3.479681in}}%
\pgfpathlineto{\pgfqpoint{2.802482in}{3.415655in}}%
\pgfpathlineto{\pgfqpoint{2.913379in}{3.351628in}}%
\pgfpathlineto{\pgfqpoint{3.024275in}{3.415655in}}%
\pgfpathlineto{\pgfqpoint{2.913379in}{3.479681in}}%
\pgfpathclose%
\pgfusepath{fill}%
\end{pgfscope}%
\begin{pgfscope}%
\pgfpathrectangle{\pgfqpoint{0.765000in}{0.660000in}}{\pgfqpoint{4.620000in}{4.620000in}}%
\pgfusepath{clip}%
\pgfsetbuttcap%
\pgfsetroundjoin%
\definecolor{currentfill}{rgb}{1.000000,0.894118,0.788235}%
\pgfsetfillcolor{currentfill}%
\pgfsetlinewidth{0.000000pt}%
\definecolor{currentstroke}{rgb}{1.000000,0.894118,0.788235}%
\pgfsetstrokecolor{currentstroke}%
\pgfsetdash{}{0pt}%
\pgfpathmoveto{\pgfqpoint{2.913379in}{3.479681in}}%
\pgfpathlineto{\pgfqpoint{2.802482in}{3.415655in}}%
\pgfpathlineto{\pgfqpoint{2.802482in}{3.543707in}}%
\pgfpathlineto{\pgfqpoint{2.913379in}{3.607733in}}%
\pgfpathlineto{\pgfqpoint{2.913379in}{3.479681in}}%
\pgfpathclose%
\pgfusepath{fill}%
\end{pgfscope}%
\begin{pgfscope}%
\pgfpathrectangle{\pgfqpoint{0.765000in}{0.660000in}}{\pgfqpoint{4.620000in}{4.620000in}}%
\pgfusepath{clip}%
\pgfsetbuttcap%
\pgfsetroundjoin%
\definecolor{currentfill}{rgb}{1.000000,0.894118,0.788235}%
\pgfsetfillcolor{currentfill}%
\pgfsetlinewidth{0.000000pt}%
\definecolor{currentstroke}{rgb}{1.000000,0.894118,0.788235}%
\pgfsetstrokecolor{currentstroke}%
\pgfsetdash{}{0pt}%
\pgfpathmoveto{\pgfqpoint{2.913379in}{3.479681in}}%
\pgfpathlineto{\pgfqpoint{3.024275in}{3.415655in}}%
\pgfpathlineto{\pgfqpoint{3.024275in}{3.543707in}}%
\pgfpathlineto{\pgfqpoint{2.913379in}{3.607733in}}%
\pgfpathlineto{\pgfqpoint{2.913379in}{3.479681in}}%
\pgfpathclose%
\pgfusepath{fill}%
\end{pgfscope}%
\begin{pgfscope}%
\pgfpathrectangle{\pgfqpoint{0.765000in}{0.660000in}}{\pgfqpoint{4.620000in}{4.620000in}}%
\pgfusepath{clip}%
\pgfsetbuttcap%
\pgfsetroundjoin%
\definecolor{currentfill}{rgb}{1.000000,0.894118,0.788235}%
\pgfsetfillcolor{currentfill}%
\pgfsetlinewidth{0.000000pt}%
\definecolor{currentstroke}{rgb}{1.000000,0.894118,0.788235}%
\pgfsetstrokecolor{currentstroke}%
\pgfsetdash{}{0pt}%
\pgfpathmoveto{\pgfqpoint{2.793000in}{3.409870in}}%
\pgfpathlineto{\pgfqpoint{2.682103in}{3.345844in}}%
\pgfpathlineto{\pgfqpoint{2.793000in}{3.281817in}}%
\pgfpathlineto{\pgfqpoint{2.903897in}{3.345844in}}%
\pgfpathlineto{\pgfqpoint{2.793000in}{3.409870in}}%
\pgfpathclose%
\pgfusepath{fill}%
\end{pgfscope}%
\begin{pgfscope}%
\pgfpathrectangle{\pgfqpoint{0.765000in}{0.660000in}}{\pgfqpoint{4.620000in}{4.620000in}}%
\pgfusepath{clip}%
\pgfsetbuttcap%
\pgfsetroundjoin%
\definecolor{currentfill}{rgb}{1.000000,0.894118,0.788235}%
\pgfsetfillcolor{currentfill}%
\pgfsetlinewidth{0.000000pt}%
\definecolor{currentstroke}{rgb}{1.000000,0.894118,0.788235}%
\pgfsetstrokecolor{currentstroke}%
\pgfsetdash{}{0pt}%
\pgfpathmoveto{\pgfqpoint{2.793000in}{3.409870in}}%
\pgfpathlineto{\pgfqpoint{2.682103in}{3.345844in}}%
\pgfpathlineto{\pgfqpoint{2.682103in}{3.473896in}}%
\pgfpathlineto{\pgfqpoint{2.793000in}{3.537922in}}%
\pgfpathlineto{\pgfqpoint{2.793000in}{3.409870in}}%
\pgfpathclose%
\pgfusepath{fill}%
\end{pgfscope}%
\begin{pgfscope}%
\pgfpathrectangle{\pgfqpoint{0.765000in}{0.660000in}}{\pgfqpoint{4.620000in}{4.620000in}}%
\pgfusepath{clip}%
\pgfsetbuttcap%
\pgfsetroundjoin%
\definecolor{currentfill}{rgb}{1.000000,0.894118,0.788235}%
\pgfsetfillcolor{currentfill}%
\pgfsetlinewidth{0.000000pt}%
\definecolor{currentstroke}{rgb}{1.000000,0.894118,0.788235}%
\pgfsetstrokecolor{currentstroke}%
\pgfsetdash{}{0pt}%
\pgfpathmoveto{\pgfqpoint{2.793000in}{3.409870in}}%
\pgfpathlineto{\pgfqpoint{2.903897in}{3.345844in}}%
\pgfpathlineto{\pgfqpoint{2.903897in}{3.473896in}}%
\pgfpathlineto{\pgfqpoint{2.793000in}{3.537922in}}%
\pgfpathlineto{\pgfqpoint{2.793000in}{3.409870in}}%
\pgfpathclose%
\pgfusepath{fill}%
\end{pgfscope}%
\begin{pgfscope}%
\pgfpathrectangle{\pgfqpoint{0.765000in}{0.660000in}}{\pgfqpoint{4.620000in}{4.620000in}}%
\pgfusepath{clip}%
\pgfsetbuttcap%
\pgfsetroundjoin%
\definecolor{currentfill}{rgb}{1.000000,0.894118,0.788235}%
\pgfsetfillcolor{currentfill}%
\pgfsetlinewidth{0.000000pt}%
\definecolor{currentstroke}{rgb}{1.000000,0.894118,0.788235}%
\pgfsetstrokecolor{currentstroke}%
\pgfsetdash{}{0pt}%
\pgfpathmoveto{\pgfqpoint{2.913379in}{3.607733in}}%
\pgfpathlineto{\pgfqpoint{2.802482in}{3.543707in}}%
\pgfpathlineto{\pgfqpoint{2.913379in}{3.479681in}}%
\pgfpathlineto{\pgfqpoint{3.024275in}{3.543707in}}%
\pgfpathlineto{\pgfqpoint{2.913379in}{3.607733in}}%
\pgfpathclose%
\pgfusepath{fill}%
\end{pgfscope}%
\begin{pgfscope}%
\pgfpathrectangle{\pgfqpoint{0.765000in}{0.660000in}}{\pgfqpoint{4.620000in}{4.620000in}}%
\pgfusepath{clip}%
\pgfsetbuttcap%
\pgfsetroundjoin%
\definecolor{currentfill}{rgb}{1.000000,0.894118,0.788235}%
\pgfsetfillcolor{currentfill}%
\pgfsetlinewidth{0.000000pt}%
\definecolor{currentstroke}{rgb}{1.000000,0.894118,0.788235}%
\pgfsetstrokecolor{currentstroke}%
\pgfsetdash{}{0pt}%
\pgfpathmoveto{\pgfqpoint{2.913379in}{3.351628in}}%
\pgfpathlineto{\pgfqpoint{3.024275in}{3.415655in}}%
\pgfpathlineto{\pgfqpoint{3.024275in}{3.543707in}}%
\pgfpathlineto{\pgfqpoint{2.913379in}{3.479681in}}%
\pgfpathlineto{\pgfqpoint{2.913379in}{3.351628in}}%
\pgfpathclose%
\pgfusepath{fill}%
\end{pgfscope}%
\begin{pgfscope}%
\pgfpathrectangle{\pgfqpoint{0.765000in}{0.660000in}}{\pgfqpoint{4.620000in}{4.620000in}}%
\pgfusepath{clip}%
\pgfsetbuttcap%
\pgfsetroundjoin%
\definecolor{currentfill}{rgb}{1.000000,0.894118,0.788235}%
\pgfsetfillcolor{currentfill}%
\pgfsetlinewidth{0.000000pt}%
\definecolor{currentstroke}{rgb}{1.000000,0.894118,0.788235}%
\pgfsetstrokecolor{currentstroke}%
\pgfsetdash{}{0pt}%
\pgfpathmoveto{\pgfqpoint{2.802482in}{3.415655in}}%
\pgfpathlineto{\pgfqpoint{2.913379in}{3.351628in}}%
\pgfpathlineto{\pgfqpoint{2.913379in}{3.479681in}}%
\pgfpathlineto{\pgfqpoint{2.802482in}{3.543707in}}%
\pgfpathlineto{\pgfqpoint{2.802482in}{3.415655in}}%
\pgfpathclose%
\pgfusepath{fill}%
\end{pgfscope}%
\begin{pgfscope}%
\pgfpathrectangle{\pgfqpoint{0.765000in}{0.660000in}}{\pgfqpoint{4.620000in}{4.620000in}}%
\pgfusepath{clip}%
\pgfsetbuttcap%
\pgfsetroundjoin%
\definecolor{currentfill}{rgb}{1.000000,0.894118,0.788235}%
\pgfsetfillcolor{currentfill}%
\pgfsetlinewidth{0.000000pt}%
\definecolor{currentstroke}{rgb}{1.000000,0.894118,0.788235}%
\pgfsetstrokecolor{currentstroke}%
\pgfsetdash{}{0pt}%
\pgfpathmoveto{\pgfqpoint{2.793000in}{3.537922in}}%
\pgfpathlineto{\pgfqpoint{2.682103in}{3.473896in}}%
\pgfpathlineto{\pgfqpoint{2.793000in}{3.409870in}}%
\pgfpathlineto{\pgfqpoint{2.903897in}{3.473896in}}%
\pgfpathlineto{\pgfqpoint{2.793000in}{3.537922in}}%
\pgfpathclose%
\pgfusepath{fill}%
\end{pgfscope}%
\begin{pgfscope}%
\pgfpathrectangle{\pgfqpoint{0.765000in}{0.660000in}}{\pgfqpoint{4.620000in}{4.620000in}}%
\pgfusepath{clip}%
\pgfsetbuttcap%
\pgfsetroundjoin%
\definecolor{currentfill}{rgb}{1.000000,0.894118,0.788235}%
\pgfsetfillcolor{currentfill}%
\pgfsetlinewidth{0.000000pt}%
\definecolor{currentstroke}{rgb}{1.000000,0.894118,0.788235}%
\pgfsetstrokecolor{currentstroke}%
\pgfsetdash{}{0pt}%
\pgfpathmoveto{\pgfqpoint{2.793000in}{3.281817in}}%
\pgfpathlineto{\pgfqpoint{2.903897in}{3.345844in}}%
\pgfpathlineto{\pgfqpoint{2.903897in}{3.473896in}}%
\pgfpathlineto{\pgfqpoint{2.793000in}{3.409870in}}%
\pgfpathlineto{\pgfqpoint{2.793000in}{3.281817in}}%
\pgfpathclose%
\pgfusepath{fill}%
\end{pgfscope}%
\begin{pgfscope}%
\pgfpathrectangle{\pgfqpoint{0.765000in}{0.660000in}}{\pgfqpoint{4.620000in}{4.620000in}}%
\pgfusepath{clip}%
\pgfsetbuttcap%
\pgfsetroundjoin%
\definecolor{currentfill}{rgb}{1.000000,0.894118,0.788235}%
\pgfsetfillcolor{currentfill}%
\pgfsetlinewidth{0.000000pt}%
\definecolor{currentstroke}{rgb}{1.000000,0.894118,0.788235}%
\pgfsetstrokecolor{currentstroke}%
\pgfsetdash{}{0pt}%
\pgfpathmoveto{\pgfqpoint{2.682103in}{3.345844in}}%
\pgfpathlineto{\pgfqpoint{2.793000in}{3.281817in}}%
\pgfpathlineto{\pgfqpoint{2.793000in}{3.409870in}}%
\pgfpathlineto{\pgfqpoint{2.682103in}{3.473896in}}%
\pgfpathlineto{\pgfqpoint{2.682103in}{3.345844in}}%
\pgfpathclose%
\pgfusepath{fill}%
\end{pgfscope}%
\begin{pgfscope}%
\pgfpathrectangle{\pgfqpoint{0.765000in}{0.660000in}}{\pgfqpoint{4.620000in}{4.620000in}}%
\pgfusepath{clip}%
\pgfsetbuttcap%
\pgfsetroundjoin%
\definecolor{currentfill}{rgb}{1.000000,0.894118,0.788235}%
\pgfsetfillcolor{currentfill}%
\pgfsetlinewidth{0.000000pt}%
\definecolor{currentstroke}{rgb}{1.000000,0.894118,0.788235}%
\pgfsetstrokecolor{currentstroke}%
\pgfsetdash{}{0pt}%
\pgfpathmoveto{\pgfqpoint{2.913379in}{3.479681in}}%
\pgfpathlineto{\pgfqpoint{2.802482in}{3.415655in}}%
\pgfpathlineto{\pgfqpoint{2.682103in}{3.345844in}}%
\pgfpathlineto{\pgfqpoint{2.793000in}{3.409870in}}%
\pgfpathlineto{\pgfqpoint{2.913379in}{3.479681in}}%
\pgfpathclose%
\pgfusepath{fill}%
\end{pgfscope}%
\begin{pgfscope}%
\pgfpathrectangle{\pgfqpoint{0.765000in}{0.660000in}}{\pgfqpoint{4.620000in}{4.620000in}}%
\pgfusepath{clip}%
\pgfsetbuttcap%
\pgfsetroundjoin%
\definecolor{currentfill}{rgb}{1.000000,0.894118,0.788235}%
\pgfsetfillcolor{currentfill}%
\pgfsetlinewidth{0.000000pt}%
\definecolor{currentstroke}{rgb}{1.000000,0.894118,0.788235}%
\pgfsetstrokecolor{currentstroke}%
\pgfsetdash{}{0pt}%
\pgfpathmoveto{\pgfqpoint{3.024275in}{3.415655in}}%
\pgfpathlineto{\pgfqpoint{2.913379in}{3.479681in}}%
\pgfpathlineto{\pgfqpoint{2.793000in}{3.409870in}}%
\pgfpathlineto{\pgfqpoint{2.903897in}{3.345844in}}%
\pgfpathlineto{\pgfqpoint{3.024275in}{3.415655in}}%
\pgfpathclose%
\pgfusepath{fill}%
\end{pgfscope}%
\begin{pgfscope}%
\pgfpathrectangle{\pgfqpoint{0.765000in}{0.660000in}}{\pgfqpoint{4.620000in}{4.620000in}}%
\pgfusepath{clip}%
\pgfsetbuttcap%
\pgfsetroundjoin%
\definecolor{currentfill}{rgb}{1.000000,0.894118,0.788235}%
\pgfsetfillcolor{currentfill}%
\pgfsetlinewidth{0.000000pt}%
\definecolor{currentstroke}{rgb}{1.000000,0.894118,0.788235}%
\pgfsetstrokecolor{currentstroke}%
\pgfsetdash{}{0pt}%
\pgfpathmoveto{\pgfqpoint{2.913379in}{3.479681in}}%
\pgfpathlineto{\pgfqpoint{2.913379in}{3.607733in}}%
\pgfpathlineto{\pgfqpoint{2.793000in}{3.537922in}}%
\pgfpathlineto{\pgfqpoint{2.903897in}{3.345844in}}%
\pgfpathlineto{\pgfqpoint{2.913379in}{3.479681in}}%
\pgfpathclose%
\pgfusepath{fill}%
\end{pgfscope}%
\begin{pgfscope}%
\pgfpathrectangle{\pgfqpoint{0.765000in}{0.660000in}}{\pgfqpoint{4.620000in}{4.620000in}}%
\pgfusepath{clip}%
\pgfsetbuttcap%
\pgfsetroundjoin%
\definecolor{currentfill}{rgb}{1.000000,0.894118,0.788235}%
\pgfsetfillcolor{currentfill}%
\pgfsetlinewidth{0.000000pt}%
\definecolor{currentstroke}{rgb}{1.000000,0.894118,0.788235}%
\pgfsetstrokecolor{currentstroke}%
\pgfsetdash{}{0pt}%
\pgfpathmoveto{\pgfqpoint{3.024275in}{3.415655in}}%
\pgfpathlineto{\pgfqpoint{3.024275in}{3.543707in}}%
\pgfpathlineto{\pgfqpoint{2.903897in}{3.473896in}}%
\pgfpathlineto{\pgfqpoint{2.793000in}{3.409870in}}%
\pgfpathlineto{\pgfqpoint{3.024275in}{3.415655in}}%
\pgfpathclose%
\pgfusepath{fill}%
\end{pgfscope}%
\begin{pgfscope}%
\pgfpathrectangle{\pgfqpoint{0.765000in}{0.660000in}}{\pgfqpoint{4.620000in}{4.620000in}}%
\pgfusepath{clip}%
\pgfsetbuttcap%
\pgfsetroundjoin%
\definecolor{currentfill}{rgb}{1.000000,0.894118,0.788235}%
\pgfsetfillcolor{currentfill}%
\pgfsetlinewidth{0.000000pt}%
\definecolor{currentstroke}{rgb}{1.000000,0.894118,0.788235}%
\pgfsetstrokecolor{currentstroke}%
\pgfsetdash{}{0pt}%
\pgfpathmoveto{\pgfqpoint{2.802482in}{3.415655in}}%
\pgfpathlineto{\pgfqpoint{2.913379in}{3.351628in}}%
\pgfpathlineto{\pgfqpoint{2.793000in}{3.281817in}}%
\pgfpathlineto{\pgfqpoint{2.682103in}{3.345844in}}%
\pgfpathlineto{\pgfqpoint{2.802482in}{3.415655in}}%
\pgfpathclose%
\pgfusepath{fill}%
\end{pgfscope}%
\begin{pgfscope}%
\pgfpathrectangle{\pgfqpoint{0.765000in}{0.660000in}}{\pgfqpoint{4.620000in}{4.620000in}}%
\pgfusepath{clip}%
\pgfsetbuttcap%
\pgfsetroundjoin%
\definecolor{currentfill}{rgb}{1.000000,0.894118,0.788235}%
\pgfsetfillcolor{currentfill}%
\pgfsetlinewidth{0.000000pt}%
\definecolor{currentstroke}{rgb}{1.000000,0.894118,0.788235}%
\pgfsetstrokecolor{currentstroke}%
\pgfsetdash{}{0pt}%
\pgfpathmoveto{\pgfqpoint{2.913379in}{3.351628in}}%
\pgfpathlineto{\pgfqpoint{3.024275in}{3.415655in}}%
\pgfpathlineto{\pgfqpoint{2.903897in}{3.345844in}}%
\pgfpathlineto{\pgfqpoint{2.793000in}{3.281817in}}%
\pgfpathlineto{\pgfqpoint{2.913379in}{3.351628in}}%
\pgfpathclose%
\pgfusepath{fill}%
\end{pgfscope}%
\begin{pgfscope}%
\pgfpathrectangle{\pgfqpoint{0.765000in}{0.660000in}}{\pgfqpoint{4.620000in}{4.620000in}}%
\pgfusepath{clip}%
\pgfsetbuttcap%
\pgfsetroundjoin%
\definecolor{currentfill}{rgb}{1.000000,0.894118,0.788235}%
\pgfsetfillcolor{currentfill}%
\pgfsetlinewidth{0.000000pt}%
\definecolor{currentstroke}{rgb}{1.000000,0.894118,0.788235}%
\pgfsetstrokecolor{currentstroke}%
\pgfsetdash{}{0pt}%
\pgfpathmoveto{\pgfqpoint{2.913379in}{3.607733in}}%
\pgfpathlineto{\pgfqpoint{2.802482in}{3.543707in}}%
\pgfpathlineto{\pgfqpoint{2.682103in}{3.473896in}}%
\pgfpathlineto{\pgfqpoint{2.793000in}{3.537922in}}%
\pgfpathlineto{\pgfqpoint{2.913379in}{3.607733in}}%
\pgfpathclose%
\pgfusepath{fill}%
\end{pgfscope}%
\begin{pgfscope}%
\pgfpathrectangle{\pgfqpoint{0.765000in}{0.660000in}}{\pgfqpoint{4.620000in}{4.620000in}}%
\pgfusepath{clip}%
\pgfsetbuttcap%
\pgfsetroundjoin%
\definecolor{currentfill}{rgb}{1.000000,0.894118,0.788235}%
\pgfsetfillcolor{currentfill}%
\pgfsetlinewidth{0.000000pt}%
\definecolor{currentstroke}{rgb}{1.000000,0.894118,0.788235}%
\pgfsetstrokecolor{currentstroke}%
\pgfsetdash{}{0pt}%
\pgfpathmoveto{\pgfqpoint{3.024275in}{3.543707in}}%
\pgfpathlineto{\pgfqpoint{2.913379in}{3.607733in}}%
\pgfpathlineto{\pgfqpoint{2.793000in}{3.537922in}}%
\pgfpathlineto{\pgfqpoint{2.903897in}{3.473896in}}%
\pgfpathlineto{\pgfqpoint{3.024275in}{3.543707in}}%
\pgfpathclose%
\pgfusepath{fill}%
\end{pgfscope}%
\begin{pgfscope}%
\pgfpathrectangle{\pgfqpoint{0.765000in}{0.660000in}}{\pgfqpoint{4.620000in}{4.620000in}}%
\pgfusepath{clip}%
\pgfsetbuttcap%
\pgfsetroundjoin%
\definecolor{currentfill}{rgb}{1.000000,0.894118,0.788235}%
\pgfsetfillcolor{currentfill}%
\pgfsetlinewidth{0.000000pt}%
\definecolor{currentstroke}{rgb}{1.000000,0.894118,0.788235}%
\pgfsetstrokecolor{currentstroke}%
\pgfsetdash{}{0pt}%
\pgfpathmoveto{\pgfqpoint{2.802482in}{3.415655in}}%
\pgfpathlineto{\pgfqpoint{2.802482in}{3.543707in}}%
\pgfpathlineto{\pgfqpoint{2.682103in}{3.473896in}}%
\pgfpathlineto{\pgfqpoint{2.793000in}{3.281817in}}%
\pgfpathlineto{\pgfqpoint{2.802482in}{3.415655in}}%
\pgfpathclose%
\pgfusepath{fill}%
\end{pgfscope}%
\begin{pgfscope}%
\pgfpathrectangle{\pgfqpoint{0.765000in}{0.660000in}}{\pgfqpoint{4.620000in}{4.620000in}}%
\pgfusepath{clip}%
\pgfsetbuttcap%
\pgfsetroundjoin%
\definecolor{currentfill}{rgb}{1.000000,0.894118,0.788235}%
\pgfsetfillcolor{currentfill}%
\pgfsetlinewidth{0.000000pt}%
\definecolor{currentstroke}{rgb}{1.000000,0.894118,0.788235}%
\pgfsetstrokecolor{currentstroke}%
\pgfsetdash{}{0pt}%
\pgfpathmoveto{\pgfqpoint{2.913379in}{3.351628in}}%
\pgfpathlineto{\pgfqpoint{2.913379in}{3.479681in}}%
\pgfpathlineto{\pgfqpoint{2.793000in}{3.409870in}}%
\pgfpathlineto{\pgfqpoint{2.682103in}{3.345844in}}%
\pgfpathlineto{\pgfqpoint{2.913379in}{3.351628in}}%
\pgfpathclose%
\pgfusepath{fill}%
\end{pgfscope}%
\begin{pgfscope}%
\pgfpathrectangle{\pgfqpoint{0.765000in}{0.660000in}}{\pgfqpoint{4.620000in}{4.620000in}}%
\pgfusepath{clip}%
\pgfsetbuttcap%
\pgfsetroundjoin%
\definecolor{currentfill}{rgb}{1.000000,0.894118,0.788235}%
\pgfsetfillcolor{currentfill}%
\pgfsetlinewidth{0.000000pt}%
\definecolor{currentstroke}{rgb}{1.000000,0.894118,0.788235}%
\pgfsetstrokecolor{currentstroke}%
\pgfsetdash{}{0pt}%
\pgfpathmoveto{\pgfqpoint{2.913379in}{3.479681in}}%
\pgfpathlineto{\pgfqpoint{3.024275in}{3.543707in}}%
\pgfpathlineto{\pgfqpoint{2.903897in}{3.473896in}}%
\pgfpathlineto{\pgfqpoint{2.793000in}{3.409870in}}%
\pgfpathlineto{\pgfqpoint{2.913379in}{3.479681in}}%
\pgfpathclose%
\pgfusepath{fill}%
\end{pgfscope}%
\begin{pgfscope}%
\pgfpathrectangle{\pgfqpoint{0.765000in}{0.660000in}}{\pgfqpoint{4.620000in}{4.620000in}}%
\pgfusepath{clip}%
\pgfsetbuttcap%
\pgfsetroundjoin%
\definecolor{currentfill}{rgb}{1.000000,0.894118,0.788235}%
\pgfsetfillcolor{currentfill}%
\pgfsetlinewidth{0.000000pt}%
\definecolor{currentstroke}{rgb}{1.000000,0.894118,0.788235}%
\pgfsetstrokecolor{currentstroke}%
\pgfsetdash{}{0pt}%
\pgfpathmoveto{\pgfqpoint{2.802482in}{3.543707in}}%
\pgfpathlineto{\pgfqpoint{2.913379in}{3.479681in}}%
\pgfpathlineto{\pgfqpoint{2.793000in}{3.409870in}}%
\pgfpathlineto{\pgfqpoint{2.682103in}{3.473896in}}%
\pgfpathlineto{\pgfqpoint{2.802482in}{3.543707in}}%
\pgfpathclose%
\pgfusepath{fill}%
\end{pgfscope}%
\begin{pgfscope}%
\pgfpathrectangle{\pgfqpoint{0.765000in}{0.660000in}}{\pgfqpoint{4.620000in}{4.620000in}}%
\pgfusepath{clip}%
\pgfsetbuttcap%
\pgfsetroundjoin%
\definecolor{currentfill}{rgb}{1.000000,0.894118,0.788235}%
\pgfsetfillcolor{currentfill}%
\pgfsetlinewidth{0.000000pt}%
\definecolor{currentstroke}{rgb}{1.000000,0.894118,0.788235}%
\pgfsetstrokecolor{currentstroke}%
\pgfsetdash{}{0pt}%
\pgfpathmoveto{\pgfqpoint{2.913379in}{3.479681in}}%
\pgfpathlineto{\pgfqpoint{2.802482in}{3.415655in}}%
\pgfpathlineto{\pgfqpoint{2.913379in}{3.351628in}}%
\pgfpathlineto{\pgfqpoint{3.024275in}{3.415655in}}%
\pgfpathlineto{\pgfqpoint{2.913379in}{3.479681in}}%
\pgfpathclose%
\pgfusepath{fill}%
\end{pgfscope}%
\begin{pgfscope}%
\pgfpathrectangle{\pgfqpoint{0.765000in}{0.660000in}}{\pgfqpoint{4.620000in}{4.620000in}}%
\pgfusepath{clip}%
\pgfsetbuttcap%
\pgfsetroundjoin%
\definecolor{currentfill}{rgb}{1.000000,0.894118,0.788235}%
\pgfsetfillcolor{currentfill}%
\pgfsetlinewidth{0.000000pt}%
\definecolor{currentstroke}{rgb}{1.000000,0.894118,0.788235}%
\pgfsetstrokecolor{currentstroke}%
\pgfsetdash{}{0pt}%
\pgfpathmoveto{\pgfqpoint{2.913379in}{3.479681in}}%
\pgfpathlineto{\pgfqpoint{2.802482in}{3.415655in}}%
\pgfpathlineto{\pgfqpoint{2.802482in}{3.543707in}}%
\pgfpathlineto{\pgfqpoint{2.913379in}{3.607733in}}%
\pgfpathlineto{\pgfqpoint{2.913379in}{3.479681in}}%
\pgfpathclose%
\pgfusepath{fill}%
\end{pgfscope}%
\begin{pgfscope}%
\pgfpathrectangle{\pgfqpoint{0.765000in}{0.660000in}}{\pgfqpoint{4.620000in}{4.620000in}}%
\pgfusepath{clip}%
\pgfsetbuttcap%
\pgfsetroundjoin%
\definecolor{currentfill}{rgb}{1.000000,0.894118,0.788235}%
\pgfsetfillcolor{currentfill}%
\pgfsetlinewidth{0.000000pt}%
\definecolor{currentstroke}{rgb}{1.000000,0.894118,0.788235}%
\pgfsetstrokecolor{currentstroke}%
\pgfsetdash{}{0pt}%
\pgfpathmoveto{\pgfqpoint{2.913379in}{3.479681in}}%
\pgfpathlineto{\pgfqpoint{3.024275in}{3.415655in}}%
\pgfpathlineto{\pgfqpoint{3.024275in}{3.543707in}}%
\pgfpathlineto{\pgfqpoint{2.913379in}{3.607733in}}%
\pgfpathlineto{\pgfqpoint{2.913379in}{3.479681in}}%
\pgfpathclose%
\pgfusepath{fill}%
\end{pgfscope}%
\begin{pgfscope}%
\pgfpathrectangle{\pgfqpoint{0.765000in}{0.660000in}}{\pgfqpoint{4.620000in}{4.620000in}}%
\pgfusepath{clip}%
\pgfsetbuttcap%
\pgfsetroundjoin%
\definecolor{currentfill}{rgb}{1.000000,0.894118,0.788235}%
\pgfsetfillcolor{currentfill}%
\pgfsetlinewidth{0.000000pt}%
\definecolor{currentstroke}{rgb}{1.000000,0.894118,0.788235}%
\pgfsetstrokecolor{currentstroke}%
\pgfsetdash{}{0pt}%
\pgfpathmoveto{\pgfqpoint{2.982382in}{3.516877in}}%
\pgfpathlineto{\pgfqpoint{2.871485in}{3.452851in}}%
\pgfpathlineto{\pgfqpoint{2.982382in}{3.388824in}}%
\pgfpathlineto{\pgfqpoint{3.093279in}{3.452851in}}%
\pgfpathlineto{\pgfqpoint{2.982382in}{3.516877in}}%
\pgfpathclose%
\pgfusepath{fill}%
\end{pgfscope}%
\begin{pgfscope}%
\pgfpathrectangle{\pgfqpoint{0.765000in}{0.660000in}}{\pgfqpoint{4.620000in}{4.620000in}}%
\pgfusepath{clip}%
\pgfsetbuttcap%
\pgfsetroundjoin%
\definecolor{currentfill}{rgb}{1.000000,0.894118,0.788235}%
\pgfsetfillcolor{currentfill}%
\pgfsetlinewidth{0.000000pt}%
\definecolor{currentstroke}{rgb}{1.000000,0.894118,0.788235}%
\pgfsetstrokecolor{currentstroke}%
\pgfsetdash{}{0pt}%
\pgfpathmoveto{\pgfqpoint{2.982382in}{3.516877in}}%
\pgfpathlineto{\pgfqpoint{2.871485in}{3.452851in}}%
\pgfpathlineto{\pgfqpoint{2.871485in}{3.580903in}}%
\pgfpathlineto{\pgfqpoint{2.982382in}{3.644929in}}%
\pgfpathlineto{\pgfqpoint{2.982382in}{3.516877in}}%
\pgfpathclose%
\pgfusepath{fill}%
\end{pgfscope}%
\begin{pgfscope}%
\pgfpathrectangle{\pgfqpoint{0.765000in}{0.660000in}}{\pgfqpoint{4.620000in}{4.620000in}}%
\pgfusepath{clip}%
\pgfsetbuttcap%
\pgfsetroundjoin%
\definecolor{currentfill}{rgb}{1.000000,0.894118,0.788235}%
\pgfsetfillcolor{currentfill}%
\pgfsetlinewidth{0.000000pt}%
\definecolor{currentstroke}{rgb}{1.000000,0.894118,0.788235}%
\pgfsetstrokecolor{currentstroke}%
\pgfsetdash{}{0pt}%
\pgfpathmoveto{\pgfqpoint{2.982382in}{3.516877in}}%
\pgfpathlineto{\pgfqpoint{3.093279in}{3.452851in}}%
\pgfpathlineto{\pgfqpoint{3.093279in}{3.580903in}}%
\pgfpathlineto{\pgfqpoint{2.982382in}{3.644929in}}%
\pgfpathlineto{\pgfqpoint{2.982382in}{3.516877in}}%
\pgfpathclose%
\pgfusepath{fill}%
\end{pgfscope}%
\begin{pgfscope}%
\pgfpathrectangle{\pgfqpoint{0.765000in}{0.660000in}}{\pgfqpoint{4.620000in}{4.620000in}}%
\pgfusepath{clip}%
\pgfsetbuttcap%
\pgfsetroundjoin%
\definecolor{currentfill}{rgb}{1.000000,0.894118,0.788235}%
\pgfsetfillcolor{currentfill}%
\pgfsetlinewidth{0.000000pt}%
\definecolor{currentstroke}{rgb}{1.000000,0.894118,0.788235}%
\pgfsetstrokecolor{currentstroke}%
\pgfsetdash{}{0pt}%
\pgfpathmoveto{\pgfqpoint{2.913379in}{3.607733in}}%
\pgfpathlineto{\pgfqpoint{2.802482in}{3.543707in}}%
\pgfpathlineto{\pgfqpoint{2.913379in}{3.479681in}}%
\pgfpathlineto{\pgfqpoint{3.024275in}{3.543707in}}%
\pgfpathlineto{\pgfqpoint{2.913379in}{3.607733in}}%
\pgfpathclose%
\pgfusepath{fill}%
\end{pgfscope}%
\begin{pgfscope}%
\pgfpathrectangle{\pgfqpoint{0.765000in}{0.660000in}}{\pgfqpoint{4.620000in}{4.620000in}}%
\pgfusepath{clip}%
\pgfsetbuttcap%
\pgfsetroundjoin%
\definecolor{currentfill}{rgb}{1.000000,0.894118,0.788235}%
\pgfsetfillcolor{currentfill}%
\pgfsetlinewidth{0.000000pt}%
\definecolor{currentstroke}{rgb}{1.000000,0.894118,0.788235}%
\pgfsetstrokecolor{currentstroke}%
\pgfsetdash{}{0pt}%
\pgfpathmoveto{\pgfqpoint{2.913379in}{3.351628in}}%
\pgfpathlineto{\pgfqpoint{3.024275in}{3.415655in}}%
\pgfpathlineto{\pgfqpoint{3.024275in}{3.543707in}}%
\pgfpathlineto{\pgfqpoint{2.913379in}{3.479681in}}%
\pgfpathlineto{\pgfqpoint{2.913379in}{3.351628in}}%
\pgfpathclose%
\pgfusepath{fill}%
\end{pgfscope}%
\begin{pgfscope}%
\pgfpathrectangle{\pgfqpoint{0.765000in}{0.660000in}}{\pgfqpoint{4.620000in}{4.620000in}}%
\pgfusepath{clip}%
\pgfsetbuttcap%
\pgfsetroundjoin%
\definecolor{currentfill}{rgb}{1.000000,0.894118,0.788235}%
\pgfsetfillcolor{currentfill}%
\pgfsetlinewidth{0.000000pt}%
\definecolor{currentstroke}{rgb}{1.000000,0.894118,0.788235}%
\pgfsetstrokecolor{currentstroke}%
\pgfsetdash{}{0pt}%
\pgfpathmoveto{\pgfqpoint{2.802482in}{3.415655in}}%
\pgfpathlineto{\pgfqpoint{2.913379in}{3.351628in}}%
\pgfpathlineto{\pgfqpoint{2.913379in}{3.479681in}}%
\pgfpathlineto{\pgfqpoint{2.802482in}{3.543707in}}%
\pgfpathlineto{\pgfqpoint{2.802482in}{3.415655in}}%
\pgfpathclose%
\pgfusepath{fill}%
\end{pgfscope}%
\begin{pgfscope}%
\pgfpathrectangle{\pgfqpoint{0.765000in}{0.660000in}}{\pgfqpoint{4.620000in}{4.620000in}}%
\pgfusepath{clip}%
\pgfsetbuttcap%
\pgfsetroundjoin%
\definecolor{currentfill}{rgb}{1.000000,0.894118,0.788235}%
\pgfsetfillcolor{currentfill}%
\pgfsetlinewidth{0.000000pt}%
\definecolor{currentstroke}{rgb}{1.000000,0.894118,0.788235}%
\pgfsetstrokecolor{currentstroke}%
\pgfsetdash{}{0pt}%
\pgfpathmoveto{\pgfqpoint{2.982382in}{3.644929in}}%
\pgfpathlineto{\pgfqpoint{2.871485in}{3.580903in}}%
\pgfpathlineto{\pgfqpoint{2.982382in}{3.516877in}}%
\pgfpathlineto{\pgfqpoint{3.093279in}{3.580903in}}%
\pgfpathlineto{\pgfqpoint{2.982382in}{3.644929in}}%
\pgfpathclose%
\pgfusepath{fill}%
\end{pgfscope}%
\begin{pgfscope}%
\pgfpathrectangle{\pgfqpoint{0.765000in}{0.660000in}}{\pgfqpoint{4.620000in}{4.620000in}}%
\pgfusepath{clip}%
\pgfsetbuttcap%
\pgfsetroundjoin%
\definecolor{currentfill}{rgb}{1.000000,0.894118,0.788235}%
\pgfsetfillcolor{currentfill}%
\pgfsetlinewidth{0.000000pt}%
\definecolor{currentstroke}{rgb}{1.000000,0.894118,0.788235}%
\pgfsetstrokecolor{currentstroke}%
\pgfsetdash{}{0pt}%
\pgfpathmoveto{\pgfqpoint{2.982382in}{3.388824in}}%
\pgfpathlineto{\pgfqpoint{3.093279in}{3.452851in}}%
\pgfpathlineto{\pgfqpoint{3.093279in}{3.580903in}}%
\pgfpathlineto{\pgfqpoint{2.982382in}{3.516877in}}%
\pgfpathlineto{\pgfqpoint{2.982382in}{3.388824in}}%
\pgfpathclose%
\pgfusepath{fill}%
\end{pgfscope}%
\begin{pgfscope}%
\pgfpathrectangle{\pgfqpoint{0.765000in}{0.660000in}}{\pgfqpoint{4.620000in}{4.620000in}}%
\pgfusepath{clip}%
\pgfsetbuttcap%
\pgfsetroundjoin%
\definecolor{currentfill}{rgb}{1.000000,0.894118,0.788235}%
\pgfsetfillcolor{currentfill}%
\pgfsetlinewidth{0.000000pt}%
\definecolor{currentstroke}{rgb}{1.000000,0.894118,0.788235}%
\pgfsetstrokecolor{currentstroke}%
\pgfsetdash{}{0pt}%
\pgfpathmoveto{\pgfqpoint{2.871485in}{3.452851in}}%
\pgfpathlineto{\pgfqpoint{2.982382in}{3.388824in}}%
\pgfpathlineto{\pgfqpoint{2.982382in}{3.516877in}}%
\pgfpathlineto{\pgfqpoint{2.871485in}{3.580903in}}%
\pgfpathlineto{\pgfqpoint{2.871485in}{3.452851in}}%
\pgfpathclose%
\pgfusepath{fill}%
\end{pgfscope}%
\begin{pgfscope}%
\pgfpathrectangle{\pgfqpoint{0.765000in}{0.660000in}}{\pgfqpoint{4.620000in}{4.620000in}}%
\pgfusepath{clip}%
\pgfsetbuttcap%
\pgfsetroundjoin%
\definecolor{currentfill}{rgb}{1.000000,0.894118,0.788235}%
\pgfsetfillcolor{currentfill}%
\pgfsetlinewidth{0.000000pt}%
\definecolor{currentstroke}{rgb}{1.000000,0.894118,0.788235}%
\pgfsetstrokecolor{currentstroke}%
\pgfsetdash{}{0pt}%
\pgfpathmoveto{\pgfqpoint{2.913379in}{3.479681in}}%
\pgfpathlineto{\pgfqpoint{2.802482in}{3.415655in}}%
\pgfpathlineto{\pgfqpoint{2.871485in}{3.452851in}}%
\pgfpathlineto{\pgfqpoint{2.982382in}{3.516877in}}%
\pgfpathlineto{\pgfqpoint{2.913379in}{3.479681in}}%
\pgfpathclose%
\pgfusepath{fill}%
\end{pgfscope}%
\begin{pgfscope}%
\pgfpathrectangle{\pgfqpoint{0.765000in}{0.660000in}}{\pgfqpoint{4.620000in}{4.620000in}}%
\pgfusepath{clip}%
\pgfsetbuttcap%
\pgfsetroundjoin%
\definecolor{currentfill}{rgb}{1.000000,0.894118,0.788235}%
\pgfsetfillcolor{currentfill}%
\pgfsetlinewidth{0.000000pt}%
\definecolor{currentstroke}{rgb}{1.000000,0.894118,0.788235}%
\pgfsetstrokecolor{currentstroke}%
\pgfsetdash{}{0pt}%
\pgfpathmoveto{\pgfqpoint{3.024275in}{3.415655in}}%
\pgfpathlineto{\pgfqpoint{2.913379in}{3.479681in}}%
\pgfpathlineto{\pgfqpoint{2.982382in}{3.516877in}}%
\pgfpathlineto{\pgfqpoint{3.093279in}{3.452851in}}%
\pgfpathlineto{\pgfqpoint{3.024275in}{3.415655in}}%
\pgfpathclose%
\pgfusepath{fill}%
\end{pgfscope}%
\begin{pgfscope}%
\pgfpathrectangle{\pgfqpoint{0.765000in}{0.660000in}}{\pgfqpoint{4.620000in}{4.620000in}}%
\pgfusepath{clip}%
\pgfsetbuttcap%
\pgfsetroundjoin%
\definecolor{currentfill}{rgb}{1.000000,0.894118,0.788235}%
\pgfsetfillcolor{currentfill}%
\pgfsetlinewidth{0.000000pt}%
\definecolor{currentstroke}{rgb}{1.000000,0.894118,0.788235}%
\pgfsetstrokecolor{currentstroke}%
\pgfsetdash{}{0pt}%
\pgfpathmoveto{\pgfqpoint{2.913379in}{3.479681in}}%
\pgfpathlineto{\pgfqpoint{2.913379in}{3.607733in}}%
\pgfpathlineto{\pgfqpoint{2.982382in}{3.644929in}}%
\pgfpathlineto{\pgfqpoint{3.093279in}{3.452851in}}%
\pgfpathlineto{\pgfqpoint{2.913379in}{3.479681in}}%
\pgfpathclose%
\pgfusepath{fill}%
\end{pgfscope}%
\begin{pgfscope}%
\pgfpathrectangle{\pgfqpoint{0.765000in}{0.660000in}}{\pgfqpoint{4.620000in}{4.620000in}}%
\pgfusepath{clip}%
\pgfsetbuttcap%
\pgfsetroundjoin%
\definecolor{currentfill}{rgb}{1.000000,0.894118,0.788235}%
\pgfsetfillcolor{currentfill}%
\pgfsetlinewidth{0.000000pt}%
\definecolor{currentstroke}{rgb}{1.000000,0.894118,0.788235}%
\pgfsetstrokecolor{currentstroke}%
\pgfsetdash{}{0pt}%
\pgfpathmoveto{\pgfqpoint{3.024275in}{3.415655in}}%
\pgfpathlineto{\pgfqpoint{3.024275in}{3.543707in}}%
\pgfpathlineto{\pgfqpoint{3.093279in}{3.580903in}}%
\pgfpathlineto{\pgfqpoint{2.982382in}{3.516877in}}%
\pgfpathlineto{\pgfqpoint{3.024275in}{3.415655in}}%
\pgfpathclose%
\pgfusepath{fill}%
\end{pgfscope}%
\begin{pgfscope}%
\pgfpathrectangle{\pgfqpoint{0.765000in}{0.660000in}}{\pgfqpoint{4.620000in}{4.620000in}}%
\pgfusepath{clip}%
\pgfsetbuttcap%
\pgfsetroundjoin%
\definecolor{currentfill}{rgb}{1.000000,0.894118,0.788235}%
\pgfsetfillcolor{currentfill}%
\pgfsetlinewidth{0.000000pt}%
\definecolor{currentstroke}{rgb}{1.000000,0.894118,0.788235}%
\pgfsetstrokecolor{currentstroke}%
\pgfsetdash{}{0pt}%
\pgfpathmoveto{\pgfqpoint{2.802482in}{3.415655in}}%
\pgfpathlineto{\pgfqpoint{2.913379in}{3.351628in}}%
\pgfpathlineto{\pgfqpoint{2.982382in}{3.388824in}}%
\pgfpathlineto{\pgfqpoint{2.871485in}{3.452851in}}%
\pgfpathlineto{\pgfqpoint{2.802482in}{3.415655in}}%
\pgfpathclose%
\pgfusepath{fill}%
\end{pgfscope}%
\begin{pgfscope}%
\pgfpathrectangle{\pgfqpoint{0.765000in}{0.660000in}}{\pgfqpoint{4.620000in}{4.620000in}}%
\pgfusepath{clip}%
\pgfsetbuttcap%
\pgfsetroundjoin%
\definecolor{currentfill}{rgb}{1.000000,0.894118,0.788235}%
\pgfsetfillcolor{currentfill}%
\pgfsetlinewidth{0.000000pt}%
\definecolor{currentstroke}{rgb}{1.000000,0.894118,0.788235}%
\pgfsetstrokecolor{currentstroke}%
\pgfsetdash{}{0pt}%
\pgfpathmoveto{\pgfqpoint{2.913379in}{3.351628in}}%
\pgfpathlineto{\pgfqpoint{3.024275in}{3.415655in}}%
\pgfpathlineto{\pgfqpoint{3.093279in}{3.452851in}}%
\pgfpathlineto{\pgfqpoint{2.982382in}{3.388824in}}%
\pgfpathlineto{\pgfqpoint{2.913379in}{3.351628in}}%
\pgfpathclose%
\pgfusepath{fill}%
\end{pgfscope}%
\begin{pgfscope}%
\pgfpathrectangle{\pgfqpoint{0.765000in}{0.660000in}}{\pgfqpoint{4.620000in}{4.620000in}}%
\pgfusepath{clip}%
\pgfsetbuttcap%
\pgfsetroundjoin%
\definecolor{currentfill}{rgb}{1.000000,0.894118,0.788235}%
\pgfsetfillcolor{currentfill}%
\pgfsetlinewidth{0.000000pt}%
\definecolor{currentstroke}{rgb}{1.000000,0.894118,0.788235}%
\pgfsetstrokecolor{currentstroke}%
\pgfsetdash{}{0pt}%
\pgfpathmoveto{\pgfqpoint{2.913379in}{3.607733in}}%
\pgfpathlineto{\pgfqpoint{2.802482in}{3.543707in}}%
\pgfpathlineto{\pgfqpoint{2.871485in}{3.580903in}}%
\pgfpathlineto{\pgfqpoint{2.982382in}{3.644929in}}%
\pgfpathlineto{\pgfqpoint{2.913379in}{3.607733in}}%
\pgfpathclose%
\pgfusepath{fill}%
\end{pgfscope}%
\begin{pgfscope}%
\pgfpathrectangle{\pgfqpoint{0.765000in}{0.660000in}}{\pgfqpoint{4.620000in}{4.620000in}}%
\pgfusepath{clip}%
\pgfsetbuttcap%
\pgfsetroundjoin%
\definecolor{currentfill}{rgb}{1.000000,0.894118,0.788235}%
\pgfsetfillcolor{currentfill}%
\pgfsetlinewidth{0.000000pt}%
\definecolor{currentstroke}{rgb}{1.000000,0.894118,0.788235}%
\pgfsetstrokecolor{currentstroke}%
\pgfsetdash{}{0pt}%
\pgfpathmoveto{\pgfqpoint{3.024275in}{3.543707in}}%
\pgfpathlineto{\pgfqpoint{2.913379in}{3.607733in}}%
\pgfpathlineto{\pgfqpoint{2.982382in}{3.644929in}}%
\pgfpathlineto{\pgfqpoint{3.093279in}{3.580903in}}%
\pgfpathlineto{\pgfqpoint{3.024275in}{3.543707in}}%
\pgfpathclose%
\pgfusepath{fill}%
\end{pgfscope}%
\begin{pgfscope}%
\pgfpathrectangle{\pgfqpoint{0.765000in}{0.660000in}}{\pgfqpoint{4.620000in}{4.620000in}}%
\pgfusepath{clip}%
\pgfsetbuttcap%
\pgfsetroundjoin%
\definecolor{currentfill}{rgb}{1.000000,0.894118,0.788235}%
\pgfsetfillcolor{currentfill}%
\pgfsetlinewidth{0.000000pt}%
\definecolor{currentstroke}{rgb}{1.000000,0.894118,0.788235}%
\pgfsetstrokecolor{currentstroke}%
\pgfsetdash{}{0pt}%
\pgfpathmoveto{\pgfqpoint{2.802482in}{3.415655in}}%
\pgfpathlineto{\pgfqpoint{2.802482in}{3.543707in}}%
\pgfpathlineto{\pgfqpoint{2.871485in}{3.580903in}}%
\pgfpathlineto{\pgfqpoint{2.982382in}{3.388824in}}%
\pgfpathlineto{\pgfqpoint{2.802482in}{3.415655in}}%
\pgfpathclose%
\pgfusepath{fill}%
\end{pgfscope}%
\begin{pgfscope}%
\pgfpathrectangle{\pgfqpoint{0.765000in}{0.660000in}}{\pgfqpoint{4.620000in}{4.620000in}}%
\pgfusepath{clip}%
\pgfsetbuttcap%
\pgfsetroundjoin%
\definecolor{currentfill}{rgb}{1.000000,0.894118,0.788235}%
\pgfsetfillcolor{currentfill}%
\pgfsetlinewidth{0.000000pt}%
\definecolor{currentstroke}{rgb}{1.000000,0.894118,0.788235}%
\pgfsetstrokecolor{currentstroke}%
\pgfsetdash{}{0pt}%
\pgfpathmoveto{\pgfqpoint{2.913379in}{3.351628in}}%
\pgfpathlineto{\pgfqpoint{2.913379in}{3.479681in}}%
\pgfpathlineto{\pgfqpoint{2.982382in}{3.516877in}}%
\pgfpathlineto{\pgfqpoint{2.871485in}{3.452851in}}%
\pgfpathlineto{\pgfqpoint{2.913379in}{3.351628in}}%
\pgfpathclose%
\pgfusepath{fill}%
\end{pgfscope}%
\begin{pgfscope}%
\pgfpathrectangle{\pgfqpoint{0.765000in}{0.660000in}}{\pgfqpoint{4.620000in}{4.620000in}}%
\pgfusepath{clip}%
\pgfsetbuttcap%
\pgfsetroundjoin%
\definecolor{currentfill}{rgb}{1.000000,0.894118,0.788235}%
\pgfsetfillcolor{currentfill}%
\pgfsetlinewidth{0.000000pt}%
\definecolor{currentstroke}{rgb}{1.000000,0.894118,0.788235}%
\pgfsetstrokecolor{currentstroke}%
\pgfsetdash{}{0pt}%
\pgfpathmoveto{\pgfqpoint{2.913379in}{3.479681in}}%
\pgfpathlineto{\pgfqpoint{3.024275in}{3.543707in}}%
\pgfpathlineto{\pgfqpoint{3.093279in}{3.580903in}}%
\pgfpathlineto{\pgfqpoint{2.982382in}{3.516877in}}%
\pgfpathlineto{\pgfqpoint{2.913379in}{3.479681in}}%
\pgfpathclose%
\pgfusepath{fill}%
\end{pgfscope}%
\begin{pgfscope}%
\pgfpathrectangle{\pgfqpoint{0.765000in}{0.660000in}}{\pgfqpoint{4.620000in}{4.620000in}}%
\pgfusepath{clip}%
\pgfsetbuttcap%
\pgfsetroundjoin%
\definecolor{currentfill}{rgb}{1.000000,0.894118,0.788235}%
\pgfsetfillcolor{currentfill}%
\pgfsetlinewidth{0.000000pt}%
\definecolor{currentstroke}{rgb}{1.000000,0.894118,0.788235}%
\pgfsetstrokecolor{currentstroke}%
\pgfsetdash{}{0pt}%
\pgfpathmoveto{\pgfqpoint{2.802482in}{3.543707in}}%
\pgfpathlineto{\pgfqpoint{2.913379in}{3.479681in}}%
\pgfpathlineto{\pgfqpoint{2.982382in}{3.516877in}}%
\pgfpathlineto{\pgfqpoint{2.871485in}{3.580903in}}%
\pgfpathlineto{\pgfqpoint{2.802482in}{3.543707in}}%
\pgfpathclose%
\pgfusepath{fill}%
\end{pgfscope}%
\begin{pgfscope}%
\pgfpathrectangle{\pgfqpoint{0.765000in}{0.660000in}}{\pgfqpoint{4.620000in}{4.620000in}}%
\pgfusepath{clip}%
\pgfsetbuttcap%
\pgfsetroundjoin%
\definecolor{currentfill}{rgb}{1.000000,0.894118,0.788235}%
\pgfsetfillcolor{currentfill}%
\pgfsetlinewidth{0.000000pt}%
\definecolor{currentstroke}{rgb}{1.000000,0.894118,0.788235}%
\pgfsetstrokecolor{currentstroke}%
\pgfsetdash{}{0pt}%
\pgfpathmoveto{\pgfqpoint{3.107502in}{3.529134in}}%
\pgfpathlineto{\pgfqpoint{2.996605in}{3.465108in}}%
\pgfpathlineto{\pgfqpoint{3.107502in}{3.401081in}}%
\pgfpathlineto{\pgfqpoint{3.218398in}{3.465108in}}%
\pgfpathlineto{\pgfqpoint{3.107502in}{3.529134in}}%
\pgfpathclose%
\pgfusepath{fill}%
\end{pgfscope}%
\begin{pgfscope}%
\pgfpathrectangle{\pgfqpoint{0.765000in}{0.660000in}}{\pgfqpoint{4.620000in}{4.620000in}}%
\pgfusepath{clip}%
\pgfsetbuttcap%
\pgfsetroundjoin%
\definecolor{currentfill}{rgb}{1.000000,0.894118,0.788235}%
\pgfsetfillcolor{currentfill}%
\pgfsetlinewidth{0.000000pt}%
\definecolor{currentstroke}{rgb}{1.000000,0.894118,0.788235}%
\pgfsetstrokecolor{currentstroke}%
\pgfsetdash{}{0pt}%
\pgfpathmoveto{\pgfqpoint{3.107502in}{3.529134in}}%
\pgfpathlineto{\pgfqpoint{2.996605in}{3.465108in}}%
\pgfpathlineto{\pgfqpoint{2.996605in}{3.593160in}}%
\pgfpathlineto{\pgfqpoint{3.107502in}{3.657186in}}%
\pgfpathlineto{\pgfqpoint{3.107502in}{3.529134in}}%
\pgfpathclose%
\pgfusepath{fill}%
\end{pgfscope}%
\begin{pgfscope}%
\pgfpathrectangle{\pgfqpoint{0.765000in}{0.660000in}}{\pgfqpoint{4.620000in}{4.620000in}}%
\pgfusepath{clip}%
\pgfsetbuttcap%
\pgfsetroundjoin%
\definecolor{currentfill}{rgb}{1.000000,0.894118,0.788235}%
\pgfsetfillcolor{currentfill}%
\pgfsetlinewidth{0.000000pt}%
\definecolor{currentstroke}{rgb}{1.000000,0.894118,0.788235}%
\pgfsetstrokecolor{currentstroke}%
\pgfsetdash{}{0pt}%
\pgfpathmoveto{\pgfqpoint{3.107502in}{3.529134in}}%
\pgfpathlineto{\pgfqpoint{3.218398in}{3.465108in}}%
\pgfpathlineto{\pgfqpoint{3.218398in}{3.593160in}}%
\pgfpathlineto{\pgfqpoint{3.107502in}{3.657186in}}%
\pgfpathlineto{\pgfqpoint{3.107502in}{3.529134in}}%
\pgfpathclose%
\pgfusepath{fill}%
\end{pgfscope}%
\begin{pgfscope}%
\pgfpathrectangle{\pgfqpoint{0.765000in}{0.660000in}}{\pgfqpoint{4.620000in}{4.620000in}}%
\pgfusepath{clip}%
\pgfsetbuttcap%
\pgfsetroundjoin%
\definecolor{currentfill}{rgb}{1.000000,0.894118,0.788235}%
\pgfsetfillcolor{currentfill}%
\pgfsetlinewidth{0.000000pt}%
\definecolor{currentstroke}{rgb}{1.000000,0.894118,0.788235}%
\pgfsetstrokecolor{currentstroke}%
\pgfsetdash{}{0pt}%
\pgfpathmoveto{\pgfqpoint{3.121227in}{3.527081in}}%
\pgfpathlineto{\pgfqpoint{3.010331in}{3.463055in}}%
\pgfpathlineto{\pgfqpoint{3.121227in}{3.399029in}}%
\pgfpathlineto{\pgfqpoint{3.232124in}{3.463055in}}%
\pgfpathlineto{\pgfqpoint{3.121227in}{3.527081in}}%
\pgfpathclose%
\pgfusepath{fill}%
\end{pgfscope}%
\begin{pgfscope}%
\pgfpathrectangle{\pgfqpoint{0.765000in}{0.660000in}}{\pgfqpoint{4.620000in}{4.620000in}}%
\pgfusepath{clip}%
\pgfsetbuttcap%
\pgfsetroundjoin%
\definecolor{currentfill}{rgb}{1.000000,0.894118,0.788235}%
\pgfsetfillcolor{currentfill}%
\pgfsetlinewidth{0.000000pt}%
\definecolor{currentstroke}{rgb}{1.000000,0.894118,0.788235}%
\pgfsetstrokecolor{currentstroke}%
\pgfsetdash{}{0pt}%
\pgfpathmoveto{\pgfqpoint{3.121227in}{3.527081in}}%
\pgfpathlineto{\pgfqpoint{3.010331in}{3.463055in}}%
\pgfpathlineto{\pgfqpoint{3.010331in}{3.591108in}}%
\pgfpathlineto{\pgfqpoint{3.121227in}{3.655134in}}%
\pgfpathlineto{\pgfqpoint{3.121227in}{3.527081in}}%
\pgfpathclose%
\pgfusepath{fill}%
\end{pgfscope}%
\begin{pgfscope}%
\pgfpathrectangle{\pgfqpoint{0.765000in}{0.660000in}}{\pgfqpoint{4.620000in}{4.620000in}}%
\pgfusepath{clip}%
\pgfsetbuttcap%
\pgfsetroundjoin%
\definecolor{currentfill}{rgb}{1.000000,0.894118,0.788235}%
\pgfsetfillcolor{currentfill}%
\pgfsetlinewidth{0.000000pt}%
\definecolor{currentstroke}{rgb}{1.000000,0.894118,0.788235}%
\pgfsetstrokecolor{currentstroke}%
\pgfsetdash{}{0pt}%
\pgfpathmoveto{\pgfqpoint{3.121227in}{3.527081in}}%
\pgfpathlineto{\pgfqpoint{3.232124in}{3.463055in}}%
\pgfpathlineto{\pgfqpoint{3.232124in}{3.591108in}}%
\pgfpathlineto{\pgfqpoint{3.121227in}{3.655134in}}%
\pgfpathlineto{\pgfqpoint{3.121227in}{3.527081in}}%
\pgfpathclose%
\pgfusepath{fill}%
\end{pgfscope}%
\begin{pgfscope}%
\pgfpathrectangle{\pgfqpoint{0.765000in}{0.660000in}}{\pgfqpoint{4.620000in}{4.620000in}}%
\pgfusepath{clip}%
\pgfsetbuttcap%
\pgfsetroundjoin%
\definecolor{currentfill}{rgb}{1.000000,0.894118,0.788235}%
\pgfsetfillcolor{currentfill}%
\pgfsetlinewidth{0.000000pt}%
\definecolor{currentstroke}{rgb}{1.000000,0.894118,0.788235}%
\pgfsetstrokecolor{currentstroke}%
\pgfsetdash{}{0pt}%
\pgfpathmoveto{\pgfqpoint{3.107502in}{3.657186in}}%
\pgfpathlineto{\pgfqpoint{2.996605in}{3.593160in}}%
\pgfpathlineto{\pgfqpoint{3.107502in}{3.529134in}}%
\pgfpathlineto{\pgfqpoint{3.218398in}{3.593160in}}%
\pgfpathlineto{\pgfqpoint{3.107502in}{3.657186in}}%
\pgfpathclose%
\pgfusepath{fill}%
\end{pgfscope}%
\begin{pgfscope}%
\pgfpathrectangle{\pgfqpoint{0.765000in}{0.660000in}}{\pgfqpoint{4.620000in}{4.620000in}}%
\pgfusepath{clip}%
\pgfsetbuttcap%
\pgfsetroundjoin%
\definecolor{currentfill}{rgb}{1.000000,0.894118,0.788235}%
\pgfsetfillcolor{currentfill}%
\pgfsetlinewidth{0.000000pt}%
\definecolor{currentstroke}{rgb}{1.000000,0.894118,0.788235}%
\pgfsetstrokecolor{currentstroke}%
\pgfsetdash{}{0pt}%
\pgfpathmoveto{\pgfqpoint{3.107502in}{3.401081in}}%
\pgfpathlineto{\pgfqpoint{3.218398in}{3.465108in}}%
\pgfpathlineto{\pgfqpoint{3.218398in}{3.593160in}}%
\pgfpathlineto{\pgfqpoint{3.107502in}{3.529134in}}%
\pgfpathlineto{\pgfqpoint{3.107502in}{3.401081in}}%
\pgfpathclose%
\pgfusepath{fill}%
\end{pgfscope}%
\begin{pgfscope}%
\pgfpathrectangle{\pgfqpoint{0.765000in}{0.660000in}}{\pgfqpoint{4.620000in}{4.620000in}}%
\pgfusepath{clip}%
\pgfsetbuttcap%
\pgfsetroundjoin%
\definecolor{currentfill}{rgb}{1.000000,0.894118,0.788235}%
\pgfsetfillcolor{currentfill}%
\pgfsetlinewidth{0.000000pt}%
\definecolor{currentstroke}{rgb}{1.000000,0.894118,0.788235}%
\pgfsetstrokecolor{currentstroke}%
\pgfsetdash{}{0pt}%
\pgfpathmoveto{\pgfqpoint{2.996605in}{3.465108in}}%
\pgfpathlineto{\pgfqpoint{3.107502in}{3.401081in}}%
\pgfpathlineto{\pgfqpoint{3.107502in}{3.529134in}}%
\pgfpathlineto{\pgfqpoint{2.996605in}{3.593160in}}%
\pgfpathlineto{\pgfqpoint{2.996605in}{3.465108in}}%
\pgfpathclose%
\pgfusepath{fill}%
\end{pgfscope}%
\begin{pgfscope}%
\pgfpathrectangle{\pgfqpoint{0.765000in}{0.660000in}}{\pgfqpoint{4.620000in}{4.620000in}}%
\pgfusepath{clip}%
\pgfsetbuttcap%
\pgfsetroundjoin%
\definecolor{currentfill}{rgb}{1.000000,0.894118,0.788235}%
\pgfsetfillcolor{currentfill}%
\pgfsetlinewidth{0.000000pt}%
\definecolor{currentstroke}{rgb}{1.000000,0.894118,0.788235}%
\pgfsetstrokecolor{currentstroke}%
\pgfsetdash{}{0pt}%
\pgfpathmoveto{\pgfqpoint{3.121227in}{3.655134in}}%
\pgfpathlineto{\pgfqpoint{3.010331in}{3.591108in}}%
\pgfpathlineto{\pgfqpoint{3.121227in}{3.527081in}}%
\pgfpathlineto{\pgfqpoint{3.232124in}{3.591108in}}%
\pgfpathlineto{\pgfqpoint{3.121227in}{3.655134in}}%
\pgfpathclose%
\pgfusepath{fill}%
\end{pgfscope}%
\begin{pgfscope}%
\pgfpathrectangle{\pgfqpoint{0.765000in}{0.660000in}}{\pgfqpoint{4.620000in}{4.620000in}}%
\pgfusepath{clip}%
\pgfsetbuttcap%
\pgfsetroundjoin%
\definecolor{currentfill}{rgb}{1.000000,0.894118,0.788235}%
\pgfsetfillcolor{currentfill}%
\pgfsetlinewidth{0.000000pt}%
\definecolor{currentstroke}{rgb}{1.000000,0.894118,0.788235}%
\pgfsetstrokecolor{currentstroke}%
\pgfsetdash{}{0pt}%
\pgfpathmoveto{\pgfqpoint{3.121227in}{3.399029in}}%
\pgfpathlineto{\pgfqpoint{3.232124in}{3.463055in}}%
\pgfpathlineto{\pgfqpoint{3.232124in}{3.591108in}}%
\pgfpathlineto{\pgfqpoint{3.121227in}{3.527081in}}%
\pgfpathlineto{\pgfqpoint{3.121227in}{3.399029in}}%
\pgfpathclose%
\pgfusepath{fill}%
\end{pgfscope}%
\begin{pgfscope}%
\pgfpathrectangle{\pgfqpoint{0.765000in}{0.660000in}}{\pgfqpoint{4.620000in}{4.620000in}}%
\pgfusepath{clip}%
\pgfsetbuttcap%
\pgfsetroundjoin%
\definecolor{currentfill}{rgb}{1.000000,0.894118,0.788235}%
\pgfsetfillcolor{currentfill}%
\pgfsetlinewidth{0.000000pt}%
\definecolor{currentstroke}{rgb}{1.000000,0.894118,0.788235}%
\pgfsetstrokecolor{currentstroke}%
\pgfsetdash{}{0pt}%
\pgfpathmoveto{\pgfqpoint{3.010331in}{3.463055in}}%
\pgfpathlineto{\pgfqpoint{3.121227in}{3.399029in}}%
\pgfpathlineto{\pgfqpoint{3.121227in}{3.527081in}}%
\pgfpathlineto{\pgfqpoint{3.010331in}{3.591108in}}%
\pgfpathlineto{\pgfqpoint{3.010331in}{3.463055in}}%
\pgfpathclose%
\pgfusepath{fill}%
\end{pgfscope}%
\begin{pgfscope}%
\pgfpathrectangle{\pgfqpoint{0.765000in}{0.660000in}}{\pgfqpoint{4.620000in}{4.620000in}}%
\pgfusepath{clip}%
\pgfsetbuttcap%
\pgfsetroundjoin%
\definecolor{currentfill}{rgb}{1.000000,0.894118,0.788235}%
\pgfsetfillcolor{currentfill}%
\pgfsetlinewidth{0.000000pt}%
\definecolor{currentstroke}{rgb}{1.000000,0.894118,0.788235}%
\pgfsetstrokecolor{currentstroke}%
\pgfsetdash{}{0pt}%
\pgfpathmoveto{\pgfqpoint{3.107502in}{3.529134in}}%
\pgfpathlineto{\pgfqpoint{2.996605in}{3.465108in}}%
\pgfpathlineto{\pgfqpoint{3.010331in}{3.463055in}}%
\pgfpathlineto{\pgfqpoint{3.121227in}{3.527081in}}%
\pgfpathlineto{\pgfqpoint{3.107502in}{3.529134in}}%
\pgfpathclose%
\pgfusepath{fill}%
\end{pgfscope}%
\begin{pgfscope}%
\pgfpathrectangle{\pgfqpoint{0.765000in}{0.660000in}}{\pgfqpoint{4.620000in}{4.620000in}}%
\pgfusepath{clip}%
\pgfsetbuttcap%
\pgfsetroundjoin%
\definecolor{currentfill}{rgb}{1.000000,0.894118,0.788235}%
\pgfsetfillcolor{currentfill}%
\pgfsetlinewidth{0.000000pt}%
\definecolor{currentstroke}{rgb}{1.000000,0.894118,0.788235}%
\pgfsetstrokecolor{currentstroke}%
\pgfsetdash{}{0pt}%
\pgfpathmoveto{\pgfqpoint{3.218398in}{3.465108in}}%
\pgfpathlineto{\pgfqpoint{3.107502in}{3.529134in}}%
\pgfpathlineto{\pgfqpoint{3.121227in}{3.527081in}}%
\pgfpathlineto{\pgfqpoint{3.232124in}{3.463055in}}%
\pgfpathlineto{\pgfqpoint{3.218398in}{3.465108in}}%
\pgfpathclose%
\pgfusepath{fill}%
\end{pgfscope}%
\begin{pgfscope}%
\pgfpathrectangle{\pgfqpoint{0.765000in}{0.660000in}}{\pgfqpoint{4.620000in}{4.620000in}}%
\pgfusepath{clip}%
\pgfsetbuttcap%
\pgfsetroundjoin%
\definecolor{currentfill}{rgb}{1.000000,0.894118,0.788235}%
\pgfsetfillcolor{currentfill}%
\pgfsetlinewidth{0.000000pt}%
\definecolor{currentstroke}{rgb}{1.000000,0.894118,0.788235}%
\pgfsetstrokecolor{currentstroke}%
\pgfsetdash{}{0pt}%
\pgfpathmoveto{\pgfqpoint{3.107502in}{3.529134in}}%
\pgfpathlineto{\pgfqpoint{3.107502in}{3.657186in}}%
\pgfpathlineto{\pgfqpoint{3.121227in}{3.655134in}}%
\pgfpathlineto{\pgfqpoint{3.232124in}{3.463055in}}%
\pgfpathlineto{\pgfqpoint{3.107502in}{3.529134in}}%
\pgfpathclose%
\pgfusepath{fill}%
\end{pgfscope}%
\begin{pgfscope}%
\pgfpathrectangle{\pgfqpoint{0.765000in}{0.660000in}}{\pgfqpoint{4.620000in}{4.620000in}}%
\pgfusepath{clip}%
\pgfsetbuttcap%
\pgfsetroundjoin%
\definecolor{currentfill}{rgb}{1.000000,0.894118,0.788235}%
\pgfsetfillcolor{currentfill}%
\pgfsetlinewidth{0.000000pt}%
\definecolor{currentstroke}{rgb}{1.000000,0.894118,0.788235}%
\pgfsetstrokecolor{currentstroke}%
\pgfsetdash{}{0pt}%
\pgfpathmoveto{\pgfqpoint{3.218398in}{3.465108in}}%
\pgfpathlineto{\pgfqpoint{3.218398in}{3.593160in}}%
\pgfpathlineto{\pgfqpoint{3.232124in}{3.591108in}}%
\pgfpathlineto{\pgfqpoint{3.121227in}{3.527081in}}%
\pgfpathlineto{\pgfqpoint{3.218398in}{3.465108in}}%
\pgfpathclose%
\pgfusepath{fill}%
\end{pgfscope}%
\begin{pgfscope}%
\pgfpathrectangle{\pgfqpoint{0.765000in}{0.660000in}}{\pgfqpoint{4.620000in}{4.620000in}}%
\pgfusepath{clip}%
\pgfsetbuttcap%
\pgfsetroundjoin%
\definecolor{currentfill}{rgb}{1.000000,0.894118,0.788235}%
\pgfsetfillcolor{currentfill}%
\pgfsetlinewidth{0.000000pt}%
\definecolor{currentstroke}{rgb}{1.000000,0.894118,0.788235}%
\pgfsetstrokecolor{currentstroke}%
\pgfsetdash{}{0pt}%
\pgfpathmoveto{\pgfqpoint{2.996605in}{3.465108in}}%
\pgfpathlineto{\pgfqpoint{3.107502in}{3.401081in}}%
\pgfpathlineto{\pgfqpoint{3.121227in}{3.399029in}}%
\pgfpathlineto{\pgfqpoint{3.010331in}{3.463055in}}%
\pgfpathlineto{\pgfqpoint{2.996605in}{3.465108in}}%
\pgfpathclose%
\pgfusepath{fill}%
\end{pgfscope}%
\begin{pgfscope}%
\pgfpathrectangle{\pgfqpoint{0.765000in}{0.660000in}}{\pgfqpoint{4.620000in}{4.620000in}}%
\pgfusepath{clip}%
\pgfsetbuttcap%
\pgfsetroundjoin%
\definecolor{currentfill}{rgb}{1.000000,0.894118,0.788235}%
\pgfsetfillcolor{currentfill}%
\pgfsetlinewidth{0.000000pt}%
\definecolor{currentstroke}{rgb}{1.000000,0.894118,0.788235}%
\pgfsetstrokecolor{currentstroke}%
\pgfsetdash{}{0pt}%
\pgfpathmoveto{\pgfqpoint{3.107502in}{3.401081in}}%
\pgfpathlineto{\pgfqpoint{3.218398in}{3.465108in}}%
\pgfpathlineto{\pgfqpoint{3.232124in}{3.463055in}}%
\pgfpathlineto{\pgfqpoint{3.121227in}{3.399029in}}%
\pgfpathlineto{\pgfqpoint{3.107502in}{3.401081in}}%
\pgfpathclose%
\pgfusepath{fill}%
\end{pgfscope}%
\begin{pgfscope}%
\pgfpathrectangle{\pgfqpoint{0.765000in}{0.660000in}}{\pgfqpoint{4.620000in}{4.620000in}}%
\pgfusepath{clip}%
\pgfsetbuttcap%
\pgfsetroundjoin%
\definecolor{currentfill}{rgb}{1.000000,0.894118,0.788235}%
\pgfsetfillcolor{currentfill}%
\pgfsetlinewidth{0.000000pt}%
\definecolor{currentstroke}{rgb}{1.000000,0.894118,0.788235}%
\pgfsetstrokecolor{currentstroke}%
\pgfsetdash{}{0pt}%
\pgfpathmoveto{\pgfqpoint{3.107502in}{3.657186in}}%
\pgfpathlineto{\pgfqpoint{2.996605in}{3.593160in}}%
\pgfpathlineto{\pgfqpoint{3.010331in}{3.591108in}}%
\pgfpathlineto{\pgfqpoint{3.121227in}{3.655134in}}%
\pgfpathlineto{\pgfqpoint{3.107502in}{3.657186in}}%
\pgfpathclose%
\pgfusepath{fill}%
\end{pgfscope}%
\begin{pgfscope}%
\pgfpathrectangle{\pgfqpoint{0.765000in}{0.660000in}}{\pgfqpoint{4.620000in}{4.620000in}}%
\pgfusepath{clip}%
\pgfsetbuttcap%
\pgfsetroundjoin%
\definecolor{currentfill}{rgb}{1.000000,0.894118,0.788235}%
\pgfsetfillcolor{currentfill}%
\pgfsetlinewidth{0.000000pt}%
\definecolor{currentstroke}{rgb}{1.000000,0.894118,0.788235}%
\pgfsetstrokecolor{currentstroke}%
\pgfsetdash{}{0pt}%
\pgfpathmoveto{\pgfqpoint{3.218398in}{3.593160in}}%
\pgfpathlineto{\pgfqpoint{3.107502in}{3.657186in}}%
\pgfpathlineto{\pgfqpoint{3.121227in}{3.655134in}}%
\pgfpathlineto{\pgfqpoint{3.232124in}{3.591108in}}%
\pgfpathlineto{\pgfqpoint{3.218398in}{3.593160in}}%
\pgfpathclose%
\pgfusepath{fill}%
\end{pgfscope}%
\begin{pgfscope}%
\pgfpathrectangle{\pgfqpoint{0.765000in}{0.660000in}}{\pgfqpoint{4.620000in}{4.620000in}}%
\pgfusepath{clip}%
\pgfsetbuttcap%
\pgfsetroundjoin%
\definecolor{currentfill}{rgb}{1.000000,0.894118,0.788235}%
\pgfsetfillcolor{currentfill}%
\pgfsetlinewidth{0.000000pt}%
\definecolor{currentstroke}{rgb}{1.000000,0.894118,0.788235}%
\pgfsetstrokecolor{currentstroke}%
\pgfsetdash{}{0pt}%
\pgfpathmoveto{\pgfqpoint{2.996605in}{3.465108in}}%
\pgfpathlineto{\pgfqpoint{2.996605in}{3.593160in}}%
\pgfpathlineto{\pgfqpoint{3.010331in}{3.591108in}}%
\pgfpathlineto{\pgfqpoint{3.121227in}{3.399029in}}%
\pgfpathlineto{\pgfqpoint{2.996605in}{3.465108in}}%
\pgfpathclose%
\pgfusepath{fill}%
\end{pgfscope}%
\begin{pgfscope}%
\pgfpathrectangle{\pgfqpoint{0.765000in}{0.660000in}}{\pgfqpoint{4.620000in}{4.620000in}}%
\pgfusepath{clip}%
\pgfsetbuttcap%
\pgfsetroundjoin%
\definecolor{currentfill}{rgb}{1.000000,0.894118,0.788235}%
\pgfsetfillcolor{currentfill}%
\pgfsetlinewidth{0.000000pt}%
\definecolor{currentstroke}{rgb}{1.000000,0.894118,0.788235}%
\pgfsetstrokecolor{currentstroke}%
\pgfsetdash{}{0pt}%
\pgfpathmoveto{\pgfqpoint{3.107502in}{3.401081in}}%
\pgfpathlineto{\pgfqpoint{3.107502in}{3.529134in}}%
\pgfpathlineto{\pgfqpoint{3.121227in}{3.527081in}}%
\pgfpathlineto{\pgfqpoint{3.010331in}{3.463055in}}%
\pgfpathlineto{\pgfqpoint{3.107502in}{3.401081in}}%
\pgfpathclose%
\pgfusepath{fill}%
\end{pgfscope}%
\begin{pgfscope}%
\pgfpathrectangle{\pgfqpoint{0.765000in}{0.660000in}}{\pgfqpoint{4.620000in}{4.620000in}}%
\pgfusepath{clip}%
\pgfsetbuttcap%
\pgfsetroundjoin%
\definecolor{currentfill}{rgb}{1.000000,0.894118,0.788235}%
\pgfsetfillcolor{currentfill}%
\pgfsetlinewidth{0.000000pt}%
\definecolor{currentstroke}{rgb}{1.000000,0.894118,0.788235}%
\pgfsetstrokecolor{currentstroke}%
\pgfsetdash{}{0pt}%
\pgfpathmoveto{\pgfqpoint{3.107502in}{3.529134in}}%
\pgfpathlineto{\pgfqpoint{3.218398in}{3.593160in}}%
\pgfpathlineto{\pgfqpoint{3.232124in}{3.591108in}}%
\pgfpathlineto{\pgfqpoint{3.121227in}{3.527081in}}%
\pgfpathlineto{\pgfqpoint{3.107502in}{3.529134in}}%
\pgfpathclose%
\pgfusepath{fill}%
\end{pgfscope}%
\begin{pgfscope}%
\pgfpathrectangle{\pgfqpoint{0.765000in}{0.660000in}}{\pgfqpoint{4.620000in}{4.620000in}}%
\pgfusepath{clip}%
\pgfsetbuttcap%
\pgfsetroundjoin%
\definecolor{currentfill}{rgb}{1.000000,0.894118,0.788235}%
\pgfsetfillcolor{currentfill}%
\pgfsetlinewidth{0.000000pt}%
\definecolor{currentstroke}{rgb}{1.000000,0.894118,0.788235}%
\pgfsetstrokecolor{currentstroke}%
\pgfsetdash{}{0pt}%
\pgfpathmoveto{\pgfqpoint{2.996605in}{3.593160in}}%
\pgfpathlineto{\pgfqpoint{3.107502in}{3.529134in}}%
\pgfpathlineto{\pgfqpoint{3.121227in}{3.527081in}}%
\pgfpathlineto{\pgfqpoint{3.010331in}{3.591108in}}%
\pgfpathlineto{\pgfqpoint{2.996605in}{3.593160in}}%
\pgfpathclose%
\pgfusepath{fill}%
\end{pgfscope}%
\begin{pgfscope}%
\pgfpathrectangle{\pgfqpoint{0.765000in}{0.660000in}}{\pgfqpoint{4.620000in}{4.620000in}}%
\pgfusepath{clip}%
\pgfsetbuttcap%
\pgfsetroundjoin%
\definecolor{currentfill}{rgb}{1.000000,0.894118,0.788235}%
\pgfsetfillcolor{currentfill}%
\pgfsetlinewidth{0.000000pt}%
\definecolor{currentstroke}{rgb}{1.000000,0.894118,0.788235}%
\pgfsetstrokecolor{currentstroke}%
\pgfsetdash{}{0pt}%
\pgfpathmoveto{\pgfqpoint{3.107502in}{3.529134in}}%
\pgfpathlineto{\pgfqpoint{2.996605in}{3.465108in}}%
\pgfpathlineto{\pgfqpoint{3.107502in}{3.401081in}}%
\pgfpathlineto{\pgfqpoint{3.218398in}{3.465108in}}%
\pgfpathlineto{\pgfqpoint{3.107502in}{3.529134in}}%
\pgfpathclose%
\pgfusepath{fill}%
\end{pgfscope}%
\begin{pgfscope}%
\pgfpathrectangle{\pgfqpoint{0.765000in}{0.660000in}}{\pgfqpoint{4.620000in}{4.620000in}}%
\pgfusepath{clip}%
\pgfsetbuttcap%
\pgfsetroundjoin%
\definecolor{currentfill}{rgb}{1.000000,0.894118,0.788235}%
\pgfsetfillcolor{currentfill}%
\pgfsetlinewidth{0.000000pt}%
\definecolor{currentstroke}{rgb}{1.000000,0.894118,0.788235}%
\pgfsetstrokecolor{currentstroke}%
\pgfsetdash{}{0pt}%
\pgfpathmoveto{\pgfqpoint{3.107502in}{3.529134in}}%
\pgfpathlineto{\pgfqpoint{2.996605in}{3.465108in}}%
\pgfpathlineto{\pgfqpoint{2.996605in}{3.593160in}}%
\pgfpathlineto{\pgfqpoint{3.107502in}{3.657186in}}%
\pgfpathlineto{\pgfqpoint{3.107502in}{3.529134in}}%
\pgfpathclose%
\pgfusepath{fill}%
\end{pgfscope}%
\begin{pgfscope}%
\pgfpathrectangle{\pgfqpoint{0.765000in}{0.660000in}}{\pgfqpoint{4.620000in}{4.620000in}}%
\pgfusepath{clip}%
\pgfsetbuttcap%
\pgfsetroundjoin%
\definecolor{currentfill}{rgb}{1.000000,0.894118,0.788235}%
\pgfsetfillcolor{currentfill}%
\pgfsetlinewidth{0.000000pt}%
\definecolor{currentstroke}{rgb}{1.000000,0.894118,0.788235}%
\pgfsetstrokecolor{currentstroke}%
\pgfsetdash{}{0pt}%
\pgfpathmoveto{\pgfqpoint{3.107502in}{3.529134in}}%
\pgfpathlineto{\pgfqpoint{3.218398in}{3.465108in}}%
\pgfpathlineto{\pgfqpoint{3.218398in}{3.593160in}}%
\pgfpathlineto{\pgfqpoint{3.107502in}{3.657186in}}%
\pgfpathlineto{\pgfqpoint{3.107502in}{3.529134in}}%
\pgfpathclose%
\pgfusepath{fill}%
\end{pgfscope}%
\begin{pgfscope}%
\pgfpathrectangle{\pgfqpoint{0.765000in}{0.660000in}}{\pgfqpoint{4.620000in}{4.620000in}}%
\pgfusepath{clip}%
\pgfsetbuttcap%
\pgfsetroundjoin%
\definecolor{currentfill}{rgb}{1.000000,0.894118,0.788235}%
\pgfsetfillcolor{currentfill}%
\pgfsetlinewidth{0.000000pt}%
\definecolor{currentstroke}{rgb}{1.000000,0.894118,0.788235}%
\pgfsetstrokecolor{currentstroke}%
\pgfsetdash{}{0pt}%
\pgfpathmoveto{\pgfqpoint{3.085349in}{3.530998in}}%
\pgfpathlineto{\pgfqpoint{2.974452in}{3.466972in}}%
\pgfpathlineto{\pgfqpoint{3.085349in}{3.402945in}}%
\pgfpathlineto{\pgfqpoint{3.196246in}{3.466972in}}%
\pgfpathlineto{\pgfqpoint{3.085349in}{3.530998in}}%
\pgfpathclose%
\pgfusepath{fill}%
\end{pgfscope}%
\begin{pgfscope}%
\pgfpathrectangle{\pgfqpoint{0.765000in}{0.660000in}}{\pgfqpoint{4.620000in}{4.620000in}}%
\pgfusepath{clip}%
\pgfsetbuttcap%
\pgfsetroundjoin%
\definecolor{currentfill}{rgb}{1.000000,0.894118,0.788235}%
\pgfsetfillcolor{currentfill}%
\pgfsetlinewidth{0.000000pt}%
\definecolor{currentstroke}{rgb}{1.000000,0.894118,0.788235}%
\pgfsetstrokecolor{currentstroke}%
\pgfsetdash{}{0pt}%
\pgfpathmoveto{\pgfqpoint{3.085349in}{3.530998in}}%
\pgfpathlineto{\pgfqpoint{2.974452in}{3.466972in}}%
\pgfpathlineto{\pgfqpoint{2.974452in}{3.595024in}}%
\pgfpathlineto{\pgfqpoint{3.085349in}{3.659050in}}%
\pgfpathlineto{\pgfqpoint{3.085349in}{3.530998in}}%
\pgfpathclose%
\pgfusepath{fill}%
\end{pgfscope}%
\begin{pgfscope}%
\pgfpathrectangle{\pgfqpoint{0.765000in}{0.660000in}}{\pgfqpoint{4.620000in}{4.620000in}}%
\pgfusepath{clip}%
\pgfsetbuttcap%
\pgfsetroundjoin%
\definecolor{currentfill}{rgb}{1.000000,0.894118,0.788235}%
\pgfsetfillcolor{currentfill}%
\pgfsetlinewidth{0.000000pt}%
\definecolor{currentstroke}{rgb}{1.000000,0.894118,0.788235}%
\pgfsetstrokecolor{currentstroke}%
\pgfsetdash{}{0pt}%
\pgfpathmoveto{\pgfqpoint{3.085349in}{3.530998in}}%
\pgfpathlineto{\pgfqpoint{3.196246in}{3.466972in}}%
\pgfpathlineto{\pgfqpoint{3.196246in}{3.595024in}}%
\pgfpathlineto{\pgfqpoint{3.085349in}{3.659050in}}%
\pgfpathlineto{\pgfqpoint{3.085349in}{3.530998in}}%
\pgfpathclose%
\pgfusepath{fill}%
\end{pgfscope}%
\begin{pgfscope}%
\pgfpathrectangle{\pgfqpoint{0.765000in}{0.660000in}}{\pgfqpoint{4.620000in}{4.620000in}}%
\pgfusepath{clip}%
\pgfsetbuttcap%
\pgfsetroundjoin%
\definecolor{currentfill}{rgb}{1.000000,0.894118,0.788235}%
\pgfsetfillcolor{currentfill}%
\pgfsetlinewidth{0.000000pt}%
\definecolor{currentstroke}{rgb}{1.000000,0.894118,0.788235}%
\pgfsetstrokecolor{currentstroke}%
\pgfsetdash{}{0pt}%
\pgfpathmoveto{\pgfqpoint{3.107502in}{3.657186in}}%
\pgfpathlineto{\pgfqpoint{2.996605in}{3.593160in}}%
\pgfpathlineto{\pgfqpoint{3.107502in}{3.529134in}}%
\pgfpathlineto{\pgfqpoint{3.218398in}{3.593160in}}%
\pgfpathlineto{\pgfqpoint{3.107502in}{3.657186in}}%
\pgfpathclose%
\pgfusepath{fill}%
\end{pgfscope}%
\begin{pgfscope}%
\pgfpathrectangle{\pgfqpoint{0.765000in}{0.660000in}}{\pgfqpoint{4.620000in}{4.620000in}}%
\pgfusepath{clip}%
\pgfsetbuttcap%
\pgfsetroundjoin%
\definecolor{currentfill}{rgb}{1.000000,0.894118,0.788235}%
\pgfsetfillcolor{currentfill}%
\pgfsetlinewidth{0.000000pt}%
\definecolor{currentstroke}{rgb}{1.000000,0.894118,0.788235}%
\pgfsetstrokecolor{currentstroke}%
\pgfsetdash{}{0pt}%
\pgfpathmoveto{\pgfqpoint{3.107502in}{3.401081in}}%
\pgfpathlineto{\pgfqpoint{3.218398in}{3.465108in}}%
\pgfpathlineto{\pgfqpoint{3.218398in}{3.593160in}}%
\pgfpathlineto{\pgfqpoint{3.107502in}{3.529134in}}%
\pgfpathlineto{\pgfqpoint{3.107502in}{3.401081in}}%
\pgfpathclose%
\pgfusepath{fill}%
\end{pgfscope}%
\begin{pgfscope}%
\pgfpathrectangle{\pgfqpoint{0.765000in}{0.660000in}}{\pgfqpoint{4.620000in}{4.620000in}}%
\pgfusepath{clip}%
\pgfsetbuttcap%
\pgfsetroundjoin%
\definecolor{currentfill}{rgb}{1.000000,0.894118,0.788235}%
\pgfsetfillcolor{currentfill}%
\pgfsetlinewidth{0.000000pt}%
\definecolor{currentstroke}{rgb}{1.000000,0.894118,0.788235}%
\pgfsetstrokecolor{currentstroke}%
\pgfsetdash{}{0pt}%
\pgfpathmoveto{\pgfqpoint{2.996605in}{3.465108in}}%
\pgfpathlineto{\pgfqpoint{3.107502in}{3.401081in}}%
\pgfpathlineto{\pgfqpoint{3.107502in}{3.529134in}}%
\pgfpathlineto{\pgfqpoint{2.996605in}{3.593160in}}%
\pgfpathlineto{\pgfqpoint{2.996605in}{3.465108in}}%
\pgfpathclose%
\pgfusepath{fill}%
\end{pgfscope}%
\begin{pgfscope}%
\pgfpathrectangle{\pgfqpoint{0.765000in}{0.660000in}}{\pgfqpoint{4.620000in}{4.620000in}}%
\pgfusepath{clip}%
\pgfsetbuttcap%
\pgfsetroundjoin%
\definecolor{currentfill}{rgb}{1.000000,0.894118,0.788235}%
\pgfsetfillcolor{currentfill}%
\pgfsetlinewidth{0.000000pt}%
\definecolor{currentstroke}{rgb}{1.000000,0.894118,0.788235}%
\pgfsetstrokecolor{currentstroke}%
\pgfsetdash{}{0pt}%
\pgfpathmoveto{\pgfqpoint{3.085349in}{3.659050in}}%
\pgfpathlineto{\pgfqpoint{2.974452in}{3.595024in}}%
\pgfpathlineto{\pgfqpoint{3.085349in}{3.530998in}}%
\pgfpathlineto{\pgfqpoint{3.196246in}{3.595024in}}%
\pgfpathlineto{\pgfqpoint{3.085349in}{3.659050in}}%
\pgfpathclose%
\pgfusepath{fill}%
\end{pgfscope}%
\begin{pgfscope}%
\pgfpathrectangle{\pgfqpoint{0.765000in}{0.660000in}}{\pgfqpoint{4.620000in}{4.620000in}}%
\pgfusepath{clip}%
\pgfsetbuttcap%
\pgfsetroundjoin%
\definecolor{currentfill}{rgb}{1.000000,0.894118,0.788235}%
\pgfsetfillcolor{currentfill}%
\pgfsetlinewidth{0.000000pt}%
\definecolor{currentstroke}{rgb}{1.000000,0.894118,0.788235}%
\pgfsetstrokecolor{currentstroke}%
\pgfsetdash{}{0pt}%
\pgfpathmoveto{\pgfqpoint{3.085349in}{3.402945in}}%
\pgfpathlineto{\pgfqpoint{3.196246in}{3.466972in}}%
\pgfpathlineto{\pgfqpoint{3.196246in}{3.595024in}}%
\pgfpathlineto{\pgfqpoint{3.085349in}{3.530998in}}%
\pgfpathlineto{\pgfqpoint{3.085349in}{3.402945in}}%
\pgfpathclose%
\pgfusepath{fill}%
\end{pgfscope}%
\begin{pgfscope}%
\pgfpathrectangle{\pgfqpoint{0.765000in}{0.660000in}}{\pgfqpoint{4.620000in}{4.620000in}}%
\pgfusepath{clip}%
\pgfsetbuttcap%
\pgfsetroundjoin%
\definecolor{currentfill}{rgb}{1.000000,0.894118,0.788235}%
\pgfsetfillcolor{currentfill}%
\pgfsetlinewidth{0.000000pt}%
\definecolor{currentstroke}{rgb}{1.000000,0.894118,0.788235}%
\pgfsetstrokecolor{currentstroke}%
\pgfsetdash{}{0pt}%
\pgfpathmoveto{\pgfqpoint{2.974452in}{3.466972in}}%
\pgfpathlineto{\pgfqpoint{3.085349in}{3.402945in}}%
\pgfpathlineto{\pgfqpoint{3.085349in}{3.530998in}}%
\pgfpathlineto{\pgfqpoint{2.974452in}{3.595024in}}%
\pgfpathlineto{\pgfqpoint{2.974452in}{3.466972in}}%
\pgfpathclose%
\pgfusepath{fill}%
\end{pgfscope}%
\begin{pgfscope}%
\pgfpathrectangle{\pgfqpoint{0.765000in}{0.660000in}}{\pgfqpoint{4.620000in}{4.620000in}}%
\pgfusepath{clip}%
\pgfsetbuttcap%
\pgfsetroundjoin%
\definecolor{currentfill}{rgb}{1.000000,0.894118,0.788235}%
\pgfsetfillcolor{currentfill}%
\pgfsetlinewidth{0.000000pt}%
\definecolor{currentstroke}{rgb}{1.000000,0.894118,0.788235}%
\pgfsetstrokecolor{currentstroke}%
\pgfsetdash{}{0pt}%
\pgfpathmoveto{\pgfqpoint{3.107502in}{3.529134in}}%
\pgfpathlineto{\pgfqpoint{2.996605in}{3.465108in}}%
\pgfpathlineto{\pgfqpoint{2.974452in}{3.466972in}}%
\pgfpathlineto{\pgfqpoint{3.085349in}{3.530998in}}%
\pgfpathlineto{\pgfqpoint{3.107502in}{3.529134in}}%
\pgfpathclose%
\pgfusepath{fill}%
\end{pgfscope}%
\begin{pgfscope}%
\pgfpathrectangle{\pgfqpoint{0.765000in}{0.660000in}}{\pgfqpoint{4.620000in}{4.620000in}}%
\pgfusepath{clip}%
\pgfsetbuttcap%
\pgfsetroundjoin%
\definecolor{currentfill}{rgb}{1.000000,0.894118,0.788235}%
\pgfsetfillcolor{currentfill}%
\pgfsetlinewidth{0.000000pt}%
\definecolor{currentstroke}{rgb}{1.000000,0.894118,0.788235}%
\pgfsetstrokecolor{currentstroke}%
\pgfsetdash{}{0pt}%
\pgfpathmoveto{\pgfqpoint{3.218398in}{3.465108in}}%
\pgfpathlineto{\pgfqpoint{3.107502in}{3.529134in}}%
\pgfpathlineto{\pgfqpoint{3.085349in}{3.530998in}}%
\pgfpathlineto{\pgfqpoint{3.196246in}{3.466972in}}%
\pgfpathlineto{\pgfqpoint{3.218398in}{3.465108in}}%
\pgfpathclose%
\pgfusepath{fill}%
\end{pgfscope}%
\begin{pgfscope}%
\pgfpathrectangle{\pgfqpoint{0.765000in}{0.660000in}}{\pgfqpoint{4.620000in}{4.620000in}}%
\pgfusepath{clip}%
\pgfsetbuttcap%
\pgfsetroundjoin%
\definecolor{currentfill}{rgb}{1.000000,0.894118,0.788235}%
\pgfsetfillcolor{currentfill}%
\pgfsetlinewidth{0.000000pt}%
\definecolor{currentstroke}{rgb}{1.000000,0.894118,0.788235}%
\pgfsetstrokecolor{currentstroke}%
\pgfsetdash{}{0pt}%
\pgfpathmoveto{\pgfqpoint{3.107502in}{3.529134in}}%
\pgfpathlineto{\pgfqpoint{3.107502in}{3.657186in}}%
\pgfpathlineto{\pgfqpoint{3.085349in}{3.659050in}}%
\pgfpathlineto{\pgfqpoint{3.196246in}{3.466972in}}%
\pgfpathlineto{\pgfqpoint{3.107502in}{3.529134in}}%
\pgfpathclose%
\pgfusepath{fill}%
\end{pgfscope}%
\begin{pgfscope}%
\pgfpathrectangle{\pgfqpoint{0.765000in}{0.660000in}}{\pgfqpoint{4.620000in}{4.620000in}}%
\pgfusepath{clip}%
\pgfsetbuttcap%
\pgfsetroundjoin%
\definecolor{currentfill}{rgb}{1.000000,0.894118,0.788235}%
\pgfsetfillcolor{currentfill}%
\pgfsetlinewidth{0.000000pt}%
\definecolor{currentstroke}{rgb}{1.000000,0.894118,0.788235}%
\pgfsetstrokecolor{currentstroke}%
\pgfsetdash{}{0pt}%
\pgfpathmoveto{\pgfqpoint{3.218398in}{3.465108in}}%
\pgfpathlineto{\pgfqpoint{3.218398in}{3.593160in}}%
\pgfpathlineto{\pgfqpoint{3.196246in}{3.595024in}}%
\pgfpathlineto{\pgfqpoint{3.085349in}{3.530998in}}%
\pgfpathlineto{\pgfqpoint{3.218398in}{3.465108in}}%
\pgfpathclose%
\pgfusepath{fill}%
\end{pgfscope}%
\begin{pgfscope}%
\pgfpathrectangle{\pgfqpoint{0.765000in}{0.660000in}}{\pgfqpoint{4.620000in}{4.620000in}}%
\pgfusepath{clip}%
\pgfsetbuttcap%
\pgfsetroundjoin%
\definecolor{currentfill}{rgb}{1.000000,0.894118,0.788235}%
\pgfsetfillcolor{currentfill}%
\pgfsetlinewidth{0.000000pt}%
\definecolor{currentstroke}{rgb}{1.000000,0.894118,0.788235}%
\pgfsetstrokecolor{currentstroke}%
\pgfsetdash{}{0pt}%
\pgfpathmoveto{\pgfqpoint{2.996605in}{3.465108in}}%
\pgfpathlineto{\pgfqpoint{3.107502in}{3.401081in}}%
\pgfpathlineto{\pgfqpoint{3.085349in}{3.402945in}}%
\pgfpathlineto{\pgfqpoint{2.974452in}{3.466972in}}%
\pgfpathlineto{\pgfqpoint{2.996605in}{3.465108in}}%
\pgfpathclose%
\pgfusepath{fill}%
\end{pgfscope}%
\begin{pgfscope}%
\pgfpathrectangle{\pgfqpoint{0.765000in}{0.660000in}}{\pgfqpoint{4.620000in}{4.620000in}}%
\pgfusepath{clip}%
\pgfsetbuttcap%
\pgfsetroundjoin%
\definecolor{currentfill}{rgb}{1.000000,0.894118,0.788235}%
\pgfsetfillcolor{currentfill}%
\pgfsetlinewidth{0.000000pt}%
\definecolor{currentstroke}{rgb}{1.000000,0.894118,0.788235}%
\pgfsetstrokecolor{currentstroke}%
\pgfsetdash{}{0pt}%
\pgfpathmoveto{\pgfqpoint{3.107502in}{3.401081in}}%
\pgfpathlineto{\pgfqpoint{3.218398in}{3.465108in}}%
\pgfpathlineto{\pgfqpoint{3.196246in}{3.466972in}}%
\pgfpathlineto{\pgfqpoint{3.085349in}{3.402945in}}%
\pgfpathlineto{\pgfqpoint{3.107502in}{3.401081in}}%
\pgfpathclose%
\pgfusepath{fill}%
\end{pgfscope}%
\begin{pgfscope}%
\pgfpathrectangle{\pgfqpoint{0.765000in}{0.660000in}}{\pgfqpoint{4.620000in}{4.620000in}}%
\pgfusepath{clip}%
\pgfsetbuttcap%
\pgfsetroundjoin%
\definecolor{currentfill}{rgb}{1.000000,0.894118,0.788235}%
\pgfsetfillcolor{currentfill}%
\pgfsetlinewidth{0.000000pt}%
\definecolor{currentstroke}{rgb}{1.000000,0.894118,0.788235}%
\pgfsetstrokecolor{currentstroke}%
\pgfsetdash{}{0pt}%
\pgfpathmoveto{\pgfqpoint{3.107502in}{3.657186in}}%
\pgfpathlineto{\pgfqpoint{2.996605in}{3.593160in}}%
\pgfpathlineto{\pgfqpoint{2.974452in}{3.595024in}}%
\pgfpathlineto{\pgfqpoint{3.085349in}{3.659050in}}%
\pgfpathlineto{\pgfqpoint{3.107502in}{3.657186in}}%
\pgfpathclose%
\pgfusepath{fill}%
\end{pgfscope}%
\begin{pgfscope}%
\pgfpathrectangle{\pgfqpoint{0.765000in}{0.660000in}}{\pgfqpoint{4.620000in}{4.620000in}}%
\pgfusepath{clip}%
\pgfsetbuttcap%
\pgfsetroundjoin%
\definecolor{currentfill}{rgb}{1.000000,0.894118,0.788235}%
\pgfsetfillcolor{currentfill}%
\pgfsetlinewidth{0.000000pt}%
\definecolor{currentstroke}{rgb}{1.000000,0.894118,0.788235}%
\pgfsetstrokecolor{currentstroke}%
\pgfsetdash{}{0pt}%
\pgfpathmoveto{\pgfqpoint{3.218398in}{3.593160in}}%
\pgfpathlineto{\pgfqpoint{3.107502in}{3.657186in}}%
\pgfpathlineto{\pgfqpoint{3.085349in}{3.659050in}}%
\pgfpathlineto{\pgfqpoint{3.196246in}{3.595024in}}%
\pgfpathlineto{\pgfqpoint{3.218398in}{3.593160in}}%
\pgfpathclose%
\pgfusepath{fill}%
\end{pgfscope}%
\begin{pgfscope}%
\pgfpathrectangle{\pgfqpoint{0.765000in}{0.660000in}}{\pgfqpoint{4.620000in}{4.620000in}}%
\pgfusepath{clip}%
\pgfsetbuttcap%
\pgfsetroundjoin%
\definecolor{currentfill}{rgb}{1.000000,0.894118,0.788235}%
\pgfsetfillcolor{currentfill}%
\pgfsetlinewidth{0.000000pt}%
\definecolor{currentstroke}{rgb}{1.000000,0.894118,0.788235}%
\pgfsetstrokecolor{currentstroke}%
\pgfsetdash{}{0pt}%
\pgfpathmoveto{\pgfqpoint{2.996605in}{3.465108in}}%
\pgfpathlineto{\pgfqpoint{2.996605in}{3.593160in}}%
\pgfpathlineto{\pgfqpoint{2.974452in}{3.595024in}}%
\pgfpathlineto{\pgfqpoint{3.085349in}{3.402945in}}%
\pgfpathlineto{\pgfqpoint{2.996605in}{3.465108in}}%
\pgfpathclose%
\pgfusepath{fill}%
\end{pgfscope}%
\begin{pgfscope}%
\pgfpathrectangle{\pgfqpoint{0.765000in}{0.660000in}}{\pgfqpoint{4.620000in}{4.620000in}}%
\pgfusepath{clip}%
\pgfsetbuttcap%
\pgfsetroundjoin%
\definecolor{currentfill}{rgb}{1.000000,0.894118,0.788235}%
\pgfsetfillcolor{currentfill}%
\pgfsetlinewidth{0.000000pt}%
\definecolor{currentstroke}{rgb}{1.000000,0.894118,0.788235}%
\pgfsetstrokecolor{currentstroke}%
\pgfsetdash{}{0pt}%
\pgfpathmoveto{\pgfqpoint{3.107502in}{3.401081in}}%
\pgfpathlineto{\pgfqpoint{3.107502in}{3.529134in}}%
\pgfpathlineto{\pgfqpoint{3.085349in}{3.530998in}}%
\pgfpathlineto{\pgfqpoint{2.974452in}{3.466972in}}%
\pgfpathlineto{\pgfqpoint{3.107502in}{3.401081in}}%
\pgfpathclose%
\pgfusepath{fill}%
\end{pgfscope}%
\begin{pgfscope}%
\pgfpathrectangle{\pgfqpoint{0.765000in}{0.660000in}}{\pgfqpoint{4.620000in}{4.620000in}}%
\pgfusepath{clip}%
\pgfsetbuttcap%
\pgfsetroundjoin%
\definecolor{currentfill}{rgb}{1.000000,0.894118,0.788235}%
\pgfsetfillcolor{currentfill}%
\pgfsetlinewidth{0.000000pt}%
\definecolor{currentstroke}{rgb}{1.000000,0.894118,0.788235}%
\pgfsetstrokecolor{currentstroke}%
\pgfsetdash{}{0pt}%
\pgfpathmoveto{\pgfqpoint{3.107502in}{3.529134in}}%
\pgfpathlineto{\pgfqpoint{3.218398in}{3.593160in}}%
\pgfpathlineto{\pgfqpoint{3.196246in}{3.595024in}}%
\pgfpathlineto{\pgfqpoint{3.085349in}{3.530998in}}%
\pgfpathlineto{\pgfqpoint{3.107502in}{3.529134in}}%
\pgfpathclose%
\pgfusepath{fill}%
\end{pgfscope}%
\begin{pgfscope}%
\pgfpathrectangle{\pgfqpoint{0.765000in}{0.660000in}}{\pgfqpoint{4.620000in}{4.620000in}}%
\pgfusepath{clip}%
\pgfsetbuttcap%
\pgfsetroundjoin%
\definecolor{currentfill}{rgb}{1.000000,0.894118,0.788235}%
\pgfsetfillcolor{currentfill}%
\pgfsetlinewidth{0.000000pt}%
\definecolor{currentstroke}{rgb}{1.000000,0.894118,0.788235}%
\pgfsetstrokecolor{currentstroke}%
\pgfsetdash{}{0pt}%
\pgfpathmoveto{\pgfqpoint{2.996605in}{3.593160in}}%
\pgfpathlineto{\pgfqpoint{3.107502in}{3.529134in}}%
\pgfpathlineto{\pgfqpoint{3.085349in}{3.530998in}}%
\pgfpathlineto{\pgfqpoint{2.974452in}{3.595024in}}%
\pgfpathlineto{\pgfqpoint{2.996605in}{3.593160in}}%
\pgfpathclose%
\pgfusepath{fill}%
\end{pgfscope}%
\begin{pgfscope}%
\pgfpathrectangle{\pgfqpoint{0.765000in}{0.660000in}}{\pgfqpoint{4.620000in}{4.620000in}}%
\pgfusepath{clip}%
\pgfsetbuttcap%
\pgfsetroundjoin%
\definecolor{currentfill}{rgb}{1.000000,0.894118,0.788235}%
\pgfsetfillcolor{currentfill}%
\pgfsetlinewidth{0.000000pt}%
\definecolor{currentstroke}{rgb}{1.000000,0.894118,0.788235}%
\pgfsetstrokecolor{currentstroke}%
\pgfsetdash{}{0pt}%
\pgfpathmoveto{\pgfqpoint{3.350280in}{3.424899in}}%
\pgfpathlineto{\pgfqpoint{3.239384in}{3.360872in}}%
\pgfpathlineto{\pgfqpoint{3.350280in}{3.296846in}}%
\pgfpathlineto{\pgfqpoint{3.461177in}{3.360872in}}%
\pgfpathlineto{\pgfqpoint{3.350280in}{3.424899in}}%
\pgfpathclose%
\pgfusepath{fill}%
\end{pgfscope}%
\begin{pgfscope}%
\pgfpathrectangle{\pgfqpoint{0.765000in}{0.660000in}}{\pgfqpoint{4.620000in}{4.620000in}}%
\pgfusepath{clip}%
\pgfsetbuttcap%
\pgfsetroundjoin%
\definecolor{currentfill}{rgb}{1.000000,0.894118,0.788235}%
\pgfsetfillcolor{currentfill}%
\pgfsetlinewidth{0.000000pt}%
\definecolor{currentstroke}{rgb}{1.000000,0.894118,0.788235}%
\pgfsetstrokecolor{currentstroke}%
\pgfsetdash{}{0pt}%
\pgfpathmoveto{\pgfqpoint{3.350280in}{3.424899in}}%
\pgfpathlineto{\pgfqpoint{3.239384in}{3.360872in}}%
\pgfpathlineto{\pgfqpoint{3.239384in}{3.488925in}}%
\pgfpathlineto{\pgfqpoint{3.350280in}{3.552951in}}%
\pgfpathlineto{\pgfqpoint{3.350280in}{3.424899in}}%
\pgfpathclose%
\pgfusepath{fill}%
\end{pgfscope}%
\begin{pgfscope}%
\pgfpathrectangle{\pgfqpoint{0.765000in}{0.660000in}}{\pgfqpoint{4.620000in}{4.620000in}}%
\pgfusepath{clip}%
\pgfsetbuttcap%
\pgfsetroundjoin%
\definecolor{currentfill}{rgb}{1.000000,0.894118,0.788235}%
\pgfsetfillcolor{currentfill}%
\pgfsetlinewidth{0.000000pt}%
\definecolor{currentstroke}{rgb}{1.000000,0.894118,0.788235}%
\pgfsetstrokecolor{currentstroke}%
\pgfsetdash{}{0pt}%
\pgfpathmoveto{\pgfqpoint{3.350280in}{3.424899in}}%
\pgfpathlineto{\pgfqpoint{3.461177in}{3.360872in}}%
\pgfpathlineto{\pgfqpoint{3.461177in}{3.488925in}}%
\pgfpathlineto{\pgfqpoint{3.350280in}{3.552951in}}%
\pgfpathlineto{\pgfqpoint{3.350280in}{3.424899in}}%
\pgfpathclose%
\pgfusepath{fill}%
\end{pgfscope}%
\begin{pgfscope}%
\pgfpathrectangle{\pgfqpoint{0.765000in}{0.660000in}}{\pgfqpoint{4.620000in}{4.620000in}}%
\pgfusepath{clip}%
\pgfsetbuttcap%
\pgfsetroundjoin%
\definecolor{currentfill}{rgb}{1.000000,0.894118,0.788235}%
\pgfsetfillcolor{currentfill}%
\pgfsetlinewidth{0.000000pt}%
\definecolor{currentstroke}{rgb}{1.000000,0.894118,0.788235}%
\pgfsetstrokecolor{currentstroke}%
\pgfsetdash{}{0pt}%
\pgfpathmoveto{\pgfqpoint{3.357740in}{3.416522in}}%
\pgfpathlineto{\pgfqpoint{3.246843in}{3.352496in}}%
\pgfpathlineto{\pgfqpoint{3.357740in}{3.288470in}}%
\pgfpathlineto{\pgfqpoint{3.468636in}{3.352496in}}%
\pgfpathlineto{\pgfqpoint{3.357740in}{3.416522in}}%
\pgfpathclose%
\pgfusepath{fill}%
\end{pgfscope}%
\begin{pgfscope}%
\pgfpathrectangle{\pgfqpoint{0.765000in}{0.660000in}}{\pgfqpoint{4.620000in}{4.620000in}}%
\pgfusepath{clip}%
\pgfsetbuttcap%
\pgfsetroundjoin%
\definecolor{currentfill}{rgb}{1.000000,0.894118,0.788235}%
\pgfsetfillcolor{currentfill}%
\pgfsetlinewidth{0.000000pt}%
\definecolor{currentstroke}{rgb}{1.000000,0.894118,0.788235}%
\pgfsetstrokecolor{currentstroke}%
\pgfsetdash{}{0pt}%
\pgfpathmoveto{\pgfqpoint{3.357740in}{3.416522in}}%
\pgfpathlineto{\pgfqpoint{3.246843in}{3.352496in}}%
\pgfpathlineto{\pgfqpoint{3.246843in}{3.480548in}}%
\pgfpathlineto{\pgfqpoint{3.357740in}{3.544575in}}%
\pgfpathlineto{\pgfqpoint{3.357740in}{3.416522in}}%
\pgfpathclose%
\pgfusepath{fill}%
\end{pgfscope}%
\begin{pgfscope}%
\pgfpathrectangle{\pgfqpoint{0.765000in}{0.660000in}}{\pgfqpoint{4.620000in}{4.620000in}}%
\pgfusepath{clip}%
\pgfsetbuttcap%
\pgfsetroundjoin%
\definecolor{currentfill}{rgb}{1.000000,0.894118,0.788235}%
\pgfsetfillcolor{currentfill}%
\pgfsetlinewidth{0.000000pt}%
\definecolor{currentstroke}{rgb}{1.000000,0.894118,0.788235}%
\pgfsetstrokecolor{currentstroke}%
\pgfsetdash{}{0pt}%
\pgfpathmoveto{\pgfqpoint{3.357740in}{3.416522in}}%
\pgfpathlineto{\pgfqpoint{3.468636in}{3.352496in}}%
\pgfpathlineto{\pgfqpoint{3.468636in}{3.480548in}}%
\pgfpathlineto{\pgfqpoint{3.357740in}{3.544575in}}%
\pgfpathlineto{\pgfqpoint{3.357740in}{3.416522in}}%
\pgfpathclose%
\pgfusepath{fill}%
\end{pgfscope}%
\begin{pgfscope}%
\pgfpathrectangle{\pgfqpoint{0.765000in}{0.660000in}}{\pgfqpoint{4.620000in}{4.620000in}}%
\pgfusepath{clip}%
\pgfsetbuttcap%
\pgfsetroundjoin%
\definecolor{currentfill}{rgb}{1.000000,0.894118,0.788235}%
\pgfsetfillcolor{currentfill}%
\pgfsetlinewidth{0.000000pt}%
\definecolor{currentstroke}{rgb}{1.000000,0.894118,0.788235}%
\pgfsetstrokecolor{currentstroke}%
\pgfsetdash{}{0pt}%
\pgfpathmoveto{\pgfqpoint{3.350280in}{3.552951in}}%
\pgfpathlineto{\pgfqpoint{3.239384in}{3.488925in}}%
\pgfpathlineto{\pgfqpoint{3.350280in}{3.424899in}}%
\pgfpathlineto{\pgfqpoint{3.461177in}{3.488925in}}%
\pgfpathlineto{\pgfqpoint{3.350280in}{3.552951in}}%
\pgfpathclose%
\pgfusepath{fill}%
\end{pgfscope}%
\begin{pgfscope}%
\pgfpathrectangle{\pgfqpoint{0.765000in}{0.660000in}}{\pgfqpoint{4.620000in}{4.620000in}}%
\pgfusepath{clip}%
\pgfsetbuttcap%
\pgfsetroundjoin%
\definecolor{currentfill}{rgb}{1.000000,0.894118,0.788235}%
\pgfsetfillcolor{currentfill}%
\pgfsetlinewidth{0.000000pt}%
\definecolor{currentstroke}{rgb}{1.000000,0.894118,0.788235}%
\pgfsetstrokecolor{currentstroke}%
\pgfsetdash{}{0pt}%
\pgfpathmoveto{\pgfqpoint{3.350280in}{3.296846in}}%
\pgfpathlineto{\pgfqpoint{3.461177in}{3.360872in}}%
\pgfpathlineto{\pgfqpoint{3.461177in}{3.488925in}}%
\pgfpathlineto{\pgfqpoint{3.350280in}{3.424899in}}%
\pgfpathlineto{\pgfqpoint{3.350280in}{3.296846in}}%
\pgfpathclose%
\pgfusepath{fill}%
\end{pgfscope}%
\begin{pgfscope}%
\pgfpathrectangle{\pgfqpoint{0.765000in}{0.660000in}}{\pgfqpoint{4.620000in}{4.620000in}}%
\pgfusepath{clip}%
\pgfsetbuttcap%
\pgfsetroundjoin%
\definecolor{currentfill}{rgb}{1.000000,0.894118,0.788235}%
\pgfsetfillcolor{currentfill}%
\pgfsetlinewidth{0.000000pt}%
\definecolor{currentstroke}{rgb}{1.000000,0.894118,0.788235}%
\pgfsetstrokecolor{currentstroke}%
\pgfsetdash{}{0pt}%
\pgfpathmoveto{\pgfqpoint{3.239384in}{3.360872in}}%
\pgfpathlineto{\pgfqpoint{3.350280in}{3.296846in}}%
\pgfpathlineto{\pgfqpoint{3.350280in}{3.424899in}}%
\pgfpathlineto{\pgfqpoint{3.239384in}{3.488925in}}%
\pgfpathlineto{\pgfqpoint{3.239384in}{3.360872in}}%
\pgfpathclose%
\pgfusepath{fill}%
\end{pgfscope}%
\begin{pgfscope}%
\pgfpathrectangle{\pgfqpoint{0.765000in}{0.660000in}}{\pgfqpoint{4.620000in}{4.620000in}}%
\pgfusepath{clip}%
\pgfsetbuttcap%
\pgfsetroundjoin%
\definecolor{currentfill}{rgb}{1.000000,0.894118,0.788235}%
\pgfsetfillcolor{currentfill}%
\pgfsetlinewidth{0.000000pt}%
\definecolor{currentstroke}{rgb}{1.000000,0.894118,0.788235}%
\pgfsetstrokecolor{currentstroke}%
\pgfsetdash{}{0pt}%
\pgfpathmoveto{\pgfqpoint{3.357740in}{3.544575in}}%
\pgfpathlineto{\pgfqpoint{3.246843in}{3.480548in}}%
\pgfpathlineto{\pgfqpoint{3.357740in}{3.416522in}}%
\pgfpathlineto{\pgfqpoint{3.468636in}{3.480548in}}%
\pgfpathlineto{\pgfqpoint{3.357740in}{3.544575in}}%
\pgfpathclose%
\pgfusepath{fill}%
\end{pgfscope}%
\begin{pgfscope}%
\pgfpathrectangle{\pgfqpoint{0.765000in}{0.660000in}}{\pgfqpoint{4.620000in}{4.620000in}}%
\pgfusepath{clip}%
\pgfsetbuttcap%
\pgfsetroundjoin%
\definecolor{currentfill}{rgb}{1.000000,0.894118,0.788235}%
\pgfsetfillcolor{currentfill}%
\pgfsetlinewidth{0.000000pt}%
\definecolor{currentstroke}{rgb}{1.000000,0.894118,0.788235}%
\pgfsetstrokecolor{currentstroke}%
\pgfsetdash{}{0pt}%
\pgfpathmoveto{\pgfqpoint{3.357740in}{3.288470in}}%
\pgfpathlineto{\pgfqpoint{3.468636in}{3.352496in}}%
\pgfpathlineto{\pgfqpoint{3.468636in}{3.480548in}}%
\pgfpathlineto{\pgfqpoint{3.357740in}{3.416522in}}%
\pgfpathlineto{\pgfqpoint{3.357740in}{3.288470in}}%
\pgfpathclose%
\pgfusepath{fill}%
\end{pgfscope}%
\begin{pgfscope}%
\pgfpathrectangle{\pgfqpoint{0.765000in}{0.660000in}}{\pgfqpoint{4.620000in}{4.620000in}}%
\pgfusepath{clip}%
\pgfsetbuttcap%
\pgfsetroundjoin%
\definecolor{currentfill}{rgb}{1.000000,0.894118,0.788235}%
\pgfsetfillcolor{currentfill}%
\pgfsetlinewidth{0.000000pt}%
\definecolor{currentstroke}{rgb}{1.000000,0.894118,0.788235}%
\pgfsetstrokecolor{currentstroke}%
\pgfsetdash{}{0pt}%
\pgfpathmoveto{\pgfqpoint{3.246843in}{3.352496in}}%
\pgfpathlineto{\pgfqpoint{3.357740in}{3.288470in}}%
\pgfpathlineto{\pgfqpoint{3.357740in}{3.416522in}}%
\pgfpathlineto{\pgfqpoint{3.246843in}{3.480548in}}%
\pgfpathlineto{\pgfqpoint{3.246843in}{3.352496in}}%
\pgfpathclose%
\pgfusepath{fill}%
\end{pgfscope}%
\begin{pgfscope}%
\pgfpathrectangle{\pgfqpoint{0.765000in}{0.660000in}}{\pgfqpoint{4.620000in}{4.620000in}}%
\pgfusepath{clip}%
\pgfsetbuttcap%
\pgfsetroundjoin%
\definecolor{currentfill}{rgb}{1.000000,0.894118,0.788235}%
\pgfsetfillcolor{currentfill}%
\pgfsetlinewidth{0.000000pt}%
\definecolor{currentstroke}{rgb}{1.000000,0.894118,0.788235}%
\pgfsetstrokecolor{currentstroke}%
\pgfsetdash{}{0pt}%
\pgfpathmoveto{\pgfqpoint{3.350280in}{3.424899in}}%
\pgfpathlineto{\pgfqpoint{3.239384in}{3.360872in}}%
\pgfpathlineto{\pgfqpoint{3.246843in}{3.352496in}}%
\pgfpathlineto{\pgfqpoint{3.357740in}{3.416522in}}%
\pgfpathlineto{\pgfqpoint{3.350280in}{3.424899in}}%
\pgfpathclose%
\pgfusepath{fill}%
\end{pgfscope}%
\begin{pgfscope}%
\pgfpathrectangle{\pgfqpoint{0.765000in}{0.660000in}}{\pgfqpoint{4.620000in}{4.620000in}}%
\pgfusepath{clip}%
\pgfsetbuttcap%
\pgfsetroundjoin%
\definecolor{currentfill}{rgb}{1.000000,0.894118,0.788235}%
\pgfsetfillcolor{currentfill}%
\pgfsetlinewidth{0.000000pt}%
\definecolor{currentstroke}{rgb}{1.000000,0.894118,0.788235}%
\pgfsetstrokecolor{currentstroke}%
\pgfsetdash{}{0pt}%
\pgfpathmoveto{\pgfqpoint{3.461177in}{3.360872in}}%
\pgfpathlineto{\pgfqpoint{3.350280in}{3.424899in}}%
\pgfpathlineto{\pgfqpoint{3.357740in}{3.416522in}}%
\pgfpathlineto{\pgfqpoint{3.468636in}{3.352496in}}%
\pgfpathlineto{\pgfqpoint{3.461177in}{3.360872in}}%
\pgfpathclose%
\pgfusepath{fill}%
\end{pgfscope}%
\begin{pgfscope}%
\pgfpathrectangle{\pgfqpoint{0.765000in}{0.660000in}}{\pgfqpoint{4.620000in}{4.620000in}}%
\pgfusepath{clip}%
\pgfsetbuttcap%
\pgfsetroundjoin%
\definecolor{currentfill}{rgb}{1.000000,0.894118,0.788235}%
\pgfsetfillcolor{currentfill}%
\pgfsetlinewidth{0.000000pt}%
\definecolor{currentstroke}{rgb}{1.000000,0.894118,0.788235}%
\pgfsetstrokecolor{currentstroke}%
\pgfsetdash{}{0pt}%
\pgfpathmoveto{\pgfqpoint{3.350280in}{3.424899in}}%
\pgfpathlineto{\pgfqpoint{3.350280in}{3.552951in}}%
\pgfpathlineto{\pgfqpoint{3.357740in}{3.544575in}}%
\pgfpathlineto{\pgfqpoint{3.468636in}{3.352496in}}%
\pgfpathlineto{\pgfqpoint{3.350280in}{3.424899in}}%
\pgfpathclose%
\pgfusepath{fill}%
\end{pgfscope}%
\begin{pgfscope}%
\pgfpathrectangle{\pgfqpoint{0.765000in}{0.660000in}}{\pgfqpoint{4.620000in}{4.620000in}}%
\pgfusepath{clip}%
\pgfsetbuttcap%
\pgfsetroundjoin%
\definecolor{currentfill}{rgb}{1.000000,0.894118,0.788235}%
\pgfsetfillcolor{currentfill}%
\pgfsetlinewidth{0.000000pt}%
\definecolor{currentstroke}{rgb}{1.000000,0.894118,0.788235}%
\pgfsetstrokecolor{currentstroke}%
\pgfsetdash{}{0pt}%
\pgfpathmoveto{\pgfqpoint{3.461177in}{3.360872in}}%
\pgfpathlineto{\pgfqpoint{3.461177in}{3.488925in}}%
\pgfpathlineto{\pgfqpoint{3.468636in}{3.480548in}}%
\pgfpathlineto{\pgfqpoint{3.357740in}{3.416522in}}%
\pgfpathlineto{\pgfqpoint{3.461177in}{3.360872in}}%
\pgfpathclose%
\pgfusepath{fill}%
\end{pgfscope}%
\begin{pgfscope}%
\pgfpathrectangle{\pgfqpoint{0.765000in}{0.660000in}}{\pgfqpoint{4.620000in}{4.620000in}}%
\pgfusepath{clip}%
\pgfsetbuttcap%
\pgfsetroundjoin%
\definecolor{currentfill}{rgb}{1.000000,0.894118,0.788235}%
\pgfsetfillcolor{currentfill}%
\pgfsetlinewidth{0.000000pt}%
\definecolor{currentstroke}{rgb}{1.000000,0.894118,0.788235}%
\pgfsetstrokecolor{currentstroke}%
\pgfsetdash{}{0pt}%
\pgfpathmoveto{\pgfqpoint{3.239384in}{3.360872in}}%
\pgfpathlineto{\pgfqpoint{3.350280in}{3.296846in}}%
\pgfpathlineto{\pgfqpoint{3.357740in}{3.288470in}}%
\pgfpathlineto{\pgfqpoint{3.246843in}{3.352496in}}%
\pgfpathlineto{\pgfqpoint{3.239384in}{3.360872in}}%
\pgfpathclose%
\pgfusepath{fill}%
\end{pgfscope}%
\begin{pgfscope}%
\pgfpathrectangle{\pgfqpoint{0.765000in}{0.660000in}}{\pgfqpoint{4.620000in}{4.620000in}}%
\pgfusepath{clip}%
\pgfsetbuttcap%
\pgfsetroundjoin%
\definecolor{currentfill}{rgb}{1.000000,0.894118,0.788235}%
\pgfsetfillcolor{currentfill}%
\pgfsetlinewidth{0.000000pt}%
\definecolor{currentstroke}{rgb}{1.000000,0.894118,0.788235}%
\pgfsetstrokecolor{currentstroke}%
\pgfsetdash{}{0pt}%
\pgfpathmoveto{\pgfqpoint{3.350280in}{3.296846in}}%
\pgfpathlineto{\pgfqpoint{3.461177in}{3.360872in}}%
\pgfpathlineto{\pgfqpoint{3.468636in}{3.352496in}}%
\pgfpathlineto{\pgfqpoint{3.357740in}{3.288470in}}%
\pgfpathlineto{\pgfqpoint{3.350280in}{3.296846in}}%
\pgfpathclose%
\pgfusepath{fill}%
\end{pgfscope}%
\begin{pgfscope}%
\pgfpathrectangle{\pgfqpoint{0.765000in}{0.660000in}}{\pgfqpoint{4.620000in}{4.620000in}}%
\pgfusepath{clip}%
\pgfsetbuttcap%
\pgfsetroundjoin%
\definecolor{currentfill}{rgb}{1.000000,0.894118,0.788235}%
\pgfsetfillcolor{currentfill}%
\pgfsetlinewidth{0.000000pt}%
\definecolor{currentstroke}{rgb}{1.000000,0.894118,0.788235}%
\pgfsetstrokecolor{currentstroke}%
\pgfsetdash{}{0pt}%
\pgfpathmoveto{\pgfqpoint{3.350280in}{3.552951in}}%
\pgfpathlineto{\pgfqpoint{3.239384in}{3.488925in}}%
\pgfpathlineto{\pgfqpoint{3.246843in}{3.480548in}}%
\pgfpathlineto{\pgfqpoint{3.357740in}{3.544575in}}%
\pgfpathlineto{\pgfqpoint{3.350280in}{3.552951in}}%
\pgfpathclose%
\pgfusepath{fill}%
\end{pgfscope}%
\begin{pgfscope}%
\pgfpathrectangle{\pgfqpoint{0.765000in}{0.660000in}}{\pgfqpoint{4.620000in}{4.620000in}}%
\pgfusepath{clip}%
\pgfsetbuttcap%
\pgfsetroundjoin%
\definecolor{currentfill}{rgb}{1.000000,0.894118,0.788235}%
\pgfsetfillcolor{currentfill}%
\pgfsetlinewidth{0.000000pt}%
\definecolor{currentstroke}{rgb}{1.000000,0.894118,0.788235}%
\pgfsetstrokecolor{currentstroke}%
\pgfsetdash{}{0pt}%
\pgfpathmoveto{\pgfqpoint{3.461177in}{3.488925in}}%
\pgfpathlineto{\pgfqpoint{3.350280in}{3.552951in}}%
\pgfpathlineto{\pgfqpoint{3.357740in}{3.544575in}}%
\pgfpathlineto{\pgfqpoint{3.468636in}{3.480548in}}%
\pgfpathlineto{\pgfqpoint{3.461177in}{3.488925in}}%
\pgfpathclose%
\pgfusepath{fill}%
\end{pgfscope}%
\begin{pgfscope}%
\pgfpathrectangle{\pgfqpoint{0.765000in}{0.660000in}}{\pgfqpoint{4.620000in}{4.620000in}}%
\pgfusepath{clip}%
\pgfsetbuttcap%
\pgfsetroundjoin%
\definecolor{currentfill}{rgb}{1.000000,0.894118,0.788235}%
\pgfsetfillcolor{currentfill}%
\pgfsetlinewidth{0.000000pt}%
\definecolor{currentstroke}{rgb}{1.000000,0.894118,0.788235}%
\pgfsetstrokecolor{currentstroke}%
\pgfsetdash{}{0pt}%
\pgfpathmoveto{\pgfqpoint{3.239384in}{3.360872in}}%
\pgfpathlineto{\pgfqpoint{3.239384in}{3.488925in}}%
\pgfpathlineto{\pgfqpoint{3.246843in}{3.480548in}}%
\pgfpathlineto{\pgfqpoint{3.357740in}{3.288470in}}%
\pgfpathlineto{\pgfqpoint{3.239384in}{3.360872in}}%
\pgfpathclose%
\pgfusepath{fill}%
\end{pgfscope}%
\begin{pgfscope}%
\pgfpathrectangle{\pgfqpoint{0.765000in}{0.660000in}}{\pgfqpoint{4.620000in}{4.620000in}}%
\pgfusepath{clip}%
\pgfsetbuttcap%
\pgfsetroundjoin%
\definecolor{currentfill}{rgb}{1.000000,0.894118,0.788235}%
\pgfsetfillcolor{currentfill}%
\pgfsetlinewidth{0.000000pt}%
\definecolor{currentstroke}{rgb}{1.000000,0.894118,0.788235}%
\pgfsetstrokecolor{currentstroke}%
\pgfsetdash{}{0pt}%
\pgfpathmoveto{\pgfqpoint{3.350280in}{3.296846in}}%
\pgfpathlineto{\pgfqpoint{3.350280in}{3.424899in}}%
\pgfpathlineto{\pgfqpoint{3.357740in}{3.416522in}}%
\pgfpathlineto{\pgfqpoint{3.246843in}{3.352496in}}%
\pgfpathlineto{\pgfqpoint{3.350280in}{3.296846in}}%
\pgfpathclose%
\pgfusepath{fill}%
\end{pgfscope}%
\begin{pgfscope}%
\pgfpathrectangle{\pgfqpoint{0.765000in}{0.660000in}}{\pgfqpoint{4.620000in}{4.620000in}}%
\pgfusepath{clip}%
\pgfsetbuttcap%
\pgfsetroundjoin%
\definecolor{currentfill}{rgb}{1.000000,0.894118,0.788235}%
\pgfsetfillcolor{currentfill}%
\pgfsetlinewidth{0.000000pt}%
\definecolor{currentstroke}{rgb}{1.000000,0.894118,0.788235}%
\pgfsetstrokecolor{currentstroke}%
\pgfsetdash{}{0pt}%
\pgfpathmoveto{\pgfqpoint{3.350280in}{3.424899in}}%
\pgfpathlineto{\pgfqpoint{3.461177in}{3.488925in}}%
\pgfpathlineto{\pgfqpoint{3.468636in}{3.480548in}}%
\pgfpathlineto{\pgfqpoint{3.357740in}{3.416522in}}%
\pgfpathlineto{\pgfqpoint{3.350280in}{3.424899in}}%
\pgfpathclose%
\pgfusepath{fill}%
\end{pgfscope}%
\begin{pgfscope}%
\pgfpathrectangle{\pgfqpoint{0.765000in}{0.660000in}}{\pgfqpoint{4.620000in}{4.620000in}}%
\pgfusepath{clip}%
\pgfsetbuttcap%
\pgfsetroundjoin%
\definecolor{currentfill}{rgb}{1.000000,0.894118,0.788235}%
\pgfsetfillcolor{currentfill}%
\pgfsetlinewidth{0.000000pt}%
\definecolor{currentstroke}{rgb}{1.000000,0.894118,0.788235}%
\pgfsetstrokecolor{currentstroke}%
\pgfsetdash{}{0pt}%
\pgfpathmoveto{\pgfqpoint{3.239384in}{3.488925in}}%
\pgfpathlineto{\pgfqpoint{3.350280in}{3.424899in}}%
\pgfpathlineto{\pgfqpoint{3.357740in}{3.416522in}}%
\pgfpathlineto{\pgfqpoint{3.246843in}{3.480548in}}%
\pgfpathlineto{\pgfqpoint{3.239384in}{3.488925in}}%
\pgfpathclose%
\pgfusepath{fill}%
\end{pgfscope}%
\begin{pgfscope}%
\pgfpathrectangle{\pgfqpoint{0.765000in}{0.660000in}}{\pgfqpoint{4.620000in}{4.620000in}}%
\pgfusepath{clip}%
\pgfsetbuttcap%
\pgfsetroundjoin%
\definecolor{currentfill}{rgb}{1.000000,0.894118,0.788235}%
\pgfsetfillcolor{currentfill}%
\pgfsetlinewidth{0.000000pt}%
\definecolor{currentstroke}{rgb}{1.000000,0.894118,0.788235}%
\pgfsetstrokecolor{currentstroke}%
\pgfsetdash{}{0pt}%
\pgfpathmoveto{\pgfqpoint{3.350280in}{3.424899in}}%
\pgfpathlineto{\pgfqpoint{3.239384in}{3.360872in}}%
\pgfpathlineto{\pgfqpoint{3.350280in}{3.296846in}}%
\pgfpathlineto{\pgfqpoint{3.461177in}{3.360872in}}%
\pgfpathlineto{\pgfqpoint{3.350280in}{3.424899in}}%
\pgfpathclose%
\pgfusepath{fill}%
\end{pgfscope}%
\begin{pgfscope}%
\pgfpathrectangle{\pgfqpoint{0.765000in}{0.660000in}}{\pgfqpoint{4.620000in}{4.620000in}}%
\pgfusepath{clip}%
\pgfsetbuttcap%
\pgfsetroundjoin%
\definecolor{currentfill}{rgb}{1.000000,0.894118,0.788235}%
\pgfsetfillcolor{currentfill}%
\pgfsetlinewidth{0.000000pt}%
\definecolor{currentstroke}{rgb}{1.000000,0.894118,0.788235}%
\pgfsetstrokecolor{currentstroke}%
\pgfsetdash{}{0pt}%
\pgfpathmoveto{\pgfqpoint{3.350280in}{3.424899in}}%
\pgfpathlineto{\pgfqpoint{3.239384in}{3.360872in}}%
\pgfpathlineto{\pgfqpoint{3.239384in}{3.488925in}}%
\pgfpathlineto{\pgfqpoint{3.350280in}{3.552951in}}%
\pgfpathlineto{\pgfqpoint{3.350280in}{3.424899in}}%
\pgfpathclose%
\pgfusepath{fill}%
\end{pgfscope}%
\begin{pgfscope}%
\pgfpathrectangle{\pgfqpoint{0.765000in}{0.660000in}}{\pgfqpoint{4.620000in}{4.620000in}}%
\pgfusepath{clip}%
\pgfsetbuttcap%
\pgfsetroundjoin%
\definecolor{currentfill}{rgb}{1.000000,0.894118,0.788235}%
\pgfsetfillcolor{currentfill}%
\pgfsetlinewidth{0.000000pt}%
\definecolor{currentstroke}{rgb}{1.000000,0.894118,0.788235}%
\pgfsetstrokecolor{currentstroke}%
\pgfsetdash{}{0pt}%
\pgfpathmoveto{\pgfqpoint{3.350280in}{3.424899in}}%
\pgfpathlineto{\pgfqpoint{3.461177in}{3.360872in}}%
\pgfpathlineto{\pgfqpoint{3.461177in}{3.488925in}}%
\pgfpathlineto{\pgfqpoint{3.350280in}{3.552951in}}%
\pgfpathlineto{\pgfqpoint{3.350280in}{3.424899in}}%
\pgfpathclose%
\pgfusepath{fill}%
\end{pgfscope}%
\begin{pgfscope}%
\pgfpathrectangle{\pgfqpoint{0.765000in}{0.660000in}}{\pgfqpoint{4.620000in}{4.620000in}}%
\pgfusepath{clip}%
\pgfsetbuttcap%
\pgfsetroundjoin%
\definecolor{currentfill}{rgb}{1.000000,0.894118,0.788235}%
\pgfsetfillcolor{currentfill}%
\pgfsetlinewidth{0.000000pt}%
\definecolor{currentstroke}{rgb}{1.000000,0.894118,0.788235}%
\pgfsetstrokecolor{currentstroke}%
\pgfsetdash{}{0pt}%
\pgfpathmoveto{\pgfqpoint{3.334893in}{3.441865in}}%
\pgfpathlineto{\pgfqpoint{3.223996in}{3.377839in}}%
\pgfpathlineto{\pgfqpoint{3.334893in}{3.313813in}}%
\pgfpathlineto{\pgfqpoint{3.445790in}{3.377839in}}%
\pgfpathlineto{\pgfqpoint{3.334893in}{3.441865in}}%
\pgfpathclose%
\pgfusepath{fill}%
\end{pgfscope}%
\begin{pgfscope}%
\pgfpathrectangle{\pgfqpoint{0.765000in}{0.660000in}}{\pgfqpoint{4.620000in}{4.620000in}}%
\pgfusepath{clip}%
\pgfsetbuttcap%
\pgfsetroundjoin%
\definecolor{currentfill}{rgb}{1.000000,0.894118,0.788235}%
\pgfsetfillcolor{currentfill}%
\pgfsetlinewidth{0.000000pt}%
\definecolor{currentstroke}{rgb}{1.000000,0.894118,0.788235}%
\pgfsetstrokecolor{currentstroke}%
\pgfsetdash{}{0pt}%
\pgfpathmoveto{\pgfqpoint{3.334893in}{3.441865in}}%
\pgfpathlineto{\pgfqpoint{3.223996in}{3.377839in}}%
\pgfpathlineto{\pgfqpoint{3.223996in}{3.505892in}}%
\pgfpathlineto{\pgfqpoint{3.334893in}{3.569918in}}%
\pgfpathlineto{\pgfqpoint{3.334893in}{3.441865in}}%
\pgfpathclose%
\pgfusepath{fill}%
\end{pgfscope}%
\begin{pgfscope}%
\pgfpathrectangle{\pgfqpoint{0.765000in}{0.660000in}}{\pgfqpoint{4.620000in}{4.620000in}}%
\pgfusepath{clip}%
\pgfsetbuttcap%
\pgfsetroundjoin%
\definecolor{currentfill}{rgb}{1.000000,0.894118,0.788235}%
\pgfsetfillcolor{currentfill}%
\pgfsetlinewidth{0.000000pt}%
\definecolor{currentstroke}{rgb}{1.000000,0.894118,0.788235}%
\pgfsetstrokecolor{currentstroke}%
\pgfsetdash{}{0pt}%
\pgfpathmoveto{\pgfqpoint{3.334893in}{3.441865in}}%
\pgfpathlineto{\pgfqpoint{3.445790in}{3.377839in}}%
\pgfpathlineto{\pgfqpoint{3.445790in}{3.505892in}}%
\pgfpathlineto{\pgfqpoint{3.334893in}{3.569918in}}%
\pgfpathlineto{\pgfqpoint{3.334893in}{3.441865in}}%
\pgfpathclose%
\pgfusepath{fill}%
\end{pgfscope}%
\begin{pgfscope}%
\pgfpathrectangle{\pgfqpoint{0.765000in}{0.660000in}}{\pgfqpoint{4.620000in}{4.620000in}}%
\pgfusepath{clip}%
\pgfsetbuttcap%
\pgfsetroundjoin%
\definecolor{currentfill}{rgb}{1.000000,0.894118,0.788235}%
\pgfsetfillcolor{currentfill}%
\pgfsetlinewidth{0.000000pt}%
\definecolor{currentstroke}{rgb}{1.000000,0.894118,0.788235}%
\pgfsetstrokecolor{currentstroke}%
\pgfsetdash{}{0pt}%
\pgfpathmoveto{\pgfqpoint{3.350280in}{3.552951in}}%
\pgfpathlineto{\pgfqpoint{3.239384in}{3.488925in}}%
\pgfpathlineto{\pgfqpoint{3.350280in}{3.424899in}}%
\pgfpathlineto{\pgfqpoint{3.461177in}{3.488925in}}%
\pgfpathlineto{\pgfqpoint{3.350280in}{3.552951in}}%
\pgfpathclose%
\pgfusepath{fill}%
\end{pgfscope}%
\begin{pgfscope}%
\pgfpathrectangle{\pgfqpoint{0.765000in}{0.660000in}}{\pgfqpoint{4.620000in}{4.620000in}}%
\pgfusepath{clip}%
\pgfsetbuttcap%
\pgfsetroundjoin%
\definecolor{currentfill}{rgb}{1.000000,0.894118,0.788235}%
\pgfsetfillcolor{currentfill}%
\pgfsetlinewidth{0.000000pt}%
\definecolor{currentstroke}{rgb}{1.000000,0.894118,0.788235}%
\pgfsetstrokecolor{currentstroke}%
\pgfsetdash{}{0pt}%
\pgfpathmoveto{\pgfqpoint{3.350280in}{3.296846in}}%
\pgfpathlineto{\pgfqpoint{3.461177in}{3.360872in}}%
\pgfpathlineto{\pgfqpoint{3.461177in}{3.488925in}}%
\pgfpathlineto{\pgfqpoint{3.350280in}{3.424899in}}%
\pgfpathlineto{\pgfqpoint{3.350280in}{3.296846in}}%
\pgfpathclose%
\pgfusepath{fill}%
\end{pgfscope}%
\begin{pgfscope}%
\pgfpathrectangle{\pgfqpoint{0.765000in}{0.660000in}}{\pgfqpoint{4.620000in}{4.620000in}}%
\pgfusepath{clip}%
\pgfsetbuttcap%
\pgfsetroundjoin%
\definecolor{currentfill}{rgb}{1.000000,0.894118,0.788235}%
\pgfsetfillcolor{currentfill}%
\pgfsetlinewidth{0.000000pt}%
\definecolor{currentstroke}{rgb}{1.000000,0.894118,0.788235}%
\pgfsetstrokecolor{currentstroke}%
\pgfsetdash{}{0pt}%
\pgfpathmoveto{\pgfqpoint{3.239384in}{3.360872in}}%
\pgfpathlineto{\pgfqpoint{3.350280in}{3.296846in}}%
\pgfpathlineto{\pgfqpoint{3.350280in}{3.424899in}}%
\pgfpathlineto{\pgfqpoint{3.239384in}{3.488925in}}%
\pgfpathlineto{\pgfqpoint{3.239384in}{3.360872in}}%
\pgfpathclose%
\pgfusepath{fill}%
\end{pgfscope}%
\begin{pgfscope}%
\pgfpathrectangle{\pgfqpoint{0.765000in}{0.660000in}}{\pgfqpoint{4.620000in}{4.620000in}}%
\pgfusepath{clip}%
\pgfsetbuttcap%
\pgfsetroundjoin%
\definecolor{currentfill}{rgb}{1.000000,0.894118,0.788235}%
\pgfsetfillcolor{currentfill}%
\pgfsetlinewidth{0.000000pt}%
\definecolor{currentstroke}{rgb}{1.000000,0.894118,0.788235}%
\pgfsetstrokecolor{currentstroke}%
\pgfsetdash{}{0pt}%
\pgfpathmoveto{\pgfqpoint{3.334893in}{3.569918in}}%
\pgfpathlineto{\pgfqpoint{3.223996in}{3.505892in}}%
\pgfpathlineto{\pgfqpoint{3.334893in}{3.441865in}}%
\pgfpathlineto{\pgfqpoint{3.445790in}{3.505892in}}%
\pgfpathlineto{\pgfqpoint{3.334893in}{3.569918in}}%
\pgfpathclose%
\pgfusepath{fill}%
\end{pgfscope}%
\begin{pgfscope}%
\pgfpathrectangle{\pgfqpoint{0.765000in}{0.660000in}}{\pgfqpoint{4.620000in}{4.620000in}}%
\pgfusepath{clip}%
\pgfsetbuttcap%
\pgfsetroundjoin%
\definecolor{currentfill}{rgb}{1.000000,0.894118,0.788235}%
\pgfsetfillcolor{currentfill}%
\pgfsetlinewidth{0.000000pt}%
\definecolor{currentstroke}{rgb}{1.000000,0.894118,0.788235}%
\pgfsetstrokecolor{currentstroke}%
\pgfsetdash{}{0pt}%
\pgfpathmoveto{\pgfqpoint{3.334893in}{3.313813in}}%
\pgfpathlineto{\pgfqpoint{3.445790in}{3.377839in}}%
\pgfpathlineto{\pgfqpoint{3.445790in}{3.505892in}}%
\pgfpathlineto{\pgfqpoint{3.334893in}{3.441865in}}%
\pgfpathlineto{\pgfqpoint{3.334893in}{3.313813in}}%
\pgfpathclose%
\pgfusepath{fill}%
\end{pgfscope}%
\begin{pgfscope}%
\pgfpathrectangle{\pgfqpoint{0.765000in}{0.660000in}}{\pgfqpoint{4.620000in}{4.620000in}}%
\pgfusepath{clip}%
\pgfsetbuttcap%
\pgfsetroundjoin%
\definecolor{currentfill}{rgb}{1.000000,0.894118,0.788235}%
\pgfsetfillcolor{currentfill}%
\pgfsetlinewidth{0.000000pt}%
\definecolor{currentstroke}{rgb}{1.000000,0.894118,0.788235}%
\pgfsetstrokecolor{currentstroke}%
\pgfsetdash{}{0pt}%
\pgfpathmoveto{\pgfqpoint{3.223996in}{3.377839in}}%
\pgfpathlineto{\pgfqpoint{3.334893in}{3.313813in}}%
\pgfpathlineto{\pgfqpoint{3.334893in}{3.441865in}}%
\pgfpathlineto{\pgfqpoint{3.223996in}{3.505892in}}%
\pgfpathlineto{\pgfqpoint{3.223996in}{3.377839in}}%
\pgfpathclose%
\pgfusepath{fill}%
\end{pgfscope}%
\begin{pgfscope}%
\pgfpathrectangle{\pgfqpoint{0.765000in}{0.660000in}}{\pgfqpoint{4.620000in}{4.620000in}}%
\pgfusepath{clip}%
\pgfsetbuttcap%
\pgfsetroundjoin%
\definecolor{currentfill}{rgb}{1.000000,0.894118,0.788235}%
\pgfsetfillcolor{currentfill}%
\pgfsetlinewidth{0.000000pt}%
\definecolor{currentstroke}{rgb}{1.000000,0.894118,0.788235}%
\pgfsetstrokecolor{currentstroke}%
\pgfsetdash{}{0pt}%
\pgfpathmoveto{\pgfqpoint{3.350280in}{3.424899in}}%
\pgfpathlineto{\pgfqpoint{3.239384in}{3.360872in}}%
\pgfpathlineto{\pgfqpoint{3.223996in}{3.377839in}}%
\pgfpathlineto{\pgfqpoint{3.334893in}{3.441865in}}%
\pgfpathlineto{\pgfqpoint{3.350280in}{3.424899in}}%
\pgfpathclose%
\pgfusepath{fill}%
\end{pgfscope}%
\begin{pgfscope}%
\pgfpathrectangle{\pgfqpoint{0.765000in}{0.660000in}}{\pgfqpoint{4.620000in}{4.620000in}}%
\pgfusepath{clip}%
\pgfsetbuttcap%
\pgfsetroundjoin%
\definecolor{currentfill}{rgb}{1.000000,0.894118,0.788235}%
\pgfsetfillcolor{currentfill}%
\pgfsetlinewidth{0.000000pt}%
\definecolor{currentstroke}{rgb}{1.000000,0.894118,0.788235}%
\pgfsetstrokecolor{currentstroke}%
\pgfsetdash{}{0pt}%
\pgfpathmoveto{\pgfqpoint{3.461177in}{3.360872in}}%
\pgfpathlineto{\pgfqpoint{3.350280in}{3.424899in}}%
\pgfpathlineto{\pgfqpoint{3.334893in}{3.441865in}}%
\pgfpathlineto{\pgfqpoint{3.445790in}{3.377839in}}%
\pgfpathlineto{\pgfqpoint{3.461177in}{3.360872in}}%
\pgfpathclose%
\pgfusepath{fill}%
\end{pgfscope}%
\begin{pgfscope}%
\pgfpathrectangle{\pgfqpoint{0.765000in}{0.660000in}}{\pgfqpoint{4.620000in}{4.620000in}}%
\pgfusepath{clip}%
\pgfsetbuttcap%
\pgfsetroundjoin%
\definecolor{currentfill}{rgb}{1.000000,0.894118,0.788235}%
\pgfsetfillcolor{currentfill}%
\pgfsetlinewidth{0.000000pt}%
\definecolor{currentstroke}{rgb}{1.000000,0.894118,0.788235}%
\pgfsetstrokecolor{currentstroke}%
\pgfsetdash{}{0pt}%
\pgfpathmoveto{\pgfqpoint{3.350280in}{3.424899in}}%
\pgfpathlineto{\pgfqpoint{3.350280in}{3.552951in}}%
\pgfpathlineto{\pgfqpoint{3.334893in}{3.569918in}}%
\pgfpathlineto{\pgfqpoint{3.445790in}{3.377839in}}%
\pgfpathlineto{\pgfqpoint{3.350280in}{3.424899in}}%
\pgfpathclose%
\pgfusepath{fill}%
\end{pgfscope}%
\begin{pgfscope}%
\pgfpathrectangle{\pgfqpoint{0.765000in}{0.660000in}}{\pgfqpoint{4.620000in}{4.620000in}}%
\pgfusepath{clip}%
\pgfsetbuttcap%
\pgfsetroundjoin%
\definecolor{currentfill}{rgb}{1.000000,0.894118,0.788235}%
\pgfsetfillcolor{currentfill}%
\pgfsetlinewidth{0.000000pt}%
\definecolor{currentstroke}{rgb}{1.000000,0.894118,0.788235}%
\pgfsetstrokecolor{currentstroke}%
\pgfsetdash{}{0pt}%
\pgfpathmoveto{\pgfqpoint{3.461177in}{3.360872in}}%
\pgfpathlineto{\pgfqpoint{3.461177in}{3.488925in}}%
\pgfpathlineto{\pgfqpoint{3.445790in}{3.505892in}}%
\pgfpathlineto{\pgfqpoint{3.334893in}{3.441865in}}%
\pgfpathlineto{\pgfqpoint{3.461177in}{3.360872in}}%
\pgfpathclose%
\pgfusepath{fill}%
\end{pgfscope}%
\begin{pgfscope}%
\pgfpathrectangle{\pgfqpoint{0.765000in}{0.660000in}}{\pgfqpoint{4.620000in}{4.620000in}}%
\pgfusepath{clip}%
\pgfsetbuttcap%
\pgfsetroundjoin%
\definecolor{currentfill}{rgb}{1.000000,0.894118,0.788235}%
\pgfsetfillcolor{currentfill}%
\pgfsetlinewidth{0.000000pt}%
\definecolor{currentstroke}{rgb}{1.000000,0.894118,0.788235}%
\pgfsetstrokecolor{currentstroke}%
\pgfsetdash{}{0pt}%
\pgfpathmoveto{\pgfqpoint{3.239384in}{3.360872in}}%
\pgfpathlineto{\pgfqpoint{3.350280in}{3.296846in}}%
\pgfpathlineto{\pgfqpoint{3.334893in}{3.313813in}}%
\pgfpathlineto{\pgfqpoint{3.223996in}{3.377839in}}%
\pgfpathlineto{\pgfqpoint{3.239384in}{3.360872in}}%
\pgfpathclose%
\pgfusepath{fill}%
\end{pgfscope}%
\begin{pgfscope}%
\pgfpathrectangle{\pgfqpoint{0.765000in}{0.660000in}}{\pgfqpoint{4.620000in}{4.620000in}}%
\pgfusepath{clip}%
\pgfsetbuttcap%
\pgfsetroundjoin%
\definecolor{currentfill}{rgb}{1.000000,0.894118,0.788235}%
\pgfsetfillcolor{currentfill}%
\pgfsetlinewidth{0.000000pt}%
\definecolor{currentstroke}{rgb}{1.000000,0.894118,0.788235}%
\pgfsetstrokecolor{currentstroke}%
\pgfsetdash{}{0pt}%
\pgfpathmoveto{\pgfqpoint{3.350280in}{3.296846in}}%
\pgfpathlineto{\pgfqpoint{3.461177in}{3.360872in}}%
\pgfpathlineto{\pgfqpoint{3.445790in}{3.377839in}}%
\pgfpathlineto{\pgfqpoint{3.334893in}{3.313813in}}%
\pgfpathlineto{\pgfqpoint{3.350280in}{3.296846in}}%
\pgfpathclose%
\pgfusepath{fill}%
\end{pgfscope}%
\begin{pgfscope}%
\pgfpathrectangle{\pgfqpoint{0.765000in}{0.660000in}}{\pgfqpoint{4.620000in}{4.620000in}}%
\pgfusepath{clip}%
\pgfsetbuttcap%
\pgfsetroundjoin%
\definecolor{currentfill}{rgb}{1.000000,0.894118,0.788235}%
\pgfsetfillcolor{currentfill}%
\pgfsetlinewidth{0.000000pt}%
\definecolor{currentstroke}{rgb}{1.000000,0.894118,0.788235}%
\pgfsetstrokecolor{currentstroke}%
\pgfsetdash{}{0pt}%
\pgfpathmoveto{\pgfqpoint{3.350280in}{3.552951in}}%
\pgfpathlineto{\pgfqpoint{3.239384in}{3.488925in}}%
\pgfpathlineto{\pgfqpoint{3.223996in}{3.505892in}}%
\pgfpathlineto{\pgfqpoint{3.334893in}{3.569918in}}%
\pgfpathlineto{\pgfqpoint{3.350280in}{3.552951in}}%
\pgfpathclose%
\pgfusepath{fill}%
\end{pgfscope}%
\begin{pgfscope}%
\pgfpathrectangle{\pgfqpoint{0.765000in}{0.660000in}}{\pgfqpoint{4.620000in}{4.620000in}}%
\pgfusepath{clip}%
\pgfsetbuttcap%
\pgfsetroundjoin%
\definecolor{currentfill}{rgb}{1.000000,0.894118,0.788235}%
\pgfsetfillcolor{currentfill}%
\pgfsetlinewidth{0.000000pt}%
\definecolor{currentstroke}{rgb}{1.000000,0.894118,0.788235}%
\pgfsetstrokecolor{currentstroke}%
\pgfsetdash{}{0pt}%
\pgfpathmoveto{\pgfqpoint{3.461177in}{3.488925in}}%
\pgfpathlineto{\pgfqpoint{3.350280in}{3.552951in}}%
\pgfpathlineto{\pgfqpoint{3.334893in}{3.569918in}}%
\pgfpathlineto{\pgfqpoint{3.445790in}{3.505892in}}%
\pgfpathlineto{\pgfqpoint{3.461177in}{3.488925in}}%
\pgfpathclose%
\pgfusepath{fill}%
\end{pgfscope}%
\begin{pgfscope}%
\pgfpathrectangle{\pgfqpoint{0.765000in}{0.660000in}}{\pgfqpoint{4.620000in}{4.620000in}}%
\pgfusepath{clip}%
\pgfsetbuttcap%
\pgfsetroundjoin%
\definecolor{currentfill}{rgb}{1.000000,0.894118,0.788235}%
\pgfsetfillcolor{currentfill}%
\pgfsetlinewidth{0.000000pt}%
\definecolor{currentstroke}{rgb}{1.000000,0.894118,0.788235}%
\pgfsetstrokecolor{currentstroke}%
\pgfsetdash{}{0pt}%
\pgfpathmoveto{\pgfqpoint{3.239384in}{3.360872in}}%
\pgfpathlineto{\pgfqpoint{3.239384in}{3.488925in}}%
\pgfpathlineto{\pgfqpoint{3.223996in}{3.505892in}}%
\pgfpathlineto{\pgfqpoint{3.334893in}{3.313813in}}%
\pgfpathlineto{\pgfqpoint{3.239384in}{3.360872in}}%
\pgfpathclose%
\pgfusepath{fill}%
\end{pgfscope}%
\begin{pgfscope}%
\pgfpathrectangle{\pgfqpoint{0.765000in}{0.660000in}}{\pgfqpoint{4.620000in}{4.620000in}}%
\pgfusepath{clip}%
\pgfsetbuttcap%
\pgfsetroundjoin%
\definecolor{currentfill}{rgb}{1.000000,0.894118,0.788235}%
\pgfsetfillcolor{currentfill}%
\pgfsetlinewidth{0.000000pt}%
\definecolor{currentstroke}{rgb}{1.000000,0.894118,0.788235}%
\pgfsetstrokecolor{currentstroke}%
\pgfsetdash{}{0pt}%
\pgfpathmoveto{\pgfqpoint{3.350280in}{3.296846in}}%
\pgfpathlineto{\pgfqpoint{3.350280in}{3.424899in}}%
\pgfpathlineto{\pgfqpoint{3.334893in}{3.441865in}}%
\pgfpathlineto{\pgfqpoint{3.223996in}{3.377839in}}%
\pgfpathlineto{\pgfqpoint{3.350280in}{3.296846in}}%
\pgfpathclose%
\pgfusepath{fill}%
\end{pgfscope}%
\begin{pgfscope}%
\pgfpathrectangle{\pgfqpoint{0.765000in}{0.660000in}}{\pgfqpoint{4.620000in}{4.620000in}}%
\pgfusepath{clip}%
\pgfsetbuttcap%
\pgfsetroundjoin%
\definecolor{currentfill}{rgb}{1.000000,0.894118,0.788235}%
\pgfsetfillcolor{currentfill}%
\pgfsetlinewidth{0.000000pt}%
\definecolor{currentstroke}{rgb}{1.000000,0.894118,0.788235}%
\pgfsetstrokecolor{currentstroke}%
\pgfsetdash{}{0pt}%
\pgfpathmoveto{\pgfqpoint{3.350280in}{3.424899in}}%
\pgfpathlineto{\pgfqpoint{3.461177in}{3.488925in}}%
\pgfpathlineto{\pgfqpoint{3.445790in}{3.505892in}}%
\pgfpathlineto{\pgfqpoint{3.334893in}{3.441865in}}%
\pgfpathlineto{\pgfqpoint{3.350280in}{3.424899in}}%
\pgfpathclose%
\pgfusepath{fill}%
\end{pgfscope}%
\begin{pgfscope}%
\pgfpathrectangle{\pgfqpoint{0.765000in}{0.660000in}}{\pgfqpoint{4.620000in}{4.620000in}}%
\pgfusepath{clip}%
\pgfsetbuttcap%
\pgfsetroundjoin%
\definecolor{currentfill}{rgb}{1.000000,0.894118,0.788235}%
\pgfsetfillcolor{currentfill}%
\pgfsetlinewidth{0.000000pt}%
\definecolor{currentstroke}{rgb}{1.000000,0.894118,0.788235}%
\pgfsetstrokecolor{currentstroke}%
\pgfsetdash{}{0pt}%
\pgfpathmoveto{\pgfqpoint{3.239384in}{3.488925in}}%
\pgfpathlineto{\pgfqpoint{3.350280in}{3.424899in}}%
\pgfpathlineto{\pgfqpoint{3.334893in}{3.441865in}}%
\pgfpathlineto{\pgfqpoint{3.223996in}{3.505892in}}%
\pgfpathlineto{\pgfqpoint{3.239384in}{3.488925in}}%
\pgfpathclose%
\pgfusepath{fill}%
\end{pgfscope}%
\begin{pgfscope}%
\pgfpathrectangle{\pgfqpoint{0.765000in}{0.660000in}}{\pgfqpoint{4.620000in}{4.620000in}}%
\pgfusepath{clip}%
\pgfsetbuttcap%
\pgfsetroundjoin%
\definecolor{currentfill}{rgb}{1.000000,0.894118,0.788235}%
\pgfsetfillcolor{currentfill}%
\pgfsetlinewidth{0.000000pt}%
\definecolor{currentstroke}{rgb}{1.000000,0.894118,0.788235}%
\pgfsetstrokecolor{currentstroke}%
\pgfsetdash{}{0pt}%
\pgfpathmoveto{\pgfqpoint{2.671300in}{3.327008in}}%
\pgfpathlineto{\pgfqpoint{2.560403in}{3.262981in}}%
\pgfpathlineto{\pgfqpoint{2.671300in}{3.198955in}}%
\pgfpathlineto{\pgfqpoint{2.782196in}{3.262981in}}%
\pgfpathlineto{\pgfqpoint{2.671300in}{3.327008in}}%
\pgfpathclose%
\pgfusepath{fill}%
\end{pgfscope}%
\begin{pgfscope}%
\pgfpathrectangle{\pgfqpoint{0.765000in}{0.660000in}}{\pgfqpoint{4.620000in}{4.620000in}}%
\pgfusepath{clip}%
\pgfsetbuttcap%
\pgfsetroundjoin%
\definecolor{currentfill}{rgb}{1.000000,0.894118,0.788235}%
\pgfsetfillcolor{currentfill}%
\pgfsetlinewidth{0.000000pt}%
\definecolor{currentstroke}{rgb}{1.000000,0.894118,0.788235}%
\pgfsetstrokecolor{currentstroke}%
\pgfsetdash{}{0pt}%
\pgfpathmoveto{\pgfqpoint{2.671300in}{3.327008in}}%
\pgfpathlineto{\pgfqpoint{2.560403in}{3.262981in}}%
\pgfpathlineto{\pgfqpoint{2.560403in}{3.391034in}}%
\pgfpathlineto{\pgfqpoint{2.671300in}{3.455060in}}%
\pgfpathlineto{\pgfqpoint{2.671300in}{3.327008in}}%
\pgfpathclose%
\pgfusepath{fill}%
\end{pgfscope}%
\begin{pgfscope}%
\pgfpathrectangle{\pgfqpoint{0.765000in}{0.660000in}}{\pgfqpoint{4.620000in}{4.620000in}}%
\pgfusepath{clip}%
\pgfsetbuttcap%
\pgfsetroundjoin%
\definecolor{currentfill}{rgb}{1.000000,0.894118,0.788235}%
\pgfsetfillcolor{currentfill}%
\pgfsetlinewidth{0.000000pt}%
\definecolor{currentstroke}{rgb}{1.000000,0.894118,0.788235}%
\pgfsetstrokecolor{currentstroke}%
\pgfsetdash{}{0pt}%
\pgfpathmoveto{\pgfqpoint{2.671300in}{3.327008in}}%
\pgfpathlineto{\pgfqpoint{2.782196in}{3.262981in}}%
\pgfpathlineto{\pgfqpoint{2.782196in}{3.391034in}}%
\pgfpathlineto{\pgfqpoint{2.671300in}{3.455060in}}%
\pgfpathlineto{\pgfqpoint{2.671300in}{3.327008in}}%
\pgfpathclose%
\pgfusepath{fill}%
\end{pgfscope}%
\begin{pgfscope}%
\pgfpathrectangle{\pgfqpoint{0.765000in}{0.660000in}}{\pgfqpoint{4.620000in}{4.620000in}}%
\pgfusepath{clip}%
\pgfsetbuttcap%
\pgfsetroundjoin%
\definecolor{currentfill}{rgb}{1.000000,0.894118,0.788235}%
\pgfsetfillcolor{currentfill}%
\pgfsetlinewidth{0.000000pt}%
\definecolor{currentstroke}{rgb}{1.000000,0.894118,0.788235}%
\pgfsetstrokecolor{currentstroke}%
\pgfsetdash{}{0pt}%
\pgfpathmoveto{\pgfqpoint{2.692300in}{3.344662in}}%
\pgfpathlineto{\pgfqpoint{2.581404in}{3.280636in}}%
\pgfpathlineto{\pgfqpoint{2.692300in}{3.216609in}}%
\pgfpathlineto{\pgfqpoint{2.803197in}{3.280636in}}%
\pgfpathlineto{\pgfqpoint{2.692300in}{3.344662in}}%
\pgfpathclose%
\pgfusepath{fill}%
\end{pgfscope}%
\begin{pgfscope}%
\pgfpathrectangle{\pgfqpoint{0.765000in}{0.660000in}}{\pgfqpoint{4.620000in}{4.620000in}}%
\pgfusepath{clip}%
\pgfsetbuttcap%
\pgfsetroundjoin%
\definecolor{currentfill}{rgb}{1.000000,0.894118,0.788235}%
\pgfsetfillcolor{currentfill}%
\pgfsetlinewidth{0.000000pt}%
\definecolor{currentstroke}{rgb}{1.000000,0.894118,0.788235}%
\pgfsetstrokecolor{currentstroke}%
\pgfsetdash{}{0pt}%
\pgfpathmoveto{\pgfqpoint{2.692300in}{3.344662in}}%
\pgfpathlineto{\pgfqpoint{2.581404in}{3.280636in}}%
\pgfpathlineto{\pgfqpoint{2.581404in}{3.408688in}}%
\pgfpathlineto{\pgfqpoint{2.692300in}{3.472714in}}%
\pgfpathlineto{\pgfqpoint{2.692300in}{3.344662in}}%
\pgfpathclose%
\pgfusepath{fill}%
\end{pgfscope}%
\begin{pgfscope}%
\pgfpathrectangle{\pgfqpoint{0.765000in}{0.660000in}}{\pgfqpoint{4.620000in}{4.620000in}}%
\pgfusepath{clip}%
\pgfsetbuttcap%
\pgfsetroundjoin%
\definecolor{currentfill}{rgb}{1.000000,0.894118,0.788235}%
\pgfsetfillcolor{currentfill}%
\pgfsetlinewidth{0.000000pt}%
\definecolor{currentstroke}{rgb}{1.000000,0.894118,0.788235}%
\pgfsetstrokecolor{currentstroke}%
\pgfsetdash{}{0pt}%
\pgfpathmoveto{\pgfqpoint{2.692300in}{3.344662in}}%
\pgfpathlineto{\pgfqpoint{2.803197in}{3.280636in}}%
\pgfpathlineto{\pgfqpoint{2.803197in}{3.408688in}}%
\pgfpathlineto{\pgfqpoint{2.692300in}{3.472714in}}%
\pgfpathlineto{\pgfqpoint{2.692300in}{3.344662in}}%
\pgfpathclose%
\pgfusepath{fill}%
\end{pgfscope}%
\begin{pgfscope}%
\pgfpathrectangle{\pgfqpoint{0.765000in}{0.660000in}}{\pgfqpoint{4.620000in}{4.620000in}}%
\pgfusepath{clip}%
\pgfsetbuttcap%
\pgfsetroundjoin%
\definecolor{currentfill}{rgb}{1.000000,0.894118,0.788235}%
\pgfsetfillcolor{currentfill}%
\pgfsetlinewidth{0.000000pt}%
\definecolor{currentstroke}{rgb}{1.000000,0.894118,0.788235}%
\pgfsetstrokecolor{currentstroke}%
\pgfsetdash{}{0pt}%
\pgfpathmoveto{\pgfqpoint{2.671300in}{3.455060in}}%
\pgfpathlineto{\pgfqpoint{2.560403in}{3.391034in}}%
\pgfpathlineto{\pgfqpoint{2.671300in}{3.327008in}}%
\pgfpathlineto{\pgfqpoint{2.782196in}{3.391034in}}%
\pgfpathlineto{\pgfqpoint{2.671300in}{3.455060in}}%
\pgfpathclose%
\pgfusepath{fill}%
\end{pgfscope}%
\begin{pgfscope}%
\pgfpathrectangle{\pgfqpoint{0.765000in}{0.660000in}}{\pgfqpoint{4.620000in}{4.620000in}}%
\pgfusepath{clip}%
\pgfsetbuttcap%
\pgfsetroundjoin%
\definecolor{currentfill}{rgb}{1.000000,0.894118,0.788235}%
\pgfsetfillcolor{currentfill}%
\pgfsetlinewidth{0.000000pt}%
\definecolor{currentstroke}{rgb}{1.000000,0.894118,0.788235}%
\pgfsetstrokecolor{currentstroke}%
\pgfsetdash{}{0pt}%
\pgfpathmoveto{\pgfqpoint{2.671300in}{3.198955in}}%
\pgfpathlineto{\pgfqpoint{2.782196in}{3.262981in}}%
\pgfpathlineto{\pgfqpoint{2.782196in}{3.391034in}}%
\pgfpathlineto{\pgfqpoint{2.671300in}{3.327008in}}%
\pgfpathlineto{\pgfqpoint{2.671300in}{3.198955in}}%
\pgfpathclose%
\pgfusepath{fill}%
\end{pgfscope}%
\begin{pgfscope}%
\pgfpathrectangle{\pgfqpoint{0.765000in}{0.660000in}}{\pgfqpoint{4.620000in}{4.620000in}}%
\pgfusepath{clip}%
\pgfsetbuttcap%
\pgfsetroundjoin%
\definecolor{currentfill}{rgb}{1.000000,0.894118,0.788235}%
\pgfsetfillcolor{currentfill}%
\pgfsetlinewidth{0.000000pt}%
\definecolor{currentstroke}{rgb}{1.000000,0.894118,0.788235}%
\pgfsetstrokecolor{currentstroke}%
\pgfsetdash{}{0pt}%
\pgfpathmoveto{\pgfqpoint{2.560403in}{3.262981in}}%
\pgfpathlineto{\pgfqpoint{2.671300in}{3.198955in}}%
\pgfpathlineto{\pgfqpoint{2.671300in}{3.327008in}}%
\pgfpathlineto{\pgfqpoint{2.560403in}{3.391034in}}%
\pgfpathlineto{\pgfqpoint{2.560403in}{3.262981in}}%
\pgfpathclose%
\pgfusepath{fill}%
\end{pgfscope}%
\begin{pgfscope}%
\pgfpathrectangle{\pgfqpoint{0.765000in}{0.660000in}}{\pgfqpoint{4.620000in}{4.620000in}}%
\pgfusepath{clip}%
\pgfsetbuttcap%
\pgfsetroundjoin%
\definecolor{currentfill}{rgb}{1.000000,0.894118,0.788235}%
\pgfsetfillcolor{currentfill}%
\pgfsetlinewidth{0.000000pt}%
\definecolor{currentstroke}{rgb}{1.000000,0.894118,0.788235}%
\pgfsetstrokecolor{currentstroke}%
\pgfsetdash{}{0pt}%
\pgfpathmoveto{\pgfqpoint{2.692300in}{3.472714in}}%
\pgfpathlineto{\pgfqpoint{2.581404in}{3.408688in}}%
\pgfpathlineto{\pgfqpoint{2.692300in}{3.344662in}}%
\pgfpathlineto{\pgfqpoint{2.803197in}{3.408688in}}%
\pgfpathlineto{\pgfqpoint{2.692300in}{3.472714in}}%
\pgfpathclose%
\pgfusepath{fill}%
\end{pgfscope}%
\begin{pgfscope}%
\pgfpathrectangle{\pgfqpoint{0.765000in}{0.660000in}}{\pgfqpoint{4.620000in}{4.620000in}}%
\pgfusepath{clip}%
\pgfsetbuttcap%
\pgfsetroundjoin%
\definecolor{currentfill}{rgb}{1.000000,0.894118,0.788235}%
\pgfsetfillcolor{currentfill}%
\pgfsetlinewidth{0.000000pt}%
\definecolor{currentstroke}{rgb}{1.000000,0.894118,0.788235}%
\pgfsetstrokecolor{currentstroke}%
\pgfsetdash{}{0pt}%
\pgfpathmoveto{\pgfqpoint{2.692300in}{3.216609in}}%
\pgfpathlineto{\pgfqpoint{2.803197in}{3.280636in}}%
\pgfpathlineto{\pgfqpoint{2.803197in}{3.408688in}}%
\pgfpathlineto{\pgfqpoint{2.692300in}{3.344662in}}%
\pgfpathlineto{\pgfqpoint{2.692300in}{3.216609in}}%
\pgfpathclose%
\pgfusepath{fill}%
\end{pgfscope}%
\begin{pgfscope}%
\pgfpathrectangle{\pgfqpoint{0.765000in}{0.660000in}}{\pgfqpoint{4.620000in}{4.620000in}}%
\pgfusepath{clip}%
\pgfsetbuttcap%
\pgfsetroundjoin%
\definecolor{currentfill}{rgb}{1.000000,0.894118,0.788235}%
\pgfsetfillcolor{currentfill}%
\pgfsetlinewidth{0.000000pt}%
\definecolor{currentstroke}{rgb}{1.000000,0.894118,0.788235}%
\pgfsetstrokecolor{currentstroke}%
\pgfsetdash{}{0pt}%
\pgfpathmoveto{\pgfqpoint{2.581404in}{3.280636in}}%
\pgfpathlineto{\pgfqpoint{2.692300in}{3.216609in}}%
\pgfpathlineto{\pgfqpoint{2.692300in}{3.344662in}}%
\pgfpathlineto{\pgfqpoint{2.581404in}{3.408688in}}%
\pgfpathlineto{\pgfqpoint{2.581404in}{3.280636in}}%
\pgfpathclose%
\pgfusepath{fill}%
\end{pgfscope}%
\begin{pgfscope}%
\pgfpathrectangle{\pgfqpoint{0.765000in}{0.660000in}}{\pgfqpoint{4.620000in}{4.620000in}}%
\pgfusepath{clip}%
\pgfsetbuttcap%
\pgfsetroundjoin%
\definecolor{currentfill}{rgb}{1.000000,0.894118,0.788235}%
\pgfsetfillcolor{currentfill}%
\pgfsetlinewidth{0.000000pt}%
\definecolor{currentstroke}{rgb}{1.000000,0.894118,0.788235}%
\pgfsetstrokecolor{currentstroke}%
\pgfsetdash{}{0pt}%
\pgfpathmoveto{\pgfqpoint{2.671300in}{3.327008in}}%
\pgfpathlineto{\pgfqpoint{2.560403in}{3.262981in}}%
\pgfpathlineto{\pgfqpoint{2.581404in}{3.280636in}}%
\pgfpathlineto{\pgfqpoint{2.692300in}{3.344662in}}%
\pgfpathlineto{\pgfqpoint{2.671300in}{3.327008in}}%
\pgfpathclose%
\pgfusepath{fill}%
\end{pgfscope}%
\begin{pgfscope}%
\pgfpathrectangle{\pgfqpoint{0.765000in}{0.660000in}}{\pgfqpoint{4.620000in}{4.620000in}}%
\pgfusepath{clip}%
\pgfsetbuttcap%
\pgfsetroundjoin%
\definecolor{currentfill}{rgb}{1.000000,0.894118,0.788235}%
\pgfsetfillcolor{currentfill}%
\pgfsetlinewidth{0.000000pt}%
\definecolor{currentstroke}{rgb}{1.000000,0.894118,0.788235}%
\pgfsetstrokecolor{currentstroke}%
\pgfsetdash{}{0pt}%
\pgfpathmoveto{\pgfqpoint{2.782196in}{3.262981in}}%
\pgfpathlineto{\pgfqpoint{2.671300in}{3.327008in}}%
\pgfpathlineto{\pgfqpoint{2.692300in}{3.344662in}}%
\pgfpathlineto{\pgfqpoint{2.803197in}{3.280636in}}%
\pgfpathlineto{\pgfqpoint{2.782196in}{3.262981in}}%
\pgfpathclose%
\pgfusepath{fill}%
\end{pgfscope}%
\begin{pgfscope}%
\pgfpathrectangle{\pgfqpoint{0.765000in}{0.660000in}}{\pgfqpoint{4.620000in}{4.620000in}}%
\pgfusepath{clip}%
\pgfsetbuttcap%
\pgfsetroundjoin%
\definecolor{currentfill}{rgb}{1.000000,0.894118,0.788235}%
\pgfsetfillcolor{currentfill}%
\pgfsetlinewidth{0.000000pt}%
\definecolor{currentstroke}{rgb}{1.000000,0.894118,0.788235}%
\pgfsetstrokecolor{currentstroke}%
\pgfsetdash{}{0pt}%
\pgfpathmoveto{\pgfqpoint{2.671300in}{3.327008in}}%
\pgfpathlineto{\pgfqpoint{2.671300in}{3.455060in}}%
\pgfpathlineto{\pgfqpoint{2.692300in}{3.472714in}}%
\pgfpathlineto{\pgfqpoint{2.803197in}{3.280636in}}%
\pgfpathlineto{\pgfqpoint{2.671300in}{3.327008in}}%
\pgfpathclose%
\pgfusepath{fill}%
\end{pgfscope}%
\begin{pgfscope}%
\pgfpathrectangle{\pgfqpoint{0.765000in}{0.660000in}}{\pgfqpoint{4.620000in}{4.620000in}}%
\pgfusepath{clip}%
\pgfsetbuttcap%
\pgfsetroundjoin%
\definecolor{currentfill}{rgb}{1.000000,0.894118,0.788235}%
\pgfsetfillcolor{currentfill}%
\pgfsetlinewidth{0.000000pt}%
\definecolor{currentstroke}{rgb}{1.000000,0.894118,0.788235}%
\pgfsetstrokecolor{currentstroke}%
\pgfsetdash{}{0pt}%
\pgfpathmoveto{\pgfqpoint{2.782196in}{3.262981in}}%
\pgfpathlineto{\pgfqpoint{2.782196in}{3.391034in}}%
\pgfpathlineto{\pgfqpoint{2.803197in}{3.408688in}}%
\pgfpathlineto{\pgfqpoint{2.692300in}{3.344662in}}%
\pgfpathlineto{\pgfqpoint{2.782196in}{3.262981in}}%
\pgfpathclose%
\pgfusepath{fill}%
\end{pgfscope}%
\begin{pgfscope}%
\pgfpathrectangle{\pgfqpoint{0.765000in}{0.660000in}}{\pgfqpoint{4.620000in}{4.620000in}}%
\pgfusepath{clip}%
\pgfsetbuttcap%
\pgfsetroundjoin%
\definecolor{currentfill}{rgb}{1.000000,0.894118,0.788235}%
\pgfsetfillcolor{currentfill}%
\pgfsetlinewidth{0.000000pt}%
\definecolor{currentstroke}{rgb}{1.000000,0.894118,0.788235}%
\pgfsetstrokecolor{currentstroke}%
\pgfsetdash{}{0pt}%
\pgfpathmoveto{\pgfqpoint{2.560403in}{3.262981in}}%
\pgfpathlineto{\pgfqpoint{2.671300in}{3.198955in}}%
\pgfpathlineto{\pgfqpoint{2.692300in}{3.216609in}}%
\pgfpathlineto{\pgfqpoint{2.581404in}{3.280636in}}%
\pgfpathlineto{\pgfqpoint{2.560403in}{3.262981in}}%
\pgfpathclose%
\pgfusepath{fill}%
\end{pgfscope}%
\begin{pgfscope}%
\pgfpathrectangle{\pgfqpoint{0.765000in}{0.660000in}}{\pgfqpoint{4.620000in}{4.620000in}}%
\pgfusepath{clip}%
\pgfsetbuttcap%
\pgfsetroundjoin%
\definecolor{currentfill}{rgb}{1.000000,0.894118,0.788235}%
\pgfsetfillcolor{currentfill}%
\pgfsetlinewidth{0.000000pt}%
\definecolor{currentstroke}{rgb}{1.000000,0.894118,0.788235}%
\pgfsetstrokecolor{currentstroke}%
\pgfsetdash{}{0pt}%
\pgfpathmoveto{\pgfqpoint{2.671300in}{3.198955in}}%
\pgfpathlineto{\pgfqpoint{2.782196in}{3.262981in}}%
\pgfpathlineto{\pgfqpoint{2.803197in}{3.280636in}}%
\pgfpathlineto{\pgfqpoint{2.692300in}{3.216609in}}%
\pgfpathlineto{\pgfqpoint{2.671300in}{3.198955in}}%
\pgfpathclose%
\pgfusepath{fill}%
\end{pgfscope}%
\begin{pgfscope}%
\pgfpathrectangle{\pgfqpoint{0.765000in}{0.660000in}}{\pgfqpoint{4.620000in}{4.620000in}}%
\pgfusepath{clip}%
\pgfsetbuttcap%
\pgfsetroundjoin%
\definecolor{currentfill}{rgb}{1.000000,0.894118,0.788235}%
\pgfsetfillcolor{currentfill}%
\pgfsetlinewidth{0.000000pt}%
\definecolor{currentstroke}{rgb}{1.000000,0.894118,0.788235}%
\pgfsetstrokecolor{currentstroke}%
\pgfsetdash{}{0pt}%
\pgfpathmoveto{\pgfqpoint{2.671300in}{3.455060in}}%
\pgfpathlineto{\pgfqpoint{2.560403in}{3.391034in}}%
\pgfpathlineto{\pgfqpoint{2.581404in}{3.408688in}}%
\pgfpathlineto{\pgfqpoint{2.692300in}{3.472714in}}%
\pgfpathlineto{\pgfqpoint{2.671300in}{3.455060in}}%
\pgfpathclose%
\pgfusepath{fill}%
\end{pgfscope}%
\begin{pgfscope}%
\pgfpathrectangle{\pgfqpoint{0.765000in}{0.660000in}}{\pgfqpoint{4.620000in}{4.620000in}}%
\pgfusepath{clip}%
\pgfsetbuttcap%
\pgfsetroundjoin%
\definecolor{currentfill}{rgb}{1.000000,0.894118,0.788235}%
\pgfsetfillcolor{currentfill}%
\pgfsetlinewidth{0.000000pt}%
\definecolor{currentstroke}{rgb}{1.000000,0.894118,0.788235}%
\pgfsetstrokecolor{currentstroke}%
\pgfsetdash{}{0pt}%
\pgfpathmoveto{\pgfqpoint{2.782196in}{3.391034in}}%
\pgfpathlineto{\pgfqpoint{2.671300in}{3.455060in}}%
\pgfpathlineto{\pgfqpoint{2.692300in}{3.472714in}}%
\pgfpathlineto{\pgfqpoint{2.803197in}{3.408688in}}%
\pgfpathlineto{\pgfqpoint{2.782196in}{3.391034in}}%
\pgfpathclose%
\pgfusepath{fill}%
\end{pgfscope}%
\begin{pgfscope}%
\pgfpathrectangle{\pgfqpoint{0.765000in}{0.660000in}}{\pgfqpoint{4.620000in}{4.620000in}}%
\pgfusepath{clip}%
\pgfsetbuttcap%
\pgfsetroundjoin%
\definecolor{currentfill}{rgb}{1.000000,0.894118,0.788235}%
\pgfsetfillcolor{currentfill}%
\pgfsetlinewidth{0.000000pt}%
\definecolor{currentstroke}{rgb}{1.000000,0.894118,0.788235}%
\pgfsetstrokecolor{currentstroke}%
\pgfsetdash{}{0pt}%
\pgfpathmoveto{\pgfqpoint{2.560403in}{3.262981in}}%
\pgfpathlineto{\pgfqpoint{2.560403in}{3.391034in}}%
\pgfpathlineto{\pgfqpoint{2.581404in}{3.408688in}}%
\pgfpathlineto{\pgfqpoint{2.692300in}{3.216609in}}%
\pgfpathlineto{\pgfqpoint{2.560403in}{3.262981in}}%
\pgfpathclose%
\pgfusepath{fill}%
\end{pgfscope}%
\begin{pgfscope}%
\pgfpathrectangle{\pgfqpoint{0.765000in}{0.660000in}}{\pgfqpoint{4.620000in}{4.620000in}}%
\pgfusepath{clip}%
\pgfsetbuttcap%
\pgfsetroundjoin%
\definecolor{currentfill}{rgb}{1.000000,0.894118,0.788235}%
\pgfsetfillcolor{currentfill}%
\pgfsetlinewidth{0.000000pt}%
\definecolor{currentstroke}{rgb}{1.000000,0.894118,0.788235}%
\pgfsetstrokecolor{currentstroke}%
\pgfsetdash{}{0pt}%
\pgfpathmoveto{\pgfqpoint{2.671300in}{3.198955in}}%
\pgfpathlineto{\pgfqpoint{2.671300in}{3.327008in}}%
\pgfpathlineto{\pgfqpoint{2.692300in}{3.344662in}}%
\pgfpathlineto{\pgfqpoint{2.581404in}{3.280636in}}%
\pgfpathlineto{\pgfqpoint{2.671300in}{3.198955in}}%
\pgfpathclose%
\pgfusepath{fill}%
\end{pgfscope}%
\begin{pgfscope}%
\pgfpathrectangle{\pgfqpoint{0.765000in}{0.660000in}}{\pgfqpoint{4.620000in}{4.620000in}}%
\pgfusepath{clip}%
\pgfsetbuttcap%
\pgfsetroundjoin%
\definecolor{currentfill}{rgb}{1.000000,0.894118,0.788235}%
\pgfsetfillcolor{currentfill}%
\pgfsetlinewidth{0.000000pt}%
\definecolor{currentstroke}{rgb}{1.000000,0.894118,0.788235}%
\pgfsetstrokecolor{currentstroke}%
\pgfsetdash{}{0pt}%
\pgfpathmoveto{\pgfqpoint{2.671300in}{3.327008in}}%
\pgfpathlineto{\pgfqpoint{2.782196in}{3.391034in}}%
\pgfpathlineto{\pgfqpoint{2.803197in}{3.408688in}}%
\pgfpathlineto{\pgfqpoint{2.692300in}{3.344662in}}%
\pgfpathlineto{\pgfqpoint{2.671300in}{3.327008in}}%
\pgfpathclose%
\pgfusepath{fill}%
\end{pgfscope}%
\begin{pgfscope}%
\pgfpathrectangle{\pgfqpoint{0.765000in}{0.660000in}}{\pgfqpoint{4.620000in}{4.620000in}}%
\pgfusepath{clip}%
\pgfsetbuttcap%
\pgfsetroundjoin%
\definecolor{currentfill}{rgb}{1.000000,0.894118,0.788235}%
\pgfsetfillcolor{currentfill}%
\pgfsetlinewidth{0.000000pt}%
\definecolor{currentstroke}{rgb}{1.000000,0.894118,0.788235}%
\pgfsetstrokecolor{currentstroke}%
\pgfsetdash{}{0pt}%
\pgfpathmoveto{\pgfqpoint{2.560403in}{3.391034in}}%
\pgfpathlineto{\pgfqpoint{2.671300in}{3.327008in}}%
\pgfpathlineto{\pgfqpoint{2.692300in}{3.344662in}}%
\pgfpathlineto{\pgfqpoint{2.581404in}{3.408688in}}%
\pgfpathlineto{\pgfqpoint{2.560403in}{3.391034in}}%
\pgfpathclose%
\pgfusepath{fill}%
\end{pgfscope}%
\begin{pgfscope}%
\pgfpathrectangle{\pgfqpoint{0.765000in}{0.660000in}}{\pgfqpoint{4.620000in}{4.620000in}}%
\pgfusepath{clip}%
\pgfsetbuttcap%
\pgfsetroundjoin%
\definecolor{currentfill}{rgb}{1.000000,0.894118,0.788235}%
\pgfsetfillcolor{currentfill}%
\pgfsetlinewidth{0.000000pt}%
\definecolor{currentstroke}{rgb}{1.000000,0.894118,0.788235}%
\pgfsetstrokecolor{currentstroke}%
\pgfsetdash{}{0pt}%
\pgfpathmoveto{\pgfqpoint{2.671300in}{3.327008in}}%
\pgfpathlineto{\pgfqpoint{2.560403in}{3.262981in}}%
\pgfpathlineto{\pgfqpoint{2.671300in}{3.198955in}}%
\pgfpathlineto{\pgfqpoint{2.782196in}{3.262981in}}%
\pgfpathlineto{\pgfqpoint{2.671300in}{3.327008in}}%
\pgfpathclose%
\pgfusepath{fill}%
\end{pgfscope}%
\begin{pgfscope}%
\pgfpathrectangle{\pgfqpoint{0.765000in}{0.660000in}}{\pgfqpoint{4.620000in}{4.620000in}}%
\pgfusepath{clip}%
\pgfsetbuttcap%
\pgfsetroundjoin%
\definecolor{currentfill}{rgb}{1.000000,0.894118,0.788235}%
\pgfsetfillcolor{currentfill}%
\pgfsetlinewidth{0.000000pt}%
\definecolor{currentstroke}{rgb}{1.000000,0.894118,0.788235}%
\pgfsetstrokecolor{currentstroke}%
\pgfsetdash{}{0pt}%
\pgfpathmoveto{\pgfqpoint{2.671300in}{3.327008in}}%
\pgfpathlineto{\pgfqpoint{2.560403in}{3.262981in}}%
\pgfpathlineto{\pgfqpoint{2.560403in}{3.391034in}}%
\pgfpathlineto{\pgfqpoint{2.671300in}{3.455060in}}%
\pgfpathlineto{\pgfqpoint{2.671300in}{3.327008in}}%
\pgfpathclose%
\pgfusepath{fill}%
\end{pgfscope}%
\begin{pgfscope}%
\pgfpathrectangle{\pgfqpoint{0.765000in}{0.660000in}}{\pgfqpoint{4.620000in}{4.620000in}}%
\pgfusepath{clip}%
\pgfsetbuttcap%
\pgfsetroundjoin%
\definecolor{currentfill}{rgb}{1.000000,0.894118,0.788235}%
\pgfsetfillcolor{currentfill}%
\pgfsetlinewidth{0.000000pt}%
\definecolor{currentstroke}{rgb}{1.000000,0.894118,0.788235}%
\pgfsetstrokecolor{currentstroke}%
\pgfsetdash{}{0pt}%
\pgfpathmoveto{\pgfqpoint{2.671300in}{3.327008in}}%
\pgfpathlineto{\pgfqpoint{2.782196in}{3.262981in}}%
\pgfpathlineto{\pgfqpoint{2.782196in}{3.391034in}}%
\pgfpathlineto{\pgfqpoint{2.671300in}{3.455060in}}%
\pgfpathlineto{\pgfqpoint{2.671300in}{3.327008in}}%
\pgfpathclose%
\pgfusepath{fill}%
\end{pgfscope}%
\begin{pgfscope}%
\pgfpathrectangle{\pgfqpoint{0.765000in}{0.660000in}}{\pgfqpoint{4.620000in}{4.620000in}}%
\pgfusepath{clip}%
\pgfsetbuttcap%
\pgfsetroundjoin%
\definecolor{currentfill}{rgb}{1.000000,0.894118,0.788235}%
\pgfsetfillcolor{currentfill}%
\pgfsetlinewidth{0.000000pt}%
\definecolor{currentstroke}{rgb}{1.000000,0.894118,0.788235}%
\pgfsetstrokecolor{currentstroke}%
\pgfsetdash{}{0pt}%
\pgfpathmoveto{\pgfqpoint{2.651156in}{3.305927in}}%
\pgfpathlineto{\pgfqpoint{2.540259in}{3.241901in}}%
\pgfpathlineto{\pgfqpoint{2.651156in}{3.177874in}}%
\pgfpathlineto{\pgfqpoint{2.762052in}{3.241901in}}%
\pgfpathlineto{\pgfqpoint{2.651156in}{3.305927in}}%
\pgfpathclose%
\pgfusepath{fill}%
\end{pgfscope}%
\begin{pgfscope}%
\pgfpathrectangle{\pgfqpoint{0.765000in}{0.660000in}}{\pgfqpoint{4.620000in}{4.620000in}}%
\pgfusepath{clip}%
\pgfsetbuttcap%
\pgfsetroundjoin%
\definecolor{currentfill}{rgb}{1.000000,0.894118,0.788235}%
\pgfsetfillcolor{currentfill}%
\pgfsetlinewidth{0.000000pt}%
\definecolor{currentstroke}{rgb}{1.000000,0.894118,0.788235}%
\pgfsetstrokecolor{currentstroke}%
\pgfsetdash{}{0pt}%
\pgfpathmoveto{\pgfqpoint{2.651156in}{3.305927in}}%
\pgfpathlineto{\pgfqpoint{2.540259in}{3.241901in}}%
\pgfpathlineto{\pgfqpoint{2.540259in}{3.369953in}}%
\pgfpathlineto{\pgfqpoint{2.651156in}{3.433979in}}%
\pgfpathlineto{\pgfqpoint{2.651156in}{3.305927in}}%
\pgfpathclose%
\pgfusepath{fill}%
\end{pgfscope}%
\begin{pgfscope}%
\pgfpathrectangle{\pgfqpoint{0.765000in}{0.660000in}}{\pgfqpoint{4.620000in}{4.620000in}}%
\pgfusepath{clip}%
\pgfsetbuttcap%
\pgfsetroundjoin%
\definecolor{currentfill}{rgb}{1.000000,0.894118,0.788235}%
\pgfsetfillcolor{currentfill}%
\pgfsetlinewidth{0.000000pt}%
\definecolor{currentstroke}{rgb}{1.000000,0.894118,0.788235}%
\pgfsetstrokecolor{currentstroke}%
\pgfsetdash{}{0pt}%
\pgfpathmoveto{\pgfqpoint{2.651156in}{3.305927in}}%
\pgfpathlineto{\pgfqpoint{2.762052in}{3.241901in}}%
\pgfpathlineto{\pgfqpoint{2.762052in}{3.369953in}}%
\pgfpathlineto{\pgfqpoint{2.651156in}{3.433979in}}%
\pgfpathlineto{\pgfqpoint{2.651156in}{3.305927in}}%
\pgfpathclose%
\pgfusepath{fill}%
\end{pgfscope}%
\begin{pgfscope}%
\pgfpathrectangle{\pgfqpoint{0.765000in}{0.660000in}}{\pgfqpoint{4.620000in}{4.620000in}}%
\pgfusepath{clip}%
\pgfsetbuttcap%
\pgfsetroundjoin%
\definecolor{currentfill}{rgb}{1.000000,0.894118,0.788235}%
\pgfsetfillcolor{currentfill}%
\pgfsetlinewidth{0.000000pt}%
\definecolor{currentstroke}{rgb}{1.000000,0.894118,0.788235}%
\pgfsetstrokecolor{currentstroke}%
\pgfsetdash{}{0pt}%
\pgfpathmoveto{\pgfqpoint{2.671300in}{3.455060in}}%
\pgfpathlineto{\pgfqpoint{2.560403in}{3.391034in}}%
\pgfpathlineto{\pgfqpoint{2.671300in}{3.327008in}}%
\pgfpathlineto{\pgfqpoint{2.782196in}{3.391034in}}%
\pgfpathlineto{\pgfqpoint{2.671300in}{3.455060in}}%
\pgfpathclose%
\pgfusepath{fill}%
\end{pgfscope}%
\begin{pgfscope}%
\pgfpathrectangle{\pgfqpoint{0.765000in}{0.660000in}}{\pgfqpoint{4.620000in}{4.620000in}}%
\pgfusepath{clip}%
\pgfsetbuttcap%
\pgfsetroundjoin%
\definecolor{currentfill}{rgb}{1.000000,0.894118,0.788235}%
\pgfsetfillcolor{currentfill}%
\pgfsetlinewidth{0.000000pt}%
\definecolor{currentstroke}{rgb}{1.000000,0.894118,0.788235}%
\pgfsetstrokecolor{currentstroke}%
\pgfsetdash{}{0pt}%
\pgfpathmoveto{\pgfqpoint{2.671300in}{3.198955in}}%
\pgfpathlineto{\pgfqpoint{2.782196in}{3.262981in}}%
\pgfpathlineto{\pgfqpoint{2.782196in}{3.391034in}}%
\pgfpathlineto{\pgfqpoint{2.671300in}{3.327008in}}%
\pgfpathlineto{\pgfqpoint{2.671300in}{3.198955in}}%
\pgfpathclose%
\pgfusepath{fill}%
\end{pgfscope}%
\begin{pgfscope}%
\pgfpathrectangle{\pgfqpoint{0.765000in}{0.660000in}}{\pgfqpoint{4.620000in}{4.620000in}}%
\pgfusepath{clip}%
\pgfsetbuttcap%
\pgfsetroundjoin%
\definecolor{currentfill}{rgb}{1.000000,0.894118,0.788235}%
\pgfsetfillcolor{currentfill}%
\pgfsetlinewidth{0.000000pt}%
\definecolor{currentstroke}{rgb}{1.000000,0.894118,0.788235}%
\pgfsetstrokecolor{currentstroke}%
\pgfsetdash{}{0pt}%
\pgfpathmoveto{\pgfqpoint{2.560403in}{3.262981in}}%
\pgfpathlineto{\pgfqpoint{2.671300in}{3.198955in}}%
\pgfpathlineto{\pgfqpoint{2.671300in}{3.327008in}}%
\pgfpathlineto{\pgfqpoint{2.560403in}{3.391034in}}%
\pgfpathlineto{\pgfqpoint{2.560403in}{3.262981in}}%
\pgfpathclose%
\pgfusepath{fill}%
\end{pgfscope}%
\begin{pgfscope}%
\pgfpathrectangle{\pgfqpoint{0.765000in}{0.660000in}}{\pgfqpoint{4.620000in}{4.620000in}}%
\pgfusepath{clip}%
\pgfsetbuttcap%
\pgfsetroundjoin%
\definecolor{currentfill}{rgb}{1.000000,0.894118,0.788235}%
\pgfsetfillcolor{currentfill}%
\pgfsetlinewidth{0.000000pt}%
\definecolor{currentstroke}{rgb}{1.000000,0.894118,0.788235}%
\pgfsetstrokecolor{currentstroke}%
\pgfsetdash{}{0pt}%
\pgfpathmoveto{\pgfqpoint{2.651156in}{3.433979in}}%
\pgfpathlineto{\pgfqpoint{2.540259in}{3.369953in}}%
\pgfpathlineto{\pgfqpoint{2.651156in}{3.305927in}}%
\pgfpathlineto{\pgfqpoint{2.762052in}{3.369953in}}%
\pgfpathlineto{\pgfqpoint{2.651156in}{3.433979in}}%
\pgfpathclose%
\pgfusepath{fill}%
\end{pgfscope}%
\begin{pgfscope}%
\pgfpathrectangle{\pgfqpoint{0.765000in}{0.660000in}}{\pgfqpoint{4.620000in}{4.620000in}}%
\pgfusepath{clip}%
\pgfsetbuttcap%
\pgfsetroundjoin%
\definecolor{currentfill}{rgb}{1.000000,0.894118,0.788235}%
\pgfsetfillcolor{currentfill}%
\pgfsetlinewidth{0.000000pt}%
\definecolor{currentstroke}{rgb}{1.000000,0.894118,0.788235}%
\pgfsetstrokecolor{currentstroke}%
\pgfsetdash{}{0pt}%
\pgfpathmoveto{\pgfqpoint{2.651156in}{3.177874in}}%
\pgfpathlineto{\pgfqpoint{2.762052in}{3.241901in}}%
\pgfpathlineto{\pgfqpoint{2.762052in}{3.369953in}}%
\pgfpathlineto{\pgfqpoint{2.651156in}{3.305927in}}%
\pgfpathlineto{\pgfqpoint{2.651156in}{3.177874in}}%
\pgfpathclose%
\pgfusepath{fill}%
\end{pgfscope}%
\begin{pgfscope}%
\pgfpathrectangle{\pgfqpoint{0.765000in}{0.660000in}}{\pgfqpoint{4.620000in}{4.620000in}}%
\pgfusepath{clip}%
\pgfsetbuttcap%
\pgfsetroundjoin%
\definecolor{currentfill}{rgb}{1.000000,0.894118,0.788235}%
\pgfsetfillcolor{currentfill}%
\pgfsetlinewidth{0.000000pt}%
\definecolor{currentstroke}{rgb}{1.000000,0.894118,0.788235}%
\pgfsetstrokecolor{currentstroke}%
\pgfsetdash{}{0pt}%
\pgfpathmoveto{\pgfqpoint{2.540259in}{3.241901in}}%
\pgfpathlineto{\pgfqpoint{2.651156in}{3.177874in}}%
\pgfpathlineto{\pgfqpoint{2.651156in}{3.305927in}}%
\pgfpathlineto{\pgfqpoint{2.540259in}{3.369953in}}%
\pgfpathlineto{\pgfqpoint{2.540259in}{3.241901in}}%
\pgfpathclose%
\pgfusepath{fill}%
\end{pgfscope}%
\begin{pgfscope}%
\pgfpathrectangle{\pgfqpoint{0.765000in}{0.660000in}}{\pgfqpoint{4.620000in}{4.620000in}}%
\pgfusepath{clip}%
\pgfsetbuttcap%
\pgfsetroundjoin%
\definecolor{currentfill}{rgb}{1.000000,0.894118,0.788235}%
\pgfsetfillcolor{currentfill}%
\pgfsetlinewidth{0.000000pt}%
\definecolor{currentstroke}{rgb}{1.000000,0.894118,0.788235}%
\pgfsetstrokecolor{currentstroke}%
\pgfsetdash{}{0pt}%
\pgfpathmoveto{\pgfqpoint{2.671300in}{3.327008in}}%
\pgfpathlineto{\pgfqpoint{2.560403in}{3.262981in}}%
\pgfpathlineto{\pgfqpoint{2.540259in}{3.241901in}}%
\pgfpathlineto{\pgfqpoint{2.651156in}{3.305927in}}%
\pgfpathlineto{\pgfqpoint{2.671300in}{3.327008in}}%
\pgfpathclose%
\pgfusepath{fill}%
\end{pgfscope}%
\begin{pgfscope}%
\pgfpathrectangle{\pgfqpoint{0.765000in}{0.660000in}}{\pgfqpoint{4.620000in}{4.620000in}}%
\pgfusepath{clip}%
\pgfsetbuttcap%
\pgfsetroundjoin%
\definecolor{currentfill}{rgb}{1.000000,0.894118,0.788235}%
\pgfsetfillcolor{currentfill}%
\pgfsetlinewidth{0.000000pt}%
\definecolor{currentstroke}{rgb}{1.000000,0.894118,0.788235}%
\pgfsetstrokecolor{currentstroke}%
\pgfsetdash{}{0pt}%
\pgfpathmoveto{\pgfqpoint{2.782196in}{3.262981in}}%
\pgfpathlineto{\pgfqpoint{2.671300in}{3.327008in}}%
\pgfpathlineto{\pgfqpoint{2.651156in}{3.305927in}}%
\pgfpathlineto{\pgfqpoint{2.762052in}{3.241901in}}%
\pgfpathlineto{\pgfqpoint{2.782196in}{3.262981in}}%
\pgfpathclose%
\pgfusepath{fill}%
\end{pgfscope}%
\begin{pgfscope}%
\pgfpathrectangle{\pgfqpoint{0.765000in}{0.660000in}}{\pgfqpoint{4.620000in}{4.620000in}}%
\pgfusepath{clip}%
\pgfsetbuttcap%
\pgfsetroundjoin%
\definecolor{currentfill}{rgb}{1.000000,0.894118,0.788235}%
\pgfsetfillcolor{currentfill}%
\pgfsetlinewidth{0.000000pt}%
\definecolor{currentstroke}{rgb}{1.000000,0.894118,0.788235}%
\pgfsetstrokecolor{currentstroke}%
\pgfsetdash{}{0pt}%
\pgfpathmoveto{\pgfqpoint{2.671300in}{3.327008in}}%
\pgfpathlineto{\pgfqpoint{2.671300in}{3.455060in}}%
\pgfpathlineto{\pgfqpoint{2.651156in}{3.433979in}}%
\pgfpathlineto{\pgfqpoint{2.762052in}{3.241901in}}%
\pgfpathlineto{\pgfqpoint{2.671300in}{3.327008in}}%
\pgfpathclose%
\pgfusepath{fill}%
\end{pgfscope}%
\begin{pgfscope}%
\pgfpathrectangle{\pgfqpoint{0.765000in}{0.660000in}}{\pgfqpoint{4.620000in}{4.620000in}}%
\pgfusepath{clip}%
\pgfsetbuttcap%
\pgfsetroundjoin%
\definecolor{currentfill}{rgb}{1.000000,0.894118,0.788235}%
\pgfsetfillcolor{currentfill}%
\pgfsetlinewidth{0.000000pt}%
\definecolor{currentstroke}{rgb}{1.000000,0.894118,0.788235}%
\pgfsetstrokecolor{currentstroke}%
\pgfsetdash{}{0pt}%
\pgfpathmoveto{\pgfqpoint{2.782196in}{3.262981in}}%
\pgfpathlineto{\pgfqpoint{2.782196in}{3.391034in}}%
\pgfpathlineto{\pgfqpoint{2.762052in}{3.369953in}}%
\pgfpathlineto{\pgfqpoint{2.651156in}{3.305927in}}%
\pgfpathlineto{\pgfqpoint{2.782196in}{3.262981in}}%
\pgfpathclose%
\pgfusepath{fill}%
\end{pgfscope}%
\begin{pgfscope}%
\pgfpathrectangle{\pgfqpoint{0.765000in}{0.660000in}}{\pgfqpoint{4.620000in}{4.620000in}}%
\pgfusepath{clip}%
\pgfsetbuttcap%
\pgfsetroundjoin%
\definecolor{currentfill}{rgb}{1.000000,0.894118,0.788235}%
\pgfsetfillcolor{currentfill}%
\pgfsetlinewidth{0.000000pt}%
\definecolor{currentstroke}{rgb}{1.000000,0.894118,0.788235}%
\pgfsetstrokecolor{currentstroke}%
\pgfsetdash{}{0pt}%
\pgfpathmoveto{\pgfqpoint{2.560403in}{3.262981in}}%
\pgfpathlineto{\pgfqpoint{2.671300in}{3.198955in}}%
\pgfpathlineto{\pgfqpoint{2.651156in}{3.177874in}}%
\pgfpathlineto{\pgfqpoint{2.540259in}{3.241901in}}%
\pgfpathlineto{\pgfqpoint{2.560403in}{3.262981in}}%
\pgfpathclose%
\pgfusepath{fill}%
\end{pgfscope}%
\begin{pgfscope}%
\pgfpathrectangle{\pgfqpoint{0.765000in}{0.660000in}}{\pgfqpoint{4.620000in}{4.620000in}}%
\pgfusepath{clip}%
\pgfsetbuttcap%
\pgfsetroundjoin%
\definecolor{currentfill}{rgb}{1.000000,0.894118,0.788235}%
\pgfsetfillcolor{currentfill}%
\pgfsetlinewidth{0.000000pt}%
\definecolor{currentstroke}{rgb}{1.000000,0.894118,0.788235}%
\pgfsetstrokecolor{currentstroke}%
\pgfsetdash{}{0pt}%
\pgfpathmoveto{\pgfqpoint{2.671300in}{3.198955in}}%
\pgfpathlineto{\pgfqpoint{2.782196in}{3.262981in}}%
\pgfpathlineto{\pgfqpoint{2.762052in}{3.241901in}}%
\pgfpathlineto{\pgfqpoint{2.651156in}{3.177874in}}%
\pgfpathlineto{\pgfqpoint{2.671300in}{3.198955in}}%
\pgfpathclose%
\pgfusepath{fill}%
\end{pgfscope}%
\begin{pgfscope}%
\pgfpathrectangle{\pgfqpoint{0.765000in}{0.660000in}}{\pgfqpoint{4.620000in}{4.620000in}}%
\pgfusepath{clip}%
\pgfsetbuttcap%
\pgfsetroundjoin%
\definecolor{currentfill}{rgb}{1.000000,0.894118,0.788235}%
\pgfsetfillcolor{currentfill}%
\pgfsetlinewidth{0.000000pt}%
\definecolor{currentstroke}{rgb}{1.000000,0.894118,0.788235}%
\pgfsetstrokecolor{currentstroke}%
\pgfsetdash{}{0pt}%
\pgfpathmoveto{\pgfqpoint{2.671300in}{3.455060in}}%
\pgfpathlineto{\pgfqpoint{2.560403in}{3.391034in}}%
\pgfpathlineto{\pgfqpoint{2.540259in}{3.369953in}}%
\pgfpathlineto{\pgfqpoint{2.651156in}{3.433979in}}%
\pgfpathlineto{\pgfqpoint{2.671300in}{3.455060in}}%
\pgfpathclose%
\pgfusepath{fill}%
\end{pgfscope}%
\begin{pgfscope}%
\pgfpathrectangle{\pgfqpoint{0.765000in}{0.660000in}}{\pgfqpoint{4.620000in}{4.620000in}}%
\pgfusepath{clip}%
\pgfsetbuttcap%
\pgfsetroundjoin%
\definecolor{currentfill}{rgb}{1.000000,0.894118,0.788235}%
\pgfsetfillcolor{currentfill}%
\pgfsetlinewidth{0.000000pt}%
\definecolor{currentstroke}{rgb}{1.000000,0.894118,0.788235}%
\pgfsetstrokecolor{currentstroke}%
\pgfsetdash{}{0pt}%
\pgfpathmoveto{\pgfqpoint{2.782196in}{3.391034in}}%
\pgfpathlineto{\pgfqpoint{2.671300in}{3.455060in}}%
\pgfpathlineto{\pgfqpoint{2.651156in}{3.433979in}}%
\pgfpathlineto{\pgfqpoint{2.762052in}{3.369953in}}%
\pgfpathlineto{\pgfqpoint{2.782196in}{3.391034in}}%
\pgfpathclose%
\pgfusepath{fill}%
\end{pgfscope}%
\begin{pgfscope}%
\pgfpathrectangle{\pgfqpoint{0.765000in}{0.660000in}}{\pgfqpoint{4.620000in}{4.620000in}}%
\pgfusepath{clip}%
\pgfsetbuttcap%
\pgfsetroundjoin%
\definecolor{currentfill}{rgb}{1.000000,0.894118,0.788235}%
\pgfsetfillcolor{currentfill}%
\pgfsetlinewidth{0.000000pt}%
\definecolor{currentstroke}{rgb}{1.000000,0.894118,0.788235}%
\pgfsetstrokecolor{currentstroke}%
\pgfsetdash{}{0pt}%
\pgfpathmoveto{\pgfqpoint{2.560403in}{3.262981in}}%
\pgfpathlineto{\pgfqpoint{2.560403in}{3.391034in}}%
\pgfpathlineto{\pgfqpoint{2.540259in}{3.369953in}}%
\pgfpathlineto{\pgfqpoint{2.651156in}{3.177874in}}%
\pgfpathlineto{\pgfqpoint{2.560403in}{3.262981in}}%
\pgfpathclose%
\pgfusepath{fill}%
\end{pgfscope}%
\begin{pgfscope}%
\pgfpathrectangle{\pgfqpoint{0.765000in}{0.660000in}}{\pgfqpoint{4.620000in}{4.620000in}}%
\pgfusepath{clip}%
\pgfsetbuttcap%
\pgfsetroundjoin%
\definecolor{currentfill}{rgb}{1.000000,0.894118,0.788235}%
\pgfsetfillcolor{currentfill}%
\pgfsetlinewidth{0.000000pt}%
\definecolor{currentstroke}{rgb}{1.000000,0.894118,0.788235}%
\pgfsetstrokecolor{currentstroke}%
\pgfsetdash{}{0pt}%
\pgfpathmoveto{\pgfqpoint{2.671300in}{3.198955in}}%
\pgfpathlineto{\pgfqpoint{2.671300in}{3.327008in}}%
\pgfpathlineto{\pgfqpoint{2.651156in}{3.305927in}}%
\pgfpathlineto{\pgfqpoint{2.540259in}{3.241901in}}%
\pgfpathlineto{\pgfqpoint{2.671300in}{3.198955in}}%
\pgfpathclose%
\pgfusepath{fill}%
\end{pgfscope}%
\begin{pgfscope}%
\pgfpathrectangle{\pgfqpoint{0.765000in}{0.660000in}}{\pgfqpoint{4.620000in}{4.620000in}}%
\pgfusepath{clip}%
\pgfsetbuttcap%
\pgfsetroundjoin%
\definecolor{currentfill}{rgb}{1.000000,0.894118,0.788235}%
\pgfsetfillcolor{currentfill}%
\pgfsetlinewidth{0.000000pt}%
\definecolor{currentstroke}{rgb}{1.000000,0.894118,0.788235}%
\pgfsetstrokecolor{currentstroke}%
\pgfsetdash{}{0pt}%
\pgfpathmoveto{\pgfqpoint{2.671300in}{3.327008in}}%
\pgfpathlineto{\pgfqpoint{2.782196in}{3.391034in}}%
\pgfpathlineto{\pgfqpoint{2.762052in}{3.369953in}}%
\pgfpathlineto{\pgfqpoint{2.651156in}{3.305927in}}%
\pgfpathlineto{\pgfqpoint{2.671300in}{3.327008in}}%
\pgfpathclose%
\pgfusepath{fill}%
\end{pgfscope}%
\begin{pgfscope}%
\pgfpathrectangle{\pgfqpoint{0.765000in}{0.660000in}}{\pgfqpoint{4.620000in}{4.620000in}}%
\pgfusepath{clip}%
\pgfsetbuttcap%
\pgfsetroundjoin%
\definecolor{currentfill}{rgb}{1.000000,0.894118,0.788235}%
\pgfsetfillcolor{currentfill}%
\pgfsetlinewidth{0.000000pt}%
\definecolor{currentstroke}{rgb}{1.000000,0.894118,0.788235}%
\pgfsetstrokecolor{currentstroke}%
\pgfsetdash{}{0pt}%
\pgfpathmoveto{\pgfqpoint{2.560403in}{3.391034in}}%
\pgfpathlineto{\pgfqpoint{2.671300in}{3.327008in}}%
\pgfpathlineto{\pgfqpoint{2.651156in}{3.305927in}}%
\pgfpathlineto{\pgfqpoint{2.540259in}{3.369953in}}%
\pgfpathlineto{\pgfqpoint{2.560403in}{3.391034in}}%
\pgfpathclose%
\pgfusepath{fill}%
\end{pgfscope}%
\begin{pgfscope}%
\pgfpathrectangle{\pgfqpoint{0.765000in}{0.660000in}}{\pgfqpoint{4.620000in}{4.620000in}}%
\pgfusepath{clip}%
\pgfsetbuttcap%
\pgfsetroundjoin%
\definecolor{currentfill}{rgb}{1.000000,0.894118,0.788235}%
\pgfsetfillcolor{currentfill}%
\pgfsetlinewidth{0.000000pt}%
\definecolor{currentstroke}{rgb}{1.000000,0.894118,0.788235}%
\pgfsetstrokecolor{currentstroke}%
\pgfsetdash{}{0pt}%
\pgfpathmoveto{\pgfqpoint{3.293363in}{2.546452in}}%
\pgfpathlineto{\pgfqpoint{3.182466in}{2.482426in}}%
\pgfpathlineto{\pgfqpoint{3.293363in}{2.418400in}}%
\pgfpathlineto{\pgfqpoint{3.404259in}{2.482426in}}%
\pgfpathlineto{\pgfqpoint{3.293363in}{2.546452in}}%
\pgfpathclose%
\pgfusepath{fill}%
\end{pgfscope}%
\begin{pgfscope}%
\pgfpathrectangle{\pgfqpoint{0.765000in}{0.660000in}}{\pgfqpoint{4.620000in}{4.620000in}}%
\pgfusepath{clip}%
\pgfsetbuttcap%
\pgfsetroundjoin%
\definecolor{currentfill}{rgb}{1.000000,0.894118,0.788235}%
\pgfsetfillcolor{currentfill}%
\pgfsetlinewidth{0.000000pt}%
\definecolor{currentstroke}{rgb}{1.000000,0.894118,0.788235}%
\pgfsetstrokecolor{currentstroke}%
\pgfsetdash{}{0pt}%
\pgfpathmoveto{\pgfqpoint{3.293363in}{2.546452in}}%
\pgfpathlineto{\pgfqpoint{3.182466in}{2.482426in}}%
\pgfpathlineto{\pgfqpoint{3.182466in}{2.610478in}}%
\pgfpathlineto{\pgfqpoint{3.293363in}{2.674505in}}%
\pgfpathlineto{\pgfqpoint{3.293363in}{2.546452in}}%
\pgfpathclose%
\pgfusepath{fill}%
\end{pgfscope}%
\begin{pgfscope}%
\pgfpathrectangle{\pgfqpoint{0.765000in}{0.660000in}}{\pgfqpoint{4.620000in}{4.620000in}}%
\pgfusepath{clip}%
\pgfsetbuttcap%
\pgfsetroundjoin%
\definecolor{currentfill}{rgb}{1.000000,0.894118,0.788235}%
\pgfsetfillcolor{currentfill}%
\pgfsetlinewidth{0.000000pt}%
\definecolor{currentstroke}{rgb}{1.000000,0.894118,0.788235}%
\pgfsetstrokecolor{currentstroke}%
\pgfsetdash{}{0pt}%
\pgfpathmoveto{\pgfqpoint{3.293363in}{2.546452in}}%
\pgfpathlineto{\pgfqpoint{3.404259in}{2.482426in}}%
\pgfpathlineto{\pgfqpoint{3.404259in}{2.610478in}}%
\pgfpathlineto{\pgfqpoint{3.293363in}{2.674505in}}%
\pgfpathlineto{\pgfqpoint{3.293363in}{2.546452in}}%
\pgfpathclose%
\pgfusepath{fill}%
\end{pgfscope}%
\begin{pgfscope}%
\pgfpathrectangle{\pgfqpoint{0.765000in}{0.660000in}}{\pgfqpoint{4.620000in}{4.620000in}}%
\pgfusepath{clip}%
\pgfsetbuttcap%
\pgfsetroundjoin%
\definecolor{currentfill}{rgb}{1.000000,0.894118,0.788235}%
\pgfsetfillcolor{currentfill}%
\pgfsetlinewidth{0.000000pt}%
\definecolor{currentstroke}{rgb}{1.000000,0.894118,0.788235}%
\pgfsetstrokecolor{currentstroke}%
\pgfsetdash{}{0pt}%
\pgfpathmoveto{\pgfqpoint{3.242912in}{2.518793in}}%
\pgfpathlineto{\pgfqpoint{3.132016in}{2.454767in}}%
\pgfpathlineto{\pgfqpoint{3.242912in}{2.390740in}}%
\pgfpathlineto{\pgfqpoint{3.353809in}{2.454767in}}%
\pgfpathlineto{\pgfqpoint{3.242912in}{2.518793in}}%
\pgfpathclose%
\pgfusepath{fill}%
\end{pgfscope}%
\begin{pgfscope}%
\pgfpathrectangle{\pgfqpoint{0.765000in}{0.660000in}}{\pgfqpoint{4.620000in}{4.620000in}}%
\pgfusepath{clip}%
\pgfsetbuttcap%
\pgfsetroundjoin%
\definecolor{currentfill}{rgb}{1.000000,0.894118,0.788235}%
\pgfsetfillcolor{currentfill}%
\pgfsetlinewidth{0.000000pt}%
\definecolor{currentstroke}{rgb}{1.000000,0.894118,0.788235}%
\pgfsetstrokecolor{currentstroke}%
\pgfsetdash{}{0pt}%
\pgfpathmoveto{\pgfqpoint{3.242912in}{2.518793in}}%
\pgfpathlineto{\pgfqpoint{3.132016in}{2.454767in}}%
\pgfpathlineto{\pgfqpoint{3.132016in}{2.582819in}}%
\pgfpathlineto{\pgfqpoint{3.242912in}{2.646845in}}%
\pgfpathlineto{\pgfqpoint{3.242912in}{2.518793in}}%
\pgfpathclose%
\pgfusepath{fill}%
\end{pgfscope}%
\begin{pgfscope}%
\pgfpathrectangle{\pgfqpoint{0.765000in}{0.660000in}}{\pgfqpoint{4.620000in}{4.620000in}}%
\pgfusepath{clip}%
\pgfsetbuttcap%
\pgfsetroundjoin%
\definecolor{currentfill}{rgb}{1.000000,0.894118,0.788235}%
\pgfsetfillcolor{currentfill}%
\pgfsetlinewidth{0.000000pt}%
\definecolor{currentstroke}{rgb}{1.000000,0.894118,0.788235}%
\pgfsetstrokecolor{currentstroke}%
\pgfsetdash{}{0pt}%
\pgfpathmoveto{\pgfqpoint{3.242912in}{2.518793in}}%
\pgfpathlineto{\pgfqpoint{3.353809in}{2.454767in}}%
\pgfpathlineto{\pgfqpoint{3.353809in}{2.582819in}}%
\pgfpathlineto{\pgfqpoint{3.242912in}{2.646845in}}%
\pgfpathlineto{\pgfqpoint{3.242912in}{2.518793in}}%
\pgfpathclose%
\pgfusepath{fill}%
\end{pgfscope}%
\begin{pgfscope}%
\pgfpathrectangle{\pgfqpoint{0.765000in}{0.660000in}}{\pgfqpoint{4.620000in}{4.620000in}}%
\pgfusepath{clip}%
\pgfsetbuttcap%
\pgfsetroundjoin%
\definecolor{currentfill}{rgb}{1.000000,0.894118,0.788235}%
\pgfsetfillcolor{currentfill}%
\pgfsetlinewidth{0.000000pt}%
\definecolor{currentstroke}{rgb}{1.000000,0.894118,0.788235}%
\pgfsetstrokecolor{currentstroke}%
\pgfsetdash{}{0pt}%
\pgfpathmoveto{\pgfqpoint{3.293363in}{2.674505in}}%
\pgfpathlineto{\pgfqpoint{3.182466in}{2.610478in}}%
\pgfpathlineto{\pgfqpoint{3.293363in}{2.546452in}}%
\pgfpathlineto{\pgfqpoint{3.404259in}{2.610478in}}%
\pgfpathlineto{\pgfqpoint{3.293363in}{2.674505in}}%
\pgfpathclose%
\pgfusepath{fill}%
\end{pgfscope}%
\begin{pgfscope}%
\pgfpathrectangle{\pgfqpoint{0.765000in}{0.660000in}}{\pgfqpoint{4.620000in}{4.620000in}}%
\pgfusepath{clip}%
\pgfsetbuttcap%
\pgfsetroundjoin%
\definecolor{currentfill}{rgb}{1.000000,0.894118,0.788235}%
\pgfsetfillcolor{currentfill}%
\pgfsetlinewidth{0.000000pt}%
\definecolor{currentstroke}{rgb}{1.000000,0.894118,0.788235}%
\pgfsetstrokecolor{currentstroke}%
\pgfsetdash{}{0pt}%
\pgfpathmoveto{\pgfqpoint{3.293363in}{2.418400in}}%
\pgfpathlineto{\pgfqpoint{3.404259in}{2.482426in}}%
\pgfpathlineto{\pgfqpoint{3.404259in}{2.610478in}}%
\pgfpathlineto{\pgfqpoint{3.293363in}{2.546452in}}%
\pgfpathlineto{\pgfqpoint{3.293363in}{2.418400in}}%
\pgfpathclose%
\pgfusepath{fill}%
\end{pgfscope}%
\begin{pgfscope}%
\pgfpathrectangle{\pgfqpoint{0.765000in}{0.660000in}}{\pgfqpoint{4.620000in}{4.620000in}}%
\pgfusepath{clip}%
\pgfsetbuttcap%
\pgfsetroundjoin%
\definecolor{currentfill}{rgb}{1.000000,0.894118,0.788235}%
\pgfsetfillcolor{currentfill}%
\pgfsetlinewidth{0.000000pt}%
\definecolor{currentstroke}{rgb}{1.000000,0.894118,0.788235}%
\pgfsetstrokecolor{currentstroke}%
\pgfsetdash{}{0pt}%
\pgfpathmoveto{\pgfqpoint{3.182466in}{2.482426in}}%
\pgfpathlineto{\pgfqpoint{3.293363in}{2.418400in}}%
\pgfpathlineto{\pgfqpoint{3.293363in}{2.546452in}}%
\pgfpathlineto{\pgfqpoint{3.182466in}{2.610478in}}%
\pgfpathlineto{\pgfqpoint{3.182466in}{2.482426in}}%
\pgfpathclose%
\pgfusepath{fill}%
\end{pgfscope}%
\begin{pgfscope}%
\pgfpathrectangle{\pgfqpoint{0.765000in}{0.660000in}}{\pgfqpoint{4.620000in}{4.620000in}}%
\pgfusepath{clip}%
\pgfsetbuttcap%
\pgfsetroundjoin%
\definecolor{currentfill}{rgb}{1.000000,0.894118,0.788235}%
\pgfsetfillcolor{currentfill}%
\pgfsetlinewidth{0.000000pt}%
\definecolor{currentstroke}{rgb}{1.000000,0.894118,0.788235}%
\pgfsetstrokecolor{currentstroke}%
\pgfsetdash{}{0pt}%
\pgfpathmoveto{\pgfqpoint{3.242912in}{2.646845in}}%
\pgfpathlineto{\pgfqpoint{3.132016in}{2.582819in}}%
\pgfpathlineto{\pgfqpoint{3.242912in}{2.518793in}}%
\pgfpathlineto{\pgfqpoint{3.353809in}{2.582819in}}%
\pgfpathlineto{\pgfqpoint{3.242912in}{2.646845in}}%
\pgfpathclose%
\pgfusepath{fill}%
\end{pgfscope}%
\begin{pgfscope}%
\pgfpathrectangle{\pgfqpoint{0.765000in}{0.660000in}}{\pgfqpoint{4.620000in}{4.620000in}}%
\pgfusepath{clip}%
\pgfsetbuttcap%
\pgfsetroundjoin%
\definecolor{currentfill}{rgb}{1.000000,0.894118,0.788235}%
\pgfsetfillcolor{currentfill}%
\pgfsetlinewidth{0.000000pt}%
\definecolor{currentstroke}{rgb}{1.000000,0.894118,0.788235}%
\pgfsetstrokecolor{currentstroke}%
\pgfsetdash{}{0pt}%
\pgfpathmoveto{\pgfqpoint{3.242912in}{2.390740in}}%
\pgfpathlineto{\pgfqpoint{3.353809in}{2.454767in}}%
\pgfpathlineto{\pgfqpoint{3.353809in}{2.582819in}}%
\pgfpathlineto{\pgfqpoint{3.242912in}{2.518793in}}%
\pgfpathlineto{\pgfqpoint{3.242912in}{2.390740in}}%
\pgfpathclose%
\pgfusepath{fill}%
\end{pgfscope}%
\begin{pgfscope}%
\pgfpathrectangle{\pgfqpoint{0.765000in}{0.660000in}}{\pgfqpoint{4.620000in}{4.620000in}}%
\pgfusepath{clip}%
\pgfsetbuttcap%
\pgfsetroundjoin%
\definecolor{currentfill}{rgb}{1.000000,0.894118,0.788235}%
\pgfsetfillcolor{currentfill}%
\pgfsetlinewidth{0.000000pt}%
\definecolor{currentstroke}{rgb}{1.000000,0.894118,0.788235}%
\pgfsetstrokecolor{currentstroke}%
\pgfsetdash{}{0pt}%
\pgfpathmoveto{\pgfqpoint{3.132016in}{2.454767in}}%
\pgfpathlineto{\pgfqpoint{3.242912in}{2.390740in}}%
\pgfpathlineto{\pgfqpoint{3.242912in}{2.518793in}}%
\pgfpathlineto{\pgfqpoint{3.132016in}{2.582819in}}%
\pgfpathlineto{\pgfqpoint{3.132016in}{2.454767in}}%
\pgfpathclose%
\pgfusepath{fill}%
\end{pgfscope}%
\begin{pgfscope}%
\pgfpathrectangle{\pgfqpoint{0.765000in}{0.660000in}}{\pgfqpoint{4.620000in}{4.620000in}}%
\pgfusepath{clip}%
\pgfsetbuttcap%
\pgfsetroundjoin%
\definecolor{currentfill}{rgb}{1.000000,0.894118,0.788235}%
\pgfsetfillcolor{currentfill}%
\pgfsetlinewidth{0.000000pt}%
\definecolor{currentstroke}{rgb}{1.000000,0.894118,0.788235}%
\pgfsetstrokecolor{currentstroke}%
\pgfsetdash{}{0pt}%
\pgfpathmoveto{\pgfqpoint{3.293363in}{2.546452in}}%
\pgfpathlineto{\pgfqpoint{3.182466in}{2.482426in}}%
\pgfpathlineto{\pgfqpoint{3.132016in}{2.454767in}}%
\pgfpathlineto{\pgfqpoint{3.242912in}{2.518793in}}%
\pgfpathlineto{\pgfqpoint{3.293363in}{2.546452in}}%
\pgfpathclose%
\pgfusepath{fill}%
\end{pgfscope}%
\begin{pgfscope}%
\pgfpathrectangle{\pgfqpoint{0.765000in}{0.660000in}}{\pgfqpoint{4.620000in}{4.620000in}}%
\pgfusepath{clip}%
\pgfsetbuttcap%
\pgfsetroundjoin%
\definecolor{currentfill}{rgb}{1.000000,0.894118,0.788235}%
\pgfsetfillcolor{currentfill}%
\pgfsetlinewidth{0.000000pt}%
\definecolor{currentstroke}{rgb}{1.000000,0.894118,0.788235}%
\pgfsetstrokecolor{currentstroke}%
\pgfsetdash{}{0pt}%
\pgfpathmoveto{\pgfqpoint{3.404259in}{2.482426in}}%
\pgfpathlineto{\pgfqpoint{3.293363in}{2.546452in}}%
\pgfpathlineto{\pgfqpoint{3.242912in}{2.518793in}}%
\pgfpathlineto{\pgfqpoint{3.353809in}{2.454767in}}%
\pgfpathlineto{\pgfqpoint{3.404259in}{2.482426in}}%
\pgfpathclose%
\pgfusepath{fill}%
\end{pgfscope}%
\begin{pgfscope}%
\pgfpathrectangle{\pgfqpoint{0.765000in}{0.660000in}}{\pgfqpoint{4.620000in}{4.620000in}}%
\pgfusepath{clip}%
\pgfsetbuttcap%
\pgfsetroundjoin%
\definecolor{currentfill}{rgb}{1.000000,0.894118,0.788235}%
\pgfsetfillcolor{currentfill}%
\pgfsetlinewidth{0.000000pt}%
\definecolor{currentstroke}{rgb}{1.000000,0.894118,0.788235}%
\pgfsetstrokecolor{currentstroke}%
\pgfsetdash{}{0pt}%
\pgfpathmoveto{\pgfqpoint{3.293363in}{2.546452in}}%
\pgfpathlineto{\pgfqpoint{3.293363in}{2.674505in}}%
\pgfpathlineto{\pgfqpoint{3.242912in}{2.646845in}}%
\pgfpathlineto{\pgfqpoint{3.353809in}{2.454767in}}%
\pgfpathlineto{\pgfqpoint{3.293363in}{2.546452in}}%
\pgfpathclose%
\pgfusepath{fill}%
\end{pgfscope}%
\begin{pgfscope}%
\pgfpathrectangle{\pgfqpoint{0.765000in}{0.660000in}}{\pgfqpoint{4.620000in}{4.620000in}}%
\pgfusepath{clip}%
\pgfsetbuttcap%
\pgfsetroundjoin%
\definecolor{currentfill}{rgb}{1.000000,0.894118,0.788235}%
\pgfsetfillcolor{currentfill}%
\pgfsetlinewidth{0.000000pt}%
\definecolor{currentstroke}{rgb}{1.000000,0.894118,0.788235}%
\pgfsetstrokecolor{currentstroke}%
\pgfsetdash{}{0pt}%
\pgfpathmoveto{\pgfqpoint{3.404259in}{2.482426in}}%
\pgfpathlineto{\pgfqpoint{3.404259in}{2.610478in}}%
\pgfpathlineto{\pgfqpoint{3.353809in}{2.582819in}}%
\pgfpathlineto{\pgfqpoint{3.242912in}{2.518793in}}%
\pgfpathlineto{\pgfqpoint{3.404259in}{2.482426in}}%
\pgfpathclose%
\pgfusepath{fill}%
\end{pgfscope}%
\begin{pgfscope}%
\pgfpathrectangle{\pgfqpoint{0.765000in}{0.660000in}}{\pgfqpoint{4.620000in}{4.620000in}}%
\pgfusepath{clip}%
\pgfsetbuttcap%
\pgfsetroundjoin%
\definecolor{currentfill}{rgb}{1.000000,0.894118,0.788235}%
\pgfsetfillcolor{currentfill}%
\pgfsetlinewidth{0.000000pt}%
\definecolor{currentstroke}{rgb}{1.000000,0.894118,0.788235}%
\pgfsetstrokecolor{currentstroke}%
\pgfsetdash{}{0pt}%
\pgfpathmoveto{\pgfqpoint{3.182466in}{2.482426in}}%
\pgfpathlineto{\pgfqpoint{3.293363in}{2.418400in}}%
\pgfpathlineto{\pgfqpoint{3.242912in}{2.390740in}}%
\pgfpathlineto{\pgfqpoint{3.132016in}{2.454767in}}%
\pgfpathlineto{\pgfqpoint{3.182466in}{2.482426in}}%
\pgfpathclose%
\pgfusepath{fill}%
\end{pgfscope}%
\begin{pgfscope}%
\pgfpathrectangle{\pgfqpoint{0.765000in}{0.660000in}}{\pgfqpoint{4.620000in}{4.620000in}}%
\pgfusepath{clip}%
\pgfsetbuttcap%
\pgfsetroundjoin%
\definecolor{currentfill}{rgb}{1.000000,0.894118,0.788235}%
\pgfsetfillcolor{currentfill}%
\pgfsetlinewidth{0.000000pt}%
\definecolor{currentstroke}{rgb}{1.000000,0.894118,0.788235}%
\pgfsetstrokecolor{currentstroke}%
\pgfsetdash{}{0pt}%
\pgfpathmoveto{\pgfqpoint{3.293363in}{2.418400in}}%
\pgfpathlineto{\pgfqpoint{3.404259in}{2.482426in}}%
\pgfpathlineto{\pgfqpoint{3.353809in}{2.454767in}}%
\pgfpathlineto{\pgfqpoint{3.242912in}{2.390740in}}%
\pgfpathlineto{\pgfqpoint{3.293363in}{2.418400in}}%
\pgfpathclose%
\pgfusepath{fill}%
\end{pgfscope}%
\begin{pgfscope}%
\pgfpathrectangle{\pgfqpoint{0.765000in}{0.660000in}}{\pgfqpoint{4.620000in}{4.620000in}}%
\pgfusepath{clip}%
\pgfsetbuttcap%
\pgfsetroundjoin%
\definecolor{currentfill}{rgb}{1.000000,0.894118,0.788235}%
\pgfsetfillcolor{currentfill}%
\pgfsetlinewidth{0.000000pt}%
\definecolor{currentstroke}{rgb}{1.000000,0.894118,0.788235}%
\pgfsetstrokecolor{currentstroke}%
\pgfsetdash{}{0pt}%
\pgfpathmoveto{\pgfqpoint{3.293363in}{2.674505in}}%
\pgfpathlineto{\pgfqpoint{3.182466in}{2.610478in}}%
\pgfpathlineto{\pgfqpoint{3.132016in}{2.582819in}}%
\pgfpathlineto{\pgfqpoint{3.242912in}{2.646845in}}%
\pgfpathlineto{\pgfqpoint{3.293363in}{2.674505in}}%
\pgfpathclose%
\pgfusepath{fill}%
\end{pgfscope}%
\begin{pgfscope}%
\pgfpathrectangle{\pgfqpoint{0.765000in}{0.660000in}}{\pgfqpoint{4.620000in}{4.620000in}}%
\pgfusepath{clip}%
\pgfsetbuttcap%
\pgfsetroundjoin%
\definecolor{currentfill}{rgb}{1.000000,0.894118,0.788235}%
\pgfsetfillcolor{currentfill}%
\pgfsetlinewidth{0.000000pt}%
\definecolor{currentstroke}{rgb}{1.000000,0.894118,0.788235}%
\pgfsetstrokecolor{currentstroke}%
\pgfsetdash{}{0pt}%
\pgfpathmoveto{\pgfqpoint{3.404259in}{2.610478in}}%
\pgfpathlineto{\pgfqpoint{3.293363in}{2.674505in}}%
\pgfpathlineto{\pgfqpoint{3.242912in}{2.646845in}}%
\pgfpathlineto{\pgfqpoint{3.353809in}{2.582819in}}%
\pgfpathlineto{\pgfqpoint{3.404259in}{2.610478in}}%
\pgfpathclose%
\pgfusepath{fill}%
\end{pgfscope}%
\begin{pgfscope}%
\pgfpathrectangle{\pgfqpoint{0.765000in}{0.660000in}}{\pgfqpoint{4.620000in}{4.620000in}}%
\pgfusepath{clip}%
\pgfsetbuttcap%
\pgfsetroundjoin%
\definecolor{currentfill}{rgb}{1.000000,0.894118,0.788235}%
\pgfsetfillcolor{currentfill}%
\pgfsetlinewidth{0.000000pt}%
\definecolor{currentstroke}{rgb}{1.000000,0.894118,0.788235}%
\pgfsetstrokecolor{currentstroke}%
\pgfsetdash{}{0pt}%
\pgfpathmoveto{\pgfqpoint{3.182466in}{2.482426in}}%
\pgfpathlineto{\pgfqpoint{3.182466in}{2.610478in}}%
\pgfpathlineto{\pgfqpoint{3.132016in}{2.582819in}}%
\pgfpathlineto{\pgfqpoint{3.242912in}{2.390740in}}%
\pgfpathlineto{\pgfqpoint{3.182466in}{2.482426in}}%
\pgfpathclose%
\pgfusepath{fill}%
\end{pgfscope}%
\begin{pgfscope}%
\pgfpathrectangle{\pgfqpoint{0.765000in}{0.660000in}}{\pgfqpoint{4.620000in}{4.620000in}}%
\pgfusepath{clip}%
\pgfsetbuttcap%
\pgfsetroundjoin%
\definecolor{currentfill}{rgb}{1.000000,0.894118,0.788235}%
\pgfsetfillcolor{currentfill}%
\pgfsetlinewidth{0.000000pt}%
\definecolor{currentstroke}{rgb}{1.000000,0.894118,0.788235}%
\pgfsetstrokecolor{currentstroke}%
\pgfsetdash{}{0pt}%
\pgfpathmoveto{\pgfqpoint{3.293363in}{2.418400in}}%
\pgfpathlineto{\pgfqpoint{3.293363in}{2.546452in}}%
\pgfpathlineto{\pgfqpoint{3.242912in}{2.518793in}}%
\pgfpathlineto{\pgfqpoint{3.132016in}{2.454767in}}%
\pgfpathlineto{\pgfqpoint{3.293363in}{2.418400in}}%
\pgfpathclose%
\pgfusepath{fill}%
\end{pgfscope}%
\begin{pgfscope}%
\pgfpathrectangle{\pgfqpoint{0.765000in}{0.660000in}}{\pgfqpoint{4.620000in}{4.620000in}}%
\pgfusepath{clip}%
\pgfsetbuttcap%
\pgfsetroundjoin%
\definecolor{currentfill}{rgb}{1.000000,0.894118,0.788235}%
\pgfsetfillcolor{currentfill}%
\pgfsetlinewidth{0.000000pt}%
\definecolor{currentstroke}{rgb}{1.000000,0.894118,0.788235}%
\pgfsetstrokecolor{currentstroke}%
\pgfsetdash{}{0pt}%
\pgfpathmoveto{\pgfqpoint{3.293363in}{2.546452in}}%
\pgfpathlineto{\pgfqpoint{3.404259in}{2.610478in}}%
\pgfpathlineto{\pgfqpoint{3.353809in}{2.582819in}}%
\pgfpathlineto{\pgfqpoint{3.242912in}{2.518793in}}%
\pgfpathlineto{\pgfqpoint{3.293363in}{2.546452in}}%
\pgfpathclose%
\pgfusepath{fill}%
\end{pgfscope}%
\begin{pgfscope}%
\pgfpathrectangle{\pgfqpoint{0.765000in}{0.660000in}}{\pgfqpoint{4.620000in}{4.620000in}}%
\pgfusepath{clip}%
\pgfsetbuttcap%
\pgfsetroundjoin%
\definecolor{currentfill}{rgb}{1.000000,0.894118,0.788235}%
\pgfsetfillcolor{currentfill}%
\pgfsetlinewidth{0.000000pt}%
\definecolor{currentstroke}{rgb}{1.000000,0.894118,0.788235}%
\pgfsetstrokecolor{currentstroke}%
\pgfsetdash{}{0pt}%
\pgfpathmoveto{\pgfqpoint{3.182466in}{2.610478in}}%
\pgfpathlineto{\pgfqpoint{3.293363in}{2.546452in}}%
\pgfpathlineto{\pgfqpoint{3.242912in}{2.518793in}}%
\pgfpathlineto{\pgfqpoint{3.132016in}{2.582819in}}%
\pgfpathlineto{\pgfqpoint{3.182466in}{2.610478in}}%
\pgfpathclose%
\pgfusepath{fill}%
\end{pgfscope}%
\begin{pgfscope}%
\pgfpathrectangle{\pgfqpoint{0.765000in}{0.660000in}}{\pgfqpoint{4.620000in}{4.620000in}}%
\pgfusepath{clip}%
\pgfsetbuttcap%
\pgfsetroundjoin%
\definecolor{currentfill}{rgb}{1.000000,0.894118,0.788235}%
\pgfsetfillcolor{currentfill}%
\pgfsetlinewidth{0.000000pt}%
\definecolor{currentstroke}{rgb}{1.000000,0.894118,0.788235}%
\pgfsetstrokecolor{currentstroke}%
\pgfsetdash{}{0pt}%
\pgfpathmoveto{\pgfqpoint{3.293363in}{2.546452in}}%
\pgfpathlineto{\pgfqpoint{3.182466in}{2.482426in}}%
\pgfpathlineto{\pgfqpoint{3.293363in}{2.418400in}}%
\pgfpathlineto{\pgfqpoint{3.404259in}{2.482426in}}%
\pgfpathlineto{\pgfqpoint{3.293363in}{2.546452in}}%
\pgfpathclose%
\pgfusepath{fill}%
\end{pgfscope}%
\begin{pgfscope}%
\pgfpathrectangle{\pgfqpoint{0.765000in}{0.660000in}}{\pgfqpoint{4.620000in}{4.620000in}}%
\pgfusepath{clip}%
\pgfsetbuttcap%
\pgfsetroundjoin%
\definecolor{currentfill}{rgb}{1.000000,0.894118,0.788235}%
\pgfsetfillcolor{currentfill}%
\pgfsetlinewidth{0.000000pt}%
\definecolor{currentstroke}{rgb}{1.000000,0.894118,0.788235}%
\pgfsetstrokecolor{currentstroke}%
\pgfsetdash{}{0pt}%
\pgfpathmoveto{\pgfqpoint{3.293363in}{2.546452in}}%
\pgfpathlineto{\pgfqpoint{3.182466in}{2.482426in}}%
\pgfpathlineto{\pgfqpoint{3.182466in}{2.610478in}}%
\pgfpathlineto{\pgfqpoint{3.293363in}{2.674505in}}%
\pgfpathlineto{\pgfqpoint{3.293363in}{2.546452in}}%
\pgfpathclose%
\pgfusepath{fill}%
\end{pgfscope}%
\begin{pgfscope}%
\pgfpathrectangle{\pgfqpoint{0.765000in}{0.660000in}}{\pgfqpoint{4.620000in}{4.620000in}}%
\pgfusepath{clip}%
\pgfsetbuttcap%
\pgfsetroundjoin%
\definecolor{currentfill}{rgb}{1.000000,0.894118,0.788235}%
\pgfsetfillcolor{currentfill}%
\pgfsetlinewidth{0.000000pt}%
\definecolor{currentstroke}{rgb}{1.000000,0.894118,0.788235}%
\pgfsetstrokecolor{currentstroke}%
\pgfsetdash{}{0pt}%
\pgfpathmoveto{\pgfqpoint{3.293363in}{2.546452in}}%
\pgfpathlineto{\pgfqpoint{3.404259in}{2.482426in}}%
\pgfpathlineto{\pgfqpoint{3.404259in}{2.610478in}}%
\pgfpathlineto{\pgfqpoint{3.293363in}{2.674505in}}%
\pgfpathlineto{\pgfqpoint{3.293363in}{2.546452in}}%
\pgfpathclose%
\pgfusepath{fill}%
\end{pgfscope}%
\begin{pgfscope}%
\pgfpathrectangle{\pgfqpoint{0.765000in}{0.660000in}}{\pgfqpoint{4.620000in}{4.620000in}}%
\pgfusepath{clip}%
\pgfsetbuttcap%
\pgfsetroundjoin%
\definecolor{currentfill}{rgb}{1.000000,0.894118,0.788235}%
\pgfsetfillcolor{currentfill}%
\pgfsetlinewidth{0.000000pt}%
\definecolor{currentstroke}{rgb}{1.000000,0.894118,0.788235}%
\pgfsetstrokecolor{currentstroke}%
\pgfsetdash{}{0pt}%
\pgfpathmoveto{\pgfqpoint{3.463186in}{2.662942in}}%
\pgfpathlineto{\pgfqpoint{3.352289in}{2.598916in}}%
\pgfpathlineto{\pgfqpoint{3.463186in}{2.534890in}}%
\pgfpathlineto{\pgfqpoint{3.574082in}{2.598916in}}%
\pgfpathlineto{\pgfqpoint{3.463186in}{2.662942in}}%
\pgfpathclose%
\pgfusepath{fill}%
\end{pgfscope}%
\begin{pgfscope}%
\pgfpathrectangle{\pgfqpoint{0.765000in}{0.660000in}}{\pgfqpoint{4.620000in}{4.620000in}}%
\pgfusepath{clip}%
\pgfsetbuttcap%
\pgfsetroundjoin%
\definecolor{currentfill}{rgb}{1.000000,0.894118,0.788235}%
\pgfsetfillcolor{currentfill}%
\pgfsetlinewidth{0.000000pt}%
\definecolor{currentstroke}{rgb}{1.000000,0.894118,0.788235}%
\pgfsetstrokecolor{currentstroke}%
\pgfsetdash{}{0pt}%
\pgfpathmoveto{\pgfqpoint{3.463186in}{2.662942in}}%
\pgfpathlineto{\pgfqpoint{3.352289in}{2.598916in}}%
\pgfpathlineto{\pgfqpoint{3.352289in}{2.726968in}}%
\pgfpathlineto{\pgfqpoint{3.463186in}{2.790995in}}%
\pgfpathlineto{\pgfqpoint{3.463186in}{2.662942in}}%
\pgfpathclose%
\pgfusepath{fill}%
\end{pgfscope}%
\begin{pgfscope}%
\pgfpathrectangle{\pgfqpoint{0.765000in}{0.660000in}}{\pgfqpoint{4.620000in}{4.620000in}}%
\pgfusepath{clip}%
\pgfsetbuttcap%
\pgfsetroundjoin%
\definecolor{currentfill}{rgb}{1.000000,0.894118,0.788235}%
\pgfsetfillcolor{currentfill}%
\pgfsetlinewidth{0.000000pt}%
\definecolor{currentstroke}{rgb}{1.000000,0.894118,0.788235}%
\pgfsetstrokecolor{currentstroke}%
\pgfsetdash{}{0pt}%
\pgfpathmoveto{\pgfqpoint{3.463186in}{2.662942in}}%
\pgfpathlineto{\pgfqpoint{3.574082in}{2.598916in}}%
\pgfpathlineto{\pgfqpoint{3.574082in}{2.726968in}}%
\pgfpathlineto{\pgfqpoint{3.463186in}{2.790995in}}%
\pgfpathlineto{\pgfqpoint{3.463186in}{2.662942in}}%
\pgfpathclose%
\pgfusepath{fill}%
\end{pgfscope}%
\begin{pgfscope}%
\pgfpathrectangle{\pgfqpoint{0.765000in}{0.660000in}}{\pgfqpoint{4.620000in}{4.620000in}}%
\pgfusepath{clip}%
\pgfsetbuttcap%
\pgfsetroundjoin%
\definecolor{currentfill}{rgb}{1.000000,0.894118,0.788235}%
\pgfsetfillcolor{currentfill}%
\pgfsetlinewidth{0.000000pt}%
\definecolor{currentstroke}{rgb}{1.000000,0.894118,0.788235}%
\pgfsetstrokecolor{currentstroke}%
\pgfsetdash{}{0pt}%
\pgfpathmoveto{\pgfqpoint{3.293363in}{2.674505in}}%
\pgfpathlineto{\pgfqpoint{3.182466in}{2.610478in}}%
\pgfpathlineto{\pgfqpoint{3.293363in}{2.546452in}}%
\pgfpathlineto{\pgfqpoint{3.404259in}{2.610478in}}%
\pgfpathlineto{\pgfqpoint{3.293363in}{2.674505in}}%
\pgfpathclose%
\pgfusepath{fill}%
\end{pgfscope}%
\begin{pgfscope}%
\pgfpathrectangle{\pgfqpoint{0.765000in}{0.660000in}}{\pgfqpoint{4.620000in}{4.620000in}}%
\pgfusepath{clip}%
\pgfsetbuttcap%
\pgfsetroundjoin%
\definecolor{currentfill}{rgb}{1.000000,0.894118,0.788235}%
\pgfsetfillcolor{currentfill}%
\pgfsetlinewidth{0.000000pt}%
\definecolor{currentstroke}{rgb}{1.000000,0.894118,0.788235}%
\pgfsetstrokecolor{currentstroke}%
\pgfsetdash{}{0pt}%
\pgfpathmoveto{\pgfqpoint{3.293363in}{2.418400in}}%
\pgfpathlineto{\pgfqpoint{3.404259in}{2.482426in}}%
\pgfpathlineto{\pgfqpoint{3.404259in}{2.610478in}}%
\pgfpathlineto{\pgfqpoint{3.293363in}{2.546452in}}%
\pgfpathlineto{\pgfqpoint{3.293363in}{2.418400in}}%
\pgfpathclose%
\pgfusepath{fill}%
\end{pgfscope}%
\begin{pgfscope}%
\pgfpathrectangle{\pgfqpoint{0.765000in}{0.660000in}}{\pgfqpoint{4.620000in}{4.620000in}}%
\pgfusepath{clip}%
\pgfsetbuttcap%
\pgfsetroundjoin%
\definecolor{currentfill}{rgb}{1.000000,0.894118,0.788235}%
\pgfsetfillcolor{currentfill}%
\pgfsetlinewidth{0.000000pt}%
\definecolor{currentstroke}{rgb}{1.000000,0.894118,0.788235}%
\pgfsetstrokecolor{currentstroke}%
\pgfsetdash{}{0pt}%
\pgfpathmoveto{\pgfqpoint{3.182466in}{2.482426in}}%
\pgfpathlineto{\pgfqpoint{3.293363in}{2.418400in}}%
\pgfpathlineto{\pgfqpoint{3.293363in}{2.546452in}}%
\pgfpathlineto{\pgfqpoint{3.182466in}{2.610478in}}%
\pgfpathlineto{\pgfqpoint{3.182466in}{2.482426in}}%
\pgfpathclose%
\pgfusepath{fill}%
\end{pgfscope}%
\begin{pgfscope}%
\pgfpathrectangle{\pgfqpoint{0.765000in}{0.660000in}}{\pgfqpoint{4.620000in}{4.620000in}}%
\pgfusepath{clip}%
\pgfsetbuttcap%
\pgfsetroundjoin%
\definecolor{currentfill}{rgb}{1.000000,0.894118,0.788235}%
\pgfsetfillcolor{currentfill}%
\pgfsetlinewidth{0.000000pt}%
\definecolor{currentstroke}{rgb}{1.000000,0.894118,0.788235}%
\pgfsetstrokecolor{currentstroke}%
\pgfsetdash{}{0pt}%
\pgfpathmoveto{\pgfqpoint{3.463186in}{2.790995in}}%
\pgfpathlineto{\pgfqpoint{3.352289in}{2.726968in}}%
\pgfpathlineto{\pgfqpoint{3.463186in}{2.662942in}}%
\pgfpathlineto{\pgfqpoint{3.574082in}{2.726968in}}%
\pgfpathlineto{\pgfqpoint{3.463186in}{2.790995in}}%
\pgfpathclose%
\pgfusepath{fill}%
\end{pgfscope}%
\begin{pgfscope}%
\pgfpathrectangle{\pgfqpoint{0.765000in}{0.660000in}}{\pgfqpoint{4.620000in}{4.620000in}}%
\pgfusepath{clip}%
\pgfsetbuttcap%
\pgfsetroundjoin%
\definecolor{currentfill}{rgb}{1.000000,0.894118,0.788235}%
\pgfsetfillcolor{currentfill}%
\pgfsetlinewidth{0.000000pt}%
\definecolor{currentstroke}{rgb}{1.000000,0.894118,0.788235}%
\pgfsetstrokecolor{currentstroke}%
\pgfsetdash{}{0pt}%
\pgfpathmoveto{\pgfqpoint{3.463186in}{2.534890in}}%
\pgfpathlineto{\pgfqpoint{3.574082in}{2.598916in}}%
\pgfpathlineto{\pgfqpoint{3.574082in}{2.726968in}}%
\pgfpathlineto{\pgfqpoint{3.463186in}{2.662942in}}%
\pgfpathlineto{\pgfqpoint{3.463186in}{2.534890in}}%
\pgfpathclose%
\pgfusepath{fill}%
\end{pgfscope}%
\begin{pgfscope}%
\pgfpathrectangle{\pgfqpoint{0.765000in}{0.660000in}}{\pgfqpoint{4.620000in}{4.620000in}}%
\pgfusepath{clip}%
\pgfsetbuttcap%
\pgfsetroundjoin%
\definecolor{currentfill}{rgb}{1.000000,0.894118,0.788235}%
\pgfsetfillcolor{currentfill}%
\pgfsetlinewidth{0.000000pt}%
\definecolor{currentstroke}{rgb}{1.000000,0.894118,0.788235}%
\pgfsetstrokecolor{currentstroke}%
\pgfsetdash{}{0pt}%
\pgfpathmoveto{\pgfqpoint{3.352289in}{2.598916in}}%
\pgfpathlineto{\pgfqpoint{3.463186in}{2.534890in}}%
\pgfpathlineto{\pgfqpoint{3.463186in}{2.662942in}}%
\pgfpathlineto{\pgfqpoint{3.352289in}{2.726968in}}%
\pgfpathlineto{\pgfqpoint{3.352289in}{2.598916in}}%
\pgfpathclose%
\pgfusepath{fill}%
\end{pgfscope}%
\begin{pgfscope}%
\pgfpathrectangle{\pgfqpoint{0.765000in}{0.660000in}}{\pgfqpoint{4.620000in}{4.620000in}}%
\pgfusepath{clip}%
\pgfsetbuttcap%
\pgfsetroundjoin%
\definecolor{currentfill}{rgb}{1.000000,0.894118,0.788235}%
\pgfsetfillcolor{currentfill}%
\pgfsetlinewidth{0.000000pt}%
\definecolor{currentstroke}{rgb}{1.000000,0.894118,0.788235}%
\pgfsetstrokecolor{currentstroke}%
\pgfsetdash{}{0pt}%
\pgfpathmoveto{\pgfqpoint{3.293363in}{2.546452in}}%
\pgfpathlineto{\pgfqpoint{3.182466in}{2.482426in}}%
\pgfpathlineto{\pgfqpoint{3.352289in}{2.598916in}}%
\pgfpathlineto{\pgfqpoint{3.463186in}{2.662942in}}%
\pgfpathlineto{\pgfqpoint{3.293363in}{2.546452in}}%
\pgfpathclose%
\pgfusepath{fill}%
\end{pgfscope}%
\begin{pgfscope}%
\pgfpathrectangle{\pgfqpoint{0.765000in}{0.660000in}}{\pgfqpoint{4.620000in}{4.620000in}}%
\pgfusepath{clip}%
\pgfsetbuttcap%
\pgfsetroundjoin%
\definecolor{currentfill}{rgb}{1.000000,0.894118,0.788235}%
\pgfsetfillcolor{currentfill}%
\pgfsetlinewidth{0.000000pt}%
\definecolor{currentstroke}{rgb}{1.000000,0.894118,0.788235}%
\pgfsetstrokecolor{currentstroke}%
\pgfsetdash{}{0pt}%
\pgfpathmoveto{\pgfqpoint{3.404259in}{2.482426in}}%
\pgfpathlineto{\pgfqpoint{3.293363in}{2.546452in}}%
\pgfpathlineto{\pgfqpoint{3.463186in}{2.662942in}}%
\pgfpathlineto{\pgfqpoint{3.574082in}{2.598916in}}%
\pgfpathlineto{\pgfqpoint{3.404259in}{2.482426in}}%
\pgfpathclose%
\pgfusepath{fill}%
\end{pgfscope}%
\begin{pgfscope}%
\pgfpathrectangle{\pgfqpoint{0.765000in}{0.660000in}}{\pgfqpoint{4.620000in}{4.620000in}}%
\pgfusepath{clip}%
\pgfsetbuttcap%
\pgfsetroundjoin%
\definecolor{currentfill}{rgb}{1.000000,0.894118,0.788235}%
\pgfsetfillcolor{currentfill}%
\pgfsetlinewidth{0.000000pt}%
\definecolor{currentstroke}{rgb}{1.000000,0.894118,0.788235}%
\pgfsetstrokecolor{currentstroke}%
\pgfsetdash{}{0pt}%
\pgfpathmoveto{\pgfqpoint{3.293363in}{2.546452in}}%
\pgfpathlineto{\pgfqpoint{3.293363in}{2.674505in}}%
\pgfpathlineto{\pgfqpoint{3.463186in}{2.790995in}}%
\pgfpathlineto{\pgfqpoint{3.574082in}{2.598916in}}%
\pgfpathlineto{\pgfqpoint{3.293363in}{2.546452in}}%
\pgfpathclose%
\pgfusepath{fill}%
\end{pgfscope}%
\begin{pgfscope}%
\pgfpathrectangle{\pgfqpoint{0.765000in}{0.660000in}}{\pgfqpoint{4.620000in}{4.620000in}}%
\pgfusepath{clip}%
\pgfsetbuttcap%
\pgfsetroundjoin%
\definecolor{currentfill}{rgb}{1.000000,0.894118,0.788235}%
\pgfsetfillcolor{currentfill}%
\pgfsetlinewidth{0.000000pt}%
\definecolor{currentstroke}{rgb}{1.000000,0.894118,0.788235}%
\pgfsetstrokecolor{currentstroke}%
\pgfsetdash{}{0pt}%
\pgfpathmoveto{\pgfqpoint{3.404259in}{2.482426in}}%
\pgfpathlineto{\pgfqpoint{3.404259in}{2.610478in}}%
\pgfpathlineto{\pgfqpoint{3.574082in}{2.726968in}}%
\pgfpathlineto{\pgfqpoint{3.463186in}{2.662942in}}%
\pgfpathlineto{\pgfqpoint{3.404259in}{2.482426in}}%
\pgfpathclose%
\pgfusepath{fill}%
\end{pgfscope}%
\begin{pgfscope}%
\pgfpathrectangle{\pgfqpoint{0.765000in}{0.660000in}}{\pgfqpoint{4.620000in}{4.620000in}}%
\pgfusepath{clip}%
\pgfsetbuttcap%
\pgfsetroundjoin%
\definecolor{currentfill}{rgb}{1.000000,0.894118,0.788235}%
\pgfsetfillcolor{currentfill}%
\pgfsetlinewidth{0.000000pt}%
\definecolor{currentstroke}{rgb}{1.000000,0.894118,0.788235}%
\pgfsetstrokecolor{currentstroke}%
\pgfsetdash{}{0pt}%
\pgfpathmoveto{\pgfqpoint{3.182466in}{2.482426in}}%
\pgfpathlineto{\pgfqpoint{3.293363in}{2.418400in}}%
\pgfpathlineto{\pgfqpoint{3.463186in}{2.534890in}}%
\pgfpathlineto{\pgfqpoint{3.352289in}{2.598916in}}%
\pgfpathlineto{\pgfqpoint{3.182466in}{2.482426in}}%
\pgfpathclose%
\pgfusepath{fill}%
\end{pgfscope}%
\begin{pgfscope}%
\pgfpathrectangle{\pgfqpoint{0.765000in}{0.660000in}}{\pgfqpoint{4.620000in}{4.620000in}}%
\pgfusepath{clip}%
\pgfsetbuttcap%
\pgfsetroundjoin%
\definecolor{currentfill}{rgb}{1.000000,0.894118,0.788235}%
\pgfsetfillcolor{currentfill}%
\pgfsetlinewidth{0.000000pt}%
\definecolor{currentstroke}{rgb}{1.000000,0.894118,0.788235}%
\pgfsetstrokecolor{currentstroke}%
\pgfsetdash{}{0pt}%
\pgfpathmoveto{\pgfqpoint{3.293363in}{2.418400in}}%
\pgfpathlineto{\pgfqpoint{3.404259in}{2.482426in}}%
\pgfpathlineto{\pgfqpoint{3.574082in}{2.598916in}}%
\pgfpathlineto{\pgfqpoint{3.463186in}{2.534890in}}%
\pgfpathlineto{\pgfqpoint{3.293363in}{2.418400in}}%
\pgfpathclose%
\pgfusepath{fill}%
\end{pgfscope}%
\begin{pgfscope}%
\pgfpathrectangle{\pgfqpoint{0.765000in}{0.660000in}}{\pgfqpoint{4.620000in}{4.620000in}}%
\pgfusepath{clip}%
\pgfsetbuttcap%
\pgfsetroundjoin%
\definecolor{currentfill}{rgb}{1.000000,0.894118,0.788235}%
\pgfsetfillcolor{currentfill}%
\pgfsetlinewidth{0.000000pt}%
\definecolor{currentstroke}{rgb}{1.000000,0.894118,0.788235}%
\pgfsetstrokecolor{currentstroke}%
\pgfsetdash{}{0pt}%
\pgfpathmoveto{\pgfqpoint{3.293363in}{2.674505in}}%
\pgfpathlineto{\pgfqpoint{3.182466in}{2.610478in}}%
\pgfpathlineto{\pgfqpoint{3.352289in}{2.726968in}}%
\pgfpathlineto{\pgfqpoint{3.463186in}{2.790995in}}%
\pgfpathlineto{\pgfqpoint{3.293363in}{2.674505in}}%
\pgfpathclose%
\pgfusepath{fill}%
\end{pgfscope}%
\begin{pgfscope}%
\pgfpathrectangle{\pgfqpoint{0.765000in}{0.660000in}}{\pgfqpoint{4.620000in}{4.620000in}}%
\pgfusepath{clip}%
\pgfsetbuttcap%
\pgfsetroundjoin%
\definecolor{currentfill}{rgb}{1.000000,0.894118,0.788235}%
\pgfsetfillcolor{currentfill}%
\pgfsetlinewidth{0.000000pt}%
\definecolor{currentstroke}{rgb}{1.000000,0.894118,0.788235}%
\pgfsetstrokecolor{currentstroke}%
\pgfsetdash{}{0pt}%
\pgfpathmoveto{\pgfqpoint{3.404259in}{2.610478in}}%
\pgfpathlineto{\pgfqpoint{3.293363in}{2.674505in}}%
\pgfpathlineto{\pgfqpoint{3.463186in}{2.790995in}}%
\pgfpathlineto{\pgfqpoint{3.574082in}{2.726968in}}%
\pgfpathlineto{\pgfqpoint{3.404259in}{2.610478in}}%
\pgfpathclose%
\pgfusepath{fill}%
\end{pgfscope}%
\begin{pgfscope}%
\pgfpathrectangle{\pgfqpoint{0.765000in}{0.660000in}}{\pgfqpoint{4.620000in}{4.620000in}}%
\pgfusepath{clip}%
\pgfsetbuttcap%
\pgfsetroundjoin%
\definecolor{currentfill}{rgb}{1.000000,0.894118,0.788235}%
\pgfsetfillcolor{currentfill}%
\pgfsetlinewidth{0.000000pt}%
\definecolor{currentstroke}{rgb}{1.000000,0.894118,0.788235}%
\pgfsetstrokecolor{currentstroke}%
\pgfsetdash{}{0pt}%
\pgfpathmoveto{\pgfqpoint{3.182466in}{2.482426in}}%
\pgfpathlineto{\pgfqpoint{3.182466in}{2.610478in}}%
\pgfpathlineto{\pgfqpoint{3.352289in}{2.726968in}}%
\pgfpathlineto{\pgfqpoint{3.463186in}{2.534890in}}%
\pgfpathlineto{\pgfqpoint{3.182466in}{2.482426in}}%
\pgfpathclose%
\pgfusepath{fill}%
\end{pgfscope}%
\begin{pgfscope}%
\pgfpathrectangle{\pgfqpoint{0.765000in}{0.660000in}}{\pgfqpoint{4.620000in}{4.620000in}}%
\pgfusepath{clip}%
\pgfsetbuttcap%
\pgfsetroundjoin%
\definecolor{currentfill}{rgb}{1.000000,0.894118,0.788235}%
\pgfsetfillcolor{currentfill}%
\pgfsetlinewidth{0.000000pt}%
\definecolor{currentstroke}{rgb}{1.000000,0.894118,0.788235}%
\pgfsetstrokecolor{currentstroke}%
\pgfsetdash{}{0pt}%
\pgfpathmoveto{\pgfqpoint{3.293363in}{2.418400in}}%
\pgfpathlineto{\pgfqpoint{3.293363in}{2.546452in}}%
\pgfpathlineto{\pgfqpoint{3.463186in}{2.662942in}}%
\pgfpathlineto{\pgfqpoint{3.352289in}{2.598916in}}%
\pgfpathlineto{\pgfqpoint{3.293363in}{2.418400in}}%
\pgfpathclose%
\pgfusepath{fill}%
\end{pgfscope}%
\begin{pgfscope}%
\pgfpathrectangle{\pgfqpoint{0.765000in}{0.660000in}}{\pgfqpoint{4.620000in}{4.620000in}}%
\pgfusepath{clip}%
\pgfsetbuttcap%
\pgfsetroundjoin%
\definecolor{currentfill}{rgb}{1.000000,0.894118,0.788235}%
\pgfsetfillcolor{currentfill}%
\pgfsetlinewidth{0.000000pt}%
\definecolor{currentstroke}{rgb}{1.000000,0.894118,0.788235}%
\pgfsetstrokecolor{currentstroke}%
\pgfsetdash{}{0pt}%
\pgfpathmoveto{\pgfqpoint{3.293363in}{2.546452in}}%
\pgfpathlineto{\pgfqpoint{3.404259in}{2.610478in}}%
\pgfpathlineto{\pgfqpoint{3.574082in}{2.726968in}}%
\pgfpathlineto{\pgfqpoint{3.463186in}{2.662942in}}%
\pgfpathlineto{\pgfqpoint{3.293363in}{2.546452in}}%
\pgfpathclose%
\pgfusepath{fill}%
\end{pgfscope}%
\begin{pgfscope}%
\pgfpathrectangle{\pgfqpoint{0.765000in}{0.660000in}}{\pgfqpoint{4.620000in}{4.620000in}}%
\pgfusepath{clip}%
\pgfsetbuttcap%
\pgfsetroundjoin%
\definecolor{currentfill}{rgb}{1.000000,0.894118,0.788235}%
\pgfsetfillcolor{currentfill}%
\pgfsetlinewidth{0.000000pt}%
\definecolor{currentstroke}{rgb}{1.000000,0.894118,0.788235}%
\pgfsetstrokecolor{currentstroke}%
\pgfsetdash{}{0pt}%
\pgfpathmoveto{\pgfqpoint{3.182466in}{2.610478in}}%
\pgfpathlineto{\pgfqpoint{3.293363in}{2.546452in}}%
\pgfpathlineto{\pgfqpoint{3.463186in}{2.662942in}}%
\pgfpathlineto{\pgfqpoint{3.352289in}{2.726968in}}%
\pgfpathlineto{\pgfqpoint{3.182466in}{2.610478in}}%
\pgfpathclose%
\pgfusepath{fill}%
\end{pgfscope}%
\begin{pgfscope}%
\pgfpathrectangle{\pgfqpoint{0.765000in}{0.660000in}}{\pgfqpoint{4.620000in}{4.620000in}}%
\pgfusepath{clip}%
\pgfsetbuttcap%
\pgfsetroundjoin%
\definecolor{currentfill}{rgb}{1.000000,0.894118,0.788235}%
\pgfsetfillcolor{currentfill}%
\pgfsetlinewidth{0.000000pt}%
\definecolor{currentstroke}{rgb}{1.000000,0.894118,0.788235}%
\pgfsetstrokecolor{currentstroke}%
\pgfsetdash{}{0pt}%
\pgfpathmoveto{\pgfqpoint{2.982382in}{3.516877in}}%
\pgfpathlineto{\pgfqpoint{2.871485in}{3.452851in}}%
\pgfpathlineto{\pgfqpoint{2.982382in}{3.388824in}}%
\pgfpathlineto{\pgfqpoint{3.093279in}{3.452851in}}%
\pgfpathlineto{\pgfqpoint{2.982382in}{3.516877in}}%
\pgfpathclose%
\pgfusepath{fill}%
\end{pgfscope}%
\begin{pgfscope}%
\pgfpathrectangle{\pgfqpoint{0.765000in}{0.660000in}}{\pgfqpoint{4.620000in}{4.620000in}}%
\pgfusepath{clip}%
\pgfsetbuttcap%
\pgfsetroundjoin%
\definecolor{currentfill}{rgb}{1.000000,0.894118,0.788235}%
\pgfsetfillcolor{currentfill}%
\pgfsetlinewidth{0.000000pt}%
\definecolor{currentstroke}{rgb}{1.000000,0.894118,0.788235}%
\pgfsetstrokecolor{currentstroke}%
\pgfsetdash{}{0pt}%
\pgfpathmoveto{\pgfqpoint{2.982382in}{3.516877in}}%
\pgfpathlineto{\pgfqpoint{2.871485in}{3.452851in}}%
\pgfpathlineto{\pgfqpoint{2.871485in}{3.580903in}}%
\pgfpathlineto{\pgfqpoint{2.982382in}{3.644929in}}%
\pgfpathlineto{\pgfqpoint{2.982382in}{3.516877in}}%
\pgfpathclose%
\pgfusepath{fill}%
\end{pgfscope}%
\begin{pgfscope}%
\pgfpathrectangle{\pgfqpoint{0.765000in}{0.660000in}}{\pgfqpoint{4.620000in}{4.620000in}}%
\pgfusepath{clip}%
\pgfsetbuttcap%
\pgfsetroundjoin%
\definecolor{currentfill}{rgb}{1.000000,0.894118,0.788235}%
\pgfsetfillcolor{currentfill}%
\pgfsetlinewidth{0.000000pt}%
\definecolor{currentstroke}{rgb}{1.000000,0.894118,0.788235}%
\pgfsetstrokecolor{currentstroke}%
\pgfsetdash{}{0pt}%
\pgfpathmoveto{\pgfqpoint{2.982382in}{3.516877in}}%
\pgfpathlineto{\pgfqpoint{3.093279in}{3.452851in}}%
\pgfpathlineto{\pgfqpoint{3.093279in}{3.580903in}}%
\pgfpathlineto{\pgfqpoint{2.982382in}{3.644929in}}%
\pgfpathlineto{\pgfqpoint{2.982382in}{3.516877in}}%
\pgfpathclose%
\pgfusepath{fill}%
\end{pgfscope}%
\begin{pgfscope}%
\pgfpathrectangle{\pgfqpoint{0.765000in}{0.660000in}}{\pgfqpoint{4.620000in}{4.620000in}}%
\pgfusepath{clip}%
\pgfsetbuttcap%
\pgfsetroundjoin%
\definecolor{currentfill}{rgb}{1.000000,0.894118,0.788235}%
\pgfsetfillcolor{currentfill}%
\pgfsetlinewidth{0.000000pt}%
\definecolor{currentstroke}{rgb}{1.000000,0.894118,0.788235}%
\pgfsetstrokecolor{currentstroke}%
\pgfsetdash{}{0pt}%
\pgfpathmoveto{\pgfqpoint{3.046316in}{3.530257in}}%
\pgfpathlineto{\pgfqpoint{2.935419in}{3.466231in}}%
\pgfpathlineto{\pgfqpoint{3.046316in}{3.402204in}}%
\pgfpathlineto{\pgfqpoint{3.157212in}{3.466231in}}%
\pgfpathlineto{\pgfqpoint{3.046316in}{3.530257in}}%
\pgfpathclose%
\pgfusepath{fill}%
\end{pgfscope}%
\begin{pgfscope}%
\pgfpathrectangle{\pgfqpoint{0.765000in}{0.660000in}}{\pgfqpoint{4.620000in}{4.620000in}}%
\pgfusepath{clip}%
\pgfsetbuttcap%
\pgfsetroundjoin%
\definecolor{currentfill}{rgb}{1.000000,0.894118,0.788235}%
\pgfsetfillcolor{currentfill}%
\pgfsetlinewidth{0.000000pt}%
\definecolor{currentstroke}{rgb}{1.000000,0.894118,0.788235}%
\pgfsetstrokecolor{currentstroke}%
\pgfsetdash{}{0pt}%
\pgfpathmoveto{\pgfqpoint{3.046316in}{3.530257in}}%
\pgfpathlineto{\pgfqpoint{2.935419in}{3.466231in}}%
\pgfpathlineto{\pgfqpoint{2.935419in}{3.594283in}}%
\pgfpathlineto{\pgfqpoint{3.046316in}{3.658309in}}%
\pgfpathlineto{\pgfqpoint{3.046316in}{3.530257in}}%
\pgfpathclose%
\pgfusepath{fill}%
\end{pgfscope}%
\begin{pgfscope}%
\pgfpathrectangle{\pgfqpoint{0.765000in}{0.660000in}}{\pgfqpoint{4.620000in}{4.620000in}}%
\pgfusepath{clip}%
\pgfsetbuttcap%
\pgfsetroundjoin%
\definecolor{currentfill}{rgb}{1.000000,0.894118,0.788235}%
\pgfsetfillcolor{currentfill}%
\pgfsetlinewidth{0.000000pt}%
\definecolor{currentstroke}{rgb}{1.000000,0.894118,0.788235}%
\pgfsetstrokecolor{currentstroke}%
\pgfsetdash{}{0pt}%
\pgfpathmoveto{\pgfqpoint{3.046316in}{3.530257in}}%
\pgfpathlineto{\pgfqpoint{3.157212in}{3.466231in}}%
\pgfpathlineto{\pgfqpoint{3.157212in}{3.594283in}}%
\pgfpathlineto{\pgfqpoint{3.046316in}{3.658309in}}%
\pgfpathlineto{\pgfqpoint{3.046316in}{3.530257in}}%
\pgfpathclose%
\pgfusepath{fill}%
\end{pgfscope}%
\begin{pgfscope}%
\pgfpathrectangle{\pgfqpoint{0.765000in}{0.660000in}}{\pgfqpoint{4.620000in}{4.620000in}}%
\pgfusepath{clip}%
\pgfsetbuttcap%
\pgfsetroundjoin%
\definecolor{currentfill}{rgb}{1.000000,0.894118,0.788235}%
\pgfsetfillcolor{currentfill}%
\pgfsetlinewidth{0.000000pt}%
\definecolor{currentstroke}{rgb}{1.000000,0.894118,0.788235}%
\pgfsetstrokecolor{currentstroke}%
\pgfsetdash{}{0pt}%
\pgfpathmoveto{\pgfqpoint{2.982382in}{3.644929in}}%
\pgfpathlineto{\pgfqpoint{2.871485in}{3.580903in}}%
\pgfpathlineto{\pgfqpoint{2.982382in}{3.516877in}}%
\pgfpathlineto{\pgfqpoint{3.093279in}{3.580903in}}%
\pgfpathlineto{\pgfqpoint{2.982382in}{3.644929in}}%
\pgfpathclose%
\pgfusepath{fill}%
\end{pgfscope}%
\begin{pgfscope}%
\pgfpathrectangle{\pgfqpoint{0.765000in}{0.660000in}}{\pgfqpoint{4.620000in}{4.620000in}}%
\pgfusepath{clip}%
\pgfsetbuttcap%
\pgfsetroundjoin%
\definecolor{currentfill}{rgb}{1.000000,0.894118,0.788235}%
\pgfsetfillcolor{currentfill}%
\pgfsetlinewidth{0.000000pt}%
\definecolor{currentstroke}{rgb}{1.000000,0.894118,0.788235}%
\pgfsetstrokecolor{currentstroke}%
\pgfsetdash{}{0pt}%
\pgfpathmoveto{\pgfqpoint{2.982382in}{3.388824in}}%
\pgfpathlineto{\pgfqpoint{3.093279in}{3.452851in}}%
\pgfpathlineto{\pgfqpoint{3.093279in}{3.580903in}}%
\pgfpathlineto{\pgfqpoint{2.982382in}{3.516877in}}%
\pgfpathlineto{\pgfqpoint{2.982382in}{3.388824in}}%
\pgfpathclose%
\pgfusepath{fill}%
\end{pgfscope}%
\begin{pgfscope}%
\pgfpathrectangle{\pgfqpoint{0.765000in}{0.660000in}}{\pgfqpoint{4.620000in}{4.620000in}}%
\pgfusepath{clip}%
\pgfsetbuttcap%
\pgfsetroundjoin%
\definecolor{currentfill}{rgb}{1.000000,0.894118,0.788235}%
\pgfsetfillcolor{currentfill}%
\pgfsetlinewidth{0.000000pt}%
\definecolor{currentstroke}{rgb}{1.000000,0.894118,0.788235}%
\pgfsetstrokecolor{currentstroke}%
\pgfsetdash{}{0pt}%
\pgfpathmoveto{\pgfqpoint{2.871485in}{3.452851in}}%
\pgfpathlineto{\pgfqpoint{2.982382in}{3.388824in}}%
\pgfpathlineto{\pgfqpoint{2.982382in}{3.516877in}}%
\pgfpathlineto{\pgfqpoint{2.871485in}{3.580903in}}%
\pgfpathlineto{\pgfqpoint{2.871485in}{3.452851in}}%
\pgfpathclose%
\pgfusepath{fill}%
\end{pgfscope}%
\begin{pgfscope}%
\pgfpathrectangle{\pgfqpoint{0.765000in}{0.660000in}}{\pgfqpoint{4.620000in}{4.620000in}}%
\pgfusepath{clip}%
\pgfsetbuttcap%
\pgfsetroundjoin%
\definecolor{currentfill}{rgb}{1.000000,0.894118,0.788235}%
\pgfsetfillcolor{currentfill}%
\pgfsetlinewidth{0.000000pt}%
\definecolor{currentstroke}{rgb}{1.000000,0.894118,0.788235}%
\pgfsetstrokecolor{currentstroke}%
\pgfsetdash{}{0pt}%
\pgfpathmoveto{\pgfqpoint{3.046316in}{3.658309in}}%
\pgfpathlineto{\pgfqpoint{2.935419in}{3.594283in}}%
\pgfpathlineto{\pgfqpoint{3.046316in}{3.530257in}}%
\pgfpathlineto{\pgfqpoint{3.157212in}{3.594283in}}%
\pgfpathlineto{\pgfqpoint{3.046316in}{3.658309in}}%
\pgfpathclose%
\pgfusepath{fill}%
\end{pgfscope}%
\begin{pgfscope}%
\pgfpathrectangle{\pgfqpoint{0.765000in}{0.660000in}}{\pgfqpoint{4.620000in}{4.620000in}}%
\pgfusepath{clip}%
\pgfsetbuttcap%
\pgfsetroundjoin%
\definecolor{currentfill}{rgb}{1.000000,0.894118,0.788235}%
\pgfsetfillcolor{currentfill}%
\pgfsetlinewidth{0.000000pt}%
\definecolor{currentstroke}{rgb}{1.000000,0.894118,0.788235}%
\pgfsetstrokecolor{currentstroke}%
\pgfsetdash{}{0pt}%
\pgfpathmoveto{\pgfqpoint{3.046316in}{3.402204in}}%
\pgfpathlineto{\pgfqpoint{3.157212in}{3.466231in}}%
\pgfpathlineto{\pgfqpoint{3.157212in}{3.594283in}}%
\pgfpathlineto{\pgfqpoint{3.046316in}{3.530257in}}%
\pgfpathlineto{\pgfqpoint{3.046316in}{3.402204in}}%
\pgfpathclose%
\pgfusepath{fill}%
\end{pgfscope}%
\begin{pgfscope}%
\pgfpathrectangle{\pgfqpoint{0.765000in}{0.660000in}}{\pgfqpoint{4.620000in}{4.620000in}}%
\pgfusepath{clip}%
\pgfsetbuttcap%
\pgfsetroundjoin%
\definecolor{currentfill}{rgb}{1.000000,0.894118,0.788235}%
\pgfsetfillcolor{currentfill}%
\pgfsetlinewidth{0.000000pt}%
\definecolor{currentstroke}{rgb}{1.000000,0.894118,0.788235}%
\pgfsetstrokecolor{currentstroke}%
\pgfsetdash{}{0pt}%
\pgfpathmoveto{\pgfqpoint{2.935419in}{3.466231in}}%
\pgfpathlineto{\pgfqpoint{3.046316in}{3.402204in}}%
\pgfpathlineto{\pgfqpoint{3.046316in}{3.530257in}}%
\pgfpathlineto{\pgfqpoint{2.935419in}{3.594283in}}%
\pgfpathlineto{\pgfqpoint{2.935419in}{3.466231in}}%
\pgfpathclose%
\pgfusepath{fill}%
\end{pgfscope}%
\begin{pgfscope}%
\pgfpathrectangle{\pgfqpoint{0.765000in}{0.660000in}}{\pgfqpoint{4.620000in}{4.620000in}}%
\pgfusepath{clip}%
\pgfsetbuttcap%
\pgfsetroundjoin%
\definecolor{currentfill}{rgb}{1.000000,0.894118,0.788235}%
\pgfsetfillcolor{currentfill}%
\pgfsetlinewidth{0.000000pt}%
\definecolor{currentstroke}{rgb}{1.000000,0.894118,0.788235}%
\pgfsetstrokecolor{currentstroke}%
\pgfsetdash{}{0pt}%
\pgfpathmoveto{\pgfqpoint{2.982382in}{3.516877in}}%
\pgfpathlineto{\pgfqpoint{2.871485in}{3.452851in}}%
\pgfpathlineto{\pgfqpoint{2.935419in}{3.466231in}}%
\pgfpathlineto{\pgfqpoint{3.046316in}{3.530257in}}%
\pgfpathlineto{\pgfqpoint{2.982382in}{3.516877in}}%
\pgfpathclose%
\pgfusepath{fill}%
\end{pgfscope}%
\begin{pgfscope}%
\pgfpathrectangle{\pgfqpoint{0.765000in}{0.660000in}}{\pgfqpoint{4.620000in}{4.620000in}}%
\pgfusepath{clip}%
\pgfsetbuttcap%
\pgfsetroundjoin%
\definecolor{currentfill}{rgb}{1.000000,0.894118,0.788235}%
\pgfsetfillcolor{currentfill}%
\pgfsetlinewidth{0.000000pt}%
\definecolor{currentstroke}{rgb}{1.000000,0.894118,0.788235}%
\pgfsetstrokecolor{currentstroke}%
\pgfsetdash{}{0pt}%
\pgfpathmoveto{\pgfqpoint{3.093279in}{3.452851in}}%
\pgfpathlineto{\pgfqpoint{2.982382in}{3.516877in}}%
\pgfpathlineto{\pgfqpoint{3.046316in}{3.530257in}}%
\pgfpathlineto{\pgfqpoint{3.157212in}{3.466231in}}%
\pgfpathlineto{\pgfqpoint{3.093279in}{3.452851in}}%
\pgfpathclose%
\pgfusepath{fill}%
\end{pgfscope}%
\begin{pgfscope}%
\pgfpathrectangle{\pgfqpoint{0.765000in}{0.660000in}}{\pgfqpoint{4.620000in}{4.620000in}}%
\pgfusepath{clip}%
\pgfsetbuttcap%
\pgfsetroundjoin%
\definecolor{currentfill}{rgb}{1.000000,0.894118,0.788235}%
\pgfsetfillcolor{currentfill}%
\pgfsetlinewidth{0.000000pt}%
\definecolor{currentstroke}{rgb}{1.000000,0.894118,0.788235}%
\pgfsetstrokecolor{currentstroke}%
\pgfsetdash{}{0pt}%
\pgfpathmoveto{\pgfqpoint{2.982382in}{3.516877in}}%
\pgfpathlineto{\pgfqpoint{2.982382in}{3.644929in}}%
\pgfpathlineto{\pgfqpoint{3.046316in}{3.658309in}}%
\pgfpathlineto{\pgfqpoint{3.157212in}{3.466231in}}%
\pgfpathlineto{\pgfqpoint{2.982382in}{3.516877in}}%
\pgfpathclose%
\pgfusepath{fill}%
\end{pgfscope}%
\begin{pgfscope}%
\pgfpathrectangle{\pgfqpoint{0.765000in}{0.660000in}}{\pgfqpoint{4.620000in}{4.620000in}}%
\pgfusepath{clip}%
\pgfsetbuttcap%
\pgfsetroundjoin%
\definecolor{currentfill}{rgb}{1.000000,0.894118,0.788235}%
\pgfsetfillcolor{currentfill}%
\pgfsetlinewidth{0.000000pt}%
\definecolor{currentstroke}{rgb}{1.000000,0.894118,0.788235}%
\pgfsetstrokecolor{currentstroke}%
\pgfsetdash{}{0pt}%
\pgfpathmoveto{\pgfqpoint{3.093279in}{3.452851in}}%
\pgfpathlineto{\pgfqpoint{3.093279in}{3.580903in}}%
\pgfpathlineto{\pgfqpoint{3.157212in}{3.594283in}}%
\pgfpathlineto{\pgfqpoint{3.046316in}{3.530257in}}%
\pgfpathlineto{\pgfqpoint{3.093279in}{3.452851in}}%
\pgfpathclose%
\pgfusepath{fill}%
\end{pgfscope}%
\begin{pgfscope}%
\pgfpathrectangle{\pgfqpoint{0.765000in}{0.660000in}}{\pgfqpoint{4.620000in}{4.620000in}}%
\pgfusepath{clip}%
\pgfsetbuttcap%
\pgfsetroundjoin%
\definecolor{currentfill}{rgb}{1.000000,0.894118,0.788235}%
\pgfsetfillcolor{currentfill}%
\pgfsetlinewidth{0.000000pt}%
\definecolor{currentstroke}{rgb}{1.000000,0.894118,0.788235}%
\pgfsetstrokecolor{currentstroke}%
\pgfsetdash{}{0pt}%
\pgfpathmoveto{\pgfqpoint{2.871485in}{3.452851in}}%
\pgfpathlineto{\pgfqpoint{2.982382in}{3.388824in}}%
\pgfpathlineto{\pgfqpoint{3.046316in}{3.402204in}}%
\pgfpathlineto{\pgfqpoint{2.935419in}{3.466231in}}%
\pgfpathlineto{\pgfqpoint{2.871485in}{3.452851in}}%
\pgfpathclose%
\pgfusepath{fill}%
\end{pgfscope}%
\begin{pgfscope}%
\pgfpathrectangle{\pgfqpoint{0.765000in}{0.660000in}}{\pgfqpoint{4.620000in}{4.620000in}}%
\pgfusepath{clip}%
\pgfsetbuttcap%
\pgfsetroundjoin%
\definecolor{currentfill}{rgb}{1.000000,0.894118,0.788235}%
\pgfsetfillcolor{currentfill}%
\pgfsetlinewidth{0.000000pt}%
\definecolor{currentstroke}{rgb}{1.000000,0.894118,0.788235}%
\pgfsetstrokecolor{currentstroke}%
\pgfsetdash{}{0pt}%
\pgfpathmoveto{\pgfqpoint{2.982382in}{3.388824in}}%
\pgfpathlineto{\pgfqpoint{3.093279in}{3.452851in}}%
\pgfpathlineto{\pgfqpoint{3.157212in}{3.466231in}}%
\pgfpathlineto{\pgfqpoint{3.046316in}{3.402204in}}%
\pgfpathlineto{\pgfqpoint{2.982382in}{3.388824in}}%
\pgfpathclose%
\pgfusepath{fill}%
\end{pgfscope}%
\begin{pgfscope}%
\pgfpathrectangle{\pgfqpoint{0.765000in}{0.660000in}}{\pgfqpoint{4.620000in}{4.620000in}}%
\pgfusepath{clip}%
\pgfsetbuttcap%
\pgfsetroundjoin%
\definecolor{currentfill}{rgb}{1.000000,0.894118,0.788235}%
\pgfsetfillcolor{currentfill}%
\pgfsetlinewidth{0.000000pt}%
\definecolor{currentstroke}{rgb}{1.000000,0.894118,0.788235}%
\pgfsetstrokecolor{currentstroke}%
\pgfsetdash{}{0pt}%
\pgfpathmoveto{\pgfqpoint{2.982382in}{3.644929in}}%
\pgfpathlineto{\pgfqpoint{2.871485in}{3.580903in}}%
\pgfpathlineto{\pgfqpoint{2.935419in}{3.594283in}}%
\pgfpathlineto{\pgfqpoint{3.046316in}{3.658309in}}%
\pgfpathlineto{\pgfqpoint{2.982382in}{3.644929in}}%
\pgfpathclose%
\pgfusepath{fill}%
\end{pgfscope}%
\begin{pgfscope}%
\pgfpathrectangle{\pgfqpoint{0.765000in}{0.660000in}}{\pgfqpoint{4.620000in}{4.620000in}}%
\pgfusepath{clip}%
\pgfsetbuttcap%
\pgfsetroundjoin%
\definecolor{currentfill}{rgb}{1.000000,0.894118,0.788235}%
\pgfsetfillcolor{currentfill}%
\pgfsetlinewidth{0.000000pt}%
\definecolor{currentstroke}{rgb}{1.000000,0.894118,0.788235}%
\pgfsetstrokecolor{currentstroke}%
\pgfsetdash{}{0pt}%
\pgfpathmoveto{\pgfqpoint{3.093279in}{3.580903in}}%
\pgfpathlineto{\pgfqpoint{2.982382in}{3.644929in}}%
\pgfpathlineto{\pgfqpoint{3.046316in}{3.658309in}}%
\pgfpathlineto{\pgfqpoint{3.157212in}{3.594283in}}%
\pgfpathlineto{\pgfqpoint{3.093279in}{3.580903in}}%
\pgfpathclose%
\pgfusepath{fill}%
\end{pgfscope}%
\begin{pgfscope}%
\pgfpathrectangle{\pgfqpoint{0.765000in}{0.660000in}}{\pgfqpoint{4.620000in}{4.620000in}}%
\pgfusepath{clip}%
\pgfsetbuttcap%
\pgfsetroundjoin%
\definecolor{currentfill}{rgb}{1.000000,0.894118,0.788235}%
\pgfsetfillcolor{currentfill}%
\pgfsetlinewidth{0.000000pt}%
\definecolor{currentstroke}{rgb}{1.000000,0.894118,0.788235}%
\pgfsetstrokecolor{currentstroke}%
\pgfsetdash{}{0pt}%
\pgfpathmoveto{\pgfqpoint{2.871485in}{3.452851in}}%
\pgfpathlineto{\pgfqpoint{2.871485in}{3.580903in}}%
\pgfpathlineto{\pgfqpoint{2.935419in}{3.594283in}}%
\pgfpathlineto{\pgfqpoint{3.046316in}{3.402204in}}%
\pgfpathlineto{\pgfqpoint{2.871485in}{3.452851in}}%
\pgfpathclose%
\pgfusepath{fill}%
\end{pgfscope}%
\begin{pgfscope}%
\pgfpathrectangle{\pgfqpoint{0.765000in}{0.660000in}}{\pgfqpoint{4.620000in}{4.620000in}}%
\pgfusepath{clip}%
\pgfsetbuttcap%
\pgfsetroundjoin%
\definecolor{currentfill}{rgb}{1.000000,0.894118,0.788235}%
\pgfsetfillcolor{currentfill}%
\pgfsetlinewidth{0.000000pt}%
\definecolor{currentstroke}{rgb}{1.000000,0.894118,0.788235}%
\pgfsetstrokecolor{currentstroke}%
\pgfsetdash{}{0pt}%
\pgfpathmoveto{\pgfqpoint{2.982382in}{3.388824in}}%
\pgfpathlineto{\pgfqpoint{2.982382in}{3.516877in}}%
\pgfpathlineto{\pgfqpoint{3.046316in}{3.530257in}}%
\pgfpathlineto{\pgfqpoint{2.935419in}{3.466231in}}%
\pgfpathlineto{\pgfqpoint{2.982382in}{3.388824in}}%
\pgfpathclose%
\pgfusepath{fill}%
\end{pgfscope}%
\begin{pgfscope}%
\pgfpathrectangle{\pgfqpoint{0.765000in}{0.660000in}}{\pgfqpoint{4.620000in}{4.620000in}}%
\pgfusepath{clip}%
\pgfsetbuttcap%
\pgfsetroundjoin%
\definecolor{currentfill}{rgb}{1.000000,0.894118,0.788235}%
\pgfsetfillcolor{currentfill}%
\pgfsetlinewidth{0.000000pt}%
\definecolor{currentstroke}{rgb}{1.000000,0.894118,0.788235}%
\pgfsetstrokecolor{currentstroke}%
\pgfsetdash{}{0pt}%
\pgfpathmoveto{\pgfqpoint{2.982382in}{3.516877in}}%
\pgfpathlineto{\pgfqpoint{3.093279in}{3.580903in}}%
\pgfpathlineto{\pgfqpoint{3.157212in}{3.594283in}}%
\pgfpathlineto{\pgfqpoint{3.046316in}{3.530257in}}%
\pgfpathlineto{\pgfqpoint{2.982382in}{3.516877in}}%
\pgfpathclose%
\pgfusepath{fill}%
\end{pgfscope}%
\begin{pgfscope}%
\pgfpathrectangle{\pgfqpoint{0.765000in}{0.660000in}}{\pgfqpoint{4.620000in}{4.620000in}}%
\pgfusepath{clip}%
\pgfsetbuttcap%
\pgfsetroundjoin%
\definecolor{currentfill}{rgb}{1.000000,0.894118,0.788235}%
\pgfsetfillcolor{currentfill}%
\pgfsetlinewidth{0.000000pt}%
\definecolor{currentstroke}{rgb}{1.000000,0.894118,0.788235}%
\pgfsetstrokecolor{currentstroke}%
\pgfsetdash{}{0pt}%
\pgfpathmoveto{\pgfqpoint{2.871485in}{3.580903in}}%
\pgfpathlineto{\pgfqpoint{2.982382in}{3.516877in}}%
\pgfpathlineto{\pgfqpoint{3.046316in}{3.530257in}}%
\pgfpathlineto{\pgfqpoint{2.935419in}{3.594283in}}%
\pgfpathlineto{\pgfqpoint{2.871485in}{3.580903in}}%
\pgfpathclose%
\pgfusepath{fill}%
\end{pgfscope}%
\begin{pgfscope}%
\pgfpathrectangle{\pgfqpoint{0.765000in}{0.660000in}}{\pgfqpoint{4.620000in}{4.620000in}}%
\pgfusepath{clip}%
\pgfsetbuttcap%
\pgfsetroundjoin%
\definecolor{currentfill}{rgb}{1.000000,0.894118,0.788235}%
\pgfsetfillcolor{currentfill}%
\pgfsetlinewidth{0.000000pt}%
\definecolor{currentstroke}{rgb}{1.000000,0.894118,0.788235}%
\pgfsetstrokecolor{currentstroke}%
\pgfsetdash{}{0pt}%
\pgfpathmoveto{\pgfqpoint{2.982382in}{3.516877in}}%
\pgfpathlineto{\pgfqpoint{2.871485in}{3.452851in}}%
\pgfpathlineto{\pgfqpoint{2.982382in}{3.388824in}}%
\pgfpathlineto{\pgfqpoint{3.093279in}{3.452851in}}%
\pgfpathlineto{\pgfqpoint{2.982382in}{3.516877in}}%
\pgfpathclose%
\pgfusepath{fill}%
\end{pgfscope}%
\begin{pgfscope}%
\pgfpathrectangle{\pgfqpoint{0.765000in}{0.660000in}}{\pgfqpoint{4.620000in}{4.620000in}}%
\pgfusepath{clip}%
\pgfsetbuttcap%
\pgfsetroundjoin%
\definecolor{currentfill}{rgb}{1.000000,0.894118,0.788235}%
\pgfsetfillcolor{currentfill}%
\pgfsetlinewidth{0.000000pt}%
\definecolor{currentstroke}{rgb}{1.000000,0.894118,0.788235}%
\pgfsetstrokecolor{currentstroke}%
\pgfsetdash{}{0pt}%
\pgfpathmoveto{\pgfqpoint{2.982382in}{3.516877in}}%
\pgfpathlineto{\pgfqpoint{2.871485in}{3.452851in}}%
\pgfpathlineto{\pgfqpoint{2.871485in}{3.580903in}}%
\pgfpathlineto{\pgfqpoint{2.982382in}{3.644929in}}%
\pgfpathlineto{\pgfqpoint{2.982382in}{3.516877in}}%
\pgfpathclose%
\pgfusepath{fill}%
\end{pgfscope}%
\begin{pgfscope}%
\pgfpathrectangle{\pgfqpoint{0.765000in}{0.660000in}}{\pgfqpoint{4.620000in}{4.620000in}}%
\pgfusepath{clip}%
\pgfsetbuttcap%
\pgfsetroundjoin%
\definecolor{currentfill}{rgb}{1.000000,0.894118,0.788235}%
\pgfsetfillcolor{currentfill}%
\pgfsetlinewidth{0.000000pt}%
\definecolor{currentstroke}{rgb}{1.000000,0.894118,0.788235}%
\pgfsetstrokecolor{currentstroke}%
\pgfsetdash{}{0pt}%
\pgfpathmoveto{\pgfqpoint{2.982382in}{3.516877in}}%
\pgfpathlineto{\pgfqpoint{3.093279in}{3.452851in}}%
\pgfpathlineto{\pgfqpoint{3.093279in}{3.580903in}}%
\pgfpathlineto{\pgfqpoint{2.982382in}{3.644929in}}%
\pgfpathlineto{\pgfqpoint{2.982382in}{3.516877in}}%
\pgfpathclose%
\pgfusepath{fill}%
\end{pgfscope}%
\begin{pgfscope}%
\pgfpathrectangle{\pgfqpoint{0.765000in}{0.660000in}}{\pgfqpoint{4.620000in}{4.620000in}}%
\pgfusepath{clip}%
\pgfsetbuttcap%
\pgfsetroundjoin%
\definecolor{currentfill}{rgb}{1.000000,0.894118,0.788235}%
\pgfsetfillcolor{currentfill}%
\pgfsetlinewidth{0.000000pt}%
\definecolor{currentstroke}{rgb}{1.000000,0.894118,0.788235}%
\pgfsetstrokecolor{currentstroke}%
\pgfsetdash{}{0pt}%
\pgfpathmoveto{\pgfqpoint{2.913379in}{3.479681in}}%
\pgfpathlineto{\pgfqpoint{2.802482in}{3.415655in}}%
\pgfpathlineto{\pgfqpoint{2.913379in}{3.351628in}}%
\pgfpathlineto{\pgfqpoint{3.024275in}{3.415655in}}%
\pgfpathlineto{\pgfqpoint{2.913379in}{3.479681in}}%
\pgfpathclose%
\pgfusepath{fill}%
\end{pgfscope}%
\begin{pgfscope}%
\pgfpathrectangle{\pgfqpoint{0.765000in}{0.660000in}}{\pgfqpoint{4.620000in}{4.620000in}}%
\pgfusepath{clip}%
\pgfsetbuttcap%
\pgfsetroundjoin%
\definecolor{currentfill}{rgb}{1.000000,0.894118,0.788235}%
\pgfsetfillcolor{currentfill}%
\pgfsetlinewidth{0.000000pt}%
\definecolor{currentstroke}{rgb}{1.000000,0.894118,0.788235}%
\pgfsetstrokecolor{currentstroke}%
\pgfsetdash{}{0pt}%
\pgfpathmoveto{\pgfqpoint{2.913379in}{3.479681in}}%
\pgfpathlineto{\pgfqpoint{2.802482in}{3.415655in}}%
\pgfpathlineto{\pgfqpoint{2.802482in}{3.543707in}}%
\pgfpathlineto{\pgfqpoint{2.913379in}{3.607733in}}%
\pgfpathlineto{\pgfqpoint{2.913379in}{3.479681in}}%
\pgfpathclose%
\pgfusepath{fill}%
\end{pgfscope}%
\begin{pgfscope}%
\pgfpathrectangle{\pgfqpoint{0.765000in}{0.660000in}}{\pgfqpoint{4.620000in}{4.620000in}}%
\pgfusepath{clip}%
\pgfsetbuttcap%
\pgfsetroundjoin%
\definecolor{currentfill}{rgb}{1.000000,0.894118,0.788235}%
\pgfsetfillcolor{currentfill}%
\pgfsetlinewidth{0.000000pt}%
\definecolor{currentstroke}{rgb}{1.000000,0.894118,0.788235}%
\pgfsetstrokecolor{currentstroke}%
\pgfsetdash{}{0pt}%
\pgfpathmoveto{\pgfqpoint{2.913379in}{3.479681in}}%
\pgfpathlineto{\pgfqpoint{3.024275in}{3.415655in}}%
\pgfpathlineto{\pgfqpoint{3.024275in}{3.543707in}}%
\pgfpathlineto{\pgfqpoint{2.913379in}{3.607733in}}%
\pgfpathlineto{\pgfqpoint{2.913379in}{3.479681in}}%
\pgfpathclose%
\pgfusepath{fill}%
\end{pgfscope}%
\begin{pgfscope}%
\pgfpathrectangle{\pgfqpoint{0.765000in}{0.660000in}}{\pgfqpoint{4.620000in}{4.620000in}}%
\pgfusepath{clip}%
\pgfsetbuttcap%
\pgfsetroundjoin%
\definecolor{currentfill}{rgb}{1.000000,0.894118,0.788235}%
\pgfsetfillcolor{currentfill}%
\pgfsetlinewidth{0.000000pt}%
\definecolor{currentstroke}{rgb}{1.000000,0.894118,0.788235}%
\pgfsetstrokecolor{currentstroke}%
\pgfsetdash{}{0pt}%
\pgfpathmoveto{\pgfqpoint{2.982382in}{3.644929in}}%
\pgfpathlineto{\pgfqpoint{2.871485in}{3.580903in}}%
\pgfpathlineto{\pgfqpoint{2.982382in}{3.516877in}}%
\pgfpathlineto{\pgfqpoint{3.093279in}{3.580903in}}%
\pgfpathlineto{\pgfqpoint{2.982382in}{3.644929in}}%
\pgfpathclose%
\pgfusepath{fill}%
\end{pgfscope}%
\begin{pgfscope}%
\pgfpathrectangle{\pgfqpoint{0.765000in}{0.660000in}}{\pgfqpoint{4.620000in}{4.620000in}}%
\pgfusepath{clip}%
\pgfsetbuttcap%
\pgfsetroundjoin%
\definecolor{currentfill}{rgb}{1.000000,0.894118,0.788235}%
\pgfsetfillcolor{currentfill}%
\pgfsetlinewidth{0.000000pt}%
\definecolor{currentstroke}{rgb}{1.000000,0.894118,0.788235}%
\pgfsetstrokecolor{currentstroke}%
\pgfsetdash{}{0pt}%
\pgfpathmoveto{\pgfqpoint{2.982382in}{3.388824in}}%
\pgfpathlineto{\pgfqpoint{3.093279in}{3.452851in}}%
\pgfpathlineto{\pgfqpoint{3.093279in}{3.580903in}}%
\pgfpathlineto{\pgfqpoint{2.982382in}{3.516877in}}%
\pgfpathlineto{\pgfqpoint{2.982382in}{3.388824in}}%
\pgfpathclose%
\pgfusepath{fill}%
\end{pgfscope}%
\begin{pgfscope}%
\pgfpathrectangle{\pgfqpoint{0.765000in}{0.660000in}}{\pgfqpoint{4.620000in}{4.620000in}}%
\pgfusepath{clip}%
\pgfsetbuttcap%
\pgfsetroundjoin%
\definecolor{currentfill}{rgb}{1.000000,0.894118,0.788235}%
\pgfsetfillcolor{currentfill}%
\pgfsetlinewidth{0.000000pt}%
\definecolor{currentstroke}{rgb}{1.000000,0.894118,0.788235}%
\pgfsetstrokecolor{currentstroke}%
\pgfsetdash{}{0pt}%
\pgfpathmoveto{\pgfqpoint{2.871485in}{3.452851in}}%
\pgfpathlineto{\pgfqpoint{2.982382in}{3.388824in}}%
\pgfpathlineto{\pgfqpoint{2.982382in}{3.516877in}}%
\pgfpathlineto{\pgfqpoint{2.871485in}{3.580903in}}%
\pgfpathlineto{\pgfqpoint{2.871485in}{3.452851in}}%
\pgfpathclose%
\pgfusepath{fill}%
\end{pgfscope}%
\begin{pgfscope}%
\pgfpathrectangle{\pgfqpoint{0.765000in}{0.660000in}}{\pgfqpoint{4.620000in}{4.620000in}}%
\pgfusepath{clip}%
\pgfsetbuttcap%
\pgfsetroundjoin%
\definecolor{currentfill}{rgb}{1.000000,0.894118,0.788235}%
\pgfsetfillcolor{currentfill}%
\pgfsetlinewidth{0.000000pt}%
\definecolor{currentstroke}{rgb}{1.000000,0.894118,0.788235}%
\pgfsetstrokecolor{currentstroke}%
\pgfsetdash{}{0pt}%
\pgfpathmoveto{\pgfqpoint{2.913379in}{3.607733in}}%
\pgfpathlineto{\pgfqpoint{2.802482in}{3.543707in}}%
\pgfpathlineto{\pgfqpoint{2.913379in}{3.479681in}}%
\pgfpathlineto{\pgfqpoint{3.024275in}{3.543707in}}%
\pgfpathlineto{\pgfqpoint{2.913379in}{3.607733in}}%
\pgfpathclose%
\pgfusepath{fill}%
\end{pgfscope}%
\begin{pgfscope}%
\pgfpathrectangle{\pgfqpoint{0.765000in}{0.660000in}}{\pgfqpoint{4.620000in}{4.620000in}}%
\pgfusepath{clip}%
\pgfsetbuttcap%
\pgfsetroundjoin%
\definecolor{currentfill}{rgb}{1.000000,0.894118,0.788235}%
\pgfsetfillcolor{currentfill}%
\pgfsetlinewidth{0.000000pt}%
\definecolor{currentstroke}{rgb}{1.000000,0.894118,0.788235}%
\pgfsetstrokecolor{currentstroke}%
\pgfsetdash{}{0pt}%
\pgfpathmoveto{\pgfqpoint{2.913379in}{3.351628in}}%
\pgfpathlineto{\pgfqpoint{3.024275in}{3.415655in}}%
\pgfpathlineto{\pgfqpoint{3.024275in}{3.543707in}}%
\pgfpathlineto{\pgfqpoint{2.913379in}{3.479681in}}%
\pgfpathlineto{\pgfqpoint{2.913379in}{3.351628in}}%
\pgfpathclose%
\pgfusepath{fill}%
\end{pgfscope}%
\begin{pgfscope}%
\pgfpathrectangle{\pgfqpoint{0.765000in}{0.660000in}}{\pgfqpoint{4.620000in}{4.620000in}}%
\pgfusepath{clip}%
\pgfsetbuttcap%
\pgfsetroundjoin%
\definecolor{currentfill}{rgb}{1.000000,0.894118,0.788235}%
\pgfsetfillcolor{currentfill}%
\pgfsetlinewidth{0.000000pt}%
\definecolor{currentstroke}{rgb}{1.000000,0.894118,0.788235}%
\pgfsetstrokecolor{currentstroke}%
\pgfsetdash{}{0pt}%
\pgfpathmoveto{\pgfqpoint{2.802482in}{3.415655in}}%
\pgfpathlineto{\pgfqpoint{2.913379in}{3.351628in}}%
\pgfpathlineto{\pgfqpoint{2.913379in}{3.479681in}}%
\pgfpathlineto{\pgfqpoint{2.802482in}{3.543707in}}%
\pgfpathlineto{\pgfqpoint{2.802482in}{3.415655in}}%
\pgfpathclose%
\pgfusepath{fill}%
\end{pgfscope}%
\begin{pgfscope}%
\pgfpathrectangle{\pgfqpoint{0.765000in}{0.660000in}}{\pgfqpoint{4.620000in}{4.620000in}}%
\pgfusepath{clip}%
\pgfsetbuttcap%
\pgfsetroundjoin%
\definecolor{currentfill}{rgb}{1.000000,0.894118,0.788235}%
\pgfsetfillcolor{currentfill}%
\pgfsetlinewidth{0.000000pt}%
\definecolor{currentstroke}{rgb}{1.000000,0.894118,0.788235}%
\pgfsetstrokecolor{currentstroke}%
\pgfsetdash{}{0pt}%
\pgfpathmoveto{\pgfqpoint{2.982382in}{3.516877in}}%
\pgfpathlineto{\pgfqpoint{2.871485in}{3.452851in}}%
\pgfpathlineto{\pgfqpoint{2.802482in}{3.415655in}}%
\pgfpathlineto{\pgfqpoint{2.913379in}{3.479681in}}%
\pgfpathlineto{\pgfqpoint{2.982382in}{3.516877in}}%
\pgfpathclose%
\pgfusepath{fill}%
\end{pgfscope}%
\begin{pgfscope}%
\pgfpathrectangle{\pgfqpoint{0.765000in}{0.660000in}}{\pgfqpoint{4.620000in}{4.620000in}}%
\pgfusepath{clip}%
\pgfsetbuttcap%
\pgfsetroundjoin%
\definecolor{currentfill}{rgb}{1.000000,0.894118,0.788235}%
\pgfsetfillcolor{currentfill}%
\pgfsetlinewidth{0.000000pt}%
\definecolor{currentstroke}{rgb}{1.000000,0.894118,0.788235}%
\pgfsetstrokecolor{currentstroke}%
\pgfsetdash{}{0pt}%
\pgfpathmoveto{\pgfqpoint{3.093279in}{3.452851in}}%
\pgfpathlineto{\pgfqpoint{2.982382in}{3.516877in}}%
\pgfpathlineto{\pgfqpoint{2.913379in}{3.479681in}}%
\pgfpathlineto{\pgfqpoint{3.024275in}{3.415655in}}%
\pgfpathlineto{\pgfqpoint{3.093279in}{3.452851in}}%
\pgfpathclose%
\pgfusepath{fill}%
\end{pgfscope}%
\begin{pgfscope}%
\pgfpathrectangle{\pgfqpoint{0.765000in}{0.660000in}}{\pgfqpoint{4.620000in}{4.620000in}}%
\pgfusepath{clip}%
\pgfsetbuttcap%
\pgfsetroundjoin%
\definecolor{currentfill}{rgb}{1.000000,0.894118,0.788235}%
\pgfsetfillcolor{currentfill}%
\pgfsetlinewidth{0.000000pt}%
\definecolor{currentstroke}{rgb}{1.000000,0.894118,0.788235}%
\pgfsetstrokecolor{currentstroke}%
\pgfsetdash{}{0pt}%
\pgfpathmoveto{\pgfqpoint{2.982382in}{3.516877in}}%
\pgfpathlineto{\pgfqpoint{2.982382in}{3.644929in}}%
\pgfpathlineto{\pgfqpoint{2.913379in}{3.607733in}}%
\pgfpathlineto{\pgfqpoint{3.024275in}{3.415655in}}%
\pgfpathlineto{\pgfqpoint{2.982382in}{3.516877in}}%
\pgfpathclose%
\pgfusepath{fill}%
\end{pgfscope}%
\begin{pgfscope}%
\pgfpathrectangle{\pgfqpoint{0.765000in}{0.660000in}}{\pgfqpoint{4.620000in}{4.620000in}}%
\pgfusepath{clip}%
\pgfsetbuttcap%
\pgfsetroundjoin%
\definecolor{currentfill}{rgb}{1.000000,0.894118,0.788235}%
\pgfsetfillcolor{currentfill}%
\pgfsetlinewidth{0.000000pt}%
\definecolor{currentstroke}{rgb}{1.000000,0.894118,0.788235}%
\pgfsetstrokecolor{currentstroke}%
\pgfsetdash{}{0pt}%
\pgfpathmoveto{\pgfqpoint{3.093279in}{3.452851in}}%
\pgfpathlineto{\pgfqpoint{3.093279in}{3.580903in}}%
\pgfpathlineto{\pgfqpoint{3.024275in}{3.543707in}}%
\pgfpathlineto{\pgfqpoint{2.913379in}{3.479681in}}%
\pgfpathlineto{\pgfqpoint{3.093279in}{3.452851in}}%
\pgfpathclose%
\pgfusepath{fill}%
\end{pgfscope}%
\begin{pgfscope}%
\pgfpathrectangle{\pgfqpoint{0.765000in}{0.660000in}}{\pgfqpoint{4.620000in}{4.620000in}}%
\pgfusepath{clip}%
\pgfsetbuttcap%
\pgfsetroundjoin%
\definecolor{currentfill}{rgb}{1.000000,0.894118,0.788235}%
\pgfsetfillcolor{currentfill}%
\pgfsetlinewidth{0.000000pt}%
\definecolor{currentstroke}{rgb}{1.000000,0.894118,0.788235}%
\pgfsetstrokecolor{currentstroke}%
\pgfsetdash{}{0pt}%
\pgfpathmoveto{\pgfqpoint{2.871485in}{3.452851in}}%
\pgfpathlineto{\pgfqpoint{2.982382in}{3.388824in}}%
\pgfpathlineto{\pgfqpoint{2.913379in}{3.351628in}}%
\pgfpathlineto{\pgfqpoint{2.802482in}{3.415655in}}%
\pgfpathlineto{\pgfqpoint{2.871485in}{3.452851in}}%
\pgfpathclose%
\pgfusepath{fill}%
\end{pgfscope}%
\begin{pgfscope}%
\pgfpathrectangle{\pgfqpoint{0.765000in}{0.660000in}}{\pgfqpoint{4.620000in}{4.620000in}}%
\pgfusepath{clip}%
\pgfsetbuttcap%
\pgfsetroundjoin%
\definecolor{currentfill}{rgb}{1.000000,0.894118,0.788235}%
\pgfsetfillcolor{currentfill}%
\pgfsetlinewidth{0.000000pt}%
\definecolor{currentstroke}{rgb}{1.000000,0.894118,0.788235}%
\pgfsetstrokecolor{currentstroke}%
\pgfsetdash{}{0pt}%
\pgfpathmoveto{\pgfqpoint{2.982382in}{3.388824in}}%
\pgfpathlineto{\pgfqpoint{3.093279in}{3.452851in}}%
\pgfpathlineto{\pgfqpoint{3.024275in}{3.415655in}}%
\pgfpathlineto{\pgfqpoint{2.913379in}{3.351628in}}%
\pgfpathlineto{\pgfqpoint{2.982382in}{3.388824in}}%
\pgfpathclose%
\pgfusepath{fill}%
\end{pgfscope}%
\begin{pgfscope}%
\pgfpathrectangle{\pgfqpoint{0.765000in}{0.660000in}}{\pgfqpoint{4.620000in}{4.620000in}}%
\pgfusepath{clip}%
\pgfsetbuttcap%
\pgfsetroundjoin%
\definecolor{currentfill}{rgb}{1.000000,0.894118,0.788235}%
\pgfsetfillcolor{currentfill}%
\pgfsetlinewidth{0.000000pt}%
\definecolor{currentstroke}{rgb}{1.000000,0.894118,0.788235}%
\pgfsetstrokecolor{currentstroke}%
\pgfsetdash{}{0pt}%
\pgfpathmoveto{\pgfqpoint{2.982382in}{3.644929in}}%
\pgfpathlineto{\pgfqpoint{2.871485in}{3.580903in}}%
\pgfpathlineto{\pgfqpoint{2.802482in}{3.543707in}}%
\pgfpathlineto{\pgfqpoint{2.913379in}{3.607733in}}%
\pgfpathlineto{\pgfqpoint{2.982382in}{3.644929in}}%
\pgfpathclose%
\pgfusepath{fill}%
\end{pgfscope}%
\begin{pgfscope}%
\pgfpathrectangle{\pgfqpoint{0.765000in}{0.660000in}}{\pgfqpoint{4.620000in}{4.620000in}}%
\pgfusepath{clip}%
\pgfsetbuttcap%
\pgfsetroundjoin%
\definecolor{currentfill}{rgb}{1.000000,0.894118,0.788235}%
\pgfsetfillcolor{currentfill}%
\pgfsetlinewidth{0.000000pt}%
\definecolor{currentstroke}{rgb}{1.000000,0.894118,0.788235}%
\pgfsetstrokecolor{currentstroke}%
\pgfsetdash{}{0pt}%
\pgfpathmoveto{\pgfqpoint{3.093279in}{3.580903in}}%
\pgfpathlineto{\pgfqpoint{2.982382in}{3.644929in}}%
\pgfpathlineto{\pgfqpoint{2.913379in}{3.607733in}}%
\pgfpathlineto{\pgfqpoint{3.024275in}{3.543707in}}%
\pgfpathlineto{\pgfqpoint{3.093279in}{3.580903in}}%
\pgfpathclose%
\pgfusepath{fill}%
\end{pgfscope}%
\begin{pgfscope}%
\pgfpathrectangle{\pgfqpoint{0.765000in}{0.660000in}}{\pgfqpoint{4.620000in}{4.620000in}}%
\pgfusepath{clip}%
\pgfsetbuttcap%
\pgfsetroundjoin%
\definecolor{currentfill}{rgb}{1.000000,0.894118,0.788235}%
\pgfsetfillcolor{currentfill}%
\pgfsetlinewidth{0.000000pt}%
\definecolor{currentstroke}{rgb}{1.000000,0.894118,0.788235}%
\pgfsetstrokecolor{currentstroke}%
\pgfsetdash{}{0pt}%
\pgfpathmoveto{\pgfqpoint{2.871485in}{3.452851in}}%
\pgfpathlineto{\pgfqpoint{2.871485in}{3.580903in}}%
\pgfpathlineto{\pgfqpoint{2.802482in}{3.543707in}}%
\pgfpathlineto{\pgfqpoint{2.913379in}{3.351628in}}%
\pgfpathlineto{\pgfqpoint{2.871485in}{3.452851in}}%
\pgfpathclose%
\pgfusepath{fill}%
\end{pgfscope}%
\begin{pgfscope}%
\pgfpathrectangle{\pgfqpoint{0.765000in}{0.660000in}}{\pgfqpoint{4.620000in}{4.620000in}}%
\pgfusepath{clip}%
\pgfsetbuttcap%
\pgfsetroundjoin%
\definecolor{currentfill}{rgb}{1.000000,0.894118,0.788235}%
\pgfsetfillcolor{currentfill}%
\pgfsetlinewidth{0.000000pt}%
\definecolor{currentstroke}{rgb}{1.000000,0.894118,0.788235}%
\pgfsetstrokecolor{currentstroke}%
\pgfsetdash{}{0pt}%
\pgfpathmoveto{\pgfqpoint{2.982382in}{3.388824in}}%
\pgfpathlineto{\pgfqpoint{2.982382in}{3.516877in}}%
\pgfpathlineto{\pgfqpoint{2.913379in}{3.479681in}}%
\pgfpathlineto{\pgfqpoint{2.802482in}{3.415655in}}%
\pgfpathlineto{\pgfqpoint{2.982382in}{3.388824in}}%
\pgfpathclose%
\pgfusepath{fill}%
\end{pgfscope}%
\begin{pgfscope}%
\pgfpathrectangle{\pgfqpoint{0.765000in}{0.660000in}}{\pgfqpoint{4.620000in}{4.620000in}}%
\pgfusepath{clip}%
\pgfsetbuttcap%
\pgfsetroundjoin%
\definecolor{currentfill}{rgb}{1.000000,0.894118,0.788235}%
\pgfsetfillcolor{currentfill}%
\pgfsetlinewidth{0.000000pt}%
\definecolor{currentstroke}{rgb}{1.000000,0.894118,0.788235}%
\pgfsetstrokecolor{currentstroke}%
\pgfsetdash{}{0pt}%
\pgfpathmoveto{\pgfqpoint{2.982382in}{3.516877in}}%
\pgfpathlineto{\pgfqpoint{3.093279in}{3.580903in}}%
\pgfpathlineto{\pgfqpoint{3.024275in}{3.543707in}}%
\pgfpathlineto{\pgfqpoint{2.913379in}{3.479681in}}%
\pgfpathlineto{\pgfqpoint{2.982382in}{3.516877in}}%
\pgfpathclose%
\pgfusepath{fill}%
\end{pgfscope}%
\begin{pgfscope}%
\pgfpathrectangle{\pgfqpoint{0.765000in}{0.660000in}}{\pgfqpoint{4.620000in}{4.620000in}}%
\pgfusepath{clip}%
\pgfsetbuttcap%
\pgfsetroundjoin%
\definecolor{currentfill}{rgb}{1.000000,0.894118,0.788235}%
\pgfsetfillcolor{currentfill}%
\pgfsetlinewidth{0.000000pt}%
\definecolor{currentstroke}{rgb}{1.000000,0.894118,0.788235}%
\pgfsetstrokecolor{currentstroke}%
\pgfsetdash{}{0pt}%
\pgfpathmoveto{\pgfqpoint{2.871485in}{3.580903in}}%
\pgfpathlineto{\pgfqpoint{2.982382in}{3.516877in}}%
\pgfpathlineto{\pgfqpoint{2.913379in}{3.479681in}}%
\pgfpathlineto{\pgfqpoint{2.802482in}{3.543707in}}%
\pgfpathlineto{\pgfqpoint{2.871485in}{3.580903in}}%
\pgfpathclose%
\pgfusepath{fill}%
\end{pgfscope}%
\begin{pgfscope}%
\pgfpathrectangle{\pgfqpoint{0.765000in}{0.660000in}}{\pgfqpoint{4.620000in}{4.620000in}}%
\pgfusepath{clip}%
\pgfsetbuttcap%
\pgfsetroundjoin%
\definecolor{currentfill}{rgb}{1.000000,0.894118,0.788235}%
\pgfsetfillcolor{currentfill}%
\pgfsetlinewidth{0.000000pt}%
\definecolor{currentstroke}{rgb}{1.000000,0.894118,0.788235}%
\pgfsetstrokecolor{currentstroke}%
\pgfsetdash{}{0pt}%
\pgfpathmoveto{\pgfqpoint{3.463186in}{2.662942in}}%
\pgfpathlineto{\pgfqpoint{3.352289in}{2.598916in}}%
\pgfpathlineto{\pgfqpoint{3.463186in}{2.534890in}}%
\pgfpathlineto{\pgfqpoint{3.574082in}{2.598916in}}%
\pgfpathlineto{\pgfqpoint{3.463186in}{2.662942in}}%
\pgfpathclose%
\pgfusepath{fill}%
\end{pgfscope}%
\begin{pgfscope}%
\pgfpathrectangle{\pgfqpoint{0.765000in}{0.660000in}}{\pgfqpoint{4.620000in}{4.620000in}}%
\pgfusepath{clip}%
\pgfsetbuttcap%
\pgfsetroundjoin%
\definecolor{currentfill}{rgb}{1.000000,0.894118,0.788235}%
\pgfsetfillcolor{currentfill}%
\pgfsetlinewidth{0.000000pt}%
\definecolor{currentstroke}{rgb}{1.000000,0.894118,0.788235}%
\pgfsetstrokecolor{currentstroke}%
\pgfsetdash{}{0pt}%
\pgfpathmoveto{\pgfqpoint{3.463186in}{2.662942in}}%
\pgfpathlineto{\pgfqpoint{3.352289in}{2.598916in}}%
\pgfpathlineto{\pgfqpoint{3.352289in}{2.726968in}}%
\pgfpathlineto{\pgfqpoint{3.463186in}{2.790995in}}%
\pgfpathlineto{\pgfqpoint{3.463186in}{2.662942in}}%
\pgfpathclose%
\pgfusepath{fill}%
\end{pgfscope}%
\begin{pgfscope}%
\pgfpathrectangle{\pgfqpoint{0.765000in}{0.660000in}}{\pgfqpoint{4.620000in}{4.620000in}}%
\pgfusepath{clip}%
\pgfsetbuttcap%
\pgfsetroundjoin%
\definecolor{currentfill}{rgb}{1.000000,0.894118,0.788235}%
\pgfsetfillcolor{currentfill}%
\pgfsetlinewidth{0.000000pt}%
\definecolor{currentstroke}{rgb}{1.000000,0.894118,0.788235}%
\pgfsetstrokecolor{currentstroke}%
\pgfsetdash{}{0pt}%
\pgfpathmoveto{\pgfqpoint{3.463186in}{2.662942in}}%
\pgfpathlineto{\pgfqpoint{3.574082in}{2.598916in}}%
\pgfpathlineto{\pgfqpoint{3.574082in}{2.726968in}}%
\pgfpathlineto{\pgfqpoint{3.463186in}{2.790995in}}%
\pgfpathlineto{\pgfqpoint{3.463186in}{2.662942in}}%
\pgfpathclose%
\pgfusepath{fill}%
\end{pgfscope}%
\begin{pgfscope}%
\pgfpathrectangle{\pgfqpoint{0.765000in}{0.660000in}}{\pgfqpoint{4.620000in}{4.620000in}}%
\pgfusepath{clip}%
\pgfsetbuttcap%
\pgfsetroundjoin%
\definecolor{currentfill}{rgb}{1.000000,0.894118,0.788235}%
\pgfsetfillcolor{currentfill}%
\pgfsetlinewidth{0.000000pt}%
\definecolor{currentstroke}{rgb}{1.000000,0.894118,0.788235}%
\pgfsetstrokecolor{currentstroke}%
\pgfsetdash{}{0pt}%
\pgfpathmoveto{\pgfqpoint{3.477043in}{2.674663in}}%
\pgfpathlineto{\pgfqpoint{3.366147in}{2.610637in}}%
\pgfpathlineto{\pgfqpoint{3.477043in}{2.546611in}}%
\pgfpathlineto{\pgfqpoint{3.587940in}{2.610637in}}%
\pgfpathlineto{\pgfqpoint{3.477043in}{2.674663in}}%
\pgfpathclose%
\pgfusepath{fill}%
\end{pgfscope}%
\begin{pgfscope}%
\pgfpathrectangle{\pgfqpoint{0.765000in}{0.660000in}}{\pgfqpoint{4.620000in}{4.620000in}}%
\pgfusepath{clip}%
\pgfsetbuttcap%
\pgfsetroundjoin%
\definecolor{currentfill}{rgb}{1.000000,0.894118,0.788235}%
\pgfsetfillcolor{currentfill}%
\pgfsetlinewidth{0.000000pt}%
\definecolor{currentstroke}{rgb}{1.000000,0.894118,0.788235}%
\pgfsetstrokecolor{currentstroke}%
\pgfsetdash{}{0pt}%
\pgfpathmoveto{\pgfqpoint{3.477043in}{2.674663in}}%
\pgfpathlineto{\pgfqpoint{3.366147in}{2.610637in}}%
\pgfpathlineto{\pgfqpoint{3.366147in}{2.738689in}}%
\pgfpathlineto{\pgfqpoint{3.477043in}{2.802716in}}%
\pgfpathlineto{\pgfqpoint{3.477043in}{2.674663in}}%
\pgfpathclose%
\pgfusepath{fill}%
\end{pgfscope}%
\begin{pgfscope}%
\pgfpathrectangle{\pgfqpoint{0.765000in}{0.660000in}}{\pgfqpoint{4.620000in}{4.620000in}}%
\pgfusepath{clip}%
\pgfsetbuttcap%
\pgfsetroundjoin%
\definecolor{currentfill}{rgb}{1.000000,0.894118,0.788235}%
\pgfsetfillcolor{currentfill}%
\pgfsetlinewidth{0.000000pt}%
\definecolor{currentstroke}{rgb}{1.000000,0.894118,0.788235}%
\pgfsetstrokecolor{currentstroke}%
\pgfsetdash{}{0pt}%
\pgfpathmoveto{\pgfqpoint{3.477043in}{2.674663in}}%
\pgfpathlineto{\pgfqpoint{3.587940in}{2.610637in}}%
\pgfpathlineto{\pgfqpoint{3.587940in}{2.738689in}}%
\pgfpathlineto{\pgfqpoint{3.477043in}{2.802716in}}%
\pgfpathlineto{\pgfqpoint{3.477043in}{2.674663in}}%
\pgfpathclose%
\pgfusepath{fill}%
\end{pgfscope}%
\begin{pgfscope}%
\pgfpathrectangle{\pgfqpoint{0.765000in}{0.660000in}}{\pgfqpoint{4.620000in}{4.620000in}}%
\pgfusepath{clip}%
\pgfsetbuttcap%
\pgfsetroundjoin%
\definecolor{currentfill}{rgb}{1.000000,0.894118,0.788235}%
\pgfsetfillcolor{currentfill}%
\pgfsetlinewidth{0.000000pt}%
\definecolor{currentstroke}{rgb}{1.000000,0.894118,0.788235}%
\pgfsetstrokecolor{currentstroke}%
\pgfsetdash{}{0pt}%
\pgfpathmoveto{\pgfqpoint{3.463186in}{2.790995in}}%
\pgfpathlineto{\pgfqpoint{3.352289in}{2.726968in}}%
\pgfpathlineto{\pgfqpoint{3.463186in}{2.662942in}}%
\pgfpathlineto{\pgfqpoint{3.574082in}{2.726968in}}%
\pgfpathlineto{\pgfqpoint{3.463186in}{2.790995in}}%
\pgfpathclose%
\pgfusepath{fill}%
\end{pgfscope}%
\begin{pgfscope}%
\pgfpathrectangle{\pgfqpoint{0.765000in}{0.660000in}}{\pgfqpoint{4.620000in}{4.620000in}}%
\pgfusepath{clip}%
\pgfsetbuttcap%
\pgfsetroundjoin%
\definecolor{currentfill}{rgb}{1.000000,0.894118,0.788235}%
\pgfsetfillcolor{currentfill}%
\pgfsetlinewidth{0.000000pt}%
\definecolor{currentstroke}{rgb}{1.000000,0.894118,0.788235}%
\pgfsetstrokecolor{currentstroke}%
\pgfsetdash{}{0pt}%
\pgfpathmoveto{\pgfqpoint{3.463186in}{2.534890in}}%
\pgfpathlineto{\pgfqpoint{3.574082in}{2.598916in}}%
\pgfpathlineto{\pgfqpoint{3.574082in}{2.726968in}}%
\pgfpathlineto{\pgfqpoint{3.463186in}{2.662942in}}%
\pgfpathlineto{\pgfqpoint{3.463186in}{2.534890in}}%
\pgfpathclose%
\pgfusepath{fill}%
\end{pgfscope}%
\begin{pgfscope}%
\pgfpathrectangle{\pgfqpoint{0.765000in}{0.660000in}}{\pgfqpoint{4.620000in}{4.620000in}}%
\pgfusepath{clip}%
\pgfsetbuttcap%
\pgfsetroundjoin%
\definecolor{currentfill}{rgb}{1.000000,0.894118,0.788235}%
\pgfsetfillcolor{currentfill}%
\pgfsetlinewidth{0.000000pt}%
\definecolor{currentstroke}{rgb}{1.000000,0.894118,0.788235}%
\pgfsetstrokecolor{currentstroke}%
\pgfsetdash{}{0pt}%
\pgfpathmoveto{\pgfqpoint{3.352289in}{2.598916in}}%
\pgfpathlineto{\pgfqpoint{3.463186in}{2.534890in}}%
\pgfpathlineto{\pgfqpoint{3.463186in}{2.662942in}}%
\pgfpathlineto{\pgfqpoint{3.352289in}{2.726968in}}%
\pgfpathlineto{\pgfqpoint{3.352289in}{2.598916in}}%
\pgfpathclose%
\pgfusepath{fill}%
\end{pgfscope}%
\begin{pgfscope}%
\pgfpathrectangle{\pgfqpoint{0.765000in}{0.660000in}}{\pgfqpoint{4.620000in}{4.620000in}}%
\pgfusepath{clip}%
\pgfsetbuttcap%
\pgfsetroundjoin%
\definecolor{currentfill}{rgb}{1.000000,0.894118,0.788235}%
\pgfsetfillcolor{currentfill}%
\pgfsetlinewidth{0.000000pt}%
\definecolor{currentstroke}{rgb}{1.000000,0.894118,0.788235}%
\pgfsetstrokecolor{currentstroke}%
\pgfsetdash{}{0pt}%
\pgfpathmoveto{\pgfqpoint{3.477043in}{2.802716in}}%
\pgfpathlineto{\pgfqpoint{3.366147in}{2.738689in}}%
\pgfpathlineto{\pgfqpoint{3.477043in}{2.674663in}}%
\pgfpathlineto{\pgfqpoint{3.587940in}{2.738689in}}%
\pgfpathlineto{\pgfqpoint{3.477043in}{2.802716in}}%
\pgfpathclose%
\pgfusepath{fill}%
\end{pgfscope}%
\begin{pgfscope}%
\pgfpathrectangle{\pgfqpoint{0.765000in}{0.660000in}}{\pgfqpoint{4.620000in}{4.620000in}}%
\pgfusepath{clip}%
\pgfsetbuttcap%
\pgfsetroundjoin%
\definecolor{currentfill}{rgb}{1.000000,0.894118,0.788235}%
\pgfsetfillcolor{currentfill}%
\pgfsetlinewidth{0.000000pt}%
\definecolor{currentstroke}{rgb}{1.000000,0.894118,0.788235}%
\pgfsetstrokecolor{currentstroke}%
\pgfsetdash{}{0pt}%
\pgfpathmoveto{\pgfqpoint{3.477043in}{2.546611in}}%
\pgfpathlineto{\pgfqpoint{3.587940in}{2.610637in}}%
\pgfpathlineto{\pgfqpoint{3.587940in}{2.738689in}}%
\pgfpathlineto{\pgfqpoint{3.477043in}{2.674663in}}%
\pgfpathlineto{\pgfqpoint{3.477043in}{2.546611in}}%
\pgfpathclose%
\pgfusepath{fill}%
\end{pgfscope}%
\begin{pgfscope}%
\pgfpathrectangle{\pgfqpoint{0.765000in}{0.660000in}}{\pgfqpoint{4.620000in}{4.620000in}}%
\pgfusepath{clip}%
\pgfsetbuttcap%
\pgfsetroundjoin%
\definecolor{currentfill}{rgb}{1.000000,0.894118,0.788235}%
\pgfsetfillcolor{currentfill}%
\pgfsetlinewidth{0.000000pt}%
\definecolor{currentstroke}{rgb}{1.000000,0.894118,0.788235}%
\pgfsetstrokecolor{currentstroke}%
\pgfsetdash{}{0pt}%
\pgfpathmoveto{\pgfqpoint{3.366147in}{2.610637in}}%
\pgfpathlineto{\pgfqpoint{3.477043in}{2.546611in}}%
\pgfpathlineto{\pgfqpoint{3.477043in}{2.674663in}}%
\pgfpathlineto{\pgfqpoint{3.366147in}{2.738689in}}%
\pgfpathlineto{\pgfqpoint{3.366147in}{2.610637in}}%
\pgfpathclose%
\pgfusepath{fill}%
\end{pgfscope}%
\begin{pgfscope}%
\pgfpathrectangle{\pgfqpoint{0.765000in}{0.660000in}}{\pgfqpoint{4.620000in}{4.620000in}}%
\pgfusepath{clip}%
\pgfsetbuttcap%
\pgfsetroundjoin%
\definecolor{currentfill}{rgb}{1.000000,0.894118,0.788235}%
\pgfsetfillcolor{currentfill}%
\pgfsetlinewidth{0.000000pt}%
\definecolor{currentstroke}{rgb}{1.000000,0.894118,0.788235}%
\pgfsetstrokecolor{currentstroke}%
\pgfsetdash{}{0pt}%
\pgfpathmoveto{\pgfqpoint{3.463186in}{2.662942in}}%
\pgfpathlineto{\pgfqpoint{3.352289in}{2.598916in}}%
\pgfpathlineto{\pgfqpoint{3.366147in}{2.610637in}}%
\pgfpathlineto{\pgfqpoint{3.477043in}{2.674663in}}%
\pgfpathlineto{\pgfqpoint{3.463186in}{2.662942in}}%
\pgfpathclose%
\pgfusepath{fill}%
\end{pgfscope}%
\begin{pgfscope}%
\pgfpathrectangle{\pgfqpoint{0.765000in}{0.660000in}}{\pgfqpoint{4.620000in}{4.620000in}}%
\pgfusepath{clip}%
\pgfsetbuttcap%
\pgfsetroundjoin%
\definecolor{currentfill}{rgb}{1.000000,0.894118,0.788235}%
\pgfsetfillcolor{currentfill}%
\pgfsetlinewidth{0.000000pt}%
\definecolor{currentstroke}{rgb}{1.000000,0.894118,0.788235}%
\pgfsetstrokecolor{currentstroke}%
\pgfsetdash{}{0pt}%
\pgfpathmoveto{\pgfqpoint{3.574082in}{2.598916in}}%
\pgfpathlineto{\pgfqpoint{3.463186in}{2.662942in}}%
\pgfpathlineto{\pgfqpoint{3.477043in}{2.674663in}}%
\pgfpathlineto{\pgfqpoint{3.587940in}{2.610637in}}%
\pgfpathlineto{\pgfqpoint{3.574082in}{2.598916in}}%
\pgfpathclose%
\pgfusepath{fill}%
\end{pgfscope}%
\begin{pgfscope}%
\pgfpathrectangle{\pgfqpoint{0.765000in}{0.660000in}}{\pgfqpoint{4.620000in}{4.620000in}}%
\pgfusepath{clip}%
\pgfsetbuttcap%
\pgfsetroundjoin%
\definecolor{currentfill}{rgb}{1.000000,0.894118,0.788235}%
\pgfsetfillcolor{currentfill}%
\pgfsetlinewidth{0.000000pt}%
\definecolor{currentstroke}{rgb}{1.000000,0.894118,0.788235}%
\pgfsetstrokecolor{currentstroke}%
\pgfsetdash{}{0pt}%
\pgfpathmoveto{\pgfqpoint{3.463186in}{2.662942in}}%
\pgfpathlineto{\pgfqpoint{3.463186in}{2.790995in}}%
\pgfpathlineto{\pgfqpoint{3.477043in}{2.802716in}}%
\pgfpathlineto{\pgfqpoint{3.587940in}{2.610637in}}%
\pgfpathlineto{\pgfqpoint{3.463186in}{2.662942in}}%
\pgfpathclose%
\pgfusepath{fill}%
\end{pgfscope}%
\begin{pgfscope}%
\pgfpathrectangle{\pgfqpoint{0.765000in}{0.660000in}}{\pgfqpoint{4.620000in}{4.620000in}}%
\pgfusepath{clip}%
\pgfsetbuttcap%
\pgfsetroundjoin%
\definecolor{currentfill}{rgb}{1.000000,0.894118,0.788235}%
\pgfsetfillcolor{currentfill}%
\pgfsetlinewidth{0.000000pt}%
\definecolor{currentstroke}{rgb}{1.000000,0.894118,0.788235}%
\pgfsetstrokecolor{currentstroke}%
\pgfsetdash{}{0pt}%
\pgfpathmoveto{\pgfqpoint{3.574082in}{2.598916in}}%
\pgfpathlineto{\pgfqpoint{3.574082in}{2.726968in}}%
\pgfpathlineto{\pgfqpoint{3.587940in}{2.738689in}}%
\pgfpathlineto{\pgfqpoint{3.477043in}{2.674663in}}%
\pgfpathlineto{\pgfqpoint{3.574082in}{2.598916in}}%
\pgfpathclose%
\pgfusepath{fill}%
\end{pgfscope}%
\begin{pgfscope}%
\pgfpathrectangle{\pgfqpoint{0.765000in}{0.660000in}}{\pgfqpoint{4.620000in}{4.620000in}}%
\pgfusepath{clip}%
\pgfsetbuttcap%
\pgfsetroundjoin%
\definecolor{currentfill}{rgb}{1.000000,0.894118,0.788235}%
\pgfsetfillcolor{currentfill}%
\pgfsetlinewidth{0.000000pt}%
\definecolor{currentstroke}{rgb}{1.000000,0.894118,0.788235}%
\pgfsetstrokecolor{currentstroke}%
\pgfsetdash{}{0pt}%
\pgfpathmoveto{\pgfqpoint{3.352289in}{2.598916in}}%
\pgfpathlineto{\pgfqpoint{3.463186in}{2.534890in}}%
\pgfpathlineto{\pgfqpoint{3.477043in}{2.546611in}}%
\pgfpathlineto{\pgfqpoint{3.366147in}{2.610637in}}%
\pgfpathlineto{\pgfqpoint{3.352289in}{2.598916in}}%
\pgfpathclose%
\pgfusepath{fill}%
\end{pgfscope}%
\begin{pgfscope}%
\pgfpathrectangle{\pgfqpoint{0.765000in}{0.660000in}}{\pgfqpoint{4.620000in}{4.620000in}}%
\pgfusepath{clip}%
\pgfsetbuttcap%
\pgfsetroundjoin%
\definecolor{currentfill}{rgb}{1.000000,0.894118,0.788235}%
\pgfsetfillcolor{currentfill}%
\pgfsetlinewidth{0.000000pt}%
\definecolor{currentstroke}{rgb}{1.000000,0.894118,0.788235}%
\pgfsetstrokecolor{currentstroke}%
\pgfsetdash{}{0pt}%
\pgfpathmoveto{\pgfqpoint{3.463186in}{2.534890in}}%
\pgfpathlineto{\pgfqpoint{3.574082in}{2.598916in}}%
\pgfpathlineto{\pgfqpoint{3.587940in}{2.610637in}}%
\pgfpathlineto{\pgfqpoint{3.477043in}{2.546611in}}%
\pgfpathlineto{\pgfqpoint{3.463186in}{2.534890in}}%
\pgfpathclose%
\pgfusepath{fill}%
\end{pgfscope}%
\begin{pgfscope}%
\pgfpathrectangle{\pgfqpoint{0.765000in}{0.660000in}}{\pgfqpoint{4.620000in}{4.620000in}}%
\pgfusepath{clip}%
\pgfsetbuttcap%
\pgfsetroundjoin%
\definecolor{currentfill}{rgb}{1.000000,0.894118,0.788235}%
\pgfsetfillcolor{currentfill}%
\pgfsetlinewidth{0.000000pt}%
\definecolor{currentstroke}{rgb}{1.000000,0.894118,0.788235}%
\pgfsetstrokecolor{currentstroke}%
\pgfsetdash{}{0pt}%
\pgfpathmoveto{\pgfqpoint{3.463186in}{2.790995in}}%
\pgfpathlineto{\pgfqpoint{3.352289in}{2.726968in}}%
\pgfpathlineto{\pgfqpoint{3.366147in}{2.738689in}}%
\pgfpathlineto{\pgfqpoint{3.477043in}{2.802716in}}%
\pgfpathlineto{\pgfqpoint{3.463186in}{2.790995in}}%
\pgfpathclose%
\pgfusepath{fill}%
\end{pgfscope}%
\begin{pgfscope}%
\pgfpathrectangle{\pgfqpoint{0.765000in}{0.660000in}}{\pgfqpoint{4.620000in}{4.620000in}}%
\pgfusepath{clip}%
\pgfsetbuttcap%
\pgfsetroundjoin%
\definecolor{currentfill}{rgb}{1.000000,0.894118,0.788235}%
\pgfsetfillcolor{currentfill}%
\pgfsetlinewidth{0.000000pt}%
\definecolor{currentstroke}{rgb}{1.000000,0.894118,0.788235}%
\pgfsetstrokecolor{currentstroke}%
\pgfsetdash{}{0pt}%
\pgfpathmoveto{\pgfqpoint{3.574082in}{2.726968in}}%
\pgfpathlineto{\pgfqpoint{3.463186in}{2.790995in}}%
\pgfpathlineto{\pgfqpoint{3.477043in}{2.802716in}}%
\pgfpathlineto{\pgfqpoint{3.587940in}{2.738689in}}%
\pgfpathlineto{\pgfqpoint{3.574082in}{2.726968in}}%
\pgfpathclose%
\pgfusepath{fill}%
\end{pgfscope}%
\begin{pgfscope}%
\pgfpathrectangle{\pgfqpoint{0.765000in}{0.660000in}}{\pgfqpoint{4.620000in}{4.620000in}}%
\pgfusepath{clip}%
\pgfsetbuttcap%
\pgfsetroundjoin%
\definecolor{currentfill}{rgb}{1.000000,0.894118,0.788235}%
\pgfsetfillcolor{currentfill}%
\pgfsetlinewidth{0.000000pt}%
\definecolor{currentstroke}{rgb}{1.000000,0.894118,0.788235}%
\pgfsetstrokecolor{currentstroke}%
\pgfsetdash{}{0pt}%
\pgfpathmoveto{\pgfqpoint{3.352289in}{2.598916in}}%
\pgfpathlineto{\pgfqpoint{3.352289in}{2.726968in}}%
\pgfpathlineto{\pgfqpoint{3.366147in}{2.738689in}}%
\pgfpathlineto{\pgfqpoint{3.477043in}{2.546611in}}%
\pgfpathlineto{\pgfqpoint{3.352289in}{2.598916in}}%
\pgfpathclose%
\pgfusepath{fill}%
\end{pgfscope}%
\begin{pgfscope}%
\pgfpathrectangle{\pgfqpoint{0.765000in}{0.660000in}}{\pgfqpoint{4.620000in}{4.620000in}}%
\pgfusepath{clip}%
\pgfsetbuttcap%
\pgfsetroundjoin%
\definecolor{currentfill}{rgb}{1.000000,0.894118,0.788235}%
\pgfsetfillcolor{currentfill}%
\pgfsetlinewidth{0.000000pt}%
\definecolor{currentstroke}{rgb}{1.000000,0.894118,0.788235}%
\pgfsetstrokecolor{currentstroke}%
\pgfsetdash{}{0pt}%
\pgfpathmoveto{\pgfqpoint{3.463186in}{2.534890in}}%
\pgfpathlineto{\pgfqpoint{3.463186in}{2.662942in}}%
\pgfpathlineto{\pgfqpoint{3.477043in}{2.674663in}}%
\pgfpathlineto{\pgfqpoint{3.366147in}{2.610637in}}%
\pgfpathlineto{\pgfqpoint{3.463186in}{2.534890in}}%
\pgfpathclose%
\pgfusepath{fill}%
\end{pgfscope}%
\begin{pgfscope}%
\pgfpathrectangle{\pgfqpoint{0.765000in}{0.660000in}}{\pgfqpoint{4.620000in}{4.620000in}}%
\pgfusepath{clip}%
\pgfsetbuttcap%
\pgfsetroundjoin%
\definecolor{currentfill}{rgb}{1.000000,0.894118,0.788235}%
\pgfsetfillcolor{currentfill}%
\pgfsetlinewidth{0.000000pt}%
\definecolor{currentstroke}{rgb}{1.000000,0.894118,0.788235}%
\pgfsetstrokecolor{currentstroke}%
\pgfsetdash{}{0pt}%
\pgfpathmoveto{\pgfqpoint{3.463186in}{2.662942in}}%
\pgfpathlineto{\pgfqpoint{3.574082in}{2.726968in}}%
\pgfpathlineto{\pgfqpoint{3.587940in}{2.738689in}}%
\pgfpathlineto{\pgfqpoint{3.477043in}{2.674663in}}%
\pgfpathlineto{\pgfqpoint{3.463186in}{2.662942in}}%
\pgfpathclose%
\pgfusepath{fill}%
\end{pgfscope}%
\begin{pgfscope}%
\pgfpathrectangle{\pgfqpoint{0.765000in}{0.660000in}}{\pgfqpoint{4.620000in}{4.620000in}}%
\pgfusepath{clip}%
\pgfsetbuttcap%
\pgfsetroundjoin%
\definecolor{currentfill}{rgb}{1.000000,0.894118,0.788235}%
\pgfsetfillcolor{currentfill}%
\pgfsetlinewidth{0.000000pt}%
\definecolor{currentstroke}{rgb}{1.000000,0.894118,0.788235}%
\pgfsetstrokecolor{currentstroke}%
\pgfsetdash{}{0pt}%
\pgfpathmoveto{\pgfqpoint{3.352289in}{2.726968in}}%
\pgfpathlineto{\pgfqpoint{3.463186in}{2.662942in}}%
\pgfpathlineto{\pgfqpoint{3.477043in}{2.674663in}}%
\pgfpathlineto{\pgfqpoint{3.366147in}{2.738689in}}%
\pgfpathlineto{\pgfqpoint{3.352289in}{2.726968in}}%
\pgfpathclose%
\pgfusepath{fill}%
\end{pgfscope}%
\begin{pgfscope}%
\pgfpathrectangle{\pgfqpoint{0.765000in}{0.660000in}}{\pgfqpoint{4.620000in}{4.620000in}}%
\pgfusepath{clip}%
\pgfsetbuttcap%
\pgfsetroundjoin%
\definecolor{currentfill}{rgb}{1.000000,0.894118,0.788235}%
\pgfsetfillcolor{currentfill}%
\pgfsetlinewidth{0.000000pt}%
\definecolor{currentstroke}{rgb}{1.000000,0.894118,0.788235}%
\pgfsetstrokecolor{currentstroke}%
\pgfsetdash{}{0pt}%
\pgfpathmoveto{\pgfqpoint{3.463186in}{2.662942in}}%
\pgfpathlineto{\pgfqpoint{3.352289in}{2.598916in}}%
\pgfpathlineto{\pgfqpoint{3.463186in}{2.534890in}}%
\pgfpathlineto{\pgfqpoint{3.574082in}{2.598916in}}%
\pgfpathlineto{\pgfqpoint{3.463186in}{2.662942in}}%
\pgfpathclose%
\pgfusepath{fill}%
\end{pgfscope}%
\begin{pgfscope}%
\pgfpathrectangle{\pgfqpoint{0.765000in}{0.660000in}}{\pgfqpoint{4.620000in}{4.620000in}}%
\pgfusepath{clip}%
\pgfsetbuttcap%
\pgfsetroundjoin%
\definecolor{currentfill}{rgb}{1.000000,0.894118,0.788235}%
\pgfsetfillcolor{currentfill}%
\pgfsetlinewidth{0.000000pt}%
\definecolor{currentstroke}{rgb}{1.000000,0.894118,0.788235}%
\pgfsetstrokecolor{currentstroke}%
\pgfsetdash{}{0pt}%
\pgfpathmoveto{\pgfqpoint{3.463186in}{2.662942in}}%
\pgfpathlineto{\pgfqpoint{3.352289in}{2.598916in}}%
\pgfpathlineto{\pgfqpoint{3.352289in}{2.726968in}}%
\pgfpathlineto{\pgfqpoint{3.463186in}{2.790995in}}%
\pgfpathlineto{\pgfqpoint{3.463186in}{2.662942in}}%
\pgfpathclose%
\pgfusepath{fill}%
\end{pgfscope}%
\begin{pgfscope}%
\pgfpathrectangle{\pgfqpoint{0.765000in}{0.660000in}}{\pgfqpoint{4.620000in}{4.620000in}}%
\pgfusepath{clip}%
\pgfsetbuttcap%
\pgfsetroundjoin%
\definecolor{currentfill}{rgb}{1.000000,0.894118,0.788235}%
\pgfsetfillcolor{currentfill}%
\pgfsetlinewidth{0.000000pt}%
\definecolor{currentstroke}{rgb}{1.000000,0.894118,0.788235}%
\pgfsetstrokecolor{currentstroke}%
\pgfsetdash{}{0pt}%
\pgfpathmoveto{\pgfqpoint{3.463186in}{2.662942in}}%
\pgfpathlineto{\pgfqpoint{3.574082in}{2.598916in}}%
\pgfpathlineto{\pgfqpoint{3.574082in}{2.726968in}}%
\pgfpathlineto{\pgfqpoint{3.463186in}{2.790995in}}%
\pgfpathlineto{\pgfqpoint{3.463186in}{2.662942in}}%
\pgfpathclose%
\pgfusepath{fill}%
\end{pgfscope}%
\begin{pgfscope}%
\pgfpathrectangle{\pgfqpoint{0.765000in}{0.660000in}}{\pgfqpoint{4.620000in}{4.620000in}}%
\pgfusepath{clip}%
\pgfsetbuttcap%
\pgfsetroundjoin%
\definecolor{currentfill}{rgb}{1.000000,0.894118,0.788235}%
\pgfsetfillcolor{currentfill}%
\pgfsetlinewidth{0.000000pt}%
\definecolor{currentstroke}{rgb}{1.000000,0.894118,0.788235}%
\pgfsetstrokecolor{currentstroke}%
\pgfsetdash{}{0pt}%
\pgfpathmoveto{\pgfqpoint{3.293363in}{2.546452in}}%
\pgfpathlineto{\pgfqpoint{3.182466in}{2.482426in}}%
\pgfpathlineto{\pgfqpoint{3.293363in}{2.418400in}}%
\pgfpathlineto{\pgfqpoint{3.404259in}{2.482426in}}%
\pgfpathlineto{\pgfqpoint{3.293363in}{2.546452in}}%
\pgfpathclose%
\pgfusepath{fill}%
\end{pgfscope}%
\begin{pgfscope}%
\pgfpathrectangle{\pgfqpoint{0.765000in}{0.660000in}}{\pgfqpoint{4.620000in}{4.620000in}}%
\pgfusepath{clip}%
\pgfsetbuttcap%
\pgfsetroundjoin%
\definecolor{currentfill}{rgb}{1.000000,0.894118,0.788235}%
\pgfsetfillcolor{currentfill}%
\pgfsetlinewidth{0.000000pt}%
\definecolor{currentstroke}{rgb}{1.000000,0.894118,0.788235}%
\pgfsetstrokecolor{currentstroke}%
\pgfsetdash{}{0pt}%
\pgfpathmoveto{\pgfqpoint{3.293363in}{2.546452in}}%
\pgfpathlineto{\pgfqpoint{3.182466in}{2.482426in}}%
\pgfpathlineto{\pgfqpoint{3.182466in}{2.610478in}}%
\pgfpathlineto{\pgfqpoint{3.293363in}{2.674505in}}%
\pgfpathlineto{\pgfqpoint{3.293363in}{2.546452in}}%
\pgfpathclose%
\pgfusepath{fill}%
\end{pgfscope}%
\begin{pgfscope}%
\pgfpathrectangle{\pgfqpoint{0.765000in}{0.660000in}}{\pgfqpoint{4.620000in}{4.620000in}}%
\pgfusepath{clip}%
\pgfsetbuttcap%
\pgfsetroundjoin%
\definecolor{currentfill}{rgb}{1.000000,0.894118,0.788235}%
\pgfsetfillcolor{currentfill}%
\pgfsetlinewidth{0.000000pt}%
\definecolor{currentstroke}{rgb}{1.000000,0.894118,0.788235}%
\pgfsetstrokecolor{currentstroke}%
\pgfsetdash{}{0pt}%
\pgfpathmoveto{\pgfqpoint{3.293363in}{2.546452in}}%
\pgfpathlineto{\pgfqpoint{3.404259in}{2.482426in}}%
\pgfpathlineto{\pgfqpoint{3.404259in}{2.610478in}}%
\pgfpathlineto{\pgfqpoint{3.293363in}{2.674505in}}%
\pgfpathlineto{\pgfqpoint{3.293363in}{2.546452in}}%
\pgfpathclose%
\pgfusepath{fill}%
\end{pgfscope}%
\begin{pgfscope}%
\pgfpathrectangle{\pgfqpoint{0.765000in}{0.660000in}}{\pgfqpoint{4.620000in}{4.620000in}}%
\pgfusepath{clip}%
\pgfsetbuttcap%
\pgfsetroundjoin%
\definecolor{currentfill}{rgb}{1.000000,0.894118,0.788235}%
\pgfsetfillcolor{currentfill}%
\pgfsetlinewidth{0.000000pt}%
\definecolor{currentstroke}{rgb}{1.000000,0.894118,0.788235}%
\pgfsetstrokecolor{currentstroke}%
\pgfsetdash{}{0pt}%
\pgfpathmoveto{\pgfqpoint{3.463186in}{2.790995in}}%
\pgfpathlineto{\pgfqpoint{3.352289in}{2.726968in}}%
\pgfpathlineto{\pgfqpoint{3.463186in}{2.662942in}}%
\pgfpathlineto{\pgfqpoint{3.574082in}{2.726968in}}%
\pgfpathlineto{\pgfqpoint{3.463186in}{2.790995in}}%
\pgfpathclose%
\pgfusepath{fill}%
\end{pgfscope}%
\begin{pgfscope}%
\pgfpathrectangle{\pgfqpoint{0.765000in}{0.660000in}}{\pgfqpoint{4.620000in}{4.620000in}}%
\pgfusepath{clip}%
\pgfsetbuttcap%
\pgfsetroundjoin%
\definecolor{currentfill}{rgb}{1.000000,0.894118,0.788235}%
\pgfsetfillcolor{currentfill}%
\pgfsetlinewidth{0.000000pt}%
\definecolor{currentstroke}{rgb}{1.000000,0.894118,0.788235}%
\pgfsetstrokecolor{currentstroke}%
\pgfsetdash{}{0pt}%
\pgfpathmoveto{\pgfqpoint{3.463186in}{2.534890in}}%
\pgfpathlineto{\pgfqpoint{3.574082in}{2.598916in}}%
\pgfpathlineto{\pgfqpoint{3.574082in}{2.726968in}}%
\pgfpathlineto{\pgfqpoint{3.463186in}{2.662942in}}%
\pgfpathlineto{\pgfqpoint{3.463186in}{2.534890in}}%
\pgfpathclose%
\pgfusepath{fill}%
\end{pgfscope}%
\begin{pgfscope}%
\pgfpathrectangle{\pgfqpoint{0.765000in}{0.660000in}}{\pgfqpoint{4.620000in}{4.620000in}}%
\pgfusepath{clip}%
\pgfsetbuttcap%
\pgfsetroundjoin%
\definecolor{currentfill}{rgb}{1.000000,0.894118,0.788235}%
\pgfsetfillcolor{currentfill}%
\pgfsetlinewidth{0.000000pt}%
\definecolor{currentstroke}{rgb}{1.000000,0.894118,0.788235}%
\pgfsetstrokecolor{currentstroke}%
\pgfsetdash{}{0pt}%
\pgfpathmoveto{\pgfqpoint{3.352289in}{2.598916in}}%
\pgfpathlineto{\pgfqpoint{3.463186in}{2.534890in}}%
\pgfpathlineto{\pgfqpoint{3.463186in}{2.662942in}}%
\pgfpathlineto{\pgfqpoint{3.352289in}{2.726968in}}%
\pgfpathlineto{\pgfqpoint{3.352289in}{2.598916in}}%
\pgfpathclose%
\pgfusepath{fill}%
\end{pgfscope}%
\begin{pgfscope}%
\pgfpathrectangle{\pgfqpoint{0.765000in}{0.660000in}}{\pgfqpoint{4.620000in}{4.620000in}}%
\pgfusepath{clip}%
\pgfsetbuttcap%
\pgfsetroundjoin%
\definecolor{currentfill}{rgb}{1.000000,0.894118,0.788235}%
\pgfsetfillcolor{currentfill}%
\pgfsetlinewidth{0.000000pt}%
\definecolor{currentstroke}{rgb}{1.000000,0.894118,0.788235}%
\pgfsetstrokecolor{currentstroke}%
\pgfsetdash{}{0pt}%
\pgfpathmoveto{\pgfqpoint{3.293363in}{2.674505in}}%
\pgfpathlineto{\pgfqpoint{3.182466in}{2.610478in}}%
\pgfpathlineto{\pgfqpoint{3.293363in}{2.546452in}}%
\pgfpathlineto{\pgfqpoint{3.404259in}{2.610478in}}%
\pgfpathlineto{\pgfqpoint{3.293363in}{2.674505in}}%
\pgfpathclose%
\pgfusepath{fill}%
\end{pgfscope}%
\begin{pgfscope}%
\pgfpathrectangle{\pgfqpoint{0.765000in}{0.660000in}}{\pgfqpoint{4.620000in}{4.620000in}}%
\pgfusepath{clip}%
\pgfsetbuttcap%
\pgfsetroundjoin%
\definecolor{currentfill}{rgb}{1.000000,0.894118,0.788235}%
\pgfsetfillcolor{currentfill}%
\pgfsetlinewidth{0.000000pt}%
\definecolor{currentstroke}{rgb}{1.000000,0.894118,0.788235}%
\pgfsetstrokecolor{currentstroke}%
\pgfsetdash{}{0pt}%
\pgfpathmoveto{\pgfqpoint{3.293363in}{2.418400in}}%
\pgfpathlineto{\pgfqpoint{3.404259in}{2.482426in}}%
\pgfpathlineto{\pgfqpoint{3.404259in}{2.610478in}}%
\pgfpathlineto{\pgfqpoint{3.293363in}{2.546452in}}%
\pgfpathlineto{\pgfqpoint{3.293363in}{2.418400in}}%
\pgfpathclose%
\pgfusepath{fill}%
\end{pgfscope}%
\begin{pgfscope}%
\pgfpathrectangle{\pgfqpoint{0.765000in}{0.660000in}}{\pgfqpoint{4.620000in}{4.620000in}}%
\pgfusepath{clip}%
\pgfsetbuttcap%
\pgfsetroundjoin%
\definecolor{currentfill}{rgb}{1.000000,0.894118,0.788235}%
\pgfsetfillcolor{currentfill}%
\pgfsetlinewidth{0.000000pt}%
\definecolor{currentstroke}{rgb}{1.000000,0.894118,0.788235}%
\pgfsetstrokecolor{currentstroke}%
\pgfsetdash{}{0pt}%
\pgfpathmoveto{\pgfqpoint{3.182466in}{2.482426in}}%
\pgfpathlineto{\pgfqpoint{3.293363in}{2.418400in}}%
\pgfpathlineto{\pgfqpoint{3.293363in}{2.546452in}}%
\pgfpathlineto{\pgfqpoint{3.182466in}{2.610478in}}%
\pgfpathlineto{\pgfqpoint{3.182466in}{2.482426in}}%
\pgfpathclose%
\pgfusepath{fill}%
\end{pgfscope}%
\begin{pgfscope}%
\pgfpathrectangle{\pgfqpoint{0.765000in}{0.660000in}}{\pgfqpoint{4.620000in}{4.620000in}}%
\pgfusepath{clip}%
\pgfsetbuttcap%
\pgfsetroundjoin%
\definecolor{currentfill}{rgb}{1.000000,0.894118,0.788235}%
\pgfsetfillcolor{currentfill}%
\pgfsetlinewidth{0.000000pt}%
\definecolor{currentstroke}{rgb}{1.000000,0.894118,0.788235}%
\pgfsetstrokecolor{currentstroke}%
\pgfsetdash{}{0pt}%
\pgfpathmoveto{\pgfqpoint{3.463186in}{2.662942in}}%
\pgfpathlineto{\pgfqpoint{3.352289in}{2.598916in}}%
\pgfpathlineto{\pgfqpoint{3.182466in}{2.482426in}}%
\pgfpathlineto{\pgfqpoint{3.293363in}{2.546452in}}%
\pgfpathlineto{\pgfqpoint{3.463186in}{2.662942in}}%
\pgfpathclose%
\pgfusepath{fill}%
\end{pgfscope}%
\begin{pgfscope}%
\pgfpathrectangle{\pgfqpoint{0.765000in}{0.660000in}}{\pgfqpoint{4.620000in}{4.620000in}}%
\pgfusepath{clip}%
\pgfsetbuttcap%
\pgfsetroundjoin%
\definecolor{currentfill}{rgb}{1.000000,0.894118,0.788235}%
\pgfsetfillcolor{currentfill}%
\pgfsetlinewidth{0.000000pt}%
\definecolor{currentstroke}{rgb}{1.000000,0.894118,0.788235}%
\pgfsetstrokecolor{currentstroke}%
\pgfsetdash{}{0pt}%
\pgfpathmoveto{\pgfqpoint{3.574082in}{2.598916in}}%
\pgfpathlineto{\pgfqpoint{3.463186in}{2.662942in}}%
\pgfpathlineto{\pgfqpoint{3.293363in}{2.546452in}}%
\pgfpathlineto{\pgfqpoint{3.404259in}{2.482426in}}%
\pgfpathlineto{\pgfqpoint{3.574082in}{2.598916in}}%
\pgfpathclose%
\pgfusepath{fill}%
\end{pgfscope}%
\begin{pgfscope}%
\pgfpathrectangle{\pgfqpoint{0.765000in}{0.660000in}}{\pgfqpoint{4.620000in}{4.620000in}}%
\pgfusepath{clip}%
\pgfsetbuttcap%
\pgfsetroundjoin%
\definecolor{currentfill}{rgb}{1.000000,0.894118,0.788235}%
\pgfsetfillcolor{currentfill}%
\pgfsetlinewidth{0.000000pt}%
\definecolor{currentstroke}{rgb}{1.000000,0.894118,0.788235}%
\pgfsetstrokecolor{currentstroke}%
\pgfsetdash{}{0pt}%
\pgfpathmoveto{\pgfqpoint{3.463186in}{2.662942in}}%
\pgfpathlineto{\pgfqpoint{3.463186in}{2.790995in}}%
\pgfpathlineto{\pgfqpoint{3.293363in}{2.674505in}}%
\pgfpathlineto{\pgfqpoint{3.404259in}{2.482426in}}%
\pgfpathlineto{\pgfqpoint{3.463186in}{2.662942in}}%
\pgfpathclose%
\pgfusepath{fill}%
\end{pgfscope}%
\begin{pgfscope}%
\pgfpathrectangle{\pgfqpoint{0.765000in}{0.660000in}}{\pgfqpoint{4.620000in}{4.620000in}}%
\pgfusepath{clip}%
\pgfsetbuttcap%
\pgfsetroundjoin%
\definecolor{currentfill}{rgb}{1.000000,0.894118,0.788235}%
\pgfsetfillcolor{currentfill}%
\pgfsetlinewidth{0.000000pt}%
\definecolor{currentstroke}{rgb}{1.000000,0.894118,0.788235}%
\pgfsetstrokecolor{currentstroke}%
\pgfsetdash{}{0pt}%
\pgfpathmoveto{\pgfqpoint{3.574082in}{2.598916in}}%
\pgfpathlineto{\pgfqpoint{3.574082in}{2.726968in}}%
\pgfpathlineto{\pgfqpoint{3.404259in}{2.610478in}}%
\pgfpathlineto{\pgfqpoint{3.293363in}{2.546452in}}%
\pgfpathlineto{\pgfqpoint{3.574082in}{2.598916in}}%
\pgfpathclose%
\pgfusepath{fill}%
\end{pgfscope}%
\begin{pgfscope}%
\pgfpathrectangle{\pgfqpoint{0.765000in}{0.660000in}}{\pgfqpoint{4.620000in}{4.620000in}}%
\pgfusepath{clip}%
\pgfsetbuttcap%
\pgfsetroundjoin%
\definecolor{currentfill}{rgb}{1.000000,0.894118,0.788235}%
\pgfsetfillcolor{currentfill}%
\pgfsetlinewidth{0.000000pt}%
\definecolor{currentstroke}{rgb}{1.000000,0.894118,0.788235}%
\pgfsetstrokecolor{currentstroke}%
\pgfsetdash{}{0pt}%
\pgfpathmoveto{\pgfqpoint{3.352289in}{2.598916in}}%
\pgfpathlineto{\pgfqpoint{3.463186in}{2.534890in}}%
\pgfpathlineto{\pgfqpoint{3.293363in}{2.418400in}}%
\pgfpathlineto{\pgfqpoint{3.182466in}{2.482426in}}%
\pgfpathlineto{\pgfqpoint{3.352289in}{2.598916in}}%
\pgfpathclose%
\pgfusepath{fill}%
\end{pgfscope}%
\begin{pgfscope}%
\pgfpathrectangle{\pgfqpoint{0.765000in}{0.660000in}}{\pgfqpoint{4.620000in}{4.620000in}}%
\pgfusepath{clip}%
\pgfsetbuttcap%
\pgfsetroundjoin%
\definecolor{currentfill}{rgb}{1.000000,0.894118,0.788235}%
\pgfsetfillcolor{currentfill}%
\pgfsetlinewidth{0.000000pt}%
\definecolor{currentstroke}{rgb}{1.000000,0.894118,0.788235}%
\pgfsetstrokecolor{currentstroke}%
\pgfsetdash{}{0pt}%
\pgfpathmoveto{\pgfqpoint{3.463186in}{2.534890in}}%
\pgfpathlineto{\pgfqpoint{3.574082in}{2.598916in}}%
\pgfpathlineto{\pgfqpoint{3.404259in}{2.482426in}}%
\pgfpathlineto{\pgfqpoint{3.293363in}{2.418400in}}%
\pgfpathlineto{\pgfqpoint{3.463186in}{2.534890in}}%
\pgfpathclose%
\pgfusepath{fill}%
\end{pgfscope}%
\begin{pgfscope}%
\pgfpathrectangle{\pgfqpoint{0.765000in}{0.660000in}}{\pgfqpoint{4.620000in}{4.620000in}}%
\pgfusepath{clip}%
\pgfsetbuttcap%
\pgfsetroundjoin%
\definecolor{currentfill}{rgb}{1.000000,0.894118,0.788235}%
\pgfsetfillcolor{currentfill}%
\pgfsetlinewidth{0.000000pt}%
\definecolor{currentstroke}{rgb}{1.000000,0.894118,0.788235}%
\pgfsetstrokecolor{currentstroke}%
\pgfsetdash{}{0pt}%
\pgfpathmoveto{\pgfqpoint{3.463186in}{2.790995in}}%
\pgfpathlineto{\pgfqpoint{3.352289in}{2.726968in}}%
\pgfpathlineto{\pgfqpoint{3.182466in}{2.610478in}}%
\pgfpathlineto{\pgfqpoint{3.293363in}{2.674505in}}%
\pgfpathlineto{\pgfqpoint{3.463186in}{2.790995in}}%
\pgfpathclose%
\pgfusepath{fill}%
\end{pgfscope}%
\begin{pgfscope}%
\pgfpathrectangle{\pgfqpoint{0.765000in}{0.660000in}}{\pgfqpoint{4.620000in}{4.620000in}}%
\pgfusepath{clip}%
\pgfsetbuttcap%
\pgfsetroundjoin%
\definecolor{currentfill}{rgb}{1.000000,0.894118,0.788235}%
\pgfsetfillcolor{currentfill}%
\pgfsetlinewidth{0.000000pt}%
\definecolor{currentstroke}{rgb}{1.000000,0.894118,0.788235}%
\pgfsetstrokecolor{currentstroke}%
\pgfsetdash{}{0pt}%
\pgfpathmoveto{\pgfqpoint{3.574082in}{2.726968in}}%
\pgfpathlineto{\pgfqpoint{3.463186in}{2.790995in}}%
\pgfpathlineto{\pgfqpoint{3.293363in}{2.674505in}}%
\pgfpathlineto{\pgfqpoint{3.404259in}{2.610478in}}%
\pgfpathlineto{\pgfqpoint{3.574082in}{2.726968in}}%
\pgfpathclose%
\pgfusepath{fill}%
\end{pgfscope}%
\begin{pgfscope}%
\pgfpathrectangle{\pgfqpoint{0.765000in}{0.660000in}}{\pgfqpoint{4.620000in}{4.620000in}}%
\pgfusepath{clip}%
\pgfsetbuttcap%
\pgfsetroundjoin%
\definecolor{currentfill}{rgb}{1.000000,0.894118,0.788235}%
\pgfsetfillcolor{currentfill}%
\pgfsetlinewidth{0.000000pt}%
\definecolor{currentstroke}{rgb}{1.000000,0.894118,0.788235}%
\pgfsetstrokecolor{currentstroke}%
\pgfsetdash{}{0pt}%
\pgfpathmoveto{\pgfqpoint{3.352289in}{2.598916in}}%
\pgfpathlineto{\pgfqpoint{3.352289in}{2.726968in}}%
\pgfpathlineto{\pgfqpoint{3.182466in}{2.610478in}}%
\pgfpathlineto{\pgfqpoint{3.293363in}{2.418400in}}%
\pgfpathlineto{\pgfqpoint{3.352289in}{2.598916in}}%
\pgfpathclose%
\pgfusepath{fill}%
\end{pgfscope}%
\begin{pgfscope}%
\pgfpathrectangle{\pgfqpoint{0.765000in}{0.660000in}}{\pgfqpoint{4.620000in}{4.620000in}}%
\pgfusepath{clip}%
\pgfsetbuttcap%
\pgfsetroundjoin%
\definecolor{currentfill}{rgb}{1.000000,0.894118,0.788235}%
\pgfsetfillcolor{currentfill}%
\pgfsetlinewidth{0.000000pt}%
\definecolor{currentstroke}{rgb}{1.000000,0.894118,0.788235}%
\pgfsetstrokecolor{currentstroke}%
\pgfsetdash{}{0pt}%
\pgfpathmoveto{\pgfqpoint{3.463186in}{2.534890in}}%
\pgfpathlineto{\pgfqpoint{3.463186in}{2.662942in}}%
\pgfpathlineto{\pgfqpoint{3.293363in}{2.546452in}}%
\pgfpathlineto{\pgfqpoint{3.182466in}{2.482426in}}%
\pgfpathlineto{\pgfqpoint{3.463186in}{2.534890in}}%
\pgfpathclose%
\pgfusepath{fill}%
\end{pgfscope}%
\begin{pgfscope}%
\pgfpathrectangle{\pgfqpoint{0.765000in}{0.660000in}}{\pgfqpoint{4.620000in}{4.620000in}}%
\pgfusepath{clip}%
\pgfsetbuttcap%
\pgfsetroundjoin%
\definecolor{currentfill}{rgb}{1.000000,0.894118,0.788235}%
\pgfsetfillcolor{currentfill}%
\pgfsetlinewidth{0.000000pt}%
\definecolor{currentstroke}{rgb}{1.000000,0.894118,0.788235}%
\pgfsetstrokecolor{currentstroke}%
\pgfsetdash{}{0pt}%
\pgfpathmoveto{\pgfqpoint{3.463186in}{2.662942in}}%
\pgfpathlineto{\pgfqpoint{3.574082in}{2.726968in}}%
\pgfpathlineto{\pgfqpoint{3.404259in}{2.610478in}}%
\pgfpathlineto{\pgfqpoint{3.293363in}{2.546452in}}%
\pgfpathlineto{\pgfqpoint{3.463186in}{2.662942in}}%
\pgfpathclose%
\pgfusepath{fill}%
\end{pgfscope}%
\begin{pgfscope}%
\pgfpathrectangle{\pgfqpoint{0.765000in}{0.660000in}}{\pgfqpoint{4.620000in}{4.620000in}}%
\pgfusepath{clip}%
\pgfsetbuttcap%
\pgfsetroundjoin%
\definecolor{currentfill}{rgb}{1.000000,0.894118,0.788235}%
\pgfsetfillcolor{currentfill}%
\pgfsetlinewidth{0.000000pt}%
\definecolor{currentstroke}{rgb}{1.000000,0.894118,0.788235}%
\pgfsetstrokecolor{currentstroke}%
\pgfsetdash{}{0pt}%
\pgfpathmoveto{\pgfqpoint{3.352289in}{2.726968in}}%
\pgfpathlineto{\pgfqpoint{3.463186in}{2.662942in}}%
\pgfpathlineto{\pgfqpoint{3.293363in}{2.546452in}}%
\pgfpathlineto{\pgfqpoint{3.182466in}{2.610478in}}%
\pgfpathlineto{\pgfqpoint{3.352289in}{2.726968in}}%
\pgfpathclose%
\pgfusepath{fill}%
\end{pgfscope}%
\begin{pgfscope}%
\pgfpathrectangle{\pgfqpoint{0.765000in}{0.660000in}}{\pgfqpoint{4.620000in}{4.620000in}}%
\pgfusepath{clip}%
\pgfsetbuttcap%
\pgfsetroundjoin%
\definecolor{currentfill}{rgb}{1.000000,0.894118,0.788235}%
\pgfsetfillcolor{currentfill}%
\pgfsetlinewidth{0.000000pt}%
\definecolor{currentstroke}{rgb}{1.000000,0.894118,0.788235}%
\pgfsetstrokecolor{currentstroke}%
\pgfsetdash{}{0pt}%
\pgfpathmoveto{\pgfqpoint{2.582969in}{3.161487in}}%
\pgfpathlineto{\pgfqpoint{2.472072in}{3.097461in}}%
\pgfpathlineto{\pgfqpoint{2.582969in}{3.033435in}}%
\pgfpathlineto{\pgfqpoint{2.693865in}{3.097461in}}%
\pgfpathlineto{\pgfqpoint{2.582969in}{3.161487in}}%
\pgfpathclose%
\pgfusepath{fill}%
\end{pgfscope}%
\begin{pgfscope}%
\pgfpathrectangle{\pgfqpoint{0.765000in}{0.660000in}}{\pgfqpoint{4.620000in}{4.620000in}}%
\pgfusepath{clip}%
\pgfsetbuttcap%
\pgfsetroundjoin%
\definecolor{currentfill}{rgb}{1.000000,0.894118,0.788235}%
\pgfsetfillcolor{currentfill}%
\pgfsetlinewidth{0.000000pt}%
\definecolor{currentstroke}{rgb}{1.000000,0.894118,0.788235}%
\pgfsetstrokecolor{currentstroke}%
\pgfsetdash{}{0pt}%
\pgfpathmoveto{\pgfqpoint{2.582969in}{3.161487in}}%
\pgfpathlineto{\pgfqpoint{2.472072in}{3.097461in}}%
\pgfpathlineto{\pgfqpoint{2.472072in}{3.225514in}}%
\pgfpathlineto{\pgfqpoint{2.582969in}{3.289540in}}%
\pgfpathlineto{\pgfqpoint{2.582969in}{3.161487in}}%
\pgfpathclose%
\pgfusepath{fill}%
\end{pgfscope}%
\begin{pgfscope}%
\pgfpathrectangle{\pgfqpoint{0.765000in}{0.660000in}}{\pgfqpoint{4.620000in}{4.620000in}}%
\pgfusepath{clip}%
\pgfsetbuttcap%
\pgfsetroundjoin%
\definecolor{currentfill}{rgb}{1.000000,0.894118,0.788235}%
\pgfsetfillcolor{currentfill}%
\pgfsetlinewidth{0.000000pt}%
\definecolor{currentstroke}{rgb}{1.000000,0.894118,0.788235}%
\pgfsetstrokecolor{currentstroke}%
\pgfsetdash{}{0pt}%
\pgfpathmoveto{\pgfqpoint{2.582969in}{3.161487in}}%
\pgfpathlineto{\pgfqpoint{2.693865in}{3.097461in}}%
\pgfpathlineto{\pgfqpoint{2.693865in}{3.225514in}}%
\pgfpathlineto{\pgfqpoint{2.582969in}{3.289540in}}%
\pgfpathlineto{\pgfqpoint{2.582969in}{3.161487in}}%
\pgfpathclose%
\pgfusepath{fill}%
\end{pgfscope}%
\begin{pgfscope}%
\pgfpathrectangle{\pgfqpoint{0.765000in}{0.660000in}}{\pgfqpoint{4.620000in}{4.620000in}}%
\pgfusepath{clip}%
\pgfsetbuttcap%
\pgfsetroundjoin%
\definecolor{currentfill}{rgb}{1.000000,0.894118,0.788235}%
\pgfsetfillcolor{currentfill}%
\pgfsetlinewidth{0.000000pt}%
\definecolor{currentstroke}{rgb}{1.000000,0.894118,0.788235}%
\pgfsetstrokecolor{currentstroke}%
\pgfsetdash{}{0pt}%
\pgfpathmoveto{\pgfqpoint{2.583171in}{3.110796in}}%
\pgfpathlineto{\pgfqpoint{2.472275in}{3.046769in}}%
\pgfpathlineto{\pgfqpoint{2.583171in}{2.982743in}}%
\pgfpathlineto{\pgfqpoint{2.694068in}{3.046769in}}%
\pgfpathlineto{\pgfqpoint{2.583171in}{3.110796in}}%
\pgfpathclose%
\pgfusepath{fill}%
\end{pgfscope}%
\begin{pgfscope}%
\pgfpathrectangle{\pgfqpoint{0.765000in}{0.660000in}}{\pgfqpoint{4.620000in}{4.620000in}}%
\pgfusepath{clip}%
\pgfsetbuttcap%
\pgfsetroundjoin%
\definecolor{currentfill}{rgb}{1.000000,0.894118,0.788235}%
\pgfsetfillcolor{currentfill}%
\pgfsetlinewidth{0.000000pt}%
\definecolor{currentstroke}{rgb}{1.000000,0.894118,0.788235}%
\pgfsetstrokecolor{currentstroke}%
\pgfsetdash{}{0pt}%
\pgfpathmoveto{\pgfqpoint{2.583171in}{3.110796in}}%
\pgfpathlineto{\pgfqpoint{2.472275in}{3.046769in}}%
\pgfpathlineto{\pgfqpoint{2.472275in}{3.174822in}}%
\pgfpathlineto{\pgfqpoint{2.583171in}{3.238848in}}%
\pgfpathlineto{\pgfqpoint{2.583171in}{3.110796in}}%
\pgfpathclose%
\pgfusepath{fill}%
\end{pgfscope}%
\begin{pgfscope}%
\pgfpathrectangle{\pgfqpoint{0.765000in}{0.660000in}}{\pgfqpoint{4.620000in}{4.620000in}}%
\pgfusepath{clip}%
\pgfsetbuttcap%
\pgfsetroundjoin%
\definecolor{currentfill}{rgb}{1.000000,0.894118,0.788235}%
\pgfsetfillcolor{currentfill}%
\pgfsetlinewidth{0.000000pt}%
\definecolor{currentstroke}{rgb}{1.000000,0.894118,0.788235}%
\pgfsetstrokecolor{currentstroke}%
\pgfsetdash{}{0pt}%
\pgfpathmoveto{\pgfqpoint{2.583171in}{3.110796in}}%
\pgfpathlineto{\pgfqpoint{2.694068in}{3.046769in}}%
\pgfpathlineto{\pgfqpoint{2.694068in}{3.174822in}}%
\pgfpathlineto{\pgfqpoint{2.583171in}{3.238848in}}%
\pgfpathlineto{\pgfqpoint{2.583171in}{3.110796in}}%
\pgfpathclose%
\pgfusepath{fill}%
\end{pgfscope}%
\begin{pgfscope}%
\pgfpathrectangle{\pgfqpoint{0.765000in}{0.660000in}}{\pgfqpoint{4.620000in}{4.620000in}}%
\pgfusepath{clip}%
\pgfsetbuttcap%
\pgfsetroundjoin%
\definecolor{currentfill}{rgb}{1.000000,0.894118,0.788235}%
\pgfsetfillcolor{currentfill}%
\pgfsetlinewidth{0.000000pt}%
\definecolor{currentstroke}{rgb}{1.000000,0.894118,0.788235}%
\pgfsetstrokecolor{currentstroke}%
\pgfsetdash{}{0pt}%
\pgfpathmoveto{\pgfqpoint{2.582969in}{3.289540in}}%
\pgfpathlineto{\pgfqpoint{2.472072in}{3.225514in}}%
\pgfpathlineto{\pgfqpoint{2.582969in}{3.161487in}}%
\pgfpathlineto{\pgfqpoint{2.693865in}{3.225514in}}%
\pgfpathlineto{\pgfqpoint{2.582969in}{3.289540in}}%
\pgfpathclose%
\pgfusepath{fill}%
\end{pgfscope}%
\begin{pgfscope}%
\pgfpathrectangle{\pgfqpoint{0.765000in}{0.660000in}}{\pgfqpoint{4.620000in}{4.620000in}}%
\pgfusepath{clip}%
\pgfsetbuttcap%
\pgfsetroundjoin%
\definecolor{currentfill}{rgb}{1.000000,0.894118,0.788235}%
\pgfsetfillcolor{currentfill}%
\pgfsetlinewidth{0.000000pt}%
\definecolor{currentstroke}{rgb}{1.000000,0.894118,0.788235}%
\pgfsetstrokecolor{currentstroke}%
\pgfsetdash{}{0pt}%
\pgfpathmoveto{\pgfqpoint{2.582969in}{3.033435in}}%
\pgfpathlineto{\pgfqpoint{2.693865in}{3.097461in}}%
\pgfpathlineto{\pgfqpoint{2.693865in}{3.225514in}}%
\pgfpathlineto{\pgfqpoint{2.582969in}{3.161487in}}%
\pgfpathlineto{\pgfqpoint{2.582969in}{3.033435in}}%
\pgfpathclose%
\pgfusepath{fill}%
\end{pgfscope}%
\begin{pgfscope}%
\pgfpathrectangle{\pgfqpoint{0.765000in}{0.660000in}}{\pgfqpoint{4.620000in}{4.620000in}}%
\pgfusepath{clip}%
\pgfsetbuttcap%
\pgfsetroundjoin%
\definecolor{currentfill}{rgb}{1.000000,0.894118,0.788235}%
\pgfsetfillcolor{currentfill}%
\pgfsetlinewidth{0.000000pt}%
\definecolor{currentstroke}{rgb}{1.000000,0.894118,0.788235}%
\pgfsetstrokecolor{currentstroke}%
\pgfsetdash{}{0pt}%
\pgfpathmoveto{\pgfqpoint{2.472072in}{3.097461in}}%
\pgfpathlineto{\pgfqpoint{2.582969in}{3.033435in}}%
\pgfpathlineto{\pgfqpoint{2.582969in}{3.161487in}}%
\pgfpathlineto{\pgfqpoint{2.472072in}{3.225514in}}%
\pgfpathlineto{\pgfqpoint{2.472072in}{3.097461in}}%
\pgfpathclose%
\pgfusepath{fill}%
\end{pgfscope}%
\begin{pgfscope}%
\pgfpathrectangle{\pgfqpoint{0.765000in}{0.660000in}}{\pgfqpoint{4.620000in}{4.620000in}}%
\pgfusepath{clip}%
\pgfsetbuttcap%
\pgfsetroundjoin%
\definecolor{currentfill}{rgb}{1.000000,0.894118,0.788235}%
\pgfsetfillcolor{currentfill}%
\pgfsetlinewidth{0.000000pt}%
\definecolor{currentstroke}{rgb}{1.000000,0.894118,0.788235}%
\pgfsetstrokecolor{currentstroke}%
\pgfsetdash{}{0pt}%
\pgfpathmoveto{\pgfqpoint{2.583171in}{3.238848in}}%
\pgfpathlineto{\pgfqpoint{2.472275in}{3.174822in}}%
\pgfpathlineto{\pgfqpoint{2.583171in}{3.110796in}}%
\pgfpathlineto{\pgfqpoint{2.694068in}{3.174822in}}%
\pgfpathlineto{\pgfqpoint{2.583171in}{3.238848in}}%
\pgfpathclose%
\pgfusepath{fill}%
\end{pgfscope}%
\begin{pgfscope}%
\pgfpathrectangle{\pgfqpoint{0.765000in}{0.660000in}}{\pgfqpoint{4.620000in}{4.620000in}}%
\pgfusepath{clip}%
\pgfsetbuttcap%
\pgfsetroundjoin%
\definecolor{currentfill}{rgb}{1.000000,0.894118,0.788235}%
\pgfsetfillcolor{currentfill}%
\pgfsetlinewidth{0.000000pt}%
\definecolor{currentstroke}{rgb}{1.000000,0.894118,0.788235}%
\pgfsetstrokecolor{currentstroke}%
\pgfsetdash{}{0pt}%
\pgfpathmoveto{\pgfqpoint{2.583171in}{2.982743in}}%
\pgfpathlineto{\pgfqpoint{2.694068in}{3.046769in}}%
\pgfpathlineto{\pgfqpoint{2.694068in}{3.174822in}}%
\pgfpathlineto{\pgfqpoint{2.583171in}{3.110796in}}%
\pgfpathlineto{\pgfqpoint{2.583171in}{2.982743in}}%
\pgfpathclose%
\pgfusepath{fill}%
\end{pgfscope}%
\begin{pgfscope}%
\pgfpathrectangle{\pgfqpoint{0.765000in}{0.660000in}}{\pgfqpoint{4.620000in}{4.620000in}}%
\pgfusepath{clip}%
\pgfsetbuttcap%
\pgfsetroundjoin%
\definecolor{currentfill}{rgb}{1.000000,0.894118,0.788235}%
\pgfsetfillcolor{currentfill}%
\pgfsetlinewidth{0.000000pt}%
\definecolor{currentstroke}{rgb}{1.000000,0.894118,0.788235}%
\pgfsetstrokecolor{currentstroke}%
\pgfsetdash{}{0pt}%
\pgfpathmoveto{\pgfqpoint{2.472275in}{3.046769in}}%
\pgfpathlineto{\pgfqpoint{2.583171in}{2.982743in}}%
\pgfpathlineto{\pgfqpoint{2.583171in}{3.110796in}}%
\pgfpathlineto{\pgfqpoint{2.472275in}{3.174822in}}%
\pgfpathlineto{\pgfqpoint{2.472275in}{3.046769in}}%
\pgfpathclose%
\pgfusepath{fill}%
\end{pgfscope}%
\begin{pgfscope}%
\pgfpathrectangle{\pgfqpoint{0.765000in}{0.660000in}}{\pgfqpoint{4.620000in}{4.620000in}}%
\pgfusepath{clip}%
\pgfsetbuttcap%
\pgfsetroundjoin%
\definecolor{currentfill}{rgb}{1.000000,0.894118,0.788235}%
\pgfsetfillcolor{currentfill}%
\pgfsetlinewidth{0.000000pt}%
\definecolor{currentstroke}{rgb}{1.000000,0.894118,0.788235}%
\pgfsetstrokecolor{currentstroke}%
\pgfsetdash{}{0pt}%
\pgfpathmoveto{\pgfqpoint{2.582969in}{3.161487in}}%
\pgfpathlineto{\pgfqpoint{2.472072in}{3.097461in}}%
\pgfpathlineto{\pgfqpoint{2.472275in}{3.046769in}}%
\pgfpathlineto{\pgfqpoint{2.583171in}{3.110796in}}%
\pgfpathlineto{\pgfqpoint{2.582969in}{3.161487in}}%
\pgfpathclose%
\pgfusepath{fill}%
\end{pgfscope}%
\begin{pgfscope}%
\pgfpathrectangle{\pgfqpoint{0.765000in}{0.660000in}}{\pgfqpoint{4.620000in}{4.620000in}}%
\pgfusepath{clip}%
\pgfsetbuttcap%
\pgfsetroundjoin%
\definecolor{currentfill}{rgb}{1.000000,0.894118,0.788235}%
\pgfsetfillcolor{currentfill}%
\pgfsetlinewidth{0.000000pt}%
\definecolor{currentstroke}{rgb}{1.000000,0.894118,0.788235}%
\pgfsetstrokecolor{currentstroke}%
\pgfsetdash{}{0pt}%
\pgfpathmoveto{\pgfqpoint{2.693865in}{3.097461in}}%
\pgfpathlineto{\pgfqpoint{2.582969in}{3.161487in}}%
\pgfpathlineto{\pgfqpoint{2.583171in}{3.110796in}}%
\pgfpathlineto{\pgfqpoint{2.694068in}{3.046769in}}%
\pgfpathlineto{\pgfqpoint{2.693865in}{3.097461in}}%
\pgfpathclose%
\pgfusepath{fill}%
\end{pgfscope}%
\begin{pgfscope}%
\pgfpathrectangle{\pgfqpoint{0.765000in}{0.660000in}}{\pgfqpoint{4.620000in}{4.620000in}}%
\pgfusepath{clip}%
\pgfsetbuttcap%
\pgfsetroundjoin%
\definecolor{currentfill}{rgb}{1.000000,0.894118,0.788235}%
\pgfsetfillcolor{currentfill}%
\pgfsetlinewidth{0.000000pt}%
\definecolor{currentstroke}{rgb}{1.000000,0.894118,0.788235}%
\pgfsetstrokecolor{currentstroke}%
\pgfsetdash{}{0pt}%
\pgfpathmoveto{\pgfqpoint{2.582969in}{3.161487in}}%
\pgfpathlineto{\pgfqpoint{2.582969in}{3.289540in}}%
\pgfpathlineto{\pgfqpoint{2.583171in}{3.238848in}}%
\pgfpathlineto{\pgfqpoint{2.694068in}{3.046769in}}%
\pgfpathlineto{\pgfqpoint{2.582969in}{3.161487in}}%
\pgfpathclose%
\pgfusepath{fill}%
\end{pgfscope}%
\begin{pgfscope}%
\pgfpathrectangle{\pgfqpoint{0.765000in}{0.660000in}}{\pgfqpoint{4.620000in}{4.620000in}}%
\pgfusepath{clip}%
\pgfsetbuttcap%
\pgfsetroundjoin%
\definecolor{currentfill}{rgb}{1.000000,0.894118,0.788235}%
\pgfsetfillcolor{currentfill}%
\pgfsetlinewidth{0.000000pt}%
\definecolor{currentstroke}{rgb}{1.000000,0.894118,0.788235}%
\pgfsetstrokecolor{currentstroke}%
\pgfsetdash{}{0pt}%
\pgfpathmoveto{\pgfqpoint{2.693865in}{3.097461in}}%
\pgfpathlineto{\pgfqpoint{2.693865in}{3.225514in}}%
\pgfpathlineto{\pgfqpoint{2.694068in}{3.174822in}}%
\pgfpathlineto{\pgfqpoint{2.583171in}{3.110796in}}%
\pgfpathlineto{\pgfqpoint{2.693865in}{3.097461in}}%
\pgfpathclose%
\pgfusepath{fill}%
\end{pgfscope}%
\begin{pgfscope}%
\pgfpathrectangle{\pgfqpoint{0.765000in}{0.660000in}}{\pgfqpoint{4.620000in}{4.620000in}}%
\pgfusepath{clip}%
\pgfsetbuttcap%
\pgfsetroundjoin%
\definecolor{currentfill}{rgb}{1.000000,0.894118,0.788235}%
\pgfsetfillcolor{currentfill}%
\pgfsetlinewidth{0.000000pt}%
\definecolor{currentstroke}{rgb}{1.000000,0.894118,0.788235}%
\pgfsetstrokecolor{currentstroke}%
\pgfsetdash{}{0pt}%
\pgfpathmoveto{\pgfqpoint{2.472072in}{3.097461in}}%
\pgfpathlineto{\pgfqpoint{2.582969in}{3.033435in}}%
\pgfpathlineto{\pgfqpoint{2.583171in}{2.982743in}}%
\pgfpathlineto{\pgfqpoint{2.472275in}{3.046769in}}%
\pgfpathlineto{\pgfqpoint{2.472072in}{3.097461in}}%
\pgfpathclose%
\pgfusepath{fill}%
\end{pgfscope}%
\begin{pgfscope}%
\pgfpathrectangle{\pgfqpoint{0.765000in}{0.660000in}}{\pgfqpoint{4.620000in}{4.620000in}}%
\pgfusepath{clip}%
\pgfsetbuttcap%
\pgfsetroundjoin%
\definecolor{currentfill}{rgb}{1.000000,0.894118,0.788235}%
\pgfsetfillcolor{currentfill}%
\pgfsetlinewidth{0.000000pt}%
\definecolor{currentstroke}{rgb}{1.000000,0.894118,0.788235}%
\pgfsetstrokecolor{currentstroke}%
\pgfsetdash{}{0pt}%
\pgfpathmoveto{\pgfqpoint{2.582969in}{3.033435in}}%
\pgfpathlineto{\pgfqpoint{2.693865in}{3.097461in}}%
\pgfpathlineto{\pgfqpoint{2.694068in}{3.046769in}}%
\pgfpathlineto{\pgfqpoint{2.583171in}{2.982743in}}%
\pgfpathlineto{\pgfqpoint{2.582969in}{3.033435in}}%
\pgfpathclose%
\pgfusepath{fill}%
\end{pgfscope}%
\begin{pgfscope}%
\pgfpathrectangle{\pgfqpoint{0.765000in}{0.660000in}}{\pgfqpoint{4.620000in}{4.620000in}}%
\pgfusepath{clip}%
\pgfsetbuttcap%
\pgfsetroundjoin%
\definecolor{currentfill}{rgb}{1.000000,0.894118,0.788235}%
\pgfsetfillcolor{currentfill}%
\pgfsetlinewidth{0.000000pt}%
\definecolor{currentstroke}{rgb}{1.000000,0.894118,0.788235}%
\pgfsetstrokecolor{currentstroke}%
\pgfsetdash{}{0pt}%
\pgfpathmoveto{\pgfqpoint{2.582969in}{3.289540in}}%
\pgfpathlineto{\pgfqpoint{2.472072in}{3.225514in}}%
\pgfpathlineto{\pgfqpoint{2.472275in}{3.174822in}}%
\pgfpathlineto{\pgfqpoint{2.583171in}{3.238848in}}%
\pgfpathlineto{\pgfqpoint{2.582969in}{3.289540in}}%
\pgfpathclose%
\pgfusepath{fill}%
\end{pgfscope}%
\begin{pgfscope}%
\pgfpathrectangle{\pgfqpoint{0.765000in}{0.660000in}}{\pgfqpoint{4.620000in}{4.620000in}}%
\pgfusepath{clip}%
\pgfsetbuttcap%
\pgfsetroundjoin%
\definecolor{currentfill}{rgb}{1.000000,0.894118,0.788235}%
\pgfsetfillcolor{currentfill}%
\pgfsetlinewidth{0.000000pt}%
\definecolor{currentstroke}{rgb}{1.000000,0.894118,0.788235}%
\pgfsetstrokecolor{currentstroke}%
\pgfsetdash{}{0pt}%
\pgfpathmoveto{\pgfqpoint{2.693865in}{3.225514in}}%
\pgfpathlineto{\pgfqpoint{2.582969in}{3.289540in}}%
\pgfpathlineto{\pgfqpoint{2.583171in}{3.238848in}}%
\pgfpathlineto{\pgfqpoint{2.694068in}{3.174822in}}%
\pgfpathlineto{\pgfqpoint{2.693865in}{3.225514in}}%
\pgfpathclose%
\pgfusepath{fill}%
\end{pgfscope}%
\begin{pgfscope}%
\pgfpathrectangle{\pgfqpoint{0.765000in}{0.660000in}}{\pgfqpoint{4.620000in}{4.620000in}}%
\pgfusepath{clip}%
\pgfsetbuttcap%
\pgfsetroundjoin%
\definecolor{currentfill}{rgb}{1.000000,0.894118,0.788235}%
\pgfsetfillcolor{currentfill}%
\pgfsetlinewidth{0.000000pt}%
\definecolor{currentstroke}{rgb}{1.000000,0.894118,0.788235}%
\pgfsetstrokecolor{currentstroke}%
\pgfsetdash{}{0pt}%
\pgfpathmoveto{\pgfqpoint{2.472072in}{3.097461in}}%
\pgfpathlineto{\pgfqpoint{2.472072in}{3.225514in}}%
\pgfpathlineto{\pgfqpoint{2.472275in}{3.174822in}}%
\pgfpathlineto{\pgfqpoint{2.583171in}{2.982743in}}%
\pgfpathlineto{\pgfqpoint{2.472072in}{3.097461in}}%
\pgfpathclose%
\pgfusepath{fill}%
\end{pgfscope}%
\begin{pgfscope}%
\pgfpathrectangle{\pgfqpoint{0.765000in}{0.660000in}}{\pgfqpoint{4.620000in}{4.620000in}}%
\pgfusepath{clip}%
\pgfsetbuttcap%
\pgfsetroundjoin%
\definecolor{currentfill}{rgb}{1.000000,0.894118,0.788235}%
\pgfsetfillcolor{currentfill}%
\pgfsetlinewidth{0.000000pt}%
\definecolor{currentstroke}{rgb}{1.000000,0.894118,0.788235}%
\pgfsetstrokecolor{currentstroke}%
\pgfsetdash{}{0pt}%
\pgfpathmoveto{\pgfqpoint{2.582969in}{3.033435in}}%
\pgfpathlineto{\pgfqpoint{2.582969in}{3.161487in}}%
\pgfpathlineto{\pgfqpoint{2.583171in}{3.110796in}}%
\pgfpathlineto{\pgfqpoint{2.472275in}{3.046769in}}%
\pgfpathlineto{\pgfqpoint{2.582969in}{3.033435in}}%
\pgfpathclose%
\pgfusepath{fill}%
\end{pgfscope}%
\begin{pgfscope}%
\pgfpathrectangle{\pgfqpoint{0.765000in}{0.660000in}}{\pgfqpoint{4.620000in}{4.620000in}}%
\pgfusepath{clip}%
\pgfsetbuttcap%
\pgfsetroundjoin%
\definecolor{currentfill}{rgb}{1.000000,0.894118,0.788235}%
\pgfsetfillcolor{currentfill}%
\pgfsetlinewidth{0.000000pt}%
\definecolor{currentstroke}{rgb}{1.000000,0.894118,0.788235}%
\pgfsetstrokecolor{currentstroke}%
\pgfsetdash{}{0pt}%
\pgfpathmoveto{\pgfqpoint{2.582969in}{3.161487in}}%
\pgfpathlineto{\pgfqpoint{2.693865in}{3.225514in}}%
\pgfpathlineto{\pgfqpoint{2.694068in}{3.174822in}}%
\pgfpathlineto{\pgfqpoint{2.583171in}{3.110796in}}%
\pgfpathlineto{\pgfqpoint{2.582969in}{3.161487in}}%
\pgfpathclose%
\pgfusepath{fill}%
\end{pgfscope}%
\begin{pgfscope}%
\pgfpathrectangle{\pgfqpoint{0.765000in}{0.660000in}}{\pgfqpoint{4.620000in}{4.620000in}}%
\pgfusepath{clip}%
\pgfsetbuttcap%
\pgfsetroundjoin%
\definecolor{currentfill}{rgb}{1.000000,0.894118,0.788235}%
\pgfsetfillcolor{currentfill}%
\pgfsetlinewidth{0.000000pt}%
\definecolor{currentstroke}{rgb}{1.000000,0.894118,0.788235}%
\pgfsetstrokecolor{currentstroke}%
\pgfsetdash{}{0pt}%
\pgfpathmoveto{\pgfqpoint{2.472072in}{3.225514in}}%
\pgfpathlineto{\pgfqpoint{2.582969in}{3.161487in}}%
\pgfpathlineto{\pgfqpoint{2.583171in}{3.110796in}}%
\pgfpathlineto{\pgfqpoint{2.472275in}{3.174822in}}%
\pgfpathlineto{\pgfqpoint{2.472072in}{3.225514in}}%
\pgfpathclose%
\pgfusepath{fill}%
\end{pgfscope}%
\begin{pgfscope}%
\pgfpathrectangle{\pgfqpoint{0.765000in}{0.660000in}}{\pgfqpoint{4.620000in}{4.620000in}}%
\pgfusepath{clip}%
\pgfsetbuttcap%
\pgfsetroundjoin%
\definecolor{currentfill}{rgb}{1.000000,0.894118,0.788235}%
\pgfsetfillcolor{currentfill}%
\pgfsetlinewidth{0.000000pt}%
\definecolor{currentstroke}{rgb}{1.000000,0.894118,0.788235}%
\pgfsetstrokecolor{currentstroke}%
\pgfsetdash{}{0pt}%
\pgfpathmoveto{\pgfqpoint{2.582969in}{3.161487in}}%
\pgfpathlineto{\pgfqpoint{2.472072in}{3.097461in}}%
\pgfpathlineto{\pgfqpoint{2.582969in}{3.033435in}}%
\pgfpathlineto{\pgfqpoint{2.693865in}{3.097461in}}%
\pgfpathlineto{\pgfqpoint{2.582969in}{3.161487in}}%
\pgfpathclose%
\pgfusepath{fill}%
\end{pgfscope}%
\begin{pgfscope}%
\pgfpathrectangle{\pgfqpoint{0.765000in}{0.660000in}}{\pgfqpoint{4.620000in}{4.620000in}}%
\pgfusepath{clip}%
\pgfsetbuttcap%
\pgfsetroundjoin%
\definecolor{currentfill}{rgb}{1.000000,0.894118,0.788235}%
\pgfsetfillcolor{currentfill}%
\pgfsetlinewidth{0.000000pt}%
\definecolor{currentstroke}{rgb}{1.000000,0.894118,0.788235}%
\pgfsetstrokecolor{currentstroke}%
\pgfsetdash{}{0pt}%
\pgfpathmoveto{\pgfqpoint{2.582969in}{3.161487in}}%
\pgfpathlineto{\pgfqpoint{2.472072in}{3.097461in}}%
\pgfpathlineto{\pgfqpoint{2.472072in}{3.225514in}}%
\pgfpathlineto{\pgfqpoint{2.582969in}{3.289540in}}%
\pgfpathlineto{\pgfqpoint{2.582969in}{3.161487in}}%
\pgfpathclose%
\pgfusepath{fill}%
\end{pgfscope}%
\begin{pgfscope}%
\pgfpathrectangle{\pgfqpoint{0.765000in}{0.660000in}}{\pgfqpoint{4.620000in}{4.620000in}}%
\pgfusepath{clip}%
\pgfsetbuttcap%
\pgfsetroundjoin%
\definecolor{currentfill}{rgb}{1.000000,0.894118,0.788235}%
\pgfsetfillcolor{currentfill}%
\pgfsetlinewidth{0.000000pt}%
\definecolor{currentstroke}{rgb}{1.000000,0.894118,0.788235}%
\pgfsetstrokecolor{currentstroke}%
\pgfsetdash{}{0pt}%
\pgfpathmoveto{\pgfqpoint{2.582969in}{3.161487in}}%
\pgfpathlineto{\pgfqpoint{2.693865in}{3.097461in}}%
\pgfpathlineto{\pgfqpoint{2.693865in}{3.225514in}}%
\pgfpathlineto{\pgfqpoint{2.582969in}{3.289540in}}%
\pgfpathlineto{\pgfqpoint{2.582969in}{3.161487in}}%
\pgfpathclose%
\pgfusepath{fill}%
\end{pgfscope}%
\begin{pgfscope}%
\pgfpathrectangle{\pgfqpoint{0.765000in}{0.660000in}}{\pgfqpoint{4.620000in}{4.620000in}}%
\pgfusepath{clip}%
\pgfsetbuttcap%
\pgfsetroundjoin%
\definecolor{currentfill}{rgb}{1.000000,0.894118,0.788235}%
\pgfsetfillcolor{currentfill}%
\pgfsetlinewidth{0.000000pt}%
\definecolor{currentstroke}{rgb}{1.000000,0.894118,0.788235}%
\pgfsetstrokecolor{currentstroke}%
\pgfsetdash{}{0pt}%
\pgfpathmoveto{\pgfqpoint{2.597727in}{3.213477in}}%
\pgfpathlineto{\pgfqpoint{2.486831in}{3.149451in}}%
\pgfpathlineto{\pgfqpoint{2.597727in}{3.085425in}}%
\pgfpathlineto{\pgfqpoint{2.708624in}{3.149451in}}%
\pgfpathlineto{\pgfqpoint{2.597727in}{3.213477in}}%
\pgfpathclose%
\pgfusepath{fill}%
\end{pgfscope}%
\begin{pgfscope}%
\pgfpathrectangle{\pgfqpoint{0.765000in}{0.660000in}}{\pgfqpoint{4.620000in}{4.620000in}}%
\pgfusepath{clip}%
\pgfsetbuttcap%
\pgfsetroundjoin%
\definecolor{currentfill}{rgb}{1.000000,0.894118,0.788235}%
\pgfsetfillcolor{currentfill}%
\pgfsetlinewidth{0.000000pt}%
\definecolor{currentstroke}{rgb}{1.000000,0.894118,0.788235}%
\pgfsetstrokecolor{currentstroke}%
\pgfsetdash{}{0pt}%
\pgfpathmoveto{\pgfqpoint{2.597727in}{3.213477in}}%
\pgfpathlineto{\pgfqpoint{2.486831in}{3.149451in}}%
\pgfpathlineto{\pgfqpoint{2.486831in}{3.277504in}}%
\pgfpathlineto{\pgfqpoint{2.597727in}{3.341530in}}%
\pgfpathlineto{\pgfqpoint{2.597727in}{3.213477in}}%
\pgfpathclose%
\pgfusepath{fill}%
\end{pgfscope}%
\begin{pgfscope}%
\pgfpathrectangle{\pgfqpoint{0.765000in}{0.660000in}}{\pgfqpoint{4.620000in}{4.620000in}}%
\pgfusepath{clip}%
\pgfsetbuttcap%
\pgfsetroundjoin%
\definecolor{currentfill}{rgb}{1.000000,0.894118,0.788235}%
\pgfsetfillcolor{currentfill}%
\pgfsetlinewidth{0.000000pt}%
\definecolor{currentstroke}{rgb}{1.000000,0.894118,0.788235}%
\pgfsetstrokecolor{currentstroke}%
\pgfsetdash{}{0pt}%
\pgfpathmoveto{\pgfqpoint{2.597727in}{3.213477in}}%
\pgfpathlineto{\pgfqpoint{2.708624in}{3.149451in}}%
\pgfpathlineto{\pgfqpoint{2.708624in}{3.277504in}}%
\pgfpathlineto{\pgfqpoint{2.597727in}{3.341530in}}%
\pgfpathlineto{\pgfqpoint{2.597727in}{3.213477in}}%
\pgfpathclose%
\pgfusepath{fill}%
\end{pgfscope}%
\begin{pgfscope}%
\pgfpathrectangle{\pgfqpoint{0.765000in}{0.660000in}}{\pgfqpoint{4.620000in}{4.620000in}}%
\pgfusepath{clip}%
\pgfsetbuttcap%
\pgfsetroundjoin%
\definecolor{currentfill}{rgb}{1.000000,0.894118,0.788235}%
\pgfsetfillcolor{currentfill}%
\pgfsetlinewidth{0.000000pt}%
\definecolor{currentstroke}{rgb}{1.000000,0.894118,0.788235}%
\pgfsetstrokecolor{currentstroke}%
\pgfsetdash{}{0pt}%
\pgfpathmoveto{\pgfqpoint{2.582969in}{3.289540in}}%
\pgfpathlineto{\pgfqpoint{2.472072in}{3.225514in}}%
\pgfpathlineto{\pgfqpoint{2.582969in}{3.161487in}}%
\pgfpathlineto{\pgfqpoint{2.693865in}{3.225514in}}%
\pgfpathlineto{\pgfqpoint{2.582969in}{3.289540in}}%
\pgfpathclose%
\pgfusepath{fill}%
\end{pgfscope}%
\begin{pgfscope}%
\pgfpathrectangle{\pgfqpoint{0.765000in}{0.660000in}}{\pgfqpoint{4.620000in}{4.620000in}}%
\pgfusepath{clip}%
\pgfsetbuttcap%
\pgfsetroundjoin%
\definecolor{currentfill}{rgb}{1.000000,0.894118,0.788235}%
\pgfsetfillcolor{currentfill}%
\pgfsetlinewidth{0.000000pt}%
\definecolor{currentstroke}{rgb}{1.000000,0.894118,0.788235}%
\pgfsetstrokecolor{currentstroke}%
\pgfsetdash{}{0pt}%
\pgfpathmoveto{\pgfqpoint{2.582969in}{3.033435in}}%
\pgfpathlineto{\pgfqpoint{2.693865in}{3.097461in}}%
\pgfpathlineto{\pgfqpoint{2.693865in}{3.225514in}}%
\pgfpathlineto{\pgfqpoint{2.582969in}{3.161487in}}%
\pgfpathlineto{\pgfqpoint{2.582969in}{3.033435in}}%
\pgfpathclose%
\pgfusepath{fill}%
\end{pgfscope}%
\begin{pgfscope}%
\pgfpathrectangle{\pgfqpoint{0.765000in}{0.660000in}}{\pgfqpoint{4.620000in}{4.620000in}}%
\pgfusepath{clip}%
\pgfsetbuttcap%
\pgfsetroundjoin%
\definecolor{currentfill}{rgb}{1.000000,0.894118,0.788235}%
\pgfsetfillcolor{currentfill}%
\pgfsetlinewidth{0.000000pt}%
\definecolor{currentstroke}{rgb}{1.000000,0.894118,0.788235}%
\pgfsetstrokecolor{currentstroke}%
\pgfsetdash{}{0pt}%
\pgfpathmoveto{\pgfqpoint{2.472072in}{3.097461in}}%
\pgfpathlineto{\pgfqpoint{2.582969in}{3.033435in}}%
\pgfpathlineto{\pgfqpoint{2.582969in}{3.161487in}}%
\pgfpathlineto{\pgfqpoint{2.472072in}{3.225514in}}%
\pgfpathlineto{\pgfqpoint{2.472072in}{3.097461in}}%
\pgfpathclose%
\pgfusepath{fill}%
\end{pgfscope}%
\begin{pgfscope}%
\pgfpathrectangle{\pgfqpoint{0.765000in}{0.660000in}}{\pgfqpoint{4.620000in}{4.620000in}}%
\pgfusepath{clip}%
\pgfsetbuttcap%
\pgfsetroundjoin%
\definecolor{currentfill}{rgb}{1.000000,0.894118,0.788235}%
\pgfsetfillcolor{currentfill}%
\pgfsetlinewidth{0.000000pt}%
\definecolor{currentstroke}{rgb}{1.000000,0.894118,0.788235}%
\pgfsetstrokecolor{currentstroke}%
\pgfsetdash{}{0pt}%
\pgfpathmoveto{\pgfqpoint{2.597727in}{3.341530in}}%
\pgfpathlineto{\pgfqpoint{2.486831in}{3.277504in}}%
\pgfpathlineto{\pgfqpoint{2.597727in}{3.213477in}}%
\pgfpathlineto{\pgfqpoint{2.708624in}{3.277504in}}%
\pgfpathlineto{\pgfqpoint{2.597727in}{3.341530in}}%
\pgfpathclose%
\pgfusepath{fill}%
\end{pgfscope}%
\begin{pgfscope}%
\pgfpathrectangle{\pgfqpoint{0.765000in}{0.660000in}}{\pgfqpoint{4.620000in}{4.620000in}}%
\pgfusepath{clip}%
\pgfsetbuttcap%
\pgfsetroundjoin%
\definecolor{currentfill}{rgb}{1.000000,0.894118,0.788235}%
\pgfsetfillcolor{currentfill}%
\pgfsetlinewidth{0.000000pt}%
\definecolor{currentstroke}{rgb}{1.000000,0.894118,0.788235}%
\pgfsetstrokecolor{currentstroke}%
\pgfsetdash{}{0pt}%
\pgfpathmoveto{\pgfqpoint{2.597727in}{3.085425in}}%
\pgfpathlineto{\pgfqpoint{2.708624in}{3.149451in}}%
\pgfpathlineto{\pgfqpoint{2.708624in}{3.277504in}}%
\pgfpathlineto{\pgfqpoint{2.597727in}{3.213477in}}%
\pgfpathlineto{\pgfqpoint{2.597727in}{3.085425in}}%
\pgfpathclose%
\pgfusepath{fill}%
\end{pgfscope}%
\begin{pgfscope}%
\pgfpathrectangle{\pgfqpoint{0.765000in}{0.660000in}}{\pgfqpoint{4.620000in}{4.620000in}}%
\pgfusepath{clip}%
\pgfsetbuttcap%
\pgfsetroundjoin%
\definecolor{currentfill}{rgb}{1.000000,0.894118,0.788235}%
\pgfsetfillcolor{currentfill}%
\pgfsetlinewidth{0.000000pt}%
\definecolor{currentstroke}{rgb}{1.000000,0.894118,0.788235}%
\pgfsetstrokecolor{currentstroke}%
\pgfsetdash{}{0pt}%
\pgfpathmoveto{\pgfqpoint{2.486831in}{3.149451in}}%
\pgfpathlineto{\pgfqpoint{2.597727in}{3.085425in}}%
\pgfpathlineto{\pgfqpoint{2.597727in}{3.213477in}}%
\pgfpathlineto{\pgfqpoint{2.486831in}{3.277504in}}%
\pgfpathlineto{\pgfqpoint{2.486831in}{3.149451in}}%
\pgfpathclose%
\pgfusepath{fill}%
\end{pgfscope}%
\begin{pgfscope}%
\pgfpathrectangle{\pgfqpoint{0.765000in}{0.660000in}}{\pgfqpoint{4.620000in}{4.620000in}}%
\pgfusepath{clip}%
\pgfsetbuttcap%
\pgfsetroundjoin%
\definecolor{currentfill}{rgb}{1.000000,0.894118,0.788235}%
\pgfsetfillcolor{currentfill}%
\pgfsetlinewidth{0.000000pt}%
\definecolor{currentstroke}{rgb}{1.000000,0.894118,0.788235}%
\pgfsetstrokecolor{currentstroke}%
\pgfsetdash{}{0pt}%
\pgfpathmoveto{\pgfqpoint{2.582969in}{3.161487in}}%
\pgfpathlineto{\pgfqpoint{2.472072in}{3.097461in}}%
\pgfpathlineto{\pgfqpoint{2.486831in}{3.149451in}}%
\pgfpathlineto{\pgfqpoint{2.597727in}{3.213477in}}%
\pgfpathlineto{\pgfqpoint{2.582969in}{3.161487in}}%
\pgfpathclose%
\pgfusepath{fill}%
\end{pgfscope}%
\begin{pgfscope}%
\pgfpathrectangle{\pgfqpoint{0.765000in}{0.660000in}}{\pgfqpoint{4.620000in}{4.620000in}}%
\pgfusepath{clip}%
\pgfsetbuttcap%
\pgfsetroundjoin%
\definecolor{currentfill}{rgb}{1.000000,0.894118,0.788235}%
\pgfsetfillcolor{currentfill}%
\pgfsetlinewidth{0.000000pt}%
\definecolor{currentstroke}{rgb}{1.000000,0.894118,0.788235}%
\pgfsetstrokecolor{currentstroke}%
\pgfsetdash{}{0pt}%
\pgfpathmoveto{\pgfqpoint{2.693865in}{3.097461in}}%
\pgfpathlineto{\pgfqpoint{2.582969in}{3.161487in}}%
\pgfpathlineto{\pgfqpoint{2.597727in}{3.213477in}}%
\pgfpathlineto{\pgfqpoint{2.708624in}{3.149451in}}%
\pgfpathlineto{\pgfqpoint{2.693865in}{3.097461in}}%
\pgfpathclose%
\pgfusepath{fill}%
\end{pgfscope}%
\begin{pgfscope}%
\pgfpathrectangle{\pgfqpoint{0.765000in}{0.660000in}}{\pgfqpoint{4.620000in}{4.620000in}}%
\pgfusepath{clip}%
\pgfsetbuttcap%
\pgfsetroundjoin%
\definecolor{currentfill}{rgb}{1.000000,0.894118,0.788235}%
\pgfsetfillcolor{currentfill}%
\pgfsetlinewidth{0.000000pt}%
\definecolor{currentstroke}{rgb}{1.000000,0.894118,0.788235}%
\pgfsetstrokecolor{currentstroke}%
\pgfsetdash{}{0pt}%
\pgfpathmoveto{\pgfqpoint{2.582969in}{3.161487in}}%
\pgfpathlineto{\pgfqpoint{2.582969in}{3.289540in}}%
\pgfpathlineto{\pgfqpoint{2.597727in}{3.341530in}}%
\pgfpathlineto{\pgfqpoint{2.708624in}{3.149451in}}%
\pgfpathlineto{\pgfqpoint{2.582969in}{3.161487in}}%
\pgfpathclose%
\pgfusepath{fill}%
\end{pgfscope}%
\begin{pgfscope}%
\pgfpathrectangle{\pgfqpoint{0.765000in}{0.660000in}}{\pgfqpoint{4.620000in}{4.620000in}}%
\pgfusepath{clip}%
\pgfsetbuttcap%
\pgfsetroundjoin%
\definecolor{currentfill}{rgb}{1.000000,0.894118,0.788235}%
\pgfsetfillcolor{currentfill}%
\pgfsetlinewidth{0.000000pt}%
\definecolor{currentstroke}{rgb}{1.000000,0.894118,0.788235}%
\pgfsetstrokecolor{currentstroke}%
\pgfsetdash{}{0pt}%
\pgfpathmoveto{\pgfqpoint{2.693865in}{3.097461in}}%
\pgfpathlineto{\pgfqpoint{2.693865in}{3.225514in}}%
\pgfpathlineto{\pgfqpoint{2.708624in}{3.277504in}}%
\pgfpathlineto{\pgfqpoint{2.597727in}{3.213477in}}%
\pgfpathlineto{\pgfqpoint{2.693865in}{3.097461in}}%
\pgfpathclose%
\pgfusepath{fill}%
\end{pgfscope}%
\begin{pgfscope}%
\pgfpathrectangle{\pgfqpoint{0.765000in}{0.660000in}}{\pgfqpoint{4.620000in}{4.620000in}}%
\pgfusepath{clip}%
\pgfsetbuttcap%
\pgfsetroundjoin%
\definecolor{currentfill}{rgb}{1.000000,0.894118,0.788235}%
\pgfsetfillcolor{currentfill}%
\pgfsetlinewidth{0.000000pt}%
\definecolor{currentstroke}{rgb}{1.000000,0.894118,0.788235}%
\pgfsetstrokecolor{currentstroke}%
\pgfsetdash{}{0pt}%
\pgfpathmoveto{\pgfqpoint{2.472072in}{3.097461in}}%
\pgfpathlineto{\pgfqpoint{2.582969in}{3.033435in}}%
\pgfpathlineto{\pgfqpoint{2.597727in}{3.085425in}}%
\pgfpathlineto{\pgfqpoint{2.486831in}{3.149451in}}%
\pgfpathlineto{\pgfqpoint{2.472072in}{3.097461in}}%
\pgfpathclose%
\pgfusepath{fill}%
\end{pgfscope}%
\begin{pgfscope}%
\pgfpathrectangle{\pgfqpoint{0.765000in}{0.660000in}}{\pgfqpoint{4.620000in}{4.620000in}}%
\pgfusepath{clip}%
\pgfsetbuttcap%
\pgfsetroundjoin%
\definecolor{currentfill}{rgb}{1.000000,0.894118,0.788235}%
\pgfsetfillcolor{currentfill}%
\pgfsetlinewidth{0.000000pt}%
\definecolor{currentstroke}{rgb}{1.000000,0.894118,0.788235}%
\pgfsetstrokecolor{currentstroke}%
\pgfsetdash{}{0pt}%
\pgfpathmoveto{\pgfqpoint{2.582969in}{3.033435in}}%
\pgfpathlineto{\pgfqpoint{2.693865in}{3.097461in}}%
\pgfpathlineto{\pgfqpoint{2.708624in}{3.149451in}}%
\pgfpathlineto{\pgfqpoint{2.597727in}{3.085425in}}%
\pgfpathlineto{\pgfqpoint{2.582969in}{3.033435in}}%
\pgfpathclose%
\pgfusepath{fill}%
\end{pgfscope}%
\begin{pgfscope}%
\pgfpathrectangle{\pgfqpoint{0.765000in}{0.660000in}}{\pgfqpoint{4.620000in}{4.620000in}}%
\pgfusepath{clip}%
\pgfsetbuttcap%
\pgfsetroundjoin%
\definecolor{currentfill}{rgb}{1.000000,0.894118,0.788235}%
\pgfsetfillcolor{currentfill}%
\pgfsetlinewidth{0.000000pt}%
\definecolor{currentstroke}{rgb}{1.000000,0.894118,0.788235}%
\pgfsetstrokecolor{currentstroke}%
\pgfsetdash{}{0pt}%
\pgfpathmoveto{\pgfqpoint{2.582969in}{3.289540in}}%
\pgfpathlineto{\pgfqpoint{2.472072in}{3.225514in}}%
\pgfpathlineto{\pgfqpoint{2.486831in}{3.277504in}}%
\pgfpathlineto{\pgfqpoint{2.597727in}{3.341530in}}%
\pgfpathlineto{\pgfqpoint{2.582969in}{3.289540in}}%
\pgfpathclose%
\pgfusepath{fill}%
\end{pgfscope}%
\begin{pgfscope}%
\pgfpathrectangle{\pgfqpoint{0.765000in}{0.660000in}}{\pgfqpoint{4.620000in}{4.620000in}}%
\pgfusepath{clip}%
\pgfsetbuttcap%
\pgfsetroundjoin%
\definecolor{currentfill}{rgb}{1.000000,0.894118,0.788235}%
\pgfsetfillcolor{currentfill}%
\pgfsetlinewidth{0.000000pt}%
\definecolor{currentstroke}{rgb}{1.000000,0.894118,0.788235}%
\pgfsetstrokecolor{currentstroke}%
\pgfsetdash{}{0pt}%
\pgfpathmoveto{\pgfqpoint{2.693865in}{3.225514in}}%
\pgfpathlineto{\pgfqpoint{2.582969in}{3.289540in}}%
\pgfpathlineto{\pgfqpoint{2.597727in}{3.341530in}}%
\pgfpathlineto{\pgfqpoint{2.708624in}{3.277504in}}%
\pgfpathlineto{\pgfqpoint{2.693865in}{3.225514in}}%
\pgfpathclose%
\pgfusepath{fill}%
\end{pgfscope}%
\begin{pgfscope}%
\pgfpathrectangle{\pgfqpoint{0.765000in}{0.660000in}}{\pgfqpoint{4.620000in}{4.620000in}}%
\pgfusepath{clip}%
\pgfsetbuttcap%
\pgfsetroundjoin%
\definecolor{currentfill}{rgb}{1.000000,0.894118,0.788235}%
\pgfsetfillcolor{currentfill}%
\pgfsetlinewidth{0.000000pt}%
\definecolor{currentstroke}{rgb}{1.000000,0.894118,0.788235}%
\pgfsetstrokecolor{currentstroke}%
\pgfsetdash{}{0pt}%
\pgfpathmoveto{\pgfqpoint{2.472072in}{3.097461in}}%
\pgfpathlineto{\pgfqpoint{2.472072in}{3.225514in}}%
\pgfpathlineto{\pgfqpoint{2.486831in}{3.277504in}}%
\pgfpathlineto{\pgfqpoint{2.597727in}{3.085425in}}%
\pgfpathlineto{\pgfqpoint{2.472072in}{3.097461in}}%
\pgfpathclose%
\pgfusepath{fill}%
\end{pgfscope}%
\begin{pgfscope}%
\pgfpathrectangle{\pgfqpoint{0.765000in}{0.660000in}}{\pgfqpoint{4.620000in}{4.620000in}}%
\pgfusepath{clip}%
\pgfsetbuttcap%
\pgfsetroundjoin%
\definecolor{currentfill}{rgb}{1.000000,0.894118,0.788235}%
\pgfsetfillcolor{currentfill}%
\pgfsetlinewidth{0.000000pt}%
\definecolor{currentstroke}{rgb}{1.000000,0.894118,0.788235}%
\pgfsetstrokecolor{currentstroke}%
\pgfsetdash{}{0pt}%
\pgfpathmoveto{\pgfqpoint{2.582969in}{3.033435in}}%
\pgfpathlineto{\pgfqpoint{2.582969in}{3.161487in}}%
\pgfpathlineto{\pgfqpoint{2.597727in}{3.213477in}}%
\pgfpathlineto{\pgfqpoint{2.486831in}{3.149451in}}%
\pgfpathlineto{\pgfqpoint{2.582969in}{3.033435in}}%
\pgfpathclose%
\pgfusepath{fill}%
\end{pgfscope}%
\begin{pgfscope}%
\pgfpathrectangle{\pgfqpoint{0.765000in}{0.660000in}}{\pgfqpoint{4.620000in}{4.620000in}}%
\pgfusepath{clip}%
\pgfsetbuttcap%
\pgfsetroundjoin%
\definecolor{currentfill}{rgb}{1.000000,0.894118,0.788235}%
\pgfsetfillcolor{currentfill}%
\pgfsetlinewidth{0.000000pt}%
\definecolor{currentstroke}{rgb}{1.000000,0.894118,0.788235}%
\pgfsetstrokecolor{currentstroke}%
\pgfsetdash{}{0pt}%
\pgfpathmoveto{\pgfqpoint{2.582969in}{3.161487in}}%
\pgfpathlineto{\pgfqpoint{2.693865in}{3.225514in}}%
\pgfpathlineto{\pgfqpoint{2.708624in}{3.277504in}}%
\pgfpathlineto{\pgfqpoint{2.597727in}{3.213477in}}%
\pgfpathlineto{\pgfqpoint{2.582969in}{3.161487in}}%
\pgfpathclose%
\pgfusepath{fill}%
\end{pgfscope}%
\begin{pgfscope}%
\pgfpathrectangle{\pgfqpoint{0.765000in}{0.660000in}}{\pgfqpoint{4.620000in}{4.620000in}}%
\pgfusepath{clip}%
\pgfsetbuttcap%
\pgfsetroundjoin%
\definecolor{currentfill}{rgb}{1.000000,0.894118,0.788235}%
\pgfsetfillcolor{currentfill}%
\pgfsetlinewidth{0.000000pt}%
\definecolor{currentstroke}{rgb}{1.000000,0.894118,0.788235}%
\pgfsetstrokecolor{currentstroke}%
\pgfsetdash{}{0pt}%
\pgfpathmoveto{\pgfqpoint{2.472072in}{3.225514in}}%
\pgfpathlineto{\pgfqpoint{2.582969in}{3.161487in}}%
\pgfpathlineto{\pgfqpoint{2.597727in}{3.213477in}}%
\pgfpathlineto{\pgfqpoint{2.486831in}{3.277504in}}%
\pgfpathlineto{\pgfqpoint{2.472072in}{3.225514in}}%
\pgfpathclose%
\pgfusepath{fill}%
\end{pgfscope}%
\begin{pgfscope}%
\pgfpathrectangle{\pgfqpoint{0.765000in}{0.660000in}}{\pgfqpoint{4.620000in}{4.620000in}}%
\pgfusepath{clip}%
\pgfsetbuttcap%
\pgfsetroundjoin%
\definecolor{currentfill}{rgb}{1.000000,0.894118,0.788235}%
\pgfsetfillcolor{currentfill}%
\pgfsetlinewidth{0.000000pt}%
\definecolor{currentstroke}{rgb}{1.000000,0.894118,0.788235}%
\pgfsetstrokecolor{currentstroke}%
\pgfsetdash{}{0pt}%
\pgfpathmoveto{\pgfqpoint{2.597727in}{3.213477in}}%
\pgfpathlineto{\pgfqpoint{2.486831in}{3.149451in}}%
\pgfpathlineto{\pgfqpoint{2.597727in}{3.085425in}}%
\pgfpathlineto{\pgfqpoint{2.708624in}{3.149451in}}%
\pgfpathlineto{\pgfqpoint{2.597727in}{3.213477in}}%
\pgfpathclose%
\pgfusepath{fill}%
\end{pgfscope}%
\begin{pgfscope}%
\pgfpathrectangle{\pgfqpoint{0.765000in}{0.660000in}}{\pgfqpoint{4.620000in}{4.620000in}}%
\pgfusepath{clip}%
\pgfsetbuttcap%
\pgfsetroundjoin%
\definecolor{currentfill}{rgb}{1.000000,0.894118,0.788235}%
\pgfsetfillcolor{currentfill}%
\pgfsetlinewidth{0.000000pt}%
\definecolor{currentstroke}{rgb}{1.000000,0.894118,0.788235}%
\pgfsetstrokecolor{currentstroke}%
\pgfsetdash{}{0pt}%
\pgfpathmoveto{\pgfqpoint{2.597727in}{3.213477in}}%
\pgfpathlineto{\pgfqpoint{2.486831in}{3.149451in}}%
\pgfpathlineto{\pgfqpoint{2.486831in}{3.277504in}}%
\pgfpathlineto{\pgfqpoint{2.597727in}{3.341530in}}%
\pgfpathlineto{\pgfqpoint{2.597727in}{3.213477in}}%
\pgfpathclose%
\pgfusepath{fill}%
\end{pgfscope}%
\begin{pgfscope}%
\pgfpathrectangle{\pgfqpoint{0.765000in}{0.660000in}}{\pgfqpoint{4.620000in}{4.620000in}}%
\pgfusepath{clip}%
\pgfsetbuttcap%
\pgfsetroundjoin%
\definecolor{currentfill}{rgb}{1.000000,0.894118,0.788235}%
\pgfsetfillcolor{currentfill}%
\pgfsetlinewidth{0.000000pt}%
\definecolor{currentstroke}{rgb}{1.000000,0.894118,0.788235}%
\pgfsetstrokecolor{currentstroke}%
\pgfsetdash{}{0pt}%
\pgfpathmoveto{\pgfqpoint{2.597727in}{3.213477in}}%
\pgfpathlineto{\pgfqpoint{2.708624in}{3.149451in}}%
\pgfpathlineto{\pgfqpoint{2.708624in}{3.277504in}}%
\pgfpathlineto{\pgfqpoint{2.597727in}{3.341530in}}%
\pgfpathlineto{\pgfqpoint{2.597727in}{3.213477in}}%
\pgfpathclose%
\pgfusepath{fill}%
\end{pgfscope}%
\begin{pgfscope}%
\pgfpathrectangle{\pgfqpoint{0.765000in}{0.660000in}}{\pgfqpoint{4.620000in}{4.620000in}}%
\pgfusepath{clip}%
\pgfsetbuttcap%
\pgfsetroundjoin%
\definecolor{currentfill}{rgb}{1.000000,0.894118,0.788235}%
\pgfsetfillcolor{currentfill}%
\pgfsetlinewidth{0.000000pt}%
\definecolor{currentstroke}{rgb}{1.000000,0.894118,0.788235}%
\pgfsetstrokecolor{currentstroke}%
\pgfsetdash{}{0pt}%
\pgfpathmoveto{\pgfqpoint{2.613910in}{3.253240in}}%
\pgfpathlineto{\pgfqpoint{2.503014in}{3.189213in}}%
\pgfpathlineto{\pgfqpoint{2.613910in}{3.125187in}}%
\pgfpathlineto{\pgfqpoint{2.724807in}{3.189213in}}%
\pgfpathlineto{\pgfqpoint{2.613910in}{3.253240in}}%
\pgfpathclose%
\pgfusepath{fill}%
\end{pgfscope}%
\begin{pgfscope}%
\pgfpathrectangle{\pgfqpoint{0.765000in}{0.660000in}}{\pgfqpoint{4.620000in}{4.620000in}}%
\pgfusepath{clip}%
\pgfsetbuttcap%
\pgfsetroundjoin%
\definecolor{currentfill}{rgb}{1.000000,0.894118,0.788235}%
\pgfsetfillcolor{currentfill}%
\pgfsetlinewidth{0.000000pt}%
\definecolor{currentstroke}{rgb}{1.000000,0.894118,0.788235}%
\pgfsetstrokecolor{currentstroke}%
\pgfsetdash{}{0pt}%
\pgfpathmoveto{\pgfqpoint{2.613910in}{3.253240in}}%
\pgfpathlineto{\pgfqpoint{2.503014in}{3.189213in}}%
\pgfpathlineto{\pgfqpoint{2.503014in}{3.317266in}}%
\pgfpathlineto{\pgfqpoint{2.613910in}{3.381292in}}%
\pgfpathlineto{\pgfqpoint{2.613910in}{3.253240in}}%
\pgfpathclose%
\pgfusepath{fill}%
\end{pgfscope}%
\begin{pgfscope}%
\pgfpathrectangle{\pgfqpoint{0.765000in}{0.660000in}}{\pgfqpoint{4.620000in}{4.620000in}}%
\pgfusepath{clip}%
\pgfsetbuttcap%
\pgfsetroundjoin%
\definecolor{currentfill}{rgb}{1.000000,0.894118,0.788235}%
\pgfsetfillcolor{currentfill}%
\pgfsetlinewidth{0.000000pt}%
\definecolor{currentstroke}{rgb}{1.000000,0.894118,0.788235}%
\pgfsetstrokecolor{currentstroke}%
\pgfsetdash{}{0pt}%
\pgfpathmoveto{\pgfqpoint{2.613910in}{3.253240in}}%
\pgfpathlineto{\pgfqpoint{2.724807in}{3.189213in}}%
\pgfpathlineto{\pgfqpoint{2.724807in}{3.317266in}}%
\pgfpathlineto{\pgfqpoint{2.613910in}{3.381292in}}%
\pgfpathlineto{\pgfqpoint{2.613910in}{3.253240in}}%
\pgfpathclose%
\pgfusepath{fill}%
\end{pgfscope}%
\begin{pgfscope}%
\pgfpathrectangle{\pgfqpoint{0.765000in}{0.660000in}}{\pgfqpoint{4.620000in}{4.620000in}}%
\pgfusepath{clip}%
\pgfsetbuttcap%
\pgfsetroundjoin%
\definecolor{currentfill}{rgb}{1.000000,0.894118,0.788235}%
\pgfsetfillcolor{currentfill}%
\pgfsetlinewidth{0.000000pt}%
\definecolor{currentstroke}{rgb}{1.000000,0.894118,0.788235}%
\pgfsetstrokecolor{currentstroke}%
\pgfsetdash{}{0pt}%
\pgfpathmoveto{\pgfqpoint{2.597727in}{3.341530in}}%
\pgfpathlineto{\pgfqpoint{2.486831in}{3.277504in}}%
\pgfpathlineto{\pgfqpoint{2.597727in}{3.213477in}}%
\pgfpathlineto{\pgfqpoint{2.708624in}{3.277504in}}%
\pgfpathlineto{\pgfqpoint{2.597727in}{3.341530in}}%
\pgfpathclose%
\pgfusepath{fill}%
\end{pgfscope}%
\begin{pgfscope}%
\pgfpathrectangle{\pgfqpoint{0.765000in}{0.660000in}}{\pgfqpoint{4.620000in}{4.620000in}}%
\pgfusepath{clip}%
\pgfsetbuttcap%
\pgfsetroundjoin%
\definecolor{currentfill}{rgb}{1.000000,0.894118,0.788235}%
\pgfsetfillcolor{currentfill}%
\pgfsetlinewidth{0.000000pt}%
\definecolor{currentstroke}{rgb}{1.000000,0.894118,0.788235}%
\pgfsetstrokecolor{currentstroke}%
\pgfsetdash{}{0pt}%
\pgfpathmoveto{\pgfqpoint{2.597727in}{3.085425in}}%
\pgfpathlineto{\pgfqpoint{2.708624in}{3.149451in}}%
\pgfpathlineto{\pgfqpoint{2.708624in}{3.277504in}}%
\pgfpathlineto{\pgfqpoint{2.597727in}{3.213477in}}%
\pgfpathlineto{\pgfqpoint{2.597727in}{3.085425in}}%
\pgfpathclose%
\pgfusepath{fill}%
\end{pgfscope}%
\begin{pgfscope}%
\pgfpathrectangle{\pgfqpoint{0.765000in}{0.660000in}}{\pgfqpoint{4.620000in}{4.620000in}}%
\pgfusepath{clip}%
\pgfsetbuttcap%
\pgfsetroundjoin%
\definecolor{currentfill}{rgb}{1.000000,0.894118,0.788235}%
\pgfsetfillcolor{currentfill}%
\pgfsetlinewidth{0.000000pt}%
\definecolor{currentstroke}{rgb}{1.000000,0.894118,0.788235}%
\pgfsetstrokecolor{currentstroke}%
\pgfsetdash{}{0pt}%
\pgfpathmoveto{\pgfqpoint{2.486831in}{3.149451in}}%
\pgfpathlineto{\pgfqpoint{2.597727in}{3.085425in}}%
\pgfpathlineto{\pgfqpoint{2.597727in}{3.213477in}}%
\pgfpathlineto{\pgfqpoint{2.486831in}{3.277504in}}%
\pgfpathlineto{\pgfqpoint{2.486831in}{3.149451in}}%
\pgfpathclose%
\pgfusepath{fill}%
\end{pgfscope}%
\begin{pgfscope}%
\pgfpathrectangle{\pgfqpoint{0.765000in}{0.660000in}}{\pgfqpoint{4.620000in}{4.620000in}}%
\pgfusepath{clip}%
\pgfsetbuttcap%
\pgfsetroundjoin%
\definecolor{currentfill}{rgb}{1.000000,0.894118,0.788235}%
\pgfsetfillcolor{currentfill}%
\pgfsetlinewidth{0.000000pt}%
\definecolor{currentstroke}{rgb}{1.000000,0.894118,0.788235}%
\pgfsetstrokecolor{currentstroke}%
\pgfsetdash{}{0pt}%
\pgfpathmoveto{\pgfqpoint{2.613910in}{3.381292in}}%
\pgfpathlineto{\pgfqpoint{2.503014in}{3.317266in}}%
\pgfpathlineto{\pgfqpoint{2.613910in}{3.253240in}}%
\pgfpathlineto{\pgfqpoint{2.724807in}{3.317266in}}%
\pgfpathlineto{\pgfqpoint{2.613910in}{3.381292in}}%
\pgfpathclose%
\pgfusepath{fill}%
\end{pgfscope}%
\begin{pgfscope}%
\pgfpathrectangle{\pgfqpoint{0.765000in}{0.660000in}}{\pgfqpoint{4.620000in}{4.620000in}}%
\pgfusepath{clip}%
\pgfsetbuttcap%
\pgfsetroundjoin%
\definecolor{currentfill}{rgb}{1.000000,0.894118,0.788235}%
\pgfsetfillcolor{currentfill}%
\pgfsetlinewidth{0.000000pt}%
\definecolor{currentstroke}{rgb}{1.000000,0.894118,0.788235}%
\pgfsetstrokecolor{currentstroke}%
\pgfsetdash{}{0pt}%
\pgfpathmoveto{\pgfqpoint{2.613910in}{3.125187in}}%
\pgfpathlineto{\pgfqpoint{2.724807in}{3.189213in}}%
\pgfpathlineto{\pgfqpoint{2.724807in}{3.317266in}}%
\pgfpathlineto{\pgfqpoint{2.613910in}{3.253240in}}%
\pgfpathlineto{\pgfqpoint{2.613910in}{3.125187in}}%
\pgfpathclose%
\pgfusepath{fill}%
\end{pgfscope}%
\begin{pgfscope}%
\pgfpathrectangle{\pgfqpoint{0.765000in}{0.660000in}}{\pgfqpoint{4.620000in}{4.620000in}}%
\pgfusepath{clip}%
\pgfsetbuttcap%
\pgfsetroundjoin%
\definecolor{currentfill}{rgb}{1.000000,0.894118,0.788235}%
\pgfsetfillcolor{currentfill}%
\pgfsetlinewidth{0.000000pt}%
\definecolor{currentstroke}{rgb}{1.000000,0.894118,0.788235}%
\pgfsetstrokecolor{currentstroke}%
\pgfsetdash{}{0pt}%
\pgfpathmoveto{\pgfqpoint{2.503014in}{3.189213in}}%
\pgfpathlineto{\pgfqpoint{2.613910in}{3.125187in}}%
\pgfpathlineto{\pgfqpoint{2.613910in}{3.253240in}}%
\pgfpathlineto{\pgfqpoint{2.503014in}{3.317266in}}%
\pgfpathlineto{\pgfqpoint{2.503014in}{3.189213in}}%
\pgfpathclose%
\pgfusepath{fill}%
\end{pgfscope}%
\begin{pgfscope}%
\pgfpathrectangle{\pgfqpoint{0.765000in}{0.660000in}}{\pgfqpoint{4.620000in}{4.620000in}}%
\pgfusepath{clip}%
\pgfsetbuttcap%
\pgfsetroundjoin%
\definecolor{currentfill}{rgb}{1.000000,0.894118,0.788235}%
\pgfsetfillcolor{currentfill}%
\pgfsetlinewidth{0.000000pt}%
\definecolor{currentstroke}{rgb}{1.000000,0.894118,0.788235}%
\pgfsetstrokecolor{currentstroke}%
\pgfsetdash{}{0pt}%
\pgfpathmoveto{\pgfqpoint{2.597727in}{3.213477in}}%
\pgfpathlineto{\pgfqpoint{2.486831in}{3.149451in}}%
\pgfpathlineto{\pgfqpoint{2.503014in}{3.189213in}}%
\pgfpathlineto{\pgfqpoint{2.613910in}{3.253240in}}%
\pgfpathlineto{\pgfqpoint{2.597727in}{3.213477in}}%
\pgfpathclose%
\pgfusepath{fill}%
\end{pgfscope}%
\begin{pgfscope}%
\pgfpathrectangle{\pgfqpoint{0.765000in}{0.660000in}}{\pgfqpoint{4.620000in}{4.620000in}}%
\pgfusepath{clip}%
\pgfsetbuttcap%
\pgfsetroundjoin%
\definecolor{currentfill}{rgb}{1.000000,0.894118,0.788235}%
\pgfsetfillcolor{currentfill}%
\pgfsetlinewidth{0.000000pt}%
\definecolor{currentstroke}{rgb}{1.000000,0.894118,0.788235}%
\pgfsetstrokecolor{currentstroke}%
\pgfsetdash{}{0pt}%
\pgfpathmoveto{\pgfqpoint{2.708624in}{3.149451in}}%
\pgfpathlineto{\pgfqpoint{2.597727in}{3.213477in}}%
\pgfpathlineto{\pgfqpoint{2.613910in}{3.253240in}}%
\pgfpathlineto{\pgfqpoint{2.724807in}{3.189213in}}%
\pgfpathlineto{\pgfqpoint{2.708624in}{3.149451in}}%
\pgfpathclose%
\pgfusepath{fill}%
\end{pgfscope}%
\begin{pgfscope}%
\pgfpathrectangle{\pgfqpoint{0.765000in}{0.660000in}}{\pgfqpoint{4.620000in}{4.620000in}}%
\pgfusepath{clip}%
\pgfsetbuttcap%
\pgfsetroundjoin%
\definecolor{currentfill}{rgb}{1.000000,0.894118,0.788235}%
\pgfsetfillcolor{currentfill}%
\pgfsetlinewidth{0.000000pt}%
\definecolor{currentstroke}{rgb}{1.000000,0.894118,0.788235}%
\pgfsetstrokecolor{currentstroke}%
\pgfsetdash{}{0pt}%
\pgfpathmoveto{\pgfqpoint{2.597727in}{3.213477in}}%
\pgfpathlineto{\pgfqpoint{2.597727in}{3.341530in}}%
\pgfpathlineto{\pgfqpoint{2.613910in}{3.381292in}}%
\pgfpathlineto{\pgfqpoint{2.724807in}{3.189213in}}%
\pgfpathlineto{\pgfqpoint{2.597727in}{3.213477in}}%
\pgfpathclose%
\pgfusepath{fill}%
\end{pgfscope}%
\begin{pgfscope}%
\pgfpathrectangle{\pgfqpoint{0.765000in}{0.660000in}}{\pgfqpoint{4.620000in}{4.620000in}}%
\pgfusepath{clip}%
\pgfsetbuttcap%
\pgfsetroundjoin%
\definecolor{currentfill}{rgb}{1.000000,0.894118,0.788235}%
\pgfsetfillcolor{currentfill}%
\pgfsetlinewidth{0.000000pt}%
\definecolor{currentstroke}{rgb}{1.000000,0.894118,0.788235}%
\pgfsetstrokecolor{currentstroke}%
\pgfsetdash{}{0pt}%
\pgfpathmoveto{\pgfqpoint{2.708624in}{3.149451in}}%
\pgfpathlineto{\pgfqpoint{2.708624in}{3.277504in}}%
\pgfpathlineto{\pgfqpoint{2.724807in}{3.317266in}}%
\pgfpathlineto{\pgfqpoint{2.613910in}{3.253240in}}%
\pgfpathlineto{\pgfqpoint{2.708624in}{3.149451in}}%
\pgfpathclose%
\pgfusepath{fill}%
\end{pgfscope}%
\begin{pgfscope}%
\pgfpathrectangle{\pgfqpoint{0.765000in}{0.660000in}}{\pgfqpoint{4.620000in}{4.620000in}}%
\pgfusepath{clip}%
\pgfsetbuttcap%
\pgfsetroundjoin%
\definecolor{currentfill}{rgb}{1.000000,0.894118,0.788235}%
\pgfsetfillcolor{currentfill}%
\pgfsetlinewidth{0.000000pt}%
\definecolor{currentstroke}{rgb}{1.000000,0.894118,0.788235}%
\pgfsetstrokecolor{currentstroke}%
\pgfsetdash{}{0pt}%
\pgfpathmoveto{\pgfqpoint{2.486831in}{3.149451in}}%
\pgfpathlineto{\pgfqpoint{2.597727in}{3.085425in}}%
\pgfpathlineto{\pgfqpoint{2.613910in}{3.125187in}}%
\pgfpathlineto{\pgfqpoint{2.503014in}{3.189213in}}%
\pgfpathlineto{\pgfqpoint{2.486831in}{3.149451in}}%
\pgfpathclose%
\pgfusepath{fill}%
\end{pgfscope}%
\begin{pgfscope}%
\pgfpathrectangle{\pgfqpoint{0.765000in}{0.660000in}}{\pgfqpoint{4.620000in}{4.620000in}}%
\pgfusepath{clip}%
\pgfsetbuttcap%
\pgfsetroundjoin%
\definecolor{currentfill}{rgb}{1.000000,0.894118,0.788235}%
\pgfsetfillcolor{currentfill}%
\pgfsetlinewidth{0.000000pt}%
\definecolor{currentstroke}{rgb}{1.000000,0.894118,0.788235}%
\pgfsetstrokecolor{currentstroke}%
\pgfsetdash{}{0pt}%
\pgfpathmoveto{\pgfqpoint{2.597727in}{3.085425in}}%
\pgfpathlineto{\pgfqpoint{2.708624in}{3.149451in}}%
\pgfpathlineto{\pgfqpoint{2.724807in}{3.189213in}}%
\pgfpathlineto{\pgfqpoint{2.613910in}{3.125187in}}%
\pgfpathlineto{\pgfqpoint{2.597727in}{3.085425in}}%
\pgfpathclose%
\pgfusepath{fill}%
\end{pgfscope}%
\begin{pgfscope}%
\pgfpathrectangle{\pgfqpoint{0.765000in}{0.660000in}}{\pgfqpoint{4.620000in}{4.620000in}}%
\pgfusepath{clip}%
\pgfsetbuttcap%
\pgfsetroundjoin%
\definecolor{currentfill}{rgb}{1.000000,0.894118,0.788235}%
\pgfsetfillcolor{currentfill}%
\pgfsetlinewidth{0.000000pt}%
\definecolor{currentstroke}{rgb}{1.000000,0.894118,0.788235}%
\pgfsetstrokecolor{currentstroke}%
\pgfsetdash{}{0pt}%
\pgfpathmoveto{\pgfqpoint{2.597727in}{3.341530in}}%
\pgfpathlineto{\pgfqpoint{2.486831in}{3.277504in}}%
\pgfpathlineto{\pgfqpoint{2.503014in}{3.317266in}}%
\pgfpathlineto{\pgfqpoint{2.613910in}{3.381292in}}%
\pgfpathlineto{\pgfqpoint{2.597727in}{3.341530in}}%
\pgfpathclose%
\pgfusepath{fill}%
\end{pgfscope}%
\begin{pgfscope}%
\pgfpathrectangle{\pgfqpoint{0.765000in}{0.660000in}}{\pgfqpoint{4.620000in}{4.620000in}}%
\pgfusepath{clip}%
\pgfsetbuttcap%
\pgfsetroundjoin%
\definecolor{currentfill}{rgb}{1.000000,0.894118,0.788235}%
\pgfsetfillcolor{currentfill}%
\pgfsetlinewidth{0.000000pt}%
\definecolor{currentstroke}{rgb}{1.000000,0.894118,0.788235}%
\pgfsetstrokecolor{currentstroke}%
\pgfsetdash{}{0pt}%
\pgfpathmoveto{\pgfqpoint{2.708624in}{3.277504in}}%
\pgfpathlineto{\pgfqpoint{2.597727in}{3.341530in}}%
\pgfpathlineto{\pgfqpoint{2.613910in}{3.381292in}}%
\pgfpathlineto{\pgfqpoint{2.724807in}{3.317266in}}%
\pgfpathlineto{\pgfqpoint{2.708624in}{3.277504in}}%
\pgfpathclose%
\pgfusepath{fill}%
\end{pgfscope}%
\begin{pgfscope}%
\pgfpathrectangle{\pgfqpoint{0.765000in}{0.660000in}}{\pgfqpoint{4.620000in}{4.620000in}}%
\pgfusepath{clip}%
\pgfsetbuttcap%
\pgfsetroundjoin%
\definecolor{currentfill}{rgb}{1.000000,0.894118,0.788235}%
\pgfsetfillcolor{currentfill}%
\pgfsetlinewidth{0.000000pt}%
\definecolor{currentstroke}{rgb}{1.000000,0.894118,0.788235}%
\pgfsetstrokecolor{currentstroke}%
\pgfsetdash{}{0pt}%
\pgfpathmoveto{\pgfqpoint{2.486831in}{3.149451in}}%
\pgfpathlineto{\pgfqpoint{2.486831in}{3.277504in}}%
\pgfpathlineto{\pgfqpoint{2.503014in}{3.317266in}}%
\pgfpathlineto{\pgfqpoint{2.613910in}{3.125187in}}%
\pgfpathlineto{\pgfqpoint{2.486831in}{3.149451in}}%
\pgfpathclose%
\pgfusepath{fill}%
\end{pgfscope}%
\begin{pgfscope}%
\pgfpathrectangle{\pgfqpoint{0.765000in}{0.660000in}}{\pgfqpoint{4.620000in}{4.620000in}}%
\pgfusepath{clip}%
\pgfsetbuttcap%
\pgfsetroundjoin%
\definecolor{currentfill}{rgb}{1.000000,0.894118,0.788235}%
\pgfsetfillcolor{currentfill}%
\pgfsetlinewidth{0.000000pt}%
\definecolor{currentstroke}{rgb}{1.000000,0.894118,0.788235}%
\pgfsetstrokecolor{currentstroke}%
\pgfsetdash{}{0pt}%
\pgfpathmoveto{\pgfqpoint{2.597727in}{3.085425in}}%
\pgfpathlineto{\pgfqpoint{2.597727in}{3.213477in}}%
\pgfpathlineto{\pgfqpoint{2.613910in}{3.253240in}}%
\pgfpathlineto{\pgfqpoint{2.503014in}{3.189213in}}%
\pgfpathlineto{\pgfqpoint{2.597727in}{3.085425in}}%
\pgfpathclose%
\pgfusepath{fill}%
\end{pgfscope}%
\begin{pgfscope}%
\pgfpathrectangle{\pgfqpoint{0.765000in}{0.660000in}}{\pgfqpoint{4.620000in}{4.620000in}}%
\pgfusepath{clip}%
\pgfsetbuttcap%
\pgfsetroundjoin%
\definecolor{currentfill}{rgb}{1.000000,0.894118,0.788235}%
\pgfsetfillcolor{currentfill}%
\pgfsetlinewidth{0.000000pt}%
\definecolor{currentstroke}{rgb}{1.000000,0.894118,0.788235}%
\pgfsetstrokecolor{currentstroke}%
\pgfsetdash{}{0pt}%
\pgfpathmoveto{\pgfqpoint{2.597727in}{3.213477in}}%
\pgfpathlineto{\pgfqpoint{2.708624in}{3.277504in}}%
\pgfpathlineto{\pgfqpoint{2.724807in}{3.317266in}}%
\pgfpathlineto{\pgfqpoint{2.613910in}{3.253240in}}%
\pgfpathlineto{\pgfqpoint{2.597727in}{3.213477in}}%
\pgfpathclose%
\pgfusepath{fill}%
\end{pgfscope}%
\begin{pgfscope}%
\pgfpathrectangle{\pgfqpoint{0.765000in}{0.660000in}}{\pgfqpoint{4.620000in}{4.620000in}}%
\pgfusepath{clip}%
\pgfsetbuttcap%
\pgfsetroundjoin%
\definecolor{currentfill}{rgb}{1.000000,0.894118,0.788235}%
\pgfsetfillcolor{currentfill}%
\pgfsetlinewidth{0.000000pt}%
\definecolor{currentstroke}{rgb}{1.000000,0.894118,0.788235}%
\pgfsetstrokecolor{currentstroke}%
\pgfsetdash{}{0pt}%
\pgfpathmoveto{\pgfqpoint{2.486831in}{3.277504in}}%
\pgfpathlineto{\pgfqpoint{2.597727in}{3.213477in}}%
\pgfpathlineto{\pgfqpoint{2.613910in}{3.253240in}}%
\pgfpathlineto{\pgfqpoint{2.503014in}{3.317266in}}%
\pgfpathlineto{\pgfqpoint{2.486831in}{3.277504in}}%
\pgfpathclose%
\pgfusepath{fill}%
\end{pgfscope}%
\begin{pgfscope}%
\pgfpathrectangle{\pgfqpoint{0.765000in}{0.660000in}}{\pgfqpoint{4.620000in}{4.620000in}}%
\pgfusepath{clip}%
\pgfsetbuttcap%
\pgfsetroundjoin%
\definecolor{currentfill}{rgb}{1.000000,0.894118,0.788235}%
\pgfsetfillcolor{currentfill}%
\pgfsetlinewidth{0.000000pt}%
\definecolor{currentstroke}{rgb}{1.000000,0.894118,0.788235}%
\pgfsetstrokecolor{currentstroke}%
\pgfsetdash{}{0pt}%
\pgfpathmoveto{\pgfqpoint{2.597727in}{3.213477in}}%
\pgfpathlineto{\pgfqpoint{2.486831in}{3.149451in}}%
\pgfpathlineto{\pgfqpoint{2.597727in}{3.085425in}}%
\pgfpathlineto{\pgfqpoint{2.708624in}{3.149451in}}%
\pgfpathlineto{\pgfqpoint{2.597727in}{3.213477in}}%
\pgfpathclose%
\pgfusepath{fill}%
\end{pgfscope}%
\begin{pgfscope}%
\pgfpathrectangle{\pgfqpoint{0.765000in}{0.660000in}}{\pgfqpoint{4.620000in}{4.620000in}}%
\pgfusepath{clip}%
\pgfsetbuttcap%
\pgfsetroundjoin%
\definecolor{currentfill}{rgb}{1.000000,0.894118,0.788235}%
\pgfsetfillcolor{currentfill}%
\pgfsetlinewidth{0.000000pt}%
\definecolor{currentstroke}{rgb}{1.000000,0.894118,0.788235}%
\pgfsetstrokecolor{currentstroke}%
\pgfsetdash{}{0pt}%
\pgfpathmoveto{\pgfqpoint{2.597727in}{3.213477in}}%
\pgfpathlineto{\pgfqpoint{2.486831in}{3.149451in}}%
\pgfpathlineto{\pgfqpoint{2.486831in}{3.277504in}}%
\pgfpathlineto{\pgfqpoint{2.597727in}{3.341530in}}%
\pgfpathlineto{\pgfqpoint{2.597727in}{3.213477in}}%
\pgfpathclose%
\pgfusepath{fill}%
\end{pgfscope}%
\begin{pgfscope}%
\pgfpathrectangle{\pgfqpoint{0.765000in}{0.660000in}}{\pgfqpoint{4.620000in}{4.620000in}}%
\pgfusepath{clip}%
\pgfsetbuttcap%
\pgfsetroundjoin%
\definecolor{currentfill}{rgb}{1.000000,0.894118,0.788235}%
\pgfsetfillcolor{currentfill}%
\pgfsetlinewidth{0.000000pt}%
\definecolor{currentstroke}{rgb}{1.000000,0.894118,0.788235}%
\pgfsetstrokecolor{currentstroke}%
\pgfsetdash{}{0pt}%
\pgfpathmoveto{\pgfqpoint{2.597727in}{3.213477in}}%
\pgfpathlineto{\pgfqpoint{2.708624in}{3.149451in}}%
\pgfpathlineto{\pgfqpoint{2.708624in}{3.277504in}}%
\pgfpathlineto{\pgfqpoint{2.597727in}{3.341530in}}%
\pgfpathlineto{\pgfqpoint{2.597727in}{3.213477in}}%
\pgfpathclose%
\pgfusepath{fill}%
\end{pgfscope}%
\begin{pgfscope}%
\pgfpathrectangle{\pgfqpoint{0.765000in}{0.660000in}}{\pgfqpoint{4.620000in}{4.620000in}}%
\pgfusepath{clip}%
\pgfsetbuttcap%
\pgfsetroundjoin%
\definecolor{currentfill}{rgb}{1.000000,0.894118,0.788235}%
\pgfsetfillcolor{currentfill}%
\pgfsetlinewidth{0.000000pt}%
\definecolor{currentstroke}{rgb}{1.000000,0.894118,0.788235}%
\pgfsetstrokecolor{currentstroke}%
\pgfsetdash{}{0pt}%
\pgfpathmoveto{\pgfqpoint{2.582969in}{3.161487in}}%
\pgfpathlineto{\pgfqpoint{2.472072in}{3.097461in}}%
\pgfpathlineto{\pgfqpoint{2.582969in}{3.033435in}}%
\pgfpathlineto{\pgfqpoint{2.693865in}{3.097461in}}%
\pgfpathlineto{\pgfqpoint{2.582969in}{3.161487in}}%
\pgfpathclose%
\pgfusepath{fill}%
\end{pgfscope}%
\begin{pgfscope}%
\pgfpathrectangle{\pgfqpoint{0.765000in}{0.660000in}}{\pgfqpoint{4.620000in}{4.620000in}}%
\pgfusepath{clip}%
\pgfsetbuttcap%
\pgfsetroundjoin%
\definecolor{currentfill}{rgb}{1.000000,0.894118,0.788235}%
\pgfsetfillcolor{currentfill}%
\pgfsetlinewidth{0.000000pt}%
\definecolor{currentstroke}{rgb}{1.000000,0.894118,0.788235}%
\pgfsetstrokecolor{currentstroke}%
\pgfsetdash{}{0pt}%
\pgfpathmoveto{\pgfqpoint{2.582969in}{3.161487in}}%
\pgfpathlineto{\pgfqpoint{2.472072in}{3.097461in}}%
\pgfpathlineto{\pgfqpoint{2.472072in}{3.225514in}}%
\pgfpathlineto{\pgfqpoint{2.582969in}{3.289540in}}%
\pgfpathlineto{\pgfqpoint{2.582969in}{3.161487in}}%
\pgfpathclose%
\pgfusepath{fill}%
\end{pgfscope}%
\begin{pgfscope}%
\pgfpathrectangle{\pgfqpoint{0.765000in}{0.660000in}}{\pgfqpoint{4.620000in}{4.620000in}}%
\pgfusepath{clip}%
\pgfsetbuttcap%
\pgfsetroundjoin%
\definecolor{currentfill}{rgb}{1.000000,0.894118,0.788235}%
\pgfsetfillcolor{currentfill}%
\pgfsetlinewidth{0.000000pt}%
\definecolor{currentstroke}{rgb}{1.000000,0.894118,0.788235}%
\pgfsetstrokecolor{currentstroke}%
\pgfsetdash{}{0pt}%
\pgfpathmoveto{\pgfqpoint{2.582969in}{3.161487in}}%
\pgfpathlineto{\pgfqpoint{2.693865in}{3.097461in}}%
\pgfpathlineto{\pgfqpoint{2.693865in}{3.225514in}}%
\pgfpathlineto{\pgfqpoint{2.582969in}{3.289540in}}%
\pgfpathlineto{\pgfqpoint{2.582969in}{3.161487in}}%
\pgfpathclose%
\pgfusepath{fill}%
\end{pgfscope}%
\begin{pgfscope}%
\pgfpathrectangle{\pgfqpoint{0.765000in}{0.660000in}}{\pgfqpoint{4.620000in}{4.620000in}}%
\pgfusepath{clip}%
\pgfsetbuttcap%
\pgfsetroundjoin%
\definecolor{currentfill}{rgb}{1.000000,0.894118,0.788235}%
\pgfsetfillcolor{currentfill}%
\pgfsetlinewidth{0.000000pt}%
\definecolor{currentstroke}{rgb}{1.000000,0.894118,0.788235}%
\pgfsetstrokecolor{currentstroke}%
\pgfsetdash{}{0pt}%
\pgfpathmoveto{\pgfqpoint{2.597727in}{3.341530in}}%
\pgfpathlineto{\pgfqpoint{2.486831in}{3.277504in}}%
\pgfpathlineto{\pgfqpoint{2.597727in}{3.213477in}}%
\pgfpathlineto{\pgfqpoint{2.708624in}{3.277504in}}%
\pgfpathlineto{\pgfqpoint{2.597727in}{3.341530in}}%
\pgfpathclose%
\pgfusepath{fill}%
\end{pgfscope}%
\begin{pgfscope}%
\pgfpathrectangle{\pgfqpoint{0.765000in}{0.660000in}}{\pgfqpoint{4.620000in}{4.620000in}}%
\pgfusepath{clip}%
\pgfsetbuttcap%
\pgfsetroundjoin%
\definecolor{currentfill}{rgb}{1.000000,0.894118,0.788235}%
\pgfsetfillcolor{currentfill}%
\pgfsetlinewidth{0.000000pt}%
\definecolor{currentstroke}{rgb}{1.000000,0.894118,0.788235}%
\pgfsetstrokecolor{currentstroke}%
\pgfsetdash{}{0pt}%
\pgfpathmoveto{\pgfqpoint{2.597727in}{3.085425in}}%
\pgfpathlineto{\pgfqpoint{2.708624in}{3.149451in}}%
\pgfpathlineto{\pgfqpoint{2.708624in}{3.277504in}}%
\pgfpathlineto{\pgfqpoint{2.597727in}{3.213477in}}%
\pgfpathlineto{\pgfqpoint{2.597727in}{3.085425in}}%
\pgfpathclose%
\pgfusepath{fill}%
\end{pgfscope}%
\begin{pgfscope}%
\pgfpathrectangle{\pgfqpoint{0.765000in}{0.660000in}}{\pgfqpoint{4.620000in}{4.620000in}}%
\pgfusepath{clip}%
\pgfsetbuttcap%
\pgfsetroundjoin%
\definecolor{currentfill}{rgb}{1.000000,0.894118,0.788235}%
\pgfsetfillcolor{currentfill}%
\pgfsetlinewidth{0.000000pt}%
\definecolor{currentstroke}{rgb}{1.000000,0.894118,0.788235}%
\pgfsetstrokecolor{currentstroke}%
\pgfsetdash{}{0pt}%
\pgfpathmoveto{\pgfqpoint{2.486831in}{3.149451in}}%
\pgfpathlineto{\pgfqpoint{2.597727in}{3.085425in}}%
\pgfpathlineto{\pgfqpoint{2.597727in}{3.213477in}}%
\pgfpathlineto{\pgfqpoint{2.486831in}{3.277504in}}%
\pgfpathlineto{\pgfqpoint{2.486831in}{3.149451in}}%
\pgfpathclose%
\pgfusepath{fill}%
\end{pgfscope}%
\begin{pgfscope}%
\pgfpathrectangle{\pgfqpoint{0.765000in}{0.660000in}}{\pgfqpoint{4.620000in}{4.620000in}}%
\pgfusepath{clip}%
\pgfsetbuttcap%
\pgfsetroundjoin%
\definecolor{currentfill}{rgb}{1.000000,0.894118,0.788235}%
\pgfsetfillcolor{currentfill}%
\pgfsetlinewidth{0.000000pt}%
\definecolor{currentstroke}{rgb}{1.000000,0.894118,0.788235}%
\pgfsetstrokecolor{currentstroke}%
\pgfsetdash{}{0pt}%
\pgfpathmoveto{\pgfqpoint{2.582969in}{3.289540in}}%
\pgfpathlineto{\pgfqpoint{2.472072in}{3.225514in}}%
\pgfpathlineto{\pgfqpoint{2.582969in}{3.161487in}}%
\pgfpathlineto{\pgfqpoint{2.693865in}{3.225514in}}%
\pgfpathlineto{\pgfqpoint{2.582969in}{3.289540in}}%
\pgfpathclose%
\pgfusepath{fill}%
\end{pgfscope}%
\begin{pgfscope}%
\pgfpathrectangle{\pgfqpoint{0.765000in}{0.660000in}}{\pgfqpoint{4.620000in}{4.620000in}}%
\pgfusepath{clip}%
\pgfsetbuttcap%
\pgfsetroundjoin%
\definecolor{currentfill}{rgb}{1.000000,0.894118,0.788235}%
\pgfsetfillcolor{currentfill}%
\pgfsetlinewidth{0.000000pt}%
\definecolor{currentstroke}{rgb}{1.000000,0.894118,0.788235}%
\pgfsetstrokecolor{currentstroke}%
\pgfsetdash{}{0pt}%
\pgfpathmoveto{\pgfqpoint{2.582969in}{3.033435in}}%
\pgfpathlineto{\pgfqpoint{2.693865in}{3.097461in}}%
\pgfpathlineto{\pgfqpoint{2.693865in}{3.225514in}}%
\pgfpathlineto{\pgfqpoint{2.582969in}{3.161487in}}%
\pgfpathlineto{\pgfqpoint{2.582969in}{3.033435in}}%
\pgfpathclose%
\pgfusepath{fill}%
\end{pgfscope}%
\begin{pgfscope}%
\pgfpathrectangle{\pgfqpoint{0.765000in}{0.660000in}}{\pgfqpoint{4.620000in}{4.620000in}}%
\pgfusepath{clip}%
\pgfsetbuttcap%
\pgfsetroundjoin%
\definecolor{currentfill}{rgb}{1.000000,0.894118,0.788235}%
\pgfsetfillcolor{currentfill}%
\pgfsetlinewidth{0.000000pt}%
\definecolor{currentstroke}{rgb}{1.000000,0.894118,0.788235}%
\pgfsetstrokecolor{currentstroke}%
\pgfsetdash{}{0pt}%
\pgfpathmoveto{\pgfqpoint{2.472072in}{3.097461in}}%
\pgfpathlineto{\pgfqpoint{2.582969in}{3.033435in}}%
\pgfpathlineto{\pgfqpoint{2.582969in}{3.161487in}}%
\pgfpathlineto{\pgfqpoint{2.472072in}{3.225514in}}%
\pgfpathlineto{\pgfqpoint{2.472072in}{3.097461in}}%
\pgfpathclose%
\pgfusepath{fill}%
\end{pgfscope}%
\begin{pgfscope}%
\pgfpathrectangle{\pgfqpoint{0.765000in}{0.660000in}}{\pgfqpoint{4.620000in}{4.620000in}}%
\pgfusepath{clip}%
\pgfsetbuttcap%
\pgfsetroundjoin%
\definecolor{currentfill}{rgb}{1.000000,0.894118,0.788235}%
\pgfsetfillcolor{currentfill}%
\pgfsetlinewidth{0.000000pt}%
\definecolor{currentstroke}{rgb}{1.000000,0.894118,0.788235}%
\pgfsetstrokecolor{currentstroke}%
\pgfsetdash{}{0pt}%
\pgfpathmoveto{\pgfqpoint{2.597727in}{3.213477in}}%
\pgfpathlineto{\pgfqpoint{2.486831in}{3.149451in}}%
\pgfpathlineto{\pgfqpoint{2.472072in}{3.097461in}}%
\pgfpathlineto{\pgfqpoint{2.582969in}{3.161487in}}%
\pgfpathlineto{\pgfqpoint{2.597727in}{3.213477in}}%
\pgfpathclose%
\pgfusepath{fill}%
\end{pgfscope}%
\begin{pgfscope}%
\pgfpathrectangle{\pgfqpoint{0.765000in}{0.660000in}}{\pgfqpoint{4.620000in}{4.620000in}}%
\pgfusepath{clip}%
\pgfsetbuttcap%
\pgfsetroundjoin%
\definecolor{currentfill}{rgb}{1.000000,0.894118,0.788235}%
\pgfsetfillcolor{currentfill}%
\pgfsetlinewidth{0.000000pt}%
\definecolor{currentstroke}{rgb}{1.000000,0.894118,0.788235}%
\pgfsetstrokecolor{currentstroke}%
\pgfsetdash{}{0pt}%
\pgfpathmoveto{\pgfqpoint{2.708624in}{3.149451in}}%
\pgfpathlineto{\pgfqpoint{2.597727in}{3.213477in}}%
\pgfpathlineto{\pgfqpoint{2.582969in}{3.161487in}}%
\pgfpathlineto{\pgfqpoint{2.693865in}{3.097461in}}%
\pgfpathlineto{\pgfqpoint{2.708624in}{3.149451in}}%
\pgfpathclose%
\pgfusepath{fill}%
\end{pgfscope}%
\begin{pgfscope}%
\pgfpathrectangle{\pgfqpoint{0.765000in}{0.660000in}}{\pgfqpoint{4.620000in}{4.620000in}}%
\pgfusepath{clip}%
\pgfsetbuttcap%
\pgfsetroundjoin%
\definecolor{currentfill}{rgb}{1.000000,0.894118,0.788235}%
\pgfsetfillcolor{currentfill}%
\pgfsetlinewidth{0.000000pt}%
\definecolor{currentstroke}{rgb}{1.000000,0.894118,0.788235}%
\pgfsetstrokecolor{currentstroke}%
\pgfsetdash{}{0pt}%
\pgfpathmoveto{\pgfqpoint{2.597727in}{3.213477in}}%
\pgfpathlineto{\pgfqpoint{2.597727in}{3.341530in}}%
\pgfpathlineto{\pgfqpoint{2.582969in}{3.289540in}}%
\pgfpathlineto{\pgfqpoint{2.693865in}{3.097461in}}%
\pgfpathlineto{\pgfqpoint{2.597727in}{3.213477in}}%
\pgfpathclose%
\pgfusepath{fill}%
\end{pgfscope}%
\begin{pgfscope}%
\pgfpathrectangle{\pgfqpoint{0.765000in}{0.660000in}}{\pgfqpoint{4.620000in}{4.620000in}}%
\pgfusepath{clip}%
\pgfsetbuttcap%
\pgfsetroundjoin%
\definecolor{currentfill}{rgb}{1.000000,0.894118,0.788235}%
\pgfsetfillcolor{currentfill}%
\pgfsetlinewidth{0.000000pt}%
\definecolor{currentstroke}{rgb}{1.000000,0.894118,0.788235}%
\pgfsetstrokecolor{currentstroke}%
\pgfsetdash{}{0pt}%
\pgfpathmoveto{\pgfqpoint{2.708624in}{3.149451in}}%
\pgfpathlineto{\pgfqpoint{2.708624in}{3.277504in}}%
\pgfpathlineto{\pgfqpoint{2.693865in}{3.225514in}}%
\pgfpathlineto{\pgfqpoint{2.582969in}{3.161487in}}%
\pgfpathlineto{\pgfqpoint{2.708624in}{3.149451in}}%
\pgfpathclose%
\pgfusepath{fill}%
\end{pgfscope}%
\begin{pgfscope}%
\pgfpathrectangle{\pgfqpoint{0.765000in}{0.660000in}}{\pgfqpoint{4.620000in}{4.620000in}}%
\pgfusepath{clip}%
\pgfsetbuttcap%
\pgfsetroundjoin%
\definecolor{currentfill}{rgb}{1.000000,0.894118,0.788235}%
\pgfsetfillcolor{currentfill}%
\pgfsetlinewidth{0.000000pt}%
\definecolor{currentstroke}{rgb}{1.000000,0.894118,0.788235}%
\pgfsetstrokecolor{currentstroke}%
\pgfsetdash{}{0pt}%
\pgfpathmoveto{\pgfqpoint{2.486831in}{3.149451in}}%
\pgfpathlineto{\pgfqpoint{2.597727in}{3.085425in}}%
\pgfpathlineto{\pgfqpoint{2.582969in}{3.033435in}}%
\pgfpathlineto{\pgfqpoint{2.472072in}{3.097461in}}%
\pgfpathlineto{\pgfqpoint{2.486831in}{3.149451in}}%
\pgfpathclose%
\pgfusepath{fill}%
\end{pgfscope}%
\begin{pgfscope}%
\pgfpathrectangle{\pgfqpoint{0.765000in}{0.660000in}}{\pgfqpoint{4.620000in}{4.620000in}}%
\pgfusepath{clip}%
\pgfsetbuttcap%
\pgfsetroundjoin%
\definecolor{currentfill}{rgb}{1.000000,0.894118,0.788235}%
\pgfsetfillcolor{currentfill}%
\pgfsetlinewidth{0.000000pt}%
\definecolor{currentstroke}{rgb}{1.000000,0.894118,0.788235}%
\pgfsetstrokecolor{currentstroke}%
\pgfsetdash{}{0pt}%
\pgfpathmoveto{\pgfqpoint{2.597727in}{3.085425in}}%
\pgfpathlineto{\pgfqpoint{2.708624in}{3.149451in}}%
\pgfpathlineto{\pgfqpoint{2.693865in}{3.097461in}}%
\pgfpathlineto{\pgfqpoint{2.582969in}{3.033435in}}%
\pgfpathlineto{\pgfqpoint{2.597727in}{3.085425in}}%
\pgfpathclose%
\pgfusepath{fill}%
\end{pgfscope}%
\begin{pgfscope}%
\pgfpathrectangle{\pgfqpoint{0.765000in}{0.660000in}}{\pgfqpoint{4.620000in}{4.620000in}}%
\pgfusepath{clip}%
\pgfsetbuttcap%
\pgfsetroundjoin%
\definecolor{currentfill}{rgb}{1.000000,0.894118,0.788235}%
\pgfsetfillcolor{currentfill}%
\pgfsetlinewidth{0.000000pt}%
\definecolor{currentstroke}{rgb}{1.000000,0.894118,0.788235}%
\pgfsetstrokecolor{currentstroke}%
\pgfsetdash{}{0pt}%
\pgfpathmoveto{\pgfqpoint{2.597727in}{3.341530in}}%
\pgfpathlineto{\pgfqpoint{2.486831in}{3.277504in}}%
\pgfpathlineto{\pgfqpoint{2.472072in}{3.225514in}}%
\pgfpathlineto{\pgfqpoint{2.582969in}{3.289540in}}%
\pgfpathlineto{\pgfqpoint{2.597727in}{3.341530in}}%
\pgfpathclose%
\pgfusepath{fill}%
\end{pgfscope}%
\begin{pgfscope}%
\pgfpathrectangle{\pgfqpoint{0.765000in}{0.660000in}}{\pgfqpoint{4.620000in}{4.620000in}}%
\pgfusepath{clip}%
\pgfsetbuttcap%
\pgfsetroundjoin%
\definecolor{currentfill}{rgb}{1.000000,0.894118,0.788235}%
\pgfsetfillcolor{currentfill}%
\pgfsetlinewidth{0.000000pt}%
\definecolor{currentstroke}{rgb}{1.000000,0.894118,0.788235}%
\pgfsetstrokecolor{currentstroke}%
\pgfsetdash{}{0pt}%
\pgfpathmoveto{\pgfqpoint{2.708624in}{3.277504in}}%
\pgfpathlineto{\pgfqpoint{2.597727in}{3.341530in}}%
\pgfpathlineto{\pgfqpoint{2.582969in}{3.289540in}}%
\pgfpathlineto{\pgfqpoint{2.693865in}{3.225514in}}%
\pgfpathlineto{\pgfqpoint{2.708624in}{3.277504in}}%
\pgfpathclose%
\pgfusepath{fill}%
\end{pgfscope}%
\begin{pgfscope}%
\pgfpathrectangle{\pgfqpoint{0.765000in}{0.660000in}}{\pgfqpoint{4.620000in}{4.620000in}}%
\pgfusepath{clip}%
\pgfsetbuttcap%
\pgfsetroundjoin%
\definecolor{currentfill}{rgb}{1.000000,0.894118,0.788235}%
\pgfsetfillcolor{currentfill}%
\pgfsetlinewidth{0.000000pt}%
\definecolor{currentstroke}{rgb}{1.000000,0.894118,0.788235}%
\pgfsetstrokecolor{currentstroke}%
\pgfsetdash{}{0pt}%
\pgfpathmoveto{\pgfqpoint{2.486831in}{3.149451in}}%
\pgfpathlineto{\pgfqpoint{2.486831in}{3.277504in}}%
\pgfpathlineto{\pgfqpoint{2.472072in}{3.225514in}}%
\pgfpathlineto{\pgfqpoint{2.582969in}{3.033435in}}%
\pgfpathlineto{\pgfqpoint{2.486831in}{3.149451in}}%
\pgfpathclose%
\pgfusepath{fill}%
\end{pgfscope}%
\begin{pgfscope}%
\pgfpathrectangle{\pgfqpoint{0.765000in}{0.660000in}}{\pgfqpoint{4.620000in}{4.620000in}}%
\pgfusepath{clip}%
\pgfsetbuttcap%
\pgfsetroundjoin%
\definecolor{currentfill}{rgb}{1.000000,0.894118,0.788235}%
\pgfsetfillcolor{currentfill}%
\pgfsetlinewidth{0.000000pt}%
\definecolor{currentstroke}{rgb}{1.000000,0.894118,0.788235}%
\pgfsetstrokecolor{currentstroke}%
\pgfsetdash{}{0pt}%
\pgfpathmoveto{\pgfqpoint{2.597727in}{3.085425in}}%
\pgfpathlineto{\pgfqpoint{2.597727in}{3.213477in}}%
\pgfpathlineto{\pgfqpoint{2.582969in}{3.161487in}}%
\pgfpathlineto{\pgfqpoint{2.472072in}{3.097461in}}%
\pgfpathlineto{\pgfqpoint{2.597727in}{3.085425in}}%
\pgfpathclose%
\pgfusepath{fill}%
\end{pgfscope}%
\begin{pgfscope}%
\pgfpathrectangle{\pgfqpoint{0.765000in}{0.660000in}}{\pgfqpoint{4.620000in}{4.620000in}}%
\pgfusepath{clip}%
\pgfsetbuttcap%
\pgfsetroundjoin%
\definecolor{currentfill}{rgb}{1.000000,0.894118,0.788235}%
\pgfsetfillcolor{currentfill}%
\pgfsetlinewidth{0.000000pt}%
\definecolor{currentstroke}{rgb}{1.000000,0.894118,0.788235}%
\pgfsetstrokecolor{currentstroke}%
\pgfsetdash{}{0pt}%
\pgfpathmoveto{\pgfqpoint{2.597727in}{3.213477in}}%
\pgfpathlineto{\pgfqpoint{2.708624in}{3.277504in}}%
\pgfpathlineto{\pgfqpoint{2.693865in}{3.225514in}}%
\pgfpathlineto{\pgfqpoint{2.582969in}{3.161487in}}%
\pgfpathlineto{\pgfqpoint{2.597727in}{3.213477in}}%
\pgfpathclose%
\pgfusepath{fill}%
\end{pgfscope}%
\begin{pgfscope}%
\pgfpathrectangle{\pgfqpoint{0.765000in}{0.660000in}}{\pgfqpoint{4.620000in}{4.620000in}}%
\pgfusepath{clip}%
\pgfsetbuttcap%
\pgfsetroundjoin%
\definecolor{currentfill}{rgb}{1.000000,0.894118,0.788235}%
\pgfsetfillcolor{currentfill}%
\pgfsetlinewidth{0.000000pt}%
\definecolor{currentstroke}{rgb}{1.000000,0.894118,0.788235}%
\pgfsetstrokecolor{currentstroke}%
\pgfsetdash{}{0pt}%
\pgfpathmoveto{\pgfqpoint{2.486831in}{3.277504in}}%
\pgfpathlineto{\pgfqpoint{2.597727in}{3.213477in}}%
\pgfpathlineto{\pgfqpoint{2.582969in}{3.161487in}}%
\pgfpathlineto{\pgfqpoint{2.472072in}{3.225514in}}%
\pgfpathlineto{\pgfqpoint{2.486831in}{3.277504in}}%
\pgfpathclose%
\pgfusepath{fill}%
\end{pgfscope}%
\begin{pgfscope}%
\pgfpathrectangle{\pgfqpoint{0.765000in}{0.660000in}}{\pgfqpoint{4.620000in}{4.620000in}}%
\pgfusepath{clip}%
\pgfsetbuttcap%
\pgfsetroundjoin%
\definecolor{currentfill}{rgb}{1.000000,0.894118,0.788235}%
\pgfsetfillcolor{currentfill}%
\pgfsetlinewidth{0.000000pt}%
\definecolor{currentstroke}{rgb}{1.000000,0.894118,0.788235}%
\pgfsetstrokecolor{currentstroke}%
\pgfsetdash{}{0pt}%
\pgfpathmoveto{\pgfqpoint{3.230111in}{3.500132in}}%
\pgfpathlineto{\pgfqpoint{3.119215in}{3.436106in}}%
\pgfpathlineto{\pgfqpoint{3.230111in}{3.372080in}}%
\pgfpathlineto{\pgfqpoint{3.341008in}{3.436106in}}%
\pgfpathlineto{\pgfqpoint{3.230111in}{3.500132in}}%
\pgfpathclose%
\pgfusepath{fill}%
\end{pgfscope}%
\begin{pgfscope}%
\pgfpathrectangle{\pgfqpoint{0.765000in}{0.660000in}}{\pgfqpoint{4.620000in}{4.620000in}}%
\pgfusepath{clip}%
\pgfsetbuttcap%
\pgfsetroundjoin%
\definecolor{currentfill}{rgb}{1.000000,0.894118,0.788235}%
\pgfsetfillcolor{currentfill}%
\pgfsetlinewidth{0.000000pt}%
\definecolor{currentstroke}{rgb}{1.000000,0.894118,0.788235}%
\pgfsetstrokecolor{currentstroke}%
\pgfsetdash{}{0pt}%
\pgfpathmoveto{\pgfqpoint{3.230111in}{3.500132in}}%
\pgfpathlineto{\pgfqpoint{3.119215in}{3.436106in}}%
\pgfpathlineto{\pgfqpoint{3.119215in}{3.564159in}}%
\pgfpathlineto{\pgfqpoint{3.230111in}{3.628185in}}%
\pgfpathlineto{\pgfqpoint{3.230111in}{3.500132in}}%
\pgfpathclose%
\pgfusepath{fill}%
\end{pgfscope}%
\begin{pgfscope}%
\pgfpathrectangle{\pgfqpoint{0.765000in}{0.660000in}}{\pgfqpoint{4.620000in}{4.620000in}}%
\pgfusepath{clip}%
\pgfsetbuttcap%
\pgfsetroundjoin%
\definecolor{currentfill}{rgb}{1.000000,0.894118,0.788235}%
\pgfsetfillcolor{currentfill}%
\pgfsetlinewidth{0.000000pt}%
\definecolor{currentstroke}{rgb}{1.000000,0.894118,0.788235}%
\pgfsetstrokecolor{currentstroke}%
\pgfsetdash{}{0pt}%
\pgfpathmoveto{\pgfqpoint{3.230111in}{3.500132in}}%
\pgfpathlineto{\pgfqpoint{3.341008in}{3.436106in}}%
\pgfpathlineto{\pgfqpoint{3.341008in}{3.564159in}}%
\pgfpathlineto{\pgfqpoint{3.230111in}{3.628185in}}%
\pgfpathlineto{\pgfqpoint{3.230111in}{3.500132in}}%
\pgfpathclose%
\pgfusepath{fill}%
\end{pgfscope}%
\begin{pgfscope}%
\pgfpathrectangle{\pgfqpoint{0.765000in}{0.660000in}}{\pgfqpoint{4.620000in}{4.620000in}}%
\pgfusepath{clip}%
\pgfsetbuttcap%
\pgfsetroundjoin%
\definecolor{currentfill}{rgb}{1.000000,0.894118,0.788235}%
\pgfsetfillcolor{currentfill}%
\pgfsetlinewidth{0.000000pt}%
\definecolor{currentstroke}{rgb}{1.000000,0.894118,0.788235}%
\pgfsetstrokecolor{currentstroke}%
\pgfsetdash{}{0pt}%
\pgfpathmoveto{\pgfqpoint{3.145667in}{3.522403in}}%
\pgfpathlineto{\pgfqpoint{3.034770in}{3.458376in}}%
\pgfpathlineto{\pgfqpoint{3.145667in}{3.394350in}}%
\pgfpathlineto{\pgfqpoint{3.256563in}{3.458376in}}%
\pgfpathlineto{\pgfqpoint{3.145667in}{3.522403in}}%
\pgfpathclose%
\pgfusepath{fill}%
\end{pgfscope}%
\begin{pgfscope}%
\pgfpathrectangle{\pgfqpoint{0.765000in}{0.660000in}}{\pgfqpoint{4.620000in}{4.620000in}}%
\pgfusepath{clip}%
\pgfsetbuttcap%
\pgfsetroundjoin%
\definecolor{currentfill}{rgb}{1.000000,0.894118,0.788235}%
\pgfsetfillcolor{currentfill}%
\pgfsetlinewidth{0.000000pt}%
\definecolor{currentstroke}{rgb}{1.000000,0.894118,0.788235}%
\pgfsetstrokecolor{currentstroke}%
\pgfsetdash{}{0pt}%
\pgfpathmoveto{\pgfqpoint{3.145667in}{3.522403in}}%
\pgfpathlineto{\pgfqpoint{3.034770in}{3.458376in}}%
\pgfpathlineto{\pgfqpoint{3.034770in}{3.586429in}}%
\pgfpathlineto{\pgfqpoint{3.145667in}{3.650455in}}%
\pgfpathlineto{\pgfqpoint{3.145667in}{3.522403in}}%
\pgfpathclose%
\pgfusepath{fill}%
\end{pgfscope}%
\begin{pgfscope}%
\pgfpathrectangle{\pgfqpoint{0.765000in}{0.660000in}}{\pgfqpoint{4.620000in}{4.620000in}}%
\pgfusepath{clip}%
\pgfsetbuttcap%
\pgfsetroundjoin%
\definecolor{currentfill}{rgb}{1.000000,0.894118,0.788235}%
\pgfsetfillcolor{currentfill}%
\pgfsetlinewidth{0.000000pt}%
\definecolor{currentstroke}{rgb}{1.000000,0.894118,0.788235}%
\pgfsetstrokecolor{currentstroke}%
\pgfsetdash{}{0pt}%
\pgfpathmoveto{\pgfqpoint{3.145667in}{3.522403in}}%
\pgfpathlineto{\pgfqpoint{3.256563in}{3.458376in}}%
\pgfpathlineto{\pgfqpoint{3.256563in}{3.586429in}}%
\pgfpathlineto{\pgfqpoint{3.145667in}{3.650455in}}%
\pgfpathlineto{\pgfqpoint{3.145667in}{3.522403in}}%
\pgfpathclose%
\pgfusepath{fill}%
\end{pgfscope}%
\begin{pgfscope}%
\pgfpathrectangle{\pgfqpoint{0.765000in}{0.660000in}}{\pgfqpoint{4.620000in}{4.620000in}}%
\pgfusepath{clip}%
\pgfsetbuttcap%
\pgfsetroundjoin%
\definecolor{currentfill}{rgb}{1.000000,0.894118,0.788235}%
\pgfsetfillcolor{currentfill}%
\pgfsetlinewidth{0.000000pt}%
\definecolor{currentstroke}{rgb}{1.000000,0.894118,0.788235}%
\pgfsetstrokecolor{currentstroke}%
\pgfsetdash{}{0pt}%
\pgfpathmoveto{\pgfqpoint{3.230111in}{3.628185in}}%
\pgfpathlineto{\pgfqpoint{3.119215in}{3.564159in}}%
\pgfpathlineto{\pgfqpoint{3.230111in}{3.500132in}}%
\pgfpathlineto{\pgfqpoint{3.341008in}{3.564159in}}%
\pgfpathlineto{\pgfqpoint{3.230111in}{3.628185in}}%
\pgfpathclose%
\pgfusepath{fill}%
\end{pgfscope}%
\begin{pgfscope}%
\pgfpathrectangle{\pgfqpoint{0.765000in}{0.660000in}}{\pgfqpoint{4.620000in}{4.620000in}}%
\pgfusepath{clip}%
\pgfsetbuttcap%
\pgfsetroundjoin%
\definecolor{currentfill}{rgb}{1.000000,0.894118,0.788235}%
\pgfsetfillcolor{currentfill}%
\pgfsetlinewidth{0.000000pt}%
\definecolor{currentstroke}{rgb}{1.000000,0.894118,0.788235}%
\pgfsetstrokecolor{currentstroke}%
\pgfsetdash{}{0pt}%
\pgfpathmoveto{\pgfqpoint{3.230111in}{3.372080in}}%
\pgfpathlineto{\pgfqpoint{3.341008in}{3.436106in}}%
\pgfpathlineto{\pgfqpoint{3.341008in}{3.564159in}}%
\pgfpathlineto{\pgfqpoint{3.230111in}{3.500132in}}%
\pgfpathlineto{\pgfqpoint{3.230111in}{3.372080in}}%
\pgfpathclose%
\pgfusepath{fill}%
\end{pgfscope}%
\begin{pgfscope}%
\pgfpathrectangle{\pgfqpoint{0.765000in}{0.660000in}}{\pgfqpoint{4.620000in}{4.620000in}}%
\pgfusepath{clip}%
\pgfsetbuttcap%
\pgfsetroundjoin%
\definecolor{currentfill}{rgb}{1.000000,0.894118,0.788235}%
\pgfsetfillcolor{currentfill}%
\pgfsetlinewidth{0.000000pt}%
\definecolor{currentstroke}{rgb}{1.000000,0.894118,0.788235}%
\pgfsetstrokecolor{currentstroke}%
\pgfsetdash{}{0pt}%
\pgfpathmoveto{\pgfqpoint{3.119215in}{3.436106in}}%
\pgfpathlineto{\pgfqpoint{3.230111in}{3.372080in}}%
\pgfpathlineto{\pgfqpoint{3.230111in}{3.500132in}}%
\pgfpathlineto{\pgfqpoint{3.119215in}{3.564159in}}%
\pgfpathlineto{\pgfqpoint{3.119215in}{3.436106in}}%
\pgfpathclose%
\pgfusepath{fill}%
\end{pgfscope}%
\begin{pgfscope}%
\pgfpathrectangle{\pgfqpoint{0.765000in}{0.660000in}}{\pgfqpoint{4.620000in}{4.620000in}}%
\pgfusepath{clip}%
\pgfsetbuttcap%
\pgfsetroundjoin%
\definecolor{currentfill}{rgb}{1.000000,0.894118,0.788235}%
\pgfsetfillcolor{currentfill}%
\pgfsetlinewidth{0.000000pt}%
\definecolor{currentstroke}{rgb}{1.000000,0.894118,0.788235}%
\pgfsetstrokecolor{currentstroke}%
\pgfsetdash{}{0pt}%
\pgfpathmoveto{\pgfqpoint{3.145667in}{3.650455in}}%
\pgfpathlineto{\pgfqpoint{3.034770in}{3.586429in}}%
\pgfpathlineto{\pgfqpoint{3.145667in}{3.522403in}}%
\pgfpathlineto{\pgfqpoint{3.256563in}{3.586429in}}%
\pgfpathlineto{\pgfqpoint{3.145667in}{3.650455in}}%
\pgfpathclose%
\pgfusepath{fill}%
\end{pgfscope}%
\begin{pgfscope}%
\pgfpathrectangle{\pgfqpoint{0.765000in}{0.660000in}}{\pgfqpoint{4.620000in}{4.620000in}}%
\pgfusepath{clip}%
\pgfsetbuttcap%
\pgfsetroundjoin%
\definecolor{currentfill}{rgb}{1.000000,0.894118,0.788235}%
\pgfsetfillcolor{currentfill}%
\pgfsetlinewidth{0.000000pt}%
\definecolor{currentstroke}{rgb}{1.000000,0.894118,0.788235}%
\pgfsetstrokecolor{currentstroke}%
\pgfsetdash{}{0pt}%
\pgfpathmoveto{\pgfqpoint{3.145667in}{3.394350in}}%
\pgfpathlineto{\pgfqpoint{3.256563in}{3.458376in}}%
\pgfpathlineto{\pgfqpoint{3.256563in}{3.586429in}}%
\pgfpathlineto{\pgfqpoint{3.145667in}{3.522403in}}%
\pgfpathlineto{\pgfqpoint{3.145667in}{3.394350in}}%
\pgfpathclose%
\pgfusepath{fill}%
\end{pgfscope}%
\begin{pgfscope}%
\pgfpathrectangle{\pgfqpoint{0.765000in}{0.660000in}}{\pgfqpoint{4.620000in}{4.620000in}}%
\pgfusepath{clip}%
\pgfsetbuttcap%
\pgfsetroundjoin%
\definecolor{currentfill}{rgb}{1.000000,0.894118,0.788235}%
\pgfsetfillcolor{currentfill}%
\pgfsetlinewidth{0.000000pt}%
\definecolor{currentstroke}{rgb}{1.000000,0.894118,0.788235}%
\pgfsetstrokecolor{currentstroke}%
\pgfsetdash{}{0pt}%
\pgfpathmoveto{\pgfqpoint{3.034770in}{3.458376in}}%
\pgfpathlineto{\pgfqpoint{3.145667in}{3.394350in}}%
\pgfpathlineto{\pgfqpoint{3.145667in}{3.522403in}}%
\pgfpathlineto{\pgfqpoint{3.034770in}{3.586429in}}%
\pgfpathlineto{\pgfqpoint{3.034770in}{3.458376in}}%
\pgfpathclose%
\pgfusepath{fill}%
\end{pgfscope}%
\begin{pgfscope}%
\pgfpathrectangle{\pgfqpoint{0.765000in}{0.660000in}}{\pgfqpoint{4.620000in}{4.620000in}}%
\pgfusepath{clip}%
\pgfsetbuttcap%
\pgfsetroundjoin%
\definecolor{currentfill}{rgb}{1.000000,0.894118,0.788235}%
\pgfsetfillcolor{currentfill}%
\pgfsetlinewidth{0.000000pt}%
\definecolor{currentstroke}{rgb}{1.000000,0.894118,0.788235}%
\pgfsetstrokecolor{currentstroke}%
\pgfsetdash{}{0pt}%
\pgfpathmoveto{\pgfqpoint{3.230111in}{3.500132in}}%
\pgfpathlineto{\pgfqpoint{3.119215in}{3.436106in}}%
\pgfpathlineto{\pgfqpoint{3.034770in}{3.458376in}}%
\pgfpathlineto{\pgfqpoint{3.145667in}{3.522403in}}%
\pgfpathlineto{\pgfqpoint{3.230111in}{3.500132in}}%
\pgfpathclose%
\pgfusepath{fill}%
\end{pgfscope}%
\begin{pgfscope}%
\pgfpathrectangle{\pgfqpoint{0.765000in}{0.660000in}}{\pgfqpoint{4.620000in}{4.620000in}}%
\pgfusepath{clip}%
\pgfsetbuttcap%
\pgfsetroundjoin%
\definecolor{currentfill}{rgb}{1.000000,0.894118,0.788235}%
\pgfsetfillcolor{currentfill}%
\pgfsetlinewidth{0.000000pt}%
\definecolor{currentstroke}{rgb}{1.000000,0.894118,0.788235}%
\pgfsetstrokecolor{currentstroke}%
\pgfsetdash{}{0pt}%
\pgfpathmoveto{\pgfqpoint{3.341008in}{3.436106in}}%
\pgfpathlineto{\pgfqpoint{3.230111in}{3.500132in}}%
\pgfpathlineto{\pgfqpoint{3.145667in}{3.522403in}}%
\pgfpathlineto{\pgfqpoint{3.256563in}{3.458376in}}%
\pgfpathlineto{\pgfqpoint{3.341008in}{3.436106in}}%
\pgfpathclose%
\pgfusepath{fill}%
\end{pgfscope}%
\begin{pgfscope}%
\pgfpathrectangle{\pgfqpoint{0.765000in}{0.660000in}}{\pgfqpoint{4.620000in}{4.620000in}}%
\pgfusepath{clip}%
\pgfsetbuttcap%
\pgfsetroundjoin%
\definecolor{currentfill}{rgb}{1.000000,0.894118,0.788235}%
\pgfsetfillcolor{currentfill}%
\pgfsetlinewidth{0.000000pt}%
\definecolor{currentstroke}{rgb}{1.000000,0.894118,0.788235}%
\pgfsetstrokecolor{currentstroke}%
\pgfsetdash{}{0pt}%
\pgfpathmoveto{\pgfqpoint{3.230111in}{3.500132in}}%
\pgfpathlineto{\pgfqpoint{3.230111in}{3.628185in}}%
\pgfpathlineto{\pgfqpoint{3.145667in}{3.650455in}}%
\pgfpathlineto{\pgfqpoint{3.256563in}{3.458376in}}%
\pgfpathlineto{\pgfqpoint{3.230111in}{3.500132in}}%
\pgfpathclose%
\pgfusepath{fill}%
\end{pgfscope}%
\begin{pgfscope}%
\pgfpathrectangle{\pgfqpoint{0.765000in}{0.660000in}}{\pgfqpoint{4.620000in}{4.620000in}}%
\pgfusepath{clip}%
\pgfsetbuttcap%
\pgfsetroundjoin%
\definecolor{currentfill}{rgb}{1.000000,0.894118,0.788235}%
\pgfsetfillcolor{currentfill}%
\pgfsetlinewidth{0.000000pt}%
\definecolor{currentstroke}{rgb}{1.000000,0.894118,0.788235}%
\pgfsetstrokecolor{currentstroke}%
\pgfsetdash{}{0pt}%
\pgfpathmoveto{\pgfqpoint{3.341008in}{3.436106in}}%
\pgfpathlineto{\pgfqpoint{3.341008in}{3.564159in}}%
\pgfpathlineto{\pgfqpoint{3.256563in}{3.586429in}}%
\pgfpathlineto{\pgfqpoint{3.145667in}{3.522403in}}%
\pgfpathlineto{\pgfqpoint{3.341008in}{3.436106in}}%
\pgfpathclose%
\pgfusepath{fill}%
\end{pgfscope}%
\begin{pgfscope}%
\pgfpathrectangle{\pgfqpoint{0.765000in}{0.660000in}}{\pgfqpoint{4.620000in}{4.620000in}}%
\pgfusepath{clip}%
\pgfsetbuttcap%
\pgfsetroundjoin%
\definecolor{currentfill}{rgb}{1.000000,0.894118,0.788235}%
\pgfsetfillcolor{currentfill}%
\pgfsetlinewidth{0.000000pt}%
\definecolor{currentstroke}{rgb}{1.000000,0.894118,0.788235}%
\pgfsetstrokecolor{currentstroke}%
\pgfsetdash{}{0pt}%
\pgfpathmoveto{\pgfqpoint{3.119215in}{3.436106in}}%
\pgfpathlineto{\pgfqpoint{3.230111in}{3.372080in}}%
\pgfpathlineto{\pgfqpoint{3.145667in}{3.394350in}}%
\pgfpathlineto{\pgfqpoint{3.034770in}{3.458376in}}%
\pgfpathlineto{\pgfqpoint{3.119215in}{3.436106in}}%
\pgfpathclose%
\pgfusepath{fill}%
\end{pgfscope}%
\begin{pgfscope}%
\pgfpathrectangle{\pgfqpoint{0.765000in}{0.660000in}}{\pgfqpoint{4.620000in}{4.620000in}}%
\pgfusepath{clip}%
\pgfsetbuttcap%
\pgfsetroundjoin%
\definecolor{currentfill}{rgb}{1.000000,0.894118,0.788235}%
\pgfsetfillcolor{currentfill}%
\pgfsetlinewidth{0.000000pt}%
\definecolor{currentstroke}{rgb}{1.000000,0.894118,0.788235}%
\pgfsetstrokecolor{currentstroke}%
\pgfsetdash{}{0pt}%
\pgfpathmoveto{\pgfqpoint{3.230111in}{3.372080in}}%
\pgfpathlineto{\pgfqpoint{3.341008in}{3.436106in}}%
\pgfpathlineto{\pgfqpoint{3.256563in}{3.458376in}}%
\pgfpathlineto{\pgfqpoint{3.145667in}{3.394350in}}%
\pgfpathlineto{\pgfqpoint{3.230111in}{3.372080in}}%
\pgfpathclose%
\pgfusepath{fill}%
\end{pgfscope}%
\begin{pgfscope}%
\pgfpathrectangle{\pgfqpoint{0.765000in}{0.660000in}}{\pgfqpoint{4.620000in}{4.620000in}}%
\pgfusepath{clip}%
\pgfsetbuttcap%
\pgfsetroundjoin%
\definecolor{currentfill}{rgb}{1.000000,0.894118,0.788235}%
\pgfsetfillcolor{currentfill}%
\pgfsetlinewidth{0.000000pt}%
\definecolor{currentstroke}{rgb}{1.000000,0.894118,0.788235}%
\pgfsetstrokecolor{currentstroke}%
\pgfsetdash{}{0pt}%
\pgfpathmoveto{\pgfqpoint{3.230111in}{3.628185in}}%
\pgfpathlineto{\pgfqpoint{3.119215in}{3.564159in}}%
\pgfpathlineto{\pgfqpoint{3.034770in}{3.586429in}}%
\pgfpathlineto{\pgfqpoint{3.145667in}{3.650455in}}%
\pgfpathlineto{\pgfqpoint{3.230111in}{3.628185in}}%
\pgfpathclose%
\pgfusepath{fill}%
\end{pgfscope}%
\begin{pgfscope}%
\pgfpathrectangle{\pgfqpoint{0.765000in}{0.660000in}}{\pgfqpoint{4.620000in}{4.620000in}}%
\pgfusepath{clip}%
\pgfsetbuttcap%
\pgfsetroundjoin%
\definecolor{currentfill}{rgb}{1.000000,0.894118,0.788235}%
\pgfsetfillcolor{currentfill}%
\pgfsetlinewidth{0.000000pt}%
\definecolor{currentstroke}{rgb}{1.000000,0.894118,0.788235}%
\pgfsetstrokecolor{currentstroke}%
\pgfsetdash{}{0pt}%
\pgfpathmoveto{\pgfqpoint{3.341008in}{3.564159in}}%
\pgfpathlineto{\pgfqpoint{3.230111in}{3.628185in}}%
\pgfpathlineto{\pgfqpoint{3.145667in}{3.650455in}}%
\pgfpathlineto{\pgfqpoint{3.256563in}{3.586429in}}%
\pgfpathlineto{\pgfqpoint{3.341008in}{3.564159in}}%
\pgfpathclose%
\pgfusepath{fill}%
\end{pgfscope}%
\begin{pgfscope}%
\pgfpathrectangle{\pgfqpoint{0.765000in}{0.660000in}}{\pgfqpoint{4.620000in}{4.620000in}}%
\pgfusepath{clip}%
\pgfsetbuttcap%
\pgfsetroundjoin%
\definecolor{currentfill}{rgb}{1.000000,0.894118,0.788235}%
\pgfsetfillcolor{currentfill}%
\pgfsetlinewidth{0.000000pt}%
\definecolor{currentstroke}{rgb}{1.000000,0.894118,0.788235}%
\pgfsetstrokecolor{currentstroke}%
\pgfsetdash{}{0pt}%
\pgfpathmoveto{\pgfqpoint{3.119215in}{3.436106in}}%
\pgfpathlineto{\pgfqpoint{3.119215in}{3.564159in}}%
\pgfpathlineto{\pgfqpoint{3.034770in}{3.586429in}}%
\pgfpathlineto{\pgfqpoint{3.145667in}{3.394350in}}%
\pgfpathlineto{\pgfqpoint{3.119215in}{3.436106in}}%
\pgfpathclose%
\pgfusepath{fill}%
\end{pgfscope}%
\begin{pgfscope}%
\pgfpathrectangle{\pgfqpoint{0.765000in}{0.660000in}}{\pgfqpoint{4.620000in}{4.620000in}}%
\pgfusepath{clip}%
\pgfsetbuttcap%
\pgfsetroundjoin%
\definecolor{currentfill}{rgb}{1.000000,0.894118,0.788235}%
\pgfsetfillcolor{currentfill}%
\pgfsetlinewidth{0.000000pt}%
\definecolor{currentstroke}{rgb}{1.000000,0.894118,0.788235}%
\pgfsetstrokecolor{currentstroke}%
\pgfsetdash{}{0pt}%
\pgfpathmoveto{\pgfqpoint{3.230111in}{3.372080in}}%
\pgfpathlineto{\pgfqpoint{3.230111in}{3.500132in}}%
\pgfpathlineto{\pgfqpoint{3.145667in}{3.522403in}}%
\pgfpathlineto{\pgfqpoint{3.034770in}{3.458376in}}%
\pgfpathlineto{\pgfqpoint{3.230111in}{3.372080in}}%
\pgfpathclose%
\pgfusepath{fill}%
\end{pgfscope}%
\begin{pgfscope}%
\pgfpathrectangle{\pgfqpoint{0.765000in}{0.660000in}}{\pgfqpoint{4.620000in}{4.620000in}}%
\pgfusepath{clip}%
\pgfsetbuttcap%
\pgfsetroundjoin%
\definecolor{currentfill}{rgb}{1.000000,0.894118,0.788235}%
\pgfsetfillcolor{currentfill}%
\pgfsetlinewidth{0.000000pt}%
\definecolor{currentstroke}{rgb}{1.000000,0.894118,0.788235}%
\pgfsetstrokecolor{currentstroke}%
\pgfsetdash{}{0pt}%
\pgfpathmoveto{\pgfqpoint{3.230111in}{3.500132in}}%
\pgfpathlineto{\pgfqpoint{3.341008in}{3.564159in}}%
\pgfpathlineto{\pgfqpoint{3.256563in}{3.586429in}}%
\pgfpathlineto{\pgfqpoint{3.145667in}{3.522403in}}%
\pgfpathlineto{\pgfqpoint{3.230111in}{3.500132in}}%
\pgfpathclose%
\pgfusepath{fill}%
\end{pgfscope}%
\begin{pgfscope}%
\pgfpathrectangle{\pgfqpoint{0.765000in}{0.660000in}}{\pgfqpoint{4.620000in}{4.620000in}}%
\pgfusepath{clip}%
\pgfsetbuttcap%
\pgfsetroundjoin%
\definecolor{currentfill}{rgb}{1.000000,0.894118,0.788235}%
\pgfsetfillcolor{currentfill}%
\pgfsetlinewidth{0.000000pt}%
\definecolor{currentstroke}{rgb}{1.000000,0.894118,0.788235}%
\pgfsetstrokecolor{currentstroke}%
\pgfsetdash{}{0pt}%
\pgfpathmoveto{\pgfqpoint{3.119215in}{3.564159in}}%
\pgfpathlineto{\pgfqpoint{3.230111in}{3.500132in}}%
\pgfpathlineto{\pgfqpoint{3.145667in}{3.522403in}}%
\pgfpathlineto{\pgfqpoint{3.034770in}{3.586429in}}%
\pgfpathlineto{\pgfqpoint{3.119215in}{3.564159in}}%
\pgfpathclose%
\pgfusepath{fill}%
\end{pgfscope}%
\begin{pgfscope}%
\pgfpathrectangle{\pgfqpoint{0.765000in}{0.660000in}}{\pgfqpoint{4.620000in}{4.620000in}}%
\pgfusepath{clip}%
\pgfsetbuttcap%
\pgfsetroundjoin%
\definecolor{currentfill}{rgb}{1.000000,0.894118,0.788235}%
\pgfsetfillcolor{currentfill}%
\pgfsetlinewidth{0.000000pt}%
\definecolor{currentstroke}{rgb}{1.000000,0.894118,0.788235}%
\pgfsetstrokecolor{currentstroke}%
\pgfsetdash{}{0pt}%
\pgfpathmoveto{\pgfqpoint{3.230111in}{3.500132in}}%
\pgfpathlineto{\pgfqpoint{3.119215in}{3.436106in}}%
\pgfpathlineto{\pgfqpoint{3.230111in}{3.372080in}}%
\pgfpathlineto{\pgfqpoint{3.341008in}{3.436106in}}%
\pgfpathlineto{\pgfqpoint{3.230111in}{3.500132in}}%
\pgfpathclose%
\pgfusepath{fill}%
\end{pgfscope}%
\begin{pgfscope}%
\pgfpathrectangle{\pgfqpoint{0.765000in}{0.660000in}}{\pgfqpoint{4.620000in}{4.620000in}}%
\pgfusepath{clip}%
\pgfsetbuttcap%
\pgfsetroundjoin%
\definecolor{currentfill}{rgb}{1.000000,0.894118,0.788235}%
\pgfsetfillcolor{currentfill}%
\pgfsetlinewidth{0.000000pt}%
\definecolor{currentstroke}{rgb}{1.000000,0.894118,0.788235}%
\pgfsetstrokecolor{currentstroke}%
\pgfsetdash{}{0pt}%
\pgfpathmoveto{\pgfqpoint{3.230111in}{3.500132in}}%
\pgfpathlineto{\pgfqpoint{3.119215in}{3.436106in}}%
\pgfpathlineto{\pgfqpoint{3.119215in}{3.564159in}}%
\pgfpathlineto{\pgfqpoint{3.230111in}{3.628185in}}%
\pgfpathlineto{\pgfqpoint{3.230111in}{3.500132in}}%
\pgfpathclose%
\pgfusepath{fill}%
\end{pgfscope}%
\begin{pgfscope}%
\pgfpathrectangle{\pgfqpoint{0.765000in}{0.660000in}}{\pgfqpoint{4.620000in}{4.620000in}}%
\pgfusepath{clip}%
\pgfsetbuttcap%
\pgfsetroundjoin%
\definecolor{currentfill}{rgb}{1.000000,0.894118,0.788235}%
\pgfsetfillcolor{currentfill}%
\pgfsetlinewidth{0.000000pt}%
\definecolor{currentstroke}{rgb}{1.000000,0.894118,0.788235}%
\pgfsetstrokecolor{currentstroke}%
\pgfsetdash{}{0pt}%
\pgfpathmoveto{\pgfqpoint{3.230111in}{3.500132in}}%
\pgfpathlineto{\pgfqpoint{3.341008in}{3.436106in}}%
\pgfpathlineto{\pgfqpoint{3.341008in}{3.564159in}}%
\pgfpathlineto{\pgfqpoint{3.230111in}{3.628185in}}%
\pgfpathlineto{\pgfqpoint{3.230111in}{3.500132in}}%
\pgfpathclose%
\pgfusepath{fill}%
\end{pgfscope}%
\begin{pgfscope}%
\pgfpathrectangle{\pgfqpoint{0.765000in}{0.660000in}}{\pgfqpoint{4.620000in}{4.620000in}}%
\pgfusepath{clip}%
\pgfsetbuttcap%
\pgfsetroundjoin%
\definecolor{currentfill}{rgb}{1.000000,0.894118,0.788235}%
\pgfsetfillcolor{currentfill}%
\pgfsetlinewidth{0.000000pt}%
\definecolor{currentstroke}{rgb}{1.000000,0.894118,0.788235}%
\pgfsetstrokecolor{currentstroke}%
\pgfsetdash{}{0pt}%
\pgfpathmoveto{\pgfqpoint{3.265487in}{3.489049in}}%
\pgfpathlineto{\pgfqpoint{3.154590in}{3.425023in}}%
\pgfpathlineto{\pgfqpoint{3.265487in}{3.360996in}}%
\pgfpathlineto{\pgfqpoint{3.376384in}{3.425023in}}%
\pgfpathlineto{\pgfqpoint{3.265487in}{3.489049in}}%
\pgfpathclose%
\pgfusepath{fill}%
\end{pgfscope}%
\begin{pgfscope}%
\pgfpathrectangle{\pgfqpoint{0.765000in}{0.660000in}}{\pgfqpoint{4.620000in}{4.620000in}}%
\pgfusepath{clip}%
\pgfsetbuttcap%
\pgfsetroundjoin%
\definecolor{currentfill}{rgb}{1.000000,0.894118,0.788235}%
\pgfsetfillcolor{currentfill}%
\pgfsetlinewidth{0.000000pt}%
\definecolor{currentstroke}{rgb}{1.000000,0.894118,0.788235}%
\pgfsetstrokecolor{currentstroke}%
\pgfsetdash{}{0pt}%
\pgfpathmoveto{\pgfqpoint{3.265487in}{3.489049in}}%
\pgfpathlineto{\pgfqpoint{3.154590in}{3.425023in}}%
\pgfpathlineto{\pgfqpoint{3.154590in}{3.553075in}}%
\pgfpathlineto{\pgfqpoint{3.265487in}{3.617101in}}%
\pgfpathlineto{\pgfqpoint{3.265487in}{3.489049in}}%
\pgfpathclose%
\pgfusepath{fill}%
\end{pgfscope}%
\begin{pgfscope}%
\pgfpathrectangle{\pgfqpoint{0.765000in}{0.660000in}}{\pgfqpoint{4.620000in}{4.620000in}}%
\pgfusepath{clip}%
\pgfsetbuttcap%
\pgfsetroundjoin%
\definecolor{currentfill}{rgb}{1.000000,0.894118,0.788235}%
\pgfsetfillcolor{currentfill}%
\pgfsetlinewidth{0.000000pt}%
\definecolor{currentstroke}{rgb}{1.000000,0.894118,0.788235}%
\pgfsetstrokecolor{currentstroke}%
\pgfsetdash{}{0pt}%
\pgfpathmoveto{\pgfqpoint{3.265487in}{3.489049in}}%
\pgfpathlineto{\pgfqpoint{3.376384in}{3.425023in}}%
\pgfpathlineto{\pgfqpoint{3.376384in}{3.553075in}}%
\pgfpathlineto{\pgfqpoint{3.265487in}{3.617101in}}%
\pgfpathlineto{\pgfqpoint{3.265487in}{3.489049in}}%
\pgfpathclose%
\pgfusepath{fill}%
\end{pgfscope}%
\begin{pgfscope}%
\pgfpathrectangle{\pgfqpoint{0.765000in}{0.660000in}}{\pgfqpoint{4.620000in}{4.620000in}}%
\pgfusepath{clip}%
\pgfsetbuttcap%
\pgfsetroundjoin%
\definecolor{currentfill}{rgb}{1.000000,0.894118,0.788235}%
\pgfsetfillcolor{currentfill}%
\pgfsetlinewidth{0.000000pt}%
\definecolor{currentstroke}{rgb}{1.000000,0.894118,0.788235}%
\pgfsetstrokecolor{currentstroke}%
\pgfsetdash{}{0pt}%
\pgfpathmoveto{\pgfqpoint{3.230111in}{3.628185in}}%
\pgfpathlineto{\pgfqpoint{3.119215in}{3.564159in}}%
\pgfpathlineto{\pgfqpoint{3.230111in}{3.500132in}}%
\pgfpathlineto{\pgfqpoint{3.341008in}{3.564159in}}%
\pgfpathlineto{\pgfqpoint{3.230111in}{3.628185in}}%
\pgfpathclose%
\pgfusepath{fill}%
\end{pgfscope}%
\begin{pgfscope}%
\pgfpathrectangle{\pgfqpoint{0.765000in}{0.660000in}}{\pgfqpoint{4.620000in}{4.620000in}}%
\pgfusepath{clip}%
\pgfsetbuttcap%
\pgfsetroundjoin%
\definecolor{currentfill}{rgb}{1.000000,0.894118,0.788235}%
\pgfsetfillcolor{currentfill}%
\pgfsetlinewidth{0.000000pt}%
\definecolor{currentstroke}{rgb}{1.000000,0.894118,0.788235}%
\pgfsetstrokecolor{currentstroke}%
\pgfsetdash{}{0pt}%
\pgfpathmoveto{\pgfqpoint{3.230111in}{3.372080in}}%
\pgfpathlineto{\pgfqpoint{3.341008in}{3.436106in}}%
\pgfpathlineto{\pgfqpoint{3.341008in}{3.564159in}}%
\pgfpathlineto{\pgfqpoint{3.230111in}{3.500132in}}%
\pgfpathlineto{\pgfqpoint{3.230111in}{3.372080in}}%
\pgfpathclose%
\pgfusepath{fill}%
\end{pgfscope}%
\begin{pgfscope}%
\pgfpathrectangle{\pgfqpoint{0.765000in}{0.660000in}}{\pgfqpoint{4.620000in}{4.620000in}}%
\pgfusepath{clip}%
\pgfsetbuttcap%
\pgfsetroundjoin%
\definecolor{currentfill}{rgb}{1.000000,0.894118,0.788235}%
\pgfsetfillcolor{currentfill}%
\pgfsetlinewidth{0.000000pt}%
\definecolor{currentstroke}{rgb}{1.000000,0.894118,0.788235}%
\pgfsetstrokecolor{currentstroke}%
\pgfsetdash{}{0pt}%
\pgfpathmoveto{\pgfqpoint{3.119215in}{3.436106in}}%
\pgfpathlineto{\pgfqpoint{3.230111in}{3.372080in}}%
\pgfpathlineto{\pgfqpoint{3.230111in}{3.500132in}}%
\pgfpathlineto{\pgfqpoint{3.119215in}{3.564159in}}%
\pgfpathlineto{\pgfqpoint{3.119215in}{3.436106in}}%
\pgfpathclose%
\pgfusepath{fill}%
\end{pgfscope}%
\begin{pgfscope}%
\pgfpathrectangle{\pgfqpoint{0.765000in}{0.660000in}}{\pgfqpoint{4.620000in}{4.620000in}}%
\pgfusepath{clip}%
\pgfsetbuttcap%
\pgfsetroundjoin%
\definecolor{currentfill}{rgb}{1.000000,0.894118,0.788235}%
\pgfsetfillcolor{currentfill}%
\pgfsetlinewidth{0.000000pt}%
\definecolor{currentstroke}{rgb}{1.000000,0.894118,0.788235}%
\pgfsetstrokecolor{currentstroke}%
\pgfsetdash{}{0pt}%
\pgfpathmoveto{\pgfqpoint{3.265487in}{3.617101in}}%
\pgfpathlineto{\pgfqpoint{3.154590in}{3.553075in}}%
\pgfpathlineto{\pgfqpoint{3.265487in}{3.489049in}}%
\pgfpathlineto{\pgfqpoint{3.376384in}{3.553075in}}%
\pgfpathlineto{\pgfqpoint{3.265487in}{3.617101in}}%
\pgfpathclose%
\pgfusepath{fill}%
\end{pgfscope}%
\begin{pgfscope}%
\pgfpathrectangle{\pgfqpoint{0.765000in}{0.660000in}}{\pgfqpoint{4.620000in}{4.620000in}}%
\pgfusepath{clip}%
\pgfsetbuttcap%
\pgfsetroundjoin%
\definecolor{currentfill}{rgb}{1.000000,0.894118,0.788235}%
\pgfsetfillcolor{currentfill}%
\pgfsetlinewidth{0.000000pt}%
\definecolor{currentstroke}{rgb}{1.000000,0.894118,0.788235}%
\pgfsetstrokecolor{currentstroke}%
\pgfsetdash{}{0pt}%
\pgfpathmoveto{\pgfqpoint{3.265487in}{3.360996in}}%
\pgfpathlineto{\pgfqpoint{3.376384in}{3.425023in}}%
\pgfpathlineto{\pgfqpoint{3.376384in}{3.553075in}}%
\pgfpathlineto{\pgfqpoint{3.265487in}{3.489049in}}%
\pgfpathlineto{\pgfqpoint{3.265487in}{3.360996in}}%
\pgfpathclose%
\pgfusepath{fill}%
\end{pgfscope}%
\begin{pgfscope}%
\pgfpathrectangle{\pgfqpoint{0.765000in}{0.660000in}}{\pgfqpoint{4.620000in}{4.620000in}}%
\pgfusepath{clip}%
\pgfsetbuttcap%
\pgfsetroundjoin%
\definecolor{currentfill}{rgb}{1.000000,0.894118,0.788235}%
\pgfsetfillcolor{currentfill}%
\pgfsetlinewidth{0.000000pt}%
\definecolor{currentstroke}{rgb}{1.000000,0.894118,0.788235}%
\pgfsetstrokecolor{currentstroke}%
\pgfsetdash{}{0pt}%
\pgfpathmoveto{\pgfqpoint{3.154590in}{3.425023in}}%
\pgfpathlineto{\pgfqpoint{3.265487in}{3.360996in}}%
\pgfpathlineto{\pgfqpoint{3.265487in}{3.489049in}}%
\pgfpathlineto{\pgfqpoint{3.154590in}{3.553075in}}%
\pgfpathlineto{\pgfqpoint{3.154590in}{3.425023in}}%
\pgfpathclose%
\pgfusepath{fill}%
\end{pgfscope}%
\begin{pgfscope}%
\pgfpathrectangle{\pgfqpoint{0.765000in}{0.660000in}}{\pgfqpoint{4.620000in}{4.620000in}}%
\pgfusepath{clip}%
\pgfsetbuttcap%
\pgfsetroundjoin%
\definecolor{currentfill}{rgb}{1.000000,0.894118,0.788235}%
\pgfsetfillcolor{currentfill}%
\pgfsetlinewidth{0.000000pt}%
\definecolor{currentstroke}{rgb}{1.000000,0.894118,0.788235}%
\pgfsetstrokecolor{currentstroke}%
\pgfsetdash{}{0pt}%
\pgfpathmoveto{\pgfqpoint{3.230111in}{3.500132in}}%
\pgfpathlineto{\pgfqpoint{3.119215in}{3.436106in}}%
\pgfpathlineto{\pgfqpoint{3.154590in}{3.425023in}}%
\pgfpathlineto{\pgfqpoint{3.265487in}{3.489049in}}%
\pgfpathlineto{\pgfqpoint{3.230111in}{3.500132in}}%
\pgfpathclose%
\pgfusepath{fill}%
\end{pgfscope}%
\begin{pgfscope}%
\pgfpathrectangle{\pgfqpoint{0.765000in}{0.660000in}}{\pgfqpoint{4.620000in}{4.620000in}}%
\pgfusepath{clip}%
\pgfsetbuttcap%
\pgfsetroundjoin%
\definecolor{currentfill}{rgb}{1.000000,0.894118,0.788235}%
\pgfsetfillcolor{currentfill}%
\pgfsetlinewidth{0.000000pt}%
\definecolor{currentstroke}{rgb}{1.000000,0.894118,0.788235}%
\pgfsetstrokecolor{currentstroke}%
\pgfsetdash{}{0pt}%
\pgfpathmoveto{\pgfqpoint{3.341008in}{3.436106in}}%
\pgfpathlineto{\pgfqpoint{3.230111in}{3.500132in}}%
\pgfpathlineto{\pgfqpoint{3.265487in}{3.489049in}}%
\pgfpathlineto{\pgfqpoint{3.376384in}{3.425023in}}%
\pgfpathlineto{\pgfqpoint{3.341008in}{3.436106in}}%
\pgfpathclose%
\pgfusepath{fill}%
\end{pgfscope}%
\begin{pgfscope}%
\pgfpathrectangle{\pgfqpoint{0.765000in}{0.660000in}}{\pgfqpoint{4.620000in}{4.620000in}}%
\pgfusepath{clip}%
\pgfsetbuttcap%
\pgfsetroundjoin%
\definecolor{currentfill}{rgb}{1.000000,0.894118,0.788235}%
\pgfsetfillcolor{currentfill}%
\pgfsetlinewidth{0.000000pt}%
\definecolor{currentstroke}{rgb}{1.000000,0.894118,0.788235}%
\pgfsetstrokecolor{currentstroke}%
\pgfsetdash{}{0pt}%
\pgfpathmoveto{\pgfqpoint{3.230111in}{3.500132in}}%
\pgfpathlineto{\pgfqpoint{3.230111in}{3.628185in}}%
\pgfpathlineto{\pgfqpoint{3.265487in}{3.617101in}}%
\pgfpathlineto{\pgfqpoint{3.376384in}{3.425023in}}%
\pgfpathlineto{\pgfqpoint{3.230111in}{3.500132in}}%
\pgfpathclose%
\pgfusepath{fill}%
\end{pgfscope}%
\begin{pgfscope}%
\pgfpathrectangle{\pgfqpoint{0.765000in}{0.660000in}}{\pgfqpoint{4.620000in}{4.620000in}}%
\pgfusepath{clip}%
\pgfsetbuttcap%
\pgfsetroundjoin%
\definecolor{currentfill}{rgb}{1.000000,0.894118,0.788235}%
\pgfsetfillcolor{currentfill}%
\pgfsetlinewidth{0.000000pt}%
\definecolor{currentstroke}{rgb}{1.000000,0.894118,0.788235}%
\pgfsetstrokecolor{currentstroke}%
\pgfsetdash{}{0pt}%
\pgfpathmoveto{\pgfqpoint{3.341008in}{3.436106in}}%
\pgfpathlineto{\pgfqpoint{3.341008in}{3.564159in}}%
\pgfpathlineto{\pgfqpoint{3.376384in}{3.553075in}}%
\pgfpathlineto{\pgfqpoint{3.265487in}{3.489049in}}%
\pgfpathlineto{\pgfqpoint{3.341008in}{3.436106in}}%
\pgfpathclose%
\pgfusepath{fill}%
\end{pgfscope}%
\begin{pgfscope}%
\pgfpathrectangle{\pgfqpoint{0.765000in}{0.660000in}}{\pgfqpoint{4.620000in}{4.620000in}}%
\pgfusepath{clip}%
\pgfsetbuttcap%
\pgfsetroundjoin%
\definecolor{currentfill}{rgb}{1.000000,0.894118,0.788235}%
\pgfsetfillcolor{currentfill}%
\pgfsetlinewidth{0.000000pt}%
\definecolor{currentstroke}{rgb}{1.000000,0.894118,0.788235}%
\pgfsetstrokecolor{currentstroke}%
\pgfsetdash{}{0pt}%
\pgfpathmoveto{\pgfqpoint{3.119215in}{3.436106in}}%
\pgfpathlineto{\pgfqpoint{3.230111in}{3.372080in}}%
\pgfpathlineto{\pgfqpoint{3.265487in}{3.360996in}}%
\pgfpathlineto{\pgfqpoint{3.154590in}{3.425023in}}%
\pgfpathlineto{\pgfqpoint{3.119215in}{3.436106in}}%
\pgfpathclose%
\pgfusepath{fill}%
\end{pgfscope}%
\begin{pgfscope}%
\pgfpathrectangle{\pgfqpoint{0.765000in}{0.660000in}}{\pgfqpoint{4.620000in}{4.620000in}}%
\pgfusepath{clip}%
\pgfsetbuttcap%
\pgfsetroundjoin%
\definecolor{currentfill}{rgb}{1.000000,0.894118,0.788235}%
\pgfsetfillcolor{currentfill}%
\pgfsetlinewidth{0.000000pt}%
\definecolor{currentstroke}{rgb}{1.000000,0.894118,0.788235}%
\pgfsetstrokecolor{currentstroke}%
\pgfsetdash{}{0pt}%
\pgfpathmoveto{\pgfqpoint{3.230111in}{3.372080in}}%
\pgfpathlineto{\pgfqpoint{3.341008in}{3.436106in}}%
\pgfpathlineto{\pgfqpoint{3.376384in}{3.425023in}}%
\pgfpathlineto{\pgfqpoint{3.265487in}{3.360996in}}%
\pgfpathlineto{\pgfqpoint{3.230111in}{3.372080in}}%
\pgfpathclose%
\pgfusepath{fill}%
\end{pgfscope}%
\begin{pgfscope}%
\pgfpathrectangle{\pgfqpoint{0.765000in}{0.660000in}}{\pgfqpoint{4.620000in}{4.620000in}}%
\pgfusepath{clip}%
\pgfsetbuttcap%
\pgfsetroundjoin%
\definecolor{currentfill}{rgb}{1.000000,0.894118,0.788235}%
\pgfsetfillcolor{currentfill}%
\pgfsetlinewidth{0.000000pt}%
\definecolor{currentstroke}{rgb}{1.000000,0.894118,0.788235}%
\pgfsetstrokecolor{currentstroke}%
\pgfsetdash{}{0pt}%
\pgfpathmoveto{\pgfqpoint{3.230111in}{3.628185in}}%
\pgfpathlineto{\pgfqpoint{3.119215in}{3.564159in}}%
\pgfpathlineto{\pgfqpoint{3.154590in}{3.553075in}}%
\pgfpathlineto{\pgfqpoint{3.265487in}{3.617101in}}%
\pgfpathlineto{\pgfqpoint{3.230111in}{3.628185in}}%
\pgfpathclose%
\pgfusepath{fill}%
\end{pgfscope}%
\begin{pgfscope}%
\pgfpathrectangle{\pgfqpoint{0.765000in}{0.660000in}}{\pgfqpoint{4.620000in}{4.620000in}}%
\pgfusepath{clip}%
\pgfsetbuttcap%
\pgfsetroundjoin%
\definecolor{currentfill}{rgb}{1.000000,0.894118,0.788235}%
\pgfsetfillcolor{currentfill}%
\pgfsetlinewidth{0.000000pt}%
\definecolor{currentstroke}{rgb}{1.000000,0.894118,0.788235}%
\pgfsetstrokecolor{currentstroke}%
\pgfsetdash{}{0pt}%
\pgfpathmoveto{\pgfqpoint{3.341008in}{3.564159in}}%
\pgfpathlineto{\pgfqpoint{3.230111in}{3.628185in}}%
\pgfpathlineto{\pgfqpoint{3.265487in}{3.617101in}}%
\pgfpathlineto{\pgfqpoint{3.376384in}{3.553075in}}%
\pgfpathlineto{\pgfqpoint{3.341008in}{3.564159in}}%
\pgfpathclose%
\pgfusepath{fill}%
\end{pgfscope}%
\begin{pgfscope}%
\pgfpathrectangle{\pgfqpoint{0.765000in}{0.660000in}}{\pgfqpoint{4.620000in}{4.620000in}}%
\pgfusepath{clip}%
\pgfsetbuttcap%
\pgfsetroundjoin%
\definecolor{currentfill}{rgb}{1.000000,0.894118,0.788235}%
\pgfsetfillcolor{currentfill}%
\pgfsetlinewidth{0.000000pt}%
\definecolor{currentstroke}{rgb}{1.000000,0.894118,0.788235}%
\pgfsetstrokecolor{currentstroke}%
\pgfsetdash{}{0pt}%
\pgfpathmoveto{\pgfqpoint{3.119215in}{3.436106in}}%
\pgfpathlineto{\pgfqpoint{3.119215in}{3.564159in}}%
\pgfpathlineto{\pgfqpoint{3.154590in}{3.553075in}}%
\pgfpathlineto{\pgfqpoint{3.265487in}{3.360996in}}%
\pgfpathlineto{\pgfqpoint{3.119215in}{3.436106in}}%
\pgfpathclose%
\pgfusepath{fill}%
\end{pgfscope}%
\begin{pgfscope}%
\pgfpathrectangle{\pgfqpoint{0.765000in}{0.660000in}}{\pgfqpoint{4.620000in}{4.620000in}}%
\pgfusepath{clip}%
\pgfsetbuttcap%
\pgfsetroundjoin%
\definecolor{currentfill}{rgb}{1.000000,0.894118,0.788235}%
\pgfsetfillcolor{currentfill}%
\pgfsetlinewidth{0.000000pt}%
\definecolor{currentstroke}{rgb}{1.000000,0.894118,0.788235}%
\pgfsetstrokecolor{currentstroke}%
\pgfsetdash{}{0pt}%
\pgfpathmoveto{\pgfqpoint{3.230111in}{3.372080in}}%
\pgfpathlineto{\pgfqpoint{3.230111in}{3.500132in}}%
\pgfpathlineto{\pgfqpoint{3.265487in}{3.489049in}}%
\pgfpathlineto{\pgfqpoint{3.154590in}{3.425023in}}%
\pgfpathlineto{\pgfqpoint{3.230111in}{3.372080in}}%
\pgfpathclose%
\pgfusepath{fill}%
\end{pgfscope}%
\begin{pgfscope}%
\pgfpathrectangle{\pgfqpoint{0.765000in}{0.660000in}}{\pgfqpoint{4.620000in}{4.620000in}}%
\pgfusepath{clip}%
\pgfsetbuttcap%
\pgfsetroundjoin%
\definecolor{currentfill}{rgb}{1.000000,0.894118,0.788235}%
\pgfsetfillcolor{currentfill}%
\pgfsetlinewidth{0.000000pt}%
\definecolor{currentstroke}{rgb}{1.000000,0.894118,0.788235}%
\pgfsetstrokecolor{currentstroke}%
\pgfsetdash{}{0pt}%
\pgfpathmoveto{\pgfqpoint{3.230111in}{3.500132in}}%
\pgfpathlineto{\pgfqpoint{3.341008in}{3.564159in}}%
\pgfpathlineto{\pgfqpoint{3.376384in}{3.553075in}}%
\pgfpathlineto{\pgfqpoint{3.265487in}{3.489049in}}%
\pgfpathlineto{\pgfqpoint{3.230111in}{3.500132in}}%
\pgfpathclose%
\pgfusepath{fill}%
\end{pgfscope}%
\begin{pgfscope}%
\pgfpathrectangle{\pgfqpoint{0.765000in}{0.660000in}}{\pgfqpoint{4.620000in}{4.620000in}}%
\pgfusepath{clip}%
\pgfsetbuttcap%
\pgfsetroundjoin%
\definecolor{currentfill}{rgb}{1.000000,0.894118,0.788235}%
\pgfsetfillcolor{currentfill}%
\pgfsetlinewidth{0.000000pt}%
\definecolor{currentstroke}{rgb}{1.000000,0.894118,0.788235}%
\pgfsetstrokecolor{currentstroke}%
\pgfsetdash{}{0pt}%
\pgfpathmoveto{\pgfqpoint{3.119215in}{3.564159in}}%
\pgfpathlineto{\pgfqpoint{3.230111in}{3.500132in}}%
\pgfpathlineto{\pgfqpoint{3.265487in}{3.489049in}}%
\pgfpathlineto{\pgfqpoint{3.154590in}{3.553075in}}%
\pgfpathlineto{\pgfqpoint{3.119215in}{3.564159in}}%
\pgfpathclose%
\pgfusepath{fill}%
\end{pgfscope}%
\begin{pgfscope}%
\pgfpathrectangle{\pgfqpoint{0.765000in}{0.660000in}}{\pgfqpoint{4.620000in}{4.620000in}}%
\pgfusepath{clip}%
\pgfsetrectcap%
\pgfsetroundjoin%
\pgfsetlinewidth{1.204500pt}%
\definecolor{currentstroke}{rgb}{1.000000,0.576471,0.309804}%
\pgfsetstrokecolor{currentstroke}%
\pgfsetdash{}{0pt}%
\pgfpathmoveto{\pgfqpoint{2.291739in}{2.573807in}}%
\pgfusepath{stroke}%
\end{pgfscope}%
\begin{pgfscope}%
\pgfpathrectangle{\pgfqpoint{0.765000in}{0.660000in}}{\pgfqpoint{4.620000in}{4.620000in}}%
\pgfusepath{clip}%
\pgfsetbuttcap%
\pgfsetroundjoin%
\definecolor{currentfill}{rgb}{1.000000,0.576471,0.309804}%
\pgfsetfillcolor{currentfill}%
\pgfsetlinewidth{1.003750pt}%
\definecolor{currentstroke}{rgb}{1.000000,0.576471,0.309804}%
\pgfsetstrokecolor{currentstroke}%
\pgfsetdash{}{0pt}%
\pgfsys@defobject{currentmarker}{\pgfqpoint{-0.033333in}{-0.033333in}}{\pgfqpoint{0.033333in}{0.033333in}}{%
\pgfpathmoveto{\pgfqpoint{0.000000in}{-0.033333in}}%
\pgfpathcurveto{\pgfqpoint{0.008840in}{-0.033333in}}{\pgfqpoint{0.017319in}{-0.029821in}}{\pgfqpoint{0.023570in}{-0.023570in}}%
\pgfpathcurveto{\pgfqpoint{0.029821in}{-0.017319in}}{\pgfqpoint{0.033333in}{-0.008840in}}{\pgfqpoint{0.033333in}{0.000000in}}%
\pgfpathcurveto{\pgfqpoint{0.033333in}{0.008840in}}{\pgfqpoint{0.029821in}{0.017319in}}{\pgfqpoint{0.023570in}{0.023570in}}%
\pgfpathcurveto{\pgfqpoint{0.017319in}{0.029821in}}{\pgfqpoint{0.008840in}{0.033333in}}{\pgfqpoint{0.000000in}{0.033333in}}%
\pgfpathcurveto{\pgfqpoint{-0.008840in}{0.033333in}}{\pgfqpoint{-0.017319in}{0.029821in}}{\pgfqpoint{-0.023570in}{0.023570in}}%
\pgfpathcurveto{\pgfqpoint{-0.029821in}{0.017319in}}{\pgfqpoint{-0.033333in}{0.008840in}}{\pgfqpoint{-0.033333in}{0.000000in}}%
\pgfpathcurveto{\pgfqpoint{-0.033333in}{-0.008840in}}{\pgfqpoint{-0.029821in}{-0.017319in}}{\pgfqpoint{-0.023570in}{-0.023570in}}%
\pgfpathcurveto{\pgfqpoint{-0.017319in}{-0.029821in}}{\pgfqpoint{-0.008840in}{-0.033333in}}{\pgfqpoint{0.000000in}{-0.033333in}}%
\pgfpathlineto{\pgfqpoint{0.000000in}{-0.033333in}}%
\pgfpathclose%
\pgfusepath{stroke,fill}%
}%
\begin{pgfscope}%
\pgfsys@transformshift{2.291739in}{2.573807in}%
\pgfsys@useobject{currentmarker}{}%
\end{pgfscope}%
\end{pgfscope}%
\begin{pgfscope}%
\pgfpathrectangle{\pgfqpoint{0.765000in}{0.660000in}}{\pgfqpoint{4.620000in}{4.620000in}}%
\pgfusepath{clip}%
\pgfsetrectcap%
\pgfsetroundjoin%
\pgfsetlinewidth{1.204500pt}%
\definecolor{currentstroke}{rgb}{1.000000,0.576471,0.309804}%
\pgfsetstrokecolor{currentstroke}%
\pgfsetdash{}{0pt}%
\pgfpathmoveto{\pgfqpoint{2.742212in}{2.153332in}}%
\pgfusepath{stroke}%
\end{pgfscope}%
\begin{pgfscope}%
\pgfpathrectangle{\pgfqpoint{0.765000in}{0.660000in}}{\pgfqpoint{4.620000in}{4.620000in}}%
\pgfusepath{clip}%
\pgfsetbuttcap%
\pgfsetroundjoin%
\definecolor{currentfill}{rgb}{1.000000,0.576471,0.309804}%
\pgfsetfillcolor{currentfill}%
\pgfsetlinewidth{1.003750pt}%
\definecolor{currentstroke}{rgb}{1.000000,0.576471,0.309804}%
\pgfsetstrokecolor{currentstroke}%
\pgfsetdash{}{0pt}%
\pgfsys@defobject{currentmarker}{\pgfqpoint{-0.033333in}{-0.033333in}}{\pgfqpoint{0.033333in}{0.033333in}}{%
\pgfpathmoveto{\pgfqpoint{0.000000in}{-0.033333in}}%
\pgfpathcurveto{\pgfqpoint{0.008840in}{-0.033333in}}{\pgfqpoint{0.017319in}{-0.029821in}}{\pgfqpoint{0.023570in}{-0.023570in}}%
\pgfpathcurveto{\pgfqpoint{0.029821in}{-0.017319in}}{\pgfqpoint{0.033333in}{-0.008840in}}{\pgfqpoint{0.033333in}{0.000000in}}%
\pgfpathcurveto{\pgfqpoint{0.033333in}{0.008840in}}{\pgfqpoint{0.029821in}{0.017319in}}{\pgfqpoint{0.023570in}{0.023570in}}%
\pgfpathcurveto{\pgfqpoint{0.017319in}{0.029821in}}{\pgfqpoint{0.008840in}{0.033333in}}{\pgfqpoint{0.000000in}{0.033333in}}%
\pgfpathcurveto{\pgfqpoint{-0.008840in}{0.033333in}}{\pgfqpoint{-0.017319in}{0.029821in}}{\pgfqpoint{-0.023570in}{0.023570in}}%
\pgfpathcurveto{\pgfqpoint{-0.029821in}{0.017319in}}{\pgfqpoint{-0.033333in}{0.008840in}}{\pgfqpoint{-0.033333in}{0.000000in}}%
\pgfpathcurveto{\pgfqpoint{-0.033333in}{-0.008840in}}{\pgfqpoint{-0.029821in}{-0.017319in}}{\pgfqpoint{-0.023570in}{-0.023570in}}%
\pgfpathcurveto{\pgfqpoint{-0.017319in}{-0.029821in}}{\pgfqpoint{-0.008840in}{-0.033333in}}{\pgfqpoint{0.000000in}{-0.033333in}}%
\pgfpathlineto{\pgfqpoint{0.000000in}{-0.033333in}}%
\pgfpathclose%
\pgfusepath{stroke,fill}%
}%
\begin{pgfscope}%
\pgfsys@transformshift{2.742212in}{2.153332in}%
\pgfsys@useobject{currentmarker}{}%
\end{pgfscope}%
\end{pgfscope}%
\begin{pgfscope}%
\pgfpathrectangle{\pgfqpoint{0.765000in}{0.660000in}}{\pgfqpoint{4.620000in}{4.620000in}}%
\pgfusepath{clip}%
\pgfsetrectcap%
\pgfsetroundjoin%
\pgfsetlinewidth{1.204500pt}%
\definecolor{currentstroke}{rgb}{1.000000,0.576471,0.309804}%
\pgfsetstrokecolor{currentstroke}%
\pgfsetdash{}{0pt}%
\pgfpathmoveto{\pgfqpoint{2.102070in}{2.567929in}}%
\pgfusepath{stroke}%
\end{pgfscope}%
\begin{pgfscope}%
\pgfpathrectangle{\pgfqpoint{0.765000in}{0.660000in}}{\pgfqpoint{4.620000in}{4.620000in}}%
\pgfusepath{clip}%
\pgfsetbuttcap%
\pgfsetroundjoin%
\definecolor{currentfill}{rgb}{1.000000,0.576471,0.309804}%
\pgfsetfillcolor{currentfill}%
\pgfsetlinewidth{1.003750pt}%
\definecolor{currentstroke}{rgb}{1.000000,0.576471,0.309804}%
\pgfsetstrokecolor{currentstroke}%
\pgfsetdash{}{0pt}%
\pgfsys@defobject{currentmarker}{\pgfqpoint{-0.033333in}{-0.033333in}}{\pgfqpoint{0.033333in}{0.033333in}}{%
\pgfpathmoveto{\pgfqpoint{0.000000in}{-0.033333in}}%
\pgfpathcurveto{\pgfqpoint{0.008840in}{-0.033333in}}{\pgfqpoint{0.017319in}{-0.029821in}}{\pgfqpoint{0.023570in}{-0.023570in}}%
\pgfpathcurveto{\pgfqpoint{0.029821in}{-0.017319in}}{\pgfqpoint{0.033333in}{-0.008840in}}{\pgfqpoint{0.033333in}{0.000000in}}%
\pgfpathcurveto{\pgfqpoint{0.033333in}{0.008840in}}{\pgfqpoint{0.029821in}{0.017319in}}{\pgfqpoint{0.023570in}{0.023570in}}%
\pgfpathcurveto{\pgfqpoint{0.017319in}{0.029821in}}{\pgfqpoint{0.008840in}{0.033333in}}{\pgfqpoint{0.000000in}{0.033333in}}%
\pgfpathcurveto{\pgfqpoint{-0.008840in}{0.033333in}}{\pgfqpoint{-0.017319in}{0.029821in}}{\pgfqpoint{-0.023570in}{0.023570in}}%
\pgfpathcurveto{\pgfqpoint{-0.029821in}{0.017319in}}{\pgfqpoint{-0.033333in}{0.008840in}}{\pgfqpoint{-0.033333in}{0.000000in}}%
\pgfpathcurveto{\pgfqpoint{-0.033333in}{-0.008840in}}{\pgfqpoint{-0.029821in}{-0.017319in}}{\pgfqpoint{-0.023570in}{-0.023570in}}%
\pgfpathcurveto{\pgfqpoint{-0.017319in}{-0.029821in}}{\pgfqpoint{-0.008840in}{-0.033333in}}{\pgfqpoint{0.000000in}{-0.033333in}}%
\pgfpathlineto{\pgfqpoint{0.000000in}{-0.033333in}}%
\pgfpathclose%
\pgfusepath{stroke,fill}%
}%
\begin{pgfscope}%
\pgfsys@transformshift{2.102070in}{2.567929in}%
\pgfsys@useobject{currentmarker}{}%
\end{pgfscope}%
\end{pgfscope}%
\begin{pgfscope}%
\pgfpathrectangle{\pgfqpoint{0.765000in}{0.660000in}}{\pgfqpoint{4.620000in}{4.620000in}}%
\pgfusepath{clip}%
\pgfsetrectcap%
\pgfsetroundjoin%
\pgfsetlinewidth{1.204500pt}%
\definecolor{currentstroke}{rgb}{1.000000,0.576471,0.309804}%
\pgfsetstrokecolor{currentstroke}%
\pgfsetdash{}{0pt}%
\pgfpathmoveto{\pgfqpoint{2.888730in}{4.501916in}}%
\pgfusepath{stroke}%
\end{pgfscope}%
\begin{pgfscope}%
\pgfpathrectangle{\pgfqpoint{0.765000in}{0.660000in}}{\pgfqpoint{4.620000in}{4.620000in}}%
\pgfusepath{clip}%
\pgfsetbuttcap%
\pgfsetroundjoin%
\definecolor{currentfill}{rgb}{1.000000,0.576471,0.309804}%
\pgfsetfillcolor{currentfill}%
\pgfsetlinewidth{1.003750pt}%
\definecolor{currentstroke}{rgb}{1.000000,0.576471,0.309804}%
\pgfsetstrokecolor{currentstroke}%
\pgfsetdash{}{0pt}%
\pgfsys@defobject{currentmarker}{\pgfqpoint{-0.033333in}{-0.033333in}}{\pgfqpoint{0.033333in}{0.033333in}}{%
\pgfpathmoveto{\pgfqpoint{0.000000in}{-0.033333in}}%
\pgfpathcurveto{\pgfqpoint{0.008840in}{-0.033333in}}{\pgfqpoint{0.017319in}{-0.029821in}}{\pgfqpoint{0.023570in}{-0.023570in}}%
\pgfpathcurveto{\pgfqpoint{0.029821in}{-0.017319in}}{\pgfqpoint{0.033333in}{-0.008840in}}{\pgfqpoint{0.033333in}{0.000000in}}%
\pgfpathcurveto{\pgfqpoint{0.033333in}{0.008840in}}{\pgfqpoint{0.029821in}{0.017319in}}{\pgfqpoint{0.023570in}{0.023570in}}%
\pgfpathcurveto{\pgfqpoint{0.017319in}{0.029821in}}{\pgfqpoint{0.008840in}{0.033333in}}{\pgfqpoint{0.000000in}{0.033333in}}%
\pgfpathcurveto{\pgfqpoint{-0.008840in}{0.033333in}}{\pgfqpoint{-0.017319in}{0.029821in}}{\pgfqpoint{-0.023570in}{0.023570in}}%
\pgfpathcurveto{\pgfqpoint{-0.029821in}{0.017319in}}{\pgfqpoint{-0.033333in}{0.008840in}}{\pgfqpoint{-0.033333in}{0.000000in}}%
\pgfpathcurveto{\pgfqpoint{-0.033333in}{-0.008840in}}{\pgfqpoint{-0.029821in}{-0.017319in}}{\pgfqpoint{-0.023570in}{-0.023570in}}%
\pgfpathcurveto{\pgfqpoint{-0.017319in}{-0.029821in}}{\pgfqpoint{-0.008840in}{-0.033333in}}{\pgfqpoint{0.000000in}{-0.033333in}}%
\pgfpathlineto{\pgfqpoint{0.000000in}{-0.033333in}}%
\pgfpathclose%
\pgfusepath{stroke,fill}%
}%
\begin{pgfscope}%
\pgfsys@transformshift{2.888730in}{4.501916in}%
\pgfsys@useobject{currentmarker}{}%
\end{pgfscope}%
\end{pgfscope}%
\begin{pgfscope}%
\pgfpathrectangle{\pgfqpoint{0.765000in}{0.660000in}}{\pgfqpoint{4.620000in}{4.620000in}}%
\pgfusepath{clip}%
\pgfsetrectcap%
\pgfsetroundjoin%
\pgfsetlinewidth{1.204500pt}%
\definecolor{currentstroke}{rgb}{1.000000,0.576471,0.309804}%
\pgfsetstrokecolor{currentstroke}%
\pgfsetdash{}{0pt}%
\pgfpathmoveto{\pgfqpoint{4.298302in}{2.705740in}}%
\pgfusepath{stroke}%
\end{pgfscope}%
\begin{pgfscope}%
\pgfpathrectangle{\pgfqpoint{0.765000in}{0.660000in}}{\pgfqpoint{4.620000in}{4.620000in}}%
\pgfusepath{clip}%
\pgfsetbuttcap%
\pgfsetroundjoin%
\definecolor{currentfill}{rgb}{1.000000,0.576471,0.309804}%
\pgfsetfillcolor{currentfill}%
\pgfsetlinewidth{1.003750pt}%
\definecolor{currentstroke}{rgb}{1.000000,0.576471,0.309804}%
\pgfsetstrokecolor{currentstroke}%
\pgfsetdash{}{0pt}%
\pgfsys@defobject{currentmarker}{\pgfqpoint{-0.033333in}{-0.033333in}}{\pgfqpoint{0.033333in}{0.033333in}}{%
\pgfpathmoveto{\pgfqpoint{0.000000in}{-0.033333in}}%
\pgfpathcurveto{\pgfqpoint{0.008840in}{-0.033333in}}{\pgfqpoint{0.017319in}{-0.029821in}}{\pgfqpoint{0.023570in}{-0.023570in}}%
\pgfpathcurveto{\pgfqpoint{0.029821in}{-0.017319in}}{\pgfqpoint{0.033333in}{-0.008840in}}{\pgfqpoint{0.033333in}{0.000000in}}%
\pgfpathcurveto{\pgfqpoint{0.033333in}{0.008840in}}{\pgfqpoint{0.029821in}{0.017319in}}{\pgfqpoint{0.023570in}{0.023570in}}%
\pgfpathcurveto{\pgfqpoint{0.017319in}{0.029821in}}{\pgfqpoint{0.008840in}{0.033333in}}{\pgfqpoint{0.000000in}{0.033333in}}%
\pgfpathcurveto{\pgfqpoint{-0.008840in}{0.033333in}}{\pgfqpoint{-0.017319in}{0.029821in}}{\pgfqpoint{-0.023570in}{0.023570in}}%
\pgfpathcurveto{\pgfqpoint{-0.029821in}{0.017319in}}{\pgfqpoint{-0.033333in}{0.008840in}}{\pgfqpoint{-0.033333in}{0.000000in}}%
\pgfpathcurveto{\pgfqpoint{-0.033333in}{-0.008840in}}{\pgfqpoint{-0.029821in}{-0.017319in}}{\pgfqpoint{-0.023570in}{-0.023570in}}%
\pgfpathcurveto{\pgfqpoint{-0.017319in}{-0.029821in}}{\pgfqpoint{-0.008840in}{-0.033333in}}{\pgfqpoint{0.000000in}{-0.033333in}}%
\pgfpathlineto{\pgfqpoint{0.000000in}{-0.033333in}}%
\pgfpathclose%
\pgfusepath{stroke,fill}%
}%
\begin{pgfscope}%
\pgfsys@transformshift{4.298302in}{2.705740in}%
\pgfsys@useobject{currentmarker}{}%
\end{pgfscope}%
\end{pgfscope}%
\begin{pgfscope}%
\pgfpathrectangle{\pgfqpoint{0.765000in}{0.660000in}}{\pgfqpoint{4.620000in}{4.620000in}}%
\pgfusepath{clip}%
\pgfsetrectcap%
\pgfsetroundjoin%
\pgfsetlinewidth{1.204500pt}%
\definecolor{currentstroke}{rgb}{1.000000,0.576471,0.309804}%
\pgfsetstrokecolor{currentstroke}%
\pgfsetdash{}{0pt}%
\pgfpathmoveto{\pgfqpoint{4.024928in}{2.450185in}}%
\pgfusepath{stroke}%
\end{pgfscope}%
\begin{pgfscope}%
\pgfpathrectangle{\pgfqpoint{0.765000in}{0.660000in}}{\pgfqpoint{4.620000in}{4.620000in}}%
\pgfusepath{clip}%
\pgfsetbuttcap%
\pgfsetroundjoin%
\definecolor{currentfill}{rgb}{1.000000,0.576471,0.309804}%
\pgfsetfillcolor{currentfill}%
\pgfsetlinewidth{1.003750pt}%
\definecolor{currentstroke}{rgb}{1.000000,0.576471,0.309804}%
\pgfsetstrokecolor{currentstroke}%
\pgfsetdash{}{0pt}%
\pgfsys@defobject{currentmarker}{\pgfqpoint{-0.033333in}{-0.033333in}}{\pgfqpoint{0.033333in}{0.033333in}}{%
\pgfpathmoveto{\pgfqpoint{0.000000in}{-0.033333in}}%
\pgfpathcurveto{\pgfqpoint{0.008840in}{-0.033333in}}{\pgfqpoint{0.017319in}{-0.029821in}}{\pgfqpoint{0.023570in}{-0.023570in}}%
\pgfpathcurveto{\pgfqpoint{0.029821in}{-0.017319in}}{\pgfqpoint{0.033333in}{-0.008840in}}{\pgfqpoint{0.033333in}{0.000000in}}%
\pgfpathcurveto{\pgfqpoint{0.033333in}{0.008840in}}{\pgfqpoint{0.029821in}{0.017319in}}{\pgfqpoint{0.023570in}{0.023570in}}%
\pgfpathcurveto{\pgfqpoint{0.017319in}{0.029821in}}{\pgfqpoint{0.008840in}{0.033333in}}{\pgfqpoint{0.000000in}{0.033333in}}%
\pgfpathcurveto{\pgfqpoint{-0.008840in}{0.033333in}}{\pgfqpoint{-0.017319in}{0.029821in}}{\pgfqpoint{-0.023570in}{0.023570in}}%
\pgfpathcurveto{\pgfqpoint{-0.029821in}{0.017319in}}{\pgfqpoint{-0.033333in}{0.008840in}}{\pgfqpoint{-0.033333in}{0.000000in}}%
\pgfpathcurveto{\pgfqpoint{-0.033333in}{-0.008840in}}{\pgfqpoint{-0.029821in}{-0.017319in}}{\pgfqpoint{-0.023570in}{-0.023570in}}%
\pgfpathcurveto{\pgfqpoint{-0.017319in}{-0.029821in}}{\pgfqpoint{-0.008840in}{-0.033333in}}{\pgfqpoint{0.000000in}{-0.033333in}}%
\pgfpathlineto{\pgfqpoint{0.000000in}{-0.033333in}}%
\pgfpathclose%
\pgfusepath{stroke,fill}%
}%
\begin{pgfscope}%
\pgfsys@transformshift{4.024928in}{2.450185in}%
\pgfsys@useobject{currentmarker}{}%
\end{pgfscope}%
\end{pgfscope}%
\begin{pgfscope}%
\pgfpathrectangle{\pgfqpoint{0.765000in}{0.660000in}}{\pgfqpoint{4.620000in}{4.620000in}}%
\pgfusepath{clip}%
\pgfsetrectcap%
\pgfsetroundjoin%
\pgfsetlinewidth{1.204500pt}%
\definecolor{currentstroke}{rgb}{1.000000,0.576471,0.309804}%
\pgfsetstrokecolor{currentstroke}%
\pgfsetdash{}{0pt}%
\pgfpathmoveto{\pgfqpoint{2.095736in}{2.808077in}}%
\pgfusepath{stroke}%
\end{pgfscope}%
\begin{pgfscope}%
\pgfpathrectangle{\pgfqpoint{0.765000in}{0.660000in}}{\pgfqpoint{4.620000in}{4.620000in}}%
\pgfusepath{clip}%
\pgfsetbuttcap%
\pgfsetroundjoin%
\definecolor{currentfill}{rgb}{1.000000,0.576471,0.309804}%
\pgfsetfillcolor{currentfill}%
\pgfsetlinewidth{1.003750pt}%
\definecolor{currentstroke}{rgb}{1.000000,0.576471,0.309804}%
\pgfsetstrokecolor{currentstroke}%
\pgfsetdash{}{0pt}%
\pgfsys@defobject{currentmarker}{\pgfqpoint{-0.033333in}{-0.033333in}}{\pgfqpoint{0.033333in}{0.033333in}}{%
\pgfpathmoveto{\pgfqpoint{0.000000in}{-0.033333in}}%
\pgfpathcurveto{\pgfqpoint{0.008840in}{-0.033333in}}{\pgfqpoint{0.017319in}{-0.029821in}}{\pgfqpoint{0.023570in}{-0.023570in}}%
\pgfpathcurveto{\pgfqpoint{0.029821in}{-0.017319in}}{\pgfqpoint{0.033333in}{-0.008840in}}{\pgfqpoint{0.033333in}{0.000000in}}%
\pgfpathcurveto{\pgfqpoint{0.033333in}{0.008840in}}{\pgfqpoint{0.029821in}{0.017319in}}{\pgfqpoint{0.023570in}{0.023570in}}%
\pgfpathcurveto{\pgfqpoint{0.017319in}{0.029821in}}{\pgfqpoint{0.008840in}{0.033333in}}{\pgfqpoint{0.000000in}{0.033333in}}%
\pgfpathcurveto{\pgfqpoint{-0.008840in}{0.033333in}}{\pgfqpoint{-0.017319in}{0.029821in}}{\pgfqpoint{-0.023570in}{0.023570in}}%
\pgfpathcurveto{\pgfqpoint{-0.029821in}{0.017319in}}{\pgfqpoint{-0.033333in}{0.008840in}}{\pgfqpoint{-0.033333in}{0.000000in}}%
\pgfpathcurveto{\pgfqpoint{-0.033333in}{-0.008840in}}{\pgfqpoint{-0.029821in}{-0.017319in}}{\pgfqpoint{-0.023570in}{-0.023570in}}%
\pgfpathcurveto{\pgfqpoint{-0.017319in}{-0.029821in}}{\pgfqpoint{-0.008840in}{-0.033333in}}{\pgfqpoint{0.000000in}{-0.033333in}}%
\pgfpathlineto{\pgfqpoint{0.000000in}{-0.033333in}}%
\pgfpathclose%
\pgfusepath{stroke,fill}%
}%
\begin{pgfscope}%
\pgfsys@transformshift{2.095736in}{2.808077in}%
\pgfsys@useobject{currentmarker}{}%
\end{pgfscope}%
\end{pgfscope}%
\begin{pgfscope}%
\pgfpathrectangle{\pgfqpoint{0.765000in}{0.660000in}}{\pgfqpoint{4.620000in}{4.620000in}}%
\pgfusepath{clip}%
\pgfsetrectcap%
\pgfsetroundjoin%
\pgfsetlinewidth{1.204500pt}%
\definecolor{currentstroke}{rgb}{1.000000,0.576471,0.309804}%
\pgfsetstrokecolor{currentstroke}%
\pgfsetdash{}{0pt}%
\pgfpathmoveto{\pgfqpoint{3.252520in}{3.963575in}}%
\pgfusepath{stroke}%
\end{pgfscope}%
\begin{pgfscope}%
\pgfpathrectangle{\pgfqpoint{0.765000in}{0.660000in}}{\pgfqpoint{4.620000in}{4.620000in}}%
\pgfusepath{clip}%
\pgfsetbuttcap%
\pgfsetroundjoin%
\definecolor{currentfill}{rgb}{1.000000,0.576471,0.309804}%
\pgfsetfillcolor{currentfill}%
\pgfsetlinewidth{1.003750pt}%
\definecolor{currentstroke}{rgb}{1.000000,0.576471,0.309804}%
\pgfsetstrokecolor{currentstroke}%
\pgfsetdash{}{0pt}%
\pgfsys@defobject{currentmarker}{\pgfqpoint{-0.033333in}{-0.033333in}}{\pgfqpoint{0.033333in}{0.033333in}}{%
\pgfpathmoveto{\pgfqpoint{0.000000in}{-0.033333in}}%
\pgfpathcurveto{\pgfqpoint{0.008840in}{-0.033333in}}{\pgfqpoint{0.017319in}{-0.029821in}}{\pgfqpoint{0.023570in}{-0.023570in}}%
\pgfpathcurveto{\pgfqpoint{0.029821in}{-0.017319in}}{\pgfqpoint{0.033333in}{-0.008840in}}{\pgfqpoint{0.033333in}{0.000000in}}%
\pgfpathcurveto{\pgfqpoint{0.033333in}{0.008840in}}{\pgfqpoint{0.029821in}{0.017319in}}{\pgfqpoint{0.023570in}{0.023570in}}%
\pgfpathcurveto{\pgfqpoint{0.017319in}{0.029821in}}{\pgfqpoint{0.008840in}{0.033333in}}{\pgfqpoint{0.000000in}{0.033333in}}%
\pgfpathcurveto{\pgfqpoint{-0.008840in}{0.033333in}}{\pgfqpoint{-0.017319in}{0.029821in}}{\pgfqpoint{-0.023570in}{0.023570in}}%
\pgfpathcurveto{\pgfqpoint{-0.029821in}{0.017319in}}{\pgfqpoint{-0.033333in}{0.008840in}}{\pgfqpoint{-0.033333in}{0.000000in}}%
\pgfpathcurveto{\pgfqpoint{-0.033333in}{-0.008840in}}{\pgfqpoint{-0.029821in}{-0.017319in}}{\pgfqpoint{-0.023570in}{-0.023570in}}%
\pgfpathcurveto{\pgfqpoint{-0.017319in}{-0.029821in}}{\pgfqpoint{-0.008840in}{-0.033333in}}{\pgfqpoint{0.000000in}{-0.033333in}}%
\pgfpathlineto{\pgfqpoint{0.000000in}{-0.033333in}}%
\pgfpathclose%
\pgfusepath{stroke,fill}%
}%
\begin{pgfscope}%
\pgfsys@transformshift{3.252520in}{3.963575in}%
\pgfsys@useobject{currentmarker}{}%
\end{pgfscope}%
\end{pgfscope}%
\begin{pgfscope}%
\pgfpathrectangle{\pgfqpoint{0.765000in}{0.660000in}}{\pgfqpoint{4.620000in}{4.620000in}}%
\pgfusepath{clip}%
\pgfsetrectcap%
\pgfsetroundjoin%
\pgfsetlinewidth{1.204500pt}%
\definecolor{currentstroke}{rgb}{1.000000,0.576471,0.309804}%
\pgfsetstrokecolor{currentstroke}%
\pgfsetdash{}{0pt}%
\pgfpathmoveto{\pgfqpoint{2.125219in}{3.625990in}}%
\pgfusepath{stroke}%
\end{pgfscope}%
\begin{pgfscope}%
\pgfpathrectangle{\pgfqpoint{0.765000in}{0.660000in}}{\pgfqpoint{4.620000in}{4.620000in}}%
\pgfusepath{clip}%
\pgfsetbuttcap%
\pgfsetroundjoin%
\definecolor{currentfill}{rgb}{1.000000,0.576471,0.309804}%
\pgfsetfillcolor{currentfill}%
\pgfsetlinewidth{1.003750pt}%
\definecolor{currentstroke}{rgb}{1.000000,0.576471,0.309804}%
\pgfsetstrokecolor{currentstroke}%
\pgfsetdash{}{0pt}%
\pgfsys@defobject{currentmarker}{\pgfqpoint{-0.033333in}{-0.033333in}}{\pgfqpoint{0.033333in}{0.033333in}}{%
\pgfpathmoveto{\pgfqpoint{0.000000in}{-0.033333in}}%
\pgfpathcurveto{\pgfqpoint{0.008840in}{-0.033333in}}{\pgfqpoint{0.017319in}{-0.029821in}}{\pgfqpoint{0.023570in}{-0.023570in}}%
\pgfpathcurveto{\pgfqpoint{0.029821in}{-0.017319in}}{\pgfqpoint{0.033333in}{-0.008840in}}{\pgfqpoint{0.033333in}{0.000000in}}%
\pgfpathcurveto{\pgfqpoint{0.033333in}{0.008840in}}{\pgfqpoint{0.029821in}{0.017319in}}{\pgfqpoint{0.023570in}{0.023570in}}%
\pgfpathcurveto{\pgfqpoint{0.017319in}{0.029821in}}{\pgfqpoint{0.008840in}{0.033333in}}{\pgfqpoint{0.000000in}{0.033333in}}%
\pgfpathcurveto{\pgfqpoint{-0.008840in}{0.033333in}}{\pgfqpoint{-0.017319in}{0.029821in}}{\pgfqpoint{-0.023570in}{0.023570in}}%
\pgfpathcurveto{\pgfqpoint{-0.029821in}{0.017319in}}{\pgfqpoint{-0.033333in}{0.008840in}}{\pgfqpoint{-0.033333in}{0.000000in}}%
\pgfpathcurveto{\pgfqpoint{-0.033333in}{-0.008840in}}{\pgfqpoint{-0.029821in}{-0.017319in}}{\pgfqpoint{-0.023570in}{-0.023570in}}%
\pgfpathcurveto{\pgfqpoint{-0.017319in}{-0.029821in}}{\pgfqpoint{-0.008840in}{-0.033333in}}{\pgfqpoint{0.000000in}{-0.033333in}}%
\pgfpathlineto{\pgfqpoint{0.000000in}{-0.033333in}}%
\pgfpathclose%
\pgfusepath{stroke,fill}%
}%
\begin{pgfscope}%
\pgfsys@transformshift{2.125219in}{3.625990in}%
\pgfsys@useobject{currentmarker}{}%
\end{pgfscope}%
\end{pgfscope}%
\begin{pgfscope}%
\pgfpathrectangle{\pgfqpoint{0.765000in}{0.660000in}}{\pgfqpoint{4.620000in}{4.620000in}}%
\pgfusepath{clip}%
\pgfsetrectcap%
\pgfsetroundjoin%
\pgfsetlinewidth{1.204500pt}%
\definecolor{currentstroke}{rgb}{1.000000,0.576471,0.309804}%
\pgfsetstrokecolor{currentstroke}%
\pgfsetdash{}{0pt}%
\pgfpathmoveto{\pgfqpoint{3.081498in}{2.055640in}}%
\pgfusepath{stroke}%
\end{pgfscope}%
\begin{pgfscope}%
\pgfpathrectangle{\pgfqpoint{0.765000in}{0.660000in}}{\pgfqpoint{4.620000in}{4.620000in}}%
\pgfusepath{clip}%
\pgfsetbuttcap%
\pgfsetroundjoin%
\definecolor{currentfill}{rgb}{1.000000,0.576471,0.309804}%
\pgfsetfillcolor{currentfill}%
\pgfsetlinewidth{1.003750pt}%
\definecolor{currentstroke}{rgb}{1.000000,0.576471,0.309804}%
\pgfsetstrokecolor{currentstroke}%
\pgfsetdash{}{0pt}%
\pgfsys@defobject{currentmarker}{\pgfqpoint{-0.033333in}{-0.033333in}}{\pgfqpoint{0.033333in}{0.033333in}}{%
\pgfpathmoveto{\pgfqpoint{0.000000in}{-0.033333in}}%
\pgfpathcurveto{\pgfqpoint{0.008840in}{-0.033333in}}{\pgfqpoint{0.017319in}{-0.029821in}}{\pgfqpoint{0.023570in}{-0.023570in}}%
\pgfpathcurveto{\pgfqpoint{0.029821in}{-0.017319in}}{\pgfqpoint{0.033333in}{-0.008840in}}{\pgfqpoint{0.033333in}{0.000000in}}%
\pgfpathcurveto{\pgfqpoint{0.033333in}{0.008840in}}{\pgfqpoint{0.029821in}{0.017319in}}{\pgfqpoint{0.023570in}{0.023570in}}%
\pgfpathcurveto{\pgfqpoint{0.017319in}{0.029821in}}{\pgfqpoint{0.008840in}{0.033333in}}{\pgfqpoint{0.000000in}{0.033333in}}%
\pgfpathcurveto{\pgfqpoint{-0.008840in}{0.033333in}}{\pgfqpoint{-0.017319in}{0.029821in}}{\pgfqpoint{-0.023570in}{0.023570in}}%
\pgfpathcurveto{\pgfqpoint{-0.029821in}{0.017319in}}{\pgfqpoint{-0.033333in}{0.008840in}}{\pgfqpoint{-0.033333in}{0.000000in}}%
\pgfpathcurveto{\pgfqpoint{-0.033333in}{-0.008840in}}{\pgfqpoint{-0.029821in}{-0.017319in}}{\pgfqpoint{-0.023570in}{-0.023570in}}%
\pgfpathcurveto{\pgfqpoint{-0.017319in}{-0.029821in}}{\pgfqpoint{-0.008840in}{-0.033333in}}{\pgfqpoint{0.000000in}{-0.033333in}}%
\pgfpathlineto{\pgfqpoint{0.000000in}{-0.033333in}}%
\pgfpathclose%
\pgfusepath{stroke,fill}%
}%
\begin{pgfscope}%
\pgfsys@transformshift{3.081498in}{2.055640in}%
\pgfsys@useobject{currentmarker}{}%
\end{pgfscope}%
\end{pgfscope}%
\begin{pgfscope}%
\pgfpathrectangle{\pgfqpoint{0.765000in}{0.660000in}}{\pgfqpoint{4.620000in}{4.620000in}}%
\pgfusepath{clip}%
\pgfsetrectcap%
\pgfsetroundjoin%
\pgfsetlinewidth{1.204500pt}%
\definecolor{currentstroke}{rgb}{1.000000,0.576471,0.309804}%
\pgfsetstrokecolor{currentstroke}%
\pgfsetdash{}{0pt}%
\pgfpathmoveto{\pgfqpoint{4.086145in}{3.908950in}}%
\pgfusepath{stroke}%
\end{pgfscope}%
\begin{pgfscope}%
\pgfpathrectangle{\pgfqpoint{0.765000in}{0.660000in}}{\pgfqpoint{4.620000in}{4.620000in}}%
\pgfusepath{clip}%
\pgfsetbuttcap%
\pgfsetroundjoin%
\definecolor{currentfill}{rgb}{1.000000,0.576471,0.309804}%
\pgfsetfillcolor{currentfill}%
\pgfsetlinewidth{1.003750pt}%
\definecolor{currentstroke}{rgb}{1.000000,0.576471,0.309804}%
\pgfsetstrokecolor{currentstroke}%
\pgfsetdash{}{0pt}%
\pgfsys@defobject{currentmarker}{\pgfqpoint{-0.033333in}{-0.033333in}}{\pgfqpoint{0.033333in}{0.033333in}}{%
\pgfpathmoveto{\pgfqpoint{0.000000in}{-0.033333in}}%
\pgfpathcurveto{\pgfqpoint{0.008840in}{-0.033333in}}{\pgfqpoint{0.017319in}{-0.029821in}}{\pgfqpoint{0.023570in}{-0.023570in}}%
\pgfpathcurveto{\pgfqpoint{0.029821in}{-0.017319in}}{\pgfqpoint{0.033333in}{-0.008840in}}{\pgfqpoint{0.033333in}{0.000000in}}%
\pgfpathcurveto{\pgfqpoint{0.033333in}{0.008840in}}{\pgfqpoint{0.029821in}{0.017319in}}{\pgfqpoint{0.023570in}{0.023570in}}%
\pgfpathcurveto{\pgfqpoint{0.017319in}{0.029821in}}{\pgfqpoint{0.008840in}{0.033333in}}{\pgfqpoint{0.000000in}{0.033333in}}%
\pgfpathcurveto{\pgfqpoint{-0.008840in}{0.033333in}}{\pgfqpoint{-0.017319in}{0.029821in}}{\pgfqpoint{-0.023570in}{0.023570in}}%
\pgfpathcurveto{\pgfqpoint{-0.029821in}{0.017319in}}{\pgfqpoint{-0.033333in}{0.008840in}}{\pgfqpoint{-0.033333in}{0.000000in}}%
\pgfpathcurveto{\pgfqpoint{-0.033333in}{-0.008840in}}{\pgfqpoint{-0.029821in}{-0.017319in}}{\pgfqpoint{-0.023570in}{-0.023570in}}%
\pgfpathcurveto{\pgfqpoint{-0.017319in}{-0.029821in}}{\pgfqpoint{-0.008840in}{-0.033333in}}{\pgfqpoint{0.000000in}{-0.033333in}}%
\pgfpathlineto{\pgfqpoint{0.000000in}{-0.033333in}}%
\pgfpathclose%
\pgfusepath{stroke,fill}%
}%
\begin{pgfscope}%
\pgfsys@transformshift{4.086145in}{3.908950in}%
\pgfsys@useobject{currentmarker}{}%
\end{pgfscope}%
\end{pgfscope}%
\begin{pgfscope}%
\pgfpathrectangle{\pgfqpoint{0.765000in}{0.660000in}}{\pgfqpoint{4.620000in}{4.620000in}}%
\pgfusepath{clip}%
\pgfsetrectcap%
\pgfsetroundjoin%
\pgfsetlinewidth{1.204500pt}%
\definecolor{currentstroke}{rgb}{1.000000,0.576471,0.309804}%
\pgfsetstrokecolor{currentstroke}%
\pgfsetdash{}{0pt}%
\pgfpathmoveto{\pgfqpoint{3.608603in}{3.981476in}}%
\pgfusepath{stroke}%
\end{pgfscope}%
\begin{pgfscope}%
\pgfpathrectangle{\pgfqpoint{0.765000in}{0.660000in}}{\pgfqpoint{4.620000in}{4.620000in}}%
\pgfusepath{clip}%
\pgfsetbuttcap%
\pgfsetroundjoin%
\definecolor{currentfill}{rgb}{1.000000,0.576471,0.309804}%
\pgfsetfillcolor{currentfill}%
\pgfsetlinewidth{1.003750pt}%
\definecolor{currentstroke}{rgb}{1.000000,0.576471,0.309804}%
\pgfsetstrokecolor{currentstroke}%
\pgfsetdash{}{0pt}%
\pgfsys@defobject{currentmarker}{\pgfqpoint{-0.033333in}{-0.033333in}}{\pgfqpoint{0.033333in}{0.033333in}}{%
\pgfpathmoveto{\pgfqpoint{0.000000in}{-0.033333in}}%
\pgfpathcurveto{\pgfqpoint{0.008840in}{-0.033333in}}{\pgfqpoint{0.017319in}{-0.029821in}}{\pgfqpoint{0.023570in}{-0.023570in}}%
\pgfpathcurveto{\pgfqpoint{0.029821in}{-0.017319in}}{\pgfqpoint{0.033333in}{-0.008840in}}{\pgfqpoint{0.033333in}{0.000000in}}%
\pgfpathcurveto{\pgfqpoint{0.033333in}{0.008840in}}{\pgfqpoint{0.029821in}{0.017319in}}{\pgfqpoint{0.023570in}{0.023570in}}%
\pgfpathcurveto{\pgfqpoint{0.017319in}{0.029821in}}{\pgfqpoint{0.008840in}{0.033333in}}{\pgfqpoint{0.000000in}{0.033333in}}%
\pgfpathcurveto{\pgfqpoint{-0.008840in}{0.033333in}}{\pgfqpoint{-0.017319in}{0.029821in}}{\pgfqpoint{-0.023570in}{0.023570in}}%
\pgfpathcurveto{\pgfqpoint{-0.029821in}{0.017319in}}{\pgfqpoint{-0.033333in}{0.008840in}}{\pgfqpoint{-0.033333in}{0.000000in}}%
\pgfpathcurveto{\pgfqpoint{-0.033333in}{-0.008840in}}{\pgfqpoint{-0.029821in}{-0.017319in}}{\pgfqpoint{-0.023570in}{-0.023570in}}%
\pgfpathcurveto{\pgfqpoint{-0.017319in}{-0.029821in}}{\pgfqpoint{-0.008840in}{-0.033333in}}{\pgfqpoint{0.000000in}{-0.033333in}}%
\pgfpathlineto{\pgfqpoint{0.000000in}{-0.033333in}}%
\pgfpathclose%
\pgfusepath{stroke,fill}%
}%
\begin{pgfscope}%
\pgfsys@transformshift{3.608603in}{3.981476in}%
\pgfsys@useobject{currentmarker}{}%
\end{pgfscope}%
\end{pgfscope}%
\begin{pgfscope}%
\pgfpathrectangle{\pgfqpoint{0.765000in}{0.660000in}}{\pgfqpoint{4.620000in}{4.620000in}}%
\pgfusepath{clip}%
\pgfsetrectcap%
\pgfsetroundjoin%
\pgfsetlinewidth{1.204500pt}%
\definecolor{currentstroke}{rgb}{1.000000,0.576471,0.309804}%
\pgfsetstrokecolor{currentstroke}%
\pgfsetdash{}{0pt}%
\pgfpathmoveto{\pgfqpoint{3.891614in}{1.195241in}}%
\pgfusepath{stroke}%
\end{pgfscope}%
\begin{pgfscope}%
\pgfpathrectangle{\pgfqpoint{0.765000in}{0.660000in}}{\pgfqpoint{4.620000in}{4.620000in}}%
\pgfusepath{clip}%
\pgfsetbuttcap%
\pgfsetroundjoin%
\definecolor{currentfill}{rgb}{1.000000,0.576471,0.309804}%
\pgfsetfillcolor{currentfill}%
\pgfsetlinewidth{1.003750pt}%
\definecolor{currentstroke}{rgb}{1.000000,0.576471,0.309804}%
\pgfsetstrokecolor{currentstroke}%
\pgfsetdash{}{0pt}%
\pgfsys@defobject{currentmarker}{\pgfqpoint{-0.033333in}{-0.033333in}}{\pgfqpoint{0.033333in}{0.033333in}}{%
\pgfpathmoveto{\pgfqpoint{0.000000in}{-0.033333in}}%
\pgfpathcurveto{\pgfqpoint{0.008840in}{-0.033333in}}{\pgfqpoint{0.017319in}{-0.029821in}}{\pgfqpoint{0.023570in}{-0.023570in}}%
\pgfpathcurveto{\pgfqpoint{0.029821in}{-0.017319in}}{\pgfqpoint{0.033333in}{-0.008840in}}{\pgfqpoint{0.033333in}{0.000000in}}%
\pgfpathcurveto{\pgfqpoint{0.033333in}{0.008840in}}{\pgfqpoint{0.029821in}{0.017319in}}{\pgfqpoint{0.023570in}{0.023570in}}%
\pgfpathcurveto{\pgfqpoint{0.017319in}{0.029821in}}{\pgfqpoint{0.008840in}{0.033333in}}{\pgfqpoint{0.000000in}{0.033333in}}%
\pgfpathcurveto{\pgfqpoint{-0.008840in}{0.033333in}}{\pgfqpoint{-0.017319in}{0.029821in}}{\pgfqpoint{-0.023570in}{0.023570in}}%
\pgfpathcurveto{\pgfqpoint{-0.029821in}{0.017319in}}{\pgfqpoint{-0.033333in}{0.008840in}}{\pgfqpoint{-0.033333in}{0.000000in}}%
\pgfpathcurveto{\pgfqpoint{-0.033333in}{-0.008840in}}{\pgfqpoint{-0.029821in}{-0.017319in}}{\pgfqpoint{-0.023570in}{-0.023570in}}%
\pgfpathcurveto{\pgfqpoint{-0.017319in}{-0.029821in}}{\pgfqpoint{-0.008840in}{-0.033333in}}{\pgfqpoint{0.000000in}{-0.033333in}}%
\pgfpathlineto{\pgfqpoint{0.000000in}{-0.033333in}}%
\pgfpathclose%
\pgfusepath{stroke,fill}%
}%
\begin{pgfscope}%
\pgfsys@transformshift{3.891614in}{1.195241in}%
\pgfsys@useobject{currentmarker}{}%
\end{pgfscope}%
\end{pgfscope}%
\begin{pgfscope}%
\pgfpathrectangle{\pgfqpoint{0.765000in}{0.660000in}}{\pgfqpoint{4.620000in}{4.620000in}}%
\pgfusepath{clip}%
\pgfsetrectcap%
\pgfsetroundjoin%
\pgfsetlinewidth{1.204500pt}%
\definecolor{currentstroke}{rgb}{1.000000,0.576471,0.309804}%
\pgfsetstrokecolor{currentstroke}%
\pgfsetdash{}{0pt}%
\pgfpathmoveto{\pgfqpoint{2.138241in}{3.457870in}}%
\pgfusepath{stroke}%
\end{pgfscope}%
\begin{pgfscope}%
\pgfpathrectangle{\pgfqpoint{0.765000in}{0.660000in}}{\pgfqpoint{4.620000in}{4.620000in}}%
\pgfusepath{clip}%
\pgfsetbuttcap%
\pgfsetroundjoin%
\definecolor{currentfill}{rgb}{1.000000,0.576471,0.309804}%
\pgfsetfillcolor{currentfill}%
\pgfsetlinewidth{1.003750pt}%
\definecolor{currentstroke}{rgb}{1.000000,0.576471,0.309804}%
\pgfsetstrokecolor{currentstroke}%
\pgfsetdash{}{0pt}%
\pgfsys@defobject{currentmarker}{\pgfqpoint{-0.033333in}{-0.033333in}}{\pgfqpoint{0.033333in}{0.033333in}}{%
\pgfpathmoveto{\pgfqpoint{0.000000in}{-0.033333in}}%
\pgfpathcurveto{\pgfqpoint{0.008840in}{-0.033333in}}{\pgfqpoint{0.017319in}{-0.029821in}}{\pgfqpoint{0.023570in}{-0.023570in}}%
\pgfpathcurveto{\pgfqpoint{0.029821in}{-0.017319in}}{\pgfqpoint{0.033333in}{-0.008840in}}{\pgfqpoint{0.033333in}{0.000000in}}%
\pgfpathcurveto{\pgfqpoint{0.033333in}{0.008840in}}{\pgfqpoint{0.029821in}{0.017319in}}{\pgfqpoint{0.023570in}{0.023570in}}%
\pgfpathcurveto{\pgfqpoint{0.017319in}{0.029821in}}{\pgfqpoint{0.008840in}{0.033333in}}{\pgfqpoint{0.000000in}{0.033333in}}%
\pgfpathcurveto{\pgfqpoint{-0.008840in}{0.033333in}}{\pgfqpoint{-0.017319in}{0.029821in}}{\pgfqpoint{-0.023570in}{0.023570in}}%
\pgfpathcurveto{\pgfqpoint{-0.029821in}{0.017319in}}{\pgfqpoint{-0.033333in}{0.008840in}}{\pgfqpoint{-0.033333in}{0.000000in}}%
\pgfpathcurveto{\pgfqpoint{-0.033333in}{-0.008840in}}{\pgfqpoint{-0.029821in}{-0.017319in}}{\pgfqpoint{-0.023570in}{-0.023570in}}%
\pgfpathcurveto{\pgfqpoint{-0.017319in}{-0.029821in}}{\pgfqpoint{-0.008840in}{-0.033333in}}{\pgfqpoint{0.000000in}{-0.033333in}}%
\pgfpathlineto{\pgfqpoint{0.000000in}{-0.033333in}}%
\pgfpathclose%
\pgfusepath{stroke,fill}%
}%
\begin{pgfscope}%
\pgfsys@transformshift{2.138241in}{3.457870in}%
\pgfsys@useobject{currentmarker}{}%
\end{pgfscope}%
\end{pgfscope}%
\begin{pgfscope}%
\pgfpathrectangle{\pgfqpoint{0.765000in}{0.660000in}}{\pgfqpoint{4.620000in}{4.620000in}}%
\pgfusepath{clip}%
\pgfsetrectcap%
\pgfsetroundjoin%
\pgfsetlinewidth{1.204500pt}%
\definecolor{currentstroke}{rgb}{1.000000,0.576471,0.309804}%
\pgfsetstrokecolor{currentstroke}%
\pgfsetdash{}{0pt}%
\pgfpathmoveto{\pgfqpoint{3.663049in}{2.267237in}}%
\pgfusepath{stroke}%
\end{pgfscope}%
\begin{pgfscope}%
\pgfpathrectangle{\pgfqpoint{0.765000in}{0.660000in}}{\pgfqpoint{4.620000in}{4.620000in}}%
\pgfusepath{clip}%
\pgfsetbuttcap%
\pgfsetroundjoin%
\definecolor{currentfill}{rgb}{1.000000,0.576471,0.309804}%
\pgfsetfillcolor{currentfill}%
\pgfsetlinewidth{1.003750pt}%
\definecolor{currentstroke}{rgb}{1.000000,0.576471,0.309804}%
\pgfsetstrokecolor{currentstroke}%
\pgfsetdash{}{0pt}%
\pgfsys@defobject{currentmarker}{\pgfqpoint{-0.033333in}{-0.033333in}}{\pgfqpoint{0.033333in}{0.033333in}}{%
\pgfpathmoveto{\pgfqpoint{0.000000in}{-0.033333in}}%
\pgfpathcurveto{\pgfqpoint{0.008840in}{-0.033333in}}{\pgfqpoint{0.017319in}{-0.029821in}}{\pgfqpoint{0.023570in}{-0.023570in}}%
\pgfpathcurveto{\pgfqpoint{0.029821in}{-0.017319in}}{\pgfqpoint{0.033333in}{-0.008840in}}{\pgfqpoint{0.033333in}{0.000000in}}%
\pgfpathcurveto{\pgfqpoint{0.033333in}{0.008840in}}{\pgfqpoint{0.029821in}{0.017319in}}{\pgfqpoint{0.023570in}{0.023570in}}%
\pgfpathcurveto{\pgfqpoint{0.017319in}{0.029821in}}{\pgfqpoint{0.008840in}{0.033333in}}{\pgfqpoint{0.000000in}{0.033333in}}%
\pgfpathcurveto{\pgfqpoint{-0.008840in}{0.033333in}}{\pgfqpoint{-0.017319in}{0.029821in}}{\pgfqpoint{-0.023570in}{0.023570in}}%
\pgfpathcurveto{\pgfqpoint{-0.029821in}{0.017319in}}{\pgfqpoint{-0.033333in}{0.008840in}}{\pgfqpoint{-0.033333in}{0.000000in}}%
\pgfpathcurveto{\pgfqpoint{-0.033333in}{-0.008840in}}{\pgfqpoint{-0.029821in}{-0.017319in}}{\pgfqpoint{-0.023570in}{-0.023570in}}%
\pgfpathcurveto{\pgfqpoint{-0.017319in}{-0.029821in}}{\pgfqpoint{-0.008840in}{-0.033333in}}{\pgfqpoint{0.000000in}{-0.033333in}}%
\pgfpathlineto{\pgfqpoint{0.000000in}{-0.033333in}}%
\pgfpathclose%
\pgfusepath{stroke,fill}%
}%
\begin{pgfscope}%
\pgfsys@transformshift{3.663049in}{2.267237in}%
\pgfsys@useobject{currentmarker}{}%
\end{pgfscope}%
\end{pgfscope}%
\begin{pgfscope}%
\pgfpathrectangle{\pgfqpoint{0.765000in}{0.660000in}}{\pgfqpoint{4.620000in}{4.620000in}}%
\pgfusepath{clip}%
\pgfsetrectcap%
\pgfsetroundjoin%
\pgfsetlinewidth{1.204500pt}%
\definecolor{currentstroke}{rgb}{1.000000,0.576471,0.309804}%
\pgfsetstrokecolor{currentstroke}%
\pgfsetdash{}{0pt}%
\pgfpathmoveto{\pgfqpoint{3.275168in}{3.921300in}}%
\pgfusepath{stroke}%
\end{pgfscope}%
\begin{pgfscope}%
\pgfpathrectangle{\pgfqpoint{0.765000in}{0.660000in}}{\pgfqpoint{4.620000in}{4.620000in}}%
\pgfusepath{clip}%
\pgfsetbuttcap%
\pgfsetroundjoin%
\definecolor{currentfill}{rgb}{1.000000,0.576471,0.309804}%
\pgfsetfillcolor{currentfill}%
\pgfsetlinewidth{1.003750pt}%
\definecolor{currentstroke}{rgb}{1.000000,0.576471,0.309804}%
\pgfsetstrokecolor{currentstroke}%
\pgfsetdash{}{0pt}%
\pgfsys@defobject{currentmarker}{\pgfqpoint{-0.033333in}{-0.033333in}}{\pgfqpoint{0.033333in}{0.033333in}}{%
\pgfpathmoveto{\pgfqpoint{0.000000in}{-0.033333in}}%
\pgfpathcurveto{\pgfqpoint{0.008840in}{-0.033333in}}{\pgfqpoint{0.017319in}{-0.029821in}}{\pgfqpoint{0.023570in}{-0.023570in}}%
\pgfpathcurveto{\pgfqpoint{0.029821in}{-0.017319in}}{\pgfqpoint{0.033333in}{-0.008840in}}{\pgfqpoint{0.033333in}{0.000000in}}%
\pgfpathcurveto{\pgfqpoint{0.033333in}{0.008840in}}{\pgfqpoint{0.029821in}{0.017319in}}{\pgfqpoint{0.023570in}{0.023570in}}%
\pgfpathcurveto{\pgfqpoint{0.017319in}{0.029821in}}{\pgfqpoint{0.008840in}{0.033333in}}{\pgfqpoint{0.000000in}{0.033333in}}%
\pgfpathcurveto{\pgfqpoint{-0.008840in}{0.033333in}}{\pgfqpoint{-0.017319in}{0.029821in}}{\pgfqpoint{-0.023570in}{0.023570in}}%
\pgfpathcurveto{\pgfqpoint{-0.029821in}{0.017319in}}{\pgfqpoint{-0.033333in}{0.008840in}}{\pgfqpoint{-0.033333in}{0.000000in}}%
\pgfpathcurveto{\pgfqpoint{-0.033333in}{-0.008840in}}{\pgfqpoint{-0.029821in}{-0.017319in}}{\pgfqpoint{-0.023570in}{-0.023570in}}%
\pgfpathcurveto{\pgfqpoint{-0.017319in}{-0.029821in}}{\pgfqpoint{-0.008840in}{-0.033333in}}{\pgfqpoint{0.000000in}{-0.033333in}}%
\pgfpathlineto{\pgfqpoint{0.000000in}{-0.033333in}}%
\pgfpathclose%
\pgfusepath{stroke,fill}%
}%
\begin{pgfscope}%
\pgfsys@transformshift{3.275168in}{3.921300in}%
\pgfsys@useobject{currentmarker}{}%
\end{pgfscope}%
\end{pgfscope}%
\begin{pgfscope}%
\pgfpathrectangle{\pgfqpoint{0.765000in}{0.660000in}}{\pgfqpoint{4.620000in}{4.620000in}}%
\pgfusepath{clip}%
\pgfsetrectcap%
\pgfsetroundjoin%
\pgfsetlinewidth{1.204500pt}%
\definecolor{currentstroke}{rgb}{1.000000,0.576471,0.309804}%
\pgfsetstrokecolor{currentstroke}%
\pgfsetdash{}{0pt}%
\pgfpathmoveto{\pgfqpoint{3.544302in}{2.377750in}}%
\pgfusepath{stroke}%
\end{pgfscope}%
\begin{pgfscope}%
\pgfpathrectangle{\pgfqpoint{0.765000in}{0.660000in}}{\pgfqpoint{4.620000in}{4.620000in}}%
\pgfusepath{clip}%
\pgfsetbuttcap%
\pgfsetroundjoin%
\definecolor{currentfill}{rgb}{1.000000,0.576471,0.309804}%
\pgfsetfillcolor{currentfill}%
\pgfsetlinewidth{1.003750pt}%
\definecolor{currentstroke}{rgb}{1.000000,0.576471,0.309804}%
\pgfsetstrokecolor{currentstroke}%
\pgfsetdash{}{0pt}%
\pgfsys@defobject{currentmarker}{\pgfqpoint{-0.033333in}{-0.033333in}}{\pgfqpoint{0.033333in}{0.033333in}}{%
\pgfpathmoveto{\pgfqpoint{0.000000in}{-0.033333in}}%
\pgfpathcurveto{\pgfqpoint{0.008840in}{-0.033333in}}{\pgfqpoint{0.017319in}{-0.029821in}}{\pgfqpoint{0.023570in}{-0.023570in}}%
\pgfpathcurveto{\pgfqpoint{0.029821in}{-0.017319in}}{\pgfqpoint{0.033333in}{-0.008840in}}{\pgfqpoint{0.033333in}{0.000000in}}%
\pgfpathcurveto{\pgfqpoint{0.033333in}{0.008840in}}{\pgfqpoint{0.029821in}{0.017319in}}{\pgfqpoint{0.023570in}{0.023570in}}%
\pgfpathcurveto{\pgfqpoint{0.017319in}{0.029821in}}{\pgfqpoint{0.008840in}{0.033333in}}{\pgfqpoint{0.000000in}{0.033333in}}%
\pgfpathcurveto{\pgfqpoint{-0.008840in}{0.033333in}}{\pgfqpoint{-0.017319in}{0.029821in}}{\pgfqpoint{-0.023570in}{0.023570in}}%
\pgfpathcurveto{\pgfqpoint{-0.029821in}{0.017319in}}{\pgfqpoint{-0.033333in}{0.008840in}}{\pgfqpoint{-0.033333in}{0.000000in}}%
\pgfpathcurveto{\pgfqpoint{-0.033333in}{-0.008840in}}{\pgfqpoint{-0.029821in}{-0.017319in}}{\pgfqpoint{-0.023570in}{-0.023570in}}%
\pgfpathcurveto{\pgfqpoint{-0.017319in}{-0.029821in}}{\pgfqpoint{-0.008840in}{-0.033333in}}{\pgfqpoint{0.000000in}{-0.033333in}}%
\pgfpathlineto{\pgfqpoint{0.000000in}{-0.033333in}}%
\pgfpathclose%
\pgfusepath{stroke,fill}%
}%
\begin{pgfscope}%
\pgfsys@transformshift{3.544302in}{2.377750in}%
\pgfsys@useobject{currentmarker}{}%
\end{pgfscope}%
\end{pgfscope}%
\begin{pgfscope}%
\pgfpathrectangle{\pgfqpoint{0.765000in}{0.660000in}}{\pgfqpoint{4.620000in}{4.620000in}}%
\pgfusepath{clip}%
\pgfsetrectcap%
\pgfsetroundjoin%
\pgfsetlinewidth{1.204500pt}%
\definecolor{currentstroke}{rgb}{1.000000,0.576471,0.309804}%
\pgfsetstrokecolor{currentstroke}%
\pgfsetdash{}{0pt}%
\pgfpathmoveto{\pgfqpoint{3.835229in}{2.523208in}}%
\pgfusepath{stroke}%
\end{pgfscope}%
\begin{pgfscope}%
\pgfpathrectangle{\pgfqpoint{0.765000in}{0.660000in}}{\pgfqpoint{4.620000in}{4.620000in}}%
\pgfusepath{clip}%
\pgfsetbuttcap%
\pgfsetroundjoin%
\definecolor{currentfill}{rgb}{1.000000,0.576471,0.309804}%
\pgfsetfillcolor{currentfill}%
\pgfsetlinewidth{1.003750pt}%
\definecolor{currentstroke}{rgb}{1.000000,0.576471,0.309804}%
\pgfsetstrokecolor{currentstroke}%
\pgfsetdash{}{0pt}%
\pgfsys@defobject{currentmarker}{\pgfqpoint{-0.033333in}{-0.033333in}}{\pgfqpoint{0.033333in}{0.033333in}}{%
\pgfpathmoveto{\pgfqpoint{0.000000in}{-0.033333in}}%
\pgfpathcurveto{\pgfqpoint{0.008840in}{-0.033333in}}{\pgfqpoint{0.017319in}{-0.029821in}}{\pgfqpoint{0.023570in}{-0.023570in}}%
\pgfpathcurveto{\pgfqpoint{0.029821in}{-0.017319in}}{\pgfqpoint{0.033333in}{-0.008840in}}{\pgfqpoint{0.033333in}{0.000000in}}%
\pgfpathcurveto{\pgfqpoint{0.033333in}{0.008840in}}{\pgfqpoint{0.029821in}{0.017319in}}{\pgfqpoint{0.023570in}{0.023570in}}%
\pgfpathcurveto{\pgfqpoint{0.017319in}{0.029821in}}{\pgfqpoint{0.008840in}{0.033333in}}{\pgfqpoint{0.000000in}{0.033333in}}%
\pgfpathcurveto{\pgfqpoint{-0.008840in}{0.033333in}}{\pgfqpoint{-0.017319in}{0.029821in}}{\pgfqpoint{-0.023570in}{0.023570in}}%
\pgfpathcurveto{\pgfqpoint{-0.029821in}{0.017319in}}{\pgfqpoint{-0.033333in}{0.008840in}}{\pgfqpoint{-0.033333in}{0.000000in}}%
\pgfpathcurveto{\pgfqpoint{-0.033333in}{-0.008840in}}{\pgfqpoint{-0.029821in}{-0.017319in}}{\pgfqpoint{-0.023570in}{-0.023570in}}%
\pgfpathcurveto{\pgfqpoint{-0.017319in}{-0.029821in}}{\pgfqpoint{-0.008840in}{-0.033333in}}{\pgfqpoint{0.000000in}{-0.033333in}}%
\pgfpathlineto{\pgfqpoint{0.000000in}{-0.033333in}}%
\pgfpathclose%
\pgfusepath{stroke,fill}%
}%
\begin{pgfscope}%
\pgfsys@transformshift{3.835229in}{2.523208in}%
\pgfsys@useobject{currentmarker}{}%
\end{pgfscope}%
\end{pgfscope}%
\begin{pgfscope}%
\pgfpathrectangle{\pgfqpoint{0.765000in}{0.660000in}}{\pgfqpoint{4.620000in}{4.620000in}}%
\pgfusepath{clip}%
\pgfsetrectcap%
\pgfsetroundjoin%
\pgfsetlinewidth{1.204500pt}%
\definecolor{currentstroke}{rgb}{1.000000,0.576471,0.309804}%
\pgfsetstrokecolor{currentstroke}%
\pgfsetdash{}{0pt}%
\pgfpathmoveto{\pgfqpoint{2.970821in}{1.951121in}}%
\pgfusepath{stroke}%
\end{pgfscope}%
\begin{pgfscope}%
\pgfpathrectangle{\pgfqpoint{0.765000in}{0.660000in}}{\pgfqpoint{4.620000in}{4.620000in}}%
\pgfusepath{clip}%
\pgfsetbuttcap%
\pgfsetroundjoin%
\definecolor{currentfill}{rgb}{1.000000,0.576471,0.309804}%
\pgfsetfillcolor{currentfill}%
\pgfsetlinewidth{1.003750pt}%
\definecolor{currentstroke}{rgb}{1.000000,0.576471,0.309804}%
\pgfsetstrokecolor{currentstroke}%
\pgfsetdash{}{0pt}%
\pgfsys@defobject{currentmarker}{\pgfqpoint{-0.033333in}{-0.033333in}}{\pgfqpoint{0.033333in}{0.033333in}}{%
\pgfpathmoveto{\pgfqpoint{0.000000in}{-0.033333in}}%
\pgfpathcurveto{\pgfqpoint{0.008840in}{-0.033333in}}{\pgfqpoint{0.017319in}{-0.029821in}}{\pgfqpoint{0.023570in}{-0.023570in}}%
\pgfpathcurveto{\pgfqpoint{0.029821in}{-0.017319in}}{\pgfqpoint{0.033333in}{-0.008840in}}{\pgfqpoint{0.033333in}{0.000000in}}%
\pgfpathcurveto{\pgfqpoint{0.033333in}{0.008840in}}{\pgfqpoint{0.029821in}{0.017319in}}{\pgfqpoint{0.023570in}{0.023570in}}%
\pgfpathcurveto{\pgfqpoint{0.017319in}{0.029821in}}{\pgfqpoint{0.008840in}{0.033333in}}{\pgfqpoint{0.000000in}{0.033333in}}%
\pgfpathcurveto{\pgfqpoint{-0.008840in}{0.033333in}}{\pgfqpoint{-0.017319in}{0.029821in}}{\pgfqpoint{-0.023570in}{0.023570in}}%
\pgfpathcurveto{\pgfqpoint{-0.029821in}{0.017319in}}{\pgfqpoint{-0.033333in}{0.008840in}}{\pgfqpoint{-0.033333in}{0.000000in}}%
\pgfpathcurveto{\pgfqpoint{-0.033333in}{-0.008840in}}{\pgfqpoint{-0.029821in}{-0.017319in}}{\pgfqpoint{-0.023570in}{-0.023570in}}%
\pgfpathcurveto{\pgfqpoint{-0.017319in}{-0.029821in}}{\pgfqpoint{-0.008840in}{-0.033333in}}{\pgfqpoint{0.000000in}{-0.033333in}}%
\pgfpathlineto{\pgfqpoint{0.000000in}{-0.033333in}}%
\pgfpathclose%
\pgfusepath{stroke,fill}%
}%
\begin{pgfscope}%
\pgfsys@transformshift{2.970821in}{1.951121in}%
\pgfsys@useobject{currentmarker}{}%
\end{pgfscope}%
\end{pgfscope}%
\begin{pgfscope}%
\pgfpathrectangle{\pgfqpoint{0.765000in}{0.660000in}}{\pgfqpoint{4.620000in}{4.620000in}}%
\pgfusepath{clip}%
\pgfsetrectcap%
\pgfsetroundjoin%
\pgfsetlinewidth{1.204500pt}%
\definecolor{currentstroke}{rgb}{1.000000,0.576471,0.309804}%
\pgfsetstrokecolor{currentstroke}%
\pgfsetdash{}{0pt}%
\pgfpathmoveto{\pgfqpoint{3.170345in}{4.403195in}}%
\pgfusepath{stroke}%
\end{pgfscope}%
\begin{pgfscope}%
\pgfpathrectangle{\pgfqpoint{0.765000in}{0.660000in}}{\pgfqpoint{4.620000in}{4.620000in}}%
\pgfusepath{clip}%
\pgfsetbuttcap%
\pgfsetroundjoin%
\definecolor{currentfill}{rgb}{1.000000,0.576471,0.309804}%
\pgfsetfillcolor{currentfill}%
\pgfsetlinewidth{1.003750pt}%
\definecolor{currentstroke}{rgb}{1.000000,0.576471,0.309804}%
\pgfsetstrokecolor{currentstroke}%
\pgfsetdash{}{0pt}%
\pgfsys@defobject{currentmarker}{\pgfqpoint{-0.033333in}{-0.033333in}}{\pgfqpoint{0.033333in}{0.033333in}}{%
\pgfpathmoveto{\pgfqpoint{0.000000in}{-0.033333in}}%
\pgfpathcurveto{\pgfqpoint{0.008840in}{-0.033333in}}{\pgfqpoint{0.017319in}{-0.029821in}}{\pgfqpoint{0.023570in}{-0.023570in}}%
\pgfpathcurveto{\pgfqpoint{0.029821in}{-0.017319in}}{\pgfqpoint{0.033333in}{-0.008840in}}{\pgfqpoint{0.033333in}{0.000000in}}%
\pgfpathcurveto{\pgfqpoint{0.033333in}{0.008840in}}{\pgfqpoint{0.029821in}{0.017319in}}{\pgfqpoint{0.023570in}{0.023570in}}%
\pgfpathcurveto{\pgfqpoint{0.017319in}{0.029821in}}{\pgfqpoint{0.008840in}{0.033333in}}{\pgfqpoint{0.000000in}{0.033333in}}%
\pgfpathcurveto{\pgfqpoint{-0.008840in}{0.033333in}}{\pgfqpoint{-0.017319in}{0.029821in}}{\pgfqpoint{-0.023570in}{0.023570in}}%
\pgfpathcurveto{\pgfqpoint{-0.029821in}{0.017319in}}{\pgfqpoint{-0.033333in}{0.008840in}}{\pgfqpoint{-0.033333in}{0.000000in}}%
\pgfpathcurveto{\pgfqpoint{-0.033333in}{-0.008840in}}{\pgfqpoint{-0.029821in}{-0.017319in}}{\pgfqpoint{-0.023570in}{-0.023570in}}%
\pgfpathcurveto{\pgfqpoint{-0.017319in}{-0.029821in}}{\pgfqpoint{-0.008840in}{-0.033333in}}{\pgfqpoint{0.000000in}{-0.033333in}}%
\pgfpathlineto{\pgfqpoint{0.000000in}{-0.033333in}}%
\pgfpathclose%
\pgfusepath{stroke,fill}%
}%
\begin{pgfscope}%
\pgfsys@transformshift{3.170345in}{4.403195in}%
\pgfsys@useobject{currentmarker}{}%
\end{pgfscope}%
\end{pgfscope}%
\begin{pgfscope}%
\pgfpathrectangle{\pgfqpoint{0.765000in}{0.660000in}}{\pgfqpoint{4.620000in}{4.620000in}}%
\pgfusepath{clip}%
\pgfsetrectcap%
\pgfsetroundjoin%
\pgfsetlinewidth{1.204500pt}%
\definecolor{currentstroke}{rgb}{1.000000,0.576471,0.309804}%
\pgfsetstrokecolor{currentstroke}%
\pgfsetdash{}{0pt}%
\pgfpathmoveto{\pgfqpoint{4.230410in}{3.478023in}}%
\pgfusepath{stroke}%
\end{pgfscope}%
\begin{pgfscope}%
\pgfpathrectangle{\pgfqpoint{0.765000in}{0.660000in}}{\pgfqpoint{4.620000in}{4.620000in}}%
\pgfusepath{clip}%
\pgfsetbuttcap%
\pgfsetroundjoin%
\definecolor{currentfill}{rgb}{1.000000,0.576471,0.309804}%
\pgfsetfillcolor{currentfill}%
\pgfsetlinewidth{1.003750pt}%
\definecolor{currentstroke}{rgb}{1.000000,0.576471,0.309804}%
\pgfsetstrokecolor{currentstroke}%
\pgfsetdash{}{0pt}%
\pgfsys@defobject{currentmarker}{\pgfqpoint{-0.033333in}{-0.033333in}}{\pgfqpoint{0.033333in}{0.033333in}}{%
\pgfpathmoveto{\pgfqpoint{0.000000in}{-0.033333in}}%
\pgfpathcurveto{\pgfqpoint{0.008840in}{-0.033333in}}{\pgfqpoint{0.017319in}{-0.029821in}}{\pgfqpoint{0.023570in}{-0.023570in}}%
\pgfpathcurveto{\pgfqpoint{0.029821in}{-0.017319in}}{\pgfqpoint{0.033333in}{-0.008840in}}{\pgfqpoint{0.033333in}{0.000000in}}%
\pgfpathcurveto{\pgfqpoint{0.033333in}{0.008840in}}{\pgfqpoint{0.029821in}{0.017319in}}{\pgfqpoint{0.023570in}{0.023570in}}%
\pgfpathcurveto{\pgfqpoint{0.017319in}{0.029821in}}{\pgfqpoint{0.008840in}{0.033333in}}{\pgfqpoint{0.000000in}{0.033333in}}%
\pgfpathcurveto{\pgfqpoint{-0.008840in}{0.033333in}}{\pgfqpoint{-0.017319in}{0.029821in}}{\pgfqpoint{-0.023570in}{0.023570in}}%
\pgfpathcurveto{\pgfqpoint{-0.029821in}{0.017319in}}{\pgfqpoint{-0.033333in}{0.008840in}}{\pgfqpoint{-0.033333in}{0.000000in}}%
\pgfpathcurveto{\pgfqpoint{-0.033333in}{-0.008840in}}{\pgfqpoint{-0.029821in}{-0.017319in}}{\pgfqpoint{-0.023570in}{-0.023570in}}%
\pgfpathcurveto{\pgfqpoint{-0.017319in}{-0.029821in}}{\pgfqpoint{-0.008840in}{-0.033333in}}{\pgfqpoint{0.000000in}{-0.033333in}}%
\pgfpathlineto{\pgfqpoint{0.000000in}{-0.033333in}}%
\pgfpathclose%
\pgfusepath{stroke,fill}%
}%
\begin{pgfscope}%
\pgfsys@transformshift{4.230410in}{3.478023in}%
\pgfsys@useobject{currentmarker}{}%
\end{pgfscope}%
\end{pgfscope}%
\begin{pgfscope}%
\pgfpathrectangle{\pgfqpoint{0.765000in}{0.660000in}}{\pgfqpoint{4.620000in}{4.620000in}}%
\pgfusepath{clip}%
\pgfsetrectcap%
\pgfsetroundjoin%
\pgfsetlinewidth{1.204500pt}%
\definecolor{currentstroke}{rgb}{1.000000,0.576471,0.309804}%
\pgfsetstrokecolor{currentstroke}%
\pgfsetdash{}{0pt}%
\pgfpathmoveto{\pgfqpoint{2.318786in}{3.861573in}}%
\pgfusepath{stroke}%
\end{pgfscope}%
\begin{pgfscope}%
\pgfpathrectangle{\pgfqpoint{0.765000in}{0.660000in}}{\pgfqpoint{4.620000in}{4.620000in}}%
\pgfusepath{clip}%
\pgfsetbuttcap%
\pgfsetroundjoin%
\definecolor{currentfill}{rgb}{1.000000,0.576471,0.309804}%
\pgfsetfillcolor{currentfill}%
\pgfsetlinewidth{1.003750pt}%
\definecolor{currentstroke}{rgb}{1.000000,0.576471,0.309804}%
\pgfsetstrokecolor{currentstroke}%
\pgfsetdash{}{0pt}%
\pgfsys@defobject{currentmarker}{\pgfqpoint{-0.033333in}{-0.033333in}}{\pgfqpoint{0.033333in}{0.033333in}}{%
\pgfpathmoveto{\pgfqpoint{0.000000in}{-0.033333in}}%
\pgfpathcurveto{\pgfqpoint{0.008840in}{-0.033333in}}{\pgfqpoint{0.017319in}{-0.029821in}}{\pgfqpoint{0.023570in}{-0.023570in}}%
\pgfpathcurveto{\pgfqpoint{0.029821in}{-0.017319in}}{\pgfqpoint{0.033333in}{-0.008840in}}{\pgfqpoint{0.033333in}{0.000000in}}%
\pgfpathcurveto{\pgfqpoint{0.033333in}{0.008840in}}{\pgfqpoint{0.029821in}{0.017319in}}{\pgfqpoint{0.023570in}{0.023570in}}%
\pgfpathcurveto{\pgfqpoint{0.017319in}{0.029821in}}{\pgfqpoint{0.008840in}{0.033333in}}{\pgfqpoint{0.000000in}{0.033333in}}%
\pgfpathcurveto{\pgfqpoint{-0.008840in}{0.033333in}}{\pgfqpoint{-0.017319in}{0.029821in}}{\pgfqpoint{-0.023570in}{0.023570in}}%
\pgfpathcurveto{\pgfqpoint{-0.029821in}{0.017319in}}{\pgfqpoint{-0.033333in}{0.008840in}}{\pgfqpoint{-0.033333in}{0.000000in}}%
\pgfpathcurveto{\pgfqpoint{-0.033333in}{-0.008840in}}{\pgfqpoint{-0.029821in}{-0.017319in}}{\pgfqpoint{-0.023570in}{-0.023570in}}%
\pgfpathcurveto{\pgfqpoint{-0.017319in}{-0.029821in}}{\pgfqpoint{-0.008840in}{-0.033333in}}{\pgfqpoint{0.000000in}{-0.033333in}}%
\pgfpathlineto{\pgfqpoint{0.000000in}{-0.033333in}}%
\pgfpathclose%
\pgfusepath{stroke,fill}%
}%
\begin{pgfscope}%
\pgfsys@transformshift{2.318786in}{3.861573in}%
\pgfsys@useobject{currentmarker}{}%
\end{pgfscope}%
\end{pgfscope}%
\begin{pgfscope}%
\pgfpathrectangle{\pgfqpoint{0.765000in}{0.660000in}}{\pgfqpoint{4.620000in}{4.620000in}}%
\pgfusepath{clip}%
\pgfsetrectcap%
\pgfsetroundjoin%
\pgfsetlinewidth{1.204500pt}%
\definecolor{currentstroke}{rgb}{1.000000,0.576471,0.309804}%
\pgfsetstrokecolor{currentstroke}%
\pgfsetdash{}{0pt}%
\pgfpathmoveto{\pgfqpoint{4.076604in}{4.486588in}}%
\pgfusepath{stroke}%
\end{pgfscope}%
\begin{pgfscope}%
\pgfpathrectangle{\pgfqpoint{0.765000in}{0.660000in}}{\pgfqpoint{4.620000in}{4.620000in}}%
\pgfusepath{clip}%
\pgfsetbuttcap%
\pgfsetroundjoin%
\definecolor{currentfill}{rgb}{1.000000,0.576471,0.309804}%
\pgfsetfillcolor{currentfill}%
\pgfsetlinewidth{1.003750pt}%
\definecolor{currentstroke}{rgb}{1.000000,0.576471,0.309804}%
\pgfsetstrokecolor{currentstroke}%
\pgfsetdash{}{0pt}%
\pgfsys@defobject{currentmarker}{\pgfqpoint{-0.033333in}{-0.033333in}}{\pgfqpoint{0.033333in}{0.033333in}}{%
\pgfpathmoveto{\pgfqpoint{0.000000in}{-0.033333in}}%
\pgfpathcurveto{\pgfqpoint{0.008840in}{-0.033333in}}{\pgfqpoint{0.017319in}{-0.029821in}}{\pgfqpoint{0.023570in}{-0.023570in}}%
\pgfpathcurveto{\pgfqpoint{0.029821in}{-0.017319in}}{\pgfqpoint{0.033333in}{-0.008840in}}{\pgfqpoint{0.033333in}{0.000000in}}%
\pgfpathcurveto{\pgfqpoint{0.033333in}{0.008840in}}{\pgfqpoint{0.029821in}{0.017319in}}{\pgfqpoint{0.023570in}{0.023570in}}%
\pgfpathcurveto{\pgfqpoint{0.017319in}{0.029821in}}{\pgfqpoint{0.008840in}{0.033333in}}{\pgfqpoint{0.000000in}{0.033333in}}%
\pgfpathcurveto{\pgfqpoint{-0.008840in}{0.033333in}}{\pgfqpoint{-0.017319in}{0.029821in}}{\pgfqpoint{-0.023570in}{0.023570in}}%
\pgfpathcurveto{\pgfqpoint{-0.029821in}{0.017319in}}{\pgfqpoint{-0.033333in}{0.008840in}}{\pgfqpoint{-0.033333in}{0.000000in}}%
\pgfpathcurveto{\pgfqpoint{-0.033333in}{-0.008840in}}{\pgfqpoint{-0.029821in}{-0.017319in}}{\pgfqpoint{-0.023570in}{-0.023570in}}%
\pgfpathcurveto{\pgfqpoint{-0.017319in}{-0.029821in}}{\pgfqpoint{-0.008840in}{-0.033333in}}{\pgfqpoint{0.000000in}{-0.033333in}}%
\pgfpathlineto{\pgfqpoint{0.000000in}{-0.033333in}}%
\pgfpathclose%
\pgfusepath{stroke,fill}%
}%
\begin{pgfscope}%
\pgfsys@transformshift{4.076604in}{4.486588in}%
\pgfsys@useobject{currentmarker}{}%
\end{pgfscope}%
\end{pgfscope}%
\begin{pgfscope}%
\pgfpathrectangle{\pgfqpoint{0.765000in}{0.660000in}}{\pgfqpoint{4.620000in}{4.620000in}}%
\pgfusepath{clip}%
\pgfsetrectcap%
\pgfsetroundjoin%
\pgfsetlinewidth{1.204500pt}%
\definecolor{currentstroke}{rgb}{1.000000,0.576471,0.309804}%
\pgfsetstrokecolor{currentstroke}%
\pgfsetdash{}{0pt}%
\pgfpathmoveto{\pgfqpoint{2.630042in}{3.660013in}}%
\pgfusepath{stroke}%
\end{pgfscope}%
\begin{pgfscope}%
\pgfpathrectangle{\pgfqpoint{0.765000in}{0.660000in}}{\pgfqpoint{4.620000in}{4.620000in}}%
\pgfusepath{clip}%
\pgfsetbuttcap%
\pgfsetroundjoin%
\definecolor{currentfill}{rgb}{1.000000,0.576471,0.309804}%
\pgfsetfillcolor{currentfill}%
\pgfsetlinewidth{1.003750pt}%
\definecolor{currentstroke}{rgb}{1.000000,0.576471,0.309804}%
\pgfsetstrokecolor{currentstroke}%
\pgfsetdash{}{0pt}%
\pgfsys@defobject{currentmarker}{\pgfqpoint{-0.033333in}{-0.033333in}}{\pgfqpoint{0.033333in}{0.033333in}}{%
\pgfpathmoveto{\pgfqpoint{0.000000in}{-0.033333in}}%
\pgfpathcurveto{\pgfqpoint{0.008840in}{-0.033333in}}{\pgfqpoint{0.017319in}{-0.029821in}}{\pgfqpoint{0.023570in}{-0.023570in}}%
\pgfpathcurveto{\pgfqpoint{0.029821in}{-0.017319in}}{\pgfqpoint{0.033333in}{-0.008840in}}{\pgfqpoint{0.033333in}{0.000000in}}%
\pgfpathcurveto{\pgfqpoint{0.033333in}{0.008840in}}{\pgfqpoint{0.029821in}{0.017319in}}{\pgfqpoint{0.023570in}{0.023570in}}%
\pgfpathcurveto{\pgfqpoint{0.017319in}{0.029821in}}{\pgfqpoint{0.008840in}{0.033333in}}{\pgfqpoint{0.000000in}{0.033333in}}%
\pgfpathcurveto{\pgfqpoint{-0.008840in}{0.033333in}}{\pgfqpoint{-0.017319in}{0.029821in}}{\pgfqpoint{-0.023570in}{0.023570in}}%
\pgfpathcurveto{\pgfqpoint{-0.029821in}{0.017319in}}{\pgfqpoint{-0.033333in}{0.008840in}}{\pgfqpoint{-0.033333in}{0.000000in}}%
\pgfpathcurveto{\pgfqpoint{-0.033333in}{-0.008840in}}{\pgfqpoint{-0.029821in}{-0.017319in}}{\pgfqpoint{-0.023570in}{-0.023570in}}%
\pgfpathcurveto{\pgfqpoint{-0.017319in}{-0.029821in}}{\pgfqpoint{-0.008840in}{-0.033333in}}{\pgfqpoint{0.000000in}{-0.033333in}}%
\pgfpathlineto{\pgfqpoint{0.000000in}{-0.033333in}}%
\pgfpathclose%
\pgfusepath{stroke,fill}%
}%
\begin{pgfscope}%
\pgfsys@transformshift{2.630042in}{3.660013in}%
\pgfsys@useobject{currentmarker}{}%
\end{pgfscope}%
\end{pgfscope}%
\begin{pgfscope}%
\pgfpathrectangle{\pgfqpoint{0.765000in}{0.660000in}}{\pgfqpoint{4.620000in}{4.620000in}}%
\pgfusepath{clip}%
\pgfsetrectcap%
\pgfsetroundjoin%
\pgfsetlinewidth{1.204500pt}%
\definecolor{currentstroke}{rgb}{1.000000,0.576471,0.309804}%
\pgfsetstrokecolor{currentstroke}%
\pgfsetdash{}{0pt}%
\pgfpathmoveto{\pgfqpoint{3.860845in}{2.713286in}}%
\pgfusepath{stroke}%
\end{pgfscope}%
\begin{pgfscope}%
\pgfpathrectangle{\pgfqpoint{0.765000in}{0.660000in}}{\pgfqpoint{4.620000in}{4.620000in}}%
\pgfusepath{clip}%
\pgfsetbuttcap%
\pgfsetroundjoin%
\definecolor{currentfill}{rgb}{1.000000,0.576471,0.309804}%
\pgfsetfillcolor{currentfill}%
\pgfsetlinewidth{1.003750pt}%
\definecolor{currentstroke}{rgb}{1.000000,0.576471,0.309804}%
\pgfsetstrokecolor{currentstroke}%
\pgfsetdash{}{0pt}%
\pgfsys@defobject{currentmarker}{\pgfqpoint{-0.033333in}{-0.033333in}}{\pgfqpoint{0.033333in}{0.033333in}}{%
\pgfpathmoveto{\pgfqpoint{0.000000in}{-0.033333in}}%
\pgfpathcurveto{\pgfqpoint{0.008840in}{-0.033333in}}{\pgfqpoint{0.017319in}{-0.029821in}}{\pgfqpoint{0.023570in}{-0.023570in}}%
\pgfpathcurveto{\pgfqpoint{0.029821in}{-0.017319in}}{\pgfqpoint{0.033333in}{-0.008840in}}{\pgfqpoint{0.033333in}{0.000000in}}%
\pgfpathcurveto{\pgfqpoint{0.033333in}{0.008840in}}{\pgfqpoint{0.029821in}{0.017319in}}{\pgfqpoint{0.023570in}{0.023570in}}%
\pgfpathcurveto{\pgfqpoint{0.017319in}{0.029821in}}{\pgfqpoint{0.008840in}{0.033333in}}{\pgfqpoint{0.000000in}{0.033333in}}%
\pgfpathcurveto{\pgfqpoint{-0.008840in}{0.033333in}}{\pgfqpoint{-0.017319in}{0.029821in}}{\pgfqpoint{-0.023570in}{0.023570in}}%
\pgfpathcurveto{\pgfqpoint{-0.029821in}{0.017319in}}{\pgfqpoint{-0.033333in}{0.008840in}}{\pgfqpoint{-0.033333in}{0.000000in}}%
\pgfpathcurveto{\pgfqpoint{-0.033333in}{-0.008840in}}{\pgfqpoint{-0.029821in}{-0.017319in}}{\pgfqpoint{-0.023570in}{-0.023570in}}%
\pgfpathcurveto{\pgfqpoint{-0.017319in}{-0.029821in}}{\pgfqpoint{-0.008840in}{-0.033333in}}{\pgfqpoint{0.000000in}{-0.033333in}}%
\pgfpathlineto{\pgfqpoint{0.000000in}{-0.033333in}}%
\pgfpathclose%
\pgfusepath{stroke,fill}%
}%
\begin{pgfscope}%
\pgfsys@transformshift{3.860845in}{2.713286in}%
\pgfsys@useobject{currentmarker}{}%
\end{pgfscope}%
\end{pgfscope}%
\begin{pgfscope}%
\pgfpathrectangle{\pgfqpoint{0.765000in}{0.660000in}}{\pgfqpoint{4.620000in}{4.620000in}}%
\pgfusepath{clip}%
\pgfsetrectcap%
\pgfsetroundjoin%
\pgfsetlinewidth{1.204500pt}%
\definecolor{currentstroke}{rgb}{1.000000,0.576471,0.309804}%
\pgfsetstrokecolor{currentstroke}%
\pgfsetdash{}{0pt}%
\pgfpathmoveto{\pgfqpoint{2.017281in}{2.669087in}}%
\pgfusepath{stroke}%
\end{pgfscope}%
\begin{pgfscope}%
\pgfpathrectangle{\pgfqpoint{0.765000in}{0.660000in}}{\pgfqpoint{4.620000in}{4.620000in}}%
\pgfusepath{clip}%
\pgfsetbuttcap%
\pgfsetroundjoin%
\definecolor{currentfill}{rgb}{1.000000,0.576471,0.309804}%
\pgfsetfillcolor{currentfill}%
\pgfsetlinewidth{1.003750pt}%
\definecolor{currentstroke}{rgb}{1.000000,0.576471,0.309804}%
\pgfsetstrokecolor{currentstroke}%
\pgfsetdash{}{0pt}%
\pgfsys@defobject{currentmarker}{\pgfqpoint{-0.033333in}{-0.033333in}}{\pgfqpoint{0.033333in}{0.033333in}}{%
\pgfpathmoveto{\pgfqpoint{0.000000in}{-0.033333in}}%
\pgfpathcurveto{\pgfqpoint{0.008840in}{-0.033333in}}{\pgfqpoint{0.017319in}{-0.029821in}}{\pgfqpoint{0.023570in}{-0.023570in}}%
\pgfpathcurveto{\pgfqpoint{0.029821in}{-0.017319in}}{\pgfqpoint{0.033333in}{-0.008840in}}{\pgfqpoint{0.033333in}{0.000000in}}%
\pgfpathcurveto{\pgfqpoint{0.033333in}{0.008840in}}{\pgfqpoint{0.029821in}{0.017319in}}{\pgfqpoint{0.023570in}{0.023570in}}%
\pgfpathcurveto{\pgfqpoint{0.017319in}{0.029821in}}{\pgfqpoint{0.008840in}{0.033333in}}{\pgfqpoint{0.000000in}{0.033333in}}%
\pgfpathcurveto{\pgfqpoint{-0.008840in}{0.033333in}}{\pgfqpoint{-0.017319in}{0.029821in}}{\pgfqpoint{-0.023570in}{0.023570in}}%
\pgfpathcurveto{\pgfqpoint{-0.029821in}{0.017319in}}{\pgfqpoint{-0.033333in}{0.008840in}}{\pgfqpoint{-0.033333in}{0.000000in}}%
\pgfpathcurveto{\pgfqpoint{-0.033333in}{-0.008840in}}{\pgfqpoint{-0.029821in}{-0.017319in}}{\pgfqpoint{-0.023570in}{-0.023570in}}%
\pgfpathcurveto{\pgfqpoint{-0.017319in}{-0.029821in}}{\pgfqpoint{-0.008840in}{-0.033333in}}{\pgfqpoint{0.000000in}{-0.033333in}}%
\pgfpathlineto{\pgfqpoint{0.000000in}{-0.033333in}}%
\pgfpathclose%
\pgfusepath{stroke,fill}%
}%
\begin{pgfscope}%
\pgfsys@transformshift{2.017281in}{2.669087in}%
\pgfsys@useobject{currentmarker}{}%
\end{pgfscope}%
\end{pgfscope}%
\begin{pgfscope}%
\pgfpathrectangle{\pgfqpoint{0.765000in}{0.660000in}}{\pgfqpoint{4.620000in}{4.620000in}}%
\pgfusepath{clip}%
\pgfsetrectcap%
\pgfsetroundjoin%
\pgfsetlinewidth{1.204500pt}%
\definecolor{currentstroke}{rgb}{1.000000,0.576471,0.309804}%
\pgfsetstrokecolor{currentstroke}%
\pgfsetdash{}{0pt}%
\pgfpathmoveto{\pgfqpoint{2.154893in}{3.297158in}}%
\pgfusepath{stroke}%
\end{pgfscope}%
\begin{pgfscope}%
\pgfpathrectangle{\pgfqpoint{0.765000in}{0.660000in}}{\pgfqpoint{4.620000in}{4.620000in}}%
\pgfusepath{clip}%
\pgfsetbuttcap%
\pgfsetroundjoin%
\definecolor{currentfill}{rgb}{1.000000,0.576471,0.309804}%
\pgfsetfillcolor{currentfill}%
\pgfsetlinewidth{1.003750pt}%
\definecolor{currentstroke}{rgb}{1.000000,0.576471,0.309804}%
\pgfsetstrokecolor{currentstroke}%
\pgfsetdash{}{0pt}%
\pgfsys@defobject{currentmarker}{\pgfqpoint{-0.033333in}{-0.033333in}}{\pgfqpoint{0.033333in}{0.033333in}}{%
\pgfpathmoveto{\pgfqpoint{0.000000in}{-0.033333in}}%
\pgfpathcurveto{\pgfqpoint{0.008840in}{-0.033333in}}{\pgfqpoint{0.017319in}{-0.029821in}}{\pgfqpoint{0.023570in}{-0.023570in}}%
\pgfpathcurveto{\pgfqpoint{0.029821in}{-0.017319in}}{\pgfqpoint{0.033333in}{-0.008840in}}{\pgfqpoint{0.033333in}{0.000000in}}%
\pgfpathcurveto{\pgfqpoint{0.033333in}{0.008840in}}{\pgfqpoint{0.029821in}{0.017319in}}{\pgfqpoint{0.023570in}{0.023570in}}%
\pgfpathcurveto{\pgfqpoint{0.017319in}{0.029821in}}{\pgfqpoint{0.008840in}{0.033333in}}{\pgfqpoint{0.000000in}{0.033333in}}%
\pgfpathcurveto{\pgfqpoint{-0.008840in}{0.033333in}}{\pgfqpoint{-0.017319in}{0.029821in}}{\pgfqpoint{-0.023570in}{0.023570in}}%
\pgfpathcurveto{\pgfqpoint{-0.029821in}{0.017319in}}{\pgfqpoint{-0.033333in}{0.008840in}}{\pgfqpoint{-0.033333in}{0.000000in}}%
\pgfpathcurveto{\pgfqpoint{-0.033333in}{-0.008840in}}{\pgfqpoint{-0.029821in}{-0.017319in}}{\pgfqpoint{-0.023570in}{-0.023570in}}%
\pgfpathcurveto{\pgfqpoint{-0.017319in}{-0.029821in}}{\pgfqpoint{-0.008840in}{-0.033333in}}{\pgfqpoint{0.000000in}{-0.033333in}}%
\pgfpathlineto{\pgfqpoint{0.000000in}{-0.033333in}}%
\pgfpathclose%
\pgfusepath{stroke,fill}%
}%
\begin{pgfscope}%
\pgfsys@transformshift{2.154893in}{3.297158in}%
\pgfsys@useobject{currentmarker}{}%
\end{pgfscope}%
\end{pgfscope}%
\begin{pgfscope}%
\pgfpathrectangle{\pgfqpoint{0.765000in}{0.660000in}}{\pgfqpoint{4.620000in}{4.620000in}}%
\pgfusepath{clip}%
\pgfsetrectcap%
\pgfsetroundjoin%
\pgfsetlinewidth{1.204500pt}%
\definecolor{currentstroke}{rgb}{1.000000,0.576471,0.309804}%
\pgfsetstrokecolor{currentstroke}%
\pgfsetdash{}{0pt}%
\pgfpathmoveto{\pgfqpoint{1.740234in}{3.241407in}}%
\pgfusepath{stroke}%
\end{pgfscope}%
\begin{pgfscope}%
\pgfpathrectangle{\pgfqpoint{0.765000in}{0.660000in}}{\pgfqpoint{4.620000in}{4.620000in}}%
\pgfusepath{clip}%
\pgfsetbuttcap%
\pgfsetroundjoin%
\definecolor{currentfill}{rgb}{1.000000,0.576471,0.309804}%
\pgfsetfillcolor{currentfill}%
\pgfsetlinewidth{1.003750pt}%
\definecolor{currentstroke}{rgb}{1.000000,0.576471,0.309804}%
\pgfsetstrokecolor{currentstroke}%
\pgfsetdash{}{0pt}%
\pgfsys@defobject{currentmarker}{\pgfqpoint{-0.033333in}{-0.033333in}}{\pgfqpoint{0.033333in}{0.033333in}}{%
\pgfpathmoveto{\pgfqpoint{0.000000in}{-0.033333in}}%
\pgfpathcurveto{\pgfqpoint{0.008840in}{-0.033333in}}{\pgfqpoint{0.017319in}{-0.029821in}}{\pgfqpoint{0.023570in}{-0.023570in}}%
\pgfpathcurveto{\pgfqpoint{0.029821in}{-0.017319in}}{\pgfqpoint{0.033333in}{-0.008840in}}{\pgfqpoint{0.033333in}{0.000000in}}%
\pgfpathcurveto{\pgfqpoint{0.033333in}{0.008840in}}{\pgfqpoint{0.029821in}{0.017319in}}{\pgfqpoint{0.023570in}{0.023570in}}%
\pgfpathcurveto{\pgfqpoint{0.017319in}{0.029821in}}{\pgfqpoint{0.008840in}{0.033333in}}{\pgfqpoint{0.000000in}{0.033333in}}%
\pgfpathcurveto{\pgfqpoint{-0.008840in}{0.033333in}}{\pgfqpoint{-0.017319in}{0.029821in}}{\pgfqpoint{-0.023570in}{0.023570in}}%
\pgfpathcurveto{\pgfqpoint{-0.029821in}{0.017319in}}{\pgfqpoint{-0.033333in}{0.008840in}}{\pgfqpoint{-0.033333in}{0.000000in}}%
\pgfpathcurveto{\pgfqpoint{-0.033333in}{-0.008840in}}{\pgfqpoint{-0.029821in}{-0.017319in}}{\pgfqpoint{-0.023570in}{-0.023570in}}%
\pgfpathcurveto{\pgfqpoint{-0.017319in}{-0.029821in}}{\pgfqpoint{-0.008840in}{-0.033333in}}{\pgfqpoint{0.000000in}{-0.033333in}}%
\pgfpathlineto{\pgfqpoint{0.000000in}{-0.033333in}}%
\pgfpathclose%
\pgfusepath{stroke,fill}%
}%
\begin{pgfscope}%
\pgfsys@transformshift{1.740234in}{3.241407in}%
\pgfsys@useobject{currentmarker}{}%
\end{pgfscope}%
\end{pgfscope}%
\begin{pgfscope}%
\pgfpathrectangle{\pgfqpoint{0.765000in}{0.660000in}}{\pgfqpoint{4.620000in}{4.620000in}}%
\pgfusepath{clip}%
\pgfsetrectcap%
\pgfsetroundjoin%
\pgfsetlinewidth{1.204500pt}%
\definecolor{currentstroke}{rgb}{1.000000,0.576471,0.309804}%
\pgfsetstrokecolor{currentstroke}%
\pgfsetdash{}{0pt}%
\pgfpathmoveto{\pgfqpoint{3.377710in}{2.090057in}}%
\pgfusepath{stroke}%
\end{pgfscope}%
\begin{pgfscope}%
\pgfpathrectangle{\pgfqpoint{0.765000in}{0.660000in}}{\pgfqpoint{4.620000in}{4.620000in}}%
\pgfusepath{clip}%
\pgfsetbuttcap%
\pgfsetroundjoin%
\definecolor{currentfill}{rgb}{1.000000,0.576471,0.309804}%
\pgfsetfillcolor{currentfill}%
\pgfsetlinewidth{1.003750pt}%
\definecolor{currentstroke}{rgb}{1.000000,0.576471,0.309804}%
\pgfsetstrokecolor{currentstroke}%
\pgfsetdash{}{0pt}%
\pgfsys@defobject{currentmarker}{\pgfqpoint{-0.033333in}{-0.033333in}}{\pgfqpoint{0.033333in}{0.033333in}}{%
\pgfpathmoveto{\pgfqpoint{0.000000in}{-0.033333in}}%
\pgfpathcurveto{\pgfqpoint{0.008840in}{-0.033333in}}{\pgfqpoint{0.017319in}{-0.029821in}}{\pgfqpoint{0.023570in}{-0.023570in}}%
\pgfpathcurveto{\pgfqpoint{0.029821in}{-0.017319in}}{\pgfqpoint{0.033333in}{-0.008840in}}{\pgfqpoint{0.033333in}{0.000000in}}%
\pgfpathcurveto{\pgfqpoint{0.033333in}{0.008840in}}{\pgfqpoint{0.029821in}{0.017319in}}{\pgfqpoint{0.023570in}{0.023570in}}%
\pgfpathcurveto{\pgfqpoint{0.017319in}{0.029821in}}{\pgfqpoint{0.008840in}{0.033333in}}{\pgfqpoint{0.000000in}{0.033333in}}%
\pgfpathcurveto{\pgfqpoint{-0.008840in}{0.033333in}}{\pgfqpoint{-0.017319in}{0.029821in}}{\pgfqpoint{-0.023570in}{0.023570in}}%
\pgfpathcurveto{\pgfqpoint{-0.029821in}{0.017319in}}{\pgfqpoint{-0.033333in}{0.008840in}}{\pgfqpoint{-0.033333in}{0.000000in}}%
\pgfpathcurveto{\pgfqpoint{-0.033333in}{-0.008840in}}{\pgfqpoint{-0.029821in}{-0.017319in}}{\pgfqpoint{-0.023570in}{-0.023570in}}%
\pgfpathcurveto{\pgfqpoint{-0.017319in}{-0.029821in}}{\pgfqpoint{-0.008840in}{-0.033333in}}{\pgfqpoint{0.000000in}{-0.033333in}}%
\pgfpathlineto{\pgfqpoint{0.000000in}{-0.033333in}}%
\pgfpathclose%
\pgfusepath{stroke,fill}%
}%
\begin{pgfscope}%
\pgfsys@transformshift{3.377710in}{2.090057in}%
\pgfsys@useobject{currentmarker}{}%
\end{pgfscope}%
\end{pgfscope}%
\begin{pgfscope}%
\pgfpathrectangle{\pgfqpoint{0.765000in}{0.660000in}}{\pgfqpoint{4.620000in}{4.620000in}}%
\pgfusepath{clip}%
\pgfsetrectcap%
\pgfsetroundjoin%
\pgfsetlinewidth{1.204500pt}%
\definecolor{currentstroke}{rgb}{1.000000,0.576471,0.309804}%
\pgfsetstrokecolor{currentstroke}%
\pgfsetdash{}{0pt}%
\pgfpathmoveto{\pgfqpoint{4.006889in}{3.440269in}}%
\pgfusepath{stroke}%
\end{pgfscope}%
\begin{pgfscope}%
\pgfpathrectangle{\pgfqpoint{0.765000in}{0.660000in}}{\pgfqpoint{4.620000in}{4.620000in}}%
\pgfusepath{clip}%
\pgfsetbuttcap%
\pgfsetroundjoin%
\definecolor{currentfill}{rgb}{1.000000,0.576471,0.309804}%
\pgfsetfillcolor{currentfill}%
\pgfsetlinewidth{1.003750pt}%
\definecolor{currentstroke}{rgb}{1.000000,0.576471,0.309804}%
\pgfsetstrokecolor{currentstroke}%
\pgfsetdash{}{0pt}%
\pgfsys@defobject{currentmarker}{\pgfqpoint{-0.033333in}{-0.033333in}}{\pgfqpoint{0.033333in}{0.033333in}}{%
\pgfpathmoveto{\pgfqpoint{0.000000in}{-0.033333in}}%
\pgfpathcurveto{\pgfqpoint{0.008840in}{-0.033333in}}{\pgfqpoint{0.017319in}{-0.029821in}}{\pgfqpoint{0.023570in}{-0.023570in}}%
\pgfpathcurveto{\pgfqpoint{0.029821in}{-0.017319in}}{\pgfqpoint{0.033333in}{-0.008840in}}{\pgfqpoint{0.033333in}{0.000000in}}%
\pgfpathcurveto{\pgfqpoint{0.033333in}{0.008840in}}{\pgfqpoint{0.029821in}{0.017319in}}{\pgfqpoint{0.023570in}{0.023570in}}%
\pgfpathcurveto{\pgfqpoint{0.017319in}{0.029821in}}{\pgfqpoint{0.008840in}{0.033333in}}{\pgfqpoint{0.000000in}{0.033333in}}%
\pgfpathcurveto{\pgfqpoint{-0.008840in}{0.033333in}}{\pgfqpoint{-0.017319in}{0.029821in}}{\pgfqpoint{-0.023570in}{0.023570in}}%
\pgfpathcurveto{\pgfqpoint{-0.029821in}{0.017319in}}{\pgfqpoint{-0.033333in}{0.008840in}}{\pgfqpoint{-0.033333in}{0.000000in}}%
\pgfpathcurveto{\pgfqpoint{-0.033333in}{-0.008840in}}{\pgfqpoint{-0.029821in}{-0.017319in}}{\pgfqpoint{-0.023570in}{-0.023570in}}%
\pgfpathcurveto{\pgfqpoint{-0.017319in}{-0.029821in}}{\pgfqpoint{-0.008840in}{-0.033333in}}{\pgfqpoint{0.000000in}{-0.033333in}}%
\pgfpathlineto{\pgfqpoint{0.000000in}{-0.033333in}}%
\pgfpathclose%
\pgfusepath{stroke,fill}%
}%
\begin{pgfscope}%
\pgfsys@transformshift{4.006889in}{3.440269in}%
\pgfsys@useobject{currentmarker}{}%
\end{pgfscope}%
\end{pgfscope}%
\begin{pgfscope}%
\pgfpathrectangle{\pgfqpoint{0.765000in}{0.660000in}}{\pgfqpoint{4.620000in}{4.620000in}}%
\pgfusepath{clip}%
\pgfsetrectcap%
\pgfsetroundjoin%
\pgfsetlinewidth{1.204500pt}%
\definecolor{currentstroke}{rgb}{1.000000,0.576471,0.309804}%
\pgfsetstrokecolor{currentstroke}%
\pgfsetdash{}{0pt}%
\pgfpathmoveto{\pgfqpoint{2.465772in}{1.758120in}}%
\pgfusepath{stroke}%
\end{pgfscope}%
\begin{pgfscope}%
\pgfpathrectangle{\pgfqpoint{0.765000in}{0.660000in}}{\pgfqpoint{4.620000in}{4.620000in}}%
\pgfusepath{clip}%
\pgfsetbuttcap%
\pgfsetroundjoin%
\definecolor{currentfill}{rgb}{1.000000,0.576471,0.309804}%
\pgfsetfillcolor{currentfill}%
\pgfsetlinewidth{1.003750pt}%
\definecolor{currentstroke}{rgb}{1.000000,0.576471,0.309804}%
\pgfsetstrokecolor{currentstroke}%
\pgfsetdash{}{0pt}%
\pgfsys@defobject{currentmarker}{\pgfqpoint{-0.033333in}{-0.033333in}}{\pgfqpoint{0.033333in}{0.033333in}}{%
\pgfpathmoveto{\pgfqpoint{0.000000in}{-0.033333in}}%
\pgfpathcurveto{\pgfqpoint{0.008840in}{-0.033333in}}{\pgfqpoint{0.017319in}{-0.029821in}}{\pgfqpoint{0.023570in}{-0.023570in}}%
\pgfpathcurveto{\pgfqpoint{0.029821in}{-0.017319in}}{\pgfqpoint{0.033333in}{-0.008840in}}{\pgfqpoint{0.033333in}{0.000000in}}%
\pgfpathcurveto{\pgfqpoint{0.033333in}{0.008840in}}{\pgfqpoint{0.029821in}{0.017319in}}{\pgfqpoint{0.023570in}{0.023570in}}%
\pgfpathcurveto{\pgfqpoint{0.017319in}{0.029821in}}{\pgfqpoint{0.008840in}{0.033333in}}{\pgfqpoint{0.000000in}{0.033333in}}%
\pgfpathcurveto{\pgfqpoint{-0.008840in}{0.033333in}}{\pgfqpoint{-0.017319in}{0.029821in}}{\pgfqpoint{-0.023570in}{0.023570in}}%
\pgfpathcurveto{\pgfqpoint{-0.029821in}{0.017319in}}{\pgfqpoint{-0.033333in}{0.008840in}}{\pgfqpoint{-0.033333in}{0.000000in}}%
\pgfpathcurveto{\pgfqpoint{-0.033333in}{-0.008840in}}{\pgfqpoint{-0.029821in}{-0.017319in}}{\pgfqpoint{-0.023570in}{-0.023570in}}%
\pgfpathcurveto{\pgfqpoint{-0.017319in}{-0.029821in}}{\pgfqpoint{-0.008840in}{-0.033333in}}{\pgfqpoint{0.000000in}{-0.033333in}}%
\pgfpathlineto{\pgfqpoint{0.000000in}{-0.033333in}}%
\pgfpathclose%
\pgfusepath{stroke,fill}%
}%
\begin{pgfscope}%
\pgfsys@transformshift{2.465772in}{1.758120in}%
\pgfsys@useobject{currentmarker}{}%
\end{pgfscope}%
\end{pgfscope}%
\begin{pgfscope}%
\pgfpathrectangle{\pgfqpoint{0.765000in}{0.660000in}}{\pgfqpoint{4.620000in}{4.620000in}}%
\pgfusepath{clip}%
\pgfsetrectcap%
\pgfsetroundjoin%
\pgfsetlinewidth{1.204500pt}%
\definecolor{currentstroke}{rgb}{1.000000,0.576471,0.309804}%
\pgfsetstrokecolor{currentstroke}%
\pgfsetdash{}{0pt}%
\pgfpathmoveto{\pgfqpoint{2.351752in}{3.577768in}}%
\pgfusepath{stroke}%
\end{pgfscope}%
\begin{pgfscope}%
\pgfpathrectangle{\pgfqpoint{0.765000in}{0.660000in}}{\pgfqpoint{4.620000in}{4.620000in}}%
\pgfusepath{clip}%
\pgfsetbuttcap%
\pgfsetroundjoin%
\definecolor{currentfill}{rgb}{1.000000,0.576471,0.309804}%
\pgfsetfillcolor{currentfill}%
\pgfsetlinewidth{1.003750pt}%
\definecolor{currentstroke}{rgb}{1.000000,0.576471,0.309804}%
\pgfsetstrokecolor{currentstroke}%
\pgfsetdash{}{0pt}%
\pgfsys@defobject{currentmarker}{\pgfqpoint{-0.033333in}{-0.033333in}}{\pgfqpoint{0.033333in}{0.033333in}}{%
\pgfpathmoveto{\pgfqpoint{0.000000in}{-0.033333in}}%
\pgfpathcurveto{\pgfqpoint{0.008840in}{-0.033333in}}{\pgfqpoint{0.017319in}{-0.029821in}}{\pgfqpoint{0.023570in}{-0.023570in}}%
\pgfpathcurveto{\pgfqpoint{0.029821in}{-0.017319in}}{\pgfqpoint{0.033333in}{-0.008840in}}{\pgfqpoint{0.033333in}{0.000000in}}%
\pgfpathcurveto{\pgfqpoint{0.033333in}{0.008840in}}{\pgfqpoint{0.029821in}{0.017319in}}{\pgfqpoint{0.023570in}{0.023570in}}%
\pgfpathcurveto{\pgfqpoint{0.017319in}{0.029821in}}{\pgfqpoint{0.008840in}{0.033333in}}{\pgfqpoint{0.000000in}{0.033333in}}%
\pgfpathcurveto{\pgfqpoint{-0.008840in}{0.033333in}}{\pgfqpoint{-0.017319in}{0.029821in}}{\pgfqpoint{-0.023570in}{0.023570in}}%
\pgfpathcurveto{\pgfqpoint{-0.029821in}{0.017319in}}{\pgfqpoint{-0.033333in}{0.008840in}}{\pgfqpoint{-0.033333in}{0.000000in}}%
\pgfpathcurveto{\pgfqpoint{-0.033333in}{-0.008840in}}{\pgfqpoint{-0.029821in}{-0.017319in}}{\pgfqpoint{-0.023570in}{-0.023570in}}%
\pgfpathcurveto{\pgfqpoint{-0.017319in}{-0.029821in}}{\pgfqpoint{-0.008840in}{-0.033333in}}{\pgfqpoint{0.000000in}{-0.033333in}}%
\pgfpathlineto{\pgfqpoint{0.000000in}{-0.033333in}}%
\pgfpathclose%
\pgfusepath{stroke,fill}%
}%
\begin{pgfscope}%
\pgfsys@transformshift{2.351752in}{3.577768in}%
\pgfsys@useobject{currentmarker}{}%
\end{pgfscope}%
\end{pgfscope}%
\begin{pgfscope}%
\pgfpathrectangle{\pgfqpoint{0.765000in}{0.660000in}}{\pgfqpoint{4.620000in}{4.620000in}}%
\pgfusepath{clip}%
\pgfsetrectcap%
\pgfsetroundjoin%
\pgfsetlinewidth{1.204500pt}%
\definecolor{currentstroke}{rgb}{1.000000,0.576471,0.309804}%
\pgfsetstrokecolor{currentstroke}%
\pgfsetdash{}{0pt}%
\pgfpathmoveto{\pgfqpoint{2.185914in}{3.494286in}}%
\pgfusepath{stroke}%
\end{pgfscope}%
\begin{pgfscope}%
\pgfpathrectangle{\pgfqpoint{0.765000in}{0.660000in}}{\pgfqpoint{4.620000in}{4.620000in}}%
\pgfusepath{clip}%
\pgfsetbuttcap%
\pgfsetroundjoin%
\definecolor{currentfill}{rgb}{1.000000,0.576471,0.309804}%
\pgfsetfillcolor{currentfill}%
\pgfsetlinewidth{1.003750pt}%
\definecolor{currentstroke}{rgb}{1.000000,0.576471,0.309804}%
\pgfsetstrokecolor{currentstroke}%
\pgfsetdash{}{0pt}%
\pgfsys@defobject{currentmarker}{\pgfqpoint{-0.033333in}{-0.033333in}}{\pgfqpoint{0.033333in}{0.033333in}}{%
\pgfpathmoveto{\pgfqpoint{0.000000in}{-0.033333in}}%
\pgfpathcurveto{\pgfqpoint{0.008840in}{-0.033333in}}{\pgfqpoint{0.017319in}{-0.029821in}}{\pgfqpoint{0.023570in}{-0.023570in}}%
\pgfpathcurveto{\pgfqpoint{0.029821in}{-0.017319in}}{\pgfqpoint{0.033333in}{-0.008840in}}{\pgfqpoint{0.033333in}{0.000000in}}%
\pgfpathcurveto{\pgfqpoint{0.033333in}{0.008840in}}{\pgfqpoint{0.029821in}{0.017319in}}{\pgfqpoint{0.023570in}{0.023570in}}%
\pgfpathcurveto{\pgfqpoint{0.017319in}{0.029821in}}{\pgfqpoint{0.008840in}{0.033333in}}{\pgfqpoint{0.000000in}{0.033333in}}%
\pgfpathcurveto{\pgfqpoint{-0.008840in}{0.033333in}}{\pgfqpoint{-0.017319in}{0.029821in}}{\pgfqpoint{-0.023570in}{0.023570in}}%
\pgfpathcurveto{\pgfqpoint{-0.029821in}{0.017319in}}{\pgfqpoint{-0.033333in}{0.008840in}}{\pgfqpoint{-0.033333in}{0.000000in}}%
\pgfpathcurveto{\pgfqpoint{-0.033333in}{-0.008840in}}{\pgfqpoint{-0.029821in}{-0.017319in}}{\pgfqpoint{-0.023570in}{-0.023570in}}%
\pgfpathcurveto{\pgfqpoint{-0.017319in}{-0.029821in}}{\pgfqpoint{-0.008840in}{-0.033333in}}{\pgfqpoint{0.000000in}{-0.033333in}}%
\pgfpathlineto{\pgfqpoint{0.000000in}{-0.033333in}}%
\pgfpathclose%
\pgfusepath{stroke,fill}%
}%
\begin{pgfscope}%
\pgfsys@transformshift{2.185914in}{3.494286in}%
\pgfsys@useobject{currentmarker}{}%
\end{pgfscope}%
\end{pgfscope}%
\begin{pgfscope}%
\pgfpathrectangle{\pgfqpoint{0.765000in}{0.660000in}}{\pgfqpoint{4.620000in}{4.620000in}}%
\pgfusepath{clip}%
\pgfsetrectcap%
\pgfsetroundjoin%
\pgfsetlinewidth{1.204500pt}%
\definecolor{currentstroke}{rgb}{1.000000,0.576471,0.309804}%
\pgfsetstrokecolor{currentstroke}%
\pgfsetdash{}{0pt}%
\pgfpathmoveto{\pgfqpoint{2.310318in}{4.226772in}}%
\pgfusepath{stroke}%
\end{pgfscope}%
\begin{pgfscope}%
\pgfpathrectangle{\pgfqpoint{0.765000in}{0.660000in}}{\pgfqpoint{4.620000in}{4.620000in}}%
\pgfusepath{clip}%
\pgfsetbuttcap%
\pgfsetroundjoin%
\definecolor{currentfill}{rgb}{1.000000,0.576471,0.309804}%
\pgfsetfillcolor{currentfill}%
\pgfsetlinewidth{1.003750pt}%
\definecolor{currentstroke}{rgb}{1.000000,0.576471,0.309804}%
\pgfsetstrokecolor{currentstroke}%
\pgfsetdash{}{0pt}%
\pgfsys@defobject{currentmarker}{\pgfqpoint{-0.033333in}{-0.033333in}}{\pgfqpoint{0.033333in}{0.033333in}}{%
\pgfpathmoveto{\pgfqpoint{0.000000in}{-0.033333in}}%
\pgfpathcurveto{\pgfqpoint{0.008840in}{-0.033333in}}{\pgfqpoint{0.017319in}{-0.029821in}}{\pgfqpoint{0.023570in}{-0.023570in}}%
\pgfpathcurveto{\pgfqpoint{0.029821in}{-0.017319in}}{\pgfqpoint{0.033333in}{-0.008840in}}{\pgfqpoint{0.033333in}{0.000000in}}%
\pgfpathcurveto{\pgfqpoint{0.033333in}{0.008840in}}{\pgfqpoint{0.029821in}{0.017319in}}{\pgfqpoint{0.023570in}{0.023570in}}%
\pgfpathcurveto{\pgfqpoint{0.017319in}{0.029821in}}{\pgfqpoint{0.008840in}{0.033333in}}{\pgfqpoint{0.000000in}{0.033333in}}%
\pgfpathcurveto{\pgfqpoint{-0.008840in}{0.033333in}}{\pgfqpoint{-0.017319in}{0.029821in}}{\pgfqpoint{-0.023570in}{0.023570in}}%
\pgfpathcurveto{\pgfqpoint{-0.029821in}{0.017319in}}{\pgfqpoint{-0.033333in}{0.008840in}}{\pgfqpoint{-0.033333in}{0.000000in}}%
\pgfpathcurveto{\pgfqpoint{-0.033333in}{-0.008840in}}{\pgfqpoint{-0.029821in}{-0.017319in}}{\pgfqpoint{-0.023570in}{-0.023570in}}%
\pgfpathcurveto{\pgfqpoint{-0.017319in}{-0.029821in}}{\pgfqpoint{-0.008840in}{-0.033333in}}{\pgfqpoint{0.000000in}{-0.033333in}}%
\pgfpathlineto{\pgfqpoint{0.000000in}{-0.033333in}}%
\pgfpathclose%
\pgfusepath{stroke,fill}%
}%
\begin{pgfscope}%
\pgfsys@transformshift{2.310318in}{4.226772in}%
\pgfsys@useobject{currentmarker}{}%
\end{pgfscope}%
\end{pgfscope}%
\begin{pgfscope}%
\pgfpathrectangle{\pgfqpoint{0.765000in}{0.660000in}}{\pgfqpoint{4.620000in}{4.620000in}}%
\pgfusepath{clip}%
\pgfsetrectcap%
\pgfsetroundjoin%
\pgfsetlinewidth{1.204500pt}%
\definecolor{currentstroke}{rgb}{1.000000,0.576471,0.309804}%
\pgfsetstrokecolor{currentstroke}%
\pgfsetdash{}{0pt}%
\pgfpathmoveto{\pgfqpoint{2.422467in}{2.531677in}}%
\pgfusepath{stroke}%
\end{pgfscope}%
\begin{pgfscope}%
\pgfpathrectangle{\pgfqpoint{0.765000in}{0.660000in}}{\pgfqpoint{4.620000in}{4.620000in}}%
\pgfusepath{clip}%
\pgfsetbuttcap%
\pgfsetroundjoin%
\definecolor{currentfill}{rgb}{1.000000,0.576471,0.309804}%
\pgfsetfillcolor{currentfill}%
\pgfsetlinewidth{1.003750pt}%
\definecolor{currentstroke}{rgb}{1.000000,0.576471,0.309804}%
\pgfsetstrokecolor{currentstroke}%
\pgfsetdash{}{0pt}%
\pgfsys@defobject{currentmarker}{\pgfqpoint{-0.033333in}{-0.033333in}}{\pgfqpoint{0.033333in}{0.033333in}}{%
\pgfpathmoveto{\pgfqpoint{0.000000in}{-0.033333in}}%
\pgfpathcurveto{\pgfqpoint{0.008840in}{-0.033333in}}{\pgfqpoint{0.017319in}{-0.029821in}}{\pgfqpoint{0.023570in}{-0.023570in}}%
\pgfpathcurveto{\pgfqpoint{0.029821in}{-0.017319in}}{\pgfqpoint{0.033333in}{-0.008840in}}{\pgfqpoint{0.033333in}{0.000000in}}%
\pgfpathcurveto{\pgfqpoint{0.033333in}{0.008840in}}{\pgfqpoint{0.029821in}{0.017319in}}{\pgfqpoint{0.023570in}{0.023570in}}%
\pgfpathcurveto{\pgfqpoint{0.017319in}{0.029821in}}{\pgfqpoint{0.008840in}{0.033333in}}{\pgfqpoint{0.000000in}{0.033333in}}%
\pgfpathcurveto{\pgfqpoint{-0.008840in}{0.033333in}}{\pgfqpoint{-0.017319in}{0.029821in}}{\pgfqpoint{-0.023570in}{0.023570in}}%
\pgfpathcurveto{\pgfqpoint{-0.029821in}{0.017319in}}{\pgfqpoint{-0.033333in}{0.008840in}}{\pgfqpoint{-0.033333in}{0.000000in}}%
\pgfpathcurveto{\pgfqpoint{-0.033333in}{-0.008840in}}{\pgfqpoint{-0.029821in}{-0.017319in}}{\pgfqpoint{-0.023570in}{-0.023570in}}%
\pgfpathcurveto{\pgfqpoint{-0.017319in}{-0.029821in}}{\pgfqpoint{-0.008840in}{-0.033333in}}{\pgfqpoint{0.000000in}{-0.033333in}}%
\pgfpathlineto{\pgfqpoint{0.000000in}{-0.033333in}}%
\pgfpathclose%
\pgfusepath{stroke,fill}%
}%
\begin{pgfscope}%
\pgfsys@transformshift{2.422467in}{2.531677in}%
\pgfsys@useobject{currentmarker}{}%
\end{pgfscope}%
\end{pgfscope}%
\begin{pgfscope}%
\pgfpathrectangle{\pgfqpoint{0.765000in}{0.660000in}}{\pgfqpoint{4.620000in}{4.620000in}}%
\pgfusepath{clip}%
\pgfsetrectcap%
\pgfsetroundjoin%
\pgfsetlinewidth{1.204500pt}%
\definecolor{currentstroke}{rgb}{1.000000,0.576471,0.309804}%
\pgfsetstrokecolor{currentstroke}%
\pgfsetdash{}{0pt}%
\pgfpathmoveto{\pgfqpoint{3.638058in}{4.067957in}}%
\pgfusepath{stroke}%
\end{pgfscope}%
\begin{pgfscope}%
\pgfpathrectangle{\pgfqpoint{0.765000in}{0.660000in}}{\pgfqpoint{4.620000in}{4.620000in}}%
\pgfusepath{clip}%
\pgfsetbuttcap%
\pgfsetroundjoin%
\definecolor{currentfill}{rgb}{1.000000,0.576471,0.309804}%
\pgfsetfillcolor{currentfill}%
\pgfsetlinewidth{1.003750pt}%
\definecolor{currentstroke}{rgb}{1.000000,0.576471,0.309804}%
\pgfsetstrokecolor{currentstroke}%
\pgfsetdash{}{0pt}%
\pgfsys@defobject{currentmarker}{\pgfqpoint{-0.033333in}{-0.033333in}}{\pgfqpoint{0.033333in}{0.033333in}}{%
\pgfpathmoveto{\pgfqpoint{0.000000in}{-0.033333in}}%
\pgfpathcurveto{\pgfqpoint{0.008840in}{-0.033333in}}{\pgfqpoint{0.017319in}{-0.029821in}}{\pgfqpoint{0.023570in}{-0.023570in}}%
\pgfpathcurveto{\pgfqpoint{0.029821in}{-0.017319in}}{\pgfqpoint{0.033333in}{-0.008840in}}{\pgfqpoint{0.033333in}{0.000000in}}%
\pgfpathcurveto{\pgfqpoint{0.033333in}{0.008840in}}{\pgfqpoint{0.029821in}{0.017319in}}{\pgfqpoint{0.023570in}{0.023570in}}%
\pgfpathcurveto{\pgfqpoint{0.017319in}{0.029821in}}{\pgfqpoint{0.008840in}{0.033333in}}{\pgfqpoint{0.000000in}{0.033333in}}%
\pgfpathcurveto{\pgfqpoint{-0.008840in}{0.033333in}}{\pgfqpoint{-0.017319in}{0.029821in}}{\pgfqpoint{-0.023570in}{0.023570in}}%
\pgfpathcurveto{\pgfqpoint{-0.029821in}{0.017319in}}{\pgfqpoint{-0.033333in}{0.008840in}}{\pgfqpoint{-0.033333in}{0.000000in}}%
\pgfpathcurveto{\pgfqpoint{-0.033333in}{-0.008840in}}{\pgfqpoint{-0.029821in}{-0.017319in}}{\pgfqpoint{-0.023570in}{-0.023570in}}%
\pgfpathcurveto{\pgfqpoint{-0.017319in}{-0.029821in}}{\pgfqpoint{-0.008840in}{-0.033333in}}{\pgfqpoint{0.000000in}{-0.033333in}}%
\pgfpathlineto{\pgfqpoint{0.000000in}{-0.033333in}}%
\pgfpathclose%
\pgfusepath{stroke,fill}%
}%
\begin{pgfscope}%
\pgfsys@transformshift{3.638058in}{4.067957in}%
\pgfsys@useobject{currentmarker}{}%
\end{pgfscope}%
\end{pgfscope}%
\begin{pgfscope}%
\pgfpathrectangle{\pgfqpoint{0.765000in}{0.660000in}}{\pgfqpoint{4.620000in}{4.620000in}}%
\pgfusepath{clip}%
\pgfsetrectcap%
\pgfsetroundjoin%
\pgfsetlinewidth{1.204500pt}%
\definecolor{currentstroke}{rgb}{1.000000,0.576471,0.309804}%
\pgfsetstrokecolor{currentstroke}%
\pgfsetdash{}{0pt}%
\pgfpathmoveto{\pgfqpoint{2.556667in}{3.552469in}}%
\pgfusepath{stroke}%
\end{pgfscope}%
\begin{pgfscope}%
\pgfpathrectangle{\pgfqpoint{0.765000in}{0.660000in}}{\pgfqpoint{4.620000in}{4.620000in}}%
\pgfusepath{clip}%
\pgfsetbuttcap%
\pgfsetroundjoin%
\definecolor{currentfill}{rgb}{1.000000,0.576471,0.309804}%
\pgfsetfillcolor{currentfill}%
\pgfsetlinewidth{1.003750pt}%
\definecolor{currentstroke}{rgb}{1.000000,0.576471,0.309804}%
\pgfsetstrokecolor{currentstroke}%
\pgfsetdash{}{0pt}%
\pgfsys@defobject{currentmarker}{\pgfqpoint{-0.033333in}{-0.033333in}}{\pgfqpoint{0.033333in}{0.033333in}}{%
\pgfpathmoveto{\pgfqpoint{0.000000in}{-0.033333in}}%
\pgfpathcurveto{\pgfqpoint{0.008840in}{-0.033333in}}{\pgfqpoint{0.017319in}{-0.029821in}}{\pgfqpoint{0.023570in}{-0.023570in}}%
\pgfpathcurveto{\pgfqpoint{0.029821in}{-0.017319in}}{\pgfqpoint{0.033333in}{-0.008840in}}{\pgfqpoint{0.033333in}{0.000000in}}%
\pgfpathcurveto{\pgfqpoint{0.033333in}{0.008840in}}{\pgfqpoint{0.029821in}{0.017319in}}{\pgfqpoint{0.023570in}{0.023570in}}%
\pgfpathcurveto{\pgfqpoint{0.017319in}{0.029821in}}{\pgfqpoint{0.008840in}{0.033333in}}{\pgfqpoint{0.000000in}{0.033333in}}%
\pgfpathcurveto{\pgfqpoint{-0.008840in}{0.033333in}}{\pgfqpoint{-0.017319in}{0.029821in}}{\pgfqpoint{-0.023570in}{0.023570in}}%
\pgfpathcurveto{\pgfqpoint{-0.029821in}{0.017319in}}{\pgfqpoint{-0.033333in}{0.008840in}}{\pgfqpoint{-0.033333in}{0.000000in}}%
\pgfpathcurveto{\pgfqpoint{-0.033333in}{-0.008840in}}{\pgfqpoint{-0.029821in}{-0.017319in}}{\pgfqpoint{-0.023570in}{-0.023570in}}%
\pgfpathcurveto{\pgfqpoint{-0.017319in}{-0.029821in}}{\pgfqpoint{-0.008840in}{-0.033333in}}{\pgfqpoint{0.000000in}{-0.033333in}}%
\pgfpathlineto{\pgfqpoint{0.000000in}{-0.033333in}}%
\pgfpathclose%
\pgfusepath{stroke,fill}%
}%
\begin{pgfscope}%
\pgfsys@transformshift{2.556667in}{3.552469in}%
\pgfsys@useobject{currentmarker}{}%
\end{pgfscope}%
\end{pgfscope}%
\begin{pgfscope}%
\pgfpathrectangle{\pgfqpoint{0.765000in}{0.660000in}}{\pgfqpoint{4.620000in}{4.620000in}}%
\pgfusepath{clip}%
\pgfsetrectcap%
\pgfsetroundjoin%
\pgfsetlinewidth{1.204500pt}%
\definecolor{currentstroke}{rgb}{1.000000,0.576471,0.309804}%
\pgfsetstrokecolor{currentstroke}%
\pgfsetdash{}{0pt}%
\pgfpathmoveto{\pgfqpoint{3.338592in}{3.977561in}}%
\pgfusepath{stroke}%
\end{pgfscope}%
\begin{pgfscope}%
\pgfpathrectangle{\pgfqpoint{0.765000in}{0.660000in}}{\pgfqpoint{4.620000in}{4.620000in}}%
\pgfusepath{clip}%
\pgfsetbuttcap%
\pgfsetroundjoin%
\definecolor{currentfill}{rgb}{1.000000,0.576471,0.309804}%
\pgfsetfillcolor{currentfill}%
\pgfsetlinewidth{1.003750pt}%
\definecolor{currentstroke}{rgb}{1.000000,0.576471,0.309804}%
\pgfsetstrokecolor{currentstroke}%
\pgfsetdash{}{0pt}%
\pgfsys@defobject{currentmarker}{\pgfqpoint{-0.033333in}{-0.033333in}}{\pgfqpoint{0.033333in}{0.033333in}}{%
\pgfpathmoveto{\pgfqpoint{0.000000in}{-0.033333in}}%
\pgfpathcurveto{\pgfqpoint{0.008840in}{-0.033333in}}{\pgfqpoint{0.017319in}{-0.029821in}}{\pgfqpoint{0.023570in}{-0.023570in}}%
\pgfpathcurveto{\pgfqpoint{0.029821in}{-0.017319in}}{\pgfqpoint{0.033333in}{-0.008840in}}{\pgfqpoint{0.033333in}{0.000000in}}%
\pgfpathcurveto{\pgfqpoint{0.033333in}{0.008840in}}{\pgfqpoint{0.029821in}{0.017319in}}{\pgfqpoint{0.023570in}{0.023570in}}%
\pgfpathcurveto{\pgfqpoint{0.017319in}{0.029821in}}{\pgfqpoint{0.008840in}{0.033333in}}{\pgfqpoint{0.000000in}{0.033333in}}%
\pgfpathcurveto{\pgfqpoint{-0.008840in}{0.033333in}}{\pgfqpoint{-0.017319in}{0.029821in}}{\pgfqpoint{-0.023570in}{0.023570in}}%
\pgfpathcurveto{\pgfqpoint{-0.029821in}{0.017319in}}{\pgfqpoint{-0.033333in}{0.008840in}}{\pgfqpoint{-0.033333in}{0.000000in}}%
\pgfpathcurveto{\pgfqpoint{-0.033333in}{-0.008840in}}{\pgfqpoint{-0.029821in}{-0.017319in}}{\pgfqpoint{-0.023570in}{-0.023570in}}%
\pgfpathcurveto{\pgfqpoint{-0.017319in}{-0.029821in}}{\pgfqpoint{-0.008840in}{-0.033333in}}{\pgfqpoint{0.000000in}{-0.033333in}}%
\pgfpathlineto{\pgfqpoint{0.000000in}{-0.033333in}}%
\pgfpathclose%
\pgfusepath{stroke,fill}%
}%
\begin{pgfscope}%
\pgfsys@transformshift{3.338592in}{3.977561in}%
\pgfsys@useobject{currentmarker}{}%
\end{pgfscope}%
\end{pgfscope}%
\begin{pgfscope}%
\pgfpathrectangle{\pgfqpoint{0.765000in}{0.660000in}}{\pgfqpoint{4.620000in}{4.620000in}}%
\pgfusepath{clip}%
\pgfsetrectcap%
\pgfsetroundjoin%
\pgfsetlinewidth{1.204500pt}%
\definecolor{currentstroke}{rgb}{1.000000,0.576471,0.309804}%
\pgfsetstrokecolor{currentstroke}%
\pgfsetdash{}{0pt}%
\pgfpathmoveto{\pgfqpoint{3.129614in}{4.893934in}}%
\pgfusepath{stroke}%
\end{pgfscope}%
\begin{pgfscope}%
\pgfpathrectangle{\pgfqpoint{0.765000in}{0.660000in}}{\pgfqpoint{4.620000in}{4.620000in}}%
\pgfusepath{clip}%
\pgfsetbuttcap%
\pgfsetroundjoin%
\definecolor{currentfill}{rgb}{1.000000,0.576471,0.309804}%
\pgfsetfillcolor{currentfill}%
\pgfsetlinewidth{1.003750pt}%
\definecolor{currentstroke}{rgb}{1.000000,0.576471,0.309804}%
\pgfsetstrokecolor{currentstroke}%
\pgfsetdash{}{0pt}%
\pgfsys@defobject{currentmarker}{\pgfqpoint{-0.033333in}{-0.033333in}}{\pgfqpoint{0.033333in}{0.033333in}}{%
\pgfpathmoveto{\pgfqpoint{0.000000in}{-0.033333in}}%
\pgfpathcurveto{\pgfqpoint{0.008840in}{-0.033333in}}{\pgfqpoint{0.017319in}{-0.029821in}}{\pgfqpoint{0.023570in}{-0.023570in}}%
\pgfpathcurveto{\pgfqpoint{0.029821in}{-0.017319in}}{\pgfqpoint{0.033333in}{-0.008840in}}{\pgfqpoint{0.033333in}{0.000000in}}%
\pgfpathcurveto{\pgfqpoint{0.033333in}{0.008840in}}{\pgfqpoint{0.029821in}{0.017319in}}{\pgfqpoint{0.023570in}{0.023570in}}%
\pgfpathcurveto{\pgfqpoint{0.017319in}{0.029821in}}{\pgfqpoint{0.008840in}{0.033333in}}{\pgfqpoint{0.000000in}{0.033333in}}%
\pgfpathcurveto{\pgfqpoint{-0.008840in}{0.033333in}}{\pgfqpoint{-0.017319in}{0.029821in}}{\pgfqpoint{-0.023570in}{0.023570in}}%
\pgfpathcurveto{\pgfqpoint{-0.029821in}{0.017319in}}{\pgfqpoint{-0.033333in}{0.008840in}}{\pgfqpoint{-0.033333in}{0.000000in}}%
\pgfpathcurveto{\pgfqpoint{-0.033333in}{-0.008840in}}{\pgfqpoint{-0.029821in}{-0.017319in}}{\pgfqpoint{-0.023570in}{-0.023570in}}%
\pgfpathcurveto{\pgfqpoint{-0.017319in}{-0.029821in}}{\pgfqpoint{-0.008840in}{-0.033333in}}{\pgfqpoint{0.000000in}{-0.033333in}}%
\pgfpathlineto{\pgfqpoint{0.000000in}{-0.033333in}}%
\pgfpathclose%
\pgfusepath{stroke,fill}%
}%
\begin{pgfscope}%
\pgfsys@transformshift{3.129614in}{4.893934in}%
\pgfsys@useobject{currentmarker}{}%
\end{pgfscope}%
\end{pgfscope}%
\begin{pgfscope}%
\pgfpathrectangle{\pgfqpoint{0.765000in}{0.660000in}}{\pgfqpoint{4.620000in}{4.620000in}}%
\pgfusepath{clip}%
\pgfsetrectcap%
\pgfsetroundjoin%
\pgfsetlinewidth{1.204500pt}%
\definecolor{currentstroke}{rgb}{1.000000,0.576471,0.309804}%
\pgfsetstrokecolor{currentstroke}%
\pgfsetdash{}{0pt}%
\pgfpathmoveto{\pgfqpoint{4.192903in}{1.370546in}}%
\pgfusepath{stroke}%
\end{pgfscope}%
\begin{pgfscope}%
\pgfpathrectangle{\pgfqpoint{0.765000in}{0.660000in}}{\pgfqpoint{4.620000in}{4.620000in}}%
\pgfusepath{clip}%
\pgfsetbuttcap%
\pgfsetroundjoin%
\definecolor{currentfill}{rgb}{1.000000,0.576471,0.309804}%
\pgfsetfillcolor{currentfill}%
\pgfsetlinewidth{1.003750pt}%
\definecolor{currentstroke}{rgb}{1.000000,0.576471,0.309804}%
\pgfsetstrokecolor{currentstroke}%
\pgfsetdash{}{0pt}%
\pgfsys@defobject{currentmarker}{\pgfqpoint{-0.033333in}{-0.033333in}}{\pgfqpoint{0.033333in}{0.033333in}}{%
\pgfpathmoveto{\pgfqpoint{0.000000in}{-0.033333in}}%
\pgfpathcurveto{\pgfqpoint{0.008840in}{-0.033333in}}{\pgfqpoint{0.017319in}{-0.029821in}}{\pgfqpoint{0.023570in}{-0.023570in}}%
\pgfpathcurveto{\pgfqpoint{0.029821in}{-0.017319in}}{\pgfqpoint{0.033333in}{-0.008840in}}{\pgfqpoint{0.033333in}{0.000000in}}%
\pgfpathcurveto{\pgfqpoint{0.033333in}{0.008840in}}{\pgfqpoint{0.029821in}{0.017319in}}{\pgfqpoint{0.023570in}{0.023570in}}%
\pgfpathcurveto{\pgfqpoint{0.017319in}{0.029821in}}{\pgfqpoint{0.008840in}{0.033333in}}{\pgfqpoint{0.000000in}{0.033333in}}%
\pgfpathcurveto{\pgfqpoint{-0.008840in}{0.033333in}}{\pgfqpoint{-0.017319in}{0.029821in}}{\pgfqpoint{-0.023570in}{0.023570in}}%
\pgfpathcurveto{\pgfqpoint{-0.029821in}{0.017319in}}{\pgfqpoint{-0.033333in}{0.008840in}}{\pgfqpoint{-0.033333in}{0.000000in}}%
\pgfpathcurveto{\pgfqpoint{-0.033333in}{-0.008840in}}{\pgfqpoint{-0.029821in}{-0.017319in}}{\pgfqpoint{-0.023570in}{-0.023570in}}%
\pgfpathcurveto{\pgfqpoint{-0.017319in}{-0.029821in}}{\pgfqpoint{-0.008840in}{-0.033333in}}{\pgfqpoint{0.000000in}{-0.033333in}}%
\pgfpathlineto{\pgfqpoint{0.000000in}{-0.033333in}}%
\pgfpathclose%
\pgfusepath{stroke,fill}%
}%
\begin{pgfscope}%
\pgfsys@transformshift{4.192903in}{1.370546in}%
\pgfsys@useobject{currentmarker}{}%
\end{pgfscope}%
\end{pgfscope}%
\begin{pgfscope}%
\pgfpathrectangle{\pgfqpoint{0.765000in}{0.660000in}}{\pgfqpoint{4.620000in}{4.620000in}}%
\pgfusepath{clip}%
\pgfsetrectcap%
\pgfsetroundjoin%
\pgfsetlinewidth{1.204500pt}%
\definecolor{currentstroke}{rgb}{1.000000,0.576471,0.309804}%
\pgfsetstrokecolor{currentstroke}%
\pgfsetdash{}{0pt}%
\pgfpathmoveto{\pgfqpoint{2.033582in}{3.052409in}}%
\pgfusepath{stroke}%
\end{pgfscope}%
\begin{pgfscope}%
\pgfpathrectangle{\pgfqpoint{0.765000in}{0.660000in}}{\pgfqpoint{4.620000in}{4.620000in}}%
\pgfusepath{clip}%
\pgfsetbuttcap%
\pgfsetroundjoin%
\definecolor{currentfill}{rgb}{1.000000,0.576471,0.309804}%
\pgfsetfillcolor{currentfill}%
\pgfsetlinewidth{1.003750pt}%
\definecolor{currentstroke}{rgb}{1.000000,0.576471,0.309804}%
\pgfsetstrokecolor{currentstroke}%
\pgfsetdash{}{0pt}%
\pgfsys@defobject{currentmarker}{\pgfqpoint{-0.033333in}{-0.033333in}}{\pgfqpoint{0.033333in}{0.033333in}}{%
\pgfpathmoveto{\pgfqpoint{0.000000in}{-0.033333in}}%
\pgfpathcurveto{\pgfqpoint{0.008840in}{-0.033333in}}{\pgfqpoint{0.017319in}{-0.029821in}}{\pgfqpoint{0.023570in}{-0.023570in}}%
\pgfpathcurveto{\pgfqpoint{0.029821in}{-0.017319in}}{\pgfqpoint{0.033333in}{-0.008840in}}{\pgfqpoint{0.033333in}{0.000000in}}%
\pgfpathcurveto{\pgfqpoint{0.033333in}{0.008840in}}{\pgfqpoint{0.029821in}{0.017319in}}{\pgfqpoint{0.023570in}{0.023570in}}%
\pgfpathcurveto{\pgfqpoint{0.017319in}{0.029821in}}{\pgfqpoint{0.008840in}{0.033333in}}{\pgfqpoint{0.000000in}{0.033333in}}%
\pgfpathcurveto{\pgfqpoint{-0.008840in}{0.033333in}}{\pgfqpoint{-0.017319in}{0.029821in}}{\pgfqpoint{-0.023570in}{0.023570in}}%
\pgfpathcurveto{\pgfqpoint{-0.029821in}{0.017319in}}{\pgfqpoint{-0.033333in}{0.008840in}}{\pgfqpoint{-0.033333in}{0.000000in}}%
\pgfpathcurveto{\pgfqpoint{-0.033333in}{-0.008840in}}{\pgfqpoint{-0.029821in}{-0.017319in}}{\pgfqpoint{-0.023570in}{-0.023570in}}%
\pgfpathcurveto{\pgfqpoint{-0.017319in}{-0.029821in}}{\pgfqpoint{-0.008840in}{-0.033333in}}{\pgfqpoint{0.000000in}{-0.033333in}}%
\pgfpathlineto{\pgfqpoint{0.000000in}{-0.033333in}}%
\pgfpathclose%
\pgfusepath{stroke,fill}%
}%
\begin{pgfscope}%
\pgfsys@transformshift{2.033582in}{3.052409in}%
\pgfsys@useobject{currentmarker}{}%
\end{pgfscope}%
\end{pgfscope}%
\begin{pgfscope}%
\pgfpathrectangle{\pgfqpoint{0.765000in}{0.660000in}}{\pgfqpoint{4.620000in}{4.620000in}}%
\pgfusepath{clip}%
\pgfsetrectcap%
\pgfsetroundjoin%
\pgfsetlinewidth{1.204500pt}%
\definecolor{currentstroke}{rgb}{1.000000,0.576471,0.309804}%
\pgfsetstrokecolor{currentstroke}%
\pgfsetdash{}{0pt}%
\pgfpathmoveto{\pgfqpoint{2.640771in}{2.146971in}}%
\pgfusepath{stroke}%
\end{pgfscope}%
\begin{pgfscope}%
\pgfpathrectangle{\pgfqpoint{0.765000in}{0.660000in}}{\pgfqpoint{4.620000in}{4.620000in}}%
\pgfusepath{clip}%
\pgfsetbuttcap%
\pgfsetroundjoin%
\definecolor{currentfill}{rgb}{1.000000,0.576471,0.309804}%
\pgfsetfillcolor{currentfill}%
\pgfsetlinewidth{1.003750pt}%
\definecolor{currentstroke}{rgb}{1.000000,0.576471,0.309804}%
\pgfsetstrokecolor{currentstroke}%
\pgfsetdash{}{0pt}%
\pgfsys@defobject{currentmarker}{\pgfqpoint{-0.033333in}{-0.033333in}}{\pgfqpoint{0.033333in}{0.033333in}}{%
\pgfpathmoveto{\pgfqpoint{0.000000in}{-0.033333in}}%
\pgfpathcurveto{\pgfqpoint{0.008840in}{-0.033333in}}{\pgfqpoint{0.017319in}{-0.029821in}}{\pgfqpoint{0.023570in}{-0.023570in}}%
\pgfpathcurveto{\pgfqpoint{0.029821in}{-0.017319in}}{\pgfqpoint{0.033333in}{-0.008840in}}{\pgfqpoint{0.033333in}{0.000000in}}%
\pgfpathcurveto{\pgfqpoint{0.033333in}{0.008840in}}{\pgfqpoint{0.029821in}{0.017319in}}{\pgfqpoint{0.023570in}{0.023570in}}%
\pgfpathcurveto{\pgfqpoint{0.017319in}{0.029821in}}{\pgfqpoint{0.008840in}{0.033333in}}{\pgfqpoint{0.000000in}{0.033333in}}%
\pgfpathcurveto{\pgfqpoint{-0.008840in}{0.033333in}}{\pgfqpoint{-0.017319in}{0.029821in}}{\pgfqpoint{-0.023570in}{0.023570in}}%
\pgfpathcurveto{\pgfqpoint{-0.029821in}{0.017319in}}{\pgfqpoint{-0.033333in}{0.008840in}}{\pgfqpoint{-0.033333in}{0.000000in}}%
\pgfpathcurveto{\pgfqpoint{-0.033333in}{-0.008840in}}{\pgfqpoint{-0.029821in}{-0.017319in}}{\pgfqpoint{-0.023570in}{-0.023570in}}%
\pgfpathcurveto{\pgfqpoint{-0.017319in}{-0.029821in}}{\pgfqpoint{-0.008840in}{-0.033333in}}{\pgfqpoint{0.000000in}{-0.033333in}}%
\pgfpathlineto{\pgfqpoint{0.000000in}{-0.033333in}}%
\pgfpathclose%
\pgfusepath{stroke,fill}%
}%
\begin{pgfscope}%
\pgfsys@transformshift{2.640771in}{2.146971in}%
\pgfsys@useobject{currentmarker}{}%
\end{pgfscope}%
\end{pgfscope}%
\begin{pgfscope}%
\pgfpathrectangle{\pgfqpoint{0.765000in}{0.660000in}}{\pgfqpoint{4.620000in}{4.620000in}}%
\pgfusepath{clip}%
\pgfsetrectcap%
\pgfsetroundjoin%
\pgfsetlinewidth{1.204500pt}%
\definecolor{currentstroke}{rgb}{1.000000,0.576471,0.309804}%
\pgfsetstrokecolor{currentstroke}%
\pgfsetdash{}{0pt}%
\pgfpathmoveto{\pgfqpoint{2.409507in}{1.178779in}}%
\pgfusepath{stroke}%
\end{pgfscope}%
\begin{pgfscope}%
\pgfpathrectangle{\pgfqpoint{0.765000in}{0.660000in}}{\pgfqpoint{4.620000in}{4.620000in}}%
\pgfusepath{clip}%
\pgfsetbuttcap%
\pgfsetroundjoin%
\definecolor{currentfill}{rgb}{1.000000,0.576471,0.309804}%
\pgfsetfillcolor{currentfill}%
\pgfsetlinewidth{1.003750pt}%
\definecolor{currentstroke}{rgb}{1.000000,0.576471,0.309804}%
\pgfsetstrokecolor{currentstroke}%
\pgfsetdash{}{0pt}%
\pgfsys@defobject{currentmarker}{\pgfqpoint{-0.033333in}{-0.033333in}}{\pgfqpoint{0.033333in}{0.033333in}}{%
\pgfpathmoveto{\pgfqpoint{0.000000in}{-0.033333in}}%
\pgfpathcurveto{\pgfqpoint{0.008840in}{-0.033333in}}{\pgfqpoint{0.017319in}{-0.029821in}}{\pgfqpoint{0.023570in}{-0.023570in}}%
\pgfpathcurveto{\pgfqpoint{0.029821in}{-0.017319in}}{\pgfqpoint{0.033333in}{-0.008840in}}{\pgfqpoint{0.033333in}{0.000000in}}%
\pgfpathcurveto{\pgfqpoint{0.033333in}{0.008840in}}{\pgfqpoint{0.029821in}{0.017319in}}{\pgfqpoint{0.023570in}{0.023570in}}%
\pgfpathcurveto{\pgfqpoint{0.017319in}{0.029821in}}{\pgfqpoint{0.008840in}{0.033333in}}{\pgfqpoint{0.000000in}{0.033333in}}%
\pgfpathcurveto{\pgfqpoint{-0.008840in}{0.033333in}}{\pgfqpoint{-0.017319in}{0.029821in}}{\pgfqpoint{-0.023570in}{0.023570in}}%
\pgfpathcurveto{\pgfqpoint{-0.029821in}{0.017319in}}{\pgfqpoint{-0.033333in}{0.008840in}}{\pgfqpoint{-0.033333in}{0.000000in}}%
\pgfpathcurveto{\pgfqpoint{-0.033333in}{-0.008840in}}{\pgfqpoint{-0.029821in}{-0.017319in}}{\pgfqpoint{-0.023570in}{-0.023570in}}%
\pgfpathcurveto{\pgfqpoint{-0.017319in}{-0.029821in}}{\pgfqpoint{-0.008840in}{-0.033333in}}{\pgfqpoint{0.000000in}{-0.033333in}}%
\pgfpathlineto{\pgfqpoint{0.000000in}{-0.033333in}}%
\pgfpathclose%
\pgfusepath{stroke,fill}%
}%
\begin{pgfscope}%
\pgfsys@transformshift{2.409507in}{1.178779in}%
\pgfsys@useobject{currentmarker}{}%
\end{pgfscope}%
\end{pgfscope}%
\begin{pgfscope}%
\pgfpathrectangle{\pgfqpoint{0.765000in}{0.660000in}}{\pgfqpoint{4.620000in}{4.620000in}}%
\pgfusepath{clip}%
\pgfsetrectcap%
\pgfsetroundjoin%
\pgfsetlinewidth{1.204500pt}%
\definecolor{currentstroke}{rgb}{1.000000,0.576471,0.309804}%
\pgfsetstrokecolor{currentstroke}%
\pgfsetdash{}{0pt}%
\pgfpathmoveto{\pgfqpoint{2.724838in}{3.656969in}}%
\pgfusepath{stroke}%
\end{pgfscope}%
\begin{pgfscope}%
\pgfpathrectangle{\pgfqpoint{0.765000in}{0.660000in}}{\pgfqpoint{4.620000in}{4.620000in}}%
\pgfusepath{clip}%
\pgfsetbuttcap%
\pgfsetroundjoin%
\definecolor{currentfill}{rgb}{1.000000,0.576471,0.309804}%
\pgfsetfillcolor{currentfill}%
\pgfsetlinewidth{1.003750pt}%
\definecolor{currentstroke}{rgb}{1.000000,0.576471,0.309804}%
\pgfsetstrokecolor{currentstroke}%
\pgfsetdash{}{0pt}%
\pgfsys@defobject{currentmarker}{\pgfqpoint{-0.033333in}{-0.033333in}}{\pgfqpoint{0.033333in}{0.033333in}}{%
\pgfpathmoveto{\pgfqpoint{0.000000in}{-0.033333in}}%
\pgfpathcurveto{\pgfqpoint{0.008840in}{-0.033333in}}{\pgfqpoint{0.017319in}{-0.029821in}}{\pgfqpoint{0.023570in}{-0.023570in}}%
\pgfpathcurveto{\pgfqpoint{0.029821in}{-0.017319in}}{\pgfqpoint{0.033333in}{-0.008840in}}{\pgfqpoint{0.033333in}{0.000000in}}%
\pgfpathcurveto{\pgfqpoint{0.033333in}{0.008840in}}{\pgfqpoint{0.029821in}{0.017319in}}{\pgfqpoint{0.023570in}{0.023570in}}%
\pgfpathcurveto{\pgfqpoint{0.017319in}{0.029821in}}{\pgfqpoint{0.008840in}{0.033333in}}{\pgfqpoint{0.000000in}{0.033333in}}%
\pgfpathcurveto{\pgfqpoint{-0.008840in}{0.033333in}}{\pgfqpoint{-0.017319in}{0.029821in}}{\pgfqpoint{-0.023570in}{0.023570in}}%
\pgfpathcurveto{\pgfqpoint{-0.029821in}{0.017319in}}{\pgfqpoint{-0.033333in}{0.008840in}}{\pgfqpoint{-0.033333in}{0.000000in}}%
\pgfpathcurveto{\pgfqpoint{-0.033333in}{-0.008840in}}{\pgfqpoint{-0.029821in}{-0.017319in}}{\pgfqpoint{-0.023570in}{-0.023570in}}%
\pgfpathcurveto{\pgfqpoint{-0.017319in}{-0.029821in}}{\pgfqpoint{-0.008840in}{-0.033333in}}{\pgfqpoint{0.000000in}{-0.033333in}}%
\pgfpathlineto{\pgfqpoint{0.000000in}{-0.033333in}}%
\pgfpathclose%
\pgfusepath{stroke,fill}%
}%
\begin{pgfscope}%
\pgfsys@transformshift{2.724838in}{3.656969in}%
\pgfsys@useobject{currentmarker}{}%
\end{pgfscope}%
\end{pgfscope}%
\begin{pgfscope}%
\pgfpathrectangle{\pgfqpoint{0.765000in}{0.660000in}}{\pgfqpoint{4.620000in}{4.620000in}}%
\pgfusepath{clip}%
\pgfsetrectcap%
\pgfsetroundjoin%
\pgfsetlinewidth{1.204500pt}%
\definecolor{currentstroke}{rgb}{1.000000,0.576471,0.309804}%
\pgfsetstrokecolor{currentstroke}%
\pgfsetdash{}{0pt}%
\pgfpathmoveto{\pgfqpoint{2.144568in}{2.965188in}}%
\pgfusepath{stroke}%
\end{pgfscope}%
\begin{pgfscope}%
\pgfpathrectangle{\pgfqpoint{0.765000in}{0.660000in}}{\pgfqpoint{4.620000in}{4.620000in}}%
\pgfusepath{clip}%
\pgfsetbuttcap%
\pgfsetroundjoin%
\definecolor{currentfill}{rgb}{1.000000,0.576471,0.309804}%
\pgfsetfillcolor{currentfill}%
\pgfsetlinewidth{1.003750pt}%
\definecolor{currentstroke}{rgb}{1.000000,0.576471,0.309804}%
\pgfsetstrokecolor{currentstroke}%
\pgfsetdash{}{0pt}%
\pgfsys@defobject{currentmarker}{\pgfqpoint{-0.033333in}{-0.033333in}}{\pgfqpoint{0.033333in}{0.033333in}}{%
\pgfpathmoveto{\pgfqpoint{0.000000in}{-0.033333in}}%
\pgfpathcurveto{\pgfqpoint{0.008840in}{-0.033333in}}{\pgfqpoint{0.017319in}{-0.029821in}}{\pgfqpoint{0.023570in}{-0.023570in}}%
\pgfpathcurveto{\pgfqpoint{0.029821in}{-0.017319in}}{\pgfqpoint{0.033333in}{-0.008840in}}{\pgfqpoint{0.033333in}{0.000000in}}%
\pgfpathcurveto{\pgfqpoint{0.033333in}{0.008840in}}{\pgfqpoint{0.029821in}{0.017319in}}{\pgfqpoint{0.023570in}{0.023570in}}%
\pgfpathcurveto{\pgfqpoint{0.017319in}{0.029821in}}{\pgfqpoint{0.008840in}{0.033333in}}{\pgfqpoint{0.000000in}{0.033333in}}%
\pgfpathcurveto{\pgfqpoint{-0.008840in}{0.033333in}}{\pgfqpoint{-0.017319in}{0.029821in}}{\pgfqpoint{-0.023570in}{0.023570in}}%
\pgfpathcurveto{\pgfqpoint{-0.029821in}{0.017319in}}{\pgfqpoint{-0.033333in}{0.008840in}}{\pgfqpoint{-0.033333in}{0.000000in}}%
\pgfpathcurveto{\pgfqpoint{-0.033333in}{-0.008840in}}{\pgfqpoint{-0.029821in}{-0.017319in}}{\pgfqpoint{-0.023570in}{-0.023570in}}%
\pgfpathcurveto{\pgfqpoint{-0.017319in}{-0.029821in}}{\pgfqpoint{-0.008840in}{-0.033333in}}{\pgfqpoint{0.000000in}{-0.033333in}}%
\pgfpathlineto{\pgfqpoint{0.000000in}{-0.033333in}}%
\pgfpathclose%
\pgfusepath{stroke,fill}%
}%
\begin{pgfscope}%
\pgfsys@transformshift{2.144568in}{2.965188in}%
\pgfsys@useobject{currentmarker}{}%
\end{pgfscope}%
\end{pgfscope}%
\begin{pgfscope}%
\pgfpathrectangle{\pgfqpoint{0.765000in}{0.660000in}}{\pgfqpoint{4.620000in}{4.620000in}}%
\pgfusepath{clip}%
\pgfsetrectcap%
\pgfsetroundjoin%
\pgfsetlinewidth{1.204500pt}%
\definecolor{currentstroke}{rgb}{1.000000,0.576471,0.309804}%
\pgfsetstrokecolor{currentstroke}%
\pgfsetdash{}{0pt}%
\pgfpathmoveto{\pgfqpoint{3.044015in}{4.338936in}}%
\pgfusepath{stroke}%
\end{pgfscope}%
\begin{pgfscope}%
\pgfpathrectangle{\pgfqpoint{0.765000in}{0.660000in}}{\pgfqpoint{4.620000in}{4.620000in}}%
\pgfusepath{clip}%
\pgfsetbuttcap%
\pgfsetroundjoin%
\definecolor{currentfill}{rgb}{1.000000,0.576471,0.309804}%
\pgfsetfillcolor{currentfill}%
\pgfsetlinewidth{1.003750pt}%
\definecolor{currentstroke}{rgb}{1.000000,0.576471,0.309804}%
\pgfsetstrokecolor{currentstroke}%
\pgfsetdash{}{0pt}%
\pgfsys@defobject{currentmarker}{\pgfqpoint{-0.033333in}{-0.033333in}}{\pgfqpoint{0.033333in}{0.033333in}}{%
\pgfpathmoveto{\pgfqpoint{0.000000in}{-0.033333in}}%
\pgfpathcurveto{\pgfqpoint{0.008840in}{-0.033333in}}{\pgfqpoint{0.017319in}{-0.029821in}}{\pgfqpoint{0.023570in}{-0.023570in}}%
\pgfpathcurveto{\pgfqpoint{0.029821in}{-0.017319in}}{\pgfqpoint{0.033333in}{-0.008840in}}{\pgfqpoint{0.033333in}{0.000000in}}%
\pgfpathcurveto{\pgfqpoint{0.033333in}{0.008840in}}{\pgfqpoint{0.029821in}{0.017319in}}{\pgfqpoint{0.023570in}{0.023570in}}%
\pgfpathcurveto{\pgfqpoint{0.017319in}{0.029821in}}{\pgfqpoint{0.008840in}{0.033333in}}{\pgfqpoint{0.000000in}{0.033333in}}%
\pgfpathcurveto{\pgfqpoint{-0.008840in}{0.033333in}}{\pgfqpoint{-0.017319in}{0.029821in}}{\pgfqpoint{-0.023570in}{0.023570in}}%
\pgfpathcurveto{\pgfqpoint{-0.029821in}{0.017319in}}{\pgfqpoint{-0.033333in}{0.008840in}}{\pgfqpoint{-0.033333in}{0.000000in}}%
\pgfpathcurveto{\pgfqpoint{-0.033333in}{-0.008840in}}{\pgfqpoint{-0.029821in}{-0.017319in}}{\pgfqpoint{-0.023570in}{-0.023570in}}%
\pgfpathcurveto{\pgfqpoint{-0.017319in}{-0.029821in}}{\pgfqpoint{-0.008840in}{-0.033333in}}{\pgfqpoint{0.000000in}{-0.033333in}}%
\pgfpathlineto{\pgfqpoint{0.000000in}{-0.033333in}}%
\pgfpathclose%
\pgfusepath{stroke,fill}%
}%
\begin{pgfscope}%
\pgfsys@transformshift{3.044015in}{4.338936in}%
\pgfsys@useobject{currentmarker}{}%
\end{pgfscope}%
\end{pgfscope}%
\begin{pgfscope}%
\pgfpathrectangle{\pgfqpoint{0.765000in}{0.660000in}}{\pgfqpoint{4.620000in}{4.620000in}}%
\pgfusepath{clip}%
\pgfsetrectcap%
\pgfsetroundjoin%
\pgfsetlinewidth{1.204500pt}%
\definecolor{currentstroke}{rgb}{1.000000,0.576471,0.309804}%
\pgfsetstrokecolor{currentstroke}%
\pgfsetdash{}{0pt}%
\pgfpathmoveto{\pgfqpoint{2.335251in}{3.797662in}}%
\pgfusepath{stroke}%
\end{pgfscope}%
\begin{pgfscope}%
\pgfpathrectangle{\pgfqpoint{0.765000in}{0.660000in}}{\pgfqpoint{4.620000in}{4.620000in}}%
\pgfusepath{clip}%
\pgfsetbuttcap%
\pgfsetroundjoin%
\definecolor{currentfill}{rgb}{1.000000,0.576471,0.309804}%
\pgfsetfillcolor{currentfill}%
\pgfsetlinewidth{1.003750pt}%
\definecolor{currentstroke}{rgb}{1.000000,0.576471,0.309804}%
\pgfsetstrokecolor{currentstroke}%
\pgfsetdash{}{0pt}%
\pgfsys@defobject{currentmarker}{\pgfqpoint{-0.033333in}{-0.033333in}}{\pgfqpoint{0.033333in}{0.033333in}}{%
\pgfpathmoveto{\pgfqpoint{0.000000in}{-0.033333in}}%
\pgfpathcurveto{\pgfqpoint{0.008840in}{-0.033333in}}{\pgfqpoint{0.017319in}{-0.029821in}}{\pgfqpoint{0.023570in}{-0.023570in}}%
\pgfpathcurveto{\pgfqpoint{0.029821in}{-0.017319in}}{\pgfqpoint{0.033333in}{-0.008840in}}{\pgfqpoint{0.033333in}{0.000000in}}%
\pgfpathcurveto{\pgfqpoint{0.033333in}{0.008840in}}{\pgfqpoint{0.029821in}{0.017319in}}{\pgfqpoint{0.023570in}{0.023570in}}%
\pgfpathcurveto{\pgfqpoint{0.017319in}{0.029821in}}{\pgfqpoint{0.008840in}{0.033333in}}{\pgfqpoint{0.000000in}{0.033333in}}%
\pgfpathcurveto{\pgfqpoint{-0.008840in}{0.033333in}}{\pgfqpoint{-0.017319in}{0.029821in}}{\pgfqpoint{-0.023570in}{0.023570in}}%
\pgfpathcurveto{\pgfqpoint{-0.029821in}{0.017319in}}{\pgfqpoint{-0.033333in}{0.008840in}}{\pgfqpoint{-0.033333in}{0.000000in}}%
\pgfpathcurveto{\pgfqpoint{-0.033333in}{-0.008840in}}{\pgfqpoint{-0.029821in}{-0.017319in}}{\pgfqpoint{-0.023570in}{-0.023570in}}%
\pgfpathcurveto{\pgfqpoint{-0.017319in}{-0.029821in}}{\pgfqpoint{-0.008840in}{-0.033333in}}{\pgfqpoint{0.000000in}{-0.033333in}}%
\pgfpathlineto{\pgfqpoint{0.000000in}{-0.033333in}}%
\pgfpathclose%
\pgfusepath{stroke,fill}%
}%
\begin{pgfscope}%
\pgfsys@transformshift{2.335251in}{3.797662in}%
\pgfsys@useobject{currentmarker}{}%
\end{pgfscope}%
\end{pgfscope}%
\begin{pgfscope}%
\pgfpathrectangle{\pgfqpoint{0.765000in}{0.660000in}}{\pgfqpoint{4.620000in}{4.620000in}}%
\pgfusepath{clip}%
\pgfsetrectcap%
\pgfsetroundjoin%
\pgfsetlinewidth{1.204500pt}%
\definecolor{currentstroke}{rgb}{1.000000,0.576471,0.309804}%
\pgfsetstrokecolor{currentstroke}%
\pgfsetdash{}{0pt}%
\pgfpathmoveto{\pgfqpoint{2.609790in}{1.353408in}}%
\pgfusepath{stroke}%
\end{pgfscope}%
\begin{pgfscope}%
\pgfpathrectangle{\pgfqpoint{0.765000in}{0.660000in}}{\pgfqpoint{4.620000in}{4.620000in}}%
\pgfusepath{clip}%
\pgfsetbuttcap%
\pgfsetroundjoin%
\definecolor{currentfill}{rgb}{1.000000,0.576471,0.309804}%
\pgfsetfillcolor{currentfill}%
\pgfsetlinewidth{1.003750pt}%
\definecolor{currentstroke}{rgb}{1.000000,0.576471,0.309804}%
\pgfsetstrokecolor{currentstroke}%
\pgfsetdash{}{0pt}%
\pgfsys@defobject{currentmarker}{\pgfqpoint{-0.033333in}{-0.033333in}}{\pgfqpoint{0.033333in}{0.033333in}}{%
\pgfpathmoveto{\pgfqpoint{0.000000in}{-0.033333in}}%
\pgfpathcurveto{\pgfqpoint{0.008840in}{-0.033333in}}{\pgfqpoint{0.017319in}{-0.029821in}}{\pgfqpoint{0.023570in}{-0.023570in}}%
\pgfpathcurveto{\pgfqpoint{0.029821in}{-0.017319in}}{\pgfqpoint{0.033333in}{-0.008840in}}{\pgfqpoint{0.033333in}{0.000000in}}%
\pgfpathcurveto{\pgfqpoint{0.033333in}{0.008840in}}{\pgfqpoint{0.029821in}{0.017319in}}{\pgfqpoint{0.023570in}{0.023570in}}%
\pgfpathcurveto{\pgfqpoint{0.017319in}{0.029821in}}{\pgfqpoint{0.008840in}{0.033333in}}{\pgfqpoint{0.000000in}{0.033333in}}%
\pgfpathcurveto{\pgfqpoint{-0.008840in}{0.033333in}}{\pgfqpoint{-0.017319in}{0.029821in}}{\pgfqpoint{-0.023570in}{0.023570in}}%
\pgfpathcurveto{\pgfqpoint{-0.029821in}{0.017319in}}{\pgfqpoint{-0.033333in}{0.008840in}}{\pgfqpoint{-0.033333in}{0.000000in}}%
\pgfpathcurveto{\pgfqpoint{-0.033333in}{-0.008840in}}{\pgfqpoint{-0.029821in}{-0.017319in}}{\pgfqpoint{-0.023570in}{-0.023570in}}%
\pgfpathcurveto{\pgfqpoint{-0.017319in}{-0.029821in}}{\pgfqpoint{-0.008840in}{-0.033333in}}{\pgfqpoint{0.000000in}{-0.033333in}}%
\pgfpathlineto{\pgfqpoint{0.000000in}{-0.033333in}}%
\pgfpathclose%
\pgfusepath{stroke,fill}%
}%
\begin{pgfscope}%
\pgfsys@transformshift{2.609790in}{1.353408in}%
\pgfsys@useobject{currentmarker}{}%
\end{pgfscope}%
\end{pgfscope}%
\begin{pgfscope}%
\pgfpathrectangle{\pgfqpoint{0.765000in}{0.660000in}}{\pgfqpoint{4.620000in}{4.620000in}}%
\pgfusepath{clip}%
\pgfsetrectcap%
\pgfsetroundjoin%
\pgfsetlinewidth{1.204500pt}%
\definecolor{currentstroke}{rgb}{1.000000,0.576471,0.309804}%
\pgfsetstrokecolor{currentstroke}%
\pgfsetdash{}{0pt}%
\pgfpathmoveto{\pgfqpoint{3.963409in}{2.452525in}}%
\pgfusepath{stroke}%
\end{pgfscope}%
\begin{pgfscope}%
\pgfpathrectangle{\pgfqpoint{0.765000in}{0.660000in}}{\pgfqpoint{4.620000in}{4.620000in}}%
\pgfusepath{clip}%
\pgfsetbuttcap%
\pgfsetroundjoin%
\definecolor{currentfill}{rgb}{1.000000,0.576471,0.309804}%
\pgfsetfillcolor{currentfill}%
\pgfsetlinewidth{1.003750pt}%
\definecolor{currentstroke}{rgb}{1.000000,0.576471,0.309804}%
\pgfsetstrokecolor{currentstroke}%
\pgfsetdash{}{0pt}%
\pgfsys@defobject{currentmarker}{\pgfqpoint{-0.033333in}{-0.033333in}}{\pgfqpoint{0.033333in}{0.033333in}}{%
\pgfpathmoveto{\pgfqpoint{0.000000in}{-0.033333in}}%
\pgfpathcurveto{\pgfqpoint{0.008840in}{-0.033333in}}{\pgfqpoint{0.017319in}{-0.029821in}}{\pgfqpoint{0.023570in}{-0.023570in}}%
\pgfpathcurveto{\pgfqpoint{0.029821in}{-0.017319in}}{\pgfqpoint{0.033333in}{-0.008840in}}{\pgfqpoint{0.033333in}{0.000000in}}%
\pgfpathcurveto{\pgfqpoint{0.033333in}{0.008840in}}{\pgfqpoint{0.029821in}{0.017319in}}{\pgfqpoint{0.023570in}{0.023570in}}%
\pgfpathcurveto{\pgfqpoint{0.017319in}{0.029821in}}{\pgfqpoint{0.008840in}{0.033333in}}{\pgfqpoint{0.000000in}{0.033333in}}%
\pgfpathcurveto{\pgfqpoint{-0.008840in}{0.033333in}}{\pgfqpoint{-0.017319in}{0.029821in}}{\pgfqpoint{-0.023570in}{0.023570in}}%
\pgfpathcurveto{\pgfqpoint{-0.029821in}{0.017319in}}{\pgfqpoint{-0.033333in}{0.008840in}}{\pgfqpoint{-0.033333in}{0.000000in}}%
\pgfpathcurveto{\pgfqpoint{-0.033333in}{-0.008840in}}{\pgfqpoint{-0.029821in}{-0.017319in}}{\pgfqpoint{-0.023570in}{-0.023570in}}%
\pgfpathcurveto{\pgfqpoint{-0.017319in}{-0.029821in}}{\pgfqpoint{-0.008840in}{-0.033333in}}{\pgfqpoint{0.000000in}{-0.033333in}}%
\pgfpathlineto{\pgfqpoint{0.000000in}{-0.033333in}}%
\pgfpathclose%
\pgfusepath{stroke,fill}%
}%
\begin{pgfscope}%
\pgfsys@transformshift{3.963409in}{2.452525in}%
\pgfsys@useobject{currentmarker}{}%
\end{pgfscope}%
\end{pgfscope}%
\begin{pgfscope}%
\pgfpathrectangle{\pgfqpoint{0.765000in}{0.660000in}}{\pgfqpoint{4.620000in}{4.620000in}}%
\pgfusepath{clip}%
\pgfsetrectcap%
\pgfsetroundjoin%
\pgfsetlinewidth{1.204500pt}%
\definecolor{currentstroke}{rgb}{1.000000,0.576471,0.309804}%
\pgfsetstrokecolor{currentstroke}%
\pgfsetdash{}{0pt}%
\pgfpathmoveto{\pgfqpoint{3.286641in}{2.212006in}}%
\pgfusepath{stroke}%
\end{pgfscope}%
\begin{pgfscope}%
\pgfpathrectangle{\pgfqpoint{0.765000in}{0.660000in}}{\pgfqpoint{4.620000in}{4.620000in}}%
\pgfusepath{clip}%
\pgfsetbuttcap%
\pgfsetroundjoin%
\definecolor{currentfill}{rgb}{1.000000,0.576471,0.309804}%
\pgfsetfillcolor{currentfill}%
\pgfsetlinewidth{1.003750pt}%
\definecolor{currentstroke}{rgb}{1.000000,0.576471,0.309804}%
\pgfsetstrokecolor{currentstroke}%
\pgfsetdash{}{0pt}%
\pgfsys@defobject{currentmarker}{\pgfqpoint{-0.033333in}{-0.033333in}}{\pgfqpoint{0.033333in}{0.033333in}}{%
\pgfpathmoveto{\pgfqpoint{0.000000in}{-0.033333in}}%
\pgfpathcurveto{\pgfqpoint{0.008840in}{-0.033333in}}{\pgfqpoint{0.017319in}{-0.029821in}}{\pgfqpoint{0.023570in}{-0.023570in}}%
\pgfpathcurveto{\pgfqpoint{0.029821in}{-0.017319in}}{\pgfqpoint{0.033333in}{-0.008840in}}{\pgfqpoint{0.033333in}{0.000000in}}%
\pgfpathcurveto{\pgfqpoint{0.033333in}{0.008840in}}{\pgfqpoint{0.029821in}{0.017319in}}{\pgfqpoint{0.023570in}{0.023570in}}%
\pgfpathcurveto{\pgfqpoint{0.017319in}{0.029821in}}{\pgfqpoint{0.008840in}{0.033333in}}{\pgfqpoint{0.000000in}{0.033333in}}%
\pgfpathcurveto{\pgfqpoint{-0.008840in}{0.033333in}}{\pgfqpoint{-0.017319in}{0.029821in}}{\pgfqpoint{-0.023570in}{0.023570in}}%
\pgfpathcurveto{\pgfqpoint{-0.029821in}{0.017319in}}{\pgfqpoint{-0.033333in}{0.008840in}}{\pgfqpoint{-0.033333in}{0.000000in}}%
\pgfpathcurveto{\pgfqpoint{-0.033333in}{-0.008840in}}{\pgfqpoint{-0.029821in}{-0.017319in}}{\pgfqpoint{-0.023570in}{-0.023570in}}%
\pgfpathcurveto{\pgfqpoint{-0.017319in}{-0.029821in}}{\pgfqpoint{-0.008840in}{-0.033333in}}{\pgfqpoint{0.000000in}{-0.033333in}}%
\pgfpathlineto{\pgfqpoint{0.000000in}{-0.033333in}}%
\pgfpathclose%
\pgfusepath{stroke,fill}%
}%
\begin{pgfscope}%
\pgfsys@transformshift{3.286641in}{2.212006in}%
\pgfsys@useobject{currentmarker}{}%
\end{pgfscope}%
\end{pgfscope}%
\begin{pgfscope}%
\pgfpathrectangle{\pgfqpoint{0.765000in}{0.660000in}}{\pgfqpoint{4.620000in}{4.620000in}}%
\pgfusepath{clip}%
\pgfsetrectcap%
\pgfsetroundjoin%
\pgfsetlinewidth{1.204500pt}%
\definecolor{currentstroke}{rgb}{1.000000,0.576471,0.309804}%
\pgfsetstrokecolor{currentstroke}%
\pgfsetdash{}{0pt}%
\pgfpathmoveto{\pgfqpoint{1.612957in}{3.401627in}}%
\pgfusepath{stroke}%
\end{pgfscope}%
\begin{pgfscope}%
\pgfpathrectangle{\pgfqpoint{0.765000in}{0.660000in}}{\pgfqpoint{4.620000in}{4.620000in}}%
\pgfusepath{clip}%
\pgfsetbuttcap%
\pgfsetroundjoin%
\definecolor{currentfill}{rgb}{1.000000,0.576471,0.309804}%
\pgfsetfillcolor{currentfill}%
\pgfsetlinewidth{1.003750pt}%
\definecolor{currentstroke}{rgb}{1.000000,0.576471,0.309804}%
\pgfsetstrokecolor{currentstroke}%
\pgfsetdash{}{0pt}%
\pgfsys@defobject{currentmarker}{\pgfqpoint{-0.033333in}{-0.033333in}}{\pgfqpoint{0.033333in}{0.033333in}}{%
\pgfpathmoveto{\pgfqpoint{0.000000in}{-0.033333in}}%
\pgfpathcurveto{\pgfqpoint{0.008840in}{-0.033333in}}{\pgfqpoint{0.017319in}{-0.029821in}}{\pgfqpoint{0.023570in}{-0.023570in}}%
\pgfpathcurveto{\pgfqpoint{0.029821in}{-0.017319in}}{\pgfqpoint{0.033333in}{-0.008840in}}{\pgfqpoint{0.033333in}{0.000000in}}%
\pgfpathcurveto{\pgfqpoint{0.033333in}{0.008840in}}{\pgfqpoint{0.029821in}{0.017319in}}{\pgfqpoint{0.023570in}{0.023570in}}%
\pgfpathcurveto{\pgfqpoint{0.017319in}{0.029821in}}{\pgfqpoint{0.008840in}{0.033333in}}{\pgfqpoint{0.000000in}{0.033333in}}%
\pgfpathcurveto{\pgfqpoint{-0.008840in}{0.033333in}}{\pgfqpoint{-0.017319in}{0.029821in}}{\pgfqpoint{-0.023570in}{0.023570in}}%
\pgfpathcurveto{\pgfqpoint{-0.029821in}{0.017319in}}{\pgfqpoint{-0.033333in}{0.008840in}}{\pgfqpoint{-0.033333in}{0.000000in}}%
\pgfpathcurveto{\pgfqpoint{-0.033333in}{-0.008840in}}{\pgfqpoint{-0.029821in}{-0.017319in}}{\pgfqpoint{-0.023570in}{-0.023570in}}%
\pgfpathcurveto{\pgfqpoint{-0.017319in}{-0.029821in}}{\pgfqpoint{-0.008840in}{-0.033333in}}{\pgfqpoint{0.000000in}{-0.033333in}}%
\pgfpathlineto{\pgfqpoint{0.000000in}{-0.033333in}}%
\pgfpathclose%
\pgfusepath{stroke,fill}%
}%
\begin{pgfscope}%
\pgfsys@transformshift{1.612957in}{3.401627in}%
\pgfsys@useobject{currentmarker}{}%
\end{pgfscope}%
\end{pgfscope}%
\begin{pgfscope}%
\pgfpathrectangle{\pgfqpoint{0.765000in}{0.660000in}}{\pgfqpoint{4.620000in}{4.620000in}}%
\pgfusepath{clip}%
\pgfsetrectcap%
\pgfsetroundjoin%
\pgfsetlinewidth{1.204500pt}%
\definecolor{currentstroke}{rgb}{1.000000,0.576471,0.309804}%
\pgfsetstrokecolor{currentstroke}%
\pgfsetdash{}{0pt}%
\pgfpathmoveto{\pgfqpoint{2.654806in}{4.368094in}}%
\pgfusepath{stroke}%
\end{pgfscope}%
\begin{pgfscope}%
\pgfpathrectangle{\pgfqpoint{0.765000in}{0.660000in}}{\pgfqpoint{4.620000in}{4.620000in}}%
\pgfusepath{clip}%
\pgfsetbuttcap%
\pgfsetroundjoin%
\definecolor{currentfill}{rgb}{1.000000,0.576471,0.309804}%
\pgfsetfillcolor{currentfill}%
\pgfsetlinewidth{1.003750pt}%
\definecolor{currentstroke}{rgb}{1.000000,0.576471,0.309804}%
\pgfsetstrokecolor{currentstroke}%
\pgfsetdash{}{0pt}%
\pgfsys@defobject{currentmarker}{\pgfqpoint{-0.033333in}{-0.033333in}}{\pgfqpoint{0.033333in}{0.033333in}}{%
\pgfpathmoveto{\pgfqpoint{0.000000in}{-0.033333in}}%
\pgfpathcurveto{\pgfqpoint{0.008840in}{-0.033333in}}{\pgfqpoint{0.017319in}{-0.029821in}}{\pgfqpoint{0.023570in}{-0.023570in}}%
\pgfpathcurveto{\pgfqpoint{0.029821in}{-0.017319in}}{\pgfqpoint{0.033333in}{-0.008840in}}{\pgfqpoint{0.033333in}{0.000000in}}%
\pgfpathcurveto{\pgfqpoint{0.033333in}{0.008840in}}{\pgfqpoint{0.029821in}{0.017319in}}{\pgfqpoint{0.023570in}{0.023570in}}%
\pgfpathcurveto{\pgfqpoint{0.017319in}{0.029821in}}{\pgfqpoint{0.008840in}{0.033333in}}{\pgfqpoint{0.000000in}{0.033333in}}%
\pgfpathcurveto{\pgfqpoint{-0.008840in}{0.033333in}}{\pgfqpoint{-0.017319in}{0.029821in}}{\pgfqpoint{-0.023570in}{0.023570in}}%
\pgfpathcurveto{\pgfqpoint{-0.029821in}{0.017319in}}{\pgfqpoint{-0.033333in}{0.008840in}}{\pgfqpoint{-0.033333in}{0.000000in}}%
\pgfpathcurveto{\pgfqpoint{-0.033333in}{-0.008840in}}{\pgfqpoint{-0.029821in}{-0.017319in}}{\pgfqpoint{-0.023570in}{-0.023570in}}%
\pgfpathcurveto{\pgfqpoint{-0.017319in}{-0.029821in}}{\pgfqpoint{-0.008840in}{-0.033333in}}{\pgfqpoint{0.000000in}{-0.033333in}}%
\pgfpathlineto{\pgfqpoint{0.000000in}{-0.033333in}}%
\pgfpathclose%
\pgfusepath{stroke,fill}%
}%
\begin{pgfscope}%
\pgfsys@transformshift{2.654806in}{4.368094in}%
\pgfsys@useobject{currentmarker}{}%
\end{pgfscope}%
\end{pgfscope}%
\begin{pgfscope}%
\pgfpathrectangle{\pgfqpoint{0.765000in}{0.660000in}}{\pgfqpoint{4.620000in}{4.620000in}}%
\pgfusepath{clip}%
\pgfsetrectcap%
\pgfsetroundjoin%
\pgfsetlinewidth{1.204500pt}%
\definecolor{currentstroke}{rgb}{1.000000,0.576471,0.309804}%
\pgfsetstrokecolor{currentstroke}%
\pgfsetdash{}{0pt}%
\pgfpathmoveto{\pgfqpoint{3.600780in}{3.643618in}}%
\pgfusepath{stroke}%
\end{pgfscope}%
\begin{pgfscope}%
\pgfpathrectangle{\pgfqpoint{0.765000in}{0.660000in}}{\pgfqpoint{4.620000in}{4.620000in}}%
\pgfusepath{clip}%
\pgfsetbuttcap%
\pgfsetroundjoin%
\definecolor{currentfill}{rgb}{1.000000,0.576471,0.309804}%
\pgfsetfillcolor{currentfill}%
\pgfsetlinewidth{1.003750pt}%
\definecolor{currentstroke}{rgb}{1.000000,0.576471,0.309804}%
\pgfsetstrokecolor{currentstroke}%
\pgfsetdash{}{0pt}%
\pgfsys@defobject{currentmarker}{\pgfqpoint{-0.033333in}{-0.033333in}}{\pgfqpoint{0.033333in}{0.033333in}}{%
\pgfpathmoveto{\pgfqpoint{0.000000in}{-0.033333in}}%
\pgfpathcurveto{\pgfqpoint{0.008840in}{-0.033333in}}{\pgfqpoint{0.017319in}{-0.029821in}}{\pgfqpoint{0.023570in}{-0.023570in}}%
\pgfpathcurveto{\pgfqpoint{0.029821in}{-0.017319in}}{\pgfqpoint{0.033333in}{-0.008840in}}{\pgfqpoint{0.033333in}{0.000000in}}%
\pgfpathcurveto{\pgfqpoint{0.033333in}{0.008840in}}{\pgfqpoint{0.029821in}{0.017319in}}{\pgfqpoint{0.023570in}{0.023570in}}%
\pgfpathcurveto{\pgfqpoint{0.017319in}{0.029821in}}{\pgfqpoint{0.008840in}{0.033333in}}{\pgfqpoint{0.000000in}{0.033333in}}%
\pgfpathcurveto{\pgfqpoint{-0.008840in}{0.033333in}}{\pgfqpoint{-0.017319in}{0.029821in}}{\pgfqpoint{-0.023570in}{0.023570in}}%
\pgfpathcurveto{\pgfqpoint{-0.029821in}{0.017319in}}{\pgfqpoint{-0.033333in}{0.008840in}}{\pgfqpoint{-0.033333in}{0.000000in}}%
\pgfpathcurveto{\pgfqpoint{-0.033333in}{-0.008840in}}{\pgfqpoint{-0.029821in}{-0.017319in}}{\pgfqpoint{-0.023570in}{-0.023570in}}%
\pgfpathcurveto{\pgfqpoint{-0.017319in}{-0.029821in}}{\pgfqpoint{-0.008840in}{-0.033333in}}{\pgfqpoint{0.000000in}{-0.033333in}}%
\pgfpathlineto{\pgfqpoint{0.000000in}{-0.033333in}}%
\pgfpathclose%
\pgfusepath{stroke,fill}%
}%
\begin{pgfscope}%
\pgfsys@transformshift{3.600780in}{3.643618in}%
\pgfsys@useobject{currentmarker}{}%
\end{pgfscope}%
\end{pgfscope}%
\begin{pgfscope}%
\pgfpathrectangle{\pgfqpoint{0.765000in}{0.660000in}}{\pgfqpoint{4.620000in}{4.620000in}}%
\pgfusepath{clip}%
\pgfsetrectcap%
\pgfsetroundjoin%
\pgfsetlinewidth{1.204500pt}%
\definecolor{currentstroke}{rgb}{1.000000,0.576471,0.309804}%
\pgfsetstrokecolor{currentstroke}%
\pgfsetdash{}{0pt}%
\pgfpathmoveto{\pgfqpoint{1.758098in}{3.695680in}}%
\pgfusepath{stroke}%
\end{pgfscope}%
\begin{pgfscope}%
\pgfpathrectangle{\pgfqpoint{0.765000in}{0.660000in}}{\pgfqpoint{4.620000in}{4.620000in}}%
\pgfusepath{clip}%
\pgfsetbuttcap%
\pgfsetroundjoin%
\definecolor{currentfill}{rgb}{1.000000,0.576471,0.309804}%
\pgfsetfillcolor{currentfill}%
\pgfsetlinewidth{1.003750pt}%
\definecolor{currentstroke}{rgb}{1.000000,0.576471,0.309804}%
\pgfsetstrokecolor{currentstroke}%
\pgfsetdash{}{0pt}%
\pgfsys@defobject{currentmarker}{\pgfqpoint{-0.033333in}{-0.033333in}}{\pgfqpoint{0.033333in}{0.033333in}}{%
\pgfpathmoveto{\pgfqpoint{0.000000in}{-0.033333in}}%
\pgfpathcurveto{\pgfqpoint{0.008840in}{-0.033333in}}{\pgfqpoint{0.017319in}{-0.029821in}}{\pgfqpoint{0.023570in}{-0.023570in}}%
\pgfpathcurveto{\pgfqpoint{0.029821in}{-0.017319in}}{\pgfqpoint{0.033333in}{-0.008840in}}{\pgfqpoint{0.033333in}{0.000000in}}%
\pgfpathcurveto{\pgfqpoint{0.033333in}{0.008840in}}{\pgfqpoint{0.029821in}{0.017319in}}{\pgfqpoint{0.023570in}{0.023570in}}%
\pgfpathcurveto{\pgfqpoint{0.017319in}{0.029821in}}{\pgfqpoint{0.008840in}{0.033333in}}{\pgfqpoint{0.000000in}{0.033333in}}%
\pgfpathcurveto{\pgfqpoint{-0.008840in}{0.033333in}}{\pgfqpoint{-0.017319in}{0.029821in}}{\pgfqpoint{-0.023570in}{0.023570in}}%
\pgfpathcurveto{\pgfqpoint{-0.029821in}{0.017319in}}{\pgfqpoint{-0.033333in}{0.008840in}}{\pgfqpoint{-0.033333in}{0.000000in}}%
\pgfpathcurveto{\pgfqpoint{-0.033333in}{-0.008840in}}{\pgfqpoint{-0.029821in}{-0.017319in}}{\pgfqpoint{-0.023570in}{-0.023570in}}%
\pgfpathcurveto{\pgfqpoint{-0.017319in}{-0.029821in}}{\pgfqpoint{-0.008840in}{-0.033333in}}{\pgfqpoint{0.000000in}{-0.033333in}}%
\pgfpathlineto{\pgfqpoint{0.000000in}{-0.033333in}}%
\pgfpathclose%
\pgfusepath{stroke,fill}%
}%
\begin{pgfscope}%
\pgfsys@transformshift{1.758098in}{3.695680in}%
\pgfsys@useobject{currentmarker}{}%
\end{pgfscope}%
\end{pgfscope}%
\begin{pgfscope}%
\pgfpathrectangle{\pgfqpoint{0.765000in}{0.660000in}}{\pgfqpoint{4.620000in}{4.620000in}}%
\pgfusepath{clip}%
\pgfsetrectcap%
\pgfsetroundjoin%
\pgfsetlinewidth{1.204500pt}%
\definecolor{currentstroke}{rgb}{1.000000,0.576471,0.309804}%
\pgfsetstrokecolor{currentstroke}%
\pgfsetdash{}{0pt}%
\pgfpathmoveto{\pgfqpoint{1.783496in}{4.135000in}}%
\pgfusepath{stroke}%
\end{pgfscope}%
\begin{pgfscope}%
\pgfpathrectangle{\pgfqpoint{0.765000in}{0.660000in}}{\pgfqpoint{4.620000in}{4.620000in}}%
\pgfusepath{clip}%
\pgfsetbuttcap%
\pgfsetroundjoin%
\definecolor{currentfill}{rgb}{1.000000,0.576471,0.309804}%
\pgfsetfillcolor{currentfill}%
\pgfsetlinewidth{1.003750pt}%
\definecolor{currentstroke}{rgb}{1.000000,0.576471,0.309804}%
\pgfsetstrokecolor{currentstroke}%
\pgfsetdash{}{0pt}%
\pgfsys@defobject{currentmarker}{\pgfqpoint{-0.033333in}{-0.033333in}}{\pgfqpoint{0.033333in}{0.033333in}}{%
\pgfpathmoveto{\pgfqpoint{0.000000in}{-0.033333in}}%
\pgfpathcurveto{\pgfqpoint{0.008840in}{-0.033333in}}{\pgfqpoint{0.017319in}{-0.029821in}}{\pgfqpoint{0.023570in}{-0.023570in}}%
\pgfpathcurveto{\pgfqpoint{0.029821in}{-0.017319in}}{\pgfqpoint{0.033333in}{-0.008840in}}{\pgfqpoint{0.033333in}{0.000000in}}%
\pgfpathcurveto{\pgfqpoint{0.033333in}{0.008840in}}{\pgfqpoint{0.029821in}{0.017319in}}{\pgfqpoint{0.023570in}{0.023570in}}%
\pgfpathcurveto{\pgfqpoint{0.017319in}{0.029821in}}{\pgfqpoint{0.008840in}{0.033333in}}{\pgfqpoint{0.000000in}{0.033333in}}%
\pgfpathcurveto{\pgfqpoint{-0.008840in}{0.033333in}}{\pgfqpoint{-0.017319in}{0.029821in}}{\pgfqpoint{-0.023570in}{0.023570in}}%
\pgfpathcurveto{\pgfqpoint{-0.029821in}{0.017319in}}{\pgfqpoint{-0.033333in}{0.008840in}}{\pgfqpoint{-0.033333in}{0.000000in}}%
\pgfpathcurveto{\pgfqpoint{-0.033333in}{-0.008840in}}{\pgfqpoint{-0.029821in}{-0.017319in}}{\pgfqpoint{-0.023570in}{-0.023570in}}%
\pgfpathcurveto{\pgfqpoint{-0.017319in}{-0.029821in}}{\pgfqpoint{-0.008840in}{-0.033333in}}{\pgfqpoint{0.000000in}{-0.033333in}}%
\pgfpathlineto{\pgfqpoint{0.000000in}{-0.033333in}}%
\pgfpathclose%
\pgfusepath{stroke,fill}%
}%
\begin{pgfscope}%
\pgfsys@transformshift{1.783496in}{4.135000in}%
\pgfsys@useobject{currentmarker}{}%
\end{pgfscope}%
\end{pgfscope}%
\begin{pgfscope}%
\pgfpathrectangle{\pgfqpoint{0.765000in}{0.660000in}}{\pgfqpoint{4.620000in}{4.620000in}}%
\pgfusepath{clip}%
\pgfsetrectcap%
\pgfsetroundjoin%
\pgfsetlinewidth{1.204500pt}%
\definecolor{currentstroke}{rgb}{1.000000,0.576471,0.309804}%
\pgfsetstrokecolor{currentstroke}%
\pgfsetdash{}{0pt}%
\pgfpathmoveto{\pgfqpoint{3.986832in}{3.959663in}}%
\pgfusepath{stroke}%
\end{pgfscope}%
\begin{pgfscope}%
\pgfpathrectangle{\pgfqpoint{0.765000in}{0.660000in}}{\pgfqpoint{4.620000in}{4.620000in}}%
\pgfusepath{clip}%
\pgfsetbuttcap%
\pgfsetroundjoin%
\definecolor{currentfill}{rgb}{1.000000,0.576471,0.309804}%
\pgfsetfillcolor{currentfill}%
\pgfsetlinewidth{1.003750pt}%
\definecolor{currentstroke}{rgb}{1.000000,0.576471,0.309804}%
\pgfsetstrokecolor{currentstroke}%
\pgfsetdash{}{0pt}%
\pgfsys@defobject{currentmarker}{\pgfqpoint{-0.033333in}{-0.033333in}}{\pgfqpoint{0.033333in}{0.033333in}}{%
\pgfpathmoveto{\pgfqpoint{0.000000in}{-0.033333in}}%
\pgfpathcurveto{\pgfqpoint{0.008840in}{-0.033333in}}{\pgfqpoint{0.017319in}{-0.029821in}}{\pgfqpoint{0.023570in}{-0.023570in}}%
\pgfpathcurveto{\pgfqpoint{0.029821in}{-0.017319in}}{\pgfqpoint{0.033333in}{-0.008840in}}{\pgfqpoint{0.033333in}{0.000000in}}%
\pgfpathcurveto{\pgfqpoint{0.033333in}{0.008840in}}{\pgfqpoint{0.029821in}{0.017319in}}{\pgfqpoint{0.023570in}{0.023570in}}%
\pgfpathcurveto{\pgfqpoint{0.017319in}{0.029821in}}{\pgfqpoint{0.008840in}{0.033333in}}{\pgfqpoint{0.000000in}{0.033333in}}%
\pgfpathcurveto{\pgfqpoint{-0.008840in}{0.033333in}}{\pgfqpoint{-0.017319in}{0.029821in}}{\pgfqpoint{-0.023570in}{0.023570in}}%
\pgfpathcurveto{\pgfqpoint{-0.029821in}{0.017319in}}{\pgfqpoint{-0.033333in}{0.008840in}}{\pgfqpoint{-0.033333in}{0.000000in}}%
\pgfpathcurveto{\pgfqpoint{-0.033333in}{-0.008840in}}{\pgfqpoint{-0.029821in}{-0.017319in}}{\pgfqpoint{-0.023570in}{-0.023570in}}%
\pgfpathcurveto{\pgfqpoint{-0.017319in}{-0.029821in}}{\pgfqpoint{-0.008840in}{-0.033333in}}{\pgfqpoint{0.000000in}{-0.033333in}}%
\pgfpathlineto{\pgfqpoint{0.000000in}{-0.033333in}}%
\pgfpathclose%
\pgfusepath{stroke,fill}%
}%
\begin{pgfscope}%
\pgfsys@transformshift{3.986832in}{3.959663in}%
\pgfsys@useobject{currentmarker}{}%
\end{pgfscope}%
\end{pgfscope}%
\begin{pgfscope}%
\pgfpathrectangle{\pgfqpoint{0.765000in}{0.660000in}}{\pgfqpoint{4.620000in}{4.620000in}}%
\pgfusepath{clip}%
\pgfsetrectcap%
\pgfsetroundjoin%
\pgfsetlinewidth{1.204500pt}%
\definecolor{currentstroke}{rgb}{1.000000,0.576471,0.309804}%
\pgfsetstrokecolor{currentstroke}%
\pgfsetdash{}{0pt}%
\pgfpathmoveto{\pgfqpoint{2.628738in}{3.644561in}}%
\pgfusepath{stroke}%
\end{pgfscope}%
\begin{pgfscope}%
\pgfpathrectangle{\pgfqpoint{0.765000in}{0.660000in}}{\pgfqpoint{4.620000in}{4.620000in}}%
\pgfusepath{clip}%
\pgfsetbuttcap%
\pgfsetroundjoin%
\definecolor{currentfill}{rgb}{1.000000,0.576471,0.309804}%
\pgfsetfillcolor{currentfill}%
\pgfsetlinewidth{1.003750pt}%
\definecolor{currentstroke}{rgb}{1.000000,0.576471,0.309804}%
\pgfsetstrokecolor{currentstroke}%
\pgfsetdash{}{0pt}%
\pgfsys@defobject{currentmarker}{\pgfqpoint{-0.033333in}{-0.033333in}}{\pgfqpoint{0.033333in}{0.033333in}}{%
\pgfpathmoveto{\pgfqpoint{0.000000in}{-0.033333in}}%
\pgfpathcurveto{\pgfqpoint{0.008840in}{-0.033333in}}{\pgfqpoint{0.017319in}{-0.029821in}}{\pgfqpoint{0.023570in}{-0.023570in}}%
\pgfpathcurveto{\pgfqpoint{0.029821in}{-0.017319in}}{\pgfqpoint{0.033333in}{-0.008840in}}{\pgfqpoint{0.033333in}{0.000000in}}%
\pgfpathcurveto{\pgfqpoint{0.033333in}{0.008840in}}{\pgfqpoint{0.029821in}{0.017319in}}{\pgfqpoint{0.023570in}{0.023570in}}%
\pgfpathcurveto{\pgfqpoint{0.017319in}{0.029821in}}{\pgfqpoint{0.008840in}{0.033333in}}{\pgfqpoint{0.000000in}{0.033333in}}%
\pgfpathcurveto{\pgfqpoint{-0.008840in}{0.033333in}}{\pgfqpoint{-0.017319in}{0.029821in}}{\pgfqpoint{-0.023570in}{0.023570in}}%
\pgfpathcurveto{\pgfqpoint{-0.029821in}{0.017319in}}{\pgfqpoint{-0.033333in}{0.008840in}}{\pgfqpoint{-0.033333in}{0.000000in}}%
\pgfpathcurveto{\pgfqpoint{-0.033333in}{-0.008840in}}{\pgfqpoint{-0.029821in}{-0.017319in}}{\pgfqpoint{-0.023570in}{-0.023570in}}%
\pgfpathcurveto{\pgfqpoint{-0.017319in}{-0.029821in}}{\pgfqpoint{-0.008840in}{-0.033333in}}{\pgfqpoint{0.000000in}{-0.033333in}}%
\pgfpathlineto{\pgfqpoint{0.000000in}{-0.033333in}}%
\pgfpathclose%
\pgfusepath{stroke,fill}%
}%
\begin{pgfscope}%
\pgfsys@transformshift{2.628738in}{3.644561in}%
\pgfsys@useobject{currentmarker}{}%
\end{pgfscope}%
\end{pgfscope}%
\begin{pgfscope}%
\pgfpathrectangle{\pgfqpoint{0.765000in}{0.660000in}}{\pgfqpoint{4.620000in}{4.620000in}}%
\pgfusepath{clip}%
\pgfsetrectcap%
\pgfsetroundjoin%
\pgfsetlinewidth{1.204500pt}%
\definecolor{currentstroke}{rgb}{0.176471,0.192157,0.258824}%
\pgfsetstrokecolor{currentstroke}%
\pgfsetdash{}{0pt}%
\pgfpathmoveto{\pgfqpoint{3.028160in}{3.143114in}}%
\pgfusepath{stroke}%
\end{pgfscope}%
\begin{pgfscope}%
\pgfpathrectangle{\pgfqpoint{0.765000in}{0.660000in}}{\pgfqpoint{4.620000in}{4.620000in}}%
\pgfusepath{clip}%
\pgfsetbuttcap%
\pgfsetmiterjoin%
\definecolor{currentfill}{rgb}{0.176471,0.192157,0.258824}%
\pgfsetfillcolor{currentfill}%
\pgfsetlinewidth{1.003750pt}%
\definecolor{currentstroke}{rgb}{0.176471,0.192157,0.258824}%
\pgfsetstrokecolor{currentstroke}%
\pgfsetdash{}{0pt}%
\pgfsys@defobject{currentmarker}{\pgfqpoint{-0.033333in}{-0.033333in}}{\pgfqpoint{0.033333in}{0.033333in}}{%
\pgfpathmoveto{\pgfqpoint{-0.000000in}{-0.033333in}}%
\pgfpathlineto{\pgfqpoint{0.033333in}{0.033333in}}%
\pgfpathlineto{\pgfqpoint{-0.033333in}{0.033333in}}%
\pgfpathlineto{\pgfqpoint{-0.000000in}{-0.033333in}}%
\pgfpathclose%
\pgfusepath{stroke,fill}%
}%
\begin{pgfscope}%
\pgfsys@transformshift{3.028160in}{3.143114in}%
\pgfsys@useobject{currentmarker}{}%
\end{pgfscope}%
\end{pgfscope}%
\begin{pgfscope}%
\pgfpathrectangle{\pgfqpoint{0.765000in}{0.660000in}}{\pgfqpoint{4.620000in}{4.620000in}}%
\pgfusepath{clip}%
\pgfsetrectcap%
\pgfsetroundjoin%
\pgfsetlinewidth{1.204500pt}%
\definecolor{currentstroke}{rgb}{0.176471,0.192157,0.258824}%
\pgfsetstrokecolor{currentstroke}%
\pgfsetdash{}{0pt}%
\pgfpathmoveto{\pgfqpoint{2.969234in}{2.964526in}}%
\pgfusepath{stroke}%
\end{pgfscope}%
\begin{pgfscope}%
\pgfpathrectangle{\pgfqpoint{0.765000in}{0.660000in}}{\pgfqpoint{4.620000in}{4.620000in}}%
\pgfusepath{clip}%
\pgfsetbuttcap%
\pgfsetmiterjoin%
\definecolor{currentfill}{rgb}{0.176471,0.192157,0.258824}%
\pgfsetfillcolor{currentfill}%
\pgfsetlinewidth{1.003750pt}%
\definecolor{currentstroke}{rgb}{0.176471,0.192157,0.258824}%
\pgfsetstrokecolor{currentstroke}%
\pgfsetdash{}{0pt}%
\pgfsys@defobject{currentmarker}{\pgfqpoint{-0.033333in}{-0.033333in}}{\pgfqpoint{0.033333in}{0.033333in}}{%
\pgfpathmoveto{\pgfqpoint{-0.000000in}{-0.033333in}}%
\pgfpathlineto{\pgfqpoint{0.033333in}{0.033333in}}%
\pgfpathlineto{\pgfqpoint{-0.033333in}{0.033333in}}%
\pgfpathlineto{\pgfqpoint{-0.000000in}{-0.033333in}}%
\pgfpathclose%
\pgfusepath{stroke,fill}%
}%
\begin{pgfscope}%
\pgfsys@transformshift{2.969234in}{2.964526in}%
\pgfsys@useobject{currentmarker}{}%
\end{pgfscope}%
\end{pgfscope}%
\begin{pgfscope}%
\pgfpathrectangle{\pgfqpoint{0.765000in}{0.660000in}}{\pgfqpoint{4.620000in}{4.620000in}}%
\pgfusepath{clip}%
\pgfsetrectcap%
\pgfsetroundjoin%
\pgfsetlinewidth{1.204500pt}%
\definecolor{currentstroke}{rgb}{0.176471,0.192157,0.258824}%
\pgfsetstrokecolor{currentstroke}%
\pgfsetdash{}{0pt}%
\pgfpathmoveto{\pgfqpoint{3.057453in}{3.259269in}}%
\pgfusepath{stroke}%
\end{pgfscope}%
\begin{pgfscope}%
\pgfpathrectangle{\pgfqpoint{0.765000in}{0.660000in}}{\pgfqpoint{4.620000in}{4.620000in}}%
\pgfusepath{clip}%
\pgfsetbuttcap%
\pgfsetmiterjoin%
\definecolor{currentfill}{rgb}{0.176471,0.192157,0.258824}%
\pgfsetfillcolor{currentfill}%
\pgfsetlinewidth{1.003750pt}%
\definecolor{currentstroke}{rgb}{0.176471,0.192157,0.258824}%
\pgfsetstrokecolor{currentstroke}%
\pgfsetdash{}{0pt}%
\pgfsys@defobject{currentmarker}{\pgfqpoint{-0.033333in}{-0.033333in}}{\pgfqpoint{0.033333in}{0.033333in}}{%
\pgfpathmoveto{\pgfqpoint{-0.000000in}{-0.033333in}}%
\pgfpathlineto{\pgfqpoint{0.033333in}{0.033333in}}%
\pgfpathlineto{\pgfqpoint{-0.033333in}{0.033333in}}%
\pgfpathlineto{\pgfqpoint{-0.000000in}{-0.033333in}}%
\pgfpathclose%
\pgfusepath{stroke,fill}%
}%
\begin{pgfscope}%
\pgfsys@transformshift{3.057453in}{3.259269in}%
\pgfsys@useobject{currentmarker}{}%
\end{pgfscope}%
\end{pgfscope}%
\begin{pgfscope}%
\pgfpathrectangle{\pgfqpoint{0.765000in}{0.660000in}}{\pgfqpoint{4.620000in}{4.620000in}}%
\pgfusepath{clip}%
\pgfsetrectcap%
\pgfsetroundjoin%
\pgfsetlinewidth{1.204500pt}%
\definecolor{currentstroke}{rgb}{0.176471,0.192157,0.258824}%
\pgfsetstrokecolor{currentstroke}%
\pgfsetdash{}{0pt}%
\pgfpathmoveto{\pgfqpoint{3.175598in}{2.810411in}}%
\pgfusepath{stroke}%
\end{pgfscope}%
\begin{pgfscope}%
\pgfpathrectangle{\pgfqpoint{0.765000in}{0.660000in}}{\pgfqpoint{4.620000in}{4.620000in}}%
\pgfusepath{clip}%
\pgfsetbuttcap%
\pgfsetmiterjoin%
\definecolor{currentfill}{rgb}{0.176471,0.192157,0.258824}%
\pgfsetfillcolor{currentfill}%
\pgfsetlinewidth{1.003750pt}%
\definecolor{currentstroke}{rgb}{0.176471,0.192157,0.258824}%
\pgfsetstrokecolor{currentstroke}%
\pgfsetdash{}{0pt}%
\pgfsys@defobject{currentmarker}{\pgfqpoint{-0.033333in}{-0.033333in}}{\pgfqpoint{0.033333in}{0.033333in}}{%
\pgfpathmoveto{\pgfqpoint{-0.000000in}{-0.033333in}}%
\pgfpathlineto{\pgfqpoint{0.033333in}{0.033333in}}%
\pgfpathlineto{\pgfqpoint{-0.033333in}{0.033333in}}%
\pgfpathlineto{\pgfqpoint{-0.000000in}{-0.033333in}}%
\pgfpathclose%
\pgfusepath{stroke,fill}%
}%
\begin{pgfscope}%
\pgfsys@transformshift{3.175598in}{2.810411in}%
\pgfsys@useobject{currentmarker}{}%
\end{pgfscope}%
\end{pgfscope}%
\begin{pgfscope}%
\pgfpathrectangle{\pgfqpoint{0.765000in}{0.660000in}}{\pgfqpoint{4.620000in}{4.620000in}}%
\pgfusepath{clip}%
\pgfsetrectcap%
\pgfsetroundjoin%
\pgfsetlinewidth{1.204500pt}%
\definecolor{currentstroke}{rgb}{0.176471,0.192157,0.258824}%
\pgfsetstrokecolor{currentstroke}%
\pgfsetdash{}{0pt}%
\pgfpathmoveto{\pgfqpoint{3.233094in}{2.885854in}}%
\pgfusepath{stroke}%
\end{pgfscope}%
\begin{pgfscope}%
\pgfpathrectangle{\pgfqpoint{0.765000in}{0.660000in}}{\pgfqpoint{4.620000in}{4.620000in}}%
\pgfusepath{clip}%
\pgfsetbuttcap%
\pgfsetmiterjoin%
\definecolor{currentfill}{rgb}{0.176471,0.192157,0.258824}%
\pgfsetfillcolor{currentfill}%
\pgfsetlinewidth{1.003750pt}%
\definecolor{currentstroke}{rgb}{0.176471,0.192157,0.258824}%
\pgfsetstrokecolor{currentstroke}%
\pgfsetdash{}{0pt}%
\pgfsys@defobject{currentmarker}{\pgfqpoint{-0.033333in}{-0.033333in}}{\pgfqpoint{0.033333in}{0.033333in}}{%
\pgfpathmoveto{\pgfqpoint{-0.000000in}{-0.033333in}}%
\pgfpathlineto{\pgfqpoint{0.033333in}{0.033333in}}%
\pgfpathlineto{\pgfqpoint{-0.033333in}{0.033333in}}%
\pgfpathlineto{\pgfqpoint{-0.000000in}{-0.033333in}}%
\pgfpathclose%
\pgfusepath{stroke,fill}%
}%
\begin{pgfscope}%
\pgfsys@transformshift{3.233094in}{2.885854in}%
\pgfsys@useobject{currentmarker}{}%
\end{pgfscope}%
\end{pgfscope}%
\begin{pgfscope}%
\pgfpathrectangle{\pgfqpoint{0.765000in}{0.660000in}}{\pgfqpoint{4.620000in}{4.620000in}}%
\pgfusepath{clip}%
\pgfsetrectcap%
\pgfsetroundjoin%
\pgfsetlinewidth{1.204500pt}%
\definecolor{currentstroke}{rgb}{0.176471,0.192157,0.258824}%
\pgfsetstrokecolor{currentstroke}%
\pgfsetdash{}{0pt}%
\pgfpathmoveto{\pgfqpoint{2.973942in}{3.024328in}}%
\pgfusepath{stroke}%
\end{pgfscope}%
\begin{pgfscope}%
\pgfpathrectangle{\pgfqpoint{0.765000in}{0.660000in}}{\pgfqpoint{4.620000in}{4.620000in}}%
\pgfusepath{clip}%
\pgfsetbuttcap%
\pgfsetmiterjoin%
\definecolor{currentfill}{rgb}{0.176471,0.192157,0.258824}%
\pgfsetfillcolor{currentfill}%
\pgfsetlinewidth{1.003750pt}%
\definecolor{currentstroke}{rgb}{0.176471,0.192157,0.258824}%
\pgfsetstrokecolor{currentstroke}%
\pgfsetdash{}{0pt}%
\pgfsys@defobject{currentmarker}{\pgfqpoint{-0.033333in}{-0.033333in}}{\pgfqpoint{0.033333in}{0.033333in}}{%
\pgfpathmoveto{\pgfqpoint{-0.000000in}{-0.033333in}}%
\pgfpathlineto{\pgfqpoint{0.033333in}{0.033333in}}%
\pgfpathlineto{\pgfqpoint{-0.033333in}{0.033333in}}%
\pgfpathlineto{\pgfqpoint{-0.000000in}{-0.033333in}}%
\pgfpathclose%
\pgfusepath{stroke,fill}%
}%
\begin{pgfscope}%
\pgfsys@transformshift{2.973942in}{3.024328in}%
\pgfsys@useobject{currentmarker}{}%
\end{pgfscope}%
\end{pgfscope}%
\begin{pgfscope}%
\pgfpathrectangle{\pgfqpoint{0.765000in}{0.660000in}}{\pgfqpoint{4.620000in}{4.620000in}}%
\pgfusepath{clip}%
\pgfsetrectcap%
\pgfsetroundjoin%
\pgfsetlinewidth{1.204500pt}%
\definecolor{currentstroke}{rgb}{0.176471,0.192157,0.258824}%
\pgfsetstrokecolor{currentstroke}%
\pgfsetdash{}{0pt}%
\pgfpathmoveto{\pgfqpoint{3.215035in}{3.002857in}}%
\pgfusepath{stroke}%
\end{pgfscope}%
\begin{pgfscope}%
\pgfpathrectangle{\pgfqpoint{0.765000in}{0.660000in}}{\pgfqpoint{4.620000in}{4.620000in}}%
\pgfusepath{clip}%
\pgfsetbuttcap%
\pgfsetmiterjoin%
\definecolor{currentfill}{rgb}{0.176471,0.192157,0.258824}%
\pgfsetfillcolor{currentfill}%
\pgfsetlinewidth{1.003750pt}%
\definecolor{currentstroke}{rgb}{0.176471,0.192157,0.258824}%
\pgfsetstrokecolor{currentstroke}%
\pgfsetdash{}{0pt}%
\pgfsys@defobject{currentmarker}{\pgfqpoint{-0.033333in}{-0.033333in}}{\pgfqpoint{0.033333in}{0.033333in}}{%
\pgfpathmoveto{\pgfqpoint{-0.000000in}{-0.033333in}}%
\pgfpathlineto{\pgfqpoint{0.033333in}{0.033333in}}%
\pgfpathlineto{\pgfqpoint{-0.033333in}{0.033333in}}%
\pgfpathlineto{\pgfqpoint{-0.000000in}{-0.033333in}}%
\pgfpathclose%
\pgfusepath{stroke,fill}%
}%
\begin{pgfscope}%
\pgfsys@transformshift{3.215035in}{3.002857in}%
\pgfsys@useobject{currentmarker}{}%
\end{pgfscope}%
\end{pgfscope}%
\begin{pgfscope}%
\pgfpathrectangle{\pgfqpoint{0.765000in}{0.660000in}}{\pgfqpoint{4.620000in}{4.620000in}}%
\pgfusepath{clip}%
\pgfsetrectcap%
\pgfsetroundjoin%
\pgfsetlinewidth{1.204500pt}%
\definecolor{currentstroke}{rgb}{0.176471,0.192157,0.258824}%
\pgfsetstrokecolor{currentstroke}%
\pgfsetdash{}{0pt}%
\pgfpathmoveto{\pgfqpoint{3.099557in}{2.824254in}}%
\pgfusepath{stroke}%
\end{pgfscope}%
\begin{pgfscope}%
\pgfpathrectangle{\pgfqpoint{0.765000in}{0.660000in}}{\pgfqpoint{4.620000in}{4.620000in}}%
\pgfusepath{clip}%
\pgfsetbuttcap%
\pgfsetmiterjoin%
\definecolor{currentfill}{rgb}{0.176471,0.192157,0.258824}%
\pgfsetfillcolor{currentfill}%
\pgfsetlinewidth{1.003750pt}%
\definecolor{currentstroke}{rgb}{0.176471,0.192157,0.258824}%
\pgfsetstrokecolor{currentstroke}%
\pgfsetdash{}{0pt}%
\pgfsys@defobject{currentmarker}{\pgfqpoint{-0.033333in}{-0.033333in}}{\pgfqpoint{0.033333in}{0.033333in}}{%
\pgfpathmoveto{\pgfqpoint{-0.000000in}{-0.033333in}}%
\pgfpathlineto{\pgfqpoint{0.033333in}{0.033333in}}%
\pgfpathlineto{\pgfqpoint{-0.033333in}{0.033333in}}%
\pgfpathlineto{\pgfqpoint{-0.000000in}{-0.033333in}}%
\pgfpathclose%
\pgfusepath{stroke,fill}%
}%
\begin{pgfscope}%
\pgfsys@transformshift{3.099557in}{2.824254in}%
\pgfsys@useobject{currentmarker}{}%
\end{pgfscope}%
\end{pgfscope}%
\begin{pgfscope}%
\pgfpathrectangle{\pgfqpoint{0.765000in}{0.660000in}}{\pgfqpoint{4.620000in}{4.620000in}}%
\pgfusepath{clip}%
\pgfsetrectcap%
\pgfsetroundjoin%
\pgfsetlinewidth{1.204500pt}%
\definecolor{currentstroke}{rgb}{0.176471,0.192157,0.258824}%
\pgfsetstrokecolor{currentstroke}%
\pgfsetdash{}{0pt}%
\pgfpathmoveto{\pgfqpoint{3.082660in}{3.266765in}}%
\pgfusepath{stroke}%
\end{pgfscope}%
\begin{pgfscope}%
\pgfpathrectangle{\pgfqpoint{0.765000in}{0.660000in}}{\pgfqpoint{4.620000in}{4.620000in}}%
\pgfusepath{clip}%
\pgfsetbuttcap%
\pgfsetmiterjoin%
\definecolor{currentfill}{rgb}{0.176471,0.192157,0.258824}%
\pgfsetfillcolor{currentfill}%
\pgfsetlinewidth{1.003750pt}%
\definecolor{currentstroke}{rgb}{0.176471,0.192157,0.258824}%
\pgfsetstrokecolor{currentstroke}%
\pgfsetdash{}{0pt}%
\pgfsys@defobject{currentmarker}{\pgfqpoint{-0.033333in}{-0.033333in}}{\pgfqpoint{0.033333in}{0.033333in}}{%
\pgfpathmoveto{\pgfqpoint{-0.000000in}{-0.033333in}}%
\pgfpathlineto{\pgfqpoint{0.033333in}{0.033333in}}%
\pgfpathlineto{\pgfqpoint{-0.033333in}{0.033333in}}%
\pgfpathlineto{\pgfqpoint{-0.000000in}{-0.033333in}}%
\pgfpathclose%
\pgfusepath{stroke,fill}%
}%
\begin{pgfscope}%
\pgfsys@transformshift{3.082660in}{3.266765in}%
\pgfsys@useobject{currentmarker}{}%
\end{pgfscope}%
\end{pgfscope}%
\begin{pgfscope}%
\pgfpathrectangle{\pgfqpoint{0.765000in}{0.660000in}}{\pgfqpoint{4.620000in}{4.620000in}}%
\pgfusepath{clip}%
\pgfsetrectcap%
\pgfsetroundjoin%
\pgfsetlinewidth{1.204500pt}%
\definecolor{currentstroke}{rgb}{0.176471,0.192157,0.258824}%
\pgfsetstrokecolor{currentstroke}%
\pgfsetdash{}{0pt}%
\pgfpathmoveto{\pgfqpoint{3.230738in}{3.234090in}}%
\pgfusepath{stroke}%
\end{pgfscope}%
\begin{pgfscope}%
\pgfpathrectangle{\pgfqpoint{0.765000in}{0.660000in}}{\pgfqpoint{4.620000in}{4.620000in}}%
\pgfusepath{clip}%
\pgfsetbuttcap%
\pgfsetmiterjoin%
\definecolor{currentfill}{rgb}{0.176471,0.192157,0.258824}%
\pgfsetfillcolor{currentfill}%
\pgfsetlinewidth{1.003750pt}%
\definecolor{currentstroke}{rgb}{0.176471,0.192157,0.258824}%
\pgfsetstrokecolor{currentstroke}%
\pgfsetdash{}{0pt}%
\pgfsys@defobject{currentmarker}{\pgfqpoint{-0.033333in}{-0.033333in}}{\pgfqpoint{0.033333in}{0.033333in}}{%
\pgfpathmoveto{\pgfqpoint{-0.000000in}{-0.033333in}}%
\pgfpathlineto{\pgfqpoint{0.033333in}{0.033333in}}%
\pgfpathlineto{\pgfqpoint{-0.033333in}{0.033333in}}%
\pgfpathlineto{\pgfqpoint{-0.000000in}{-0.033333in}}%
\pgfpathclose%
\pgfusepath{stroke,fill}%
}%
\begin{pgfscope}%
\pgfsys@transformshift{3.230738in}{3.234090in}%
\pgfsys@useobject{currentmarker}{}%
\end{pgfscope}%
\end{pgfscope}%
\begin{pgfscope}%
\pgfpathrectangle{\pgfqpoint{0.765000in}{0.660000in}}{\pgfqpoint{4.620000in}{4.620000in}}%
\pgfusepath{clip}%
\pgfsetrectcap%
\pgfsetroundjoin%
\pgfsetlinewidth{1.204500pt}%
\definecolor{currentstroke}{rgb}{0.176471,0.192157,0.258824}%
\pgfsetstrokecolor{currentstroke}%
\pgfsetdash{}{0pt}%
\pgfpathmoveto{\pgfqpoint{3.175631in}{2.915262in}}%
\pgfusepath{stroke}%
\end{pgfscope}%
\begin{pgfscope}%
\pgfpathrectangle{\pgfqpoint{0.765000in}{0.660000in}}{\pgfqpoint{4.620000in}{4.620000in}}%
\pgfusepath{clip}%
\pgfsetbuttcap%
\pgfsetmiterjoin%
\definecolor{currentfill}{rgb}{0.176471,0.192157,0.258824}%
\pgfsetfillcolor{currentfill}%
\pgfsetlinewidth{1.003750pt}%
\definecolor{currentstroke}{rgb}{0.176471,0.192157,0.258824}%
\pgfsetstrokecolor{currentstroke}%
\pgfsetdash{}{0pt}%
\pgfsys@defobject{currentmarker}{\pgfqpoint{-0.033333in}{-0.033333in}}{\pgfqpoint{0.033333in}{0.033333in}}{%
\pgfpathmoveto{\pgfqpoint{-0.000000in}{-0.033333in}}%
\pgfpathlineto{\pgfqpoint{0.033333in}{0.033333in}}%
\pgfpathlineto{\pgfqpoint{-0.033333in}{0.033333in}}%
\pgfpathlineto{\pgfqpoint{-0.000000in}{-0.033333in}}%
\pgfpathclose%
\pgfusepath{stroke,fill}%
}%
\begin{pgfscope}%
\pgfsys@transformshift{3.175631in}{2.915262in}%
\pgfsys@useobject{currentmarker}{}%
\end{pgfscope}%
\end{pgfscope}%
\begin{pgfscope}%
\pgfpathrectangle{\pgfqpoint{0.765000in}{0.660000in}}{\pgfqpoint{4.620000in}{4.620000in}}%
\pgfusepath{clip}%
\pgfsetrectcap%
\pgfsetroundjoin%
\pgfsetlinewidth{1.204500pt}%
\definecolor{currentstroke}{rgb}{0.176471,0.192157,0.258824}%
\pgfsetstrokecolor{currentstroke}%
\pgfsetdash{}{0pt}%
\pgfpathmoveto{\pgfqpoint{3.299840in}{3.038101in}}%
\pgfusepath{stroke}%
\end{pgfscope}%
\begin{pgfscope}%
\pgfpathrectangle{\pgfqpoint{0.765000in}{0.660000in}}{\pgfqpoint{4.620000in}{4.620000in}}%
\pgfusepath{clip}%
\pgfsetbuttcap%
\pgfsetmiterjoin%
\definecolor{currentfill}{rgb}{0.176471,0.192157,0.258824}%
\pgfsetfillcolor{currentfill}%
\pgfsetlinewidth{1.003750pt}%
\definecolor{currentstroke}{rgb}{0.176471,0.192157,0.258824}%
\pgfsetstrokecolor{currentstroke}%
\pgfsetdash{}{0pt}%
\pgfsys@defobject{currentmarker}{\pgfqpoint{-0.033333in}{-0.033333in}}{\pgfqpoint{0.033333in}{0.033333in}}{%
\pgfpathmoveto{\pgfqpoint{-0.000000in}{-0.033333in}}%
\pgfpathlineto{\pgfqpoint{0.033333in}{0.033333in}}%
\pgfpathlineto{\pgfqpoint{-0.033333in}{0.033333in}}%
\pgfpathlineto{\pgfqpoint{-0.000000in}{-0.033333in}}%
\pgfpathclose%
\pgfusepath{stroke,fill}%
}%
\begin{pgfscope}%
\pgfsys@transformshift{3.299840in}{3.038101in}%
\pgfsys@useobject{currentmarker}{}%
\end{pgfscope}%
\end{pgfscope}%
\begin{pgfscope}%
\pgfpathrectangle{\pgfqpoint{0.765000in}{0.660000in}}{\pgfqpoint{4.620000in}{4.620000in}}%
\pgfusepath{clip}%
\pgfsetrectcap%
\pgfsetroundjoin%
\pgfsetlinewidth{1.204500pt}%
\definecolor{currentstroke}{rgb}{0.176471,0.192157,0.258824}%
\pgfsetstrokecolor{currentstroke}%
\pgfsetdash{}{0pt}%
\pgfpathmoveto{\pgfqpoint{3.012296in}{2.919010in}}%
\pgfusepath{stroke}%
\end{pgfscope}%
\begin{pgfscope}%
\pgfpathrectangle{\pgfqpoint{0.765000in}{0.660000in}}{\pgfqpoint{4.620000in}{4.620000in}}%
\pgfusepath{clip}%
\pgfsetbuttcap%
\pgfsetmiterjoin%
\definecolor{currentfill}{rgb}{0.176471,0.192157,0.258824}%
\pgfsetfillcolor{currentfill}%
\pgfsetlinewidth{1.003750pt}%
\definecolor{currentstroke}{rgb}{0.176471,0.192157,0.258824}%
\pgfsetstrokecolor{currentstroke}%
\pgfsetdash{}{0pt}%
\pgfsys@defobject{currentmarker}{\pgfqpoint{-0.033333in}{-0.033333in}}{\pgfqpoint{0.033333in}{0.033333in}}{%
\pgfpathmoveto{\pgfqpoint{-0.000000in}{-0.033333in}}%
\pgfpathlineto{\pgfqpoint{0.033333in}{0.033333in}}%
\pgfpathlineto{\pgfqpoint{-0.033333in}{0.033333in}}%
\pgfpathlineto{\pgfqpoint{-0.000000in}{-0.033333in}}%
\pgfpathclose%
\pgfusepath{stroke,fill}%
}%
\begin{pgfscope}%
\pgfsys@transformshift{3.012296in}{2.919010in}%
\pgfsys@useobject{currentmarker}{}%
\end{pgfscope}%
\end{pgfscope}%
\begin{pgfscope}%
\pgfpathrectangle{\pgfqpoint{0.765000in}{0.660000in}}{\pgfqpoint{4.620000in}{4.620000in}}%
\pgfusepath{clip}%
\pgfsetrectcap%
\pgfsetroundjoin%
\pgfsetlinewidth{1.204500pt}%
\definecolor{currentstroke}{rgb}{0.176471,0.192157,0.258824}%
\pgfsetstrokecolor{currentstroke}%
\pgfsetdash{}{0pt}%
\pgfpathmoveto{\pgfqpoint{2.992656in}{3.169763in}}%
\pgfusepath{stroke}%
\end{pgfscope}%
\begin{pgfscope}%
\pgfpathrectangle{\pgfqpoint{0.765000in}{0.660000in}}{\pgfqpoint{4.620000in}{4.620000in}}%
\pgfusepath{clip}%
\pgfsetbuttcap%
\pgfsetmiterjoin%
\definecolor{currentfill}{rgb}{0.176471,0.192157,0.258824}%
\pgfsetfillcolor{currentfill}%
\pgfsetlinewidth{1.003750pt}%
\definecolor{currentstroke}{rgb}{0.176471,0.192157,0.258824}%
\pgfsetstrokecolor{currentstroke}%
\pgfsetdash{}{0pt}%
\pgfsys@defobject{currentmarker}{\pgfqpoint{-0.033333in}{-0.033333in}}{\pgfqpoint{0.033333in}{0.033333in}}{%
\pgfpathmoveto{\pgfqpoint{-0.000000in}{-0.033333in}}%
\pgfpathlineto{\pgfqpoint{0.033333in}{0.033333in}}%
\pgfpathlineto{\pgfqpoint{-0.033333in}{0.033333in}}%
\pgfpathlineto{\pgfqpoint{-0.000000in}{-0.033333in}}%
\pgfpathclose%
\pgfusepath{stroke,fill}%
}%
\begin{pgfscope}%
\pgfsys@transformshift{2.992656in}{3.169763in}%
\pgfsys@useobject{currentmarker}{}%
\end{pgfscope}%
\end{pgfscope}%
\begin{pgfscope}%
\pgfpathrectangle{\pgfqpoint{0.765000in}{0.660000in}}{\pgfqpoint{4.620000in}{4.620000in}}%
\pgfusepath{clip}%
\pgfsetrectcap%
\pgfsetroundjoin%
\pgfsetlinewidth{1.204500pt}%
\definecolor{currentstroke}{rgb}{0.176471,0.192157,0.258824}%
\pgfsetstrokecolor{currentstroke}%
\pgfsetdash{}{0pt}%
\pgfpathmoveto{\pgfqpoint{3.118777in}{2.854793in}}%
\pgfusepath{stroke}%
\end{pgfscope}%
\begin{pgfscope}%
\pgfpathrectangle{\pgfqpoint{0.765000in}{0.660000in}}{\pgfqpoint{4.620000in}{4.620000in}}%
\pgfusepath{clip}%
\pgfsetbuttcap%
\pgfsetmiterjoin%
\definecolor{currentfill}{rgb}{0.176471,0.192157,0.258824}%
\pgfsetfillcolor{currentfill}%
\pgfsetlinewidth{1.003750pt}%
\definecolor{currentstroke}{rgb}{0.176471,0.192157,0.258824}%
\pgfsetstrokecolor{currentstroke}%
\pgfsetdash{}{0pt}%
\pgfsys@defobject{currentmarker}{\pgfqpoint{-0.033333in}{-0.033333in}}{\pgfqpoint{0.033333in}{0.033333in}}{%
\pgfpathmoveto{\pgfqpoint{-0.000000in}{-0.033333in}}%
\pgfpathlineto{\pgfqpoint{0.033333in}{0.033333in}}%
\pgfpathlineto{\pgfqpoint{-0.033333in}{0.033333in}}%
\pgfpathlineto{\pgfqpoint{-0.000000in}{-0.033333in}}%
\pgfpathclose%
\pgfusepath{stroke,fill}%
}%
\begin{pgfscope}%
\pgfsys@transformshift{3.118777in}{2.854793in}%
\pgfsys@useobject{currentmarker}{}%
\end{pgfscope}%
\end{pgfscope}%
\begin{pgfscope}%
\pgfpathrectangle{\pgfqpoint{0.765000in}{0.660000in}}{\pgfqpoint{4.620000in}{4.620000in}}%
\pgfusepath{clip}%
\pgfsetrectcap%
\pgfsetroundjoin%
\pgfsetlinewidth{1.204500pt}%
\definecolor{currentstroke}{rgb}{0.176471,0.192157,0.258824}%
\pgfsetstrokecolor{currentstroke}%
\pgfsetdash{}{0pt}%
\pgfpathmoveto{\pgfqpoint{3.123443in}{3.007043in}}%
\pgfusepath{stroke}%
\end{pgfscope}%
\begin{pgfscope}%
\pgfpathrectangle{\pgfqpoint{0.765000in}{0.660000in}}{\pgfqpoint{4.620000in}{4.620000in}}%
\pgfusepath{clip}%
\pgfsetbuttcap%
\pgfsetmiterjoin%
\definecolor{currentfill}{rgb}{0.176471,0.192157,0.258824}%
\pgfsetfillcolor{currentfill}%
\pgfsetlinewidth{1.003750pt}%
\definecolor{currentstroke}{rgb}{0.176471,0.192157,0.258824}%
\pgfsetstrokecolor{currentstroke}%
\pgfsetdash{}{0pt}%
\pgfsys@defobject{currentmarker}{\pgfqpoint{-0.033333in}{-0.033333in}}{\pgfqpoint{0.033333in}{0.033333in}}{%
\pgfpathmoveto{\pgfqpoint{-0.000000in}{-0.033333in}}%
\pgfpathlineto{\pgfqpoint{0.033333in}{0.033333in}}%
\pgfpathlineto{\pgfqpoint{-0.033333in}{0.033333in}}%
\pgfpathlineto{\pgfqpoint{-0.000000in}{-0.033333in}}%
\pgfpathclose%
\pgfusepath{stroke,fill}%
}%
\begin{pgfscope}%
\pgfsys@transformshift{3.123443in}{3.007043in}%
\pgfsys@useobject{currentmarker}{}%
\end{pgfscope}%
\end{pgfscope}%
\begin{pgfscope}%
\pgfpathrectangle{\pgfqpoint{0.765000in}{0.660000in}}{\pgfqpoint{4.620000in}{4.620000in}}%
\pgfusepath{clip}%
\pgfsetrectcap%
\pgfsetroundjoin%
\pgfsetlinewidth{1.204500pt}%
\definecolor{currentstroke}{rgb}{0.176471,0.192157,0.258824}%
\pgfsetstrokecolor{currentstroke}%
\pgfsetdash{}{0pt}%
\pgfpathmoveto{\pgfqpoint{2.941470in}{3.131352in}}%
\pgfusepath{stroke}%
\end{pgfscope}%
\begin{pgfscope}%
\pgfpathrectangle{\pgfqpoint{0.765000in}{0.660000in}}{\pgfqpoint{4.620000in}{4.620000in}}%
\pgfusepath{clip}%
\pgfsetbuttcap%
\pgfsetmiterjoin%
\definecolor{currentfill}{rgb}{0.176471,0.192157,0.258824}%
\pgfsetfillcolor{currentfill}%
\pgfsetlinewidth{1.003750pt}%
\definecolor{currentstroke}{rgb}{0.176471,0.192157,0.258824}%
\pgfsetstrokecolor{currentstroke}%
\pgfsetdash{}{0pt}%
\pgfsys@defobject{currentmarker}{\pgfqpoint{-0.033333in}{-0.033333in}}{\pgfqpoint{0.033333in}{0.033333in}}{%
\pgfpathmoveto{\pgfqpoint{-0.000000in}{-0.033333in}}%
\pgfpathlineto{\pgfqpoint{0.033333in}{0.033333in}}%
\pgfpathlineto{\pgfqpoint{-0.033333in}{0.033333in}}%
\pgfpathlineto{\pgfqpoint{-0.000000in}{-0.033333in}}%
\pgfpathclose%
\pgfusepath{stroke,fill}%
}%
\begin{pgfscope}%
\pgfsys@transformshift{2.941470in}{3.131352in}%
\pgfsys@useobject{currentmarker}{}%
\end{pgfscope}%
\end{pgfscope}%
\begin{pgfscope}%
\pgfpathrectangle{\pgfqpoint{0.765000in}{0.660000in}}{\pgfqpoint{4.620000in}{4.620000in}}%
\pgfusepath{clip}%
\pgfsetrectcap%
\pgfsetroundjoin%
\pgfsetlinewidth{1.204500pt}%
\definecolor{currentstroke}{rgb}{0.176471,0.192157,0.258824}%
\pgfsetstrokecolor{currentstroke}%
\pgfsetdash{}{0pt}%
\pgfpathmoveto{\pgfqpoint{3.056021in}{3.132614in}}%
\pgfusepath{stroke}%
\end{pgfscope}%
\begin{pgfscope}%
\pgfpathrectangle{\pgfqpoint{0.765000in}{0.660000in}}{\pgfqpoint{4.620000in}{4.620000in}}%
\pgfusepath{clip}%
\pgfsetbuttcap%
\pgfsetmiterjoin%
\definecolor{currentfill}{rgb}{0.176471,0.192157,0.258824}%
\pgfsetfillcolor{currentfill}%
\pgfsetlinewidth{1.003750pt}%
\definecolor{currentstroke}{rgb}{0.176471,0.192157,0.258824}%
\pgfsetstrokecolor{currentstroke}%
\pgfsetdash{}{0pt}%
\pgfsys@defobject{currentmarker}{\pgfqpoint{-0.033333in}{-0.033333in}}{\pgfqpoint{0.033333in}{0.033333in}}{%
\pgfpathmoveto{\pgfqpoint{-0.000000in}{-0.033333in}}%
\pgfpathlineto{\pgfqpoint{0.033333in}{0.033333in}}%
\pgfpathlineto{\pgfqpoint{-0.033333in}{0.033333in}}%
\pgfpathlineto{\pgfqpoint{-0.000000in}{-0.033333in}}%
\pgfpathclose%
\pgfusepath{stroke,fill}%
}%
\begin{pgfscope}%
\pgfsys@transformshift{3.056021in}{3.132614in}%
\pgfsys@useobject{currentmarker}{}%
\end{pgfscope}%
\end{pgfscope}%
\begin{pgfscope}%
\pgfpathrectangle{\pgfqpoint{0.765000in}{0.660000in}}{\pgfqpoint{4.620000in}{4.620000in}}%
\pgfusepath{clip}%
\pgfsetrectcap%
\pgfsetroundjoin%
\pgfsetlinewidth{1.204500pt}%
\definecolor{currentstroke}{rgb}{0.176471,0.192157,0.258824}%
\pgfsetstrokecolor{currentstroke}%
\pgfsetdash{}{0pt}%
\pgfpathmoveto{\pgfqpoint{3.207459in}{3.040819in}}%
\pgfusepath{stroke}%
\end{pgfscope}%
\begin{pgfscope}%
\pgfpathrectangle{\pgfqpoint{0.765000in}{0.660000in}}{\pgfqpoint{4.620000in}{4.620000in}}%
\pgfusepath{clip}%
\pgfsetbuttcap%
\pgfsetmiterjoin%
\definecolor{currentfill}{rgb}{0.176471,0.192157,0.258824}%
\pgfsetfillcolor{currentfill}%
\pgfsetlinewidth{1.003750pt}%
\definecolor{currentstroke}{rgb}{0.176471,0.192157,0.258824}%
\pgfsetstrokecolor{currentstroke}%
\pgfsetdash{}{0pt}%
\pgfsys@defobject{currentmarker}{\pgfqpoint{-0.033333in}{-0.033333in}}{\pgfqpoint{0.033333in}{0.033333in}}{%
\pgfpathmoveto{\pgfqpoint{-0.000000in}{-0.033333in}}%
\pgfpathlineto{\pgfqpoint{0.033333in}{0.033333in}}%
\pgfpathlineto{\pgfqpoint{-0.033333in}{0.033333in}}%
\pgfpathlineto{\pgfqpoint{-0.000000in}{-0.033333in}}%
\pgfpathclose%
\pgfusepath{stroke,fill}%
}%
\begin{pgfscope}%
\pgfsys@transformshift{3.207459in}{3.040819in}%
\pgfsys@useobject{currentmarker}{}%
\end{pgfscope}%
\end{pgfscope}%
\begin{pgfscope}%
\pgfpathrectangle{\pgfqpoint{0.765000in}{0.660000in}}{\pgfqpoint{4.620000in}{4.620000in}}%
\pgfusepath{clip}%
\pgfsetrectcap%
\pgfsetroundjoin%
\pgfsetlinewidth{1.204500pt}%
\definecolor{currentstroke}{rgb}{0.176471,0.192157,0.258824}%
\pgfsetstrokecolor{currentstroke}%
\pgfsetdash{}{0pt}%
\pgfpathmoveto{\pgfqpoint{3.196394in}{3.218235in}}%
\pgfusepath{stroke}%
\end{pgfscope}%
\begin{pgfscope}%
\pgfpathrectangle{\pgfqpoint{0.765000in}{0.660000in}}{\pgfqpoint{4.620000in}{4.620000in}}%
\pgfusepath{clip}%
\pgfsetbuttcap%
\pgfsetmiterjoin%
\definecolor{currentfill}{rgb}{0.176471,0.192157,0.258824}%
\pgfsetfillcolor{currentfill}%
\pgfsetlinewidth{1.003750pt}%
\definecolor{currentstroke}{rgb}{0.176471,0.192157,0.258824}%
\pgfsetstrokecolor{currentstroke}%
\pgfsetdash{}{0pt}%
\pgfsys@defobject{currentmarker}{\pgfqpoint{-0.033333in}{-0.033333in}}{\pgfqpoint{0.033333in}{0.033333in}}{%
\pgfpathmoveto{\pgfqpoint{-0.000000in}{-0.033333in}}%
\pgfpathlineto{\pgfqpoint{0.033333in}{0.033333in}}%
\pgfpathlineto{\pgfqpoint{-0.033333in}{0.033333in}}%
\pgfpathlineto{\pgfqpoint{-0.000000in}{-0.033333in}}%
\pgfpathclose%
\pgfusepath{stroke,fill}%
}%
\begin{pgfscope}%
\pgfsys@transformshift{3.196394in}{3.218235in}%
\pgfsys@useobject{currentmarker}{}%
\end{pgfscope}%
\end{pgfscope}%
\begin{pgfscope}%
\pgfpathrectangle{\pgfqpoint{0.765000in}{0.660000in}}{\pgfqpoint{4.620000in}{4.620000in}}%
\pgfusepath{clip}%
\pgfsetrectcap%
\pgfsetroundjoin%
\pgfsetlinewidth{1.204500pt}%
\definecolor{currentstroke}{rgb}{0.176471,0.192157,0.258824}%
\pgfsetstrokecolor{currentstroke}%
\pgfsetdash{}{0pt}%
\pgfpathmoveto{\pgfqpoint{3.144253in}{2.877742in}}%
\pgfusepath{stroke}%
\end{pgfscope}%
\begin{pgfscope}%
\pgfpathrectangle{\pgfqpoint{0.765000in}{0.660000in}}{\pgfqpoint{4.620000in}{4.620000in}}%
\pgfusepath{clip}%
\pgfsetbuttcap%
\pgfsetmiterjoin%
\definecolor{currentfill}{rgb}{0.176471,0.192157,0.258824}%
\pgfsetfillcolor{currentfill}%
\pgfsetlinewidth{1.003750pt}%
\definecolor{currentstroke}{rgb}{0.176471,0.192157,0.258824}%
\pgfsetstrokecolor{currentstroke}%
\pgfsetdash{}{0pt}%
\pgfsys@defobject{currentmarker}{\pgfqpoint{-0.033333in}{-0.033333in}}{\pgfqpoint{0.033333in}{0.033333in}}{%
\pgfpathmoveto{\pgfqpoint{-0.000000in}{-0.033333in}}%
\pgfpathlineto{\pgfqpoint{0.033333in}{0.033333in}}%
\pgfpathlineto{\pgfqpoint{-0.033333in}{0.033333in}}%
\pgfpathlineto{\pgfqpoint{-0.000000in}{-0.033333in}}%
\pgfpathclose%
\pgfusepath{stroke,fill}%
}%
\begin{pgfscope}%
\pgfsys@transformshift{3.144253in}{2.877742in}%
\pgfsys@useobject{currentmarker}{}%
\end{pgfscope}%
\end{pgfscope}%
\begin{pgfscope}%
\pgfpathrectangle{\pgfqpoint{0.765000in}{0.660000in}}{\pgfqpoint{4.620000in}{4.620000in}}%
\pgfusepath{clip}%
\pgfsetrectcap%
\pgfsetroundjoin%
\pgfsetlinewidth{1.204500pt}%
\definecolor{currentstroke}{rgb}{0.176471,0.192157,0.258824}%
\pgfsetstrokecolor{currentstroke}%
\pgfsetdash{}{0pt}%
\pgfpathmoveto{\pgfqpoint{3.138672in}{2.878014in}}%
\pgfusepath{stroke}%
\end{pgfscope}%
\begin{pgfscope}%
\pgfpathrectangle{\pgfqpoint{0.765000in}{0.660000in}}{\pgfqpoint{4.620000in}{4.620000in}}%
\pgfusepath{clip}%
\pgfsetbuttcap%
\pgfsetmiterjoin%
\definecolor{currentfill}{rgb}{0.176471,0.192157,0.258824}%
\pgfsetfillcolor{currentfill}%
\pgfsetlinewidth{1.003750pt}%
\definecolor{currentstroke}{rgb}{0.176471,0.192157,0.258824}%
\pgfsetstrokecolor{currentstroke}%
\pgfsetdash{}{0pt}%
\pgfsys@defobject{currentmarker}{\pgfqpoint{-0.033333in}{-0.033333in}}{\pgfqpoint{0.033333in}{0.033333in}}{%
\pgfpathmoveto{\pgfqpoint{-0.000000in}{-0.033333in}}%
\pgfpathlineto{\pgfqpoint{0.033333in}{0.033333in}}%
\pgfpathlineto{\pgfqpoint{-0.033333in}{0.033333in}}%
\pgfpathlineto{\pgfqpoint{-0.000000in}{-0.033333in}}%
\pgfpathclose%
\pgfusepath{stroke,fill}%
}%
\begin{pgfscope}%
\pgfsys@transformshift{3.138672in}{2.878014in}%
\pgfsys@useobject{currentmarker}{}%
\end{pgfscope}%
\end{pgfscope}%
\begin{pgfscope}%
\pgfpathrectangle{\pgfqpoint{0.765000in}{0.660000in}}{\pgfqpoint{4.620000in}{4.620000in}}%
\pgfusepath{clip}%
\pgfsetrectcap%
\pgfsetroundjoin%
\pgfsetlinewidth{1.204500pt}%
\definecolor{currentstroke}{rgb}{0.176471,0.192157,0.258824}%
\pgfsetstrokecolor{currentstroke}%
\pgfsetdash{}{0pt}%
\pgfpathmoveto{\pgfqpoint{3.194018in}{3.291834in}}%
\pgfusepath{stroke}%
\end{pgfscope}%
\begin{pgfscope}%
\pgfpathrectangle{\pgfqpoint{0.765000in}{0.660000in}}{\pgfqpoint{4.620000in}{4.620000in}}%
\pgfusepath{clip}%
\pgfsetbuttcap%
\pgfsetmiterjoin%
\definecolor{currentfill}{rgb}{0.176471,0.192157,0.258824}%
\pgfsetfillcolor{currentfill}%
\pgfsetlinewidth{1.003750pt}%
\definecolor{currentstroke}{rgb}{0.176471,0.192157,0.258824}%
\pgfsetstrokecolor{currentstroke}%
\pgfsetdash{}{0pt}%
\pgfsys@defobject{currentmarker}{\pgfqpoint{-0.033333in}{-0.033333in}}{\pgfqpoint{0.033333in}{0.033333in}}{%
\pgfpathmoveto{\pgfqpoint{-0.000000in}{-0.033333in}}%
\pgfpathlineto{\pgfqpoint{0.033333in}{0.033333in}}%
\pgfpathlineto{\pgfqpoint{-0.033333in}{0.033333in}}%
\pgfpathlineto{\pgfqpoint{-0.000000in}{-0.033333in}}%
\pgfpathclose%
\pgfusepath{stroke,fill}%
}%
\begin{pgfscope}%
\pgfsys@transformshift{3.194018in}{3.291834in}%
\pgfsys@useobject{currentmarker}{}%
\end{pgfscope}%
\end{pgfscope}%
\begin{pgfscope}%
\pgfpathrectangle{\pgfqpoint{0.765000in}{0.660000in}}{\pgfqpoint{4.620000in}{4.620000in}}%
\pgfusepath{clip}%
\pgfsetrectcap%
\pgfsetroundjoin%
\pgfsetlinewidth{1.204500pt}%
\definecolor{currentstroke}{rgb}{0.176471,0.192157,0.258824}%
\pgfsetstrokecolor{currentstroke}%
\pgfsetdash{}{0pt}%
\pgfpathmoveto{\pgfqpoint{3.048272in}{2.985681in}}%
\pgfusepath{stroke}%
\end{pgfscope}%
\begin{pgfscope}%
\pgfpathrectangle{\pgfqpoint{0.765000in}{0.660000in}}{\pgfqpoint{4.620000in}{4.620000in}}%
\pgfusepath{clip}%
\pgfsetbuttcap%
\pgfsetmiterjoin%
\definecolor{currentfill}{rgb}{0.176471,0.192157,0.258824}%
\pgfsetfillcolor{currentfill}%
\pgfsetlinewidth{1.003750pt}%
\definecolor{currentstroke}{rgb}{0.176471,0.192157,0.258824}%
\pgfsetstrokecolor{currentstroke}%
\pgfsetdash{}{0pt}%
\pgfsys@defobject{currentmarker}{\pgfqpoint{-0.033333in}{-0.033333in}}{\pgfqpoint{0.033333in}{0.033333in}}{%
\pgfpathmoveto{\pgfqpoint{-0.000000in}{-0.033333in}}%
\pgfpathlineto{\pgfqpoint{0.033333in}{0.033333in}}%
\pgfpathlineto{\pgfqpoint{-0.033333in}{0.033333in}}%
\pgfpathlineto{\pgfqpoint{-0.000000in}{-0.033333in}}%
\pgfpathclose%
\pgfusepath{stroke,fill}%
}%
\begin{pgfscope}%
\pgfsys@transformshift{3.048272in}{2.985681in}%
\pgfsys@useobject{currentmarker}{}%
\end{pgfscope}%
\end{pgfscope}%
\begin{pgfscope}%
\pgfpathrectangle{\pgfqpoint{0.765000in}{0.660000in}}{\pgfqpoint{4.620000in}{4.620000in}}%
\pgfusepath{clip}%
\pgfsetrectcap%
\pgfsetroundjoin%
\pgfsetlinewidth{1.204500pt}%
\definecolor{currentstroke}{rgb}{0.176471,0.192157,0.258824}%
\pgfsetstrokecolor{currentstroke}%
\pgfsetdash{}{0pt}%
\pgfpathmoveto{\pgfqpoint{3.246341in}{3.201645in}}%
\pgfusepath{stroke}%
\end{pgfscope}%
\begin{pgfscope}%
\pgfpathrectangle{\pgfqpoint{0.765000in}{0.660000in}}{\pgfqpoint{4.620000in}{4.620000in}}%
\pgfusepath{clip}%
\pgfsetbuttcap%
\pgfsetmiterjoin%
\definecolor{currentfill}{rgb}{0.176471,0.192157,0.258824}%
\pgfsetfillcolor{currentfill}%
\pgfsetlinewidth{1.003750pt}%
\definecolor{currentstroke}{rgb}{0.176471,0.192157,0.258824}%
\pgfsetstrokecolor{currentstroke}%
\pgfsetdash{}{0pt}%
\pgfsys@defobject{currentmarker}{\pgfqpoint{-0.033333in}{-0.033333in}}{\pgfqpoint{0.033333in}{0.033333in}}{%
\pgfpathmoveto{\pgfqpoint{-0.000000in}{-0.033333in}}%
\pgfpathlineto{\pgfqpoint{0.033333in}{0.033333in}}%
\pgfpathlineto{\pgfqpoint{-0.033333in}{0.033333in}}%
\pgfpathlineto{\pgfqpoint{-0.000000in}{-0.033333in}}%
\pgfpathclose%
\pgfusepath{stroke,fill}%
}%
\begin{pgfscope}%
\pgfsys@transformshift{3.246341in}{3.201645in}%
\pgfsys@useobject{currentmarker}{}%
\end{pgfscope}%
\end{pgfscope}%
\begin{pgfscope}%
\pgfpathrectangle{\pgfqpoint{0.765000in}{0.660000in}}{\pgfqpoint{4.620000in}{4.620000in}}%
\pgfusepath{clip}%
\pgfsetrectcap%
\pgfsetroundjoin%
\pgfsetlinewidth{1.204500pt}%
\definecolor{currentstroke}{rgb}{0.176471,0.192157,0.258824}%
\pgfsetstrokecolor{currentstroke}%
\pgfsetdash{}{0pt}%
\pgfpathmoveto{\pgfqpoint{3.216446in}{2.958357in}}%
\pgfusepath{stroke}%
\end{pgfscope}%
\begin{pgfscope}%
\pgfpathrectangle{\pgfqpoint{0.765000in}{0.660000in}}{\pgfqpoint{4.620000in}{4.620000in}}%
\pgfusepath{clip}%
\pgfsetbuttcap%
\pgfsetmiterjoin%
\definecolor{currentfill}{rgb}{0.176471,0.192157,0.258824}%
\pgfsetfillcolor{currentfill}%
\pgfsetlinewidth{1.003750pt}%
\definecolor{currentstroke}{rgb}{0.176471,0.192157,0.258824}%
\pgfsetstrokecolor{currentstroke}%
\pgfsetdash{}{0pt}%
\pgfsys@defobject{currentmarker}{\pgfqpoint{-0.033333in}{-0.033333in}}{\pgfqpoint{0.033333in}{0.033333in}}{%
\pgfpathmoveto{\pgfqpoint{-0.000000in}{-0.033333in}}%
\pgfpathlineto{\pgfqpoint{0.033333in}{0.033333in}}%
\pgfpathlineto{\pgfqpoint{-0.033333in}{0.033333in}}%
\pgfpathlineto{\pgfqpoint{-0.000000in}{-0.033333in}}%
\pgfpathclose%
\pgfusepath{stroke,fill}%
}%
\begin{pgfscope}%
\pgfsys@transformshift{3.216446in}{2.958357in}%
\pgfsys@useobject{currentmarker}{}%
\end{pgfscope}%
\end{pgfscope}%
\begin{pgfscope}%
\pgfpathrectangle{\pgfqpoint{0.765000in}{0.660000in}}{\pgfqpoint{4.620000in}{4.620000in}}%
\pgfusepath{clip}%
\pgfsetrectcap%
\pgfsetroundjoin%
\pgfsetlinewidth{1.204500pt}%
\definecolor{currentstroke}{rgb}{0.176471,0.192157,0.258824}%
\pgfsetstrokecolor{currentstroke}%
\pgfsetdash{}{0pt}%
\pgfpathmoveto{\pgfqpoint{3.016890in}{3.164394in}}%
\pgfusepath{stroke}%
\end{pgfscope}%
\begin{pgfscope}%
\pgfpathrectangle{\pgfqpoint{0.765000in}{0.660000in}}{\pgfqpoint{4.620000in}{4.620000in}}%
\pgfusepath{clip}%
\pgfsetbuttcap%
\pgfsetmiterjoin%
\definecolor{currentfill}{rgb}{0.176471,0.192157,0.258824}%
\pgfsetfillcolor{currentfill}%
\pgfsetlinewidth{1.003750pt}%
\definecolor{currentstroke}{rgb}{0.176471,0.192157,0.258824}%
\pgfsetstrokecolor{currentstroke}%
\pgfsetdash{}{0pt}%
\pgfsys@defobject{currentmarker}{\pgfqpoint{-0.033333in}{-0.033333in}}{\pgfqpoint{0.033333in}{0.033333in}}{%
\pgfpathmoveto{\pgfqpoint{-0.000000in}{-0.033333in}}%
\pgfpathlineto{\pgfqpoint{0.033333in}{0.033333in}}%
\pgfpathlineto{\pgfqpoint{-0.033333in}{0.033333in}}%
\pgfpathlineto{\pgfqpoint{-0.000000in}{-0.033333in}}%
\pgfpathclose%
\pgfusepath{stroke,fill}%
}%
\begin{pgfscope}%
\pgfsys@transformshift{3.016890in}{3.164394in}%
\pgfsys@useobject{currentmarker}{}%
\end{pgfscope}%
\end{pgfscope}%
\begin{pgfscope}%
\pgfpathrectangle{\pgfqpoint{0.765000in}{0.660000in}}{\pgfqpoint{4.620000in}{4.620000in}}%
\pgfusepath{clip}%
\pgfsetrectcap%
\pgfsetroundjoin%
\pgfsetlinewidth{1.204500pt}%
\definecolor{currentstroke}{rgb}{0.176471,0.192157,0.258824}%
\pgfsetstrokecolor{currentstroke}%
\pgfsetdash{}{0pt}%
\pgfpathmoveto{\pgfqpoint{3.042281in}{2.865179in}}%
\pgfusepath{stroke}%
\end{pgfscope}%
\begin{pgfscope}%
\pgfpathrectangle{\pgfqpoint{0.765000in}{0.660000in}}{\pgfqpoint{4.620000in}{4.620000in}}%
\pgfusepath{clip}%
\pgfsetbuttcap%
\pgfsetmiterjoin%
\definecolor{currentfill}{rgb}{0.176471,0.192157,0.258824}%
\pgfsetfillcolor{currentfill}%
\pgfsetlinewidth{1.003750pt}%
\definecolor{currentstroke}{rgb}{0.176471,0.192157,0.258824}%
\pgfsetstrokecolor{currentstroke}%
\pgfsetdash{}{0pt}%
\pgfsys@defobject{currentmarker}{\pgfqpoint{-0.033333in}{-0.033333in}}{\pgfqpoint{0.033333in}{0.033333in}}{%
\pgfpathmoveto{\pgfqpoint{-0.000000in}{-0.033333in}}%
\pgfpathlineto{\pgfqpoint{0.033333in}{0.033333in}}%
\pgfpathlineto{\pgfqpoint{-0.033333in}{0.033333in}}%
\pgfpathlineto{\pgfqpoint{-0.000000in}{-0.033333in}}%
\pgfpathclose%
\pgfusepath{stroke,fill}%
}%
\begin{pgfscope}%
\pgfsys@transformshift{3.042281in}{2.865179in}%
\pgfsys@useobject{currentmarker}{}%
\end{pgfscope}%
\end{pgfscope}%
\begin{pgfscope}%
\pgfpathrectangle{\pgfqpoint{0.765000in}{0.660000in}}{\pgfqpoint{4.620000in}{4.620000in}}%
\pgfusepath{clip}%
\pgfsetrectcap%
\pgfsetroundjoin%
\pgfsetlinewidth{1.204500pt}%
\definecolor{currentstroke}{rgb}{0.176471,0.192157,0.258824}%
\pgfsetstrokecolor{currentstroke}%
\pgfsetdash{}{0pt}%
\pgfpathmoveto{\pgfqpoint{2.897149in}{3.183561in}}%
\pgfusepath{stroke}%
\end{pgfscope}%
\begin{pgfscope}%
\pgfpathrectangle{\pgfqpoint{0.765000in}{0.660000in}}{\pgfqpoint{4.620000in}{4.620000in}}%
\pgfusepath{clip}%
\pgfsetbuttcap%
\pgfsetmiterjoin%
\definecolor{currentfill}{rgb}{0.176471,0.192157,0.258824}%
\pgfsetfillcolor{currentfill}%
\pgfsetlinewidth{1.003750pt}%
\definecolor{currentstroke}{rgb}{0.176471,0.192157,0.258824}%
\pgfsetstrokecolor{currentstroke}%
\pgfsetdash{}{0pt}%
\pgfsys@defobject{currentmarker}{\pgfqpoint{-0.033333in}{-0.033333in}}{\pgfqpoint{0.033333in}{0.033333in}}{%
\pgfpathmoveto{\pgfqpoint{-0.000000in}{-0.033333in}}%
\pgfpathlineto{\pgfqpoint{0.033333in}{0.033333in}}%
\pgfpathlineto{\pgfqpoint{-0.033333in}{0.033333in}}%
\pgfpathlineto{\pgfqpoint{-0.000000in}{-0.033333in}}%
\pgfpathclose%
\pgfusepath{stroke,fill}%
}%
\begin{pgfscope}%
\pgfsys@transformshift{2.897149in}{3.183561in}%
\pgfsys@useobject{currentmarker}{}%
\end{pgfscope}%
\end{pgfscope}%
\begin{pgfscope}%
\pgfpathrectangle{\pgfqpoint{0.765000in}{0.660000in}}{\pgfqpoint{4.620000in}{4.620000in}}%
\pgfusepath{clip}%
\pgfsetrectcap%
\pgfsetroundjoin%
\pgfsetlinewidth{1.204500pt}%
\definecolor{currentstroke}{rgb}{0.176471,0.192157,0.258824}%
\pgfsetstrokecolor{currentstroke}%
\pgfsetdash{}{0pt}%
\pgfpathmoveto{\pgfqpoint{2.927431in}{3.131107in}}%
\pgfusepath{stroke}%
\end{pgfscope}%
\begin{pgfscope}%
\pgfpathrectangle{\pgfqpoint{0.765000in}{0.660000in}}{\pgfqpoint{4.620000in}{4.620000in}}%
\pgfusepath{clip}%
\pgfsetbuttcap%
\pgfsetmiterjoin%
\definecolor{currentfill}{rgb}{0.176471,0.192157,0.258824}%
\pgfsetfillcolor{currentfill}%
\pgfsetlinewidth{1.003750pt}%
\definecolor{currentstroke}{rgb}{0.176471,0.192157,0.258824}%
\pgfsetstrokecolor{currentstroke}%
\pgfsetdash{}{0pt}%
\pgfsys@defobject{currentmarker}{\pgfqpoint{-0.033333in}{-0.033333in}}{\pgfqpoint{0.033333in}{0.033333in}}{%
\pgfpathmoveto{\pgfqpoint{-0.000000in}{-0.033333in}}%
\pgfpathlineto{\pgfqpoint{0.033333in}{0.033333in}}%
\pgfpathlineto{\pgfqpoint{-0.033333in}{0.033333in}}%
\pgfpathlineto{\pgfqpoint{-0.000000in}{-0.033333in}}%
\pgfpathclose%
\pgfusepath{stroke,fill}%
}%
\begin{pgfscope}%
\pgfsys@transformshift{2.927431in}{3.131107in}%
\pgfsys@useobject{currentmarker}{}%
\end{pgfscope}%
\end{pgfscope}%
\begin{pgfscope}%
\pgfpathrectangle{\pgfqpoint{0.765000in}{0.660000in}}{\pgfqpoint{4.620000in}{4.620000in}}%
\pgfusepath{clip}%
\pgfsetrectcap%
\pgfsetroundjoin%
\pgfsetlinewidth{1.204500pt}%
\definecolor{currentstroke}{rgb}{0.176471,0.192157,0.258824}%
\pgfsetstrokecolor{currentstroke}%
\pgfsetdash{}{0pt}%
\pgfpathmoveto{\pgfqpoint{3.211963in}{3.099725in}}%
\pgfusepath{stroke}%
\end{pgfscope}%
\begin{pgfscope}%
\pgfpathrectangle{\pgfqpoint{0.765000in}{0.660000in}}{\pgfqpoint{4.620000in}{4.620000in}}%
\pgfusepath{clip}%
\pgfsetbuttcap%
\pgfsetmiterjoin%
\definecolor{currentfill}{rgb}{0.176471,0.192157,0.258824}%
\pgfsetfillcolor{currentfill}%
\pgfsetlinewidth{1.003750pt}%
\definecolor{currentstroke}{rgb}{0.176471,0.192157,0.258824}%
\pgfsetstrokecolor{currentstroke}%
\pgfsetdash{}{0pt}%
\pgfsys@defobject{currentmarker}{\pgfqpoint{-0.033333in}{-0.033333in}}{\pgfqpoint{0.033333in}{0.033333in}}{%
\pgfpathmoveto{\pgfqpoint{-0.000000in}{-0.033333in}}%
\pgfpathlineto{\pgfqpoint{0.033333in}{0.033333in}}%
\pgfpathlineto{\pgfqpoint{-0.033333in}{0.033333in}}%
\pgfpathlineto{\pgfqpoint{-0.000000in}{-0.033333in}}%
\pgfpathclose%
\pgfusepath{stroke,fill}%
}%
\begin{pgfscope}%
\pgfsys@transformshift{3.211963in}{3.099725in}%
\pgfsys@useobject{currentmarker}{}%
\end{pgfscope}%
\end{pgfscope}%
\begin{pgfscope}%
\pgfpathrectangle{\pgfqpoint{0.765000in}{0.660000in}}{\pgfqpoint{4.620000in}{4.620000in}}%
\pgfusepath{clip}%
\pgfsetrectcap%
\pgfsetroundjoin%
\pgfsetlinewidth{1.204500pt}%
\definecolor{currentstroke}{rgb}{0.176471,0.192157,0.258824}%
\pgfsetstrokecolor{currentstroke}%
\pgfsetdash{}{0pt}%
\pgfpathmoveto{\pgfqpoint{3.218699in}{3.075680in}}%
\pgfusepath{stroke}%
\end{pgfscope}%
\begin{pgfscope}%
\pgfpathrectangle{\pgfqpoint{0.765000in}{0.660000in}}{\pgfqpoint{4.620000in}{4.620000in}}%
\pgfusepath{clip}%
\pgfsetbuttcap%
\pgfsetmiterjoin%
\definecolor{currentfill}{rgb}{0.176471,0.192157,0.258824}%
\pgfsetfillcolor{currentfill}%
\pgfsetlinewidth{1.003750pt}%
\definecolor{currentstroke}{rgb}{0.176471,0.192157,0.258824}%
\pgfsetstrokecolor{currentstroke}%
\pgfsetdash{}{0pt}%
\pgfsys@defobject{currentmarker}{\pgfqpoint{-0.033333in}{-0.033333in}}{\pgfqpoint{0.033333in}{0.033333in}}{%
\pgfpathmoveto{\pgfqpoint{-0.000000in}{-0.033333in}}%
\pgfpathlineto{\pgfqpoint{0.033333in}{0.033333in}}%
\pgfpathlineto{\pgfqpoint{-0.033333in}{0.033333in}}%
\pgfpathlineto{\pgfqpoint{-0.000000in}{-0.033333in}}%
\pgfpathclose%
\pgfusepath{stroke,fill}%
}%
\begin{pgfscope}%
\pgfsys@transformshift{3.218699in}{3.075680in}%
\pgfsys@useobject{currentmarker}{}%
\end{pgfscope}%
\end{pgfscope}%
\begin{pgfscope}%
\pgfpathrectangle{\pgfqpoint{0.765000in}{0.660000in}}{\pgfqpoint{4.620000in}{4.620000in}}%
\pgfusepath{clip}%
\pgfsetrectcap%
\pgfsetroundjoin%
\pgfsetlinewidth{1.204500pt}%
\definecolor{currentstroke}{rgb}{0.176471,0.192157,0.258824}%
\pgfsetstrokecolor{currentstroke}%
\pgfsetdash{}{0pt}%
\pgfpathmoveto{\pgfqpoint{2.975479in}{3.165570in}}%
\pgfusepath{stroke}%
\end{pgfscope}%
\begin{pgfscope}%
\pgfpathrectangle{\pgfqpoint{0.765000in}{0.660000in}}{\pgfqpoint{4.620000in}{4.620000in}}%
\pgfusepath{clip}%
\pgfsetbuttcap%
\pgfsetmiterjoin%
\definecolor{currentfill}{rgb}{0.176471,0.192157,0.258824}%
\pgfsetfillcolor{currentfill}%
\pgfsetlinewidth{1.003750pt}%
\definecolor{currentstroke}{rgb}{0.176471,0.192157,0.258824}%
\pgfsetstrokecolor{currentstroke}%
\pgfsetdash{}{0pt}%
\pgfsys@defobject{currentmarker}{\pgfqpoint{-0.033333in}{-0.033333in}}{\pgfqpoint{0.033333in}{0.033333in}}{%
\pgfpathmoveto{\pgfqpoint{-0.000000in}{-0.033333in}}%
\pgfpathlineto{\pgfqpoint{0.033333in}{0.033333in}}%
\pgfpathlineto{\pgfqpoint{-0.033333in}{0.033333in}}%
\pgfpathlineto{\pgfqpoint{-0.000000in}{-0.033333in}}%
\pgfpathclose%
\pgfusepath{stroke,fill}%
}%
\begin{pgfscope}%
\pgfsys@transformshift{2.975479in}{3.165570in}%
\pgfsys@useobject{currentmarker}{}%
\end{pgfscope}%
\end{pgfscope}%
\begin{pgfscope}%
\pgfpathrectangle{\pgfqpoint{0.765000in}{0.660000in}}{\pgfqpoint{4.620000in}{4.620000in}}%
\pgfusepath{clip}%
\pgfsetrectcap%
\pgfsetroundjoin%
\pgfsetlinewidth{1.204500pt}%
\definecolor{currentstroke}{rgb}{0.176471,0.192157,0.258824}%
\pgfsetstrokecolor{currentstroke}%
\pgfsetdash{}{0pt}%
\pgfpathmoveto{\pgfqpoint{3.214146in}{3.060766in}}%
\pgfusepath{stroke}%
\end{pgfscope}%
\begin{pgfscope}%
\pgfpathrectangle{\pgfqpoint{0.765000in}{0.660000in}}{\pgfqpoint{4.620000in}{4.620000in}}%
\pgfusepath{clip}%
\pgfsetbuttcap%
\pgfsetmiterjoin%
\definecolor{currentfill}{rgb}{0.176471,0.192157,0.258824}%
\pgfsetfillcolor{currentfill}%
\pgfsetlinewidth{1.003750pt}%
\definecolor{currentstroke}{rgb}{0.176471,0.192157,0.258824}%
\pgfsetstrokecolor{currentstroke}%
\pgfsetdash{}{0pt}%
\pgfsys@defobject{currentmarker}{\pgfqpoint{-0.033333in}{-0.033333in}}{\pgfqpoint{0.033333in}{0.033333in}}{%
\pgfpathmoveto{\pgfqpoint{-0.000000in}{-0.033333in}}%
\pgfpathlineto{\pgfqpoint{0.033333in}{0.033333in}}%
\pgfpathlineto{\pgfqpoint{-0.033333in}{0.033333in}}%
\pgfpathlineto{\pgfqpoint{-0.000000in}{-0.033333in}}%
\pgfpathclose%
\pgfusepath{stroke,fill}%
}%
\begin{pgfscope}%
\pgfsys@transformshift{3.214146in}{3.060766in}%
\pgfsys@useobject{currentmarker}{}%
\end{pgfscope}%
\end{pgfscope}%
\begin{pgfscope}%
\pgfpathrectangle{\pgfqpoint{0.765000in}{0.660000in}}{\pgfqpoint{4.620000in}{4.620000in}}%
\pgfusepath{clip}%
\pgfsetrectcap%
\pgfsetroundjoin%
\pgfsetlinewidth{1.204500pt}%
\definecolor{currentstroke}{rgb}{0.176471,0.192157,0.258824}%
\pgfsetstrokecolor{currentstroke}%
\pgfsetdash{}{0pt}%
\pgfpathmoveto{\pgfqpoint{3.350563in}{2.926523in}}%
\pgfusepath{stroke}%
\end{pgfscope}%
\begin{pgfscope}%
\pgfpathrectangle{\pgfqpoint{0.765000in}{0.660000in}}{\pgfqpoint{4.620000in}{4.620000in}}%
\pgfusepath{clip}%
\pgfsetbuttcap%
\pgfsetmiterjoin%
\definecolor{currentfill}{rgb}{0.176471,0.192157,0.258824}%
\pgfsetfillcolor{currentfill}%
\pgfsetlinewidth{1.003750pt}%
\definecolor{currentstroke}{rgb}{0.176471,0.192157,0.258824}%
\pgfsetstrokecolor{currentstroke}%
\pgfsetdash{}{0pt}%
\pgfsys@defobject{currentmarker}{\pgfqpoint{-0.033333in}{-0.033333in}}{\pgfqpoint{0.033333in}{0.033333in}}{%
\pgfpathmoveto{\pgfqpoint{-0.000000in}{-0.033333in}}%
\pgfpathlineto{\pgfqpoint{0.033333in}{0.033333in}}%
\pgfpathlineto{\pgfqpoint{-0.033333in}{0.033333in}}%
\pgfpathlineto{\pgfqpoint{-0.000000in}{-0.033333in}}%
\pgfpathclose%
\pgfusepath{stroke,fill}%
}%
\begin{pgfscope}%
\pgfsys@transformshift{3.350563in}{2.926523in}%
\pgfsys@useobject{currentmarker}{}%
\end{pgfscope}%
\end{pgfscope}%
\begin{pgfscope}%
\pgfpathrectangle{\pgfqpoint{0.765000in}{0.660000in}}{\pgfqpoint{4.620000in}{4.620000in}}%
\pgfusepath{clip}%
\pgfsetrectcap%
\pgfsetroundjoin%
\pgfsetlinewidth{1.204500pt}%
\definecolor{currentstroke}{rgb}{0.176471,0.192157,0.258824}%
\pgfsetstrokecolor{currentstroke}%
\pgfsetdash{}{0pt}%
\pgfpathmoveto{\pgfqpoint{3.101871in}{2.913666in}}%
\pgfusepath{stroke}%
\end{pgfscope}%
\begin{pgfscope}%
\pgfpathrectangle{\pgfqpoint{0.765000in}{0.660000in}}{\pgfqpoint{4.620000in}{4.620000in}}%
\pgfusepath{clip}%
\pgfsetbuttcap%
\pgfsetmiterjoin%
\definecolor{currentfill}{rgb}{0.176471,0.192157,0.258824}%
\pgfsetfillcolor{currentfill}%
\pgfsetlinewidth{1.003750pt}%
\definecolor{currentstroke}{rgb}{0.176471,0.192157,0.258824}%
\pgfsetstrokecolor{currentstroke}%
\pgfsetdash{}{0pt}%
\pgfsys@defobject{currentmarker}{\pgfqpoint{-0.033333in}{-0.033333in}}{\pgfqpoint{0.033333in}{0.033333in}}{%
\pgfpathmoveto{\pgfqpoint{-0.000000in}{-0.033333in}}%
\pgfpathlineto{\pgfqpoint{0.033333in}{0.033333in}}%
\pgfpathlineto{\pgfqpoint{-0.033333in}{0.033333in}}%
\pgfpathlineto{\pgfqpoint{-0.000000in}{-0.033333in}}%
\pgfpathclose%
\pgfusepath{stroke,fill}%
}%
\begin{pgfscope}%
\pgfsys@transformshift{3.101871in}{2.913666in}%
\pgfsys@useobject{currentmarker}{}%
\end{pgfscope}%
\end{pgfscope}%
\begin{pgfscope}%
\pgfpathrectangle{\pgfqpoint{0.765000in}{0.660000in}}{\pgfqpoint{4.620000in}{4.620000in}}%
\pgfusepath{clip}%
\pgfsetrectcap%
\pgfsetroundjoin%
\pgfsetlinewidth{1.204500pt}%
\definecolor{currentstroke}{rgb}{0.176471,0.192157,0.258824}%
\pgfsetstrokecolor{currentstroke}%
\pgfsetdash{}{0pt}%
\pgfpathmoveto{\pgfqpoint{3.236664in}{3.233008in}}%
\pgfusepath{stroke}%
\end{pgfscope}%
\begin{pgfscope}%
\pgfpathrectangle{\pgfqpoint{0.765000in}{0.660000in}}{\pgfqpoint{4.620000in}{4.620000in}}%
\pgfusepath{clip}%
\pgfsetbuttcap%
\pgfsetmiterjoin%
\definecolor{currentfill}{rgb}{0.176471,0.192157,0.258824}%
\pgfsetfillcolor{currentfill}%
\pgfsetlinewidth{1.003750pt}%
\definecolor{currentstroke}{rgb}{0.176471,0.192157,0.258824}%
\pgfsetstrokecolor{currentstroke}%
\pgfsetdash{}{0pt}%
\pgfsys@defobject{currentmarker}{\pgfqpoint{-0.033333in}{-0.033333in}}{\pgfqpoint{0.033333in}{0.033333in}}{%
\pgfpathmoveto{\pgfqpoint{-0.000000in}{-0.033333in}}%
\pgfpathlineto{\pgfqpoint{0.033333in}{0.033333in}}%
\pgfpathlineto{\pgfqpoint{-0.033333in}{0.033333in}}%
\pgfpathlineto{\pgfqpoint{-0.000000in}{-0.033333in}}%
\pgfpathclose%
\pgfusepath{stroke,fill}%
}%
\begin{pgfscope}%
\pgfsys@transformshift{3.236664in}{3.233008in}%
\pgfsys@useobject{currentmarker}{}%
\end{pgfscope}%
\end{pgfscope}%
\begin{pgfscope}%
\pgfpathrectangle{\pgfqpoint{0.765000in}{0.660000in}}{\pgfqpoint{4.620000in}{4.620000in}}%
\pgfusepath{clip}%
\pgfsetrectcap%
\pgfsetroundjoin%
\pgfsetlinewidth{1.204500pt}%
\definecolor{currentstroke}{rgb}{0.176471,0.192157,0.258824}%
\pgfsetstrokecolor{currentstroke}%
\pgfsetdash{}{0pt}%
\pgfpathmoveto{\pgfqpoint{2.971495in}{3.231013in}}%
\pgfusepath{stroke}%
\end{pgfscope}%
\begin{pgfscope}%
\pgfpathrectangle{\pgfqpoint{0.765000in}{0.660000in}}{\pgfqpoint{4.620000in}{4.620000in}}%
\pgfusepath{clip}%
\pgfsetbuttcap%
\pgfsetmiterjoin%
\definecolor{currentfill}{rgb}{0.176471,0.192157,0.258824}%
\pgfsetfillcolor{currentfill}%
\pgfsetlinewidth{1.003750pt}%
\definecolor{currentstroke}{rgb}{0.176471,0.192157,0.258824}%
\pgfsetstrokecolor{currentstroke}%
\pgfsetdash{}{0pt}%
\pgfsys@defobject{currentmarker}{\pgfqpoint{-0.033333in}{-0.033333in}}{\pgfqpoint{0.033333in}{0.033333in}}{%
\pgfpathmoveto{\pgfqpoint{-0.000000in}{-0.033333in}}%
\pgfpathlineto{\pgfqpoint{0.033333in}{0.033333in}}%
\pgfpathlineto{\pgfqpoint{-0.033333in}{0.033333in}}%
\pgfpathlineto{\pgfqpoint{-0.000000in}{-0.033333in}}%
\pgfpathclose%
\pgfusepath{stroke,fill}%
}%
\begin{pgfscope}%
\pgfsys@transformshift{2.971495in}{3.231013in}%
\pgfsys@useobject{currentmarker}{}%
\end{pgfscope}%
\end{pgfscope}%
\begin{pgfscope}%
\pgfpathrectangle{\pgfqpoint{0.765000in}{0.660000in}}{\pgfqpoint{4.620000in}{4.620000in}}%
\pgfusepath{clip}%
\pgfsetrectcap%
\pgfsetroundjoin%
\pgfsetlinewidth{1.204500pt}%
\definecolor{currentstroke}{rgb}{0.176471,0.192157,0.258824}%
\pgfsetstrokecolor{currentstroke}%
\pgfsetdash{}{0pt}%
\pgfpathmoveto{\pgfqpoint{3.002064in}{3.106786in}}%
\pgfusepath{stroke}%
\end{pgfscope}%
\begin{pgfscope}%
\pgfpathrectangle{\pgfqpoint{0.765000in}{0.660000in}}{\pgfqpoint{4.620000in}{4.620000in}}%
\pgfusepath{clip}%
\pgfsetbuttcap%
\pgfsetmiterjoin%
\definecolor{currentfill}{rgb}{0.176471,0.192157,0.258824}%
\pgfsetfillcolor{currentfill}%
\pgfsetlinewidth{1.003750pt}%
\definecolor{currentstroke}{rgb}{0.176471,0.192157,0.258824}%
\pgfsetstrokecolor{currentstroke}%
\pgfsetdash{}{0pt}%
\pgfsys@defobject{currentmarker}{\pgfqpoint{-0.033333in}{-0.033333in}}{\pgfqpoint{0.033333in}{0.033333in}}{%
\pgfpathmoveto{\pgfqpoint{-0.000000in}{-0.033333in}}%
\pgfpathlineto{\pgfqpoint{0.033333in}{0.033333in}}%
\pgfpathlineto{\pgfqpoint{-0.033333in}{0.033333in}}%
\pgfpathlineto{\pgfqpoint{-0.000000in}{-0.033333in}}%
\pgfpathclose%
\pgfusepath{stroke,fill}%
}%
\begin{pgfscope}%
\pgfsys@transformshift{3.002064in}{3.106786in}%
\pgfsys@useobject{currentmarker}{}%
\end{pgfscope}%
\end{pgfscope}%
\begin{pgfscope}%
\pgfpathrectangle{\pgfqpoint{0.765000in}{0.660000in}}{\pgfqpoint{4.620000in}{4.620000in}}%
\pgfusepath{clip}%
\pgfsetrectcap%
\pgfsetroundjoin%
\pgfsetlinewidth{1.204500pt}%
\definecolor{currentstroke}{rgb}{0.176471,0.192157,0.258824}%
\pgfsetstrokecolor{currentstroke}%
\pgfsetdash{}{0pt}%
\pgfpathmoveto{\pgfqpoint{3.208241in}{3.203556in}}%
\pgfusepath{stroke}%
\end{pgfscope}%
\begin{pgfscope}%
\pgfpathrectangle{\pgfqpoint{0.765000in}{0.660000in}}{\pgfqpoint{4.620000in}{4.620000in}}%
\pgfusepath{clip}%
\pgfsetbuttcap%
\pgfsetmiterjoin%
\definecolor{currentfill}{rgb}{0.176471,0.192157,0.258824}%
\pgfsetfillcolor{currentfill}%
\pgfsetlinewidth{1.003750pt}%
\definecolor{currentstroke}{rgb}{0.176471,0.192157,0.258824}%
\pgfsetstrokecolor{currentstroke}%
\pgfsetdash{}{0pt}%
\pgfsys@defobject{currentmarker}{\pgfqpoint{-0.033333in}{-0.033333in}}{\pgfqpoint{0.033333in}{0.033333in}}{%
\pgfpathmoveto{\pgfqpoint{-0.000000in}{-0.033333in}}%
\pgfpathlineto{\pgfqpoint{0.033333in}{0.033333in}}%
\pgfpathlineto{\pgfqpoint{-0.033333in}{0.033333in}}%
\pgfpathlineto{\pgfqpoint{-0.000000in}{-0.033333in}}%
\pgfpathclose%
\pgfusepath{stroke,fill}%
}%
\begin{pgfscope}%
\pgfsys@transformshift{3.208241in}{3.203556in}%
\pgfsys@useobject{currentmarker}{}%
\end{pgfscope}%
\end{pgfscope}%
\begin{pgfscope}%
\pgfpathrectangle{\pgfqpoint{0.765000in}{0.660000in}}{\pgfqpoint{4.620000in}{4.620000in}}%
\pgfusepath{clip}%
\pgfsetrectcap%
\pgfsetroundjoin%
\pgfsetlinewidth{1.204500pt}%
\definecolor{currentstroke}{rgb}{0.176471,0.192157,0.258824}%
\pgfsetstrokecolor{currentstroke}%
\pgfsetdash{}{0pt}%
\pgfpathmoveto{\pgfqpoint{2.985480in}{3.050363in}}%
\pgfusepath{stroke}%
\end{pgfscope}%
\begin{pgfscope}%
\pgfpathrectangle{\pgfqpoint{0.765000in}{0.660000in}}{\pgfqpoint{4.620000in}{4.620000in}}%
\pgfusepath{clip}%
\pgfsetbuttcap%
\pgfsetmiterjoin%
\definecolor{currentfill}{rgb}{0.176471,0.192157,0.258824}%
\pgfsetfillcolor{currentfill}%
\pgfsetlinewidth{1.003750pt}%
\definecolor{currentstroke}{rgb}{0.176471,0.192157,0.258824}%
\pgfsetstrokecolor{currentstroke}%
\pgfsetdash{}{0pt}%
\pgfsys@defobject{currentmarker}{\pgfqpoint{-0.033333in}{-0.033333in}}{\pgfqpoint{0.033333in}{0.033333in}}{%
\pgfpathmoveto{\pgfqpoint{-0.000000in}{-0.033333in}}%
\pgfpathlineto{\pgfqpoint{0.033333in}{0.033333in}}%
\pgfpathlineto{\pgfqpoint{-0.033333in}{0.033333in}}%
\pgfpathlineto{\pgfqpoint{-0.000000in}{-0.033333in}}%
\pgfpathclose%
\pgfusepath{stroke,fill}%
}%
\begin{pgfscope}%
\pgfsys@transformshift{2.985480in}{3.050363in}%
\pgfsys@useobject{currentmarker}{}%
\end{pgfscope}%
\end{pgfscope}%
\begin{pgfscope}%
\pgfpathrectangle{\pgfqpoint{0.765000in}{0.660000in}}{\pgfqpoint{4.620000in}{4.620000in}}%
\pgfusepath{clip}%
\pgfsetrectcap%
\pgfsetroundjoin%
\pgfsetlinewidth{1.204500pt}%
\definecolor{currentstroke}{rgb}{0.176471,0.192157,0.258824}%
\pgfsetstrokecolor{currentstroke}%
\pgfsetdash{}{0pt}%
\pgfpathmoveto{\pgfqpoint{3.113707in}{3.312267in}}%
\pgfusepath{stroke}%
\end{pgfscope}%
\begin{pgfscope}%
\pgfpathrectangle{\pgfqpoint{0.765000in}{0.660000in}}{\pgfqpoint{4.620000in}{4.620000in}}%
\pgfusepath{clip}%
\pgfsetbuttcap%
\pgfsetmiterjoin%
\definecolor{currentfill}{rgb}{0.176471,0.192157,0.258824}%
\pgfsetfillcolor{currentfill}%
\pgfsetlinewidth{1.003750pt}%
\definecolor{currentstroke}{rgb}{0.176471,0.192157,0.258824}%
\pgfsetstrokecolor{currentstroke}%
\pgfsetdash{}{0pt}%
\pgfsys@defobject{currentmarker}{\pgfqpoint{-0.033333in}{-0.033333in}}{\pgfqpoint{0.033333in}{0.033333in}}{%
\pgfpathmoveto{\pgfqpoint{-0.000000in}{-0.033333in}}%
\pgfpathlineto{\pgfqpoint{0.033333in}{0.033333in}}%
\pgfpathlineto{\pgfqpoint{-0.033333in}{0.033333in}}%
\pgfpathlineto{\pgfqpoint{-0.000000in}{-0.033333in}}%
\pgfpathclose%
\pgfusepath{stroke,fill}%
}%
\begin{pgfscope}%
\pgfsys@transformshift{3.113707in}{3.312267in}%
\pgfsys@useobject{currentmarker}{}%
\end{pgfscope}%
\end{pgfscope}%
\begin{pgfscope}%
\pgfpathrectangle{\pgfqpoint{0.765000in}{0.660000in}}{\pgfqpoint{4.620000in}{4.620000in}}%
\pgfusepath{clip}%
\pgfsetrectcap%
\pgfsetroundjoin%
\pgfsetlinewidth{1.204500pt}%
\definecolor{currentstroke}{rgb}{0.176471,0.192157,0.258824}%
\pgfsetstrokecolor{currentstroke}%
\pgfsetdash{}{0pt}%
\pgfpathmoveto{\pgfqpoint{3.209310in}{3.069953in}}%
\pgfusepath{stroke}%
\end{pgfscope}%
\begin{pgfscope}%
\pgfpathrectangle{\pgfqpoint{0.765000in}{0.660000in}}{\pgfqpoint{4.620000in}{4.620000in}}%
\pgfusepath{clip}%
\pgfsetbuttcap%
\pgfsetmiterjoin%
\definecolor{currentfill}{rgb}{0.176471,0.192157,0.258824}%
\pgfsetfillcolor{currentfill}%
\pgfsetlinewidth{1.003750pt}%
\definecolor{currentstroke}{rgb}{0.176471,0.192157,0.258824}%
\pgfsetstrokecolor{currentstroke}%
\pgfsetdash{}{0pt}%
\pgfsys@defobject{currentmarker}{\pgfqpoint{-0.033333in}{-0.033333in}}{\pgfqpoint{0.033333in}{0.033333in}}{%
\pgfpathmoveto{\pgfqpoint{-0.000000in}{-0.033333in}}%
\pgfpathlineto{\pgfqpoint{0.033333in}{0.033333in}}%
\pgfpathlineto{\pgfqpoint{-0.033333in}{0.033333in}}%
\pgfpathlineto{\pgfqpoint{-0.000000in}{-0.033333in}}%
\pgfpathclose%
\pgfusepath{stroke,fill}%
}%
\begin{pgfscope}%
\pgfsys@transformshift{3.209310in}{3.069953in}%
\pgfsys@useobject{currentmarker}{}%
\end{pgfscope}%
\end{pgfscope}%
\begin{pgfscope}%
\pgfpathrectangle{\pgfqpoint{0.765000in}{0.660000in}}{\pgfqpoint{4.620000in}{4.620000in}}%
\pgfusepath{clip}%
\pgfsetrectcap%
\pgfsetroundjoin%
\pgfsetlinewidth{1.204500pt}%
\definecolor{currentstroke}{rgb}{0.000000,0.000000,0.000000}%
\pgfsetstrokecolor{currentstroke}%
\pgfsetdash{}{0pt}%
\pgfpathmoveto{\pgfqpoint{3.121227in}{3.527081in}}%
\pgfusepath{stroke}%
\end{pgfscope}%
\begin{pgfscope}%
\pgfpathrectangle{\pgfqpoint{0.765000in}{0.660000in}}{\pgfqpoint{4.620000in}{4.620000in}}%
\pgfusepath{clip}%
\pgfsetbuttcap%
\pgfsetroundjoin%
\definecolor{currentfill}{rgb}{0.000000,0.000000,0.000000}%
\pgfsetfillcolor{currentfill}%
\pgfsetlinewidth{1.003750pt}%
\definecolor{currentstroke}{rgb}{0.000000,0.000000,0.000000}%
\pgfsetstrokecolor{currentstroke}%
\pgfsetdash{}{0pt}%
\pgfsys@defobject{currentmarker}{\pgfqpoint{-0.016667in}{-0.016667in}}{\pgfqpoint{0.016667in}{0.016667in}}{%
\pgfpathmoveto{\pgfqpoint{0.000000in}{-0.016667in}}%
\pgfpathcurveto{\pgfqpoint{0.004420in}{-0.016667in}}{\pgfqpoint{0.008660in}{-0.014911in}}{\pgfqpoint{0.011785in}{-0.011785in}}%
\pgfpathcurveto{\pgfqpoint{0.014911in}{-0.008660in}}{\pgfqpoint{0.016667in}{-0.004420in}}{\pgfqpoint{0.016667in}{0.000000in}}%
\pgfpathcurveto{\pgfqpoint{0.016667in}{0.004420in}}{\pgfqpoint{0.014911in}{0.008660in}}{\pgfqpoint{0.011785in}{0.011785in}}%
\pgfpathcurveto{\pgfqpoint{0.008660in}{0.014911in}}{\pgfqpoint{0.004420in}{0.016667in}}{\pgfqpoint{0.000000in}{0.016667in}}%
\pgfpathcurveto{\pgfqpoint{-0.004420in}{0.016667in}}{\pgfqpoint{-0.008660in}{0.014911in}}{\pgfqpoint{-0.011785in}{0.011785in}}%
\pgfpathcurveto{\pgfqpoint{-0.014911in}{0.008660in}}{\pgfqpoint{-0.016667in}{0.004420in}}{\pgfqpoint{-0.016667in}{0.000000in}}%
\pgfpathcurveto{\pgfqpoint{-0.016667in}{-0.004420in}}{\pgfqpoint{-0.014911in}{-0.008660in}}{\pgfqpoint{-0.011785in}{-0.011785in}}%
\pgfpathcurveto{\pgfqpoint{-0.008660in}{-0.014911in}}{\pgfqpoint{-0.004420in}{-0.016667in}}{\pgfqpoint{0.000000in}{-0.016667in}}%
\pgfpathlineto{\pgfqpoint{0.000000in}{-0.016667in}}%
\pgfpathclose%
\pgfusepath{stroke,fill}%
}%
\begin{pgfscope}%
\pgfsys@transformshift{3.121227in}{3.527081in}%
\pgfsys@useobject{currentmarker}{}%
\end{pgfscope}%
\end{pgfscope}%
\begin{pgfscope}%
\pgfpathrectangle{\pgfqpoint{0.765000in}{0.660000in}}{\pgfqpoint{4.620000in}{4.620000in}}%
\pgfusepath{clip}%
\pgfsetrectcap%
\pgfsetroundjoin%
\pgfsetlinewidth{1.204500pt}%
\definecolor{currentstroke}{rgb}{0.000000,0.000000,0.000000}%
\pgfsetstrokecolor{currentstroke}%
\pgfsetdash{}{0pt}%
\pgfpathmoveto{\pgfqpoint{3.121227in}{3.527081in}}%
\pgfpathlineto{\pgfqpoint{3.145667in}{3.522403in}}%
\pgfusepath{stroke}%
\end{pgfscope}%
\begin{pgfscope}%
\pgfpathrectangle{\pgfqpoint{0.765000in}{0.660000in}}{\pgfqpoint{4.620000in}{4.620000in}}%
\pgfusepath{clip}%
\pgfsetrectcap%
\pgfsetroundjoin%
\pgfsetlinewidth{1.204500pt}%
\definecolor{currentstroke}{rgb}{0.000000,0.000000,0.000000}%
\pgfsetstrokecolor{currentstroke}%
\pgfsetdash{}{0pt}%
\pgfpathmoveto{\pgfqpoint{3.121227in}{3.527081in}}%
\pgfpathlineto{\pgfqpoint{3.107502in}{3.529134in}}%
\pgfusepath{stroke}%
\end{pgfscope}%
\begin{pgfscope}%
\pgfpathrectangle{\pgfqpoint{0.765000in}{0.660000in}}{\pgfqpoint{4.620000in}{4.620000in}}%
\pgfusepath{clip}%
\pgfsetbuttcap%
\pgfsetroundjoin%
\pgfsetlinewidth{1.204500pt}%
\definecolor{currentstroke}{rgb}{0.827451,0.827451,0.827451}%
\pgfsetstrokecolor{currentstroke}%
\pgfsetdash{{1.200000pt}{1.980000pt}}{0.000000pt}%
\pgfpathmoveto{\pgfqpoint{3.121227in}{3.527081in}}%
\pgfpathlineto{\pgfqpoint{3.157562in}{4.564965in}}%
\pgfusepath{stroke}%
\end{pgfscope}%
\begin{pgfscope}%
\pgfpathrectangle{\pgfqpoint{0.765000in}{0.660000in}}{\pgfqpoint{4.620000in}{4.620000in}}%
\pgfusepath{clip}%
\pgfsetrectcap%
\pgfsetroundjoin%
\pgfsetlinewidth{1.204500pt}%
\definecolor{currentstroke}{rgb}{0.000000,0.000000,0.000000}%
\pgfsetstrokecolor{currentstroke}%
\pgfsetdash{}{0pt}%
\pgfpathmoveto{\pgfqpoint{3.145667in}{3.522403in}}%
\pgfusepath{stroke}%
\end{pgfscope}%
\begin{pgfscope}%
\pgfpathrectangle{\pgfqpoint{0.765000in}{0.660000in}}{\pgfqpoint{4.620000in}{4.620000in}}%
\pgfusepath{clip}%
\pgfsetbuttcap%
\pgfsetroundjoin%
\definecolor{currentfill}{rgb}{0.000000,0.000000,0.000000}%
\pgfsetfillcolor{currentfill}%
\pgfsetlinewidth{1.003750pt}%
\definecolor{currentstroke}{rgb}{0.000000,0.000000,0.000000}%
\pgfsetstrokecolor{currentstroke}%
\pgfsetdash{}{0pt}%
\pgfsys@defobject{currentmarker}{\pgfqpoint{-0.016667in}{-0.016667in}}{\pgfqpoint{0.016667in}{0.016667in}}{%
\pgfpathmoveto{\pgfqpoint{0.000000in}{-0.016667in}}%
\pgfpathcurveto{\pgfqpoint{0.004420in}{-0.016667in}}{\pgfqpoint{0.008660in}{-0.014911in}}{\pgfqpoint{0.011785in}{-0.011785in}}%
\pgfpathcurveto{\pgfqpoint{0.014911in}{-0.008660in}}{\pgfqpoint{0.016667in}{-0.004420in}}{\pgfqpoint{0.016667in}{0.000000in}}%
\pgfpathcurveto{\pgfqpoint{0.016667in}{0.004420in}}{\pgfqpoint{0.014911in}{0.008660in}}{\pgfqpoint{0.011785in}{0.011785in}}%
\pgfpathcurveto{\pgfqpoint{0.008660in}{0.014911in}}{\pgfqpoint{0.004420in}{0.016667in}}{\pgfqpoint{0.000000in}{0.016667in}}%
\pgfpathcurveto{\pgfqpoint{-0.004420in}{0.016667in}}{\pgfqpoint{-0.008660in}{0.014911in}}{\pgfqpoint{-0.011785in}{0.011785in}}%
\pgfpathcurveto{\pgfqpoint{-0.014911in}{0.008660in}}{\pgfqpoint{-0.016667in}{0.004420in}}{\pgfqpoint{-0.016667in}{0.000000in}}%
\pgfpathcurveto{\pgfqpoint{-0.016667in}{-0.004420in}}{\pgfqpoint{-0.014911in}{-0.008660in}}{\pgfqpoint{-0.011785in}{-0.011785in}}%
\pgfpathcurveto{\pgfqpoint{-0.008660in}{-0.014911in}}{\pgfqpoint{-0.004420in}{-0.016667in}}{\pgfqpoint{0.000000in}{-0.016667in}}%
\pgfpathlineto{\pgfqpoint{0.000000in}{-0.016667in}}%
\pgfpathclose%
\pgfusepath{stroke,fill}%
}%
\begin{pgfscope}%
\pgfsys@transformshift{3.145667in}{3.522403in}%
\pgfsys@useobject{currentmarker}{}%
\end{pgfscope}%
\end{pgfscope}%
\begin{pgfscope}%
\pgfpathrectangle{\pgfqpoint{0.765000in}{0.660000in}}{\pgfqpoint{4.620000in}{4.620000in}}%
\pgfusepath{clip}%
\pgfsetrectcap%
\pgfsetroundjoin%
\pgfsetlinewidth{1.204500pt}%
\definecolor{currentstroke}{rgb}{0.000000,0.000000,0.000000}%
\pgfsetstrokecolor{currentstroke}%
\pgfsetdash{}{0pt}%
\pgfpathmoveto{\pgfqpoint{3.145667in}{3.522403in}}%
\pgfpathlineto{\pgfqpoint{3.121227in}{3.527081in}}%
\pgfusepath{stroke}%
\end{pgfscope}%
\begin{pgfscope}%
\pgfpathrectangle{\pgfqpoint{0.765000in}{0.660000in}}{\pgfqpoint{4.620000in}{4.620000in}}%
\pgfusepath{clip}%
\pgfsetrectcap%
\pgfsetroundjoin%
\pgfsetlinewidth{1.204500pt}%
\definecolor{currentstroke}{rgb}{0.000000,0.000000,0.000000}%
\pgfsetstrokecolor{currentstroke}%
\pgfsetdash{}{0pt}%
\pgfpathmoveto{\pgfqpoint{3.145667in}{3.522403in}}%
\pgfpathlineto{\pgfqpoint{3.230111in}{3.500132in}}%
\pgfusepath{stroke}%
\end{pgfscope}%
\begin{pgfscope}%
\pgfpathrectangle{\pgfqpoint{0.765000in}{0.660000in}}{\pgfqpoint{4.620000in}{4.620000in}}%
\pgfusepath{clip}%
\pgfsetbuttcap%
\pgfsetroundjoin%
\pgfsetlinewidth{1.204500pt}%
\definecolor{currentstroke}{rgb}{0.827451,0.827451,0.827451}%
\pgfsetstrokecolor{currentstroke}%
\pgfsetdash{{1.200000pt}{1.980000pt}}{0.000000pt}%
\pgfpathmoveto{\pgfqpoint{3.145667in}{3.522403in}}%
\pgfpathlineto{\pgfqpoint{3.324262in}{5.274056in}}%
\pgfusepath{stroke}%
\end{pgfscope}%
\begin{pgfscope}%
\pgfpathrectangle{\pgfqpoint{0.765000in}{0.660000in}}{\pgfqpoint{4.620000in}{4.620000in}}%
\pgfusepath{clip}%
\pgfsetrectcap%
\pgfsetroundjoin%
\pgfsetlinewidth{1.204500pt}%
\definecolor{currentstroke}{rgb}{0.000000,0.000000,0.000000}%
\pgfsetstrokecolor{currentstroke}%
\pgfsetdash{}{0pt}%
\pgfpathmoveto{\pgfqpoint{2.979627in}{2.512096in}}%
\pgfusepath{stroke}%
\end{pgfscope}%
\begin{pgfscope}%
\pgfpathrectangle{\pgfqpoint{0.765000in}{0.660000in}}{\pgfqpoint{4.620000in}{4.620000in}}%
\pgfusepath{clip}%
\pgfsetbuttcap%
\pgfsetroundjoin%
\definecolor{currentfill}{rgb}{0.000000,0.000000,0.000000}%
\pgfsetfillcolor{currentfill}%
\pgfsetlinewidth{1.003750pt}%
\definecolor{currentstroke}{rgb}{0.000000,0.000000,0.000000}%
\pgfsetstrokecolor{currentstroke}%
\pgfsetdash{}{0pt}%
\pgfsys@defobject{currentmarker}{\pgfqpoint{-0.016667in}{-0.016667in}}{\pgfqpoint{0.016667in}{0.016667in}}{%
\pgfpathmoveto{\pgfqpoint{0.000000in}{-0.016667in}}%
\pgfpathcurveto{\pgfqpoint{0.004420in}{-0.016667in}}{\pgfqpoint{0.008660in}{-0.014911in}}{\pgfqpoint{0.011785in}{-0.011785in}}%
\pgfpathcurveto{\pgfqpoint{0.014911in}{-0.008660in}}{\pgfqpoint{0.016667in}{-0.004420in}}{\pgfqpoint{0.016667in}{0.000000in}}%
\pgfpathcurveto{\pgfqpoint{0.016667in}{0.004420in}}{\pgfqpoint{0.014911in}{0.008660in}}{\pgfqpoint{0.011785in}{0.011785in}}%
\pgfpathcurveto{\pgfqpoint{0.008660in}{0.014911in}}{\pgfqpoint{0.004420in}{0.016667in}}{\pgfqpoint{0.000000in}{0.016667in}}%
\pgfpathcurveto{\pgfqpoint{-0.004420in}{0.016667in}}{\pgfqpoint{-0.008660in}{0.014911in}}{\pgfqpoint{-0.011785in}{0.011785in}}%
\pgfpathcurveto{\pgfqpoint{-0.014911in}{0.008660in}}{\pgfqpoint{-0.016667in}{0.004420in}}{\pgfqpoint{-0.016667in}{0.000000in}}%
\pgfpathcurveto{\pgfqpoint{-0.016667in}{-0.004420in}}{\pgfqpoint{-0.014911in}{-0.008660in}}{\pgfqpoint{-0.011785in}{-0.011785in}}%
\pgfpathcurveto{\pgfqpoint{-0.008660in}{-0.014911in}}{\pgfqpoint{-0.004420in}{-0.016667in}}{\pgfqpoint{0.000000in}{-0.016667in}}%
\pgfpathlineto{\pgfqpoint{0.000000in}{-0.016667in}}%
\pgfpathclose%
\pgfusepath{stroke,fill}%
}%
\begin{pgfscope}%
\pgfsys@transformshift{2.979627in}{2.512096in}%
\pgfsys@useobject{currentmarker}{}%
\end{pgfscope}%
\end{pgfscope}%
\begin{pgfscope}%
\pgfpathrectangle{\pgfqpoint{0.765000in}{0.660000in}}{\pgfqpoint{4.620000in}{4.620000in}}%
\pgfusepath{clip}%
\pgfsetrectcap%
\pgfsetroundjoin%
\pgfsetlinewidth{1.204500pt}%
\definecolor{currentstroke}{rgb}{0.000000,0.000000,0.000000}%
\pgfsetstrokecolor{currentstroke}%
\pgfsetdash{}{0pt}%
\pgfpathmoveto{\pgfqpoint{2.979627in}{2.512096in}}%
\pgfpathlineto{\pgfqpoint{2.952980in}{2.520265in}}%
\pgfusepath{stroke}%
\end{pgfscope}%
\begin{pgfscope}%
\pgfpathrectangle{\pgfqpoint{0.765000in}{0.660000in}}{\pgfqpoint{4.620000in}{4.620000in}}%
\pgfusepath{clip}%
\pgfsetrectcap%
\pgfsetroundjoin%
\pgfsetlinewidth{1.204500pt}%
\definecolor{currentstroke}{rgb}{0.000000,0.000000,0.000000}%
\pgfsetstrokecolor{currentstroke}%
\pgfsetdash{}{0pt}%
\pgfpathmoveto{\pgfqpoint{2.979627in}{2.512096in}}%
\pgfpathlineto{\pgfqpoint{3.071936in}{2.492937in}}%
\pgfusepath{stroke}%
\end{pgfscope}%
\begin{pgfscope}%
\pgfpathrectangle{\pgfqpoint{0.765000in}{0.660000in}}{\pgfqpoint{4.620000in}{4.620000in}}%
\pgfusepath{clip}%
\pgfsetbuttcap%
\pgfsetroundjoin%
\pgfsetlinewidth{1.204500pt}%
\definecolor{currentstroke}{rgb}{0.827451,0.827451,0.827451}%
\pgfsetstrokecolor{currentstroke}%
\pgfsetdash{{1.200000pt}{1.980000pt}}{0.000000pt}%
\pgfpathmoveto{\pgfqpoint{2.979627in}{2.512096in}}%
\pgfpathlineto{\pgfqpoint{2.436056in}{1.097272in}}%
\pgfusepath{stroke}%
\end{pgfscope}%
\begin{pgfscope}%
\pgfpathrectangle{\pgfqpoint{0.765000in}{0.660000in}}{\pgfqpoint{4.620000in}{4.620000in}}%
\pgfusepath{clip}%
\pgfsetrectcap%
\pgfsetroundjoin%
\pgfsetlinewidth{1.204500pt}%
\definecolor{currentstroke}{rgb}{0.000000,0.000000,0.000000}%
\pgfsetstrokecolor{currentstroke}%
\pgfsetdash{}{0pt}%
\pgfpathmoveto{\pgfqpoint{2.952980in}{2.520265in}}%
\pgfusepath{stroke}%
\end{pgfscope}%
\begin{pgfscope}%
\pgfpathrectangle{\pgfqpoint{0.765000in}{0.660000in}}{\pgfqpoint{4.620000in}{4.620000in}}%
\pgfusepath{clip}%
\pgfsetbuttcap%
\pgfsetroundjoin%
\definecolor{currentfill}{rgb}{0.000000,0.000000,0.000000}%
\pgfsetfillcolor{currentfill}%
\pgfsetlinewidth{1.003750pt}%
\definecolor{currentstroke}{rgb}{0.000000,0.000000,0.000000}%
\pgfsetstrokecolor{currentstroke}%
\pgfsetdash{}{0pt}%
\pgfsys@defobject{currentmarker}{\pgfqpoint{-0.016667in}{-0.016667in}}{\pgfqpoint{0.016667in}{0.016667in}}{%
\pgfpathmoveto{\pgfqpoint{0.000000in}{-0.016667in}}%
\pgfpathcurveto{\pgfqpoint{0.004420in}{-0.016667in}}{\pgfqpoint{0.008660in}{-0.014911in}}{\pgfqpoint{0.011785in}{-0.011785in}}%
\pgfpathcurveto{\pgfqpoint{0.014911in}{-0.008660in}}{\pgfqpoint{0.016667in}{-0.004420in}}{\pgfqpoint{0.016667in}{0.000000in}}%
\pgfpathcurveto{\pgfqpoint{0.016667in}{0.004420in}}{\pgfqpoint{0.014911in}{0.008660in}}{\pgfqpoint{0.011785in}{0.011785in}}%
\pgfpathcurveto{\pgfqpoint{0.008660in}{0.014911in}}{\pgfqpoint{0.004420in}{0.016667in}}{\pgfqpoint{0.000000in}{0.016667in}}%
\pgfpathcurveto{\pgfqpoint{-0.004420in}{0.016667in}}{\pgfqpoint{-0.008660in}{0.014911in}}{\pgfqpoint{-0.011785in}{0.011785in}}%
\pgfpathcurveto{\pgfqpoint{-0.014911in}{0.008660in}}{\pgfqpoint{-0.016667in}{0.004420in}}{\pgfqpoint{-0.016667in}{0.000000in}}%
\pgfpathcurveto{\pgfqpoint{-0.016667in}{-0.004420in}}{\pgfqpoint{-0.014911in}{-0.008660in}}{\pgfqpoint{-0.011785in}{-0.011785in}}%
\pgfpathcurveto{\pgfqpoint{-0.008660in}{-0.014911in}}{\pgfqpoint{-0.004420in}{-0.016667in}}{\pgfqpoint{0.000000in}{-0.016667in}}%
\pgfpathlineto{\pgfqpoint{0.000000in}{-0.016667in}}%
\pgfpathclose%
\pgfusepath{stroke,fill}%
}%
\begin{pgfscope}%
\pgfsys@transformshift{2.952980in}{2.520265in}%
\pgfsys@useobject{currentmarker}{}%
\end{pgfscope}%
\end{pgfscope}%
\begin{pgfscope}%
\pgfpathrectangle{\pgfqpoint{0.765000in}{0.660000in}}{\pgfqpoint{4.620000in}{4.620000in}}%
\pgfusepath{clip}%
\pgfsetrectcap%
\pgfsetroundjoin%
\pgfsetlinewidth{1.204500pt}%
\definecolor{currentstroke}{rgb}{0.000000,0.000000,0.000000}%
\pgfsetstrokecolor{currentstroke}%
\pgfsetdash{}{0pt}%
\pgfpathmoveto{\pgfqpoint{2.952980in}{2.520265in}}%
\pgfpathlineto{\pgfqpoint{2.979627in}{2.512096in}}%
\pgfusepath{stroke}%
\end{pgfscope}%
\begin{pgfscope}%
\pgfpathrectangle{\pgfqpoint{0.765000in}{0.660000in}}{\pgfqpoint{4.620000in}{4.620000in}}%
\pgfusepath{clip}%
\pgfsetrectcap%
\pgfsetroundjoin%
\pgfsetlinewidth{1.204500pt}%
\definecolor{currentstroke}{rgb}{0.000000,0.000000,0.000000}%
\pgfsetstrokecolor{currentstroke}%
\pgfsetdash{}{0pt}%
\pgfpathmoveto{\pgfqpoint{2.952980in}{2.520265in}}%
\pgfpathlineto{\pgfqpoint{2.929883in}{2.530541in}}%
\pgfusepath{stroke}%
\end{pgfscope}%
\begin{pgfscope}%
\pgfpathrectangle{\pgfqpoint{0.765000in}{0.660000in}}{\pgfqpoint{4.620000in}{4.620000in}}%
\pgfusepath{clip}%
\pgfsetbuttcap%
\pgfsetroundjoin%
\pgfsetlinewidth{1.204500pt}%
\definecolor{currentstroke}{rgb}{0.827451,0.827451,0.827451}%
\pgfsetstrokecolor{currentstroke}%
\pgfsetdash{{1.200000pt}{1.980000pt}}{0.000000pt}%
\pgfpathmoveto{\pgfqpoint{2.952980in}{2.520265in}}%
\pgfpathlineto{\pgfqpoint{2.105960in}{0.763619in}}%
\pgfusepath{stroke}%
\end{pgfscope}%
\begin{pgfscope}%
\pgfpathrectangle{\pgfqpoint{0.765000in}{0.660000in}}{\pgfqpoint{4.620000in}{4.620000in}}%
\pgfusepath{clip}%
\pgfsetrectcap%
\pgfsetroundjoin%
\pgfsetlinewidth{1.204500pt}%
\definecolor{currentstroke}{rgb}{0.000000,0.000000,0.000000}%
\pgfsetstrokecolor{currentstroke}%
\pgfsetdash{}{0pt}%
\pgfpathmoveto{\pgfqpoint{3.519640in}{2.712796in}}%
\pgfusepath{stroke}%
\end{pgfscope}%
\begin{pgfscope}%
\pgfpathrectangle{\pgfqpoint{0.765000in}{0.660000in}}{\pgfqpoint{4.620000in}{4.620000in}}%
\pgfusepath{clip}%
\pgfsetbuttcap%
\pgfsetroundjoin%
\definecolor{currentfill}{rgb}{0.000000,0.000000,0.000000}%
\pgfsetfillcolor{currentfill}%
\pgfsetlinewidth{1.003750pt}%
\definecolor{currentstroke}{rgb}{0.000000,0.000000,0.000000}%
\pgfsetstrokecolor{currentstroke}%
\pgfsetdash{}{0pt}%
\pgfsys@defobject{currentmarker}{\pgfqpoint{-0.016667in}{-0.016667in}}{\pgfqpoint{0.016667in}{0.016667in}}{%
\pgfpathmoveto{\pgfqpoint{0.000000in}{-0.016667in}}%
\pgfpathcurveto{\pgfqpoint{0.004420in}{-0.016667in}}{\pgfqpoint{0.008660in}{-0.014911in}}{\pgfqpoint{0.011785in}{-0.011785in}}%
\pgfpathcurveto{\pgfqpoint{0.014911in}{-0.008660in}}{\pgfqpoint{0.016667in}{-0.004420in}}{\pgfqpoint{0.016667in}{0.000000in}}%
\pgfpathcurveto{\pgfqpoint{0.016667in}{0.004420in}}{\pgfqpoint{0.014911in}{0.008660in}}{\pgfqpoint{0.011785in}{0.011785in}}%
\pgfpathcurveto{\pgfqpoint{0.008660in}{0.014911in}}{\pgfqpoint{0.004420in}{0.016667in}}{\pgfqpoint{0.000000in}{0.016667in}}%
\pgfpathcurveto{\pgfqpoint{-0.004420in}{0.016667in}}{\pgfqpoint{-0.008660in}{0.014911in}}{\pgfqpoint{-0.011785in}{0.011785in}}%
\pgfpathcurveto{\pgfqpoint{-0.014911in}{0.008660in}}{\pgfqpoint{-0.016667in}{0.004420in}}{\pgfqpoint{-0.016667in}{0.000000in}}%
\pgfpathcurveto{\pgfqpoint{-0.016667in}{-0.004420in}}{\pgfqpoint{-0.014911in}{-0.008660in}}{\pgfqpoint{-0.011785in}{-0.011785in}}%
\pgfpathcurveto{\pgfqpoint{-0.008660in}{-0.014911in}}{\pgfqpoint{-0.004420in}{-0.016667in}}{\pgfqpoint{0.000000in}{-0.016667in}}%
\pgfpathlineto{\pgfqpoint{0.000000in}{-0.016667in}}%
\pgfpathclose%
\pgfusepath{stroke,fill}%
}%
\begin{pgfscope}%
\pgfsys@transformshift{3.519640in}{2.712796in}%
\pgfsys@useobject{currentmarker}{}%
\end{pgfscope}%
\end{pgfscope}%
\begin{pgfscope}%
\pgfpathrectangle{\pgfqpoint{0.765000in}{0.660000in}}{\pgfqpoint{4.620000in}{4.620000in}}%
\pgfusepath{clip}%
\pgfsetrectcap%
\pgfsetroundjoin%
\pgfsetlinewidth{1.204500pt}%
\definecolor{currentstroke}{rgb}{0.000000,0.000000,0.000000}%
\pgfsetstrokecolor{currentstroke}%
\pgfsetdash{}{0pt}%
\pgfpathmoveto{\pgfqpoint{3.519640in}{2.712796in}}%
\pgfpathlineto{\pgfqpoint{3.546263in}{2.765531in}}%
\pgfusepath{stroke}%
\end{pgfscope}%
\begin{pgfscope}%
\pgfpathrectangle{\pgfqpoint{0.765000in}{0.660000in}}{\pgfqpoint{4.620000in}{4.620000in}}%
\pgfusepath{clip}%
\pgfsetrectcap%
\pgfsetroundjoin%
\pgfsetlinewidth{1.204500pt}%
\definecolor{currentstroke}{rgb}{0.000000,0.000000,0.000000}%
\pgfsetstrokecolor{currentstroke}%
\pgfsetdash{}{0pt}%
\pgfpathmoveto{\pgfqpoint{3.519640in}{2.712796in}}%
\pgfpathlineto{\pgfqpoint{3.477043in}{2.674663in}}%
\pgfusepath{stroke}%
\end{pgfscope}%
\begin{pgfscope}%
\pgfpathrectangle{\pgfqpoint{0.765000in}{0.660000in}}{\pgfqpoint{4.620000in}{4.620000in}}%
\pgfusepath{clip}%
\pgfsetbuttcap%
\pgfsetroundjoin%
\pgfsetlinewidth{1.204500pt}%
\definecolor{currentstroke}{rgb}{0.827451,0.827451,0.827451}%
\pgfsetstrokecolor{currentstroke}%
\pgfsetdash{{1.200000pt}{1.980000pt}}{0.000000pt}%
\pgfpathmoveto{\pgfqpoint{3.519640in}{2.712796in}}%
\pgfpathlineto{\pgfqpoint{5.151498in}{0.650000in}}%
\pgfusepath{stroke}%
\end{pgfscope}%
\begin{pgfscope}%
\pgfpathrectangle{\pgfqpoint{0.765000in}{0.660000in}}{\pgfqpoint{4.620000in}{4.620000in}}%
\pgfusepath{clip}%
\pgfsetrectcap%
\pgfsetroundjoin%
\pgfsetlinewidth{1.204500pt}%
\definecolor{currentstroke}{rgb}{0.000000,0.000000,0.000000}%
\pgfsetstrokecolor{currentstroke}%
\pgfsetdash{}{0pt}%
\pgfpathmoveto{\pgfqpoint{3.546263in}{2.765531in}}%
\pgfusepath{stroke}%
\end{pgfscope}%
\begin{pgfscope}%
\pgfpathrectangle{\pgfqpoint{0.765000in}{0.660000in}}{\pgfqpoint{4.620000in}{4.620000in}}%
\pgfusepath{clip}%
\pgfsetbuttcap%
\pgfsetroundjoin%
\definecolor{currentfill}{rgb}{0.000000,0.000000,0.000000}%
\pgfsetfillcolor{currentfill}%
\pgfsetlinewidth{1.003750pt}%
\definecolor{currentstroke}{rgb}{0.000000,0.000000,0.000000}%
\pgfsetstrokecolor{currentstroke}%
\pgfsetdash{}{0pt}%
\pgfsys@defobject{currentmarker}{\pgfqpoint{-0.016667in}{-0.016667in}}{\pgfqpoint{0.016667in}{0.016667in}}{%
\pgfpathmoveto{\pgfqpoint{0.000000in}{-0.016667in}}%
\pgfpathcurveto{\pgfqpoint{0.004420in}{-0.016667in}}{\pgfqpoint{0.008660in}{-0.014911in}}{\pgfqpoint{0.011785in}{-0.011785in}}%
\pgfpathcurveto{\pgfqpoint{0.014911in}{-0.008660in}}{\pgfqpoint{0.016667in}{-0.004420in}}{\pgfqpoint{0.016667in}{0.000000in}}%
\pgfpathcurveto{\pgfqpoint{0.016667in}{0.004420in}}{\pgfqpoint{0.014911in}{0.008660in}}{\pgfqpoint{0.011785in}{0.011785in}}%
\pgfpathcurveto{\pgfqpoint{0.008660in}{0.014911in}}{\pgfqpoint{0.004420in}{0.016667in}}{\pgfqpoint{0.000000in}{0.016667in}}%
\pgfpathcurveto{\pgfqpoint{-0.004420in}{0.016667in}}{\pgfqpoint{-0.008660in}{0.014911in}}{\pgfqpoint{-0.011785in}{0.011785in}}%
\pgfpathcurveto{\pgfqpoint{-0.014911in}{0.008660in}}{\pgfqpoint{-0.016667in}{0.004420in}}{\pgfqpoint{-0.016667in}{0.000000in}}%
\pgfpathcurveto{\pgfqpoint{-0.016667in}{-0.004420in}}{\pgfqpoint{-0.014911in}{-0.008660in}}{\pgfqpoint{-0.011785in}{-0.011785in}}%
\pgfpathcurveto{\pgfqpoint{-0.008660in}{-0.014911in}}{\pgfqpoint{-0.004420in}{-0.016667in}}{\pgfqpoint{0.000000in}{-0.016667in}}%
\pgfpathlineto{\pgfqpoint{0.000000in}{-0.016667in}}%
\pgfpathclose%
\pgfusepath{stroke,fill}%
}%
\begin{pgfscope}%
\pgfsys@transformshift{3.546263in}{2.765531in}%
\pgfsys@useobject{currentmarker}{}%
\end{pgfscope}%
\end{pgfscope}%
\begin{pgfscope}%
\pgfpathrectangle{\pgfqpoint{0.765000in}{0.660000in}}{\pgfqpoint{4.620000in}{4.620000in}}%
\pgfusepath{clip}%
\pgfsetrectcap%
\pgfsetroundjoin%
\pgfsetlinewidth{1.204500pt}%
\definecolor{currentstroke}{rgb}{0.000000,0.000000,0.000000}%
\pgfsetstrokecolor{currentstroke}%
\pgfsetdash{}{0pt}%
\pgfpathmoveto{\pgfqpoint{3.546263in}{2.765531in}}%
\pgfpathlineto{\pgfqpoint{3.519640in}{2.712796in}}%
\pgfusepath{stroke}%
\end{pgfscope}%
\begin{pgfscope}%
\pgfpathrectangle{\pgfqpoint{0.765000in}{0.660000in}}{\pgfqpoint{4.620000in}{4.620000in}}%
\pgfusepath{clip}%
\pgfsetrectcap%
\pgfsetroundjoin%
\pgfsetlinewidth{1.204500pt}%
\definecolor{currentstroke}{rgb}{0.000000,0.000000,0.000000}%
\pgfsetstrokecolor{currentstroke}%
\pgfsetdash{}{0pt}%
\pgfpathmoveto{\pgfqpoint{3.546263in}{2.765531in}}%
\pgfpathlineto{\pgfqpoint{3.551407in}{2.776901in}}%
\pgfusepath{stroke}%
\end{pgfscope}%
\begin{pgfscope}%
\pgfpathrectangle{\pgfqpoint{0.765000in}{0.660000in}}{\pgfqpoint{4.620000in}{4.620000in}}%
\pgfusepath{clip}%
\pgfsetbuttcap%
\pgfsetroundjoin%
\pgfsetlinewidth{1.204500pt}%
\definecolor{currentstroke}{rgb}{0.827451,0.827451,0.827451}%
\pgfsetstrokecolor{currentstroke}%
\pgfsetdash{{1.200000pt}{1.980000pt}}{0.000000pt}%
\pgfpathmoveto{\pgfqpoint{3.546263in}{2.765531in}}%
\pgfpathlineto{\pgfqpoint{4.287284in}{2.155930in}}%
\pgfusepath{stroke}%
\end{pgfscope}%
\begin{pgfscope}%
\pgfpathrectangle{\pgfqpoint{0.765000in}{0.660000in}}{\pgfqpoint{4.620000in}{4.620000in}}%
\pgfusepath{clip}%
\pgfsetrectcap%
\pgfsetroundjoin%
\pgfsetlinewidth{1.204500pt}%
\definecolor{currentstroke}{rgb}{0.000000,0.000000,0.000000}%
\pgfsetstrokecolor{currentstroke}%
\pgfsetdash{}{0pt}%
\pgfpathmoveto{\pgfqpoint{2.708745in}{3.356240in}}%
\pgfusepath{stroke}%
\end{pgfscope}%
\begin{pgfscope}%
\pgfpathrectangle{\pgfqpoint{0.765000in}{0.660000in}}{\pgfqpoint{4.620000in}{4.620000in}}%
\pgfusepath{clip}%
\pgfsetbuttcap%
\pgfsetroundjoin%
\definecolor{currentfill}{rgb}{0.000000,0.000000,0.000000}%
\pgfsetfillcolor{currentfill}%
\pgfsetlinewidth{1.003750pt}%
\definecolor{currentstroke}{rgb}{0.000000,0.000000,0.000000}%
\pgfsetstrokecolor{currentstroke}%
\pgfsetdash{}{0pt}%
\pgfsys@defobject{currentmarker}{\pgfqpoint{-0.016667in}{-0.016667in}}{\pgfqpoint{0.016667in}{0.016667in}}{%
\pgfpathmoveto{\pgfqpoint{0.000000in}{-0.016667in}}%
\pgfpathcurveto{\pgfqpoint{0.004420in}{-0.016667in}}{\pgfqpoint{0.008660in}{-0.014911in}}{\pgfqpoint{0.011785in}{-0.011785in}}%
\pgfpathcurveto{\pgfqpoint{0.014911in}{-0.008660in}}{\pgfqpoint{0.016667in}{-0.004420in}}{\pgfqpoint{0.016667in}{0.000000in}}%
\pgfpathcurveto{\pgfqpoint{0.016667in}{0.004420in}}{\pgfqpoint{0.014911in}{0.008660in}}{\pgfqpoint{0.011785in}{0.011785in}}%
\pgfpathcurveto{\pgfqpoint{0.008660in}{0.014911in}}{\pgfqpoint{0.004420in}{0.016667in}}{\pgfqpoint{0.000000in}{0.016667in}}%
\pgfpathcurveto{\pgfqpoint{-0.004420in}{0.016667in}}{\pgfqpoint{-0.008660in}{0.014911in}}{\pgfqpoint{-0.011785in}{0.011785in}}%
\pgfpathcurveto{\pgfqpoint{-0.014911in}{0.008660in}}{\pgfqpoint{-0.016667in}{0.004420in}}{\pgfqpoint{-0.016667in}{0.000000in}}%
\pgfpathcurveto{\pgfqpoint{-0.016667in}{-0.004420in}}{\pgfqpoint{-0.014911in}{-0.008660in}}{\pgfqpoint{-0.011785in}{-0.011785in}}%
\pgfpathcurveto{\pgfqpoint{-0.008660in}{-0.014911in}}{\pgfqpoint{-0.004420in}{-0.016667in}}{\pgfqpoint{0.000000in}{-0.016667in}}%
\pgfpathlineto{\pgfqpoint{0.000000in}{-0.016667in}}%
\pgfpathclose%
\pgfusepath{stroke,fill}%
}%
\begin{pgfscope}%
\pgfsys@transformshift{2.708745in}{3.356240in}%
\pgfsys@useobject{currentmarker}{}%
\end{pgfscope}%
\end{pgfscope}%
\begin{pgfscope}%
\pgfpathrectangle{\pgfqpoint{0.765000in}{0.660000in}}{\pgfqpoint{4.620000in}{4.620000in}}%
\pgfusepath{clip}%
\pgfsetrectcap%
\pgfsetroundjoin%
\pgfsetlinewidth{1.204500pt}%
\definecolor{currentstroke}{rgb}{0.000000,0.000000,0.000000}%
\pgfsetstrokecolor{currentstroke}%
\pgfsetdash{}{0pt}%
\pgfpathmoveto{\pgfqpoint{2.708745in}{3.356240in}}%
\pgfpathlineto{\pgfqpoint{2.692300in}{3.344662in}}%
\pgfusepath{stroke}%
\end{pgfscope}%
\begin{pgfscope}%
\pgfpathrectangle{\pgfqpoint{0.765000in}{0.660000in}}{\pgfqpoint{4.620000in}{4.620000in}}%
\pgfusepath{clip}%
\pgfsetrectcap%
\pgfsetroundjoin%
\pgfsetlinewidth{1.204500pt}%
\definecolor{currentstroke}{rgb}{0.000000,0.000000,0.000000}%
\pgfsetstrokecolor{currentstroke}%
\pgfsetdash{}{0pt}%
\pgfpathmoveto{\pgfqpoint{2.708745in}{3.356240in}}%
\pgfpathlineto{\pgfqpoint{2.788056in}{3.406767in}}%
\pgfusepath{stroke}%
\end{pgfscope}%
\begin{pgfscope}%
\pgfpathrectangle{\pgfqpoint{0.765000in}{0.660000in}}{\pgfqpoint{4.620000in}{4.620000in}}%
\pgfusepath{clip}%
\pgfsetbuttcap%
\pgfsetroundjoin%
\pgfsetlinewidth{1.204500pt}%
\definecolor{currentstroke}{rgb}{0.827451,0.827451,0.827451}%
\pgfsetstrokecolor{currentstroke}%
\pgfsetdash{{1.200000pt}{1.980000pt}}{0.000000pt}%
\pgfpathmoveto{\pgfqpoint{2.708745in}{3.356240in}}%
\pgfpathlineto{\pgfqpoint{1.943771in}{4.048839in}}%
\pgfusepath{stroke}%
\end{pgfscope}%
\begin{pgfscope}%
\pgfpathrectangle{\pgfqpoint{0.765000in}{0.660000in}}{\pgfqpoint{4.620000in}{4.620000in}}%
\pgfusepath{clip}%
\pgfsetrectcap%
\pgfsetroundjoin%
\pgfsetlinewidth{1.204500pt}%
\definecolor{currentstroke}{rgb}{0.000000,0.000000,0.000000}%
\pgfsetstrokecolor{currentstroke}%
\pgfsetdash{}{0pt}%
\pgfpathmoveto{\pgfqpoint{2.692300in}{3.344662in}}%
\pgfusepath{stroke}%
\end{pgfscope}%
\begin{pgfscope}%
\pgfpathrectangle{\pgfqpoint{0.765000in}{0.660000in}}{\pgfqpoint{4.620000in}{4.620000in}}%
\pgfusepath{clip}%
\pgfsetbuttcap%
\pgfsetroundjoin%
\definecolor{currentfill}{rgb}{0.000000,0.000000,0.000000}%
\pgfsetfillcolor{currentfill}%
\pgfsetlinewidth{1.003750pt}%
\definecolor{currentstroke}{rgb}{0.000000,0.000000,0.000000}%
\pgfsetstrokecolor{currentstroke}%
\pgfsetdash{}{0pt}%
\pgfsys@defobject{currentmarker}{\pgfqpoint{-0.016667in}{-0.016667in}}{\pgfqpoint{0.016667in}{0.016667in}}{%
\pgfpathmoveto{\pgfqpoint{0.000000in}{-0.016667in}}%
\pgfpathcurveto{\pgfqpoint{0.004420in}{-0.016667in}}{\pgfqpoint{0.008660in}{-0.014911in}}{\pgfqpoint{0.011785in}{-0.011785in}}%
\pgfpathcurveto{\pgfqpoint{0.014911in}{-0.008660in}}{\pgfqpoint{0.016667in}{-0.004420in}}{\pgfqpoint{0.016667in}{0.000000in}}%
\pgfpathcurveto{\pgfqpoint{0.016667in}{0.004420in}}{\pgfqpoint{0.014911in}{0.008660in}}{\pgfqpoint{0.011785in}{0.011785in}}%
\pgfpathcurveto{\pgfqpoint{0.008660in}{0.014911in}}{\pgfqpoint{0.004420in}{0.016667in}}{\pgfqpoint{0.000000in}{0.016667in}}%
\pgfpathcurveto{\pgfqpoint{-0.004420in}{0.016667in}}{\pgfqpoint{-0.008660in}{0.014911in}}{\pgfqpoint{-0.011785in}{0.011785in}}%
\pgfpathcurveto{\pgfqpoint{-0.014911in}{0.008660in}}{\pgfqpoint{-0.016667in}{0.004420in}}{\pgfqpoint{-0.016667in}{0.000000in}}%
\pgfpathcurveto{\pgfqpoint{-0.016667in}{-0.004420in}}{\pgfqpoint{-0.014911in}{-0.008660in}}{\pgfqpoint{-0.011785in}{-0.011785in}}%
\pgfpathcurveto{\pgfqpoint{-0.008660in}{-0.014911in}}{\pgfqpoint{-0.004420in}{-0.016667in}}{\pgfqpoint{0.000000in}{-0.016667in}}%
\pgfpathlineto{\pgfqpoint{0.000000in}{-0.016667in}}%
\pgfpathclose%
\pgfusepath{stroke,fill}%
}%
\begin{pgfscope}%
\pgfsys@transformshift{2.692300in}{3.344662in}%
\pgfsys@useobject{currentmarker}{}%
\end{pgfscope}%
\end{pgfscope}%
\begin{pgfscope}%
\pgfpathrectangle{\pgfqpoint{0.765000in}{0.660000in}}{\pgfqpoint{4.620000in}{4.620000in}}%
\pgfusepath{clip}%
\pgfsetrectcap%
\pgfsetroundjoin%
\pgfsetlinewidth{1.204500pt}%
\definecolor{currentstroke}{rgb}{0.000000,0.000000,0.000000}%
\pgfsetstrokecolor{currentstroke}%
\pgfsetdash{}{0pt}%
\pgfpathmoveto{\pgfqpoint{2.692300in}{3.344662in}}%
\pgfpathlineto{\pgfqpoint{2.708745in}{3.356240in}}%
\pgfusepath{stroke}%
\end{pgfscope}%
\begin{pgfscope}%
\pgfpathrectangle{\pgfqpoint{0.765000in}{0.660000in}}{\pgfqpoint{4.620000in}{4.620000in}}%
\pgfusepath{clip}%
\pgfsetrectcap%
\pgfsetroundjoin%
\pgfsetlinewidth{1.204500pt}%
\definecolor{currentstroke}{rgb}{0.000000,0.000000,0.000000}%
\pgfsetstrokecolor{currentstroke}%
\pgfsetdash{}{0pt}%
\pgfpathmoveto{\pgfqpoint{2.692300in}{3.344662in}}%
\pgfpathlineto{\pgfqpoint{2.671300in}{3.327008in}}%
\pgfusepath{stroke}%
\end{pgfscope}%
\begin{pgfscope}%
\pgfpathrectangle{\pgfqpoint{0.765000in}{0.660000in}}{\pgfqpoint{4.620000in}{4.620000in}}%
\pgfusepath{clip}%
\pgfsetbuttcap%
\pgfsetroundjoin%
\pgfsetlinewidth{1.204500pt}%
\definecolor{currentstroke}{rgb}{0.827451,0.827451,0.827451}%
\pgfsetstrokecolor{currentstroke}%
\pgfsetdash{{1.200000pt}{1.980000pt}}{0.000000pt}%
\pgfpathmoveto{\pgfqpoint{2.692300in}{3.344662in}}%
\pgfpathlineto{\pgfqpoint{1.212209in}{4.508665in}}%
\pgfusepath{stroke}%
\end{pgfscope}%
\begin{pgfscope}%
\pgfpathrectangle{\pgfqpoint{0.765000in}{0.660000in}}{\pgfqpoint{4.620000in}{4.620000in}}%
\pgfusepath{clip}%
\pgfsetrectcap%
\pgfsetroundjoin%
\pgfsetlinewidth{1.204500pt}%
\definecolor{currentstroke}{rgb}{0.000000,0.000000,0.000000}%
\pgfsetstrokecolor{currentstroke}%
\pgfsetdash{}{0pt}%
\pgfpathmoveto{\pgfqpoint{3.425935in}{3.326669in}}%
\pgfusepath{stroke}%
\end{pgfscope}%
\begin{pgfscope}%
\pgfpathrectangle{\pgfqpoint{0.765000in}{0.660000in}}{\pgfqpoint{4.620000in}{4.620000in}}%
\pgfusepath{clip}%
\pgfsetbuttcap%
\pgfsetroundjoin%
\definecolor{currentfill}{rgb}{0.000000,0.000000,0.000000}%
\pgfsetfillcolor{currentfill}%
\pgfsetlinewidth{1.003750pt}%
\definecolor{currentstroke}{rgb}{0.000000,0.000000,0.000000}%
\pgfsetstrokecolor{currentstroke}%
\pgfsetdash{}{0pt}%
\pgfsys@defobject{currentmarker}{\pgfqpoint{-0.016667in}{-0.016667in}}{\pgfqpoint{0.016667in}{0.016667in}}{%
\pgfpathmoveto{\pgfqpoint{0.000000in}{-0.016667in}}%
\pgfpathcurveto{\pgfqpoint{0.004420in}{-0.016667in}}{\pgfqpoint{0.008660in}{-0.014911in}}{\pgfqpoint{0.011785in}{-0.011785in}}%
\pgfpathcurveto{\pgfqpoint{0.014911in}{-0.008660in}}{\pgfqpoint{0.016667in}{-0.004420in}}{\pgfqpoint{0.016667in}{0.000000in}}%
\pgfpathcurveto{\pgfqpoint{0.016667in}{0.004420in}}{\pgfqpoint{0.014911in}{0.008660in}}{\pgfqpoint{0.011785in}{0.011785in}}%
\pgfpathcurveto{\pgfqpoint{0.008660in}{0.014911in}}{\pgfqpoint{0.004420in}{0.016667in}}{\pgfqpoint{0.000000in}{0.016667in}}%
\pgfpathcurveto{\pgfqpoint{-0.004420in}{0.016667in}}{\pgfqpoint{-0.008660in}{0.014911in}}{\pgfqpoint{-0.011785in}{0.011785in}}%
\pgfpathcurveto{\pgfqpoint{-0.014911in}{0.008660in}}{\pgfqpoint{-0.016667in}{0.004420in}}{\pgfqpoint{-0.016667in}{0.000000in}}%
\pgfpathcurveto{\pgfqpoint{-0.016667in}{-0.004420in}}{\pgfqpoint{-0.014911in}{-0.008660in}}{\pgfqpoint{-0.011785in}{-0.011785in}}%
\pgfpathcurveto{\pgfqpoint{-0.008660in}{-0.014911in}}{\pgfqpoint{-0.004420in}{-0.016667in}}{\pgfqpoint{0.000000in}{-0.016667in}}%
\pgfpathlineto{\pgfqpoint{0.000000in}{-0.016667in}}%
\pgfpathclose%
\pgfusepath{stroke,fill}%
}%
\begin{pgfscope}%
\pgfsys@transformshift{3.425935in}{3.326669in}%
\pgfsys@useobject{currentmarker}{}%
\end{pgfscope}%
\end{pgfscope}%
\begin{pgfscope}%
\pgfpathrectangle{\pgfqpoint{0.765000in}{0.660000in}}{\pgfqpoint{4.620000in}{4.620000in}}%
\pgfusepath{clip}%
\pgfsetrectcap%
\pgfsetroundjoin%
\pgfsetlinewidth{1.204500pt}%
\definecolor{currentstroke}{rgb}{0.000000,0.000000,0.000000}%
\pgfsetstrokecolor{currentstroke}%
\pgfsetdash{}{0pt}%
\pgfpathmoveto{\pgfqpoint{3.425935in}{3.326669in}}%
\pgfpathlineto{\pgfqpoint{3.357740in}{3.416522in}}%
\pgfusepath{stroke}%
\end{pgfscope}%
\begin{pgfscope}%
\pgfpathrectangle{\pgfqpoint{0.765000in}{0.660000in}}{\pgfqpoint{4.620000in}{4.620000in}}%
\pgfusepath{clip}%
\pgfsetrectcap%
\pgfsetroundjoin%
\pgfsetlinewidth{1.204500pt}%
\definecolor{currentstroke}{rgb}{0.000000,0.000000,0.000000}%
\pgfsetstrokecolor{currentstroke}%
\pgfsetdash{}{0pt}%
\pgfpathmoveto{\pgfqpoint{3.425935in}{3.326669in}}%
\pgfpathlineto{\pgfqpoint{3.575586in}{2.945282in}}%
\pgfusepath{stroke}%
\end{pgfscope}%
\begin{pgfscope}%
\pgfpathrectangle{\pgfqpoint{0.765000in}{0.660000in}}{\pgfqpoint{4.620000in}{4.620000in}}%
\pgfusepath{clip}%
\pgfsetbuttcap%
\pgfsetroundjoin%
\pgfsetlinewidth{1.204500pt}%
\definecolor{currentstroke}{rgb}{0.827451,0.827451,0.827451}%
\pgfsetstrokecolor{currentstroke}%
\pgfsetdash{{1.200000pt}{1.980000pt}}{0.000000pt}%
\pgfpathmoveto{\pgfqpoint{3.425935in}{3.326669in}}%
\pgfpathlineto{\pgfqpoint{5.395000in}{4.234008in}}%
\pgfusepath{stroke}%
\end{pgfscope}%
\begin{pgfscope}%
\pgfpathrectangle{\pgfqpoint{0.765000in}{0.660000in}}{\pgfqpoint{4.620000in}{4.620000in}}%
\pgfusepath{clip}%
\pgfsetrectcap%
\pgfsetroundjoin%
\pgfsetlinewidth{1.204500pt}%
\definecolor{currentstroke}{rgb}{0.000000,0.000000,0.000000}%
\pgfsetstrokecolor{currentstroke}%
\pgfsetdash{}{0pt}%
\pgfpathmoveto{\pgfqpoint{3.357740in}{3.416522in}}%
\pgfusepath{stroke}%
\end{pgfscope}%
\begin{pgfscope}%
\pgfpathrectangle{\pgfqpoint{0.765000in}{0.660000in}}{\pgfqpoint{4.620000in}{4.620000in}}%
\pgfusepath{clip}%
\pgfsetbuttcap%
\pgfsetroundjoin%
\definecolor{currentfill}{rgb}{0.000000,0.000000,0.000000}%
\pgfsetfillcolor{currentfill}%
\pgfsetlinewidth{1.003750pt}%
\definecolor{currentstroke}{rgb}{0.000000,0.000000,0.000000}%
\pgfsetstrokecolor{currentstroke}%
\pgfsetdash{}{0pt}%
\pgfsys@defobject{currentmarker}{\pgfqpoint{-0.016667in}{-0.016667in}}{\pgfqpoint{0.016667in}{0.016667in}}{%
\pgfpathmoveto{\pgfqpoint{0.000000in}{-0.016667in}}%
\pgfpathcurveto{\pgfqpoint{0.004420in}{-0.016667in}}{\pgfqpoint{0.008660in}{-0.014911in}}{\pgfqpoint{0.011785in}{-0.011785in}}%
\pgfpathcurveto{\pgfqpoint{0.014911in}{-0.008660in}}{\pgfqpoint{0.016667in}{-0.004420in}}{\pgfqpoint{0.016667in}{0.000000in}}%
\pgfpathcurveto{\pgfqpoint{0.016667in}{0.004420in}}{\pgfqpoint{0.014911in}{0.008660in}}{\pgfqpoint{0.011785in}{0.011785in}}%
\pgfpathcurveto{\pgfqpoint{0.008660in}{0.014911in}}{\pgfqpoint{0.004420in}{0.016667in}}{\pgfqpoint{0.000000in}{0.016667in}}%
\pgfpathcurveto{\pgfqpoint{-0.004420in}{0.016667in}}{\pgfqpoint{-0.008660in}{0.014911in}}{\pgfqpoint{-0.011785in}{0.011785in}}%
\pgfpathcurveto{\pgfqpoint{-0.014911in}{0.008660in}}{\pgfqpoint{-0.016667in}{0.004420in}}{\pgfqpoint{-0.016667in}{0.000000in}}%
\pgfpathcurveto{\pgfqpoint{-0.016667in}{-0.004420in}}{\pgfqpoint{-0.014911in}{-0.008660in}}{\pgfqpoint{-0.011785in}{-0.011785in}}%
\pgfpathcurveto{\pgfqpoint{-0.008660in}{-0.014911in}}{\pgfqpoint{-0.004420in}{-0.016667in}}{\pgfqpoint{0.000000in}{-0.016667in}}%
\pgfpathlineto{\pgfqpoint{0.000000in}{-0.016667in}}%
\pgfpathclose%
\pgfusepath{stroke,fill}%
}%
\begin{pgfscope}%
\pgfsys@transformshift{3.357740in}{3.416522in}%
\pgfsys@useobject{currentmarker}{}%
\end{pgfscope}%
\end{pgfscope}%
\begin{pgfscope}%
\pgfpathrectangle{\pgfqpoint{0.765000in}{0.660000in}}{\pgfqpoint{4.620000in}{4.620000in}}%
\pgfusepath{clip}%
\pgfsetrectcap%
\pgfsetroundjoin%
\pgfsetlinewidth{1.204500pt}%
\definecolor{currentstroke}{rgb}{0.000000,0.000000,0.000000}%
\pgfsetstrokecolor{currentstroke}%
\pgfsetdash{}{0pt}%
\pgfpathmoveto{\pgfqpoint{3.357740in}{3.416522in}}%
\pgfpathlineto{\pgfqpoint{3.425935in}{3.326669in}}%
\pgfusepath{stroke}%
\end{pgfscope}%
\begin{pgfscope}%
\pgfpathrectangle{\pgfqpoint{0.765000in}{0.660000in}}{\pgfqpoint{4.620000in}{4.620000in}}%
\pgfusepath{clip}%
\pgfsetrectcap%
\pgfsetroundjoin%
\pgfsetlinewidth{1.204500pt}%
\definecolor{currentstroke}{rgb}{0.000000,0.000000,0.000000}%
\pgfsetstrokecolor{currentstroke}%
\pgfsetdash{}{0pt}%
\pgfpathmoveto{\pgfqpoint{3.357740in}{3.416522in}}%
\pgfpathlineto{\pgfqpoint{3.350280in}{3.424899in}}%
\pgfusepath{stroke}%
\end{pgfscope}%
\begin{pgfscope}%
\pgfpathrectangle{\pgfqpoint{0.765000in}{0.660000in}}{\pgfqpoint{4.620000in}{4.620000in}}%
\pgfusepath{clip}%
\pgfsetbuttcap%
\pgfsetroundjoin%
\pgfsetlinewidth{1.204500pt}%
\definecolor{currentstroke}{rgb}{0.827451,0.827451,0.827451}%
\pgfsetstrokecolor{currentstroke}%
\pgfsetdash{{1.200000pt}{1.980000pt}}{0.000000pt}%
\pgfpathmoveto{\pgfqpoint{3.357740in}{3.416522in}}%
\pgfpathlineto{\pgfqpoint{4.990350in}{4.659893in}}%
\pgfusepath{stroke}%
\end{pgfscope}%
\begin{pgfscope}%
\pgfpathrectangle{\pgfqpoint{0.765000in}{0.660000in}}{\pgfqpoint{4.620000in}{4.620000in}}%
\pgfusepath{clip}%
\pgfsetrectcap%
\pgfsetroundjoin%
\pgfsetlinewidth{1.204500pt}%
\definecolor{currentstroke}{rgb}{0.000000,0.000000,0.000000}%
\pgfsetstrokecolor{currentstroke}%
\pgfsetdash{}{0pt}%
\pgfpathmoveto{\pgfqpoint{2.604377in}{3.013317in}}%
\pgfusepath{stroke}%
\end{pgfscope}%
\begin{pgfscope}%
\pgfpathrectangle{\pgfqpoint{0.765000in}{0.660000in}}{\pgfqpoint{4.620000in}{4.620000in}}%
\pgfusepath{clip}%
\pgfsetbuttcap%
\pgfsetroundjoin%
\definecolor{currentfill}{rgb}{0.000000,0.000000,0.000000}%
\pgfsetfillcolor{currentfill}%
\pgfsetlinewidth{1.003750pt}%
\definecolor{currentstroke}{rgb}{0.000000,0.000000,0.000000}%
\pgfsetstrokecolor{currentstroke}%
\pgfsetdash{}{0pt}%
\pgfsys@defobject{currentmarker}{\pgfqpoint{-0.016667in}{-0.016667in}}{\pgfqpoint{0.016667in}{0.016667in}}{%
\pgfpathmoveto{\pgfqpoint{0.000000in}{-0.016667in}}%
\pgfpathcurveto{\pgfqpoint{0.004420in}{-0.016667in}}{\pgfqpoint{0.008660in}{-0.014911in}}{\pgfqpoint{0.011785in}{-0.011785in}}%
\pgfpathcurveto{\pgfqpoint{0.014911in}{-0.008660in}}{\pgfqpoint{0.016667in}{-0.004420in}}{\pgfqpoint{0.016667in}{0.000000in}}%
\pgfpathcurveto{\pgfqpoint{0.016667in}{0.004420in}}{\pgfqpoint{0.014911in}{0.008660in}}{\pgfqpoint{0.011785in}{0.011785in}}%
\pgfpathcurveto{\pgfqpoint{0.008660in}{0.014911in}}{\pgfqpoint{0.004420in}{0.016667in}}{\pgfqpoint{0.000000in}{0.016667in}}%
\pgfpathcurveto{\pgfqpoint{-0.004420in}{0.016667in}}{\pgfqpoint{-0.008660in}{0.014911in}}{\pgfqpoint{-0.011785in}{0.011785in}}%
\pgfpathcurveto{\pgfqpoint{-0.014911in}{0.008660in}}{\pgfqpoint{-0.016667in}{0.004420in}}{\pgfqpoint{-0.016667in}{0.000000in}}%
\pgfpathcurveto{\pgfqpoint{-0.016667in}{-0.004420in}}{\pgfqpoint{-0.014911in}{-0.008660in}}{\pgfqpoint{-0.011785in}{-0.011785in}}%
\pgfpathcurveto{\pgfqpoint{-0.008660in}{-0.014911in}}{\pgfqpoint{-0.004420in}{-0.016667in}}{\pgfqpoint{0.000000in}{-0.016667in}}%
\pgfpathlineto{\pgfqpoint{0.000000in}{-0.016667in}}%
\pgfpathclose%
\pgfusepath{stroke,fill}%
}%
\begin{pgfscope}%
\pgfsys@transformshift{2.604377in}{3.013317in}%
\pgfsys@useobject{currentmarker}{}%
\end{pgfscope}%
\end{pgfscope}%
\begin{pgfscope}%
\pgfpathrectangle{\pgfqpoint{0.765000in}{0.660000in}}{\pgfqpoint{4.620000in}{4.620000in}}%
\pgfusepath{clip}%
\pgfsetrectcap%
\pgfsetroundjoin%
\pgfsetlinewidth{1.204500pt}%
\definecolor{currentstroke}{rgb}{0.000000,0.000000,0.000000}%
\pgfsetstrokecolor{currentstroke}%
\pgfsetdash{}{0pt}%
\pgfpathmoveto{\pgfqpoint{2.604377in}{3.013317in}}%
\pgfpathlineto{\pgfqpoint{2.672370in}{2.810270in}}%
\pgfusepath{stroke}%
\end{pgfscope}%
\begin{pgfscope}%
\pgfpathrectangle{\pgfqpoint{0.765000in}{0.660000in}}{\pgfqpoint{4.620000in}{4.620000in}}%
\pgfusepath{clip}%
\pgfsetrectcap%
\pgfsetroundjoin%
\pgfsetlinewidth{1.204500pt}%
\definecolor{currentstroke}{rgb}{0.000000,0.000000,0.000000}%
\pgfsetstrokecolor{currentstroke}%
\pgfsetdash{}{0pt}%
\pgfpathmoveto{\pgfqpoint{2.604377in}{3.013317in}}%
\pgfpathlineto{\pgfqpoint{2.586744in}{3.073269in}}%
\pgfusepath{stroke}%
\end{pgfscope}%
\begin{pgfscope}%
\pgfpathrectangle{\pgfqpoint{0.765000in}{0.660000in}}{\pgfqpoint{4.620000in}{4.620000in}}%
\pgfusepath{clip}%
\pgfsetbuttcap%
\pgfsetroundjoin%
\pgfsetlinewidth{1.204500pt}%
\definecolor{currentstroke}{rgb}{0.827451,0.827451,0.827451}%
\pgfsetstrokecolor{currentstroke}%
\pgfsetdash{{1.200000pt}{1.980000pt}}{0.000000pt}%
\pgfpathmoveto{\pgfqpoint{2.604377in}{3.013317in}}%
\pgfpathlineto{\pgfqpoint{2.050533in}{2.757798in}}%
\pgfusepath{stroke}%
\end{pgfscope}%
\begin{pgfscope}%
\pgfpathrectangle{\pgfqpoint{0.765000in}{0.660000in}}{\pgfqpoint{4.620000in}{4.620000in}}%
\pgfusepath{clip}%
\pgfsetrectcap%
\pgfsetroundjoin%
\pgfsetlinewidth{1.204500pt}%
\definecolor{currentstroke}{rgb}{0.000000,0.000000,0.000000}%
\pgfsetstrokecolor{currentstroke}%
\pgfsetdash{}{0pt}%
\pgfpathmoveto{\pgfqpoint{2.672370in}{2.810270in}}%
\pgfusepath{stroke}%
\end{pgfscope}%
\begin{pgfscope}%
\pgfpathrectangle{\pgfqpoint{0.765000in}{0.660000in}}{\pgfqpoint{4.620000in}{4.620000in}}%
\pgfusepath{clip}%
\pgfsetbuttcap%
\pgfsetroundjoin%
\definecolor{currentfill}{rgb}{0.000000,0.000000,0.000000}%
\pgfsetfillcolor{currentfill}%
\pgfsetlinewidth{1.003750pt}%
\definecolor{currentstroke}{rgb}{0.000000,0.000000,0.000000}%
\pgfsetstrokecolor{currentstroke}%
\pgfsetdash{}{0pt}%
\pgfsys@defobject{currentmarker}{\pgfqpoint{-0.016667in}{-0.016667in}}{\pgfqpoint{0.016667in}{0.016667in}}{%
\pgfpathmoveto{\pgfqpoint{0.000000in}{-0.016667in}}%
\pgfpathcurveto{\pgfqpoint{0.004420in}{-0.016667in}}{\pgfqpoint{0.008660in}{-0.014911in}}{\pgfqpoint{0.011785in}{-0.011785in}}%
\pgfpathcurveto{\pgfqpoint{0.014911in}{-0.008660in}}{\pgfqpoint{0.016667in}{-0.004420in}}{\pgfqpoint{0.016667in}{0.000000in}}%
\pgfpathcurveto{\pgfqpoint{0.016667in}{0.004420in}}{\pgfqpoint{0.014911in}{0.008660in}}{\pgfqpoint{0.011785in}{0.011785in}}%
\pgfpathcurveto{\pgfqpoint{0.008660in}{0.014911in}}{\pgfqpoint{0.004420in}{0.016667in}}{\pgfqpoint{0.000000in}{0.016667in}}%
\pgfpathcurveto{\pgfqpoint{-0.004420in}{0.016667in}}{\pgfqpoint{-0.008660in}{0.014911in}}{\pgfqpoint{-0.011785in}{0.011785in}}%
\pgfpathcurveto{\pgfqpoint{-0.014911in}{0.008660in}}{\pgfqpoint{-0.016667in}{0.004420in}}{\pgfqpoint{-0.016667in}{0.000000in}}%
\pgfpathcurveto{\pgfqpoint{-0.016667in}{-0.004420in}}{\pgfqpoint{-0.014911in}{-0.008660in}}{\pgfqpoint{-0.011785in}{-0.011785in}}%
\pgfpathcurveto{\pgfqpoint{-0.008660in}{-0.014911in}}{\pgfqpoint{-0.004420in}{-0.016667in}}{\pgfqpoint{0.000000in}{-0.016667in}}%
\pgfpathlineto{\pgfqpoint{0.000000in}{-0.016667in}}%
\pgfpathclose%
\pgfusepath{stroke,fill}%
}%
\begin{pgfscope}%
\pgfsys@transformshift{2.672370in}{2.810270in}%
\pgfsys@useobject{currentmarker}{}%
\end{pgfscope}%
\end{pgfscope}%
\begin{pgfscope}%
\pgfpathrectangle{\pgfqpoint{0.765000in}{0.660000in}}{\pgfqpoint{4.620000in}{4.620000in}}%
\pgfusepath{clip}%
\pgfsetrectcap%
\pgfsetroundjoin%
\pgfsetlinewidth{1.204500pt}%
\definecolor{currentstroke}{rgb}{0.000000,0.000000,0.000000}%
\pgfsetstrokecolor{currentstroke}%
\pgfsetdash{}{0pt}%
\pgfpathmoveto{\pgfqpoint{2.672370in}{2.810270in}}%
\pgfpathlineto{\pgfqpoint{2.604377in}{3.013317in}}%
\pgfusepath{stroke}%
\end{pgfscope}%
\begin{pgfscope}%
\pgfpathrectangle{\pgfqpoint{0.765000in}{0.660000in}}{\pgfqpoint{4.620000in}{4.620000in}}%
\pgfusepath{clip}%
\pgfsetrectcap%
\pgfsetroundjoin%
\pgfsetlinewidth{1.204500pt}%
\definecolor{currentstroke}{rgb}{0.000000,0.000000,0.000000}%
\pgfsetstrokecolor{currentstroke}%
\pgfsetdash{}{0pt}%
\pgfpathmoveto{\pgfqpoint{2.672370in}{2.810270in}}%
\pgfpathlineto{\pgfqpoint{2.688236in}{2.773437in}}%
\pgfusepath{stroke}%
\end{pgfscope}%
\begin{pgfscope}%
\pgfpathrectangle{\pgfqpoint{0.765000in}{0.660000in}}{\pgfqpoint{4.620000in}{4.620000in}}%
\pgfusepath{clip}%
\pgfsetbuttcap%
\pgfsetroundjoin%
\pgfsetlinewidth{1.204500pt}%
\definecolor{currentstroke}{rgb}{0.827451,0.827451,0.827451}%
\pgfsetstrokecolor{currentstroke}%
\pgfsetdash{{1.200000pt}{1.980000pt}}{0.000000pt}%
\pgfpathmoveto{\pgfqpoint{2.672370in}{2.810270in}}%
\pgfpathlineto{\pgfqpoint{1.462297in}{2.181492in}}%
\pgfusepath{stroke}%
\end{pgfscope}%
\begin{pgfscope}%
\pgfpathrectangle{\pgfqpoint{0.765000in}{0.660000in}}{\pgfqpoint{4.620000in}{4.620000in}}%
\pgfusepath{clip}%
\pgfsetrectcap%
\pgfsetroundjoin%
\pgfsetlinewidth{1.204500pt}%
\definecolor{currentstroke}{rgb}{0.000000,0.000000,0.000000}%
\pgfsetstrokecolor{currentstroke}%
\pgfsetdash{}{0pt}%
\pgfpathmoveto{\pgfqpoint{3.575586in}{2.945282in}}%
\pgfusepath{stroke}%
\end{pgfscope}%
\begin{pgfscope}%
\pgfpathrectangle{\pgfqpoint{0.765000in}{0.660000in}}{\pgfqpoint{4.620000in}{4.620000in}}%
\pgfusepath{clip}%
\pgfsetbuttcap%
\pgfsetroundjoin%
\definecolor{currentfill}{rgb}{0.000000,0.000000,0.000000}%
\pgfsetfillcolor{currentfill}%
\pgfsetlinewidth{1.003750pt}%
\definecolor{currentstroke}{rgb}{0.000000,0.000000,0.000000}%
\pgfsetstrokecolor{currentstroke}%
\pgfsetdash{}{0pt}%
\pgfsys@defobject{currentmarker}{\pgfqpoint{-0.016667in}{-0.016667in}}{\pgfqpoint{0.016667in}{0.016667in}}{%
\pgfpathmoveto{\pgfqpoint{0.000000in}{-0.016667in}}%
\pgfpathcurveto{\pgfqpoint{0.004420in}{-0.016667in}}{\pgfqpoint{0.008660in}{-0.014911in}}{\pgfqpoint{0.011785in}{-0.011785in}}%
\pgfpathcurveto{\pgfqpoint{0.014911in}{-0.008660in}}{\pgfqpoint{0.016667in}{-0.004420in}}{\pgfqpoint{0.016667in}{0.000000in}}%
\pgfpathcurveto{\pgfqpoint{0.016667in}{0.004420in}}{\pgfqpoint{0.014911in}{0.008660in}}{\pgfqpoint{0.011785in}{0.011785in}}%
\pgfpathcurveto{\pgfqpoint{0.008660in}{0.014911in}}{\pgfqpoint{0.004420in}{0.016667in}}{\pgfqpoint{0.000000in}{0.016667in}}%
\pgfpathcurveto{\pgfqpoint{-0.004420in}{0.016667in}}{\pgfqpoint{-0.008660in}{0.014911in}}{\pgfqpoint{-0.011785in}{0.011785in}}%
\pgfpathcurveto{\pgfqpoint{-0.014911in}{0.008660in}}{\pgfqpoint{-0.016667in}{0.004420in}}{\pgfqpoint{-0.016667in}{0.000000in}}%
\pgfpathcurveto{\pgfqpoint{-0.016667in}{-0.004420in}}{\pgfqpoint{-0.014911in}{-0.008660in}}{\pgfqpoint{-0.011785in}{-0.011785in}}%
\pgfpathcurveto{\pgfqpoint{-0.008660in}{-0.014911in}}{\pgfqpoint{-0.004420in}{-0.016667in}}{\pgfqpoint{0.000000in}{-0.016667in}}%
\pgfpathlineto{\pgfqpoint{0.000000in}{-0.016667in}}%
\pgfpathclose%
\pgfusepath{stroke,fill}%
}%
\begin{pgfscope}%
\pgfsys@transformshift{3.575586in}{2.945282in}%
\pgfsys@useobject{currentmarker}{}%
\end{pgfscope}%
\end{pgfscope}%
\begin{pgfscope}%
\pgfpathrectangle{\pgfqpoint{0.765000in}{0.660000in}}{\pgfqpoint{4.620000in}{4.620000in}}%
\pgfusepath{clip}%
\pgfsetrectcap%
\pgfsetroundjoin%
\pgfsetlinewidth{1.204500pt}%
\definecolor{currentstroke}{rgb}{0.000000,0.000000,0.000000}%
\pgfsetstrokecolor{currentstroke}%
\pgfsetdash{}{0pt}%
\pgfpathmoveto{\pgfqpoint{3.575586in}{2.945282in}}%
\pgfpathlineto{\pgfqpoint{3.425935in}{3.326669in}}%
\pgfusepath{stroke}%
\end{pgfscope}%
\begin{pgfscope}%
\pgfpathrectangle{\pgfqpoint{0.765000in}{0.660000in}}{\pgfqpoint{4.620000in}{4.620000in}}%
\pgfusepath{clip}%
\pgfsetrectcap%
\pgfsetroundjoin%
\pgfsetlinewidth{1.204500pt}%
\definecolor{currentstroke}{rgb}{0.000000,0.000000,0.000000}%
\pgfsetstrokecolor{currentstroke}%
\pgfsetdash{}{0pt}%
\pgfpathmoveto{\pgfqpoint{3.575586in}{2.945282in}}%
\pgfpathlineto{\pgfqpoint{3.572472in}{2.854798in}}%
\pgfusepath{stroke}%
\end{pgfscope}%
\begin{pgfscope}%
\pgfpathrectangle{\pgfqpoint{0.765000in}{0.660000in}}{\pgfqpoint{4.620000in}{4.620000in}}%
\pgfusepath{clip}%
\pgfsetbuttcap%
\pgfsetroundjoin%
\pgfsetlinewidth{1.204500pt}%
\definecolor{currentstroke}{rgb}{0.827451,0.827451,0.827451}%
\pgfsetstrokecolor{currentstroke}%
\pgfsetdash{{1.200000pt}{1.980000pt}}{0.000000pt}%
\pgfpathmoveto{\pgfqpoint{3.575586in}{2.945282in}}%
\pgfpathlineto{\pgfqpoint{5.395000in}{2.976511in}}%
\pgfusepath{stroke}%
\end{pgfscope}%
\begin{pgfscope}%
\pgfpathrectangle{\pgfqpoint{0.765000in}{0.660000in}}{\pgfqpoint{4.620000in}{4.620000in}}%
\pgfusepath{clip}%
\pgfsetrectcap%
\pgfsetroundjoin%
\pgfsetlinewidth{1.204500pt}%
\definecolor{currentstroke}{rgb}{0.000000,0.000000,0.000000}%
\pgfsetstrokecolor{currentstroke}%
\pgfsetdash{}{0pt}%
\pgfpathmoveto{\pgfqpoint{2.586744in}{3.073269in}}%
\pgfusepath{stroke}%
\end{pgfscope}%
\begin{pgfscope}%
\pgfpathrectangle{\pgfqpoint{0.765000in}{0.660000in}}{\pgfqpoint{4.620000in}{4.620000in}}%
\pgfusepath{clip}%
\pgfsetbuttcap%
\pgfsetroundjoin%
\definecolor{currentfill}{rgb}{0.000000,0.000000,0.000000}%
\pgfsetfillcolor{currentfill}%
\pgfsetlinewidth{1.003750pt}%
\definecolor{currentstroke}{rgb}{0.000000,0.000000,0.000000}%
\pgfsetstrokecolor{currentstroke}%
\pgfsetdash{}{0pt}%
\pgfsys@defobject{currentmarker}{\pgfqpoint{-0.016667in}{-0.016667in}}{\pgfqpoint{0.016667in}{0.016667in}}{%
\pgfpathmoveto{\pgfqpoint{0.000000in}{-0.016667in}}%
\pgfpathcurveto{\pgfqpoint{0.004420in}{-0.016667in}}{\pgfqpoint{0.008660in}{-0.014911in}}{\pgfqpoint{0.011785in}{-0.011785in}}%
\pgfpathcurveto{\pgfqpoint{0.014911in}{-0.008660in}}{\pgfqpoint{0.016667in}{-0.004420in}}{\pgfqpoint{0.016667in}{0.000000in}}%
\pgfpathcurveto{\pgfqpoint{0.016667in}{0.004420in}}{\pgfqpoint{0.014911in}{0.008660in}}{\pgfqpoint{0.011785in}{0.011785in}}%
\pgfpathcurveto{\pgfqpoint{0.008660in}{0.014911in}}{\pgfqpoint{0.004420in}{0.016667in}}{\pgfqpoint{0.000000in}{0.016667in}}%
\pgfpathcurveto{\pgfqpoint{-0.004420in}{0.016667in}}{\pgfqpoint{-0.008660in}{0.014911in}}{\pgfqpoint{-0.011785in}{0.011785in}}%
\pgfpathcurveto{\pgfqpoint{-0.014911in}{0.008660in}}{\pgfqpoint{-0.016667in}{0.004420in}}{\pgfqpoint{-0.016667in}{0.000000in}}%
\pgfpathcurveto{\pgfqpoint{-0.016667in}{-0.004420in}}{\pgfqpoint{-0.014911in}{-0.008660in}}{\pgfqpoint{-0.011785in}{-0.011785in}}%
\pgfpathcurveto{\pgfqpoint{-0.008660in}{-0.014911in}}{\pgfqpoint{-0.004420in}{-0.016667in}}{\pgfqpoint{0.000000in}{-0.016667in}}%
\pgfpathlineto{\pgfqpoint{0.000000in}{-0.016667in}}%
\pgfpathclose%
\pgfusepath{stroke,fill}%
}%
\begin{pgfscope}%
\pgfsys@transformshift{2.586744in}{3.073269in}%
\pgfsys@useobject{currentmarker}{}%
\end{pgfscope}%
\end{pgfscope}%
\begin{pgfscope}%
\pgfpathrectangle{\pgfqpoint{0.765000in}{0.660000in}}{\pgfqpoint{4.620000in}{4.620000in}}%
\pgfusepath{clip}%
\pgfsetrectcap%
\pgfsetroundjoin%
\pgfsetlinewidth{1.204500pt}%
\definecolor{currentstroke}{rgb}{0.000000,0.000000,0.000000}%
\pgfsetstrokecolor{currentstroke}%
\pgfsetdash{}{0pt}%
\pgfpathmoveto{\pgfqpoint{2.586744in}{3.073269in}}%
\pgfpathlineto{\pgfqpoint{2.604377in}{3.013317in}}%
\pgfusepath{stroke}%
\end{pgfscope}%
\begin{pgfscope}%
\pgfpathrectangle{\pgfqpoint{0.765000in}{0.660000in}}{\pgfqpoint{4.620000in}{4.620000in}}%
\pgfusepath{clip}%
\pgfsetrectcap%
\pgfsetroundjoin%
\pgfsetlinewidth{1.204500pt}%
\definecolor{currentstroke}{rgb}{0.000000,0.000000,0.000000}%
\pgfsetstrokecolor{currentstroke}%
\pgfsetdash{}{0pt}%
\pgfpathmoveto{\pgfqpoint{2.586744in}{3.073269in}}%
\pgfpathlineto{\pgfqpoint{2.583171in}{3.110796in}}%
\pgfusepath{stroke}%
\end{pgfscope}%
\begin{pgfscope}%
\pgfpathrectangle{\pgfqpoint{0.765000in}{0.660000in}}{\pgfqpoint{4.620000in}{4.620000in}}%
\pgfusepath{clip}%
\pgfsetbuttcap%
\pgfsetroundjoin%
\pgfsetlinewidth{1.204500pt}%
\definecolor{currentstroke}{rgb}{0.827451,0.827451,0.827451}%
\pgfsetstrokecolor{currentstroke}%
\pgfsetdash{{1.200000pt}{1.980000pt}}{0.000000pt}%
\pgfpathmoveto{\pgfqpoint{2.586744in}{3.073269in}}%
\pgfpathlineto{\pgfqpoint{0.755000in}{2.422421in}}%
\pgfusepath{stroke}%
\end{pgfscope}%
\begin{pgfscope}%
\pgfpathrectangle{\pgfqpoint{0.765000in}{0.660000in}}{\pgfqpoint{4.620000in}{4.620000in}}%
\pgfusepath{clip}%
\pgfsetrectcap%
\pgfsetroundjoin%
\pgfsetlinewidth{1.204500pt}%
\definecolor{currentstroke}{rgb}{0.000000,0.000000,0.000000}%
\pgfsetstrokecolor{currentstroke}%
\pgfsetdash{}{0pt}%
\pgfpathmoveto{\pgfqpoint{2.635438in}{3.285868in}}%
\pgfusepath{stroke}%
\end{pgfscope}%
\begin{pgfscope}%
\pgfpathrectangle{\pgfqpoint{0.765000in}{0.660000in}}{\pgfqpoint{4.620000in}{4.620000in}}%
\pgfusepath{clip}%
\pgfsetbuttcap%
\pgfsetroundjoin%
\definecolor{currentfill}{rgb}{0.000000,0.000000,0.000000}%
\pgfsetfillcolor{currentfill}%
\pgfsetlinewidth{1.003750pt}%
\definecolor{currentstroke}{rgb}{0.000000,0.000000,0.000000}%
\pgfsetstrokecolor{currentstroke}%
\pgfsetdash{}{0pt}%
\pgfsys@defobject{currentmarker}{\pgfqpoint{-0.016667in}{-0.016667in}}{\pgfqpoint{0.016667in}{0.016667in}}{%
\pgfpathmoveto{\pgfqpoint{0.000000in}{-0.016667in}}%
\pgfpathcurveto{\pgfqpoint{0.004420in}{-0.016667in}}{\pgfqpoint{0.008660in}{-0.014911in}}{\pgfqpoint{0.011785in}{-0.011785in}}%
\pgfpathcurveto{\pgfqpoint{0.014911in}{-0.008660in}}{\pgfqpoint{0.016667in}{-0.004420in}}{\pgfqpoint{0.016667in}{0.000000in}}%
\pgfpathcurveto{\pgfqpoint{0.016667in}{0.004420in}}{\pgfqpoint{0.014911in}{0.008660in}}{\pgfqpoint{0.011785in}{0.011785in}}%
\pgfpathcurveto{\pgfqpoint{0.008660in}{0.014911in}}{\pgfqpoint{0.004420in}{0.016667in}}{\pgfqpoint{0.000000in}{0.016667in}}%
\pgfpathcurveto{\pgfqpoint{-0.004420in}{0.016667in}}{\pgfqpoint{-0.008660in}{0.014911in}}{\pgfqpoint{-0.011785in}{0.011785in}}%
\pgfpathcurveto{\pgfqpoint{-0.014911in}{0.008660in}}{\pgfqpoint{-0.016667in}{0.004420in}}{\pgfqpoint{-0.016667in}{0.000000in}}%
\pgfpathcurveto{\pgfqpoint{-0.016667in}{-0.004420in}}{\pgfqpoint{-0.014911in}{-0.008660in}}{\pgfqpoint{-0.011785in}{-0.011785in}}%
\pgfpathcurveto{\pgfqpoint{-0.008660in}{-0.014911in}}{\pgfqpoint{-0.004420in}{-0.016667in}}{\pgfqpoint{0.000000in}{-0.016667in}}%
\pgfpathlineto{\pgfqpoint{0.000000in}{-0.016667in}}%
\pgfpathclose%
\pgfusepath{stroke,fill}%
}%
\begin{pgfscope}%
\pgfsys@transformshift{2.635438in}{3.285868in}%
\pgfsys@useobject{currentmarker}{}%
\end{pgfscope}%
\end{pgfscope}%
\begin{pgfscope}%
\pgfpathrectangle{\pgfqpoint{0.765000in}{0.660000in}}{\pgfqpoint{4.620000in}{4.620000in}}%
\pgfusepath{clip}%
\pgfsetrectcap%
\pgfsetroundjoin%
\pgfsetlinewidth{1.204500pt}%
\definecolor{currentstroke}{rgb}{0.000000,0.000000,0.000000}%
\pgfsetstrokecolor{currentstroke}%
\pgfsetdash{}{0pt}%
\pgfpathmoveto{\pgfqpoint{2.635438in}{3.285868in}}%
\pgfpathlineto{\pgfqpoint{2.613910in}{3.253240in}}%
\pgfusepath{stroke}%
\end{pgfscope}%
\begin{pgfscope}%
\pgfpathrectangle{\pgfqpoint{0.765000in}{0.660000in}}{\pgfqpoint{4.620000in}{4.620000in}}%
\pgfusepath{clip}%
\pgfsetrectcap%
\pgfsetroundjoin%
\pgfsetlinewidth{1.204500pt}%
\definecolor{currentstroke}{rgb}{0.000000,0.000000,0.000000}%
\pgfsetstrokecolor{currentstroke}%
\pgfsetdash{}{0pt}%
\pgfpathmoveto{\pgfqpoint{2.635438in}{3.285868in}}%
\pgfpathlineto{\pgfqpoint{2.642560in}{3.294969in}}%
\pgfusepath{stroke}%
\end{pgfscope}%
\begin{pgfscope}%
\pgfpathrectangle{\pgfqpoint{0.765000in}{0.660000in}}{\pgfqpoint{4.620000in}{4.620000in}}%
\pgfusepath{clip}%
\pgfsetbuttcap%
\pgfsetroundjoin%
\pgfsetlinewidth{1.204500pt}%
\definecolor{currentstroke}{rgb}{0.827451,0.827451,0.827451}%
\pgfsetstrokecolor{currentstroke}%
\pgfsetdash{{1.200000pt}{1.980000pt}}{0.000000pt}%
\pgfpathmoveto{\pgfqpoint{2.635438in}{3.285868in}}%
\pgfpathlineto{\pgfqpoint{1.098702in}{3.886606in}}%
\pgfusepath{stroke}%
\end{pgfscope}%
\begin{pgfscope}%
\pgfpathrectangle{\pgfqpoint{0.765000in}{0.660000in}}{\pgfqpoint{4.620000in}{4.620000in}}%
\pgfusepath{clip}%
\pgfsetrectcap%
\pgfsetroundjoin%
\pgfsetlinewidth{1.204500pt}%
\definecolor{currentstroke}{rgb}{0.000000,0.000000,0.000000}%
\pgfsetstrokecolor{currentstroke}%
\pgfsetdash{}{0pt}%
\pgfpathmoveto{\pgfqpoint{2.613910in}{3.253240in}}%
\pgfusepath{stroke}%
\end{pgfscope}%
\begin{pgfscope}%
\pgfpathrectangle{\pgfqpoint{0.765000in}{0.660000in}}{\pgfqpoint{4.620000in}{4.620000in}}%
\pgfusepath{clip}%
\pgfsetbuttcap%
\pgfsetroundjoin%
\definecolor{currentfill}{rgb}{0.000000,0.000000,0.000000}%
\pgfsetfillcolor{currentfill}%
\pgfsetlinewidth{1.003750pt}%
\definecolor{currentstroke}{rgb}{0.000000,0.000000,0.000000}%
\pgfsetstrokecolor{currentstroke}%
\pgfsetdash{}{0pt}%
\pgfsys@defobject{currentmarker}{\pgfqpoint{-0.016667in}{-0.016667in}}{\pgfqpoint{0.016667in}{0.016667in}}{%
\pgfpathmoveto{\pgfqpoint{0.000000in}{-0.016667in}}%
\pgfpathcurveto{\pgfqpoint{0.004420in}{-0.016667in}}{\pgfqpoint{0.008660in}{-0.014911in}}{\pgfqpoint{0.011785in}{-0.011785in}}%
\pgfpathcurveto{\pgfqpoint{0.014911in}{-0.008660in}}{\pgfqpoint{0.016667in}{-0.004420in}}{\pgfqpoint{0.016667in}{0.000000in}}%
\pgfpathcurveto{\pgfqpoint{0.016667in}{0.004420in}}{\pgfqpoint{0.014911in}{0.008660in}}{\pgfqpoint{0.011785in}{0.011785in}}%
\pgfpathcurveto{\pgfqpoint{0.008660in}{0.014911in}}{\pgfqpoint{0.004420in}{0.016667in}}{\pgfqpoint{0.000000in}{0.016667in}}%
\pgfpathcurveto{\pgfqpoint{-0.004420in}{0.016667in}}{\pgfqpoint{-0.008660in}{0.014911in}}{\pgfqpoint{-0.011785in}{0.011785in}}%
\pgfpathcurveto{\pgfqpoint{-0.014911in}{0.008660in}}{\pgfqpoint{-0.016667in}{0.004420in}}{\pgfqpoint{-0.016667in}{0.000000in}}%
\pgfpathcurveto{\pgfqpoint{-0.016667in}{-0.004420in}}{\pgfqpoint{-0.014911in}{-0.008660in}}{\pgfqpoint{-0.011785in}{-0.011785in}}%
\pgfpathcurveto{\pgfqpoint{-0.008660in}{-0.014911in}}{\pgfqpoint{-0.004420in}{-0.016667in}}{\pgfqpoint{0.000000in}{-0.016667in}}%
\pgfpathlineto{\pgfqpoint{0.000000in}{-0.016667in}}%
\pgfpathclose%
\pgfusepath{stroke,fill}%
}%
\begin{pgfscope}%
\pgfsys@transformshift{2.613910in}{3.253240in}%
\pgfsys@useobject{currentmarker}{}%
\end{pgfscope}%
\end{pgfscope}%
\begin{pgfscope}%
\pgfpathrectangle{\pgfqpoint{0.765000in}{0.660000in}}{\pgfqpoint{4.620000in}{4.620000in}}%
\pgfusepath{clip}%
\pgfsetrectcap%
\pgfsetroundjoin%
\pgfsetlinewidth{1.204500pt}%
\definecolor{currentstroke}{rgb}{0.000000,0.000000,0.000000}%
\pgfsetstrokecolor{currentstroke}%
\pgfsetdash{}{0pt}%
\pgfpathmoveto{\pgfqpoint{2.613910in}{3.253240in}}%
\pgfpathlineto{\pgfqpoint{2.635438in}{3.285868in}}%
\pgfusepath{stroke}%
\end{pgfscope}%
\begin{pgfscope}%
\pgfpathrectangle{\pgfqpoint{0.765000in}{0.660000in}}{\pgfqpoint{4.620000in}{4.620000in}}%
\pgfusepath{clip}%
\pgfsetrectcap%
\pgfsetroundjoin%
\pgfsetlinewidth{1.204500pt}%
\definecolor{currentstroke}{rgb}{0.000000,0.000000,0.000000}%
\pgfsetstrokecolor{currentstroke}%
\pgfsetdash{}{0pt}%
\pgfpathmoveto{\pgfqpoint{2.613910in}{3.253240in}}%
\pgfpathlineto{\pgfqpoint{2.597727in}{3.213477in}}%
\pgfusepath{stroke}%
\end{pgfscope}%
\begin{pgfscope}%
\pgfpathrectangle{\pgfqpoint{0.765000in}{0.660000in}}{\pgfqpoint{4.620000in}{4.620000in}}%
\pgfusepath{clip}%
\pgfsetbuttcap%
\pgfsetroundjoin%
\pgfsetlinewidth{1.204500pt}%
\definecolor{currentstroke}{rgb}{0.827451,0.827451,0.827451}%
\pgfsetstrokecolor{currentstroke}%
\pgfsetdash{{1.200000pt}{1.980000pt}}{0.000000pt}%
\pgfpathmoveto{\pgfqpoint{2.613910in}{3.253240in}}%
\pgfpathlineto{\pgfqpoint{0.755000in}{3.710545in}}%
\pgfusepath{stroke}%
\end{pgfscope}%
\begin{pgfscope}%
\pgfpathrectangle{\pgfqpoint{0.765000in}{0.660000in}}{\pgfqpoint{4.620000in}{4.620000in}}%
\pgfusepath{clip}%
\pgfsetrectcap%
\pgfsetroundjoin%
\pgfsetlinewidth{1.204500pt}%
\definecolor{currentstroke}{rgb}{0.000000,0.000000,0.000000}%
\pgfsetstrokecolor{currentstroke}%
\pgfsetdash{}{0pt}%
\pgfpathmoveto{\pgfqpoint{3.477043in}{2.674663in}}%
\pgfusepath{stroke}%
\end{pgfscope}%
\begin{pgfscope}%
\pgfpathrectangle{\pgfqpoint{0.765000in}{0.660000in}}{\pgfqpoint{4.620000in}{4.620000in}}%
\pgfusepath{clip}%
\pgfsetbuttcap%
\pgfsetroundjoin%
\definecolor{currentfill}{rgb}{0.000000,0.000000,0.000000}%
\pgfsetfillcolor{currentfill}%
\pgfsetlinewidth{1.003750pt}%
\definecolor{currentstroke}{rgb}{0.000000,0.000000,0.000000}%
\pgfsetstrokecolor{currentstroke}%
\pgfsetdash{}{0pt}%
\pgfsys@defobject{currentmarker}{\pgfqpoint{-0.016667in}{-0.016667in}}{\pgfqpoint{0.016667in}{0.016667in}}{%
\pgfpathmoveto{\pgfqpoint{0.000000in}{-0.016667in}}%
\pgfpathcurveto{\pgfqpoint{0.004420in}{-0.016667in}}{\pgfqpoint{0.008660in}{-0.014911in}}{\pgfqpoint{0.011785in}{-0.011785in}}%
\pgfpathcurveto{\pgfqpoint{0.014911in}{-0.008660in}}{\pgfqpoint{0.016667in}{-0.004420in}}{\pgfqpoint{0.016667in}{0.000000in}}%
\pgfpathcurveto{\pgfqpoint{0.016667in}{0.004420in}}{\pgfqpoint{0.014911in}{0.008660in}}{\pgfqpoint{0.011785in}{0.011785in}}%
\pgfpathcurveto{\pgfqpoint{0.008660in}{0.014911in}}{\pgfqpoint{0.004420in}{0.016667in}}{\pgfqpoint{0.000000in}{0.016667in}}%
\pgfpathcurveto{\pgfqpoint{-0.004420in}{0.016667in}}{\pgfqpoint{-0.008660in}{0.014911in}}{\pgfqpoint{-0.011785in}{0.011785in}}%
\pgfpathcurveto{\pgfqpoint{-0.014911in}{0.008660in}}{\pgfqpoint{-0.016667in}{0.004420in}}{\pgfqpoint{-0.016667in}{0.000000in}}%
\pgfpathcurveto{\pgfqpoint{-0.016667in}{-0.004420in}}{\pgfqpoint{-0.014911in}{-0.008660in}}{\pgfqpoint{-0.011785in}{-0.011785in}}%
\pgfpathcurveto{\pgfqpoint{-0.008660in}{-0.014911in}}{\pgfqpoint{-0.004420in}{-0.016667in}}{\pgfqpoint{0.000000in}{-0.016667in}}%
\pgfpathlineto{\pgfqpoint{0.000000in}{-0.016667in}}%
\pgfpathclose%
\pgfusepath{stroke,fill}%
}%
\begin{pgfscope}%
\pgfsys@transformshift{3.477043in}{2.674663in}%
\pgfsys@useobject{currentmarker}{}%
\end{pgfscope}%
\end{pgfscope}%
\begin{pgfscope}%
\pgfpathrectangle{\pgfqpoint{0.765000in}{0.660000in}}{\pgfqpoint{4.620000in}{4.620000in}}%
\pgfusepath{clip}%
\pgfsetrectcap%
\pgfsetroundjoin%
\pgfsetlinewidth{1.204500pt}%
\definecolor{currentstroke}{rgb}{0.000000,0.000000,0.000000}%
\pgfsetstrokecolor{currentstroke}%
\pgfsetdash{}{0pt}%
\pgfpathmoveto{\pgfqpoint{3.477043in}{2.674663in}}%
\pgfpathlineto{\pgfqpoint{3.519640in}{2.712796in}}%
\pgfusepath{stroke}%
\end{pgfscope}%
\begin{pgfscope}%
\pgfpathrectangle{\pgfqpoint{0.765000in}{0.660000in}}{\pgfqpoint{4.620000in}{4.620000in}}%
\pgfusepath{clip}%
\pgfsetrectcap%
\pgfsetroundjoin%
\pgfsetlinewidth{1.204500pt}%
\definecolor{currentstroke}{rgb}{0.000000,0.000000,0.000000}%
\pgfsetstrokecolor{currentstroke}%
\pgfsetdash{}{0pt}%
\pgfpathmoveto{\pgfqpoint{3.477043in}{2.674663in}}%
\pgfpathlineto{\pgfqpoint{3.463186in}{2.662942in}}%
\pgfusepath{stroke}%
\end{pgfscope}%
\begin{pgfscope}%
\pgfpathrectangle{\pgfqpoint{0.765000in}{0.660000in}}{\pgfqpoint{4.620000in}{4.620000in}}%
\pgfusepath{clip}%
\pgfsetbuttcap%
\pgfsetroundjoin%
\pgfsetlinewidth{1.204500pt}%
\definecolor{currentstroke}{rgb}{0.827451,0.827451,0.827451}%
\pgfsetstrokecolor{currentstroke}%
\pgfsetdash{{1.200000pt}{1.980000pt}}{0.000000pt}%
\pgfpathmoveto{\pgfqpoint{3.477043in}{2.674663in}}%
\pgfpathlineto{\pgfqpoint{3.733722in}{2.163541in}}%
\pgfusepath{stroke}%
\end{pgfscope}%
\begin{pgfscope}%
\pgfpathrectangle{\pgfqpoint{0.765000in}{0.660000in}}{\pgfqpoint{4.620000in}{4.620000in}}%
\pgfusepath{clip}%
\pgfsetrectcap%
\pgfsetroundjoin%
\pgfsetlinewidth{1.204500pt}%
\definecolor{currentstroke}{rgb}{0.000000,0.000000,0.000000}%
\pgfsetstrokecolor{currentstroke}%
\pgfsetdash{}{0pt}%
\pgfpathmoveto{\pgfqpoint{2.793000in}{3.409870in}}%
\pgfusepath{stroke}%
\end{pgfscope}%
\begin{pgfscope}%
\pgfpathrectangle{\pgfqpoint{0.765000in}{0.660000in}}{\pgfqpoint{4.620000in}{4.620000in}}%
\pgfusepath{clip}%
\pgfsetbuttcap%
\pgfsetroundjoin%
\definecolor{currentfill}{rgb}{0.000000,0.000000,0.000000}%
\pgfsetfillcolor{currentfill}%
\pgfsetlinewidth{1.003750pt}%
\definecolor{currentstroke}{rgb}{0.000000,0.000000,0.000000}%
\pgfsetstrokecolor{currentstroke}%
\pgfsetdash{}{0pt}%
\pgfsys@defobject{currentmarker}{\pgfqpoint{-0.016667in}{-0.016667in}}{\pgfqpoint{0.016667in}{0.016667in}}{%
\pgfpathmoveto{\pgfqpoint{0.000000in}{-0.016667in}}%
\pgfpathcurveto{\pgfqpoint{0.004420in}{-0.016667in}}{\pgfqpoint{0.008660in}{-0.014911in}}{\pgfqpoint{0.011785in}{-0.011785in}}%
\pgfpathcurveto{\pgfqpoint{0.014911in}{-0.008660in}}{\pgfqpoint{0.016667in}{-0.004420in}}{\pgfqpoint{0.016667in}{0.000000in}}%
\pgfpathcurveto{\pgfqpoint{0.016667in}{0.004420in}}{\pgfqpoint{0.014911in}{0.008660in}}{\pgfqpoint{0.011785in}{0.011785in}}%
\pgfpathcurveto{\pgfqpoint{0.008660in}{0.014911in}}{\pgfqpoint{0.004420in}{0.016667in}}{\pgfqpoint{0.000000in}{0.016667in}}%
\pgfpathcurveto{\pgfqpoint{-0.004420in}{0.016667in}}{\pgfqpoint{-0.008660in}{0.014911in}}{\pgfqpoint{-0.011785in}{0.011785in}}%
\pgfpathcurveto{\pgfqpoint{-0.014911in}{0.008660in}}{\pgfqpoint{-0.016667in}{0.004420in}}{\pgfqpoint{-0.016667in}{0.000000in}}%
\pgfpathcurveto{\pgfqpoint{-0.016667in}{-0.004420in}}{\pgfqpoint{-0.014911in}{-0.008660in}}{\pgfqpoint{-0.011785in}{-0.011785in}}%
\pgfpathcurveto{\pgfqpoint{-0.008660in}{-0.014911in}}{\pgfqpoint{-0.004420in}{-0.016667in}}{\pgfqpoint{0.000000in}{-0.016667in}}%
\pgfpathlineto{\pgfqpoint{0.000000in}{-0.016667in}}%
\pgfpathclose%
\pgfusepath{stroke,fill}%
}%
\begin{pgfscope}%
\pgfsys@transformshift{2.793000in}{3.409870in}%
\pgfsys@useobject{currentmarker}{}%
\end{pgfscope}%
\end{pgfscope}%
\begin{pgfscope}%
\pgfpathrectangle{\pgfqpoint{0.765000in}{0.660000in}}{\pgfqpoint{4.620000in}{4.620000in}}%
\pgfusepath{clip}%
\pgfsetrectcap%
\pgfsetroundjoin%
\pgfsetlinewidth{1.204500pt}%
\definecolor{currentstroke}{rgb}{0.000000,0.000000,0.000000}%
\pgfsetstrokecolor{currentstroke}%
\pgfsetdash{}{0pt}%
\pgfpathmoveto{\pgfqpoint{2.793000in}{3.409870in}}%
\pgfpathlineto{\pgfqpoint{2.788056in}{3.406767in}}%
\pgfusepath{stroke}%
\end{pgfscope}%
\begin{pgfscope}%
\pgfpathrectangle{\pgfqpoint{0.765000in}{0.660000in}}{\pgfqpoint{4.620000in}{4.620000in}}%
\pgfusepath{clip}%
\pgfsetrectcap%
\pgfsetroundjoin%
\pgfsetlinewidth{1.204500pt}%
\definecolor{currentstroke}{rgb}{0.000000,0.000000,0.000000}%
\pgfsetstrokecolor{currentstroke}%
\pgfsetdash{}{0pt}%
\pgfpathmoveto{\pgfqpoint{2.793000in}{3.409870in}}%
\pgfpathlineto{\pgfqpoint{2.913379in}{3.479681in}}%
\pgfusepath{stroke}%
\end{pgfscope}%
\begin{pgfscope}%
\pgfpathrectangle{\pgfqpoint{0.765000in}{0.660000in}}{\pgfqpoint{4.620000in}{4.620000in}}%
\pgfusepath{clip}%
\pgfsetbuttcap%
\pgfsetroundjoin%
\pgfsetlinewidth{1.204500pt}%
\definecolor{currentstroke}{rgb}{0.827451,0.827451,0.827451}%
\pgfsetstrokecolor{currentstroke}%
\pgfsetdash{{1.200000pt}{1.980000pt}}{0.000000pt}%
\pgfpathmoveto{\pgfqpoint{2.793000in}{3.409870in}}%
\pgfpathlineto{\pgfqpoint{2.236628in}{3.967368in}}%
\pgfusepath{stroke}%
\end{pgfscope}%
\begin{pgfscope}%
\pgfpathrectangle{\pgfqpoint{0.765000in}{0.660000in}}{\pgfqpoint{4.620000in}{4.620000in}}%
\pgfusepath{clip}%
\pgfsetrectcap%
\pgfsetroundjoin%
\pgfsetlinewidth{1.204500pt}%
\definecolor{currentstroke}{rgb}{0.000000,0.000000,0.000000}%
\pgfsetstrokecolor{currentstroke}%
\pgfsetdash{}{0pt}%
\pgfpathmoveto{\pgfqpoint{2.788056in}{3.406767in}}%
\pgfusepath{stroke}%
\end{pgfscope}%
\begin{pgfscope}%
\pgfpathrectangle{\pgfqpoint{0.765000in}{0.660000in}}{\pgfqpoint{4.620000in}{4.620000in}}%
\pgfusepath{clip}%
\pgfsetbuttcap%
\pgfsetroundjoin%
\definecolor{currentfill}{rgb}{0.000000,0.000000,0.000000}%
\pgfsetfillcolor{currentfill}%
\pgfsetlinewidth{1.003750pt}%
\definecolor{currentstroke}{rgb}{0.000000,0.000000,0.000000}%
\pgfsetstrokecolor{currentstroke}%
\pgfsetdash{}{0pt}%
\pgfsys@defobject{currentmarker}{\pgfqpoint{-0.016667in}{-0.016667in}}{\pgfqpoint{0.016667in}{0.016667in}}{%
\pgfpathmoveto{\pgfqpoint{0.000000in}{-0.016667in}}%
\pgfpathcurveto{\pgfqpoint{0.004420in}{-0.016667in}}{\pgfqpoint{0.008660in}{-0.014911in}}{\pgfqpoint{0.011785in}{-0.011785in}}%
\pgfpathcurveto{\pgfqpoint{0.014911in}{-0.008660in}}{\pgfqpoint{0.016667in}{-0.004420in}}{\pgfqpoint{0.016667in}{0.000000in}}%
\pgfpathcurveto{\pgfqpoint{0.016667in}{0.004420in}}{\pgfqpoint{0.014911in}{0.008660in}}{\pgfqpoint{0.011785in}{0.011785in}}%
\pgfpathcurveto{\pgfqpoint{0.008660in}{0.014911in}}{\pgfqpoint{0.004420in}{0.016667in}}{\pgfqpoint{0.000000in}{0.016667in}}%
\pgfpathcurveto{\pgfqpoint{-0.004420in}{0.016667in}}{\pgfqpoint{-0.008660in}{0.014911in}}{\pgfqpoint{-0.011785in}{0.011785in}}%
\pgfpathcurveto{\pgfqpoint{-0.014911in}{0.008660in}}{\pgfqpoint{-0.016667in}{0.004420in}}{\pgfqpoint{-0.016667in}{0.000000in}}%
\pgfpathcurveto{\pgfqpoint{-0.016667in}{-0.004420in}}{\pgfqpoint{-0.014911in}{-0.008660in}}{\pgfqpoint{-0.011785in}{-0.011785in}}%
\pgfpathcurveto{\pgfqpoint{-0.008660in}{-0.014911in}}{\pgfqpoint{-0.004420in}{-0.016667in}}{\pgfqpoint{0.000000in}{-0.016667in}}%
\pgfpathlineto{\pgfqpoint{0.000000in}{-0.016667in}}%
\pgfpathclose%
\pgfusepath{stroke,fill}%
}%
\begin{pgfscope}%
\pgfsys@transformshift{2.788056in}{3.406767in}%
\pgfsys@useobject{currentmarker}{}%
\end{pgfscope}%
\end{pgfscope}%
\begin{pgfscope}%
\pgfpathrectangle{\pgfqpoint{0.765000in}{0.660000in}}{\pgfqpoint{4.620000in}{4.620000in}}%
\pgfusepath{clip}%
\pgfsetrectcap%
\pgfsetroundjoin%
\pgfsetlinewidth{1.204500pt}%
\definecolor{currentstroke}{rgb}{0.000000,0.000000,0.000000}%
\pgfsetstrokecolor{currentstroke}%
\pgfsetdash{}{0pt}%
\pgfpathmoveto{\pgfqpoint{2.788056in}{3.406767in}}%
\pgfpathlineto{\pgfqpoint{2.708745in}{3.356240in}}%
\pgfusepath{stroke}%
\end{pgfscope}%
\begin{pgfscope}%
\pgfpathrectangle{\pgfqpoint{0.765000in}{0.660000in}}{\pgfqpoint{4.620000in}{4.620000in}}%
\pgfusepath{clip}%
\pgfsetrectcap%
\pgfsetroundjoin%
\pgfsetlinewidth{1.204500pt}%
\definecolor{currentstroke}{rgb}{0.000000,0.000000,0.000000}%
\pgfsetstrokecolor{currentstroke}%
\pgfsetdash{}{0pt}%
\pgfpathmoveto{\pgfqpoint{2.788056in}{3.406767in}}%
\pgfpathlineto{\pgfqpoint{2.793000in}{3.409870in}}%
\pgfusepath{stroke}%
\end{pgfscope}%
\begin{pgfscope}%
\pgfpathrectangle{\pgfqpoint{0.765000in}{0.660000in}}{\pgfqpoint{4.620000in}{4.620000in}}%
\pgfusepath{clip}%
\pgfsetbuttcap%
\pgfsetroundjoin%
\pgfsetlinewidth{1.204500pt}%
\definecolor{currentstroke}{rgb}{0.827451,0.827451,0.827451}%
\pgfsetstrokecolor{currentstroke}%
\pgfsetdash{{1.200000pt}{1.980000pt}}{0.000000pt}%
\pgfpathmoveto{\pgfqpoint{2.788056in}{3.406767in}}%
\pgfpathlineto{\pgfqpoint{2.676791in}{3.513370in}}%
\pgfusepath{stroke}%
\end{pgfscope}%
\begin{pgfscope}%
\pgfpathrectangle{\pgfqpoint{0.765000in}{0.660000in}}{\pgfqpoint{4.620000in}{4.620000in}}%
\pgfusepath{clip}%
\pgfsetrectcap%
\pgfsetroundjoin%
\pgfsetlinewidth{1.204500pt}%
\definecolor{currentstroke}{rgb}{0.000000,0.000000,0.000000}%
\pgfsetstrokecolor{currentstroke}%
\pgfsetdash{}{0pt}%
\pgfpathmoveto{\pgfqpoint{3.563846in}{2.808203in}}%
\pgfusepath{stroke}%
\end{pgfscope}%
\begin{pgfscope}%
\pgfpathrectangle{\pgfqpoint{0.765000in}{0.660000in}}{\pgfqpoint{4.620000in}{4.620000in}}%
\pgfusepath{clip}%
\pgfsetbuttcap%
\pgfsetroundjoin%
\definecolor{currentfill}{rgb}{0.000000,0.000000,0.000000}%
\pgfsetfillcolor{currentfill}%
\pgfsetlinewidth{1.003750pt}%
\definecolor{currentstroke}{rgb}{0.000000,0.000000,0.000000}%
\pgfsetstrokecolor{currentstroke}%
\pgfsetdash{}{0pt}%
\pgfsys@defobject{currentmarker}{\pgfqpoint{-0.016667in}{-0.016667in}}{\pgfqpoint{0.016667in}{0.016667in}}{%
\pgfpathmoveto{\pgfqpoint{0.000000in}{-0.016667in}}%
\pgfpathcurveto{\pgfqpoint{0.004420in}{-0.016667in}}{\pgfqpoint{0.008660in}{-0.014911in}}{\pgfqpoint{0.011785in}{-0.011785in}}%
\pgfpathcurveto{\pgfqpoint{0.014911in}{-0.008660in}}{\pgfqpoint{0.016667in}{-0.004420in}}{\pgfqpoint{0.016667in}{0.000000in}}%
\pgfpathcurveto{\pgfqpoint{0.016667in}{0.004420in}}{\pgfqpoint{0.014911in}{0.008660in}}{\pgfqpoint{0.011785in}{0.011785in}}%
\pgfpathcurveto{\pgfqpoint{0.008660in}{0.014911in}}{\pgfqpoint{0.004420in}{0.016667in}}{\pgfqpoint{0.000000in}{0.016667in}}%
\pgfpathcurveto{\pgfqpoint{-0.004420in}{0.016667in}}{\pgfqpoint{-0.008660in}{0.014911in}}{\pgfqpoint{-0.011785in}{0.011785in}}%
\pgfpathcurveto{\pgfqpoint{-0.014911in}{0.008660in}}{\pgfqpoint{-0.016667in}{0.004420in}}{\pgfqpoint{-0.016667in}{0.000000in}}%
\pgfpathcurveto{\pgfqpoint{-0.016667in}{-0.004420in}}{\pgfqpoint{-0.014911in}{-0.008660in}}{\pgfqpoint{-0.011785in}{-0.011785in}}%
\pgfpathcurveto{\pgfqpoint{-0.008660in}{-0.014911in}}{\pgfqpoint{-0.004420in}{-0.016667in}}{\pgfqpoint{0.000000in}{-0.016667in}}%
\pgfpathlineto{\pgfqpoint{0.000000in}{-0.016667in}}%
\pgfpathclose%
\pgfusepath{stroke,fill}%
}%
\begin{pgfscope}%
\pgfsys@transformshift{3.563846in}{2.808203in}%
\pgfsys@useobject{currentmarker}{}%
\end{pgfscope}%
\end{pgfscope}%
\begin{pgfscope}%
\pgfpathrectangle{\pgfqpoint{0.765000in}{0.660000in}}{\pgfqpoint{4.620000in}{4.620000in}}%
\pgfusepath{clip}%
\pgfsetrectcap%
\pgfsetroundjoin%
\pgfsetlinewidth{1.204500pt}%
\definecolor{currentstroke}{rgb}{0.000000,0.000000,0.000000}%
\pgfsetstrokecolor{currentstroke}%
\pgfsetdash{}{0pt}%
\pgfpathmoveto{\pgfqpoint{3.563846in}{2.808203in}}%
\pgfpathlineto{\pgfqpoint{3.551407in}{2.776901in}}%
\pgfusepath{stroke}%
\end{pgfscope}%
\begin{pgfscope}%
\pgfpathrectangle{\pgfqpoint{0.765000in}{0.660000in}}{\pgfqpoint{4.620000in}{4.620000in}}%
\pgfusepath{clip}%
\pgfsetrectcap%
\pgfsetroundjoin%
\pgfsetlinewidth{1.204500pt}%
\definecolor{currentstroke}{rgb}{0.000000,0.000000,0.000000}%
\pgfsetstrokecolor{currentstroke}%
\pgfsetdash{}{0pt}%
\pgfpathmoveto{\pgfqpoint{3.563846in}{2.808203in}}%
\pgfpathlineto{\pgfqpoint{3.572472in}{2.854798in}}%
\pgfusepath{stroke}%
\end{pgfscope}%
\begin{pgfscope}%
\pgfpathrectangle{\pgfqpoint{0.765000in}{0.660000in}}{\pgfqpoint{4.620000in}{4.620000in}}%
\pgfusepath{clip}%
\pgfsetbuttcap%
\pgfsetroundjoin%
\pgfsetlinewidth{1.204500pt}%
\definecolor{currentstroke}{rgb}{0.827451,0.827451,0.827451}%
\pgfsetstrokecolor{currentstroke}%
\pgfsetdash{{1.200000pt}{1.980000pt}}{0.000000pt}%
\pgfpathmoveto{\pgfqpoint{3.563846in}{2.808203in}}%
\pgfpathlineto{\pgfqpoint{5.395000in}{1.792354in}}%
\pgfusepath{stroke}%
\end{pgfscope}%
\begin{pgfscope}%
\pgfpathrectangle{\pgfqpoint{0.765000in}{0.660000in}}{\pgfqpoint{4.620000in}{4.620000in}}%
\pgfusepath{clip}%
\pgfsetrectcap%
\pgfsetroundjoin%
\pgfsetlinewidth{1.204500pt}%
\definecolor{currentstroke}{rgb}{0.000000,0.000000,0.000000}%
\pgfsetstrokecolor{currentstroke}%
\pgfsetdash{}{0pt}%
\pgfpathmoveto{\pgfqpoint{3.551407in}{2.776901in}}%
\pgfusepath{stroke}%
\end{pgfscope}%
\begin{pgfscope}%
\pgfpathrectangle{\pgfqpoint{0.765000in}{0.660000in}}{\pgfqpoint{4.620000in}{4.620000in}}%
\pgfusepath{clip}%
\pgfsetbuttcap%
\pgfsetroundjoin%
\definecolor{currentfill}{rgb}{0.000000,0.000000,0.000000}%
\pgfsetfillcolor{currentfill}%
\pgfsetlinewidth{1.003750pt}%
\definecolor{currentstroke}{rgb}{0.000000,0.000000,0.000000}%
\pgfsetstrokecolor{currentstroke}%
\pgfsetdash{}{0pt}%
\pgfsys@defobject{currentmarker}{\pgfqpoint{-0.016667in}{-0.016667in}}{\pgfqpoint{0.016667in}{0.016667in}}{%
\pgfpathmoveto{\pgfqpoint{0.000000in}{-0.016667in}}%
\pgfpathcurveto{\pgfqpoint{0.004420in}{-0.016667in}}{\pgfqpoint{0.008660in}{-0.014911in}}{\pgfqpoint{0.011785in}{-0.011785in}}%
\pgfpathcurveto{\pgfqpoint{0.014911in}{-0.008660in}}{\pgfqpoint{0.016667in}{-0.004420in}}{\pgfqpoint{0.016667in}{0.000000in}}%
\pgfpathcurveto{\pgfqpoint{0.016667in}{0.004420in}}{\pgfqpoint{0.014911in}{0.008660in}}{\pgfqpoint{0.011785in}{0.011785in}}%
\pgfpathcurveto{\pgfqpoint{0.008660in}{0.014911in}}{\pgfqpoint{0.004420in}{0.016667in}}{\pgfqpoint{0.000000in}{0.016667in}}%
\pgfpathcurveto{\pgfqpoint{-0.004420in}{0.016667in}}{\pgfqpoint{-0.008660in}{0.014911in}}{\pgfqpoint{-0.011785in}{0.011785in}}%
\pgfpathcurveto{\pgfqpoint{-0.014911in}{0.008660in}}{\pgfqpoint{-0.016667in}{0.004420in}}{\pgfqpoint{-0.016667in}{0.000000in}}%
\pgfpathcurveto{\pgfqpoint{-0.016667in}{-0.004420in}}{\pgfqpoint{-0.014911in}{-0.008660in}}{\pgfqpoint{-0.011785in}{-0.011785in}}%
\pgfpathcurveto{\pgfqpoint{-0.008660in}{-0.014911in}}{\pgfqpoint{-0.004420in}{-0.016667in}}{\pgfqpoint{0.000000in}{-0.016667in}}%
\pgfpathlineto{\pgfqpoint{0.000000in}{-0.016667in}}%
\pgfpathclose%
\pgfusepath{stroke,fill}%
}%
\begin{pgfscope}%
\pgfsys@transformshift{3.551407in}{2.776901in}%
\pgfsys@useobject{currentmarker}{}%
\end{pgfscope}%
\end{pgfscope}%
\begin{pgfscope}%
\pgfpathrectangle{\pgfqpoint{0.765000in}{0.660000in}}{\pgfqpoint{4.620000in}{4.620000in}}%
\pgfusepath{clip}%
\pgfsetrectcap%
\pgfsetroundjoin%
\pgfsetlinewidth{1.204500pt}%
\definecolor{currentstroke}{rgb}{0.000000,0.000000,0.000000}%
\pgfsetstrokecolor{currentstroke}%
\pgfsetdash{}{0pt}%
\pgfpathmoveto{\pgfqpoint{3.551407in}{2.776901in}}%
\pgfpathlineto{\pgfqpoint{3.546263in}{2.765531in}}%
\pgfusepath{stroke}%
\end{pgfscope}%
\begin{pgfscope}%
\pgfpathrectangle{\pgfqpoint{0.765000in}{0.660000in}}{\pgfqpoint{4.620000in}{4.620000in}}%
\pgfusepath{clip}%
\pgfsetrectcap%
\pgfsetroundjoin%
\pgfsetlinewidth{1.204500pt}%
\definecolor{currentstroke}{rgb}{0.000000,0.000000,0.000000}%
\pgfsetstrokecolor{currentstroke}%
\pgfsetdash{}{0pt}%
\pgfpathmoveto{\pgfqpoint{3.551407in}{2.776901in}}%
\pgfpathlineto{\pgfqpoint{3.563846in}{2.808203in}}%
\pgfusepath{stroke}%
\end{pgfscope}%
\begin{pgfscope}%
\pgfpathrectangle{\pgfqpoint{0.765000in}{0.660000in}}{\pgfqpoint{4.620000in}{4.620000in}}%
\pgfusepath{clip}%
\pgfsetbuttcap%
\pgfsetroundjoin%
\pgfsetlinewidth{1.204500pt}%
\definecolor{currentstroke}{rgb}{0.827451,0.827451,0.827451}%
\pgfsetstrokecolor{currentstroke}%
\pgfsetdash{{1.200000pt}{1.980000pt}}{0.000000pt}%
\pgfpathmoveto{\pgfqpoint{3.551407in}{2.776901in}}%
\pgfpathlineto{\pgfqpoint{4.404009in}{2.141740in}}%
\pgfusepath{stroke}%
\end{pgfscope}%
\begin{pgfscope}%
\pgfpathrectangle{\pgfqpoint{0.765000in}{0.660000in}}{\pgfqpoint{4.620000in}{4.620000in}}%
\pgfusepath{clip}%
\pgfsetrectcap%
\pgfsetroundjoin%
\pgfsetlinewidth{1.204500pt}%
\definecolor{currentstroke}{rgb}{0.000000,0.000000,0.000000}%
\pgfsetstrokecolor{currentstroke}%
\pgfsetdash{}{0pt}%
\pgfpathmoveto{\pgfqpoint{3.085349in}{3.530998in}}%
\pgfusepath{stroke}%
\end{pgfscope}%
\begin{pgfscope}%
\pgfpathrectangle{\pgfqpoint{0.765000in}{0.660000in}}{\pgfqpoint{4.620000in}{4.620000in}}%
\pgfusepath{clip}%
\pgfsetbuttcap%
\pgfsetroundjoin%
\definecolor{currentfill}{rgb}{0.000000,0.000000,0.000000}%
\pgfsetfillcolor{currentfill}%
\pgfsetlinewidth{1.003750pt}%
\definecolor{currentstroke}{rgb}{0.000000,0.000000,0.000000}%
\pgfsetstrokecolor{currentstroke}%
\pgfsetdash{}{0pt}%
\pgfsys@defobject{currentmarker}{\pgfqpoint{-0.016667in}{-0.016667in}}{\pgfqpoint{0.016667in}{0.016667in}}{%
\pgfpathmoveto{\pgfqpoint{0.000000in}{-0.016667in}}%
\pgfpathcurveto{\pgfqpoint{0.004420in}{-0.016667in}}{\pgfqpoint{0.008660in}{-0.014911in}}{\pgfqpoint{0.011785in}{-0.011785in}}%
\pgfpathcurveto{\pgfqpoint{0.014911in}{-0.008660in}}{\pgfqpoint{0.016667in}{-0.004420in}}{\pgfqpoint{0.016667in}{0.000000in}}%
\pgfpathcurveto{\pgfqpoint{0.016667in}{0.004420in}}{\pgfqpoint{0.014911in}{0.008660in}}{\pgfqpoint{0.011785in}{0.011785in}}%
\pgfpathcurveto{\pgfqpoint{0.008660in}{0.014911in}}{\pgfqpoint{0.004420in}{0.016667in}}{\pgfqpoint{0.000000in}{0.016667in}}%
\pgfpathcurveto{\pgfqpoint{-0.004420in}{0.016667in}}{\pgfqpoint{-0.008660in}{0.014911in}}{\pgfqpoint{-0.011785in}{0.011785in}}%
\pgfpathcurveto{\pgfqpoint{-0.014911in}{0.008660in}}{\pgfqpoint{-0.016667in}{0.004420in}}{\pgfqpoint{-0.016667in}{0.000000in}}%
\pgfpathcurveto{\pgfqpoint{-0.016667in}{-0.004420in}}{\pgfqpoint{-0.014911in}{-0.008660in}}{\pgfqpoint{-0.011785in}{-0.011785in}}%
\pgfpathcurveto{\pgfqpoint{-0.008660in}{-0.014911in}}{\pgfqpoint{-0.004420in}{-0.016667in}}{\pgfqpoint{0.000000in}{-0.016667in}}%
\pgfpathlineto{\pgfqpoint{0.000000in}{-0.016667in}}%
\pgfpathclose%
\pgfusepath{stroke,fill}%
}%
\begin{pgfscope}%
\pgfsys@transformshift{3.085349in}{3.530998in}%
\pgfsys@useobject{currentmarker}{}%
\end{pgfscope}%
\end{pgfscope}%
\begin{pgfscope}%
\pgfpathrectangle{\pgfqpoint{0.765000in}{0.660000in}}{\pgfqpoint{4.620000in}{4.620000in}}%
\pgfusepath{clip}%
\pgfsetrectcap%
\pgfsetroundjoin%
\pgfsetlinewidth{1.204500pt}%
\definecolor{currentstroke}{rgb}{0.000000,0.000000,0.000000}%
\pgfsetstrokecolor{currentstroke}%
\pgfsetdash{}{0pt}%
\pgfpathmoveto{\pgfqpoint{3.085349in}{3.530998in}}%
\pgfpathlineto{\pgfqpoint{3.046316in}{3.530257in}}%
\pgfusepath{stroke}%
\end{pgfscope}%
\begin{pgfscope}%
\pgfpathrectangle{\pgfqpoint{0.765000in}{0.660000in}}{\pgfqpoint{4.620000in}{4.620000in}}%
\pgfusepath{clip}%
\pgfsetrectcap%
\pgfsetroundjoin%
\pgfsetlinewidth{1.204500pt}%
\definecolor{currentstroke}{rgb}{0.000000,0.000000,0.000000}%
\pgfsetstrokecolor{currentstroke}%
\pgfsetdash{}{0pt}%
\pgfpathmoveto{\pgfqpoint{3.085349in}{3.530998in}}%
\pgfpathlineto{\pgfqpoint{3.107502in}{3.529134in}}%
\pgfusepath{stroke}%
\end{pgfscope}%
\begin{pgfscope}%
\pgfpathrectangle{\pgfqpoint{0.765000in}{0.660000in}}{\pgfqpoint{4.620000in}{4.620000in}}%
\pgfusepath{clip}%
\pgfsetbuttcap%
\pgfsetroundjoin%
\pgfsetlinewidth{1.204500pt}%
\definecolor{currentstroke}{rgb}{0.827451,0.827451,0.827451}%
\pgfsetstrokecolor{currentstroke}%
\pgfsetdash{{1.200000pt}{1.980000pt}}{0.000000pt}%
\pgfpathmoveto{\pgfqpoint{3.085349in}{3.530998in}}%
\pgfpathlineto{\pgfqpoint{2.857298in}{5.290000in}}%
\pgfusepath{stroke}%
\end{pgfscope}%
\begin{pgfscope}%
\pgfpathrectangle{\pgfqpoint{0.765000in}{0.660000in}}{\pgfqpoint{4.620000in}{4.620000in}}%
\pgfusepath{clip}%
\pgfsetrectcap%
\pgfsetroundjoin%
\pgfsetlinewidth{1.204500pt}%
\definecolor{currentstroke}{rgb}{0.000000,0.000000,0.000000}%
\pgfsetstrokecolor{currentstroke}%
\pgfsetdash{}{0pt}%
\pgfpathmoveto{\pgfqpoint{3.046316in}{3.530257in}}%
\pgfusepath{stroke}%
\end{pgfscope}%
\begin{pgfscope}%
\pgfpathrectangle{\pgfqpoint{0.765000in}{0.660000in}}{\pgfqpoint{4.620000in}{4.620000in}}%
\pgfusepath{clip}%
\pgfsetbuttcap%
\pgfsetroundjoin%
\definecolor{currentfill}{rgb}{0.000000,0.000000,0.000000}%
\pgfsetfillcolor{currentfill}%
\pgfsetlinewidth{1.003750pt}%
\definecolor{currentstroke}{rgb}{0.000000,0.000000,0.000000}%
\pgfsetstrokecolor{currentstroke}%
\pgfsetdash{}{0pt}%
\pgfsys@defobject{currentmarker}{\pgfqpoint{-0.016667in}{-0.016667in}}{\pgfqpoint{0.016667in}{0.016667in}}{%
\pgfpathmoveto{\pgfqpoint{0.000000in}{-0.016667in}}%
\pgfpathcurveto{\pgfqpoint{0.004420in}{-0.016667in}}{\pgfqpoint{0.008660in}{-0.014911in}}{\pgfqpoint{0.011785in}{-0.011785in}}%
\pgfpathcurveto{\pgfqpoint{0.014911in}{-0.008660in}}{\pgfqpoint{0.016667in}{-0.004420in}}{\pgfqpoint{0.016667in}{0.000000in}}%
\pgfpathcurveto{\pgfqpoint{0.016667in}{0.004420in}}{\pgfqpoint{0.014911in}{0.008660in}}{\pgfqpoint{0.011785in}{0.011785in}}%
\pgfpathcurveto{\pgfqpoint{0.008660in}{0.014911in}}{\pgfqpoint{0.004420in}{0.016667in}}{\pgfqpoint{0.000000in}{0.016667in}}%
\pgfpathcurveto{\pgfqpoint{-0.004420in}{0.016667in}}{\pgfqpoint{-0.008660in}{0.014911in}}{\pgfqpoint{-0.011785in}{0.011785in}}%
\pgfpathcurveto{\pgfqpoint{-0.014911in}{0.008660in}}{\pgfqpoint{-0.016667in}{0.004420in}}{\pgfqpoint{-0.016667in}{0.000000in}}%
\pgfpathcurveto{\pgfqpoint{-0.016667in}{-0.004420in}}{\pgfqpoint{-0.014911in}{-0.008660in}}{\pgfqpoint{-0.011785in}{-0.011785in}}%
\pgfpathcurveto{\pgfqpoint{-0.008660in}{-0.014911in}}{\pgfqpoint{-0.004420in}{-0.016667in}}{\pgfqpoint{0.000000in}{-0.016667in}}%
\pgfpathlineto{\pgfqpoint{0.000000in}{-0.016667in}}%
\pgfpathclose%
\pgfusepath{stroke,fill}%
}%
\begin{pgfscope}%
\pgfsys@transformshift{3.046316in}{3.530257in}%
\pgfsys@useobject{currentmarker}{}%
\end{pgfscope}%
\end{pgfscope}%
\begin{pgfscope}%
\pgfpathrectangle{\pgfqpoint{0.765000in}{0.660000in}}{\pgfqpoint{4.620000in}{4.620000in}}%
\pgfusepath{clip}%
\pgfsetrectcap%
\pgfsetroundjoin%
\pgfsetlinewidth{1.204500pt}%
\definecolor{currentstroke}{rgb}{0.000000,0.000000,0.000000}%
\pgfsetstrokecolor{currentstroke}%
\pgfsetdash{}{0pt}%
\pgfpathmoveto{\pgfqpoint{3.046316in}{3.530257in}}%
\pgfpathlineto{\pgfqpoint{3.085349in}{3.530998in}}%
\pgfusepath{stroke}%
\end{pgfscope}%
\begin{pgfscope}%
\pgfpathrectangle{\pgfqpoint{0.765000in}{0.660000in}}{\pgfqpoint{4.620000in}{4.620000in}}%
\pgfusepath{clip}%
\pgfsetrectcap%
\pgfsetroundjoin%
\pgfsetlinewidth{1.204500pt}%
\definecolor{currentstroke}{rgb}{0.000000,0.000000,0.000000}%
\pgfsetstrokecolor{currentstroke}%
\pgfsetdash{}{0pt}%
\pgfpathmoveto{\pgfqpoint{3.046316in}{3.530257in}}%
\pgfpathlineto{\pgfqpoint{2.982382in}{3.516877in}}%
\pgfusepath{stroke}%
\end{pgfscope}%
\begin{pgfscope}%
\pgfpathrectangle{\pgfqpoint{0.765000in}{0.660000in}}{\pgfqpoint{4.620000in}{4.620000in}}%
\pgfusepath{clip}%
\pgfsetbuttcap%
\pgfsetroundjoin%
\pgfsetlinewidth{1.204500pt}%
\definecolor{currentstroke}{rgb}{0.827451,0.827451,0.827451}%
\pgfsetstrokecolor{currentstroke}%
\pgfsetdash{{1.200000pt}{1.980000pt}}{0.000000pt}%
\pgfpathmoveto{\pgfqpoint{3.046316in}{3.530257in}}%
\pgfpathlineto{\pgfqpoint{2.502292in}{5.290000in}}%
\pgfusepath{stroke}%
\end{pgfscope}%
\begin{pgfscope}%
\pgfpathrectangle{\pgfqpoint{0.765000in}{0.660000in}}{\pgfqpoint{4.620000in}{4.620000in}}%
\pgfusepath{clip}%
\pgfsetrectcap%
\pgfsetroundjoin%
\pgfsetlinewidth{1.204500pt}%
\definecolor{currentstroke}{rgb}{0.000000,0.000000,0.000000}%
\pgfsetstrokecolor{currentstroke}%
\pgfsetdash{}{0pt}%
\pgfpathmoveto{\pgfqpoint{3.242912in}{2.518793in}}%
\pgfusepath{stroke}%
\end{pgfscope}%
\begin{pgfscope}%
\pgfpathrectangle{\pgfqpoint{0.765000in}{0.660000in}}{\pgfqpoint{4.620000in}{4.620000in}}%
\pgfusepath{clip}%
\pgfsetbuttcap%
\pgfsetroundjoin%
\definecolor{currentfill}{rgb}{0.000000,0.000000,0.000000}%
\pgfsetfillcolor{currentfill}%
\pgfsetlinewidth{1.003750pt}%
\definecolor{currentstroke}{rgb}{0.000000,0.000000,0.000000}%
\pgfsetstrokecolor{currentstroke}%
\pgfsetdash{}{0pt}%
\pgfsys@defobject{currentmarker}{\pgfqpoint{-0.016667in}{-0.016667in}}{\pgfqpoint{0.016667in}{0.016667in}}{%
\pgfpathmoveto{\pgfqpoint{0.000000in}{-0.016667in}}%
\pgfpathcurveto{\pgfqpoint{0.004420in}{-0.016667in}}{\pgfqpoint{0.008660in}{-0.014911in}}{\pgfqpoint{0.011785in}{-0.011785in}}%
\pgfpathcurveto{\pgfqpoint{0.014911in}{-0.008660in}}{\pgfqpoint{0.016667in}{-0.004420in}}{\pgfqpoint{0.016667in}{0.000000in}}%
\pgfpathcurveto{\pgfqpoint{0.016667in}{0.004420in}}{\pgfqpoint{0.014911in}{0.008660in}}{\pgfqpoint{0.011785in}{0.011785in}}%
\pgfpathcurveto{\pgfqpoint{0.008660in}{0.014911in}}{\pgfqpoint{0.004420in}{0.016667in}}{\pgfqpoint{0.000000in}{0.016667in}}%
\pgfpathcurveto{\pgfqpoint{-0.004420in}{0.016667in}}{\pgfqpoint{-0.008660in}{0.014911in}}{\pgfqpoint{-0.011785in}{0.011785in}}%
\pgfpathcurveto{\pgfqpoint{-0.014911in}{0.008660in}}{\pgfqpoint{-0.016667in}{0.004420in}}{\pgfqpoint{-0.016667in}{0.000000in}}%
\pgfpathcurveto{\pgfqpoint{-0.016667in}{-0.004420in}}{\pgfqpoint{-0.014911in}{-0.008660in}}{\pgfqpoint{-0.011785in}{-0.011785in}}%
\pgfpathcurveto{\pgfqpoint{-0.008660in}{-0.014911in}}{\pgfqpoint{-0.004420in}{-0.016667in}}{\pgfqpoint{0.000000in}{-0.016667in}}%
\pgfpathlineto{\pgfqpoint{0.000000in}{-0.016667in}}%
\pgfpathclose%
\pgfusepath{stroke,fill}%
}%
\begin{pgfscope}%
\pgfsys@transformshift{3.242912in}{2.518793in}%
\pgfsys@useobject{currentmarker}{}%
\end{pgfscope}%
\end{pgfscope}%
\begin{pgfscope}%
\pgfpathrectangle{\pgfqpoint{0.765000in}{0.660000in}}{\pgfqpoint{4.620000in}{4.620000in}}%
\pgfusepath{clip}%
\pgfsetrectcap%
\pgfsetroundjoin%
\pgfsetlinewidth{1.204500pt}%
\definecolor{currentstroke}{rgb}{0.000000,0.000000,0.000000}%
\pgfsetstrokecolor{currentstroke}%
\pgfsetdash{}{0pt}%
\pgfpathmoveto{\pgfqpoint{3.242912in}{2.518793in}}%
\pgfpathlineto{\pgfqpoint{3.186392in}{2.493495in}}%
\pgfusepath{stroke}%
\end{pgfscope}%
\begin{pgfscope}%
\pgfpathrectangle{\pgfqpoint{0.765000in}{0.660000in}}{\pgfqpoint{4.620000in}{4.620000in}}%
\pgfusepath{clip}%
\pgfsetrectcap%
\pgfsetroundjoin%
\pgfsetlinewidth{1.204500pt}%
\definecolor{currentstroke}{rgb}{0.000000,0.000000,0.000000}%
\pgfsetstrokecolor{currentstroke}%
\pgfsetdash{}{0pt}%
\pgfpathmoveto{\pgfqpoint{3.242912in}{2.518793in}}%
\pgfpathlineto{\pgfqpoint{3.293363in}{2.546452in}}%
\pgfusepath{stroke}%
\end{pgfscope}%
\begin{pgfscope}%
\pgfpathrectangle{\pgfqpoint{0.765000in}{0.660000in}}{\pgfqpoint{4.620000in}{4.620000in}}%
\pgfusepath{clip}%
\pgfsetbuttcap%
\pgfsetroundjoin%
\pgfsetlinewidth{1.204500pt}%
\definecolor{currentstroke}{rgb}{0.827451,0.827451,0.827451}%
\pgfsetstrokecolor{currentstroke}%
\pgfsetdash{{1.200000pt}{1.980000pt}}{0.000000pt}%
\pgfpathmoveto{\pgfqpoint{3.242912in}{2.518793in}}%
\pgfpathlineto{\pgfqpoint{3.587880in}{1.024339in}}%
\pgfusepath{stroke}%
\end{pgfscope}%
\begin{pgfscope}%
\pgfpathrectangle{\pgfqpoint{0.765000in}{0.660000in}}{\pgfqpoint{4.620000in}{4.620000in}}%
\pgfusepath{clip}%
\pgfsetrectcap%
\pgfsetroundjoin%
\pgfsetlinewidth{1.204500pt}%
\definecolor{currentstroke}{rgb}{0.000000,0.000000,0.000000}%
\pgfsetstrokecolor{currentstroke}%
\pgfsetdash{}{0pt}%
\pgfpathmoveto{\pgfqpoint{3.186392in}{2.493495in}}%
\pgfusepath{stroke}%
\end{pgfscope}%
\begin{pgfscope}%
\pgfpathrectangle{\pgfqpoint{0.765000in}{0.660000in}}{\pgfqpoint{4.620000in}{4.620000in}}%
\pgfusepath{clip}%
\pgfsetbuttcap%
\pgfsetroundjoin%
\definecolor{currentfill}{rgb}{0.000000,0.000000,0.000000}%
\pgfsetfillcolor{currentfill}%
\pgfsetlinewidth{1.003750pt}%
\definecolor{currentstroke}{rgb}{0.000000,0.000000,0.000000}%
\pgfsetstrokecolor{currentstroke}%
\pgfsetdash{}{0pt}%
\pgfsys@defobject{currentmarker}{\pgfqpoint{-0.016667in}{-0.016667in}}{\pgfqpoint{0.016667in}{0.016667in}}{%
\pgfpathmoveto{\pgfqpoint{0.000000in}{-0.016667in}}%
\pgfpathcurveto{\pgfqpoint{0.004420in}{-0.016667in}}{\pgfqpoint{0.008660in}{-0.014911in}}{\pgfqpoint{0.011785in}{-0.011785in}}%
\pgfpathcurveto{\pgfqpoint{0.014911in}{-0.008660in}}{\pgfqpoint{0.016667in}{-0.004420in}}{\pgfqpoint{0.016667in}{0.000000in}}%
\pgfpathcurveto{\pgfqpoint{0.016667in}{0.004420in}}{\pgfqpoint{0.014911in}{0.008660in}}{\pgfqpoint{0.011785in}{0.011785in}}%
\pgfpathcurveto{\pgfqpoint{0.008660in}{0.014911in}}{\pgfqpoint{0.004420in}{0.016667in}}{\pgfqpoint{0.000000in}{0.016667in}}%
\pgfpathcurveto{\pgfqpoint{-0.004420in}{0.016667in}}{\pgfqpoint{-0.008660in}{0.014911in}}{\pgfqpoint{-0.011785in}{0.011785in}}%
\pgfpathcurveto{\pgfqpoint{-0.014911in}{0.008660in}}{\pgfqpoint{-0.016667in}{0.004420in}}{\pgfqpoint{-0.016667in}{0.000000in}}%
\pgfpathcurveto{\pgfqpoint{-0.016667in}{-0.004420in}}{\pgfqpoint{-0.014911in}{-0.008660in}}{\pgfqpoint{-0.011785in}{-0.011785in}}%
\pgfpathcurveto{\pgfqpoint{-0.008660in}{-0.014911in}}{\pgfqpoint{-0.004420in}{-0.016667in}}{\pgfqpoint{0.000000in}{-0.016667in}}%
\pgfpathlineto{\pgfqpoint{0.000000in}{-0.016667in}}%
\pgfpathclose%
\pgfusepath{stroke,fill}%
}%
\begin{pgfscope}%
\pgfsys@transformshift{3.186392in}{2.493495in}%
\pgfsys@useobject{currentmarker}{}%
\end{pgfscope}%
\end{pgfscope}%
\begin{pgfscope}%
\pgfpathrectangle{\pgfqpoint{0.765000in}{0.660000in}}{\pgfqpoint{4.620000in}{4.620000in}}%
\pgfusepath{clip}%
\pgfsetrectcap%
\pgfsetroundjoin%
\pgfsetlinewidth{1.204500pt}%
\definecolor{currentstroke}{rgb}{0.000000,0.000000,0.000000}%
\pgfsetstrokecolor{currentstroke}%
\pgfsetdash{}{0pt}%
\pgfpathmoveto{\pgfqpoint{3.186392in}{2.493495in}}%
\pgfpathlineto{\pgfqpoint{3.242912in}{2.518793in}}%
\pgfusepath{stroke}%
\end{pgfscope}%
\begin{pgfscope}%
\pgfpathrectangle{\pgfqpoint{0.765000in}{0.660000in}}{\pgfqpoint{4.620000in}{4.620000in}}%
\pgfusepath{clip}%
\pgfsetrectcap%
\pgfsetroundjoin%
\pgfsetlinewidth{1.204500pt}%
\definecolor{currentstroke}{rgb}{0.000000,0.000000,0.000000}%
\pgfsetstrokecolor{currentstroke}%
\pgfsetdash{}{0pt}%
\pgfpathmoveto{\pgfqpoint{3.186392in}{2.493495in}}%
\pgfpathlineto{\pgfqpoint{3.120119in}{2.489790in}}%
\pgfusepath{stroke}%
\end{pgfscope}%
\begin{pgfscope}%
\pgfpathrectangle{\pgfqpoint{0.765000in}{0.660000in}}{\pgfqpoint{4.620000in}{4.620000in}}%
\pgfusepath{clip}%
\pgfsetbuttcap%
\pgfsetroundjoin%
\pgfsetlinewidth{1.204500pt}%
\definecolor{currentstroke}{rgb}{0.827451,0.827451,0.827451}%
\pgfsetstrokecolor{currentstroke}%
\pgfsetdash{{1.200000pt}{1.980000pt}}{0.000000pt}%
\pgfpathmoveto{\pgfqpoint{3.186392in}{2.493495in}}%
\pgfpathlineto{\pgfqpoint{3.249185in}{0.650000in}}%
\pgfusepath{stroke}%
\end{pgfscope}%
\begin{pgfscope}%
\pgfpathrectangle{\pgfqpoint{0.765000in}{0.660000in}}{\pgfqpoint{4.620000in}{4.620000in}}%
\pgfusepath{clip}%
\pgfsetrectcap%
\pgfsetroundjoin%
\pgfsetlinewidth{1.204500pt}%
\definecolor{currentstroke}{rgb}{0.000000,0.000000,0.000000}%
\pgfsetstrokecolor{currentstroke}%
\pgfsetdash{}{0pt}%
\pgfpathmoveto{\pgfqpoint{3.071936in}{2.492937in}}%
\pgfusepath{stroke}%
\end{pgfscope}%
\begin{pgfscope}%
\pgfpathrectangle{\pgfqpoint{0.765000in}{0.660000in}}{\pgfqpoint{4.620000in}{4.620000in}}%
\pgfusepath{clip}%
\pgfsetbuttcap%
\pgfsetroundjoin%
\definecolor{currentfill}{rgb}{0.000000,0.000000,0.000000}%
\pgfsetfillcolor{currentfill}%
\pgfsetlinewidth{1.003750pt}%
\definecolor{currentstroke}{rgb}{0.000000,0.000000,0.000000}%
\pgfsetstrokecolor{currentstroke}%
\pgfsetdash{}{0pt}%
\pgfsys@defobject{currentmarker}{\pgfqpoint{-0.016667in}{-0.016667in}}{\pgfqpoint{0.016667in}{0.016667in}}{%
\pgfpathmoveto{\pgfqpoint{0.000000in}{-0.016667in}}%
\pgfpathcurveto{\pgfqpoint{0.004420in}{-0.016667in}}{\pgfqpoint{0.008660in}{-0.014911in}}{\pgfqpoint{0.011785in}{-0.011785in}}%
\pgfpathcurveto{\pgfqpoint{0.014911in}{-0.008660in}}{\pgfqpoint{0.016667in}{-0.004420in}}{\pgfqpoint{0.016667in}{0.000000in}}%
\pgfpathcurveto{\pgfqpoint{0.016667in}{0.004420in}}{\pgfqpoint{0.014911in}{0.008660in}}{\pgfqpoint{0.011785in}{0.011785in}}%
\pgfpathcurveto{\pgfqpoint{0.008660in}{0.014911in}}{\pgfqpoint{0.004420in}{0.016667in}}{\pgfqpoint{0.000000in}{0.016667in}}%
\pgfpathcurveto{\pgfqpoint{-0.004420in}{0.016667in}}{\pgfqpoint{-0.008660in}{0.014911in}}{\pgfqpoint{-0.011785in}{0.011785in}}%
\pgfpathcurveto{\pgfqpoint{-0.014911in}{0.008660in}}{\pgfqpoint{-0.016667in}{0.004420in}}{\pgfqpoint{-0.016667in}{0.000000in}}%
\pgfpathcurveto{\pgfqpoint{-0.016667in}{-0.004420in}}{\pgfqpoint{-0.014911in}{-0.008660in}}{\pgfqpoint{-0.011785in}{-0.011785in}}%
\pgfpathcurveto{\pgfqpoint{-0.008660in}{-0.014911in}}{\pgfqpoint{-0.004420in}{-0.016667in}}{\pgfqpoint{0.000000in}{-0.016667in}}%
\pgfpathlineto{\pgfqpoint{0.000000in}{-0.016667in}}%
\pgfpathclose%
\pgfusepath{stroke,fill}%
}%
\begin{pgfscope}%
\pgfsys@transformshift{3.071936in}{2.492937in}%
\pgfsys@useobject{currentmarker}{}%
\end{pgfscope}%
\end{pgfscope}%
\begin{pgfscope}%
\pgfpathrectangle{\pgfqpoint{0.765000in}{0.660000in}}{\pgfqpoint{4.620000in}{4.620000in}}%
\pgfusepath{clip}%
\pgfsetrectcap%
\pgfsetroundjoin%
\pgfsetlinewidth{1.204500pt}%
\definecolor{currentstroke}{rgb}{0.000000,0.000000,0.000000}%
\pgfsetstrokecolor{currentstroke}%
\pgfsetdash{}{0pt}%
\pgfpathmoveto{\pgfqpoint{3.071936in}{2.492937in}}%
\pgfpathlineto{\pgfqpoint{2.979627in}{2.512096in}}%
\pgfusepath{stroke}%
\end{pgfscope}%
\begin{pgfscope}%
\pgfpathrectangle{\pgfqpoint{0.765000in}{0.660000in}}{\pgfqpoint{4.620000in}{4.620000in}}%
\pgfusepath{clip}%
\pgfsetrectcap%
\pgfsetroundjoin%
\pgfsetlinewidth{1.204500pt}%
\definecolor{currentstroke}{rgb}{0.000000,0.000000,0.000000}%
\pgfsetstrokecolor{currentstroke}%
\pgfsetdash{}{0pt}%
\pgfpathmoveto{\pgfqpoint{3.071936in}{2.492937in}}%
\pgfpathlineto{\pgfqpoint{3.120119in}{2.489790in}}%
\pgfusepath{stroke}%
\end{pgfscope}%
\begin{pgfscope}%
\pgfpathrectangle{\pgfqpoint{0.765000in}{0.660000in}}{\pgfqpoint{4.620000in}{4.620000in}}%
\pgfusepath{clip}%
\pgfsetbuttcap%
\pgfsetroundjoin%
\pgfsetlinewidth{1.204500pt}%
\definecolor{currentstroke}{rgb}{0.827451,0.827451,0.827451}%
\pgfsetstrokecolor{currentstroke}%
\pgfsetdash{{1.200000pt}{1.980000pt}}{0.000000pt}%
\pgfpathmoveto{\pgfqpoint{3.071936in}{2.492937in}}%
\pgfpathlineto{\pgfqpoint{2.550804in}{0.650000in}}%
\pgfusepath{stroke}%
\end{pgfscope}%
\begin{pgfscope}%
\pgfpathrectangle{\pgfqpoint{0.765000in}{0.660000in}}{\pgfqpoint{4.620000in}{4.620000in}}%
\pgfusepath{clip}%
\pgfsetrectcap%
\pgfsetroundjoin%
\pgfsetlinewidth{1.204500pt}%
\definecolor{currentstroke}{rgb}{0.000000,0.000000,0.000000}%
\pgfsetstrokecolor{currentstroke}%
\pgfsetdash{}{0pt}%
\pgfpathmoveto{\pgfqpoint{3.120119in}{2.489790in}}%
\pgfusepath{stroke}%
\end{pgfscope}%
\begin{pgfscope}%
\pgfpathrectangle{\pgfqpoint{0.765000in}{0.660000in}}{\pgfqpoint{4.620000in}{4.620000in}}%
\pgfusepath{clip}%
\pgfsetbuttcap%
\pgfsetroundjoin%
\definecolor{currentfill}{rgb}{0.000000,0.000000,0.000000}%
\pgfsetfillcolor{currentfill}%
\pgfsetlinewidth{1.003750pt}%
\definecolor{currentstroke}{rgb}{0.000000,0.000000,0.000000}%
\pgfsetstrokecolor{currentstroke}%
\pgfsetdash{}{0pt}%
\pgfsys@defobject{currentmarker}{\pgfqpoint{-0.016667in}{-0.016667in}}{\pgfqpoint{0.016667in}{0.016667in}}{%
\pgfpathmoveto{\pgfqpoint{0.000000in}{-0.016667in}}%
\pgfpathcurveto{\pgfqpoint{0.004420in}{-0.016667in}}{\pgfqpoint{0.008660in}{-0.014911in}}{\pgfqpoint{0.011785in}{-0.011785in}}%
\pgfpathcurveto{\pgfqpoint{0.014911in}{-0.008660in}}{\pgfqpoint{0.016667in}{-0.004420in}}{\pgfqpoint{0.016667in}{0.000000in}}%
\pgfpathcurveto{\pgfqpoint{0.016667in}{0.004420in}}{\pgfqpoint{0.014911in}{0.008660in}}{\pgfqpoint{0.011785in}{0.011785in}}%
\pgfpathcurveto{\pgfqpoint{0.008660in}{0.014911in}}{\pgfqpoint{0.004420in}{0.016667in}}{\pgfqpoint{0.000000in}{0.016667in}}%
\pgfpathcurveto{\pgfqpoint{-0.004420in}{0.016667in}}{\pgfqpoint{-0.008660in}{0.014911in}}{\pgfqpoint{-0.011785in}{0.011785in}}%
\pgfpathcurveto{\pgfqpoint{-0.014911in}{0.008660in}}{\pgfqpoint{-0.016667in}{0.004420in}}{\pgfqpoint{-0.016667in}{0.000000in}}%
\pgfpathcurveto{\pgfqpoint{-0.016667in}{-0.004420in}}{\pgfqpoint{-0.014911in}{-0.008660in}}{\pgfqpoint{-0.011785in}{-0.011785in}}%
\pgfpathcurveto{\pgfqpoint{-0.008660in}{-0.014911in}}{\pgfqpoint{-0.004420in}{-0.016667in}}{\pgfqpoint{0.000000in}{-0.016667in}}%
\pgfpathlineto{\pgfqpoint{0.000000in}{-0.016667in}}%
\pgfpathclose%
\pgfusepath{stroke,fill}%
}%
\begin{pgfscope}%
\pgfsys@transformshift{3.120119in}{2.489790in}%
\pgfsys@useobject{currentmarker}{}%
\end{pgfscope}%
\end{pgfscope}%
\begin{pgfscope}%
\pgfpathrectangle{\pgfqpoint{0.765000in}{0.660000in}}{\pgfqpoint{4.620000in}{4.620000in}}%
\pgfusepath{clip}%
\pgfsetrectcap%
\pgfsetroundjoin%
\pgfsetlinewidth{1.204500pt}%
\definecolor{currentstroke}{rgb}{0.000000,0.000000,0.000000}%
\pgfsetstrokecolor{currentstroke}%
\pgfsetdash{}{0pt}%
\pgfpathmoveto{\pgfqpoint{3.120119in}{2.489790in}}%
\pgfpathlineto{\pgfqpoint{3.186392in}{2.493495in}}%
\pgfusepath{stroke}%
\end{pgfscope}%
\begin{pgfscope}%
\pgfpathrectangle{\pgfqpoint{0.765000in}{0.660000in}}{\pgfqpoint{4.620000in}{4.620000in}}%
\pgfusepath{clip}%
\pgfsetrectcap%
\pgfsetroundjoin%
\pgfsetlinewidth{1.204500pt}%
\definecolor{currentstroke}{rgb}{0.000000,0.000000,0.000000}%
\pgfsetstrokecolor{currentstroke}%
\pgfsetdash{}{0pt}%
\pgfpathmoveto{\pgfqpoint{3.120119in}{2.489790in}}%
\pgfpathlineto{\pgfqpoint{3.071936in}{2.492937in}}%
\pgfusepath{stroke}%
\end{pgfscope}%
\begin{pgfscope}%
\pgfpathrectangle{\pgfqpoint{0.765000in}{0.660000in}}{\pgfqpoint{4.620000in}{4.620000in}}%
\pgfusepath{clip}%
\pgfsetbuttcap%
\pgfsetroundjoin%
\pgfsetlinewidth{1.204500pt}%
\definecolor{currentstroke}{rgb}{0.827451,0.827451,0.827451}%
\pgfsetstrokecolor{currentstroke}%
\pgfsetdash{{1.200000pt}{1.980000pt}}{0.000000pt}%
\pgfpathmoveto{\pgfqpoint{3.120119in}{2.489790in}}%
\pgfpathlineto{\pgfqpoint{2.801771in}{0.650000in}}%
\pgfusepath{stroke}%
\end{pgfscope}%
\begin{pgfscope}%
\pgfpathrectangle{\pgfqpoint{0.765000in}{0.660000in}}{\pgfqpoint{4.620000in}{4.620000in}}%
\pgfusepath{clip}%
\pgfsetrectcap%
\pgfsetroundjoin%
\pgfsetlinewidth{1.204500pt}%
\definecolor{currentstroke}{rgb}{0.000000,0.000000,0.000000}%
\pgfsetstrokecolor{currentstroke}%
\pgfsetdash{}{0pt}%
\pgfpathmoveto{\pgfqpoint{3.296975in}{3.470058in}}%
\pgfusepath{stroke}%
\end{pgfscope}%
\begin{pgfscope}%
\pgfpathrectangle{\pgfqpoint{0.765000in}{0.660000in}}{\pgfqpoint{4.620000in}{4.620000in}}%
\pgfusepath{clip}%
\pgfsetbuttcap%
\pgfsetroundjoin%
\definecolor{currentfill}{rgb}{0.000000,0.000000,0.000000}%
\pgfsetfillcolor{currentfill}%
\pgfsetlinewidth{1.003750pt}%
\definecolor{currentstroke}{rgb}{0.000000,0.000000,0.000000}%
\pgfsetstrokecolor{currentstroke}%
\pgfsetdash{}{0pt}%
\pgfsys@defobject{currentmarker}{\pgfqpoint{-0.016667in}{-0.016667in}}{\pgfqpoint{0.016667in}{0.016667in}}{%
\pgfpathmoveto{\pgfqpoint{0.000000in}{-0.016667in}}%
\pgfpathcurveto{\pgfqpoint{0.004420in}{-0.016667in}}{\pgfqpoint{0.008660in}{-0.014911in}}{\pgfqpoint{0.011785in}{-0.011785in}}%
\pgfpathcurveto{\pgfqpoint{0.014911in}{-0.008660in}}{\pgfqpoint{0.016667in}{-0.004420in}}{\pgfqpoint{0.016667in}{0.000000in}}%
\pgfpathcurveto{\pgfqpoint{0.016667in}{0.004420in}}{\pgfqpoint{0.014911in}{0.008660in}}{\pgfqpoint{0.011785in}{0.011785in}}%
\pgfpathcurveto{\pgfqpoint{0.008660in}{0.014911in}}{\pgfqpoint{0.004420in}{0.016667in}}{\pgfqpoint{0.000000in}{0.016667in}}%
\pgfpathcurveto{\pgfqpoint{-0.004420in}{0.016667in}}{\pgfqpoint{-0.008660in}{0.014911in}}{\pgfqpoint{-0.011785in}{0.011785in}}%
\pgfpathcurveto{\pgfqpoint{-0.014911in}{0.008660in}}{\pgfqpoint{-0.016667in}{0.004420in}}{\pgfqpoint{-0.016667in}{0.000000in}}%
\pgfpathcurveto{\pgfqpoint{-0.016667in}{-0.004420in}}{\pgfqpoint{-0.014911in}{-0.008660in}}{\pgfqpoint{-0.011785in}{-0.011785in}}%
\pgfpathcurveto{\pgfqpoint{-0.008660in}{-0.014911in}}{\pgfqpoint{-0.004420in}{-0.016667in}}{\pgfqpoint{0.000000in}{-0.016667in}}%
\pgfpathlineto{\pgfqpoint{0.000000in}{-0.016667in}}%
\pgfpathclose%
\pgfusepath{stroke,fill}%
}%
\begin{pgfscope}%
\pgfsys@transformshift{3.296975in}{3.470058in}%
\pgfsys@useobject{currentmarker}{}%
\end{pgfscope}%
\end{pgfscope}%
\begin{pgfscope}%
\pgfpathrectangle{\pgfqpoint{0.765000in}{0.660000in}}{\pgfqpoint{4.620000in}{4.620000in}}%
\pgfusepath{clip}%
\pgfsetrectcap%
\pgfsetroundjoin%
\pgfsetlinewidth{1.204500pt}%
\definecolor{currentstroke}{rgb}{0.000000,0.000000,0.000000}%
\pgfsetstrokecolor{currentstroke}%
\pgfsetdash{}{0pt}%
\pgfpathmoveto{\pgfqpoint{3.296975in}{3.470058in}}%
\pgfpathlineto{\pgfqpoint{3.265487in}{3.489049in}}%
\pgfusepath{stroke}%
\end{pgfscope}%
\begin{pgfscope}%
\pgfpathrectangle{\pgfqpoint{0.765000in}{0.660000in}}{\pgfqpoint{4.620000in}{4.620000in}}%
\pgfusepath{clip}%
\pgfsetrectcap%
\pgfsetroundjoin%
\pgfsetlinewidth{1.204500pt}%
\definecolor{currentstroke}{rgb}{0.000000,0.000000,0.000000}%
\pgfsetstrokecolor{currentstroke}%
\pgfsetdash{}{0pt}%
\pgfpathmoveto{\pgfqpoint{3.296975in}{3.470058in}}%
\pgfpathlineto{\pgfqpoint{3.310017in}{3.461366in}}%
\pgfusepath{stroke}%
\end{pgfscope}%
\begin{pgfscope}%
\pgfpathrectangle{\pgfqpoint{0.765000in}{0.660000in}}{\pgfqpoint{4.620000in}{4.620000in}}%
\pgfusepath{clip}%
\pgfsetbuttcap%
\pgfsetroundjoin%
\pgfsetlinewidth{1.204500pt}%
\definecolor{currentstroke}{rgb}{0.827451,0.827451,0.827451}%
\pgfsetstrokecolor{currentstroke}%
\pgfsetdash{{1.200000pt}{1.980000pt}}{0.000000pt}%
\pgfpathmoveto{\pgfqpoint{3.296975in}{3.470058in}}%
\pgfpathlineto{\pgfqpoint{3.888703in}{4.494361in}}%
\pgfusepath{stroke}%
\end{pgfscope}%
\begin{pgfscope}%
\pgfpathrectangle{\pgfqpoint{0.765000in}{0.660000in}}{\pgfqpoint{4.620000in}{4.620000in}}%
\pgfusepath{clip}%
\pgfsetrectcap%
\pgfsetroundjoin%
\pgfsetlinewidth{1.204500pt}%
\definecolor{currentstroke}{rgb}{0.000000,0.000000,0.000000}%
\pgfsetstrokecolor{currentstroke}%
\pgfsetdash{}{0pt}%
\pgfpathmoveto{\pgfqpoint{3.265487in}{3.489049in}}%
\pgfusepath{stroke}%
\end{pgfscope}%
\begin{pgfscope}%
\pgfpathrectangle{\pgfqpoint{0.765000in}{0.660000in}}{\pgfqpoint{4.620000in}{4.620000in}}%
\pgfusepath{clip}%
\pgfsetbuttcap%
\pgfsetroundjoin%
\definecolor{currentfill}{rgb}{0.000000,0.000000,0.000000}%
\pgfsetfillcolor{currentfill}%
\pgfsetlinewidth{1.003750pt}%
\definecolor{currentstroke}{rgb}{0.000000,0.000000,0.000000}%
\pgfsetstrokecolor{currentstroke}%
\pgfsetdash{}{0pt}%
\pgfsys@defobject{currentmarker}{\pgfqpoint{-0.016667in}{-0.016667in}}{\pgfqpoint{0.016667in}{0.016667in}}{%
\pgfpathmoveto{\pgfqpoint{0.000000in}{-0.016667in}}%
\pgfpathcurveto{\pgfqpoint{0.004420in}{-0.016667in}}{\pgfqpoint{0.008660in}{-0.014911in}}{\pgfqpoint{0.011785in}{-0.011785in}}%
\pgfpathcurveto{\pgfqpoint{0.014911in}{-0.008660in}}{\pgfqpoint{0.016667in}{-0.004420in}}{\pgfqpoint{0.016667in}{0.000000in}}%
\pgfpathcurveto{\pgfqpoint{0.016667in}{0.004420in}}{\pgfqpoint{0.014911in}{0.008660in}}{\pgfqpoint{0.011785in}{0.011785in}}%
\pgfpathcurveto{\pgfqpoint{0.008660in}{0.014911in}}{\pgfqpoint{0.004420in}{0.016667in}}{\pgfqpoint{0.000000in}{0.016667in}}%
\pgfpathcurveto{\pgfqpoint{-0.004420in}{0.016667in}}{\pgfqpoint{-0.008660in}{0.014911in}}{\pgfqpoint{-0.011785in}{0.011785in}}%
\pgfpathcurveto{\pgfqpoint{-0.014911in}{0.008660in}}{\pgfqpoint{-0.016667in}{0.004420in}}{\pgfqpoint{-0.016667in}{0.000000in}}%
\pgfpathcurveto{\pgfqpoint{-0.016667in}{-0.004420in}}{\pgfqpoint{-0.014911in}{-0.008660in}}{\pgfqpoint{-0.011785in}{-0.011785in}}%
\pgfpathcurveto{\pgfqpoint{-0.008660in}{-0.014911in}}{\pgfqpoint{-0.004420in}{-0.016667in}}{\pgfqpoint{0.000000in}{-0.016667in}}%
\pgfpathlineto{\pgfqpoint{0.000000in}{-0.016667in}}%
\pgfpathclose%
\pgfusepath{stroke,fill}%
}%
\begin{pgfscope}%
\pgfsys@transformshift{3.265487in}{3.489049in}%
\pgfsys@useobject{currentmarker}{}%
\end{pgfscope}%
\end{pgfscope}%
\begin{pgfscope}%
\pgfpathrectangle{\pgfqpoint{0.765000in}{0.660000in}}{\pgfqpoint{4.620000in}{4.620000in}}%
\pgfusepath{clip}%
\pgfsetrectcap%
\pgfsetroundjoin%
\pgfsetlinewidth{1.204500pt}%
\definecolor{currentstroke}{rgb}{0.000000,0.000000,0.000000}%
\pgfsetstrokecolor{currentstroke}%
\pgfsetdash{}{0pt}%
\pgfpathmoveto{\pgfqpoint{3.265487in}{3.489049in}}%
\pgfpathlineto{\pgfqpoint{3.296975in}{3.470058in}}%
\pgfusepath{stroke}%
\end{pgfscope}%
\begin{pgfscope}%
\pgfpathrectangle{\pgfqpoint{0.765000in}{0.660000in}}{\pgfqpoint{4.620000in}{4.620000in}}%
\pgfusepath{clip}%
\pgfsetrectcap%
\pgfsetroundjoin%
\pgfsetlinewidth{1.204500pt}%
\definecolor{currentstroke}{rgb}{0.000000,0.000000,0.000000}%
\pgfsetstrokecolor{currentstroke}%
\pgfsetdash{}{0pt}%
\pgfpathmoveto{\pgfqpoint{3.265487in}{3.489049in}}%
\pgfpathlineto{\pgfqpoint{3.230111in}{3.500132in}}%
\pgfusepath{stroke}%
\end{pgfscope}%
\begin{pgfscope}%
\pgfpathrectangle{\pgfqpoint{0.765000in}{0.660000in}}{\pgfqpoint{4.620000in}{4.620000in}}%
\pgfusepath{clip}%
\pgfsetbuttcap%
\pgfsetroundjoin%
\pgfsetlinewidth{1.204500pt}%
\definecolor{currentstroke}{rgb}{0.827451,0.827451,0.827451}%
\pgfsetstrokecolor{currentstroke}%
\pgfsetdash{{1.200000pt}{1.980000pt}}{0.000000pt}%
\pgfpathmoveto{\pgfqpoint{3.265487in}{3.489049in}}%
\pgfpathlineto{\pgfqpoint{3.918140in}{5.290000in}}%
\pgfusepath{stroke}%
\end{pgfscope}%
\begin{pgfscope}%
\pgfpathrectangle{\pgfqpoint{0.765000in}{0.660000in}}{\pgfqpoint{4.620000in}{4.620000in}}%
\pgfusepath{clip}%
\pgfsetrectcap%
\pgfsetroundjoin%
\pgfsetlinewidth{1.204500pt}%
\definecolor{currentstroke}{rgb}{0.000000,0.000000,0.000000}%
\pgfsetstrokecolor{currentstroke}%
\pgfsetdash{}{0pt}%
\pgfpathmoveto{\pgfqpoint{2.929883in}{2.530541in}}%
\pgfusepath{stroke}%
\end{pgfscope}%
\begin{pgfscope}%
\pgfpathrectangle{\pgfqpoint{0.765000in}{0.660000in}}{\pgfqpoint{4.620000in}{4.620000in}}%
\pgfusepath{clip}%
\pgfsetbuttcap%
\pgfsetroundjoin%
\definecolor{currentfill}{rgb}{0.000000,0.000000,0.000000}%
\pgfsetfillcolor{currentfill}%
\pgfsetlinewidth{1.003750pt}%
\definecolor{currentstroke}{rgb}{0.000000,0.000000,0.000000}%
\pgfsetstrokecolor{currentstroke}%
\pgfsetdash{}{0pt}%
\pgfsys@defobject{currentmarker}{\pgfqpoint{-0.016667in}{-0.016667in}}{\pgfqpoint{0.016667in}{0.016667in}}{%
\pgfpathmoveto{\pgfqpoint{0.000000in}{-0.016667in}}%
\pgfpathcurveto{\pgfqpoint{0.004420in}{-0.016667in}}{\pgfqpoint{0.008660in}{-0.014911in}}{\pgfqpoint{0.011785in}{-0.011785in}}%
\pgfpathcurveto{\pgfqpoint{0.014911in}{-0.008660in}}{\pgfqpoint{0.016667in}{-0.004420in}}{\pgfqpoint{0.016667in}{0.000000in}}%
\pgfpathcurveto{\pgfqpoint{0.016667in}{0.004420in}}{\pgfqpoint{0.014911in}{0.008660in}}{\pgfqpoint{0.011785in}{0.011785in}}%
\pgfpathcurveto{\pgfqpoint{0.008660in}{0.014911in}}{\pgfqpoint{0.004420in}{0.016667in}}{\pgfqpoint{0.000000in}{0.016667in}}%
\pgfpathcurveto{\pgfqpoint{-0.004420in}{0.016667in}}{\pgfqpoint{-0.008660in}{0.014911in}}{\pgfqpoint{-0.011785in}{0.011785in}}%
\pgfpathcurveto{\pgfqpoint{-0.014911in}{0.008660in}}{\pgfqpoint{-0.016667in}{0.004420in}}{\pgfqpoint{-0.016667in}{0.000000in}}%
\pgfpathcurveto{\pgfqpoint{-0.016667in}{-0.004420in}}{\pgfqpoint{-0.014911in}{-0.008660in}}{\pgfqpoint{-0.011785in}{-0.011785in}}%
\pgfpathcurveto{\pgfqpoint{-0.008660in}{-0.014911in}}{\pgfqpoint{-0.004420in}{-0.016667in}}{\pgfqpoint{0.000000in}{-0.016667in}}%
\pgfpathlineto{\pgfqpoint{0.000000in}{-0.016667in}}%
\pgfpathclose%
\pgfusepath{stroke,fill}%
}%
\begin{pgfscope}%
\pgfsys@transformshift{2.929883in}{2.530541in}%
\pgfsys@useobject{currentmarker}{}%
\end{pgfscope}%
\end{pgfscope}%
\begin{pgfscope}%
\pgfpathrectangle{\pgfqpoint{0.765000in}{0.660000in}}{\pgfqpoint{4.620000in}{4.620000in}}%
\pgfusepath{clip}%
\pgfsetrectcap%
\pgfsetroundjoin%
\pgfsetlinewidth{1.204500pt}%
\definecolor{currentstroke}{rgb}{0.000000,0.000000,0.000000}%
\pgfsetstrokecolor{currentstroke}%
\pgfsetdash{}{0pt}%
\pgfpathmoveto{\pgfqpoint{2.929883in}{2.530541in}}%
\pgfpathlineto{\pgfqpoint{2.952980in}{2.520265in}}%
\pgfusepath{stroke}%
\end{pgfscope}%
\begin{pgfscope}%
\pgfpathrectangle{\pgfqpoint{0.765000in}{0.660000in}}{\pgfqpoint{4.620000in}{4.620000in}}%
\pgfusepath{clip}%
\pgfsetrectcap%
\pgfsetroundjoin%
\pgfsetlinewidth{1.204500pt}%
\definecolor{currentstroke}{rgb}{0.000000,0.000000,0.000000}%
\pgfsetstrokecolor{currentstroke}%
\pgfsetdash{}{0pt}%
\pgfpathmoveto{\pgfqpoint{2.929883in}{2.530541in}}%
\pgfpathlineto{\pgfqpoint{2.915238in}{2.538405in}}%
\pgfusepath{stroke}%
\end{pgfscope}%
\begin{pgfscope}%
\pgfpathrectangle{\pgfqpoint{0.765000in}{0.660000in}}{\pgfqpoint{4.620000in}{4.620000in}}%
\pgfusepath{clip}%
\pgfsetbuttcap%
\pgfsetroundjoin%
\pgfsetlinewidth{1.204500pt}%
\definecolor{currentstroke}{rgb}{0.827451,0.827451,0.827451}%
\pgfsetstrokecolor{currentstroke}%
\pgfsetdash{{1.200000pt}{1.980000pt}}{0.000000pt}%
\pgfpathmoveto{\pgfqpoint{2.929883in}{2.530541in}}%
\pgfpathlineto{\pgfqpoint{2.338155in}{1.506238in}}%
\pgfusepath{stroke}%
\end{pgfscope}%
\begin{pgfscope}%
\pgfpathrectangle{\pgfqpoint{0.765000in}{0.660000in}}{\pgfqpoint{4.620000in}{4.620000in}}%
\pgfusepath{clip}%
\pgfsetrectcap%
\pgfsetroundjoin%
\pgfsetlinewidth{1.204500pt}%
\definecolor{currentstroke}{rgb}{0.000000,0.000000,0.000000}%
\pgfsetstrokecolor{currentstroke}%
\pgfsetdash{}{0pt}%
\pgfpathmoveto{\pgfqpoint{3.334893in}{3.441865in}}%
\pgfusepath{stroke}%
\end{pgfscope}%
\begin{pgfscope}%
\pgfpathrectangle{\pgfqpoint{0.765000in}{0.660000in}}{\pgfqpoint{4.620000in}{4.620000in}}%
\pgfusepath{clip}%
\pgfsetbuttcap%
\pgfsetroundjoin%
\definecolor{currentfill}{rgb}{0.000000,0.000000,0.000000}%
\pgfsetfillcolor{currentfill}%
\pgfsetlinewidth{1.003750pt}%
\definecolor{currentstroke}{rgb}{0.000000,0.000000,0.000000}%
\pgfsetstrokecolor{currentstroke}%
\pgfsetdash{}{0pt}%
\pgfsys@defobject{currentmarker}{\pgfqpoint{-0.016667in}{-0.016667in}}{\pgfqpoint{0.016667in}{0.016667in}}{%
\pgfpathmoveto{\pgfqpoint{0.000000in}{-0.016667in}}%
\pgfpathcurveto{\pgfqpoint{0.004420in}{-0.016667in}}{\pgfqpoint{0.008660in}{-0.014911in}}{\pgfqpoint{0.011785in}{-0.011785in}}%
\pgfpathcurveto{\pgfqpoint{0.014911in}{-0.008660in}}{\pgfqpoint{0.016667in}{-0.004420in}}{\pgfqpoint{0.016667in}{0.000000in}}%
\pgfpathcurveto{\pgfqpoint{0.016667in}{0.004420in}}{\pgfqpoint{0.014911in}{0.008660in}}{\pgfqpoint{0.011785in}{0.011785in}}%
\pgfpathcurveto{\pgfqpoint{0.008660in}{0.014911in}}{\pgfqpoint{0.004420in}{0.016667in}}{\pgfqpoint{0.000000in}{0.016667in}}%
\pgfpathcurveto{\pgfqpoint{-0.004420in}{0.016667in}}{\pgfqpoint{-0.008660in}{0.014911in}}{\pgfqpoint{-0.011785in}{0.011785in}}%
\pgfpathcurveto{\pgfqpoint{-0.014911in}{0.008660in}}{\pgfqpoint{-0.016667in}{0.004420in}}{\pgfqpoint{-0.016667in}{0.000000in}}%
\pgfpathcurveto{\pgfqpoint{-0.016667in}{-0.004420in}}{\pgfqpoint{-0.014911in}{-0.008660in}}{\pgfqpoint{-0.011785in}{-0.011785in}}%
\pgfpathcurveto{\pgfqpoint{-0.008660in}{-0.014911in}}{\pgfqpoint{-0.004420in}{-0.016667in}}{\pgfqpoint{0.000000in}{-0.016667in}}%
\pgfpathlineto{\pgfqpoint{0.000000in}{-0.016667in}}%
\pgfpathclose%
\pgfusepath{stroke,fill}%
}%
\begin{pgfscope}%
\pgfsys@transformshift{3.334893in}{3.441865in}%
\pgfsys@useobject{currentmarker}{}%
\end{pgfscope}%
\end{pgfscope}%
\begin{pgfscope}%
\pgfpathrectangle{\pgfqpoint{0.765000in}{0.660000in}}{\pgfqpoint{4.620000in}{4.620000in}}%
\pgfusepath{clip}%
\pgfsetrectcap%
\pgfsetroundjoin%
\pgfsetlinewidth{1.204500pt}%
\definecolor{currentstroke}{rgb}{0.000000,0.000000,0.000000}%
\pgfsetstrokecolor{currentstroke}%
\pgfsetdash{}{0pt}%
\pgfpathmoveto{\pgfqpoint{3.334893in}{3.441865in}}%
\pgfpathlineto{\pgfqpoint{3.310017in}{3.461366in}}%
\pgfusepath{stroke}%
\end{pgfscope}%
\begin{pgfscope}%
\pgfpathrectangle{\pgfqpoint{0.765000in}{0.660000in}}{\pgfqpoint{4.620000in}{4.620000in}}%
\pgfusepath{clip}%
\pgfsetrectcap%
\pgfsetroundjoin%
\pgfsetlinewidth{1.204500pt}%
\definecolor{currentstroke}{rgb}{0.000000,0.000000,0.000000}%
\pgfsetstrokecolor{currentstroke}%
\pgfsetdash{}{0pt}%
\pgfpathmoveto{\pgfqpoint{3.334893in}{3.441865in}}%
\pgfpathlineto{\pgfqpoint{3.350280in}{3.424899in}}%
\pgfusepath{stroke}%
\end{pgfscope}%
\begin{pgfscope}%
\pgfpathrectangle{\pgfqpoint{0.765000in}{0.660000in}}{\pgfqpoint{4.620000in}{4.620000in}}%
\pgfusepath{clip}%
\pgfsetbuttcap%
\pgfsetroundjoin%
\pgfsetlinewidth{1.204500pt}%
\definecolor{currentstroke}{rgb}{0.827451,0.827451,0.827451}%
\pgfsetstrokecolor{currentstroke}%
\pgfsetdash{{1.200000pt}{1.980000pt}}{0.000000pt}%
\pgfpathmoveto{\pgfqpoint{3.334893in}{3.441865in}}%
\pgfpathlineto{\pgfqpoint{5.070241in}{5.290000in}}%
\pgfusepath{stroke}%
\end{pgfscope}%
\begin{pgfscope}%
\pgfpathrectangle{\pgfqpoint{0.765000in}{0.660000in}}{\pgfqpoint{4.620000in}{4.620000in}}%
\pgfusepath{clip}%
\pgfsetrectcap%
\pgfsetroundjoin%
\pgfsetlinewidth{1.204500pt}%
\definecolor{currentstroke}{rgb}{0.000000,0.000000,0.000000}%
\pgfsetstrokecolor{currentstroke}%
\pgfsetdash{}{0pt}%
\pgfpathmoveto{\pgfqpoint{3.310017in}{3.461366in}}%
\pgfusepath{stroke}%
\end{pgfscope}%
\begin{pgfscope}%
\pgfpathrectangle{\pgfqpoint{0.765000in}{0.660000in}}{\pgfqpoint{4.620000in}{4.620000in}}%
\pgfusepath{clip}%
\pgfsetbuttcap%
\pgfsetroundjoin%
\definecolor{currentfill}{rgb}{0.000000,0.000000,0.000000}%
\pgfsetfillcolor{currentfill}%
\pgfsetlinewidth{1.003750pt}%
\definecolor{currentstroke}{rgb}{0.000000,0.000000,0.000000}%
\pgfsetstrokecolor{currentstroke}%
\pgfsetdash{}{0pt}%
\pgfsys@defobject{currentmarker}{\pgfqpoint{-0.016667in}{-0.016667in}}{\pgfqpoint{0.016667in}{0.016667in}}{%
\pgfpathmoveto{\pgfqpoint{0.000000in}{-0.016667in}}%
\pgfpathcurveto{\pgfqpoint{0.004420in}{-0.016667in}}{\pgfqpoint{0.008660in}{-0.014911in}}{\pgfqpoint{0.011785in}{-0.011785in}}%
\pgfpathcurveto{\pgfqpoint{0.014911in}{-0.008660in}}{\pgfqpoint{0.016667in}{-0.004420in}}{\pgfqpoint{0.016667in}{0.000000in}}%
\pgfpathcurveto{\pgfqpoint{0.016667in}{0.004420in}}{\pgfqpoint{0.014911in}{0.008660in}}{\pgfqpoint{0.011785in}{0.011785in}}%
\pgfpathcurveto{\pgfqpoint{0.008660in}{0.014911in}}{\pgfqpoint{0.004420in}{0.016667in}}{\pgfqpoint{0.000000in}{0.016667in}}%
\pgfpathcurveto{\pgfqpoint{-0.004420in}{0.016667in}}{\pgfqpoint{-0.008660in}{0.014911in}}{\pgfqpoint{-0.011785in}{0.011785in}}%
\pgfpathcurveto{\pgfqpoint{-0.014911in}{0.008660in}}{\pgfqpoint{-0.016667in}{0.004420in}}{\pgfqpoint{-0.016667in}{0.000000in}}%
\pgfpathcurveto{\pgfqpoint{-0.016667in}{-0.004420in}}{\pgfqpoint{-0.014911in}{-0.008660in}}{\pgfqpoint{-0.011785in}{-0.011785in}}%
\pgfpathcurveto{\pgfqpoint{-0.008660in}{-0.014911in}}{\pgfqpoint{-0.004420in}{-0.016667in}}{\pgfqpoint{0.000000in}{-0.016667in}}%
\pgfpathlineto{\pgfqpoint{0.000000in}{-0.016667in}}%
\pgfpathclose%
\pgfusepath{stroke,fill}%
}%
\begin{pgfscope}%
\pgfsys@transformshift{3.310017in}{3.461366in}%
\pgfsys@useobject{currentmarker}{}%
\end{pgfscope}%
\end{pgfscope}%
\begin{pgfscope}%
\pgfpathrectangle{\pgfqpoint{0.765000in}{0.660000in}}{\pgfqpoint{4.620000in}{4.620000in}}%
\pgfusepath{clip}%
\pgfsetrectcap%
\pgfsetroundjoin%
\pgfsetlinewidth{1.204500pt}%
\definecolor{currentstroke}{rgb}{0.000000,0.000000,0.000000}%
\pgfsetstrokecolor{currentstroke}%
\pgfsetdash{}{0pt}%
\pgfpathmoveto{\pgfqpoint{3.310017in}{3.461366in}}%
\pgfpathlineto{\pgfqpoint{3.296975in}{3.470058in}}%
\pgfusepath{stroke}%
\end{pgfscope}%
\begin{pgfscope}%
\pgfpathrectangle{\pgfqpoint{0.765000in}{0.660000in}}{\pgfqpoint{4.620000in}{4.620000in}}%
\pgfusepath{clip}%
\pgfsetrectcap%
\pgfsetroundjoin%
\pgfsetlinewidth{1.204500pt}%
\definecolor{currentstroke}{rgb}{0.000000,0.000000,0.000000}%
\pgfsetstrokecolor{currentstroke}%
\pgfsetdash{}{0pt}%
\pgfpathmoveto{\pgfqpoint{3.310017in}{3.461366in}}%
\pgfpathlineto{\pgfqpoint{3.334893in}{3.441865in}}%
\pgfusepath{stroke}%
\end{pgfscope}%
\begin{pgfscope}%
\pgfpathrectangle{\pgfqpoint{0.765000in}{0.660000in}}{\pgfqpoint{4.620000in}{4.620000in}}%
\pgfusepath{clip}%
\pgfsetbuttcap%
\pgfsetroundjoin%
\pgfsetlinewidth{1.204500pt}%
\definecolor{currentstroke}{rgb}{0.827451,0.827451,0.827451}%
\pgfsetstrokecolor{currentstroke}%
\pgfsetdash{{1.200000pt}{1.980000pt}}{0.000000pt}%
\pgfpathmoveto{\pgfqpoint{3.310017in}{3.461366in}}%
\pgfpathlineto{\pgfqpoint{4.450489in}{5.128149in}}%
\pgfusepath{stroke}%
\end{pgfscope}%
\begin{pgfscope}%
\pgfpathrectangle{\pgfqpoint{0.765000in}{0.660000in}}{\pgfqpoint{4.620000in}{4.620000in}}%
\pgfusepath{clip}%
\pgfsetrectcap%
\pgfsetroundjoin%
\pgfsetlinewidth{1.204500pt}%
\definecolor{currentstroke}{rgb}{0.000000,0.000000,0.000000}%
\pgfsetstrokecolor{currentstroke}%
\pgfsetdash{}{0pt}%
\pgfpathmoveto{\pgfqpoint{2.815582in}{2.599184in}}%
\pgfusepath{stroke}%
\end{pgfscope}%
\begin{pgfscope}%
\pgfpathrectangle{\pgfqpoint{0.765000in}{0.660000in}}{\pgfqpoint{4.620000in}{4.620000in}}%
\pgfusepath{clip}%
\pgfsetbuttcap%
\pgfsetroundjoin%
\definecolor{currentfill}{rgb}{0.000000,0.000000,0.000000}%
\pgfsetfillcolor{currentfill}%
\pgfsetlinewidth{1.003750pt}%
\definecolor{currentstroke}{rgb}{0.000000,0.000000,0.000000}%
\pgfsetstrokecolor{currentstroke}%
\pgfsetdash{}{0pt}%
\pgfsys@defobject{currentmarker}{\pgfqpoint{-0.016667in}{-0.016667in}}{\pgfqpoint{0.016667in}{0.016667in}}{%
\pgfpathmoveto{\pgfqpoint{0.000000in}{-0.016667in}}%
\pgfpathcurveto{\pgfqpoint{0.004420in}{-0.016667in}}{\pgfqpoint{0.008660in}{-0.014911in}}{\pgfqpoint{0.011785in}{-0.011785in}}%
\pgfpathcurveto{\pgfqpoint{0.014911in}{-0.008660in}}{\pgfqpoint{0.016667in}{-0.004420in}}{\pgfqpoint{0.016667in}{0.000000in}}%
\pgfpathcurveto{\pgfqpoint{0.016667in}{0.004420in}}{\pgfqpoint{0.014911in}{0.008660in}}{\pgfqpoint{0.011785in}{0.011785in}}%
\pgfpathcurveto{\pgfqpoint{0.008660in}{0.014911in}}{\pgfqpoint{0.004420in}{0.016667in}}{\pgfqpoint{0.000000in}{0.016667in}}%
\pgfpathcurveto{\pgfqpoint{-0.004420in}{0.016667in}}{\pgfqpoint{-0.008660in}{0.014911in}}{\pgfqpoint{-0.011785in}{0.011785in}}%
\pgfpathcurveto{\pgfqpoint{-0.014911in}{0.008660in}}{\pgfqpoint{-0.016667in}{0.004420in}}{\pgfqpoint{-0.016667in}{0.000000in}}%
\pgfpathcurveto{\pgfqpoint{-0.016667in}{-0.004420in}}{\pgfqpoint{-0.014911in}{-0.008660in}}{\pgfqpoint{-0.011785in}{-0.011785in}}%
\pgfpathcurveto{\pgfqpoint{-0.008660in}{-0.014911in}}{\pgfqpoint{-0.004420in}{-0.016667in}}{\pgfqpoint{0.000000in}{-0.016667in}}%
\pgfpathlineto{\pgfqpoint{0.000000in}{-0.016667in}}%
\pgfpathclose%
\pgfusepath{stroke,fill}%
}%
\begin{pgfscope}%
\pgfsys@transformshift{2.815582in}{2.599184in}%
\pgfsys@useobject{currentmarker}{}%
\end{pgfscope}%
\end{pgfscope}%
\begin{pgfscope}%
\pgfpathrectangle{\pgfqpoint{0.765000in}{0.660000in}}{\pgfqpoint{4.620000in}{4.620000in}}%
\pgfusepath{clip}%
\pgfsetrectcap%
\pgfsetroundjoin%
\pgfsetlinewidth{1.204500pt}%
\definecolor{currentstroke}{rgb}{0.000000,0.000000,0.000000}%
\pgfsetstrokecolor{currentstroke}%
\pgfsetdash{}{0pt}%
\pgfpathmoveto{\pgfqpoint{2.815582in}{2.599184in}}%
\pgfpathlineto{\pgfqpoint{2.915238in}{2.538405in}}%
\pgfusepath{stroke}%
\end{pgfscope}%
\begin{pgfscope}%
\pgfpathrectangle{\pgfqpoint{0.765000in}{0.660000in}}{\pgfqpoint{4.620000in}{4.620000in}}%
\pgfusepath{clip}%
\pgfsetrectcap%
\pgfsetroundjoin%
\pgfsetlinewidth{1.204500pt}%
\definecolor{currentstroke}{rgb}{0.000000,0.000000,0.000000}%
\pgfsetstrokecolor{currentstroke}%
\pgfsetdash{}{0pt}%
\pgfpathmoveto{\pgfqpoint{2.815582in}{2.599184in}}%
\pgfpathlineto{\pgfqpoint{2.688236in}{2.773437in}}%
\pgfusepath{stroke}%
\end{pgfscope}%
\begin{pgfscope}%
\pgfpathrectangle{\pgfqpoint{0.765000in}{0.660000in}}{\pgfqpoint{4.620000in}{4.620000in}}%
\pgfusepath{clip}%
\pgfsetbuttcap%
\pgfsetroundjoin%
\pgfsetlinewidth{1.204500pt}%
\definecolor{currentstroke}{rgb}{0.827451,0.827451,0.827451}%
\pgfsetstrokecolor{currentstroke}%
\pgfsetdash{{1.200000pt}{1.980000pt}}{0.000000pt}%
\pgfpathmoveto{\pgfqpoint{2.815582in}{2.599184in}}%
\pgfpathlineto{\pgfqpoint{1.033110in}{0.650000in}}%
\pgfusepath{stroke}%
\end{pgfscope}%
\begin{pgfscope}%
\pgfpathrectangle{\pgfqpoint{0.765000in}{0.660000in}}{\pgfqpoint{4.620000in}{4.620000in}}%
\pgfusepath{clip}%
\pgfsetrectcap%
\pgfsetroundjoin%
\pgfsetlinewidth{1.204500pt}%
\definecolor{currentstroke}{rgb}{0.000000,0.000000,0.000000}%
\pgfsetstrokecolor{currentstroke}%
\pgfsetdash{}{0pt}%
\pgfpathmoveto{\pgfqpoint{2.915238in}{2.538405in}}%
\pgfusepath{stroke}%
\end{pgfscope}%
\begin{pgfscope}%
\pgfpathrectangle{\pgfqpoint{0.765000in}{0.660000in}}{\pgfqpoint{4.620000in}{4.620000in}}%
\pgfusepath{clip}%
\pgfsetbuttcap%
\pgfsetroundjoin%
\definecolor{currentfill}{rgb}{0.000000,0.000000,0.000000}%
\pgfsetfillcolor{currentfill}%
\pgfsetlinewidth{1.003750pt}%
\definecolor{currentstroke}{rgb}{0.000000,0.000000,0.000000}%
\pgfsetstrokecolor{currentstroke}%
\pgfsetdash{}{0pt}%
\pgfsys@defobject{currentmarker}{\pgfqpoint{-0.016667in}{-0.016667in}}{\pgfqpoint{0.016667in}{0.016667in}}{%
\pgfpathmoveto{\pgfqpoint{0.000000in}{-0.016667in}}%
\pgfpathcurveto{\pgfqpoint{0.004420in}{-0.016667in}}{\pgfqpoint{0.008660in}{-0.014911in}}{\pgfqpoint{0.011785in}{-0.011785in}}%
\pgfpathcurveto{\pgfqpoint{0.014911in}{-0.008660in}}{\pgfqpoint{0.016667in}{-0.004420in}}{\pgfqpoint{0.016667in}{0.000000in}}%
\pgfpathcurveto{\pgfqpoint{0.016667in}{0.004420in}}{\pgfqpoint{0.014911in}{0.008660in}}{\pgfqpoint{0.011785in}{0.011785in}}%
\pgfpathcurveto{\pgfqpoint{0.008660in}{0.014911in}}{\pgfqpoint{0.004420in}{0.016667in}}{\pgfqpoint{0.000000in}{0.016667in}}%
\pgfpathcurveto{\pgfqpoint{-0.004420in}{0.016667in}}{\pgfqpoint{-0.008660in}{0.014911in}}{\pgfqpoint{-0.011785in}{0.011785in}}%
\pgfpathcurveto{\pgfqpoint{-0.014911in}{0.008660in}}{\pgfqpoint{-0.016667in}{0.004420in}}{\pgfqpoint{-0.016667in}{0.000000in}}%
\pgfpathcurveto{\pgfqpoint{-0.016667in}{-0.004420in}}{\pgfqpoint{-0.014911in}{-0.008660in}}{\pgfqpoint{-0.011785in}{-0.011785in}}%
\pgfpathcurveto{\pgfqpoint{-0.008660in}{-0.014911in}}{\pgfqpoint{-0.004420in}{-0.016667in}}{\pgfqpoint{0.000000in}{-0.016667in}}%
\pgfpathlineto{\pgfqpoint{0.000000in}{-0.016667in}}%
\pgfpathclose%
\pgfusepath{stroke,fill}%
}%
\begin{pgfscope}%
\pgfsys@transformshift{2.915238in}{2.538405in}%
\pgfsys@useobject{currentmarker}{}%
\end{pgfscope}%
\end{pgfscope}%
\begin{pgfscope}%
\pgfpathrectangle{\pgfqpoint{0.765000in}{0.660000in}}{\pgfqpoint{4.620000in}{4.620000in}}%
\pgfusepath{clip}%
\pgfsetrectcap%
\pgfsetroundjoin%
\pgfsetlinewidth{1.204500pt}%
\definecolor{currentstroke}{rgb}{0.000000,0.000000,0.000000}%
\pgfsetstrokecolor{currentstroke}%
\pgfsetdash{}{0pt}%
\pgfpathmoveto{\pgfqpoint{2.915238in}{2.538405in}}%
\pgfpathlineto{\pgfqpoint{2.929883in}{2.530541in}}%
\pgfusepath{stroke}%
\end{pgfscope}%
\begin{pgfscope}%
\pgfpathrectangle{\pgfqpoint{0.765000in}{0.660000in}}{\pgfqpoint{4.620000in}{4.620000in}}%
\pgfusepath{clip}%
\pgfsetrectcap%
\pgfsetroundjoin%
\pgfsetlinewidth{1.204500pt}%
\definecolor{currentstroke}{rgb}{0.000000,0.000000,0.000000}%
\pgfsetstrokecolor{currentstroke}%
\pgfsetdash{}{0pt}%
\pgfpathmoveto{\pgfqpoint{2.915238in}{2.538405in}}%
\pgfpathlineto{\pgfqpoint{2.815582in}{2.599184in}}%
\pgfusepath{stroke}%
\end{pgfscope}%
\begin{pgfscope}%
\pgfpathrectangle{\pgfqpoint{0.765000in}{0.660000in}}{\pgfqpoint{4.620000in}{4.620000in}}%
\pgfusepath{clip}%
\pgfsetbuttcap%
\pgfsetroundjoin%
\pgfsetlinewidth{1.204500pt}%
\definecolor{currentstroke}{rgb}{0.827451,0.827451,0.827451}%
\pgfsetstrokecolor{currentstroke}%
\pgfsetdash{{1.200000pt}{1.980000pt}}{0.000000pt}%
\pgfpathmoveto{\pgfqpoint{2.915238in}{2.538405in}}%
\pgfpathlineto{\pgfqpoint{2.443272in}{1.806464in}}%
\pgfusepath{stroke}%
\end{pgfscope}%
\begin{pgfscope}%
\pgfpathrectangle{\pgfqpoint{0.765000in}{0.660000in}}{\pgfqpoint{4.620000in}{4.620000in}}%
\pgfusepath{clip}%
\pgfsetrectcap%
\pgfsetroundjoin%
\pgfsetlinewidth{1.204500pt}%
\definecolor{currentstroke}{rgb}{0.000000,0.000000,0.000000}%
\pgfsetstrokecolor{currentstroke}%
\pgfsetdash{}{0pt}%
\pgfpathmoveto{\pgfqpoint{2.583171in}{3.110796in}}%
\pgfusepath{stroke}%
\end{pgfscope}%
\begin{pgfscope}%
\pgfpathrectangle{\pgfqpoint{0.765000in}{0.660000in}}{\pgfqpoint{4.620000in}{4.620000in}}%
\pgfusepath{clip}%
\pgfsetbuttcap%
\pgfsetroundjoin%
\definecolor{currentfill}{rgb}{0.000000,0.000000,0.000000}%
\pgfsetfillcolor{currentfill}%
\pgfsetlinewidth{1.003750pt}%
\definecolor{currentstroke}{rgb}{0.000000,0.000000,0.000000}%
\pgfsetstrokecolor{currentstroke}%
\pgfsetdash{}{0pt}%
\pgfsys@defobject{currentmarker}{\pgfqpoint{-0.016667in}{-0.016667in}}{\pgfqpoint{0.016667in}{0.016667in}}{%
\pgfpathmoveto{\pgfqpoint{0.000000in}{-0.016667in}}%
\pgfpathcurveto{\pgfqpoint{0.004420in}{-0.016667in}}{\pgfqpoint{0.008660in}{-0.014911in}}{\pgfqpoint{0.011785in}{-0.011785in}}%
\pgfpathcurveto{\pgfqpoint{0.014911in}{-0.008660in}}{\pgfqpoint{0.016667in}{-0.004420in}}{\pgfqpoint{0.016667in}{0.000000in}}%
\pgfpathcurveto{\pgfqpoint{0.016667in}{0.004420in}}{\pgfqpoint{0.014911in}{0.008660in}}{\pgfqpoint{0.011785in}{0.011785in}}%
\pgfpathcurveto{\pgfqpoint{0.008660in}{0.014911in}}{\pgfqpoint{0.004420in}{0.016667in}}{\pgfqpoint{0.000000in}{0.016667in}}%
\pgfpathcurveto{\pgfqpoint{-0.004420in}{0.016667in}}{\pgfqpoint{-0.008660in}{0.014911in}}{\pgfqpoint{-0.011785in}{0.011785in}}%
\pgfpathcurveto{\pgfqpoint{-0.014911in}{0.008660in}}{\pgfqpoint{-0.016667in}{0.004420in}}{\pgfqpoint{-0.016667in}{0.000000in}}%
\pgfpathcurveto{\pgfqpoint{-0.016667in}{-0.004420in}}{\pgfqpoint{-0.014911in}{-0.008660in}}{\pgfqpoint{-0.011785in}{-0.011785in}}%
\pgfpathcurveto{\pgfqpoint{-0.008660in}{-0.014911in}}{\pgfqpoint{-0.004420in}{-0.016667in}}{\pgfqpoint{0.000000in}{-0.016667in}}%
\pgfpathlineto{\pgfqpoint{0.000000in}{-0.016667in}}%
\pgfpathclose%
\pgfusepath{stroke,fill}%
}%
\begin{pgfscope}%
\pgfsys@transformshift{2.583171in}{3.110796in}%
\pgfsys@useobject{currentmarker}{}%
\end{pgfscope}%
\end{pgfscope}%
\begin{pgfscope}%
\pgfpathrectangle{\pgfqpoint{0.765000in}{0.660000in}}{\pgfqpoint{4.620000in}{4.620000in}}%
\pgfusepath{clip}%
\pgfsetrectcap%
\pgfsetroundjoin%
\pgfsetlinewidth{1.204500pt}%
\definecolor{currentstroke}{rgb}{0.000000,0.000000,0.000000}%
\pgfsetstrokecolor{currentstroke}%
\pgfsetdash{}{0pt}%
\pgfpathmoveto{\pgfqpoint{2.583171in}{3.110796in}}%
\pgfpathlineto{\pgfqpoint{2.586744in}{3.073269in}}%
\pgfusepath{stroke}%
\end{pgfscope}%
\begin{pgfscope}%
\pgfpathrectangle{\pgfqpoint{0.765000in}{0.660000in}}{\pgfqpoint{4.620000in}{4.620000in}}%
\pgfusepath{clip}%
\pgfsetrectcap%
\pgfsetroundjoin%
\pgfsetlinewidth{1.204500pt}%
\definecolor{currentstroke}{rgb}{0.000000,0.000000,0.000000}%
\pgfsetstrokecolor{currentstroke}%
\pgfsetdash{}{0pt}%
\pgfpathmoveto{\pgfqpoint{2.583171in}{3.110796in}}%
\pgfpathlineto{\pgfqpoint{2.582969in}{3.161487in}}%
\pgfusepath{stroke}%
\end{pgfscope}%
\begin{pgfscope}%
\pgfpathrectangle{\pgfqpoint{0.765000in}{0.660000in}}{\pgfqpoint{4.620000in}{4.620000in}}%
\pgfusepath{clip}%
\pgfsetbuttcap%
\pgfsetroundjoin%
\pgfsetlinewidth{1.204500pt}%
\definecolor{currentstroke}{rgb}{0.827451,0.827451,0.827451}%
\pgfsetstrokecolor{currentstroke}%
\pgfsetdash{{1.200000pt}{1.980000pt}}{0.000000pt}%
\pgfpathmoveto{\pgfqpoint{2.583171in}{3.110796in}}%
\pgfpathlineto{\pgfqpoint{1.055089in}{2.756659in}}%
\pgfusepath{stroke}%
\end{pgfscope}%
\begin{pgfscope}%
\pgfpathrectangle{\pgfqpoint{0.765000in}{0.660000in}}{\pgfqpoint{4.620000in}{4.620000in}}%
\pgfusepath{clip}%
\pgfsetrectcap%
\pgfsetroundjoin%
\pgfsetlinewidth{1.204500pt}%
\definecolor{currentstroke}{rgb}{0.000000,0.000000,0.000000}%
\pgfsetstrokecolor{currentstroke}%
\pgfsetdash{}{0pt}%
\pgfpathmoveto{\pgfqpoint{2.688236in}{2.773437in}}%
\pgfusepath{stroke}%
\end{pgfscope}%
\begin{pgfscope}%
\pgfpathrectangle{\pgfqpoint{0.765000in}{0.660000in}}{\pgfqpoint{4.620000in}{4.620000in}}%
\pgfusepath{clip}%
\pgfsetbuttcap%
\pgfsetroundjoin%
\definecolor{currentfill}{rgb}{0.000000,0.000000,0.000000}%
\pgfsetfillcolor{currentfill}%
\pgfsetlinewidth{1.003750pt}%
\definecolor{currentstroke}{rgb}{0.000000,0.000000,0.000000}%
\pgfsetstrokecolor{currentstroke}%
\pgfsetdash{}{0pt}%
\pgfsys@defobject{currentmarker}{\pgfqpoint{-0.016667in}{-0.016667in}}{\pgfqpoint{0.016667in}{0.016667in}}{%
\pgfpathmoveto{\pgfqpoint{0.000000in}{-0.016667in}}%
\pgfpathcurveto{\pgfqpoint{0.004420in}{-0.016667in}}{\pgfqpoint{0.008660in}{-0.014911in}}{\pgfqpoint{0.011785in}{-0.011785in}}%
\pgfpathcurveto{\pgfqpoint{0.014911in}{-0.008660in}}{\pgfqpoint{0.016667in}{-0.004420in}}{\pgfqpoint{0.016667in}{0.000000in}}%
\pgfpathcurveto{\pgfqpoint{0.016667in}{0.004420in}}{\pgfqpoint{0.014911in}{0.008660in}}{\pgfqpoint{0.011785in}{0.011785in}}%
\pgfpathcurveto{\pgfqpoint{0.008660in}{0.014911in}}{\pgfqpoint{0.004420in}{0.016667in}}{\pgfqpoint{0.000000in}{0.016667in}}%
\pgfpathcurveto{\pgfqpoint{-0.004420in}{0.016667in}}{\pgfqpoint{-0.008660in}{0.014911in}}{\pgfqpoint{-0.011785in}{0.011785in}}%
\pgfpathcurveto{\pgfqpoint{-0.014911in}{0.008660in}}{\pgfqpoint{-0.016667in}{0.004420in}}{\pgfqpoint{-0.016667in}{0.000000in}}%
\pgfpathcurveto{\pgfqpoint{-0.016667in}{-0.004420in}}{\pgfqpoint{-0.014911in}{-0.008660in}}{\pgfqpoint{-0.011785in}{-0.011785in}}%
\pgfpathcurveto{\pgfqpoint{-0.008660in}{-0.014911in}}{\pgfqpoint{-0.004420in}{-0.016667in}}{\pgfqpoint{0.000000in}{-0.016667in}}%
\pgfpathlineto{\pgfqpoint{0.000000in}{-0.016667in}}%
\pgfpathclose%
\pgfusepath{stroke,fill}%
}%
\begin{pgfscope}%
\pgfsys@transformshift{2.688236in}{2.773437in}%
\pgfsys@useobject{currentmarker}{}%
\end{pgfscope}%
\end{pgfscope}%
\begin{pgfscope}%
\pgfpathrectangle{\pgfqpoint{0.765000in}{0.660000in}}{\pgfqpoint{4.620000in}{4.620000in}}%
\pgfusepath{clip}%
\pgfsetrectcap%
\pgfsetroundjoin%
\pgfsetlinewidth{1.204500pt}%
\definecolor{currentstroke}{rgb}{0.000000,0.000000,0.000000}%
\pgfsetstrokecolor{currentstroke}%
\pgfsetdash{}{0pt}%
\pgfpathmoveto{\pgfqpoint{2.688236in}{2.773437in}}%
\pgfpathlineto{\pgfqpoint{2.672370in}{2.810270in}}%
\pgfusepath{stroke}%
\end{pgfscope}%
\begin{pgfscope}%
\pgfpathrectangle{\pgfqpoint{0.765000in}{0.660000in}}{\pgfqpoint{4.620000in}{4.620000in}}%
\pgfusepath{clip}%
\pgfsetrectcap%
\pgfsetroundjoin%
\pgfsetlinewidth{1.204500pt}%
\definecolor{currentstroke}{rgb}{0.000000,0.000000,0.000000}%
\pgfsetstrokecolor{currentstroke}%
\pgfsetdash{}{0pt}%
\pgfpathmoveto{\pgfqpoint{2.688236in}{2.773437in}}%
\pgfpathlineto{\pgfqpoint{2.815582in}{2.599184in}}%
\pgfusepath{stroke}%
\end{pgfscope}%
\begin{pgfscope}%
\pgfpathrectangle{\pgfqpoint{0.765000in}{0.660000in}}{\pgfqpoint{4.620000in}{4.620000in}}%
\pgfusepath{clip}%
\pgfsetbuttcap%
\pgfsetroundjoin%
\pgfsetlinewidth{1.204500pt}%
\definecolor{currentstroke}{rgb}{0.827451,0.827451,0.827451}%
\pgfsetstrokecolor{currentstroke}%
\pgfsetdash{{1.200000pt}{1.980000pt}}{0.000000pt}%
\pgfpathmoveto{\pgfqpoint{2.688236in}{2.773437in}}%
\pgfpathlineto{\pgfqpoint{0.755000in}{1.458767in}}%
\pgfusepath{stroke}%
\end{pgfscope}%
\begin{pgfscope}%
\pgfpathrectangle{\pgfqpoint{0.765000in}{0.660000in}}{\pgfqpoint{4.620000in}{4.620000in}}%
\pgfusepath{clip}%
\pgfsetrectcap%
\pgfsetroundjoin%
\pgfsetlinewidth{1.204500pt}%
\definecolor{currentstroke}{rgb}{0.000000,0.000000,0.000000}%
\pgfsetstrokecolor{currentstroke}%
\pgfsetdash{}{0pt}%
\pgfpathmoveto{\pgfqpoint{3.572472in}{2.854798in}}%
\pgfusepath{stroke}%
\end{pgfscope}%
\begin{pgfscope}%
\pgfpathrectangle{\pgfqpoint{0.765000in}{0.660000in}}{\pgfqpoint{4.620000in}{4.620000in}}%
\pgfusepath{clip}%
\pgfsetbuttcap%
\pgfsetroundjoin%
\definecolor{currentfill}{rgb}{0.000000,0.000000,0.000000}%
\pgfsetfillcolor{currentfill}%
\pgfsetlinewidth{1.003750pt}%
\definecolor{currentstroke}{rgb}{0.000000,0.000000,0.000000}%
\pgfsetstrokecolor{currentstroke}%
\pgfsetdash{}{0pt}%
\pgfsys@defobject{currentmarker}{\pgfqpoint{-0.016667in}{-0.016667in}}{\pgfqpoint{0.016667in}{0.016667in}}{%
\pgfpathmoveto{\pgfqpoint{0.000000in}{-0.016667in}}%
\pgfpathcurveto{\pgfqpoint{0.004420in}{-0.016667in}}{\pgfqpoint{0.008660in}{-0.014911in}}{\pgfqpoint{0.011785in}{-0.011785in}}%
\pgfpathcurveto{\pgfqpoint{0.014911in}{-0.008660in}}{\pgfqpoint{0.016667in}{-0.004420in}}{\pgfqpoint{0.016667in}{0.000000in}}%
\pgfpathcurveto{\pgfqpoint{0.016667in}{0.004420in}}{\pgfqpoint{0.014911in}{0.008660in}}{\pgfqpoint{0.011785in}{0.011785in}}%
\pgfpathcurveto{\pgfqpoint{0.008660in}{0.014911in}}{\pgfqpoint{0.004420in}{0.016667in}}{\pgfqpoint{0.000000in}{0.016667in}}%
\pgfpathcurveto{\pgfqpoint{-0.004420in}{0.016667in}}{\pgfqpoint{-0.008660in}{0.014911in}}{\pgfqpoint{-0.011785in}{0.011785in}}%
\pgfpathcurveto{\pgfqpoint{-0.014911in}{0.008660in}}{\pgfqpoint{-0.016667in}{0.004420in}}{\pgfqpoint{-0.016667in}{0.000000in}}%
\pgfpathcurveto{\pgfqpoint{-0.016667in}{-0.004420in}}{\pgfqpoint{-0.014911in}{-0.008660in}}{\pgfqpoint{-0.011785in}{-0.011785in}}%
\pgfpathcurveto{\pgfqpoint{-0.008660in}{-0.014911in}}{\pgfqpoint{-0.004420in}{-0.016667in}}{\pgfqpoint{0.000000in}{-0.016667in}}%
\pgfpathlineto{\pgfqpoint{0.000000in}{-0.016667in}}%
\pgfpathclose%
\pgfusepath{stroke,fill}%
}%
\begin{pgfscope}%
\pgfsys@transformshift{3.572472in}{2.854798in}%
\pgfsys@useobject{currentmarker}{}%
\end{pgfscope}%
\end{pgfscope}%
\begin{pgfscope}%
\pgfpathrectangle{\pgfqpoint{0.765000in}{0.660000in}}{\pgfqpoint{4.620000in}{4.620000in}}%
\pgfusepath{clip}%
\pgfsetrectcap%
\pgfsetroundjoin%
\pgfsetlinewidth{1.204500pt}%
\definecolor{currentstroke}{rgb}{0.000000,0.000000,0.000000}%
\pgfsetstrokecolor{currentstroke}%
\pgfsetdash{}{0pt}%
\pgfpathmoveto{\pgfqpoint{3.572472in}{2.854798in}}%
\pgfpathlineto{\pgfqpoint{3.575586in}{2.945282in}}%
\pgfusepath{stroke}%
\end{pgfscope}%
\begin{pgfscope}%
\pgfpathrectangle{\pgfqpoint{0.765000in}{0.660000in}}{\pgfqpoint{4.620000in}{4.620000in}}%
\pgfusepath{clip}%
\pgfsetrectcap%
\pgfsetroundjoin%
\pgfsetlinewidth{1.204500pt}%
\definecolor{currentstroke}{rgb}{0.000000,0.000000,0.000000}%
\pgfsetstrokecolor{currentstroke}%
\pgfsetdash{}{0pt}%
\pgfpathmoveto{\pgfqpoint{3.572472in}{2.854798in}}%
\pgfpathlineto{\pgfqpoint{3.563846in}{2.808203in}}%
\pgfusepath{stroke}%
\end{pgfscope}%
\begin{pgfscope}%
\pgfpathrectangle{\pgfqpoint{0.765000in}{0.660000in}}{\pgfqpoint{4.620000in}{4.620000in}}%
\pgfusepath{clip}%
\pgfsetbuttcap%
\pgfsetroundjoin%
\pgfsetlinewidth{1.204500pt}%
\definecolor{currentstroke}{rgb}{0.827451,0.827451,0.827451}%
\pgfsetstrokecolor{currentstroke}%
\pgfsetdash{{1.200000pt}{1.980000pt}}{0.000000pt}%
\pgfpathmoveto{\pgfqpoint{3.572472in}{2.854798in}}%
\pgfpathlineto{\pgfqpoint{5.395000in}{2.265723in}}%
\pgfusepath{stroke}%
\end{pgfscope}%
\begin{pgfscope}%
\pgfpathrectangle{\pgfqpoint{0.765000in}{0.660000in}}{\pgfqpoint{4.620000in}{4.620000in}}%
\pgfusepath{clip}%
\pgfsetrectcap%
\pgfsetroundjoin%
\pgfsetlinewidth{1.204500pt}%
\definecolor{currentstroke}{rgb}{0.000000,0.000000,0.000000}%
\pgfsetstrokecolor{currentstroke}%
\pgfsetdash{}{0pt}%
\pgfpathmoveto{\pgfqpoint{2.651156in}{3.305927in}}%
\pgfusepath{stroke}%
\end{pgfscope}%
\begin{pgfscope}%
\pgfpathrectangle{\pgfqpoint{0.765000in}{0.660000in}}{\pgfqpoint{4.620000in}{4.620000in}}%
\pgfusepath{clip}%
\pgfsetbuttcap%
\pgfsetroundjoin%
\definecolor{currentfill}{rgb}{0.000000,0.000000,0.000000}%
\pgfsetfillcolor{currentfill}%
\pgfsetlinewidth{1.003750pt}%
\definecolor{currentstroke}{rgb}{0.000000,0.000000,0.000000}%
\pgfsetstrokecolor{currentstroke}%
\pgfsetdash{}{0pt}%
\pgfsys@defobject{currentmarker}{\pgfqpoint{-0.016667in}{-0.016667in}}{\pgfqpoint{0.016667in}{0.016667in}}{%
\pgfpathmoveto{\pgfqpoint{0.000000in}{-0.016667in}}%
\pgfpathcurveto{\pgfqpoint{0.004420in}{-0.016667in}}{\pgfqpoint{0.008660in}{-0.014911in}}{\pgfqpoint{0.011785in}{-0.011785in}}%
\pgfpathcurveto{\pgfqpoint{0.014911in}{-0.008660in}}{\pgfqpoint{0.016667in}{-0.004420in}}{\pgfqpoint{0.016667in}{0.000000in}}%
\pgfpathcurveto{\pgfqpoint{0.016667in}{0.004420in}}{\pgfqpoint{0.014911in}{0.008660in}}{\pgfqpoint{0.011785in}{0.011785in}}%
\pgfpathcurveto{\pgfqpoint{0.008660in}{0.014911in}}{\pgfqpoint{0.004420in}{0.016667in}}{\pgfqpoint{0.000000in}{0.016667in}}%
\pgfpathcurveto{\pgfqpoint{-0.004420in}{0.016667in}}{\pgfqpoint{-0.008660in}{0.014911in}}{\pgfqpoint{-0.011785in}{0.011785in}}%
\pgfpathcurveto{\pgfqpoint{-0.014911in}{0.008660in}}{\pgfqpoint{-0.016667in}{0.004420in}}{\pgfqpoint{-0.016667in}{0.000000in}}%
\pgfpathcurveto{\pgfqpoint{-0.016667in}{-0.004420in}}{\pgfqpoint{-0.014911in}{-0.008660in}}{\pgfqpoint{-0.011785in}{-0.011785in}}%
\pgfpathcurveto{\pgfqpoint{-0.008660in}{-0.014911in}}{\pgfqpoint{-0.004420in}{-0.016667in}}{\pgfqpoint{0.000000in}{-0.016667in}}%
\pgfpathlineto{\pgfqpoint{0.000000in}{-0.016667in}}%
\pgfpathclose%
\pgfusepath{stroke,fill}%
}%
\begin{pgfscope}%
\pgfsys@transformshift{2.651156in}{3.305927in}%
\pgfsys@useobject{currentmarker}{}%
\end{pgfscope}%
\end{pgfscope}%
\begin{pgfscope}%
\pgfpathrectangle{\pgfqpoint{0.765000in}{0.660000in}}{\pgfqpoint{4.620000in}{4.620000in}}%
\pgfusepath{clip}%
\pgfsetrectcap%
\pgfsetroundjoin%
\pgfsetlinewidth{1.204500pt}%
\definecolor{currentstroke}{rgb}{0.000000,0.000000,0.000000}%
\pgfsetstrokecolor{currentstroke}%
\pgfsetdash{}{0pt}%
\pgfpathmoveto{\pgfqpoint{2.651156in}{3.305927in}}%
\pgfpathlineto{\pgfqpoint{2.642560in}{3.294969in}}%
\pgfusepath{stroke}%
\end{pgfscope}%
\begin{pgfscope}%
\pgfpathrectangle{\pgfqpoint{0.765000in}{0.660000in}}{\pgfqpoint{4.620000in}{4.620000in}}%
\pgfusepath{clip}%
\pgfsetrectcap%
\pgfsetroundjoin%
\pgfsetlinewidth{1.204500pt}%
\definecolor{currentstroke}{rgb}{0.000000,0.000000,0.000000}%
\pgfsetstrokecolor{currentstroke}%
\pgfsetdash{}{0pt}%
\pgfpathmoveto{\pgfqpoint{2.651156in}{3.305927in}}%
\pgfpathlineto{\pgfqpoint{2.671300in}{3.327008in}}%
\pgfusepath{stroke}%
\end{pgfscope}%
\begin{pgfscope}%
\pgfpathrectangle{\pgfqpoint{0.765000in}{0.660000in}}{\pgfqpoint{4.620000in}{4.620000in}}%
\pgfusepath{clip}%
\pgfsetbuttcap%
\pgfsetroundjoin%
\pgfsetlinewidth{1.204500pt}%
\definecolor{currentstroke}{rgb}{0.827451,0.827451,0.827451}%
\pgfsetstrokecolor{currentstroke}%
\pgfsetdash{{1.200000pt}{1.980000pt}}{0.000000pt}%
\pgfpathmoveto{\pgfqpoint{2.651156in}{3.305927in}}%
\pgfpathlineto{\pgfqpoint{0.825817in}{4.213917in}}%
\pgfusepath{stroke}%
\end{pgfscope}%
\begin{pgfscope}%
\pgfpathrectangle{\pgfqpoint{0.765000in}{0.660000in}}{\pgfqpoint{4.620000in}{4.620000in}}%
\pgfusepath{clip}%
\pgfsetrectcap%
\pgfsetroundjoin%
\pgfsetlinewidth{1.204500pt}%
\definecolor{currentstroke}{rgb}{0.000000,0.000000,0.000000}%
\pgfsetstrokecolor{currentstroke}%
\pgfsetdash{}{0pt}%
\pgfpathmoveto{\pgfqpoint{2.642560in}{3.294969in}}%
\pgfusepath{stroke}%
\end{pgfscope}%
\begin{pgfscope}%
\pgfpathrectangle{\pgfqpoint{0.765000in}{0.660000in}}{\pgfqpoint{4.620000in}{4.620000in}}%
\pgfusepath{clip}%
\pgfsetbuttcap%
\pgfsetroundjoin%
\definecolor{currentfill}{rgb}{0.000000,0.000000,0.000000}%
\pgfsetfillcolor{currentfill}%
\pgfsetlinewidth{1.003750pt}%
\definecolor{currentstroke}{rgb}{0.000000,0.000000,0.000000}%
\pgfsetstrokecolor{currentstroke}%
\pgfsetdash{}{0pt}%
\pgfsys@defobject{currentmarker}{\pgfqpoint{-0.016667in}{-0.016667in}}{\pgfqpoint{0.016667in}{0.016667in}}{%
\pgfpathmoveto{\pgfqpoint{0.000000in}{-0.016667in}}%
\pgfpathcurveto{\pgfqpoint{0.004420in}{-0.016667in}}{\pgfqpoint{0.008660in}{-0.014911in}}{\pgfqpoint{0.011785in}{-0.011785in}}%
\pgfpathcurveto{\pgfqpoint{0.014911in}{-0.008660in}}{\pgfqpoint{0.016667in}{-0.004420in}}{\pgfqpoint{0.016667in}{0.000000in}}%
\pgfpathcurveto{\pgfqpoint{0.016667in}{0.004420in}}{\pgfqpoint{0.014911in}{0.008660in}}{\pgfqpoint{0.011785in}{0.011785in}}%
\pgfpathcurveto{\pgfqpoint{0.008660in}{0.014911in}}{\pgfqpoint{0.004420in}{0.016667in}}{\pgfqpoint{0.000000in}{0.016667in}}%
\pgfpathcurveto{\pgfqpoint{-0.004420in}{0.016667in}}{\pgfqpoint{-0.008660in}{0.014911in}}{\pgfqpoint{-0.011785in}{0.011785in}}%
\pgfpathcurveto{\pgfqpoint{-0.014911in}{0.008660in}}{\pgfqpoint{-0.016667in}{0.004420in}}{\pgfqpoint{-0.016667in}{0.000000in}}%
\pgfpathcurveto{\pgfqpoint{-0.016667in}{-0.004420in}}{\pgfqpoint{-0.014911in}{-0.008660in}}{\pgfqpoint{-0.011785in}{-0.011785in}}%
\pgfpathcurveto{\pgfqpoint{-0.008660in}{-0.014911in}}{\pgfqpoint{-0.004420in}{-0.016667in}}{\pgfqpoint{0.000000in}{-0.016667in}}%
\pgfpathlineto{\pgfqpoint{0.000000in}{-0.016667in}}%
\pgfpathclose%
\pgfusepath{stroke,fill}%
}%
\begin{pgfscope}%
\pgfsys@transformshift{2.642560in}{3.294969in}%
\pgfsys@useobject{currentmarker}{}%
\end{pgfscope}%
\end{pgfscope}%
\begin{pgfscope}%
\pgfpathrectangle{\pgfqpoint{0.765000in}{0.660000in}}{\pgfqpoint{4.620000in}{4.620000in}}%
\pgfusepath{clip}%
\pgfsetrectcap%
\pgfsetroundjoin%
\pgfsetlinewidth{1.204500pt}%
\definecolor{currentstroke}{rgb}{0.000000,0.000000,0.000000}%
\pgfsetstrokecolor{currentstroke}%
\pgfsetdash{}{0pt}%
\pgfpathmoveto{\pgfqpoint{2.642560in}{3.294969in}}%
\pgfpathlineto{\pgfqpoint{2.635438in}{3.285868in}}%
\pgfusepath{stroke}%
\end{pgfscope}%
\begin{pgfscope}%
\pgfpathrectangle{\pgfqpoint{0.765000in}{0.660000in}}{\pgfqpoint{4.620000in}{4.620000in}}%
\pgfusepath{clip}%
\pgfsetrectcap%
\pgfsetroundjoin%
\pgfsetlinewidth{1.204500pt}%
\definecolor{currentstroke}{rgb}{0.000000,0.000000,0.000000}%
\pgfsetstrokecolor{currentstroke}%
\pgfsetdash{}{0pt}%
\pgfpathmoveto{\pgfqpoint{2.642560in}{3.294969in}}%
\pgfpathlineto{\pgfqpoint{2.651156in}{3.305927in}}%
\pgfusepath{stroke}%
\end{pgfscope}%
\begin{pgfscope}%
\pgfpathrectangle{\pgfqpoint{0.765000in}{0.660000in}}{\pgfqpoint{4.620000in}{4.620000in}}%
\pgfusepath{clip}%
\pgfsetbuttcap%
\pgfsetroundjoin%
\pgfsetlinewidth{1.204500pt}%
\definecolor{currentstroke}{rgb}{0.827451,0.827451,0.827451}%
\pgfsetstrokecolor{currentstroke}%
\pgfsetdash{{1.200000pt}{1.980000pt}}{0.000000pt}%
\pgfpathmoveto{\pgfqpoint{2.642560in}{3.294969in}}%
\pgfpathlineto{\pgfqpoint{2.620333in}{3.304696in}}%
\pgfusepath{stroke}%
\end{pgfscope}%
\begin{pgfscope}%
\pgfpathrectangle{\pgfqpoint{0.765000in}{0.660000in}}{\pgfqpoint{4.620000in}{4.620000in}}%
\pgfusepath{clip}%
\pgfsetrectcap%
\pgfsetroundjoin%
\pgfsetlinewidth{1.204500pt}%
\definecolor{currentstroke}{rgb}{0.000000,0.000000,0.000000}%
\pgfsetstrokecolor{currentstroke}%
\pgfsetdash{}{0pt}%
\pgfpathmoveto{\pgfqpoint{2.913379in}{3.479681in}}%
\pgfusepath{stroke}%
\end{pgfscope}%
\begin{pgfscope}%
\pgfpathrectangle{\pgfqpoint{0.765000in}{0.660000in}}{\pgfqpoint{4.620000in}{4.620000in}}%
\pgfusepath{clip}%
\pgfsetbuttcap%
\pgfsetroundjoin%
\definecolor{currentfill}{rgb}{0.000000,0.000000,0.000000}%
\pgfsetfillcolor{currentfill}%
\pgfsetlinewidth{1.003750pt}%
\definecolor{currentstroke}{rgb}{0.000000,0.000000,0.000000}%
\pgfsetstrokecolor{currentstroke}%
\pgfsetdash{}{0pt}%
\pgfsys@defobject{currentmarker}{\pgfqpoint{-0.016667in}{-0.016667in}}{\pgfqpoint{0.016667in}{0.016667in}}{%
\pgfpathmoveto{\pgfqpoint{0.000000in}{-0.016667in}}%
\pgfpathcurveto{\pgfqpoint{0.004420in}{-0.016667in}}{\pgfqpoint{0.008660in}{-0.014911in}}{\pgfqpoint{0.011785in}{-0.011785in}}%
\pgfpathcurveto{\pgfqpoint{0.014911in}{-0.008660in}}{\pgfqpoint{0.016667in}{-0.004420in}}{\pgfqpoint{0.016667in}{0.000000in}}%
\pgfpathcurveto{\pgfqpoint{0.016667in}{0.004420in}}{\pgfqpoint{0.014911in}{0.008660in}}{\pgfqpoint{0.011785in}{0.011785in}}%
\pgfpathcurveto{\pgfqpoint{0.008660in}{0.014911in}}{\pgfqpoint{0.004420in}{0.016667in}}{\pgfqpoint{0.000000in}{0.016667in}}%
\pgfpathcurveto{\pgfqpoint{-0.004420in}{0.016667in}}{\pgfqpoint{-0.008660in}{0.014911in}}{\pgfqpoint{-0.011785in}{0.011785in}}%
\pgfpathcurveto{\pgfqpoint{-0.014911in}{0.008660in}}{\pgfqpoint{-0.016667in}{0.004420in}}{\pgfqpoint{-0.016667in}{0.000000in}}%
\pgfpathcurveto{\pgfqpoint{-0.016667in}{-0.004420in}}{\pgfqpoint{-0.014911in}{-0.008660in}}{\pgfqpoint{-0.011785in}{-0.011785in}}%
\pgfpathcurveto{\pgfqpoint{-0.008660in}{-0.014911in}}{\pgfqpoint{-0.004420in}{-0.016667in}}{\pgfqpoint{0.000000in}{-0.016667in}}%
\pgfpathlineto{\pgfqpoint{0.000000in}{-0.016667in}}%
\pgfpathclose%
\pgfusepath{stroke,fill}%
}%
\begin{pgfscope}%
\pgfsys@transformshift{2.913379in}{3.479681in}%
\pgfsys@useobject{currentmarker}{}%
\end{pgfscope}%
\end{pgfscope}%
\begin{pgfscope}%
\pgfpathrectangle{\pgfqpoint{0.765000in}{0.660000in}}{\pgfqpoint{4.620000in}{4.620000in}}%
\pgfusepath{clip}%
\pgfsetrectcap%
\pgfsetroundjoin%
\pgfsetlinewidth{1.204500pt}%
\definecolor{currentstroke}{rgb}{0.000000,0.000000,0.000000}%
\pgfsetstrokecolor{currentstroke}%
\pgfsetdash{}{0pt}%
\pgfpathmoveto{\pgfqpoint{2.913379in}{3.479681in}}%
\pgfpathlineto{\pgfqpoint{2.793000in}{3.409870in}}%
\pgfusepath{stroke}%
\end{pgfscope}%
\begin{pgfscope}%
\pgfpathrectangle{\pgfqpoint{0.765000in}{0.660000in}}{\pgfqpoint{4.620000in}{4.620000in}}%
\pgfusepath{clip}%
\pgfsetrectcap%
\pgfsetroundjoin%
\pgfsetlinewidth{1.204500pt}%
\definecolor{currentstroke}{rgb}{0.000000,0.000000,0.000000}%
\pgfsetstrokecolor{currentstroke}%
\pgfsetdash{}{0pt}%
\pgfpathmoveto{\pgfqpoint{2.913379in}{3.479681in}}%
\pgfpathlineto{\pgfqpoint{2.982382in}{3.516877in}}%
\pgfusepath{stroke}%
\end{pgfscope}%
\begin{pgfscope}%
\pgfpathrectangle{\pgfqpoint{0.765000in}{0.660000in}}{\pgfqpoint{4.620000in}{4.620000in}}%
\pgfusepath{clip}%
\pgfsetbuttcap%
\pgfsetroundjoin%
\pgfsetlinewidth{1.204500pt}%
\definecolor{currentstroke}{rgb}{0.827451,0.827451,0.827451}%
\pgfsetstrokecolor{currentstroke}%
\pgfsetdash{{1.200000pt}{1.980000pt}}{0.000000pt}%
\pgfpathmoveto{\pgfqpoint{2.913379in}{3.479681in}}%
\pgfpathlineto{\pgfqpoint{2.430942in}{3.998803in}}%
\pgfusepath{stroke}%
\end{pgfscope}%
\begin{pgfscope}%
\pgfpathrectangle{\pgfqpoint{0.765000in}{0.660000in}}{\pgfqpoint{4.620000in}{4.620000in}}%
\pgfusepath{clip}%
\pgfsetrectcap%
\pgfsetroundjoin%
\pgfsetlinewidth{1.204500pt}%
\definecolor{currentstroke}{rgb}{0.000000,0.000000,0.000000}%
\pgfsetstrokecolor{currentstroke}%
\pgfsetdash{}{0pt}%
\pgfpathmoveto{\pgfqpoint{3.107502in}{3.529134in}}%
\pgfusepath{stroke}%
\end{pgfscope}%
\begin{pgfscope}%
\pgfpathrectangle{\pgfqpoint{0.765000in}{0.660000in}}{\pgfqpoint{4.620000in}{4.620000in}}%
\pgfusepath{clip}%
\pgfsetbuttcap%
\pgfsetroundjoin%
\definecolor{currentfill}{rgb}{0.000000,0.000000,0.000000}%
\pgfsetfillcolor{currentfill}%
\pgfsetlinewidth{1.003750pt}%
\definecolor{currentstroke}{rgb}{0.000000,0.000000,0.000000}%
\pgfsetstrokecolor{currentstroke}%
\pgfsetdash{}{0pt}%
\pgfsys@defobject{currentmarker}{\pgfqpoint{-0.016667in}{-0.016667in}}{\pgfqpoint{0.016667in}{0.016667in}}{%
\pgfpathmoveto{\pgfqpoint{0.000000in}{-0.016667in}}%
\pgfpathcurveto{\pgfqpoint{0.004420in}{-0.016667in}}{\pgfqpoint{0.008660in}{-0.014911in}}{\pgfqpoint{0.011785in}{-0.011785in}}%
\pgfpathcurveto{\pgfqpoint{0.014911in}{-0.008660in}}{\pgfqpoint{0.016667in}{-0.004420in}}{\pgfqpoint{0.016667in}{0.000000in}}%
\pgfpathcurveto{\pgfqpoint{0.016667in}{0.004420in}}{\pgfqpoint{0.014911in}{0.008660in}}{\pgfqpoint{0.011785in}{0.011785in}}%
\pgfpathcurveto{\pgfqpoint{0.008660in}{0.014911in}}{\pgfqpoint{0.004420in}{0.016667in}}{\pgfqpoint{0.000000in}{0.016667in}}%
\pgfpathcurveto{\pgfqpoint{-0.004420in}{0.016667in}}{\pgfqpoint{-0.008660in}{0.014911in}}{\pgfqpoint{-0.011785in}{0.011785in}}%
\pgfpathcurveto{\pgfqpoint{-0.014911in}{0.008660in}}{\pgfqpoint{-0.016667in}{0.004420in}}{\pgfqpoint{-0.016667in}{0.000000in}}%
\pgfpathcurveto{\pgfqpoint{-0.016667in}{-0.004420in}}{\pgfqpoint{-0.014911in}{-0.008660in}}{\pgfqpoint{-0.011785in}{-0.011785in}}%
\pgfpathcurveto{\pgfqpoint{-0.008660in}{-0.014911in}}{\pgfqpoint{-0.004420in}{-0.016667in}}{\pgfqpoint{0.000000in}{-0.016667in}}%
\pgfpathlineto{\pgfqpoint{0.000000in}{-0.016667in}}%
\pgfpathclose%
\pgfusepath{stroke,fill}%
}%
\begin{pgfscope}%
\pgfsys@transformshift{3.107502in}{3.529134in}%
\pgfsys@useobject{currentmarker}{}%
\end{pgfscope}%
\end{pgfscope}%
\begin{pgfscope}%
\pgfpathrectangle{\pgfqpoint{0.765000in}{0.660000in}}{\pgfqpoint{4.620000in}{4.620000in}}%
\pgfusepath{clip}%
\pgfsetrectcap%
\pgfsetroundjoin%
\pgfsetlinewidth{1.204500pt}%
\definecolor{currentstroke}{rgb}{0.000000,0.000000,0.000000}%
\pgfsetstrokecolor{currentstroke}%
\pgfsetdash{}{0pt}%
\pgfpathmoveto{\pgfqpoint{3.107502in}{3.529134in}}%
\pgfpathlineto{\pgfqpoint{3.121227in}{3.527081in}}%
\pgfusepath{stroke}%
\end{pgfscope}%
\begin{pgfscope}%
\pgfpathrectangle{\pgfqpoint{0.765000in}{0.660000in}}{\pgfqpoint{4.620000in}{4.620000in}}%
\pgfusepath{clip}%
\pgfsetrectcap%
\pgfsetroundjoin%
\pgfsetlinewidth{1.204500pt}%
\definecolor{currentstroke}{rgb}{0.000000,0.000000,0.000000}%
\pgfsetstrokecolor{currentstroke}%
\pgfsetdash{}{0pt}%
\pgfpathmoveto{\pgfqpoint{3.107502in}{3.529134in}}%
\pgfpathlineto{\pgfqpoint{3.085349in}{3.530998in}}%
\pgfusepath{stroke}%
\end{pgfscope}%
\begin{pgfscope}%
\pgfpathrectangle{\pgfqpoint{0.765000in}{0.660000in}}{\pgfqpoint{4.620000in}{4.620000in}}%
\pgfusepath{clip}%
\pgfsetbuttcap%
\pgfsetroundjoin%
\pgfsetlinewidth{1.204500pt}%
\definecolor{currentstroke}{rgb}{0.827451,0.827451,0.827451}%
\pgfsetstrokecolor{currentstroke}%
\pgfsetdash{{1.200000pt}{1.980000pt}}{0.000000pt}%
\pgfpathmoveto{\pgfqpoint{3.107502in}{3.529134in}}%
\pgfpathlineto{\pgfqpoint{3.060758in}{5.162711in}}%
\pgfusepath{stroke}%
\end{pgfscope}%
\begin{pgfscope}%
\pgfpathrectangle{\pgfqpoint{0.765000in}{0.660000in}}{\pgfqpoint{4.620000in}{4.620000in}}%
\pgfusepath{clip}%
\pgfsetrectcap%
\pgfsetroundjoin%
\pgfsetlinewidth{1.204500pt}%
\definecolor{currentstroke}{rgb}{0.000000,0.000000,0.000000}%
\pgfsetstrokecolor{currentstroke}%
\pgfsetdash{}{0pt}%
\pgfpathmoveto{\pgfqpoint{3.350280in}{3.424899in}}%
\pgfusepath{stroke}%
\end{pgfscope}%
\begin{pgfscope}%
\pgfpathrectangle{\pgfqpoint{0.765000in}{0.660000in}}{\pgfqpoint{4.620000in}{4.620000in}}%
\pgfusepath{clip}%
\pgfsetbuttcap%
\pgfsetroundjoin%
\definecolor{currentfill}{rgb}{0.000000,0.000000,0.000000}%
\pgfsetfillcolor{currentfill}%
\pgfsetlinewidth{1.003750pt}%
\definecolor{currentstroke}{rgb}{0.000000,0.000000,0.000000}%
\pgfsetstrokecolor{currentstroke}%
\pgfsetdash{}{0pt}%
\pgfsys@defobject{currentmarker}{\pgfqpoint{-0.016667in}{-0.016667in}}{\pgfqpoint{0.016667in}{0.016667in}}{%
\pgfpathmoveto{\pgfqpoint{0.000000in}{-0.016667in}}%
\pgfpathcurveto{\pgfqpoint{0.004420in}{-0.016667in}}{\pgfqpoint{0.008660in}{-0.014911in}}{\pgfqpoint{0.011785in}{-0.011785in}}%
\pgfpathcurveto{\pgfqpoint{0.014911in}{-0.008660in}}{\pgfqpoint{0.016667in}{-0.004420in}}{\pgfqpoint{0.016667in}{0.000000in}}%
\pgfpathcurveto{\pgfqpoint{0.016667in}{0.004420in}}{\pgfqpoint{0.014911in}{0.008660in}}{\pgfqpoint{0.011785in}{0.011785in}}%
\pgfpathcurveto{\pgfqpoint{0.008660in}{0.014911in}}{\pgfqpoint{0.004420in}{0.016667in}}{\pgfqpoint{0.000000in}{0.016667in}}%
\pgfpathcurveto{\pgfqpoint{-0.004420in}{0.016667in}}{\pgfqpoint{-0.008660in}{0.014911in}}{\pgfqpoint{-0.011785in}{0.011785in}}%
\pgfpathcurveto{\pgfqpoint{-0.014911in}{0.008660in}}{\pgfqpoint{-0.016667in}{0.004420in}}{\pgfqpoint{-0.016667in}{0.000000in}}%
\pgfpathcurveto{\pgfqpoint{-0.016667in}{-0.004420in}}{\pgfqpoint{-0.014911in}{-0.008660in}}{\pgfqpoint{-0.011785in}{-0.011785in}}%
\pgfpathcurveto{\pgfqpoint{-0.008660in}{-0.014911in}}{\pgfqpoint{-0.004420in}{-0.016667in}}{\pgfqpoint{0.000000in}{-0.016667in}}%
\pgfpathlineto{\pgfqpoint{0.000000in}{-0.016667in}}%
\pgfpathclose%
\pgfusepath{stroke,fill}%
}%
\begin{pgfscope}%
\pgfsys@transformshift{3.350280in}{3.424899in}%
\pgfsys@useobject{currentmarker}{}%
\end{pgfscope}%
\end{pgfscope}%
\begin{pgfscope}%
\pgfpathrectangle{\pgfqpoint{0.765000in}{0.660000in}}{\pgfqpoint{4.620000in}{4.620000in}}%
\pgfusepath{clip}%
\pgfsetrectcap%
\pgfsetroundjoin%
\pgfsetlinewidth{1.204500pt}%
\definecolor{currentstroke}{rgb}{0.000000,0.000000,0.000000}%
\pgfsetstrokecolor{currentstroke}%
\pgfsetdash{}{0pt}%
\pgfpathmoveto{\pgfqpoint{3.350280in}{3.424899in}}%
\pgfpathlineto{\pgfqpoint{3.357740in}{3.416522in}}%
\pgfusepath{stroke}%
\end{pgfscope}%
\begin{pgfscope}%
\pgfpathrectangle{\pgfqpoint{0.765000in}{0.660000in}}{\pgfqpoint{4.620000in}{4.620000in}}%
\pgfusepath{clip}%
\pgfsetrectcap%
\pgfsetroundjoin%
\pgfsetlinewidth{1.204500pt}%
\definecolor{currentstroke}{rgb}{0.000000,0.000000,0.000000}%
\pgfsetstrokecolor{currentstroke}%
\pgfsetdash{}{0pt}%
\pgfpathmoveto{\pgfqpoint{3.350280in}{3.424899in}}%
\pgfpathlineto{\pgfqpoint{3.334893in}{3.441865in}}%
\pgfusepath{stroke}%
\end{pgfscope}%
\begin{pgfscope}%
\pgfpathrectangle{\pgfqpoint{0.765000in}{0.660000in}}{\pgfqpoint{4.620000in}{4.620000in}}%
\pgfusepath{clip}%
\pgfsetbuttcap%
\pgfsetroundjoin%
\pgfsetlinewidth{1.204500pt}%
\definecolor{currentstroke}{rgb}{0.827451,0.827451,0.827451}%
\pgfsetstrokecolor{currentstroke}%
\pgfsetdash{{1.200000pt}{1.980000pt}}{0.000000pt}%
\pgfpathmoveto{\pgfqpoint{3.350280in}{3.424899in}}%
\pgfpathlineto{\pgfqpoint{3.531807in}{3.579477in}}%
\pgfusepath{stroke}%
\end{pgfscope}%
\begin{pgfscope}%
\pgfpathrectangle{\pgfqpoint{0.765000in}{0.660000in}}{\pgfqpoint{4.620000in}{4.620000in}}%
\pgfusepath{clip}%
\pgfsetrectcap%
\pgfsetroundjoin%
\pgfsetlinewidth{1.204500pt}%
\definecolor{currentstroke}{rgb}{0.000000,0.000000,0.000000}%
\pgfsetstrokecolor{currentstroke}%
\pgfsetdash{}{0pt}%
\pgfpathmoveto{\pgfqpoint{2.671300in}{3.327008in}}%
\pgfusepath{stroke}%
\end{pgfscope}%
\begin{pgfscope}%
\pgfpathrectangle{\pgfqpoint{0.765000in}{0.660000in}}{\pgfqpoint{4.620000in}{4.620000in}}%
\pgfusepath{clip}%
\pgfsetbuttcap%
\pgfsetroundjoin%
\definecolor{currentfill}{rgb}{0.000000,0.000000,0.000000}%
\pgfsetfillcolor{currentfill}%
\pgfsetlinewidth{1.003750pt}%
\definecolor{currentstroke}{rgb}{0.000000,0.000000,0.000000}%
\pgfsetstrokecolor{currentstroke}%
\pgfsetdash{}{0pt}%
\pgfsys@defobject{currentmarker}{\pgfqpoint{-0.016667in}{-0.016667in}}{\pgfqpoint{0.016667in}{0.016667in}}{%
\pgfpathmoveto{\pgfqpoint{0.000000in}{-0.016667in}}%
\pgfpathcurveto{\pgfqpoint{0.004420in}{-0.016667in}}{\pgfqpoint{0.008660in}{-0.014911in}}{\pgfqpoint{0.011785in}{-0.011785in}}%
\pgfpathcurveto{\pgfqpoint{0.014911in}{-0.008660in}}{\pgfqpoint{0.016667in}{-0.004420in}}{\pgfqpoint{0.016667in}{0.000000in}}%
\pgfpathcurveto{\pgfqpoint{0.016667in}{0.004420in}}{\pgfqpoint{0.014911in}{0.008660in}}{\pgfqpoint{0.011785in}{0.011785in}}%
\pgfpathcurveto{\pgfqpoint{0.008660in}{0.014911in}}{\pgfqpoint{0.004420in}{0.016667in}}{\pgfqpoint{0.000000in}{0.016667in}}%
\pgfpathcurveto{\pgfqpoint{-0.004420in}{0.016667in}}{\pgfqpoint{-0.008660in}{0.014911in}}{\pgfqpoint{-0.011785in}{0.011785in}}%
\pgfpathcurveto{\pgfqpoint{-0.014911in}{0.008660in}}{\pgfqpoint{-0.016667in}{0.004420in}}{\pgfqpoint{-0.016667in}{0.000000in}}%
\pgfpathcurveto{\pgfqpoint{-0.016667in}{-0.004420in}}{\pgfqpoint{-0.014911in}{-0.008660in}}{\pgfqpoint{-0.011785in}{-0.011785in}}%
\pgfpathcurveto{\pgfqpoint{-0.008660in}{-0.014911in}}{\pgfqpoint{-0.004420in}{-0.016667in}}{\pgfqpoint{0.000000in}{-0.016667in}}%
\pgfpathlineto{\pgfqpoint{0.000000in}{-0.016667in}}%
\pgfpathclose%
\pgfusepath{stroke,fill}%
}%
\begin{pgfscope}%
\pgfsys@transformshift{2.671300in}{3.327008in}%
\pgfsys@useobject{currentmarker}{}%
\end{pgfscope}%
\end{pgfscope}%
\begin{pgfscope}%
\pgfpathrectangle{\pgfqpoint{0.765000in}{0.660000in}}{\pgfqpoint{4.620000in}{4.620000in}}%
\pgfusepath{clip}%
\pgfsetrectcap%
\pgfsetroundjoin%
\pgfsetlinewidth{1.204500pt}%
\definecolor{currentstroke}{rgb}{0.000000,0.000000,0.000000}%
\pgfsetstrokecolor{currentstroke}%
\pgfsetdash{}{0pt}%
\pgfpathmoveto{\pgfqpoint{2.671300in}{3.327008in}}%
\pgfpathlineto{\pgfqpoint{2.692300in}{3.344662in}}%
\pgfusepath{stroke}%
\end{pgfscope}%
\begin{pgfscope}%
\pgfpathrectangle{\pgfqpoint{0.765000in}{0.660000in}}{\pgfqpoint{4.620000in}{4.620000in}}%
\pgfusepath{clip}%
\pgfsetrectcap%
\pgfsetroundjoin%
\pgfsetlinewidth{1.204500pt}%
\definecolor{currentstroke}{rgb}{0.000000,0.000000,0.000000}%
\pgfsetstrokecolor{currentstroke}%
\pgfsetdash{}{0pt}%
\pgfpathmoveto{\pgfqpoint{2.671300in}{3.327008in}}%
\pgfpathlineto{\pgfqpoint{2.651156in}{3.305927in}}%
\pgfusepath{stroke}%
\end{pgfscope}%
\begin{pgfscope}%
\pgfpathrectangle{\pgfqpoint{0.765000in}{0.660000in}}{\pgfqpoint{4.620000in}{4.620000in}}%
\pgfusepath{clip}%
\pgfsetbuttcap%
\pgfsetroundjoin%
\pgfsetlinewidth{1.204500pt}%
\definecolor{currentstroke}{rgb}{0.827451,0.827451,0.827451}%
\pgfsetstrokecolor{currentstroke}%
\pgfsetdash{{1.200000pt}{1.980000pt}}{0.000000pt}%
\pgfpathmoveto{\pgfqpoint{2.671300in}{3.327008in}}%
\pgfpathlineto{\pgfqpoint{0.755000in}{4.544134in}}%
\pgfusepath{stroke}%
\end{pgfscope}%
\begin{pgfscope}%
\pgfpathrectangle{\pgfqpoint{0.765000in}{0.660000in}}{\pgfqpoint{4.620000in}{4.620000in}}%
\pgfusepath{clip}%
\pgfsetrectcap%
\pgfsetroundjoin%
\pgfsetlinewidth{1.204500pt}%
\definecolor{currentstroke}{rgb}{0.000000,0.000000,0.000000}%
\pgfsetstrokecolor{currentstroke}%
\pgfsetdash{}{0pt}%
\pgfpathmoveto{\pgfqpoint{3.293363in}{2.546452in}}%
\pgfusepath{stroke}%
\end{pgfscope}%
\begin{pgfscope}%
\pgfpathrectangle{\pgfqpoint{0.765000in}{0.660000in}}{\pgfqpoint{4.620000in}{4.620000in}}%
\pgfusepath{clip}%
\pgfsetbuttcap%
\pgfsetroundjoin%
\definecolor{currentfill}{rgb}{0.000000,0.000000,0.000000}%
\pgfsetfillcolor{currentfill}%
\pgfsetlinewidth{1.003750pt}%
\definecolor{currentstroke}{rgb}{0.000000,0.000000,0.000000}%
\pgfsetstrokecolor{currentstroke}%
\pgfsetdash{}{0pt}%
\pgfsys@defobject{currentmarker}{\pgfqpoint{-0.016667in}{-0.016667in}}{\pgfqpoint{0.016667in}{0.016667in}}{%
\pgfpathmoveto{\pgfqpoint{0.000000in}{-0.016667in}}%
\pgfpathcurveto{\pgfqpoint{0.004420in}{-0.016667in}}{\pgfqpoint{0.008660in}{-0.014911in}}{\pgfqpoint{0.011785in}{-0.011785in}}%
\pgfpathcurveto{\pgfqpoint{0.014911in}{-0.008660in}}{\pgfqpoint{0.016667in}{-0.004420in}}{\pgfqpoint{0.016667in}{0.000000in}}%
\pgfpathcurveto{\pgfqpoint{0.016667in}{0.004420in}}{\pgfqpoint{0.014911in}{0.008660in}}{\pgfqpoint{0.011785in}{0.011785in}}%
\pgfpathcurveto{\pgfqpoint{0.008660in}{0.014911in}}{\pgfqpoint{0.004420in}{0.016667in}}{\pgfqpoint{0.000000in}{0.016667in}}%
\pgfpathcurveto{\pgfqpoint{-0.004420in}{0.016667in}}{\pgfqpoint{-0.008660in}{0.014911in}}{\pgfqpoint{-0.011785in}{0.011785in}}%
\pgfpathcurveto{\pgfqpoint{-0.014911in}{0.008660in}}{\pgfqpoint{-0.016667in}{0.004420in}}{\pgfqpoint{-0.016667in}{0.000000in}}%
\pgfpathcurveto{\pgfqpoint{-0.016667in}{-0.004420in}}{\pgfqpoint{-0.014911in}{-0.008660in}}{\pgfqpoint{-0.011785in}{-0.011785in}}%
\pgfpathcurveto{\pgfqpoint{-0.008660in}{-0.014911in}}{\pgfqpoint{-0.004420in}{-0.016667in}}{\pgfqpoint{0.000000in}{-0.016667in}}%
\pgfpathlineto{\pgfqpoint{0.000000in}{-0.016667in}}%
\pgfpathclose%
\pgfusepath{stroke,fill}%
}%
\begin{pgfscope}%
\pgfsys@transformshift{3.293363in}{2.546452in}%
\pgfsys@useobject{currentmarker}{}%
\end{pgfscope}%
\end{pgfscope}%
\begin{pgfscope}%
\pgfpathrectangle{\pgfqpoint{0.765000in}{0.660000in}}{\pgfqpoint{4.620000in}{4.620000in}}%
\pgfusepath{clip}%
\pgfsetrectcap%
\pgfsetroundjoin%
\pgfsetlinewidth{1.204500pt}%
\definecolor{currentstroke}{rgb}{0.000000,0.000000,0.000000}%
\pgfsetstrokecolor{currentstroke}%
\pgfsetdash{}{0pt}%
\pgfpathmoveto{\pgfqpoint{3.293363in}{2.546452in}}%
\pgfpathlineto{\pgfqpoint{3.242912in}{2.518793in}}%
\pgfusepath{stroke}%
\end{pgfscope}%
\begin{pgfscope}%
\pgfpathrectangle{\pgfqpoint{0.765000in}{0.660000in}}{\pgfqpoint{4.620000in}{4.620000in}}%
\pgfusepath{clip}%
\pgfsetrectcap%
\pgfsetroundjoin%
\pgfsetlinewidth{1.204500pt}%
\definecolor{currentstroke}{rgb}{0.000000,0.000000,0.000000}%
\pgfsetstrokecolor{currentstroke}%
\pgfsetdash{}{0pt}%
\pgfpathmoveto{\pgfqpoint{3.293363in}{2.546452in}}%
\pgfpathlineto{\pgfqpoint{3.463186in}{2.662942in}}%
\pgfusepath{stroke}%
\end{pgfscope}%
\begin{pgfscope}%
\pgfpathrectangle{\pgfqpoint{0.765000in}{0.660000in}}{\pgfqpoint{4.620000in}{4.620000in}}%
\pgfusepath{clip}%
\pgfsetbuttcap%
\pgfsetroundjoin%
\pgfsetlinewidth{1.204500pt}%
\definecolor{currentstroke}{rgb}{0.827451,0.827451,0.827451}%
\pgfsetstrokecolor{currentstroke}%
\pgfsetdash{{1.200000pt}{1.980000pt}}{0.000000pt}%
\pgfpathmoveto{\pgfqpoint{3.293363in}{2.546452in}}%
\pgfpathlineto{\pgfqpoint{3.885134in}{0.696412in}}%
\pgfusepath{stroke}%
\end{pgfscope}%
\begin{pgfscope}%
\pgfpathrectangle{\pgfqpoint{0.765000in}{0.660000in}}{\pgfqpoint{4.620000in}{4.620000in}}%
\pgfusepath{clip}%
\pgfsetrectcap%
\pgfsetroundjoin%
\pgfsetlinewidth{1.204500pt}%
\definecolor{currentstroke}{rgb}{0.000000,0.000000,0.000000}%
\pgfsetstrokecolor{currentstroke}%
\pgfsetdash{}{0pt}%
\pgfpathmoveto{\pgfqpoint{2.982382in}{3.516877in}}%
\pgfusepath{stroke}%
\end{pgfscope}%
\begin{pgfscope}%
\pgfpathrectangle{\pgfqpoint{0.765000in}{0.660000in}}{\pgfqpoint{4.620000in}{4.620000in}}%
\pgfusepath{clip}%
\pgfsetbuttcap%
\pgfsetroundjoin%
\definecolor{currentfill}{rgb}{0.000000,0.000000,0.000000}%
\pgfsetfillcolor{currentfill}%
\pgfsetlinewidth{1.003750pt}%
\definecolor{currentstroke}{rgb}{0.000000,0.000000,0.000000}%
\pgfsetstrokecolor{currentstroke}%
\pgfsetdash{}{0pt}%
\pgfsys@defobject{currentmarker}{\pgfqpoint{-0.016667in}{-0.016667in}}{\pgfqpoint{0.016667in}{0.016667in}}{%
\pgfpathmoveto{\pgfqpoint{0.000000in}{-0.016667in}}%
\pgfpathcurveto{\pgfqpoint{0.004420in}{-0.016667in}}{\pgfqpoint{0.008660in}{-0.014911in}}{\pgfqpoint{0.011785in}{-0.011785in}}%
\pgfpathcurveto{\pgfqpoint{0.014911in}{-0.008660in}}{\pgfqpoint{0.016667in}{-0.004420in}}{\pgfqpoint{0.016667in}{0.000000in}}%
\pgfpathcurveto{\pgfqpoint{0.016667in}{0.004420in}}{\pgfqpoint{0.014911in}{0.008660in}}{\pgfqpoint{0.011785in}{0.011785in}}%
\pgfpathcurveto{\pgfqpoint{0.008660in}{0.014911in}}{\pgfqpoint{0.004420in}{0.016667in}}{\pgfqpoint{0.000000in}{0.016667in}}%
\pgfpathcurveto{\pgfqpoint{-0.004420in}{0.016667in}}{\pgfqpoint{-0.008660in}{0.014911in}}{\pgfqpoint{-0.011785in}{0.011785in}}%
\pgfpathcurveto{\pgfqpoint{-0.014911in}{0.008660in}}{\pgfqpoint{-0.016667in}{0.004420in}}{\pgfqpoint{-0.016667in}{0.000000in}}%
\pgfpathcurveto{\pgfqpoint{-0.016667in}{-0.004420in}}{\pgfqpoint{-0.014911in}{-0.008660in}}{\pgfqpoint{-0.011785in}{-0.011785in}}%
\pgfpathcurveto{\pgfqpoint{-0.008660in}{-0.014911in}}{\pgfqpoint{-0.004420in}{-0.016667in}}{\pgfqpoint{0.000000in}{-0.016667in}}%
\pgfpathlineto{\pgfqpoint{0.000000in}{-0.016667in}}%
\pgfpathclose%
\pgfusepath{stroke,fill}%
}%
\begin{pgfscope}%
\pgfsys@transformshift{2.982382in}{3.516877in}%
\pgfsys@useobject{currentmarker}{}%
\end{pgfscope}%
\end{pgfscope}%
\begin{pgfscope}%
\pgfpathrectangle{\pgfqpoint{0.765000in}{0.660000in}}{\pgfqpoint{4.620000in}{4.620000in}}%
\pgfusepath{clip}%
\pgfsetrectcap%
\pgfsetroundjoin%
\pgfsetlinewidth{1.204500pt}%
\definecolor{currentstroke}{rgb}{0.000000,0.000000,0.000000}%
\pgfsetstrokecolor{currentstroke}%
\pgfsetdash{}{0pt}%
\pgfpathmoveto{\pgfqpoint{2.982382in}{3.516877in}}%
\pgfpathlineto{\pgfqpoint{3.046316in}{3.530257in}}%
\pgfusepath{stroke}%
\end{pgfscope}%
\begin{pgfscope}%
\pgfpathrectangle{\pgfqpoint{0.765000in}{0.660000in}}{\pgfqpoint{4.620000in}{4.620000in}}%
\pgfusepath{clip}%
\pgfsetrectcap%
\pgfsetroundjoin%
\pgfsetlinewidth{1.204500pt}%
\definecolor{currentstroke}{rgb}{0.000000,0.000000,0.000000}%
\pgfsetstrokecolor{currentstroke}%
\pgfsetdash{}{0pt}%
\pgfpathmoveto{\pgfqpoint{2.982382in}{3.516877in}}%
\pgfpathlineto{\pgfqpoint{2.913379in}{3.479681in}}%
\pgfusepath{stroke}%
\end{pgfscope}%
\begin{pgfscope}%
\pgfpathrectangle{\pgfqpoint{0.765000in}{0.660000in}}{\pgfqpoint{4.620000in}{4.620000in}}%
\pgfusepath{clip}%
\pgfsetbuttcap%
\pgfsetroundjoin%
\pgfsetlinewidth{1.204500pt}%
\definecolor{currentstroke}{rgb}{0.827451,0.827451,0.827451}%
\pgfsetstrokecolor{currentstroke}%
\pgfsetdash{{1.200000pt}{1.980000pt}}{0.000000pt}%
\pgfpathmoveto{\pgfqpoint{2.982382in}{3.516877in}}%
\pgfpathlineto{\pgfqpoint{1.844335in}{5.290000in}}%
\pgfusepath{stroke}%
\end{pgfscope}%
\begin{pgfscope}%
\pgfpathrectangle{\pgfqpoint{0.765000in}{0.660000in}}{\pgfqpoint{4.620000in}{4.620000in}}%
\pgfusepath{clip}%
\pgfsetrectcap%
\pgfsetroundjoin%
\pgfsetlinewidth{1.204500pt}%
\definecolor{currentstroke}{rgb}{0.000000,0.000000,0.000000}%
\pgfsetstrokecolor{currentstroke}%
\pgfsetdash{}{0pt}%
\pgfpathmoveto{\pgfqpoint{3.463186in}{2.662942in}}%
\pgfusepath{stroke}%
\end{pgfscope}%
\begin{pgfscope}%
\pgfpathrectangle{\pgfqpoint{0.765000in}{0.660000in}}{\pgfqpoint{4.620000in}{4.620000in}}%
\pgfusepath{clip}%
\pgfsetbuttcap%
\pgfsetroundjoin%
\definecolor{currentfill}{rgb}{0.000000,0.000000,0.000000}%
\pgfsetfillcolor{currentfill}%
\pgfsetlinewidth{1.003750pt}%
\definecolor{currentstroke}{rgb}{0.000000,0.000000,0.000000}%
\pgfsetstrokecolor{currentstroke}%
\pgfsetdash{}{0pt}%
\pgfsys@defobject{currentmarker}{\pgfqpoint{-0.016667in}{-0.016667in}}{\pgfqpoint{0.016667in}{0.016667in}}{%
\pgfpathmoveto{\pgfqpoint{0.000000in}{-0.016667in}}%
\pgfpathcurveto{\pgfqpoint{0.004420in}{-0.016667in}}{\pgfqpoint{0.008660in}{-0.014911in}}{\pgfqpoint{0.011785in}{-0.011785in}}%
\pgfpathcurveto{\pgfqpoint{0.014911in}{-0.008660in}}{\pgfqpoint{0.016667in}{-0.004420in}}{\pgfqpoint{0.016667in}{0.000000in}}%
\pgfpathcurveto{\pgfqpoint{0.016667in}{0.004420in}}{\pgfqpoint{0.014911in}{0.008660in}}{\pgfqpoint{0.011785in}{0.011785in}}%
\pgfpathcurveto{\pgfqpoint{0.008660in}{0.014911in}}{\pgfqpoint{0.004420in}{0.016667in}}{\pgfqpoint{0.000000in}{0.016667in}}%
\pgfpathcurveto{\pgfqpoint{-0.004420in}{0.016667in}}{\pgfqpoint{-0.008660in}{0.014911in}}{\pgfqpoint{-0.011785in}{0.011785in}}%
\pgfpathcurveto{\pgfqpoint{-0.014911in}{0.008660in}}{\pgfqpoint{-0.016667in}{0.004420in}}{\pgfqpoint{-0.016667in}{0.000000in}}%
\pgfpathcurveto{\pgfqpoint{-0.016667in}{-0.004420in}}{\pgfqpoint{-0.014911in}{-0.008660in}}{\pgfqpoint{-0.011785in}{-0.011785in}}%
\pgfpathcurveto{\pgfqpoint{-0.008660in}{-0.014911in}}{\pgfqpoint{-0.004420in}{-0.016667in}}{\pgfqpoint{0.000000in}{-0.016667in}}%
\pgfpathlineto{\pgfqpoint{0.000000in}{-0.016667in}}%
\pgfpathclose%
\pgfusepath{stroke,fill}%
}%
\begin{pgfscope}%
\pgfsys@transformshift{3.463186in}{2.662942in}%
\pgfsys@useobject{currentmarker}{}%
\end{pgfscope}%
\end{pgfscope}%
\begin{pgfscope}%
\pgfpathrectangle{\pgfqpoint{0.765000in}{0.660000in}}{\pgfqpoint{4.620000in}{4.620000in}}%
\pgfusepath{clip}%
\pgfsetrectcap%
\pgfsetroundjoin%
\pgfsetlinewidth{1.204500pt}%
\definecolor{currentstroke}{rgb}{0.000000,0.000000,0.000000}%
\pgfsetstrokecolor{currentstroke}%
\pgfsetdash{}{0pt}%
\pgfpathmoveto{\pgfqpoint{3.463186in}{2.662942in}}%
\pgfpathlineto{\pgfqpoint{3.477043in}{2.674663in}}%
\pgfusepath{stroke}%
\end{pgfscope}%
\begin{pgfscope}%
\pgfpathrectangle{\pgfqpoint{0.765000in}{0.660000in}}{\pgfqpoint{4.620000in}{4.620000in}}%
\pgfusepath{clip}%
\pgfsetrectcap%
\pgfsetroundjoin%
\pgfsetlinewidth{1.204500pt}%
\definecolor{currentstroke}{rgb}{0.000000,0.000000,0.000000}%
\pgfsetstrokecolor{currentstroke}%
\pgfsetdash{}{0pt}%
\pgfpathmoveto{\pgfqpoint{3.463186in}{2.662942in}}%
\pgfpathlineto{\pgfqpoint{3.293363in}{2.546452in}}%
\pgfusepath{stroke}%
\end{pgfscope}%
\begin{pgfscope}%
\pgfpathrectangle{\pgfqpoint{0.765000in}{0.660000in}}{\pgfqpoint{4.620000in}{4.620000in}}%
\pgfusepath{clip}%
\pgfsetbuttcap%
\pgfsetroundjoin%
\pgfsetlinewidth{1.204500pt}%
\definecolor{currentstroke}{rgb}{0.827451,0.827451,0.827451}%
\pgfsetstrokecolor{currentstroke}%
\pgfsetdash{{1.200000pt}{1.980000pt}}{0.000000pt}%
\pgfpathmoveto{\pgfqpoint{3.463186in}{2.662942in}}%
\pgfpathlineto{\pgfqpoint{4.255399in}{0.808856in}}%
\pgfusepath{stroke}%
\end{pgfscope}%
\begin{pgfscope}%
\pgfpathrectangle{\pgfqpoint{0.765000in}{0.660000in}}{\pgfqpoint{4.620000in}{4.620000in}}%
\pgfusepath{clip}%
\pgfsetrectcap%
\pgfsetroundjoin%
\pgfsetlinewidth{1.204500pt}%
\definecolor{currentstroke}{rgb}{0.000000,0.000000,0.000000}%
\pgfsetstrokecolor{currentstroke}%
\pgfsetdash{}{0pt}%
\pgfpathmoveto{\pgfqpoint{2.582969in}{3.161487in}}%
\pgfusepath{stroke}%
\end{pgfscope}%
\begin{pgfscope}%
\pgfpathrectangle{\pgfqpoint{0.765000in}{0.660000in}}{\pgfqpoint{4.620000in}{4.620000in}}%
\pgfusepath{clip}%
\pgfsetbuttcap%
\pgfsetroundjoin%
\definecolor{currentfill}{rgb}{0.000000,0.000000,0.000000}%
\pgfsetfillcolor{currentfill}%
\pgfsetlinewidth{1.003750pt}%
\definecolor{currentstroke}{rgb}{0.000000,0.000000,0.000000}%
\pgfsetstrokecolor{currentstroke}%
\pgfsetdash{}{0pt}%
\pgfsys@defobject{currentmarker}{\pgfqpoint{-0.016667in}{-0.016667in}}{\pgfqpoint{0.016667in}{0.016667in}}{%
\pgfpathmoveto{\pgfqpoint{0.000000in}{-0.016667in}}%
\pgfpathcurveto{\pgfqpoint{0.004420in}{-0.016667in}}{\pgfqpoint{0.008660in}{-0.014911in}}{\pgfqpoint{0.011785in}{-0.011785in}}%
\pgfpathcurveto{\pgfqpoint{0.014911in}{-0.008660in}}{\pgfqpoint{0.016667in}{-0.004420in}}{\pgfqpoint{0.016667in}{0.000000in}}%
\pgfpathcurveto{\pgfqpoint{0.016667in}{0.004420in}}{\pgfqpoint{0.014911in}{0.008660in}}{\pgfqpoint{0.011785in}{0.011785in}}%
\pgfpathcurveto{\pgfqpoint{0.008660in}{0.014911in}}{\pgfqpoint{0.004420in}{0.016667in}}{\pgfqpoint{0.000000in}{0.016667in}}%
\pgfpathcurveto{\pgfqpoint{-0.004420in}{0.016667in}}{\pgfqpoint{-0.008660in}{0.014911in}}{\pgfqpoint{-0.011785in}{0.011785in}}%
\pgfpathcurveto{\pgfqpoint{-0.014911in}{0.008660in}}{\pgfqpoint{-0.016667in}{0.004420in}}{\pgfqpoint{-0.016667in}{0.000000in}}%
\pgfpathcurveto{\pgfqpoint{-0.016667in}{-0.004420in}}{\pgfqpoint{-0.014911in}{-0.008660in}}{\pgfqpoint{-0.011785in}{-0.011785in}}%
\pgfpathcurveto{\pgfqpoint{-0.008660in}{-0.014911in}}{\pgfqpoint{-0.004420in}{-0.016667in}}{\pgfqpoint{0.000000in}{-0.016667in}}%
\pgfpathlineto{\pgfqpoint{0.000000in}{-0.016667in}}%
\pgfpathclose%
\pgfusepath{stroke,fill}%
}%
\begin{pgfscope}%
\pgfsys@transformshift{2.582969in}{3.161487in}%
\pgfsys@useobject{currentmarker}{}%
\end{pgfscope}%
\end{pgfscope}%
\begin{pgfscope}%
\pgfpathrectangle{\pgfqpoint{0.765000in}{0.660000in}}{\pgfqpoint{4.620000in}{4.620000in}}%
\pgfusepath{clip}%
\pgfsetrectcap%
\pgfsetroundjoin%
\pgfsetlinewidth{1.204500pt}%
\definecolor{currentstroke}{rgb}{0.000000,0.000000,0.000000}%
\pgfsetstrokecolor{currentstroke}%
\pgfsetdash{}{0pt}%
\pgfpathmoveto{\pgfqpoint{2.582969in}{3.161487in}}%
\pgfpathlineto{\pgfqpoint{2.583171in}{3.110796in}}%
\pgfusepath{stroke}%
\end{pgfscope}%
\begin{pgfscope}%
\pgfpathrectangle{\pgfqpoint{0.765000in}{0.660000in}}{\pgfqpoint{4.620000in}{4.620000in}}%
\pgfusepath{clip}%
\pgfsetrectcap%
\pgfsetroundjoin%
\pgfsetlinewidth{1.204500pt}%
\definecolor{currentstroke}{rgb}{0.000000,0.000000,0.000000}%
\pgfsetstrokecolor{currentstroke}%
\pgfsetdash{}{0pt}%
\pgfpathmoveto{\pgfqpoint{2.582969in}{3.161487in}}%
\pgfpathlineto{\pgfqpoint{2.597727in}{3.213477in}}%
\pgfusepath{stroke}%
\end{pgfscope}%
\begin{pgfscope}%
\pgfpathrectangle{\pgfqpoint{0.765000in}{0.660000in}}{\pgfqpoint{4.620000in}{4.620000in}}%
\pgfusepath{clip}%
\pgfsetbuttcap%
\pgfsetroundjoin%
\pgfsetlinewidth{1.204500pt}%
\definecolor{currentstroke}{rgb}{0.827451,0.827451,0.827451}%
\pgfsetstrokecolor{currentstroke}%
\pgfsetdash{{1.200000pt}{1.980000pt}}{0.000000pt}%
\pgfpathmoveto{\pgfqpoint{2.582969in}{3.161487in}}%
\pgfpathlineto{\pgfqpoint{0.755000in}{3.033059in}}%
\pgfusepath{stroke}%
\end{pgfscope}%
\begin{pgfscope}%
\pgfpathrectangle{\pgfqpoint{0.765000in}{0.660000in}}{\pgfqpoint{4.620000in}{4.620000in}}%
\pgfusepath{clip}%
\pgfsetrectcap%
\pgfsetroundjoin%
\pgfsetlinewidth{1.204500pt}%
\definecolor{currentstroke}{rgb}{0.000000,0.000000,0.000000}%
\pgfsetstrokecolor{currentstroke}%
\pgfsetdash{}{0pt}%
\pgfpathmoveto{\pgfqpoint{2.597727in}{3.213477in}}%
\pgfusepath{stroke}%
\end{pgfscope}%
\begin{pgfscope}%
\pgfpathrectangle{\pgfqpoint{0.765000in}{0.660000in}}{\pgfqpoint{4.620000in}{4.620000in}}%
\pgfusepath{clip}%
\pgfsetbuttcap%
\pgfsetroundjoin%
\definecolor{currentfill}{rgb}{0.000000,0.000000,0.000000}%
\pgfsetfillcolor{currentfill}%
\pgfsetlinewidth{1.003750pt}%
\definecolor{currentstroke}{rgb}{0.000000,0.000000,0.000000}%
\pgfsetstrokecolor{currentstroke}%
\pgfsetdash{}{0pt}%
\pgfsys@defobject{currentmarker}{\pgfqpoint{-0.016667in}{-0.016667in}}{\pgfqpoint{0.016667in}{0.016667in}}{%
\pgfpathmoveto{\pgfqpoint{0.000000in}{-0.016667in}}%
\pgfpathcurveto{\pgfqpoint{0.004420in}{-0.016667in}}{\pgfqpoint{0.008660in}{-0.014911in}}{\pgfqpoint{0.011785in}{-0.011785in}}%
\pgfpathcurveto{\pgfqpoint{0.014911in}{-0.008660in}}{\pgfqpoint{0.016667in}{-0.004420in}}{\pgfqpoint{0.016667in}{0.000000in}}%
\pgfpathcurveto{\pgfqpoint{0.016667in}{0.004420in}}{\pgfqpoint{0.014911in}{0.008660in}}{\pgfqpoint{0.011785in}{0.011785in}}%
\pgfpathcurveto{\pgfqpoint{0.008660in}{0.014911in}}{\pgfqpoint{0.004420in}{0.016667in}}{\pgfqpoint{0.000000in}{0.016667in}}%
\pgfpathcurveto{\pgfqpoint{-0.004420in}{0.016667in}}{\pgfqpoint{-0.008660in}{0.014911in}}{\pgfqpoint{-0.011785in}{0.011785in}}%
\pgfpathcurveto{\pgfqpoint{-0.014911in}{0.008660in}}{\pgfqpoint{-0.016667in}{0.004420in}}{\pgfqpoint{-0.016667in}{0.000000in}}%
\pgfpathcurveto{\pgfqpoint{-0.016667in}{-0.004420in}}{\pgfqpoint{-0.014911in}{-0.008660in}}{\pgfqpoint{-0.011785in}{-0.011785in}}%
\pgfpathcurveto{\pgfqpoint{-0.008660in}{-0.014911in}}{\pgfqpoint{-0.004420in}{-0.016667in}}{\pgfqpoint{0.000000in}{-0.016667in}}%
\pgfpathlineto{\pgfqpoint{0.000000in}{-0.016667in}}%
\pgfpathclose%
\pgfusepath{stroke,fill}%
}%
\begin{pgfscope}%
\pgfsys@transformshift{2.597727in}{3.213477in}%
\pgfsys@useobject{currentmarker}{}%
\end{pgfscope}%
\end{pgfscope}%
\begin{pgfscope}%
\pgfpathrectangle{\pgfqpoint{0.765000in}{0.660000in}}{\pgfqpoint{4.620000in}{4.620000in}}%
\pgfusepath{clip}%
\pgfsetrectcap%
\pgfsetroundjoin%
\pgfsetlinewidth{1.204500pt}%
\definecolor{currentstroke}{rgb}{0.000000,0.000000,0.000000}%
\pgfsetstrokecolor{currentstroke}%
\pgfsetdash{}{0pt}%
\pgfpathmoveto{\pgfqpoint{2.597727in}{3.213477in}}%
\pgfpathlineto{\pgfqpoint{2.613910in}{3.253240in}}%
\pgfusepath{stroke}%
\end{pgfscope}%
\begin{pgfscope}%
\pgfpathrectangle{\pgfqpoint{0.765000in}{0.660000in}}{\pgfqpoint{4.620000in}{4.620000in}}%
\pgfusepath{clip}%
\pgfsetrectcap%
\pgfsetroundjoin%
\pgfsetlinewidth{1.204500pt}%
\definecolor{currentstroke}{rgb}{0.000000,0.000000,0.000000}%
\pgfsetstrokecolor{currentstroke}%
\pgfsetdash{}{0pt}%
\pgfpathmoveto{\pgfqpoint{2.597727in}{3.213477in}}%
\pgfpathlineto{\pgfqpoint{2.582969in}{3.161487in}}%
\pgfusepath{stroke}%
\end{pgfscope}%
\begin{pgfscope}%
\pgfpathrectangle{\pgfqpoint{0.765000in}{0.660000in}}{\pgfqpoint{4.620000in}{4.620000in}}%
\pgfusepath{clip}%
\pgfsetbuttcap%
\pgfsetroundjoin%
\pgfsetlinewidth{1.204500pt}%
\definecolor{currentstroke}{rgb}{0.827451,0.827451,0.827451}%
\pgfsetstrokecolor{currentstroke}%
\pgfsetdash{{1.200000pt}{1.980000pt}}{0.000000pt}%
\pgfpathmoveto{\pgfqpoint{2.597727in}{3.213477in}}%
\pgfpathlineto{\pgfqpoint{0.755000in}{3.397593in}}%
\pgfusepath{stroke}%
\end{pgfscope}%
\begin{pgfscope}%
\pgfpathrectangle{\pgfqpoint{0.765000in}{0.660000in}}{\pgfqpoint{4.620000in}{4.620000in}}%
\pgfusepath{clip}%
\pgfsetrectcap%
\pgfsetroundjoin%
\pgfsetlinewidth{1.204500pt}%
\definecolor{currentstroke}{rgb}{0.000000,0.000000,0.000000}%
\pgfsetstrokecolor{currentstroke}%
\pgfsetdash{}{0pt}%
\pgfpathmoveto{\pgfqpoint{3.230111in}{3.500132in}}%
\pgfusepath{stroke}%
\end{pgfscope}%
\begin{pgfscope}%
\pgfpathrectangle{\pgfqpoint{0.765000in}{0.660000in}}{\pgfqpoint{4.620000in}{4.620000in}}%
\pgfusepath{clip}%
\pgfsetbuttcap%
\pgfsetroundjoin%
\definecolor{currentfill}{rgb}{0.000000,0.000000,0.000000}%
\pgfsetfillcolor{currentfill}%
\pgfsetlinewidth{1.003750pt}%
\definecolor{currentstroke}{rgb}{0.000000,0.000000,0.000000}%
\pgfsetstrokecolor{currentstroke}%
\pgfsetdash{}{0pt}%
\pgfsys@defobject{currentmarker}{\pgfqpoint{-0.016667in}{-0.016667in}}{\pgfqpoint{0.016667in}{0.016667in}}{%
\pgfpathmoveto{\pgfqpoint{0.000000in}{-0.016667in}}%
\pgfpathcurveto{\pgfqpoint{0.004420in}{-0.016667in}}{\pgfqpoint{0.008660in}{-0.014911in}}{\pgfqpoint{0.011785in}{-0.011785in}}%
\pgfpathcurveto{\pgfqpoint{0.014911in}{-0.008660in}}{\pgfqpoint{0.016667in}{-0.004420in}}{\pgfqpoint{0.016667in}{0.000000in}}%
\pgfpathcurveto{\pgfqpoint{0.016667in}{0.004420in}}{\pgfqpoint{0.014911in}{0.008660in}}{\pgfqpoint{0.011785in}{0.011785in}}%
\pgfpathcurveto{\pgfqpoint{0.008660in}{0.014911in}}{\pgfqpoint{0.004420in}{0.016667in}}{\pgfqpoint{0.000000in}{0.016667in}}%
\pgfpathcurveto{\pgfqpoint{-0.004420in}{0.016667in}}{\pgfqpoint{-0.008660in}{0.014911in}}{\pgfqpoint{-0.011785in}{0.011785in}}%
\pgfpathcurveto{\pgfqpoint{-0.014911in}{0.008660in}}{\pgfqpoint{-0.016667in}{0.004420in}}{\pgfqpoint{-0.016667in}{0.000000in}}%
\pgfpathcurveto{\pgfqpoint{-0.016667in}{-0.004420in}}{\pgfqpoint{-0.014911in}{-0.008660in}}{\pgfqpoint{-0.011785in}{-0.011785in}}%
\pgfpathcurveto{\pgfqpoint{-0.008660in}{-0.014911in}}{\pgfqpoint{-0.004420in}{-0.016667in}}{\pgfqpoint{0.000000in}{-0.016667in}}%
\pgfpathlineto{\pgfqpoint{0.000000in}{-0.016667in}}%
\pgfpathclose%
\pgfusepath{stroke,fill}%
}%
\begin{pgfscope}%
\pgfsys@transformshift{3.230111in}{3.500132in}%
\pgfsys@useobject{currentmarker}{}%
\end{pgfscope}%
\end{pgfscope}%
\begin{pgfscope}%
\pgfpathrectangle{\pgfqpoint{0.765000in}{0.660000in}}{\pgfqpoint{4.620000in}{4.620000in}}%
\pgfusepath{clip}%
\pgfsetrectcap%
\pgfsetroundjoin%
\pgfsetlinewidth{1.204500pt}%
\definecolor{currentstroke}{rgb}{0.000000,0.000000,0.000000}%
\pgfsetstrokecolor{currentstroke}%
\pgfsetdash{}{0pt}%
\pgfpathmoveto{\pgfqpoint{3.230111in}{3.500132in}}%
\pgfpathlineto{\pgfqpoint{3.145667in}{3.522403in}}%
\pgfusepath{stroke}%
\end{pgfscope}%
\begin{pgfscope}%
\pgfpathrectangle{\pgfqpoint{0.765000in}{0.660000in}}{\pgfqpoint{4.620000in}{4.620000in}}%
\pgfusepath{clip}%
\pgfsetrectcap%
\pgfsetroundjoin%
\pgfsetlinewidth{1.204500pt}%
\definecolor{currentstroke}{rgb}{0.000000,0.000000,0.000000}%
\pgfsetstrokecolor{currentstroke}%
\pgfsetdash{}{0pt}%
\pgfpathmoveto{\pgfqpoint{3.230111in}{3.500132in}}%
\pgfpathlineto{\pgfqpoint{3.265487in}{3.489049in}}%
\pgfusepath{stroke}%
\end{pgfscope}%
\begin{pgfscope}%
\pgfpathrectangle{\pgfqpoint{0.765000in}{0.660000in}}{\pgfqpoint{4.620000in}{4.620000in}}%
\pgfusepath{clip}%
\pgfsetbuttcap%
\pgfsetroundjoin%
\pgfsetlinewidth{1.204500pt}%
\definecolor{currentstroke}{rgb}{0.827451,0.827451,0.827451}%
\pgfsetstrokecolor{currentstroke}%
\pgfsetdash{{1.200000pt}{1.980000pt}}{0.000000pt}%
\pgfpathmoveto{\pgfqpoint{3.230111in}{3.500132in}}%
\pgfpathlineto{\pgfqpoint{3.431496in}{4.662332in}}%
\pgfusepath{stroke}%
\end{pgfscope}%
\end{pgfpicture}%
\makeatother%
\endgroup%
}}
        
        \caption{Sampling $10$ points from each class}
        \label{fig:featureeng}
    \end{subfigure}
    
    \bigskip
    \centering
    \begin{subfigure}{0.45\textwidth}
        \centering
        \resizebox{\linewidth}{!}{%
        \centering
            \clipbox{0.15\width{} 0.3\height{} 0.15\width{} 0.2\height{}}{%% Creator: Matplotlib, PGF backend
%%
%% To include the figure in your LaTeX document, write
%%   \input{<filename>.pgf}
%%
%% Make sure the required packages are loaded in your preamble
%%   \usepackage{pgf}
%%
%% Also ensure that all the required font packages are loaded; for instance,
%% the lmodern package is sometimes necessary when using math font.
%%   \usepackage{lmodern}
%%
%% Figures using additional raster images can only be included by \input if
%% they are in the same directory as the main LaTeX file. For loading figures
%% from other directories you can use the `import` package
%%   \usepackage{import}
%%
%% and then include the figures with
%%   \import{<path to file>}{<filename>.pgf}
%%
%% Matplotlib used the following preamble
%%   
%%   \usepackage{fontspec}
%%   \setmainfont{DejaVuSerif.ttf}[Path=\detokenize{/Users/sam/Library/Python/3.9/lib/python/site-packages/matplotlib/mpl-data/fonts/ttf/}]
%%   \setsansfont{DejaVuSans.ttf}[Path=\detokenize{/Users/sam/Library/Python/3.9/lib/python/site-packages/matplotlib/mpl-data/fonts/ttf/}]
%%   \setmonofont{DejaVuSansMono.ttf}[Path=\detokenize{/Users/sam/Library/Python/3.9/lib/python/site-packages/matplotlib/mpl-data/fonts/ttf/}]
%%   \makeatletter\@ifpackageloaded{underscore}{}{\usepackage[strings]{underscore}}\makeatother
%%
\begingroup%
\makeatletter%
\begin{pgfpicture}%
\pgfpathrectangle{\pgfpointorigin}{\pgfqpoint{6.000000in}{6.000000in}}%
\pgfusepath{use as bounding box, clip}%
\begin{pgfscope}%
\pgfsetbuttcap%
\pgfsetmiterjoin%
\definecolor{currentfill}{rgb}{1.000000,1.000000,1.000000}%
\pgfsetfillcolor{currentfill}%
\pgfsetlinewidth{0.000000pt}%
\definecolor{currentstroke}{rgb}{1.000000,1.000000,1.000000}%
\pgfsetstrokecolor{currentstroke}%
\pgfsetdash{}{0pt}%
\pgfpathmoveto{\pgfqpoint{0.000000in}{0.000000in}}%
\pgfpathlineto{\pgfqpoint{6.000000in}{0.000000in}}%
\pgfpathlineto{\pgfqpoint{6.000000in}{6.000000in}}%
\pgfpathlineto{\pgfqpoint{0.000000in}{6.000000in}}%
\pgfpathlineto{\pgfqpoint{0.000000in}{0.000000in}}%
\pgfpathclose%
\pgfusepath{fill}%
\end{pgfscope}%
\begin{pgfscope}%
\pgfsetbuttcap%
\pgfsetmiterjoin%
\definecolor{currentfill}{rgb}{1.000000,1.000000,1.000000}%
\pgfsetfillcolor{currentfill}%
\pgfsetlinewidth{0.000000pt}%
\definecolor{currentstroke}{rgb}{0.000000,0.000000,0.000000}%
\pgfsetstrokecolor{currentstroke}%
\pgfsetstrokeopacity{0.000000}%
\pgfsetdash{}{0pt}%
\pgfpathmoveto{\pgfqpoint{0.765000in}{0.660000in}}%
\pgfpathlineto{\pgfqpoint{5.385000in}{0.660000in}}%
\pgfpathlineto{\pgfqpoint{5.385000in}{5.280000in}}%
\pgfpathlineto{\pgfqpoint{0.765000in}{5.280000in}}%
\pgfpathlineto{\pgfqpoint{0.765000in}{0.660000in}}%
\pgfpathclose%
\pgfusepath{fill}%
\end{pgfscope}%
\begin{pgfscope}%
\pgfpathrectangle{\pgfqpoint{0.765000in}{0.660000in}}{\pgfqpoint{4.620000in}{4.620000in}}%
\pgfusepath{clip}%
\pgfsetrectcap%
\pgfsetroundjoin%
\pgfsetlinewidth{1.204500pt}%
\definecolor{currentstroke}{rgb}{1.000000,0.576471,0.309804}%
\pgfsetstrokecolor{currentstroke}%
\pgfsetdash{}{0pt}%
\pgfpathmoveto{\pgfqpoint{3.985626in}{2.728727in}}%
\pgfusepath{stroke}%
\end{pgfscope}%
\begin{pgfscope}%
\pgfpathrectangle{\pgfqpoint{0.765000in}{0.660000in}}{\pgfqpoint{4.620000in}{4.620000in}}%
\pgfusepath{clip}%
\pgfsetbuttcap%
\pgfsetroundjoin%
\definecolor{currentfill}{rgb}{1.000000,0.576471,0.309804}%
\pgfsetfillcolor{currentfill}%
\pgfsetlinewidth{1.003750pt}%
\definecolor{currentstroke}{rgb}{1.000000,0.576471,0.309804}%
\pgfsetstrokecolor{currentstroke}%
\pgfsetdash{}{0pt}%
\pgfsys@defobject{currentmarker}{\pgfqpoint{-0.033333in}{-0.033333in}}{\pgfqpoint{0.033333in}{0.033333in}}{%
\pgfpathmoveto{\pgfqpoint{0.000000in}{-0.033333in}}%
\pgfpathcurveto{\pgfqpoint{0.008840in}{-0.033333in}}{\pgfqpoint{0.017319in}{-0.029821in}}{\pgfqpoint{0.023570in}{-0.023570in}}%
\pgfpathcurveto{\pgfqpoint{0.029821in}{-0.017319in}}{\pgfqpoint{0.033333in}{-0.008840in}}{\pgfqpoint{0.033333in}{0.000000in}}%
\pgfpathcurveto{\pgfqpoint{0.033333in}{0.008840in}}{\pgfqpoint{0.029821in}{0.017319in}}{\pgfqpoint{0.023570in}{0.023570in}}%
\pgfpathcurveto{\pgfqpoint{0.017319in}{0.029821in}}{\pgfqpoint{0.008840in}{0.033333in}}{\pgfqpoint{0.000000in}{0.033333in}}%
\pgfpathcurveto{\pgfqpoint{-0.008840in}{0.033333in}}{\pgfqpoint{-0.017319in}{0.029821in}}{\pgfqpoint{-0.023570in}{0.023570in}}%
\pgfpathcurveto{\pgfqpoint{-0.029821in}{0.017319in}}{\pgfqpoint{-0.033333in}{0.008840in}}{\pgfqpoint{-0.033333in}{0.000000in}}%
\pgfpathcurveto{\pgfqpoint{-0.033333in}{-0.008840in}}{\pgfqpoint{-0.029821in}{-0.017319in}}{\pgfqpoint{-0.023570in}{-0.023570in}}%
\pgfpathcurveto{\pgfqpoint{-0.017319in}{-0.029821in}}{\pgfqpoint{-0.008840in}{-0.033333in}}{\pgfqpoint{0.000000in}{-0.033333in}}%
\pgfpathlineto{\pgfqpoint{0.000000in}{-0.033333in}}%
\pgfpathclose%
\pgfusepath{stroke,fill}%
}%
\begin{pgfscope}%
\pgfsys@transformshift{3.985626in}{2.728727in}%
\pgfsys@useobject{currentmarker}{}%
\end{pgfscope}%
\end{pgfscope}%
\begin{pgfscope}%
\pgfpathrectangle{\pgfqpoint{0.765000in}{0.660000in}}{\pgfqpoint{4.620000in}{4.620000in}}%
\pgfusepath{clip}%
\pgfsetrectcap%
\pgfsetroundjoin%
\pgfsetlinewidth{1.204500pt}%
\definecolor{currentstroke}{rgb}{1.000000,0.576471,0.309804}%
\pgfsetstrokecolor{currentstroke}%
\pgfsetdash{}{0pt}%
\pgfpathmoveto{\pgfqpoint{4.372321in}{2.641107in}}%
\pgfusepath{stroke}%
\end{pgfscope}%
\begin{pgfscope}%
\pgfpathrectangle{\pgfqpoint{0.765000in}{0.660000in}}{\pgfqpoint{4.620000in}{4.620000in}}%
\pgfusepath{clip}%
\pgfsetbuttcap%
\pgfsetroundjoin%
\definecolor{currentfill}{rgb}{1.000000,0.576471,0.309804}%
\pgfsetfillcolor{currentfill}%
\pgfsetlinewidth{1.003750pt}%
\definecolor{currentstroke}{rgb}{1.000000,0.576471,0.309804}%
\pgfsetstrokecolor{currentstroke}%
\pgfsetdash{}{0pt}%
\pgfsys@defobject{currentmarker}{\pgfqpoint{-0.033333in}{-0.033333in}}{\pgfqpoint{0.033333in}{0.033333in}}{%
\pgfpathmoveto{\pgfqpoint{0.000000in}{-0.033333in}}%
\pgfpathcurveto{\pgfqpoint{0.008840in}{-0.033333in}}{\pgfqpoint{0.017319in}{-0.029821in}}{\pgfqpoint{0.023570in}{-0.023570in}}%
\pgfpathcurveto{\pgfqpoint{0.029821in}{-0.017319in}}{\pgfqpoint{0.033333in}{-0.008840in}}{\pgfqpoint{0.033333in}{0.000000in}}%
\pgfpathcurveto{\pgfqpoint{0.033333in}{0.008840in}}{\pgfqpoint{0.029821in}{0.017319in}}{\pgfqpoint{0.023570in}{0.023570in}}%
\pgfpathcurveto{\pgfqpoint{0.017319in}{0.029821in}}{\pgfqpoint{0.008840in}{0.033333in}}{\pgfqpoint{0.000000in}{0.033333in}}%
\pgfpathcurveto{\pgfqpoint{-0.008840in}{0.033333in}}{\pgfqpoint{-0.017319in}{0.029821in}}{\pgfqpoint{-0.023570in}{0.023570in}}%
\pgfpathcurveto{\pgfqpoint{-0.029821in}{0.017319in}}{\pgfqpoint{-0.033333in}{0.008840in}}{\pgfqpoint{-0.033333in}{0.000000in}}%
\pgfpathcurveto{\pgfqpoint{-0.033333in}{-0.008840in}}{\pgfqpoint{-0.029821in}{-0.017319in}}{\pgfqpoint{-0.023570in}{-0.023570in}}%
\pgfpathcurveto{\pgfqpoint{-0.017319in}{-0.029821in}}{\pgfqpoint{-0.008840in}{-0.033333in}}{\pgfqpoint{0.000000in}{-0.033333in}}%
\pgfpathlineto{\pgfqpoint{0.000000in}{-0.033333in}}%
\pgfpathclose%
\pgfusepath{stroke,fill}%
}%
\begin{pgfscope}%
\pgfsys@transformshift{4.372321in}{2.641107in}%
\pgfsys@useobject{currentmarker}{}%
\end{pgfscope}%
\end{pgfscope}%
\begin{pgfscope}%
\pgfpathrectangle{\pgfqpoint{0.765000in}{0.660000in}}{\pgfqpoint{4.620000in}{4.620000in}}%
\pgfusepath{clip}%
\pgfsetrectcap%
\pgfsetroundjoin%
\pgfsetlinewidth{1.204500pt}%
\definecolor{currentstroke}{rgb}{1.000000,0.576471,0.309804}%
\pgfsetstrokecolor{currentstroke}%
\pgfsetdash{}{0pt}%
\pgfpathmoveto{\pgfqpoint{3.797765in}{2.807778in}}%
\pgfusepath{stroke}%
\end{pgfscope}%
\begin{pgfscope}%
\pgfpathrectangle{\pgfqpoint{0.765000in}{0.660000in}}{\pgfqpoint{4.620000in}{4.620000in}}%
\pgfusepath{clip}%
\pgfsetbuttcap%
\pgfsetroundjoin%
\definecolor{currentfill}{rgb}{1.000000,0.576471,0.309804}%
\pgfsetfillcolor{currentfill}%
\pgfsetlinewidth{1.003750pt}%
\definecolor{currentstroke}{rgb}{1.000000,0.576471,0.309804}%
\pgfsetstrokecolor{currentstroke}%
\pgfsetdash{}{0pt}%
\pgfsys@defobject{currentmarker}{\pgfqpoint{-0.033333in}{-0.033333in}}{\pgfqpoint{0.033333in}{0.033333in}}{%
\pgfpathmoveto{\pgfqpoint{0.000000in}{-0.033333in}}%
\pgfpathcurveto{\pgfqpoint{0.008840in}{-0.033333in}}{\pgfqpoint{0.017319in}{-0.029821in}}{\pgfqpoint{0.023570in}{-0.023570in}}%
\pgfpathcurveto{\pgfqpoint{0.029821in}{-0.017319in}}{\pgfqpoint{0.033333in}{-0.008840in}}{\pgfqpoint{0.033333in}{0.000000in}}%
\pgfpathcurveto{\pgfqpoint{0.033333in}{0.008840in}}{\pgfqpoint{0.029821in}{0.017319in}}{\pgfqpoint{0.023570in}{0.023570in}}%
\pgfpathcurveto{\pgfqpoint{0.017319in}{0.029821in}}{\pgfqpoint{0.008840in}{0.033333in}}{\pgfqpoint{0.000000in}{0.033333in}}%
\pgfpathcurveto{\pgfqpoint{-0.008840in}{0.033333in}}{\pgfqpoint{-0.017319in}{0.029821in}}{\pgfqpoint{-0.023570in}{0.023570in}}%
\pgfpathcurveto{\pgfqpoint{-0.029821in}{0.017319in}}{\pgfqpoint{-0.033333in}{0.008840in}}{\pgfqpoint{-0.033333in}{0.000000in}}%
\pgfpathcurveto{\pgfqpoint{-0.033333in}{-0.008840in}}{\pgfqpoint{-0.029821in}{-0.017319in}}{\pgfqpoint{-0.023570in}{-0.023570in}}%
\pgfpathcurveto{\pgfqpoint{-0.017319in}{-0.029821in}}{\pgfqpoint{-0.008840in}{-0.033333in}}{\pgfqpoint{0.000000in}{-0.033333in}}%
\pgfpathlineto{\pgfqpoint{0.000000in}{-0.033333in}}%
\pgfpathclose%
\pgfusepath{stroke,fill}%
}%
\begin{pgfscope}%
\pgfsys@transformshift{3.797765in}{2.807778in}%
\pgfsys@useobject{currentmarker}{}%
\end{pgfscope}%
\end{pgfscope}%
\begin{pgfscope}%
\pgfpathrectangle{\pgfqpoint{0.765000in}{0.660000in}}{\pgfqpoint{4.620000in}{4.620000in}}%
\pgfusepath{clip}%
\pgfsetrectcap%
\pgfsetroundjoin%
\pgfsetlinewidth{1.204500pt}%
\definecolor{currentstroke}{rgb}{1.000000,0.576471,0.309804}%
\pgfsetstrokecolor{currentstroke}%
\pgfsetdash{}{0pt}%
\pgfpathmoveto{\pgfqpoint{5.369646in}{3.172612in}}%
\pgfusepath{stroke}%
\end{pgfscope}%
\begin{pgfscope}%
\pgfpathrectangle{\pgfqpoint{0.765000in}{0.660000in}}{\pgfqpoint{4.620000in}{4.620000in}}%
\pgfusepath{clip}%
\pgfsetbuttcap%
\pgfsetroundjoin%
\definecolor{currentfill}{rgb}{1.000000,0.576471,0.309804}%
\pgfsetfillcolor{currentfill}%
\pgfsetlinewidth{1.003750pt}%
\definecolor{currentstroke}{rgb}{1.000000,0.576471,0.309804}%
\pgfsetstrokecolor{currentstroke}%
\pgfsetdash{}{0pt}%
\pgfsys@defobject{currentmarker}{\pgfqpoint{-0.033333in}{-0.033333in}}{\pgfqpoint{0.033333in}{0.033333in}}{%
\pgfpathmoveto{\pgfqpoint{0.000000in}{-0.033333in}}%
\pgfpathcurveto{\pgfqpoint{0.008840in}{-0.033333in}}{\pgfqpoint{0.017319in}{-0.029821in}}{\pgfqpoint{0.023570in}{-0.023570in}}%
\pgfpathcurveto{\pgfqpoint{0.029821in}{-0.017319in}}{\pgfqpoint{0.033333in}{-0.008840in}}{\pgfqpoint{0.033333in}{0.000000in}}%
\pgfpathcurveto{\pgfqpoint{0.033333in}{0.008840in}}{\pgfqpoint{0.029821in}{0.017319in}}{\pgfqpoint{0.023570in}{0.023570in}}%
\pgfpathcurveto{\pgfqpoint{0.017319in}{0.029821in}}{\pgfqpoint{0.008840in}{0.033333in}}{\pgfqpoint{0.000000in}{0.033333in}}%
\pgfpathcurveto{\pgfqpoint{-0.008840in}{0.033333in}}{\pgfqpoint{-0.017319in}{0.029821in}}{\pgfqpoint{-0.023570in}{0.023570in}}%
\pgfpathcurveto{\pgfqpoint{-0.029821in}{0.017319in}}{\pgfqpoint{-0.033333in}{0.008840in}}{\pgfqpoint{-0.033333in}{0.000000in}}%
\pgfpathcurveto{\pgfqpoint{-0.033333in}{-0.008840in}}{\pgfqpoint{-0.029821in}{-0.017319in}}{\pgfqpoint{-0.023570in}{-0.023570in}}%
\pgfpathcurveto{\pgfqpoint{-0.017319in}{-0.029821in}}{\pgfqpoint{-0.008840in}{-0.033333in}}{\pgfqpoint{0.000000in}{-0.033333in}}%
\pgfpathlineto{\pgfqpoint{0.000000in}{-0.033333in}}%
\pgfpathclose%
\pgfusepath{stroke,fill}%
}%
\begin{pgfscope}%
\pgfsys@transformshift{5.369646in}{3.172612in}%
\pgfsys@useobject{currentmarker}{}%
\end{pgfscope}%
\end{pgfscope}%
\begin{pgfscope}%
\pgfpathrectangle{\pgfqpoint{0.765000in}{0.660000in}}{\pgfqpoint{4.620000in}{4.620000in}}%
\pgfusepath{clip}%
\pgfsetrectcap%
\pgfsetroundjoin%
\pgfsetlinewidth{1.204500pt}%
\definecolor{currentstroke}{rgb}{1.000000,0.576471,0.309804}%
\pgfsetstrokecolor{currentstroke}%
\pgfsetdash{}{0pt}%
\pgfpathmoveto{\pgfqpoint{3.995413in}{2.808234in}}%
\pgfusepath{stroke}%
\end{pgfscope}%
\begin{pgfscope}%
\pgfpathrectangle{\pgfqpoint{0.765000in}{0.660000in}}{\pgfqpoint{4.620000in}{4.620000in}}%
\pgfusepath{clip}%
\pgfsetbuttcap%
\pgfsetroundjoin%
\definecolor{currentfill}{rgb}{1.000000,0.576471,0.309804}%
\pgfsetfillcolor{currentfill}%
\pgfsetlinewidth{1.003750pt}%
\definecolor{currentstroke}{rgb}{1.000000,0.576471,0.309804}%
\pgfsetstrokecolor{currentstroke}%
\pgfsetdash{}{0pt}%
\pgfsys@defobject{currentmarker}{\pgfqpoint{-0.033333in}{-0.033333in}}{\pgfqpoint{0.033333in}{0.033333in}}{%
\pgfpathmoveto{\pgfqpoint{0.000000in}{-0.033333in}}%
\pgfpathcurveto{\pgfqpoint{0.008840in}{-0.033333in}}{\pgfqpoint{0.017319in}{-0.029821in}}{\pgfqpoint{0.023570in}{-0.023570in}}%
\pgfpathcurveto{\pgfqpoint{0.029821in}{-0.017319in}}{\pgfqpoint{0.033333in}{-0.008840in}}{\pgfqpoint{0.033333in}{0.000000in}}%
\pgfpathcurveto{\pgfqpoint{0.033333in}{0.008840in}}{\pgfqpoint{0.029821in}{0.017319in}}{\pgfqpoint{0.023570in}{0.023570in}}%
\pgfpathcurveto{\pgfqpoint{0.017319in}{0.029821in}}{\pgfqpoint{0.008840in}{0.033333in}}{\pgfqpoint{0.000000in}{0.033333in}}%
\pgfpathcurveto{\pgfqpoint{-0.008840in}{0.033333in}}{\pgfqpoint{-0.017319in}{0.029821in}}{\pgfqpoint{-0.023570in}{0.023570in}}%
\pgfpathcurveto{\pgfqpoint{-0.029821in}{0.017319in}}{\pgfqpoint{-0.033333in}{0.008840in}}{\pgfqpoint{-0.033333in}{0.000000in}}%
\pgfpathcurveto{\pgfqpoint{-0.033333in}{-0.008840in}}{\pgfqpoint{-0.029821in}{-0.017319in}}{\pgfqpoint{-0.023570in}{-0.023570in}}%
\pgfpathcurveto{\pgfqpoint{-0.017319in}{-0.029821in}}{\pgfqpoint{-0.008840in}{-0.033333in}}{\pgfqpoint{0.000000in}{-0.033333in}}%
\pgfpathlineto{\pgfqpoint{0.000000in}{-0.033333in}}%
\pgfpathclose%
\pgfusepath{stroke,fill}%
}%
\begin{pgfscope}%
\pgfsys@transformshift{3.995413in}{2.808234in}%
\pgfsys@useobject{currentmarker}{}%
\end{pgfscope}%
\end{pgfscope}%
\begin{pgfscope}%
\pgfpathrectangle{\pgfqpoint{0.765000in}{0.660000in}}{\pgfqpoint{4.620000in}{4.620000in}}%
\pgfusepath{clip}%
\pgfsetrectcap%
\pgfsetroundjoin%
\pgfsetlinewidth{1.204500pt}%
\definecolor{currentstroke}{rgb}{1.000000,0.576471,0.309804}%
\pgfsetstrokecolor{currentstroke}%
\pgfsetdash{}{0pt}%
\pgfpathmoveto{\pgfqpoint{3.214844in}{2.619659in}}%
\pgfusepath{stroke}%
\end{pgfscope}%
\begin{pgfscope}%
\pgfpathrectangle{\pgfqpoint{0.765000in}{0.660000in}}{\pgfqpoint{4.620000in}{4.620000in}}%
\pgfusepath{clip}%
\pgfsetbuttcap%
\pgfsetroundjoin%
\definecolor{currentfill}{rgb}{1.000000,0.576471,0.309804}%
\pgfsetfillcolor{currentfill}%
\pgfsetlinewidth{1.003750pt}%
\definecolor{currentstroke}{rgb}{1.000000,0.576471,0.309804}%
\pgfsetstrokecolor{currentstroke}%
\pgfsetdash{}{0pt}%
\pgfsys@defobject{currentmarker}{\pgfqpoint{-0.033333in}{-0.033333in}}{\pgfqpoint{0.033333in}{0.033333in}}{%
\pgfpathmoveto{\pgfqpoint{0.000000in}{-0.033333in}}%
\pgfpathcurveto{\pgfqpoint{0.008840in}{-0.033333in}}{\pgfqpoint{0.017319in}{-0.029821in}}{\pgfqpoint{0.023570in}{-0.023570in}}%
\pgfpathcurveto{\pgfqpoint{0.029821in}{-0.017319in}}{\pgfqpoint{0.033333in}{-0.008840in}}{\pgfqpoint{0.033333in}{0.000000in}}%
\pgfpathcurveto{\pgfqpoint{0.033333in}{0.008840in}}{\pgfqpoint{0.029821in}{0.017319in}}{\pgfqpoint{0.023570in}{0.023570in}}%
\pgfpathcurveto{\pgfqpoint{0.017319in}{0.029821in}}{\pgfqpoint{0.008840in}{0.033333in}}{\pgfqpoint{0.000000in}{0.033333in}}%
\pgfpathcurveto{\pgfqpoint{-0.008840in}{0.033333in}}{\pgfqpoint{-0.017319in}{0.029821in}}{\pgfqpoint{-0.023570in}{0.023570in}}%
\pgfpathcurveto{\pgfqpoint{-0.029821in}{0.017319in}}{\pgfqpoint{-0.033333in}{0.008840in}}{\pgfqpoint{-0.033333in}{0.000000in}}%
\pgfpathcurveto{\pgfqpoint{-0.033333in}{-0.008840in}}{\pgfqpoint{-0.029821in}{-0.017319in}}{\pgfqpoint{-0.023570in}{-0.023570in}}%
\pgfpathcurveto{\pgfqpoint{-0.017319in}{-0.029821in}}{\pgfqpoint{-0.008840in}{-0.033333in}}{\pgfqpoint{0.000000in}{-0.033333in}}%
\pgfpathlineto{\pgfqpoint{0.000000in}{-0.033333in}}%
\pgfpathclose%
\pgfusepath{stroke,fill}%
}%
\begin{pgfscope}%
\pgfsys@transformshift{3.214844in}{2.619659in}%
\pgfsys@useobject{currentmarker}{}%
\end{pgfscope}%
\end{pgfscope}%
\begin{pgfscope}%
\pgfpathrectangle{\pgfqpoint{0.765000in}{0.660000in}}{\pgfqpoint{4.620000in}{4.620000in}}%
\pgfusepath{clip}%
\pgfsetrectcap%
\pgfsetroundjoin%
\pgfsetlinewidth{1.204500pt}%
\definecolor{currentstroke}{rgb}{1.000000,0.576471,0.309804}%
\pgfsetstrokecolor{currentstroke}%
\pgfsetdash{}{0pt}%
\pgfpathmoveto{\pgfqpoint{2.840712in}{2.612490in}}%
\pgfusepath{stroke}%
\end{pgfscope}%
\begin{pgfscope}%
\pgfpathrectangle{\pgfqpoint{0.765000in}{0.660000in}}{\pgfqpoint{4.620000in}{4.620000in}}%
\pgfusepath{clip}%
\pgfsetbuttcap%
\pgfsetroundjoin%
\definecolor{currentfill}{rgb}{1.000000,0.576471,0.309804}%
\pgfsetfillcolor{currentfill}%
\pgfsetlinewidth{1.003750pt}%
\definecolor{currentstroke}{rgb}{1.000000,0.576471,0.309804}%
\pgfsetstrokecolor{currentstroke}%
\pgfsetdash{}{0pt}%
\pgfsys@defobject{currentmarker}{\pgfqpoint{-0.033333in}{-0.033333in}}{\pgfqpoint{0.033333in}{0.033333in}}{%
\pgfpathmoveto{\pgfqpoint{0.000000in}{-0.033333in}}%
\pgfpathcurveto{\pgfqpoint{0.008840in}{-0.033333in}}{\pgfqpoint{0.017319in}{-0.029821in}}{\pgfqpoint{0.023570in}{-0.023570in}}%
\pgfpathcurveto{\pgfqpoint{0.029821in}{-0.017319in}}{\pgfqpoint{0.033333in}{-0.008840in}}{\pgfqpoint{0.033333in}{0.000000in}}%
\pgfpathcurveto{\pgfqpoint{0.033333in}{0.008840in}}{\pgfqpoint{0.029821in}{0.017319in}}{\pgfqpoint{0.023570in}{0.023570in}}%
\pgfpathcurveto{\pgfqpoint{0.017319in}{0.029821in}}{\pgfqpoint{0.008840in}{0.033333in}}{\pgfqpoint{0.000000in}{0.033333in}}%
\pgfpathcurveto{\pgfqpoint{-0.008840in}{0.033333in}}{\pgfqpoint{-0.017319in}{0.029821in}}{\pgfqpoint{-0.023570in}{0.023570in}}%
\pgfpathcurveto{\pgfqpoint{-0.029821in}{0.017319in}}{\pgfqpoint{-0.033333in}{0.008840in}}{\pgfqpoint{-0.033333in}{0.000000in}}%
\pgfpathcurveto{\pgfqpoint{-0.033333in}{-0.008840in}}{\pgfqpoint{-0.029821in}{-0.017319in}}{\pgfqpoint{-0.023570in}{-0.023570in}}%
\pgfpathcurveto{\pgfqpoint{-0.017319in}{-0.029821in}}{\pgfqpoint{-0.008840in}{-0.033333in}}{\pgfqpoint{0.000000in}{-0.033333in}}%
\pgfpathlineto{\pgfqpoint{0.000000in}{-0.033333in}}%
\pgfpathclose%
\pgfusepath{stroke,fill}%
}%
\begin{pgfscope}%
\pgfsys@transformshift{2.840712in}{2.612490in}%
\pgfsys@useobject{currentmarker}{}%
\end{pgfscope}%
\end{pgfscope}%
\begin{pgfscope}%
\pgfpathrectangle{\pgfqpoint{0.765000in}{0.660000in}}{\pgfqpoint{4.620000in}{4.620000in}}%
\pgfusepath{clip}%
\pgfsetrectcap%
\pgfsetroundjoin%
\pgfsetlinewidth{1.204500pt}%
\definecolor{currentstroke}{rgb}{1.000000,0.576471,0.309804}%
\pgfsetstrokecolor{currentstroke}%
\pgfsetdash{}{0pt}%
\pgfpathmoveto{\pgfqpoint{4.655930in}{2.880516in}}%
\pgfusepath{stroke}%
\end{pgfscope}%
\begin{pgfscope}%
\pgfpathrectangle{\pgfqpoint{0.765000in}{0.660000in}}{\pgfqpoint{4.620000in}{4.620000in}}%
\pgfusepath{clip}%
\pgfsetbuttcap%
\pgfsetroundjoin%
\definecolor{currentfill}{rgb}{1.000000,0.576471,0.309804}%
\pgfsetfillcolor{currentfill}%
\pgfsetlinewidth{1.003750pt}%
\definecolor{currentstroke}{rgb}{1.000000,0.576471,0.309804}%
\pgfsetstrokecolor{currentstroke}%
\pgfsetdash{}{0pt}%
\pgfsys@defobject{currentmarker}{\pgfqpoint{-0.033333in}{-0.033333in}}{\pgfqpoint{0.033333in}{0.033333in}}{%
\pgfpathmoveto{\pgfqpoint{0.000000in}{-0.033333in}}%
\pgfpathcurveto{\pgfqpoint{0.008840in}{-0.033333in}}{\pgfqpoint{0.017319in}{-0.029821in}}{\pgfqpoint{0.023570in}{-0.023570in}}%
\pgfpathcurveto{\pgfqpoint{0.029821in}{-0.017319in}}{\pgfqpoint{0.033333in}{-0.008840in}}{\pgfqpoint{0.033333in}{0.000000in}}%
\pgfpathcurveto{\pgfqpoint{0.033333in}{0.008840in}}{\pgfqpoint{0.029821in}{0.017319in}}{\pgfqpoint{0.023570in}{0.023570in}}%
\pgfpathcurveto{\pgfqpoint{0.017319in}{0.029821in}}{\pgfqpoint{0.008840in}{0.033333in}}{\pgfqpoint{0.000000in}{0.033333in}}%
\pgfpathcurveto{\pgfqpoint{-0.008840in}{0.033333in}}{\pgfqpoint{-0.017319in}{0.029821in}}{\pgfqpoint{-0.023570in}{0.023570in}}%
\pgfpathcurveto{\pgfqpoint{-0.029821in}{0.017319in}}{\pgfqpoint{-0.033333in}{0.008840in}}{\pgfqpoint{-0.033333in}{0.000000in}}%
\pgfpathcurveto{\pgfqpoint{-0.033333in}{-0.008840in}}{\pgfqpoint{-0.029821in}{-0.017319in}}{\pgfqpoint{-0.023570in}{-0.023570in}}%
\pgfpathcurveto{\pgfqpoint{-0.017319in}{-0.029821in}}{\pgfqpoint{-0.008840in}{-0.033333in}}{\pgfqpoint{0.000000in}{-0.033333in}}%
\pgfpathlineto{\pgfqpoint{0.000000in}{-0.033333in}}%
\pgfpathclose%
\pgfusepath{stroke,fill}%
}%
\begin{pgfscope}%
\pgfsys@transformshift{4.655930in}{2.880516in}%
\pgfsys@useobject{currentmarker}{}%
\end{pgfscope}%
\end{pgfscope}%
\begin{pgfscope}%
\pgfpathrectangle{\pgfqpoint{0.765000in}{0.660000in}}{\pgfqpoint{4.620000in}{4.620000in}}%
\pgfusepath{clip}%
\pgfsetrectcap%
\pgfsetroundjoin%
\pgfsetlinewidth{1.204500pt}%
\definecolor{currentstroke}{rgb}{1.000000,0.576471,0.309804}%
\pgfsetstrokecolor{currentstroke}%
\pgfsetdash{}{0pt}%
\pgfpathmoveto{\pgfqpoint{2.981139in}{3.132003in}}%
\pgfusepath{stroke}%
\end{pgfscope}%
\begin{pgfscope}%
\pgfpathrectangle{\pgfqpoint{0.765000in}{0.660000in}}{\pgfqpoint{4.620000in}{4.620000in}}%
\pgfusepath{clip}%
\pgfsetbuttcap%
\pgfsetroundjoin%
\definecolor{currentfill}{rgb}{1.000000,0.576471,0.309804}%
\pgfsetfillcolor{currentfill}%
\pgfsetlinewidth{1.003750pt}%
\definecolor{currentstroke}{rgb}{1.000000,0.576471,0.309804}%
\pgfsetstrokecolor{currentstroke}%
\pgfsetdash{}{0pt}%
\pgfsys@defobject{currentmarker}{\pgfqpoint{-0.033333in}{-0.033333in}}{\pgfqpoint{0.033333in}{0.033333in}}{%
\pgfpathmoveto{\pgfqpoint{0.000000in}{-0.033333in}}%
\pgfpathcurveto{\pgfqpoint{0.008840in}{-0.033333in}}{\pgfqpoint{0.017319in}{-0.029821in}}{\pgfqpoint{0.023570in}{-0.023570in}}%
\pgfpathcurveto{\pgfqpoint{0.029821in}{-0.017319in}}{\pgfqpoint{0.033333in}{-0.008840in}}{\pgfqpoint{0.033333in}{0.000000in}}%
\pgfpathcurveto{\pgfqpoint{0.033333in}{0.008840in}}{\pgfqpoint{0.029821in}{0.017319in}}{\pgfqpoint{0.023570in}{0.023570in}}%
\pgfpathcurveto{\pgfqpoint{0.017319in}{0.029821in}}{\pgfqpoint{0.008840in}{0.033333in}}{\pgfqpoint{0.000000in}{0.033333in}}%
\pgfpathcurveto{\pgfqpoint{-0.008840in}{0.033333in}}{\pgfqpoint{-0.017319in}{0.029821in}}{\pgfqpoint{-0.023570in}{0.023570in}}%
\pgfpathcurveto{\pgfqpoint{-0.029821in}{0.017319in}}{\pgfqpoint{-0.033333in}{0.008840in}}{\pgfqpoint{-0.033333in}{0.000000in}}%
\pgfpathcurveto{\pgfqpoint{-0.033333in}{-0.008840in}}{\pgfqpoint{-0.029821in}{-0.017319in}}{\pgfqpoint{-0.023570in}{-0.023570in}}%
\pgfpathcurveto{\pgfqpoint{-0.017319in}{-0.029821in}}{\pgfqpoint{-0.008840in}{-0.033333in}}{\pgfqpoint{0.000000in}{-0.033333in}}%
\pgfpathlineto{\pgfqpoint{0.000000in}{-0.033333in}}%
\pgfpathclose%
\pgfusepath{stroke,fill}%
}%
\begin{pgfscope}%
\pgfsys@transformshift{2.981139in}{3.132003in}%
\pgfsys@useobject{currentmarker}{}%
\end{pgfscope}%
\end{pgfscope}%
\begin{pgfscope}%
\pgfpathrectangle{\pgfqpoint{0.765000in}{0.660000in}}{\pgfqpoint{4.620000in}{4.620000in}}%
\pgfusepath{clip}%
\pgfsetrectcap%
\pgfsetroundjoin%
\pgfsetlinewidth{1.204500pt}%
\definecolor{currentstroke}{rgb}{1.000000,0.576471,0.309804}%
\pgfsetstrokecolor{currentstroke}%
\pgfsetdash{}{0pt}%
\pgfpathmoveto{\pgfqpoint{2.938390in}{3.094295in}}%
\pgfusepath{stroke}%
\end{pgfscope}%
\begin{pgfscope}%
\pgfpathrectangle{\pgfqpoint{0.765000in}{0.660000in}}{\pgfqpoint{4.620000in}{4.620000in}}%
\pgfusepath{clip}%
\pgfsetbuttcap%
\pgfsetroundjoin%
\definecolor{currentfill}{rgb}{1.000000,0.576471,0.309804}%
\pgfsetfillcolor{currentfill}%
\pgfsetlinewidth{1.003750pt}%
\definecolor{currentstroke}{rgb}{1.000000,0.576471,0.309804}%
\pgfsetstrokecolor{currentstroke}%
\pgfsetdash{}{0pt}%
\pgfsys@defobject{currentmarker}{\pgfqpoint{-0.033333in}{-0.033333in}}{\pgfqpoint{0.033333in}{0.033333in}}{%
\pgfpathmoveto{\pgfqpoint{0.000000in}{-0.033333in}}%
\pgfpathcurveto{\pgfqpoint{0.008840in}{-0.033333in}}{\pgfqpoint{0.017319in}{-0.029821in}}{\pgfqpoint{0.023570in}{-0.023570in}}%
\pgfpathcurveto{\pgfqpoint{0.029821in}{-0.017319in}}{\pgfqpoint{0.033333in}{-0.008840in}}{\pgfqpoint{0.033333in}{0.000000in}}%
\pgfpathcurveto{\pgfqpoint{0.033333in}{0.008840in}}{\pgfqpoint{0.029821in}{0.017319in}}{\pgfqpoint{0.023570in}{0.023570in}}%
\pgfpathcurveto{\pgfqpoint{0.017319in}{0.029821in}}{\pgfqpoint{0.008840in}{0.033333in}}{\pgfqpoint{0.000000in}{0.033333in}}%
\pgfpathcurveto{\pgfqpoint{-0.008840in}{0.033333in}}{\pgfqpoint{-0.017319in}{0.029821in}}{\pgfqpoint{-0.023570in}{0.023570in}}%
\pgfpathcurveto{\pgfqpoint{-0.029821in}{0.017319in}}{\pgfqpoint{-0.033333in}{0.008840in}}{\pgfqpoint{-0.033333in}{0.000000in}}%
\pgfpathcurveto{\pgfqpoint{-0.033333in}{-0.008840in}}{\pgfqpoint{-0.029821in}{-0.017319in}}{\pgfqpoint{-0.023570in}{-0.023570in}}%
\pgfpathcurveto{\pgfqpoint{-0.017319in}{-0.029821in}}{\pgfqpoint{-0.008840in}{-0.033333in}}{\pgfqpoint{0.000000in}{-0.033333in}}%
\pgfpathlineto{\pgfqpoint{0.000000in}{-0.033333in}}%
\pgfpathclose%
\pgfusepath{stroke,fill}%
}%
\begin{pgfscope}%
\pgfsys@transformshift{2.938390in}{3.094295in}%
\pgfsys@useobject{currentmarker}{}%
\end{pgfscope}%
\end{pgfscope}%
\begin{pgfscope}%
\pgfpathrectangle{\pgfqpoint{0.765000in}{0.660000in}}{\pgfqpoint{4.620000in}{4.620000in}}%
\pgfusepath{clip}%
\pgfsetrectcap%
\pgfsetroundjoin%
\pgfsetlinewidth{1.204500pt}%
\definecolor{currentstroke}{rgb}{1.000000,0.576471,0.309804}%
\pgfsetstrokecolor{currentstroke}%
\pgfsetdash{}{0pt}%
\pgfpathmoveto{\pgfqpoint{2.751359in}{3.081578in}}%
\pgfusepath{stroke}%
\end{pgfscope}%
\begin{pgfscope}%
\pgfpathrectangle{\pgfqpoint{0.765000in}{0.660000in}}{\pgfqpoint{4.620000in}{4.620000in}}%
\pgfusepath{clip}%
\pgfsetbuttcap%
\pgfsetroundjoin%
\definecolor{currentfill}{rgb}{1.000000,0.576471,0.309804}%
\pgfsetfillcolor{currentfill}%
\pgfsetlinewidth{1.003750pt}%
\definecolor{currentstroke}{rgb}{1.000000,0.576471,0.309804}%
\pgfsetstrokecolor{currentstroke}%
\pgfsetdash{}{0pt}%
\pgfsys@defobject{currentmarker}{\pgfqpoint{-0.033333in}{-0.033333in}}{\pgfqpoint{0.033333in}{0.033333in}}{%
\pgfpathmoveto{\pgfqpoint{0.000000in}{-0.033333in}}%
\pgfpathcurveto{\pgfqpoint{0.008840in}{-0.033333in}}{\pgfqpoint{0.017319in}{-0.029821in}}{\pgfqpoint{0.023570in}{-0.023570in}}%
\pgfpathcurveto{\pgfqpoint{0.029821in}{-0.017319in}}{\pgfqpoint{0.033333in}{-0.008840in}}{\pgfqpoint{0.033333in}{0.000000in}}%
\pgfpathcurveto{\pgfqpoint{0.033333in}{0.008840in}}{\pgfqpoint{0.029821in}{0.017319in}}{\pgfqpoint{0.023570in}{0.023570in}}%
\pgfpathcurveto{\pgfqpoint{0.017319in}{0.029821in}}{\pgfqpoint{0.008840in}{0.033333in}}{\pgfqpoint{0.000000in}{0.033333in}}%
\pgfpathcurveto{\pgfqpoint{-0.008840in}{0.033333in}}{\pgfqpoint{-0.017319in}{0.029821in}}{\pgfqpoint{-0.023570in}{0.023570in}}%
\pgfpathcurveto{\pgfqpoint{-0.029821in}{0.017319in}}{\pgfqpoint{-0.033333in}{0.008840in}}{\pgfqpoint{-0.033333in}{0.000000in}}%
\pgfpathcurveto{\pgfqpoint{-0.033333in}{-0.008840in}}{\pgfqpoint{-0.029821in}{-0.017319in}}{\pgfqpoint{-0.023570in}{-0.023570in}}%
\pgfpathcurveto{\pgfqpoint{-0.017319in}{-0.029821in}}{\pgfqpoint{-0.008840in}{-0.033333in}}{\pgfqpoint{0.000000in}{-0.033333in}}%
\pgfpathlineto{\pgfqpoint{0.000000in}{-0.033333in}}%
\pgfpathclose%
\pgfusepath{stroke,fill}%
}%
\begin{pgfscope}%
\pgfsys@transformshift{2.751359in}{3.081578in}%
\pgfsys@useobject{currentmarker}{}%
\end{pgfscope}%
\end{pgfscope}%
\begin{pgfscope}%
\pgfpathrectangle{\pgfqpoint{0.765000in}{0.660000in}}{\pgfqpoint{4.620000in}{4.620000in}}%
\pgfusepath{clip}%
\pgfsetrectcap%
\pgfsetroundjoin%
\pgfsetlinewidth{1.204500pt}%
\definecolor{currentstroke}{rgb}{1.000000,0.576471,0.309804}%
\pgfsetstrokecolor{currentstroke}%
\pgfsetdash{}{0pt}%
\pgfpathmoveto{\pgfqpoint{3.588569in}{2.415841in}}%
\pgfusepath{stroke}%
\end{pgfscope}%
\begin{pgfscope}%
\pgfpathrectangle{\pgfqpoint{0.765000in}{0.660000in}}{\pgfqpoint{4.620000in}{4.620000in}}%
\pgfusepath{clip}%
\pgfsetbuttcap%
\pgfsetroundjoin%
\definecolor{currentfill}{rgb}{1.000000,0.576471,0.309804}%
\pgfsetfillcolor{currentfill}%
\pgfsetlinewidth{1.003750pt}%
\definecolor{currentstroke}{rgb}{1.000000,0.576471,0.309804}%
\pgfsetstrokecolor{currentstroke}%
\pgfsetdash{}{0pt}%
\pgfsys@defobject{currentmarker}{\pgfqpoint{-0.033333in}{-0.033333in}}{\pgfqpoint{0.033333in}{0.033333in}}{%
\pgfpathmoveto{\pgfqpoint{0.000000in}{-0.033333in}}%
\pgfpathcurveto{\pgfqpoint{0.008840in}{-0.033333in}}{\pgfqpoint{0.017319in}{-0.029821in}}{\pgfqpoint{0.023570in}{-0.023570in}}%
\pgfpathcurveto{\pgfqpoint{0.029821in}{-0.017319in}}{\pgfqpoint{0.033333in}{-0.008840in}}{\pgfqpoint{0.033333in}{0.000000in}}%
\pgfpathcurveto{\pgfqpoint{0.033333in}{0.008840in}}{\pgfqpoint{0.029821in}{0.017319in}}{\pgfqpoint{0.023570in}{0.023570in}}%
\pgfpathcurveto{\pgfqpoint{0.017319in}{0.029821in}}{\pgfqpoint{0.008840in}{0.033333in}}{\pgfqpoint{0.000000in}{0.033333in}}%
\pgfpathcurveto{\pgfqpoint{-0.008840in}{0.033333in}}{\pgfqpoint{-0.017319in}{0.029821in}}{\pgfqpoint{-0.023570in}{0.023570in}}%
\pgfpathcurveto{\pgfqpoint{-0.029821in}{0.017319in}}{\pgfqpoint{-0.033333in}{0.008840in}}{\pgfqpoint{-0.033333in}{0.000000in}}%
\pgfpathcurveto{\pgfqpoint{-0.033333in}{-0.008840in}}{\pgfqpoint{-0.029821in}{-0.017319in}}{\pgfqpoint{-0.023570in}{-0.023570in}}%
\pgfpathcurveto{\pgfqpoint{-0.017319in}{-0.029821in}}{\pgfqpoint{-0.008840in}{-0.033333in}}{\pgfqpoint{0.000000in}{-0.033333in}}%
\pgfpathlineto{\pgfqpoint{0.000000in}{-0.033333in}}%
\pgfpathclose%
\pgfusepath{stroke,fill}%
}%
\begin{pgfscope}%
\pgfsys@transformshift{3.588569in}{2.415841in}%
\pgfsys@useobject{currentmarker}{}%
\end{pgfscope}%
\end{pgfscope}%
\begin{pgfscope}%
\pgfpathrectangle{\pgfqpoint{0.765000in}{0.660000in}}{\pgfqpoint{4.620000in}{4.620000in}}%
\pgfusepath{clip}%
\pgfsetrectcap%
\pgfsetroundjoin%
\pgfsetlinewidth{1.204500pt}%
\definecolor{currentstroke}{rgb}{1.000000,0.576471,0.309804}%
\pgfsetstrokecolor{currentstroke}%
\pgfsetdash{}{0pt}%
\pgfpathmoveto{\pgfqpoint{2.987670in}{2.528670in}}%
\pgfusepath{stroke}%
\end{pgfscope}%
\begin{pgfscope}%
\pgfpathrectangle{\pgfqpoint{0.765000in}{0.660000in}}{\pgfqpoint{4.620000in}{4.620000in}}%
\pgfusepath{clip}%
\pgfsetbuttcap%
\pgfsetroundjoin%
\definecolor{currentfill}{rgb}{1.000000,0.576471,0.309804}%
\pgfsetfillcolor{currentfill}%
\pgfsetlinewidth{1.003750pt}%
\definecolor{currentstroke}{rgb}{1.000000,0.576471,0.309804}%
\pgfsetstrokecolor{currentstroke}%
\pgfsetdash{}{0pt}%
\pgfsys@defobject{currentmarker}{\pgfqpoint{-0.033333in}{-0.033333in}}{\pgfqpoint{0.033333in}{0.033333in}}{%
\pgfpathmoveto{\pgfqpoint{0.000000in}{-0.033333in}}%
\pgfpathcurveto{\pgfqpoint{0.008840in}{-0.033333in}}{\pgfqpoint{0.017319in}{-0.029821in}}{\pgfqpoint{0.023570in}{-0.023570in}}%
\pgfpathcurveto{\pgfqpoint{0.029821in}{-0.017319in}}{\pgfqpoint{0.033333in}{-0.008840in}}{\pgfqpoint{0.033333in}{0.000000in}}%
\pgfpathcurveto{\pgfqpoint{0.033333in}{0.008840in}}{\pgfqpoint{0.029821in}{0.017319in}}{\pgfqpoint{0.023570in}{0.023570in}}%
\pgfpathcurveto{\pgfqpoint{0.017319in}{0.029821in}}{\pgfqpoint{0.008840in}{0.033333in}}{\pgfqpoint{0.000000in}{0.033333in}}%
\pgfpathcurveto{\pgfqpoint{-0.008840in}{0.033333in}}{\pgfqpoint{-0.017319in}{0.029821in}}{\pgfqpoint{-0.023570in}{0.023570in}}%
\pgfpathcurveto{\pgfqpoint{-0.029821in}{0.017319in}}{\pgfqpoint{-0.033333in}{0.008840in}}{\pgfqpoint{-0.033333in}{0.000000in}}%
\pgfpathcurveto{\pgfqpoint{-0.033333in}{-0.008840in}}{\pgfqpoint{-0.029821in}{-0.017319in}}{\pgfqpoint{-0.023570in}{-0.023570in}}%
\pgfpathcurveto{\pgfqpoint{-0.017319in}{-0.029821in}}{\pgfqpoint{-0.008840in}{-0.033333in}}{\pgfqpoint{0.000000in}{-0.033333in}}%
\pgfpathlineto{\pgfqpoint{0.000000in}{-0.033333in}}%
\pgfpathclose%
\pgfusepath{stroke,fill}%
}%
\begin{pgfscope}%
\pgfsys@transformshift{2.987670in}{2.528670in}%
\pgfsys@useobject{currentmarker}{}%
\end{pgfscope}%
\end{pgfscope}%
\begin{pgfscope}%
\pgfpathrectangle{\pgfqpoint{0.765000in}{0.660000in}}{\pgfqpoint{4.620000in}{4.620000in}}%
\pgfusepath{clip}%
\pgfsetrectcap%
\pgfsetroundjoin%
\pgfsetlinewidth{1.204500pt}%
\definecolor{currentstroke}{rgb}{1.000000,0.576471,0.309804}%
\pgfsetstrokecolor{currentstroke}%
\pgfsetdash{}{0pt}%
\pgfpathmoveto{\pgfqpoint{4.969267in}{3.109766in}}%
\pgfusepath{stroke}%
\end{pgfscope}%
\begin{pgfscope}%
\pgfpathrectangle{\pgfqpoint{0.765000in}{0.660000in}}{\pgfqpoint{4.620000in}{4.620000in}}%
\pgfusepath{clip}%
\pgfsetbuttcap%
\pgfsetroundjoin%
\definecolor{currentfill}{rgb}{1.000000,0.576471,0.309804}%
\pgfsetfillcolor{currentfill}%
\pgfsetlinewidth{1.003750pt}%
\definecolor{currentstroke}{rgb}{1.000000,0.576471,0.309804}%
\pgfsetstrokecolor{currentstroke}%
\pgfsetdash{}{0pt}%
\pgfsys@defobject{currentmarker}{\pgfqpoint{-0.033333in}{-0.033333in}}{\pgfqpoint{0.033333in}{0.033333in}}{%
\pgfpathmoveto{\pgfqpoint{0.000000in}{-0.033333in}}%
\pgfpathcurveto{\pgfqpoint{0.008840in}{-0.033333in}}{\pgfqpoint{0.017319in}{-0.029821in}}{\pgfqpoint{0.023570in}{-0.023570in}}%
\pgfpathcurveto{\pgfqpoint{0.029821in}{-0.017319in}}{\pgfqpoint{0.033333in}{-0.008840in}}{\pgfqpoint{0.033333in}{0.000000in}}%
\pgfpathcurveto{\pgfqpoint{0.033333in}{0.008840in}}{\pgfqpoint{0.029821in}{0.017319in}}{\pgfqpoint{0.023570in}{0.023570in}}%
\pgfpathcurveto{\pgfqpoint{0.017319in}{0.029821in}}{\pgfqpoint{0.008840in}{0.033333in}}{\pgfqpoint{0.000000in}{0.033333in}}%
\pgfpathcurveto{\pgfqpoint{-0.008840in}{0.033333in}}{\pgfqpoint{-0.017319in}{0.029821in}}{\pgfqpoint{-0.023570in}{0.023570in}}%
\pgfpathcurveto{\pgfqpoint{-0.029821in}{0.017319in}}{\pgfqpoint{-0.033333in}{0.008840in}}{\pgfqpoint{-0.033333in}{0.000000in}}%
\pgfpathcurveto{\pgfqpoint{-0.033333in}{-0.008840in}}{\pgfqpoint{-0.029821in}{-0.017319in}}{\pgfqpoint{-0.023570in}{-0.023570in}}%
\pgfpathcurveto{\pgfqpoint{-0.017319in}{-0.029821in}}{\pgfqpoint{-0.008840in}{-0.033333in}}{\pgfqpoint{0.000000in}{-0.033333in}}%
\pgfpathlineto{\pgfqpoint{0.000000in}{-0.033333in}}%
\pgfpathclose%
\pgfusepath{stroke,fill}%
}%
\begin{pgfscope}%
\pgfsys@transformshift{4.969267in}{3.109766in}%
\pgfsys@useobject{currentmarker}{}%
\end{pgfscope}%
\end{pgfscope}%
\begin{pgfscope}%
\pgfpathrectangle{\pgfqpoint{0.765000in}{0.660000in}}{\pgfqpoint{4.620000in}{4.620000in}}%
\pgfusepath{clip}%
\pgfsetrectcap%
\pgfsetroundjoin%
\pgfsetlinewidth{1.204500pt}%
\definecolor{currentstroke}{rgb}{1.000000,0.576471,0.309804}%
\pgfsetstrokecolor{currentstroke}%
\pgfsetdash{}{0pt}%
\pgfpathmoveto{\pgfqpoint{2.565224in}{2.574721in}}%
\pgfusepath{stroke}%
\end{pgfscope}%
\begin{pgfscope}%
\pgfpathrectangle{\pgfqpoint{0.765000in}{0.660000in}}{\pgfqpoint{4.620000in}{4.620000in}}%
\pgfusepath{clip}%
\pgfsetbuttcap%
\pgfsetroundjoin%
\definecolor{currentfill}{rgb}{1.000000,0.576471,0.309804}%
\pgfsetfillcolor{currentfill}%
\pgfsetlinewidth{1.003750pt}%
\definecolor{currentstroke}{rgb}{1.000000,0.576471,0.309804}%
\pgfsetstrokecolor{currentstroke}%
\pgfsetdash{}{0pt}%
\pgfsys@defobject{currentmarker}{\pgfqpoint{-0.033333in}{-0.033333in}}{\pgfqpoint{0.033333in}{0.033333in}}{%
\pgfpathmoveto{\pgfqpoint{0.000000in}{-0.033333in}}%
\pgfpathcurveto{\pgfqpoint{0.008840in}{-0.033333in}}{\pgfqpoint{0.017319in}{-0.029821in}}{\pgfqpoint{0.023570in}{-0.023570in}}%
\pgfpathcurveto{\pgfqpoint{0.029821in}{-0.017319in}}{\pgfqpoint{0.033333in}{-0.008840in}}{\pgfqpoint{0.033333in}{0.000000in}}%
\pgfpathcurveto{\pgfqpoint{0.033333in}{0.008840in}}{\pgfqpoint{0.029821in}{0.017319in}}{\pgfqpoint{0.023570in}{0.023570in}}%
\pgfpathcurveto{\pgfqpoint{0.017319in}{0.029821in}}{\pgfqpoint{0.008840in}{0.033333in}}{\pgfqpoint{0.000000in}{0.033333in}}%
\pgfpathcurveto{\pgfqpoint{-0.008840in}{0.033333in}}{\pgfqpoint{-0.017319in}{0.029821in}}{\pgfqpoint{-0.023570in}{0.023570in}}%
\pgfpathcurveto{\pgfqpoint{-0.029821in}{0.017319in}}{\pgfqpoint{-0.033333in}{0.008840in}}{\pgfqpoint{-0.033333in}{0.000000in}}%
\pgfpathcurveto{\pgfqpoint{-0.033333in}{-0.008840in}}{\pgfqpoint{-0.029821in}{-0.017319in}}{\pgfqpoint{-0.023570in}{-0.023570in}}%
\pgfpathcurveto{\pgfqpoint{-0.017319in}{-0.029821in}}{\pgfqpoint{-0.008840in}{-0.033333in}}{\pgfqpoint{0.000000in}{-0.033333in}}%
\pgfpathlineto{\pgfqpoint{0.000000in}{-0.033333in}}%
\pgfpathclose%
\pgfusepath{stroke,fill}%
}%
\begin{pgfscope}%
\pgfsys@transformshift{2.565224in}{2.574721in}%
\pgfsys@useobject{currentmarker}{}%
\end{pgfscope}%
\end{pgfscope}%
\begin{pgfscope}%
\pgfpathrectangle{\pgfqpoint{0.765000in}{0.660000in}}{\pgfqpoint{4.620000in}{4.620000in}}%
\pgfusepath{clip}%
\pgfsetrectcap%
\pgfsetroundjoin%
\pgfsetlinewidth{1.204500pt}%
\definecolor{currentstroke}{rgb}{1.000000,0.576471,0.309804}%
\pgfsetstrokecolor{currentstroke}%
\pgfsetdash{}{0pt}%
\pgfpathmoveto{\pgfqpoint{4.583410in}{2.939705in}}%
\pgfusepath{stroke}%
\end{pgfscope}%
\begin{pgfscope}%
\pgfpathrectangle{\pgfqpoint{0.765000in}{0.660000in}}{\pgfqpoint{4.620000in}{4.620000in}}%
\pgfusepath{clip}%
\pgfsetbuttcap%
\pgfsetroundjoin%
\definecolor{currentfill}{rgb}{1.000000,0.576471,0.309804}%
\pgfsetfillcolor{currentfill}%
\pgfsetlinewidth{1.003750pt}%
\definecolor{currentstroke}{rgb}{1.000000,0.576471,0.309804}%
\pgfsetstrokecolor{currentstroke}%
\pgfsetdash{}{0pt}%
\pgfsys@defobject{currentmarker}{\pgfqpoint{-0.033333in}{-0.033333in}}{\pgfqpoint{0.033333in}{0.033333in}}{%
\pgfpathmoveto{\pgfqpoint{0.000000in}{-0.033333in}}%
\pgfpathcurveto{\pgfqpoint{0.008840in}{-0.033333in}}{\pgfqpoint{0.017319in}{-0.029821in}}{\pgfqpoint{0.023570in}{-0.023570in}}%
\pgfpathcurveto{\pgfqpoint{0.029821in}{-0.017319in}}{\pgfqpoint{0.033333in}{-0.008840in}}{\pgfqpoint{0.033333in}{0.000000in}}%
\pgfpathcurveto{\pgfqpoint{0.033333in}{0.008840in}}{\pgfqpoint{0.029821in}{0.017319in}}{\pgfqpoint{0.023570in}{0.023570in}}%
\pgfpathcurveto{\pgfqpoint{0.017319in}{0.029821in}}{\pgfqpoint{0.008840in}{0.033333in}}{\pgfqpoint{0.000000in}{0.033333in}}%
\pgfpathcurveto{\pgfqpoint{-0.008840in}{0.033333in}}{\pgfqpoint{-0.017319in}{0.029821in}}{\pgfqpoint{-0.023570in}{0.023570in}}%
\pgfpathcurveto{\pgfqpoint{-0.029821in}{0.017319in}}{\pgfqpoint{-0.033333in}{0.008840in}}{\pgfqpoint{-0.033333in}{0.000000in}}%
\pgfpathcurveto{\pgfqpoint{-0.033333in}{-0.008840in}}{\pgfqpoint{-0.029821in}{-0.017319in}}{\pgfqpoint{-0.023570in}{-0.023570in}}%
\pgfpathcurveto{\pgfqpoint{-0.017319in}{-0.029821in}}{\pgfqpoint{-0.008840in}{-0.033333in}}{\pgfqpoint{0.000000in}{-0.033333in}}%
\pgfpathlineto{\pgfqpoint{0.000000in}{-0.033333in}}%
\pgfpathclose%
\pgfusepath{stroke,fill}%
}%
\begin{pgfscope}%
\pgfsys@transformshift{4.583410in}{2.939705in}%
\pgfsys@useobject{currentmarker}{}%
\end{pgfscope}%
\end{pgfscope}%
\begin{pgfscope}%
\pgfpathrectangle{\pgfqpoint{0.765000in}{0.660000in}}{\pgfqpoint{4.620000in}{4.620000in}}%
\pgfusepath{clip}%
\pgfsetrectcap%
\pgfsetroundjoin%
\pgfsetlinewidth{1.204500pt}%
\definecolor{currentstroke}{rgb}{1.000000,0.576471,0.309804}%
\pgfsetstrokecolor{currentstroke}%
\pgfsetdash{}{0pt}%
\pgfpathmoveto{\pgfqpoint{3.307753in}{2.846877in}}%
\pgfusepath{stroke}%
\end{pgfscope}%
\begin{pgfscope}%
\pgfpathrectangle{\pgfqpoint{0.765000in}{0.660000in}}{\pgfqpoint{4.620000in}{4.620000in}}%
\pgfusepath{clip}%
\pgfsetbuttcap%
\pgfsetroundjoin%
\definecolor{currentfill}{rgb}{1.000000,0.576471,0.309804}%
\pgfsetfillcolor{currentfill}%
\pgfsetlinewidth{1.003750pt}%
\definecolor{currentstroke}{rgb}{1.000000,0.576471,0.309804}%
\pgfsetstrokecolor{currentstroke}%
\pgfsetdash{}{0pt}%
\pgfsys@defobject{currentmarker}{\pgfqpoint{-0.033333in}{-0.033333in}}{\pgfqpoint{0.033333in}{0.033333in}}{%
\pgfpathmoveto{\pgfqpoint{0.000000in}{-0.033333in}}%
\pgfpathcurveto{\pgfqpoint{0.008840in}{-0.033333in}}{\pgfqpoint{0.017319in}{-0.029821in}}{\pgfqpoint{0.023570in}{-0.023570in}}%
\pgfpathcurveto{\pgfqpoint{0.029821in}{-0.017319in}}{\pgfqpoint{0.033333in}{-0.008840in}}{\pgfqpoint{0.033333in}{0.000000in}}%
\pgfpathcurveto{\pgfqpoint{0.033333in}{0.008840in}}{\pgfqpoint{0.029821in}{0.017319in}}{\pgfqpoint{0.023570in}{0.023570in}}%
\pgfpathcurveto{\pgfqpoint{0.017319in}{0.029821in}}{\pgfqpoint{0.008840in}{0.033333in}}{\pgfqpoint{0.000000in}{0.033333in}}%
\pgfpathcurveto{\pgfqpoint{-0.008840in}{0.033333in}}{\pgfqpoint{-0.017319in}{0.029821in}}{\pgfqpoint{-0.023570in}{0.023570in}}%
\pgfpathcurveto{\pgfqpoint{-0.029821in}{0.017319in}}{\pgfqpoint{-0.033333in}{0.008840in}}{\pgfqpoint{-0.033333in}{0.000000in}}%
\pgfpathcurveto{\pgfqpoint{-0.033333in}{-0.008840in}}{\pgfqpoint{-0.029821in}{-0.017319in}}{\pgfqpoint{-0.023570in}{-0.023570in}}%
\pgfpathcurveto{\pgfqpoint{-0.017319in}{-0.029821in}}{\pgfqpoint{-0.008840in}{-0.033333in}}{\pgfqpoint{0.000000in}{-0.033333in}}%
\pgfpathlineto{\pgfqpoint{0.000000in}{-0.033333in}}%
\pgfpathclose%
\pgfusepath{stroke,fill}%
}%
\begin{pgfscope}%
\pgfsys@transformshift{3.307753in}{2.846877in}%
\pgfsys@useobject{currentmarker}{}%
\end{pgfscope}%
\end{pgfscope}%
\begin{pgfscope}%
\pgfpathrectangle{\pgfqpoint{0.765000in}{0.660000in}}{\pgfqpoint{4.620000in}{4.620000in}}%
\pgfusepath{clip}%
\pgfsetrectcap%
\pgfsetroundjoin%
\pgfsetlinewidth{1.204500pt}%
\definecolor{currentstroke}{rgb}{1.000000,0.576471,0.309804}%
\pgfsetstrokecolor{currentstroke}%
\pgfsetdash{}{0pt}%
\pgfusepath{stroke}%
\end{pgfscope}%
\begin{pgfscope}%
\pgfpathrectangle{\pgfqpoint{0.765000in}{0.660000in}}{\pgfqpoint{4.620000in}{4.620000in}}%
\pgfusepath{clip}%
\pgfsetbuttcap%
\pgfsetroundjoin%
\definecolor{currentfill}{rgb}{1.000000,0.576471,0.309804}%
\pgfsetfillcolor{currentfill}%
\pgfsetlinewidth{1.003750pt}%
\definecolor{currentstroke}{rgb}{1.000000,0.576471,0.309804}%
\pgfsetstrokecolor{currentstroke}%
\pgfsetdash{}{0pt}%
\pgfsys@defobject{currentmarker}{\pgfqpoint{-0.033333in}{-0.033333in}}{\pgfqpoint{0.033333in}{0.033333in}}{%
\pgfpathmoveto{\pgfqpoint{0.000000in}{-0.033333in}}%
\pgfpathcurveto{\pgfqpoint{0.008840in}{-0.033333in}}{\pgfqpoint{0.017319in}{-0.029821in}}{\pgfqpoint{0.023570in}{-0.023570in}}%
\pgfpathcurveto{\pgfqpoint{0.029821in}{-0.017319in}}{\pgfqpoint{0.033333in}{-0.008840in}}{\pgfqpoint{0.033333in}{0.000000in}}%
\pgfpathcurveto{\pgfqpoint{0.033333in}{0.008840in}}{\pgfqpoint{0.029821in}{0.017319in}}{\pgfqpoint{0.023570in}{0.023570in}}%
\pgfpathcurveto{\pgfqpoint{0.017319in}{0.029821in}}{\pgfqpoint{0.008840in}{0.033333in}}{\pgfqpoint{0.000000in}{0.033333in}}%
\pgfpathcurveto{\pgfqpoint{-0.008840in}{0.033333in}}{\pgfqpoint{-0.017319in}{0.029821in}}{\pgfqpoint{-0.023570in}{0.023570in}}%
\pgfpathcurveto{\pgfqpoint{-0.029821in}{0.017319in}}{\pgfqpoint{-0.033333in}{0.008840in}}{\pgfqpoint{-0.033333in}{0.000000in}}%
\pgfpathcurveto{\pgfqpoint{-0.033333in}{-0.008840in}}{\pgfqpoint{-0.029821in}{-0.017319in}}{\pgfqpoint{-0.023570in}{-0.023570in}}%
\pgfpathcurveto{\pgfqpoint{-0.017319in}{-0.029821in}}{\pgfqpoint{-0.008840in}{-0.033333in}}{\pgfqpoint{0.000000in}{-0.033333in}}%
\pgfpathlineto{\pgfqpoint{0.000000in}{-0.033333in}}%
\pgfpathclose%
\pgfusepath{stroke,fill}%
}%
\begin{pgfscope}%
\pgfsys@transformshift{5.415615in}{3.242173in}%
\pgfsys@useobject{currentmarker}{}%
\end{pgfscope}%
\end{pgfscope}%
\begin{pgfscope}%
\pgfpathrectangle{\pgfqpoint{0.765000in}{0.660000in}}{\pgfqpoint{4.620000in}{4.620000in}}%
\pgfusepath{clip}%
\pgfsetrectcap%
\pgfsetroundjoin%
\pgfsetlinewidth{1.204500pt}%
\definecolor{currentstroke}{rgb}{1.000000,0.576471,0.309804}%
\pgfsetstrokecolor{currentstroke}%
\pgfsetdash{}{0pt}%
\pgfpathmoveto{\pgfqpoint{2.144482in}{2.826821in}}%
\pgfusepath{stroke}%
\end{pgfscope}%
\begin{pgfscope}%
\pgfpathrectangle{\pgfqpoint{0.765000in}{0.660000in}}{\pgfqpoint{4.620000in}{4.620000in}}%
\pgfusepath{clip}%
\pgfsetbuttcap%
\pgfsetroundjoin%
\definecolor{currentfill}{rgb}{1.000000,0.576471,0.309804}%
\pgfsetfillcolor{currentfill}%
\pgfsetlinewidth{1.003750pt}%
\definecolor{currentstroke}{rgb}{1.000000,0.576471,0.309804}%
\pgfsetstrokecolor{currentstroke}%
\pgfsetdash{}{0pt}%
\pgfsys@defobject{currentmarker}{\pgfqpoint{-0.033333in}{-0.033333in}}{\pgfqpoint{0.033333in}{0.033333in}}{%
\pgfpathmoveto{\pgfqpoint{0.000000in}{-0.033333in}}%
\pgfpathcurveto{\pgfqpoint{0.008840in}{-0.033333in}}{\pgfqpoint{0.017319in}{-0.029821in}}{\pgfqpoint{0.023570in}{-0.023570in}}%
\pgfpathcurveto{\pgfqpoint{0.029821in}{-0.017319in}}{\pgfqpoint{0.033333in}{-0.008840in}}{\pgfqpoint{0.033333in}{0.000000in}}%
\pgfpathcurveto{\pgfqpoint{0.033333in}{0.008840in}}{\pgfqpoint{0.029821in}{0.017319in}}{\pgfqpoint{0.023570in}{0.023570in}}%
\pgfpathcurveto{\pgfqpoint{0.017319in}{0.029821in}}{\pgfqpoint{0.008840in}{0.033333in}}{\pgfqpoint{0.000000in}{0.033333in}}%
\pgfpathcurveto{\pgfqpoint{-0.008840in}{0.033333in}}{\pgfqpoint{-0.017319in}{0.029821in}}{\pgfqpoint{-0.023570in}{0.023570in}}%
\pgfpathcurveto{\pgfqpoint{-0.029821in}{0.017319in}}{\pgfqpoint{-0.033333in}{0.008840in}}{\pgfqpoint{-0.033333in}{0.000000in}}%
\pgfpathcurveto{\pgfqpoint{-0.033333in}{-0.008840in}}{\pgfqpoint{-0.029821in}{-0.017319in}}{\pgfqpoint{-0.023570in}{-0.023570in}}%
\pgfpathcurveto{\pgfqpoint{-0.017319in}{-0.029821in}}{\pgfqpoint{-0.008840in}{-0.033333in}}{\pgfqpoint{0.000000in}{-0.033333in}}%
\pgfpathlineto{\pgfqpoint{0.000000in}{-0.033333in}}%
\pgfpathclose%
\pgfusepath{stroke,fill}%
}%
\begin{pgfscope}%
\pgfsys@transformshift{2.144482in}{2.826821in}%
\pgfsys@useobject{currentmarker}{}%
\end{pgfscope}%
\end{pgfscope}%
\begin{pgfscope}%
\pgfpathrectangle{\pgfqpoint{0.765000in}{0.660000in}}{\pgfqpoint{4.620000in}{4.620000in}}%
\pgfusepath{clip}%
\pgfsetrectcap%
\pgfsetroundjoin%
\pgfsetlinewidth{1.204500pt}%
\definecolor{currentstroke}{rgb}{1.000000,0.576471,0.309804}%
\pgfsetstrokecolor{currentstroke}%
\pgfsetdash{}{0pt}%
\pgfpathmoveto{\pgfqpoint{4.664293in}{3.048055in}}%
\pgfusepath{stroke}%
\end{pgfscope}%
\begin{pgfscope}%
\pgfpathrectangle{\pgfqpoint{0.765000in}{0.660000in}}{\pgfqpoint{4.620000in}{4.620000in}}%
\pgfusepath{clip}%
\pgfsetbuttcap%
\pgfsetroundjoin%
\definecolor{currentfill}{rgb}{1.000000,0.576471,0.309804}%
\pgfsetfillcolor{currentfill}%
\pgfsetlinewidth{1.003750pt}%
\definecolor{currentstroke}{rgb}{1.000000,0.576471,0.309804}%
\pgfsetstrokecolor{currentstroke}%
\pgfsetdash{}{0pt}%
\pgfsys@defobject{currentmarker}{\pgfqpoint{-0.033333in}{-0.033333in}}{\pgfqpoint{0.033333in}{0.033333in}}{%
\pgfpathmoveto{\pgfqpoint{0.000000in}{-0.033333in}}%
\pgfpathcurveto{\pgfqpoint{0.008840in}{-0.033333in}}{\pgfqpoint{0.017319in}{-0.029821in}}{\pgfqpoint{0.023570in}{-0.023570in}}%
\pgfpathcurveto{\pgfqpoint{0.029821in}{-0.017319in}}{\pgfqpoint{0.033333in}{-0.008840in}}{\pgfqpoint{0.033333in}{0.000000in}}%
\pgfpathcurveto{\pgfqpoint{0.033333in}{0.008840in}}{\pgfqpoint{0.029821in}{0.017319in}}{\pgfqpoint{0.023570in}{0.023570in}}%
\pgfpathcurveto{\pgfqpoint{0.017319in}{0.029821in}}{\pgfqpoint{0.008840in}{0.033333in}}{\pgfqpoint{0.000000in}{0.033333in}}%
\pgfpathcurveto{\pgfqpoint{-0.008840in}{0.033333in}}{\pgfqpoint{-0.017319in}{0.029821in}}{\pgfqpoint{-0.023570in}{0.023570in}}%
\pgfpathcurveto{\pgfqpoint{-0.029821in}{0.017319in}}{\pgfqpoint{-0.033333in}{0.008840in}}{\pgfqpoint{-0.033333in}{0.000000in}}%
\pgfpathcurveto{\pgfqpoint{-0.033333in}{-0.008840in}}{\pgfqpoint{-0.029821in}{-0.017319in}}{\pgfqpoint{-0.023570in}{-0.023570in}}%
\pgfpathcurveto{\pgfqpoint{-0.017319in}{-0.029821in}}{\pgfqpoint{-0.008840in}{-0.033333in}}{\pgfqpoint{0.000000in}{-0.033333in}}%
\pgfpathlineto{\pgfqpoint{0.000000in}{-0.033333in}}%
\pgfpathclose%
\pgfusepath{stroke,fill}%
}%
\begin{pgfscope}%
\pgfsys@transformshift{4.664293in}{3.048055in}%
\pgfsys@useobject{currentmarker}{}%
\end{pgfscope}%
\end{pgfscope}%
\begin{pgfscope}%
\pgfpathrectangle{\pgfqpoint{0.765000in}{0.660000in}}{\pgfqpoint{4.620000in}{4.620000in}}%
\pgfusepath{clip}%
\pgfsetrectcap%
\pgfsetroundjoin%
\pgfsetlinewidth{1.204500pt}%
\definecolor{currentstroke}{rgb}{1.000000,0.576471,0.309804}%
\pgfsetstrokecolor{currentstroke}%
\pgfsetdash{}{0pt}%
\pgfpathmoveto{\pgfqpoint{4.749270in}{3.004696in}}%
\pgfusepath{stroke}%
\end{pgfscope}%
\begin{pgfscope}%
\pgfpathrectangle{\pgfqpoint{0.765000in}{0.660000in}}{\pgfqpoint{4.620000in}{4.620000in}}%
\pgfusepath{clip}%
\pgfsetbuttcap%
\pgfsetroundjoin%
\definecolor{currentfill}{rgb}{1.000000,0.576471,0.309804}%
\pgfsetfillcolor{currentfill}%
\pgfsetlinewidth{1.003750pt}%
\definecolor{currentstroke}{rgb}{1.000000,0.576471,0.309804}%
\pgfsetstrokecolor{currentstroke}%
\pgfsetdash{}{0pt}%
\pgfsys@defobject{currentmarker}{\pgfqpoint{-0.033333in}{-0.033333in}}{\pgfqpoint{0.033333in}{0.033333in}}{%
\pgfpathmoveto{\pgfqpoint{0.000000in}{-0.033333in}}%
\pgfpathcurveto{\pgfqpoint{0.008840in}{-0.033333in}}{\pgfqpoint{0.017319in}{-0.029821in}}{\pgfqpoint{0.023570in}{-0.023570in}}%
\pgfpathcurveto{\pgfqpoint{0.029821in}{-0.017319in}}{\pgfqpoint{0.033333in}{-0.008840in}}{\pgfqpoint{0.033333in}{0.000000in}}%
\pgfpathcurveto{\pgfqpoint{0.033333in}{0.008840in}}{\pgfqpoint{0.029821in}{0.017319in}}{\pgfqpoint{0.023570in}{0.023570in}}%
\pgfpathcurveto{\pgfqpoint{0.017319in}{0.029821in}}{\pgfqpoint{0.008840in}{0.033333in}}{\pgfqpoint{0.000000in}{0.033333in}}%
\pgfpathcurveto{\pgfqpoint{-0.008840in}{0.033333in}}{\pgfqpoint{-0.017319in}{0.029821in}}{\pgfqpoint{-0.023570in}{0.023570in}}%
\pgfpathcurveto{\pgfqpoint{-0.029821in}{0.017319in}}{\pgfqpoint{-0.033333in}{0.008840in}}{\pgfqpoint{-0.033333in}{0.000000in}}%
\pgfpathcurveto{\pgfqpoint{-0.033333in}{-0.008840in}}{\pgfqpoint{-0.029821in}{-0.017319in}}{\pgfqpoint{-0.023570in}{-0.023570in}}%
\pgfpathcurveto{\pgfqpoint{-0.017319in}{-0.029821in}}{\pgfqpoint{-0.008840in}{-0.033333in}}{\pgfqpoint{0.000000in}{-0.033333in}}%
\pgfpathlineto{\pgfqpoint{0.000000in}{-0.033333in}}%
\pgfpathclose%
\pgfusepath{stroke,fill}%
}%
\begin{pgfscope}%
\pgfsys@transformshift{4.749270in}{3.004696in}%
\pgfsys@useobject{currentmarker}{}%
\end{pgfscope}%
\end{pgfscope}%
\begin{pgfscope}%
\pgfpathrectangle{\pgfqpoint{0.765000in}{0.660000in}}{\pgfqpoint{4.620000in}{4.620000in}}%
\pgfusepath{clip}%
\pgfsetrectcap%
\pgfsetroundjoin%
\pgfsetlinewidth{1.204500pt}%
\definecolor{currentstroke}{rgb}{1.000000,0.576471,0.309804}%
\pgfsetstrokecolor{currentstroke}%
\pgfsetdash{}{0pt}%
\pgfpathmoveto{\pgfqpoint{2.328750in}{2.746484in}}%
\pgfusepath{stroke}%
\end{pgfscope}%
\begin{pgfscope}%
\pgfpathrectangle{\pgfqpoint{0.765000in}{0.660000in}}{\pgfqpoint{4.620000in}{4.620000in}}%
\pgfusepath{clip}%
\pgfsetbuttcap%
\pgfsetroundjoin%
\definecolor{currentfill}{rgb}{1.000000,0.576471,0.309804}%
\pgfsetfillcolor{currentfill}%
\pgfsetlinewidth{1.003750pt}%
\definecolor{currentstroke}{rgb}{1.000000,0.576471,0.309804}%
\pgfsetstrokecolor{currentstroke}%
\pgfsetdash{}{0pt}%
\pgfsys@defobject{currentmarker}{\pgfqpoint{-0.033333in}{-0.033333in}}{\pgfqpoint{0.033333in}{0.033333in}}{%
\pgfpathmoveto{\pgfqpoint{0.000000in}{-0.033333in}}%
\pgfpathcurveto{\pgfqpoint{0.008840in}{-0.033333in}}{\pgfqpoint{0.017319in}{-0.029821in}}{\pgfqpoint{0.023570in}{-0.023570in}}%
\pgfpathcurveto{\pgfqpoint{0.029821in}{-0.017319in}}{\pgfqpoint{0.033333in}{-0.008840in}}{\pgfqpoint{0.033333in}{0.000000in}}%
\pgfpathcurveto{\pgfqpoint{0.033333in}{0.008840in}}{\pgfqpoint{0.029821in}{0.017319in}}{\pgfqpoint{0.023570in}{0.023570in}}%
\pgfpathcurveto{\pgfqpoint{0.017319in}{0.029821in}}{\pgfqpoint{0.008840in}{0.033333in}}{\pgfqpoint{0.000000in}{0.033333in}}%
\pgfpathcurveto{\pgfqpoint{-0.008840in}{0.033333in}}{\pgfqpoint{-0.017319in}{0.029821in}}{\pgfqpoint{-0.023570in}{0.023570in}}%
\pgfpathcurveto{\pgfqpoint{-0.029821in}{0.017319in}}{\pgfqpoint{-0.033333in}{0.008840in}}{\pgfqpoint{-0.033333in}{0.000000in}}%
\pgfpathcurveto{\pgfqpoint{-0.033333in}{-0.008840in}}{\pgfqpoint{-0.029821in}{-0.017319in}}{\pgfqpoint{-0.023570in}{-0.023570in}}%
\pgfpathcurveto{\pgfqpoint{-0.017319in}{-0.029821in}}{\pgfqpoint{-0.008840in}{-0.033333in}}{\pgfqpoint{0.000000in}{-0.033333in}}%
\pgfpathlineto{\pgfqpoint{0.000000in}{-0.033333in}}%
\pgfpathclose%
\pgfusepath{stroke,fill}%
}%
\begin{pgfscope}%
\pgfsys@transformshift{2.328750in}{2.746484in}%
\pgfsys@useobject{currentmarker}{}%
\end{pgfscope}%
\end{pgfscope}%
\begin{pgfscope}%
\pgfpathrectangle{\pgfqpoint{0.765000in}{0.660000in}}{\pgfqpoint{4.620000in}{4.620000in}}%
\pgfusepath{clip}%
\pgfsetrectcap%
\pgfsetroundjoin%
\pgfsetlinewidth{1.204500pt}%
\definecolor{currentstroke}{rgb}{1.000000,0.576471,0.309804}%
\pgfsetstrokecolor{currentstroke}%
\pgfsetdash{}{0pt}%
\pgfpathmoveto{\pgfqpoint{4.435415in}{2.789876in}}%
\pgfusepath{stroke}%
\end{pgfscope}%
\begin{pgfscope}%
\pgfpathrectangle{\pgfqpoint{0.765000in}{0.660000in}}{\pgfqpoint{4.620000in}{4.620000in}}%
\pgfusepath{clip}%
\pgfsetbuttcap%
\pgfsetroundjoin%
\definecolor{currentfill}{rgb}{1.000000,0.576471,0.309804}%
\pgfsetfillcolor{currentfill}%
\pgfsetlinewidth{1.003750pt}%
\definecolor{currentstroke}{rgb}{1.000000,0.576471,0.309804}%
\pgfsetstrokecolor{currentstroke}%
\pgfsetdash{}{0pt}%
\pgfsys@defobject{currentmarker}{\pgfqpoint{-0.033333in}{-0.033333in}}{\pgfqpoint{0.033333in}{0.033333in}}{%
\pgfpathmoveto{\pgfqpoint{0.000000in}{-0.033333in}}%
\pgfpathcurveto{\pgfqpoint{0.008840in}{-0.033333in}}{\pgfqpoint{0.017319in}{-0.029821in}}{\pgfqpoint{0.023570in}{-0.023570in}}%
\pgfpathcurveto{\pgfqpoint{0.029821in}{-0.017319in}}{\pgfqpoint{0.033333in}{-0.008840in}}{\pgfqpoint{0.033333in}{0.000000in}}%
\pgfpathcurveto{\pgfqpoint{0.033333in}{0.008840in}}{\pgfqpoint{0.029821in}{0.017319in}}{\pgfqpoint{0.023570in}{0.023570in}}%
\pgfpathcurveto{\pgfqpoint{0.017319in}{0.029821in}}{\pgfqpoint{0.008840in}{0.033333in}}{\pgfqpoint{0.000000in}{0.033333in}}%
\pgfpathcurveto{\pgfqpoint{-0.008840in}{0.033333in}}{\pgfqpoint{-0.017319in}{0.029821in}}{\pgfqpoint{-0.023570in}{0.023570in}}%
\pgfpathcurveto{\pgfqpoint{-0.029821in}{0.017319in}}{\pgfqpoint{-0.033333in}{0.008840in}}{\pgfqpoint{-0.033333in}{0.000000in}}%
\pgfpathcurveto{\pgfqpoint{-0.033333in}{-0.008840in}}{\pgfqpoint{-0.029821in}{-0.017319in}}{\pgfqpoint{-0.023570in}{-0.023570in}}%
\pgfpathcurveto{\pgfqpoint{-0.017319in}{-0.029821in}}{\pgfqpoint{-0.008840in}{-0.033333in}}{\pgfqpoint{0.000000in}{-0.033333in}}%
\pgfpathlineto{\pgfqpoint{0.000000in}{-0.033333in}}%
\pgfpathclose%
\pgfusepath{stroke,fill}%
}%
\begin{pgfscope}%
\pgfsys@transformshift{4.435415in}{2.789876in}%
\pgfsys@useobject{currentmarker}{}%
\end{pgfscope}%
\end{pgfscope}%
\begin{pgfscope}%
\pgfpathrectangle{\pgfqpoint{0.765000in}{0.660000in}}{\pgfqpoint{4.620000in}{4.620000in}}%
\pgfusepath{clip}%
\pgfsetrectcap%
\pgfsetroundjoin%
\pgfsetlinewidth{1.204500pt}%
\definecolor{currentstroke}{rgb}{1.000000,0.576471,0.309804}%
\pgfsetstrokecolor{currentstroke}%
\pgfsetdash{}{0pt}%
\pgfpathmoveto{\pgfqpoint{2.704602in}{3.179003in}}%
\pgfusepath{stroke}%
\end{pgfscope}%
\begin{pgfscope}%
\pgfpathrectangle{\pgfqpoint{0.765000in}{0.660000in}}{\pgfqpoint{4.620000in}{4.620000in}}%
\pgfusepath{clip}%
\pgfsetbuttcap%
\pgfsetroundjoin%
\definecolor{currentfill}{rgb}{1.000000,0.576471,0.309804}%
\pgfsetfillcolor{currentfill}%
\pgfsetlinewidth{1.003750pt}%
\definecolor{currentstroke}{rgb}{1.000000,0.576471,0.309804}%
\pgfsetstrokecolor{currentstroke}%
\pgfsetdash{}{0pt}%
\pgfsys@defobject{currentmarker}{\pgfqpoint{-0.033333in}{-0.033333in}}{\pgfqpoint{0.033333in}{0.033333in}}{%
\pgfpathmoveto{\pgfqpoint{0.000000in}{-0.033333in}}%
\pgfpathcurveto{\pgfqpoint{0.008840in}{-0.033333in}}{\pgfqpoint{0.017319in}{-0.029821in}}{\pgfqpoint{0.023570in}{-0.023570in}}%
\pgfpathcurveto{\pgfqpoint{0.029821in}{-0.017319in}}{\pgfqpoint{0.033333in}{-0.008840in}}{\pgfqpoint{0.033333in}{0.000000in}}%
\pgfpathcurveto{\pgfqpoint{0.033333in}{0.008840in}}{\pgfqpoint{0.029821in}{0.017319in}}{\pgfqpoint{0.023570in}{0.023570in}}%
\pgfpathcurveto{\pgfqpoint{0.017319in}{0.029821in}}{\pgfqpoint{0.008840in}{0.033333in}}{\pgfqpoint{0.000000in}{0.033333in}}%
\pgfpathcurveto{\pgfqpoint{-0.008840in}{0.033333in}}{\pgfqpoint{-0.017319in}{0.029821in}}{\pgfqpoint{-0.023570in}{0.023570in}}%
\pgfpathcurveto{\pgfqpoint{-0.029821in}{0.017319in}}{\pgfqpoint{-0.033333in}{0.008840in}}{\pgfqpoint{-0.033333in}{0.000000in}}%
\pgfpathcurveto{\pgfqpoint{-0.033333in}{-0.008840in}}{\pgfqpoint{-0.029821in}{-0.017319in}}{\pgfqpoint{-0.023570in}{-0.023570in}}%
\pgfpathcurveto{\pgfqpoint{-0.017319in}{-0.029821in}}{\pgfqpoint{-0.008840in}{-0.033333in}}{\pgfqpoint{0.000000in}{-0.033333in}}%
\pgfpathlineto{\pgfqpoint{0.000000in}{-0.033333in}}%
\pgfpathclose%
\pgfusepath{stroke,fill}%
}%
\begin{pgfscope}%
\pgfsys@transformshift{2.704602in}{3.179003in}%
\pgfsys@useobject{currentmarker}{}%
\end{pgfscope}%
\end{pgfscope}%
\begin{pgfscope}%
\pgfpathrectangle{\pgfqpoint{0.765000in}{0.660000in}}{\pgfqpoint{4.620000in}{4.620000in}}%
\pgfusepath{clip}%
\pgfsetrectcap%
\pgfsetroundjoin%
\pgfsetlinewidth{1.204500pt}%
\definecolor{currentstroke}{rgb}{1.000000,0.576471,0.309804}%
\pgfsetstrokecolor{currentstroke}%
\pgfsetdash{}{0pt}%
\pgfpathmoveto{\pgfqpoint{3.816314in}{2.637750in}}%
\pgfusepath{stroke}%
\end{pgfscope}%
\begin{pgfscope}%
\pgfpathrectangle{\pgfqpoint{0.765000in}{0.660000in}}{\pgfqpoint{4.620000in}{4.620000in}}%
\pgfusepath{clip}%
\pgfsetbuttcap%
\pgfsetroundjoin%
\definecolor{currentfill}{rgb}{1.000000,0.576471,0.309804}%
\pgfsetfillcolor{currentfill}%
\pgfsetlinewidth{1.003750pt}%
\definecolor{currentstroke}{rgb}{1.000000,0.576471,0.309804}%
\pgfsetstrokecolor{currentstroke}%
\pgfsetdash{}{0pt}%
\pgfsys@defobject{currentmarker}{\pgfqpoint{-0.033333in}{-0.033333in}}{\pgfqpoint{0.033333in}{0.033333in}}{%
\pgfpathmoveto{\pgfqpoint{0.000000in}{-0.033333in}}%
\pgfpathcurveto{\pgfqpoint{0.008840in}{-0.033333in}}{\pgfqpoint{0.017319in}{-0.029821in}}{\pgfqpoint{0.023570in}{-0.023570in}}%
\pgfpathcurveto{\pgfqpoint{0.029821in}{-0.017319in}}{\pgfqpoint{0.033333in}{-0.008840in}}{\pgfqpoint{0.033333in}{0.000000in}}%
\pgfpathcurveto{\pgfqpoint{0.033333in}{0.008840in}}{\pgfqpoint{0.029821in}{0.017319in}}{\pgfqpoint{0.023570in}{0.023570in}}%
\pgfpathcurveto{\pgfqpoint{0.017319in}{0.029821in}}{\pgfqpoint{0.008840in}{0.033333in}}{\pgfqpoint{0.000000in}{0.033333in}}%
\pgfpathcurveto{\pgfqpoint{-0.008840in}{0.033333in}}{\pgfqpoint{-0.017319in}{0.029821in}}{\pgfqpoint{-0.023570in}{0.023570in}}%
\pgfpathcurveto{\pgfqpoint{-0.029821in}{0.017319in}}{\pgfqpoint{-0.033333in}{0.008840in}}{\pgfqpoint{-0.033333in}{0.000000in}}%
\pgfpathcurveto{\pgfqpoint{-0.033333in}{-0.008840in}}{\pgfqpoint{-0.029821in}{-0.017319in}}{\pgfqpoint{-0.023570in}{-0.023570in}}%
\pgfpathcurveto{\pgfqpoint{-0.017319in}{-0.029821in}}{\pgfqpoint{-0.008840in}{-0.033333in}}{\pgfqpoint{0.000000in}{-0.033333in}}%
\pgfpathlineto{\pgfqpoint{0.000000in}{-0.033333in}}%
\pgfpathclose%
\pgfusepath{stroke,fill}%
}%
\begin{pgfscope}%
\pgfsys@transformshift{3.816314in}{2.637750in}%
\pgfsys@useobject{currentmarker}{}%
\end{pgfscope}%
\end{pgfscope}%
\begin{pgfscope}%
\pgfpathrectangle{\pgfqpoint{0.765000in}{0.660000in}}{\pgfqpoint{4.620000in}{4.620000in}}%
\pgfusepath{clip}%
\pgfsetrectcap%
\pgfsetroundjoin%
\pgfsetlinewidth{1.204500pt}%
\definecolor{currentstroke}{rgb}{1.000000,0.576471,0.309804}%
\pgfsetstrokecolor{currentstroke}%
\pgfsetdash{}{0pt}%
\pgfpathmoveto{\pgfqpoint{3.754878in}{2.640924in}}%
\pgfusepath{stroke}%
\end{pgfscope}%
\begin{pgfscope}%
\pgfpathrectangle{\pgfqpoint{0.765000in}{0.660000in}}{\pgfqpoint{4.620000in}{4.620000in}}%
\pgfusepath{clip}%
\pgfsetbuttcap%
\pgfsetroundjoin%
\definecolor{currentfill}{rgb}{1.000000,0.576471,0.309804}%
\pgfsetfillcolor{currentfill}%
\pgfsetlinewidth{1.003750pt}%
\definecolor{currentstroke}{rgb}{1.000000,0.576471,0.309804}%
\pgfsetstrokecolor{currentstroke}%
\pgfsetdash{}{0pt}%
\pgfsys@defobject{currentmarker}{\pgfqpoint{-0.033333in}{-0.033333in}}{\pgfqpoint{0.033333in}{0.033333in}}{%
\pgfpathmoveto{\pgfqpoint{0.000000in}{-0.033333in}}%
\pgfpathcurveto{\pgfqpoint{0.008840in}{-0.033333in}}{\pgfqpoint{0.017319in}{-0.029821in}}{\pgfqpoint{0.023570in}{-0.023570in}}%
\pgfpathcurveto{\pgfqpoint{0.029821in}{-0.017319in}}{\pgfqpoint{0.033333in}{-0.008840in}}{\pgfqpoint{0.033333in}{0.000000in}}%
\pgfpathcurveto{\pgfqpoint{0.033333in}{0.008840in}}{\pgfqpoint{0.029821in}{0.017319in}}{\pgfqpoint{0.023570in}{0.023570in}}%
\pgfpathcurveto{\pgfqpoint{0.017319in}{0.029821in}}{\pgfqpoint{0.008840in}{0.033333in}}{\pgfqpoint{0.000000in}{0.033333in}}%
\pgfpathcurveto{\pgfqpoint{-0.008840in}{0.033333in}}{\pgfqpoint{-0.017319in}{0.029821in}}{\pgfqpoint{-0.023570in}{0.023570in}}%
\pgfpathcurveto{\pgfqpoint{-0.029821in}{0.017319in}}{\pgfqpoint{-0.033333in}{0.008840in}}{\pgfqpoint{-0.033333in}{0.000000in}}%
\pgfpathcurveto{\pgfqpoint{-0.033333in}{-0.008840in}}{\pgfqpoint{-0.029821in}{-0.017319in}}{\pgfqpoint{-0.023570in}{-0.023570in}}%
\pgfpathcurveto{\pgfqpoint{-0.017319in}{-0.029821in}}{\pgfqpoint{-0.008840in}{-0.033333in}}{\pgfqpoint{0.000000in}{-0.033333in}}%
\pgfpathlineto{\pgfqpoint{0.000000in}{-0.033333in}}%
\pgfpathclose%
\pgfusepath{stroke,fill}%
}%
\begin{pgfscope}%
\pgfsys@transformshift{3.754878in}{2.640924in}%
\pgfsys@useobject{currentmarker}{}%
\end{pgfscope}%
\end{pgfscope}%
\begin{pgfscope}%
\pgfpathrectangle{\pgfqpoint{0.765000in}{0.660000in}}{\pgfqpoint{4.620000in}{4.620000in}}%
\pgfusepath{clip}%
\pgfsetrectcap%
\pgfsetroundjoin%
\pgfsetlinewidth{1.204500pt}%
\definecolor{currentstroke}{rgb}{1.000000,0.576471,0.309804}%
\pgfsetstrokecolor{currentstroke}%
\pgfsetdash{}{0pt}%
\pgfpathmoveto{\pgfqpoint{4.498473in}{2.881780in}}%
\pgfusepath{stroke}%
\end{pgfscope}%
\begin{pgfscope}%
\pgfpathrectangle{\pgfqpoint{0.765000in}{0.660000in}}{\pgfqpoint{4.620000in}{4.620000in}}%
\pgfusepath{clip}%
\pgfsetbuttcap%
\pgfsetroundjoin%
\definecolor{currentfill}{rgb}{1.000000,0.576471,0.309804}%
\pgfsetfillcolor{currentfill}%
\pgfsetlinewidth{1.003750pt}%
\definecolor{currentstroke}{rgb}{1.000000,0.576471,0.309804}%
\pgfsetstrokecolor{currentstroke}%
\pgfsetdash{}{0pt}%
\pgfsys@defobject{currentmarker}{\pgfqpoint{-0.033333in}{-0.033333in}}{\pgfqpoint{0.033333in}{0.033333in}}{%
\pgfpathmoveto{\pgfqpoint{0.000000in}{-0.033333in}}%
\pgfpathcurveto{\pgfqpoint{0.008840in}{-0.033333in}}{\pgfqpoint{0.017319in}{-0.029821in}}{\pgfqpoint{0.023570in}{-0.023570in}}%
\pgfpathcurveto{\pgfqpoint{0.029821in}{-0.017319in}}{\pgfqpoint{0.033333in}{-0.008840in}}{\pgfqpoint{0.033333in}{0.000000in}}%
\pgfpathcurveto{\pgfqpoint{0.033333in}{0.008840in}}{\pgfqpoint{0.029821in}{0.017319in}}{\pgfqpoint{0.023570in}{0.023570in}}%
\pgfpathcurveto{\pgfqpoint{0.017319in}{0.029821in}}{\pgfqpoint{0.008840in}{0.033333in}}{\pgfqpoint{0.000000in}{0.033333in}}%
\pgfpathcurveto{\pgfqpoint{-0.008840in}{0.033333in}}{\pgfqpoint{-0.017319in}{0.029821in}}{\pgfqpoint{-0.023570in}{0.023570in}}%
\pgfpathcurveto{\pgfqpoint{-0.029821in}{0.017319in}}{\pgfqpoint{-0.033333in}{0.008840in}}{\pgfqpoint{-0.033333in}{0.000000in}}%
\pgfpathcurveto{\pgfqpoint{-0.033333in}{-0.008840in}}{\pgfqpoint{-0.029821in}{-0.017319in}}{\pgfqpoint{-0.023570in}{-0.023570in}}%
\pgfpathcurveto{\pgfqpoint{-0.017319in}{-0.029821in}}{\pgfqpoint{-0.008840in}{-0.033333in}}{\pgfqpoint{0.000000in}{-0.033333in}}%
\pgfpathlineto{\pgfqpoint{0.000000in}{-0.033333in}}%
\pgfpathclose%
\pgfusepath{stroke,fill}%
}%
\begin{pgfscope}%
\pgfsys@transformshift{4.498473in}{2.881780in}%
\pgfsys@useobject{currentmarker}{}%
\end{pgfscope}%
\end{pgfscope}%
\begin{pgfscope}%
\pgfpathrectangle{\pgfqpoint{0.765000in}{0.660000in}}{\pgfqpoint{4.620000in}{4.620000in}}%
\pgfusepath{clip}%
\pgfsetrectcap%
\pgfsetroundjoin%
\pgfsetlinewidth{1.204500pt}%
\definecolor{currentstroke}{rgb}{1.000000,0.576471,0.309804}%
\pgfsetstrokecolor{currentstroke}%
\pgfsetdash{}{0pt}%
\pgfpathmoveto{\pgfqpoint{4.857841in}{3.430614in}}%
\pgfusepath{stroke}%
\end{pgfscope}%
\begin{pgfscope}%
\pgfpathrectangle{\pgfqpoint{0.765000in}{0.660000in}}{\pgfqpoint{4.620000in}{4.620000in}}%
\pgfusepath{clip}%
\pgfsetbuttcap%
\pgfsetroundjoin%
\definecolor{currentfill}{rgb}{1.000000,0.576471,0.309804}%
\pgfsetfillcolor{currentfill}%
\pgfsetlinewidth{1.003750pt}%
\definecolor{currentstroke}{rgb}{1.000000,0.576471,0.309804}%
\pgfsetstrokecolor{currentstroke}%
\pgfsetdash{}{0pt}%
\pgfsys@defobject{currentmarker}{\pgfqpoint{-0.033333in}{-0.033333in}}{\pgfqpoint{0.033333in}{0.033333in}}{%
\pgfpathmoveto{\pgfqpoint{0.000000in}{-0.033333in}}%
\pgfpathcurveto{\pgfqpoint{0.008840in}{-0.033333in}}{\pgfqpoint{0.017319in}{-0.029821in}}{\pgfqpoint{0.023570in}{-0.023570in}}%
\pgfpathcurveto{\pgfqpoint{0.029821in}{-0.017319in}}{\pgfqpoint{0.033333in}{-0.008840in}}{\pgfqpoint{0.033333in}{0.000000in}}%
\pgfpathcurveto{\pgfqpoint{0.033333in}{0.008840in}}{\pgfqpoint{0.029821in}{0.017319in}}{\pgfqpoint{0.023570in}{0.023570in}}%
\pgfpathcurveto{\pgfqpoint{0.017319in}{0.029821in}}{\pgfqpoint{0.008840in}{0.033333in}}{\pgfqpoint{0.000000in}{0.033333in}}%
\pgfpathcurveto{\pgfqpoint{-0.008840in}{0.033333in}}{\pgfqpoint{-0.017319in}{0.029821in}}{\pgfqpoint{-0.023570in}{0.023570in}}%
\pgfpathcurveto{\pgfqpoint{-0.029821in}{0.017319in}}{\pgfqpoint{-0.033333in}{0.008840in}}{\pgfqpoint{-0.033333in}{0.000000in}}%
\pgfpathcurveto{\pgfqpoint{-0.033333in}{-0.008840in}}{\pgfqpoint{-0.029821in}{-0.017319in}}{\pgfqpoint{-0.023570in}{-0.023570in}}%
\pgfpathcurveto{\pgfqpoint{-0.017319in}{-0.029821in}}{\pgfqpoint{-0.008840in}{-0.033333in}}{\pgfqpoint{0.000000in}{-0.033333in}}%
\pgfpathlineto{\pgfqpoint{0.000000in}{-0.033333in}}%
\pgfpathclose%
\pgfusepath{stroke,fill}%
}%
\begin{pgfscope}%
\pgfsys@transformshift{4.857841in}{3.430614in}%
\pgfsys@useobject{currentmarker}{}%
\end{pgfscope}%
\end{pgfscope}%
\begin{pgfscope}%
\pgfpathrectangle{\pgfqpoint{0.765000in}{0.660000in}}{\pgfqpoint{4.620000in}{4.620000in}}%
\pgfusepath{clip}%
\pgfsetrectcap%
\pgfsetroundjoin%
\pgfsetlinewidth{1.204500pt}%
\definecolor{currentstroke}{rgb}{1.000000,0.576471,0.309804}%
\pgfsetstrokecolor{currentstroke}%
\pgfsetdash{}{0pt}%
\pgfpathmoveto{\pgfqpoint{2.628720in}{3.144735in}}%
\pgfusepath{stroke}%
\end{pgfscope}%
\begin{pgfscope}%
\pgfpathrectangle{\pgfqpoint{0.765000in}{0.660000in}}{\pgfqpoint{4.620000in}{4.620000in}}%
\pgfusepath{clip}%
\pgfsetbuttcap%
\pgfsetroundjoin%
\definecolor{currentfill}{rgb}{1.000000,0.576471,0.309804}%
\pgfsetfillcolor{currentfill}%
\pgfsetlinewidth{1.003750pt}%
\definecolor{currentstroke}{rgb}{1.000000,0.576471,0.309804}%
\pgfsetstrokecolor{currentstroke}%
\pgfsetdash{}{0pt}%
\pgfsys@defobject{currentmarker}{\pgfqpoint{-0.033333in}{-0.033333in}}{\pgfqpoint{0.033333in}{0.033333in}}{%
\pgfpathmoveto{\pgfqpoint{0.000000in}{-0.033333in}}%
\pgfpathcurveto{\pgfqpoint{0.008840in}{-0.033333in}}{\pgfqpoint{0.017319in}{-0.029821in}}{\pgfqpoint{0.023570in}{-0.023570in}}%
\pgfpathcurveto{\pgfqpoint{0.029821in}{-0.017319in}}{\pgfqpoint{0.033333in}{-0.008840in}}{\pgfqpoint{0.033333in}{0.000000in}}%
\pgfpathcurveto{\pgfqpoint{0.033333in}{0.008840in}}{\pgfqpoint{0.029821in}{0.017319in}}{\pgfqpoint{0.023570in}{0.023570in}}%
\pgfpathcurveto{\pgfqpoint{0.017319in}{0.029821in}}{\pgfqpoint{0.008840in}{0.033333in}}{\pgfqpoint{0.000000in}{0.033333in}}%
\pgfpathcurveto{\pgfqpoint{-0.008840in}{0.033333in}}{\pgfqpoint{-0.017319in}{0.029821in}}{\pgfqpoint{-0.023570in}{0.023570in}}%
\pgfpathcurveto{\pgfqpoint{-0.029821in}{0.017319in}}{\pgfqpoint{-0.033333in}{0.008840in}}{\pgfqpoint{-0.033333in}{0.000000in}}%
\pgfpathcurveto{\pgfqpoint{-0.033333in}{-0.008840in}}{\pgfqpoint{-0.029821in}{-0.017319in}}{\pgfqpoint{-0.023570in}{-0.023570in}}%
\pgfpathcurveto{\pgfqpoint{-0.017319in}{-0.029821in}}{\pgfqpoint{-0.008840in}{-0.033333in}}{\pgfqpoint{0.000000in}{-0.033333in}}%
\pgfpathlineto{\pgfqpoint{0.000000in}{-0.033333in}}%
\pgfpathclose%
\pgfusepath{stroke,fill}%
}%
\begin{pgfscope}%
\pgfsys@transformshift{2.628720in}{3.144735in}%
\pgfsys@useobject{currentmarker}{}%
\end{pgfscope}%
\end{pgfscope}%
\begin{pgfscope}%
\pgfpathrectangle{\pgfqpoint{0.765000in}{0.660000in}}{\pgfqpoint{4.620000in}{4.620000in}}%
\pgfusepath{clip}%
\pgfsetrectcap%
\pgfsetroundjoin%
\pgfsetlinewidth{1.204500pt}%
\definecolor{currentstroke}{rgb}{1.000000,0.576471,0.309804}%
\pgfsetstrokecolor{currentstroke}%
\pgfsetdash{}{0pt}%
\pgfpathmoveto{\pgfqpoint{2.865189in}{2.896017in}}%
\pgfusepath{stroke}%
\end{pgfscope}%
\begin{pgfscope}%
\pgfpathrectangle{\pgfqpoint{0.765000in}{0.660000in}}{\pgfqpoint{4.620000in}{4.620000in}}%
\pgfusepath{clip}%
\pgfsetbuttcap%
\pgfsetroundjoin%
\definecolor{currentfill}{rgb}{1.000000,0.576471,0.309804}%
\pgfsetfillcolor{currentfill}%
\pgfsetlinewidth{1.003750pt}%
\definecolor{currentstroke}{rgb}{1.000000,0.576471,0.309804}%
\pgfsetstrokecolor{currentstroke}%
\pgfsetdash{}{0pt}%
\pgfsys@defobject{currentmarker}{\pgfqpoint{-0.033333in}{-0.033333in}}{\pgfqpoint{0.033333in}{0.033333in}}{%
\pgfpathmoveto{\pgfqpoint{0.000000in}{-0.033333in}}%
\pgfpathcurveto{\pgfqpoint{0.008840in}{-0.033333in}}{\pgfqpoint{0.017319in}{-0.029821in}}{\pgfqpoint{0.023570in}{-0.023570in}}%
\pgfpathcurveto{\pgfqpoint{0.029821in}{-0.017319in}}{\pgfqpoint{0.033333in}{-0.008840in}}{\pgfqpoint{0.033333in}{0.000000in}}%
\pgfpathcurveto{\pgfqpoint{0.033333in}{0.008840in}}{\pgfqpoint{0.029821in}{0.017319in}}{\pgfqpoint{0.023570in}{0.023570in}}%
\pgfpathcurveto{\pgfqpoint{0.017319in}{0.029821in}}{\pgfqpoint{0.008840in}{0.033333in}}{\pgfqpoint{0.000000in}{0.033333in}}%
\pgfpathcurveto{\pgfqpoint{-0.008840in}{0.033333in}}{\pgfqpoint{-0.017319in}{0.029821in}}{\pgfqpoint{-0.023570in}{0.023570in}}%
\pgfpathcurveto{\pgfqpoint{-0.029821in}{0.017319in}}{\pgfqpoint{-0.033333in}{0.008840in}}{\pgfqpoint{-0.033333in}{0.000000in}}%
\pgfpathcurveto{\pgfqpoint{-0.033333in}{-0.008840in}}{\pgfqpoint{-0.029821in}{-0.017319in}}{\pgfqpoint{-0.023570in}{-0.023570in}}%
\pgfpathcurveto{\pgfqpoint{-0.017319in}{-0.029821in}}{\pgfqpoint{-0.008840in}{-0.033333in}}{\pgfqpoint{0.000000in}{-0.033333in}}%
\pgfpathlineto{\pgfqpoint{0.000000in}{-0.033333in}}%
\pgfpathclose%
\pgfusepath{stroke,fill}%
}%
\begin{pgfscope}%
\pgfsys@transformshift{2.865189in}{2.896017in}%
\pgfsys@useobject{currentmarker}{}%
\end{pgfscope}%
\end{pgfscope}%
\begin{pgfscope}%
\pgfpathrectangle{\pgfqpoint{0.765000in}{0.660000in}}{\pgfqpoint{4.620000in}{4.620000in}}%
\pgfusepath{clip}%
\pgfsetrectcap%
\pgfsetroundjoin%
\pgfsetlinewidth{1.204500pt}%
\definecolor{currentstroke}{rgb}{1.000000,0.576471,0.309804}%
\pgfsetstrokecolor{currentstroke}%
\pgfsetdash{}{0pt}%
\pgfpathmoveto{\pgfqpoint{4.524308in}{2.506842in}}%
\pgfusepath{stroke}%
\end{pgfscope}%
\begin{pgfscope}%
\pgfpathrectangle{\pgfqpoint{0.765000in}{0.660000in}}{\pgfqpoint{4.620000in}{4.620000in}}%
\pgfusepath{clip}%
\pgfsetbuttcap%
\pgfsetroundjoin%
\definecolor{currentfill}{rgb}{1.000000,0.576471,0.309804}%
\pgfsetfillcolor{currentfill}%
\pgfsetlinewidth{1.003750pt}%
\definecolor{currentstroke}{rgb}{1.000000,0.576471,0.309804}%
\pgfsetstrokecolor{currentstroke}%
\pgfsetdash{}{0pt}%
\pgfsys@defobject{currentmarker}{\pgfqpoint{-0.033333in}{-0.033333in}}{\pgfqpoint{0.033333in}{0.033333in}}{%
\pgfpathmoveto{\pgfqpoint{0.000000in}{-0.033333in}}%
\pgfpathcurveto{\pgfqpoint{0.008840in}{-0.033333in}}{\pgfqpoint{0.017319in}{-0.029821in}}{\pgfqpoint{0.023570in}{-0.023570in}}%
\pgfpathcurveto{\pgfqpoint{0.029821in}{-0.017319in}}{\pgfqpoint{0.033333in}{-0.008840in}}{\pgfqpoint{0.033333in}{0.000000in}}%
\pgfpathcurveto{\pgfqpoint{0.033333in}{0.008840in}}{\pgfqpoint{0.029821in}{0.017319in}}{\pgfqpoint{0.023570in}{0.023570in}}%
\pgfpathcurveto{\pgfqpoint{0.017319in}{0.029821in}}{\pgfqpoint{0.008840in}{0.033333in}}{\pgfqpoint{0.000000in}{0.033333in}}%
\pgfpathcurveto{\pgfqpoint{-0.008840in}{0.033333in}}{\pgfqpoint{-0.017319in}{0.029821in}}{\pgfqpoint{-0.023570in}{0.023570in}}%
\pgfpathcurveto{\pgfqpoint{-0.029821in}{0.017319in}}{\pgfqpoint{-0.033333in}{0.008840in}}{\pgfqpoint{-0.033333in}{0.000000in}}%
\pgfpathcurveto{\pgfqpoint{-0.033333in}{-0.008840in}}{\pgfqpoint{-0.029821in}{-0.017319in}}{\pgfqpoint{-0.023570in}{-0.023570in}}%
\pgfpathcurveto{\pgfqpoint{-0.017319in}{-0.029821in}}{\pgfqpoint{-0.008840in}{-0.033333in}}{\pgfqpoint{0.000000in}{-0.033333in}}%
\pgfpathlineto{\pgfqpoint{0.000000in}{-0.033333in}}%
\pgfpathclose%
\pgfusepath{stroke,fill}%
}%
\begin{pgfscope}%
\pgfsys@transformshift{4.524308in}{2.506842in}%
\pgfsys@useobject{currentmarker}{}%
\end{pgfscope}%
\end{pgfscope}%
\begin{pgfscope}%
\pgfpathrectangle{\pgfqpoint{0.765000in}{0.660000in}}{\pgfqpoint{4.620000in}{4.620000in}}%
\pgfusepath{clip}%
\pgfsetrectcap%
\pgfsetroundjoin%
\pgfsetlinewidth{1.204500pt}%
\definecolor{currentstroke}{rgb}{1.000000,0.576471,0.309804}%
\pgfsetstrokecolor{currentstroke}%
\pgfsetdash{}{0pt}%
\pgfpathmoveto{\pgfqpoint{3.380366in}{2.439375in}}%
\pgfusepath{stroke}%
\end{pgfscope}%
\begin{pgfscope}%
\pgfpathrectangle{\pgfqpoint{0.765000in}{0.660000in}}{\pgfqpoint{4.620000in}{4.620000in}}%
\pgfusepath{clip}%
\pgfsetbuttcap%
\pgfsetroundjoin%
\definecolor{currentfill}{rgb}{1.000000,0.576471,0.309804}%
\pgfsetfillcolor{currentfill}%
\pgfsetlinewidth{1.003750pt}%
\definecolor{currentstroke}{rgb}{1.000000,0.576471,0.309804}%
\pgfsetstrokecolor{currentstroke}%
\pgfsetdash{}{0pt}%
\pgfsys@defobject{currentmarker}{\pgfqpoint{-0.033333in}{-0.033333in}}{\pgfqpoint{0.033333in}{0.033333in}}{%
\pgfpathmoveto{\pgfqpoint{0.000000in}{-0.033333in}}%
\pgfpathcurveto{\pgfqpoint{0.008840in}{-0.033333in}}{\pgfqpoint{0.017319in}{-0.029821in}}{\pgfqpoint{0.023570in}{-0.023570in}}%
\pgfpathcurveto{\pgfqpoint{0.029821in}{-0.017319in}}{\pgfqpoint{0.033333in}{-0.008840in}}{\pgfqpoint{0.033333in}{0.000000in}}%
\pgfpathcurveto{\pgfqpoint{0.033333in}{0.008840in}}{\pgfqpoint{0.029821in}{0.017319in}}{\pgfqpoint{0.023570in}{0.023570in}}%
\pgfpathcurveto{\pgfqpoint{0.017319in}{0.029821in}}{\pgfqpoint{0.008840in}{0.033333in}}{\pgfqpoint{0.000000in}{0.033333in}}%
\pgfpathcurveto{\pgfqpoint{-0.008840in}{0.033333in}}{\pgfqpoint{-0.017319in}{0.029821in}}{\pgfqpoint{-0.023570in}{0.023570in}}%
\pgfpathcurveto{\pgfqpoint{-0.029821in}{0.017319in}}{\pgfqpoint{-0.033333in}{0.008840in}}{\pgfqpoint{-0.033333in}{0.000000in}}%
\pgfpathcurveto{\pgfqpoint{-0.033333in}{-0.008840in}}{\pgfqpoint{-0.029821in}{-0.017319in}}{\pgfqpoint{-0.023570in}{-0.023570in}}%
\pgfpathcurveto{\pgfqpoint{-0.017319in}{-0.029821in}}{\pgfqpoint{-0.008840in}{-0.033333in}}{\pgfqpoint{0.000000in}{-0.033333in}}%
\pgfpathlineto{\pgfqpoint{0.000000in}{-0.033333in}}%
\pgfpathclose%
\pgfusepath{stroke,fill}%
}%
\begin{pgfscope}%
\pgfsys@transformshift{3.380366in}{2.439375in}%
\pgfsys@useobject{currentmarker}{}%
\end{pgfscope}%
\end{pgfscope}%
\begin{pgfscope}%
\pgfpathrectangle{\pgfqpoint{0.765000in}{0.660000in}}{\pgfqpoint{4.620000in}{4.620000in}}%
\pgfusepath{clip}%
\pgfsetrectcap%
\pgfsetroundjoin%
\pgfsetlinewidth{1.204500pt}%
\definecolor{currentstroke}{rgb}{1.000000,0.576471,0.309804}%
\pgfsetstrokecolor{currentstroke}%
\pgfsetdash{}{0pt}%
\pgfpathmoveto{\pgfqpoint{2.925124in}{2.675622in}}%
\pgfusepath{stroke}%
\end{pgfscope}%
\begin{pgfscope}%
\pgfpathrectangle{\pgfqpoint{0.765000in}{0.660000in}}{\pgfqpoint{4.620000in}{4.620000in}}%
\pgfusepath{clip}%
\pgfsetbuttcap%
\pgfsetroundjoin%
\definecolor{currentfill}{rgb}{1.000000,0.576471,0.309804}%
\pgfsetfillcolor{currentfill}%
\pgfsetlinewidth{1.003750pt}%
\definecolor{currentstroke}{rgb}{1.000000,0.576471,0.309804}%
\pgfsetstrokecolor{currentstroke}%
\pgfsetdash{}{0pt}%
\pgfsys@defobject{currentmarker}{\pgfqpoint{-0.033333in}{-0.033333in}}{\pgfqpoint{0.033333in}{0.033333in}}{%
\pgfpathmoveto{\pgfqpoint{0.000000in}{-0.033333in}}%
\pgfpathcurveto{\pgfqpoint{0.008840in}{-0.033333in}}{\pgfqpoint{0.017319in}{-0.029821in}}{\pgfqpoint{0.023570in}{-0.023570in}}%
\pgfpathcurveto{\pgfqpoint{0.029821in}{-0.017319in}}{\pgfqpoint{0.033333in}{-0.008840in}}{\pgfqpoint{0.033333in}{0.000000in}}%
\pgfpathcurveto{\pgfqpoint{0.033333in}{0.008840in}}{\pgfqpoint{0.029821in}{0.017319in}}{\pgfqpoint{0.023570in}{0.023570in}}%
\pgfpathcurveto{\pgfqpoint{0.017319in}{0.029821in}}{\pgfqpoint{0.008840in}{0.033333in}}{\pgfqpoint{0.000000in}{0.033333in}}%
\pgfpathcurveto{\pgfqpoint{-0.008840in}{0.033333in}}{\pgfqpoint{-0.017319in}{0.029821in}}{\pgfqpoint{-0.023570in}{0.023570in}}%
\pgfpathcurveto{\pgfqpoint{-0.029821in}{0.017319in}}{\pgfqpoint{-0.033333in}{0.008840in}}{\pgfqpoint{-0.033333in}{0.000000in}}%
\pgfpathcurveto{\pgfqpoint{-0.033333in}{-0.008840in}}{\pgfqpoint{-0.029821in}{-0.017319in}}{\pgfqpoint{-0.023570in}{-0.023570in}}%
\pgfpathcurveto{\pgfqpoint{-0.017319in}{-0.029821in}}{\pgfqpoint{-0.008840in}{-0.033333in}}{\pgfqpoint{0.000000in}{-0.033333in}}%
\pgfpathlineto{\pgfqpoint{0.000000in}{-0.033333in}}%
\pgfpathclose%
\pgfusepath{stroke,fill}%
}%
\begin{pgfscope}%
\pgfsys@transformshift{2.925124in}{2.675622in}%
\pgfsys@useobject{currentmarker}{}%
\end{pgfscope}%
\end{pgfscope}%
\begin{pgfscope}%
\pgfpathrectangle{\pgfqpoint{0.765000in}{0.660000in}}{\pgfqpoint{4.620000in}{4.620000in}}%
\pgfusepath{clip}%
\pgfsetrectcap%
\pgfsetroundjoin%
\pgfsetlinewidth{1.204500pt}%
\definecolor{currentstroke}{rgb}{1.000000,0.576471,0.309804}%
\pgfsetstrokecolor{currentstroke}%
\pgfsetdash{}{0pt}%
\pgfpathmoveto{\pgfqpoint{2.896549in}{2.540388in}}%
\pgfusepath{stroke}%
\end{pgfscope}%
\begin{pgfscope}%
\pgfpathrectangle{\pgfqpoint{0.765000in}{0.660000in}}{\pgfqpoint{4.620000in}{4.620000in}}%
\pgfusepath{clip}%
\pgfsetbuttcap%
\pgfsetroundjoin%
\definecolor{currentfill}{rgb}{1.000000,0.576471,0.309804}%
\pgfsetfillcolor{currentfill}%
\pgfsetlinewidth{1.003750pt}%
\definecolor{currentstroke}{rgb}{1.000000,0.576471,0.309804}%
\pgfsetstrokecolor{currentstroke}%
\pgfsetdash{}{0pt}%
\pgfsys@defobject{currentmarker}{\pgfqpoint{-0.033333in}{-0.033333in}}{\pgfqpoint{0.033333in}{0.033333in}}{%
\pgfpathmoveto{\pgfqpoint{0.000000in}{-0.033333in}}%
\pgfpathcurveto{\pgfqpoint{0.008840in}{-0.033333in}}{\pgfqpoint{0.017319in}{-0.029821in}}{\pgfqpoint{0.023570in}{-0.023570in}}%
\pgfpathcurveto{\pgfqpoint{0.029821in}{-0.017319in}}{\pgfqpoint{0.033333in}{-0.008840in}}{\pgfqpoint{0.033333in}{0.000000in}}%
\pgfpathcurveto{\pgfqpoint{0.033333in}{0.008840in}}{\pgfqpoint{0.029821in}{0.017319in}}{\pgfqpoint{0.023570in}{0.023570in}}%
\pgfpathcurveto{\pgfqpoint{0.017319in}{0.029821in}}{\pgfqpoint{0.008840in}{0.033333in}}{\pgfqpoint{0.000000in}{0.033333in}}%
\pgfpathcurveto{\pgfqpoint{-0.008840in}{0.033333in}}{\pgfqpoint{-0.017319in}{0.029821in}}{\pgfqpoint{-0.023570in}{0.023570in}}%
\pgfpathcurveto{\pgfqpoint{-0.029821in}{0.017319in}}{\pgfqpoint{-0.033333in}{0.008840in}}{\pgfqpoint{-0.033333in}{0.000000in}}%
\pgfpathcurveto{\pgfqpoint{-0.033333in}{-0.008840in}}{\pgfqpoint{-0.029821in}{-0.017319in}}{\pgfqpoint{-0.023570in}{-0.023570in}}%
\pgfpathcurveto{\pgfqpoint{-0.017319in}{-0.029821in}}{\pgfqpoint{-0.008840in}{-0.033333in}}{\pgfqpoint{0.000000in}{-0.033333in}}%
\pgfpathlineto{\pgfqpoint{0.000000in}{-0.033333in}}%
\pgfpathclose%
\pgfusepath{stroke,fill}%
}%
\begin{pgfscope}%
\pgfsys@transformshift{2.896549in}{2.540388in}%
\pgfsys@useobject{currentmarker}{}%
\end{pgfscope}%
\end{pgfscope}%
\begin{pgfscope}%
\pgfpathrectangle{\pgfqpoint{0.765000in}{0.660000in}}{\pgfqpoint{4.620000in}{4.620000in}}%
\pgfusepath{clip}%
\pgfsetrectcap%
\pgfsetroundjoin%
\pgfsetlinewidth{1.204500pt}%
\definecolor{currentstroke}{rgb}{1.000000,0.576471,0.309804}%
\pgfsetstrokecolor{currentstroke}%
\pgfsetdash{}{0pt}%
\pgfpathmoveto{\pgfqpoint{4.697213in}{2.792804in}}%
\pgfusepath{stroke}%
\end{pgfscope}%
\begin{pgfscope}%
\pgfpathrectangle{\pgfqpoint{0.765000in}{0.660000in}}{\pgfqpoint{4.620000in}{4.620000in}}%
\pgfusepath{clip}%
\pgfsetbuttcap%
\pgfsetroundjoin%
\definecolor{currentfill}{rgb}{1.000000,0.576471,0.309804}%
\pgfsetfillcolor{currentfill}%
\pgfsetlinewidth{1.003750pt}%
\definecolor{currentstroke}{rgb}{1.000000,0.576471,0.309804}%
\pgfsetstrokecolor{currentstroke}%
\pgfsetdash{}{0pt}%
\pgfsys@defobject{currentmarker}{\pgfqpoint{-0.033333in}{-0.033333in}}{\pgfqpoint{0.033333in}{0.033333in}}{%
\pgfpathmoveto{\pgfqpoint{0.000000in}{-0.033333in}}%
\pgfpathcurveto{\pgfqpoint{0.008840in}{-0.033333in}}{\pgfqpoint{0.017319in}{-0.029821in}}{\pgfqpoint{0.023570in}{-0.023570in}}%
\pgfpathcurveto{\pgfqpoint{0.029821in}{-0.017319in}}{\pgfqpoint{0.033333in}{-0.008840in}}{\pgfqpoint{0.033333in}{0.000000in}}%
\pgfpathcurveto{\pgfqpoint{0.033333in}{0.008840in}}{\pgfqpoint{0.029821in}{0.017319in}}{\pgfqpoint{0.023570in}{0.023570in}}%
\pgfpathcurveto{\pgfqpoint{0.017319in}{0.029821in}}{\pgfqpoint{0.008840in}{0.033333in}}{\pgfqpoint{0.000000in}{0.033333in}}%
\pgfpathcurveto{\pgfqpoint{-0.008840in}{0.033333in}}{\pgfqpoint{-0.017319in}{0.029821in}}{\pgfqpoint{-0.023570in}{0.023570in}}%
\pgfpathcurveto{\pgfqpoint{-0.029821in}{0.017319in}}{\pgfqpoint{-0.033333in}{0.008840in}}{\pgfqpoint{-0.033333in}{0.000000in}}%
\pgfpathcurveto{\pgfqpoint{-0.033333in}{-0.008840in}}{\pgfqpoint{-0.029821in}{-0.017319in}}{\pgfqpoint{-0.023570in}{-0.023570in}}%
\pgfpathcurveto{\pgfqpoint{-0.017319in}{-0.029821in}}{\pgfqpoint{-0.008840in}{-0.033333in}}{\pgfqpoint{0.000000in}{-0.033333in}}%
\pgfpathlineto{\pgfqpoint{0.000000in}{-0.033333in}}%
\pgfpathclose%
\pgfusepath{stroke,fill}%
}%
\begin{pgfscope}%
\pgfsys@transformshift{4.697213in}{2.792804in}%
\pgfsys@useobject{currentmarker}{}%
\end{pgfscope}%
\end{pgfscope}%
\begin{pgfscope}%
\pgfpathrectangle{\pgfqpoint{0.765000in}{0.660000in}}{\pgfqpoint{4.620000in}{4.620000in}}%
\pgfusepath{clip}%
\pgfsetrectcap%
\pgfsetroundjoin%
\pgfsetlinewidth{1.204500pt}%
\definecolor{currentstroke}{rgb}{1.000000,0.576471,0.309804}%
\pgfsetstrokecolor{currentstroke}%
\pgfsetdash{}{0pt}%
\pgfpathmoveto{\pgfqpoint{2.742567in}{2.544196in}}%
\pgfusepath{stroke}%
\end{pgfscope}%
\begin{pgfscope}%
\pgfpathrectangle{\pgfqpoint{0.765000in}{0.660000in}}{\pgfqpoint{4.620000in}{4.620000in}}%
\pgfusepath{clip}%
\pgfsetbuttcap%
\pgfsetroundjoin%
\definecolor{currentfill}{rgb}{1.000000,0.576471,0.309804}%
\pgfsetfillcolor{currentfill}%
\pgfsetlinewidth{1.003750pt}%
\definecolor{currentstroke}{rgb}{1.000000,0.576471,0.309804}%
\pgfsetstrokecolor{currentstroke}%
\pgfsetdash{}{0pt}%
\pgfsys@defobject{currentmarker}{\pgfqpoint{-0.033333in}{-0.033333in}}{\pgfqpoint{0.033333in}{0.033333in}}{%
\pgfpathmoveto{\pgfqpoint{0.000000in}{-0.033333in}}%
\pgfpathcurveto{\pgfqpoint{0.008840in}{-0.033333in}}{\pgfqpoint{0.017319in}{-0.029821in}}{\pgfqpoint{0.023570in}{-0.023570in}}%
\pgfpathcurveto{\pgfqpoint{0.029821in}{-0.017319in}}{\pgfqpoint{0.033333in}{-0.008840in}}{\pgfqpoint{0.033333in}{0.000000in}}%
\pgfpathcurveto{\pgfqpoint{0.033333in}{0.008840in}}{\pgfqpoint{0.029821in}{0.017319in}}{\pgfqpoint{0.023570in}{0.023570in}}%
\pgfpathcurveto{\pgfqpoint{0.017319in}{0.029821in}}{\pgfqpoint{0.008840in}{0.033333in}}{\pgfqpoint{0.000000in}{0.033333in}}%
\pgfpathcurveto{\pgfqpoint{-0.008840in}{0.033333in}}{\pgfqpoint{-0.017319in}{0.029821in}}{\pgfqpoint{-0.023570in}{0.023570in}}%
\pgfpathcurveto{\pgfqpoint{-0.029821in}{0.017319in}}{\pgfqpoint{-0.033333in}{0.008840in}}{\pgfqpoint{-0.033333in}{0.000000in}}%
\pgfpathcurveto{\pgfqpoint{-0.033333in}{-0.008840in}}{\pgfqpoint{-0.029821in}{-0.017319in}}{\pgfqpoint{-0.023570in}{-0.023570in}}%
\pgfpathcurveto{\pgfqpoint{-0.017319in}{-0.029821in}}{\pgfqpoint{-0.008840in}{-0.033333in}}{\pgfqpoint{0.000000in}{-0.033333in}}%
\pgfpathlineto{\pgfqpoint{0.000000in}{-0.033333in}}%
\pgfpathclose%
\pgfusepath{stroke,fill}%
}%
\begin{pgfscope}%
\pgfsys@transformshift{2.742567in}{2.544196in}%
\pgfsys@useobject{currentmarker}{}%
\end{pgfscope}%
\end{pgfscope}%
\begin{pgfscope}%
\pgfpathrectangle{\pgfqpoint{0.765000in}{0.660000in}}{\pgfqpoint{4.620000in}{4.620000in}}%
\pgfusepath{clip}%
\pgfsetrectcap%
\pgfsetroundjoin%
\pgfsetlinewidth{1.204500pt}%
\definecolor{currentstroke}{rgb}{1.000000,0.576471,0.309804}%
\pgfsetstrokecolor{currentstroke}%
\pgfsetdash{}{0pt}%
\pgfpathmoveto{\pgfqpoint{4.376535in}{2.753859in}}%
\pgfusepath{stroke}%
\end{pgfscope}%
\begin{pgfscope}%
\pgfpathrectangle{\pgfqpoint{0.765000in}{0.660000in}}{\pgfqpoint{4.620000in}{4.620000in}}%
\pgfusepath{clip}%
\pgfsetbuttcap%
\pgfsetroundjoin%
\definecolor{currentfill}{rgb}{1.000000,0.576471,0.309804}%
\pgfsetfillcolor{currentfill}%
\pgfsetlinewidth{1.003750pt}%
\definecolor{currentstroke}{rgb}{1.000000,0.576471,0.309804}%
\pgfsetstrokecolor{currentstroke}%
\pgfsetdash{}{0pt}%
\pgfsys@defobject{currentmarker}{\pgfqpoint{-0.033333in}{-0.033333in}}{\pgfqpoint{0.033333in}{0.033333in}}{%
\pgfpathmoveto{\pgfqpoint{0.000000in}{-0.033333in}}%
\pgfpathcurveto{\pgfqpoint{0.008840in}{-0.033333in}}{\pgfqpoint{0.017319in}{-0.029821in}}{\pgfqpoint{0.023570in}{-0.023570in}}%
\pgfpathcurveto{\pgfqpoint{0.029821in}{-0.017319in}}{\pgfqpoint{0.033333in}{-0.008840in}}{\pgfqpoint{0.033333in}{0.000000in}}%
\pgfpathcurveto{\pgfqpoint{0.033333in}{0.008840in}}{\pgfqpoint{0.029821in}{0.017319in}}{\pgfqpoint{0.023570in}{0.023570in}}%
\pgfpathcurveto{\pgfqpoint{0.017319in}{0.029821in}}{\pgfqpoint{0.008840in}{0.033333in}}{\pgfqpoint{0.000000in}{0.033333in}}%
\pgfpathcurveto{\pgfqpoint{-0.008840in}{0.033333in}}{\pgfqpoint{-0.017319in}{0.029821in}}{\pgfqpoint{-0.023570in}{0.023570in}}%
\pgfpathcurveto{\pgfqpoint{-0.029821in}{0.017319in}}{\pgfqpoint{-0.033333in}{0.008840in}}{\pgfqpoint{-0.033333in}{0.000000in}}%
\pgfpathcurveto{\pgfqpoint{-0.033333in}{-0.008840in}}{\pgfqpoint{-0.029821in}{-0.017319in}}{\pgfqpoint{-0.023570in}{-0.023570in}}%
\pgfpathcurveto{\pgfqpoint{-0.017319in}{-0.029821in}}{\pgfqpoint{-0.008840in}{-0.033333in}}{\pgfqpoint{0.000000in}{-0.033333in}}%
\pgfpathlineto{\pgfqpoint{0.000000in}{-0.033333in}}%
\pgfpathclose%
\pgfusepath{stroke,fill}%
}%
\begin{pgfscope}%
\pgfsys@transformshift{4.376535in}{2.753859in}%
\pgfsys@useobject{currentmarker}{}%
\end{pgfscope}%
\end{pgfscope}%
\begin{pgfscope}%
\pgfpathrectangle{\pgfqpoint{0.765000in}{0.660000in}}{\pgfqpoint{4.620000in}{4.620000in}}%
\pgfusepath{clip}%
\pgfsetrectcap%
\pgfsetroundjoin%
\pgfsetlinewidth{1.204500pt}%
\definecolor{currentstroke}{rgb}{1.000000,0.576471,0.309804}%
\pgfsetstrokecolor{currentstroke}%
\pgfsetdash{}{0pt}%
\pgfpathmoveto{\pgfqpoint{2.695216in}{3.051958in}}%
\pgfusepath{stroke}%
\end{pgfscope}%
\begin{pgfscope}%
\pgfpathrectangle{\pgfqpoint{0.765000in}{0.660000in}}{\pgfqpoint{4.620000in}{4.620000in}}%
\pgfusepath{clip}%
\pgfsetbuttcap%
\pgfsetroundjoin%
\definecolor{currentfill}{rgb}{1.000000,0.576471,0.309804}%
\pgfsetfillcolor{currentfill}%
\pgfsetlinewidth{1.003750pt}%
\definecolor{currentstroke}{rgb}{1.000000,0.576471,0.309804}%
\pgfsetstrokecolor{currentstroke}%
\pgfsetdash{}{0pt}%
\pgfsys@defobject{currentmarker}{\pgfqpoint{-0.033333in}{-0.033333in}}{\pgfqpoint{0.033333in}{0.033333in}}{%
\pgfpathmoveto{\pgfqpoint{0.000000in}{-0.033333in}}%
\pgfpathcurveto{\pgfqpoint{0.008840in}{-0.033333in}}{\pgfqpoint{0.017319in}{-0.029821in}}{\pgfqpoint{0.023570in}{-0.023570in}}%
\pgfpathcurveto{\pgfqpoint{0.029821in}{-0.017319in}}{\pgfqpoint{0.033333in}{-0.008840in}}{\pgfqpoint{0.033333in}{0.000000in}}%
\pgfpathcurveto{\pgfqpoint{0.033333in}{0.008840in}}{\pgfqpoint{0.029821in}{0.017319in}}{\pgfqpoint{0.023570in}{0.023570in}}%
\pgfpathcurveto{\pgfqpoint{0.017319in}{0.029821in}}{\pgfqpoint{0.008840in}{0.033333in}}{\pgfqpoint{0.000000in}{0.033333in}}%
\pgfpathcurveto{\pgfqpoint{-0.008840in}{0.033333in}}{\pgfqpoint{-0.017319in}{0.029821in}}{\pgfqpoint{-0.023570in}{0.023570in}}%
\pgfpathcurveto{\pgfqpoint{-0.029821in}{0.017319in}}{\pgfqpoint{-0.033333in}{0.008840in}}{\pgfqpoint{-0.033333in}{0.000000in}}%
\pgfpathcurveto{\pgfqpoint{-0.033333in}{-0.008840in}}{\pgfqpoint{-0.029821in}{-0.017319in}}{\pgfqpoint{-0.023570in}{-0.023570in}}%
\pgfpathcurveto{\pgfqpoint{-0.017319in}{-0.029821in}}{\pgfqpoint{-0.008840in}{-0.033333in}}{\pgfqpoint{0.000000in}{-0.033333in}}%
\pgfpathlineto{\pgfqpoint{0.000000in}{-0.033333in}}%
\pgfpathclose%
\pgfusepath{stroke,fill}%
}%
\begin{pgfscope}%
\pgfsys@transformshift{2.695216in}{3.051958in}%
\pgfsys@useobject{currentmarker}{}%
\end{pgfscope}%
\end{pgfscope}%
\begin{pgfscope}%
\pgfpathrectangle{\pgfqpoint{0.765000in}{0.660000in}}{\pgfqpoint{4.620000in}{4.620000in}}%
\pgfusepath{clip}%
\pgfsetrectcap%
\pgfsetroundjoin%
\pgfsetlinewidth{1.204500pt}%
\definecolor{currentstroke}{rgb}{1.000000,0.576471,0.309804}%
\pgfsetstrokecolor{currentstroke}%
\pgfsetdash{}{0pt}%
\pgfpathmoveto{\pgfqpoint{2.697487in}{2.594797in}}%
\pgfusepath{stroke}%
\end{pgfscope}%
\begin{pgfscope}%
\pgfpathrectangle{\pgfqpoint{0.765000in}{0.660000in}}{\pgfqpoint{4.620000in}{4.620000in}}%
\pgfusepath{clip}%
\pgfsetbuttcap%
\pgfsetroundjoin%
\definecolor{currentfill}{rgb}{1.000000,0.576471,0.309804}%
\pgfsetfillcolor{currentfill}%
\pgfsetlinewidth{1.003750pt}%
\definecolor{currentstroke}{rgb}{1.000000,0.576471,0.309804}%
\pgfsetstrokecolor{currentstroke}%
\pgfsetdash{}{0pt}%
\pgfsys@defobject{currentmarker}{\pgfqpoint{-0.033333in}{-0.033333in}}{\pgfqpoint{0.033333in}{0.033333in}}{%
\pgfpathmoveto{\pgfqpoint{0.000000in}{-0.033333in}}%
\pgfpathcurveto{\pgfqpoint{0.008840in}{-0.033333in}}{\pgfqpoint{0.017319in}{-0.029821in}}{\pgfqpoint{0.023570in}{-0.023570in}}%
\pgfpathcurveto{\pgfqpoint{0.029821in}{-0.017319in}}{\pgfqpoint{0.033333in}{-0.008840in}}{\pgfqpoint{0.033333in}{0.000000in}}%
\pgfpathcurveto{\pgfqpoint{0.033333in}{0.008840in}}{\pgfqpoint{0.029821in}{0.017319in}}{\pgfqpoint{0.023570in}{0.023570in}}%
\pgfpathcurveto{\pgfqpoint{0.017319in}{0.029821in}}{\pgfqpoint{0.008840in}{0.033333in}}{\pgfqpoint{0.000000in}{0.033333in}}%
\pgfpathcurveto{\pgfqpoint{-0.008840in}{0.033333in}}{\pgfqpoint{-0.017319in}{0.029821in}}{\pgfqpoint{-0.023570in}{0.023570in}}%
\pgfpathcurveto{\pgfqpoint{-0.029821in}{0.017319in}}{\pgfqpoint{-0.033333in}{0.008840in}}{\pgfqpoint{-0.033333in}{0.000000in}}%
\pgfpathcurveto{\pgfqpoint{-0.033333in}{-0.008840in}}{\pgfqpoint{-0.029821in}{-0.017319in}}{\pgfqpoint{-0.023570in}{-0.023570in}}%
\pgfpathcurveto{\pgfqpoint{-0.017319in}{-0.029821in}}{\pgfqpoint{-0.008840in}{-0.033333in}}{\pgfqpoint{0.000000in}{-0.033333in}}%
\pgfpathlineto{\pgfqpoint{0.000000in}{-0.033333in}}%
\pgfpathclose%
\pgfusepath{stroke,fill}%
}%
\begin{pgfscope}%
\pgfsys@transformshift{2.697487in}{2.594797in}%
\pgfsys@useobject{currentmarker}{}%
\end{pgfscope}%
\end{pgfscope}%
\begin{pgfscope}%
\pgfpathrectangle{\pgfqpoint{0.765000in}{0.660000in}}{\pgfqpoint{4.620000in}{4.620000in}}%
\pgfusepath{clip}%
\pgfsetrectcap%
\pgfsetroundjoin%
\pgfsetlinewidth{1.204500pt}%
\definecolor{currentstroke}{rgb}{1.000000,0.576471,0.309804}%
\pgfsetstrokecolor{currentstroke}%
\pgfsetdash{}{0pt}%
\pgfpathmoveto{\pgfqpoint{2.717079in}{2.636880in}}%
\pgfusepath{stroke}%
\end{pgfscope}%
\begin{pgfscope}%
\pgfpathrectangle{\pgfqpoint{0.765000in}{0.660000in}}{\pgfqpoint{4.620000in}{4.620000in}}%
\pgfusepath{clip}%
\pgfsetbuttcap%
\pgfsetroundjoin%
\definecolor{currentfill}{rgb}{1.000000,0.576471,0.309804}%
\pgfsetfillcolor{currentfill}%
\pgfsetlinewidth{1.003750pt}%
\definecolor{currentstroke}{rgb}{1.000000,0.576471,0.309804}%
\pgfsetstrokecolor{currentstroke}%
\pgfsetdash{}{0pt}%
\pgfsys@defobject{currentmarker}{\pgfqpoint{-0.033333in}{-0.033333in}}{\pgfqpoint{0.033333in}{0.033333in}}{%
\pgfpathmoveto{\pgfqpoint{0.000000in}{-0.033333in}}%
\pgfpathcurveto{\pgfqpoint{0.008840in}{-0.033333in}}{\pgfqpoint{0.017319in}{-0.029821in}}{\pgfqpoint{0.023570in}{-0.023570in}}%
\pgfpathcurveto{\pgfqpoint{0.029821in}{-0.017319in}}{\pgfqpoint{0.033333in}{-0.008840in}}{\pgfqpoint{0.033333in}{0.000000in}}%
\pgfpathcurveto{\pgfqpoint{0.033333in}{0.008840in}}{\pgfqpoint{0.029821in}{0.017319in}}{\pgfqpoint{0.023570in}{0.023570in}}%
\pgfpathcurveto{\pgfqpoint{0.017319in}{0.029821in}}{\pgfqpoint{0.008840in}{0.033333in}}{\pgfqpoint{0.000000in}{0.033333in}}%
\pgfpathcurveto{\pgfqpoint{-0.008840in}{0.033333in}}{\pgfqpoint{-0.017319in}{0.029821in}}{\pgfqpoint{-0.023570in}{0.023570in}}%
\pgfpathcurveto{\pgfqpoint{-0.029821in}{0.017319in}}{\pgfqpoint{-0.033333in}{0.008840in}}{\pgfqpoint{-0.033333in}{0.000000in}}%
\pgfpathcurveto{\pgfqpoint{-0.033333in}{-0.008840in}}{\pgfqpoint{-0.029821in}{-0.017319in}}{\pgfqpoint{-0.023570in}{-0.023570in}}%
\pgfpathcurveto{\pgfqpoint{-0.017319in}{-0.029821in}}{\pgfqpoint{-0.008840in}{-0.033333in}}{\pgfqpoint{0.000000in}{-0.033333in}}%
\pgfpathlineto{\pgfqpoint{0.000000in}{-0.033333in}}%
\pgfpathclose%
\pgfusepath{stroke,fill}%
}%
\begin{pgfscope}%
\pgfsys@transformshift{2.717079in}{2.636880in}%
\pgfsys@useobject{currentmarker}{}%
\end{pgfscope}%
\end{pgfscope}%
\begin{pgfscope}%
\pgfpathrectangle{\pgfqpoint{0.765000in}{0.660000in}}{\pgfqpoint{4.620000in}{4.620000in}}%
\pgfusepath{clip}%
\pgfsetrectcap%
\pgfsetroundjoin%
\pgfsetlinewidth{1.204500pt}%
\definecolor{currentstroke}{rgb}{1.000000,0.576471,0.309804}%
\pgfsetstrokecolor{currentstroke}%
\pgfsetdash{}{0pt}%
\pgfpathmoveto{\pgfqpoint{4.923136in}{3.289740in}}%
\pgfusepath{stroke}%
\end{pgfscope}%
\begin{pgfscope}%
\pgfpathrectangle{\pgfqpoint{0.765000in}{0.660000in}}{\pgfqpoint{4.620000in}{4.620000in}}%
\pgfusepath{clip}%
\pgfsetbuttcap%
\pgfsetroundjoin%
\definecolor{currentfill}{rgb}{1.000000,0.576471,0.309804}%
\pgfsetfillcolor{currentfill}%
\pgfsetlinewidth{1.003750pt}%
\definecolor{currentstroke}{rgb}{1.000000,0.576471,0.309804}%
\pgfsetstrokecolor{currentstroke}%
\pgfsetdash{}{0pt}%
\pgfsys@defobject{currentmarker}{\pgfqpoint{-0.033333in}{-0.033333in}}{\pgfqpoint{0.033333in}{0.033333in}}{%
\pgfpathmoveto{\pgfqpoint{0.000000in}{-0.033333in}}%
\pgfpathcurveto{\pgfqpoint{0.008840in}{-0.033333in}}{\pgfqpoint{0.017319in}{-0.029821in}}{\pgfqpoint{0.023570in}{-0.023570in}}%
\pgfpathcurveto{\pgfqpoint{0.029821in}{-0.017319in}}{\pgfqpoint{0.033333in}{-0.008840in}}{\pgfqpoint{0.033333in}{0.000000in}}%
\pgfpathcurveto{\pgfqpoint{0.033333in}{0.008840in}}{\pgfqpoint{0.029821in}{0.017319in}}{\pgfqpoint{0.023570in}{0.023570in}}%
\pgfpathcurveto{\pgfqpoint{0.017319in}{0.029821in}}{\pgfqpoint{0.008840in}{0.033333in}}{\pgfqpoint{0.000000in}{0.033333in}}%
\pgfpathcurveto{\pgfqpoint{-0.008840in}{0.033333in}}{\pgfqpoint{-0.017319in}{0.029821in}}{\pgfqpoint{-0.023570in}{0.023570in}}%
\pgfpathcurveto{\pgfqpoint{-0.029821in}{0.017319in}}{\pgfqpoint{-0.033333in}{0.008840in}}{\pgfqpoint{-0.033333in}{0.000000in}}%
\pgfpathcurveto{\pgfqpoint{-0.033333in}{-0.008840in}}{\pgfqpoint{-0.029821in}{-0.017319in}}{\pgfqpoint{-0.023570in}{-0.023570in}}%
\pgfpathcurveto{\pgfqpoint{-0.017319in}{-0.029821in}}{\pgfqpoint{-0.008840in}{-0.033333in}}{\pgfqpoint{0.000000in}{-0.033333in}}%
\pgfpathlineto{\pgfqpoint{0.000000in}{-0.033333in}}%
\pgfpathclose%
\pgfusepath{stroke,fill}%
}%
\begin{pgfscope}%
\pgfsys@transformshift{4.923136in}{3.289740in}%
\pgfsys@useobject{currentmarker}{}%
\end{pgfscope}%
\end{pgfscope}%
\begin{pgfscope}%
\pgfpathrectangle{\pgfqpoint{0.765000in}{0.660000in}}{\pgfqpoint{4.620000in}{4.620000in}}%
\pgfusepath{clip}%
\pgfsetrectcap%
\pgfsetroundjoin%
\pgfsetlinewidth{1.204500pt}%
\definecolor{currentstroke}{rgb}{1.000000,0.576471,0.309804}%
\pgfsetstrokecolor{currentstroke}%
\pgfsetdash{}{0pt}%
\pgfpathmoveto{\pgfqpoint{3.196396in}{2.426574in}}%
\pgfusepath{stroke}%
\end{pgfscope}%
\begin{pgfscope}%
\pgfpathrectangle{\pgfqpoint{0.765000in}{0.660000in}}{\pgfqpoint{4.620000in}{4.620000in}}%
\pgfusepath{clip}%
\pgfsetbuttcap%
\pgfsetroundjoin%
\definecolor{currentfill}{rgb}{1.000000,0.576471,0.309804}%
\pgfsetfillcolor{currentfill}%
\pgfsetlinewidth{1.003750pt}%
\definecolor{currentstroke}{rgb}{1.000000,0.576471,0.309804}%
\pgfsetstrokecolor{currentstroke}%
\pgfsetdash{}{0pt}%
\pgfsys@defobject{currentmarker}{\pgfqpoint{-0.033333in}{-0.033333in}}{\pgfqpoint{0.033333in}{0.033333in}}{%
\pgfpathmoveto{\pgfqpoint{0.000000in}{-0.033333in}}%
\pgfpathcurveto{\pgfqpoint{0.008840in}{-0.033333in}}{\pgfqpoint{0.017319in}{-0.029821in}}{\pgfqpoint{0.023570in}{-0.023570in}}%
\pgfpathcurveto{\pgfqpoint{0.029821in}{-0.017319in}}{\pgfqpoint{0.033333in}{-0.008840in}}{\pgfqpoint{0.033333in}{0.000000in}}%
\pgfpathcurveto{\pgfqpoint{0.033333in}{0.008840in}}{\pgfqpoint{0.029821in}{0.017319in}}{\pgfqpoint{0.023570in}{0.023570in}}%
\pgfpathcurveto{\pgfqpoint{0.017319in}{0.029821in}}{\pgfqpoint{0.008840in}{0.033333in}}{\pgfqpoint{0.000000in}{0.033333in}}%
\pgfpathcurveto{\pgfqpoint{-0.008840in}{0.033333in}}{\pgfqpoint{-0.017319in}{0.029821in}}{\pgfqpoint{-0.023570in}{0.023570in}}%
\pgfpathcurveto{\pgfqpoint{-0.029821in}{0.017319in}}{\pgfqpoint{-0.033333in}{0.008840in}}{\pgfqpoint{-0.033333in}{0.000000in}}%
\pgfpathcurveto{\pgfqpoint{-0.033333in}{-0.008840in}}{\pgfqpoint{-0.029821in}{-0.017319in}}{\pgfqpoint{-0.023570in}{-0.023570in}}%
\pgfpathcurveto{\pgfqpoint{-0.017319in}{-0.029821in}}{\pgfqpoint{-0.008840in}{-0.033333in}}{\pgfqpoint{0.000000in}{-0.033333in}}%
\pgfpathlineto{\pgfqpoint{0.000000in}{-0.033333in}}%
\pgfpathclose%
\pgfusepath{stroke,fill}%
}%
\begin{pgfscope}%
\pgfsys@transformshift{3.196396in}{2.426574in}%
\pgfsys@useobject{currentmarker}{}%
\end{pgfscope}%
\end{pgfscope}%
\begin{pgfscope}%
\pgfpathrectangle{\pgfqpoint{0.765000in}{0.660000in}}{\pgfqpoint{4.620000in}{4.620000in}}%
\pgfusepath{clip}%
\pgfsetrectcap%
\pgfsetroundjoin%
\pgfsetlinewidth{1.204500pt}%
\definecolor{currentstroke}{rgb}{1.000000,0.576471,0.309804}%
\pgfsetstrokecolor{currentstroke}%
\pgfsetdash{}{0pt}%
\pgfpathmoveto{\pgfqpoint{4.169633in}{2.725769in}}%
\pgfusepath{stroke}%
\end{pgfscope}%
\begin{pgfscope}%
\pgfpathrectangle{\pgfqpoint{0.765000in}{0.660000in}}{\pgfqpoint{4.620000in}{4.620000in}}%
\pgfusepath{clip}%
\pgfsetbuttcap%
\pgfsetroundjoin%
\definecolor{currentfill}{rgb}{1.000000,0.576471,0.309804}%
\pgfsetfillcolor{currentfill}%
\pgfsetlinewidth{1.003750pt}%
\definecolor{currentstroke}{rgb}{1.000000,0.576471,0.309804}%
\pgfsetstrokecolor{currentstroke}%
\pgfsetdash{}{0pt}%
\pgfsys@defobject{currentmarker}{\pgfqpoint{-0.033333in}{-0.033333in}}{\pgfqpoint{0.033333in}{0.033333in}}{%
\pgfpathmoveto{\pgfqpoint{0.000000in}{-0.033333in}}%
\pgfpathcurveto{\pgfqpoint{0.008840in}{-0.033333in}}{\pgfqpoint{0.017319in}{-0.029821in}}{\pgfqpoint{0.023570in}{-0.023570in}}%
\pgfpathcurveto{\pgfqpoint{0.029821in}{-0.017319in}}{\pgfqpoint{0.033333in}{-0.008840in}}{\pgfqpoint{0.033333in}{0.000000in}}%
\pgfpathcurveto{\pgfqpoint{0.033333in}{0.008840in}}{\pgfqpoint{0.029821in}{0.017319in}}{\pgfqpoint{0.023570in}{0.023570in}}%
\pgfpathcurveto{\pgfqpoint{0.017319in}{0.029821in}}{\pgfqpoint{0.008840in}{0.033333in}}{\pgfqpoint{0.000000in}{0.033333in}}%
\pgfpathcurveto{\pgfqpoint{-0.008840in}{0.033333in}}{\pgfqpoint{-0.017319in}{0.029821in}}{\pgfqpoint{-0.023570in}{0.023570in}}%
\pgfpathcurveto{\pgfqpoint{-0.029821in}{0.017319in}}{\pgfqpoint{-0.033333in}{0.008840in}}{\pgfqpoint{-0.033333in}{0.000000in}}%
\pgfpathcurveto{\pgfqpoint{-0.033333in}{-0.008840in}}{\pgfqpoint{-0.029821in}{-0.017319in}}{\pgfqpoint{-0.023570in}{-0.023570in}}%
\pgfpathcurveto{\pgfqpoint{-0.017319in}{-0.029821in}}{\pgfqpoint{-0.008840in}{-0.033333in}}{\pgfqpoint{0.000000in}{-0.033333in}}%
\pgfpathlineto{\pgfqpoint{0.000000in}{-0.033333in}}%
\pgfpathclose%
\pgfusepath{stroke,fill}%
}%
\begin{pgfscope}%
\pgfsys@transformshift{4.169633in}{2.725769in}%
\pgfsys@useobject{currentmarker}{}%
\end{pgfscope}%
\end{pgfscope}%
\begin{pgfscope}%
\pgfpathrectangle{\pgfqpoint{0.765000in}{0.660000in}}{\pgfqpoint{4.620000in}{4.620000in}}%
\pgfusepath{clip}%
\pgfsetrectcap%
\pgfsetroundjoin%
\pgfsetlinewidth{1.204500pt}%
\definecolor{currentstroke}{rgb}{1.000000,0.576471,0.309804}%
\pgfsetstrokecolor{currentstroke}%
\pgfsetdash{}{0pt}%
\pgfpathmoveto{\pgfqpoint{3.047359in}{3.553061in}}%
\pgfusepath{stroke}%
\end{pgfscope}%
\begin{pgfscope}%
\pgfpathrectangle{\pgfqpoint{0.765000in}{0.660000in}}{\pgfqpoint{4.620000in}{4.620000in}}%
\pgfusepath{clip}%
\pgfsetbuttcap%
\pgfsetroundjoin%
\definecolor{currentfill}{rgb}{1.000000,0.576471,0.309804}%
\pgfsetfillcolor{currentfill}%
\pgfsetlinewidth{1.003750pt}%
\definecolor{currentstroke}{rgb}{1.000000,0.576471,0.309804}%
\pgfsetstrokecolor{currentstroke}%
\pgfsetdash{}{0pt}%
\pgfsys@defobject{currentmarker}{\pgfqpoint{-0.033333in}{-0.033333in}}{\pgfqpoint{0.033333in}{0.033333in}}{%
\pgfpathmoveto{\pgfqpoint{0.000000in}{-0.033333in}}%
\pgfpathcurveto{\pgfqpoint{0.008840in}{-0.033333in}}{\pgfqpoint{0.017319in}{-0.029821in}}{\pgfqpoint{0.023570in}{-0.023570in}}%
\pgfpathcurveto{\pgfqpoint{0.029821in}{-0.017319in}}{\pgfqpoint{0.033333in}{-0.008840in}}{\pgfqpoint{0.033333in}{0.000000in}}%
\pgfpathcurveto{\pgfqpoint{0.033333in}{0.008840in}}{\pgfqpoint{0.029821in}{0.017319in}}{\pgfqpoint{0.023570in}{0.023570in}}%
\pgfpathcurveto{\pgfqpoint{0.017319in}{0.029821in}}{\pgfqpoint{0.008840in}{0.033333in}}{\pgfqpoint{0.000000in}{0.033333in}}%
\pgfpathcurveto{\pgfqpoint{-0.008840in}{0.033333in}}{\pgfqpoint{-0.017319in}{0.029821in}}{\pgfqpoint{-0.023570in}{0.023570in}}%
\pgfpathcurveto{\pgfqpoint{-0.029821in}{0.017319in}}{\pgfqpoint{-0.033333in}{0.008840in}}{\pgfqpoint{-0.033333in}{0.000000in}}%
\pgfpathcurveto{\pgfqpoint{-0.033333in}{-0.008840in}}{\pgfqpoint{-0.029821in}{-0.017319in}}{\pgfqpoint{-0.023570in}{-0.023570in}}%
\pgfpathcurveto{\pgfqpoint{-0.017319in}{-0.029821in}}{\pgfqpoint{-0.008840in}{-0.033333in}}{\pgfqpoint{0.000000in}{-0.033333in}}%
\pgfpathlineto{\pgfqpoint{0.000000in}{-0.033333in}}%
\pgfpathclose%
\pgfusepath{stroke,fill}%
}%
\begin{pgfscope}%
\pgfsys@transformshift{3.047359in}{3.553061in}%
\pgfsys@useobject{currentmarker}{}%
\end{pgfscope}%
\end{pgfscope}%
\begin{pgfscope}%
\pgfpathrectangle{\pgfqpoint{0.765000in}{0.660000in}}{\pgfqpoint{4.620000in}{4.620000in}}%
\pgfusepath{clip}%
\pgfsetrectcap%
\pgfsetroundjoin%
\pgfsetlinewidth{1.204500pt}%
\definecolor{currentstroke}{rgb}{1.000000,0.576471,0.309804}%
\pgfsetstrokecolor{currentstroke}%
\pgfsetdash{}{0pt}%
\pgfpathmoveto{\pgfqpoint{3.424540in}{2.719085in}}%
\pgfusepath{stroke}%
\end{pgfscope}%
\begin{pgfscope}%
\pgfpathrectangle{\pgfqpoint{0.765000in}{0.660000in}}{\pgfqpoint{4.620000in}{4.620000in}}%
\pgfusepath{clip}%
\pgfsetbuttcap%
\pgfsetroundjoin%
\definecolor{currentfill}{rgb}{1.000000,0.576471,0.309804}%
\pgfsetfillcolor{currentfill}%
\pgfsetlinewidth{1.003750pt}%
\definecolor{currentstroke}{rgb}{1.000000,0.576471,0.309804}%
\pgfsetstrokecolor{currentstroke}%
\pgfsetdash{}{0pt}%
\pgfsys@defobject{currentmarker}{\pgfqpoint{-0.033333in}{-0.033333in}}{\pgfqpoint{0.033333in}{0.033333in}}{%
\pgfpathmoveto{\pgfqpoint{0.000000in}{-0.033333in}}%
\pgfpathcurveto{\pgfqpoint{0.008840in}{-0.033333in}}{\pgfqpoint{0.017319in}{-0.029821in}}{\pgfqpoint{0.023570in}{-0.023570in}}%
\pgfpathcurveto{\pgfqpoint{0.029821in}{-0.017319in}}{\pgfqpoint{0.033333in}{-0.008840in}}{\pgfqpoint{0.033333in}{0.000000in}}%
\pgfpathcurveto{\pgfqpoint{0.033333in}{0.008840in}}{\pgfqpoint{0.029821in}{0.017319in}}{\pgfqpoint{0.023570in}{0.023570in}}%
\pgfpathcurveto{\pgfqpoint{0.017319in}{0.029821in}}{\pgfqpoint{0.008840in}{0.033333in}}{\pgfqpoint{0.000000in}{0.033333in}}%
\pgfpathcurveto{\pgfqpoint{-0.008840in}{0.033333in}}{\pgfqpoint{-0.017319in}{0.029821in}}{\pgfqpoint{-0.023570in}{0.023570in}}%
\pgfpathcurveto{\pgfqpoint{-0.029821in}{0.017319in}}{\pgfqpoint{-0.033333in}{0.008840in}}{\pgfqpoint{-0.033333in}{0.000000in}}%
\pgfpathcurveto{\pgfqpoint{-0.033333in}{-0.008840in}}{\pgfqpoint{-0.029821in}{-0.017319in}}{\pgfqpoint{-0.023570in}{-0.023570in}}%
\pgfpathcurveto{\pgfqpoint{-0.017319in}{-0.029821in}}{\pgfqpoint{-0.008840in}{-0.033333in}}{\pgfqpoint{0.000000in}{-0.033333in}}%
\pgfpathlineto{\pgfqpoint{0.000000in}{-0.033333in}}%
\pgfpathclose%
\pgfusepath{stroke,fill}%
}%
\begin{pgfscope}%
\pgfsys@transformshift{3.424540in}{2.719085in}%
\pgfsys@useobject{currentmarker}{}%
\end{pgfscope}%
\end{pgfscope}%
\begin{pgfscope}%
\pgfpathrectangle{\pgfqpoint{0.765000in}{0.660000in}}{\pgfqpoint{4.620000in}{4.620000in}}%
\pgfusepath{clip}%
\pgfsetrectcap%
\pgfsetroundjoin%
\pgfsetlinewidth{1.204500pt}%
\definecolor{currentstroke}{rgb}{1.000000,0.576471,0.309804}%
\pgfsetstrokecolor{currentstroke}%
\pgfsetdash{}{0pt}%
\pgfpathmoveto{\pgfqpoint{3.811663in}{2.678737in}}%
\pgfusepath{stroke}%
\end{pgfscope}%
\begin{pgfscope}%
\pgfpathrectangle{\pgfqpoint{0.765000in}{0.660000in}}{\pgfqpoint{4.620000in}{4.620000in}}%
\pgfusepath{clip}%
\pgfsetbuttcap%
\pgfsetroundjoin%
\definecolor{currentfill}{rgb}{1.000000,0.576471,0.309804}%
\pgfsetfillcolor{currentfill}%
\pgfsetlinewidth{1.003750pt}%
\definecolor{currentstroke}{rgb}{1.000000,0.576471,0.309804}%
\pgfsetstrokecolor{currentstroke}%
\pgfsetdash{}{0pt}%
\pgfsys@defobject{currentmarker}{\pgfqpoint{-0.033333in}{-0.033333in}}{\pgfqpoint{0.033333in}{0.033333in}}{%
\pgfpathmoveto{\pgfqpoint{0.000000in}{-0.033333in}}%
\pgfpathcurveto{\pgfqpoint{0.008840in}{-0.033333in}}{\pgfqpoint{0.017319in}{-0.029821in}}{\pgfqpoint{0.023570in}{-0.023570in}}%
\pgfpathcurveto{\pgfqpoint{0.029821in}{-0.017319in}}{\pgfqpoint{0.033333in}{-0.008840in}}{\pgfqpoint{0.033333in}{0.000000in}}%
\pgfpathcurveto{\pgfqpoint{0.033333in}{0.008840in}}{\pgfqpoint{0.029821in}{0.017319in}}{\pgfqpoint{0.023570in}{0.023570in}}%
\pgfpathcurveto{\pgfqpoint{0.017319in}{0.029821in}}{\pgfqpoint{0.008840in}{0.033333in}}{\pgfqpoint{0.000000in}{0.033333in}}%
\pgfpathcurveto{\pgfqpoint{-0.008840in}{0.033333in}}{\pgfqpoint{-0.017319in}{0.029821in}}{\pgfqpoint{-0.023570in}{0.023570in}}%
\pgfpathcurveto{\pgfqpoint{-0.029821in}{0.017319in}}{\pgfqpoint{-0.033333in}{0.008840in}}{\pgfqpoint{-0.033333in}{0.000000in}}%
\pgfpathcurveto{\pgfqpoint{-0.033333in}{-0.008840in}}{\pgfqpoint{-0.029821in}{-0.017319in}}{\pgfqpoint{-0.023570in}{-0.023570in}}%
\pgfpathcurveto{\pgfqpoint{-0.017319in}{-0.029821in}}{\pgfqpoint{-0.008840in}{-0.033333in}}{\pgfqpoint{0.000000in}{-0.033333in}}%
\pgfpathlineto{\pgfqpoint{0.000000in}{-0.033333in}}%
\pgfpathclose%
\pgfusepath{stroke,fill}%
}%
\begin{pgfscope}%
\pgfsys@transformshift{3.811663in}{2.678737in}%
\pgfsys@useobject{currentmarker}{}%
\end{pgfscope}%
\end{pgfscope}%
\begin{pgfscope}%
\pgfpathrectangle{\pgfqpoint{0.765000in}{0.660000in}}{\pgfqpoint{4.620000in}{4.620000in}}%
\pgfusepath{clip}%
\pgfsetrectcap%
\pgfsetroundjoin%
\pgfsetlinewidth{1.204500pt}%
\definecolor{currentstroke}{rgb}{1.000000,0.576471,0.309804}%
\pgfsetstrokecolor{currentstroke}%
\pgfsetdash{}{0pt}%
\pgfpathmoveto{\pgfqpoint{2.707118in}{2.475054in}}%
\pgfusepath{stroke}%
\end{pgfscope}%
\begin{pgfscope}%
\pgfpathrectangle{\pgfqpoint{0.765000in}{0.660000in}}{\pgfqpoint{4.620000in}{4.620000in}}%
\pgfusepath{clip}%
\pgfsetbuttcap%
\pgfsetroundjoin%
\definecolor{currentfill}{rgb}{1.000000,0.576471,0.309804}%
\pgfsetfillcolor{currentfill}%
\pgfsetlinewidth{1.003750pt}%
\definecolor{currentstroke}{rgb}{1.000000,0.576471,0.309804}%
\pgfsetstrokecolor{currentstroke}%
\pgfsetdash{}{0pt}%
\pgfsys@defobject{currentmarker}{\pgfqpoint{-0.033333in}{-0.033333in}}{\pgfqpoint{0.033333in}{0.033333in}}{%
\pgfpathmoveto{\pgfqpoint{0.000000in}{-0.033333in}}%
\pgfpathcurveto{\pgfqpoint{0.008840in}{-0.033333in}}{\pgfqpoint{0.017319in}{-0.029821in}}{\pgfqpoint{0.023570in}{-0.023570in}}%
\pgfpathcurveto{\pgfqpoint{0.029821in}{-0.017319in}}{\pgfqpoint{0.033333in}{-0.008840in}}{\pgfqpoint{0.033333in}{0.000000in}}%
\pgfpathcurveto{\pgfqpoint{0.033333in}{0.008840in}}{\pgfqpoint{0.029821in}{0.017319in}}{\pgfqpoint{0.023570in}{0.023570in}}%
\pgfpathcurveto{\pgfqpoint{0.017319in}{0.029821in}}{\pgfqpoint{0.008840in}{0.033333in}}{\pgfqpoint{0.000000in}{0.033333in}}%
\pgfpathcurveto{\pgfqpoint{-0.008840in}{0.033333in}}{\pgfqpoint{-0.017319in}{0.029821in}}{\pgfqpoint{-0.023570in}{0.023570in}}%
\pgfpathcurveto{\pgfqpoint{-0.029821in}{0.017319in}}{\pgfqpoint{-0.033333in}{0.008840in}}{\pgfqpoint{-0.033333in}{0.000000in}}%
\pgfpathcurveto{\pgfqpoint{-0.033333in}{-0.008840in}}{\pgfqpoint{-0.029821in}{-0.017319in}}{\pgfqpoint{-0.023570in}{-0.023570in}}%
\pgfpathcurveto{\pgfqpoint{-0.017319in}{-0.029821in}}{\pgfqpoint{-0.008840in}{-0.033333in}}{\pgfqpoint{0.000000in}{-0.033333in}}%
\pgfpathlineto{\pgfqpoint{0.000000in}{-0.033333in}}%
\pgfpathclose%
\pgfusepath{stroke,fill}%
}%
\begin{pgfscope}%
\pgfsys@transformshift{2.707118in}{2.475054in}%
\pgfsys@useobject{currentmarker}{}%
\end{pgfscope}%
\end{pgfscope}%
\begin{pgfscope}%
\pgfpathrectangle{\pgfqpoint{0.765000in}{0.660000in}}{\pgfqpoint{4.620000in}{4.620000in}}%
\pgfusepath{clip}%
\pgfsetrectcap%
\pgfsetroundjoin%
\pgfsetlinewidth{1.204500pt}%
\definecolor{currentstroke}{rgb}{1.000000,0.576471,0.309804}%
\pgfsetstrokecolor{currentstroke}%
\pgfsetdash{}{0pt}%
\pgfpathmoveto{\pgfqpoint{3.857217in}{2.697247in}}%
\pgfusepath{stroke}%
\end{pgfscope}%
\begin{pgfscope}%
\pgfpathrectangle{\pgfqpoint{0.765000in}{0.660000in}}{\pgfqpoint{4.620000in}{4.620000in}}%
\pgfusepath{clip}%
\pgfsetbuttcap%
\pgfsetroundjoin%
\definecolor{currentfill}{rgb}{1.000000,0.576471,0.309804}%
\pgfsetfillcolor{currentfill}%
\pgfsetlinewidth{1.003750pt}%
\definecolor{currentstroke}{rgb}{1.000000,0.576471,0.309804}%
\pgfsetstrokecolor{currentstroke}%
\pgfsetdash{}{0pt}%
\pgfsys@defobject{currentmarker}{\pgfqpoint{-0.033333in}{-0.033333in}}{\pgfqpoint{0.033333in}{0.033333in}}{%
\pgfpathmoveto{\pgfqpoint{0.000000in}{-0.033333in}}%
\pgfpathcurveto{\pgfqpoint{0.008840in}{-0.033333in}}{\pgfqpoint{0.017319in}{-0.029821in}}{\pgfqpoint{0.023570in}{-0.023570in}}%
\pgfpathcurveto{\pgfqpoint{0.029821in}{-0.017319in}}{\pgfqpoint{0.033333in}{-0.008840in}}{\pgfqpoint{0.033333in}{0.000000in}}%
\pgfpathcurveto{\pgfqpoint{0.033333in}{0.008840in}}{\pgfqpoint{0.029821in}{0.017319in}}{\pgfqpoint{0.023570in}{0.023570in}}%
\pgfpathcurveto{\pgfqpoint{0.017319in}{0.029821in}}{\pgfqpoint{0.008840in}{0.033333in}}{\pgfqpoint{0.000000in}{0.033333in}}%
\pgfpathcurveto{\pgfqpoint{-0.008840in}{0.033333in}}{\pgfqpoint{-0.017319in}{0.029821in}}{\pgfqpoint{-0.023570in}{0.023570in}}%
\pgfpathcurveto{\pgfqpoint{-0.029821in}{0.017319in}}{\pgfqpoint{-0.033333in}{0.008840in}}{\pgfqpoint{-0.033333in}{0.000000in}}%
\pgfpathcurveto{\pgfqpoint{-0.033333in}{-0.008840in}}{\pgfqpoint{-0.029821in}{-0.017319in}}{\pgfqpoint{-0.023570in}{-0.023570in}}%
\pgfpathcurveto{\pgfqpoint{-0.017319in}{-0.029821in}}{\pgfqpoint{-0.008840in}{-0.033333in}}{\pgfqpoint{0.000000in}{-0.033333in}}%
\pgfpathlineto{\pgfqpoint{0.000000in}{-0.033333in}}%
\pgfpathclose%
\pgfusepath{stroke,fill}%
}%
\begin{pgfscope}%
\pgfsys@transformshift{3.857217in}{2.697247in}%
\pgfsys@useobject{currentmarker}{}%
\end{pgfscope}%
\end{pgfscope}%
\begin{pgfscope}%
\pgfpathrectangle{\pgfqpoint{0.765000in}{0.660000in}}{\pgfqpoint{4.620000in}{4.620000in}}%
\pgfusepath{clip}%
\pgfsetrectcap%
\pgfsetroundjoin%
\pgfsetlinewidth{1.204500pt}%
\definecolor{currentstroke}{rgb}{1.000000,0.576471,0.309804}%
\pgfsetstrokecolor{currentstroke}%
\pgfsetdash{}{0pt}%
\pgfpathmoveto{\pgfqpoint{3.358780in}{2.339945in}}%
\pgfusepath{stroke}%
\end{pgfscope}%
\begin{pgfscope}%
\pgfpathrectangle{\pgfqpoint{0.765000in}{0.660000in}}{\pgfqpoint{4.620000in}{4.620000in}}%
\pgfusepath{clip}%
\pgfsetbuttcap%
\pgfsetroundjoin%
\definecolor{currentfill}{rgb}{1.000000,0.576471,0.309804}%
\pgfsetfillcolor{currentfill}%
\pgfsetlinewidth{1.003750pt}%
\definecolor{currentstroke}{rgb}{1.000000,0.576471,0.309804}%
\pgfsetstrokecolor{currentstroke}%
\pgfsetdash{}{0pt}%
\pgfsys@defobject{currentmarker}{\pgfqpoint{-0.033333in}{-0.033333in}}{\pgfqpoint{0.033333in}{0.033333in}}{%
\pgfpathmoveto{\pgfqpoint{0.000000in}{-0.033333in}}%
\pgfpathcurveto{\pgfqpoint{0.008840in}{-0.033333in}}{\pgfqpoint{0.017319in}{-0.029821in}}{\pgfqpoint{0.023570in}{-0.023570in}}%
\pgfpathcurveto{\pgfqpoint{0.029821in}{-0.017319in}}{\pgfqpoint{0.033333in}{-0.008840in}}{\pgfqpoint{0.033333in}{0.000000in}}%
\pgfpathcurveto{\pgfqpoint{0.033333in}{0.008840in}}{\pgfqpoint{0.029821in}{0.017319in}}{\pgfqpoint{0.023570in}{0.023570in}}%
\pgfpathcurveto{\pgfqpoint{0.017319in}{0.029821in}}{\pgfqpoint{0.008840in}{0.033333in}}{\pgfqpoint{0.000000in}{0.033333in}}%
\pgfpathcurveto{\pgfqpoint{-0.008840in}{0.033333in}}{\pgfqpoint{-0.017319in}{0.029821in}}{\pgfqpoint{-0.023570in}{0.023570in}}%
\pgfpathcurveto{\pgfqpoint{-0.029821in}{0.017319in}}{\pgfqpoint{-0.033333in}{0.008840in}}{\pgfqpoint{-0.033333in}{0.000000in}}%
\pgfpathcurveto{\pgfqpoint{-0.033333in}{-0.008840in}}{\pgfqpoint{-0.029821in}{-0.017319in}}{\pgfqpoint{-0.023570in}{-0.023570in}}%
\pgfpathcurveto{\pgfqpoint{-0.017319in}{-0.029821in}}{\pgfqpoint{-0.008840in}{-0.033333in}}{\pgfqpoint{0.000000in}{-0.033333in}}%
\pgfpathlineto{\pgfqpoint{0.000000in}{-0.033333in}}%
\pgfpathclose%
\pgfusepath{stroke,fill}%
}%
\begin{pgfscope}%
\pgfsys@transformshift{3.358780in}{2.339945in}%
\pgfsys@useobject{currentmarker}{}%
\end{pgfscope}%
\end{pgfscope}%
\begin{pgfscope}%
\pgfpathrectangle{\pgfqpoint{0.765000in}{0.660000in}}{\pgfqpoint{4.620000in}{4.620000in}}%
\pgfusepath{clip}%
\pgfsetrectcap%
\pgfsetroundjoin%
\pgfsetlinewidth{1.204500pt}%
\definecolor{currentstroke}{rgb}{1.000000,0.576471,0.309804}%
\pgfsetstrokecolor{currentstroke}%
\pgfsetdash{}{0pt}%
\pgfpathmoveto{\pgfqpoint{2.731879in}{2.093345in}}%
\pgfusepath{stroke}%
\end{pgfscope}%
\begin{pgfscope}%
\pgfpathrectangle{\pgfqpoint{0.765000in}{0.660000in}}{\pgfqpoint{4.620000in}{4.620000in}}%
\pgfusepath{clip}%
\pgfsetbuttcap%
\pgfsetroundjoin%
\definecolor{currentfill}{rgb}{1.000000,0.576471,0.309804}%
\pgfsetfillcolor{currentfill}%
\pgfsetlinewidth{1.003750pt}%
\definecolor{currentstroke}{rgb}{1.000000,0.576471,0.309804}%
\pgfsetstrokecolor{currentstroke}%
\pgfsetdash{}{0pt}%
\pgfsys@defobject{currentmarker}{\pgfqpoint{-0.033333in}{-0.033333in}}{\pgfqpoint{0.033333in}{0.033333in}}{%
\pgfpathmoveto{\pgfqpoint{0.000000in}{-0.033333in}}%
\pgfpathcurveto{\pgfqpoint{0.008840in}{-0.033333in}}{\pgfqpoint{0.017319in}{-0.029821in}}{\pgfqpoint{0.023570in}{-0.023570in}}%
\pgfpathcurveto{\pgfqpoint{0.029821in}{-0.017319in}}{\pgfqpoint{0.033333in}{-0.008840in}}{\pgfqpoint{0.033333in}{0.000000in}}%
\pgfpathcurveto{\pgfqpoint{0.033333in}{0.008840in}}{\pgfqpoint{0.029821in}{0.017319in}}{\pgfqpoint{0.023570in}{0.023570in}}%
\pgfpathcurveto{\pgfqpoint{0.017319in}{0.029821in}}{\pgfqpoint{0.008840in}{0.033333in}}{\pgfqpoint{0.000000in}{0.033333in}}%
\pgfpathcurveto{\pgfqpoint{-0.008840in}{0.033333in}}{\pgfqpoint{-0.017319in}{0.029821in}}{\pgfqpoint{-0.023570in}{0.023570in}}%
\pgfpathcurveto{\pgfqpoint{-0.029821in}{0.017319in}}{\pgfqpoint{-0.033333in}{0.008840in}}{\pgfqpoint{-0.033333in}{0.000000in}}%
\pgfpathcurveto{\pgfqpoint{-0.033333in}{-0.008840in}}{\pgfqpoint{-0.029821in}{-0.017319in}}{\pgfqpoint{-0.023570in}{-0.023570in}}%
\pgfpathcurveto{\pgfqpoint{-0.017319in}{-0.029821in}}{\pgfqpoint{-0.008840in}{-0.033333in}}{\pgfqpoint{0.000000in}{-0.033333in}}%
\pgfpathlineto{\pgfqpoint{0.000000in}{-0.033333in}}%
\pgfpathclose%
\pgfusepath{stroke,fill}%
}%
\begin{pgfscope}%
\pgfsys@transformshift{2.731879in}{2.093345in}%
\pgfsys@useobject{currentmarker}{}%
\end{pgfscope}%
\end{pgfscope}%
\begin{pgfscope}%
\pgfpathrectangle{\pgfqpoint{0.765000in}{0.660000in}}{\pgfqpoint{4.620000in}{4.620000in}}%
\pgfusepath{clip}%
\pgfsetrectcap%
\pgfsetroundjoin%
\pgfsetlinewidth{1.204500pt}%
\definecolor{currentstroke}{rgb}{0.176471,0.192157,0.258824}%
\pgfsetstrokecolor{currentstroke}%
\pgfsetdash{}{0pt}%
\pgfpathmoveto{\pgfqpoint{2.376513in}{3.241847in}}%
\pgfusepath{stroke}%
\end{pgfscope}%
\begin{pgfscope}%
\pgfpathrectangle{\pgfqpoint{0.765000in}{0.660000in}}{\pgfqpoint{4.620000in}{4.620000in}}%
\pgfusepath{clip}%
\pgfsetbuttcap%
\pgfsetmiterjoin%
\definecolor{currentfill}{rgb}{0.176471,0.192157,0.258824}%
\pgfsetfillcolor{currentfill}%
\pgfsetlinewidth{1.003750pt}%
\definecolor{currentstroke}{rgb}{0.176471,0.192157,0.258824}%
\pgfsetstrokecolor{currentstroke}%
\pgfsetdash{}{0pt}%
\pgfsys@defobject{currentmarker}{\pgfqpoint{-0.033333in}{-0.033333in}}{\pgfqpoint{0.033333in}{0.033333in}}{%
\pgfpathmoveto{\pgfqpoint{-0.000000in}{-0.033333in}}%
\pgfpathlineto{\pgfqpoint{0.033333in}{0.033333in}}%
\pgfpathlineto{\pgfqpoint{-0.033333in}{0.033333in}}%
\pgfpathlineto{\pgfqpoint{-0.000000in}{-0.033333in}}%
\pgfpathclose%
\pgfusepath{stroke,fill}%
}%
\begin{pgfscope}%
\pgfsys@transformshift{2.376513in}{3.241847in}%
\pgfsys@useobject{currentmarker}{}%
\end{pgfscope}%
\end{pgfscope}%
\begin{pgfscope}%
\pgfpathrectangle{\pgfqpoint{0.765000in}{0.660000in}}{\pgfqpoint{4.620000in}{4.620000in}}%
\pgfusepath{clip}%
\pgfsetrectcap%
\pgfsetroundjoin%
\pgfsetlinewidth{1.204500pt}%
\definecolor{currentstroke}{rgb}{0.176471,0.192157,0.258824}%
\pgfsetstrokecolor{currentstroke}%
\pgfsetdash{}{0pt}%
\pgfpathmoveto{\pgfqpoint{3.598776in}{3.501804in}}%
\pgfusepath{stroke}%
\end{pgfscope}%
\begin{pgfscope}%
\pgfpathrectangle{\pgfqpoint{0.765000in}{0.660000in}}{\pgfqpoint{4.620000in}{4.620000in}}%
\pgfusepath{clip}%
\pgfsetbuttcap%
\pgfsetmiterjoin%
\definecolor{currentfill}{rgb}{0.176471,0.192157,0.258824}%
\pgfsetfillcolor{currentfill}%
\pgfsetlinewidth{1.003750pt}%
\definecolor{currentstroke}{rgb}{0.176471,0.192157,0.258824}%
\pgfsetstrokecolor{currentstroke}%
\pgfsetdash{}{0pt}%
\pgfsys@defobject{currentmarker}{\pgfqpoint{-0.033333in}{-0.033333in}}{\pgfqpoint{0.033333in}{0.033333in}}{%
\pgfpathmoveto{\pgfqpoint{-0.000000in}{-0.033333in}}%
\pgfpathlineto{\pgfqpoint{0.033333in}{0.033333in}}%
\pgfpathlineto{\pgfqpoint{-0.033333in}{0.033333in}}%
\pgfpathlineto{\pgfqpoint{-0.000000in}{-0.033333in}}%
\pgfpathclose%
\pgfusepath{stroke,fill}%
}%
\begin{pgfscope}%
\pgfsys@transformshift{3.598776in}{3.501804in}%
\pgfsys@useobject{currentmarker}{}%
\end{pgfscope}%
\end{pgfscope}%
\begin{pgfscope}%
\pgfpathrectangle{\pgfqpoint{0.765000in}{0.660000in}}{\pgfqpoint{4.620000in}{4.620000in}}%
\pgfusepath{clip}%
\pgfsetrectcap%
\pgfsetroundjoin%
\pgfsetlinewidth{1.204500pt}%
\definecolor{currentstroke}{rgb}{0.176471,0.192157,0.258824}%
\pgfsetstrokecolor{currentstroke}%
\pgfsetdash{}{0pt}%
\pgfpathmoveto{\pgfqpoint{2.083911in}{3.716298in}}%
\pgfusepath{stroke}%
\end{pgfscope}%
\begin{pgfscope}%
\pgfpathrectangle{\pgfqpoint{0.765000in}{0.660000in}}{\pgfqpoint{4.620000in}{4.620000in}}%
\pgfusepath{clip}%
\pgfsetbuttcap%
\pgfsetmiterjoin%
\definecolor{currentfill}{rgb}{0.176471,0.192157,0.258824}%
\pgfsetfillcolor{currentfill}%
\pgfsetlinewidth{1.003750pt}%
\definecolor{currentstroke}{rgb}{0.176471,0.192157,0.258824}%
\pgfsetstrokecolor{currentstroke}%
\pgfsetdash{}{0pt}%
\pgfsys@defobject{currentmarker}{\pgfqpoint{-0.033333in}{-0.033333in}}{\pgfqpoint{0.033333in}{0.033333in}}{%
\pgfpathmoveto{\pgfqpoint{-0.000000in}{-0.033333in}}%
\pgfpathlineto{\pgfqpoint{0.033333in}{0.033333in}}%
\pgfpathlineto{\pgfqpoint{-0.033333in}{0.033333in}}%
\pgfpathlineto{\pgfqpoint{-0.000000in}{-0.033333in}}%
\pgfpathclose%
\pgfusepath{stroke,fill}%
}%
\begin{pgfscope}%
\pgfsys@transformshift{2.083911in}{3.716298in}%
\pgfsys@useobject{currentmarker}{}%
\end{pgfscope}%
\end{pgfscope}%
\begin{pgfscope}%
\pgfpathrectangle{\pgfqpoint{0.765000in}{0.660000in}}{\pgfqpoint{4.620000in}{4.620000in}}%
\pgfusepath{clip}%
\pgfsetrectcap%
\pgfsetroundjoin%
\pgfsetlinewidth{1.204500pt}%
\definecolor{currentstroke}{rgb}{0.176471,0.192157,0.258824}%
\pgfsetstrokecolor{currentstroke}%
\pgfsetdash{}{0pt}%
\pgfpathmoveto{\pgfqpoint{2.034318in}{3.482362in}}%
\pgfusepath{stroke}%
\end{pgfscope}%
\begin{pgfscope}%
\pgfpathrectangle{\pgfqpoint{0.765000in}{0.660000in}}{\pgfqpoint{4.620000in}{4.620000in}}%
\pgfusepath{clip}%
\pgfsetbuttcap%
\pgfsetmiterjoin%
\definecolor{currentfill}{rgb}{0.176471,0.192157,0.258824}%
\pgfsetfillcolor{currentfill}%
\pgfsetlinewidth{1.003750pt}%
\definecolor{currentstroke}{rgb}{0.176471,0.192157,0.258824}%
\pgfsetstrokecolor{currentstroke}%
\pgfsetdash{}{0pt}%
\pgfsys@defobject{currentmarker}{\pgfqpoint{-0.033333in}{-0.033333in}}{\pgfqpoint{0.033333in}{0.033333in}}{%
\pgfpathmoveto{\pgfqpoint{-0.000000in}{-0.033333in}}%
\pgfpathlineto{\pgfqpoint{0.033333in}{0.033333in}}%
\pgfpathlineto{\pgfqpoint{-0.033333in}{0.033333in}}%
\pgfpathlineto{\pgfqpoint{-0.000000in}{-0.033333in}}%
\pgfpathclose%
\pgfusepath{stroke,fill}%
}%
\begin{pgfscope}%
\pgfsys@transformshift{2.034318in}{3.482362in}%
\pgfsys@useobject{currentmarker}{}%
\end{pgfscope}%
\end{pgfscope}%
\begin{pgfscope}%
\pgfpathrectangle{\pgfqpoint{0.765000in}{0.660000in}}{\pgfqpoint{4.620000in}{4.620000in}}%
\pgfusepath{clip}%
\pgfsetrectcap%
\pgfsetroundjoin%
\pgfsetlinewidth{1.204500pt}%
\definecolor{currentstroke}{rgb}{0.176471,0.192157,0.258824}%
\pgfsetstrokecolor{currentstroke}%
\pgfsetdash{}{0pt}%
\pgfpathmoveto{\pgfqpoint{2.888262in}{3.592853in}}%
\pgfusepath{stroke}%
\end{pgfscope}%
\begin{pgfscope}%
\pgfpathrectangle{\pgfqpoint{0.765000in}{0.660000in}}{\pgfqpoint{4.620000in}{4.620000in}}%
\pgfusepath{clip}%
\pgfsetbuttcap%
\pgfsetmiterjoin%
\definecolor{currentfill}{rgb}{0.176471,0.192157,0.258824}%
\pgfsetfillcolor{currentfill}%
\pgfsetlinewidth{1.003750pt}%
\definecolor{currentstroke}{rgb}{0.176471,0.192157,0.258824}%
\pgfsetstrokecolor{currentstroke}%
\pgfsetdash{}{0pt}%
\pgfsys@defobject{currentmarker}{\pgfqpoint{-0.033333in}{-0.033333in}}{\pgfqpoint{0.033333in}{0.033333in}}{%
\pgfpathmoveto{\pgfqpoint{-0.000000in}{-0.033333in}}%
\pgfpathlineto{\pgfqpoint{0.033333in}{0.033333in}}%
\pgfpathlineto{\pgfqpoint{-0.033333in}{0.033333in}}%
\pgfpathlineto{\pgfqpoint{-0.000000in}{-0.033333in}}%
\pgfpathclose%
\pgfusepath{stroke,fill}%
}%
\begin{pgfscope}%
\pgfsys@transformshift{2.888262in}{3.592853in}%
\pgfsys@useobject{currentmarker}{}%
\end{pgfscope}%
\end{pgfscope}%
\begin{pgfscope}%
\pgfpathrectangle{\pgfqpoint{0.765000in}{0.660000in}}{\pgfqpoint{4.620000in}{4.620000in}}%
\pgfusepath{clip}%
\pgfsetrectcap%
\pgfsetroundjoin%
\pgfsetlinewidth{1.204500pt}%
\definecolor{currentstroke}{rgb}{0.176471,0.192157,0.258824}%
\pgfsetstrokecolor{currentstroke}%
\pgfsetdash{}{0pt}%
\pgfpathmoveto{\pgfqpoint{2.213511in}{3.603999in}}%
\pgfusepath{stroke}%
\end{pgfscope}%
\begin{pgfscope}%
\pgfpathrectangle{\pgfqpoint{0.765000in}{0.660000in}}{\pgfqpoint{4.620000in}{4.620000in}}%
\pgfusepath{clip}%
\pgfsetbuttcap%
\pgfsetmiterjoin%
\definecolor{currentfill}{rgb}{0.176471,0.192157,0.258824}%
\pgfsetfillcolor{currentfill}%
\pgfsetlinewidth{1.003750pt}%
\definecolor{currentstroke}{rgb}{0.176471,0.192157,0.258824}%
\pgfsetstrokecolor{currentstroke}%
\pgfsetdash{}{0pt}%
\pgfsys@defobject{currentmarker}{\pgfqpoint{-0.033333in}{-0.033333in}}{\pgfqpoint{0.033333in}{0.033333in}}{%
\pgfpathmoveto{\pgfqpoint{-0.000000in}{-0.033333in}}%
\pgfpathlineto{\pgfqpoint{0.033333in}{0.033333in}}%
\pgfpathlineto{\pgfqpoint{-0.033333in}{0.033333in}}%
\pgfpathlineto{\pgfqpoint{-0.000000in}{-0.033333in}}%
\pgfpathclose%
\pgfusepath{stroke,fill}%
}%
\begin{pgfscope}%
\pgfsys@transformshift{2.213511in}{3.603999in}%
\pgfsys@useobject{currentmarker}{}%
\end{pgfscope}%
\end{pgfscope}%
\begin{pgfscope}%
\pgfpathrectangle{\pgfqpoint{0.765000in}{0.660000in}}{\pgfqpoint{4.620000in}{4.620000in}}%
\pgfusepath{clip}%
\pgfsetrectcap%
\pgfsetroundjoin%
\pgfsetlinewidth{1.204500pt}%
\definecolor{currentstroke}{rgb}{0.176471,0.192157,0.258824}%
\pgfsetstrokecolor{currentstroke}%
\pgfsetdash{}{0pt}%
\pgfpathmoveto{\pgfqpoint{1.862606in}{3.364769in}}%
\pgfusepath{stroke}%
\end{pgfscope}%
\begin{pgfscope}%
\pgfpathrectangle{\pgfqpoint{0.765000in}{0.660000in}}{\pgfqpoint{4.620000in}{4.620000in}}%
\pgfusepath{clip}%
\pgfsetbuttcap%
\pgfsetmiterjoin%
\definecolor{currentfill}{rgb}{0.176471,0.192157,0.258824}%
\pgfsetfillcolor{currentfill}%
\pgfsetlinewidth{1.003750pt}%
\definecolor{currentstroke}{rgb}{0.176471,0.192157,0.258824}%
\pgfsetstrokecolor{currentstroke}%
\pgfsetdash{}{0pt}%
\pgfsys@defobject{currentmarker}{\pgfqpoint{-0.033333in}{-0.033333in}}{\pgfqpoint{0.033333in}{0.033333in}}{%
\pgfpathmoveto{\pgfqpoint{-0.000000in}{-0.033333in}}%
\pgfpathlineto{\pgfqpoint{0.033333in}{0.033333in}}%
\pgfpathlineto{\pgfqpoint{-0.033333in}{0.033333in}}%
\pgfpathlineto{\pgfqpoint{-0.000000in}{-0.033333in}}%
\pgfpathclose%
\pgfusepath{stroke,fill}%
}%
\begin{pgfscope}%
\pgfsys@transformshift{1.862606in}{3.364769in}%
\pgfsys@useobject{currentmarker}{}%
\end{pgfscope}%
\end{pgfscope}%
\begin{pgfscope}%
\pgfpathrectangle{\pgfqpoint{0.765000in}{0.660000in}}{\pgfqpoint{4.620000in}{4.620000in}}%
\pgfusepath{clip}%
\pgfsetrectcap%
\pgfsetroundjoin%
\pgfsetlinewidth{1.204500pt}%
\definecolor{currentstroke}{rgb}{0.176471,0.192157,0.258824}%
\pgfsetstrokecolor{currentstroke}%
\pgfsetdash{}{0pt}%
\pgfpathmoveto{\pgfqpoint{3.711268in}{3.488344in}}%
\pgfusepath{stroke}%
\end{pgfscope}%
\begin{pgfscope}%
\pgfpathrectangle{\pgfqpoint{0.765000in}{0.660000in}}{\pgfqpoint{4.620000in}{4.620000in}}%
\pgfusepath{clip}%
\pgfsetbuttcap%
\pgfsetmiterjoin%
\definecolor{currentfill}{rgb}{0.176471,0.192157,0.258824}%
\pgfsetfillcolor{currentfill}%
\pgfsetlinewidth{1.003750pt}%
\definecolor{currentstroke}{rgb}{0.176471,0.192157,0.258824}%
\pgfsetstrokecolor{currentstroke}%
\pgfsetdash{}{0pt}%
\pgfsys@defobject{currentmarker}{\pgfqpoint{-0.033333in}{-0.033333in}}{\pgfqpoint{0.033333in}{0.033333in}}{%
\pgfpathmoveto{\pgfqpoint{-0.000000in}{-0.033333in}}%
\pgfpathlineto{\pgfqpoint{0.033333in}{0.033333in}}%
\pgfpathlineto{\pgfqpoint{-0.033333in}{0.033333in}}%
\pgfpathlineto{\pgfqpoint{-0.000000in}{-0.033333in}}%
\pgfpathclose%
\pgfusepath{stroke,fill}%
}%
\begin{pgfscope}%
\pgfsys@transformshift{3.711268in}{3.488344in}%
\pgfsys@useobject{currentmarker}{}%
\end{pgfscope}%
\end{pgfscope}%
\begin{pgfscope}%
\pgfpathrectangle{\pgfqpoint{0.765000in}{0.660000in}}{\pgfqpoint{4.620000in}{4.620000in}}%
\pgfusepath{clip}%
\pgfsetrectcap%
\pgfsetroundjoin%
\pgfsetlinewidth{1.204500pt}%
\definecolor{currentstroke}{rgb}{0.176471,0.192157,0.258824}%
\pgfsetstrokecolor{currentstroke}%
\pgfsetdash{}{0pt}%
\pgfpathmoveto{\pgfqpoint{3.393700in}{2.984977in}}%
\pgfusepath{stroke}%
\end{pgfscope}%
\begin{pgfscope}%
\pgfpathrectangle{\pgfqpoint{0.765000in}{0.660000in}}{\pgfqpoint{4.620000in}{4.620000in}}%
\pgfusepath{clip}%
\pgfsetbuttcap%
\pgfsetmiterjoin%
\definecolor{currentfill}{rgb}{0.176471,0.192157,0.258824}%
\pgfsetfillcolor{currentfill}%
\pgfsetlinewidth{1.003750pt}%
\definecolor{currentstroke}{rgb}{0.176471,0.192157,0.258824}%
\pgfsetstrokecolor{currentstroke}%
\pgfsetdash{}{0pt}%
\pgfsys@defobject{currentmarker}{\pgfqpoint{-0.033333in}{-0.033333in}}{\pgfqpoint{0.033333in}{0.033333in}}{%
\pgfpathmoveto{\pgfqpoint{-0.000000in}{-0.033333in}}%
\pgfpathlineto{\pgfqpoint{0.033333in}{0.033333in}}%
\pgfpathlineto{\pgfqpoint{-0.033333in}{0.033333in}}%
\pgfpathlineto{\pgfqpoint{-0.000000in}{-0.033333in}}%
\pgfpathclose%
\pgfusepath{stroke,fill}%
}%
\begin{pgfscope}%
\pgfsys@transformshift{3.393700in}{2.984977in}%
\pgfsys@useobject{currentmarker}{}%
\end{pgfscope}%
\end{pgfscope}%
\begin{pgfscope}%
\pgfpathrectangle{\pgfqpoint{0.765000in}{0.660000in}}{\pgfqpoint{4.620000in}{4.620000in}}%
\pgfusepath{clip}%
\pgfsetrectcap%
\pgfsetroundjoin%
\pgfsetlinewidth{1.204500pt}%
\definecolor{currentstroke}{rgb}{0.176471,0.192157,0.258824}%
\pgfsetstrokecolor{currentstroke}%
\pgfsetdash{}{0pt}%
\pgfpathmoveto{\pgfqpoint{3.388538in}{3.505737in}}%
\pgfusepath{stroke}%
\end{pgfscope}%
\begin{pgfscope}%
\pgfpathrectangle{\pgfqpoint{0.765000in}{0.660000in}}{\pgfqpoint{4.620000in}{4.620000in}}%
\pgfusepath{clip}%
\pgfsetbuttcap%
\pgfsetmiterjoin%
\definecolor{currentfill}{rgb}{0.176471,0.192157,0.258824}%
\pgfsetfillcolor{currentfill}%
\pgfsetlinewidth{1.003750pt}%
\definecolor{currentstroke}{rgb}{0.176471,0.192157,0.258824}%
\pgfsetstrokecolor{currentstroke}%
\pgfsetdash{}{0pt}%
\pgfsys@defobject{currentmarker}{\pgfqpoint{-0.033333in}{-0.033333in}}{\pgfqpoint{0.033333in}{0.033333in}}{%
\pgfpathmoveto{\pgfqpoint{-0.000000in}{-0.033333in}}%
\pgfpathlineto{\pgfqpoint{0.033333in}{0.033333in}}%
\pgfpathlineto{\pgfqpoint{-0.033333in}{0.033333in}}%
\pgfpathlineto{\pgfqpoint{-0.000000in}{-0.033333in}}%
\pgfpathclose%
\pgfusepath{stroke,fill}%
}%
\begin{pgfscope}%
\pgfsys@transformshift{3.388538in}{3.505737in}%
\pgfsys@useobject{currentmarker}{}%
\end{pgfscope}%
\end{pgfscope}%
\begin{pgfscope}%
\pgfpathrectangle{\pgfqpoint{0.765000in}{0.660000in}}{\pgfqpoint{4.620000in}{4.620000in}}%
\pgfusepath{clip}%
\pgfsetrectcap%
\pgfsetroundjoin%
\pgfsetlinewidth{1.204500pt}%
\definecolor{currentstroke}{rgb}{0.176471,0.192157,0.258824}%
\pgfsetstrokecolor{currentstroke}%
\pgfsetdash{}{0pt}%
\pgfpathmoveto{\pgfqpoint{2.176875in}{3.446714in}}%
\pgfusepath{stroke}%
\end{pgfscope}%
\begin{pgfscope}%
\pgfpathrectangle{\pgfqpoint{0.765000in}{0.660000in}}{\pgfqpoint{4.620000in}{4.620000in}}%
\pgfusepath{clip}%
\pgfsetbuttcap%
\pgfsetmiterjoin%
\definecolor{currentfill}{rgb}{0.176471,0.192157,0.258824}%
\pgfsetfillcolor{currentfill}%
\pgfsetlinewidth{1.003750pt}%
\definecolor{currentstroke}{rgb}{0.176471,0.192157,0.258824}%
\pgfsetstrokecolor{currentstroke}%
\pgfsetdash{}{0pt}%
\pgfsys@defobject{currentmarker}{\pgfqpoint{-0.033333in}{-0.033333in}}{\pgfqpoint{0.033333in}{0.033333in}}{%
\pgfpathmoveto{\pgfqpoint{-0.000000in}{-0.033333in}}%
\pgfpathlineto{\pgfqpoint{0.033333in}{0.033333in}}%
\pgfpathlineto{\pgfqpoint{-0.033333in}{0.033333in}}%
\pgfpathlineto{\pgfqpoint{-0.000000in}{-0.033333in}}%
\pgfpathclose%
\pgfusepath{stroke,fill}%
}%
\begin{pgfscope}%
\pgfsys@transformshift{2.176875in}{3.446714in}%
\pgfsys@useobject{currentmarker}{}%
\end{pgfscope}%
\end{pgfscope}%
\begin{pgfscope}%
\pgfpathrectangle{\pgfqpoint{0.765000in}{0.660000in}}{\pgfqpoint{4.620000in}{4.620000in}}%
\pgfusepath{clip}%
\pgfsetrectcap%
\pgfsetroundjoin%
\pgfsetlinewidth{1.204500pt}%
\definecolor{currentstroke}{rgb}{0.176471,0.192157,0.258824}%
\pgfsetstrokecolor{currentstroke}%
\pgfsetdash{}{0pt}%
\pgfpathmoveto{\pgfqpoint{2.887136in}{3.564853in}}%
\pgfusepath{stroke}%
\end{pgfscope}%
\begin{pgfscope}%
\pgfpathrectangle{\pgfqpoint{0.765000in}{0.660000in}}{\pgfqpoint{4.620000in}{4.620000in}}%
\pgfusepath{clip}%
\pgfsetbuttcap%
\pgfsetmiterjoin%
\definecolor{currentfill}{rgb}{0.176471,0.192157,0.258824}%
\pgfsetfillcolor{currentfill}%
\pgfsetlinewidth{1.003750pt}%
\definecolor{currentstroke}{rgb}{0.176471,0.192157,0.258824}%
\pgfsetstrokecolor{currentstroke}%
\pgfsetdash{}{0pt}%
\pgfsys@defobject{currentmarker}{\pgfqpoint{-0.033333in}{-0.033333in}}{\pgfqpoint{0.033333in}{0.033333in}}{%
\pgfpathmoveto{\pgfqpoint{-0.000000in}{-0.033333in}}%
\pgfpathlineto{\pgfqpoint{0.033333in}{0.033333in}}%
\pgfpathlineto{\pgfqpoint{-0.033333in}{0.033333in}}%
\pgfpathlineto{\pgfqpoint{-0.000000in}{-0.033333in}}%
\pgfpathclose%
\pgfusepath{stroke,fill}%
}%
\begin{pgfscope}%
\pgfsys@transformshift{2.887136in}{3.564853in}%
\pgfsys@useobject{currentmarker}{}%
\end{pgfscope}%
\end{pgfscope}%
\begin{pgfscope}%
\pgfpathrectangle{\pgfqpoint{0.765000in}{0.660000in}}{\pgfqpoint{4.620000in}{4.620000in}}%
\pgfusepath{clip}%
\pgfsetrectcap%
\pgfsetroundjoin%
\pgfsetlinewidth{1.204500pt}%
\definecolor{currentstroke}{rgb}{0.176471,0.192157,0.258824}%
\pgfsetstrokecolor{currentstroke}%
\pgfsetdash{}{0pt}%
\pgfpathmoveto{\pgfqpoint{1.026858in}{2.886571in}}%
\pgfusepath{stroke}%
\end{pgfscope}%
\begin{pgfscope}%
\pgfpathrectangle{\pgfqpoint{0.765000in}{0.660000in}}{\pgfqpoint{4.620000in}{4.620000in}}%
\pgfusepath{clip}%
\pgfsetbuttcap%
\pgfsetmiterjoin%
\definecolor{currentfill}{rgb}{0.176471,0.192157,0.258824}%
\pgfsetfillcolor{currentfill}%
\pgfsetlinewidth{1.003750pt}%
\definecolor{currentstroke}{rgb}{0.176471,0.192157,0.258824}%
\pgfsetstrokecolor{currentstroke}%
\pgfsetdash{}{0pt}%
\pgfsys@defobject{currentmarker}{\pgfqpoint{-0.033333in}{-0.033333in}}{\pgfqpoint{0.033333in}{0.033333in}}{%
\pgfpathmoveto{\pgfqpoint{-0.000000in}{-0.033333in}}%
\pgfpathlineto{\pgfqpoint{0.033333in}{0.033333in}}%
\pgfpathlineto{\pgfqpoint{-0.033333in}{0.033333in}}%
\pgfpathlineto{\pgfqpoint{-0.000000in}{-0.033333in}}%
\pgfpathclose%
\pgfusepath{stroke,fill}%
}%
\begin{pgfscope}%
\pgfsys@transformshift{1.026858in}{2.886571in}%
\pgfsys@useobject{currentmarker}{}%
\end{pgfscope}%
\end{pgfscope}%
\begin{pgfscope}%
\pgfpathrectangle{\pgfqpoint{0.765000in}{0.660000in}}{\pgfqpoint{4.620000in}{4.620000in}}%
\pgfusepath{clip}%
\pgfsetrectcap%
\pgfsetroundjoin%
\pgfsetlinewidth{1.204500pt}%
\definecolor{currentstroke}{rgb}{0.176471,0.192157,0.258824}%
\pgfsetstrokecolor{currentstroke}%
\pgfsetdash{}{0pt}%
\pgfpathmoveto{\pgfqpoint{2.044090in}{3.571430in}}%
\pgfusepath{stroke}%
\end{pgfscope}%
\begin{pgfscope}%
\pgfpathrectangle{\pgfqpoint{0.765000in}{0.660000in}}{\pgfqpoint{4.620000in}{4.620000in}}%
\pgfusepath{clip}%
\pgfsetbuttcap%
\pgfsetmiterjoin%
\definecolor{currentfill}{rgb}{0.176471,0.192157,0.258824}%
\pgfsetfillcolor{currentfill}%
\pgfsetlinewidth{1.003750pt}%
\definecolor{currentstroke}{rgb}{0.176471,0.192157,0.258824}%
\pgfsetstrokecolor{currentstroke}%
\pgfsetdash{}{0pt}%
\pgfsys@defobject{currentmarker}{\pgfqpoint{-0.033333in}{-0.033333in}}{\pgfqpoint{0.033333in}{0.033333in}}{%
\pgfpathmoveto{\pgfqpoint{-0.000000in}{-0.033333in}}%
\pgfpathlineto{\pgfqpoint{0.033333in}{0.033333in}}%
\pgfpathlineto{\pgfqpoint{-0.033333in}{0.033333in}}%
\pgfpathlineto{\pgfqpoint{-0.000000in}{-0.033333in}}%
\pgfpathclose%
\pgfusepath{stroke,fill}%
}%
\begin{pgfscope}%
\pgfsys@transformshift{2.044090in}{3.571430in}%
\pgfsys@useobject{currentmarker}{}%
\end{pgfscope}%
\end{pgfscope}%
\begin{pgfscope}%
\pgfpathrectangle{\pgfqpoint{0.765000in}{0.660000in}}{\pgfqpoint{4.620000in}{4.620000in}}%
\pgfusepath{clip}%
\pgfsetrectcap%
\pgfsetroundjoin%
\pgfsetlinewidth{1.204500pt}%
\definecolor{currentstroke}{rgb}{0.176471,0.192157,0.258824}%
\pgfsetstrokecolor{currentstroke}%
\pgfsetdash{}{0pt}%
\pgfpathmoveto{\pgfqpoint{1.922885in}{3.304537in}}%
\pgfusepath{stroke}%
\end{pgfscope}%
\begin{pgfscope}%
\pgfpathrectangle{\pgfqpoint{0.765000in}{0.660000in}}{\pgfqpoint{4.620000in}{4.620000in}}%
\pgfusepath{clip}%
\pgfsetbuttcap%
\pgfsetmiterjoin%
\definecolor{currentfill}{rgb}{0.176471,0.192157,0.258824}%
\pgfsetfillcolor{currentfill}%
\pgfsetlinewidth{1.003750pt}%
\definecolor{currentstroke}{rgb}{0.176471,0.192157,0.258824}%
\pgfsetstrokecolor{currentstroke}%
\pgfsetdash{}{0pt}%
\pgfsys@defobject{currentmarker}{\pgfqpoint{-0.033333in}{-0.033333in}}{\pgfqpoint{0.033333in}{0.033333in}}{%
\pgfpathmoveto{\pgfqpoint{-0.000000in}{-0.033333in}}%
\pgfpathlineto{\pgfqpoint{0.033333in}{0.033333in}}%
\pgfpathlineto{\pgfqpoint{-0.033333in}{0.033333in}}%
\pgfpathlineto{\pgfqpoint{-0.000000in}{-0.033333in}}%
\pgfpathclose%
\pgfusepath{stroke,fill}%
}%
\begin{pgfscope}%
\pgfsys@transformshift{1.922885in}{3.304537in}%
\pgfsys@useobject{currentmarker}{}%
\end{pgfscope}%
\end{pgfscope}%
\begin{pgfscope}%
\pgfpathrectangle{\pgfqpoint{0.765000in}{0.660000in}}{\pgfqpoint{4.620000in}{4.620000in}}%
\pgfusepath{clip}%
\pgfsetrectcap%
\pgfsetroundjoin%
\pgfsetlinewidth{1.204500pt}%
\definecolor{currentstroke}{rgb}{0.176471,0.192157,0.258824}%
\pgfsetstrokecolor{currentstroke}%
\pgfsetdash{}{0pt}%
\pgfpathmoveto{\pgfqpoint{1.310540in}{2.853537in}}%
\pgfusepath{stroke}%
\end{pgfscope}%
\begin{pgfscope}%
\pgfpathrectangle{\pgfqpoint{0.765000in}{0.660000in}}{\pgfqpoint{4.620000in}{4.620000in}}%
\pgfusepath{clip}%
\pgfsetbuttcap%
\pgfsetmiterjoin%
\definecolor{currentfill}{rgb}{0.176471,0.192157,0.258824}%
\pgfsetfillcolor{currentfill}%
\pgfsetlinewidth{1.003750pt}%
\definecolor{currentstroke}{rgb}{0.176471,0.192157,0.258824}%
\pgfsetstrokecolor{currentstroke}%
\pgfsetdash{}{0pt}%
\pgfsys@defobject{currentmarker}{\pgfqpoint{-0.033333in}{-0.033333in}}{\pgfqpoint{0.033333in}{0.033333in}}{%
\pgfpathmoveto{\pgfqpoint{-0.000000in}{-0.033333in}}%
\pgfpathlineto{\pgfqpoint{0.033333in}{0.033333in}}%
\pgfpathlineto{\pgfqpoint{-0.033333in}{0.033333in}}%
\pgfpathlineto{\pgfqpoint{-0.000000in}{-0.033333in}}%
\pgfpathclose%
\pgfusepath{stroke,fill}%
}%
\begin{pgfscope}%
\pgfsys@transformshift{1.310540in}{2.853537in}%
\pgfsys@useobject{currentmarker}{}%
\end{pgfscope}%
\end{pgfscope}%
\begin{pgfscope}%
\pgfpathrectangle{\pgfqpoint{0.765000in}{0.660000in}}{\pgfqpoint{4.620000in}{4.620000in}}%
\pgfusepath{clip}%
\pgfsetrectcap%
\pgfsetroundjoin%
\pgfsetlinewidth{1.204500pt}%
\definecolor{currentstroke}{rgb}{0.176471,0.192157,0.258824}%
\pgfsetstrokecolor{currentstroke}%
\pgfsetdash{}{0pt}%
\pgfpathmoveto{\pgfqpoint{3.116587in}{3.684682in}}%
\pgfusepath{stroke}%
\end{pgfscope}%
\begin{pgfscope}%
\pgfpathrectangle{\pgfqpoint{0.765000in}{0.660000in}}{\pgfqpoint{4.620000in}{4.620000in}}%
\pgfusepath{clip}%
\pgfsetbuttcap%
\pgfsetmiterjoin%
\definecolor{currentfill}{rgb}{0.176471,0.192157,0.258824}%
\pgfsetfillcolor{currentfill}%
\pgfsetlinewidth{1.003750pt}%
\definecolor{currentstroke}{rgb}{0.176471,0.192157,0.258824}%
\pgfsetstrokecolor{currentstroke}%
\pgfsetdash{}{0pt}%
\pgfsys@defobject{currentmarker}{\pgfqpoint{-0.033333in}{-0.033333in}}{\pgfqpoint{0.033333in}{0.033333in}}{%
\pgfpathmoveto{\pgfqpoint{-0.000000in}{-0.033333in}}%
\pgfpathlineto{\pgfqpoint{0.033333in}{0.033333in}}%
\pgfpathlineto{\pgfqpoint{-0.033333in}{0.033333in}}%
\pgfpathlineto{\pgfqpoint{-0.000000in}{-0.033333in}}%
\pgfpathclose%
\pgfusepath{stroke,fill}%
}%
\begin{pgfscope}%
\pgfsys@transformshift{3.116587in}{3.684682in}%
\pgfsys@useobject{currentmarker}{}%
\end{pgfscope}%
\end{pgfscope}%
\begin{pgfscope}%
\pgfpathrectangle{\pgfqpoint{0.765000in}{0.660000in}}{\pgfqpoint{4.620000in}{4.620000in}}%
\pgfusepath{clip}%
\pgfsetrectcap%
\pgfsetroundjoin%
\pgfsetlinewidth{1.204500pt}%
\definecolor{currentstroke}{rgb}{0.176471,0.192157,0.258824}%
\pgfsetstrokecolor{currentstroke}%
\pgfsetdash{}{0pt}%
\pgfpathmoveto{\pgfqpoint{2.310226in}{3.734334in}}%
\pgfusepath{stroke}%
\end{pgfscope}%
\begin{pgfscope}%
\pgfpathrectangle{\pgfqpoint{0.765000in}{0.660000in}}{\pgfqpoint{4.620000in}{4.620000in}}%
\pgfusepath{clip}%
\pgfsetbuttcap%
\pgfsetmiterjoin%
\definecolor{currentfill}{rgb}{0.176471,0.192157,0.258824}%
\pgfsetfillcolor{currentfill}%
\pgfsetlinewidth{1.003750pt}%
\definecolor{currentstroke}{rgb}{0.176471,0.192157,0.258824}%
\pgfsetstrokecolor{currentstroke}%
\pgfsetdash{}{0pt}%
\pgfsys@defobject{currentmarker}{\pgfqpoint{-0.033333in}{-0.033333in}}{\pgfqpoint{0.033333in}{0.033333in}}{%
\pgfpathmoveto{\pgfqpoint{-0.000000in}{-0.033333in}}%
\pgfpathlineto{\pgfqpoint{0.033333in}{0.033333in}}%
\pgfpathlineto{\pgfqpoint{-0.033333in}{0.033333in}}%
\pgfpathlineto{\pgfqpoint{-0.000000in}{-0.033333in}}%
\pgfpathclose%
\pgfusepath{stroke,fill}%
}%
\begin{pgfscope}%
\pgfsys@transformshift{2.310226in}{3.734334in}%
\pgfsys@useobject{currentmarker}{}%
\end{pgfscope}%
\end{pgfscope}%
\begin{pgfscope}%
\pgfpathrectangle{\pgfqpoint{0.765000in}{0.660000in}}{\pgfqpoint{4.620000in}{4.620000in}}%
\pgfusepath{clip}%
\pgfsetrectcap%
\pgfsetroundjoin%
\pgfsetlinewidth{1.204500pt}%
\definecolor{currentstroke}{rgb}{0.176471,0.192157,0.258824}%
\pgfsetstrokecolor{currentstroke}%
\pgfsetdash{}{0pt}%
\pgfpathmoveto{\pgfqpoint{3.799586in}{3.182808in}}%
\pgfusepath{stroke}%
\end{pgfscope}%
\begin{pgfscope}%
\pgfpathrectangle{\pgfqpoint{0.765000in}{0.660000in}}{\pgfqpoint{4.620000in}{4.620000in}}%
\pgfusepath{clip}%
\pgfsetbuttcap%
\pgfsetmiterjoin%
\definecolor{currentfill}{rgb}{0.176471,0.192157,0.258824}%
\pgfsetfillcolor{currentfill}%
\pgfsetlinewidth{1.003750pt}%
\definecolor{currentstroke}{rgb}{0.176471,0.192157,0.258824}%
\pgfsetstrokecolor{currentstroke}%
\pgfsetdash{}{0pt}%
\pgfsys@defobject{currentmarker}{\pgfqpoint{-0.033333in}{-0.033333in}}{\pgfqpoint{0.033333in}{0.033333in}}{%
\pgfpathmoveto{\pgfqpoint{-0.000000in}{-0.033333in}}%
\pgfpathlineto{\pgfqpoint{0.033333in}{0.033333in}}%
\pgfpathlineto{\pgfqpoint{-0.033333in}{0.033333in}}%
\pgfpathlineto{\pgfqpoint{-0.000000in}{-0.033333in}}%
\pgfpathclose%
\pgfusepath{stroke,fill}%
}%
\begin{pgfscope}%
\pgfsys@transformshift{3.799586in}{3.182808in}%
\pgfsys@useobject{currentmarker}{}%
\end{pgfscope}%
\end{pgfscope}%
\begin{pgfscope}%
\pgfpathrectangle{\pgfqpoint{0.765000in}{0.660000in}}{\pgfqpoint{4.620000in}{4.620000in}}%
\pgfusepath{clip}%
\pgfsetrectcap%
\pgfsetroundjoin%
\pgfsetlinewidth{1.204500pt}%
\definecolor{currentstroke}{rgb}{0.176471,0.192157,0.258824}%
\pgfsetstrokecolor{currentstroke}%
\pgfsetdash{}{0pt}%
\pgfpathmoveto{\pgfqpoint{1.942951in}{3.439868in}}%
\pgfusepath{stroke}%
\end{pgfscope}%
\begin{pgfscope}%
\pgfpathrectangle{\pgfqpoint{0.765000in}{0.660000in}}{\pgfqpoint{4.620000in}{4.620000in}}%
\pgfusepath{clip}%
\pgfsetbuttcap%
\pgfsetmiterjoin%
\definecolor{currentfill}{rgb}{0.176471,0.192157,0.258824}%
\pgfsetfillcolor{currentfill}%
\pgfsetlinewidth{1.003750pt}%
\definecolor{currentstroke}{rgb}{0.176471,0.192157,0.258824}%
\pgfsetstrokecolor{currentstroke}%
\pgfsetdash{}{0pt}%
\pgfsys@defobject{currentmarker}{\pgfqpoint{-0.033333in}{-0.033333in}}{\pgfqpoint{0.033333in}{0.033333in}}{%
\pgfpathmoveto{\pgfqpoint{-0.000000in}{-0.033333in}}%
\pgfpathlineto{\pgfqpoint{0.033333in}{0.033333in}}%
\pgfpathlineto{\pgfqpoint{-0.033333in}{0.033333in}}%
\pgfpathlineto{\pgfqpoint{-0.000000in}{-0.033333in}}%
\pgfpathclose%
\pgfusepath{stroke,fill}%
}%
\begin{pgfscope}%
\pgfsys@transformshift{1.942951in}{3.439868in}%
\pgfsys@useobject{currentmarker}{}%
\end{pgfscope}%
\end{pgfscope}%
\begin{pgfscope}%
\pgfpathrectangle{\pgfqpoint{0.765000in}{0.660000in}}{\pgfqpoint{4.620000in}{4.620000in}}%
\pgfusepath{clip}%
\pgfsetrectcap%
\pgfsetroundjoin%
\pgfsetlinewidth{1.204500pt}%
\definecolor{currentstroke}{rgb}{0.176471,0.192157,0.258824}%
\pgfsetstrokecolor{currentstroke}%
\pgfsetdash{}{0pt}%
\pgfpathmoveto{\pgfqpoint{1.647734in}{3.036230in}}%
\pgfusepath{stroke}%
\end{pgfscope}%
\begin{pgfscope}%
\pgfpathrectangle{\pgfqpoint{0.765000in}{0.660000in}}{\pgfqpoint{4.620000in}{4.620000in}}%
\pgfusepath{clip}%
\pgfsetbuttcap%
\pgfsetmiterjoin%
\definecolor{currentfill}{rgb}{0.176471,0.192157,0.258824}%
\pgfsetfillcolor{currentfill}%
\pgfsetlinewidth{1.003750pt}%
\definecolor{currentstroke}{rgb}{0.176471,0.192157,0.258824}%
\pgfsetstrokecolor{currentstroke}%
\pgfsetdash{}{0pt}%
\pgfsys@defobject{currentmarker}{\pgfqpoint{-0.033333in}{-0.033333in}}{\pgfqpoint{0.033333in}{0.033333in}}{%
\pgfpathmoveto{\pgfqpoint{-0.000000in}{-0.033333in}}%
\pgfpathlineto{\pgfqpoint{0.033333in}{0.033333in}}%
\pgfpathlineto{\pgfqpoint{-0.033333in}{0.033333in}}%
\pgfpathlineto{\pgfqpoint{-0.000000in}{-0.033333in}}%
\pgfpathclose%
\pgfusepath{stroke,fill}%
}%
\begin{pgfscope}%
\pgfsys@transformshift{1.647734in}{3.036230in}%
\pgfsys@useobject{currentmarker}{}%
\end{pgfscope}%
\end{pgfscope}%
\begin{pgfscope}%
\pgfpathrectangle{\pgfqpoint{0.765000in}{0.660000in}}{\pgfqpoint{4.620000in}{4.620000in}}%
\pgfusepath{clip}%
\pgfsetrectcap%
\pgfsetroundjoin%
\pgfsetlinewidth{1.204500pt}%
\definecolor{currentstroke}{rgb}{0.176471,0.192157,0.258824}%
\pgfsetstrokecolor{currentstroke}%
\pgfsetdash{}{0pt}%
\pgfpathmoveto{\pgfqpoint{3.322501in}{3.749250in}}%
\pgfusepath{stroke}%
\end{pgfscope}%
\begin{pgfscope}%
\pgfpathrectangle{\pgfqpoint{0.765000in}{0.660000in}}{\pgfqpoint{4.620000in}{4.620000in}}%
\pgfusepath{clip}%
\pgfsetbuttcap%
\pgfsetmiterjoin%
\definecolor{currentfill}{rgb}{0.176471,0.192157,0.258824}%
\pgfsetfillcolor{currentfill}%
\pgfsetlinewidth{1.003750pt}%
\definecolor{currentstroke}{rgb}{0.176471,0.192157,0.258824}%
\pgfsetstrokecolor{currentstroke}%
\pgfsetdash{}{0pt}%
\pgfsys@defobject{currentmarker}{\pgfqpoint{-0.033333in}{-0.033333in}}{\pgfqpoint{0.033333in}{0.033333in}}{%
\pgfpathmoveto{\pgfqpoint{-0.000000in}{-0.033333in}}%
\pgfpathlineto{\pgfqpoint{0.033333in}{0.033333in}}%
\pgfpathlineto{\pgfqpoint{-0.033333in}{0.033333in}}%
\pgfpathlineto{\pgfqpoint{-0.000000in}{-0.033333in}}%
\pgfpathclose%
\pgfusepath{stroke,fill}%
}%
\begin{pgfscope}%
\pgfsys@transformshift{3.322501in}{3.749250in}%
\pgfsys@useobject{currentmarker}{}%
\end{pgfscope}%
\end{pgfscope}%
\begin{pgfscope}%
\pgfpathrectangle{\pgfqpoint{0.765000in}{0.660000in}}{\pgfqpoint{4.620000in}{4.620000in}}%
\pgfusepath{clip}%
\pgfsetrectcap%
\pgfsetroundjoin%
\pgfsetlinewidth{1.204500pt}%
\definecolor{currentstroke}{rgb}{0.176471,0.192157,0.258824}%
\pgfsetstrokecolor{currentstroke}%
\pgfsetdash{}{0pt}%
\pgfpathmoveto{\pgfqpoint{1.975323in}{3.493702in}}%
\pgfusepath{stroke}%
\end{pgfscope}%
\begin{pgfscope}%
\pgfpathrectangle{\pgfqpoint{0.765000in}{0.660000in}}{\pgfqpoint{4.620000in}{4.620000in}}%
\pgfusepath{clip}%
\pgfsetbuttcap%
\pgfsetmiterjoin%
\definecolor{currentfill}{rgb}{0.176471,0.192157,0.258824}%
\pgfsetfillcolor{currentfill}%
\pgfsetlinewidth{1.003750pt}%
\definecolor{currentstroke}{rgb}{0.176471,0.192157,0.258824}%
\pgfsetstrokecolor{currentstroke}%
\pgfsetdash{}{0pt}%
\pgfsys@defobject{currentmarker}{\pgfqpoint{-0.033333in}{-0.033333in}}{\pgfqpoint{0.033333in}{0.033333in}}{%
\pgfpathmoveto{\pgfqpoint{-0.000000in}{-0.033333in}}%
\pgfpathlineto{\pgfqpoint{0.033333in}{0.033333in}}%
\pgfpathlineto{\pgfqpoint{-0.033333in}{0.033333in}}%
\pgfpathlineto{\pgfqpoint{-0.000000in}{-0.033333in}}%
\pgfpathclose%
\pgfusepath{stroke,fill}%
}%
\begin{pgfscope}%
\pgfsys@transformshift{1.975323in}{3.493702in}%
\pgfsys@useobject{currentmarker}{}%
\end{pgfscope}%
\end{pgfscope}%
\begin{pgfscope}%
\pgfpathrectangle{\pgfqpoint{0.765000in}{0.660000in}}{\pgfqpoint{4.620000in}{4.620000in}}%
\pgfusepath{clip}%
\pgfsetrectcap%
\pgfsetroundjoin%
\pgfsetlinewidth{1.204500pt}%
\definecolor{currentstroke}{rgb}{0.176471,0.192157,0.258824}%
\pgfsetstrokecolor{currentstroke}%
\pgfsetdash{}{0pt}%
\pgfpathmoveto{\pgfqpoint{3.218012in}{3.993230in}}%
\pgfusepath{stroke}%
\end{pgfscope}%
\begin{pgfscope}%
\pgfpathrectangle{\pgfqpoint{0.765000in}{0.660000in}}{\pgfqpoint{4.620000in}{4.620000in}}%
\pgfusepath{clip}%
\pgfsetbuttcap%
\pgfsetmiterjoin%
\definecolor{currentfill}{rgb}{0.176471,0.192157,0.258824}%
\pgfsetfillcolor{currentfill}%
\pgfsetlinewidth{1.003750pt}%
\definecolor{currentstroke}{rgb}{0.176471,0.192157,0.258824}%
\pgfsetstrokecolor{currentstroke}%
\pgfsetdash{}{0pt}%
\pgfsys@defobject{currentmarker}{\pgfqpoint{-0.033333in}{-0.033333in}}{\pgfqpoint{0.033333in}{0.033333in}}{%
\pgfpathmoveto{\pgfqpoint{-0.000000in}{-0.033333in}}%
\pgfpathlineto{\pgfqpoint{0.033333in}{0.033333in}}%
\pgfpathlineto{\pgfqpoint{-0.033333in}{0.033333in}}%
\pgfpathlineto{\pgfqpoint{-0.000000in}{-0.033333in}}%
\pgfpathclose%
\pgfusepath{stroke,fill}%
}%
\begin{pgfscope}%
\pgfsys@transformshift{3.218012in}{3.993230in}%
\pgfsys@useobject{currentmarker}{}%
\end{pgfscope}%
\end{pgfscope}%
\begin{pgfscope}%
\pgfpathrectangle{\pgfqpoint{0.765000in}{0.660000in}}{\pgfqpoint{4.620000in}{4.620000in}}%
\pgfusepath{clip}%
\pgfsetrectcap%
\pgfsetroundjoin%
\pgfsetlinewidth{1.204500pt}%
\definecolor{currentstroke}{rgb}{0.176471,0.192157,0.258824}%
\pgfsetstrokecolor{currentstroke}%
\pgfsetdash{}{0pt}%
\pgfpathmoveto{\pgfqpoint{4.109341in}{3.247221in}}%
\pgfusepath{stroke}%
\end{pgfscope}%
\begin{pgfscope}%
\pgfpathrectangle{\pgfqpoint{0.765000in}{0.660000in}}{\pgfqpoint{4.620000in}{4.620000in}}%
\pgfusepath{clip}%
\pgfsetbuttcap%
\pgfsetmiterjoin%
\definecolor{currentfill}{rgb}{0.176471,0.192157,0.258824}%
\pgfsetfillcolor{currentfill}%
\pgfsetlinewidth{1.003750pt}%
\definecolor{currentstroke}{rgb}{0.176471,0.192157,0.258824}%
\pgfsetstrokecolor{currentstroke}%
\pgfsetdash{}{0pt}%
\pgfsys@defobject{currentmarker}{\pgfqpoint{-0.033333in}{-0.033333in}}{\pgfqpoint{0.033333in}{0.033333in}}{%
\pgfpathmoveto{\pgfqpoint{-0.000000in}{-0.033333in}}%
\pgfpathlineto{\pgfqpoint{0.033333in}{0.033333in}}%
\pgfpathlineto{\pgfqpoint{-0.033333in}{0.033333in}}%
\pgfpathlineto{\pgfqpoint{-0.000000in}{-0.033333in}}%
\pgfpathclose%
\pgfusepath{stroke,fill}%
}%
\begin{pgfscope}%
\pgfsys@transformshift{4.109341in}{3.247221in}%
\pgfsys@useobject{currentmarker}{}%
\end{pgfscope}%
\end{pgfscope}%
\begin{pgfscope}%
\pgfpathrectangle{\pgfqpoint{0.765000in}{0.660000in}}{\pgfqpoint{4.620000in}{4.620000in}}%
\pgfusepath{clip}%
\pgfsetrectcap%
\pgfsetroundjoin%
\pgfsetlinewidth{1.204500pt}%
\definecolor{currentstroke}{rgb}{0.176471,0.192157,0.258824}%
\pgfsetstrokecolor{currentstroke}%
\pgfsetdash{}{0pt}%
\pgfpathmoveto{\pgfqpoint{1.443774in}{2.982997in}}%
\pgfusepath{stroke}%
\end{pgfscope}%
\begin{pgfscope}%
\pgfpathrectangle{\pgfqpoint{0.765000in}{0.660000in}}{\pgfqpoint{4.620000in}{4.620000in}}%
\pgfusepath{clip}%
\pgfsetbuttcap%
\pgfsetmiterjoin%
\definecolor{currentfill}{rgb}{0.176471,0.192157,0.258824}%
\pgfsetfillcolor{currentfill}%
\pgfsetlinewidth{1.003750pt}%
\definecolor{currentstroke}{rgb}{0.176471,0.192157,0.258824}%
\pgfsetstrokecolor{currentstroke}%
\pgfsetdash{}{0pt}%
\pgfsys@defobject{currentmarker}{\pgfqpoint{-0.033333in}{-0.033333in}}{\pgfqpoint{0.033333in}{0.033333in}}{%
\pgfpathmoveto{\pgfqpoint{-0.000000in}{-0.033333in}}%
\pgfpathlineto{\pgfqpoint{0.033333in}{0.033333in}}%
\pgfpathlineto{\pgfqpoint{-0.033333in}{0.033333in}}%
\pgfpathlineto{\pgfqpoint{-0.000000in}{-0.033333in}}%
\pgfpathclose%
\pgfusepath{stroke,fill}%
}%
\begin{pgfscope}%
\pgfsys@transformshift{1.443774in}{2.982997in}%
\pgfsys@useobject{currentmarker}{}%
\end{pgfscope}%
\end{pgfscope}%
\begin{pgfscope}%
\pgfpathrectangle{\pgfqpoint{0.765000in}{0.660000in}}{\pgfqpoint{4.620000in}{4.620000in}}%
\pgfusepath{clip}%
\pgfsetrectcap%
\pgfsetroundjoin%
\pgfsetlinewidth{1.204500pt}%
\definecolor{currentstroke}{rgb}{0.176471,0.192157,0.258824}%
\pgfsetstrokecolor{currentstroke}%
\pgfsetdash{}{0pt}%
\pgfpathmoveto{\pgfqpoint{3.658773in}{3.657330in}}%
\pgfusepath{stroke}%
\end{pgfscope}%
\begin{pgfscope}%
\pgfpathrectangle{\pgfqpoint{0.765000in}{0.660000in}}{\pgfqpoint{4.620000in}{4.620000in}}%
\pgfusepath{clip}%
\pgfsetbuttcap%
\pgfsetmiterjoin%
\definecolor{currentfill}{rgb}{0.176471,0.192157,0.258824}%
\pgfsetfillcolor{currentfill}%
\pgfsetlinewidth{1.003750pt}%
\definecolor{currentstroke}{rgb}{0.176471,0.192157,0.258824}%
\pgfsetstrokecolor{currentstroke}%
\pgfsetdash{}{0pt}%
\pgfsys@defobject{currentmarker}{\pgfqpoint{-0.033333in}{-0.033333in}}{\pgfqpoint{0.033333in}{0.033333in}}{%
\pgfpathmoveto{\pgfqpoint{-0.000000in}{-0.033333in}}%
\pgfpathlineto{\pgfqpoint{0.033333in}{0.033333in}}%
\pgfpathlineto{\pgfqpoint{-0.033333in}{0.033333in}}%
\pgfpathlineto{\pgfqpoint{-0.000000in}{-0.033333in}}%
\pgfpathclose%
\pgfusepath{stroke,fill}%
}%
\begin{pgfscope}%
\pgfsys@transformshift{3.658773in}{3.657330in}%
\pgfsys@useobject{currentmarker}{}%
\end{pgfscope}%
\end{pgfscope}%
\begin{pgfscope}%
\pgfpathrectangle{\pgfqpoint{0.765000in}{0.660000in}}{\pgfqpoint{4.620000in}{4.620000in}}%
\pgfusepath{clip}%
\pgfsetrectcap%
\pgfsetroundjoin%
\pgfsetlinewidth{1.204500pt}%
\definecolor{currentstroke}{rgb}{0.176471,0.192157,0.258824}%
\pgfsetstrokecolor{currentstroke}%
\pgfsetdash{}{0pt}%
\pgfpathmoveto{\pgfqpoint{3.734816in}{3.019265in}}%
\pgfusepath{stroke}%
\end{pgfscope}%
\begin{pgfscope}%
\pgfpathrectangle{\pgfqpoint{0.765000in}{0.660000in}}{\pgfqpoint{4.620000in}{4.620000in}}%
\pgfusepath{clip}%
\pgfsetbuttcap%
\pgfsetmiterjoin%
\definecolor{currentfill}{rgb}{0.176471,0.192157,0.258824}%
\pgfsetfillcolor{currentfill}%
\pgfsetlinewidth{1.003750pt}%
\definecolor{currentstroke}{rgb}{0.176471,0.192157,0.258824}%
\pgfsetstrokecolor{currentstroke}%
\pgfsetdash{}{0pt}%
\pgfsys@defobject{currentmarker}{\pgfqpoint{-0.033333in}{-0.033333in}}{\pgfqpoint{0.033333in}{0.033333in}}{%
\pgfpathmoveto{\pgfqpoint{-0.000000in}{-0.033333in}}%
\pgfpathlineto{\pgfqpoint{0.033333in}{0.033333in}}%
\pgfpathlineto{\pgfqpoint{-0.033333in}{0.033333in}}%
\pgfpathlineto{\pgfqpoint{-0.000000in}{-0.033333in}}%
\pgfpathclose%
\pgfusepath{stroke,fill}%
}%
\begin{pgfscope}%
\pgfsys@transformshift{3.734816in}{3.019265in}%
\pgfsys@useobject{currentmarker}{}%
\end{pgfscope}%
\end{pgfscope}%
\begin{pgfscope}%
\pgfpathrectangle{\pgfqpoint{0.765000in}{0.660000in}}{\pgfqpoint{4.620000in}{4.620000in}}%
\pgfusepath{clip}%
\pgfsetrectcap%
\pgfsetroundjoin%
\pgfsetlinewidth{1.204500pt}%
\definecolor{currentstroke}{rgb}{0.176471,0.192157,0.258824}%
\pgfsetstrokecolor{currentstroke}%
\pgfsetdash{}{0pt}%
\pgfpathmoveto{\pgfqpoint{2.673707in}{3.614585in}}%
\pgfusepath{stroke}%
\end{pgfscope}%
\begin{pgfscope}%
\pgfpathrectangle{\pgfqpoint{0.765000in}{0.660000in}}{\pgfqpoint{4.620000in}{4.620000in}}%
\pgfusepath{clip}%
\pgfsetbuttcap%
\pgfsetmiterjoin%
\definecolor{currentfill}{rgb}{0.176471,0.192157,0.258824}%
\pgfsetfillcolor{currentfill}%
\pgfsetlinewidth{1.003750pt}%
\definecolor{currentstroke}{rgb}{0.176471,0.192157,0.258824}%
\pgfsetstrokecolor{currentstroke}%
\pgfsetdash{}{0pt}%
\pgfsys@defobject{currentmarker}{\pgfqpoint{-0.033333in}{-0.033333in}}{\pgfqpoint{0.033333in}{0.033333in}}{%
\pgfpathmoveto{\pgfqpoint{-0.000000in}{-0.033333in}}%
\pgfpathlineto{\pgfqpoint{0.033333in}{0.033333in}}%
\pgfpathlineto{\pgfqpoint{-0.033333in}{0.033333in}}%
\pgfpathlineto{\pgfqpoint{-0.000000in}{-0.033333in}}%
\pgfpathclose%
\pgfusepath{stroke,fill}%
}%
\begin{pgfscope}%
\pgfsys@transformshift{2.673707in}{3.614585in}%
\pgfsys@useobject{currentmarker}{}%
\end{pgfscope}%
\end{pgfscope}%
\begin{pgfscope}%
\pgfpathrectangle{\pgfqpoint{0.765000in}{0.660000in}}{\pgfqpoint{4.620000in}{4.620000in}}%
\pgfusepath{clip}%
\pgfsetrectcap%
\pgfsetroundjoin%
\pgfsetlinewidth{1.204500pt}%
\definecolor{currentstroke}{rgb}{0.176471,0.192157,0.258824}%
\pgfsetstrokecolor{currentstroke}%
\pgfsetdash{}{0pt}%
\pgfpathmoveto{\pgfqpoint{2.030724in}{3.262091in}}%
\pgfusepath{stroke}%
\end{pgfscope}%
\begin{pgfscope}%
\pgfpathrectangle{\pgfqpoint{0.765000in}{0.660000in}}{\pgfqpoint{4.620000in}{4.620000in}}%
\pgfusepath{clip}%
\pgfsetbuttcap%
\pgfsetmiterjoin%
\definecolor{currentfill}{rgb}{0.176471,0.192157,0.258824}%
\pgfsetfillcolor{currentfill}%
\pgfsetlinewidth{1.003750pt}%
\definecolor{currentstroke}{rgb}{0.176471,0.192157,0.258824}%
\pgfsetstrokecolor{currentstroke}%
\pgfsetdash{}{0pt}%
\pgfsys@defobject{currentmarker}{\pgfqpoint{-0.033333in}{-0.033333in}}{\pgfqpoint{0.033333in}{0.033333in}}{%
\pgfpathmoveto{\pgfqpoint{-0.000000in}{-0.033333in}}%
\pgfpathlineto{\pgfqpoint{0.033333in}{0.033333in}}%
\pgfpathlineto{\pgfqpoint{-0.033333in}{0.033333in}}%
\pgfpathlineto{\pgfqpoint{-0.000000in}{-0.033333in}}%
\pgfpathclose%
\pgfusepath{stroke,fill}%
}%
\begin{pgfscope}%
\pgfsys@transformshift{2.030724in}{3.262091in}%
\pgfsys@useobject{currentmarker}{}%
\end{pgfscope}%
\end{pgfscope}%
\begin{pgfscope}%
\pgfpathrectangle{\pgfqpoint{0.765000in}{0.660000in}}{\pgfqpoint{4.620000in}{4.620000in}}%
\pgfusepath{clip}%
\pgfsetrectcap%
\pgfsetroundjoin%
\pgfsetlinewidth{1.204500pt}%
\definecolor{currentstroke}{rgb}{0.176471,0.192157,0.258824}%
\pgfsetstrokecolor{currentstroke}%
\pgfsetdash{}{0pt}%
\pgfpathmoveto{\pgfqpoint{1.151863in}{3.062159in}}%
\pgfusepath{stroke}%
\end{pgfscope}%
\begin{pgfscope}%
\pgfpathrectangle{\pgfqpoint{0.765000in}{0.660000in}}{\pgfqpoint{4.620000in}{4.620000in}}%
\pgfusepath{clip}%
\pgfsetbuttcap%
\pgfsetmiterjoin%
\definecolor{currentfill}{rgb}{0.176471,0.192157,0.258824}%
\pgfsetfillcolor{currentfill}%
\pgfsetlinewidth{1.003750pt}%
\definecolor{currentstroke}{rgb}{0.176471,0.192157,0.258824}%
\pgfsetstrokecolor{currentstroke}%
\pgfsetdash{}{0pt}%
\pgfsys@defobject{currentmarker}{\pgfqpoint{-0.033333in}{-0.033333in}}{\pgfqpoint{0.033333in}{0.033333in}}{%
\pgfpathmoveto{\pgfqpoint{-0.000000in}{-0.033333in}}%
\pgfpathlineto{\pgfqpoint{0.033333in}{0.033333in}}%
\pgfpathlineto{\pgfqpoint{-0.033333in}{0.033333in}}%
\pgfpathlineto{\pgfqpoint{-0.000000in}{-0.033333in}}%
\pgfpathclose%
\pgfusepath{stroke,fill}%
}%
\begin{pgfscope}%
\pgfsys@transformshift{1.151863in}{3.062159in}%
\pgfsys@useobject{currentmarker}{}%
\end{pgfscope}%
\end{pgfscope}%
\begin{pgfscope}%
\pgfpathrectangle{\pgfqpoint{0.765000in}{0.660000in}}{\pgfqpoint{4.620000in}{4.620000in}}%
\pgfusepath{clip}%
\pgfsetrectcap%
\pgfsetroundjoin%
\pgfsetlinewidth{1.204500pt}%
\definecolor{currentstroke}{rgb}{0.176471,0.192157,0.258824}%
\pgfsetstrokecolor{currentstroke}%
\pgfsetdash{}{0pt}%
\pgfpathmoveto{\pgfqpoint{3.873934in}{3.382896in}}%
\pgfusepath{stroke}%
\end{pgfscope}%
\begin{pgfscope}%
\pgfpathrectangle{\pgfqpoint{0.765000in}{0.660000in}}{\pgfqpoint{4.620000in}{4.620000in}}%
\pgfusepath{clip}%
\pgfsetbuttcap%
\pgfsetmiterjoin%
\definecolor{currentfill}{rgb}{0.176471,0.192157,0.258824}%
\pgfsetfillcolor{currentfill}%
\pgfsetlinewidth{1.003750pt}%
\definecolor{currentstroke}{rgb}{0.176471,0.192157,0.258824}%
\pgfsetstrokecolor{currentstroke}%
\pgfsetdash{}{0pt}%
\pgfsys@defobject{currentmarker}{\pgfqpoint{-0.033333in}{-0.033333in}}{\pgfqpoint{0.033333in}{0.033333in}}{%
\pgfpathmoveto{\pgfqpoint{-0.000000in}{-0.033333in}}%
\pgfpathlineto{\pgfqpoint{0.033333in}{0.033333in}}%
\pgfpathlineto{\pgfqpoint{-0.033333in}{0.033333in}}%
\pgfpathlineto{\pgfqpoint{-0.000000in}{-0.033333in}}%
\pgfpathclose%
\pgfusepath{stroke,fill}%
}%
\begin{pgfscope}%
\pgfsys@transformshift{3.873934in}{3.382896in}%
\pgfsys@useobject{currentmarker}{}%
\end{pgfscope}%
\end{pgfscope}%
\begin{pgfscope}%
\pgfpathrectangle{\pgfqpoint{0.765000in}{0.660000in}}{\pgfqpoint{4.620000in}{4.620000in}}%
\pgfusepath{clip}%
\pgfsetrectcap%
\pgfsetroundjoin%
\pgfsetlinewidth{1.204500pt}%
\definecolor{currentstroke}{rgb}{0.176471,0.192157,0.258824}%
\pgfsetstrokecolor{currentstroke}%
\pgfsetdash{}{0pt}%
\pgfpathmoveto{\pgfqpoint{3.751687in}{2.898405in}}%
\pgfusepath{stroke}%
\end{pgfscope}%
\begin{pgfscope}%
\pgfpathrectangle{\pgfqpoint{0.765000in}{0.660000in}}{\pgfqpoint{4.620000in}{4.620000in}}%
\pgfusepath{clip}%
\pgfsetbuttcap%
\pgfsetmiterjoin%
\definecolor{currentfill}{rgb}{0.176471,0.192157,0.258824}%
\pgfsetfillcolor{currentfill}%
\pgfsetlinewidth{1.003750pt}%
\definecolor{currentstroke}{rgb}{0.176471,0.192157,0.258824}%
\pgfsetstrokecolor{currentstroke}%
\pgfsetdash{}{0pt}%
\pgfsys@defobject{currentmarker}{\pgfqpoint{-0.033333in}{-0.033333in}}{\pgfqpoint{0.033333in}{0.033333in}}{%
\pgfpathmoveto{\pgfqpoint{-0.000000in}{-0.033333in}}%
\pgfpathlineto{\pgfqpoint{0.033333in}{0.033333in}}%
\pgfpathlineto{\pgfqpoint{-0.033333in}{0.033333in}}%
\pgfpathlineto{\pgfqpoint{-0.000000in}{-0.033333in}}%
\pgfpathclose%
\pgfusepath{stroke,fill}%
}%
\begin{pgfscope}%
\pgfsys@transformshift{3.751687in}{2.898405in}%
\pgfsys@useobject{currentmarker}{}%
\end{pgfscope}%
\end{pgfscope}%
\begin{pgfscope}%
\pgfpathrectangle{\pgfqpoint{0.765000in}{0.660000in}}{\pgfqpoint{4.620000in}{4.620000in}}%
\pgfusepath{clip}%
\pgfsetrectcap%
\pgfsetroundjoin%
\pgfsetlinewidth{1.204500pt}%
\definecolor{currentstroke}{rgb}{0.176471,0.192157,0.258824}%
\pgfsetstrokecolor{currentstroke}%
\pgfsetdash{}{0pt}%
\pgfpathmoveto{\pgfqpoint{3.504135in}{3.912421in}}%
\pgfusepath{stroke}%
\end{pgfscope}%
\begin{pgfscope}%
\pgfpathrectangle{\pgfqpoint{0.765000in}{0.660000in}}{\pgfqpoint{4.620000in}{4.620000in}}%
\pgfusepath{clip}%
\pgfsetbuttcap%
\pgfsetmiterjoin%
\definecolor{currentfill}{rgb}{0.176471,0.192157,0.258824}%
\pgfsetfillcolor{currentfill}%
\pgfsetlinewidth{1.003750pt}%
\definecolor{currentstroke}{rgb}{0.176471,0.192157,0.258824}%
\pgfsetstrokecolor{currentstroke}%
\pgfsetdash{}{0pt}%
\pgfsys@defobject{currentmarker}{\pgfqpoint{-0.033333in}{-0.033333in}}{\pgfqpoint{0.033333in}{0.033333in}}{%
\pgfpathmoveto{\pgfqpoint{-0.000000in}{-0.033333in}}%
\pgfpathlineto{\pgfqpoint{0.033333in}{0.033333in}}%
\pgfpathlineto{\pgfqpoint{-0.033333in}{0.033333in}}%
\pgfpathlineto{\pgfqpoint{-0.000000in}{-0.033333in}}%
\pgfpathclose%
\pgfusepath{stroke,fill}%
}%
\begin{pgfscope}%
\pgfsys@transformshift{3.504135in}{3.912421in}%
\pgfsys@useobject{currentmarker}{}%
\end{pgfscope}%
\end{pgfscope}%
\begin{pgfscope}%
\pgfpathrectangle{\pgfqpoint{0.765000in}{0.660000in}}{\pgfqpoint{4.620000in}{4.620000in}}%
\pgfusepath{clip}%
\pgfsetrectcap%
\pgfsetroundjoin%
\pgfsetlinewidth{1.204500pt}%
\definecolor{currentstroke}{rgb}{0.176471,0.192157,0.258824}%
\pgfsetstrokecolor{currentstroke}%
\pgfsetdash{}{0pt}%
\pgfpathmoveto{\pgfqpoint{4.098812in}{3.295655in}}%
\pgfusepath{stroke}%
\end{pgfscope}%
\begin{pgfscope}%
\pgfpathrectangle{\pgfqpoint{0.765000in}{0.660000in}}{\pgfqpoint{4.620000in}{4.620000in}}%
\pgfusepath{clip}%
\pgfsetbuttcap%
\pgfsetmiterjoin%
\definecolor{currentfill}{rgb}{0.176471,0.192157,0.258824}%
\pgfsetfillcolor{currentfill}%
\pgfsetlinewidth{1.003750pt}%
\definecolor{currentstroke}{rgb}{0.176471,0.192157,0.258824}%
\pgfsetstrokecolor{currentstroke}%
\pgfsetdash{}{0pt}%
\pgfsys@defobject{currentmarker}{\pgfqpoint{-0.033333in}{-0.033333in}}{\pgfqpoint{0.033333in}{0.033333in}}{%
\pgfpathmoveto{\pgfqpoint{-0.000000in}{-0.033333in}}%
\pgfpathlineto{\pgfqpoint{0.033333in}{0.033333in}}%
\pgfpathlineto{\pgfqpoint{-0.033333in}{0.033333in}}%
\pgfpathlineto{\pgfqpoint{-0.000000in}{-0.033333in}}%
\pgfpathclose%
\pgfusepath{stroke,fill}%
}%
\begin{pgfscope}%
\pgfsys@transformshift{4.098812in}{3.295655in}%
\pgfsys@useobject{currentmarker}{}%
\end{pgfscope}%
\end{pgfscope}%
\begin{pgfscope}%
\pgfpathrectangle{\pgfqpoint{0.765000in}{0.660000in}}{\pgfqpoint{4.620000in}{4.620000in}}%
\pgfusepath{clip}%
\pgfsetrectcap%
\pgfsetroundjoin%
\pgfsetlinewidth{1.204500pt}%
\definecolor{currentstroke}{rgb}{0.176471,0.192157,0.258824}%
\pgfsetstrokecolor{currentstroke}%
\pgfsetdash{}{0pt}%
\pgfpathmoveto{\pgfqpoint{3.509425in}{3.287215in}}%
\pgfusepath{stroke}%
\end{pgfscope}%
\begin{pgfscope}%
\pgfpathrectangle{\pgfqpoint{0.765000in}{0.660000in}}{\pgfqpoint{4.620000in}{4.620000in}}%
\pgfusepath{clip}%
\pgfsetbuttcap%
\pgfsetmiterjoin%
\definecolor{currentfill}{rgb}{0.176471,0.192157,0.258824}%
\pgfsetfillcolor{currentfill}%
\pgfsetlinewidth{1.003750pt}%
\definecolor{currentstroke}{rgb}{0.176471,0.192157,0.258824}%
\pgfsetstrokecolor{currentstroke}%
\pgfsetdash{}{0pt}%
\pgfsys@defobject{currentmarker}{\pgfqpoint{-0.033333in}{-0.033333in}}{\pgfqpoint{0.033333in}{0.033333in}}{%
\pgfpathmoveto{\pgfqpoint{-0.000000in}{-0.033333in}}%
\pgfpathlineto{\pgfqpoint{0.033333in}{0.033333in}}%
\pgfpathlineto{\pgfqpoint{-0.033333in}{0.033333in}}%
\pgfpathlineto{\pgfqpoint{-0.000000in}{-0.033333in}}%
\pgfpathclose%
\pgfusepath{stroke,fill}%
}%
\begin{pgfscope}%
\pgfsys@transformshift{3.509425in}{3.287215in}%
\pgfsys@useobject{currentmarker}{}%
\end{pgfscope}%
\end{pgfscope}%
\begin{pgfscope}%
\pgfpathrectangle{\pgfqpoint{0.765000in}{0.660000in}}{\pgfqpoint{4.620000in}{4.620000in}}%
\pgfusepath{clip}%
\pgfsetrectcap%
\pgfsetroundjoin%
\pgfsetlinewidth{1.204500pt}%
\definecolor{currentstroke}{rgb}{0.176471,0.192157,0.258824}%
\pgfsetstrokecolor{currentstroke}%
\pgfsetdash{}{0pt}%
\pgfpathmoveto{\pgfqpoint{3.473620in}{3.266050in}}%
\pgfusepath{stroke}%
\end{pgfscope}%
\begin{pgfscope}%
\pgfpathrectangle{\pgfqpoint{0.765000in}{0.660000in}}{\pgfqpoint{4.620000in}{4.620000in}}%
\pgfusepath{clip}%
\pgfsetbuttcap%
\pgfsetmiterjoin%
\definecolor{currentfill}{rgb}{0.176471,0.192157,0.258824}%
\pgfsetfillcolor{currentfill}%
\pgfsetlinewidth{1.003750pt}%
\definecolor{currentstroke}{rgb}{0.176471,0.192157,0.258824}%
\pgfsetstrokecolor{currentstroke}%
\pgfsetdash{}{0pt}%
\pgfsys@defobject{currentmarker}{\pgfqpoint{-0.033333in}{-0.033333in}}{\pgfqpoint{0.033333in}{0.033333in}}{%
\pgfpathmoveto{\pgfqpoint{-0.000000in}{-0.033333in}}%
\pgfpathlineto{\pgfqpoint{0.033333in}{0.033333in}}%
\pgfpathlineto{\pgfqpoint{-0.033333in}{0.033333in}}%
\pgfpathlineto{\pgfqpoint{-0.000000in}{-0.033333in}}%
\pgfpathclose%
\pgfusepath{stroke,fill}%
}%
\begin{pgfscope}%
\pgfsys@transformshift{3.473620in}{3.266050in}%
\pgfsys@useobject{currentmarker}{}%
\end{pgfscope}%
\end{pgfscope}%
\begin{pgfscope}%
\pgfpathrectangle{\pgfqpoint{0.765000in}{0.660000in}}{\pgfqpoint{4.620000in}{4.620000in}}%
\pgfusepath{clip}%
\pgfsetrectcap%
\pgfsetroundjoin%
\pgfsetlinewidth{1.204500pt}%
\definecolor{currentstroke}{rgb}{0.176471,0.192157,0.258824}%
\pgfsetstrokecolor{currentstroke}%
\pgfsetdash{}{0pt}%
\pgfpathmoveto{\pgfqpoint{2.426243in}{3.568364in}}%
\pgfusepath{stroke}%
\end{pgfscope}%
\begin{pgfscope}%
\pgfpathrectangle{\pgfqpoint{0.765000in}{0.660000in}}{\pgfqpoint{4.620000in}{4.620000in}}%
\pgfusepath{clip}%
\pgfsetbuttcap%
\pgfsetmiterjoin%
\definecolor{currentfill}{rgb}{0.176471,0.192157,0.258824}%
\pgfsetfillcolor{currentfill}%
\pgfsetlinewidth{1.003750pt}%
\definecolor{currentstroke}{rgb}{0.176471,0.192157,0.258824}%
\pgfsetstrokecolor{currentstroke}%
\pgfsetdash{}{0pt}%
\pgfsys@defobject{currentmarker}{\pgfqpoint{-0.033333in}{-0.033333in}}{\pgfqpoint{0.033333in}{0.033333in}}{%
\pgfpathmoveto{\pgfqpoint{-0.000000in}{-0.033333in}}%
\pgfpathlineto{\pgfqpoint{0.033333in}{0.033333in}}%
\pgfpathlineto{\pgfqpoint{-0.033333in}{0.033333in}}%
\pgfpathlineto{\pgfqpoint{-0.000000in}{-0.033333in}}%
\pgfpathclose%
\pgfusepath{stroke,fill}%
}%
\begin{pgfscope}%
\pgfsys@transformshift{2.426243in}{3.568364in}%
\pgfsys@useobject{currentmarker}{}%
\end{pgfscope}%
\end{pgfscope}%
\begin{pgfscope}%
\pgfpathrectangle{\pgfqpoint{0.765000in}{0.660000in}}{\pgfqpoint{4.620000in}{4.620000in}}%
\pgfusepath{clip}%
\pgfsetrectcap%
\pgfsetroundjoin%
\pgfsetlinewidth{1.204500pt}%
\definecolor{currentstroke}{rgb}{0.176471,0.192157,0.258824}%
\pgfsetstrokecolor{currentstroke}%
\pgfsetdash{}{0pt}%
\pgfpathmoveto{\pgfqpoint{2.830877in}{3.623423in}}%
\pgfusepath{stroke}%
\end{pgfscope}%
\begin{pgfscope}%
\pgfpathrectangle{\pgfqpoint{0.765000in}{0.660000in}}{\pgfqpoint{4.620000in}{4.620000in}}%
\pgfusepath{clip}%
\pgfsetbuttcap%
\pgfsetmiterjoin%
\definecolor{currentfill}{rgb}{0.176471,0.192157,0.258824}%
\pgfsetfillcolor{currentfill}%
\pgfsetlinewidth{1.003750pt}%
\definecolor{currentstroke}{rgb}{0.176471,0.192157,0.258824}%
\pgfsetstrokecolor{currentstroke}%
\pgfsetdash{}{0pt}%
\pgfsys@defobject{currentmarker}{\pgfqpoint{-0.033333in}{-0.033333in}}{\pgfqpoint{0.033333in}{0.033333in}}{%
\pgfpathmoveto{\pgfqpoint{-0.000000in}{-0.033333in}}%
\pgfpathlineto{\pgfqpoint{0.033333in}{0.033333in}}%
\pgfpathlineto{\pgfqpoint{-0.033333in}{0.033333in}}%
\pgfpathlineto{\pgfqpoint{-0.000000in}{-0.033333in}}%
\pgfpathclose%
\pgfusepath{stroke,fill}%
}%
\begin{pgfscope}%
\pgfsys@transformshift{2.830877in}{3.623423in}%
\pgfsys@useobject{currentmarker}{}%
\end{pgfscope}%
\end{pgfscope}%
\begin{pgfscope}%
\pgfpathrectangle{\pgfqpoint{0.765000in}{0.660000in}}{\pgfqpoint{4.620000in}{4.620000in}}%
\pgfusepath{clip}%
\pgfsetrectcap%
\pgfsetroundjoin%
\pgfsetlinewidth{1.204500pt}%
\definecolor{currentstroke}{rgb}{0.176471,0.192157,0.258824}%
\pgfsetstrokecolor{currentstroke}%
\pgfsetdash{}{0pt}%
\pgfpathmoveto{\pgfqpoint{3.720555in}{3.262853in}}%
\pgfusepath{stroke}%
\end{pgfscope}%
\begin{pgfscope}%
\pgfpathrectangle{\pgfqpoint{0.765000in}{0.660000in}}{\pgfqpoint{4.620000in}{4.620000in}}%
\pgfusepath{clip}%
\pgfsetbuttcap%
\pgfsetmiterjoin%
\definecolor{currentfill}{rgb}{0.176471,0.192157,0.258824}%
\pgfsetfillcolor{currentfill}%
\pgfsetlinewidth{1.003750pt}%
\definecolor{currentstroke}{rgb}{0.176471,0.192157,0.258824}%
\pgfsetstrokecolor{currentstroke}%
\pgfsetdash{}{0pt}%
\pgfsys@defobject{currentmarker}{\pgfqpoint{-0.033333in}{-0.033333in}}{\pgfqpoint{0.033333in}{0.033333in}}{%
\pgfpathmoveto{\pgfqpoint{-0.000000in}{-0.033333in}}%
\pgfpathlineto{\pgfqpoint{0.033333in}{0.033333in}}%
\pgfpathlineto{\pgfqpoint{-0.033333in}{0.033333in}}%
\pgfpathlineto{\pgfqpoint{-0.000000in}{-0.033333in}}%
\pgfpathclose%
\pgfusepath{stroke,fill}%
}%
\begin{pgfscope}%
\pgfsys@transformshift{3.720555in}{3.262853in}%
\pgfsys@useobject{currentmarker}{}%
\end{pgfscope}%
\end{pgfscope}%
\begin{pgfscope}%
\pgfpathrectangle{\pgfqpoint{0.765000in}{0.660000in}}{\pgfqpoint{4.620000in}{4.620000in}}%
\pgfusepath{clip}%
\pgfsetrectcap%
\pgfsetroundjoin%
\pgfsetlinewidth{1.204500pt}%
\definecolor{currentstroke}{rgb}{0.176471,0.192157,0.258824}%
\pgfsetstrokecolor{currentstroke}%
\pgfsetdash{}{0pt}%
\pgfpathmoveto{\pgfqpoint{3.347171in}{3.451717in}}%
\pgfusepath{stroke}%
\end{pgfscope}%
\begin{pgfscope}%
\pgfpathrectangle{\pgfqpoint{0.765000in}{0.660000in}}{\pgfqpoint{4.620000in}{4.620000in}}%
\pgfusepath{clip}%
\pgfsetbuttcap%
\pgfsetmiterjoin%
\definecolor{currentfill}{rgb}{0.176471,0.192157,0.258824}%
\pgfsetfillcolor{currentfill}%
\pgfsetlinewidth{1.003750pt}%
\definecolor{currentstroke}{rgb}{0.176471,0.192157,0.258824}%
\pgfsetstrokecolor{currentstroke}%
\pgfsetdash{}{0pt}%
\pgfsys@defobject{currentmarker}{\pgfqpoint{-0.033333in}{-0.033333in}}{\pgfqpoint{0.033333in}{0.033333in}}{%
\pgfpathmoveto{\pgfqpoint{-0.000000in}{-0.033333in}}%
\pgfpathlineto{\pgfqpoint{0.033333in}{0.033333in}}%
\pgfpathlineto{\pgfqpoint{-0.033333in}{0.033333in}}%
\pgfpathlineto{\pgfqpoint{-0.000000in}{-0.033333in}}%
\pgfpathclose%
\pgfusepath{stroke,fill}%
}%
\begin{pgfscope}%
\pgfsys@transformshift{3.347171in}{3.451717in}%
\pgfsys@useobject{currentmarker}{}%
\end{pgfscope}%
\end{pgfscope}%
\begin{pgfscope}%
\pgfpathrectangle{\pgfqpoint{0.765000in}{0.660000in}}{\pgfqpoint{4.620000in}{4.620000in}}%
\pgfusepath{clip}%
\pgfsetrectcap%
\pgfsetroundjoin%
\pgfsetlinewidth{1.204500pt}%
\definecolor{currentstroke}{rgb}{0.176471,0.192157,0.258824}%
\pgfsetstrokecolor{currentstroke}%
\pgfsetdash{}{0pt}%
\pgfpathmoveto{\pgfqpoint{3.284686in}{3.825545in}}%
\pgfusepath{stroke}%
\end{pgfscope}%
\begin{pgfscope}%
\pgfpathrectangle{\pgfqpoint{0.765000in}{0.660000in}}{\pgfqpoint{4.620000in}{4.620000in}}%
\pgfusepath{clip}%
\pgfsetbuttcap%
\pgfsetmiterjoin%
\definecolor{currentfill}{rgb}{0.176471,0.192157,0.258824}%
\pgfsetfillcolor{currentfill}%
\pgfsetlinewidth{1.003750pt}%
\definecolor{currentstroke}{rgb}{0.176471,0.192157,0.258824}%
\pgfsetstrokecolor{currentstroke}%
\pgfsetdash{}{0pt}%
\pgfsys@defobject{currentmarker}{\pgfqpoint{-0.033333in}{-0.033333in}}{\pgfqpoint{0.033333in}{0.033333in}}{%
\pgfpathmoveto{\pgfqpoint{-0.000000in}{-0.033333in}}%
\pgfpathlineto{\pgfqpoint{0.033333in}{0.033333in}}%
\pgfpathlineto{\pgfqpoint{-0.033333in}{0.033333in}}%
\pgfpathlineto{\pgfqpoint{-0.000000in}{-0.033333in}}%
\pgfpathclose%
\pgfusepath{stroke,fill}%
}%
\begin{pgfscope}%
\pgfsys@transformshift{3.284686in}{3.825545in}%
\pgfsys@useobject{currentmarker}{}%
\end{pgfscope}%
\end{pgfscope}%
\begin{pgfscope}%
\pgfpathrectangle{\pgfqpoint{0.765000in}{0.660000in}}{\pgfqpoint{4.620000in}{4.620000in}}%
\pgfusepath{clip}%
\pgfsetrectcap%
\pgfsetroundjoin%
\pgfsetlinewidth{1.204500pt}%
\definecolor{currentstroke}{rgb}{0.176471,0.192157,0.258824}%
\pgfsetstrokecolor{currentstroke}%
\pgfsetdash{}{0pt}%
\pgfpathmoveto{\pgfqpoint{3.371455in}{3.730738in}}%
\pgfusepath{stroke}%
\end{pgfscope}%
\begin{pgfscope}%
\pgfpathrectangle{\pgfqpoint{0.765000in}{0.660000in}}{\pgfqpoint{4.620000in}{4.620000in}}%
\pgfusepath{clip}%
\pgfsetbuttcap%
\pgfsetmiterjoin%
\definecolor{currentfill}{rgb}{0.176471,0.192157,0.258824}%
\pgfsetfillcolor{currentfill}%
\pgfsetlinewidth{1.003750pt}%
\definecolor{currentstroke}{rgb}{0.176471,0.192157,0.258824}%
\pgfsetstrokecolor{currentstroke}%
\pgfsetdash{}{0pt}%
\pgfsys@defobject{currentmarker}{\pgfqpoint{-0.033333in}{-0.033333in}}{\pgfqpoint{0.033333in}{0.033333in}}{%
\pgfpathmoveto{\pgfqpoint{-0.000000in}{-0.033333in}}%
\pgfpathlineto{\pgfqpoint{0.033333in}{0.033333in}}%
\pgfpathlineto{\pgfqpoint{-0.033333in}{0.033333in}}%
\pgfpathlineto{\pgfqpoint{-0.000000in}{-0.033333in}}%
\pgfpathclose%
\pgfusepath{stroke,fill}%
}%
\begin{pgfscope}%
\pgfsys@transformshift{3.371455in}{3.730738in}%
\pgfsys@useobject{currentmarker}{}%
\end{pgfscope}%
\end{pgfscope}%
\begin{pgfscope}%
\pgfpathrectangle{\pgfqpoint{0.765000in}{0.660000in}}{\pgfqpoint{4.620000in}{4.620000in}}%
\pgfusepath{clip}%
\pgfsetrectcap%
\pgfsetroundjoin%
\pgfsetlinewidth{1.204500pt}%
\definecolor{currentstroke}{rgb}{0.176471,0.192157,0.258824}%
\pgfsetstrokecolor{currentstroke}%
\pgfsetdash{}{0pt}%
\pgfpathmoveto{\pgfqpoint{3.751612in}{3.122098in}}%
\pgfusepath{stroke}%
\end{pgfscope}%
\begin{pgfscope}%
\pgfpathrectangle{\pgfqpoint{0.765000in}{0.660000in}}{\pgfqpoint{4.620000in}{4.620000in}}%
\pgfusepath{clip}%
\pgfsetbuttcap%
\pgfsetmiterjoin%
\definecolor{currentfill}{rgb}{0.176471,0.192157,0.258824}%
\pgfsetfillcolor{currentfill}%
\pgfsetlinewidth{1.003750pt}%
\definecolor{currentstroke}{rgb}{0.176471,0.192157,0.258824}%
\pgfsetstrokecolor{currentstroke}%
\pgfsetdash{}{0pt}%
\pgfsys@defobject{currentmarker}{\pgfqpoint{-0.033333in}{-0.033333in}}{\pgfqpoint{0.033333in}{0.033333in}}{%
\pgfpathmoveto{\pgfqpoint{-0.000000in}{-0.033333in}}%
\pgfpathlineto{\pgfqpoint{0.033333in}{0.033333in}}%
\pgfpathlineto{\pgfqpoint{-0.033333in}{0.033333in}}%
\pgfpathlineto{\pgfqpoint{-0.000000in}{-0.033333in}}%
\pgfpathclose%
\pgfusepath{stroke,fill}%
}%
\begin{pgfscope}%
\pgfsys@transformshift{3.751612in}{3.122098in}%
\pgfsys@useobject{currentmarker}{}%
\end{pgfscope}%
\end{pgfscope}%
\begin{pgfscope}%
\pgfpathrectangle{\pgfqpoint{0.765000in}{0.660000in}}{\pgfqpoint{4.620000in}{4.620000in}}%
\pgfusepath{clip}%
\pgfsetrectcap%
\pgfsetroundjoin%
\pgfsetlinewidth{1.204500pt}%
\definecolor{currentstroke}{rgb}{0.176471,0.192157,0.258824}%
\pgfsetstrokecolor{currentstroke}%
\pgfsetdash{}{0pt}%
\pgfpathmoveto{\pgfqpoint{3.846694in}{2.901213in}}%
\pgfusepath{stroke}%
\end{pgfscope}%
\begin{pgfscope}%
\pgfpathrectangle{\pgfqpoint{0.765000in}{0.660000in}}{\pgfqpoint{4.620000in}{4.620000in}}%
\pgfusepath{clip}%
\pgfsetbuttcap%
\pgfsetmiterjoin%
\definecolor{currentfill}{rgb}{0.176471,0.192157,0.258824}%
\pgfsetfillcolor{currentfill}%
\pgfsetlinewidth{1.003750pt}%
\definecolor{currentstroke}{rgb}{0.176471,0.192157,0.258824}%
\pgfsetstrokecolor{currentstroke}%
\pgfsetdash{}{0pt}%
\pgfsys@defobject{currentmarker}{\pgfqpoint{-0.033333in}{-0.033333in}}{\pgfqpoint{0.033333in}{0.033333in}}{%
\pgfpathmoveto{\pgfqpoint{-0.000000in}{-0.033333in}}%
\pgfpathlineto{\pgfqpoint{0.033333in}{0.033333in}}%
\pgfpathlineto{\pgfqpoint{-0.033333in}{0.033333in}}%
\pgfpathlineto{\pgfqpoint{-0.000000in}{-0.033333in}}%
\pgfpathclose%
\pgfusepath{stroke,fill}%
}%
\begin{pgfscope}%
\pgfsys@transformshift{3.846694in}{2.901213in}%
\pgfsys@useobject{currentmarker}{}%
\end{pgfscope}%
\end{pgfscope}%
\begin{pgfscope}%
\pgfpathrectangle{\pgfqpoint{0.765000in}{0.660000in}}{\pgfqpoint{4.620000in}{4.620000in}}%
\pgfusepath{clip}%
\pgfsetrectcap%
\pgfsetroundjoin%
\pgfsetlinewidth{1.204500pt}%
\definecolor{currentstroke}{rgb}{0.176471,0.192157,0.258824}%
\pgfsetstrokecolor{currentstroke}%
\pgfsetdash{}{0pt}%
\pgfpathmoveto{\pgfqpoint{3.626640in}{3.505332in}}%
\pgfusepath{stroke}%
\end{pgfscope}%
\begin{pgfscope}%
\pgfpathrectangle{\pgfqpoint{0.765000in}{0.660000in}}{\pgfqpoint{4.620000in}{4.620000in}}%
\pgfusepath{clip}%
\pgfsetbuttcap%
\pgfsetmiterjoin%
\definecolor{currentfill}{rgb}{0.176471,0.192157,0.258824}%
\pgfsetfillcolor{currentfill}%
\pgfsetlinewidth{1.003750pt}%
\definecolor{currentstroke}{rgb}{0.176471,0.192157,0.258824}%
\pgfsetstrokecolor{currentstroke}%
\pgfsetdash{}{0pt}%
\pgfsys@defobject{currentmarker}{\pgfqpoint{-0.033333in}{-0.033333in}}{\pgfqpoint{0.033333in}{0.033333in}}{%
\pgfpathmoveto{\pgfqpoint{-0.000000in}{-0.033333in}}%
\pgfpathlineto{\pgfqpoint{0.033333in}{0.033333in}}%
\pgfpathlineto{\pgfqpoint{-0.033333in}{0.033333in}}%
\pgfpathlineto{\pgfqpoint{-0.000000in}{-0.033333in}}%
\pgfpathclose%
\pgfusepath{stroke,fill}%
}%
\begin{pgfscope}%
\pgfsys@transformshift{3.626640in}{3.505332in}%
\pgfsys@useobject{currentmarker}{}%
\end{pgfscope}%
\end{pgfscope}%
\begin{pgfscope}%
\pgfpathrectangle{\pgfqpoint{0.765000in}{0.660000in}}{\pgfqpoint{4.620000in}{4.620000in}}%
\pgfusepath{clip}%
\pgfsetrectcap%
\pgfsetroundjoin%
\pgfsetlinewidth{1.204500pt}%
\definecolor{currentstroke}{rgb}{0.176471,0.192157,0.258824}%
\pgfsetstrokecolor{currentstroke}%
\pgfsetdash{}{0pt}%
\pgfpathmoveto{\pgfqpoint{3.907771in}{3.206128in}}%
\pgfusepath{stroke}%
\end{pgfscope}%
\begin{pgfscope}%
\pgfpathrectangle{\pgfqpoint{0.765000in}{0.660000in}}{\pgfqpoint{4.620000in}{4.620000in}}%
\pgfusepath{clip}%
\pgfsetbuttcap%
\pgfsetmiterjoin%
\definecolor{currentfill}{rgb}{0.176471,0.192157,0.258824}%
\pgfsetfillcolor{currentfill}%
\pgfsetlinewidth{1.003750pt}%
\definecolor{currentstroke}{rgb}{0.176471,0.192157,0.258824}%
\pgfsetstrokecolor{currentstroke}%
\pgfsetdash{}{0pt}%
\pgfsys@defobject{currentmarker}{\pgfqpoint{-0.033333in}{-0.033333in}}{\pgfqpoint{0.033333in}{0.033333in}}{%
\pgfpathmoveto{\pgfqpoint{-0.000000in}{-0.033333in}}%
\pgfpathlineto{\pgfqpoint{0.033333in}{0.033333in}}%
\pgfpathlineto{\pgfqpoint{-0.033333in}{0.033333in}}%
\pgfpathlineto{\pgfqpoint{-0.000000in}{-0.033333in}}%
\pgfpathclose%
\pgfusepath{stroke,fill}%
}%
\begin{pgfscope}%
\pgfsys@transformshift{3.907771in}{3.206128in}%
\pgfsys@useobject{currentmarker}{}%
\end{pgfscope}%
\end{pgfscope}%
\begin{pgfscope}%
\pgfpathrectangle{\pgfqpoint{0.765000in}{0.660000in}}{\pgfqpoint{4.620000in}{4.620000in}}%
\pgfusepath{clip}%
\pgfsetrectcap%
\pgfsetroundjoin%
\pgfsetlinewidth{1.204500pt}%
\definecolor{currentstroke}{rgb}{0.176471,0.192157,0.258824}%
\pgfsetstrokecolor{currentstroke}%
\pgfsetdash{}{0pt}%
\pgfpathmoveto{\pgfqpoint{3.470588in}{3.321701in}}%
\pgfusepath{stroke}%
\end{pgfscope}%
\begin{pgfscope}%
\pgfpathrectangle{\pgfqpoint{0.765000in}{0.660000in}}{\pgfqpoint{4.620000in}{4.620000in}}%
\pgfusepath{clip}%
\pgfsetbuttcap%
\pgfsetmiterjoin%
\definecolor{currentfill}{rgb}{0.176471,0.192157,0.258824}%
\pgfsetfillcolor{currentfill}%
\pgfsetlinewidth{1.003750pt}%
\definecolor{currentstroke}{rgb}{0.176471,0.192157,0.258824}%
\pgfsetstrokecolor{currentstroke}%
\pgfsetdash{}{0pt}%
\pgfsys@defobject{currentmarker}{\pgfqpoint{-0.033333in}{-0.033333in}}{\pgfqpoint{0.033333in}{0.033333in}}{%
\pgfpathmoveto{\pgfqpoint{-0.000000in}{-0.033333in}}%
\pgfpathlineto{\pgfqpoint{0.033333in}{0.033333in}}%
\pgfpathlineto{\pgfqpoint{-0.033333in}{0.033333in}}%
\pgfpathlineto{\pgfqpoint{-0.000000in}{-0.033333in}}%
\pgfpathclose%
\pgfusepath{stroke,fill}%
}%
\begin{pgfscope}%
\pgfsys@transformshift{3.470588in}{3.321701in}%
\pgfsys@useobject{currentmarker}{}%
\end{pgfscope}%
\end{pgfscope}%
\begin{pgfscope}%
\pgfpathrectangle{\pgfqpoint{0.765000in}{0.660000in}}{\pgfqpoint{4.620000in}{4.620000in}}%
\pgfusepath{clip}%
\pgfsetrectcap%
\pgfsetroundjoin%
\pgfsetlinewidth{1.204500pt}%
\definecolor{currentstroke}{rgb}{0.176471,0.192157,0.258824}%
\pgfsetstrokecolor{currentstroke}%
\pgfsetdash{}{0pt}%
\pgfpathmoveto{\pgfqpoint{4.251402in}{3.479248in}}%
\pgfusepath{stroke}%
\end{pgfscope}%
\begin{pgfscope}%
\pgfpathrectangle{\pgfqpoint{0.765000in}{0.660000in}}{\pgfqpoint{4.620000in}{4.620000in}}%
\pgfusepath{clip}%
\pgfsetbuttcap%
\pgfsetmiterjoin%
\definecolor{currentfill}{rgb}{0.176471,0.192157,0.258824}%
\pgfsetfillcolor{currentfill}%
\pgfsetlinewidth{1.003750pt}%
\definecolor{currentstroke}{rgb}{0.176471,0.192157,0.258824}%
\pgfsetstrokecolor{currentstroke}%
\pgfsetdash{}{0pt}%
\pgfsys@defobject{currentmarker}{\pgfqpoint{-0.033333in}{-0.033333in}}{\pgfqpoint{0.033333in}{0.033333in}}{%
\pgfpathmoveto{\pgfqpoint{-0.000000in}{-0.033333in}}%
\pgfpathlineto{\pgfqpoint{0.033333in}{0.033333in}}%
\pgfpathlineto{\pgfqpoint{-0.033333in}{0.033333in}}%
\pgfpathlineto{\pgfqpoint{-0.000000in}{-0.033333in}}%
\pgfpathclose%
\pgfusepath{stroke,fill}%
}%
\begin{pgfscope}%
\pgfsys@transformshift{4.251402in}{3.479248in}%
\pgfsys@useobject{currentmarker}{}%
\end{pgfscope}%
\end{pgfscope}%
\begin{pgfscope}%
\pgfpathrectangle{\pgfqpoint{0.765000in}{0.660000in}}{\pgfqpoint{4.620000in}{4.620000in}}%
\pgfusepath{clip}%
\pgfsetrectcap%
\pgfsetroundjoin%
\pgfsetlinewidth{1.204500pt}%
\definecolor{currentstroke}{rgb}{0.176471,0.192157,0.258824}%
\pgfsetstrokecolor{currentstroke}%
\pgfsetdash{}{0pt}%
\pgfpathmoveto{\pgfqpoint{1.894935in}{3.618149in}}%
\pgfusepath{stroke}%
\end{pgfscope}%
\begin{pgfscope}%
\pgfpathrectangle{\pgfqpoint{0.765000in}{0.660000in}}{\pgfqpoint{4.620000in}{4.620000in}}%
\pgfusepath{clip}%
\pgfsetbuttcap%
\pgfsetmiterjoin%
\definecolor{currentfill}{rgb}{0.176471,0.192157,0.258824}%
\pgfsetfillcolor{currentfill}%
\pgfsetlinewidth{1.003750pt}%
\definecolor{currentstroke}{rgb}{0.176471,0.192157,0.258824}%
\pgfsetstrokecolor{currentstroke}%
\pgfsetdash{}{0pt}%
\pgfsys@defobject{currentmarker}{\pgfqpoint{-0.033333in}{-0.033333in}}{\pgfqpoint{0.033333in}{0.033333in}}{%
\pgfpathmoveto{\pgfqpoint{-0.000000in}{-0.033333in}}%
\pgfpathlineto{\pgfqpoint{0.033333in}{0.033333in}}%
\pgfpathlineto{\pgfqpoint{-0.033333in}{0.033333in}}%
\pgfpathlineto{\pgfqpoint{-0.000000in}{-0.033333in}}%
\pgfpathclose%
\pgfusepath{stroke,fill}%
}%
\begin{pgfscope}%
\pgfsys@transformshift{1.894935in}{3.618149in}%
\pgfsys@useobject{currentmarker}{}%
\end{pgfscope}%
\end{pgfscope}%
\begin{pgfscope}%
\pgfpathrectangle{\pgfqpoint{0.765000in}{0.660000in}}{\pgfqpoint{4.620000in}{4.620000in}}%
\pgfusepath{clip}%
\pgfsetrectcap%
\pgfsetroundjoin%
\pgfsetlinewidth{1.204500pt}%
\definecolor{currentstroke}{rgb}{0.000000,0.000000,0.000000}%
\pgfsetstrokecolor{currentstroke}%
\pgfsetdash{}{0pt}%
\pgfpathmoveto{\pgfqpoint{2.840481in}{3.244213in}}%
\pgfusepath{stroke}%
\end{pgfscope}%
\begin{pgfscope}%
\pgfpathrectangle{\pgfqpoint{0.765000in}{0.660000in}}{\pgfqpoint{4.620000in}{4.620000in}}%
\pgfusepath{clip}%
\pgfsetbuttcap%
\pgfsetroundjoin%
\definecolor{currentfill}{rgb}{0.000000,0.000000,0.000000}%
\pgfsetfillcolor{currentfill}%
\pgfsetlinewidth{1.003750pt}%
\definecolor{currentstroke}{rgb}{0.000000,0.000000,0.000000}%
\pgfsetstrokecolor{currentstroke}%
\pgfsetdash{}{0pt}%
\pgfsys@defobject{currentmarker}{\pgfqpoint{-0.016667in}{-0.016667in}}{\pgfqpoint{0.016667in}{0.016667in}}{%
\pgfpathmoveto{\pgfqpoint{0.000000in}{-0.016667in}}%
\pgfpathcurveto{\pgfqpoint{0.004420in}{-0.016667in}}{\pgfqpoint{0.008660in}{-0.014911in}}{\pgfqpoint{0.011785in}{-0.011785in}}%
\pgfpathcurveto{\pgfqpoint{0.014911in}{-0.008660in}}{\pgfqpoint{0.016667in}{-0.004420in}}{\pgfqpoint{0.016667in}{0.000000in}}%
\pgfpathcurveto{\pgfqpoint{0.016667in}{0.004420in}}{\pgfqpoint{0.014911in}{0.008660in}}{\pgfqpoint{0.011785in}{0.011785in}}%
\pgfpathcurveto{\pgfqpoint{0.008660in}{0.014911in}}{\pgfqpoint{0.004420in}{0.016667in}}{\pgfqpoint{0.000000in}{0.016667in}}%
\pgfpathcurveto{\pgfqpoint{-0.004420in}{0.016667in}}{\pgfqpoint{-0.008660in}{0.014911in}}{\pgfqpoint{-0.011785in}{0.011785in}}%
\pgfpathcurveto{\pgfqpoint{-0.014911in}{0.008660in}}{\pgfqpoint{-0.016667in}{0.004420in}}{\pgfqpoint{-0.016667in}{0.000000in}}%
\pgfpathcurveto{\pgfqpoint{-0.016667in}{-0.004420in}}{\pgfqpoint{-0.014911in}{-0.008660in}}{\pgfqpoint{-0.011785in}{-0.011785in}}%
\pgfpathcurveto{\pgfqpoint{-0.008660in}{-0.014911in}}{\pgfqpoint{-0.004420in}{-0.016667in}}{\pgfqpoint{0.000000in}{-0.016667in}}%
\pgfpathlineto{\pgfqpoint{0.000000in}{-0.016667in}}%
\pgfpathclose%
\pgfusepath{stroke,fill}%
}%
\begin{pgfscope}%
\pgfsys@transformshift{2.840481in}{3.244213in}%
\pgfsys@useobject{currentmarker}{}%
\end{pgfscope}%
\end{pgfscope}%
\begin{pgfscope}%
\pgfpathrectangle{\pgfqpoint{0.765000in}{0.660000in}}{\pgfqpoint{4.620000in}{4.620000in}}%
\pgfusepath{clip}%
\pgfsetrectcap%
\pgfsetroundjoin%
\pgfsetlinewidth{1.204500pt}%
\definecolor{currentstroke}{rgb}{0.000000,0.000000,0.000000}%
\pgfsetstrokecolor{currentstroke}%
\pgfsetdash{}{0pt}%
\pgfpathmoveto{\pgfqpoint{2.840481in}{3.244213in}}%
\pgfpathlineto{\pgfqpoint{3.665320in}{2.857644in}}%
\pgfusepath{stroke}%
\end{pgfscope}%
\begin{pgfscope}%
\pgfpathrectangle{\pgfqpoint{0.765000in}{0.660000in}}{\pgfqpoint{4.620000in}{4.620000in}}%
\pgfusepath{clip}%
\pgfsetbuttcap%
\pgfsetroundjoin%
\pgfsetlinewidth{1.204500pt}%
\definecolor{currentstroke}{rgb}{0.827451,0.827451,0.827451}%
\pgfsetstrokecolor{currentstroke}%
\pgfsetdash{{1.200000pt}{1.980000pt}}{0.000000pt}%
\pgfpathmoveto{\pgfqpoint{2.840481in}{3.244213in}}%
\pgfpathlineto{\pgfqpoint{2.840481in}{3.456103in}}%
\pgfusepath{stroke}%
\end{pgfscope}%
\begin{pgfscope}%
\pgfpathrectangle{\pgfqpoint{0.765000in}{0.660000in}}{\pgfqpoint{4.620000in}{4.620000in}}%
\pgfusepath{clip}%
\pgfsetrectcap%
\pgfsetroundjoin%
\pgfsetlinewidth{1.204500pt}%
\definecolor{currentstroke}{rgb}{0.000000,0.000000,0.000000}%
\pgfsetstrokecolor{currentstroke}%
\pgfsetdash{}{0pt}%
\pgfpathmoveto{\pgfqpoint{2.840481in}{3.244213in}}%
\pgfpathlineto{\pgfqpoint{2.144233in}{2.917910in}}%
\pgfusepath{stroke}%
\end{pgfscope}%
\begin{pgfscope}%
\pgfpathrectangle{\pgfqpoint{0.765000in}{0.660000in}}{\pgfqpoint{4.620000in}{4.620000in}}%
\pgfusepath{clip}%
\pgfsetrectcap%
\pgfsetroundjoin%
\pgfsetlinewidth{1.204500pt}%
\definecolor{currentstroke}{rgb}{0.000000,0.000000,0.000000}%
\pgfsetstrokecolor{currentstroke}%
\pgfsetdash{}{0pt}%
\pgfpathmoveto{\pgfqpoint{3.665320in}{2.857644in}}%
\pgfusepath{stroke}%
\end{pgfscope}%
\begin{pgfscope}%
\pgfpathrectangle{\pgfqpoint{0.765000in}{0.660000in}}{\pgfqpoint{4.620000in}{4.620000in}}%
\pgfusepath{clip}%
\pgfsetbuttcap%
\pgfsetroundjoin%
\definecolor{currentfill}{rgb}{0.000000,0.000000,0.000000}%
\pgfsetfillcolor{currentfill}%
\pgfsetlinewidth{1.003750pt}%
\definecolor{currentstroke}{rgb}{0.000000,0.000000,0.000000}%
\pgfsetstrokecolor{currentstroke}%
\pgfsetdash{}{0pt}%
\pgfsys@defobject{currentmarker}{\pgfqpoint{-0.016667in}{-0.016667in}}{\pgfqpoint{0.016667in}{0.016667in}}{%
\pgfpathmoveto{\pgfqpoint{0.000000in}{-0.016667in}}%
\pgfpathcurveto{\pgfqpoint{0.004420in}{-0.016667in}}{\pgfqpoint{0.008660in}{-0.014911in}}{\pgfqpoint{0.011785in}{-0.011785in}}%
\pgfpathcurveto{\pgfqpoint{0.014911in}{-0.008660in}}{\pgfqpoint{0.016667in}{-0.004420in}}{\pgfqpoint{0.016667in}{0.000000in}}%
\pgfpathcurveto{\pgfqpoint{0.016667in}{0.004420in}}{\pgfqpoint{0.014911in}{0.008660in}}{\pgfqpoint{0.011785in}{0.011785in}}%
\pgfpathcurveto{\pgfqpoint{0.008660in}{0.014911in}}{\pgfqpoint{0.004420in}{0.016667in}}{\pgfqpoint{0.000000in}{0.016667in}}%
\pgfpathcurveto{\pgfqpoint{-0.004420in}{0.016667in}}{\pgfqpoint{-0.008660in}{0.014911in}}{\pgfqpoint{-0.011785in}{0.011785in}}%
\pgfpathcurveto{\pgfqpoint{-0.014911in}{0.008660in}}{\pgfqpoint{-0.016667in}{0.004420in}}{\pgfqpoint{-0.016667in}{0.000000in}}%
\pgfpathcurveto{\pgfqpoint{-0.016667in}{-0.004420in}}{\pgfqpoint{-0.014911in}{-0.008660in}}{\pgfqpoint{-0.011785in}{-0.011785in}}%
\pgfpathcurveto{\pgfqpoint{-0.008660in}{-0.014911in}}{\pgfqpoint{-0.004420in}{-0.016667in}}{\pgfqpoint{0.000000in}{-0.016667in}}%
\pgfpathlineto{\pgfqpoint{0.000000in}{-0.016667in}}%
\pgfpathclose%
\pgfusepath{stroke,fill}%
}%
\begin{pgfscope}%
\pgfsys@transformshift{3.665320in}{2.857644in}%
\pgfsys@useobject{currentmarker}{}%
\end{pgfscope}%
\end{pgfscope}%
\begin{pgfscope}%
\pgfpathrectangle{\pgfqpoint{0.765000in}{0.660000in}}{\pgfqpoint{4.620000in}{4.620000in}}%
\pgfusepath{clip}%
\pgfsetrectcap%
\pgfsetroundjoin%
\pgfsetlinewidth{1.204500pt}%
\definecolor{currentstroke}{rgb}{0.000000,0.000000,0.000000}%
\pgfsetstrokecolor{currentstroke}%
\pgfsetdash{}{0pt}%
\pgfpathmoveto{\pgfqpoint{3.665320in}{2.857644in}}%
\pgfpathlineto{\pgfqpoint{2.840481in}{3.244213in}}%
\pgfusepath{stroke}%
\end{pgfscope}%
\begin{pgfscope}%
\pgfpathrectangle{\pgfqpoint{0.765000in}{0.660000in}}{\pgfqpoint{4.620000in}{4.620000in}}%
\pgfusepath{clip}%
\pgfsetrectcap%
\pgfsetroundjoin%
\pgfsetlinewidth{1.204500pt}%
\definecolor{currentstroke}{rgb}{0.000000,0.000000,0.000000}%
\pgfsetstrokecolor{currentstroke}%
\pgfsetdash{}{0pt}%
\pgfpathmoveto{\pgfqpoint{3.665320in}{2.857644in}}%
\pgfpathlineto{\pgfqpoint{3.846694in}{2.942647in}}%
\pgfusepath{stroke}%
\end{pgfscope}%
\begin{pgfscope}%
\pgfpathrectangle{\pgfqpoint{0.765000in}{0.660000in}}{\pgfqpoint{4.620000in}{4.620000in}}%
\pgfusepath{clip}%
\pgfsetbuttcap%
\pgfsetroundjoin%
\pgfsetlinewidth{1.204500pt}%
\definecolor{currentstroke}{rgb}{0.827451,0.827451,0.827451}%
\pgfsetstrokecolor{currentstroke}%
\pgfsetdash{{1.200000pt}{1.980000pt}}{0.000000pt}%
\pgfpathmoveto{\pgfqpoint{3.665320in}{2.857644in}}%
\pgfpathlineto{\pgfqpoint{3.665320in}{0.650000in}}%
\pgfusepath{stroke}%
\end{pgfscope}%
\begin{pgfscope}%
\pgfpathrectangle{\pgfqpoint{0.765000in}{0.660000in}}{\pgfqpoint{4.620000in}{4.620000in}}%
\pgfusepath{clip}%
\pgfsetrectcap%
\pgfsetroundjoin%
\pgfsetlinewidth{1.204500pt}%
\definecolor{currentstroke}{rgb}{0.000000,0.000000,0.000000}%
\pgfsetstrokecolor{currentstroke}%
\pgfsetdash{}{0pt}%
\pgfpathmoveto{\pgfqpoint{2.840481in}{3.456103in}}%
\pgfusepath{stroke}%
\end{pgfscope}%
\begin{pgfscope}%
\pgfpathrectangle{\pgfqpoint{0.765000in}{0.660000in}}{\pgfqpoint{4.620000in}{4.620000in}}%
\pgfusepath{clip}%
\pgfsetbuttcap%
\pgfsetroundjoin%
\definecolor{currentfill}{rgb}{0.000000,0.000000,0.000000}%
\pgfsetfillcolor{currentfill}%
\pgfsetlinewidth{1.003750pt}%
\definecolor{currentstroke}{rgb}{0.000000,0.000000,0.000000}%
\pgfsetstrokecolor{currentstroke}%
\pgfsetdash{}{0pt}%
\pgfsys@defobject{currentmarker}{\pgfqpoint{-0.016667in}{-0.016667in}}{\pgfqpoint{0.016667in}{0.016667in}}{%
\pgfpathmoveto{\pgfqpoint{0.000000in}{-0.016667in}}%
\pgfpathcurveto{\pgfqpoint{0.004420in}{-0.016667in}}{\pgfqpoint{0.008660in}{-0.014911in}}{\pgfqpoint{0.011785in}{-0.011785in}}%
\pgfpathcurveto{\pgfqpoint{0.014911in}{-0.008660in}}{\pgfqpoint{0.016667in}{-0.004420in}}{\pgfqpoint{0.016667in}{0.000000in}}%
\pgfpathcurveto{\pgfqpoint{0.016667in}{0.004420in}}{\pgfqpoint{0.014911in}{0.008660in}}{\pgfqpoint{0.011785in}{0.011785in}}%
\pgfpathcurveto{\pgfqpoint{0.008660in}{0.014911in}}{\pgfqpoint{0.004420in}{0.016667in}}{\pgfqpoint{0.000000in}{0.016667in}}%
\pgfpathcurveto{\pgfqpoint{-0.004420in}{0.016667in}}{\pgfqpoint{-0.008660in}{0.014911in}}{\pgfqpoint{-0.011785in}{0.011785in}}%
\pgfpathcurveto{\pgfqpoint{-0.014911in}{0.008660in}}{\pgfqpoint{-0.016667in}{0.004420in}}{\pgfqpoint{-0.016667in}{0.000000in}}%
\pgfpathcurveto{\pgfqpoint{-0.016667in}{-0.004420in}}{\pgfqpoint{-0.014911in}{-0.008660in}}{\pgfqpoint{-0.011785in}{-0.011785in}}%
\pgfpathcurveto{\pgfqpoint{-0.008660in}{-0.014911in}}{\pgfqpoint{-0.004420in}{-0.016667in}}{\pgfqpoint{0.000000in}{-0.016667in}}%
\pgfpathlineto{\pgfqpoint{0.000000in}{-0.016667in}}%
\pgfpathclose%
\pgfusepath{stroke,fill}%
}%
\begin{pgfscope}%
\pgfsys@transformshift{2.840481in}{3.456103in}%
\pgfsys@useobject{currentmarker}{}%
\end{pgfscope}%
\end{pgfscope}%
\begin{pgfscope}%
\pgfpathrectangle{\pgfqpoint{0.765000in}{0.660000in}}{\pgfqpoint{4.620000in}{4.620000in}}%
\pgfusepath{clip}%
\pgfsetbuttcap%
\pgfsetroundjoin%
\pgfsetlinewidth{1.204500pt}%
\definecolor{currentstroke}{rgb}{0.827451,0.827451,0.827451}%
\pgfsetstrokecolor{currentstroke}%
\pgfsetdash{{1.200000pt}{1.980000pt}}{0.000000pt}%
\pgfpathmoveto{\pgfqpoint{2.840481in}{3.456103in}}%
\pgfpathlineto{\pgfqpoint{2.840481in}{3.244213in}}%
\pgfusepath{stroke}%
\end{pgfscope}%
\begin{pgfscope}%
\pgfpathrectangle{\pgfqpoint{0.765000in}{0.660000in}}{\pgfqpoint{4.620000in}{4.620000in}}%
\pgfusepath{clip}%
\pgfsetbuttcap%
\pgfsetroundjoin%
\pgfsetlinewidth{1.204500pt}%
\definecolor{currentstroke}{rgb}{0.827451,0.827451,0.827451}%
\pgfsetstrokecolor{currentstroke}%
\pgfsetdash{{1.200000pt}{1.980000pt}}{0.000000pt}%
\pgfpathmoveto{\pgfqpoint{2.840481in}{3.456103in}}%
\pgfpathlineto{\pgfqpoint{3.047359in}{3.553059in}}%
\pgfusepath{stroke}%
\end{pgfscope}%
\begin{pgfscope}%
\pgfpathrectangle{\pgfqpoint{0.765000in}{0.660000in}}{\pgfqpoint{4.620000in}{4.620000in}}%
\pgfusepath{clip}%
\pgfsetbuttcap%
\pgfsetroundjoin%
\pgfsetlinewidth{1.204500pt}%
\definecolor{currentstroke}{rgb}{0.827451,0.827451,0.827451}%
\pgfsetstrokecolor{currentstroke}%
\pgfsetdash{{1.200000pt}{1.980000pt}}{0.000000pt}%
\pgfpathmoveto{\pgfqpoint{2.840481in}{3.456103in}}%
\pgfpathlineto{\pgfqpoint{3.047359in}{3.553061in}}%
\pgfusepath{stroke}%
\end{pgfscope}%
\begin{pgfscope}%
\pgfpathrectangle{\pgfqpoint{0.765000in}{0.660000in}}{\pgfqpoint{4.620000in}{4.620000in}}%
\pgfusepath{clip}%
\pgfsetbuttcap%
\pgfsetroundjoin%
\pgfsetlinewidth{1.204500pt}%
\definecolor{currentstroke}{rgb}{0.827451,0.827451,0.827451}%
\pgfsetstrokecolor{currentstroke}%
\pgfsetdash{{1.200000pt}{1.980000pt}}{0.000000pt}%
\pgfpathmoveto{\pgfqpoint{2.840481in}{3.456103in}}%
\pgfpathlineto{\pgfqpoint{5.395000in}{4.653303in}}%
\pgfusepath{stroke}%
\end{pgfscope}%
\begin{pgfscope}%
\pgfpathrectangle{\pgfqpoint{0.765000in}{0.660000in}}{\pgfqpoint{4.620000in}{4.620000in}}%
\pgfusepath{clip}%
\pgfsetbuttcap%
\pgfsetroundjoin%
\pgfsetlinewidth{1.204500pt}%
\definecolor{currentstroke}{rgb}{0.827451,0.827451,0.827451}%
\pgfsetstrokecolor{currentstroke}%
\pgfsetdash{{1.200000pt}{1.980000pt}}{0.000000pt}%
\pgfpathmoveto{\pgfqpoint{2.840481in}{3.456103in}}%
\pgfpathlineto{\pgfqpoint{0.755000in}{4.433484in}}%
\pgfusepath{stroke}%
\end{pgfscope}%
\begin{pgfscope}%
\pgfpathrectangle{\pgfqpoint{0.765000in}{0.660000in}}{\pgfqpoint{4.620000in}{4.620000in}}%
\pgfusepath{clip}%
\pgfsetrectcap%
\pgfsetroundjoin%
\pgfsetlinewidth{1.204500pt}%
\definecolor{currentstroke}{rgb}{0.000000,0.000000,0.000000}%
\pgfsetstrokecolor{currentstroke}%
\pgfsetdash{}{0pt}%
\pgfpathmoveto{\pgfqpoint{3.861519in}{3.171494in}}%
\pgfusepath{stroke}%
\end{pgfscope}%
\begin{pgfscope}%
\pgfpathrectangle{\pgfqpoint{0.765000in}{0.660000in}}{\pgfqpoint{4.620000in}{4.620000in}}%
\pgfusepath{clip}%
\pgfsetbuttcap%
\pgfsetroundjoin%
\definecolor{currentfill}{rgb}{0.000000,0.000000,0.000000}%
\pgfsetfillcolor{currentfill}%
\pgfsetlinewidth{1.003750pt}%
\definecolor{currentstroke}{rgb}{0.000000,0.000000,0.000000}%
\pgfsetstrokecolor{currentstroke}%
\pgfsetdash{}{0pt}%
\pgfsys@defobject{currentmarker}{\pgfqpoint{-0.016667in}{-0.016667in}}{\pgfqpoint{0.016667in}{0.016667in}}{%
\pgfpathmoveto{\pgfqpoint{0.000000in}{-0.016667in}}%
\pgfpathcurveto{\pgfqpoint{0.004420in}{-0.016667in}}{\pgfqpoint{0.008660in}{-0.014911in}}{\pgfqpoint{0.011785in}{-0.011785in}}%
\pgfpathcurveto{\pgfqpoint{0.014911in}{-0.008660in}}{\pgfqpoint{0.016667in}{-0.004420in}}{\pgfqpoint{0.016667in}{0.000000in}}%
\pgfpathcurveto{\pgfqpoint{0.016667in}{0.004420in}}{\pgfqpoint{0.014911in}{0.008660in}}{\pgfqpoint{0.011785in}{0.011785in}}%
\pgfpathcurveto{\pgfqpoint{0.008660in}{0.014911in}}{\pgfqpoint{0.004420in}{0.016667in}}{\pgfqpoint{0.000000in}{0.016667in}}%
\pgfpathcurveto{\pgfqpoint{-0.004420in}{0.016667in}}{\pgfqpoint{-0.008660in}{0.014911in}}{\pgfqpoint{-0.011785in}{0.011785in}}%
\pgfpathcurveto{\pgfqpoint{-0.014911in}{0.008660in}}{\pgfqpoint{-0.016667in}{0.004420in}}{\pgfqpoint{-0.016667in}{0.000000in}}%
\pgfpathcurveto{\pgfqpoint{-0.016667in}{-0.004420in}}{\pgfqpoint{-0.014911in}{-0.008660in}}{\pgfqpoint{-0.011785in}{-0.011785in}}%
\pgfpathcurveto{\pgfqpoint{-0.008660in}{-0.014911in}}{\pgfqpoint{-0.004420in}{-0.016667in}}{\pgfqpoint{0.000000in}{-0.016667in}}%
\pgfpathlineto{\pgfqpoint{0.000000in}{-0.016667in}}%
\pgfpathclose%
\pgfusepath{stroke,fill}%
}%
\begin{pgfscope}%
\pgfsys@transformshift{3.861519in}{3.171494in}%
\pgfsys@useobject{currentmarker}{}%
\end{pgfscope}%
\end{pgfscope}%
\begin{pgfscope}%
\pgfpathrectangle{\pgfqpoint{0.765000in}{0.660000in}}{\pgfqpoint{4.620000in}{4.620000in}}%
\pgfusepath{clip}%
\pgfsetrectcap%
\pgfsetroundjoin%
\pgfsetlinewidth{1.204500pt}%
\definecolor{currentstroke}{rgb}{0.000000,0.000000,0.000000}%
\pgfsetstrokecolor{currentstroke}%
\pgfsetdash{}{0pt}%
\pgfpathmoveto{\pgfqpoint{3.861519in}{3.171494in}}%
\pgfpathlineto{\pgfqpoint{3.861519in}{2.963491in}}%
\pgfusepath{stroke}%
\end{pgfscope}%
\begin{pgfscope}%
\pgfpathrectangle{\pgfqpoint{0.765000in}{0.660000in}}{\pgfqpoint{4.620000in}{4.620000in}}%
\pgfusepath{clip}%
\pgfsetbuttcap%
\pgfsetroundjoin%
\pgfsetlinewidth{1.204500pt}%
\definecolor{currentstroke}{rgb}{0.827451,0.827451,0.827451}%
\pgfsetstrokecolor{currentstroke}%
\pgfsetdash{{1.200000pt}{1.980000pt}}{0.000000pt}%
\pgfpathmoveto{\pgfqpoint{3.861519in}{3.171494in}}%
\pgfpathlineto{\pgfqpoint{3.047359in}{3.553059in}}%
\pgfusepath{stroke}%
\end{pgfscope}%
\begin{pgfscope}%
\pgfpathrectangle{\pgfqpoint{0.765000in}{0.660000in}}{\pgfqpoint{4.620000in}{4.620000in}}%
\pgfusepath{clip}%
\pgfsetbuttcap%
\pgfsetroundjoin%
\pgfsetlinewidth{1.204500pt}%
\definecolor{currentstroke}{rgb}{0.827451,0.827451,0.827451}%
\pgfsetstrokecolor{currentstroke}%
\pgfsetdash{{1.200000pt}{1.980000pt}}{0.000000pt}%
\pgfpathmoveto{\pgfqpoint{3.861519in}{3.171494in}}%
\pgfpathlineto{\pgfqpoint{3.047359in}{3.553061in}}%
\pgfusepath{stroke}%
\end{pgfscope}%
\begin{pgfscope}%
\pgfpathrectangle{\pgfqpoint{0.765000in}{0.660000in}}{\pgfqpoint{4.620000in}{4.620000in}}%
\pgfusepath{clip}%
\pgfsetbuttcap%
\pgfsetroundjoin%
\pgfsetlinewidth{1.204500pt}%
\definecolor{currentstroke}{rgb}{0.827451,0.827451,0.827451}%
\pgfsetstrokecolor{currentstroke}%
\pgfsetdash{{1.200000pt}{1.980000pt}}{0.000000pt}%
\pgfpathmoveto{\pgfqpoint{3.861519in}{3.171494in}}%
\pgfpathlineto{\pgfqpoint{0.755000in}{4.627395in}}%
\pgfusepath{stroke}%
\end{pgfscope}%
\begin{pgfscope}%
\pgfpathrectangle{\pgfqpoint{0.765000in}{0.660000in}}{\pgfqpoint{4.620000in}{4.620000in}}%
\pgfusepath{clip}%
\pgfsetrectcap%
\pgfsetroundjoin%
\pgfsetlinewidth{1.204500pt}%
\definecolor{currentstroke}{rgb}{0.000000,0.000000,0.000000}%
\pgfsetstrokecolor{currentstroke}%
\pgfsetdash{}{0pt}%
\pgfpathmoveto{\pgfqpoint{3.861519in}{3.171494in}}%
\pgfpathlineto{\pgfqpoint{5.395000in}{3.890175in}}%
\pgfusepath{stroke}%
\end{pgfscope}%
\begin{pgfscope}%
\pgfpathrectangle{\pgfqpoint{0.765000in}{0.660000in}}{\pgfqpoint{4.620000in}{4.620000in}}%
\pgfusepath{clip}%
\pgfsetrectcap%
\pgfsetroundjoin%
\pgfsetlinewidth{1.204500pt}%
\definecolor{currentstroke}{rgb}{0.000000,0.000000,0.000000}%
\pgfsetstrokecolor{currentstroke}%
\pgfsetdash{}{0pt}%
\pgfpathmoveto{\pgfqpoint{3.861519in}{2.963491in}}%
\pgfusepath{stroke}%
\end{pgfscope}%
\begin{pgfscope}%
\pgfpathrectangle{\pgfqpoint{0.765000in}{0.660000in}}{\pgfqpoint{4.620000in}{4.620000in}}%
\pgfusepath{clip}%
\pgfsetbuttcap%
\pgfsetroundjoin%
\definecolor{currentfill}{rgb}{0.000000,0.000000,0.000000}%
\pgfsetfillcolor{currentfill}%
\pgfsetlinewidth{1.003750pt}%
\definecolor{currentstroke}{rgb}{0.000000,0.000000,0.000000}%
\pgfsetstrokecolor{currentstroke}%
\pgfsetdash{}{0pt}%
\pgfsys@defobject{currentmarker}{\pgfqpoint{-0.016667in}{-0.016667in}}{\pgfqpoint{0.016667in}{0.016667in}}{%
\pgfpathmoveto{\pgfqpoint{0.000000in}{-0.016667in}}%
\pgfpathcurveto{\pgfqpoint{0.004420in}{-0.016667in}}{\pgfqpoint{0.008660in}{-0.014911in}}{\pgfqpoint{0.011785in}{-0.011785in}}%
\pgfpathcurveto{\pgfqpoint{0.014911in}{-0.008660in}}{\pgfqpoint{0.016667in}{-0.004420in}}{\pgfqpoint{0.016667in}{0.000000in}}%
\pgfpathcurveto{\pgfqpoint{0.016667in}{0.004420in}}{\pgfqpoint{0.014911in}{0.008660in}}{\pgfqpoint{0.011785in}{0.011785in}}%
\pgfpathcurveto{\pgfqpoint{0.008660in}{0.014911in}}{\pgfqpoint{0.004420in}{0.016667in}}{\pgfqpoint{0.000000in}{0.016667in}}%
\pgfpathcurveto{\pgfqpoint{-0.004420in}{0.016667in}}{\pgfqpoint{-0.008660in}{0.014911in}}{\pgfqpoint{-0.011785in}{0.011785in}}%
\pgfpathcurveto{\pgfqpoint{-0.014911in}{0.008660in}}{\pgfqpoint{-0.016667in}{0.004420in}}{\pgfqpoint{-0.016667in}{0.000000in}}%
\pgfpathcurveto{\pgfqpoint{-0.016667in}{-0.004420in}}{\pgfqpoint{-0.014911in}{-0.008660in}}{\pgfqpoint{-0.011785in}{-0.011785in}}%
\pgfpathcurveto{\pgfqpoint{-0.008660in}{-0.014911in}}{\pgfqpoint{-0.004420in}{-0.016667in}}{\pgfqpoint{0.000000in}{-0.016667in}}%
\pgfpathlineto{\pgfqpoint{0.000000in}{-0.016667in}}%
\pgfpathclose%
\pgfusepath{stroke,fill}%
}%
\begin{pgfscope}%
\pgfsys@transformshift{3.861519in}{2.963491in}%
\pgfsys@useobject{currentmarker}{}%
\end{pgfscope}%
\end{pgfscope}%
\begin{pgfscope}%
\pgfpathrectangle{\pgfqpoint{0.765000in}{0.660000in}}{\pgfqpoint{4.620000in}{4.620000in}}%
\pgfusepath{clip}%
\pgfsetrectcap%
\pgfsetroundjoin%
\pgfsetlinewidth{1.204500pt}%
\definecolor{currentstroke}{rgb}{0.000000,0.000000,0.000000}%
\pgfsetstrokecolor{currentstroke}%
\pgfsetdash{}{0pt}%
\pgfpathmoveto{\pgfqpoint{3.861519in}{2.963491in}}%
\pgfpathlineto{\pgfqpoint{3.861519in}{3.171494in}}%
\pgfusepath{stroke}%
\end{pgfscope}%
\begin{pgfscope}%
\pgfpathrectangle{\pgfqpoint{0.765000in}{0.660000in}}{\pgfqpoint{4.620000in}{4.620000in}}%
\pgfusepath{clip}%
\pgfsetrectcap%
\pgfsetroundjoin%
\pgfsetlinewidth{1.204500pt}%
\definecolor{currentstroke}{rgb}{0.000000,0.000000,0.000000}%
\pgfsetstrokecolor{currentstroke}%
\pgfsetdash{}{0pt}%
\pgfpathmoveto{\pgfqpoint{3.861519in}{2.963491in}}%
\pgfpathlineto{\pgfqpoint{3.846694in}{2.942647in}}%
\pgfusepath{stroke}%
\end{pgfscope}%
\begin{pgfscope}%
\pgfpathrectangle{\pgfqpoint{0.765000in}{0.660000in}}{\pgfqpoint{4.620000in}{4.620000in}}%
\pgfusepath{clip}%
\pgfsetbuttcap%
\pgfsetroundjoin%
\pgfsetlinewidth{1.204500pt}%
\definecolor{currentstroke}{rgb}{0.827451,0.827451,0.827451}%
\pgfsetstrokecolor{currentstroke}%
\pgfsetdash{{1.200000pt}{1.980000pt}}{0.000000pt}%
\pgfpathmoveto{\pgfqpoint{3.861519in}{2.963491in}}%
\pgfpathlineto{\pgfqpoint{5.395000in}{3.682172in}}%
\pgfusepath{stroke}%
\end{pgfscope}%
\begin{pgfscope}%
\pgfpathrectangle{\pgfqpoint{0.765000in}{0.660000in}}{\pgfqpoint{4.620000in}{4.620000in}}%
\pgfusepath{clip}%
\pgfsetrectcap%
\pgfsetroundjoin%
\pgfsetlinewidth{1.204500pt}%
\definecolor{currentstroke}{rgb}{0.000000,0.000000,0.000000}%
\pgfsetstrokecolor{currentstroke}%
\pgfsetdash{}{0pt}%
\pgfpathmoveto{\pgfqpoint{3.846694in}{2.942647in}}%
\pgfusepath{stroke}%
\end{pgfscope}%
\begin{pgfscope}%
\pgfpathrectangle{\pgfqpoint{0.765000in}{0.660000in}}{\pgfqpoint{4.620000in}{4.620000in}}%
\pgfusepath{clip}%
\pgfsetbuttcap%
\pgfsetroundjoin%
\definecolor{currentfill}{rgb}{0.000000,0.000000,0.000000}%
\pgfsetfillcolor{currentfill}%
\pgfsetlinewidth{1.003750pt}%
\definecolor{currentstroke}{rgb}{0.000000,0.000000,0.000000}%
\pgfsetstrokecolor{currentstroke}%
\pgfsetdash{}{0pt}%
\pgfsys@defobject{currentmarker}{\pgfqpoint{-0.016667in}{-0.016667in}}{\pgfqpoint{0.016667in}{0.016667in}}{%
\pgfpathmoveto{\pgfqpoint{0.000000in}{-0.016667in}}%
\pgfpathcurveto{\pgfqpoint{0.004420in}{-0.016667in}}{\pgfqpoint{0.008660in}{-0.014911in}}{\pgfqpoint{0.011785in}{-0.011785in}}%
\pgfpathcurveto{\pgfqpoint{0.014911in}{-0.008660in}}{\pgfqpoint{0.016667in}{-0.004420in}}{\pgfqpoint{0.016667in}{0.000000in}}%
\pgfpathcurveto{\pgfqpoint{0.016667in}{0.004420in}}{\pgfqpoint{0.014911in}{0.008660in}}{\pgfqpoint{0.011785in}{0.011785in}}%
\pgfpathcurveto{\pgfqpoint{0.008660in}{0.014911in}}{\pgfqpoint{0.004420in}{0.016667in}}{\pgfqpoint{0.000000in}{0.016667in}}%
\pgfpathcurveto{\pgfqpoint{-0.004420in}{0.016667in}}{\pgfqpoint{-0.008660in}{0.014911in}}{\pgfqpoint{-0.011785in}{0.011785in}}%
\pgfpathcurveto{\pgfqpoint{-0.014911in}{0.008660in}}{\pgfqpoint{-0.016667in}{0.004420in}}{\pgfqpoint{-0.016667in}{0.000000in}}%
\pgfpathcurveto{\pgfqpoint{-0.016667in}{-0.004420in}}{\pgfqpoint{-0.014911in}{-0.008660in}}{\pgfqpoint{-0.011785in}{-0.011785in}}%
\pgfpathcurveto{\pgfqpoint{-0.008660in}{-0.014911in}}{\pgfqpoint{-0.004420in}{-0.016667in}}{\pgfqpoint{0.000000in}{-0.016667in}}%
\pgfpathlineto{\pgfqpoint{0.000000in}{-0.016667in}}%
\pgfpathclose%
\pgfusepath{stroke,fill}%
}%
\begin{pgfscope}%
\pgfsys@transformshift{3.846694in}{2.942647in}%
\pgfsys@useobject{currentmarker}{}%
\end{pgfscope}%
\end{pgfscope}%
\begin{pgfscope}%
\pgfpathrectangle{\pgfqpoint{0.765000in}{0.660000in}}{\pgfqpoint{4.620000in}{4.620000in}}%
\pgfusepath{clip}%
\pgfsetrectcap%
\pgfsetroundjoin%
\pgfsetlinewidth{1.204500pt}%
\definecolor{currentstroke}{rgb}{0.000000,0.000000,0.000000}%
\pgfsetstrokecolor{currentstroke}%
\pgfsetdash{}{0pt}%
\pgfpathmoveto{\pgfqpoint{3.846694in}{2.942647in}}%
\pgfpathlineto{\pgfqpoint{3.665320in}{2.857644in}}%
\pgfusepath{stroke}%
\end{pgfscope}%
\begin{pgfscope}%
\pgfpathrectangle{\pgfqpoint{0.765000in}{0.660000in}}{\pgfqpoint{4.620000in}{4.620000in}}%
\pgfusepath{clip}%
\pgfsetrectcap%
\pgfsetroundjoin%
\pgfsetlinewidth{1.204500pt}%
\definecolor{currentstroke}{rgb}{0.000000,0.000000,0.000000}%
\pgfsetstrokecolor{currentstroke}%
\pgfsetdash{}{0pt}%
\pgfpathmoveto{\pgfqpoint{3.846694in}{2.942647in}}%
\pgfpathlineto{\pgfqpoint{3.861519in}{2.963491in}}%
\pgfusepath{stroke}%
\end{pgfscope}%
\begin{pgfscope}%
\pgfpathrectangle{\pgfqpoint{0.765000in}{0.660000in}}{\pgfqpoint{4.620000in}{4.620000in}}%
\pgfusepath{clip}%
\pgfsetbuttcap%
\pgfsetroundjoin%
\pgfsetlinewidth{1.204500pt}%
\definecolor{currentstroke}{rgb}{0.827451,0.827451,0.827451}%
\pgfsetstrokecolor{currentstroke}%
\pgfsetdash{{1.200000pt}{1.980000pt}}{0.000000pt}%
\pgfpathmoveto{\pgfqpoint{3.846694in}{2.942647in}}%
\pgfpathlineto{\pgfqpoint{3.846694in}{0.650000in}}%
\pgfusepath{stroke}%
\end{pgfscope}%
\begin{pgfscope}%
\pgfpathrectangle{\pgfqpoint{0.765000in}{0.660000in}}{\pgfqpoint{4.620000in}{4.620000in}}%
\pgfusepath{clip}%
\pgfsetrectcap%
\pgfsetroundjoin%
\pgfsetlinewidth{1.204500pt}%
\definecolor{currentstroke}{rgb}{0.000000,0.000000,0.000000}%
\pgfsetstrokecolor{currentstroke}%
\pgfsetdash{}{0pt}%
\pgfpathmoveto{\pgfqpoint{3.047359in}{3.553059in}}%
\pgfusepath{stroke}%
\end{pgfscope}%
\begin{pgfscope}%
\pgfpathrectangle{\pgfqpoint{0.765000in}{0.660000in}}{\pgfqpoint{4.620000in}{4.620000in}}%
\pgfusepath{clip}%
\pgfsetbuttcap%
\pgfsetroundjoin%
\definecolor{currentfill}{rgb}{0.000000,0.000000,0.000000}%
\pgfsetfillcolor{currentfill}%
\pgfsetlinewidth{1.003750pt}%
\definecolor{currentstroke}{rgb}{0.000000,0.000000,0.000000}%
\pgfsetstrokecolor{currentstroke}%
\pgfsetdash{}{0pt}%
\pgfsys@defobject{currentmarker}{\pgfqpoint{-0.016667in}{-0.016667in}}{\pgfqpoint{0.016667in}{0.016667in}}{%
\pgfpathmoveto{\pgfqpoint{0.000000in}{-0.016667in}}%
\pgfpathcurveto{\pgfqpoint{0.004420in}{-0.016667in}}{\pgfqpoint{0.008660in}{-0.014911in}}{\pgfqpoint{0.011785in}{-0.011785in}}%
\pgfpathcurveto{\pgfqpoint{0.014911in}{-0.008660in}}{\pgfqpoint{0.016667in}{-0.004420in}}{\pgfqpoint{0.016667in}{0.000000in}}%
\pgfpathcurveto{\pgfqpoint{0.016667in}{0.004420in}}{\pgfqpoint{0.014911in}{0.008660in}}{\pgfqpoint{0.011785in}{0.011785in}}%
\pgfpathcurveto{\pgfqpoint{0.008660in}{0.014911in}}{\pgfqpoint{0.004420in}{0.016667in}}{\pgfqpoint{0.000000in}{0.016667in}}%
\pgfpathcurveto{\pgfqpoint{-0.004420in}{0.016667in}}{\pgfqpoint{-0.008660in}{0.014911in}}{\pgfqpoint{-0.011785in}{0.011785in}}%
\pgfpathcurveto{\pgfqpoint{-0.014911in}{0.008660in}}{\pgfqpoint{-0.016667in}{0.004420in}}{\pgfqpoint{-0.016667in}{0.000000in}}%
\pgfpathcurveto{\pgfqpoint{-0.016667in}{-0.004420in}}{\pgfqpoint{-0.014911in}{-0.008660in}}{\pgfqpoint{-0.011785in}{-0.011785in}}%
\pgfpathcurveto{\pgfqpoint{-0.008660in}{-0.014911in}}{\pgfqpoint{-0.004420in}{-0.016667in}}{\pgfqpoint{0.000000in}{-0.016667in}}%
\pgfpathlineto{\pgfqpoint{0.000000in}{-0.016667in}}%
\pgfpathclose%
\pgfusepath{stroke,fill}%
}%
\begin{pgfscope}%
\pgfsys@transformshift{3.047359in}{3.553059in}%
\pgfsys@useobject{currentmarker}{}%
\end{pgfscope}%
\end{pgfscope}%
\begin{pgfscope}%
\pgfpathrectangle{\pgfqpoint{0.765000in}{0.660000in}}{\pgfqpoint{4.620000in}{4.620000in}}%
\pgfusepath{clip}%
\pgfsetbuttcap%
\pgfsetroundjoin%
\pgfsetlinewidth{1.204500pt}%
\definecolor{currentstroke}{rgb}{0.827451,0.827451,0.827451}%
\pgfsetstrokecolor{currentstroke}%
\pgfsetdash{{1.200000pt}{1.980000pt}}{0.000000pt}%
\pgfpathmoveto{\pgfqpoint{3.047359in}{3.553059in}}%
\pgfpathlineto{\pgfqpoint{2.840481in}{3.456103in}}%
\pgfusepath{stroke}%
\end{pgfscope}%
\begin{pgfscope}%
\pgfpathrectangle{\pgfqpoint{0.765000in}{0.660000in}}{\pgfqpoint{4.620000in}{4.620000in}}%
\pgfusepath{clip}%
\pgfsetbuttcap%
\pgfsetroundjoin%
\pgfsetlinewidth{1.204500pt}%
\definecolor{currentstroke}{rgb}{0.827451,0.827451,0.827451}%
\pgfsetstrokecolor{currentstroke}%
\pgfsetdash{{1.200000pt}{1.980000pt}}{0.000000pt}%
\pgfpathmoveto{\pgfqpoint{3.047359in}{3.553059in}}%
\pgfpathlineto{\pgfqpoint{3.861519in}{3.171494in}}%
\pgfusepath{stroke}%
\end{pgfscope}%
\begin{pgfscope}%
\pgfpathrectangle{\pgfqpoint{0.765000in}{0.660000in}}{\pgfqpoint{4.620000in}{4.620000in}}%
\pgfusepath{clip}%
\pgfsetbuttcap%
\pgfsetroundjoin%
\pgfsetlinewidth{1.204500pt}%
\definecolor{currentstroke}{rgb}{0.827451,0.827451,0.827451}%
\pgfsetstrokecolor{currentstroke}%
\pgfsetdash{{1.200000pt}{1.980000pt}}{0.000000pt}%
\pgfpathmoveto{\pgfqpoint{3.047359in}{3.553059in}}%
\pgfpathlineto{\pgfqpoint{3.047359in}{3.553061in}}%
\pgfusepath{stroke}%
\end{pgfscope}%
\begin{pgfscope}%
\pgfpathrectangle{\pgfqpoint{0.765000in}{0.660000in}}{\pgfqpoint{4.620000in}{4.620000in}}%
\pgfusepath{clip}%
\pgfsetbuttcap%
\pgfsetroundjoin%
\pgfsetlinewidth{1.204500pt}%
\definecolor{currentstroke}{rgb}{0.827451,0.827451,0.827451}%
\pgfsetstrokecolor{currentstroke}%
\pgfsetdash{{1.200000pt}{1.980000pt}}{0.000000pt}%
\pgfpathmoveto{\pgfqpoint{3.047359in}{3.553059in}}%
\pgfpathlineto{\pgfqpoint{5.395000in}{4.653303in}}%
\pgfusepath{stroke}%
\end{pgfscope}%
\begin{pgfscope}%
\pgfpathrectangle{\pgfqpoint{0.765000in}{0.660000in}}{\pgfqpoint{4.620000in}{4.620000in}}%
\pgfusepath{clip}%
\pgfsetbuttcap%
\pgfsetroundjoin%
\pgfsetlinewidth{1.204500pt}%
\definecolor{currentstroke}{rgb}{0.827451,0.827451,0.827451}%
\pgfsetstrokecolor{currentstroke}%
\pgfsetdash{{1.200000pt}{1.980000pt}}{0.000000pt}%
\pgfpathmoveto{\pgfqpoint{3.047359in}{3.553059in}}%
\pgfpathlineto{\pgfqpoint{0.755000in}{4.627395in}}%
\pgfusepath{stroke}%
\end{pgfscope}%
\begin{pgfscope}%
\pgfpathrectangle{\pgfqpoint{0.765000in}{0.660000in}}{\pgfqpoint{4.620000in}{4.620000in}}%
\pgfusepath{clip}%
\pgfsetrectcap%
\pgfsetroundjoin%
\pgfsetlinewidth{1.204500pt}%
\definecolor{currentstroke}{rgb}{0.000000,0.000000,0.000000}%
\pgfsetstrokecolor{currentstroke}%
\pgfsetdash{}{0pt}%
\pgfpathmoveto{\pgfqpoint{2.144233in}{2.917910in}}%
\pgfusepath{stroke}%
\end{pgfscope}%
\begin{pgfscope}%
\pgfpathrectangle{\pgfqpoint{0.765000in}{0.660000in}}{\pgfqpoint{4.620000in}{4.620000in}}%
\pgfusepath{clip}%
\pgfsetbuttcap%
\pgfsetroundjoin%
\definecolor{currentfill}{rgb}{0.000000,0.000000,0.000000}%
\pgfsetfillcolor{currentfill}%
\pgfsetlinewidth{1.003750pt}%
\definecolor{currentstroke}{rgb}{0.000000,0.000000,0.000000}%
\pgfsetstrokecolor{currentstroke}%
\pgfsetdash{}{0pt}%
\pgfsys@defobject{currentmarker}{\pgfqpoint{-0.016667in}{-0.016667in}}{\pgfqpoint{0.016667in}{0.016667in}}{%
\pgfpathmoveto{\pgfqpoint{0.000000in}{-0.016667in}}%
\pgfpathcurveto{\pgfqpoint{0.004420in}{-0.016667in}}{\pgfqpoint{0.008660in}{-0.014911in}}{\pgfqpoint{0.011785in}{-0.011785in}}%
\pgfpathcurveto{\pgfqpoint{0.014911in}{-0.008660in}}{\pgfqpoint{0.016667in}{-0.004420in}}{\pgfqpoint{0.016667in}{0.000000in}}%
\pgfpathcurveto{\pgfqpoint{0.016667in}{0.004420in}}{\pgfqpoint{0.014911in}{0.008660in}}{\pgfqpoint{0.011785in}{0.011785in}}%
\pgfpathcurveto{\pgfqpoint{0.008660in}{0.014911in}}{\pgfqpoint{0.004420in}{0.016667in}}{\pgfqpoint{0.000000in}{0.016667in}}%
\pgfpathcurveto{\pgfqpoint{-0.004420in}{0.016667in}}{\pgfqpoint{-0.008660in}{0.014911in}}{\pgfqpoint{-0.011785in}{0.011785in}}%
\pgfpathcurveto{\pgfqpoint{-0.014911in}{0.008660in}}{\pgfqpoint{-0.016667in}{0.004420in}}{\pgfqpoint{-0.016667in}{0.000000in}}%
\pgfpathcurveto{\pgfqpoint{-0.016667in}{-0.004420in}}{\pgfqpoint{-0.014911in}{-0.008660in}}{\pgfqpoint{-0.011785in}{-0.011785in}}%
\pgfpathcurveto{\pgfqpoint{-0.008660in}{-0.014911in}}{\pgfqpoint{-0.004420in}{-0.016667in}}{\pgfqpoint{0.000000in}{-0.016667in}}%
\pgfpathlineto{\pgfqpoint{0.000000in}{-0.016667in}}%
\pgfpathclose%
\pgfusepath{stroke,fill}%
}%
\begin{pgfscope}%
\pgfsys@transformshift{2.144233in}{2.917910in}%
\pgfsys@useobject{currentmarker}{}%
\end{pgfscope}%
\end{pgfscope}%
\begin{pgfscope}%
\pgfpathrectangle{\pgfqpoint{0.765000in}{0.660000in}}{\pgfqpoint{4.620000in}{4.620000in}}%
\pgfusepath{clip}%
\pgfsetrectcap%
\pgfsetroundjoin%
\pgfsetlinewidth{1.204500pt}%
\definecolor{currentstroke}{rgb}{0.000000,0.000000,0.000000}%
\pgfsetstrokecolor{currentstroke}%
\pgfsetdash{}{0pt}%
\pgfpathmoveto{\pgfqpoint{2.144233in}{2.917910in}}%
\pgfpathlineto{\pgfqpoint{2.840481in}{3.244213in}}%
\pgfusepath{stroke}%
\end{pgfscope}%
\begin{pgfscope}%
\pgfpathrectangle{\pgfqpoint{0.765000in}{0.660000in}}{\pgfqpoint{4.620000in}{4.620000in}}%
\pgfusepath{clip}%
\pgfsetrectcap%
\pgfsetroundjoin%
\pgfsetlinewidth{1.204500pt}%
\definecolor{currentstroke}{rgb}{0.000000,0.000000,0.000000}%
\pgfsetstrokecolor{currentstroke}%
\pgfsetdash{}{0pt}%
\pgfpathmoveto{\pgfqpoint{2.144233in}{2.917910in}}%
\pgfpathlineto{\pgfqpoint{2.144233in}{0.650000in}}%
\pgfusepath{stroke}%
\end{pgfscope}%
\begin{pgfscope}%
\pgfpathrectangle{\pgfqpoint{0.765000in}{0.660000in}}{\pgfqpoint{4.620000in}{4.620000in}}%
\pgfusepath{clip}%
\pgfsetbuttcap%
\pgfsetroundjoin%
\pgfsetlinewidth{1.204500pt}%
\definecolor{currentstroke}{rgb}{0.827451,0.827451,0.827451}%
\pgfsetstrokecolor{currentstroke}%
\pgfsetdash{{1.200000pt}{1.980000pt}}{0.000000pt}%
\pgfpathmoveto{\pgfqpoint{2.144233in}{2.917910in}}%
\pgfpathlineto{\pgfqpoint{0.755000in}{3.568987in}}%
\pgfusepath{stroke}%
\end{pgfscope}%
\begin{pgfscope}%
\pgfpathrectangle{\pgfqpoint{0.765000in}{0.660000in}}{\pgfqpoint{4.620000in}{4.620000in}}%
\pgfusepath{clip}%
\pgfsetrectcap%
\pgfsetroundjoin%
\pgfsetlinewidth{1.204500pt}%
\definecolor{currentstroke}{rgb}{0.000000,0.000000,0.000000}%
\pgfsetstrokecolor{currentstroke}%
\pgfsetdash{}{0pt}%
\pgfpathmoveto{\pgfqpoint{3.047359in}{3.553061in}}%
\pgfusepath{stroke}%
\end{pgfscope}%
\begin{pgfscope}%
\pgfpathrectangle{\pgfqpoint{0.765000in}{0.660000in}}{\pgfqpoint{4.620000in}{4.620000in}}%
\pgfusepath{clip}%
\pgfsetbuttcap%
\pgfsetroundjoin%
\definecolor{currentfill}{rgb}{0.000000,0.000000,0.000000}%
\pgfsetfillcolor{currentfill}%
\pgfsetlinewidth{1.003750pt}%
\definecolor{currentstroke}{rgb}{0.000000,0.000000,0.000000}%
\pgfsetstrokecolor{currentstroke}%
\pgfsetdash{}{0pt}%
\pgfsys@defobject{currentmarker}{\pgfqpoint{-0.016667in}{-0.016667in}}{\pgfqpoint{0.016667in}{0.016667in}}{%
\pgfpathmoveto{\pgfqpoint{0.000000in}{-0.016667in}}%
\pgfpathcurveto{\pgfqpoint{0.004420in}{-0.016667in}}{\pgfqpoint{0.008660in}{-0.014911in}}{\pgfqpoint{0.011785in}{-0.011785in}}%
\pgfpathcurveto{\pgfqpoint{0.014911in}{-0.008660in}}{\pgfqpoint{0.016667in}{-0.004420in}}{\pgfqpoint{0.016667in}{0.000000in}}%
\pgfpathcurveto{\pgfqpoint{0.016667in}{0.004420in}}{\pgfqpoint{0.014911in}{0.008660in}}{\pgfqpoint{0.011785in}{0.011785in}}%
\pgfpathcurveto{\pgfqpoint{0.008660in}{0.014911in}}{\pgfqpoint{0.004420in}{0.016667in}}{\pgfqpoint{0.000000in}{0.016667in}}%
\pgfpathcurveto{\pgfqpoint{-0.004420in}{0.016667in}}{\pgfqpoint{-0.008660in}{0.014911in}}{\pgfqpoint{-0.011785in}{0.011785in}}%
\pgfpathcurveto{\pgfqpoint{-0.014911in}{0.008660in}}{\pgfqpoint{-0.016667in}{0.004420in}}{\pgfqpoint{-0.016667in}{0.000000in}}%
\pgfpathcurveto{\pgfqpoint{-0.016667in}{-0.004420in}}{\pgfqpoint{-0.014911in}{-0.008660in}}{\pgfqpoint{-0.011785in}{-0.011785in}}%
\pgfpathcurveto{\pgfqpoint{-0.008660in}{-0.014911in}}{\pgfqpoint{-0.004420in}{-0.016667in}}{\pgfqpoint{0.000000in}{-0.016667in}}%
\pgfpathlineto{\pgfqpoint{0.000000in}{-0.016667in}}%
\pgfpathclose%
\pgfusepath{stroke,fill}%
}%
\begin{pgfscope}%
\pgfsys@transformshift{3.047359in}{3.553061in}%
\pgfsys@useobject{currentmarker}{}%
\end{pgfscope}%
\end{pgfscope}%
\begin{pgfscope}%
\pgfpathrectangle{\pgfqpoint{0.765000in}{0.660000in}}{\pgfqpoint{4.620000in}{4.620000in}}%
\pgfusepath{clip}%
\pgfsetbuttcap%
\pgfsetroundjoin%
\pgfsetlinewidth{1.204500pt}%
\definecolor{currentstroke}{rgb}{0.827451,0.827451,0.827451}%
\pgfsetstrokecolor{currentstroke}%
\pgfsetdash{{1.200000pt}{1.980000pt}}{0.000000pt}%
\pgfpathmoveto{\pgfqpoint{3.047359in}{3.553061in}}%
\pgfpathlineto{\pgfqpoint{2.840481in}{3.456103in}}%
\pgfusepath{stroke}%
\end{pgfscope}%
\begin{pgfscope}%
\pgfpathrectangle{\pgfqpoint{0.765000in}{0.660000in}}{\pgfqpoint{4.620000in}{4.620000in}}%
\pgfusepath{clip}%
\pgfsetbuttcap%
\pgfsetroundjoin%
\pgfsetlinewidth{1.204500pt}%
\definecolor{currentstroke}{rgb}{0.827451,0.827451,0.827451}%
\pgfsetstrokecolor{currentstroke}%
\pgfsetdash{{1.200000pt}{1.980000pt}}{0.000000pt}%
\pgfpathmoveto{\pgfqpoint{3.047359in}{3.553061in}}%
\pgfpathlineto{\pgfqpoint{3.861519in}{3.171494in}}%
\pgfusepath{stroke}%
\end{pgfscope}%
\begin{pgfscope}%
\pgfpathrectangle{\pgfqpoint{0.765000in}{0.660000in}}{\pgfqpoint{4.620000in}{4.620000in}}%
\pgfusepath{clip}%
\pgfsetbuttcap%
\pgfsetroundjoin%
\pgfsetlinewidth{1.204500pt}%
\definecolor{currentstroke}{rgb}{0.827451,0.827451,0.827451}%
\pgfsetstrokecolor{currentstroke}%
\pgfsetdash{{1.200000pt}{1.980000pt}}{0.000000pt}%
\pgfpathmoveto{\pgfqpoint{3.047359in}{3.553061in}}%
\pgfpathlineto{\pgfqpoint{3.047359in}{3.553059in}}%
\pgfusepath{stroke}%
\end{pgfscope}%
\begin{pgfscope}%
\pgfpathrectangle{\pgfqpoint{0.765000in}{0.660000in}}{\pgfqpoint{4.620000in}{4.620000in}}%
\pgfusepath{clip}%
\pgfsetbuttcap%
\pgfsetroundjoin%
\pgfsetlinewidth{1.204500pt}%
\definecolor{currentstroke}{rgb}{0.827451,0.827451,0.827451}%
\pgfsetstrokecolor{currentstroke}%
\pgfsetdash{{1.200000pt}{1.980000pt}}{0.000000pt}%
\pgfpathmoveto{\pgfqpoint{3.047359in}{3.553061in}}%
\pgfpathlineto{\pgfqpoint{0.755000in}{4.627397in}}%
\pgfusepath{stroke}%
\end{pgfscope}%
\begin{pgfscope}%
\pgfpathrectangle{\pgfqpoint{0.765000in}{0.660000in}}{\pgfqpoint{4.620000in}{4.620000in}}%
\pgfusepath{clip}%
\pgfsetbuttcap%
\pgfsetroundjoin%
\pgfsetlinewidth{1.204500pt}%
\definecolor{currentstroke}{rgb}{0.827451,0.827451,0.827451}%
\pgfsetstrokecolor{currentstroke}%
\pgfsetdash{{1.200000pt}{1.980000pt}}{0.000000pt}%
\pgfpathmoveto{\pgfqpoint{3.047359in}{3.553061in}}%
\pgfpathlineto{\pgfqpoint{5.395000in}{4.653306in}}%
\pgfusepath{stroke}%
\end{pgfscope}%
\end{pgfpicture}%
\makeatother%
\endgroup%
}}
        \caption{Using $\mathcal{A}_3$}
        \label{fig:nofeatureeng}
    \end{subfigure}
    \hfill
    \centering
    \begin{subfigure}{0.45\textwidth}
        \centering
        \resizebox{\linewidth}{!}{%
        \centering
            \clipbox{0.15\width{} 0.3\height{} 0.15\width{} 0.2\height{}}{%% Creator: Matplotlib, PGF backend
%%
%% To include the figure in your LaTeX document, write
%%   \input{<filename>.pgf}
%%
%% Make sure the required packages are loaded in your preamble
%%   \usepackage{pgf}
%%
%% Also ensure that all the required font packages are loaded; for instance,
%% the lmodern package is sometimes necessary when using math font.
%%   \usepackage{lmodern}
%%
%% Figures using additional raster images can only be included by \input if
%% they are in the same directory as the main LaTeX file. For loading figures
%% from other directories you can use the `import` package
%%   \usepackage{import}
%%
%% and then include the figures with
%%   \import{<path to file>}{<filename>.pgf}
%%
%% Matplotlib used the following preamble
%%   
%%   \usepackage{fontspec}
%%   \setmainfont{DejaVuSerif.ttf}[Path=\detokenize{/Users/sam/Library/Python/3.9/lib/python/site-packages/matplotlib/mpl-data/fonts/ttf/}]
%%   \setsansfont{DejaVuSans.ttf}[Path=\detokenize{/Users/sam/Library/Python/3.9/lib/python/site-packages/matplotlib/mpl-data/fonts/ttf/}]
%%   \setmonofont{DejaVuSansMono.ttf}[Path=\detokenize{/Users/sam/Library/Python/3.9/lib/python/site-packages/matplotlib/mpl-data/fonts/ttf/}]
%%   \makeatletter\@ifpackageloaded{underscore}{}{\usepackage[strings]{underscore}}\makeatother
%%
\begingroup%
\makeatletter%
\begin{pgfpicture}%
\pgfpathrectangle{\pgfpointorigin}{\pgfqpoint{6.000000in}{6.000000in}}%
\pgfusepath{use as bounding box, clip}%
\begin{pgfscope}%
\pgfsetbuttcap%
\pgfsetmiterjoin%
\definecolor{currentfill}{rgb}{1.000000,1.000000,1.000000}%
\pgfsetfillcolor{currentfill}%
\pgfsetlinewidth{0.000000pt}%
\definecolor{currentstroke}{rgb}{1.000000,1.000000,1.000000}%
\pgfsetstrokecolor{currentstroke}%
\pgfsetdash{}{0pt}%
\pgfpathmoveto{\pgfqpoint{0.000000in}{0.000000in}}%
\pgfpathlineto{\pgfqpoint{6.000000in}{0.000000in}}%
\pgfpathlineto{\pgfqpoint{6.000000in}{6.000000in}}%
\pgfpathlineto{\pgfqpoint{0.000000in}{6.000000in}}%
\pgfpathlineto{\pgfqpoint{0.000000in}{0.000000in}}%
\pgfpathclose%
\pgfusepath{fill}%
\end{pgfscope}%
\begin{pgfscope}%
\pgfsetbuttcap%
\pgfsetmiterjoin%
\definecolor{currentfill}{rgb}{1.000000,1.000000,1.000000}%
\pgfsetfillcolor{currentfill}%
\pgfsetlinewidth{0.000000pt}%
\definecolor{currentstroke}{rgb}{0.000000,0.000000,0.000000}%
\pgfsetstrokecolor{currentstroke}%
\pgfsetstrokeopacity{0.000000}%
\pgfsetdash{}{0pt}%
\pgfpathmoveto{\pgfqpoint{0.765000in}{0.660000in}}%
\pgfpathlineto{\pgfqpoint{5.385000in}{0.660000in}}%
\pgfpathlineto{\pgfqpoint{5.385000in}{5.280000in}}%
\pgfpathlineto{\pgfqpoint{0.765000in}{5.280000in}}%
\pgfpathlineto{\pgfqpoint{0.765000in}{0.660000in}}%
\pgfpathclose%
\pgfusepath{fill}%
\end{pgfscope}%
\begin{pgfscope}%
\pgfpathrectangle{\pgfqpoint{0.765000in}{0.660000in}}{\pgfqpoint{4.620000in}{4.620000in}}%
\pgfusepath{clip}%
\pgfsetbuttcap%
\pgfsetroundjoin%
\definecolor{currentfill}{rgb}{1.000000,0.894118,0.788235}%
\pgfsetfillcolor{currentfill}%
\pgfsetlinewidth{0.000000pt}%
\definecolor{currentstroke}{rgb}{1.000000,0.894118,0.788235}%
\pgfsetstrokecolor{currentstroke}%
\pgfsetdash{}{0pt}%
\pgfpathmoveto{\pgfqpoint{3.325787in}{2.910799in}}%
\pgfpathlineto{\pgfqpoint{3.298839in}{2.895241in}}%
\pgfpathlineto{\pgfqpoint{3.325787in}{2.879683in}}%
\pgfpathlineto{\pgfqpoint{3.352734in}{2.895241in}}%
\pgfpathlineto{\pgfqpoint{3.325787in}{2.910799in}}%
\pgfpathclose%
\pgfusepath{fill}%
\end{pgfscope}%
\begin{pgfscope}%
\pgfpathrectangle{\pgfqpoint{0.765000in}{0.660000in}}{\pgfqpoint{4.620000in}{4.620000in}}%
\pgfusepath{clip}%
\pgfsetbuttcap%
\pgfsetroundjoin%
\definecolor{currentfill}{rgb}{1.000000,0.894118,0.788235}%
\pgfsetfillcolor{currentfill}%
\pgfsetlinewidth{0.000000pt}%
\definecolor{currentstroke}{rgb}{1.000000,0.894118,0.788235}%
\pgfsetstrokecolor{currentstroke}%
\pgfsetdash{}{0pt}%
\pgfpathmoveto{\pgfqpoint{3.325787in}{2.910799in}}%
\pgfpathlineto{\pgfqpoint{3.298839in}{2.895241in}}%
\pgfpathlineto{\pgfqpoint{3.298839in}{2.926357in}}%
\pgfpathlineto{\pgfqpoint{3.325787in}{2.941915in}}%
\pgfpathlineto{\pgfqpoint{3.325787in}{2.910799in}}%
\pgfpathclose%
\pgfusepath{fill}%
\end{pgfscope}%
\begin{pgfscope}%
\pgfpathrectangle{\pgfqpoint{0.765000in}{0.660000in}}{\pgfqpoint{4.620000in}{4.620000in}}%
\pgfusepath{clip}%
\pgfsetbuttcap%
\pgfsetroundjoin%
\definecolor{currentfill}{rgb}{1.000000,0.894118,0.788235}%
\pgfsetfillcolor{currentfill}%
\pgfsetlinewidth{0.000000pt}%
\definecolor{currentstroke}{rgb}{1.000000,0.894118,0.788235}%
\pgfsetstrokecolor{currentstroke}%
\pgfsetdash{}{0pt}%
\pgfpathmoveto{\pgfqpoint{3.325787in}{2.910799in}}%
\pgfpathlineto{\pgfqpoint{3.352734in}{2.895241in}}%
\pgfpathlineto{\pgfqpoint{3.352734in}{2.926357in}}%
\pgfpathlineto{\pgfqpoint{3.325787in}{2.941915in}}%
\pgfpathlineto{\pgfqpoint{3.325787in}{2.910799in}}%
\pgfpathclose%
\pgfusepath{fill}%
\end{pgfscope}%
\begin{pgfscope}%
\pgfpathrectangle{\pgfqpoint{0.765000in}{0.660000in}}{\pgfqpoint{4.620000in}{4.620000in}}%
\pgfusepath{clip}%
\pgfsetbuttcap%
\pgfsetroundjoin%
\definecolor{currentfill}{rgb}{1.000000,0.894118,0.788235}%
\pgfsetfillcolor{currentfill}%
\pgfsetlinewidth{0.000000pt}%
\definecolor{currentstroke}{rgb}{1.000000,0.894118,0.788235}%
\pgfsetstrokecolor{currentstroke}%
\pgfsetdash{}{0pt}%
\pgfpathmoveto{\pgfqpoint{3.538039in}{2.783646in}}%
\pgfpathlineto{\pgfqpoint{3.511092in}{2.768088in}}%
\pgfpathlineto{\pgfqpoint{3.538039in}{2.752530in}}%
\pgfpathlineto{\pgfqpoint{3.564986in}{2.768088in}}%
\pgfpathlineto{\pgfqpoint{3.538039in}{2.783646in}}%
\pgfpathclose%
\pgfusepath{fill}%
\end{pgfscope}%
\begin{pgfscope}%
\pgfpathrectangle{\pgfqpoint{0.765000in}{0.660000in}}{\pgfqpoint{4.620000in}{4.620000in}}%
\pgfusepath{clip}%
\pgfsetbuttcap%
\pgfsetroundjoin%
\definecolor{currentfill}{rgb}{1.000000,0.894118,0.788235}%
\pgfsetfillcolor{currentfill}%
\pgfsetlinewidth{0.000000pt}%
\definecolor{currentstroke}{rgb}{1.000000,0.894118,0.788235}%
\pgfsetstrokecolor{currentstroke}%
\pgfsetdash{}{0pt}%
\pgfpathmoveto{\pgfqpoint{3.538039in}{2.783646in}}%
\pgfpathlineto{\pgfqpoint{3.511092in}{2.768088in}}%
\pgfpathlineto{\pgfqpoint{3.511092in}{2.799204in}}%
\pgfpathlineto{\pgfqpoint{3.538039in}{2.814762in}}%
\pgfpathlineto{\pgfqpoint{3.538039in}{2.783646in}}%
\pgfpathclose%
\pgfusepath{fill}%
\end{pgfscope}%
\begin{pgfscope}%
\pgfpathrectangle{\pgfqpoint{0.765000in}{0.660000in}}{\pgfqpoint{4.620000in}{4.620000in}}%
\pgfusepath{clip}%
\pgfsetbuttcap%
\pgfsetroundjoin%
\definecolor{currentfill}{rgb}{1.000000,0.894118,0.788235}%
\pgfsetfillcolor{currentfill}%
\pgfsetlinewidth{0.000000pt}%
\definecolor{currentstroke}{rgb}{1.000000,0.894118,0.788235}%
\pgfsetstrokecolor{currentstroke}%
\pgfsetdash{}{0pt}%
\pgfpathmoveto{\pgfqpoint{3.538039in}{2.783646in}}%
\pgfpathlineto{\pgfqpoint{3.564986in}{2.768088in}}%
\pgfpathlineto{\pgfqpoint{3.564986in}{2.799204in}}%
\pgfpathlineto{\pgfqpoint{3.538039in}{2.814762in}}%
\pgfpathlineto{\pgfqpoint{3.538039in}{2.783646in}}%
\pgfpathclose%
\pgfusepath{fill}%
\end{pgfscope}%
\begin{pgfscope}%
\pgfpathrectangle{\pgfqpoint{0.765000in}{0.660000in}}{\pgfqpoint{4.620000in}{4.620000in}}%
\pgfusepath{clip}%
\pgfsetbuttcap%
\pgfsetroundjoin%
\definecolor{currentfill}{rgb}{1.000000,0.894118,0.788235}%
\pgfsetfillcolor{currentfill}%
\pgfsetlinewidth{0.000000pt}%
\definecolor{currentstroke}{rgb}{1.000000,0.894118,0.788235}%
\pgfsetstrokecolor{currentstroke}%
\pgfsetdash{}{0pt}%
\pgfpathmoveto{\pgfqpoint{3.325787in}{2.941915in}}%
\pgfpathlineto{\pgfqpoint{3.298839in}{2.926357in}}%
\pgfpathlineto{\pgfqpoint{3.325787in}{2.910799in}}%
\pgfpathlineto{\pgfqpoint{3.352734in}{2.926357in}}%
\pgfpathlineto{\pgfqpoint{3.325787in}{2.941915in}}%
\pgfpathclose%
\pgfusepath{fill}%
\end{pgfscope}%
\begin{pgfscope}%
\pgfpathrectangle{\pgfqpoint{0.765000in}{0.660000in}}{\pgfqpoint{4.620000in}{4.620000in}}%
\pgfusepath{clip}%
\pgfsetbuttcap%
\pgfsetroundjoin%
\definecolor{currentfill}{rgb}{1.000000,0.894118,0.788235}%
\pgfsetfillcolor{currentfill}%
\pgfsetlinewidth{0.000000pt}%
\definecolor{currentstroke}{rgb}{1.000000,0.894118,0.788235}%
\pgfsetstrokecolor{currentstroke}%
\pgfsetdash{}{0pt}%
\pgfpathmoveto{\pgfqpoint{3.325787in}{2.879683in}}%
\pgfpathlineto{\pgfqpoint{3.352734in}{2.895241in}}%
\pgfpathlineto{\pgfqpoint{3.352734in}{2.926357in}}%
\pgfpathlineto{\pgfqpoint{3.325787in}{2.910799in}}%
\pgfpathlineto{\pgfqpoint{3.325787in}{2.879683in}}%
\pgfpathclose%
\pgfusepath{fill}%
\end{pgfscope}%
\begin{pgfscope}%
\pgfpathrectangle{\pgfqpoint{0.765000in}{0.660000in}}{\pgfqpoint{4.620000in}{4.620000in}}%
\pgfusepath{clip}%
\pgfsetbuttcap%
\pgfsetroundjoin%
\definecolor{currentfill}{rgb}{1.000000,0.894118,0.788235}%
\pgfsetfillcolor{currentfill}%
\pgfsetlinewidth{0.000000pt}%
\definecolor{currentstroke}{rgb}{1.000000,0.894118,0.788235}%
\pgfsetstrokecolor{currentstroke}%
\pgfsetdash{}{0pt}%
\pgfpathmoveto{\pgfqpoint{3.298839in}{2.895241in}}%
\pgfpathlineto{\pgfqpoint{3.325787in}{2.879683in}}%
\pgfpathlineto{\pgfqpoint{3.325787in}{2.910799in}}%
\pgfpathlineto{\pgfqpoint{3.298839in}{2.926357in}}%
\pgfpathlineto{\pgfqpoint{3.298839in}{2.895241in}}%
\pgfpathclose%
\pgfusepath{fill}%
\end{pgfscope}%
\begin{pgfscope}%
\pgfpathrectangle{\pgfqpoint{0.765000in}{0.660000in}}{\pgfqpoint{4.620000in}{4.620000in}}%
\pgfusepath{clip}%
\pgfsetbuttcap%
\pgfsetroundjoin%
\definecolor{currentfill}{rgb}{1.000000,0.894118,0.788235}%
\pgfsetfillcolor{currentfill}%
\pgfsetlinewidth{0.000000pt}%
\definecolor{currentstroke}{rgb}{1.000000,0.894118,0.788235}%
\pgfsetstrokecolor{currentstroke}%
\pgfsetdash{}{0pt}%
\pgfpathmoveto{\pgfqpoint{3.538039in}{2.814762in}}%
\pgfpathlineto{\pgfqpoint{3.511092in}{2.799204in}}%
\pgfpathlineto{\pgfqpoint{3.538039in}{2.783646in}}%
\pgfpathlineto{\pgfqpoint{3.564986in}{2.799204in}}%
\pgfpathlineto{\pgfqpoint{3.538039in}{2.814762in}}%
\pgfpathclose%
\pgfusepath{fill}%
\end{pgfscope}%
\begin{pgfscope}%
\pgfpathrectangle{\pgfqpoint{0.765000in}{0.660000in}}{\pgfqpoint{4.620000in}{4.620000in}}%
\pgfusepath{clip}%
\pgfsetbuttcap%
\pgfsetroundjoin%
\definecolor{currentfill}{rgb}{1.000000,0.894118,0.788235}%
\pgfsetfillcolor{currentfill}%
\pgfsetlinewidth{0.000000pt}%
\definecolor{currentstroke}{rgb}{1.000000,0.894118,0.788235}%
\pgfsetstrokecolor{currentstroke}%
\pgfsetdash{}{0pt}%
\pgfpathmoveto{\pgfqpoint{3.538039in}{2.752530in}}%
\pgfpathlineto{\pgfqpoint{3.564986in}{2.768088in}}%
\pgfpathlineto{\pgfqpoint{3.564986in}{2.799204in}}%
\pgfpathlineto{\pgfqpoint{3.538039in}{2.783646in}}%
\pgfpathlineto{\pgfqpoint{3.538039in}{2.752530in}}%
\pgfpathclose%
\pgfusepath{fill}%
\end{pgfscope}%
\begin{pgfscope}%
\pgfpathrectangle{\pgfqpoint{0.765000in}{0.660000in}}{\pgfqpoint{4.620000in}{4.620000in}}%
\pgfusepath{clip}%
\pgfsetbuttcap%
\pgfsetroundjoin%
\definecolor{currentfill}{rgb}{1.000000,0.894118,0.788235}%
\pgfsetfillcolor{currentfill}%
\pgfsetlinewidth{0.000000pt}%
\definecolor{currentstroke}{rgb}{1.000000,0.894118,0.788235}%
\pgfsetstrokecolor{currentstroke}%
\pgfsetdash{}{0pt}%
\pgfpathmoveto{\pgfqpoint{3.511092in}{2.768088in}}%
\pgfpathlineto{\pgfqpoint{3.538039in}{2.752530in}}%
\pgfpathlineto{\pgfqpoint{3.538039in}{2.783646in}}%
\pgfpathlineto{\pgfqpoint{3.511092in}{2.799204in}}%
\pgfpathlineto{\pgfqpoint{3.511092in}{2.768088in}}%
\pgfpathclose%
\pgfusepath{fill}%
\end{pgfscope}%
\begin{pgfscope}%
\pgfpathrectangle{\pgfqpoint{0.765000in}{0.660000in}}{\pgfqpoint{4.620000in}{4.620000in}}%
\pgfusepath{clip}%
\pgfsetbuttcap%
\pgfsetroundjoin%
\definecolor{currentfill}{rgb}{1.000000,0.894118,0.788235}%
\pgfsetfillcolor{currentfill}%
\pgfsetlinewidth{0.000000pt}%
\definecolor{currentstroke}{rgb}{1.000000,0.894118,0.788235}%
\pgfsetstrokecolor{currentstroke}%
\pgfsetdash{}{0pt}%
\pgfpathmoveto{\pgfqpoint{3.325787in}{2.910799in}}%
\pgfpathlineto{\pgfqpoint{3.298839in}{2.895241in}}%
\pgfpathlineto{\pgfqpoint{3.511092in}{2.768088in}}%
\pgfpathlineto{\pgfqpoint{3.538039in}{2.783646in}}%
\pgfpathlineto{\pgfqpoint{3.325787in}{2.910799in}}%
\pgfpathclose%
\pgfusepath{fill}%
\end{pgfscope}%
\begin{pgfscope}%
\pgfpathrectangle{\pgfqpoint{0.765000in}{0.660000in}}{\pgfqpoint{4.620000in}{4.620000in}}%
\pgfusepath{clip}%
\pgfsetbuttcap%
\pgfsetroundjoin%
\definecolor{currentfill}{rgb}{1.000000,0.894118,0.788235}%
\pgfsetfillcolor{currentfill}%
\pgfsetlinewidth{0.000000pt}%
\definecolor{currentstroke}{rgb}{1.000000,0.894118,0.788235}%
\pgfsetstrokecolor{currentstroke}%
\pgfsetdash{}{0pt}%
\pgfpathmoveto{\pgfqpoint{3.352734in}{2.895241in}}%
\pgfpathlineto{\pgfqpoint{3.325787in}{2.910799in}}%
\pgfpathlineto{\pgfqpoint{3.538039in}{2.783646in}}%
\pgfpathlineto{\pgfqpoint{3.564986in}{2.768088in}}%
\pgfpathlineto{\pgfqpoint{3.352734in}{2.895241in}}%
\pgfpathclose%
\pgfusepath{fill}%
\end{pgfscope}%
\begin{pgfscope}%
\pgfpathrectangle{\pgfqpoint{0.765000in}{0.660000in}}{\pgfqpoint{4.620000in}{4.620000in}}%
\pgfusepath{clip}%
\pgfsetbuttcap%
\pgfsetroundjoin%
\definecolor{currentfill}{rgb}{1.000000,0.894118,0.788235}%
\pgfsetfillcolor{currentfill}%
\pgfsetlinewidth{0.000000pt}%
\definecolor{currentstroke}{rgb}{1.000000,0.894118,0.788235}%
\pgfsetstrokecolor{currentstroke}%
\pgfsetdash{}{0pt}%
\pgfpathmoveto{\pgfqpoint{3.325787in}{2.910799in}}%
\pgfpathlineto{\pgfqpoint{3.325787in}{2.941915in}}%
\pgfpathlineto{\pgfqpoint{3.538039in}{2.814762in}}%
\pgfpathlineto{\pgfqpoint{3.564986in}{2.768088in}}%
\pgfpathlineto{\pgfqpoint{3.325787in}{2.910799in}}%
\pgfpathclose%
\pgfusepath{fill}%
\end{pgfscope}%
\begin{pgfscope}%
\pgfpathrectangle{\pgfqpoint{0.765000in}{0.660000in}}{\pgfqpoint{4.620000in}{4.620000in}}%
\pgfusepath{clip}%
\pgfsetbuttcap%
\pgfsetroundjoin%
\definecolor{currentfill}{rgb}{1.000000,0.894118,0.788235}%
\pgfsetfillcolor{currentfill}%
\pgfsetlinewidth{0.000000pt}%
\definecolor{currentstroke}{rgb}{1.000000,0.894118,0.788235}%
\pgfsetstrokecolor{currentstroke}%
\pgfsetdash{}{0pt}%
\pgfpathmoveto{\pgfqpoint{3.352734in}{2.895241in}}%
\pgfpathlineto{\pgfqpoint{3.352734in}{2.926357in}}%
\pgfpathlineto{\pgfqpoint{3.564986in}{2.799204in}}%
\pgfpathlineto{\pgfqpoint{3.538039in}{2.783646in}}%
\pgfpathlineto{\pgfqpoint{3.352734in}{2.895241in}}%
\pgfpathclose%
\pgfusepath{fill}%
\end{pgfscope}%
\begin{pgfscope}%
\pgfpathrectangle{\pgfqpoint{0.765000in}{0.660000in}}{\pgfqpoint{4.620000in}{4.620000in}}%
\pgfusepath{clip}%
\pgfsetbuttcap%
\pgfsetroundjoin%
\definecolor{currentfill}{rgb}{1.000000,0.894118,0.788235}%
\pgfsetfillcolor{currentfill}%
\pgfsetlinewidth{0.000000pt}%
\definecolor{currentstroke}{rgb}{1.000000,0.894118,0.788235}%
\pgfsetstrokecolor{currentstroke}%
\pgfsetdash{}{0pt}%
\pgfpathmoveto{\pgfqpoint{3.298839in}{2.895241in}}%
\pgfpathlineto{\pgfqpoint{3.325787in}{2.879683in}}%
\pgfpathlineto{\pgfqpoint{3.538039in}{2.752530in}}%
\pgfpathlineto{\pgfqpoint{3.511092in}{2.768088in}}%
\pgfpathlineto{\pgfqpoint{3.298839in}{2.895241in}}%
\pgfpathclose%
\pgfusepath{fill}%
\end{pgfscope}%
\begin{pgfscope}%
\pgfpathrectangle{\pgfqpoint{0.765000in}{0.660000in}}{\pgfqpoint{4.620000in}{4.620000in}}%
\pgfusepath{clip}%
\pgfsetbuttcap%
\pgfsetroundjoin%
\definecolor{currentfill}{rgb}{1.000000,0.894118,0.788235}%
\pgfsetfillcolor{currentfill}%
\pgfsetlinewidth{0.000000pt}%
\definecolor{currentstroke}{rgb}{1.000000,0.894118,0.788235}%
\pgfsetstrokecolor{currentstroke}%
\pgfsetdash{}{0pt}%
\pgfpathmoveto{\pgfqpoint{3.325787in}{2.879683in}}%
\pgfpathlineto{\pgfqpoint{3.352734in}{2.895241in}}%
\pgfpathlineto{\pgfqpoint{3.564986in}{2.768088in}}%
\pgfpathlineto{\pgfqpoint{3.538039in}{2.752530in}}%
\pgfpathlineto{\pgfqpoint{3.325787in}{2.879683in}}%
\pgfpathclose%
\pgfusepath{fill}%
\end{pgfscope}%
\begin{pgfscope}%
\pgfpathrectangle{\pgfqpoint{0.765000in}{0.660000in}}{\pgfqpoint{4.620000in}{4.620000in}}%
\pgfusepath{clip}%
\pgfsetbuttcap%
\pgfsetroundjoin%
\definecolor{currentfill}{rgb}{1.000000,0.894118,0.788235}%
\pgfsetfillcolor{currentfill}%
\pgfsetlinewidth{0.000000pt}%
\definecolor{currentstroke}{rgb}{1.000000,0.894118,0.788235}%
\pgfsetstrokecolor{currentstroke}%
\pgfsetdash{}{0pt}%
\pgfpathmoveto{\pgfqpoint{3.325787in}{2.941915in}}%
\pgfpathlineto{\pgfqpoint{3.298839in}{2.926357in}}%
\pgfpathlineto{\pgfqpoint{3.511092in}{2.799204in}}%
\pgfpathlineto{\pgfqpoint{3.538039in}{2.814762in}}%
\pgfpathlineto{\pgfqpoint{3.325787in}{2.941915in}}%
\pgfpathclose%
\pgfusepath{fill}%
\end{pgfscope}%
\begin{pgfscope}%
\pgfpathrectangle{\pgfqpoint{0.765000in}{0.660000in}}{\pgfqpoint{4.620000in}{4.620000in}}%
\pgfusepath{clip}%
\pgfsetbuttcap%
\pgfsetroundjoin%
\definecolor{currentfill}{rgb}{1.000000,0.894118,0.788235}%
\pgfsetfillcolor{currentfill}%
\pgfsetlinewidth{0.000000pt}%
\definecolor{currentstroke}{rgb}{1.000000,0.894118,0.788235}%
\pgfsetstrokecolor{currentstroke}%
\pgfsetdash{}{0pt}%
\pgfpathmoveto{\pgfqpoint{3.352734in}{2.926357in}}%
\pgfpathlineto{\pgfqpoint{3.325787in}{2.941915in}}%
\pgfpathlineto{\pgfqpoint{3.538039in}{2.814762in}}%
\pgfpathlineto{\pgfqpoint{3.564986in}{2.799204in}}%
\pgfpathlineto{\pgfqpoint{3.352734in}{2.926357in}}%
\pgfpathclose%
\pgfusepath{fill}%
\end{pgfscope}%
\begin{pgfscope}%
\pgfpathrectangle{\pgfqpoint{0.765000in}{0.660000in}}{\pgfqpoint{4.620000in}{4.620000in}}%
\pgfusepath{clip}%
\pgfsetbuttcap%
\pgfsetroundjoin%
\definecolor{currentfill}{rgb}{1.000000,0.894118,0.788235}%
\pgfsetfillcolor{currentfill}%
\pgfsetlinewidth{0.000000pt}%
\definecolor{currentstroke}{rgb}{1.000000,0.894118,0.788235}%
\pgfsetstrokecolor{currentstroke}%
\pgfsetdash{}{0pt}%
\pgfpathmoveto{\pgfqpoint{3.298839in}{2.895241in}}%
\pgfpathlineto{\pgfqpoint{3.298839in}{2.926357in}}%
\pgfpathlineto{\pgfqpoint{3.511092in}{2.799204in}}%
\pgfpathlineto{\pgfqpoint{3.538039in}{2.752530in}}%
\pgfpathlineto{\pgfqpoint{3.298839in}{2.895241in}}%
\pgfpathclose%
\pgfusepath{fill}%
\end{pgfscope}%
\begin{pgfscope}%
\pgfpathrectangle{\pgfqpoint{0.765000in}{0.660000in}}{\pgfqpoint{4.620000in}{4.620000in}}%
\pgfusepath{clip}%
\pgfsetbuttcap%
\pgfsetroundjoin%
\definecolor{currentfill}{rgb}{1.000000,0.894118,0.788235}%
\pgfsetfillcolor{currentfill}%
\pgfsetlinewidth{0.000000pt}%
\definecolor{currentstroke}{rgb}{1.000000,0.894118,0.788235}%
\pgfsetstrokecolor{currentstroke}%
\pgfsetdash{}{0pt}%
\pgfpathmoveto{\pgfqpoint{3.325787in}{2.879683in}}%
\pgfpathlineto{\pgfqpoint{3.325787in}{2.910799in}}%
\pgfpathlineto{\pgfqpoint{3.538039in}{2.783646in}}%
\pgfpathlineto{\pgfqpoint{3.511092in}{2.768088in}}%
\pgfpathlineto{\pgfqpoint{3.325787in}{2.879683in}}%
\pgfpathclose%
\pgfusepath{fill}%
\end{pgfscope}%
\begin{pgfscope}%
\pgfpathrectangle{\pgfqpoint{0.765000in}{0.660000in}}{\pgfqpoint{4.620000in}{4.620000in}}%
\pgfusepath{clip}%
\pgfsetbuttcap%
\pgfsetroundjoin%
\definecolor{currentfill}{rgb}{1.000000,0.894118,0.788235}%
\pgfsetfillcolor{currentfill}%
\pgfsetlinewidth{0.000000pt}%
\definecolor{currentstroke}{rgb}{1.000000,0.894118,0.788235}%
\pgfsetstrokecolor{currentstroke}%
\pgfsetdash{}{0pt}%
\pgfpathmoveto{\pgfqpoint{3.325787in}{2.910799in}}%
\pgfpathlineto{\pgfqpoint{3.352734in}{2.926357in}}%
\pgfpathlineto{\pgfqpoint{3.564986in}{2.799204in}}%
\pgfpathlineto{\pgfqpoint{3.538039in}{2.783646in}}%
\pgfpathlineto{\pgfqpoint{3.325787in}{2.910799in}}%
\pgfpathclose%
\pgfusepath{fill}%
\end{pgfscope}%
\begin{pgfscope}%
\pgfpathrectangle{\pgfqpoint{0.765000in}{0.660000in}}{\pgfqpoint{4.620000in}{4.620000in}}%
\pgfusepath{clip}%
\pgfsetbuttcap%
\pgfsetroundjoin%
\definecolor{currentfill}{rgb}{1.000000,0.894118,0.788235}%
\pgfsetfillcolor{currentfill}%
\pgfsetlinewidth{0.000000pt}%
\definecolor{currentstroke}{rgb}{1.000000,0.894118,0.788235}%
\pgfsetstrokecolor{currentstroke}%
\pgfsetdash{}{0pt}%
\pgfpathmoveto{\pgfqpoint{3.298839in}{2.926357in}}%
\pgfpathlineto{\pgfqpoint{3.325787in}{2.910799in}}%
\pgfpathlineto{\pgfqpoint{3.538039in}{2.783646in}}%
\pgfpathlineto{\pgfqpoint{3.511092in}{2.799204in}}%
\pgfpathlineto{\pgfqpoint{3.298839in}{2.926357in}}%
\pgfpathclose%
\pgfusepath{fill}%
\end{pgfscope}%
\begin{pgfscope}%
\pgfpathrectangle{\pgfqpoint{0.765000in}{0.660000in}}{\pgfqpoint{4.620000in}{4.620000in}}%
\pgfusepath{clip}%
\pgfsetbuttcap%
\pgfsetroundjoin%
\definecolor{currentfill}{rgb}{1.000000,0.894118,0.788235}%
\pgfsetfillcolor{currentfill}%
\pgfsetlinewidth{0.000000pt}%
\definecolor{currentstroke}{rgb}{1.000000,0.894118,0.788235}%
\pgfsetstrokecolor{currentstroke}%
\pgfsetdash{}{0pt}%
\pgfpathmoveto{\pgfqpoint{3.325787in}{2.910799in}}%
\pgfpathlineto{\pgfqpoint{3.298839in}{2.895241in}}%
\pgfpathlineto{\pgfqpoint{3.325787in}{2.879683in}}%
\pgfpathlineto{\pgfqpoint{3.352734in}{2.895241in}}%
\pgfpathlineto{\pgfqpoint{3.325787in}{2.910799in}}%
\pgfpathclose%
\pgfusepath{fill}%
\end{pgfscope}%
\begin{pgfscope}%
\pgfpathrectangle{\pgfqpoint{0.765000in}{0.660000in}}{\pgfqpoint{4.620000in}{4.620000in}}%
\pgfusepath{clip}%
\pgfsetbuttcap%
\pgfsetroundjoin%
\definecolor{currentfill}{rgb}{1.000000,0.894118,0.788235}%
\pgfsetfillcolor{currentfill}%
\pgfsetlinewidth{0.000000pt}%
\definecolor{currentstroke}{rgb}{1.000000,0.894118,0.788235}%
\pgfsetstrokecolor{currentstroke}%
\pgfsetdash{}{0pt}%
\pgfpathmoveto{\pgfqpoint{3.325787in}{2.910799in}}%
\pgfpathlineto{\pgfqpoint{3.298839in}{2.895241in}}%
\pgfpathlineto{\pgfqpoint{3.298839in}{2.926357in}}%
\pgfpathlineto{\pgfqpoint{3.325787in}{2.941915in}}%
\pgfpathlineto{\pgfqpoint{3.325787in}{2.910799in}}%
\pgfpathclose%
\pgfusepath{fill}%
\end{pgfscope}%
\begin{pgfscope}%
\pgfpathrectangle{\pgfqpoint{0.765000in}{0.660000in}}{\pgfqpoint{4.620000in}{4.620000in}}%
\pgfusepath{clip}%
\pgfsetbuttcap%
\pgfsetroundjoin%
\definecolor{currentfill}{rgb}{1.000000,0.894118,0.788235}%
\pgfsetfillcolor{currentfill}%
\pgfsetlinewidth{0.000000pt}%
\definecolor{currentstroke}{rgb}{1.000000,0.894118,0.788235}%
\pgfsetstrokecolor{currentstroke}%
\pgfsetdash{}{0pt}%
\pgfpathmoveto{\pgfqpoint{3.325787in}{2.910799in}}%
\pgfpathlineto{\pgfqpoint{3.352734in}{2.895241in}}%
\pgfpathlineto{\pgfqpoint{3.352734in}{2.926357in}}%
\pgfpathlineto{\pgfqpoint{3.325787in}{2.941915in}}%
\pgfpathlineto{\pgfqpoint{3.325787in}{2.910799in}}%
\pgfpathclose%
\pgfusepath{fill}%
\end{pgfscope}%
\begin{pgfscope}%
\pgfpathrectangle{\pgfqpoint{0.765000in}{0.660000in}}{\pgfqpoint{4.620000in}{4.620000in}}%
\pgfusepath{clip}%
\pgfsetbuttcap%
\pgfsetroundjoin%
\definecolor{currentfill}{rgb}{1.000000,0.894118,0.788235}%
\pgfsetfillcolor{currentfill}%
\pgfsetlinewidth{0.000000pt}%
\definecolor{currentstroke}{rgb}{1.000000,0.894118,0.788235}%
\pgfsetstrokecolor{currentstroke}%
\pgfsetdash{}{0pt}%
\pgfpathmoveto{\pgfqpoint{3.318632in}{2.917586in}}%
\pgfpathlineto{\pgfqpoint{3.291685in}{2.902028in}}%
\pgfpathlineto{\pgfqpoint{3.318632in}{2.886470in}}%
\pgfpathlineto{\pgfqpoint{3.345579in}{2.902028in}}%
\pgfpathlineto{\pgfqpoint{3.318632in}{2.917586in}}%
\pgfpathclose%
\pgfusepath{fill}%
\end{pgfscope}%
\begin{pgfscope}%
\pgfpathrectangle{\pgfqpoint{0.765000in}{0.660000in}}{\pgfqpoint{4.620000in}{4.620000in}}%
\pgfusepath{clip}%
\pgfsetbuttcap%
\pgfsetroundjoin%
\definecolor{currentfill}{rgb}{1.000000,0.894118,0.788235}%
\pgfsetfillcolor{currentfill}%
\pgfsetlinewidth{0.000000pt}%
\definecolor{currentstroke}{rgb}{1.000000,0.894118,0.788235}%
\pgfsetstrokecolor{currentstroke}%
\pgfsetdash{}{0pt}%
\pgfpathmoveto{\pgfqpoint{3.318632in}{2.917586in}}%
\pgfpathlineto{\pgfqpoint{3.291685in}{2.902028in}}%
\pgfpathlineto{\pgfqpoint{3.291685in}{2.933144in}}%
\pgfpathlineto{\pgfqpoint{3.318632in}{2.948702in}}%
\pgfpathlineto{\pgfqpoint{3.318632in}{2.917586in}}%
\pgfpathclose%
\pgfusepath{fill}%
\end{pgfscope}%
\begin{pgfscope}%
\pgfpathrectangle{\pgfqpoint{0.765000in}{0.660000in}}{\pgfqpoint{4.620000in}{4.620000in}}%
\pgfusepath{clip}%
\pgfsetbuttcap%
\pgfsetroundjoin%
\definecolor{currentfill}{rgb}{1.000000,0.894118,0.788235}%
\pgfsetfillcolor{currentfill}%
\pgfsetlinewidth{0.000000pt}%
\definecolor{currentstroke}{rgb}{1.000000,0.894118,0.788235}%
\pgfsetstrokecolor{currentstroke}%
\pgfsetdash{}{0pt}%
\pgfpathmoveto{\pgfqpoint{3.318632in}{2.917586in}}%
\pgfpathlineto{\pgfqpoint{3.345579in}{2.902028in}}%
\pgfpathlineto{\pgfqpoint{3.345579in}{2.933144in}}%
\pgfpathlineto{\pgfqpoint{3.318632in}{2.948702in}}%
\pgfpathlineto{\pgfqpoint{3.318632in}{2.917586in}}%
\pgfpathclose%
\pgfusepath{fill}%
\end{pgfscope}%
\begin{pgfscope}%
\pgfpathrectangle{\pgfqpoint{0.765000in}{0.660000in}}{\pgfqpoint{4.620000in}{4.620000in}}%
\pgfusepath{clip}%
\pgfsetbuttcap%
\pgfsetroundjoin%
\definecolor{currentfill}{rgb}{1.000000,0.894118,0.788235}%
\pgfsetfillcolor{currentfill}%
\pgfsetlinewidth{0.000000pt}%
\definecolor{currentstroke}{rgb}{1.000000,0.894118,0.788235}%
\pgfsetstrokecolor{currentstroke}%
\pgfsetdash{}{0pt}%
\pgfpathmoveto{\pgfqpoint{3.325787in}{2.941915in}}%
\pgfpathlineto{\pgfqpoint{3.298839in}{2.926357in}}%
\pgfpathlineto{\pgfqpoint{3.325787in}{2.910799in}}%
\pgfpathlineto{\pgfqpoint{3.352734in}{2.926357in}}%
\pgfpathlineto{\pgfqpoint{3.325787in}{2.941915in}}%
\pgfpathclose%
\pgfusepath{fill}%
\end{pgfscope}%
\begin{pgfscope}%
\pgfpathrectangle{\pgfqpoint{0.765000in}{0.660000in}}{\pgfqpoint{4.620000in}{4.620000in}}%
\pgfusepath{clip}%
\pgfsetbuttcap%
\pgfsetroundjoin%
\definecolor{currentfill}{rgb}{1.000000,0.894118,0.788235}%
\pgfsetfillcolor{currentfill}%
\pgfsetlinewidth{0.000000pt}%
\definecolor{currentstroke}{rgb}{1.000000,0.894118,0.788235}%
\pgfsetstrokecolor{currentstroke}%
\pgfsetdash{}{0pt}%
\pgfpathmoveto{\pgfqpoint{3.325787in}{2.879683in}}%
\pgfpathlineto{\pgfqpoint{3.352734in}{2.895241in}}%
\pgfpathlineto{\pgfqpoint{3.352734in}{2.926357in}}%
\pgfpathlineto{\pgfqpoint{3.325787in}{2.910799in}}%
\pgfpathlineto{\pgfqpoint{3.325787in}{2.879683in}}%
\pgfpathclose%
\pgfusepath{fill}%
\end{pgfscope}%
\begin{pgfscope}%
\pgfpathrectangle{\pgfqpoint{0.765000in}{0.660000in}}{\pgfqpoint{4.620000in}{4.620000in}}%
\pgfusepath{clip}%
\pgfsetbuttcap%
\pgfsetroundjoin%
\definecolor{currentfill}{rgb}{1.000000,0.894118,0.788235}%
\pgfsetfillcolor{currentfill}%
\pgfsetlinewidth{0.000000pt}%
\definecolor{currentstroke}{rgb}{1.000000,0.894118,0.788235}%
\pgfsetstrokecolor{currentstroke}%
\pgfsetdash{}{0pt}%
\pgfpathmoveto{\pgfqpoint{3.298839in}{2.895241in}}%
\pgfpathlineto{\pgfqpoint{3.325787in}{2.879683in}}%
\pgfpathlineto{\pgfqpoint{3.325787in}{2.910799in}}%
\pgfpathlineto{\pgfqpoint{3.298839in}{2.926357in}}%
\pgfpathlineto{\pgfqpoint{3.298839in}{2.895241in}}%
\pgfpathclose%
\pgfusepath{fill}%
\end{pgfscope}%
\begin{pgfscope}%
\pgfpathrectangle{\pgfqpoint{0.765000in}{0.660000in}}{\pgfqpoint{4.620000in}{4.620000in}}%
\pgfusepath{clip}%
\pgfsetbuttcap%
\pgfsetroundjoin%
\definecolor{currentfill}{rgb}{1.000000,0.894118,0.788235}%
\pgfsetfillcolor{currentfill}%
\pgfsetlinewidth{0.000000pt}%
\definecolor{currentstroke}{rgb}{1.000000,0.894118,0.788235}%
\pgfsetstrokecolor{currentstroke}%
\pgfsetdash{}{0pt}%
\pgfpathmoveto{\pgfqpoint{3.318632in}{2.948702in}}%
\pgfpathlineto{\pgfqpoint{3.291685in}{2.933144in}}%
\pgfpathlineto{\pgfqpoint{3.318632in}{2.917586in}}%
\pgfpathlineto{\pgfqpoint{3.345579in}{2.933144in}}%
\pgfpathlineto{\pgfqpoint{3.318632in}{2.948702in}}%
\pgfpathclose%
\pgfusepath{fill}%
\end{pgfscope}%
\begin{pgfscope}%
\pgfpathrectangle{\pgfqpoint{0.765000in}{0.660000in}}{\pgfqpoint{4.620000in}{4.620000in}}%
\pgfusepath{clip}%
\pgfsetbuttcap%
\pgfsetroundjoin%
\definecolor{currentfill}{rgb}{1.000000,0.894118,0.788235}%
\pgfsetfillcolor{currentfill}%
\pgfsetlinewidth{0.000000pt}%
\definecolor{currentstroke}{rgb}{1.000000,0.894118,0.788235}%
\pgfsetstrokecolor{currentstroke}%
\pgfsetdash{}{0pt}%
\pgfpathmoveto{\pgfqpoint{3.318632in}{2.886470in}}%
\pgfpathlineto{\pgfqpoint{3.345579in}{2.902028in}}%
\pgfpathlineto{\pgfqpoint{3.345579in}{2.933144in}}%
\pgfpathlineto{\pgfqpoint{3.318632in}{2.917586in}}%
\pgfpathlineto{\pgfqpoint{3.318632in}{2.886470in}}%
\pgfpathclose%
\pgfusepath{fill}%
\end{pgfscope}%
\begin{pgfscope}%
\pgfpathrectangle{\pgfqpoint{0.765000in}{0.660000in}}{\pgfqpoint{4.620000in}{4.620000in}}%
\pgfusepath{clip}%
\pgfsetbuttcap%
\pgfsetroundjoin%
\definecolor{currentfill}{rgb}{1.000000,0.894118,0.788235}%
\pgfsetfillcolor{currentfill}%
\pgfsetlinewidth{0.000000pt}%
\definecolor{currentstroke}{rgb}{1.000000,0.894118,0.788235}%
\pgfsetstrokecolor{currentstroke}%
\pgfsetdash{}{0pt}%
\pgfpathmoveto{\pgfqpoint{3.291685in}{2.902028in}}%
\pgfpathlineto{\pgfqpoint{3.318632in}{2.886470in}}%
\pgfpathlineto{\pgfqpoint{3.318632in}{2.917586in}}%
\pgfpathlineto{\pgfqpoint{3.291685in}{2.933144in}}%
\pgfpathlineto{\pgfqpoint{3.291685in}{2.902028in}}%
\pgfpathclose%
\pgfusepath{fill}%
\end{pgfscope}%
\begin{pgfscope}%
\pgfpathrectangle{\pgfqpoint{0.765000in}{0.660000in}}{\pgfqpoint{4.620000in}{4.620000in}}%
\pgfusepath{clip}%
\pgfsetbuttcap%
\pgfsetroundjoin%
\definecolor{currentfill}{rgb}{1.000000,0.894118,0.788235}%
\pgfsetfillcolor{currentfill}%
\pgfsetlinewidth{0.000000pt}%
\definecolor{currentstroke}{rgb}{1.000000,0.894118,0.788235}%
\pgfsetstrokecolor{currentstroke}%
\pgfsetdash{}{0pt}%
\pgfpathmoveto{\pgfqpoint{3.325787in}{2.910799in}}%
\pgfpathlineto{\pgfqpoint{3.298839in}{2.895241in}}%
\pgfpathlineto{\pgfqpoint{3.291685in}{2.902028in}}%
\pgfpathlineto{\pgfqpoint{3.318632in}{2.917586in}}%
\pgfpathlineto{\pgfqpoint{3.325787in}{2.910799in}}%
\pgfpathclose%
\pgfusepath{fill}%
\end{pgfscope}%
\begin{pgfscope}%
\pgfpathrectangle{\pgfqpoint{0.765000in}{0.660000in}}{\pgfqpoint{4.620000in}{4.620000in}}%
\pgfusepath{clip}%
\pgfsetbuttcap%
\pgfsetroundjoin%
\definecolor{currentfill}{rgb}{1.000000,0.894118,0.788235}%
\pgfsetfillcolor{currentfill}%
\pgfsetlinewidth{0.000000pt}%
\definecolor{currentstroke}{rgb}{1.000000,0.894118,0.788235}%
\pgfsetstrokecolor{currentstroke}%
\pgfsetdash{}{0pt}%
\pgfpathmoveto{\pgfqpoint{3.352734in}{2.895241in}}%
\pgfpathlineto{\pgfqpoint{3.325787in}{2.910799in}}%
\pgfpathlineto{\pgfqpoint{3.318632in}{2.917586in}}%
\pgfpathlineto{\pgfqpoint{3.345579in}{2.902028in}}%
\pgfpathlineto{\pgfqpoint{3.352734in}{2.895241in}}%
\pgfpathclose%
\pgfusepath{fill}%
\end{pgfscope}%
\begin{pgfscope}%
\pgfpathrectangle{\pgfqpoint{0.765000in}{0.660000in}}{\pgfqpoint{4.620000in}{4.620000in}}%
\pgfusepath{clip}%
\pgfsetbuttcap%
\pgfsetroundjoin%
\definecolor{currentfill}{rgb}{1.000000,0.894118,0.788235}%
\pgfsetfillcolor{currentfill}%
\pgfsetlinewidth{0.000000pt}%
\definecolor{currentstroke}{rgb}{1.000000,0.894118,0.788235}%
\pgfsetstrokecolor{currentstroke}%
\pgfsetdash{}{0pt}%
\pgfpathmoveto{\pgfqpoint{3.325787in}{2.910799in}}%
\pgfpathlineto{\pgfqpoint{3.325787in}{2.941915in}}%
\pgfpathlineto{\pgfqpoint{3.318632in}{2.948702in}}%
\pgfpathlineto{\pgfqpoint{3.345579in}{2.902028in}}%
\pgfpathlineto{\pgfqpoint{3.325787in}{2.910799in}}%
\pgfpathclose%
\pgfusepath{fill}%
\end{pgfscope}%
\begin{pgfscope}%
\pgfpathrectangle{\pgfqpoint{0.765000in}{0.660000in}}{\pgfqpoint{4.620000in}{4.620000in}}%
\pgfusepath{clip}%
\pgfsetbuttcap%
\pgfsetroundjoin%
\definecolor{currentfill}{rgb}{1.000000,0.894118,0.788235}%
\pgfsetfillcolor{currentfill}%
\pgfsetlinewidth{0.000000pt}%
\definecolor{currentstroke}{rgb}{1.000000,0.894118,0.788235}%
\pgfsetstrokecolor{currentstroke}%
\pgfsetdash{}{0pt}%
\pgfpathmoveto{\pgfqpoint{3.352734in}{2.895241in}}%
\pgfpathlineto{\pgfqpoint{3.352734in}{2.926357in}}%
\pgfpathlineto{\pgfqpoint{3.345579in}{2.933144in}}%
\pgfpathlineto{\pgfqpoint{3.318632in}{2.917586in}}%
\pgfpathlineto{\pgfqpoint{3.352734in}{2.895241in}}%
\pgfpathclose%
\pgfusepath{fill}%
\end{pgfscope}%
\begin{pgfscope}%
\pgfpathrectangle{\pgfqpoint{0.765000in}{0.660000in}}{\pgfqpoint{4.620000in}{4.620000in}}%
\pgfusepath{clip}%
\pgfsetbuttcap%
\pgfsetroundjoin%
\definecolor{currentfill}{rgb}{1.000000,0.894118,0.788235}%
\pgfsetfillcolor{currentfill}%
\pgfsetlinewidth{0.000000pt}%
\definecolor{currentstroke}{rgb}{1.000000,0.894118,0.788235}%
\pgfsetstrokecolor{currentstroke}%
\pgfsetdash{}{0pt}%
\pgfpathmoveto{\pgfqpoint{3.298839in}{2.895241in}}%
\pgfpathlineto{\pgfqpoint{3.325787in}{2.879683in}}%
\pgfpathlineto{\pgfqpoint{3.318632in}{2.886470in}}%
\pgfpathlineto{\pgfqpoint{3.291685in}{2.902028in}}%
\pgfpathlineto{\pgfqpoint{3.298839in}{2.895241in}}%
\pgfpathclose%
\pgfusepath{fill}%
\end{pgfscope}%
\begin{pgfscope}%
\pgfpathrectangle{\pgfqpoint{0.765000in}{0.660000in}}{\pgfqpoint{4.620000in}{4.620000in}}%
\pgfusepath{clip}%
\pgfsetbuttcap%
\pgfsetroundjoin%
\definecolor{currentfill}{rgb}{1.000000,0.894118,0.788235}%
\pgfsetfillcolor{currentfill}%
\pgfsetlinewidth{0.000000pt}%
\definecolor{currentstroke}{rgb}{1.000000,0.894118,0.788235}%
\pgfsetstrokecolor{currentstroke}%
\pgfsetdash{}{0pt}%
\pgfpathmoveto{\pgfqpoint{3.325787in}{2.879683in}}%
\pgfpathlineto{\pgfqpoint{3.352734in}{2.895241in}}%
\pgfpathlineto{\pgfqpoint{3.345579in}{2.902028in}}%
\pgfpathlineto{\pgfqpoint{3.318632in}{2.886470in}}%
\pgfpathlineto{\pgfqpoint{3.325787in}{2.879683in}}%
\pgfpathclose%
\pgfusepath{fill}%
\end{pgfscope}%
\begin{pgfscope}%
\pgfpathrectangle{\pgfqpoint{0.765000in}{0.660000in}}{\pgfqpoint{4.620000in}{4.620000in}}%
\pgfusepath{clip}%
\pgfsetbuttcap%
\pgfsetroundjoin%
\definecolor{currentfill}{rgb}{1.000000,0.894118,0.788235}%
\pgfsetfillcolor{currentfill}%
\pgfsetlinewidth{0.000000pt}%
\definecolor{currentstroke}{rgb}{1.000000,0.894118,0.788235}%
\pgfsetstrokecolor{currentstroke}%
\pgfsetdash{}{0pt}%
\pgfpathmoveto{\pgfqpoint{3.325787in}{2.941915in}}%
\pgfpathlineto{\pgfqpoint{3.298839in}{2.926357in}}%
\pgfpathlineto{\pgfqpoint{3.291685in}{2.933144in}}%
\pgfpathlineto{\pgfqpoint{3.318632in}{2.948702in}}%
\pgfpathlineto{\pgfqpoint{3.325787in}{2.941915in}}%
\pgfpathclose%
\pgfusepath{fill}%
\end{pgfscope}%
\begin{pgfscope}%
\pgfpathrectangle{\pgfqpoint{0.765000in}{0.660000in}}{\pgfqpoint{4.620000in}{4.620000in}}%
\pgfusepath{clip}%
\pgfsetbuttcap%
\pgfsetroundjoin%
\definecolor{currentfill}{rgb}{1.000000,0.894118,0.788235}%
\pgfsetfillcolor{currentfill}%
\pgfsetlinewidth{0.000000pt}%
\definecolor{currentstroke}{rgb}{1.000000,0.894118,0.788235}%
\pgfsetstrokecolor{currentstroke}%
\pgfsetdash{}{0pt}%
\pgfpathmoveto{\pgfqpoint{3.352734in}{2.926357in}}%
\pgfpathlineto{\pgfqpoint{3.325787in}{2.941915in}}%
\pgfpathlineto{\pgfqpoint{3.318632in}{2.948702in}}%
\pgfpathlineto{\pgfqpoint{3.345579in}{2.933144in}}%
\pgfpathlineto{\pgfqpoint{3.352734in}{2.926357in}}%
\pgfpathclose%
\pgfusepath{fill}%
\end{pgfscope}%
\begin{pgfscope}%
\pgfpathrectangle{\pgfqpoint{0.765000in}{0.660000in}}{\pgfqpoint{4.620000in}{4.620000in}}%
\pgfusepath{clip}%
\pgfsetbuttcap%
\pgfsetroundjoin%
\definecolor{currentfill}{rgb}{1.000000,0.894118,0.788235}%
\pgfsetfillcolor{currentfill}%
\pgfsetlinewidth{0.000000pt}%
\definecolor{currentstroke}{rgb}{1.000000,0.894118,0.788235}%
\pgfsetstrokecolor{currentstroke}%
\pgfsetdash{}{0pt}%
\pgfpathmoveto{\pgfqpoint{3.298839in}{2.895241in}}%
\pgfpathlineto{\pgfqpoint{3.298839in}{2.926357in}}%
\pgfpathlineto{\pgfqpoint{3.291685in}{2.933144in}}%
\pgfpathlineto{\pgfqpoint{3.318632in}{2.886470in}}%
\pgfpathlineto{\pgfqpoint{3.298839in}{2.895241in}}%
\pgfpathclose%
\pgfusepath{fill}%
\end{pgfscope}%
\begin{pgfscope}%
\pgfpathrectangle{\pgfqpoint{0.765000in}{0.660000in}}{\pgfqpoint{4.620000in}{4.620000in}}%
\pgfusepath{clip}%
\pgfsetbuttcap%
\pgfsetroundjoin%
\definecolor{currentfill}{rgb}{1.000000,0.894118,0.788235}%
\pgfsetfillcolor{currentfill}%
\pgfsetlinewidth{0.000000pt}%
\definecolor{currentstroke}{rgb}{1.000000,0.894118,0.788235}%
\pgfsetstrokecolor{currentstroke}%
\pgfsetdash{}{0pt}%
\pgfpathmoveto{\pgfqpoint{3.325787in}{2.879683in}}%
\pgfpathlineto{\pgfqpoint{3.325787in}{2.910799in}}%
\pgfpathlineto{\pgfqpoint{3.318632in}{2.917586in}}%
\pgfpathlineto{\pgfqpoint{3.291685in}{2.902028in}}%
\pgfpathlineto{\pgfqpoint{3.325787in}{2.879683in}}%
\pgfpathclose%
\pgfusepath{fill}%
\end{pgfscope}%
\begin{pgfscope}%
\pgfpathrectangle{\pgfqpoint{0.765000in}{0.660000in}}{\pgfqpoint{4.620000in}{4.620000in}}%
\pgfusepath{clip}%
\pgfsetbuttcap%
\pgfsetroundjoin%
\definecolor{currentfill}{rgb}{1.000000,0.894118,0.788235}%
\pgfsetfillcolor{currentfill}%
\pgfsetlinewidth{0.000000pt}%
\definecolor{currentstroke}{rgb}{1.000000,0.894118,0.788235}%
\pgfsetstrokecolor{currentstroke}%
\pgfsetdash{}{0pt}%
\pgfpathmoveto{\pgfqpoint{3.325787in}{2.910799in}}%
\pgfpathlineto{\pgfqpoint{3.352734in}{2.926357in}}%
\pgfpathlineto{\pgfqpoint{3.345579in}{2.933144in}}%
\pgfpathlineto{\pgfqpoint{3.318632in}{2.917586in}}%
\pgfpathlineto{\pgfqpoint{3.325787in}{2.910799in}}%
\pgfpathclose%
\pgfusepath{fill}%
\end{pgfscope}%
\begin{pgfscope}%
\pgfpathrectangle{\pgfqpoint{0.765000in}{0.660000in}}{\pgfqpoint{4.620000in}{4.620000in}}%
\pgfusepath{clip}%
\pgfsetbuttcap%
\pgfsetroundjoin%
\definecolor{currentfill}{rgb}{1.000000,0.894118,0.788235}%
\pgfsetfillcolor{currentfill}%
\pgfsetlinewidth{0.000000pt}%
\definecolor{currentstroke}{rgb}{1.000000,0.894118,0.788235}%
\pgfsetstrokecolor{currentstroke}%
\pgfsetdash{}{0pt}%
\pgfpathmoveto{\pgfqpoint{3.298839in}{2.926357in}}%
\pgfpathlineto{\pgfqpoint{3.325787in}{2.910799in}}%
\pgfpathlineto{\pgfqpoint{3.318632in}{2.917586in}}%
\pgfpathlineto{\pgfqpoint{3.291685in}{2.933144in}}%
\pgfpathlineto{\pgfqpoint{3.298839in}{2.926357in}}%
\pgfpathclose%
\pgfusepath{fill}%
\end{pgfscope}%
\begin{pgfscope}%
\pgfpathrectangle{\pgfqpoint{0.765000in}{0.660000in}}{\pgfqpoint{4.620000in}{4.620000in}}%
\pgfusepath{clip}%
\pgfsetbuttcap%
\pgfsetroundjoin%
\definecolor{currentfill}{rgb}{1.000000,0.894118,0.788235}%
\pgfsetfillcolor{currentfill}%
\pgfsetlinewidth{0.000000pt}%
\definecolor{currentstroke}{rgb}{1.000000,0.894118,0.788235}%
\pgfsetstrokecolor{currentstroke}%
\pgfsetdash{}{0pt}%
\pgfpathmoveto{\pgfqpoint{3.538039in}{2.783646in}}%
\pgfpathlineto{\pgfqpoint{3.511092in}{2.768088in}}%
\pgfpathlineto{\pgfqpoint{3.538039in}{2.752530in}}%
\pgfpathlineto{\pgfqpoint{3.564986in}{2.768088in}}%
\pgfpathlineto{\pgfqpoint{3.538039in}{2.783646in}}%
\pgfpathclose%
\pgfusepath{fill}%
\end{pgfscope}%
\begin{pgfscope}%
\pgfpathrectangle{\pgfqpoint{0.765000in}{0.660000in}}{\pgfqpoint{4.620000in}{4.620000in}}%
\pgfusepath{clip}%
\pgfsetbuttcap%
\pgfsetroundjoin%
\definecolor{currentfill}{rgb}{1.000000,0.894118,0.788235}%
\pgfsetfillcolor{currentfill}%
\pgfsetlinewidth{0.000000pt}%
\definecolor{currentstroke}{rgb}{1.000000,0.894118,0.788235}%
\pgfsetstrokecolor{currentstroke}%
\pgfsetdash{}{0pt}%
\pgfpathmoveto{\pgfqpoint{3.538039in}{2.783646in}}%
\pgfpathlineto{\pgfqpoint{3.511092in}{2.768088in}}%
\pgfpathlineto{\pgfqpoint{3.511092in}{2.799204in}}%
\pgfpathlineto{\pgfqpoint{3.538039in}{2.814762in}}%
\pgfpathlineto{\pgfqpoint{3.538039in}{2.783646in}}%
\pgfpathclose%
\pgfusepath{fill}%
\end{pgfscope}%
\begin{pgfscope}%
\pgfpathrectangle{\pgfqpoint{0.765000in}{0.660000in}}{\pgfqpoint{4.620000in}{4.620000in}}%
\pgfusepath{clip}%
\pgfsetbuttcap%
\pgfsetroundjoin%
\definecolor{currentfill}{rgb}{1.000000,0.894118,0.788235}%
\pgfsetfillcolor{currentfill}%
\pgfsetlinewidth{0.000000pt}%
\definecolor{currentstroke}{rgb}{1.000000,0.894118,0.788235}%
\pgfsetstrokecolor{currentstroke}%
\pgfsetdash{}{0pt}%
\pgfpathmoveto{\pgfqpoint{3.538039in}{2.783646in}}%
\pgfpathlineto{\pgfqpoint{3.564986in}{2.768088in}}%
\pgfpathlineto{\pgfqpoint{3.564986in}{2.799204in}}%
\pgfpathlineto{\pgfqpoint{3.538039in}{2.814762in}}%
\pgfpathlineto{\pgfqpoint{3.538039in}{2.783646in}}%
\pgfpathclose%
\pgfusepath{fill}%
\end{pgfscope}%
\begin{pgfscope}%
\pgfpathrectangle{\pgfqpoint{0.765000in}{0.660000in}}{\pgfqpoint{4.620000in}{4.620000in}}%
\pgfusepath{clip}%
\pgfsetbuttcap%
\pgfsetroundjoin%
\definecolor{currentfill}{rgb}{1.000000,0.894118,0.788235}%
\pgfsetfillcolor{currentfill}%
\pgfsetlinewidth{0.000000pt}%
\definecolor{currentstroke}{rgb}{1.000000,0.894118,0.788235}%
\pgfsetstrokecolor{currentstroke}%
\pgfsetdash{}{0pt}%
\pgfpathmoveto{\pgfqpoint{3.325787in}{2.910799in}}%
\pgfpathlineto{\pgfqpoint{3.298839in}{2.895241in}}%
\pgfpathlineto{\pgfqpoint{3.325787in}{2.879683in}}%
\pgfpathlineto{\pgfqpoint{3.352734in}{2.895241in}}%
\pgfpathlineto{\pgfqpoint{3.325787in}{2.910799in}}%
\pgfpathclose%
\pgfusepath{fill}%
\end{pgfscope}%
\begin{pgfscope}%
\pgfpathrectangle{\pgfqpoint{0.765000in}{0.660000in}}{\pgfqpoint{4.620000in}{4.620000in}}%
\pgfusepath{clip}%
\pgfsetbuttcap%
\pgfsetroundjoin%
\definecolor{currentfill}{rgb}{1.000000,0.894118,0.788235}%
\pgfsetfillcolor{currentfill}%
\pgfsetlinewidth{0.000000pt}%
\definecolor{currentstroke}{rgb}{1.000000,0.894118,0.788235}%
\pgfsetstrokecolor{currentstroke}%
\pgfsetdash{}{0pt}%
\pgfpathmoveto{\pgfqpoint{3.325787in}{2.910799in}}%
\pgfpathlineto{\pgfqpoint{3.298839in}{2.895241in}}%
\pgfpathlineto{\pgfqpoint{3.298839in}{2.926357in}}%
\pgfpathlineto{\pgfqpoint{3.325787in}{2.941915in}}%
\pgfpathlineto{\pgfqpoint{3.325787in}{2.910799in}}%
\pgfpathclose%
\pgfusepath{fill}%
\end{pgfscope}%
\begin{pgfscope}%
\pgfpathrectangle{\pgfqpoint{0.765000in}{0.660000in}}{\pgfqpoint{4.620000in}{4.620000in}}%
\pgfusepath{clip}%
\pgfsetbuttcap%
\pgfsetroundjoin%
\definecolor{currentfill}{rgb}{1.000000,0.894118,0.788235}%
\pgfsetfillcolor{currentfill}%
\pgfsetlinewidth{0.000000pt}%
\definecolor{currentstroke}{rgb}{1.000000,0.894118,0.788235}%
\pgfsetstrokecolor{currentstroke}%
\pgfsetdash{}{0pt}%
\pgfpathmoveto{\pgfqpoint{3.325787in}{2.910799in}}%
\pgfpathlineto{\pgfqpoint{3.352734in}{2.895241in}}%
\pgfpathlineto{\pgfqpoint{3.352734in}{2.926357in}}%
\pgfpathlineto{\pgfqpoint{3.325787in}{2.941915in}}%
\pgfpathlineto{\pgfqpoint{3.325787in}{2.910799in}}%
\pgfpathclose%
\pgfusepath{fill}%
\end{pgfscope}%
\begin{pgfscope}%
\pgfpathrectangle{\pgfqpoint{0.765000in}{0.660000in}}{\pgfqpoint{4.620000in}{4.620000in}}%
\pgfusepath{clip}%
\pgfsetbuttcap%
\pgfsetroundjoin%
\definecolor{currentfill}{rgb}{1.000000,0.894118,0.788235}%
\pgfsetfillcolor{currentfill}%
\pgfsetlinewidth{0.000000pt}%
\definecolor{currentstroke}{rgb}{1.000000,0.894118,0.788235}%
\pgfsetstrokecolor{currentstroke}%
\pgfsetdash{}{0pt}%
\pgfpathmoveto{\pgfqpoint{3.538039in}{2.814762in}}%
\pgfpathlineto{\pgfqpoint{3.511092in}{2.799204in}}%
\pgfpathlineto{\pgfqpoint{3.538039in}{2.783646in}}%
\pgfpathlineto{\pgfqpoint{3.564986in}{2.799204in}}%
\pgfpathlineto{\pgfqpoint{3.538039in}{2.814762in}}%
\pgfpathclose%
\pgfusepath{fill}%
\end{pgfscope}%
\begin{pgfscope}%
\pgfpathrectangle{\pgfqpoint{0.765000in}{0.660000in}}{\pgfqpoint{4.620000in}{4.620000in}}%
\pgfusepath{clip}%
\pgfsetbuttcap%
\pgfsetroundjoin%
\definecolor{currentfill}{rgb}{1.000000,0.894118,0.788235}%
\pgfsetfillcolor{currentfill}%
\pgfsetlinewidth{0.000000pt}%
\definecolor{currentstroke}{rgb}{1.000000,0.894118,0.788235}%
\pgfsetstrokecolor{currentstroke}%
\pgfsetdash{}{0pt}%
\pgfpathmoveto{\pgfqpoint{3.538039in}{2.752530in}}%
\pgfpathlineto{\pgfqpoint{3.564986in}{2.768088in}}%
\pgfpathlineto{\pgfqpoint{3.564986in}{2.799204in}}%
\pgfpathlineto{\pgfqpoint{3.538039in}{2.783646in}}%
\pgfpathlineto{\pgfqpoint{3.538039in}{2.752530in}}%
\pgfpathclose%
\pgfusepath{fill}%
\end{pgfscope}%
\begin{pgfscope}%
\pgfpathrectangle{\pgfqpoint{0.765000in}{0.660000in}}{\pgfqpoint{4.620000in}{4.620000in}}%
\pgfusepath{clip}%
\pgfsetbuttcap%
\pgfsetroundjoin%
\definecolor{currentfill}{rgb}{1.000000,0.894118,0.788235}%
\pgfsetfillcolor{currentfill}%
\pgfsetlinewidth{0.000000pt}%
\definecolor{currentstroke}{rgb}{1.000000,0.894118,0.788235}%
\pgfsetstrokecolor{currentstroke}%
\pgfsetdash{}{0pt}%
\pgfpathmoveto{\pgfqpoint{3.511092in}{2.768088in}}%
\pgfpathlineto{\pgfqpoint{3.538039in}{2.752530in}}%
\pgfpathlineto{\pgfqpoint{3.538039in}{2.783646in}}%
\pgfpathlineto{\pgfqpoint{3.511092in}{2.799204in}}%
\pgfpathlineto{\pgfqpoint{3.511092in}{2.768088in}}%
\pgfpathclose%
\pgfusepath{fill}%
\end{pgfscope}%
\begin{pgfscope}%
\pgfpathrectangle{\pgfqpoint{0.765000in}{0.660000in}}{\pgfqpoint{4.620000in}{4.620000in}}%
\pgfusepath{clip}%
\pgfsetbuttcap%
\pgfsetroundjoin%
\definecolor{currentfill}{rgb}{1.000000,0.894118,0.788235}%
\pgfsetfillcolor{currentfill}%
\pgfsetlinewidth{0.000000pt}%
\definecolor{currentstroke}{rgb}{1.000000,0.894118,0.788235}%
\pgfsetstrokecolor{currentstroke}%
\pgfsetdash{}{0pt}%
\pgfpathmoveto{\pgfqpoint{3.325787in}{2.941915in}}%
\pgfpathlineto{\pgfqpoint{3.298839in}{2.926357in}}%
\pgfpathlineto{\pgfqpoint{3.325787in}{2.910799in}}%
\pgfpathlineto{\pgfqpoint{3.352734in}{2.926357in}}%
\pgfpathlineto{\pgfqpoint{3.325787in}{2.941915in}}%
\pgfpathclose%
\pgfusepath{fill}%
\end{pgfscope}%
\begin{pgfscope}%
\pgfpathrectangle{\pgfqpoint{0.765000in}{0.660000in}}{\pgfqpoint{4.620000in}{4.620000in}}%
\pgfusepath{clip}%
\pgfsetbuttcap%
\pgfsetroundjoin%
\definecolor{currentfill}{rgb}{1.000000,0.894118,0.788235}%
\pgfsetfillcolor{currentfill}%
\pgfsetlinewidth{0.000000pt}%
\definecolor{currentstroke}{rgb}{1.000000,0.894118,0.788235}%
\pgfsetstrokecolor{currentstroke}%
\pgfsetdash{}{0pt}%
\pgfpathmoveto{\pgfqpoint{3.325787in}{2.879683in}}%
\pgfpathlineto{\pgfqpoint{3.352734in}{2.895241in}}%
\pgfpathlineto{\pgfqpoint{3.352734in}{2.926357in}}%
\pgfpathlineto{\pgfqpoint{3.325787in}{2.910799in}}%
\pgfpathlineto{\pgfqpoint{3.325787in}{2.879683in}}%
\pgfpathclose%
\pgfusepath{fill}%
\end{pgfscope}%
\begin{pgfscope}%
\pgfpathrectangle{\pgfqpoint{0.765000in}{0.660000in}}{\pgfqpoint{4.620000in}{4.620000in}}%
\pgfusepath{clip}%
\pgfsetbuttcap%
\pgfsetroundjoin%
\definecolor{currentfill}{rgb}{1.000000,0.894118,0.788235}%
\pgfsetfillcolor{currentfill}%
\pgfsetlinewidth{0.000000pt}%
\definecolor{currentstroke}{rgb}{1.000000,0.894118,0.788235}%
\pgfsetstrokecolor{currentstroke}%
\pgfsetdash{}{0pt}%
\pgfpathmoveto{\pgfqpoint{3.298839in}{2.895241in}}%
\pgfpathlineto{\pgfqpoint{3.325787in}{2.879683in}}%
\pgfpathlineto{\pgfqpoint{3.325787in}{2.910799in}}%
\pgfpathlineto{\pgfqpoint{3.298839in}{2.926357in}}%
\pgfpathlineto{\pgfqpoint{3.298839in}{2.895241in}}%
\pgfpathclose%
\pgfusepath{fill}%
\end{pgfscope}%
\begin{pgfscope}%
\pgfpathrectangle{\pgfqpoint{0.765000in}{0.660000in}}{\pgfqpoint{4.620000in}{4.620000in}}%
\pgfusepath{clip}%
\pgfsetbuttcap%
\pgfsetroundjoin%
\definecolor{currentfill}{rgb}{1.000000,0.894118,0.788235}%
\pgfsetfillcolor{currentfill}%
\pgfsetlinewidth{0.000000pt}%
\definecolor{currentstroke}{rgb}{1.000000,0.894118,0.788235}%
\pgfsetstrokecolor{currentstroke}%
\pgfsetdash{}{0pt}%
\pgfpathmoveto{\pgfqpoint{3.538039in}{2.783646in}}%
\pgfpathlineto{\pgfqpoint{3.511092in}{2.768088in}}%
\pgfpathlineto{\pgfqpoint{3.298839in}{2.895241in}}%
\pgfpathlineto{\pgfqpoint{3.325787in}{2.910799in}}%
\pgfpathlineto{\pgfqpoint{3.538039in}{2.783646in}}%
\pgfpathclose%
\pgfusepath{fill}%
\end{pgfscope}%
\begin{pgfscope}%
\pgfpathrectangle{\pgfqpoint{0.765000in}{0.660000in}}{\pgfqpoint{4.620000in}{4.620000in}}%
\pgfusepath{clip}%
\pgfsetbuttcap%
\pgfsetroundjoin%
\definecolor{currentfill}{rgb}{1.000000,0.894118,0.788235}%
\pgfsetfillcolor{currentfill}%
\pgfsetlinewidth{0.000000pt}%
\definecolor{currentstroke}{rgb}{1.000000,0.894118,0.788235}%
\pgfsetstrokecolor{currentstroke}%
\pgfsetdash{}{0pt}%
\pgfpathmoveto{\pgfqpoint{3.564986in}{2.768088in}}%
\pgfpathlineto{\pgfqpoint{3.538039in}{2.783646in}}%
\pgfpathlineto{\pgfqpoint{3.325787in}{2.910799in}}%
\pgfpathlineto{\pgfqpoint{3.352734in}{2.895241in}}%
\pgfpathlineto{\pgfqpoint{3.564986in}{2.768088in}}%
\pgfpathclose%
\pgfusepath{fill}%
\end{pgfscope}%
\begin{pgfscope}%
\pgfpathrectangle{\pgfqpoint{0.765000in}{0.660000in}}{\pgfqpoint{4.620000in}{4.620000in}}%
\pgfusepath{clip}%
\pgfsetbuttcap%
\pgfsetroundjoin%
\definecolor{currentfill}{rgb}{1.000000,0.894118,0.788235}%
\pgfsetfillcolor{currentfill}%
\pgfsetlinewidth{0.000000pt}%
\definecolor{currentstroke}{rgb}{1.000000,0.894118,0.788235}%
\pgfsetstrokecolor{currentstroke}%
\pgfsetdash{}{0pt}%
\pgfpathmoveto{\pgfqpoint{3.538039in}{2.783646in}}%
\pgfpathlineto{\pgfqpoint{3.538039in}{2.814762in}}%
\pgfpathlineto{\pgfqpoint{3.325787in}{2.941915in}}%
\pgfpathlineto{\pgfqpoint{3.352734in}{2.895241in}}%
\pgfpathlineto{\pgfqpoint{3.538039in}{2.783646in}}%
\pgfpathclose%
\pgfusepath{fill}%
\end{pgfscope}%
\begin{pgfscope}%
\pgfpathrectangle{\pgfqpoint{0.765000in}{0.660000in}}{\pgfqpoint{4.620000in}{4.620000in}}%
\pgfusepath{clip}%
\pgfsetbuttcap%
\pgfsetroundjoin%
\definecolor{currentfill}{rgb}{1.000000,0.894118,0.788235}%
\pgfsetfillcolor{currentfill}%
\pgfsetlinewidth{0.000000pt}%
\definecolor{currentstroke}{rgb}{1.000000,0.894118,0.788235}%
\pgfsetstrokecolor{currentstroke}%
\pgfsetdash{}{0pt}%
\pgfpathmoveto{\pgfqpoint{3.564986in}{2.768088in}}%
\pgfpathlineto{\pgfqpoint{3.564986in}{2.799204in}}%
\pgfpathlineto{\pgfqpoint{3.352734in}{2.926357in}}%
\pgfpathlineto{\pgfqpoint{3.325787in}{2.910799in}}%
\pgfpathlineto{\pgfqpoint{3.564986in}{2.768088in}}%
\pgfpathclose%
\pgfusepath{fill}%
\end{pgfscope}%
\begin{pgfscope}%
\pgfpathrectangle{\pgfqpoint{0.765000in}{0.660000in}}{\pgfqpoint{4.620000in}{4.620000in}}%
\pgfusepath{clip}%
\pgfsetbuttcap%
\pgfsetroundjoin%
\definecolor{currentfill}{rgb}{1.000000,0.894118,0.788235}%
\pgfsetfillcolor{currentfill}%
\pgfsetlinewidth{0.000000pt}%
\definecolor{currentstroke}{rgb}{1.000000,0.894118,0.788235}%
\pgfsetstrokecolor{currentstroke}%
\pgfsetdash{}{0pt}%
\pgfpathmoveto{\pgfqpoint{3.511092in}{2.768088in}}%
\pgfpathlineto{\pgfqpoint{3.538039in}{2.752530in}}%
\pgfpathlineto{\pgfqpoint{3.325787in}{2.879683in}}%
\pgfpathlineto{\pgfqpoint{3.298839in}{2.895241in}}%
\pgfpathlineto{\pgfqpoint{3.511092in}{2.768088in}}%
\pgfpathclose%
\pgfusepath{fill}%
\end{pgfscope}%
\begin{pgfscope}%
\pgfpathrectangle{\pgfqpoint{0.765000in}{0.660000in}}{\pgfqpoint{4.620000in}{4.620000in}}%
\pgfusepath{clip}%
\pgfsetbuttcap%
\pgfsetroundjoin%
\definecolor{currentfill}{rgb}{1.000000,0.894118,0.788235}%
\pgfsetfillcolor{currentfill}%
\pgfsetlinewidth{0.000000pt}%
\definecolor{currentstroke}{rgb}{1.000000,0.894118,0.788235}%
\pgfsetstrokecolor{currentstroke}%
\pgfsetdash{}{0pt}%
\pgfpathmoveto{\pgfqpoint{3.538039in}{2.752530in}}%
\pgfpathlineto{\pgfqpoint{3.564986in}{2.768088in}}%
\pgfpathlineto{\pgfqpoint{3.352734in}{2.895241in}}%
\pgfpathlineto{\pgfqpoint{3.325787in}{2.879683in}}%
\pgfpathlineto{\pgfqpoint{3.538039in}{2.752530in}}%
\pgfpathclose%
\pgfusepath{fill}%
\end{pgfscope}%
\begin{pgfscope}%
\pgfpathrectangle{\pgfqpoint{0.765000in}{0.660000in}}{\pgfqpoint{4.620000in}{4.620000in}}%
\pgfusepath{clip}%
\pgfsetbuttcap%
\pgfsetroundjoin%
\definecolor{currentfill}{rgb}{1.000000,0.894118,0.788235}%
\pgfsetfillcolor{currentfill}%
\pgfsetlinewidth{0.000000pt}%
\definecolor{currentstroke}{rgb}{1.000000,0.894118,0.788235}%
\pgfsetstrokecolor{currentstroke}%
\pgfsetdash{}{0pt}%
\pgfpathmoveto{\pgfqpoint{3.538039in}{2.814762in}}%
\pgfpathlineto{\pgfqpoint{3.511092in}{2.799204in}}%
\pgfpathlineto{\pgfqpoint{3.298839in}{2.926357in}}%
\pgfpathlineto{\pgfqpoint{3.325787in}{2.941915in}}%
\pgfpathlineto{\pgfqpoint{3.538039in}{2.814762in}}%
\pgfpathclose%
\pgfusepath{fill}%
\end{pgfscope}%
\begin{pgfscope}%
\pgfpathrectangle{\pgfqpoint{0.765000in}{0.660000in}}{\pgfqpoint{4.620000in}{4.620000in}}%
\pgfusepath{clip}%
\pgfsetbuttcap%
\pgfsetroundjoin%
\definecolor{currentfill}{rgb}{1.000000,0.894118,0.788235}%
\pgfsetfillcolor{currentfill}%
\pgfsetlinewidth{0.000000pt}%
\definecolor{currentstroke}{rgb}{1.000000,0.894118,0.788235}%
\pgfsetstrokecolor{currentstroke}%
\pgfsetdash{}{0pt}%
\pgfpathmoveto{\pgfqpoint{3.564986in}{2.799204in}}%
\pgfpathlineto{\pgfqpoint{3.538039in}{2.814762in}}%
\pgfpathlineto{\pgfqpoint{3.325787in}{2.941915in}}%
\pgfpathlineto{\pgfqpoint{3.352734in}{2.926357in}}%
\pgfpathlineto{\pgfqpoint{3.564986in}{2.799204in}}%
\pgfpathclose%
\pgfusepath{fill}%
\end{pgfscope}%
\begin{pgfscope}%
\pgfpathrectangle{\pgfqpoint{0.765000in}{0.660000in}}{\pgfqpoint{4.620000in}{4.620000in}}%
\pgfusepath{clip}%
\pgfsetbuttcap%
\pgfsetroundjoin%
\definecolor{currentfill}{rgb}{1.000000,0.894118,0.788235}%
\pgfsetfillcolor{currentfill}%
\pgfsetlinewidth{0.000000pt}%
\definecolor{currentstroke}{rgb}{1.000000,0.894118,0.788235}%
\pgfsetstrokecolor{currentstroke}%
\pgfsetdash{}{0pt}%
\pgfpathmoveto{\pgfqpoint{3.511092in}{2.768088in}}%
\pgfpathlineto{\pgfqpoint{3.511092in}{2.799204in}}%
\pgfpathlineto{\pgfqpoint{3.298839in}{2.926357in}}%
\pgfpathlineto{\pgfqpoint{3.325787in}{2.879683in}}%
\pgfpathlineto{\pgfqpoint{3.511092in}{2.768088in}}%
\pgfpathclose%
\pgfusepath{fill}%
\end{pgfscope}%
\begin{pgfscope}%
\pgfpathrectangle{\pgfqpoint{0.765000in}{0.660000in}}{\pgfqpoint{4.620000in}{4.620000in}}%
\pgfusepath{clip}%
\pgfsetbuttcap%
\pgfsetroundjoin%
\definecolor{currentfill}{rgb}{1.000000,0.894118,0.788235}%
\pgfsetfillcolor{currentfill}%
\pgfsetlinewidth{0.000000pt}%
\definecolor{currentstroke}{rgb}{1.000000,0.894118,0.788235}%
\pgfsetstrokecolor{currentstroke}%
\pgfsetdash{}{0pt}%
\pgfpathmoveto{\pgfqpoint{3.538039in}{2.752530in}}%
\pgfpathlineto{\pgfqpoint{3.538039in}{2.783646in}}%
\pgfpathlineto{\pgfqpoint{3.325787in}{2.910799in}}%
\pgfpathlineto{\pgfqpoint{3.298839in}{2.895241in}}%
\pgfpathlineto{\pgfqpoint{3.538039in}{2.752530in}}%
\pgfpathclose%
\pgfusepath{fill}%
\end{pgfscope}%
\begin{pgfscope}%
\pgfpathrectangle{\pgfqpoint{0.765000in}{0.660000in}}{\pgfqpoint{4.620000in}{4.620000in}}%
\pgfusepath{clip}%
\pgfsetbuttcap%
\pgfsetroundjoin%
\definecolor{currentfill}{rgb}{1.000000,0.894118,0.788235}%
\pgfsetfillcolor{currentfill}%
\pgfsetlinewidth{0.000000pt}%
\definecolor{currentstroke}{rgb}{1.000000,0.894118,0.788235}%
\pgfsetstrokecolor{currentstroke}%
\pgfsetdash{}{0pt}%
\pgfpathmoveto{\pgfqpoint{3.538039in}{2.783646in}}%
\pgfpathlineto{\pgfqpoint{3.564986in}{2.799204in}}%
\pgfpathlineto{\pgfqpoint{3.352734in}{2.926357in}}%
\pgfpathlineto{\pgfqpoint{3.325787in}{2.910799in}}%
\pgfpathlineto{\pgfqpoint{3.538039in}{2.783646in}}%
\pgfpathclose%
\pgfusepath{fill}%
\end{pgfscope}%
\begin{pgfscope}%
\pgfpathrectangle{\pgfqpoint{0.765000in}{0.660000in}}{\pgfqpoint{4.620000in}{4.620000in}}%
\pgfusepath{clip}%
\pgfsetbuttcap%
\pgfsetroundjoin%
\definecolor{currentfill}{rgb}{1.000000,0.894118,0.788235}%
\pgfsetfillcolor{currentfill}%
\pgfsetlinewidth{0.000000pt}%
\definecolor{currentstroke}{rgb}{1.000000,0.894118,0.788235}%
\pgfsetstrokecolor{currentstroke}%
\pgfsetdash{}{0pt}%
\pgfpathmoveto{\pgfqpoint{3.511092in}{2.799204in}}%
\pgfpathlineto{\pgfqpoint{3.538039in}{2.783646in}}%
\pgfpathlineto{\pgfqpoint{3.325787in}{2.910799in}}%
\pgfpathlineto{\pgfqpoint{3.298839in}{2.926357in}}%
\pgfpathlineto{\pgfqpoint{3.511092in}{2.799204in}}%
\pgfpathclose%
\pgfusepath{fill}%
\end{pgfscope}%
\begin{pgfscope}%
\pgfpathrectangle{\pgfqpoint{0.765000in}{0.660000in}}{\pgfqpoint{4.620000in}{4.620000in}}%
\pgfusepath{clip}%
\pgfsetbuttcap%
\pgfsetroundjoin%
\definecolor{currentfill}{rgb}{1.000000,0.894118,0.788235}%
\pgfsetfillcolor{currentfill}%
\pgfsetlinewidth{0.000000pt}%
\definecolor{currentstroke}{rgb}{1.000000,0.894118,0.788235}%
\pgfsetstrokecolor{currentstroke}%
\pgfsetdash{}{0pt}%
\pgfpathmoveto{\pgfqpoint{3.538039in}{2.783646in}}%
\pgfpathlineto{\pgfqpoint{3.511092in}{2.768088in}}%
\pgfpathlineto{\pgfqpoint{3.538039in}{2.752530in}}%
\pgfpathlineto{\pgfqpoint{3.564986in}{2.768088in}}%
\pgfpathlineto{\pgfqpoint{3.538039in}{2.783646in}}%
\pgfpathclose%
\pgfusepath{fill}%
\end{pgfscope}%
\begin{pgfscope}%
\pgfpathrectangle{\pgfqpoint{0.765000in}{0.660000in}}{\pgfqpoint{4.620000in}{4.620000in}}%
\pgfusepath{clip}%
\pgfsetbuttcap%
\pgfsetroundjoin%
\definecolor{currentfill}{rgb}{1.000000,0.894118,0.788235}%
\pgfsetfillcolor{currentfill}%
\pgfsetlinewidth{0.000000pt}%
\definecolor{currentstroke}{rgb}{1.000000,0.894118,0.788235}%
\pgfsetstrokecolor{currentstroke}%
\pgfsetdash{}{0pt}%
\pgfpathmoveto{\pgfqpoint{3.538039in}{2.783646in}}%
\pgfpathlineto{\pgfqpoint{3.511092in}{2.768088in}}%
\pgfpathlineto{\pgfqpoint{3.511092in}{2.799204in}}%
\pgfpathlineto{\pgfqpoint{3.538039in}{2.814762in}}%
\pgfpathlineto{\pgfqpoint{3.538039in}{2.783646in}}%
\pgfpathclose%
\pgfusepath{fill}%
\end{pgfscope}%
\begin{pgfscope}%
\pgfpathrectangle{\pgfqpoint{0.765000in}{0.660000in}}{\pgfqpoint{4.620000in}{4.620000in}}%
\pgfusepath{clip}%
\pgfsetbuttcap%
\pgfsetroundjoin%
\definecolor{currentfill}{rgb}{1.000000,0.894118,0.788235}%
\pgfsetfillcolor{currentfill}%
\pgfsetlinewidth{0.000000pt}%
\definecolor{currentstroke}{rgb}{1.000000,0.894118,0.788235}%
\pgfsetstrokecolor{currentstroke}%
\pgfsetdash{}{0pt}%
\pgfpathmoveto{\pgfqpoint{3.538039in}{2.783646in}}%
\pgfpathlineto{\pgfqpoint{3.564986in}{2.768088in}}%
\pgfpathlineto{\pgfqpoint{3.564986in}{2.799204in}}%
\pgfpathlineto{\pgfqpoint{3.538039in}{2.814762in}}%
\pgfpathlineto{\pgfqpoint{3.538039in}{2.783646in}}%
\pgfpathclose%
\pgfusepath{fill}%
\end{pgfscope}%
\begin{pgfscope}%
\pgfpathrectangle{\pgfqpoint{0.765000in}{0.660000in}}{\pgfqpoint{4.620000in}{4.620000in}}%
\pgfusepath{clip}%
\pgfsetbuttcap%
\pgfsetroundjoin%
\definecolor{currentfill}{rgb}{1.000000,0.894118,0.788235}%
\pgfsetfillcolor{currentfill}%
\pgfsetlinewidth{0.000000pt}%
\definecolor{currentstroke}{rgb}{1.000000,0.894118,0.788235}%
\pgfsetstrokecolor{currentstroke}%
\pgfsetdash{}{0pt}%
\pgfpathmoveto{\pgfqpoint{3.693066in}{2.834300in}}%
\pgfpathlineto{\pgfqpoint{3.666119in}{2.818742in}}%
\pgfpathlineto{\pgfqpoint{3.693066in}{2.803184in}}%
\pgfpathlineto{\pgfqpoint{3.720013in}{2.818742in}}%
\pgfpathlineto{\pgfqpoint{3.693066in}{2.834300in}}%
\pgfpathclose%
\pgfusepath{fill}%
\end{pgfscope}%
\begin{pgfscope}%
\pgfpathrectangle{\pgfqpoint{0.765000in}{0.660000in}}{\pgfqpoint{4.620000in}{4.620000in}}%
\pgfusepath{clip}%
\pgfsetbuttcap%
\pgfsetroundjoin%
\definecolor{currentfill}{rgb}{1.000000,0.894118,0.788235}%
\pgfsetfillcolor{currentfill}%
\pgfsetlinewidth{0.000000pt}%
\definecolor{currentstroke}{rgb}{1.000000,0.894118,0.788235}%
\pgfsetstrokecolor{currentstroke}%
\pgfsetdash{}{0pt}%
\pgfpathmoveto{\pgfqpoint{3.693066in}{2.834300in}}%
\pgfpathlineto{\pgfqpoint{3.666119in}{2.818742in}}%
\pgfpathlineto{\pgfqpoint{3.666119in}{2.849858in}}%
\pgfpathlineto{\pgfqpoint{3.693066in}{2.865416in}}%
\pgfpathlineto{\pgfqpoint{3.693066in}{2.834300in}}%
\pgfpathclose%
\pgfusepath{fill}%
\end{pgfscope}%
\begin{pgfscope}%
\pgfpathrectangle{\pgfqpoint{0.765000in}{0.660000in}}{\pgfqpoint{4.620000in}{4.620000in}}%
\pgfusepath{clip}%
\pgfsetbuttcap%
\pgfsetroundjoin%
\definecolor{currentfill}{rgb}{1.000000,0.894118,0.788235}%
\pgfsetfillcolor{currentfill}%
\pgfsetlinewidth{0.000000pt}%
\definecolor{currentstroke}{rgb}{1.000000,0.894118,0.788235}%
\pgfsetstrokecolor{currentstroke}%
\pgfsetdash{}{0pt}%
\pgfpathmoveto{\pgfqpoint{3.693066in}{2.834300in}}%
\pgfpathlineto{\pgfqpoint{3.720013in}{2.818742in}}%
\pgfpathlineto{\pgfqpoint{3.720013in}{2.849858in}}%
\pgfpathlineto{\pgfqpoint{3.693066in}{2.865416in}}%
\pgfpathlineto{\pgfqpoint{3.693066in}{2.834300in}}%
\pgfpathclose%
\pgfusepath{fill}%
\end{pgfscope}%
\begin{pgfscope}%
\pgfpathrectangle{\pgfqpoint{0.765000in}{0.660000in}}{\pgfqpoint{4.620000in}{4.620000in}}%
\pgfusepath{clip}%
\pgfsetbuttcap%
\pgfsetroundjoin%
\definecolor{currentfill}{rgb}{1.000000,0.894118,0.788235}%
\pgfsetfillcolor{currentfill}%
\pgfsetlinewidth{0.000000pt}%
\definecolor{currentstroke}{rgb}{1.000000,0.894118,0.788235}%
\pgfsetstrokecolor{currentstroke}%
\pgfsetdash{}{0pt}%
\pgfpathmoveto{\pgfqpoint{3.538039in}{2.814762in}}%
\pgfpathlineto{\pgfqpoint{3.511092in}{2.799204in}}%
\pgfpathlineto{\pgfqpoint{3.538039in}{2.783646in}}%
\pgfpathlineto{\pgfqpoint{3.564986in}{2.799204in}}%
\pgfpathlineto{\pgfqpoint{3.538039in}{2.814762in}}%
\pgfpathclose%
\pgfusepath{fill}%
\end{pgfscope}%
\begin{pgfscope}%
\pgfpathrectangle{\pgfqpoint{0.765000in}{0.660000in}}{\pgfqpoint{4.620000in}{4.620000in}}%
\pgfusepath{clip}%
\pgfsetbuttcap%
\pgfsetroundjoin%
\definecolor{currentfill}{rgb}{1.000000,0.894118,0.788235}%
\pgfsetfillcolor{currentfill}%
\pgfsetlinewidth{0.000000pt}%
\definecolor{currentstroke}{rgb}{1.000000,0.894118,0.788235}%
\pgfsetstrokecolor{currentstroke}%
\pgfsetdash{}{0pt}%
\pgfpathmoveto{\pgfqpoint{3.538039in}{2.752530in}}%
\pgfpathlineto{\pgfqpoint{3.564986in}{2.768088in}}%
\pgfpathlineto{\pgfqpoint{3.564986in}{2.799204in}}%
\pgfpathlineto{\pgfqpoint{3.538039in}{2.783646in}}%
\pgfpathlineto{\pgfqpoint{3.538039in}{2.752530in}}%
\pgfpathclose%
\pgfusepath{fill}%
\end{pgfscope}%
\begin{pgfscope}%
\pgfpathrectangle{\pgfqpoint{0.765000in}{0.660000in}}{\pgfqpoint{4.620000in}{4.620000in}}%
\pgfusepath{clip}%
\pgfsetbuttcap%
\pgfsetroundjoin%
\definecolor{currentfill}{rgb}{1.000000,0.894118,0.788235}%
\pgfsetfillcolor{currentfill}%
\pgfsetlinewidth{0.000000pt}%
\definecolor{currentstroke}{rgb}{1.000000,0.894118,0.788235}%
\pgfsetstrokecolor{currentstroke}%
\pgfsetdash{}{0pt}%
\pgfpathmoveto{\pgfqpoint{3.511092in}{2.768088in}}%
\pgfpathlineto{\pgfqpoint{3.538039in}{2.752530in}}%
\pgfpathlineto{\pgfqpoint{3.538039in}{2.783646in}}%
\pgfpathlineto{\pgfqpoint{3.511092in}{2.799204in}}%
\pgfpathlineto{\pgfqpoint{3.511092in}{2.768088in}}%
\pgfpathclose%
\pgfusepath{fill}%
\end{pgfscope}%
\begin{pgfscope}%
\pgfpathrectangle{\pgfqpoint{0.765000in}{0.660000in}}{\pgfqpoint{4.620000in}{4.620000in}}%
\pgfusepath{clip}%
\pgfsetbuttcap%
\pgfsetroundjoin%
\definecolor{currentfill}{rgb}{1.000000,0.894118,0.788235}%
\pgfsetfillcolor{currentfill}%
\pgfsetlinewidth{0.000000pt}%
\definecolor{currentstroke}{rgb}{1.000000,0.894118,0.788235}%
\pgfsetstrokecolor{currentstroke}%
\pgfsetdash{}{0pt}%
\pgfpathmoveto{\pgfqpoint{3.693066in}{2.865416in}}%
\pgfpathlineto{\pgfqpoint{3.666119in}{2.849858in}}%
\pgfpathlineto{\pgfqpoint{3.693066in}{2.834300in}}%
\pgfpathlineto{\pgfqpoint{3.720013in}{2.849858in}}%
\pgfpathlineto{\pgfqpoint{3.693066in}{2.865416in}}%
\pgfpathclose%
\pgfusepath{fill}%
\end{pgfscope}%
\begin{pgfscope}%
\pgfpathrectangle{\pgfqpoint{0.765000in}{0.660000in}}{\pgfqpoint{4.620000in}{4.620000in}}%
\pgfusepath{clip}%
\pgfsetbuttcap%
\pgfsetroundjoin%
\definecolor{currentfill}{rgb}{1.000000,0.894118,0.788235}%
\pgfsetfillcolor{currentfill}%
\pgfsetlinewidth{0.000000pt}%
\definecolor{currentstroke}{rgb}{1.000000,0.894118,0.788235}%
\pgfsetstrokecolor{currentstroke}%
\pgfsetdash{}{0pt}%
\pgfpathmoveto{\pgfqpoint{3.693066in}{2.803184in}}%
\pgfpathlineto{\pgfqpoint{3.720013in}{2.818742in}}%
\pgfpathlineto{\pgfqpoint{3.720013in}{2.849858in}}%
\pgfpathlineto{\pgfqpoint{3.693066in}{2.834300in}}%
\pgfpathlineto{\pgfqpoint{3.693066in}{2.803184in}}%
\pgfpathclose%
\pgfusepath{fill}%
\end{pgfscope}%
\begin{pgfscope}%
\pgfpathrectangle{\pgfqpoint{0.765000in}{0.660000in}}{\pgfqpoint{4.620000in}{4.620000in}}%
\pgfusepath{clip}%
\pgfsetbuttcap%
\pgfsetroundjoin%
\definecolor{currentfill}{rgb}{1.000000,0.894118,0.788235}%
\pgfsetfillcolor{currentfill}%
\pgfsetlinewidth{0.000000pt}%
\definecolor{currentstroke}{rgb}{1.000000,0.894118,0.788235}%
\pgfsetstrokecolor{currentstroke}%
\pgfsetdash{}{0pt}%
\pgfpathmoveto{\pgfqpoint{3.666119in}{2.818742in}}%
\pgfpathlineto{\pgfqpoint{3.693066in}{2.803184in}}%
\pgfpathlineto{\pgfqpoint{3.693066in}{2.834300in}}%
\pgfpathlineto{\pgfqpoint{3.666119in}{2.849858in}}%
\pgfpathlineto{\pgfqpoint{3.666119in}{2.818742in}}%
\pgfpathclose%
\pgfusepath{fill}%
\end{pgfscope}%
\begin{pgfscope}%
\pgfpathrectangle{\pgfqpoint{0.765000in}{0.660000in}}{\pgfqpoint{4.620000in}{4.620000in}}%
\pgfusepath{clip}%
\pgfsetbuttcap%
\pgfsetroundjoin%
\definecolor{currentfill}{rgb}{1.000000,0.894118,0.788235}%
\pgfsetfillcolor{currentfill}%
\pgfsetlinewidth{0.000000pt}%
\definecolor{currentstroke}{rgb}{1.000000,0.894118,0.788235}%
\pgfsetstrokecolor{currentstroke}%
\pgfsetdash{}{0pt}%
\pgfpathmoveto{\pgfqpoint{3.538039in}{2.783646in}}%
\pgfpathlineto{\pgfqpoint{3.511092in}{2.768088in}}%
\pgfpathlineto{\pgfqpoint{3.666119in}{2.818742in}}%
\pgfpathlineto{\pgfqpoint{3.693066in}{2.834300in}}%
\pgfpathlineto{\pgfqpoint{3.538039in}{2.783646in}}%
\pgfpathclose%
\pgfusepath{fill}%
\end{pgfscope}%
\begin{pgfscope}%
\pgfpathrectangle{\pgfqpoint{0.765000in}{0.660000in}}{\pgfqpoint{4.620000in}{4.620000in}}%
\pgfusepath{clip}%
\pgfsetbuttcap%
\pgfsetroundjoin%
\definecolor{currentfill}{rgb}{1.000000,0.894118,0.788235}%
\pgfsetfillcolor{currentfill}%
\pgfsetlinewidth{0.000000pt}%
\definecolor{currentstroke}{rgb}{1.000000,0.894118,0.788235}%
\pgfsetstrokecolor{currentstroke}%
\pgfsetdash{}{0pt}%
\pgfpathmoveto{\pgfqpoint{3.564986in}{2.768088in}}%
\pgfpathlineto{\pgfqpoint{3.538039in}{2.783646in}}%
\pgfpathlineto{\pgfqpoint{3.693066in}{2.834300in}}%
\pgfpathlineto{\pgfqpoint{3.720013in}{2.818742in}}%
\pgfpathlineto{\pgfqpoint{3.564986in}{2.768088in}}%
\pgfpathclose%
\pgfusepath{fill}%
\end{pgfscope}%
\begin{pgfscope}%
\pgfpathrectangle{\pgfqpoint{0.765000in}{0.660000in}}{\pgfqpoint{4.620000in}{4.620000in}}%
\pgfusepath{clip}%
\pgfsetbuttcap%
\pgfsetroundjoin%
\definecolor{currentfill}{rgb}{1.000000,0.894118,0.788235}%
\pgfsetfillcolor{currentfill}%
\pgfsetlinewidth{0.000000pt}%
\definecolor{currentstroke}{rgb}{1.000000,0.894118,0.788235}%
\pgfsetstrokecolor{currentstroke}%
\pgfsetdash{}{0pt}%
\pgfpathmoveto{\pgfqpoint{3.538039in}{2.783646in}}%
\pgfpathlineto{\pgfqpoint{3.538039in}{2.814762in}}%
\pgfpathlineto{\pgfqpoint{3.693066in}{2.865416in}}%
\pgfpathlineto{\pgfqpoint{3.720013in}{2.818742in}}%
\pgfpathlineto{\pgfqpoint{3.538039in}{2.783646in}}%
\pgfpathclose%
\pgfusepath{fill}%
\end{pgfscope}%
\begin{pgfscope}%
\pgfpathrectangle{\pgfqpoint{0.765000in}{0.660000in}}{\pgfqpoint{4.620000in}{4.620000in}}%
\pgfusepath{clip}%
\pgfsetbuttcap%
\pgfsetroundjoin%
\definecolor{currentfill}{rgb}{1.000000,0.894118,0.788235}%
\pgfsetfillcolor{currentfill}%
\pgfsetlinewidth{0.000000pt}%
\definecolor{currentstroke}{rgb}{1.000000,0.894118,0.788235}%
\pgfsetstrokecolor{currentstroke}%
\pgfsetdash{}{0pt}%
\pgfpathmoveto{\pgfqpoint{3.564986in}{2.768088in}}%
\pgfpathlineto{\pgfqpoint{3.564986in}{2.799204in}}%
\pgfpathlineto{\pgfqpoint{3.720013in}{2.849858in}}%
\pgfpathlineto{\pgfqpoint{3.693066in}{2.834300in}}%
\pgfpathlineto{\pgfqpoint{3.564986in}{2.768088in}}%
\pgfpathclose%
\pgfusepath{fill}%
\end{pgfscope}%
\begin{pgfscope}%
\pgfpathrectangle{\pgfqpoint{0.765000in}{0.660000in}}{\pgfqpoint{4.620000in}{4.620000in}}%
\pgfusepath{clip}%
\pgfsetbuttcap%
\pgfsetroundjoin%
\definecolor{currentfill}{rgb}{1.000000,0.894118,0.788235}%
\pgfsetfillcolor{currentfill}%
\pgfsetlinewidth{0.000000pt}%
\definecolor{currentstroke}{rgb}{1.000000,0.894118,0.788235}%
\pgfsetstrokecolor{currentstroke}%
\pgfsetdash{}{0pt}%
\pgfpathmoveto{\pgfqpoint{3.511092in}{2.768088in}}%
\pgfpathlineto{\pgfqpoint{3.538039in}{2.752530in}}%
\pgfpathlineto{\pgfqpoint{3.693066in}{2.803184in}}%
\pgfpathlineto{\pgfqpoint{3.666119in}{2.818742in}}%
\pgfpathlineto{\pgfqpoint{3.511092in}{2.768088in}}%
\pgfpathclose%
\pgfusepath{fill}%
\end{pgfscope}%
\begin{pgfscope}%
\pgfpathrectangle{\pgfqpoint{0.765000in}{0.660000in}}{\pgfqpoint{4.620000in}{4.620000in}}%
\pgfusepath{clip}%
\pgfsetbuttcap%
\pgfsetroundjoin%
\definecolor{currentfill}{rgb}{1.000000,0.894118,0.788235}%
\pgfsetfillcolor{currentfill}%
\pgfsetlinewidth{0.000000pt}%
\definecolor{currentstroke}{rgb}{1.000000,0.894118,0.788235}%
\pgfsetstrokecolor{currentstroke}%
\pgfsetdash{}{0pt}%
\pgfpathmoveto{\pgfqpoint{3.538039in}{2.752530in}}%
\pgfpathlineto{\pgfqpoint{3.564986in}{2.768088in}}%
\pgfpathlineto{\pgfqpoint{3.720013in}{2.818742in}}%
\pgfpathlineto{\pgfqpoint{3.693066in}{2.803184in}}%
\pgfpathlineto{\pgfqpoint{3.538039in}{2.752530in}}%
\pgfpathclose%
\pgfusepath{fill}%
\end{pgfscope}%
\begin{pgfscope}%
\pgfpathrectangle{\pgfqpoint{0.765000in}{0.660000in}}{\pgfqpoint{4.620000in}{4.620000in}}%
\pgfusepath{clip}%
\pgfsetbuttcap%
\pgfsetroundjoin%
\definecolor{currentfill}{rgb}{1.000000,0.894118,0.788235}%
\pgfsetfillcolor{currentfill}%
\pgfsetlinewidth{0.000000pt}%
\definecolor{currentstroke}{rgb}{1.000000,0.894118,0.788235}%
\pgfsetstrokecolor{currentstroke}%
\pgfsetdash{}{0pt}%
\pgfpathmoveto{\pgfqpoint{3.538039in}{2.814762in}}%
\pgfpathlineto{\pgfqpoint{3.511092in}{2.799204in}}%
\pgfpathlineto{\pgfqpoint{3.666119in}{2.849858in}}%
\pgfpathlineto{\pgfqpoint{3.693066in}{2.865416in}}%
\pgfpathlineto{\pgfqpoint{3.538039in}{2.814762in}}%
\pgfpathclose%
\pgfusepath{fill}%
\end{pgfscope}%
\begin{pgfscope}%
\pgfpathrectangle{\pgfqpoint{0.765000in}{0.660000in}}{\pgfqpoint{4.620000in}{4.620000in}}%
\pgfusepath{clip}%
\pgfsetbuttcap%
\pgfsetroundjoin%
\definecolor{currentfill}{rgb}{1.000000,0.894118,0.788235}%
\pgfsetfillcolor{currentfill}%
\pgfsetlinewidth{0.000000pt}%
\definecolor{currentstroke}{rgb}{1.000000,0.894118,0.788235}%
\pgfsetstrokecolor{currentstroke}%
\pgfsetdash{}{0pt}%
\pgfpathmoveto{\pgfqpoint{3.564986in}{2.799204in}}%
\pgfpathlineto{\pgfqpoint{3.538039in}{2.814762in}}%
\pgfpathlineto{\pgfqpoint{3.693066in}{2.865416in}}%
\pgfpathlineto{\pgfqpoint{3.720013in}{2.849858in}}%
\pgfpathlineto{\pgfqpoint{3.564986in}{2.799204in}}%
\pgfpathclose%
\pgfusepath{fill}%
\end{pgfscope}%
\begin{pgfscope}%
\pgfpathrectangle{\pgfqpoint{0.765000in}{0.660000in}}{\pgfqpoint{4.620000in}{4.620000in}}%
\pgfusepath{clip}%
\pgfsetbuttcap%
\pgfsetroundjoin%
\definecolor{currentfill}{rgb}{1.000000,0.894118,0.788235}%
\pgfsetfillcolor{currentfill}%
\pgfsetlinewidth{0.000000pt}%
\definecolor{currentstroke}{rgb}{1.000000,0.894118,0.788235}%
\pgfsetstrokecolor{currentstroke}%
\pgfsetdash{}{0pt}%
\pgfpathmoveto{\pgfqpoint{3.511092in}{2.768088in}}%
\pgfpathlineto{\pgfqpoint{3.511092in}{2.799204in}}%
\pgfpathlineto{\pgfqpoint{3.666119in}{2.849858in}}%
\pgfpathlineto{\pgfqpoint{3.693066in}{2.803184in}}%
\pgfpathlineto{\pgfqpoint{3.511092in}{2.768088in}}%
\pgfpathclose%
\pgfusepath{fill}%
\end{pgfscope}%
\begin{pgfscope}%
\pgfpathrectangle{\pgfqpoint{0.765000in}{0.660000in}}{\pgfqpoint{4.620000in}{4.620000in}}%
\pgfusepath{clip}%
\pgfsetbuttcap%
\pgfsetroundjoin%
\definecolor{currentfill}{rgb}{1.000000,0.894118,0.788235}%
\pgfsetfillcolor{currentfill}%
\pgfsetlinewidth{0.000000pt}%
\definecolor{currentstroke}{rgb}{1.000000,0.894118,0.788235}%
\pgfsetstrokecolor{currentstroke}%
\pgfsetdash{}{0pt}%
\pgfpathmoveto{\pgfqpoint{3.538039in}{2.752530in}}%
\pgfpathlineto{\pgfqpoint{3.538039in}{2.783646in}}%
\pgfpathlineto{\pgfqpoint{3.693066in}{2.834300in}}%
\pgfpathlineto{\pgfqpoint{3.666119in}{2.818742in}}%
\pgfpathlineto{\pgfqpoint{3.538039in}{2.752530in}}%
\pgfpathclose%
\pgfusepath{fill}%
\end{pgfscope}%
\begin{pgfscope}%
\pgfpathrectangle{\pgfqpoint{0.765000in}{0.660000in}}{\pgfqpoint{4.620000in}{4.620000in}}%
\pgfusepath{clip}%
\pgfsetbuttcap%
\pgfsetroundjoin%
\definecolor{currentfill}{rgb}{1.000000,0.894118,0.788235}%
\pgfsetfillcolor{currentfill}%
\pgfsetlinewidth{0.000000pt}%
\definecolor{currentstroke}{rgb}{1.000000,0.894118,0.788235}%
\pgfsetstrokecolor{currentstroke}%
\pgfsetdash{}{0pt}%
\pgfpathmoveto{\pgfqpoint{3.538039in}{2.783646in}}%
\pgfpathlineto{\pgfqpoint{3.564986in}{2.799204in}}%
\pgfpathlineto{\pgfqpoint{3.720013in}{2.849858in}}%
\pgfpathlineto{\pgfqpoint{3.693066in}{2.834300in}}%
\pgfpathlineto{\pgfqpoint{3.538039in}{2.783646in}}%
\pgfpathclose%
\pgfusepath{fill}%
\end{pgfscope}%
\begin{pgfscope}%
\pgfpathrectangle{\pgfqpoint{0.765000in}{0.660000in}}{\pgfqpoint{4.620000in}{4.620000in}}%
\pgfusepath{clip}%
\pgfsetbuttcap%
\pgfsetroundjoin%
\definecolor{currentfill}{rgb}{1.000000,0.894118,0.788235}%
\pgfsetfillcolor{currentfill}%
\pgfsetlinewidth{0.000000pt}%
\definecolor{currentstroke}{rgb}{1.000000,0.894118,0.788235}%
\pgfsetstrokecolor{currentstroke}%
\pgfsetdash{}{0pt}%
\pgfpathmoveto{\pgfqpoint{3.511092in}{2.799204in}}%
\pgfpathlineto{\pgfqpoint{3.538039in}{2.783646in}}%
\pgfpathlineto{\pgfqpoint{3.693066in}{2.834300in}}%
\pgfpathlineto{\pgfqpoint{3.666119in}{2.849858in}}%
\pgfpathlineto{\pgfqpoint{3.511092in}{2.799204in}}%
\pgfpathclose%
\pgfusepath{fill}%
\end{pgfscope}%
\begin{pgfscope}%
\pgfpathrectangle{\pgfqpoint{0.765000in}{0.660000in}}{\pgfqpoint{4.620000in}{4.620000in}}%
\pgfusepath{clip}%
\pgfsetbuttcap%
\pgfsetroundjoin%
\definecolor{currentfill}{rgb}{1.000000,0.894118,0.788235}%
\pgfsetfillcolor{currentfill}%
\pgfsetlinewidth{0.000000pt}%
\definecolor{currentstroke}{rgb}{1.000000,0.894118,0.788235}%
\pgfsetstrokecolor{currentstroke}%
\pgfsetdash{}{0pt}%
\pgfpathmoveto{\pgfqpoint{3.201279in}{3.052489in}}%
\pgfpathlineto{\pgfqpoint{3.174331in}{3.036931in}}%
\pgfpathlineto{\pgfqpoint{3.201279in}{3.021373in}}%
\pgfpathlineto{\pgfqpoint{3.228226in}{3.036931in}}%
\pgfpathlineto{\pgfqpoint{3.201279in}{3.052489in}}%
\pgfpathclose%
\pgfusepath{fill}%
\end{pgfscope}%
\begin{pgfscope}%
\pgfpathrectangle{\pgfqpoint{0.765000in}{0.660000in}}{\pgfqpoint{4.620000in}{4.620000in}}%
\pgfusepath{clip}%
\pgfsetbuttcap%
\pgfsetroundjoin%
\definecolor{currentfill}{rgb}{1.000000,0.894118,0.788235}%
\pgfsetfillcolor{currentfill}%
\pgfsetlinewidth{0.000000pt}%
\definecolor{currentstroke}{rgb}{1.000000,0.894118,0.788235}%
\pgfsetstrokecolor{currentstroke}%
\pgfsetdash{}{0pt}%
\pgfpathmoveto{\pgfqpoint{3.201279in}{3.052489in}}%
\pgfpathlineto{\pgfqpoint{3.174331in}{3.036931in}}%
\pgfpathlineto{\pgfqpoint{3.174331in}{3.068047in}}%
\pgfpathlineto{\pgfqpoint{3.201279in}{3.083605in}}%
\pgfpathlineto{\pgfqpoint{3.201279in}{3.052489in}}%
\pgfpathclose%
\pgfusepath{fill}%
\end{pgfscope}%
\begin{pgfscope}%
\pgfpathrectangle{\pgfqpoint{0.765000in}{0.660000in}}{\pgfqpoint{4.620000in}{4.620000in}}%
\pgfusepath{clip}%
\pgfsetbuttcap%
\pgfsetroundjoin%
\definecolor{currentfill}{rgb}{1.000000,0.894118,0.788235}%
\pgfsetfillcolor{currentfill}%
\pgfsetlinewidth{0.000000pt}%
\definecolor{currentstroke}{rgb}{1.000000,0.894118,0.788235}%
\pgfsetstrokecolor{currentstroke}%
\pgfsetdash{}{0pt}%
\pgfpathmoveto{\pgfqpoint{3.201279in}{3.052489in}}%
\pgfpathlineto{\pgfqpoint{3.228226in}{3.036931in}}%
\pgfpathlineto{\pgfqpoint{3.228226in}{3.068047in}}%
\pgfpathlineto{\pgfqpoint{3.201279in}{3.083605in}}%
\pgfpathlineto{\pgfqpoint{3.201279in}{3.052489in}}%
\pgfpathclose%
\pgfusepath{fill}%
\end{pgfscope}%
\begin{pgfscope}%
\pgfpathrectangle{\pgfqpoint{0.765000in}{0.660000in}}{\pgfqpoint{4.620000in}{4.620000in}}%
\pgfusepath{clip}%
\pgfsetbuttcap%
\pgfsetroundjoin%
\definecolor{currentfill}{rgb}{1.000000,0.894118,0.788235}%
\pgfsetfillcolor{currentfill}%
\pgfsetlinewidth{0.000000pt}%
\definecolor{currentstroke}{rgb}{1.000000,0.894118,0.788235}%
\pgfsetstrokecolor{currentstroke}%
\pgfsetdash{}{0pt}%
\pgfpathmoveto{\pgfqpoint{3.233018in}{3.305233in}}%
\pgfpathlineto{\pgfqpoint{3.206071in}{3.289675in}}%
\pgfpathlineto{\pgfqpoint{3.233018in}{3.274117in}}%
\pgfpathlineto{\pgfqpoint{3.259965in}{3.289675in}}%
\pgfpathlineto{\pgfqpoint{3.233018in}{3.305233in}}%
\pgfpathclose%
\pgfusepath{fill}%
\end{pgfscope}%
\begin{pgfscope}%
\pgfpathrectangle{\pgfqpoint{0.765000in}{0.660000in}}{\pgfqpoint{4.620000in}{4.620000in}}%
\pgfusepath{clip}%
\pgfsetbuttcap%
\pgfsetroundjoin%
\definecolor{currentfill}{rgb}{1.000000,0.894118,0.788235}%
\pgfsetfillcolor{currentfill}%
\pgfsetlinewidth{0.000000pt}%
\definecolor{currentstroke}{rgb}{1.000000,0.894118,0.788235}%
\pgfsetstrokecolor{currentstroke}%
\pgfsetdash{}{0pt}%
\pgfpathmoveto{\pgfqpoint{3.233018in}{3.305233in}}%
\pgfpathlineto{\pgfqpoint{3.206071in}{3.289675in}}%
\pgfpathlineto{\pgfqpoint{3.206071in}{3.320791in}}%
\pgfpathlineto{\pgfqpoint{3.233018in}{3.336349in}}%
\pgfpathlineto{\pgfqpoint{3.233018in}{3.305233in}}%
\pgfpathclose%
\pgfusepath{fill}%
\end{pgfscope}%
\begin{pgfscope}%
\pgfpathrectangle{\pgfqpoint{0.765000in}{0.660000in}}{\pgfqpoint{4.620000in}{4.620000in}}%
\pgfusepath{clip}%
\pgfsetbuttcap%
\pgfsetroundjoin%
\definecolor{currentfill}{rgb}{1.000000,0.894118,0.788235}%
\pgfsetfillcolor{currentfill}%
\pgfsetlinewidth{0.000000pt}%
\definecolor{currentstroke}{rgb}{1.000000,0.894118,0.788235}%
\pgfsetstrokecolor{currentstroke}%
\pgfsetdash{}{0pt}%
\pgfpathmoveto{\pgfqpoint{3.233018in}{3.305233in}}%
\pgfpathlineto{\pgfqpoint{3.259965in}{3.289675in}}%
\pgfpathlineto{\pgfqpoint{3.259965in}{3.320791in}}%
\pgfpathlineto{\pgfqpoint{3.233018in}{3.336349in}}%
\pgfpathlineto{\pgfqpoint{3.233018in}{3.305233in}}%
\pgfpathclose%
\pgfusepath{fill}%
\end{pgfscope}%
\begin{pgfscope}%
\pgfpathrectangle{\pgfqpoint{0.765000in}{0.660000in}}{\pgfqpoint{4.620000in}{4.620000in}}%
\pgfusepath{clip}%
\pgfsetbuttcap%
\pgfsetroundjoin%
\definecolor{currentfill}{rgb}{1.000000,0.894118,0.788235}%
\pgfsetfillcolor{currentfill}%
\pgfsetlinewidth{0.000000pt}%
\definecolor{currentstroke}{rgb}{1.000000,0.894118,0.788235}%
\pgfsetstrokecolor{currentstroke}%
\pgfsetdash{}{0pt}%
\pgfpathmoveto{\pgfqpoint{3.201279in}{3.083605in}}%
\pgfpathlineto{\pgfqpoint{3.174331in}{3.068047in}}%
\pgfpathlineto{\pgfqpoint{3.201279in}{3.052489in}}%
\pgfpathlineto{\pgfqpoint{3.228226in}{3.068047in}}%
\pgfpathlineto{\pgfqpoint{3.201279in}{3.083605in}}%
\pgfpathclose%
\pgfusepath{fill}%
\end{pgfscope}%
\begin{pgfscope}%
\pgfpathrectangle{\pgfqpoint{0.765000in}{0.660000in}}{\pgfqpoint{4.620000in}{4.620000in}}%
\pgfusepath{clip}%
\pgfsetbuttcap%
\pgfsetroundjoin%
\definecolor{currentfill}{rgb}{1.000000,0.894118,0.788235}%
\pgfsetfillcolor{currentfill}%
\pgfsetlinewidth{0.000000pt}%
\definecolor{currentstroke}{rgb}{1.000000,0.894118,0.788235}%
\pgfsetstrokecolor{currentstroke}%
\pgfsetdash{}{0pt}%
\pgfpathmoveto{\pgfqpoint{3.201279in}{3.021373in}}%
\pgfpathlineto{\pgfqpoint{3.228226in}{3.036931in}}%
\pgfpathlineto{\pgfqpoint{3.228226in}{3.068047in}}%
\pgfpathlineto{\pgfqpoint{3.201279in}{3.052489in}}%
\pgfpathlineto{\pgfqpoint{3.201279in}{3.021373in}}%
\pgfpathclose%
\pgfusepath{fill}%
\end{pgfscope}%
\begin{pgfscope}%
\pgfpathrectangle{\pgfqpoint{0.765000in}{0.660000in}}{\pgfqpoint{4.620000in}{4.620000in}}%
\pgfusepath{clip}%
\pgfsetbuttcap%
\pgfsetroundjoin%
\definecolor{currentfill}{rgb}{1.000000,0.894118,0.788235}%
\pgfsetfillcolor{currentfill}%
\pgfsetlinewidth{0.000000pt}%
\definecolor{currentstroke}{rgb}{1.000000,0.894118,0.788235}%
\pgfsetstrokecolor{currentstroke}%
\pgfsetdash{}{0pt}%
\pgfpathmoveto{\pgfqpoint{3.174331in}{3.036931in}}%
\pgfpathlineto{\pgfqpoint{3.201279in}{3.021373in}}%
\pgfpathlineto{\pgfqpoint{3.201279in}{3.052489in}}%
\pgfpathlineto{\pgfqpoint{3.174331in}{3.068047in}}%
\pgfpathlineto{\pgfqpoint{3.174331in}{3.036931in}}%
\pgfpathclose%
\pgfusepath{fill}%
\end{pgfscope}%
\begin{pgfscope}%
\pgfpathrectangle{\pgfqpoint{0.765000in}{0.660000in}}{\pgfqpoint{4.620000in}{4.620000in}}%
\pgfusepath{clip}%
\pgfsetbuttcap%
\pgfsetroundjoin%
\definecolor{currentfill}{rgb}{1.000000,0.894118,0.788235}%
\pgfsetfillcolor{currentfill}%
\pgfsetlinewidth{0.000000pt}%
\definecolor{currentstroke}{rgb}{1.000000,0.894118,0.788235}%
\pgfsetstrokecolor{currentstroke}%
\pgfsetdash{}{0pt}%
\pgfpathmoveto{\pgfqpoint{3.233018in}{3.336349in}}%
\pgfpathlineto{\pgfqpoint{3.206071in}{3.320791in}}%
\pgfpathlineto{\pgfqpoint{3.233018in}{3.305233in}}%
\pgfpathlineto{\pgfqpoint{3.259965in}{3.320791in}}%
\pgfpathlineto{\pgfqpoint{3.233018in}{3.336349in}}%
\pgfpathclose%
\pgfusepath{fill}%
\end{pgfscope}%
\begin{pgfscope}%
\pgfpathrectangle{\pgfqpoint{0.765000in}{0.660000in}}{\pgfqpoint{4.620000in}{4.620000in}}%
\pgfusepath{clip}%
\pgfsetbuttcap%
\pgfsetroundjoin%
\definecolor{currentfill}{rgb}{1.000000,0.894118,0.788235}%
\pgfsetfillcolor{currentfill}%
\pgfsetlinewidth{0.000000pt}%
\definecolor{currentstroke}{rgb}{1.000000,0.894118,0.788235}%
\pgfsetstrokecolor{currentstroke}%
\pgfsetdash{}{0pt}%
\pgfpathmoveto{\pgfqpoint{3.233018in}{3.274117in}}%
\pgfpathlineto{\pgfqpoint{3.259965in}{3.289675in}}%
\pgfpathlineto{\pgfqpoint{3.259965in}{3.320791in}}%
\pgfpathlineto{\pgfqpoint{3.233018in}{3.305233in}}%
\pgfpathlineto{\pgfqpoint{3.233018in}{3.274117in}}%
\pgfpathclose%
\pgfusepath{fill}%
\end{pgfscope}%
\begin{pgfscope}%
\pgfpathrectangle{\pgfqpoint{0.765000in}{0.660000in}}{\pgfqpoint{4.620000in}{4.620000in}}%
\pgfusepath{clip}%
\pgfsetbuttcap%
\pgfsetroundjoin%
\definecolor{currentfill}{rgb}{1.000000,0.894118,0.788235}%
\pgfsetfillcolor{currentfill}%
\pgfsetlinewidth{0.000000pt}%
\definecolor{currentstroke}{rgb}{1.000000,0.894118,0.788235}%
\pgfsetstrokecolor{currentstroke}%
\pgfsetdash{}{0pt}%
\pgfpathmoveto{\pgfqpoint{3.206071in}{3.289675in}}%
\pgfpathlineto{\pgfqpoint{3.233018in}{3.274117in}}%
\pgfpathlineto{\pgfqpoint{3.233018in}{3.305233in}}%
\pgfpathlineto{\pgfqpoint{3.206071in}{3.320791in}}%
\pgfpathlineto{\pgfqpoint{3.206071in}{3.289675in}}%
\pgfpathclose%
\pgfusepath{fill}%
\end{pgfscope}%
\begin{pgfscope}%
\pgfpathrectangle{\pgfqpoint{0.765000in}{0.660000in}}{\pgfqpoint{4.620000in}{4.620000in}}%
\pgfusepath{clip}%
\pgfsetbuttcap%
\pgfsetroundjoin%
\definecolor{currentfill}{rgb}{1.000000,0.894118,0.788235}%
\pgfsetfillcolor{currentfill}%
\pgfsetlinewidth{0.000000pt}%
\definecolor{currentstroke}{rgb}{1.000000,0.894118,0.788235}%
\pgfsetstrokecolor{currentstroke}%
\pgfsetdash{}{0pt}%
\pgfpathmoveto{\pgfqpoint{3.201279in}{3.052489in}}%
\pgfpathlineto{\pgfqpoint{3.174331in}{3.036931in}}%
\pgfpathlineto{\pgfqpoint{3.206071in}{3.289675in}}%
\pgfpathlineto{\pgfqpoint{3.233018in}{3.305233in}}%
\pgfpathlineto{\pgfqpoint{3.201279in}{3.052489in}}%
\pgfpathclose%
\pgfusepath{fill}%
\end{pgfscope}%
\begin{pgfscope}%
\pgfpathrectangle{\pgfqpoint{0.765000in}{0.660000in}}{\pgfqpoint{4.620000in}{4.620000in}}%
\pgfusepath{clip}%
\pgfsetbuttcap%
\pgfsetroundjoin%
\definecolor{currentfill}{rgb}{1.000000,0.894118,0.788235}%
\pgfsetfillcolor{currentfill}%
\pgfsetlinewidth{0.000000pt}%
\definecolor{currentstroke}{rgb}{1.000000,0.894118,0.788235}%
\pgfsetstrokecolor{currentstroke}%
\pgfsetdash{}{0pt}%
\pgfpathmoveto{\pgfqpoint{3.228226in}{3.036931in}}%
\pgfpathlineto{\pgfqpoint{3.201279in}{3.052489in}}%
\pgfpathlineto{\pgfqpoint{3.233018in}{3.305233in}}%
\pgfpathlineto{\pgfqpoint{3.259965in}{3.289675in}}%
\pgfpathlineto{\pgfqpoint{3.228226in}{3.036931in}}%
\pgfpathclose%
\pgfusepath{fill}%
\end{pgfscope}%
\begin{pgfscope}%
\pgfpathrectangle{\pgfqpoint{0.765000in}{0.660000in}}{\pgfqpoint{4.620000in}{4.620000in}}%
\pgfusepath{clip}%
\pgfsetbuttcap%
\pgfsetroundjoin%
\definecolor{currentfill}{rgb}{1.000000,0.894118,0.788235}%
\pgfsetfillcolor{currentfill}%
\pgfsetlinewidth{0.000000pt}%
\definecolor{currentstroke}{rgb}{1.000000,0.894118,0.788235}%
\pgfsetstrokecolor{currentstroke}%
\pgfsetdash{}{0pt}%
\pgfpathmoveto{\pgfqpoint{3.201279in}{3.052489in}}%
\pgfpathlineto{\pgfqpoint{3.201279in}{3.083605in}}%
\pgfpathlineto{\pgfqpoint{3.233018in}{3.336349in}}%
\pgfpathlineto{\pgfqpoint{3.259965in}{3.289675in}}%
\pgfpathlineto{\pgfqpoint{3.201279in}{3.052489in}}%
\pgfpathclose%
\pgfusepath{fill}%
\end{pgfscope}%
\begin{pgfscope}%
\pgfpathrectangle{\pgfqpoint{0.765000in}{0.660000in}}{\pgfqpoint{4.620000in}{4.620000in}}%
\pgfusepath{clip}%
\pgfsetbuttcap%
\pgfsetroundjoin%
\definecolor{currentfill}{rgb}{1.000000,0.894118,0.788235}%
\pgfsetfillcolor{currentfill}%
\pgfsetlinewidth{0.000000pt}%
\definecolor{currentstroke}{rgb}{1.000000,0.894118,0.788235}%
\pgfsetstrokecolor{currentstroke}%
\pgfsetdash{}{0pt}%
\pgfpathmoveto{\pgfqpoint{3.228226in}{3.036931in}}%
\pgfpathlineto{\pgfqpoint{3.228226in}{3.068047in}}%
\pgfpathlineto{\pgfqpoint{3.259965in}{3.320791in}}%
\pgfpathlineto{\pgfqpoint{3.233018in}{3.305233in}}%
\pgfpathlineto{\pgfqpoint{3.228226in}{3.036931in}}%
\pgfpathclose%
\pgfusepath{fill}%
\end{pgfscope}%
\begin{pgfscope}%
\pgfpathrectangle{\pgfqpoint{0.765000in}{0.660000in}}{\pgfqpoint{4.620000in}{4.620000in}}%
\pgfusepath{clip}%
\pgfsetbuttcap%
\pgfsetroundjoin%
\definecolor{currentfill}{rgb}{1.000000,0.894118,0.788235}%
\pgfsetfillcolor{currentfill}%
\pgfsetlinewidth{0.000000pt}%
\definecolor{currentstroke}{rgb}{1.000000,0.894118,0.788235}%
\pgfsetstrokecolor{currentstroke}%
\pgfsetdash{}{0pt}%
\pgfpathmoveto{\pgfqpoint{3.174331in}{3.036931in}}%
\pgfpathlineto{\pgfqpoint{3.201279in}{3.021373in}}%
\pgfpathlineto{\pgfqpoint{3.233018in}{3.274117in}}%
\pgfpathlineto{\pgfqpoint{3.206071in}{3.289675in}}%
\pgfpathlineto{\pgfqpoint{3.174331in}{3.036931in}}%
\pgfpathclose%
\pgfusepath{fill}%
\end{pgfscope}%
\begin{pgfscope}%
\pgfpathrectangle{\pgfqpoint{0.765000in}{0.660000in}}{\pgfqpoint{4.620000in}{4.620000in}}%
\pgfusepath{clip}%
\pgfsetbuttcap%
\pgfsetroundjoin%
\definecolor{currentfill}{rgb}{1.000000,0.894118,0.788235}%
\pgfsetfillcolor{currentfill}%
\pgfsetlinewidth{0.000000pt}%
\definecolor{currentstroke}{rgb}{1.000000,0.894118,0.788235}%
\pgfsetstrokecolor{currentstroke}%
\pgfsetdash{}{0pt}%
\pgfpathmoveto{\pgfqpoint{3.201279in}{3.021373in}}%
\pgfpathlineto{\pgfqpoint{3.228226in}{3.036931in}}%
\pgfpathlineto{\pgfqpoint{3.259965in}{3.289675in}}%
\pgfpathlineto{\pgfqpoint{3.233018in}{3.274117in}}%
\pgfpathlineto{\pgfqpoint{3.201279in}{3.021373in}}%
\pgfpathclose%
\pgfusepath{fill}%
\end{pgfscope}%
\begin{pgfscope}%
\pgfpathrectangle{\pgfqpoint{0.765000in}{0.660000in}}{\pgfqpoint{4.620000in}{4.620000in}}%
\pgfusepath{clip}%
\pgfsetbuttcap%
\pgfsetroundjoin%
\definecolor{currentfill}{rgb}{1.000000,0.894118,0.788235}%
\pgfsetfillcolor{currentfill}%
\pgfsetlinewidth{0.000000pt}%
\definecolor{currentstroke}{rgb}{1.000000,0.894118,0.788235}%
\pgfsetstrokecolor{currentstroke}%
\pgfsetdash{}{0pt}%
\pgfpathmoveto{\pgfqpoint{3.201279in}{3.083605in}}%
\pgfpathlineto{\pgfqpoint{3.174331in}{3.068047in}}%
\pgfpathlineto{\pgfqpoint{3.206071in}{3.320791in}}%
\pgfpathlineto{\pgfqpoint{3.233018in}{3.336349in}}%
\pgfpathlineto{\pgfqpoint{3.201279in}{3.083605in}}%
\pgfpathclose%
\pgfusepath{fill}%
\end{pgfscope}%
\begin{pgfscope}%
\pgfpathrectangle{\pgfqpoint{0.765000in}{0.660000in}}{\pgfqpoint{4.620000in}{4.620000in}}%
\pgfusepath{clip}%
\pgfsetbuttcap%
\pgfsetroundjoin%
\definecolor{currentfill}{rgb}{1.000000,0.894118,0.788235}%
\pgfsetfillcolor{currentfill}%
\pgfsetlinewidth{0.000000pt}%
\definecolor{currentstroke}{rgb}{1.000000,0.894118,0.788235}%
\pgfsetstrokecolor{currentstroke}%
\pgfsetdash{}{0pt}%
\pgfpathmoveto{\pgfqpoint{3.228226in}{3.068047in}}%
\pgfpathlineto{\pgfqpoint{3.201279in}{3.083605in}}%
\pgfpathlineto{\pgfqpoint{3.233018in}{3.336349in}}%
\pgfpathlineto{\pgfqpoint{3.259965in}{3.320791in}}%
\pgfpathlineto{\pgfqpoint{3.228226in}{3.068047in}}%
\pgfpathclose%
\pgfusepath{fill}%
\end{pgfscope}%
\begin{pgfscope}%
\pgfpathrectangle{\pgfqpoint{0.765000in}{0.660000in}}{\pgfqpoint{4.620000in}{4.620000in}}%
\pgfusepath{clip}%
\pgfsetbuttcap%
\pgfsetroundjoin%
\definecolor{currentfill}{rgb}{1.000000,0.894118,0.788235}%
\pgfsetfillcolor{currentfill}%
\pgfsetlinewidth{0.000000pt}%
\definecolor{currentstroke}{rgb}{1.000000,0.894118,0.788235}%
\pgfsetstrokecolor{currentstroke}%
\pgfsetdash{}{0pt}%
\pgfpathmoveto{\pgfqpoint{3.174331in}{3.036931in}}%
\pgfpathlineto{\pgfqpoint{3.174331in}{3.068047in}}%
\pgfpathlineto{\pgfqpoint{3.206071in}{3.320791in}}%
\pgfpathlineto{\pgfqpoint{3.233018in}{3.274117in}}%
\pgfpathlineto{\pgfqpoint{3.174331in}{3.036931in}}%
\pgfpathclose%
\pgfusepath{fill}%
\end{pgfscope}%
\begin{pgfscope}%
\pgfpathrectangle{\pgfqpoint{0.765000in}{0.660000in}}{\pgfqpoint{4.620000in}{4.620000in}}%
\pgfusepath{clip}%
\pgfsetbuttcap%
\pgfsetroundjoin%
\definecolor{currentfill}{rgb}{1.000000,0.894118,0.788235}%
\pgfsetfillcolor{currentfill}%
\pgfsetlinewidth{0.000000pt}%
\definecolor{currentstroke}{rgb}{1.000000,0.894118,0.788235}%
\pgfsetstrokecolor{currentstroke}%
\pgfsetdash{}{0pt}%
\pgfpathmoveto{\pgfqpoint{3.201279in}{3.021373in}}%
\pgfpathlineto{\pgfqpoint{3.201279in}{3.052489in}}%
\pgfpathlineto{\pgfqpoint{3.233018in}{3.305233in}}%
\pgfpathlineto{\pgfqpoint{3.206071in}{3.289675in}}%
\pgfpathlineto{\pgfqpoint{3.201279in}{3.021373in}}%
\pgfpathclose%
\pgfusepath{fill}%
\end{pgfscope}%
\begin{pgfscope}%
\pgfpathrectangle{\pgfqpoint{0.765000in}{0.660000in}}{\pgfqpoint{4.620000in}{4.620000in}}%
\pgfusepath{clip}%
\pgfsetbuttcap%
\pgfsetroundjoin%
\definecolor{currentfill}{rgb}{1.000000,0.894118,0.788235}%
\pgfsetfillcolor{currentfill}%
\pgfsetlinewidth{0.000000pt}%
\definecolor{currentstroke}{rgb}{1.000000,0.894118,0.788235}%
\pgfsetstrokecolor{currentstroke}%
\pgfsetdash{}{0pt}%
\pgfpathmoveto{\pgfqpoint{3.201279in}{3.052489in}}%
\pgfpathlineto{\pgfqpoint{3.228226in}{3.068047in}}%
\pgfpathlineto{\pgfqpoint{3.259965in}{3.320791in}}%
\pgfpathlineto{\pgfqpoint{3.233018in}{3.305233in}}%
\pgfpathlineto{\pgfqpoint{3.201279in}{3.052489in}}%
\pgfpathclose%
\pgfusepath{fill}%
\end{pgfscope}%
\begin{pgfscope}%
\pgfpathrectangle{\pgfqpoint{0.765000in}{0.660000in}}{\pgfqpoint{4.620000in}{4.620000in}}%
\pgfusepath{clip}%
\pgfsetbuttcap%
\pgfsetroundjoin%
\definecolor{currentfill}{rgb}{1.000000,0.894118,0.788235}%
\pgfsetfillcolor{currentfill}%
\pgfsetlinewidth{0.000000pt}%
\definecolor{currentstroke}{rgb}{1.000000,0.894118,0.788235}%
\pgfsetstrokecolor{currentstroke}%
\pgfsetdash{}{0pt}%
\pgfpathmoveto{\pgfqpoint{3.174331in}{3.068047in}}%
\pgfpathlineto{\pgfqpoint{3.201279in}{3.052489in}}%
\pgfpathlineto{\pgfqpoint{3.233018in}{3.305233in}}%
\pgfpathlineto{\pgfqpoint{3.206071in}{3.320791in}}%
\pgfpathlineto{\pgfqpoint{3.174331in}{3.068047in}}%
\pgfpathclose%
\pgfusepath{fill}%
\end{pgfscope}%
\begin{pgfscope}%
\pgfpathrectangle{\pgfqpoint{0.765000in}{0.660000in}}{\pgfqpoint{4.620000in}{4.620000in}}%
\pgfusepath{clip}%
\pgfsetbuttcap%
\pgfsetroundjoin%
\definecolor{currentfill}{rgb}{1.000000,0.894118,0.788235}%
\pgfsetfillcolor{currentfill}%
\pgfsetlinewidth{0.000000pt}%
\definecolor{currentstroke}{rgb}{1.000000,0.894118,0.788235}%
\pgfsetstrokecolor{currentstroke}%
\pgfsetdash{}{0pt}%
\pgfpathmoveto{\pgfqpoint{3.201279in}{3.052489in}}%
\pgfpathlineto{\pgfqpoint{3.174331in}{3.036931in}}%
\pgfpathlineto{\pgfqpoint{3.201279in}{3.021373in}}%
\pgfpathlineto{\pgfqpoint{3.228226in}{3.036931in}}%
\pgfpathlineto{\pgfqpoint{3.201279in}{3.052489in}}%
\pgfpathclose%
\pgfusepath{fill}%
\end{pgfscope}%
\begin{pgfscope}%
\pgfpathrectangle{\pgfqpoint{0.765000in}{0.660000in}}{\pgfqpoint{4.620000in}{4.620000in}}%
\pgfusepath{clip}%
\pgfsetbuttcap%
\pgfsetroundjoin%
\definecolor{currentfill}{rgb}{1.000000,0.894118,0.788235}%
\pgfsetfillcolor{currentfill}%
\pgfsetlinewidth{0.000000pt}%
\definecolor{currentstroke}{rgb}{1.000000,0.894118,0.788235}%
\pgfsetstrokecolor{currentstroke}%
\pgfsetdash{}{0pt}%
\pgfpathmoveto{\pgfqpoint{3.201279in}{3.052489in}}%
\pgfpathlineto{\pgfqpoint{3.174331in}{3.036931in}}%
\pgfpathlineto{\pgfqpoint{3.174331in}{3.068047in}}%
\pgfpathlineto{\pgfqpoint{3.201279in}{3.083605in}}%
\pgfpathlineto{\pgfqpoint{3.201279in}{3.052489in}}%
\pgfpathclose%
\pgfusepath{fill}%
\end{pgfscope}%
\begin{pgfscope}%
\pgfpathrectangle{\pgfqpoint{0.765000in}{0.660000in}}{\pgfqpoint{4.620000in}{4.620000in}}%
\pgfusepath{clip}%
\pgfsetbuttcap%
\pgfsetroundjoin%
\definecolor{currentfill}{rgb}{1.000000,0.894118,0.788235}%
\pgfsetfillcolor{currentfill}%
\pgfsetlinewidth{0.000000pt}%
\definecolor{currentstroke}{rgb}{1.000000,0.894118,0.788235}%
\pgfsetstrokecolor{currentstroke}%
\pgfsetdash{}{0pt}%
\pgfpathmoveto{\pgfqpoint{3.201279in}{3.052489in}}%
\pgfpathlineto{\pgfqpoint{3.228226in}{3.036931in}}%
\pgfpathlineto{\pgfqpoint{3.228226in}{3.068047in}}%
\pgfpathlineto{\pgfqpoint{3.201279in}{3.083605in}}%
\pgfpathlineto{\pgfqpoint{3.201279in}{3.052489in}}%
\pgfpathclose%
\pgfusepath{fill}%
\end{pgfscope}%
\begin{pgfscope}%
\pgfpathrectangle{\pgfqpoint{0.765000in}{0.660000in}}{\pgfqpoint{4.620000in}{4.620000in}}%
\pgfusepath{clip}%
\pgfsetbuttcap%
\pgfsetroundjoin%
\definecolor{currentfill}{rgb}{1.000000,0.894118,0.788235}%
\pgfsetfillcolor{currentfill}%
\pgfsetlinewidth{0.000000pt}%
\definecolor{currentstroke}{rgb}{1.000000,0.894118,0.788235}%
\pgfsetstrokecolor{currentstroke}%
\pgfsetdash{}{0pt}%
\pgfpathmoveto{\pgfqpoint{3.318632in}{2.917586in}}%
\pgfpathlineto{\pgfqpoint{3.291685in}{2.902028in}}%
\pgfpathlineto{\pgfqpoint{3.318632in}{2.886470in}}%
\pgfpathlineto{\pgfqpoint{3.345579in}{2.902028in}}%
\pgfpathlineto{\pgfqpoint{3.318632in}{2.917586in}}%
\pgfpathclose%
\pgfusepath{fill}%
\end{pgfscope}%
\begin{pgfscope}%
\pgfpathrectangle{\pgfqpoint{0.765000in}{0.660000in}}{\pgfqpoint{4.620000in}{4.620000in}}%
\pgfusepath{clip}%
\pgfsetbuttcap%
\pgfsetroundjoin%
\definecolor{currentfill}{rgb}{1.000000,0.894118,0.788235}%
\pgfsetfillcolor{currentfill}%
\pgfsetlinewidth{0.000000pt}%
\definecolor{currentstroke}{rgb}{1.000000,0.894118,0.788235}%
\pgfsetstrokecolor{currentstroke}%
\pgfsetdash{}{0pt}%
\pgfpathmoveto{\pgfqpoint{3.318632in}{2.917586in}}%
\pgfpathlineto{\pgfqpoint{3.291685in}{2.902028in}}%
\pgfpathlineto{\pgfqpoint{3.291685in}{2.933144in}}%
\pgfpathlineto{\pgfqpoint{3.318632in}{2.948702in}}%
\pgfpathlineto{\pgfqpoint{3.318632in}{2.917586in}}%
\pgfpathclose%
\pgfusepath{fill}%
\end{pgfscope}%
\begin{pgfscope}%
\pgfpathrectangle{\pgfqpoint{0.765000in}{0.660000in}}{\pgfqpoint{4.620000in}{4.620000in}}%
\pgfusepath{clip}%
\pgfsetbuttcap%
\pgfsetroundjoin%
\definecolor{currentfill}{rgb}{1.000000,0.894118,0.788235}%
\pgfsetfillcolor{currentfill}%
\pgfsetlinewidth{0.000000pt}%
\definecolor{currentstroke}{rgb}{1.000000,0.894118,0.788235}%
\pgfsetstrokecolor{currentstroke}%
\pgfsetdash{}{0pt}%
\pgfpathmoveto{\pgfqpoint{3.318632in}{2.917586in}}%
\pgfpathlineto{\pgfqpoint{3.345579in}{2.902028in}}%
\pgfpathlineto{\pgfqpoint{3.345579in}{2.933144in}}%
\pgfpathlineto{\pgfqpoint{3.318632in}{2.948702in}}%
\pgfpathlineto{\pgfqpoint{3.318632in}{2.917586in}}%
\pgfpathclose%
\pgfusepath{fill}%
\end{pgfscope}%
\begin{pgfscope}%
\pgfpathrectangle{\pgfqpoint{0.765000in}{0.660000in}}{\pgfqpoint{4.620000in}{4.620000in}}%
\pgfusepath{clip}%
\pgfsetbuttcap%
\pgfsetroundjoin%
\definecolor{currentfill}{rgb}{1.000000,0.894118,0.788235}%
\pgfsetfillcolor{currentfill}%
\pgfsetlinewidth{0.000000pt}%
\definecolor{currentstroke}{rgb}{1.000000,0.894118,0.788235}%
\pgfsetstrokecolor{currentstroke}%
\pgfsetdash{}{0pt}%
\pgfpathmoveto{\pgfqpoint{3.201279in}{3.083605in}}%
\pgfpathlineto{\pgfqpoint{3.174331in}{3.068047in}}%
\pgfpathlineto{\pgfqpoint{3.201279in}{3.052489in}}%
\pgfpathlineto{\pgfqpoint{3.228226in}{3.068047in}}%
\pgfpathlineto{\pgfqpoint{3.201279in}{3.083605in}}%
\pgfpathclose%
\pgfusepath{fill}%
\end{pgfscope}%
\begin{pgfscope}%
\pgfpathrectangle{\pgfqpoint{0.765000in}{0.660000in}}{\pgfqpoint{4.620000in}{4.620000in}}%
\pgfusepath{clip}%
\pgfsetbuttcap%
\pgfsetroundjoin%
\definecolor{currentfill}{rgb}{1.000000,0.894118,0.788235}%
\pgfsetfillcolor{currentfill}%
\pgfsetlinewidth{0.000000pt}%
\definecolor{currentstroke}{rgb}{1.000000,0.894118,0.788235}%
\pgfsetstrokecolor{currentstroke}%
\pgfsetdash{}{0pt}%
\pgfpathmoveto{\pgfqpoint{3.201279in}{3.021373in}}%
\pgfpathlineto{\pgfqpoint{3.228226in}{3.036931in}}%
\pgfpathlineto{\pgfqpoint{3.228226in}{3.068047in}}%
\pgfpathlineto{\pgfqpoint{3.201279in}{3.052489in}}%
\pgfpathlineto{\pgfqpoint{3.201279in}{3.021373in}}%
\pgfpathclose%
\pgfusepath{fill}%
\end{pgfscope}%
\begin{pgfscope}%
\pgfpathrectangle{\pgfqpoint{0.765000in}{0.660000in}}{\pgfqpoint{4.620000in}{4.620000in}}%
\pgfusepath{clip}%
\pgfsetbuttcap%
\pgfsetroundjoin%
\definecolor{currentfill}{rgb}{1.000000,0.894118,0.788235}%
\pgfsetfillcolor{currentfill}%
\pgfsetlinewidth{0.000000pt}%
\definecolor{currentstroke}{rgb}{1.000000,0.894118,0.788235}%
\pgfsetstrokecolor{currentstroke}%
\pgfsetdash{}{0pt}%
\pgfpathmoveto{\pgfqpoint{3.174331in}{3.036931in}}%
\pgfpathlineto{\pgfqpoint{3.201279in}{3.021373in}}%
\pgfpathlineto{\pgfqpoint{3.201279in}{3.052489in}}%
\pgfpathlineto{\pgfqpoint{3.174331in}{3.068047in}}%
\pgfpathlineto{\pgfqpoint{3.174331in}{3.036931in}}%
\pgfpathclose%
\pgfusepath{fill}%
\end{pgfscope}%
\begin{pgfscope}%
\pgfpathrectangle{\pgfqpoint{0.765000in}{0.660000in}}{\pgfqpoint{4.620000in}{4.620000in}}%
\pgfusepath{clip}%
\pgfsetbuttcap%
\pgfsetroundjoin%
\definecolor{currentfill}{rgb}{1.000000,0.894118,0.788235}%
\pgfsetfillcolor{currentfill}%
\pgfsetlinewidth{0.000000pt}%
\definecolor{currentstroke}{rgb}{1.000000,0.894118,0.788235}%
\pgfsetstrokecolor{currentstroke}%
\pgfsetdash{}{0pt}%
\pgfpathmoveto{\pgfqpoint{3.318632in}{2.948702in}}%
\pgfpathlineto{\pgfqpoint{3.291685in}{2.933144in}}%
\pgfpathlineto{\pgfqpoint{3.318632in}{2.917586in}}%
\pgfpathlineto{\pgfqpoint{3.345579in}{2.933144in}}%
\pgfpathlineto{\pgfqpoint{3.318632in}{2.948702in}}%
\pgfpathclose%
\pgfusepath{fill}%
\end{pgfscope}%
\begin{pgfscope}%
\pgfpathrectangle{\pgfqpoint{0.765000in}{0.660000in}}{\pgfqpoint{4.620000in}{4.620000in}}%
\pgfusepath{clip}%
\pgfsetbuttcap%
\pgfsetroundjoin%
\definecolor{currentfill}{rgb}{1.000000,0.894118,0.788235}%
\pgfsetfillcolor{currentfill}%
\pgfsetlinewidth{0.000000pt}%
\definecolor{currentstroke}{rgb}{1.000000,0.894118,0.788235}%
\pgfsetstrokecolor{currentstroke}%
\pgfsetdash{}{0pt}%
\pgfpathmoveto{\pgfqpoint{3.318632in}{2.886470in}}%
\pgfpathlineto{\pgfqpoint{3.345579in}{2.902028in}}%
\pgfpathlineto{\pgfqpoint{3.345579in}{2.933144in}}%
\pgfpathlineto{\pgfqpoint{3.318632in}{2.917586in}}%
\pgfpathlineto{\pgfqpoint{3.318632in}{2.886470in}}%
\pgfpathclose%
\pgfusepath{fill}%
\end{pgfscope}%
\begin{pgfscope}%
\pgfpathrectangle{\pgfqpoint{0.765000in}{0.660000in}}{\pgfqpoint{4.620000in}{4.620000in}}%
\pgfusepath{clip}%
\pgfsetbuttcap%
\pgfsetroundjoin%
\definecolor{currentfill}{rgb}{1.000000,0.894118,0.788235}%
\pgfsetfillcolor{currentfill}%
\pgfsetlinewidth{0.000000pt}%
\definecolor{currentstroke}{rgb}{1.000000,0.894118,0.788235}%
\pgfsetstrokecolor{currentstroke}%
\pgfsetdash{}{0pt}%
\pgfpathmoveto{\pgfqpoint{3.291685in}{2.902028in}}%
\pgfpathlineto{\pgfqpoint{3.318632in}{2.886470in}}%
\pgfpathlineto{\pgfqpoint{3.318632in}{2.917586in}}%
\pgfpathlineto{\pgfqpoint{3.291685in}{2.933144in}}%
\pgfpathlineto{\pgfqpoint{3.291685in}{2.902028in}}%
\pgfpathclose%
\pgfusepath{fill}%
\end{pgfscope}%
\begin{pgfscope}%
\pgfpathrectangle{\pgfqpoint{0.765000in}{0.660000in}}{\pgfqpoint{4.620000in}{4.620000in}}%
\pgfusepath{clip}%
\pgfsetbuttcap%
\pgfsetroundjoin%
\definecolor{currentfill}{rgb}{1.000000,0.894118,0.788235}%
\pgfsetfillcolor{currentfill}%
\pgfsetlinewidth{0.000000pt}%
\definecolor{currentstroke}{rgb}{1.000000,0.894118,0.788235}%
\pgfsetstrokecolor{currentstroke}%
\pgfsetdash{}{0pt}%
\pgfpathmoveto{\pgfqpoint{3.201279in}{3.052489in}}%
\pgfpathlineto{\pgfqpoint{3.174331in}{3.036931in}}%
\pgfpathlineto{\pgfqpoint{3.291685in}{2.902028in}}%
\pgfpathlineto{\pgfqpoint{3.318632in}{2.917586in}}%
\pgfpathlineto{\pgfqpoint{3.201279in}{3.052489in}}%
\pgfpathclose%
\pgfusepath{fill}%
\end{pgfscope}%
\begin{pgfscope}%
\pgfpathrectangle{\pgfqpoint{0.765000in}{0.660000in}}{\pgfqpoint{4.620000in}{4.620000in}}%
\pgfusepath{clip}%
\pgfsetbuttcap%
\pgfsetroundjoin%
\definecolor{currentfill}{rgb}{1.000000,0.894118,0.788235}%
\pgfsetfillcolor{currentfill}%
\pgfsetlinewidth{0.000000pt}%
\definecolor{currentstroke}{rgb}{1.000000,0.894118,0.788235}%
\pgfsetstrokecolor{currentstroke}%
\pgfsetdash{}{0pt}%
\pgfpathmoveto{\pgfqpoint{3.228226in}{3.036931in}}%
\pgfpathlineto{\pgfqpoint{3.201279in}{3.052489in}}%
\pgfpathlineto{\pgfqpoint{3.318632in}{2.917586in}}%
\pgfpathlineto{\pgfqpoint{3.345579in}{2.902028in}}%
\pgfpathlineto{\pgfqpoint{3.228226in}{3.036931in}}%
\pgfpathclose%
\pgfusepath{fill}%
\end{pgfscope}%
\begin{pgfscope}%
\pgfpathrectangle{\pgfqpoint{0.765000in}{0.660000in}}{\pgfqpoint{4.620000in}{4.620000in}}%
\pgfusepath{clip}%
\pgfsetbuttcap%
\pgfsetroundjoin%
\definecolor{currentfill}{rgb}{1.000000,0.894118,0.788235}%
\pgfsetfillcolor{currentfill}%
\pgfsetlinewidth{0.000000pt}%
\definecolor{currentstroke}{rgb}{1.000000,0.894118,0.788235}%
\pgfsetstrokecolor{currentstroke}%
\pgfsetdash{}{0pt}%
\pgfpathmoveto{\pgfqpoint{3.201279in}{3.052489in}}%
\pgfpathlineto{\pgfqpoint{3.201279in}{3.083605in}}%
\pgfpathlineto{\pgfqpoint{3.318632in}{2.948702in}}%
\pgfpathlineto{\pgfqpoint{3.345579in}{2.902028in}}%
\pgfpathlineto{\pgfqpoint{3.201279in}{3.052489in}}%
\pgfpathclose%
\pgfusepath{fill}%
\end{pgfscope}%
\begin{pgfscope}%
\pgfpathrectangle{\pgfqpoint{0.765000in}{0.660000in}}{\pgfqpoint{4.620000in}{4.620000in}}%
\pgfusepath{clip}%
\pgfsetbuttcap%
\pgfsetroundjoin%
\definecolor{currentfill}{rgb}{1.000000,0.894118,0.788235}%
\pgfsetfillcolor{currentfill}%
\pgfsetlinewidth{0.000000pt}%
\definecolor{currentstroke}{rgb}{1.000000,0.894118,0.788235}%
\pgfsetstrokecolor{currentstroke}%
\pgfsetdash{}{0pt}%
\pgfpathmoveto{\pgfqpoint{3.228226in}{3.036931in}}%
\pgfpathlineto{\pgfqpoint{3.228226in}{3.068047in}}%
\pgfpathlineto{\pgfqpoint{3.345579in}{2.933144in}}%
\pgfpathlineto{\pgfqpoint{3.318632in}{2.917586in}}%
\pgfpathlineto{\pgfqpoint{3.228226in}{3.036931in}}%
\pgfpathclose%
\pgfusepath{fill}%
\end{pgfscope}%
\begin{pgfscope}%
\pgfpathrectangle{\pgfqpoint{0.765000in}{0.660000in}}{\pgfqpoint{4.620000in}{4.620000in}}%
\pgfusepath{clip}%
\pgfsetbuttcap%
\pgfsetroundjoin%
\definecolor{currentfill}{rgb}{1.000000,0.894118,0.788235}%
\pgfsetfillcolor{currentfill}%
\pgfsetlinewidth{0.000000pt}%
\definecolor{currentstroke}{rgb}{1.000000,0.894118,0.788235}%
\pgfsetstrokecolor{currentstroke}%
\pgfsetdash{}{0pt}%
\pgfpathmoveto{\pgfqpoint{3.174331in}{3.036931in}}%
\pgfpathlineto{\pgfqpoint{3.201279in}{3.021373in}}%
\pgfpathlineto{\pgfqpoint{3.318632in}{2.886470in}}%
\pgfpathlineto{\pgfqpoint{3.291685in}{2.902028in}}%
\pgfpathlineto{\pgfqpoint{3.174331in}{3.036931in}}%
\pgfpathclose%
\pgfusepath{fill}%
\end{pgfscope}%
\begin{pgfscope}%
\pgfpathrectangle{\pgfqpoint{0.765000in}{0.660000in}}{\pgfqpoint{4.620000in}{4.620000in}}%
\pgfusepath{clip}%
\pgfsetbuttcap%
\pgfsetroundjoin%
\definecolor{currentfill}{rgb}{1.000000,0.894118,0.788235}%
\pgfsetfillcolor{currentfill}%
\pgfsetlinewidth{0.000000pt}%
\definecolor{currentstroke}{rgb}{1.000000,0.894118,0.788235}%
\pgfsetstrokecolor{currentstroke}%
\pgfsetdash{}{0pt}%
\pgfpathmoveto{\pgfqpoint{3.201279in}{3.021373in}}%
\pgfpathlineto{\pgfqpoint{3.228226in}{3.036931in}}%
\pgfpathlineto{\pgfqpoint{3.345579in}{2.902028in}}%
\pgfpathlineto{\pgfqpoint{3.318632in}{2.886470in}}%
\pgfpathlineto{\pgfqpoint{3.201279in}{3.021373in}}%
\pgfpathclose%
\pgfusepath{fill}%
\end{pgfscope}%
\begin{pgfscope}%
\pgfpathrectangle{\pgfqpoint{0.765000in}{0.660000in}}{\pgfqpoint{4.620000in}{4.620000in}}%
\pgfusepath{clip}%
\pgfsetbuttcap%
\pgfsetroundjoin%
\definecolor{currentfill}{rgb}{1.000000,0.894118,0.788235}%
\pgfsetfillcolor{currentfill}%
\pgfsetlinewidth{0.000000pt}%
\definecolor{currentstroke}{rgb}{1.000000,0.894118,0.788235}%
\pgfsetstrokecolor{currentstroke}%
\pgfsetdash{}{0pt}%
\pgfpathmoveto{\pgfqpoint{3.201279in}{3.083605in}}%
\pgfpathlineto{\pgfqpoint{3.174331in}{3.068047in}}%
\pgfpathlineto{\pgfqpoint{3.291685in}{2.933144in}}%
\pgfpathlineto{\pgfqpoint{3.318632in}{2.948702in}}%
\pgfpathlineto{\pgfqpoint{3.201279in}{3.083605in}}%
\pgfpathclose%
\pgfusepath{fill}%
\end{pgfscope}%
\begin{pgfscope}%
\pgfpathrectangle{\pgfqpoint{0.765000in}{0.660000in}}{\pgfqpoint{4.620000in}{4.620000in}}%
\pgfusepath{clip}%
\pgfsetbuttcap%
\pgfsetroundjoin%
\definecolor{currentfill}{rgb}{1.000000,0.894118,0.788235}%
\pgfsetfillcolor{currentfill}%
\pgfsetlinewidth{0.000000pt}%
\definecolor{currentstroke}{rgb}{1.000000,0.894118,0.788235}%
\pgfsetstrokecolor{currentstroke}%
\pgfsetdash{}{0pt}%
\pgfpathmoveto{\pgfqpoint{3.228226in}{3.068047in}}%
\pgfpathlineto{\pgfqpoint{3.201279in}{3.083605in}}%
\pgfpathlineto{\pgfqpoint{3.318632in}{2.948702in}}%
\pgfpathlineto{\pgfqpoint{3.345579in}{2.933144in}}%
\pgfpathlineto{\pgfqpoint{3.228226in}{3.068047in}}%
\pgfpathclose%
\pgfusepath{fill}%
\end{pgfscope}%
\begin{pgfscope}%
\pgfpathrectangle{\pgfqpoint{0.765000in}{0.660000in}}{\pgfqpoint{4.620000in}{4.620000in}}%
\pgfusepath{clip}%
\pgfsetbuttcap%
\pgfsetroundjoin%
\definecolor{currentfill}{rgb}{1.000000,0.894118,0.788235}%
\pgfsetfillcolor{currentfill}%
\pgfsetlinewidth{0.000000pt}%
\definecolor{currentstroke}{rgb}{1.000000,0.894118,0.788235}%
\pgfsetstrokecolor{currentstroke}%
\pgfsetdash{}{0pt}%
\pgfpathmoveto{\pgfqpoint{3.174331in}{3.036931in}}%
\pgfpathlineto{\pgfqpoint{3.174331in}{3.068047in}}%
\pgfpathlineto{\pgfqpoint{3.291685in}{2.933144in}}%
\pgfpathlineto{\pgfqpoint{3.318632in}{2.886470in}}%
\pgfpathlineto{\pgfqpoint{3.174331in}{3.036931in}}%
\pgfpathclose%
\pgfusepath{fill}%
\end{pgfscope}%
\begin{pgfscope}%
\pgfpathrectangle{\pgfqpoint{0.765000in}{0.660000in}}{\pgfqpoint{4.620000in}{4.620000in}}%
\pgfusepath{clip}%
\pgfsetbuttcap%
\pgfsetroundjoin%
\definecolor{currentfill}{rgb}{1.000000,0.894118,0.788235}%
\pgfsetfillcolor{currentfill}%
\pgfsetlinewidth{0.000000pt}%
\definecolor{currentstroke}{rgb}{1.000000,0.894118,0.788235}%
\pgfsetstrokecolor{currentstroke}%
\pgfsetdash{}{0pt}%
\pgfpathmoveto{\pgfqpoint{3.201279in}{3.021373in}}%
\pgfpathlineto{\pgfqpoint{3.201279in}{3.052489in}}%
\pgfpathlineto{\pgfqpoint{3.318632in}{2.917586in}}%
\pgfpathlineto{\pgfqpoint{3.291685in}{2.902028in}}%
\pgfpathlineto{\pgfqpoint{3.201279in}{3.021373in}}%
\pgfpathclose%
\pgfusepath{fill}%
\end{pgfscope}%
\begin{pgfscope}%
\pgfpathrectangle{\pgfqpoint{0.765000in}{0.660000in}}{\pgfqpoint{4.620000in}{4.620000in}}%
\pgfusepath{clip}%
\pgfsetbuttcap%
\pgfsetroundjoin%
\definecolor{currentfill}{rgb}{1.000000,0.894118,0.788235}%
\pgfsetfillcolor{currentfill}%
\pgfsetlinewidth{0.000000pt}%
\definecolor{currentstroke}{rgb}{1.000000,0.894118,0.788235}%
\pgfsetstrokecolor{currentstroke}%
\pgfsetdash{}{0pt}%
\pgfpathmoveto{\pgfqpoint{3.201279in}{3.052489in}}%
\pgfpathlineto{\pgfqpoint{3.228226in}{3.068047in}}%
\pgfpathlineto{\pgfqpoint{3.345579in}{2.933144in}}%
\pgfpathlineto{\pgfqpoint{3.318632in}{2.917586in}}%
\pgfpathlineto{\pgfqpoint{3.201279in}{3.052489in}}%
\pgfpathclose%
\pgfusepath{fill}%
\end{pgfscope}%
\begin{pgfscope}%
\pgfpathrectangle{\pgfqpoint{0.765000in}{0.660000in}}{\pgfqpoint{4.620000in}{4.620000in}}%
\pgfusepath{clip}%
\pgfsetbuttcap%
\pgfsetroundjoin%
\definecolor{currentfill}{rgb}{1.000000,0.894118,0.788235}%
\pgfsetfillcolor{currentfill}%
\pgfsetlinewidth{0.000000pt}%
\definecolor{currentstroke}{rgb}{1.000000,0.894118,0.788235}%
\pgfsetstrokecolor{currentstroke}%
\pgfsetdash{}{0pt}%
\pgfpathmoveto{\pgfqpoint{3.174331in}{3.068047in}}%
\pgfpathlineto{\pgfqpoint{3.201279in}{3.052489in}}%
\pgfpathlineto{\pgfqpoint{3.318632in}{2.917586in}}%
\pgfpathlineto{\pgfqpoint{3.291685in}{2.933144in}}%
\pgfpathlineto{\pgfqpoint{3.174331in}{3.068047in}}%
\pgfpathclose%
\pgfusepath{fill}%
\end{pgfscope}%
\begin{pgfscope}%
\pgfpathrectangle{\pgfqpoint{0.765000in}{0.660000in}}{\pgfqpoint{4.620000in}{4.620000in}}%
\pgfusepath{clip}%
\pgfsetbuttcap%
\pgfsetroundjoin%
\definecolor{currentfill}{rgb}{1.000000,0.894118,0.788235}%
\pgfsetfillcolor{currentfill}%
\pgfsetlinewidth{0.000000pt}%
\definecolor{currentstroke}{rgb}{1.000000,0.894118,0.788235}%
\pgfsetstrokecolor{currentstroke}%
\pgfsetdash{}{0pt}%
\pgfpathmoveto{\pgfqpoint{3.233018in}{3.305233in}}%
\pgfpathlineto{\pgfqpoint{3.206071in}{3.289675in}}%
\pgfpathlineto{\pgfqpoint{3.233018in}{3.274117in}}%
\pgfpathlineto{\pgfqpoint{3.259965in}{3.289675in}}%
\pgfpathlineto{\pgfqpoint{3.233018in}{3.305233in}}%
\pgfpathclose%
\pgfusepath{fill}%
\end{pgfscope}%
\begin{pgfscope}%
\pgfpathrectangle{\pgfqpoint{0.765000in}{0.660000in}}{\pgfqpoint{4.620000in}{4.620000in}}%
\pgfusepath{clip}%
\pgfsetbuttcap%
\pgfsetroundjoin%
\definecolor{currentfill}{rgb}{1.000000,0.894118,0.788235}%
\pgfsetfillcolor{currentfill}%
\pgfsetlinewidth{0.000000pt}%
\definecolor{currentstroke}{rgb}{1.000000,0.894118,0.788235}%
\pgfsetstrokecolor{currentstroke}%
\pgfsetdash{}{0pt}%
\pgfpathmoveto{\pgfqpoint{3.233018in}{3.305233in}}%
\pgfpathlineto{\pgfqpoint{3.206071in}{3.289675in}}%
\pgfpathlineto{\pgfqpoint{3.206071in}{3.320791in}}%
\pgfpathlineto{\pgfqpoint{3.233018in}{3.336349in}}%
\pgfpathlineto{\pgfqpoint{3.233018in}{3.305233in}}%
\pgfpathclose%
\pgfusepath{fill}%
\end{pgfscope}%
\begin{pgfscope}%
\pgfpathrectangle{\pgfqpoint{0.765000in}{0.660000in}}{\pgfqpoint{4.620000in}{4.620000in}}%
\pgfusepath{clip}%
\pgfsetbuttcap%
\pgfsetroundjoin%
\definecolor{currentfill}{rgb}{1.000000,0.894118,0.788235}%
\pgfsetfillcolor{currentfill}%
\pgfsetlinewidth{0.000000pt}%
\definecolor{currentstroke}{rgb}{1.000000,0.894118,0.788235}%
\pgfsetstrokecolor{currentstroke}%
\pgfsetdash{}{0pt}%
\pgfpathmoveto{\pgfqpoint{3.233018in}{3.305233in}}%
\pgfpathlineto{\pgfqpoint{3.259965in}{3.289675in}}%
\pgfpathlineto{\pgfqpoint{3.259965in}{3.320791in}}%
\pgfpathlineto{\pgfqpoint{3.233018in}{3.336349in}}%
\pgfpathlineto{\pgfqpoint{3.233018in}{3.305233in}}%
\pgfpathclose%
\pgfusepath{fill}%
\end{pgfscope}%
\begin{pgfscope}%
\pgfpathrectangle{\pgfqpoint{0.765000in}{0.660000in}}{\pgfqpoint{4.620000in}{4.620000in}}%
\pgfusepath{clip}%
\pgfsetbuttcap%
\pgfsetroundjoin%
\definecolor{currentfill}{rgb}{1.000000,0.894118,0.788235}%
\pgfsetfillcolor{currentfill}%
\pgfsetlinewidth{0.000000pt}%
\definecolor{currentstroke}{rgb}{1.000000,0.894118,0.788235}%
\pgfsetstrokecolor{currentstroke}%
\pgfsetdash{}{0pt}%
\pgfpathmoveto{\pgfqpoint{3.201279in}{3.052489in}}%
\pgfpathlineto{\pgfqpoint{3.174331in}{3.036931in}}%
\pgfpathlineto{\pgfqpoint{3.201279in}{3.021373in}}%
\pgfpathlineto{\pgfqpoint{3.228226in}{3.036931in}}%
\pgfpathlineto{\pgfqpoint{3.201279in}{3.052489in}}%
\pgfpathclose%
\pgfusepath{fill}%
\end{pgfscope}%
\begin{pgfscope}%
\pgfpathrectangle{\pgfqpoint{0.765000in}{0.660000in}}{\pgfqpoint{4.620000in}{4.620000in}}%
\pgfusepath{clip}%
\pgfsetbuttcap%
\pgfsetroundjoin%
\definecolor{currentfill}{rgb}{1.000000,0.894118,0.788235}%
\pgfsetfillcolor{currentfill}%
\pgfsetlinewidth{0.000000pt}%
\definecolor{currentstroke}{rgb}{1.000000,0.894118,0.788235}%
\pgfsetstrokecolor{currentstroke}%
\pgfsetdash{}{0pt}%
\pgfpathmoveto{\pgfqpoint{3.201279in}{3.052489in}}%
\pgfpathlineto{\pgfqpoint{3.174331in}{3.036931in}}%
\pgfpathlineto{\pgfqpoint{3.174331in}{3.068047in}}%
\pgfpathlineto{\pgfqpoint{3.201279in}{3.083605in}}%
\pgfpathlineto{\pgfqpoint{3.201279in}{3.052489in}}%
\pgfpathclose%
\pgfusepath{fill}%
\end{pgfscope}%
\begin{pgfscope}%
\pgfpathrectangle{\pgfqpoint{0.765000in}{0.660000in}}{\pgfqpoint{4.620000in}{4.620000in}}%
\pgfusepath{clip}%
\pgfsetbuttcap%
\pgfsetroundjoin%
\definecolor{currentfill}{rgb}{1.000000,0.894118,0.788235}%
\pgfsetfillcolor{currentfill}%
\pgfsetlinewidth{0.000000pt}%
\definecolor{currentstroke}{rgb}{1.000000,0.894118,0.788235}%
\pgfsetstrokecolor{currentstroke}%
\pgfsetdash{}{0pt}%
\pgfpathmoveto{\pgfqpoint{3.201279in}{3.052489in}}%
\pgfpathlineto{\pgfqpoint{3.228226in}{3.036931in}}%
\pgfpathlineto{\pgfqpoint{3.228226in}{3.068047in}}%
\pgfpathlineto{\pgfqpoint{3.201279in}{3.083605in}}%
\pgfpathlineto{\pgfqpoint{3.201279in}{3.052489in}}%
\pgfpathclose%
\pgfusepath{fill}%
\end{pgfscope}%
\begin{pgfscope}%
\pgfpathrectangle{\pgfqpoint{0.765000in}{0.660000in}}{\pgfqpoint{4.620000in}{4.620000in}}%
\pgfusepath{clip}%
\pgfsetbuttcap%
\pgfsetroundjoin%
\definecolor{currentfill}{rgb}{1.000000,0.894118,0.788235}%
\pgfsetfillcolor{currentfill}%
\pgfsetlinewidth{0.000000pt}%
\definecolor{currentstroke}{rgb}{1.000000,0.894118,0.788235}%
\pgfsetstrokecolor{currentstroke}%
\pgfsetdash{}{0pt}%
\pgfpathmoveto{\pgfqpoint{3.233018in}{3.336349in}}%
\pgfpathlineto{\pgfqpoint{3.206071in}{3.320791in}}%
\pgfpathlineto{\pgfqpoint{3.233018in}{3.305233in}}%
\pgfpathlineto{\pgfqpoint{3.259965in}{3.320791in}}%
\pgfpathlineto{\pgfqpoint{3.233018in}{3.336349in}}%
\pgfpathclose%
\pgfusepath{fill}%
\end{pgfscope}%
\begin{pgfscope}%
\pgfpathrectangle{\pgfqpoint{0.765000in}{0.660000in}}{\pgfqpoint{4.620000in}{4.620000in}}%
\pgfusepath{clip}%
\pgfsetbuttcap%
\pgfsetroundjoin%
\definecolor{currentfill}{rgb}{1.000000,0.894118,0.788235}%
\pgfsetfillcolor{currentfill}%
\pgfsetlinewidth{0.000000pt}%
\definecolor{currentstroke}{rgb}{1.000000,0.894118,0.788235}%
\pgfsetstrokecolor{currentstroke}%
\pgfsetdash{}{0pt}%
\pgfpathmoveto{\pgfqpoint{3.233018in}{3.274117in}}%
\pgfpathlineto{\pgfqpoint{3.259965in}{3.289675in}}%
\pgfpathlineto{\pgfqpoint{3.259965in}{3.320791in}}%
\pgfpathlineto{\pgfqpoint{3.233018in}{3.305233in}}%
\pgfpathlineto{\pgfqpoint{3.233018in}{3.274117in}}%
\pgfpathclose%
\pgfusepath{fill}%
\end{pgfscope}%
\begin{pgfscope}%
\pgfpathrectangle{\pgfqpoint{0.765000in}{0.660000in}}{\pgfqpoint{4.620000in}{4.620000in}}%
\pgfusepath{clip}%
\pgfsetbuttcap%
\pgfsetroundjoin%
\definecolor{currentfill}{rgb}{1.000000,0.894118,0.788235}%
\pgfsetfillcolor{currentfill}%
\pgfsetlinewidth{0.000000pt}%
\definecolor{currentstroke}{rgb}{1.000000,0.894118,0.788235}%
\pgfsetstrokecolor{currentstroke}%
\pgfsetdash{}{0pt}%
\pgfpathmoveto{\pgfqpoint{3.206071in}{3.289675in}}%
\pgfpathlineto{\pgfqpoint{3.233018in}{3.274117in}}%
\pgfpathlineto{\pgfqpoint{3.233018in}{3.305233in}}%
\pgfpathlineto{\pgfqpoint{3.206071in}{3.320791in}}%
\pgfpathlineto{\pgfqpoint{3.206071in}{3.289675in}}%
\pgfpathclose%
\pgfusepath{fill}%
\end{pgfscope}%
\begin{pgfscope}%
\pgfpathrectangle{\pgfqpoint{0.765000in}{0.660000in}}{\pgfqpoint{4.620000in}{4.620000in}}%
\pgfusepath{clip}%
\pgfsetbuttcap%
\pgfsetroundjoin%
\definecolor{currentfill}{rgb}{1.000000,0.894118,0.788235}%
\pgfsetfillcolor{currentfill}%
\pgfsetlinewidth{0.000000pt}%
\definecolor{currentstroke}{rgb}{1.000000,0.894118,0.788235}%
\pgfsetstrokecolor{currentstroke}%
\pgfsetdash{}{0pt}%
\pgfpathmoveto{\pgfqpoint{3.201279in}{3.083605in}}%
\pgfpathlineto{\pgfqpoint{3.174331in}{3.068047in}}%
\pgfpathlineto{\pgfqpoint{3.201279in}{3.052489in}}%
\pgfpathlineto{\pgfqpoint{3.228226in}{3.068047in}}%
\pgfpathlineto{\pgfqpoint{3.201279in}{3.083605in}}%
\pgfpathclose%
\pgfusepath{fill}%
\end{pgfscope}%
\begin{pgfscope}%
\pgfpathrectangle{\pgfqpoint{0.765000in}{0.660000in}}{\pgfqpoint{4.620000in}{4.620000in}}%
\pgfusepath{clip}%
\pgfsetbuttcap%
\pgfsetroundjoin%
\definecolor{currentfill}{rgb}{1.000000,0.894118,0.788235}%
\pgfsetfillcolor{currentfill}%
\pgfsetlinewidth{0.000000pt}%
\definecolor{currentstroke}{rgb}{1.000000,0.894118,0.788235}%
\pgfsetstrokecolor{currentstroke}%
\pgfsetdash{}{0pt}%
\pgfpathmoveto{\pgfqpoint{3.201279in}{3.021373in}}%
\pgfpathlineto{\pgfqpoint{3.228226in}{3.036931in}}%
\pgfpathlineto{\pgfqpoint{3.228226in}{3.068047in}}%
\pgfpathlineto{\pgfqpoint{3.201279in}{3.052489in}}%
\pgfpathlineto{\pgfqpoint{3.201279in}{3.021373in}}%
\pgfpathclose%
\pgfusepath{fill}%
\end{pgfscope}%
\begin{pgfscope}%
\pgfpathrectangle{\pgfqpoint{0.765000in}{0.660000in}}{\pgfqpoint{4.620000in}{4.620000in}}%
\pgfusepath{clip}%
\pgfsetbuttcap%
\pgfsetroundjoin%
\definecolor{currentfill}{rgb}{1.000000,0.894118,0.788235}%
\pgfsetfillcolor{currentfill}%
\pgfsetlinewidth{0.000000pt}%
\definecolor{currentstroke}{rgb}{1.000000,0.894118,0.788235}%
\pgfsetstrokecolor{currentstroke}%
\pgfsetdash{}{0pt}%
\pgfpathmoveto{\pgfqpoint{3.174331in}{3.036931in}}%
\pgfpathlineto{\pgfqpoint{3.201279in}{3.021373in}}%
\pgfpathlineto{\pgfqpoint{3.201279in}{3.052489in}}%
\pgfpathlineto{\pgfqpoint{3.174331in}{3.068047in}}%
\pgfpathlineto{\pgfqpoint{3.174331in}{3.036931in}}%
\pgfpathclose%
\pgfusepath{fill}%
\end{pgfscope}%
\begin{pgfscope}%
\pgfpathrectangle{\pgfqpoint{0.765000in}{0.660000in}}{\pgfqpoint{4.620000in}{4.620000in}}%
\pgfusepath{clip}%
\pgfsetbuttcap%
\pgfsetroundjoin%
\definecolor{currentfill}{rgb}{1.000000,0.894118,0.788235}%
\pgfsetfillcolor{currentfill}%
\pgfsetlinewidth{0.000000pt}%
\definecolor{currentstroke}{rgb}{1.000000,0.894118,0.788235}%
\pgfsetstrokecolor{currentstroke}%
\pgfsetdash{}{0pt}%
\pgfpathmoveto{\pgfqpoint{3.233018in}{3.305233in}}%
\pgfpathlineto{\pgfqpoint{3.206071in}{3.289675in}}%
\pgfpathlineto{\pgfqpoint{3.174331in}{3.036931in}}%
\pgfpathlineto{\pgfqpoint{3.201279in}{3.052489in}}%
\pgfpathlineto{\pgfqpoint{3.233018in}{3.305233in}}%
\pgfpathclose%
\pgfusepath{fill}%
\end{pgfscope}%
\begin{pgfscope}%
\pgfpathrectangle{\pgfqpoint{0.765000in}{0.660000in}}{\pgfqpoint{4.620000in}{4.620000in}}%
\pgfusepath{clip}%
\pgfsetbuttcap%
\pgfsetroundjoin%
\definecolor{currentfill}{rgb}{1.000000,0.894118,0.788235}%
\pgfsetfillcolor{currentfill}%
\pgfsetlinewidth{0.000000pt}%
\definecolor{currentstroke}{rgb}{1.000000,0.894118,0.788235}%
\pgfsetstrokecolor{currentstroke}%
\pgfsetdash{}{0pt}%
\pgfpathmoveto{\pgfqpoint{3.259965in}{3.289675in}}%
\pgfpathlineto{\pgfqpoint{3.233018in}{3.305233in}}%
\pgfpathlineto{\pgfqpoint{3.201279in}{3.052489in}}%
\pgfpathlineto{\pgfqpoint{3.228226in}{3.036931in}}%
\pgfpathlineto{\pgfqpoint{3.259965in}{3.289675in}}%
\pgfpathclose%
\pgfusepath{fill}%
\end{pgfscope}%
\begin{pgfscope}%
\pgfpathrectangle{\pgfqpoint{0.765000in}{0.660000in}}{\pgfqpoint{4.620000in}{4.620000in}}%
\pgfusepath{clip}%
\pgfsetbuttcap%
\pgfsetroundjoin%
\definecolor{currentfill}{rgb}{1.000000,0.894118,0.788235}%
\pgfsetfillcolor{currentfill}%
\pgfsetlinewidth{0.000000pt}%
\definecolor{currentstroke}{rgb}{1.000000,0.894118,0.788235}%
\pgfsetstrokecolor{currentstroke}%
\pgfsetdash{}{0pt}%
\pgfpathmoveto{\pgfqpoint{3.233018in}{3.305233in}}%
\pgfpathlineto{\pgfqpoint{3.233018in}{3.336349in}}%
\pgfpathlineto{\pgfqpoint{3.201279in}{3.083605in}}%
\pgfpathlineto{\pgfqpoint{3.228226in}{3.036931in}}%
\pgfpathlineto{\pgfqpoint{3.233018in}{3.305233in}}%
\pgfpathclose%
\pgfusepath{fill}%
\end{pgfscope}%
\begin{pgfscope}%
\pgfpathrectangle{\pgfqpoint{0.765000in}{0.660000in}}{\pgfqpoint{4.620000in}{4.620000in}}%
\pgfusepath{clip}%
\pgfsetbuttcap%
\pgfsetroundjoin%
\definecolor{currentfill}{rgb}{1.000000,0.894118,0.788235}%
\pgfsetfillcolor{currentfill}%
\pgfsetlinewidth{0.000000pt}%
\definecolor{currentstroke}{rgb}{1.000000,0.894118,0.788235}%
\pgfsetstrokecolor{currentstroke}%
\pgfsetdash{}{0pt}%
\pgfpathmoveto{\pgfqpoint{3.259965in}{3.289675in}}%
\pgfpathlineto{\pgfqpoint{3.259965in}{3.320791in}}%
\pgfpathlineto{\pgfqpoint{3.228226in}{3.068047in}}%
\pgfpathlineto{\pgfqpoint{3.201279in}{3.052489in}}%
\pgfpathlineto{\pgfqpoint{3.259965in}{3.289675in}}%
\pgfpathclose%
\pgfusepath{fill}%
\end{pgfscope}%
\begin{pgfscope}%
\pgfpathrectangle{\pgfqpoint{0.765000in}{0.660000in}}{\pgfqpoint{4.620000in}{4.620000in}}%
\pgfusepath{clip}%
\pgfsetbuttcap%
\pgfsetroundjoin%
\definecolor{currentfill}{rgb}{1.000000,0.894118,0.788235}%
\pgfsetfillcolor{currentfill}%
\pgfsetlinewidth{0.000000pt}%
\definecolor{currentstroke}{rgb}{1.000000,0.894118,0.788235}%
\pgfsetstrokecolor{currentstroke}%
\pgfsetdash{}{0pt}%
\pgfpathmoveto{\pgfqpoint{3.206071in}{3.289675in}}%
\pgfpathlineto{\pgfqpoint{3.233018in}{3.274117in}}%
\pgfpathlineto{\pgfqpoint{3.201279in}{3.021373in}}%
\pgfpathlineto{\pgfqpoint{3.174331in}{3.036931in}}%
\pgfpathlineto{\pgfqpoint{3.206071in}{3.289675in}}%
\pgfpathclose%
\pgfusepath{fill}%
\end{pgfscope}%
\begin{pgfscope}%
\pgfpathrectangle{\pgfqpoint{0.765000in}{0.660000in}}{\pgfqpoint{4.620000in}{4.620000in}}%
\pgfusepath{clip}%
\pgfsetbuttcap%
\pgfsetroundjoin%
\definecolor{currentfill}{rgb}{1.000000,0.894118,0.788235}%
\pgfsetfillcolor{currentfill}%
\pgfsetlinewidth{0.000000pt}%
\definecolor{currentstroke}{rgb}{1.000000,0.894118,0.788235}%
\pgfsetstrokecolor{currentstroke}%
\pgfsetdash{}{0pt}%
\pgfpathmoveto{\pgfqpoint{3.233018in}{3.274117in}}%
\pgfpathlineto{\pgfqpoint{3.259965in}{3.289675in}}%
\pgfpathlineto{\pgfqpoint{3.228226in}{3.036931in}}%
\pgfpathlineto{\pgfqpoint{3.201279in}{3.021373in}}%
\pgfpathlineto{\pgfqpoint{3.233018in}{3.274117in}}%
\pgfpathclose%
\pgfusepath{fill}%
\end{pgfscope}%
\begin{pgfscope}%
\pgfpathrectangle{\pgfqpoint{0.765000in}{0.660000in}}{\pgfqpoint{4.620000in}{4.620000in}}%
\pgfusepath{clip}%
\pgfsetbuttcap%
\pgfsetroundjoin%
\definecolor{currentfill}{rgb}{1.000000,0.894118,0.788235}%
\pgfsetfillcolor{currentfill}%
\pgfsetlinewidth{0.000000pt}%
\definecolor{currentstroke}{rgb}{1.000000,0.894118,0.788235}%
\pgfsetstrokecolor{currentstroke}%
\pgfsetdash{}{0pt}%
\pgfpathmoveto{\pgfqpoint{3.233018in}{3.336349in}}%
\pgfpathlineto{\pgfqpoint{3.206071in}{3.320791in}}%
\pgfpathlineto{\pgfqpoint{3.174331in}{3.068047in}}%
\pgfpathlineto{\pgfqpoint{3.201279in}{3.083605in}}%
\pgfpathlineto{\pgfqpoint{3.233018in}{3.336349in}}%
\pgfpathclose%
\pgfusepath{fill}%
\end{pgfscope}%
\begin{pgfscope}%
\pgfpathrectangle{\pgfqpoint{0.765000in}{0.660000in}}{\pgfqpoint{4.620000in}{4.620000in}}%
\pgfusepath{clip}%
\pgfsetbuttcap%
\pgfsetroundjoin%
\definecolor{currentfill}{rgb}{1.000000,0.894118,0.788235}%
\pgfsetfillcolor{currentfill}%
\pgfsetlinewidth{0.000000pt}%
\definecolor{currentstroke}{rgb}{1.000000,0.894118,0.788235}%
\pgfsetstrokecolor{currentstroke}%
\pgfsetdash{}{0pt}%
\pgfpathmoveto{\pgfqpoint{3.259965in}{3.320791in}}%
\pgfpathlineto{\pgfqpoint{3.233018in}{3.336349in}}%
\pgfpathlineto{\pgfqpoint{3.201279in}{3.083605in}}%
\pgfpathlineto{\pgfqpoint{3.228226in}{3.068047in}}%
\pgfpathlineto{\pgfqpoint{3.259965in}{3.320791in}}%
\pgfpathclose%
\pgfusepath{fill}%
\end{pgfscope}%
\begin{pgfscope}%
\pgfpathrectangle{\pgfqpoint{0.765000in}{0.660000in}}{\pgfqpoint{4.620000in}{4.620000in}}%
\pgfusepath{clip}%
\pgfsetbuttcap%
\pgfsetroundjoin%
\definecolor{currentfill}{rgb}{1.000000,0.894118,0.788235}%
\pgfsetfillcolor{currentfill}%
\pgfsetlinewidth{0.000000pt}%
\definecolor{currentstroke}{rgb}{1.000000,0.894118,0.788235}%
\pgfsetstrokecolor{currentstroke}%
\pgfsetdash{}{0pt}%
\pgfpathmoveto{\pgfqpoint{3.206071in}{3.289675in}}%
\pgfpathlineto{\pgfqpoint{3.206071in}{3.320791in}}%
\pgfpathlineto{\pgfqpoint{3.174331in}{3.068047in}}%
\pgfpathlineto{\pgfqpoint{3.201279in}{3.021373in}}%
\pgfpathlineto{\pgfqpoint{3.206071in}{3.289675in}}%
\pgfpathclose%
\pgfusepath{fill}%
\end{pgfscope}%
\begin{pgfscope}%
\pgfpathrectangle{\pgfqpoint{0.765000in}{0.660000in}}{\pgfqpoint{4.620000in}{4.620000in}}%
\pgfusepath{clip}%
\pgfsetbuttcap%
\pgfsetroundjoin%
\definecolor{currentfill}{rgb}{1.000000,0.894118,0.788235}%
\pgfsetfillcolor{currentfill}%
\pgfsetlinewidth{0.000000pt}%
\definecolor{currentstroke}{rgb}{1.000000,0.894118,0.788235}%
\pgfsetstrokecolor{currentstroke}%
\pgfsetdash{}{0pt}%
\pgfpathmoveto{\pgfqpoint{3.233018in}{3.274117in}}%
\pgfpathlineto{\pgfqpoint{3.233018in}{3.305233in}}%
\pgfpathlineto{\pgfqpoint{3.201279in}{3.052489in}}%
\pgfpathlineto{\pgfqpoint{3.174331in}{3.036931in}}%
\pgfpathlineto{\pgfqpoint{3.233018in}{3.274117in}}%
\pgfpathclose%
\pgfusepath{fill}%
\end{pgfscope}%
\begin{pgfscope}%
\pgfpathrectangle{\pgfqpoint{0.765000in}{0.660000in}}{\pgfqpoint{4.620000in}{4.620000in}}%
\pgfusepath{clip}%
\pgfsetbuttcap%
\pgfsetroundjoin%
\definecolor{currentfill}{rgb}{1.000000,0.894118,0.788235}%
\pgfsetfillcolor{currentfill}%
\pgfsetlinewidth{0.000000pt}%
\definecolor{currentstroke}{rgb}{1.000000,0.894118,0.788235}%
\pgfsetstrokecolor{currentstroke}%
\pgfsetdash{}{0pt}%
\pgfpathmoveto{\pgfqpoint{3.233018in}{3.305233in}}%
\pgfpathlineto{\pgfqpoint{3.259965in}{3.320791in}}%
\pgfpathlineto{\pgfqpoint{3.228226in}{3.068047in}}%
\pgfpathlineto{\pgfqpoint{3.201279in}{3.052489in}}%
\pgfpathlineto{\pgfqpoint{3.233018in}{3.305233in}}%
\pgfpathclose%
\pgfusepath{fill}%
\end{pgfscope}%
\begin{pgfscope}%
\pgfpathrectangle{\pgfqpoint{0.765000in}{0.660000in}}{\pgfqpoint{4.620000in}{4.620000in}}%
\pgfusepath{clip}%
\pgfsetbuttcap%
\pgfsetroundjoin%
\definecolor{currentfill}{rgb}{1.000000,0.894118,0.788235}%
\pgfsetfillcolor{currentfill}%
\pgfsetlinewidth{0.000000pt}%
\definecolor{currentstroke}{rgb}{1.000000,0.894118,0.788235}%
\pgfsetstrokecolor{currentstroke}%
\pgfsetdash{}{0pt}%
\pgfpathmoveto{\pgfqpoint{3.206071in}{3.320791in}}%
\pgfpathlineto{\pgfqpoint{3.233018in}{3.305233in}}%
\pgfpathlineto{\pgfqpoint{3.201279in}{3.052489in}}%
\pgfpathlineto{\pgfqpoint{3.174331in}{3.068047in}}%
\pgfpathlineto{\pgfqpoint{3.206071in}{3.320791in}}%
\pgfpathclose%
\pgfusepath{fill}%
\end{pgfscope}%
\begin{pgfscope}%
\pgfpathrectangle{\pgfqpoint{0.765000in}{0.660000in}}{\pgfqpoint{4.620000in}{4.620000in}}%
\pgfusepath{clip}%
\pgfsetbuttcap%
\pgfsetroundjoin%
\definecolor{currentfill}{rgb}{1.000000,0.894118,0.788235}%
\pgfsetfillcolor{currentfill}%
\pgfsetlinewidth{0.000000pt}%
\definecolor{currentstroke}{rgb}{1.000000,0.894118,0.788235}%
\pgfsetstrokecolor{currentstroke}%
\pgfsetdash{}{0pt}%
\pgfpathmoveto{\pgfqpoint{3.318632in}{2.917586in}}%
\pgfpathlineto{\pgfqpoint{3.291685in}{2.902028in}}%
\pgfpathlineto{\pgfqpoint{3.318632in}{2.886470in}}%
\pgfpathlineto{\pgfqpoint{3.345579in}{2.902028in}}%
\pgfpathlineto{\pgfqpoint{3.318632in}{2.917586in}}%
\pgfpathclose%
\pgfusepath{fill}%
\end{pgfscope}%
\begin{pgfscope}%
\pgfpathrectangle{\pgfqpoint{0.765000in}{0.660000in}}{\pgfqpoint{4.620000in}{4.620000in}}%
\pgfusepath{clip}%
\pgfsetbuttcap%
\pgfsetroundjoin%
\definecolor{currentfill}{rgb}{1.000000,0.894118,0.788235}%
\pgfsetfillcolor{currentfill}%
\pgfsetlinewidth{0.000000pt}%
\definecolor{currentstroke}{rgb}{1.000000,0.894118,0.788235}%
\pgfsetstrokecolor{currentstroke}%
\pgfsetdash{}{0pt}%
\pgfpathmoveto{\pgfqpoint{3.318632in}{2.917586in}}%
\pgfpathlineto{\pgfqpoint{3.291685in}{2.902028in}}%
\pgfpathlineto{\pgfqpoint{3.291685in}{2.933144in}}%
\pgfpathlineto{\pgfqpoint{3.318632in}{2.948702in}}%
\pgfpathlineto{\pgfqpoint{3.318632in}{2.917586in}}%
\pgfpathclose%
\pgfusepath{fill}%
\end{pgfscope}%
\begin{pgfscope}%
\pgfpathrectangle{\pgfqpoint{0.765000in}{0.660000in}}{\pgfqpoint{4.620000in}{4.620000in}}%
\pgfusepath{clip}%
\pgfsetbuttcap%
\pgfsetroundjoin%
\definecolor{currentfill}{rgb}{1.000000,0.894118,0.788235}%
\pgfsetfillcolor{currentfill}%
\pgfsetlinewidth{0.000000pt}%
\definecolor{currentstroke}{rgb}{1.000000,0.894118,0.788235}%
\pgfsetstrokecolor{currentstroke}%
\pgfsetdash{}{0pt}%
\pgfpathmoveto{\pgfqpoint{3.318632in}{2.917586in}}%
\pgfpathlineto{\pgfqpoint{3.345579in}{2.902028in}}%
\pgfpathlineto{\pgfqpoint{3.345579in}{2.933144in}}%
\pgfpathlineto{\pgfqpoint{3.318632in}{2.948702in}}%
\pgfpathlineto{\pgfqpoint{3.318632in}{2.917586in}}%
\pgfpathclose%
\pgfusepath{fill}%
\end{pgfscope}%
\begin{pgfscope}%
\pgfpathrectangle{\pgfqpoint{0.765000in}{0.660000in}}{\pgfqpoint{4.620000in}{4.620000in}}%
\pgfusepath{clip}%
\pgfsetbuttcap%
\pgfsetroundjoin%
\definecolor{currentfill}{rgb}{1.000000,0.894118,0.788235}%
\pgfsetfillcolor{currentfill}%
\pgfsetlinewidth{0.000000pt}%
\definecolor{currentstroke}{rgb}{1.000000,0.894118,0.788235}%
\pgfsetstrokecolor{currentstroke}%
\pgfsetdash{}{0pt}%
\pgfpathmoveto{\pgfqpoint{3.325787in}{2.910799in}}%
\pgfpathlineto{\pgfqpoint{3.298839in}{2.895241in}}%
\pgfpathlineto{\pgfqpoint{3.325787in}{2.879683in}}%
\pgfpathlineto{\pgfqpoint{3.352734in}{2.895241in}}%
\pgfpathlineto{\pgfqpoint{3.325787in}{2.910799in}}%
\pgfpathclose%
\pgfusepath{fill}%
\end{pgfscope}%
\begin{pgfscope}%
\pgfpathrectangle{\pgfqpoint{0.765000in}{0.660000in}}{\pgfqpoint{4.620000in}{4.620000in}}%
\pgfusepath{clip}%
\pgfsetbuttcap%
\pgfsetroundjoin%
\definecolor{currentfill}{rgb}{1.000000,0.894118,0.788235}%
\pgfsetfillcolor{currentfill}%
\pgfsetlinewidth{0.000000pt}%
\definecolor{currentstroke}{rgb}{1.000000,0.894118,0.788235}%
\pgfsetstrokecolor{currentstroke}%
\pgfsetdash{}{0pt}%
\pgfpathmoveto{\pgfqpoint{3.325787in}{2.910799in}}%
\pgfpathlineto{\pgfqpoint{3.298839in}{2.895241in}}%
\pgfpathlineto{\pgfqpoint{3.298839in}{2.926357in}}%
\pgfpathlineto{\pgfqpoint{3.325787in}{2.941915in}}%
\pgfpathlineto{\pgfqpoint{3.325787in}{2.910799in}}%
\pgfpathclose%
\pgfusepath{fill}%
\end{pgfscope}%
\begin{pgfscope}%
\pgfpathrectangle{\pgfqpoint{0.765000in}{0.660000in}}{\pgfqpoint{4.620000in}{4.620000in}}%
\pgfusepath{clip}%
\pgfsetbuttcap%
\pgfsetroundjoin%
\definecolor{currentfill}{rgb}{1.000000,0.894118,0.788235}%
\pgfsetfillcolor{currentfill}%
\pgfsetlinewidth{0.000000pt}%
\definecolor{currentstroke}{rgb}{1.000000,0.894118,0.788235}%
\pgfsetstrokecolor{currentstroke}%
\pgfsetdash{}{0pt}%
\pgfpathmoveto{\pgfqpoint{3.325787in}{2.910799in}}%
\pgfpathlineto{\pgfqpoint{3.352734in}{2.895241in}}%
\pgfpathlineto{\pgfqpoint{3.352734in}{2.926357in}}%
\pgfpathlineto{\pgfqpoint{3.325787in}{2.941915in}}%
\pgfpathlineto{\pgfqpoint{3.325787in}{2.910799in}}%
\pgfpathclose%
\pgfusepath{fill}%
\end{pgfscope}%
\begin{pgfscope}%
\pgfpathrectangle{\pgfqpoint{0.765000in}{0.660000in}}{\pgfqpoint{4.620000in}{4.620000in}}%
\pgfusepath{clip}%
\pgfsetbuttcap%
\pgfsetroundjoin%
\definecolor{currentfill}{rgb}{1.000000,0.894118,0.788235}%
\pgfsetfillcolor{currentfill}%
\pgfsetlinewidth{0.000000pt}%
\definecolor{currentstroke}{rgb}{1.000000,0.894118,0.788235}%
\pgfsetstrokecolor{currentstroke}%
\pgfsetdash{}{0pt}%
\pgfpathmoveto{\pgfqpoint{3.318632in}{2.948702in}}%
\pgfpathlineto{\pgfqpoint{3.291685in}{2.933144in}}%
\pgfpathlineto{\pgfqpoint{3.318632in}{2.917586in}}%
\pgfpathlineto{\pgfqpoint{3.345579in}{2.933144in}}%
\pgfpathlineto{\pgfqpoint{3.318632in}{2.948702in}}%
\pgfpathclose%
\pgfusepath{fill}%
\end{pgfscope}%
\begin{pgfscope}%
\pgfpathrectangle{\pgfqpoint{0.765000in}{0.660000in}}{\pgfqpoint{4.620000in}{4.620000in}}%
\pgfusepath{clip}%
\pgfsetbuttcap%
\pgfsetroundjoin%
\definecolor{currentfill}{rgb}{1.000000,0.894118,0.788235}%
\pgfsetfillcolor{currentfill}%
\pgfsetlinewidth{0.000000pt}%
\definecolor{currentstroke}{rgb}{1.000000,0.894118,0.788235}%
\pgfsetstrokecolor{currentstroke}%
\pgfsetdash{}{0pt}%
\pgfpathmoveto{\pgfqpoint{3.318632in}{2.886470in}}%
\pgfpathlineto{\pgfqpoint{3.345579in}{2.902028in}}%
\pgfpathlineto{\pgfqpoint{3.345579in}{2.933144in}}%
\pgfpathlineto{\pgfqpoint{3.318632in}{2.917586in}}%
\pgfpathlineto{\pgfqpoint{3.318632in}{2.886470in}}%
\pgfpathclose%
\pgfusepath{fill}%
\end{pgfscope}%
\begin{pgfscope}%
\pgfpathrectangle{\pgfqpoint{0.765000in}{0.660000in}}{\pgfqpoint{4.620000in}{4.620000in}}%
\pgfusepath{clip}%
\pgfsetbuttcap%
\pgfsetroundjoin%
\definecolor{currentfill}{rgb}{1.000000,0.894118,0.788235}%
\pgfsetfillcolor{currentfill}%
\pgfsetlinewidth{0.000000pt}%
\definecolor{currentstroke}{rgb}{1.000000,0.894118,0.788235}%
\pgfsetstrokecolor{currentstroke}%
\pgfsetdash{}{0pt}%
\pgfpathmoveto{\pgfqpoint{3.291685in}{2.902028in}}%
\pgfpathlineto{\pgfqpoint{3.318632in}{2.886470in}}%
\pgfpathlineto{\pgfqpoint{3.318632in}{2.917586in}}%
\pgfpathlineto{\pgfqpoint{3.291685in}{2.933144in}}%
\pgfpathlineto{\pgfqpoint{3.291685in}{2.902028in}}%
\pgfpathclose%
\pgfusepath{fill}%
\end{pgfscope}%
\begin{pgfscope}%
\pgfpathrectangle{\pgfqpoint{0.765000in}{0.660000in}}{\pgfqpoint{4.620000in}{4.620000in}}%
\pgfusepath{clip}%
\pgfsetbuttcap%
\pgfsetroundjoin%
\definecolor{currentfill}{rgb}{1.000000,0.894118,0.788235}%
\pgfsetfillcolor{currentfill}%
\pgfsetlinewidth{0.000000pt}%
\definecolor{currentstroke}{rgb}{1.000000,0.894118,0.788235}%
\pgfsetstrokecolor{currentstroke}%
\pgfsetdash{}{0pt}%
\pgfpathmoveto{\pgfqpoint{3.325787in}{2.941915in}}%
\pgfpathlineto{\pgfqpoint{3.298839in}{2.926357in}}%
\pgfpathlineto{\pgfqpoint{3.325787in}{2.910799in}}%
\pgfpathlineto{\pgfqpoint{3.352734in}{2.926357in}}%
\pgfpathlineto{\pgfqpoint{3.325787in}{2.941915in}}%
\pgfpathclose%
\pgfusepath{fill}%
\end{pgfscope}%
\begin{pgfscope}%
\pgfpathrectangle{\pgfqpoint{0.765000in}{0.660000in}}{\pgfqpoint{4.620000in}{4.620000in}}%
\pgfusepath{clip}%
\pgfsetbuttcap%
\pgfsetroundjoin%
\definecolor{currentfill}{rgb}{1.000000,0.894118,0.788235}%
\pgfsetfillcolor{currentfill}%
\pgfsetlinewidth{0.000000pt}%
\definecolor{currentstroke}{rgb}{1.000000,0.894118,0.788235}%
\pgfsetstrokecolor{currentstroke}%
\pgfsetdash{}{0pt}%
\pgfpathmoveto{\pgfqpoint{3.325787in}{2.879683in}}%
\pgfpathlineto{\pgfqpoint{3.352734in}{2.895241in}}%
\pgfpathlineto{\pgfqpoint{3.352734in}{2.926357in}}%
\pgfpathlineto{\pgfqpoint{3.325787in}{2.910799in}}%
\pgfpathlineto{\pgfqpoint{3.325787in}{2.879683in}}%
\pgfpathclose%
\pgfusepath{fill}%
\end{pgfscope}%
\begin{pgfscope}%
\pgfpathrectangle{\pgfqpoint{0.765000in}{0.660000in}}{\pgfqpoint{4.620000in}{4.620000in}}%
\pgfusepath{clip}%
\pgfsetbuttcap%
\pgfsetroundjoin%
\definecolor{currentfill}{rgb}{1.000000,0.894118,0.788235}%
\pgfsetfillcolor{currentfill}%
\pgfsetlinewidth{0.000000pt}%
\definecolor{currentstroke}{rgb}{1.000000,0.894118,0.788235}%
\pgfsetstrokecolor{currentstroke}%
\pgfsetdash{}{0pt}%
\pgfpathmoveto{\pgfqpoint{3.298839in}{2.895241in}}%
\pgfpathlineto{\pgfqpoint{3.325787in}{2.879683in}}%
\pgfpathlineto{\pgfqpoint{3.325787in}{2.910799in}}%
\pgfpathlineto{\pgfqpoint{3.298839in}{2.926357in}}%
\pgfpathlineto{\pgfqpoint{3.298839in}{2.895241in}}%
\pgfpathclose%
\pgfusepath{fill}%
\end{pgfscope}%
\begin{pgfscope}%
\pgfpathrectangle{\pgfqpoint{0.765000in}{0.660000in}}{\pgfqpoint{4.620000in}{4.620000in}}%
\pgfusepath{clip}%
\pgfsetbuttcap%
\pgfsetroundjoin%
\definecolor{currentfill}{rgb}{1.000000,0.894118,0.788235}%
\pgfsetfillcolor{currentfill}%
\pgfsetlinewidth{0.000000pt}%
\definecolor{currentstroke}{rgb}{1.000000,0.894118,0.788235}%
\pgfsetstrokecolor{currentstroke}%
\pgfsetdash{}{0pt}%
\pgfpathmoveto{\pgfqpoint{3.318632in}{2.917586in}}%
\pgfpathlineto{\pgfqpoint{3.291685in}{2.902028in}}%
\pgfpathlineto{\pgfqpoint{3.298839in}{2.895241in}}%
\pgfpathlineto{\pgfqpoint{3.325787in}{2.910799in}}%
\pgfpathlineto{\pgfqpoint{3.318632in}{2.917586in}}%
\pgfpathclose%
\pgfusepath{fill}%
\end{pgfscope}%
\begin{pgfscope}%
\pgfpathrectangle{\pgfqpoint{0.765000in}{0.660000in}}{\pgfqpoint{4.620000in}{4.620000in}}%
\pgfusepath{clip}%
\pgfsetbuttcap%
\pgfsetroundjoin%
\definecolor{currentfill}{rgb}{1.000000,0.894118,0.788235}%
\pgfsetfillcolor{currentfill}%
\pgfsetlinewidth{0.000000pt}%
\definecolor{currentstroke}{rgb}{1.000000,0.894118,0.788235}%
\pgfsetstrokecolor{currentstroke}%
\pgfsetdash{}{0pt}%
\pgfpathmoveto{\pgfqpoint{3.345579in}{2.902028in}}%
\pgfpathlineto{\pgfqpoint{3.318632in}{2.917586in}}%
\pgfpathlineto{\pgfqpoint{3.325787in}{2.910799in}}%
\pgfpathlineto{\pgfqpoint{3.352734in}{2.895241in}}%
\pgfpathlineto{\pgfqpoint{3.345579in}{2.902028in}}%
\pgfpathclose%
\pgfusepath{fill}%
\end{pgfscope}%
\begin{pgfscope}%
\pgfpathrectangle{\pgfqpoint{0.765000in}{0.660000in}}{\pgfqpoint{4.620000in}{4.620000in}}%
\pgfusepath{clip}%
\pgfsetbuttcap%
\pgfsetroundjoin%
\definecolor{currentfill}{rgb}{1.000000,0.894118,0.788235}%
\pgfsetfillcolor{currentfill}%
\pgfsetlinewidth{0.000000pt}%
\definecolor{currentstroke}{rgb}{1.000000,0.894118,0.788235}%
\pgfsetstrokecolor{currentstroke}%
\pgfsetdash{}{0pt}%
\pgfpathmoveto{\pgfqpoint{3.318632in}{2.917586in}}%
\pgfpathlineto{\pgfqpoint{3.318632in}{2.948702in}}%
\pgfpathlineto{\pgfqpoint{3.325787in}{2.941915in}}%
\pgfpathlineto{\pgfqpoint{3.352734in}{2.895241in}}%
\pgfpathlineto{\pgfqpoint{3.318632in}{2.917586in}}%
\pgfpathclose%
\pgfusepath{fill}%
\end{pgfscope}%
\begin{pgfscope}%
\pgfpathrectangle{\pgfqpoint{0.765000in}{0.660000in}}{\pgfqpoint{4.620000in}{4.620000in}}%
\pgfusepath{clip}%
\pgfsetbuttcap%
\pgfsetroundjoin%
\definecolor{currentfill}{rgb}{1.000000,0.894118,0.788235}%
\pgfsetfillcolor{currentfill}%
\pgfsetlinewidth{0.000000pt}%
\definecolor{currentstroke}{rgb}{1.000000,0.894118,0.788235}%
\pgfsetstrokecolor{currentstroke}%
\pgfsetdash{}{0pt}%
\pgfpathmoveto{\pgfqpoint{3.345579in}{2.902028in}}%
\pgfpathlineto{\pgfqpoint{3.345579in}{2.933144in}}%
\pgfpathlineto{\pgfqpoint{3.352734in}{2.926357in}}%
\pgfpathlineto{\pgfqpoint{3.325787in}{2.910799in}}%
\pgfpathlineto{\pgfqpoint{3.345579in}{2.902028in}}%
\pgfpathclose%
\pgfusepath{fill}%
\end{pgfscope}%
\begin{pgfscope}%
\pgfpathrectangle{\pgfqpoint{0.765000in}{0.660000in}}{\pgfqpoint{4.620000in}{4.620000in}}%
\pgfusepath{clip}%
\pgfsetbuttcap%
\pgfsetroundjoin%
\definecolor{currentfill}{rgb}{1.000000,0.894118,0.788235}%
\pgfsetfillcolor{currentfill}%
\pgfsetlinewidth{0.000000pt}%
\definecolor{currentstroke}{rgb}{1.000000,0.894118,0.788235}%
\pgfsetstrokecolor{currentstroke}%
\pgfsetdash{}{0pt}%
\pgfpathmoveto{\pgfqpoint{3.291685in}{2.902028in}}%
\pgfpathlineto{\pgfqpoint{3.318632in}{2.886470in}}%
\pgfpathlineto{\pgfqpoint{3.325787in}{2.879683in}}%
\pgfpathlineto{\pgfqpoint{3.298839in}{2.895241in}}%
\pgfpathlineto{\pgfqpoint{3.291685in}{2.902028in}}%
\pgfpathclose%
\pgfusepath{fill}%
\end{pgfscope}%
\begin{pgfscope}%
\pgfpathrectangle{\pgfqpoint{0.765000in}{0.660000in}}{\pgfqpoint{4.620000in}{4.620000in}}%
\pgfusepath{clip}%
\pgfsetbuttcap%
\pgfsetroundjoin%
\definecolor{currentfill}{rgb}{1.000000,0.894118,0.788235}%
\pgfsetfillcolor{currentfill}%
\pgfsetlinewidth{0.000000pt}%
\definecolor{currentstroke}{rgb}{1.000000,0.894118,0.788235}%
\pgfsetstrokecolor{currentstroke}%
\pgfsetdash{}{0pt}%
\pgfpathmoveto{\pgfqpoint{3.318632in}{2.886470in}}%
\pgfpathlineto{\pgfqpoint{3.345579in}{2.902028in}}%
\pgfpathlineto{\pgfqpoint{3.352734in}{2.895241in}}%
\pgfpathlineto{\pgfqpoint{3.325787in}{2.879683in}}%
\pgfpathlineto{\pgfqpoint{3.318632in}{2.886470in}}%
\pgfpathclose%
\pgfusepath{fill}%
\end{pgfscope}%
\begin{pgfscope}%
\pgfpathrectangle{\pgfqpoint{0.765000in}{0.660000in}}{\pgfqpoint{4.620000in}{4.620000in}}%
\pgfusepath{clip}%
\pgfsetbuttcap%
\pgfsetroundjoin%
\definecolor{currentfill}{rgb}{1.000000,0.894118,0.788235}%
\pgfsetfillcolor{currentfill}%
\pgfsetlinewidth{0.000000pt}%
\definecolor{currentstroke}{rgb}{1.000000,0.894118,0.788235}%
\pgfsetstrokecolor{currentstroke}%
\pgfsetdash{}{0pt}%
\pgfpathmoveto{\pgfqpoint{3.318632in}{2.948702in}}%
\pgfpathlineto{\pgfqpoint{3.291685in}{2.933144in}}%
\pgfpathlineto{\pgfqpoint{3.298839in}{2.926357in}}%
\pgfpathlineto{\pgfqpoint{3.325787in}{2.941915in}}%
\pgfpathlineto{\pgfqpoint{3.318632in}{2.948702in}}%
\pgfpathclose%
\pgfusepath{fill}%
\end{pgfscope}%
\begin{pgfscope}%
\pgfpathrectangle{\pgfqpoint{0.765000in}{0.660000in}}{\pgfqpoint{4.620000in}{4.620000in}}%
\pgfusepath{clip}%
\pgfsetbuttcap%
\pgfsetroundjoin%
\definecolor{currentfill}{rgb}{1.000000,0.894118,0.788235}%
\pgfsetfillcolor{currentfill}%
\pgfsetlinewidth{0.000000pt}%
\definecolor{currentstroke}{rgb}{1.000000,0.894118,0.788235}%
\pgfsetstrokecolor{currentstroke}%
\pgfsetdash{}{0pt}%
\pgfpathmoveto{\pgfqpoint{3.345579in}{2.933144in}}%
\pgfpathlineto{\pgfqpoint{3.318632in}{2.948702in}}%
\pgfpathlineto{\pgfqpoint{3.325787in}{2.941915in}}%
\pgfpathlineto{\pgfqpoint{3.352734in}{2.926357in}}%
\pgfpathlineto{\pgfqpoint{3.345579in}{2.933144in}}%
\pgfpathclose%
\pgfusepath{fill}%
\end{pgfscope}%
\begin{pgfscope}%
\pgfpathrectangle{\pgfqpoint{0.765000in}{0.660000in}}{\pgfqpoint{4.620000in}{4.620000in}}%
\pgfusepath{clip}%
\pgfsetbuttcap%
\pgfsetroundjoin%
\definecolor{currentfill}{rgb}{1.000000,0.894118,0.788235}%
\pgfsetfillcolor{currentfill}%
\pgfsetlinewidth{0.000000pt}%
\definecolor{currentstroke}{rgb}{1.000000,0.894118,0.788235}%
\pgfsetstrokecolor{currentstroke}%
\pgfsetdash{}{0pt}%
\pgfpathmoveto{\pgfqpoint{3.291685in}{2.902028in}}%
\pgfpathlineto{\pgfqpoint{3.291685in}{2.933144in}}%
\pgfpathlineto{\pgfqpoint{3.298839in}{2.926357in}}%
\pgfpathlineto{\pgfqpoint{3.325787in}{2.879683in}}%
\pgfpathlineto{\pgfqpoint{3.291685in}{2.902028in}}%
\pgfpathclose%
\pgfusepath{fill}%
\end{pgfscope}%
\begin{pgfscope}%
\pgfpathrectangle{\pgfqpoint{0.765000in}{0.660000in}}{\pgfqpoint{4.620000in}{4.620000in}}%
\pgfusepath{clip}%
\pgfsetbuttcap%
\pgfsetroundjoin%
\definecolor{currentfill}{rgb}{1.000000,0.894118,0.788235}%
\pgfsetfillcolor{currentfill}%
\pgfsetlinewidth{0.000000pt}%
\definecolor{currentstroke}{rgb}{1.000000,0.894118,0.788235}%
\pgfsetstrokecolor{currentstroke}%
\pgfsetdash{}{0pt}%
\pgfpathmoveto{\pgfqpoint{3.318632in}{2.886470in}}%
\pgfpathlineto{\pgfqpoint{3.318632in}{2.917586in}}%
\pgfpathlineto{\pgfqpoint{3.325787in}{2.910799in}}%
\pgfpathlineto{\pgfqpoint{3.298839in}{2.895241in}}%
\pgfpathlineto{\pgfqpoint{3.318632in}{2.886470in}}%
\pgfpathclose%
\pgfusepath{fill}%
\end{pgfscope}%
\begin{pgfscope}%
\pgfpathrectangle{\pgfqpoint{0.765000in}{0.660000in}}{\pgfqpoint{4.620000in}{4.620000in}}%
\pgfusepath{clip}%
\pgfsetbuttcap%
\pgfsetroundjoin%
\definecolor{currentfill}{rgb}{1.000000,0.894118,0.788235}%
\pgfsetfillcolor{currentfill}%
\pgfsetlinewidth{0.000000pt}%
\definecolor{currentstroke}{rgb}{1.000000,0.894118,0.788235}%
\pgfsetstrokecolor{currentstroke}%
\pgfsetdash{}{0pt}%
\pgfpathmoveto{\pgfqpoint{3.318632in}{2.917586in}}%
\pgfpathlineto{\pgfqpoint{3.345579in}{2.933144in}}%
\pgfpathlineto{\pgfqpoint{3.352734in}{2.926357in}}%
\pgfpathlineto{\pgfqpoint{3.325787in}{2.910799in}}%
\pgfpathlineto{\pgfqpoint{3.318632in}{2.917586in}}%
\pgfpathclose%
\pgfusepath{fill}%
\end{pgfscope}%
\begin{pgfscope}%
\pgfpathrectangle{\pgfqpoint{0.765000in}{0.660000in}}{\pgfqpoint{4.620000in}{4.620000in}}%
\pgfusepath{clip}%
\pgfsetbuttcap%
\pgfsetroundjoin%
\definecolor{currentfill}{rgb}{1.000000,0.894118,0.788235}%
\pgfsetfillcolor{currentfill}%
\pgfsetlinewidth{0.000000pt}%
\definecolor{currentstroke}{rgb}{1.000000,0.894118,0.788235}%
\pgfsetstrokecolor{currentstroke}%
\pgfsetdash{}{0pt}%
\pgfpathmoveto{\pgfqpoint{3.291685in}{2.933144in}}%
\pgfpathlineto{\pgfqpoint{3.318632in}{2.917586in}}%
\pgfpathlineto{\pgfqpoint{3.325787in}{2.910799in}}%
\pgfpathlineto{\pgfqpoint{3.298839in}{2.926357in}}%
\pgfpathlineto{\pgfqpoint{3.291685in}{2.933144in}}%
\pgfpathclose%
\pgfusepath{fill}%
\end{pgfscope}%
\begin{pgfscope}%
\pgfpathrectangle{\pgfqpoint{0.765000in}{0.660000in}}{\pgfqpoint{4.620000in}{4.620000in}}%
\pgfusepath{clip}%
\pgfsetbuttcap%
\pgfsetroundjoin%
\definecolor{currentfill}{rgb}{1.000000,0.894118,0.788235}%
\pgfsetfillcolor{currentfill}%
\pgfsetlinewidth{0.000000pt}%
\definecolor{currentstroke}{rgb}{1.000000,0.894118,0.788235}%
\pgfsetstrokecolor{currentstroke}%
\pgfsetdash{}{0pt}%
\pgfpathmoveto{\pgfqpoint{3.318632in}{2.917586in}}%
\pgfpathlineto{\pgfqpoint{3.291685in}{2.902028in}}%
\pgfpathlineto{\pgfqpoint{3.318632in}{2.886470in}}%
\pgfpathlineto{\pgfqpoint{3.345579in}{2.902028in}}%
\pgfpathlineto{\pgfqpoint{3.318632in}{2.917586in}}%
\pgfpathclose%
\pgfusepath{fill}%
\end{pgfscope}%
\begin{pgfscope}%
\pgfpathrectangle{\pgfqpoint{0.765000in}{0.660000in}}{\pgfqpoint{4.620000in}{4.620000in}}%
\pgfusepath{clip}%
\pgfsetbuttcap%
\pgfsetroundjoin%
\definecolor{currentfill}{rgb}{1.000000,0.894118,0.788235}%
\pgfsetfillcolor{currentfill}%
\pgfsetlinewidth{0.000000pt}%
\definecolor{currentstroke}{rgb}{1.000000,0.894118,0.788235}%
\pgfsetstrokecolor{currentstroke}%
\pgfsetdash{}{0pt}%
\pgfpathmoveto{\pgfqpoint{3.318632in}{2.917586in}}%
\pgfpathlineto{\pgfqpoint{3.291685in}{2.902028in}}%
\pgfpathlineto{\pgfqpoint{3.291685in}{2.933144in}}%
\pgfpathlineto{\pgfqpoint{3.318632in}{2.948702in}}%
\pgfpathlineto{\pgfqpoint{3.318632in}{2.917586in}}%
\pgfpathclose%
\pgfusepath{fill}%
\end{pgfscope}%
\begin{pgfscope}%
\pgfpathrectangle{\pgfqpoint{0.765000in}{0.660000in}}{\pgfqpoint{4.620000in}{4.620000in}}%
\pgfusepath{clip}%
\pgfsetbuttcap%
\pgfsetroundjoin%
\definecolor{currentfill}{rgb}{1.000000,0.894118,0.788235}%
\pgfsetfillcolor{currentfill}%
\pgfsetlinewidth{0.000000pt}%
\definecolor{currentstroke}{rgb}{1.000000,0.894118,0.788235}%
\pgfsetstrokecolor{currentstroke}%
\pgfsetdash{}{0pt}%
\pgfpathmoveto{\pgfqpoint{3.318632in}{2.917586in}}%
\pgfpathlineto{\pgfqpoint{3.345579in}{2.902028in}}%
\pgfpathlineto{\pgfqpoint{3.345579in}{2.933144in}}%
\pgfpathlineto{\pgfqpoint{3.318632in}{2.948702in}}%
\pgfpathlineto{\pgfqpoint{3.318632in}{2.917586in}}%
\pgfpathclose%
\pgfusepath{fill}%
\end{pgfscope}%
\begin{pgfscope}%
\pgfpathrectangle{\pgfqpoint{0.765000in}{0.660000in}}{\pgfqpoint{4.620000in}{4.620000in}}%
\pgfusepath{clip}%
\pgfsetbuttcap%
\pgfsetroundjoin%
\definecolor{currentfill}{rgb}{1.000000,0.894118,0.788235}%
\pgfsetfillcolor{currentfill}%
\pgfsetlinewidth{0.000000pt}%
\definecolor{currentstroke}{rgb}{1.000000,0.894118,0.788235}%
\pgfsetstrokecolor{currentstroke}%
\pgfsetdash{}{0pt}%
\pgfpathmoveto{\pgfqpoint{3.201279in}{3.052489in}}%
\pgfpathlineto{\pgfqpoint{3.174331in}{3.036931in}}%
\pgfpathlineto{\pgfqpoint{3.201279in}{3.021373in}}%
\pgfpathlineto{\pgfqpoint{3.228226in}{3.036931in}}%
\pgfpathlineto{\pgfqpoint{3.201279in}{3.052489in}}%
\pgfpathclose%
\pgfusepath{fill}%
\end{pgfscope}%
\begin{pgfscope}%
\pgfpathrectangle{\pgfqpoint{0.765000in}{0.660000in}}{\pgfqpoint{4.620000in}{4.620000in}}%
\pgfusepath{clip}%
\pgfsetbuttcap%
\pgfsetroundjoin%
\definecolor{currentfill}{rgb}{1.000000,0.894118,0.788235}%
\pgfsetfillcolor{currentfill}%
\pgfsetlinewidth{0.000000pt}%
\definecolor{currentstroke}{rgb}{1.000000,0.894118,0.788235}%
\pgfsetstrokecolor{currentstroke}%
\pgfsetdash{}{0pt}%
\pgfpathmoveto{\pgfqpoint{3.201279in}{3.052489in}}%
\pgfpathlineto{\pgfqpoint{3.174331in}{3.036931in}}%
\pgfpathlineto{\pgfqpoint{3.174331in}{3.068047in}}%
\pgfpathlineto{\pgfqpoint{3.201279in}{3.083605in}}%
\pgfpathlineto{\pgfqpoint{3.201279in}{3.052489in}}%
\pgfpathclose%
\pgfusepath{fill}%
\end{pgfscope}%
\begin{pgfscope}%
\pgfpathrectangle{\pgfqpoint{0.765000in}{0.660000in}}{\pgfqpoint{4.620000in}{4.620000in}}%
\pgfusepath{clip}%
\pgfsetbuttcap%
\pgfsetroundjoin%
\definecolor{currentfill}{rgb}{1.000000,0.894118,0.788235}%
\pgfsetfillcolor{currentfill}%
\pgfsetlinewidth{0.000000pt}%
\definecolor{currentstroke}{rgb}{1.000000,0.894118,0.788235}%
\pgfsetstrokecolor{currentstroke}%
\pgfsetdash{}{0pt}%
\pgfpathmoveto{\pgfqpoint{3.201279in}{3.052489in}}%
\pgfpathlineto{\pgfqpoint{3.228226in}{3.036931in}}%
\pgfpathlineto{\pgfqpoint{3.228226in}{3.068047in}}%
\pgfpathlineto{\pgfqpoint{3.201279in}{3.083605in}}%
\pgfpathlineto{\pgfqpoint{3.201279in}{3.052489in}}%
\pgfpathclose%
\pgfusepath{fill}%
\end{pgfscope}%
\begin{pgfscope}%
\pgfpathrectangle{\pgfqpoint{0.765000in}{0.660000in}}{\pgfqpoint{4.620000in}{4.620000in}}%
\pgfusepath{clip}%
\pgfsetbuttcap%
\pgfsetroundjoin%
\definecolor{currentfill}{rgb}{1.000000,0.894118,0.788235}%
\pgfsetfillcolor{currentfill}%
\pgfsetlinewidth{0.000000pt}%
\definecolor{currentstroke}{rgb}{1.000000,0.894118,0.788235}%
\pgfsetstrokecolor{currentstroke}%
\pgfsetdash{}{0pt}%
\pgfpathmoveto{\pgfqpoint{3.318632in}{2.948702in}}%
\pgfpathlineto{\pgfqpoint{3.291685in}{2.933144in}}%
\pgfpathlineto{\pgfqpoint{3.318632in}{2.917586in}}%
\pgfpathlineto{\pgfqpoint{3.345579in}{2.933144in}}%
\pgfpathlineto{\pgfqpoint{3.318632in}{2.948702in}}%
\pgfpathclose%
\pgfusepath{fill}%
\end{pgfscope}%
\begin{pgfscope}%
\pgfpathrectangle{\pgfqpoint{0.765000in}{0.660000in}}{\pgfqpoint{4.620000in}{4.620000in}}%
\pgfusepath{clip}%
\pgfsetbuttcap%
\pgfsetroundjoin%
\definecolor{currentfill}{rgb}{1.000000,0.894118,0.788235}%
\pgfsetfillcolor{currentfill}%
\pgfsetlinewidth{0.000000pt}%
\definecolor{currentstroke}{rgb}{1.000000,0.894118,0.788235}%
\pgfsetstrokecolor{currentstroke}%
\pgfsetdash{}{0pt}%
\pgfpathmoveto{\pgfqpoint{3.318632in}{2.886470in}}%
\pgfpathlineto{\pgfqpoint{3.345579in}{2.902028in}}%
\pgfpathlineto{\pgfqpoint{3.345579in}{2.933144in}}%
\pgfpathlineto{\pgfqpoint{3.318632in}{2.917586in}}%
\pgfpathlineto{\pgfqpoint{3.318632in}{2.886470in}}%
\pgfpathclose%
\pgfusepath{fill}%
\end{pgfscope}%
\begin{pgfscope}%
\pgfpathrectangle{\pgfqpoint{0.765000in}{0.660000in}}{\pgfqpoint{4.620000in}{4.620000in}}%
\pgfusepath{clip}%
\pgfsetbuttcap%
\pgfsetroundjoin%
\definecolor{currentfill}{rgb}{1.000000,0.894118,0.788235}%
\pgfsetfillcolor{currentfill}%
\pgfsetlinewidth{0.000000pt}%
\definecolor{currentstroke}{rgb}{1.000000,0.894118,0.788235}%
\pgfsetstrokecolor{currentstroke}%
\pgfsetdash{}{0pt}%
\pgfpathmoveto{\pgfqpoint{3.291685in}{2.902028in}}%
\pgfpathlineto{\pgfqpoint{3.318632in}{2.886470in}}%
\pgfpathlineto{\pgfqpoint{3.318632in}{2.917586in}}%
\pgfpathlineto{\pgfqpoint{3.291685in}{2.933144in}}%
\pgfpathlineto{\pgfqpoint{3.291685in}{2.902028in}}%
\pgfpathclose%
\pgfusepath{fill}%
\end{pgfscope}%
\begin{pgfscope}%
\pgfpathrectangle{\pgfqpoint{0.765000in}{0.660000in}}{\pgfqpoint{4.620000in}{4.620000in}}%
\pgfusepath{clip}%
\pgfsetbuttcap%
\pgfsetroundjoin%
\definecolor{currentfill}{rgb}{1.000000,0.894118,0.788235}%
\pgfsetfillcolor{currentfill}%
\pgfsetlinewidth{0.000000pt}%
\definecolor{currentstroke}{rgb}{1.000000,0.894118,0.788235}%
\pgfsetstrokecolor{currentstroke}%
\pgfsetdash{}{0pt}%
\pgfpathmoveto{\pgfqpoint{3.201279in}{3.083605in}}%
\pgfpathlineto{\pgfqpoint{3.174331in}{3.068047in}}%
\pgfpathlineto{\pgfqpoint{3.201279in}{3.052489in}}%
\pgfpathlineto{\pgfqpoint{3.228226in}{3.068047in}}%
\pgfpathlineto{\pgfqpoint{3.201279in}{3.083605in}}%
\pgfpathclose%
\pgfusepath{fill}%
\end{pgfscope}%
\begin{pgfscope}%
\pgfpathrectangle{\pgfqpoint{0.765000in}{0.660000in}}{\pgfqpoint{4.620000in}{4.620000in}}%
\pgfusepath{clip}%
\pgfsetbuttcap%
\pgfsetroundjoin%
\definecolor{currentfill}{rgb}{1.000000,0.894118,0.788235}%
\pgfsetfillcolor{currentfill}%
\pgfsetlinewidth{0.000000pt}%
\definecolor{currentstroke}{rgb}{1.000000,0.894118,0.788235}%
\pgfsetstrokecolor{currentstroke}%
\pgfsetdash{}{0pt}%
\pgfpathmoveto{\pgfqpoint{3.201279in}{3.021373in}}%
\pgfpathlineto{\pgfqpoint{3.228226in}{3.036931in}}%
\pgfpathlineto{\pgfqpoint{3.228226in}{3.068047in}}%
\pgfpathlineto{\pgfqpoint{3.201279in}{3.052489in}}%
\pgfpathlineto{\pgfqpoint{3.201279in}{3.021373in}}%
\pgfpathclose%
\pgfusepath{fill}%
\end{pgfscope}%
\begin{pgfscope}%
\pgfpathrectangle{\pgfqpoint{0.765000in}{0.660000in}}{\pgfqpoint{4.620000in}{4.620000in}}%
\pgfusepath{clip}%
\pgfsetbuttcap%
\pgfsetroundjoin%
\definecolor{currentfill}{rgb}{1.000000,0.894118,0.788235}%
\pgfsetfillcolor{currentfill}%
\pgfsetlinewidth{0.000000pt}%
\definecolor{currentstroke}{rgb}{1.000000,0.894118,0.788235}%
\pgfsetstrokecolor{currentstroke}%
\pgfsetdash{}{0pt}%
\pgfpathmoveto{\pgfqpoint{3.174331in}{3.036931in}}%
\pgfpathlineto{\pgfqpoint{3.201279in}{3.021373in}}%
\pgfpathlineto{\pgfqpoint{3.201279in}{3.052489in}}%
\pgfpathlineto{\pgfqpoint{3.174331in}{3.068047in}}%
\pgfpathlineto{\pgfqpoint{3.174331in}{3.036931in}}%
\pgfpathclose%
\pgfusepath{fill}%
\end{pgfscope}%
\begin{pgfscope}%
\pgfpathrectangle{\pgfqpoint{0.765000in}{0.660000in}}{\pgfqpoint{4.620000in}{4.620000in}}%
\pgfusepath{clip}%
\pgfsetbuttcap%
\pgfsetroundjoin%
\definecolor{currentfill}{rgb}{1.000000,0.894118,0.788235}%
\pgfsetfillcolor{currentfill}%
\pgfsetlinewidth{0.000000pt}%
\definecolor{currentstroke}{rgb}{1.000000,0.894118,0.788235}%
\pgfsetstrokecolor{currentstroke}%
\pgfsetdash{}{0pt}%
\pgfpathmoveto{\pgfqpoint{3.318632in}{2.917586in}}%
\pgfpathlineto{\pgfqpoint{3.291685in}{2.902028in}}%
\pgfpathlineto{\pgfqpoint{3.174331in}{3.036931in}}%
\pgfpathlineto{\pgfqpoint{3.201279in}{3.052489in}}%
\pgfpathlineto{\pgfqpoint{3.318632in}{2.917586in}}%
\pgfpathclose%
\pgfusepath{fill}%
\end{pgfscope}%
\begin{pgfscope}%
\pgfpathrectangle{\pgfqpoint{0.765000in}{0.660000in}}{\pgfqpoint{4.620000in}{4.620000in}}%
\pgfusepath{clip}%
\pgfsetbuttcap%
\pgfsetroundjoin%
\definecolor{currentfill}{rgb}{1.000000,0.894118,0.788235}%
\pgfsetfillcolor{currentfill}%
\pgfsetlinewidth{0.000000pt}%
\definecolor{currentstroke}{rgb}{1.000000,0.894118,0.788235}%
\pgfsetstrokecolor{currentstroke}%
\pgfsetdash{}{0pt}%
\pgfpathmoveto{\pgfqpoint{3.345579in}{2.902028in}}%
\pgfpathlineto{\pgfqpoint{3.318632in}{2.917586in}}%
\pgfpathlineto{\pgfqpoint{3.201279in}{3.052489in}}%
\pgfpathlineto{\pgfqpoint{3.228226in}{3.036931in}}%
\pgfpathlineto{\pgfqpoint{3.345579in}{2.902028in}}%
\pgfpathclose%
\pgfusepath{fill}%
\end{pgfscope}%
\begin{pgfscope}%
\pgfpathrectangle{\pgfqpoint{0.765000in}{0.660000in}}{\pgfqpoint{4.620000in}{4.620000in}}%
\pgfusepath{clip}%
\pgfsetbuttcap%
\pgfsetroundjoin%
\definecolor{currentfill}{rgb}{1.000000,0.894118,0.788235}%
\pgfsetfillcolor{currentfill}%
\pgfsetlinewidth{0.000000pt}%
\definecolor{currentstroke}{rgb}{1.000000,0.894118,0.788235}%
\pgfsetstrokecolor{currentstroke}%
\pgfsetdash{}{0pt}%
\pgfpathmoveto{\pgfqpoint{3.318632in}{2.917586in}}%
\pgfpathlineto{\pgfqpoint{3.318632in}{2.948702in}}%
\pgfpathlineto{\pgfqpoint{3.201279in}{3.083605in}}%
\pgfpathlineto{\pgfqpoint{3.228226in}{3.036931in}}%
\pgfpathlineto{\pgfqpoint{3.318632in}{2.917586in}}%
\pgfpathclose%
\pgfusepath{fill}%
\end{pgfscope}%
\begin{pgfscope}%
\pgfpathrectangle{\pgfqpoint{0.765000in}{0.660000in}}{\pgfqpoint{4.620000in}{4.620000in}}%
\pgfusepath{clip}%
\pgfsetbuttcap%
\pgfsetroundjoin%
\definecolor{currentfill}{rgb}{1.000000,0.894118,0.788235}%
\pgfsetfillcolor{currentfill}%
\pgfsetlinewidth{0.000000pt}%
\definecolor{currentstroke}{rgb}{1.000000,0.894118,0.788235}%
\pgfsetstrokecolor{currentstroke}%
\pgfsetdash{}{0pt}%
\pgfpathmoveto{\pgfqpoint{3.345579in}{2.902028in}}%
\pgfpathlineto{\pgfqpoint{3.345579in}{2.933144in}}%
\pgfpathlineto{\pgfqpoint{3.228226in}{3.068047in}}%
\pgfpathlineto{\pgfqpoint{3.201279in}{3.052489in}}%
\pgfpathlineto{\pgfqpoint{3.345579in}{2.902028in}}%
\pgfpathclose%
\pgfusepath{fill}%
\end{pgfscope}%
\begin{pgfscope}%
\pgfpathrectangle{\pgfqpoint{0.765000in}{0.660000in}}{\pgfqpoint{4.620000in}{4.620000in}}%
\pgfusepath{clip}%
\pgfsetbuttcap%
\pgfsetroundjoin%
\definecolor{currentfill}{rgb}{1.000000,0.894118,0.788235}%
\pgfsetfillcolor{currentfill}%
\pgfsetlinewidth{0.000000pt}%
\definecolor{currentstroke}{rgb}{1.000000,0.894118,0.788235}%
\pgfsetstrokecolor{currentstroke}%
\pgfsetdash{}{0pt}%
\pgfpathmoveto{\pgfqpoint{3.291685in}{2.902028in}}%
\pgfpathlineto{\pgfqpoint{3.318632in}{2.886470in}}%
\pgfpathlineto{\pgfqpoint{3.201279in}{3.021373in}}%
\pgfpathlineto{\pgfqpoint{3.174331in}{3.036931in}}%
\pgfpathlineto{\pgfqpoint{3.291685in}{2.902028in}}%
\pgfpathclose%
\pgfusepath{fill}%
\end{pgfscope}%
\begin{pgfscope}%
\pgfpathrectangle{\pgfqpoint{0.765000in}{0.660000in}}{\pgfqpoint{4.620000in}{4.620000in}}%
\pgfusepath{clip}%
\pgfsetbuttcap%
\pgfsetroundjoin%
\definecolor{currentfill}{rgb}{1.000000,0.894118,0.788235}%
\pgfsetfillcolor{currentfill}%
\pgfsetlinewidth{0.000000pt}%
\definecolor{currentstroke}{rgb}{1.000000,0.894118,0.788235}%
\pgfsetstrokecolor{currentstroke}%
\pgfsetdash{}{0pt}%
\pgfpathmoveto{\pgfqpoint{3.318632in}{2.886470in}}%
\pgfpathlineto{\pgfqpoint{3.345579in}{2.902028in}}%
\pgfpathlineto{\pgfqpoint{3.228226in}{3.036931in}}%
\pgfpathlineto{\pgfqpoint{3.201279in}{3.021373in}}%
\pgfpathlineto{\pgfqpoint{3.318632in}{2.886470in}}%
\pgfpathclose%
\pgfusepath{fill}%
\end{pgfscope}%
\begin{pgfscope}%
\pgfpathrectangle{\pgfqpoint{0.765000in}{0.660000in}}{\pgfqpoint{4.620000in}{4.620000in}}%
\pgfusepath{clip}%
\pgfsetbuttcap%
\pgfsetroundjoin%
\definecolor{currentfill}{rgb}{1.000000,0.894118,0.788235}%
\pgfsetfillcolor{currentfill}%
\pgfsetlinewidth{0.000000pt}%
\definecolor{currentstroke}{rgb}{1.000000,0.894118,0.788235}%
\pgfsetstrokecolor{currentstroke}%
\pgfsetdash{}{0pt}%
\pgfpathmoveto{\pgfqpoint{3.318632in}{2.948702in}}%
\pgfpathlineto{\pgfqpoint{3.291685in}{2.933144in}}%
\pgfpathlineto{\pgfqpoint{3.174331in}{3.068047in}}%
\pgfpathlineto{\pgfqpoint{3.201279in}{3.083605in}}%
\pgfpathlineto{\pgfqpoint{3.318632in}{2.948702in}}%
\pgfpathclose%
\pgfusepath{fill}%
\end{pgfscope}%
\begin{pgfscope}%
\pgfpathrectangle{\pgfqpoint{0.765000in}{0.660000in}}{\pgfqpoint{4.620000in}{4.620000in}}%
\pgfusepath{clip}%
\pgfsetbuttcap%
\pgfsetroundjoin%
\definecolor{currentfill}{rgb}{1.000000,0.894118,0.788235}%
\pgfsetfillcolor{currentfill}%
\pgfsetlinewidth{0.000000pt}%
\definecolor{currentstroke}{rgb}{1.000000,0.894118,0.788235}%
\pgfsetstrokecolor{currentstroke}%
\pgfsetdash{}{0pt}%
\pgfpathmoveto{\pgfqpoint{3.345579in}{2.933144in}}%
\pgfpathlineto{\pgfqpoint{3.318632in}{2.948702in}}%
\pgfpathlineto{\pgfqpoint{3.201279in}{3.083605in}}%
\pgfpathlineto{\pgfqpoint{3.228226in}{3.068047in}}%
\pgfpathlineto{\pgfqpoint{3.345579in}{2.933144in}}%
\pgfpathclose%
\pgfusepath{fill}%
\end{pgfscope}%
\begin{pgfscope}%
\pgfpathrectangle{\pgfqpoint{0.765000in}{0.660000in}}{\pgfqpoint{4.620000in}{4.620000in}}%
\pgfusepath{clip}%
\pgfsetbuttcap%
\pgfsetroundjoin%
\definecolor{currentfill}{rgb}{1.000000,0.894118,0.788235}%
\pgfsetfillcolor{currentfill}%
\pgfsetlinewidth{0.000000pt}%
\definecolor{currentstroke}{rgb}{1.000000,0.894118,0.788235}%
\pgfsetstrokecolor{currentstroke}%
\pgfsetdash{}{0pt}%
\pgfpathmoveto{\pgfqpoint{3.291685in}{2.902028in}}%
\pgfpathlineto{\pgfqpoint{3.291685in}{2.933144in}}%
\pgfpathlineto{\pgfqpoint{3.174331in}{3.068047in}}%
\pgfpathlineto{\pgfqpoint{3.201279in}{3.021373in}}%
\pgfpathlineto{\pgfqpoint{3.291685in}{2.902028in}}%
\pgfpathclose%
\pgfusepath{fill}%
\end{pgfscope}%
\begin{pgfscope}%
\pgfpathrectangle{\pgfqpoint{0.765000in}{0.660000in}}{\pgfqpoint{4.620000in}{4.620000in}}%
\pgfusepath{clip}%
\pgfsetbuttcap%
\pgfsetroundjoin%
\definecolor{currentfill}{rgb}{1.000000,0.894118,0.788235}%
\pgfsetfillcolor{currentfill}%
\pgfsetlinewidth{0.000000pt}%
\definecolor{currentstroke}{rgb}{1.000000,0.894118,0.788235}%
\pgfsetstrokecolor{currentstroke}%
\pgfsetdash{}{0pt}%
\pgfpathmoveto{\pgfqpoint{3.318632in}{2.886470in}}%
\pgfpathlineto{\pgfqpoint{3.318632in}{2.917586in}}%
\pgfpathlineto{\pgfqpoint{3.201279in}{3.052489in}}%
\pgfpathlineto{\pgfqpoint{3.174331in}{3.036931in}}%
\pgfpathlineto{\pgfqpoint{3.318632in}{2.886470in}}%
\pgfpathclose%
\pgfusepath{fill}%
\end{pgfscope}%
\begin{pgfscope}%
\pgfpathrectangle{\pgfqpoint{0.765000in}{0.660000in}}{\pgfqpoint{4.620000in}{4.620000in}}%
\pgfusepath{clip}%
\pgfsetbuttcap%
\pgfsetroundjoin%
\definecolor{currentfill}{rgb}{1.000000,0.894118,0.788235}%
\pgfsetfillcolor{currentfill}%
\pgfsetlinewidth{0.000000pt}%
\definecolor{currentstroke}{rgb}{1.000000,0.894118,0.788235}%
\pgfsetstrokecolor{currentstroke}%
\pgfsetdash{}{0pt}%
\pgfpathmoveto{\pgfqpoint{3.318632in}{2.917586in}}%
\pgfpathlineto{\pgfqpoint{3.345579in}{2.933144in}}%
\pgfpathlineto{\pgfqpoint{3.228226in}{3.068047in}}%
\pgfpathlineto{\pgfqpoint{3.201279in}{3.052489in}}%
\pgfpathlineto{\pgfqpoint{3.318632in}{2.917586in}}%
\pgfpathclose%
\pgfusepath{fill}%
\end{pgfscope}%
\begin{pgfscope}%
\pgfpathrectangle{\pgfqpoint{0.765000in}{0.660000in}}{\pgfqpoint{4.620000in}{4.620000in}}%
\pgfusepath{clip}%
\pgfsetbuttcap%
\pgfsetroundjoin%
\definecolor{currentfill}{rgb}{1.000000,0.894118,0.788235}%
\pgfsetfillcolor{currentfill}%
\pgfsetlinewidth{0.000000pt}%
\definecolor{currentstroke}{rgb}{1.000000,0.894118,0.788235}%
\pgfsetstrokecolor{currentstroke}%
\pgfsetdash{}{0pt}%
\pgfpathmoveto{\pgfqpoint{3.291685in}{2.933144in}}%
\pgfpathlineto{\pgfqpoint{3.318632in}{2.917586in}}%
\pgfpathlineto{\pgfqpoint{3.201279in}{3.052489in}}%
\pgfpathlineto{\pgfqpoint{3.174331in}{3.068047in}}%
\pgfpathlineto{\pgfqpoint{3.291685in}{2.933144in}}%
\pgfpathclose%
\pgfusepath{fill}%
\end{pgfscope}%
\begin{pgfscope}%
\pgfpathrectangle{\pgfqpoint{0.765000in}{0.660000in}}{\pgfqpoint{4.620000in}{4.620000in}}%
\pgfusepath{clip}%
\pgfsetbuttcap%
\pgfsetroundjoin%
\definecolor{currentfill}{rgb}{1.000000,0.894118,0.788235}%
\pgfsetfillcolor{currentfill}%
\pgfsetlinewidth{0.000000pt}%
\definecolor{currentstroke}{rgb}{1.000000,0.894118,0.788235}%
\pgfsetstrokecolor{currentstroke}%
\pgfsetdash{}{0pt}%
\pgfpathmoveto{\pgfqpoint{3.693066in}{2.834300in}}%
\pgfpathlineto{\pgfqpoint{3.666119in}{2.818742in}}%
\pgfpathlineto{\pgfqpoint{3.693066in}{2.803184in}}%
\pgfpathlineto{\pgfqpoint{3.720013in}{2.818742in}}%
\pgfpathlineto{\pgfqpoint{3.693066in}{2.834300in}}%
\pgfpathclose%
\pgfusepath{fill}%
\end{pgfscope}%
\begin{pgfscope}%
\pgfpathrectangle{\pgfqpoint{0.765000in}{0.660000in}}{\pgfqpoint{4.620000in}{4.620000in}}%
\pgfusepath{clip}%
\pgfsetbuttcap%
\pgfsetroundjoin%
\definecolor{currentfill}{rgb}{1.000000,0.894118,0.788235}%
\pgfsetfillcolor{currentfill}%
\pgfsetlinewidth{0.000000pt}%
\definecolor{currentstroke}{rgb}{1.000000,0.894118,0.788235}%
\pgfsetstrokecolor{currentstroke}%
\pgfsetdash{}{0pt}%
\pgfpathmoveto{\pgfqpoint{3.693066in}{2.834300in}}%
\pgfpathlineto{\pgfqpoint{3.666119in}{2.818742in}}%
\pgfpathlineto{\pgfqpoint{3.666119in}{2.849858in}}%
\pgfpathlineto{\pgfqpoint{3.693066in}{2.865416in}}%
\pgfpathlineto{\pgfqpoint{3.693066in}{2.834300in}}%
\pgfpathclose%
\pgfusepath{fill}%
\end{pgfscope}%
\begin{pgfscope}%
\pgfpathrectangle{\pgfqpoint{0.765000in}{0.660000in}}{\pgfqpoint{4.620000in}{4.620000in}}%
\pgfusepath{clip}%
\pgfsetbuttcap%
\pgfsetroundjoin%
\definecolor{currentfill}{rgb}{1.000000,0.894118,0.788235}%
\pgfsetfillcolor{currentfill}%
\pgfsetlinewidth{0.000000pt}%
\definecolor{currentstroke}{rgb}{1.000000,0.894118,0.788235}%
\pgfsetstrokecolor{currentstroke}%
\pgfsetdash{}{0pt}%
\pgfpathmoveto{\pgfqpoint{3.693066in}{2.834300in}}%
\pgfpathlineto{\pgfqpoint{3.720013in}{2.818742in}}%
\pgfpathlineto{\pgfqpoint{3.720013in}{2.849858in}}%
\pgfpathlineto{\pgfqpoint{3.693066in}{2.865416in}}%
\pgfpathlineto{\pgfqpoint{3.693066in}{2.834300in}}%
\pgfpathclose%
\pgfusepath{fill}%
\end{pgfscope}%
\begin{pgfscope}%
\pgfpathrectangle{\pgfqpoint{0.765000in}{0.660000in}}{\pgfqpoint{4.620000in}{4.620000in}}%
\pgfusepath{clip}%
\pgfsetbuttcap%
\pgfsetroundjoin%
\definecolor{currentfill}{rgb}{1.000000,0.894118,0.788235}%
\pgfsetfillcolor{currentfill}%
\pgfsetlinewidth{0.000000pt}%
\definecolor{currentstroke}{rgb}{1.000000,0.894118,0.788235}%
\pgfsetstrokecolor{currentstroke}%
\pgfsetdash{}{0pt}%
\pgfpathmoveto{\pgfqpoint{3.538039in}{2.783646in}}%
\pgfpathlineto{\pgfqpoint{3.511092in}{2.768088in}}%
\pgfpathlineto{\pgfqpoint{3.538039in}{2.752530in}}%
\pgfpathlineto{\pgfqpoint{3.564986in}{2.768088in}}%
\pgfpathlineto{\pgfqpoint{3.538039in}{2.783646in}}%
\pgfpathclose%
\pgfusepath{fill}%
\end{pgfscope}%
\begin{pgfscope}%
\pgfpathrectangle{\pgfqpoint{0.765000in}{0.660000in}}{\pgfqpoint{4.620000in}{4.620000in}}%
\pgfusepath{clip}%
\pgfsetbuttcap%
\pgfsetroundjoin%
\definecolor{currentfill}{rgb}{1.000000,0.894118,0.788235}%
\pgfsetfillcolor{currentfill}%
\pgfsetlinewidth{0.000000pt}%
\definecolor{currentstroke}{rgb}{1.000000,0.894118,0.788235}%
\pgfsetstrokecolor{currentstroke}%
\pgfsetdash{}{0pt}%
\pgfpathmoveto{\pgfqpoint{3.538039in}{2.783646in}}%
\pgfpathlineto{\pgfqpoint{3.511092in}{2.768088in}}%
\pgfpathlineto{\pgfqpoint{3.511092in}{2.799204in}}%
\pgfpathlineto{\pgfqpoint{3.538039in}{2.814762in}}%
\pgfpathlineto{\pgfqpoint{3.538039in}{2.783646in}}%
\pgfpathclose%
\pgfusepath{fill}%
\end{pgfscope}%
\begin{pgfscope}%
\pgfpathrectangle{\pgfqpoint{0.765000in}{0.660000in}}{\pgfqpoint{4.620000in}{4.620000in}}%
\pgfusepath{clip}%
\pgfsetbuttcap%
\pgfsetroundjoin%
\definecolor{currentfill}{rgb}{1.000000,0.894118,0.788235}%
\pgfsetfillcolor{currentfill}%
\pgfsetlinewidth{0.000000pt}%
\definecolor{currentstroke}{rgb}{1.000000,0.894118,0.788235}%
\pgfsetstrokecolor{currentstroke}%
\pgfsetdash{}{0pt}%
\pgfpathmoveto{\pgfqpoint{3.538039in}{2.783646in}}%
\pgfpathlineto{\pgfqpoint{3.564986in}{2.768088in}}%
\pgfpathlineto{\pgfqpoint{3.564986in}{2.799204in}}%
\pgfpathlineto{\pgfqpoint{3.538039in}{2.814762in}}%
\pgfpathlineto{\pgfqpoint{3.538039in}{2.783646in}}%
\pgfpathclose%
\pgfusepath{fill}%
\end{pgfscope}%
\begin{pgfscope}%
\pgfpathrectangle{\pgfqpoint{0.765000in}{0.660000in}}{\pgfqpoint{4.620000in}{4.620000in}}%
\pgfusepath{clip}%
\pgfsetbuttcap%
\pgfsetroundjoin%
\definecolor{currentfill}{rgb}{1.000000,0.894118,0.788235}%
\pgfsetfillcolor{currentfill}%
\pgfsetlinewidth{0.000000pt}%
\definecolor{currentstroke}{rgb}{1.000000,0.894118,0.788235}%
\pgfsetstrokecolor{currentstroke}%
\pgfsetdash{}{0pt}%
\pgfpathmoveto{\pgfqpoint{3.693066in}{2.865416in}}%
\pgfpathlineto{\pgfqpoint{3.666119in}{2.849858in}}%
\pgfpathlineto{\pgfqpoint{3.693066in}{2.834300in}}%
\pgfpathlineto{\pgfqpoint{3.720013in}{2.849858in}}%
\pgfpathlineto{\pgfqpoint{3.693066in}{2.865416in}}%
\pgfpathclose%
\pgfusepath{fill}%
\end{pgfscope}%
\begin{pgfscope}%
\pgfpathrectangle{\pgfqpoint{0.765000in}{0.660000in}}{\pgfqpoint{4.620000in}{4.620000in}}%
\pgfusepath{clip}%
\pgfsetbuttcap%
\pgfsetroundjoin%
\definecolor{currentfill}{rgb}{1.000000,0.894118,0.788235}%
\pgfsetfillcolor{currentfill}%
\pgfsetlinewidth{0.000000pt}%
\definecolor{currentstroke}{rgb}{1.000000,0.894118,0.788235}%
\pgfsetstrokecolor{currentstroke}%
\pgfsetdash{}{0pt}%
\pgfpathmoveto{\pgfqpoint{3.693066in}{2.803184in}}%
\pgfpathlineto{\pgfqpoint{3.720013in}{2.818742in}}%
\pgfpathlineto{\pgfqpoint{3.720013in}{2.849858in}}%
\pgfpathlineto{\pgfqpoint{3.693066in}{2.834300in}}%
\pgfpathlineto{\pgfqpoint{3.693066in}{2.803184in}}%
\pgfpathclose%
\pgfusepath{fill}%
\end{pgfscope}%
\begin{pgfscope}%
\pgfpathrectangle{\pgfqpoint{0.765000in}{0.660000in}}{\pgfqpoint{4.620000in}{4.620000in}}%
\pgfusepath{clip}%
\pgfsetbuttcap%
\pgfsetroundjoin%
\definecolor{currentfill}{rgb}{1.000000,0.894118,0.788235}%
\pgfsetfillcolor{currentfill}%
\pgfsetlinewidth{0.000000pt}%
\definecolor{currentstroke}{rgb}{1.000000,0.894118,0.788235}%
\pgfsetstrokecolor{currentstroke}%
\pgfsetdash{}{0pt}%
\pgfpathmoveto{\pgfqpoint{3.666119in}{2.818742in}}%
\pgfpathlineto{\pgfqpoint{3.693066in}{2.803184in}}%
\pgfpathlineto{\pgfqpoint{3.693066in}{2.834300in}}%
\pgfpathlineto{\pgfqpoint{3.666119in}{2.849858in}}%
\pgfpathlineto{\pgfqpoint{3.666119in}{2.818742in}}%
\pgfpathclose%
\pgfusepath{fill}%
\end{pgfscope}%
\begin{pgfscope}%
\pgfpathrectangle{\pgfqpoint{0.765000in}{0.660000in}}{\pgfqpoint{4.620000in}{4.620000in}}%
\pgfusepath{clip}%
\pgfsetbuttcap%
\pgfsetroundjoin%
\definecolor{currentfill}{rgb}{1.000000,0.894118,0.788235}%
\pgfsetfillcolor{currentfill}%
\pgfsetlinewidth{0.000000pt}%
\definecolor{currentstroke}{rgb}{1.000000,0.894118,0.788235}%
\pgfsetstrokecolor{currentstroke}%
\pgfsetdash{}{0pt}%
\pgfpathmoveto{\pgfqpoint{3.538039in}{2.814762in}}%
\pgfpathlineto{\pgfqpoint{3.511092in}{2.799204in}}%
\pgfpathlineto{\pgfqpoint{3.538039in}{2.783646in}}%
\pgfpathlineto{\pgfqpoint{3.564986in}{2.799204in}}%
\pgfpathlineto{\pgfqpoint{3.538039in}{2.814762in}}%
\pgfpathclose%
\pgfusepath{fill}%
\end{pgfscope}%
\begin{pgfscope}%
\pgfpathrectangle{\pgfqpoint{0.765000in}{0.660000in}}{\pgfqpoint{4.620000in}{4.620000in}}%
\pgfusepath{clip}%
\pgfsetbuttcap%
\pgfsetroundjoin%
\definecolor{currentfill}{rgb}{1.000000,0.894118,0.788235}%
\pgfsetfillcolor{currentfill}%
\pgfsetlinewidth{0.000000pt}%
\definecolor{currentstroke}{rgb}{1.000000,0.894118,0.788235}%
\pgfsetstrokecolor{currentstroke}%
\pgfsetdash{}{0pt}%
\pgfpathmoveto{\pgfqpoint{3.538039in}{2.752530in}}%
\pgfpathlineto{\pgfqpoint{3.564986in}{2.768088in}}%
\pgfpathlineto{\pgfqpoint{3.564986in}{2.799204in}}%
\pgfpathlineto{\pgfqpoint{3.538039in}{2.783646in}}%
\pgfpathlineto{\pgfqpoint{3.538039in}{2.752530in}}%
\pgfpathclose%
\pgfusepath{fill}%
\end{pgfscope}%
\begin{pgfscope}%
\pgfpathrectangle{\pgfqpoint{0.765000in}{0.660000in}}{\pgfqpoint{4.620000in}{4.620000in}}%
\pgfusepath{clip}%
\pgfsetbuttcap%
\pgfsetroundjoin%
\definecolor{currentfill}{rgb}{1.000000,0.894118,0.788235}%
\pgfsetfillcolor{currentfill}%
\pgfsetlinewidth{0.000000pt}%
\definecolor{currentstroke}{rgb}{1.000000,0.894118,0.788235}%
\pgfsetstrokecolor{currentstroke}%
\pgfsetdash{}{0pt}%
\pgfpathmoveto{\pgfqpoint{3.511092in}{2.768088in}}%
\pgfpathlineto{\pgfqpoint{3.538039in}{2.752530in}}%
\pgfpathlineto{\pgfqpoint{3.538039in}{2.783646in}}%
\pgfpathlineto{\pgfqpoint{3.511092in}{2.799204in}}%
\pgfpathlineto{\pgfqpoint{3.511092in}{2.768088in}}%
\pgfpathclose%
\pgfusepath{fill}%
\end{pgfscope}%
\begin{pgfscope}%
\pgfpathrectangle{\pgfqpoint{0.765000in}{0.660000in}}{\pgfqpoint{4.620000in}{4.620000in}}%
\pgfusepath{clip}%
\pgfsetbuttcap%
\pgfsetroundjoin%
\definecolor{currentfill}{rgb}{1.000000,0.894118,0.788235}%
\pgfsetfillcolor{currentfill}%
\pgfsetlinewidth{0.000000pt}%
\definecolor{currentstroke}{rgb}{1.000000,0.894118,0.788235}%
\pgfsetstrokecolor{currentstroke}%
\pgfsetdash{}{0pt}%
\pgfpathmoveto{\pgfqpoint{3.693066in}{2.834300in}}%
\pgfpathlineto{\pgfqpoint{3.666119in}{2.818742in}}%
\pgfpathlineto{\pgfqpoint{3.511092in}{2.768088in}}%
\pgfpathlineto{\pgfqpoint{3.538039in}{2.783646in}}%
\pgfpathlineto{\pgfqpoint{3.693066in}{2.834300in}}%
\pgfpathclose%
\pgfusepath{fill}%
\end{pgfscope}%
\begin{pgfscope}%
\pgfpathrectangle{\pgfqpoint{0.765000in}{0.660000in}}{\pgfqpoint{4.620000in}{4.620000in}}%
\pgfusepath{clip}%
\pgfsetbuttcap%
\pgfsetroundjoin%
\definecolor{currentfill}{rgb}{1.000000,0.894118,0.788235}%
\pgfsetfillcolor{currentfill}%
\pgfsetlinewidth{0.000000pt}%
\definecolor{currentstroke}{rgb}{1.000000,0.894118,0.788235}%
\pgfsetstrokecolor{currentstroke}%
\pgfsetdash{}{0pt}%
\pgfpathmoveto{\pgfqpoint{3.720013in}{2.818742in}}%
\pgfpathlineto{\pgfqpoint{3.693066in}{2.834300in}}%
\pgfpathlineto{\pgfqpoint{3.538039in}{2.783646in}}%
\pgfpathlineto{\pgfqpoint{3.564986in}{2.768088in}}%
\pgfpathlineto{\pgfqpoint{3.720013in}{2.818742in}}%
\pgfpathclose%
\pgfusepath{fill}%
\end{pgfscope}%
\begin{pgfscope}%
\pgfpathrectangle{\pgfqpoint{0.765000in}{0.660000in}}{\pgfqpoint{4.620000in}{4.620000in}}%
\pgfusepath{clip}%
\pgfsetbuttcap%
\pgfsetroundjoin%
\definecolor{currentfill}{rgb}{1.000000,0.894118,0.788235}%
\pgfsetfillcolor{currentfill}%
\pgfsetlinewidth{0.000000pt}%
\definecolor{currentstroke}{rgb}{1.000000,0.894118,0.788235}%
\pgfsetstrokecolor{currentstroke}%
\pgfsetdash{}{0pt}%
\pgfpathmoveto{\pgfqpoint{3.693066in}{2.834300in}}%
\pgfpathlineto{\pgfqpoint{3.693066in}{2.865416in}}%
\pgfpathlineto{\pgfqpoint{3.538039in}{2.814762in}}%
\pgfpathlineto{\pgfqpoint{3.564986in}{2.768088in}}%
\pgfpathlineto{\pgfqpoint{3.693066in}{2.834300in}}%
\pgfpathclose%
\pgfusepath{fill}%
\end{pgfscope}%
\begin{pgfscope}%
\pgfpathrectangle{\pgfqpoint{0.765000in}{0.660000in}}{\pgfqpoint{4.620000in}{4.620000in}}%
\pgfusepath{clip}%
\pgfsetbuttcap%
\pgfsetroundjoin%
\definecolor{currentfill}{rgb}{1.000000,0.894118,0.788235}%
\pgfsetfillcolor{currentfill}%
\pgfsetlinewidth{0.000000pt}%
\definecolor{currentstroke}{rgb}{1.000000,0.894118,0.788235}%
\pgfsetstrokecolor{currentstroke}%
\pgfsetdash{}{0pt}%
\pgfpathmoveto{\pgfqpoint{3.720013in}{2.818742in}}%
\pgfpathlineto{\pgfqpoint{3.720013in}{2.849858in}}%
\pgfpathlineto{\pgfqpoint{3.564986in}{2.799204in}}%
\pgfpathlineto{\pgfqpoint{3.538039in}{2.783646in}}%
\pgfpathlineto{\pgfqpoint{3.720013in}{2.818742in}}%
\pgfpathclose%
\pgfusepath{fill}%
\end{pgfscope}%
\begin{pgfscope}%
\pgfpathrectangle{\pgfqpoint{0.765000in}{0.660000in}}{\pgfqpoint{4.620000in}{4.620000in}}%
\pgfusepath{clip}%
\pgfsetbuttcap%
\pgfsetroundjoin%
\definecolor{currentfill}{rgb}{1.000000,0.894118,0.788235}%
\pgfsetfillcolor{currentfill}%
\pgfsetlinewidth{0.000000pt}%
\definecolor{currentstroke}{rgb}{1.000000,0.894118,0.788235}%
\pgfsetstrokecolor{currentstroke}%
\pgfsetdash{}{0pt}%
\pgfpathmoveto{\pgfqpoint{3.666119in}{2.818742in}}%
\pgfpathlineto{\pgfqpoint{3.693066in}{2.803184in}}%
\pgfpathlineto{\pgfqpoint{3.538039in}{2.752530in}}%
\pgfpathlineto{\pgfqpoint{3.511092in}{2.768088in}}%
\pgfpathlineto{\pgfqpoint{3.666119in}{2.818742in}}%
\pgfpathclose%
\pgfusepath{fill}%
\end{pgfscope}%
\begin{pgfscope}%
\pgfpathrectangle{\pgfqpoint{0.765000in}{0.660000in}}{\pgfqpoint{4.620000in}{4.620000in}}%
\pgfusepath{clip}%
\pgfsetbuttcap%
\pgfsetroundjoin%
\definecolor{currentfill}{rgb}{1.000000,0.894118,0.788235}%
\pgfsetfillcolor{currentfill}%
\pgfsetlinewidth{0.000000pt}%
\definecolor{currentstroke}{rgb}{1.000000,0.894118,0.788235}%
\pgfsetstrokecolor{currentstroke}%
\pgfsetdash{}{0pt}%
\pgfpathmoveto{\pgfqpoint{3.693066in}{2.803184in}}%
\pgfpathlineto{\pgfqpoint{3.720013in}{2.818742in}}%
\pgfpathlineto{\pgfqpoint{3.564986in}{2.768088in}}%
\pgfpathlineto{\pgfqpoint{3.538039in}{2.752530in}}%
\pgfpathlineto{\pgfqpoint{3.693066in}{2.803184in}}%
\pgfpathclose%
\pgfusepath{fill}%
\end{pgfscope}%
\begin{pgfscope}%
\pgfpathrectangle{\pgfqpoint{0.765000in}{0.660000in}}{\pgfqpoint{4.620000in}{4.620000in}}%
\pgfusepath{clip}%
\pgfsetbuttcap%
\pgfsetroundjoin%
\definecolor{currentfill}{rgb}{1.000000,0.894118,0.788235}%
\pgfsetfillcolor{currentfill}%
\pgfsetlinewidth{0.000000pt}%
\definecolor{currentstroke}{rgb}{1.000000,0.894118,0.788235}%
\pgfsetstrokecolor{currentstroke}%
\pgfsetdash{}{0pt}%
\pgfpathmoveto{\pgfqpoint{3.693066in}{2.865416in}}%
\pgfpathlineto{\pgfqpoint{3.666119in}{2.849858in}}%
\pgfpathlineto{\pgfqpoint{3.511092in}{2.799204in}}%
\pgfpathlineto{\pgfqpoint{3.538039in}{2.814762in}}%
\pgfpathlineto{\pgfqpoint{3.693066in}{2.865416in}}%
\pgfpathclose%
\pgfusepath{fill}%
\end{pgfscope}%
\begin{pgfscope}%
\pgfpathrectangle{\pgfqpoint{0.765000in}{0.660000in}}{\pgfqpoint{4.620000in}{4.620000in}}%
\pgfusepath{clip}%
\pgfsetbuttcap%
\pgfsetroundjoin%
\definecolor{currentfill}{rgb}{1.000000,0.894118,0.788235}%
\pgfsetfillcolor{currentfill}%
\pgfsetlinewidth{0.000000pt}%
\definecolor{currentstroke}{rgb}{1.000000,0.894118,0.788235}%
\pgfsetstrokecolor{currentstroke}%
\pgfsetdash{}{0pt}%
\pgfpathmoveto{\pgfqpoint{3.720013in}{2.849858in}}%
\pgfpathlineto{\pgfqpoint{3.693066in}{2.865416in}}%
\pgfpathlineto{\pgfqpoint{3.538039in}{2.814762in}}%
\pgfpathlineto{\pgfqpoint{3.564986in}{2.799204in}}%
\pgfpathlineto{\pgfqpoint{3.720013in}{2.849858in}}%
\pgfpathclose%
\pgfusepath{fill}%
\end{pgfscope}%
\begin{pgfscope}%
\pgfpathrectangle{\pgfqpoint{0.765000in}{0.660000in}}{\pgfqpoint{4.620000in}{4.620000in}}%
\pgfusepath{clip}%
\pgfsetbuttcap%
\pgfsetroundjoin%
\definecolor{currentfill}{rgb}{1.000000,0.894118,0.788235}%
\pgfsetfillcolor{currentfill}%
\pgfsetlinewidth{0.000000pt}%
\definecolor{currentstroke}{rgb}{1.000000,0.894118,0.788235}%
\pgfsetstrokecolor{currentstroke}%
\pgfsetdash{}{0pt}%
\pgfpathmoveto{\pgfqpoint{3.666119in}{2.818742in}}%
\pgfpathlineto{\pgfqpoint{3.666119in}{2.849858in}}%
\pgfpathlineto{\pgfqpoint{3.511092in}{2.799204in}}%
\pgfpathlineto{\pgfqpoint{3.538039in}{2.752530in}}%
\pgfpathlineto{\pgfqpoint{3.666119in}{2.818742in}}%
\pgfpathclose%
\pgfusepath{fill}%
\end{pgfscope}%
\begin{pgfscope}%
\pgfpathrectangle{\pgfqpoint{0.765000in}{0.660000in}}{\pgfqpoint{4.620000in}{4.620000in}}%
\pgfusepath{clip}%
\pgfsetbuttcap%
\pgfsetroundjoin%
\definecolor{currentfill}{rgb}{1.000000,0.894118,0.788235}%
\pgfsetfillcolor{currentfill}%
\pgfsetlinewidth{0.000000pt}%
\definecolor{currentstroke}{rgb}{1.000000,0.894118,0.788235}%
\pgfsetstrokecolor{currentstroke}%
\pgfsetdash{}{0pt}%
\pgfpathmoveto{\pgfqpoint{3.693066in}{2.803184in}}%
\pgfpathlineto{\pgfqpoint{3.693066in}{2.834300in}}%
\pgfpathlineto{\pgfqpoint{3.538039in}{2.783646in}}%
\pgfpathlineto{\pgfqpoint{3.511092in}{2.768088in}}%
\pgfpathlineto{\pgfqpoint{3.693066in}{2.803184in}}%
\pgfpathclose%
\pgfusepath{fill}%
\end{pgfscope}%
\begin{pgfscope}%
\pgfpathrectangle{\pgfqpoint{0.765000in}{0.660000in}}{\pgfqpoint{4.620000in}{4.620000in}}%
\pgfusepath{clip}%
\pgfsetbuttcap%
\pgfsetroundjoin%
\definecolor{currentfill}{rgb}{1.000000,0.894118,0.788235}%
\pgfsetfillcolor{currentfill}%
\pgfsetlinewidth{0.000000pt}%
\definecolor{currentstroke}{rgb}{1.000000,0.894118,0.788235}%
\pgfsetstrokecolor{currentstroke}%
\pgfsetdash{}{0pt}%
\pgfpathmoveto{\pgfqpoint{3.693066in}{2.834300in}}%
\pgfpathlineto{\pgfqpoint{3.720013in}{2.849858in}}%
\pgfpathlineto{\pgfqpoint{3.564986in}{2.799204in}}%
\pgfpathlineto{\pgfqpoint{3.538039in}{2.783646in}}%
\pgfpathlineto{\pgfqpoint{3.693066in}{2.834300in}}%
\pgfpathclose%
\pgfusepath{fill}%
\end{pgfscope}%
\begin{pgfscope}%
\pgfpathrectangle{\pgfqpoint{0.765000in}{0.660000in}}{\pgfqpoint{4.620000in}{4.620000in}}%
\pgfusepath{clip}%
\pgfsetbuttcap%
\pgfsetroundjoin%
\definecolor{currentfill}{rgb}{1.000000,0.894118,0.788235}%
\pgfsetfillcolor{currentfill}%
\pgfsetlinewidth{0.000000pt}%
\definecolor{currentstroke}{rgb}{1.000000,0.894118,0.788235}%
\pgfsetstrokecolor{currentstroke}%
\pgfsetdash{}{0pt}%
\pgfpathmoveto{\pgfqpoint{3.666119in}{2.849858in}}%
\pgfpathlineto{\pgfqpoint{3.693066in}{2.834300in}}%
\pgfpathlineto{\pgfqpoint{3.538039in}{2.783646in}}%
\pgfpathlineto{\pgfqpoint{3.511092in}{2.799204in}}%
\pgfpathlineto{\pgfqpoint{3.666119in}{2.849858in}}%
\pgfpathclose%
\pgfusepath{fill}%
\end{pgfscope}%
\begin{pgfscope}%
\pgfpathrectangle{\pgfqpoint{0.765000in}{0.660000in}}{\pgfqpoint{4.620000in}{4.620000in}}%
\pgfusepath{clip}%
\pgfsetbuttcap%
\pgfsetroundjoin%
\definecolor{currentfill}{rgb}{1.000000,0.894118,0.788235}%
\pgfsetfillcolor{currentfill}%
\pgfsetlinewidth{0.000000pt}%
\definecolor{currentstroke}{rgb}{1.000000,0.894118,0.788235}%
\pgfsetstrokecolor{currentstroke}%
\pgfsetdash{}{0pt}%
\pgfpathmoveto{\pgfqpoint{3.176663in}{3.599980in}}%
\pgfpathlineto{\pgfqpoint{3.149716in}{3.584422in}}%
\pgfpathlineto{\pgfqpoint{3.176663in}{3.568864in}}%
\pgfpathlineto{\pgfqpoint{3.203610in}{3.584422in}}%
\pgfpathlineto{\pgfqpoint{3.176663in}{3.599980in}}%
\pgfpathclose%
\pgfusepath{fill}%
\end{pgfscope}%
\begin{pgfscope}%
\pgfpathrectangle{\pgfqpoint{0.765000in}{0.660000in}}{\pgfqpoint{4.620000in}{4.620000in}}%
\pgfusepath{clip}%
\pgfsetbuttcap%
\pgfsetroundjoin%
\definecolor{currentfill}{rgb}{1.000000,0.894118,0.788235}%
\pgfsetfillcolor{currentfill}%
\pgfsetlinewidth{0.000000pt}%
\definecolor{currentstroke}{rgb}{1.000000,0.894118,0.788235}%
\pgfsetstrokecolor{currentstroke}%
\pgfsetdash{}{0pt}%
\pgfpathmoveto{\pgfqpoint{3.176663in}{3.599980in}}%
\pgfpathlineto{\pgfqpoint{3.149716in}{3.584422in}}%
\pgfpathlineto{\pgfqpoint{3.149716in}{3.615538in}}%
\pgfpathlineto{\pgfqpoint{3.176663in}{3.631096in}}%
\pgfpathlineto{\pgfqpoint{3.176663in}{3.599980in}}%
\pgfpathclose%
\pgfusepath{fill}%
\end{pgfscope}%
\begin{pgfscope}%
\pgfpathrectangle{\pgfqpoint{0.765000in}{0.660000in}}{\pgfqpoint{4.620000in}{4.620000in}}%
\pgfusepath{clip}%
\pgfsetbuttcap%
\pgfsetroundjoin%
\definecolor{currentfill}{rgb}{1.000000,0.894118,0.788235}%
\pgfsetfillcolor{currentfill}%
\pgfsetlinewidth{0.000000pt}%
\definecolor{currentstroke}{rgb}{1.000000,0.894118,0.788235}%
\pgfsetstrokecolor{currentstroke}%
\pgfsetdash{}{0pt}%
\pgfpathmoveto{\pgfqpoint{3.176663in}{3.599980in}}%
\pgfpathlineto{\pgfqpoint{3.203610in}{3.584422in}}%
\pgfpathlineto{\pgfqpoint{3.203610in}{3.615538in}}%
\pgfpathlineto{\pgfqpoint{3.176663in}{3.631096in}}%
\pgfpathlineto{\pgfqpoint{3.176663in}{3.599980in}}%
\pgfpathclose%
\pgfusepath{fill}%
\end{pgfscope}%
\begin{pgfscope}%
\pgfpathrectangle{\pgfqpoint{0.765000in}{0.660000in}}{\pgfqpoint{4.620000in}{4.620000in}}%
\pgfusepath{clip}%
\pgfsetbuttcap%
\pgfsetroundjoin%
\definecolor{currentfill}{rgb}{1.000000,0.894118,0.788235}%
\pgfsetfillcolor{currentfill}%
\pgfsetlinewidth{0.000000pt}%
\definecolor{currentstroke}{rgb}{1.000000,0.894118,0.788235}%
\pgfsetstrokecolor{currentstroke}%
\pgfsetdash{}{0pt}%
\pgfpathmoveto{\pgfqpoint{3.006193in}{3.769934in}}%
\pgfpathlineto{\pgfqpoint{2.979245in}{3.754376in}}%
\pgfpathlineto{\pgfqpoint{3.006193in}{3.738818in}}%
\pgfpathlineto{\pgfqpoint{3.033140in}{3.754376in}}%
\pgfpathlineto{\pgfqpoint{3.006193in}{3.769934in}}%
\pgfpathclose%
\pgfusepath{fill}%
\end{pgfscope}%
\begin{pgfscope}%
\pgfpathrectangle{\pgfqpoint{0.765000in}{0.660000in}}{\pgfqpoint{4.620000in}{4.620000in}}%
\pgfusepath{clip}%
\pgfsetbuttcap%
\pgfsetroundjoin%
\definecolor{currentfill}{rgb}{1.000000,0.894118,0.788235}%
\pgfsetfillcolor{currentfill}%
\pgfsetlinewidth{0.000000pt}%
\definecolor{currentstroke}{rgb}{1.000000,0.894118,0.788235}%
\pgfsetstrokecolor{currentstroke}%
\pgfsetdash{}{0pt}%
\pgfpathmoveto{\pgfqpoint{3.006193in}{3.769934in}}%
\pgfpathlineto{\pgfqpoint{2.979245in}{3.754376in}}%
\pgfpathlineto{\pgfqpoint{2.979245in}{3.785492in}}%
\pgfpathlineto{\pgfqpoint{3.006193in}{3.801050in}}%
\pgfpathlineto{\pgfqpoint{3.006193in}{3.769934in}}%
\pgfpathclose%
\pgfusepath{fill}%
\end{pgfscope}%
\begin{pgfscope}%
\pgfpathrectangle{\pgfqpoint{0.765000in}{0.660000in}}{\pgfqpoint{4.620000in}{4.620000in}}%
\pgfusepath{clip}%
\pgfsetbuttcap%
\pgfsetroundjoin%
\definecolor{currentfill}{rgb}{1.000000,0.894118,0.788235}%
\pgfsetfillcolor{currentfill}%
\pgfsetlinewidth{0.000000pt}%
\definecolor{currentstroke}{rgb}{1.000000,0.894118,0.788235}%
\pgfsetstrokecolor{currentstroke}%
\pgfsetdash{}{0pt}%
\pgfpathmoveto{\pgfqpoint{3.006193in}{3.769934in}}%
\pgfpathlineto{\pgfqpoint{3.033140in}{3.754376in}}%
\pgfpathlineto{\pgfqpoint{3.033140in}{3.785492in}}%
\pgfpathlineto{\pgfqpoint{3.006193in}{3.801050in}}%
\pgfpathlineto{\pgfqpoint{3.006193in}{3.769934in}}%
\pgfpathclose%
\pgfusepath{fill}%
\end{pgfscope}%
\begin{pgfscope}%
\pgfpathrectangle{\pgfqpoint{0.765000in}{0.660000in}}{\pgfqpoint{4.620000in}{4.620000in}}%
\pgfusepath{clip}%
\pgfsetbuttcap%
\pgfsetroundjoin%
\definecolor{currentfill}{rgb}{1.000000,0.894118,0.788235}%
\pgfsetfillcolor{currentfill}%
\pgfsetlinewidth{0.000000pt}%
\definecolor{currentstroke}{rgb}{1.000000,0.894118,0.788235}%
\pgfsetstrokecolor{currentstroke}%
\pgfsetdash{}{0pt}%
\pgfpathmoveto{\pgfqpoint{3.176663in}{3.631096in}}%
\pgfpathlineto{\pgfqpoint{3.149716in}{3.615538in}}%
\pgfpathlineto{\pgfqpoint{3.176663in}{3.599980in}}%
\pgfpathlineto{\pgfqpoint{3.203610in}{3.615538in}}%
\pgfpathlineto{\pgfqpoint{3.176663in}{3.631096in}}%
\pgfpathclose%
\pgfusepath{fill}%
\end{pgfscope}%
\begin{pgfscope}%
\pgfpathrectangle{\pgfqpoint{0.765000in}{0.660000in}}{\pgfqpoint{4.620000in}{4.620000in}}%
\pgfusepath{clip}%
\pgfsetbuttcap%
\pgfsetroundjoin%
\definecolor{currentfill}{rgb}{1.000000,0.894118,0.788235}%
\pgfsetfillcolor{currentfill}%
\pgfsetlinewidth{0.000000pt}%
\definecolor{currentstroke}{rgb}{1.000000,0.894118,0.788235}%
\pgfsetstrokecolor{currentstroke}%
\pgfsetdash{}{0pt}%
\pgfpathmoveto{\pgfqpoint{3.176663in}{3.568864in}}%
\pgfpathlineto{\pgfqpoint{3.203610in}{3.584422in}}%
\pgfpathlineto{\pgfqpoint{3.203610in}{3.615538in}}%
\pgfpathlineto{\pgfqpoint{3.176663in}{3.599980in}}%
\pgfpathlineto{\pgfqpoint{3.176663in}{3.568864in}}%
\pgfpathclose%
\pgfusepath{fill}%
\end{pgfscope}%
\begin{pgfscope}%
\pgfpathrectangle{\pgfqpoint{0.765000in}{0.660000in}}{\pgfqpoint{4.620000in}{4.620000in}}%
\pgfusepath{clip}%
\pgfsetbuttcap%
\pgfsetroundjoin%
\definecolor{currentfill}{rgb}{1.000000,0.894118,0.788235}%
\pgfsetfillcolor{currentfill}%
\pgfsetlinewidth{0.000000pt}%
\definecolor{currentstroke}{rgb}{1.000000,0.894118,0.788235}%
\pgfsetstrokecolor{currentstroke}%
\pgfsetdash{}{0pt}%
\pgfpathmoveto{\pgfqpoint{3.149716in}{3.584422in}}%
\pgfpathlineto{\pgfqpoint{3.176663in}{3.568864in}}%
\pgfpathlineto{\pgfqpoint{3.176663in}{3.599980in}}%
\pgfpathlineto{\pgfqpoint{3.149716in}{3.615538in}}%
\pgfpathlineto{\pgfqpoint{3.149716in}{3.584422in}}%
\pgfpathclose%
\pgfusepath{fill}%
\end{pgfscope}%
\begin{pgfscope}%
\pgfpathrectangle{\pgfqpoint{0.765000in}{0.660000in}}{\pgfqpoint{4.620000in}{4.620000in}}%
\pgfusepath{clip}%
\pgfsetbuttcap%
\pgfsetroundjoin%
\definecolor{currentfill}{rgb}{1.000000,0.894118,0.788235}%
\pgfsetfillcolor{currentfill}%
\pgfsetlinewidth{0.000000pt}%
\definecolor{currentstroke}{rgb}{1.000000,0.894118,0.788235}%
\pgfsetstrokecolor{currentstroke}%
\pgfsetdash{}{0pt}%
\pgfpathmoveto{\pgfqpoint{3.006193in}{3.801050in}}%
\pgfpathlineto{\pgfqpoint{2.979245in}{3.785492in}}%
\pgfpathlineto{\pgfqpoint{3.006193in}{3.769934in}}%
\pgfpathlineto{\pgfqpoint{3.033140in}{3.785492in}}%
\pgfpathlineto{\pgfqpoint{3.006193in}{3.801050in}}%
\pgfpathclose%
\pgfusepath{fill}%
\end{pgfscope}%
\begin{pgfscope}%
\pgfpathrectangle{\pgfqpoint{0.765000in}{0.660000in}}{\pgfqpoint{4.620000in}{4.620000in}}%
\pgfusepath{clip}%
\pgfsetbuttcap%
\pgfsetroundjoin%
\definecolor{currentfill}{rgb}{1.000000,0.894118,0.788235}%
\pgfsetfillcolor{currentfill}%
\pgfsetlinewidth{0.000000pt}%
\definecolor{currentstroke}{rgb}{1.000000,0.894118,0.788235}%
\pgfsetstrokecolor{currentstroke}%
\pgfsetdash{}{0pt}%
\pgfpathmoveto{\pgfqpoint{3.006193in}{3.738818in}}%
\pgfpathlineto{\pgfqpoint{3.033140in}{3.754376in}}%
\pgfpathlineto{\pgfqpoint{3.033140in}{3.785492in}}%
\pgfpathlineto{\pgfqpoint{3.006193in}{3.769934in}}%
\pgfpathlineto{\pgfqpoint{3.006193in}{3.738818in}}%
\pgfpathclose%
\pgfusepath{fill}%
\end{pgfscope}%
\begin{pgfscope}%
\pgfpathrectangle{\pgfqpoint{0.765000in}{0.660000in}}{\pgfqpoint{4.620000in}{4.620000in}}%
\pgfusepath{clip}%
\pgfsetbuttcap%
\pgfsetroundjoin%
\definecolor{currentfill}{rgb}{1.000000,0.894118,0.788235}%
\pgfsetfillcolor{currentfill}%
\pgfsetlinewidth{0.000000pt}%
\definecolor{currentstroke}{rgb}{1.000000,0.894118,0.788235}%
\pgfsetstrokecolor{currentstroke}%
\pgfsetdash{}{0pt}%
\pgfpathmoveto{\pgfqpoint{2.979245in}{3.754376in}}%
\pgfpathlineto{\pgfqpoint{3.006193in}{3.738818in}}%
\pgfpathlineto{\pgfqpoint{3.006193in}{3.769934in}}%
\pgfpathlineto{\pgfqpoint{2.979245in}{3.785492in}}%
\pgfpathlineto{\pgfqpoint{2.979245in}{3.754376in}}%
\pgfpathclose%
\pgfusepath{fill}%
\end{pgfscope}%
\begin{pgfscope}%
\pgfpathrectangle{\pgfqpoint{0.765000in}{0.660000in}}{\pgfqpoint{4.620000in}{4.620000in}}%
\pgfusepath{clip}%
\pgfsetbuttcap%
\pgfsetroundjoin%
\definecolor{currentfill}{rgb}{1.000000,0.894118,0.788235}%
\pgfsetfillcolor{currentfill}%
\pgfsetlinewidth{0.000000pt}%
\definecolor{currentstroke}{rgb}{1.000000,0.894118,0.788235}%
\pgfsetstrokecolor{currentstroke}%
\pgfsetdash{}{0pt}%
\pgfpathmoveto{\pgfqpoint{3.176663in}{3.599980in}}%
\pgfpathlineto{\pgfqpoint{3.149716in}{3.584422in}}%
\pgfpathlineto{\pgfqpoint{2.979245in}{3.754376in}}%
\pgfpathlineto{\pgfqpoint{3.006193in}{3.769934in}}%
\pgfpathlineto{\pgfqpoint{3.176663in}{3.599980in}}%
\pgfpathclose%
\pgfusepath{fill}%
\end{pgfscope}%
\begin{pgfscope}%
\pgfpathrectangle{\pgfqpoint{0.765000in}{0.660000in}}{\pgfqpoint{4.620000in}{4.620000in}}%
\pgfusepath{clip}%
\pgfsetbuttcap%
\pgfsetroundjoin%
\definecolor{currentfill}{rgb}{1.000000,0.894118,0.788235}%
\pgfsetfillcolor{currentfill}%
\pgfsetlinewidth{0.000000pt}%
\definecolor{currentstroke}{rgb}{1.000000,0.894118,0.788235}%
\pgfsetstrokecolor{currentstroke}%
\pgfsetdash{}{0pt}%
\pgfpathmoveto{\pgfqpoint{3.203610in}{3.584422in}}%
\pgfpathlineto{\pgfqpoint{3.176663in}{3.599980in}}%
\pgfpathlineto{\pgfqpoint{3.006193in}{3.769934in}}%
\pgfpathlineto{\pgfqpoint{3.033140in}{3.754376in}}%
\pgfpathlineto{\pgfqpoint{3.203610in}{3.584422in}}%
\pgfpathclose%
\pgfusepath{fill}%
\end{pgfscope}%
\begin{pgfscope}%
\pgfpathrectangle{\pgfqpoint{0.765000in}{0.660000in}}{\pgfqpoint{4.620000in}{4.620000in}}%
\pgfusepath{clip}%
\pgfsetbuttcap%
\pgfsetroundjoin%
\definecolor{currentfill}{rgb}{1.000000,0.894118,0.788235}%
\pgfsetfillcolor{currentfill}%
\pgfsetlinewidth{0.000000pt}%
\definecolor{currentstroke}{rgb}{1.000000,0.894118,0.788235}%
\pgfsetstrokecolor{currentstroke}%
\pgfsetdash{}{0pt}%
\pgfpathmoveto{\pgfqpoint{3.176663in}{3.599980in}}%
\pgfpathlineto{\pgfqpoint{3.176663in}{3.631096in}}%
\pgfpathlineto{\pgfqpoint{3.006193in}{3.801050in}}%
\pgfpathlineto{\pgfqpoint{3.033140in}{3.754376in}}%
\pgfpathlineto{\pgfqpoint{3.176663in}{3.599980in}}%
\pgfpathclose%
\pgfusepath{fill}%
\end{pgfscope}%
\begin{pgfscope}%
\pgfpathrectangle{\pgfqpoint{0.765000in}{0.660000in}}{\pgfqpoint{4.620000in}{4.620000in}}%
\pgfusepath{clip}%
\pgfsetbuttcap%
\pgfsetroundjoin%
\definecolor{currentfill}{rgb}{1.000000,0.894118,0.788235}%
\pgfsetfillcolor{currentfill}%
\pgfsetlinewidth{0.000000pt}%
\definecolor{currentstroke}{rgb}{1.000000,0.894118,0.788235}%
\pgfsetstrokecolor{currentstroke}%
\pgfsetdash{}{0pt}%
\pgfpathmoveto{\pgfqpoint{3.203610in}{3.584422in}}%
\pgfpathlineto{\pgfqpoint{3.203610in}{3.615538in}}%
\pgfpathlineto{\pgfqpoint{3.033140in}{3.785492in}}%
\pgfpathlineto{\pgfqpoint{3.006193in}{3.769934in}}%
\pgfpathlineto{\pgfqpoint{3.203610in}{3.584422in}}%
\pgfpathclose%
\pgfusepath{fill}%
\end{pgfscope}%
\begin{pgfscope}%
\pgfpathrectangle{\pgfqpoint{0.765000in}{0.660000in}}{\pgfqpoint{4.620000in}{4.620000in}}%
\pgfusepath{clip}%
\pgfsetbuttcap%
\pgfsetroundjoin%
\definecolor{currentfill}{rgb}{1.000000,0.894118,0.788235}%
\pgfsetfillcolor{currentfill}%
\pgfsetlinewidth{0.000000pt}%
\definecolor{currentstroke}{rgb}{1.000000,0.894118,0.788235}%
\pgfsetstrokecolor{currentstroke}%
\pgfsetdash{}{0pt}%
\pgfpathmoveto{\pgfqpoint{3.149716in}{3.584422in}}%
\pgfpathlineto{\pgfqpoint{3.176663in}{3.568864in}}%
\pgfpathlineto{\pgfqpoint{3.006193in}{3.738818in}}%
\pgfpathlineto{\pgfqpoint{2.979245in}{3.754376in}}%
\pgfpathlineto{\pgfqpoint{3.149716in}{3.584422in}}%
\pgfpathclose%
\pgfusepath{fill}%
\end{pgfscope}%
\begin{pgfscope}%
\pgfpathrectangle{\pgfqpoint{0.765000in}{0.660000in}}{\pgfqpoint{4.620000in}{4.620000in}}%
\pgfusepath{clip}%
\pgfsetbuttcap%
\pgfsetroundjoin%
\definecolor{currentfill}{rgb}{1.000000,0.894118,0.788235}%
\pgfsetfillcolor{currentfill}%
\pgfsetlinewidth{0.000000pt}%
\definecolor{currentstroke}{rgb}{1.000000,0.894118,0.788235}%
\pgfsetstrokecolor{currentstroke}%
\pgfsetdash{}{0pt}%
\pgfpathmoveto{\pgfqpoint{3.176663in}{3.568864in}}%
\pgfpathlineto{\pgfqpoint{3.203610in}{3.584422in}}%
\pgfpathlineto{\pgfqpoint{3.033140in}{3.754376in}}%
\pgfpathlineto{\pgfqpoint{3.006193in}{3.738818in}}%
\pgfpathlineto{\pgfqpoint{3.176663in}{3.568864in}}%
\pgfpathclose%
\pgfusepath{fill}%
\end{pgfscope}%
\begin{pgfscope}%
\pgfpathrectangle{\pgfqpoint{0.765000in}{0.660000in}}{\pgfqpoint{4.620000in}{4.620000in}}%
\pgfusepath{clip}%
\pgfsetbuttcap%
\pgfsetroundjoin%
\definecolor{currentfill}{rgb}{1.000000,0.894118,0.788235}%
\pgfsetfillcolor{currentfill}%
\pgfsetlinewidth{0.000000pt}%
\definecolor{currentstroke}{rgb}{1.000000,0.894118,0.788235}%
\pgfsetstrokecolor{currentstroke}%
\pgfsetdash{}{0pt}%
\pgfpathmoveto{\pgfqpoint{3.176663in}{3.631096in}}%
\pgfpathlineto{\pgfqpoint{3.149716in}{3.615538in}}%
\pgfpathlineto{\pgfqpoint{2.979245in}{3.785492in}}%
\pgfpathlineto{\pgfqpoint{3.006193in}{3.801050in}}%
\pgfpathlineto{\pgfqpoint{3.176663in}{3.631096in}}%
\pgfpathclose%
\pgfusepath{fill}%
\end{pgfscope}%
\begin{pgfscope}%
\pgfpathrectangle{\pgfqpoint{0.765000in}{0.660000in}}{\pgfqpoint{4.620000in}{4.620000in}}%
\pgfusepath{clip}%
\pgfsetbuttcap%
\pgfsetroundjoin%
\definecolor{currentfill}{rgb}{1.000000,0.894118,0.788235}%
\pgfsetfillcolor{currentfill}%
\pgfsetlinewidth{0.000000pt}%
\definecolor{currentstroke}{rgb}{1.000000,0.894118,0.788235}%
\pgfsetstrokecolor{currentstroke}%
\pgfsetdash{}{0pt}%
\pgfpathmoveto{\pgfqpoint{3.203610in}{3.615538in}}%
\pgfpathlineto{\pgfqpoint{3.176663in}{3.631096in}}%
\pgfpathlineto{\pgfqpoint{3.006193in}{3.801050in}}%
\pgfpathlineto{\pgfqpoint{3.033140in}{3.785492in}}%
\pgfpathlineto{\pgfqpoint{3.203610in}{3.615538in}}%
\pgfpathclose%
\pgfusepath{fill}%
\end{pgfscope}%
\begin{pgfscope}%
\pgfpathrectangle{\pgfqpoint{0.765000in}{0.660000in}}{\pgfqpoint{4.620000in}{4.620000in}}%
\pgfusepath{clip}%
\pgfsetbuttcap%
\pgfsetroundjoin%
\definecolor{currentfill}{rgb}{1.000000,0.894118,0.788235}%
\pgfsetfillcolor{currentfill}%
\pgfsetlinewidth{0.000000pt}%
\definecolor{currentstroke}{rgb}{1.000000,0.894118,0.788235}%
\pgfsetstrokecolor{currentstroke}%
\pgfsetdash{}{0pt}%
\pgfpathmoveto{\pgfqpoint{3.149716in}{3.584422in}}%
\pgfpathlineto{\pgfqpoint{3.149716in}{3.615538in}}%
\pgfpathlineto{\pgfqpoint{2.979245in}{3.785492in}}%
\pgfpathlineto{\pgfqpoint{3.006193in}{3.738818in}}%
\pgfpathlineto{\pgfqpoint{3.149716in}{3.584422in}}%
\pgfpathclose%
\pgfusepath{fill}%
\end{pgfscope}%
\begin{pgfscope}%
\pgfpathrectangle{\pgfqpoint{0.765000in}{0.660000in}}{\pgfqpoint{4.620000in}{4.620000in}}%
\pgfusepath{clip}%
\pgfsetbuttcap%
\pgfsetroundjoin%
\definecolor{currentfill}{rgb}{1.000000,0.894118,0.788235}%
\pgfsetfillcolor{currentfill}%
\pgfsetlinewidth{0.000000pt}%
\definecolor{currentstroke}{rgb}{1.000000,0.894118,0.788235}%
\pgfsetstrokecolor{currentstroke}%
\pgfsetdash{}{0pt}%
\pgfpathmoveto{\pgfqpoint{3.176663in}{3.568864in}}%
\pgfpathlineto{\pgfqpoint{3.176663in}{3.599980in}}%
\pgfpathlineto{\pgfqpoint{3.006193in}{3.769934in}}%
\pgfpathlineto{\pgfqpoint{2.979245in}{3.754376in}}%
\pgfpathlineto{\pgfqpoint{3.176663in}{3.568864in}}%
\pgfpathclose%
\pgfusepath{fill}%
\end{pgfscope}%
\begin{pgfscope}%
\pgfpathrectangle{\pgfqpoint{0.765000in}{0.660000in}}{\pgfqpoint{4.620000in}{4.620000in}}%
\pgfusepath{clip}%
\pgfsetbuttcap%
\pgfsetroundjoin%
\definecolor{currentfill}{rgb}{1.000000,0.894118,0.788235}%
\pgfsetfillcolor{currentfill}%
\pgfsetlinewidth{0.000000pt}%
\definecolor{currentstroke}{rgb}{1.000000,0.894118,0.788235}%
\pgfsetstrokecolor{currentstroke}%
\pgfsetdash{}{0pt}%
\pgfpathmoveto{\pgfqpoint{3.176663in}{3.599980in}}%
\pgfpathlineto{\pgfqpoint{3.203610in}{3.615538in}}%
\pgfpathlineto{\pgfqpoint{3.033140in}{3.785492in}}%
\pgfpathlineto{\pgfqpoint{3.006193in}{3.769934in}}%
\pgfpathlineto{\pgfqpoint{3.176663in}{3.599980in}}%
\pgfpathclose%
\pgfusepath{fill}%
\end{pgfscope}%
\begin{pgfscope}%
\pgfpathrectangle{\pgfqpoint{0.765000in}{0.660000in}}{\pgfqpoint{4.620000in}{4.620000in}}%
\pgfusepath{clip}%
\pgfsetbuttcap%
\pgfsetroundjoin%
\definecolor{currentfill}{rgb}{1.000000,0.894118,0.788235}%
\pgfsetfillcolor{currentfill}%
\pgfsetlinewidth{0.000000pt}%
\definecolor{currentstroke}{rgb}{1.000000,0.894118,0.788235}%
\pgfsetstrokecolor{currentstroke}%
\pgfsetdash{}{0pt}%
\pgfpathmoveto{\pgfqpoint{3.149716in}{3.615538in}}%
\pgfpathlineto{\pgfqpoint{3.176663in}{3.599980in}}%
\pgfpathlineto{\pgfqpoint{3.006193in}{3.769934in}}%
\pgfpathlineto{\pgfqpoint{2.979245in}{3.785492in}}%
\pgfpathlineto{\pgfqpoint{3.149716in}{3.615538in}}%
\pgfpathclose%
\pgfusepath{fill}%
\end{pgfscope}%
\begin{pgfscope}%
\pgfpathrectangle{\pgfqpoint{0.765000in}{0.660000in}}{\pgfqpoint{4.620000in}{4.620000in}}%
\pgfusepath{clip}%
\pgfsetbuttcap%
\pgfsetroundjoin%
\definecolor{currentfill}{rgb}{1.000000,0.894118,0.788235}%
\pgfsetfillcolor{currentfill}%
\pgfsetlinewidth{0.000000pt}%
\definecolor{currentstroke}{rgb}{1.000000,0.894118,0.788235}%
\pgfsetstrokecolor{currentstroke}%
\pgfsetdash{}{0pt}%
\pgfpathmoveto{\pgfqpoint{3.006193in}{3.769934in}}%
\pgfpathlineto{\pgfqpoint{2.979245in}{3.754376in}}%
\pgfpathlineto{\pgfqpoint{3.006193in}{3.738818in}}%
\pgfpathlineto{\pgfqpoint{3.033140in}{3.754376in}}%
\pgfpathlineto{\pgfqpoint{3.006193in}{3.769934in}}%
\pgfpathclose%
\pgfusepath{fill}%
\end{pgfscope}%
\begin{pgfscope}%
\pgfpathrectangle{\pgfqpoint{0.765000in}{0.660000in}}{\pgfqpoint{4.620000in}{4.620000in}}%
\pgfusepath{clip}%
\pgfsetbuttcap%
\pgfsetroundjoin%
\definecolor{currentfill}{rgb}{1.000000,0.894118,0.788235}%
\pgfsetfillcolor{currentfill}%
\pgfsetlinewidth{0.000000pt}%
\definecolor{currentstroke}{rgb}{1.000000,0.894118,0.788235}%
\pgfsetstrokecolor{currentstroke}%
\pgfsetdash{}{0pt}%
\pgfpathmoveto{\pgfqpoint{3.006193in}{3.769934in}}%
\pgfpathlineto{\pgfqpoint{2.979245in}{3.754376in}}%
\pgfpathlineto{\pgfqpoint{2.979245in}{3.785492in}}%
\pgfpathlineto{\pgfqpoint{3.006193in}{3.801050in}}%
\pgfpathlineto{\pgfqpoint{3.006193in}{3.769934in}}%
\pgfpathclose%
\pgfusepath{fill}%
\end{pgfscope}%
\begin{pgfscope}%
\pgfpathrectangle{\pgfqpoint{0.765000in}{0.660000in}}{\pgfqpoint{4.620000in}{4.620000in}}%
\pgfusepath{clip}%
\pgfsetbuttcap%
\pgfsetroundjoin%
\definecolor{currentfill}{rgb}{1.000000,0.894118,0.788235}%
\pgfsetfillcolor{currentfill}%
\pgfsetlinewidth{0.000000pt}%
\definecolor{currentstroke}{rgb}{1.000000,0.894118,0.788235}%
\pgfsetstrokecolor{currentstroke}%
\pgfsetdash{}{0pt}%
\pgfpathmoveto{\pgfqpoint{3.006193in}{3.769934in}}%
\pgfpathlineto{\pgfqpoint{3.033140in}{3.754376in}}%
\pgfpathlineto{\pgfqpoint{3.033140in}{3.785492in}}%
\pgfpathlineto{\pgfqpoint{3.006193in}{3.801050in}}%
\pgfpathlineto{\pgfqpoint{3.006193in}{3.769934in}}%
\pgfpathclose%
\pgfusepath{fill}%
\end{pgfscope}%
\begin{pgfscope}%
\pgfpathrectangle{\pgfqpoint{0.765000in}{0.660000in}}{\pgfqpoint{4.620000in}{4.620000in}}%
\pgfusepath{clip}%
\pgfsetbuttcap%
\pgfsetroundjoin%
\definecolor{currentfill}{rgb}{1.000000,0.894118,0.788235}%
\pgfsetfillcolor{currentfill}%
\pgfsetlinewidth{0.000000pt}%
\definecolor{currentstroke}{rgb}{1.000000,0.894118,0.788235}%
\pgfsetstrokecolor{currentstroke}%
\pgfsetdash{}{0pt}%
\pgfpathmoveto{\pgfqpoint{3.176663in}{3.599980in}}%
\pgfpathlineto{\pgfqpoint{3.149716in}{3.584422in}}%
\pgfpathlineto{\pgfqpoint{3.176663in}{3.568864in}}%
\pgfpathlineto{\pgfqpoint{3.203610in}{3.584422in}}%
\pgfpathlineto{\pgfqpoint{3.176663in}{3.599980in}}%
\pgfpathclose%
\pgfusepath{fill}%
\end{pgfscope}%
\begin{pgfscope}%
\pgfpathrectangle{\pgfqpoint{0.765000in}{0.660000in}}{\pgfqpoint{4.620000in}{4.620000in}}%
\pgfusepath{clip}%
\pgfsetbuttcap%
\pgfsetroundjoin%
\definecolor{currentfill}{rgb}{1.000000,0.894118,0.788235}%
\pgfsetfillcolor{currentfill}%
\pgfsetlinewidth{0.000000pt}%
\definecolor{currentstroke}{rgb}{1.000000,0.894118,0.788235}%
\pgfsetstrokecolor{currentstroke}%
\pgfsetdash{}{0pt}%
\pgfpathmoveto{\pgfqpoint{3.176663in}{3.599980in}}%
\pgfpathlineto{\pgfqpoint{3.149716in}{3.584422in}}%
\pgfpathlineto{\pgfqpoint{3.149716in}{3.615538in}}%
\pgfpathlineto{\pgfqpoint{3.176663in}{3.631096in}}%
\pgfpathlineto{\pgfqpoint{3.176663in}{3.599980in}}%
\pgfpathclose%
\pgfusepath{fill}%
\end{pgfscope}%
\begin{pgfscope}%
\pgfpathrectangle{\pgfqpoint{0.765000in}{0.660000in}}{\pgfqpoint{4.620000in}{4.620000in}}%
\pgfusepath{clip}%
\pgfsetbuttcap%
\pgfsetroundjoin%
\definecolor{currentfill}{rgb}{1.000000,0.894118,0.788235}%
\pgfsetfillcolor{currentfill}%
\pgfsetlinewidth{0.000000pt}%
\definecolor{currentstroke}{rgb}{1.000000,0.894118,0.788235}%
\pgfsetstrokecolor{currentstroke}%
\pgfsetdash{}{0pt}%
\pgfpathmoveto{\pgfqpoint{3.176663in}{3.599980in}}%
\pgfpathlineto{\pgfqpoint{3.203610in}{3.584422in}}%
\pgfpathlineto{\pgfqpoint{3.203610in}{3.615538in}}%
\pgfpathlineto{\pgfqpoint{3.176663in}{3.631096in}}%
\pgfpathlineto{\pgfqpoint{3.176663in}{3.599980in}}%
\pgfpathclose%
\pgfusepath{fill}%
\end{pgfscope}%
\begin{pgfscope}%
\pgfpathrectangle{\pgfqpoint{0.765000in}{0.660000in}}{\pgfqpoint{4.620000in}{4.620000in}}%
\pgfusepath{clip}%
\pgfsetbuttcap%
\pgfsetroundjoin%
\definecolor{currentfill}{rgb}{1.000000,0.894118,0.788235}%
\pgfsetfillcolor{currentfill}%
\pgfsetlinewidth{0.000000pt}%
\definecolor{currentstroke}{rgb}{1.000000,0.894118,0.788235}%
\pgfsetstrokecolor{currentstroke}%
\pgfsetdash{}{0pt}%
\pgfpathmoveto{\pgfqpoint{3.006193in}{3.801050in}}%
\pgfpathlineto{\pgfqpoint{2.979245in}{3.785492in}}%
\pgfpathlineto{\pgfqpoint{3.006193in}{3.769934in}}%
\pgfpathlineto{\pgfqpoint{3.033140in}{3.785492in}}%
\pgfpathlineto{\pgfqpoint{3.006193in}{3.801050in}}%
\pgfpathclose%
\pgfusepath{fill}%
\end{pgfscope}%
\begin{pgfscope}%
\pgfpathrectangle{\pgfqpoint{0.765000in}{0.660000in}}{\pgfqpoint{4.620000in}{4.620000in}}%
\pgfusepath{clip}%
\pgfsetbuttcap%
\pgfsetroundjoin%
\definecolor{currentfill}{rgb}{1.000000,0.894118,0.788235}%
\pgfsetfillcolor{currentfill}%
\pgfsetlinewidth{0.000000pt}%
\definecolor{currentstroke}{rgb}{1.000000,0.894118,0.788235}%
\pgfsetstrokecolor{currentstroke}%
\pgfsetdash{}{0pt}%
\pgfpathmoveto{\pgfqpoint{3.006193in}{3.738818in}}%
\pgfpathlineto{\pgfqpoint{3.033140in}{3.754376in}}%
\pgfpathlineto{\pgfqpoint{3.033140in}{3.785492in}}%
\pgfpathlineto{\pgfqpoint{3.006193in}{3.769934in}}%
\pgfpathlineto{\pgfqpoint{3.006193in}{3.738818in}}%
\pgfpathclose%
\pgfusepath{fill}%
\end{pgfscope}%
\begin{pgfscope}%
\pgfpathrectangle{\pgfqpoint{0.765000in}{0.660000in}}{\pgfqpoint{4.620000in}{4.620000in}}%
\pgfusepath{clip}%
\pgfsetbuttcap%
\pgfsetroundjoin%
\definecolor{currentfill}{rgb}{1.000000,0.894118,0.788235}%
\pgfsetfillcolor{currentfill}%
\pgfsetlinewidth{0.000000pt}%
\definecolor{currentstroke}{rgb}{1.000000,0.894118,0.788235}%
\pgfsetstrokecolor{currentstroke}%
\pgfsetdash{}{0pt}%
\pgfpathmoveto{\pgfqpoint{2.979245in}{3.754376in}}%
\pgfpathlineto{\pgfqpoint{3.006193in}{3.738818in}}%
\pgfpathlineto{\pgfqpoint{3.006193in}{3.769934in}}%
\pgfpathlineto{\pgfqpoint{2.979245in}{3.785492in}}%
\pgfpathlineto{\pgfqpoint{2.979245in}{3.754376in}}%
\pgfpathclose%
\pgfusepath{fill}%
\end{pgfscope}%
\begin{pgfscope}%
\pgfpathrectangle{\pgfqpoint{0.765000in}{0.660000in}}{\pgfqpoint{4.620000in}{4.620000in}}%
\pgfusepath{clip}%
\pgfsetbuttcap%
\pgfsetroundjoin%
\definecolor{currentfill}{rgb}{1.000000,0.894118,0.788235}%
\pgfsetfillcolor{currentfill}%
\pgfsetlinewidth{0.000000pt}%
\definecolor{currentstroke}{rgb}{1.000000,0.894118,0.788235}%
\pgfsetstrokecolor{currentstroke}%
\pgfsetdash{}{0pt}%
\pgfpathmoveto{\pgfqpoint{3.176663in}{3.631096in}}%
\pgfpathlineto{\pgfqpoint{3.149716in}{3.615538in}}%
\pgfpathlineto{\pgfqpoint{3.176663in}{3.599980in}}%
\pgfpathlineto{\pgfqpoint{3.203610in}{3.615538in}}%
\pgfpathlineto{\pgfqpoint{3.176663in}{3.631096in}}%
\pgfpathclose%
\pgfusepath{fill}%
\end{pgfscope}%
\begin{pgfscope}%
\pgfpathrectangle{\pgfqpoint{0.765000in}{0.660000in}}{\pgfqpoint{4.620000in}{4.620000in}}%
\pgfusepath{clip}%
\pgfsetbuttcap%
\pgfsetroundjoin%
\definecolor{currentfill}{rgb}{1.000000,0.894118,0.788235}%
\pgfsetfillcolor{currentfill}%
\pgfsetlinewidth{0.000000pt}%
\definecolor{currentstroke}{rgb}{1.000000,0.894118,0.788235}%
\pgfsetstrokecolor{currentstroke}%
\pgfsetdash{}{0pt}%
\pgfpathmoveto{\pgfqpoint{3.176663in}{3.568864in}}%
\pgfpathlineto{\pgfqpoint{3.203610in}{3.584422in}}%
\pgfpathlineto{\pgfqpoint{3.203610in}{3.615538in}}%
\pgfpathlineto{\pgfqpoint{3.176663in}{3.599980in}}%
\pgfpathlineto{\pgfqpoint{3.176663in}{3.568864in}}%
\pgfpathclose%
\pgfusepath{fill}%
\end{pgfscope}%
\begin{pgfscope}%
\pgfpathrectangle{\pgfqpoint{0.765000in}{0.660000in}}{\pgfqpoint{4.620000in}{4.620000in}}%
\pgfusepath{clip}%
\pgfsetbuttcap%
\pgfsetroundjoin%
\definecolor{currentfill}{rgb}{1.000000,0.894118,0.788235}%
\pgfsetfillcolor{currentfill}%
\pgfsetlinewidth{0.000000pt}%
\definecolor{currentstroke}{rgb}{1.000000,0.894118,0.788235}%
\pgfsetstrokecolor{currentstroke}%
\pgfsetdash{}{0pt}%
\pgfpathmoveto{\pgfqpoint{3.149716in}{3.584422in}}%
\pgfpathlineto{\pgfqpoint{3.176663in}{3.568864in}}%
\pgfpathlineto{\pgfqpoint{3.176663in}{3.599980in}}%
\pgfpathlineto{\pgfqpoint{3.149716in}{3.615538in}}%
\pgfpathlineto{\pgfqpoint{3.149716in}{3.584422in}}%
\pgfpathclose%
\pgfusepath{fill}%
\end{pgfscope}%
\begin{pgfscope}%
\pgfpathrectangle{\pgfqpoint{0.765000in}{0.660000in}}{\pgfqpoint{4.620000in}{4.620000in}}%
\pgfusepath{clip}%
\pgfsetbuttcap%
\pgfsetroundjoin%
\definecolor{currentfill}{rgb}{1.000000,0.894118,0.788235}%
\pgfsetfillcolor{currentfill}%
\pgfsetlinewidth{0.000000pt}%
\definecolor{currentstroke}{rgb}{1.000000,0.894118,0.788235}%
\pgfsetstrokecolor{currentstroke}%
\pgfsetdash{}{0pt}%
\pgfpathmoveto{\pgfqpoint{3.006193in}{3.769934in}}%
\pgfpathlineto{\pgfqpoint{2.979245in}{3.754376in}}%
\pgfpathlineto{\pgfqpoint{3.149716in}{3.584422in}}%
\pgfpathlineto{\pgfqpoint{3.176663in}{3.599980in}}%
\pgfpathlineto{\pgfqpoint{3.006193in}{3.769934in}}%
\pgfpathclose%
\pgfusepath{fill}%
\end{pgfscope}%
\begin{pgfscope}%
\pgfpathrectangle{\pgfqpoint{0.765000in}{0.660000in}}{\pgfqpoint{4.620000in}{4.620000in}}%
\pgfusepath{clip}%
\pgfsetbuttcap%
\pgfsetroundjoin%
\definecolor{currentfill}{rgb}{1.000000,0.894118,0.788235}%
\pgfsetfillcolor{currentfill}%
\pgfsetlinewidth{0.000000pt}%
\definecolor{currentstroke}{rgb}{1.000000,0.894118,0.788235}%
\pgfsetstrokecolor{currentstroke}%
\pgfsetdash{}{0pt}%
\pgfpathmoveto{\pgfqpoint{3.033140in}{3.754376in}}%
\pgfpathlineto{\pgfqpoint{3.006193in}{3.769934in}}%
\pgfpathlineto{\pgfqpoint{3.176663in}{3.599980in}}%
\pgfpathlineto{\pgfqpoint{3.203610in}{3.584422in}}%
\pgfpathlineto{\pgfqpoint{3.033140in}{3.754376in}}%
\pgfpathclose%
\pgfusepath{fill}%
\end{pgfscope}%
\begin{pgfscope}%
\pgfpathrectangle{\pgfqpoint{0.765000in}{0.660000in}}{\pgfqpoint{4.620000in}{4.620000in}}%
\pgfusepath{clip}%
\pgfsetbuttcap%
\pgfsetroundjoin%
\definecolor{currentfill}{rgb}{1.000000,0.894118,0.788235}%
\pgfsetfillcolor{currentfill}%
\pgfsetlinewidth{0.000000pt}%
\definecolor{currentstroke}{rgb}{1.000000,0.894118,0.788235}%
\pgfsetstrokecolor{currentstroke}%
\pgfsetdash{}{0pt}%
\pgfpathmoveto{\pgfqpoint{3.006193in}{3.769934in}}%
\pgfpathlineto{\pgfqpoint{3.006193in}{3.801050in}}%
\pgfpathlineto{\pgfqpoint{3.176663in}{3.631096in}}%
\pgfpathlineto{\pgfqpoint{3.203610in}{3.584422in}}%
\pgfpathlineto{\pgfqpoint{3.006193in}{3.769934in}}%
\pgfpathclose%
\pgfusepath{fill}%
\end{pgfscope}%
\begin{pgfscope}%
\pgfpathrectangle{\pgfqpoint{0.765000in}{0.660000in}}{\pgfqpoint{4.620000in}{4.620000in}}%
\pgfusepath{clip}%
\pgfsetbuttcap%
\pgfsetroundjoin%
\definecolor{currentfill}{rgb}{1.000000,0.894118,0.788235}%
\pgfsetfillcolor{currentfill}%
\pgfsetlinewidth{0.000000pt}%
\definecolor{currentstroke}{rgb}{1.000000,0.894118,0.788235}%
\pgfsetstrokecolor{currentstroke}%
\pgfsetdash{}{0pt}%
\pgfpathmoveto{\pgfqpoint{3.033140in}{3.754376in}}%
\pgfpathlineto{\pgfqpoint{3.033140in}{3.785492in}}%
\pgfpathlineto{\pgfqpoint{3.203610in}{3.615538in}}%
\pgfpathlineto{\pgfqpoint{3.176663in}{3.599980in}}%
\pgfpathlineto{\pgfqpoint{3.033140in}{3.754376in}}%
\pgfpathclose%
\pgfusepath{fill}%
\end{pgfscope}%
\begin{pgfscope}%
\pgfpathrectangle{\pgfqpoint{0.765000in}{0.660000in}}{\pgfqpoint{4.620000in}{4.620000in}}%
\pgfusepath{clip}%
\pgfsetbuttcap%
\pgfsetroundjoin%
\definecolor{currentfill}{rgb}{1.000000,0.894118,0.788235}%
\pgfsetfillcolor{currentfill}%
\pgfsetlinewidth{0.000000pt}%
\definecolor{currentstroke}{rgb}{1.000000,0.894118,0.788235}%
\pgfsetstrokecolor{currentstroke}%
\pgfsetdash{}{0pt}%
\pgfpathmoveto{\pgfqpoint{2.979245in}{3.754376in}}%
\pgfpathlineto{\pgfqpoint{3.006193in}{3.738818in}}%
\pgfpathlineto{\pgfqpoint{3.176663in}{3.568864in}}%
\pgfpathlineto{\pgfqpoint{3.149716in}{3.584422in}}%
\pgfpathlineto{\pgfqpoint{2.979245in}{3.754376in}}%
\pgfpathclose%
\pgfusepath{fill}%
\end{pgfscope}%
\begin{pgfscope}%
\pgfpathrectangle{\pgfqpoint{0.765000in}{0.660000in}}{\pgfqpoint{4.620000in}{4.620000in}}%
\pgfusepath{clip}%
\pgfsetbuttcap%
\pgfsetroundjoin%
\definecolor{currentfill}{rgb}{1.000000,0.894118,0.788235}%
\pgfsetfillcolor{currentfill}%
\pgfsetlinewidth{0.000000pt}%
\definecolor{currentstroke}{rgb}{1.000000,0.894118,0.788235}%
\pgfsetstrokecolor{currentstroke}%
\pgfsetdash{}{0pt}%
\pgfpathmoveto{\pgfqpoint{3.006193in}{3.738818in}}%
\pgfpathlineto{\pgfqpoint{3.033140in}{3.754376in}}%
\pgfpathlineto{\pgfqpoint{3.203610in}{3.584422in}}%
\pgfpathlineto{\pgfqpoint{3.176663in}{3.568864in}}%
\pgfpathlineto{\pgfqpoint{3.006193in}{3.738818in}}%
\pgfpathclose%
\pgfusepath{fill}%
\end{pgfscope}%
\begin{pgfscope}%
\pgfpathrectangle{\pgfqpoint{0.765000in}{0.660000in}}{\pgfqpoint{4.620000in}{4.620000in}}%
\pgfusepath{clip}%
\pgfsetbuttcap%
\pgfsetroundjoin%
\definecolor{currentfill}{rgb}{1.000000,0.894118,0.788235}%
\pgfsetfillcolor{currentfill}%
\pgfsetlinewidth{0.000000pt}%
\definecolor{currentstroke}{rgb}{1.000000,0.894118,0.788235}%
\pgfsetstrokecolor{currentstroke}%
\pgfsetdash{}{0pt}%
\pgfpathmoveto{\pgfqpoint{3.006193in}{3.801050in}}%
\pgfpathlineto{\pgfqpoint{2.979245in}{3.785492in}}%
\pgfpathlineto{\pgfqpoint{3.149716in}{3.615538in}}%
\pgfpathlineto{\pgfqpoint{3.176663in}{3.631096in}}%
\pgfpathlineto{\pgfqpoint{3.006193in}{3.801050in}}%
\pgfpathclose%
\pgfusepath{fill}%
\end{pgfscope}%
\begin{pgfscope}%
\pgfpathrectangle{\pgfqpoint{0.765000in}{0.660000in}}{\pgfqpoint{4.620000in}{4.620000in}}%
\pgfusepath{clip}%
\pgfsetbuttcap%
\pgfsetroundjoin%
\definecolor{currentfill}{rgb}{1.000000,0.894118,0.788235}%
\pgfsetfillcolor{currentfill}%
\pgfsetlinewidth{0.000000pt}%
\definecolor{currentstroke}{rgb}{1.000000,0.894118,0.788235}%
\pgfsetstrokecolor{currentstroke}%
\pgfsetdash{}{0pt}%
\pgfpathmoveto{\pgfqpoint{3.033140in}{3.785492in}}%
\pgfpathlineto{\pgfqpoint{3.006193in}{3.801050in}}%
\pgfpathlineto{\pgfqpoint{3.176663in}{3.631096in}}%
\pgfpathlineto{\pgfqpoint{3.203610in}{3.615538in}}%
\pgfpathlineto{\pgfqpoint{3.033140in}{3.785492in}}%
\pgfpathclose%
\pgfusepath{fill}%
\end{pgfscope}%
\begin{pgfscope}%
\pgfpathrectangle{\pgfqpoint{0.765000in}{0.660000in}}{\pgfqpoint{4.620000in}{4.620000in}}%
\pgfusepath{clip}%
\pgfsetbuttcap%
\pgfsetroundjoin%
\definecolor{currentfill}{rgb}{1.000000,0.894118,0.788235}%
\pgfsetfillcolor{currentfill}%
\pgfsetlinewidth{0.000000pt}%
\definecolor{currentstroke}{rgb}{1.000000,0.894118,0.788235}%
\pgfsetstrokecolor{currentstroke}%
\pgfsetdash{}{0pt}%
\pgfpathmoveto{\pgfqpoint{2.979245in}{3.754376in}}%
\pgfpathlineto{\pgfqpoint{2.979245in}{3.785492in}}%
\pgfpathlineto{\pgfqpoint{3.149716in}{3.615538in}}%
\pgfpathlineto{\pgfqpoint{3.176663in}{3.568864in}}%
\pgfpathlineto{\pgfqpoint{2.979245in}{3.754376in}}%
\pgfpathclose%
\pgfusepath{fill}%
\end{pgfscope}%
\begin{pgfscope}%
\pgfpathrectangle{\pgfqpoint{0.765000in}{0.660000in}}{\pgfqpoint{4.620000in}{4.620000in}}%
\pgfusepath{clip}%
\pgfsetbuttcap%
\pgfsetroundjoin%
\definecolor{currentfill}{rgb}{1.000000,0.894118,0.788235}%
\pgfsetfillcolor{currentfill}%
\pgfsetlinewidth{0.000000pt}%
\definecolor{currentstroke}{rgb}{1.000000,0.894118,0.788235}%
\pgfsetstrokecolor{currentstroke}%
\pgfsetdash{}{0pt}%
\pgfpathmoveto{\pgfqpoint{3.006193in}{3.738818in}}%
\pgfpathlineto{\pgfqpoint{3.006193in}{3.769934in}}%
\pgfpathlineto{\pgfqpoint{3.176663in}{3.599980in}}%
\pgfpathlineto{\pgfqpoint{3.149716in}{3.584422in}}%
\pgfpathlineto{\pgfqpoint{3.006193in}{3.738818in}}%
\pgfpathclose%
\pgfusepath{fill}%
\end{pgfscope}%
\begin{pgfscope}%
\pgfpathrectangle{\pgfqpoint{0.765000in}{0.660000in}}{\pgfqpoint{4.620000in}{4.620000in}}%
\pgfusepath{clip}%
\pgfsetbuttcap%
\pgfsetroundjoin%
\definecolor{currentfill}{rgb}{1.000000,0.894118,0.788235}%
\pgfsetfillcolor{currentfill}%
\pgfsetlinewidth{0.000000pt}%
\definecolor{currentstroke}{rgb}{1.000000,0.894118,0.788235}%
\pgfsetstrokecolor{currentstroke}%
\pgfsetdash{}{0pt}%
\pgfpathmoveto{\pgfqpoint{3.006193in}{3.769934in}}%
\pgfpathlineto{\pgfqpoint{3.033140in}{3.785492in}}%
\pgfpathlineto{\pgfqpoint{3.203610in}{3.615538in}}%
\pgfpathlineto{\pgfqpoint{3.176663in}{3.599980in}}%
\pgfpathlineto{\pgfqpoint{3.006193in}{3.769934in}}%
\pgfpathclose%
\pgfusepath{fill}%
\end{pgfscope}%
\begin{pgfscope}%
\pgfpathrectangle{\pgfqpoint{0.765000in}{0.660000in}}{\pgfqpoint{4.620000in}{4.620000in}}%
\pgfusepath{clip}%
\pgfsetbuttcap%
\pgfsetroundjoin%
\definecolor{currentfill}{rgb}{1.000000,0.894118,0.788235}%
\pgfsetfillcolor{currentfill}%
\pgfsetlinewidth{0.000000pt}%
\definecolor{currentstroke}{rgb}{1.000000,0.894118,0.788235}%
\pgfsetstrokecolor{currentstroke}%
\pgfsetdash{}{0pt}%
\pgfpathmoveto{\pgfqpoint{2.979245in}{3.785492in}}%
\pgfpathlineto{\pgfqpoint{3.006193in}{3.769934in}}%
\pgfpathlineto{\pgfqpoint{3.176663in}{3.599980in}}%
\pgfpathlineto{\pgfqpoint{3.149716in}{3.615538in}}%
\pgfpathlineto{\pgfqpoint{2.979245in}{3.785492in}}%
\pgfpathclose%
\pgfusepath{fill}%
\end{pgfscope}%
\begin{pgfscope}%
\pgfpathrectangle{\pgfqpoint{0.765000in}{0.660000in}}{\pgfqpoint{4.620000in}{4.620000in}}%
\pgfusepath{clip}%
\pgfsetbuttcap%
\pgfsetroundjoin%
\definecolor{currentfill}{rgb}{1.000000,0.894118,0.788235}%
\pgfsetfillcolor{currentfill}%
\pgfsetlinewidth{0.000000pt}%
\definecolor{currentstroke}{rgb}{1.000000,0.894118,0.788235}%
\pgfsetstrokecolor{currentstroke}%
\pgfsetdash{}{0pt}%
\pgfpathmoveto{\pgfqpoint{3.006193in}{3.769934in}}%
\pgfpathlineto{\pgfqpoint{2.979245in}{3.754376in}}%
\pgfpathlineto{\pgfqpoint{3.006193in}{3.738818in}}%
\pgfpathlineto{\pgfqpoint{3.033140in}{3.754376in}}%
\pgfpathlineto{\pgfqpoint{3.006193in}{3.769934in}}%
\pgfpathclose%
\pgfusepath{fill}%
\end{pgfscope}%
\begin{pgfscope}%
\pgfpathrectangle{\pgfqpoint{0.765000in}{0.660000in}}{\pgfqpoint{4.620000in}{4.620000in}}%
\pgfusepath{clip}%
\pgfsetbuttcap%
\pgfsetroundjoin%
\definecolor{currentfill}{rgb}{1.000000,0.894118,0.788235}%
\pgfsetfillcolor{currentfill}%
\pgfsetlinewidth{0.000000pt}%
\definecolor{currentstroke}{rgb}{1.000000,0.894118,0.788235}%
\pgfsetstrokecolor{currentstroke}%
\pgfsetdash{}{0pt}%
\pgfpathmoveto{\pgfqpoint{3.006193in}{3.769934in}}%
\pgfpathlineto{\pgfqpoint{2.979245in}{3.754376in}}%
\pgfpathlineto{\pgfqpoint{2.979245in}{3.785492in}}%
\pgfpathlineto{\pgfqpoint{3.006193in}{3.801050in}}%
\pgfpathlineto{\pgfqpoint{3.006193in}{3.769934in}}%
\pgfpathclose%
\pgfusepath{fill}%
\end{pgfscope}%
\begin{pgfscope}%
\pgfpathrectangle{\pgfqpoint{0.765000in}{0.660000in}}{\pgfqpoint{4.620000in}{4.620000in}}%
\pgfusepath{clip}%
\pgfsetbuttcap%
\pgfsetroundjoin%
\definecolor{currentfill}{rgb}{1.000000,0.894118,0.788235}%
\pgfsetfillcolor{currentfill}%
\pgfsetlinewidth{0.000000pt}%
\definecolor{currentstroke}{rgb}{1.000000,0.894118,0.788235}%
\pgfsetstrokecolor{currentstroke}%
\pgfsetdash{}{0pt}%
\pgfpathmoveto{\pgfqpoint{3.006193in}{3.769934in}}%
\pgfpathlineto{\pgfqpoint{3.033140in}{3.754376in}}%
\pgfpathlineto{\pgfqpoint{3.033140in}{3.785492in}}%
\pgfpathlineto{\pgfqpoint{3.006193in}{3.801050in}}%
\pgfpathlineto{\pgfqpoint{3.006193in}{3.769934in}}%
\pgfpathclose%
\pgfusepath{fill}%
\end{pgfscope}%
\begin{pgfscope}%
\pgfpathrectangle{\pgfqpoint{0.765000in}{0.660000in}}{\pgfqpoint{4.620000in}{4.620000in}}%
\pgfusepath{clip}%
\pgfsetbuttcap%
\pgfsetroundjoin%
\definecolor{currentfill}{rgb}{1.000000,0.894118,0.788235}%
\pgfsetfillcolor{currentfill}%
\pgfsetlinewidth{0.000000pt}%
\definecolor{currentstroke}{rgb}{1.000000,0.894118,0.788235}%
\pgfsetstrokecolor{currentstroke}%
\pgfsetdash{}{0pt}%
\pgfpathmoveto{\pgfqpoint{2.981592in}{3.650866in}}%
\pgfpathlineto{\pgfqpoint{2.954645in}{3.635308in}}%
\pgfpathlineto{\pgfqpoint{2.981592in}{3.619750in}}%
\pgfpathlineto{\pgfqpoint{3.008539in}{3.635308in}}%
\pgfpathlineto{\pgfqpoint{2.981592in}{3.650866in}}%
\pgfpathclose%
\pgfusepath{fill}%
\end{pgfscope}%
\begin{pgfscope}%
\pgfpathrectangle{\pgfqpoint{0.765000in}{0.660000in}}{\pgfqpoint{4.620000in}{4.620000in}}%
\pgfusepath{clip}%
\pgfsetbuttcap%
\pgfsetroundjoin%
\definecolor{currentfill}{rgb}{1.000000,0.894118,0.788235}%
\pgfsetfillcolor{currentfill}%
\pgfsetlinewidth{0.000000pt}%
\definecolor{currentstroke}{rgb}{1.000000,0.894118,0.788235}%
\pgfsetstrokecolor{currentstroke}%
\pgfsetdash{}{0pt}%
\pgfpathmoveto{\pgfqpoint{2.981592in}{3.650866in}}%
\pgfpathlineto{\pgfqpoint{2.954645in}{3.635308in}}%
\pgfpathlineto{\pgfqpoint{2.954645in}{3.666424in}}%
\pgfpathlineto{\pgfqpoint{2.981592in}{3.681982in}}%
\pgfpathlineto{\pgfqpoint{2.981592in}{3.650866in}}%
\pgfpathclose%
\pgfusepath{fill}%
\end{pgfscope}%
\begin{pgfscope}%
\pgfpathrectangle{\pgfqpoint{0.765000in}{0.660000in}}{\pgfqpoint{4.620000in}{4.620000in}}%
\pgfusepath{clip}%
\pgfsetbuttcap%
\pgfsetroundjoin%
\definecolor{currentfill}{rgb}{1.000000,0.894118,0.788235}%
\pgfsetfillcolor{currentfill}%
\pgfsetlinewidth{0.000000pt}%
\definecolor{currentstroke}{rgb}{1.000000,0.894118,0.788235}%
\pgfsetstrokecolor{currentstroke}%
\pgfsetdash{}{0pt}%
\pgfpathmoveto{\pgfqpoint{2.981592in}{3.650866in}}%
\pgfpathlineto{\pgfqpoint{3.008539in}{3.635308in}}%
\pgfpathlineto{\pgfqpoint{3.008539in}{3.666424in}}%
\pgfpathlineto{\pgfqpoint{2.981592in}{3.681982in}}%
\pgfpathlineto{\pgfqpoint{2.981592in}{3.650866in}}%
\pgfpathclose%
\pgfusepath{fill}%
\end{pgfscope}%
\begin{pgfscope}%
\pgfpathrectangle{\pgfqpoint{0.765000in}{0.660000in}}{\pgfqpoint{4.620000in}{4.620000in}}%
\pgfusepath{clip}%
\pgfsetbuttcap%
\pgfsetroundjoin%
\definecolor{currentfill}{rgb}{1.000000,0.894118,0.788235}%
\pgfsetfillcolor{currentfill}%
\pgfsetlinewidth{0.000000pt}%
\definecolor{currentstroke}{rgb}{1.000000,0.894118,0.788235}%
\pgfsetstrokecolor{currentstroke}%
\pgfsetdash{}{0pt}%
\pgfpathmoveto{\pgfqpoint{3.006193in}{3.801050in}}%
\pgfpathlineto{\pgfqpoint{2.979245in}{3.785492in}}%
\pgfpathlineto{\pgfqpoint{3.006193in}{3.769934in}}%
\pgfpathlineto{\pgfqpoint{3.033140in}{3.785492in}}%
\pgfpathlineto{\pgfqpoint{3.006193in}{3.801050in}}%
\pgfpathclose%
\pgfusepath{fill}%
\end{pgfscope}%
\begin{pgfscope}%
\pgfpathrectangle{\pgfqpoint{0.765000in}{0.660000in}}{\pgfqpoint{4.620000in}{4.620000in}}%
\pgfusepath{clip}%
\pgfsetbuttcap%
\pgfsetroundjoin%
\definecolor{currentfill}{rgb}{1.000000,0.894118,0.788235}%
\pgfsetfillcolor{currentfill}%
\pgfsetlinewidth{0.000000pt}%
\definecolor{currentstroke}{rgb}{1.000000,0.894118,0.788235}%
\pgfsetstrokecolor{currentstroke}%
\pgfsetdash{}{0pt}%
\pgfpathmoveto{\pgfqpoint{3.006193in}{3.738818in}}%
\pgfpathlineto{\pgfqpoint{3.033140in}{3.754376in}}%
\pgfpathlineto{\pgfqpoint{3.033140in}{3.785492in}}%
\pgfpathlineto{\pgfqpoint{3.006193in}{3.769934in}}%
\pgfpathlineto{\pgfqpoint{3.006193in}{3.738818in}}%
\pgfpathclose%
\pgfusepath{fill}%
\end{pgfscope}%
\begin{pgfscope}%
\pgfpathrectangle{\pgfqpoint{0.765000in}{0.660000in}}{\pgfqpoint{4.620000in}{4.620000in}}%
\pgfusepath{clip}%
\pgfsetbuttcap%
\pgfsetroundjoin%
\definecolor{currentfill}{rgb}{1.000000,0.894118,0.788235}%
\pgfsetfillcolor{currentfill}%
\pgfsetlinewidth{0.000000pt}%
\definecolor{currentstroke}{rgb}{1.000000,0.894118,0.788235}%
\pgfsetstrokecolor{currentstroke}%
\pgfsetdash{}{0pt}%
\pgfpathmoveto{\pgfqpoint{2.979245in}{3.754376in}}%
\pgfpathlineto{\pgfqpoint{3.006193in}{3.738818in}}%
\pgfpathlineto{\pgfqpoint{3.006193in}{3.769934in}}%
\pgfpathlineto{\pgfqpoint{2.979245in}{3.785492in}}%
\pgfpathlineto{\pgfqpoint{2.979245in}{3.754376in}}%
\pgfpathclose%
\pgfusepath{fill}%
\end{pgfscope}%
\begin{pgfscope}%
\pgfpathrectangle{\pgfqpoint{0.765000in}{0.660000in}}{\pgfqpoint{4.620000in}{4.620000in}}%
\pgfusepath{clip}%
\pgfsetbuttcap%
\pgfsetroundjoin%
\definecolor{currentfill}{rgb}{1.000000,0.894118,0.788235}%
\pgfsetfillcolor{currentfill}%
\pgfsetlinewidth{0.000000pt}%
\definecolor{currentstroke}{rgb}{1.000000,0.894118,0.788235}%
\pgfsetstrokecolor{currentstroke}%
\pgfsetdash{}{0pt}%
\pgfpathmoveto{\pgfqpoint{2.981592in}{3.681982in}}%
\pgfpathlineto{\pgfqpoint{2.954645in}{3.666424in}}%
\pgfpathlineto{\pgfqpoint{2.981592in}{3.650866in}}%
\pgfpathlineto{\pgfqpoint{3.008539in}{3.666424in}}%
\pgfpathlineto{\pgfqpoint{2.981592in}{3.681982in}}%
\pgfpathclose%
\pgfusepath{fill}%
\end{pgfscope}%
\begin{pgfscope}%
\pgfpathrectangle{\pgfqpoint{0.765000in}{0.660000in}}{\pgfqpoint{4.620000in}{4.620000in}}%
\pgfusepath{clip}%
\pgfsetbuttcap%
\pgfsetroundjoin%
\definecolor{currentfill}{rgb}{1.000000,0.894118,0.788235}%
\pgfsetfillcolor{currentfill}%
\pgfsetlinewidth{0.000000pt}%
\definecolor{currentstroke}{rgb}{1.000000,0.894118,0.788235}%
\pgfsetstrokecolor{currentstroke}%
\pgfsetdash{}{0pt}%
\pgfpathmoveto{\pgfqpoint{2.981592in}{3.619750in}}%
\pgfpathlineto{\pgfqpoint{3.008539in}{3.635308in}}%
\pgfpathlineto{\pgfqpoint{3.008539in}{3.666424in}}%
\pgfpathlineto{\pgfqpoint{2.981592in}{3.650866in}}%
\pgfpathlineto{\pgfqpoint{2.981592in}{3.619750in}}%
\pgfpathclose%
\pgfusepath{fill}%
\end{pgfscope}%
\begin{pgfscope}%
\pgfpathrectangle{\pgfqpoint{0.765000in}{0.660000in}}{\pgfqpoint{4.620000in}{4.620000in}}%
\pgfusepath{clip}%
\pgfsetbuttcap%
\pgfsetroundjoin%
\definecolor{currentfill}{rgb}{1.000000,0.894118,0.788235}%
\pgfsetfillcolor{currentfill}%
\pgfsetlinewidth{0.000000pt}%
\definecolor{currentstroke}{rgb}{1.000000,0.894118,0.788235}%
\pgfsetstrokecolor{currentstroke}%
\pgfsetdash{}{0pt}%
\pgfpathmoveto{\pgfqpoint{2.954645in}{3.635308in}}%
\pgfpathlineto{\pgfqpoint{2.981592in}{3.619750in}}%
\pgfpathlineto{\pgfqpoint{2.981592in}{3.650866in}}%
\pgfpathlineto{\pgfqpoint{2.954645in}{3.666424in}}%
\pgfpathlineto{\pgfqpoint{2.954645in}{3.635308in}}%
\pgfpathclose%
\pgfusepath{fill}%
\end{pgfscope}%
\begin{pgfscope}%
\pgfpathrectangle{\pgfqpoint{0.765000in}{0.660000in}}{\pgfqpoint{4.620000in}{4.620000in}}%
\pgfusepath{clip}%
\pgfsetbuttcap%
\pgfsetroundjoin%
\definecolor{currentfill}{rgb}{1.000000,0.894118,0.788235}%
\pgfsetfillcolor{currentfill}%
\pgfsetlinewidth{0.000000pt}%
\definecolor{currentstroke}{rgb}{1.000000,0.894118,0.788235}%
\pgfsetstrokecolor{currentstroke}%
\pgfsetdash{}{0pt}%
\pgfpathmoveto{\pgfqpoint{3.006193in}{3.769934in}}%
\pgfpathlineto{\pgfqpoint{2.979245in}{3.754376in}}%
\pgfpathlineto{\pgfqpoint{2.954645in}{3.635308in}}%
\pgfpathlineto{\pgfqpoint{2.981592in}{3.650866in}}%
\pgfpathlineto{\pgfqpoint{3.006193in}{3.769934in}}%
\pgfpathclose%
\pgfusepath{fill}%
\end{pgfscope}%
\begin{pgfscope}%
\pgfpathrectangle{\pgfqpoint{0.765000in}{0.660000in}}{\pgfqpoint{4.620000in}{4.620000in}}%
\pgfusepath{clip}%
\pgfsetbuttcap%
\pgfsetroundjoin%
\definecolor{currentfill}{rgb}{1.000000,0.894118,0.788235}%
\pgfsetfillcolor{currentfill}%
\pgfsetlinewidth{0.000000pt}%
\definecolor{currentstroke}{rgb}{1.000000,0.894118,0.788235}%
\pgfsetstrokecolor{currentstroke}%
\pgfsetdash{}{0pt}%
\pgfpathmoveto{\pgfqpoint{3.033140in}{3.754376in}}%
\pgfpathlineto{\pgfqpoint{3.006193in}{3.769934in}}%
\pgfpathlineto{\pgfqpoint{2.981592in}{3.650866in}}%
\pgfpathlineto{\pgfqpoint{3.008539in}{3.635308in}}%
\pgfpathlineto{\pgfqpoint{3.033140in}{3.754376in}}%
\pgfpathclose%
\pgfusepath{fill}%
\end{pgfscope}%
\begin{pgfscope}%
\pgfpathrectangle{\pgfqpoint{0.765000in}{0.660000in}}{\pgfqpoint{4.620000in}{4.620000in}}%
\pgfusepath{clip}%
\pgfsetbuttcap%
\pgfsetroundjoin%
\definecolor{currentfill}{rgb}{1.000000,0.894118,0.788235}%
\pgfsetfillcolor{currentfill}%
\pgfsetlinewidth{0.000000pt}%
\definecolor{currentstroke}{rgb}{1.000000,0.894118,0.788235}%
\pgfsetstrokecolor{currentstroke}%
\pgfsetdash{}{0pt}%
\pgfpathmoveto{\pgfqpoint{3.006193in}{3.769934in}}%
\pgfpathlineto{\pgfqpoint{3.006193in}{3.801050in}}%
\pgfpathlineto{\pgfqpoint{2.981592in}{3.681982in}}%
\pgfpathlineto{\pgfqpoint{3.008539in}{3.635308in}}%
\pgfpathlineto{\pgfqpoint{3.006193in}{3.769934in}}%
\pgfpathclose%
\pgfusepath{fill}%
\end{pgfscope}%
\begin{pgfscope}%
\pgfpathrectangle{\pgfqpoint{0.765000in}{0.660000in}}{\pgfqpoint{4.620000in}{4.620000in}}%
\pgfusepath{clip}%
\pgfsetbuttcap%
\pgfsetroundjoin%
\definecolor{currentfill}{rgb}{1.000000,0.894118,0.788235}%
\pgfsetfillcolor{currentfill}%
\pgfsetlinewidth{0.000000pt}%
\definecolor{currentstroke}{rgb}{1.000000,0.894118,0.788235}%
\pgfsetstrokecolor{currentstroke}%
\pgfsetdash{}{0pt}%
\pgfpathmoveto{\pgfqpoint{3.033140in}{3.754376in}}%
\pgfpathlineto{\pgfqpoint{3.033140in}{3.785492in}}%
\pgfpathlineto{\pgfqpoint{3.008539in}{3.666424in}}%
\pgfpathlineto{\pgfqpoint{2.981592in}{3.650866in}}%
\pgfpathlineto{\pgfqpoint{3.033140in}{3.754376in}}%
\pgfpathclose%
\pgfusepath{fill}%
\end{pgfscope}%
\begin{pgfscope}%
\pgfpathrectangle{\pgfqpoint{0.765000in}{0.660000in}}{\pgfqpoint{4.620000in}{4.620000in}}%
\pgfusepath{clip}%
\pgfsetbuttcap%
\pgfsetroundjoin%
\definecolor{currentfill}{rgb}{1.000000,0.894118,0.788235}%
\pgfsetfillcolor{currentfill}%
\pgfsetlinewidth{0.000000pt}%
\definecolor{currentstroke}{rgb}{1.000000,0.894118,0.788235}%
\pgfsetstrokecolor{currentstroke}%
\pgfsetdash{}{0pt}%
\pgfpathmoveto{\pgfqpoint{2.979245in}{3.754376in}}%
\pgfpathlineto{\pgfqpoint{3.006193in}{3.738818in}}%
\pgfpathlineto{\pgfqpoint{2.981592in}{3.619750in}}%
\pgfpathlineto{\pgfqpoint{2.954645in}{3.635308in}}%
\pgfpathlineto{\pgfqpoint{2.979245in}{3.754376in}}%
\pgfpathclose%
\pgfusepath{fill}%
\end{pgfscope}%
\begin{pgfscope}%
\pgfpathrectangle{\pgfqpoint{0.765000in}{0.660000in}}{\pgfqpoint{4.620000in}{4.620000in}}%
\pgfusepath{clip}%
\pgfsetbuttcap%
\pgfsetroundjoin%
\definecolor{currentfill}{rgb}{1.000000,0.894118,0.788235}%
\pgfsetfillcolor{currentfill}%
\pgfsetlinewidth{0.000000pt}%
\definecolor{currentstroke}{rgb}{1.000000,0.894118,0.788235}%
\pgfsetstrokecolor{currentstroke}%
\pgfsetdash{}{0pt}%
\pgfpathmoveto{\pgfqpoint{3.006193in}{3.738818in}}%
\pgfpathlineto{\pgfqpoint{3.033140in}{3.754376in}}%
\pgfpathlineto{\pgfqpoint{3.008539in}{3.635308in}}%
\pgfpathlineto{\pgfqpoint{2.981592in}{3.619750in}}%
\pgfpathlineto{\pgfqpoint{3.006193in}{3.738818in}}%
\pgfpathclose%
\pgfusepath{fill}%
\end{pgfscope}%
\begin{pgfscope}%
\pgfpathrectangle{\pgfqpoint{0.765000in}{0.660000in}}{\pgfqpoint{4.620000in}{4.620000in}}%
\pgfusepath{clip}%
\pgfsetbuttcap%
\pgfsetroundjoin%
\definecolor{currentfill}{rgb}{1.000000,0.894118,0.788235}%
\pgfsetfillcolor{currentfill}%
\pgfsetlinewidth{0.000000pt}%
\definecolor{currentstroke}{rgb}{1.000000,0.894118,0.788235}%
\pgfsetstrokecolor{currentstroke}%
\pgfsetdash{}{0pt}%
\pgfpathmoveto{\pgfqpoint{3.006193in}{3.801050in}}%
\pgfpathlineto{\pgfqpoint{2.979245in}{3.785492in}}%
\pgfpathlineto{\pgfqpoint{2.954645in}{3.666424in}}%
\pgfpathlineto{\pgfqpoint{2.981592in}{3.681982in}}%
\pgfpathlineto{\pgfqpoint{3.006193in}{3.801050in}}%
\pgfpathclose%
\pgfusepath{fill}%
\end{pgfscope}%
\begin{pgfscope}%
\pgfpathrectangle{\pgfqpoint{0.765000in}{0.660000in}}{\pgfqpoint{4.620000in}{4.620000in}}%
\pgfusepath{clip}%
\pgfsetbuttcap%
\pgfsetroundjoin%
\definecolor{currentfill}{rgb}{1.000000,0.894118,0.788235}%
\pgfsetfillcolor{currentfill}%
\pgfsetlinewidth{0.000000pt}%
\definecolor{currentstroke}{rgb}{1.000000,0.894118,0.788235}%
\pgfsetstrokecolor{currentstroke}%
\pgfsetdash{}{0pt}%
\pgfpathmoveto{\pgfqpoint{3.033140in}{3.785492in}}%
\pgfpathlineto{\pgfqpoint{3.006193in}{3.801050in}}%
\pgfpathlineto{\pgfqpoint{2.981592in}{3.681982in}}%
\pgfpathlineto{\pgfqpoint{3.008539in}{3.666424in}}%
\pgfpathlineto{\pgfqpoint{3.033140in}{3.785492in}}%
\pgfpathclose%
\pgfusepath{fill}%
\end{pgfscope}%
\begin{pgfscope}%
\pgfpathrectangle{\pgfqpoint{0.765000in}{0.660000in}}{\pgfqpoint{4.620000in}{4.620000in}}%
\pgfusepath{clip}%
\pgfsetbuttcap%
\pgfsetroundjoin%
\definecolor{currentfill}{rgb}{1.000000,0.894118,0.788235}%
\pgfsetfillcolor{currentfill}%
\pgfsetlinewidth{0.000000pt}%
\definecolor{currentstroke}{rgb}{1.000000,0.894118,0.788235}%
\pgfsetstrokecolor{currentstroke}%
\pgfsetdash{}{0pt}%
\pgfpathmoveto{\pgfqpoint{2.979245in}{3.754376in}}%
\pgfpathlineto{\pgfqpoint{2.979245in}{3.785492in}}%
\pgfpathlineto{\pgfqpoint{2.954645in}{3.666424in}}%
\pgfpathlineto{\pgfqpoint{2.981592in}{3.619750in}}%
\pgfpathlineto{\pgfqpoint{2.979245in}{3.754376in}}%
\pgfpathclose%
\pgfusepath{fill}%
\end{pgfscope}%
\begin{pgfscope}%
\pgfpathrectangle{\pgfqpoint{0.765000in}{0.660000in}}{\pgfqpoint{4.620000in}{4.620000in}}%
\pgfusepath{clip}%
\pgfsetbuttcap%
\pgfsetroundjoin%
\definecolor{currentfill}{rgb}{1.000000,0.894118,0.788235}%
\pgfsetfillcolor{currentfill}%
\pgfsetlinewidth{0.000000pt}%
\definecolor{currentstroke}{rgb}{1.000000,0.894118,0.788235}%
\pgfsetstrokecolor{currentstroke}%
\pgfsetdash{}{0pt}%
\pgfpathmoveto{\pgfqpoint{3.006193in}{3.738818in}}%
\pgfpathlineto{\pgfqpoint{3.006193in}{3.769934in}}%
\pgfpathlineto{\pgfqpoint{2.981592in}{3.650866in}}%
\pgfpathlineto{\pgfqpoint{2.954645in}{3.635308in}}%
\pgfpathlineto{\pgfqpoint{3.006193in}{3.738818in}}%
\pgfpathclose%
\pgfusepath{fill}%
\end{pgfscope}%
\begin{pgfscope}%
\pgfpathrectangle{\pgfqpoint{0.765000in}{0.660000in}}{\pgfqpoint{4.620000in}{4.620000in}}%
\pgfusepath{clip}%
\pgfsetbuttcap%
\pgfsetroundjoin%
\definecolor{currentfill}{rgb}{1.000000,0.894118,0.788235}%
\pgfsetfillcolor{currentfill}%
\pgfsetlinewidth{0.000000pt}%
\definecolor{currentstroke}{rgb}{1.000000,0.894118,0.788235}%
\pgfsetstrokecolor{currentstroke}%
\pgfsetdash{}{0pt}%
\pgfpathmoveto{\pgfqpoint{3.006193in}{3.769934in}}%
\pgfpathlineto{\pgfqpoint{3.033140in}{3.785492in}}%
\pgfpathlineto{\pgfqpoint{3.008539in}{3.666424in}}%
\pgfpathlineto{\pgfqpoint{2.981592in}{3.650866in}}%
\pgfpathlineto{\pgfqpoint{3.006193in}{3.769934in}}%
\pgfpathclose%
\pgfusepath{fill}%
\end{pgfscope}%
\begin{pgfscope}%
\pgfpathrectangle{\pgfqpoint{0.765000in}{0.660000in}}{\pgfqpoint{4.620000in}{4.620000in}}%
\pgfusepath{clip}%
\pgfsetbuttcap%
\pgfsetroundjoin%
\definecolor{currentfill}{rgb}{1.000000,0.894118,0.788235}%
\pgfsetfillcolor{currentfill}%
\pgfsetlinewidth{0.000000pt}%
\definecolor{currentstroke}{rgb}{1.000000,0.894118,0.788235}%
\pgfsetstrokecolor{currentstroke}%
\pgfsetdash{}{0pt}%
\pgfpathmoveto{\pgfqpoint{2.979245in}{3.785492in}}%
\pgfpathlineto{\pgfqpoint{3.006193in}{3.769934in}}%
\pgfpathlineto{\pgfqpoint{2.981592in}{3.650866in}}%
\pgfpathlineto{\pgfqpoint{2.954645in}{3.666424in}}%
\pgfpathlineto{\pgfqpoint{2.979245in}{3.785492in}}%
\pgfpathclose%
\pgfusepath{fill}%
\end{pgfscope}%
\begin{pgfscope}%
\pgfpathrectangle{\pgfqpoint{0.765000in}{0.660000in}}{\pgfqpoint{4.620000in}{4.620000in}}%
\pgfusepath{clip}%
\pgfsetbuttcap%
\pgfsetroundjoin%
\definecolor{currentfill}{rgb}{1.000000,0.894118,0.788235}%
\pgfsetfillcolor{currentfill}%
\pgfsetlinewidth{0.000000pt}%
\definecolor{currentstroke}{rgb}{1.000000,0.894118,0.788235}%
\pgfsetstrokecolor{currentstroke}%
\pgfsetdash{}{0pt}%
\pgfpathmoveto{\pgfqpoint{2.973794in}{3.546621in}}%
\pgfpathlineto{\pgfqpoint{2.946847in}{3.531063in}}%
\pgfpathlineto{\pgfqpoint{2.973794in}{3.515505in}}%
\pgfpathlineto{\pgfqpoint{3.000742in}{3.531063in}}%
\pgfpathlineto{\pgfqpoint{2.973794in}{3.546621in}}%
\pgfpathclose%
\pgfusepath{fill}%
\end{pgfscope}%
\begin{pgfscope}%
\pgfpathrectangle{\pgfqpoint{0.765000in}{0.660000in}}{\pgfqpoint{4.620000in}{4.620000in}}%
\pgfusepath{clip}%
\pgfsetbuttcap%
\pgfsetroundjoin%
\definecolor{currentfill}{rgb}{1.000000,0.894118,0.788235}%
\pgfsetfillcolor{currentfill}%
\pgfsetlinewidth{0.000000pt}%
\definecolor{currentstroke}{rgb}{1.000000,0.894118,0.788235}%
\pgfsetstrokecolor{currentstroke}%
\pgfsetdash{}{0pt}%
\pgfpathmoveto{\pgfqpoint{2.973794in}{3.546621in}}%
\pgfpathlineto{\pgfqpoint{2.946847in}{3.531063in}}%
\pgfpathlineto{\pgfqpoint{2.946847in}{3.562179in}}%
\pgfpathlineto{\pgfqpoint{2.973794in}{3.577737in}}%
\pgfpathlineto{\pgfqpoint{2.973794in}{3.546621in}}%
\pgfpathclose%
\pgfusepath{fill}%
\end{pgfscope}%
\begin{pgfscope}%
\pgfpathrectangle{\pgfqpoint{0.765000in}{0.660000in}}{\pgfqpoint{4.620000in}{4.620000in}}%
\pgfusepath{clip}%
\pgfsetbuttcap%
\pgfsetroundjoin%
\definecolor{currentfill}{rgb}{1.000000,0.894118,0.788235}%
\pgfsetfillcolor{currentfill}%
\pgfsetlinewidth{0.000000pt}%
\definecolor{currentstroke}{rgb}{1.000000,0.894118,0.788235}%
\pgfsetstrokecolor{currentstroke}%
\pgfsetdash{}{0pt}%
\pgfpathmoveto{\pgfqpoint{2.973794in}{3.546621in}}%
\pgfpathlineto{\pgfqpoint{3.000742in}{3.531063in}}%
\pgfpathlineto{\pgfqpoint{3.000742in}{3.562179in}}%
\pgfpathlineto{\pgfqpoint{2.973794in}{3.577737in}}%
\pgfpathlineto{\pgfqpoint{2.973794in}{3.546621in}}%
\pgfpathclose%
\pgfusepath{fill}%
\end{pgfscope}%
\begin{pgfscope}%
\pgfpathrectangle{\pgfqpoint{0.765000in}{0.660000in}}{\pgfqpoint{4.620000in}{4.620000in}}%
\pgfusepath{clip}%
\pgfsetbuttcap%
\pgfsetroundjoin%
\definecolor{currentfill}{rgb}{1.000000,0.894118,0.788235}%
\pgfsetfillcolor{currentfill}%
\pgfsetlinewidth{0.000000pt}%
\definecolor{currentstroke}{rgb}{1.000000,0.894118,0.788235}%
\pgfsetstrokecolor{currentstroke}%
\pgfsetdash{}{0pt}%
\pgfpathmoveto{\pgfqpoint{2.748715in}{3.300819in}}%
\pgfpathlineto{\pgfqpoint{2.721768in}{3.285261in}}%
\pgfpathlineto{\pgfqpoint{2.748715in}{3.269703in}}%
\pgfpathlineto{\pgfqpoint{2.775662in}{3.285261in}}%
\pgfpathlineto{\pgfqpoint{2.748715in}{3.300819in}}%
\pgfpathclose%
\pgfusepath{fill}%
\end{pgfscope}%
\begin{pgfscope}%
\pgfpathrectangle{\pgfqpoint{0.765000in}{0.660000in}}{\pgfqpoint{4.620000in}{4.620000in}}%
\pgfusepath{clip}%
\pgfsetbuttcap%
\pgfsetroundjoin%
\definecolor{currentfill}{rgb}{1.000000,0.894118,0.788235}%
\pgfsetfillcolor{currentfill}%
\pgfsetlinewidth{0.000000pt}%
\definecolor{currentstroke}{rgb}{1.000000,0.894118,0.788235}%
\pgfsetstrokecolor{currentstroke}%
\pgfsetdash{}{0pt}%
\pgfpathmoveto{\pgfqpoint{2.748715in}{3.300819in}}%
\pgfpathlineto{\pgfqpoint{2.721768in}{3.285261in}}%
\pgfpathlineto{\pgfqpoint{2.721768in}{3.316377in}}%
\pgfpathlineto{\pgfqpoint{2.748715in}{3.331935in}}%
\pgfpathlineto{\pgfqpoint{2.748715in}{3.300819in}}%
\pgfpathclose%
\pgfusepath{fill}%
\end{pgfscope}%
\begin{pgfscope}%
\pgfpathrectangle{\pgfqpoint{0.765000in}{0.660000in}}{\pgfqpoint{4.620000in}{4.620000in}}%
\pgfusepath{clip}%
\pgfsetbuttcap%
\pgfsetroundjoin%
\definecolor{currentfill}{rgb}{1.000000,0.894118,0.788235}%
\pgfsetfillcolor{currentfill}%
\pgfsetlinewidth{0.000000pt}%
\definecolor{currentstroke}{rgb}{1.000000,0.894118,0.788235}%
\pgfsetstrokecolor{currentstroke}%
\pgfsetdash{}{0pt}%
\pgfpathmoveto{\pgfqpoint{2.748715in}{3.300819in}}%
\pgfpathlineto{\pgfqpoint{2.775662in}{3.285261in}}%
\pgfpathlineto{\pgfqpoint{2.775662in}{3.316377in}}%
\pgfpathlineto{\pgfqpoint{2.748715in}{3.331935in}}%
\pgfpathlineto{\pgfqpoint{2.748715in}{3.300819in}}%
\pgfpathclose%
\pgfusepath{fill}%
\end{pgfscope}%
\begin{pgfscope}%
\pgfpathrectangle{\pgfqpoint{0.765000in}{0.660000in}}{\pgfqpoint{4.620000in}{4.620000in}}%
\pgfusepath{clip}%
\pgfsetbuttcap%
\pgfsetroundjoin%
\definecolor{currentfill}{rgb}{1.000000,0.894118,0.788235}%
\pgfsetfillcolor{currentfill}%
\pgfsetlinewidth{0.000000pt}%
\definecolor{currentstroke}{rgb}{1.000000,0.894118,0.788235}%
\pgfsetstrokecolor{currentstroke}%
\pgfsetdash{}{0pt}%
\pgfpathmoveto{\pgfqpoint{2.973794in}{3.577737in}}%
\pgfpathlineto{\pgfqpoint{2.946847in}{3.562179in}}%
\pgfpathlineto{\pgfqpoint{2.973794in}{3.546621in}}%
\pgfpathlineto{\pgfqpoint{3.000742in}{3.562179in}}%
\pgfpathlineto{\pgfqpoint{2.973794in}{3.577737in}}%
\pgfpathclose%
\pgfusepath{fill}%
\end{pgfscope}%
\begin{pgfscope}%
\pgfpathrectangle{\pgfqpoint{0.765000in}{0.660000in}}{\pgfqpoint{4.620000in}{4.620000in}}%
\pgfusepath{clip}%
\pgfsetbuttcap%
\pgfsetroundjoin%
\definecolor{currentfill}{rgb}{1.000000,0.894118,0.788235}%
\pgfsetfillcolor{currentfill}%
\pgfsetlinewidth{0.000000pt}%
\definecolor{currentstroke}{rgb}{1.000000,0.894118,0.788235}%
\pgfsetstrokecolor{currentstroke}%
\pgfsetdash{}{0pt}%
\pgfpathmoveto{\pgfqpoint{2.973794in}{3.515505in}}%
\pgfpathlineto{\pgfqpoint{3.000742in}{3.531063in}}%
\pgfpathlineto{\pgfqpoint{3.000742in}{3.562179in}}%
\pgfpathlineto{\pgfqpoint{2.973794in}{3.546621in}}%
\pgfpathlineto{\pgfqpoint{2.973794in}{3.515505in}}%
\pgfpathclose%
\pgfusepath{fill}%
\end{pgfscope}%
\begin{pgfscope}%
\pgfpathrectangle{\pgfqpoint{0.765000in}{0.660000in}}{\pgfqpoint{4.620000in}{4.620000in}}%
\pgfusepath{clip}%
\pgfsetbuttcap%
\pgfsetroundjoin%
\definecolor{currentfill}{rgb}{1.000000,0.894118,0.788235}%
\pgfsetfillcolor{currentfill}%
\pgfsetlinewidth{0.000000pt}%
\definecolor{currentstroke}{rgb}{1.000000,0.894118,0.788235}%
\pgfsetstrokecolor{currentstroke}%
\pgfsetdash{}{0pt}%
\pgfpathmoveto{\pgfqpoint{2.946847in}{3.531063in}}%
\pgfpathlineto{\pgfqpoint{2.973794in}{3.515505in}}%
\pgfpathlineto{\pgfqpoint{2.973794in}{3.546621in}}%
\pgfpathlineto{\pgfqpoint{2.946847in}{3.562179in}}%
\pgfpathlineto{\pgfqpoint{2.946847in}{3.531063in}}%
\pgfpathclose%
\pgfusepath{fill}%
\end{pgfscope}%
\begin{pgfscope}%
\pgfpathrectangle{\pgfqpoint{0.765000in}{0.660000in}}{\pgfqpoint{4.620000in}{4.620000in}}%
\pgfusepath{clip}%
\pgfsetbuttcap%
\pgfsetroundjoin%
\definecolor{currentfill}{rgb}{1.000000,0.894118,0.788235}%
\pgfsetfillcolor{currentfill}%
\pgfsetlinewidth{0.000000pt}%
\definecolor{currentstroke}{rgb}{1.000000,0.894118,0.788235}%
\pgfsetstrokecolor{currentstroke}%
\pgfsetdash{}{0pt}%
\pgfpathmoveto{\pgfqpoint{2.748715in}{3.331935in}}%
\pgfpathlineto{\pgfqpoint{2.721768in}{3.316377in}}%
\pgfpathlineto{\pgfqpoint{2.748715in}{3.300819in}}%
\pgfpathlineto{\pgfqpoint{2.775662in}{3.316377in}}%
\pgfpathlineto{\pgfqpoint{2.748715in}{3.331935in}}%
\pgfpathclose%
\pgfusepath{fill}%
\end{pgfscope}%
\begin{pgfscope}%
\pgfpathrectangle{\pgfqpoint{0.765000in}{0.660000in}}{\pgfqpoint{4.620000in}{4.620000in}}%
\pgfusepath{clip}%
\pgfsetbuttcap%
\pgfsetroundjoin%
\definecolor{currentfill}{rgb}{1.000000,0.894118,0.788235}%
\pgfsetfillcolor{currentfill}%
\pgfsetlinewidth{0.000000pt}%
\definecolor{currentstroke}{rgb}{1.000000,0.894118,0.788235}%
\pgfsetstrokecolor{currentstroke}%
\pgfsetdash{}{0pt}%
\pgfpathmoveto{\pgfqpoint{2.748715in}{3.269703in}}%
\pgfpathlineto{\pgfqpoint{2.775662in}{3.285261in}}%
\pgfpathlineto{\pgfqpoint{2.775662in}{3.316377in}}%
\pgfpathlineto{\pgfqpoint{2.748715in}{3.300819in}}%
\pgfpathlineto{\pgfqpoint{2.748715in}{3.269703in}}%
\pgfpathclose%
\pgfusepath{fill}%
\end{pgfscope}%
\begin{pgfscope}%
\pgfpathrectangle{\pgfqpoint{0.765000in}{0.660000in}}{\pgfqpoint{4.620000in}{4.620000in}}%
\pgfusepath{clip}%
\pgfsetbuttcap%
\pgfsetroundjoin%
\definecolor{currentfill}{rgb}{1.000000,0.894118,0.788235}%
\pgfsetfillcolor{currentfill}%
\pgfsetlinewidth{0.000000pt}%
\definecolor{currentstroke}{rgb}{1.000000,0.894118,0.788235}%
\pgfsetstrokecolor{currentstroke}%
\pgfsetdash{}{0pt}%
\pgfpathmoveto{\pgfqpoint{2.721768in}{3.285261in}}%
\pgfpathlineto{\pgfqpoint{2.748715in}{3.269703in}}%
\pgfpathlineto{\pgfqpoint{2.748715in}{3.300819in}}%
\pgfpathlineto{\pgfqpoint{2.721768in}{3.316377in}}%
\pgfpathlineto{\pgfqpoint{2.721768in}{3.285261in}}%
\pgfpathclose%
\pgfusepath{fill}%
\end{pgfscope}%
\begin{pgfscope}%
\pgfpathrectangle{\pgfqpoint{0.765000in}{0.660000in}}{\pgfqpoint{4.620000in}{4.620000in}}%
\pgfusepath{clip}%
\pgfsetbuttcap%
\pgfsetroundjoin%
\definecolor{currentfill}{rgb}{1.000000,0.894118,0.788235}%
\pgfsetfillcolor{currentfill}%
\pgfsetlinewidth{0.000000pt}%
\definecolor{currentstroke}{rgb}{1.000000,0.894118,0.788235}%
\pgfsetstrokecolor{currentstroke}%
\pgfsetdash{}{0pt}%
\pgfpathmoveto{\pgfqpoint{2.973794in}{3.546621in}}%
\pgfpathlineto{\pgfqpoint{2.946847in}{3.531063in}}%
\pgfpathlineto{\pgfqpoint{2.721768in}{3.285261in}}%
\pgfpathlineto{\pgfqpoint{2.748715in}{3.300819in}}%
\pgfpathlineto{\pgfqpoint{2.973794in}{3.546621in}}%
\pgfpathclose%
\pgfusepath{fill}%
\end{pgfscope}%
\begin{pgfscope}%
\pgfpathrectangle{\pgfqpoint{0.765000in}{0.660000in}}{\pgfqpoint{4.620000in}{4.620000in}}%
\pgfusepath{clip}%
\pgfsetbuttcap%
\pgfsetroundjoin%
\definecolor{currentfill}{rgb}{1.000000,0.894118,0.788235}%
\pgfsetfillcolor{currentfill}%
\pgfsetlinewidth{0.000000pt}%
\definecolor{currentstroke}{rgb}{1.000000,0.894118,0.788235}%
\pgfsetstrokecolor{currentstroke}%
\pgfsetdash{}{0pt}%
\pgfpathmoveto{\pgfqpoint{3.000742in}{3.531063in}}%
\pgfpathlineto{\pgfqpoint{2.973794in}{3.546621in}}%
\pgfpathlineto{\pgfqpoint{2.748715in}{3.300819in}}%
\pgfpathlineto{\pgfqpoint{2.775662in}{3.285261in}}%
\pgfpathlineto{\pgfqpoint{3.000742in}{3.531063in}}%
\pgfpathclose%
\pgfusepath{fill}%
\end{pgfscope}%
\begin{pgfscope}%
\pgfpathrectangle{\pgfqpoint{0.765000in}{0.660000in}}{\pgfqpoint{4.620000in}{4.620000in}}%
\pgfusepath{clip}%
\pgfsetbuttcap%
\pgfsetroundjoin%
\definecolor{currentfill}{rgb}{1.000000,0.894118,0.788235}%
\pgfsetfillcolor{currentfill}%
\pgfsetlinewidth{0.000000pt}%
\definecolor{currentstroke}{rgb}{1.000000,0.894118,0.788235}%
\pgfsetstrokecolor{currentstroke}%
\pgfsetdash{}{0pt}%
\pgfpathmoveto{\pgfqpoint{2.973794in}{3.546621in}}%
\pgfpathlineto{\pgfqpoint{2.973794in}{3.577737in}}%
\pgfpathlineto{\pgfqpoint{2.748715in}{3.331935in}}%
\pgfpathlineto{\pgfqpoint{2.775662in}{3.285261in}}%
\pgfpathlineto{\pgfqpoint{2.973794in}{3.546621in}}%
\pgfpathclose%
\pgfusepath{fill}%
\end{pgfscope}%
\begin{pgfscope}%
\pgfpathrectangle{\pgfqpoint{0.765000in}{0.660000in}}{\pgfqpoint{4.620000in}{4.620000in}}%
\pgfusepath{clip}%
\pgfsetbuttcap%
\pgfsetroundjoin%
\definecolor{currentfill}{rgb}{1.000000,0.894118,0.788235}%
\pgfsetfillcolor{currentfill}%
\pgfsetlinewidth{0.000000pt}%
\definecolor{currentstroke}{rgb}{1.000000,0.894118,0.788235}%
\pgfsetstrokecolor{currentstroke}%
\pgfsetdash{}{0pt}%
\pgfpathmoveto{\pgfqpoint{3.000742in}{3.531063in}}%
\pgfpathlineto{\pgfqpoint{3.000742in}{3.562179in}}%
\pgfpathlineto{\pgfqpoint{2.775662in}{3.316377in}}%
\pgfpathlineto{\pgfqpoint{2.748715in}{3.300819in}}%
\pgfpathlineto{\pgfqpoint{3.000742in}{3.531063in}}%
\pgfpathclose%
\pgfusepath{fill}%
\end{pgfscope}%
\begin{pgfscope}%
\pgfpathrectangle{\pgfqpoint{0.765000in}{0.660000in}}{\pgfqpoint{4.620000in}{4.620000in}}%
\pgfusepath{clip}%
\pgfsetbuttcap%
\pgfsetroundjoin%
\definecolor{currentfill}{rgb}{1.000000,0.894118,0.788235}%
\pgfsetfillcolor{currentfill}%
\pgfsetlinewidth{0.000000pt}%
\definecolor{currentstroke}{rgb}{1.000000,0.894118,0.788235}%
\pgfsetstrokecolor{currentstroke}%
\pgfsetdash{}{0pt}%
\pgfpathmoveto{\pgfqpoint{2.946847in}{3.531063in}}%
\pgfpathlineto{\pgfqpoint{2.973794in}{3.515505in}}%
\pgfpathlineto{\pgfqpoint{2.748715in}{3.269703in}}%
\pgfpathlineto{\pgfqpoint{2.721768in}{3.285261in}}%
\pgfpathlineto{\pgfqpoint{2.946847in}{3.531063in}}%
\pgfpathclose%
\pgfusepath{fill}%
\end{pgfscope}%
\begin{pgfscope}%
\pgfpathrectangle{\pgfqpoint{0.765000in}{0.660000in}}{\pgfqpoint{4.620000in}{4.620000in}}%
\pgfusepath{clip}%
\pgfsetbuttcap%
\pgfsetroundjoin%
\definecolor{currentfill}{rgb}{1.000000,0.894118,0.788235}%
\pgfsetfillcolor{currentfill}%
\pgfsetlinewidth{0.000000pt}%
\definecolor{currentstroke}{rgb}{1.000000,0.894118,0.788235}%
\pgfsetstrokecolor{currentstroke}%
\pgfsetdash{}{0pt}%
\pgfpathmoveto{\pgfqpoint{2.973794in}{3.515505in}}%
\pgfpathlineto{\pgfqpoint{3.000742in}{3.531063in}}%
\pgfpathlineto{\pgfqpoint{2.775662in}{3.285261in}}%
\pgfpathlineto{\pgfqpoint{2.748715in}{3.269703in}}%
\pgfpathlineto{\pgfqpoint{2.973794in}{3.515505in}}%
\pgfpathclose%
\pgfusepath{fill}%
\end{pgfscope}%
\begin{pgfscope}%
\pgfpathrectangle{\pgfqpoint{0.765000in}{0.660000in}}{\pgfqpoint{4.620000in}{4.620000in}}%
\pgfusepath{clip}%
\pgfsetbuttcap%
\pgfsetroundjoin%
\definecolor{currentfill}{rgb}{1.000000,0.894118,0.788235}%
\pgfsetfillcolor{currentfill}%
\pgfsetlinewidth{0.000000pt}%
\definecolor{currentstroke}{rgb}{1.000000,0.894118,0.788235}%
\pgfsetstrokecolor{currentstroke}%
\pgfsetdash{}{0pt}%
\pgfpathmoveto{\pgfqpoint{2.973794in}{3.577737in}}%
\pgfpathlineto{\pgfqpoint{2.946847in}{3.562179in}}%
\pgfpathlineto{\pgfqpoint{2.721768in}{3.316377in}}%
\pgfpathlineto{\pgfqpoint{2.748715in}{3.331935in}}%
\pgfpathlineto{\pgfqpoint{2.973794in}{3.577737in}}%
\pgfpathclose%
\pgfusepath{fill}%
\end{pgfscope}%
\begin{pgfscope}%
\pgfpathrectangle{\pgfqpoint{0.765000in}{0.660000in}}{\pgfqpoint{4.620000in}{4.620000in}}%
\pgfusepath{clip}%
\pgfsetbuttcap%
\pgfsetroundjoin%
\definecolor{currentfill}{rgb}{1.000000,0.894118,0.788235}%
\pgfsetfillcolor{currentfill}%
\pgfsetlinewidth{0.000000pt}%
\definecolor{currentstroke}{rgb}{1.000000,0.894118,0.788235}%
\pgfsetstrokecolor{currentstroke}%
\pgfsetdash{}{0pt}%
\pgfpathmoveto{\pgfqpoint{3.000742in}{3.562179in}}%
\pgfpathlineto{\pgfqpoint{2.973794in}{3.577737in}}%
\pgfpathlineto{\pgfqpoint{2.748715in}{3.331935in}}%
\pgfpathlineto{\pgfqpoint{2.775662in}{3.316377in}}%
\pgfpathlineto{\pgfqpoint{3.000742in}{3.562179in}}%
\pgfpathclose%
\pgfusepath{fill}%
\end{pgfscope}%
\begin{pgfscope}%
\pgfpathrectangle{\pgfqpoint{0.765000in}{0.660000in}}{\pgfqpoint{4.620000in}{4.620000in}}%
\pgfusepath{clip}%
\pgfsetbuttcap%
\pgfsetroundjoin%
\definecolor{currentfill}{rgb}{1.000000,0.894118,0.788235}%
\pgfsetfillcolor{currentfill}%
\pgfsetlinewidth{0.000000pt}%
\definecolor{currentstroke}{rgb}{1.000000,0.894118,0.788235}%
\pgfsetstrokecolor{currentstroke}%
\pgfsetdash{}{0pt}%
\pgfpathmoveto{\pgfqpoint{2.946847in}{3.531063in}}%
\pgfpathlineto{\pgfqpoint{2.946847in}{3.562179in}}%
\pgfpathlineto{\pgfqpoint{2.721768in}{3.316377in}}%
\pgfpathlineto{\pgfqpoint{2.748715in}{3.269703in}}%
\pgfpathlineto{\pgfqpoint{2.946847in}{3.531063in}}%
\pgfpathclose%
\pgfusepath{fill}%
\end{pgfscope}%
\begin{pgfscope}%
\pgfpathrectangle{\pgfqpoint{0.765000in}{0.660000in}}{\pgfqpoint{4.620000in}{4.620000in}}%
\pgfusepath{clip}%
\pgfsetbuttcap%
\pgfsetroundjoin%
\definecolor{currentfill}{rgb}{1.000000,0.894118,0.788235}%
\pgfsetfillcolor{currentfill}%
\pgfsetlinewidth{0.000000pt}%
\definecolor{currentstroke}{rgb}{1.000000,0.894118,0.788235}%
\pgfsetstrokecolor{currentstroke}%
\pgfsetdash{}{0pt}%
\pgfpathmoveto{\pgfqpoint{2.973794in}{3.515505in}}%
\pgfpathlineto{\pgfqpoint{2.973794in}{3.546621in}}%
\pgfpathlineto{\pgfqpoint{2.748715in}{3.300819in}}%
\pgfpathlineto{\pgfqpoint{2.721768in}{3.285261in}}%
\pgfpathlineto{\pgfqpoint{2.973794in}{3.515505in}}%
\pgfpathclose%
\pgfusepath{fill}%
\end{pgfscope}%
\begin{pgfscope}%
\pgfpathrectangle{\pgfqpoint{0.765000in}{0.660000in}}{\pgfqpoint{4.620000in}{4.620000in}}%
\pgfusepath{clip}%
\pgfsetbuttcap%
\pgfsetroundjoin%
\definecolor{currentfill}{rgb}{1.000000,0.894118,0.788235}%
\pgfsetfillcolor{currentfill}%
\pgfsetlinewidth{0.000000pt}%
\definecolor{currentstroke}{rgb}{1.000000,0.894118,0.788235}%
\pgfsetstrokecolor{currentstroke}%
\pgfsetdash{}{0pt}%
\pgfpathmoveto{\pgfqpoint{2.973794in}{3.546621in}}%
\pgfpathlineto{\pgfqpoint{3.000742in}{3.562179in}}%
\pgfpathlineto{\pgfqpoint{2.775662in}{3.316377in}}%
\pgfpathlineto{\pgfqpoint{2.748715in}{3.300819in}}%
\pgfpathlineto{\pgfqpoint{2.973794in}{3.546621in}}%
\pgfpathclose%
\pgfusepath{fill}%
\end{pgfscope}%
\begin{pgfscope}%
\pgfpathrectangle{\pgfqpoint{0.765000in}{0.660000in}}{\pgfqpoint{4.620000in}{4.620000in}}%
\pgfusepath{clip}%
\pgfsetbuttcap%
\pgfsetroundjoin%
\definecolor{currentfill}{rgb}{1.000000,0.894118,0.788235}%
\pgfsetfillcolor{currentfill}%
\pgfsetlinewidth{0.000000pt}%
\definecolor{currentstroke}{rgb}{1.000000,0.894118,0.788235}%
\pgfsetstrokecolor{currentstroke}%
\pgfsetdash{}{0pt}%
\pgfpathmoveto{\pgfqpoint{2.946847in}{3.562179in}}%
\pgfpathlineto{\pgfqpoint{2.973794in}{3.546621in}}%
\pgfpathlineto{\pgfqpoint{2.748715in}{3.300819in}}%
\pgfpathlineto{\pgfqpoint{2.721768in}{3.316377in}}%
\pgfpathlineto{\pgfqpoint{2.946847in}{3.562179in}}%
\pgfpathclose%
\pgfusepath{fill}%
\end{pgfscope}%
\begin{pgfscope}%
\pgfpathrectangle{\pgfqpoint{0.765000in}{0.660000in}}{\pgfqpoint{4.620000in}{4.620000in}}%
\pgfusepath{clip}%
\pgfsetbuttcap%
\pgfsetroundjoin%
\definecolor{currentfill}{rgb}{1.000000,0.894118,0.788235}%
\pgfsetfillcolor{currentfill}%
\pgfsetlinewidth{0.000000pt}%
\definecolor{currentstroke}{rgb}{1.000000,0.894118,0.788235}%
\pgfsetstrokecolor{currentstroke}%
\pgfsetdash{}{0pt}%
\pgfpathmoveto{\pgfqpoint{2.748715in}{3.300819in}}%
\pgfpathlineto{\pgfqpoint{2.721768in}{3.285261in}}%
\pgfpathlineto{\pgfqpoint{2.748715in}{3.269703in}}%
\pgfpathlineto{\pgfqpoint{2.775662in}{3.285261in}}%
\pgfpathlineto{\pgfqpoint{2.748715in}{3.300819in}}%
\pgfpathclose%
\pgfusepath{fill}%
\end{pgfscope}%
\begin{pgfscope}%
\pgfpathrectangle{\pgfqpoint{0.765000in}{0.660000in}}{\pgfqpoint{4.620000in}{4.620000in}}%
\pgfusepath{clip}%
\pgfsetbuttcap%
\pgfsetroundjoin%
\definecolor{currentfill}{rgb}{1.000000,0.894118,0.788235}%
\pgfsetfillcolor{currentfill}%
\pgfsetlinewidth{0.000000pt}%
\definecolor{currentstroke}{rgb}{1.000000,0.894118,0.788235}%
\pgfsetstrokecolor{currentstroke}%
\pgfsetdash{}{0pt}%
\pgfpathmoveto{\pgfqpoint{2.748715in}{3.300819in}}%
\pgfpathlineto{\pgfqpoint{2.721768in}{3.285261in}}%
\pgfpathlineto{\pgfqpoint{2.721768in}{3.316377in}}%
\pgfpathlineto{\pgfqpoint{2.748715in}{3.331935in}}%
\pgfpathlineto{\pgfqpoint{2.748715in}{3.300819in}}%
\pgfpathclose%
\pgfusepath{fill}%
\end{pgfscope}%
\begin{pgfscope}%
\pgfpathrectangle{\pgfqpoint{0.765000in}{0.660000in}}{\pgfqpoint{4.620000in}{4.620000in}}%
\pgfusepath{clip}%
\pgfsetbuttcap%
\pgfsetroundjoin%
\definecolor{currentfill}{rgb}{1.000000,0.894118,0.788235}%
\pgfsetfillcolor{currentfill}%
\pgfsetlinewidth{0.000000pt}%
\definecolor{currentstroke}{rgb}{1.000000,0.894118,0.788235}%
\pgfsetstrokecolor{currentstroke}%
\pgfsetdash{}{0pt}%
\pgfpathmoveto{\pgfqpoint{2.748715in}{3.300819in}}%
\pgfpathlineto{\pgfqpoint{2.775662in}{3.285261in}}%
\pgfpathlineto{\pgfqpoint{2.775662in}{3.316377in}}%
\pgfpathlineto{\pgfqpoint{2.748715in}{3.331935in}}%
\pgfpathlineto{\pgfqpoint{2.748715in}{3.300819in}}%
\pgfpathclose%
\pgfusepath{fill}%
\end{pgfscope}%
\begin{pgfscope}%
\pgfpathrectangle{\pgfqpoint{0.765000in}{0.660000in}}{\pgfqpoint{4.620000in}{4.620000in}}%
\pgfusepath{clip}%
\pgfsetbuttcap%
\pgfsetroundjoin%
\definecolor{currentfill}{rgb}{1.000000,0.894118,0.788235}%
\pgfsetfillcolor{currentfill}%
\pgfsetlinewidth{0.000000pt}%
\definecolor{currentstroke}{rgb}{1.000000,0.894118,0.788235}%
\pgfsetstrokecolor{currentstroke}%
\pgfsetdash{}{0pt}%
\pgfpathmoveto{\pgfqpoint{2.973794in}{3.546621in}}%
\pgfpathlineto{\pgfqpoint{2.946847in}{3.531063in}}%
\pgfpathlineto{\pgfqpoint{2.973794in}{3.515505in}}%
\pgfpathlineto{\pgfqpoint{3.000742in}{3.531063in}}%
\pgfpathlineto{\pgfqpoint{2.973794in}{3.546621in}}%
\pgfpathclose%
\pgfusepath{fill}%
\end{pgfscope}%
\begin{pgfscope}%
\pgfpathrectangle{\pgfqpoint{0.765000in}{0.660000in}}{\pgfqpoint{4.620000in}{4.620000in}}%
\pgfusepath{clip}%
\pgfsetbuttcap%
\pgfsetroundjoin%
\definecolor{currentfill}{rgb}{1.000000,0.894118,0.788235}%
\pgfsetfillcolor{currentfill}%
\pgfsetlinewidth{0.000000pt}%
\definecolor{currentstroke}{rgb}{1.000000,0.894118,0.788235}%
\pgfsetstrokecolor{currentstroke}%
\pgfsetdash{}{0pt}%
\pgfpathmoveto{\pgfqpoint{2.973794in}{3.546621in}}%
\pgfpathlineto{\pgfqpoint{2.946847in}{3.531063in}}%
\pgfpathlineto{\pgfqpoint{2.946847in}{3.562179in}}%
\pgfpathlineto{\pgfqpoint{2.973794in}{3.577737in}}%
\pgfpathlineto{\pgfqpoint{2.973794in}{3.546621in}}%
\pgfpathclose%
\pgfusepath{fill}%
\end{pgfscope}%
\begin{pgfscope}%
\pgfpathrectangle{\pgfqpoint{0.765000in}{0.660000in}}{\pgfqpoint{4.620000in}{4.620000in}}%
\pgfusepath{clip}%
\pgfsetbuttcap%
\pgfsetroundjoin%
\definecolor{currentfill}{rgb}{1.000000,0.894118,0.788235}%
\pgfsetfillcolor{currentfill}%
\pgfsetlinewidth{0.000000pt}%
\definecolor{currentstroke}{rgb}{1.000000,0.894118,0.788235}%
\pgfsetstrokecolor{currentstroke}%
\pgfsetdash{}{0pt}%
\pgfpathmoveto{\pgfqpoint{2.973794in}{3.546621in}}%
\pgfpathlineto{\pgfqpoint{3.000742in}{3.531063in}}%
\pgfpathlineto{\pgfqpoint{3.000742in}{3.562179in}}%
\pgfpathlineto{\pgfqpoint{2.973794in}{3.577737in}}%
\pgfpathlineto{\pgfqpoint{2.973794in}{3.546621in}}%
\pgfpathclose%
\pgfusepath{fill}%
\end{pgfscope}%
\begin{pgfscope}%
\pgfpathrectangle{\pgfqpoint{0.765000in}{0.660000in}}{\pgfqpoint{4.620000in}{4.620000in}}%
\pgfusepath{clip}%
\pgfsetbuttcap%
\pgfsetroundjoin%
\definecolor{currentfill}{rgb}{1.000000,0.894118,0.788235}%
\pgfsetfillcolor{currentfill}%
\pgfsetlinewidth{0.000000pt}%
\definecolor{currentstroke}{rgb}{1.000000,0.894118,0.788235}%
\pgfsetstrokecolor{currentstroke}%
\pgfsetdash{}{0pt}%
\pgfpathmoveto{\pgfqpoint{2.748715in}{3.331935in}}%
\pgfpathlineto{\pgfqpoint{2.721768in}{3.316377in}}%
\pgfpathlineto{\pgfqpoint{2.748715in}{3.300819in}}%
\pgfpathlineto{\pgfqpoint{2.775662in}{3.316377in}}%
\pgfpathlineto{\pgfqpoint{2.748715in}{3.331935in}}%
\pgfpathclose%
\pgfusepath{fill}%
\end{pgfscope}%
\begin{pgfscope}%
\pgfpathrectangle{\pgfqpoint{0.765000in}{0.660000in}}{\pgfqpoint{4.620000in}{4.620000in}}%
\pgfusepath{clip}%
\pgfsetbuttcap%
\pgfsetroundjoin%
\definecolor{currentfill}{rgb}{1.000000,0.894118,0.788235}%
\pgfsetfillcolor{currentfill}%
\pgfsetlinewidth{0.000000pt}%
\definecolor{currentstroke}{rgb}{1.000000,0.894118,0.788235}%
\pgfsetstrokecolor{currentstroke}%
\pgfsetdash{}{0pt}%
\pgfpathmoveto{\pgfqpoint{2.748715in}{3.269703in}}%
\pgfpathlineto{\pgfqpoint{2.775662in}{3.285261in}}%
\pgfpathlineto{\pgfqpoint{2.775662in}{3.316377in}}%
\pgfpathlineto{\pgfqpoint{2.748715in}{3.300819in}}%
\pgfpathlineto{\pgfqpoint{2.748715in}{3.269703in}}%
\pgfpathclose%
\pgfusepath{fill}%
\end{pgfscope}%
\begin{pgfscope}%
\pgfpathrectangle{\pgfqpoint{0.765000in}{0.660000in}}{\pgfqpoint{4.620000in}{4.620000in}}%
\pgfusepath{clip}%
\pgfsetbuttcap%
\pgfsetroundjoin%
\definecolor{currentfill}{rgb}{1.000000,0.894118,0.788235}%
\pgfsetfillcolor{currentfill}%
\pgfsetlinewidth{0.000000pt}%
\definecolor{currentstroke}{rgb}{1.000000,0.894118,0.788235}%
\pgfsetstrokecolor{currentstroke}%
\pgfsetdash{}{0pt}%
\pgfpathmoveto{\pgfqpoint{2.721768in}{3.285261in}}%
\pgfpathlineto{\pgfqpoint{2.748715in}{3.269703in}}%
\pgfpathlineto{\pgfqpoint{2.748715in}{3.300819in}}%
\pgfpathlineto{\pgfqpoint{2.721768in}{3.316377in}}%
\pgfpathlineto{\pgfqpoint{2.721768in}{3.285261in}}%
\pgfpathclose%
\pgfusepath{fill}%
\end{pgfscope}%
\begin{pgfscope}%
\pgfpathrectangle{\pgfqpoint{0.765000in}{0.660000in}}{\pgfqpoint{4.620000in}{4.620000in}}%
\pgfusepath{clip}%
\pgfsetbuttcap%
\pgfsetroundjoin%
\definecolor{currentfill}{rgb}{1.000000,0.894118,0.788235}%
\pgfsetfillcolor{currentfill}%
\pgfsetlinewidth{0.000000pt}%
\definecolor{currentstroke}{rgb}{1.000000,0.894118,0.788235}%
\pgfsetstrokecolor{currentstroke}%
\pgfsetdash{}{0pt}%
\pgfpathmoveto{\pgfqpoint{2.973794in}{3.577737in}}%
\pgfpathlineto{\pgfqpoint{2.946847in}{3.562179in}}%
\pgfpathlineto{\pgfqpoint{2.973794in}{3.546621in}}%
\pgfpathlineto{\pgfqpoint{3.000742in}{3.562179in}}%
\pgfpathlineto{\pgfqpoint{2.973794in}{3.577737in}}%
\pgfpathclose%
\pgfusepath{fill}%
\end{pgfscope}%
\begin{pgfscope}%
\pgfpathrectangle{\pgfqpoint{0.765000in}{0.660000in}}{\pgfqpoint{4.620000in}{4.620000in}}%
\pgfusepath{clip}%
\pgfsetbuttcap%
\pgfsetroundjoin%
\definecolor{currentfill}{rgb}{1.000000,0.894118,0.788235}%
\pgfsetfillcolor{currentfill}%
\pgfsetlinewidth{0.000000pt}%
\definecolor{currentstroke}{rgb}{1.000000,0.894118,0.788235}%
\pgfsetstrokecolor{currentstroke}%
\pgfsetdash{}{0pt}%
\pgfpathmoveto{\pgfqpoint{2.973794in}{3.515505in}}%
\pgfpathlineto{\pgfqpoint{3.000742in}{3.531063in}}%
\pgfpathlineto{\pgfqpoint{3.000742in}{3.562179in}}%
\pgfpathlineto{\pgfqpoint{2.973794in}{3.546621in}}%
\pgfpathlineto{\pgfqpoint{2.973794in}{3.515505in}}%
\pgfpathclose%
\pgfusepath{fill}%
\end{pgfscope}%
\begin{pgfscope}%
\pgfpathrectangle{\pgfqpoint{0.765000in}{0.660000in}}{\pgfqpoint{4.620000in}{4.620000in}}%
\pgfusepath{clip}%
\pgfsetbuttcap%
\pgfsetroundjoin%
\definecolor{currentfill}{rgb}{1.000000,0.894118,0.788235}%
\pgfsetfillcolor{currentfill}%
\pgfsetlinewidth{0.000000pt}%
\definecolor{currentstroke}{rgb}{1.000000,0.894118,0.788235}%
\pgfsetstrokecolor{currentstroke}%
\pgfsetdash{}{0pt}%
\pgfpathmoveto{\pgfqpoint{2.946847in}{3.531063in}}%
\pgfpathlineto{\pgfqpoint{2.973794in}{3.515505in}}%
\pgfpathlineto{\pgfqpoint{2.973794in}{3.546621in}}%
\pgfpathlineto{\pgfqpoint{2.946847in}{3.562179in}}%
\pgfpathlineto{\pgfqpoint{2.946847in}{3.531063in}}%
\pgfpathclose%
\pgfusepath{fill}%
\end{pgfscope}%
\begin{pgfscope}%
\pgfpathrectangle{\pgfqpoint{0.765000in}{0.660000in}}{\pgfqpoint{4.620000in}{4.620000in}}%
\pgfusepath{clip}%
\pgfsetbuttcap%
\pgfsetroundjoin%
\definecolor{currentfill}{rgb}{1.000000,0.894118,0.788235}%
\pgfsetfillcolor{currentfill}%
\pgfsetlinewidth{0.000000pt}%
\definecolor{currentstroke}{rgb}{1.000000,0.894118,0.788235}%
\pgfsetstrokecolor{currentstroke}%
\pgfsetdash{}{0pt}%
\pgfpathmoveto{\pgfqpoint{2.748715in}{3.300819in}}%
\pgfpathlineto{\pgfqpoint{2.721768in}{3.285261in}}%
\pgfpathlineto{\pgfqpoint{2.946847in}{3.531063in}}%
\pgfpathlineto{\pgfqpoint{2.973794in}{3.546621in}}%
\pgfpathlineto{\pgfqpoint{2.748715in}{3.300819in}}%
\pgfpathclose%
\pgfusepath{fill}%
\end{pgfscope}%
\begin{pgfscope}%
\pgfpathrectangle{\pgfqpoint{0.765000in}{0.660000in}}{\pgfqpoint{4.620000in}{4.620000in}}%
\pgfusepath{clip}%
\pgfsetbuttcap%
\pgfsetroundjoin%
\definecolor{currentfill}{rgb}{1.000000,0.894118,0.788235}%
\pgfsetfillcolor{currentfill}%
\pgfsetlinewidth{0.000000pt}%
\definecolor{currentstroke}{rgb}{1.000000,0.894118,0.788235}%
\pgfsetstrokecolor{currentstroke}%
\pgfsetdash{}{0pt}%
\pgfpathmoveto{\pgfqpoint{2.775662in}{3.285261in}}%
\pgfpathlineto{\pgfqpoint{2.748715in}{3.300819in}}%
\pgfpathlineto{\pgfqpoint{2.973794in}{3.546621in}}%
\pgfpathlineto{\pgfqpoint{3.000742in}{3.531063in}}%
\pgfpathlineto{\pgfqpoint{2.775662in}{3.285261in}}%
\pgfpathclose%
\pgfusepath{fill}%
\end{pgfscope}%
\begin{pgfscope}%
\pgfpathrectangle{\pgfqpoint{0.765000in}{0.660000in}}{\pgfqpoint{4.620000in}{4.620000in}}%
\pgfusepath{clip}%
\pgfsetbuttcap%
\pgfsetroundjoin%
\definecolor{currentfill}{rgb}{1.000000,0.894118,0.788235}%
\pgfsetfillcolor{currentfill}%
\pgfsetlinewidth{0.000000pt}%
\definecolor{currentstroke}{rgb}{1.000000,0.894118,0.788235}%
\pgfsetstrokecolor{currentstroke}%
\pgfsetdash{}{0pt}%
\pgfpathmoveto{\pgfqpoint{2.748715in}{3.300819in}}%
\pgfpathlineto{\pgfqpoint{2.748715in}{3.331935in}}%
\pgfpathlineto{\pgfqpoint{2.973794in}{3.577737in}}%
\pgfpathlineto{\pgfqpoint{3.000742in}{3.531063in}}%
\pgfpathlineto{\pgfqpoint{2.748715in}{3.300819in}}%
\pgfpathclose%
\pgfusepath{fill}%
\end{pgfscope}%
\begin{pgfscope}%
\pgfpathrectangle{\pgfqpoint{0.765000in}{0.660000in}}{\pgfqpoint{4.620000in}{4.620000in}}%
\pgfusepath{clip}%
\pgfsetbuttcap%
\pgfsetroundjoin%
\definecolor{currentfill}{rgb}{1.000000,0.894118,0.788235}%
\pgfsetfillcolor{currentfill}%
\pgfsetlinewidth{0.000000pt}%
\definecolor{currentstroke}{rgb}{1.000000,0.894118,0.788235}%
\pgfsetstrokecolor{currentstroke}%
\pgfsetdash{}{0pt}%
\pgfpathmoveto{\pgfqpoint{2.775662in}{3.285261in}}%
\pgfpathlineto{\pgfqpoint{2.775662in}{3.316377in}}%
\pgfpathlineto{\pgfqpoint{3.000742in}{3.562179in}}%
\pgfpathlineto{\pgfqpoint{2.973794in}{3.546621in}}%
\pgfpathlineto{\pgfqpoint{2.775662in}{3.285261in}}%
\pgfpathclose%
\pgfusepath{fill}%
\end{pgfscope}%
\begin{pgfscope}%
\pgfpathrectangle{\pgfqpoint{0.765000in}{0.660000in}}{\pgfqpoint{4.620000in}{4.620000in}}%
\pgfusepath{clip}%
\pgfsetbuttcap%
\pgfsetroundjoin%
\definecolor{currentfill}{rgb}{1.000000,0.894118,0.788235}%
\pgfsetfillcolor{currentfill}%
\pgfsetlinewidth{0.000000pt}%
\definecolor{currentstroke}{rgb}{1.000000,0.894118,0.788235}%
\pgfsetstrokecolor{currentstroke}%
\pgfsetdash{}{0pt}%
\pgfpathmoveto{\pgfqpoint{2.721768in}{3.285261in}}%
\pgfpathlineto{\pgfqpoint{2.748715in}{3.269703in}}%
\pgfpathlineto{\pgfqpoint{2.973794in}{3.515505in}}%
\pgfpathlineto{\pgfqpoint{2.946847in}{3.531063in}}%
\pgfpathlineto{\pgfqpoint{2.721768in}{3.285261in}}%
\pgfpathclose%
\pgfusepath{fill}%
\end{pgfscope}%
\begin{pgfscope}%
\pgfpathrectangle{\pgfqpoint{0.765000in}{0.660000in}}{\pgfqpoint{4.620000in}{4.620000in}}%
\pgfusepath{clip}%
\pgfsetbuttcap%
\pgfsetroundjoin%
\definecolor{currentfill}{rgb}{1.000000,0.894118,0.788235}%
\pgfsetfillcolor{currentfill}%
\pgfsetlinewidth{0.000000pt}%
\definecolor{currentstroke}{rgb}{1.000000,0.894118,0.788235}%
\pgfsetstrokecolor{currentstroke}%
\pgfsetdash{}{0pt}%
\pgfpathmoveto{\pgfqpoint{2.748715in}{3.269703in}}%
\pgfpathlineto{\pgfqpoint{2.775662in}{3.285261in}}%
\pgfpathlineto{\pgfqpoint{3.000742in}{3.531063in}}%
\pgfpathlineto{\pgfqpoint{2.973794in}{3.515505in}}%
\pgfpathlineto{\pgfqpoint{2.748715in}{3.269703in}}%
\pgfpathclose%
\pgfusepath{fill}%
\end{pgfscope}%
\begin{pgfscope}%
\pgfpathrectangle{\pgfqpoint{0.765000in}{0.660000in}}{\pgfqpoint{4.620000in}{4.620000in}}%
\pgfusepath{clip}%
\pgfsetbuttcap%
\pgfsetroundjoin%
\definecolor{currentfill}{rgb}{1.000000,0.894118,0.788235}%
\pgfsetfillcolor{currentfill}%
\pgfsetlinewidth{0.000000pt}%
\definecolor{currentstroke}{rgb}{1.000000,0.894118,0.788235}%
\pgfsetstrokecolor{currentstroke}%
\pgfsetdash{}{0pt}%
\pgfpathmoveto{\pgfqpoint{2.748715in}{3.331935in}}%
\pgfpathlineto{\pgfqpoint{2.721768in}{3.316377in}}%
\pgfpathlineto{\pgfqpoint{2.946847in}{3.562179in}}%
\pgfpathlineto{\pgfqpoint{2.973794in}{3.577737in}}%
\pgfpathlineto{\pgfqpoint{2.748715in}{3.331935in}}%
\pgfpathclose%
\pgfusepath{fill}%
\end{pgfscope}%
\begin{pgfscope}%
\pgfpathrectangle{\pgfqpoint{0.765000in}{0.660000in}}{\pgfqpoint{4.620000in}{4.620000in}}%
\pgfusepath{clip}%
\pgfsetbuttcap%
\pgfsetroundjoin%
\definecolor{currentfill}{rgb}{1.000000,0.894118,0.788235}%
\pgfsetfillcolor{currentfill}%
\pgfsetlinewidth{0.000000pt}%
\definecolor{currentstroke}{rgb}{1.000000,0.894118,0.788235}%
\pgfsetstrokecolor{currentstroke}%
\pgfsetdash{}{0pt}%
\pgfpathmoveto{\pgfqpoint{2.775662in}{3.316377in}}%
\pgfpathlineto{\pgfqpoint{2.748715in}{3.331935in}}%
\pgfpathlineto{\pgfqpoint{2.973794in}{3.577737in}}%
\pgfpathlineto{\pgfqpoint{3.000742in}{3.562179in}}%
\pgfpathlineto{\pgfqpoint{2.775662in}{3.316377in}}%
\pgfpathclose%
\pgfusepath{fill}%
\end{pgfscope}%
\begin{pgfscope}%
\pgfpathrectangle{\pgfqpoint{0.765000in}{0.660000in}}{\pgfqpoint{4.620000in}{4.620000in}}%
\pgfusepath{clip}%
\pgfsetbuttcap%
\pgfsetroundjoin%
\definecolor{currentfill}{rgb}{1.000000,0.894118,0.788235}%
\pgfsetfillcolor{currentfill}%
\pgfsetlinewidth{0.000000pt}%
\definecolor{currentstroke}{rgb}{1.000000,0.894118,0.788235}%
\pgfsetstrokecolor{currentstroke}%
\pgfsetdash{}{0pt}%
\pgfpathmoveto{\pgfqpoint{2.721768in}{3.285261in}}%
\pgfpathlineto{\pgfqpoint{2.721768in}{3.316377in}}%
\pgfpathlineto{\pgfqpoint{2.946847in}{3.562179in}}%
\pgfpathlineto{\pgfqpoint{2.973794in}{3.515505in}}%
\pgfpathlineto{\pgfqpoint{2.721768in}{3.285261in}}%
\pgfpathclose%
\pgfusepath{fill}%
\end{pgfscope}%
\begin{pgfscope}%
\pgfpathrectangle{\pgfqpoint{0.765000in}{0.660000in}}{\pgfqpoint{4.620000in}{4.620000in}}%
\pgfusepath{clip}%
\pgfsetbuttcap%
\pgfsetroundjoin%
\definecolor{currentfill}{rgb}{1.000000,0.894118,0.788235}%
\pgfsetfillcolor{currentfill}%
\pgfsetlinewidth{0.000000pt}%
\definecolor{currentstroke}{rgb}{1.000000,0.894118,0.788235}%
\pgfsetstrokecolor{currentstroke}%
\pgfsetdash{}{0pt}%
\pgfpathmoveto{\pgfqpoint{2.748715in}{3.269703in}}%
\pgfpathlineto{\pgfqpoint{2.748715in}{3.300819in}}%
\pgfpathlineto{\pgfqpoint{2.973794in}{3.546621in}}%
\pgfpathlineto{\pgfqpoint{2.946847in}{3.531063in}}%
\pgfpathlineto{\pgfqpoint{2.748715in}{3.269703in}}%
\pgfpathclose%
\pgfusepath{fill}%
\end{pgfscope}%
\begin{pgfscope}%
\pgfpathrectangle{\pgfqpoint{0.765000in}{0.660000in}}{\pgfqpoint{4.620000in}{4.620000in}}%
\pgfusepath{clip}%
\pgfsetbuttcap%
\pgfsetroundjoin%
\definecolor{currentfill}{rgb}{1.000000,0.894118,0.788235}%
\pgfsetfillcolor{currentfill}%
\pgfsetlinewidth{0.000000pt}%
\definecolor{currentstroke}{rgb}{1.000000,0.894118,0.788235}%
\pgfsetstrokecolor{currentstroke}%
\pgfsetdash{}{0pt}%
\pgfpathmoveto{\pgfqpoint{2.748715in}{3.300819in}}%
\pgfpathlineto{\pgfqpoint{2.775662in}{3.316377in}}%
\pgfpathlineto{\pgfqpoint{3.000742in}{3.562179in}}%
\pgfpathlineto{\pgfqpoint{2.973794in}{3.546621in}}%
\pgfpathlineto{\pgfqpoint{2.748715in}{3.300819in}}%
\pgfpathclose%
\pgfusepath{fill}%
\end{pgfscope}%
\begin{pgfscope}%
\pgfpathrectangle{\pgfqpoint{0.765000in}{0.660000in}}{\pgfqpoint{4.620000in}{4.620000in}}%
\pgfusepath{clip}%
\pgfsetbuttcap%
\pgfsetroundjoin%
\definecolor{currentfill}{rgb}{1.000000,0.894118,0.788235}%
\pgfsetfillcolor{currentfill}%
\pgfsetlinewidth{0.000000pt}%
\definecolor{currentstroke}{rgb}{1.000000,0.894118,0.788235}%
\pgfsetstrokecolor{currentstroke}%
\pgfsetdash{}{0pt}%
\pgfpathmoveto{\pgfqpoint{2.721768in}{3.316377in}}%
\pgfpathlineto{\pgfqpoint{2.748715in}{3.300819in}}%
\pgfpathlineto{\pgfqpoint{2.973794in}{3.546621in}}%
\pgfpathlineto{\pgfqpoint{2.946847in}{3.562179in}}%
\pgfpathlineto{\pgfqpoint{2.721768in}{3.316377in}}%
\pgfpathclose%
\pgfusepath{fill}%
\end{pgfscope}%
\begin{pgfscope}%
\pgfpathrectangle{\pgfqpoint{0.765000in}{0.660000in}}{\pgfqpoint{4.620000in}{4.620000in}}%
\pgfusepath{clip}%
\pgfsetbuttcap%
\pgfsetroundjoin%
\definecolor{currentfill}{rgb}{1.000000,0.894118,0.788235}%
\pgfsetfillcolor{currentfill}%
\pgfsetlinewidth{0.000000pt}%
\definecolor{currentstroke}{rgb}{1.000000,0.894118,0.788235}%
\pgfsetstrokecolor{currentstroke}%
\pgfsetdash{}{0pt}%
\pgfpathmoveto{\pgfqpoint{2.748715in}{3.300819in}}%
\pgfpathlineto{\pgfqpoint{2.721768in}{3.285261in}}%
\pgfpathlineto{\pgfqpoint{2.748715in}{3.269703in}}%
\pgfpathlineto{\pgfqpoint{2.775662in}{3.285261in}}%
\pgfpathlineto{\pgfqpoint{2.748715in}{3.300819in}}%
\pgfpathclose%
\pgfusepath{fill}%
\end{pgfscope}%
\begin{pgfscope}%
\pgfpathrectangle{\pgfqpoint{0.765000in}{0.660000in}}{\pgfqpoint{4.620000in}{4.620000in}}%
\pgfusepath{clip}%
\pgfsetbuttcap%
\pgfsetroundjoin%
\definecolor{currentfill}{rgb}{1.000000,0.894118,0.788235}%
\pgfsetfillcolor{currentfill}%
\pgfsetlinewidth{0.000000pt}%
\definecolor{currentstroke}{rgb}{1.000000,0.894118,0.788235}%
\pgfsetstrokecolor{currentstroke}%
\pgfsetdash{}{0pt}%
\pgfpathmoveto{\pgfqpoint{2.748715in}{3.300819in}}%
\pgfpathlineto{\pgfqpoint{2.721768in}{3.285261in}}%
\pgfpathlineto{\pgfqpoint{2.721768in}{3.316377in}}%
\pgfpathlineto{\pgfqpoint{2.748715in}{3.331935in}}%
\pgfpathlineto{\pgfqpoint{2.748715in}{3.300819in}}%
\pgfpathclose%
\pgfusepath{fill}%
\end{pgfscope}%
\begin{pgfscope}%
\pgfpathrectangle{\pgfqpoint{0.765000in}{0.660000in}}{\pgfqpoint{4.620000in}{4.620000in}}%
\pgfusepath{clip}%
\pgfsetbuttcap%
\pgfsetroundjoin%
\definecolor{currentfill}{rgb}{1.000000,0.894118,0.788235}%
\pgfsetfillcolor{currentfill}%
\pgfsetlinewidth{0.000000pt}%
\definecolor{currentstroke}{rgb}{1.000000,0.894118,0.788235}%
\pgfsetstrokecolor{currentstroke}%
\pgfsetdash{}{0pt}%
\pgfpathmoveto{\pgfqpoint{2.748715in}{3.300819in}}%
\pgfpathlineto{\pgfqpoint{2.775662in}{3.285261in}}%
\pgfpathlineto{\pgfqpoint{2.775662in}{3.316377in}}%
\pgfpathlineto{\pgfqpoint{2.748715in}{3.331935in}}%
\pgfpathlineto{\pgfqpoint{2.748715in}{3.300819in}}%
\pgfpathclose%
\pgfusepath{fill}%
\end{pgfscope}%
\begin{pgfscope}%
\pgfpathrectangle{\pgfqpoint{0.765000in}{0.660000in}}{\pgfqpoint{4.620000in}{4.620000in}}%
\pgfusepath{clip}%
\pgfsetbuttcap%
\pgfsetroundjoin%
\definecolor{currentfill}{rgb}{1.000000,0.894118,0.788235}%
\pgfsetfillcolor{currentfill}%
\pgfsetlinewidth{0.000000pt}%
\definecolor{currentstroke}{rgb}{1.000000,0.894118,0.788235}%
\pgfsetstrokecolor{currentstroke}%
\pgfsetdash{}{0pt}%
\pgfpathmoveto{\pgfqpoint{2.716919in}{3.259885in}}%
\pgfpathlineto{\pgfqpoint{2.689971in}{3.244327in}}%
\pgfpathlineto{\pgfqpoint{2.716919in}{3.228769in}}%
\pgfpathlineto{\pgfqpoint{2.743866in}{3.244327in}}%
\pgfpathlineto{\pgfqpoint{2.716919in}{3.259885in}}%
\pgfpathclose%
\pgfusepath{fill}%
\end{pgfscope}%
\begin{pgfscope}%
\pgfpathrectangle{\pgfqpoint{0.765000in}{0.660000in}}{\pgfqpoint{4.620000in}{4.620000in}}%
\pgfusepath{clip}%
\pgfsetbuttcap%
\pgfsetroundjoin%
\definecolor{currentfill}{rgb}{1.000000,0.894118,0.788235}%
\pgfsetfillcolor{currentfill}%
\pgfsetlinewidth{0.000000pt}%
\definecolor{currentstroke}{rgb}{1.000000,0.894118,0.788235}%
\pgfsetstrokecolor{currentstroke}%
\pgfsetdash{}{0pt}%
\pgfpathmoveto{\pgfqpoint{2.716919in}{3.259885in}}%
\pgfpathlineto{\pgfqpoint{2.689971in}{3.244327in}}%
\pgfpathlineto{\pgfqpoint{2.689971in}{3.275443in}}%
\pgfpathlineto{\pgfqpoint{2.716919in}{3.291001in}}%
\pgfpathlineto{\pgfqpoint{2.716919in}{3.259885in}}%
\pgfpathclose%
\pgfusepath{fill}%
\end{pgfscope}%
\begin{pgfscope}%
\pgfpathrectangle{\pgfqpoint{0.765000in}{0.660000in}}{\pgfqpoint{4.620000in}{4.620000in}}%
\pgfusepath{clip}%
\pgfsetbuttcap%
\pgfsetroundjoin%
\definecolor{currentfill}{rgb}{1.000000,0.894118,0.788235}%
\pgfsetfillcolor{currentfill}%
\pgfsetlinewidth{0.000000pt}%
\definecolor{currentstroke}{rgb}{1.000000,0.894118,0.788235}%
\pgfsetstrokecolor{currentstroke}%
\pgfsetdash{}{0pt}%
\pgfpathmoveto{\pgfqpoint{2.716919in}{3.259885in}}%
\pgfpathlineto{\pgfqpoint{2.743866in}{3.244327in}}%
\pgfpathlineto{\pgfqpoint{2.743866in}{3.275443in}}%
\pgfpathlineto{\pgfqpoint{2.716919in}{3.291001in}}%
\pgfpathlineto{\pgfqpoint{2.716919in}{3.259885in}}%
\pgfpathclose%
\pgfusepath{fill}%
\end{pgfscope}%
\begin{pgfscope}%
\pgfpathrectangle{\pgfqpoint{0.765000in}{0.660000in}}{\pgfqpoint{4.620000in}{4.620000in}}%
\pgfusepath{clip}%
\pgfsetbuttcap%
\pgfsetroundjoin%
\definecolor{currentfill}{rgb}{1.000000,0.894118,0.788235}%
\pgfsetfillcolor{currentfill}%
\pgfsetlinewidth{0.000000pt}%
\definecolor{currentstroke}{rgb}{1.000000,0.894118,0.788235}%
\pgfsetstrokecolor{currentstroke}%
\pgfsetdash{}{0pt}%
\pgfpathmoveto{\pgfqpoint{2.748715in}{3.331935in}}%
\pgfpathlineto{\pgfqpoint{2.721768in}{3.316377in}}%
\pgfpathlineto{\pgfqpoint{2.748715in}{3.300819in}}%
\pgfpathlineto{\pgfqpoint{2.775662in}{3.316377in}}%
\pgfpathlineto{\pgfqpoint{2.748715in}{3.331935in}}%
\pgfpathclose%
\pgfusepath{fill}%
\end{pgfscope}%
\begin{pgfscope}%
\pgfpathrectangle{\pgfqpoint{0.765000in}{0.660000in}}{\pgfqpoint{4.620000in}{4.620000in}}%
\pgfusepath{clip}%
\pgfsetbuttcap%
\pgfsetroundjoin%
\definecolor{currentfill}{rgb}{1.000000,0.894118,0.788235}%
\pgfsetfillcolor{currentfill}%
\pgfsetlinewidth{0.000000pt}%
\definecolor{currentstroke}{rgb}{1.000000,0.894118,0.788235}%
\pgfsetstrokecolor{currentstroke}%
\pgfsetdash{}{0pt}%
\pgfpathmoveto{\pgfqpoint{2.748715in}{3.269703in}}%
\pgfpathlineto{\pgfqpoint{2.775662in}{3.285261in}}%
\pgfpathlineto{\pgfqpoint{2.775662in}{3.316377in}}%
\pgfpathlineto{\pgfqpoint{2.748715in}{3.300819in}}%
\pgfpathlineto{\pgfqpoint{2.748715in}{3.269703in}}%
\pgfpathclose%
\pgfusepath{fill}%
\end{pgfscope}%
\begin{pgfscope}%
\pgfpathrectangle{\pgfqpoint{0.765000in}{0.660000in}}{\pgfqpoint{4.620000in}{4.620000in}}%
\pgfusepath{clip}%
\pgfsetbuttcap%
\pgfsetroundjoin%
\definecolor{currentfill}{rgb}{1.000000,0.894118,0.788235}%
\pgfsetfillcolor{currentfill}%
\pgfsetlinewidth{0.000000pt}%
\definecolor{currentstroke}{rgb}{1.000000,0.894118,0.788235}%
\pgfsetstrokecolor{currentstroke}%
\pgfsetdash{}{0pt}%
\pgfpathmoveto{\pgfqpoint{2.721768in}{3.285261in}}%
\pgfpathlineto{\pgfqpoint{2.748715in}{3.269703in}}%
\pgfpathlineto{\pgfqpoint{2.748715in}{3.300819in}}%
\pgfpathlineto{\pgfqpoint{2.721768in}{3.316377in}}%
\pgfpathlineto{\pgfqpoint{2.721768in}{3.285261in}}%
\pgfpathclose%
\pgfusepath{fill}%
\end{pgfscope}%
\begin{pgfscope}%
\pgfpathrectangle{\pgfqpoint{0.765000in}{0.660000in}}{\pgfqpoint{4.620000in}{4.620000in}}%
\pgfusepath{clip}%
\pgfsetbuttcap%
\pgfsetroundjoin%
\definecolor{currentfill}{rgb}{1.000000,0.894118,0.788235}%
\pgfsetfillcolor{currentfill}%
\pgfsetlinewidth{0.000000pt}%
\definecolor{currentstroke}{rgb}{1.000000,0.894118,0.788235}%
\pgfsetstrokecolor{currentstroke}%
\pgfsetdash{}{0pt}%
\pgfpathmoveto{\pgfqpoint{2.716919in}{3.291001in}}%
\pgfpathlineto{\pgfqpoint{2.689971in}{3.275443in}}%
\pgfpathlineto{\pgfqpoint{2.716919in}{3.259885in}}%
\pgfpathlineto{\pgfqpoint{2.743866in}{3.275443in}}%
\pgfpathlineto{\pgfqpoint{2.716919in}{3.291001in}}%
\pgfpathclose%
\pgfusepath{fill}%
\end{pgfscope}%
\begin{pgfscope}%
\pgfpathrectangle{\pgfqpoint{0.765000in}{0.660000in}}{\pgfqpoint{4.620000in}{4.620000in}}%
\pgfusepath{clip}%
\pgfsetbuttcap%
\pgfsetroundjoin%
\definecolor{currentfill}{rgb}{1.000000,0.894118,0.788235}%
\pgfsetfillcolor{currentfill}%
\pgfsetlinewidth{0.000000pt}%
\definecolor{currentstroke}{rgb}{1.000000,0.894118,0.788235}%
\pgfsetstrokecolor{currentstroke}%
\pgfsetdash{}{0pt}%
\pgfpathmoveto{\pgfqpoint{2.716919in}{3.228769in}}%
\pgfpathlineto{\pgfqpoint{2.743866in}{3.244327in}}%
\pgfpathlineto{\pgfqpoint{2.743866in}{3.275443in}}%
\pgfpathlineto{\pgfqpoint{2.716919in}{3.259885in}}%
\pgfpathlineto{\pgfqpoint{2.716919in}{3.228769in}}%
\pgfpathclose%
\pgfusepath{fill}%
\end{pgfscope}%
\begin{pgfscope}%
\pgfpathrectangle{\pgfqpoint{0.765000in}{0.660000in}}{\pgfqpoint{4.620000in}{4.620000in}}%
\pgfusepath{clip}%
\pgfsetbuttcap%
\pgfsetroundjoin%
\definecolor{currentfill}{rgb}{1.000000,0.894118,0.788235}%
\pgfsetfillcolor{currentfill}%
\pgfsetlinewidth{0.000000pt}%
\definecolor{currentstroke}{rgb}{1.000000,0.894118,0.788235}%
\pgfsetstrokecolor{currentstroke}%
\pgfsetdash{}{0pt}%
\pgfpathmoveto{\pgfqpoint{2.689971in}{3.244327in}}%
\pgfpathlineto{\pgfqpoint{2.716919in}{3.228769in}}%
\pgfpathlineto{\pgfqpoint{2.716919in}{3.259885in}}%
\pgfpathlineto{\pgfqpoint{2.689971in}{3.275443in}}%
\pgfpathlineto{\pgfqpoint{2.689971in}{3.244327in}}%
\pgfpathclose%
\pgfusepath{fill}%
\end{pgfscope}%
\begin{pgfscope}%
\pgfpathrectangle{\pgfqpoint{0.765000in}{0.660000in}}{\pgfqpoint{4.620000in}{4.620000in}}%
\pgfusepath{clip}%
\pgfsetbuttcap%
\pgfsetroundjoin%
\definecolor{currentfill}{rgb}{1.000000,0.894118,0.788235}%
\pgfsetfillcolor{currentfill}%
\pgfsetlinewidth{0.000000pt}%
\definecolor{currentstroke}{rgb}{1.000000,0.894118,0.788235}%
\pgfsetstrokecolor{currentstroke}%
\pgfsetdash{}{0pt}%
\pgfpathmoveto{\pgfqpoint{2.748715in}{3.300819in}}%
\pgfpathlineto{\pgfqpoint{2.721768in}{3.285261in}}%
\pgfpathlineto{\pgfqpoint{2.689971in}{3.244327in}}%
\pgfpathlineto{\pgfqpoint{2.716919in}{3.259885in}}%
\pgfpathlineto{\pgfqpoint{2.748715in}{3.300819in}}%
\pgfpathclose%
\pgfusepath{fill}%
\end{pgfscope}%
\begin{pgfscope}%
\pgfpathrectangle{\pgfqpoint{0.765000in}{0.660000in}}{\pgfqpoint{4.620000in}{4.620000in}}%
\pgfusepath{clip}%
\pgfsetbuttcap%
\pgfsetroundjoin%
\definecolor{currentfill}{rgb}{1.000000,0.894118,0.788235}%
\pgfsetfillcolor{currentfill}%
\pgfsetlinewidth{0.000000pt}%
\definecolor{currentstroke}{rgb}{1.000000,0.894118,0.788235}%
\pgfsetstrokecolor{currentstroke}%
\pgfsetdash{}{0pt}%
\pgfpathmoveto{\pgfqpoint{2.775662in}{3.285261in}}%
\pgfpathlineto{\pgfqpoint{2.748715in}{3.300819in}}%
\pgfpathlineto{\pgfqpoint{2.716919in}{3.259885in}}%
\pgfpathlineto{\pgfqpoint{2.743866in}{3.244327in}}%
\pgfpathlineto{\pgfqpoint{2.775662in}{3.285261in}}%
\pgfpathclose%
\pgfusepath{fill}%
\end{pgfscope}%
\begin{pgfscope}%
\pgfpathrectangle{\pgfqpoint{0.765000in}{0.660000in}}{\pgfqpoint{4.620000in}{4.620000in}}%
\pgfusepath{clip}%
\pgfsetbuttcap%
\pgfsetroundjoin%
\definecolor{currentfill}{rgb}{1.000000,0.894118,0.788235}%
\pgfsetfillcolor{currentfill}%
\pgfsetlinewidth{0.000000pt}%
\definecolor{currentstroke}{rgb}{1.000000,0.894118,0.788235}%
\pgfsetstrokecolor{currentstroke}%
\pgfsetdash{}{0pt}%
\pgfpathmoveto{\pgfqpoint{2.748715in}{3.300819in}}%
\pgfpathlineto{\pgfqpoint{2.748715in}{3.331935in}}%
\pgfpathlineto{\pgfqpoint{2.716919in}{3.291001in}}%
\pgfpathlineto{\pgfqpoint{2.743866in}{3.244327in}}%
\pgfpathlineto{\pgfqpoint{2.748715in}{3.300819in}}%
\pgfpathclose%
\pgfusepath{fill}%
\end{pgfscope}%
\begin{pgfscope}%
\pgfpathrectangle{\pgfqpoint{0.765000in}{0.660000in}}{\pgfqpoint{4.620000in}{4.620000in}}%
\pgfusepath{clip}%
\pgfsetbuttcap%
\pgfsetroundjoin%
\definecolor{currentfill}{rgb}{1.000000,0.894118,0.788235}%
\pgfsetfillcolor{currentfill}%
\pgfsetlinewidth{0.000000pt}%
\definecolor{currentstroke}{rgb}{1.000000,0.894118,0.788235}%
\pgfsetstrokecolor{currentstroke}%
\pgfsetdash{}{0pt}%
\pgfpathmoveto{\pgfqpoint{2.775662in}{3.285261in}}%
\pgfpathlineto{\pgfqpoint{2.775662in}{3.316377in}}%
\pgfpathlineto{\pgfqpoint{2.743866in}{3.275443in}}%
\pgfpathlineto{\pgfqpoint{2.716919in}{3.259885in}}%
\pgfpathlineto{\pgfqpoint{2.775662in}{3.285261in}}%
\pgfpathclose%
\pgfusepath{fill}%
\end{pgfscope}%
\begin{pgfscope}%
\pgfpathrectangle{\pgfqpoint{0.765000in}{0.660000in}}{\pgfqpoint{4.620000in}{4.620000in}}%
\pgfusepath{clip}%
\pgfsetbuttcap%
\pgfsetroundjoin%
\definecolor{currentfill}{rgb}{1.000000,0.894118,0.788235}%
\pgfsetfillcolor{currentfill}%
\pgfsetlinewidth{0.000000pt}%
\definecolor{currentstroke}{rgb}{1.000000,0.894118,0.788235}%
\pgfsetstrokecolor{currentstroke}%
\pgfsetdash{}{0pt}%
\pgfpathmoveto{\pgfqpoint{2.721768in}{3.285261in}}%
\pgfpathlineto{\pgfqpoint{2.748715in}{3.269703in}}%
\pgfpathlineto{\pgfqpoint{2.716919in}{3.228769in}}%
\pgfpathlineto{\pgfqpoint{2.689971in}{3.244327in}}%
\pgfpathlineto{\pgfqpoint{2.721768in}{3.285261in}}%
\pgfpathclose%
\pgfusepath{fill}%
\end{pgfscope}%
\begin{pgfscope}%
\pgfpathrectangle{\pgfqpoint{0.765000in}{0.660000in}}{\pgfqpoint{4.620000in}{4.620000in}}%
\pgfusepath{clip}%
\pgfsetbuttcap%
\pgfsetroundjoin%
\definecolor{currentfill}{rgb}{1.000000,0.894118,0.788235}%
\pgfsetfillcolor{currentfill}%
\pgfsetlinewidth{0.000000pt}%
\definecolor{currentstroke}{rgb}{1.000000,0.894118,0.788235}%
\pgfsetstrokecolor{currentstroke}%
\pgfsetdash{}{0pt}%
\pgfpathmoveto{\pgfqpoint{2.748715in}{3.269703in}}%
\pgfpathlineto{\pgfqpoint{2.775662in}{3.285261in}}%
\pgfpathlineto{\pgfqpoint{2.743866in}{3.244327in}}%
\pgfpathlineto{\pgfqpoint{2.716919in}{3.228769in}}%
\pgfpathlineto{\pgfqpoint{2.748715in}{3.269703in}}%
\pgfpathclose%
\pgfusepath{fill}%
\end{pgfscope}%
\begin{pgfscope}%
\pgfpathrectangle{\pgfqpoint{0.765000in}{0.660000in}}{\pgfqpoint{4.620000in}{4.620000in}}%
\pgfusepath{clip}%
\pgfsetbuttcap%
\pgfsetroundjoin%
\definecolor{currentfill}{rgb}{1.000000,0.894118,0.788235}%
\pgfsetfillcolor{currentfill}%
\pgfsetlinewidth{0.000000pt}%
\definecolor{currentstroke}{rgb}{1.000000,0.894118,0.788235}%
\pgfsetstrokecolor{currentstroke}%
\pgfsetdash{}{0pt}%
\pgfpathmoveto{\pgfqpoint{2.748715in}{3.331935in}}%
\pgfpathlineto{\pgfqpoint{2.721768in}{3.316377in}}%
\pgfpathlineto{\pgfqpoint{2.689971in}{3.275443in}}%
\pgfpathlineto{\pgfqpoint{2.716919in}{3.291001in}}%
\pgfpathlineto{\pgfqpoint{2.748715in}{3.331935in}}%
\pgfpathclose%
\pgfusepath{fill}%
\end{pgfscope}%
\begin{pgfscope}%
\pgfpathrectangle{\pgfqpoint{0.765000in}{0.660000in}}{\pgfqpoint{4.620000in}{4.620000in}}%
\pgfusepath{clip}%
\pgfsetbuttcap%
\pgfsetroundjoin%
\definecolor{currentfill}{rgb}{1.000000,0.894118,0.788235}%
\pgfsetfillcolor{currentfill}%
\pgfsetlinewidth{0.000000pt}%
\definecolor{currentstroke}{rgb}{1.000000,0.894118,0.788235}%
\pgfsetstrokecolor{currentstroke}%
\pgfsetdash{}{0pt}%
\pgfpathmoveto{\pgfqpoint{2.775662in}{3.316377in}}%
\pgfpathlineto{\pgfqpoint{2.748715in}{3.331935in}}%
\pgfpathlineto{\pgfqpoint{2.716919in}{3.291001in}}%
\pgfpathlineto{\pgfqpoint{2.743866in}{3.275443in}}%
\pgfpathlineto{\pgfqpoint{2.775662in}{3.316377in}}%
\pgfpathclose%
\pgfusepath{fill}%
\end{pgfscope}%
\begin{pgfscope}%
\pgfpathrectangle{\pgfqpoint{0.765000in}{0.660000in}}{\pgfqpoint{4.620000in}{4.620000in}}%
\pgfusepath{clip}%
\pgfsetbuttcap%
\pgfsetroundjoin%
\definecolor{currentfill}{rgb}{1.000000,0.894118,0.788235}%
\pgfsetfillcolor{currentfill}%
\pgfsetlinewidth{0.000000pt}%
\definecolor{currentstroke}{rgb}{1.000000,0.894118,0.788235}%
\pgfsetstrokecolor{currentstroke}%
\pgfsetdash{}{0pt}%
\pgfpathmoveto{\pgfqpoint{2.721768in}{3.285261in}}%
\pgfpathlineto{\pgfqpoint{2.721768in}{3.316377in}}%
\pgfpathlineto{\pgfqpoint{2.689971in}{3.275443in}}%
\pgfpathlineto{\pgfqpoint{2.716919in}{3.228769in}}%
\pgfpathlineto{\pgfqpoint{2.721768in}{3.285261in}}%
\pgfpathclose%
\pgfusepath{fill}%
\end{pgfscope}%
\begin{pgfscope}%
\pgfpathrectangle{\pgfqpoint{0.765000in}{0.660000in}}{\pgfqpoint{4.620000in}{4.620000in}}%
\pgfusepath{clip}%
\pgfsetbuttcap%
\pgfsetroundjoin%
\definecolor{currentfill}{rgb}{1.000000,0.894118,0.788235}%
\pgfsetfillcolor{currentfill}%
\pgfsetlinewidth{0.000000pt}%
\definecolor{currentstroke}{rgb}{1.000000,0.894118,0.788235}%
\pgfsetstrokecolor{currentstroke}%
\pgfsetdash{}{0pt}%
\pgfpathmoveto{\pgfqpoint{2.748715in}{3.269703in}}%
\pgfpathlineto{\pgfqpoint{2.748715in}{3.300819in}}%
\pgfpathlineto{\pgfqpoint{2.716919in}{3.259885in}}%
\pgfpathlineto{\pgfqpoint{2.689971in}{3.244327in}}%
\pgfpathlineto{\pgfqpoint{2.748715in}{3.269703in}}%
\pgfpathclose%
\pgfusepath{fill}%
\end{pgfscope}%
\begin{pgfscope}%
\pgfpathrectangle{\pgfqpoint{0.765000in}{0.660000in}}{\pgfqpoint{4.620000in}{4.620000in}}%
\pgfusepath{clip}%
\pgfsetbuttcap%
\pgfsetroundjoin%
\definecolor{currentfill}{rgb}{1.000000,0.894118,0.788235}%
\pgfsetfillcolor{currentfill}%
\pgfsetlinewidth{0.000000pt}%
\definecolor{currentstroke}{rgb}{1.000000,0.894118,0.788235}%
\pgfsetstrokecolor{currentstroke}%
\pgfsetdash{}{0pt}%
\pgfpathmoveto{\pgfqpoint{2.748715in}{3.300819in}}%
\pgfpathlineto{\pgfqpoint{2.775662in}{3.316377in}}%
\pgfpathlineto{\pgfqpoint{2.743866in}{3.275443in}}%
\pgfpathlineto{\pgfqpoint{2.716919in}{3.259885in}}%
\pgfpathlineto{\pgfqpoint{2.748715in}{3.300819in}}%
\pgfpathclose%
\pgfusepath{fill}%
\end{pgfscope}%
\begin{pgfscope}%
\pgfpathrectangle{\pgfqpoint{0.765000in}{0.660000in}}{\pgfqpoint{4.620000in}{4.620000in}}%
\pgfusepath{clip}%
\pgfsetbuttcap%
\pgfsetroundjoin%
\definecolor{currentfill}{rgb}{1.000000,0.894118,0.788235}%
\pgfsetfillcolor{currentfill}%
\pgfsetlinewidth{0.000000pt}%
\definecolor{currentstroke}{rgb}{1.000000,0.894118,0.788235}%
\pgfsetstrokecolor{currentstroke}%
\pgfsetdash{}{0pt}%
\pgfpathmoveto{\pgfqpoint{2.721768in}{3.316377in}}%
\pgfpathlineto{\pgfqpoint{2.748715in}{3.300819in}}%
\pgfpathlineto{\pgfqpoint{2.716919in}{3.259885in}}%
\pgfpathlineto{\pgfqpoint{2.689971in}{3.275443in}}%
\pgfpathlineto{\pgfqpoint{2.721768in}{3.316377in}}%
\pgfpathclose%
\pgfusepath{fill}%
\end{pgfscope}%
\begin{pgfscope}%
\pgfpathrectangle{\pgfqpoint{0.765000in}{0.660000in}}{\pgfqpoint{4.620000in}{4.620000in}}%
\pgfusepath{clip}%
\pgfsetbuttcap%
\pgfsetroundjoin%
\definecolor{currentfill}{rgb}{1.000000,0.894118,0.788235}%
\pgfsetfillcolor{currentfill}%
\pgfsetlinewidth{0.000000pt}%
\definecolor{currentstroke}{rgb}{1.000000,0.894118,0.788235}%
\pgfsetstrokecolor{currentstroke}%
\pgfsetdash{}{0pt}%
\pgfpathmoveto{\pgfqpoint{3.220833in}{3.327632in}}%
\pgfpathlineto{\pgfqpoint{3.193886in}{3.312074in}}%
\pgfpathlineto{\pgfqpoint{3.220833in}{3.296516in}}%
\pgfpathlineto{\pgfqpoint{3.247780in}{3.312074in}}%
\pgfpathlineto{\pgfqpoint{3.220833in}{3.327632in}}%
\pgfpathclose%
\pgfusepath{fill}%
\end{pgfscope}%
\begin{pgfscope}%
\pgfpathrectangle{\pgfqpoint{0.765000in}{0.660000in}}{\pgfqpoint{4.620000in}{4.620000in}}%
\pgfusepath{clip}%
\pgfsetbuttcap%
\pgfsetroundjoin%
\definecolor{currentfill}{rgb}{1.000000,0.894118,0.788235}%
\pgfsetfillcolor{currentfill}%
\pgfsetlinewidth{0.000000pt}%
\definecolor{currentstroke}{rgb}{1.000000,0.894118,0.788235}%
\pgfsetstrokecolor{currentstroke}%
\pgfsetdash{}{0pt}%
\pgfpathmoveto{\pgfqpoint{3.220833in}{3.327632in}}%
\pgfpathlineto{\pgfqpoint{3.193886in}{3.312074in}}%
\pgfpathlineto{\pgfqpoint{3.193886in}{3.343190in}}%
\pgfpathlineto{\pgfqpoint{3.220833in}{3.358748in}}%
\pgfpathlineto{\pgfqpoint{3.220833in}{3.327632in}}%
\pgfpathclose%
\pgfusepath{fill}%
\end{pgfscope}%
\begin{pgfscope}%
\pgfpathrectangle{\pgfqpoint{0.765000in}{0.660000in}}{\pgfqpoint{4.620000in}{4.620000in}}%
\pgfusepath{clip}%
\pgfsetbuttcap%
\pgfsetroundjoin%
\definecolor{currentfill}{rgb}{1.000000,0.894118,0.788235}%
\pgfsetfillcolor{currentfill}%
\pgfsetlinewidth{0.000000pt}%
\definecolor{currentstroke}{rgb}{1.000000,0.894118,0.788235}%
\pgfsetstrokecolor{currentstroke}%
\pgfsetdash{}{0pt}%
\pgfpathmoveto{\pgfqpoint{3.220833in}{3.327632in}}%
\pgfpathlineto{\pgfqpoint{3.247780in}{3.312074in}}%
\pgfpathlineto{\pgfqpoint{3.247780in}{3.343190in}}%
\pgfpathlineto{\pgfqpoint{3.220833in}{3.358748in}}%
\pgfpathlineto{\pgfqpoint{3.220833in}{3.327632in}}%
\pgfpathclose%
\pgfusepath{fill}%
\end{pgfscope}%
\begin{pgfscope}%
\pgfpathrectangle{\pgfqpoint{0.765000in}{0.660000in}}{\pgfqpoint{4.620000in}{4.620000in}}%
\pgfusepath{clip}%
\pgfsetbuttcap%
\pgfsetroundjoin%
\definecolor{currentfill}{rgb}{1.000000,0.894118,0.788235}%
\pgfsetfillcolor{currentfill}%
\pgfsetlinewidth{0.000000pt}%
\definecolor{currentstroke}{rgb}{1.000000,0.894118,0.788235}%
\pgfsetstrokecolor{currentstroke}%
\pgfsetdash{}{0pt}%
\pgfpathmoveto{\pgfqpoint{3.233018in}{3.305233in}}%
\pgfpathlineto{\pgfqpoint{3.206071in}{3.289675in}}%
\pgfpathlineto{\pgfqpoint{3.233018in}{3.274117in}}%
\pgfpathlineto{\pgfqpoint{3.259965in}{3.289675in}}%
\pgfpathlineto{\pgfqpoint{3.233018in}{3.305233in}}%
\pgfpathclose%
\pgfusepath{fill}%
\end{pgfscope}%
\begin{pgfscope}%
\pgfpathrectangle{\pgfqpoint{0.765000in}{0.660000in}}{\pgfqpoint{4.620000in}{4.620000in}}%
\pgfusepath{clip}%
\pgfsetbuttcap%
\pgfsetroundjoin%
\definecolor{currentfill}{rgb}{1.000000,0.894118,0.788235}%
\pgfsetfillcolor{currentfill}%
\pgfsetlinewidth{0.000000pt}%
\definecolor{currentstroke}{rgb}{1.000000,0.894118,0.788235}%
\pgfsetstrokecolor{currentstroke}%
\pgfsetdash{}{0pt}%
\pgfpathmoveto{\pgfqpoint{3.233018in}{3.305233in}}%
\pgfpathlineto{\pgfqpoint{3.206071in}{3.289675in}}%
\pgfpathlineto{\pgfqpoint{3.206071in}{3.320791in}}%
\pgfpathlineto{\pgfqpoint{3.233018in}{3.336349in}}%
\pgfpathlineto{\pgfqpoint{3.233018in}{3.305233in}}%
\pgfpathclose%
\pgfusepath{fill}%
\end{pgfscope}%
\begin{pgfscope}%
\pgfpathrectangle{\pgfqpoint{0.765000in}{0.660000in}}{\pgfqpoint{4.620000in}{4.620000in}}%
\pgfusepath{clip}%
\pgfsetbuttcap%
\pgfsetroundjoin%
\definecolor{currentfill}{rgb}{1.000000,0.894118,0.788235}%
\pgfsetfillcolor{currentfill}%
\pgfsetlinewidth{0.000000pt}%
\definecolor{currentstroke}{rgb}{1.000000,0.894118,0.788235}%
\pgfsetstrokecolor{currentstroke}%
\pgfsetdash{}{0pt}%
\pgfpathmoveto{\pgfqpoint{3.233018in}{3.305233in}}%
\pgfpathlineto{\pgfqpoint{3.259965in}{3.289675in}}%
\pgfpathlineto{\pgfqpoint{3.259965in}{3.320791in}}%
\pgfpathlineto{\pgfqpoint{3.233018in}{3.336349in}}%
\pgfpathlineto{\pgfqpoint{3.233018in}{3.305233in}}%
\pgfpathclose%
\pgfusepath{fill}%
\end{pgfscope}%
\begin{pgfscope}%
\pgfpathrectangle{\pgfqpoint{0.765000in}{0.660000in}}{\pgfqpoint{4.620000in}{4.620000in}}%
\pgfusepath{clip}%
\pgfsetbuttcap%
\pgfsetroundjoin%
\definecolor{currentfill}{rgb}{1.000000,0.894118,0.788235}%
\pgfsetfillcolor{currentfill}%
\pgfsetlinewidth{0.000000pt}%
\definecolor{currentstroke}{rgb}{1.000000,0.894118,0.788235}%
\pgfsetstrokecolor{currentstroke}%
\pgfsetdash{}{0pt}%
\pgfpathmoveto{\pgfqpoint{3.220833in}{3.358748in}}%
\pgfpathlineto{\pgfqpoint{3.193886in}{3.343190in}}%
\pgfpathlineto{\pgfqpoint{3.220833in}{3.327632in}}%
\pgfpathlineto{\pgfqpoint{3.247780in}{3.343190in}}%
\pgfpathlineto{\pgfqpoint{3.220833in}{3.358748in}}%
\pgfpathclose%
\pgfusepath{fill}%
\end{pgfscope}%
\begin{pgfscope}%
\pgfpathrectangle{\pgfqpoint{0.765000in}{0.660000in}}{\pgfqpoint{4.620000in}{4.620000in}}%
\pgfusepath{clip}%
\pgfsetbuttcap%
\pgfsetroundjoin%
\definecolor{currentfill}{rgb}{1.000000,0.894118,0.788235}%
\pgfsetfillcolor{currentfill}%
\pgfsetlinewidth{0.000000pt}%
\definecolor{currentstroke}{rgb}{1.000000,0.894118,0.788235}%
\pgfsetstrokecolor{currentstroke}%
\pgfsetdash{}{0pt}%
\pgfpathmoveto{\pgfqpoint{3.220833in}{3.296516in}}%
\pgfpathlineto{\pgfqpoint{3.247780in}{3.312074in}}%
\pgfpathlineto{\pgfqpoint{3.247780in}{3.343190in}}%
\pgfpathlineto{\pgfqpoint{3.220833in}{3.327632in}}%
\pgfpathlineto{\pgfqpoint{3.220833in}{3.296516in}}%
\pgfpathclose%
\pgfusepath{fill}%
\end{pgfscope}%
\begin{pgfscope}%
\pgfpathrectangle{\pgfqpoint{0.765000in}{0.660000in}}{\pgfqpoint{4.620000in}{4.620000in}}%
\pgfusepath{clip}%
\pgfsetbuttcap%
\pgfsetroundjoin%
\definecolor{currentfill}{rgb}{1.000000,0.894118,0.788235}%
\pgfsetfillcolor{currentfill}%
\pgfsetlinewidth{0.000000pt}%
\definecolor{currentstroke}{rgb}{1.000000,0.894118,0.788235}%
\pgfsetstrokecolor{currentstroke}%
\pgfsetdash{}{0pt}%
\pgfpathmoveto{\pgfqpoint{3.193886in}{3.312074in}}%
\pgfpathlineto{\pgfqpoint{3.220833in}{3.296516in}}%
\pgfpathlineto{\pgfqpoint{3.220833in}{3.327632in}}%
\pgfpathlineto{\pgfqpoint{3.193886in}{3.343190in}}%
\pgfpathlineto{\pgfqpoint{3.193886in}{3.312074in}}%
\pgfpathclose%
\pgfusepath{fill}%
\end{pgfscope}%
\begin{pgfscope}%
\pgfpathrectangle{\pgfqpoint{0.765000in}{0.660000in}}{\pgfqpoint{4.620000in}{4.620000in}}%
\pgfusepath{clip}%
\pgfsetbuttcap%
\pgfsetroundjoin%
\definecolor{currentfill}{rgb}{1.000000,0.894118,0.788235}%
\pgfsetfillcolor{currentfill}%
\pgfsetlinewidth{0.000000pt}%
\definecolor{currentstroke}{rgb}{1.000000,0.894118,0.788235}%
\pgfsetstrokecolor{currentstroke}%
\pgfsetdash{}{0pt}%
\pgfpathmoveto{\pgfqpoint{3.233018in}{3.336349in}}%
\pgfpathlineto{\pgfqpoint{3.206071in}{3.320791in}}%
\pgfpathlineto{\pgfqpoint{3.233018in}{3.305233in}}%
\pgfpathlineto{\pgfqpoint{3.259965in}{3.320791in}}%
\pgfpathlineto{\pgfqpoint{3.233018in}{3.336349in}}%
\pgfpathclose%
\pgfusepath{fill}%
\end{pgfscope}%
\begin{pgfscope}%
\pgfpathrectangle{\pgfqpoint{0.765000in}{0.660000in}}{\pgfqpoint{4.620000in}{4.620000in}}%
\pgfusepath{clip}%
\pgfsetbuttcap%
\pgfsetroundjoin%
\definecolor{currentfill}{rgb}{1.000000,0.894118,0.788235}%
\pgfsetfillcolor{currentfill}%
\pgfsetlinewidth{0.000000pt}%
\definecolor{currentstroke}{rgb}{1.000000,0.894118,0.788235}%
\pgfsetstrokecolor{currentstroke}%
\pgfsetdash{}{0pt}%
\pgfpathmoveto{\pgfqpoint{3.233018in}{3.274117in}}%
\pgfpathlineto{\pgfqpoint{3.259965in}{3.289675in}}%
\pgfpathlineto{\pgfqpoint{3.259965in}{3.320791in}}%
\pgfpathlineto{\pgfqpoint{3.233018in}{3.305233in}}%
\pgfpathlineto{\pgfqpoint{3.233018in}{3.274117in}}%
\pgfpathclose%
\pgfusepath{fill}%
\end{pgfscope}%
\begin{pgfscope}%
\pgfpathrectangle{\pgfqpoint{0.765000in}{0.660000in}}{\pgfqpoint{4.620000in}{4.620000in}}%
\pgfusepath{clip}%
\pgfsetbuttcap%
\pgfsetroundjoin%
\definecolor{currentfill}{rgb}{1.000000,0.894118,0.788235}%
\pgfsetfillcolor{currentfill}%
\pgfsetlinewidth{0.000000pt}%
\definecolor{currentstroke}{rgb}{1.000000,0.894118,0.788235}%
\pgfsetstrokecolor{currentstroke}%
\pgfsetdash{}{0pt}%
\pgfpathmoveto{\pgfqpoint{3.206071in}{3.289675in}}%
\pgfpathlineto{\pgfqpoint{3.233018in}{3.274117in}}%
\pgfpathlineto{\pgfqpoint{3.233018in}{3.305233in}}%
\pgfpathlineto{\pgfqpoint{3.206071in}{3.320791in}}%
\pgfpathlineto{\pgfqpoint{3.206071in}{3.289675in}}%
\pgfpathclose%
\pgfusepath{fill}%
\end{pgfscope}%
\begin{pgfscope}%
\pgfpathrectangle{\pgfqpoint{0.765000in}{0.660000in}}{\pgfqpoint{4.620000in}{4.620000in}}%
\pgfusepath{clip}%
\pgfsetbuttcap%
\pgfsetroundjoin%
\definecolor{currentfill}{rgb}{1.000000,0.894118,0.788235}%
\pgfsetfillcolor{currentfill}%
\pgfsetlinewidth{0.000000pt}%
\definecolor{currentstroke}{rgb}{1.000000,0.894118,0.788235}%
\pgfsetstrokecolor{currentstroke}%
\pgfsetdash{}{0pt}%
\pgfpathmoveto{\pgfqpoint{3.220833in}{3.327632in}}%
\pgfpathlineto{\pgfqpoint{3.193886in}{3.312074in}}%
\pgfpathlineto{\pgfqpoint{3.206071in}{3.289675in}}%
\pgfpathlineto{\pgfqpoint{3.233018in}{3.305233in}}%
\pgfpathlineto{\pgfqpoint{3.220833in}{3.327632in}}%
\pgfpathclose%
\pgfusepath{fill}%
\end{pgfscope}%
\begin{pgfscope}%
\pgfpathrectangle{\pgfqpoint{0.765000in}{0.660000in}}{\pgfqpoint{4.620000in}{4.620000in}}%
\pgfusepath{clip}%
\pgfsetbuttcap%
\pgfsetroundjoin%
\definecolor{currentfill}{rgb}{1.000000,0.894118,0.788235}%
\pgfsetfillcolor{currentfill}%
\pgfsetlinewidth{0.000000pt}%
\definecolor{currentstroke}{rgb}{1.000000,0.894118,0.788235}%
\pgfsetstrokecolor{currentstroke}%
\pgfsetdash{}{0pt}%
\pgfpathmoveto{\pgfqpoint{3.247780in}{3.312074in}}%
\pgfpathlineto{\pgfqpoint{3.220833in}{3.327632in}}%
\pgfpathlineto{\pgfqpoint{3.233018in}{3.305233in}}%
\pgfpathlineto{\pgfqpoint{3.259965in}{3.289675in}}%
\pgfpathlineto{\pgfqpoint{3.247780in}{3.312074in}}%
\pgfpathclose%
\pgfusepath{fill}%
\end{pgfscope}%
\begin{pgfscope}%
\pgfpathrectangle{\pgfqpoint{0.765000in}{0.660000in}}{\pgfqpoint{4.620000in}{4.620000in}}%
\pgfusepath{clip}%
\pgfsetbuttcap%
\pgfsetroundjoin%
\definecolor{currentfill}{rgb}{1.000000,0.894118,0.788235}%
\pgfsetfillcolor{currentfill}%
\pgfsetlinewidth{0.000000pt}%
\definecolor{currentstroke}{rgb}{1.000000,0.894118,0.788235}%
\pgfsetstrokecolor{currentstroke}%
\pgfsetdash{}{0pt}%
\pgfpathmoveto{\pgfqpoint{3.220833in}{3.327632in}}%
\pgfpathlineto{\pgfqpoint{3.220833in}{3.358748in}}%
\pgfpathlineto{\pgfqpoint{3.233018in}{3.336349in}}%
\pgfpathlineto{\pgfqpoint{3.259965in}{3.289675in}}%
\pgfpathlineto{\pgfqpoint{3.220833in}{3.327632in}}%
\pgfpathclose%
\pgfusepath{fill}%
\end{pgfscope}%
\begin{pgfscope}%
\pgfpathrectangle{\pgfqpoint{0.765000in}{0.660000in}}{\pgfqpoint{4.620000in}{4.620000in}}%
\pgfusepath{clip}%
\pgfsetbuttcap%
\pgfsetroundjoin%
\definecolor{currentfill}{rgb}{1.000000,0.894118,0.788235}%
\pgfsetfillcolor{currentfill}%
\pgfsetlinewidth{0.000000pt}%
\definecolor{currentstroke}{rgb}{1.000000,0.894118,0.788235}%
\pgfsetstrokecolor{currentstroke}%
\pgfsetdash{}{0pt}%
\pgfpathmoveto{\pgfqpoint{3.247780in}{3.312074in}}%
\pgfpathlineto{\pgfqpoint{3.247780in}{3.343190in}}%
\pgfpathlineto{\pgfqpoint{3.259965in}{3.320791in}}%
\pgfpathlineto{\pgfqpoint{3.233018in}{3.305233in}}%
\pgfpathlineto{\pgfqpoint{3.247780in}{3.312074in}}%
\pgfpathclose%
\pgfusepath{fill}%
\end{pgfscope}%
\begin{pgfscope}%
\pgfpathrectangle{\pgfqpoint{0.765000in}{0.660000in}}{\pgfqpoint{4.620000in}{4.620000in}}%
\pgfusepath{clip}%
\pgfsetbuttcap%
\pgfsetroundjoin%
\definecolor{currentfill}{rgb}{1.000000,0.894118,0.788235}%
\pgfsetfillcolor{currentfill}%
\pgfsetlinewidth{0.000000pt}%
\definecolor{currentstroke}{rgb}{1.000000,0.894118,0.788235}%
\pgfsetstrokecolor{currentstroke}%
\pgfsetdash{}{0pt}%
\pgfpathmoveto{\pgfqpoint{3.193886in}{3.312074in}}%
\pgfpathlineto{\pgfqpoint{3.220833in}{3.296516in}}%
\pgfpathlineto{\pgfqpoint{3.233018in}{3.274117in}}%
\pgfpathlineto{\pgfqpoint{3.206071in}{3.289675in}}%
\pgfpathlineto{\pgfqpoint{3.193886in}{3.312074in}}%
\pgfpathclose%
\pgfusepath{fill}%
\end{pgfscope}%
\begin{pgfscope}%
\pgfpathrectangle{\pgfqpoint{0.765000in}{0.660000in}}{\pgfqpoint{4.620000in}{4.620000in}}%
\pgfusepath{clip}%
\pgfsetbuttcap%
\pgfsetroundjoin%
\definecolor{currentfill}{rgb}{1.000000,0.894118,0.788235}%
\pgfsetfillcolor{currentfill}%
\pgfsetlinewidth{0.000000pt}%
\definecolor{currentstroke}{rgb}{1.000000,0.894118,0.788235}%
\pgfsetstrokecolor{currentstroke}%
\pgfsetdash{}{0pt}%
\pgfpathmoveto{\pgfqpoint{3.220833in}{3.296516in}}%
\pgfpathlineto{\pgfqpoint{3.247780in}{3.312074in}}%
\pgfpathlineto{\pgfqpoint{3.259965in}{3.289675in}}%
\pgfpathlineto{\pgfqpoint{3.233018in}{3.274117in}}%
\pgfpathlineto{\pgfqpoint{3.220833in}{3.296516in}}%
\pgfpathclose%
\pgfusepath{fill}%
\end{pgfscope}%
\begin{pgfscope}%
\pgfpathrectangle{\pgfqpoint{0.765000in}{0.660000in}}{\pgfqpoint{4.620000in}{4.620000in}}%
\pgfusepath{clip}%
\pgfsetbuttcap%
\pgfsetroundjoin%
\definecolor{currentfill}{rgb}{1.000000,0.894118,0.788235}%
\pgfsetfillcolor{currentfill}%
\pgfsetlinewidth{0.000000pt}%
\definecolor{currentstroke}{rgb}{1.000000,0.894118,0.788235}%
\pgfsetstrokecolor{currentstroke}%
\pgfsetdash{}{0pt}%
\pgfpathmoveto{\pgfqpoint{3.220833in}{3.358748in}}%
\pgfpathlineto{\pgfqpoint{3.193886in}{3.343190in}}%
\pgfpathlineto{\pgfqpoint{3.206071in}{3.320791in}}%
\pgfpathlineto{\pgfqpoint{3.233018in}{3.336349in}}%
\pgfpathlineto{\pgfqpoint{3.220833in}{3.358748in}}%
\pgfpathclose%
\pgfusepath{fill}%
\end{pgfscope}%
\begin{pgfscope}%
\pgfpathrectangle{\pgfqpoint{0.765000in}{0.660000in}}{\pgfqpoint{4.620000in}{4.620000in}}%
\pgfusepath{clip}%
\pgfsetbuttcap%
\pgfsetroundjoin%
\definecolor{currentfill}{rgb}{1.000000,0.894118,0.788235}%
\pgfsetfillcolor{currentfill}%
\pgfsetlinewidth{0.000000pt}%
\definecolor{currentstroke}{rgb}{1.000000,0.894118,0.788235}%
\pgfsetstrokecolor{currentstroke}%
\pgfsetdash{}{0pt}%
\pgfpathmoveto{\pgfqpoint{3.247780in}{3.343190in}}%
\pgfpathlineto{\pgfqpoint{3.220833in}{3.358748in}}%
\pgfpathlineto{\pgfqpoint{3.233018in}{3.336349in}}%
\pgfpathlineto{\pgfqpoint{3.259965in}{3.320791in}}%
\pgfpathlineto{\pgfqpoint{3.247780in}{3.343190in}}%
\pgfpathclose%
\pgfusepath{fill}%
\end{pgfscope}%
\begin{pgfscope}%
\pgfpathrectangle{\pgfqpoint{0.765000in}{0.660000in}}{\pgfqpoint{4.620000in}{4.620000in}}%
\pgfusepath{clip}%
\pgfsetbuttcap%
\pgfsetroundjoin%
\definecolor{currentfill}{rgb}{1.000000,0.894118,0.788235}%
\pgfsetfillcolor{currentfill}%
\pgfsetlinewidth{0.000000pt}%
\definecolor{currentstroke}{rgb}{1.000000,0.894118,0.788235}%
\pgfsetstrokecolor{currentstroke}%
\pgfsetdash{}{0pt}%
\pgfpathmoveto{\pgfqpoint{3.193886in}{3.312074in}}%
\pgfpathlineto{\pgfqpoint{3.193886in}{3.343190in}}%
\pgfpathlineto{\pgfqpoint{3.206071in}{3.320791in}}%
\pgfpathlineto{\pgfqpoint{3.233018in}{3.274117in}}%
\pgfpathlineto{\pgfqpoint{3.193886in}{3.312074in}}%
\pgfpathclose%
\pgfusepath{fill}%
\end{pgfscope}%
\begin{pgfscope}%
\pgfpathrectangle{\pgfqpoint{0.765000in}{0.660000in}}{\pgfqpoint{4.620000in}{4.620000in}}%
\pgfusepath{clip}%
\pgfsetbuttcap%
\pgfsetroundjoin%
\definecolor{currentfill}{rgb}{1.000000,0.894118,0.788235}%
\pgfsetfillcolor{currentfill}%
\pgfsetlinewidth{0.000000pt}%
\definecolor{currentstroke}{rgb}{1.000000,0.894118,0.788235}%
\pgfsetstrokecolor{currentstroke}%
\pgfsetdash{}{0pt}%
\pgfpathmoveto{\pgfqpoint{3.220833in}{3.296516in}}%
\pgfpathlineto{\pgfqpoint{3.220833in}{3.327632in}}%
\pgfpathlineto{\pgfqpoint{3.233018in}{3.305233in}}%
\pgfpathlineto{\pgfqpoint{3.206071in}{3.289675in}}%
\pgfpathlineto{\pgfqpoint{3.220833in}{3.296516in}}%
\pgfpathclose%
\pgfusepath{fill}%
\end{pgfscope}%
\begin{pgfscope}%
\pgfpathrectangle{\pgfqpoint{0.765000in}{0.660000in}}{\pgfqpoint{4.620000in}{4.620000in}}%
\pgfusepath{clip}%
\pgfsetbuttcap%
\pgfsetroundjoin%
\definecolor{currentfill}{rgb}{1.000000,0.894118,0.788235}%
\pgfsetfillcolor{currentfill}%
\pgfsetlinewidth{0.000000pt}%
\definecolor{currentstroke}{rgb}{1.000000,0.894118,0.788235}%
\pgfsetstrokecolor{currentstroke}%
\pgfsetdash{}{0pt}%
\pgfpathmoveto{\pgfqpoint{3.220833in}{3.327632in}}%
\pgfpathlineto{\pgfqpoint{3.247780in}{3.343190in}}%
\pgfpathlineto{\pgfqpoint{3.259965in}{3.320791in}}%
\pgfpathlineto{\pgfqpoint{3.233018in}{3.305233in}}%
\pgfpathlineto{\pgfqpoint{3.220833in}{3.327632in}}%
\pgfpathclose%
\pgfusepath{fill}%
\end{pgfscope}%
\begin{pgfscope}%
\pgfpathrectangle{\pgfqpoint{0.765000in}{0.660000in}}{\pgfqpoint{4.620000in}{4.620000in}}%
\pgfusepath{clip}%
\pgfsetbuttcap%
\pgfsetroundjoin%
\definecolor{currentfill}{rgb}{1.000000,0.894118,0.788235}%
\pgfsetfillcolor{currentfill}%
\pgfsetlinewidth{0.000000pt}%
\definecolor{currentstroke}{rgb}{1.000000,0.894118,0.788235}%
\pgfsetstrokecolor{currentstroke}%
\pgfsetdash{}{0pt}%
\pgfpathmoveto{\pgfqpoint{3.193886in}{3.343190in}}%
\pgfpathlineto{\pgfqpoint{3.220833in}{3.327632in}}%
\pgfpathlineto{\pgfqpoint{3.233018in}{3.305233in}}%
\pgfpathlineto{\pgfqpoint{3.206071in}{3.320791in}}%
\pgfpathlineto{\pgfqpoint{3.193886in}{3.343190in}}%
\pgfpathclose%
\pgfusepath{fill}%
\end{pgfscope}%
\begin{pgfscope}%
\pgfpathrectangle{\pgfqpoint{0.765000in}{0.660000in}}{\pgfqpoint{4.620000in}{4.620000in}}%
\pgfusepath{clip}%
\pgfsetbuttcap%
\pgfsetroundjoin%
\definecolor{currentfill}{rgb}{1.000000,0.894118,0.788235}%
\pgfsetfillcolor{currentfill}%
\pgfsetlinewidth{0.000000pt}%
\definecolor{currentstroke}{rgb}{1.000000,0.894118,0.788235}%
\pgfsetstrokecolor{currentstroke}%
\pgfsetdash{}{0pt}%
\pgfpathmoveto{\pgfqpoint{3.220833in}{3.327632in}}%
\pgfpathlineto{\pgfqpoint{3.193886in}{3.312074in}}%
\pgfpathlineto{\pgfqpoint{3.220833in}{3.296516in}}%
\pgfpathlineto{\pgfqpoint{3.247780in}{3.312074in}}%
\pgfpathlineto{\pgfqpoint{3.220833in}{3.327632in}}%
\pgfpathclose%
\pgfusepath{fill}%
\end{pgfscope}%
\begin{pgfscope}%
\pgfpathrectangle{\pgfqpoint{0.765000in}{0.660000in}}{\pgfqpoint{4.620000in}{4.620000in}}%
\pgfusepath{clip}%
\pgfsetbuttcap%
\pgfsetroundjoin%
\definecolor{currentfill}{rgb}{1.000000,0.894118,0.788235}%
\pgfsetfillcolor{currentfill}%
\pgfsetlinewidth{0.000000pt}%
\definecolor{currentstroke}{rgb}{1.000000,0.894118,0.788235}%
\pgfsetstrokecolor{currentstroke}%
\pgfsetdash{}{0pt}%
\pgfpathmoveto{\pgfqpoint{3.220833in}{3.327632in}}%
\pgfpathlineto{\pgfqpoint{3.193886in}{3.312074in}}%
\pgfpathlineto{\pgfqpoint{3.193886in}{3.343190in}}%
\pgfpathlineto{\pgfqpoint{3.220833in}{3.358748in}}%
\pgfpathlineto{\pgfqpoint{3.220833in}{3.327632in}}%
\pgfpathclose%
\pgfusepath{fill}%
\end{pgfscope}%
\begin{pgfscope}%
\pgfpathrectangle{\pgfqpoint{0.765000in}{0.660000in}}{\pgfqpoint{4.620000in}{4.620000in}}%
\pgfusepath{clip}%
\pgfsetbuttcap%
\pgfsetroundjoin%
\definecolor{currentfill}{rgb}{1.000000,0.894118,0.788235}%
\pgfsetfillcolor{currentfill}%
\pgfsetlinewidth{0.000000pt}%
\definecolor{currentstroke}{rgb}{1.000000,0.894118,0.788235}%
\pgfsetstrokecolor{currentstroke}%
\pgfsetdash{}{0pt}%
\pgfpathmoveto{\pgfqpoint{3.220833in}{3.327632in}}%
\pgfpathlineto{\pgfqpoint{3.247780in}{3.312074in}}%
\pgfpathlineto{\pgfqpoint{3.247780in}{3.343190in}}%
\pgfpathlineto{\pgfqpoint{3.220833in}{3.358748in}}%
\pgfpathlineto{\pgfqpoint{3.220833in}{3.327632in}}%
\pgfpathclose%
\pgfusepath{fill}%
\end{pgfscope}%
\begin{pgfscope}%
\pgfpathrectangle{\pgfqpoint{0.765000in}{0.660000in}}{\pgfqpoint{4.620000in}{4.620000in}}%
\pgfusepath{clip}%
\pgfsetbuttcap%
\pgfsetroundjoin%
\definecolor{currentfill}{rgb}{1.000000,0.894118,0.788235}%
\pgfsetfillcolor{currentfill}%
\pgfsetlinewidth{0.000000pt}%
\definecolor{currentstroke}{rgb}{1.000000,0.894118,0.788235}%
\pgfsetstrokecolor{currentstroke}%
\pgfsetdash{}{0pt}%
\pgfpathmoveto{\pgfqpoint{3.176663in}{3.599980in}}%
\pgfpathlineto{\pgfqpoint{3.149716in}{3.584422in}}%
\pgfpathlineto{\pgfqpoint{3.176663in}{3.568864in}}%
\pgfpathlineto{\pgfqpoint{3.203610in}{3.584422in}}%
\pgfpathlineto{\pgfqpoint{3.176663in}{3.599980in}}%
\pgfpathclose%
\pgfusepath{fill}%
\end{pgfscope}%
\begin{pgfscope}%
\pgfpathrectangle{\pgfqpoint{0.765000in}{0.660000in}}{\pgfqpoint{4.620000in}{4.620000in}}%
\pgfusepath{clip}%
\pgfsetbuttcap%
\pgfsetroundjoin%
\definecolor{currentfill}{rgb}{1.000000,0.894118,0.788235}%
\pgfsetfillcolor{currentfill}%
\pgfsetlinewidth{0.000000pt}%
\definecolor{currentstroke}{rgb}{1.000000,0.894118,0.788235}%
\pgfsetstrokecolor{currentstroke}%
\pgfsetdash{}{0pt}%
\pgfpathmoveto{\pgfqpoint{3.176663in}{3.599980in}}%
\pgfpathlineto{\pgfqpoint{3.149716in}{3.584422in}}%
\pgfpathlineto{\pgfqpoint{3.149716in}{3.615538in}}%
\pgfpathlineto{\pgfqpoint{3.176663in}{3.631096in}}%
\pgfpathlineto{\pgfqpoint{3.176663in}{3.599980in}}%
\pgfpathclose%
\pgfusepath{fill}%
\end{pgfscope}%
\begin{pgfscope}%
\pgfpathrectangle{\pgfqpoint{0.765000in}{0.660000in}}{\pgfqpoint{4.620000in}{4.620000in}}%
\pgfusepath{clip}%
\pgfsetbuttcap%
\pgfsetroundjoin%
\definecolor{currentfill}{rgb}{1.000000,0.894118,0.788235}%
\pgfsetfillcolor{currentfill}%
\pgfsetlinewidth{0.000000pt}%
\definecolor{currentstroke}{rgb}{1.000000,0.894118,0.788235}%
\pgfsetstrokecolor{currentstroke}%
\pgfsetdash{}{0pt}%
\pgfpathmoveto{\pgfqpoint{3.176663in}{3.599980in}}%
\pgfpathlineto{\pgfqpoint{3.203610in}{3.584422in}}%
\pgfpathlineto{\pgfqpoint{3.203610in}{3.615538in}}%
\pgfpathlineto{\pgfqpoint{3.176663in}{3.631096in}}%
\pgfpathlineto{\pgfqpoint{3.176663in}{3.599980in}}%
\pgfpathclose%
\pgfusepath{fill}%
\end{pgfscope}%
\begin{pgfscope}%
\pgfpathrectangle{\pgfqpoint{0.765000in}{0.660000in}}{\pgfqpoint{4.620000in}{4.620000in}}%
\pgfusepath{clip}%
\pgfsetbuttcap%
\pgfsetroundjoin%
\definecolor{currentfill}{rgb}{1.000000,0.894118,0.788235}%
\pgfsetfillcolor{currentfill}%
\pgfsetlinewidth{0.000000pt}%
\definecolor{currentstroke}{rgb}{1.000000,0.894118,0.788235}%
\pgfsetstrokecolor{currentstroke}%
\pgfsetdash{}{0pt}%
\pgfpathmoveto{\pgfqpoint{3.220833in}{3.358748in}}%
\pgfpathlineto{\pgfqpoint{3.193886in}{3.343190in}}%
\pgfpathlineto{\pgfqpoint{3.220833in}{3.327632in}}%
\pgfpathlineto{\pgfqpoint{3.247780in}{3.343190in}}%
\pgfpathlineto{\pgfqpoint{3.220833in}{3.358748in}}%
\pgfpathclose%
\pgfusepath{fill}%
\end{pgfscope}%
\begin{pgfscope}%
\pgfpathrectangle{\pgfqpoint{0.765000in}{0.660000in}}{\pgfqpoint{4.620000in}{4.620000in}}%
\pgfusepath{clip}%
\pgfsetbuttcap%
\pgfsetroundjoin%
\definecolor{currentfill}{rgb}{1.000000,0.894118,0.788235}%
\pgfsetfillcolor{currentfill}%
\pgfsetlinewidth{0.000000pt}%
\definecolor{currentstroke}{rgb}{1.000000,0.894118,0.788235}%
\pgfsetstrokecolor{currentstroke}%
\pgfsetdash{}{0pt}%
\pgfpathmoveto{\pgfqpoint{3.220833in}{3.296516in}}%
\pgfpathlineto{\pgfqpoint{3.247780in}{3.312074in}}%
\pgfpathlineto{\pgfqpoint{3.247780in}{3.343190in}}%
\pgfpathlineto{\pgfqpoint{3.220833in}{3.327632in}}%
\pgfpathlineto{\pgfqpoint{3.220833in}{3.296516in}}%
\pgfpathclose%
\pgfusepath{fill}%
\end{pgfscope}%
\begin{pgfscope}%
\pgfpathrectangle{\pgfqpoint{0.765000in}{0.660000in}}{\pgfqpoint{4.620000in}{4.620000in}}%
\pgfusepath{clip}%
\pgfsetbuttcap%
\pgfsetroundjoin%
\definecolor{currentfill}{rgb}{1.000000,0.894118,0.788235}%
\pgfsetfillcolor{currentfill}%
\pgfsetlinewidth{0.000000pt}%
\definecolor{currentstroke}{rgb}{1.000000,0.894118,0.788235}%
\pgfsetstrokecolor{currentstroke}%
\pgfsetdash{}{0pt}%
\pgfpathmoveto{\pgfqpoint{3.193886in}{3.312074in}}%
\pgfpathlineto{\pgfqpoint{3.220833in}{3.296516in}}%
\pgfpathlineto{\pgfqpoint{3.220833in}{3.327632in}}%
\pgfpathlineto{\pgfqpoint{3.193886in}{3.343190in}}%
\pgfpathlineto{\pgfqpoint{3.193886in}{3.312074in}}%
\pgfpathclose%
\pgfusepath{fill}%
\end{pgfscope}%
\begin{pgfscope}%
\pgfpathrectangle{\pgfqpoint{0.765000in}{0.660000in}}{\pgfqpoint{4.620000in}{4.620000in}}%
\pgfusepath{clip}%
\pgfsetbuttcap%
\pgfsetroundjoin%
\definecolor{currentfill}{rgb}{1.000000,0.894118,0.788235}%
\pgfsetfillcolor{currentfill}%
\pgfsetlinewidth{0.000000pt}%
\definecolor{currentstroke}{rgb}{1.000000,0.894118,0.788235}%
\pgfsetstrokecolor{currentstroke}%
\pgfsetdash{}{0pt}%
\pgfpathmoveto{\pgfqpoint{3.176663in}{3.631096in}}%
\pgfpathlineto{\pgfqpoint{3.149716in}{3.615538in}}%
\pgfpathlineto{\pgfqpoint{3.176663in}{3.599980in}}%
\pgfpathlineto{\pgfqpoint{3.203610in}{3.615538in}}%
\pgfpathlineto{\pgfqpoint{3.176663in}{3.631096in}}%
\pgfpathclose%
\pgfusepath{fill}%
\end{pgfscope}%
\begin{pgfscope}%
\pgfpathrectangle{\pgfqpoint{0.765000in}{0.660000in}}{\pgfqpoint{4.620000in}{4.620000in}}%
\pgfusepath{clip}%
\pgfsetbuttcap%
\pgfsetroundjoin%
\definecolor{currentfill}{rgb}{1.000000,0.894118,0.788235}%
\pgfsetfillcolor{currentfill}%
\pgfsetlinewidth{0.000000pt}%
\definecolor{currentstroke}{rgb}{1.000000,0.894118,0.788235}%
\pgfsetstrokecolor{currentstroke}%
\pgfsetdash{}{0pt}%
\pgfpathmoveto{\pgfqpoint{3.176663in}{3.568864in}}%
\pgfpathlineto{\pgfqpoint{3.203610in}{3.584422in}}%
\pgfpathlineto{\pgfqpoint{3.203610in}{3.615538in}}%
\pgfpathlineto{\pgfqpoint{3.176663in}{3.599980in}}%
\pgfpathlineto{\pgfqpoint{3.176663in}{3.568864in}}%
\pgfpathclose%
\pgfusepath{fill}%
\end{pgfscope}%
\begin{pgfscope}%
\pgfpathrectangle{\pgfqpoint{0.765000in}{0.660000in}}{\pgfqpoint{4.620000in}{4.620000in}}%
\pgfusepath{clip}%
\pgfsetbuttcap%
\pgfsetroundjoin%
\definecolor{currentfill}{rgb}{1.000000,0.894118,0.788235}%
\pgfsetfillcolor{currentfill}%
\pgfsetlinewidth{0.000000pt}%
\definecolor{currentstroke}{rgb}{1.000000,0.894118,0.788235}%
\pgfsetstrokecolor{currentstroke}%
\pgfsetdash{}{0pt}%
\pgfpathmoveto{\pgfqpoint{3.149716in}{3.584422in}}%
\pgfpathlineto{\pgfqpoint{3.176663in}{3.568864in}}%
\pgfpathlineto{\pgfqpoint{3.176663in}{3.599980in}}%
\pgfpathlineto{\pgfqpoint{3.149716in}{3.615538in}}%
\pgfpathlineto{\pgfqpoint{3.149716in}{3.584422in}}%
\pgfpathclose%
\pgfusepath{fill}%
\end{pgfscope}%
\begin{pgfscope}%
\pgfpathrectangle{\pgfqpoint{0.765000in}{0.660000in}}{\pgfqpoint{4.620000in}{4.620000in}}%
\pgfusepath{clip}%
\pgfsetbuttcap%
\pgfsetroundjoin%
\definecolor{currentfill}{rgb}{1.000000,0.894118,0.788235}%
\pgfsetfillcolor{currentfill}%
\pgfsetlinewidth{0.000000pt}%
\definecolor{currentstroke}{rgb}{1.000000,0.894118,0.788235}%
\pgfsetstrokecolor{currentstroke}%
\pgfsetdash{}{0pt}%
\pgfpathmoveto{\pgfqpoint{3.220833in}{3.327632in}}%
\pgfpathlineto{\pgfqpoint{3.193886in}{3.312074in}}%
\pgfpathlineto{\pgfqpoint{3.149716in}{3.584422in}}%
\pgfpathlineto{\pgfqpoint{3.176663in}{3.599980in}}%
\pgfpathlineto{\pgfqpoint{3.220833in}{3.327632in}}%
\pgfpathclose%
\pgfusepath{fill}%
\end{pgfscope}%
\begin{pgfscope}%
\pgfpathrectangle{\pgfqpoint{0.765000in}{0.660000in}}{\pgfqpoint{4.620000in}{4.620000in}}%
\pgfusepath{clip}%
\pgfsetbuttcap%
\pgfsetroundjoin%
\definecolor{currentfill}{rgb}{1.000000,0.894118,0.788235}%
\pgfsetfillcolor{currentfill}%
\pgfsetlinewidth{0.000000pt}%
\definecolor{currentstroke}{rgb}{1.000000,0.894118,0.788235}%
\pgfsetstrokecolor{currentstroke}%
\pgfsetdash{}{0pt}%
\pgfpathmoveto{\pgfqpoint{3.247780in}{3.312074in}}%
\pgfpathlineto{\pgfqpoint{3.220833in}{3.327632in}}%
\pgfpathlineto{\pgfqpoint{3.176663in}{3.599980in}}%
\pgfpathlineto{\pgfqpoint{3.203610in}{3.584422in}}%
\pgfpathlineto{\pgfqpoint{3.247780in}{3.312074in}}%
\pgfpathclose%
\pgfusepath{fill}%
\end{pgfscope}%
\begin{pgfscope}%
\pgfpathrectangle{\pgfqpoint{0.765000in}{0.660000in}}{\pgfqpoint{4.620000in}{4.620000in}}%
\pgfusepath{clip}%
\pgfsetbuttcap%
\pgfsetroundjoin%
\definecolor{currentfill}{rgb}{1.000000,0.894118,0.788235}%
\pgfsetfillcolor{currentfill}%
\pgfsetlinewidth{0.000000pt}%
\definecolor{currentstroke}{rgb}{1.000000,0.894118,0.788235}%
\pgfsetstrokecolor{currentstroke}%
\pgfsetdash{}{0pt}%
\pgfpathmoveto{\pgfqpoint{3.220833in}{3.327632in}}%
\pgfpathlineto{\pgfqpoint{3.220833in}{3.358748in}}%
\pgfpathlineto{\pgfqpoint{3.176663in}{3.631096in}}%
\pgfpathlineto{\pgfqpoint{3.203610in}{3.584422in}}%
\pgfpathlineto{\pgfqpoint{3.220833in}{3.327632in}}%
\pgfpathclose%
\pgfusepath{fill}%
\end{pgfscope}%
\begin{pgfscope}%
\pgfpathrectangle{\pgfqpoint{0.765000in}{0.660000in}}{\pgfqpoint{4.620000in}{4.620000in}}%
\pgfusepath{clip}%
\pgfsetbuttcap%
\pgfsetroundjoin%
\definecolor{currentfill}{rgb}{1.000000,0.894118,0.788235}%
\pgfsetfillcolor{currentfill}%
\pgfsetlinewidth{0.000000pt}%
\definecolor{currentstroke}{rgb}{1.000000,0.894118,0.788235}%
\pgfsetstrokecolor{currentstroke}%
\pgfsetdash{}{0pt}%
\pgfpathmoveto{\pgfqpoint{3.247780in}{3.312074in}}%
\pgfpathlineto{\pgfqpoint{3.247780in}{3.343190in}}%
\pgfpathlineto{\pgfqpoint{3.203610in}{3.615538in}}%
\pgfpathlineto{\pgfqpoint{3.176663in}{3.599980in}}%
\pgfpathlineto{\pgfqpoint{3.247780in}{3.312074in}}%
\pgfpathclose%
\pgfusepath{fill}%
\end{pgfscope}%
\begin{pgfscope}%
\pgfpathrectangle{\pgfqpoint{0.765000in}{0.660000in}}{\pgfqpoint{4.620000in}{4.620000in}}%
\pgfusepath{clip}%
\pgfsetbuttcap%
\pgfsetroundjoin%
\definecolor{currentfill}{rgb}{1.000000,0.894118,0.788235}%
\pgfsetfillcolor{currentfill}%
\pgfsetlinewidth{0.000000pt}%
\definecolor{currentstroke}{rgb}{1.000000,0.894118,0.788235}%
\pgfsetstrokecolor{currentstroke}%
\pgfsetdash{}{0pt}%
\pgfpathmoveto{\pgfqpoint{3.193886in}{3.312074in}}%
\pgfpathlineto{\pgfqpoint{3.220833in}{3.296516in}}%
\pgfpathlineto{\pgfqpoint{3.176663in}{3.568864in}}%
\pgfpathlineto{\pgfqpoint{3.149716in}{3.584422in}}%
\pgfpathlineto{\pgfqpoint{3.193886in}{3.312074in}}%
\pgfpathclose%
\pgfusepath{fill}%
\end{pgfscope}%
\begin{pgfscope}%
\pgfpathrectangle{\pgfqpoint{0.765000in}{0.660000in}}{\pgfqpoint{4.620000in}{4.620000in}}%
\pgfusepath{clip}%
\pgfsetbuttcap%
\pgfsetroundjoin%
\definecolor{currentfill}{rgb}{1.000000,0.894118,0.788235}%
\pgfsetfillcolor{currentfill}%
\pgfsetlinewidth{0.000000pt}%
\definecolor{currentstroke}{rgb}{1.000000,0.894118,0.788235}%
\pgfsetstrokecolor{currentstroke}%
\pgfsetdash{}{0pt}%
\pgfpathmoveto{\pgfqpoint{3.220833in}{3.296516in}}%
\pgfpathlineto{\pgfqpoint{3.247780in}{3.312074in}}%
\pgfpathlineto{\pgfqpoint{3.203610in}{3.584422in}}%
\pgfpathlineto{\pgfqpoint{3.176663in}{3.568864in}}%
\pgfpathlineto{\pgfqpoint{3.220833in}{3.296516in}}%
\pgfpathclose%
\pgfusepath{fill}%
\end{pgfscope}%
\begin{pgfscope}%
\pgfpathrectangle{\pgfqpoint{0.765000in}{0.660000in}}{\pgfqpoint{4.620000in}{4.620000in}}%
\pgfusepath{clip}%
\pgfsetbuttcap%
\pgfsetroundjoin%
\definecolor{currentfill}{rgb}{1.000000,0.894118,0.788235}%
\pgfsetfillcolor{currentfill}%
\pgfsetlinewidth{0.000000pt}%
\definecolor{currentstroke}{rgb}{1.000000,0.894118,0.788235}%
\pgfsetstrokecolor{currentstroke}%
\pgfsetdash{}{0pt}%
\pgfpathmoveto{\pgfqpoint{3.220833in}{3.358748in}}%
\pgfpathlineto{\pgfqpoint{3.193886in}{3.343190in}}%
\pgfpathlineto{\pgfqpoint{3.149716in}{3.615538in}}%
\pgfpathlineto{\pgfqpoint{3.176663in}{3.631096in}}%
\pgfpathlineto{\pgfqpoint{3.220833in}{3.358748in}}%
\pgfpathclose%
\pgfusepath{fill}%
\end{pgfscope}%
\begin{pgfscope}%
\pgfpathrectangle{\pgfqpoint{0.765000in}{0.660000in}}{\pgfqpoint{4.620000in}{4.620000in}}%
\pgfusepath{clip}%
\pgfsetbuttcap%
\pgfsetroundjoin%
\definecolor{currentfill}{rgb}{1.000000,0.894118,0.788235}%
\pgfsetfillcolor{currentfill}%
\pgfsetlinewidth{0.000000pt}%
\definecolor{currentstroke}{rgb}{1.000000,0.894118,0.788235}%
\pgfsetstrokecolor{currentstroke}%
\pgfsetdash{}{0pt}%
\pgfpathmoveto{\pgfqpoint{3.247780in}{3.343190in}}%
\pgfpathlineto{\pgfqpoint{3.220833in}{3.358748in}}%
\pgfpathlineto{\pgfqpoint{3.176663in}{3.631096in}}%
\pgfpathlineto{\pgfqpoint{3.203610in}{3.615538in}}%
\pgfpathlineto{\pgfqpoint{3.247780in}{3.343190in}}%
\pgfpathclose%
\pgfusepath{fill}%
\end{pgfscope}%
\begin{pgfscope}%
\pgfpathrectangle{\pgfqpoint{0.765000in}{0.660000in}}{\pgfqpoint{4.620000in}{4.620000in}}%
\pgfusepath{clip}%
\pgfsetbuttcap%
\pgfsetroundjoin%
\definecolor{currentfill}{rgb}{1.000000,0.894118,0.788235}%
\pgfsetfillcolor{currentfill}%
\pgfsetlinewidth{0.000000pt}%
\definecolor{currentstroke}{rgb}{1.000000,0.894118,0.788235}%
\pgfsetstrokecolor{currentstroke}%
\pgfsetdash{}{0pt}%
\pgfpathmoveto{\pgfqpoint{3.193886in}{3.312074in}}%
\pgfpathlineto{\pgfqpoint{3.193886in}{3.343190in}}%
\pgfpathlineto{\pgfqpoint{3.149716in}{3.615538in}}%
\pgfpathlineto{\pgfqpoint{3.176663in}{3.568864in}}%
\pgfpathlineto{\pgfqpoint{3.193886in}{3.312074in}}%
\pgfpathclose%
\pgfusepath{fill}%
\end{pgfscope}%
\begin{pgfscope}%
\pgfpathrectangle{\pgfqpoint{0.765000in}{0.660000in}}{\pgfqpoint{4.620000in}{4.620000in}}%
\pgfusepath{clip}%
\pgfsetbuttcap%
\pgfsetroundjoin%
\definecolor{currentfill}{rgb}{1.000000,0.894118,0.788235}%
\pgfsetfillcolor{currentfill}%
\pgfsetlinewidth{0.000000pt}%
\definecolor{currentstroke}{rgb}{1.000000,0.894118,0.788235}%
\pgfsetstrokecolor{currentstroke}%
\pgfsetdash{}{0pt}%
\pgfpathmoveto{\pgfqpoint{3.220833in}{3.296516in}}%
\pgfpathlineto{\pgfqpoint{3.220833in}{3.327632in}}%
\pgfpathlineto{\pgfqpoint{3.176663in}{3.599980in}}%
\pgfpathlineto{\pgfqpoint{3.149716in}{3.584422in}}%
\pgfpathlineto{\pgfqpoint{3.220833in}{3.296516in}}%
\pgfpathclose%
\pgfusepath{fill}%
\end{pgfscope}%
\begin{pgfscope}%
\pgfpathrectangle{\pgfqpoint{0.765000in}{0.660000in}}{\pgfqpoint{4.620000in}{4.620000in}}%
\pgfusepath{clip}%
\pgfsetbuttcap%
\pgfsetroundjoin%
\definecolor{currentfill}{rgb}{1.000000,0.894118,0.788235}%
\pgfsetfillcolor{currentfill}%
\pgfsetlinewidth{0.000000pt}%
\definecolor{currentstroke}{rgb}{1.000000,0.894118,0.788235}%
\pgfsetstrokecolor{currentstroke}%
\pgfsetdash{}{0pt}%
\pgfpathmoveto{\pgfqpoint{3.220833in}{3.327632in}}%
\pgfpathlineto{\pgfqpoint{3.247780in}{3.343190in}}%
\pgfpathlineto{\pgfqpoint{3.203610in}{3.615538in}}%
\pgfpathlineto{\pgfqpoint{3.176663in}{3.599980in}}%
\pgfpathlineto{\pgfqpoint{3.220833in}{3.327632in}}%
\pgfpathclose%
\pgfusepath{fill}%
\end{pgfscope}%
\begin{pgfscope}%
\pgfpathrectangle{\pgfqpoint{0.765000in}{0.660000in}}{\pgfqpoint{4.620000in}{4.620000in}}%
\pgfusepath{clip}%
\pgfsetbuttcap%
\pgfsetroundjoin%
\definecolor{currentfill}{rgb}{1.000000,0.894118,0.788235}%
\pgfsetfillcolor{currentfill}%
\pgfsetlinewidth{0.000000pt}%
\definecolor{currentstroke}{rgb}{1.000000,0.894118,0.788235}%
\pgfsetstrokecolor{currentstroke}%
\pgfsetdash{}{0pt}%
\pgfpathmoveto{\pgfqpoint{3.193886in}{3.343190in}}%
\pgfpathlineto{\pgfqpoint{3.220833in}{3.327632in}}%
\pgfpathlineto{\pgfqpoint{3.176663in}{3.599980in}}%
\pgfpathlineto{\pgfqpoint{3.149716in}{3.615538in}}%
\pgfpathlineto{\pgfqpoint{3.193886in}{3.343190in}}%
\pgfpathclose%
\pgfusepath{fill}%
\end{pgfscope}%
\begin{pgfscope}%
\pgfpathrectangle{\pgfqpoint{0.765000in}{0.660000in}}{\pgfqpoint{4.620000in}{4.620000in}}%
\pgfusepath{clip}%
\pgfsetbuttcap%
\pgfsetroundjoin%
\definecolor{currentfill}{rgb}{1.000000,0.894118,0.788235}%
\pgfsetfillcolor{currentfill}%
\pgfsetlinewidth{0.000000pt}%
\definecolor{currentstroke}{rgb}{1.000000,0.894118,0.788235}%
\pgfsetstrokecolor{currentstroke}%
\pgfsetdash{}{0pt}%
\pgfpathmoveto{\pgfqpoint{2.595009in}{3.223964in}}%
\pgfpathlineto{\pgfqpoint{2.568062in}{3.208406in}}%
\pgfpathlineto{\pgfqpoint{2.595009in}{3.192848in}}%
\pgfpathlineto{\pgfqpoint{2.621957in}{3.208406in}}%
\pgfpathlineto{\pgfqpoint{2.595009in}{3.223964in}}%
\pgfpathclose%
\pgfusepath{fill}%
\end{pgfscope}%
\begin{pgfscope}%
\pgfpathrectangle{\pgfqpoint{0.765000in}{0.660000in}}{\pgfqpoint{4.620000in}{4.620000in}}%
\pgfusepath{clip}%
\pgfsetbuttcap%
\pgfsetroundjoin%
\definecolor{currentfill}{rgb}{1.000000,0.894118,0.788235}%
\pgfsetfillcolor{currentfill}%
\pgfsetlinewidth{0.000000pt}%
\definecolor{currentstroke}{rgb}{1.000000,0.894118,0.788235}%
\pgfsetstrokecolor{currentstroke}%
\pgfsetdash{}{0pt}%
\pgfpathmoveto{\pgfqpoint{2.595009in}{3.223964in}}%
\pgfpathlineto{\pgfqpoint{2.568062in}{3.208406in}}%
\pgfpathlineto{\pgfqpoint{2.568062in}{3.239522in}}%
\pgfpathlineto{\pgfqpoint{2.595009in}{3.255080in}}%
\pgfpathlineto{\pgfqpoint{2.595009in}{3.223964in}}%
\pgfpathclose%
\pgfusepath{fill}%
\end{pgfscope}%
\begin{pgfscope}%
\pgfpathrectangle{\pgfqpoint{0.765000in}{0.660000in}}{\pgfqpoint{4.620000in}{4.620000in}}%
\pgfusepath{clip}%
\pgfsetbuttcap%
\pgfsetroundjoin%
\definecolor{currentfill}{rgb}{1.000000,0.894118,0.788235}%
\pgfsetfillcolor{currentfill}%
\pgfsetlinewidth{0.000000pt}%
\definecolor{currentstroke}{rgb}{1.000000,0.894118,0.788235}%
\pgfsetstrokecolor{currentstroke}%
\pgfsetdash{}{0pt}%
\pgfpathmoveto{\pgfqpoint{2.595009in}{3.223964in}}%
\pgfpathlineto{\pgfqpoint{2.621957in}{3.208406in}}%
\pgfpathlineto{\pgfqpoint{2.621957in}{3.239522in}}%
\pgfpathlineto{\pgfqpoint{2.595009in}{3.255080in}}%
\pgfpathlineto{\pgfqpoint{2.595009in}{3.223964in}}%
\pgfpathclose%
\pgfusepath{fill}%
\end{pgfscope}%
\begin{pgfscope}%
\pgfpathrectangle{\pgfqpoint{0.765000in}{0.660000in}}{\pgfqpoint{4.620000in}{4.620000in}}%
\pgfusepath{clip}%
\pgfsetbuttcap%
\pgfsetroundjoin%
\definecolor{currentfill}{rgb}{1.000000,0.894118,0.788235}%
\pgfsetfillcolor{currentfill}%
\pgfsetlinewidth{0.000000pt}%
\definecolor{currentstroke}{rgb}{1.000000,0.894118,0.788235}%
\pgfsetstrokecolor{currentstroke}%
\pgfsetdash{}{0pt}%
\pgfpathmoveto{\pgfqpoint{2.537289in}{3.204720in}}%
\pgfpathlineto{\pgfqpoint{2.510341in}{3.189162in}}%
\pgfpathlineto{\pgfqpoint{2.537289in}{3.173604in}}%
\pgfpathlineto{\pgfqpoint{2.564236in}{3.189162in}}%
\pgfpathlineto{\pgfqpoint{2.537289in}{3.204720in}}%
\pgfpathclose%
\pgfusepath{fill}%
\end{pgfscope}%
\begin{pgfscope}%
\pgfpathrectangle{\pgfqpoint{0.765000in}{0.660000in}}{\pgfqpoint{4.620000in}{4.620000in}}%
\pgfusepath{clip}%
\pgfsetbuttcap%
\pgfsetroundjoin%
\definecolor{currentfill}{rgb}{1.000000,0.894118,0.788235}%
\pgfsetfillcolor{currentfill}%
\pgfsetlinewidth{0.000000pt}%
\definecolor{currentstroke}{rgb}{1.000000,0.894118,0.788235}%
\pgfsetstrokecolor{currentstroke}%
\pgfsetdash{}{0pt}%
\pgfpathmoveto{\pgfqpoint{2.537289in}{3.204720in}}%
\pgfpathlineto{\pgfqpoint{2.510341in}{3.189162in}}%
\pgfpathlineto{\pgfqpoint{2.510341in}{3.220278in}}%
\pgfpathlineto{\pgfqpoint{2.537289in}{3.235836in}}%
\pgfpathlineto{\pgfqpoint{2.537289in}{3.204720in}}%
\pgfpathclose%
\pgfusepath{fill}%
\end{pgfscope}%
\begin{pgfscope}%
\pgfpathrectangle{\pgfqpoint{0.765000in}{0.660000in}}{\pgfqpoint{4.620000in}{4.620000in}}%
\pgfusepath{clip}%
\pgfsetbuttcap%
\pgfsetroundjoin%
\definecolor{currentfill}{rgb}{1.000000,0.894118,0.788235}%
\pgfsetfillcolor{currentfill}%
\pgfsetlinewidth{0.000000pt}%
\definecolor{currentstroke}{rgb}{1.000000,0.894118,0.788235}%
\pgfsetstrokecolor{currentstroke}%
\pgfsetdash{}{0pt}%
\pgfpathmoveto{\pgfqpoint{2.537289in}{3.204720in}}%
\pgfpathlineto{\pgfqpoint{2.564236in}{3.189162in}}%
\pgfpathlineto{\pgfqpoint{2.564236in}{3.220278in}}%
\pgfpathlineto{\pgfqpoint{2.537289in}{3.235836in}}%
\pgfpathlineto{\pgfqpoint{2.537289in}{3.204720in}}%
\pgfpathclose%
\pgfusepath{fill}%
\end{pgfscope}%
\begin{pgfscope}%
\pgfpathrectangle{\pgfqpoint{0.765000in}{0.660000in}}{\pgfqpoint{4.620000in}{4.620000in}}%
\pgfusepath{clip}%
\pgfsetbuttcap%
\pgfsetroundjoin%
\definecolor{currentfill}{rgb}{1.000000,0.894118,0.788235}%
\pgfsetfillcolor{currentfill}%
\pgfsetlinewidth{0.000000pt}%
\definecolor{currentstroke}{rgb}{1.000000,0.894118,0.788235}%
\pgfsetstrokecolor{currentstroke}%
\pgfsetdash{}{0pt}%
\pgfpathmoveto{\pgfqpoint{2.595009in}{3.255080in}}%
\pgfpathlineto{\pgfqpoint{2.568062in}{3.239522in}}%
\pgfpathlineto{\pgfqpoint{2.595009in}{3.223964in}}%
\pgfpathlineto{\pgfqpoint{2.621957in}{3.239522in}}%
\pgfpathlineto{\pgfqpoint{2.595009in}{3.255080in}}%
\pgfpathclose%
\pgfusepath{fill}%
\end{pgfscope}%
\begin{pgfscope}%
\pgfpathrectangle{\pgfqpoint{0.765000in}{0.660000in}}{\pgfqpoint{4.620000in}{4.620000in}}%
\pgfusepath{clip}%
\pgfsetbuttcap%
\pgfsetroundjoin%
\definecolor{currentfill}{rgb}{1.000000,0.894118,0.788235}%
\pgfsetfillcolor{currentfill}%
\pgfsetlinewidth{0.000000pt}%
\definecolor{currentstroke}{rgb}{1.000000,0.894118,0.788235}%
\pgfsetstrokecolor{currentstroke}%
\pgfsetdash{}{0pt}%
\pgfpathmoveto{\pgfqpoint{2.595009in}{3.192848in}}%
\pgfpathlineto{\pgfqpoint{2.621957in}{3.208406in}}%
\pgfpathlineto{\pgfqpoint{2.621957in}{3.239522in}}%
\pgfpathlineto{\pgfqpoint{2.595009in}{3.223964in}}%
\pgfpathlineto{\pgfqpoint{2.595009in}{3.192848in}}%
\pgfpathclose%
\pgfusepath{fill}%
\end{pgfscope}%
\begin{pgfscope}%
\pgfpathrectangle{\pgfqpoint{0.765000in}{0.660000in}}{\pgfqpoint{4.620000in}{4.620000in}}%
\pgfusepath{clip}%
\pgfsetbuttcap%
\pgfsetroundjoin%
\definecolor{currentfill}{rgb}{1.000000,0.894118,0.788235}%
\pgfsetfillcolor{currentfill}%
\pgfsetlinewidth{0.000000pt}%
\definecolor{currentstroke}{rgb}{1.000000,0.894118,0.788235}%
\pgfsetstrokecolor{currentstroke}%
\pgfsetdash{}{0pt}%
\pgfpathmoveto{\pgfqpoint{2.568062in}{3.208406in}}%
\pgfpathlineto{\pgfqpoint{2.595009in}{3.192848in}}%
\pgfpathlineto{\pgfqpoint{2.595009in}{3.223964in}}%
\pgfpathlineto{\pgfqpoint{2.568062in}{3.239522in}}%
\pgfpathlineto{\pgfqpoint{2.568062in}{3.208406in}}%
\pgfpathclose%
\pgfusepath{fill}%
\end{pgfscope}%
\begin{pgfscope}%
\pgfpathrectangle{\pgfqpoint{0.765000in}{0.660000in}}{\pgfqpoint{4.620000in}{4.620000in}}%
\pgfusepath{clip}%
\pgfsetbuttcap%
\pgfsetroundjoin%
\definecolor{currentfill}{rgb}{1.000000,0.894118,0.788235}%
\pgfsetfillcolor{currentfill}%
\pgfsetlinewidth{0.000000pt}%
\definecolor{currentstroke}{rgb}{1.000000,0.894118,0.788235}%
\pgfsetstrokecolor{currentstroke}%
\pgfsetdash{}{0pt}%
\pgfpathmoveto{\pgfqpoint{2.537289in}{3.235836in}}%
\pgfpathlineto{\pgfqpoint{2.510341in}{3.220278in}}%
\pgfpathlineto{\pgfqpoint{2.537289in}{3.204720in}}%
\pgfpathlineto{\pgfqpoint{2.564236in}{3.220278in}}%
\pgfpathlineto{\pgfqpoint{2.537289in}{3.235836in}}%
\pgfpathclose%
\pgfusepath{fill}%
\end{pgfscope}%
\begin{pgfscope}%
\pgfpathrectangle{\pgfqpoint{0.765000in}{0.660000in}}{\pgfqpoint{4.620000in}{4.620000in}}%
\pgfusepath{clip}%
\pgfsetbuttcap%
\pgfsetroundjoin%
\definecolor{currentfill}{rgb}{1.000000,0.894118,0.788235}%
\pgfsetfillcolor{currentfill}%
\pgfsetlinewidth{0.000000pt}%
\definecolor{currentstroke}{rgb}{1.000000,0.894118,0.788235}%
\pgfsetstrokecolor{currentstroke}%
\pgfsetdash{}{0pt}%
\pgfpathmoveto{\pgfqpoint{2.537289in}{3.173604in}}%
\pgfpathlineto{\pgfqpoint{2.564236in}{3.189162in}}%
\pgfpathlineto{\pgfqpoint{2.564236in}{3.220278in}}%
\pgfpathlineto{\pgfqpoint{2.537289in}{3.204720in}}%
\pgfpathlineto{\pgfqpoint{2.537289in}{3.173604in}}%
\pgfpathclose%
\pgfusepath{fill}%
\end{pgfscope}%
\begin{pgfscope}%
\pgfpathrectangle{\pgfqpoint{0.765000in}{0.660000in}}{\pgfqpoint{4.620000in}{4.620000in}}%
\pgfusepath{clip}%
\pgfsetbuttcap%
\pgfsetroundjoin%
\definecolor{currentfill}{rgb}{1.000000,0.894118,0.788235}%
\pgfsetfillcolor{currentfill}%
\pgfsetlinewidth{0.000000pt}%
\definecolor{currentstroke}{rgb}{1.000000,0.894118,0.788235}%
\pgfsetstrokecolor{currentstroke}%
\pgfsetdash{}{0pt}%
\pgfpathmoveto{\pgfqpoint{2.510341in}{3.189162in}}%
\pgfpathlineto{\pgfqpoint{2.537289in}{3.173604in}}%
\pgfpathlineto{\pgfqpoint{2.537289in}{3.204720in}}%
\pgfpathlineto{\pgfqpoint{2.510341in}{3.220278in}}%
\pgfpathlineto{\pgfqpoint{2.510341in}{3.189162in}}%
\pgfpathclose%
\pgfusepath{fill}%
\end{pgfscope}%
\begin{pgfscope}%
\pgfpathrectangle{\pgfqpoint{0.765000in}{0.660000in}}{\pgfqpoint{4.620000in}{4.620000in}}%
\pgfusepath{clip}%
\pgfsetbuttcap%
\pgfsetroundjoin%
\definecolor{currentfill}{rgb}{1.000000,0.894118,0.788235}%
\pgfsetfillcolor{currentfill}%
\pgfsetlinewidth{0.000000pt}%
\definecolor{currentstroke}{rgb}{1.000000,0.894118,0.788235}%
\pgfsetstrokecolor{currentstroke}%
\pgfsetdash{}{0pt}%
\pgfpathmoveto{\pgfqpoint{2.595009in}{3.223964in}}%
\pgfpathlineto{\pgfqpoint{2.568062in}{3.208406in}}%
\pgfpathlineto{\pgfqpoint{2.510341in}{3.189162in}}%
\pgfpathlineto{\pgfqpoint{2.537289in}{3.204720in}}%
\pgfpathlineto{\pgfqpoint{2.595009in}{3.223964in}}%
\pgfpathclose%
\pgfusepath{fill}%
\end{pgfscope}%
\begin{pgfscope}%
\pgfpathrectangle{\pgfqpoint{0.765000in}{0.660000in}}{\pgfqpoint{4.620000in}{4.620000in}}%
\pgfusepath{clip}%
\pgfsetbuttcap%
\pgfsetroundjoin%
\definecolor{currentfill}{rgb}{1.000000,0.894118,0.788235}%
\pgfsetfillcolor{currentfill}%
\pgfsetlinewidth{0.000000pt}%
\definecolor{currentstroke}{rgb}{1.000000,0.894118,0.788235}%
\pgfsetstrokecolor{currentstroke}%
\pgfsetdash{}{0pt}%
\pgfpathmoveto{\pgfqpoint{2.621957in}{3.208406in}}%
\pgfpathlineto{\pgfqpoint{2.595009in}{3.223964in}}%
\pgfpathlineto{\pgfqpoint{2.537289in}{3.204720in}}%
\pgfpathlineto{\pgfqpoint{2.564236in}{3.189162in}}%
\pgfpathlineto{\pgfqpoint{2.621957in}{3.208406in}}%
\pgfpathclose%
\pgfusepath{fill}%
\end{pgfscope}%
\begin{pgfscope}%
\pgfpathrectangle{\pgfqpoint{0.765000in}{0.660000in}}{\pgfqpoint{4.620000in}{4.620000in}}%
\pgfusepath{clip}%
\pgfsetbuttcap%
\pgfsetroundjoin%
\definecolor{currentfill}{rgb}{1.000000,0.894118,0.788235}%
\pgfsetfillcolor{currentfill}%
\pgfsetlinewidth{0.000000pt}%
\definecolor{currentstroke}{rgb}{1.000000,0.894118,0.788235}%
\pgfsetstrokecolor{currentstroke}%
\pgfsetdash{}{0pt}%
\pgfpathmoveto{\pgfqpoint{2.595009in}{3.223964in}}%
\pgfpathlineto{\pgfqpoint{2.595009in}{3.255080in}}%
\pgfpathlineto{\pgfqpoint{2.537289in}{3.235836in}}%
\pgfpathlineto{\pgfqpoint{2.564236in}{3.189162in}}%
\pgfpathlineto{\pgfqpoint{2.595009in}{3.223964in}}%
\pgfpathclose%
\pgfusepath{fill}%
\end{pgfscope}%
\begin{pgfscope}%
\pgfpathrectangle{\pgfqpoint{0.765000in}{0.660000in}}{\pgfqpoint{4.620000in}{4.620000in}}%
\pgfusepath{clip}%
\pgfsetbuttcap%
\pgfsetroundjoin%
\definecolor{currentfill}{rgb}{1.000000,0.894118,0.788235}%
\pgfsetfillcolor{currentfill}%
\pgfsetlinewidth{0.000000pt}%
\definecolor{currentstroke}{rgb}{1.000000,0.894118,0.788235}%
\pgfsetstrokecolor{currentstroke}%
\pgfsetdash{}{0pt}%
\pgfpathmoveto{\pgfqpoint{2.621957in}{3.208406in}}%
\pgfpathlineto{\pgfqpoint{2.621957in}{3.239522in}}%
\pgfpathlineto{\pgfqpoint{2.564236in}{3.220278in}}%
\pgfpathlineto{\pgfqpoint{2.537289in}{3.204720in}}%
\pgfpathlineto{\pgfqpoint{2.621957in}{3.208406in}}%
\pgfpathclose%
\pgfusepath{fill}%
\end{pgfscope}%
\begin{pgfscope}%
\pgfpathrectangle{\pgfqpoint{0.765000in}{0.660000in}}{\pgfqpoint{4.620000in}{4.620000in}}%
\pgfusepath{clip}%
\pgfsetbuttcap%
\pgfsetroundjoin%
\definecolor{currentfill}{rgb}{1.000000,0.894118,0.788235}%
\pgfsetfillcolor{currentfill}%
\pgfsetlinewidth{0.000000pt}%
\definecolor{currentstroke}{rgb}{1.000000,0.894118,0.788235}%
\pgfsetstrokecolor{currentstroke}%
\pgfsetdash{}{0pt}%
\pgfpathmoveto{\pgfqpoint{2.568062in}{3.208406in}}%
\pgfpathlineto{\pgfqpoint{2.595009in}{3.192848in}}%
\pgfpathlineto{\pgfqpoint{2.537289in}{3.173604in}}%
\pgfpathlineto{\pgfqpoint{2.510341in}{3.189162in}}%
\pgfpathlineto{\pgfqpoint{2.568062in}{3.208406in}}%
\pgfpathclose%
\pgfusepath{fill}%
\end{pgfscope}%
\begin{pgfscope}%
\pgfpathrectangle{\pgfqpoint{0.765000in}{0.660000in}}{\pgfqpoint{4.620000in}{4.620000in}}%
\pgfusepath{clip}%
\pgfsetbuttcap%
\pgfsetroundjoin%
\definecolor{currentfill}{rgb}{1.000000,0.894118,0.788235}%
\pgfsetfillcolor{currentfill}%
\pgfsetlinewidth{0.000000pt}%
\definecolor{currentstroke}{rgb}{1.000000,0.894118,0.788235}%
\pgfsetstrokecolor{currentstroke}%
\pgfsetdash{}{0pt}%
\pgfpathmoveto{\pgfqpoint{2.595009in}{3.192848in}}%
\pgfpathlineto{\pgfqpoint{2.621957in}{3.208406in}}%
\pgfpathlineto{\pgfqpoint{2.564236in}{3.189162in}}%
\pgfpathlineto{\pgfqpoint{2.537289in}{3.173604in}}%
\pgfpathlineto{\pgfqpoint{2.595009in}{3.192848in}}%
\pgfpathclose%
\pgfusepath{fill}%
\end{pgfscope}%
\begin{pgfscope}%
\pgfpathrectangle{\pgfqpoint{0.765000in}{0.660000in}}{\pgfqpoint{4.620000in}{4.620000in}}%
\pgfusepath{clip}%
\pgfsetbuttcap%
\pgfsetroundjoin%
\definecolor{currentfill}{rgb}{1.000000,0.894118,0.788235}%
\pgfsetfillcolor{currentfill}%
\pgfsetlinewidth{0.000000pt}%
\definecolor{currentstroke}{rgb}{1.000000,0.894118,0.788235}%
\pgfsetstrokecolor{currentstroke}%
\pgfsetdash{}{0pt}%
\pgfpathmoveto{\pgfqpoint{2.595009in}{3.255080in}}%
\pgfpathlineto{\pgfqpoint{2.568062in}{3.239522in}}%
\pgfpathlineto{\pgfqpoint{2.510341in}{3.220278in}}%
\pgfpathlineto{\pgfqpoint{2.537289in}{3.235836in}}%
\pgfpathlineto{\pgfqpoint{2.595009in}{3.255080in}}%
\pgfpathclose%
\pgfusepath{fill}%
\end{pgfscope}%
\begin{pgfscope}%
\pgfpathrectangle{\pgfqpoint{0.765000in}{0.660000in}}{\pgfqpoint{4.620000in}{4.620000in}}%
\pgfusepath{clip}%
\pgfsetbuttcap%
\pgfsetroundjoin%
\definecolor{currentfill}{rgb}{1.000000,0.894118,0.788235}%
\pgfsetfillcolor{currentfill}%
\pgfsetlinewidth{0.000000pt}%
\definecolor{currentstroke}{rgb}{1.000000,0.894118,0.788235}%
\pgfsetstrokecolor{currentstroke}%
\pgfsetdash{}{0pt}%
\pgfpathmoveto{\pgfqpoint{2.621957in}{3.239522in}}%
\pgfpathlineto{\pgfqpoint{2.595009in}{3.255080in}}%
\pgfpathlineto{\pgfqpoint{2.537289in}{3.235836in}}%
\pgfpathlineto{\pgfqpoint{2.564236in}{3.220278in}}%
\pgfpathlineto{\pgfqpoint{2.621957in}{3.239522in}}%
\pgfpathclose%
\pgfusepath{fill}%
\end{pgfscope}%
\begin{pgfscope}%
\pgfpathrectangle{\pgfqpoint{0.765000in}{0.660000in}}{\pgfqpoint{4.620000in}{4.620000in}}%
\pgfusepath{clip}%
\pgfsetbuttcap%
\pgfsetroundjoin%
\definecolor{currentfill}{rgb}{1.000000,0.894118,0.788235}%
\pgfsetfillcolor{currentfill}%
\pgfsetlinewidth{0.000000pt}%
\definecolor{currentstroke}{rgb}{1.000000,0.894118,0.788235}%
\pgfsetstrokecolor{currentstroke}%
\pgfsetdash{}{0pt}%
\pgfpathmoveto{\pgfqpoint{2.568062in}{3.208406in}}%
\pgfpathlineto{\pgfqpoint{2.568062in}{3.239522in}}%
\pgfpathlineto{\pgfqpoint{2.510341in}{3.220278in}}%
\pgfpathlineto{\pgfqpoint{2.537289in}{3.173604in}}%
\pgfpathlineto{\pgfqpoint{2.568062in}{3.208406in}}%
\pgfpathclose%
\pgfusepath{fill}%
\end{pgfscope}%
\begin{pgfscope}%
\pgfpathrectangle{\pgfqpoint{0.765000in}{0.660000in}}{\pgfqpoint{4.620000in}{4.620000in}}%
\pgfusepath{clip}%
\pgfsetbuttcap%
\pgfsetroundjoin%
\definecolor{currentfill}{rgb}{1.000000,0.894118,0.788235}%
\pgfsetfillcolor{currentfill}%
\pgfsetlinewidth{0.000000pt}%
\definecolor{currentstroke}{rgb}{1.000000,0.894118,0.788235}%
\pgfsetstrokecolor{currentstroke}%
\pgfsetdash{}{0pt}%
\pgfpathmoveto{\pgfqpoint{2.595009in}{3.192848in}}%
\pgfpathlineto{\pgfqpoint{2.595009in}{3.223964in}}%
\pgfpathlineto{\pgfqpoint{2.537289in}{3.204720in}}%
\pgfpathlineto{\pgfqpoint{2.510341in}{3.189162in}}%
\pgfpathlineto{\pgfqpoint{2.595009in}{3.192848in}}%
\pgfpathclose%
\pgfusepath{fill}%
\end{pgfscope}%
\begin{pgfscope}%
\pgfpathrectangle{\pgfqpoint{0.765000in}{0.660000in}}{\pgfqpoint{4.620000in}{4.620000in}}%
\pgfusepath{clip}%
\pgfsetbuttcap%
\pgfsetroundjoin%
\definecolor{currentfill}{rgb}{1.000000,0.894118,0.788235}%
\pgfsetfillcolor{currentfill}%
\pgfsetlinewidth{0.000000pt}%
\definecolor{currentstroke}{rgb}{1.000000,0.894118,0.788235}%
\pgfsetstrokecolor{currentstroke}%
\pgfsetdash{}{0pt}%
\pgfpathmoveto{\pgfqpoint{2.595009in}{3.223964in}}%
\pgfpathlineto{\pgfqpoint{2.621957in}{3.239522in}}%
\pgfpathlineto{\pgfqpoint{2.564236in}{3.220278in}}%
\pgfpathlineto{\pgfqpoint{2.537289in}{3.204720in}}%
\pgfpathlineto{\pgfqpoint{2.595009in}{3.223964in}}%
\pgfpathclose%
\pgfusepath{fill}%
\end{pgfscope}%
\begin{pgfscope}%
\pgfpathrectangle{\pgfqpoint{0.765000in}{0.660000in}}{\pgfqpoint{4.620000in}{4.620000in}}%
\pgfusepath{clip}%
\pgfsetbuttcap%
\pgfsetroundjoin%
\definecolor{currentfill}{rgb}{1.000000,0.894118,0.788235}%
\pgfsetfillcolor{currentfill}%
\pgfsetlinewidth{0.000000pt}%
\definecolor{currentstroke}{rgb}{1.000000,0.894118,0.788235}%
\pgfsetstrokecolor{currentstroke}%
\pgfsetdash{}{0pt}%
\pgfpathmoveto{\pgfqpoint{2.568062in}{3.239522in}}%
\pgfpathlineto{\pgfqpoint{2.595009in}{3.223964in}}%
\pgfpathlineto{\pgfqpoint{2.537289in}{3.204720in}}%
\pgfpathlineto{\pgfqpoint{2.510341in}{3.220278in}}%
\pgfpathlineto{\pgfqpoint{2.568062in}{3.239522in}}%
\pgfpathclose%
\pgfusepath{fill}%
\end{pgfscope}%
\begin{pgfscope}%
\pgfpathrectangle{\pgfqpoint{0.765000in}{0.660000in}}{\pgfqpoint{4.620000in}{4.620000in}}%
\pgfusepath{clip}%
\pgfsetbuttcap%
\pgfsetroundjoin%
\definecolor{currentfill}{rgb}{1.000000,0.894118,0.788235}%
\pgfsetfillcolor{currentfill}%
\pgfsetlinewidth{0.000000pt}%
\definecolor{currentstroke}{rgb}{1.000000,0.894118,0.788235}%
\pgfsetstrokecolor{currentstroke}%
\pgfsetdash{}{0pt}%
\pgfpathmoveto{\pgfqpoint{2.595009in}{3.223964in}}%
\pgfpathlineto{\pgfqpoint{2.568062in}{3.208406in}}%
\pgfpathlineto{\pgfqpoint{2.595009in}{3.192848in}}%
\pgfpathlineto{\pgfqpoint{2.621957in}{3.208406in}}%
\pgfpathlineto{\pgfqpoint{2.595009in}{3.223964in}}%
\pgfpathclose%
\pgfusepath{fill}%
\end{pgfscope}%
\begin{pgfscope}%
\pgfpathrectangle{\pgfqpoint{0.765000in}{0.660000in}}{\pgfqpoint{4.620000in}{4.620000in}}%
\pgfusepath{clip}%
\pgfsetbuttcap%
\pgfsetroundjoin%
\definecolor{currentfill}{rgb}{1.000000,0.894118,0.788235}%
\pgfsetfillcolor{currentfill}%
\pgfsetlinewidth{0.000000pt}%
\definecolor{currentstroke}{rgb}{1.000000,0.894118,0.788235}%
\pgfsetstrokecolor{currentstroke}%
\pgfsetdash{}{0pt}%
\pgfpathmoveto{\pgfqpoint{2.595009in}{3.223964in}}%
\pgfpathlineto{\pgfqpoint{2.568062in}{3.208406in}}%
\pgfpathlineto{\pgfqpoint{2.568062in}{3.239522in}}%
\pgfpathlineto{\pgfqpoint{2.595009in}{3.255080in}}%
\pgfpathlineto{\pgfqpoint{2.595009in}{3.223964in}}%
\pgfpathclose%
\pgfusepath{fill}%
\end{pgfscope}%
\begin{pgfscope}%
\pgfpathrectangle{\pgfqpoint{0.765000in}{0.660000in}}{\pgfqpoint{4.620000in}{4.620000in}}%
\pgfusepath{clip}%
\pgfsetbuttcap%
\pgfsetroundjoin%
\definecolor{currentfill}{rgb}{1.000000,0.894118,0.788235}%
\pgfsetfillcolor{currentfill}%
\pgfsetlinewidth{0.000000pt}%
\definecolor{currentstroke}{rgb}{1.000000,0.894118,0.788235}%
\pgfsetstrokecolor{currentstroke}%
\pgfsetdash{}{0pt}%
\pgfpathmoveto{\pgfqpoint{2.595009in}{3.223964in}}%
\pgfpathlineto{\pgfqpoint{2.621957in}{3.208406in}}%
\pgfpathlineto{\pgfqpoint{2.621957in}{3.239522in}}%
\pgfpathlineto{\pgfqpoint{2.595009in}{3.255080in}}%
\pgfpathlineto{\pgfqpoint{2.595009in}{3.223964in}}%
\pgfpathclose%
\pgfusepath{fill}%
\end{pgfscope}%
\begin{pgfscope}%
\pgfpathrectangle{\pgfqpoint{0.765000in}{0.660000in}}{\pgfqpoint{4.620000in}{4.620000in}}%
\pgfusepath{clip}%
\pgfsetbuttcap%
\pgfsetroundjoin%
\definecolor{currentfill}{rgb}{1.000000,0.894118,0.788235}%
\pgfsetfillcolor{currentfill}%
\pgfsetlinewidth{0.000000pt}%
\definecolor{currentstroke}{rgb}{1.000000,0.894118,0.788235}%
\pgfsetstrokecolor{currentstroke}%
\pgfsetdash{}{0pt}%
\pgfpathmoveto{\pgfqpoint{2.605373in}{3.227174in}}%
\pgfpathlineto{\pgfqpoint{2.578426in}{3.211616in}}%
\pgfpathlineto{\pgfqpoint{2.605373in}{3.196058in}}%
\pgfpathlineto{\pgfqpoint{2.632320in}{3.211616in}}%
\pgfpathlineto{\pgfqpoint{2.605373in}{3.227174in}}%
\pgfpathclose%
\pgfusepath{fill}%
\end{pgfscope}%
\begin{pgfscope}%
\pgfpathrectangle{\pgfqpoint{0.765000in}{0.660000in}}{\pgfqpoint{4.620000in}{4.620000in}}%
\pgfusepath{clip}%
\pgfsetbuttcap%
\pgfsetroundjoin%
\definecolor{currentfill}{rgb}{1.000000,0.894118,0.788235}%
\pgfsetfillcolor{currentfill}%
\pgfsetlinewidth{0.000000pt}%
\definecolor{currentstroke}{rgb}{1.000000,0.894118,0.788235}%
\pgfsetstrokecolor{currentstroke}%
\pgfsetdash{}{0pt}%
\pgfpathmoveto{\pgfqpoint{2.605373in}{3.227174in}}%
\pgfpathlineto{\pgfqpoint{2.578426in}{3.211616in}}%
\pgfpathlineto{\pgfqpoint{2.578426in}{3.242732in}}%
\pgfpathlineto{\pgfqpoint{2.605373in}{3.258290in}}%
\pgfpathlineto{\pgfqpoint{2.605373in}{3.227174in}}%
\pgfpathclose%
\pgfusepath{fill}%
\end{pgfscope}%
\begin{pgfscope}%
\pgfpathrectangle{\pgfqpoint{0.765000in}{0.660000in}}{\pgfqpoint{4.620000in}{4.620000in}}%
\pgfusepath{clip}%
\pgfsetbuttcap%
\pgfsetroundjoin%
\definecolor{currentfill}{rgb}{1.000000,0.894118,0.788235}%
\pgfsetfillcolor{currentfill}%
\pgfsetlinewidth{0.000000pt}%
\definecolor{currentstroke}{rgb}{1.000000,0.894118,0.788235}%
\pgfsetstrokecolor{currentstroke}%
\pgfsetdash{}{0pt}%
\pgfpathmoveto{\pgfqpoint{2.605373in}{3.227174in}}%
\pgfpathlineto{\pgfqpoint{2.632320in}{3.211616in}}%
\pgfpathlineto{\pgfqpoint{2.632320in}{3.242732in}}%
\pgfpathlineto{\pgfqpoint{2.605373in}{3.258290in}}%
\pgfpathlineto{\pgfqpoint{2.605373in}{3.227174in}}%
\pgfpathclose%
\pgfusepath{fill}%
\end{pgfscope}%
\begin{pgfscope}%
\pgfpathrectangle{\pgfqpoint{0.765000in}{0.660000in}}{\pgfqpoint{4.620000in}{4.620000in}}%
\pgfusepath{clip}%
\pgfsetbuttcap%
\pgfsetroundjoin%
\definecolor{currentfill}{rgb}{1.000000,0.894118,0.788235}%
\pgfsetfillcolor{currentfill}%
\pgfsetlinewidth{0.000000pt}%
\definecolor{currentstroke}{rgb}{1.000000,0.894118,0.788235}%
\pgfsetstrokecolor{currentstroke}%
\pgfsetdash{}{0pt}%
\pgfpathmoveto{\pgfqpoint{2.595009in}{3.255080in}}%
\pgfpathlineto{\pgfqpoint{2.568062in}{3.239522in}}%
\pgfpathlineto{\pgfqpoint{2.595009in}{3.223964in}}%
\pgfpathlineto{\pgfqpoint{2.621957in}{3.239522in}}%
\pgfpathlineto{\pgfqpoint{2.595009in}{3.255080in}}%
\pgfpathclose%
\pgfusepath{fill}%
\end{pgfscope}%
\begin{pgfscope}%
\pgfpathrectangle{\pgfqpoint{0.765000in}{0.660000in}}{\pgfqpoint{4.620000in}{4.620000in}}%
\pgfusepath{clip}%
\pgfsetbuttcap%
\pgfsetroundjoin%
\definecolor{currentfill}{rgb}{1.000000,0.894118,0.788235}%
\pgfsetfillcolor{currentfill}%
\pgfsetlinewidth{0.000000pt}%
\definecolor{currentstroke}{rgb}{1.000000,0.894118,0.788235}%
\pgfsetstrokecolor{currentstroke}%
\pgfsetdash{}{0pt}%
\pgfpathmoveto{\pgfqpoint{2.595009in}{3.192848in}}%
\pgfpathlineto{\pgfqpoint{2.621957in}{3.208406in}}%
\pgfpathlineto{\pgfqpoint{2.621957in}{3.239522in}}%
\pgfpathlineto{\pgfqpoint{2.595009in}{3.223964in}}%
\pgfpathlineto{\pgfqpoint{2.595009in}{3.192848in}}%
\pgfpathclose%
\pgfusepath{fill}%
\end{pgfscope}%
\begin{pgfscope}%
\pgfpathrectangle{\pgfqpoint{0.765000in}{0.660000in}}{\pgfqpoint{4.620000in}{4.620000in}}%
\pgfusepath{clip}%
\pgfsetbuttcap%
\pgfsetroundjoin%
\definecolor{currentfill}{rgb}{1.000000,0.894118,0.788235}%
\pgfsetfillcolor{currentfill}%
\pgfsetlinewidth{0.000000pt}%
\definecolor{currentstroke}{rgb}{1.000000,0.894118,0.788235}%
\pgfsetstrokecolor{currentstroke}%
\pgfsetdash{}{0pt}%
\pgfpathmoveto{\pgfqpoint{2.568062in}{3.208406in}}%
\pgfpathlineto{\pgfqpoint{2.595009in}{3.192848in}}%
\pgfpathlineto{\pgfqpoint{2.595009in}{3.223964in}}%
\pgfpathlineto{\pgfqpoint{2.568062in}{3.239522in}}%
\pgfpathlineto{\pgfqpoint{2.568062in}{3.208406in}}%
\pgfpathclose%
\pgfusepath{fill}%
\end{pgfscope}%
\begin{pgfscope}%
\pgfpathrectangle{\pgfqpoint{0.765000in}{0.660000in}}{\pgfqpoint{4.620000in}{4.620000in}}%
\pgfusepath{clip}%
\pgfsetbuttcap%
\pgfsetroundjoin%
\definecolor{currentfill}{rgb}{1.000000,0.894118,0.788235}%
\pgfsetfillcolor{currentfill}%
\pgfsetlinewidth{0.000000pt}%
\definecolor{currentstroke}{rgb}{1.000000,0.894118,0.788235}%
\pgfsetstrokecolor{currentstroke}%
\pgfsetdash{}{0pt}%
\pgfpathmoveto{\pgfqpoint{2.605373in}{3.258290in}}%
\pgfpathlineto{\pgfqpoint{2.578426in}{3.242732in}}%
\pgfpathlineto{\pgfqpoint{2.605373in}{3.227174in}}%
\pgfpathlineto{\pgfqpoint{2.632320in}{3.242732in}}%
\pgfpathlineto{\pgfqpoint{2.605373in}{3.258290in}}%
\pgfpathclose%
\pgfusepath{fill}%
\end{pgfscope}%
\begin{pgfscope}%
\pgfpathrectangle{\pgfqpoint{0.765000in}{0.660000in}}{\pgfqpoint{4.620000in}{4.620000in}}%
\pgfusepath{clip}%
\pgfsetbuttcap%
\pgfsetroundjoin%
\definecolor{currentfill}{rgb}{1.000000,0.894118,0.788235}%
\pgfsetfillcolor{currentfill}%
\pgfsetlinewidth{0.000000pt}%
\definecolor{currentstroke}{rgb}{1.000000,0.894118,0.788235}%
\pgfsetstrokecolor{currentstroke}%
\pgfsetdash{}{0pt}%
\pgfpathmoveto{\pgfqpoint{2.605373in}{3.196058in}}%
\pgfpathlineto{\pgfqpoint{2.632320in}{3.211616in}}%
\pgfpathlineto{\pgfqpoint{2.632320in}{3.242732in}}%
\pgfpathlineto{\pgfqpoint{2.605373in}{3.227174in}}%
\pgfpathlineto{\pgfqpoint{2.605373in}{3.196058in}}%
\pgfpathclose%
\pgfusepath{fill}%
\end{pgfscope}%
\begin{pgfscope}%
\pgfpathrectangle{\pgfqpoint{0.765000in}{0.660000in}}{\pgfqpoint{4.620000in}{4.620000in}}%
\pgfusepath{clip}%
\pgfsetbuttcap%
\pgfsetroundjoin%
\definecolor{currentfill}{rgb}{1.000000,0.894118,0.788235}%
\pgfsetfillcolor{currentfill}%
\pgfsetlinewidth{0.000000pt}%
\definecolor{currentstroke}{rgb}{1.000000,0.894118,0.788235}%
\pgfsetstrokecolor{currentstroke}%
\pgfsetdash{}{0pt}%
\pgfpathmoveto{\pgfqpoint{2.578426in}{3.211616in}}%
\pgfpathlineto{\pgfqpoint{2.605373in}{3.196058in}}%
\pgfpathlineto{\pgfqpoint{2.605373in}{3.227174in}}%
\pgfpathlineto{\pgfqpoint{2.578426in}{3.242732in}}%
\pgfpathlineto{\pgfqpoint{2.578426in}{3.211616in}}%
\pgfpathclose%
\pgfusepath{fill}%
\end{pgfscope}%
\begin{pgfscope}%
\pgfpathrectangle{\pgfqpoint{0.765000in}{0.660000in}}{\pgfqpoint{4.620000in}{4.620000in}}%
\pgfusepath{clip}%
\pgfsetbuttcap%
\pgfsetroundjoin%
\definecolor{currentfill}{rgb}{1.000000,0.894118,0.788235}%
\pgfsetfillcolor{currentfill}%
\pgfsetlinewidth{0.000000pt}%
\definecolor{currentstroke}{rgb}{1.000000,0.894118,0.788235}%
\pgfsetstrokecolor{currentstroke}%
\pgfsetdash{}{0pt}%
\pgfpathmoveto{\pgfqpoint{2.595009in}{3.223964in}}%
\pgfpathlineto{\pgfqpoint{2.568062in}{3.208406in}}%
\pgfpathlineto{\pgfqpoint{2.578426in}{3.211616in}}%
\pgfpathlineto{\pgfqpoint{2.605373in}{3.227174in}}%
\pgfpathlineto{\pgfqpoint{2.595009in}{3.223964in}}%
\pgfpathclose%
\pgfusepath{fill}%
\end{pgfscope}%
\begin{pgfscope}%
\pgfpathrectangle{\pgfqpoint{0.765000in}{0.660000in}}{\pgfqpoint{4.620000in}{4.620000in}}%
\pgfusepath{clip}%
\pgfsetbuttcap%
\pgfsetroundjoin%
\definecolor{currentfill}{rgb}{1.000000,0.894118,0.788235}%
\pgfsetfillcolor{currentfill}%
\pgfsetlinewidth{0.000000pt}%
\definecolor{currentstroke}{rgb}{1.000000,0.894118,0.788235}%
\pgfsetstrokecolor{currentstroke}%
\pgfsetdash{}{0pt}%
\pgfpathmoveto{\pgfqpoint{2.621957in}{3.208406in}}%
\pgfpathlineto{\pgfqpoint{2.595009in}{3.223964in}}%
\pgfpathlineto{\pgfqpoint{2.605373in}{3.227174in}}%
\pgfpathlineto{\pgfqpoint{2.632320in}{3.211616in}}%
\pgfpathlineto{\pgfqpoint{2.621957in}{3.208406in}}%
\pgfpathclose%
\pgfusepath{fill}%
\end{pgfscope}%
\begin{pgfscope}%
\pgfpathrectangle{\pgfqpoint{0.765000in}{0.660000in}}{\pgfqpoint{4.620000in}{4.620000in}}%
\pgfusepath{clip}%
\pgfsetbuttcap%
\pgfsetroundjoin%
\definecolor{currentfill}{rgb}{1.000000,0.894118,0.788235}%
\pgfsetfillcolor{currentfill}%
\pgfsetlinewidth{0.000000pt}%
\definecolor{currentstroke}{rgb}{1.000000,0.894118,0.788235}%
\pgfsetstrokecolor{currentstroke}%
\pgfsetdash{}{0pt}%
\pgfpathmoveto{\pgfqpoint{2.595009in}{3.223964in}}%
\pgfpathlineto{\pgfqpoint{2.595009in}{3.255080in}}%
\pgfpathlineto{\pgfqpoint{2.605373in}{3.258290in}}%
\pgfpathlineto{\pgfqpoint{2.632320in}{3.211616in}}%
\pgfpathlineto{\pgfqpoint{2.595009in}{3.223964in}}%
\pgfpathclose%
\pgfusepath{fill}%
\end{pgfscope}%
\begin{pgfscope}%
\pgfpathrectangle{\pgfqpoint{0.765000in}{0.660000in}}{\pgfqpoint{4.620000in}{4.620000in}}%
\pgfusepath{clip}%
\pgfsetbuttcap%
\pgfsetroundjoin%
\definecolor{currentfill}{rgb}{1.000000,0.894118,0.788235}%
\pgfsetfillcolor{currentfill}%
\pgfsetlinewidth{0.000000pt}%
\definecolor{currentstroke}{rgb}{1.000000,0.894118,0.788235}%
\pgfsetstrokecolor{currentstroke}%
\pgfsetdash{}{0pt}%
\pgfpathmoveto{\pgfqpoint{2.621957in}{3.208406in}}%
\pgfpathlineto{\pgfqpoint{2.621957in}{3.239522in}}%
\pgfpathlineto{\pgfqpoint{2.632320in}{3.242732in}}%
\pgfpathlineto{\pgfqpoint{2.605373in}{3.227174in}}%
\pgfpathlineto{\pgfqpoint{2.621957in}{3.208406in}}%
\pgfpathclose%
\pgfusepath{fill}%
\end{pgfscope}%
\begin{pgfscope}%
\pgfpathrectangle{\pgfqpoint{0.765000in}{0.660000in}}{\pgfqpoint{4.620000in}{4.620000in}}%
\pgfusepath{clip}%
\pgfsetbuttcap%
\pgfsetroundjoin%
\definecolor{currentfill}{rgb}{1.000000,0.894118,0.788235}%
\pgfsetfillcolor{currentfill}%
\pgfsetlinewidth{0.000000pt}%
\definecolor{currentstroke}{rgb}{1.000000,0.894118,0.788235}%
\pgfsetstrokecolor{currentstroke}%
\pgfsetdash{}{0pt}%
\pgfpathmoveto{\pgfqpoint{2.568062in}{3.208406in}}%
\pgfpathlineto{\pgfqpoint{2.595009in}{3.192848in}}%
\pgfpathlineto{\pgfqpoint{2.605373in}{3.196058in}}%
\pgfpathlineto{\pgfqpoint{2.578426in}{3.211616in}}%
\pgfpathlineto{\pgfqpoint{2.568062in}{3.208406in}}%
\pgfpathclose%
\pgfusepath{fill}%
\end{pgfscope}%
\begin{pgfscope}%
\pgfpathrectangle{\pgfqpoint{0.765000in}{0.660000in}}{\pgfqpoint{4.620000in}{4.620000in}}%
\pgfusepath{clip}%
\pgfsetbuttcap%
\pgfsetroundjoin%
\definecolor{currentfill}{rgb}{1.000000,0.894118,0.788235}%
\pgfsetfillcolor{currentfill}%
\pgfsetlinewidth{0.000000pt}%
\definecolor{currentstroke}{rgb}{1.000000,0.894118,0.788235}%
\pgfsetstrokecolor{currentstroke}%
\pgfsetdash{}{0pt}%
\pgfpathmoveto{\pgfqpoint{2.595009in}{3.192848in}}%
\pgfpathlineto{\pgfqpoint{2.621957in}{3.208406in}}%
\pgfpathlineto{\pgfqpoint{2.632320in}{3.211616in}}%
\pgfpathlineto{\pgfqpoint{2.605373in}{3.196058in}}%
\pgfpathlineto{\pgfqpoint{2.595009in}{3.192848in}}%
\pgfpathclose%
\pgfusepath{fill}%
\end{pgfscope}%
\begin{pgfscope}%
\pgfpathrectangle{\pgfqpoint{0.765000in}{0.660000in}}{\pgfqpoint{4.620000in}{4.620000in}}%
\pgfusepath{clip}%
\pgfsetbuttcap%
\pgfsetroundjoin%
\definecolor{currentfill}{rgb}{1.000000,0.894118,0.788235}%
\pgfsetfillcolor{currentfill}%
\pgfsetlinewidth{0.000000pt}%
\definecolor{currentstroke}{rgb}{1.000000,0.894118,0.788235}%
\pgfsetstrokecolor{currentstroke}%
\pgfsetdash{}{0pt}%
\pgfpathmoveto{\pgfqpoint{2.595009in}{3.255080in}}%
\pgfpathlineto{\pgfqpoint{2.568062in}{3.239522in}}%
\pgfpathlineto{\pgfqpoint{2.578426in}{3.242732in}}%
\pgfpathlineto{\pgfqpoint{2.605373in}{3.258290in}}%
\pgfpathlineto{\pgfqpoint{2.595009in}{3.255080in}}%
\pgfpathclose%
\pgfusepath{fill}%
\end{pgfscope}%
\begin{pgfscope}%
\pgfpathrectangle{\pgfqpoint{0.765000in}{0.660000in}}{\pgfqpoint{4.620000in}{4.620000in}}%
\pgfusepath{clip}%
\pgfsetbuttcap%
\pgfsetroundjoin%
\definecolor{currentfill}{rgb}{1.000000,0.894118,0.788235}%
\pgfsetfillcolor{currentfill}%
\pgfsetlinewidth{0.000000pt}%
\definecolor{currentstroke}{rgb}{1.000000,0.894118,0.788235}%
\pgfsetstrokecolor{currentstroke}%
\pgfsetdash{}{0pt}%
\pgfpathmoveto{\pgfqpoint{2.621957in}{3.239522in}}%
\pgfpathlineto{\pgfqpoint{2.595009in}{3.255080in}}%
\pgfpathlineto{\pgfqpoint{2.605373in}{3.258290in}}%
\pgfpathlineto{\pgfqpoint{2.632320in}{3.242732in}}%
\pgfpathlineto{\pgfqpoint{2.621957in}{3.239522in}}%
\pgfpathclose%
\pgfusepath{fill}%
\end{pgfscope}%
\begin{pgfscope}%
\pgfpathrectangle{\pgfqpoint{0.765000in}{0.660000in}}{\pgfqpoint{4.620000in}{4.620000in}}%
\pgfusepath{clip}%
\pgfsetbuttcap%
\pgfsetroundjoin%
\definecolor{currentfill}{rgb}{1.000000,0.894118,0.788235}%
\pgfsetfillcolor{currentfill}%
\pgfsetlinewidth{0.000000pt}%
\definecolor{currentstroke}{rgb}{1.000000,0.894118,0.788235}%
\pgfsetstrokecolor{currentstroke}%
\pgfsetdash{}{0pt}%
\pgfpathmoveto{\pgfqpoint{2.568062in}{3.208406in}}%
\pgfpathlineto{\pgfqpoint{2.568062in}{3.239522in}}%
\pgfpathlineto{\pgfqpoint{2.578426in}{3.242732in}}%
\pgfpathlineto{\pgfqpoint{2.605373in}{3.196058in}}%
\pgfpathlineto{\pgfqpoint{2.568062in}{3.208406in}}%
\pgfpathclose%
\pgfusepath{fill}%
\end{pgfscope}%
\begin{pgfscope}%
\pgfpathrectangle{\pgfqpoint{0.765000in}{0.660000in}}{\pgfqpoint{4.620000in}{4.620000in}}%
\pgfusepath{clip}%
\pgfsetbuttcap%
\pgfsetroundjoin%
\definecolor{currentfill}{rgb}{1.000000,0.894118,0.788235}%
\pgfsetfillcolor{currentfill}%
\pgfsetlinewidth{0.000000pt}%
\definecolor{currentstroke}{rgb}{1.000000,0.894118,0.788235}%
\pgfsetstrokecolor{currentstroke}%
\pgfsetdash{}{0pt}%
\pgfpathmoveto{\pgfqpoint{2.595009in}{3.192848in}}%
\pgfpathlineto{\pgfqpoint{2.595009in}{3.223964in}}%
\pgfpathlineto{\pgfqpoint{2.605373in}{3.227174in}}%
\pgfpathlineto{\pgfqpoint{2.578426in}{3.211616in}}%
\pgfpathlineto{\pgfqpoint{2.595009in}{3.192848in}}%
\pgfpathclose%
\pgfusepath{fill}%
\end{pgfscope}%
\begin{pgfscope}%
\pgfpathrectangle{\pgfqpoint{0.765000in}{0.660000in}}{\pgfqpoint{4.620000in}{4.620000in}}%
\pgfusepath{clip}%
\pgfsetbuttcap%
\pgfsetroundjoin%
\definecolor{currentfill}{rgb}{1.000000,0.894118,0.788235}%
\pgfsetfillcolor{currentfill}%
\pgfsetlinewidth{0.000000pt}%
\definecolor{currentstroke}{rgb}{1.000000,0.894118,0.788235}%
\pgfsetstrokecolor{currentstroke}%
\pgfsetdash{}{0pt}%
\pgfpathmoveto{\pgfqpoint{2.595009in}{3.223964in}}%
\pgfpathlineto{\pgfqpoint{2.621957in}{3.239522in}}%
\pgfpathlineto{\pgfqpoint{2.632320in}{3.242732in}}%
\pgfpathlineto{\pgfqpoint{2.605373in}{3.227174in}}%
\pgfpathlineto{\pgfqpoint{2.595009in}{3.223964in}}%
\pgfpathclose%
\pgfusepath{fill}%
\end{pgfscope}%
\begin{pgfscope}%
\pgfpathrectangle{\pgfqpoint{0.765000in}{0.660000in}}{\pgfqpoint{4.620000in}{4.620000in}}%
\pgfusepath{clip}%
\pgfsetbuttcap%
\pgfsetroundjoin%
\definecolor{currentfill}{rgb}{1.000000,0.894118,0.788235}%
\pgfsetfillcolor{currentfill}%
\pgfsetlinewidth{0.000000pt}%
\definecolor{currentstroke}{rgb}{1.000000,0.894118,0.788235}%
\pgfsetstrokecolor{currentstroke}%
\pgfsetdash{}{0pt}%
\pgfpathmoveto{\pgfqpoint{2.568062in}{3.239522in}}%
\pgfpathlineto{\pgfqpoint{2.595009in}{3.223964in}}%
\pgfpathlineto{\pgfqpoint{2.605373in}{3.227174in}}%
\pgfpathlineto{\pgfqpoint{2.578426in}{3.242732in}}%
\pgfpathlineto{\pgfqpoint{2.568062in}{3.239522in}}%
\pgfpathclose%
\pgfusepath{fill}%
\end{pgfscope}%
\begin{pgfscope}%
\pgfpathrectangle{\pgfqpoint{0.765000in}{0.660000in}}{\pgfqpoint{4.620000in}{4.620000in}}%
\pgfusepath{clip}%
\pgfsetbuttcap%
\pgfsetroundjoin%
\definecolor{currentfill}{rgb}{1.000000,0.894118,0.788235}%
\pgfsetfillcolor{currentfill}%
\pgfsetlinewidth{0.000000pt}%
\definecolor{currentstroke}{rgb}{1.000000,0.894118,0.788235}%
\pgfsetstrokecolor{currentstroke}%
\pgfsetdash{}{0pt}%
\pgfpathmoveto{\pgfqpoint{2.537289in}{3.204720in}}%
\pgfpathlineto{\pgfqpoint{2.510341in}{3.189162in}}%
\pgfpathlineto{\pgfqpoint{2.537289in}{3.173604in}}%
\pgfpathlineto{\pgfqpoint{2.564236in}{3.189162in}}%
\pgfpathlineto{\pgfqpoint{2.537289in}{3.204720in}}%
\pgfpathclose%
\pgfusepath{fill}%
\end{pgfscope}%
\begin{pgfscope}%
\pgfpathrectangle{\pgfqpoint{0.765000in}{0.660000in}}{\pgfqpoint{4.620000in}{4.620000in}}%
\pgfusepath{clip}%
\pgfsetbuttcap%
\pgfsetroundjoin%
\definecolor{currentfill}{rgb}{1.000000,0.894118,0.788235}%
\pgfsetfillcolor{currentfill}%
\pgfsetlinewidth{0.000000pt}%
\definecolor{currentstroke}{rgb}{1.000000,0.894118,0.788235}%
\pgfsetstrokecolor{currentstroke}%
\pgfsetdash{}{0pt}%
\pgfpathmoveto{\pgfqpoint{2.537289in}{3.204720in}}%
\pgfpathlineto{\pgfqpoint{2.510341in}{3.189162in}}%
\pgfpathlineto{\pgfqpoint{2.510341in}{3.220278in}}%
\pgfpathlineto{\pgfqpoint{2.537289in}{3.235836in}}%
\pgfpathlineto{\pgfqpoint{2.537289in}{3.204720in}}%
\pgfpathclose%
\pgfusepath{fill}%
\end{pgfscope}%
\begin{pgfscope}%
\pgfpathrectangle{\pgfqpoint{0.765000in}{0.660000in}}{\pgfqpoint{4.620000in}{4.620000in}}%
\pgfusepath{clip}%
\pgfsetbuttcap%
\pgfsetroundjoin%
\definecolor{currentfill}{rgb}{1.000000,0.894118,0.788235}%
\pgfsetfillcolor{currentfill}%
\pgfsetlinewidth{0.000000pt}%
\definecolor{currentstroke}{rgb}{1.000000,0.894118,0.788235}%
\pgfsetstrokecolor{currentstroke}%
\pgfsetdash{}{0pt}%
\pgfpathmoveto{\pgfqpoint{2.537289in}{3.204720in}}%
\pgfpathlineto{\pgfqpoint{2.564236in}{3.189162in}}%
\pgfpathlineto{\pgfqpoint{2.564236in}{3.220278in}}%
\pgfpathlineto{\pgfqpoint{2.537289in}{3.235836in}}%
\pgfpathlineto{\pgfqpoint{2.537289in}{3.204720in}}%
\pgfpathclose%
\pgfusepath{fill}%
\end{pgfscope}%
\begin{pgfscope}%
\pgfpathrectangle{\pgfqpoint{0.765000in}{0.660000in}}{\pgfqpoint{4.620000in}{4.620000in}}%
\pgfusepath{clip}%
\pgfsetbuttcap%
\pgfsetroundjoin%
\definecolor{currentfill}{rgb}{1.000000,0.894118,0.788235}%
\pgfsetfillcolor{currentfill}%
\pgfsetlinewidth{0.000000pt}%
\definecolor{currentstroke}{rgb}{1.000000,0.894118,0.788235}%
\pgfsetstrokecolor{currentstroke}%
\pgfsetdash{}{0pt}%
\pgfpathmoveto{\pgfqpoint{2.595009in}{3.223964in}}%
\pgfpathlineto{\pgfqpoint{2.568062in}{3.208406in}}%
\pgfpathlineto{\pgfqpoint{2.595009in}{3.192848in}}%
\pgfpathlineto{\pgfqpoint{2.621957in}{3.208406in}}%
\pgfpathlineto{\pgfqpoint{2.595009in}{3.223964in}}%
\pgfpathclose%
\pgfusepath{fill}%
\end{pgfscope}%
\begin{pgfscope}%
\pgfpathrectangle{\pgfqpoint{0.765000in}{0.660000in}}{\pgfqpoint{4.620000in}{4.620000in}}%
\pgfusepath{clip}%
\pgfsetbuttcap%
\pgfsetroundjoin%
\definecolor{currentfill}{rgb}{1.000000,0.894118,0.788235}%
\pgfsetfillcolor{currentfill}%
\pgfsetlinewidth{0.000000pt}%
\definecolor{currentstroke}{rgb}{1.000000,0.894118,0.788235}%
\pgfsetstrokecolor{currentstroke}%
\pgfsetdash{}{0pt}%
\pgfpathmoveto{\pgfqpoint{2.595009in}{3.223964in}}%
\pgfpathlineto{\pgfqpoint{2.568062in}{3.208406in}}%
\pgfpathlineto{\pgfqpoint{2.568062in}{3.239522in}}%
\pgfpathlineto{\pgfqpoint{2.595009in}{3.255080in}}%
\pgfpathlineto{\pgfqpoint{2.595009in}{3.223964in}}%
\pgfpathclose%
\pgfusepath{fill}%
\end{pgfscope}%
\begin{pgfscope}%
\pgfpathrectangle{\pgfqpoint{0.765000in}{0.660000in}}{\pgfqpoint{4.620000in}{4.620000in}}%
\pgfusepath{clip}%
\pgfsetbuttcap%
\pgfsetroundjoin%
\definecolor{currentfill}{rgb}{1.000000,0.894118,0.788235}%
\pgfsetfillcolor{currentfill}%
\pgfsetlinewidth{0.000000pt}%
\definecolor{currentstroke}{rgb}{1.000000,0.894118,0.788235}%
\pgfsetstrokecolor{currentstroke}%
\pgfsetdash{}{0pt}%
\pgfpathmoveto{\pgfqpoint{2.595009in}{3.223964in}}%
\pgfpathlineto{\pgfqpoint{2.621957in}{3.208406in}}%
\pgfpathlineto{\pgfqpoint{2.621957in}{3.239522in}}%
\pgfpathlineto{\pgfqpoint{2.595009in}{3.255080in}}%
\pgfpathlineto{\pgfqpoint{2.595009in}{3.223964in}}%
\pgfpathclose%
\pgfusepath{fill}%
\end{pgfscope}%
\begin{pgfscope}%
\pgfpathrectangle{\pgfqpoint{0.765000in}{0.660000in}}{\pgfqpoint{4.620000in}{4.620000in}}%
\pgfusepath{clip}%
\pgfsetbuttcap%
\pgfsetroundjoin%
\definecolor{currentfill}{rgb}{1.000000,0.894118,0.788235}%
\pgfsetfillcolor{currentfill}%
\pgfsetlinewidth{0.000000pt}%
\definecolor{currentstroke}{rgb}{1.000000,0.894118,0.788235}%
\pgfsetstrokecolor{currentstroke}%
\pgfsetdash{}{0pt}%
\pgfpathmoveto{\pgfqpoint{2.537289in}{3.235836in}}%
\pgfpathlineto{\pgfqpoint{2.510341in}{3.220278in}}%
\pgfpathlineto{\pgfqpoint{2.537289in}{3.204720in}}%
\pgfpathlineto{\pgfqpoint{2.564236in}{3.220278in}}%
\pgfpathlineto{\pgfqpoint{2.537289in}{3.235836in}}%
\pgfpathclose%
\pgfusepath{fill}%
\end{pgfscope}%
\begin{pgfscope}%
\pgfpathrectangle{\pgfqpoint{0.765000in}{0.660000in}}{\pgfqpoint{4.620000in}{4.620000in}}%
\pgfusepath{clip}%
\pgfsetbuttcap%
\pgfsetroundjoin%
\definecolor{currentfill}{rgb}{1.000000,0.894118,0.788235}%
\pgfsetfillcolor{currentfill}%
\pgfsetlinewidth{0.000000pt}%
\definecolor{currentstroke}{rgb}{1.000000,0.894118,0.788235}%
\pgfsetstrokecolor{currentstroke}%
\pgfsetdash{}{0pt}%
\pgfpathmoveto{\pgfqpoint{2.537289in}{3.173604in}}%
\pgfpathlineto{\pgfqpoint{2.564236in}{3.189162in}}%
\pgfpathlineto{\pgfqpoint{2.564236in}{3.220278in}}%
\pgfpathlineto{\pgfqpoint{2.537289in}{3.204720in}}%
\pgfpathlineto{\pgfqpoint{2.537289in}{3.173604in}}%
\pgfpathclose%
\pgfusepath{fill}%
\end{pgfscope}%
\begin{pgfscope}%
\pgfpathrectangle{\pgfqpoint{0.765000in}{0.660000in}}{\pgfqpoint{4.620000in}{4.620000in}}%
\pgfusepath{clip}%
\pgfsetbuttcap%
\pgfsetroundjoin%
\definecolor{currentfill}{rgb}{1.000000,0.894118,0.788235}%
\pgfsetfillcolor{currentfill}%
\pgfsetlinewidth{0.000000pt}%
\definecolor{currentstroke}{rgb}{1.000000,0.894118,0.788235}%
\pgfsetstrokecolor{currentstroke}%
\pgfsetdash{}{0pt}%
\pgfpathmoveto{\pgfqpoint{2.510341in}{3.189162in}}%
\pgfpathlineto{\pgfqpoint{2.537289in}{3.173604in}}%
\pgfpathlineto{\pgfqpoint{2.537289in}{3.204720in}}%
\pgfpathlineto{\pgfqpoint{2.510341in}{3.220278in}}%
\pgfpathlineto{\pgfqpoint{2.510341in}{3.189162in}}%
\pgfpathclose%
\pgfusepath{fill}%
\end{pgfscope}%
\begin{pgfscope}%
\pgfpathrectangle{\pgfqpoint{0.765000in}{0.660000in}}{\pgfqpoint{4.620000in}{4.620000in}}%
\pgfusepath{clip}%
\pgfsetbuttcap%
\pgfsetroundjoin%
\definecolor{currentfill}{rgb}{1.000000,0.894118,0.788235}%
\pgfsetfillcolor{currentfill}%
\pgfsetlinewidth{0.000000pt}%
\definecolor{currentstroke}{rgb}{1.000000,0.894118,0.788235}%
\pgfsetstrokecolor{currentstroke}%
\pgfsetdash{}{0pt}%
\pgfpathmoveto{\pgfqpoint{2.595009in}{3.255080in}}%
\pgfpathlineto{\pgfqpoint{2.568062in}{3.239522in}}%
\pgfpathlineto{\pgfqpoint{2.595009in}{3.223964in}}%
\pgfpathlineto{\pgfqpoint{2.621957in}{3.239522in}}%
\pgfpathlineto{\pgfqpoint{2.595009in}{3.255080in}}%
\pgfpathclose%
\pgfusepath{fill}%
\end{pgfscope}%
\begin{pgfscope}%
\pgfpathrectangle{\pgfqpoint{0.765000in}{0.660000in}}{\pgfqpoint{4.620000in}{4.620000in}}%
\pgfusepath{clip}%
\pgfsetbuttcap%
\pgfsetroundjoin%
\definecolor{currentfill}{rgb}{1.000000,0.894118,0.788235}%
\pgfsetfillcolor{currentfill}%
\pgfsetlinewidth{0.000000pt}%
\definecolor{currentstroke}{rgb}{1.000000,0.894118,0.788235}%
\pgfsetstrokecolor{currentstroke}%
\pgfsetdash{}{0pt}%
\pgfpathmoveto{\pgfqpoint{2.595009in}{3.192848in}}%
\pgfpathlineto{\pgfqpoint{2.621957in}{3.208406in}}%
\pgfpathlineto{\pgfqpoint{2.621957in}{3.239522in}}%
\pgfpathlineto{\pgfqpoint{2.595009in}{3.223964in}}%
\pgfpathlineto{\pgfqpoint{2.595009in}{3.192848in}}%
\pgfpathclose%
\pgfusepath{fill}%
\end{pgfscope}%
\begin{pgfscope}%
\pgfpathrectangle{\pgfqpoint{0.765000in}{0.660000in}}{\pgfqpoint{4.620000in}{4.620000in}}%
\pgfusepath{clip}%
\pgfsetbuttcap%
\pgfsetroundjoin%
\definecolor{currentfill}{rgb}{1.000000,0.894118,0.788235}%
\pgfsetfillcolor{currentfill}%
\pgfsetlinewidth{0.000000pt}%
\definecolor{currentstroke}{rgb}{1.000000,0.894118,0.788235}%
\pgfsetstrokecolor{currentstroke}%
\pgfsetdash{}{0pt}%
\pgfpathmoveto{\pgfqpoint{2.568062in}{3.208406in}}%
\pgfpathlineto{\pgfqpoint{2.595009in}{3.192848in}}%
\pgfpathlineto{\pgfqpoint{2.595009in}{3.223964in}}%
\pgfpathlineto{\pgfqpoint{2.568062in}{3.239522in}}%
\pgfpathlineto{\pgfqpoint{2.568062in}{3.208406in}}%
\pgfpathclose%
\pgfusepath{fill}%
\end{pgfscope}%
\begin{pgfscope}%
\pgfpathrectangle{\pgfqpoint{0.765000in}{0.660000in}}{\pgfqpoint{4.620000in}{4.620000in}}%
\pgfusepath{clip}%
\pgfsetbuttcap%
\pgfsetroundjoin%
\definecolor{currentfill}{rgb}{1.000000,0.894118,0.788235}%
\pgfsetfillcolor{currentfill}%
\pgfsetlinewidth{0.000000pt}%
\definecolor{currentstroke}{rgb}{1.000000,0.894118,0.788235}%
\pgfsetstrokecolor{currentstroke}%
\pgfsetdash{}{0pt}%
\pgfpathmoveto{\pgfqpoint{2.537289in}{3.204720in}}%
\pgfpathlineto{\pgfqpoint{2.510341in}{3.189162in}}%
\pgfpathlineto{\pgfqpoint{2.568062in}{3.208406in}}%
\pgfpathlineto{\pgfqpoint{2.595009in}{3.223964in}}%
\pgfpathlineto{\pgfqpoint{2.537289in}{3.204720in}}%
\pgfpathclose%
\pgfusepath{fill}%
\end{pgfscope}%
\begin{pgfscope}%
\pgfpathrectangle{\pgfqpoint{0.765000in}{0.660000in}}{\pgfqpoint{4.620000in}{4.620000in}}%
\pgfusepath{clip}%
\pgfsetbuttcap%
\pgfsetroundjoin%
\definecolor{currentfill}{rgb}{1.000000,0.894118,0.788235}%
\pgfsetfillcolor{currentfill}%
\pgfsetlinewidth{0.000000pt}%
\definecolor{currentstroke}{rgb}{1.000000,0.894118,0.788235}%
\pgfsetstrokecolor{currentstroke}%
\pgfsetdash{}{0pt}%
\pgfpathmoveto{\pgfqpoint{2.564236in}{3.189162in}}%
\pgfpathlineto{\pgfqpoint{2.537289in}{3.204720in}}%
\pgfpathlineto{\pgfqpoint{2.595009in}{3.223964in}}%
\pgfpathlineto{\pgfqpoint{2.621957in}{3.208406in}}%
\pgfpathlineto{\pgfqpoint{2.564236in}{3.189162in}}%
\pgfpathclose%
\pgfusepath{fill}%
\end{pgfscope}%
\begin{pgfscope}%
\pgfpathrectangle{\pgfqpoint{0.765000in}{0.660000in}}{\pgfqpoint{4.620000in}{4.620000in}}%
\pgfusepath{clip}%
\pgfsetbuttcap%
\pgfsetroundjoin%
\definecolor{currentfill}{rgb}{1.000000,0.894118,0.788235}%
\pgfsetfillcolor{currentfill}%
\pgfsetlinewidth{0.000000pt}%
\definecolor{currentstroke}{rgb}{1.000000,0.894118,0.788235}%
\pgfsetstrokecolor{currentstroke}%
\pgfsetdash{}{0pt}%
\pgfpathmoveto{\pgfqpoint{2.537289in}{3.204720in}}%
\pgfpathlineto{\pgfqpoint{2.537289in}{3.235836in}}%
\pgfpathlineto{\pgfqpoint{2.595009in}{3.255080in}}%
\pgfpathlineto{\pgfqpoint{2.621957in}{3.208406in}}%
\pgfpathlineto{\pgfqpoint{2.537289in}{3.204720in}}%
\pgfpathclose%
\pgfusepath{fill}%
\end{pgfscope}%
\begin{pgfscope}%
\pgfpathrectangle{\pgfqpoint{0.765000in}{0.660000in}}{\pgfqpoint{4.620000in}{4.620000in}}%
\pgfusepath{clip}%
\pgfsetbuttcap%
\pgfsetroundjoin%
\definecolor{currentfill}{rgb}{1.000000,0.894118,0.788235}%
\pgfsetfillcolor{currentfill}%
\pgfsetlinewidth{0.000000pt}%
\definecolor{currentstroke}{rgb}{1.000000,0.894118,0.788235}%
\pgfsetstrokecolor{currentstroke}%
\pgfsetdash{}{0pt}%
\pgfpathmoveto{\pgfqpoint{2.564236in}{3.189162in}}%
\pgfpathlineto{\pgfqpoint{2.564236in}{3.220278in}}%
\pgfpathlineto{\pgfqpoint{2.621957in}{3.239522in}}%
\pgfpathlineto{\pgfqpoint{2.595009in}{3.223964in}}%
\pgfpathlineto{\pgfqpoint{2.564236in}{3.189162in}}%
\pgfpathclose%
\pgfusepath{fill}%
\end{pgfscope}%
\begin{pgfscope}%
\pgfpathrectangle{\pgfqpoint{0.765000in}{0.660000in}}{\pgfqpoint{4.620000in}{4.620000in}}%
\pgfusepath{clip}%
\pgfsetbuttcap%
\pgfsetroundjoin%
\definecolor{currentfill}{rgb}{1.000000,0.894118,0.788235}%
\pgfsetfillcolor{currentfill}%
\pgfsetlinewidth{0.000000pt}%
\definecolor{currentstroke}{rgb}{1.000000,0.894118,0.788235}%
\pgfsetstrokecolor{currentstroke}%
\pgfsetdash{}{0pt}%
\pgfpathmoveto{\pgfqpoint{2.510341in}{3.189162in}}%
\pgfpathlineto{\pgfqpoint{2.537289in}{3.173604in}}%
\pgfpathlineto{\pgfqpoint{2.595009in}{3.192848in}}%
\pgfpathlineto{\pgfqpoint{2.568062in}{3.208406in}}%
\pgfpathlineto{\pgfqpoint{2.510341in}{3.189162in}}%
\pgfpathclose%
\pgfusepath{fill}%
\end{pgfscope}%
\begin{pgfscope}%
\pgfpathrectangle{\pgfqpoint{0.765000in}{0.660000in}}{\pgfqpoint{4.620000in}{4.620000in}}%
\pgfusepath{clip}%
\pgfsetbuttcap%
\pgfsetroundjoin%
\definecolor{currentfill}{rgb}{1.000000,0.894118,0.788235}%
\pgfsetfillcolor{currentfill}%
\pgfsetlinewidth{0.000000pt}%
\definecolor{currentstroke}{rgb}{1.000000,0.894118,0.788235}%
\pgfsetstrokecolor{currentstroke}%
\pgfsetdash{}{0pt}%
\pgfpathmoveto{\pgfqpoint{2.537289in}{3.173604in}}%
\pgfpathlineto{\pgfqpoint{2.564236in}{3.189162in}}%
\pgfpathlineto{\pgfqpoint{2.621957in}{3.208406in}}%
\pgfpathlineto{\pgfqpoint{2.595009in}{3.192848in}}%
\pgfpathlineto{\pgfqpoint{2.537289in}{3.173604in}}%
\pgfpathclose%
\pgfusepath{fill}%
\end{pgfscope}%
\begin{pgfscope}%
\pgfpathrectangle{\pgfqpoint{0.765000in}{0.660000in}}{\pgfqpoint{4.620000in}{4.620000in}}%
\pgfusepath{clip}%
\pgfsetbuttcap%
\pgfsetroundjoin%
\definecolor{currentfill}{rgb}{1.000000,0.894118,0.788235}%
\pgfsetfillcolor{currentfill}%
\pgfsetlinewidth{0.000000pt}%
\definecolor{currentstroke}{rgb}{1.000000,0.894118,0.788235}%
\pgfsetstrokecolor{currentstroke}%
\pgfsetdash{}{0pt}%
\pgfpathmoveto{\pgfqpoint{2.537289in}{3.235836in}}%
\pgfpathlineto{\pgfqpoint{2.510341in}{3.220278in}}%
\pgfpathlineto{\pgfqpoint{2.568062in}{3.239522in}}%
\pgfpathlineto{\pgfqpoint{2.595009in}{3.255080in}}%
\pgfpathlineto{\pgfqpoint{2.537289in}{3.235836in}}%
\pgfpathclose%
\pgfusepath{fill}%
\end{pgfscope}%
\begin{pgfscope}%
\pgfpathrectangle{\pgfqpoint{0.765000in}{0.660000in}}{\pgfqpoint{4.620000in}{4.620000in}}%
\pgfusepath{clip}%
\pgfsetbuttcap%
\pgfsetroundjoin%
\definecolor{currentfill}{rgb}{1.000000,0.894118,0.788235}%
\pgfsetfillcolor{currentfill}%
\pgfsetlinewidth{0.000000pt}%
\definecolor{currentstroke}{rgb}{1.000000,0.894118,0.788235}%
\pgfsetstrokecolor{currentstroke}%
\pgfsetdash{}{0pt}%
\pgfpathmoveto{\pgfqpoint{2.564236in}{3.220278in}}%
\pgfpathlineto{\pgfqpoint{2.537289in}{3.235836in}}%
\pgfpathlineto{\pgfqpoint{2.595009in}{3.255080in}}%
\pgfpathlineto{\pgfqpoint{2.621957in}{3.239522in}}%
\pgfpathlineto{\pgfqpoint{2.564236in}{3.220278in}}%
\pgfpathclose%
\pgfusepath{fill}%
\end{pgfscope}%
\begin{pgfscope}%
\pgfpathrectangle{\pgfqpoint{0.765000in}{0.660000in}}{\pgfqpoint{4.620000in}{4.620000in}}%
\pgfusepath{clip}%
\pgfsetbuttcap%
\pgfsetroundjoin%
\definecolor{currentfill}{rgb}{1.000000,0.894118,0.788235}%
\pgfsetfillcolor{currentfill}%
\pgfsetlinewidth{0.000000pt}%
\definecolor{currentstroke}{rgb}{1.000000,0.894118,0.788235}%
\pgfsetstrokecolor{currentstroke}%
\pgfsetdash{}{0pt}%
\pgfpathmoveto{\pgfqpoint{2.510341in}{3.189162in}}%
\pgfpathlineto{\pgfqpoint{2.510341in}{3.220278in}}%
\pgfpathlineto{\pgfqpoint{2.568062in}{3.239522in}}%
\pgfpathlineto{\pgfqpoint{2.595009in}{3.192848in}}%
\pgfpathlineto{\pgfqpoint{2.510341in}{3.189162in}}%
\pgfpathclose%
\pgfusepath{fill}%
\end{pgfscope}%
\begin{pgfscope}%
\pgfpathrectangle{\pgfqpoint{0.765000in}{0.660000in}}{\pgfqpoint{4.620000in}{4.620000in}}%
\pgfusepath{clip}%
\pgfsetbuttcap%
\pgfsetroundjoin%
\definecolor{currentfill}{rgb}{1.000000,0.894118,0.788235}%
\pgfsetfillcolor{currentfill}%
\pgfsetlinewidth{0.000000pt}%
\definecolor{currentstroke}{rgb}{1.000000,0.894118,0.788235}%
\pgfsetstrokecolor{currentstroke}%
\pgfsetdash{}{0pt}%
\pgfpathmoveto{\pgfqpoint{2.537289in}{3.173604in}}%
\pgfpathlineto{\pgfqpoint{2.537289in}{3.204720in}}%
\pgfpathlineto{\pgfqpoint{2.595009in}{3.223964in}}%
\pgfpathlineto{\pgfqpoint{2.568062in}{3.208406in}}%
\pgfpathlineto{\pgfqpoint{2.537289in}{3.173604in}}%
\pgfpathclose%
\pgfusepath{fill}%
\end{pgfscope}%
\begin{pgfscope}%
\pgfpathrectangle{\pgfqpoint{0.765000in}{0.660000in}}{\pgfqpoint{4.620000in}{4.620000in}}%
\pgfusepath{clip}%
\pgfsetbuttcap%
\pgfsetroundjoin%
\definecolor{currentfill}{rgb}{1.000000,0.894118,0.788235}%
\pgfsetfillcolor{currentfill}%
\pgfsetlinewidth{0.000000pt}%
\definecolor{currentstroke}{rgb}{1.000000,0.894118,0.788235}%
\pgfsetstrokecolor{currentstroke}%
\pgfsetdash{}{0pt}%
\pgfpathmoveto{\pgfqpoint{2.537289in}{3.204720in}}%
\pgfpathlineto{\pgfqpoint{2.564236in}{3.220278in}}%
\pgfpathlineto{\pgfqpoint{2.621957in}{3.239522in}}%
\pgfpathlineto{\pgfqpoint{2.595009in}{3.223964in}}%
\pgfpathlineto{\pgfqpoint{2.537289in}{3.204720in}}%
\pgfpathclose%
\pgfusepath{fill}%
\end{pgfscope}%
\begin{pgfscope}%
\pgfpathrectangle{\pgfqpoint{0.765000in}{0.660000in}}{\pgfqpoint{4.620000in}{4.620000in}}%
\pgfusepath{clip}%
\pgfsetbuttcap%
\pgfsetroundjoin%
\definecolor{currentfill}{rgb}{1.000000,0.894118,0.788235}%
\pgfsetfillcolor{currentfill}%
\pgfsetlinewidth{0.000000pt}%
\definecolor{currentstroke}{rgb}{1.000000,0.894118,0.788235}%
\pgfsetstrokecolor{currentstroke}%
\pgfsetdash{}{0pt}%
\pgfpathmoveto{\pgfqpoint{2.510341in}{3.220278in}}%
\pgfpathlineto{\pgfqpoint{2.537289in}{3.204720in}}%
\pgfpathlineto{\pgfqpoint{2.595009in}{3.223964in}}%
\pgfpathlineto{\pgfqpoint{2.568062in}{3.239522in}}%
\pgfpathlineto{\pgfqpoint{2.510341in}{3.220278in}}%
\pgfpathclose%
\pgfusepath{fill}%
\end{pgfscope}%
\begin{pgfscope}%
\pgfpathrectangle{\pgfqpoint{0.765000in}{0.660000in}}{\pgfqpoint{4.620000in}{4.620000in}}%
\pgfusepath{clip}%
\pgfsetbuttcap%
\pgfsetroundjoin%
\definecolor{currentfill}{rgb}{1.000000,0.894118,0.788235}%
\pgfsetfillcolor{currentfill}%
\pgfsetlinewidth{0.000000pt}%
\definecolor{currentstroke}{rgb}{1.000000,0.894118,0.788235}%
\pgfsetstrokecolor{currentstroke}%
\pgfsetdash{}{0pt}%
\pgfpathmoveto{\pgfqpoint{2.537289in}{3.204720in}}%
\pgfpathlineto{\pgfqpoint{2.510341in}{3.189162in}}%
\pgfpathlineto{\pgfqpoint{2.537289in}{3.173604in}}%
\pgfpathlineto{\pgfqpoint{2.564236in}{3.189162in}}%
\pgfpathlineto{\pgfqpoint{2.537289in}{3.204720in}}%
\pgfpathclose%
\pgfusepath{fill}%
\end{pgfscope}%
\begin{pgfscope}%
\pgfpathrectangle{\pgfqpoint{0.765000in}{0.660000in}}{\pgfqpoint{4.620000in}{4.620000in}}%
\pgfusepath{clip}%
\pgfsetbuttcap%
\pgfsetroundjoin%
\definecolor{currentfill}{rgb}{1.000000,0.894118,0.788235}%
\pgfsetfillcolor{currentfill}%
\pgfsetlinewidth{0.000000pt}%
\definecolor{currentstroke}{rgb}{1.000000,0.894118,0.788235}%
\pgfsetstrokecolor{currentstroke}%
\pgfsetdash{}{0pt}%
\pgfpathmoveto{\pgfqpoint{2.537289in}{3.204720in}}%
\pgfpathlineto{\pgfqpoint{2.510341in}{3.189162in}}%
\pgfpathlineto{\pgfqpoint{2.510341in}{3.220278in}}%
\pgfpathlineto{\pgfqpoint{2.537289in}{3.235836in}}%
\pgfpathlineto{\pgfqpoint{2.537289in}{3.204720in}}%
\pgfpathclose%
\pgfusepath{fill}%
\end{pgfscope}%
\begin{pgfscope}%
\pgfpathrectangle{\pgfqpoint{0.765000in}{0.660000in}}{\pgfqpoint{4.620000in}{4.620000in}}%
\pgfusepath{clip}%
\pgfsetbuttcap%
\pgfsetroundjoin%
\definecolor{currentfill}{rgb}{1.000000,0.894118,0.788235}%
\pgfsetfillcolor{currentfill}%
\pgfsetlinewidth{0.000000pt}%
\definecolor{currentstroke}{rgb}{1.000000,0.894118,0.788235}%
\pgfsetstrokecolor{currentstroke}%
\pgfsetdash{}{0pt}%
\pgfpathmoveto{\pgfqpoint{2.537289in}{3.204720in}}%
\pgfpathlineto{\pgfqpoint{2.564236in}{3.189162in}}%
\pgfpathlineto{\pgfqpoint{2.564236in}{3.220278in}}%
\pgfpathlineto{\pgfqpoint{2.537289in}{3.235836in}}%
\pgfpathlineto{\pgfqpoint{2.537289in}{3.204720in}}%
\pgfpathclose%
\pgfusepath{fill}%
\end{pgfscope}%
\begin{pgfscope}%
\pgfpathrectangle{\pgfqpoint{0.765000in}{0.660000in}}{\pgfqpoint{4.620000in}{4.620000in}}%
\pgfusepath{clip}%
\pgfsetbuttcap%
\pgfsetroundjoin%
\definecolor{currentfill}{rgb}{1.000000,0.894118,0.788235}%
\pgfsetfillcolor{currentfill}%
\pgfsetlinewidth{0.000000pt}%
\definecolor{currentstroke}{rgb}{1.000000,0.894118,0.788235}%
\pgfsetstrokecolor{currentstroke}%
\pgfsetdash{}{0pt}%
\pgfpathmoveto{\pgfqpoint{2.421256in}{3.149423in}}%
\pgfpathlineto{\pgfqpoint{2.394309in}{3.133865in}}%
\pgfpathlineto{\pgfqpoint{2.421256in}{3.118307in}}%
\pgfpathlineto{\pgfqpoint{2.448203in}{3.133865in}}%
\pgfpathlineto{\pgfqpoint{2.421256in}{3.149423in}}%
\pgfpathclose%
\pgfusepath{fill}%
\end{pgfscope}%
\begin{pgfscope}%
\pgfpathrectangle{\pgfqpoint{0.765000in}{0.660000in}}{\pgfqpoint{4.620000in}{4.620000in}}%
\pgfusepath{clip}%
\pgfsetbuttcap%
\pgfsetroundjoin%
\definecolor{currentfill}{rgb}{1.000000,0.894118,0.788235}%
\pgfsetfillcolor{currentfill}%
\pgfsetlinewidth{0.000000pt}%
\definecolor{currentstroke}{rgb}{1.000000,0.894118,0.788235}%
\pgfsetstrokecolor{currentstroke}%
\pgfsetdash{}{0pt}%
\pgfpathmoveto{\pgfqpoint{2.421256in}{3.149423in}}%
\pgfpathlineto{\pgfqpoint{2.394309in}{3.133865in}}%
\pgfpathlineto{\pgfqpoint{2.394309in}{3.164981in}}%
\pgfpathlineto{\pgfqpoint{2.421256in}{3.180539in}}%
\pgfpathlineto{\pgfqpoint{2.421256in}{3.149423in}}%
\pgfpathclose%
\pgfusepath{fill}%
\end{pgfscope}%
\begin{pgfscope}%
\pgfpathrectangle{\pgfqpoint{0.765000in}{0.660000in}}{\pgfqpoint{4.620000in}{4.620000in}}%
\pgfusepath{clip}%
\pgfsetbuttcap%
\pgfsetroundjoin%
\definecolor{currentfill}{rgb}{1.000000,0.894118,0.788235}%
\pgfsetfillcolor{currentfill}%
\pgfsetlinewidth{0.000000pt}%
\definecolor{currentstroke}{rgb}{1.000000,0.894118,0.788235}%
\pgfsetstrokecolor{currentstroke}%
\pgfsetdash{}{0pt}%
\pgfpathmoveto{\pgfqpoint{2.421256in}{3.149423in}}%
\pgfpathlineto{\pgfqpoint{2.448203in}{3.133865in}}%
\pgfpathlineto{\pgfqpoint{2.448203in}{3.164981in}}%
\pgfpathlineto{\pgfqpoint{2.421256in}{3.180539in}}%
\pgfpathlineto{\pgfqpoint{2.421256in}{3.149423in}}%
\pgfpathclose%
\pgfusepath{fill}%
\end{pgfscope}%
\begin{pgfscope}%
\pgfpathrectangle{\pgfqpoint{0.765000in}{0.660000in}}{\pgfqpoint{4.620000in}{4.620000in}}%
\pgfusepath{clip}%
\pgfsetbuttcap%
\pgfsetroundjoin%
\definecolor{currentfill}{rgb}{1.000000,0.894118,0.788235}%
\pgfsetfillcolor{currentfill}%
\pgfsetlinewidth{0.000000pt}%
\definecolor{currentstroke}{rgb}{1.000000,0.894118,0.788235}%
\pgfsetstrokecolor{currentstroke}%
\pgfsetdash{}{0pt}%
\pgfpathmoveto{\pgfqpoint{2.537289in}{3.235836in}}%
\pgfpathlineto{\pgfqpoint{2.510341in}{3.220278in}}%
\pgfpathlineto{\pgfqpoint{2.537289in}{3.204720in}}%
\pgfpathlineto{\pgfqpoint{2.564236in}{3.220278in}}%
\pgfpathlineto{\pgfqpoint{2.537289in}{3.235836in}}%
\pgfpathclose%
\pgfusepath{fill}%
\end{pgfscope}%
\begin{pgfscope}%
\pgfpathrectangle{\pgfqpoint{0.765000in}{0.660000in}}{\pgfqpoint{4.620000in}{4.620000in}}%
\pgfusepath{clip}%
\pgfsetbuttcap%
\pgfsetroundjoin%
\definecolor{currentfill}{rgb}{1.000000,0.894118,0.788235}%
\pgfsetfillcolor{currentfill}%
\pgfsetlinewidth{0.000000pt}%
\definecolor{currentstroke}{rgb}{1.000000,0.894118,0.788235}%
\pgfsetstrokecolor{currentstroke}%
\pgfsetdash{}{0pt}%
\pgfpathmoveto{\pgfqpoint{2.537289in}{3.173604in}}%
\pgfpathlineto{\pgfqpoint{2.564236in}{3.189162in}}%
\pgfpathlineto{\pgfqpoint{2.564236in}{3.220278in}}%
\pgfpathlineto{\pgfqpoint{2.537289in}{3.204720in}}%
\pgfpathlineto{\pgfqpoint{2.537289in}{3.173604in}}%
\pgfpathclose%
\pgfusepath{fill}%
\end{pgfscope}%
\begin{pgfscope}%
\pgfpathrectangle{\pgfqpoint{0.765000in}{0.660000in}}{\pgfqpoint{4.620000in}{4.620000in}}%
\pgfusepath{clip}%
\pgfsetbuttcap%
\pgfsetroundjoin%
\definecolor{currentfill}{rgb}{1.000000,0.894118,0.788235}%
\pgfsetfillcolor{currentfill}%
\pgfsetlinewidth{0.000000pt}%
\definecolor{currentstroke}{rgb}{1.000000,0.894118,0.788235}%
\pgfsetstrokecolor{currentstroke}%
\pgfsetdash{}{0pt}%
\pgfpathmoveto{\pgfqpoint{2.510341in}{3.189162in}}%
\pgfpathlineto{\pgfqpoint{2.537289in}{3.173604in}}%
\pgfpathlineto{\pgfqpoint{2.537289in}{3.204720in}}%
\pgfpathlineto{\pgfqpoint{2.510341in}{3.220278in}}%
\pgfpathlineto{\pgfqpoint{2.510341in}{3.189162in}}%
\pgfpathclose%
\pgfusepath{fill}%
\end{pgfscope}%
\begin{pgfscope}%
\pgfpathrectangle{\pgfqpoint{0.765000in}{0.660000in}}{\pgfqpoint{4.620000in}{4.620000in}}%
\pgfusepath{clip}%
\pgfsetbuttcap%
\pgfsetroundjoin%
\definecolor{currentfill}{rgb}{1.000000,0.894118,0.788235}%
\pgfsetfillcolor{currentfill}%
\pgfsetlinewidth{0.000000pt}%
\definecolor{currentstroke}{rgb}{1.000000,0.894118,0.788235}%
\pgfsetstrokecolor{currentstroke}%
\pgfsetdash{}{0pt}%
\pgfpathmoveto{\pgfqpoint{2.421256in}{3.180539in}}%
\pgfpathlineto{\pgfqpoint{2.394309in}{3.164981in}}%
\pgfpathlineto{\pgfqpoint{2.421256in}{3.149423in}}%
\pgfpathlineto{\pgfqpoint{2.448203in}{3.164981in}}%
\pgfpathlineto{\pgfqpoint{2.421256in}{3.180539in}}%
\pgfpathclose%
\pgfusepath{fill}%
\end{pgfscope}%
\begin{pgfscope}%
\pgfpathrectangle{\pgfqpoint{0.765000in}{0.660000in}}{\pgfqpoint{4.620000in}{4.620000in}}%
\pgfusepath{clip}%
\pgfsetbuttcap%
\pgfsetroundjoin%
\definecolor{currentfill}{rgb}{1.000000,0.894118,0.788235}%
\pgfsetfillcolor{currentfill}%
\pgfsetlinewidth{0.000000pt}%
\definecolor{currentstroke}{rgb}{1.000000,0.894118,0.788235}%
\pgfsetstrokecolor{currentstroke}%
\pgfsetdash{}{0pt}%
\pgfpathmoveto{\pgfqpoint{2.421256in}{3.118307in}}%
\pgfpathlineto{\pgfqpoint{2.448203in}{3.133865in}}%
\pgfpathlineto{\pgfqpoint{2.448203in}{3.164981in}}%
\pgfpathlineto{\pgfqpoint{2.421256in}{3.149423in}}%
\pgfpathlineto{\pgfqpoint{2.421256in}{3.118307in}}%
\pgfpathclose%
\pgfusepath{fill}%
\end{pgfscope}%
\begin{pgfscope}%
\pgfpathrectangle{\pgfqpoint{0.765000in}{0.660000in}}{\pgfqpoint{4.620000in}{4.620000in}}%
\pgfusepath{clip}%
\pgfsetbuttcap%
\pgfsetroundjoin%
\definecolor{currentfill}{rgb}{1.000000,0.894118,0.788235}%
\pgfsetfillcolor{currentfill}%
\pgfsetlinewidth{0.000000pt}%
\definecolor{currentstroke}{rgb}{1.000000,0.894118,0.788235}%
\pgfsetstrokecolor{currentstroke}%
\pgfsetdash{}{0pt}%
\pgfpathmoveto{\pgfqpoint{2.394309in}{3.133865in}}%
\pgfpathlineto{\pgfqpoint{2.421256in}{3.118307in}}%
\pgfpathlineto{\pgfqpoint{2.421256in}{3.149423in}}%
\pgfpathlineto{\pgfqpoint{2.394309in}{3.164981in}}%
\pgfpathlineto{\pgfqpoint{2.394309in}{3.133865in}}%
\pgfpathclose%
\pgfusepath{fill}%
\end{pgfscope}%
\begin{pgfscope}%
\pgfpathrectangle{\pgfqpoint{0.765000in}{0.660000in}}{\pgfqpoint{4.620000in}{4.620000in}}%
\pgfusepath{clip}%
\pgfsetbuttcap%
\pgfsetroundjoin%
\definecolor{currentfill}{rgb}{1.000000,0.894118,0.788235}%
\pgfsetfillcolor{currentfill}%
\pgfsetlinewidth{0.000000pt}%
\definecolor{currentstroke}{rgb}{1.000000,0.894118,0.788235}%
\pgfsetstrokecolor{currentstroke}%
\pgfsetdash{}{0pt}%
\pgfpathmoveto{\pgfqpoint{2.537289in}{3.204720in}}%
\pgfpathlineto{\pgfqpoint{2.510341in}{3.189162in}}%
\pgfpathlineto{\pgfqpoint{2.394309in}{3.133865in}}%
\pgfpathlineto{\pgfqpoint{2.421256in}{3.149423in}}%
\pgfpathlineto{\pgfqpoint{2.537289in}{3.204720in}}%
\pgfpathclose%
\pgfusepath{fill}%
\end{pgfscope}%
\begin{pgfscope}%
\pgfpathrectangle{\pgfqpoint{0.765000in}{0.660000in}}{\pgfqpoint{4.620000in}{4.620000in}}%
\pgfusepath{clip}%
\pgfsetbuttcap%
\pgfsetroundjoin%
\definecolor{currentfill}{rgb}{1.000000,0.894118,0.788235}%
\pgfsetfillcolor{currentfill}%
\pgfsetlinewidth{0.000000pt}%
\definecolor{currentstroke}{rgb}{1.000000,0.894118,0.788235}%
\pgfsetstrokecolor{currentstroke}%
\pgfsetdash{}{0pt}%
\pgfpathmoveto{\pgfqpoint{2.564236in}{3.189162in}}%
\pgfpathlineto{\pgfqpoint{2.537289in}{3.204720in}}%
\pgfpathlineto{\pgfqpoint{2.421256in}{3.149423in}}%
\pgfpathlineto{\pgfqpoint{2.448203in}{3.133865in}}%
\pgfpathlineto{\pgfqpoint{2.564236in}{3.189162in}}%
\pgfpathclose%
\pgfusepath{fill}%
\end{pgfscope}%
\begin{pgfscope}%
\pgfpathrectangle{\pgfqpoint{0.765000in}{0.660000in}}{\pgfqpoint{4.620000in}{4.620000in}}%
\pgfusepath{clip}%
\pgfsetbuttcap%
\pgfsetroundjoin%
\definecolor{currentfill}{rgb}{1.000000,0.894118,0.788235}%
\pgfsetfillcolor{currentfill}%
\pgfsetlinewidth{0.000000pt}%
\definecolor{currentstroke}{rgb}{1.000000,0.894118,0.788235}%
\pgfsetstrokecolor{currentstroke}%
\pgfsetdash{}{0pt}%
\pgfpathmoveto{\pgfqpoint{2.537289in}{3.204720in}}%
\pgfpathlineto{\pgfqpoint{2.537289in}{3.235836in}}%
\pgfpathlineto{\pgfqpoint{2.421256in}{3.180539in}}%
\pgfpathlineto{\pgfqpoint{2.448203in}{3.133865in}}%
\pgfpathlineto{\pgfqpoint{2.537289in}{3.204720in}}%
\pgfpathclose%
\pgfusepath{fill}%
\end{pgfscope}%
\begin{pgfscope}%
\pgfpathrectangle{\pgfqpoint{0.765000in}{0.660000in}}{\pgfqpoint{4.620000in}{4.620000in}}%
\pgfusepath{clip}%
\pgfsetbuttcap%
\pgfsetroundjoin%
\definecolor{currentfill}{rgb}{1.000000,0.894118,0.788235}%
\pgfsetfillcolor{currentfill}%
\pgfsetlinewidth{0.000000pt}%
\definecolor{currentstroke}{rgb}{1.000000,0.894118,0.788235}%
\pgfsetstrokecolor{currentstroke}%
\pgfsetdash{}{0pt}%
\pgfpathmoveto{\pgfqpoint{2.564236in}{3.189162in}}%
\pgfpathlineto{\pgfqpoint{2.564236in}{3.220278in}}%
\pgfpathlineto{\pgfqpoint{2.448203in}{3.164981in}}%
\pgfpathlineto{\pgfqpoint{2.421256in}{3.149423in}}%
\pgfpathlineto{\pgfqpoint{2.564236in}{3.189162in}}%
\pgfpathclose%
\pgfusepath{fill}%
\end{pgfscope}%
\begin{pgfscope}%
\pgfpathrectangle{\pgfqpoint{0.765000in}{0.660000in}}{\pgfqpoint{4.620000in}{4.620000in}}%
\pgfusepath{clip}%
\pgfsetbuttcap%
\pgfsetroundjoin%
\definecolor{currentfill}{rgb}{1.000000,0.894118,0.788235}%
\pgfsetfillcolor{currentfill}%
\pgfsetlinewidth{0.000000pt}%
\definecolor{currentstroke}{rgb}{1.000000,0.894118,0.788235}%
\pgfsetstrokecolor{currentstroke}%
\pgfsetdash{}{0pt}%
\pgfpathmoveto{\pgfqpoint{2.510341in}{3.189162in}}%
\pgfpathlineto{\pgfqpoint{2.537289in}{3.173604in}}%
\pgfpathlineto{\pgfqpoint{2.421256in}{3.118307in}}%
\pgfpathlineto{\pgfqpoint{2.394309in}{3.133865in}}%
\pgfpathlineto{\pgfqpoint{2.510341in}{3.189162in}}%
\pgfpathclose%
\pgfusepath{fill}%
\end{pgfscope}%
\begin{pgfscope}%
\pgfpathrectangle{\pgfqpoint{0.765000in}{0.660000in}}{\pgfqpoint{4.620000in}{4.620000in}}%
\pgfusepath{clip}%
\pgfsetbuttcap%
\pgfsetroundjoin%
\definecolor{currentfill}{rgb}{1.000000,0.894118,0.788235}%
\pgfsetfillcolor{currentfill}%
\pgfsetlinewidth{0.000000pt}%
\definecolor{currentstroke}{rgb}{1.000000,0.894118,0.788235}%
\pgfsetstrokecolor{currentstroke}%
\pgfsetdash{}{0pt}%
\pgfpathmoveto{\pgfqpoint{2.537289in}{3.173604in}}%
\pgfpathlineto{\pgfqpoint{2.564236in}{3.189162in}}%
\pgfpathlineto{\pgfqpoint{2.448203in}{3.133865in}}%
\pgfpathlineto{\pgfqpoint{2.421256in}{3.118307in}}%
\pgfpathlineto{\pgfqpoint{2.537289in}{3.173604in}}%
\pgfpathclose%
\pgfusepath{fill}%
\end{pgfscope}%
\begin{pgfscope}%
\pgfpathrectangle{\pgfqpoint{0.765000in}{0.660000in}}{\pgfqpoint{4.620000in}{4.620000in}}%
\pgfusepath{clip}%
\pgfsetbuttcap%
\pgfsetroundjoin%
\definecolor{currentfill}{rgb}{1.000000,0.894118,0.788235}%
\pgfsetfillcolor{currentfill}%
\pgfsetlinewidth{0.000000pt}%
\definecolor{currentstroke}{rgb}{1.000000,0.894118,0.788235}%
\pgfsetstrokecolor{currentstroke}%
\pgfsetdash{}{0pt}%
\pgfpathmoveto{\pgfqpoint{2.537289in}{3.235836in}}%
\pgfpathlineto{\pgfqpoint{2.510341in}{3.220278in}}%
\pgfpathlineto{\pgfqpoint{2.394309in}{3.164981in}}%
\pgfpathlineto{\pgfqpoint{2.421256in}{3.180539in}}%
\pgfpathlineto{\pgfqpoint{2.537289in}{3.235836in}}%
\pgfpathclose%
\pgfusepath{fill}%
\end{pgfscope}%
\begin{pgfscope}%
\pgfpathrectangle{\pgfqpoint{0.765000in}{0.660000in}}{\pgfqpoint{4.620000in}{4.620000in}}%
\pgfusepath{clip}%
\pgfsetbuttcap%
\pgfsetroundjoin%
\definecolor{currentfill}{rgb}{1.000000,0.894118,0.788235}%
\pgfsetfillcolor{currentfill}%
\pgfsetlinewidth{0.000000pt}%
\definecolor{currentstroke}{rgb}{1.000000,0.894118,0.788235}%
\pgfsetstrokecolor{currentstroke}%
\pgfsetdash{}{0pt}%
\pgfpathmoveto{\pgfqpoint{2.564236in}{3.220278in}}%
\pgfpathlineto{\pgfqpoint{2.537289in}{3.235836in}}%
\pgfpathlineto{\pgfqpoint{2.421256in}{3.180539in}}%
\pgfpathlineto{\pgfqpoint{2.448203in}{3.164981in}}%
\pgfpathlineto{\pgfqpoint{2.564236in}{3.220278in}}%
\pgfpathclose%
\pgfusepath{fill}%
\end{pgfscope}%
\begin{pgfscope}%
\pgfpathrectangle{\pgfqpoint{0.765000in}{0.660000in}}{\pgfqpoint{4.620000in}{4.620000in}}%
\pgfusepath{clip}%
\pgfsetbuttcap%
\pgfsetroundjoin%
\definecolor{currentfill}{rgb}{1.000000,0.894118,0.788235}%
\pgfsetfillcolor{currentfill}%
\pgfsetlinewidth{0.000000pt}%
\definecolor{currentstroke}{rgb}{1.000000,0.894118,0.788235}%
\pgfsetstrokecolor{currentstroke}%
\pgfsetdash{}{0pt}%
\pgfpathmoveto{\pgfqpoint{2.510341in}{3.189162in}}%
\pgfpathlineto{\pgfqpoint{2.510341in}{3.220278in}}%
\pgfpathlineto{\pgfqpoint{2.394309in}{3.164981in}}%
\pgfpathlineto{\pgfqpoint{2.421256in}{3.118307in}}%
\pgfpathlineto{\pgfqpoint{2.510341in}{3.189162in}}%
\pgfpathclose%
\pgfusepath{fill}%
\end{pgfscope}%
\begin{pgfscope}%
\pgfpathrectangle{\pgfqpoint{0.765000in}{0.660000in}}{\pgfqpoint{4.620000in}{4.620000in}}%
\pgfusepath{clip}%
\pgfsetbuttcap%
\pgfsetroundjoin%
\definecolor{currentfill}{rgb}{1.000000,0.894118,0.788235}%
\pgfsetfillcolor{currentfill}%
\pgfsetlinewidth{0.000000pt}%
\definecolor{currentstroke}{rgb}{1.000000,0.894118,0.788235}%
\pgfsetstrokecolor{currentstroke}%
\pgfsetdash{}{0pt}%
\pgfpathmoveto{\pgfqpoint{2.537289in}{3.173604in}}%
\pgfpathlineto{\pgfqpoint{2.537289in}{3.204720in}}%
\pgfpathlineto{\pgfqpoint{2.421256in}{3.149423in}}%
\pgfpathlineto{\pgfqpoint{2.394309in}{3.133865in}}%
\pgfpathlineto{\pgfqpoint{2.537289in}{3.173604in}}%
\pgfpathclose%
\pgfusepath{fill}%
\end{pgfscope}%
\begin{pgfscope}%
\pgfpathrectangle{\pgfqpoint{0.765000in}{0.660000in}}{\pgfqpoint{4.620000in}{4.620000in}}%
\pgfusepath{clip}%
\pgfsetbuttcap%
\pgfsetroundjoin%
\definecolor{currentfill}{rgb}{1.000000,0.894118,0.788235}%
\pgfsetfillcolor{currentfill}%
\pgfsetlinewidth{0.000000pt}%
\definecolor{currentstroke}{rgb}{1.000000,0.894118,0.788235}%
\pgfsetstrokecolor{currentstroke}%
\pgfsetdash{}{0pt}%
\pgfpathmoveto{\pgfqpoint{2.537289in}{3.204720in}}%
\pgfpathlineto{\pgfqpoint{2.564236in}{3.220278in}}%
\pgfpathlineto{\pgfqpoint{2.448203in}{3.164981in}}%
\pgfpathlineto{\pgfqpoint{2.421256in}{3.149423in}}%
\pgfpathlineto{\pgfqpoint{2.537289in}{3.204720in}}%
\pgfpathclose%
\pgfusepath{fill}%
\end{pgfscope}%
\begin{pgfscope}%
\pgfpathrectangle{\pgfqpoint{0.765000in}{0.660000in}}{\pgfqpoint{4.620000in}{4.620000in}}%
\pgfusepath{clip}%
\pgfsetbuttcap%
\pgfsetroundjoin%
\definecolor{currentfill}{rgb}{1.000000,0.894118,0.788235}%
\pgfsetfillcolor{currentfill}%
\pgfsetlinewidth{0.000000pt}%
\definecolor{currentstroke}{rgb}{1.000000,0.894118,0.788235}%
\pgfsetstrokecolor{currentstroke}%
\pgfsetdash{}{0pt}%
\pgfpathmoveto{\pgfqpoint{2.510341in}{3.220278in}}%
\pgfpathlineto{\pgfqpoint{2.537289in}{3.204720in}}%
\pgfpathlineto{\pgfqpoint{2.421256in}{3.149423in}}%
\pgfpathlineto{\pgfqpoint{2.394309in}{3.164981in}}%
\pgfpathlineto{\pgfqpoint{2.510341in}{3.220278in}}%
\pgfpathclose%
\pgfusepath{fill}%
\end{pgfscope}%
\begin{pgfscope}%
\pgfpathrectangle{\pgfqpoint{0.765000in}{0.660000in}}{\pgfqpoint{4.620000in}{4.620000in}}%
\pgfusepath{clip}%
\pgfsetbuttcap%
\pgfsetroundjoin%
\definecolor{currentfill}{rgb}{1.000000,0.894118,0.788235}%
\pgfsetfillcolor{currentfill}%
\pgfsetlinewidth{0.000000pt}%
\definecolor{currentstroke}{rgb}{1.000000,0.894118,0.788235}%
\pgfsetstrokecolor{currentstroke}%
\pgfsetdash{}{0pt}%
\pgfpathmoveto{\pgfqpoint{3.890923in}{2.871313in}}%
\pgfpathlineto{\pgfqpoint{3.863975in}{2.855755in}}%
\pgfpathlineto{\pgfqpoint{3.890923in}{2.840197in}}%
\pgfpathlineto{\pgfqpoint{3.917870in}{2.855755in}}%
\pgfpathlineto{\pgfqpoint{3.890923in}{2.871313in}}%
\pgfpathclose%
\pgfusepath{fill}%
\end{pgfscope}%
\begin{pgfscope}%
\pgfpathrectangle{\pgfqpoint{0.765000in}{0.660000in}}{\pgfqpoint{4.620000in}{4.620000in}}%
\pgfusepath{clip}%
\pgfsetbuttcap%
\pgfsetroundjoin%
\definecolor{currentfill}{rgb}{1.000000,0.894118,0.788235}%
\pgfsetfillcolor{currentfill}%
\pgfsetlinewidth{0.000000pt}%
\definecolor{currentstroke}{rgb}{1.000000,0.894118,0.788235}%
\pgfsetstrokecolor{currentstroke}%
\pgfsetdash{}{0pt}%
\pgfpathmoveto{\pgfqpoint{3.890923in}{2.871313in}}%
\pgfpathlineto{\pgfqpoint{3.863975in}{2.855755in}}%
\pgfpathlineto{\pgfqpoint{3.863975in}{2.886871in}}%
\pgfpathlineto{\pgfqpoint{3.890923in}{2.902429in}}%
\pgfpathlineto{\pgfqpoint{3.890923in}{2.871313in}}%
\pgfpathclose%
\pgfusepath{fill}%
\end{pgfscope}%
\begin{pgfscope}%
\pgfpathrectangle{\pgfqpoint{0.765000in}{0.660000in}}{\pgfqpoint{4.620000in}{4.620000in}}%
\pgfusepath{clip}%
\pgfsetbuttcap%
\pgfsetroundjoin%
\definecolor{currentfill}{rgb}{1.000000,0.894118,0.788235}%
\pgfsetfillcolor{currentfill}%
\pgfsetlinewidth{0.000000pt}%
\definecolor{currentstroke}{rgb}{1.000000,0.894118,0.788235}%
\pgfsetstrokecolor{currentstroke}%
\pgfsetdash{}{0pt}%
\pgfpathmoveto{\pgfqpoint{3.890923in}{2.871313in}}%
\pgfpathlineto{\pgfqpoint{3.917870in}{2.855755in}}%
\pgfpathlineto{\pgfqpoint{3.917870in}{2.886871in}}%
\pgfpathlineto{\pgfqpoint{3.890923in}{2.902429in}}%
\pgfpathlineto{\pgfqpoint{3.890923in}{2.871313in}}%
\pgfpathclose%
\pgfusepath{fill}%
\end{pgfscope}%
\begin{pgfscope}%
\pgfpathrectangle{\pgfqpoint{0.765000in}{0.660000in}}{\pgfqpoint{4.620000in}{4.620000in}}%
\pgfusepath{clip}%
\pgfsetbuttcap%
\pgfsetroundjoin%
\definecolor{currentfill}{rgb}{1.000000,0.894118,0.788235}%
\pgfsetfillcolor{currentfill}%
\pgfsetlinewidth{0.000000pt}%
\definecolor{currentstroke}{rgb}{1.000000,0.894118,0.788235}%
\pgfsetstrokecolor{currentstroke}%
\pgfsetdash{}{0pt}%
\pgfpathmoveto{\pgfqpoint{3.886454in}{2.865404in}}%
\pgfpathlineto{\pgfqpoint{3.859507in}{2.849846in}}%
\pgfpathlineto{\pgfqpoint{3.886454in}{2.834288in}}%
\pgfpathlineto{\pgfqpoint{3.913402in}{2.849846in}}%
\pgfpathlineto{\pgfqpoint{3.886454in}{2.865404in}}%
\pgfpathclose%
\pgfusepath{fill}%
\end{pgfscope}%
\begin{pgfscope}%
\pgfpathrectangle{\pgfqpoint{0.765000in}{0.660000in}}{\pgfqpoint{4.620000in}{4.620000in}}%
\pgfusepath{clip}%
\pgfsetbuttcap%
\pgfsetroundjoin%
\definecolor{currentfill}{rgb}{1.000000,0.894118,0.788235}%
\pgfsetfillcolor{currentfill}%
\pgfsetlinewidth{0.000000pt}%
\definecolor{currentstroke}{rgb}{1.000000,0.894118,0.788235}%
\pgfsetstrokecolor{currentstroke}%
\pgfsetdash{}{0pt}%
\pgfpathmoveto{\pgfqpoint{3.886454in}{2.865404in}}%
\pgfpathlineto{\pgfqpoint{3.859507in}{2.849846in}}%
\pgfpathlineto{\pgfqpoint{3.859507in}{2.880962in}}%
\pgfpathlineto{\pgfqpoint{3.886454in}{2.896520in}}%
\pgfpathlineto{\pgfqpoint{3.886454in}{2.865404in}}%
\pgfpathclose%
\pgfusepath{fill}%
\end{pgfscope}%
\begin{pgfscope}%
\pgfpathrectangle{\pgfqpoint{0.765000in}{0.660000in}}{\pgfqpoint{4.620000in}{4.620000in}}%
\pgfusepath{clip}%
\pgfsetbuttcap%
\pgfsetroundjoin%
\definecolor{currentfill}{rgb}{1.000000,0.894118,0.788235}%
\pgfsetfillcolor{currentfill}%
\pgfsetlinewidth{0.000000pt}%
\definecolor{currentstroke}{rgb}{1.000000,0.894118,0.788235}%
\pgfsetstrokecolor{currentstroke}%
\pgfsetdash{}{0pt}%
\pgfpathmoveto{\pgfqpoint{3.886454in}{2.865404in}}%
\pgfpathlineto{\pgfqpoint{3.913402in}{2.849846in}}%
\pgfpathlineto{\pgfqpoint{3.913402in}{2.880962in}}%
\pgfpathlineto{\pgfqpoint{3.886454in}{2.896520in}}%
\pgfpathlineto{\pgfqpoint{3.886454in}{2.865404in}}%
\pgfpathclose%
\pgfusepath{fill}%
\end{pgfscope}%
\begin{pgfscope}%
\pgfpathrectangle{\pgfqpoint{0.765000in}{0.660000in}}{\pgfqpoint{4.620000in}{4.620000in}}%
\pgfusepath{clip}%
\pgfsetbuttcap%
\pgfsetroundjoin%
\definecolor{currentfill}{rgb}{1.000000,0.894118,0.788235}%
\pgfsetfillcolor{currentfill}%
\pgfsetlinewidth{0.000000pt}%
\definecolor{currentstroke}{rgb}{1.000000,0.894118,0.788235}%
\pgfsetstrokecolor{currentstroke}%
\pgfsetdash{}{0pt}%
\pgfpathmoveto{\pgfqpoint{3.890923in}{2.902429in}}%
\pgfpathlineto{\pgfqpoint{3.863975in}{2.886871in}}%
\pgfpathlineto{\pgfqpoint{3.890923in}{2.871313in}}%
\pgfpathlineto{\pgfqpoint{3.917870in}{2.886871in}}%
\pgfpathlineto{\pgfqpoint{3.890923in}{2.902429in}}%
\pgfpathclose%
\pgfusepath{fill}%
\end{pgfscope}%
\begin{pgfscope}%
\pgfpathrectangle{\pgfqpoint{0.765000in}{0.660000in}}{\pgfqpoint{4.620000in}{4.620000in}}%
\pgfusepath{clip}%
\pgfsetbuttcap%
\pgfsetroundjoin%
\definecolor{currentfill}{rgb}{1.000000,0.894118,0.788235}%
\pgfsetfillcolor{currentfill}%
\pgfsetlinewidth{0.000000pt}%
\definecolor{currentstroke}{rgb}{1.000000,0.894118,0.788235}%
\pgfsetstrokecolor{currentstroke}%
\pgfsetdash{}{0pt}%
\pgfpathmoveto{\pgfqpoint{3.890923in}{2.840197in}}%
\pgfpathlineto{\pgfqpoint{3.917870in}{2.855755in}}%
\pgfpathlineto{\pgfqpoint{3.917870in}{2.886871in}}%
\pgfpathlineto{\pgfqpoint{3.890923in}{2.871313in}}%
\pgfpathlineto{\pgfqpoint{3.890923in}{2.840197in}}%
\pgfpathclose%
\pgfusepath{fill}%
\end{pgfscope}%
\begin{pgfscope}%
\pgfpathrectangle{\pgfqpoint{0.765000in}{0.660000in}}{\pgfqpoint{4.620000in}{4.620000in}}%
\pgfusepath{clip}%
\pgfsetbuttcap%
\pgfsetroundjoin%
\definecolor{currentfill}{rgb}{1.000000,0.894118,0.788235}%
\pgfsetfillcolor{currentfill}%
\pgfsetlinewidth{0.000000pt}%
\definecolor{currentstroke}{rgb}{1.000000,0.894118,0.788235}%
\pgfsetstrokecolor{currentstroke}%
\pgfsetdash{}{0pt}%
\pgfpathmoveto{\pgfqpoint{3.863975in}{2.855755in}}%
\pgfpathlineto{\pgfqpoint{3.890923in}{2.840197in}}%
\pgfpathlineto{\pgfqpoint{3.890923in}{2.871313in}}%
\pgfpathlineto{\pgfqpoint{3.863975in}{2.886871in}}%
\pgfpathlineto{\pgfqpoint{3.863975in}{2.855755in}}%
\pgfpathclose%
\pgfusepath{fill}%
\end{pgfscope}%
\begin{pgfscope}%
\pgfpathrectangle{\pgfqpoint{0.765000in}{0.660000in}}{\pgfqpoint{4.620000in}{4.620000in}}%
\pgfusepath{clip}%
\pgfsetbuttcap%
\pgfsetroundjoin%
\definecolor{currentfill}{rgb}{1.000000,0.894118,0.788235}%
\pgfsetfillcolor{currentfill}%
\pgfsetlinewidth{0.000000pt}%
\definecolor{currentstroke}{rgb}{1.000000,0.894118,0.788235}%
\pgfsetstrokecolor{currentstroke}%
\pgfsetdash{}{0pt}%
\pgfpathmoveto{\pgfqpoint{3.886454in}{2.896520in}}%
\pgfpathlineto{\pgfqpoint{3.859507in}{2.880962in}}%
\pgfpathlineto{\pgfqpoint{3.886454in}{2.865404in}}%
\pgfpathlineto{\pgfqpoint{3.913402in}{2.880962in}}%
\pgfpathlineto{\pgfqpoint{3.886454in}{2.896520in}}%
\pgfpathclose%
\pgfusepath{fill}%
\end{pgfscope}%
\begin{pgfscope}%
\pgfpathrectangle{\pgfqpoint{0.765000in}{0.660000in}}{\pgfqpoint{4.620000in}{4.620000in}}%
\pgfusepath{clip}%
\pgfsetbuttcap%
\pgfsetroundjoin%
\definecolor{currentfill}{rgb}{1.000000,0.894118,0.788235}%
\pgfsetfillcolor{currentfill}%
\pgfsetlinewidth{0.000000pt}%
\definecolor{currentstroke}{rgb}{1.000000,0.894118,0.788235}%
\pgfsetstrokecolor{currentstroke}%
\pgfsetdash{}{0pt}%
\pgfpathmoveto{\pgfqpoint{3.886454in}{2.834288in}}%
\pgfpathlineto{\pgfqpoint{3.913402in}{2.849846in}}%
\pgfpathlineto{\pgfqpoint{3.913402in}{2.880962in}}%
\pgfpathlineto{\pgfqpoint{3.886454in}{2.865404in}}%
\pgfpathlineto{\pgfqpoint{3.886454in}{2.834288in}}%
\pgfpathclose%
\pgfusepath{fill}%
\end{pgfscope}%
\begin{pgfscope}%
\pgfpathrectangle{\pgfqpoint{0.765000in}{0.660000in}}{\pgfqpoint{4.620000in}{4.620000in}}%
\pgfusepath{clip}%
\pgfsetbuttcap%
\pgfsetroundjoin%
\definecolor{currentfill}{rgb}{1.000000,0.894118,0.788235}%
\pgfsetfillcolor{currentfill}%
\pgfsetlinewidth{0.000000pt}%
\definecolor{currentstroke}{rgb}{1.000000,0.894118,0.788235}%
\pgfsetstrokecolor{currentstroke}%
\pgfsetdash{}{0pt}%
\pgfpathmoveto{\pgfqpoint{3.859507in}{2.849846in}}%
\pgfpathlineto{\pgfqpoint{3.886454in}{2.834288in}}%
\pgfpathlineto{\pgfqpoint{3.886454in}{2.865404in}}%
\pgfpathlineto{\pgfqpoint{3.859507in}{2.880962in}}%
\pgfpathlineto{\pgfqpoint{3.859507in}{2.849846in}}%
\pgfpathclose%
\pgfusepath{fill}%
\end{pgfscope}%
\begin{pgfscope}%
\pgfpathrectangle{\pgfqpoint{0.765000in}{0.660000in}}{\pgfqpoint{4.620000in}{4.620000in}}%
\pgfusepath{clip}%
\pgfsetbuttcap%
\pgfsetroundjoin%
\definecolor{currentfill}{rgb}{1.000000,0.894118,0.788235}%
\pgfsetfillcolor{currentfill}%
\pgfsetlinewidth{0.000000pt}%
\definecolor{currentstroke}{rgb}{1.000000,0.894118,0.788235}%
\pgfsetstrokecolor{currentstroke}%
\pgfsetdash{}{0pt}%
\pgfpathmoveto{\pgfqpoint{3.890923in}{2.871313in}}%
\pgfpathlineto{\pgfqpoint{3.863975in}{2.855755in}}%
\pgfpathlineto{\pgfqpoint{3.859507in}{2.849846in}}%
\pgfpathlineto{\pgfqpoint{3.886454in}{2.865404in}}%
\pgfpathlineto{\pgfqpoint{3.890923in}{2.871313in}}%
\pgfpathclose%
\pgfusepath{fill}%
\end{pgfscope}%
\begin{pgfscope}%
\pgfpathrectangle{\pgfqpoint{0.765000in}{0.660000in}}{\pgfqpoint{4.620000in}{4.620000in}}%
\pgfusepath{clip}%
\pgfsetbuttcap%
\pgfsetroundjoin%
\definecolor{currentfill}{rgb}{1.000000,0.894118,0.788235}%
\pgfsetfillcolor{currentfill}%
\pgfsetlinewidth{0.000000pt}%
\definecolor{currentstroke}{rgb}{1.000000,0.894118,0.788235}%
\pgfsetstrokecolor{currentstroke}%
\pgfsetdash{}{0pt}%
\pgfpathmoveto{\pgfqpoint{3.917870in}{2.855755in}}%
\pgfpathlineto{\pgfqpoint{3.890923in}{2.871313in}}%
\pgfpathlineto{\pgfqpoint{3.886454in}{2.865404in}}%
\pgfpathlineto{\pgfqpoint{3.913402in}{2.849846in}}%
\pgfpathlineto{\pgfqpoint{3.917870in}{2.855755in}}%
\pgfpathclose%
\pgfusepath{fill}%
\end{pgfscope}%
\begin{pgfscope}%
\pgfpathrectangle{\pgfqpoint{0.765000in}{0.660000in}}{\pgfqpoint{4.620000in}{4.620000in}}%
\pgfusepath{clip}%
\pgfsetbuttcap%
\pgfsetroundjoin%
\definecolor{currentfill}{rgb}{1.000000,0.894118,0.788235}%
\pgfsetfillcolor{currentfill}%
\pgfsetlinewidth{0.000000pt}%
\definecolor{currentstroke}{rgb}{1.000000,0.894118,0.788235}%
\pgfsetstrokecolor{currentstroke}%
\pgfsetdash{}{0pt}%
\pgfpathmoveto{\pgfqpoint{3.890923in}{2.871313in}}%
\pgfpathlineto{\pgfqpoint{3.890923in}{2.902429in}}%
\pgfpathlineto{\pgfqpoint{3.886454in}{2.896520in}}%
\pgfpathlineto{\pgfqpoint{3.913402in}{2.849846in}}%
\pgfpathlineto{\pgfqpoint{3.890923in}{2.871313in}}%
\pgfpathclose%
\pgfusepath{fill}%
\end{pgfscope}%
\begin{pgfscope}%
\pgfpathrectangle{\pgfqpoint{0.765000in}{0.660000in}}{\pgfqpoint{4.620000in}{4.620000in}}%
\pgfusepath{clip}%
\pgfsetbuttcap%
\pgfsetroundjoin%
\definecolor{currentfill}{rgb}{1.000000,0.894118,0.788235}%
\pgfsetfillcolor{currentfill}%
\pgfsetlinewidth{0.000000pt}%
\definecolor{currentstroke}{rgb}{1.000000,0.894118,0.788235}%
\pgfsetstrokecolor{currentstroke}%
\pgfsetdash{}{0pt}%
\pgfpathmoveto{\pgfqpoint{3.917870in}{2.855755in}}%
\pgfpathlineto{\pgfqpoint{3.917870in}{2.886871in}}%
\pgfpathlineto{\pgfqpoint{3.913402in}{2.880962in}}%
\pgfpathlineto{\pgfqpoint{3.886454in}{2.865404in}}%
\pgfpathlineto{\pgfqpoint{3.917870in}{2.855755in}}%
\pgfpathclose%
\pgfusepath{fill}%
\end{pgfscope}%
\begin{pgfscope}%
\pgfpathrectangle{\pgfqpoint{0.765000in}{0.660000in}}{\pgfqpoint{4.620000in}{4.620000in}}%
\pgfusepath{clip}%
\pgfsetbuttcap%
\pgfsetroundjoin%
\definecolor{currentfill}{rgb}{1.000000,0.894118,0.788235}%
\pgfsetfillcolor{currentfill}%
\pgfsetlinewidth{0.000000pt}%
\definecolor{currentstroke}{rgb}{1.000000,0.894118,0.788235}%
\pgfsetstrokecolor{currentstroke}%
\pgfsetdash{}{0pt}%
\pgfpathmoveto{\pgfqpoint{3.863975in}{2.855755in}}%
\pgfpathlineto{\pgfqpoint{3.890923in}{2.840197in}}%
\pgfpathlineto{\pgfqpoint{3.886454in}{2.834288in}}%
\pgfpathlineto{\pgfqpoint{3.859507in}{2.849846in}}%
\pgfpathlineto{\pgfqpoint{3.863975in}{2.855755in}}%
\pgfpathclose%
\pgfusepath{fill}%
\end{pgfscope}%
\begin{pgfscope}%
\pgfpathrectangle{\pgfqpoint{0.765000in}{0.660000in}}{\pgfqpoint{4.620000in}{4.620000in}}%
\pgfusepath{clip}%
\pgfsetbuttcap%
\pgfsetroundjoin%
\definecolor{currentfill}{rgb}{1.000000,0.894118,0.788235}%
\pgfsetfillcolor{currentfill}%
\pgfsetlinewidth{0.000000pt}%
\definecolor{currentstroke}{rgb}{1.000000,0.894118,0.788235}%
\pgfsetstrokecolor{currentstroke}%
\pgfsetdash{}{0pt}%
\pgfpathmoveto{\pgfqpoint{3.890923in}{2.840197in}}%
\pgfpathlineto{\pgfqpoint{3.917870in}{2.855755in}}%
\pgfpathlineto{\pgfqpoint{3.913402in}{2.849846in}}%
\pgfpathlineto{\pgfqpoint{3.886454in}{2.834288in}}%
\pgfpathlineto{\pgfqpoint{3.890923in}{2.840197in}}%
\pgfpathclose%
\pgfusepath{fill}%
\end{pgfscope}%
\begin{pgfscope}%
\pgfpathrectangle{\pgfqpoint{0.765000in}{0.660000in}}{\pgfqpoint{4.620000in}{4.620000in}}%
\pgfusepath{clip}%
\pgfsetbuttcap%
\pgfsetroundjoin%
\definecolor{currentfill}{rgb}{1.000000,0.894118,0.788235}%
\pgfsetfillcolor{currentfill}%
\pgfsetlinewidth{0.000000pt}%
\definecolor{currentstroke}{rgb}{1.000000,0.894118,0.788235}%
\pgfsetstrokecolor{currentstroke}%
\pgfsetdash{}{0pt}%
\pgfpathmoveto{\pgfqpoint{3.890923in}{2.902429in}}%
\pgfpathlineto{\pgfqpoint{3.863975in}{2.886871in}}%
\pgfpathlineto{\pgfqpoint{3.859507in}{2.880962in}}%
\pgfpathlineto{\pgfqpoint{3.886454in}{2.896520in}}%
\pgfpathlineto{\pgfqpoint{3.890923in}{2.902429in}}%
\pgfpathclose%
\pgfusepath{fill}%
\end{pgfscope}%
\begin{pgfscope}%
\pgfpathrectangle{\pgfqpoint{0.765000in}{0.660000in}}{\pgfqpoint{4.620000in}{4.620000in}}%
\pgfusepath{clip}%
\pgfsetbuttcap%
\pgfsetroundjoin%
\definecolor{currentfill}{rgb}{1.000000,0.894118,0.788235}%
\pgfsetfillcolor{currentfill}%
\pgfsetlinewidth{0.000000pt}%
\definecolor{currentstroke}{rgb}{1.000000,0.894118,0.788235}%
\pgfsetstrokecolor{currentstroke}%
\pgfsetdash{}{0pt}%
\pgfpathmoveto{\pgfqpoint{3.917870in}{2.886871in}}%
\pgfpathlineto{\pgfqpoint{3.890923in}{2.902429in}}%
\pgfpathlineto{\pgfqpoint{3.886454in}{2.896520in}}%
\pgfpathlineto{\pgfqpoint{3.913402in}{2.880962in}}%
\pgfpathlineto{\pgfqpoint{3.917870in}{2.886871in}}%
\pgfpathclose%
\pgfusepath{fill}%
\end{pgfscope}%
\begin{pgfscope}%
\pgfpathrectangle{\pgfqpoint{0.765000in}{0.660000in}}{\pgfqpoint{4.620000in}{4.620000in}}%
\pgfusepath{clip}%
\pgfsetbuttcap%
\pgfsetroundjoin%
\definecolor{currentfill}{rgb}{1.000000,0.894118,0.788235}%
\pgfsetfillcolor{currentfill}%
\pgfsetlinewidth{0.000000pt}%
\definecolor{currentstroke}{rgb}{1.000000,0.894118,0.788235}%
\pgfsetstrokecolor{currentstroke}%
\pgfsetdash{}{0pt}%
\pgfpathmoveto{\pgfqpoint{3.863975in}{2.855755in}}%
\pgfpathlineto{\pgfqpoint{3.863975in}{2.886871in}}%
\pgfpathlineto{\pgfqpoint{3.859507in}{2.880962in}}%
\pgfpathlineto{\pgfqpoint{3.886454in}{2.834288in}}%
\pgfpathlineto{\pgfqpoint{3.863975in}{2.855755in}}%
\pgfpathclose%
\pgfusepath{fill}%
\end{pgfscope}%
\begin{pgfscope}%
\pgfpathrectangle{\pgfqpoint{0.765000in}{0.660000in}}{\pgfqpoint{4.620000in}{4.620000in}}%
\pgfusepath{clip}%
\pgfsetbuttcap%
\pgfsetroundjoin%
\definecolor{currentfill}{rgb}{1.000000,0.894118,0.788235}%
\pgfsetfillcolor{currentfill}%
\pgfsetlinewidth{0.000000pt}%
\definecolor{currentstroke}{rgb}{1.000000,0.894118,0.788235}%
\pgfsetstrokecolor{currentstroke}%
\pgfsetdash{}{0pt}%
\pgfpathmoveto{\pgfqpoint{3.890923in}{2.840197in}}%
\pgfpathlineto{\pgfqpoint{3.890923in}{2.871313in}}%
\pgfpathlineto{\pgfqpoint{3.886454in}{2.865404in}}%
\pgfpathlineto{\pgfqpoint{3.859507in}{2.849846in}}%
\pgfpathlineto{\pgfqpoint{3.890923in}{2.840197in}}%
\pgfpathclose%
\pgfusepath{fill}%
\end{pgfscope}%
\begin{pgfscope}%
\pgfpathrectangle{\pgfqpoint{0.765000in}{0.660000in}}{\pgfqpoint{4.620000in}{4.620000in}}%
\pgfusepath{clip}%
\pgfsetbuttcap%
\pgfsetroundjoin%
\definecolor{currentfill}{rgb}{1.000000,0.894118,0.788235}%
\pgfsetfillcolor{currentfill}%
\pgfsetlinewidth{0.000000pt}%
\definecolor{currentstroke}{rgb}{1.000000,0.894118,0.788235}%
\pgfsetstrokecolor{currentstroke}%
\pgfsetdash{}{0pt}%
\pgfpathmoveto{\pgfqpoint{3.890923in}{2.871313in}}%
\pgfpathlineto{\pgfqpoint{3.917870in}{2.886871in}}%
\pgfpathlineto{\pgfqpoint{3.913402in}{2.880962in}}%
\pgfpathlineto{\pgfqpoint{3.886454in}{2.865404in}}%
\pgfpathlineto{\pgfqpoint{3.890923in}{2.871313in}}%
\pgfpathclose%
\pgfusepath{fill}%
\end{pgfscope}%
\begin{pgfscope}%
\pgfpathrectangle{\pgfqpoint{0.765000in}{0.660000in}}{\pgfqpoint{4.620000in}{4.620000in}}%
\pgfusepath{clip}%
\pgfsetbuttcap%
\pgfsetroundjoin%
\definecolor{currentfill}{rgb}{1.000000,0.894118,0.788235}%
\pgfsetfillcolor{currentfill}%
\pgfsetlinewidth{0.000000pt}%
\definecolor{currentstroke}{rgb}{1.000000,0.894118,0.788235}%
\pgfsetstrokecolor{currentstroke}%
\pgfsetdash{}{0pt}%
\pgfpathmoveto{\pgfqpoint{3.863975in}{2.886871in}}%
\pgfpathlineto{\pgfqpoint{3.890923in}{2.871313in}}%
\pgfpathlineto{\pgfqpoint{3.886454in}{2.865404in}}%
\pgfpathlineto{\pgfqpoint{3.859507in}{2.880962in}}%
\pgfpathlineto{\pgfqpoint{3.863975in}{2.886871in}}%
\pgfpathclose%
\pgfusepath{fill}%
\end{pgfscope}%
\begin{pgfscope}%
\pgfpathrectangle{\pgfqpoint{0.765000in}{0.660000in}}{\pgfqpoint{4.620000in}{4.620000in}}%
\pgfusepath{clip}%
\pgfsetbuttcap%
\pgfsetroundjoin%
\definecolor{currentfill}{rgb}{1.000000,0.894118,0.788235}%
\pgfsetfillcolor{currentfill}%
\pgfsetlinewidth{0.000000pt}%
\definecolor{currentstroke}{rgb}{1.000000,0.894118,0.788235}%
\pgfsetstrokecolor{currentstroke}%
\pgfsetdash{}{0pt}%
\pgfpathmoveto{\pgfqpoint{3.890923in}{2.871313in}}%
\pgfpathlineto{\pgfqpoint{3.863975in}{2.855755in}}%
\pgfpathlineto{\pgfqpoint{3.890923in}{2.840197in}}%
\pgfpathlineto{\pgfqpoint{3.917870in}{2.855755in}}%
\pgfpathlineto{\pgfqpoint{3.890923in}{2.871313in}}%
\pgfpathclose%
\pgfusepath{fill}%
\end{pgfscope}%
\begin{pgfscope}%
\pgfpathrectangle{\pgfqpoint{0.765000in}{0.660000in}}{\pgfqpoint{4.620000in}{4.620000in}}%
\pgfusepath{clip}%
\pgfsetbuttcap%
\pgfsetroundjoin%
\definecolor{currentfill}{rgb}{1.000000,0.894118,0.788235}%
\pgfsetfillcolor{currentfill}%
\pgfsetlinewidth{0.000000pt}%
\definecolor{currentstroke}{rgb}{1.000000,0.894118,0.788235}%
\pgfsetstrokecolor{currentstroke}%
\pgfsetdash{}{0pt}%
\pgfpathmoveto{\pgfqpoint{3.890923in}{2.871313in}}%
\pgfpathlineto{\pgfqpoint{3.863975in}{2.855755in}}%
\pgfpathlineto{\pgfqpoint{3.863975in}{2.886871in}}%
\pgfpathlineto{\pgfqpoint{3.890923in}{2.902429in}}%
\pgfpathlineto{\pgfqpoint{3.890923in}{2.871313in}}%
\pgfpathclose%
\pgfusepath{fill}%
\end{pgfscope}%
\begin{pgfscope}%
\pgfpathrectangle{\pgfqpoint{0.765000in}{0.660000in}}{\pgfqpoint{4.620000in}{4.620000in}}%
\pgfusepath{clip}%
\pgfsetbuttcap%
\pgfsetroundjoin%
\definecolor{currentfill}{rgb}{1.000000,0.894118,0.788235}%
\pgfsetfillcolor{currentfill}%
\pgfsetlinewidth{0.000000pt}%
\definecolor{currentstroke}{rgb}{1.000000,0.894118,0.788235}%
\pgfsetstrokecolor{currentstroke}%
\pgfsetdash{}{0pt}%
\pgfpathmoveto{\pgfqpoint{3.890923in}{2.871313in}}%
\pgfpathlineto{\pgfqpoint{3.917870in}{2.855755in}}%
\pgfpathlineto{\pgfqpoint{3.917870in}{2.886871in}}%
\pgfpathlineto{\pgfqpoint{3.890923in}{2.902429in}}%
\pgfpathlineto{\pgfqpoint{3.890923in}{2.871313in}}%
\pgfpathclose%
\pgfusepath{fill}%
\end{pgfscope}%
\begin{pgfscope}%
\pgfpathrectangle{\pgfqpoint{0.765000in}{0.660000in}}{\pgfqpoint{4.620000in}{4.620000in}}%
\pgfusepath{clip}%
\pgfsetbuttcap%
\pgfsetroundjoin%
\definecolor{currentfill}{rgb}{1.000000,0.894118,0.788235}%
\pgfsetfillcolor{currentfill}%
\pgfsetlinewidth{0.000000pt}%
\definecolor{currentstroke}{rgb}{1.000000,0.894118,0.788235}%
\pgfsetstrokecolor{currentstroke}%
\pgfsetdash{}{0pt}%
\pgfpathmoveto{\pgfqpoint{4.007374in}{3.052356in}}%
\pgfpathlineto{\pgfqpoint{3.980427in}{3.036798in}}%
\pgfpathlineto{\pgfqpoint{4.007374in}{3.021240in}}%
\pgfpathlineto{\pgfqpoint{4.034321in}{3.036798in}}%
\pgfpathlineto{\pgfqpoint{4.007374in}{3.052356in}}%
\pgfpathclose%
\pgfusepath{fill}%
\end{pgfscope}%
\begin{pgfscope}%
\pgfpathrectangle{\pgfqpoint{0.765000in}{0.660000in}}{\pgfqpoint{4.620000in}{4.620000in}}%
\pgfusepath{clip}%
\pgfsetbuttcap%
\pgfsetroundjoin%
\definecolor{currentfill}{rgb}{1.000000,0.894118,0.788235}%
\pgfsetfillcolor{currentfill}%
\pgfsetlinewidth{0.000000pt}%
\definecolor{currentstroke}{rgb}{1.000000,0.894118,0.788235}%
\pgfsetstrokecolor{currentstroke}%
\pgfsetdash{}{0pt}%
\pgfpathmoveto{\pgfqpoint{4.007374in}{3.052356in}}%
\pgfpathlineto{\pgfqpoint{3.980427in}{3.036798in}}%
\pgfpathlineto{\pgfqpoint{3.980427in}{3.067914in}}%
\pgfpathlineto{\pgfqpoint{4.007374in}{3.083472in}}%
\pgfpathlineto{\pgfqpoint{4.007374in}{3.052356in}}%
\pgfpathclose%
\pgfusepath{fill}%
\end{pgfscope}%
\begin{pgfscope}%
\pgfpathrectangle{\pgfqpoint{0.765000in}{0.660000in}}{\pgfqpoint{4.620000in}{4.620000in}}%
\pgfusepath{clip}%
\pgfsetbuttcap%
\pgfsetroundjoin%
\definecolor{currentfill}{rgb}{1.000000,0.894118,0.788235}%
\pgfsetfillcolor{currentfill}%
\pgfsetlinewidth{0.000000pt}%
\definecolor{currentstroke}{rgb}{1.000000,0.894118,0.788235}%
\pgfsetstrokecolor{currentstroke}%
\pgfsetdash{}{0pt}%
\pgfpathmoveto{\pgfqpoint{4.007374in}{3.052356in}}%
\pgfpathlineto{\pgfqpoint{4.034321in}{3.036798in}}%
\pgfpathlineto{\pgfqpoint{4.034321in}{3.067914in}}%
\pgfpathlineto{\pgfqpoint{4.007374in}{3.083472in}}%
\pgfpathlineto{\pgfqpoint{4.007374in}{3.052356in}}%
\pgfpathclose%
\pgfusepath{fill}%
\end{pgfscope}%
\begin{pgfscope}%
\pgfpathrectangle{\pgfqpoint{0.765000in}{0.660000in}}{\pgfqpoint{4.620000in}{4.620000in}}%
\pgfusepath{clip}%
\pgfsetbuttcap%
\pgfsetroundjoin%
\definecolor{currentfill}{rgb}{1.000000,0.894118,0.788235}%
\pgfsetfillcolor{currentfill}%
\pgfsetlinewidth{0.000000pt}%
\definecolor{currentstroke}{rgb}{1.000000,0.894118,0.788235}%
\pgfsetstrokecolor{currentstroke}%
\pgfsetdash{}{0pt}%
\pgfpathmoveto{\pgfqpoint{3.890923in}{2.902429in}}%
\pgfpathlineto{\pgfqpoint{3.863975in}{2.886871in}}%
\pgfpathlineto{\pgfqpoint{3.890923in}{2.871313in}}%
\pgfpathlineto{\pgfqpoint{3.917870in}{2.886871in}}%
\pgfpathlineto{\pgfqpoint{3.890923in}{2.902429in}}%
\pgfpathclose%
\pgfusepath{fill}%
\end{pgfscope}%
\begin{pgfscope}%
\pgfpathrectangle{\pgfqpoint{0.765000in}{0.660000in}}{\pgfqpoint{4.620000in}{4.620000in}}%
\pgfusepath{clip}%
\pgfsetbuttcap%
\pgfsetroundjoin%
\definecolor{currentfill}{rgb}{1.000000,0.894118,0.788235}%
\pgfsetfillcolor{currentfill}%
\pgfsetlinewidth{0.000000pt}%
\definecolor{currentstroke}{rgb}{1.000000,0.894118,0.788235}%
\pgfsetstrokecolor{currentstroke}%
\pgfsetdash{}{0pt}%
\pgfpathmoveto{\pgfqpoint{3.890923in}{2.840197in}}%
\pgfpathlineto{\pgfqpoint{3.917870in}{2.855755in}}%
\pgfpathlineto{\pgfqpoint{3.917870in}{2.886871in}}%
\pgfpathlineto{\pgfqpoint{3.890923in}{2.871313in}}%
\pgfpathlineto{\pgfqpoint{3.890923in}{2.840197in}}%
\pgfpathclose%
\pgfusepath{fill}%
\end{pgfscope}%
\begin{pgfscope}%
\pgfpathrectangle{\pgfqpoint{0.765000in}{0.660000in}}{\pgfqpoint{4.620000in}{4.620000in}}%
\pgfusepath{clip}%
\pgfsetbuttcap%
\pgfsetroundjoin%
\definecolor{currentfill}{rgb}{1.000000,0.894118,0.788235}%
\pgfsetfillcolor{currentfill}%
\pgfsetlinewidth{0.000000pt}%
\definecolor{currentstroke}{rgb}{1.000000,0.894118,0.788235}%
\pgfsetstrokecolor{currentstroke}%
\pgfsetdash{}{0pt}%
\pgfpathmoveto{\pgfqpoint{3.863975in}{2.855755in}}%
\pgfpathlineto{\pgfqpoint{3.890923in}{2.840197in}}%
\pgfpathlineto{\pgfqpoint{3.890923in}{2.871313in}}%
\pgfpathlineto{\pgfqpoint{3.863975in}{2.886871in}}%
\pgfpathlineto{\pgfqpoint{3.863975in}{2.855755in}}%
\pgfpathclose%
\pgfusepath{fill}%
\end{pgfscope}%
\begin{pgfscope}%
\pgfpathrectangle{\pgfqpoint{0.765000in}{0.660000in}}{\pgfqpoint{4.620000in}{4.620000in}}%
\pgfusepath{clip}%
\pgfsetbuttcap%
\pgfsetroundjoin%
\definecolor{currentfill}{rgb}{1.000000,0.894118,0.788235}%
\pgfsetfillcolor{currentfill}%
\pgfsetlinewidth{0.000000pt}%
\definecolor{currentstroke}{rgb}{1.000000,0.894118,0.788235}%
\pgfsetstrokecolor{currentstroke}%
\pgfsetdash{}{0pt}%
\pgfpathmoveto{\pgfqpoint{4.007374in}{3.083472in}}%
\pgfpathlineto{\pgfqpoint{3.980427in}{3.067914in}}%
\pgfpathlineto{\pgfqpoint{4.007374in}{3.052356in}}%
\pgfpathlineto{\pgfqpoint{4.034321in}{3.067914in}}%
\pgfpathlineto{\pgfqpoint{4.007374in}{3.083472in}}%
\pgfpathclose%
\pgfusepath{fill}%
\end{pgfscope}%
\begin{pgfscope}%
\pgfpathrectangle{\pgfqpoint{0.765000in}{0.660000in}}{\pgfqpoint{4.620000in}{4.620000in}}%
\pgfusepath{clip}%
\pgfsetbuttcap%
\pgfsetroundjoin%
\definecolor{currentfill}{rgb}{1.000000,0.894118,0.788235}%
\pgfsetfillcolor{currentfill}%
\pgfsetlinewidth{0.000000pt}%
\definecolor{currentstroke}{rgb}{1.000000,0.894118,0.788235}%
\pgfsetstrokecolor{currentstroke}%
\pgfsetdash{}{0pt}%
\pgfpathmoveto{\pgfqpoint{4.007374in}{3.021240in}}%
\pgfpathlineto{\pgfqpoint{4.034321in}{3.036798in}}%
\pgfpathlineto{\pgfqpoint{4.034321in}{3.067914in}}%
\pgfpathlineto{\pgfqpoint{4.007374in}{3.052356in}}%
\pgfpathlineto{\pgfqpoint{4.007374in}{3.021240in}}%
\pgfpathclose%
\pgfusepath{fill}%
\end{pgfscope}%
\begin{pgfscope}%
\pgfpathrectangle{\pgfqpoint{0.765000in}{0.660000in}}{\pgfqpoint{4.620000in}{4.620000in}}%
\pgfusepath{clip}%
\pgfsetbuttcap%
\pgfsetroundjoin%
\definecolor{currentfill}{rgb}{1.000000,0.894118,0.788235}%
\pgfsetfillcolor{currentfill}%
\pgfsetlinewidth{0.000000pt}%
\definecolor{currentstroke}{rgb}{1.000000,0.894118,0.788235}%
\pgfsetstrokecolor{currentstroke}%
\pgfsetdash{}{0pt}%
\pgfpathmoveto{\pgfqpoint{3.980427in}{3.036798in}}%
\pgfpathlineto{\pgfqpoint{4.007374in}{3.021240in}}%
\pgfpathlineto{\pgfqpoint{4.007374in}{3.052356in}}%
\pgfpathlineto{\pgfqpoint{3.980427in}{3.067914in}}%
\pgfpathlineto{\pgfqpoint{3.980427in}{3.036798in}}%
\pgfpathclose%
\pgfusepath{fill}%
\end{pgfscope}%
\begin{pgfscope}%
\pgfpathrectangle{\pgfqpoint{0.765000in}{0.660000in}}{\pgfqpoint{4.620000in}{4.620000in}}%
\pgfusepath{clip}%
\pgfsetbuttcap%
\pgfsetroundjoin%
\definecolor{currentfill}{rgb}{1.000000,0.894118,0.788235}%
\pgfsetfillcolor{currentfill}%
\pgfsetlinewidth{0.000000pt}%
\definecolor{currentstroke}{rgb}{1.000000,0.894118,0.788235}%
\pgfsetstrokecolor{currentstroke}%
\pgfsetdash{}{0pt}%
\pgfpathmoveto{\pgfqpoint{3.890923in}{2.871313in}}%
\pgfpathlineto{\pgfqpoint{3.863975in}{2.855755in}}%
\pgfpathlineto{\pgfqpoint{3.980427in}{3.036798in}}%
\pgfpathlineto{\pgfqpoint{4.007374in}{3.052356in}}%
\pgfpathlineto{\pgfqpoint{3.890923in}{2.871313in}}%
\pgfpathclose%
\pgfusepath{fill}%
\end{pgfscope}%
\begin{pgfscope}%
\pgfpathrectangle{\pgfqpoint{0.765000in}{0.660000in}}{\pgfqpoint{4.620000in}{4.620000in}}%
\pgfusepath{clip}%
\pgfsetbuttcap%
\pgfsetroundjoin%
\definecolor{currentfill}{rgb}{1.000000,0.894118,0.788235}%
\pgfsetfillcolor{currentfill}%
\pgfsetlinewidth{0.000000pt}%
\definecolor{currentstroke}{rgb}{1.000000,0.894118,0.788235}%
\pgfsetstrokecolor{currentstroke}%
\pgfsetdash{}{0pt}%
\pgfpathmoveto{\pgfqpoint{3.917870in}{2.855755in}}%
\pgfpathlineto{\pgfqpoint{3.890923in}{2.871313in}}%
\pgfpathlineto{\pgfqpoint{4.007374in}{3.052356in}}%
\pgfpathlineto{\pgfqpoint{4.034321in}{3.036798in}}%
\pgfpathlineto{\pgfqpoint{3.917870in}{2.855755in}}%
\pgfpathclose%
\pgfusepath{fill}%
\end{pgfscope}%
\begin{pgfscope}%
\pgfpathrectangle{\pgfqpoint{0.765000in}{0.660000in}}{\pgfqpoint{4.620000in}{4.620000in}}%
\pgfusepath{clip}%
\pgfsetbuttcap%
\pgfsetroundjoin%
\definecolor{currentfill}{rgb}{1.000000,0.894118,0.788235}%
\pgfsetfillcolor{currentfill}%
\pgfsetlinewidth{0.000000pt}%
\definecolor{currentstroke}{rgb}{1.000000,0.894118,0.788235}%
\pgfsetstrokecolor{currentstroke}%
\pgfsetdash{}{0pt}%
\pgfpathmoveto{\pgfqpoint{3.890923in}{2.871313in}}%
\pgfpathlineto{\pgfqpoint{3.890923in}{2.902429in}}%
\pgfpathlineto{\pgfqpoint{4.007374in}{3.083472in}}%
\pgfpathlineto{\pgfqpoint{4.034321in}{3.036798in}}%
\pgfpathlineto{\pgfqpoint{3.890923in}{2.871313in}}%
\pgfpathclose%
\pgfusepath{fill}%
\end{pgfscope}%
\begin{pgfscope}%
\pgfpathrectangle{\pgfqpoint{0.765000in}{0.660000in}}{\pgfqpoint{4.620000in}{4.620000in}}%
\pgfusepath{clip}%
\pgfsetbuttcap%
\pgfsetroundjoin%
\definecolor{currentfill}{rgb}{1.000000,0.894118,0.788235}%
\pgfsetfillcolor{currentfill}%
\pgfsetlinewidth{0.000000pt}%
\definecolor{currentstroke}{rgb}{1.000000,0.894118,0.788235}%
\pgfsetstrokecolor{currentstroke}%
\pgfsetdash{}{0pt}%
\pgfpathmoveto{\pgfqpoint{3.917870in}{2.855755in}}%
\pgfpathlineto{\pgfqpoint{3.917870in}{2.886871in}}%
\pgfpathlineto{\pgfqpoint{4.034321in}{3.067914in}}%
\pgfpathlineto{\pgfqpoint{4.007374in}{3.052356in}}%
\pgfpathlineto{\pgfqpoint{3.917870in}{2.855755in}}%
\pgfpathclose%
\pgfusepath{fill}%
\end{pgfscope}%
\begin{pgfscope}%
\pgfpathrectangle{\pgfqpoint{0.765000in}{0.660000in}}{\pgfqpoint{4.620000in}{4.620000in}}%
\pgfusepath{clip}%
\pgfsetbuttcap%
\pgfsetroundjoin%
\definecolor{currentfill}{rgb}{1.000000,0.894118,0.788235}%
\pgfsetfillcolor{currentfill}%
\pgfsetlinewidth{0.000000pt}%
\definecolor{currentstroke}{rgb}{1.000000,0.894118,0.788235}%
\pgfsetstrokecolor{currentstroke}%
\pgfsetdash{}{0pt}%
\pgfpathmoveto{\pgfqpoint{3.863975in}{2.855755in}}%
\pgfpathlineto{\pgfqpoint{3.890923in}{2.840197in}}%
\pgfpathlineto{\pgfqpoint{4.007374in}{3.021240in}}%
\pgfpathlineto{\pgfqpoint{3.980427in}{3.036798in}}%
\pgfpathlineto{\pgfqpoint{3.863975in}{2.855755in}}%
\pgfpathclose%
\pgfusepath{fill}%
\end{pgfscope}%
\begin{pgfscope}%
\pgfpathrectangle{\pgfqpoint{0.765000in}{0.660000in}}{\pgfqpoint{4.620000in}{4.620000in}}%
\pgfusepath{clip}%
\pgfsetbuttcap%
\pgfsetroundjoin%
\definecolor{currentfill}{rgb}{1.000000,0.894118,0.788235}%
\pgfsetfillcolor{currentfill}%
\pgfsetlinewidth{0.000000pt}%
\definecolor{currentstroke}{rgb}{1.000000,0.894118,0.788235}%
\pgfsetstrokecolor{currentstroke}%
\pgfsetdash{}{0pt}%
\pgfpathmoveto{\pgfqpoint{3.890923in}{2.840197in}}%
\pgfpathlineto{\pgfqpoint{3.917870in}{2.855755in}}%
\pgfpathlineto{\pgfqpoint{4.034321in}{3.036798in}}%
\pgfpathlineto{\pgfqpoint{4.007374in}{3.021240in}}%
\pgfpathlineto{\pgfqpoint{3.890923in}{2.840197in}}%
\pgfpathclose%
\pgfusepath{fill}%
\end{pgfscope}%
\begin{pgfscope}%
\pgfpathrectangle{\pgfqpoint{0.765000in}{0.660000in}}{\pgfqpoint{4.620000in}{4.620000in}}%
\pgfusepath{clip}%
\pgfsetbuttcap%
\pgfsetroundjoin%
\definecolor{currentfill}{rgb}{1.000000,0.894118,0.788235}%
\pgfsetfillcolor{currentfill}%
\pgfsetlinewidth{0.000000pt}%
\definecolor{currentstroke}{rgb}{1.000000,0.894118,0.788235}%
\pgfsetstrokecolor{currentstroke}%
\pgfsetdash{}{0pt}%
\pgfpathmoveto{\pgfqpoint{3.890923in}{2.902429in}}%
\pgfpathlineto{\pgfqpoint{3.863975in}{2.886871in}}%
\pgfpathlineto{\pgfqpoint{3.980427in}{3.067914in}}%
\pgfpathlineto{\pgfqpoint{4.007374in}{3.083472in}}%
\pgfpathlineto{\pgfqpoint{3.890923in}{2.902429in}}%
\pgfpathclose%
\pgfusepath{fill}%
\end{pgfscope}%
\begin{pgfscope}%
\pgfpathrectangle{\pgfqpoint{0.765000in}{0.660000in}}{\pgfqpoint{4.620000in}{4.620000in}}%
\pgfusepath{clip}%
\pgfsetbuttcap%
\pgfsetroundjoin%
\definecolor{currentfill}{rgb}{1.000000,0.894118,0.788235}%
\pgfsetfillcolor{currentfill}%
\pgfsetlinewidth{0.000000pt}%
\definecolor{currentstroke}{rgb}{1.000000,0.894118,0.788235}%
\pgfsetstrokecolor{currentstroke}%
\pgfsetdash{}{0pt}%
\pgfpathmoveto{\pgfqpoint{3.917870in}{2.886871in}}%
\pgfpathlineto{\pgfqpoint{3.890923in}{2.902429in}}%
\pgfpathlineto{\pgfqpoint{4.007374in}{3.083472in}}%
\pgfpathlineto{\pgfqpoint{4.034321in}{3.067914in}}%
\pgfpathlineto{\pgfqpoint{3.917870in}{2.886871in}}%
\pgfpathclose%
\pgfusepath{fill}%
\end{pgfscope}%
\begin{pgfscope}%
\pgfpathrectangle{\pgfqpoint{0.765000in}{0.660000in}}{\pgfqpoint{4.620000in}{4.620000in}}%
\pgfusepath{clip}%
\pgfsetbuttcap%
\pgfsetroundjoin%
\definecolor{currentfill}{rgb}{1.000000,0.894118,0.788235}%
\pgfsetfillcolor{currentfill}%
\pgfsetlinewidth{0.000000pt}%
\definecolor{currentstroke}{rgb}{1.000000,0.894118,0.788235}%
\pgfsetstrokecolor{currentstroke}%
\pgfsetdash{}{0pt}%
\pgfpathmoveto{\pgfqpoint{3.863975in}{2.855755in}}%
\pgfpathlineto{\pgfqpoint{3.863975in}{2.886871in}}%
\pgfpathlineto{\pgfqpoint{3.980427in}{3.067914in}}%
\pgfpathlineto{\pgfqpoint{4.007374in}{3.021240in}}%
\pgfpathlineto{\pgfqpoint{3.863975in}{2.855755in}}%
\pgfpathclose%
\pgfusepath{fill}%
\end{pgfscope}%
\begin{pgfscope}%
\pgfpathrectangle{\pgfqpoint{0.765000in}{0.660000in}}{\pgfqpoint{4.620000in}{4.620000in}}%
\pgfusepath{clip}%
\pgfsetbuttcap%
\pgfsetroundjoin%
\definecolor{currentfill}{rgb}{1.000000,0.894118,0.788235}%
\pgfsetfillcolor{currentfill}%
\pgfsetlinewidth{0.000000pt}%
\definecolor{currentstroke}{rgb}{1.000000,0.894118,0.788235}%
\pgfsetstrokecolor{currentstroke}%
\pgfsetdash{}{0pt}%
\pgfpathmoveto{\pgfqpoint{3.890923in}{2.840197in}}%
\pgfpathlineto{\pgfqpoint{3.890923in}{2.871313in}}%
\pgfpathlineto{\pgfqpoint{4.007374in}{3.052356in}}%
\pgfpathlineto{\pgfqpoint{3.980427in}{3.036798in}}%
\pgfpathlineto{\pgfqpoint{3.890923in}{2.840197in}}%
\pgfpathclose%
\pgfusepath{fill}%
\end{pgfscope}%
\begin{pgfscope}%
\pgfpathrectangle{\pgfqpoint{0.765000in}{0.660000in}}{\pgfqpoint{4.620000in}{4.620000in}}%
\pgfusepath{clip}%
\pgfsetbuttcap%
\pgfsetroundjoin%
\definecolor{currentfill}{rgb}{1.000000,0.894118,0.788235}%
\pgfsetfillcolor{currentfill}%
\pgfsetlinewidth{0.000000pt}%
\definecolor{currentstroke}{rgb}{1.000000,0.894118,0.788235}%
\pgfsetstrokecolor{currentstroke}%
\pgfsetdash{}{0pt}%
\pgfpathmoveto{\pgfqpoint{3.890923in}{2.871313in}}%
\pgfpathlineto{\pgfqpoint{3.917870in}{2.886871in}}%
\pgfpathlineto{\pgfqpoint{4.034321in}{3.067914in}}%
\pgfpathlineto{\pgfqpoint{4.007374in}{3.052356in}}%
\pgfpathlineto{\pgfqpoint{3.890923in}{2.871313in}}%
\pgfpathclose%
\pgfusepath{fill}%
\end{pgfscope}%
\begin{pgfscope}%
\pgfpathrectangle{\pgfqpoint{0.765000in}{0.660000in}}{\pgfqpoint{4.620000in}{4.620000in}}%
\pgfusepath{clip}%
\pgfsetbuttcap%
\pgfsetroundjoin%
\definecolor{currentfill}{rgb}{1.000000,0.894118,0.788235}%
\pgfsetfillcolor{currentfill}%
\pgfsetlinewidth{0.000000pt}%
\definecolor{currentstroke}{rgb}{1.000000,0.894118,0.788235}%
\pgfsetstrokecolor{currentstroke}%
\pgfsetdash{}{0pt}%
\pgfpathmoveto{\pgfqpoint{3.863975in}{2.886871in}}%
\pgfpathlineto{\pgfqpoint{3.890923in}{2.871313in}}%
\pgfpathlineto{\pgfqpoint{4.007374in}{3.052356in}}%
\pgfpathlineto{\pgfqpoint{3.980427in}{3.067914in}}%
\pgfpathlineto{\pgfqpoint{3.863975in}{2.886871in}}%
\pgfpathclose%
\pgfusepath{fill}%
\end{pgfscope}%
\begin{pgfscope}%
\pgfpathrectangle{\pgfqpoint{0.765000in}{0.660000in}}{\pgfqpoint{4.620000in}{4.620000in}}%
\pgfusepath{clip}%
\pgfsetbuttcap%
\pgfsetroundjoin%
\definecolor{currentfill}{rgb}{1.000000,0.894118,0.788235}%
\pgfsetfillcolor{currentfill}%
\pgfsetlinewidth{0.000000pt}%
\definecolor{currentstroke}{rgb}{1.000000,0.894118,0.788235}%
\pgfsetstrokecolor{currentstroke}%
\pgfsetdash{}{0pt}%
\pgfpathmoveto{\pgfqpoint{3.890923in}{2.871313in}}%
\pgfpathlineto{\pgfqpoint{3.863975in}{2.855755in}}%
\pgfpathlineto{\pgfqpoint{3.890923in}{2.840197in}}%
\pgfpathlineto{\pgfqpoint{3.917870in}{2.855755in}}%
\pgfpathlineto{\pgfqpoint{3.890923in}{2.871313in}}%
\pgfpathclose%
\pgfusepath{fill}%
\end{pgfscope}%
\begin{pgfscope}%
\pgfpathrectangle{\pgfqpoint{0.765000in}{0.660000in}}{\pgfqpoint{4.620000in}{4.620000in}}%
\pgfusepath{clip}%
\pgfsetbuttcap%
\pgfsetroundjoin%
\definecolor{currentfill}{rgb}{1.000000,0.894118,0.788235}%
\pgfsetfillcolor{currentfill}%
\pgfsetlinewidth{0.000000pt}%
\definecolor{currentstroke}{rgb}{1.000000,0.894118,0.788235}%
\pgfsetstrokecolor{currentstroke}%
\pgfsetdash{}{0pt}%
\pgfpathmoveto{\pgfqpoint{3.890923in}{2.871313in}}%
\pgfpathlineto{\pgfqpoint{3.863975in}{2.855755in}}%
\pgfpathlineto{\pgfqpoint{3.863975in}{2.886871in}}%
\pgfpathlineto{\pgfqpoint{3.890923in}{2.902429in}}%
\pgfpathlineto{\pgfqpoint{3.890923in}{2.871313in}}%
\pgfpathclose%
\pgfusepath{fill}%
\end{pgfscope}%
\begin{pgfscope}%
\pgfpathrectangle{\pgfqpoint{0.765000in}{0.660000in}}{\pgfqpoint{4.620000in}{4.620000in}}%
\pgfusepath{clip}%
\pgfsetbuttcap%
\pgfsetroundjoin%
\definecolor{currentfill}{rgb}{1.000000,0.894118,0.788235}%
\pgfsetfillcolor{currentfill}%
\pgfsetlinewidth{0.000000pt}%
\definecolor{currentstroke}{rgb}{1.000000,0.894118,0.788235}%
\pgfsetstrokecolor{currentstroke}%
\pgfsetdash{}{0pt}%
\pgfpathmoveto{\pgfqpoint{3.890923in}{2.871313in}}%
\pgfpathlineto{\pgfqpoint{3.917870in}{2.855755in}}%
\pgfpathlineto{\pgfqpoint{3.917870in}{2.886871in}}%
\pgfpathlineto{\pgfqpoint{3.890923in}{2.902429in}}%
\pgfpathlineto{\pgfqpoint{3.890923in}{2.871313in}}%
\pgfpathclose%
\pgfusepath{fill}%
\end{pgfscope}%
\begin{pgfscope}%
\pgfpathrectangle{\pgfqpoint{0.765000in}{0.660000in}}{\pgfqpoint{4.620000in}{4.620000in}}%
\pgfusepath{clip}%
\pgfsetbuttcap%
\pgfsetroundjoin%
\definecolor{currentfill}{rgb}{1.000000,0.894118,0.788235}%
\pgfsetfillcolor{currentfill}%
\pgfsetlinewidth{0.000000pt}%
\definecolor{currentstroke}{rgb}{1.000000,0.894118,0.788235}%
\pgfsetstrokecolor{currentstroke}%
\pgfsetdash{}{0pt}%
\pgfpathmoveto{\pgfqpoint{4.007207in}{3.052100in}}%
\pgfpathlineto{\pgfqpoint{3.980260in}{3.036542in}}%
\pgfpathlineto{\pgfqpoint{4.007207in}{3.020984in}}%
\pgfpathlineto{\pgfqpoint{4.034154in}{3.036542in}}%
\pgfpathlineto{\pgfqpoint{4.007207in}{3.052100in}}%
\pgfpathclose%
\pgfusepath{fill}%
\end{pgfscope}%
\begin{pgfscope}%
\pgfpathrectangle{\pgfqpoint{0.765000in}{0.660000in}}{\pgfqpoint{4.620000in}{4.620000in}}%
\pgfusepath{clip}%
\pgfsetbuttcap%
\pgfsetroundjoin%
\definecolor{currentfill}{rgb}{1.000000,0.894118,0.788235}%
\pgfsetfillcolor{currentfill}%
\pgfsetlinewidth{0.000000pt}%
\definecolor{currentstroke}{rgb}{1.000000,0.894118,0.788235}%
\pgfsetstrokecolor{currentstroke}%
\pgfsetdash{}{0pt}%
\pgfpathmoveto{\pgfqpoint{4.007207in}{3.052100in}}%
\pgfpathlineto{\pgfqpoint{3.980260in}{3.036542in}}%
\pgfpathlineto{\pgfqpoint{3.980260in}{3.067658in}}%
\pgfpathlineto{\pgfqpoint{4.007207in}{3.083216in}}%
\pgfpathlineto{\pgfqpoint{4.007207in}{3.052100in}}%
\pgfpathclose%
\pgfusepath{fill}%
\end{pgfscope}%
\begin{pgfscope}%
\pgfpathrectangle{\pgfqpoint{0.765000in}{0.660000in}}{\pgfqpoint{4.620000in}{4.620000in}}%
\pgfusepath{clip}%
\pgfsetbuttcap%
\pgfsetroundjoin%
\definecolor{currentfill}{rgb}{1.000000,0.894118,0.788235}%
\pgfsetfillcolor{currentfill}%
\pgfsetlinewidth{0.000000pt}%
\definecolor{currentstroke}{rgb}{1.000000,0.894118,0.788235}%
\pgfsetstrokecolor{currentstroke}%
\pgfsetdash{}{0pt}%
\pgfpathmoveto{\pgfqpoint{4.007207in}{3.052100in}}%
\pgfpathlineto{\pgfqpoint{4.034154in}{3.036542in}}%
\pgfpathlineto{\pgfqpoint{4.034154in}{3.067658in}}%
\pgfpathlineto{\pgfqpoint{4.007207in}{3.083216in}}%
\pgfpathlineto{\pgfqpoint{4.007207in}{3.052100in}}%
\pgfpathclose%
\pgfusepath{fill}%
\end{pgfscope}%
\begin{pgfscope}%
\pgfpathrectangle{\pgfqpoint{0.765000in}{0.660000in}}{\pgfqpoint{4.620000in}{4.620000in}}%
\pgfusepath{clip}%
\pgfsetbuttcap%
\pgfsetroundjoin%
\definecolor{currentfill}{rgb}{1.000000,0.894118,0.788235}%
\pgfsetfillcolor{currentfill}%
\pgfsetlinewidth{0.000000pt}%
\definecolor{currentstroke}{rgb}{1.000000,0.894118,0.788235}%
\pgfsetstrokecolor{currentstroke}%
\pgfsetdash{}{0pt}%
\pgfpathmoveto{\pgfqpoint{3.890923in}{2.902429in}}%
\pgfpathlineto{\pgfqpoint{3.863975in}{2.886871in}}%
\pgfpathlineto{\pgfqpoint{3.890923in}{2.871313in}}%
\pgfpathlineto{\pgfqpoint{3.917870in}{2.886871in}}%
\pgfpathlineto{\pgfqpoint{3.890923in}{2.902429in}}%
\pgfpathclose%
\pgfusepath{fill}%
\end{pgfscope}%
\begin{pgfscope}%
\pgfpathrectangle{\pgfqpoint{0.765000in}{0.660000in}}{\pgfqpoint{4.620000in}{4.620000in}}%
\pgfusepath{clip}%
\pgfsetbuttcap%
\pgfsetroundjoin%
\definecolor{currentfill}{rgb}{1.000000,0.894118,0.788235}%
\pgfsetfillcolor{currentfill}%
\pgfsetlinewidth{0.000000pt}%
\definecolor{currentstroke}{rgb}{1.000000,0.894118,0.788235}%
\pgfsetstrokecolor{currentstroke}%
\pgfsetdash{}{0pt}%
\pgfpathmoveto{\pgfqpoint{3.890923in}{2.840197in}}%
\pgfpathlineto{\pgfqpoint{3.917870in}{2.855755in}}%
\pgfpathlineto{\pgfqpoint{3.917870in}{2.886871in}}%
\pgfpathlineto{\pgfqpoint{3.890923in}{2.871313in}}%
\pgfpathlineto{\pgfqpoint{3.890923in}{2.840197in}}%
\pgfpathclose%
\pgfusepath{fill}%
\end{pgfscope}%
\begin{pgfscope}%
\pgfpathrectangle{\pgfqpoint{0.765000in}{0.660000in}}{\pgfqpoint{4.620000in}{4.620000in}}%
\pgfusepath{clip}%
\pgfsetbuttcap%
\pgfsetroundjoin%
\definecolor{currentfill}{rgb}{1.000000,0.894118,0.788235}%
\pgfsetfillcolor{currentfill}%
\pgfsetlinewidth{0.000000pt}%
\definecolor{currentstroke}{rgb}{1.000000,0.894118,0.788235}%
\pgfsetstrokecolor{currentstroke}%
\pgfsetdash{}{0pt}%
\pgfpathmoveto{\pgfqpoint{3.863975in}{2.855755in}}%
\pgfpathlineto{\pgfqpoint{3.890923in}{2.840197in}}%
\pgfpathlineto{\pgfqpoint{3.890923in}{2.871313in}}%
\pgfpathlineto{\pgfqpoint{3.863975in}{2.886871in}}%
\pgfpathlineto{\pgfqpoint{3.863975in}{2.855755in}}%
\pgfpathclose%
\pgfusepath{fill}%
\end{pgfscope}%
\begin{pgfscope}%
\pgfpathrectangle{\pgfqpoint{0.765000in}{0.660000in}}{\pgfqpoint{4.620000in}{4.620000in}}%
\pgfusepath{clip}%
\pgfsetbuttcap%
\pgfsetroundjoin%
\definecolor{currentfill}{rgb}{1.000000,0.894118,0.788235}%
\pgfsetfillcolor{currentfill}%
\pgfsetlinewidth{0.000000pt}%
\definecolor{currentstroke}{rgb}{1.000000,0.894118,0.788235}%
\pgfsetstrokecolor{currentstroke}%
\pgfsetdash{}{0pt}%
\pgfpathmoveto{\pgfqpoint{4.007207in}{3.083216in}}%
\pgfpathlineto{\pgfqpoint{3.980260in}{3.067658in}}%
\pgfpathlineto{\pgfqpoint{4.007207in}{3.052100in}}%
\pgfpathlineto{\pgfqpoint{4.034154in}{3.067658in}}%
\pgfpathlineto{\pgfqpoint{4.007207in}{3.083216in}}%
\pgfpathclose%
\pgfusepath{fill}%
\end{pgfscope}%
\begin{pgfscope}%
\pgfpathrectangle{\pgfqpoint{0.765000in}{0.660000in}}{\pgfqpoint{4.620000in}{4.620000in}}%
\pgfusepath{clip}%
\pgfsetbuttcap%
\pgfsetroundjoin%
\definecolor{currentfill}{rgb}{1.000000,0.894118,0.788235}%
\pgfsetfillcolor{currentfill}%
\pgfsetlinewidth{0.000000pt}%
\definecolor{currentstroke}{rgb}{1.000000,0.894118,0.788235}%
\pgfsetstrokecolor{currentstroke}%
\pgfsetdash{}{0pt}%
\pgfpathmoveto{\pgfqpoint{4.007207in}{3.020984in}}%
\pgfpathlineto{\pgfqpoint{4.034154in}{3.036542in}}%
\pgfpathlineto{\pgfqpoint{4.034154in}{3.067658in}}%
\pgfpathlineto{\pgfqpoint{4.007207in}{3.052100in}}%
\pgfpathlineto{\pgfqpoint{4.007207in}{3.020984in}}%
\pgfpathclose%
\pgfusepath{fill}%
\end{pgfscope}%
\begin{pgfscope}%
\pgfpathrectangle{\pgfqpoint{0.765000in}{0.660000in}}{\pgfqpoint{4.620000in}{4.620000in}}%
\pgfusepath{clip}%
\pgfsetbuttcap%
\pgfsetroundjoin%
\definecolor{currentfill}{rgb}{1.000000,0.894118,0.788235}%
\pgfsetfillcolor{currentfill}%
\pgfsetlinewidth{0.000000pt}%
\definecolor{currentstroke}{rgb}{1.000000,0.894118,0.788235}%
\pgfsetstrokecolor{currentstroke}%
\pgfsetdash{}{0pt}%
\pgfpathmoveto{\pgfqpoint{3.980260in}{3.036542in}}%
\pgfpathlineto{\pgfqpoint{4.007207in}{3.020984in}}%
\pgfpathlineto{\pgfqpoint{4.007207in}{3.052100in}}%
\pgfpathlineto{\pgfqpoint{3.980260in}{3.067658in}}%
\pgfpathlineto{\pgfqpoint{3.980260in}{3.036542in}}%
\pgfpathclose%
\pgfusepath{fill}%
\end{pgfscope}%
\begin{pgfscope}%
\pgfpathrectangle{\pgfqpoint{0.765000in}{0.660000in}}{\pgfqpoint{4.620000in}{4.620000in}}%
\pgfusepath{clip}%
\pgfsetbuttcap%
\pgfsetroundjoin%
\definecolor{currentfill}{rgb}{1.000000,0.894118,0.788235}%
\pgfsetfillcolor{currentfill}%
\pgfsetlinewidth{0.000000pt}%
\definecolor{currentstroke}{rgb}{1.000000,0.894118,0.788235}%
\pgfsetstrokecolor{currentstroke}%
\pgfsetdash{}{0pt}%
\pgfpathmoveto{\pgfqpoint{3.890923in}{2.871313in}}%
\pgfpathlineto{\pgfqpoint{3.863975in}{2.855755in}}%
\pgfpathlineto{\pgfqpoint{3.980260in}{3.036542in}}%
\pgfpathlineto{\pgfqpoint{4.007207in}{3.052100in}}%
\pgfpathlineto{\pgfqpoint{3.890923in}{2.871313in}}%
\pgfpathclose%
\pgfusepath{fill}%
\end{pgfscope}%
\begin{pgfscope}%
\pgfpathrectangle{\pgfqpoint{0.765000in}{0.660000in}}{\pgfqpoint{4.620000in}{4.620000in}}%
\pgfusepath{clip}%
\pgfsetbuttcap%
\pgfsetroundjoin%
\definecolor{currentfill}{rgb}{1.000000,0.894118,0.788235}%
\pgfsetfillcolor{currentfill}%
\pgfsetlinewidth{0.000000pt}%
\definecolor{currentstroke}{rgb}{1.000000,0.894118,0.788235}%
\pgfsetstrokecolor{currentstroke}%
\pgfsetdash{}{0pt}%
\pgfpathmoveto{\pgfqpoint{3.917870in}{2.855755in}}%
\pgfpathlineto{\pgfqpoint{3.890923in}{2.871313in}}%
\pgfpathlineto{\pgfqpoint{4.007207in}{3.052100in}}%
\pgfpathlineto{\pgfqpoint{4.034154in}{3.036542in}}%
\pgfpathlineto{\pgfqpoint{3.917870in}{2.855755in}}%
\pgfpathclose%
\pgfusepath{fill}%
\end{pgfscope}%
\begin{pgfscope}%
\pgfpathrectangle{\pgfqpoint{0.765000in}{0.660000in}}{\pgfqpoint{4.620000in}{4.620000in}}%
\pgfusepath{clip}%
\pgfsetbuttcap%
\pgfsetroundjoin%
\definecolor{currentfill}{rgb}{1.000000,0.894118,0.788235}%
\pgfsetfillcolor{currentfill}%
\pgfsetlinewidth{0.000000pt}%
\definecolor{currentstroke}{rgb}{1.000000,0.894118,0.788235}%
\pgfsetstrokecolor{currentstroke}%
\pgfsetdash{}{0pt}%
\pgfpathmoveto{\pgfqpoint{3.890923in}{2.871313in}}%
\pgfpathlineto{\pgfqpoint{3.890923in}{2.902429in}}%
\pgfpathlineto{\pgfqpoint{4.007207in}{3.083216in}}%
\pgfpathlineto{\pgfqpoint{4.034154in}{3.036542in}}%
\pgfpathlineto{\pgfqpoint{3.890923in}{2.871313in}}%
\pgfpathclose%
\pgfusepath{fill}%
\end{pgfscope}%
\begin{pgfscope}%
\pgfpathrectangle{\pgfqpoint{0.765000in}{0.660000in}}{\pgfqpoint{4.620000in}{4.620000in}}%
\pgfusepath{clip}%
\pgfsetbuttcap%
\pgfsetroundjoin%
\definecolor{currentfill}{rgb}{1.000000,0.894118,0.788235}%
\pgfsetfillcolor{currentfill}%
\pgfsetlinewidth{0.000000pt}%
\definecolor{currentstroke}{rgb}{1.000000,0.894118,0.788235}%
\pgfsetstrokecolor{currentstroke}%
\pgfsetdash{}{0pt}%
\pgfpathmoveto{\pgfqpoint{3.917870in}{2.855755in}}%
\pgfpathlineto{\pgfqpoint{3.917870in}{2.886871in}}%
\pgfpathlineto{\pgfqpoint{4.034154in}{3.067658in}}%
\pgfpathlineto{\pgfqpoint{4.007207in}{3.052100in}}%
\pgfpathlineto{\pgfqpoint{3.917870in}{2.855755in}}%
\pgfpathclose%
\pgfusepath{fill}%
\end{pgfscope}%
\begin{pgfscope}%
\pgfpathrectangle{\pgfqpoint{0.765000in}{0.660000in}}{\pgfqpoint{4.620000in}{4.620000in}}%
\pgfusepath{clip}%
\pgfsetbuttcap%
\pgfsetroundjoin%
\definecolor{currentfill}{rgb}{1.000000,0.894118,0.788235}%
\pgfsetfillcolor{currentfill}%
\pgfsetlinewidth{0.000000pt}%
\definecolor{currentstroke}{rgb}{1.000000,0.894118,0.788235}%
\pgfsetstrokecolor{currentstroke}%
\pgfsetdash{}{0pt}%
\pgfpathmoveto{\pgfqpoint{3.863975in}{2.855755in}}%
\pgfpathlineto{\pgfqpoint{3.890923in}{2.840197in}}%
\pgfpathlineto{\pgfqpoint{4.007207in}{3.020984in}}%
\pgfpathlineto{\pgfqpoint{3.980260in}{3.036542in}}%
\pgfpathlineto{\pgfqpoint{3.863975in}{2.855755in}}%
\pgfpathclose%
\pgfusepath{fill}%
\end{pgfscope}%
\begin{pgfscope}%
\pgfpathrectangle{\pgfqpoint{0.765000in}{0.660000in}}{\pgfqpoint{4.620000in}{4.620000in}}%
\pgfusepath{clip}%
\pgfsetbuttcap%
\pgfsetroundjoin%
\definecolor{currentfill}{rgb}{1.000000,0.894118,0.788235}%
\pgfsetfillcolor{currentfill}%
\pgfsetlinewidth{0.000000pt}%
\definecolor{currentstroke}{rgb}{1.000000,0.894118,0.788235}%
\pgfsetstrokecolor{currentstroke}%
\pgfsetdash{}{0pt}%
\pgfpathmoveto{\pgfqpoint{3.890923in}{2.840197in}}%
\pgfpathlineto{\pgfqpoint{3.917870in}{2.855755in}}%
\pgfpathlineto{\pgfqpoint{4.034154in}{3.036542in}}%
\pgfpathlineto{\pgfqpoint{4.007207in}{3.020984in}}%
\pgfpathlineto{\pgfqpoint{3.890923in}{2.840197in}}%
\pgfpathclose%
\pgfusepath{fill}%
\end{pgfscope}%
\begin{pgfscope}%
\pgfpathrectangle{\pgfqpoint{0.765000in}{0.660000in}}{\pgfqpoint{4.620000in}{4.620000in}}%
\pgfusepath{clip}%
\pgfsetbuttcap%
\pgfsetroundjoin%
\definecolor{currentfill}{rgb}{1.000000,0.894118,0.788235}%
\pgfsetfillcolor{currentfill}%
\pgfsetlinewidth{0.000000pt}%
\definecolor{currentstroke}{rgb}{1.000000,0.894118,0.788235}%
\pgfsetstrokecolor{currentstroke}%
\pgfsetdash{}{0pt}%
\pgfpathmoveto{\pgfqpoint{3.890923in}{2.902429in}}%
\pgfpathlineto{\pgfqpoint{3.863975in}{2.886871in}}%
\pgfpathlineto{\pgfqpoint{3.980260in}{3.067658in}}%
\pgfpathlineto{\pgfqpoint{4.007207in}{3.083216in}}%
\pgfpathlineto{\pgfqpoint{3.890923in}{2.902429in}}%
\pgfpathclose%
\pgfusepath{fill}%
\end{pgfscope}%
\begin{pgfscope}%
\pgfpathrectangle{\pgfqpoint{0.765000in}{0.660000in}}{\pgfqpoint{4.620000in}{4.620000in}}%
\pgfusepath{clip}%
\pgfsetbuttcap%
\pgfsetroundjoin%
\definecolor{currentfill}{rgb}{1.000000,0.894118,0.788235}%
\pgfsetfillcolor{currentfill}%
\pgfsetlinewidth{0.000000pt}%
\definecolor{currentstroke}{rgb}{1.000000,0.894118,0.788235}%
\pgfsetstrokecolor{currentstroke}%
\pgfsetdash{}{0pt}%
\pgfpathmoveto{\pgfqpoint{3.917870in}{2.886871in}}%
\pgfpathlineto{\pgfqpoint{3.890923in}{2.902429in}}%
\pgfpathlineto{\pgfqpoint{4.007207in}{3.083216in}}%
\pgfpathlineto{\pgfqpoint{4.034154in}{3.067658in}}%
\pgfpathlineto{\pgfqpoint{3.917870in}{2.886871in}}%
\pgfpathclose%
\pgfusepath{fill}%
\end{pgfscope}%
\begin{pgfscope}%
\pgfpathrectangle{\pgfqpoint{0.765000in}{0.660000in}}{\pgfqpoint{4.620000in}{4.620000in}}%
\pgfusepath{clip}%
\pgfsetbuttcap%
\pgfsetroundjoin%
\definecolor{currentfill}{rgb}{1.000000,0.894118,0.788235}%
\pgfsetfillcolor{currentfill}%
\pgfsetlinewidth{0.000000pt}%
\definecolor{currentstroke}{rgb}{1.000000,0.894118,0.788235}%
\pgfsetstrokecolor{currentstroke}%
\pgfsetdash{}{0pt}%
\pgfpathmoveto{\pgfqpoint{3.863975in}{2.855755in}}%
\pgfpathlineto{\pgfqpoint{3.863975in}{2.886871in}}%
\pgfpathlineto{\pgfqpoint{3.980260in}{3.067658in}}%
\pgfpathlineto{\pgfqpoint{4.007207in}{3.020984in}}%
\pgfpathlineto{\pgfqpoint{3.863975in}{2.855755in}}%
\pgfpathclose%
\pgfusepath{fill}%
\end{pgfscope}%
\begin{pgfscope}%
\pgfpathrectangle{\pgfqpoint{0.765000in}{0.660000in}}{\pgfqpoint{4.620000in}{4.620000in}}%
\pgfusepath{clip}%
\pgfsetbuttcap%
\pgfsetroundjoin%
\definecolor{currentfill}{rgb}{1.000000,0.894118,0.788235}%
\pgfsetfillcolor{currentfill}%
\pgfsetlinewidth{0.000000pt}%
\definecolor{currentstroke}{rgb}{1.000000,0.894118,0.788235}%
\pgfsetstrokecolor{currentstroke}%
\pgfsetdash{}{0pt}%
\pgfpathmoveto{\pgfqpoint{3.890923in}{2.840197in}}%
\pgfpathlineto{\pgfqpoint{3.890923in}{2.871313in}}%
\pgfpathlineto{\pgfqpoint{4.007207in}{3.052100in}}%
\pgfpathlineto{\pgfqpoint{3.980260in}{3.036542in}}%
\pgfpathlineto{\pgfqpoint{3.890923in}{2.840197in}}%
\pgfpathclose%
\pgfusepath{fill}%
\end{pgfscope}%
\begin{pgfscope}%
\pgfpathrectangle{\pgfqpoint{0.765000in}{0.660000in}}{\pgfqpoint{4.620000in}{4.620000in}}%
\pgfusepath{clip}%
\pgfsetbuttcap%
\pgfsetroundjoin%
\definecolor{currentfill}{rgb}{1.000000,0.894118,0.788235}%
\pgfsetfillcolor{currentfill}%
\pgfsetlinewidth{0.000000pt}%
\definecolor{currentstroke}{rgb}{1.000000,0.894118,0.788235}%
\pgfsetstrokecolor{currentstroke}%
\pgfsetdash{}{0pt}%
\pgfpathmoveto{\pgfqpoint{3.890923in}{2.871313in}}%
\pgfpathlineto{\pgfqpoint{3.917870in}{2.886871in}}%
\pgfpathlineto{\pgfqpoint{4.034154in}{3.067658in}}%
\pgfpathlineto{\pgfqpoint{4.007207in}{3.052100in}}%
\pgfpathlineto{\pgfqpoint{3.890923in}{2.871313in}}%
\pgfpathclose%
\pgfusepath{fill}%
\end{pgfscope}%
\begin{pgfscope}%
\pgfpathrectangle{\pgfqpoint{0.765000in}{0.660000in}}{\pgfqpoint{4.620000in}{4.620000in}}%
\pgfusepath{clip}%
\pgfsetbuttcap%
\pgfsetroundjoin%
\definecolor{currentfill}{rgb}{1.000000,0.894118,0.788235}%
\pgfsetfillcolor{currentfill}%
\pgfsetlinewidth{0.000000pt}%
\definecolor{currentstroke}{rgb}{1.000000,0.894118,0.788235}%
\pgfsetstrokecolor{currentstroke}%
\pgfsetdash{}{0pt}%
\pgfpathmoveto{\pgfqpoint{3.863975in}{2.886871in}}%
\pgfpathlineto{\pgfqpoint{3.890923in}{2.871313in}}%
\pgfpathlineto{\pgfqpoint{4.007207in}{3.052100in}}%
\pgfpathlineto{\pgfqpoint{3.980260in}{3.067658in}}%
\pgfpathlineto{\pgfqpoint{3.863975in}{2.886871in}}%
\pgfpathclose%
\pgfusepath{fill}%
\end{pgfscope}%
\begin{pgfscope}%
\pgfpathrectangle{\pgfqpoint{0.765000in}{0.660000in}}{\pgfqpoint{4.620000in}{4.620000in}}%
\pgfusepath{clip}%
\pgfsetbuttcap%
\pgfsetroundjoin%
\definecolor{currentfill}{rgb}{1.000000,0.894118,0.788235}%
\pgfsetfillcolor{currentfill}%
\pgfsetlinewidth{0.000000pt}%
\definecolor{currentstroke}{rgb}{1.000000,0.894118,0.788235}%
\pgfsetstrokecolor{currentstroke}%
\pgfsetdash{}{0pt}%
\pgfpathmoveto{\pgfqpoint{3.886454in}{2.865404in}}%
\pgfpathlineto{\pgfqpoint{3.859507in}{2.849846in}}%
\pgfpathlineto{\pgfqpoint{3.886454in}{2.834288in}}%
\pgfpathlineto{\pgfqpoint{3.913402in}{2.849846in}}%
\pgfpathlineto{\pgfqpoint{3.886454in}{2.865404in}}%
\pgfpathclose%
\pgfusepath{fill}%
\end{pgfscope}%
\begin{pgfscope}%
\pgfpathrectangle{\pgfqpoint{0.765000in}{0.660000in}}{\pgfqpoint{4.620000in}{4.620000in}}%
\pgfusepath{clip}%
\pgfsetbuttcap%
\pgfsetroundjoin%
\definecolor{currentfill}{rgb}{1.000000,0.894118,0.788235}%
\pgfsetfillcolor{currentfill}%
\pgfsetlinewidth{0.000000pt}%
\definecolor{currentstroke}{rgb}{1.000000,0.894118,0.788235}%
\pgfsetstrokecolor{currentstroke}%
\pgfsetdash{}{0pt}%
\pgfpathmoveto{\pgfqpoint{3.886454in}{2.865404in}}%
\pgfpathlineto{\pgfqpoint{3.859507in}{2.849846in}}%
\pgfpathlineto{\pgfqpoint{3.859507in}{2.880962in}}%
\pgfpathlineto{\pgfqpoint{3.886454in}{2.896520in}}%
\pgfpathlineto{\pgfqpoint{3.886454in}{2.865404in}}%
\pgfpathclose%
\pgfusepath{fill}%
\end{pgfscope}%
\begin{pgfscope}%
\pgfpathrectangle{\pgfqpoint{0.765000in}{0.660000in}}{\pgfqpoint{4.620000in}{4.620000in}}%
\pgfusepath{clip}%
\pgfsetbuttcap%
\pgfsetroundjoin%
\definecolor{currentfill}{rgb}{1.000000,0.894118,0.788235}%
\pgfsetfillcolor{currentfill}%
\pgfsetlinewidth{0.000000pt}%
\definecolor{currentstroke}{rgb}{1.000000,0.894118,0.788235}%
\pgfsetstrokecolor{currentstroke}%
\pgfsetdash{}{0pt}%
\pgfpathmoveto{\pgfqpoint{3.886454in}{2.865404in}}%
\pgfpathlineto{\pgfqpoint{3.913402in}{2.849846in}}%
\pgfpathlineto{\pgfqpoint{3.913402in}{2.880962in}}%
\pgfpathlineto{\pgfqpoint{3.886454in}{2.896520in}}%
\pgfpathlineto{\pgfqpoint{3.886454in}{2.865404in}}%
\pgfpathclose%
\pgfusepath{fill}%
\end{pgfscope}%
\begin{pgfscope}%
\pgfpathrectangle{\pgfqpoint{0.765000in}{0.660000in}}{\pgfqpoint{4.620000in}{4.620000in}}%
\pgfusepath{clip}%
\pgfsetbuttcap%
\pgfsetroundjoin%
\definecolor{currentfill}{rgb}{1.000000,0.894118,0.788235}%
\pgfsetfillcolor{currentfill}%
\pgfsetlinewidth{0.000000pt}%
\definecolor{currentstroke}{rgb}{1.000000,0.894118,0.788235}%
\pgfsetstrokecolor{currentstroke}%
\pgfsetdash{}{0pt}%
\pgfpathmoveto{\pgfqpoint{3.890923in}{2.871313in}}%
\pgfpathlineto{\pgfqpoint{3.863975in}{2.855755in}}%
\pgfpathlineto{\pgfqpoint{3.890923in}{2.840197in}}%
\pgfpathlineto{\pgfqpoint{3.917870in}{2.855755in}}%
\pgfpathlineto{\pgfqpoint{3.890923in}{2.871313in}}%
\pgfpathclose%
\pgfusepath{fill}%
\end{pgfscope}%
\begin{pgfscope}%
\pgfpathrectangle{\pgfqpoint{0.765000in}{0.660000in}}{\pgfqpoint{4.620000in}{4.620000in}}%
\pgfusepath{clip}%
\pgfsetbuttcap%
\pgfsetroundjoin%
\definecolor{currentfill}{rgb}{1.000000,0.894118,0.788235}%
\pgfsetfillcolor{currentfill}%
\pgfsetlinewidth{0.000000pt}%
\definecolor{currentstroke}{rgb}{1.000000,0.894118,0.788235}%
\pgfsetstrokecolor{currentstroke}%
\pgfsetdash{}{0pt}%
\pgfpathmoveto{\pgfqpoint{3.890923in}{2.871313in}}%
\pgfpathlineto{\pgfqpoint{3.863975in}{2.855755in}}%
\pgfpathlineto{\pgfqpoint{3.863975in}{2.886871in}}%
\pgfpathlineto{\pgfqpoint{3.890923in}{2.902429in}}%
\pgfpathlineto{\pgfqpoint{3.890923in}{2.871313in}}%
\pgfpathclose%
\pgfusepath{fill}%
\end{pgfscope}%
\begin{pgfscope}%
\pgfpathrectangle{\pgfqpoint{0.765000in}{0.660000in}}{\pgfqpoint{4.620000in}{4.620000in}}%
\pgfusepath{clip}%
\pgfsetbuttcap%
\pgfsetroundjoin%
\definecolor{currentfill}{rgb}{1.000000,0.894118,0.788235}%
\pgfsetfillcolor{currentfill}%
\pgfsetlinewidth{0.000000pt}%
\definecolor{currentstroke}{rgb}{1.000000,0.894118,0.788235}%
\pgfsetstrokecolor{currentstroke}%
\pgfsetdash{}{0pt}%
\pgfpathmoveto{\pgfqpoint{3.890923in}{2.871313in}}%
\pgfpathlineto{\pgfqpoint{3.917870in}{2.855755in}}%
\pgfpathlineto{\pgfqpoint{3.917870in}{2.886871in}}%
\pgfpathlineto{\pgfqpoint{3.890923in}{2.902429in}}%
\pgfpathlineto{\pgfqpoint{3.890923in}{2.871313in}}%
\pgfpathclose%
\pgfusepath{fill}%
\end{pgfscope}%
\begin{pgfscope}%
\pgfpathrectangle{\pgfqpoint{0.765000in}{0.660000in}}{\pgfqpoint{4.620000in}{4.620000in}}%
\pgfusepath{clip}%
\pgfsetbuttcap%
\pgfsetroundjoin%
\definecolor{currentfill}{rgb}{1.000000,0.894118,0.788235}%
\pgfsetfillcolor{currentfill}%
\pgfsetlinewidth{0.000000pt}%
\definecolor{currentstroke}{rgb}{1.000000,0.894118,0.788235}%
\pgfsetstrokecolor{currentstroke}%
\pgfsetdash{}{0pt}%
\pgfpathmoveto{\pgfqpoint{3.886454in}{2.896520in}}%
\pgfpathlineto{\pgfqpoint{3.859507in}{2.880962in}}%
\pgfpathlineto{\pgfqpoint{3.886454in}{2.865404in}}%
\pgfpathlineto{\pgfqpoint{3.913402in}{2.880962in}}%
\pgfpathlineto{\pgfqpoint{3.886454in}{2.896520in}}%
\pgfpathclose%
\pgfusepath{fill}%
\end{pgfscope}%
\begin{pgfscope}%
\pgfpathrectangle{\pgfqpoint{0.765000in}{0.660000in}}{\pgfqpoint{4.620000in}{4.620000in}}%
\pgfusepath{clip}%
\pgfsetbuttcap%
\pgfsetroundjoin%
\definecolor{currentfill}{rgb}{1.000000,0.894118,0.788235}%
\pgfsetfillcolor{currentfill}%
\pgfsetlinewidth{0.000000pt}%
\definecolor{currentstroke}{rgb}{1.000000,0.894118,0.788235}%
\pgfsetstrokecolor{currentstroke}%
\pgfsetdash{}{0pt}%
\pgfpathmoveto{\pgfqpoint{3.886454in}{2.834288in}}%
\pgfpathlineto{\pgfqpoint{3.913402in}{2.849846in}}%
\pgfpathlineto{\pgfqpoint{3.913402in}{2.880962in}}%
\pgfpathlineto{\pgfqpoint{3.886454in}{2.865404in}}%
\pgfpathlineto{\pgfqpoint{3.886454in}{2.834288in}}%
\pgfpathclose%
\pgfusepath{fill}%
\end{pgfscope}%
\begin{pgfscope}%
\pgfpathrectangle{\pgfqpoint{0.765000in}{0.660000in}}{\pgfqpoint{4.620000in}{4.620000in}}%
\pgfusepath{clip}%
\pgfsetbuttcap%
\pgfsetroundjoin%
\definecolor{currentfill}{rgb}{1.000000,0.894118,0.788235}%
\pgfsetfillcolor{currentfill}%
\pgfsetlinewidth{0.000000pt}%
\definecolor{currentstroke}{rgb}{1.000000,0.894118,0.788235}%
\pgfsetstrokecolor{currentstroke}%
\pgfsetdash{}{0pt}%
\pgfpathmoveto{\pgfqpoint{3.859507in}{2.849846in}}%
\pgfpathlineto{\pgfqpoint{3.886454in}{2.834288in}}%
\pgfpathlineto{\pgfqpoint{3.886454in}{2.865404in}}%
\pgfpathlineto{\pgfqpoint{3.859507in}{2.880962in}}%
\pgfpathlineto{\pgfqpoint{3.859507in}{2.849846in}}%
\pgfpathclose%
\pgfusepath{fill}%
\end{pgfscope}%
\begin{pgfscope}%
\pgfpathrectangle{\pgfqpoint{0.765000in}{0.660000in}}{\pgfqpoint{4.620000in}{4.620000in}}%
\pgfusepath{clip}%
\pgfsetbuttcap%
\pgfsetroundjoin%
\definecolor{currentfill}{rgb}{1.000000,0.894118,0.788235}%
\pgfsetfillcolor{currentfill}%
\pgfsetlinewidth{0.000000pt}%
\definecolor{currentstroke}{rgb}{1.000000,0.894118,0.788235}%
\pgfsetstrokecolor{currentstroke}%
\pgfsetdash{}{0pt}%
\pgfpathmoveto{\pgfqpoint{3.890923in}{2.902429in}}%
\pgfpathlineto{\pgfqpoint{3.863975in}{2.886871in}}%
\pgfpathlineto{\pgfqpoint{3.890923in}{2.871313in}}%
\pgfpathlineto{\pgfqpoint{3.917870in}{2.886871in}}%
\pgfpathlineto{\pgfqpoint{3.890923in}{2.902429in}}%
\pgfpathclose%
\pgfusepath{fill}%
\end{pgfscope}%
\begin{pgfscope}%
\pgfpathrectangle{\pgfqpoint{0.765000in}{0.660000in}}{\pgfqpoint{4.620000in}{4.620000in}}%
\pgfusepath{clip}%
\pgfsetbuttcap%
\pgfsetroundjoin%
\definecolor{currentfill}{rgb}{1.000000,0.894118,0.788235}%
\pgfsetfillcolor{currentfill}%
\pgfsetlinewidth{0.000000pt}%
\definecolor{currentstroke}{rgb}{1.000000,0.894118,0.788235}%
\pgfsetstrokecolor{currentstroke}%
\pgfsetdash{}{0pt}%
\pgfpathmoveto{\pgfqpoint{3.890923in}{2.840197in}}%
\pgfpathlineto{\pgfqpoint{3.917870in}{2.855755in}}%
\pgfpathlineto{\pgfqpoint{3.917870in}{2.886871in}}%
\pgfpathlineto{\pgfqpoint{3.890923in}{2.871313in}}%
\pgfpathlineto{\pgfqpoint{3.890923in}{2.840197in}}%
\pgfpathclose%
\pgfusepath{fill}%
\end{pgfscope}%
\begin{pgfscope}%
\pgfpathrectangle{\pgfqpoint{0.765000in}{0.660000in}}{\pgfqpoint{4.620000in}{4.620000in}}%
\pgfusepath{clip}%
\pgfsetbuttcap%
\pgfsetroundjoin%
\definecolor{currentfill}{rgb}{1.000000,0.894118,0.788235}%
\pgfsetfillcolor{currentfill}%
\pgfsetlinewidth{0.000000pt}%
\definecolor{currentstroke}{rgb}{1.000000,0.894118,0.788235}%
\pgfsetstrokecolor{currentstroke}%
\pgfsetdash{}{0pt}%
\pgfpathmoveto{\pgfqpoint{3.863975in}{2.855755in}}%
\pgfpathlineto{\pgfqpoint{3.890923in}{2.840197in}}%
\pgfpathlineto{\pgfqpoint{3.890923in}{2.871313in}}%
\pgfpathlineto{\pgfqpoint{3.863975in}{2.886871in}}%
\pgfpathlineto{\pgfqpoint{3.863975in}{2.855755in}}%
\pgfpathclose%
\pgfusepath{fill}%
\end{pgfscope}%
\begin{pgfscope}%
\pgfpathrectangle{\pgfqpoint{0.765000in}{0.660000in}}{\pgfqpoint{4.620000in}{4.620000in}}%
\pgfusepath{clip}%
\pgfsetbuttcap%
\pgfsetroundjoin%
\definecolor{currentfill}{rgb}{1.000000,0.894118,0.788235}%
\pgfsetfillcolor{currentfill}%
\pgfsetlinewidth{0.000000pt}%
\definecolor{currentstroke}{rgb}{1.000000,0.894118,0.788235}%
\pgfsetstrokecolor{currentstroke}%
\pgfsetdash{}{0pt}%
\pgfpathmoveto{\pgfqpoint{3.886454in}{2.865404in}}%
\pgfpathlineto{\pgfqpoint{3.859507in}{2.849846in}}%
\pgfpathlineto{\pgfqpoint{3.863975in}{2.855755in}}%
\pgfpathlineto{\pgfqpoint{3.890923in}{2.871313in}}%
\pgfpathlineto{\pgfqpoint{3.886454in}{2.865404in}}%
\pgfpathclose%
\pgfusepath{fill}%
\end{pgfscope}%
\begin{pgfscope}%
\pgfpathrectangle{\pgfqpoint{0.765000in}{0.660000in}}{\pgfqpoint{4.620000in}{4.620000in}}%
\pgfusepath{clip}%
\pgfsetbuttcap%
\pgfsetroundjoin%
\definecolor{currentfill}{rgb}{1.000000,0.894118,0.788235}%
\pgfsetfillcolor{currentfill}%
\pgfsetlinewidth{0.000000pt}%
\definecolor{currentstroke}{rgb}{1.000000,0.894118,0.788235}%
\pgfsetstrokecolor{currentstroke}%
\pgfsetdash{}{0pt}%
\pgfpathmoveto{\pgfqpoint{3.913402in}{2.849846in}}%
\pgfpathlineto{\pgfqpoint{3.886454in}{2.865404in}}%
\pgfpathlineto{\pgfqpoint{3.890923in}{2.871313in}}%
\pgfpathlineto{\pgfqpoint{3.917870in}{2.855755in}}%
\pgfpathlineto{\pgfqpoint{3.913402in}{2.849846in}}%
\pgfpathclose%
\pgfusepath{fill}%
\end{pgfscope}%
\begin{pgfscope}%
\pgfpathrectangle{\pgfqpoint{0.765000in}{0.660000in}}{\pgfqpoint{4.620000in}{4.620000in}}%
\pgfusepath{clip}%
\pgfsetbuttcap%
\pgfsetroundjoin%
\definecolor{currentfill}{rgb}{1.000000,0.894118,0.788235}%
\pgfsetfillcolor{currentfill}%
\pgfsetlinewidth{0.000000pt}%
\definecolor{currentstroke}{rgb}{1.000000,0.894118,0.788235}%
\pgfsetstrokecolor{currentstroke}%
\pgfsetdash{}{0pt}%
\pgfpathmoveto{\pgfqpoint{3.886454in}{2.865404in}}%
\pgfpathlineto{\pgfqpoint{3.886454in}{2.896520in}}%
\pgfpathlineto{\pgfqpoint{3.890923in}{2.902429in}}%
\pgfpathlineto{\pgfqpoint{3.917870in}{2.855755in}}%
\pgfpathlineto{\pgfqpoint{3.886454in}{2.865404in}}%
\pgfpathclose%
\pgfusepath{fill}%
\end{pgfscope}%
\begin{pgfscope}%
\pgfpathrectangle{\pgfqpoint{0.765000in}{0.660000in}}{\pgfqpoint{4.620000in}{4.620000in}}%
\pgfusepath{clip}%
\pgfsetbuttcap%
\pgfsetroundjoin%
\definecolor{currentfill}{rgb}{1.000000,0.894118,0.788235}%
\pgfsetfillcolor{currentfill}%
\pgfsetlinewidth{0.000000pt}%
\definecolor{currentstroke}{rgb}{1.000000,0.894118,0.788235}%
\pgfsetstrokecolor{currentstroke}%
\pgfsetdash{}{0pt}%
\pgfpathmoveto{\pgfqpoint{3.913402in}{2.849846in}}%
\pgfpathlineto{\pgfqpoint{3.913402in}{2.880962in}}%
\pgfpathlineto{\pgfqpoint{3.917870in}{2.886871in}}%
\pgfpathlineto{\pgfqpoint{3.890923in}{2.871313in}}%
\pgfpathlineto{\pgfqpoint{3.913402in}{2.849846in}}%
\pgfpathclose%
\pgfusepath{fill}%
\end{pgfscope}%
\begin{pgfscope}%
\pgfpathrectangle{\pgfqpoint{0.765000in}{0.660000in}}{\pgfqpoint{4.620000in}{4.620000in}}%
\pgfusepath{clip}%
\pgfsetbuttcap%
\pgfsetroundjoin%
\definecolor{currentfill}{rgb}{1.000000,0.894118,0.788235}%
\pgfsetfillcolor{currentfill}%
\pgfsetlinewidth{0.000000pt}%
\definecolor{currentstroke}{rgb}{1.000000,0.894118,0.788235}%
\pgfsetstrokecolor{currentstroke}%
\pgfsetdash{}{0pt}%
\pgfpathmoveto{\pgfqpoint{3.859507in}{2.849846in}}%
\pgfpathlineto{\pgfqpoint{3.886454in}{2.834288in}}%
\pgfpathlineto{\pgfqpoint{3.890923in}{2.840197in}}%
\pgfpathlineto{\pgfqpoint{3.863975in}{2.855755in}}%
\pgfpathlineto{\pgfqpoint{3.859507in}{2.849846in}}%
\pgfpathclose%
\pgfusepath{fill}%
\end{pgfscope}%
\begin{pgfscope}%
\pgfpathrectangle{\pgfqpoint{0.765000in}{0.660000in}}{\pgfqpoint{4.620000in}{4.620000in}}%
\pgfusepath{clip}%
\pgfsetbuttcap%
\pgfsetroundjoin%
\definecolor{currentfill}{rgb}{1.000000,0.894118,0.788235}%
\pgfsetfillcolor{currentfill}%
\pgfsetlinewidth{0.000000pt}%
\definecolor{currentstroke}{rgb}{1.000000,0.894118,0.788235}%
\pgfsetstrokecolor{currentstroke}%
\pgfsetdash{}{0pt}%
\pgfpathmoveto{\pgfqpoint{3.886454in}{2.834288in}}%
\pgfpathlineto{\pgfqpoint{3.913402in}{2.849846in}}%
\pgfpathlineto{\pgfqpoint{3.917870in}{2.855755in}}%
\pgfpathlineto{\pgfqpoint{3.890923in}{2.840197in}}%
\pgfpathlineto{\pgfqpoint{3.886454in}{2.834288in}}%
\pgfpathclose%
\pgfusepath{fill}%
\end{pgfscope}%
\begin{pgfscope}%
\pgfpathrectangle{\pgfqpoint{0.765000in}{0.660000in}}{\pgfqpoint{4.620000in}{4.620000in}}%
\pgfusepath{clip}%
\pgfsetbuttcap%
\pgfsetroundjoin%
\definecolor{currentfill}{rgb}{1.000000,0.894118,0.788235}%
\pgfsetfillcolor{currentfill}%
\pgfsetlinewidth{0.000000pt}%
\definecolor{currentstroke}{rgb}{1.000000,0.894118,0.788235}%
\pgfsetstrokecolor{currentstroke}%
\pgfsetdash{}{0pt}%
\pgfpathmoveto{\pgfqpoint{3.886454in}{2.896520in}}%
\pgfpathlineto{\pgfqpoint{3.859507in}{2.880962in}}%
\pgfpathlineto{\pgfqpoint{3.863975in}{2.886871in}}%
\pgfpathlineto{\pgfqpoint{3.890923in}{2.902429in}}%
\pgfpathlineto{\pgfqpoint{3.886454in}{2.896520in}}%
\pgfpathclose%
\pgfusepath{fill}%
\end{pgfscope}%
\begin{pgfscope}%
\pgfpathrectangle{\pgfqpoint{0.765000in}{0.660000in}}{\pgfqpoint{4.620000in}{4.620000in}}%
\pgfusepath{clip}%
\pgfsetbuttcap%
\pgfsetroundjoin%
\definecolor{currentfill}{rgb}{1.000000,0.894118,0.788235}%
\pgfsetfillcolor{currentfill}%
\pgfsetlinewidth{0.000000pt}%
\definecolor{currentstroke}{rgb}{1.000000,0.894118,0.788235}%
\pgfsetstrokecolor{currentstroke}%
\pgfsetdash{}{0pt}%
\pgfpathmoveto{\pgfqpoint{3.913402in}{2.880962in}}%
\pgfpathlineto{\pgfqpoint{3.886454in}{2.896520in}}%
\pgfpathlineto{\pgfqpoint{3.890923in}{2.902429in}}%
\pgfpathlineto{\pgfqpoint{3.917870in}{2.886871in}}%
\pgfpathlineto{\pgfqpoint{3.913402in}{2.880962in}}%
\pgfpathclose%
\pgfusepath{fill}%
\end{pgfscope}%
\begin{pgfscope}%
\pgfpathrectangle{\pgfqpoint{0.765000in}{0.660000in}}{\pgfqpoint{4.620000in}{4.620000in}}%
\pgfusepath{clip}%
\pgfsetbuttcap%
\pgfsetroundjoin%
\definecolor{currentfill}{rgb}{1.000000,0.894118,0.788235}%
\pgfsetfillcolor{currentfill}%
\pgfsetlinewidth{0.000000pt}%
\definecolor{currentstroke}{rgb}{1.000000,0.894118,0.788235}%
\pgfsetstrokecolor{currentstroke}%
\pgfsetdash{}{0pt}%
\pgfpathmoveto{\pgfqpoint{3.859507in}{2.849846in}}%
\pgfpathlineto{\pgfqpoint{3.859507in}{2.880962in}}%
\pgfpathlineto{\pgfqpoint{3.863975in}{2.886871in}}%
\pgfpathlineto{\pgfqpoint{3.890923in}{2.840197in}}%
\pgfpathlineto{\pgfqpoint{3.859507in}{2.849846in}}%
\pgfpathclose%
\pgfusepath{fill}%
\end{pgfscope}%
\begin{pgfscope}%
\pgfpathrectangle{\pgfqpoint{0.765000in}{0.660000in}}{\pgfqpoint{4.620000in}{4.620000in}}%
\pgfusepath{clip}%
\pgfsetbuttcap%
\pgfsetroundjoin%
\definecolor{currentfill}{rgb}{1.000000,0.894118,0.788235}%
\pgfsetfillcolor{currentfill}%
\pgfsetlinewidth{0.000000pt}%
\definecolor{currentstroke}{rgb}{1.000000,0.894118,0.788235}%
\pgfsetstrokecolor{currentstroke}%
\pgfsetdash{}{0pt}%
\pgfpathmoveto{\pgfqpoint{3.886454in}{2.834288in}}%
\pgfpathlineto{\pgfqpoint{3.886454in}{2.865404in}}%
\pgfpathlineto{\pgfqpoint{3.890923in}{2.871313in}}%
\pgfpathlineto{\pgfqpoint{3.863975in}{2.855755in}}%
\pgfpathlineto{\pgfqpoint{3.886454in}{2.834288in}}%
\pgfpathclose%
\pgfusepath{fill}%
\end{pgfscope}%
\begin{pgfscope}%
\pgfpathrectangle{\pgfqpoint{0.765000in}{0.660000in}}{\pgfqpoint{4.620000in}{4.620000in}}%
\pgfusepath{clip}%
\pgfsetbuttcap%
\pgfsetroundjoin%
\definecolor{currentfill}{rgb}{1.000000,0.894118,0.788235}%
\pgfsetfillcolor{currentfill}%
\pgfsetlinewidth{0.000000pt}%
\definecolor{currentstroke}{rgb}{1.000000,0.894118,0.788235}%
\pgfsetstrokecolor{currentstroke}%
\pgfsetdash{}{0pt}%
\pgfpathmoveto{\pgfqpoint{3.886454in}{2.865404in}}%
\pgfpathlineto{\pgfqpoint{3.913402in}{2.880962in}}%
\pgfpathlineto{\pgfqpoint{3.917870in}{2.886871in}}%
\pgfpathlineto{\pgfqpoint{3.890923in}{2.871313in}}%
\pgfpathlineto{\pgfqpoint{3.886454in}{2.865404in}}%
\pgfpathclose%
\pgfusepath{fill}%
\end{pgfscope}%
\begin{pgfscope}%
\pgfpathrectangle{\pgfqpoint{0.765000in}{0.660000in}}{\pgfqpoint{4.620000in}{4.620000in}}%
\pgfusepath{clip}%
\pgfsetbuttcap%
\pgfsetroundjoin%
\definecolor{currentfill}{rgb}{1.000000,0.894118,0.788235}%
\pgfsetfillcolor{currentfill}%
\pgfsetlinewidth{0.000000pt}%
\definecolor{currentstroke}{rgb}{1.000000,0.894118,0.788235}%
\pgfsetstrokecolor{currentstroke}%
\pgfsetdash{}{0pt}%
\pgfpathmoveto{\pgfqpoint{3.859507in}{2.880962in}}%
\pgfpathlineto{\pgfqpoint{3.886454in}{2.865404in}}%
\pgfpathlineto{\pgfqpoint{3.890923in}{2.871313in}}%
\pgfpathlineto{\pgfqpoint{3.863975in}{2.886871in}}%
\pgfpathlineto{\pgfqpoint{3.859507in}{2.880962in}}%
\pgfpathclose%
\pgfusepath{fill}%
\end{pgfscope}%
\begin{pgfscope}%
\pgfpathrectangle{\pgfqpoint{0.765000in}{0.660000in}}{\pgfqpoint{4.620000in}{4.620000in}}%
\pgfusepath{clip}%
\pgfsetbuttcap%
\pgfsetroundjoin%
\definecolor{currentfill}{rgb}{1.000000,0.894118,0.788235}%
\pgfsetfillcolor{currentfill}%
\pgfsetlinewidth{0.000000pt}%
\definecolor{currentstroke}{rgb}{1.000000,0.894118,0.788235}%
\pgfsetstrokecolor{currentstroke}%
\pgfsetdash{}{0pt}%
\pgfpathmoveto{\pgfqpoint{3.886454in}{2.865404in}}%
\pgfpathlineto{\pgfqpoint{3.859507in}{2.849846in}}%
\pgfpathlineto{\pgfqpoint{3.886454in}{2.834288in}}%
\pgfpathlineto{\pgfqpoint{3.913402in}{2.849846in}}%
\pgfpathlineto{\pgfqpoint{3.886454in}{2.865404in}}%
\pgfpathclose%
\pgfusepath{fill}%
\end{pgfscope}%
\begin{pgfscope}%
\pgfpathrectangle{\pgfqpoint{0.765000in}{0.660000in}}{\pgfqpoint{4.620000in}{4.620000in}}%
\pgfusepath{clip}%
\pgfsetbuttcap%
\pgfsetroundjoin%
\definecolor{currentfill}{rgb}{1.000000,0.894118,0.788235}%
\pgfsetfillcolor{currentfill}%
\pgfsetlinewidth{0.000000pt}%
\definecolor{currentstroke}{rgb}{1.000000,0.894118,0.788235}%
\pgfsetstrokecolor{currentstroke}%
\pgfsetdash{}{0pt}%
\pgfpathmoveto{\pgfqpoint{3.886454in}{2.865404in}}%
\pgfpathlineto{\pgfqpoint{3.859507in}{2.849846in}}%
\pgfpathlineto{\pgfqpoint{3.859507in}{2.880962in}}%
\pgfpathlineto{\pgfqpoint{3.886454in}{2.896520in}}%
\pgfpathlineto{\pgfqpoint{3.886454in}{2.865404in}}%
\pgfpathclose%
\pgfusepath{fill}%
\end{pgfscope}%
\begin{pgfscope}%
\pgfpathrectangle{\pgfqpoint{0.765000in}{0.660000in}}{\pgfqpoint{4.620000in}{4.620000in}}%
\pgfusepath{clip}%
\pgfsetbuttcap%
\pgfsetroundjoin%
\definecolor{currentfill}{rgb}{1.000000,0.894118,0.788235}%
\pgfsetfillcolor{currentfill}%
\pgfsetlinewidth{0.000000pt}%
\definecolor{currentstroke}{rgb}{1.000000,0.894118,0.788235}%
\pgfsetstrokecolor{currentstroke}%
\pgfsetdash{}{0pt}%
\pgfpathmoveto{\pgfqpoint{3.886454in}{2.865404in}}%
\pgfpathlineto{\pgfqpoint{3.913402in}{2.849846in}}%
\pgfpathlineto{\pgfqpoint{3.913402in}{2.880962in}}%
\pgfpathlineto{\pgfqpoint{3.886454in}{2.896520in}}%
\pgfpathlineto{\pgfqpoint{3.886454in}{2.865404in}}%
\pgfpathclose%
\pgfusepath{fill}%
\end{pgfscope}%
\begin{pgfscope}%
\pgfpathrectangle{\pgfqpoint{0.765000in}{0.660000in}}{\pgfqpoint{4.620000in}{4.620000in}}%
\pgfusepath{clip}%
\pgfsetbuttcap%
\pgfsetroundjoin%
\definecolor{currentfill}{rgb}{1.000000,0.894118,0.788235}%
\pgfsetfillcolor{currentfill}%
\pgfsetlinewidth{0.000000pt}%
\definecolor{currentstroke}{rgb}{1.000000,0.894118,0.788235}%
\pgfsetstrokecolor{currentstroke}%
\pgfsetdash{}{0pt}%
\pgfpathmoveto{\pgfqpoint{3.880311in}{2.858581in}}%
\pgfpathlineto{\pgfqpoint{3.853364in}{2.843023in}}%
\pgfpathlineto{\pgfqpoint{3.880311in}{2.827465in}}%
\pgfpathlineto{\pgfqpoint{3.907258in}{2.843023in}}%
\pgfpathlineto{\pgfqpoint{3.880311in}{2.858581in}}%
\pgfpathclose%
\pgfusepath{fill}%
\end{pgfscope}%
\begin{pgfscope}%
\pgfpathrectangle{\pgfqpoint{0.765000in}{0.660000in}}{\pgfqpoint{4.620000in}{4.620000in}}%
\pgfusepath{clip}%
\pgfsetbuttcap%
\pgfsetroundjoin%
\definecolor{currentfill}{rgb}{1.000000,0.894118,0.788235}%
\pgfsetfillcolor{currentfill}%
\pgfsetlinewidth{0.000000pt}%
\definecolor{currentstroke}{rgb}{1.000000,0.894118,0.788235}%
\pgfsetstrokecolor{currentstroke}%
\pgfsetdash{}{0pt}%
\pgfpathmoveto{\pgfqpoint{3.880311in}{2.858581in}}%
\pgfpathlineto{\pgfqpoint{3.853364in}{2.843023in}}%
\pgfpathlineto{\pgfqpoint{3.853364in}{2.874139in}}%
\pgfpathlineto{\pgfqpoint{3.880311in}{2.889697in}}%
\pgfpathlineto{\pgfqpoint{3.880311in}{2.858581in}}%
\pgfpathclose%
\pgfusepath{fill}%
\end{pgfscope}%
\begin{pgfscope}%
\pgfpathrectangle{\pgfqpoint{0.765000in}{0.660000in}}{\pgfqpoint{4.620000in}{4.620000in}}%
\pgfusepath{clip}%
\pgfsetbuttcap%
\pgfsetroundjoin%
\definecolor{currentfill}{rgb}{1.000000,0.894118,0.788235}%
\pgfsetfillcolor{currentfill}%
\pgfsetlinewidth{0.000000pt}%
\definecolor{currentstroke}{rgb}{1.000000,0.894118,0.788235}%
\pgfsetstrokecolor{currentstroke}%
\pgfsetdash{}{0pt}%
\pgfpathmoveto{\pgfqpoint{3.880311in}{2.858581in}}%
\pgfpathlineto{\pgfqpoint{3.907258in}{2.843023in}}%
\pgfpathlineto{\pgfqpoint{3.907258in}{2.874139in}}%
\pgfpathlineto{\pgfqpoint{3.880311in}{2.889697in}}%
\pgfpathlineto{\pgfqpoint{3.880311in}{2.858581in}}%
\pgfpathclose%
\pgfusepath{fill}%
\end{pgfscope}%
\begin{pgfscope}%
\pgfpathrectangle{\pgfqpoint{0.765000in}{0.660000in}}{\pgfqpoint{4.620000in}{4.620000in}}%
\pgfusepath{clip}%
\pgfsetbuttcap%
\pgfsetroundjoin%
\definecolor{currentfill}{rgb}{1.000000,0.894118,0.788235}%
\pgfsetfillcolor{currentfill}%
\pgfsetlinewidth{0.000000pt}%
\definecolor{currentstroke}{rgb}{1.000000,0.894118,0.788235}%
\pgfsetstrokecolor{currentstroke}%
\pgfsetdash{}{0pt}%
\pgfpathmoveto{\pgfqpoint{3.886454in}{2.896520in}}%
\pgfpathlineto{\pgfqpoint{3.859507in}{2.880962in}}%
\pgfpathlineto{\pgfqpoint{3.886454in}{2.865404in}}%
\pgfpathlineto{\pgfqpoint{3.913402in}{2.880962in}}%
\pgfpathlineto{\pgfqpoint{3.886454in}{2.896520in}}%
\pgfpathclose%
\pgfusepath{fill}%
\end{pgfscope}%
\begin{pgfscope}%
\pgfpathrectangle{\pgfqpoint{0.765000in}{0.660000in}}{\pgfqpoint{4.620000in}{4.620000in}}%
\pgfusepath{clip}%
\pgfsetbuttcap%
\pgfsetroundjoin%
\definecolor{currentfill}{rgb}{1.000000,0.894118,0.788235}%
\pgfsetfillcolor{currentfill}%
\pgfsetlinewidth{0.000000pt}%
\definecolor{currentstroke}{rgb}{1.000000,0.894118,0.788235}%
\pgfsetstrokecolor{currentstroke}%
\pgfsetdash{}{0pt}%
\pgfpathmoveto{\pgfqpoint{3.886454in}{2.834288in}}%
\pgfpathlineto{\pgfqpoint{3.913402in}{2.849846in}}%
\pgfpathlineto{\pgfqpoint{3.913402in}{2.880962in}}%
\pgfpathlineto{\pgfqpoint{3.886454in}{2.865404in}}%
\pgfpathlineto{\pgfqpoint{3.886454in}{2.834288in}}%
\pgfpathclose%
\pgfusepath{fill}%
\end{pgfscope}%
\begin{pgfscope}%
\pgfpathrectangle{\pgfqpoint{0.765000in}{0.660000in}}{\pgfqpoint{4.620000in}{4.620000in}}%
\pgfusepath{clip}%
\pgfsetbuttcap%
\pgfsetroundjoin%
\definecolor{currentfill}{rgb}{1.000000,0.894118,0.788235}%
\pgfsetfillcolor{currentfill}%
\pgfsetlinewidth{0.000000pt}%
\definecolor{currentstroke}{rgb}{1.000000,0.894118,0.788235}%
\pgfsetstrokecolor{currentstroke}%
\pgfsetdash{}{0pt}%
\pgfpathmoveto{\pgfqpoint{3.859507in}{2.849846in}}%
\pgfpathlineto{\pgfqpoint{3.886454in}{2.834288in}}%
\pgfpathlineto{\pgfqpoint{3.886454in}{2.865404in}}%
\pgfpathlineto{\pgfqpoint{3.859507in}{2.880962in}}%
\pgfpathlineto{\pgfqpoint{3.859507in}{2.849846in}}%
\pgfpathclose%
\pgfusepath{fill}%
\end{pgfscope}%
\begin{pgfscope}%
\pgfpathrectangle{\pgfqpoint{0.765000in}{0.660000in}}{\pgfqpoint{4.620000in}{4.620000in}}%
\pgfusepath{clip}%
\pgfsetbuttcap%
\pgfsetroundjoin%
\definecolor{currentfill}{rgb}{1.000000,0.894118,0.788235}%
\pgfsetfillcolor{currentfill}%
\pgfsetlinewidth{0.000000pt}%
\definecolor{currentstroke}{rgb}{1.000000,0.894118,0.788235}%
\pgfsetstrokecolor{currentstroke}%
\pgfsetdash{}{0pt}%
\pgfpathmoveto{\pgfqpoint{3.880311in}{2.889697in}}%
\pgfpathlineto{\pgfqpoint{3.853364in}{2.874139in}}%
\pgfpathlineto{\pgfqpoint{3.880311in}{2.858581in}}%
\pgfpathlineto{\pgfqpoint{3.907258in}{2.874139in}}%
\pgfpathlineto{\pgfqpoint{3.880311in}{2.889697in}}%
\pgfpathclose%
\pgfusepath{fill}%
\end{pgfscope}%
\begin{pgfscope}%
\pgfpathrectangle{\pgfqpoint{0.765000in}{0.660000in}}{\pgfqpoint{4.620000in}{4.620000in}}%
\pgfusepath{clip}%
\pgfsetbuttcap%
\pgfsetroundjoin%
\definecolor{currentfill}{rgb}{1.000000,0.894118,0.788235}%
\pgfsetfillcolor{currentfill}%
\pgfsetlinewidth{0.000000pt}%
\definecolor{currentstroke}{rgb}{1.000000,0.894118,0.788235}%
\pgfsetstrokecolor{currentstroke}%
\pgfsetdash{}{0pt}%
\pgfpathmoveto{\pgfqpoint{3.880311in}{2.827465in}}%
\pgfpathlineto{\pgfqpoint{3.907258in}{2.843023in}}%
\pgfpathlineto{\pgfqpoint{3.907258in}{2.874139in}}%
\pgfpathlineto{\pgfqpoint{3.880311in}{2.858581in}}%
\pgfpathlineto{\pgfqpoint{3.880311in}{2.827465in}}%
\pgfpathclose%
\pgfusepath{fill}%
\end{pgfscope}%
\begin{pgfscope}%
\pgfpathrectangle{\pgfqpoint{0.765000in}{0.660000in}}{\pgfqpoint{4.620000in}{4.620000in}}%
\pgfusepath{clip}%
\pgfsetbuttcap%
\pgfsetroundjoin%
\definecolor{currentfill}{rgb}{1.000000,0.894118,0.788235}%
\pgfsetfillcolor{currentfill}%
\pgfsetlinewidth{0.000000pt}%
\definecolor{currentstroke}{rgb}{1.000000,0.894118,0.788235}%
\pgfsetstrokecolor{currentstroke}%
\pgfsetdash{}{0pt}%
\pgfpathmoveto{\pgfqpoint{3.853364in}{2.843023in}}%
\pgfpathlineto{\pgfqpoint{3.880311in}{2.827465in}}%
\pgfpathlineto{\pgfqpoint{3.880311in}{2.858581in}}%
\pgfpathlineto{\pgfqpoint{3.853364in}{2.874139in}}%
\pgfpathlineto{\pgfqpoint{3.853364in}{2.843023in}}%
\pgfpathclose%
\pgfusepath{fill}%
\end{pgfscope}%
\begin{pgfscope}%
\pgfpathrectangle{\pgfqpoint{0.765000in}{0.660000in}}{\pgfqpoint{4.620000in}{4.620000in}}%
\pgfusepath{clip}%
\pgfsetbuttcap%
\pgfsetroundjoin%
\definecolor{currentfill}{rgb}{1.000000,0.894118,0.788235}%
\pgfsetfillcolor{currentfill}%
\pgfsetlinewidth{0.000000pt}%
\definecolor{currentstroke}{rgb}{1.000000,0.894118,0.788235}%
\pgfsetstrokecolor{currentstroke}%
\pgfsetdash{}{0pt}%
\pgfpathmoveto{\pgfqpoint{3.886454in}{2.865404in}}%
\pgfpathlineto{\pgfqpoint{3.859507in}{2.849846in}}%
\pgfpathlineto{\pgfqpoint{3.853364in}{2.843023in}}%
\pgfpathlineto{\pgfqpoint{3.880311in}{2.858581in}}%
\pgfpathlineto{\pgfqpoint{3.886454in}{2.865404in}}%
\pgfpathclose%
\pgfusepath{fill}%
\end{pgfscope}%
\begin{pgfscope}%
\pgfpathrectangle{\pgfqpoint{0.765000in}{0.660000in}}{\pgfqpoint{4.620000in}{4.620000in}}%
\pgfusepath{clip}%
\pgfsetbuttcap%
\pgfsetroundjoin%
\definecolor{currentfill}{rgb}{1.000000,0.894118,0.788235}%
\pgfsetfillcolor{currentfill}%
\pgfsetlinewidth{0.000000pt}%
\definecolor{currentstroke}{rgb}{1.000000,0.894118,0.788235}%
\pgfsetstrokecolor{currentstroke}%
\pgfsetdash{}{0pt}%
\pgfpathmoveto{\pgfqpoint{3.913402in}{2.849846in}}%
\pgfpathlineto{\pgfqpoint{3.886454in}{2.865404in}}%
\pgfpathlineto{\pgfqpoint{3.880311in}{2.858581in}}%
\pgfpathlineto{\pgfqpoint{3.907258in}{2.843023in}}%
\pgfpathlineto{\pgfqpoint{3.913402in}{2.849846in}}%
\pgfpathclose%
\pgfusepath{fill}%
\end{pgfscope}%
\begin{pgfscope}%
\pgfpathrectangle{\pgfqpoint{0.765000in}{0.660000in}}{\pgfqpoint{4.620000in}{4.620000in}}%
\pgfusepath{clip}%
\pgfsetbuttcap%
\pgfsetroundjoin%
\definecolor{currentfill}{rgb}{1.000000,0.894118,0.788235}%
\pgfsetfillcolor{currentfill}%
\pgfsetlinewidth{0.000000pt}%
\definecolor{currentstroke}{rgb}{1.000000,0.894118,0.788235}%
\pgfsetstrokecolor{currentstroke}%
\pgfsetdash{}{0pt}%
\pgfpathmoveto{\pgfqpoint{3.886454in}{2.865404in}}%
\pgfpathlineto{\pgfqpoint{3.886454in}{2.896520in}}%
\pgfpathlineto{\pgfqpoint{3.880311in}{2.889697in}}%
\pgfpathlineto{\pgfqpoint{3.907258in}{2.843023in}}%
\pgfpathlineto{\pgfqpoint{3.886454in}{2.865404in}}%
\pgfpathclose%
\pgfusepath{fill}%
\end{pgfscope}%
\begin{pgfscope}%
\pgfpathrectangle{\pgfqpoint{0.765000in}{0.660000in}}{\pgfqpoint{4.620000in}{4.620000in}}%
\pgfusepath{clip}%
\pgfsetbuttcap%
\pgfsetroundjoin%
\definecolor{currentfill}{rgb}{1.000000,0.894118,0.788235}%
\pgfsetfillcolor{currentfill}%
\pgfsetlinewidth{0.000000pt}%
\definecolor{currentstroke}{rgb}{1.000000,0.894118,0.788235}%
\pgfsetstrokecolor{currentstroke}%
\pgfsetdash{}{0pt}%
\pgfpathmoveto{\pgfqpoint{3.913402in}{2.849846in}}%
\pgfpathlineto{\pgfqpoint{3.913402in}{2.880962in}}%
\pgfpathlineto{\pgfqpoint{3.907258in}{2.874139in}}%
\pgfpathlineto{\pgfqpoint{3.880311in}{2.858581in}}%
\pgfpathlineto{\pgfqpoint{3.913402in}{2.849846in}}%
\pgfpathclose%
\pgfusepath{fill}%
\end{pgfscope}%
\begin{pgfscope}%
\pgfpathrectangle{\pgfqpoint{0.765000in}{0.660000in}}{\pgfqpoint{4.620000in}{4.620000in}}%
\pgfusepath{clip}%
\pgfsetbuttcap%
\pgfsetroundjoin%
\definecolor{currentfill}{rgb}{1.000000,0.894118,0.788235}%
\pgfsetfillcolor{currentfill}%
\pgfsetlinewidth{0.000000pt}%
\definecolor{currentstroke}{rgb}{1.000000,0.894118,0.788235}%
\pgfsetstrokecolor{currentstroke}%
\pgfsetdash{}{0pt}%
\pgfpathmoveto{\pgfqpoint{3.859507in}{2.849846in}}%
\pgfpathlineto{\pgfqpoint{3.886454in}{2.834288in}}%
\pgfpathlineto{\pgfqpoint{3.880311in}{2.827465in}}%
\pgfpathlineto{\pgfqpoint{3.853364in}{2.843023in}}%
\pgfpathlineto{\pgfqpoint{3.859507in}{2.849846in}}%
\pgfpathclose%
\pgfusepath{fill}%
\end{pgfscope}%
\begin{pgfscope}%
\pgfpathrectangle{\pgfqpoint{0.765000in}{0.660000in}}{\pgfqpoint{4.620000in}{4.620000in}}%
\pgfusepath{clip}%
\pgfsetbuttcap%
\pgfsetroundjoin%
\definecolor{currentfill}{rgb}{1.000000,0.894118,0.788235}%
\pgfsetfillcolor{currentfill}%
\pgfsetlinewidth{0.000000pt}%
\definecolor{currentstroke}{rgb}{1.000000,0.894118,0.788235}%
\pgfsetstrokecolor{currentstroke}%
\pgfsetdash{}{0pt}%
\pgfpathmoveto{\pgfqpoint{3.886454in}{2.834288in}}%
\pgfpathlineto{\pgfqpoint{3.913402in}{2.849846in}}%
\pgfpathlineto{\pgfqpoint{3.907258in}{2.843023in}}%
\pgfpathlineto{\pgfqpoint{3.880311in}{2.827465in}}%
\pgfpathlineto{\pgfqpoint{3.886454in}{2.834288in}}%
\pgfpathclose%
\pgfusepath{fill}%
\end{pgfscope}%
\begin{pgfscope}%
\pgfpathrectangle{\pgfqpoint{0.765000in}{0.660000in}}{\pgfqpoint{4.620000in}{4.620000in}}%
\pgfusepath{clip}%
\pgfsetbuttcap%
\pgfsetroundjoin%
\definecolor{currentfill}{rgb}{1.000000,0.894118,0.788235}%
\pgfsetfillcolor{currentfill}%
\pgfsetlinewidth{0.000000pt}%
\definecolor{currentstroke}{rgb}{1.000000,0.894118,0.788235}%
\pgfsetstrokecolor{currentstroke}%
\pgfsetdash{}{0pt}%
\pgfpathmoveto{\pgfqpoint{3.886454in}{2.896520in}}%
\pgfpathlineto{\pgfqpoint{3.859507in}{2.880962in}}%
\pgfpathlineto{\pgfqpoint{3.853364in}{2.874139in}}%
\pgfpathlineto{\pgfqpoint{3.880311in}{2.889697in}}%
\pgfpathlineto{\pgfqpoint{3.886454in}{2.896520in}}%
\pgfpathclose%
\pgfusepath{fill}%
\end{pgfscope}%
\begin{pgfscope}%
\pgfpathrectangle{\pgfqpoint{0.765000in}{0.660000in}}{\pgfqpoint{4.620000in}{4.620000in}}%
\pgfusepath{clip}%
\pgfsetbuttcap%
\pgfsetroundjoin%
\definecolor{currentfill}{rgb}{1.000000,0.894118,0.788235}%
\pgfsetfillcolor{currentfill}%
\pgfsetlinewidth{0.000000pt}%
\definecolor{currentstroke}{rgb}{1.000000,0.894118,0.788235}%
\pgfsetstrokecolor{currentstroke}%
\pgfsetdash{}{0pt}%
\pgfpathmoveto{\pgfqpoint{3.913402in}{2.880962in}}%
\pgfpathlineto{\pgfqpoint{3.886454in}{2.896520in}}%
\pgfpathlineto{\pgfqpoint{3.880311in}{2.889697in}}%
\pgfpathlineto{\pgfqpoint{3.907258in}{2.874139in}}%
\pgfpathlineto{\pgfqpoint{3.913402in}{2.880962in}}%
\pgfpathclose%
\pgfusepath{fill}%
\end{pgfscope}%
\begin{pgfscope}%
\pgfpathrectangle{\pgfqpoint{0.765000in}{0.660000in}}{\pgfqpoint{4.620000in}{4.620000in}}%
\pgfusepath{clip}%
\pgfsetbuttcap%
\pgfsetroundjoin%
\definecolor{currentfill}{rgb}{1.000000,0.894118,0.788235}%
\pgfsetfillcolor{currentfill}%
\pgfsetlinewidth{0.000000pt}%
\definecolor{currentstroke}{rgb}{1.000000,0.894118,0.788235}%
\pgfsetstrokecolor{currentstroke}%
\pgfsetdash{}{0pt}%
\pgfpathmoveto{\pgfqpoint{3.859507in}{2.849846in}}%
\pgfpathlineto{\pgfqpoint{3.859507in}{2.880962in}}%
\pgfpathlineto{\pgfqpoint{3.853364in}{2.874139in}}%
\pgfpathlineto{\pgfqpoint{3.880311in}{2.827465in}}%
\pgfpathlineto{\pgfqpoint{3.859507in}{2.849846in}}%
\pgfpathclose%
\pgfusepath{fill}%
\end{pgfscope}%
\begin{pgfscope}%
\pgfpathrectangle{\pgfqpoint{0.765000in}{0.660000in}}{\pgfqpoint{4.620000in}{4.620000in}}%
\pgfusepath{clip}%
\pgfsetbuttcap%
\pgfsetroundjoin%
\definecolor{currentfill}{rgb}{1.000000,0.894118,0.788235}%
\pgfsetfillcolor{currentfill}%
\pgfsetlinewidth{0.000000pt}%
\definecolor{currentstroke}{rgb}{1.000000,0.894118,0.788235}%
\pgfsetstrokecolor{currentstroke}%
\pgfsetdash{}{0pt}%
\pgfpathmoveto{\pgfqpoint{3.886454in}{2.834288in}}%
\pgfpathlineto{\pgfqpoint{3.886454in}{2.865404in}}%
\pgfpathlineto{\pgfqpoint{3.880311in}{2.858581in}}%
\pgfpathlineto{\pgfqpoint{3.853364in}{2.843023in}}%
\pgfpathlineto{\pgfqpoint{3.886454in}{2.834288in}}%
\pgfpathclose%
\pgfusepath{fill}%
\end{pgfscope}%
\begin{pgfscope}%
\pgfpathrectangle{\pgfqpoint{0.765000in}{0.660000in}}{\pgfqpoint{4.620000in}{4.620000in}}%
\pgfusepath{clip}%
\pgfsetbuttcap%
\pgfsetroundjoin%
\definecolor{currentfill}{rgb}{1.000000,0.894118,0.788235}%
\pgfsetfillcolor{currentfill}%
\pgfsetlinewidth{0.000000pt}%
\definecolor{currentstroke}{rgb}{1.000000,0.894118,0.788235}%
\pgfsetstrokecolor{currentstroke}%
\pgfsetdash{}{0pt}%
\pgfpathmoveto{\pgfqpoint{3.886454in}{2.865404in}}%
\pgfpathlineto{\pgfqpoint{3.913402in}{2.880962in}}%
\pgfpathlineto{\pgfqpoint{3.907258in}{2.874139in}}%
\pgfpathlineto{\pgfqpoint{3.880311in}{2.858581in}}%
\pgfpathlineto{\pgfqpoint{3.886454in}{2.865404in}}%
\pgfpathclose%
\pgfusepath{fill}%
\end{pgfscope}%
\begin{pgfscope}%
\pgfpathrectangle{\pgfqpoint{0.765000in}{0.660000in}}{\pgfqpoint{4.620000in}{4.620000in}}%
\pgfusepath{clip}%
\pgfsetbuttcap%
\pgfsetroundjoin%
\definecolor{currentfill}{rgb}{1.000000,0.894118,0.788235}%
\pgfsetfillcolor{currentfill}%
\pgfsetlinewidth{0.000000pt}%
\definecolor{currentstroke}{rgb}{1.000000,0.894118,0.788235}%
\pgfsetstrokecolor{currentstroke}%
\pgfsetdash{}{0pt}%
\pgfpathmoveto{\pgfqpoint{3.859507in}{2.880962in}}%
\pgfpathlineto{\pgfqpoint{3.886454in}{2.865404in}}%
\pgfpathlineto{\pgfqpoint{3.880311in}{2.858581in}}%
\pgfpathlineto{\pgfqpoint{3.853364in}{2.874139in}}%
\pgfpathlineto{\pgfqpoint{3.859507in}{2.880962in}}%
\pgfpathclose%
\pgfusepath{fill}%
\end{pgfscope}%
\begin{pgfscope}%
\pgfpathrectangle{\pgfqpoint{0.765000in}{0.660000in}}{\pgfqpoint{4.620000in}{4.620000in}}%
\pgfusepath{clip}%
\pgfsetbuttcap%
\pgfsetroundjoin%
\definecolor{currentfill}{rgb}{1.000000,0.894118,0.788235}%
\pgfsetfillcolor{currentfill}%
\pgfsetlinewidth{0.000000pt}%
\definecolor{currentstroke}{rgb}{1.000000,0.894118,0.788235}%
\pgfsetstrokecolor{currentstroke}%
\pgfsetdash{}{0pt}%
\pgfpathmoveto{\pgfqpoint{2.981238in}{3.646599in}}%
\pgfpathlineto{\pgfqpoint{2.954291in}{3.631041in}}%
\pgfpathlineto{\pgfqpoint{2.981238in}{3.615483in}}%
\pgfpathlineto{\pgfqpoint{3.008185in}{3.631041in}}%
\pgfpathlineto{\pgfqpoint{2.981238in}{3.646599in}}%
\pgfpathclose%
\pgfusepath{fill}%
\end{pgfscope}%
\begin{pgfscope}%
\pgfpathrectangle{\pgfqpoint{0.765000in}{0.660000in}}{\pgfqpoint{4.620000in}{4.620000in}}%
\pgfusepath{clip}%
\pgfsetbuttcap%
\pgfsetroundjoin%
\definecolor{currentfill}{rgb}{1.000000,0.894118,0.788235}%
\pgfsetfillcolor{currentfill}%
\pgfsetlinewidth{0.000000pt}%
\definecolor{currentstroke}{rgb}{1.000000,0.894118,0.788235}%
\pgfsetstrokecolor{currentstroke}%
\pgfsetdash{}{0pt}%
\pgfpathmoveto{\pgfqpoint{2.981238in}{3.646599in}}%
\pgfpathlineto{\pgfqpoint{2.954291in}{3.631041in}}%
\pgfpathlineto{\pgfqpoint{2.954291in}{3.662157in}}%
\pgfpathlineto{\pgfqpoint{2.981238in}{3.677715in}}%
\pgfpathlineto{\pgfqpoint{2.981238in}{3.646599in}}%
\pgfpathclose%
\pgfusepath{fill}%
\end{pgfscope}%
\begin{pgfscope}%
\pgfpathrectangle{\pgfqpoint{0.765000in}{0.660000in}}{\pgfqpoint{4.620000in}{4.620000in}}%
\pgfusepath{clip}%
\pgfsetbuttcap%
\pgfsetroundjoin%
\definecolor{currentfill}{rgb}{1.000000,0.894118,0.788235}%
\pgfsetfillcolor{currentfill}%
\pgfsetlinewidth{0.000000pt}%
\definecolor{currentstroke}{rgb}{1.000000,0.894118,0.788235}%
\pgfsetstrokecolor{currentstroke}%
\pgfsetdash{}{0pt}%
\pgfpathmoveto{\pgfqpoint{2.981238in}{3.646599in}}%
\pgfpathlineto{\pgfqpoint{3.008185in}{3.631041in}}%
\pgfpathlineto{\pgfqpoint{3.008185in}{3.662157in}}%
\pgfpathlineto{\pgfqpoint{2.981238in}{3.677715in}}%
\pgfpathlineto{\pgfqpoint{2.981238in}{3.646599in}}%
\pgfpathclose%
\pgfusepath{fill}%
\end{pgfscope}%
\begin{pgfscope}%
\pgfpathrectangle{\pgfqpoint{0.765000in}{0.660000in}}{\pgfqpoint{4.620000in}{4.620000in}}%
\pgfusepath{clip}%
\pgfsetbuttcap%
\pgfsetroundjoin%
\definecolor{currentfill}{rgb}{1.000000,0.894118,0.788235}%
\pgfsetfillcolor{currentfill}%
\pgfsetlinewidth{0.000000pt}%
\definecolor{currentstroke}{rgb}{1.000000,0.894118,0.788235}%
\pgfsetstrokecolor{currentstroke}%
\pgfsetdash{}{0pt}%
\pgfpathmoveto{\pgfqpoint{2.973794in}{3.546621in}}%
\pgfpathlineto{\pgfqpoint{2.946847in}{3.531063in}}%
\pgfpathlineto{\pgfqpoint{2.973794in}{3.515505in}}%
\pgfpathlineto{\pgfqpoint{3.000742in}{3.531063in}}%
\pgfpathlineto{\pgfqpoint{2.973794in}{3.546621in}}%
\pgfpathclose%
\pgfusepath{fill}%
\end{pgfscope}%
\begin{pgfscope}%
\pgfpathrectangle{\pgfqpoint{0.765000in}{0.660000in}}{\pgfqpoint{4.620000in}{4.620000in}}%
\pgfusepath{clip}%
\pgfsetbuttcap%
\pgfsetroundjoin%
\definecolor{currentfill}{rgb}{1.000000,0.894118,0.788235}%
\pgfsetfillcolor{currentfill}%
\pgfsetlinewidth{0.000000pt}%
\definecolor{currentstroke}{rgb}{1.000000,0.894118,0.788235}%
\pgfsetstrokecolor{currentstroke}%
\pgfsetdash{}{0pt}%
\pgfpathmoveto{\pgfqpoint{2.973794in}{3.546621in}}%
\pgfpathlineto{\pgfqpoint{2.946847in}{3.531063in}}%
\pgfpathlineto{\pgfqpoint{2.946847in}{3.562179in}}%
\pgfpathlineto{\pgfqpoint{2.973794in}{3.577737in}}%
\pgfpathlineto{\pgfqpoint{2.973794in}{3.546621in}}%
\pgfpathclose%
\pgfusepath{fill}%
\end{pgfscope}%
\begin{pgfscope}%
\pgfpathrectangle{\pgfqpoint{0.765000in}{0.660000in}}{\pgfqpoint{4.620000in}{4.620000in}}%
\pgfusepath{clip}%
\pgfsetbuttcap%
\pgfsetroundjoin%
\definecolor{currentfill}{rgb}{1.000000,0.894118,0.788235}%
\pgfsetfillcolor{currentfill}%
\pgfsetlinewidth{0.000000pt}%
\definecolor{currentstroke}{rgb}{1.000000,0.894118,0.788235}%
\pgfsetstrokecolor{currentstroke}%
\pgfsetdash{}{0pt}%
\pgfpathmoveto{\pgfqpoint{2.973794in}{3.546621in}}%
\pgfpathlineto{\pgfqpoint{3.000742in}{3.531063in}}%
\pgfpathlineto{\pgfqpoint{3.000742in}{3.562179in}}%
\pgfpathlineto{\pgfqpoint{2.973794in}{3.577737in}}%
\pgfpathlineto{\pgfqpoint{2.973794in}{3.546621in}}%
\pgfpathclose%
\pgfusepath{fill}%
\end{pgfscope}%
\begin{pgfscope}%
\pgfpathrectangle{\pgfqpoint{0.765000in}{0.660000in}}{\pgfqpoint{4.620000in}{4.620000in}}%
\pgfusepath{clip}%
\pgfsetbuttcap%
\pgfsetroundjoin%
\definecolor{currentfill}{rgb}{1.000000,0.894118,0.788235}%
\pgfsetfillcolor{currentfill}%
\pgfsetlinewidth{0.000000pt}%
\definecolor{currentstroke}{rgb}{1.000000,0.894118,0.788235}%
\pgfsetstrokecolor{currentstroke}%
\pgfsetdash{}{0pt}%
\pgfpathmoveto{\pgfqpoint{2.981238in}{3.677715in}}%
\pgfpathlineto{\pgfqpoint{2.954291in}{3.662157in}}%
\pgfpathlineto{\pgfqpoint{2.981238in}{3.646599in}}%
\pgfpathlineto{\pgfqpoint{3.008185in}{3.662157in}}%
\pgfpathlineto{\pgfqpoint{2.981238in}{3.677715in}}%
\pgfpathclose%
\pgfusepath{fill}%
\end{pgfscope}%
\begin{pgfscope}%
\pgfpathrectangle{\pgfqpoint{0.765000in}{0.660000in}}{\pgfqpoint{4.620000in}{4.620000in}}%
\pgfusepath{clip}%
\pgfsetbuttcap%
\pgfsetroundjoin%
\definecolor{currentfill}{rgb}{1.000000,0.894118,0.788235}%
\pgfsetfillcolor{currentfill}%
\pgfsetlinewidth{0.000000pt}%
\definecolor{currentstroke}{rgb}{1.000000,0.894118,0.788235}%
\pgfsetstrokecolor{currentstroke}%
\pgfsetdash{}{0pt}%
\pgfpathmoveto{\pgfqpoint{2.981238in}{3.615483in}}%
\pgfpathlineto{\pgfqpoint{3.008185in}{3.631041in}}%
\pgfpathlineto{\pgfqpoint{3.008185in}{3.662157in}}%
\pgfpathlineto{\pgfqpoint{2.981238in}{3.646599in}}%
\pgfpathlineto{\pgfqpoint{2.981238in}{3.615483in}}%
\pgfpathclose%
\pgfusepath{fill}%
\end{pgfscope}%
\begin{pgfscope}%
\pgfpathrectangle{\pgfqpoint{0.765000in}{0.660000in}}{\pgfqpoint{4.620000in}{4.620000in}}%
\pgfusepath{clip}%
\pgfsetbuttcap%
\pgfsetroundjoin%
\definecolor{currentfill}{rgb}{1.000000,0.894118,0.788235}%
\pgfsetfillcolor{currentfill}%
\pgfsetlinewidth{0.000000pt}%
\definecolor{currentstroke}{rgb}{1.000000,0.894118,0.788235}%
\pgfsetstrokecolor{currentstroke}%
\pgfsetdash{}{0pt}%
\pgfpathmoveto{\pgfqpoint{2.954291in}{3.631041in}}%
\pgfpathlineto{\pgfqpoint{2.981238in}{3.615483in}}%
\pgfpathlineto{\pgfqpoint{2.981238in}{3.646599in}}%
\pgfpathlineto{\pgfqpoint{2.954291in}{3.662157in}}%
\pgfpathlineto{\pgfqpoint{2.954291in}{3.631041in}}%
\pgfpathclose%
\pgfusepath{fill}%
\end{pgfscope}%
\begin{pgfscope}%
\pgfpathrectangle{\pgfqpoint{0.765000in}{0.660000in}}{\pgfqpoint{4.620000in}{4.620000in}}%
\pgfusepath{clip}%
\pgfsetbuttcap%
\pgfsetroundjoin%
\definecolor{currentfill}{rgb}{1.000000,0.894118,0.788235}%
\pgfsetfillcolor{currentfill}%
\pgfsetlinewidth{0.000000pt}%
\definecolor{currentstroke}{rgb}{1.000000,0.894118,0.788235}%
\pgfsetstrokecolor{currentstroke}%
\pgfsetdash{}{0pt}%
\pgfpathmoveto{\pgfqpoint{2.973794in}{3.577737in}}%
\pgfpathlineto{\pgfqpoint{2.946847in}{3.562179in}}%
\pgfpathlineto{\pgfqpoint{2.973794in}{3.546621in}}%
\pgfpathlineto{\pgfqpoint{3.000742in}{3.562179in}}%
\pgfpathlineto{\pgfqpoint{2.973794in}{3.577737in}}%
\pgfpathclose%
\pgfusepath{fill}%
\end{pgfscope}%
\begin{pgfscope}%
\pgfpathrectangle{\pgfqpoint{0.765000in}{0.660000in}}{\pgfqpoint{4.620000in}{4.620000in}}%
\pgfusepath{clip}%
\pgfsetbuttcap%
\pgfsetroundjoin%
\definecolor{currentfill}{rgb}{1.000000,0.894118,0.788235}%
\pgfsetfillcolor{currentfill}%
\pgfsetlinewidth{0.000000pt}%
\definecolor{currentstroke}{rgb}{1.000000,0.894118,0.788235}%
\pgfsetstrokecolor{currentstroke}%
\pgfsetdash{}{0pt}%
\pgfpathmoveto{\pgfqpoint{2.973794in}{3.515505in}}%
\pgfpathlineto{\pgfqpoint{3.000742in}{3.531063in}}%
\pgfpathlineto{\pgfqpoint{3.000742in}{3.562179in}}%
\pgfpathlineto{\pgfqpoint{2.973794in}{3.546621in}}%
\pgfpathlineto{\pgfqpoint{2.973794in}{3.515505in}}%
\pgfpathclose%
\pgfusepath{fill}%
\end{pgfscope}%
\begin{pgfscope}%
\pgfpathrectangle{\pgfqpoint{0.765000in}{0.660000in}}{\pgfqpoint{4.620000in}{4.620000in}}%
\pgfusepath{clip}%
\pgfsetbuttcap%
\pgfsetroundjoin%
\definecolor{currentfill}{rgb}{1.000000,0.894118,0.788235}%
\pgfsetfillcolor{currentfill}%
\pgfsetlinewidth{0.000000pt}%
\definecolor{currentstroke}{rgb}{1.000000,0.894118,0.788235}%
\pgfsetstrokecolor{currentstroke}%
\pgfsetdash{}{0pt}%
\pgfpathmoveto{\pgfqpoint{2.946847in}{3.531063in}}%
\pgfpathlineto{\pgfqpoint{2.973794in}{3.515505in}}%
\pgfpathlineto{\pgfqpoint{2.973794in}{3.546621in}}%
\pgfpathlineto{\pgfqpoint{2.946847in}{3.562179in}}%
\pgfpathlineto{\pgfqpoint{2.946847in}{3.531063in}}%
\pgfpathclose%
\pgfusepath{fill}%
\end{pgfscope}%
\begin{pgfscope}%
\pgfpathrectangle{\pgfqpoint{0.765000in}{0.660000in}}{\pgfqpoint{4.620000in}{4.620000in}}%
\pgfusepath{clip}%
\pgfsetbuttcap%
\pgfsetroundjoin%
\definecolor{currentfill}{rgb}{1.000000,0.894118,0.788235}%
\pgfsetfillcolor{currentfill}%
\pgfsetlinewidth{0.000000pt}%
\definecolor{currentstroke}{rgb}{1.000000,0.894118,0.788235}%
\pgfsetstrokecolor{currentstroke}%
\pgfsetdash{}{0pt}%
\pgfpathmoveto{\pgfqpoint{2.981238in}{3.646599in}}%
\pgfpathlineto{\pgfqpoint{2.954291in}{3.631041in}}%
\pgfpathlineto{\pgfqpoint{2.946847in}{3.531063in}}%
\pgfpathlineto{\pgfqpoint{2.973794in}{3.546621in}}%
\pgfpathlineto{\pgfqpoint{2.981238in}{3.646599in}}%
\pgfpathclose%
\pgfusepath{fill}%
\end{pgfscope}%
\begin{pgfscope}%
\pgfpathrectangle{\pgfqpoint{0.765000in}{0.660000in}}{\pgfqpoint{4.620000in}{4.620000in}}%
\pgfusepath{clip}%
\pgfsetbuttcap%
\pgfsetroundjoin%
\definecolor{currentfill}{rgb}{1.000000,0.894118,0.788235}%
\pgfsetfillcolor{currentfill}%
\pgfsetlinewidth{0.000000pt}%
\definecolor{currentstroke}{rgb}{1.000000,0.894118,0.788235}%
\pgfsetstrokecolor{currentstroke}%
\pgfsetdash{}{0pt}%
\pgfpathmoveto{\pgfqpoint{3.008185in}{3.631041in}}%
\pgfpathlineto{\pgfqpoint{2.981238in}{3.646599in}}%
\pgfpathlineto{\pgfqpoint{2.973794in}{3.546621in}}%
\pgfpathlineto{\pgfqpoint{3.000742in}{3.531063in}}%
\pgfpathlineto{\pgfqpoint{3.008185in}{3.631041in}}%
\pgfpathclose%
\pgfusepath{fill}%
\end{pgfscope}%
\begin{pgfscope}%
\pgfpathrectangle{\pgfqpoint{0.765000in}{0.660000in}}{\pgfqpoint{4.620000in}{4.620000in}}%
\pgfusepath{clip}%
\pgfsetbuttcap%
\pgfsetroundjoin%
\definecolor{currentfill}{rgb}{1.000000,0.894118,0.788235}%
\pgfsetfillcolor{currentfill}%
\pgfsetlinewidth{0.000000pt}%
\definecolor{currentstroke}{rgb}{1.000000,0.894118,0.788235}%
\pgfsetstrokecolor{currentstroke}%
\pgfsetdash{}{0pt}%
\pgfpathmoveto{\pgfqpoint{2.981238in}{3.646599in}}%
\pgfpathlineto{\pgfqpoint{2.981238in}{3.677715in}}%
\pgfpathlineto{\pgfqpoint{2.973794in}{3.577737in}}%
\pgfpathlineto{\pgfqpoint{3.000742in}{3.531063in}}%
\pgfpathlineto{\pgfqpoint{2.981238in}{3.646599in}}%
\pgfpathclose%
\pgfusepath{fill}%
\end{pgfscope}%
\begin{pgfscope}%
\pgfpathrectangle{\pgfqpoint{0.765000in}{0.660000in}}{\pgfqpoint{4.620000in}{4.620000in}}%
\pgfusepath{clip}%
\pgfsetbuttcap%
\pgfsetroundjoin%
\definecolor{currentfill}{rgb}{1.000000,0.894118,0.788235}%
\pgfsetfillcolor{currentfill}%
\pgfsetlinewidth{0.000000pt}%
\definecolor{currentstroke}{rgb}{1.000000,0.894118,0.788235}%
\pgfsetstrokecolor{currentstroke}%
\pgfsetdash{}{0pt}%
\pgfpathmoveto{\pgfqpoint{3.008185in}{3.631041in}}%
\pgfpathlineto{\pgfqpoint{3.008185in}{3.662157in}}%
\pgfpathlineto{\pgfqpoint{3.000742in}{3.562179in}}%
\pgfpathlineto{\pgfqpoint{2.973794in}{3.546621in}}%
\pgfpathlineto{\pgfqpoint{3.008185in}{3.631041in}}%
\pgfpathclose%
\pgfusepath{fill}%
\end{pgfscope}%
\begin{pgfscope}%
\pgfpathrectangle{\pgfqpoint{0.765000in}{0.660000in}}{\pgfqpoint{4.620000in}{4.620000in}}%
\pgfusepath{clip}%
\pgfsetbuttcap%
\pgfsetroundjoin%
\definecolor{currentfill}{rgb}{1.000000,0.894118,0.788235}%
\pgfsetfillcolor{currentfill}%
\pgfsetlinewidth{0.000000pt}%
\definecolor{currentstroke}{rgb}{1.000000,0.894118,0.788235}%
\pgfsetstrokecolor{currentstroke}%
\pgfsetdash{}{0pt}%
\pgfpathmoveto{\pgfqpoint{2.954291in}{3.631041in}}%
\pgfpathlineto{\pgfqpoint{2.981238in}{3.615483in}}%
\pgfpathlineto{\pgfqpoint{2.973794in}{3.515505in}}%
\pgfpathlineto{\pgfqpoint{2.946847in}{3.531063in}}%
\pgfpathlineto{\pgfqpoint{2.954291in}{3.631041in}}%
\pgfpathclose%
\pgfusepath{fill}%
\end{pgfscope}%
\begin{pgfscope}%
\pgfpathrectangle{\pgfqpoint{0.765000in}{0.660000in}}{\pgfqpoint{4.620000in}{4.620000in}}%
\pgfusepath{clip}%
\pgfsetbuttcap%
\pgfsetroundjoin%
\definecolor{currentfill}{rgb}{1.000000,0.894118,0.788235}%
\pgfsetfillcolor{currentfill}%
\pgfsetlinewidth{0.000000pt}%
\definecolor{currentstroke}{rgb}{1.000000,0.894118,0.788235}%
\pgfsetstrokecolor{currentstroke}%
\pgfsetdash{}{0pt}%
\pgfpathmoveto{\pgfqpoint{2.981238in}{3.615483in}}%
\pgfpathlineto{\pgfqpoint{3.008185in}{3.631041in}}%
\pgfpathlineto{\pgfqpoint{3.000742in}{3.531063in}}%
\pgfpathlineto{\pgfqpoint{2.973794in}{3.515505in}}%
\pgfpathlineto{\pgfqpoint{2.981238in}{3.615483in}}%
\pgfpathclose%
\pgfusepath{fill}%
\end{pgfscope}%
\begin{pgfscope}%
\pgfpathrectangle{\pgfqpoint{0.765000in}{0.660000in}}{\pgfqpoint{4.620000in}{4.620000in}}%
\pgfusepath{clip}%
\pgfsetbuttcap%
\pgfsetroundjoin%
\definecolor{currentfill}{rgb}{1.000000,0.894118,0.788235}%
\pgfsetfillcolor{currentfill}%
\pgfsetlinewidth{0.000000pt}%
\definecolor{currentstroke}{rgb}{1.000000,0.894118,0.788235}%
\pgfsetstrokecolor{currentstroke}%
\pgfsetdash{}{0pt}%
\pgfpathmoveto{\pgfqpoint{2.981238in}{3.677715in}}%
\pgfpathlineto{\pgfqpoint{2.954291in}{3.662157in}}%
\pgfpathlineto{\pgfqpoint{2.946847in}{3.562179in}}%
\pgfpathlineto{\pgfqpoint{2.973794in}{3.577737in}}%
\pgfpathlineto{\pgfqpoint{2.981238in}{3.677715in}}%
\pgfpathclose%
\pgfusepath{fill}%
\end{pgfscope}%
\begin{pgfscope}%
\pgfpathrectangle{\pgfqpoint{0.765000in}{0.660000in}}{\pgfqpoint{4.620000in}{4.620000in}}%
\pgfusepath{clip}%
\pgfsetbuttcap%
\pgfsetroundjoin%
\definecolor{currentfill}{rgb}{1.000000,0.894118,0.788235}%
\pgfsetfillcolor{currentfill}%
\pgfsetlinewidth{0.000000pt}%
\definecolor{currentstroke}{rgb}{1.000000,0.894118,0.788235}%
\pgfsetstrokecolor{currentstroke}%
\pgfsetdash{}{0pt}%
\pgfpathmoveto{\pgfqpoint{3.008185in}{3.662157in}}%
\pgfpathlineto{\pgfqpoint{2.981238in}{3.677715in}}%
\pgfpathlineto{\pgfqpoint{2.973794in}{3.577737in}}%
\pgfpathlineto{\pgfqpoint{3.000742in}{3.562179in}}%
\pgfpathlineto{\pgfqpoint{3.008185in}{3.662157in}}%
\pgfpathclose%
\pgfusepath{fill}%
\end{pgfscope}%
\begin{pgfscope}%
\pgfpathrectangle{\pgfqpoint{0.765000in}{0.660000in}}{\pgfqpoint{4.620000in}{4.620000in}}%
\pgfusepath{clip}%
\pgfsetbuttcap%
\pgfsetroundjoin%
\definecolor{currentfill}{rgb}{1.000000,0.894118,0.788235}%
\pgfsetfillcolor{currentfill}%
\pgfsetlinewidth{0.000000pt}%
\definecolor{currentstroke}{rgb}{1.000000,0.894118,0.788235}%
\pgfsetstrokecolor{currentstroke}%
\pgfsetdash{}{0pt}%
\pgfpathmoveto{\pgfqpoint{2.954291in}{3.631041in}}%
\pgfpathlineto{\pgfqpoint{2.954291in}{3.662157in}}%
\pgfpathlineto{\pgfqpoint{2.946847in}{3.562179in}}%
\pgfpathlineto{\pgfqpoint{2.973794in}{3.515505in}}%
\pgfpathlineto{\pgfqpoint{2.954291in}{3.631041in}}%
\pgfpathclose%
\pgfusepath{fill}%
\end{pgfscope}%
\begin{pgfscope}%
\pgfpathrectangle{\pgfqpoint{0.765000in}{0.660000in}}{\pgfqpoint{4.620000in}{4.620000in}}%
\pgfusepath{clip}%
\pgfsetbuttcap%
\pgfsetroundjoin%
\definecolor{currentfill}{rgb}{1.000000,0.894118,0.788235}%
\pgfsetfillcolor{currentfill}%
\pgfsetlinewidth{0.000000pt}%
\definecolor{currentstroke}{rgb}{1.000000,0.894118,0.788235}%
\pgfsetstrokecolor{currentstroke}%
\pgfsetdash{}{0pt}%
\pgfpathmoveto{\pgfqpoint{2.981238in}{3.615483in}}%
\pgfpathlineto{\pgfqpoint{2.981238in}{3.646599in}}%
\pgfpathlineto{\pgfqpoint{2.973794in}{3.546621in}}%
\pgfpathlineto{\pgfqpoint{2.946847in}{3.531063in}}%
\pgfpathlineto{\pgfqpoint{2.981238in}{3.615483in}}%
\pgfpathclose%
\pgfusepath{fill}%
\end{pgfscope}%
\begin{pgfscope}%
\pgfpathrectangle{\pgfqpoint{0.765000in}{0.660000in}}{\pgfqpoint{4.620000in}{4.620000in}}%
\pgfusepath{clip}%
\pgfsetbuttcap%
\pgfsetroundjoin%
\definecolor{currentfill}{rgb}{1.000000,0.894118,0.788235}%
\pgfsetfillcolor{currentfill}%
\pgfsetlinewidth{0.000000pt}%
\definecolor{currentstroke}{rgb}{1.000000,0.894118,0.788235}%
\pgfsetstrokecolor{currentstroke}%
\pgfsetdash{}{0pt}%
\pgfpathmoveto{\pgfqpoint{2.981238in}{3.646599in}}%
\pgfpathlineto{\pgfqpoint{3.008185in}{3.662157in}}%
\pgfpathlineto{\pgfqpoint{3.000742in}{3.562179in}}%
\pgfpathlineto{\pgfqpoint{2.973794in}{3.546621in}}%
\pgfpathlineto{\pgfqpoint{2.981238in}{3.646599in}}%
\pgfpathclose%
\pgfusepath{fill}%
\end{pgfscope}%
\begin{pgfscope}%
\pgfpathrectangle{\pgfqpoint{0.765000in}{0.660000in}}{\pgfqpoint{4.620000in}{4.620000in}}%
\pgfusepath{clip}%
\pgfsetbuttcap%
\pgfsetroundjoin%
\definecolor{currentfill}{rgb}{1.000000,0.894118,0.788235}%
\pgfsetfillcolor{currentfill}%
\pgfsetlinewidth{0.000000pt}%
\definecolor{currentstroke}{rgb}{1.000000,0.894118,0.788235}%
\pgfsetstrokecolor{currentstroke}%
\pgfsetdash{}{0pt}%
\pgfpathmoveto{\pgfqpoint{2.954291in}{3.662157in}}%
\pgfpathlineto{\pgfqpoint{2.981238in}{3.646599in}}%
\pgfpathlineto{\pgfqpoint{2.973794in}{3.546621in}}%
\pgfpathlineto{\pgfqpoint{2.946847in}{3.562179in}}%
\pgfpathlineto{\pgfqpoint{2.954291in}{3.662157in}}%
\pgfpathclose%
\pgfusepath{fill}%
\end{pgfscope}%
\begin{pgfscope}%
\pgfpathrectangle{\pgfqpoint{0.765000in}{0.660000in}}{\pgfqpoint{4.620000in}{4.620000in}}%
\pgfusepath{clip}%
\pgfsetbuttcap%
\pgfsetroundjoin%
\definecolor{currentfill}{rgb}{1.000000,0.894118,0.788235}%
\pgfsetfillcolor{currentfill}%
\pgfsetlinewidth{0.000000pt}%
\definecolor{currentstroke}{rgb}{1.000000,0.894118,0.788235}%
\pgfsetstrokecolor{currentstroke}%
\pgfsetdash{}{0pt}%
\pgfpathmoveto{\pgfqpoint{2.981238in}{3.646599in}}%
\pgfpathlineto{\pgfqpoint{2.954291in}{3.631041in}}%
\pgfpathlineto{\pgfqpoint{2.981238in}{3.615483in}}%
\pgfpathlineto{\pgfqpoint{3.008185in}{3.631041in}}%
\pgfpathlineto{\pgfqpoint{2.981238in}{3.646599in}}%
\pgfpathclose%
\pgfusepath{fill}%
\end{pgfscope}%
\begin{pgfscope}%
\pgfpathrectangle{\pgfqpoint{0.765000in}{0.660000in}}{\pgfqpoint{4.620000in}{4.620000in}}%
\pgfusepath{clip}%
\pgfsetbuttcap%
\pgfsetroundjoin%
\definecolor{currentfill}{rgb}{1.000000,0.894118,0.788235}%
\pgfsetfillcolor{currentfill}%
\pgfsetlinewidth{0.000000pt}%
\definecolor{currentstroke}{rgb}{1.000000,0.894118,0.788235}%
\pgfsetstrokecolor{currentstroke}%
\pgfsetdash{}{0pt}%
\pgfpathmoveto{\pgfqpoint{2.981238in}{3.646599in}}%
\pgfpathlineto{\pgfqpoint{2.954291in}{3.631041in}}%
\pgfpathlineto{\pgfqpoint{2.954291in}{3.662157in}}%
\pgfpathlineto{\pgfqpoint{2.981238in}{3.677715in}}%
\pgfpathlineto{\pgfqpoint{2.981238in}{3.646599in}}%
\pgfpathclose%
\pgfusepath{fill}%
\end{pgfscope}%
\begin{pgfscope}%
\pgfpathrectangle{\pgfqpoint{0.765000in}{0.660000in}}{\pgfqpoint{4.620000in}{4.620000in}}%
\pgfusepath{clip}%
\pgfsetbuttcap%
\pgfsetroundjoin%
\definecolor{currentfill}{rgb}{1.000000,0.894118,0.788235}%
\pgfsetfillcolor{currentfill}%
\pgfsetlinewidth{0.000000pt}%
\definecolor{currentstroke}{rgb}{1.000000,0.894118,0.788235}%
\pgfsetstrokecolor{currentstroke}%
\pgfsetdash{}{0pt}%
\pgfpathmoveto{\pgfqpoint{2.981238in}{3.646599in}}%
\pgfpathlineto{\pgfqpoint{3.008185in}{3.631041in}}%
\pgfpathlineto{\pgfqpoint{3.008185in}{3.662157in}}%
\pgfpathlineto{\pgfqpoint{2.981238in}{3.677715in}}%
\pgfpathlineto{\pgfqpoint{2.981238in}{3.646599in}}%
\pgfpathclose%
\pgfusepath{fill}%
\end{pgfscope}%
\begin{pgfscope}%
\pgfpathrectangle{\pgfqpoint{0.765000in}{0.660000in}}{\pgfqpoint{4.620000in}{4.620000in}}%
\pgfusepath{clip}%
\pgfsetbuttcap%
\pgfsetroundjoin%
\definecolor{currentfill}{rgb}{1.000000,0.894118,0.788235}%
\pgfsetfillcolor{currentfill}%
\pgfsetlinewidth{0.000000pt}%
\definecolor{currentstroke}{rgb}{1.000000,0.894118,0.788235}%
\pgfsetstrokecolor{currentstroke}%
\pgfsetdash{}{0pt}%
\pgfpathmoveto{\pgfqpoint{2.981592in}{3.650866in}}%
\pgfpathlineto{\pgfqpoint{2.954645in}{3.635308in}}%
\pgfpathlineto{\pgfqpoint{2.981592in}{3.619750in}}%
\pgfpathlineto{\pgfqpoint{3.008539in}{3.635308in}}%
\pgfpathlineto{\pgfqpoint{2.981592in}{3.650866in}}%
\pgfpathclose%
\pgfusepath{fill}%
\end{pgfscope}%
\begin{pgfscope}%
\pgfpathrectangle{\pgfqpoint{0.765000in}{0.660000in}}{\pgfqpoint{4.620000in}{4.620000in}}%
\pgfusepath{clip}%
\pgfsetbuttcap%
\pgfsetroundjoin%
\definecolor{currentfill}{rgb}{1.000000,0.894118,0.788235}%
\pgfsetfillcolor{currentfill}%
\pgfsetlinewidth{0.000000pt}%
\definecolor{currentstroke}{rgb}{1.000000,0.894118,0.788235}%
\pgfsetstrokecolor{currentstroke}%
\pgfsetdash{}{0pt}%
\pgfpathmoveto{\pgfqpoint{2.981592in}{3.650866in}}%
\pgfpathlineto{\pgfqpoint{2.954645in}{3.635308in}}%
\pgfpathlineto{\pgfqpoint{2.954645in}{3.666424in}}%
\pgfpathlineto{\pgfqpoint{2.981592in}{3.681982in}}%
\pgfpathlineto{\pgfqpoint{2.981592in}{3.650866in}}%
\pgfpathclose%
\pgfusepath{fill}%
\end{pgfscope}%
\begin{pgfscope}%
\pgfpathrectangle{\pgfqpoint{0.765000in}{0.660000in}}{\pgfqpoint{4.620000in}{4.620000in}}%
\pgfusepath{clip}%
\pgfsetbuttcap%
\pgfsetroundjoin%
\definecolor{currentfill}{rgb}{1.000000,0.894118,0.788235}%
\pgfsetfillcolor{currentfill}%
\pgfsetlinewidth{0.000000pt}%
\definecolor{currentstroke}{rgb}{1.000000,0.894118,0.788235}%
\pgfsetstrokecolor{currentstroke}%
\pgfsetdash{}{0pt}%
\pgfpathmoveto{\pgfqpoint{2.981592in}{3.650866in}}%
\pgfpathlineto{\pgfqpoint{3.008539in}{3.635308in}}%
\pgfpathlineto{\pgfqpoint{3.008539in}{3.666424in}}%
\pgfpathlineto{\pgfqpoint{2.981592in}{3.681982in}}%
\pgfpathlineto{\pgfqpoint{2.981592in}{3.650866in}}%
\pgfpathclose%
\pgfusepath{fill}%
\end{pgfscope}%
\begin{pgfscope}%
\pgfpathrectangle{\pgfqpoint{0.765000in}{0.660000in}}{\pgfqpoint{4.620000in}{4.620000in}}%
\pgfusepath{clip}%
\pgfsetbuttcap%
\pgfsetroundjoin%
\definecolor{currentfill}{rgb}{1.000000,0.894118,0.788235}%
\pgfsetfillcolor{currentfill}%
\pgfsetlinewidth{0.000000pt}%
\definecolor{currentstroke}{rgb}{1.000000,0.894118,0.788235}%
\pgfsetstrokecolor{currentstroke}%
\pgfsetdash{}{0pt}%
\pgfpathmoveto{\pgfqpoint{2.981238in}{3.677715in}}%
\pgfpathlineto{\pgfqpoint{2.954291in}{3.662157in}}%
\pgfpathlineto{\pgfqpoint{2.981238in}{3.646599in}}%
\pgfpathlineto{\pgfqpoint{3.008185in}{3.662157in}}%
\pgfpathlineto{\pgfqpoint{2.981238in}{3.677715in}}%
\pgfpathclose%
\pgfusepath{fill}%
\end{pgfscope}%
\begin{pgfscope}%
\pgfpathrectangle{\pgfqpoint{0.765000in}{0.660000in}}{\pgfqpoint{4.620000in}{4.620000in}}%
\pgfusepath{clip}%
\pgfsetbuttcap%
\pgfsetroundjoin%
\definecolor{currentfill}{rgb}{1.000000,0.894118,0.788235}%
\pgfsetfillcolor{currentfill}%
\pgfsetlinewidth{0.000000pt}%
\definecolor{currentstroke}{rgb}{1.000000,0.894118,0.788235}%
\pgfsetstrokecolor{currentstroke}%
\pgfsetdash{}{0pt}%
\pgfpathmoveto{\pgfqpoint{2.981238in}{3.615483in}}%
\pgfpathlineto{\pgfqpoint{3.008185in}{3.631041in}}%
\pgfpathlineto{\pgfqpoint{3.008185in}{3.662157in}}%
\pgfpathlineto{\pgfqpoint{2.981238in}{3.646599in}}%
\pgfpathlineto{\pgfqpoint{2.981238in}{3.615483in}}%
\pgfpathclose%
\pgfusepath{fill}%
\end{pgfscope}%
\begin{pgfscope}%
\pgfpathrectangle{\pgfqpoint{0.765000in}{0.660000in}}{\pgfqpoint{4.620000in}{4.620000in}}%
\pgfusepath{clip}%
\pgfsetbuttcap%
\pgfsetroundjoin%
\definecolor{currentfill}{rgb}{1.000000,0.894118,0.788235}%
\pgfsetfillcolor{currentfill}%
\pgfsetlinewidth{0.000000pt}%
\definecolor{currentstroke}{rgb}{1.000000,0.894118,0.788235}%
\pgfsetstrokecolor{currentstroke}%
\pgfsetdash{}{0pt}%
\pgfpathmoveto{\pgfqpoint{2.954291in}{3.631041in}}%
\pgfpathlineto{\pgfqpoint{2.981238in}{3.615483in}}%
\pgfpathlineto{\pgfqpoint{2.981238in}{3.646599in}}%
\pgfpathlineto{\pgfqpoint{2.954291in}{3.662157in}}%
\pgfpathlineto{\pgfqpoint{2.954291in}{3.631041in}}%
\pgfpathclose%
\pgfusepath{fill}%
\end{pgfscope}%
\begin{pgfscope}%
\pgfpathrectangle{\pgfqpoint{0.765000in}{0.660000in}}{\pgfqpoint{4.620000in}{4.620000in}}%
\pgfusepath{clip}%
\pgfsetbuttcap%
\pgfsetroundjoin%
\definecolor{currentfill}{rgb}{1.000000,0.894118,0.788235}%
\pgfsetfillcolor{currentfill}%
\pgfsetlinewidth{0.000000pt}%
\definecolor{currentstroke}{rgb}{1.000000,0.894118,0.788235}%
\pgfsetstrokecolor{currentstroke}%
\pgfsetdash{}{0pt}%
\pgfpathmoveto{\pgfqpoint{2.981592in}{3.681982in}}%
\pgfpathlineto{\pgfqpoint{2.954645in}{3.666424in}}%
\pgfpathlineto{\pgfqpoint{2.981592in}{3.650866in}}%
\pgfpathlineto{\pgfqpoint{3.008539in}{3.666424in}}%
\pgfpathlineto{\pgfqpoint{2.981592in}{3.681982in}}%
\pgfpathclose%
\pgfusepath{fill}%
\end{pgfscope}%
\begin{pgfscope}%
\pgfpathrectangle{\pgfqpoint{0.765000in}{0.660000in}}{\pgfqpoint{4.620000in}{4.620000in}}%
\pgfusepath{clip}%
\pgfsetbuttcap%
\pgfsetroundjoin%
\definecolor{currentfill}{rgb}{1.000000,0.894118,0.788235}%
\pgfsetfillcolor{currentfill}%
\pgfsetlinewidth{0.000000pt}%
\definecolor{currentstroke}{rgb}{1.000000,0.894118,0.788235}%
\pgfsetstrokecolor{currentstroke}%
\pgfsetdash{}{0pt}%
\pgfpathmoveto{\pgfqpoint{2.981592in}{3.619750in}}%
\pgfpathlineto{\pgfqpoint{3.008539in}{3.635308in}}%
\pgfpathlineto{\pgfqpoint{3.008539in}{3.666424in}}%
\pgfpathlineto{\pgfqpoint{2.981592in}{3.650866in}}%
\pgfpathlineto{\pgfqpoint{2.981592in}{3.619750in}}%
\pgfpathclose%
\pgfusepath{fill}%
\end{pgfscope}%
\begin{pgfscope}%
\pgfpathrectangle{\pgfqpoint{0.765000in}{0.660000in}}{\pgfqpoint{4.620000in}{4.620000in}}%
\pgfusepath{clip}%
\pgfsetbuttcap%
\pgfsetroundjoin%
\definecolor{currentfill}{rgb}{1.000000,0.894118,0.788235}%
\pgfsetfillcolor{currentfill}%
\pgfsetlinewidth{0.000000pt}%
\definecolor{currentstroke}{rgb}{1.000000,0.894118,0.788235}%
\pgfsetstrokecolor{currentstroke}%
\pgfsetdash{}{0pt}%
\pgfpathmoveto{\pgfqpoint{2.954645in}{3.635308in}}%
\pgfpathlineto{\pgfqpoint{2.981592in}{3.619750in}}%
\pgfpathlineto{\pgfqpoint{2.981592in}{3.650866in}}%
\pgfpathlineto{\pgfqpoint{2.954645in}{3.666424in}}%
\pgfpathlineto{\pgfqpoint{2.954645in}{3.635308in}}%
\pgfpathclose%
\pgfusepath{fill}%
\end{pgfscope}%
\begin{pgfscope}%
\pgfpathrectangle{\pgfqpoint{0.765000in}{0.660000in}}{\pgfqpoint{4.620000in}{4.620000in}}%
\pgfusepath{clip}%
\pgfsetbuttcap%
\pgfsetroundjoin%
\definecolor{currentfill}{rgb}{1.000000,0.894118,0.788235}%
\pgfsetfillcolor{currentfill}%
\pgfsetlinewidth{0.000000pt}%
\definecolor{currentstroke}{rgb}{1.000000,0.894118,0.788235}%
\pgfsetstrokecolor{currentstroke}%
\pgfsetdash{}{0pt}%
\pgfpathmoveto{\pgfqpoint{2.981238in}{3.646599in}}%
\pgfpathlineto{\pgfqpoint{2.954291in}{3.631041in}}%
\pgfpathlineto{\pgfqpoint{2.954645in}{3.635308in}}%
\pgfpathlineto{\pgfqpoint{2.981592in}{3.650866in}}%
\pgfpathlineto{\pgfqpoint{2.981238in}{3.646599in}}%
\pgfpathclose%
\pgfusepath{fill}%
\end{pgfscope}%
\begin{pgfscope}%
\pgfpathrectangle{\pgfqpoint{0.765000in}{0.660000in}}{\pgfqpoint{4.620000in}{4.620000in}}%
\pgfusepath{clip}%
\pgfsetbuttcap%
\pgfsetroundjoin%
\definecolor{currentfill}{rgb}{1.000000,0.894118,0.788235}%
\pgfsetfillcolor{currentfill}%
\pgfsetlinewidth{0.000000pt}%
\definecolor{currentstroke}{rgb}{1.000000,0.894118,0.788235}%
\pgfsetstrokecolor{currentstroke}%
\pgfsetdash{}{0pt}%
\pgfpathmoveto{\pgfqpoint{3.008185in}{3.631041in}}%
\pgfpathlineto{\pgfqpoint{2.981238in}{3.646599in}}%
\pgfpathlineto{\pgfqpoint{2.981592in}{3.650866in}}%
\pgfpathlineto{\pgfqpoint{3.008539in}{3.635308in}}%
\pgfpathlineto{\pgfqpoint{3.008185in}{3.631041in}}%
\pgfpathclose%
\pgfusepath{fill}%
\end{pgfscope}%
\begin{pgfscope}%
\pgfpathrectangle{\pgfqpoint{0.765000in}{0.660000in}}{\pgfqpoint{4.620000in}{4.620000in}}%
\pgfusepath{clip}%
\pgfsetbuttcap%
\pgfsetroundjoin%
\definecolor{currentfill}{rgb}{1.000000,0.894118,0.788235}%
\pgfsetfillcolor{currentfill}%
\pgfsetlinewidth{0.000000pt}%
\definecolor{currentstroke}{rgb}{1.000000,0.894118,0.788235}%
\pgfsetstrokecolor{currentstroke}%
\pgfsetdash{}{0pt}%
\pgfpathmoveto{\pgfqpoint{2.981238in}{3.646599in}}%
\pgfpathlineto{\pgfqpoint{2.981238in}{3.677715in}}%
\pgfpathlineto{\pgfqpoint{2.981592in}{3.681982in}}%
\pgfpathlineto{\pgfqpoint{3.008539in}{3.635308in}}%
\pgfpathlineto{\pgfqpoint{2.981238in}{3.646599in}}%
\pgfpathclose%
\pgfusepath{fill}%
\end{pgfscope}%
\begin{pgfscope}%
\pgfpathrectangle{\pgfqpoint{0.765000in}{0.660000in}}{\pgfqpoint{4.620000in}{4.620000in}}%
\pgfusepath{clip}%
\pgfsetbuttcap%
\pgfsetroundjoin%
\definecolor{currentfill}{rgb}{1.000000,0.894118,0.788235}%
\pgfsetfillcolor{currentfill}%
\pgfsetlinewidth{0.000000pt}%
\definecolor{currentstroke}{rgb}{1.000000,0.894118,0.788235}%
\pgfsetstrokecolor{currentstroke}%
\pgfsetdash{}{0pt}%
\pgfpathmoveto{\pgfqpoint{3.008185in}{3.631041in}}%
\pgfpathlineto{\pgfqpoint{3.008185in}{3.662157in}}%
\pgfpathlineto{\pgfqpoint{3.008539in}{3.666424in}}%
\pgfpathlineto{\pgfqpoint{2.981592in}{3.650866in}}%
\pgfpathlineto{\pgfqpoint{3.008185in}{3.631041in}}%
\pgfpathclose%
\pgfusepath{fill}%
\end{pgfscope}%
\begin{pgfscope}%
\pgfpathrectangle{\pgfqpoint{0.765000in}{0.660000in}}{\pgfqpoint{4.620000in}{4.620000in}}%
\pgfusepath{clip}%
\pgfsetbuttcap%
\pgfsetroundjoin%
\definecolor{currentfill}{rgb}{1.000000,0.894118,0.788235}%
\pgfsetfillcolor{currentfill}%
\pgfsetlinewidth{0.000000pt}%
\definecolor{currentstroke}{rgb}{1.000000,0.894118,0.788235}%
\pgfsetstrokecolor{currentstroke}%
\pgfsetdash{}{0pt}%
\pgfpathmoveto{\pgfqpoint{2.954291in}{3.631041in}}%
\pgfpathlineto{\pgfqpoint{2.981238in}{3.615483in}}%
\pgfpathlineto{\pgfqpoint{2.981592in}{3.619750in}}%
\pgfpathlineto{\pgfqpoint{2.954645in}{3.635308in}}%
\pgfpathlineto{\pgfqpoint{2.954291in}{3.631041in}}%
\pgfpathclose%
\pgfusepath{fill}%
\end{pgfscope}%
\begin{pgfscope}%
\pgfpathrectangle{\pgfqpoint{0.765000in}{0.660000in}}{\pgfqpoint{4.620000in}{4.620000in}}%
\pgfusepath{clip}%
\pgfsetbuttcap%
\pgfsetroundjoin%
\definecolor{currentfill}{rgb}{1.000000,0.894118,0.788235}%
\pgfsetfillcolor{currentfill}%
\pgfsetlinewidth{0.000000pt}%
\definecolor{currentstroke}{rgb}{1.000000,0.894118,0.788235}%
\pgfsetstrokecolor{currentstroke}%
\pgfsetdash{}{0pt}%
\pgfpathmoveto{\pgfqpoint{2.981238in}{3.615483in}}%
\pgfpathlineto{\pgfqpoint{3.008185in}{3.631041in}}%
\pgfpathlineto{\pgfqpoint{3.008539in}{3.635308in}}%
\pgfpathlineto{\pgfqpoint{2.981592in}{3.619750in}}%
\pgfpathlineto{\pgfqpoint{2.981238in}{3.615483in}}%
\pgfpathclose%
\pgfusepath{fill}%
\end{pgfscope}%
\begin{pgfscope}%
\pgfpathrectangle{\pgfqpoint{0.765000in}{0.660000in}}{\pgfqpoint{4.620000in}{4.620000in}}%
\pgfusepath{clip}%
\pgfsetbuttcap%
\pgfsetroundjoin%
\definecolor{currentfill}{rgb}{1.000000,0.894118,0.788235}%
\pgfsetfillcolor{currentfill}%
\pgfsetlinewidth{0.000000pt}%
\definecolor{currentstroke}{rgb}{1.000000,0.894118,0.788235}%
\pgfsetstrokecolor{currentstroke}%
\pgfsetdash{}{0pt}%
\pgfpathmoveto{\pgfqpoint{2.981238in}{3.677715in}}%
\pgfpathlineto{\pgfqpoint{2.954291in}{3.662157in}}%
\pgfpathlineto{\pgfqpoint{2.954645in}{3.666424in}}%
\pgfpathlineto{\pgfqpoint{2.981592in}{3.681982in}}%
\pgfpathlineto{\pgfqpoint{2.981238in}{3.677715in}}%
\pgfpathclose%
\pgfusepath{fill}%
\end{pgfscope}%
\begin{pgfscope}%
\pgfpathrectangle{\pgfqpoint{0.765000in}{0.660000in}}{\pgfqpoint{4.620000in}{4.620000in}}%
\pgfusepath{clip}%
\pgfsetbuttcap%
\pgfsetroundjoin%
\definecolor{currentfill}{rgb}{1.000000,0.894118,0.788235}%
\pgfsetfillcolor{currentfill}%
\pgfsetlinewidth{0.000000pt}%
\definecolor{currentstroke}{rgb}{1.000000,0.894118,0.788235}%
\pgfsetstrokecolor{currentstroke}%
\pgfsetdash{}{0pt}%
\pgfpathmoveto{\pgfqpoint{3.008185in}{3.662157in}}%
\pgfpathlineto{\pgfqpoint{2.981238in}{3.677715in}}%
\pgfpathlineto{\pgfqpoint{2.981592in}{3.681982in}}%
\pgfpathlineto{\pgfqpoint{3.008539in}{3.666424in}}%
\pgfpathlineto{\pgfqpoint{3.008185in}{3.662157in}}%
\pgfpathclose%
\pgfusepath{fill}%
\end{pgfscope}%
\begin{pgfscope}%
\pgfpathrectangle{\pgfqpoint{0.765000in}{0.660000in}}{\pgfqpoint{4.620000in}{4.620000in}}%
\pgfusepath{clip}%
\pgfsetbuttcap%
\pgfsetroundjoin%
\definecolor{currentfill}{rgb}{1.000000,0.894118,0.788235}%
\pgfsetfillcolor{currentfill}%
\pgfsetlinewidth{0.000000pt}%
\definecolor{currentstroke}{rgb}{1.000000,0.894118,0.788235}%
\pgfsetstrokecolor{currentstroke}%
\pgfsetdash{}{0pt}%
\pgfpathmoveto{\pgfqpoint{2.954291in}{3.631041in}}%
\pgfpathlineto{\pgfqpoint{2.954291in}{3.662157in}}%
\pgfpathlineto{\pgfqpoint{2.954645in}{3.666424in}}%
\pgfpathlineto{\pgfqpoint{2.981592in}{3.619750in}}%
\pgfpathlineto{\pgfqpoint{2.954291in}{3.631041in}}%
\pgfpathclose%
\pgfusepath{fill}%
\end{pgfscope}%
\begin{pgfscope}%
\pgfpathrectangle{\pgfqpoint{0.765000in}{0.660000in}}{\pgfqpoint{4.620000in}{4.620000in}}%
\pgfusepath{clip}%
\pgfsetbuttcap%
\pgfsetroundjoin%
\definecolor{currentfill}{rgb}{1.000000,0.894118,0.788235}%
\pgfsetfillcolor{currentfill}%
\pgfsetlinewidth{0.000000pt}%
\definecolor{currentstroke}{rgb}{1.000000,0.894118,0.788235}%
\pgfsetstrokecolor{currentstroke}%
\pgfsetdash{}{0pt}%
\pgfpathmoveto{\pgfqpoint{2.981238in}{3.615483in}}%
\pgfpathlineto{\pgfqpoint{2.981238in}{3.646599in}}%
\pgfpathlineto{\pgfqpoint{2.981592in}{3.650866in}}%
\pgfpathlineto{\pgfqpoint{2.954645in}{3.635308in}}%
\pgfpathlineto{\pgfqpoint{2.981238in}{3.615483in}}%
\pgfpathclose%
\pgfusepath{fill}%
\end{pgfscope}%
\begin{pgfscope}%
\pgfpathrectangle{\pgfqpoint{0.765000in}{0.660000in}}{\pgfqpoint{4.620000in}{4.620000in}}%
\pgfusepath{clip}%
\pgfsetbuttcap%
\pgfsetroundjoin%
\definecolor{currentfill}{rgb}{1.000000,0.894118,0.788235}%
\pgfsetfillcolor{currentfill}%
\pgfsetlinewidth{0.000000pt}%
\definecolor{currentstroke}{rgb}{1.000000,0.894118,0.788235}%
\pgfsetstrokecolor{currentstroke}%
\pgfsetdash{}{0pt}%
\pgfpathmoveto{\pgfqpoint{2.981238in}{3.646599in}}%
\pgfpathlineto{\pgfqpoint{3.008185in}{3.662157in}}%
\pgfpathlineto{\pgfqpoint{3.008539in}{3.666424in}}%
\pgfpathlineto{\pgfqpoint{2.981592in}{3.650866in}}%
\pgfpathlineto{\pgfqpoint{2.981238in}{3.646599in}}%
\pgfpathclose%
\pgfusepath{fill}%
\end{pgfscope}%
\begin{pgfscope}%
\pgfpathrectangle{\pgfqpoint{0.765000in}{0.660000in}}{\pgfqpoint{4.620000in}{4.620000in}}%
\pgfusepath{clip}%
\pgfsetbuttcap%
\pgfsetroundjoin%
\definecolor{currentfill}{rgb}{1.000000,0.894118,0.788235}%
\pgfsetfillcolor{currentfill}%
\pgfsetlinewidth{0.000000pt}%
\definecolor{currentstroke}{rgb}{1.000000,0.894118,0.788235}%
\pgfsetstrokecolor{currentstroke}%
\pgfsetdash{}{0pt}%
\pgfpathmoveto{\pgfqpoint{2.954291in}{3.662157in}}%
\pgfpathlineto{\pgfqpoint{2.981238in}{3.646599in}}%
\pgfpathlineto{\pgfqpoint{2.981592in}{3.650866in}}%
\pgfpathlineto{\pgfqpoint{2.954645in}{3.666424in}}%
\pgfpathlineto{\pgfqpoint{2.954291in}{3.662157in}}%
\pgfpathclose%
\pgfusepath{fill}%
\end{pgfscope}%
\begin{pgfscope}%
\pgfpathrectangle{\pgfqpoint{0.765000in}{0.660000in}}{\pgfqpoint{4.620000in}{4.620000in}}%
\pgfusepath{clip}%
\pgfsetbuttcap%
\pgfsetroundjoin%
\definecolor{currentfill}{rgb}{1.000000,0.894118,0.788235}%
\pgfsetfillcolor{currentfill}%
\pgfsetlinewidth{0.000000pt}%
\definecolor{currentstroke}{rgb}{1.000000,0.894118,0.788235}%
\pgfsetstrokecolor{currentstroke}%
\pgfsetdash{}{0pt}%
\pgfpathmoveto{\pgfqpoint{4.007374in}{3.052356in}}%
\pgfpathlineto{\pgfqpoint{3.980427in}{3.036798in}}%
\pgfpathlineto{\pgfqpoint{4.007374in}{3.021240in}}%
\pgfpathlineto{\pgfqpoint{4.034321in}{3.036798in}}%
\pgfpathlineto{\pgfqpoint{4.007374in}{3.052356in}}%
\pgfpathclose%
\pgfusepath{fill}%
\end{pgfscope}%
\begin{pgfscope}%
\pgfpathrectangle{\pgfqpoint{0.765000in}{0.660000in}}{\pgfqpoint{4.620000in}{4.620000in}}%
\pgfusepath{clip}%
\pgfsetbuttcap%
\pgfsetroundjoin%
\definecolor{currentfill}{rgb}{1.000000,0.894118,0.788235}%
\pgfsetfillcolor{currentfill}%
\pgfsetlinewidth{0.000000pt}%
\definecolor{currentstroke}{rgb}{1.000000,0.894118,0.788235}%
\pgfsetstrokecolor{currentstroke}%
\pgfsetdash{}{0pt}%
\pgfpathmoveto{\pgfqpoint{4.007374in}{3.052356in}}%
\pgfpathlineto{\pgfqpoint{3.980427in}{3.036798in}}%
\pgfpathlineto{\pgfqpoint{3.980427in}{3.067914in}}%
\pgfpathlineto{\pgfqpoint{4.007374in}{3.083472in}}%
\pgfpathlineto{\pgfqpoint{4.007374in}{3.052356in}}%
\pgfpathclose%
\pgfusepath{fill}%
\end{pgfscope}%
\begin{pgfscope}%
\pgfpathrectangle{\pgfqpoint{0.765000in}{0.660000in}}{\pgfqpoint{4.620000in}{4.620000in}}%
\pgfusepath{clip}%
\pgfsetbuttcap%
\pgfsetroundjoin%
\definecolor{currentfill}{rgb}{1.000000,0.894118,0.788235}%
\pgfsetfillcolor{currentfill}%
\pgfsetlinewidth{0.000000pt}%
\definecolor{currentstroke}{rgb}{1.000000,0.894118,0.788235}%
\pgfsetstrokecolor{currentstroke}%
\pgfsetdash{}{0pt}%
\pgfpathmoveto{\pgfqpoint{4.007374in}{3.052356in}}%
\pgfpathlineto{\pgfqpoint{4.034321in}{3.036798in}}%
\pgfpathlineto{\pgfqpoint{4.034321in}{3.067914in}}%
\pgfpathlineto{\pgfqpoint{4.007374in}{3.083472in}}%
\pgfpathlineto{\pgfqpoint{4.007374in}{3.052356in}}%
\pgfpathclose%
\pgfusepath{fill}%
\end{pgfscope}%
\begin{pgfscope}%
\pgfpathrectangle{\pgfqpoint{0.765000in}{0.660000in}}{\pgfqpoint{4.620000in}{4.620000in}}%
\pgfusepath{clip}%
\pgfsetbuttcap%
\pgfsetroundjoin%
\definecolor{currentfill}{rgb}{1.000000,0.894118,0.788235}%
\pgfsetfillcolor{currentfill}%
\pgfsetlinewidth{0.000000pt}%
\definecolor{currentstroke}{rgb}{1.000000,0.894118,0.788235}%
\pgfsetstrokecolor{currentstroke}%
\pgfsetdash{}{0pt}%
\pgfpathmoveto{\pgfqpoint{3.890923in}{2.871313in}}%
\pgfpathlineto{\pgfqpoint{3.863975in}{2.855755in}}%
\pgfpathlineto{\pgfqpoint{3.890923in}{2.840197in}}%
\pgfpathlineto{\pgfqpoint{3.917870in}{2.855755in}}%
\pgfpathlineto{\pgfqpoint{3.890923in}{2.871313in}}%
\pgfpathclose%
\pgfusepath{fill}%
\end{pgfscope}%
\begin{pgfscope}%
\pgfpathrectangle{\pgfqpoint{0.765000in}{0.660000in}}{\pgfqpoint{4.620000in}{4.620000in}}%
\pgfusepath{clip}%
\pgfsetbuttcap%
\pgfsetroundjoin%
\definecolor{currentfill}{rgb}{1.000000,0.894118,0.788235}%
\pgfsetfillcolor{currentfill}%
\pgfsetlinewidth{0.000000pt}%
\definecolor{currentstroke}{rgb}{1.000000,0.894118,0.788235}%
\pgfsetstrokecolor{currentstroke}%
\pgfsetdash{}{0pt}%
\pgfpathmoveto{\pgfqpoint{3.890923in}{2.871313in}}%
\pgfpathlineto{\pgfqpoint{3.863975in}{2.855755in}}%
\pgfpathlineto{\pgfqpoint{3.863975in}{2.886871in}}%
\pgfpathlineto{\pgfqpoint{3.890923in}{2.902429in}}%
\pgfpathlineto{\pgfqpoint{3.890923in}{2.871313in}}%
\pgfpathclose%
\pgfusepath{fill}%
\end{pgfscope}%
\begin{pgfscope}%
\pgfpathrectangle{\pgfqpoint{0.765000in}{0.660000in}}{\pgfqpoint{4.620000in}{4.620000in}}%
\pgfusepath{clip}%
\pgfsetbuttcap%
\pgfsetroundjoin%
\definecolor{currentfill}{rgb}{1.000000,0.894118,0.788235}%
\pgfsetfillcolor{currentfill}%
\pgfsetlinewidth{0.000000pt}%
\definecolor{currentstroke}{rgb}{1.000000,0.894118,0.788235}%
\pgfsetstrokecolor{currentstroke}%
\pgfsetdash{}{0pt}%
\pgfpathmoveto{\pgfqpoint{3.890923in}{2.871313in}}%
\pgfpathlineto{\pgfqpoint{3.917870in}{2.855755in}}%
\pgfpathlineto{\pgfqpoint{3.917870in}{2.886871in}}%
\pgfpathlineto{\pgfqpoint{3.890923in}{2.902429in}}%
\pgfpathlineto{\pgfqpoint{3.890923in}{2.871313in}}%
\pgfpathclose%
\pgfusepath{fill}%
\end{pgfscope}%
\begin{pgfscope}%
\pgfpathrectangle{\pgfqpoint{0.765000in}{0.660000in}}{\pgfqpoint{4.620000in}{4.620000in}}%
\pgfusepath{clip}%
\pgfsetbuttcap%
\pgfsetroundjoin%
\definecolor{currentfill}{rgb}{1.000000,0.894118,0.788235}%
\pgfsetfillcolor{currentfill}%
\pgfsetlinewidth{0.000000pt}%
\definecolor{currentstroke}{rgb}{1.000000,0.894118,0.788235}%
\pgfsetstrokecolor{currentstroke}%
\pgfsetdash{}{0pt}%
\pgfpathmoveto{\pgfqpoint{4.007374in}{3.083472in}}%
\pgfpathlineto{\pgfqpoint{3.980427in}{3.067914in}}%
\pgfpathlineto{\pgfqpoint{4.007374in}{3.052356in}}%
\pgfpathlineto{\pgfqpoint{4.034321in}{3.067914in}}%
\pgfpathlineto{\pgfqpoint{4.007374in}{3.083472in}}%
\pgfpathclose%
\pgfusepath{fill}%
\end{pgfscope}%
\begin{pgfscope}%
\pgfpathrectangle{\pgfqpoint{0.765000in}{0.660000in}}{\pgfqpoint{4.620000in}{4.620000in}}%
\pgfusepath{clip}%
\pgfsetbuttcap%
\pgfsetroundjoin%
\definecolor{currentfill}{rgb}{1.000000,0.894118,0.788235}%
\pgfsetfillcolor{currentfill}%
\pgfsetlinewidth{0.000000pt}%
\definecolor{currentstroke}{rgb}{1.000000,0.894118,0.788235}%
\pgfsetstrokecolor{currentstroke}%
\pgfsetdash{}{0pt}%
\pgfpathmoveto{\pgfqpoint{4.007374in}{3.021240in}}%
\pgfpathlineto{\pgfqpoint{4.034321in}{3.036798in}}%
\pgfpathlineto{\pgfqpoint{4.034321in}{3.067914in}}%
\pgfpathlineto{\pgfqpoint{4.007374in}{3.052356in}}%
\pgfpathlineto{\pgfqpoint{4.007374in}{3.021240in}}%
\pgfpathclose%
\pgfusepath{fill}%
\end{pgfscope}%
\begin{pgfscope}%
\pgfpathrectangle{\pgfqpoint{0.765000in}{0.660000in}}{\pgfqpoint{4.620000in}{4.620000in}}%
\pgfusepath{clip}%
\pgfsetbuttcap%
\pgfsetroundjoin%
\definecolor{currentfill}{rgb}{1.000000,0.894118,0.788235}%
\pgfsetfillcolor{currentfill}%
\pgfsetlinewidth{0.000000pt}%
\definecolor{currentstroke}{rgb}{1.000000,0.894118,0.788235}%
\pgfsetstrokecolor{currentstroke}%
\pgfsetdash{}{0pt}%
\pgfpathmoveto{\pgfqpoint{3.980427in}{3.036798in}}%
\pgfpathlineto{\pgfqpoint{4.007374in}{3.021240in}}%
\pgfpathlineto{\pgfqpoint{4.007374in}{3.052356in}}%
\pgfpathlineto{\pgfqpoint{3.980427in}{3.067914in}}%
\pgfpathlineto{\pgfqpoint{3.980427in}{3.036798in}}%
\pgfpathclose%
\pgfusepath{fill}%
\end{pgfscope}%
\begin{pgfscope}%
\pgfpathrectangle{\pgfqpoint{0.765000in}{0.660000in}}{\pgfqpoint{4.620000in}{4.620000in}}%
\pgfusepath{clip}%
\pgfsetbuttcap%
\pgfsetroundjoin%
\definecolor{currentfill}{rgb}{1.000000,0.894118,0.788235}%
\pgfsetfillcolor{currentfill}%
\pgfsetlinewidth{0.000000pt}%
\definecolor{currentstroke}{rgb}{1.000000,0.894118,0.788235}%
\pgfsetstrokecolor{currentstroke}%
\pgfsetdash{}{0pt}%
\pgfpathmoveto{\pgfqpoint{3.890923in}{2.902429in}}%
\pgfpathlineto{\pgfqpoint{3.863975in}{2.886871in}}%
\pgfpathlineto{\pgfqpoint{3.890923in}{2.871313in}}%
\pgfpathlineto{\pgfqpoint{3.917870in}{2.886871in}}%
\pgfpathlineto{\pgfqpoint{3.890923in}{2.902429in}}%
\pgfpathclose%
\pgfusepath{fill}%
\end{pgfscope}%
\begin{pgfscope}%
\pgfpathrectangle{\pgfqpoint{0.765000in}{0.660000in}}{\pgfqpoint{4.620000in}{4.620000in}}%
\pgfusepath{clip}%
\pgfsetbuttcap%
\pgfsetroundjoin%
\definecolor{currentfill}{rgb}{1.000000,0.894118,0.788235}%
\pgfsetfillcolor{currentfill}%
\pgfsetlinewidth{0.000000pt}%
\definecolor{currentstroke}{rgb}{1.000000,0.894118,0.788235}%
\pgfsetstrokecolor{currentstroke}%
\pgfsetdash{}{0pt}%
\pgfpathmoveto{\pgfqpoint{3.890923in}{2.840197in}}%
\pgfpathlineto{\pgfqpoint{3.917870in}{2.855755in}}%
\pgfpathlineto{\pgfqpoint{3.917870in}{2.886871in}}%
\pgfpathlineto{\pgfqpoint{3.890923in}{2.871313in}}%
\pgfpathlineto{\pgfqpoint{3.890923in}{2.840197in}}%
\pgfpathclose%
\pgfusepath{fill}%
\end{pgfscope}%
\begin{pgfscope}%
\pgfpathrectangle{\pgfqpoint{0.765000in}{0.660000in}}{\pgfqpoint{4.620000in}{4.620000in}}%
\pgfusepath{clip}%
\pgfsetbuttcap%
\pgfsetroundjoin%
\definecolor{currentfill}{rgb}{1.000000,0.894118,0.788235}%
\pgfsetfillcolor{currentfill}%
\pgfsetlinewidth{0.000000pt}%
\definecolor{currentstroke}{rgb}{1.000000,0.894118,0.788235}%
\pgfsetstrokecolor{currentstroke}%
\pgfsetdash{}{0pt}%
\pgfpathmoveto{\pgfqpoint{3.863975in}{2.855755in}}%
\pgfpathlineto{\pgfqpoint{3.890923in}{2.840197in}}%
\pgfpathlineto{\pgfqpoint{3.890923in}{2.871313in}}%
\pgfpathlineto{\pgfqpoint{3.863975in}{2.886871in}}%
\pgfpathlineto{\pgfqpoint{3.863975in}{2.855755in}}%
\pgfpathclose%
\pgfusepath{fill}%
\end{pgfscope}%
\begin{pgfscope}%
\pgfpathrectangle{\pgfqpoint{0.765000in}{0.660000in}}{\pgfqpoint{4.620000in}{4.620000in}}%
\pgfusepath{clip}%
\pgfsetbuttcap%
\pgfsetroundjoin%
\definecolor{currentfill}{rgb}{1.000000,0.894118,0.788235}%
\pgfsetfillcolor{currentfill}%
\pgfsetlinewidth{0.000000pt}%
\definecolor{currentstroke}{rgb}{1.000000,0.894118,0.788235}%
\pgfsetstrokecolor{currentstroke}%
\pgfsetdash{}{0pt}%
\pgfpathmoveto{\pgfqpoint{4.007374in}{3.052356in}}%
\pgfpathlineto{\pgfqpoint{3.980427in}{3.036798in}}%
\pgfpathlineto{\pgfqpoint{3.863975in}{2.855755in}}%
\pgfpathlineto{\pgfqpoint{3.890923in}{2.871313in}}%
\pgfpathlineto{\pgfqpoint{4.007374in}{3.052356in}}%
\pgfpathclose%
\pgfusepath{fill}%
\end{pgfscope}%
\begin{pgfscope}%
\pgfpathrectangle{\pgfqpoint{0.765000in}{0.660000in}}{\pgfqpoint{4.620000in}{4.620000in}}%
\pgfusepath{clip}%
\pgfsetbuttcap%
\pgfsetroundjoin%
\definecolor{currentfill}{rgb}{1.000000,0.894118,0.788235}%
\pgfsetfillcolor{currentfill}%
\pgfsetlinewidth{0.000000pt}%
\definecolor{currentstroke}{rgb}{1.000000,0.894118,0.788235}%
\pgfsetstrokecolor{currentstroke}%
\pgfsetdash{}{0pt}%
\pgfpathmoveto{\pgfqpoint{4.034321in}{3.036798in}}%
\pgfpathlineto{\pgfqpoint{4.007374in}{3.052356in}}%
\pgfpathlineto{\pgfqpoint{3.890923in}{2.871313in}}%
\pgfpathlineto{\pgfqpoint{3.917870in}{2.855755in}}%
\pgfpathlineto{\pgfqpoint{4.034321in}{3.036798in}}%
\pgfpathclose%
\pgfusepath{fill}%
\end{pgfscope}%
\begin{pgfscope}%
\pgfpathrectangle{\pgfqpoint{0.765000in}{0.660000in}}{\pgfqpoint{4.620000in}{4.620000in}}%
\pgfusepath{clip}%
\pgfsetbuttcap%
\pgfsetroundjoin%
\definecolor{currentfill}{rgb}{1.000000,0.894118,0.788235}%
\pgfsetfillcolor{currentfill}%
\pgfsetlinewidth{0.000000pt}%
\definecolor{currentstroke}{rgb}{1.000000,0.894118,0.788235}%
\pgfsetstrokecolor{currentstroke}%
\pgfsetdash{}{0pt}%
\pgfpathmoveto{\pgfqpoint{4.007374in}{3.052356in}}%
\pgfpathlineto{\pgfqpoint{4.007374in}{3.083472in}}%
\pgfpathlineto{\pgfqpoint{3.890923in}{2.902429in}}%
\pgfpathlineto{\pgfqpoint{3.917870in}{2.855755in}}%
\pgfpathlineto{\pgfqpoint{4.007374in}{3.052356in}}%
\pgfpathclose%
\pgfusepath{fill}%
\end{pgfscope}%
\begin{pgfscope}%
\pgfpathrectangle{\pgfqpoint{0.765000in}{0.660000in}}{\pgfqpoint{4.620000in}{4.620000in}}%
\pgfusepath{clip}%
\pgfsetbuttcap%
\pgfsetroundjoin%
\definecolor{currentfill}{rgb}{1.000000,0.894118,0.788235}%
\pgfsetfillcolor{currentfill}%
\pgfsetlinewidth{0.000000pt}%
\definecolor{currentstroke}{rgb}{1.000000,0.894118,0.788235}%
\pgfsetstrokecolor{currentstroke}%
\pgfsetdash{}{0pt}%
\pgfpathmoveto{\pgfqpoint{4.034321in}{3.036798in}}%
\pgfpathlineto{\pgfqpoint{4.034321in}{3.067914in}}%
\pgfpathlineto{\pgfqpoint{3.917870in}{2.886871in}}%
\pgfpathlineto{\pgfqpoint{3.890923in}{2.871313in}}%
\pgfpathlineto{\pgfqpoint{4.034321in}{3.036798in}}%
\pgfpathclose%
\pgfusepath{fill}%
\end{pgfscope}%
\begin{pgfscope}%
\pgfpathrectangle{\pgfqpoint{0.765000in}{0.660000in}}{\pgfqpoint{4.620000in}{4.620000in}}%
\pgfusepath{clip}%
\pgfsetbuttcap%
\pgfsetroundjoin%
\definecolor{currentfill}{rgb}{1.000000,0.894118,0.788235}%
\pgfsetfillcolor{currentfill}%
\pgfsetlinewidth{0.000000pt}%
\definecolor{currentstroke}{rgb}{1.000000,0.894118,0.788235}%
\pgfsetstrokecolor{currentstroke}%
\pgfsetdash{}{0pt}%
\pgfpathmoveto{\pgfqpoint{3.980427in}{3.036798in}}%
\pgfpathlineto{\pgfqpoint{4.007374in}{3.021240in}}%
\pgfpathlineto{\pgfqpoint{3.890923in}{2.840197in}}%
\pgfpathlineto{\pgfqpoint{3.863975in}{2.855755in}}%
\pgfpathlineto{\pgfqpoint{3.980427in}{3.036798in}}%
\pgfpathclose%
\pgfusepath{fill}%
\end{pgfscope}%
\begin{pgfscope}%
\pgfpathrectangle{\pgfqpoint{0.765000in}{0.660000in}}{\pgfqpoint{4.620000in}{4.620000in}}%
\pgfusepath{clip}%
\pgfsetbuttcap%
\pgfsetroundjoin%
\definecolor{currentfill}{rgb}{1.000000,0.894118,0.788235}%
\pgfsetfillcolor{currentfill}%
\pgfsetlinewidth{0.000000pt}%
\definecolor{currentstroke}{rgb}{1.000000,0.894118,0.788235}%
\pgfsetstrokecolor{currentstroke}%
\pgfsetdash{}{0pt}%
\pgfpathmoveto{\pgfqpoint{4.007374in}{3.021240in}}%
\pgfpathlineto{\pgfqpoint{4.034321in}{3.036798in}}%
\pgfpathlineto{\pgfqpoint{3.917870in}{2.855755in}}%
\pgfpathlineto{\pgfqpoint{3.890923in}{2.840197in}}%
\pgfpathlineto{\pgfqpoint{4.007374in}{3.021240in}}%
\pgfpathclose%
\pgfusepath{fill}%
\end{pgfscope}%
\begin{pgfscope}%
\pgfpathrectangle{\pgfqpoint{0.765000in}{0.660000in}}{\pgfqpoint{4.620000in}{4.620000in}}%
\pgfusepath{clip}%
\pgfsetbuttcap%
\pgfsetroundjoin%
\definecolor{currentfill}{rgb}{1.000000,0.894118,0.788235}%
\pgfsetfillcolor{currentfill}%
\pgfsetlinewidth{0.000000pt}%
\definecolor{currentstroke}{rgb}{1.000000,0.894118,0.788235}%
\pgfsetstrokecolor{currentstroke}%
\pgfsetdash{}{0pt}%
\pgfpathmoveto{\pgfqpoint{4.007374in}{3.083472in}}%
\pgfpathlineto{\pgfqpoint{3.980427in}{3.067914in}}%
\pgfpathlineto{\pgfqpoint{3.863975in}{2.886871in}}%
\pgfpathlineto{\pgfqpoint{3.890923in}{2.902429in}}%
\pgfpathlineto{\pgfqpoint{4.007374in}{3.083472in}}%
\pgfpathclose%
\pgfusepath{fill}%
\end{pgfscope}%
\begin{pgfscope}%
\pgfpathrectangle{\pgfqpoint{0.765000in}{0.660000in}}{\pgfqpoint{4.620000in}{4.620000in}}%
\pgfusepath{clip}%
\pgfsetbuttcap%
\pgfsetroundjoin%
\definecolor{currentfill}{rgb}{1.000000,0.894118,0.788235}%
\pgfsetfillcolor{currentfill}%
\pgfsetlinewidth{0.000000pt}%
\definecolor{currentstroke}{rgb}{1.000000,0.894118,0.788235}%
\pgfsetstrokecolor{currentstroke}%
\pgfsetdash{}{0pt}%
\pgfpathmoveto{\pgfqpoint{4.034321in}{3.067914in}}%
\pgfpathlineto{\pgfqpoint{4.007374in}{3.083472in}}%
\pgfpathlineto{\pgfqpoint{3.890923in}{2.902429in}}%
\pgfpathlineto{\pgfqpoint{3.917870in}{2.886871in}}%
\pgfpathlineto{\pgfqpoint{4.034321in}{3.067914in}}%
\pgfpathclose%
\pgfusepath{fill}%
\end{pgfscope}%
\begin{pgfscope}%
\pgfpathrectangle{\pgfqpoint{0.765000in}{0.660000in}}{\pgfqpoint{4.620000in}{4.620000in}}%
\pgfusepath{clip}%
\pgfsetbuttcap%
\pgfsetroundjoin%
\definecolor{currentfill}{rgb}{1.000000,0.894118,0.788235}%
\pgfsetfillcolor{currentfill}%
\pgfsetlinewidth{0.000000pt}%
\definecolor{currentstroke}{rgb}{1.000000,0.894118,0.788235}%
\pgfsetstrokecolor{currentstroke}%
\pgfsetdash{}{0pt}%
\pgfpathmoveto{\pgfqpoint{3.980427in}{3.036798in}}%
\pgfpathlineto{\pgfqpoint{3.980427in}{3.067914in}}%
\pgfpathlineto{\pgfqpoint{3.863975in}{2.886871in}}%
\pgfpathlineto{\pgfqpoint{3.890923in}{2.840197in}}%
\pgfpathlineto{\pgfqpoint{3.980427in}{3.036798in}}%
\pgfpathclose%
\pgfusepath{fill}%
\end{pgfscope}%
\begin{pgfscope}%
\pgfpathrectangle{\pgfqpoint{0.765000in}{0.660000in}}{\pgfqpoint{4.620000in}{4.620000in}}%
\pgfusepath{clip}%
\pgfsetbuttcap%
\pgfsetroundjoin%
\definecolor{currentfill}{rgb}{1.000000,0.894118,0.788235}%
\pgfsetfillcolor{currentfill}%
\pgfsetlinewidth{0.000000pt}%
\definecolor{currentstroke}{rgb}{1.000000,0.894118,0.788235}%
\pgfsetstrokecolor{currentstroke}%
\pgfsetdash{}{0pt}%
\pgfpathmoveto{\pgfqpoint{4.007374in}{3.021240in}}%
\pgfpathlineto{\pgfqpoint{4.007374in}{3.052356in}}%
\pgfpathlineto{\pgfqpoint{3.890923in}{2.871313in}}%
\pgfpathlineto{\pgfqpoint{3.863975in}{2.855755in}}%
\pgfpathlineto{\pgfqpoint{4.007374in}{3.021240in}}%
\pgfpathclose%
\pgfusepath{fill}%
\end{pgfscope}%
\begin{pgfscope}%
\pgfpathrectangle{\pgfqpoint{0.765000in}{0.660000in}}{\pgfqpoint{4.620000in}{4.620000in}}%
\pgfusepath{clip}%
\pgfsetbuttcap%
\pgfsetroundjoin%
\definecolor{currentfill}{rgb}{1.000000,0.894118,0.788235}%
\pgfsetfillcolor{currentfill}%
\pgfsetlinewidth{0.000000pt}%
\definecolor{currentstroke}{rgb}{1.000000,0.894118,0.788235}%
\pgfsetstrokecolor{currentstroke}%
\pgfsetdash{}{0pt}%
\pgfpathmoveto{\pgfqpoint{4.007374in}{3.052356in}}%
\pgfpathlineto{\pgfqpoint{4.034321in}{3.067914in}}%
\pgfpathlineto{\pgfqpoint{3.917870in}{2.886871in}}%
\pgfpathlineto{\pgfqpoint{3.890923in}{2.871313in}}%
\pgfpathlineto{\pgfqpoint{4.007374in}{3.052356in}}%
\pgfpathclose%
\pgfusepath{fill}%
\end{pgfscope}%
\begin{pgfscope}%
\pgfpathrectangle{\pgfqpoint{0.765000in}{0.660000in}}{\pgfqpoint{4.620000in}{4.620000in}}%
\pgfusepath{clip}%
\pgfsetbuttcap%
\pgfsetroundjoin%
\definecolor{currentfill}{rgb}{1.000000,0.894118,0.788235}%
\pgfsetfillcolor{currentfill}%
\pgfsetlinewidth{0.000000pt}%
\definecolor{currentstroke}{rgb}{1.000000,0.894118,0.788235}%
\pgfsetstrokecolor{currentstroke}%
\pgfsetdash{}{0pt}%
\pgfpathmoveto{\pgfqpoint{3.980427in}{3.067914in}}%
\pgfpathlineto{\pgfqpoint{4.007374in}{3.052356in}}%
\pgfpathlineto{\pgfqpoint{3.890923in}{2.871313in}}%
\pgfpathlineto{\pgfqpoint{3.863975in}{2.886871in}}%
\pgfpathlineto{\pgfqpoint{3.980427in}{3.067914in}}%
\pgfpathclose%
\pgfusepath{fill}%
\end{pgfscope}%
\begin{pgfscope}%
\pgfpathrectangle{\pgfqpoint{0.765000in}{0.660000in}}{\pgfqpoint{4.620000in}{4.620000in}}%
\pgfusepath{clip}%
\pgfsetbuttcap%
\pgfsetroundjoin%
\definecolor{currentfill}{rgb}{1.000000,0.894118,0.788235}%
\pgfsetfillcolor{currentfill}%
\pgfsetlinewidth{0.000000pt}%
\definecolor{currentstroke}{rgb}{1.000000,0.894118,0.788235}%
\pgfsetstrokecolor{currentstroke}%
\pgfsetdash{}{0pt}%
\pgfpathmoveto{\pgfqpoint{4.007374in}{3.052356in}}%
\pgfpathlineto{\pgfqpoint{3.980427in}{3.036798in}}%
\pgfpathlineto{\pgfqpoint{4.007374in}{3.021240in}}%
\pgfpathlineto{\pgfqpoint{4.034321in}{3.036798in}}%
\pgfpathlineto{\pgfqpoint{4.007374in}{3.052356in}}%
\pgfpathclose%
\pgfusepath{fill}%
\end{pgfscope}%
\begin{pgfscope}%
\pgfpathrectangle{\pgfqpoint{0.765000in}{0.660000in}}{\pgfqpoint{4.620000in}{4.620000in}}%
\pgfusepath{clip}%
\pgfsetbuttcap%
\pgfsetroundjoin%
\definecolor{currentfill}{rgb}{1.000000,0.894118,0.788235}%
\pgfsetfillcolor{currentfill}%
\pgfsetlinewidth{0.000000pt}%
\definecolor{currentstroke}{rgb}{1.000000,0.894118,0.788235}%
\pgfsetstrokecolor{currentstroke}%
\pgfsetdash{}{0pt}%
\pgfpathmoveto{\pgfqpoint{4.007374in}{3.052356in}}%
\pgfpathlineto{\pgfqpoint{3.980427in}{3.036798in}}%
\pgfpathlineto{\pgfqpoint{3.980427in}{3.067914in}}%
\pgfpathlineto{\pgfqpoint{4.007374in}{3.083472in}}%
\pgfpathlineto{\pgfqpoint{4.007374in}{3.052356in}}%
\pgfpathclose%
\pgfusepath{fill}%
\end{pgfscope}%
\begin{pgfscope}%
\pgfpathrectangle{\pgfqpoint{0.765000in}{0.660000in}}{\pgfqpoint{4.620000in}{4.620000in}}%
\pgfusepath{clip}%
\pgfsetbuttcap%
\pgfsetroundjoin%
\definecolor{currentfill}{rgb}{1.000000,0.894118,0.788235}%
\pgfsetfillcolor{currentfill}%
\pgfsetlinewidth{0.000000pt}%
\definecolor{currentstroke}{rgb}{1.000000,0.894118,0.788235}%
\pgfsetstrokecolor{currentstroke}%
\pgfsetdash{}{0pt}%
\pgfpathmoveto{\pgfqpoint{4.007374in}{3.052356in}}%
\pgfpathlineto{\pgfqpoint{4.034321in}{3.036798in}}%
\pgfpathlineto{\pgfqpoint{4.034321in}{3.067914in}}%
\pgfpathlineto{\pgfqpoint{4.007374in}{3.083472in}}%
\pgfpathlineto{\pgfqpoint{4.007374in}{3.052356in}}%
\pgfpathclose%
\pgfusepath{fill}%
\end{pgfscope}%
\begin{pgfscope}%
\pgfpathrectangle{\pgfqpoint{0.765000in}{0.660000in}}{\pgfqpoint{4.620000in}{4.620000in}}%
\pgfusepath{clip}%
\pgfsetbuttcap%
\pgfsetroundjoin%
\definecolor{currentfill}{rgb}{1.000000,0.894118,0.788235}%
\pgfsetfillcolor{currentfill}%
\pgfsetlinewidth{0.000000pt}%
\definecolor{currentstroke}{rgb}{1.000000,0.894118,0.788235}%
\pgfsetstrokecolor{currentstroke}%
\pgfsetdash{}{0pt}%
\pgfpathmoveto{\pgfqpoint{4.007207in}{3.052100in}}%
\pgfpathlineto{\pgfqpoint{3.980260in}{3.036542in}}%
\pgfpathlineto{\pgfqpoint{4.007207in}{3.020984in}}%
\pgfpathlineto{\pgfqpoint{4.034154in}{3.036542in}}%
\pgfpathlineto{\pgfqpoint{4.007207in}{3.052100in}}%
\pgfpathclose%
\pgfusepath{fill}%
\end{pgfscope}%
\begin{pgfscope}%
\pgfpathrectangle{\pgfqpoint{0.765000in}{0.660000in}}{\pgfqpoint{4.620000in}{4.620000in}}%
\pgfusepath{clip}%
\pgfsetbuttcap%
\pgfsetroundjoin%
\definecolor{currentfill}{rgb}{1.000000,0.894118,0.788235}%
\pgfsetfillcolor{currentfill}%
\pgfsetlinewidth{0.000000pt}%
\definecolor{currentstroke}{rgb}{1.000000,0.894118,0.788235}%
\pgfsetstrokecolor{currentstroke}%
\pgfsetdash{}{0pt}%
\pgfpathmoveto{\pgfqpoint{4.007207in}{3.052100in}}%
\pgfpathlineto{\pgfqpoint{3.980260in}{3.036542in}}%
\pgfpathlineto{\pgfqpoint{3.980260in}{3.067658in}}%
\pgfpathlineto{\pgfqpoint{4.007207in}{3.083216in}}%
\pgfpathlineto{\pgfqpoint{4.007207in}{3.052100in}}%
\pgfpathclose%
\pgfusepath{fill}%
\end{pgfscope}%
\begin{pgfscope}%
\pgfpathrectangle{\pgfqpoint{0.765000in}{0.660000in}}{\pgfqpoint{4.620000in}{4.620000in}}%
\pgfusepath{clip}%
\pgfsetbuttcap%
\pgfsetroundjoin%
\definecolor{currentfill}{rgb}{1.000000,0.894118,0.788235}%
\pgfsetfillcolor{currentfill}%
\pgfsetlinewidth{0.000000pt}%
\definecolor{currentstroke}{rgb}{1.000000,0.894118,0.788235}%
\pgfsetstrokecolor{currentstroke}%
\pgfsetdash{}{0pt}%
\pgfpathmoveto{\pgfqpoint{4.007207in}{3.052100in}}%
\pgfpathlineto{\pgfqpoint{4.034154in}{3.036542in}}%
\pgfpathlineto{\pgfqpoint{4.034154in}{3.067658in}}%
\pgfpathlineto{\pgfqpoint{4.007207in}{3.083216in}}%
\pgfpathlineto{\pgfqpoint{4.007207in}{3.052100in}}%
\pgfpathclose%
\pgfusepath{fill}%
\end{pgfscope}%
\begin{pgfscope}%
\pgfpathrectangle{\pgfqpoint{0.765000in}{0.660000in}}{\pgfqpoint{4.620000in}{4.620000in}}%
\pgfusepath{clip}%
\pgfsetbuttcap%
\pgfsetroundjoin%
\definecolor{currentfill}{rgb}{1.000000,0.894118,0.788235}%
\pgfsetfillcolor{currentfill}%
\pgfsetlinewidth{0.000000pt}%
\definecolor{currentstroke}{rgb}{1.000000,0.894118,0.788235}%
\pgfsetstrokecolor{currentstroke}%
\pgfsetdash{}{0pt}%
\pgfpathmoveto{\pgfqpoint{4.007374in}{3.083472in}}%
\pgfpathlineto{\pgfqpoint{3.980427in}{3.067914in}}%
\pgfpathlineto{\pgfqpoint{4.007374in}{3.052356in}}%
\pgfpathlineto{\pgfqpoint{4.034321in}{3.067914in}}%
\pgfpathlineto{\pgfqpoint{4.007374in}{3.083472in}}%
\pgfpathclose%
\pgfusepath{fill}%
\end{pgfscope}%
\begin{pgfscope}%
\pgfpathrectangle{\pgfqpoint{0.765000in}{0.660000in}}{\pgfqpoint{4.620000in}{4.620000in}}%
\pgfusepath{clip}%
\pgfsetbuttcap%
\pgfsetroundjoin%
\definecolor{currentfill}{rgb}{1.000000,0.894118,0.788235}%
\pgfsetfillcolor{currentfill}%
\pgfsetlinewidth{0.000000pt}%
\definecolor{currentstroke}{rgb}{1.000000,0.894118,0.788235}%
\pgfsetstrokecolor{currentstroke}%
\pgfsetdash{}{0pt}%
\pgfpathmoveto{\pgfqpoint{4.007374in}{3.021240in}}%
\pgfpathlineto{\pgfqpoint{4.034321in}{3.036798in}}%
\pgfpathlineto{\pgfqpoint{4.034321in}{3.067914in}}%
\pgfpathlineto{\pgfqpoint{4.007374in}{3.052356in}}%
\pgfpathlineto{\pgfqpoint{4.007374in}{3.021240in}}%
\pgfpathclose%
\pgfusepath{fill}%
\end{pgfscope}%
\begin{pgfscope}%
\pgfpathrectangle{\pgfqpoint{0.765000in}{0.660000in}}{\pgfqpoint{4.620000in}{4.620000in}}%
\pgfusepath{clip}%
\pgfsetbuttcap%
\pgfsetroundjoin%
\definecolor{currentfill}{rgb}{1.000000,0.894118,0.788235}%
\pgfsetfillcolor{currentfill}%
\pgfsetlinewidth{0.000000pt}%
\definecolor{currentstroke}{rgb}{1.000000,0.894118,0.788235}%
\pgfsetstrokecolor{currentstroke}%
\pgfsetdash{}{0pt}%
\pgfpathmoveto{\pgfqpoint{3.980427in}{3.036798in}}%
\pgfpathlineto{\pgfqpoint{4.007374in}{3.021240in}}%
\pgfpathlineto{\pgfqpoint{4.007374in}{3.052356in}}%
\pgfpathlineto{\pgfqpoint{3.980427in}{3.067914in}}%
\pgfpathlineto{\pgfqpoint{3.980427in}{3.036798in}}%
\pgfpathclose%
\pgfusepath{fill}%
\end{pgfscope}%
\begin{pgfscope}%
\pgfpathrectangle{\pgfqpoint{0.765000in}{0.660000in}}{\pgfqpoint{4.620000in}{4.620000in}}%
\pgfusepath{clip}%
\pgfsetbuttcap%
\pgfsetroundjoin%
\definecolor{currentfill}{rgb}{1.000000,0.894118,0.788235}%
\pgfsetfillcolor{currentfill}%
\pgfsetlinewidth{0.000000pt}%
\definecolor{currentstroke}{rgb}{1.000000,0.894118,0.788235}%
\pgfsetstrokecolor{currentstroke}%
\pgfsetdash{}{0pt}%
\pgfpathmoveto{\pgfqpoint{4.007207in}{3.083216in}}%
\pgfpathlineto{\pgfqpoint{3.980260in}{3.067658in}}%
\pgfpathlineto{\pgfqpoint{4.007207in}{3.052100in}}%
\pgfpathlineto{\pgfqpoint{4.034154in}{3.067658in}}%
\pgfpathlineto{\pgfqpoint{4.007207in}{3.083216in}}%
\pgfpathclose%
\pgfusepath{fill}%
\end{pgfscope}%
\begin{pgfscope}%
\pgfpathrectangle{\pgfqpoint{0.765000in}{0.660000in}}{\pgfqpoint{4.620000in}{4.620000in}}%
\pgfusepath{clip}%
\pgfsetbuttcap%
\pgfsetroundjoin%
\definecolor{currentfill}{rgb}{1.000000,0.894118,0.788235}%
\pgfsetfillcolor{currentfill}%
\pgfsetlinewidth{0.000000pt}%
\definecolor{currentstroke}{rgb}{1.000000,0.894118,0.788235}%
\pgfsetstrokecolor{currentstroke}%
\pgfsetdash{}{0pt}%
\pgfpathmoveto{\pgfqpoint{4.007207in}{3.020984in}}%
\pgfpathlineto{\pgfqpoint{4.034154in}{3.036542in}}%
\pgfpathlineto{\pgfqpoint{4.034154in}{3.067658in}}%
\pgfpathlineto{\pgfqpoint{4.007207in}{3.052100in}}%
\pgfpathlineto{\pgfqpoint{4.007207in}{3.020984in}}%
\pgfpathclose%
\pgfusepath{fill}%
\end{pgfscope}%
\begin{pgfscope}%
\pgfpathrectangle{\pgfqpoint{0.765000in}{0.660000in}}{\pgfqpoint{4.620000in}{4.620000in}}%
\pgfusepath{clip}%
\pgfsetbuttcap%
\pgfsetroundjoin%
\definecolor{currentfill}{rgb}{1.000000,0.894118,0.788235}%
\pgfsetfillcolor{currentfill}%
\pgfsetlinewidth{0.000000pt}%
\definecolor{currentstroke}{rgb}{1.000000,0.894118,0.788235}%
\pgfsetstrokecolor{currentstroke}%
\pgfsetdash{}{0pt}%
\pgfpathmoveto{\pgfqpoint{3.980260in}{3.036542in}}%
\pgfpathlineto{\pgfqpoint{4.007207in}{3.020984in}}%
\pgfpathlineto{\pgfqpoint{4.007207in}{3.052100in}}%
\pgfpathlineto{\pgfqpoint{3.980260in}{3.067658in}}%
\pgfpathlineto{\pgfqpoint{3.980260in}{3.036542in}}%
\pgfpathclose%
\pgfusepath{fill}%
\end{pgfscope}%
\begin{pgfscope}%
\pgfpathrectangle{\pgfqpoint{0.765000in}{0.660000in}}{\pgfqpoint{4.620000in}{4.620000in}}%
\pgfusepath{clip}%
\pgfsetbuttcap%
\pgfsetroundjoin%
\definecolor{currentfill}{rgb}{1.000000,0.894118,0.788235}%
\pgfsetfillcolor{currentfill}%
\pgfsetlinewidth{0.000000pt}%
\definecolor{currentstroke}{rgb}{1.000000,0.894118,0.788235}%
\pgfsetstrokecolor{currentstroke}%
\pgfsetdash{}{0pt}%
\pgfpathmoveto{\pgfqpoint{4.007374in}{3.052356in}}%
\pgfpathlineto{\pgfqpoint{3.980427in}{3.036798in}}%
\pgfpathlineto{\pgfqpoint{3.980260in}{3.036542in}}%
\pgfpathlineto{\pgfqpoint{4.007207in}{3.052100in}}%
\pgfpathlineto{\pgfqpoint{4.007374in}{3.052356in}}%
\pgfpathclose%
\pgfusepath{fill}%
\end{pgfscope}%
\begin{pgfscope}%
\pgfpathrectangle{\pgfqpoint{0.765000in}{0.660000in}}{\pgfqpoint{4.620000in}{4.620000in}}%
\pgfusepath{clip}%
\pgfsetbuttcap%
\pgfsetroundjoin%
\definecolor{currentfill}{rgb}{1.000000,0.894118,0.788235}%
\pgfsetfillcolor{currentfill}%
\pgfsetlinewidth{0.000000pt}%
\definecolor{currentstroke}{rgb}{1.000000,0.894118,0.788235}%
\pgfsetstrokecolor{currentstroke}%
\pgfsetdash{}{0pt}%
\pgfpathmoveto{\pgfqpoint{4.034321in}{3.036798in}}%
\pgfpathlineto{\pgfqpoint{4.007374in}{3.052356in}}%
\pgfpathlineto{\pgfqpoint{4.007207in}{3.052100in}}%
\pgfpathlineto{\pgfqpoint{4.034154in}{3.036542in}}%
\pgfpathlineto{\pgfqpoint{4.034321in}{3.036798in}}%
\pgfpathclose%
\pgfusepath{fill}%
\end{pgfscope}%
\begin{pgfscope}%
\pgfpathrectangle{\pgfqpoint{0.765000in}{0.660000in}}{\pgfqpoint{4.620000in}{4.620000in}}%
\pgfusepath{clip}%
\pgfsetbuttcap%
\pgfsetroundjoin%
\definecolor{currentfill}{rgb}{1.000000,0.894118,0.788235}%
\pgfsetfillcolor{currentfill}%
\pgfsetlinewidth{0.000000pt}%
\definecolor{currentstroke}{rgb}{1.000000,0.894118,0.788235}%
\pgfsetstrokecolor{currentstroke}%
\pgfsetdash{}{0pt}%
\pgfpathmoveto{\pgfqpoint{4.007374in}{3.052356in}}%
\pgfpathlineto{\pgfqpoint{4.007374in}{3.083472in}}%
\pgfpathlineto{\pgfqpoint{4.007207in}{3.083216in}}%
\pgfpathlineto{\pgfqpoint{4.034154in}{3.036542in}}%
\pgfpathlineto{\pgfqpoint{4.007374in}{3.052356in}}%
\pgfpathclose%
\pgfusepath{fill}%
\end{pgfscope}%
\begin{pgfscope}%
\pgfpathrectangle{\pgfqpoint{0.765000in}{0.660000in}}{\pgfqpoint{4.620000in}{4.620000in}}%
\pgfusepath{clip}%
\pgfsetbuttcap%
\pgfsetroundjoin%
\definecolor{currentfill}{rgb}{1.000000,0.894118,0.788235}%
\pgfsetfillcolor{currentfill}%
\pgfsetlinewidth{0.000000pt}%
\definecolor{currentstroke}{rgb}{1.000000,0.894118,0.788235}%
\pgfsetstrokecolor{currentstroke}%
\pgfsetdash{}{0pt}%
\pgfpathmoveto{\pgfqpoint{4.034321in}{3.036798in}}%
\pgfpathlineto{\pgfqpoint{4.034321in}{3.067914in}}%
\pgfpathlineto{\pgfqpoint{4.034154in}{3.067658in}}%
\pgfpathlineto{\pgfqpoint{4.007207in}{3.052100in}}%
\pgfpathlineto{\pgfqpoint{4.034321in}{3.036798in}}%
\pgfpathclose%
\pgfusepath{fill}%
\end{pgfscope}%
\begin{pgfscope}%
\pgfpathrectangle{\pgfqpoint{0.765000in}{0.660000in}}{\pgfqpoint{4.620000in}{4.620000in}}%
\pgfusepath{clip}%
\pgfsetbuttcap%
\pgfsetroundjoin%
\definecolor{currentfill}{rgb}{1.000000,0.894118,0.788235}%
\pgfsetfillcolor{currentfill}%
\pgfsetlinewidth{0.000000pt}%
\definecolor{currentstroke}{rgb}{1.000000,0.894118,0.788235}%
\pgfsetstrokecolor{currentstroke}%
\pgfsetdash{}{0pt}%
\pgfpathmoveto{\pgfqpoint{3.980427in}{3.036798in}}%
\pgfpathlineto{\pgfqpoint{4.007374in}{3.021240in}}%
\pgfpathlineto{\pgfqpoint{4.007207in}{3.020984in}}%
\pgfpathlineto{\pgfqpoint{3.980260in}{3.036542in}}%
\pgfpathlineto{\pgfqpoint{3.980427in}{3.036798in}}%
\pgfpathclose%
\pgfusepath{fill}%
\end{pgfscope}%
\begin{pgfscope}%
\pgfpathrectangle{\pgfqpoint{0.765000in}{0.660000in}}{\pgfqpoint{4.620000in}{4.620000in}}%
\pgfusepath{clip}%
\pgfsetbuttcap%
\pgfsetroundjoin%
\definecolor{currentfill}{rgb}{1.000000,0.894118,0.788235}%
\pgfsetfillcolor{currentfill}%
\pgfsetlinewidth{0.000000pt}%
\definecolor{currentstroke}{rgb}{1.000000,0.894118,0.788235}%
\pgfsetstrokecolor{currentstroke}%
\pgfsetdash{}{0pt}%
\pgfpathmoveto{\pgfqpoint{4.007374in}{3.021240in}}%
\pgfpathlineto{\pgfqpoint{4.034321in}{3.036798in}}%
\pgfpathlineto{\pgfqpoint{4.034154in}{3.036542in}}%
\pgfpathlineto{\pgfqpoint{4.007207in}{3.020984in}}%
\pgfpathlineto{\pgfqpoint{4.007374in}{3.021240in}}%
\pgfpathclose%
\pgfusepath{fill}%
\end{pgfscope}%
\begin{pgfscope}%
\pgfpathrectangle{\pgfqpoint{0.765000in}{0.660000in}}{\pgfqpoint{4.620000in}{4.620000in}}%
\pgfusepath{clip}%
\pgfsetbuttcap%
\pgfsetroundjoin%
\definecolor{currentfill}{rgb}{1.000000,0.894118,0.788235}%
\pgfsetfillcolor{currentfill}%
\pgfsetlinewidth{0.000000pt}%
\definecolor{currentstroke}{rgb}{1.000000,0.894118,0.788235}%
\pgfsetstrokecolor{currentstroke}%
\pgfsetdash{}{0pt}%
\pgfpathmoveto{\pgfqpoint{4.007374in}{3.083472in}}%
\pgfpathlineto{\pgfqpoint{3.980427in}{3.067914in}}%
\pgfpathlineto{\pgfqpoint{3.980260in}{3.067658in}}%
\pgfpathlineto{\pgfqpoint{4.007207in}{3.083216in}}%
\pgfpathlineto{\pgfqpoint{4.007374in}{3.083472in}}%
\pgfpathclose%
\pgfusepath{fill}%
\end{pgfscope}%
\begin{pgfscope}%
\pgfpathrectangle{\pgfqpoint{0.765000in}{0.660000in}}{\pgfqpoint{4.620000in}{4.620000in}}%
\pgfusepath{clip}%
\pgfsetbuttcap%
\pgfsetroundjoin%
\definecolor{currentfill}{rgb}{1.000000,0.894118,0.788235}%
\pgfsetfillcolor{currentfill}%
\pgfsetlinewidth{0.000000pt}%
\definecolor{currentstroke}{rgb}{1.000000,0.894118,0.788235}%
\pgfsetstrokecolor{currentstroke}%
\pgfsetdash{}{0pt}%
\pgfpathmoveto{\pgfqpoint{4.034321in}{3.067914in}}%
\pgfpathlineto{\pgfqpoint{4.007374in}{3.083472in}}%
\pgfpathlineto{\pgfqpoint{4.007207in}{3.083216in}}%
\pgfpathlineto{\pgfqpoint{4.034154in}{3.067658in}}%
\pgfpathlineto{\pgfqpoint{4.034321in}{3.067914in}}%
\pgfpathclose%
\pgfusepath{fill}%
\end{pgfscope}%
\begin{pgfscope}%
\pgfpathrectangle{\pgfqpoint{0.765000in}{0.660000in}}{\pgfqpoint{4.620000in}{4.620000in}}%
\pgfusepath{clip}%
\pgfsetbuttcap%
\pgfsetroundjoin%
\definecolor{currentfill}{rgb}{1.000000,0.894118,0.788235}%
\pgfsetfillcolor{currentfill}%
\pgfsetlinewidth{0.000000pt}%
\definecolor{currentstroke}{rgb}{1.000000,0.894118,0.788235}%
\pgfsetstrokecolor{currentstroke}%
\pgfsetdash{}{0pt}%
\pgfpathmoveto{\pgfqpoint{3.980427in}{3.036798in}}%
\pgfpathlineto{\pgfqpoint{3.980427in}{3.067914in}}%
\pgfpathlineto{\pgfqpoint{3.980260in}{3.067658in}}%
\pgfpathlineto{\pgfqpoint{4.007207in}{3.020984in}}%
\pgfpathlineto{\pgfqpoint{3.980427in}{3.036798in}}%
\pgfpathclose%
\pgfusepath{fill}%
\end{pgfscope}%
\begin{pgfscope}%
\pgfpathrectangle{\pgfqpoint{0.765000in}{0.660000in}}{\pgfqpoint{4.620000in}{4.620000in}}%
\pgfusepath{clip}%
\pgfsetbuttcap%
\pgfsetroundjoin%
\definecolor{currentfill}{rgb}{1.000000,0.894118,0.788235}%
\pgfsetfillcolor{currentfill}%
\pgfsetlinewidth{0.000000pt}%
\definecolor{currentstroke}{rgb}{1.000000,0.894118,0.788235}%
\pgfsetstrokecolor{currentstroke}%
\pgfsetdash{}{0pt}%
\pgfpathmoveto{\pgfqpoint{4.007374in}{3.021240in}}%
\pgfpathlineto{\pgfqpoint{4.007374in}{3.052356in}}%
\pgfpathlineto{\pgfqpoint{4.007207in}{3.052100in}}%
\pgfpathlineto{\pgfqpoint{3.980260in}{3.036542in}}%
\pgfpathlineto{\pgfqpoint{4.007374in}{3.021240in}}%
\pgfpathclose%
\pgfusepath{fill}%
\end{pgfscope}%
\begin{pgfscope}%
\pgfpathrectangle{\pgfqpoint{0.765000in}{0.660000in}}{\pgfqpoint{4.620000in}{4.620000in}}%
\pgfusepath{clip}%
\pgfsetbuttcap%
\pgfsetroundjoin%
\definecolor{currentfill}{rgb}{1.000000,0.894118,0.788235}%
\pgfsetfillcolor{currentfill}%
\pgfsetlinewidth{0.000000pt}%
\definecolor{currentstroke}{rgb}{1.000000,0.894118,0.788235}%
\pgfsetstrokecolor{currentstroke}%
\pgfsetdash{}{0pt}%
\pgfpathmoveto{\pgfqpoint{4.007374in}{3.052356in}}%
\pgfpathlineto{\pgfqpoint{4.034321in}{3.067914in}}%
\pgfpathlineto{\pgfqpoint{4.034154in}{3.067658in}}%
\pgfpathlineto{\pgfqpoint{4.007207in}{3.052100in}}%
\pgfpathlineto{\pgfqpoint{4.007374in}{3.052356in}}%
\pgfpathclose%
\pgfusepath{fill}%
\end{pgfscope}%
\begin{pgfscope}%
\pgfpathrectangle{\pgfqpoint{0.765000in}{0.660000in}}{\pgfqpoint{4.620000in}{4.620000in}}%
\pgfusepath{clip}%
\pgfsetbuttcap%
\pgfsetroundjoin%
\definecolor{currentfill}{rgb}{1.000000,0.894118,0.788235}%
\pgfsetfillcolor{currentfill}%
\pgfsetlinewidth{0.000000pt}%
\definecolor{currentstroke}{rgb}{1.000000,0.894118,0.788235}%
\pgfsetstrokecolor{currentstroke}%
\pgfsetdash{}{0pt}%
\pgfpathmoveto{\pgfqpoint{3.980427in}{3.067914in}}%
\pgfpathlineto{\pgfqpoint{4.007374in}{3.052356in}}%
\pgfpathlineto{\pgfqpoint{4.007207in}{3.052100in}}%
\pgfpathlineto{\pgfqpoint{3.980260in}{3.067658in}}%
\pgfpathlineto{\pgfqpoint{3.980427in}{3.067914in}}%
\pgfpathclose%
\pgfusepath{fill}%
\end{pgfscope}%
\begin{pgfscope}%
\pgfpathrectangle{\pgfqpoint{0.765000in}{0.660000in}}{\pgfqpoint{4.620000in}{4.620000in}}%
\pgfusepath{clip}%
\pgfsetbuttcap%
\pgfsetroundjoin%
\definecolor{currentfill}{rgb}{1.000000,0.894118,0.788235}%
\pgfsetfillcolor{currentfill}%
\pgfsetlinewidth{0.000000pt}%
\definecolor{currentstroke}{rgb}{1.000000,0.894118,0.788235}%
\pgfsetstrokecolor{currentstroke}%
\pgfsetdash{}{0pt}%
\pgfpathmoveto{\pgfqpoint{4.007374in}{3.052356in}}%
\pgfpathlineto{\pgfqpoint{3.980427in}{3.036798in}}%
\pgfpathlineto{\pgfqpoint{4.007374in}{3.021240in}}%
\pgfpathlineto{\pgfqpoint{4.034321in}{3.036798in}}%
\pgfpathlineto{\pgfqpoint{4.007374in}{3.052356in}}%
\pgfpathclose%
\pgfusepath{fill}%
\end{pgfscope}%
\begin{pgfscope}%
\pgfpathrectangle{\pgfqpoint{0.765000in}{0.660000in}}{\pgfqpoint{4.620000in}{4.620000in}}%
\pgfusepath{clip}%
\pgfsetbuttcap%
\pgfsetroundjoin%
\definecolor{currentfill}{rgb}{1.000000,0.894118,0.788235}%
\pgfsetfillcolor{currentfill}%
\pgfsetlinewidth{0.000000pt}%
\definecolor{currentstroke}{rgb}{1.000000,0.894118,0.788235}%
\pgfsetstrokecolor{currentstroke}%
\pgfsetdash{}{0pt}%
\pgfpathmoveto{\pgfqpoint{4.007374in}{3.052356in}}%
\pgfpathlineto{\pgfqpoint{3.980427in}{3.036798in}}%
\pgfpathlineto{\pgfqpoint{3.980427in}{3.067914in}}%
\pgfpathlineto{\pgfqpoint{4.007374in}{3.083472in}}%
\pgfpathlineto{\pgfqpoint{4.007374in}{3.052356in}}%
\pgfpathclose%
\pgfusepath{fill}%
\end{pgfscope}%
\begin{pgfscope}%
\pgfpathrectangle{\pgfqpoint{0.765000in}{0.660000in}}{\pgfqpoint{4.620000in}{4.620000in}}%
\pgfusepath{clip}%
\pgfsetbuttcap%
\pgfsetroundjoin%
\definecolor{currentfill}{rgb}{1.000000,0.894118,0.788235}%
\pgfsetfillcolor{currentfill}%
\pgfsetlinewidth{0.000000pt}%
\definecolor{currentstroke}{rgb}{1.000000,0.894118,0.788235}%
\pgfsetstrokecolor{currentstroke}%
\pgfsetdash{}{0pt}%
\pgfpathmoveto{\pgfqpoint{4.007374in}{3.052356in}}%
\pgfpathlineto{\pgfqpoint{4.034321in}{3.036798in}}%
\pgfpathlineto{\pgfqpoint{4.034321in}{3.067914in}}%
\pgfpathlineto{\pgfqpoint{4.007374in}{3.083472in}}%
\pgfpathlineto{\pgfqpoint{4.007374in}{3.052356in}}%
\pgfpathclose%
\pgfusepath{fill}%
\end{pgfscope}%
\begin{pgfscope}%
\pgfpathrectangle{\pgfqpoint{0.765000in}{0.660000in}}{\pgfqpoint{4.620000in}{4.620000in}}%
\pgfusepath{clip}%
\pgfsetbuttcap%
\pgfsetroundjoin%
\definecolor{currentfill}{rgb}{1.000000,0.894118,0.788235}%
\pgfsetfillcolor{currentfill}%
\pgfsetlinewidth{0.000000pt}%
\definecolor{currentstroke}{rgb}{1.000000,0.894118,0.788235}%
\pgfsetstrokecolor{currentstroke}%
\pgfsetdash{}{0pt}%
\pgfpathmoveto{\pgfqpoint{4.123332in}{3.249597in}}%
\pgfpathlineto{\pgfqpoint{4.096385in}{3.234039in}}%
\pgfpathlineto{\pgfqpoint{4.123332in}{3.218481in}}%
\pgfpathlineto{\pgfqpoint{4.150279in}{3.234039in}}%
\pgfpathlineto{\pgfqpoint{4.123332in}{3.249597in}}%
\pgfpathclose%
\pgfusepath{fill}%
\end{pgfscope}%
\begin{pgfscope}%
\pgfpathrectangle{\pgfqpoint{0.765000in}{0.660000in}}{\pgfqpoint{4.620000in}{4.620000in}}%
\pgfusepath{clip}%
\pgfsetbuttcap%
\pgfsetroundjoin%
\definecolor{currentfill}{rgb}{1.000000,0.894118,0.788235}%
\pgfsetfillcolor{currentfill}%
\pgfsetlinewidth{0.000000pt}%
\definecolor{currentstroke}{rgb}{1.000000,0.894118,0.788235}%
\pgfsetstrokecolor{currentstroke}%
\pgfsetdash{}{0pt}%
\pgfpathmoveto{\pgfqpoint{4.123332in}{3.249597in}}%
\pgfpathlineto{\pgfqpoint{4.096385in}{3.234039in}}%
\pgfpathlineto{\pgfqpoint{4.096385in}{3.265155in}}%
\pgfpathlineto{\pgfqpoint{4.123332in}{3.280713in}}%
\pgfpathlineto{\pgfqpoint{4.123332in}{3.249597in}}%
\pgfpathclose%
\pgfusepath{fill}%
\end{pgfscope}%
\begin{pgfscope}%
\pgfpathrectangle{\pgfqpoint{0.765000in}{0.660000in}}{\pgfqpoint{4.620000in}{4.620000in}}%
\pgfusepath{clip}%
\pgfsetbuttcap%
\pgfsetroundjoin%
\definecolor{currentfill}{rgb}{1.000000,0.894118,0.788235}%
\pgfsetfillcolor{currentfill}%
\pgfsetlinewidth{0.000000pt}%
\definecolor{currentstroke}{rgb}{1.000000,0.894118,0.788235}%
\pgfsetstrokecolor{currentstroke}%
\pgfsetdash{}{0pt}%
\pgfpathmoveto{\pgfqpoint{4.123332in}{3.249597in}}%
\pgfpathlineto{\pgfqpoint{4.150279in}{3.234039in}}%
\pgfpathlineto{\pgfqpoint{4.150279in}{3.265155in}}%
\pgfpathlineto{\pgfqpoint{4.123332in}{3.280713in}}%
\pgfpathlineto{\pgfqpoint{4.123332in}{3.249597in}}%
\pgfpathclose%
\pgfusepath{fill}%
\end{pgfscope}%
\begin{pgfscope}%
\pgfpathrectangle{\pgfqpoint{0.765000in}{0.660000in}}{\pgfqpoint{4.620000in}{4.620000in}}%
\pgfusepath{clip}%
\pgfsetbuttcap%
\pgfsetroundjoin%
\definecolor{currentfill}{rgb}{1.000000,0.894118,0.788235}%
\pgfsetfillcolor{currentfill}%
\pgfsetlinewidth{0.000000pt}%
\definecolor{currentstroke}{rgb}{1.000000,0.894118,0.788235}%
\pgfsetstrokecolor{currentstroke}%
\pgfsetdash{}{0pt}%
\pgfpathmoveto{\pgfqpoint{4.007374in}{3.083472in}}%
\pgfpathlineto{\pgfqpoint{3.980427in}{3.067914in}}%
\pgfpathlineto{\pgfqpoint{4.007374in}{3.052356in}}%
\pgfpathlineto{\pgfqpoint{4.034321in}{3.067914in}}%
\pgfpathlineto{\pgfqpoint{4.007374in}{3.083472in}}%
\pgfpathclose%
\pgfusepath{fill}%
\end{pgfscope}%
\begin{pgfscope}%
\pgfpathrectangle{\pgfqpoint{0.765000in}{0.660000in}}{\pgfqpoint{4.620000in}{4.620000in}}%
\pgfusepath{clip}%
\pgfsetbuttcap%
\pgfsetroundjoin%
\definecolor{currentfill}{rgb}{1.000000,0.894118,0.788235}%
\pgfsetfillcolor{currentfill}%
\pgfsetlinewidth{0.000000pt}%
\definecolor{currentstroke}{rgb}{1.000000,0.894118,0.788235}%
\pgfsetstrokecolor{currentstroke}%
\pgfsetdash{}{0pt}%
\pgfpathmoveto{\pgfqpoint{4.007374in}{3.021240in}}%
\pgfpathlineto{\pgfqpoint{4.034321in}{3.036798in}}%
\pgfpathlineto{\pgfqpoint{4.034321in}{3.067914in}}%
\pgfpathlineto{\pgfqpoint{4.007374in}{3.052356in}}%
\pgfpathlineto{\pgfqpoint{4.007374in}{3.021240in}}%
\pgfpathclose%
\pgfusepath{fill}%
\end{pgfscope}%
\begin{pgfscope}%
\pgfpathrectangle{\pgfqpoint{0.765000in}{0.660000in}}{\pgfqpoint{4.620000in}{4.620000in}}%
\pgfusepath{clip}%
\pgfsetbuttcap%
\pgfsetroundjoin%
\definecolor{currentfill}{rgb}{1.000000,0.894118,0.788235}%
\pgfsetfillcolor{currentfill}%
\pgfsetlinewidth{0.000000pt}%
\definecolor{currentstroke}{rgb}{1.000000,0.894118,0.788235}%
\pgfsetstrokecolor{currentstroke}%
\pgfsetdash{}{0pt}%
\pgfpathmoveto{\pgfqpoint{3.980427in}{3.036798in}}%
\pgfpathlineto{\pgfqpoint{4.007374in}{3.021240in}}%
\pgfpathlineto{\pgfqpoint{4.007374in}{3.052356in}}%
\pgfpathlineto{\pgfqpoint{3.980427in}{3.067914in}}%
\pgfpathlineto{\pgfqpoint{3.980427in}{3.036798in}}%
\pgfpathclose%
\pgfusepath{fill}%
\end{pgfscope}%
\begin{pgfscope}%
\pgfpathrectangle{\pgfqpoint{0.765000in}{0.660000in}}{\pgfqpoint{4.620000in}{4.620000in}}%
\pgfusepath{clip}%
\pgfsetbuttcap%
\pgfsetroundjoin%
\definecolor{currentfill}{rgb}{1.000000,0.894118,0.788235}%
\pgfsetfillcolor{currentfill}%
\pgfsetlinewidth{0.000000pt}%
\definecolor{currentstroke}{rgb}{1.000000,0.894118,0.788235}%
\pgfsetstrokecolor{currentstroke}%
\pgfsetdash{}{0pt}%
\pgfpathmoveto{\pgfqpoint{4.123332in}{3.280713in}}%
\pgfpathlineto{\pgfqpoint{4.096385in}{3.265155in}}%
\pgfpathlineto{\pgfqpoint{4.123332in}{3.249597in}}%
\pgfpathlineto{\pgfqpoint{4.150279in}{3.265155in}}%
\pgfpathlineto{\pgfqpoint{4.123332in}{3.280713in}}%
\pgfpathclose%
\pgfusepath{fill}%
\end{pgfscope}%
\begin{pgfscope}%
\pgfpathrectangle{\pgfqpoint{0.765000in}{0.660000in}}{\pgfqpoint{4.620000in}{4.620000in}}%
\pgfusepath{clip}%
\pgfsetbuttcap%
\pgfsetroundjoin%
\definecolor{currentfill}{rgb}{1.000000,0.894118,0.788235}%
\pgfsetfillcolor{currentfill}%
\pgfsetlinewidth{0.000000pt}%
\definecolor{currentstroke}{rgb}{1.000000,0.894118,0.788235}%
\pgfsetstrokecolor{currentstroke}%
\pgfsetdash{}{0pt}%
\pgfpathmoveto{\pgfqpoint{4.123332in}{3.218481in}}%
\pgfpathlineto{\pgfqpoint{4.150279in}{3.234039in}}%
\pgfpathlineto{\pgfqpoint{4.150279in}{3.265155in}}%
\pgfpathlineto{\pgfqpoint{4.123332in}{3.249597in}}%
\pgfpathlineto{\pgfqpoint{4.123332in}{3.218481in}}%
\pgfpathclose%
\pgfusepath{fill}%
\end{pgfscope}%
\begin{pgfscope}%
\pgfpathrectangle{\pgfqpoint{0.765000in}{0.660000in}}{\pgfqpoint{4.620000in}{4.620000in}}%
\pgfusepath{clip}%
\pgfsetbuttcap%
\pgfsetroundjoin%
\definecolor{currentfill}{rgb}{1.000000,0.894118,0.788235}%
\pgfsetfillcolor{currentfill}%
\pgfsetlinewidth{0.000000pt}%
\definecolor{currentstroke}{rgb}{1.000000,0.894118,0.788235}%
\pgfsetstrokecolor{currentstroke}%
\pgfsetdash{}{0pt}%
\pgfpathmoveto{\pgfqpoint{4.096385in}{3.234039in}}%
\pgfpathlineto{\pgfqpoint{4.123332in}{3.218481in}}%
\pgfpathlineto{\pgfqpoint{4.123332in}{3.249597in}}%
\pgfpathlineto{\pgfqpoint{4.096385in}{3.265155in}}%
\pgfpathlineto{\pgfqpoint{4.096385in}{3.234039in}}%
\pgfpathclose%
\pgfusepath{fill}%
\end{pgfscope}%
\begin{pgfscope}%
\pgfpathrectangle{\pgfqpoint{0.765000in}{0.660000in}}{\pgfqpoint{4.620000in}{4.620000in}}%
\pgfusepath{clip}%
\pgfsetbuttcap%
\pgfsetroundjoin%
\definecolor{currentfill}{rgb}{1.000000,0.894118,0.788235}%
\pgfsetfillcolor{currentfill}%
\pgfsetlinewidth{0.000000pt}%
\definecolor{currentstroke}{rgb}{1.000000,0.894118,0.788235}%
\pgfsetstrokecolor{currentstroke}%
\pgfsetdash{}{0pt}%
\pgfpathmoveto{\pgfqpoint{4.007374in}{3.052356in}}%
\pgfpathlineto{\pgfqpoint{3.980427in}{3.036798in}}%
\pgfpathlineto{\pgfqpoint{4.096385in}{3.234039in}}%
\pgfpathlineto{\pgfqpoint{4.123332in}{3.249597in}}%
\pgfpathlineto{\pgfqpoint{4.007374in}{3.052356in}}%
\pgfpathclose%
\pgfusepath{fill}%
\end{pgfscope}%
\begin{pgfscope}%
\pgfpathrectangle{\pgfqpoint{0.765000in}{0.660000in}}{\pgfqpoint{4.620000in}{4.620000in}}%
\pgfusepath{clip}%
\pgfsetbuttcap%
\pgfsetroundjoin%
\definecolor{currentfill}{rgb}{1.000000,0.894118,0.788235}%
\pgfsetfillcolor{currentfill}%
\pgfsetlinewidth{0.000000pt}%
\definecolor{currentstroke}{rgb}{1.000000,0.894118,0.788235}%
\pgfsetstrokecolor{currentstroke}%
\pgfsetdash{}{0pt}%
\pgfpathmoveto{\pgfqpoint{4.034321in}{3.036798in}}%
\pgfpathlineto{\pgfqpoint{4.007374in}{3.052356in}}%
\pgfpathlineto{\pgfqpoint{4.123332in}{3.249597in}}%
\pgfpathlineto{\pgfqpoint{4.150279in}{3.234039in}}%
\pgfpathlineto{\pgfqpoint{4.034321in}{3.036798in}}%
\pgfpathclose%
\pgfusepath{fill}%
\end{pgfscope}%
\begin{pgfscope}%
\pgfpathrectangle{\pgfqpoint{0.765000in}{0.660000in}}{\pgfqpoint{4.620000in}{4.620000in}}%
\pgfusepath{clip}%
\pgfsetbuttcap%
\pgfsetroundjoin%
\definecolor{currentfill}{rgb}{1.000000,0.894118,0.788235}%
\pgfsetfillcolor{currentfill}%
\pgfsetlinewidth{0.000000pt}%
\definecolor{currentstroke}{rgb}{1.000000,0.894118,0.788235}%
\pgfsetstrokecolor{currentstroke}%
\pgfsetdash{}{0pt}%
\pgfpathmoveto{\pgfqpoint{4.007374in}{3.052356in}}%
\pgfpathlineto{\pgfqpoint{4.007374in}{3.083472in}}%
\pgfpathlineto{\pgfqpoint{4.123332in}{3.280713in}}%
\pgfpathlineto{\pgfqpoint{4.150279in}{3.234039in}}%
\pgfpathlineto{\pgfqpoint{4.007374in}{3.052356in}}%
\pgfpathclose%
\pgfusepath{fill}%
\end{pgfscope}%
\begin{pgfscope}%
\pgfpathrectangle{\pgfqpoint{0.765000in}{0.660000in}}{\pgfqpoint{4.620000in}{4.620000in}}%
\pgfusepath{clip}%
\pgfsetbuttcap%
\pgfsetroundjoin%
\definecolor{currentfill}{rgb}{1.000000,0.894118,0.788235}%
\pgfsetfillcolor{currentfill}%
\pgfsetlinewidth{0.000000pt}%
\definecolor{currentstroke}{rgb}{1.000000,0.894118,0.788235}%
\pgfsetstrokecolor{currentstroke}%
\pgfsetdash{}{0pt}%
\pgfpathmoveto{\pgfqpoint{4.034321in}{3.036798in}}%
\pgfpathlineto{\pgfqpoint{4.034321in}{3.067914in}}%
\pgfpathlineto{\pgfqpoint{4.150279in}{3.265155in}}%
\pgfpathlineto{\pgfqpoint{4.123332in}{3.249597in}}%
\pgfpathlineto{\pgfqpoint{4.034321in}{3.036798in}}%
\pgfpathclose%
\pgfusepath{fill}%
\end{pgfscope}%
\begin{pgfscope}%
\pgfpathrectangle{\pgfqpoint{0.765000in}{0.660000in}}{\pgfqpoint{4.620000in}{4.620000in}}%
\pgfusepath{clip}%
\pgfsetbuttcap%
\pgfsetroundjoin%
\definecolor{currentfill}{rgb}{1.000000,0.894118,0.788235}%
\pgfsetfillcolor{currentfill}%
\pgfsetlinewidth{0.000000pt}%
\definecolor{currentstroke}{rgb}{1.000000,0.894118,0.788235}%
\pgfsetstrokecolor{currentstroke}%
\pgfsetdash{}{0pt}%
\pgfpathmoveto{\pgfqpoint{3.980427in}{3.036798in}}%
\pgfpathlineto{\pgfqpoint{4.007374in}{3.021240in}}%
\pgfpathlineto{\pgfqpoint{4.123332in}{3.218481in}}%
\pgfpathlineto{\pgfqpoint{4.096385in}{3.234039in}}%
\pgfpathlineto{\pgfqpoint{3.980427in}{3.036798in}}%
\pgfpathclose%
\pgfusepath{fill}%
\end{pgfscope}%
\begin{pgfscope}%
\pgfpathrectangle{\pgfqpoint{0.765000in}{0.660000in}}{\pgfqpoint{4.620000in}{4.620000in}}%
\pgfusepath{clip}%
\pgfsetbuttcap%
\pgfsetroundjoin%
\definecolor{currentfill}{rgb}{1.000000,0.894118,0.788235}%
\pgfsetfillcolor{currentfill}%
\pgfsetlinewidth{0.000000pt}%
\definecolor{currentstroke}{rgb}{1.000000,0.894118,0.788235}%
\pgfsetstrokecolor{currentstroke}%
\pgfsetdash{}{0pt}%
\pgfpathmoveto{\pgfqpoint{4.007374in}{3.021240in}}%
\pgfpathlineto{\pgfqpoint{4.034321in}{3.036798in}}%
\pgfpathlineto{\pgfqpoint{4.150279in}{3.234039in}}%
\pgfpathlineto{\pgfqpoint{4.123332in}{3.218481in}}%
\pgfpathlineto{\pgfqpoint{4.007374in}{3.021240in}}%
\pgfpathclose%
\pgfusepath{fill}%
\end{pgfscope}%
\begin{pgfscope}%
\pgfpathrectangle{\pgfqpoint{0.765000in}{0.660000in}}{\pgfqpoint{4.620000in}{4.620000in}}%
\pgfusepath{clip}%
\pgfsetbuttcap%
\pgfsetroundjoin%
\definecolor{currentfill}{rgb}{1.000000,0.894118,0.788235}%
\pgfsetfillcolor{currentfill}%
\pgfsetlinewidth{0.000000pt}%
\definecolor{currentstroke}{rgb}{1.000000,0.894118,0.788235}%
\pgfsetstrokecolor{currentstroke}%
\pgfsetdash{}{0pt}%
\pgfpathmoveto{\pgfqpoint{4.007374in}{3.083472in}}%
\pgfpathlineto{\pgfqpoint{3.980427in}{3.067914in}}%
\pgfpathlineto{\pgfqpoint{4.096385in}{3.265155in}}%
\pgfpathlineto{\pgfqpoint{4.123332in}{3.280713in}}%
\pgfpathlineto{\pgfqpoint{4.007374in}{3.083472in}}%
\pgfpathclose%
\pgfusepath{fill}%
\end{pgfscope}%
\begin{pgfscope}%
\pgfpathrectangle{\pgfqpoint{0.765000in}{0.660000in}}{\pgfqpoint{4.620000in}{4.620000in}}%
\pgfusepath{clip}%
\pgfsetbuttcap%
\pgfsetroundjoin%
\definecolor{currentfill}{rgb}{1.000000,0.894118,0.788235}%
\pgfsetfillcolor{currentfill}%
\pgfsetlinewidth{0.000000pt}%
\definecolor{currentstroke}{rgb}{1.000000,0.894118,0.788235}%
\pgfsetstrokecolor{currentstroke}%
\pgfsetdash{}{0pt}%
\pgfpathmoveto{\pgfqpoint{4.034321in}{3.067914in}}%
\pgfpathlineto{\pgfqpoint{4.007374in}{3.083472in}}%
\pgfpathlineto{\pgfqpoint{4.123332in}{3.280713in}}%
\pgfpathlineto{\pgfqpoint{4.150279in}{3.265155in}}%
\pgfpathlineto{\pgfqpoint{4.034321in}{3.067914in}}%
\pgfpathclose%
\pgfusepath{fill}%
\end{pgfscope}%
\begin{pgfscope}%
\pgfpathrectangle{\pgfqpoint{0.765000in}{0.660000in}}{\pgfqpoint{4.620000in}{4.620000in}}%
\pgfusepath{clip}%
\pgfsetbuttcap%
\pgfsetroundjoin%
\definecolor{currentfill}{rgb}{1.000000,0.894118,0.788235}%
\pgfsetfillcolor{currentfill}%
\pgfsetlinewidth{0.000000pt}%
\definecolor{currentstroke}{rgb}{1.000000,0.894118,0.788235}%
\pgfsetstrokecolor{currentstroke}%
\pgfsetdash{}{0pt}%
\pgfpathmoveto{\pgfqpoint{3.980427in}{3.036798in}}%
\pgfpathlineto{\pgfqpoint{3.980427in}{3.067914in}}%
\pgfpathlineto{\pgfqpoint{4.096385in}{3.265155in}}%
\pgfpathlineto{\pgfqpoint{4.123332in}{3.218481in}}%
\pgfpathlineto{\pgfqpoint{3.980427in}{3.036798in}}%
\pgfpathclose%
\pgfusepath{fill}%
\end{pgfscope}%
\begin{pgfscope}%
\pgfpathrectangle{\pgfqpoint{0.765000in}{0.660000in}}{\pgfqpoint{4.620000in}{4.620000in}}%
\pgfusepath{clip}%
\pgfsetbuttcap%
\pgfsetroundjoin%
\definecolor{currentfill}{rgb}{1.000000,0.894118,0.788235}%
\pgfsetfillcolor{currentfill}%
\pgfsetlinewidth{0.000000pt}%
\definecolor{currentstroke}{rgb}{1.000000,0.894118,0.788235}%
\pgfsetstrokecolor{currentstroke}%
\pgfsetdash{}{0pt}%
\pgfpathmoveto{\pgfqpoint{4.007374in}{3.021240in}}%
\pgfpathlineto{\pgfqpoint{4.007374in}{3.052356in}}%
\pgfpathlineto{\pgfqpoint{4.123332in}{3.249597in}}%
\pgfpathlineto{\pgfqpoint{4.096385in}{3.234039in}}%
\pgfpathlineto{\pgfqpoint{4.007374in}{3.021240in}}%
\pgfpathclose%
\pgfusepath{fill}%
\end{pgfscope}%
\begin{pgfscope}%
\pgfpathrectangle{\pgfqpoint{0.765000in}{0.660000in}}{\pgfqpoint{4.620000in}{4.620000in}}%
\pgfusepath{clip}%
\pgfsetbuttcap%
\pgfsetroundjoin%
\definecolor{currentfill}{rgb}{1.000000,0.894118,0.788235}%
\pgfsetfillcolor{currentfill}%
\pgfsetlinewidth{0.000000pt}%
\definecolor{currentstroke}{rgb}{1.000000,0.894118,0.788235}%
\pgfsetstrokecolor{currentstroke}%
\pgfsetdash{}{0pt}%
\pgfpathmoveto{\pgfqpoint{4.007374in}{3.052356in}}%
\pgfpathlineto{\pgfqpoint{4.034321in}{3.067914in}}%
\pgfpathlineto{\pgfqpoint{4.150279in}{3.265155in}}%
\pgfpathlineto{\pgfqpoint{4.123332in}{3.249597in}}%
\pgfpathlineto{\pgfqpoint{4.007374in}{3.052356in}}%
\pgfpathclose%
\pgfusepath{fill}%
\end{pgfscope}%
\begin{pgfscope}%
\pgfpathrectangle{\pgfqpoint{0.765000in}{0.660000in}}{\pgfqpoint{4.620000in}{4.620000in}}%
\pgfusepath{clip}%
\pgfsetbuttcap%
\pgfsetroundjoin%
\definecolor{currentfill}{rgb}{1.000000,0.894118,0.788235}%
\pgfsetfillcolor{currentfill}%
\pgfsetlinewidth{0.000000pt}%
\definecolor{currentstroke}{rgb}{1.000000,0.894118,0.788235}%
\pgfsetstrokecolor{currentstroke}%
\pgfsetdash{}{0pt}%
\pgfpathmoveto{\pgfqpoint{3.980427in}{3.067914in}}%
\pgfpathlineto{\pgfqpoint{4.007374in}{3.052356in}}%
\pgfpathlineto{\pgfqpoint{4.123332in}{3.249597in}}%
\pgfpathlineto{\pgfqpoint{4.096385in}{3.265155in}}%
\pgfpathlineto{\pgfqpoint{3.980427in}{3.067914in}}%
\pgfpathclose%
\pgfusepath{fill}%
\end{pgfscope}%
\begin{pgfscope}%
\pgfpathrectangle{\pgfqpoint{0.765000in}{0.660000in}}{\pgfqpoint{4.620000in}{4.620000in}}%
\pgfusepath{clip}%
\pgfsetbuttcap%
\pgfsetroundjoin%
\definecolor{currentfill}{rgb}{1.000000,0.894118,0.788235}%
\pgfsetfillcolor{currentfill}%
\pgfsetlinewidth{0.000000pt}%
\definecolor{currentstroke}{rgb}{1.000000,0.894118,0.788235}%
\pgfsetstrokecolor{currentstroke}%
\pgfsetdash{}{0pt}%
\pgfpathmoveto{\pgfqpoint{4.007207in}{3.052100in}}%
\pgfpathlineto{\pgfqpoint{3.980260in}{3.036542in}}%
\pgfpathlineto{\pgfqpoint{4.007207in}{3.020984in}}%
\pgfpathlineto{\pgfqpoint{4.034154in}{3.036542in}}%
\pgfpathlineto{\pgfqpoint{4.007207in}{3.052100in}}%
\pgfpathclose%
\pgfusepath{fill}%
\end{pgfscope}%
\begin{pgfscope}%
\pgfpathrectangle{\pgfqpoint{0.765000in}{0.660000in}}{\pgfqpoint{4.620000in}{4.620000in}}%
\pgfusepath{clip}%
\pgfsetbuttcap%
\pgfsetroundjoin%
\definecolor{currentfill}{rgb}{1.000000,0.894118,0.788235}%
\pgfsetfillcolor{currentfill}%
\pgfsetlinewidth{0.000000pt}%
\definecolor{currentstroke}{rgb}{1.000000,0.894118,0.788235}%
\pgfsetstrokecolor{currentstroke}%
\pgfsetdash{}{0pt}%
\pgfpathmoveto{\pgfqpoint{4.007207in}{3.052100in}}%
\pgfpathlineto{\pgfqpoint{3.980260in}{3.036542in}}%
\pgfpathlineto{\pgfqpoint{3.980260in}{3.067658in}}%
\pgfpathlineto{\pgfqpoint{4.007207in}{3.083216in}}%
\pgfpathlineto{\pgfqpoint{4.007207in}{3.052100in}}%
\pgfpathclose%
\pgfusepath{fill}%
\end{pgfscope}%
\begin{pgfscope}%
\pgfpathrectangle{\pgfqpoint{0.765000in}{0.660000in}}{\pgfqpoint{4.620000in}{4.620000in}}%
\pgfusepath{clip}%
\pgfsetbuttcap%
\pgfsetroundjoin%
\definecolor{currentfill}{rgb}{1.000000,0.894118,0.788235}%
\pgfsetfillcolor{currentfill}%
\pgfsetlinewidth{0.000000pt}%
\definecolor{currentstroke}{rgb}{1.000000,0.894118,0.788235}%
\pgfsetstrokecolor{currentstroke}%
\pgfsetdash{}{0pt}%
\pgfpathmoveto{\pgfqpoint{4.007207in}{3.052100in}}%
\pgfpathlineto{\pgfqpoint{4.034154in}{3.036542in}}%
\pgfpathlineto{\pgfqpoint{4.034154in}{3.067658in}}%
\pgfpathlineto{\pgfqpoint{4.007207in}{3.083216in}}%
\pgfpathlineto{\pgfqpoint{4.007207in}{3.052100in}}%
\pgfpathclose%
\pgfusepath{fill}%
\end{pgfscope}%
\begin{pgfscope}%
\pgfpathrectangle{\pgfqpoint{0.765000in}{0.660000in}}{\pgfqpoint{4.620000in}{4.620000in}}%
\pgfusepath{clip}%
\pgfsetbuttcap%
\pgfsetroundjoin%
\definecolor{currentfill}{rgb}{1.000000,0.894118,0.788235}%
\pgfsetfillcolor{currentfill}%
\pgfsetlinewidth{0.000000pt}%
\definecolor{currentstroke}{rgb}{1.000000,0.894118,0.788235}%
\pgfsetstrokecolor{currentstroke}%
\pgfsetdash{}{0pt}%
\pgfpathmoveto{\pgfqpoint{3.890923in}{2.871313in}}%
\pgfpathlineto{\pgfqpoint{3.863975in}{2.855755in}}%
\pgfpathlineto{\pgfqpoint{3.890923in}{2.840197in}}%
\pgfpathlineto{\pgfqpoint{3.917870in}{2.855755in}}%
\pgfpathlineto{\pgfqpoint{3.890923in}{2.871313in}}%
\pgfpathclose%
\pgfusepath{fill}%
\end{pgfscope}%
\begin{pgfscope}%
\pgfpathrectangle{\pgfqpoint{0.765000in}{0.660000in}}{\pgfqpoint{4.620000in}{4.620000in}}%
\pgfusepath{clip}%
\pgfsetbuttcap%
\pgfsetroundjoin%
\definecolor{currentfill}{rgb}{1.000000,0.894118,0.788235}%
\pgfsetfillcolor{currentfill}%
\pgfsetlinewidth{0.000000pt}%
\definecolor{currentstroke}{rgb}{1.000000,0.894118,0.788235}%
\pgfsetstrokecolor{currentstroke}%
\pgfsetdash{}{0pt}%
\pgfpathmoveto{\pgfqpoint{3.890923in}{2.871313in}}%
\pgfpathlineto{\pgfqpoint{3.863975in}{2.855755in}}%
\pgfpathlineto{\pgfqpoint{3.863975in}{2.886871in}}%
\pgfpathlineto{\pgfqpoint{3.890923in}{2.902429in}}%
\pgfpathlineto{\pgfqpoint{3.890923in}{2.871313in}}%
\pgfpathclose%
\pgfusepath{fill}%
\end{pgfscope}%
\begin{pgfscope}%
\pgfpathrectangle{\pgfqpoint{0.765000in}{0.660000in}}{\pgfqpoint{4.620000in}{4.620000in}}%
\pgfusepath{clip}%
\pgfsetbuttcap%
\pgfsetroundjoin%
\definecolor{currentfill}{rgb}{1.000000,0.894118,0.788235}%
\pgfsetfillcolor{currentfill}%
\pgfsetlinewidth{0.000000pt}%
\definecolor{currentstroke}{rgb}{1.000000,0.894118,0.788235}%
\pgfsetstrokecolor{currentstroke}%
\pgfsetdash{}{0pt}%
\pgfpathmoveto{\pgfqpoint{3.890923in}{2.871313in}}%
\pgfpathlineto{\pgfqpoint{3.917870in}{2.855755in}}%
\pgfpathlineto{\pgfqpoint{3.917870in}{2.886871in}}%
\pgfpathlineto{\pgfqpoint{3.890923in}{2.902429in}}%
\pgfpathlineto{\pgfqpoint{3.890923in}{2.871313in}}%
\pgfpathclose%
\pgfusepath{fill}%
\end{pgfscope}%
\begin{pgfscope}%
\pgfpathrectangle{\pgfqpoint{0.765000in}{0.660000in}}{\pgfqpoint{4.620000in}{4.620000in}}%
\pgfusepath{clip}%
\pgfsetbuttcap%
\pgfsetroundjoin%
\definecolor{currentfill}{rgb}{1.000000,0.894118,0.788235}%
\pgfsetfillcolor{currentfill}%
\pgfsetlinewidth{0.000000pt}%
\definecolor{currentstroke}{rgb}{1.000000,0.894118,0.788235}%
\pgfsetstrokecolor{currentstroke}%
\pgfsetdash{}{0pt}%
\pgfpathmoveto{\pgfqpoint{4.007207in}{3.083216in}}%
\pgfpathlineto{\pgfqpoint{3.980260in}{3.067658in}}%
\pgfpathlineto{\pgfqpoint{4.007207in}{3.052100in}}%
\pgfpathlineto{\pgfqpoint{4.034154in}{3.067658in}}%
\pgfpathlineto{\pgfqpoint{4.007207in}{3.083216in}}%
\pgfpathclose%
\pgfusepath{fill}%
\end{pgfscope}%
\begin{pgfscope}%
\pgfpathrectangle{\pgfqpoint{0.765000in}{0.660000in}}{\pgfqpoint{4.620000in}{4.620000in}}%
\pgfusepath{clip}%
\pgfsetbuttcap%
\pgfsetroundjoin%
\definecolor{currentfill}{rgb}{1.000000,0.894118,0.788235}%
\pgfsetfillcolor{currentfill}%
\pgfsetlinewidth{0.000000pt}%
\definecolor{currentstroke}{rgb}{1.000000,0.894118,0.788235}%
\pgfsetstrokecolor{currentstroke}%
\pgfsetdash{}{0pt}%
\pgfpathmoveto{\pgfqpoint{4.007207in}{3.020984in}}%
\pgfpathlineto{\pgfqpoint{4.034154in}{3.036542in}}%
\pgfpathlineto{\pgfqpoint{4.034154in}{3.067658in}}%
\pgfpathlineto{\pgfqpoint{4.007207in}{3.052100in}}%
\pgfpathlineto{\pgfqpoint{4.007207in}{3.020984in}}%
\pgfpathclose%
\pgfusepath{fill}%
\end{pgfscope}%
\begin{pgfscope}%
\pgfpathrectangle{\pgfqpoint{0.765000in}{0.660000in}}{\pgfqpoint{4.620000in}{4.620000in}}%
\pgfusepath{clip}%
\pgfsetbuttcap%
\pgfsetroundjoin%
\definecolor{currentfill}{rgb}{1.000000,0.894118,0.788235}%
\pgfsetfillcolor{currentfill}%
\pgfsetlinewidth{0.000000pt}%
\definecolor{currentstroke}{rgb}{1.000000,0.894118,0.788235}%
\pgfsetstrokecolor{currentstroke}%
\pgfsetdash{}{0pt}%
\pgfpathmoveto{\pgfqpoint{3.980260in}{3.036542in}}%
\pgfpathlineto{\pgfqpoint{4.007207in}{3.020984in}}%
\pgfpathlineto{\pgfqpoint{4.007207in}{3.052100in}}%
\pgfpathlineto{\pgfqpoint{3.980260in}{3.067658in}}%
\pgfpathlineto{\pgfqpoint{3.980260in}{3.036542in}}%
\pgfpathclose%
\pgfusepath{fill}%
\end{pgfscope}%
\begin{pgfscope}%
\pgfpathrectangle{\pgfqpoint{0.765000in}{0.660000in}}{\pgfqpoint{4.620000in}{4.620000in}}%
\pgfusepath{clip}%
\pgfsetbuttcap%
\pgfsetroundjoin%
\definecolor{currentfill}{rgb}{1.000000,0.894118,0.788235}%
\pgfsetfillcolor{currentfill}%
\pgfsetlinewidth{0.000000pt}%
\definecolor{currentstroke}{rgb}{1.000000,0.894118,0.788235}%
\pgfsetstrokecolor{currentstroke}%
\pgfsetdash{}{0pt}%
\pgfpathmoveto{\pgfqpoint{3.890923in}{2.902429in}}%
\pgfpathlineto{\pgfqpoint{3.863975in}{2.886871in}}%
\pgfpathlineto{\pgfqpoint{3.890923in}{2.871313in}}%
\pgfpathlineto{\pgfqpoint{3.917870in}{2.886871in}}%
\pgfpathlineto{\pgfqpoint{3.890923in}{2.902429in}}%
\pgfpathclose%
\pgfusepath{fill}%
\end{pgfscope}%
\begin{pgfscope}%
\pgfpathrectangle{\pgfqpoint{0.765000in}{0.660000in}}{\pgfqpoint{4.620000in}{4.620000in}}%
\pgfusepath{clip}%
\pgfsetbuttcap%
\pgfsetroundjoin%
\definecolor{currentfill}{rgb}{1.000000,0.894118,0.788235}%
\pgfsetfillcolor{currentfill}%
\pgfsetlinewidth{0.000000pt}%
\definecolor{currentstroke}{rgb}{1.000000,0.894118,0.788235}%
\pgfsetstrokecolor{currentstroke}%
\pgfsetdash{}{0pt}%
\pgfpathmoveto{\pgfqpoint{3.890923in}{2.840197in}}%
\pgfpathlineto{\pgfqpoint{3.917870in}{2.855755in}}%
\pgfpathlineto{\pgfqpoint{3.917870in}{2.886871in}}%
\pgfpathlineto{\pgfqpoint{3.890923in}{2.871313in}}%
\pgfpathlineto{\pgfqpoint{3.890923in}{2.840197in}}%
\pgfpathclose%
\pgfusepath{fill}%
\end{pgfscope}%
\begin{pgfscope}%
\pgfpathrectangle{\pgfqpoint{0.765000in}{0.660000in}}{\pgfqpoint{4.620000in}{4.620000in}}%
\pgfusepath{clip}%
\pgfsetbuttcap%
\pgfsetroundjoin%
\definecolor{currentfill}{rgb}{1.000000,0.894118,0.788235}%
\pgfsetfillcolor{currentfill}%
\pgfsetlinewidth{0.000000pt}%
\definecolor{currentstroke}{rgb}{1.000000,0.894118,0.788235}%
\pgfsetstrokecolor{currentstroke}%
\pgfsetdash{}{0pt}%
\pgfpathmoveto{\pgfqpoint{3.863975in}{2.855755in}}%
\pgfpathlineto{\pgfqpoint{3.890923in}{2.840197in}}%
\pgfpathlineto{\pgfqpoint{3.890923in}{2.871313in}}%
\pgfpathlineto{\pgfqpoint{3.863975in}{2.886871in}}%
\pgfpathlineto{\pgfqpoint{3.863975in}{2.855755in}}%
\pgfpathclose%
\pgfusepath{fill}%
\end{pgfscope}%
\begin{pgfscope}%
\pgfpathrectangle{\pgfqpoint{0.765000in}{0.660000in}}{\pgfqpoint{4.620000in}{4.620000in}}%
\pgfusepath{clip}%
\pgfsetbuttcap%
\pgfsetroundjoin%
\definecolor{currentfill}{rgb}{1.000000,0.894118,0.788235}%
\pgfsetfillcolor{currentfill}%
\pgfsetlinewidth{0.000000pt}%
\definecolor{currentstroke}{rgb}{1.000000,0.894118,0.788235}%
\pgfsetstrokecolor{currentstroke}%
\pgfsetdash{}{0pt}%
\pgfpathmoveto{\pgfqpoint{4.007207in}{3.052100in}}%
\pgfpathlineto{\pgfqpoint{3.980260in}{3.036542in}}%
\pgfpathlineto{\pgfqpoint{3.863975in}{2.855755in}}%
\pgfpathlineto{\pgfqpoint{3.890923in}{2.871313in}}%
\pgfpathlineto{\pgfqpoint{4.007207in}{3.052100in}}%
\pgfpathclose%
\pgfusepath{fill}%
\end{pgfscope}%
\begin{pgfscope}%
\pgfpathrectangle{\pgfqpoint{0.765000in}{0.660000in}}{\pgfqpoint{4.620000in}{4.620000in}}%
\pgfusepath{clip}%
\pgfsetbuttcap%
\pgfsetroundjoin%
\definecolor{currentfill}{rgb}{1.000000,0.894118,0.788235}%
\pgfsetfillcolor{currentfill}%
\pgfsetlinewidth{0.000000pt}%
\definecolor{currentstroke}{rgb}{1.000000,0.894118,0.788235}%
\pgfsetstrokecolor{currentstroke}%
\pgfsetdash{}{0pt}%
\pgfpathmoveto{\pgfqpoint{4.034154in}{3.036542in}}%
\pgfpathlineto{\pgfqpoint{4.007207in}{3.052100in}}%
\pgfpathlineto{\pgfqpoint{3.890923in}{2.871313in}}%
\pgfpathlineto{\pgfqpoint{3.917870in}{2.855755in}}%
\pgfpathlineto{\pgfqpoint{4.034154in}{3.036542in}}%
\pgfpathclose%
\pgfusepath{fill}%
\end{pgfscope}%
\begin{pgfscope}%
\pgfpathrectangle{\pgfqpoint{0.765000in}{0.660000in}}{\pgfqpoint{4.620000in}{4.620000in}}%
\pgfusepath{clip}%
\pgfsetbuttcap%
\pgfsetroundjoin%
\definecolor{currentfill}{rgb}{1.000000,0.894118,0.788235}%
\pgfsetfillcolor{currentfill}%
\pgfsetlinewidth{0.000000pt}%
\definecolor{currentstroke}{rgb}{1.000000,0.894118,0.788235}%
\pgfsetstrokecolor{currentstroke}%
\pgfsetdash{}{0pt}%
\pgfpathmoveto{\pgfqpoint{4.007207in}{3.052100in}}%
\pgfpathlineto{\pgfqpoint{4.007207in}{3.083216in}}%
\pgfpathlineto{\pgfqpoint{3.890923in}{2.902429in}}%
\pgfpathlineto{\pgfqpoint{3.917870in}{2.855755in}}%
\pgfpathlineto{\pgfqpoint{4.007207in}{3.052100in}}%
\pgfpathclose%
\pgfusepath{fill}%
\end{pgfscope}%
\begin{pgfscope}%
\pgfpathrectangle{\pgfqpoint{0.765000in}{0.660000in}}{\pgfqpoint{4.620000in}{4.620000in}}%
\pgfusepath{clip}%
\pgfsetbuttcap%
\pgfsetroundjoin%
\definecolor{currentfill}{rgb}{1.000000,0.894118,0.788235}%
\pgfsetfillcolor{currentfill}%
\pgfsetlinewidth{0.000000pt}%
\definecolor{currentstroke}{rgb}{1.000000,0.894118,0.788235}%
\pgfsetstrokecolor{currentstroke}%
\pgfsetdash{}{0pt}%
\pgfpathmoveto{\pgfqpoint{4.034154in}{3.036542in}}%
\pgfpathlineto{\pgfqpoint{4.034154in}{3.067658in}}%
\pgfpathlineto{\pgfqpoint{3.917870in}{2.886871in}}%
\pgfpathlineto{\pgfqpoint{3.890923in}{2.871313in}}%
\pgfpathlineto{\pgfqpoint{4.034154in}{3.036542in}}%
\pgfpathclose%
\pgfusepath{fill}%
\end{pgfscope}%
\begin{pgfscope}%
\pgfpathrectangle{\pgfqpoint{0.765000in}{0.660000in}}{\pgfqpoint{4.620000in}{4.620000in}}%
\pgfusepath{clip}%
\pgfsetbuttcap%
\pgfsetroundjoin%
\definecolor{currentfill}{rgb}{1.000000,0.894118,0.788235}%
\pgfsetfillcolor{currentfill}%
\pgfsetlinewidth{0.000000pt}%
\definecolor{currentstroke}{rgb}{1.000000,0.894118,0.788235}%
\pgfsetstrokecolor{currentstroke}%
\pgfsetdash{}{0pt}%
\pgfpathmoveto{\pgfqpoint{3.980260in}{3.036542in}}%
\pgfpathlineto{\pgfqpoint{4.007207in}{3.020984in}}%
\pgfpathlineto{\pgfqpoint{3.890923in}{2.840197in}}%
\pgfpathlineto{\pgfqpoint{3.863975in}{2.855755in}}%
\pgfpathlineto{\pgfqpoint{3.980260in}{3.036542in}}%
\pgfpathclose%
\pgfusepath{fill}%
\end{pgfscope}%
\begin{pgfscope}%
\pgfpathrectangle{\pgfqpoint{0.765000in}{0.660000in}}{\pgfqpoint{4.620000in}{4.620000in}}%
\pgfusepath{clip}%
\pgfsetbuttcap%
\pgfsetroundjoin%
\definecolor{currentfill}{rgb}{1.000000,0.894118,0.788235}%
\pgfsetfillcolor{currentfill}%
\pgfsetlinewidth{0.000000pt}%
\definecolor{currentstroke}{rgb}{1.000000,0.894118,0.788235}%
\pgfsetstrokecolor{currentstroke}%
\pgfsetdash{}{0pt}%
\pgfpathmoveto{\pgfqpoint{4.007207in}{3.020984in}}%
\pgfpathlineto{\pgfqpoint{4.034154in}{3.036542in}}%
\pgfpathlineto{\pgfqpoint{3.917870in}{2.855755in}}%
\pgfpathlineto{\pgfqpoint{3.890923in}{2.840197in}}%
\pgfpathlineto{\pgfqpoint{4.007207in}{3.020984in}}%
\pgfpathclose%
\pgfusepath{fill}%
\end{pgfscope}%
\begin{pgfscope}%
\pgfpathrectangle{\pgfqpoint{0.765000in}{0.660000in}}{\pgfqpoint{4.620000in}{4.620000in}}%
\pgfusepath{clip}%
\pgfsetbuttcap%
\pgfsetroundjoin%
\definecolor{currentfill}{rgb}{1.000000,0.894118,0.788235}%
\pgfsetfillcolor{currentfill}%
\pgfsetlinewidth{0.000000pt}%
\definecolor{currentstroke}{rgb}{1.000000,0.894118,0.788235}%
\pgfsetstrokecolor{currentstroke}%
\pgfsetdash{}{0pt}%
\pgfpathmoveto{\pgfqpoint{4.007207in}{3.083216in}}%
\pgfpathlineto{\pgfqpoint{3.980260in}{3.067658in}}%
\pgfpathlineto{\pgfqpoint{3.863975in}{2.886871in}}%
\pgfpathlineto{\pgfqpoint{3.890923in}{2.902429in}}%
\pgfpathlineto{\pgfqpoint{4.007207in}{3.083216in}}%
\pgfpathclose%
\pgfusepath{fill}%
\end{pgfscope}%
\begin{pgfscope}%
\pgfpathrectangle{\pgfqpoint{0.765000in}{0.660000in}}{\pgfqpoint{4.620000in}{4.620000in}}%
\pgfusepath{clip}%
\pgfsetbuttcap%
\pgfsetroundjoin%
\definecolor{currentfill}{rgb}{1.000000,0.894118,0.788235}%
\pgfsetfillcolor{currentfill}%
\pgfsetlinewidth{0.000000pt}%
\definecolor{currentstroke}{rgb}{1.000000,0.894118,0.788235}%
\pgfsetstrokecolor{currentstroke}%
\pgfsetdash{}{0pt}%
\pgfpathmoveto{\pgfqpoint{4.034154in}{3.067658in}}%
\pgfpathlineto{\pgfqpoint{4.007207in}{3.083216in}}%
\pgfpathlineto{\pgfqpoint{3.890923in}{2.902429in}}%
\pgfpathlineto{\pgfqpoint{3.917870in}{2.886871in}}%
\pgfpathlineto{\pgfqpoint{4.034154in}{3.067658in}}%
\pgfpathclose%
\pgfusepath{fill}%
\end{pgfscope}%
\begin{pgfscope}%
\pgfpathrectangle{\pgfqpoint{0.765000in}{0.660000in}}{\pgfqpoint{4.620000in}{4.620000in}}%
\pgfusepath{clip}%
\pgfsetbuttcap%
\pgfsetroundjoin%
\definecolor{currentfill}{rgb}{1.000000,0.894118,0.788235}%
\pgfsetfillcolor{currentfill}%
\pgfsetlinewidth{0.000000pt}%
\definecolor{currentstroke}{rgb}{1.000000,0.894118,0.788235}%
\pgfsetstrokecolor{currentstroke}%
\pgfsetdash{}{0pt}%
\pgfpathmoveto{\pgfqpoint{3.980260in}{3.036542in}}%
\pgfpathlineto{\pgfqpoint{3.980260in}{3.067658in}}%
\pgfpathlineto{\pgfqpoint{3.863975in}{2.886871in}}%
\pgfpathlineto{\pgfqpoint{3.890923in}{2.840197in}}%
\pgfpathlineto{\pgfqpoint{3.980260in}{3.036542in}}%
\pgfpathclose%
\pgfusepath{fill}%
\end{pgfscope}%
\begin{pgfscope}%
\pgfpathrectangle{\pgfqpoint{0.765000in}{0.660000in}}{\pgfqpoint{4.620000in}{4.620000in}}%
\pgfusepath{clip}%
\pgfsetbuttcap%
\pgfsetroundjoin%
\definecolor{currentfill}{rgb}{1.000000,0.894118,0.788235}%
\pgfsetfillcolor{currentfill}%
\pgfsetlinewidth{0.000000pt}%
\definecolor{currentstroke}{rgb}{1.000000,0.894118,0.788235}%
\pgfsetstrokecolor{currentstroke}%
\pgfsetdash{}{0pt}%
\pgfpathmoveto{\pgfqpoint{4.007207in}{3.020984in}}%
\pgfpathlineto{\pgfqpoint{4.007207in}{3.052100in}}%
\pgfpathlineto{\pgfqpoint{3.890923in}{2.871313in}}%
\pgfpathlineto{\pgfqpoint{3.863975in}{2.855755in}}%
\pgfpathlineto{\pgfqpoint{4.007207in}{3.020984in}}%
\pgfpathclose%
\pgfusepath{fill}%
\end{pgfscope}%
\begin{pgfscope}%
\pgfpathrectangle{\pgfqpoint{0.765000in}{0.660000in}}{\pgfqpoint{4.620000in}{4.620000in}}%
\pgfusepath{clip}%
\pgfsetbuttcap%
\pgfsetroundjoin%
\definecolor{currentfill}{rgb}{1.000000,0.894118,0.788235}%
\pgfsetfillcolor{currentfill}%
\pgfsetlinewidth{0.000000pt}%
\definecolor{currentstroke}{rgb}{1.000000,0.894118,0.788235}%
\pgfsetstrokecolor{currentstroke}%
\pgfsetdash{}{0pt}%
\pgfpathmoveto{\pgfqpoint{4.007207in}{3.052100in}}%
\pgfpathlineto{\pgfqpoint{4.034154in}{3.067658in}}%
\pgfpathlineto{\pgfqpoint{3.917870in}{2.886871in}}%
\pgfpathlineto{\pgfqpoint{3.890923in}{2.871313in}}%
\pgfpathlineto{\pgfqpoint{4.007207in}{3.052100in}}%
\pgfpathclose%
\pgfusepath{fill}%
\end{pgfscope}%
\begin{pgfscope}%
\pgfpathrectangle{\pgfqpoint{0.765000in}{0.660000in}}{\pgfqpoint{4.620000in}{4.620000in}}%
\pgfusepath{clip}%
\pgfsetbuttcap%
\pgfsetroundjoin%
\definecolor{currentfill}{rgb}{1.000000,0.894118,0.788235}%
\pgfsetfillcolor{currentfill}%
\pgfsetlinewidth{0.000000pt}%
\definecolor{currentstroke}{rgb}{1.000000,0.894118,0.788235}%
\pgfsetstrokecolor{currentstroke}%
\pgfsetdash{}{0pt}%
\pgfpathmoveto{\pgfqpoint{3.980260in}{3.067658in}}%
\pgfpathlineto{\pgfqpoint{4.007207in}{3.052100in}}%
\pgfpathlineto{\pgfqpoint{3.890923in}{2.871313in}}%
\pgfpathlineto{\pgfqpoint{3.863975in}{2.886871in}}%
\pgfpathlineto{\pgfqpoint{3.980260in}{3.067658in}}%
\pgfpathclose%
\pgfusepath{fill}%
\end{pgfscope}%
\begin{pgfscope}%
\pgfpathrectangle{\pgfqpoint{0.765000in}{0.660000in}}{\pgfqpoint{4.620000in}{4.620000in}}%
\pgfusepath{clip}%
\pgfsetbuttcap%
\pgfsetroundjoin%
\definecolor{currentfill}{rgb}{1.000000,0.894118,0.788235}%
\pgfsetfillcolor{currentfill}%
\pgfsetlinewidth{0.000000pt}%
\definecolor{currentstroke}{rgb}{1.000000,0.894118,0.788235}%
\pgfsetstrokecolor{currentstroke}%
\pgfsetdash{}{0pt}%
\pgfpathmoveto{\pgfqpoint{4.007207in}{3.052100in}}%
\pgfpathlineto{\pgfqpoint{3.980260in}{3.036542in}}%
\pgfpathlineto{\pgfqpoint{4.007207in}{3.020984in}}%
\pgfpathlineto{\pgfqpoint{4.034154in}{3.036542in}}%
\pgfpathlineto{\pgfqpoint{4.007207in}{3.052100in}}%
\pgfpathclose%
\pgfusepath{fill}%
\end{pgfscope}%
\begin{pgfscope}%
\pgfpathrectangle{\pgfqpoint{0.765000in}{0.660000in}}{\pgfqpoint{4.620000in}{4.620000in}}%
\pgfusepath{clip}%
\pgfsetbuttcap%
\pgfsetroundjoin%
\definecolor{currentfill}{rgb}{1.000000,0.894118,0.788235}%
\pgfsetfillcolor{currentfill}%
\pgfsetlinewidth{0.000000pt}%
\definecolor{currentstroke}{rgb}{1.000000,0.894118,0.788235}%
\pgfsetstrokecolor{currentstroke}%
\pgfsetdash{}{0pt}%
\pgfpathmoveto{\pgfqpoint{4.007207in}{3.052100in}}%
\pgfpathlineto{\pgfqpoint{3.980260in}{3.036542in}}%
\pgfpathlineto{\pgfqpoint{3.980260in}{3.067658in}}%
\pgfpathlineto{\pgfqpoint{4.007207in}{3.083216in}}%
\pgfpathlineto{\pgfqpoint{4.007207in}{3.052100in}}%
\pgfpathclose%
\pgfusepath{fill}%
\end{pgfscope}%
\begin{pgfscope}%
\pgfpathrectangle{\pgfqpoint{0.765000in}{0.660000in}}{\pgfqpoint{4.620000in}{4.620000in}}%
\pgfusepath{clip}%
\pgfsetbuttcap%
\pgfsetroundjoin%
\definecolor{currentfill}{rgb}{1.000000,0.894118,0.788235}%
\pgfsetfillcolor{currentfill}%
\pgfsetlinewidth{0.000000pt}%
\definecolor{currentstroke}{rgb}{1.000000,0.894118,0.788235}%
\pgfsetstrokecolor{currentstroke}%
\pgfsetdash{}{0pt}%
\pgfpathmoveto{\pgfqpoint{4.007207in}{3.052100in}}%
\pgfpathlineto{\pgfqpoint{4.034154in}{3.036542in}}%
\pgfpathlineto{\pgfqpoint{4.034154in}{3.067658in}}%
\pgfpathlineto{\pgfqpoint{4.007207in}{3.083216in}}%
\pgfpathlineto{\pgfqpoint{4.007207in}{3.052100in}}%
\pgfpathclose%
\pgfusepath{fill}%
\end{pgfscope}%
\begin{pgfscope}%
\pgfpathrectangle{\pgfqpoint{0.765000in}{0.660000in}}{\pgfqpoint{4.620000in}{4.620000in}}%
\pgfusepath{clip}%
\pgfsetbuttcap%
\pgfsetroundjoin%
\definecolor{currentfill}{rgb}{1.000000,0.894118,0.788235}%
\pgfsetfillcolor{currentfill}%
\pgfsetlinewidth{0.000000pt}%
\definecolor{currentstroke}{rgb}{1.000000,0.894118,0.788235}%
\pgfsetstrokecolor{currentstroke}%
\pgfsetdash{}{0pt}%
\pgfpathmoveto{\pgfqpoint{4.007374in}{3.052356in}}%
\pgfpathlineto{\pgfqpoint{3.980427in}{3.036798in}}%
\pgfpathlineto{\pgfqpoint{4.007374in}{3.021240in}}%
\pgfpathlineto{\pgfqpoint{4.034321in}{3.036798in}}%
\pgfpathlineto{\pgfqpoint{4.007374in}{3.052356in}}%
\pgfpathclose%
\pgfusepath{fill}%
\end{pgfscope}%
\begin{pgfscope}%
\pgfpathrectangle{\pgfqpoint{0.765000in}{0.660000in}}{\pgfqpoint{4.620000in}{4.620000in}}%
\pgfusepath{clip}%
\pgfsetbuttcap%
\pgfsetroundjoin%
\definecolor{currentfill}{rgb}{1.000000,0.894118,0.788235}%
\pgfsetfillcolor{currentfill}%
\pgfsetlinewidth{0.000000pt}%
\definecolor{currentstroke}{rgb}{1.000000,0.894118,0.788235}%
\pgfsetstrokecolor{currentstroke}%
\pgfsetdash{}{0pt}%
\pgfpathmoveto{\pgfqpoint{4.007374in}{3.052356in}}%
\pgfpathlineto{\pgfqpoint{3.980427in}{3.036798in}}%
\pgfpathlineto{\pgfqpoint{3.980427in}{3.067914in}}%
\pgfpathlineto{\pgfqpoint{4.007374in}{3.083472in}}%
\pgfpathlineto{\pgfqpoint{4.007374in}{3.052356in}}%
\pgfpathclose%
\pgfusepath{fill}%
\end{pgfscope}%
\begin{pgfscope}%
\pgfpathrectangle{\pgfqpoint{0.765000in}{0.660000in}}{\pgfqpoint{4.620000in}{4.620000in}}%
\pgfusepath{clip}%
\pgfsetbuttcap%
\pgfsetroundjoin%
\definecolor{currentfill}{rgb}{1.000000,0.894118,0.788235}%
\pgfsetfillcolor{currentfill}%
\pgfsetlinewidth{0.000000pt}%
\definecolor{currentstroke}{rgb}{1.000000,0.894118,0.788235}%
\pgfsetstrokecolor{currentstroke}%
\pgfsetdash{}{0pt}%
\pgfpathmoveto{\pgfqpoint{4.007374in}{3.052356in}}%
\pgfpathlineto{\pgfqpoint{4.034321in}{3.036798in}}%
\pgfpathlineto{\pgfqpoint{4.034321in}{3.067914in}}%
\pgfpathlineto{\pgfqpoint{4.007374in}{3.083472in}}%
\pgfpathlineto{\pgfqpoint{4.007374in}{3.052356in}}%
\pgfpathclose%
\pgfusepath{fill}%
\end{pgfscope}%
\begin{pgfscope}%
\pgfpathrectangle{\pgfqpoint{0.765000in}{0.660000in}}{\pgfqpoint{4.620000in}{4.620000in}}%
\pgfusepath{clip}%
\pgfsetbuttcap%
\pgfsetroundjoin%
\definecolor{currentfill}{rgb}{1.000000,0.894118,0.788235}%
\pgfsetfillcolor{currentfill}%
\pgfsetlinewidth{0.000000pt}%
\definecolor{currentstroke}{rgb}{1.000000,0.894118,0.788235}%
\pgfsetstrokecolor{currentstroke}%
\pgfsetdash{}{0pt}%
\pgfpathmoveto{\pgfqpoint{4.007207in}{3.083216in}}%
\pgfpathlineto{\pgfqpoint{3.980260in}{3.067658in}}%
\pgfpathlineto{\pgfqpoint{4.007207in}{3.052100in}}%
\pgfpathlineto{\pgfqpoint{4.034154in}{3.067658in}}%
\pgfpathlineto{\pgfqpoint{4.007207in}{3.083216in}}%
\pgfpathclose%
\pgfusepath{fill}%
\end{pgfscope}%
\begin{pgfscope}%
\pgfpathrectangle{\pgfqpoint{0.765000in}{0.660000in}}{\pgfqpoint{4.620000in}{4.620000in}}%
\pgfusepath{clip}%
\pgfsetbuttcap%
\pgfsetroundjoin%
\definecolor{currentfill}{rgb}{1.000000,0.894118,0.788235}%
\pgfsetfillcolor{currentfill}%
\pgfsetlinewidth{0.000000pt}%
\definecolor{currentstroke}{rgb}{1.000000,0.894118,0.788235}%
\pgfsetstrokecolor{currentstroke}%
\pgfsetdash{}{0pt}%
\pgfpathmoveto{\pgfqpoint{4.007207in}{3.020984in}}%
\pgfpathlineto{\pgfqpoint{4.034154in}{3.036542in}}%
\pgfpathlineto{\pgfqpoint{4.034154in}{3.067658in}}%
\pgfpathlineto{\pgfqpoint{4.007207in}{3.052100in}}%
\pgfpathlineto{\pgfqpoint{4.007207in}{3.020984in}}%
\pgfpathclose%
\pgfusepath{fill}%
\end{pgfscope}%
\begin{pgfscope}%
\pgfpathrectangle{\pgfqpoint{0.765000in}{0.660000in}}{\pgfqpoint{4.620000in}{4.620000in}}%
\pgfusepath{clip}%
\pgfsetbuttcap%
\pgfsetroundjoin%
\definecolor{currentfill}{rgb}{1.000000,0.894118,0.788235}%
\pgfsetfillcolor{currentfill}%
\pgfsetlinewidth{0.000000pt}%
\definecolor{currentstroke}{rgb}{1.000000,0.894118,0.788235}%
\pgfsetstrokecolor{currentstroke}%
\pgfsetdash{}{0pt}%
\pgfpathmoveto{\pgfqpoint{3.980260in}{3.036542in}}%
\pgfpathlineto{\pgfqpoint{4.007207in}{3.020984in}}%
\pgfpathlineto{\pgfqpoint{4.007207in}{3.052100in}}%
\pgfpathlineto{\pgfqpoint{3.980260in}{3.067658in}}%
\pgfpathlineto{\pgfqpoint{3.980260in}{3.036542in}}%
\pgfpathclose%
\pgfusepath{fill}%
\end{pgfscope}%
\begin{pgfscope}%
\pgfpathrectangle{\pgfqpoint{0.765000in}{0.660000in}}{\pgfqpoint{4.620000in}{4.620000in}}%
\pgfusepath{clip}%
\pgfsetbuttcap%
\pgfsetroundjoin%
\definecolor{currentfill}{rgb}{1.000000,0.894118,0.788235}%
\pgfsetfillcolor{currentfill}%
\pgfsetlinewidth{0.000000pt}%
\definecolor{currentstroke}{rgb}{1.000000,0.894118,0.788235}%
\pgfsetstrokecolor{currentstroke}%
\pgfsetdash{}{0pt}%
\pgfpathmoveto{\pgfqpoint{4.007374in}{3.083472in}}%
\pgfpathlineto{\pgfqpoint{3.980427in}{3.067914in}}%
\pgfpathlineto{\pgfqpoint{4.007374in}{3.052356in}}%
\pgfpathlineto{\pgfqpoint{4.034321in}{3.067914in}}%
\pgfpathlineto{\pgfqpoint{4.007374in}{3.083472in}}%
\pgfpathclose%
\pgfusepath{fill}%
\end{pgfscope}%
\begin{pgfscope}%
\pgfpathrectangle{\pgfqpoint{0.765000in}{0.660000in}}{\pgfqpoint{4.620000in}{4.620000in}}%
\pgfusepath{clip}%
\pgfsetbuttcap%
\pgfsetroundjoin%
\definecolor{currentfill}{rgb}{1.000000,0.894118,0.788235}%
\pgfsetfillcolor{currentfill}%
\pgfsetlinewidth{0.000000pt}%
\definecolor{currentstroke}{rgb}{1.000000,0.894118,0.788235}%
\pgfsetstrokecolor{currentstroke}%
\pgfsetdash{}{0pt}%
\pgfpathmoveto{\pgfqpoint{4.007374in}{3.021240in}}%
\pgfpathlineto{\pgfqpoint{4.034321in}{3.036798in}}%
\pgfpathlineto{\pgfqpoint{4.034321in}{3.067914in}}%
\pgfpathlineto{\pgfqpoint{4.007374in}{3.052356in}}%
\pgfpathlineto{\pgfqpoint{4.007374in}{3.021240in}}%
\pgfpathclose%
\pgfusepath{fill}%
\end{pgfscope}%
\begin{pgfscope}%
\pgfpathrectangle{\pgfqpoint{0.765000in}{0.660000in}}{\pgfqpoint{4.620000in}{4.620000in}}%
\pgfusepath{clip}%
\pgfsetbuttcap%
\pgfsetroundjoin%
\definecolor{currentfill}{rgb}{1.000000,0.894118,0.788235}%
\pgfsetfillcolor{currentfill}%
\pgfsetlinewidth{0.000000pt}%
\definecolor{currentstroke}{rgb}{1.000000,0.894118,0.788235}%
\pgfsetstrokecolor{currentstroke}%
\pgfsetdash{}{0pt}%
\pgfpathmoveto{\pgfqpoint{3.980427in}{3.036798in}}%
\pgfpathlineto{\pgfqpoint{4.007374in}{3.021240in}}%
\pgfpathlineto{\pgfqpoint{4.007374in}{3.052356in}}%
\pgfpathlineto{\pgfqpoint{3.980427in}{3.067914in}}%
\pgfpathlineto{\pgfqpoint{3.980427in}{3.036798in}}%
\pgfpathclose%
\pgfusepath{fill}%
\end{pgfscope}%
\begin{pgfscope}%
\pgfpathrectangle{\pgfqpoint{0.765000in}{0.660000in}}{\pgfqpoint{4.620000in}{4.620000in}}%
\pgfusepath{clip}%
\pgfsetbuttcap%
\pgfsetroundjoin%
\definecolor{currentfill}{rgb}{1.000000,0.894118,0.788235}%
\pgfsetfillcolor{currentfill}%
\pgfsetlinewidth{0.000000pt}%
\definecolor{currentstroke}{rgb}{1.000000,0.894118,0.788235}%
\pgfsetstrokecolor{currentstroke}%
\pgfsetdash{}{0pt}%
\pgfpathmoveto{\pgfqpoint{4.007207in}{3.052100in}}%
\pgfpathlineto{\pgfqpoint{3.980260in}{3.036542in}}%
\pgfpathlineto{\pgfqpoint{3.980427in}{3.036798in}}%
\pgfpathlineto{\pgfqpoint{4.007374in}{3.052356in}}%
\pgfpathlineto{\pgfqpoint{4.007207in}{3.052100in}}%
\pgfpathclose%
\pgfusepath{fill}%
\end{pgfscope}%
\begin{pgfscope}%
\pgfpathrectangle{\pgfqpoint{0.765000in}{0.660000in}}{\pgfqpoint{4.620000in}{4.620000in}}%
\pgfusepath{clip}%
\pgfsetbuttcap%
\pgfsetroundjoin%
\definecolor{currentfill}{rgb}{1.000000,0.894118,0.788235}%
\pgfsetfillcolor{currentfill}%
\pgfsetlinewidth{0.000000pt}%
\definecolor{currentstroke}{rgb}{1.000000,0.894118,0.788235}%
\pgfsetstrokecolor{currentstroke}%
\pgfsetdash{}{0pt}%
\pgfpathmoveto{\pgfqpoint{4.034154in}{3.036542in}}%
\pgfpathlineto{\pgfqpoint{4.007207in}{3.052100in}}%
\pgfpathlineto{\pgfqpoint{4.007374in}{3.052356in}}%
\pgfpathlineto{\pgfqpoint{4.034321in}{3.036798in}}%
\pgfpathlineto{\pgfqpoint{4.034154in}{3.036542in}}%
\pgfpathclose%
\pgfusepath{fill}%
\end{pgfscope}%
\begin{pgfscope}%
\pgfpathrectangle{\pgfqpoint{0.765000in}{0.660000in}}{\pgfqpoint{4.620000in}{4.620000in}}%
\pgfusepath{clip}%
\pgfsetbuttcap%
\pgfsetroundjoin%
\definecolor{currentfill}{rgb}{1.000000,0.894118,0.788235}%
\pgfsetfillcolor{currentfill}%
\pgfsetlinewidth{0.000000pt}%
\definecolor{currentstroke}{rgb}{1.000000,0.894118,0.788235}%
\pgfsetstrokecolor{currentstroke}%
\pgfsetdash{}{0pt}%
\pgfpathmoveto{\pgfqpoint{4.007207in}{3.052100in}}%
\pgfpathlineto{\pgfqpoint{4.007207in}{3.083216in}}%
\pgfpathlineto{\pgfqpoint{4.007374in}{3.083472in}}%
\pgfpathlineto{\pgfqpoint{4.034321in}{3.036798in}}%
\pgfpathlineto{\pgfqpoint{4.007207in}{3.052100in}}%
\pgfpathclose%
\pgfusepath{fill}%
\end{pgfscope}%
\begin{pgfscope}%
\pgfpathrectangle{\pgfqpoint{0.765000in}{0.660000in}}{\pgfqpoint{4.620000in}{4.620000in}}%
\pgfusepath{clip}%
\pgfsetbuttcap%
\pgfsetroundjoin%
\definecolor{currentfill}{rgb}{1.000000,0.894118,0.788235}%
\pgfsetfillcolor{currentfill}%
\pgfsetlinewidth{0.000000pt}%
\definecolor{currentstroke}{rgb}{1.000000,0.894118,0.788235}%
\pgfsetstrokecolor{currentstroke}%
\pgfsetdash{}{0pt}%
\pgfpathmoveto{\pgfqpoint{4.034154in}{3.036542in}}%
\pgfpathlineto{\pgfqpoint{4.034154in}{3.067658in}}%
\pgfpathlineto{\pgfqpoint{4.034321in}{3.067914in}}%
\pgfpathlineto{\pgfqpoint{4.007374in}{3.052356in}}%
\pgfpathlineto{\pgfqpoint{4.034154in}{3.036542in}}%
\pgfpathclose%
\pgfusepath{fill}%
\end{pgfscope}%
\begin{pgfscope}%
\pgfpathrectangle{\pgfqpoint{0.765000in}{0.660000in}}{\pgfqpoint{4.620000in}{4.620000in}}%
\pgfusepath{clip}%
\pgfsetbuttcap%
\pgfsetroundjoin%
\definecolor{currentfill}{rgb}{1.000000,0.894118,0.788235}%
\pgfsetfillcolor{currentfill}%
\pgfsetlinewidth{0.000000pt}%
\definecolor{currentstroke}{rgb}{1.000000,0.894118,0.788235}%
\pgfsetstrokecolor{currentstroke}%
\pgfsetdash{}{0pt}%
\pgfpathmoveto{\pgfqpoint{3.980260in}{3.036542in}}%
\pgfpathlineto{\pgfqpoint{4.007207in}{3.020984in}}%
\pgfpathlineto{\pgfqpoint{4.007374in}{3.021240in}}%
\pgfpathlineto{\pgfqpoint{3.980427in}{3.036798in}}%
\pgfpathlineto{\pgfqpoint{3.980260in}{3.036542in}}%
\pgfpathclose%
\pgfusepath{fill}%
\end{pgfscope}%
\begin{pgfscope}%
\pgfpathrectangle{\pgfqpoint{0.765000in}{0.660000in}}{\pgfqpoint{4.620000in}{4.620000in}}%
\pgfusepath{clip}%
\pgfsetbuttcap%
\pgfsetroundjoin%
\definecolor{currentfill}{rgb}{1.000000,0.894118,0.788235}%
\pgfsetfillcolor{currentfill}%
\pgfsetlinewidth{0.000000pt}%
\definecolor{currentstroke}{rgb}{1.000000,0.894118,0.788235}%
\pgfsetstrokecolor{currentstroke}%
\pgfsetdash{}{0pt}%
\pgfpathmoveto{\pgfqpoint{4.007207in}{3.020984in}}%
\pgfpathlineto{\pgfqpoint{4.034154in}{3.036542in}}%
\pgfpathlineto{\pgfqpoint{4.034321in}{3.036798in}}%
\pgfpathlineto{\pgfqpoint{4.007374in}{3.021240in}}%
\pgfpathlineto{\pgfqpoint{4.007207in}{3.020984in}}%
\pgfpathclose%
\pgfusepath{fill}%
\end{pgfscope}%
\begin{pgfscope}%
\pgfpathrectangle{\pgfqpoint{0.765000in}{0.660000in}}{\pgfqpoint{4.620000in}{4.620000in}}%
\pgfusepath{clip}%
\pgfsetbuttcap%
\pgfsetroundjoin%
\definecolor{currentfill}{rgb}{1.000000,0.894118,0.788235}%
\pgfsetfillcolor{currentfill}%
\pgfsetlinewidth{0.000000pt}%
\definecolor{currentstroke}{rgb}{1.000000,0.894118,0.788235}%
\pgfsetstrokecolor{currentstroke}%
\pgfsetdash{}{0pt}%
\pgfpathmoveto{\pgfqpoint{4.007207in}{3.083216in}}%
\pgfpathlineto{\pgfqpoint{3.980260in}{3.067658in}}%
\pgfpathlineto{\pgfqpoint{3.980427in}{3.067914in}}%
\pgfpathlineto{\pgfqpoint{4.007374in}{3.083472in}}%
\pgfpathlineto{\pgfqpoint{4.007207in}{3.083216in}}%
\pgfpathclose%
\pgfusepath{fill}%
\end{pgfscope}%
\begin{pgfscope}%
\pgfpathrectangle{\pgfqpoint{0.765000in}{0.660000in}}{\pgfqpoint{4.620000in}{4.620000in}}%
\pgfusepath{clip}%
\pgfsetbuttcap%
\pgfsetroundjoin%
\definecolor{currentfill}{rgb}{1.000000,0.894118,0.788235}%
\pgfsetfillcolor{currentfill}%
\pgfsetlinewidth{0.000000pt}%
\definecolor{currentstroke}{rgb}{1.000000,0.894118,0.788235}%
\pgfsetstrokecolor{currentstroke}%
\pgfsetdash{}{0pt}%
\pgfpathmoveto{\pgfqpoint{4.034154in}{3.067658in}}%
\pgfpathlineto{\pgfqpoint{4.007207in}{3.083216in}}%
\pgfpathlineto{\pgfqpoint{4.007374in}{3.083472in}}%
\pgfpathlineto{\pgfqpoint{4.034321in}{3.067914in}}%
\pgfpathlineto{\pgfqpoint{4.034154in}{3.067658in}}%
\pgfpathclose%
\pgfusepath{fill}%
\end{pgfscope}%
\begin{pgfscope}%
\pgfpathrectangle{\pgfqpoint{0.765000in}{0.660000in}}{\pgfqpoint{4.620000in}{4.620000in}}%
\pgfusepath{clip}%
\pgfsetbuttcap%
\pgfsetroundjoin%
\definecolor{currentfill}{rgb}{1.000000,0.894118,0.788235}%
\pgfsetfillcolor{currentfill}%
\pgfsetlinewidth{0.000000pt}%
\definecolor{currentstroke}{rgb}{1.000000,0.894118,0.788235}%
\pgfsetstrokecolor{currentstroke}%
\pgfsetdash{}{0pt}%
\pgfpathmoveto{\pgfqpoint{3.980260in}{3.036542in}}%
\pgfpathlineto{\pgfqpoint{3.980260in}{3.067658in}}%
\pgfpathlineto{\pgfqpoint{3.980427in}{3.067914in}}%
\pgfpathlineto{\pgfqpoint{4.007374in}{3.021240in}}%
\pgfpathlineto{\pgfqpoint{3.980260in}{3.036542in}}%
\pgfpathclose%
\pgfusepath{fill}%
\end{pgfscope}%
\begin{pgfscope}%
\pgfpathrectangle{\pgfqpoint{0.765000in}{0.660000in}}{\pgfqpoint{4.620000in}{4.620000in}}%
\pgfusepath{clip}%
\pgfsetbuttcap%
\pgfsetroundjoin%
\definecolor{currentfill}{rgb}{1.000000,0.894118,0.788235}%
\pgfsetfillcolor{currentfill}%
\pgfsetlinewidth{0.000000pt}%
\definecolor{currentstroke}{rgb}{1.000000,0.894118,0.788235}%
\pgfsetstrokecolor{currentstroke}%
\pgfsetdash{}{0pt}%
\pgfpathmoveto{\pgfqpoint{4.007207in}{3.020984in}}%
\pgfpathlineto{\pgfqpoint{4.007207in}{3.052100in}}%
\pgfpathlineto{\pgfqpoint{4.007374in}{3.052356in}}%
\pgfpathlineto{\pgfqpoint{3.980427in}{3.036798in}}%
\pgfpathlineto{\pgfqpoint{4.007207in}{3.020984in}}%
\pgfpathclose%
\pgfusepath{fill}%
\end{pgfscope}%
\begin{pgfscope}%
\pgfpathrectangle{\pgfqpoint{0.765000in}{0.660000in}}{\pgfqpoint{4.620000in}{4.620000in}}%
\pgfusepath{clip}%
\pgfsetbuttcap%
\pgfsetroundjoin%
\definecolor{currentfill}{rgb}{1.000000,0.894118,0.788235}%
\pgfsetfillcolor{currentfill}%
\pgfsetlinewidth{0.000000pt}%
\definecolor{currentstroke}{rgb}{1.000000,0.894118,0.788235}%
\pgfsetstrokecolor{currentstroke}%
\pgfsetdash{}{0pt}%
\pgfpathmoveto{\pgfqpoint{4.007207in}{3.052100in}}%
\pgfpathlineto{\pgfqpoint{4.034154in}{3.067658in}}%
\pgfpathlineto{\pgfqpoint{4.034321in}{3.067914in}}%
\pgfpathlineto{\pgfqpoint{4.007374in}{3.052356in}}%
\pgfpathlineto{\pgfqpoint{4.007207in}{3.052100in}}%
\pgfpathclose%
\pgfusepath{fill}%
\end{pgfscope}%
\begin{pgfscope}%
\pgfpathrectangle{\pgfqpoint{0.765000in}{0.660000in}}{\pgfqpoint{4.620000in}{4.620000in}}%
\pgfusepath{clip}%
\pgfsetbuttcap%
\pgfsetroundjoin%
\definecolor{currentfill}{rgb}{1.000000,0.894118,0.788235}%
\pgfsetfillcolor{currentfill}%
\pgfsetlinewidth{0.000000pt}%
\definecolor{currentstroke}{rgb}{1.000000,0.894118,0.788235}%
\pgfsetstrokecolor{currentstroke}%
\pgfsetdash{}{0pt}%
\pgfpathmoveto{\pgfqpoint{3.980260in}{3.067658in}}%
\pgfpathlineto{\pgfqpoint{4.007207in}{3.052100in}}%
\pgfpathlineto{\pgfqpoint{4.007374in}{3.052356in}}%
\pgfpathlineto{\pgfqpoint{3.980427in}{3.067914in}}%
\pgfpathlineto{\pgfqpoint{3.980260in}{3.067658in}}%
\pgfpathclose%
\pgfusepath{fill}%
\end{pgfscope}%
\begin{pgfscope}%
\pgfpathrectangle{\pgfqpoint{0.765000in}{0.660000in}}{\pgfqpoint{4.620000in}{4.620000in}}%
\pgfusepath{clip}%
\pgfsetbuttcap%
\pgfsetroundjoin%
\definecolor{currentfill}{rgb}{1.000000,0.894118,0.788235}%
\pgfsetfillcolor{currentfill}%
\pgfsetlinewidth{0.000000pt}%
\definecolor{currentstroke}{rgb}{1.000000,0.894118,0.788235}%
\pgfsetstrokecolor{currentstroke}%
\pgfsetdash{}{0pt}%
\pgfpathmoveto{\pgfqpoint{4.260633in}{3.313475in}}%
\pgfpathlineto{\pgfqpoint{4.233686in}{3.297917in}}%
\pgfpathlineto{\pgfqpoint{4.260633in}{3.282359in}}%
\pgfpathlineto{\pgfqpoint{4.287581in}{3.297917in}}%
\pgfpathlineto{\pgfqpoint{4.260633in}{3.313475in}}%
\pgfpathclose%
\pgfusepath{fill}%
\end{pgfscope}%
\begin{pgfscope}%
\pgfpathrectangle{\pgfqpoint{0.765000in}{0.660000in}}{\pgfqpoint{4.620000in}{4.620000in}}%
\pgfusepath{clip}%
\pgfsetbuttcap%
\pgfsetroundjoin%
\definecolor{currentfill}{rgb}{1.000000,0.894118,0.788235}%
\pgfsetfillcolor{currentfill}%
\pgfsetlinewidth{0.000000pt}%
\definecolor{currentstroke}{rgb}{1.000000,0.894118,0.788235}%
\pgfsetstrokecolor{currentstroke}%
\pgfsetdash{}{0pt}%
\pgfpathmoveto{\pgfqpoint{4.260633in}{3.313475in}}%
\pgfpathlineto{\pgfqpoint{4.233686in}{3.297917in}}%
\pgfpathlineto{\pgfqpoint{4.233686in}{3.329033in}}%
\pgfpathlineto{\pgfqpoint{4.260633in}{3.344591in}}%
\pgfpathlineto{\pgfqpoint{4.260633in}{3.313475in}}%
\pgfpathclose%
\pgfusepath{fill}%
\end{pgfscope}%
\begin{pgfscope}%
\pgfpathrectangle{\pgfqpoint{0.765000in}{0.660000in}}{\pgfqpoint{4.620000in}{4.620000in}}%
\pgfusepath{clip}%
\pgfsetbuttcap%
\pgfsetroundjoin%
\definecolor{currentfill}{rgb}{1.000000,0.894118,0.788235}%
\pgfsetfillcolor{currentfill}%
\pgfsetlinewidth{0.000000pt}%
\definecolor{currentstroke}{rgb}{1.000000,0.894118,0.788235}%
\pgfsetstrokecolor{currentstroke}%
\pgfsetdash{}{0pt}%
\pgfpathmoveto{\pgfqpoint{4.260633in}{3.313475in}}%
\pgfpathlineto{\pgfqpoint{4.287581in}{3.297917in}}%
\pgfpathlineto{\pgfqpoint{4.287581in}{3.329033in}}%
\pgfpathlineto{\pgfqpoint{4.260633in}{3.344591in}}%
\pgfpathlineto{\pgfqpoint{4.260633in}{3.313475in}}%
\pgfpathclose%
\pgfusepath{fill}%
\end{pgfscope}%
\begin{pgfscope}%
\pgfpathrectangle{\pgfqpoint{0.765000in}{0.660000in}}{\pgfqpoint{4.620000in}{4.620000in}}%
\pgfusepath{clip}%
\pgfsetbuttcap%
\pgfsetroundjoin%
\definecolor{currentfill}{rgb}{1.000000,0.894118,0.788235}%
\pgfsetfillcolor{currentfill}%
\pgfsetlinewidth{0.000000pt}%
\definecolor{currentstroke}{rgb}{1.000000,0.894118,0.788235}%
\pgfsetstrokecolor{currentstroke}%
\pgfsetdash{}{0pt}%
\pgfpathmoveto{\pgfqpoint{4.192694in}{3.279892in}}%
\pgfpathlineto{\pgfqpoint{4.165747in}{3.264334in}}%
\pgfpathlineto{\pgfqpoint{4.192694in}{3.248776in}}%
\pgfpathlineto{\pgfqpoint{4.219642in}{3.264334in}}%
\pgfpathlineto{\pgfqpoint{4.192694in}{3.279892in}}%
\pgfpathclose%
\pgfusepath{fill}%
\end{pgfscope}%
\begin{pgfscope}%
\pgfpathrectangle{\pgfqpoint{0.765000in}{0.660000in}}{\pgfqpoint{4.620000in}{4.620000in}}%
\pgfusepath{clip}%
\pgfsetbuttcap%
\pgfsetroundjoin%
\definecolor{currentfill}{rgb}{1.000000,0.894118,0.788235}%
\pgfsetfillcolor{currentfill}%
\pgfsetlinewidth{0.000000pt}%
\definecolor{currentstroke}{rgb}{1.000000,0.894118,0.788235}%
\pgfsetstrokecolor{currentstroke}%
\pgfsetdash{}{0pt}%
\pgfpathmoveto{\pgfqpoint{4.192694in}{3.279892in}}%
\pgfpathlineto{\pgfqpoint{4.165747in}{3.264334in}}%
\pgfpathlineto{\pgfqpoint{4.165747in}{3.295450in}}%
\pgfpathlineto{\pgfqpoint{4.192694in}{3.311008in}}%
\pgfpathlineto{\pgfqpoint{4.192694in}{3.279892in}}%
\pgfpathclose%
\pgfusepath{fill}%
\end{pgfscope}%
\begin{pgfscope}%
\pgfpathrectangle{\pgfqpoint{0.765000in}{0.660000in}}{\pgfqpoint{4.620000in}{4.620000in}}%
\pgfusepath{clip}%
\pgfsetbuttcap%
\pgfsetroundjoin%
\definecolor{currentfill}{rgb}{1.000000,0.894118,0.788235}%
\pgfsetfillcolor{currentfill}%
\pgfsetlinewidth{0.000000pt}%
\definecolor{currentstroke}{rgb}{1.000000,0.894118,0.788235}%
\pgfsetstrokecolor{currentstroke}%
\pgfsetdash{}{0pt}%
\pgfpathmoveto{\pgfqpoint{4.192694in}{3.279892in}}%
\pgfpathlineto{\pgfqpoint{4.219642in}{3.264334in}}%
\pgfpathlineto{\pgfqpoint{4.219642in}{3.295450in}}%
\pgfpathlineto{\pgfqpoint{4.192694in}{3.311008in}}%
\pgfpathlineto{\pgfqpoint{4.192694in}{3.279892in}}%
\pgfpathclose%
\pgfusepath{fill}%
\end{pgfscope}%
\begin{pgfscope}%
\pgfpathrectangle{\pgfqpoint{0.765000in}{0.660000in}}{\pgfqpoint{4.620000in}{4.620000in}}%
\pgfusepath{clip}%
\pgfsetbuttcap%
\pgfsetroundjoin%
\definecolor{currentfill}{rgb}{1.000000,0.894118,0.788235}%
\pgfsetfillcolor{currentfill}%
\pgfsetlinewidth{0.000000pt}%
\definecolor{currentstroke}{rgb}{1.000000,0.894118,0.788235}%
\pgfsetstrokecolor{currentstroke}%
\pgfsetdash{}{0pt}%
\pgfpathmoveto{\pgfqpoint{4.260633in}{3.344591in}}%
\pgfpathlineto{\pgfqpoint{4.233686in}{3.329033in}}%
\pgfpathlineto{\pgfqpoint{4.260633in}{3.313475in}}%
\pgfpathlineto{\pgfqpoint{4.287581in}{3.329033in}}%
\pgfpathlineto{\pgfqpoint{4.260633in}{3.344591in}}%
\pgfpathclose%
\pgfusepath{fill}%
\end{pgfscope}%
\begin{pgfscope}%
\pgfpathrectangle{\pgfqpoint{0.765000in}{0.660000in}}{\pgfqpoint{4.620000in}{4.620000in}}%
\pgfusepath{clip}%
\pgfsetbuttcap%
\pgfsetroundjoin%
\definecolor{currentfill}{rgb}{1.000000,0.894118,0.788235}%
\pgfsetfillcolor{currentfill}%
\pgfsetlinewidth{0.000000pt}%
\definecolor{currentstroke}{rgb}{1.000000,0.894118,0.788235}%
\pgfsetstrokecolor{currentstroke}%
\pgfsetdash{}{0pt}%
\pgfpathmoveto{\pgfqpoint{4.260633in}{3.282359in}}%
\pgfpathlineto{\pgfqpoint{4.287581in}{3.297917in}}%
\pgfpathlineto{\pgfqpoint{4.287581in}{3.329033in}}%
\pgfpathlineto{\pgfqpoint{4.260633in}{3.313475in}}%
\pgfpathlineto{\pgfqpoint{4.260633in}{3.282359in}}%
\pgfpathclose%
\pgfusepath{fill}%
\end{pgfscope}%
\begin{pgfscope}%
\pgfpathrectangle{\pgfqpoint{0.765000in}{0.660000in}}{\pgfqpoint{4.620000in}{4.620000in}}%
\pgfusepath{clip}%
\pgfsetbuttcap%
\pgfsetroundjoin%
\definecolor{currentfill}{rgb}{1.000000,0.894118,0.788235}%
\pgfsetfillcolor{currentfill}%
\pgfsetlinewidth{0.000000pt}%
\definecolor{currentstroke}{rgb}{1.000000,0.894118,0.788235}%
\pgfsetstrokecolor{currentstroke}%
\pgfsetdash{}{0pt}%
\pgfpathmoveto{\pgfqpoint{4.233686in}{3.297917in}}%
\pgfpathlineto{\pgfqpoint{4.260633in}{3.282359in}}%
\pgfpathlineto{\pgfqpoint{4.260633in}{3.313475in}}%
\pgfpathlineto{\pgfqpoint{4.233686in}{3.329033in}}%
\pgfpathlineto{\pgfqpoint{4.233686in}{3.297917in}}%
\pgfpathclose%
\pgfusepath{fill}%
\end{pgfscope}%
\begin{pgfscope}%
\pgfpathrectangle{\pgfqpoint{0.765000in}{0.660000in}}{\pgfqpoint{4.620000in}{4.620000in}}%
\pgfusepath{clip}%
\pgfsetbuttcap%
\pgfsetroundjoin%
\definecolor{currentfill}{rgb}{1.000000,0.894118,0.788235}%
\pgfsetfillcolor{currentfill}%
\pgfsetlinewidth{0.000000pt}%
\definecolor{currentstroke}{rgb}{1.000000,0.894118,0.788235}%
\pgfsetstrokecolor{currentstroke}%
\pgfsetdash{}{0pt}%
\pgfpathmoveto{\pgfqpoint{4.192694in}{3.311008in}}%
\pgfpathlineto{\pgfqpoint{4.165747in}{3.295450in}}%
\pgfpathlineto{\pgfqpoint{4.192694in}{3.279892in}}%
\pgfpathlineto{\pgfqpoint{4.219642in}{3.295450in}}%
\pgfpathlineto{\pgfqpoint{4.192694in}{3.311008in}}%
\pgfpathclose%
\pgfusepath{fill}%
\end{pgfscope}%
\begin{pgfscope}%
\pgfpathrectangle{\pgfqpoint{0.765000in}{0.660000in}}{\pgfqpoint{4.620000in}{4.620000in}}%
\pgfusepath{clip}%
\pgfsetbuttcap%
\pgfsetroundjoin%
\definecolor{currentfill}{rgb}{1.000000,0.894118,0.788235}%
\pgfsetfillcolor{currentfill}%
\pgfsetlinewidth{0.000000pt}%
\definecolor{currentstroke}{rgb}{1.000000,0.894118,0.788235}%
\pgfsetstrokecolor{currentstroke}%
\pgfsetdash{}{0pt}%
\pgfpathmoveto{\pgfqpoint{4.192694in}{3.248776in}}%
\pgfpathlineto{\pgfqpoint{4.219642in}{3.264334in}}%
\pgfpathlineto{\pgfqpoint{4.219642in}{3.295450in}}%
\pgfpathlineto{\pgfqpoint{4.192694in}{3.279892in}}%
\pgfpathlineto{\pgfqpoint{4.192694in}{3.248776in}}%
\pgfpathclose%
\pgfusepath{fill}%
\end{pgfscope}%
\begin{pgfscope}%
\pgfpathrectangle{\pgfqpoint{0.765000in}{0.660000in}}{\pgfqpoint{4.620000in}{4.620000in}}%
\pgfusepath{clip}%
\pgfsetbuttcap%
\pgfsetroundjoin%
\definecolor{currentfill}{rgb}{1.000000,0.894118,0.788235}%
\pgfsetfillcolor{currentfill}%
\pgfsetlinewidth{0.000000pt}%
\definecolor{currentstroke}{rgb}{1.000000,0.894118,0.788235}%
\pgfsetstrokecolor{currentstroke}%
\pgfsetdash{}{0pt}%
\pgfpathmoveto{\pgfqpoint{4.165747in}{3.264334in}}%
\pgfpathlineto{\pgfqpoint{4.192694in}{3.248776in}}%
\pgfpathlineto{\pgfqpoint{4.192694in}{3.279892in}}%
\pgfpathlineto{\pgfqpoint{4.165747in}{3.295450in}}%
\pgfpathlineto{\pgfqpoint{4.165747in}{3.264334in}}%
\pgfpathclose%
\pgfusepath{fill}%
\end{pgfscope}%
\begin{pgfscope}%
\pgfpathrectangle{\pgfqpoint{0.765000in}{0.660000in}}{\pgfqpoint{4.620000in}{4.620000in}}%
\pgfusepath{clip}%
\pgfsetbuttcap%
\pgfsetroundjoin%
\definecolor{currentfill}{rgb}{1.000000,0.894118,0.788235}%
\pgfsetfillcolor{currentfill}%
\pgfsetlinewidth{0.000000pt}%
\definecolor{currentstroke}{rgb}{1.000000,0.894118,0.788235}%
\pgfsetstrokecolor{currentstroke}%
\pgfsetdash{}{0pt}%
\pgfpathmoveto{\pgfqpoint{4.260633in}{3.313475in}}%
\pgfpathlineto{\pgfqpoint{4.233686in}{3.297917in}}%
\pgfpathlineto{\pgfqpoint{4.165747in}{3.264334in}}%
\pgfpathlineto{\pgfqpoint{4.192694in}{3.279892in}}%
\pgfpathlineto{\pgfqpoint{4.260633in}{3.313475in}}%
\pgfpathclose%
\pgfusepath{fill}%
\end{pgfscope}%
\begin{pgfscope}%
\pgfpathrectangle{\pgfqpoint{0.765000in}{0.660000in}}{\pgfqpoint{4.620000in}{4.620000in}}%
\pgfusepath{clip}%
\pgfsetbuttcap%
\pgfsetroundjoin%
\definecolor{currentfill}{rgb}{1.000000,0.894118,0.788235}%
\pgfsetfillcolor{currentfill}%
\pgfsetlinewidth{0.000000pt}%
\definecolor{currentstroke}{rgb}{1.000000,0.894118,0.788235}%
\pgfsetstrokecolor{currentstroke}%
\pgfsetdash{}{0pt}%
\pgfpathmoveto{\pgfqpoint{4.287581in}{3.297917in}}%
\pgfpathlineto{\pgfqpoint{4.260633in}{3.313475in}}%
\pgfpathlineto{\pgfqpoint{4.192694in}{3.279892in}}%
\pgfpathlineto{\pgfqpoint{4.219642in}{3.264334in}}%
\pgfpathlineto{\pgfqpoint{4.287581in}{3.297917in}}%
\pgfpathclose%
\pgfusepath{fill}%
\end{pgfscope}%
\begin{pgfscope}%
\pgfpathrectangle{\pgfqpoint{0.765000in}{0.660000in}}{\pgfqpoint{4.620000in}{4.620000in}}%
\pgfusepath{clip}%
\pgfsetbuttcap%
\pgfsetroundjoin%
\definecolor{currentfill}{rgb}{1.000000,0.894118,0.788235}%
\pgfsetfillcolor{currentfill}%
\pgfsetlinewidth{0.000000pt}%
\definecolor{currentstroke}{rgb}{1.000000,0.894118,0.788235}%
\pgfsetstrokecolor{currentstroke}%
\pgfsetdash{}{0pt}%
\pgfpathmoveto{\pgfqpoint{4.260633in}{3.313475in}}%
\pgfpathlineto{\pgfqpoint{4.260633in}{3.344591in}}%
\pgfpathlineto{\pgfqpoint{4.192694in}{3.311008in}}%
\pgfpathlineto{\pgfqpoint{4.219642in}{3.264334in}}%
\pgfpathlineto{\pgfqpoint{4.260633in}{3.313475in}}%
\pgfpathclose%
\pgfusepath{fill}%
\end{pgfscope}%
\begin{pgfscope}%
\pgfpathrectangle{\pgfqpoint{0.765000in}{0.660000in}}{\pgfqpoint{4.620000in}{4.620000in}}%
\pgfusepath{clip}%
\pgfsetbuttcap%
\pgfsetroundjoin%
\definecolor{currentfill}{rgb}{1.000000,0.894118,0.788235}%
\pgfsetfillcolor{currentfill}%
\pgfsetlinewidth{0.000000pt}%
\definecolor{currentstroke}{rgb}{1.000000,0.894118,0.788235}%
\pgfsetstrokecolor{currentstroke}%
\pgfsetdash{}{0pt}%
\pgfpathmoveto{\pgfqpoint{4.287581in}{3.297917in}}%
\pgfpathlineto{\pgfqpoint{4.287581in}{3.329033in}}%
\pgfpathlineto{\pgfqpoint{4.219642in}{3.295450in}}%
\pgfpathlineto{\pgfqpoint{4.192694in}{3.279892in}}%
\pgfpathlineto{\pgfqpoint{4.287581in}{3.297917in}}%
\pgfpathclose%
\pgfusepath{fill}%
\end{pgfscope}%
\begin{pgfscope}%
\pgfpathrectangle{\pgfqpoint{0.765000in}{0.660000in}}{\pgfqpoint{4.620000in}{4.620000in}}%
\pgfusepath{clip}%
\pgfsetbuttcap%
\pgfsetroundjoin%
\definecolor{currentfill}{rgb}{1.000000,0.894118,0.788235}%
\pgfsetfillcolor{currentfill}%
\pgfsetlinewidth{0.000000pt}%
\definecolor{currentstroke}{rgb}{1.000000,0.894118,0.788235}%
\pgfsetstrokecolor{currentstroke}%
\pgfsetdash{}{0pt}%
\pgfpathmoveto{\pgfqpoint{4.233686in}{3.297917in}}%
\pgfpathlineto{\pgfqpoint{4.260633in}{3.282359in}}%
\pgfpathlineto{\pgfqpoint{4.192694in}{3.248776in}}%
\pgfpathlineto{\pgfqpoint{4.165747in}{3.264334in}}%
\pgfpathlineto{\pgfqpoint{4.233686in}{3.297917in}}%
\pgfpathclose%
\pgfusepath{fill}%
\end{pgfscope}%
\begin{pgfscope}%
\pgfpathrectangle{\pgfqpoint{0.765000in}{0.660000in}}{\pgfqpoint{4.620000in}{4.620000in}}%
\pgfusepath{clip}%
\pgfsetbuttcap%
\pgfsetroundjoin%
\definecolor{currentfill}{rgb}{1.000000,0.894118,0.788235}%
\pgfsetfillcolor{currentfill}%
\pgfsetlinewidth{0.000000pt}%
\definecolor{currentstroke}{rgb}{1.000000,0.894118,0.788235}%
\pgfsetstrokecolor{currentstroke}%
\pgfsetdash{}{0pt}%
\pgfpathmoveto{\pgfqpoint{4.260633in}{3.282359in}}%
\pgfpathlineto{\pgfqpoint{4.287581in}{3.297917in}}%
\pgfpathlineto{\pgfqpoint{4.219642in}{3.264334in}}%
\pgfpathlineto{\pgfqpoint{4.192694in}{3.248776in}}%
\pgfpathlineto{\pgfqpoint{4.260633in}{3.282359in}}%
\pgfpathclose%
\pgfusepath{fill}%
\end{pgfscope}%
\begin{pgfscope}%
\pgfpathrectangle{\pgfqpoint{0.765000in}{0.660000in}}{\pgfqpoint{4.620000in}{4.620000in}}%
\pgfusepath{clip}%
\pgfsetbuttcap%
\pgfsetroundjoin%
\definecolor{currentfill}{rgb}{1.000000,0.894118,0.788235}%
\pgfsetfillcolor{currentfill}%
\pgfsetlinewidth{0.000000pt}%
\definecolor{currentstroke}{rgb}{1.000000,0.894118,0.788235}%
\pgfsetstrokecolor{currentstroke}%
\pgfsetdash{}{0pt}%
\pgfpathmoveto{\pgfqpoint{4.260633in}{3.344591in}}%
\pgfpathlineto{\pgfqpoint{4.233686in}{3.329033in}}%
\pgfpathlineto{\pgfqpoint{4.165747in}{3.295450in}}%
\pgfpathlineto{\pgfqpoint{4.192694in}{3.311008in}}%
\pgfpathlineto{\pgfqpoint{4.260633in}{3.344591in}}%
\pgfpathclose%
\pgfusepath{fill}%
\end{pgfscope}%
\begin{pgfscope}%
\pgfpathrectangle{\pgfqpoint{0.765000in}{0.660000in}}{\pgfqpoint{4.620000in}{4.620000in}}%
\pgfusepath{clip}%
\pgfsetbuttcap%
\pgfsetroundjoin%
\definecolor{currentfill}{rgb}{1.000000,0.894118,0.788235}%
\pgfsetfillcolor{currentfill}%
\pgfsetlinewidth{0.000000pt}%
\definecolor{currentstroke}{rgb}{1.000000,0.894118,0.788235}%
\pgfsetstrokecolor{currentstroke}%
\pgfsetdash{}{0pt}%
\pgfpathmoveto{\pgfqpoint{4.287581in}{3.329033in}}%
\pgfpathlineto{\pgfqpoint{4.260633in}{3.344591in}}%
\pgfpathlineto{\pgfqpoint{4.192694in}{3.311008in}}%
\pgfpathlineto{\pgfqpoint{4.219642in}{3.295450in}}%
\pgfpathlineto{\pgfqpoint{4.287581in}{3.329033in}}%
\pgfpathclose%
\pgfusepath{fill}%
\end{pgfscope}%
\begin{pgfscope}%
\pgfpathrectangle{\pgfqpoint{0.765000in}{0.660000in}}{\pgfqpoint{4.620000in}{4.620000in}}%
\pgfusepath{clip}%
\pgfsetbuttcap%
\pgfsetroundjoin%
\definecolor{currentfill}{rgb}{1.000000,0.894118,0.788235}%
\pgfsetfillcolor{currentfill}%
\pgfsetlinewidth{0.000000pt}%
\definecolor{currentstroke}{rgb}{1.000000,0.894118,0.788235}%
\pgfsetstrokecolor{currentstroke}%
\pgfsetdash{}{0pt}%
\pgfpathmoveto{\pgfqpoint{4.233686in}{3.297917in}}%
\pgfpathlineto{\pgfqpoint{4.233686in}{3.329033in}}%
\pgfpathlineto{\pgfqpoint{4.165747in}{3.295450in}}%
\pgfpathlineto{\pgfqpoint{4.192694in}{3.248776in}}%
\pgfpathlineto{\pgfqpoint{4.233686in}{3.297917in}}%
\pgfpathclose%
\pgfusepath{fill}%
\end{pgfscope}%
\begin{pgfscope}%
\pgfpathrectangle{\pgfqpoint{0.765000in}{0.660000in}}{\pgfqpoint{4.620000in}{4.620000in}}%
\pgfusepath{clip}%
\pgfsetbuttcap%
\pgfsetroundjoin%
\definecolor{currentfill}{rgb}{1.000000,0.894118,0.788235}%
\pgfsetfillcolor{currentfill}%
\pgfsetlinewidth{0.000000pt}%
\definecolor{currentstroke}{rgb}{1.000000,0.894118,0.788235}%
\pgfsetstrokecolor{currentstroke}%
\pgfsetdash{}{0pt}%
\pgfpathmoveto{\pgfqpoint{4.260633in}{3.282359in}}%
\pgfpathlineto{\pgfqpoint{4.260633in}{3.313475in}}%
\pgfpathlineto{\pgfqpoint{4.192694in}{3.279892in}}%
\pgfpathlineto{\pgfqpoint{4.165747in}{3.264334in}}%
\pgfpathlineto{\pgfqpoint{4.260633in}{3.282359in}}%
\pgfpathclose%
\pgfusepath{fill}%
\end{pgfscope}%
\begin{pgfscope}%
\pgfpathrectangle{\pgfqpoint{0.765000in}{0.660000in}}{\pgfqpoint{4.620000in}{4.620000in}}%
\pgfusepath{clip}%
\pgfsetbuttcap%
\pgfsetroundjoin%
\definecolor{currentfill}{rgb}{1.000000,0.894118,0.788235}%
\pgfsetfillcolor{currentfill}%
\pgfsetlinewidth{0.000000pt}%
\definecolor{currentstroke}{rgb}{1.000000,0.894118,0.788235}%
\pgfsetstrokecolor{currentstroke}%
\pgfsetdash{}{0pt}%
\pgfpathmoveto{\pgfqpoint{4.260633in}{3.313475in}}%
\pgfpathlineto{\pgfqpoint{4.287581in}{3.329033in}}%
\pgfpathlineto{\pgfqpoint{4.219642in}{3.295450in}}%
\pgfpathlineto{\pgfqpoint{4.192694in}{3.279892in}}%
\pgfpathlineto{\pgfqpoint{4.260633in}{3.313475in}}%
\pgfpathclose%
\pgfusepath{fill}%
\end{pgfscope}%
\begin{pgfscope}%
\pgfpathrectangle{\pgfqpoint{0.765000in}{0.660000in}}{\pgfqpoint{4.620000in}{4.620000in}}%
\pgfusepath{clip}%
\pgfsetbuttcap%
\pgfsetroundjoin%
\definecolor{currentfill}{rgb}{1.000000,0.894118,0.788235}%
\pgfsetfillcolor{currentfill}%
\pgfsetlinewidth{0.000000pt}%
\definecolor{currentstroke}{rgb}{1.000000,0.894118,0.788235}%
\pgfsetstrokecolor{currentstroke}%
\pgfsetdash{}{0pt}%
\pgfpathmoveto{\pgfqpoint{4.233686in}{3.329033in}}%
\pgfpathlineto{\pgfqpoint{4.260633in}{3.313475in}}%
\pgfpathlineto{\pgfqpoint{4.192694in}{3.279892in}}%
\pgfpathlineto{\pgfqpoint{4.165747in}{3.295450in}}%
\pgfpathlineto{\pgfqpoint{4.233686in}{3.329033in}}%
\pgfpathclose%
\pgfusepath{fill}%
\end{pgfscope}%
\begin{pgfscope}%
\pgfpathrectangle{\pgfqpoint{0.765000in}{0.660000in}}{\pgfqpoint{4.620000in}{4.620000in}}%
\pgfusepath{clip}%
\pgfsetbuttcap%
\pgfsetroundjoin%
\definecolor{currentfill}{rgb}{1.000000,0.894118,0.788235}%
\pgfsetfillcolor{currentfill}%
\pgfsetlinewidth{0.000000pt}%
\definecolor{currentstroke}{rgb}{1.000000,0.894118,0.788235}%
\pgfsetstrokecolor{currentstroke}%
\pgfsetdash{}{0pt}%
\pgfpathmoveto{\pgfqpoint{14.674810in}{9.957844in}}%
\pgfpathlineto{\pgfqpoint{14.647863in}{9.942286in}}%
\pgfpathlineto{\pgfqpoint{14.674810in}{9.926728in}}%
\pgfpathlineto{\pgfqpoint{14.701758in}{9.942286in}}%
\pgfpathlineto{\pgfqpoint{14.674810in}{9.957844in}}%
\pgfpathclose%
\pgfusepath{fill}%
\end{pgfscope}%
\begin{pgfscope}%
\pgfpathrectangle{\pgfqpoint{0.765000in}{0.660000in}}{\pgfqpoint{4.620000in}{4.620000in}}%
\pgfusepath{clip}%
\pgfsetbuttcap%
\pgfsetroundjoin%
\definecolor{currentfill}{rgb}{1.000000,0.894118,0.788235}%
\pgfsetfillcolor{currentfill}%
\pgfsetlinewidth{0.000000pt}%
\definecolor{currentstroke}{rgb}{1.000000,0.894118,0.788235}%
\pgfsetstrokecolor{currentstroke}%
\pgfsetdash{}{0pt}%
\pgfpathmoveto{\pgfqpoint{14.674810in}{9.957844in}}%
\pgfpathlineto{\pgfqpoint{14.647863in}{9.942286in}}%
\pgfpathlineto{\pgfqpoint{14.647863in}{9.973402in}}%
\pgfpathlineto{\pgfqpoint{14.674810in}{9.988960in}}%
\pgfpathlineto{\pgfqpoint{14.674810in}{9.957844in}}%
\pgfpathclose%
\pgfusepath{fill}%
\end{pgfscope}%
\begin{pgfscope}%
\pgfpathrectangle{\pgfqpoint{0.765000in}{0.660000in}}{\pgfqpoint{4.620000in}{4.620000in}}%
\pgfusepath{clip}%
\pgfsetbuttcap%
\pgfsetroundjoin%
\definecolor{currentfill}{rgb}{1.000000,0.894118,0.788235}%
\pgfsetfillcolor{currentfill}%
\pgfsetlinewidth{0.000000pt}%
\definecolor{currentstroke}{rgb}{1.000000,0.894118,0.788235}%
\pgfsetstrokecolor{currentstroke}%
\pgfsetdash{}{0pt}%
\pgfpathmoveto{\pgfqpoint{14.674810in}{9.957844in}}%
\pgfpathlineto{\pgfqpoint{14.701758in}{9.942286in}}%
\pgfpathlineto{\pgfqpoint{14.701758in}{9.973402in}}%
\pgfpathlineto{\pgfqpoint{14.674810in}{9.988960in}}%
\pgfpathlineto{\pgfqpoint{14.674810in}{9.957844in}}%
\pgfpathclose%
\pgfusepath{fill}%
\end{pgfscope}%
\begin{pgfscope}%
\pgfpathrectangle{\pgfqpoint{0.765000in}{0.660000in}}{\pgfqpoint{4.620000in}{4.620000in}}%
\pgfusepath{clip}%
\pgfsetbuttcap%
\pgfsetroundjoin%
\definecolor{currentfill}{rgb}{1.000000,0.894118,0.788235}%
\pgfsetfillcolor{currentfill}%
\pgfsetlinewidth{0.000000pt}%
\definecolor{currentstroke}{rgb}{1.000000,0.894118,0.788235}%
\pgfsetstrokecolor{currentstroke}%
\pgfsetdash{}{0pt}%
\pgfpathmoveto{\pgfqpoint{4.260633in}{3.313475in}}%
\pgfpathlineto{\pgfqpoint{4.233686in}{3.297917in}}%
\pgfpathlineto{\pgfqpoint{4.260633in}{3.282359in}}%
\pgfpathlineto{\pgfqpoint{4.287581in}{3.297917in}}%
\pgfpathlineto{\pgfqpoint{4.260633in}{3.313475in}}%
\pgfpathclose%
\pgfusepath{fill}%
\end{pgfscope}%
\begin{pgfscope}%
\pgfpathrectangle{\pgfqpoint{0.765000in}{0.660000in}}{\pgfqpoint{4.620000in}{4.620000in}}%
\pgfusepath{clip}%
\pgfsetbuttcap%
\pgfsetroundjoin%
\definecolor{currentfill}{rgb}{1.000000,0.894118,0.788235}%
\pgfsetfillcolor{currentfill}%
\pgfsetlinewidth{0.000000pt}%
\definecolor{currentstroke}{rgb}{1.000000,0.894118,0.788235}%
\pgfsetstrokecolor{currentstroke}%
\pgfsetdash{}{0pt}%
\pgfpathmoveto{\pgfqpoint{4.260633in}{3.313475in}}%
\pgfpathlineto{\pgfqpoint{4.233686in}{3.297917in}}%
\pgfpathlineto{\pgfqpoint{4.233686in}{3.329033in}}%
\pgfpathlineto{\pgfqpoint{4.260633in}{3.344591in}}%
\pgfpathlineto{\pgfqpoint{4.260633in}{3.313475in}}%
\pgfpathclose%
\pgfusepath{fill}%
\end{pgfscope}%
\begin{pgfscope}%
\pgfpathrectangle{\pgfqpoint{0.765000in}{0.660000in}}{\pgfqpoint{4.620000in}{4.620000in}}%
\pgfusepath{clip}%
\pgfsetbuttcap%
\pgfsetroundjoin%
\definecolor{currentfill}{rgb}{1.000000,0.894118,0.788235}%
\pgfsetfillcolor{currentfill}%
\pgfsetlinewidth{0.000000pt}%
\definecolor{currentstroke}{rgb}{1.000000,0.894118,0.788235}%
\pgfsetstrokecolor{currentstroke}%
\pgfsetdash{}{0pt}%
\pgfpathmoveto{\pgfqpoint{4.260633in}{3.313475in}}%
\pgfpathlineto{\pgfqpoint{4.287581in}{3.297917in}}%
\pgfpathlineto{\pgfqpoint{4.287581in}{3.329033in}}%
\pgfpathlineto{\pgfqpoint{4.260633in}{3.344591in}}%
\pgfpathlineto{\pgfqpoint{4.260633in}{3.313475in}}%
\pgfpathclose%
\pgfusepath{fill}%
\end{pgfscope}%
\begin{pgfscope}%
\pgfpathrectangle{\pgfqpoint{0.765000in}{0.660000in}}{\pgfqpoint{4.620000in}{4.620000in}}%
\pgfusepath{clip}%
\pgfsetbuttcap%
\pgfsetroundjoin%
\definecolor{currentfill}{rgb}{1.000000,0.894118,0.788235}%
\pgfsetfillcolor{currentfill}%
\pgfsetlinewidth{0.000000pt}%
\definecolor{currentstroke}{rgb}{1.000000,0.894118,0.788235}%
\pgfsetstrokecolor{currentstroke}%
\pgfsetdash{}{0pt}%
\pgfpathmoveto{\pgfqpoint{4.260633in}{3.344591in}}%
\pgfpathlineto{\pgfqpoint{4.233686in}{3.329033in}}%
\pgfpathlineto{\pgfqpoint{4.260633in}{3.313475in}}%
\pgfpathlineto{\pgfqpoint{4.287581in}{3.329033in}}%
\pgfpathlineto{\pgfqpoint{4.260633in}{3.344591in}}%
\pgfpathclose%
\pgfusepath{fill}%
\end{pgfscope}%
\begin{pgfscope}%
\pgfpathrectangle{\pgfqpoint{0.765000in}{0.660000in}}{\pgfqpoint{4.620000in}{4.620000in}}%
\pgfusepath{clip}%
\pgfsetbuttcap%
\pgfsetroundjoin%
\definecolor{currentfill}{rgb}{1.000000,0.894118,0.788235}%
\pgfsetfillcolor{currentfill}%
\pgfsetlinewidth{0.000000pt}%
\definecolor{currentstroke}{rgb}{1.000000,0.894118,0.788235}%
\pgfsetstrokecolor{currentstroke}%
\pgfsetdash{}{0pt}%
\pgfpathmoveto{\pgfqpoint{4.260633in}{3.282359in}}%
\pgfpathlineto{\pgfqpoint{4.287581in}{3.297917in}}%
\pgfpathlineto{\pgfqpoint{4.287581in}{3.329033in}}%
\pgfpathlineto{\pgfqpoint{4.260633in}{3.313475in}}%
\pgfpathlineto{\pgfqpoint{4.260633in}{3.282359in}}%
\pgfpathclose%
\pgfusepath{fill}%
\end{pgfscope}%
\begin{pgfscope}%
\pgfpathrectangle{\pgfqpoint{0.765000in}{0.660000in}}{\pgfqpoint{4.620000in}{4.620000in}}%
\pgfusepath{clip}%
\pgfsetbuttcap%
\pgfsetroundjoin%
\definecolor{currentfill}{rgb}{1.000000,0.894118,0.788235}%
\pgfsetfillcolor{currentfill}%
\pgfsetlinewidth{0.000000pt}%
\definecolor{currentstroke}{rgb}{1.000000,0.894118,0.788235}%
\pgfsetstrokecolor{currentstroke}%
\pgfsetdash{}{0pt}%
\pgfpathmoveto{\pgfqpoint{4.233686in}{3.297917in}}%
\pgfpathlineto{\pgfqpoint{4.260633in}{3.282359in}}%
\pgfpathlineto{\pgfqpoint{4.260633in}{3.313475in}}%
\pgfpathlineto{\pgfqpoint{4.233686in}{3.329033in}}%
\pgfpathlineto{\pgfqpoint{4.233686in}{3.297917in}}%
\pgfpathclose%
\pgfusepath{fill}%
\end{pgfscope}%
\begin{pgfscope}%
\pgfpathrectangle{\pgfqpoint{0.765000in}{0.660000in}}{\pgfqpoint{4.620000in}{4.620000in}}%
\pgfusepath{clip}%
\pgfsetbuttcap%
\pgfsetroundjoin%
\definecolor{currentfill}{rgb}{1.000000,0.894118,0.788235}%
\pgfsetfillcolor{currentfill}%
\pgfsetlinewidth{0.000000pt}%
\definecolor{currentstroke}{rgb}{1.000000,0.894118,0.788235}%
\pgfsetstrokecolor{currentstroke}%
\pgfsetdash{}{0pt}%
\pgfpathmoveto{\pgfqpoint{14.674810in}{9.988960in}}%
\pgfpathlineto{\pgfqpoint{14.647863in}{9.973402in}}%
\pgfpathlineto{\pgfqpoint{14.674810in}{9.957844in}}%
\pgfpathlineto{\pgfqpoint{14.701758in}{9.973402in}}%
\pgfpathlineto{\pgfqpoint{14.674810in}{9.988960in}}%
\pgfpathclose%
\pgfusepath{fill}%
\end{pgfscope}%
\begin{pgfscope}%
\pgfpathrectangle{\pgfqpoint{0.765000in}{0.660000in}}{\pgfqpoint{4.620000in}{4.620000in}}%
\pgfusepath{clip}%
\pgfsetbuttcap%
\pgfsetroundjoin%
\definecolor{currentfill}{rgb}{1.000000,0.894118,0.788235}%
\pgfsetfillcolor{currentfill}%
\pgfsetlinewidth{0.000000pt}%
\definecolor{currentstroke}{rgb}{1.000000,0.894118,0.788235}%
\pgfsetstrokecolor{currentstroke}%
\pgfsetdash{}{0pt}%
\pgfpathmoveto{\pgfqpoint{14.674810in}{9.926728in}}%
\pgfpathlineto{\pgfqpoint{14.701758in}{9.942286in}}%
\pgfpathlineto{\pgfqpoint{14.701758in}{9.973402in}}%
\pgfpathlineto{\pgfqpoint{14.674810in}{9.957844in}}%
\pgfpathlineto{\pgfqpoint{14.674810in}{9.926728in}}%
\pgfpathclose%
\pgfusepath{fill}%
\end{pgfscope}%
\begin{pgfscope}%
\pgfpathrectangle{\pgfqpoint{0.765000in}{0.660000in}}{\pgfqpoint{4.620000in}{4.620000in}}%
\pgfusepath{clip}%
\pgfsetbuttcap%
\pgfsetroundjoin%
\definecolor{currentfill}{rgb}{1.000000,0.894118,0.788235}%
\pgfsetfillcolor{currentfill}%
\pgfsetlinewidth{0.000000pt}%
\definecolor{currentstroke}{rgb}{1.000000,0.894118,0.788235}%
\pgfsetstrokecolor{currentstroke}%
\pgfsetdash{}{0pt}%
\pgfpathmoveto{\pgfqpoint{14.647863in}{9.942286in}}%
\pgfpathlineto{\pgfqpoint{14.674810in}{9.926728in}}%
\pgfpathlineto{\pgfqpoint{14.674810in}{9.957844in}}%
\pgfpathlineto{\pgfqpoint{14.647863in}{9.973402in}}%
\pgfpathlineto{\pgfqpoint{14.647863in}{9.942286in}}%
\pgfpathclose%
\pgfusepath{fill}%
\end{pgfscope}%
\begin{pgfscope}%
\pgfpathrectangle{\pgfqpoint{0.765000in}{0.660000in}}{\pgfqpoint{4.620000in}{4.620000in}}%
\pgfusepath{clip}%
\pgfsetbuttcap%
\pgfsetroundjoin%
\definecolor{currentfill}{rgb}{1.000000,0.894118,0.788235}%
\pgfsetfillcolor{currentfill}%
\pgfsetlinewidth{0.000000pt}%
\definecolor{currentstroke}{rgb}{1.000000,0.894118,0.788235}%
\pgfsetstrokecolor{currentstroke}%
\pgfsetdash{}{0pt}%
\pgfpathmoveto{\pgfqpoint{4.287581in}{3.297917in}}%
\pgfpathlineto{\pgfqpoint{4.260633in}{3.313475in}}%
\pgfpathlineto{\pgfqpoint{14.674810in}{9.957844in}}%
\pgfpathlineto{\pgfqpoint{14.701758in}{9.942286in}}%
\pgfpathlineto{\pgfqpoint{4.287581in}{3.297917in}}%
\pgfpathclose%
\pgfusepath{fill}%
\end{pgfscope}%
\begin{pgfscope}%
\pgfpathrectangle{\pgfqpoint{0.765000in}{0.660000in}}{\pgfqpoint{4.620000in}{4.620000in}}%
\pgfusepath{clip}%
\pgfsetbuttcap%
\pgfsetroundjoin%
\definecolor{currentfill}{rgb}{1.000000,0.894118,0.788235}%
\pgfsetfillcolor{currentfill}%
\pgfsetlinewidth{0.000000pt}%
\definecolor{currentstroke}{rgb}{1.000000,0.894118,0.788235}%
\pgfsetstrokecolor{currentstroke}%
\pgfsetdash{}{0pt}%
\pgfpathmoveto{\pgfqpoint{4.260633in}{3.313475in}}%
\pgfpathlineto{\pgfqpoint{4.233686in}{3.297917in}}%
\pgfpathlineto{\pgfqpoint{14.647863in}{9.942286in}}%
\pgfpathlineto{\pgfqpoint{14.674810in}{9.957844in}}%
\pgfpathlineto{\pgfqpoint{4.260633in}{3.313475in}}%
\pgfpathclose%
\pgfusepath{fill}%
\end{pgfscope}%
\begin{pgfscope}%
\pgfpathrectangle{\pgfqpoint{0.765000in}{0.660000in}}{\pgfqpoint{4.620000in}{4.620000in}}%
\pgfusepath{clip}%
\pgfsetbuttcap%
\pgfsetroundjoin%
\definecolor{currentfill}{rgb}{1.000000,0.894118,0.788235}%
\pgfsetfillcolor{currentfill}%
\pgfsetlinewidth{0.000000pt}%
\definecolor{currentstroke}{rgb}{1.000000,0.894118,0.788235}%
\pgfsetstrokecolor{currentstroke}%
\pgfsetdash{}{0pt}%
\pgfpathmoveto{\pgfqpoint{4.260633in}{3.313475in}}%
\pgfpathlineto{\pgfqpoint{4.260633in}{3.344591in}}%
\pgfpathlineto{\pgfqpoint{14.674810in}{9.988960in}}%
\pgfpathlineto{\pgfqpoint{14.701758in}{9.942286in}}%
\pgfpathlineto{\pgfqpoint{4.260633in}{3.313475in}}%
\pgfpathclose%
\pgfusepath{fill}%
\end{pgfscope}%
\begin{pgfscope}%
\pgfpathrectangle{\pgfqpoint{0.765000in}{0.660000in}}{\pgfqpoint{4.620000in}{4.620000in}}%
\pgfusepath{clip}%
\pgfsetbuttcap%
\pgfsetroundjoin%
\definecolor{currentfill}{rgb}{1.000000,0.894118,0.788235}%
\pgfsetfillcolor{currentfill}%
\pgfsetlinewidth{0.000000pt}%
\definecolor{currentstroke}{rgb}{1.000000,0.894118,0.788235}%
\pgfsetstrokecolor{currentstroke}%
\pgfsetdash{}{0pt}%
\pgfpathmoveto{\pgfqpoint{4.287581in}{3.297917in}}%
\pgfpathlineto{\pgfqpoint{4.287581in}{3.329033in}}%
\pgfpathlineto{\pgfqpoint{14.701758in}{9.973402in}}%
\pgfpathlineto{\pgfqpoint{14.674810in}{9.957844in}}%
\pgfpathlineto{\pgfqpoint{4.287581in}{3.297917in}}%
\pgfpathclose%
\pgfusepath{fill}%
\end{pgfscope}%
\begin{pgfscope}%
\pgfpathrectangle{\pgfqpoint{0.765000in}{0.660000in}}{\pgfqpoint{4.620000in}{4.620000in}}%
\pgfusepath{clip}%
\pgfsetbuttcap%
\pgfsetroundjoin%
\definecolor{currentfill}{rgb}{1.000000,0.894118,0.788235}%
\pgfsetfillcolor{currentfill}%
\pgfsetlinewidth{0.000000pt}%
\definecolor{currentstroke}{rgb}{1.000000,0.894118,0.788235}%
\pgfsetstrokecolor{currentstroke}%
\pgfsetdash{}{0pt}%
\pgfpathmoveto{\pgfqpoint{4.233686in}{3.297917in}}%
\pgfpathlineto{\pgfqpoint{4.260633in}{3.282359in}}%
\pgfpathlineto{\pgfqpoint{14.674810in}{9.926728in}}%
\pgfpathlineto{\pgfqpoint{14.647863in}{9.942286in}}%
\pgfpathlineto{\pgfqpoint{4.233686in}{3.297917in}}%
\pgfpathclose%
\pgfusepath{fill}%
\end{pgfscope}%
\begin{pgfscope}%
\pgfpathrectangle{\pgfqpoint{0.765000in}{0.660000in}}{\pgfqpoint{4.620000in}{4.620000in}}%
\pgfusepath{clip}%
\pgfsetbuttcap%
\pgfsetroundjoin%
\definecolor{currentfill}{rgb}{1.000000,0.894118,0.788235}%
\pgfsetfillcolor{currentfill}%
\pgfsetlinewidth{0.000000pt}%
\definecolor{currentstroke}{rgb}{1.000000,0.894118,0.788235}%
\pgfsetstrokecolor{currentstroke}%
\pgfsetdash{}{0pt}%
\pgfpathmoveto{\pgfqpoint{4.260633in}{3.282359in}}%
\pgfpathlineto{\pgfqpoint{4.287581in}{3.297917in}}%
\pgfpathlineto{\pgfqpoint{14.701758in}{9.942286in}}%
\pgfpathlineto{\pgfqpoint{14.674810in}{9.926728in}}%
\pgfpathlineto{\pgfqpoint{4.260633in}{3.282359in}}%
\pgfpathclose%
\pgfusepath{fill}%
\end{pgfscope}%
\begin{pgfscope}%
\pgfpathrectangle{\pgfqpoint{0.765000in}{0.660000in}}{\pgfqpoint{4.620000in}{4.620000in}}%
\pgfusepath{clip}%
\pgfsetbuttcap%
\pgfsetroundjoin%
\definecolor{currentfill}{rgb}{1.000000,0.894118,0.788235}%
\pgfsetfillcolor{currentfill}%
\pgfsetlinewidth{0.000000pt}%
\definecolor{currentstroke}{rgb}{1.000000,0.894118,0.788235}%
\pgfsetstrokecolor{currentstroke}%
\pgfsetdash{}{0pt}%
\pgfpathmoveto{\pgfqpoint{4.260633in}{3.344591in}}%
\pgfpathlineto{\pgfqpoint{4.233686in}{3.329033in}}%
\pgfpathlineto{\pgfqpoint{14.647863in}{9.973402in}}%
\pgfpathlineto{\pgfqpoint{14.674810in}{9.988960in}}%
\pgfpathlineto{\pgfqpoint{4.260633in}{3.344591in}}%
\pgfpathclose%
\pgfusepath{fill}%
\end{pgfscope}%
\begin{pgfscope}%
\pgfpathrectangle{\pgfqpoint{0.765000in}{0.660000in}}{\pgfqpoint{4.620000in}{4.620000in}}%
\pgfusepath{clip}%
\pgfsetbuttcap%
\pgfsetroundjoin%
\definecolor{currentfill}{rgb}{1.000000,0.894118,0.788235}%
\pgfsetfillcolor{currentfill}%
\pgfsetlinewidth{0.000000pt}%
\definecolor{currentstroke}{rgb}{1.000000,0.894118,0.788235}%
\pgfsetstrokecolor{currentstroke}%
\pgfsetdash{}{0pt}%
\pgfpathmoveto{\pgfqpoint{4.287581in}{3.329033in}}%
\pgfpathlineto{\pgfqpoint{4.260633in}{3.344591in}}%
\pgfpathlineto{\pgfqpoint{14.674810in}{9.988960in}}%
\pgfpathlineto{\pgfqpoint{14.701758in}{9.973402in}}%
\pgfpathlineto{\pgfqpoint{4.287581in}{3.329033in}}%
\pgfpathclose%
\pgfusepath{fill}%
\end{pgfscope}%
\begin{pgfscope}%
\pgfpathrectangle{\pgfqpoint{0.765000in}{0.660000in}}{\pgfqpoint{4.620000in}{4.620000in}}%
\pgfusepath{clip}%
\pgfsetbuttcap%
\pgfsetroundjoin%
\definecolor{currentfill}{rgb}{1.000000,0.894118,0.788235}%
\pgfsetfillcolor{currentfill}%
\pgfsetlinewidth{0.000000pt}%
\definecolor{currentstroke}{rgb}{1.000000,0.894118,0.788235}%
\pgfsetstrokecolor{currentstroke}%
\pgfsetdash{}{0pt}%
\pgfpathmoveto{\pgfqpoint{4.233686in}{3.297917in}}%
\pgfpathlineto{\pgfqpoint{4.233686in}{3.329033in}}%
\pgfpathlineto{\pgfqpoint{14.647863in}{9.973402in}}%
\pgfpathlineto{\pgfqpoint{14.674810in}{9.926728in}}%
\pgfpathlineto{\pgfqpoint{4.233686in}{3.297917in}}%
\pgfpathclose%
\pgfusepath{fill}%
\end{pgfscope}%
\begin{pgfscope}%
\pgfpathrectangle{\pgfqpoint{0.765000in}{0.660000in}}{\pgfqpoint{4.620000in}{4.620000in}}%
\pgfusepath{clip}%
\pgfsetbuttcap%
\pgfsetroundjoin%
\definecolor{currentfill}{rgb}{1.000000,0.894118,0.788235}%
\pgfsetfillcolor{currentfill}%
\pgfsetlinewidth{0.000000pt}%
\definecolor{currentstroke}{rgb}{1.000000,0.894118,0.788235}%
\pgfsetstrokecolor{currentstroke}%
\pgfsetdash{}{0pt}%
\pgfpathmoveto{\pgfqpoint{4.260633in}{3.282359in}}%
\pgfpathlineto{\pgfqpoint{4.260633in}{3.313475in}}%
\pgfpathlineto{\pgfqpoint{14.674810in}{9.957844in}}%
\pgfpathlineto{\pgfqpoint{14.647863in}{9.942286in}}%
\pgfpathlineto{\pgfqpoint{4.260633in}{3.282359in}}%
\pgfpathclose%
\pgfusepath{fill}%
\end{pgfscope}%
\begin{pgfscope}%
\pgfpathrectangle{\pgfqpoint{0.765000in}{0.660000in}}{\pgfqpoint{4.620000in}{4.620000in}}%
\pgfusepath{clip}%
\pgfsetbuttcap%
\pgfsetroundjoin%
\definecolor{currentfill}{rgb}{1.000000,0.894118,0.788235}%
\pgfsetfillcolor{currentfill}%
\pgfsetlinewidth{0.000000pt}%
\definecolor{currentstroke}{rgb}{1.000000,0.894118,0.788235}%
\pgfsetstrokecolor{currentstroke}%
\pgfsetdash{}{0pt}%
\pgfpathmoveto{\pgfqpoint{4.260633in}{3.313475in}}%
\pgfpathlineto{\pgfqpoint{4.287581in}{3.329033in}}%
\pgfpathlineto{\pgfqpoint{14.701758in}{9.973402in}}%
\pgfpathlineto{\pgfqpoint{14.674810in}{9.957844in}}%
\pgfpathlineto{\pgfqpoint{4.260633in}{3.313475in}}%
\pgfpathclose%
\pgfusepath{fill}%
\end{pgfscope}%
\begin{pgfscope}%
\pgfpathrectangle{\pgfqpoint{0.765000in}{0.660000in}}{\pgfqpoint{4.620000in}{4.620000in}}%
\pgfusepath{clip}%
\pgfsetbuttcap%
\pgfsetroundjoin%
\definecolor{currentfill}{rgb}{1.000000,0.894118,0.788235}%
\pgfsetfillcolor{currentfill}%
\pgfsetlinewidth{0.000000pt}%
\definecolor{currentstroke}{rgb}{1.000000,0.894118,0.788235}%
\pgfsetstrokecolor{currentstroke}%
\pgfsetdash{}{0pt}%
\pgfpathmoveto{\pgfqpoint{4.233686in}{3.329033in}}%
\pgfpathlineto{\pgfqpoint{4.260633in}{3.313475in}}%
\pgfpathlineto{\pgfqpoint{14.674810in}{9.957844in}}%
\pgfpathlineto{\pgfqpoint{14.647863in}{9.973402in}}%
\pgfpathlineto{\pgfqpoint{4.233686in}{3.329033in}}%
\pgfpathclose%
\pgfusepath{fill}%
\end{pgfscope}%
\begin{pgfscope}%
\pgfpathrectangle{\pgfqpoint{0.765000in}{0.660000in}}{\pgfqpoint{4.620000in}{4.620000in}}%
\pgfusepath{clip}%
\pgfsetbuttcap%
\pgfsetroundjoin%
\definecolor{currentfill}{rgb}{1.000000,0.894118,0.788235}%
\pgfsetfillcolor{currentfill}%
\pgfsetlinewidth{0.000000pt}%
\definecolor{currentstroke}{rgb}{1.000000,0.894118,0.788235}%
\pgfsetstrokecolor{currentstroke}%
\pgfsetdash{}{0pt}%
\pgfpathmoveto{\pgfqpoint{2.263574in}{3.062189in}}%
\pgfpathlineto{\pgfqpoint{2.236626in}{3.046631in}}%
\pgfpathlineto{\pgfqpoint{2.263574in}{3.031073in}}%
\pgfpathlineto{\pgfqpoint{2.290521in}{3.046631in}}%
\pgfpathlineto{\pgfqpoint{2.263574in}{3.062189in}}%
\pgfpathclose%
\pgfusepath{fill}%
\end{pgfscope}%
\begin{pgfscope}%
\pgfpathrectangle{\pgfqpoint{0.765000in}{0.660000in}}{\pgfqpoint{4.620000in}{4.620000in}}%
\pgfusepath{clip}%
\pgfsetbuttcap%
\pgfsetroundjoin%
\definecolor{currentfill}{rgb}{1.000000,0.894118,0.788235}%
\pgfsetfillcolor{currentfill}%
\pgfsetlinewidth{0.000000pt}%
\definecolor{currentstroke}{rgb}{1.000000,0.894118,0.788235}%
\pgfsetstrokecolor{currentstroke}%
\pgfsetdash{}{0pt}%
\pgfpathmoveto{\pgfqpoint{2.263574in}{3.062189in}}%
\pgfpathlineto{\pgfqpoint{2.236626in}{3.046631in}}%
\pgfpathlineto{\pgfqpoint{2.236626in}{3.077747in}}%
\pgfpathlineto{\pgfqpoint{2.263574in}{3.093305in}}%
\pgfpathlineto{\pgfqpoint{2.263574in}{3.062189in}}%
\pgfpathclose%
\pgfusepath{fill}%
\end{pgfscope}%
\begin{pgfscope}%
\pgfpathrectangle{\pgfqpoint{0.765000in}{0.660000in}}{\pgfqpoint{4.620000in}{4.620000in}}%
\pgfusepath{clip}%
\pgfsetbuttcap%
\pgfsetroundjoin%
\definecolor{currentfill}{rgb}{1.000000,0.894118,0.788235}%
\pgfsetfillcolor{currentfill}%
\pgfsetlinewidth{0.000000pt}%
\definecolor{currentstroke}{rgb}{1.000000,0.894118,0.788235}%
\pgfsetstrokecolor{currentstroke}%
\pgfsetdash{}{0pt}%
\pgfpathmoveto{\pgfqpoint{2.263574in}{3.062189in}}%
\pgfpathlineto{\pgfqpoint{2.290521in}{3.046631in}}%
\pgfpathlineto{\pgfqpoint{2.290521in}{3.077747in}}%
\pgfpathlineto{\pgfqpoint{2.263574in}{3.093305in}}%
\pgfpathlineto{\pgfqpoint{2.263574in}{3.062189in}}%
\pgfpathclose%
\pgfusepath{fill}%
\end{pgfscope}%
\begin{pgfscope}%
\pgfpathrectangle{\pgfqpoint{0.765000in}{0.660000in}}{\pgfqpoint{4.620000in}{4.620000in}}%
\pgfusepath{clip}%
\pgfsetbuttcap%
\pgfsetroundjoin%
\definecolor{currentfill}{rgb}{1.000000,0.894118,0.788235}%
\pgfsetfillcolor{currentfill}%
\pgfsetlinewidth{0.000000pt}%
\definecolor{currentstroke}{rgb}{1.000000,0.894118,0.788235}%
\pgfsetstrokecolor{currentstroke}%
\pgfsetdash{}{0pt}%
\pgfpathmoveto{\pgfqpoint{2.421256in}{3.149423in}}%
\pgfpathlineto{\pgfqpoint{2.394309in}{3.133865in}}%
\pgfpathlineto{\pgfqpoint{2.421256in}{3.118307in}}%
\pgfpathlineto{\pgfqpoint{2.448203in}{3.133865in}}%
\pgfpathlineto{\pgfqpoint{2.421256in}{3.149423in}}%
\pgfpathclose%
\pgfusepath{fill}%
\end{pgfscope}%
\begin{pgfscope}%
\pgfpathrectangle{\pgfqpoint{0.765000in}{0.660000in}}{\pgfqpoint{4.620000in}{4.620000in}}%
\pgfusepath{clip}%
\pgfsetbuttcap%
\pgfsetroundjoin%
\definecolor{currentfill}{rgb}{1.000000,0.894118,0.788235}%
\pgfsetfillcolor{currentfill}%
\pgfsetlinewidth{0.000000pt}%
\definecolor{currentstroke}{rgb}{1.000000,0.894118,0.788235}%
\pgfsetstrokecolor{currentstroke}%
\pgfsetdash{}{0pt}%
\pgfpathmoveto{\pgfqpoint{2.421256in}{3.149423in}}%
\pgfpathlineto{\pgfqpoint{2.394309in}{3.133865in}}%
\pgfpathlineto{\pgfqpoint{2.394309in}{3.164981in}}%
\pgfpathlineto{\pgfqpoint{2.421256in}{3.180539in}}%
\pgfpathlineto{\pgfqpoint{2.421256in}{3.149423in}}%
\pgfpathclose%
\pgfusepath{fill}%
\end{pgfscope}%
\begin{pgfscope}%
\pgfpathrectangle{\pgfqpoint{0.765000in}{0.660000in}}{\pgfqpoint{4.620000in}{4.620000in}}%
\pgfusepath{clip}%
\pgfsetbuttcap%
\pgfsetroundjoin%
\definecolor{currentfill}{rgb}{1.000000,0.894118,0.788235}%
\pgfsetfillcolor{currentfill}%
\pgfsetlinewidth{0.000000pt}%
\definecolor{currentstroke}{rgb}{1.000000,0.894118,0.788235}%
\pgfsetstrokecolor{currentstroke}%
\pgfsetdash{}{0pt}%
\pgfpathmoveto{\pgfqpoint{2.421256in}{3.149423in}}%
\pgfpathlineto{\pgfqpoint{2.448203in}{3.133865in}}%
\pgfpathlineto{\pgfqpoint{2.448203in}{3.164981in}}%
\pgfpathlineto{\pgfqpoint{2.421256in}{3.180539in}}%
\pgfpathlineto{\pgfqpoint{2.421256in}{3.149423in}}%
\pgfpathclose%
\pgfusepath{fill}%
\end{pgfscope}%
\begin{pgfscope}%
\pgfpathrectangle{\pgfqpoint{0.765000in}{0.660000in}}{\pgfqpoint{4.620000in}{4.620000in}}%
\pgfusepath{clip}%
\pgfsetbuttcap%
\pgfsetroundjoin%
\definecolor{currentfill}{rgb}{1.000000,0.894118,0.788235}%
\pgfsetfillcolor{currentfill}%
\pgfsetlinewidth{0.000000pt}%
\definecolor{currentstroke}{rgb}{1.000000,0.894118,0.788235}%
\pgfsetstrokecolor{currentstroke}%
\pgfsetdash{}{0pt}%
\pgfpathmoveto{\pgfqpoint{2.263574in}{3.093305in}}%
\pgfpathlineto{\pgfqpoint{2.236626in}{3.077747in}}%
\pgfpathlineto{\pgfqpoint{2.263574in}{3.062189in}}%
\pgfpathlineto{\pgfqpoint{2.290521in}{3.077747in}}%
\pgfpathlineto{\pgfqpoint{2.263574in}{3.093305in}}%
\pgfpathclose%
\pgfusepath{fill}%
\end{pgfscope}%
\begin{pgfscope}%
\pgfpathrectangle{\pgfqpoint{0.765000in}{0.660000in}}{\pgfqpoint{4.620000in}{4.620000in}}%
\pgfusepath{clip}%
\pgfsetbuttcap%
\pgfsetroundjoin%
\definecolor{currentfill}{rgb}{1.000000,0.894118,0.788235}%
\pgfsetfillcolor{currentfill}%
\pgfsetlinewidth{0.000000pt}%
\definecolor{currentstroke}{rgb}{1.000000,0.894118,0.788235}%
\pgfsetstrokecolor{currentstroke}%
\pgfsetdash{}{0pt}%
\pgfpathmoveto{\pgfqpoint{2.263574in}{3.031073in}}%
\pgfpathlineto{\pgfqpoint{2.290521in}{3.046631in}}%
\pgfpathlineto{\pgfqpoint{2.290521in}{3.077747in}}%
\pgfpathlineto{\pgfqpoint{2.263574in}{3.062189in}}%
\pgfpathlineto{\pgfqpoint{2.263574in}{3.031073in}}%
\pgfpathclose%
\pgfusepath{fill}%
\end{pgfscope}%
\begin{pgfscope}%
\pgfpathrectangle{\pgfqpoint{0.765000in}{0.660000in}}{\pgfqpoint{4.620000in}{4.620000in}}%
\pgfusepath{clip}%
\pgfsetbuttcap%
\pgfsetroundjoin%
\definecolor{currentfill}{rgb}{1.000000,0.894118,0.788235}%
\pgfsetfillcolor{currentfill}%
\pgfsetlinewidth{0.000000pt}%
\definecolor{currentstroke}{rgb}{1.000000,0.894118,0.788235}%
\pgfsetstrokecolor{currentstroke}%
\pgfsetdash{}{0pt}%
\pgfpathmoveto{\pgfqpoint{2.236626in}{3.046631in}}%
\pgfpathlineto{\pgfqpoint{2.263574in}{3.031073in}}%
\pgfpathlineto{\pgfqpoint{2.263574in}{3.062189in}}%
\pgfpathlineto{\pgfqpoint{2.236626in}{3.077747in}}%
\pgfpathlineto{\pgfqpoint{2.236626in}{3.046631in}}%
\pgfpathclose%
\pgfusepath{fill}%
\end{pgfscope}%
\begin{pgfscope}%
\pgfpathrectangle{\pgfqpoint{0.765000in}{0.660000in}}{\pgfqpoint{4.620000in}{4.620000in}}%
\pgfusepath{clip}%
\pgfsetbuttcap%
\pgfsetroundjoin%
\definecolor{currentfill}{rgb}{1.000000,0.894118,0.788235}%
\pgfsetfillcolor{currentfill}%
\pgfsetlinewidth{0.000000pt}%
\definecolor{currentstroke}{rgb}{1.000000,0.894118,0.788235}%
\pgfsetstrokecolor{currentstroke}%
\pgfsetdash{}{0pt}%
\pgfpathmoveto{\pgfqpoint{2.421256in}{3.180539in}}%
\pgfpathlineto{\pgfqpoint{2.394309in}{3.164981in}}%
\pgfpathlineto{\pgfqpoint{2.421256in}{3.149423in}}%
\pgfpathlineto{\pgfqpoint{2.448203in}{3.164981in}}%
\pgfpathlineto{\pgfqpoint{2.421256in}{3.180539in}}%
\pgfpathclose%
\pgfusepath{fill}%
\end{pgfscope}%
\begin{pgfscope}%
\pgfpathrectangle{\pgfqpoint{0.765000in}{0.660000in}}{\pgfqpoint{4.620000in}{4.620000in}}%
\pgfusepath{clip}%
\pgfsetbuttcap%
\pgfsetroundjoin%
\definecolor{currentfill}{rgb}{1.000000,0.894118,0.788235}%
\pgfsetfillcolor{currentfill}%
\pgfsetlinewidth{0.000000pt}%
\definecolor{currentstroke}{rgb}{1.000000,0.894118,0.788235}%
\pgfsetstrokecolor{currentstroke}%
\pgfsetdash{}{0pt}%
\pgfpathmoveto{\pgfqpoint{2.421256in}{3.118307in}}%
\pgfpathlineto{\pgfqpoint{2.448203in}{3.133865in}}%
\pgfpathlineto{\pgfqpoint{2.448203in}{3.164981in}}%
\pgfpathlineto{\pgfqpoint{2.421256in}{3.149423in}}%
\pgfpathlineto{\pgfqpoint{2.421256in}{3.118307in}}%
\pgfpathclose%
\pgfusepath{fill}%
\end{pgfscope}%
\begin{pgfscope}%
\pgfpathrectangle{\pgfqpoint{0.765000in}{0.660000in}}{\pgfqpoint{4.620000in}{4.620000in}}%
\pgfusepath{clip}%
\pgfsetbuttcap%
\pgfsetroundjoin%
\definecolor{currentfill}{rgb}{1.000000,0.894118,0.788235}%
\pgfsetfillcolor{currentfill}%
\pgfsetlinewidth{0.000000pt}%
\definecolor{currentstroke}{rgb}{1.000000,0.894118,0.788235}%
\pgfsetstrokecolor{currentstroke}%
\pgfsetdash{}{0pt}%
\pgfpathmoveto{\pgfqpoint{2.394309in}{3.133865in}}%
\pgfpathlineto{\pgfqpoint{2.421256in}{3.118307in}}%
\pgfpathlineto{\pgfqpoint{2.421256in}{3.149423in}}%
\pgfpathlineto{\pgfqpoint{2.394309in}{3.164981in}}%
\pgfpathlineto{\pgfqpoint{2.394309in}{3.133865in}}%
\pgfpathclose%
\pgfusepath{fill}%
\end{pgfscope}%
\begin{pgfscope}%
\pgfpathrectangle{\pgfqpoint{0.765000in}{0.660000in}}{\pgfqpoint{4.620000in}{4.620000in}}%
\pgfusepath{clip}%
\pgfsetbuttcap%
\pgfsetroundjoin%
\definecolor{currentfill}{rgb}{1.000000,0.894118,0.788235}%
\pgfsetfillcolor{currentfill}%
\pgfsetlinewidth{0.000000pt}%
\definecolor{currentstroke}{rgb}{1.000000,0.894118,0.788235}%
\pgfsetstrokecolor{currentstroke}%
\pgfsetdash{}{0pt}%
\pgfpathmoveto{\pgfqpoint{2.263574in}{3.062189in}}%
\pgfpathlineto{\pgfqpoint{2.236626in}{3.046631in}}%
\pgfpathlineto{\pgfqpoint{2.394309in}{3.133865in}}%
\pgfpathlineto{\pgfqpoint{2.421256in}{3.149423in}}%
\pgfpathlineto{\pgfqpoint{2.263574in}{3.062189in}}%
\pgfpathclose%
\pgfusepath{fill}%
\end{pgfscope}%
\begin{pgfscope}%
\pgfpathrectangle{\pgfqpoint{0.765000in}{0.660000in}}{\pgfqpoint{4.620000in}{4.620000in}}%
\pgfusepath{clip}%
\pgfsetbuttcap%
\pgfsetroundjoin%
\definecolor{currentfill}{rgb}{1.000000,0.894118,0.788235}%
\pgfsetfillcolor{currentfill}%
\pgfsetlinewidth{0.000000pt}%
\definecolor{currentstroke}{rgb}{1.000000,0.894118,0.788235}%
\pgfsetstrokecolor{currentstroke}%
\pgfsetdash{}{0pt}%
\pgfpathmoveto{\pgfqpoint{2.290521in}{3.046631in}}%
\pgfpathlineto{\pgfqpoint{2.263574in}{3.062189in}}%
\pgfpathlineto{\pgfqpoint{2.421256in}{3.149423in}}%
\pgfpathlineto{\pgfqpoint{2.448203in}{3.133865in}}%
\pgfpathlineto{\pgfqpoint{2.290521in}{3.046631in}}%
\pgfpathclose%
\pgfusepath{fill}%
\end{pgfscope}%
\begin{pgfscope}%
\pgfpathrectangle{\pgfqpoint{0.765000in}{0.660000in}}{\pgfqpoint{4.620000in}{4.620000in}}%
\pgfusepath{clip}%
\pgfsetbuttcap%
\pgfsetroundjoin%
\definecolor{currentfill}{rgb}{1.000000,0.894118,0.788235}%
\pgfsetfillcolor{currentfill}%
\pgfsetlinewidth{0.000000pt}%
\definecolor{currentstroke}{rgb}{1.000000,0.894118,0.788235}%
\pgfsetstrokecolor{currentstroke}%
\pgfsetdash{}{0pt}%
\pgfpathmoveto{\pgfqpoint{2.263574in}{3.062189in}}%
\pgfpathlineto{\pgfqpoint{2.263574in}{3.093305in}}%
\pgfpathlineto{\pgfqpoint{2.421256in}{3.180539in}}%
\pgfpathlineto{\pgfqpoint{2.448203in}{3.133865in}}%
\pgfpathlineto{\pgfqpoint{2.263574in}{3.062189in}}%
\pgfpathclose%
\pgfusepath{fill}%
\end{pgfscope}%
\begin{pgfscope}%
\pgfpathrectangle{\pgfqpoint{0.765000in}{0.660000in}}{\pgfqpoint{4.620000in}{4.620000in}}%
\pgfusepath{clip}%
\pgfsetbuttcap%
\pgfsetroundjoin%
\definecolor{currentfill}{rgb}{1.000000,0.894118,0.788235}%
\pgfsetfillcolor{currentfill}%
\pgfsetlinewidth{0.000000pt}%
\definecolor{currentstroke}{rgb}{1.000000,0.894118,0.788235}%
\pgfsetstrokecolor{currentstroke}%
\pgfsetdash{}{0pt}%
\pgfpathmoveto{\pgfqpoint{2.290521in}{3.046631in}}%
\pgfpathlineto{\pgfqpoint{2.290521in}{3.077747in}}%
\pgfpathlineto{\pgfqpoint{2.448203in}{3.164981in}}%
\pgfpathlineto{\pgfqpoint{2.421256in}{3.149423in}}%
\pgfpathlineto{\pgfqpoint{2.290521in}{3.046631in}}%
\pgfpathclose%
\pgfusepath{fill}%
\end{pgfscope}%
\begin{pgfscope}%
\pgfpathrectangle{\pgfqpoint{0.765000in}{0.660000in}}{\pgfqpoint{4.620000in}{4.620000in}}%
\pgfusepath{clip}%
\pgfsetbuttcap%
\pgfsetroundjoin%
\definecolor{currentfill}{rgb}{1.000000,0.894118,0.788235}%
\pgfsetfillcolor{currentfill}%
\pgfsetlinewidth{0.000000pt}%
\definecolor{currentstroke}{rgb}{1.000000,0.894118,0.788235}%
\pgfsetstrokecolor{currentstroke}%
\pgfsetdash{}{0pt}%
\pgfpathmoveto{\pgfqpoint{2.236626in}{3.046631in}}%
\pgfpathlineto{\pgfqpoint{2.263574in}{3.031073in}}%
\pgfpathlineto{\pgfqpoint{2.421256in}{3.118307in}}%
\pgfpathlineto{\pgfqpoint{2.394309in}{3.133865in}}%
\pgfpathlineto{\pgfqpoint{2.236626in}{3.046631in}}%
\pgfpathclose%
\pgfusepath{fill}%
\end{pgfscope}%
\begin{pgfscope}%
\pgfpathrectangle{\pgfqpoint{0.765000in}{0.660000in}}{\pgfqpoint{4.620000in}{4.620000in}}%
\pgfusepath{clip}%
\pgfsetbuttcap%
\pgfsetroundjoin%
\definecolor{currentfill}{rgb}{1.000000,0.894118,0.788235}%
\pgfsetfillcolor{currentfill}%
\pgfsetlinewidth{0.000000pt}%
\definecolor{currentstroke}{rgb}{1.000000,0.894118,0.788235}%
\pgfsetstrokecolor{currentstroke}%
\pgfsetdash{}{0pt}%
\pgfpathmoveto{\pgfqpoint{2.263574in}{3.031073in}}%
\pgfpathlineto{\pgfqpoint{2.290521in}{3.046631in}}%
\pgfpathlineto{\pgfqpoint{2.448203in}{3.133865in}}%
\pgfpathlineto{\pgfqpoint{2.421256in}{3.118307in}}%
\pgfpathlineto{\pgfqpoint{2.263574in}{3.031073in}}%
\pgfpathclose%
\pgfusepath{fill}%
\end{pgfscope}%
\begin{pgfscope}%
\pgfpathrectangle{\pgfqpoint{0.765000in}{0.660000in}}{\pgfqpoint{4.620000in}{4.620000in}}%
\pgfusepath{clip}%
\pgfsetbuttcap%
\pgfsetroundjoin%
\definecolor{currentfill}{rgb}{1.000000,0.894118,0.788235}%
\pgfsetfillcolor{currentfill}%
\pgfsetlinewidth{0.000000pt}%
\definecolor{currentstroke}{rgb}{1.000000,0.894118,0.788235}%
\pgfsetstrokecolor{currentstroke}%
\pgfsetdash{}{0pt}%
\pgfpathmoveto{\pgfqpoint{2.263574in}{3.093305in}}%
\pgfpathlineto{\pgfqpoint{2.236626in}{3.077747in}}%
\pgfpathlineto{\pgfqpoint{2.394309in}{3.164981in}}%
\pgfpathlineto{\pgfqpoint{2.421256in}{3.180539in}}%
\pgfpathlineto{\pgfqpoint{2.263574in}{3.093305in}}%
\pgfpathclose%
\pgfusepath{fill}%
\end{pgfscope}%
\begin{pgfscope}%
\pgfpathrectangle{\pgfqpoint{0.765000in}{0.660000in}}{\pgfqpoint{4.620000in}{4.620000in}}%
\pgfusepath{clip}%
\pgfsetbuttcap%
\pgfsetroundjoin%
\definecolor{currentfill}{rgb}{1.000000,0.894118,0.788235}%
\pgfsetfillcolor{currentfill}%
\pgfsetlinewidth{0.000000pt}%
\definecolor{currentstroke}{rgb}{1.000000,0.894118,0.788235}%
\pgfsetstrokecolor{currentstroke}%
\pgfsetdash{}{0pt}%
\pgfpathmoveto{\pgfqpoint{2.290521in}{3.077747in}}%
\pgfpathlineto{\pgfqpoint{2.263574in}{3.093305in}}%
\pgfpathlineto{\pgfqpoint{2.421256in}{3.180539in}}%
\pgfpathlineto{\pgfqpoint{2.448203in}{3.164981in}}%
\pgfpathlineto{\pgfqpoint{2.290521in}{3.077747in}}%
\pgfpathclose%
\pgfusepath{fill}%
\end{pgfscope}%
\begin{pgfscope}%
\pgfpathrectangle{\pgfqpoint{0.765000in}{0.660000in}}{\pgfqpoint{4.620000in}{4.620000in}}%
\pgfusepath{clip}%
\pgfsetbuttcap%
\pgfsetroundjoin%
\definecolor{currentfill}{rgb}{1.000000,0.894118,0.788235}%
\pgfsetfillcolor{currentfill}%
\pgfsetlinewidth{0.000000pt}%
\definecolor{currentstroke}{rgb}{1.000000,0.894118,0.788235}%
\pgfsetstrokecolor{currentstroke}%
\pgfsetdash{}{0pt}%
\pgfpathmoveto{\pgfqpoint{2.236626in}{3.046631in}}%
\pgfpathlineto{\pgfqpoint{2.236626in}{3.077747in}}%
\pgfpathlineto{\pgfqpoint{2.394309in}{3.164981in}}%
\pgfpathlineto{\pgfqpoint{2.421256in}{3.118307in}}%
\pgfpathlineto{\pgfqpoint{2.236626in}{3.046631in}}%
\pgfpathclose%
\pgfusepath{fill}%
\end{pgfscope}%
\begin{pgfscope}%
\pgfpathrectangle{\pgfqpoint{0.765000in}{0.660000in}}{\pgfqpoint{4.620000in}{4.620000in}}%
\pgfusepath{clip}%
\pgfsetbuttcap%
\pgfsetroundjoin%
\definecolor{currentfill}{rgb}{1.000000,0.894118,0.788235}%
\pgfsetfillcolor{currentfill}%
\pgfsetlinewidth{0.000000pt}%
\definecolor{currentstroke}{rgb}{1.000000,0.894118,0.788235}%
\pgfsetstrokecolor{currentstroke}%
\pgfsetdash{}{0pt}%
\pgfpathmoveto{\pgfqpoint{2.263574in}{3.031073in}}%
\pgfpathlineto{\pgfqpoint{2.263574in}{3.062189in}}%
\pgfpathlineto{\pgfqpoint{2.421256in}{3.149423in}}%
\pgfpathlineto{\pgfqpoint{2.394309in}{3.133865in}}%
\pgfpathlineto{\pgfqpoint{2.263574in}{3.031073in}}%
\pgfpathclose%
\pgfusepath{fill}%
\end{pgfscope}%
\begin{pgfscope}%
\pgfpathrectangle{\pgfqpoint{0.765000in}{0.660000in}}{\pgfqpoint{4.620000in}{4.620000in}}%
\pgfusepath{clip}%
\pgfsetbuttcap%
\pgfsetroundjoin%
\definecolor{currentfill}{rgb}{1.000000,0.894118,0.788235}%
\pgfsetfillcolor{currentfill}%
\pgfsetlinewidth{0.000000pt}%
\definecolor{currentstroke}{rgb}{1.000000,0.894118,0.788235}%
\pgfsetstrokecolor{currentstroke}%
\pgfsetdash{}{0pt}%
\pgfpathmoveto{\pgfqpoint{2.263574in}{3.062189in}}%
\pgfpathlineto{\pgfqpoint{2.290521in}{3.077747in}}%
\pgfpathlineto{\pgfqpoint{2.448203in}{3.164981in}}%
\pgfpathlineto{\pgfqpoint{2.421256in}{3.149423in}}%
\pgfpathlineto{\pgfqpoint{2.263574in}{3.062189in}}%
\pgfpathclose%
\pgfusepath{fill}%
\end{pgfscope}%
\begin{pgfscope}%
\pgfpathrectangle{\pgfqpoint{0.765000in}{0.660000in}}{\pgfqpoint{4.620000in}{4.620000in}}%
\pgfusepath{clip}%
\pgfsetbuttcap%
\pgfsetroundjoin%
\definecolor{currentfill}{rgb}{1.000000,0.894118,0.788235}%
\pgfsetfillcolor{currentfill}%
\pgfsetlinewidth{0.000000pt}%
\definecolor{currentstroke}{rgb}{1.000000,0.894118,0.788235}%
\pgfsetstrokecolor{currentstroke}%
\pgfsetdash{}{0pt}%
\pgfpathmoveto{\pgfqpoint{2.236626in}{3.077747in}}%
\pgfpathlineto{\pgfqpoint{2.263574in}{3.062189in}}%
\pgfpathlineto{\pgfqpoint{2.421256in}{3.149423in}}%
\pgfpathlineto{\pgfqpoint{2.394309in}{3.164981in}}%
\pgfpathlineto{\pgfqpoint{2.236626in}{3.077747in}}%
\pgfpathclose%
\pgfusepath{fill}%
\end{pgfscope}%
\begin{pgfscope}%
\pgfpathrectangle{\pgfqpoint{0.765000in}{0.660000in}}{\pgfqpoint{4.620000in}{4.620000in}}%
\pgfusepath{clip}%
\pgfsetbuttcap%
\pgfsetroundjoin%
\definecolor{currentfill}{rgb}{1.000000,0.894118,0.788235}%
\pgfsetfillcolor{currentfill}%
\pgfsetlinewidth{0.000000pt}%
\definecolor{currentstroke}{rgb}{1.000000,0.894118,0.788235}%
\pgfsetstrokecolor{currentstroke}%
\pgfsetdash{}{0pt}%
\pgfpathmoveto{\pgfqpoint{2.263574in}{3.062189in}}%
\pgfpathlineto{\pgfqpoint{2.236626in}{3.046631in}}%
\pgfpathlineto{\pgfqpoint{2.263574in}{3.031073in}}%
\pgfpathlineto{\pgfqpoint{2.290521in}{3.046631in}}%
\pgfpathlineto{\pgfqpoint{2.263574in}{3.062189in}}%
\pgfpathclose%
\pgfusepath{fill}%
\end{pgfscope}%
\begin{pgfscope}%
\pgfpathrectangle{\pgfqpoint{0.765000in}{0.660000in}}{\pgfqpoint{4.620000in}{4.620000in}}%
\pgfusepath{clip}%
\pgfsetbuttcap%
\pgfsetroundjoin%
\definecolor{currentfill}{rgb}{1.000000,0.894118,0.788235}%
\pgfsetfillcolor{currentfill}%
\pgfsetlinewidth{0.000000pt}%
\definecolor{currentstroke}{rgb}{1.000000,0.894118,0.788235}%
\pgfsetstrokecolor{currentstroke}%
\pgfsetdash{}{0pt}%
\pgfpathmoveto{\pgfqpoint{2.263574in}{3.062189in}}%
\pgfpathlineto{\pgfqpoint{2.236626in}{3.046631in}}%
\pgfpathlineto{\pgfqpoint{2.236626in}{3.077747in}}%
\pgfpathlineto{\pgfqpoint{2.263574in}{3.093305in}}%
\pgfpathlineto{\pgfqpoint{2.263574in}{3.062189in}}%
\pgfpathclose%
\pgfusepath{fill}%
\end{pgfscope}%
\begin{pgfscope}%
\pgfpathrectangle{\pgfqpoint{0.765000in}{0.660000in}}{\pgfqpoint{4.620000in}{4.620000in}}%
\pgfusepath{clip}%
\pgfsetbuttcap%
\pgfsetroundjoin%
\definecolor{currentfill}{rgb}{1.000000,0.894118,0.788235}%
\pgfsetfillcolor{currentfill}%
\pgfsetlinewidth{0.000000pt}%
\definecolor{currentstroke}{rgb}{1.000000,0.894118,0.788235}%
\pgfsetstrokecolor{currentstroke}%
\pgfsetdash{}{0pt}%
\pgfpathmoveto{\pgfqpoint{2.263574in}{3.062189in}}%
\pgfpathlineto{\pgfqpoint{2.290521in}{3.046631in}}%
\pgfpathlineto{\pgfqpoint{2.290521in}{3.077747in}}%
\pgfpathlineto{\pgfqpoint{2.263574in}{3.093305in}}%
\pgfpathlineto{\pgfqpoint{2.263574in}{3.062189in}}%
\pgfpathclose%
\pgfusepath{fill}%
\end{pgfscope}%
\begin{pgfscope}%
\pgfpathrectangle{\pgfqpoint{0.765000in}{0.660000in}}{\pgfqpoint{4.620000in}{4.620000in}}%
\pgfusepath{clip}%
\pgfsetbuttcap%
\pgfsetroundjoin%
\definecolor{currentfill}{rgb}{1.000000,0.894118,0.788235}%
\pgfsetfillcolor{currentfill}%
\pgfsetlinewidth{0.000000pt}%
\definecolor{currentstroke}{rgb}{1.000000,0.894118,0.788235}%
\pgfsetstrokecolor{currentstroke}%
\pgfsetdash{}{0pt}%
\pgfpathmoveto{\pgfqpoint{-7.119467in}{-2.455033in}}%
\pgfpathlineto{\pgfqpoint{-7.146414in}{-2.470591in}}%
\pgfpathlineto{\pgfqpoint{-7.119467in}{-2.486149in}}%
\pgfpathlineto{\pgfqpoint{-7.092520in}{-2.470591in}}%
\pgfpathlineto{\pgfqpoint{-7.119467in}{-2.455033in}}%
\pgfpathclose%
\pgfusepath{fill}%
\end{pgfscope}%
\begin{pgfscope}%
\pgfpathrectangle{\pgfqpoint{0.765000in}{0.660000in}}{\pgfqpoint{4.620000in}{4.620000in}}%
\pgfusepath{clip}%
\pgfsetbuttcap%
\pgfsetroundjoin%
\definecolor{currentfill}{rgb}{1.000000,0.894118,0.788235}%
\pgfsetfillcolor{currentfill}%
\pgfsetlinewidth{0.000000pt}%
\definecolor{currentstroke}{rgb}{1.000000,0.894118,0.788235}%
\pgfsetstrokecolor{currentstroke}%
\pgfsetdash{}{0pt}%
\pgfpathmoveto{\pgfqpoint{-7.119467in}{-2.455033in}}%
\pgfpathlineto{\pgfqpoint{-7.146414in}{-2.470591in}}%
\pgfpathlineto{\pgfqpoint{-7.146414in}{-2.439475in}}%
\pgfpathlineto{\pgfqpoint{-7.119467in}{-2.423917in}}%
\pgfpathlineto{\pgfqpoint{-7.119467in}{-2.455033in}}%
\pgfpathclose%
\pgfusepath{fill}%
\end{pgfscope}%
\begin{pgfscope}%
\pgfpathrectangle{\pgfqpoint{0.765000in}{0.660000in}}{\pgfqpoint{4.620000in}{4.620000in}}%
\pgfusepath{clip}%
\pgfsetbuttcap%
\pgfsetroundjoin%
\definecolor{currentfill}{rgb}{1.000000,0.894118,0.788235}%
\pgfsetfillcolor{currentfill}%
\pgfsetlinewidth{0.000000pt}%
\definecolor{currentstroke}{rgb}{1.000000,0.894118,0.788235}%
\pgfsetstrokecolor{currentstroke}%
\pgfsetdash{}{0pt}%
\pgfpathmoveto{\pgfqpoint{-7.119467in}{-2.455033in}}%
\pgfpathlineto{\pgfqpoint{-7.092520in}{-2.470591in}}%
\pgfpathlineto{\pgfqpoint{-7.092520in}{-2.439475in}}%
\pgfpathlineto{\pgfqpoint{-7.119467in}{-2.423917in}}%
\pgfpathlineto{\pgfqpoint{-7.119467in}{-2.455033in}}%
\pgfpathclose%
\pgfusepath{fill}%
\end{pgfscope}%
\begin{pgfscope}%
\pgfpathrectangle{\pgfqpoint{0.765000in}{0.660000in}}{\pgfqpoint{4.620000in}{4.620000in}}%
\pgfusepath{clip}%
\pgfsetbuttcap%
\pgfsetroundjoin%
\definecolor{currentfill}{rgb}{1.000000,0.894118,0.788235}%
\pgfsetfillcolor{currentfill}%
\pgfsetlinewidth{0.000000pt}%
\definecolor{currentstroke}{rgb}{1.000000,0.894118,0.788235}%
\pgfsetstrokecolor{currentstroke}%
\pgfsetdash{}{0pt}%
\pgfpathmoveto{\pgfqpoint{2.263574in}{3.093305in}}%
\pgfpathlineto{\pgfqpoint{2.236626in}{3.077747in}}%
\pgfpathlineto{\pgfqpoint{2.263574in}{3.062189in}}%
\pgfpathlineto{\pgfqpoint{2.290521in}{3.077747in}}%
\pgfpathlineto{\pgfqpoint{2.263574in}{3.093305in}}%
\pgfpathclose%
\pgfusepath{fill}%
\end{pgfscope}%
\begin{pgfscope}%
\pgfpathrectangle{\pgfqpoint{0.765000in}{0.660000in}}{\pgfqpoint{4.620000in}{4.620000in}}%
\pgfusepath{clip}%
\pgfsetbuttcap%
\pgfsetroundjoin%
\definecolor{currentfill}{rgb}{1.000000,0.894118,0.788235}%
\pgfsetfillcolor{currentfill}%
\pgfsetlinewidth{0.000000pt}%
\definecolor{currentstroke}{rgb}{1.000000,0.894118,0.788235}%
\pgfsetstrokecolor{currentstroke}%
\pgfsetdash{}{0pt}%
\pgfpathmoveto{\pgfqpoint{2.263574in}{3.031073in}}%
\pgfpathlineto{\pgfqpoint{2.290521in}{3.046631in}}%
\pgfpathlineto{\pgfqpoint{2.290521in}{3.077747in}}%
\pgfpathlineto{\pgfqpoint{2.263574in}{3.062189in}}%
\pgfpathlineto{\pgfqpoint{2.263574in}{3.031073in}}%
\pgfpathclose%
\pgfusepath{fill}%
\end{pgfscope}%
\begin{pgfscope}%
\pgfpathrectangle{\pgfqpoint{0.765000in}{0.660000in}}{\pgfqpoint{4.620000in}{4.620000in}}%
\pgfusepath{clip}%
\pgfsetbuttcap%
\pgfsetroundjoin%
\definecolor{currentfill}{rgb}{1.000000,0.894118,0.788235}%
\pgfsetfillcolor{currentfill}%
\pgfsetlinewidth{0.000000pt}%
\definecolor{currentstroke}{rgb}{1.000000,0.894118,0.788235}%
\pgfsetstrokecolor{currentstroke}%
\pgfsetdash{}{0pt}%
\pgfpathmoveto{\pgfqpoint{2.236626in}{3.046631in}}%
\pgfpathlineto{\pgfqpoint{2.263574in}{3.031073in}}%
\pgfpathlineto{\pgfqpoint{2.263574in}{3.062189in}}%
\pgfpathlineto{\pgfqpoint{2.236626in}{3.077747in}}%
\pgfpathlineto{\pgfqpoint{2.236626in}{3.046631in}}%
\pgfpathclose%
\pgfusepath{fill}%
\end{pgfscope}%
\begin{pgfscope}%
\pgfpathrectangle{\pgfqpoint{0.765000in}{0.660000in}}{\pgfqpoint{4.620000in}{4.620000in}}%
\pgfusepath{clip}%
\pgfsetbuttcap%
\pgfsetroundjoin%
\definecolor{currentfill}{rgb}{1.000000,0.894118,0.788235}%
\pgfsetfillcolor{currentfill}%
\pgfsetlinewidth{0.000000pt}%
\definecolor{currentstroke}{rgb}{1.000000,0.894118,0.788235}%
\pgfsetstrokecolor{currentstroke}%
\pgfsetdash{}{0pt}%
\pgfpathmoveto{\pgfqpoint{-7.119467in}{-2.423917in}}%
\pgfpathlineto{\pgfqpoint{-7.146414in}{-2.439475in}}%
\pgfpathlineto{\pgfqpoint{-7.119467in}{-2.455033in}}%
\pgfpathlineto{\pgfqpoint{-7.092520in}{-2.439475in}}%
\pgfpathlineto{\pgfqpoint{-7.119467in}{-2.423917in}}%
\pgfpathclose%
\pgfusepath{fill}%
\end{pgfscope}%
\begin{pgfscope}%
\pgfpathrectangle{\pgfqpoint{0.765000in}{0.660000in}}{\pgfqpoint{4.620000in}{4.620000in}}%
\pgfusepath{clip}%
\pgfsetbuttcap%
\pgfsetroundjoin%
\definecolor{currentfill}{rgb}{1.000000,0.894118,0.788235}%
\pgfsetfillcolor{currentfill}%
\pgfsetlinewidth{0.000000pt}%
\definecolor{currentstroke}{rgb}{1.000000,0.894118,0.788235}%
\pgfsetstrokecolor{currentstroke}%
\pgfsetdash{}{0pt}%
\pgfpathmoveto{\pgfqpoint{-7.119467in}{-2.486149in}}%
\pgfpathlineto{\pgfqpoint{-7.092520in}{-2.470591in}}%
\pgfpathlineto{\pgfqpoint{-7.092520in}{-2.439475in}}%
\pgfpathlineto{\pgfqpoint{-7.119467in}{-2.455033in}}%
\pgfpathlineto{\pgfqpoint{-7.119467in}{-2.486149in}}%
\pgfpathclose%
\pgfusepath{fill}%
\end{pgfscope}%
\begin{pgfscope}%
\pgfpathrectangle{\pgfqpoint{0.765000in}{0.660000in}}{\pgfqpoint{4.620000in}{4.620000in}}%
\pgfusepath{clip}%
\pgfsetbuttcap%
\pgfsetroundjoin%
\definecolor{currentfill}{rgb}{1.000000,0.894118,0.788235}%
\pgfsetfillcolor{currentfill}%
\pgfsetlinewidth{0.000000pt}%
\definecolor{currentstroke}{rgb}{1.000000,0.894118,0.788235}%
\pgfsetstrokecolor{currentstroke}%
\pgfsetdash{}{0pt}%
\pgfpathmoveto{\pgfqpoint{-7.146414in}{-2.470591in}}%
\pgfpathlineto{\pgfqpoint{-7.119467in}{-2.486149in}}%
\pgfpathlineto{\pgfqpoint{-7.119467in}{-2.455033in}}%
\pgfpathlineto{\pgfqpoint{-7.146414in}{-2.439475in}}%
\pgfpathlineto{\pgfqpoint{-7.146414in}{-2.470591in}}%
\pgfpathclose%
\pgfusepath{fill}%
\end{pgfscope}%
\begin{pgfscope}%
\pgfpathrectangle{\pgfqpoint{0.765000in}{0.660000in}}{\pgfqpoint{4.620000in}{4.620000in}}%
\pgfusepath{clip}%
\pgfsetbuttcap%
\pgfsetroundjoin%
\definecolor{currentfill}{rgb}{1.000000,0.894118,0.788235}%
\pgfsetfillcolor{currentfill}%
\pgfsetlinewidth{0.000000pt}%
\definecolor{currentstroke}{rgb}{1.000000,0.894118,0.788235}%
\pgfsetstrokecolor{currentstroke}%
\pgfsetdash{}{0pt}%
\pgfpathmoveto{\pgfqpoint{2.263574in}{3.062189in}}%
\pgfpathlineto{\pgfqpoint{2.236626in}{3.046631in}}%
\pgfpathlineto{\pgfqpoint{-7.146414in}{-2.470591in}}%
\pgfpathlineto{\pgfqpoint{-7.119467in}{-2.455033in}}%
\pgfpathlineto{\pgfqpoint{2.263574in}{3.062189in}}%
\pgfpathclose%
\pgfusepath{fill}%
\end{pgfscope}%
\begin{pgfscope}%
\pgfpathrectangle{\pgfqpoint{0.765000in}{0.660000in}}{\pgfqpoint{4.620000in}{4.620000in}}%
\pgfusepath{clip}%
\pgfsetbuttcap%
\pgfsetroundjoin%
\definecolor{currentfill}{rgb}{1.000000,0.894118,0.788235}%
\pgfsetfillcolor{currentfill}%
\pgfsetlinewidth{0.000000pt}%
\definecolor{currentstroke}{rgb}{1.000000,0.894118,0.788235}%
\pgfsetstrokecolor{currentstroke}%
\pgfsetdash{}{0pt}%
\pgfpathmoveto{\pgfqpoint{2.290521in}{3.046631in}}%
\pgfpathlineto{\pgfqpoint{2.263574in}{3.062189in}}%
\pgfpathlineto{\pgfqpoint{-7.119467in}{-2.455033in}}%
\pgfpathlineto{\pgfqpoint{-7.092520in}{-2.470591in}}%
\pgfpathlineto{\pgfqpoint{2.290521in}{3.046631in}}%
\pgfpathclose%
\pgfusepath{fill}%
\end{pgfscope}%
\begin{pgfscope}%
\pgfpathrectangle{\pgfqpoint{0.765000in}{0.660000in}}{\pgfqpoint{4.620000in}{4.620000in}}%
\pgfusepath{clip}%
\pgfsetbuttcap%
\pgfsetroundjoin%
\definecolor{currentfill}{rgb}{1.000000,0.894118,0.788235}%
\pgfsetfillcolor{currentfill}%
\pgfsetlinewidth{0.000000pt}%
\definecolor{currentstroke}{rgb}{1.000000,0.894118,0.788235}%
\pgfsetstrokecolor{currentstroke}%
\pgfsetdash{}{0pt}%
\pgfpathmoveto{\pgfqpoint{2.263574in}{3.062189in}}%
\pgfpathlineto{\pgfqpoint{2.263574in}{3.093305in}}%
\pgfpathlineto{\pgfqpoint{-7.119467in}{-2.423917in}}%
\pgfpathlineto{\pgfqpoint{-7.092520in}{-2.470591in}}%
\pgfpathlineto{\pgfqpoint{2.263574in}{3.062189in}}%
\pgfpathclose%
\pgfusepath{fill}%
\end{pgfscope}%
\begin{pgfscope}%
\pgfpathrectangle{\pgfqpoint{0.765000in}{0.660000in}}{\pgfqpoint{4.620000in}{4.620000in}}%
\pgfusepath{clip}%
\pgfsetbuttcap%
\pgfsetroundjoin%
\definecolor{currentfill}{rgb}{1.000000,0.894118,0.788235}%
\pgfsetfillcolor{currentfill}%
\pgfsetlinewidth{0.000000pt}%
\definecolor{currentstroke}{rgb}{1.000000,0.894118,0.788235}%
\pgfsetstrokecolor{currentstroke}%
\pgfsetdash{}{0pt}%
\pgfpathmoveto{\pgfqpoint{2.290521in}{3.046631in}}%
\pgfpathlineto{\pgfqpoint{2.290521in}{3.077747in}}%
\pgfpathlineto{\pgfqpoint{-7.092520in}{-2.439475in}}%
\pgfpathlineto{\pgfqpoint{-7.119467in}{-2.455033in}}%
\pgfpathlineto{\pgfqpoint{2.290521in}{3.046631in}}%
\pgfpathclose%
\pgfusepath{fill}%
\end{pgfscope}%
\begin{pgfscope}%
\pgfpathrectangle{\pgfqpoint{0.765000in}{0.660000in}}{\pgfqpoint{4.620000in}{4.620000in}}%
\pgfusepath{clip}%
\pgfsetbuttcap%
\pgfsetroundjoin%
\definecolor{currentfill}{rgb}{1.000000,0.894118,0.788235}%
\pgfsetfillcolor{currentfill}%
\pgfsetlinewidth{0.000000pt}%
\definecolor{currentstroke}{rgb}{1.000000,0.894118,0.788235}%
\pgfsetstrokecolor{currentstroke}%
\pgfsetdash{}{0pt}%
\pgfpathmoveto{\pgfqpoint{2.236626in}{3.046631in}}%
\pgfpathlineto{\pgfqpoint{2.263574in}{3.031073in}}%
\pgfpathlineto{\pgfqpoint{-7.119467in}{-2.486149in}}%
\pgfpathlineto{\pgfqpoint{-7.146414in}{-2.470591in}}%
\pgfpathlineto{\pgfqpoint{2.236626in}{3.046631in}}%
\pgfpathclose%
\pgfusepath{fill}%
\end{pgfscope}%
\begin{pgfscope}%
\pgfpathrectangle{\pgfqpoint{0.765000in}{0.660000in}}{\pgfqpoint{4.620000in}{4.620000in}}%
\pgfusepath{clip}%
\pgfsetbuttcap%
\pgfsetroundjoin%
\definecolor{currentfill}{rgb}{1.000000,0.894118,0.788235}%
\pgfsetfillcolor{currentfill}%
\pgfsetlinewidth{0.000000pt}%
\definecolor{currentstroke}{rgb}{1.000000,0.894118,0.788235}%
\pgfsetstrokecolor{currentstroke}%
\pgfsetdash{}{0pt}%
\pgfpathmoveto{\pgfqpoint{2.263574in}{3.031073in}}%
\pgfpathlineto{\pgfqpoint{2.290521in}{3.046631in}}%
\pgfpathlineto{\pgfqpoint{-7.092520in}{-2.470591in}}%
\pgfpathlineto{\pgfqpoint{-7.119467in}{-2.486149in}}%
\pgfpathlineto{\pgfqpoint{2.263574in}{3.031073in}}%
\pgfpathclose%
\pgfusepath{fill}%
\end{pgfscope}%
\begin{pgfscope}%
\pgfpathrectangle{\pgfqpoint{0.765000in}{0.660000in}}{\pgfqpoint{4.620000in}{4.620000in}}%
\pgfusepath{clip}%
\pgfsetbuttcap%
\pgfsetroundjoin%
\definecolor{currentfill}{rgb}{1.000000,0.894118,0.788235}%
\pgfsetfillcolor{currentfill}%
\pgfsetlinewidth{0.000000pt}%
\definecolor{currentstroke}{rgb}{1.000000,0.894118,0.788235}%
\pgfsetstrokecolor{currentstroke}%
\pgfsetdash{}{0pt}%
\pgfpathmoveto{\pgfqpoint{2.263574in}{3.093305in}}%
\pgfpathlineto{\pgfqpoint{2.236626in}{3.077747in}}%
\pgfpathlineto{\pgfqpoint{-7.146414in}{-2.439475in}}%
\pgfpathlineto{\pgfqpoint{-7.119467in}{-2.423917in}}%
\pgfpathlineto{\pgfqpoint{2.263574in}{3.093305in}}%
\pgfpathclose%
\pgfusepath{fill}%
\end{pgfscope}%
\begin{pgfscope}%
\pgfpathrectangle{\pgfqpoint{0.765000in}{0.660000in}}{\pgfqpoint{4.620000in}{4.620000in}}%
\pgfusepath{clip}%
\pgfsetbuttcap%
\pgfsetroundjoin%
\definecolor{currentfill}{rgb}{1.000000,0.894118,0.788235}%
\pgfsetfillcolor{currentfill}%
\pgfsetlinewidth{0.000000pt}%
\definecolor{currentstroke}{rgb}{1.000000,0.894118,0.788235}%
\pgfsetstrokecolor{currentstroke}%
\pgfsetdash{}{0pt}%
\pgfpathmoveto{\pgfqpoint{2.290521in}{3.077747in}}%
\pgfpathlineto{\pgfqpoint{2.263574in}{3.093305in}}%
\pgfpathlineto{\pgfqpoint{-7.119467in}{-2.423917in}}%
\pgfpathlineto{\pgfqpoint{-7.092520in}{-2.439475in}}%
\pgfpathlineto{\pgfqpoint{2.290521in}{3.077747in}}%
\pgfpathclose%
\pgfusepath{fill}%
\end{pgfscope}%
\begin{pgfscope}%
\pgfpathrectangle{\pgfqpoint{0.765000in}{0.660000in}}{\pgfqpoint{4.620000in}{4.620000in}}%
\pgfusepath{clip}%
\pgfsetbuttcap%
\pgfsetroundjoin%
\definecolor{currentfill}{rgb}{1.000000,0.894118,0.788235}%
\pgfsetfillcolor{currentfill}%
\pgfsetlinewidth{0.000000pt}%
\definecolor{currentstroke}{rgb}{1.000000,0.894118,0.788235}%
\pgfsetstrokecolor{currentstroke}%
\pgfsetdash{}{0pt}%
\pgfpathmoveto{\pgfqpoint{2.236626in}{3.046631in}}%
\pgfpathlineto{\pgfqpoint{2.236626in}{3.077747in}}%
\pgfpathlineto{\pgfqpoint{-7.146414in}{-2.439475in}}%
\pgfpathlineto{\pgfqpoint{-7.119467in}{-2.486149in}}%
\pgfpathlineto{\pgfqpoint{2.236626in}{3.046631in}}%
\pgfpathclose%
\pgfusepath{fill}%
\end{pgfscope}%
\begin{pgfscope}%
\pgfpathrectangle{\pgfqpoint{0.765000in}{0.660000in}}{\pgfqpoint{4.620000in}{4.620000in}}%
\pgfusepath{clip}%
\pgfsetbuttcap%
\pgfsetroundjoin%
\definecolor{currentfill}{rgb}{1.000000,0.894118,0.788235}%
\pgfsetfillcolor{currentfill}%
\pgfsetlinewidth{0.000000pt}%
\definecolor{currentstroke}{rgb}{1.000000,0.894118,0.788235}%
\pgfsetstrokecolor{currentstroke}%
\pgfsetdash{}{0pt}%
\pgfpathmoveto{\pgfqpoint{2.263574in}{3.031073in}}%
\pgfpathlineto{\pgfqpoint{2.263574in}{3.062189in}}%
\pgfpathlineto{\pgfqpoint{-7.119467in}{-2.455033in}}%
\pgfpathlineto{\pgfqpoint{-7.146414in}{-2.470591in}}%
\pgfpathlineto{\pgfqpoint{2.263574in}{3.031073in}}%
\pgfpathclose%
\pgfusepath{fill}%
\end{pgfscope}%
\begin{pgfscope}%
\pgfpathrectangle{\pgfqpoint{0.765000in}{0.660000in}}{\pgfqpoint{4.620000in}{4.620000in}}%
\pgfusepath{clip}%
\pgfsetbuttcap%
\pgfsetroundjoin%
\definecolor{currentfill}{rgb}{1.000000,0.894118,0.788235}%
\pgfsetfillcolor{currentfill}%
\pgfsetlinewidth{0.000000pt}%
\definecolor{currentstroke}{rgb}{1.000000,0.894118,0.788235}%
\pgfsetstrokecolor{currentstroke}%
\pgfsetdash{}{0pt}%
\pgfpathmoveto{\pgfqpoint{2.236626in}{3.077747in}}%
\pgfpathlineto{\pgfqpoint{2.263574in}{3.062189in}}%
\pgfpathlineto{\pgfqpoint{-7.119467in}{-2.455033in}}%
\pgfpathlineto{\pgfqpoint{-7.146414in}{-2.439475in}}%
\pgfpathlineto{\pgfqpoint{2.236626in}{3.077747in}}%
\pgfpathclose%
\pgfusepath{fill}%
\end{pgfscope}%
\begin{pgfscope}%
\pgfpathrectangle{\pgfqpoint{0.765000in}{0.660000in}}{\pgfqpoint{4.620000in}{4.620000in}}%
\pgfusepath{clip}%
\pgfsetbuttcap%
\pgfsetroundjoin%
\definecolor{currentfill}{rgb}{1.000000,0.894118,0.788235}%
\pgfsetfillcolor{currentfill}%
\pgfsetlinewidth{0.000000pt}%
\definecolor{currentstroke}{rgb}{1.000000,0.894118,0.788235}%
\pgfsetstrokecolor{currentstroke}%
\pgfsetdash{}{0pt}%
\pgfpathmoveto{\pgfqpoint{2.263574in}{3.062189in}}%
\pgfpathlineto{\pgfqpoint{2.290521in}{3.077747in}}%
\pgfpathlineto{\pgfqpoint{-7.092520in}{-2.439475in}}%
\pgfpathlineto{\pgfqpoint{-7.119467in}{-2.455033in}}%
\pgfpathlineto{\pgfqpoint{2.263574in}{3.062189in}}%
\pgfpathclose%
\pgfusepath{fill}%
\end{pgfscope}%
\begin{pgfscope}%
\pgfpathrectangle{\pgfqpoint{0.765000in}{0.660000in}}{\pgfqpoint{4.620000in}{4.620000in}}%
\pgfusepath{clip}%
\pgfsetbuttcap%
\pgfsetroundjoin%
\definecolor{currentfill}{rgb}{1.000000,0.894118,0.788235}%
\pgfsetfillcolor{currentfill}%
\pgfsetlinewidth{0.000000pt}%
\definecolor{currentstroke}{rgb}{1.000000,0.894118,0.788235}%
\pgfsetstrokecolor{currentstroke}%
\pgfsetdash{}{0pt}%
\pgfpathmoveto{\pgfqpoint{2.981592in}{3.650866in}}%
\pgfpathlineto{\pgfqpoint{2.954645in}{3.635308in}}%
\pgfpathlineto{\pgfqpoint{2.981592in}{3.619750in}}%
\pgfpathlineto{\pgfqpoint{3.008539in}{3.635308in}}%
\pgfpathlineto{\pgfqpoint{2.981592in}{3.650866in}}%
\pgfpathclose%
\pgfusepath{fill}%
\end{pgfscope}%
\begin{pgfscope}%
\pgfpathrectangle{\pgfqpoint{0.765000in}{0.660000in}}{\pgfqpoint{4.620000in}{4.620000in}}%
\pgfusepath{clip}%
\pgfsetbuttcap%
\pgfsetroundjoin%
\definecolor{currentfill}{rgb}{1.000000,0.894118,0.788235}%
\pgfsetfillcolor{currentfill}%
\pgfsetlinewidth{0.000000pt}%
\definecolor{currentstroke}{rgb}{1.000000,0.894118,0.788235}%
\pgfsetstrokecolor{currentstroke}%
\pgfsetdash{}{0pt}%
\pgfpathmoveto{\pgfqpoint{2.981592in}{3.650866in}}%
\pgfpathlineto{\pgfqpoint{2.954645in}{3.635308in}}%
\pgfpathlineto{\pgfqpoint{2.954645in}{3.666424in}}%
\pgfpathlineto{\pgfqpoint{2.981592in}{3.681982in}}%
\pgfpathlineto{\pgfqpoint{2.981592in}{3.650866in}}%
\pgfpathclose%
\pgfusepath{fill}%
\end{pgfscope}%
\begin{pgfscope}%
\pgfpathrectangle{\pgfqpoint{0.765000in}{0.660000in}}{\pgfqpoint{4.620000in}{4.620000in}}%
\pgfusepath{clip}%
\pgfsetbuttcap%
\pgfsetroundjoin%
\definecolor{currentfill}{rgb}{1.000000,0.894118,0.788235}%
\pgfsetfillcolor{currentfill}%
\pgfsetlinewidth{0.000000pt}%
\definecolor{currentstroke}{rgb}{1.000000,0.894118,0.788235}%
\pgfsetstrokecolor{currentstroke}%
\pgfsetdash{}{0pt}%
\pgfpathmoveto{\pgfqpoint{2.981592in}{3.650866in}}%
\pgfpathlineto{\pgfqpoint{3.008539in}{3.635308in}}%
\pgfpathlineto{\pgfqpoint{3.008539in}{3.666424in}}%
\pgfpathlineto{\pgfqpoint{2.981592in}{3.681982in}}%
\pgfpathlineto{\pgfqpoint{2.981592in}{3.650866in}}%
\pgfpathclose%
\pgfusepath{fill}%
\end{pgfscope}%
\begin{pgfscope}%
\pgfpathrectangle{\pgfqpoint{0.765000in}{0.660000in}}{\pgfqpoint{4.620000in}{4.620000in}}%
\pgfusepath{clip}%
\pgfsetbuttcap%
\pgfsetroundjoin%
\definecolor{currentfill}{rgb}{1.000000,0.894118,0.788235}%
\pgfsetfillcolor{currentfill}%
\pgfsetlinewidth{0.000000pt}%
\definecolor{currentstroke}{rgb}{1.000000,0.894118,0.788235}%
\pgfsetstrokecolor{currentstroke}%
\pgfsetdash{}{0pt}%
\pgfpathmoveto{\pgfqpoint{3.006193in}{3.769934in}}%
\pgfpathlineto{\pgfqpoint{2.979245in}{3.754376in}}%
\pgfpathlineto{\pgfqpoint{3.006193in}{3.738818in}}%
\pgfpathlineto{\pgfqpoint{3.033140in}{3.754376in}}%
\pgfpathlineto{\pgfqpoint{3.006193in}{3.769934in}}%
\pgfpathclose%
\pgfusepath{fill}%
\end{pgfscope}%
\begin{pgfscope}%
\pgfpathrectangle{\pgfqpoint{0.765000in}{0.660000in}}{\pgfqpoint{4.620000in}{4.620000in}}%
\pgfusepath{clip}%
\pgfsetbuttcap%
\pgfsetroundjoin%
\definecolor{currentfill}{rgb}{1.000000,0.894118,0.788235}%
\pgfsetfillcolor{currentfill}%
\pgfsetlinewidth{0.000000pt}%
\definecolor{currentstroke}{rgb}{1.000000,0.894118,0.788235}%
\pgfsetstrokecolor{currentstroke}%
\pgfsetdash{}{0pt}%
\pgfpathmoveto{\pgfqpoint{3.006193in}{3.769934in}}%
\pgfpathlineto{\pgfqpoint{2.979245in}{3.754376in}}%
\pgfpathlineto{\pgfqpoint{2.979245in}{3.785492in}}%
\pgfpathlineto{\pgfqpoint{3.006193in}{3.801050in}}%
\pgfpathlineto{\pgfqpoint{3.006193in}{3.769934in}}%
\pgfpathclose%
\pgfusepath{fill}%
\end{pgfscope}%
\begin{pgfscope}%
\pgfpathrectangle{\pgfqpoint{0.765000in}{0.660000in}}{\pgfqpoint{4.620000in}{4.620000in}}%
\pgfusepath{clip}%
\pgfsetbuttcap%
\pgfsetroundjoin%
\definecolor{currentfill}{rgb}{1.000000,0.894118,0.788235}%
\pgfsetfillcolor{currentfill}%
\pgfsetlinewidth{0.000000pt}%
\definecolor{currentstroke}{rgb}{1.000000,0.894118,0.788235}%
\pgfsetstrokecolor{currentstroke}%
\pgfsetdash{}{0pt}%
\pgfpathmoveto{\pgfqpoint{3.006193in}{3.769934in}}%
\pgfpathlineto{\pgfqpoint{3.033140in}{3.754376in}}%
\pgfpathlineto{\pgfqpoint{3.033140in}{3.785492in}}%
\pgfpathlineto{\pgfqpoint{3.006193in}{3.801050in}}%
\pgfpathlineto{\pgfqpoint{3.006193in}{3.769934in}}%
\pgfpathclose%
\pgfusepath{fill}%
\end{pgfscope}%
\begin{pgfscope}%
\pgfpathrectangle{\pgfqpoint{0.765000in}{0.660000in}}{\pgfqpoint{4.620000in}{4.620000in}}%
\pgfusepath{clip}%
\pgfsetbuttcap%
\pgfsetroundjoin%
\definecolor{currentfill}{rgb}{1.000000,0.894118,0.788235}%
\pgfsetfillcolor{currentfill}%
\pgfsetlinewidth{0.000000pt}%
\definecolor{currentstroke}{rgb}{1.000000,0.894118,0.788235}%
\pgfsetstrokecolor{currentstroke}%
\pgfsetdash{}{0pt}%
\pgfpathmoveto{\pgfqpoint{2.981592in}{3.681982in}}%
\pgfpathlineto{\pgfqpoint{2.954645in}{3.666424in}}%
\pgfpathlineto{\pgfqpoint{2.981592in}{3.650866in}}%
\pgfpathlineto{\pgfqpoint{3.008539in}{3.666424in}}%
\pgfpathlineto{\pgfqpoint{2.981592in}{3.681982in}}%
\pgfpathclose%
\pgfusepath{fill}%
\end{pgfscope}%
\begin{pgfscope}%
\pgfpathrectangle{\pgfqpoint{0.765000in}{0.660000in}}{\pgfqpoint{4.620000in}{4.620000in}}%
\pgfusepath{clip}%
\pgfsetbuttcap%
\pgfsetroundjoin%
\definecolor{currentfill}{rgb}{1.000000,0.894118,0.788235}%
\pgfsetfillcolor{currentfill}%
\pgfsetlinewidth{0.000000pt}%
\definecolor{currentstroke}{rgb}{1.000000,0.894118,0.788235}%
\pgfsetstrokecolor{currentstroke}%
\pgfsetdash{}{0pt}%
\pgfpathmoveto{\pgfqpoint{2.981592in}{3.619750in}}%
\pgfpathlineto{\pgfqpoint{3.008539in}{3.635308in}}%
\pgfpathlineto{\pgfqpoint{3.008539in}{3.666424in}}%
\pgfpathlineto{\pgfqpoint{2.981592in}{3.650866in}}%
\pgfpathlineto{\pgfqpoint{2.981592in}{3.619750in}}%
\pgfpathclose%
\pgfusepath{fill}%
\end{pgfscope}%
\begin{pgfscope}%
\pgfpathrectangle{\pgfqpoint{0.765000in}{0.660000in}}{\pgfqpoint{4.620000in}{4.620000in}}%
\pgfusepath{clip}%
\pgfsetbuttcap%
\pgfsetroundjoin%
\definecolor{currentfill}{rgb}{1.000000,0.894118,0.788235}%
\pgfsetfillcolor{currentfill}%
\pgfsetlinewidth{0.000000pt}%
\definecolor{currentstroke}{rgb}{1.000000,0.894118,0.788235}%
\pgfsetstrokecolor{currentstroke}%
\pgfsetdash{}{0pt}%
\pgfpathmoveto{\pgfqpoint{2.954645in}{3.635308in}}%
\pgfpathlineto{\pgfqpoint{2.981592in}{3.619750in}}%
\pgfpathlineto{\pgfqpoint{2.981592in}{3.650866in}}%
\pgfpathlineto{\pgfqpoint{2.954645in}{3.666424in}}%
\pgfpathlineto{\pgfqpoint{2.954645in}{3.635308in}}%
\pgfpathclose%
\pgfusepath{fill}%
\end{pgfscope}%
\begin{pgfscope}%
\pgfpathrectangle{\pgfqpoint{0.765000in}{0.660000in}}{\pgfqpoint{4.620000in}{4.620000in}}%
\pgfusepath{clip}%
\pgfsetbuttcap%
\pgfsetroundjoin%
\definecolor{currentfill}{rgb}{1.000000,0.894118,0.788235}%
\pgfsetfillcolor{currentfill}%
\pgfsetlinewidth{0.000000pt}%
\definecolor{currentstroke}{rgb}{1.000000,0.894118,0.788235}%
\pgfsetstrokecolor{currentstroke}%
\pgfsetdash{}{0pt}%
\pgfpathmoveto{\pgfqpoint{3.006193in}{3.801050in}}%
\pgfpathlineto{\pgfqpoint{2.979245in}{3.785492in}}%
\pgfpathlineto{\pgfqpoint{3.006193in}{3.769934in}}%
\pgfpathlineto{\pgfqpoint{3.033140in}{3.785492in}}%
\pgfpathlineto{\pgfqpoint{3.006193in}{3.801050in}}%
\pgfpathclose%
\pgfusepath{fill}%
\end{pgfscope}%
\begin{pgfscope}%
\pgfpathrectangle{\pgfqpoint{0.765000in}{0.660000in}}{\pgfqpoint{4.620000in}{4.620000in}}%
\pgfusepath{clip}%
\pgfsetbuttcap%
\pgfsetroundjoin%
\definecolor{currentfill}{rgb}{1.000000,0.894118,0.788235}%
\pgfsetfillcolor{currentfill}%
\pgfsetlinewidth{0.000000pt}%
\definecolor{currentstroke}{rgb}{1.000000,0.894118,0.788235}%
\pgfsetstrokecolor{currentstroke}%
\pgfsetdash{}{0pt}%
\pgfpathmoveto{\pgfqpoint{3.006193in}{3.738818in}}%
\pgfpathlineto{\pgfqpoint{3.033140in}{3.754376in}}%
\pgfpathlineto{\pgfqpoint{3.033140in}{3.785492in}}%
\pgfpathlineto{\pgfqpoint{3.006193in}{3.769934in}}%
\pgfpathlineto{\pgfqpoint{3.006193in}{3.738818in}}%
\pgfpathclose%
\pgfusepath{fill}%
\end{pgfscope}%
\begin{pgfscope}%
\pgfpathrectangle{\pgfqpoint{0.765000in}{0.660000in}}{\pgfqpoint{4.620000in}{4.620000in}}%
\pgfusepath{clip}%
\pgfsetbuttcap%
\pgfsetroundjoin%
\definecolor{currentfill}{rgb}{1.000000,0.894118,0.788235}%
\pgfsetfillcolor{currentfill}%
\pgfsetlinewidth{0.000000pt}%
\definecolor{currentstroke}{rgb}{1.000000,0.894118,0.788235}%
\pgfsetstrokecolor{currentstroke}%
\pgfsetdash{}{0pt}%
\pgfpathmoveto{\pgfqpoint{2.979245in}{3.754376in}}%
\pgfpathlineto{\pgfqpoint{3.006193in}{3.738818in}}%
\pgfpathlineto{\pgfqpoint{3.006193in}{3.769934in}}%
\pgfpathlineto{\pgfqpoint{2.979245in}{3.785492in}}%
\pgfpathlineto{\pgfqpoint{2.979245in}{3.754376in}}%
\pgfpathclose%
\pgfusepath{fill}%
\end{pgfscope}%
\begin{pgfscope}%
\pgfpathrectangle{\pgfqpoint{0.765000in}{0.660000in}}{\pgfqpoint{4.620000in}{4.620000in}}%
\pgfusepath{clip}%
\pgfsetbuttcap%
\pgfsetroundjoin%
\definecolor{currentfill}{rgb}{1.000000,0.894118,0.788235}%
\pgfsetfillcolor{currentfill}%
\pgfsetlinewidth{0.000000pt}%
\definecolor{currentstroke}{rgb}{1.000000,0.894118,0.788235}%
\pgfsetstrokecolor{currentstroke}%
\pgfsetdash{}{0pt}%
\pgfpathmoveto{\pgfqpoint{2.981592in}{3.650866in}}%
\pgfpathlineto{\pgfqpoint{2.954645in}{3.635308in}}%
\pgfpathlineto{\pgfqpoint{2.979245in}{3.754376in}}%
\pgfpathlineto{\pgfqpoint{3.006193in}{3.769934in}}%
\pgfpathlineto{\pgfqpoint{2.981592in}{3.650866in}}%
\pgfpathclose%
\pgfusepath{fill}%
\end{pgfscope}%
\begin{pgfscope}%
\pgfpathrectangle{\pgfqpoint{0.765000in}{0.660000in}}{\pgfqpoint{4.620000in}{4.620000in}}%
\pgfusepath{clip}%
\pgfsetbuttcap%
\pgfsetroundjoin%
\definecolor{currentfill}{rgb}{1.000000,0.894118,0.788235}%
\pgfsetfillcolor{currentfill}%
\pgfsetlinewidth{0.000000pt}%
\definecolor{currentstroke}{rgb}{1.000000,0.894118,0.788235}%
\pgfsetstrokecolor{currentstroke}%
\pgfsetdash{}{0pt}%
\pgfpathmoveto{\pgfqpoint{3.008539in}{3.635308in}}%
\pgfpathlineto{\pgfqpoint{2.981592in}{3.650866in}}%
\pgfpathlineto{\pgfqpoint{3.006193in}{3.769934in}}%
\pgfpathlineto{\pgfqpoint{3.033140in}{3.754376in}}%
\pgfpathlineto{\pgfqpoint{3.008539in}{3.635308in}}%
\pgfpathclose%
\pgfusepath{fill}%
\end{pgfscope}%
\begin{pgfscope}%
\pgfpathrectangle{\pgfqpoint{0.765000in}{0.660000in}}{\pgfqpoint{4.620000in}{4.620000in}}%
\pgfusepath{clip}%
\pgfsetbuttcap%
\pgfsetroundjoin%
\definecolor{currentfill}{rgb}{1.000000,0.894118,0.788235}%
\pgfsetfillcolor{currentfill}%
\pgfsetlinewidth{0.000000pt}%
\definecolor{currentstroke}{rgb}{1.000000,0.894118,0.788235}%
\pgfsetstrokecolor{currentstroke}%
\pgfsetdash{}{0pt}%
\pgfpathmoveto{\pgfqpoint{2.981592in}{3.650866in}}%
\pgfpathlineto{\pgfqpoint{2.981592in}{3.681982in}}%
\pgfpathlineto{\pgfqpoint{3.006193in}{3.801050in}}%
\pgfpathlineto{\pgfqpoint{3.033140in}{3.754376in}}%
\pgfpathlineto{\pgfqpoint{2.981592in}{3.650866in}}%
\pgfpathclose%
\pgfusepath{fill}%
\end{pgfscope}%
\begin{pgfscope}%
\pgfpathrectangle{\pgfqpoint{0.765000in}{0.660000in}}{\pgfqpoint{4.620000in}{4.620000in}}%
\pgfusepath{clip}%
\pgfsetbuttcap%
\pgfsetroundjoin%
\definecolor{currentfill}{rgb}{1.000000,0.894118,0.788235}%
\pgfsetfillcolor{currentfill}%
\pgfsetlinewidth{0.000000pt}%
\definecolor{currentstroke}{rgb}{1.000000,0.894118,0.788235}%
\pgfsetstrokecolor{currentstroke}%
\pgfsetdash{}{0pt}%
\pgfpathmoveto{\pgfqpoint{3.008539in}{3.635308in}}%
\pgfpathlineto{\pgfqpoint{3.008539in}{3.666424in}}%
\pgfpathlineto{\pgfqpoint{3.033140in}{3.785492in}}%
\pgfpathlineto{\pgfqpoint{3.006193in}{3.769934in}}%
\pgfpathlineto{\pgfqpoint{3.008539in}{3.635308in}}%
\pgfpathclose%
\pgfusepath{fill}%
\end{pgfscope}%
\begin{pgfscope}%
\pgfpathrectangle{\pgfqpoint{0.765000in}{0.660000in}}{\pgfqpoint{4.620000in}{4.620000in}}%
\pgfusepath{clip}%
\pgfsetbuttcap%
\pgfsetroundjoin%
\definecolor{currentfill}{rgb}{1.000000,0.894118,0.788235}%
\pgfsetfillcolor{currentfill}%
\pgfsetlinewidth{0.000000pt}%
\definecolor{currentstroke}{rgb}{1.000000,0.894118,0.788235}%
\pgfsetstrokecolor{currentstroke}%
\pgfsetdash{}{0pt}%
\pgfpathmoveto{\pgfqpoint{2.954645in}{3.635308in}}%
\pgfpathlineto{\pgfqpoint{2.981592in}{3.619750in}}%
\pgfpathlineto{\pgfqpoint{3.006193in}{3.738818in}}%
\pgfpathlineto{\pgfqpoint{2.979245in}{3.754376in}}%
\pgfpathlineto{\pgfqpoint{2.954645in}{3.635308in}}%
\pgfpathclose%
\pgfusepath{fill}%
\end{pgfscope}%
\begin{pgfscope}%
\pgfpathrectangle{\pgfqpoint{0.765000in}{0.660000in}}{\pgfqpoint{4.620000in}{4.620000in}}%
\pgfusepath{clip}%
\pgfsetbuttcap%
\pgfsetroundjoin%
\definecolor{currentfill}{rgb}{1.000000,0.894118,0.788235}%
\pgfsetfillcolor{currentfill}%
\pgfsetlinewidth{0.000000pt}%
\definecolor{currentstroke}{rgb}{1.000000,0.894118,0.788235}%
\pgfsetstrokecolor{currentstroke}%
\pgfsetdash{}{0pt}%
\pgfpathmoveto{\pgfqpoint{2.981592in}{3.619750in}}%
\pgfpathlineto{\pgfqpoint{3.008539in}{3.635308in}}%
\pgfpathlineto{\pgfqpoint{3.033140in}{3.754376in}}%
\pgfpathlineto{\pgfqpoint{3.006193in}{3.738818in}}%
\pgfpathlineto{\pgfqpoint{2.981592in}{3.619750in}}%
\pgfpathclose%
\pgfusepath{fill}%
\end{pgfscope}%
\begin{pgfscope}%
\pgfpathrectangle{\pgfqpoint{0.765000in}{0.660000in}}{\pgfqpoint{4.620000in}{4.620000in}}%
\pgfusepath{clip}%
\pgfsetbuttcap%
\pgfsetroundjoin%
\definecolor{currentfill}{rgb}{1.000000,0.894118,0.788235}%
\pgfsetfillcolor{currentfill}%
\pgfsetlinewidth{0.000000pt}%
\definecolor{currentstroke}{rgb}{1.000000,0.894118,0.788235}%
\pgfsetstrokecolor{currentstroke}%
\pgfsetdash{}{0pt}%
\pgfpathmoveto{\pgfqpoint{2.981592in}{3.681982in}}%
\pgfpathlineto{\pgfqpoint{2.954645in}{3.666424in}}%
\pgfpathlineto{\pgfqpoint{2.979245in}{3.785492in}}%
\pgfpathlineto{\pgfqpoint{3.006193in}{3.801050in}}%
\pgfpathlineto{\pgfqpoint{2.981592in}{3.681982in}}%
\pgfpathclose%
\pgfusepath{fill}%
\end{pgfscope}%
\begin{pgfscope}%
\pgfpathrectangle{\pgfqpoint{0.765000in}{0.660000in}}{\pgfqpoint{4.620000in}{4.620000in}}%
\pgfusepath{clip}%
\pgfsetbuttcap%
\pgfsetroundjoin%
\definecolor{currentfill}{rgb}{1.000000,0.894118,0.788235}%
\pgfsetfillcolor{currentfill}%
\pgfsetlinewidth{0.000000pt}%
\definecolor{currentstroke}{rgb}{1.000000,0.894118,0.788235}%
\pgfsetstrokecolor{currentstroke}%
\pgfsetdash{}{0pt}%
\pgfpathmoveto{\pgfqpoint{3.008539in}{3.666424in}}%
\pgfpathlineto{\pgfqpoint{2.981592in}{3.681982in}}%
\pgfpathlineto{\pgfqpoint{3.006193in}{3.801050in}}%
\pgfpathlineto{\pgfqpoint{3.033140in}{3.785492in}}%
\pgfpathlineto{\pgfqpoint{3.008539in}{3.666424in}}%
\pgfpathclose%
\pgfusepath{fill}%
\end{pgfscope}%
\begin{pgfscope}%
\pgfpathrectangle{\pgfqpoint{0.765000in}{0.660000in}}{\pgfqpoint{4.620000in}{4.620000in}}%
\pgfusepath{clip}%
\pgfsetbuttcap%
\pgfsetroundjoin%
\definecolor{currentfill}{rgb}{1.000000,0.894118,0.788235}%
\pgfsetfillcolor{currentfill}%
\pgfsetlinewidth{0.000000pt}%
\definecolor{currentstroke}{rgb}{1.000000,0.894118,0.788235}%
\pgfsetstrokecolor{currentstroke}%
\pgfsetdash{}{0pt}%
\pgfpathmoveto{\pgfqpoint{2.954645in}{3.635308in}}%
\pgfpathlineto{\pgfqpoint{2.954645in}{3.666424in}}%
\pgfpathlineto{\pgfqpoint{2.979245in}{3.785492in}}%
\pgfpathlineto{\pgfqpoint{3.006193in}{3.738818in}}%
\pgfpathlineto{\pgfqpoint{2.954645in}{3.635308in}}%
\pgfpathclose%
\pgfusepath{fill}%
\end{pgfscope}%
\begin{pgfscope}%
\pgfpathrectangle{\pgfqpoint{0.765000in}{0.660000in}}{\pgfqpoint{4.620000in}{4.620000in}}%
\pgfusepath{clip}%
\pgfsetbuttcap%
\pgfsetroundjoin%
\definecolor{currentfill}{rgb}{1.000000,0.894118,0.788235}%
\pgfsetfillcolor{currentfill}%
\pgfsetlinewidth{0.000000pt}%
\definecolor{currentstroke}{rgb}{1.000000,0.894118,0.788235}%
\pgfsetstrokecolor{currentstroke}%
\pgfsetdash{}{0pt}%
\pgfpathmoveto{\pgfqpoint{2.981592in}{3.619750in}}%
\pgfpathlineto{\pgfqpoint{2.981592in}{3.650866in}}%
\pgfpathlineto{\pgfqpoint{3.006193in}{3.769934in}}%
\pgfpathlineto{\pgfqpoint{2.979245in}{3.754376in}}%
\pgfpathlineto{\pgfqpoint{2.981592in}{3.619750in}}%
\pgfpathclose%
\pgfusepath{fill}%
\end{pgfscope}%
\begin{pgfscope}%
\pgfpathrectangle{\pgfqpoint{0.765000in}{0.660000in}}{\pgfqpoint{4.620000in}{4.620000in}}%
\pgfusepath{clip}%
\pgfsetbuttcap%
\pgfsetroundjoin%
\definecolor{currentfill}{rgb}{1.000000,0.894118,0.788235}%
\pgfsetfillcolor{currentfill}%
\pgfsetlinewidth{0.000000pt}%
\definecolor{currentstroke}{rgb}{1.000000,0.894118,0.788235}%
\pgfsetstrokecolor{currentstroke}%
\pgfsetdash{}{0pt}%
\pgfpathmoveto{\pgfqpoint{2.981592in}{3.650866in}}%
\pgfpathlineto{\pgfqpoint{3.008539in}{3.666424in}}%
\pgfpathlineto{\pgfqpoint{3.033140in}{3.785492in}}%
\pgfpathlineto{\pgfqpoint{3.006193in}{3.769934in}}%
\pgfpathlineto{\pgfqpoint{2.981592in}{3.650866in}}%
\pgfpathclose%
\pgfusepath{fill}%
\end{pgfscope}%
\begin{pgfscope}%
\pgfpathrectangle{\pgfqpoint{0.765000in}{0.660000in}}{\pgfqpoint{4.620000in}{4.620000in}}%
\pgfusepath{clip}%
\pgfsetbuttcap%
\pgfsetroundjoin%
\definecolor{currentfill}{rgb}{1.000000,0.894118,0.788235}%
\pgfsetfillcolor{currentfill}%
\pgfsetlinewidth{0.000000pt}%
\definecolor{currentstroke}{rgb}{1.000000,0.894118,0.788235}%
\pgfsetstrokecolor{currentstroke}%
\pgfsetdash{}{0pt}%
\pgfpathmoveto{\pgfqpoint{2.954645in}{3.666424in}}%
\pgfpathlineto{\pgfqpoint{2.981592in}{3.650866in}}%
\pgfpathlineto{\pgfqpoint{3.006193in}{3.769934in}}%
\pgfpathlineto{\pgfqpoint{2.979245in}{3.785492in}}%
\pgfpathlineto{\pgfqpoint{2.954645in}{3.666424in}}%
\pgfpathclose%
\pgfusepath{fill}%
\end{pgfscope}%
\begin{pgfscope}%
\pgfpathrectangle{\pgfqpoint{0.765000in}{0.660000in}}{\pgfqpoint{4.620000in}{4.620000in}}%
\pgfusepath{clip}%
\pgfsetbuttcap%
\pgfsetroundjoin%
\definecolor{currentfill}{rgb}{1.000000,0.894118,0.788235}%
\pgfsetfillcolor{currentfill}%
\pgfsetlinewidth{0.000000pt}%
\definecolor{currentstroke}{rgb}{1.000000,0.894118,0.788235}%
\pgfsetstrokecolor{currentstroke}%
\pgfsetdash{}{0pt}%
\pgfpathmoveto{\pgfqpoint{2.981592in}{3.650866in}}%
\pgfpathlineto{\pgfqpoint{2.954645in}{3.635308in}}%
\pgfpathlineto{\pgfqpoint{2.981592in}{3.619750in}}%
\pgfpathlineto{\pgfqpoint{3.008539in}{3.635308in}}%
\pgfpathlineto{\pgfqpoint{2.981592in}{3.650866in}}%
\pgfpathclose%
\pgfusepath{fill}%
\end{pgfscope}%
\begin{pgfscope}%
\pgfpathrectangle{\pgfqpoint{0.765000in}{0.660000in}}{\pgfqpoint{4.620000in}{4.620000in}}%
\pgfusepath{clip}%
\pgfsetbuttcap%
\pgfsetroundjoin%
\definecolor{currentfill}{rgb}{1.000000,0.894118,0.788235}%
\pgfsetfillcolor{currentfill}%
\pgfsetlinewidth{0.000000pt}%
\definecolor{currentstroke}{rgb}{1.000000,0.894118,0.788235}%
\pgfsetstrokecolor{currentstroke}%
\pgfsetdash{}{0pt}%
\pgfpathmoveto{\pgfqpoint{2.981592in}{3.650866in}}%
\pgfpathlineto{\pgfqpoint{2.954645in}{3.635308in}}%
\pgfpathlineto{\pgfqpoint{2.954645in}{3.666424in}}%
\pgfpathlineto{\pgfqpoint{2.981592in}{3.681982in}}%
\pgfpathlineto{\pgfqpoint{2.981592in}{3.650866in}}%
\pgfpathclose%
\pgfusepath{fill}%
\end{pgfscope}%
\begin{pgfscope}%
\pgfpathrectangle{\pgfqpoint{0.765000in}{0.660000in}}{\pgfqpoint{4.620000in}{4.620000in}}%
\pgfusepath{clip}%
\pgfsetbuttcap%
\pgfsetroundjoin%
\definecolor{currentfill}{rgb}{1.000000,0.894118,0.788235}%
\pgfsetfillcolor{currentfill}%
\pgfsetlinewidth{0.000000pt}%
\definecolor{currentstroke}{rgb}{1.000000,0.894118,0.788235}%
\pgfsetstrokecolor{currentstroke}%
\pgfsetdash{}{0pt}%
\pgfpathmoveto{\pgfqpoint{2.981592in}{3.650866in}}%
\pgfpathlineto{\pgfqpoint{3.008539in}{3.635308in}}%
\pgfpathlineto{\pgfqpoint{3.008539in}{3.666424in}}%
\pgfpathlineto{\pgfqpoint{2.981592in}{3.681982in}}%
\pgfpathlineto{\pgfqpoint{2.981592in}{3.650866in}}%
\pgfpathclose%
\pgfusepath{fill}%
\end{pgfscope}%
\begin{pgfscope}%
\pgfpathrectangle{\pgfqpoint{0.765000in}{0.660000in}}{\pgfqpoint{4.620000in}{4.620000in}}%
\pgfusepath{clip}%
\pgfsetbuttcap%
\pgfsetroundjoin%
\definecolor{currentfill}{rgb}{1.000000,0.894118,0.788235}%
\pgfsetfillcolor{currentfill}%
\pgfsetlinewidth{0.000000pt}%
\definecolor{currentstroke}{rgb}{1.000000,0.894118,0.788235}%
\pgfsetstrokecolor{currentstroke}%
\pgfsetdash{}{0pt}%
\pgfpathmoveto{\pgfqpoint{2.981238in}{3.646599in}}%
\pgfpathlineto{\pgfqpoint{2.954291in}{3.631041in}}%
\pgfpathlineto{\pgfqpoint{2.981238in}{3.615483in}}%
\pgfpathlineto{\pgfqpoint{3.008185in}{3.631041in}}%
\pgfpathlineto{\pgfqpoint{2.981238in}{3.646599in}}%
\pgfpathclose%
\pgfusepath{fill}%
\end{pgfscope}%
\begin{pgfscope}%
\pgfpathrectangle{\pgfqpoint{0.765000in}{0.660000in}}{\pgfqpoint{4.620000in}{4.620000in}}%
\pgfusepath{clip}%
\pgfsetbuttcap%
\pgfsetroundjoin%
\definecolor{currentfill}{rgb}{1.000000,0.894118,0.788235}%
\pgfsetfillcolor{currentfill}%
\pgfsetlinewidth{0.000000pt}%
\definecolor{currentstroke}{rgb}{1.000000,0.894118,0.788235}%
\pgfsetstrokecolor{currentstroke}%
\pgfsetdash{}{0pt}%
\pgfpathmoveto{\pgfqpoint{2.981238in}{3.646599in}}%
\pgfpathlineto{\pgfqpoint{2.954291in}{3.631041in}}%
\pgfpathlineto{\pgfqpoint{2.954291in}{3.662157in}}%
\pgfpathlineto{\pgfqpoint{2.981238in}{3.677715in}}%
\pgfpathlineto{\pgfqpoint{2.981238in}{3.646599in}}%
\pgfpathclose%
\pgfusepath{fill}%
\end{pgfscope}%
\begin{pgfscope}%
\pgfpathrectangle{\pgfqpoint{0.765000in}{0.660000in}}{\pgfqpoint{4.620000in}{4.620000in}}%
\pgfusepath{clip}%
\pgfsetbuttcap%
\pgfsetroundjoin%
\definecolor{currentfill}{rgb}{1.000000,0.894118,0.788235}%
\pgfsetfillcolor{currentfill}%
\pgfsetlinewidth{0.000000pt}%
\definecolor{currentstroke}{rgb}{1.000000,0.894118,0.788235}%
\pgfsetstrokecolor{currentstroke}%
\pgfsetdash{}{0pt}%
\pgfpathmoveto{\pgfqpoint{2.981238in}{3.646599in}}%
\pgfpathlineto{\pgfqpoint{3.008185in}{3.631041in}}%
\pgfpathlineto{\pgfqpoint{3.008185in}{3.662157in}}%
\pgfpathlineto{\pgfqpoint{2.981238in}{3.677715in}}%
\pgfpathlineto{\pgfqpoint{2.981238in}{3.646599in}}%
\pgfpathclose%
\pgfusepath{fill}%
\end{pgfscope}%
\begin{pgfscope}%
\pgfpathrectangle{\pgfqpoint{0.765000in}{0.660000in}}{\pgfqpoint{4.620000in}{4.620000in}}%
\pgfusepath{clip}%
\pgfsetbuttcap%
\pgfsetroundjoin%
\definecolor{currentfill}{rgb}{1.000000,0.894118,0.788235}%
\pgfsetfillcolor{currentfill}%
\pgfsetlinewidth{0.000000pt}%
\definecolor{currentstroke}{rgb}{1.000000,0.894118,0.788235}%
\pgfsetstrokecolor{currentstroke}%
\pgfsetdash{}{0pt}%
\pgfpathmoveto{\pgfqpoint{2.981592in}{3.681982in}}%
\pgfpathlineto{\pgfqpoint{2.954645in}{3.666424in}}%
\pgfpathlineto{\pgfqpoint{2.981592in}{3.650866in}}%
\pgfpathlineto{\pgfqpoint{3.008539in}{3.666424in}}%
\pgfpathlineto{\pgfqpoint{2.981592in}{3.681982in}}%
\pgfpathclose%
\pgfusepath{fill}%
\end{pgfscope}%
\begin{pgfscope}%
\pgfpathrectangle{\pgfqpoint{0.765000in}{0.660000in}}{\pgfqpoint{4.620000in}{4.620000in}}%
\pgfusepath{clip}%
\pgfsetbuttcap%
\pgfsetroundjoin%
\definecolor{currentfill}{rgb}{1.000000,0.894118,0.788235}%
\pgfsetfillcolor{currentfill}%
\pgfsetlinewidth{0.000000pt}%
\definecolor{currentstroke}{rgb}{1.000000,0.894118,0.788235}%
\pgfsetstrokecolor{currentstroke}%
\pgfsetdash{}{0pt}%
\pgfpathmoveto{\pgfqpoint{2.981592in}{3.619750in}}%
\pgfpathlineto{\pgfqpoint{3.008539in}{3.635308in}}%
\pgfpathlineto{\pgfqpoint{3.008539in}{3.666424in}}%
\pgfpathlineto{\pgfqpoint{2.981592in}{3.650866in}}%
\pgfpathlineto{\pgfqpoint{2.981592in}{3.619750in}}%
\pgfpathclose%
\pgfusepath{fill}%
\end{pgfscope}%
\begin{pgfscope}%
\pgfpathrectangle{\pgfqpoint{0.765000in}{0.660000in}}{\pgfqpoint{4.620000in}{4.620000in}}%
\pgfusepath{clip}%
\pgfsetbuttcap%
\pgfsetroundjoin%
\definecolor{currentfill}{rgb}{1.000000,0.894118,0.788235}%
\pgfsetfillcolor{currentfill}%
\pgfsetlinewidth{0.000000pt}%
\definecolor{currentstroke}{rgb}{1.000000,0.894118,0.788235}%
\pgfsetstrokecolor{currentstroke}%
\pgfsetdash{}{0pt}%
\pgfpathmoveto{\pgfqpoint{2.954645in}{3.635308in}}%
\pgfpathlineto{\pgfqpoint{2.981592in}{3.619750in}}%
\pgfpathlineto{\pgfqpoint{2.981592in}{3.650866in}}%
\pgfpathlineto{\pgfqpoint{2.954645in}{3.666424in}}%
\pgfpathlineto{\pgfqpoint{2.954645in}{3.635308in}}%
\pgfpathclose%
\pgfusepath{fill}%
\end{pgfscope}%
\begin{pgfscope}%
\pgfpathrectangle{\pgfqpoint{0.765000in}{0.660000in}}{\pgfqpoint{4.620000in}{4.620000in}}%
\pgfusepath{clip}%
\pgfsetbuttcap%
\pgfsetroundjoin%
\definecolor{currentfill}{rgb}{1.000000,0.894118,0.788235}%
\pgfsetfillcolor{currentfill}%
\pgfsetlinewidth{0.000000pt}%
\definecolor{currentstroke}{rgb}{1.000000,0.894118,0.788235}%
\pgfsetstrokecolor{currentstroke}%
\pgfsetdash{}{0pt}%
\pgfpathmoveto{\pgfqpoint{2.981238in}{3.677715in}}%
\pgfpathlineto{\pgfqpoint{2.954291in}{3.662157in}}%
\pgfpathlineto{\pgfqpoint{2.981238in}{3.646599in}}%
\pgfpathlineto{\pgfqpoint{3.008185in}{3.662157in}}%
\pgfpathlineto{\pgfqpoint{2.981238in}{3.677715in}}%
\pgfpathclose%
\pgfusepath{fill}%
\end{pgfscope}%
\begin{pgfscope}%
\pgfpathrectangle{\pgfqpoint{0.765000in}{0.660000in}}{\pgfqpoint{4.620000in}{4.620000in}}%
\pgfusepath{clip}%
\pgfsetbuttcap%
\pgfsetroundjoin%
\definecolor{currentfill}{rgb}{1.000000,0.894118,0.788235}%
\pgfsetfillcolor{currentfill}%
\pgfsetlinewidth{0.000000pt}%
\definecolor{currentstroke}{rgb}{1.000000,0.894118,0.788235}%
\pgfsetstrokecolor{currentstroke}%
\pgfsetdash{}{0pt}%
\pgfpathmoveto{\pgfqpoint{2.981238in}{3.615483in}}%
\pgfpathlineto{\pgfqpoint{3.008185in}{3.631041in}}%
\pgfpathlineto{\pgfqpoint{3.008185in}{3.662157in}}%
\pgfpathlineto{\pgfqpoint{2.981238in}{3.646599in}}%
\pgfpathlineto{\pgfqpoint{2.981238in}{3.615483in}}%
\pgfpathclose%
\pgfusepath{fill}%
\end{pgfscope}%
\begin{pgfscope}%
\pgfpathrectangle{\pgfqpoint{0.765000in}{0.660000in}}{\pgfqpoint{4.620000in}{4.620000in}}%
\pgfusepath{clip}%
\pgfsetbuttcap%
\pgfsetroundjoin%
\definecolor{currentfill}{rgb}{1.000000,0.894118,0.788235}%
\pgfsetfillcolor{currentfill}%
\pgfsetlinewidth{0.000000pt}%
\definecolor{currentstroke}{rgb}{1.000000,0.894118,0.788235}%
\pgfsetstrokecolor{currentstroke}%
\pgfsetdash{}{0pt}%
\pgfpathmoveto{\pgfqpoint{2.954291in}{3.631041in}}%
\pgfpathlineto{\pgfqpoint{2.981238in}{3.615483in}}%
\pgfpathlineto{\pgfqpoint{2.981238in}{3.646599in}}%
\pgfpathlineto{\pgfqpoint{2.954291in}{3.662157in}}%
\pgfpathlineto{\pgfqpoint{2.954291in}{3.631041in}}%
\pgfpathclose%
\pgfusepath{fill}%
\end{pgfscope}%
\begin{pgfscope}%
\pgfpathrectangle{\pgfqpoint{0.765000in}{0.660000in}}{\pgfqpoint{4.620000in}{4.620000in}}%
\pgfusepath{clip}%
\pgfsetbuttcap%
\pgfsetroundjoin%
\definecolor{currentfill}{rgb}{1.000000,0.894118,0.788235}%
\pgfsetfillcolor{currentfill}%
\pgfsetlinewidth{0.000000pt}%
\definecolor{currentstroke}{rgb}{1.000000,0.894118,0.788235}%
\pgfsetstrokecolor{currentstroke}%
\pgfsetdash{}{0pt}%
\pgfpathmoveto{\pgfqpoint{2.981592in}{3.650866in}}%
\pgfpathlineto{\pgfqpoint{2.954645in}{3.635308in}}%
\pgfpathlineto{\pgfqpoint{2.954291in}{3.631041in}}%
\pgfpathlineto{\pgfqpoint{2.981238in}{3.646599in}}%
\pgfpathlineto{\pgfqpoint{2.981592in}{3.650866in}}%
\pgfpathclose%
\pgfusepath{fill}%
\end{pgfscope}%
\begin{pgfscope}%
\pgfpathrectangle{\pgfqpoint{0.765000in}{0.660000in}}{\pgfqpoint{4.620000in}{4.620000in}}%
\pgfusepath{clip}%
\pgfsetbuttcap%
\pgfsetroundjoin%
\definecolor{currentfill}{rgb}{1.000000,0.894118,0.788235}%
\pgfsetfillcolor{currentfill}%
\pgfsetlinewidth{0.000000pt}%
\definecolor{currentstroke}{rgb}{1.000000,0.894118,0.788235}%
\pgfsetstrokecolor{currentstroke}%
\pgfsetdash{}{0pt}%
\pgfpathmoveto{\pgfqpoint{3.008539in}{3.635308in}}%
\pgfpathlineto{\pgfqpoint{2.981592in}{3.650866in}}%
\pgfpathlineto{\pgfqpoint{2.981238in}{3.646599in}}%
\pgfpathlineto{\pgfqpoint{3.008185in}{3.631041in}}%
\pgfpathlineto{\pgfqpoint{3.008539in}{3.635308in}}%
\pgfpathclose%
\pgfusepath{fill}%
\end{pgfscope}%
\begin{pgfscope}%
\pgfpathrectangle{\pgfqpoint{0.765000in}{0.660000in}}{\pgfqpoint{4.620000in}{4.620000in}}%
\pgfusepath{clip}%
\pgfsetbuttcap%
\pgfsetroundjoin%
\definecolor{currentfill}{rgb}{1.000000,0.894118,0.788235}%
\pgfsetfillcolor{currentfill}%
\pgfsetlinewidth{0.000000pt}%
\definecolor{currentstroke}{rgb}{1.000000,0.894118,0.788235}%
\pgfsetstrokecolor{currentstroke}%
\pgfsetdash{}{0pt}%
\pgfpathmoveto{\pgfqpoint{2.981592in}{3.650866in}}%
\pgfpathlineto{\pgfqpoint{2.981592in}{3.681982in}}%
\pgfpathlineto{\pgfqpoint{2.981238in}{3.677715in}}%
\pgfpathlineto{\pgfqpoint{3.008185in}{3.631041in}}%
\pgfpathlineto{\pgfqpoint{2.981592in}{3.650866in}}%
\pgfpathclose%
\pgfusepath{fill}%
\end{pgfscope}%
\begin{pgfscope}%
\pgfpathrectangle{\pgfqpoint{0.765000in}{0.660000in}}{\pgfqpoint{4.620000in}{4.620000in}}%
\pgfusepath{clip}%
\pgfsetbuttcap%
\pgfsetroundjoin%
\definecolor{currentfill}{rgb}{1.000000,0.894118,0.788235}%
\pgfsetfillcolor{currentfill}%
\pgfsetlinewidth{0.000000pt}%
\definecolor{currentstroke}{rgb}{1.000000,0.894118,0.788235}%
\pgfsetstrokecolor{currentstroke}%
\pgfsetdash{}{0pt}%
\pgfpathmoveto{\pgfqpoint{3.008539in}{3.635308in}}%
\pgfpathlineto{\pgfqpoint{3.008539in}{3.666424in}}%
\pgfpathlineto{\pgfqpoint{3.008185in}{3.662157in}}%
\pgfpathlineto{\pgfqpoint{2.981238in}{3.646599in}}%
\pgfpathlineto{\pgfqpoint{3.008539in}{3.635308in}}%
\pgfpathclose%
\pgfusepath{fill}%
\end{pgfscope}%
\begin{pgfscope}%
\pgfpathrectangle{\pgfqpoint{0.765000in}{0.660000in}}{\pgfqpoint{4.620000in}{4.620000in}}%
\pgfusepath{clip}%
\pgfsetbuttcap%
\pgfsetroundjoin%
\definecolor{currentfill}{rgb}{1.000000,0.894118,0.788235}%
\pgfsetfillcolor{currentfill}%
\pgfsetlinewidth{0.000000pt}%
\definecolor{currentstroke}{rgb}{1.000000,0.894118,0.788235}%
\pgfsetstrokecolor{currentstroke}%
\pgfsetdash{}{0pt}%
\pgfpathmoveto{\pgfqpoint{2.954645in}{3.635308in}}%
\pgfpathlineto{\pgfqpoint{2.981592in}{3.619750in}}%
\pgfpathlineto{\pgfqpoint{2.981238in}{3.615483in}}%
\pgfpathlineto{\pgfqpoint{2.954291in}{3.631041in}}%
\pgfpathlineto{\pgfqpoint{2.954645in}{3.635308in}}%
\pgfpathclose%
\pgfusepath{fill}%
\end{pgfscope}%
\begin{pgfscope}%
\pgfpathrectangle{\pgfqpoint{0.765000in}{0.660000in}}{\pgfqpoint{4.620000in}{4.620000in}}%
\pgfusepath{clip}%
\pgfsetbuttcap%
\pgfsetroundjoin%
\definecolor{currentfill}{rgb}{1.000000,0.894118,0.788235}%
\pgfsetfillcolor{currentfill}%
\pgfsetlinewidth{0.000000pt}%
\definecolor{currentstroke}{rgb}{1.000000,0.894118,0.788235}%
\pgfsetstrokecolor{currentstroke}%
\pgfsetdash{}{0pt}%
\pgfpathmoveto{\pgfqpoint{2.981592in}{3.619750in}}%
\pgfpathlineto{\pgfqpoint{3.008539in}{3.635308in}}%
\pgfpathlineto{\pgfqpoint{3.008185in}{3.631041in}}%
\pgfpathlineto{\pgfqpoint{2.981238in}{3.615483in}}%
\pgfpathlineto{\pgfqpoint{2.981592in}{3.619750in}}%
\pgfpathclose%
\pgfusepath{fill}%
\end{pgfscope}%
\begin{pgfscope}%
\pgfpathrectangle{\pgfqpoint{0.765000in}{0.660000in}}{\pgfqpoint{4.620000in}{4.620000in}}%
\pgfusepath{clip}%
\pgfsetbuttcap%
\pgfsetroundjoin%
\definecolor{currentfill}{rgb}{1.000000,0.894118,0.788235}%
\pgfsetfillcolor{currentfill}%
\pgfsetlinewidth{0.000000pt}%
\definecolor{currentstroke}{rgb}{1.000000,0.894118,0.788235}%
\pgfsetstrokecolor{currentstroke}%
\pgfsetdash{}{0pt}%
\pgfpathmoveto{\pgfqpoint{2.981592in}{3.681982in}}%
\pgfpathlineto{\pgfqpoint{2.954645in}{3.666424in}}%
\pgfpathlineto{\pgfqpoint{2.954291in}{3.662157in}}%
\pgfpathlineto{\pgfqpoint{2.981238in}{3.677715in}}%
\pgfpathlineto{\pgfqpoint{2.981592in}{3.681982in}}%
\pgfpathclose%
\pgfusepath{fill}%
\end{pgfscope}%
\begin{pgfscope}%
\pgfpathrectangle{\pgfqpoint{0.765000in}{0.660000in}}{\pgfqpoint{4.620000in}{4.620000in}}%
\pgfusepath{clip}%
\pgfsetbuttcap%
\pgfsetroundjoin%
\definecolor{currentfill}{rgb}{1.000000,0.894118,0.788235}%
\pgfsetfillcolor{currentfill}%
\pgfsetlinewidth{0.000000pt}%
\definecolor{currentstroke}{rgb}{1.000000,0.894118,0.788235}%
\pgfsetstrokecolor{currentstroke}%
\pgfsetdash{}{0pt}%
\pgfpathmoveto{\pgfqpoint{3.008539in}{3.666424in}}%
\pgfpathlineto{\pgfqpoint{2.981592in}{3.681982in}}%
\pgfpathlineto{\pgfqpoint{2.981238in}{3.677715in}}%
\pgfpathlineto{\pgfqpoint{3.008185in}{3.662157in}}%
\pgfpathlineto{\pgfqpoint{3.008539in}{3.666424in}}%
\pgfpathclose%
\pgfusepath{fill}%
\end{pgfscope}%
\begin{pgfscope}%
\pgfpathrectangle{\pgfqpoint{0.765000in}{0.660000in}}{\pgfqpoint{4.620000in}{4.620000in}}%
\pgfusepath{clip}%
\pgfsetbuttcap%
\pgfsetroundjoin%
\definecolor{currentfill}{rgb}{1.000000,0.894118,0.788235}%
\pgfsetfillcolor{currentfill}%
\pgfsetlinewidth{0.000000pt}%
\definecolor{currentstroke}{rgb}{1.000000,0.894118,0.788235}%
\pgfsetstrokecolor{currentstroke}%
\pgfsetdash{}{0pt}%
\pgfpathmoveto{\pgfqpoint{2.954645in}{3.635308in}}%
\pgfpathlineto{\pgfqpoint{2.954645in}{3.666424in}}%
\pgfpathlineto{\pgfqpoint{2.954291in}{3.662157in}}%
\pgfpathlineto{\pgfqpoint{2.981238in}{3.615483in}}%
\pgfpathlineto{\pgfqpoint{2.954645in}{3.635308in}}%
\pgfpathclose%
\pgfusepath{fill}%
\end{pgfscope}%
\begin{pgfscope}%
\pgfpathrectangle{\pgfqpoint{0.765000in}{0.660000in}}{\pgfqpoint{4.620000in}{4.620000in}}%
\pgfusepath{clip}%
\pgfsetbuttcap%
\pgfsetroundjoin%
\definecolor{currentfill}{rgb}{1.000000,0.894118,0.788235}%
\pgfsetfillcolor{currentfill}%
\pgfsetlinewidth{0.000000pt}%
\definecolor{currentstroke}{rgb}{1.000000,0.894118,0.788235}%
\pgfsetstrokecolor{currentstroke}%
\pgfsetdash{}{0pt}%
\pgfpathmoveto{\pgfqpoint{2.981592in}{3.619750in}}%
\pgfpathlineto{\pgfqpoint{2.981592in}{3.650866in}}%
\pgfpathlineto{\pgfqpoint{2.981238in}{3.646599in}}%
\pgfpathlineto{\pgfqpoint{2.954291in}{3.631041in}}%
\pgfpathlineto{\pgfqpoint{2.981592in}{3.619750in}}%
\pgfpathclose%
\pgfusepath{fill}%
\end{pgfscope}%
\begin{pgfscope}%
\pgfpathrectangle{\pgfqpoint{0.765000in}{0.660000in}}{\pgfqpoint{4.620000in}{4.620000in}}%
\pgfusepath{clip}%
\pgfsetbuttcap%
\pgfsetroundjoin%
\definecolor{currentfill}{rgb}{1.000000,0.894118,0.788235}%
\pgfsetfillcolor{currentfill}%
\pgfsetlinewidth{0.000000pt}%
\definecolor{currentstroke}{rgb}{1.000000,0.894118,0.788235}%
\pgfsetstrokecolor{currentstroke}%
\pgfsetdash{}{0pt}%
\pgfpathmoveto{\pgfqpoint{2.981592in}{3.650866in}}%
\pgfpathlineto{\pgfqpoint{3.008539in}{3.666424in}}%
\pgfpathlineto{\pgfqpoint{3.008185in}{3.662157in}}%
\pgfpathlineto{\pgfqpoint{2.981238in}{3.646599in}}%
\pgfpathlineto{\pgfqpoint{2.981592in}{3.650866in}}%
\pgfpathclose%
\pgfusepath{fill}%
\end{pgfscope}%
\begin{pgfscope}%
\pgfpathrectangle{\pgfqpoint{0.765000in}{0.660000in}}{\pgfqpoint{4.620000in}{4.620000in}}%
\pgfusepath{clip}%
\pgfsetbuttcap%
\pgfsetroundjoin%
\definecolor{currentfill}{rgb}{1.000000,0.894118,0.788235}%
\pgfsetfillcolor{currentfill}%
\pgfsetlinewidth{0.000000pt}%
\definecolor{currentstroke}{rgb}{1.000000,0.894118,0.788235}%
\pgfsetstrokecolor{currentstroke}%
\pgfsetdash{}{0pt}%
\pgfpathmoveto{\pgfqpoint{2.954645in}{3.666424in}}%
\pgfpathlineto{\pgfqpoint{2.981592in}{3.650866in}}%
\pgfpathlineto{\pgfqpoint{2.981238in}{3.646599in}}%
\pgfpathlineto{\pgfqpoint{2.954291in}{3.662157in}}%
\pgfpathlineto{\pgfqpoint{2.954645in}{3.666424in}}%
\pgfpathclose%
\pgfusepath{fill}%
\end{pgfscope}%
\begin{pgfscope}%
\pgfpathrectangle{\pgfqpoint{0.765000in}{0.660000in}}{\pgfqpoint{4.620000in}{4.620000in}}%
\pgfusepath{clip}%
\pgfsetbuttcap%
\pgfsetroundjoin%
\definecolor{currentfill}{rgb}{1.000000,0.894118,0.788235}%
\pgfsetfillcolor{currentfill}%
\pgfsetlinewidth{0.000000pt}%
\definecolor{currentstroke}{rgb}{1.000000,0.894118,0.788235}%
\pgfsetstrokecolor{currentstroke}%
\pgfsetdash{}{0pt}%
\pgfpathmoveto{\pgfqpoint{3.839490in}{2.833104in}}%
\pgfpathlineto{\pgfqpoint{3.812542in}{2.817546in}}%
\pgfpathlineto{\pgfqpoint{3.839490in}{2.801988in}}%
\pgfpathlineto{\pgfqpoint{3.866437in}{2.817546in}}%
\pgfpathlineto{\pgfqpoint{3.839490in}{2.833104in}}%
\pgfpathclose%
\pgfusepath{fill}%
\end{pgfscope}%
\begin{pgfscope}%
\pgfpathrectangle{\pgfqpoint{0.765000in}{0.660000in}}{\pgfqpoint{4.620000in}{4.620000in}}%
\pgfusepath{clip}%
\pgfsetbuttcap%
\pgfsetroundjoin%
\definecolor{currentfill}{rgb}{1.000000,0.894118,0.788235}%
\pgfsetfillcolor{currentfill}%
\pgfsetlinewidth{0.000000pt}%
\definecolor{currentstroke}{rgb}{1.000000,0.894118,0.788235}%
\pgfsetstrokecolor{currentstroke}%
\pgfsetdash{}{0pt}%
\pgfpathmoveto{\pgfqpoint{3.839490in}{2.833104in}}%
\pgfpathlineto{\pgfqpoint{3.812542in}{2.817546in}}%
\pgfpathlineto{\pgfqpoint{3.812542in}{2.848662in}}%
\pgfpathlineto{\pgfqpoint{3.839490in}{2.864220in}}%
\pgfpathlineto{\pgfqpoint{3.839490in}{2.833104in}}%
\pgfpathclose%
\pgfusepath{fill}%
\end{pgfscope}%
\begin{pgfscope}%
\pgfpathrectangle{\pgfqpoint{0.765000in}{0.660000in}}{\pgfqpoint{4.620000in}{4.620000in}}%
\pgfusepath{clip}%
\pgfsetbuttcap%
\pgfsetroundjoin%
\definecolor{currentfill}{rgb}{1.000000,0.894118,0.788235}%
\pgfsetfillcolor{currentfill}%
\pgfsetlinewidth{0.000000pt}%
\definecolor{currentstroke}{rgb}{1.000000,0.894118,0.788235}%
\pgfsetstrokecolor{currentstroke}%
\pgfsetdash{}{0pt}%
\pgfpathmoveto{\pgfqpoint{3.839490in}{2.833104in}}%
\pgfpathlineto{\pgfqpoint{3.866437in}{2.817546in}}%
\pgfpathlineto{\pgfqpoint{3.866437in}{2.848662in}}%
\pgfpathlineto{\pgfqpoint{3.839490in}{2.864220in}}%
\pgfpathlineto{\pgfqpoint{3.839490in}{2.833104in}}%
\pgfpathclose%
\pgfusepath{fill}%
\end{pgfscope}%
\begin{pgfscope}%
\pgfpathrectangle{\pgfqpoint{0.765000in}{0.660000in}}{\pgfqpoint{4.620000in}{4.620000in}}%
\pgfusepath{clip}%
\pgfsetbuttcap%
\pgfsetroundjoin%
\definecolor{currentfill}{rgb}{1.000000,0.894118,0.788235}%
\pgfsetfillcolor{currentfill}%
\pgfsetlinewidth{0.000000pt}%
\definecolor{currentstroke}{rgb}{1.000000,0.894118,0.788235}%
\pgfsetstrokecolor{currentstroke}%
\pgfsetdash{}{0pt}%
\pgfpathmoveto{\pgfqpoint{3.755404in}{2.825971in}}%
\pgfpathlineto{\pgfqpoint{3.728456in}{2.810413in}}%
\pgfpathlineto{\pgfqpoint{3.755404in}{2.794855in}}%
\pgfpathlineto{\pgfqpoint{3.782351in}{2.810413in}}%
\pgfpathlineto{\pgfqpoint{3.755404in}{2.825971in}}%
\pgfpathclose%
\pgfusepath{fill}%
\end{pgfscope}%
\begin{pgfscope}%
\pgfpathrectangle{\pgfqpoint{0.765000in}{0.660000in}}{\pgfqpoint{4.620000in}{4.620000in}}%
\pgfusepath{clip}%
\pgfsetbuttcap%
\pgfsetroundjoin%
\definecolor{currentfill}{rgb}{1.000000,0.894118,0.788235}%
\pgfsetfillcolor{currentfill}%
\pgfsetlinewidth{0.000000pt}%
\definecolor{currentstroke}{rgb}{1.000000,0.894118,0.788235}%
\pgfsetstrokecolor{currentstroke}%
\pgfsetdash{}{0pt}%
\pgfpathmoveto{\pgfqpoint{3.755404in}{2.825971in}}%
\pgfpathlineto{\pgfqpoint{3.728456in}{2.810413in}}%
\pgfpathlineto{\pgfqpoint{3.728456in}{2.841529in}}%
\pgfpathlineto{\pgfqpoint{3.755404in}{2.857087in}}%
\pgfpathlineto{\pgfqpoint{3.755404in}{2.825971in}}%
\pgfpathclose%
\pgfusepath{fill}%
\end{pgfscope}%
\begin{pgfscope}%
\pgfpathrectangle{\pgfqpoint{0.765000in}{0.660000in}}{\pgfqpoint{4.620000in}{4.620000in}}%
\pgfusepath{clip}%
\pgfsetbuttcap%
\pgfsetroundjoin%
\definecolor{currentfill}{rgb}{1.000000,0.894118,0.788235}%
\pgfsetfillcolor{currentfill}%
\pgfsetlinewidth{0.000000pt}%
\definecolor{currentstroke}{rgb}{1.000000,0.894118,0.788235}%
\pgfsetstrokecolor{currentstroke}%
\pgfsetdash{}{0pt}%
\pgfpathmoveto{\pgfqpoint{3.755404in}{2.825971in}}%
\pgfpathlineto{\pgfqpoint{3.782351in}{2.810413in}}%
\pgfpathlineto{\pgfqpoint{3.782351in}{2.841529in}}%
\pgfpathlineto{\pgfqpoint{3.755404in}{2.857087in}}%
\pgfpathlineto{\pgfqpoint{3.755404in}{2.825971in}}%
\pgfpathclose%
\pgfusepath{fill}%
\end{pgfscope}%
\begin{pgfscope}%
\pgfpathrectangle{\pgfqpoint{0.765000in}{0.660000in}}{\pgfqpoint{4.620000in}{4.620000in}}%
\pgfusepath{clip}%
\pgfsetbuttcap%
\pgfsetroundjoin%
\definecolor{currentfill}{rgb}{1.000000,0.894118,0.788235}%
\pgfsetfillcolor{currentfill}%
\pgfsetlinewidth{0.000000pt}%
\definecolor{currentstroke}{rgb}{1.000000,0.894118,0.788235}%
\pgfsetstrokecolor{currentstroke}%
\pgfsetdash{}{0pt}%
\pgfpathmoveto{\pgfqpoint{3.839490in}{2.864220in}}%
\pgfpathlineto{\pgfqpoint{3.812542in}{2.848662in}}%
\pgfpathlineto{\pgfqpoint{3.839490in}{2.833104in}}%
\pgfpathlineto{\pgfqpoint{3.866437in}{2.848662in}}%
\pgfpathlineto{\pgfqpoint{3.839490in}{2.864220in}}%
\pgfpathclose%
\pgfusepath{fill}%
\end{pgfscope}%
\begin{pgfscope}%
\pgfpathrectangle{\pgfqpoint{0.765000in}{0.660000in}}{\pgfqpoint{4.620000in}{4.620000in}}%
\pgfusepath{clip}%
\pgfsetbuttcap%
\pgfsetroundjoin%
\definecolor{currentfill}{rgb}{1.000000,0.894118,0.788235}%
\pgfsetfillcolor{currentfill}%
\pgfsetlinewidth{0.000000pt}%
\definecolor{currentstroke}{rgb}{1.000000,0.894118,0.788235}%
\pgfsetstrokecolor{currentstroke}%
\pgfsetdash{}{0pt}%
\pgfpathmoveto{\pgfqpoint{3.839490in}{2.801988in}}%
\pgfpathlineto{\pgfqpoint{3.866437in}{2.817546in}}%
\pgfpathlineto{\pgfqpoint{3.866437in}{2.848662in}}%
\pgfpathlineto{\pgfqpoint{3.839490in}{2.833104in}}%
\pgfpathlineto{\pgfqpoint{3.839490in}{2.801988in}}%
\pgfpathclose%
\pgfusepath{fill}%
\end{pgfscope}%
\begin{pgfscope}%
\pgfpathrectangle{\pgfqpoint{0.765000in}{0.660000in}}{\pgfqpoint{4.620000in}{4.620000in}}%
\pgfusepath{clip}%
\pgfsetbuttcap%
\pgfsetroundjoin%
\definecolor{currentfill}{rgb}{1.000000,0.894118,0.788235}%
\pgfsetfillcolor{currentfill}%
\pgfsetlinewidth{0.000000pt}%
\definecolor{currentstroke}{rgb}{1.000000,0.894118,0.788235}%
\pgfsetstrokecolor{currentstroke}%
\pgfsetdash{}{0pt}%
\pgfpathmoveto{\pgfqpoint{3.812542in}{2.817546in}}%
\pgfpathlineto{\pgfqpoint{3.839490in}{2.801988in}}%
\pgfpathlineto{\pgfqpoint{3.839490in}{2.833104in}}%
\pgfpathlineto{\pgfqpoint{3.812542in}{2.848662in}}%
\pgfpathlineto{\pgfqpoint{3.812542in}{2.817546in}}%
\pgfpathclose%
\pgfusepath{fill}%
\end{pgfscope}%
\begin{pgfscope}%
\pgfpathrectangle{\pgfqpoint{0.765000in}{0.660000in}}{\pgfqpoint{4.620000in}{4.620000in}}%
\pgfusepath{clip}%
\pgfsetbuttcap%
\pgfsetroundjoin%
\definecolor{currentfill}{rgb}{1.000000,0.894118,0.788235}%
\pgfsetfillcolor{currentfill}%
\pgfsetlinewidth{0.000000pt}%
\definecolor{currentstroke}{rgb}{1.000000,0.894118,0.788235}%
\pgfsetstrokecolor{currentstroke}%
\pgfsetdash{}{0pt}%
\pgfpathmoveto{\pgfqpoint{3.755404in}{2.857087in}}%
\pgfpathlineto{\pgfqpoint{3.728456in}{2.841529in}}%
\pgfpathlineto{\pgfqpoint{3.755404in}{2.825971in}}%
\pgfpathlineto{\pgfqpoint{3.782351in}{2.841529in}}%
\pgfpathlineto{\pgfqpoint{3.755404in}{2.857087in}}%
\pgfpathclose%
\pgfusepath{fill}%
\end{pgfscope}%
\begin{pgfscope}%
\pgfpathrectangle{\pgfqpoint{0.765000in}{0.660000in}}{\pgfqpoint{4.620000in}{4.620000in}}%
\pgfusepath{clip}%
\pgfsetbuttcap%
\pgfsetroundjoin%
\definecolor{currentfill}{rgb}{1.000000,0.894118,0.788235}%
\pgfsetfillcolor{currentfill}%
\pgfsetlinewidth{0.000000pt}%
\definecolor{currentstroke}{rgb}{1.000000,0.894118,0.788235}%
\pgfsetstrokecolor{currentstroke}%
\pgfsetdash{}{0pt}%
\pgfpathmoveto{\pgfqpoint{3.755404in}{2.794855in}}%
\pgfpathlineto{\pgfqpoint{3.782351in}{2.810413in}}%
\pgfpathlineto{\pgfqpoint{3.782351in}{2.841529in}}%
\pgfpathlineto{\pgfqpoint{3.755404in}{2.825971in}}%
\pgfpathlineto{\pgfqpoint{3.755404in}{2.794855in}}%
\pgfpathclose%
\pgfusepath{fill}%
\end{pgfscope}%
\begin{pgfscope}%
\pgfpathrectangle{\pgfqpoint{0.765000in}{0.660000in}}{\pgfqpoint{4.620000in}{4.620000in}}%
\pgfusepath{clip}%
\pgfsetbuttcap%
\pgfsetroundjoin%
\definecolor{currentfill}{rgb}{1.000000,0.894118,0.788235}%
\pgfsetfillcolor{currentfill}%
\pgfsetlinewidth{0.000000pt}%
\definecolor{currentstroke}{rgb}{1.000000,0.894118,0.788235}%
\pgfsetstrokecolor{currentstroke}%
\pgfsetdash{}{0pt}%
\pgfpathmoveto{\pgfqpoint{3.728456in}{2.810413in}}%
\pgfpathlineto{\pgfqpoint{3.755404in}{2.794855in}}%
\pgfpathlineto{\pgfqpoint{3.755404in}{2.825971in}}%
\pgfpathlineto{\pgfqpoint{3.728456in}{2.841529in}}%
\pgfpathlineto{\pgfqpoint{3.728456in}{2.810413in}}%
\pgfpathclose%
\pgfusepath{fill}%
\end{pgfscope}%
\begin{pgfscope}%
\pgfpathrectangle{\pgfqpoint{0.765000in}{0.660000in}}{\pgfqpoint{4.620000in}{4.620000in}}%
\pgfusepath{clip}%
\pgfsetbuttcap%
\pgfsetroundjoin%
\definecolor{currentfill}{rgb}{1.000000,0.894118,0.788235}%
\pgfsetfillcolor{currentfill}%
\pgfsetlinewidth{0.000000pt}%
\definecolor{currentstroke}{rgb}{1.000000,0.894118,0.788235}%
\pgfsetstrokecolor{currentstroke}%
\pgfsetdash{}{0pt}%
\pgfpathmoveto{\pgfqpoint{3.839490in}{2.833104in}}%
\pgfpathlineto{\pgfqpoint{3.812542in}{2.817546in}}%
\pgfpathlineto{\pgfqpoint{3.728456in}{2.810413in}}%
\pgfpathlineto{\pgfqpoint{3.755404in}{2.825971in}}%
\pgfpathlineto{\pgfqpoint{3.839490in}{2.833104in}}%
\pgfpathclose%
\pgfusepath{fill}%
\end{pgfscope}%
\begin{pgfscope}%
\pgfpathrectangle{\pgfqpoint{0.765000in}{0.660000in}}{\pgfqpoint{4.620000in}{4.620000in}}%
\pgfusepath{clip}%
\pgfsetbuttcap%
\pgfsetroundjoin%
\definecolor{currentfill}{rgb}{1.000000,0.894118,0.788235}%
\pgfsetfillcolor{currentfill}%
\pgfsetlinewidth{0.000000pt}%
\definecolor{currentstroke}{rgb}{1.000000,0.894118,0.788235}%
\pgfsetstrokecolor{currentstroke}%
\pgfsetdash{}{0pt}%
\pgfpathmoveto{\pgfqpoint{3.866437in}{2.817546in}}%
\pgfpathlineto{\pgfqpoint{3.839490in}{2.833104in}}%
\pgfpathlineto{\pgfqpoint{3.755404in}{2.825971in}}%
\pgfpathlineto{\pgfqpoint{3.782351in}{2.810413in}}%
\pgfpathlineto{\pgfqpoint{3.866437in}{2.817546in}}%
\pgfpathclose%
\pgfusepath{fill}%
\end{pgfscope}%
\begin{pgfscope}%
\pgfpathrectangle{\pgfqpoint{0.765000in}{0.660000in}}{\pgfqpoint{4.620000in}{4.620000in}}%
\pgfusepath{clip}%
\pgfsetbuttcap%
\pgfsetroundjoin%
\definecolor{currentfill}{rgb}{1.000000,0.894118,0.788235}%
\pgfsetfillcolor{currentfill}%
\pgfsetlinewidth{0.000000pt}%
\definecolor{currentstroke}{rgb}{1.000000,0.894118,0.788235}%
\pgfsetstrokecolor{currentstroke}%
\pgfsetdash{}{0pt}%
\pgfpathmoveto{\pgfqpoint{3.839490in}{2.833104in}}%
\pgfpathlineto{\pgfqpoint{3.839490in}{2.864220in}}%
\pgfpathlineto{\pgfqpoint{3.755404in}{2.857087in}}%
\pgfpathlineto{\pgfqpoint{3.782351in}{2.810413in}}%
\pgfpathlineto{\pgfqpoint{3.839490in}{2.833104in}}%
\pgfpathclose%
\pgfusepath{fill}%
\end{pgfscope}%
\begin{pgfscope}%
\pgfpathrectangle{\pgfqpoint{0.765000in}{0.660000in}}{\pgfqpoint{4.620000in}{4.620000in}}%
\pgfusepath{clip}%
\pgfsetbuttcap%
\pgfsetroundjoin%
\definecolor{currentfill}{rgb}{1.000000,0.894118,0.788235}%
\pgfsetfillcolor{currentfill}%
\pgfsetlinewidth{0.000000pt}%
\definecolor{currentstroke}{rgb}{1.000000,0.894118,0.788235}%
\pgfsetstrokecolor{currentstroke}%
\pgfsetdash{}{0pt}%
\pgfpathmoveto{\pgfqpoint{3.866437in}{2.817546in}}%
\pgfpathlineto{\pgfqpoint{3.866437in}{2.848662in}}%
\pgfpathlineto{\pgfqpoint{3.782351in}{2.841529in}}%
\pgfpathlineto{\pgfqpoint{3.755404in}{2.825971in}}%
\pgfpathlineto{\pgfqpoint{3.866437in}{2.817546in}}%
\pgfpathclose%
\pgfusepath{fill}%
\end{pgfscope}%
\begin{pgfscope}%
\pgfpathrectangle{\pgfqpoint{0.765000in}{0.660000in}}{\pgfqpoint{4.620000in}{4.620000in}}%
\pgfusepath{clip}%
\pgfsetbuttcap%
\pgfsetroundjoin%
\definecolor{currentfill}{rgb}{1.000000,0.894118,0.788235}%
\pgfsetfillcolor{currentfill}%
\pgfsetlinewidth{0.000000pt}%
\definecolor{currentstroke}{rgb}{1.000000,0.894118,0.788235}%
\pgfsetstrokecolor{currentstroke}%
\pgfsetdash{}{0pt}%
\pgfpathmoveto{\pgfqpoint{3.812542in}{2.817546in}}%
\pgfpathlineto{\pgfqpoint{3.839490in}{2.801988in}}%
\pgfpathlineto{\pgfqpoint{3.755404in}{2.794855in}}%
\pgfpathlineto{\pgfqpoint{3.728456in}{2.810413in}}%
\pgfpathlineto{\pgfqpoint{3.812542in}{2.817546in}}%
\pgfpathclose%
\pgfusepath{fill}%
\end{pgfscope}%
\begin{pgfscope}%
\pgfpathrectangle{\pgfqpoint{0.765000in}{0.660000in}}{\pgfqpoint{4.620000in}{4.620000in}}%
\pgfusepath{clip}%
\pgfsetbuttcap%
\pgfsetroundjoin%
\definecolor{currentfill}{rgb}{1.000000,0.894118,0.788235}%
\pgfsetfillcolor{currentfill}%
\pgfsetlinewidth{0.000000pt}%
\definecolor{currentstroke}{rgb}{1.000000,0.894118,0.788235}%
\pgfsetstrokecolor{currentstroke}%
\pgfsetdash{}{0pt}%
\pgfpathmoveto{\pgfqpoint{3.839490in}{2.801988in}}%
\pgfpathlineto{\pgfqpoint{3.866437in}{2.817546in}}%
\pgfpathlineto{\pgfqpoint{3.782351in}{2.810413in}}%
\pgfpathlineto{\pgfqpoint{3.755404in}{2.794855in}}%
\pgfpathlineto{\pgfqpoint{3.839490in}{2.801988in}}%
\pgfpathclose%
\pgfusepath{fill}%
\end{pgfscope}%
\begin{pgfscope}%
\pgfpathrectangle{\pgfqpoint{0.765000in}{0.660000in}}{\pgfqpoint{4.620000in}{4.620000in}}%
\pgfusepath{clip}%
\pgfsetbuttcap%
\pgfsetroundjoin%
\definecolor{currentfill}{rgb}{1.000000,0.894118,0.788235}%
\pgfsetfillcolor{currentfill}%
\pgfsetlinewidth{0.000000pt}%
\definecolor{currentstroke}{rgb}{1.000000,0.894118,0.788235}%
\pgfsetstrokecolor{currentstroke}%
\pgfsetdash{}{0pt}%
\pgfpathmoveto{\pgfqpoint{3.839490in}{2.864220in}}%
\pgfpathlineto{\pgfqpoint{3.812542in}{2.848662in}}%
\pgfpathlineto{\pgfqpoint{3.728456in}{2.841529in}}%
\pgfpathlineto{\pgfqpoint{3.755404in}{2.857087in}}%
\pgfpathlineto{\pgfqpoint{3.839490in}{2.864220in}}%
\pgfpathclose%
\pgfusepath{fill}%
\end{pgfscope}%
\begin{pgfscope}%
\pgfpathrectangle{\pgfqpoint{0.765000in}{0.660000in}}{\pgfqpoint{4.620000in}{4.620000in}}%
\pgfusepath{clip}%
\pgfsetbuttcap%
\pgfsetroundjoin%
\definecolor{currentfill}{rgb}{1.000000,0.894118,0.788235}%
\pgfsetfillcolor{currentfill}%
\pgfsetlinewidth{0.000000pt}%
\definecolor{currentstroke}{rgb}{1.000000,0.894118,0.788235}%
\pgfsetstrokecolor{currentstroke}%
\pgfsetdash{}{0pt}%
\pgfpathmoveto{\pgfqpoint{3.866437in}{2.848662in}}%
\pgfpathlineto{\pgfqpoint{3.839490in}{2.864220in}}%
\pgfpathlineto{\pgfqpoint{3.755404in}{2.857087in}}%
\pgfpathlineto{\pgfqpoint{3.782351in}{2.841529in}}%
\pgfpathlineto{\pgfqpoint{3.866437in}{2.848662in}}%
\pgfpathclose%
\pgfusepath{fill}%
\end{pgfscope}%
\begin{pgfscope}%
\pgfpathrectangle{\pgfqpoint{0.765000in}{0.660000in}}{\pgfqpoint{4.620000in}{4.620000in}}%
\pgfusepath{clip}%
\pgfsetbuttcap%
\pgfsetroundjoin%
\definecolor{currentfill}{rgb}{1.000000,0.894118,0.788235}%
\pgfsetfillcolor{currentfill}%
\pgfsetlinewidth{0.000000pt}%
\definecolor{currentstroke}{rgb}{1.000000,0.894118,0.788235}%
\pgfsetstrokecolor{currentstroke}%
\pgfsetdash{}{0pt}%
\pgfpathmoveto{\pgfqpoint{3.812542in}{2.817546in}}%
\pgfpathlineto{\pgfqpoint{3.812542in}{2.848662in}}%
\pgfpathlineto{\pgfqpoint{3.728456in}{2.841529in}}%
\pgfpathlineto{\pgfqpoint{3.755404in}{2.794855in}}%
\pgfpathlineto{\pgfqpoint{3.812542in}{2.817546in}}%
\pgfpathclose%
\pgfusepath{fill}%
\end{pgfscope}%
\begin{pgfscope}%
\pgfpathrectangle{\pgfqpoint{0.765000in}{0.660000in}}{\pgfqpoint{4.620000in}{4.620000in}}%
\pgfusepath{clip}%
\pgfsetbuttcap%
\pgfsetroundjoin%
\definecolor{currentfill}{rgb}{1.000000,0.894118,0.788235}%
\pgfsetfillcolor{currentfill}%
\pgfsetlinewidth{0.000000pt}%
\definecolor{currentstroke}{rgb}{1.000000,0.894118,0.788235}%
\pgfsetstrokecolor{currentstroke}%
\pgfsetdash{}{0pt}%
\pgfpathmoveto{\pgfqpoint{3.839490in}{2.801988in}}%
\pgfpathlineto{\pgfqpoint{3.839490in}{2.833104in}}%
\pgfpathlineto{\pgfqpoint{3.755404in}{2.825971in}}%
\pgfpathlineto{\pgfqpoint{3.728456in}{2.810413in}}%
\pgfpathlineto{\pgfqpoint{3.839490in}{2.801988in}}%
\pgfpathclose%
\pgfusepath{fill}%
\end{pgfscope}%
\begin{pgfscope}%
\pgfpathrectangle{\pgfqpoint{0.765000in}{0.660000in}}{\pgfqpoint{4.620000in}{4.620000in}}%
\pgfusepath{clip}%
\pgfsetbuttcap%
\pgfsetroundjoin%
\definecolor{currentfill}{rgb}{1.000000,0.894118,0.788235}%
\pgfsetfillcolor{currentfill}%
\pgfsetlinewidth{0.000000pt}%
\definecolor{currentstroke}{rgb}{1.000000,0.894118,0.788235}%
\pgfsetstrokecolor{currentstroke}%
\pgfsetdash{}{0pt}%
\pgfpathmoveto{\pgfqpoint{3.839490in}{2.833104in}}%
\pgfpathlineto{\pgfqpoint{3.866437in}{2.848662in}}%
\pgfpathlineto{\pgfqpoint{3.782351in}{2.841529in}}%
\pgfpathlineto{\pgfqpoint{3.755404in}{2.825971in}}%
\pgfpathlineto{\pgfqpoint{3.839490in}{2.833104in}}%
\pgfpathclose%
\pgfusepath{fill}%
\end{pgfscope}%
\begin{pgfscope}%
\pgfpathrectangle{\pgfqpoint{0.765000in}{0.660000in}}{\pgfqpoint{4.620000in}{4.620000in}}%
\pgfusepath{clip}%
\pgfsetbuttcap%
\pgfsetroundjoin%
\definecolor{currentfill}{rgb}{1.000000,0.894118,0.788235}%
\pgfsetfillcolor{currentfill}%
\pgfsetlinewidth{0.000000pt}%
\definecolor{currentstroke}{rgb}{1.000000,0.894118,0.788235}%
\pgfsetstrokecolor{currentstroke}%
\pgfsetdash{}{0pt}%
\pgfpathmoveto{\pgfqpoint{3.812542in}{2.848662in}}%
\pgfpathlineto{\pgfqpoint{3.839490in}{2.833104in}}%
\pgfpathlineto{\pgfqpoint{3.755404in}{2.825971in}}%
\pgfpathlineto{\pgfqpoint{3.728456in}{2.841529in}}%
\pgfpathlineto{\pgfqpoint{3.812542in}{2.848662in}}%
\pgfpathclose%
\pgfusepath{fill}%
\end{pgfscope}%
\begin{pgfscope}%
\pgfpathrectangle{\pgfqpoint{0.765000in}{0.660000in}}{\pgfqpoint{4.620000in}{4.620000in}}%
\pgfusepath{clip}%
\pgfsetbuttcap%
\pgfsetroundjoin%
\definecolor{currentfill}{rgb}{1.000000,0.894118,0.788235}%
\pgfsetfillcolor{currentfill}%
\pgfsetlinewidth{0.000000pt}%
\definecolor{currentstroke}{rgb}{1.000000,0.894118,0.788235}%
\pgfsetstrokecolor{currentstroke}%
\pgfsetdash{}{0pt}%
\pgfpathmoveto{\pgfqpoint{3.839490in}{2.833104in}}%
\pgfpathlineto{\pgfqpoint{3.812542in}{2.817546in}}%
\pgfpathlineto{\pgfqpoint{3.839490in}{2.801988in}}%
\pgfpathlineto{\pgfqpoint{3.866437in}{2.817546in}}%
\pgfpathlineto{\pgfqpoint{3.839490in}{2.833104in}}%
\pgfpathclose%
\pgfusepath{fill}%
\end{pgfscope}%
\begin{pgfscope}%
\pgfpathrectangle{\pgfqpoint{0.765000in}{0.660000in}}{\pgfqpoint{4.620000in}{4.620000in}}%
\pgfusepath{clip}%
\pgfsetbuttcap%
\pgfsetroundjoin%
\definecolor{currentfill}{rgb}{1.000000,0.894118,0.788235}%
\pgfsetfillcolor{currentfill}%
\pgfsetlinewidth{0.000000pt}%
\definecolor{currentstroke}{rgb}{1.000000,0.894118,0.788235}%
\pgfsetstrokecolor{currentstroke}%
\pgfsetdash{}{0pt}%
\pgfpathmoveto{\pgfqpoint{3.839490in}{2.833104in}}%
\pgfpathlineto{\pgfqpoint{3.812542in}{2.817546in}}%
\pgfpathlineto{\pgfqpoint{3.812542in}{2.848662in}}%
\pgfpathlineto{\pgfqpoint{3.839490in}{2.864220in}}%
\pgfpathlineto{\pgfqpoint{3.839490in}{2.833104in}}%
\pgfpathclose%
\pgfusepath{fill}%
\end{pgfscope}%
\begin{pgfscope}%
\pgfpathrectangle{\pgfqpoint{0.765000in}{0.660000in}}{\pgfqpoint{4.620000in}{4.620000in}}%
\pgfusepath{clip}%
\pgfsetbuttcap%
\pgfsetroundjoin%
\definecolor{currentfill}{rgb}{1.000000,0.894118,0.788235}%
\pgfsetfillcolor{currentfill}%
\pgfsetlinewidth{0.000000pt}%
\definecolor{currentstroke}{rgb}{1.000000,0.894118,0.788235}%
\pgfsetstrokecolor{currentstroke}%
\pgfsetdash{}{0pt}%
\pgfpathmoveto{\pgfqpoint{3.839490in}{2.833104in}}%
\pgfpathlineto{\pgfqpoint{3.866437in}{2.817546in}}%
\pgfpathlineto{\pgfqpoint{3.866437in}{2.848662in}}%
\pgfpathlineto{\pgfqpoint{3.839490in}{2.864220in}}%
\pgfpathlineto{\pgfqpoint{3.839490in}{2.833104in}}%
\pgfpathclose%
\pgfusepath{fill}%
\end{pgfscope}%
\begin{pgfscope}%
\pgfpathrectangle{\pgfqpoint{0.765000in}{0.660000in}}{\pgfqpoint{4.620000in}{4.620000in}}%
\pgfusepath{clip}%
\pgfsetbuttcap%
\pgfsetroundjoin%
\definecolor{currentfill}{rgb}{1.000000,0.894118,0.788235}%
\pgfsetfillcolor{currentfill}%
\pgfsetlinewidth{0.000000pt}%
\definecolor{currentstroke}{rgb}{1.000000,0.894118,0.788235}%
\pgfsetstrokecolor{currentstroke}%
\pgfsetdash{}{0pt}%
\pgfpathmoveto{\pgfqpoint{3.856631in}{2.841507in}}%
\pgfpathlineto{\pgfqpoint{3.829684in}{2.825949in}}%
\pgfpathlineto{\pgfqpoint{3.856631in}{2.810391in}}%
\pgfpathlineto{\pgfqpoint{3.883578in}{2.825949in}}%
\pgfpathlineto{\pgfqpoint{3.856631in}{2.841507in}}%
\pgfpathclose%
\pgfusepath{fill}%
\end{pgfscope}%
\begin{pgfscope}%
\pgfpathrectangle{\pgfqpoint{0.765000in}{0.660000in}}{\pgfqpoint{4.620000in}{4.620000in}}%
\pgfusepath{clip}%
\pgfsetbuttcap%
\pgfsetroundjoin%
\definecolor{currentfill}{rgb}{1.000000,0.894118,0.788235}%
\pgfsetfillcolor{currentfill}%
\pgfsetlinewidth{0.000000pt}%
\definecolor{currentstroke}{rgb}{1.000000,0.894118,0.788235}%
\pgfsetstrokecolor{currentstroke}%
\pgfsetdash{}{0pt}%
\pgfpathmoveto{\pgfqpoint{3.856631in}{2.841507in}}%
\pgfpathlineto{\pgfqpoint{3.829684in}{2.825949in}}%
\pgfpathlineto{\pgfqpoint{3.829684in}{2.857065in}}%
\pgfpathlineto{\pgfqpoint{3.856631in}{2.872623in}}%
\pgfpathlineto{\pgfqpoint{3.856631in}{2.841507in}}%
\pgfpathclose%
\pgfusepath{fill}%
\end{pgfscope}%
\begin{pgfscope}%
\pgfpathrectangle{\pgfqpoint{0.765000in}{0.660000in}}{\pgfqpoint{4.620000in}{4.620000in}}%
\pgfusepath{clip}%
\pgfsetbuttcap%
\pgfsetroundjoin%
\definecolor{currentfill}{rgb}{1.000000,0.894118,0.788235}%
\pgfsetfillcolor{currentfill}%
\pgfsetlinewidth{0.000000pt}%
\definecolor{currentstroke}{rgb}{1.000000,0.894118,0.788235}%
\pgfsetstrokecolor{currentstroke}%
\pgfsetdash{}{0pt}%
\pgfpathmoveto{\pgfqpoint{3.856631in}{2.841507in}}%
\pgfpathlineto{\pgfqpoint{3.883578in}{2.825949in}}%
\pgfpathlineto{\pgfqpoint{3.883578in}{2.857065in}}%
\pgfpathlineto{\pgfqpoint{3.856631in}{2.872623in}}%
\pgfpathlineto{\pgfqpoint{3.856631in}{2.841507in}}%
\pgfpathclose%
\pgfusepath{fill}%
\end{pgfscope}%
\begin{pgfscope}%
\pgfpathrectangle{\pgfqpoint{0.765000in}{0.660000in}}{\pgfqpoint{4.620000in}{4.620000in}}%
\pgfusepath{clip}%
\pgfsetbuttcap%
\pgfsetroundjoin%
\definecolor{currentfill}{rgb}{1.000000,0.894118,0.788235}%
\pgfsetfillcolor{currentfill}%
\pgfsetlinewidth{0.000000pt}%
\definecolor{currentstroke}{rgb}{1.000000,0.894118,0.788235}%
\pgfsetstrokecolor{currentstroke}%
\pgfsetdash{}{0pt}%
\pgfpathmoveto{\pgfqpoint{3.839490in}{2.864220in}}%
\pgfpathlineto{\pgfqpoint{3.812542in}{2.848662in}}%
\pgfpathlineto{\pgfqpoint{3.839490in}{2.833104in}}%
\pgfpathlineto{\pgfqpoint{3.866437in}{2.848662in}}%
\pgfpathlineto{\pgfqpoint{3.839490in}{2.864220in}}%
\pgfpathclose%
\pgfusepath{fill}%
\end{pgfscope}%
\begin{pgfscope}%
\pgfpathrectangle{\pgfqpoint{0.765000in}{0.660000in}}{\pgfqpoint{4.620000in}{4.620000in}}%
\pgfusepath{clip}%
\pgfsetbuttcap%
\pgfsetroundjoin%
\definecolor{currentfill}{rgb}{1.000000,0.894118,0.788235}%
\pgfsetfillcolor{currentfill}%
\pgfsetlinewidth{0.000000pt}%
\definecolor{currentstroke}{rgb}{1.000000,0.894118,0.788235}%
\pgfsetstrokecolor{currentstroke}%
\pgfsetdash{}{0pt}%
\pgfpathmoveto{\pgfqpoint{3.839490in}{2.801988in}}%
\pgfpathlineto{\pgfqpoint{3.866437in}{2.817546in}}%
\pgfpathlineto{\pgfqpoint{3.866437in}{2.848662in}}%
\pgfpathlineto{\pgfqpoint{3.839490in}{2.833104in}}%
\pgfpathlineto{\pgfqpoint{3.839490in}{2.801988in}}%
\pgfpathclose%
\pgfusepath{fill}%
\end{pgfscope}%
\begin{pgfscope}%
\pgfpathrectangle{\pgfqpoint{0.765000in}{0.660000in}}{\pgfqpoint{4.620000in}{4.620000in}}%
\pgfusepath{clip}%
\pgfsetbuttcap%
\pgfsetroundjoin%
\definecolor{currentfill}{rgb}{1.000000,0.894118,0.788235}%
\pgfsetfillcolor{currentfill}%
\pgfsetlinewidth{0.000000pt}%
\definecolor{currentstroke}{rgb}{1.000000,0.894118,0.788235}%
\pgfsetstrokecolor{currentstroke}%
\pgfsetdash{}{0pt}%
\pgfpathmoveto{\pgfqpoint{3.812542in}{2.817546in}}%
\pgfpathlineto{\pgfqpoint{3.839490in}{2.801988in}}%
\pgfpathlineto{\pgfqpoint{3.839490in}{2.833104in}}%
\pgfpathlineto{\pgfqpoint{3.812542in}{2.848662in}}%
\pgfpathlineto{\pgfqpoint{3.812542in}{2.817546in}}%
\pgfpathclose%
\pgfusepath{fill}%
\end{pgfscope}%
\begin{pgfscope}%
\pgfpathrectangle{\pgfqpoint{0.765000in}{0.660000in}}{\pgfqpoint{4.620000in}{4.620000in}}%
\pgfusepath{clip}%
\pgfsetbuttcap%
\pgfsetroundjoin%
\definecolor{currentfill}{rgb}{1.000000,0.894118,0.788235}%
\pgfsetfillcolor{currentfill}%
\pgfsetlinewidth{0.000000pt}%
\definecolor{currentstroke}{rgb}{1.000000,0.894118,0.788235}%
\pgfsetstrokecolor{currentstroke}%
\pgfsetdash{}{0pt}%
\pgfpathmoveto{\pgfqpoint{3.856631in}{2.872623in}}%
\pgfpathlineto{\pgfqpoint{3.829684in}{2.857065in}}%
\pgfpathlineto{\pgfqpoint{3.856631in}{2.841507in}}%
\pgfpathlineto{\pgfqpoint{3.883578in}{2.857065in}}%
\pgfpathlineto{\pgfqpoint{3.856631in}{2.872623in}}%
\pgfpathclose%
\pgfusepath{fill}%
\end{pgfscope}%
\begin{pgfscope}%
\pgfpathrectangle{\pgfqpoint{0.765000in}{0.660000in}}{\pgfqpoint{4.620000in}{4.620000in}}%
\pgfusepath{clip}%
\pgfsetbuttcap%
\pgfsetroundjoin%
\definecolor{currentfill}{rgb}{1.000000,0.894118,0.788235}%
\pgfsetfillcolor{currentfill}%
\pgfsetlinewidth{0.000000pt}%
\definecolor{currentstroke}{rgb}{1.000000,0.894118,0.788235}%
\pgfsetstrokecolor{currentstroke}%
\pgfsetdash{}{0pt}%
\pgfpathmoveto{\pgfqpoint{3.856631in}{2.810391in}}%
\pgfpathlineto{\pgfqpoint{3.883578in}{2.825949in}}%
\pgfpathlineto{\pgfqpoint{3.883578in}{2.857065in}}%
\pgfpathlineto{\pgfqpoint{3.856631in}{2.841507in}}%
\pgfpathlineto{\pgfqpoint{3.856631in}{2.810391in}}%
\pgfpathclose%
\pgfusepath{fill}%
\end{pgfscope}%
\begin{pgfscope}%
\pgfpathrectangle{\pgfqpoint{0.765000in}{0.660000in}}{\pgfqpoint{4.620000in}{4.620000in}}%
\pgfusepath{clip}%
\pgfsetbuttcap%
\pgfsetroundjoin%
\definecolor{currentfill}{rgb}{1.000000,0.894118,0.788235}%
\pgfsetfillcolor{currentfill}%
\pgfsetlinewidth{0.000000pt}%
\definecolor{currentstroke}{rgb}{1.000000,0.894118,0.788235}%
\pgfsetstrokecolor{currentstroke}%
\pgfsetdash{}{0pt}%
\pgfpathmoveto{\pgfqpoint{3.829684in}{2.825949in}}%
\pgfpathlineto{\pgfqpoint{3.856631in}{2.810391in}}%
\pgfpathlineto{\pgfqpoint{3.856631in}{2.841507in}}%
\pgfpathlineto{\pgfqpoint{3.829684in}{2.857065in}}%
\pgfpathlineto{\pgfqpoint{3.829684in}{2.825949in}}%
\pgfpathclose%
\pgfusepath{fill}%
\end{pgfscope}%
\begin{pgfscope}%
\pgfpathrectangle{\pgfqpoint{0.765000in}{0.660000in}}{\pgfqpoint{4.620000in}{4.620000in}}%
\pgfusepath{clip}%
\pgfsetbuttcap%
\pgfsetroundjoin%
\definecolor{currentfill}{rgb}{1.000000,0.894118,0.788235}%
\pgfsetfillcolor{currentfill}%
\pgfsetlinewidth{0.000000pt}%
\definecolor{currentstroke}{rgb}{1.000000,0.894118,0.788235}%
\pgfsetstrokecolor{currentstroke}%
\pgfsetdash{}{0pt}%
\pgfpathmoveto{\pgfqpoint{3.839490in}{2.833104in}}%
\pgfpathlineto{\pgfqpoint{3.812542in}{2.817546in}}%
\pgfpathlineto{\pgfqpoint{3.829684in}{2.825949in}}%
\pgfpathlineto{\pgfqpoint{3.856631in}{2.841507in}}%
\pgfpathlineto{\pgfqpoint{3.839490in}{2.833104in}}%
\pgfpathclose%
\pgfusepath{fill}%
\end{pgfscope}%
\begin{pgfscope}%
\pgfpathrectangle{\pgfqpoint{0.765000in}{0.660000in}}{\pgfqpoint{4.620000in}{4.620000in}}%
\pgfusepath{clip}%
\pgfsetbuttcap%
\pgfsetroundjoin%
\definecolor{currentfill}{rgb}{1.000000,0.894118,0.788235}%
\pgfsetfillcolor{currentfill}%
\pgfsetlinewidth{0.000000pt}%
\definecolor{currentstroke}{rgb}{1.000000,0.894118,0.788235}%
\pgfsetstrokecolor{currentstroke}%
\pgfsetdash{}{0pt}%
\pgfpathmoveto{\pgfqpoint{3.866437in}{2.817546in}}%
\pgfpathlineto{\pgfqpoint{3.839490in}{2.833104in}}%
\pgfpathlineto{\pgfqpoint{3.856631in}{2.841507in}}%
\pgfpathlineto{\pgfqpoint{3.883578in}{2.825949in}}%
\pgfpathlineto{\pgfqpoint{3.866437in}{2.817546in}}%
\pgfpathclose%
\pgfusepath{fill}%
\end{pgfscope}%
\begin{pgfscope}%
\pgfpathrectangle{\pgfqpoint{0.765000in}{0.660000in}}{\pgfqpoint{4.620000in}{4.620000in}}%
\pgfusepath{clip}%
\pgfsetbuttcap%
\pgfsetroundjoin%
\definecolor{currentfill}{rgb}{1.000000,0.894118,0.788235}%
\pgfsetfillcolor{currentfill}%
\pgfsetlinewidth{0.000000pt}%
\definecolor{currentstroke}{rgb}{1.000000,0.894118,0.788235}%
\pgfsetstrokecolor{currentstroke}%
\pgfsetdash{}{0pt}%
\pgfpathmoveto{\pgfqpoint{3.839490in}{2.833104in}}%
\pgfpathlineto{\pgfqpoint{3.839490in}{2.864220in}}%
\pgfpathlineto{\pgfqpoint{3.856631in}{2.872623in}}%
\pgfpathlineto{\pgfqpoint{3.883578in}{2.825949in}}%
\pgfpathlineto{\pgfqpoint{3.839490in}{2.833104in}}%
\pgfpathclose%
\pgfusepath{fill}%
\end{pgfscope}%
\begin{pgfscope}%
\pgfpathrectangle{\pgfqpoint{0.765000in}{0.660000in}}{\pgfqpoint{4.620000in}{4.620000in}}%
\pgfusepath{clip}%
\pgfsetbuttcap%
\pgfsetroundjoin%
\definecolor{currentfill}{rgb}{1.000000,0.894118,0.788235}%
\pgfsetfillcolor{currentfill}%
\pgfsetlinewidth{0.000000pt}%
\definecolor{currentstroke}{rgb}{1.000000,0.894118,0.788235}%
\pgfsetstrokecolor{currentstroke}%
\pgfsetdash{}{0pt}%
\pgfpathmoveto{\pgfqpoint{3.866437in}{2.817546in}}%
\pgfpathlineto{\pgfqpoint{3.866437in}{2.848662in}}%
\pgfpathlineto{\pgfqpoint{3.883578in}{2.857065in}}%
\pgfpathlineto{\pgfqpoint{3.856631in}{2.841507in}}%
\pgfpathlineto{\pgfqpoint{3.866437in}{2.817546in}}%
\pgfpathclose%
\pgfusepath{fill}%
\end{pgfscope}%
\begin{pgfscope}%
\pgfpathrectangle{\pgfqpoint{0.765000in}{0.660000in}}{\pgfqpoint{4.620000in}{4.620000in}}%
\pgfusepath{clip}%
\pgfsetbuttcap%
\pgfsetroundjoin%
\definecolor{currentfill}{rgb}{1.000000,0.894118,0.788235}%
\pgfsetfillcolor{currentfill}%
\pgfsetlinewidth{0.000000pt}%
\definecolor{currentstroke}{rgb}{1.000000,0.894118,0.788235}%
\pgfsetstrokecolor{currentstroke}%
\pgfsetdash{}{0pt}%
\pgfpathmoveto{\pgfqpoint{3.812542in}{2.817546in}}%
\pgfpathlineto{\pgfqpoint{3.839490in}{2.801988in}}%
\pgfpathlineto{\pgfqpoint{3.856631in}{2.810391in}}%
\pgfpathlineto{\pgfqpoint{3.829684in}{2.825949in}}%
\pgfpathlineto{\pgfqpoint{3.812542in}{2.817546in}}%
\pgfpathclose%
\pgfusepath{fill}%
\end{pgfscope}%
\begin{pgfscope}%
\pgfpathrectangle{\pgfqpoint{0.765000in}{0.660000in}}{\pgfqpoint{4.620000in}{4.620000in}}%
\pgfusepath{clip}%
\pgfsetbuttcap%
\pgfsetroundjoin%
\definecolor{currentfill}{rgb}{1.000000,0.894118,0.788235}%
\pgfsetfillcolor{currentfill}%
\pgfsetlinewidth{0.000000pt}%
\definecolor{currentstroke}{rgb}{1.000000,0.894118,0.788235}%
\pgfsetstrokecolor{currentstroke}%
\pgfsetdash{}{0pt}%
\pgfpathmoveto{\pgfqpoint{3.839490in}{2.801988in}}%
\pgfpathlineto{\pgfqpoint{3.866437in}{2.817546in}}%
\pgfpathlineto{\pgfqpoint{3.883578in}{2.825949in}}%
\pgfpathlineto{\pgfqpoint{3.856631in}{2.810391in}}%
\pgfpathlineto{\pgfqpoint{3.839490in}{2.801988in}}%
\pgfpathclose%
\pgfusepath{fill}%
\end{pgfscope}%
\begin{pgfscope}%
\pgfpathrectangle{\pgfqpoint{0.765000in}{0.660000in}}{\pgfqpoint{4.620000in}{4.620000in}}%
\pgfusepath{clip}%
\pgfsetbuttcap%
\pgfsetroundjoin%
\definecolor{currentfill}{rgb}{1.000000,0.894118,0.788235}%
\pgfsetfillcolor{currentfill}%
\pgfsetlinewidth{0.000000pt}%
\definecolor{currentstroke}{rgb}{1.000000,0.894118,0.788235}%
\pgfsetstrokecolor{currentstroke}%
\pgfsetdash{}{0pt}%
\pgfpathmoveto{\pgfqpoint{3.839490in}{2.864220in}}%
\pgfpathlineto{\pgfqpoint{3.812542in}{2.848662in}}%
\pgfpathlineto{\pgfqpoint{3.829684in}{2.857065in}}%
\pgfpathlineto{\pgfqpoint{3.856631in}{2.872623in}}%
\pgfpathlineto{\pgfqpoint{3.839490in}{2.864220in}}%
\pgfpathclose%
\pgfusepath{fill}%
\end{pgfscope}%
\begin{pgfscope}%
\pgfpathrectangle{\pgfqpoint{0.765000in}{0.660000in}}{\pgfqpoint{4.620000in}{4.620000in}}%
\pgfusepath{clip}%
\pgfsetbuttcap%
\pgfsetroundjoin%
\definecolor{currentfill}{rgb}{1.000000,0.894118,0.788235}%
\pgfsetfillcolor{currentfill}%
\pgfsetlinewidth{0.000000pt}%
\definecolor{currentstroke}{rgb}{1.000000,0.894118,0.788235}%
\pgfsetstrokecolor{currentstroke}%
\pgfsetdash{}{0pt}%
\pgfpathmoveto{\pgfqpoint{3.866437in}{2.848662in}}%
\pgfpathlineto{\pgfqpoint{3.839490in}{2.864220in}}%
\pgfpathlineto{\pgfqpoint{3.856631in}{2.872623in}}%
\pgfpathlineto{\pgfqpoint{3.883578in}{2.857065in}}%
\pgfpathlineto{\pgfqpoint{3.866437in}{2.848662in}}%
\pgfpathclose%
\pgfusepath{fill}%
\end{pgfscope}%
\begin{pgfscope}%
\pgfpathrectangle{\pgfqpoint{0.765000in}{0.660000in}}{\pgfqpoint{4.620000in}{4.620000in}}%
\pgfusepath{clip}%
\pgfsetbuttcap%
\pgfsetroundjoin%
\definecolor{currentfill}{rgb}{1.000000,0.894118,0.788235}%
\pgfsetfillcolor{currentfill}%
\pgfsetlinewidth{0.000000pt}%
\definecolor{currentstroke}{rgb}{1.000000,0.894118,0.788235}%
\pgfsetstrokecolor{currentstroke}%
\pgfsetdash{}{0pt}%
\pgfpathmoveto{\pgfqpoint{3.812542in}{2.817546in}}%
\pgfpathlineto{\pgfqpoint{3.812542in}{2.848662in}}%
\pgfpathlineto{\pgfqpoint{3.829684in}{2.857065in}}%
\pgfpathlineto{\pgfqpoint{3.856631in}{2.810391in}}%
\pgfpathlineto{\pgfqpoint{3.812542in}{2.817546in}}%
\pgfpathclose%
\pgfusepath{fill}%
\end{pgfscope}%
\begin{pgfscope}%
\pgfpathrectangle{\pgfqpoint{0.765000in}{0.660000in}}{\pgfqpoint{4.620000in}{4.620000in}}%
\pgfusepath{clip}%
\pgfsetbuttcap%
\pgfsetroundjoin%
\definecolor{currentfill}{rgb}{1.000000,0.894118,0.788235}%
\pgfsetfillcolor{currentfill}%
\pgfsetlinewidth{0.000000pt}%
\definecolor{currentstroke}{rgb}{1.000000,0.894118,0.788235}%
\pgfsetstrokecolor{currentstroke}%
\pgfsetdash{}{0pt}%
\pgfpathmoveto{\pgfqpoint{3.839490in}{2.801988in}}%
\pgfpathlineto{\pgfqpoint{3.839490in}{2.833104in}}%
\pgfpathlineto{\pgfqpoint{3.856631in}{2.841507in}}%
\pgfpathlineto{\pgfqpoint{3.829684in}{2.825949in}}%
\pgfpathlineto{\pgfqpoint{3.839490in}{2.801988in}}%
\pgfpathclose%
\pgfusepath{fill}%
\end{pgfscope}%
\begin{pgfscope}%
\pgfpathrectangle{\pgfqpoint{0.765000in}{0.660000in}}{\pgfqpoint{4.620000in}{4.620000in}}%
\pgfusepath{clip}%
\pgfsetbuttcap%
\pgfsetroundjoin%
\definecolor{currentfill}{rgb}{1.000000,0.894118,0.788235}%
\pgfsetfillcolor{currentfill}%
\pgfsetlinewidth{0.000000pt}%
\definecolor{currentstroke}{rgb}{1.000000,0.894118,0.788235}%
\pgfsetstrokecolor{currentstroke}%
\pgfsetdash{}{0pt}%
\pgfpathmoveto{\pgfqpoint{3.839490in}{2.833104in}}%
\pgfpathlineto{\pgfqpoint{3.866437in}{2.848662in}}%
\pgfpathlineto{\pgfqpoint{3.883578in}{2.857065in}}%
\pgfpathlineto{\pgfqpoint{3.856631in}{2.841507in}}%
\pgfpathlineto{\pgfqpoint{3.839490in}{2.833104in}}%
\pgfpathclose%
\pgfusepath{fill}%
\end{pgfscope}%
\begin{pgfscope}%
\pgfpathrectangle{\pgfqpoint{0.765000in}{0.660000in}}{\pgfqpoint{4.620000in}{4.620000in}}%
\pgfusepath{clip}%
\pgfsetbuttcap%
\pgfsetroundjoin%
\definecolor{currentfill}{rgb}{1.000000,0.894118,0.788235}%
\pgfsetfillcolor{currentfill}%
\pgfsetlinewidth{0.000000pt}%
\definecolor{currentstroke}{rgb}{1.000000,0.894118,0.788235}%
\pgfsetstrokecolor{currentstroke}%
\pgfsetdash{}{0pt}%
\pgfpathmoveto{\pgfqpoint{3.812542in}{2.848662in}}%
\pgfpathlineto{\pgfqpoint{3.839490in}{2.833104in}}%
\pgfpathlineto{\pgfqpoint{3.856631in}{2.841507in}}%
\pgfpathlineto{\pgfqpoint{3.829684in}{2.857065in}}%
\pgfpathlineto{\pgfqpoint{3.812542in}{2.848662in}}%
\pgfpathclose%
\pgfusepath{fill}%
\end{pgfscope}%
\begin{pgfscope}%
\pgfpathrectangle{\pgfqpoint{0.765000in}{0.660000in}}{\pgfqpoint{4.620000in}{4.620000in}}%
\pgfusepath{clip}%
\pgfsetbuttcap%
\pgfsetroundjoin%
\definecolor{currentfill}{rgb}{1.000000,0.894118,0.788235}%
\pgfsetfillcolor{currentfill}%
\pgfsetlinewidth{0.000000pt}%
\definecolor{currentstroke}{rgb}{1.000000,0.894118,0.788235}%
\pgfsetstrokecolor{currentstroke}%
\pgfsetdash{}{0pt}%
\pgfpathmoveto{\pgfqpoint{3.755404in}{2.825971in}}%
\pgfpathlineto{\pgfqpoint{3.728456in}{2.810413in}}%
\pgfpathlineto{\pgfqpoint{3.755404in}{2.794855in}}%
\pgfpathlineto{\pgfqpoint{3.782351in}{2.810413in}}%
\pgfpathlineto{\pgfqpoint{3.755404in}{2.825971in}}%
\pgfpathclose%
\pgfusepath{fill}%
\end{pgfscope}%
\begin{pgfscope}%
\pgfpathrectangle{\pgfqpoint{0.765000in}{0.660000in}}{\pgfqpoint{4.620000in}{4.620000in}}%
\pgfusepath{clip}%
\pgfsetbuttcap%
\pgfsetroundjoin%
\definecolor{currentfill}{rgb}{1.000000,0.894118,0.788235}%
\pgfsetfillcolor{currentfill}%
\pgfsetlinewidth{0.000000pt}%
\definecolor{currentstroke}{rgb}{1.000000,0.894118,0.788235}%
\pgfsetstrokecolor{currentstroke}%
\pgfsetdash{}{0pt}%
\pgfpathmoveto{\pgfqpoint{3.755404in}{2.825971in}}%
\pgfpathlineto{\pgfqpoint{3.728456in}{2.810413in}}%
\pgfpathlineto{\pgfqpoint{3.728456in}{2.841529in}}%
\pgfpathlineto{\pgfqpoint{3.755404in}{2.857087in}}%
\pgfpathlineto{\pgfqpoint{3.755404in}{2.825971in}}%
\pgfpathclose%
\pgfusepath{fill}%
\end{pgfscope}%
\begin{pgfscope}%
\pgfpathrectangle{\pgfqpoint{0.765000in}{0.660000in}}{\pgfqpoint{4.620000in}{4.620000in}}%
\pgfusepath{clip}%
\pgfsetbuttcap%
\pgfsetroundjoin%
\definecolor{currentfill}{rgb}{1.000000,0.894118,0.788235}%
\pgfsetfillcolor{currentfill}%
\pgfsetlinewidth{0.000000pt}%
\definecolor{currentstroke}{rgb}{1.000000,0.894118,0.788235}%
\pgfsetstrokecolor{currentstroke}%
\pgfsetdash{}{0pt}%
\pgfpathmoveto{\pgfqpoint{3.755404in}{2.825971in}}%
\pgfpathlineto{\pgfqpoint{3.782351in}{2.810413in}}%
\pgfpathlineto{\pgfqpoint{3.782351in}{2.841529in}}%
\pgfpathlineto{\pgfqpoint{3.755404in}{2.857087in}}%
\pgfpathlineto{\pgfqpoint{3.755404in}{2.825971in}}%
\pgfpathclose%
\pgfusepath{fill}%
\end{pgfscope}%
\begin{pgfscope}%
\pgfpathrectangle{\pgfqpoint{0.765000in}{0.660000in}}{\pgfqpoint{4.620000in}{4.620000in}}%
\pgfusepath{clip}%
\pgfsetbuttcap%
\pgfsetroundjoin%
\definecolor{currentfill}{rgb}{1.000000,0.894118,0.788235}%
\pgfsetfillcolor{currentfill}%
\pgfsetlinewidth{0.000000pt}%
\definecolor{currentstroke}{rgb}{1.000000,0.894118,0.788235}%
\pgfsetstrokecolor{currentstroke}%
\pgfsetdash{}{0pt}%
\pgfpathmoveto{\pgfqpoint{3.839490in}{2.833104in}}%
\pgfpathlineto{\pgfqpoint{3.812542in}{2.817546in}}%
\pgfpathlineto{\pgfqpoint{3.839490in}{2.801988in}}%
\pgfpathlineto{\pgfqpoint{3.866437in}{2.817546in}}%
\pgfpathlineto{\pgfqpoint{3.839490in}{2.833104in}}%
\pgfpathclose%
\pgfusepath{fill}%
\end{pgfscope}%
\begin{pgfscope}%
\pgfpathrectangle{\pgfqpoint{0.765000in}{0.660000in}}{\pgfqpoint{4.620000in}{4.620000in}}%
\pgfusepath{clip}%
\pgfsetbuttcap%
\pgfsetroundjoin%
\definecolor{currentfill}{rgb}{1.000000,0.894118,0.788235}%
\pgfsetfillcolor{currentfill}%
\pgfsetlinewidth{0.000000pt}%
\definecolor{currentstroke}{rgb}{1.000000,0.894118,0.788235}%
\pgfsetstrokecolor{currentstroke}%
\pgfsetdash{}{0pt}%
\pgfpathmoveto{\pgfqpoint{3.839490in}{2.833104in}}%
\pgfpathlineto{\pgfqpoint{3.812542in}{2.817546in}}%
\pgfpathlineto{\pgfqpoint{3.812542in}{2.848662in}}%
\pgfpathlineto{\pgfqpoint{3.839490in}{2.864220in}}%
\pgfpathlineto{\pgfqpoint{3.839490in}{2.833104in}}%
\pgfpathclose%
\pgfusepath{fill}%
\end{pgfscope}%
\begin{pgfscope}%
\pgfpathrectangle{\pgfqpoint{0.765000in}{0.660000in}}{\pgfqpoint{4.620000in}{4.620000in}}%
\pgfusepath{clip}%
\pgfsetbuttcap%
\pgfsetroundjoin%
\definecolor{currentfill}{rgb}{1.000000,0.894118,0.788235}%
\pgfsetfillcolor{currentfill}%
\pgfsetlinewidth{0.000000pt}%
\definecolor{currentstroke}{rgb}{1.000000,0.894118,0.788235}%
\pgfsetstrokecolor{currentstroke}%
\pgfsetdash{}{0pt}%
\pgfpathmoveto{\pgfqpoint{3.839490in}{2.833104in}}%
\pgfpathlineto{\pgfqpoint{3.866437in}{2.817546in}}%
\pgfpathlineto{\pgfqpoint{3.866437in}{2.848662in}}%
\pgfpathlineto{\pgfqpoint{3.839490in}{2.864220in}}%
\pgfpathlineto{\pgfqpoint{3.839490in}{2.833104in}}%
\pgfpathclose%
\pgfusepath{fill}%
\end{pgfscope}%
\begin{pgfscope}%
\pgfpathrectangle{\pgfqpoint{0.765000in}{0.660000in}}{\pgfqpoint{4.620000in}{4.620000in}}%
\pgfusepath{clip}%
\pgfsetbuttcap%
\pgfsetroundjoin%
\definecolor{currentfill}{rgb}{1.000000,0.894118,0.788235}%
\pgfsetfillcolor{currentfill}%
\pgfsetlinewidth{0.000000pt}%
\definecolor{currentstroke}{rgb}{1.000000,0.894118,0.788235}%
\pgfsetstrokecolor{currentstroke}%
\pgfsetdash{}{0pt}%
\pgfpathmoveto{\pgfqpoint{3.755404in}{2.857087in}}%
\pgfpathlineto{\pgfqpoint{3.728456in}{2.841529in}}%
\pgfpathlineto{\pgfqpoint{3.755404in}{2.825971in}}%
\pgfpathlineto{\pgfqpoint{3.782351in}{2.841529in}}%
\pgfpathlineto{\pgfqpoint{3.755404in}{2.857087in}}%
\pgfpathclose%
\pgfusepath{fill}%
\end{pgfscope}%
\begin{pgfscope}%
\pgfpathrectangle{\pgfqpoint{0.765000in}{0.660000in}}{\pgfqpoint{4.620000in}{4.620000in}}%
\pgfusepath{clip}%
\pgfsetbuttcap%
\pgfsetroundjoin%
\definecolor{currentfill}{rgb}{1.000000,0.894118,0.788235}%
\pgfsetfillcolor{currentfill}%
\pgfsetlinewidth{0.000000pt}%
\definecolor{currentstroke}{rgb}{1.000000,0.894118,0.788235}%
\pgfsetstrokecolor{currentstroke}%
\pgfsetdash{}{0pt}%
\pgfpathmoveto{\pgfqpoint{3.755404in}{2.794855in}}%
\pgfpathlineto{\pgfqpoint{3.782351in}{2.810413in}}%
\pgfpathlineto{\pgfqpoint{3.782351in}{2.841529in}}%
\pgfpathlineto{\pgfqpoint{3.755404in}{2.825971in}}%
\pgfpathlineto{\pgfqpoint{3.755404in}{2.794855in}}%
\pgfpathclose%
\pgfusepath{fill}%
\end{pgfscope}%
\begin{pgfscope}%
\pgfpathrectangle{\pgfqpoint{0.765000in}{0.660000in}}{\pgfqpoint{4.620000in}{4.620000in}}%
\pgfusepath{clip}%
\pgfsetbuttcap%
\pgfsetroundjoin%
\definecolor{currentfill}{rgb}{1.000000,0.894118,0.788235}%
\pgfsetfillcolor{currentfill}%
\pgfsetlinewidth{0.000000pt}%
\definecolor{currentstroke}{rgb}{1.000000,0.894118,0.788235}%
\pgfsetstrokecolor{currentstroke}%
\pgfsetdash{}{0pt}%
\pgfpathmoveto{\pgfqpoint{3.728456in}{2.810413in}}%
\pgfpathlineto{\pgfqpoint{3.755404in}{2.794855in}}%
\pgfpathlineto{\pgfqpoint{3.755404in}{2.825971in}}%
\pgfpathlineto{\pgfqpoint{3.728456in}{2.841529in}}%
\pgfpathlineto{\pgfqpoint{3.728456in}{2.810413in}}%
\pgfpathclose%
\pgfusepath{fill}%
\end{pgfscope}%
\begin{pgfscope}%
\pgfpathrectangle{\pgfqpoint{0.765000in}{0.660000in}}{\pgfqpoint{4.620000in}{4.620000in}}%
\pgfusepath{clip}%
\pgfsetbuttcap%
\pgfsetroundjoin%
\definecolor{currentfill}{rgb}{1.000000,0.894118,0.788235}%
\pgfsetfillcolor{currentfill}%
\pgfsetlinewidth{0.000000pt}%
\definecolor{currentstroke}{rgb}{1.000000,0.894118,0.788235}%
\pgfsetstrokecolor{currentstroke}%
\pgfsetdash{}{0pt}%
\pgfpathmoveto{\pgfqpoint{3.839490in}{2.864220in}}%
\pgfpathlineto{\pgfqpoint{3.812542in}{2.848662in}}%
\pgfpathlineto{\pgfqpoint{3.839490in}{2.833104in}}%
\pgfpathlineto{\pgfqpoint{3.866437in}{2.848662in}}%
\pgfpathlineto{\pgfqpoint{3.839490in}{2.864220in}}%
\pgfpathclose%
\pgfusepath{fill}%
\end{pgfscope}%
\begin{pgfscope}%
\pgfpathrectangle{\pgfqpoint{0.765000in}{0.660000in}}{\pgfqpoint{4.620000in}{4.620000in}}%
\pgfusepath{clip}%
\pgfsetbuttcap%
\pgfsetroundjoin%
\definecolor{currentfill}{rgb}{1.000000,0.894118,0.788235}%
\pgfsetfillcolor{currentfill}%
\pgfsetlinewidth{0.000000pt}%
\definecolor{currentstroke}{rgb}{1.000000,0.894118,0.788235}%
\pgfsetstrokecolor{currentstroke}%
\pgfsetdash{}{0pt}%
\pgfpathmoveto{\pgfqpoint{3.839490in}{2.801988in}}%
\pgfpathlineto{\pgfqpoint{3.866437in}{2.817546in}}%
\pgfpathlineto{\pgfqpoint{3.866437in}{2.848662in}}%
\pgfpathlineto{\pgfqpoint{3.839490in}{2.833104in}}%
\pgfpathlineto{\pgfqpoint{3.839490in}{2.801988in}}%
\pgfpathclose%
\pgfusepath{fill}%
\end{pgfscope}%
\begin{pgfscope}%
\pgfpathrectangle{\pgfqpoint{0.765000in}{0.660000in}}{\pgfqpoint{4.620000in}{4.620000in}}%
\pgfusepath{clip}%
\pgfsetbuttcap%
\pgfsetroundjoin%
\definecolor{currentfill}{rgb}{1.000000,0.894118,0.788235}%
\pgfsetfillcolor{currentfill}%
\pgfsetlinewidth{0.000000pt}%
\definecolor{currentstroke}{rgb}{1.000000,0.894118,0.788235}%
\pgfsetstrokecolor{currentstroke}%
\pgfsetdash{}{0pt}%
\pgfpathmoveto{\pgfqpoint{3.812542in}{2.817546in}}%
\pgfpathlineto{\pgfqpoint{3.839490in}{2.801988in}}%
\pgfpathlineto{\pgfqpoint{3.839490in}{2.833104in}}%
\pgfpathlineto{\pgfqpoint{3.812542in}{2.848662in}}%
\pgfpathlineto{\pgfqpoint{3.812542in}{2.817546in}}%
\pgfpathclose%
\pgfusepath{fill}%
\end{pgfscope}%
\begin{pgfscope}%
\pgfpathrectangle{\pgfqpoint{0.765000in}{0.660000in}}{\pgfqpoint{4.620000in}{4.620000in}}%
\pgfusepath{clip}%
\pgfsetbuttcap%
\pgfsetroundjoin%
\definecolor{currentfill}{rgb}{1.000000,0.894118,0.788235}%
\pgfsetfillcolor{currentfill}%
\pgfsetlinewidth{0.000000pt}%
\definecolor{currentstroke}{rgb}{1.000000,0.894118,0.788235}%
\pgfsetstrokecolor{currentstroke}%
\pgfsetdash{}{0pt}%
\pgfpathmoveto{\pgfqpoint{3.755404in}{2.825971in}}%
\pgfpathlineto{\pgfqpoint{3.728456in}{2.810413in}}%
\pgfpathlineto{\pgfqpoint{3.812542in}{2.817546in}}%
\pgfpathlineto{\pgfqpoint{3.839490in}{2.833104in}}%
\pgfpathlineto{\pgfqpoint{3.755404in}{2.825971in}}%
\pgfpathclose%
\pgfusepath{fill}%
\end{pgfscope}%
\begin{pgfscope}%
\pgfpathrectangle{\pgfqpoint{0.765000in}{0.660000in}}{\pgfqpoint{4.620000in}{4.620000in}}%
\pgfusepath{clip}%
\pgfsetbuttcap%
\pgfsetroundjoin%
\definecolor{currentfill}{rgb}{1.000000,0.894118,0.788235}%
\pgfsetfillcolor{currentfill}%
\pgfsetlinewidth{0.000000pt}%
\definecolor{currentstroke}{rgb}{1.000000,0.894118,0.788235}%
\pgfsetstrokecolor{currentstroke}%
\pgfsetdash{}{0pt}%
\pgfpathmoveto{\pgfqpoint{3.782351in}{2.810413in}}%
\pgfpathlineto{\pgfqpoint{3.755404in}{2.825971in}}%
\pgfpathlineto{\pgfqpoint{3.839490in}{2.833104in}}%
\pgfpathlineto{\pgfqpoint{3.866437in}{2.817546in}}%
\pgfpathlineto{\pgfqpoint{3.782351in}{2.810413in}}%
\pgfpathclose%
\pgfusepath{fill}%
\end{pgfscope}%
\begin{pgfscope}%
\pgfpathrectangle{\pgfqpoint{0.765000in}{0.660000in}}{\pgfqpoint{4.620000in}{4.620000in}}%
\pgfusepath{clip}%
\pgfsetbuttcap%
\pgfsetroundjoin%
\definecolor{currentfill}{rgb}{1.000000,0.894118,0.788235}%
\pgfsetfillcolor{currentfill}%
\pgfsetlinewidth{0.000000pt}%
\definecolor{currentstroke}{rgb}{1.000000,0.894118,0.788235}%
\pgfsetstrokecolor{currentstroke}%
\pgfsetdash{}{0pt}%
\pgfpathmoveto{\pgfqpoint{3.755404in}{2.825971in}}%
\pgfpathlineto{\pgfqpoint{3.755404in}{2.857087in}}%
\pgfpathlineto{\pgfqpoint{3.839490in}{2.864220in}}%
\pgfpathlineto{\pgfqpoint{3.866437in}{2.817546in}}%
\pgfpathlineto{\pgfqpoint{3.755404in}{2.825971in}}%
\pgfpathclose%
\pgfusepath{fill}%
\end{pgfscope}%
\begin{pgfscope}%
\pgfpathrectangle{\pgfqpoint{0.765000in}{0.660000in}}{\pgfqpoint{4.620000in}{4.620000in}}%
\pgfusepath{clip}%
\pgfsetbuttcap%
\pgfsetroundjoin%
\definecolor{currentfill}{rgb}{1.000000,0.894118,0.788235}%
\pgfsetfillcolor{currentfill}%
\pgfsetlinewidth{0.000000pt}%
\definecolor{currentstroke}{rgb}{1.000000,0.894118,0.788235}%
\pgfsetstrokecolor{currentstroke}%
\pgfsetdash{}{0pt}%
\pgfpathmoveto{\pgfqpoint{3.782351in}{2.810413in}}%
\pgfpathlineto{\pgfqpoint{3.782351in}{2.841529in}}%
\pgfpathlineto{\pgfqpoint{3.866437in}{2.848662in}}%
\pgfpathlineto{\pgfqpoint{3.839490in}{2.833104in}}%
\pgfpathlineto{\pgfqpoint{3.782351in}{2.810413in}}%
\pgfpathclose%
\pgfusepath{fill}%
\end{pgfscope}%
\begin{pgfscope}%
\pgfpathrectangle{\pgfqpoint{0.765000in}{0.660000in}}{\pgfqpoint{4.620000in}{4.620000in}}%
\pgfusepath{clip}%
\pgfsetbuttcap%
\pgfsetroundjoin%
\definecolor{currentfill}{rgb}{1.000000,0.894118,0.788235}%
\pgfsetfillcolor{currentfill}%
\pgfsetlinewidth{0.000000pt}%
\definecolor{currentstroke}{rgb}{1.000000,0.894118,0.788235}%
\pgfsetstrokecolor{currentstroke}%
\pgfsetdash{}{0pt}%
\pgfpathmoveto{\pgfqpoint{3.728456in}{2.810413in}}%
\pgfpathlineto{\pgfqpoint{3.755404in}{2.794855in}}%
\pgfpathlineto{\pgfqpoint{3.839490in}{2.801988in}}%
\pgfpathlineto{\pgfqpoint{3.812542in}{2.817546in}}%
\pgfpathlineto{\pgfqpoint{3.728456in}{2.810413in}}%
\pgfpathclose%
\pgfusepath{fill}%
\end{pgfscope}%
\begin{pgfscope}%
\pgfpathrectangle{\pgfqpoint{0.765000in}{0.660000in}}{\pgfqpoint{4.620000in}{4.620000in}}%
\pgfusepath{clip}%
\pgfsetbuttcap%
\pgfsetroundjoin%
\definecolor{currentfill}{rgb}{1.000000,0.894118,0.788235}%
\pgfsetfillcolor{currentfill}%
\pgfsetlinewidth{0.000000pt}%
\definecolor{currentstroke}{rgb}{1.000000,0.894118,0.788235}%
\pgfsetstrokecolor{currentstroke}%
\pgfsetdash{}{0pt}%
\pgfpathmoveto{\pgfqpoint{3.755404in}{2.794855in}}%
\pgfpathlineto{\pgfqpoint{3.782351in}{2.810413in}}%
\pgfpathlineto{\pgfqpoint{3.866437in}{2.817546in}}%
\pgfpathlineto{\pgfqpoint{3.839490in}{2.801988in}}%
\pgfpathlineto{\pgfqpoint{3.755404in}{2.794855in}}%
\pgfpathclose%
\pgfusepath{fill}%
\end{pgfscope}%
\begin{pgfscope}%
\pgfpathrectangle{\pgfqpoint{0.765000in}{0.660000in}}{\pgfqpoint{4.620000in}{4.620000in}}%
\pgfusepath{clip}%
\pgfsetbuttcap%
\pgfsetroundjoin%
\definecolor{currentfill}{rgb}{1.000000,0.894118,0.788235}%
\pgfsetfillcolor{currentfill}%
\pgfsetlinewidth{0.000000pt}%
\definecolor{currentstroke}{rgb}{1.000000,0.894118,0.788235}%
\pgfsetstrokecolor{currentstroke}%
\pgfsetdash{}{0pt}%
\pgfpathmoveto{\pgfqpoint{3.755404in}{2.857087in}}%
\pgfpathlineto{\pgfqpoint{3.728456in}{2.841529in}}%
\pgfpathlineto{\pgfqpoint{3.812542in}{2.848662in}}%
\pgfpathlineto{\pgfqpoint{3.839490in}{2.864220in}}%
\pgfpathlineto{\pgfqpoint{3.755404in}{2.857087in}}%
\pgfpathclose%
\pgfusepath{fill}%
\end{pgfscope}%
\begin{pgfscope}%
\pgfpathrectangle{\pgfqpoint{0.765000in}{0.660000in}}{\pgfqpoint{4.620000in}{4.620000in}}%
\pgfusepath{clip}%
\pgfsetbuttcap%
\pgfsetroundjoin%
\definecolor{currentfill}{rgb}{1.000000,0.894118,0.788235}%
\pgfsetfillcolor{currentfill}%
\pgfsetlinewidth{0.000000pt}%
\definecolor{currentstroke}{rgb}{1.000000,0.894118,0.788235}%
\pgfsetstrokecolor{currentstroke}%
\pgfsetdash{}{0pt}%
\pgfpathmoveto{\pgfqpoint{3.782351in}{2.841529in}}%
\pgfpathlineto{\pgfqpoint{3.755404in}{2.857087in}}%
\pgfpathlineto{\pgfqpoint{3.839490in}{2.864220in}}%
\pgfpathlineto{\pgfqpoint{3.866437in}{2.848662in}}%
\pgfpathlineto{\pgfqpoint{3.782351in}{2.841529in}}%
\pgfpathclose%
\pgfusepath{fill}%
\end{pgfscope}%
\begin{pgfscope}%
\pgfpathrectangle{\pgfqpoint{0.765000in}{0.660000in}}{\pgfqpoint{4.620000in}{4.620000in}}%
\pgfusepath{clip}%
\pgfsetbuttcap%
\pgfsetroundjoin%
\definecolor{currentfill}{rgb}{1.000000,0.894118,0.788235}%
\pgfsetfillcolor{currentfill}%
\pgfsetlinewidth{0.000000pt}%
\definecolor{currentstroke}{rgb}{1.000000,0.894118,0.788235}%
\pgfsetstrokecolor{currentstroke}%
\pgfsetdash{}{0pt}%
\pgfpathmoveto{\pgfqpoint{3.728456in}{2.810413in}}%
\pgfpathlineto{\pgfqpoint{3.728456in}{2.841529in}}%
\pgfpathlineto{\pgfqpoint{3.812542in}{2.848662in}}%
\pgfpathlineto{\pgfqpoint{3.839490in}{2.801988in}}%
\pgfpathlineto{\pgfqpoint{3.728456in}{2.810413in}}%
\pgfpathclose%
\pgfusepath{fill}%
\end{pgfscope}%
\begin{pgfscope}%
\pgfpathrectangle{\pgfqpoint{0.765000in}{0.660000in}}{\pgfqpoint{4.620000in}{4.620000in}}%
\pgfusepath{clip}%
\pgfsetbuttcap%
\pgfsetroundjoin%
\definecolor{currentfill}{rgb}{1.000000,0.894118,0.788235}%
\pgfsetfillcolor{currentfill}%
\pgfsetlinewidth{0.000000pt}%
\definecolor{currentstroke}{rgb}{1.000000,0.894118,0.788235}%
\pgfsetstrokecolor{currentstroke}%
\pgfsetdash{}{0pt}%
\pgfpathmoveto{\pgfqpoint{3.755404in}{2.794855in}}%
\pgfpathlineto{\pgfqpoint{3.755404in}{2.825971in}}%
\pgfpathlineto{\pgfqpoint{3.839490in}{2.833104in}}%
\pgfpathlineto{\pgfqpoint{3.812542in}{2.817546in}}%
\pgfpathlineto{\pgfqpoint{3.755404in}{2.794855in}}%
\pgfpathclose%
\pgfusepath{fill}%
\end{pgfscope}%
\begin{pgfscope}%
\pgfpathrectangle{\pgfqpoint{0.765000in}{0.660000in}}{\pgfqpoint{4.620000in}{4.620000in}}%
\pgfusepath{clip}%
\pgfsetbuttcap%
\pgfsetroundjoin%
\definecolor{currentfill}{rgb}{1.000000,0.894118,0.788235}%
\pgfsetfillcolor{currentfill}%
\pgfsetlinewidth{0.000000pt}%
\definecolor{currentstroke}{rgb}{1.000000,0.894118,0.788235}%
\pgfsetstrokecolor{currentstroke}%
\pgfsetdash{}{0pt}%
\pgfpathmoveto{\pgfqpoint{3.755404in}{2.825971in}}%
\pgfpathlineto{\pgfqpoint{3.782351in}{2.841529in}}%
\pgfpathlineto{\pgfqpoint{3.866437in}{2.848662in}}%
\pgfpathlineto{\pgfqpoint{3.839490in}{2.833104in}}%
\pgfpathlineto{\pgfqpoint{3.755404in}{2.825971in}}%
\pgfpathclose%
\pgfusepath{fill}%
\end{pgfscope}%
\begin{pgfscope}%
\pgfpathrectangle{\pgfqpoint{0.765000in}{0.660000in}}{\pgfqpoint{4.620000in}{4.620000in}}%
\pgfusepath{clip}%
\pgfsetbuttcap%
\pgfsetroundjoin%
\definecolor{currentfill}{rgb}{1.000000,0.894118,0.788235}%
\pgfsetfillcolor{currentfill}%
\pgfsetlinewidth{0.000000pt}%
\definecolor{currentstroke}{rgb}{1.000000,0.894118,0.788235}%
\pgfsetstrokecolor{currentstroke}%
\pgfsetdash{}{0pt}%
\pgfpathmoveto{\pgfqpoint{3.728456in}{2.841529in}}%
\pgfpathlineto{\pgfqpoint{3.755404in}{2.825971in}}%
\pgfpathlineto{\pgfqpoint{3.839490in}{2.833104in}}%
\pgfpathlineto{\pgfqpoint{3.812542in}{2.848662in}}%
\pgfpathlineto{\pgfqpoint{3.728456in}{2.841529in}}%
\pgfpathclose%
\pgfusepath{fill}%
\end{pgfscope}%
\begin{pgfscope}%
\pgfpathrectangle{\pgfqpoint{0.765000in}{0.660000in}}{\pgfqpoint{4.620000in}{4.620000in}}%
\pgfusepath{clip}%
\pgfsetbuttcap%
\pgfsetroundjoin%
\definecolor{currentfill}{rgb}{1.000000,0.894118,0.788235}%
\pgfsetfillcolor{currentfill}%
\pgfsetlinewidth{0.000000pt}%
\definecolor{currentstroke}{rgb}{1.000000,0.894118,0.788235}%
\pgfsetstrokecolor{currentstroke}%
\pgfsetdash{}{0pt}%
\pgfpathmoveto{\pgfqpoint{3.755404in}{2.825971in}}%
\pgfpathlineto{\pgfqpoint{3.728456in}{2.810413in}}%
\pgfpathlineto{\pgfqpoint{3.755404in}{2.794855in}}%
\pgfpathlineto{\pgfqpoint{3.782351in}{2.810413in}}%
\pgfpathlineto{\pgfqpoint{3.755404in}{2.825971in}}%
\pgfpathclose%
\pgfusepath{fill}%
\end{pgfscope}%
\begin{pgfscope}%
\pgfpathrectangle{\pgfqpoint{0.765000in}{0.660000in}}{\pgfqpoint{4.620000in}{4.620000in}}%
\pgfusepath{clip}%
\pgfsetbuttcap%
\pgfsetroundjoin%
\definecolor{currentfill}{rgb}{1.000000,0.894118,0.788235}%
\pgfsetfillcolor{currentfill}%
\pgfsetlinewidth{0.000000pt}%
\definecolor{currentstroke}{rgb}{1.000000,0.894118,0.788235}%
\pgfsetstrokecolor{currentstroke}%
\pgfsetdash{}{0pt}%
\pgfpathmoveto{\pgfqpoint{3.755404in}{2.825971in}}%
\pgfpathlineto{\pgfqpoint{3.728456in}{2.810413in}}%
\pgfpathlineto{\pgfqpoint{3.728456in}{2.841529in}}%
\pgfpathlineto{\pgfqpoint{3.755404in}{2.857087in}}%
\pgfpathlineto{\pgfqpoint{3.755404in}{2.825971in}}%
\pgfpathclose%
\pgfusepath{fill}%
\end{pgfscope}%
\begin{pgfscope}%
\pgfpathrectangle{\pgfqpoint{0.765000in}{0.660000in}}{\pgfqpoint{4.620000in}{4.620000in}}%
\pgfusepath{clip}%
\pgfsetbuttcap%
\pgfsetroundjoin%
\definecolor{currentfill}{rgb}{1.000000,0.894118,0.788235}%
\pgfsetfillcolor{currentfill}%
\pgfsetlinewidth{0.000000pt}%
\definecolor{currentstroke}{rgb}{1.000000,0.894118,0.788235}%
\pgfsetstrokecolor{currentstroke}%
\pgfsetdash{}{0pt}%
\pgfpathmoveto{\pgfqpoint{3.755404in}{2.825971in}}%
\pgfpathlineto{\pgfqpoint{3.782351in}{2.810413in}}%
\pgfpathlineto{\pgfqpoint{3.782351in}{2.841529in}}%
\pgfpathlineto{\pgfqpoint{3.755404in}{2.857087in}}%
\pgfpathlineto{\pgfqpoint{3.755404in}{2.825971in}}%
\pgfpathclose%
\pgfusepath{fill}%
\end{pgfscope}%
\begin{pgfscope}%
\pgfpathrectangle{\pgfqpoint{0.765000in}{0.660000in}}{\pgfqpoint{4.620000in}{4.620000in}}%
\pgfusepath{clip}%
\pgfsetbuttcap%
\pgfsetroundjoin%
\definecolor{currentfill}{rgb}{1.000000,0.894118,0.788235}%
\pgfsetfillcolor{currentfill}%
\pgfsetlinewidth{0.000000pt}%
\definecolor{currentstroke}{rgb}{1.000000,0.894118,0.788235}%
\pgfsetstrokecolor{currentstroke}%
\pgfsetdash{}{0pt}%
\pgfpathmoveto{\pgfqpoint{3.717641in}{2.829213in}}%
\pgfpathlineto{\pgfqpoint{3.690694in}{2.813655in}}%
\pgfpathlineto{\pgfqpoint{3.717641in}{2.798097in}}%
\pgfpathlineto{\pgfqpoint{3.744588in}{2.813655in}}%
\pgfpathlineto{\pgfqpoint{3.717641in}{2.829213in}}%
\pgfpathclose%
\pgfusepath{fill}%
\end{pgfscope}%
\begin{pgfscope}%
\pgfpathrectangle{\pgfqpoint{0.765000in}{0.660000in}}{\pgfqpoint{4.620000in}{4.620000in}}%
\pgfusepath{clip}%
\pgfsetbuttcap%
\pgfsetroundjoin%
\definecolor{currentfill}{rgb}{1.000000,0.894118,0.788235}%
\pgfsetfillcolor{currentfill}%
\pgfsetlinewidth{0.000000pt}%
\definecolor{currentstroke}{rgb}{1.000000,0.894118,0.788235}%
\pgfsetstrokecolor{currentstroke}%
\pgfsetdash{}{0pt}%
\pgfpathmoveto{\pgfqpoint{3.717641in}{2.829213in}}%
\pgfpathlineto{\pgfqpoint{3.690694in}{2.813655in}}%
\pgfpathlineto{\pgfqpoint{3.690694in}{2.844771in}}%
\pgfpathlineto{\pgfqpoint{3.717641in}{2.860329in}}%
\pgfpathlineto{\pgfqpoint{3.717641in}{2.829213in}}%
\pgfpathclose%
\pgfusepath{fill}%
\end{pgfscope}%
\begin{pgfscope}%
\pgfpathrectangle{\pgfqpoint{0.765000in}{0.660000in}}{\pgfqpoint{4.620000in}{4.620000in}}%
\pgfusepath{clip}%
\pgfsetbuttcap%
\pgfsetroundjoin%
\definecolor{currentfill}{rgb}{1.000000,0.894118,0.788235}%
\pgfsetfillcolor{currentfill}%
\pgfsetlinewidth{0.000000pt}%
\definecolor{currentstroke}{rgb}{1.000000,0.894118,0.788235}%
\pgfsetstrokecolor{currentstroke}%
\pgfsetdash{}{0pt}%
\pgfpathmoveto{\pgfqpoint{3.717641in}{2.829213in}}%
\pgfpathlineto{\pgfqpoint{3.744588in}{2.813655in}}%
\pgfpathlineto{\pgfqpoint{3.744588in}{2.844771in}}%
\pgfpathlineto{\pgfqpoint{3.717641in}{2.860329in}}%
\pgfpathlineto{\pgfqpoint{3.717641in}{2.829213in}}%
\pgfpathclose%
\pgfusepath{fill}%
\end{pgfscope}%
\begin{pgfscope}%
\pgfpathrectangle{\pgfqpoint{0.765000in}{0.660000in}}{\pgfqpoint{4.620000in}{4.620000in}}%
\pgfusepath{clip}%
\pgfsetbuttcap%
\pgfsetroundjoin%
\definecolor{currentfill}{rgb}{1.000000,0.894118,0.788235}%
\pgfsetfillcolor{currentfill}%
\pgfsetlinewidth{0.000000pt}%
\definecolor{currentstroke}{rgb}{1.000000,0.894118,0.788235}%
\pgfsetstrokecolor{currentstroke}%
\pgfsetdash{}{0pt}%
\pgfpathmoveto{\pgfqpoint{3.755404in}{2.857087in}}%
\pgfpathlineto{\pgfqpoint{3.728456in}{2.841529in}}%
\pgfpathlineto{\pgfqpoint{3.755404in}{2.825971in}}%
\pgfpathlineto{\pgfqpoint{3.782351in}{2.841529in}}%
\pgfpathlineto{\pgfqpoint{3.755404in}{2.857087in}}%
\pgfpathclose%
\pgfusepath{fill}%
\end{pgfscope}%
\begin{pgfscope}%
\pgfpathrectangle{\pgfqpoint{0.765000in}{0.660000in}}{\pgfqpoint{4.620000in}{4.620000in}}%
\pgfusepath{clip}%
\pgfsetbuttcap%
\pgfsetroundjoin%
\definecolor{currentfill}{rgb}{1.000000,0.894118,0.788235}%
\pgfsetfillcolor{currentfill}%
\pgfsetlinewidth{0.000000pt}%
\definecolor{currentstroke}{rgb}{1.000000,0.894118,0.788235}%
\pgfsetstrokecolor{currentstroke}%
\pgfsetdash{}{0pt}%
\pgfpathmoveto{\pgfqpoint{3.755404in}{2.794855in}}%
\pgfpathlineto{\pgfqpoint{3.782351in}{2.810413in}}%
\pgfpathlineto{\pgfqpoint{3.782351in}{2.841529in}}%
\pgfpathlineto{\pgfqpoint{3.755404in}{2.825971in}}%
\pgfpathlineto{\pgfqpoint{3.755404in}{2.794855in}}%
\pgfpathclose%
\pgfusepath{fill}%
\end{pgfscope}%
\begin{pgfscope}%
\pgfpathrectangle{\pgfqpoint{0.765000in}{0.660000in}}{\pgfqpoint{4.620000in}{4.620000in}}%
\pgfusepath{clip}%
\pgfsetbuttcap%
\pgfsetroundjoin%
\definecolor{currentfill}{rgb}{1.000000,0.894118,0.788235}%
\pgfsetfillcolor{currentfill}%
\pgfsetlinewidth{0.000000pt}%
\definecolor{currentstroke}{rgb}{1.000000,0.894118,0.788235}%
\pgfsetstrokecolor{currentstroke}%
\pgfsetdash{}{0pt}%
\pgfpathmoveto{\pgfqpoint{3.728456in}{2.810413in}}%
\pgfpathlineto{\pgfqpoint{3.755404in}{2.794855in}}%
\pgfpathlineto{\pgfqpoint{3.755404in}{2.825971in}}%
\pgfpathlineto{\pgfqpoint{3.728456in}{2.841529in}}%
\pgfpathlineto{\pgfqpoint{3.728456in}{2.810413in}}%
\pgfpathclose%
\pgfusepath{fill}%
\end{pgfscope}%
\begin{pgfscope}%
\pgfpathrectangle{\pgfqpoint{0.765000in}{0.660000in}}{\pgfqpoint{4.620000in}{4.620000in}}%
\pgfusepath{clip}%
\pgfsetbuttcap%
\pgfsetroundjoin%
\definecolor{currentfill}{rgb}{1.000000,0.894118,0.788235}%
\pgfsetfillcolor{currentfill}%
\pgfsetlinewidth{0.000000pt}%
\definecolor{currentstroke}{rgb}{1.000000,0.894118,0.788235}%
\pgfsetstrokecolor{currentstroke}%
\pgfsetdash{}{0pt}%
\pgfpathmoveto{\pgfqpoint{3.717641in}{2.860329in}}%
\pgfpathlineto{\pgfqpoint{3.690694in}{2.844771in}}%
\pgfpathlineto{\pgfqpoint{3.717641in}{2.829213in}}%
\pgfpathlineto{\pgfqpoint{3.744588in}{2.844771in}}%
\pgfpathlineto{\pgfqpoint{3.717641in}{2.860329in}}%
\pgfpathclose%
\pgfusepath{fill}%
\end{pgfscope}%
\begin{pgfscope}%
\pgfpathrectangle{\pgfqpoint{0.765000in}{0.660000in}}{\pgfqpoint{4.620000in}{4.620000in}}%
\pgfusepath{clip}%
\pgfsetbuttcap%
\pgfsetroundjoin%
\definecolor{currentfill}{rgb}{1.000000,0.894118,0.788235}%
\pgfsetfillcolor{currentfill}%
\pgfsetlinewidth{0.000000pt}%
\definecolor{currentstroke}{rgb}{1.000000,0.894118,0.788235}%
\pgfsetstrokecolor{currentstroke}%
\pgfsetdash{}{0pt}%
\pgfpathmoveto{\pgfqpoint{3.717641in}{2.798097in}}%
\pgfpathlineto{\pgfqpoint{3.744588in}{2.813655in}}%
\pgfpathlineto{\pgfqpoint{3.744588in}{2.844771in}}%
\pgfpathlineto{\pgfqpoint{3.717641in}{2.829213in}}%
\pgfpathlineto{\pgfqpoint{3.717641in}{2.798097in}}%
\pgfpathclose%
\pgfusepath{fill}%
\end{pgfscope}%
\begin{pgfscope}%
\pgfpathrectangle{\pgfqpoint{0.765000in}{0.660000in}}{\pgfqpoint{4.620000in}{4.620000in}}%
\pgfusepath{clip}%
\pgfsetbuttcap%
\pgfsetroundjoin%
\definecolor{currentfill}{rgb}{1.000000,0.894118,0.788235}%
\pgfsetfillcolor{currentfill}%
\pgfsetlinewidth{0.000000pt}%
\definecolor{currentstroke}{rgb}{1.000000,0.894118,0.788235}%
\pgfsetstrokecolor{currentstroke}%
\pgfsetdash{}{0pt}%
\pgfpathmoveto{\pgfqpoint{3.690694in}{2.813655in}}%
\pgfpathlineto{\pgfqpoint{3.717641in}{2.798097in}}%
\pgfpathlineto{\pgfqpoint{3.717641in}{2.829213in}}%
\pgfpathlineto{\pgfqpoint{3.690694in}{2.844771in}}%
\pgfpathlineto{\pgfqpoint{3.690694in}{2.813655in}}%
\pgfpathclose%
\pgfusepath{fill}%
\end{pgfscope}%
\begin{pgfscope}%
\pgfpathrectangle{\pgfqpoint{0.765000in}{0.660000in}}{\pgfqpoint{4.620000in}{4.620000in}}%
\pgfusepath{clip}%
\pgfsetbuttcap%
\pgfsetroundjoin%
\definecolor{currentfill}{rgb}{1.000000,0.894118,0.788235}%
\pgfsetfillcolor{currentfill}%
\pgfsetlinewidth{0.000000pt}%
\definecolor{currentstroke}{rgb}{1.000000,0.894118,0.788235}%
\pgfsetstrokecolor{currentstroke}%
\pgfsetdash{}{0pt}%
\pgfpathmoveto{\pgfqpoint{3.755404in}{2.825971in}}%
\pgfpathlineto{\pgfqpoint{3.728456in}{2.810413in}}%
\pgfpathlineto{\pgfqpoint{3.690694in}{2.813655in}}%
\pgfpathlineto{\pgfqpoint{3.717641in}{2.829213in}}%
\pgfpathlineto{\pgfqpoint{3.755404in}{2.825971in}}%
\pgfpathclose%
\pgfusepath{fill}%
\end{pgfscope}%
\begin{pgfscope}%
\pgfpathrectangle{\pgfqpoint{0.765000in}{0.660000in}}{\pgfqpoint{4.620000in}{4.620000in}}%
\pgfusepath{clip}%
\pgfsetbuttcap%
\pgfsetroundjoin%
\definecolor{currentfill}{rgb}{1.000000,0.894118,0.788235}%
\pgfsetfillcolor{currentfill}%
\pgfsetlinewidth{0.000000pt}%
\definecolor{currentstroke}{rgb}{1.000000,0.894118,0.788235}%
\pgfsetstrokecolor{currentstroke}%
\pgfsetdash{}{0pt}%
\pgfpathmoveto{\pgfqpoint{3.782351in}{2.810413in}}%
\pgfpathlineto{\pgfqpoint{3.755404in}{2.825971in}}%
\pgfpathlineto{\pgfqpoint{3.717641in}{2.829213in}}%
\pgfpathlineto{\pgfqpoint{3.744588in}{2.813655in}}%
\pgfpathlineto{\pgfqpoint{3.782351in}{2.810413in}}%
\pgfpathclose%
\pgfusepath{fill}%
\end{pgfscope}%
\begin{pgfscope}%
\pgfpathrectangle{\pgfqpoint{0.765000in}{0.660000in}}{\pgfqpoint{4.620000in}{4.620000in}}%
\pgfusepath{clip}%
\pgfsetbuttcap%
\pgfsetroundjoin%
\definecolor{currentfill}{rgb}{1.000000,0.894118,0.788235}%
\pgfsetfillcolor{currentfill}%
\pgfsetlinewidth{0.000000pt}%
\definecolor{currentstroke}{rgb}{1.000000,0.894118,0.788235}%
\pgfsetstrokecolor{currentstroke}%
\pgfsetdash{}{0pt}%
\pgfpathmoveto{\pgfqpoint{3.755404in}{2.825971in}}%
\pgfpathlineto{\pgfqpoint{3.755404in}{2.857087in}}%
\pgfpathlineto{\pgfqpoint{3.717641in}{2.860329in}}%
\pgfpathlineto{\pgfqpoint{3.744588in}{2.813655in}}%
\pgfpathlineto{\pgfqpoint{3.755404in}{2.825971in}}%
\pgfpathclose%
\pgfusepath{fill}%
\end{pgfscope}%
\begin{pgfscope}%
\pgfpathrectangle{\pgfqpoint{0.765000in}{0.660000in}}{\pgfqpoint{4.620000in}{4.620000in}}%
\pgfusepath{clip}%
\pgfsetbuttcap%
\pgfsetroundjoin%
\definecolor{currentfill}{rgb}{1.000000,0.894118,0.788235}%
\pgfsetfillcolor{currentfill}%
\pgfsetlinewidth{0.000000pt}%
\definecolor{currentstroke}{rgb}{1.000000,0.894118,0.788235}%
\pgfsetstrokecolor{currentstroke}%
\pgfsetdash{}{0pt}%
\pgfpathmoveto{\pgfqpoint{3.782351in}{2.810413in}}%
\pgfpathlineto{\pgfqpoint{3.782351in}{2.841529in}}%
\pgfpathlineto{\pgfqpoint{3.744588in}{2.844771in}}%
\pgfpathlineto{\pgfqpoint{3.717641in}{2.829213in}}%
\pgfpathlineto{\pgfqpoint{3.782351in}{2.810413in}}%
\pgfpathclose%
\pgfusepath{fill}%
\end{pgfscope}%
\begin{pgfscope}%
\pgfpathrectangle{\pgfqpoint{0.765000in}{0.660000in}}{\pgfqpoint{4.620000in}{4.620000in}}%
\pgfusepath{clip}%
\pgfsetbuttcap%
\pgfsetroundjoin%
\definecolor{currentfill}{rgb}{1.000000,0.894118,0.788235}%
\pgfsetfillcolor{currentfill}%
\pgfsetlinewidth{0.000000pt}%
\definecolor{currentstroke}{rgb}{1.000000,0.894118,0.788235}%
\pgfsetstrokecolor{currentstroke}%
\pgfsetdash{}{0pt}%
\pgfpathmoveto{\pgfqpoint{3.728456in}{2.810413in}}%
\pgfpathlineto{\pgfqpoint{3.755404in}{2.794855in}}%
\pgfpathlineto{\pgfqpoint{3.717641in}{2.798097in}}%
\pgfpathlineto{\pgfqpoint{3.690694in}{2.813655in}}%
\pgfpathlineto{\pgfqpoint{3.728456in}{2.810413in}}%
\pgfpathclose%
\pgfusepath{fill}%
\end{pgfscope}%
\begin{pgfscope}%
\pgfpathrectangle{\pgfqpoint{0.765000in}{0.660000in}}{\pgfqpoint{4.620000in}{4.620000in}}%
\pgfusepath{clip}%
\pgfsetbuttcap%
\pgfsetroundjoin%
\definecolor{currentfill}{rgb}{1.000000,0.894118,0.788235}%
\pgfsetfillcolor{currentfill}%
\pgfsetlinewidth{0.000000pt}%
\definecolor{currentstroke}{rgb}{1.000000,0.894118,0.788235}%
\pgfsetstrokecolor{currentstroke}%
\pgfsetdash{}{0pt}%
\pgfpathmoveto{\pgfqpoint{3.755404in}{2.794855in}}%
\pgfpathlineto{\pgfqpoint{3.782351in}{2.810413in}}%
\pgfpathlineto{\pgfqpoint{3.744588in}{2.813655in}}%
\pgfpathlineto{\pgfqpoint{3.717641in}{2.798097in}}%
\pgfpathlineto{\pgfqpoint{3.755404in}{2.794855in}}%
\pgfpathclose%
\pgfusepath{fill}%
\end{pgfscope}%
\begin{pgfscope}%
\pgfpathrectangle{\pgfqpoint{0.765000in}{0.660000in}}{\pgfqpoint{4.620000in}{4.620000in}}%
\pgfusepath{clip}%
\pgfsetbuttcap%
\pgfsetroundjoin%
\definecolor{currentfill}{rgb}{1.000000,0.894118,0.788235}%
\pgfsetfillcolor{currentfill}%
\pgfsetlinewidth{0.000000pt}%
\definecolor{currentstroke}{rgb}{1.000000,0.894118,0.788235}%
\pgfsetstrokecolor{currentstroke}%
\pgfsetdash{}{0pt}%
\pgfpathmoveto{\pgfqpoint{3.755404in}{2.857087in}}%
\pgfpathlineto{\pgfqpoint{3.728456in}{2.841529in}}%
\pgfpathlineto{\pgfqpoint{3.690694in}{2.844771in}}%
\pgfpathlineto{\pgfqpoint{3.717641in}{2.860329in}}%
\pgfpathlineto{\pgfqpoint{3.755404in}{2.857087in}}%
\pgfpathclose%
\pgfusepath{fill}%
\end{pgfscope}%
\begin{pgfscope}%
\pgfpathrectangle{\pgfqpoint{0.765000in}{0.660000in}}{\pgfqpoint{4.620000in}{4.620000in}}%
\pgfusepath{clip}%
\pgfsetbuttcap%
\pgfsetroundjoin%
\definecolor{currentfill}{rgb}{1.000000,0.894118,0.788235}%
\pgfsetfillcolor{currentfill}%
\pgfsetlinewidth{0.000000pt}%
\definecolor{currentstroke}{rgb}{1.000000,0.894118,0.788235}%
\pgfsetstrokecolor{currentstroke}%
\pgfsetdash{}{0pt}%
\pgfpathmoveto{\pgfqpoint{3.782351in}{2.841529in}}%
\pgfpathlineto{\pgfqpoint{3.755404in}{2.857087in}}%
\pgfpathlineto{\pgfqpoint{3.717641in}{2.860329in}}%
\pgfpathlineto{\pgfqpoint{3.744588in}{2.844771in}}%
\pgfpathlineto{\pgfqpoint{3.782351in}{2.841529in}}%
\pgfpathclose%
\pgfusepath{fill}%
\end{pgfscope}%
\begin{pgfscope}%
\pgfpathrectangle{\pgfqpoint{0.765000in}{0.660000in}}{\pgfqpoint{4.620000in}{4.620000in}}%
\pgfusepath{clip}%
\pgfsetbuttcap%
\pgfsetroundjoin%
\definecolor{currentfill}{rgb}{1.000000,0.894118,0.788235}%
\pgfsetfillcolor{currentfill}%
\pgfsetlinewidth{0.000000pt}%
\definecolor{currentstroke}{rgb}{1.000000,0.894118,0.788235}%
\pgfsetstrokecolor{currentstroke}%
\pgfsetdash{}{0pt}%
\pgfpathmoveto{\pgfqpoint{3.728456in}{2.810413in}}%
\pgfpathlineto{\pgfqpoint{3.728456in}{2.841529in}}%
\pgfpathlineto{\pgfqpoint{3.690694in}{2.844771in}}%
\pgfpathlineto{\pgfqpoint{3.717641in}{2.798097in}}%
\pgfpathlineto{\pgfqpoint{3.728456in}{2.810413in}}%
\pgfpathclose%
\pgfusepath{fill}%
\end{pgfscope}%
\begin{pgfscope}%
\pgfpathrectangle{\pgfqpoint{0.765000in}{0.660000in}}{\pgfqpoint{4.620000in}{4.620000in}}%
\pgfusepath{clip}%
\pgfsetbuttcap%
\pgfsetroundjoin%
\definecolor{currentfill}{rgb}{1.000000,0.894118,0.788235}%
\pgfsetfillcolor{currentfill}%
\pgfsetlinewidth{0.000000pt}%
\definecolor{currentstroke}{rgb}{1.000000,0.894118,0.788235}%
\pgfsetstrokecolor{currentstroke}%
\pgfsetdash{}{0pt}%
\pgfpathmoveto{\pgfqpoint{3.755404in}{2.794855in}}%
\pgfpathlineto{\pgfqpoint{3.755404in}{2.825971in}}%
\pgfpathlineto{\pgfqpoint{3.717641in}{2.829213in}}%
\pgfpathlineto{\pgfqpoint{3.690694in}{2.813655in}}%
\pgfpathlineto{\pgfqpoint{3.755404in}{2.794855in}}%
\pgfpathclose%
\pgfusepath{fill}%
\end{pgfscope}%
\begin{pgfscope}%
\pgfpathrectangle{\pgfqpoint{0.765000in}{0.660000in}}{\pgfqpoint{4.620000in}{4.620000in}}%
\pgfusepath{clip}%
\pgfsetbuttcap%
\pgfsetroundjoin%
\definecolor{currentfill}{rgb}{1.000000,0.894118,0.788235}%
\pgfsetfillcolor{currentfill}%
\pgfsetlinewidth{0.000000pt}%
\definecolor{currentstroke}{rgb}{1.000000,0.894118,0.788235}%
\pgfsetstrokecolor{currentstroke}%
\pgfsetdash{}{0pt}%
\pgfpathmoveto{\pgfqpoint{3.755404in}{2.825971in}}%
\pgfpathlineto{\pgfqpoint{3.782351in}{2.841529in}}%
\pgfpathlineto{\pgfqpoint{3.744588in}{2.844771in}}%
\pgfpathlineto{\pgfqpoint{3.717641in}{2.829213in}}%
\pgfpathlineto{\pgfqpoint{3.755404in}{2.825971in}}%
\pgfpathclose%
\pgfusepath{fill}%
\end{pgfscope}%
\begin{pgfscope}%
\pgfpathrectangle{\pgfqpoint{0.765000in}{0.660000in}}{\pgfqpoint{4.620000in}{4.620000in}}%
\pgfusepath{clip}%
\pgfsetbuttcap%
\pgfsetroundjoin%
\definecolor{currentfill}{rgb}{1.000000,0.894118,0.788235}%
\pgfsetfillcolor{currentfill}%
\pgfsetlinewidth{0.000000pt}%
\definecolor{currentstroke}{rgb}{1.000000,0.894118,0.788235}%
\pgfsetstrokecolor{currentstroke}%
\pgfsetdash{}{0pt}%
\pgfpathmoveto{\pgfqpoint{3.728456in}{2.841529in}}%
\pgfpathlineto{\pgfqpoint{3.755404in}{2.825971in}}%
\pgfpathlineto{\pgfqpoint{3.717641in}{2.829213in}}%
\pgfpathlineto{\pgfqpoint{3.690694in}{2.844771in}}%
\pgfpathlineto{\pgfqpoint{3.728456in}{2.841529in}}%
\pgfpathclose%
\pgfusepath{fill}%
\end{pgfscope}%
\begin{pgfscope}%
\pgfpathrectangle{\pgfqpoint{0.765000in}{0.660000in}}{\pgfqpoint{4.620000in}{4.620000in}}%
\pgfusepath{clip}%
\pgfsetbuttcap%
\pgfsetroundjoin%
\definecolor{currentfill}{rgb}{1.000000,0.894118,0.788235}%
\pgfsetfillcolor{currentfill}%
\pgfsetlinewidth{0.000000pt}%
\definecolor{currentstroke}{rgb}{1.000000,0.894118,0.788235}%
\pgfsetstrokecolor{currentstroke}%
\pgfsetdash{}{0pt}%
\pgfpathmoveto{\pgfqpoint{3.868079in}{2.847690in}}%
\pgfpathlineto{\pgfqpoint{3.841132in}{2.832132in}}%
\pgfpathlineto{\pgfqpoint{3.868079in}{2.816574in}}%
\pgfpathlineto{\pgfqpoint{3.895026in}{2.832132in}}%
\pgfpathlineto{\pgfqpoint{3.868079in}{2.847690in}}%
\pgfpathclose%
\pgfusepath{fill}%
\end{pgfscope}%
\begin{pgfscope}%
\pgfpathrectangle{\pgfqpoint{0.765000in}{0.660000in}}{\pgfqpoint{4.620000in}{4.620000in}}%
\pgfusepath{clip}%
\pgfsetbuttcap%
\pgfsetroundjoin%
\definecolor{currentfill}{rgb}{1.000000,0.894118,0.788235}%
\pgfsetfillcolor{currentfill}%
\pgfsetlinewidth{0.000000pt}%
\definecolor{currentstroke}{rgb}{1.000000,0.894118,0.788235}%
\pgfsetstrokecolor{currentstroke}%
\pgfsetdash{}{0pt}%
\pgfpathmoveto{\pgfqpoint{3.868079in}{2.847690in}}%
\pgfpathlineto{\pgfqpoint{3.841132in}{2.832132in}}%
\pgfpathlineto{\pgfqpoint{3.841132in}{2.863248in}}%
\pgfpathlineto{\pgfqpoint{3.868079in}{2.878806in}}%
\pgfpathlineto{\pgfqpoint{3.868079in}{2.847690in}}%
\pgfpathclose%
\pgfusepath{fill}%
\end{pgfscope}%
\begin{pgfscope}%
\pgfpathrectangle{\pgfqpoint{0.765000in}{0.660000in}}{\pgfqpoint{4.620000in}{4.620000in}}%
\pgfusepath{clip}%
\pgfsetbuttcap%
\pgfsetroundjoin%
\definecolor{currentfill}{rgb}{1.000000,0.894118,0.788235}%
\pgfsetfillcolor{currentfill}%
\pgfsetlinewidth{0.000000pt}%
\definecolor{currentstroke}{rgb}{1.000000,0.894118,0.788235}%
\pgfsetstrokecolor{currentstroke}%
\pgfsetdash{}{0pt}%
\pgfpathmoveto{\pgfqpoint{3.868079in}{2.847690in}}%
\pgfpathlineto{\pgfqpoint{3.895026in}{2.832132in}}%
\pgfpathlineto{\pgfqpoint{3.895026in}{2.863248in}}%
\pgfpathlineto{\pgfqpoint{3.868079in}{2.878806in}}%
\pgfpathlineto{\pgfqpoint{3.868079in}{2.847690in}}%
\pgfpathclose%
\pgfusepath{fill}%
\end{pgfscope}%
\begin{pgfscope}%
\pgfpathrectangle{\pgfqpoint{0.765000in}{0.660000in}}{\pgfqpoint{4.620000in}{4.620000in}}%
\pgfusepath{clip}%
\pgfsetbuttcap%
\pgfsetroundjoin%
\definecolor{currentfill}{rgb}{1.000000,0.894118,0.788235}%
\pgfsetfillcolor{currentfill}%
\pgfsetlinewidth{0.000000pt}%
\definecolor{currentstroke}{rgb}{1.000000,0.894118,0.788235}%
\pgfsetstrokecolor{currentstroke}%
\pgfsetdash{}{0pt}%
\pgfpathmoveto{\pgfqpoint{3.880311in}{2.858581in}}%
\pgfpathlineto{\pgfqpoint{3.853364in}{2.843023in}}%
\pgfpathlineto{\pgfqpoint{3.880311in}{2.827465in}}%
\pgfpathlineto{\pgfqpoint{3.907258in}{2.843023in}}%
\pgfpathlineto{\pgfqpoint{3.880311in}{2.858581in}}%
\pgfpathclose%
\pgfusepath{fill}%
\end{pgfscope}%
\begin{pgfscope}%
\pgfpathrectangle{\pgfqpoint{0.765000in}{0.660000in}}{\pgfqpoint{4.620000in}{4.620000in}}%
\pgfusepath{clip}%
\pgfsetbuttcap%
\pgfsetroundjoin%
\definecolor{currentfill}{rgb}{1.000000,0.894118,0.788235}%
\pgfsetfillcolor{currentfill}%
\pgfsetlinewidth{0.000000pt}%
\definecolor{currentstroke}{rgb}{1.000000,0.894118,0.788235}%
\pgfsetstrokecolor{currentstroke}%
\pgfsetdash{}{0pt}%
\pgfpathmoveto{\pgfqpoint{3.880311in}{2.858581in}}%
\pgfpathlineto{\pgfqpoint{3.853364in}{2.843023in}}%
\pgfpathlineto{\pgfqpoint{3.853364in}{2.874139in}}%
\pgfpathlineto{\pgfqpoint{3.880311in}{2.889697in}}%
\pgfpathlineto{\pgfqpoint{3.880311in}{2.858581in}}%
\pgfpathclose%
\pgfusepath{fill}%
\end{pgfscope}%
\begin{pgfscope}%
\pgfpathrectangle{\pgfqpoint{0.765000in}{0.660000in}}{\pgfqpoint{4.620000in}{4.620000in}}%
\pgfusepath{clip}%
\pgfsetbuttcap%
\pgfsetroundjoin%
\definecolor{currentfill}{rgb}{1.000000,0.894118,0.788235}%
\pgfsetfillcolor{currentfill}%
\pgfsetlinewidth{0.000000pt}%
\definecolor{currentstroke}{rgb}{1.000000,0.894118,0.788235}%
\pgfsetstrokecolor{currentstroke}%
\pgfsetdash{}{0pt}%
\pgfpathmoveto{\pgfqpoint{3.880311in}{2.858581in}}%
\pgfpathlineto{\pgfqpoint{3.907258in}{2.843023in}}%
\pgfpathlineto{\pgfqpoint{3.907258in}{2.874139in}}%
\pgfpathlineto{\pgfqpoint{3.880311in}{2.889697in}}%
\pgfpathlineto{\pgfqpoint{3.880311in}{2.858581in}}%
\pgfpathclose%
\pgfusepath{fill}%
\end{pgfscope}%
\begin{pgfscope}%
\pgfpathrectangle{\pgfqpoint{0.765000in}{0.660000in}}{\pgfqpoint{4.620000in}{4.620000in}}%
\pgfusepath{clip}%
\pgfsetbuttcap%
\pgfsetroundjoin%
\definecolor{currentfill}{rgb}{1.000000,0.894118,0.788235}%
\pgfsetfillcolor{currentfill}%
\pgfsetlinewidth{0.000000pt}%
\definecolor{currentstroke}{rgb}{1.000000,0.894118,0.788235}%
\pgfsetstrokecolor{currentstroke}%
\pgfsetdash{}{0pt}%
\pgfpathmoveto{\pgfqpoint{3.868079in}{2.878806in}}%
\pgfpathlineto{\pgfqpoint{3.841132in}{2.863248in}}%
\pgfpathlineto{\pgfqpoint{3.868079in}{2.847690in}}%
\pgfpathlineto{\pgfqpoint{3.895026in}{2.863248in}}%
\pgfpathlineto{\pgfqpoint{3.868079in}{2.878806in}}%
\pgfpathclose%
\pgfusepath{fill}%
\end{pgfscope}%
\begin{pgfscope}%
\pgfpathrectangle{\pgfqpoint{0.765000in}{0.660000in}}{\pgfqpoint{4.620000in}{4.620000in}}%
\pgfusepath{clip}%
\pgfsetbuttcap%
\pgfsetroundjoin%
\definecolor{currentfill}{rgb}{1.000000,0.894118,0.788235}%
\pgfsetfillcolor{currentfill}%
\pgfsetlinewidth{0.000000pt}%
\definecolor{currentstroke}{rgb}{1.000000,0.894118,0.788235}%
\pgfsetstrokecolor{currentstroke}%
\pgfsetdash{}{0pt}%
\pgfpathmoveto{\pgfqpoint{3.868079in}{2.816574in}}%
\pgfpathlineto{\pgfqpoint{3.895026in}{2.832132in}}%
\pgfpathlineto{\pgfqpoint{3.895026in}{2.863248in}}%
\pgfpathlineto{\pgfqpoint{3.868079in}{2.847690in}}%
\pgfpathlineto{\pgfqpoint{3.868079in}{2.816574in}}%
\pgfpathclose%
\pgfusepath{fill}%
\end{pgfscope}%
\begin{pgfscope}%
\pgfpathrectangle{\pgfqpoint{0.765000in}{0.660000in}}{\pgfqpoint{4.620000in}{4.620000in}}%
\pgfusepath{clip}%
\pgfsetbuttcap%
\pgfsetroundjoin%
\definecolor{currentfill}{rgb}{1.000000,0.894118,0.788235}%
\pgfsetfillcolor{currentfill}%
\pgfsetlinewidth{0.000000pt}%
\definecolor{currentstroke}{rgb}{1.000000,0.894118,0.788235}%
\pgfsetstrokecolor{currentstroke}%
\pgfsetdash{}{0pt}%
\pgfpathmoveto{\pgfqpoint{3.841132in}{2.832132in}}%
\pgfpathlineto{\pgfqpoint{3.868079in}{2.816574in}}%
\pgfpathlineto{\pgfqpoint{3.868079in}{2.847690in}}%
\pgfpathlineto{\pgfqpoint{3.841132in}{2.863248in}}%
\pgfpathlineto{\pgfqpoint{3.841132in}{2.832132in}}%
\pgfpathclose%
\pgfusepath{fill}%
\end{pgfscope}%
\begin{pgfscope}%
\pgfpathrectangle{\pgfqpoint{0.765000in}{0.660000in}}{\pgfqpoint{4.620000in}{4.620000in}}%
\pgfusepath{clip}%
\pgfsetbuttcap%
\pgfsetroundjoin%
\definecolor{currentfill}{rgb}{1.000000,0.894118,0.788235}%
\pgfsetfillcolor{currentfill}%
\pgfsetlinewidth{0.000000pt}%
\definecolor{currentstroke}{rgb}{1.000000,0.894118,0.788235}%
\pgfsetstrokecolor{currentstroke}%
\pgfsetdash{}{0pt}%
\pgfpathmoveto{\pgfqpoint{3.880311in}{2.889697in}}%
\pgfpathlineto{\pgfqpoint{3.853364in}{2.874139in}}%
\pgfpathlineto{\pgfqpoint{3.880311in}{2.858581in}}%
\pgfpathlineto{\pgfqpoint{3.907258in}{2.874139in}}%
\pgfpathlineto{\pgfqpoint{3.880311in}{2.889697in}}%
\pgfpathclose%
\pgfusepath{fill}%
\end{pgfscope}%
\begin{pgfscope}%
\pgfpathrectangle{\pgfqpoint{0.765000in}{0.660000in}}{\pgfqpoint{4.620000in}{4.620000in}}%
\pgfusepath{clip}%
\pgfsetbuttcap%
\pgfsetroundjoin%
\definecolor{currentfill}{rgb}{1.000000,0.894118,0.788235}%
\pgfsetfillcolor{currentfill}%
\pgfsetlinewidth{0.000000pt}%
\definecolor{currentstroke}{rgb}{1.000000,0.894118,0.788235}%
\pgfsetstrokecolor{currentstroke}%
\pgfsetdash{}{0pt}%
\pgfpathmoveto{\pgfqpoint{3.880311in}{2.827465in}}%
\pgfpathlineto{\pgfqpoint{3.907258in}{2.843023in}}%
\pgfpathlineto{\pgfqpoint{3.907258in}{2.874139in}}%
\pgfpathlineto{\pgfqpoint{3.880311in}{2.858581in}}%
\pgfpathlineto{\pgfqpoint{3.880311in}{2.827465in}}%
\pgfpathclose%
\pgfusepath{fill}%
\end{pgfscope}%
\begin{pgfscope}%
\pgfpathrectangle{\pgfqpoint{0.765000in}{0.660000in}}{\pgfqpoint{4.620000in}{4.620000in}}%
\pgfusepath{clip}%
\pgfsetbuttcap%
\pgfsetroundjoin%
\definecolor{currentfill}{rgb}{1.000000,0.894118,0.788235}%
\pgfsetfillcolor{currentfill}%
\pgfsetlinewidth{0.000000pt}%
\definecolor{currentstroke}{rgb}{1.000000,0.894118,0.788235}%
\pgfsetstrokecolor{currentstroke}%
\pgfsetdash{}{0pt}%
\pgfpathmoveto{\pgfqpoint{3.853364in}{2.843023in}}%
\pgfpathlineto{\pgfqpoint{3.880311in}{2.827465in}}%
\pgfpathlineto{\pgfqpoint{3.880311in}{2.858581in}}%
\pgfpathlineto{\pgfqpoint{3.853364in}{2.874139in}}%
\pgfpathlineto{\pgfqpoint{3.853364in}{2.843023in}}%
\pgfpathclose%
\pgfusepath{fill}%
\end{pgfscope}%
\begin{pgfscope}%
\pgfpathrectangle{\pgfqpoint{0.765000in}{0.660000in}}{\pgfqpoint{4.620000in}{4.620000in}}%
\pgfusepath{clip}%
\pgfsetbuttcap%
\pgfsetroundjoin%
\definecolor{currentfill}{rgb}{1.000000,0.894118,0.788235}%
\pgfsetfillcolor{currentfill}%
\pgfsetlinewidth{0.000000pt}%
\definecolor{currentstroke}{rgb}{1.000000,0.894118,0.788235}%
\pgfsetstrokecolor{currentstroke}%
\pgfsetdash{}{0pt}%
\pgfpathmoveto{\pgfqpoint{3.868079in}{2.847690in}}%
\pgfpathlineto{\pgfqpoint{3.841132in}{2.832132in}}%
\pgfpathlineto{\pgfqpoint{3.853364in}{2.843023in}}%
\pgfpathlineto{\pgfqpoint{3.880311in}{2.858581in}}%
\pgfpathlineto{\pgfqpoint{3.868079in}{2.847690in}}%
\pgfpathclose%
\pgfusepath{fill}%
\end{pgfscope}%
\begin{pgfscope}%
\pgfpathrectangle{\pgfqpoint{0.765000in}{0.660000in}}{\pgfqpoint{4.620000in}{4.620000in}}%
\pgfusepath{clip}%
\pgfsetbuttcap%
\pgfsetroundjoin%
\definecolor{currentfill}{rgb}{1.000000,0.894118,0.788235}%
\pgfsetfillcolor{currentfill}%
\pgfsetlinewidth{0.000000pt}%
\definecolor{currentstroke}{rgb}{1.000000,0.894118,0.788235}%
\pgfsetstrokecolor{currentstroke}%
\pgfsetdash{}{0pt}%
\pgfpathmoveto{\pgfqpoint{3.895026in}{2.832132in}}%
\pgfpathlineto{\pgfqpoint{3.868079in}{2.847690in}}%
\pgfpathlineto{\pgfqpoint{3.880311in}{2.858581in}}%
\pgfpathlineto{\pgfqpoint{3.907258in}{2.843023in}}%
\pgfpathlineto{\pgfqpoint{3.895026in}{2.832132in}}%
\pgfpathclose%
\pgfusepath{fill}%
\end{pgfscope}%
\begin{pgfscope}%
\pgfpathrectangle{\pgfqpoint{0.765000in}{0.660000in}}{\pgfqpoint{4.620000in}{4.620000in}}%
\pgfusepath{clip}%
\pgfsetbuttcap%
\pgfsetroundjoin%
\definecolor{currentfill}{rgb}{1.000000,0.894118,0.788235}%
\pgfsetfillcolor{currentfill}%
\pgfsetlinewidth{0.000000pt}%
\definecolor{currentstroke}{rgb}{1.000000,0.894118,0.788235}%
\pgfsetstrokecolor{currentstroke}%
\pgfsetdash{}{0pt}%
\pgfpathmoveto{\pgfqpoint{3.868079in}{2.847690in}}%
\pgfpathlineto{\pgfqpoint{3.868079in}{2.878806in}}%
\pgfpathlineto{\pgfqpoint{3.880311in}{2.889697in}}%
\pgfpathlineto{\pgfqpoint{3.907258in}{2.843023in}}%
\pgfpathlineto{\pgfqpoint{3.868079in}{2.847690in}}%
\pgfpathclose%
\pgfusepath{fill}%
\end{pgfscope}%
\begin{pgfscope}%
\pgfpathrectangle{\pgfqpoint{0.765000in}{0.660000in}}{\pgfqpoint{4.620000in}{4.620000in}}%
\pgfusepath{clip}%
\pgfsetbuttcap%
\pgfsetroundjoin%
\definecolor{currentfill}{rgb}{1.000000,0.894118,0.788235}%
\pgfsetfillcolor{currentfill}%
\pgfsetlinewidth{0.000000pt}%
\definecolor{currentstroke}{rgb}{1.000000,0.894118,0.788235}%
\pgfsetstrokecolor{currentstroke}%
\pgfsetdash{}{0pt}%
\pgfpathmoveto{\pgfqpoint{3.895026in}{2.832132in}}%
\pgfpathlineto{\pgfqpoint{3.895026in}{2.863248in}}%
\pgfpathlineto{\pgfqpoint{3.907258in}{2.874139in}}%
\pgfpathlineto{\pgfqpoint{3.880311in}{2.858581in}}%
\pgfpathlineto{\pgfqpoint{3.895026in}{2.832132in}}%
\pgfpathclose%
\pgfusepath{fill}%
\end{pgfscope}%
\begin{pgfscope}%
\pgfpathrectangle{\pgfqpoint{0.765000in}{0.660000in}}{\pgfqpoint{4.620000in}{4.620000in}}%
\pgfusepath{clip}%
\pgfsetbuttcap%
\pgfsetroundjoin%
\definecolor{currentfill}{rgb}{1.000000,0.894118,0.788235}%
\pgfsetfillcolor{currentfill}%
\pgfsetlinewidth{0.000000pt}%
\definecolor{currentstroke}{rgb}{1.000000,0.894118,0.788235}%
\pgfsetstrokecolor{currentstroke}%
\pgfsetdash{}{0pt}%
\pgfpathmoveto{\pgfqpoint{3.841132in}{2.832132in}}%
\pgfpathlineto{\pgfqpoint{3.868079in}{2.816574in}}%
\pgfpathlineto{\pgfqpoint{3.880311in}{2.827465in}}%
\pgfpathlineto{\pgfqpoint{3.853364in}{2.843023in}}%
\pgfpathlineto{\pgfqpoint{3.841132in}{2.832132in}}%
\pgfpathclose%
\pgfusepath{fill}%
\end{pgfscope}%
\begin{pgfscope}%
\pgfpathrectangle{\pgfqpoint{0.765000in}{0.660000in}}{\pgfqpoint{4.620000in}{4.620000in}}%
\pgfusepath{clip}%
\pgfsetbuttcap%
\pgfsetroundjoin%
\definecolor{currentfill}{rgb}{1.000000,0.894118,0.788235}%
\pgfsetfillcolor{currentfill}%
\pgfsetlinewidth{0.000000pt}%
\definecolor{currentstroke}{rgb}{1.000000,0.894118,0.788235}%
\pgfsetstrokecolor{currentstroke}%
\pgfsetdash{}{0pt}%
\pgfpathmoveto{\pgfqpoint{3.868079in}{2.816574in}}%
\pgfpathlineto{\pgfqpoint{3.895026in}{2.832132in}}%
\pgfpathlineto{\pgfqpoint{3.907258in}{2.843023in}}%
\pgfpathlineto{\pgfqpoint{3.880311in}{2.827465in}}%
\pgfpathlineto{\pgfqpoint{3.868079in}{2.816574in}}%
\pgfpathclose%
\pgfusepath{fill}%
\end{pgfscope}%
\begin{pgfscope}%
\pgfpathrectangle{\pgfqpoint{0.765000in}{0.660000in}}{\pgfqpoint{4.620000in}{4.620000in}}%
\pgfusepath{clip}%
\pgfsetbuttcap%
\pgfsetroundjoin%
\definecolor{currentfill}{rgb}{1.000000,0.894118,0.788235}%
\pgfsetfillcolor{currentfill}%
\pgfsetlinewidth{0.000000pt}%
\definecolor{currentstroke}{rgb}{1.000000,0.894118,0.788235}%
\pgfsetstrokecolor{currentstroke}%
\pgfsetdash{}{0pt}%
\pgfpathmoveto{\pgfqpoint{3.868079in}{2.878806in}}%
\pgfpathlineto{\pgfqpoint{3.841132in}{2.863248in}}%
\pgfpathlineto{\pgfqpoint{3.853364in}{2.874139in}}%
\pgfpathlineto{\pgfqpoint{3.880311in}{2.889697in}}%
\pgfpathlineto{\pgfqpoint{3.868079in}{2.878806in}}%
\pgfpathclose%
\pgfusepath{fill}%
\end{pgfscope}%
\begin{pgfscope}%
\pgfpathrectangle{\pgfqpoint{0.765000in}{0.660000in}}{\pgfqpoint{4.620000in}{4.620000in}}%
\pgfusepath{clip}%
\pgfsetbuttcap%
\pgfsetroundjoin%
\definecolor{currentfill}{rgb}{1.000000,0.894118,0.788235}%
\pgfsetfillcolor{currentfill}%
\pgfsetlinewidth{0.000000pt}%
\definecolor{currentstroke}{rgb}{1.000000,0.894118,0.788235}%
\pgfsetstrokecolor{currentstroke}%
\pgfsetdash{}{0pt}%
\pgfpathmoveto{\pgfqpoint{3.895026in}{2.863248in}}%
\pgfpathlineto{\pgfqpoint{3.868079in}{2.878806in}}%
\pgfpathlineto{\pgfqpoint{3.880311in}{2.889697in}}%
\pgfpathlineto{\pgfqpoint{3.907258in}{2.874139in}}%
\pgfpathlineto{\pgfqpoint{3.895026in}{2.863248in}}%
\pgfpathclose%
\pgfusepath{fill}%
\end{pgfscope}%
\begin{pgfscope}%
\pgfpathrectangle{\pgfqpoint{0.765000in}{0.660000in}}{\pgfqpoint{4.620000in}{4.620000in}}%
\pgfusepath{clip}%
\pgfsetbuttcap%
\pgfsetroundjoin%
\definecolor{currentfill}{rgb}{1.000000,0.894118,0.788235}%
\pgfsetfillcolor{currentfill}%
\pgfsetlinewidth{0.000000pt}%
\definecolor{currentstroke}{rgb}{1.000000,0.894118,0.788235}%
\pgfsetstrokecolor{currentstroke}%
\pgfsetdash{}{0pt}%
\pgfpathmoveto{\pgfqpoint{3.841132in}{2.832132in}}%
\pgfpathlineto{\pgfqpoint{3.841132in}{2.863248in}}%
\pgfpathlineto{\pgfqpoint{3.853364in}{2.874139in}}%
\pgfpathlineto{\pgfqpoint{3.880311in}{2.827465in}}%
\pgfpathlineto{\pgfqpoint{3.841132in}{2.832132in}}%
\pgfpathclose%
\pgfusepath{fill}%
\end{pgfscope}%
\begin{pgfscope}%
\pgfpathrectangle{\pgfqpoint{0.765000in}{0.660000in}}{\pgfqpoint{4.620000in}{4.620000in}}%
\pgfusepath{clip}%
\pgfsetbuttcap%
\pgfsetroundjoin%
\definecolor{currentfill}{rgb}{1.000000,0.894118,0.788235}%
\pgfsetfillcolor{currentfill}%
\pgfsetlinewidth{0.000000pt}%
\definecolor{currentstroke}{rgb}{1.000000,0.894118,0.788235}%
\pgfsetstrokecolor{currentstroke}%
\pgfsetdash{}{0pt}%
\pgfpathmoveto{\pgfqpoint{3.868079in}{2.816574in}}%
\pgfpathlineto{\pgfqpoint{3.868079in}{2.847690in}}%
\pgfpathlineto{\pgfqpoint{3.880311in}{2.858581in}}%
\pgfpathlineto{\pgfqpoint{3.853364in}{2.843023in}}%
\pgfpathlineto{\pgfqpoint{3.868079in}{2.816574in}}%
\pgfpathclose%
\pgfusepath{fill}%
\end{pgfscope}%
\begin{pgfscope}%
\pgfpathrectangle{\pgfqpoint{0.765000in}{0.660000in}}{\pgfqpoint{4.620000in}{4.620000in}}%
\pgfusepath{clip}%
\pgfsetbuttcap%
\pgfsetroundjoin%
\definecolor{currentfill}{rgb}{1.000000,0.894118,0.788235}%
\pgfsetfillcolor{currentfill}%
\pgfsetlinewidth{0.000000pt}%
\definecolor{currentstroke}{rgb}{1.000000,0.894118,0.788235}%
\pgfsetstrokecolor{currentstroke}%
\pgfsetdash{}{0pt}%
\pgfpathmoveto{\pgfqpoint{3.868079in}{2.847690in}}%
\pgfpathlineto{\pgfqpoint{3.895026in}{2.863248in}}%
\pgfpathlineto{\pgfqpoint{3.907258in}{2.874139in}}%
\pgfpathlineto{\pgfqpoint{3.880311in}{2.858581in}}%
\pgfpathlineto{\pgfqpoint{3.868079in}{2.847690in}}%
\pgfpathclose%
\pgfusepath{fill}%
\end{pgfscope}%
\begin{pgfscope}%
\pgfpathrectangle{\pgfqpoint{0.765000in}{0.660000in}}{\pgfqpoint{4.620000in}{4.620000in}}%
\pgfusepath{clip}%
\pgfsetbuttcap%
\pgfsetroundjoin%
\definecolor{currentfill}{rgb}{1.000000,0.894118,0.788235}%
\pgfsetfillcolor{currentfill}%
\pgfsetlinewidth{0.000000pt}%
\definecolor{currentstroke}{rgb}{1.000000,0.894118,0.788235}%
\pgfsetstrokecolor{currentstroke}%
\pgfsetdash{}{0pt}%
\pgfpathmoveto{\pgfqpoint{3.841132in}{2.863248in}}%
\pgfpathlineto{\pgfqpoint{3.868079in}{2.847690in}}%
\pgfpathlineto{\pgfqpoint{3.880311in}{2.858581in}}%
\pgfpathlineto{\pgfqpoint{3.853364in}{2.874139in}}%
\pgfpathlineto{\pgfqpoint{3.841132in}{2.863248in}}%
\pgfpathclose%
\pgfusepath{fill}%
\end{pgfscope}%
\begin{pgfscope}%
\pgfpathrectangle{\pgfqpoint{0.765000in}{0.660000in}}{\pgfqpoint{4.620000in}{4.620000in}}%
\pgfusepath{clip}%
\pgfsetbuttcap%
\pgfsetroundjoin%
\definecolor{currentfill}{rgb}{1.000000,0.894118,0.788235}%
\pgfsetfillcolor{currentfill}%
\pgfsetlinewidth{0.000000pt}%
\definecolor{currentstroke}{rgb}{1.000000,0.894118,0.788235}%
\pgfsetstrokecolor{currentstroke}%
\pgfsetdash{}{0pt}%
\pgfpathmoveto{\pgfqpoint{3.868079in}{2.847690in}}%
\pgfpathlineto{\pgfqpoint{3.841132in}{2.832132in}}%
\pgfpathlineto{\pgfqpoint{3.868079in}{2.816574in}}%
\pgfpathlineto{\pgfqpoint{3.895026in}{2.832132in}}%
\pgfpathlineto{\pgfqpoint{3.868079in}{2.847690in}}%
\pgfpathclose%
\pgfusepath{fill}%
\end{pgfscope}%
\begin{pgfscope}%
\pgfpathrectangle{\pgfqpoint{0.765000in}{0.660000in}}{\pgfqpoint{4.620000in}{4.620000in}}%
\pgfusepath{clip}%
\pgfsetbuttcap%
\pgfsetroundjoin%
\definecolor{currentfill}{rgb}{1.000000,0.894118,0.788235}%
\pgfsetfillcolor{currentfill}%
\pgfsetlinewidth{0.000000pt}%
\definecolor{currentstroke}{rgb}{1.000000,0.894118,0.788235}%
\pgfsetstrokecolor{currentstroke}%
\pgfsetdash{}{0pt}%
\pgfpathmoveto{\pgfqpoint{3.868079in}{2.847690in}}%
\pgfpathlineto{\pgfqpoint{3.841132in}{2.832132in}}%
\pgfpathlineto{\pgfqpoint{3.841132in}{2.863248in}}%
\pgfpathlineto{\pgfqpoint{3.868079in}{2.878806in}}%
\pgfpathlineto{\pgfqpoint{3.868079in}{2.847690in}}%
\pgfpathclose%
\pgfusepath{fill}%
\end{pgfscope}%
\begin{pgfscope}%
\pgfpathrectangle{\pgfqpoint{0.765000in}{0.660000in}}{\pgfqpoint{4.620000in}{4.620000in}}%
\pgfusepath{clip}%
\pgfsetbuttcap%
\pgfsetroundjoin%
\definecolor{currentfill}{rgb}{1.000000,0.894118,0.788235}%
\pgfsetfillcolor{currentfill}%
\pgfsetlinewidth{0.000000pt}%
\definecolor{currentstroke}{rgb}{1.000000,0.894118,0.788235}%
\pgfsetstrokecolor{currentstroke}%
\pgfsetdash{}{0pt}%
\pgfpathmoveto{\pgfqpoint{3.868079in}{2.847690in}}%
\pgfpathlineto{\pgfqpoint{3.895026in}{2.832132in}}%
\pgfpathlineto{\pgfqpoint{3.895026in}{2.863248in}}%
\pgfpathlineto{\pgfqpoint{3.868079in}{2.878806in}}%
\pgfpathlineto{\pgfqpoint{3.868079in}{2.847690in}}%
\pgfpathclose%
\pgfusepath{fill}%
\end{pgfscope}%
\begin{pgfscope}%
\pgfpathrectangle{\pgfqpoint{0.765000in}{0.660000in}}{\pgfqpoint{4.620000in}{4.620000in}}%
\pgfusepath{clip}%
\pgfsetbuttcap%
\pgfsetroundjoin%
\definecolor{currentfill}{rgb}{1.000000,0.894118,0.788235}%
\pgfsetfillcolor{currentfill}%
\pgfsetlinewidth{0.000000pt}%
\definecolor{currentstroke}{rgb}{1.000000,0.894118,0.788235}%
\pgfsetstrokecolor{currentstroke}%
\pgfsetdash{}{0pt}%
\pgfpathmoveto{\pgfqpoint{3.856631in}{2.841507in}}%
\pgfpathlineto{\pgfqpoint{3.829684in}{2.825949in}}%
\pgfpathlineto{\pgfqpoint{3.856631in}{2.810391in}}%
\pgfpathlineto{\pgfqpoint{3.883578in}{2.825949in}}%
\pgfpathlineto{\pgfqpoint{3.856631in}{2.841507in}}%
\pgfpathclose%
\pgfusepath{fill}%
\end{pgfscope}%
\begin{pgfscope}%
\pgfpathrectangle{\pgfqpoint{0.765000in}{0.660000in}}{\pgfqpoint{4.620000in}{4.620000in}}%
\pgfusepath{clip}%
\pgfsetbuttcap%
\pgfsetroundjoin%
\definecolor{currentfill}{rgb}{1.000000,0.894118,0.788235}%
\pgfsetfillcolor{currentfill}%
\pgfsetlinewidth{0.000000pt}%
\definecolor{currentstroke}{rgb}{1.000000,0.894118,0.788235}%
\pgfsetstrokecolor{currentstroke}%
\pgfsetdash{}{0pt}%
\pgfpathmoveto{\pgfqpoint{3.856631in}{2.841507in}}%
\pgfpathlineto{\pgfqpoint{3.829684in}{2.825949in}}%
\pgfpathlineto{\pgfqpoint{3.829684in}{2.857065in}}%
\pgfpathlineto{\pgfqpoint{3.856631in}{2.872623in}}%
\pgfpathlineto{\pgfqpoint{3.856631in}{2.841507in}}%
\pgfpathclose%
\pgfusepath{fill}%
\end{pgfscope}%
\begin{pgfscope}%
\pgfpathrectangle{\pgfqpoint{0.765000in}{0.660000in}}{\pgfqpoint{4.620000in}{4.620000in}}%
\pgfusepath{clip}%
\pgfsetbuttcap%
\pgfsetroundjoin%
\definecolor{currentfill}{rgb}{1.000000,0.894118,0.788235}%
\pgfsetfillcolor{currentfill}%
\pgfsetlinewidth{0.000000pt}%
\definecolor{currentstroke}{rgb}{1.000000,0.894118,0.788235}%
\pgfsetstrokecolor{currentstroke}%
\pgfsetdash{}{0pt}%
\pgfpathmoveto{\pgfqpoint{3.856631in}{2.841507in}}%
\pgfpathlineto{\pgfqpoint{3.883578in}{2.825949in}}%
\pgfpathlineto{\pgfqpoint{3.883578in}{2.857065in}}%
\pgfpathlineto{\pgfqpoint{3.856631in}{2.872623in}}%
\pgfpathlineto{\pgfqpoint{3.856631in}{2.841507in}}%
\pgfpathclose%
\pgfusepath{fill}%
\end{pgfscope}%
\begin{pgfscope}%
\pgfpathrectangle{\pgfqpoint{0.765000in}{0.660000in}}{\pgfqpoint{4.620000in}{4.620000in}}%
\pgfusepath{clip}%
\pgfsetbuttcap%
\pgfsetroundjoin%
\definecolor{currentfill}{rgb}{1.000000,0.894118,0.788235}%
\pgfsetfillcolor{currentfill}%
\pgfsetlinewidth{0.000000pt}%
\definecolor{currentstroke}{rgb}{1.000000,0.894118,0.788235}%
\pgfsetstrokecolor{currentstroke}%
\pgfsetdash{}{0pt}%
\pgfpathmoveto{\pgfqpoint{3.868079in}{2.878806in}}%
\pgfpathlineto{\pgfqpoint{3.841132in}{2.863248in}}%
\pgfpathlineto{\pgfqpoint{3.868079in}{2.847690in}}%
\pgfpathlineto{\pgfqpoint{3.895026in}{2.863248in}}%
\pgfpathlineto{\pgfqpoint{3.868079in}{2.878806in}}%
\pgfpathclose%
\pgfusepath{fill}%
\end{pgfscope}%
\begin{pgfscope}%
\pgfpathrectangle{\pgfqpoint{0.765000in}{0.660000in}}{\pgfqpoint{4.620000in}{4.620000in}}%
\pgfusepath{clip}%
\pgfsetbuttcap%
\pgfsetroundjoin%
\definecolor{currentfill}{rgb}{1.000000,0.894118,0.788235}%
\pgfsetfillcolor{currentfill}%
\pgfsetlinewidth{0.000000pt}%
\definecolor{currentstroke}{rgb}{1.000000,0.894118,0.788235}%
\pgfsetstrokecolor{currentstroke}%
\pgfsetdash{}{0pt}%
\pgfpathmoveto{\pgfqpoint{3.868079in}{2.816574in}}%
\pgfpathlineto{\pgfqpoint{3.895026in}{2.832132in}}%
\pgfpathlineto{\pgfqpoint{3.895026in}{2.863248in}}%
\pgfpathlineto{\pgfqpoint{3.868079in}{2.847690in}}%
\pgfpathlineto{\pgfqpoint{3.868079in}{2.816574in}}%
\pgfpathclose%
\pgfusepath{fill}%
\end{pgfscope}%
\begin{pgfscope}%
\pgfpathrectangle{\pgfqpoint{0.765000in}{0.660000in}}{\pgfqpoint{4.620000in}{4.620000in}}%
\pgfusepath{clip}%
\pgfsetbuttcap%
\pgfsetroundjoin%
\definecolor{currentfill}{rgb}{1.000000,0.894118,0.788235}%
\pgfsetfillcolor{currentfill}%
\pgfsetlinewidth{0.000000pt}%
\definecolor{currentstroke}{rgb}{1.000000,0.894118,0.788235}%
\pgfsetstrokecolor{currentstroke}%
\pgfsetdash{}{0pt}%
\pgfpathmoveto{\pgfqpoint{3.841132in}{2.832132in}}%
\pgfpathlineto{\pgfqpoint{3.868079in}{2.816574in}}%
\pgfpathlineto{\pgfqpoint{3.868079in}{2.847690in}}%
\pgfpathlineto{\pgfqpoint{3.841132in}{2.863248in}}%
\pgfpathlineto{\pgfqpoint{3.841132in}{2.832132in}}%
\pgfpathclose%
\pgfusepath{fill}%
\end{pgfscope}%
\begin{pgfscope}%
\pgfpathrectangle{\pgfqpoint{0.765000in}{0.660000in}}{\pgfqpoint{4.620000in}{4.620000in}}%
\pgfusepath{clip}%
\pgfsetbuttcap%
\pgfsetroundjoin%
\definecolor{currentfill}{rgb}{1.000000,0.894118,0.788235}%
\pgfsetfillcolor{currentfill}%
\pgfsetlinewidth{0.000000pt}%
\definecolor{currentstroke}{rgb}{1.000000,0.894118,0.788235}%
\pgfsetstrokecolor{currentstroke}%
\pgfsetdash{}{0pt}%
\pgfpathmoveto{\pgfqpoint{3.856631in}{2.872623in}}%
\pgfpathlineto{\pgfqpoint{3.829684in}{2.857065in}}%
\pgfpathlineto{\pgfqpoint{3.856631in}{2.841507in}}%
\pgfpathlineto{\pgfqpoint{3.883578in}{2.857065in}}%
\pgfpathlineto{\pgfqpoint{3.856631in}{2.872623in}}%
\pgfpathclose%
\pgfusepath{fill}%
\end{pgfscope}%
\begin{pgfscope}%
\pgfpathrectangle{\pgfqpoint{0.765000in}{0.660000in}}{\pgfqpoint{4.620000in}{4.620000in}}%
\pgfusepath{clip}%
\pgfsetbuttcap%
\pgfsetroundjoin%
\definecolor{currentfill}{rgb}{1.000000,0.894118,0.788235}%
\pgfsetfillcolor{currentfill}%
\pgfsetlinewidth{0.000000pt}%
\definecolor{currentstroke}{rgb}{1.000000,0.894118,0.788235}%
\pgfsetstrokecolor{currentstroke}%
\pgfsetdash{}{0pt}%
\pgfpathmoveto{\pgfqpoint{3.856631in}{2.810391in}}%
\pgfpathlineto{\pgfqpoint{3.883578in}{2.825949in}}%
\pgfpathlineto{\pgfqpoint{3.883578in}{2.857065in}}%
\pgfpathlineto{\pgfqpoint{3.856631in}{2.841507in}}%
\pgfpathlineto{\pgfqpoint{3.856631in}{2.810391in}}%
\pgfpathclose%
\pgfusepath{fill}%
\end{pgfscope}%
\begin{pgfscope}%
\pgfpathrectangle{\pgfqpoint{0.765000in}{0.660000in}}{\pgfqpoint{4.620000in}{4.620000in}}%
\pgfusepath{clip}%
\pgfsetbuttcap%
\pgfsetroundjoin%
\definecolor{currentfill}{rgb}{1.000000,0.894118,0.788235}%
\pgfsetfillcolor{currentfill}%
\pgfsetlinewidth{0.000000pt}%
\definecolor{currentstroke}{rgb}{1.000000,0.894118,0.788235}%
\pgfsetstrokecolor{currentstroke}%
\pgfsetdash{}{0pt}%
\pgfpathmoveto{\pgfqpoint{3.829684in}{2.825949in}}%
\pgfpathlineto{\pgfqpoint{3.856631in}{2.810391in}}%
\pgfpathlineto{\pgfqpoint{3.856631in}{2.841507in}}%
\pgfpathlineto{\pgfqpoint{3.829684in}{2.857065in}}%
\pgfpathlineto{\pgfqpoint{3.829684in}{2.825949in}}%
\pgfpathclose%
\pgfusepath{fill}%
\end{pgfscope}%
\begin{pgfscope}%
\pgfpathrectangle{\pgfqpoint{0.765000in}{0.660000in}}{\pgfqpoint{4.620000in}{4.620000in}}%
\pgfusepath{clip}%
\pgfsetbuttcap%
\pgfsetroundjoin%
\definecolor{currentfill}{rgb}{1.000000,0.894118,0.788235}%
\pgfsetfillcolor{currentfill}%
\pgfsetlinewidth{0.000000pt}%
\definecolor{currentstroke}{rgb}{1.000000,0.894118,0.788235}%
\pgfsetstrokecolor{currentstroke}%
\pgfsetdash{}{0pt}%
\pgfpathmoveto{\pgfqpoint{3.868079in}{2.847690in}}%
\pgfpathlineto{\pgfqpoint{3.841132in}{2.832132in}}%
\pgfpathlineto{\pgfqpoint{3.829684in}{2.825949in}}%
\pgfpathlineto{\pgfqpoint{3.856631in}{2.841507in}}%
\pgfpathlineto{\pgfqpoint{3.868079in}{2.847690in}}%
\pgfpathclose%
\pgfusepath{fill}%
\end{pgfscope}%
\begin{pgfscope}%
\pgfpathrectangle{\pgfqpoint{0.765000in}{0.660000in}}{\pgfqpoint{4.620000in}{4.620000in}}%
\pgfusepath{clip}%
\pgfsetbuttcap%
\pgfsetroundjoin%
\definecolor{currentfill}{rgb}{1.000000,0.894118,0.788235}%
\pgfsetfillcolor{currentfill}%
\pgfsetlinewidth{0.000000pt}%
\definecolor{currentstroke}{rgb}{1.000000,0.894118,0.788235}%
\pgfsetstrokecolor{currentstroke}%
\pgfsetdash{}{0pt}%
\pgfpathmoveto{\pgfqpoint{3.895026in}{2.832132in}}%
\pgfpathlineto{\pgfqpoint{3.868079in}{2.847690in}}%
\pgfpathlineto{\pgfqpoint{3.856631in}{2.841507in}}%
\pgfpathlineto{\pgfqpoint{3.883578in}{2.825949in}}%
\pgfpathlineto{\pgfqpoint{3.895026in}{2.832132in}}%
\pgfpathclose%
\pgfusepath{fill}%
\end{pgfscope}%
\begin{pgfscope}%
\pgfpathrectangle{\pgfqpoint{0.765000in}{0.660000in}}{\pgfqpoint{4.620000in}{4.620000in}}%
\pgfusepath{clip}%
\pgfsetbuttcap%
\pgfsetroundjoin%
\definecolor{currentfill}{rgb}{1.000000,0.894118,0.788235}%
\pgfsetfillcolor{currentfill}%
\pgfsetlinewidth{0.000000pt}%
\definecolor{currentstroke}{rgb}{1.000000,0.894118,0.788235}%
\pgfsetstrokecolor{currentstroke}%
\pgfsetdash{}{0pt}%
\pgfpathmoveto{\pgfqpoint{3.868079in}{2.847690in}}%
\pgfpathlineto{\pgfqpoint{3.868079in}{2.878806in}}%
\pgfpathlineto{\pgfqpoint{3.856631in}{2.872623in}}%
\pgfpathlineto{\pgfqpoint{3.883578in}{2.825949in}}%
\pgfpathlineto{\pgfqpoint{3.868079in}{2.847690in}}%
\pgfpathclose%
\pgfusepath{fill}%
\end{pgfscope}%
\begin{pgfscope}%
\pgfpathrectangle{\pgfqpoint{0.765000in}{0.660000in}}{\pgfqpoint{4.620000in}{4.620000in}}%
\pgfusepath{clip}%
\pgfsetbuttcap%
\pgfsetroundjoin%
\definecolor{currentfill}{rgb}{1.000000,0.894118,0.788235}%
\pgfsetfillcolor{currentfill}%
\pgfsetlinewidth{0.000000pt}%
\definecolor{currentstroke}{rgb}{1.000000,0.894118,0.788235}%
\pgfsetstrokecolor{currentstroke}%
\pgfsetdash{}{0pt}%
\pgfpathmoveto{\pgfqpoint{3.895026in}{2.832132in}}%
\pgfpathlineto{\pgfqpoint{3.895026in}{2.863248in}}%
\pgfpathlineto{\pgfqpoint{3.883578in}{2.857065in}}%
\pgfpathlineto{\pgfqpoint{3.856631in}{2.841507in}}%
\pgfpathlineto{\pgfqpoint{3.895026in}{2.832132in}}%
\pgfpathclose%
\pgfusepath{fill}%
\end{pgfscope}%
\begin{pgfscope}%
\pgfpathrectangle{\pgfqpoint{0.765000in}{0.660000in}}{\pgfqpoint{4.620000in}{4.620000in}}%
\pgfusepath{clip}%
\pgfsetbuttcap%
\pgfsetroundjoin%
\definecolor{currentfill}{rgb}{1.000000,0.894118,0.788235}%
\pgfsetfillcolor{currentfill}%
\pgfsetlinewidth{0.000000pt}%
\definecolor{currentstroke}{rgb}{1.000000,0.894118,0.788235}%
\pgfsetstrokecolor{currentstroke}%
\pgfsetdash{}{0pt}%
\pgfpathmoveto{\pgfqpoint{3.841132in}{2.832132in}}%
\pgfpathlineto{\pgfqpoint{3.868079in}{2.816574in}}%
\pgfpathlineto{\pgfqpoint{3.856631in}{2.810391in}}%
\pgfpathlineto{\pgfqpoint{3.829684in}{2.825949in}}%
\pgfpathlineto{\pgfqpoint{3.841132in}{2.832132in}}%
\pgfpathclose%
\pgfusepath{fill}%
\end{pgfscope}%
\begin{pgfscope}%
\pgfpathrectangle{\pgfqpoint{0.765000in}{0.660000in}}{\pgfqpoint{4.620000in}{4.620000in}}%
\pgfusepath{clip}%
\pgfsetbuttcap%
\pgfsetroundjoin%
\definecolor{currentfill}{rgb}{1.000000,0.894118,0.788235}%
\pgfsetfillcolor{currentfill}%
\pgfsetlinewidth{0.000000pt}%
\definecolor{currentstroke}{rgb}{1.000000,0.894118,0.788235}%
\pgfsetstrokecolor{currentstroke}%
\pgfsetdash{}{0pt}%
\pgfpathmoveto{\pgfqpoint{3.868079in}{2.816574in}}%
\pgfpathlineto{\pgfqpoint{3.895026in}{2.832132in}}%
\pgfpathlineto{\pgfqpoint{3.883578in}{2.825949in}}%
\pgfpathlineto{\pgfqpoint{3.856631in}{2.810391in}}%
\pgfpathlineto{\pgfqpoint{3.868079in}{2.816574in}}%
\pgfpathclose%
\pgfusepath{fill}%
\end{pgfscope}%
\begin{pgfscope}%
\pgfpathrectangle{\pgfqpoint{0.765000in}{0.660000in}}{\pgfqpoint{4.620000in}{4.620000in}}%
\pgfusepath{clip}%
\pgfsetbuttcap%
\pgfsetroundjoin%
\definecolor{currentfill}{rgb}{1.000000,0.894118,0.788235}%
\pgfsetfillcolor{currentfill}%
\pgfsetlinewidth{0.000000pt}%
\definecolor{currentstroke}{rgb}{1.000000,0.894118,0.788235}%
\pgfsetstrokecolor{currentstroke}%
\pgfsetdash{}{0pt}%
\pgfpathmoveto{\pgfqpoint{3.868079in}{2.878806in}}%
\pgfpathlineto{\pgfqpoint{3.841132in}{2.863248in}}%
\pgfpathlineto{\pgfqpoint{3.829684in}{2.857065in}}%
\pgfpathlineto{\pgfqpoint{3.856631in}{2.872623in}}%
\pgfpathlineto{\pgfqpoint{3.868079in}{2.878806in}}%
\pgfpathclose%
\pgfusepath{fill}%
\end{pgfscope}%
\begin{pgfscope}%
\pgfpathrectangle{\pgfqpoint{0.765000in}{0.660000in}}{\pgfqpoint{4.620000in}{4.620000in}}%
\pgfusepath{clip}%
\pgfsetbuttcap%
\pgfsetroundjoin%
\definecolor{currentfill}{rgb}{1.000000,0.894118,0.788235}%
\pgfsetfillcolor{currentfill}%
\pgfsetlinewidth{0.000000pt}%
\definecolor{currentstroke}{rgb}{1.000000,0.894118,0.788235}%
\pgfsetstrokecolor{currentstroke}%
\pgfsetdash{}{0pt}%
\pgfpathmoveto{\pgfqpoint{3.895026in}{2.863248in}}%
\pgfpathlineto{\pgfqpoint{3.868079in}{2.878806in}}%
\pgfpathlineto{\pgfqpoint{3.856631in}{2.872623in}}%
\pgfpathlineto{\pgfqpoint{3.883578in}{2.857065in}}%
\pgfpathlineto{\pgfqpoint{3.895026in}{2.863248in}}%
\pgfpathclose%
\pgfusepath{fill}%
\end{pgfscope}%
\begin{pgfscope}%
\pgfpathrectangle{\pgfqpoint{0.765000in}{0.660000in}}{\pgfqpoint{4.620000in}{4.620000in}}%
\pgfusepath{clip}%
\pgfsetbuttcap%
\pgfsetroundjoin%
\definecolor{currentfill}{rgb}{1.000000,0.894118,0.788235}%
\pgfsetfillcolor{currentfill}%
\pgfsetlinewidth{0.000000pt}%
\definecolor{currentstroke}{rgb}{1.000000,0.894118,0.788235}%
\pgfsetstrokecolor{currentstroke}%
\pgfsetdash{}{0pt}%
\pgfpathmoveto{\pgfqpoint{3.841132in}{2.832132in}}%
\pgfpathlineto{\pgfqpoint{3.841132in}{2.863248in}}%
\pgfpathlineto{\pgfqpoint{3.829684in}{2.857065in}}%
\pgfpathlineto{\pgfqpoint{3.856631in}{2.810391in}}%
\pgfpathlineto{\pgfqpoint{3.841132in}{2.832132in}}%
\pgfpathclose%
\pgfusepath{fill}%
\end{pgfscope}%
\begin{pgfscope}%
\pgfpathrectangle{\pgfqpoint{0.765000in}{0.660000in}}{\pgfqpoint{4.620000in}{4.620000in}}%
\pgfusepath{clip}%
\pgfsetbuttcap%
\pgfsetroundjoin%
\definecolor{currentfill}{rgb}{1.000000,0.894118,0.788235}%
\pgfsetfillcolor{currentfill}%
\pgfsetlinewidth{0.000000pt}%
\definecolor{currentstroke}{rgb}{1.000000,0.894118,0.788235}%
\pgfsetstrokecolor{currentstroke}%
\pgfsetdash{}{0pt}%
\pgfpathmoveto{\pgfqpoint{3.868079in}{2.816574in}}%
\pgfpathlineto{\pgfqpoint{3.868079in}{2.847690in}}%
\pgfpathlineto{\pgfqpoint{3.856631in}{2.841507in}}%
\pgfpathlineto{\pgfqpoint{3.829684in}{2.825949in}}%
\pgfpathlineto{\pgfqpoint{3.868079in}{2.816574in}}%
\pgfpathclose%
\pgfusepath{fill}%
\end{pgfscope}%
\begin{pgfscope}%
\pgfpathrectangle{\pgfqpoint{0.765000in}{0.660000in}}{\pgfqpoint{4.620000in}{4.620000in}}%
\pgfusepath{clip}%
\pgfsetbuttcap%
\pgfsetroundjoin%
\definecolor{currentfill}{rgb}{1.000000,0.894118,0.788235}%
\pgfsetfillcolor{currentfill}%
\pgfsetlinewidth{0.000000pt}%
\definecolor{currentstroke}{rgb}{1.000000,0.894118,0.788235}%
\pgfsetstrokecolor{currentstroke}%
\pgfsetdash{}{0pt}%
\pgfpathmoveto{\pgfqpoint{3.868079in}{2.847690in}}%
\pgfpathlineto{\pgfqpoint{3.895026in}{2.863248in}}%
\pgfpathlineto{\pgfqpoint{3.883578in}{2.857065in}}%
\pgfpathlineto{\pgfqpoint{3.856631in}{2.841507in}}%
\pgfpathlineto{\pgfqpoint{3.868079in}{2.847690in}}%
\pgfpathclose%
\pgfusepath{fill}%
\end{pgfscope}%
\begin{pgfscope}%
\pgfpathrectangle{\pgfqpoint{0.765000in}{0.660000in}}{\pgfqpoint{4.620000in}{4.620000in}}%
\pgfusepath{clip}%
\pgfsetbuttcap%
\pgfsetroundjoin%
\definecolor{currentfill}{rgb}{1.000000,0.894118,0.788235}%
\pgfsetfillcolor{currentfill}%
\pgfsetlinewidth{0.000000pt}%
\definecolor{currentstroke}{rgb}{1.000000,0.894118,0.788235}%
\pgfsetstrokecolor{currentstroke}%
\pgfsetdash{}{0pt}%
\pgfpathmoveto{\pgfqpoint{3.841132in}{2.863248in}}%
\pgfpathlineto{\pgfqpoint{3.868079in}{2.847690in}}%
\pgfpathlineto{\pgfqpoint{3.856631in}{2.841507in}}%
\pgfpathlineto{\pgfqpoint{3.829684in}{2.857065in}}%
\pgfpathlineto{\pgfqpoint{3.841132in}{2.863248in}}%
\pgfpathclose%
\pgfusepath{fill}%
\end{pgfscope}%
\begin{pgfscope}%
\pgfpathrectangle{\pgfqpoint{0.765000in}{0.660000in}}{\pgfqpoint{4.620000in}{4.620000in}}%
\pgfusepath{clip}%
\pgfsetbuttcap%
\pgfsetroundjoin%
\definecolor{currentfill}{rgb}{1.000000,0.894118,0.788235}%
\pgfsetfillcolor{currentfill}%
\pgfsetlinewidth{0.000000pt}%
\definecolor{currentstroke}{rgb}{1.000000,0.894118,0.788235}%
\pgfsetstrokecolor{currentstroke}%
\pgfsetdash{}{0pt}%
\pgfpathmoveto{\pgfqpoint{3.880311in}{2.858581in}}%
\pgfpathlineto{\pgfqpoint{3.853364in}{2.843023in}}%
\pgfpathlineto{\pgfqpoint{3.880311in}{2.827465in}}%
\pgfpathlineto{\pgfqpoint{3.907258in}{2.843023in}}%
\pgfpathlineto{\pgfqpoint{3.880311in}{2.858581in}}%
\pgfpathclose%
\pgfusepath{fill}%
\end{pgfscope}%
\begin{pgfscope}%
\pgfpathrectangle{\pgfqpoint{0.765000in}{0.660000in}}{\pgfqpoint{4.620000in}{4.620000in}}%
\pgfusepath{clip}%
\pgfsetbuttcap%
\pgfsetroundjoin%
\definecolor{currentfill}{rgb}{1.000000,0.894118,0.788235}%
\pgfsetfillcolor{currentfill}%
\pgfsetlinewidth{0.000000pt}%
\definecolor{currentstroke}{rgb}{1.000000,0.894118,0.788235}%
\pgfsetstrokecolor{currentstroke}%
\pgfsetdash{}{0pt}%
\pgfpathmoveto{\pgfqpoint{3.880311in}{2.858581in}}%
\pgfpathlineto{\pgfqpoint{3.853364in}{2.843023in}}%
\pgfpathlineto{\pgfqpoint{3.853364in}{2.874139in}}%
\pgfpathlineto{\pgfqpoint{3.880311in}{2.889697in}}%
\pgfpathlineto{\pgfqpoint{3.880311in}{2.858581in}}%
\pgfpathclose%
\pgfusepath{fill}%
\end{pgfscope}%
\begin{pgfscope}%
\pgfpathrectangle{\pgfqpoint{0.765000in}{0.660000in}}{\pgfqpoint{4.620000in}{4.620000in}}%
\pgfusepath{clip}%
\pgfsetbuttcap%
\pgfsetroundjoin%
\definecolor{currentfill}{rgb}{1.000000,0.894118,0.788235}%
\pgfsetfillcolor{currentfill}%
\pgfsetlinewidth{0.000000pt}%
\definecolor{currentstroke}{rgb}{1.000000,0.894118,0.788235}%
\pgfsetstrokecolor{currentstroke}%
\pgfsetdash{}{0pt}%
\pgfpathmoveto{\pgfqpoint{3.880311in}{2.858581in}}%
\pgfpathlineto{\pgfqpoint{3.907258in}{2.843023in}}%
\pgfpathlineto{\pgfqpoint{3.907258in}{2.874139in}}%
\pgfpathlineto{\pgfqpoint{3.880311in}{2.889697in}}%
\pgfpathlineto{\pgfqpoint{3.880311in}{2.858581in}}%
\pgfpathclose%
\pgfusepath{fill}%
\end{pgfscope}%
\begin{pgfscope}%
\pgfpathrectangle{\pgfqpoint{0.765000in}{0.660000in}}{\pgfqpoint{4.620000in}{4.620000in}}%
\pgfusepath{clip}%
\pgfsetbuttcap%
\pgfsetroundjoin%
\definecolor{currentfill}{rgb}{1.000000,0.894118,0.788235}%
\pgfsetfillcolor{currentfill}%
\pgfsetlinewidth{0.000000pt}%
\definecolor{currentstroke}{rgb}{1.000000,0.894118,0.788235}%
\pgfsetstrokecolor{currentstroke}%
\pgfsetdash{}{0pt}%
\pgfpathmoveto{\pgfqpoint{3.886454in}{2.865404in}}%
\pgfpathlineto{\pgfqpoint{3.859507in}{2.849846in}}%
\pgfpathlineto{\pgfqpoint{3.886454in}{2.834288in}}%
\pgfpathlineto{\pgfqpoint{3.913402in}{2.849846in}}%
\pgfpathlineto{\pgfqpoint{3.886454in}{2.865404in}}%
\pgfpathclose%
\pgfusepath{fill}%
\end{pgfscope}%
\begin{pgfscope}%
\pgfpathrectangle{\pgfqpoint{0.765000in}{0.660000in}}{\pgfqpoint{4.620000in}{4.620000in}}%
\pgfusepath{clip}%
\pgfsetbuttcap%
\pgfsetroundjoin%
\definecolor{currentfill}{rgb}{1.000000,0.894118,0.788235}%
\pgfsetfillcolor{currentfill}%
\pgfsetlinewidth{0.000000pt}%
\definecolor{currentstroke}{rgb}{1.000000,0.894118,0.788235}%
\pgfsetstrokecolor{currentstroke}%
\pgfsetdash{}{0pt}%
\pgfpathmoveto{\pgfqpoint{3.886454in}{2.865404in}}%
\pgfpathlineto{\pgfqpoint{3.859507in}{2.849846in}}%
\pgfpathlineto{\pgfqpoint{3.859507in}{2.880962in}}%
\pgfpathlineto{\pgfqpoint{3.886454in}{2.896520in}}%
\pgfpathlineto{\pgfqpoint{3.886454in}{2.865404in}}%
\pgfpathclose%
\pgfusepath{fill}%
\end{pgfscope}%
\begin{pgfscope}%
\pgfpathrectangle{\pgfqpoint{0.765000in}{0.660000in}}{\pgfqpoint{4.620000in}{4.620000in}}%
\pgfusepath{clip}%
\pgfsetbuttcap%
\pgfsetroundjoin%
\definecolor{currentfill}{rgb}{1.000000,0.894118,0.788235}%
\pgfsetfillcolor{currentfill}%
\pgfsetlinewidth{0.000000pt}%
\definecolor{currentstroke}{rgb}{1.000000,0.894118,0.788235}%
\pgfsetstrokecolor{currentstroke}%
\pgfsetdash{}{0pt}%
\pgfpathmoveto{\pgfqpoint{3.886454in}{2.865404in}}%
\pgfpathlineto{\pgfqpoint{3.913402in}{2.849846in}}%
\pgfpathlineto{\pgfqpoint{3.913402in}{2.880962in}}%
\pgfpathlineto{\pgfqpoint{3.886454in}{2.896520in}}%
\pgfpathlineto{\pgfqpoint{3.886454in}{2.865404in}}%
\pgfpathclose%
\pgfusepath{fill}%
\end{pgfscope}%
\begin{pgfscope}%
\pgfpathrectangle{\pgfqpoint{0.765000in}{0.660000in}}{\pgfqpoint{4.620000in}{4.620000in}}%
\pgfusepath{clip}%
\pgfsetbuttcap%
\pgfsetroundjoin%
\definecolor{currentfill}{rgb}{1.000000,0.894118,0.788235}%
\pgfsetfillcolor{currentfill}%
\pgfsetlinewidth{0.000000pt}%
\definecolor{currentstroke}{rgb}{1.000000,0.894118,0.788235}%
\pgfsetstrokecolor{currentstroke}%
\pgfsetdash{}{0pt}%
\pgfpathmoveto{\pgfqpoint{3.880311in}{2.889697in}}%
\pgfpathlineto{\pgfqpoint{3.853364in}{2.874139in}}%
\pgfpathlineto{\pgfqpoint{3.880311in}{2.858581in}}%
\pgfpathlineto{\pgfqpoint{3.907258in}{2.874139in}}%
\pgfpathlineto{\pgfqpoint{3.880311in}{2.889697in}}%
\pgfpathclose%
\pgfusepath{fill}%
\end{pgfscope}%
\begin{pgfscope}%
\pgfpathrectangle{\pgfqpoint{0.765000in}{0.660000in}}{\pgfqpoint{4.620000in}{4.620000in}}%
\pgfusepath{clip}%
\pgfsetbuttcap%
\pgfsetroundjoin%
\definecolor{currentfill}{rgb}{1.000000,0.894118,0.788235}%
\pgfsetfillcolor{currentfill}%
\pgfsetlinewidth{0.000000pt}%
\definecolor{currentstroke}{rgb}{1.000000,0.894118,0.788235}%
\pgfsetstrokecolor{currentstroke}%
\pgfsetdash{}{0pt}%
\pgfpathmoveto{\pgfqpoint{3.880311in}{2.827465in}}%
\pgfpathlineto{\pgfqpoint{3.907258in}{2.843023in}}%
\pgfpathlineto{\pgfqpoint{3.907258in}{2.874139in}}%
\pgfpathlineto{\pgfqpoint{3.880311in}{2.858581in}}%
\pgfpathlineto{\pgfqpoint{3.880311in}{2.827465in}}%
\pgfpathclose%
\pgfusepath{fill}%
\end{pgfscope}%
\begin{pgfscope}%
\pgfpathrectangle{\pgfqpoint{0.765000in}{0.660000in}}{\pgfqpoint{4.620000in}{4.620000in}}%
\pgfusepath{clip}%
\pgfsetbuttcap%
\pgfsetroundjoin%
\definecolor{currentfill}{rgb}{1.000000,0.894118,0.788235}%
\pgfsetfillcolor{currentfill}%
\pgfsetlinewidth{0.000000pt}%
\definecolor{currentstroke}{rgb}{1.000000,0.894118,0.788235}%
\pgfsetstrokecolor{currentstroke}%
\pgfsetdash{}{0pt}%
\pgfpathmoveto{\pgfqpoint{3.853364in}{2.843023in}}%
\pgfpathlineto{\pgfqpoint{3.880311in}{2.827465in}}%
\pgfpathlineto{\pgfqpoint{3.880311in}{2.858581in}}%
\pgfpathlineto{\pgfqpoint{3.853364in}{2.874139in}}%
\pgfpathlineto{\pgfqpoint{3.853364in}{2.843023in}}%
\pgfpathclose%
\pgfusepath{fill}%
\end{pgfscope}%
\begin{pgfscope}%
\pgfpathrectangle{\pgfqpoint{0.765000in}{0.660000in}}{\pgfqpoint{4.620000in}{4.620000in}}%
\pgfusepath{clip}%
\pgfsetbuttcap%
\pgfsetroundjoin%
\definecolor{currentfill}{rgb}{1.000000,0.894118,0.788235}%
\pgfsetfillcolor{currentfill}%
\pgfsetlinewidth{0.000000pt}%
\definecolor{currentstroke}{rgb}{1.000000,0.894118,0.788235}%
\pgfsetstrokecolor{currentstroke}%
\pgfsetdash{}{0pt}%
\pgfpathmoveto{\pgfqpoint{3.886454in}{2.896520in}}%
\pgfpathlineto{\pgfqpoint{3.859507in}{2.880962in}}%
\pgfpathlineto{\pgfqpoint{3.886454in}{2.865404in}}%
\pgfpathlineto{\pgfqpoint{3.913402in}{2.880962in}}%
\pgfpathlineto{\pgfqpoint{3.886454in}{2.896520in}}%
\pgfpathclose%
\pgfusepath{fill}%
\end{pgfscope}%
\begin{pgfscope}%
\pgfpathrectangle{\pgfqpoint{0.765000in}{0.660000in}}{\pgfqpoint{4.620000in}{4.620000in}}%
\pgfusepath{clip}%
\pgfsetbuttcap%
\pgfsetroundjoin%
\definecolor{currentfill}{rgb}{1.000000,0.894118,0.788235}%
\pgfsetfillcolor{currentfill}%
\pgfsetlinewidth{0.000000pt}%
\definecolor{currentstroke}{rgb}{1.000000,0.894118,0.788235}%
\pgfsetstrokecolor{currentstroke}%
\pgfsetdash{}{0pt}%
\pgfpathmoveto{\pgfqpoint{3.886454in}{2.834288in}}%
\pgfpathlineto{\pgfqpoint{3.913402in}{2.849846in}}%
\pgfpathlineto{\pgfqpoint{3.913402in}{2.880962in}}%
\pgfpathlineto{\pgfqpoint{3.886454in}{2.865404in}}%
\pgfpathlineto{\pgfqpoint{3.886454in}{2.834288in}}%
\pgfpathclose%
\pgfusepath{fill}%
\end{pgfscope}%
\begin{pgfscope}%
\pgfpathrectangle{\pgfqpoint{0.765000in}{0.660000in}}{\pgfqpoint{4.620000in}{4.620000in}}%
\pgfusepath{clip}%
\pgfsetbuttcap%
\pgfsetroundjoin%
\definecolor{currentfill}{rgb}{1.000000,0.894118,0.788235}%
\pgfsetfillcolor{currentfill}%
\pgfsetlinewidth{0.000000pt}%
\definecolor{currentstroke}{rgb}{1.000000,0.894118,0.788235}%
\pgfsetstrokecolor{currentstroke}%
\pgfsetdash{}{0pt}%
\pgfpathmoveto{\pgfqpoint{3.859507in}{2.849846in}}%
\pgfpathlineto{\pgfqpoint{3.886454in}{2.834288in}}%
\pgfpathlineto{\pgfqpoint{3.886454in}{2.865404in}}%
\pgfpathlineto{\pgfqpoint{3.859507in}{2.880962in}}%
\pgfpathlineto{\pgfqpoint{3.859507in}{2.849846in}}%
\pgfpathclose%
\pgfusepath{fill}%
\end{pgfscope}%
\begin{pgfscope}%
\pgfpathrectangle{\pgfqpoint{0.765000in}{0.660000in}}{\pgfqpoint{4.620000in}{4.620000in}}%
\pgfusepath{clip}%
\pgfsetbuttcap%
\pgfsetroundjoin%
\definecolor{currentfill}{rgb}{1.000000,0.894118,0.788235}%
\pgfsetfillcolor{currentfill}%
\pgfsetlinewidth{0.000000pt}%
\definecolor{currentstroke}{rgb}{1.000000,0.894118,0.788235}%
\pgfsetstrokecolor{currentstroke}%
\pgfsetdash{}{0pt}%
\pgfpathmoveto{\pgfqpoint{3.880311in}{2.858581in}}%
\pgfpathlineto{\pgfqpoint{3.853364in}{2.843023in}}%
\pgfpathlineto{\pgfqpoint{3.859507in}{2.849846in}}%
\pgfpathlineto{\pgfqpoint{3.886454in}{2.865404in}}%
\pgfpathlineto{\pgfqpoint{3.880311in}{2.858581in}}%
\pgfpathclose%
\pgfusepath{fill}%
\end{pgfscope}%
\begin{pgfscope}%
\pgfpathrectangle{\pgfqpoint{0.765000in}{0.660000in}}{\pgfqpoint{4.620000in}{4.620000in}}%
\pgfusepath{clip}%
\pgfsetbuttcap%
\pgfsetroundjoin%
\definecolor{currentfill}{rgb}{1.000000,0.894118,0.788235}%
\pgfsetfillcolor{currentfill}%
\pgfsetlinewidth{0.000000pt}%
\definecolor{currentstroke}{rgb}{1.000000,0.894118,0.788235}%
\pgfsetstrokecolor{currentstroke}%
\pgfsetdash{}{0pt}%
\pgfpathmoveto{\pgfqpoint{3.907258in}{2.843023in}}%
\pgfpathlineto{\pgfqpoint{3.880311in}{2.858581in}}%
\pgfpathlineto{\pgfqpoint{3.886454in}{2.865404in}}%
\pgfpathlineto{\pgfqpoint{3.913402in}{2.849846in}}%
\pgfpathlineto{\pgfqpoint{3.907258in}{2.843023in}}%
\pgfpathclose%
\pgfusepath{fill}%
\end{pgfscope}%
\begin{pgfscope}%
\pgfpathrectangle{\pgfqpoint{0.765000in}{0.660000in}}{\pgfqpoint{4.620000in}{4.620000in}}%
\pgfusepath{clip}%
\pgfsetbuttcap%
\pgfsetroundjoin%
\definecolor{currentfill}{rgb}{1.000000,0.894118,0.788235}%
\pgfsetfillcolor{currentfill}%
\pgfsetlinewidth{0.000000pt}%
\definecolor{currentstroke}{rgb}{1.000000,0.894118,0.788235}%
\pgfsetstrokecolor{currentstroke}%
\pgfsetdash{}{0pt}%
\pgfpathmoveto{\pgfqpoint{3.880311in}{2.858581in}}%
\pgfpathlineto{\pgfqpoint{3.880311in}{2.889697in}}%
\pgfpathlineto{\pgfqpoint{3.886454in}{2.896520in}}%
\pgfpathlineto{\pgfqpoint{3.913402in}{2.849846in}}%
\pgfpathlineto{\pgfqpoint{3.880311in}{2.858581in}}%
\pgfpathclose%
\pgfusepath{fill}%
\end{pgfscope}%
\begin{pgfscope}%
\pgfpathrectangle{\pgfqpoint{0.765000in}{0.660000in}}{\pgfqpoint{4.620000in}{4.620000in}}%
\pgfusepath{clip}%
\pgfsetbuttcap%
\pgfsetroundjoin%
\definecolor{currentfill}{rgb}{1.000000,0.894118,0.788235}%
\pgfsetfillcolor{currentfill}%
\pgfsetlinewidth{0.000000pt}%
\definecolor{currentstroke}{rgb}{1.000000,0.894118,0.788235}%
\pgfsetstrokecolor{currentstroke}%
\pgfsetdash{}{0pt}%
\pgfpathmoveto{\pgfqpoint{3.907258in}{2.843023in}}%
\pgfpathlineto{\pgfqpoint{3.907258in}{2.874139in}}%
\pgfpathlineto{\pgfqpoint{3.913402in}{2.880962in}}%
\pgfpathlineto{\pgfqpoint{3.886454in}{2.865404in}}%
\pgfpathlineto{\pgfqpoint{3.907258in}{2.843023in}}%
\pgfpathclose%
\pgfusepath{fill}%
\end{pgfscope}%
\begin{pgfscope}%
\pgfpathrectangle{\pgfqpoint{0.765000in}{0.660000in}}{\pgfqpoint{4.620000in}{4.620000in}}%
\pgfusepath{clip}%
\pgfsetbuttcap%
\pgfsetroundjoin%
\definecolor{currentfill}{rgb}{1.000000,0.894118,0.788235}%
\pgfsetfillcolor{currentfill}%
\pgfsetlinewidth{0.000000pt}%
\definecolor{currentstroke}{rgb}{1.000000,0.894118,0.788235}%
\pgfsetstrokecolor{currentstroke}%
\pgfsetdash{}{0pt}%
\pgfpathmoveto{\pgfqpoint{3.853364in}{2.843023in}}%
\pgfpathlineto{\pgfqpoint{3.880311in}{2.827465in}}%
\pgfpathlineto{\pgfqpoint{3.886454in}{2.834288in}}%
\pgfpathlineto{\pgfqpoint{3.859507in}{2.849846in}}%
\pgfpathlineto{\pgfqpoint{3.853364in}{2.843023in}}%
\pgfpathclose%
\pgfusepath{fill}%
\end{pgfscope}%
\begin{pgfscope}%
\pgfpathrectangle{\pgfqpoint{0.765000in}{0.660000in}}{\pgfqpoint{4.620000in}{4.620000in}}%
\pgfusepath{clip}%
\pgfsetbuttcap%
\pgfsetroundjoin%
\definecolor{currentfill}{rgb}{1.000000,0.894118,0.788235}%
\pgfsetfillcolor{currentfill}%
\pgfsetlinewidth{0.000000pt}%
\definecolor{currentstroke}{rgb}{1.000000,0.894118,0.788235}%
\pgfsetstrokecolor{currentstroke}%
\pgfsetdash{}{0pt}%
\pgfpathmoveto{\pgfqpoint{3.880311in}{2.827465in}}%
\pgfpathlineto{\pgfqpoint{3.907258in}{2.843023in}}%
\pgfpathlineto{\pgfqpoint{3.913402in}{2.849846in}}%
\pgfpathlineto{\pgfqpoint{3.886454in}{2.834288in}}%
\pgfpathlineto{\pgfqpoint{3.880311in}{2.827465in}}%
\pgfpathclose%
\pgfusepath{fill}%
\end{pgfscope}%
\begin{pgfscope}%
\pgfpathrectangle{\pgfqpoint{0.765000in}{0.660000in}}{\pgfqpoint{4.620000in}{4.620000in}}%
\pgfusepath{clip}%
\pgfsetbuttcap%
\pgfsetroundjoin%
\definecolor{currentfill}{rgb}{1.000000,0.894118,0.788235}%
\pgfsetfillcolor{currentfill}%
\pgfsetlinewidth{0.000000pt}%
\definecolor{currentstroke}{rgb}{1.000000,0.894118,0.788235}%
\pgfsetstrokecolor{currentstroke}%
\pgfsetdash{}{0pt}%
\pgfpathmoveto{\pgfqpoint{3.880311in}{2.889697in}}%
\pgfpathlineto{\pgfqpoint{3.853364in}{2.874139in}}%
\pgfpathlineto{\pgfqpoint{3.859507in}{2.880962in}}%
\pgfpathlineto{\pgfqpoint{3.886454in}{2.896520in}}%
\pgfpathlineto{\pgfqpoint{3.880311in}{2.889697in}}%
\pgfpathclose%
\pgfusepath{fill}%
\end{pgfscope}%
\begin{pgfscope}%
\pgfpathrectangle{\pgfqpoint{0.765000in}{0.660000in}}{\pgfqpoint{4.620000in}{4.620000in}}%
\pgfusepath{clip}%
\pgfsetbuttcap%
\pgfsetroundjoin%
\definecolor{currentfill}{rgb}{1.000000,0.894118,0.788235}%
\pgfsetfillcolor{currentfill}%
\pgfsetlinewidth{0.000000pt}%
\definecolor{currentstroke}{rgb}{1.000000,0.894118,0.788235}%
\pgfsetstrokecolor{currentstroke}%
\pgfsetdash{}{0pt}%
\pgfpathmoveto{\pgfqpoint{3.907258in}{2.874139in}}%
\pgfpathlineto{\pgfqpoint{3.880311in}{2.889697in}}%
\pgfpathlineto{\pgfqpoint{3.886454in}{2.896520in}}%
\pgfpathlineto{\pgfqpoint{3.913402in}{2.880962in}}%
\pgfpathlineto{\pgfqpoint{3.907258in}{2.874139in}}%
\pgfpathclose%
\pgfusepath{fill}%
\end{pgfscope}%
\begin{pgfscope}%
\pgfpathrectangle{\pgfqpoint{0.765000in}{0.660000in}}{\pgfqpoint{4.620000in}{4.620000in}}%
\pgfusepath{clip}%
\pgfsetbuttcap%
\pgfsetroundjoin%
\definecolor{currentfill}{rgb}{1.000000,0.894118,0.788235}%
\pgfsetfillcolor{currentfill}%
\pgfsetlinewidth{0.000000pt}%
\definecolor{currentstroke}{rgb}{1.000000,0.894118,0.788235}%
\pgfsetstrokecolor{currentstroke}%
\pgfsetdash{}{0pt}%
\pgfpathmoveto{\pgfqpoint{3.853364in}{2.843023in}}%
\pgfpathlineto{\pgfqpoint{3.853364in}{2.874139in}}%
\pgfpathlineto{\pgfqpoint{3.859507in}{2.880962in}}%
\pgfpathlineto{\pgfqpoint{3.886454in}{2.834288in}}%
\pgfpathlineto{\pgfqpoint{3.853364in}{2.843023in}}%
\pgfpathclose%
\pgfusepath{fill}%
\end{pgfscope}%
\begin{pgfscope}%
\pgfpathrectangle{\pgfqpoint{0.765000in}{0.660000in}}{\pgfqpoint{4.620000in}{4.620000in}}%
\pgfusepath{clip}%
\pgfsetbuttcap%
\pgfsetroundjoin%
\definecolor{currentfill}{rgb}{1.000000,0.894118,0.788235}%
\pgfsetfillcolor{currentfill}%
\pgfsetlinewidth{0.000000pt}%
\definecolor{currentstroke}{rgb}{1.000000,0.894118,0.788235}%
\pgfsetstrokecolor{currentstroke}%
\pgfsetdash{}{0pt}%
\pgfpathmoveto{\pgfqpoint{3.880311in}{2.827465in}}%
\pgfpathlineto{\pgfqpoint{3.880311in}{2.858581in}}%
\pgfpathlineto{\pgfqpoint{3.886454in}{2.865404in}}%
\pgfpathlineto{\pgfqpoint{3.859507in}{2.849846in}}%
\pgfpathlineto{\pgfqpoint{3.880311in}{2.827465in}}%
\pgfpathclose%
\pgfusepath{fill}%
\end{pgfscope}%
\begin{pgfscope}%
\pgfpathrectangle{\pgfqpoint{0.765000in}{0.660000in}}{\pgfqpoint{4.620000in}{4.620000in}}%
\pgfusepath{clip}%
\pgfsetbuttcap%
\pgfsetroundjoin%
\definecolor{currentfill}{rgb}{1.000000,0.894118,0.788235}%
\pgfsetfillcolor{currentfill}%
\pgfsetlinewidth{0.000000pt}%
\definecolor{currentstroke}{rgb}{1.000000,0.894118,0.788235}%
\pgfsetstrokecolor{currentstroke}%
\pgfsetdash{}{0pt}%
\pgfpathmoveto{\pgfqpoint{3.880311in}{2.858581in}}%
\pgfpathlineto{\pgfqpoint{3.907258in}{2.874139in}}%
\pgfpathlineto{\pgfqpoint{3.913402in}{2.880962in}}%
\pgfpathlineto{\pgfqpoint{3.886454in}{2.865404in}}%
\pgfpathlineto{\pgfqpoint{3.880311in}{2.858581in}}%
\pgfpathclose%
\pgfusepath{fill}%
\end{pgfscope}%
\begin{pgfscope}%
\pgfpathrectangle{\pgfqpoint{0.765000in}{0.660000in}}{\pgfqpoint{4.620000in}{4.620000in}}%
\pgfusepath{clip}%
\pgfsetbuttcap%
\pgfsetroundjoin%
\definecolor{currentfill}{rgb}{1.000000,0.894118,0.788235}%
\pgfsetfillcolor{currentfill}%
\pgfsetlinewidth{0.000000pt}%
\definecolor{currentstroke}{rgb}{1.000000,0.894118,0.788235}%
\pgfsetstrokecolor{currentstroke}%
\pgfsetdash{}{0pt}%
\pgfpathmoveto{\pgfqpoint{3.853364in}{2.874139in}}%
\pgfpathlineto{\pgfqpoint{3.880311in}{2.858581in}}%
\pgfpathlineto{\pgfqpoint{3.886454in}{2.865404in}}%
\pgfpathlineto{\pgfqpoint{3.859507in}{2.880962in}}%
\pgfpathlineto{\pgfqpoint{3.853364in}{2.874139in}}%
\pgfpathclose%
\pgfusepath{fill}%
\end{pgfscope}%
\begin{pgfscope}%
\pgfpathrectangle{\pgfqpoint{0.765000in}{0.660000in}}{\pgfqpoint{4.620000in}{4.620000in}}%
\pgfusepath{clip}%
\pgfsetbuttcap%
\pgfsetroundjoin%
\definecolor{currentfill}{rgb}{1.000000,0.894118,0.788235}%
\pgfsetfillcolor{currentfill}%
\pgfsetlinewidth{0.000000pt}%
\definecolor{currentstroke}{rgb}{1.000000,0.894118,0.788235}%
\pgfsetstrokecolor{currentstroke}%
\pgfsetdash{}{0pt}%
\pgfpathmoveto{\pgfqpoint{3.880311in}{2.858581in}}%
\pgfpathlineto{\pgfqpoint{3.853364in}{2.843023in}}%
\pgfpathlineto{\pgfqpoint{3.880311in}{2.827465in}}%
\pgfpathlineto{\pgfqpoint{3.907258in}{2.843023in}}%
\pgfpathlineto{\pgfqpoint{3.880311in}{2.858581in}}%
\pgfpathclose%
\pgfusepath{fill}%
\end{pgfscope}%
\begin{pgfscope}%
\pgfpathrectangle{\pgfqpoint{0.765000in}{0.660000in}}{\pgfqpoint{4.620000in}{4.620000in}}%
\pgfusepath{clip}%
\pgfsetbuttcap%
\pgfsetroundjoin%
\definecolor{currentfill}{rgb}{1.000000,0.894118,0.788235}%
\pgfsetfillcolor{currentfill}%
\pgfsetlinewidth{0.000000pt}%
\definecolor{currentstroke}{rgb}{1.000000,0.894118,0.788235}%
\pgfsetstrokecolor{currentstroke}%
\pgfsetdash{}{0pt}%
\pgfpathmoveto{\pgfqpoint{3.880311in}{2.858581in}}%
\pgfpathlineto{\pgfqpoint{3.853364in}{2.843023in}}%
\pgfpathlineto{\pgfqpoint{3.853364in}{2.874139in}}%
\pgfpathlineto{\pgfqpoint{3.880311in}{2.889697in}}%
\pgfpathlineto{\pgfqpoint{3.880311in}{2.858581in}}%
\pgfpathclose%
\pgfusepath{fill}%
\end{pgfscope}%
\begin{pgfscope}%
\pgfpathrectangle{\pgfqpoint{0.765000in}{0.660000in}}{\pgfqpoint{4.620000in}{4.620000in}}%
\pgfusepath{clip}%
\pgfsetbuttcap%
\pgfsetroundjoin%
\definecolor{currentfill}{rgb}{1.000000,0.894118,0.788235}%
\pgfsetfillcolor{currentfill}%
\pgfsetlinewidth{0.000000pt}%
\definecolor{currentstroke}{rgb}{1.000000,0.894118,0.788235}%
\pgfsetstrokecolor{currentstroke}%
\pgfsetdash{}{0pt}%
\pgfpathmoveto{\pgfqpoint{3.880311in}{2.858581in}}%
\pgfpathlineto{\pgfqpoint{3.907258in}{2.843023in}}%
\pgfpathlineto{\pgfqpoint{3.907258in}{2.874139in}}%
\pgfpathlineto{\pgfqpoint{3.880311in}{2.889697in}}%
\pgfpathlineto{\pgfqpoint{3.880311in}{2.858581in}}%
\pgfpathclose%
\pgfusepath{fill}%
\end{pgfscope}%
\begin{pgfscope}%
\pgfpathrectangle{\pgfqpoint{0.765000in}{0.660000in}}{\pgfqpoint{4.620000in}{4.620000in}}%
\pgfusepath{clip}%
\pgfsetbuttcap%
\pgfsetroundjoin%
\definecolor{currentfill}{rgb}{1.000000,0.894118,0.788235}%
\pgfsetfillcolor{currentfill}%
\pgfsetlinewidth{0.000000pt}%
\definecolor{currentstroke}{rgb}{1.000000,0.894118,0.788235}%
\pgfsetstrokecolor{currentstroke}%
\pgfsetdash{}{0pt}%
\pgfpathmoveto{\pgfqpoint{3.868079in}{2.847690in}}%
\pgfpathlineto{\pgfqpoint{3.841132in}{2.832132in}}%
\pgfpathlineto{\pgfqpoint{3.868079in}{2.816574in}}%
\pgfpathlineto{\pgfqpoint{3.895026in}{2.832132in}}%
\pgfpathlineto{\pgfqpoint{3.868079in}{2.847690in}}%
\pgfpathclose%
\pgfusepath{fill}%
\end{pgfscope}%
\begin{pgfscope}%
\pgfpathrectangle{\pgfqpoint{0.765000in}{0.660000in}}{\pgfqpoint{4.620000in}{4.620000in}}%
\pgfusepath{clip}%
\pgfsetbuttcap%
\pgfsetroundjoin%
\definecolor{currentfill}{rgb}{1.000000,0.894118,0.788235}%
\pgfsetfillcolor{currentfill}%
\pgfsetlinewidth{0.000000pt}%
\definecolor{currentstroke}{rgb}{1.000000,0.894118,0.788235}%
\pgfsetstrokecolor{currentstroke}%
\pgfsetdash{}{0pt}%
\pgfpathmoveto{\pgfqpoint{3.868079in}{2.847690in}}%
\pgfpathlineto{\pgfqpoint{3.841132in}{2.832132in}}%
\pgfpathlineto{\pgfqpoint{3.841132in}{2.863248in}}%
\pgfpathlineto{\pgfqpoint{3.868079in}{2.878806in}}%
\pgfpathlineto{\pgfqpoint{3.868079in}{2.847690in}}%
\pgfpathclose%
\pgfusepath{fill}%
\end{pgfscope}%
\begin{pgfscope}%
\pgfpathrectangle{\pgfqpoint{0.765000in}{0.660000in}}{\pgfqpoint{4.620000in}{4.620000in}}%
\pgfusepath{clip}%
\pgfsetbuttcap%
\pgfsetroundjoin%
\definecolor{currentfill}{rgb}{1.000000,0.894118,0.788235}%
\pgfsetfillcolor{currentfill}%
\pgfsetlinewidth{0.000000pt}%
\definecolor{currentstroke}{rgb}{1.000000,0.894118,0.788235}%
\pgfsetstrokecolor{currentstroke}%
\pgfsetdash{}{0pt}%
\pgfpathmoveto{\pgfqpoint{3.868079in}{2.847690in}}%
\pgfpathlineto{\pgfqpoint{3.895026in}{2.832132in}}%
\pgfpathlineto{\pgfqpoint{3.895026in}{2.863248in}}%
\pgfpathlineto{\pgfqpoint{3.868079in}{2.878806in}}%
\pgfpathlineto{\pgfqpoint{3.868079in}{2.847690in}}%
\pgfpathclose%
\pgfusepath{fill}%
\end{pgfscope}%
\begin{pgfscope}%
\pgfpathrectangle{\pgfqpoint{0.765000in}{0.660000in}}{\pgfqpoint{4.620000in}{4.620000in}}%
\pgfusepath{clip}%
\pgfsetbuttcap%
\pgfsetroundjoin%
\definecolor{currentfill}{rgb}{1.000000,0.894118,0.788235}%
\pgfsetfillcolor{currentfill}%
\pgfsetlinewidth{0.000000pt}%
\definecolor{currentstroke}{rgb}{1.000000,0.894118,0.788235}%
\pgfsetstrokecolor{currentstroke}%
\pgfsetdash{}{0pt}%
\pgfpathmoveto{\pgfqpoint{3.880311in}{2.889697in}}%
\pgfpathlineto{\pgfqpoint{3.853364in}{2.874139in}}%
\pgfpathlineto{\pgfqpoint{3.880311in}{2.858581in}}%
\pgfpathlineto{\pgfqpoint{3.907258in}{2.874139in}}%
\pgfpathlineto{\pgfqpoint{3.880311in}{2.889697in}}%
\pgfpathclose%
\pgfusepath{fill}%
\end{pgfscope}%
\begin{pgfscope}%
\pgfpathrectangle{\pgfqpoint{0.765000in}{0.660000in}}{\pgfqpoint{4.620000in}{4.620000in}}%
\pgfusepath{clip}%
\pgfsetbuttcap%
\pgfsetroundjoin%
\definecolor{currentfill}{rgb}{1.000000,0.894118,0.788235}%
\pgfsetfillcolor{currentfill}%
\pgfsetlinewidth{0.000000pt}%
\definecolor{currentstroke}{rgb}{1.000000,0.894118,0.788235}%
\pgfsetstrokecolor{currentstroke}%
\pgfsetdash{}{0pt}%
\pgfpathmoveto{\pgfqpoint{3.880311in}{2.827465in}}%
\pgfpathlineto{\pgfqpoint{3.907258in}{2.843023in}}%
\pgfpathlineto{\pgfqpoint{3.907258in}{2.874139in}}%
\pgfpathlineto{\pgfqpoint{3.880311in}{2.858581in}}%
\pgfpathlineto{\pgfqpoint{3.880311in}{2.827465in}}%
\pgfpathclose%
\pgfusepath{fill}%
\end{pgfscope}%
\begin{pgfscope}%
\pgfpathrectangle{\pgfqpoint{0.765000in}{0.660000in}}{\pgfqpoint{4.620000in}{4.620000in}}%
\pgfusepath{clip}%
\pgfsetbuttcap%
\pgfsetroundjoin%
\definecolor{currentfill}{rgb}{1.000000,0.894118,0.788235}%
\pgfsetfillcolor{currentfill}%
\pgfsetlinewidth{0.000000pt}%
\definecolor{currentstroke}{rgb}{1.000000,0.894118,0.788235}%
\pgfsetstrokecolor{currentstroke}%
\pgfsetdash{}{0pt}%
\pgfpathmoveto{\pgfqpoint{3.853364in}{2.843023in}}%
\pgfpathlineto{\pgfqpoint{3.880311in}{2.827465in}}%
\pgfpathlineto{\pgfqpoint{3.880311in}{2.858581in}}%
\pgfpathlineto{\pgfqpoint{3.853364in}{2.874139in}}%
\pgfpathlineto{\pgfqpoint{3.853364in}{2.843023in}}%
\pgfpathclose%
\pgfusepath{fill}%
\end{pgfscope}%
\begin{pgfscope}%
\pgfpathrectangle{\pgfqpoint{0.765000in}{0.660000in}}{\pgfqpoint{4.620000in}{4.620000in}}%
\pgfusepath{clip}%
\pgfsetbuttcap%
\pgfsetroundjoin%
\definecolor{currentfill}{rgb}{1.000000,0.894118,0.788235}%
\pgfsetfillcolor{currentfill}%
\pgfsetlinewidth{0.000000pt}%
\definecolor{currentstroke}{rgb}{1.000000,0.894118,0.788235}%
\pgfsetstrokecolor{currentstroke}%
\pgfsetdash{}{0pt}%
\pgfpathmoveto{\pgfqpoint{3.868079in}{2.878806in}}%
\pgfpathlineto{\pgfqpoint{3.841132in}{2.863248in}}%
\pgfpathlineto{\pgfqpoint{3.868079in}{2.847690in}}%
\pgfpathlineto{\pgfqpoint{3.895026in}{2.863248in}}%
\pgfpathlineto{\pgfqpoint{3.868079in}{2.878806in}}%
\pgfpathclose%
\pgfusepath{fill}%
\end{pgfscope}%
\begin{pgfscope}%
\pgfpathrectangle{\pgfqpoint{0.765000in}{0.660000in}}{\pgfqpoint{4.620000in}{4.620000in}}%
\pgfusepath{clip}%
\pgfsetbuttcap%
\pgfsetroundjoin%
\definecolor{currentfill}{rgb}{1.000000,0.894118,0.788235}%
\pgfsetfillcolor{currentfill}%
\pgfsetlinewidth{0.000000pt}%
\definecolor{currentstroke}{rgb}{1.000000,0.894118,0.788235}%
\pgfsetstrokecolor{currentstroke}%
\pgfsetdash{}{0pt}%
\pgfpathmoveto{\pgfqpoint{3.868079in}{2.816574in}}%
\pgfpathlineto{\pgfqpoint{3.895026in}{2.832132in}}%
\pgfpathlineto{\pgfqpoint{3.895026in}{2.863248in}}%
\pgfpathlineto{\pgfqpoint{3.868079in}{2.847690in}}%
\pgfpathlineto{\pgfqpoint{3.868079in}{2.816574in}}%
\pgfpathclose%
\pgfusepath{fill}%
\end{pgfscope}%
\begin{pgfscope}%
\pgfpathrectangle{\pgfqpoint{0.765000in}{0.660000in}}{\pgfqpoint{4.620000in}{4.620000in}}%
\pgfusepath{clip}%
\pgfsetbuttcap%
\pgfsetroundjoin%
\definecolor{currentfill}{rgb}{1.000000,0.894118,0.788235}%
\pgfsetfillcolor{currentfill}%
\pgfsetlinewidth{0.000000pt}%
\definecolor{currentstroke}{rgb}{1.000000,0.894118,0.788235}%
\pgfsetstrokecolor{currentstroke}%
\pgfsetdash{}{0pt}%
\pgfpathmoveto{\pgfqpoint{3.841132in}{2.832132in}}%
\pgfpathlineto{\pgfqpoint{3.868079in}{2.816574in}}%
\pgfpathlineto{\pgfqpoint{3.868079in}{2.847690in}}%
\pgfpathlineto{\pgfqpoint{3.841132in}{2.863248in}}%
\pgfpathlineto{\pgfqpoint{3.841132in}{2.832132in}}%
\pgfpathclose%
\pgfusepath{fill}%
\end{pgfscope}%
\begin{pgfscope}%
\pgfpathrectangle{\pgfqpoint{0.765000in}{0.660000in}}{\pgfqpoint{4.620000in}{4.620000in}}%
\pgfusepath{clip}%
\pgfsetbuttcap%
\pgfsetroundjoin%
\definecolor{currentfill}{rgb}{1.000000,0.894118,0.788235}%
\pgfsetfillcolor{currentfill}%
\pgfsetlinewidth{0.000000pt}%
\definecolor{currentstroke}{rgb}{1.000000,0.894118,0.788235}%
\pgfsetstrokecolor{currentstroke}%
\pgfsetdash{}{0pt}%
\pgfpathmoveto{\pgfqpoint{3.880311in}{2.858581in}}%
\pgfpathlineto{\pgfqpoint{3.853364in}{2.843023in}}%
\pgfpathlineto{\pgfqpoint{3.841132in}{2.832132in}}%
\pgfpathlineto{\pgfqpoint{3.868079in}{2.847690in}}%
\pgfpathlineto{\pgfqpoint{3.880311in}{2.858581in}}%
\pgfpathclose%
\pgfusepath{fill}%
\end{pgfscope}%
\begin{pgfscope}%
\pgfpathrectangle{\pgfqpoint{0.765000in}{0.660000in}}{\pgfqpoint{4.620000in}{4.620000in}}%
\pgfusepath{clip}%
\pgfsetbuttcap%
\pgfsetroundjoin%
\definecolor{currentfill}{rgb}{1.000000,0.894118,0.788235}%
\pgfsetfillcolor{currentfill}%
\pgfsetlinewidth{0.000000pt}%
\definecolor{currentstroke}{rgb}{1.000000,0.894118,0.788235}%
\pgfsetstrokecolor{currentstroke}%
\pgfsetdash{}{0pt}%
\pgfpathmoveto{\pgfqpoint{3.907258in}{2.843023in}}%
\pgfpathlineto{\pgfqpoint{3.880311in}{2.858581in}}%
\pgfpathlineto{\pgfqpoint{3.868079in}{2.847690in}}%
\pgfpathlineto{\pgfqpoint{3.895026in}{2.832132in}}%
\pgfpathlineto{\pgfqpoint{3.907258in}{2.843023in}}%
\pgfpathclose%
\pgfusepath{fill}%
\end{pgfscope}%
\begin{pgfscope}%
\pgfpathrectangle{\pgfqpoint{0.765000in}{0.660000in}}{\pgfqpoint{4.620000in}{4.620000in}}%
\pgfusepath{clip}%
\pgfsetbuttcap%
\pgfsetroundjoin%
\definecolor{currentfill}{rgb}{1.000000,0.894118,0.788235}%
\pgfsetfillcolor{currentfill}%
\pgfsetlinewidth{0.000000pt}%
\definecolor{currentstroke}{rgb}{1.000000,0.894118,0.788235}%
\pgfsetstrokecolor{currentstroke}%
\pgfsetdash{}{0pt}%
\pgfpathmoveto{\pgfqpoint{3.880311in}{2.858581in}}%
\pgfpathlineto{\pgfqpoint{3.880311in}{2.889697in}}%
\pgfpathlineto{\pgfqpoint{3.868079in}{2.878806in}}%
\pgfpathlineto{\pgfqpoint{3.895026in}{2.832132in}}%
\pgfpathlineto{\pgfqpoint{3.880311in}{2.858581in}}%
\pgfpathclose%
\pgfusepath{fill}%
\end{pgfscope}%
\begin{pgfscope}%
\pgfpathrectangle{\pgfqpoint{0.765000in}{0.660000in}}{\pgfqpoint{4.620000in}{4.620000in}}%
\pgfusepath{clip}%
\pgfsetbuttcap%
\pgfsetroundjoin%
\definecolor{currentfill}{rgb}{1.000000,0.894118,0.788235}%
\pgfsetfillcolor{currentfill}%
\pgfsetlinewidth{0.000000pt}%
\definecolor{currentstroke}{rgb}{1.000000,0.894118,0.788235}%
\pgfsetstrokecolor{currentstroke}%
\pgfsetdash{}{0pt}%
\pgfpathmoveto{\pgfqpoint{3.907258in}{2.843023in}}%
\pgfpathlineto{\pgfqpoint{3.907258in}{2.874139in}}%
\pgfpathlineto{\pgfqpoint{3.895026in}{2.863248in}}%
\pgfpathlineto{\pgfqpoint{3.868079in}{2.847690in}}%
\pgfpathlineto{\pgfqpoint{3.907258in}{2.843023in}}%
\pgfpathclose%
\pgfusepath{fill}%
\end{pgfscope}%
\begin{pgfscope}%
\pgfpathrectangle{\pgfqpoint{0.765000in}{0.660000in}}{\pgfqpoint{4.620000in}{4.620000in}}%
\pgfusepath{clip}%
\pgfsetbuttcap%
\pgfsetroundjoin%
\definecolor{currentfill}{rgb}{1.000000,0.894118,0.788235}%
\pgfsetfillcolor{currentfill}%
\pgfsetlinewidth{0.000000pt}%
\definecolor{currentstroke}{rgb}{1.000000,0.894118,0.788235}%
\pgfsetstrokecolor{currentstroke}%
\pgfsetdash{}{0pt}%
\pgfpathmoveto{\pgfqpoint{3.853364in}{2.843023in}}%
\pgfpathlineto{\pgfqpoint{3.880311in}{2.827465in}}%
\pgfpathlineto{\pgfqpoint{3.868079in}{2.816574in}}%
\pgfpathlineto{\pgfqpoint{3.841132in}{2.832132in}}%
\pgfpathlineto{\pgfqpoint{3.853364in}{2.843023in}}%
\pgfpathclose%
\pgfusepath{fill}%
\end{pgfscope}%
\begin{pgfscope}%
\pgfpathrectangle{\pgfqpoint{0.765000in}{0.660000in}}{\pgfqpoint{4.620000in}{4.620000in}}%
\pgfusepath{clip}%
\pgfsetbuttcap%
\pgfsetroundjoin%
\definecolor{currentfill}{rgb}{1.000000,0.894118,0.788235}%
\pgfsetfillcolor{currentfill}%
\pgfsetlinewidth{0.000000pt}%
\definecolor{currentstroke}{rgb}{1.000000,0.894118,0.788235}%
\pgfsetstrokecolor{currentstroke}%
\pgfsetdash{}{0pt}%
\pgfpathmoveto{\pgfqpoint{3.880311in}{2.827465in}}%
\pgfpathlineto{\pgfqpoint{3.907258in}{2.843023in}}%
\pgfpathlineto{\pgfqpoint{3.895026in}{2.832132in}}%
\pgfpathlineto{\pgfqpoint{3.868079in}{2.816574in}}%
\pgfpathlineto{\pgfqpoint{3.880311in}{2.827465in}}%
\pgfpathclose%
\pgfusepath{fill}%
\end{pgfscope}%
\begin{pgfscope}%
\pgfpathrectangle{\pgfqpoint{0.765000in}{0.660000in}}{\pgfqpoint{4.620000in}{4.620000in}}%
\pgfusepath{clip}%
\pgfsetbuttcap%
\pgfsetroundjoin%
\definecolor{currentfill}{rgb}{1.000000,0.894118,0.788235}%
\pgfsetfillcolor{currentfill}%
\pgfsetlinewidth{0.000000pt}%
\definecolor{currentstroke}{rgb}{1.000000,0.894118,0.788235}%
\pgfsetstrokecolor{currentstroke}%
\pgfsetdash{}{0pt}%
\pgfpathmoveto{\pgfqpoint{3.880311in}{2.889697in}}%
\pgfpathlineto{\pgfqpoint{3.853364in}{2.874139in}}%
\pgfpathlineto{\pgfqpoint{3.841132in}{2.863248in}}%
\pgfpathlineto{\pgfqpoint{3.868079in}{2.878806in}}%
\pgfpathlineto{\pgfqpoint{3.880311in}{2.889697in}}%
\pgfpathclose%
\pgfusepath{fill}%
\end{pgfscope}%
\begin{pgfscope}%
\pgfpathrectangle{\pgfqpoint{0.765000in}{0.660000in}}{\pgfqpoint{4.620000in}{4.620000in}}%
\pgfusepath{clip}%
\pgfsetbuttcap%
\pgfsetroundjoin%
\definecolor{currentfill}{rgb}{1.000000,0.894118,0.788235}%
\pgfsetfillcolor{currentfill}%
\pgfsetlinewidth{0.000000pt}%
\definecolor{currentstroke}{rgb}{1.000000,0.894118,0.788235}%
\pgfsetstrokecolor{currentstroke}%
\pgfsetdash{}{0pt}%
\pgfpathmoveto{\pgfqpoint{3.907258in}{2.874139in}}%
\pgfpathlineto{\pgfqpoint{3.880311in}{2.889697in}}%
\pgfpathlineto{\pgfqpoint{3.868079in}{2.878806in}}%
\pgfpathlineto{\pgfqpoint{3.895026in}{2.863248in}}%
\pgfpathlineto{\pgfqpoint{3.907258in}{2.874139in}}%
\pgfpathclose%
\pgfusepath{fill}%
\end{pgfscope}%
\begin{pgfscope}%
\pgfpathrectangle{\pgfqpoint{0.765000in}{0.660000in}}{\pgfqpoint{4.620000in}{4.620000in}}%
\pgfusepath{clip}%
\pgfsetbuttcap%
\pgfsetroundjoin%
\definecolor{currentfill}{rgb}{1.000000,0.894118,0.788235}%
\pgfsetfillcolor{currentfill}%
\pgfsetlinewidth{0.000000pt}%
\definecolor{currentstroke}{rgb}{1.000000,0.894118,0.788235}%
\pgfsetstrokecolor{currentstroke}%
\pgfsetdash{}{0pt}%
\pgfpathmoveto{\pgfqpoint{3.853364in}{2.843023in}}%
\pgfpathlineto{\pgfqpoint{3.853364in}{2.874139in}}%
\pgfpathlineto{\pgfqpoint{3.841132in}{2.863248in}}%
\pgfpathlineto{\pgfqpoint{3.868079in}{2.816574in}}%
\pgfpathlineto{\pgfqpoint{3.853364in}{2.843023in}}%
\pgfpathclose%
\pgfusepath{fill}%
\end{pgfscope}%
\begin{pgfscope}%
\pgfpathrectangle{\pgfqpoint{0.765000in}{0.660000in}}{\pgfqpoint{4.620000in}{4.620000in}}%
\pgfusepath{clip}%
\pgfsetbuttcap%
\pgfsetroundjoin%
\definecolor{currentfill}{rgb}{1.000000,0.894118,0.788235}%
\pgfsetfillcolor{currentfill}%
\pgfsetlinewidth{0.000000pt}%
\definecolor{currentstroke}{rgb}{1.000000,0.894118,0.788235}%
\pgfsetstrokecolor{currentstroke}%
\pgfsetdash{}{0pt}%
\pgfpathmoveto{\pgfqpoint{3.880311in}{2.827465in}}%
\pgfpathlineto{\pgfqpoint{3.880311in}{2.858581in}}%
\pgfpathlineto{\pgfqpoint{3.868079in}{2.847690in}}%
\pgfpathlineto{\pgfqpoint{3.841132in}{2.832132in}}%
\pgfpathlineto{\pgfqpoint{3.880311in}{2.827465in}}%
\pgfpathclose%
\pgfusepath{fill}%
\end{pgfscope}%
\begin{pgfscope}%
\pgfpathrectangle{\pgfqpoint{0.765000in}{0.660000in}}{\pgfqpoint{4.620000in}{4.620000in}}%
\pgfusepath{clip}%
\pgfsetbuttcap%
\pgfsetroundjoin%
\definecolor{currentfill}{rgb}{1.000000,0.894118,0.788235}%
\pgfsetfillcolor{currentfill}%
\pgfsetlinewidth{0.000000pt}%
\definecolor{currentstroke}{rgb}{1.000000,0.894118,0.788235}%
\pgfsetstrokecolor{currentstroke}%
\pgfsetdash{}{0pt}%
\pgfpathmoveto{\pgfqpoint{3.880311in}{2.858581in}}%
\pgfpathlineto{\pgfqpoint{3.907258in}{2.874139in}}%
\pgfpathlineto{\pgfqpoint{3.895026in}{2.863248in}}%
\pgfpathlineto{\pgfqpoint{3.868079in}{2.847690in}}%
\pgfpathlineto{\pgfqpoint{3.880311in}{2.858581in}}%
\pgfpathclose%
\pgfusepath{fill}%
\end{pgfscope}%
\begin{pgfscope}%
\pgfpathrectangle{\pgfqpoint{0.765000in}{0.660000in}}{\pgfqpoint{4.620000in}{4.620000in}}%
\pgfusepath{clip}%
\pgfsetbuttcap%
\pgfsetroundjoin%
\definecolor{currentfill}{rgb}{1.000000,0.894118,0.788235}%
\pgfsetfillcolor{currentfill}%
\pgfsetlinewidth{0.000000pt}%
\definecolor{currentstroke}{rgb}{1.000000,0.894118,0.788235}%
\pgfsetstrokecolor{currentstroke}%
\pgfsetdash{}{0pt}%
\pgfpathmoveto{\pgfqpoint{3.853364in}{2.874139in}}%
\pgfpathlineto{\pgfqpoint{3.880311in}{2.858581in}}%
\pgfpathlineto{\pgfqpoint{3.868079in}{2.847690in}}%
\pgfpathlineto{\pgfqpoint{3.841132in}{2.863248in}}%
\pgfpathlineto{\pgfqpoint{3.853364in}{2.874139in}}%
\pgfpathclose%
\pgfusepath{fill}%
\end{pgfscope}%
\begin{pgfscope}%
\pgfpathrectangle{\pgfqpoint{0.765000in}{0.660000in}}{\pgfqpoint{4.620000in}{4.620000in}}%
\pgfusepath{clip}%
\pgfsetbuttcap%
\pgfsetroundjoin%
\definecolor{currentfill}{rgb}{1.000000,0.894118,0.788235}%
\pgfsetfillcolor{currentfill}%
\pgfsetlinewidth{0.000000pt}%
\definecolor{currentstroke}{rgb}{1.000000,0.894118,0.788235}%
\pgfsetstrokecolor{currentstroke}%
\pgfsetdash{}{0pt}%
\pgfpathmoveto{\pgfqpoint{4.192694in}{3.279892in}}%
\pgfpathlineto{\pgfqpoint{4.165747in}{3.264334in}}%
\pgfpathlineto{\pgfqpoint{4.192694in}{3.248776in}}%
\pgfpathlineto{\pgfqpoint{4.219642in}{3.264334in}}%
\pgfpathlineto{\pgfqpoint{4.192694in}{3.279892in}}%
\pgfpathclose%
\pgfusepath{fill}%
\end{pgfscope}%
\begin{pgfscope}%
\pgfpathrectangle{\pgfqpoint{0.765000in}{0.660000in}}{\pgfqpoint{4.620000in}{4.620000in}}%
\pgfusepath{clip}%
\pgfsetbuttcap%
\pgfsetroundjoin%
\definecolor{currentfill}{rgb}{1.000000,0.894118,0.788235}%
\pgfsetfillcolor{currentfill}%
\pgfsetlinewidth{0.000000pt}%
\definecolor{currentstroke}{rgb}{1.000000,0.894118,0.788235}%
\pgfsetstrokecolor{currentstroke}%
\pgfsetdash{}{0pt}%
\pgfpathmoveto{\pgfqpoint{4.192694in}{3.279892in}}%
\pgfpathlineto{\pgfqpoint{4.165747in}{3.264334in}}%
\pgfpathlineto{\pgfqpoint{4.165747in}{3.295450in}}%
\pgfpathlineto{\pgfqpoint{4.192694in}{3.311008in}}%
\pgfpathlineto{\pgfqpoint{4.192694in}{3.279892in}}%
\pgfpathclose%
\pgfusepath{fill}%
\end{pgfscope}%
\begin{pgfscope}%
\pgfpathrectangle{\pgfqpoint{0.765000in}{0.660000in}}{\pgfqpoint{4.620000in}{4.620000in}}%
\pgfusepath{clip}%
\pgfsetbuttcap%
\pgfsetroundjoin%
\definecolor{currentfill}{rgb}{1.000000,0.894118,0.788235}%
\pgfsetfillcolor{currentfill}%
\pgfsetlinewidth{0.000000pt}%
\definecolor{currentstroke}{rgb}{1.000000,0.894118,0.788235}%
\pgfsetstrokecolor{currentstroke}%
\pgfsetdash{}{0pt}%
\pgfpathmoveto{\pgfqpoint{4.192694in}{3.279892in}}%
\pgfpathlineto{\pgfqpoint{4.219642in}{3.264334in}}%
\pgfpathlineto{\pgfqpoint{4.219642in}{3.295450in}}%
\pgfpathlineto{\pgfqpoint{4.192694in}{3.311008in}}%
\pgfpathlineto{\pgfqpoint{4.192694in}{3.279892in}}%
\pgfpathclose%
\pgfusepath{fill}%
\end{pgfscope}%
\begin{pgfscope}%
\pgfpathrectangle{\pgfqpoint{0.765000in}{0.660000in}}{\pgfqpoint{4.620000in}{4.620000in}}%
\pgfusepath{clip}%
\pgfsetbuttcap%
\pgfsetroundjoin%
\definecolor{currentfill}{rgb}{1.000000,0.894118,0.788235}%
\pgfsetfillcolor{currentfill}%
\pgfsetlinewidth{0.000000pt}%
\definecolor{currentstroke}{rgb}{1.000000,0.894118,0.788235}%
\pgfsetstrokecolor{currentstroke}%
\pgfsetdash{}{0pt}%
\pgfpathmoveto{\pgfqpoint{4.260633in}{3.313475in}}%
\pgfpathlineto{\pgfqpoint{4.233686in}{3.297917in}}%
\pgfpathlineto{\pgfqpoint{4.260633in}{3.282359in}}%
\pgfpathlineto{\pgfqpoint{4.287581in}{3.297917in}}%
\pgfpathlineto{\pgfqpoint{4.260633in}{3.313475in}}%
\pgfpathclose%
\pgfusepath{fill}%
\end{pgfscope}%
\begin{pgfscope}%
\pgfpathrectangle{\pgfqpoint{0.765000in}{0.660000in}}{\pgfqpoint{4.620000in}{4.620000in}}%
\pgfusepath{clip}%
\pgfsetbuttcap%
\pgfsetroundjoin%
\definecolor{currentfill}{rgb}{1.000000,0.894118,0.788235}%
\pgfsetfillcolor{currentfill}%
\pgfsetlinewidth{0.000000pt}%
\definecolor{currentstroke}{rgb}{1.000000,0.894118,0.788235}%
\pgfsetstrokecolor{currentstroke}%
\pgfsetdash{}{0pt}%
\pgfpathmoveto{\pgfqpoint{4.260633in}{3.313475in}}%
\pgfpathlineto{\pgfqpoint{4.233686in}{3.297917in}}%
\pgfpathlineto{\pgfqpoint{4.233686in}{3.329033in}}%
\pgfpathlineto{\pgfqpoint{4.260633in}{3.344591in}}%
\pgfpathlineto{\pgfqpoint{4.260633in}{3.313475in}}%
\pgfpathclose%
\pgfusepath{fill}%
\end{pgfscope}%
\begin{pgfscope}%
\pgfpathrectangle{\pgfqpoint{0.765000in}{0.660000in}}{\pgfqpoint{4.620000in}{4.620000in}}%
\pgfusepath{clip}%
\pgfsetbuttcap%
\pgfsetroundjoin%
\definecolor{currentfill}{rgb}{1.000000,0.894118,0.788235}%
\pgfsetfillcolor{currentfill}%
\pgfsetlinewidth{0.000000pt}%
\definecolor{currentstroke}{rgb}{1.000000,0.894118,0.788235}%
\pgfsetstrokecolor{currentstroke}%
\pgfsetdash{}{0pt}%
\pgfpathmoveto{\pgfqpoint{4.260633in}{3.313475in}}%
\pgfpathlineto{\pgfqpoint{4.287581in}{3.297917in}}%
\pgfpathlineto{\pgfqpoint{4.287581in}{3.329033in}}%
\pgfpathlineto{\pgfqpoint{4.260633in}{3.344591in}}%
\pgfpathlineto{\pgfqpoint{4.260633in}{3.313475in}}%
\pgfpathclose%
\pgfusepath{fill}%
\end{pgfscope}%
\begin{pgfscope}%
\pgfpathrectangle{\pgfqpoint{0.765000in}{0.660000in}}{\pgfqpoint{4.620000in}{4.620000in}}%
\pgfusepath{clip}%
\pgfsetbuttcap%
\pgfsetroundjoin%
\definecolor{currentfill}{rgb}{1.000000,0.894118,0.788235}%
\pgfsetfillcolor{currentfill}%
\pgfsetlinewidth{0.000000pt}%
\definecolor{currentstroke}{rgb}{1.000000,0.894118,0.788235}%
\pgfsetstrokecolor{currentstroke}%
\pgfsetdash{}{0pt}%
\pgfpathmoveto{\pgfqpoint{4.192694in}{3.311008in}}%
\pgfpathlineto{\pgfqpoint{4.165747in}{3.295450in}}%
\pgfpathlineto{\pgfqpoint{4.192694in}{3.279892in}}%
\pgfpathlineto{\pgfqpoint{4.219642in}{3.295450in}}%
\pgfpathlineto{\pgfqpoint{4.192694in}{3.311008in}}%
\pgfpathclose%
\pgfusepath{fill}%
\end{pgfscope}%
\begin{pgfscope}%
\pgfpathrectangle{\pgfqpoint{0.765000in}{0.660000in}}{\pgfqpoint{4.620000in}{4.620000in}}%
\pgfusepath{clip}%
\pgfsetbuttcap%
\pgfsetroundjoin%
\definecolor{currentfill}{rgb}{1.000000,0.894118,0.788235}%
\pgfsetfillcolor{currentfill}%
\pgfsetlinewidth{0.000000pt}%
\definecolor{currentstroke}{rgb}{1.000000,0.894118,0.788235}%
\pgfsetstrokecolor{currentstroke}%
\pgfsetdash{}{0pt}%
\pgfpathmoveto{\pgfqpoint{4.192694in}{3.248776in}}%
\pgfpathlineto{\pgfqpoint{4.219642in}{3.264334in}}%
\pgfpathlineto{\pgfqpoint{4.219642in}{3.295450in}}%
\pgfpathlineto{\pgfqpoint{4.192694in}{3.279892in}}%
\pgfpathlineto{\pgfqpoint{4.192694in}{3.248776in}}%
\pgfpathclose%
\pgfusepath{fill}%
\end{pgfscope}%
\begin{pgfscope}%
\pgfpathrectangle{\pgfqpoint{0.765000in}{0.660000in}}{\pgfqpoint{4.620000in}{4.620000in}}%
\pgfusepath{clip}%
\pgfsetbuttcap%
\pgfsetroundjoin%
\definecolor{currentfill}{rgb}{1.000000,0.894118,0.788235}%
\pgfsetfillcolor{currentfill}%
\pgfsetlinewidth{0.000000pt}%
\definecolor{currentstroke}{rgb}{1.000000,0.894118,0.788235}%
\pgfsetstrokecolor{currentstroke}%
\pgfsetdash{}{0pt}%
\pgfpathmoveto{\pgfqpoint{4.165747in}{3.264334in}}%
\pgfpathlineto{\pgfqpoint{4.192694in}{3.248776in}}%
\pgfpathlineto{\pgfqpoint{4.192694in}{3.279892in}}%
\pgfpathlineto{\pgfqpoint{4.165747in}{3.295450in}}%
\pgfpathlineto{\pgfqpoint{4.165747in}{3.264334in}}%
\pgfpathclose%
\pgfusepath{fill}%
\end{pgfscope}%
\begin{pgfscope}%
\pgfpathrectangle{\pgfqpoint{0.765000in}{0.660000in}}{\pgfqpoint{4.620000in}{4.620000in}}%
\pgfusepath{clip}%
\pgfsetbuttcap%
\pgfsetroundjoin%
\definecolor{currentfill}{rgb}{1.000000,0.894118,0.788235}%
\pgfsetfillcolor{currentfill}%
\pgfsetlinewidth{0.000000pt}%
\definecolor{currentstroke}{rgb}{1.000000,0.894118,0.788235}%
\pgfsetstrokecolor{currentstroke}%
\pgfsetdash{}{0pt}%
\pgfpathmoveto{\pgfqpoint{4.260633in}{3.344591in}}%
\pgfpathlineto{\pgfqpoint{4.233686in}{3.329033in}}%
\pgfpathlineto{\pgfqpoint{4.260633in}{3.313475in}}%
\pgfpathlineto{\pgfqpoint{4.287581in}{3.329033in}}%
\pgfpathlineto{\pgfqpoint{4.260633in}{3.344591in}}%
\pgfpathclose%
\pgfusepath{fill}%
\end{pgfscope}%
\begin{pgfscope}%
\pgfpathrectangle{\pgfqpoint{0.765000in}{0.660000in}}{\pgfqpoint{4.620000in}{4.620000in}}%
\pgfusepath{clip}%
\pgfsetbuttcap%
\pgfsetroundjoin%
\definecolor{currentfill}{rgb}{1.000000,0.894118,0.788235}%
\pgfsetfillcolor{currentfill}%
\pgfsetlinewidth{0.000000pt}%
\definecolor{currentstroke}{rgb}{1.000000,0.894118,0.788235}%
\pgfsetstrokecolor{currentstroke}%
\pgfsetdash{}{0pt}%
\pgfpathmoveto{\pgfqpoint{4.260633in}{3.282359in}}%
\pgfpathlineto{\pgfqpoint{4.287581in}{3.297917in}}%
\pgfpathlineto{\pgfqpoint{4.287581in}{3.329033in}}%
\pgfpathlineto{\pgfqpoint{4.260633in}{3.313475in}}%
\pgfpathlineto{\pgfqpoint{4.260633in}{3.282359in}}%
\pgfpathclose%
\pgfusepath{fill}%
\end{pgfscope}%
\begin{pgfscope}%
\pgfpathrectangle{\pgfqpoint{0.765000in}{0.660000in}}{\pgfqpoint{4.620000in}{4.620000in}}%
\pgfusepath{clip}%
\pgfsetbuttcap%
\pgfsetroundjoin%
\definecolor{currentfill}{rgb}{1.000000,0.894118,0.788235}%
\pgfsetfillcolor{currentfill}%
\pgfsetlinewidth{0.000000pt}%
\definecolor{currentstroke}{rgb}{1.000000,0.894118,0.788235}%
\pgfsetstrokecolor{currentstroke}%
\pgfsetdash{}{0pt}%
\pgfpathmoveto{\pgfqpoint{4.233686in}{3.297917in}}%
\pgfpathlineto{\pgfqpoint{4.260633in}{3.282359in}}%
\pgfpathlineto{\pgfqpoint{4.260633in}{3.313475in}}%
\pgfpathlineto{\pgfqpoint{4.233686in}{3.329033in}}%
\pgfpathlineto{\pgfqpoint{4.233686in}{3.297917in}}%
\pgfpathclose%
\pgfusepath{fill}%
\end{pgfscope}%
\begin{pgfscope}%
\pgfpathrectangle{\pgfqpoint{0.765000in}{0.660000in}}{\pgfqpoint{4.620000in}{4.620000in}}%
\pgfusepath{clip}%
\pgfsetbuttcap%
\pgfsetroundjoin%
\definecolor{currentfill}{rgb}{1.000000,0.894118,0.788235}%
\pgfsetfillcolor{currentfill}%
\pgfsetlinewidth{0.000000pt}%
\definecolor{currentstroke}{rgb}{1.000000,0.894118,0.788235}%
\pgfsetstrokecolor{currentstroke}%
\pgfsetdash{}{0pt}%
\pgfpathmoveto{\pgfqpoint{4.192694in}{3.279892in}}%
\pgfpathlineto{\pgfqpoint{4.165747in}{3.264334in}}%
\pgfpathlineto{\pgfqpoint{4.233686in}{3.297917in}}%
\pgfpathlineto{\pgfqpoint{4.260633in}{3.313475in}}%
\pgfpathlineto{\pgfqpoint{4.192694in}{3.279892in}}%
\pgfpathclose%
\pgfusepath{fill}%
\end{pgfscope}%
\begin{pgfscope}%
\pgfpathrectangle{\pgfqpoint{0.765000in}{0.660000in}}{\pgfqpoint{4.620000in}{4.620000in}}%
\pgfusepath{clip}%
\pgfsetbuttcap%
\pgfsetroundjoin%
\definecolor{currentfill}{rgb}{1.000000,0.894118,0.788235}%
\pgfsetfillcolor{currentfill}%
\pgfsetlinewidth{0.000000pt}%
\definecolor{currentstroke}{rgb}{1.000000,0.894118,0.788235}%
\pgfsetstrokecolor{currentstroke}%
\pgfsetdash{}{0pt}%
\pgfpathmoveto{\pgfqpoint{4.219642in}{3.264334in}}%
\pgfpathlineto{\pgfqpoint{4.192694in}{3.279892in}}%
\pgfpathlineto{\pgfqpoint{4.260633in}{3.313475in}}%
\pgfpathlineto{\pgfqpoint{4.287581in}{3.297917in}}%
\pgfpathlineto{\pgfqpoint{4.219642in}{3.264334in}}%
\pgfpathclose%
\pgfusepath{fill}%
\end{pgfscope}%
\begin{pgfscope}%
\pgfpathrectangle{\pgfqpoint{0.765000in}{0.660000in}}{\pgfqpoint{4.620000in}{4.620000in}}%
\pgfusepath{clip}%
\pgfsetbuttcap%
\pgfsetroundjoin%
\definecolor{currentfill}{rgb}{1.000000,0.894118,0.788235}%
\pgfsetfillcolor{currentfill}%
\pgfsetlinewidth{0.000000pt}%
\definecolor{currentstroke}{rgb}{1.000000,0.894118,0.788235}%
\pgfsetstrokecolor{currentstroke}%
\pgfsetdash{}{0pt}%
\pgfpathmoveto{\pgfqpoint{4.192694in}{3.279892in}}%
\pgfpathlineto{\pgfqpoint{4.192694in}{3.311008in}}%
\pgfpathlineto{\pgfqpoint{4.260633in}{3.344591in}}%
\pgfpathlineto{\pgfqpoint{4.287581in}{3.297917in}}%
\pgfpathlineto{\pgfqpoint{4.192694in}{3.279892in}}%
\pgfpathclose%
\pgfusepath{fill}%
\end{pgfscope}%
\begin{pgfscope}%
\pgfpathrectangle{\pgfqpoint{0.765000in}{0.660000in}}{\pgfqpoint{4.620000in}{4.620000in}}%
\pgfusepath{clip}%
\pgfsetbuttcap%
\pgfsetroundjoin%
\definecolor{currentfill}{rgb}{1.000000,0.894118,0.788235}%
\pgfsetfillcolor{currentfill}%
\pgfsetlinewidth{0.000000pt}%
\definecolor{currentstroke}{rgb}{1.000000,0.894118,0.788235}%
\pgfsetstrokecolor{currentstroke}%
\pgfsetdash{}{0pt}%
\pgfpathmoveto{\pgfqpoint{4.219642in}{3.264334in}}%
\pgfpathlineto{\pgfqpoint{4.219642in}{3.295450in}}%
\pgfpathlineto{\pgfqpoint{4.287581in}{3.329033in}}%
\pgfpathlineto{\pgfqpoint{4.260633in}{3.313475in}}%
\pgfpathlineto{\pgfqpoint{4.219642in}{3.264334in}}%
\pgfpathclose%
\pgfusepath{fill}%
\end{pgfscope}%
\begin{pgfscope}%
\pgfpathrectangle{\pgfqpoint{0.765000in}{0.660000in}}{\pgfqpoint{4.620000in}{4.620000in}}%
\pgfusepath{clip}%
\pgfsetbuttcap%
\pgfsetroundjoin%
\definecolor{currentfill}{rgb}{1.000000,0.894118,0.788235}%
\pgfsetfillcolor{currentfill}%
\pgfsetlinewidth{0.000000pt}%
\definecolor{currentstroke}{rgb}{1.000000,0.894118,0.788235}%
\pgfsetstrokecolor{currentstroke}%
\pgfsetdash{}{0pt}%
\pgfpathmoveto{\pgfqpoint{4.165747in}{3.264334in}}%
\pgfpathlineto{\pgfqpoint{4.192694in}{3.248776in}}%
\pgfpathlineto{\pgfqpoint{4.260633in}{3.282359in}}%
\pgfpathlineto{\pgfqpoint{4.233686in}{3.297917in}}%
\pgfpathlineto{\pgfqpoint{4.165747in}{3.264334in}}%
\pgfpathclose%
\pgfusepath{fill}%
\end{pgfscope}%
\begin{pgfscope}%
\pgfpathrectangle{\pgfqpoint{0.765000in}{0.660000in}}{\pgfqpoint{4.620000in}{4.620000in}}%
\pgfusepath{clip}%
\pgfsetbuttcap%
\pgfsetroundjoin%
\definecolor{currentfill}{rgb}{1.000000,0.894118,0.788235}%
\pgfsetfillcolor{currentfill}%
\pgfsetlinewidth{0.000000pt}%
\definecolor{currentstroke}{rgb}{1.000000,0.894118,0.788235}%
\pgfsetstrokecolor{currentstroke}%
\pgfsetdash{}{0pt}%
\pgfpathmoveto{\pgfqpoint{4.192694in}{3.248776in}}%
\pgfpathlineto{\pgfqpoint{4.219642in}{3.264334in}}%
\pgfpathlineto{\pgfqpoint{4.287581in}{3.297917in}}%
\pgfpathlineto{\pgfqpoint{4.260633in}{3.282359in}}%
\pgfpathlineto{\pgfqpoint{4.192694in}{3.248776in}}%
\pgfpathclose%
\pgfusepath{fill}%
\end{pgfscope}%
\begin{pgfscope}%
\pgfpathrectangle{\pgfqpoint{0.765000in}{0.660000in}}{\pgfqpoint{4.620000in}{4.620000in}}%
\pgfusepath{clip}%
\pgfsetbuttcap%
\pgfsetroundjoin%
\definecolor{currentfill}{rgb}{1.000000,0.894118,0.788235}%
\pgfsetfillcolor{currentfill}%
\pgfsetlinewidth{0.000000pt}%
\definecolor{currentstroke}{rgb}{1.000000,0.894118,0.788235}%
\pgfsetstrokecolor{currentstroke}%
\pgfsetdash{}{0pt}%
\pgfpathmoveto{\pgfqpoint{4.192694in}{3.311008in}}%
\pgfpathlineto{\pgfqpoint{4.165747in}{3.295450in}}%
\pgfpathlineto{\pgfqpoint{4.233686in}{3.329033in}}%
\pgfpathlineto{\pgfqpoint{4.260633in}{3.344591in}}%
\pgfpathlineto{\pgfqpoint{4.192694in}{3.311008in}}%
\pgfpathclose%
\pgfusepath{fill}%
\end{pgfscope}%
\begin{pgfscope}%
\pgfpathrectangle{\pgfqpoint{0.765000in}{0.660000in}}{\pgfqpoint{4.620000in}{4.620000in}}%
\pgfusepath{clip}%
\pgfsetbuttcap%
\pgfsetroundjoin%
\definecolor{currentfill}{rgb}{1.000000,0.894118,0.788235}%
\pgfsetfillcolor{currentfill}%
\pgfsetlinewidth{0.000000pt}%
\definecolor{currentstroke}{rgb}{1.000000,0.894118,0.788235}%
\pgfsetstrokecolor{currentstroke}%
\pgfsetdash{}{0pt}%
\pgfpathmoveto{\pgfqpoint{4.219642in}{3.295450in}}%
\pgfpathlineto{\pgfqpoint{4.192694in}{3.311008in}}%
\pgfpathlineto{\pgfqpoint{4.260633in}{3.344591in}}%
\pgfpathlineto{\pgfqpoint{4.287581in}{3.329033in}}%
\pgfpathlineto{\pgfqpoint{4.219642in}{3.295450in}}%
\pgfpathclose%
\pgfusepath{fill}%
\end{pgfscope}%
\begin{pgfscope}%
\pgfpathrectangle{\pgfqpoint{0.765000in}{0.660000in}}{\pgfqpoint{4.620000in}{4.620000in}}%
\pgfusepath{clip}%
\pgfsetbuttcap%
\pgfsetroundjoin%
\definecolor{currentfill}{rgb}{1.000000,0.894118,0.788235}%
\pgfsetfillcolor{currentfill}%
\pgfsetlinewidth{0.000000pt}%
\definecolor{currentstroke}{rgb}{1.000000,0.894118,0.788235}%
\pgfsetstrokecolor{currentstroke}%
\pgfsetdash{}{0pt}%
\pgfpathmoveto{\pgfqpoint{4.165747in}{3.264334in}}%
\pgfpathlineto{\pgfqpoint{4.165747in}{3.295450in}}%
\pgfpathlineto{\pgfqpoint{4.233686in}{3.329033in}}%
\pgfpathlineto{\pgfqpoint{4.260633in}{3.282359in}}%
\pgfpathlineto{\pgfqpoint{4.165747in}{3.264334in}}%
\pgfpathclose%
\pgfusepath{fill}%
\end{pgfscope}%
\begin{pgfscope}%
\pgfpathrectangle{\pgfqpoint{0.765000in}{0.660000in}}{\pgfqpoint{4.620000in}{4.620000in}}%
\pgfusepath{clip}%
\pgfsetbuttcap%
\pgfsetroundjoin%
\definecolor{currentfill}{rgb}{1.000000,0.894118,0.788235}%
\pgfsetfillcolor{currentfill}%
\pgfsetlinewidth{0.000000pt}%
\definecolor{currentstroke}{rgb}{1.000000,0.894118,0.788235}%
\pgfsetstrokecolor{currentstroke}%
\pgfsetdash{}{0pt}%
\pgfpathmoveto{\pgfqpoint{4.192694in}{3.248776in}}%
\pgfpathlineto{\pgfqpoint{4.192694in}{3.279892in}}%
\pgfpathlineto{\pgfqpoint{4.260633in}{3.313475in}}%
\pgfpathlineto{\pgfqpoint{4.233686in}{3.297917in}}%
\pgfpathlineto{\pgfqpoint{4.192694in}{3.248776in}}%
\pgfpathclose%
\pgfusepath{fill}%
\end{pgfscope}%
\begin{pgfscope}%
\pgfpathrectangle{\pgfqpoint{0.765000in}{0.660000in}}{\pgfqpoint{4.620000in}{4.620000in}}%
\pgfusepath{clip}%
\pgfsetbuttcap%
\pgfsetroundjoin%
\definecolor{currentfill}{rgb}{1.000000,0.894118,0.788235}%
\pgfsetfillcolor{currentfill}%
\pgfsetlinewidth{0.000000pt}%
\definecolor{currentstroke}{rgb}{1.000000,0.894118,0.788235}%
\pgfsetstrokecolor{currentstroke}%
\pgfsetdash{}{0pt}%
\pgfpathmoveto{\pgfqpoint{4.192694in}{3.279892in}}%
\pgfpathlineto{\pgfqpoint{4.219642in}{3.295450in}}%
\pgfpathlineto{\pgfqpoint{4.287581in}{3.329033in}}%
\pgfpathlineto{\pgfqpoint{4.260633in}{3.313475in}}%
\pgfpathlineto{\pgfqpoint{4.192694in}{3.279892in}}%
\pgfpathclose%
\pgfusepath{fill}%
\end{pgfscope}%
\begin{pgfscope}%
\pgfpathrectangle{\pgfqpoint{0.765000in}{0.660000in}}{\pgfqpoint{4.620000in}{4.620000in}}%
\pgfusepath{clip}%
\pgfsetbuttcap%
\pgfsetroundjoin%
\definecolor{currentfill}{rgb}{1.000000,0.894118,0.788235}%
\pgfsetfillcolor{currentfill}%
\pgfsetlinewidth{0.000000pt}%
\definecolor{currentstroke}{rgb}{1.000000,0.894118,0.788235}%
\pgfsetstrokecolor{currentstroke}%
\pgfsetdash{}{0pt}%
\pgfpathmoveto{\pgfqpoint{4.165747in}{3.295450in}}%
\pgfpathlineto{\pgfqpoint{4.192694in}{3.279892in}}%
\pgfpathlineto{\pgfqpoint{4.260633in}{3.313475in}}%
\pgfpathlineto{\pgfqpoint{4.233686in}{3.329033in}}%
\pgfpathlineto{\pgfqpoint{4.165747in}{3.295450in}}%
\pgfpathclose%
\pgfusepath{fill}%
\end{pgfscope}%
\begin{pgfscope}%
\pgfpathrectangle{\pgfqpoint{0.765000in}{0.660000in}}{\pgfqpoint{4.620000in}{4.620000in}}%
\pgfusepath{clip}%
\pgfsetbuttcap%
\pgfsetroundjoin%
\definecolor{currentfill}{rgb}{1.000000,0.894118,0.788235}%
\pgfsetfillcolor{currentfill}%
\pgfsetlinewidth{0.000000pt}%
\definecolor{currentstroke}{rgb}{1.000000,0.894118,0.788235}%
\pgfsetstrokecolor{currentstroke}%
\pgfsetdash{}{0pt}%
\pgfpathmoveto{\pgfqpoint{4.192694in}{3.279892in}}%
\pgfpathlineto{\pgfqpoint{4.165747in}{3.264334in}}%
\pgfpathlineto{\pgfqpoint{4.192694in}{3.248776in}}%
\pgfpathlineto{\pgfqpoint{4.219642in}{3.264334in}}%
\pgfpathlineto{\pgfqpoint{4.192694in}{3.279892in}}%
\pgfpathclose%
\pgfusepath{fill}%
\end{pgfscope}%
\begin{pgfscope}%
\pgfpathrectangle{\pgfqpoint{0.765000in}{0.660000in}}{\pgfqpoint{4.620000in}{4.620000in}}%
\pgfusepath{clip}%
\pgfsetbuttcap%
\pgfsetroundjoin%
\definecolor{currentfill}{rgb}{1.000000,0.894118,0.788235}%
\pgfsetfillcolor{currentfill}%
\pgfsetlinewidth{0.000000pt}%
\definecolor{currentstroke}{rgb}{1.000000,0.894118,0.788235}%
\pgfsetstrokecolor{currentstroke}%
\pgfsetdash{}{0pt}%
\pgfpathmoveto{\pgfqpoint{4.192694in}{3.279892in}}%
\pgfpathlineto{\pgfqpoint{4.165747in}{3.264334in}}%
\pgfpathlineto{\pgfqpoint{4.165747in}{3.295450in}}%
\pgfpathlineto{\pgfqpoint{4.192694in}{3.311008in}}%
\pgfpathlineto{\pgfqpoint{4.192694in}{3.279892in}}%
\pgfpathclose%
\pgfusepath{fill}%
\end{pgfscope}%
\begin{pgfscope}%
\pgfpathrectangle{\pgfqpoint{0.765000in}{0.660000in}}{\pgfqpoint{4.620000in}{4.620000in}}%
\pgfusepath{clip}%
\pgfsetbuttcap%
\pgfsetroundjoin%
\definecolor{currentfill}{rgb}{1.000000,0.894118,0.788235}%
\pgfsetfillcolor{currentfill}%
\pgfsetlinewidth{0.000000pt}%
\definecolor{currentstroke}{rgb}{1.000000,0.894118,0.788235}%
\pgfsetstrokecolor{currentstroke}%
\pgfsetdash{}{0pt}%
\pgfpathmoveto{\pgfqpoint{4.192694in}{3.279892in}}%
\pgfpathlineto{\pgfqpoint{4.219642in}{3.264334in}}%
\pgfpathlineto{\pgfqpoint{4.219642in}{3.295450in}}%
\pgfpathlineto{\pgfqpoint{4.192694in}{3.311008in}}%
\pgfpathlineto{\pgfqpoint{4.192694in}{3.279892in}}%
\pgfpathclose%
\pgfusepath{fill}%
\end{pgfscope}%
\begin{pgfscope}%
\pgfpathrectangle{\pgfqpoint{0.765000in}{0.660000in}}{\pgfqpoint{4.620000in}{4.620000in}}%
\pgfusepath{clip}%
\pgfsetbuttcap%
\pgfsetroundjoin%
\definecolor{currentfill}{rgb}{1.000000,0.894118,0.788235}%
\pgfsetfillcolor{currentfill}%
\pgfsetlinewidth{0.000000pt}%
\definecolor{currentstroke}{rgb}{1.000000,0.894118,0.788235}%
\pgfsetstrokecolor{currentstroke}%
\pgfsetdash{}{0pt}%
\pgfpathmoveto{\pgfqpoint{4.176009in}{3.272329in}}%
\pgfpathlineto{\pgfqpoint{4.149062in}{3.256771in}}%
\pgfpathlineto{\pgfqpoint{4.176009in}{3.241213in}}%
\pgfpathlineto{\pgfqpoint{4.202957in}{3.256771in}}%
\pgfpathlineto{\pgfqpoint{4.176009in}{3.272329in}}%
\pgfpathclose%
\pgfusepath{fill}%
\end{pgfscope}%
\begin{pgfscope}%
\pgfpathrectangle{\pgfqpoint{0.765000in}{0.660000in}}{\pgfqpoint{4.620000in}{4.620000in}}%
\pgfusepath{clip}%
\pgfsetbuttcap%
\pgfsetroundjoin%
\definecolor{currentfill}{rgb}{1.000000,0.894118,0.788235}%
\pgfsetfillcolor{currentfill}%
\pgfsetlinewidth{0.000000pt}%
\definecolor{currentstroke}{rgb}{1.000000,0.894118,0.788235}%
\pgfsetstrokecolor{currentstroke}%
\pgfsetdash{}{0pt}%
\pgfpathmoveto{\pgfqpoint{4.176009in}{3.272329in}}%
\pgfpathlineto{\pgfqpoint{4.149062in}{3.256771in}}%
\pgfpathlineto{\pgfqpoint{4.149062in}{3.287887in}}%
\pgfpathlineto{\pgfqpoint{4.176009in}{3.303445in}}%
\pgfpathlineto{\pgfqpoint{4.176009in}{3.272329in}}%
\pgfpathclose%
\pgfusepath{fill}%
\end{pgfscope}%
\begin{pgfscope}%
\pgfpathrectangle{\pgfqpoint{0.765000in}{0.660000in}}{\pgfqpoint{4.620000in}{4.620000in}}%
\pgfusepath{clip}%
\pgfsetbuttcap%
\pgfsetroundjoin%
\definecolor{currentfill}{rgb}{1.000000,0.894118,0.788235}%
\pgfsetfillcolor{currentfill}%
\pgfsetlinewidth{0.000000pt}%
\definecolor{currentstroke}{rgb}{1.000000,0.894118,0.788235}%
\pgfsetstrokecolor{currentstroke}%
\pgfsetdash{}{0pt}%
\pgfpathmoveto{\pgfqpoint{4.176009in}{3.272329in}}%
\pgfpathlineto{\pgfqpoint{4.202957in}{3.256771in}}%
\pgfpathlineto{\pgfqpoint{4.202957in}{3.287887in}}%
\pgfpathlineto{\pgfqpoint{4.176009in}{3.303445in}}%
\pgfpathlineto{\pgfqpoint{4.176009in}{3.272329in}}%
\pgfpathclose%
\pgfusepath{fill}%
\end{pgfscope}%
\begin{pgfscope}%
\pgfpathrectangle{\pgfqpoint{0.765000in}{0.660000in}}{\pgfqpoint{4.620000in}{4.620000in}}%
\pgfusepath{clip}%
\pgfsetbuttcap%
\pgfsetroundjoin%
\definecolor{currentfill}{rgb}{1.000000,0.894118,0.788235}%
\pgfsetfillcolor{currentfill}%
\pgfsetlinewidth{0.000000pt}%
\definecolor{currentstroke}{rgb}{1.000000,0.894118,0.788235}%
\pgfsetstrokecolor{currentstroke}%
\pgfsetdash{}{0pt}%
\pgfpathmoveto{\pgfqpoint{4.192694in}{3.311008in}}%
\pgfpathlineto{\pgfqpoint{4.165747in}{3.295450in}}%
\pgfpathlineto{\pgfqpoint{4.192694in}{3.279892in}}%
\pgfpathlineto{\pgfqpoint{4.219642in}{3.295450in}}%
\pgfpathlineto{\pgfqpoint{4.192694in}{3.311008in}}%
\pgfpathclose%
\pgfusepath{fill}%
\end{pgfscope}%
\begin{pgfscope}%
\pgfpathrectangle{\pgfqpoint{0.765000in}{0.660000in}}{\pgfqpoint{4.620000in}{4.620000in}}%
\pgfusepath{clip}%
\pgfsetbuttcap%
\pgfsetroundjoin%
\definecolor{currentfill}{rgb}{1.000000,0.894118,0.788235}%
\pgfsetfillcolor{currentfill}%
\pgfsetlinewidth{0.000000pt}%
\definecolor{currentstroke}{rgb}{1.000000,0.894118,0.788235}%
\pgfsetstrokecolor{currentstroke}%
\pgfsetdash{}{0pt}%
\pgfpathmoveto{\pgfqpoint{4.192694in}{3.248776in}}%
\pgfpathlineto{\pgfqpoint{4.219642in}{3.264334in}}%
\pgfpathlineto{\pgfqpoint{4.219642in}{3.295450in}}%
\pgfpathlineto{\pgfqpoint{4.192694in}{3.279892in}}%
\pgfpathlineto{\pgfqpoint{4.192694in}{3.248776in}}%
\pgfpathclose%
\pgfusepath{fill}%
\end{pgfscope}%
\begin{pgfscope}%
\pgfpathrectangle{\pgfqpoint{0.765000in}{0.660000in}}{\pgfqpoint{4.620000in}{4.620000in}}%
\pgfusepath{clip}%
\pgfsetbuttcap%
\pgfsetroundjoin%
\definecolor{currentfill}{rgb}{1.000000,0.894118,0.788235}%
\pgfsetfillcolor{currentfill}%
\pgfsetlinewidth{0.000000pt}%
\definecolor{currentstroke}{rgb}{1.000000,0.894118,0.788235}%
\pgfsetstrokecolor{currentstroke}%
\pgfsetdash{}{0pt}%
\pgfpathmoveto{\pgfqpoint{4.165747in}{3.264334in}}%
\pgfpathlineto{\pgfqpoint{4.192694in}{3.248776in}}%
\pgfpathlineto{\pgfqpoint{4.192694in}{3.279892in}}%
\pgfpathlineto{\pgfqpoint{4.165747in}{3.295450in}}%
\pgfpathlineto{\pgfqpoint{4.165747in}{3.264334in}}%
\pgfpathclose%
\pgfusepath{fill}%
\end{pgfscope}%
\begin{pgfscope}%
\pgfpathrectangle{\pgfqpoint{0.765000in}{0.660000in}}{\pgfqpoint{4.620000in}{4.620000in}}%
\pgfusepath{clip}%
\pgfsetbuttcap%
\pgfsetroundjoin%
\definecolor{currentfill}{rgb}{1.000000,0.894118,0.788235}%
\pgfsetfillcolor{currentfill}%
\pgfsetlinewidth{0.000000pt}%
\definecolor{currentstroke}{rgb}{1.000000,0.894118,0.788235}%
\pgfsetstrokecolor{currentstroke}%
\pgfsetdash{}{0pt}%
\pgfpathmoveto{\pgfqpoint{4.176009in}{3.303445in}}%
\pgfpathlineto{\pgfqpoint{4.149062in}{3.287887in}}%
\pgfpathlineto{\pgfqpoint{4.176009in}{3.272329in}}%
\pgfpathlineto{\pgfqpoint{4.202957in}{3.287887in}}%
\pgfpathlineto{\pgfqpoint{4.176009in}{3.303445in}}%
\pgfpathclose%
\pgfusepath{fill}%
\end{pgfscope}%
\begin{pgfscope}%
\pgfpathrectangle{\pgfqpoint{0.765000in}{0.660000in}}{\pgfqpoint{4.620000in}{4.620000in}}%
\pgfusepath{clip}%
\pgfsetbuttcap%
\pgfsetroundjoin%
\definecolor{currentfill}{rgb}{1.000000,0.894118,0.788235}%
\pgfsetfillcolor{currentfill}%
\pgfsetlinewidth{0.000000pt}%
\definecolor{currentstroke}{rgb}{1.000000,0.894118,0.788235}%
\pgfsetstrokecolor{currentstroke}%
\pgfsetdash{}{0pt}%
\pgfpathmoveto{\pgfqpoint{4.176009in}{3.241213in}}%
\pgfpathlineto{\pgfqpoint{4.202957in}{3.256771in}}%
\pgfpathlineto{\pgfqpoint{4.202957in}{3.287887in}}%
\pgfpathlineto{\pgfqpoint{4.176009in}{3.272329in}}%
\pgfpathlineto{\pgfqpoint{4.176009in}{3.241213in}}%
\pgfpathclose%
\pgfusepath{fill}%
\end{pgfscope}%
\begin{pgfscope}%
\pgfpathrectangle{\pgfqpoint{0.765000in}{0.660000in}}{\pgfqpoint{4.620000in}{4.620000in}}%
\pgfusepath{clip}%
\pgfsetbuttcap%
\pgfsetroundjoin%
\definecolor{currentfill}{rgb}{1.000000,0.894118,0.788235}%
\pgfsetfillcolor{currentfill}%
\pgfsetlinewidth{0.000000pt}%
\definecolor{currentstroke}{rgb}{1.000000,0.894118,0.788235}%
\pgfsetstrokecolor{currentstroke}%
\pgfsetdash{}{0pt}%
\pgfpathmoveto{\pgfqpoint{4.149062in}{3.256771in}}%
\pgfpathlineto{\pgfqpoint{4.176009in}{3.241213in}}%
\pgfpathlineto{\pgfqpoint{4.176009in}{3.272329in}}%
\pgfpathlineto{\pgfqpoint{4.149062in}{3.287887in}}%
\pgfpathlineto{\pgfqpoint{4.149062in}{3.256771in}}%
\pgfpathclose%
\pgfusepath{fill}%
\end{pgfscope}%
\begin{pgfscope}%
\pgfpathrectangle{\pgfqpoint{0.765000in}{0.660000in}}{\pgfqpoint{4.620000in}{4.620000in}}%
\pgfusepath{clip}%
\pgfsetbuttcap%
\pgfsetroundjoin%
\definecolor{currentfill}{rgb}{1.000000,0.894118,0.788235}%
\pgfsetfillcolor{currentfill}%
\pgfsetlinewidth{0.000000pt}%
\definecolor{currentstroke}{rgb}{1.000000,0.894118,0.788235}%
\pgfsetstrokecolor{currentstroke}%
\pgfsetdash{}{0pt}%
\pgfpathmoveto{\pgfqpoint{4.192694in}{3.279892in}}%
\pgfpathlineto{\pgfqpoint{4.165747in}{3.264334in}}%
\pgfpathlineto{\pgfqpoint{4.149062in}{3.256771in}}%
\pgfpathlineto{\pgfqpoint{4.176009in}{3.272329in}}%
\pgfpathlineto{\pgfqpoint{4.192694in}{3.279892in}}%
\pgfpathclose%
\pgfusepath{fill}%
\end{pgfscope}%
\begin{pgfscope}%
\pgfpathrectangle{\pgfqpoint{0.765000in}{0.660000in}}{\pgfqpoint{4.620000in}{4.620000in}}%
\pgfusepath{clip}%
\pgfsetbuttcap%
\pgfsetroundjoin%
\definecolor{currentfill}{rgb}{1.000000,0.894118,0.788235}%
\pgfsetfillcolor{currentfill}%
\pgfsetlinewidth{0.000000pt}%
\definecolor{currentstroke}{rgb}{1.000000,0.894118,0.788235}%
\pgfsetstrokecolor{currentstroke}%
\pgfsetdash{}{0pt}%
\pgfpathmoveto{\pgfqpoint{4.219642in}{3.264334in}}%
\pgfpathlineto{\pgfqpoint{4.192694in}{3.279892in}}%
\pgfpathlineto{\pgfqpoint{4.176009in}{3.272329in}}%
\pgfpathlineto{\pgfqpoint{4.202957in}{3.256771in}}%
\pgfpathlineto{\pgfqpoint{4.219642in}{3.264334in}}%
\pgfpathclose%
\pgfusepath{fill}%
\end{pgfscope}%
\begin{pgfscope}%
\pgfpathrectangle{\pgfqpoint{0.765000in}{0.660000in}}{\pgfqpoint{4.620000in}{4.620000in}}%
\pgfusepath{clip}%
\pgfsetbuttcap%
\pgfsetroundjoin%
\definecolor{currentfill}{rgb}{1.000000,0.894118,0.788235}%
\pgfsetfillcolor{currentfill}%
\pgfsetlinewidth{0.000000pt}%
\definecolor{currentstroke}{rgb}{1.000000,0.894118,0.788235}%
\pgfsetstrokecolor{currentstroke}%
\pgfsetdash{}{0pt}%
\pgfpathmoveto{\pgfqpoint{4.192694in}{3.279892in}}%
\pgfpathlineto{\pgfqpoint{4.192694in}{3.311008in}}%
\pgfpathlineto{\pgfqpoint{4.176009in}{3.303445in}}%
\pgfpathlineto{\pgfqpoint{4.202957in}{3.256771in}}%
\pgfpathlineto{\pgfqpoint{4.192694in}{3.279892in}}%
\pgfpathclose%
\pgfusepath{fill}%
\end{pgfscope}%
\begin{pgfscope}%
\pgfpathrectangle{\pgfqpoint{0.765000in}{0.660000in}}{\pgfqpoint{4.620000in}{4.620000in}}%
\pgfusepath{clip}%
\pgfsetbuttcap%
\pgfsetroundjoin%
\definecolor{currentfill}{rgb}{1.000000,0.894118,0.788235}%
\pgfsetfillcolor{currentfill}%
\pgfsetlinewidth{0.000000pt}%
\definecolor{currentstroke}{rgb}{1.000000,0.894118,0.788235}%
\pgfsetstrokecolor{currentstroke}%
\pgfsetdash{}{0pt}%
\pgfpathmoveto{\pgfqpoint{4.219642in}{3.264334in}}%
\pgfpathlineto{\pgfqpoint{4.219642in}{3.295450in}}%
\pgfpathlineto{\pgfqpoint{4.202957in}{3.287887in}}%
\pgfpathlineto{\pgfqpoint{4.176009in}{3.272329in}}%
\pgfpathlineto{\pgfqpoint{4.219642in}{3.264334in}}%
\pgfpathclose%
\pgfusepath{fill}%
\end{pgfscope}%
\begin{pgfscope}%
\pgfpathrectangle{\pgfqpoint{0.765000in}{0.660000in}}{\pgfqpoint{4.620000in}{4.620000in}}%
\pgfusepath{clip}%
\pgfsetbuttcap%
\pgfsetroundjoin%
\definecolor{currentfill}{rgb}{1.000000,0.894118,0.788235}%
\pgfsetfillcolor{currentfill}%
\pgfsetlinewidth{0.000000pt}%
\definecolor{currentstroke}{rgb}{1.000000,0.894118,0.788235}%
\pgfsetstrokecolor{currentstroke}%
\pgfsetdash{}{0pt}%
\pgfpathmoveto{\pgfqpoint{4.165747in}{3.264334in}}%
\pgfpathlineto{\pgfqpoint{4.192694in}{3.248776in}}%
\pgfpathlineto{\pgfqpoint{4.176009in}{3.241213in}}%
\pgfpathlineto{\pgfqpoint{4.149062in}{3.256771in}}%
\pgfpathlineto{\pgfqpoint{4.165747in}{3.264334in}}%
\pgfpathclose%
\pgfusepath{fill}%
\end{pgfscope}%
\begin{pgfscope}%
\pgfpathrectangle{\pgfqpoint{0.765000in}{0.660000in}}{\pgfqpoint{4.620000in}{4.620000in}}%
\pgfusepath{clip}%
\pgfsetbuttcap%
\pgfsetroundjoin%
\definecolor{currentfill}{rgb}{1.000000,0.894118,0.788235}%
\pgfsetfillcolor{currentfill}%
\pgfsetlinewidth{0.000000pt}%
\definecolor{currentstroke}{rgb}{1.000000,0.894118,0.788235}%
\pgfsetstrokecolor{currentstroke}%
\pgfsetdash{}{0pt}%
\pgfpathmoveto{\pgfqpoint{4.192694in}{3.248776in}}%
\pgfpathlineto{\pgfqpoint{4.219642in}{3.264334in}}%
\pgfpathlineto{\pgfqpoint{4.202957in}{3.256771in}}%
\pgfpathlineto{\pgfqpoint{4.176009in}{3.241213in}}%
\pgfpathlineto{\pgfqpoint{4.192694in}{3.248776in}}%
\pgfpathclose%
\pgfusepath{fill}%
\end{pgfscope}%
\begin{pgfscope}%
\pgfpathrectangle{\pgfqpoint{0.765000in}{0.660000in}}{\pgfqpoint{4.620000in}{4.620000in}}%
\pgfusepath{clip}%
\pgfsetbuttcap%
\pgfsetroundjoin%
\definecolor{currentfill}{rgb}{1.000000,0.894118,0.788235}%
\pgfsetfillcolor{currentfill}%
\pgfsetlinewidth{0.000000pt}%
\definecolor{currentstroke}{rgb}{1.000000,0.894118,0.788235}%
\pgfsetstrokecolor{currentstroke}%
\pgfsetdash{}{0pt}%
\pgfpathmoveto{\pgfqpoint{4.192694in}{3.311008in}}%
\pgfpathlineto{\pgfqpoint{4.165747in}{3.295450in}}%
\pgfpathlineto{\pgfqpoint{4.149062in}{3.287887in}}%
\pgfpathlineto{\pgfqpoint{4.176009in}{3.303445in}}%
\pgfpathlineto{\pgfqpoint{4.192694in}{3.311008in}}%
\pgfpathclose%
\pgfusepath{fill}%
\end{pgfscope}%
\begin{pgfscope}%
\pgfpathrectangle{\pgfqpoint{0.765000in}{0.660000in}}{\pgfqpoint{4.620000in}{4.620000in}}%
\pgfusepath{clip}%
\pgfsetbuttcap%
\pgfsetroundjoin%
\definecolor{currentfill}{rgb}{1.000000,0.894118,0.788235}%
\pgfsetfillcolor{currentfill}%
\pgfsetlinewidth{0.000000pt}%
\definecolor{currentstroke}{rgb}{1.000000,0.894118,0.788235}%
\pgfsetstrokecolor{currentstroke}%
\pgfsetdash{}{0pt}%
\pgfpathmoveto{\pgfqpoint{4.219642in}{3.295450in}}%
\pgfpathlineto{\pgfqpoint{4.192694in}{3.311008in}}%
\pgfpathlineto{\pgfqpoint{4.176009in}{3.303445in}}%
\pgfpathlineto{\pgfqpoint{4.202957in}{3.287887in}}%
\pgfpathlineto{\pgfqpoint{4.219642in}{3.295450in}}%
\pgfpathclose%
\pgfusepath{fill}%
\end{pgfscope}%
\begin{pgfscope}%
\pgfpathrectangle{\pgfqpoint{0.765000in}{0.660000in}}{\pgfqpoint{4.620000in}{4.620000in}}%
\pgfusepath{clip}%
\pgfsetbuttcap%
\pgfsetroundjoin%
\definecolor{currentfill}{rgb}{1.000000,0.894118,0.788235}%
\pgfsetfillcolor{currentfill}%
\pgfsetlinewidth{0.000000pt}%
\definecolor{currentstroke}{rgb}{1.000000,0.894118,0.788235}%
\pgfsetstrokecolor{currentstroke}%
\pgfsetdash{}{0pt}%
\pgfpathmoveto{\pgfqpoint{4.165747in}{3.264334in}}%
\pgfpathlineto{\pgfqpoint{4.165747in}{3.295450in}}%
\pgfpathlineto{\pgfqpoint{4.149062in}{3.287887in}}%
\pgfpathlineto{\pgfqpoint{4.176009in}{3.241213in}}%
\pgfpathlineto{\pgfqpoint{4.165747in}{3.264334in}}%
\pgfpathclose%
\pgfusepath{fill}%
\end{pgfscope}%
\begin{pgfscope}%
\pgfpathrectangle{\pgfqpoint{0.765000in}{0.660000in}}{\pgfqpoint{4.620000in}{4.620000in}}%
\pgfusepath{clip}%
\pgfsetbuttcap%
\pgfsetroundjoin%
\definecolor{currentfill}{rgb}{1.000000,0.894118,0.788235}%
\pgfsetfillcolor{currentfill}%
\pgfsetlinewidth{0.000000pt}%
\definecolor{currentstroke}{rgb}{1.000000,0.894118,0.788235}%
\pgfsetstrokecolor{currentstroke}%
\pgfsetdash{}{0pt}%
\pgfpathmoveto{\pgfqpoint{4.192694in}{3.248776in}}%
\pgfpathlineto{\pgfqpoint{4.192694in}{3.279892in}}%
\pgfpathlineto{\pgfqpoint{4.176009in}{3.272329in}}%
\pgfpathlineto{\pgfqpoint{4.149062in}{3.256771in}}%
\pgfpathlineto{\pgfqpoint{4.192694in}{3.248776in}}%
\pgfpathclose%
\pgfusepath{fill}%
\end{pgfscope}%
\begin{pgfscope}%
\pgfpathrectangle{\pgfqpoint{0.765000in}{0.660000in}}{\pgfqpoint{4.620000in}{4.620000in}}%
\pgfusepath{clip}%
\pgfsetbuttcap%
\pgfsetroundjoin%
\definecolor{currentfill}{rgb}{1.000000,0.894118,0.788235}%
\pgfsetfillcolor{currentfill}%
\pgfsetlinewidth{0.000000pt}%
\definecolor{currentstroke}{rgb}{1.000000,0.894118,0.788235}%
\pgfsetstrokecolor{currentstroke}%
\pgfsetdash{}{0pt}%
\pgfpathmoveto{\pgfqpoint{4.192694in}{3.279892in}}%
\pgfpathlineto{\pgfqpoint{4.219642in}{3.295450in}}%
\pgfpathlineto{\pgfqpoint{4.202957in}{3.287887in}}%
\pgfpathlineto{\pgfqpoint{4.176009in}{3.272329in}}%
\pgfpathlineto{\pgfqpoint{4.192694in}{3.279892in}}%
\pgfpathclose%
\pgfusepath{fill}%
\end{pgfscope}%
\begin{pgfscope}%
\pgfpathrectangle{\pgfqpoint{0.765000in}{0.660000in}}{\pgfqpoint{4.620000in}{4.620000in}}%
\pgfusepath{clip}%
\pgfsetbuttcap%
\pgfsetroundjoin%
\definecolor{currentfill}{rgb}{1.000000,0.894118,0.788235}%
\pgfsetfillcolor{currentfill}%
\pgfsetlinewidth{0.000000pt}%
\definecolor{currentstroke}{rgb}{1.000000,0.894118,0.788235}%
\pgfsetstrokecolor{currentstroke}%
\pgfsetdash{}{0pt}%
\pgfpathmoveto{\pgfqpoint{4.165747in}{3.295450in}}%
\pgfpathlineto{\pgfqpoint{4.192694in}{3.279892in}}%
\pgfpathlineto{\pgfqpoint{4.176009in}{3.272329in}}%
\pgfpathlineto{\pgfqpoint{4.149062in}{3.287887in}}%
\pgfpathlineto{\pgfqpoint{4.165747in}{3.295450in}}%
\pgfpathclose%
\pgfusepath{fill}%
\end{pgfscope}%
\begin{pgfscope}%
\pgfpathrectangle{\pgfqpoint{0.765000in}{0.660000in}}{\pgfqpoint{4.620000in}{4.620000in}}%
\pgfusepath{clip}%
\pgfsetbuttcap%
\pgfsetroundjoin%
\definecolor{currentfill}{rgb}{1.000000,0.894118,0.788235}%
\pgfsetfillcolor{currentfill}%
\pgfsetlinewidth{0.000000pt}%
\definecolor{currentstroke}{rgb}{1.000000,0.894118,0.788235}%
\pgfsetstrokecolor{currentstroke}%
\pgfsetdash{}{0pt}%
\pgfpathmoveto{\pgfqpoint{4.176009in}{3.272329in}}%
\pgfpathlineto{\pgfqpoint{4.149062in}{3.256771in}}%
\pgfpathlineto{\pgfqpoint{4.176009in}{3.241213in}}%
\pgfpathlineto{\pgfqpoint{4.202957in}{3.256771in}}%
\pgfpathlineto{\pgfqpoint{4.176009in}{3.272329in}}%
\pgfpathclose%
\pgfusepath{fill}%
\end{pgfscope}%
\begin{pgfscope}%
\pgfpathrectangle{\pgfqpoint{0.765000in}{0.660000in}}{\pgfqpoint{4.620000in}{4.620000in}}%
\pgfusepath{clip}%
\pgfsetbuttcap%
\pgfsetroundjoin%
\definecolor{currentfill}{rgb}{1.000000,0.894118,0.788235}%
\pgfsetfillcolor{currentfill}%
\pgfsetlinewidth{0.000000pt}%
\definecolor{currentstroke}{rgb}{1.000000,0.894118,0.788235}%
\pgfsetstrokecolor{currentstroke}%
\pgfsetdash{}{0pt}%
\pgfpathmoveto{\pgfqpoint{4.176009in}{3.272329in}}%
\pgfpathlineto{\pgfqpoint{4.149062in}{3.256771in}}%
\pgfpathlineto{\pgfqpoint{4.149062in}{3.287887in}}%
\pgfpathlineto{\pgfqpoint{4.176009in}{3.303445in}}%
\pgfpathlineto{\pgfqpoint{4.176009in}{3.272329in}}%
\pgfpathclose%
\pgfusepath{fill}%
\end{pgfscope}%
\begin{pgfscope}%
\pgfpathrectangle{\pgfqpoint{0.765000in}{0.660000in}}{\pgfqpoint{4.620000in}{4.620000in}}%
\pgfusepath{clip}%
\pgfsetbuttcap%
\pgfsetroundjoin%
\definecolor{currentfill}{rgb}{1.000000,0.894118,0.788235}%
\pgfsetfillcolor{currentfill}%
\pgfsetlinewidth{0.000000pt}%
\definecolor{currentstroke}{rgb}{1.000000,0.894118,0.788235}%
\pgfsetstrokecolor{currentstroke}%
\pgfsetdash{}{0pt}%
\pgfpathmoveto{\pgfqpoint{4.176009in}{3.272329in}}%
\pgfpathlineto{\pgfqpoint{4.202957in}{3.256771in}}%
\pgfpathlineto{\pgfqpoint{4.202957in}{3.287887in}}%
\pgfpathlineto{\pgfqpoint{4.176009in}{3.303445in}}%
\pgfpathlineto{\pgfqpoint{4.176009in}{3.272329in}}%
\pgfpathclose%
\pgfusepath{fill}%
\end{pgfscope}%
\begin{pgfscope}%
\pgfpathrectangle{\pgfqpoint{0.765000in}{0.660000in}}{\pgfqpoint{4.620000in}{4.620000in}}%
\pgfusepath{clip}%
\pgfsetbuttcap%
\pgfsetroundjoin%
\definecolor{currentfill}{rgb}{1.000000,0.894118,0.788235}%
\pgfsetfillcolor{currentfill}%
\pgfsetlinewidth{0.000000pt}%
\definecolor{currentstroke}{rgb}{1.000000,0.894118,0.788235}%
\pgfsetstrokecolor{currentstroke}%
\pgfsetdash{}{0pt}%
\pgfpathmoveto{\pgfqpoint{4.192694in}{3.279892in}}%
\pgfpathlineto{\pgfqpoint{4.165747in}{3.264334in}}%
\pgfpathlineto{\pgfqpoint{4.192694in}{3.248776in}}%
\pgfpathlineto{\pgfqpoint{4.219642in}{3.264334in}}%
\pgfpathlineto{\pgfqpoint{4.192694in}{3.279892in}}%
\pgfpathclose%
\pgfusepath{fill}%
\end{pgfscope}%
\begin{pgfscope}%
\pgfpathrectangle{\pgfqpoint{0.765000in}{0.660000in}}{\pgfqpoint{4.620000in}{4.620000in}}%
\pgfusepath{clip}%
\pgfsetbuttcap%
\pgfsetroundjoin%
\definecolor{currentfill}{rgb}{1.000000,0.894118,0.788235}%
\pgfsetfillcolor{currentfill}%
\pgfsetlinewidth{0.000000pt}%
\definecolor{currentstroke}{rgb}{1.000000,0.894118,0.788235}%
\pgfsetstrokecolor{currentstroke}%
\pgfsetdash{}{0pt}%
\pgfpathmoveto{\pgfqpoint{4.192694in}{3.279892in}}%
\pgfpathlineto{\pgfqpoint{4.165747in}{3.264334in}}%
\pgfpathlineto{\pgfqpoint{4.165747in}{3.295450in}}%
\pgfpathlineto{\pgfqpoint{4.192694in}{3.311008in}}%
\pgfpathlineto{\pgfqpoint{4.192694in}{3.279892in}}%
\pgfpathclose%
\pgfusepath{fill}%
\end{pgfscope}%
\begin{pgfscope}%
\pgfpathrectangle{\pgfqpoint{0.765000in}{0.660000in}}{\pgfqpoint{4.620000in}{4.620000in}}%
\pgfusepath{clip}%
\pgfsetbuttcap%
\pgfsetroundjoin%
\definecolor{currentfill}{rgb}{1.000000,0.894118,0.788235}%
\pgfsetfillcolor{currentfill}%
\pgfsetlinewidth{0.000000pt}%
\definecolor{currentstroke}{rgb}{1.000000,0.894118,0.788235}%
\pgfsetstrokecolor{currentstroke}%
\pgfsetdash{}{0pt}%
\pgfpathmoveto{\pgfqpoint{4.192694in}{3.279892in}}%
\pgfpathlineto{\pgfqpoint{4.219642in}{3.264334in}}%
\pgfpathlineto{\pgfqpoint{4.219642in}{3.295450in}}%
\pgfpathlineto{\pgfqpoint{4.192694in}{3.311008in}}%
\pgfpathlineto{\pgfqpoint{4.192694in}{3.279892in}}%
\pgfpathclose%
\pgfusepath{fill}%
\end{pgfscope}%
\begin{pgfscope}%
\pgfpathrectangle{\pgfqpoint{0.765000in}{0.660000in}}{\pgfqpoint{4.620000in}{4.620000in}}%
\pgfusepath{clip}%
\pgfsetbuttcap%
\pgfsetroundjoin%
\definecolor{currentfill}{rgb}{1.000000,0.894118,0.788235}%
\pgfsetfillcolor{currentfill}%
\pgfsetlinewidth{0.000000pt}%
\definecolor{currentstroke}{rgb}{1.000000,0.894118,0.788235}%
\pgfsetstrokecolor{currentstroke}%
\pgfsetdash{}{0pt}%
\pgfpathmoveto{\pgfqpoint{4.176009in}{3.303445in}}%
\pgfpathlineto{\pgfqpoint{4.149062in}{3.287887in}}%
\pgfpathlineto{\pgfqpoint{4.176009in}{3.272329in}}%
\pgfpathlineto{\pgfqpoint{4.202957in}{3.287887in}}%
\pgfpathlineto{\pgfqpoint{4.176009in}{3.303445in}}%
\pgfpathclose%
\pgfusepath{fill}%
\end{pgfscope}%
\begin{pgfscope}%
\pgfpathrectangle{\pgfqpoint{0.765000in}{0.660000in}}{\pgfqpoint{4.620000in}{4.620000in}}%
\pgfusepath{clip}%
\pgfsetbuttcap%
\pgfsetroundjoin%
\definecolor{currentfill}{rgb}{1.000000,0.894118,0.788235}%
\pgfsetfillcolor{currentfill}%
\pgfsetlinewidth{0.000000pt}%
\definecolor{currentstroke}{rgb}{1.000000,0.894118,0.788235}%
\pgfsetstrokecolor{currentstroke}%
\pgfsetdash{}{0pt}%
\pgfpathmoveto{\pgfqpoint{4.176009in}{3.241213in}}%
\pgfpathlineto{\pgfqpoint{4.202957in}{3.256771in}}%
\pgfpathlineto{\pgfqpoint{4.202957in}{3.287887in}}%
\pgfpathlineto{\pgfqpoint{4.176009in}{3.272329in}}%
\pgfpathlineto{\pgfqpoint{4.176009in}{3.241213in}}%
\pgfpathclose%
\pgfusepath{fill}%
\end{pgfscope}%
\begin{pgfscope}%
\pgfpathrectangle{\pgfqpoint{0.765000in}{0.660000in}}{\pgfqpoint{4.620000in}{4.620000in}}%
\pgfusepath{clip}%
\pgfsetbuttcap%
\pgfsetroundjoin%
\definecolor{currentfill}{rgb}{1.000000,0.894118,0.788235}%
\pgfsetfillcolor{currentfill}%
\pgfsetlinewidth{0.000000pt}%
\definecolor{currentstroke}{rgb}{1.000000,0.894118,0.788235}%
\pgfsetstrokecolor{currentstroke}%
\pgfsetdash{}{0pt}%
\pgfpathmoveto{\pgfqpoint{4.149062in}{3.256771in}}%
\pgfpathlineto{\pgfqpoint{4.176009in}{3.241213in}}%
\pgfpathlineto{\pgfqpoint{4.176009in}{3.272329in}}%
\pgfpathlineto{\pgfqpoint{4.149062in}{3.287887in}}%
\pgfpathlineto{\pgfqpoint{4.149062in}{3.256771in}}%
\pgfpathclose%
\pgfusepath{fill}%
\end{pgfscope}%
\begin{pgfscope}%
\pgfpathrectangle{\pgfqpoint{0.765000in}{0.660000in}}{\pgfqpoint{4.620000in}{4.620000in}}%
\pgfusepath{clip}%
\pgfsetbuttcap%
\pgfsetroundjoin%
\definecolor{currentfill}{rgb}{1.000000,0.894118,0.788235}%
\pgfsetfillcolor{currentfill}%
\pgfsetlinewidth{0.000000pt}%
\definecolor{currentstroke}{rgb}{1.000000,0.894118,0.788235}%
\pgfsetstrokecolor{currentstroke}%
\pgfsetdash{}{0pt}%
\pgfpathmoveto{\pgfqpoint{4.192694in}{3.311008in}}%
\pgfpathlineto{\pgfqpoint{4.165747in}{3.295450in}}%
\pgfpathlineto{\pgfqpoint{4.192694in}{3.279892in}}%
\pgfpathlineto{\pgfqpoint{4.219642in}{3.295450in}}%
\pgfpathlineto{\pgfqpoint{4.192694in}{3.311008in}}%
\pgfpathclose%
\pgfusepath{fill}%
\end{pgfscope}%
\begin{pgfscope}%
\pgfpathrectangle{\pgfqpoint{0.765000in}{0.660000in}}{\pgfqpoint{4.620000in}{4.620000in}}%
\pgfusepath{clip}%
\pgfsetbuttcap%
\pgfsetroundjoin%
\definecolor{currentfill}{rgb}{1.000000,0.894118,0.788235}%
\pgfsetfillcolor{currentfill}%
\pgfsetlinewidth{0.000000pt}%
\definecolor{currentstroke}{rgb}{1.000000,0.894118,0.788235}%
\pgfsetstrokecolor{currentstroke}%
\pgfsetdash{}{0pt}%
\pgfpathmoveto{\pgfqpoint{4.192694in}{3.248776in}}%
\pgfpathlineto{\pgfqpoint{4.219642in}{3.264334in}}%
\pgfpathlineto{\pgfqpoint{4.219642in}{3.295450in}}%
\pgfpathlineto{\pgfqpoint{4.192694in}{3.279892in}}%
\pgfpathlineto{\pgfqpoint{4.192694in}{3.248776in}}%
\pgfpathclose%
\pgfusepath{fill}%
\end{pgfscope}%
\begin{pgfscope}%
\pgfpathrectangle{\pgfqpoint{0.765000in}{0.660000in}}{\pgfqpoint{4.620000in}{4.620000in}}%
\pgfusepath{clip}%
\pgfsetbuttcap%
\pgfsetroundjoin%
\definecolor{currentfill}{rgb}{1.000000,0.894118,0.788235}%
\pgfsetfillcolor{currentfill}%
\pgfsetlinewidth{0.000000pt}%
\definecolor{currentstroke}{rgb}{1.000000,0.894118,0.788235}%
\pgfsetstrokecolor{currentstroke}%
\pgfsetdash{}{0pt}%
\pgfpathmoveto{\pgfqpoint{4.165747in}{3.264334in}}%
\pgfpathlineto{\pgfqpoint{4.192694in}{3.248776in}}%
\pgfpathlineto{\pgfqpoint{4.192694in}{3.279892in}}%
\pgfpathlineto{\pgfqpoint{4.165747in}{3.295450in}}%
\pgfpathlineto{\pgfqpoint{4.165747in}{3.264334in}}%
\pgfpathclose%
\pgfusepath{fill}%
\end{pgfscope}%
\begin{pgfscope}%
\pgfpathrectangle{\pgfqpoint{0.765000in}{0.660000in}}{\pgfqpoint{4.620000in}{4.620000in}}%
\pgfusepath{clip}%
\pgfsetbuttcap%
\pgfsetroundjoin%
\definecolor{currentfill}{rgb}{1.000000,0.894118,0.788235}%
\pgfsetfillcolor{currentfill}%
\pgfsetlinewidth{0.000000pt}%
\definecolor{currentstroke}{rgb}{1.000000,0.894118,0.788235}%
\pgfsetstrokecolor{currentstroke}%
\pgfsetdash{}{0pt}%
\pgfpathmoveto{\pgfqpoint{4.176009in}{3.272329in}}%
\pgfpathlineto{\pgfqpoint{4.149062in}{3.256771in}}%
\pgfpathlineto{\pgfqpoint{4.165747in}{3.264334in}}%
\pgfpathlineto{\pgfqpoint{4.192694in}{3.279892in}}%
\pgfpathlineto{\pgfqpoint{4.176009in}{3.272329in}}%
\pgfpathclose%
\pgfusepath{fill}%
\end{pgfscope}%
\begin{pgfscope}%
\pgfpathrectangle{\pgfqpoint{0.765000in}{0.660000in}}{\pgfqpoint{4.620000in}{4.620000in}}%
\pgfusepath{clip}%
\pgfsetbuttcap%
\pgfsetroundjoin%
\definecolor{currentfill}{rgb}{1.000000,0.894118,0.788235}%
\pgfsetfillcolor{currentfill}%
\pgfsetlinewidth{0.000000pt}%
\definecolor{currentstroke}{rgb}{1.000000,0.894118,0.788235}%
\pgfsetstrokecolor{currentstroke}%
\pgfsetdash{}{0pt}%
\pgfpathmoveto{\pgfqpoint{4.202957in}{3.256771in}}%
\pgfpathlineto{\pgfqpoint{4.176009in}{3.272329in}}%
\pgfpathlineto{\pgfqpoint{4.192694in}{3.279892in}}%
\pgfpathlineto{\pgfqpoint{4.219642in}{3.264334in}}%
\pgfpathlineto{\pgfqpoint{4.202957in}{3.256771in}}%
\pgfpathclose%
\pgfusepath{fill}%
\end{pgfscope}%
\begin{pgfscope}%
\pgfpathrectangle{\pgfqpoint{0.765000in}{0.660000in}}{\pgfqpoint{4.620000in}{4.620000in}}%
\pgfusepath{clip}%
\pgfsetbuttcap%
\pgfsetroundjoin%
\definecolor{currentfill}{rgb}{1.000000,0.894118,0.788235}%
\pgfsetfillcolor{currentfill}%
\pgfsetlinewidth{0.000000pt}%
\definecolor{currentstroke}{rgb}{1.000000,0.894118,0.788235}%
\pgfsetstrokecolor{currentstroke}%
\pgfsetdash{}{0pt}%
\pgfpathmoveto{\pgfqpoint{4.176009in}{3.272329in}}%
\pgfpathlineto{\pgfqpoint{4.176009in}{3.303445in}}%
\pgfpathlineto{\pgfqpoint{4.192694in}{3.311008in}}%
\pgfpathlineto{\pgfqpoint{4.219642in}{3.264334in}}%
\pgfpathlineto{\pgfqpoint{4.176009in}{3.272329in}}%
\pgfpathclose%
\pgfusepath{fill}%
\end{pgfscope}%
\begin{pgfscope}%
\pgfpathrectangle{\pgfqpoint{0.765000in}{0.660000in}}{\pgfqpoint{4.620000in}{4.620000in}}%
\pgfusepath{clip}%
\pgfsetbuttcap%
\pgfsetroundjoin%
\definecolor{currentfill}{rgb}{1.000000,0.894118,0.788235}%
\pgfsetfillcolor{currentfill}%
\pgfsetlinewidth{0.000000pt}%
\definecolor{currentstroke}{rgb}{1.000000,0.894118,0.788235}%
\pgfsetstrokecolor{currentstroke}%
\pgfsetdash{}{0pt}%
\pgfpathmoveto{\pgfqpoint{4.202957in}{3.256771in}}%
\pgfpathlineto{\pgfqpoint{4.202957in}{3.287887in}}%
\pgfpathlineto{\pgfqpoint{4.219642in}{3.295450in}}%
\pgfpathlineto{\pgfqpoint{4.192694in}{3.279892in}}%
\pgfpathlineto{\pgfqpoint{4.202957in}{3.256771in}}%
\pgfpathclose%
\pgfusepath{fill}%
\end{pgfscope}%
\begin{pgfscope}%
\pgfpathrectangle{\pgfqpoint{0.765000in}{0.660000in}}{\pgfqpoint{4.620000in}{4.620000in}}%
\pgfusepath{clip}%
\pgfsetbuttcap%
\pgfsetroundjoin%
\definecolor{currentfill}{rgb}{1.000000,0.894118,0.788235}%
\pgfsetfillcolor{currentfill}%
\pgfsetlinewidth{0.000000pt}%
\definecolor{currentstroke}{rgb}{1.000000,0.894118,0.788235}%
\pgfsetstrokecolor{currentstroke}%
\pgfsetdash{}{0pt}%
\pgfpathmoveto{\pgfqpoint{4.149062in}{3.256771in}}%
\pgfpathlineto{\pgfqpoint{4.176009in}{3.241213in}}%
\pgfpathlineto{\pgfqpoint{4.192694in}{3.248776in}}%
\pgfpathlineto{\pgfqpoint{4.165747in}{3.264334in}}%
\pgfpathlineto{\pgfqpoint{4.149062in}{3.256771in}}%
\pgfpathclose%
\pgfusepath{fill}%
\end{pgfscope}%
\begin{pgfscope}%
\pgfpathrectangle{\pgfqpoint{0.765000in}{0.660000in}}{\pgfqpoint{4.620000in}{4.620000in}}%
\pgfusepath{clip}%
\pgfsetbuttcap%
\pgfsetroundjoin%
\definecolor{currentfill}{rgb}{1.000000,0.894118,0.788235}%
\pgfsetfillcolor{currentfill}%
\pgfsetlinewidth{0.000000pt}%
\definecolor{currentstroke}{rgb}{1.000000,0.894118,0.788235}%
\pgfsetstrokecolor{currentstroke}%
\pgfsetdash{}{0pt}%
\pgfpathmoveto{\pgfqpoint{4.176009in}{3.241213in}}%
\pgfpathlineto{\pgfqpoint{4.202957in}{3.256771in}}%
\pgfpathlineto{\pgfqpoint{4.219642in}{3.264334in}}%
\pgfpathlineto{\pgfqpoint{4.192694in}{3.248776in}}%
\pgfpathlineto{\pgfqpoint{4.176009in}{3.241213in}}%
\pgfpathclose%
\pgfusepath{fill}%
\end{pgfscope}%
\begin{pgfscope}%
\pgfpathrectangle{\pgfqpoint{0.765000in}{0.660000in}}{\pgfqpoint{4.620000in}{4.620000in}}%
\pgfusepath{clip}%
\pgfsetbuttcap%
\pgfsetroundjoin%
\definecolor{currentfill}{rgb}{1.000000,0.894118,0.788235}%
\pgfsetfillcolor{currentfill}%
\pgfsetlinewidth{0.000000pt}%
\definecolor{currentstroke}{rgb}{1.000000,0.894118,0.788235}%
\pgfsetstrokecolor{currentstroke}%
\pgfsetdash{}{0pt}%
\pgfpathmoveto{\pgfqpoint{4.176009in}{3.303445in}}%
\pgfpathlineto{\pgfqpoint{4.149062in}{3.287887in}}%
\pgfpathlineto{\pgfqpoint{4.165747in}{3.295450in}}%
\pgfpathlineto{\pgfqpoint{4.192694in}{3.311008in}}%
\pgfpathlineto{\pgfqpoint{4.176009in}{3.303445in}}%
\pgfpathclose%
\pgfusepath{fill}%
\end{pgfscope}%
\begin{pgfscope}%
\pgfpathrectangle{\pgfqpoint{0.765000in}{0.660000in}}{\pgfqpoint{4.620000in}{4.620000in}}%
\pgfusepath{clip}%
\pgfsetbuttcap%
\pgfsetroundjoin%
\definecolor{currentfill}{rgb}{1.000000,0.894118,0.788235}%
\pgfsetfillcolor{currentfill}%
\pgfsetlinewidth{0.000000pt}%
\definecolor{currentstroke}{rgb}{1.000000,0.894118,0.788235}%
\pgfsetstrokecolor{currentstroke}%
\pgfsetdash{}{0pt}%
\pgfpathmoveto{\pgfqpoint{4.202957in}{3.287887in}}%
\pgfpathlineto{\pgfqpoint{4.176009in}{3.303445in}}%
\pgfpathlineto{\pgfqpoint{4.192694in}{3.311008in}}%
\pgfpathlineto{\pgfqpoint{4.219642in}{3.295450in}}%
\pgfpathlineto{\pgfqpoint{4.202957in}{3.287887in}}%
\pgfpathclose%
\pgfusepath{fill}%
\end{pgfscope}%
\begin{pgfscope}%
\pgfpathrectangle{\pgfqpoint{0.765000in}{0.660000in}}{\pgfqpoint{4.620000in}{4.620000in}}%
\pgfusepath{clip}%
\pgfsetbuttcap%
\pgfsetroundjoin%
\definecolor{currentfill}{rgb}{1.000000,0.894118,0.788235}%
\pgfsetfillcolor{currentfill}%
\pgfsetlinewidth{0.000000pt}%
\definecolor{currentstroke}{rgb}{1.000000,0.894118,0.788235}%
\pgfsetstrokecolor{currentstroke}%
\pgfsetdash{}{0pt}%
\pgfpathmoveto{\pgfqpoint{4.149062in}{3.256771in}}%
\pgfpathlineto{\pgfqpoint{4.149062in}{3.287887in}}%
\pgfpathlineto{\pgfqpoint{4.165747in}{3.295450in}}%
\pgfpathlineto{\pgfqpoint{4.192694in}{3.248776in}}%
\pgfpathlineto{\pgfqpoint{4.149062in}{3.256771in}}%
\pgfpathclose%
\pgfusepath{fill}%
\end{pgfscope}%
\begin{pgfscope}%
\pgfpathrectangle{\pgfqpoint{0.765000in}{0.660000in}}{\pgfqpoint{4.620000in}{4.620000in}}%
\pgfusepath{clip}%
\pgfsetbuttcap%
\pgfsetroundjoin%
\definecolor{currentfill}{rgb}{1.000000,0.894118,0.788235}%
\pgfsetfillcolor{currentfill}%
\pgfsetlinewidth{0.000000pt}%
\definecolor{currentstroke}{rgb}{1.000000,0.894118,0.788235}%
\pgfsetstrokecolor{currentstroke}%
\pgfsetdash{}{0pt}%
\pgfpathmoveto{\pgfqpoint{4.176009in}{3.241213in}}%
\pgfpathlineto{\pgfqpoint{4.176009in}{3.272329in}}%
\pgfpathlineto{\pgfqpoint{4.192694in}{3.279892in}}%
\pgfpathlineto{\pgfqpoint{4.165747in}{3.264334in}}%
\pgfpathlineto{\pgfqpoint{4.176009in}{3.241213in}}%
\pgfpathclose%
\pgfusepath{fill}%
\end{pgfscope}%
\begin{pgfscope}%
\pgfpathrectangle{\pgfqpoint{0.765000in}{0.660000in}}{\pgfqpoint{4.620000in}{4.620000in}}%
\pgfusepath{clip}%
\pgfsetbuttcap%
\pgfsetroundjoin%
\definecolor{currentfill}{rgb}{1.000000,0.894118,0.788235}%
\pgfsetfillcolor{currentfill}%
\pgfsetlinewidth{0.000000pt}%
\definecolor{currentstroke}{rgb}{1.000000,0.894118,0.788235}%
\pgfsetstrokecolor{currentstroke}%
\pgfsetdash{}{0pt}%
\pgfpathmoveto{\pgfqpoint{4.176009in}{3.272329in}}%
\pgfpathlineto{\pgfqpoint{4.202957in}{3.287887in}}%
\pgfpathlineto{\pgfqpoint{4.219642in}{3.295450in}}%
\pgfpathlineto{\pgfqpoint{4.192694in}{3.279892in}}%
\pgfpathlineto{\pgfqpoint{4.176009in}{3.272329in}}%
\pgfpathclose%
\pgfusepath{fill}%
\end{pgfscope}%
\begin{pgfscope}%
\pgfpathrectangle{\pgfqpoint{0.765000in}{0.660000in}}{\pgfqpoint{4.620000in}{4.620000in}}%
\pgfusepath{clip}%
\pgfsetbuttcap%
\pgfsetroundjoin%
\definecolor{currentfill}{rgb}{1.000000,0.894118,0.788235}%
\pgfsetfillcolor{currentfill}%
\pgfsetlinewidth{0.000000pt}%
\definecolor{currentstroke}{rgb}{1.000000,0.894118,0.788235}%
\pgfsetstrokecolor{currentstroke}%
\pgfsetdash{}{0pt}%
\pgfpathmoveto{\pgfqpoint{4.149062in}{3.287887in}}%
\pgfpathlineto{\pgfqpoint{4.176009in}{3.272329in}}%
\pgfpathlineto{\pgfqpoint{4.192694in}{3.279892in}}%
\pgfpathlineto{\pgfqpoint{4.165747in}{3.295450in}}%
\pgfpathlineto{\pgfqpoint{4.149062in}{3.287887in}}%
\pgfpathclose%
\pgfusepath{fill}%
\end{pgfscope}%
\begin{pgfscope}%
\pgfpathrectangle{\pgfqpoint{0.765000in}{0.660000in}}{\pgfqpoint{4.620000in}{4.620000in}}%
\pgfusepath{clip}%
\pgfsetbuttcap%
\pgfsetroundjoin%
\definecolor{currentfill}{rgb}{1.000000,0.894118,0.788235}%
\pgfsetfillcolor{currentfill}%
\pgfsetlinewidth{0.000000pt}%
\definecolor{currentstroke}{rgb}{1.000000,0.894118,0.788235}%
\pgfsetstrokecolor{currentstroke}%
\pgfsetdash{}{0pt}%
\pgfpathmoveto{\pgfqpoint{4.176009in}{3.272329in}}%
\pgfpathlineto{\pgfqpoint{4.149062in}{3.256771in}}%
\pgfpathlineto{\pgfqpoint{4.176009in}{3.241213in}}%
\pgfpathlineto{\pgfqpoint{4.202957in}{3.256771in}}%
\pgfpathlineto{\pgfqpoint{4.176009in}{3.272329in}}%
\pgfpathclose%
\pgfusepath{fill}%
\end{pgfscope}%
\begin{pgfscope}%
\pgfpathrectangle{\pgfqpoint{0.765000in}{0.660000in}}{\pgfqpoint{4.620000in}{4.620000in}}%
\pgfusepath{clip}%
\pgfsetbuttcap%
\pgfsetroundjoin%
\definecolor{currentfill}{rgb}{1.000000,0.894118,0.788235}%
\pgfsetfillcolor{currentfill}%
\pgfsetlinewidth{0.000000pt}%
\definecolor{currentstroke}{rgb}{1.000000,0.894118,0.788235}%
\pgfsetstrokecolor{currentstroke}%
\pgfsetdash{}{0pt}%
\pgfpathmoveto{\pgfqpoint{4.176009in}{3.272329in}}%
\pgfpathlineto{\pgfqpoint{4.149062in}{3.256771in}}%
\pgfpathlineto{\pgfqpoint{4.149062in}{3.287887in}}%
\pgfpathlineto{\pgfqpoint{4.176009in}{3.303445in}}%
\pgfpathlineto{\pgfqpoint{4.176009in}{3.272329in}}%
\pgfpathclose%
\pgfusepath{fill}%
\end{pgfscope}%
\begin{pgfscope}%
\pgfpathrectangle{\pgfqpoint{0.765000in}{0.660000in}}{\pgfqpoint{4.620000in}{4.620000in}}%
\pgfusepath{clip}%
\pgfsetbuttcap%
\pgfsetroundjoin%
\definecolor{currentfill}{rgb}{1.000000,0.894118,0.788235}%
\pgfsetfillcolor{currentfill}%
\pgfsetlinewidth{0.000000pt}%
\definecolor{currentstroke}{rgb}{1.000000,0.894118,0.788235}%
\pgfsetstrokecolor{currentstroke}%
\pgfsetdash{}{0pt}%
\pgfpathmoveto{\pgfqpoint{4.176009in}{3.272329in}}%
\pgfpathlineto{\pgfqpoint{4.202957in}{3.256771in}}%
\pgfpathlineto{\pgfqpoint{4.202957in}{3.287887in}}%
\pgfpathlineto{\pgfqpoint{4.176009in}{3.303445in}}%
\pgfpathlineto{\pgfqpoint{4.176009in}{3.272329in}}%
\pgfpathclose%
\pgfusepath{fill}%
\end{pgfscope}%
\begin{pgfscope}%
\pgfpathrectangle{\pgfqpoint{0.765000in}{0.660000in}}{\pgfqpoint{4.620000in}{4.620000in}}%
\pgfusepath{clip}%
\pgfsetbuttcap%
\pgfsetroundjoin%
\definecolor{currentfill}{rgb}{1.000000,0.894118,0.788235}%
\pgfsetfillcolor{currentfill}%
\pgfsetlinewidth{0.000000pt}%
\definecolor{currentstroke}{rgb}{1.000000,0.894118,0.788235}%
\pgfsetstrokecolor{currentstroke}%
\pgfsetdash{}{0pt}%
\pgfpathmoveto{\pgfqpoint{4.123332in}{3.249597in}}%
\pgfpathlineto{\pgfqpoint{4.096385in}{3.234039in}}%
\pgfpathlineto{\pgfqpoint{4.123332in}{3.218481in}}%
\pgfpathlineto{\pgfqpoint{4.150279in}{3.234039in}}%
\pgfpathlineto{\pgfqpoint{4.123332in}{3.249597in}}%
\pgfpathclose%
\pgfusepath{fill}%
\end{pgfscope}%
\begin{pgfscope}%
\pgfpathrectangle{\pgfqpoint{0.765000in}{0.660000in}}{\pgfqpoint{4.620000in}{4.620000in}}%
\pgfusepath{clip}%
\pgfsetbuttcap%
\pgfsetroundjoin%
\definecolor{currentfill}{rgb}{1.000000,0.894118,0.788235}%
\pgfsetfillcolor{currentfill}%
\pgfsetlinewidth{0.000000pt}%
\definecolor{currentstroke}{rgb}{1.000000,0.894118,0.788235}%
\pgfsetstrokecolor{currentstroke}%
\pgfsetdash{}{0pt}%
\pgfpathmoveto{\pgfqpoint{4.123332in}{3.249597in}}%
\pgfpathlineto{\pgfqpoint{4.096385in}{3.234039in}}%
\pgfpathlineto{\pgfqpoint{4.096385in}{3.265155in}}%
\pgfpathlineto{\pgfqpoint{4.123332in}{3.280713in}}%
\pgfpathlineto{\pgfqpoint{4.123332in}{3.249597in}}%
\pgfpathclose%
\pgfusepath{fill}%
\end{pgfscope}%
\begin{pgfscope}%
\pgfpathrectangle{\pgfqpoint{0.765000in}{0.660000in}}{\pgfqpoint{4.620000in}{4.620000in}}%
\pgfusepath{clip}%
\pgfsetbuttcap%
\pgfsetroundjoin%
\definecolor{currentfill}{rgb}{1.000000,0.894118,0.788235}%
\pgfsetfillcolor{currentfill}%
\pgfsetlinewidth{0.000000pt}%
\definecolor{currentstroke}{rgb}{1.000000,0.894118,0.788235}%
\pgfsetstrokecolor{currentstroke}%
\pgfsetdash{}{0pt}%
\pgfpathmoveto{\pgfqpoint{4.123332in}{3.249597in}}%
\pgfpathlineto{\pgfqpoint{4.150279in}{3.234039in}}%
\pgfpathlineto{\pgfqpoint{4.150279in}{3.265155in}}%
\pgfpathlineto{\pgfqpoint{4.123332in}{3.280713in}}%
\pgfpathlineto{\pgfqpoint{4.123332in}{3.249597in}}%
\pgfpathclose%
\pgfusepath{fill}%
\end{pgfscope}%
\begin{pgfscope}%
\pgfpathrectangle{\pgfqpoint{0.765000in}{0.660000in}}{\pgfqpoint{4.620000in}{4.620000in}}%
\pgfusepath{clip}%
\pgfsetbuttcap%
\pgfsetroundjoin%
\definecolor{currentfill}{rgb}{1.000000,0.894118,0.788235}%
\pgfsetfillcolor{currentfill}%
\pgfsetlinewidth{0.000000pt}%
\definecolor{currentstroke}{rgb}{1.000000,0.894118,0.788235}%
\pgfsetstrokecolor{currentstroke}%
\pgfsetdash{}{0pt}%
\pgfpathmoveto{\pgfqpoint{4.176009in}{3.303445in}}%
\pgfpathlineto{\pgfqpoint{4.149062in}{3.287887in}}%
\pgfpathlineto{\pgfqpoint{4.176009in}{3.272329in}}%
\pgfpathlineto{\pgfqpoint{4.202957in}{3.287887in}}%
\pgfpathlineto{\pgfqpoint{4.176009in}{3.303445in}}%
\pgfpathclose%
\pgfusepath{fill}%
\end{pgfscope}%
\begin{pgfscope}%
\pgfpathrectangle{\pgfqpoint{0.765000in}{0.660000in}}{\pgfqpoint{4.620000in}{4.620000in}}%
\pgfusepath{clip}%
\pgfsetbuttcap%
\pgfsetroundjoin%
\definecolor{currentfill}{rgb}{1.000000,0.894118,0.788235}%
\pgfsetfillcolor{currentfill}%
\pgfsetlinewidth{0.000000pt}%
\definecolor{currentstroke}{rgb}{1.000000,0.894118,0.788235}%
\pgfsetstrokecolor{currentstroke}%
\pgfsetdash{}{0pt}%
\pgfpathmoveto{\pgfqpoint{4.176009in}{3.241213in}}%
\pgfpathlineto{\pgfqpoint{4.202957in}{3.256771in}}%
\pgfpathlineto{\pgfqpoint{4.202957in}{3.287887in}}%
\pgfpathlineto{\pgfqpoint{4.176009in}{3.272329in}}%
\pgfpathlineto{\pgfqpoint{4.176009in}{3.241213in}}%
\pgfpathclose%
\pgfusepath{fill}%
\end{pgfscope}%
\begin{pgfscope}%
\pgfpathrectangle{\pgfqpoint{0.765000in}{0.660000in}}{\pgfqpoint{4.620000in}{4.620000in}}%
\pgfusepath{clip}%
\pgfsetbuttcap%
\pgfsetroundjoin%
\definecolor{currentfill}{rgb}{1.000000,0.894118,0.788235}%
\pgfsetfillcolor{currentfill}%
\pgfsetlinewidth{0.000000pt}%
\definecolor{currentstroke}{rgb}{1.000000,0.894118,0.788235}%
\pgfsetstrokecolor{currentstroke}%
\pgfsetdash{}{0pt}%
\pgfpathmoveto{\pgfqpoint{4.149062in}{3.256771in}}%
\pgfpathlineto{\pgfqpoint{4.176009in}{3.241213in}}%
\pgfpathlineto{\pgfqpoint{4.176009in}{3.272329in}}%
\pgfpathlineto{\pgfqpoint{4.149062in}{3.287887in}}%
\pgfpathlineto{\pgfqpoint{4.149062in}{3.256771in}}%
\pgfpathclose%
\pgfusepath{fill}%
\end{pgfscope}%
\begin{pgfscope}%
\pgfpathrectangle{\pgfqpoint{0.765000in}{0.660000in}}{\pgfqpoint{4.620000in}{4.620000in}}%
\pgfusepath{clip}%
\pgfsetbuttcap%
\pgfsetroundjoin%
\definecolor{currentfill}{rgb}{1.000000,0.894118,0.788235}%
\pgfsetfillcolor{currentfill}%
\pgfsetlinewidth{0.000000pt}%
\definecolor{currentstroke}{rgb}{1.000000,0.894118,0.788235}%
\pgfsetstrokecolor{currentstroke}%
\pgfsetdash{}{0pt}%
\pgfpathmoveto{\pgfqpoint{4.123332in}{3.280713in}}%
\pgfpathlineto{\pgfqpoint{4.096385in}{3.265155in}}%
\pgfpathlineto{\pgfqpoint{4.123332in}{3.249597in}}%
\pgfpathlineto{\pgfqpoint{4.150279in}{3.265155in}}%
\pgfpathlineto{\pgfqpoint{4.123332in}{3.280713in}}%
\pgfpathclose%
\pgfusepath{fill}%
\end{pgfscope}%
\begin{pgfscope}%
\pgfpathrectangle{\pgfqpoint{0.765000in}{0.660000in}}{\pgfqpoint{4.620000in}{4.620000in}}%
\pgfusepath{clip}%
\pgfsetbuttcap%
\pgfsetroundjoin%
\definecolor{currentfill}{rgb}{1.000000,0.894118,0.788235}%
\pgfsetfillcolor{currentfill}%
\pgfsetlinewidth{0.000000pt}%
\definecolor{currentstroke}{rgb}{1.000000,0.894118,0.788235}%
\pgfsetstrokecolor{currentstroke}%
\pgfsetdash{}{0pt}%
\pgfpathmoveto{\pgfqpoint{4.123332in}{3.218481in}}%
\pgfpathlineto{\pgfqpoint{4.150279in}{3.234039in}}%
\pgfpathlineto{\pgfqpoint{4.150279in}{3.265155in}}%
\pgfpathlineto{\pgfqpoint{4.123332in}{3.249597in}}%
\pgfpathlineto{\pgfqpoint{4.123332in}{3.218481in}}%
\pgfpathclose%
\pgfusepath{fill}%
\end{pgfscope}%
\begin{pgfscope}%
\pgfpathrectangle{\pgfqpoint{0.765000in}{0.660000in}}{\pgfqpoint{4.620000in}{4.620000in}}%
\pgfusepath{clip}%
\pgfsetbuttcap%
\pgfsetroundjoin%
\definecolor{currentfill}{rgb}{1.000000,0.894118,0.788235}%
\pgfsetfillcolor{currentfill}%
\pgfsetlinewidth{0.000000pt}%
\definecolor{currentstroke}{rgb}{1.000000,0.894118,0.788235}%
\pgfsetstrokecolor{currentstroke}%
\pgfsetdash{}{0pt}%
\pgfpathmoveto{\pgfqpoint{4.096385in}{3.234039in}}%
\pgfpathlineto{\pgfqpoint{4.123332in}{3.218481in}}%
\pgfpathlineto{\pgfqpoint{4.123332in}{3.249597in}}%
\pgfpathlineto{\pgfqpoint{4.096385in}{3.265155in}}%
\pgfpathlineto{\pgfqpoint{4.096385in}{3.234039in}}%
\pgfpathclose%
\pgfusepath{fill}%
\end{pgfscope}%
\begin{pgfscope}%
\pgfpathrectangle{\pgfqpoint{0.765000in}{0.660000in}}{\pgfqpoint{4.620000in}{4.620000in}}%
\pgfusepath{clip}%
\pgfsetbuttcap%
\pgfsetroundjoin%
\definecolor{currentfill}{rgb}{1.000000,0.894118,0.788235}%
\pgfsetfillcolor{currentfill}%
\pgfsetlinewidth{0.000000pt}%
\definecolor{currentstroke}{rgb}{1.000000,0.894118,0.788235}%
\pgfsetstrokecolor{currentstroke}%
\pgfsetdash{}{0pt}%
\pgfpathmoveto{\pgfqpoint{4.176009in}{3.272329in}}%
\pgfpathlineto{\pgfqpoint{4.149062in}{3.256771in}}%
\pgfpathlineto{\pgfqpoint{4.096385in}{3.234039in}}%
\pgfpathlineto{\pgfqpoint{4.123332in}{3.249597in}}%
\pgfpathlineto{\pgfqpoint{4.176009in}{3.272329in}}%
\pgfpathclose%
\pgfusepath{fill}%
\end{pgfscope}%
\begin{pgfscope}%
\pgfpathrectangle{\pgfqpoint{0.765000in}{0.660000in}}{\pgfqpoint{4.620000in}{4.620000in}}%
\pgfusepath{clip}%
\pgfsetbuttcap%
\pgfsetroundjoin%
\definecolor{currentfill}{rgb}{1.000000,0.894118,0.788235}%
\pgfsetfillcolor{currentfill}%
\pgfsetlinewidth{0.000000pt}%
\definecolor{currentstroke}{rgb}{1.000000,0.894118,0.788235}%
\pgfsetstrokecolor{currentstroke}%
\pgfsetdash{}{0pt}%
\pgfpathmoveto{\pgfqpoint{4.202957in}{3.256771in}}%
\pgfpathlineto{\pgfqpoint{4.176009in}{3.272329in}}%
\pgfpathlineto{\pgfqpoint{4.123332in}{3.249597in}}%
\pgfpathlineto{\pgfqpoint{4.150279in}{3.234039in}}%
\pgfpathlineto{\pgfqpoint{4.202957in}{3.256771in}}%
\pgfpathclose%
\pgfusepath{fill}%
\end{pgfscope}%
\begin{pgfscope}%
\pgfpathrectangle{\pgfqpoint{0.765000in}{0.660000in}}{\pgfqpoint{4.620000in}{4.620000in}}%
\pgfusepath{clip}%
\pgfsetbuttcap%
\pgfsetroundjoin%
\definecolor{currentfill}{rgb}{1.000000,0.894118,0.788235}%
\pgfsetfillcolor{currentfill}%
\pgfsetlinewidth{0.000000pt}%
\definecolor{currentstroke}{rgb}{1.000000,0.894118,0.788235}%
\pgfsetstrokecolor{currentstroke}%
\pgfsetdash{}{0pt}%
\pgfpathmoveto{\pgfqpoint{4.176009in}{3.272329in}}%
\pgfpathlineto{\pgfqpoint{4.176009in}{3.303445in}}%
\pgfpathlineto{\pgfqpoint{4.123332in}{3.280713in}}%
\pgfpathlineto{\pgfqpoint{4.150279in}{3.234039in}}%
\pgfpathlineto{\pgfqpoint{4.176009in}{3.272329in}}%
\pgfpathclose%
\pgfusepath{fill}%
\end{pgfscope}%
\begin{pgfscope}%
\pgfpathrectangle{\pgfqpoint{0.765000in}{0.660000in}}{\pgfqpoint{4.620000in}{4.620000in}}%
\pgfusepath{clip}%
\pgfsetbuttcap%
\pgfsetroundjoin%
\definecolor{currentfill}{rgb}{1.000000,0.894118,0.788235}%
\pgfsetfillcolor{currentfill}%
\pgfsetlinewidth{0.000000pt}%
\definecolor{currentstroke}{rgb}{1.000000,0.894118,0.788235}%
\pgfsetstrokecolor{currentstroke}%
\pgfsetdash{}{0pt}%
\pgfpathmoveto{\pgfqpoint{4.202957in}{3.256771in}}%
\pgfpathlineto{\pgfqpoint{4.202957in}{3.287887in}}%
\pgfpathlineto{\pgfqpoint{4.150279in}{3.265155in}}%
\pgfpathlineto{\pgfqpoint{4.123332in}{3.249597in}}%
\pgfpathlineto{\pgfqpoint{4.202957in}{3.256771in}}%
\pgfpathclose%
\pgfusepath{fill}%
\end{pgfscope}%
\begin{pgfscope}%
\pgfpathrectangle{\pgfqpoint{0.765000in}{0.660000in}}{\pgfqpoint{4.620000in}{4.620000in}}%
\pgfusepath{clip}%
\pgfsetbuttcap%
\pgfsetroundjoin%
\definecolor{currentfill}{rgb}{1.000000,0.894118,0.788235}%
\pgfsetfillcolor{currentfill}%
\pgfsetlinewidth{0.000000pt}%
\definecolor{currentstroke}{rgb}{1.000000,0.894118,0.788235}%
\pgfsetstrokecolor{currentstroke}%
\pgfsetdash{}{0pt}%
\pgfpathmoveto{\pgfqpoint{4.149062in}{3.256771in}}%
\pgfpathlineto{\pgfqpoint{4.176009in}{3.241213in}}%
\pgfpathlineto{\pgfqpoint{4.123332in}{3.218481in}}%
\pgfpathlineto{\pgfqpoint{4.096385in}{3.234039in}}%
\pgfpathlineto{\pgfqpoint{4.149062in}{3.256771in}}%
\pgfpathclose%
\pgfusepath{fill}%
\end{pgfscope}%
\begin{pgfscope}%
\pgfpathrectangle{\pgfqpoint{0.765000in}{0.660000in}}{\pgfqpoint{4.620000in}{4.620000in}}%
\pgfusepath{clip}%
\pgfsetbuttcap%
\pgfsetroundjoin%
\definecolor{currentfill}{rgb}{1.000000,0.894118,0.788235}%
\pgfsetfillcolor{currentfill}%
\pgfsetlinewidth{0.000000pt}%
\definecolor{currentstroke}{rgb}{1.000000,0.894118,0.788235}%
\pgfsetstrokecolor{currentstroke}%
\pgfsetdash{}{0pt}%
\pgfpathmoveto{\pgfqpoint{4.176009in}{3.241213in}}%
\pgfpathlineto{\pgfqpoint{4.202957in}{3.256771in}}%
\pgfpathlineto{\pgfqpoint{4.150279in}{3.234039in}}%
\pgfpathlineto{\pgfqpoint{4.123332in}{3.218481in}}%
\pgfpathlineto{\pgfqpoint{4.176009in}{3.241213in}}%
\pgfpathclose%
\pgfusepath{fill}%
\end{pgfscope}%
\begin{pgfscope}%
\pgfpathrectangle{\pgfqpoint{0.765000in}{0.660000in}}{\pgfqpoint{4.620000in}{4.620000in}}%
\pgfusepath{clip}%
\pgfsetbuttcap%
\pgfsetroundjoin%
\definecolor{currentfill}{rgb}{1.000000,0.894118,0.788235}%
\pgfsetfillcolor{currentfill}%
\pgfsetlinewidth{0.000000pt}%
\definecolor{currentstroke}{rgb}{1.000000,0.894118,0.788235}%
\pgfsetstrokecolor{currentstroke}%
\pgfsetdash{}{0pt}%
\pgfpathmoveto{\pgfqpoint{4.176009in}{3.303445in}}%
\pgfpathlineto{\pgfqpoint{4.149062in}{3.287887in}}%
\pgfpathlineto{\pgfqpoint{4.096385in}{3.265155in}}%
\pgfpathlineto{\pgfqpoint{4.123332in}{3.280713in}}%
\pgfpathlineto{\pgfqpoint{4.176009in}{3.303445in}}%
\pgfpathclose%
\pgfusepath{fill}%
\end{pgfscope}%
\begin{pgfscope}%
\pgfpathrectangle{\pgfqpoint{0.765000in}{0.660000in}}{\pgfqpoint{4.620000in}{4.620000in}}%
\pgfusepath{clip}%
\pgfsetbuttcap%
\pgfsetroundjoin%
\definecolor{currentfill}{rgb}{1.000000,0.894118,0.788235}%
\pgfsetfillcolor{currentfill}%
\pgfsetlinewidth{0.000000pt}%
\definecolor{currentstroke}{rgb}{1.000000,0.894118,0.788235}%
\pgfsetstrokecolor{currentstroke}%
\pgfsetdash{}{0pt}%
\pgfpathmoveto{\pgfqpoint{4.202957in}{3.287887in}}%
\pgfpathlineto{\pgfqpoint{4.176009in}{3.303445in}}%
\pgfpathlineto{\pgfqpoint{4.123332in}{3.280713in}}%
\pgfpathlineto{\pgfqpoint{4.150279in}{3.265155in}}%
\pgfpathlineto{\pgfqpoint{4.202957in}{3.287887in}}%
\pgfpathclose%
\pgfusepath{fill}%
\end{pgfscope}%
\begin{pgfscope}%
\pgfpathrectangle{\pgfqpoint{0.765000in}{0.660000in}}{\pgfqpoint{4.620000in}{4.620000in}}%
\pgfusepath{clip}%
\pgfsetbuttcap%
\pgfsetroundjoin%
\definecolor{currentfill}{rgb}{1.000000,0.894118,0.788235}%
\pgfsetfillcolor{currentfill}%
\pgfsetlinewidth{0.000000pt}%
\definecolor{currentstroke}{rgb}{1.000000,0.894118,0.788235}%
\pgfsetstrokecolor{currentstroke}%
\pgfsetdash{}{0pt}%
\pgfpathmoveto{\pgfqpoint{4.149062in}{3.256771in}}%
\pgfpathlineto{\pgfqpoint{4.149062in}{3.287887in}}%
\pgfpathlineto{\pgfqpoint{4.096385in}{3.265155in}}%
\pgfpathlineto{\pgfqpoint{4.123332in}{3.218481in}}%
\pgfpathlineto{\pgfqpoint{4.149062in}{3.256771in}}%
\pgfpathclose%
\pgfusepath{fill}%
\end{pgfscope}%
\begin{pgfscope}%
\pgfpathrectangle{\pgfqpoint{0.765000in}{0.660000in}}{\pgfqpoint{4.620000in}{4.620000in}}%
\pgfusepath{clip}%
\pgfsetbuttcap%
\pgfsetroundjoin%
\definecolor{currentfill}{rgb}{1.000000,0.894118,0.788235}%
\pgfsetfillcolor{currentfill}%
\pgfsetlinewidth{0.000000pt}%
\definecolor{currentstroke}{rgb}{1.000000,0.894118,0.788235}%
\pgfsetstrokecolor{currentstroke}%
\pgfsetdash{}{0pt}%
\pgfpathmoveto{\pgfqpoint{4.176009in}{3.241213in}}%
\pgfpathlineto{\pgfqpoint{4.176009in}{3.272329in}}%
\pgfpathlineto{\pgfqpoint{4.123332in}{3.249597in}}%
\pgfpathlineto{\pgfqpoint{4.096385in}{3.234039in}}%
\pgfpathlineto{\pgfqpoint{4.176009in}{3.241213in}}%
\pgfpathclose%
\pgfusepath{fill}%
\end{pgfscope}%
\begin{pgfscope}%
\pgfpathrectangle{\pgfqpoint{0.765000in}{0.660000in}}{\pgfqpoint{4.620000in}{4.620000in}}%
\pgfusepath{clip}%
\pgfsetbuttcap%
\pgfsetroundjoin%
\definecolor{currentfill}{rgb}{1.000000,0.894118,0.788235}%
\pgfsetfillcolor{currentfill}%
\pgfsetlinewidth{0.000000pt}%
\definecolor{currentstroke}{rgb}{1.000000,0.894118,0.788235}%
\pgfsetstrokecolor{currentstroke}%
\pgfsetdash{}{0pt}%
\pgfpathmoveto{\pgfqpoint{4.176009in}{3.272329in}}%
\pgfpathlineto{\pgfqpoint{4.202957in}{3.287887in}}%
\pgfpathlineto{\pgfqpoint{4.150279in}{3.265155in}}%
\pgfpathlineto{\pgfqpoint{4.123332in}{3.249597in}}%
\pgfpathlineto{\pgfqpoint{4.176009in}{3.272329in}}%
\pgfpathclose%
\pgfusepath{fill}%
\end{pgfscope}%
\begin{pgfscope}%
\pgfpathrectangle{\pgfqpoint{0.765000in}{0.660000in}}{\pgfqpoint{4.620000in}{4.620000in}}%
\pgfusepath{clip}%
\pgfsetbuttcap%
\pgfsetroundjoin%
\definecolor{currentfill}{rgb}{1.000000,0.894118,0.788235}%
\pgfsetfillcolor{currentfill}%
\pgfsetlinewidth{0.000000pt}%
\definecolor{currentstroke}{rgb}{1.000000,0.894118,0.788235}%
\pgfsetstrokecolor{currentstroke}%
\pgfsetdash{}{0pt}%
\pgfpathmoveto{\pgfqpoint{4.149062in}{3.287887in}}%
\pgfpathlineto{\pgfqpoint{4.176009in}{3.272329in}}%
\pgfpathlineto{\pgfqpoint{4.123332in}{3.249597in}}%
\pgfpathlineto{\pgfqpoint{4.096385in}{3.265155in}}%
\pgfpathlineto{\pgfqpoint{4.149062in}{3.287887in}}%
\pgfpathclose%
\pgfusepath{fill}%
\end{pgfscope}%
\begin{pgfscope}%
\pgfpathrectangle{\pgfqpoint{0.765000in}{0.660000in}}{\pgfqpoint{4.620000in}{4.620000in}}%
\pgfusepath{clip}%
\pgfsetbuttcap%
\pgfsetroundjoin%
\definecolor{currentfill}{rgb}{1.000000,0.894118,0.788235}%
\pgfsetfillcolor{currentfill}%
\pgfsetlinewidth{0.000000pt}%
\definecolor{currentstroke}{rgb}{1.000000,0.894118,0.788235}%
\pgfsetstrokecolor{currentstroke}%
\pgfsetdash{}{0pt}%
\pgfpathmoveto{\pgfqpoint{2.605373in}{3.227174in}}%
\pgfpathlineto{\pgfqpoint{2.578426in}{3.211616in}}%
\pgfpathlineto{\pgfqpoint{2.605373in}{3.196058in}}%
\pgfpathlineto{\pgfqpoint{2.632320in}{3.211616in}}%
\pgfpathlineto{\pgfqpoint{2.605373in}{3.227174in}}%
\pgfpathclose%
\pgfusepath{fill}%
\end{pgfscope}%
\begin{pgfscope}%
\pgfpathrectangle{\pgfqpoint{0.765000in}{0.660000in}}{\pgfqpoint{4.620000in}{4.620000in}}%
\pgfusepath{clip}%
\pgfsetbuttcap%
\pgfsetroundjoin%
\definecolor{currentfill}{rgb}{1.000000,0.894118,0.788235}%
\pgfsetfillcolor{currentfill}%
\pgfsetlinewidth{0.000000pt}%
\definecolor{currentstroke}{rgb}{1.000000,0.894118,0.788235}%
\pgfsetstrokecolor{currentstroke}%
\pgfsetdash{}{0pt}%
\pgfpathmoveto{\pgfqpoint{2.605373in}{3.227174in}}%
\pgfpathlineto{\pgfqpoint{2.578426in}{3.211616in}}%
\pgfpathlineto{\pgfqpoint{2.578426in}{3.242732in}}%
\pgfpathlineto{\pgfqpoint{2.605373in}{3.258290in}}%
\pgfpathlineto{\pgfqpoint{2.605373in}{3.227174in}}%
\pgfpathclose%
\pgfusepath{fill}%
\end{pgfscope}%
\begin{pgfscope}%
\pgfpathrectangle{\pgfqpoint{0.765000in}{0.660000in}}{\pgfqpoint{4.620000in}{4.620000in}}%
\pgfusepath{clip}%
\pgfsetbuttcap%
\pgfsetroundjoin%
\definecolor{currentfill}{rgb}{1.000000,0.894118,0.788235}%
\pgfsetfillcolor{currentfill}%
\pgfsetlinewidth{0.000000pt}%
\definecolor{currentstroke}{rgb}{1.000000,0.894118,0.788235}%
\pgfsetstrokecolor{currentstroke}%
\pgfsetdash{}{0pt}%
\pgfpathmoveto{\pgfqpoint{2.605373in}{3.227174in}}%
\pgfpathlineto{\pgfqpoint{2.632320in}{3.211616in}}%
\pgfpathlineto{\pgfqpoint{2.632320in}{3.242732in}}%
\pgfpathlineto{\pgfqpoint{2.605373in}{3.258290in}}%
\pgfpathlineto{\pgfqpoint{2.605373in}{3.227174in}}%
\pgfpathclose%
\pgfusepath{fill}%
\end{pgfscope}%
\begin{pgfscope}%
\pgfpathrectangle{\pgfqpoint{0.765000in}{0.660000in}}{\pgfqpoint{4.620000in}{4.620000in}}%
\pgfusepath{clip}%
\pgfsetbuttcap%
\pgfsetroundjoin%
\definecolor{currentfill}{rgb}{1.000000,0.894118,0.788235}%
\pgfsetfillcolor{currentfill}%
\pgfsetlinewidth{0.000000pt}%
\definecolor{currentstroke}{rgb}{1.000000,0.894118,0.788235}%
\pgfsetstrokecolor{currentstroke}%
\pgfsetdash{}{0pt}%
\pgfpathmoveto{\pgfqpoint{2.716919in}{3.259885in}}%
\pgfpathlineto{\pgfqpoint{2.689971in}{3.244327in}}%
\pgfpathlineto{\pgfqpoint{2.716919in}{3.228769in}}%
\pgfpathlineto{\pgfqpoint{2.743866in}{3.244327in}}%
\pgfpathlineto{\pgfqpoint{2.716919in}{3.259885in}}%
\pgfpathclose%
\pgfusepath{fill}%
\end{pgfscope}%
\begin{pgfscope}%
\pgfpathrectangle{\pgfqpoint{0.765000in}{0.660000in}}{\pgfqpoint{4.620000in}{4.620000in}}%
\pgfusepath{clip}%
\pgfsetbuttcap%
\pgfsetroundjoin%
\definecolor{currentfill}{rgb}{1.000000,0.894118,0.788235}%
\pgfsetfillcolor{currentfill}%
\pgfsetlinewidth{0.000000pt}%
\definecolor{currentstroke}{rgb}{1.000000,0.894118,0.788235}%
\pgfsetstrokecolor{currentstroke}%
\pgfsetdash{}{0pt}%
\pgfpathmoveto{\pgfqpoint{2.716919in}{3.259885in}}%
\pgfpathlineto{\pgfqpoint{2.689971in}{3.244327in}}%
\pgfpathlineto{\pgfqpoint{2.689971in}{3.275443in}}%
\pgfpathlineto{\pgfqpoint{2.716919in}{3.291001in}}%
\pgfpathlineto{\pgfqpoint{2.716919in}{3.259885in}}%
\pgfpathclose%
\pgfusepath{fill}%
\end{pgfscope}%
\begin{pgfscope}%
\pgfpathrectangle{\pgfqpoint{0.765000in}{0.660000in}}{\pgfqpoint{4.620000in}{4.620000in}}%
\pgfusepath{clip}%
\pgfsetbuttcap%
\pgfsetroundjoin%
\definecolor{currentfill}{rgb}{1.000000,0.894118,0.788235}%
\pgfsetfillcolor{currentfill}%
\pgfsetlinewidth{0.000000pt}%
\definecolor{currentstroke}{rgb}{1.000000,0.894118,0.788235}%
\pgfsetstrokecolor{currentstroke}%
\pgfsetdash{}{0pt}%
\pgfpathmoveto{\pgfqpoint{2.716919in}{3.259885in}}%
\pgfpathlineto{\pgfqpoint{2.743866in}{3.244327in}}%
\pgfpathlineto{\pgfqpoint{2.743866in}{3.275443in}}%
\pgfpathlineto{\pgfqpoint{2.716919in}{3.291001in}}%
\pgfpathlineto{\pgfqpoint{2.716919in}{3.259885in}}%
\pgfpathclose%
\pgfusepath{fill}%
\end{pgfscope}%
\begin{pgfscope}%
\pgfpathrectangle{\pgfqpoint{0.765000in}{0.660000in}}{\pgfqpoint{4.620000in}{4.620000in}}%
\pgfusepath{clip}%
\pgfsetbuttcap%
\pgfsetroundjoin%
\definecolor{currentfill}{rgb}{1.000000,0.894118,0.788235}%
\pgfsetfillcolor{currentfill}%
\pgfsetlinewidth{0.000000pt}%
\definecolor{currentstroke}{rgb}{1.000000,0.894118,0.788235}%
\pgfsetstrokecolor{currentstroke}%
\pgfsetdash{}{0pt}%
\pgfpathmoveto{\pgfqpoint{2.605373in}{3.258290in}}%
\pgfpathlineto{\pgfqpoint{2.578426in}{3.242732in}}%
\pgfpathlineto{\pgfqpoint{2.605373in}{3.227174in}}%
\pgfpathlineto{\pgfqpoint{2.632320in}{3.242732in}}%
\pgfpathlineto{\pgfqpoint{2.605373in}{3.258290in}}%
\pgfpathclose%
\pgfusepath{fill}%
\end{pgfscope}%
\begin{pgfscope}%
\pgfpathrectangle{\pgfqpoint{0.765000in}{0.660000in}}{\pgfqpoint{4.620000in}{4.620000in}}%
\pgfusepath{clip}%
\pgfsetbuttcap%
\pgfsetroundjoin%
\definecolor{currentfill}{rgb}{1.000000,0.894118,0.788235}%
\pgfsetfillcolor{currentfill}%
\pgfsetlinewidth{0.000000pt}%
\definecolor{currentstroke}{rgb}{1.000000,0.894118,0.788235}%
\pgfsetstrokecolor{currentstroke}%
\pgfsetdash{}{0pt}%
\pgfpathmoveto{\pgfqpoint{2.605373in}{3.196058in}}%
\pgfpathlineto{\pgfqpoint{2.632320in}{3.211616in}}%
\pgfpathlineto{\pgfqpoint{2.632320in}{3.242732in}}%
\pgfpathlineto{\pgfqpoint{2.605373in}{3.227174in}}%
\pgfpathlineto{\pgfqpoint{2.605373in}{3.196058in}}%
\pgfpathclose%
\pgfusepath{fill}%
\end{pgfscope}%
\begin{pgfscope}%
\pgfpathrectangle{\pgfqpoint{0.765000in}{0.660000in}}{\pgfqpoint{4.620000in}{4.620000in}}%
\pgfusepath{clip}%
\pgfsetbuttcap%
\pgfsetroundjoin%
\definecolor{currentfill}{rgb}{1.000000,0.894118,0.788235}%
\pgfsetfillcolor{currentfill}%
\pgfsetlinewidth{0.000000pt}%
\definecolor{currentstroke}{rgb}{1.000000,0.894118,0.788235}%
\pgfsetstrokecolor{currentstroke}%
\pgfsetdash{}{0pt}%
\pgfpathmoveto{\pgfqpoint{2.578426in}{3.211616in}}%
\pgfpathlineto{\pgfqpoint{2.605373in}{3.196058in}}%
\pgfpathlineto{\pgfqpoint{2.605373in}{3.227174in}}%
\pgfpathlineto{\pgfqpoint{2.578426in}{3.242732in}}%
\pgfpathlineto{\pgfqpoint{2.578426in}{3.211616in}}%
\pgfpathclose%
\pgfusepath{fill}%
\end{pgfscope}%
\begin{pgfscope}%
\pgfpathrectangle{\pgfqpoint{0.765000in}{0.660000in}}{\pgfqpoint{4.620000in}{4.620000in}}%
\pgfusepath{clip}%
\pgfsetbuttcap%
\pgfsetroundjoin%
\definecolor{currentfill}{rgb}{1.000000,0.894118,0.788235}%
\pgfsetfillcolor{currentfill}%
\pgfsetlinewidth{0.000000pt}%
\definecolor{currentstroke}{rgb}{1.000000,0.894118,0.788235}%
\pgfsetstrokecolor{currentstroke}%
\pgfsetdash{}{0pt}%
\pgfpathmoveto{\pgfqpoint{2.716919in}{3.291001in}}%
\pgfpathlineto{\pgfqpoint{2.689971in}{3.275443in}}%
\pgfpathlineto{\pgfqpoint{2.716919in}{3.259885in}}%
\pgfpathlineto{\pgfqpoint{2.743866in}{3.275443in}}%
\pgfpathlineto{\pgfqpoint{2.716919in}{3.291001in}}%
\pgfpathclose%
\pgfusepath{fill}%
\end{pgfscope}%
\begin{pgfscope}%
\pgfpathrectangle{\pgfqpoint{0.765000in}{0.660000in}}{\pgfqpoint{4.620000in}{4.620000in}}%
\pgfusepath{clip}%
\pgfsetbuttcap%
\pgfsetroundjoin%
\definecolor{currentfill}{rgb}{1.000000,0.894118,0.788235}%
\pgfsetfillcolor{currentfill}%
\pgfsetlinewidth{0.000000pt}%
\definecolor{currentstroke}{rgb}{1.000000,0.894118,0.788235}%
\pgfsetstrokecolor{currentstroke}%
\pgfsetdash{}{0pt}%
\pgfpathmoveto{\pgfqpoint{2.716919in}{3.228769in}}%
\pgfpathlineto{\pgfqpoint{2.743866in}{3.244327in}}%
\pgfpathlineto{\pgfqpoint{2.743866in}{3.275443in}}%
\pgfpathlineto{\pgfqpoint{2.716919in}{3.259885in}}%
\pgfpathlineto{\pgfqpoint{2.716919in}{3.228769in}}%
\pgfpathclose%
\pgfusepath{fill}%
\end{pgfscope}%
\begin{pgfscope}%
\pgfpathrectangle{\pgfqpoint{0.765000in}{0.660000in}}{\pgfqpoint{4.620000in}{4.620000in}}%
\pgfusepath{clip}%
\pgfsetbuttcap%
\pgfsetroundjoin%
\definecolor{currentfill}{rgb}{1.000000,0.894118,0.788235}%
\pgfsetfillcolor{currentfill}%
\pgfsetlinewidth{0.000000pt}%
\definecolor{currentstroke}{rgb}{1.000000,0.894118,0.788235}%
\pgfsetstrokecolor{currentstroke}%
\pgfsetdash{}{0pt}%
\pgfpathmoveto{\pgfqpoint{2.689971in}{3.244327in}}%
\pgfpathlineto{\pgfqpoint{2.716919in}{3.228769in}}%
\pgfpathlineto{\pgfqpoint{2.716919in}{3.259885in}}%
\pgfpathlineto{\pgfqpoint{2.689971in}{3.275443in}}%
\pgfpathlineto{\pgfqpoint{2.689971in}{3.244327in}}%
\pgfpathclose%
\pgfusepath{fill}%
\end{pgfscope}%
\begin{pgfscope}%
\pgfpathrectangle{\pgfqpoint{0.765000in}{0.660000in}}{\pgfqpoint{4.620000in}{4.620000in}}%
\pgfusepath{clip}%
\pgfsetbuttcap%
\pgfsetroundjoin%
\definecolor{currentfill}{rgb}{1.000000,0.894118,0.788235}%
\pgfsetfillcolor{currentfill}%
\pgfsetlinewidth{0.000000pt}%
\definecolor{currentstroke}{rgb}{1.000000,0.894118,0.788235}%
\pgfsetstrokecolor{currentstroke}%
\pgfsetdash{}{0pt}%
\pgfpathmoveto{\pgfqpoint{2.605373in}{3.227174in}}%
\pgfpathlineto{\pgfqpoint{2.578426in}{3.211616in}}%
\pgfpathlineto{\pgfqpoint{2.689971in}{3.244327in}}%
\pgfpathlineto{\pgfqpoint{2.716919in}{3.259885in}}%
\pgfpathlineto{\pgfqpoint{2.605373in}{3.227174in}}%
\pgfpathclose%
\pgfusepath{fill}%
\end{pgfscope}%
\begin{pgfscope}%
\pgfpathrectangle{\pgfqpoint{0.765000in}{0.660000in}}{\pgfqpoint{4.620000in}{4.620000in}}%
\pgfusepath{clip}%
\pgfsetbuttcap%
\pgfsetroundjoin%
\definecolor{currentfill}{rgb}{1.000000,0.894118,0.788235}%
\pgfsetfillcolor{currentfill}%
\pgfsetlinewidth{0.000000pt}%
\definecolor{currentstroke}{rgb}{1.000000,0.894118,0.788235}%
\pgfsetstrokecolor{currentstroke}%
\pgfsetdash{}{0pt}%
\pgfpathmoveto{\pgfqpoint{2.632320in}{3.211616in}}%
\pgfpathlineto{\pgfqpoint{2.605373in}{3.227174in}}%
\pgfpathlineto{\pgfqpoint{2.716919in}{3.259885in}}%
\pgfpathlineto{\pgfqpoint{2.743866in}{3.244327in}}%
\pgfpathlineto{\pgfqpoint{2.632320in}{3.211616in}}%
\pgfpathclose%
\pgfusepath{fill}%
\end{pgfscope}%
\begin{pgfscope}%
\pgfpathrectangle{\pgfqpoint{0.765000in}{0.660000in}}{\pgfqpoint{4.620000in}{4.620000in}}%
\pgfusepath{clip}%
\pgfsetbuttcap%
\pgfsetroundjoin%
\definecolor{currentfill}{rgb}{1.000000,0.894118,0.788235}%
\pgfsetfillcolor{currentfill}%
\pgfsetlinewidth{0.000000pt}%
\definecolor{currentstroke}{rgb}{1.000000,0.894118,0.788235}%
\pgfsetstrokecolor{currentstroke}%
\pgfsetdash{}{0pt}%
\pgfpathmoveto{\pgfqpoint{2.605373in}{3.227174in}}%
\pgfpathlineto{\pgfqpoint{2.605373in}{3.258290in}}%
\pgfpathlineto{\pgfqpoint{2.716919in}{3.291001in}}%
\pgfpathlineto{\pgfqpoint{2.743866in}{3.244327in}}%
\pgfpathlineto{\pgfqpoint{2.605373in}{3.227174in}}%
\pgfpathclose%
\pgfusepath{fill}%
\end{pgfscope}%
\begin{pgfscope}%
\pgfpathrectangle{\pgfqpoint{0.765000in}{0.660000in}}{\pgfqpoint{4.620000in}{4.620000in}}%
\pgfusepath{clip}%
\pgfsetbuttcap%
\pgfsetroundjoin%
\definecolor{currentfill}{rgb}{1.000000,0.894118,0.788235}%
\pgfsetfillcolor{currentfill}%
\pgfsetlinewidth{0.000000pt}%
\definecolor{currentstroke}{rgb}{1.000000,0.894118,0.788235}%
\pgfsetstrokecolor{currentstroke}%
\pgfsetdash{}{0pt}%
\pgfpathmoveto{\pgfqpoint{2.632320in}{3.211616in}}%
\pgfpathlineto{\pgfqpoint{2.632320in}{3.242732in}}%
\pgfpathlineto{\pgfqpoint{2.743866in}{3.275443in}}%
\pgfpathlineto{\pgfqpoint{2.716919in}{3.259885in}}%
\pgfpathlineto{\pgfqpoint{2.632320in}{3.211616in}}%
\pgfpathclose%
\pgfusepath{fill}%
\end{pgfscope}%
\begin{pgfscope}%
\pgfpathrectangle{\pgfqpoint{0.765000in}{0.660000in}}{\pgfqpoint{4.620000in}{4.620000in}}%
\pgfusepath{clip}%
\pgfsetbuttcap%
\pgfsetroundjoin%
\definecolor{currentfill}{rgb}{1.000000,0.894118,0.788235}%
\pgfsetfillcolor{currentfill}%
\pgfsetlinewidth{0.000000pt}%
\definecolor{currentstroke}{rgb}{1.000000,0.894118,0.788235}%
\pgfsetstrokecolor{currentstroke}%
\pgfsetdash{}{0pt}%
\pgfpathmoveto{\pgfqpoint{2.578426in}{3.211616in}}%
\pgfpathlineto{\pgfqpoint{2.605373in}{3.196058in}}%
\pgfpathlineto{\pgfqpoint{2.716919in}{3.228769in}}%
\pgfpathlineto{\pgfqpoint{2.689971in}{3.244327in}}%
\pgfpathlineto{\pgfqpoint{2.578426in}{3.211616in}}%
\pgfpathclose%
\pgfusepath{fill}%
\end{pgfscope}%
\begin{pgfscope}%
\pgfpathrectangle{\pgfqpoint{0.765000in}{0.660000in}}{\pgfqpoint{4.620000in}{4.620000in}}%
\pgfusepath{clip}%
\pgfsetbuttcap%
\pgfsetroundjoin%
\definecolor{currentfill}{rgb}{1.000000,0.894118,0.788235}%
\pgfsetfillcolor{currentfill}%
\pgfsetlinewidth{0.000000pt}%
\definecolor{currentstroke}{rgb}{1.000000,0.894118,0.788235}%
\pgfsetstrokecolor{currentstroke}%
\pgfsetdash{}{0pt}%
\pgfpathmoveto{\pgfqpoint{2.605373in}{3.196058in}}%
\pgfpathlineto{\pgfqpoint{2.632320in}{3.211616in}}%
\pgfpathlineto{\pgfqpoint{2.743866in}{3.244327in}}%
\pgfpathlineto{\pgfqpoint{2.716919in}{3.228769in}}%
\pgfpathlineto{\pgfqpoint{2.605373in}{3.196058in}}%
\pgfpathclose%
\pgfusepath{fill}%
\end{pgfscope}%
\begin{pgfscope}%
\pgfpathrectangle{\pgfqpoint{0.765000in}{0.660000in}}{\pgfqpoint{4.620000in}{4.620000in}}%
\pgfusepath{clip}%
\pgfsetbuttcap%
\pgfsetroundjoin%
\definecolor{currentfill}{rgb}{1.000000,0.894118,0.788235}%
\pgfsetfillcolor{currentfill}%
\pgfsetlinewidth{0.000000pt}%
\definecolor{currentstroke}{rgb}{1.000000,0.894118,0.788235}%
\pgfsetstrokecolor{currentstroke}%
\pgfsetdash{}{0pt}%
\pgfpathmoveto{\pgfqpoint{2.605373in}{3.258290in}}%
\pgfpathlineto{\pgfqpoint{2.578426in}{3.242732in}}%
\pgfpathlineto{\pgfqpoint{2.689971in}{3.275443in}}%
\pgfpathlineto{\pgfqpoint{2.716919in}{3.291001in}}%
\pgfpathlineto{\pgfqpoint{2.605373in}{3.258290in}}%
\pgfpathclose%
\pgfusepath{fill}%
\end{pgfscope}%
\begin{pgfscope}%
\pgfpathrectangle{\pgfqpoint{0.765000in}{0.660000in}}{\pgfqpoint{4.620000in}{4.620000in}}%
\pgfusepath{clip}%
\pgfsetbuttcap%
\pgfsetroundjoin%
\definecolor{currentfill}{rgb}{1.000000,0.894118,0.788235}%
\pgfsetfillcolor{currentfill}%
\pgfsetlinewidth{0.000000pt}%
\definecolor{currentstroke}{rgb}{1.000000,0.894118,0.788235}%
\pgfsetstrokecolor{currentstroke}%
\pgfsetdash{}{0pt}%
\pgfpathmoveto{\pgfqpoint{2.632320in}{3.242732in}}%
\pgfpathlineto{\pgfqpoint{2.605373in}{3.258290in}}%
\pgfpathlineto{\pgfqpoint{2.716919in}{3.291001in}}%
\pgfpathlineto{\pgfqpoint{2.743866in}{3.275443in}}%
\pgfpathlineto{\pgfqpoint{2.632320in}{3.242732in}}%
\pgfpathclose%
\pgfusepath{fill}%
\end{pgfscope}%
\begin{pgfscope}%
\pgfpathrectangle{\pgfqpoint{0.765000in}{0.660000in}}{\pgfqpoint{4.620000in}{4.620000in}}%
\pgfusepath{clip}%
\pgfsetbuttcap%
\pgfsetroundjoin%
\definecolor{currentfill}{rgb}{1.000000,0.894118,0.788235}%
\pgfsetfillcolor{currentfill}%
\pgfsetlinewidth{0.000000pt}%
\definecolor{currentstroke}{rgb}{1.000000,0.894118,0.788235}%
\pgfsetstrokecolor{currentstroke}%
\pgfsetdash{}{0pt}%
\pgfpathmoveto{\pgfqpoint{2.578426in}{3.211616in}}%
\pgfpathlineto{\pgfqpoint{2.578426in}{3.242732in}}%
\pgfpathlineto{\pgfqpoint{2.689971in}{3.275443in}}%
\pgfpathlineto{\pgfqpoint{2.716919in}{3.228769in}}%
\pgfpathlineto{\pgfqpoint{2.578426in}{3.211616in}}%
\pgfpathclose%
\pgfusepath{fill}%
\end{pgfscope}%
\begin{pgfscope}%
\pgfpathrectangle{\pgfqpoint{0.765000in}{0.660000in}}{\pgfqpoint{4.620000in}{4.620000in}}%
\pgfusepath{clip}%
\pgfsetbuttcap%
\pgfsetroundjoin%
\definecolor{currentfill}{rgb}{1.000000,0.894118,0.788235}%
\pgfsetfillcolor{currentfill}%
\pgfsetlinewidth{0.000000pt}%
\definecolor{currentstroke}{rgb}{1.000000,0.894118,0.788235}%
\pgfsetstrokecolor{currentstroke}%
\pgfsetdash{}{0pt}%
\pgfpathmoveto{\pgfqpoint{2.605373in}{3.196058in}}%
\pgfpathlineto{\pgfqpoint{2.605373in}{3.227174in}}%
\pgfpathlineto{\pgfqpoint{2.716919in}{3.259885in}}%
\pgfpathlineto{\pgfqpoint{2.689971in}{3.244327in}}%
\pgfpathlineto{\pgfqpoint{2.605373in}{3.196058in}}%
\pgfpathclose%
\pgfusepath{fill}%
\end{pgfscope}%
\begin{pgfscope}%
\pgfpathrectangle{\pgfqpoint{0.765000in}{0.660000in}}{\pgfqpoint{4.620000in}{4.620000in}}%
\pgfusepath{clip}%
\pgfsetbuttcap%
\pgfsetroundjoin%
\definecolor{currentfill}{rgb}{1.000000,0.894118,0.788235}%
\pgfsetfillcolor{currentfill}%
\pgfsetlinewidth{0.000000pt}%
\definecolor{currentstroke}{rgb}{1.000000,0.894118,0.788235}%
\pgfsetstrokecolor{currentstroke}%
\pgfsetdash{}{0pt}%
\pgfpathmoveto{\pgfqpoint{2.605373in}{3.227174in}}%
\pgfpathlineto{\pgfqpoint{2.632320in}{3.242732in}}%
\pgfpathlineto{\pgfqpoint{2.743866in}{3.275443in}}%
\pgfpathlineto{\pgfqpoint{2.716919in}{3.259885in}}%
\pgfpathlineto{\pgfqpoint{2.605373in}{3.227174in}}%
\pgfpathclose%
\pgfusepath{fill}%
\end{pgfscope}%
\begin{pgfscope}%
\pgfpathrectangle{\pgfqpoint{0.765000in}{0.660000in}}{\pgfqpoint{4.620000in}{4.620000in}}%
\pgfusepath{clip}%
\pgfsetbuttcap%
\pgfsetroundjoin%
\definecolor{currentfill}{rgb}{1.000000,0.894118,0.788235}%
\pgfsetfillcolor{currentfill}%
\pgfsetlinewidth{0.000000pt}%
\definecolor{currentstroke}{rgb}{1.000000,0.894118,0.788235}%
\pgfsetstrokecolor{currentstroke}%
\pgfsetdash{}{0pt}%
\pgfpathmoveto{\pgfqpoint{2.578426in}{3.242732in}}%
\pgfpathlineto{\pgfqpoint{2.605373in}{3.227174in}}%
\pgfpathlineto{\pgfqpoint{2.716919in}{3.259885in}}%
\pgfpathlineto{\pgfqpoint{2.689971in}{3.275443in}}%
\pgfpathlineto{\pgfqpoint{2.578426in}{3.242732in}}%
\pgfpathclose%
\pgfusepath{fill}%
\end{pgfscope}%
\begin{pgfscope}%
\pgfpathrectangle{\pgfqpoint{0.765000in}{0.660000in}}{\pgfqpoint{4.620000in}{4.620000in}}%
\pgfusepath{clip}%
\pgfsetbuttcap%
\pgfsetroundjoin%
\definecolor{currentfill}{rgb}{1.000000,0.894118,0.788235}%
\pgfsetfillcolor{currentfill}%
\pgfsetlinewidth{0.000000pt}%
\definecolor{currentstroke}{rgb}{1.000000,0.894118,0.788235}%
\pgfsetstrokecolor{currentstroke}%
\pgfsetdash{}{0pt}%
\pgfpathmoveto{\pgfqpoint{2.605373in}{3.227174in}}%
\pgfpathlineto{\pgfqpoint{2.578426in}{3.211616in}}%
\pgfpathlineto{\pgfqpoint{2.605373in}{3.196058in}}%
\pgfpathlineto{\pgfqpoint{2.632320in}{3.211616in}}%
\pgfpathlineto{\pgfqpoint{2.605373in}{3.227174in}}%
\pgfpathclose%
\pgfusepath{fill}%
\end{pgfscope}%
\begin{pgfscope}%
\pgfpathrectangle{\pgfqpoint{0.765000in}{0.660000in}}{\pgfqpoint{4.620000in}{4.620000in}}%
\pgfusepath{clip}%
\pgfsetbuttcap%
\pgfsetroundjoin%
\definecolor{currentfill}{rgb}{1.000000,0.894118,0.788235}%
\pgfsetfillcolor{currentfill}%
\pgfsetlinewidth{0.000000pt}%
\definecolor{currentstroke}{rgb}{1.000000,0.894118,0.788235}%
\pgfsetstrokecolor{currentstroke}%
\pgfsetdash{}{0pt}%
\pgfpathmoveto{\pgfqpoint{2.605373in}{3.227174in}}%
\pgfpathlineto{\pgfqpoint{2.578426in}{3.211616in}}%
\pgfpathlineto{\pgfqpoint{2.578426in}{3.242732in}}%
\pgfpathlineto{\pgfqpoint{2.605373in}{3.258290in}}%
\pgfpathlineto{\pgfqpoint{2.605373in}{3.227174in}}%
\pgfpathclose%
\pgfusepath{fill}%
\end{pgfscope}%
\begin{pgfscope}%
\pgfpathrectangle{\pgfqpoint{0.765000in}{0.660000in}}{\pgfqpoint{4.620000in}{4.620000in}}%
\pgfusepath{clip}%
\pgfsetbuttcap%
\pgfsetroundjoin%
\definecolor{currentfill}{rgb}{1.000000,0.894118,0.788235}%
\pgfsetfillcolor{currentfill}%
\pgfsetlinewidth{0.000000pt}%
\definecolor{currentstroke}{rgb}{1.000000,0.894118,0.788235}%
\pgfsetstrokecolor{currentstroke}%
\pgfsetdash{}{0pt}%
\pgfpathmoveto{\pgfqpoint{2.605373in}{3.227174in}}%
\pgfpathlineto{\pgfqpoint{2.632320in}{3.211616in}}%
\pgfpathlineto{\pgfqpoint{2.632320in}{3.242732in}}%
\pgfpathlineto{\pgfqpoint{2.605373in}{3.258290in}}%
\pgfpathlineto{\pgfqpoint{2.605373in}{3.227174in}}%
\pgfpathclose%
\pgfusepath{fill}%
\end{pgfscope}%
\begin{pgfscope}%
\pgfpathrectangle{\pgfqpoint{0.765000in}{0.660000in}}{\pgfqpoint{4.620000in}{4.620000in}}%
\pgfusepath{clip}%
\pgfsetbuttcap%
\pgfsetroundjoin%
\definecolor{currentfill}{rgb}{1.000000,0.894118,0.788235}%
\pgfsetfillcolor{currentfill}%
\pgfsetlinewidth{0.000000pt}%
\definecolor{currentstroke}{rgb}{1.000000,0.894118,0.788235}%
\pgfsetstrokecolor{currentstroke}%
\pgfsetdash{}{0pt}%
\pgfpathmoveto{\pgfqpoint{2.595009in}{3.223964in}}%
\pgfpathlineto{\pgfqpoint{2.568062in}{3.208406in}}%
\pgfpathlineto{\pgfqpoint{2.595009in}{3.192848in}}%
\pgfpathlineto{\pgfqpoint{2.621957in}{3.208406in}}%
\pgfpathlineto{\pgfqpoint{2.595009in}{3.223964in}}%
\pgfpathclose%
\pgfusepath{fill}%
\end{pgfscope}%
\begin{pgfscope}%
\pgfpathrectangle{\pgfqpoint{0.765000in}{0.660000in}}{\pgfqpoint{4.620000in}{4.620000in}}%
\pgfusepath{clip}%
\pgfsetbuttcap%
\pgfsetroundjoin%
\definecolor{currentfill}{rgb}{1.000000,0.894118,0.788235}%
\pgfsetfillcolor{currentfill}%
\pgfsetlinewidth{0.000000pt}%
\definecolor{currentstroke}{rgb}{1.000000,0.894118,0.788235}%
\pgfsetstrokecolor{currentstroke}%
\pgfsetdash{}{0pt}%
\pgfpathmoveto{\pgfqpoint{2.595009in}{3.223964in}}%
\pgfpathlineto{\pgfqpoint{2.568062in}{3.208406in}}%
\pgfpathlineto{\pgfqpoint{2.568062in}{3.239522in}}%
\pgfpathlineto{\pgfqpoint{2.595009in}{3.255080in}}%
\pgfpathlineto{\pgfqpoint{2.595009in}{3.223964in}}%
\pgfpathclose%
\pgfusepath{fill}%
\end{pgfscope}%
\begin{pgfscope}%
\pgfpathrectangle{\pgfqpoint{0.765000in}{0.660000in}}{\pgfqpoint{4.620000in}{4.620000in}}%
\pgfusepath{clip}%
\pgfsetbuttcap%
\pgfsetroundjoin%
\definecolor{currentfill}{rgb}{1.000000,0.894118,0.788235}%
\pgfsetfillcolor{currentfill}%
\pgfsetlinewidth{0.000000pt}%
\definecolor{currentstroke}{rgb}{1.000000,0.894118,0.788235}%
\pgfsetstrokecolor{currentstroke}%
\pgfsetdash{}{0pt}%
\pgfpathmoveto{\pgfqpoint{2.595009in}{3.223964in}}%
\pgfpathlineto{\pgfqpoint{2.621957in}{3.208406in}}%
\pgfpathlineto{\pgfqpoint{2.621957in}{3.239522in}}%
\pgfpathlineto{\pgfqpoint{2.595009in}{3.255080in}}%
\pgfpathlineto{\pgfqpoint{2.595009in}{3.223964in}}%
\pgfpathclose%
\pgfusepath{fill}%
\end{pgfscope}%
\begin{pgfscope}%
\pgfpathrectangle{\pgfqpoint{0.765000in}{0.660000in}}{\pgfqpoint{4.620000in}{4.620000in}}%
\pgfusepath{clip}%
\pgfsetbuttcap%
\pgfsetroundjoin%
\definecolor{currentfill}{rgb}{1.000000,0.894118,0.788235}%
\pgfsetfillcolor{currentfill}%
\pgfsetlinewidth{0.000000pt}%
\definecolor{currentstroke}{rgb}{1.000000,0.894118,0.788235}%
\pgfsetstrokecolor{currentstroke}%
\pgfsetdash{}{0pt}%
\pgfpathmoveto{\pgfqpoint{2.605373in}{3.258290in}}%
\pgfpathlineto{\pgfqpoint{2.578426in}{3.242732in}}%
\pgfpathlineto{\pgfqpoint{2.605373in}{3.227174in}}%
\pgfpathlineto{\pgfqpoint{2.632320in}{3.242732in}}%
\pgfpathlineto{\pgfqpoint{2.605373in}{3.258290in}}%
\pgfpathclose%
\pgfusepath{fill}%
\end{pgfscope}%
\begin{pgfscope}%
\pgfpathrectangle{\pgfqpoint{0.765000in}{0.660000in}}{\pgfqpoint{4.620000in}{4.620000in}}%
\pgfusepath{clip}%
\pgfsetbuttcap%
\pgfsetroundjoin%
\definecolor{currentfill}{rgb}{1.000000,0.894118,0.788235}%
\pgfsetfillcolor{currentfill}%
\pgfsetlinewidth{0.000000pt}%
\definecolor{currentstroke}{rgb}{1.000000,0.894118,0.788235}%
\pgfsetstrokecolor{currentstroke}%
\pgfsetdash{}{0pt}%
\pgfpathmoveto{\pgfqpoint{2.605373in}{3.196058in}}%
\pgfpathlineto{\pgfqpoint{2.632320in}{3.211616in}}%
\pgfpathlineto{\pgfqpoint{2.632320in}{3.242732in}}%
\pgfpathlineto{\pgfqpoint{2.605373in}{3.227174in}}%
\pgfpathlineto{\pgfqpoint{2.605373in}{3.196058in}}%
\pgfpathclose%
\pgfusepath{fill}%
\end{pgfscope}%
\begin{pgfscope}%
\pgfpathrectangle{\pgfqpoint{0.765000in}{0.660000in}}{\pgfqpoint{4.620000in}{4.620000in}}%
\pgfusepath{clip}%
\pgfsetbuttcap%
\pgfsetroundjoin%
\definecolor{currentfill}{rgb}{1.000000,0.894118,0.788235}%
\pgfsetfillcolor{currentfill}%
\pgfsetlinewidth{0.000000pt}%
\definecolor{currentstroke}{rgb}{1.000000,0.894118,0.788235}%
\pgfsetstrokecolor{currentstroke}%
\pgfsetdash{}{0pt}%
\pgfpathmoveto{\pgfqpoint{2.578426in}{3.211616in}}%
\pgfpathlineto{\pgfqpoint{2.605373in}{3.196058in}}%
\pgfpathlineto{\pgfqpoint{2.605373in}{3.227174in}}%
\pgfpathlineto{\pgfqpoint{2.578426in}{3.242732in}}%
\pgfpathlineto{\pgfqpoint{2.578426in}{3.211616in}}%
\pgfpathclose%
\pgfusepath{fill}%
\end{pgfscope}%
\begin{pgfscope}%
\pgfpathrectangle{\pgfqpoint{0.765000in}{0.660000in}}{\pgfqpoint{4.620000in}{4.620000in}}%
\pgfusepath{clip}%
\pgfsetbuttcap%
\pgfsetroundjoin%
\definecolor{currentfill}{rgb}{1.000000,0.894118,0.788235}%
\pgfsetfillcolor{currentfill}%
\pgfsetlinewidth{0.000000pt}%
\definecolor{currentstroke}{rgb}{1.000000,0.894118,0.788235}%
\pgfsetstrokecolor{currentstroke}%
\pgfsetdash{}{0pt}%
\pgfpathmoveto{\pgfqpoint{2.595009in}{3.255080in}}%
\pgfpathlineto{\pgfqpoint{2.568062in}{3.239522in}}%
\pgfpathlineto{\pgfqpoint{2.595009in}{3.223964in}}%
\pgfpathlineto{\pgfqpoint{2.621957in}{3.239522in}}%
\pgfpathlineto{\pgfqpoint{2.595009in}{3.255080in}}%
\pgfpathclose%
\pgfusepath{fill}%
\end{pgfscope}%
\begin{pgfscope}%
\pgfpathrectangle{\pgfqpoint{0.765000in}{0.660000in}}{\pgfqpoint{4.620000in}{4.620000in}}%
\pgfusepath{clip}%
\pgfsetbuttcap%
\pgfsetroundjoin%
\definecolor{currentfill}{rgb}{1.000000,0.894118,0.788235}%
\pgfsetfillcolor{currentfill}%
\pgfsetlinewidth{0.000000pt}%
\definecolor{currentstroke}{rgb}{1.000000,0.894118,0.788235}%
\pgfsetstrokecolor{currentstroke}%
\pgfsetdash{}{0pt}%
\pgfpathmoveto{\pgfqpoint{2.595009in}{3.192848in}}%
\pgfpathlineto{\pgfqpoint{2.621957in}{3.208406in}}%
\pgfpathlineto{\pgfqpoint{2.621957in}{3.239522in}}%
\pgfpathlineto{\pgfqpoint{2.595009in}{3.223964in}}%
\pgfpathlineto{\pgfqpoint{2.595009in}{3.192848in}}%
\pgfpathclose%
\pgfusepath{fill}%
\end{pgfscope}%
\begin{pgfscope}%
\pgfpathrectangle{\pgfqpoint{0.765000in}{0.660000in}}{\pgfqpoint{4.620000in}{4.620000in}}%
\pgfusepath{clip}%
\pgfsetbuttcap%
\pgfsetroundjoin%
\definecolor{currentfill}{rgb}{1.000000,0.894118,0.788235}%
\pgfsetfillcolor{currentfill}%
\pgfsetlinewidth{0.000000pt}%
\definecolor{currentstroke}{rgb}{1.000000,0.894118,0.788235}%
\pgfsetstrokecolor{currentstroke}%
\pgfsetdash{}{0pt}%
\pgfpathmoveto{\pgfqpoint{2.568062in}{3.208406in}}%
\pgfpathlineto{\pgfqpoint{2.595009in}{3.192848in}}%
\pgfpathlineto{\pgfqpoint{2.595009in}{3.223964in}}%
\pgfpathlineto{\pgfqpoint{2.568062in}{3.239522in}}%
\pgfpathlineto{\pgfqpoint{2.568062in}{3.208406in}}%
\pgfpathclose%
\pgfusepath{fill}%
\end{pgfscope}%
\begin{pgfscope}%
\pgfpathrectangle{\pgfqpoint{0.765000in}{0.660000in}}{\pgfqpoint{4.620000in}{4.620000in}}%
\pgfusepath{clip}%
\pgfsetbuttcap%
\pgfsetroundjoin%
\definecolor{currentfill}{rgb}{1.000000,0.894118,0.788235}%
\pgfsetfillcolor{currentfill}%
\pgfsetlinewidth{0.000000pt}%
\definecolor{currentstroke}{rgb}{1.000000,0.894118,0.788235}%
\pgfsetstrokecolor{currentstroke}%
\pgfsetdash{}{0pt}%
\pgfpathmoveto{\pgfqpoint{2.605373in}{3.227174in}}%
\pgfpathlineto{\pgfqpoint{2.578426in}{3.211616in}}%
\pgfpathlineto{\pgfqpoint{2.568062in}{3.208406in}}%
\pgfpathlineto{\pgfqpoint{2.595009in}{3.223964in}}%
\pgfpathlineto{\pgfqpoint{2.605373in}{3.227174in}}%
\pgfpathclose%
\pgfusepath{fill}%
\end{pgfscope}%
\begin{pgfscope}%
\pgfpathrectangle{\pgfqpoint{0.765000in}{0.660000in}}{\pgfqpoint{4.620000in}{4.620000in}}%
\pgfusepath{clip}%
\pgfsetbuttcap%
\pgfsetroundjoin%
\definecolor{currentfill}{rgb}{1.000000,0.894118,0.788235}%
\pgfsetfillcolor{currentfill}%
\pgfsetlinewidth{0.000000pt}%
\definecolor{currentstroke}{rgb}{1.000000,0.894118,0.788235}%
\pgfsetstrokecolor{currentstroke}%
\pgfsetdash{}{0pt}%
\pgfpathmoveto{\pgfqpoint{2.632320in}{3.211616in}}%
\pgfpathlineto{\pgfqpoint{2.605373in}{3.227174in}}%
\pgfpathlineto{\pgfqpoint{2.595009in}{3.223964in}}%
\pgfpathlineto{\pgfqpoint{2.621957in}{3.208406in}}%
\pgfpathlineto{\pgfqpoint{2.632320in}{3.211616in}}%
\pgfpathclose%
\pgfusepath{fill}%
\end{pgfscope}%
\begin{pgfscope}%
\pgfpathrectangle{\pgfqpoint{0.765000in}{0.660000in}}{\pgfqpoint{4.620000in}{4.620000in}}%
\pgfusepath{clip}%
\pgfsetbuttcap%
\pgfsetroundjoin%
\definecolor{currentfill}{rgb}{1.000000,0.894118,0.788235}%
\pgfsetfillcolor{currentfill}%
\pgfsetlinewidth{0.000000pt}%
\definecolor{currentstroke}{rgb}{1.000000,0.894118,0.788235}%
\pgfsetstrokecolor{currentstroke}%
\pgfsetdash{}{0pt}%
\pgfpathmoveto{\pgfqpoint{2.605373in}{3.227174in}}%
\pgfpathlineto{\pgfqpoint{2.605373in}{3.258290in}}%
\pgfpathlineto{\pgfqpoint{2.595009in}{3.255080in}}%
\pgfpathlineto{\pgfqpoint{2.621957in}{3.208406in}}%
\pgfpathlineto{\pgfqpoint{2.605373in}{3.227174in}}%
\pgfpathclose%
\pgfusepath{fill}%
\end{pgfscope}%
\begin{pgfscope}%
\pgfpathrectangle{\pgfqpoint{0.765000in}{0.660000in}}{\pgfqpoint{4.620000in}{4.620000in}}%
\pgfusepath{clip}%
\pgfsetbuttcap%
\pgfsetroundjoin%
\definecolor{currentfill}{rgb}{1.000000,0.894118,0.788235}%
\pgfsetfillcolor{currentfill}%
\pgfsetlinewidth{0.000000pt}%
\definecolor{currentstroke}{rgb}{1.000000,0.894118,0.788235}%
\pgfsetstrokecolor{currentstroke}%
\pgfsetdash{}{0pt}%
\pgfpathmoveto{\pgfqpoint{2.632320in}{3.211616in}}%
\pgfpathlineto{\pgfqpoint{2.632320in}{3.242732in}}%
\pgfpathlineto{\pgfqpoint{2.621957in}{3.239522in}}%
\pgfpathlineto{\pgfqpoint{2.595009in}{3.223964in}}%
\pgfpathlineto{\pgfqpoint{2.632320in}{3.211616in}}%
\pgfpathclose%
\pgfusepath{fill}%
\end{pgfscope}%
\begin{pgfscope}%
\pgfpathrectangle{\pgfqpoint{0.765000in}{0.660000in}}{\pgfqpoint{4.620000in}{4.620000in}}%
\pgfusepath{clip}%
\pgfsetbuttcap%
\pgfsetroundjoin%
\definecolor{currentfill}{rgb}{1.000000,0.894118,0.788235}%
\pgfsetfillcolor{currentfill}%
\pgfsetlinewidth{0.000000pt}%
\definecolor{currentstroke}{rgb}{1.000000,0.894118,0.788235}%
\pgfsetstrokecolor{currentstroke}%
\pgfsetdash{}{0pt}%
\pgfpathmoveto{\pgfqpoint{2.578426in}{3.211616in}}%
\pgfpathlineto{\pgfqpoint{2.605373in}{3.196058in}}%
\pgfpathlineto{\pgfqpoint{2.595009in}{3.192848in}}%
\pgfpathlineto{\pgfqpoint{2.568062in}{3.208406in}}%
\pgfpathlineto{\pgfqpoint{2.578426in}{3.211616in}}%
\pgfpathclose%
\pgfusepath{fill}%
\end{pgfscope}%
\begin{pgfscope}%
\pgfpathrectangle{\pgfqpoint{0.765000in}{0.660000in}}{\pgfqpoint{4.620000in}{4.620000in}}%
\pgfusepath{clip}%
\pgfsetbuttcap%
\pgfsetroundjoin%
\definecolor{currentfill}{rgb}{1.000000,0.894118,0.788235}%
\pgfsetfillcolor{currentfill}%
\pgfsetlinewidth{0.000000pt}%
\definecolor{currentstroke}{rgb}{1.000000,0.894118,0.788235}%
\pgfsetstrokecolor{currentstroke}%
\pgfsetdash{}{0pt}%
\pgfpathmoveto{\pgfqpoint{2.605373in}{3.196058in}}%
\pgfpathlineto{\pgfqpoint{2.632320in}{3.211616in}}%
\pgfpathlineto{\pgfqpoint{2.621957in}{3.208406in}}%
\pgfpathlineto{\pgfqpoint{2.595009in}{3.192848in}}%
\pgfpathlineto{\pgfqpoint{2.605373in}{3.196058in}}%
\pgfpathclose%
\pgfusepath{fill}%
\end{pgfscope}%
\begin{pgfscope}%
\pgfpathrectangle{\pgfqpoint{0.765000in}{0.660000in}}{\pgfqpoint{4.620000in}{4.620000in}}%
\pgfusepath{clip}%
\pgfsetbuttcap%
\pgfsetroundjoin%
\definecolor{currentfill}{rgb}{1.000000,0.894118,0.788235}%
\pgfsetfillcolor{currentfill}%
\pgfsetlinewidth{0.000000pt}%
\definecolor{currentstroke}{rgb}{1.000000,0.894118,0.788235}%
\pgfsetstrokecolor{currentstroke}%
\pgfsetdash{}{0pt}%
\pgfpathmoveto{\pgfqpoint{2.605373in}{3.258290in}}%
\pgfpathlineto{\pgfqpoint{2.578426in}{3.242732in}}%
\pgfpathlineto{\pgfqpoint{2.568062in}{3.239522in}}%
\pgfpathlineto{\pgfqpoint{2.595009in}{3.255080in}}%
\pgfpathlineto{\pgfqpoint{2.605373in}{3.258290in}}%
\pgfpathclose%
\pgfusepath{fill}%
\end{pgfscope}%
\begin{pgfscope}%
\pgfpathrectangle{\pgfqpoint{0.765000in}{0.660000in}}{\pgfqpoint{4.620000in}{4.620000in}}%
\pgfusepath{clip}%
\pgfsetbuttcap%
\pgfsetroundjoin%
\definecolor{currentfill}{rgb}{1.000000,0.894118,0.788235}%
\pgfsetfillcolor{currentfill}%
\pgfsetlinewidth{0.000000pt}%
\definecolor{currentstroke}{rgb}{1.000000,0.894118,0.788235}%
\pgfsetstrokecolor{currentstroke}%
\pgfsetdash{}{0pt}%
\pgfpathmoveto{\pgfqpoint{2.632320in}{3.242732in}}%
\pgfpathlineto{\pgfqpoint{2.605373in}{3.258290in}}%
\pgfpathlineto{\pgfqpoint{2.595009in}{3.255080in}}%
\pgfpathlineto{\pgfqpoint{2.621957in}{3.239522in}}%
\pgfpathlineto{\pgfqpoint{2.632320in}{3.242732in}}%
\pgfpathclose%
\pgfusepath{fill}%
\end{pgfscope}%
\begin{pgfscope}%
\pgfpathrectangle{\pgfqpoint{0.765000in}{0.660000in}}{\pgfqpoint{4.620000in}{4.620000in}}%
\pgfusepath{clip}%
\pgfsetbuttcap%
\pgfsetroundjoin%
\definecolor{currentfill}{rgb}{1.000000,0.894118,0.788235}%
\pgfsetfillcolor{currentfill}%
\pgfsetlinewidth{0.000000pt}%
\definecolor{currentstroke}{rgb}{1.000000,0.894118,0.788235}%
\pgfsetstrokecolor{currentstroke}%
\pgfsetdash{}{0pt}%
\pgfpathmoveto{\pgfqpoint{2.578426in}{3.211616in}}%
\pgfpathlineto{\pgfqpoint{2.578426in}{3.242732in}}%
\pgfpathlineto{\pgfqpoint{2.568062in}{3.239522in}}%
\pgfpathlineto{\pgfqpoint{2.595009in}{3.192848in}}%
\pgfpathlineto{\pgfqpoint{2.578426in}{3.211616in}}%
\pgfpathclose%
\pgfusepath{fill}%
\end{pgfscope}%
\begin{pgfscope}%
\pgfpathrectangle{\pgfqpoint{0.765000in}{0.660000in}}{\pgfqpoint{4.620000in}{4.620000in}}%
\pgfusepath{clip}%
\pgfsetbuttcap%
\pgfsetroundjoin%
\definecolor{currentfill}{rgb}{1.000000,0.894118,0.788235}%
\pgfsetfillcolor{currentfill}%
\pgfsetlinewidth{0.000000pt}%
\definecolor{currentstroke}{rgb}{1.000000,0.894118,0.788235}%
\pgfsetstrokecolor{currentstroke}%
\pgfsetdash{}{0pt}%
\pgfpathmoveto{\pgfqpoint{2.605373in}{3.196058in}}%
\pgfpathlineto{\pgfqpoint{2.605373in}{3.227174in}}%
\pgfpathlineto{\pgfqpoint{2.595009in}{3.223964in}}%
\pgfpathlineto{\pgfqpoint{2.568062in}{3.208406in}}%
\pgfpathlineto{\pgfqpoint{2.605373in}{3.196058in}}%
\pgfpathclose%
\pgfusepath{fill}%
\end{pgfscope}%
\begin{pgfscope}%
\pgfpathrectangle{\pgfqpoint{0.765000in}{0.660000in}}{\pgfqpoint{4.620000in}{4.620000in}}%
\pgfusepath{clip}%
\pgfsetbuttcap%
\pgfsetroundjoin%
\definecolor{currentfill}{rgb}{1.000000,0.894118,0.788235}%
\pgfsetfillcolor{currentfill}%
\pgfsetlinewidth{0.000000pt}%
\definecolor{currentstroke}{rgb}{1.000000,0.894118,0.788235}%
\pgfsetstrokecolor{currentstroke}%
\pgfsetdash{}{0pt}%
\pgfpathmoveto{\pgfqpoint{2.605373in}{3.227174in}}%
\pgfpathlineto{\pgfqpoint{2.632320in}{3.242732in}}%
\pgfpathlineto{\pgfqpoint{2.621957in}{3.239522in}}%
\pgfpathlineto{\pgfqpoint{2.595009in}{3.223964in}}%
\pgfpathlineto{\pgfqpoint{2.605373in}{3.227174in}}%
\pgfpathclose%
\pgfusepath{fill}%
\end{pgfscope}%
\begin{pgfscope}%
\pgfpathrectangle{\pgfqpoint{0.765000in}{0.660000in}}{\pgfqpoint{4.620000in}{4.620000in}}%
\pgfusepath{clip}%
\pgfsetbuttcap%
\pgfsetroundjoin%
\definecolor{currentfill}{rgb}{1.000000,0.894118,0.788235}%
\pgfsetfillcolor{currentfill}%
\pgfsetlinewidth{0.000000pt}%
\definecolor{currentstroke}{rgb}{1.000000,0.894118,0.788235}%
\pgfsetstrokecolor{currentstroke}%
\pgfsetdash{}{0pt}%
\pgfpathmoveto{\pgfqpoint{2.578426in}{3.242732in}}%
\pgfpathlineto{\pgfqpoint{2.605373in}{3.227174in}}%
\pgfpathlineto{\pgfqpoint{2.595009in}{3.223964in}}%
\pgfpathlineto{\pgfqpoint{2.568062in}{3.239522in}}%
\pgfpathlineto{\pgfqpoint{2.578426in}{3.242732in}}%
\pgfpathclose%
\pgfusepath{fill}%
\end{pgfscope}%
\begin{pgfscope}%
\pgfpathrectangle{\pgfqpoint{0.765000in}{0.660000in}}{\pgfqpoint{4.620000in}{4.620000in}}%
\pgfusepath{clip}%
\pgfsetbuttcap%
\pgfsetroundjoin%
\definecolor{currentfill}{rgb}{1.000000,0.894118,0.788235}%
\pgfsetfillcolor{currentfill}%
\pgfsetlinewidth{0.000000pt}%
\definecolor{currentstroke}{rgb}{1.000000,0.894118,0.788235}%
\pgfsetstrokecolor{currentstroke}%
\pgfsetdash{}{0pt}%
\pgfpathmoveto{\pgfqpoint{2.421256in}{3.149423in}}%
\pgfpathlineto{\pgfqpoint{2.394309in}{3.133865in}}%
\pgfpathlineto{\pgfqpoint{2.421256in}{3.118307in}}%
\pgfpathlineto{\pgfqpoint{2.448203in}{3.133865in}}%
\pgfpathlineto{\pgfqpoint{2.421256in}{3.149423in}}%
\pgfpathclose%
\pgfusepath{fill}%
\end{pgfscope}%
\begin{pgfscope}%
\pgfpathrectangle{\pgfqpoint{0.765000in}{0.660000in}}{\pgfqpoint{4.620000in}{4.620000in}}%
\pgfusepath{clip}%
\pgfsetbuttcap%
\pgfsetroundjoin%
\definecolor{currentfill}{rgb}{1.000000,0.894118,0.788235}%
\pgfsetfillcolor{currentfill}%
\pgfsetlinewidth{0.000000pt}%
\definecolor{currentstroke}{rgb}{1.000000,0.894118,0.788235}%
\pgfsetstrokecolor{currentstroke}%
\pgfsetdash{}{0pt}%
\pgfpathmoveto{\pgfqpoint{2.421256in}{3.149423in}}%
\pgfpathlineto{\pgfqpoint{2.394309in}{3.133865in}}%
\pgfpathlineto{\pgfqpoint{2.394309in}{3.164981in}}%
\pgfpathlineto{\pgfqpoint{2.421256in}{3.180539in}}%
\pgfpathlineto{\pgfqpoint{2.421256in}{3.149423in}}%
\pgfpathclose%
\pgfusepath{fill}%
\end{pgfscope}%
\begin{pgfscope}%
\pgfpathrectangle{\pgfqpoint{0.765000in}{0.660000in}}{\pgfqpoint{4.620000in}{4.620000in}}%
\pgfusepath{clip}%
\pgfsetbuttcap%
\pgfsetroundjoin%
\definecolor{currentfill}{rgb}{1.000000,0.894118,0.788235}%
\pgfsetfillcolor{currentfill}%
\pgfsetlinewidth{0.000000pt}%
\definecolor{currentstroke}{rgb}{1.000000,0.894118,0.788235}%
\pgfsetstrokecolor{currentstroke}%
\pgfsetdash{}{0pt}%
\pgfpathmoveto{\pgfqpoint{2.421256in}{3.149423in}}%
\pgfpathlineto{\pgfqpoint{2.448203in}{3.133865in}}%
\pgfpathlineto{\pgfqpoint{2.448203in}{3.164981in}}%
\pgfpathlineto{\pgfqpoint{2.421256in}{3.180539in}}%
\pgfpathlineto{\pgfqpoint{2.421256in}{3.149423in}}%
\pgfpathclose%
\pgfusepath{fill}%
\end{pgfscope}%
\begin{pgfscope}%
\pgfpathrectangle{\pgfqpoint{0.765000in}{0.660000in}}{\pgfqpoint{4.620000in}{4.620000in}}%
\pgfusepath{clip}%
\pgfsetbuttcap%
\pgfsetroundjoin%
\definecolor{currentfill}{rgb}{1.000000,0.894118,0.788235}%
\pgfsetfillcolor{currentfill}%
\pgfsetlinewidth{0.000000pt}%
\definecolor{currentstroke}{rgb}{1.000000,0.894118,0.788235}%
\pgfsetstrokecolor{currentstroke}%
\pgfsetdash{}{0pt}%
\pgfpathmoveto{\pgfqpoint{2.537289in}{3.204720in}}%
\pgfpathlineto{\pgfqpoint{2.510341in}{3.189162in}}%
\pgfpathlineto{\pgfqpoint{2.537289in}{3.173604in}}%
\pgfpathlineto{\pgfqpoint{2.564236in}{3.189162in}}%
\pgfpathlineto{\pgfqpoint{2.537289in}{3.204720in}}%
\pgfpathclose%
\pgfusepath{fill}%
\end{pgfscope}%
\begin{pgfscope}%
\pgfpathrectangle{\pgfqpoint{0.765000in}{0.660000in}}{\pgfqpoint{4.620000in}{4.620000in}}%
\pgfusepath{clip}%
\pgfsetbuttcap%
\pgfsetroundjoin%
\definecolor{currentfill}{rgb}{1.000000,0.894118,0.788235}%
\pgfsetfillcolor{currentfill}%
\pgfsetlinewidth{0.000000pt}%
\definecolor{currentstroke}{rgb}{1.000000,0.894118,0.788235}%
\pgfsetstrokecolor{currentstroke}%
\pgfsetdash{}{0pt}%
\pgfpathmoveto{\pgfqpoint{2.537289in}{3.204720in}}%
\pgfpathlineto{\pgfqpoint{2.510341in}{3.189162in}}%
\pgfpathlineto{\pgfqpoint{2.510341in}{3.220278in}}%
\pgfpathlineto{\pgfqpoint{2.537289in}{3.235836in}}%
\pgfpathlineto{\pgfqpoint{2.537289in}{3.204720in}}%
\pgfpathclose%
\pgfusepath{fill}%
\end{pgfscope}%
\begin{pgfscope}%
\pgfpathrectangle{\pgfqpoint{0.765000in}{0.660000in}}{\pgfqpoint{4.620000in}{4.620000in}}%
\pgfusepath{clip}%
\pgfsetbuttcap%
\pgfsetroundjoin%
\definecolor{currentfill}{rgb}{1.000000,0.894118,0.788235}%
\pgfsetfillcolor{currentfill}%
\pgfsetlinewidth{0.000000pt}%
\definecolor{currentstroke}{rgb}{1.000000,0.894118,0.788235}%
\pgfsetstrokecolor{currentstroke}%
\pgfsetdash{}{0pt}%
\pgfpathmoveto{\pgfqpoint{2.537289in}{3.204720in}}%
\pgfpathlineto{\pgfqpoint{2.564236in}{3.189162in}}%
\pgfpathlineto{\pgfqpoint{2.564236in}{3.220278in}}%
\pgfpathlineto{\pgfqpoint{2.537289in}{3.235836in}}%
\pgfpathlineto{\pgfqpoint{2.537289in}{3.204720in}}%
\pgfpathclose%
\pgfusepath{fill}%
\end{pgfscope}%
\begin{pgfscope}%
\pgfpathrectangle{\pgfqpoint{0.765000in}{0.660000in}}{\pgfqpoint{4.620000in}{4.620000in}}%
\pgfusepath{clip}%
\pgfsetbuttcap%
\pgfsetroundjoin%
\definecolor{currentfill}{rgb}{1.000000,0.894118,0.788235}%
\pgfsetfillcolor{currentfill}%
\pgfsetlinewidth{0.000000pt}%
\definecolor{currentstroke}{rgb}{1.000000,0.894118,0.788235}%
\pgfsetstrokecolor{currentstroke}%
\pgfsetdash{}{0pt}%
\pgfpathmoveto{\pgfqpoint{2.421256in}{3.180539in}}%
\pgfpathlineto{\pgfqpoint{2.394309in}{3.164981in}}%
\pgfpathlineto{\pgfqpoint{2.421256in}{3.149423in}}%
\pgfpathlineto{\pgfqpoint{2.448203in}{3.164981in}}%
\pgfpathlineto{\pgfqpoint{2.421256in}{3.180539in}}%
\pgfpathclose%
\pgfusepath{fill}%
\end{pgfscope}%
\begin{pgfscope}%
\pgfpathrectangle{\pgfqpoint{0.765000in}{0.660000in}}{\pgfqpoint{4.620000in}{4.620000in}}%
\pgfusepath{clip}%
\pgfsetbuttcap%
\pgfsetroundjoin%
\definecolor{currentfill}{rgb}{1.000000,0.894118,0.788235}%
\pgfsetfillcolor{currentfill}%
\pgfsetlinewidth{0.000000pt}%
\definecolor{currentstroke}{rgb}{1.000000,0.894118,0.788235}%
\pgfsetstrokecolor{currentstroke}%
\pgfsetdash{}{0pt}%
\pgfpathmoveto{\pgfqpoint{2.421256in}{3.118307in}}%
\pgfpathlineto{\pgfqpoint{2.448203in}{3.133865in}}%
\pgfpathlineto{\pgfqpoint{2.448203in}{3.164981in}}%
\pgfpathlineto{\pgfqpoint{2.421256in}{3.149423in}}%
\pgfpathlineto{\pgfqpoint{2.421256in}{3.118307in}}%
\pgfpathclose%
\pgfusepath{fill}%
\end{pgfscope}%
\begin{pgfscope}%
\pgfpathrectangle{\pgfqpoint{0.765000in}{0.660000in}}{\pgfqpoint{4.620000in}{4.620000in}}%
\pgfusepath{clip}%
\pgfsetbuttcap%
\pgfsetroundjoin%
\definecolor{currentfill}{rgb}{1.000000,0.894118,0.788235}%
\pgfsetfillcolor{currentfill}%
\pgfsetlinewidth{0.000000pt}%
\definecolor{currentstroke}{rgb}{1.000000,0.894118,0.788235}%
\pgfsetstrokecolor{currentstroke}%
\pgfsetdash{}{0pt}%
\pgfpathmoveto{\pgfqpoint{2.394309in}{3.133865in}}%
\pgfpathlineto{\pgfqpoint{2.421256in}{3.118307in}}%
\pgfpathlineto{\pgfqpoint{2.421256in}{3.149423in}}%
\pgfpathlineto{\pgfqpoint{2.394309in}{3.164981in}}%
\pgfpathlineto{\pgfqpoint{2.394309in}{3.133865in}}%
\pgfpathclose%
\pgfusepath{fill}%
\end{pgfscope}%
\begin{pgfscope}%
\pgfpathrectangle{\pgfqpoint{0.765000in}{0.660000in}}{\pgfqpoint{4.620000in}{4.620000in}}%
\pgfusepath{clip}%
\pgfsetbuttcap%
\pgfsetroundjoin%
\definecolor{currentfill}{rgb}{1.000000,0.894118,0.788235}%
\pgfsetfillcolor{currentfill}%
\pgfsetlinewidth{0.000000pt}%
\definecolor{currentstroke}{rgb}{1.000000,0.894118,0.788235}%
\pgfsetstrokecolor{currentstroke}%
\pgfsetdash{}{0pt}%
\pgfpathmoveto{\pgfqpoint{2.537289in}{3.235836in}}%
\pgfpathlineto{\pgfqpoint{2.510341in}{3.220278in}}%
\pgfpathlineto{\pgfqpoint{2.537289in}{3.204720in}}%
\pgfpathlineto{\pgfqpoint{2.564236in}{3.220278in}}%
\pgfpathlineto{\pgfqpoint{2.537289in}{3.235836in}}%
\pgfpathclose%
\pgfusepath{fill}%
\end{pgfscope}%
\begin{pgfscope}%
\pgfpathrectangle{\pgfqpoint{0.765000in}{0.660000in}}{\pgfqpoint{4.620000in}{4.620000in}}%
\pgfusepath{clip}%
\pgfsetbuttcap%
\pgfsetroundjoin%
\definecolor{currentfill}{rgb}{1.000000,0.894118,0.788235}%
\pgfsetfillcolor{currentfill}%
\pgfsetlinewidth{0.000000pt}%
\definecolor{currentstroke}{rgb}{1.000000,0.894118,0.788235}%
\pgfsetstrokecolor{currentstroke}%
\pgfsetdash{}{0pt}%
\pgfpathmoveto{\pgfqpoint{2.537289in}{3.173604in}}%
\pgfpathlineto{\pgfqpoint{2.564236in}{3.189162in}}%
\pgfpathlineto{\pgfqpoint{2.564236in}{3.220278in}}%
\pgfpathlineto{\pgfqpoint{2.537289in}{3.204720in}}%
\pgfpathlineto{\pgfqpoint{2.537289in}{3.173604in}}%
\pgfpathclose%
\pgfusepath{fill}%
\end{pgfscope}%
\begin{pgfscope}%
\pgfpathrectangle{\pgfqpoint{0.765000in}{0.660000in}}{\pgfqpoint{4.620000in}{4.620000in}}%
\pgfusepath{clip}%
\pgfsetbuttcap%
\pgfsetroundjoin%
\definecolor{currentfill}{rgb}{1.000000,0.894118,0.788235}%
\pgfsetfillcolor{currentfill}%
\pgfsetlinewidth{0.000000pt}%
\definecolor{currentstroke}{rgb}{1.000000,0.894118,0.788235}%
\pgfsetstrokecolor{currentstroke}%
\pgfsetdash{}{0pt}%
\pgfpathmoveto{\pgfqpoint{2.510341in}{3.189162in}}%
\pgfpathlineto{\pgfqpoint{2.537289in}{3.173604in}}%
\pgfpathlineto{\pgfqpoint{2.537289in}{3.204720in}}%
\pgfpathlineto{\pgfqpoint{2.510341in}{3.220278in}}%
\pgfpathlineto{\pgfqpoint{2.510341in}{3.189162in}}%
\pgfpathclose%
\pgfusepath{fill}%
\end{pgfscope}%
\begin{pgfscope}%
\pgfpathrectangle{\pgfqpoint{0.765000in}{0.660000in}}{\pgfqpoint{4.620000in}{4.620000in}}%
\pgfusepath{clip}%
\pgfsetbuttcap%
\pgfsetroundjoin%
\definecolor{currentfill}{rgb}{1.000000,0.894118,0.788235}%
\pgfsetfillcolor{currentfill}%
\pgfsetlinewidth{0.000000pt}%
\definecolor{currentstroke}{rgb}{1.000000,0.894118,0.788235}%
\pgfsetstrokecolor{currentstroke}%
\pgfsetdash{}{0pt}%
\pgfpathmoveto{\pgfqpoint{2.421256in}{3.149423in}}%
\pgfpathlineto{\pgfqpoint{2.394309in}{3.133865in}}%
\pgfpathlineto{\pgfqpoint{2.510341in}{3.189162in}}%
\pgfpathlineto{\pgfqpoint{2.537289in}{3.204720in}}%
\pgfpathlineto{\pgfqpoint{2.421256in}{3.149423in}}%
\pgfpathclose%
\pgfusepath{fill}%
\end{pgfscope}%
\begin{pgfscope}%
\pgfpathrectangle{\pgfqpoint{0.765000in}{0.660000in}}{\pgfqpoint{4.620000in}{4.620000in}}%
\pgfusepath{clip}%
\pgfsetbuttcap%
\pgfsetroundjoin%
\definecolor{currentfill}{rgb}{1.000000,0.894118,0.788235}%
\pgfsetfillcolor{currentfill}%
\pgfsetlinewidth{0.000000pt}%
\definecolor{currentstroke}{rgb}{1.000000,0.894118,0.788235}%
\pgfsetstrokecolor{currentstroke}%
\pgfsetdash{}{0pt}%
\pgfpathmoveto{\pgfqpoint{2.448203in}{3.133865in}}%
\pgfpathlineto{\pgfqpoint{2.421256in}{3.149423in}}%
\pgfpathlineto{\pgfqpoint{2.537289in}{3.204720in}}%
\pgfpathlineto{\pgfqpoint{2.564236in}{3.189162in}}%
\pgfpathlineto{\pgfqpoint{2.448203in}{3.133865in}}%
\pgfpathclose%
\pgfusepath{fill}%
\end{pgfscope}%
\begin{pgfscope}%
\pgfpathrectangle{\pgfqpoint{0.765000in}{0.660000in}}{\pgfqpoint{4.620000in}{4.620000in}}%
\pgfusepath{clip}%
\pgfsetbuttcap%
\pgfsetroundjoin%
\definecolor{currentfill}{rgb}{1.000000,0.894118,0.788235}%
\pgfsetfillcolor{currentfill}%
\pgfsetlinewidth{0.000000pt}%
\definecolor{currentstroke}{rgb}{1.000000,0.894118,0.788235}%
\pgfsetstrokecolor{currentstroke}%
\pgfsetdash{}{0pt}%
\pgfpathmoveto{\pgfqpoint{2.421256in}{3.149423in}}%
\pgfpathlineto{\pgfqpoint{2.421256in}{3.180539in}}%
\pgfpathlineto{\pgfqpoint{2.537289in}{3.235836in}}%
\pgfpathlineto{\pgfqpoint{2.564236in}{3.189162in}}%
\pgfpathlineto{\pgfqpoint{2.421256in}{3.149423in}}%
\pgfpathclose%
\pgfusepath{fill}%
\end{pgfscope}%
\begin{pgfscope}%
\pgfpathrectangle{\pgfqpoint{0.765000in}{0.660000in}}{\pgfqpoint{4.620000in}{4.620000in}}%
\pgfusepath{clip}%
\pgfsetbuttcap%
\pgfsetroundjoin%
\definecolor{currentfill}{rgb}{1.000000,0.894118,0.788235}%
\pgfsetfillcolor{currentfill}%
\pgfsetlinewidth{0.000000pt}%
\definecolor{currentstroke}{rgb}{1.000000,0.894118,0.788235}%
\pgfsetstrokecolor{currentstroke}%
\pgfsetdash{}{0pt}%
\pgfpathmoveto{\pgfqpoint{2.448203in}{3.133865in}}%
\pgfpathlineto{\pgfqpoint{2.448203in}{3.164981in}}%
\pgfpathlineto{\pgfqpoint{2.564236in}{3.220278in}}%
\pgfpathlineto{\pgfqpoint{2.537289in}{3.204720in}}%
\pgfpathlineto{\pgfqpoint{2.448203in}{3.133865in}}%
\pgfpathclose%
\pgfusepath{fill}%
\end{pgfscope}%
\begin{pgfscope}%
\pgfpathrectangle{\pgfqpoint{0.765000in}{0.660000in}}{\pgfqpoint{4.620000in}{4.620000in}}%
\pgfusepath{clip}%
\pgfsetbuttcap%
\pgfsetroundjoin%
\definecolor{currentfill}{rgb}{1.000000,0.894118,0.788235}%
\pgfsetfillcolor{currentfill}%
\pgfsetlinewidth{0.000000pt}%
\definecolor{currentstroke}{rgb}{1.000000,0.894118,0.788235}%
\pgfsetstrokecolor{currentstroke}%
\pgfsetdash{}{0pt}%
\pgfpathmoveto{\pgfqpoint{2.394309in}{3.133865in}}%
\pgfpathlineto{\pgfqpoint{2.421256in}{3.118307in}}%
\pgfpathlineto{\pgfqpoint{2.537289in}{3.173604in}}%
\pgfpathlineto{\pgfqpoint{2.510341in}{3.189162in}}%
\pgfpathlineto{\pgfqpoint{2.394309in}{3.133865in}}%
\pgfpathclose%
\pgfusepath{fill}%
\end{pgfscope}%
\begin{pgfscope}%
\pgfpathrectangle{\pgfqpoint{0.765000in}{0.660000in}}{\pgfqpoint{4.620000in}{4.620000in}}%
\pgfusepath{clip}%
\pgfsetbuttcap%
\pgfsetroundjoin%
\definecolor{currentfill}{rgb}{1.000000,0.894118,0.788235}%
\pgfsetfillcolor{currentfill}%
\pgfsetlinewidth{0.000000pt}%
\definecolor{currentstroke}{rgb}{1.000000,0.894118,0.788235}%
\pgfsetstrokecolor{currentstroke}%
\pgfsetdash{}{0pt}%
\pgfpathmoveto{\pgfqpoint{2.421256in}{3.118307in}}%
\pgfpathlineto{\pgfqpoint{2.448203in}{3.133865in}}%
\pgfpathlineto{\pgfqpoint{2.564236in}{3.189162in}}%
\pgfpathlineto{\pgfqpoint{2.537289in}{3.173604in}}%
\pgfpathlineto{\pgfqpoint{2.421256in}{3.118307in}}%
\pgfpathclose%
\pgfusepath{fill}%
\end{pgfscope}%
\begin{pgfscope}%
\pgfpathrectangle{\pgfqpoint{0.765000in}{0.660000in}}{\pgfqpoint{4.620000in}{4.620000in}}%
\pgfusepath{clip}%
\pgfsetbuttcap%
\pgfsetroundjoin%
\definecolor{currentfill}{rgb}{1.000000,0.894118,0.788235}%
\pgfsetfillcolor{currentfill}%
\pgfsetlinewidth{0.000000pt}%
\definecolor{currentstroke}{rgb}{1.000000,0.894118,0.788235}%
\pgfsetstrokecolor{currentstroke}%
\pgfsetdash{}{0pt}%
\pgfpathmoveto{\pgfqpoint{2.421256in}{3.180539in}}%
\pgfpathlineto{\pgfqpoint{2.394309in}{3.164981in}}%
\pgfpathlineto{\pgfqpoint{2.510341in}{3.220278in}}%
\pgfpathlineto{\pgfqpoint{2.537289in}{3.235836in}}%
\pgfpathlineto{\pgfqpoint{2.421256in}{3.180539in}}%
\pgfpathclose%
\pgfusepath{fill}%
\end{pgfscope}%
\begin{pgfscope}%
\pgfpathrectangle{\pgfqpoint{0.765000in}{0.660000in}}{\pgfqpoint{4.620000in}{4.620000in}}%
\pgfusepath{clip}%
\pgfsetbuttcap%
\pgfsetroundjoin%
\definecolor{currentfill}{rgb}{1.000000,0.894118,0.788235}%
\pgfsetfillcolor{currentfill}%
\pgfsetlinewidth{0.000000pt}%
\definecolor{currentstroke}{rgb}{1.000000,0.894118,0.788235}%
\pgfsetstrokecolor{currentstroke}%
\pgfsetdash{}{0pt}%
\pgfpathmoveto{\pgfqpoint{2.448203in}{3.164981in}}%
\pgfpathlineto{\pgfqpoint{2.421256in}{3.180539in}}%
\pgfpathlineto{\pgfqpoint{2.537289in}{3.235836in}}%
\pgfpathlineto{\pgfqpoint{2.564236in}{3.220278in}}%
\pgfpathlineto{\pgfqpoint{2.448203in}{3.164981in}}%
\pgfpathclose%
\pgfusepath{fill}%
\end{pgfscope}%
\begin{pgfscope}%
\pgfpathrectangle{\pgfqpoint{0.765000in}{0.660000in}}{\pgfqpoint{4.620000in}{4.620000in}}%
\pgfusepath{clip}%
\pgfsetbuttcap%
\pgfsetroundjoin%
\definecolor{currentfill}{rgb}{1.000000,0.894118,0.788235}%
\pgfsetfillcolor{currentfill}%
\pgfsetlinewidth{0.000000pt}%
\definecolor{currentstroke}{rgb}{1.000000,0.894118,0.788235}%
\pgfsetstrokecolor{currentstroke}%
\pgfsetdash{}{0pt}%
\pgfpathmoveto{\pgfqpoint{2.394309in}{3.133865in}}%
\pgfpathlineto{\pgfqpoint{2.394309in}{3.164981in}}%
\pgfpathlineto{\pgfqpoint{2.510341in}{3.220278in}}%
\pgfpathlineto{\pgfqpoint{2.537289in}{3.173604in}}%
\pgfpathlineto{\pgfqpoint{2.394309in}{3.133865in}}%
\pgfpathclose%
\pgfusepath{fill}%
\end{pgfscope}%
\begin{pgfscope}%
\pgfpathrectangle{\pgfqpoint{0.765000in}{0.660000in}}{\pgfqpoint{4.620000in}{4.620000in}}%
\pgfusepath{clip}%
\pgfsetbuttcap%
\pgfsetroundjoin%
\definecolor{currentfill}{rgb}{1.000000,0.894118,0.788235}%
\pgfsetfillcolor{currentfill}%
\pgfsetlinewidth{0.000000pt}%
\definecolor{currentstroke}{rgb}{1.000000,0.894118,0.788235}%
\pgfsetstrokecolor{currentstroke}%
\pgfsetdash{}{0pt}%
\pgfpathmoveto{\pgfqpoint{2.421256in}{3.118307in}}%
\pgfpathlineto{\pgfqpoint{2.421256in}{3.149423in}}%
\pgfpathlineto{\pgfqpoint{2.537289in}{3.204720in}}%
\pgfpathlineto{\pgfqpoint{2.510341in}{3.189162in}}%
\pgfpathlineto{\pgfqpoint{2.421256in}{3.118307in}}%
\pgfpathclose%
\pgfusepath{fill}%
\end{pgfscope}%
\begin{pgfscope}%
\pgfpathrectangle{\pgfqpoint{0.765000in}{0.660000in}}{\pgfqpoint{4.620000in}{4.620000in}}%
\pgfusepath{clip}%
\pgfsetbuttcap%
\pgfsetroundjoin%
\definecolor{currentfill}{rgb}{1.000000,0.894118,0.788235}%
\pgfsetfillcolor{currentfill}%
\pgfsetlinewidth{0.000000pt}%
\definecolor{currentstroke}{rgb}{1.000000,0.894118,0.788235}%
\pgfsetstrokecolor{currentstroke}%
\pgfsetdash{}{0pt}%
\pgfpathmoveto{\pgfqpoint{2.421256in}{3.149423in}}%
\pgfpathlineto{\pgfqpoint{2.448203in}{3.164981in}}%
\pgfpathlineto{\pgfqpoint{2.564236in}{3.220278in}}%
\pgfpathlineto{\pgfqpoint{2.537289in}{3.204720in}}%
\pgfpathlineto{\pgfqpoint{2.421256in}{3.149423in}}%
\pgfpathclose%
\pgfusepath{fill}%
\end{pgfscope}%
\begin{pgfscope}%
\pgfpathrectangle{\pgfqpoint{0.765000in}{0.660000in}}{\pgfqpoint{4.620000in}{4.620000in}}%
\pgfusepath{clip}%
\pgfsetbuttcap%
\pgfsetroundjoin%
\definecolor{currentfill}{rgb}{1.000000,0.894118,0.788235}%
\pgfsetfillcolor{currentfill}%
\pgfsetlinewidth{0.000000pt}%
\definecolor{currentstroke}{rgb}{1.000000,0.894118,0.788235}%
\pgfsetstrokecolor{currentstroke}%
\pgfsetdash{}{0pt}%
\pgfpathmoveto{\pgfqpoint{2.394309in}{3.164981in}}%
\pgfpathlineto{\pgfqpoint{2.421256in}{3.149423in}}%
\pgfpathlineto{\pgfqpoint{2.537289in}{3.204720in}}%
\pgfpathlineto{\pgfqpoint{2.510341in}{3.220278in}}%
\pgfpathlineto{\pgfqpoint{2.394309in}{3.164981in}}%
\pgfpathclose%
\pgfusepath{fill}%
\end{pgfscope}%
\begin{pgfscope}%
\pgfpathrectangle{\pgfqpoint{0.765000in}{0.660000in}}{\pgfqpoint{4.620000in}{4.620000in}}%
\pgfusepath{clip}%
\pgfsetbuttcap%
\pgfsetroundjoin%
\definecolor{currentfill}{rgb}{1.000000,0.894118,0.788235}%
\pgfsetfillcolor{currentfill}%
\pgfsetlinewidth{0.000000pt}%
\definecolor{currentstroke}{rgb}{1.000000,0.894118,0.788235}%
\pgfsetstrokecolor{currentstroke}%
\pgfsetdash{}{0pt}%
\pgfpathmoveto{\pgfqpoint{2.421256in}{3.149423in}}%
\pgfpathlineto{\pgfqpoint{2.394309in}{3.133865in}}%
\pgfpathlineto{\pgfqpoint{2.421256in}{3.118307in}}%
\pgfpathlineto{\pgfqpoint{2.448203in}{3.133865in}}%
\pgfpathlineto{\pgfqpoint{2.421256in}{3.149423in}}%
\pgfpathclose%
\pgfusepath{fill}%
\end{pgfscope}%
\begin{pgfscope}%
\pgfpathrectangle{\pgfqpoint{0.765000in}{0.660000in}}{\pgfqpoint{4.620000in}{4.620000in}}%
\pgfusepath{clip}%
\pgfsetbuttcap%
\pgfsetroundjoin%
\definecolor{currentfill}{rgb}{1.000000,0.894118,0.788235}%
\pgfsetfillcolor{currentfill}%
\pgfsetlinewidth{0.000000pt}%
\definecolor{currentstroke}{rgb}{1.000000,0.894118,0.788235}%
\pgfsetstrokecolor{currentstroke}%
\pgfsetdash{}{0pt}%
\pgfpathmoveto{\pgfqpoint{2.421256in}{3.149423in}}%
\pgfpathlineto{\pgfqpoint{2.394309in}{3.133865in}}%
\pgfpathlineto{\pgfqpoint{2.394309in}{3.164981in}}%
\pgfpathlineto{\pgfqpoint{2.421256in}{3.180539in}}%
\pgfpathlineto{\pgfqpoint{2.421256in}{3.149423in}}%
\pgfpathclose%
\pgfusepath{fill}%
\end{pgfscope}%
\begin{pgfscope}%
\pgfpathrectangle{\pgfqpoint{0.765000in}{0.660000in}}{\pgfqpoint{4.620000in}{4.620000in}}%
\pgfusepath{clip}%
\pgfsetbuttcap%
\pgfsetroundjoin%
\definecolor{currentfill}{rgb}{1.000000,0.894118,0.788235}%
\pgfsetfillcolor{currentfill}%
\pgfsetlinewidth{0.000000pt}%
\definecolor{currentstroke}{rgb}{1.000000,0.894118,0.788235}%
\pgfsetstrokecolor{currentstroke}%
\pgfsetdash{}{0pt}%
\pgfpathmoveto{\pgfqpoint{2.421256in}{3.149423in}}%
\pgfpathlineto{\pgfqpoint{2.448203in}{3.133865in}}%
\pgfpathlineto{\pgfqpoint{2.448203in}{3.164981in}}%
\pgfpathlineto{\pgfqpoint{2.421256in}{3.180539in}}%
\pgfpathlineto{\pgfqpoint{2.421256in}{3.149423in}}%
\pgfpathclose%
\pgfusepath{fill}%
\end{pgfscope}%
\begin{pgfscope}%
\pgfpathrectangle{\pgfqpoint{0.765000in}{0.660000in}}{\pgfqpoint{4.620000in}{4.620000in}}%
\pgfusepath{clip}%
\pgfsetbuttcap%
\pgfsetroundjoin%
\definecolor{currentfill}{rgb}{1.000000,0.894118,0.788235}%
\pgfsetfillcolor{currentfill}%
\pgfsetlinewidth{0.000000pt}%
\definecolor{currentstroke}{rgb}{1.000000,0.894118,0.788235}%
\pgfsetstrokecolor{currentstroke}%
\pgfsetdash{}{0pt}%
\pgfpathmoveto{\pgfqpoint{2.263574in}{3.062189in}}%
\pgfpathlineto{\pgfqpoint{2.236626in}{3.046631in}}%
\pgfpathlineto{\pgfqpoint{2.263574in}{3.031073in}}%
\pgfpathlineto{\pgfqpoint{2.290521in}{3.046631in}}%
\pgfpathlineto{\pgfqpoint{2.263574in}{3.062189in}}%
\pgfpathclose%
\pgfusepath{fill}%
\end{pgfscope}%
\begin{pgfscope}%
\pgfpathrectangle{\pgfqpoint{0.765000in}{0.660000in}}{\pgfqpoint{4.620000in}{4.620000in}}%
\pgfusepath{clip}%
\pgfsetbuttcap%
\pgfsetroundjoin%
\definecolor{currentfill}{rgb}{1.000000,0.894118,0.788235}%
\pgfsetfillcolor{currentfill}%
\pgfsetlinewidth{0.000000pt}%
\definecolor{currentstroke}{rgb}{1.000000,0.894118,0.788235}%
\pgfsetstrokecolor{currentstroke}%
\pgfsetdash{}{0pt}%
\pgfpathmoveto{\pgfqpoint{2.263574in}{3.062189in}}%
\pgfpathlineto{\pgfqpoint{2.236626in}{3.046631in}}%
\pgfpathlineto{\pgfqpoint{2.236626in}{3.077747in}}%
\pgfpathlineto{\pgfqpoint{2.263574in}{3.093305in}}%
\pgfpathlineto{\pgfqpoint{2.263574in}{3.062189in}}%
\pgfpathclose%
\pgfusepath{fill}%
\end{pgfscope}%
\begin{pgfscope}%
\pgfpathrectangle{\pgfqpoint{0.765000in}{0.660000in}}{\pgfqpoint{4.620000in}{4.620000in}}%
\pgfusepath{clip}%
\pgfsetbuttcap%
\pgfsetroundjoin%
\definecolor{currentfill}{rgb}{1.000000,0.894118,0.788235}%
\pgfsetfillcolor{currentfill}%
\pgfsetlinewidth{0.000000pt}%
\definecolor{currentstroke}{rgb}{1.000000,0.894118,0.788235}%
\pgfsetstrokecolor{currentstroke}%
\pgfsetdash{}{0pt}%
\pgfpathmoveto{\pgfqpoint{2.263574in}{3.062189in}}%
\pgfpathlineto{\pgfqpoint{2.290521in}{3.046631in}}%
\pgfpathlineto{\pgfqpoint{2.290521in}{3.077747in}}%
\pgfpathlineto{\pgfqpoint{2.263574in}{3.093305in}}%
\pgfpathlineto{\pgfqpoint{2.263574in}{3.062189in}}%
\pgfpathclose%
\pgfusepath{fill}%
\end{pgfscope}%
\begin{pgfscope}%
\pgfpathrectangle{\pgfqpoint{0.765000in}{0.660000in}}{\pgfqpoint{4.620000in}{4.620000in}}%
\pgfusepath{clip}%
\pgfsetbuttcap%
\pgfsetroundjoin%
\definecolor{currentfill}{rgb}{1.000000,0.894118,0.788235}%
\pgfsetfillcolor{currentfill}%
\pgfsetlinewidth{0.000000pt}%
\definecolor{currentstroke}{rgb}{1.000000,0.894118,0.788235}%
\pgfsetstrokecolor{currentstroke}%
\pgfsetdash{}{0pt}%
\pgfpathmoveto{\pgfqpoint{2.421256in}{3.180539in}}%
\pgfpathlineto{\pgfqpoint{2.394309in}{3.164981in}}%
\pgfpathlineto{\pgfqpoint{2.421256in}{3.149423in}}%
\pgfpathlineto{\pgfqpoint{2.448203in}{3.164981in}}%
\pgfpathlineto{\pgfqpoint{2.421256in}{3.180539in}}%
\pgfpathclose%
\pgfusepath{fill}%
\end{pgfscope}%
\begin{pgfscope}%
\pgfpathrectangle{\pgfqpoint{0.765000in}{0.660000in}}{\pgfqpoint{4.620000in}{4.620000in}}%
\pgfusepath{clip}%
\pgfsetbuttcap%
\pgfsetroundjoin%
\definecolor{currentfill}{rgb}{1.000000,0.894118,0.788235}%
\pgfsetfillcolor{currentfill}%
\pgfsetlinewidth{0.000000pt}%
\definecolor{currentstroke}{rgb}{1.000000,0.894118,0.788235}%
\pgfsetstrokecolor{currentstroke}%
\pgfsetdash{}{0pt}%
\pgfpathmoveto{\pgfqpoint{2.421256in}{3.118307in}}%
\pgfpathlineto{\pgfqpoint{2.448203in}{3.133865in}}%
\pgfpathlineto{\pgfqpoint{2.448203in}{3.164981in}}%
\pgfpathlineto{\pgfqpoint{2.421256in}{3.149423in}}%
\pgfpathlineto{\pgfqpoint{2.421256in}{3.118307in}}%
\pgfpathclose%
\pgfusepath{fill}%
\end{pgfscope}%
\begin{pgfscope}%
\pgfpathrectangle{\pgfqpoint{0.765000in}{0.660000in}}{\pgfqpoint{4.620000in}{4.620000in}}%
\pgfusepath{clip}%
\pgfsetbuttcap%
\pgfsetroundjoin%
\definecolor{currentfill}{rgb}{1.000000,0.894118,0.788235}%
\pgfsetfillcolor{currentfill}%
\pgfsetlinewidth{0.000000pt}%
\definecolor{currentstroke}{rgb}{1.000000,0.894118,0.788235}%
\pgfsetstrokecolor{currentstroke}%
\pgfsetdash{}{0pt}%
\pgfpathmoveto{\pgfqpoint{2.394309in}{3.133865in}}%
\pgfpathlineto{\pgfqpoint{2.421256in}{3.118307in}}%
\pgfpathlineto{\pgfqpoint{2.421256in}{3.149423in}}%
\pgfpathlineto{\pgfqpoint{2.394309in}{3.164981in}}%
\pgfpathlineto{\pgfqpoint{2.394309in}{3.133865in}}%
\pgfpathclose%
\pgfusepath{fill}%
\end{pgfscope}%
\begin{pgfscope}%
\pgfpathrectangle{\pgfqpoint{0.765000in}{0.660000in}}{\pgfqpoint{4.620000in}{4.620000in}}%
\pgfusepath{clip}%
\pgfsetbuttcap%
\pgfsetroundjoin%
\definecolor{currentfill}{rgb}{1.000000,0.894118,0.788235}%
\pgfsetfillcolor{currentfill}%
\pgfsetlinewidth{0.000000pt}%
\definecolor{currentstroke}{rgb}{1.000000,0.894118,0.788235}%
\pgfsetstrokecolor{currentstroke}%
\pgfsetdash{}{0pt}%
\pgfpathmoveto{\pgfqpoint{2.263574in}{3.093305in}}%
\pgfpathlineto{\pgfqpoint{2.236626in}{3.077747in}}%
\pgfpathlineto{\pgfqpoint{2.263574in}{3.062189in}}%
\pgfpathlineto{\pgfqpoint{2.290521in}{3.077747in}}%
\pgfpathlineto{\pgfqpoint{2.263574in}{3.093305in}}%
\pgfpathclose%
\pgfusepath{fill}%
\end{pgfscope}%
\begin{pgfscope}%
\pgfpathrectangle{\pgfqpoint{0.765000in}{0.660000in}}{\pgfqpoint{4.620000in}{4.620000in}}%
\pgfusepath{clip}%
\pgfsetbuttcap%
\pgfsetroundjoin%
\definecolor{currentfill}{rgb}{1.000000,0.894118,0.788235}%
\pgfsetfillcolor{currentfill}%
\pgfsetlinewidth{0.000000pt}%
\definecolor{currentstroke}{rgb}{1.000000,0.894118,0.788235}%
\pgfsetstrokecolor{currentstroke}%
\pgfsetdash{}{0pt}%
\pgfpathmoveto{\pgfqpoint{2.263574in}{3.031073in}}%
\pgfpathlineto{\pgfqpoint{2.290521in}{3.046631in}}%
\pgfpathlineto{\pgfqpoint{2.290521in}{3.077747in}}%
\pgfpathlineto{\pgfqpoint{2.263574in}{3.062189in}}%
\pgfpathlineto{\pgfqpoint{2.263574in}{3.031073in}}%
\pgfpathclose%
\pgfusepath{fill}%
\end{pgfscope}%
\begin{pgfscope}%
\pgfpathrectangle{\pgfqpoint{0.765000in}{0.660000in}}{\pgfqpoint{4.620000in}{4.620000in}}%
\pgfusepath{clip}%
\pgfsetbuttcap%
\pgfsetroundjoin%
\definecolor{currentfill}{rgb}{1.000000,0.894118,0.788235}%
\pgfsetfillcolor{currentfill}%
\pgfsetlinewidth{0.000000pt}%
\definecolor{currentstroke}{rgb}{1.000000,0.894118,0.788235}%
\pgfsetstrokecolor{currentstroke}%
\pgfsetdash{}{0pt}%
\pgfpathmoveto{\pgfqpoint{2.236626in}{3.046631in}}%
\pgfpathlineto{\pgfqpoint{2.263574in}{3.031073in}}%
\pgfpathlineto{\pgfqpoint{2.263574in}{3.062189in}}%
\pgfpathlineto{\pgfqpoint{2.236626in}{3.077747in}}%
\pgfpathlineto{\pgfqpoint{2.236626in}{3.046631in}}%
\pgfpathclose%
\pgfusepath{fill}%
\end{pgfscope}%
\begin{pgfscope}%
\pgfpathrectangle{\pgfqpoint{0.765000in}{0.660000in}}{\pgfqpoint{4.620000in}{4.620000in}}%
\pgfusepath{clip}%
\pgfsetbuttcap%
\pgfsetroundjoin%
\definecolor{currentfill}{rgb}{1.000000,0.894118,0.788235}%
\pgfsetfillcolor{currentfill}%
\pgfsetlinewidth{0.000000pt}%
\definecolor{currentstroke}{rgb}{1.000000,0.894118,0.788235}%
\pgfsetstrokecolor{currentstroke}%
\pgfsetdash{}{0pt}%
\pgfpathmoveto{\pgfqpoint{2.421256in}{3.149423in}}%
\pgfpathlineto{\pgfqpoint{2.394309in}{3.133865in}}%
\pgfpathlineto{\pgfqpoint{2.236626in}{3.046631in}}%
\pgfpathlineto{\pgfqpoint{2.263574in}{3.062189in}}%
\pgfpathlineto{\pgfqpoint{2.421256in}{3.149423in}}%
\pgfpathclose%
\pgfusepath{fill}%
\end{pgfscope}%
\begin{pgfscope}%
\pgfpathrectangle{\pgfqpoint{0.765000in}{0.660000in}}{\pgfqpoint{4.620000in}{4.620000in}}%
\pgfusepath{clip}%
\pgfsetbuttcap%
\pgfsetroundjoin%
\definecolor{currentfill}{rgb}{1.000000,0.894118,0.788235}%
\pgfsetfillcolor{currentfill}%
\pgfsetlinewidth{0.000000pt}%
\definecolor{currentstroke}{rgb}{1.000000,0.894118,0.788235}%
\pgfsetstrokecolor{currentstroke}%
\pgfsetdash{}{0pt}%
\pgfpathmoveto{\pgfqpoint{2.448203in}{3.133865in}}%
\pgfpathlineto{\pgfqpoint{2.421256in}{3.149423in}}%
\pgfpathlineto{\pgfqpoint{2.263574in}{3.062189in}}%
\pgfpathlineto{\pgfqpoint{2.290521in}{3.046631in}}%
\pgfpathlineto{\pgfqpoint{2.448203in}{3.133865in}}%
\pgfpathclose%
\pgfusepath{fill}%
\end{pgfscope}%
\begin{pgfscope}%
\pgfpathrectangle{\pgfqpoint{0.765000in}{0.660000in}}{\pgfqpoint{4.620000in}{4.620000in}}%
\pgfusepath{clip}%
\pgfsetbuttcap%
\pgfsetroundjoin%
\definecolor{currentfill}{rgb}{1.000000,0.894118,0.788235}%
\pgfsetfillcolor{currentfill}%
\pgfsetlinewidth{0.000000pt}%
\definecolor{currentstroke}{rgb}{1.000000,0.894118,0.788235}%
\pgfsetstrokecolor{currentstroke}%
\pgfsetdash{}{0pt}%
\pgfpathmoveto{\pgfqpoint{2.421256in}{3.149423in}}%
\pgfpathlineto{\pgfqpoint{2.421256in}{3.180539in}}%
\pgfpathlineto{\pgfqpoint{2.263574in}{3.093305in}}%
\pgfpathlineto{\pgfqpoint{2.290521in}{3.046631in}}%
\pgfpathlineto{\pgfqpoint{2.421256in}{3.149423in}}%
\pgfpathclose%
\pgfusepath{fill}%
\end{pgfscope}%
\begin{pgfscope}%
\pgfpathrectangle{\pgfqpoint{0.765000in}{0.660000in}}{\pgfqpoint{4.620000in}{4.620000in}}%
\pgfusepath{clip}%
\pgfsetbuttcap%
\pgfsetroundjoin%
\definecolor{currentfill}{rgb}{1.000000,0.894118,0.788235}%
\pgfsetfillcolor{currentfill}%
\pgfsetlinewidth{0.000000pt}%
\definecolor{currentstroke}{rgb}{1.000000,0.894118,0.788235}%
\pgfsetstrokecolor{currentstroke}%
\pgfsetdash{}{0pt}%
\pgfpathmoveto{\pgfqpoint{2.448203in}{3.133865in}}%
\pgfpathlineto{\pgfqpoint{2.448203in}{3.164981in}}%
\pgfpathlineto{\pgfqpoint{2.290521in}{3.077747in}}%
\pgfpathlineto{\pgfqpoint{2.263574in}{3.062189in}}%
\pgfpathlineto{\pgfqpoint{2.448203in}{3.133865in}}%
\pgfpathclose%
\pgfusepath{fill}%
\end{pgfscope}%
\begin{pgfscope}%
\pgfpathrectangle{\pgfqpoint{0.765000in}{0.660000in}}{\pgfqpoint{4.620000in}{4.620000in}}%
\pgfusepath{clip}%
\pgfsetbuttcap%
\pgfsetroundjoin%
\definecolor{currentfill}{rgb}{1.000000,0.894118,0.788235}%
\pgfsetfillcolor{currentfill}%
\pgfsetlinewidth{0.000000pt}%
\definecolor{currentstroke}{rgb}{1.000000,0.894118,0.788235}%
\pgfsetstrokecolor{currentstroke}%
\pgfsetdash{}{0pt}%
\pgfpathmoveto{\pgfqpoint{2.394309in}{3.133865in}}%
\pgfpathlineto{\pgfqpoint{2.421256in}{3.118307in}}%
\pgfpathlineto{\pgfqpoint{2.263574in}{3.031073in}}%
\pgfpathlineto{\pgfqpoint{2.236626in}{3.046631in}}%
\pgfpathlineto{\pgfqpoint{2.394309in}{3.133865in}}%
\pgfpathclose%
\pgfusepath{fill}%
\end{pgfscope}%
\begin{pgfscope}%
\pgfpathrectangle{\pgfqpoint{0.765000in}{0.660000in}}{\pgfqpoint{4.620000in}{4.620000in}}%
\pgfusepath{clip}%
\pgfsetbuttcap%
\pgfsetroundjoin%
\definecolor{currentfill}{rgb}{1.000000,0.894118,0.788235}%
\pgfsetfillcolor{currentfill}%
\pgfsetlinewidth{0.000000pt}%
\definecolor{currentstroke}{rgb}{1.000000,0.894118,0.788235}%
\pgfsetstrokecolor{currentstroke}%
\pgfsetdash{}{0pt}%
\pgfpathmoveto{\pgfqpoint{2.421256in}{3.118307in}}%
\pgfpathlineto{\pgfqpoint{2.448203in}{3.133865in}}%
\pgfpathlineto{\pgfqpoint{2.290521in}{3.046631in}}%
\pgfpathlineto{\pgfqpoint{2.263574in}{3.031073in}}%
\pgfpathlineto{\pgfqpoint{2.421256in}{3.118307in}}%
\pgfpathclose%
\pgfusepath{fill}%
\end{pgfscope}%
\begin{pgfscope}%
\pgfpathrectangle{\pgfqpoint{0.765000in}{0.660000in}}{\pgfqpoint{4.620000in}{4.620000in}}%
\pgfusepath{clip}%
\pgfsetbuttcap%
\pgfsetroundjoin%
\definecolor{currentfill}{rgb}{1.000000,0.894118,0.788235}%
\pgfsetfillcolor{currentfill}%
\pgfsetlinewidth{0.000000pt}%
\definecolor{currentstroke}{rgb}{1.000000,0.894118,0.788235}%
\pgfsetstrokecolor{currentstroke}%
\pgfsetdash{}{0pt}%
\pgfpathmoveto{\pgfqpoint{2.421256in}{3.180539in}}%
\pgfpathlineto{\pgfqpoint{2.394309in}{3.164981in}}%
\pgfpathlineto{\pgfqpoint{2.236626in}{3.077747in}}%
\pgfpathlineto{\pgfqpoint{2.263574in}{3.093305in}}%
\pgfpathlineto{\pgfqpoint{2.421256in}{3.180539in}}%
\pgfpathclose%
\pgfusepath{fill}%
\end{pgfscope}%
\begin{pgfscope}%
\pgfpathrectangle{\pgfqpoint{0.765000in}{0.660000in}}{\pgfqpoint{4.620000in}{4.620000in}}%
\pgfusepath{clip}%
\pgfsetbuttcap%
\pgfsetroundjoin%
\definecolor{currentfill}{rgb}{1.000000,0.894118,0.788235}%
\pgfsetfillcolor{currentfill}%
\pgfsetlinewidth{0.000000pt}%
\definecolor{currentstroke}{rgb}{1.000000,0.894118,0.788235}%
\pgfsetstrokecolor{currentstroke}%
\pgfsetdash{}{0pt}%
\pgfpathmoveto{\pgfqpoint{2.448203in}{3.164981in}}%
\pgfpathlineto{\pgfqpoint{2.421256in}{3.180539in}}%
\pgfpathlineto{\pgfqpoint{2.263574in}{3.093305in}}%
\pgfpathlineto{\pgfqpoint{2.290521in}{3.077747in}}%
\pgfpathlineto{\pgfqpoint{2.448203in}{3.164981in}}%
\pgfpathclose%
\pgfusepath{fill}%
\end{pgfscope}%
\begin{pgfscope}%
\pgfpathrectangle{\pgfqpoint{0.765000in}{0.660000in}}{\pgfqpoint{4.620000in}{4.620000in}}%
\pgfusepath{clip}%
\pgfsetbuttcap%
\pgfsetroundjoin%
\definecolor{currentfill}{rgb}{1.000000,0.894118,0.788235}%
\pgfsetfillcolor{currentfill}%
\pgfsetlinewidth{0.000000pt}%
\definecolor{currentstroke}{rgb}{1.000000,0.894118,0.788235}%
\pgfsetstrokecolor{currentstroke}%
\pgfsetdash{}{0pt}%
\pgfpathmoveto{\pgfqpoint{2.394309in}{3.133865in}}%
\pgfpathlineto{\pgfqpoint{2.394309in}{3.164981in}}%
\pgfpathlineto{\pgfqpoint{2.236626in}{3.077747in}}%
\pgfpathlineto{\pgfqpoint{2.263574in}{3.031073in}}%
\pgfpathlineto{\pgfqpoint{2.394309in}{3.133865in}}%
\pgfpathclose%
\pgfusepath{fill}%
\end{pgfscope}%
\begin{pgfscope}%
\pgfpathrectangle{\pgfqpoint{0.765000in}{0.660000in}}{\pgfqpoint{4.620000in}{4.620000in}}%
\pgfusepath{clip}%
\pgfsetbuttcap%
\pgfsetroundjoin%
\definecolor{currentfill}{rgb}{1.000000,0.894118,0.788235}%
\pgfsetfillcolor{currentfill}%
\pgfsetlinewidth{0.000000pt}%
\definecolor{currentstroke}{rgb}{1.000000,0.894118,0.788235}%
\pgfsetstrokecolor{currentstroke}%
\pgfsetdash{}{0pt}%
\pgfpathmoveto{\pgfqpoint{2.421256in}{3.118307in}}%
\pgfpathlineto{\pgfqpoint{2.421256in}{3.149423in}}%
\pgfpathlineto{\pgfqpoint{2.263574in}{3.062189in}}%
\pgfpathlineto{\pgfqpoint{2.236626in}{3.046631in}}%
\pgfpathlineto{\pgfqpoint{2.421256in}{3.118307in}}%
\pgfpathclose%
\pgfusepath{fill}%
\end{pgfscope}%
\begin{pgfscope}%
\pgfpathrectangle{\pgfqpoint{0.765000in}{0.660000in}}{\pgfqpoint{4.620000in}{4.620000in}}%
\pgfusepath{clip}%
\pgfsetbuttcap%
\pgfsetroundjoin%
\definecolor{currentfill}{rgb}{1.000000,0.894118,0.788235}%
\pgfsetfillcolor{currentfill}%
\pgfsetlinewidth{0.000000pt}%
\definecolor{currentstroke}{rgb}{1.000000,0.894118,0.788235}%
\pgfsetstrokecolor{currentstroke}%
\pgfsetdash{}{0pt}%
\pgfpathmoveto{\pgfqpoint{2.421256in}{3.149423in}}%
\pgfpathlineto{\pgfqpoint{2.448203in}{3.164981in}}%
\pgfpathlineto{\pgfqpoint{2.290521in}{3.077747in}}%
\pgfpathlineto{\pgfqpoint{2.263574in}{3.062189in}}%
\pgfpathlineto{\pgfqpoint{2.421256in}{3.149423in}}%
\pgfpathclose%
\pgfusepath{fill}%
\end{pgfscope}%
\begin{pgfscope}%
\pgfpathrectangle{\pgfqpoint{0.765000in}{0.660000in}}{\pgfqpoint{4.620000in}{4.620000in}}%
\pgfusepath{clip}%
\pgfsetbuttcap%
\pgfsetroundjoin%
\definecolor{currentfill}{rgb}{1.000000,0.894118,0.788235}%
\pgfsetfillcolor{currentfill}%
\pgfsetlinewidth{0.000000pt}%
\definecolor{currentstroke}{rgb}{1.000000,0.894118,0.788235}%
\pgfsetstrokecolor{currentstroke}%
\pgfsetdash{}{0pt}%
\pgfpathmoveto{\pgfqpoint{2.394309in}{3.164981in}}%
\pgfpathlineto{\pgfqpoint{2.421256in}{3.149423in}}%
\pgfpathlineto{\pgfqpoint{2.263574in}{3.062189in}}%
\pgfpathlineto{\pgfqpoint{2.236626in}{3.077747in}}%
\pgfpathlineto{\pgfqpoint{2.394309in}{3.164981in}}%
\pgfpathclose%
\pgfusepath{fill}%
\end{pgfscope}%
\begin{pgfscope}%
\pgfpathrectangle{\pgfqpoint{0.765000in}{0.660000in}}{\pgfqpoint{4.620000in}{4.620000in}}%
\pgfusepath{clip}%
\pgfsetbuttcap%
\pgfsetroundjoin%
\definecolor{currentfill}{rgb}{1.000000,0.894118,0.788235}%
\pgfsetfillcolor{currentfill}%
\pgfsetlinewidth{0.000000pt}%
\definecolor{currentstroke}{rgb}{1.000000,0.894118,0.788235}%
\pgfsetstrokecolor{currentstroke}%
\pgfsetdash{}{0pt}%
\pgfpathmoveto{\pgfqpoint{3.856631in}{2.841507in}}%
\pgfpathlineto{\pgfqpoint{3.829684in}{2.825949in}}%
\pgfpathlineto{\pgfqpoint{3.856631in}{2.810391in}}%
\pgfpathlineto{\pgfqpoint{3.883578in}{2.825949in}}%
\pgfpathlineto{\pgfqpoint{3.856631in}{2.841507in}}%
\pgfpathclose%
\pgfusepath{fill}%
\end{pgfscope}%
\begin{pgfscope}%
\pgfpathrectangle{\pgfqpoint{0.765000in}{0.660000in}}{\pgfqpoint{4.620000in}{4.620000in}}%
\pgfusepath{clip}%
\pgfsetbuttcap%
\pgfsetroundjoin%
\definecolor{currentfill}{rgb}{1.000000,0.894118,0.788235}%
\pgfsetfillcolor{currentfill}%
\pgfsetlinewidth{0.000000pt}%
\definecolor{currentstroke}{rgb}{1.000000,0.894118,0.788235}%
\pgfsetstrokecolor{currentstroke}%
\pgfsetdash{}{0pt}%
\pgfpathmoveto{\pgfqpoint{3.856631in}{2.841507in}}%
\pgfpathlineto{\pgfqpoint{3.829684in}{2.825949in}}%
\pgfpathlineto{\pgfqpoint{3.829684in}{2.857065in}}%
\pgfpathlineto{\pgfqpoint{3.856631in}{2.872623in}}%
\pgfpathlineto{\pgfqpoint{3.856631in}{2.841507in}}%
\pgfpathclose%
\pgfusepath{fill}%
\end{pgfscope}%
\begin{pgfscope}%
\pgfpathrectangle{\pgfqpoint{0.765000in}{0.660000in}}{\pgfqpoint{4.620000in}{4.620000in}}%
\pgfusepath{clip}%
\pgfsetbuttcap%
\pgfsetroundjoin%
\definecolor{currentfill}{rgb}{1.000000,0.894118,0.788235}%
\pgfsetfillcolor{currentfill}%
\pgfsetlinewidth{0.000000pt}%
\definecolor{currentstroke}{rgb}{1.000000,0.894118,0.788235}%
\pgfsetstrokecolor{currentstroke}%
\pgfsetdash{}{0pt}%
\pgfpathmoveto{\pgfqpoint{3.856631in}{2.841507in}}%
\pgfpathlineto{\pgfqpoint{3.883578in}{2.825949in}}%
\pgfpathlineto{\pgfqpoint{3.883578in}{2.857065in}}%
\pgfpathlineto{\pgfqpoint{3.856631in}{2.872623in}}%
\pgfpathlineto{\pgfqpoint{3.856631in}{2.841507in}}%
\pgfpathclose%
\pgfusepath{fill}%
\end{pgfscope}%
\begin{pgfscope}%
\pgfpathrectangle{\pgfqpoint{0.765000in}{0.660000in}}{\pgfqpoint{4.620000in}{4.620000in}}%
\pgfusepath{clip}%
\pgfsetbuttcap%
\pgfsetroundjoin%
\definecolor{currentfill}{rgb}{1.000000,0.894118,0.788235}%
\pgfsetfillcolor{currentfill}%
\pgfsetlinewidth{0.000000pt}%
\definecolor{currentstroke}{rgb}{1.000000,0.894118,0.788235}%
\pgfsetstrokecolor{currentstroke}%
\pgfsetdash{}{0pt}%
\pgfpathmoveto{\pgfqpoint{3.839490in}{2.833104in}}%
\pgfpathlineto{\pgfqpoint{3.812542in}{2.817546in}}%
\pgfpathlineto{\pgfqpoint{3.839490in}{2.801988in}}%
\pgfpathlineto{\pgfqpoint{3.866437in}{2.817546in}}%
\pgfpathlineto{\pgfqpoint{3.839490in}{2.833104in}}%
\pgfpathclose%
\pgfusepath{fill}%
\end{pgfscope}%
\begin{pgfscope}%
\pgfpathrectangle{\pgfqpoint{0.765000in}{0.660000in}}{\pgfqpoint{4.620000in}{4.620000in}}%
\pgfusepath{clip}%
\pgfsetbuttcap%
\pgfsetroundjoin%
\definecolor{currentfill}{rgb}{1.000000,0.894118,0.788235}%
\pgfsetfillcolor{currentfill}%
\pgfsetlinewidth{0.000000pt}%
\definecolor{currentstroke}{rgb}{1.000000,0.894118,0.788235}%
\pgfsetstrokecolor{currentstroke}%
\pgfsetdash{}{0pt}%
\pgfpathmoveto{\pgfqpoint{3.839490in}{2.833104in}}%
\pgfpathlineto{\pgfqpoint{3.812542in}{2.817546in}}%
\pgfpathlineto{\pgfqpoint{3.812542in}{2.848662in}}%
\pgfpathlineto{\pgfqpoint{3.839490in}{2.864220in}}%
\pgfpathlineto{\pgfqpoint{3.839490in}{2.833104in}}%
\pgfpathclose%
\pgfusepath{fill}%
\end{pgfscope}%
\begin{pgfscope}%
\pgfpathrectangle{\pgfqpoint{0.765000in}{0.660000in}}{\pgfqpoint{4.620000in}{4.620000in}}%
\pgfusepath{clip}%
\pgfsetbuttcap%
\pgfsetroundjoin%
\definecolor{currentfill}{rgb}{1.000000,0.894118,0.788235}%
\pgfsetfillcolor{currentfill}%
\pgfsetlinewidth{0.000000pt}%
\definecolor{currentstroke}{rgb}{1.000000,0.894118,0.788235}%
\pgfsetstrokecolor{currentstroke}%
\pgfsetdash{}{0pt}%
\pgfpathmoveto{\pgfqpoint{3.839490in}{2.833104in}}%
\pgfpathlineto{\pgfqpoint{3.866437in}{2.817546in}}%
\pgfpathlineto{\pgfqpoint{3.866437in}{2.848662in}}%
\pgfpathlineto{\pgfqpoint{3.839490in}{2.864220in}}%
\pgfpathlineto{\pgfqpoint{3.839490in}{2.833104in}}%
\pgfpathclose%
\pgfusepath{fill}%
\end{pgfscope}%
\begin{pgfscope}%
\pgfpathrectangle{\pgfqpoint{0.765000in}{0.660000in}}{\pgfqpoint{4.620000in}{4.620000in}}%
\pgfusepath{clip}%
\pgfsetbuttcap%
\pgfsetroundjoin%
\definecolor{currentfill}{rgb}{1.000000,0.894118,0.788235}%
\pgfsetfillcolor{currentfill}%
\pgfsetlinewidth{0.000000pt}%
\definecolor{currentstroke}{rgb}{1.000000,0.894118,0.788235}%
\pgfsetstrokecolor{currentstroke}%
\pgfsetdash{}{0pt}%
\pgfpathmoveto{\pgfqpoint{3.856631in}{2.872623in}}%
\pgfpathlineto{\pgfqpoint{3.829684in}{2.857065in}}%
\pgfpathlineto{\pgfqpoint{3.856631in}{2.841507in}}%
\pgfpathlineto{\pgfqpoint{3.883578in}{2.857065in}}%
\pgfpathlineto{\pgfqpoint{3.856631in}{2.872623in}}%
\pgfpathclose%
\pgfusepath{fill}%
\end{pgfscope}%
\begin{pgfscope}%
\pgfpathrectangle{\pgfqpoint{0.765000in}{0.660000in}}{\pgfqpoint{4.620000in}{4.620000in}}%
\pgfusepath{clip}%
\pgfsetbuttcap%
\pgfsetroundjoin%
\definecolor{currentfill}{rgb}{1.000000,0.894118,0.788235}%
\pgfsetfillcolor{currentfill}%
\pgfsetlinewidth{0.000000pt}%
\definecolor{currentstroke}{rgb}{1.000000,0.894118,0.788235}%
\pgfsetstrokecolor{currentstroke}%
\pgfsetdash{}{0pt}%
\pgfpathmoveto{\pgfqpoint{3.856631in}{2.810391in}}%
\pgfpathlineto{\pgfqpoint{3.883578in}{2.825949in}}%
\pgfpathlineto{\pgfqpoint{3.883578in}{2.857065in}}%
\pgfpathlineto{\pgfqpoint{3.856631in}{2.841507in}}%
\pgfpathlineto{\pgfqpoint{3.856631in}{2.810391in}}%
\pgfpathclose%
\pgfusepath{fill}%
\end{pgfscope}%
\begin{pgfscope}%
\pgfpathrectangle{\pgfqpoint{0.765000in}{0.660000in}}{\pgfqpoint{4.620000in}{4.620000in}}%
\pgfusepath{clip}%
\pgfsetbuttcap%
\pgfsetroundjoin%
\definecolor{currentfill}{rgb}{1.000000,0.894118,0.788235}%
\pgfsetfillcolor{currentfill}%
\pgfsetlinewidth{0.000000pt}%
\definecolor{currentstroke}{rgb}{1.000000,0.894118,0.788235}%
\pgfsetstrokecolor{currentstroke}%
\pgfsetdash{}{0pt}%
\pgfpathmoveto{\pgfqpoint{3.829684in}{2.825949in}}%
\pgfpathlineto{\pgfqpoint{3.856631in}{2.810391in}}%
\pgfpathlineto{\pgfqpoint{3.856631in}{2.841507in}}%
\pgfpathlineto{\pgfqpoint{3.829684in}{2.857065in}}%
\pgfpathlineto{\pgfqpoint{3.829684in}{2.825949in}}%
\pgfpathclose%
\pgfusepath{fill}%
\end{pgfscope}%
\begin{pgfscope}%
\pgfpathrectangle{\pgfqpoint{0.765000in}{0.660000in}}{\pgfqpoint{4.620000in}{4.620000in}}%
\pgfusepath{clip}%
\pgfsetbuttcap%
\pgfsetroundjoin%
\definecolor{currentfill}{rgb}{1.000000,0.894118,0.788235}%
\pgfsetfillcolor{currentfill}%
\pgfsetlinewidth{0.000000pt}%
\definecolor{currentstroke}{rgb}{1.000000,0.894118,0.788235}%
\pgfsetstrokecolor{currentstroke}%
\pgfsetdash{}{0pt}%
\pgfpathmoveto{\pgfqpoint{3.839490in}{2.864220in}}%
\pgfpathlineto{\pgfqpoint{3.812542in}{2.848662in}}%
\pgfpathlineto{\pgfqpoint{3.839490in}{2.833104in}}%
\pgfpathlineto{\pgfqpoint{3.866437in}{2.848662in}}%
\pgfpathlineto{\pgfqpoint{3.839490in}{2.864220in}}%
\pgfpathclose%
\pgfusepath{fill}%
\end{pgfscope}%
\begin{pgfscope}%
\pgfpathrectangle{\pgfqpoint{0.765000in}{0.660000in}}{\pgfqpoint{4.620000in}{4.620000in}}%
\pgfusepath{clip}%
\pgfsetbuttcap%
\pgfsetroundjoin%
\definecolor{currentfill}{rgb}{1.000000,0.894118,0.788235}%
\pgfsetfillcolor{currentfill}%
\pgfsetlinewidth{0.000000pt}%
\definecolor{currentstroke}{rgb}{1.000000,0.894118,0.788235}%
\pgfsetstrokecolor{currentstroke}%
\pgfsetdash{}{0pt}%
\pgfpathmoveto{\pgfqpoint{3.839490in}{2.801988in}}%
\pgfpathlineto{\pgfqpoint{3.866437in}{2.817546in}}%
\pgfpathlineto{\pgfqpoint{3.866437in}{2.848662in}}%
\pgfpathlineto{\pgfqpoint{3.839490in}{2.833104in}}%
\pgfpathlineto{\pgfqpoint{3.839490in}{2.801988in}}%
\pgfpathclose%
\pgfusepath{fill}%
\end{pgfscope}%
\begin{pgfscope}%
\pgfpathrectangle{\pgfqpoint{0.765000in}{0.660000in}}{\pgfqpoint{4.620000in}{4.620000in}}%
\pgfusepath{clip}%
\pgfsetbuttcap%
\pgfsetroundjoin%
\definecolor{currentfill}{rgb}{1.000000,0.894118,0.788235}%
\pgfsetfillcolor{currentfill}%
\pgfsetlinewidth{0.000000pt}%
\definecolor{currentstroke}{rgb}{1.000000,0.894118,0.788235}%
\pgfsetstrokecolor{currentstroke}%
\pgfsetdash{}{0pt}%
\pgfpathmoveto{\pgfqpoint{3.812542in}{2.817546in}}%
\pgfpathlineto{\pgfqpoint{3.839490in}{2.801988in}}%
\pgfpathlineto{\pgfqpoint{3.839490in}{2.833104in}}%
\pgfpathlineto{\pgfqpoint{3.812542in}{2.848662in}}%
\pgfpathlineto{\pgfqpoint{3.812542in}{2.817546in}}%
\pgfpathclose%
\pgfusepath{fill}%
\end{pgfscope}%
\begin{pgfscope}%
\pgfpathrectangle{\pgfqpoint{0.765000in}{0.660000in}}{\pgfqpoint{4.620000in}{4.620000in}}%
\pgfusepath{clip}%
\pgfsetbuttcap%
\pgfsetroundjoin%
\definecolor{currentfill}{rgb}{1.000000,0.894118,0.788235}%
\pgfsetfillcolor{currentfill}%
\pgfsetlinewidth{0.000000pt}%
\definecolor{currentstroke}{rgb}{1.000000,0.894118,0.788235}%
\pgfsetstrokecolor{currentstroke}%
\pgfsetdash{}{0pt}%
\pgfpathmoveto{\pgfqpoint{3.856631in}{2.841507in}}%
\pgfpathlineto{\pgfqpoint{3.829684in}{2.825949in}}%
\pgfpathlineto{\pgfqpoint{3.812542in}{2.817546in}}%
\pgfpathlineto{\pgfqpoint{3.839490in}{2.833104in}}%
\pgfpathlineto{\pgfqpoint{3.856631in}{2.841507in}}%
\pgfpathclose%
\pgfusepath{fill}%
\end{pgfscope}%
\begin{pgfscope}%
\pgfpathrectangle{\pgfqpoint{0.765000in}{0.660000in}}{\pgfqpoint{4.620000in}{4.620000in}}%
\pgfusepath{clip}%
\pgfsetbuttcap%
\pgfsetroundjoin%
\definecolor{currentfill}{rgb}{1.000000,0.894118,0.788235}%
\pgfsetfillcolor{currentfill}%
\pgfsetlinewidth{0.000000pt}%
\definecolor{currentstroke}{rgb}{1.000000,0.894118,0.788235}%
\pgfsetstrokecolor{currentstroke}%
\pgfsetdash{}{0pt}%
\pgfpathmoveto{\pgfqpoint{3.883578in}{2.825949in}}%
\pgfpathlineto{\pgfqpoint{3.856631in}{2.841507in}}%
\pgfpathlineto{\pgfqpoint{3.839490in}{2.833104in}}%
\pgfpathlineto{\pgfqpoint{3.866437in}{2.817546in}}%
\pgfpathlineto{\pgfqpoint{3.883578in}{2.825949in}}%
\pgfpathclose%
\pgfusepath{fill}%
\end{pgfscope}%
\begin{pgfscope}%
\pgfpathrectangle{\pgfqpoint{0.765000in}{0.660000in}}{\pgfqpoint{4.620000in}{4.620000in}}%
\pgfusepath{clip}%
\pgfsetbuttcap%
\pgfsetroundjoin%
\definecolor{currentfill}{rgb}{1.000000,0.894118,0.788235}%
\pgfsetfillcolor{currentfill}%
\pgfsetlinewidth{0.000000pt}%
\definecolor{currentstroke}{rgb}{1.000000,0.894118,0.788235}%
\pgfsetstrokecolor{currentstroke}%
\pgfsetdash{}{0pt}%
\pgfpathmoveto{\pgfqpoint{3.856631in}{2.841507in}}%
\pgfpathlineto{\pgfqpoint{3.856631in}{2.872623in}}%
\pgfpathlineto{\pgfqpoint{3.839490in}{2.864220in}}%
\pgfpathlineto{\pgfqpoint{3.866437in}{2.817546in}}%
\pgfpathlineto{\pgfqpoint{3.856631in}{2.841507in}}%
\pgfpathclose%
\pgfusepath{fill}%
\end{pgfscope}%
\begin{pgfscope}%
\pgfpathrectangle{\pgfqpoint{0.765000in}{0.660000in}}{\pgfqpoint{4.620000in}{4.620000in}}%
\pgfusepath{clip}%
\pgfsetbuttcap%
\pgfsetroundjoin%
\definecolor{currentfill}{rgb}{1.000000,0.894118,0.788235}%
\pgfsetfillcolor{currentfill}%
\pgfsetlinewidth{0.000000pt}%
\definecolor{currentstroke}{rgb}{1.000000,0.894118,0.788235}%
\pgfsetstrokecolor{currentstroke}%
\pgfsetdash{}{0pt}%
\pgfpathmoveto{\pgfqpoint{3.883578in}{2.825949in}}%
\pgfpathlineto{\pgfqpoint{3.883578in}{2.857065in}}%
\pgfpathlineto{\pgfqpoint{3.866437in}{2.848662in}}%
\pgfpathlineto{\pgfqpoint{3.839490in}{2.833104in}}%
\pgfpathlineto{\pgfqpoint{3.883578in}{2.825949in}}%
\pgfpathclose%
\pgfusepath{fill}%
\end{pgfscope}%
\begin{pgfscope}%
\pgfpathrectangle{\pgfqpoint{0.765000in}{0.660000in}}{\pgfqpoint{4.620000in}{4.620000in}}%
\pgfusepath{clip}%
\pgfsetbuttcap%
\pgfsetroundjoin%
\definecolor{currentfill}{rgb}{1.000000,0.894118,0.788235}%
\pgfsetfillcolor{currentfill}%
\pgfsetlinewidth{0.000000pt}%
\definecolor{currentstroke}{rgb}{1.000000,0.894118,0.788235}%
\pgfsetstrokecolor{currentstroke}%
\pgfsetdash{}{0pt}%
\pgfpathmoveto{\pgfqpoint{3.829684in}{2.825949in}}%
\pgfpathlineto{\pgfqpoint{3.856631in}{2.810391in}}%
\pgfpathlineto{\pgfqpoint{3.839490in}{2.801988in}}%
\pgfpathlineto{\pgfqpoint{3.812542in}{2.817546in}}%
\pgfpathlineto{\pgfqpoint{3.829684in}{2.825949in}}%
\pgfpathclose%
\pgfusepath{fill}%
\end{pgfscope}%
\begin{pgfscope}%
\pgfpathrectangle{\pgfqpoint{0.765000in}{0.660000in}}{\pgfqpoint{4.620000in}{4.620000in}}%
\pgfusepath{clip}%
\pgfsetbuttcap%
\pgfsetroundjoin%
\definecolor{currentfill}{rgb}{1.000000,0.894118,0.788235}%
\pgfsetfillcolor{currentfill}%
\pgfsetlinewidth{0.000000pt}%
\definecolor{currentstroke}{rgb}{1.000000,0.894118,0.788235}%
\pgfsetstrokecolor{currentstroke}%
\pgfsetdash{}{0pt}%
\pgfpathmoveto{\pgfqpoint{3.856631in}{2.810391in}}%
\pgfpathlineto{\pgfqpoint{3.883578in}{2.825949in}}%
\pgfpathlineto{\pgfqpoint{3.866437in}{2.817546in}}%
\pgfpathlineto{\pgfqpoint{3.839490in}{2.801988in}}%
\pgfpathlineto{\pgfqpoint{3.856631in}{2.810391in}}%
\pgfpathclose%
\pgfusepath{fill}%
\end{pgfscope}%
\begin{pgfscope}%
\pgfpathrectangle{\pgfqpoint{0.765000in}{0.660000in}}{\pgfqpoint{4.620000in}{4.620000in}}%
\pgfusepath{clip}%
\pgfsetbuttcap%
\pgfsetroundjoin%
\definecolor{currentfill}{rgb}{1.000000,0.894118,0.788235}%
\pgfsetfillcolor{currentfill}%
\pgfsetlinewidth{0.000000pt}%
\definecolor{currentstroke}{rgb}{1.000000,0.894118,0.788235}%
\pgfsetstrokecolor{currentstroke}%
\pgfsetdash{}{0pt}%
\pgfpathmoveto{\pgfqpoint{3.856631in}{2.872623in}}%
\pgfpathlineto{\pgfqpoint{3.829684in}{2.857065in}}%
\pgfpathlineto{\pgfqpoint{3.812542in}{2.848662in}}%
\pgfpathlineto{\pgfqpoint{3.839490in}{2.864220in}}%
\pgfpathlineto{\pgfqpoint{3.856631in}{2.872623in}}%
\pgfpathclose%
\pgfusepath{fill}%
\end{pgfscope}%
\begin{pgfscope}%
\pgfpathrectangle{\pgfqpoint{0.765000in}{0.660000in}}{\pgfqpoint{4.620000in}{4.620000in}}%
\pgfusepath{clip}%
\pgfsetbuttcap%
\pgfsetroundjoin%
\definecolor{currentfill}{rgb}{1.000000,0.894118,0.788235}%
\pgfsetfillcolor{currentfill}%
\pgfsetlinewidth{0.000000pt}%
\definecolor{currentstroke}{rgb}{1.000000,0.894118,0.788235}%
\pgfsetstrokecolor{currentstroke}%
\pgfsetdash{}{0pt}%
\pgfpathmoveto{\pgfqpoint{3.883578in}{2.857065in}}%
\pgfpathlineto{\pgfqpoint{3.856631in}{2.872623in}}%
\pgfpathlineto{\pgfqpoint{3.839490in}{2.864220in}}%
\pgfpathlineto{\pgfqpoint{3.866437in}{2.848662in}}%
\pgfpathlineto{\pgfqpoint{3.883578in}{2.857065in}}%
\pgfpathclose%
\pgfusepath{fill}%
\end{pgfscope}%
\begin{pgfscope}%
\pgfpathrectangle{\pgfqpoint{0.765000in}{0.660000in}}{\pgfqpoint{4.620000in}{4.620000in}}%
\pgfusepath{clip}%
\pgfsetbuttcap%
\pgfsetroundjoin%
\definecolor{currentfill}{rgb}{1.000000,0.894118,0.788235}%
\pgfsetfillcolor{currentfill}%
\pgfsetlinewidth{0.000000pt}%
\definecolor{currentstroke}{rgb}{1.000000,0.894118,0.788235}%
\pgfsetstrokecolor{currentstroke}%
\pgfsetdash{}{0pt}%
\pgfpathmoveto{\pgfqpoint{3.829684in}{2.825949in}}%
\pgfpathlineto{\pgfqpoint{3.829684in}{2.857065in}}%
\pgfpathlineto{\pgfqpoint{3.812542in}{2.848662in}}%
\pgfpathlineto{\pgfqpoint{3.839490in}{2.801988in}}%
\pgfpathlineto{\pgfqpoint{3.829684in}{2.825949in}}%
\pgfpathclose%
\pgfusepath{fill}%
\end{pgfscope}%
\begin{pgfscope}%
\pgfpathrectangle{\pgfqpoint{0.765000in}{0.660000in}}{\pgfqpoint{4.620000in}{4.620000in}}%
\pgfusepath{clip}%
\pgfsetbuttcap%
\pgfsetroundjoin%
\definecolor{currentfill}{rgb}{1.000000,0.894118,0.788235}%
\pgfsetfillcolor{currentfill}%
\pgfsetlinewidth{0.000000pt}%
\definecolor{currentstroke}{rgb}{1.000000,0.894118,0.788235}%
\pgfsetstrokecolor{currentstroke}%
\pgfsetdash{}{0pt}%
\pgfpathmoveto{\pgfqpoint{3.856631in}{2.810391in}}%
\pgfpathlineto{\pgfqpoint{3.856631in}{2.841507in}}%
\pgfpathlineto{\pgfqpoint{3.839490in}{2.833104in}}%
\pgfpathlineto{\pgfqpoint{3.812542in}{2.817546in}}%
\pgfpathlineto{\pgfqpoint{3.856631in}{2.810391in}}%
\pgfpathclose%
\pgfusepath{fill}%
\end{pgfscope}%
\begin{pgfscope}%
\pgfpathrectangle{\pgfqpoint{0.765000in}{0.660000in}}{\pgfqpoint{4.620000in}{4.620000in}}%
\pgfusepath{clip}%
\pgfsetbuttcap%
\pgfsetroundjoin%
\definecolor{currentfill}{rgb}{1.000000,0.894118,0.788235}%
\pgfsetfillcolor{currentfill}%
\pgfsetlinewidth{0.000000pt}%
\definecolor{currentstroke}{rgb}{1.000000,0.894118,0.788235}%
\pgfsetstrokecolor{currentstroke}%
\pgfsetdash{}{0pt}%
\pgfpathmoveto{\pgfqpoint{3.856631in}{2.841507in}}%
\pgfpathlineto{\pgfqpoint{3.883578in}{2.857065in}}%
\pgfpathlineto{\pgfqpoint{3.866437in}{2.848662in}}%
\pgfpathlineto{\pgfqpoint{3.839490in}{2.833104in}}%
\pgfpathlineto{\pgfqpoint{3.856631in}{2.841507in}}%
\pgfpathclose%
\pgfusepath{fill}%
\end{pgfscope}%
\begin{pgfscope}%
\pgfpathrectangle{\pgfqpoint{0.765000in}{0.660000in}}{\pgfqpoint{4.620000in}{4.620000in}}%
\pgfusepath{clip}%
\pgfsetbuttcap%
\pgfsetroundjoin%
\definecolor{currentfill}{rgb}{1.000000,0.894118,0.788235}%
\pgfsetfillcolor{currentfill}%
\pgfsetlinewidth{0.000000pt}%
\definecolor{currentstroke}{rgb}{1.000000,0.894118,0.788235}%
\pgfsetstrokecolor{currentstroke}%
\pgfsetdash{}{0pt}%
\pgfpathmoveto{\pgfqpoint{3.829684in}{2.857065in}}%
\pgfpathlineto{\pgfqpoint{3.856631in}{2.841507in}}%
\pgfpathlineto{\pgfqpoint{3.839490in}{2.833104in}}%
\pgfpathlineto{\pgfqpoint{3.812542in}{2.848662in}}%
\pgfpathlineto{\pgfqpoint{3.829684in}{2.857065in}}%
\pgfpathclose%
\pgfusepath{fill}%
\end{pgfscope}%
\begin{pgfscope}%
\pgfpathrectangle{\pgfqpoint{0.765000in}{0.660000in}}{\pgfqpoint{4.620000in}{4.620000in}}%
\pgfusepath{clip}%
\pgfsetbuttcap%
\pgfsetroundjoin%
\definecolor{currentfill}{rgb}{1.000000,0.894118,0.788235}%
\pgfsetfillcolor{currentfill}%
\pgfsetlinewidth{0.000000pt}%
\definecolor{currentstroke}{rgb}{1.000000,0.894118,0.788235}%
\pgfsetstrokecolor{currentstroke}%
\pgfsetdash{}{0pt}%
\pgfpathmoveto{\pgfqpoint{3.856631in}{2.841507in}}%
\pgfpathlineto{\pgfqpoint{3.829684in}{2.825949in}}%
\pgfpathlineto{\pgfqpoint{3.856631in}{2.810391in}}%
\pgfpathlineto{\pgfqpoint{3.883578in}{2.825949in}}%
\pgfpathlineto{\pgfqpoint{3.856631in}{2.841507in}}%
\pgfpathclose%
\pgfusepath{fill}%
\end{pgfscope}%
\begin{pgfscope}%
\pgfpathrectangle{\pgfqpoint{0.765000in}{0.660000in}}{\pgfqpoint{4.620000in}{4.620000in}}%
\pgfusepath{clip}%
\pgfsetbuttcap%
\pgfsetroundjoin%
\definecolor{currentfill}{rgb}{1.000000,0.894118,0.788235}%
\pgfsetfillcolor{currentfill}%
\pgfsetlinewidth{0.000000pt}%
\definecolor{currentstroke}{rgb}{1.000000,0.894118,0.788235}%
\pgfsetstrokecolor{currentstroke}%
\pgfsetdash{}{0pt}%
\pgfpathmoveto{\pgfqpoint{3.856631in}{2.841507in}}%
\pgfpathlineto{\pgfqpoint{3.829684in}{2.825949in}}%
\pgfpathlineto{\pgfqpoint{3.829684in}{2.857065in}}%
\pgfpathlineto{\pgfqpoint{3.856631in}{2.872623in}}%
\pgfpathlineto{\pgfqpoint{3.856631in}{2.841507in}}%
\pgfpathclose%
\pgfusepath{fill}%
\end{pgfscope}%
\begin{pgfscope}%
\pgfpathrectangle{\pgfqpoint{0.765000in}{0.660000in}}{\pgfqpoint{4.620000in}{4.620000in}}%
\pgfusepath{clip}%
\pgfsetbuttcap%
\pgfsetroundjoin%
\definecolor{currentfill}{rgb}{1.000000,0.894118,0.788235}%
\pgfsetfillcolor{currentfill}%
\pgfsetlinewidth{0.000000pt}%
\definecolor{currentstroke}{rgb}{1.000000,0.894118,0.788235}%
\pgfsetstrokecolor{currentstroke}%
\pgfsetdash{}{0pt}%
\pgfpathmoveto{\pgfqpoint{3.856631in}{2.841507in}}%
\pgfpathlineto{\pgfqpoint{3.883578in}{2.825949in}}%
\pgfpathlineto{\pgfqpoint{3.883578in}{2.857065in}}%
\pgfpathlineto{\pgfqpoint{3.856631in}{2.872623in}}%
\pgfpathlineto{\pgfqpoint{3.856631in}{2.841507in}}%
\pgfpathclose%
\pgfusepath{fill}%
\end{pgfscope}%
\begin{pgfscope}%
\pgfpathrectangle{\pgfqpoint{0.765000in}{0.660000in}}{\pgfqpoint{4.620000in}{4.620000in}}%
\pgfusepath{clip}%
\pgfsetbuttcap%
\pgfsetroundjoin%
\definecolor{currentfill}{rgb}{1.000000,0.894118,0.788235}%
\pgfsetfillcolor{currentfill}%
\pgfsetlinewidth{0.000000pt}%
\definecolor{currentstroke}{rgb}{1.000000,0.894118,0.788235}%
\pgfsetstrokecolor{currentstroke}%
\pgfsetdash{}{0pt}%
\pgfpathmoveto{\pgfqpoint{3.868079in}{2.847690in}}%
\pgfpathlineto{\pgfqpoint{3.841132in}{2.832132in}}%
\pgfpathlineto{\pgfqpoint{3.868079in}{2.816574in}}%
\pgfpathlineto{\pgfqpoint{3.895026in}{2.832132in}}%
\pgfpathlineto{\pgfqpoint{3.868079in}{2.847690in}}%
\pgfpathclose%
\pgfusepath{fill}%
\end{pgfscope}%
\begin{pgfscope}%
\pgfpathrectangle{\pgfqpoint{0.765000in}{0.660000in}}{\pgfqpoint{4.620000in}{4.620000in}}%
\pgfusepath{clip}%
\pgfsetbuttcap%
\pgfsetroundjoin%
\definecolor{currentfill}{rgb}{1.000000,0.894118,0.788235}%
\pgfsetfillcolor{currentfill}%
\pgfsetlinewidth{0.000000pt}%
\definecolor{currentstroke}{rgb}{1.000000,0.894118,0.788235}%
\pgfsetstrokecolor{currentstroke}%
\pgfsetdash{}{0pt}%
\pgfpathmoveto{\pgfqpoint{3.868079in}{2.847690in}}%
\pgfpathlineto{\pgfqpoint{3.841132in}{2.832132in}}%
\pgfpathlineto{\pgfqpoint{3.841132in}{2.863248in}}%
\pgfpathlineto{\pgfqpoint{3.868079in}{2.878806in}}%
\pgfpathlineto{\pgfqpoint{3.868079in}{2.847690in}}%
\pgfpathclose%
\pgfusepath{fill}%
\end{pgfscope}%
\begin{pgfscope}%
\pgfpathrectangle{\pgfqpoint{0.765000in}{0.660000in}}{\pgfqpoint{4.620000in}{4.620000in}}%
\pgfusepath{clip}%
\pgfsetbuttcap%
\pgfsetroundjoin%
\definecolor{currentfill}{rgb}{1.000000,0.894118,0.788235}%
\pgfsetfillcolor{currentfill}%
\pgfsetlinewidth{0.000000pt}%
\definecolor{currentstroke}{rgb}{1.000000,0.894118,0.788235}%
\pgfsetstrokecolor{currentstroke}%
\pgfsetdash{}{0pt}%
\pgfpathmoveto{\pgfqpoint{3.868079in}{2.847690in}}%
\pgfpathlineto{\pgfqpoint{3.895026in}{2.832132in}}%
\pgfpathlineto{\pgfqpoint{3.895026in}{2.863248in}}%
\pgfpathlineto{\pgfqpoint{3.868079in}{2.878806in}}%
\pgfpathlineto{\pgfqpoint{3.868079in}{2.847690in}}%
\pgfpathclose%
\pgfusepath{fill}%
\end{pgfscope}%
\begin{pgfscope}%
\pgfpathrectangle{\pgfqpoint{0.765000in}{0.660000in}}{\pgfqpoint{4.620000in}{4.620000in}}%
\pgfusepath{clip}%
\pgfsetbuttcap%
\pgfsetroundjoin%
\definecolor{currentfill}{rgb}{1.000000,0.894118,0.788235}%
\pgfsetfillcolor{currentfill}%
\pgfsetlinewidth{0.000000pt}%
\definecolor{currentstroke}{rgb}{1.000000,0.894118,0.788235}%
\pgfsetstrokecolor{currentstroke}%
\pgfsetdash{}{0pt}%
\pgfpathmoveto{\pgfqpoint{3.856631in}{2.872623in}}%
\pgfpathlineto{\pgfqpoint{3.829684in}{2.857065in}}%
\pgfpathlineto{\pgfqpoint{3.856631in}{2.841507in}}%
\pgfpathlineto{\pgfqpoint{3.883578in}{2.857065in}}%
\pgfpathlineto{\pgfqpoint{3.856631in}{2.872623in}}%
\pgfpathclose%
\pgfusepath{fill}%
\end{pgfscope}%
\begin{pgfscope}%
\pgfpathrectangle{\pgfqpoint{0.765000in}{0.660000in}}{\pgfqpoint{4.620000in}{4.620000in}}%
\pgfusepath{clip}%
\pgfsetbuttcap%
\pgfsetroundjoin%
\definecolor{currentfill}{rgb}{1.000000,0.894118,0.788235}%
\pgfsetfillcolor{currentfill}%
\pgfsetlinewidth{0.000000pt}%
\definecolor{currentstroke}{rgb}{1.000000,0.894118,0.788235}%
\pgfsetstrokecolor{currentstroke}%
\pgfsetdash{}{0pt}%
\pgfpathmoveto{\pgfqpoint{3.856631in}{2.810391in}}%
\pgfpathlineto{\pgfqpoint{3.883578in}{2.825949in}}%
\pgfpathlineto{\pgfqpoint{3.883578in}{2.857065in}}%
\pgfpathlineto{\pgfqpoint{3.856631in}{2.841507in}}%
\pgfpathlineto{\pgfqpoint{3.856631in}{2.810391in}}%
\pgfpathclose%
\pgfusepath{fill}%
\end{pgfscope}%
\begin{pgfscope}%
\pgfpathrectangle{\pgfqpoint{0.765000in}{0.660000in}}{\pgfqpoint{4.620000in}{4.620000in}}%
\pgfusepath{clip}%
\pgfsetbuttcap%
\pgfsetroundjoin%
\definecolor{currentfill}{rgb}{1.000000,0.894118,0.788235}%
\pgfsetfillcolor{currentfill}%
\pgfsetlinewidth{0.000000pt}%
\definecolor{currentstroke}{rgb}{1.000000,0.894118,0.788235}%
\pgfsetstrokecolor{currentstroke}%
\pgfsetdash{}{0pt}%
\pgfpathmoveto{\pgfqpoint{3.829684in}{2.825949in}}%
\pgfpathlineto{\pgfqpoint{3.856631in}{2.810391in}}%
\pgfpathlineto{\pgfqpoint{3.856631in}{2.841507in}}%
\pgfpathlineto{\pgfqpoint{3.829684in}{2.857065in}}%
\pgfpathlineto{\pgfqpoint{3.829684in}{2.825949in}}%
\pgfpathclose%
\pgfusepath{fill}%
\end{pgfscope}%
\begin{pgfscope}%
\pgfpathrectangle{\pgfqpoint{0.765000in}{0.660000in}}{\pgfqpoint{4.620000in}{4.620000in}}%
\pgfusepath{clip}%
\pgfsetbuttcap%
\pgfsetroundjoin%
\definecolor{currentfill}{rgb}{1.000000,0.894118,0.788235}%
\pgfsetfillcolor{currentfill}%
\pgfsetlinewidth{0.000000pt}%
\definecolor{currentstroke}{rgb}{1.000000,0.894118,0.788235}%
\pgfsetstrokecolor{currentstroke}%
\pgfsetdash{}{0pt}%
\pgfpathmoveto{\pgfqpoint{3.868079in}{2.878806in}}%
\pgfpathlineto{\pgfqpoint{3.841132in}{2.863248in}}%
\pgfpathlineto{\pgfqpoint{3.868079in}{2.847690in}}%
\pgfpathlineto{\pgfqpoint{3.895026in}{2.863248in}}%
\pgfpathlineto{\pgfqpoint{3.868079in}{2.878806in}}%
\pgfpathclose%
\pgfusepath{fill}%
\end{pgfscope}%
\begin{pgfscope}%
\pgfpathrectangle{\pgfqpoint{0.765000in}{0.660000in}}{\pgfqpoint{4.620000in}{4.620000in}}%
\pgfusepath{clip}%
\pgfsetbuttcap%
\pgfsetroundjoin%
\definecolor{currentfill}{rgb}{1.000000,0.894118,0.788235}%
\pgfsetfillcolor{currentfill}%
\pgfsetlinewidth{0.000000pt}%
\definecolor{currentstroke}{rgb}{1.000000,0.894118,0.788235}%
\pgfsetstrokecolor{currentstroke}%
\pgfsetdash{}{0pt}%
\pgfpathmoveto{\pgfqpoint{3.868079in}{2.816574in}}%
\pgfpathlineto{\pgfqpoint{3.895026in}{2.832132in}}%
\pgfpathlineto{\pgfqpoint{3.895026in}{2.863248in}}%
\pgfpathlineto{\pgfqpoint{3.868079in}{2.847690in}}%
\pgfpathlineto{\pgfqpoint{3.868079in}{2.816574in}}%
\pgfpathclose%
\pgfusepath{fill}%
\end{pgfscope}%
\begin{pgfscope}%
\pgfpathrectangle{\pgfqpoint{0.765000in}{0.660000in}}{\pgfqpoint{4.620000in}{4.620000in}}%
\pgfusepath{clip}%
\pgfsetbuttcap%
\pgfsetroundjoin%
\definecolor{currentfill}{rgb}{1.000000,0.894118,0.788235}%
\pgfsetfillcolor{currentfill}%
\pgfsetlinewidth{0.000000pt}%
\definecolor{currentstroke}{rgb}{1.000000,0.894118,0.788235}%
\pgfsetstrokecolor{currentstroke}%
\pgfsetdash{}{0pt}%
\pgfpathmoveto{\pgfqpoint{3.841132in}{2.832132in}}%
\pgfpathlineto{\pgfqpoint{3.868079in}{2.816574in}}%
\pgfpathlineto{\pgfqpoint{3.868079in}{2.847690in}}%
\pgfpathlineto{\pgfqpoint{3.841132in}{2.863248in}}%
\pgfpathlineto{\pgfqpoint{3.841132in}{2.832132in}}%
\pgfpathclose%
\pgfusepath{fill}%
\end{pgfscope}%
\begin{pgfscope}%
\pgfpathrectangle{\pgfqpoint{0.765000in}{0.660000in}}{\pgfqpoint{4.620000in}{4.620000in}}%
\pgfusepath{clip}%
\pgfsetbuttcap%
\pgfsetroundjoin%
\definecolor{currentfill}{rgb}{1.000000,0.894118,0.788235}%
\pgfsetfillcolor{currentfill}%
\pgfsetlinewidth{0.000000pt}%
\definecolor{currentstroke}{rgb}{1.000000,0.894118,0.788235}%
\pgfsetstrokecolor{currentstroke}%
\pgfsetdash{}{0pt}%
\pgfpathmoveto{\pgfqpoint{3.856631in}{2.841507in}}%
\pgfpathlineto{\pgfqpoint{3.829684in}{2.825949in}}%
\pgfpathlineto{\pgfqpoint{3.841132in}{2.832132in}}%
\pgfpathlineto{\pgfqpoint{3.868079in}{2.847690in}}%
\pgfpathlineto{\pgfqpoint{3.856631in}{2.841507in}}%
\pgfpathclose%
\pgfusepath{fill}%
\end{pgfscope}%
\begin{pgfscope}%
\pgfpathrectangle{\pgfqpoint{0.765000in}{0.660000in}}{\pgfqpoint{4.620000in}{4.620000in}}%
\pgfusepath{clip}%
\pgfsetbuttcap%
\pgfsetroundjoin%
\definecolor{currentfill}{rgb}{1.000000,0.894118,0.788235}%
\pgfsetfillcolor{currentfill}%
\pgfsetlinewidth{0.000000pt}%
\definecolor{currentstroke}{rgb}{1.000000,0.894118,0.788235}%
\pgfsetstrokecolor{currentstroke}%
\pgfsetdash{}{0pt}%
\pgfpathmoveto{\pgfqpoint{3.883578in}{2.825949in}}%
\pgfpathlineto{\pgfqpoint{3.856631in}{2.841507in}}%
\pgfpathlineto{\pgfqpoint{3.868079in}{2.847690in}}%
\pgfpathlineto{\pgfqpoint{3.895026in}{2.832132in}}%
\pgfpathlineto{\pgfqpoint{3.883578in}{2.825949in}}%
\pgfpathclose%
\pgfusepath{fill}%
\end{pgfscope}%
\begin{pgfscope}%
\pgfpathrectangle{\pgfqpoint{0.765000in}{0.660000in}}{\pgfqpoint{4.620000in}{4.620000in}}%
\pgfusepath{clip}%
\pgfsetbuttcap%
\pgfsetroundjoin%
\definecolor{currentfill}{rgb}{1.000000,0.894118,0.788235}%
\pgfsetfillcolor{currentfill}%
\pgfsetlinewidth{0.000000pt}%
\definecolor{currentstroke}{rgb}{1.000000,0.894118,0.788235}%
\pgfsetstrokecolor{currentstroke}%
\pgfsetdash{}{0pt}%
\pgfpathmoveto{\pgfqpoint{3.856631in}{2.841507in}}%
\pgfpathlineto{\pgfqpoint{3.856631in}{2.872623in}}%
\pgfpathlineto{\pgfqpoint{3.868079in}{2.878806in}}%
\pgfpathlineto{\pgfqpoint{3.895026in}{2.832132in}}%
\pgfpathlineto{\pgfqpoint{3.856631in}{2.841507in}}%
\pgfpathclose%
\pgfusepath{fill}%
\end{pgfscope}%
\begin{pgfscope}%
\pgfpathrectangle{\pgfqpoint{0.765000in}{0.660000in}}{\pgfqpoint{4.620000in}{4.620000in}}%
\pgfusepath{clip}%
\pgfsetbuttcap%
\pgfsetroundjoin%
\definecolor{currentfill}{rgb}{1.000000,0.894118,0.788235}%
\pgfsetfillcolor{currentfill}%
\pgfsetlinewidth{0.000000pt}%
\definecolor{currentstroke}{rgb}{1.000000,0.894118,0.788235}%
\pgfsetstrokecolor{currentstroke}%
\pgfsetdash{}{0pt}%
\pgfpathmoveto{\pgfqpoint{3.883578in}{2.825949in}}%
\pgfpathlineto{\pgfqpoint{3.883578in}{2.857065in}}%
\pgfpathlineto{\pgfqpoint{3.895026in}{2.863248in}}%
\pgfpathlineto{\pgfqpoint{3.868079in}{2.847690in}}%
\pgfpathlineto{\pgfqpoint{3.883578in}{2.825949in}}%
\pgfpathclose%
\pgfusepath{fill}%
\end{pgfscope}%
\begin{pgfscope}%
\pgfpathrectangle{\pgfqpoint{0.765000in}{0.660000in}}{\pgfqpoint{4.620000in}{4.620000in}}%
\pgfusepath{clip}%
\pgfsetbuttcap%
\pgfsetroundjoin%
\definecolor{currentfill}{rgb}{1.000000,0.894118,0.788235}%
\pgfsetfillcolor{currentfill}%
\pgfsetlinewidth{0.000000pt}%
\definecolor{currentstroke}{rgb}{1.000000,0.894118,0.788235}%
\pgfsetstrokecolor{currentstroke}%
\pgfsetdash{}{0pt}%
\pgfpathmoveto{\pgfqpoint{3.829684in}{2.825949in}}%
\pgfpathlineto{\pgfqpoint{3.856631in}{2.810391in}}%
\pgfpathlineto{\pgfqpoint{3.868079in}{2.816574in}}%
\pgfpathlineto{\pgfqpoint{3.841132in}{2.832132in}}%
\pgfpathlineto{\pgfqpoint{3.829684in}{2.825949in}}%
\pgfpathclose%
\pgfusepath{fill}%
\end{pgfscope}%
\begin{pgfscope}%
\pgfpathrectangle{\pgfqpoint{0.765000in}{0.660000in}}{\pgfqpoint{4.620000in}{4.620000in}}%
\pgfusepath{clip}%
\pgfsetbuttcap%
\pgfsetroundjoin%
\definecolor{currentfill}{rgb}{1.000000,0.894118,0.788235}%
\pgfsetfillcolor{currentfill}%
\pgfsetlinewidth{0.000000pt}%
\definecolor{currentstroke}{rgb}{1.000000,0.894118,0.788235}%
\pgfsetstrokecolor{currentstroke}%
\pgfsetdash{}{0pt}%
\pgfpathmoveto{\pgfqpoint{3.856631in}{2.810391in}}%
\pgfpathlineto{\pgfqpoint{3.883578in}{2.825949in}}%
\pgfpathlineto{\pgfqpoint{3.895026in}{2.832132in}}%
\pgfpathlineto{\pgfqpoint{3.868079in}{2.816574in}}%
\pgfpathlineto{\pgfqpoint{3.856631in}{2.810391in}}%
\pgfpathclose%
\pgfusepath{fill}%
\end{pgfscope}%
\begin{pgfscope}%
\pgfpathrectangle{\pgfqpoint{0.765000in}{0.660000in}}{\pgfqpoint{4.620000in}{4.620000in}}%
\pgfusepath{clip}%
\pgfsetbuttcap%
\pgfsetroundjoin%
\definecolor{currentfill}{rgb}{1.000000,0.894118,0.788235}%
\pgfsetfillcolor{currentfill}%
\pgfsetlinewidth{0.000000pt}%
\definecolor{currentstroke}{rgb}{1.000000,0.894118,0.788235}%
\pgfsetstrokecolor{currentstroke}%
\pgfsetdash{}{0pt}%
\pgfpathmoveto{\pgfqpoint{3.856631in}{2.872623in}}%
\pgfpathlineto{\pgfqpoint{3.829684in}{2.857065in}}%
\pgfpathlineto{\pgfqpoint{3.841132in}{2.863248in}}%
\pgfpathlineto{\pgfqpoint{3.868079in}{2.878806in}}%
\pgfpathlineto{\pgfqpoint{3.856631in}{2.872623in}}%
\pgfpathclose%
\pgfusepath{fill}%
\end{pgfscope}%
\begin{pgfscope}%
\pgfpathrectangle{\pgfqpoint{0.765000in}{0.660000in}}{\pgfqpoint{4.620000in}{4.620000in}}%
\pgfusepath{clip}%
\pgfsetbuttcap%
\pgfsetroundjoin%
\definecolor{currentfill}{rgb}{1.000000,0.894118,0.788235}%
\pgfsetfillcolor{currentfill}%
\pgfsetlinewidth{0.000000pt}%
\definecolor{currentstroke}{rgb}{1.000000,0.894118,0.788235}%
\pgfsetstrokecolor{currentstroke}%
\pgfsetdash{}{0pt}%
\pgfpathmoveto{\pgfqpoint{3.883578in}{2.857065in}}%
\pgfpathlineto{\pgfqpoint{3.856631in}{2.872623in}}%
\pgfpathlineto{\pgfqpoint{3.868079in}{2.878806in}}%
\pgfpathlineto{\pgfqpoint{3.895026in}{2.863248in}}%
\pgfpathlineto{\pgfqpoint{3.883578in}{2.857065in}}%
\pgfpathclose%
\pgfusepath{fill}%
\end{pgfscope}%
\begin{pgfscope}%
\pgfpathrectangle{\pgfqpoint{0.765000in}{0.660000in}}{\pgfqpoint{4.620000in}{4.620000in}}%
\pgfusepath{clip}%
\pgfsetbuttcap%
\pgfsetroundjoin%
\definecolor{currentfill}{rgb}{1.000000,0.894118,0.788235}%
\pgfsetfillcolor{currentfill}%
\pgfsetlinewidth{0.000000pt}%
\definecolor{currentstroke}{rgb}{1.000000,0.894118,0.788235}%
\pgfsetstrokecolor{currentstroke}%
\pgfsetdash{}{0pt}%
\pgfpathmoveto{\pgfqpoint{3.829684in}{2.825949in}}%
\pgfpathlineto{\pgfqpoint{3.829684in}{2.857065in}}%
\pgfpathlineto{\pgfqpoint{3.841132in}{2.863248in}}%
\pgfpathlineto{\pgfqpoint{3.868079in}{2.816574in}}%
\pgfpathlineto{\pgfqpoint{3.829684in}{2.825949in}}%
\pgfpathclose%
\pgfusepath{fill}%
\end{pgfscope}%
\begin{pgfscope}%
\pgfpathrectangle{\pgfqpoint{0.765000in}{0.660000in}}{\pgfqpoint{4.620000in}{4.620000in}}%
\pgfusepath{clip}%
\pgfsetbuttcap%
\pgfsetroundjoin%
\definecolor{currentfill}{rgb}{1.000000,0.894118,0.788235}%
\pgfsetfillcolor{currentfill}%
\pgfsetlinewidth{0.000000pt}%
\definecolor{currentstroke}{rgb}{1.000000,0.894118,0.788235}%
\pgfsetstrokecolor{currentstroke}%
\pgfsetdash{}{0pt}%
\pgfpathmoveto{\pgfqpoint{3.856631in}{2.810391in}}%
\pgfpathlineto{\pgfqpoint{3.856631in}{2.841507in}}%
\pgfpathlineto{\pgfqpoint{3.868079in}{2.847690in}}%
\pgfpathlineto{\pgfqpoint{3.841132in}{2.832132in}}%
\pgfpathlineto{\pgfqpoint{3.856631in}{2.810391in}}%
\pgfpathclose%
\pgfusepath{fill}%
\end{pgfscope}%
\begin{pgfscope}%
\pgfpathrectangle{\pgfqpoint{0.765000in}{0.660000in}}{\pgfqpoint{4.620000in}{4.620000in}}%
\pgfusepath{clip}%
\pgfsetbuttcap%
\pgfsetroundjoin%
\definecolor{currentfill}{rgb}{1.000000,0.894118,0.788235}%
\pgfsetfillcolor{currentfill}%
\pgfsetlinewidth{0.000000pt}%
\definecolor{currentstroke}{rgb}{1.000000,0.894118,0.788235}%
\pgfsetstrokecolor{currentstroke}%
\pgfsetdash{}{0pt}%
\pgfpathmoveto{\pgfqpoint{3.856631in}{2.841507in}}%
\pgfpathlineto{\pgfqpoint{3.883578in}{2.857065in}}%
\pgfpathlineto{\pgfqpoint{3.895026in}{2.863248in}}%
\pgfpathlineto{\pgfqpoint{3.868079in}{2.847690in}}%
\pgfpathlineto{\pgfqpoint{3.856631in}{2.841507in}}%
\pgfpathclose%
\pgfusepath{fill}%
\end{pgfscope}%
\begin{pgfscope}%
\pgfpathrectangle{\pgfqpoint{0.765000in}{0.660000in}}{\pgfqpoint{4.620000in}{4.620000in}}%
\pgfusepath{clip}%
\pgfsetbuttcap%
\pgfsetroundjoin%
\definecolor{currentfill}{rgb}{1.000000,0.894118,0.788235}%
\pgfsetfillcolor{currentfill}%
\pgfsetlinewidth{0.000000pt}%
\definecolor{currentstroke}{rgb}{1.000000,0.894118,0.788235}%
\pgfsetstrokecolor{currentstroke}%
\pgfsetdash{}{0pt}%
\pgfpathmoveto{\pgfqpoint{3.829684in}{2.857065in}}%
\pgfpathlineto{\pgfqpoint{3.856631in}{2.841507in}}%
\pgfpathlineto{\pgfqpoint{3.868079in}{2.847690in}}%
\pgfpathlineto{\pgfqpoint{3.841132in}{2.863248in}}%
\pgfpathlineto{\pgfqpoint{3.829684in}{2.857065in}}%
\pgfpathclose%
\pgfusepath{fill}%
\end{pgfscope}%
\begin{pgfscope}%
\pgfpathrectangle{\pgfqpoint{0.765000in}{0.660000in}}{\pgfqpoint{4.620000in}{4.620000in}}%
\pgfusepath{clip}%
\pgfsetbuttcap%
\pgfsetroundjoin%
\definecolor{currentfill}{rgb}{1.000000,0.894118,0.788235}%
\pgfsetfillcolor{currentfill}%
\pgfsetlinewidth{0.000000pt}%
\definecolor{currentstroke}{rgb}{1.000000,0.894118,0.788235}%
\pgfsetstrokecolor{currentstroke}%
\pgfsetdash{}{0pt}%
\pgfpathmoveto{\pgfqpoint{3.712800in}{2.830079in}}%
\pgfpathlineto{\pgfqpoint{3.685852in}{2.814521in}}%
\pgfpathlineto{\pgfqpoint{3.712800in}{2.798963in}}%
\pgfpathlineto{\pgfqpoint{3.739747in}{2.814521in}}%
\pgfpathlineto{\pgfqpoint{3.712800in}{2.830079in}}%
\pgfpathclose%
\pgfusepath{fill}%
\end{pgfscope}%
\begin{pgfscope}%
\pgfpathrectangle{\pgfqpoint{0.765000in}{0.660000in}}{\pgfqpoint{4.620000in}{4.620000in}}%
\pgfusepath{clip}%
\pgfsetbuttcap%
\pgfsetroundjoin%
\definecolor{currentfill}{rgb}{1.000000,0.894118,0.788235}%
\pgfsetfillcolor{currentfill}%
\pgfsetlinewidth{0.000000pt}%
\definecolor{currentstroke}{rgb}{1.000000,0.894118,0.788235}%
\pgfsetstrokecolor{currentstroke}%
\pgfsetdash{}{0pt}%
\pgfpathmoveto{\pgfqpoint{3.712800in}{2.830079in}}%
\pgfpathlineto{\pgfqpoint{3.685852in}{2.814521in}}%
\pgfpathlineto{\pgfqpoint{3.685852in}{2.845636in}}%
\pgfpathlineto{\pgfqpoint{3.712800in}{2.861194in}}%
\pgfpathlineto{\pgfqpoint{3.712800in}{2.830079in}}%
\pgfpathclose%
\pgfusepath{fill}%
\end{pgfscope}%
\begin{pgfscope}%
\pgfpathrectangle{\pgfqpoint{0.765000in}{0.660000in}}{\pgfqpoint{4.620000in}{4.620000in}}%
\pgfusepath{clip}%
\pgfsetbuttcap%
\pgfsetroundjoin%
\definecolor{currentfill}{rgb}{1.000000,0.894118,0.788235}%
\pgfsetfillcolor{currentfill}%
\pgfsetlinewidth{0.000000pt}%
\definecolor{currentstroke}{rgb}{1.000000,0.894118,0.788235}%
\pgfsetstrokecolor{currentstroke}%
\pgfsetdash{}{0pt}%
\pgfpathmoveto{\pgfqpoint{3.712800in}{2.830079in}}%
\pgfpathlineto{\pgfqpoint{3.739747in}{2.814521in}}%
\pgfpathlineto{\pgfqpoint{3.739747in}{2.845636in}}%
\pgfpathlineto{\pgfqpoint{3.712800in}{2.861194in}}%
\pgfpathlineto{\pgfqpoint{3.712800in}{2.830079in}}%
\pgfpathclose%
\pgfusepath{fill}%
\end{pgfscope}%
\begin{pgfscope}%
\pgfpathrectangle{\pgfqpoint{0.765000in}{0.660000in}}{\pgfqpoint{4.620000in}{4.620000in}}%
\pgfusepath{clip}%
\pgfsetbuttcap%
\pgfsetroundjoin%
\definecolor{currentfill}{rgb}{1.000000,0.894118,0.788235}%
\pgfsetfillcolor{currentfill}%
\pgfsetlinewidth{0.000000pt}%
\definecolor{currentstroke}{rgb}{1.000000,0.894118,0.788235}%
\pgfsetstrokecolor{currentstroke}%
\pgfsetdash{}{0pt}%
\pgfpathmoveto{\pgfqpoint{3.706670in}{2.831422in}}%
\pgfpathlineto{\pgfqpoint{3.679722in}{2.815864in}}%
\pgfpathlineto{\pgfqpoint{3.706670in}{2.800306in}}%
\pgfpathlineto{\pgfqpoint{3.733617in}{2.815864in}}%
\pgfpathlineto{\pgfqpoint{3.706670in}{2.831422in}}%
\pgfpathclose%
\pgfusepath{fill}%
\end{pgfscope}%
\begin{pgfscope}%
\pgfpathrectangle{\pgfqpoint{0.765000in}{0.660000in}}{\pgfqpoint{4.620000in}{4.620000in}}%
\pgfusepath{clip}%
\pgfsetbuttcap%
\pgfsetroundjoin%
\definecolor{currentfill}{rgb}{1.000000,0.894118,0.788235}%
\pgfsetfillcolor{currentfill}%
\pgfsetlinewidth{0.000000pt}%
\definecolor{currentstroke}{rgb}{1.000000,0.894118,0.788235}%
\pgfsetstrokecolor{currentstroke}%
\pgfsetdash{}{0pt}%
\pgfpathmoveto{\pgfqpoint{3.706670in}{2.831422in}}%
\pgfpathlineto{\pgfqpoint{3.679722in}{2.815864in}}%
\pgfpathlineto{\pgfqpoint{3.679722in}{2.846980in}}%
\pgfpathlineto{\pgfqpoint{3.706670in}{2.862538in}}%
\pgfpathlineto{\pgfqpoint{3.706670in}{2.831422in}}%
\pgfpathclose%
\pgfusepath{fill}%
\end{pgfscope}%
\begin{pgfscope}%
\pgfpathrectangle{\pgfqpoint{0.765000in}{0.660000in}}{\pgfqpoint{4.620000in}{4.620000in}}%
\pgfusepath{clip}%
\pgfsetbuttcap%
\pgfsetroundjoin%
\definecolor{currentfill}{rgb}{1.000000,0.894118,0.788235}%
\pgfsetfillcolor{currentfill}%
\pgfsetlinewidth{0.000000pt}%
\definecolor{currentstroke}{rgb}{1.000000,0.894118,0.788235}%
\pgfsetstrokecolor{currentstroke}%
\pgfsetdash{}{0pt}%
\pgfpathmoveto{\pgfqpoint{3.706670in}{2.831422in}}%
\pgfpathlineto{\pgfqpoint{3.733617in}{2.815864in}}%
\pgfpathlineto{\pgfqpoint{3.733617in}{2.846980in}}%
\pgfpathlineto{\pgfqpoint{3.706670in}{2.862538in}}%
\pgfpathlineto{\pgfqpoint{3.706670in}{2.831422in}}%
\pgfpathclose%
\pgfusepath{fill}%
\end{pgfscope}%
\begin{pgfscope}%
\pgfpathrectangle{\pgfqpoint{0.765000in}{0.660000in}}{\pgfqpoint{4.620000in}{4.620000in}}%
\pgfusepath{clip}%
\pgfsetbuttcap%
\pgfsetroundjoin%
\definecolor{currentfill}{rgb}{1.000000,0.894118,0.788235}%
\pgfsetfillcolor{currentfill}%
\pgfsetlinewidth{0.000000pt}%
\definecolor{currentstroke}{rgb}{1.000000,0.894118,0.788235}%
\pgfsetstrokecolor{currentstroke}%
\pgfsetdash{}{0pt}%
\pgfpathmoveto{\pgfqpoint{3.712800in}{2.861194in}}%
\pgfpathlineto{\pgfqpoint{3.685852in}{2.845636in}}%
\pgfpathlineto{\pgfqpoint{3.712800in}{2.830079in}}%
\pgfpathlineto{\pgfqpoint{3.739747in}{2.845636in}}%
\pgfpathlineto{\pgfqpoint{3.712800in}{2.861194in}}%
\pgfpathclose%
\pgfusepath{fill}%
\end{pgfscope}%
\begin{pgfscope}%
\pgfpathrectangle{\pgfqpoint{0.765000in}{0.660000in}}{\pgfqpoint{4.620000in}{4.620000in}}%
\pgfusepath{clip}%
\pgfsetbuttcap%
\pgfsetroundjoin%
\definecolor{currentfill}{rgb}{1.000000,0.894118,0.788235}%
\pgfsetfillcolor{currentfill}%
\pgfsetlinewidth{0.000000pt}%
\definecolor{currentstroke}{rgb}{1.000000,0.894118,0.788235}%
\pgfsetstrokecolor{currentstroke}%
\pgfsetdash{}{0pt}%
\pgfpathmoveto{\pgfqpoint{3.712800in}{2.798963in}}%
\pgfpathlineto{\pgfqpoint{3.739747in}{2.814521in}}%
\pgfpathlineto{\pgfqpoint{3.739747in}{2.845636in}}%
\pgfpathlineto{\pgfqpoint{3.712800in}{2.830079in}}%
\pgfpathlineto{\pgfqpoint{3.712800in}{2.798963in}}%
\pgfpathclose%
\pgfusepath{fill}%
\end{pgfscope}%
\begin{pgfscope}%
\pgfpathrectangle{\pgfqpoint{0.765000in}{0.660000in}}{\pgfqpoint{4.620000in}{4.620000in}}%
\pgfusepath{clip}%
\pgfsetbuttcap%
\pgfsetroundjoin%
\definecolor{currentfill}{rgb}{1.000000,0.894118,0.788235}%
\pgfsetfillcolor{currentfill}%
\pgfsetlinewidth{0.000000pt}%
\definecolor{currentstroke}{rgb}{1.000000,0.894118,0.788235}%
\pgfsetstrokecolor{currentstroke}%
\pgfsetdash{}{0pt}%
\pgfpathmoveto{\pgfqpoint{3.685852in}{2.814521in}}%
\pgfpathlineto{\pgfqpoint{3.712800in}{2.798963in}}%
\pgfpathlineto{\pgfqpoint{3.712800in}{2.830079in}}%
\pgfpathlineto{\pgfqpoint{3.685852in}{2.845636in}}%
\pgfpathlineto{\pgfqpoint{3.685852in}{2.814521in}}%
\pgfpathclose%
\pgfusepath{fill}%
\end{pgfscope}%
\begin{pgfscope}%
\pgfpathrectangle{\pgfqpoint{0.765000in}{0.660000in}}{\pgfqpoint{4.620000in}{4.620000in}}%
\pgfusepath{clip}%
\pgfsetbuttcap%
\pgfsetroundjoin%
\definecolor{currentfill}{rgb}{1.000000,0.894118,0.788235}%
\pgfsetfillcolor{currentfill}%
\pgfsetlinewidth{0.000000pt}%
\definecolor{currentstroke}{rgb}{1.000000,0.894118,0.788235}%
\pgfsetstrokecolor{currentstroke}%
\pgfsetdash{}{0pt}%
\pgfpathmoveto{\pgfqpoint{3.706670in}{2.862538in}}%
\pgfpathlineto{\pgfqpoint{3.679722in}{2.846980in}}%
\pgfpathlineto{\pgfqpoint{3.706670in}{2.831422in}}%
\pgfpathlineto{\pgfqpoint{3.733617in}{2.846980in}}%
\pgfpathlineto{\pgfqpoint{3.706670in}{2.862538in}}%
\pgfpathclose%
\pgfusepath{fill}%
\end{pgfscope}%
\begin{pgfscope}%
\pgfpathrectangle{\pgfqpoint{0.765000in}{0.660000in}}{\pgfqpoint{4.620000in}{4.620000in}}%
\pgfusepath{clip}%
\pgfsetbuttcap%
\pgfsetroundjoin%
\definecolor{currentfill}{rgb}{1.000000,0.894118,0.788235}%
\pgfsetfillcolor{currentfill}%
\pgfsetlinewidth{0.000000pt}%
\definecolor{currentstroke}{rgb}{1.000000,0.894118,0.788235}%
\pgfsetstrokecolor{currentstroke}%
\pgfsetdash{}{0pt}%
\pgfpathmoveto{\pgfqpoint{3.706670in}{2.800306in}}%
\pgfpathlineto{\pgfqpoint{3.733617in}{2.815864in}}%
\pgfpathlineto{\pgfqpoint{3.733617in}{2.846980in}}%
\pgfpathlineto{\pgfqpoint{3.706670in}{2.831422in}}%
\pgfpathlineto{\pgfqpoint{3.706670in}{2.800306in}}%
\pgfpathclose%
\pgfusepath{fill}%
\end{pgfscope}%
\begin{pgfscope}%
\pgfpathrectangle{\pgfqpoint{0.765000in}{0.660000in}}{\pgfqpoint{4.620000in}{4.620000in}}%
\pgfusepath{clip}%
\pgfsetbuttcap%
\pgfsetroundjoin%
\definecolor{currentfill}{rgb}{1.000000,0.894118,0.788235}%
\pgfsetfillcolor{currentfill}%
\pgfsetlinewidth{0.000000pt}%
\definecolor{currentstroke}{rgb}{1.000000,0.894118,0.788235}%
\pgfsetstrokecolor{currentstroke}%
\pgfsetdash{}{0pt}%
\pgfpathmoveto{\pgfqpoint{3.679722in}{2.815864in}}%
\pgfpathlineto{\pgfqpoint{3.706670in}{2.800306in}}%
\pgfpathlineto{\pgfqpoint{3.706670in}{2.831422in}}%
\pgfpathlineto{\pgfqpoint{3.679722in}{2.846980in}}%
\pgfpathlineto{\pgfqpoint{3.679722in}{2.815864in}}%
\pgfpathclose%
\pgfusepath{fill}%
\end{pgfscope}%
\begin{pgfscope}%
\pgfpathrectangle{\pgfqpoint{0.765000in}{0.660000in}}{\pgfqpoint{4.620000in}{4.620000in}}%
\pgfusepath{clip}%
\pgfsetbuttcap%
\pgfsetroundjoin%
\definecolor{currentfill}{rgb}{1.000000,0.894118,0.788235}%
\pgfsetfillcolor{currentfill}%
\pgfsetlinewidth{0.000000pt}%
\definecolor{currentstroke}{rgb}{1.000000,0.894118,0.788235}%
\pgfsetstrokecolor{currentstroke}%
\pgfsetdash{}{0pt}%
\pgfpathmoveto{\pgfqpoint{3.712800in}{2.830079in}}%
\pgfpathlineto{\pgfqpoint{3.685852in}{2.814521in}}%
\pgfpathlineto{\pgfqpoint{3.679722in}{2.815864in}}%
\pgfpathlineto{\pgfqpoint{3.706670in}{2.831422in}}%
\pgfpathlineto{\pgfqpoint{3.712800in}{2.830079in}}%
\pgfpathclose%
\pgfusepath{fill}%
\end{pgfscope}%
\begin{pgfscope}%
\pgfpathrectangle{\pgfqpoint{0.765000in}{0.660000in}}{\pgfqpoint{4.620000in}{4.620000in}}%
\pgfusepath{clip}%
\pgfsetbuttcap%
\pgfsetroundjoin%
\definecolor{currentfill}{rgb}{1.000000,0.894118,0.788235}%
\pgfsetfillcolor{currentfill}%
\pgfsetlinewidth{0.000000pt}%
\definecolor{currentstroke}{rgb}{1.000000,0.894118,0.788235}%
\pgfsetstrokecolor{currentstroke}%
\pgfsetdash{}{0pt}%
\pgfpathmoveto{\pgfqpoint{3.739747in}{2.814521in}}%
\pgfpathlineto{\pgfqpoint{3.712800in}{2.830079in}}%
\pgfpathlineto{\pgfqpoint{3.706670in}{2.831422in}}%
\pgfpathlineto{\pgfqpoint{3.733617in}{2.815864in}}%
\pgfpathlineto{\pgfqpoint{3.739747in}{2.814521in}}%
\pgfpathclose%
\pgfusepath{fill}%
\end{pgfscope}%
\begin{pgfscope}%
\pgfpathrectangle{\pgfqpoint{0.765000in}{0.660000in}}{\pgfqpoint{4.620000in}{4.620000in}}%
\pgfusepath{clip}%
\pgfsetbuttcap%
\pgfsetroundjoin%
\definecolor{currentfill}{rgb}{1.000000,0.894118,0.788235}%
\pgfsetfillcolor{currentfill}%
\pgfsetlinewidth{0.000000pt}%
\definecolor{currentstroke}{rgb}{1.000000,0.894118,0.788235}%
\pgfsetstrokecolor{currentstroke}%
\pgfsetdash{}{0pt}%
\pgfpathmoveto{\pgfqpoint{3.712800in}{2.830079in}}%
\pgfpathlineto{\pgfqpoint{3.712800in}{2.861194in}}%
\pgfpathlineto{\pgfqpoint{3.706670in}{2.862538in}}%
\pgfpathlineto{\pgfqpoint{3.733617in}{2.815864in}}%
\pgfpathlineto{\pgfqpoint{3.712800in}{2.830079in}}%
\pgfpathclose%
\pgfusepath{fill}%
\end{pgfscope}%
\begin{pgfscope}%
\pgfpathrectangle{\pgfqpoint{0.765000in}{0.660000in}}{\pgfqpoint{4.620000in}{4.620000in}}%
\pgfusepath{clip}%
\pgfsetbuttcap%
\pgfsetroundjoin%
\definecolor{currentfill}{rgb}{1.000000,0.894118,0.788235}%
\pgfsetfillcolor{currentfill}%
\pgfsetlinewidth{0.000000pt}%
\definecolor{currentstroke}{rgb}{1.000000,0.894118,0.788235}%
\pgfsetstrokecolor{currentstroke}%
\pgfsetdash{}{0pt}%
\pgfpathmoveto{\pgfqpoint{3.739747in}{2.814521in}}%
\pgfpathlineto{\pgfqpoint{3.739747in}{2.845636in}}%
\pgfpathlineto{\pgfqpoint{3.733617in}{2.846980in}}%
\pgfpathlineto{\pgfqpoint{3.706670in}{2.831422in}}%
\pgfpathlineto{\pgfqpoint{3.739747in}{2.814521in}}%
\pgfpathclose%
\pgfusepath{fill}%
\end{pgfscope}%
\begin{pgfscope}%
\pgfpathrectangle{\pgfqpoint{0.765000in}{0.660000in}}{\pgfqpoint{4.620000in}{4.620000in}}%
\pgfusepath{clip}%
\pgfsetbuttcap%
\pgfsetroundjoin%
\definecolor{currentfill}{rgb}{1.000000,0.894118,0.788235}%
\pgfsetfillcolor{currentfill}%
\pgfsetlinewidth{0.000000pt}%
\definecolor{currentstroke}{rgb}{1.000000,0.894118,0.788235}%
\pgfsetstrokecolor{currentstroke}%
\pgfsetdash{}{0pt}%
\pgfpathmoveto{\pgfqpoint{3.685852in}{2.814521in}}%
\pgfpathlineto{\pgfqpoint{3.712800in}{2.798963in}}%
\pgfpathlineto{\pgfqpoint{3.706670in}{2.800306in}}%
\pgfpathlineto{\pgfqpoint{3.679722in}{2.815864in}}%
\pgfpathlineto{\pgfqpoint{3.685852in}{2.814521in}}%
\pgfpathclose%
\pgfusepath{fill}%
\end{pgfscope}%
\begin{pgfscope}%
\pgfpathrectangle{\pgfqpoint{0.765000in}{0.660000in}}{\pgfqpoint{4.620000in}{4.620000in}}%
\pgfusepath{clip}%
\pgfsetbuttcap%
\pgfsetroundjoin%
\definecolor{currentfill}{rgb}{1.000000,0.894118,0.788235}%
\pgfsetfillcolor{currentfill}%
\pgfsetlinewidth{0.000000pt}%
\definecolor{currentstroke}{rgb}{1.000000,0.894118,0.788235}%
\pgfsetstrokecolor{currentstroke}%
\pgfsetdash{}{0pt}%
\pgfpathmoveto{\pgfqpoint{3.712800in}{2.798963in}}%
\pgfpathlineto{\pgfqpoint{3.739747in}{2.814521in}}%
\pgfpathlineto{\pgfqpoint{3.733617in}{2.815864in}}%
\pgfpathlineto{\pgfqpoint{3.706670in}{2.800306in}}%
\pgfpathlineto{\pgfqpoint{3.712800in}{2.798963in}}%
\pgfpathclose%
\pgfusepath{fill}%
\end{pgfscope}%
\begin{pgfscope}%
\pgfpathrectangle{\pgfqpoint{0.765000in}{0.660000in}}{\pgfqpoint{4.620000in}{4.620000in}}%
\pgfusepath{clip}%
\pgfsetbuttcap%
\pgfsetroundjoin%
\definecolor{currentfill}{rgb}{1.000000,0.894118,0.788235}%
\pgfsetfillcolor{currentfill}%
\pgfsetlinewidth{0.000000pt}%
\definecolor{currentstroke}{rgb}{1.000000,0.894118,0.788235}%
\pgfsetstrokecolor{currentstroke}%
\pgfsetdash{}{0pt}%
\pgfpathmoveto{\pgfqpoint{3.712800in}{2.861194in}}%
\pgfpathlineto{\pgfqpoint{3.685852in}{2.845636in}}%
\pgfpathlineto{\pgfqpoint{3.679722in}{2.846980in}}%
\pgfpathlineto{\pgfqpoint{3.706670in}{2.862538in}}%
\pgfpathlineto{\pgfqpoint{3.712800in}{2.861194in}}%
\pgfpathclose%
\pgfusepath{fill}%
\end{pgfscope}%
\begin{pgfscope}%
\pgfpathrectangle{\pgfqpoint{0.765000in}{0.660000in}}{\pgfqpoint{4.620000in}{4.620000in}}%
\pgfusepath{clip}%
\pgfsetbuttcap%
\pgfsetroundjoin%
\definecolor{currentfill}{rgb}{1.000000,0.894118,0.788235}%
\pgfsetfillcolor{currentfill}%
\pgfsetlinewidth{0.000000pt}%
\definecolor{currentstroke}{rgb}{1.000000,0.894118,0.788235}%
\pgfsetstrokecolor{currentstroke}%
\pgfsetdash{}{0pt}%
\pgfpathmoveto{\pgfqpoint{3.739747in}{2.845636in}}%
\pgfpathlineto{\pgfqpoint{3.712800in}{2.861194in}}%
\pgfpathlineto{\pgfqpoint{3.706670in}{2.862538in}}%
\pgfpathlineto{\pgfqpoint{3.733617in}{2.846980in}}%
\pgfpathlineto{\pgfqpoint{3.739747in}{2.845636in}}%
\pgfpathclose%
\pgfusepath{fill}%
\end{pgfscope}%
\begin{pgfscope}%
\pgfpathrectangle{\pgfqpoint{0.765000in}{0.660000in}}{\pgfqpoint{4.620000in}{4.620000in}}%
\pgfusepath{clip}%
\pgfsetbuttcap%
\pgfsetroundjoin%
\definecolor{currentfill}{rgb}{1.000000,0.894118,0.788235}%
\pgfsetfillcolor{currentfill}%
\pgfsetlinewidth{0.000000pt}%
\definecolor{currentstroke}{rgb}{1.000000,0.894118,0.788235}%
\pgfsetstrokecolor{currentstroke}%
\pgfsetdash{}{0pt}%
\pgfpathmoveto{\pgfqpoint{3.685852in}{2.814521in}}%
\pgfpathlineto{\pgfqpoint{3.685852in}{2.845636in}}%
\pgfpathlineto{\pgfqpoint{3.679722in}{2.846980in}}%
\pgfpathlineto{\pgfqpoint{3.706670in}{2.800306in}}%
\pgfpathlineto{\pgfqpoint{3.685852in}{2.814521in}}%
\pgfpathclose%
\pgfusepath{fill}%
\end{pgfscope}%
\begin{pgfscope}%
\pgfpathrectangle{\pgfqpoint{0.765000in}{0.660000in}}{\pgfqpoint{4.620000in}{4.620000in}}%
\pgfusepath{clip}%
\pgfsetbuttcap%
\pgfsetroundjoin%
\definecolor{currentfill}{rgb}{1.000000,0.894118,0.788235}%
\pgfsetfillcolor{currentfill}%
\pgfsetlinewidth{0.000000pt}%
\definecolor{currentstroke}{rgb}{1.000000,0.894118,0.788235}%
\pgfsetstrokecolor{currentstroke}%
\pgfsetdash{}{0pt}%
\pgfpathmoveto{\pgfqpoint{3.712800in}{2.798963in}}%
\pgfpathlineto{\pgfqpoint{3.712800in}{2.830079in}}%
\pgfpathlineto{\pgfqpoint{3.706670in}{2.831422in}}%
\pgfpathlineto{\pgfqpoint{3.679722in}{2.815864in}}%
\pgfpathlineto{\pgfqpoint{3.712800in}{2.798963in}}%
\pgfpathclose%
\pgfusepath{fill}%
\end{pgfscope}%
\begin{pgfscope}%
\pgfpathrectangle{\pgfqpoint{0.765000in}{0.660000in}}{\pgfqpoint{4.620000in}{4.620000in}}%
\pgfusepath{clip}%
\pgfsetbuttcap%
\pgfsetroundjoin%
\definecolor{currentfill}{rgb}{1.000000,0.894118,0.788235}%
\pgfsetfillcolor{currentfill}%
\pgfsetlinewidth{0.000000pt}%
\definecolor{currentstroke}{rgb}{1.000000,0.894118,0.788235}%
\pgfsetstrokecolor{currentstroke}%
\pgfsetdash{}{0pt}%
\pgfpathmoveto{\pgfqpoint{3.712800in}{2.830079in}}%
\pgfpathlineto{\pgfqpoint{3.739747in}{2.845636in}}%
\pgfpathlineto{\pgfqpoint{3.733617in}{2.846980in}}%
\pgfpathlineto{\pgfqpoint{3.706670in}{2.831422in}}%
\pgfpathlineto{\pgfqpoint{3.712800in}{2.830079in}}%
\pgfpathclose%
\pgfusepath{fill}%
\end{pgfscope}%
\begin{pgfscope}%
\pgfpathrectangle{\pgfqpoint{0.765000in}{0.660000in}}{\pgfqpoint{4.620000in}{4.620000in}}%
\pgfusepath{clip}%
\pgfsetbuttcap%
\pgfsetroundjoin%
\definecolor{currentfill}{rgb}{1.000000,0.894118,0.788235}%
\pgfsetfillcolor{currentfill}%
\pgfsetlinewidth{0.000000pt}%
\definecolor{currentstroke}{rgb}{1.000000,0.894118,0.788235}%
\pgfsetstrokecolor{currentstroke}%
\pgfsetdash{}{0pt}%
\pgfpathmoveto{\pgfqpoint{3.685852in}{2.845636in}}%
\pgfpathlineto{\pgfqpoint{3.712800in}{2.830079in}}%
\pgfpathlineto{\pgfqpoint{3.706670in}{2.831422in}}%
\pgfpathlineto{\pgfqpoint{3.679722in}{2.846980in}}%
\pgfpathlineto{\pgfqpoint{3.685852in}{2.845636in}}%
\pgfpathclose%
\pgfusepath{fill}%
\end{pgfscope}%
\begin{pgfscope}%
\pgfpathrectangle{\pgfqpoint{0.765000in}{0.660000in}}{\pgfqpoint{4.620000in}{4.620000in}}%
\pgfusepath{clip}%
\pgfsetbuttcap%
\pgfsetroundjoin%
\definecolor{currentfill}{rgb}{1.000000,0.894118,0.788235}%
\pgfsetfillcolor{currentfill}%
\pgfsetlinewidth{0.000000pt}%
\definecolor{currentstroke}{rgb}{1.000000,0.894118,0.788235}%
\pgfsetstrokecolor{currentstroke}%
\pgfsetdash{}{0pt}%
\pgfpathmoveto{\pgfqpoint{3.712800in}{2.830079in}}%
\pgfpathlineto{\pgfqpoint{3.685852in}{2.814521in}}%
\pgfpathlineto{\pgfqpoint{3.712800in}{2.798963in}}%
\pgfpathlineto{\pgfqpoint{3.739747in}{2.814521in}}%
\pgfpathlineto{\pgfqpoint{3.712800in}{2.830079in}}%
\pgfpathclose%
\pgfusepath{fill}%
\end{pgfscope}%
\begin{pgfscope}%
\pgfpathrectangle{\pgfqpoint{0.765000in}{0.660000in}}{\pgfqpoint{4.620000in}{4.620000in}}%
\pgfusepath{clip}%
\pgfsetbuttcap%
\pgfsetroundjoin%
\definecolor{currentfill}{rgb}{1.000000,0.894118,0.788235}%
\pgfsetfillcolor{currentfill}%
\pgfsetlinewidth{0.000000pt}%
\definecolor{currentstroke}{rgb}{1.000000,0.894118,0.788235}%
\pgfsetstrokecolor{currentstroke}%
\pgfsetdash{}{0pt}%
\pgfpathmoveto{\pgfqpoint{3.712800in}{2.830079in}}%
\pgfpathlineto{\pgfqpoint{3.685852in}{2.814521in}}%
\pgfpathlineto{\pgfqpoint{3.685852in}{2.845636in}}%
\pgfpathlineto{\pgfqpoint{3.712800in}{2.861194in}}%
\pgfpathlineto{\pgfqpoint{3.712800in}{2.830079in}}%
\pgfpathclose%
\pgfusepath{fill}%
\end{pgfscope}%
\begin{pgfscope}%
\pgfpathrectangle{\pgfqpoint{0.765000in}{0.660000in}}{\pgfqpoint{4.620000in}{4.620000in}}%
\pgfusepath{clip}%
\pgfsetbuttcap%
\pgfsetroundjoin%
\definecolor{currentfill}{rgb}{1.000000,0.894118,0.788235}%
\pgfsetfillcolor{currentfill}%
\pgfsetlinewidth{0.000000pt}%
\definecolor{currentstroke}{rgb}{1.000000,0.894118,0.788235}%
\pgfsetstrokecolor{currentstroke}%
\pgfsetdash{}{0pt}%
\pgfpathmoveto{\pgfqpoint{3.712800in}{2.830079in}}%
\pgfpathlineto{\pgfqpoint{3.739747in}{2.814521in}}%
\pgfpathlineto{\pgfqpoint{3.739747in}{2.845636in}}%
\pgfpathlineto{\pgfqpoint{3.712800in}{2.861194in}}%
\pgfpathlineto{\pgfqpoint{3.712800in}{2.830079in}}%
\pgfpathclose%
\pgfusepath{fill}%
\end{pgfscope}%
\begin{pgfscope}%
\pgfpathrectangle{\pgfqpoint{0.765000in}{0.660000in}}{\pgfqpoint{4.620000in}{4.620000in}}%
\pgfusepath{clip}%
\pgfsetbuttcap%
\pgfsetroundjoin%
\definecolor{currentfill}{rgb}{1.000000,0.894118,0.788235}%
\pgfsetfillcolor{currentfill}%
\pgfsetlinewidth{0.000000pt}%
\definecolor{currentstroke}{rgb}{1.000000,0.894118,0.788235}%
\pgfsetstrokecolor{currentstroke}%
\pgfsetdash{}{0pt}%
\pgfpathmoveto{\pgfqpoint{3.717641in}{2.829213in}}%
\pgfpathlineto{\pgfqpoint{3.690694in}{2.813655in}}%
\pgfpathlineto{\pgfqpoint{3.717641in}{2.798097in}}%
\pgfpathlineto{\pgfqpoint{3.744588in}{2.813655in}}%
\pgfpathlineto{\pgfqpoint{3.717641in}{2.829213in}}%
\pgfpathclose%
\pgfusepath{fill}%
\end{pgfscope}%
\begin{pgfscope}%
\pgfpathrectangle{\pgfqpoint{0.765000in}{0.660000in}}{\pgfqpoint{4.620000in}{4.620000in}}%
\pgfusepath{clip}%
\pgfsetbuttcap%
\pgfsetroundjoin%
\definecolor{currentfill}{rgb}{1.000000,0.894118,0.788235}%
\pgfsetfillcolor{currentfill}%
\pgfsetlinewidth{0.000000pt}%
\definecolor{currentstroke}{rgb}{1.000000,0.894118,0.788235}%
\pgfsetstrokecolor{currentstroke}%
\pgfsetdash{}{0pt}%
\pgfpathmoveto{\pgfqpoint{3.717641in}{2.829213in}}%
\pgfpathlineto{\pgfqpoint{3.690694in}{2.813655in}}%
\pgfpathlineto{\pgfqpoint{3.690694in}{2.844771in}}%
\pgfpathlineto{\pgfqpoint{3.717641in}{2.860329in}}%
\pgfpathlineto{\pgfqpoint{3.717641in}{2.829213in}}%
\pgfpathclose%
\pgfusepath{fill}%
\end{pgfscope}%
\begin{pgfscope}%
\pgfpathrectangle{\pgfqpoint{0.765000in}{0.660000in}}{\pgfqpoint{4.620000in}{4.620000in}}%
\pgfusepath{clip}%
\pgfsetbuttcap%
\pgfsetroundjoin%
\definecolor{currentfill}{rgb}{1.000000,0.894118,0.788235}%
\pgfsetfillcolor{currentfill}%
\pgfsetlinewidth{0.000000pt}%
\definecolor{currentstroke}{rgb}{1.000000,0.894118,0.788235}%
\pgfsetstrokecolor{currentstroke}%
\pgfsetdash{}{0pt}%
\pgfpathmoveto{\pgfqpoint{3.717641in}{2.829213in}}%
\pgfpathlineto{\pgfqpoint{3.744588in}{2.813655in}}%
\pgfpathlineto{\pgfqpoint{3.744588in}{2.844771in}}%
\pgfpathlineto{\pgfqpoint{3.717641in}{2.860329in}}%
\pgfpathlineto{\pgfqpoint{3.717641in}{2.829213in}}%
\pgfpathclose%
\pgfusepath{fill}%
\end{pgfscope}%
\begin{pgfscope}%
\pgfpathrectangle{\pgfqpoint{0.765000in}{0.660000in}}{\pgfqpoint{4.620000in}{4.620000in}}%
\pgfusepath{clip}%
\pgfsetbuttcap%
\pgfsetroundjoin%
\definecolor{currentfill}{rgb}{1.000000,0.894118,0.788235}%
\pgfsetfillcolor{currentfill}%
\pgfsetlinewidth{0.000000pt}%
\definecolor{currentstroke}{rgb}{1.000000,0.894118,0.788235}%
\pgfsetstrokecolor{currentstroke}%
\pgfsetdash{}{0pt}%
\pgfpathmoveto{\pgfqpoint{3.712800in}{2.861194in}}%
\pgfpathlineto{\pgfqpoint{3.685852in}{2.845636in}}%
\pgfpathlineto{\pgfqpoint{3.712800in}{2.830079in}}%
\pgfpathlineto{\pgfqpoint{3.739747in}{2.845636in}}%
\pgfpathlineto{\pgfqpoint{3.712800in}{2.861194in}}%
\pgfpathclose%
\pgfusepath{fill}%
\end{pgfscope}%
\begin{pgfscope}%
\pgfpathrectangle{\pgfqpoint{0.765000in}{0.660000in}}{\pgfqpoint{4.620000in}{4.620000in}}%
\pgfusepath{clip}%
\pgfsetbuttcap%
\pgfsetroundjoin%
\definecolor{currentfill}{rgb}{1.000000,0.894118,0.788235}%
\pgfsetfillcolor{currentfill}%
\pgfsetlinewidth{0.000000pt}%
\definecolor{currentstroke}{rgb}{1.000000,0.894118,0.788235}%
\pgfsetstrokecolor{currentstroke}%
\pgfsetdash{}{0pt}%
\pgfpathmoveto{\pgfqpoint{3.712800in}{2.798963in}}%
\pgfpathlineto{\pgfqpoint{3.739747in}{2.814521in}}%
\pgfpathlineto{\pgfqpoint{3.739747in}{2.845636in}}%
\pgfpathlineto{\pgfqpoint{3.712800in}{2.830079in}}%
\pgfpathlineto{\pgfqpoint{3.712800in}{2.798963in}}%
\pgfpathclose%
\pgfusepath{fill}%
\end{pgfscope}%
\begin{pgfscope}%
\pgfpathrectangle{\pgfqpoint{0.765000in}{0.660000in}}{\pgfqpoint{4.620000in}{4.620000in}}%
\pgfusepath{clip}%
\pgfsetbuttcap%
\pgfsetroundjoin%
\definecolor{currentfill}{rgb}{1.000000,0.894118,0.788235}%
\pgfsetfillcolor{currentfill}%
\pgfsetlinewidth{0.000000pt}%
\definecolor{currentstroke}{rgb}{1.000000,0.894118,0.788235}%
\pgfsetstrokecolor{currentstroke}%
\pgfsetdash{}{0pt}%
\pgfpathmoveto{\pgfqpoint{3.685852in}{2.814521in}}%
\pgfpathlineto{\pgfqpoint{3.712800in}{2.798963in}}%
\pgfpathlineto{\pgfqpoint{3.712800in}{2.830079in}}%
\pgfpathlineto{\pgfqpoint{3.685852in}{2.845636in}}%
\pgfpathlineto{\pgfqpoint{3.685852in}{2.814521in}}%
\pgfpathclose%
\pgfusepath{fill}%
\end{pgfscope}%
\begin{pgfscope}%
\pgfpathrectangle{\pgfqpoint{0.765000in}{0.660000in}}{\pgfqpoint{4.620000in}{4.620000in}}%
\pgfusepath{clip}%
\pgfsetbuttcap%
\pgfsetroundjoin%
\definecolor{currentfill}{rgb}{1.000000,0.894118,0.788235}%
\pgfsetfillcolor{currentfill}%
\pgfsetlinewidth{0.000000pt}%
\definecolor{currentstroke}{rgb}{1.000000,0.894118,0.788235}%
\pgfsetstrokecolor{currentstroke}%
\pgfsetdash{}{0pt}%
\pgfpathmoveto{\pgfqpoint{3.717641in}{2.860329in}}%
\pgfpathlineto{\pgfqpoint{3.690694in}{2.844771in}}%
\pgfpathlineto{\pgfqpoint{3.717641in}{2.829213in}}%
\pgfpathlineto{\pgfqpoint{3.744588in}{2.844771in}}%
\pgfpathlineto{\pgfqpoint{3.717641in}{2.860329in}}%
\pgfpathclose%
\pgfusepath{fill}%
\end{pgfscope}%
\begin{pgfscope}%
\pgfpathrectangle{\pgfqpoint{0.765000in}{0.660000in}}{\pgfqpoint{4.620000in}{4.620000in}}%
\pgfusepath{clip}%
\pgfsetbuttcap%
\pgfsetroundjoin%
\definecolor{currentfill}{rgb}{1.000000,0.894118,0.788235}%
\pgfsetfillcolor{currentfill}%
\pgfsetlinewidth{0.000000pt}%
\definecolor{currentstroke}{rgb}{1.000000,0.894118,0.788235}%
\pgfsetstrokecolor{currentstroke}%
\pgfsetdash{}{0pt}%
\pgfpathmoveto{\pgfqpoint{3.717641in}{2.798097in}}%
\pgfpathlineto{\pgfqpoint{3.744588in}{2.813655in}}%
\pgfpathlineto{\pgfqpoint{3.744588in}{2.844771in}}%
\pgfpathlineto{\pgfqpoint{3.717641in}{2.829213in}}%
\pgfpathlineto{\pgfqpoint{3.717641in}{2.798097in}}%
\pgfpathclose%
\pgfusepath{fill}%
\end{pgfscope}%
\begin{pgfscope}%
\pgfpathrectangle{\pgfqpoint{0.765000in}{0.660000in}}{\pgfqpoint{4.620000in}{4.620000in}}%
\pgfusepath{clip}%
\pgfsetbuttcap%
\pgfsetroundjoin%
\definecolor{currentfill}{rgb}{1.000000,0.894118,0.788235}%
\pgfsetfillcolor{currentfill}%
\pgfsetlinewidth{0.000000pt}%
\definecolor{currentstroke}{rgb}{1.000000,0.894118,0.788235}%
\pgfsetstrokecolor{currentstroke}%
\pgfsetdash{}{0pt}%
\pgfpathmoveto{\pgfqpoint{3.690694in}{2.813655in}}%
\pgfpathlineto{\pgfqpoint{3.717641in}{2.798097in}}%
\pgfpathlineto{\pgfqpoint{3.717641in}{2.829213in}}%
\pgfpathlineto{\pgfqpoint{3.690694in}{2.844771in}}%
\pgfpathlineto{\pgfqpoint{3.690694in}{2.813655in}}%
\pgfpathclose%
\pgfusepath{fill}%
\end{pgfscope}%
\begin{pgfscope}%
\pgfpathrectangle{\pgfqpoint{0.765000in}{0.660000in}}{\pgfqpoint{4.620000in}{4.620000in}}%
\pgfusepath{clip}%
\pgfsetbuttcap%
\pgfsetroundjoin%
\definecolor{currentfill}{rgb}{1.000000,0.894118,0.788235}%
\pgfsetfillcolor{currentfill}%
\pgfsetlinewidth{0.000000pt}%
\definecolor{currentstroke}{rgb}{1.000000,0.894118,0.788235}%
\pgfsetstrokecolor{currentstroke}%
\pgfsetdash{}{0pt}%
\pgfpathmoveto{\pgfqpoint{3.712800in}{2.830079in}}%
\pgfpathlineto{\pgfqpoint{3.685852in}{2.814521in}}%
\pgfpathlineto{\pgfqpoint{3.690694in}{2.813655in}}%
\pgfpathlineto{\pgfqpoint{3.717641in}{2.829213in}}%
\pgfpathlineto{\pgfqpoint{3.712800in}{2.830079in}}%
\pgfpathclose%
\pgfusepath{fill}%
\end{pgfscope}%
\begin{pgfscope}%
\pgfpathrectangle{\pgfqpoint{0.765000in}{0.660000in}}{\pgfqpoint{4.620000in}{4.620000in}}%
\pgfusepath{clip}%
\pgfsetbuttcap%
\pgfsetroundjoin%
\definecolor{currentfill}{rgb}{1.000000,0.894118,0.788235}%
\pgfsetfillcolor{currentfill}%
\pgfsetlinewidth{0.000000pt}%
\definecolor{currentstroke}{rgb}{1.000000,0.894118,0.788235}%
\pgfsetstrokecolor{currentstroke}%
\pgfsetdash{}{0pt}%
\pgfpathmoveto{\pgfqpoint{3.739747in}{2.814521in}}%
\pgfpathlineto{\pgfqpoint{3.712800in}{2.830079in}}%
\pgfpathlineto{\pgfqpoint{3.717641in}{2.829213in}}%
\pgfpathlineto{\pgfqpoint{3.744588in}{2.813655in}}%
\pgfpathlineto{\pgfqpoint{3.739747in}{2.814521in}}%
\pgfpathclose%
\pgfusepath{fill}%
\end{pgfscope}%
\begin{pgfscope}%
\pgfpathrectangle{\pgfqpoint{0.765000in}{0.660000in}}{\pgfqpoint{4.620000in}{4.620000in}}%
\pgfusepath{clip}%
\pgfsetbuttcap%
\pgfsetroundjoin%
\definecolor{currentfill}{rgb}{1.000000,0.894118,0.788235}%
\pgfsetfillcolor{currentfill}%
\pgfsetlinewidth{0.000000pt}%
\definecolor{currentstroke}{rgb}{1.000000,0.894118,0.788235}%
\pgfsetstrokecolor{currentstroke}%
\pgfsetdash{}{0pt}%
\pgfpathmoveto{\pgfqpoint{3.712800in}{2.830079in}}%
\pgfpathlineto{\pgfqpoint{3.712800in}{2.861194in}}%
\pgfpathlineto{\pgfqpoint{3.717641in}{2.860329in}}%
\pgfpathlineto{\pgfqpoint{3.744588in}{2.813655in}}%
\pgfpathlineto{\pgfqpoint{3.712800in}{2.830079in}}%
\pgfpathclose%
\pgfusepath{fill}%
\end{pgfscope}%
\begin{pgfscope}%
\pgfpathrectangle{\pgfqpoint{0.765000in}{0.660000in}}{\pgfqpoint{4.620000in}{4.620000in}}%
\pgfusepath{clip}%
\pgfsetbuttcap%
\pgfsetroundjoin%
\definecolor{currentfill}{rgb}{1.000000,0.894118,0.788235}%
\pgfsetfillcolor{currentfill}%
\pgfsetlinewidth{0.000000pt}%
\definecolor{currentstroke}{rgb}{1.000000,0.894118,0.788235}%
\pgfsetstrokecolor{currentstroke}%
\pgfsetdash{}{0pt}%
\pgfpathmoveto{\pgfqpoint{3.739747in}{2.814521in}}%
\pgfpathlineto{\pgfqpoint{3.739747in}{2.845636in}}%
\pgfpathlineto{\pgfqpoint{3.744588in}{2.844771in}}%
\pgfpathlineto{\pgfqpoint{3.717641in}{2.829213in}}%
\pgfpathlineto{\pgfqpoint{3.739747in}{2.814521in}}%
\pgfpathclose%
\pgfusepath{fill}%
\end{pgfscope}%
\begin{pgfscope}%
\pgfpathrectangle{\pgfqpoint{0.765000in}{0.660000in}}{\pgfqpoint{4.620000in}{4.620000in}}%
\pgfusepath{clip}%
\pgfsetbuttcap%
\pgfsetroundjoin%
\definecolor{currentfill}{rgb}{1.000000,0.894118,0.788235}%
\pgfsetfillcolor{currentfill}%
\pgfsetlinewidth{0.000000pt}%
\definecolor{currentstroke}{rgb}{1.000000,0.894118,0.788235}%
\pgfsetstrokecolor{currentstroke}%
\pgfsetdash{}{0pt}%
\pgfpathmoveto{\pgfqpoint{3.685852in}{2.814521in}}%
\pgfpathlineto{\pgfqpoint{3.712800in}{2.798963in}}%
\pgfpathlineto{\pgfqpoint{3.717641in}{2.798097in}}%
\pgfpathlineto{\pgfqpoint{3.690694in}{2.813655in}}%
\pgfpathlineto{\pgfqpoint{3.685852in}{2.814521in}}%
\pgfpathclose%
\pgfusepath{fill}%
\end{pgfscope}%
\begin{pgfscope}%
\pgfpathrectangle{\pgfqpoint{0.765000in}{0.660000in}}{\pgfqpoint{4.620000in}{4.620000in}}%
\pgfusepath{clip}%
\pgfsetbuttcap%
\pgfsetroundjoin%
\definecolor{currentfill}{rgb}{1.000000,0.894118,0.788235}%
\pgfsetfillcolor{currentfill}%
\pgfsetlinewidth{0.000000pt}%
\definecolor{currentstroke}{rgb}{1.000000,0.894118,0.788235}%
\pgfsetstrokecolor{currentstroke}%
\pgfsetdash{}{0pt}%
\pgfpathmoveto{\pgfqpoint{3.712800in}{2.798963in}}%
\pgfpathlineto{\pgfqpoint{3.739747in}{2.814521in}}%
\pgfpathlineto{\pgfqpoint{3.744588in}{2.813655in}}%
\pgfpathlineto{\pgfqpoint{3.717641in}{2.798097in}}%
\pgfpathlineto{\pgfqpoint{3.712800in}{2.798963in}}%
\pgfpathclose%
\pgfusepath{fill}%
\end{pgfscope}%
\begin{pgfscope}%
\pgfpathrectangle{\pgfqpoint{0.765000in}{0.660000in}}{\pgfqpoint{4.620000in}{4.620000in}}%
\pgfusepath{clip}%
\pgfsetbuttcap%
\pgfsetroundjoin%
\definecolor{currentfill}{rgb}{1.000000,0.894118,0.788235}%
\pgfsetfillcolor{currentfill}%
\pgfsetlinewidth{0.000000pt}%
\definecolor{currentstroke}{rgb}{1.000000,0.894118,0.788235}%
\pgfsetstrokecolor{currentstroke}%
\pgfsetdash{}{0pt}%
\pgfpathmoveto{\pgfqpoint{3.712800in}{2.861194in}}%
\pgfpathlineto{\pgfqpoint{3.685852in}{2.845636in}}%
\pgfpathlineto{\pgfqpoint{3.690694in}{2.844771in}}%
\pgfpathlineto{\pgfqpoint{3.717641in}{2.860329in}}%
\pgfpathlineto{\pgfqpoint{3.712800in}{2.861194in}}%
\pgfpathclose%
\pgfusepath{fill}%
\end{pgfscope}%
\begin{pgfscope}%
\pgfpathrectangle{\pgfqpoint{0.765000in}{0.660000in}}{\pgfqpoint{4.620000in}{4.620000in}}%
\pgfusepath{clip}%
\pgfsetbuttcap%
\pgfsetroundjoin%
\definecolor{currentfill}{rgb}{1.000000,0.894118,0.788235}%
\pgfsetfillcolor{currentfill}%
\pgfsetlinewidth{0.000000pt}%
\definecolor{currentstroke}{rgb}{1.000000,0.894118,0.788235}%
\pgfsetstrokecolor{currentstroke}%
\pgfsetdash{}{0pt}%
\pgfpathmoveto{\pgfqpoint{3.739747in}{2.845636in}}%
\pgfpathlineto{\pgfqpoint{3.712800in}{2.861194in}}%
\pgfpathlineto{\pgfqpoint{3.717641in}{2.860329in}}%
\pgfpathlineto{\pgfqpoint{3.744588in}{2.844771in}}%
\pgfpathlineto{\pgfqpoint{3.739747in}{2.845636in}}%
\pgfpathclose%
\pgfusepath{fill}%
\end{pgfscope}%
\begin{pgfscope}%
\pgfpathrectangle{\pgfqpoint{0.765000in}{0.660000in}}{\pgfqpoint{4.620000in}{4.620000in}}%
\pgfusepath{clip}%
\pgfsetbuttcap%
\pgfsetroundjoin%
\definecolor{currentfill}{rgb}{1.000000,0.894118,0.788235}%
\pgfsetfillcolor{currentfill}%
\pgfsetlinewidth{0.000000pt}%
\definecolor{currentstroke}{rgb}{1.000000,0.894118,0.788235}%
\pgfsetstrokecolor{currentstroke}%
\pgfsetdash{}{0pt}%
\pgfpathmoveto{\pgfqpoint{3.685852in}{2.814521in}}%
\pgfpathlineto{\pgfqpoint{3.685852in}{2.845636in}}%
\pgfpathlineto{\pgfqpoint{3.690694in}{2.844771in}}%
\pgfpathlineto{\pgfqpoint{3.717641in}{2.798097in}}%
\pgfpathlineto{\pgfqpoint{3.685852in}{2.814521in}}%
\pgfpathclose%
\pgfusepath{fill}%
\end{pgfscope}%
\begin{pgfscope}%
\pgfpathrectangle{\pgfqpoint{0.765000in}{0.660000in}}{\pgfqpoint{4.620000in}{4.620000in}}%
\pgfusepath{clip}%
\pgfsetbuttcap%
\pgfsetroundjoin%
\definecolor{currentfill}{rgb}{1.000000,0.894118,0.788235}%
\pgfsetfillcolor{currentfill}%
\pgfsetlinewidth{0.000000pt}%
\definecolor{currentstroke}{rgb}{1.000000,0.894118,0.788235}%
\pgfsetstrokecolor{currentstroke}%
\pgfsetdash{}{0pt}%
\pgfpathmoveto{\pgfqpoint{3.712800in}{2.798963in}}%
\pgfpathlineto{\pgfqpoint{3.712800in}{2.830079in}}%
\pgfpathlineto{\pgfqpoint{3.717641in}{2.829213in}}%
\pgfpathlineto{\pgfqpoint{3.690694in}{2.813655in}}%
\pgfpathlineto{\pgfqpoint{3.712800in}{2.798963in}}%
\pgfpathclose%
\pgfusepath{fill}%
\end{pgfscope}%
\begin{pgfscope}%
\pgfpathrectangle{\pgfqpoint{0.765000in}{0.660000in}}{\pgfqpoint{4.620000in}{4.620000in}}%
\pgfusepath{clip}%
\pgfsetbuttcap%
\pgfsetroundjoin%
\definecolor{currentfill}{rgb}{1.000000,0.894118,0.788235}%
\pgfsetfillcolor{currentfill}%
\pgfsetlinewidth{0.000000pt}%
\definecolor{currentstroke}{rgb}{1.000000,0.894118,0.788235}%
\pgfsetstrokecolor{currentstroke}%
\pgfsetdash{}{0pt}%
\pgfpathmoveto{\pgfqpoint{3.712800in}{2.830079in}}%
\pgfpathlineto{\pgfqpoint{3.739747in}{2.845636in}}%
\pgfpathlineto{\pgfqpoint{3.744588in}{2.844771in}}%
\pgfpathlineto{\pgfqpoint{3.717641in}{2.829213in}}%
\pgfpathlineto{\pgfqpoint{3.712800in}{2.830079in}}%
\pgfpathclose%
\pgfusepath{fill}%
\end{pgfscope}%
\begin{pgfscope}%
\pgfpathrectangle{\pgfqpoint{0.765000in}{0.660000in}}{\pgfqpoint{4.620000in}{4.620000in}}%
\pgfusepath{clip}%
\pgfsetbuttcap%
\pgfsetroundjoin%
\definecolor{currentfill}{rgb}{1.000000,0.894118,0.788235}%
\pgfsetfillcolor{currentfill}%
\pgfsetlinewidth{0.000000pt}%
\definecolor{currentstroke}{rgb}{1.000000,0.894118,0.788235}%
\pgfsetstrokecolor{currentstroke}%
\pgfsetdash{}{0pt}%
\pgfpathmoveto{\pgfqpoint{3.685852in}{2.845636in}}%
\pgfpathlineto{\pgfqpoint{3.712800in}{2.830079in}}%
\pgfpathlineto{\pgfqpoint{3.717641in}{2.829213in}}%
\pgfpathlineto{\pgfqpoint{3.690694in}{2.844771in}}%
\pgfpathlineto{\pgfqpoint{3.685852in}{2.845636in}}%
\pgfpathclose%
\pgfusepath{fill}%
\end{pgfscope}%
\begin{pgfscope}%
\pgfpathrectangle{\pgfqpoint{0.765000in}{0.660000in}}{\pgfqpoint{4.620000in}{4.620000in}}%
\pgfusepath{clip}%
\pgfsetbuttcap%
\pgfsetroundjoin%
\definecolor{currentfill}{rgb}{1.000000,0.894118,0.788235}%
\pgfsetfillcolor{currentfill}%
\pgfsetlinewidth{0.000000pt}%
\definecolor{currentstroke}{rgb}{1.000000,0.894118,0.788235}%
\pgfsetstrokecolor{currentstroke}%
\pgfsetdash{}{0pt}%
\pgfpathmoveto{\pgfqpoint{3.706670in}{2.831422in}}%
\pgfpathlineto{\pgfqpoint{3.679722in}{2.815864in}}%
\pgfpathlineto{\pgfqpoint{3.706670in}{2.800306in}}%
\pgfpathlineto{\pgfqpoint{3.733617in}{2.815864in}}%
\pgfpathlineto{\pgfqpoint{3.706670in}{2.831422in}}%
\pgfpathclose%
\pgfusepath{fill}%
\end{pgfscope}%
\begin{pgfscope}%
\pgfpathrectangle{\pgfqpoint{0.765000in}{0.660000in}}{\pgfqpoint{4.620000in}{4.620000in}}%
\pgfusepath{clip}%
\pgfsetbuttcap%
\pgfsetroundjoin%
\definecolor{currentfill}{rgb}{1.000000,0.894118,0.788235}%
\pgfsetfillcolor{currentfill}%
\pgfsetlinewidth{0.000000pt}%
\definecolor{currentstroke}{rgb}{1.000000,0.894118,0.788235}%
\pgfsetstrokecolor{currentstroke}%
\pgfsetdash{}{0pt}%
\pgfpathmoveto{\pgfqpoint{3.706670in}{2.831422in}}%
\pgfpathlineto{\pgfqpoint{3.679722in}{2.815864in}}%
\pgfpathlineto{\pgfqpoint{3.679722in}{2.846980in}}%
\pgfpathlineto{\pgfqpoint{3.706670in}{2.862538in}}%
\pgfpathlineto{\pgfqpoint{3.706670in}{2.831422in}}%
\pgfpathclose%
\pgfusepath{fill}%
\end{pgfscope}%
\begin{pgfscope}%
\pgfpathrectangle{\pgfqpoint{0.765000in}{0.660000in}}{\pgfqpoint{4.620000in}{4.620000in}}%
\pgfusepath{clip}%
\pgfsetbuttcap%
\pgfsetroundjoin%
\definecolor{currentfill}{rgb}{1.000000,0.894118,0.788235}%
\pgfsetfillcolor{currentfill}%
\pgfsetlinewidth{0.000000pt}%
\definecolor{currentstroke}{rgb}{1.000000,0.894118,0.788235}%
\pgfsetstrokecolor{currentstroke}%
\pgfsetdash{}{0pt}%
\pgfpathmoveto{\pgfqpoint{3.706670in}{2.831422in}}%
\pgfpathlineto{\pgfqpoint{3.733617in}{2.815864in}}%
\pgfpathlineto{\pgfqpoint{3.733617in}{2.846980in}}%
\pgfpathlineto{\pgfqpoint{3.706670in}{2.862538in}}%
\pgfpathlineto{\pgfqpoint{3.706670in}{2.831422in}}%
\pgfpathclose%
\pgfusepath{fill}%
\end{pgfscope}%
\begin{pgfscope}%
\pgfpathrectangle{\pgfqpoint{0.765000in}{0.660000in}}{\pgfqpoint{4.620000in}{4.620000in}}%
\pgfusepath{clip}%
\pgfsetbuttcap%
\pgfsetroundjoin%
\definecolor{currentfill}{rgb}{1.000000,0.894118,0.788235}%
\pgfsetfillcolor{currentfill}%
\pgfsetlinewidth{0.000000pt}%
\definecolor{currentstroke}{rgb}{1.000000,0.894118,0.788235}%
\pgfsetstrokecolor{currentstroke}%
\pgfsetdash{}{0pt}%
\pgfpathmoveto{\pgfqpoint{3.693066in}{2.834300in}}%
\pgfpathlineto{\pgfqpoint{3.666119in}{2.818742in}}%
\pgfpathlineto{\pgfqpoint{3.693066in}{2.803184in}}%
\pgfpathlineto{\pgfqpoint{3.720013in}{2.818742in}}%
\pgfpathlineto{\pgfqpoint{3.693066in}{2.834300in}}%
\pgfpathclose%
\pgfusepath{fill}%
\end{pgfscope}%
\begin{pgfscope}%
\pgfpathrectangle{\pgfqpoint{0.765000in}{0.660000in}}{\pgfqpoint{4.620000in}{4.620000in}}%
\pgfusepath{clip}%
\pgfsetbuttcap%
\pgfsetroundjoin%
\definecolor{currentfill}{rgb}{1.000000,0.894118,0.788235}%
\pgfsetfillcolor{currentfill}%
\pgfsetlinewidth{0.000000pt}%
\definecolor{currentstroke}{rgb}{1.000000,0.894118,0.788235}%
\pgfsetstrokecolor{currentstroke}%
\pgfsetdash{}{0pt}%
\pgfpathmoveto{\pgfqpoint{3.693066in}{2.834300in}}%
\pgfpathlineto{\pgfqpoint{3.666119in}{2.818742in}}%
\pgfpathlineto{\pgfqpoint{3.666119in}{2.849858in}}%
\pgfpathlineto{\pgfqpoint{3.693066in}{2.865416in}}%
\pgfpathlineto{\pgfqpoint{3.693066in}{2.834300in}}%
\pgfpathclose%
\pgfusepath{fill}%
\end{pgfscope}%
\begin{pgfscope}%
\pgfpathrectangle{\pgfqpoint{0.765000in}{0.660000in}}{\pgfqpoint{4.620000in}{4.620000in}}%
\pgfusepath{clip}%
\pgfsetbuttcap%
\pgfsetroundjoin%
\definecolor{currentfill}{rgb}{1.000000,0.894118,0.788235}%
\pgfsetfillcolor{currentfill}%
\pgfsetlinewidth{0.000000pt}%
\definecolor{currentstroke}{rgb}{1.000000,0.894118,0.788235}%
\pgfsetstrokecolor{currentstroke}%
\pgfsetdash{}{0pt}%
\pgfpathmoveto{\pgfqpoint{3.693066in}{2.834300in}}%
\pgfpathlineto{\pgfqpoint{3.720013in}{2.818742in}}%
\pgfpathlineto{\pgfqpoint{3.720013in}{2.849858in}}%
\pgfpathlineto{\pgfqpoint{3.693066in}{2.865416in}}%
\pgfpathlineto{\pgfqpoint{3.693066in}{2.834300in}}%
\pgfpathclose%
\pgfusepath{fill}%
\end{pgfscope}%
\begin{pgfscope}%
\pgfpathrectangle{\pgfqpoint{0.765000in}{0.660000in}}{\pgfqpoint{4.620000in}{4.620000in}}%
\pgfusepath{clip}%
\pgfsetbuttcap%
\pgfsetroundjoin%
\definecolor{currentfill}{rgb}{1.000000,0.894118,0.788235}%
\pgfsetfillcolor{currentfill}%
\pgfsetlinewidth{0.000000pt}%
\definecolor{currentstroke}{rgb}{1.000000,0.894118,0.788235}%
\pgfsetstrokecolor{currentstroke}%
\pgfsetdash{}{0pt}%
\pgfpathmoveto{\pgfqpoint{3.706670in}{2.862538in}}%
\pgfpathlineto{\pgfqpoint{3.679722in}{2.846980in}}%
\pgfpathlineto{\pgfqpoint{3.706670in}{2.831422in}}%
\pgfpathlineto{\pgfqpoint{3.733617in}{2.846980in}}%
\pgfpathlineto{\pgfqpoint{3.706670in}{2.862538in}}%
\pgfpathclose%
\pgfusepath{fill}%
\end{pgfscope}%
\begin{pgfscope}%
\pgfpathrectangle{\pgfqpoint{0.765000in}{0.660000in}}{\pgfqpoint{4.620000in}{4.620000in}}%
\pgfusepath{clip}%
\pgfsetbuttcap%
\pgfsetroundjoin%
\definecolor{currentfill}{rgb}{1.000000,0.894118,0.788235}%
\pgfsetfillcolor{currentfill}%
\pgfsetlinewidth{0.000000pt}%
\definecolor{currentstroke}{rgb}{1.000000,0.894118,0.788235}%
\pgfsetstrokecolor{currentstroke}%
\pgfsetdash{}{0pt}%
\pgfpathmoveto{\pgfqpoint{3.706670in}{2.800306in}}%
\pgfpathlineto{\pgfqpoint{3.733617in}{2.815864in}}%
\pgfpathlineto{\pgfqpoint{3.733617in}{2.846980in}}%
\pgfpathlineto{\pgfqpoint{3.706670in}{2.831422in}}%
\pgfpathlineto{\pgfqpoint{3.706670in}{2.800306in}}%
\pgfpathclose%
\pgfusepath{fill}%
\end{pgfscope}%
\begin{pgfscope}%
\pgfpathrectangle{\pgfqpoint{0.765000in}{0.660000in}}{\pgfqpoint{4.620000in}{4.620000in}}%
\pgfusepath{clip}%
\pgfsetbuttcap%
\pgfsetroundjoin%
\definecolor{currentfill}{rgb}{1.000000,0.894118,0.788235}%
\pgfsetfillcolor{currentfill}%
\pgfsetlinewidth{0.000000pt}%
\definecolor{currentstroke}{rgb}{1.000000,0.894118,0.788235}%
\pgfsetstrokecolor{currentstroke}%
\pgfsetdash{}{0pt}%
\pgfpathmoveto{\pgfqpoint{3.679722in}{2.815864in}}%
\pgfpathlineto{\pgfqpoint{3.706670in}{2.800306in}}%
\pgfpathlineto{\pgfqpoint{3.706670in}{2.831422in}}%
\pgfpathlineto{\pgfqpoint{3.679722in}{2.846980in}}%
\pgfpathlineto{\pgfqpoint{3.679722in}{2.815864in}}%
\pgfpathclose%
\pgfusepath{fill}%
\end{pgfscope}%
\begin{pgfscope}%
\pgfpathrectangle{\pgfqpoint{0.765000in}{0.660000in}}{\pgfqpoint{4.620000in}{4.620000in}}%
\pgfusepath{clip}%
\pgfsetbuttcap%
\pgfsetroundjoin%
\definecolor{currentfill}{rgb}{1.000000,0.894118,0.788235}%
\pgfsetfillcolor{currentfill}%
\pgfsetlinewidth{0.000000pt}%
\definecolor{currentstroke}{rgb}{1.000000,0.894118,0.788235}%
\pgfsetstrokecolor{currentstroke}%
\pgfsetdash{}{0pt}%
\pgfpathmoveto{\pgfqpoint{3.693066in}{2.865416in}}%
\pgfpathlineto{\pgfqpoint{3.666119in}{2.849858in}}%
\pgfpathlineto{\pgfqpoint{3.693066in}{2.834300in}}%
\pgfpathlineto{\pgfqpoint{3.720013in}{2.849858in}}%
\pgfpathlineto{\pgfqpoint{3.693066in}{2.865416in}}%
\pgfpathclose%
\pgfusepath{fill}%
\end{pgfscope}%
\begin{pgfscope}%
\pgfpathrectangle{\pgfqpoint{0.765000in}{0.660000in}}{\pgfqpoint{4.620000in}{4.620000in}}%
\pgfusepath{clip}%
\pgfsetbuttcap%
\pgfsetroundjoin%
\definecolor{currentfill}{rgb}{1.000000,0.894118,0.788235}%
\pgfsetfillcolor{currentfill}%
\pgfsetlinewidth{0.000000pt}%
\definecolor{currentstroke}{rgb}{1.000000,0.894118,0.788235}%
\pgfsetstrokecolor{currentstroke}%
\pgfsetdash{}{0pt}%
\pgfpathmoveto{\pgfqpoint{3.693066in}{2.803184in}}%
\pgfpathlineto{\pgfqpoint{3.720013in}{2.818742in}}%
\pgfpathlineto{\pgfqpoint{3.720013in}{2.849858in}}%
\pgfpathlineto{\pgfqpoint{3.693066in}{2.834300in}}%
\pgfpathlineto{\pgfqpoint{3.693066in}{2.803184in}}%
\pgfpathclose%
\pgfusepath{fill}%
\end{pgfscope}%
\begin{pgfscope}%
\pgfpathrectangle{\pgfqpoint{0.765000in}{0.660000in}}{\pgfqpoint{4.620000in}{4.620000in}}%
\pgfusepath{clip}%
\pgfsetbuttcap%
\pgfsetroundjoin%
\definecolor{currentfill}{rgb}{1.000000,0.894118,0.788235}%
\pgfsetfillcolor{currentfill}%
\pgfsetlinewidth{0.000000pt}%
\definecolor{currentstroke}{rgb}{1.000000,0.894118,0.788235}%
\pgfsetstrokecolor{currentstroke}%
\pgfsetdash{}{0pt}%
\pgfpathmoveto{\pgfqpoint{3.666119in}{2.818742in}}%
\pgfpathlineto{\pgfqpoint{3.693066in}{2.803184in}}%
\pgfpathlineto{\pgfqpoint{3.693066in}{2.834300in}}%
\pgfpathlineto{\pgfqpoint{3.666119in}{2.849858in}}%
\pgfpathlineto{\pgfqpoint{3.666119in}{2.818742in}}%
\pgfpathclose%
\pgfusepath{fill}%
\end{pgfscope}%
\begin{pgfscope}%
\pgfpathrectangle{\pgfqpoint{0.765000in}{0.660000in}}{\pgfqpoint{4.620000in}{4.620000in}}%
\pgfusepath{clip}%
\pgfsetbuttcap%
\pgfsetroundjoin%
\definecolor{currentfill}{rgb}{1.000000,0.894118,0.788235}%
\pgfsetfillcolor{currentfill}%
\pgfsetlinewidth{0.000000pt}%
\definecolor{currentstroke}{rgb}{1.000000,0.894118,0.788235}%
\pgfsetstrokecolor{currentstroke}%
\pgfsetdash{}{0pt}%
\pgfpathmoveto{\pgfqpoint{3.706670in}{2.831422in}}%
\pgfpathlineto{\pgfqpoint{3.679722in}{2.815864in}}%
\pgfpathlineto{\pgfqpoint{3.666119in}{2.818742in}}%
\pgfpathlineto{\pgfqpoint{3.693066in}{2.834300in}}%
\pgfpathlineto{\pgfqpoint{3.706670in}{2.831422in}}%
\pgfpathclose%
\pgfusepath{fill}%
\end{pgfscope}%
\begin{pgfscope}%
\pgfpathrectangle{\pgfqpoint{0.765000in}{0.660000in}}{\pgfqpoint{4.620000in}{4.620000in}}%
\pgfusepath{clip}%
\pgfsetbuttcap%
\pgfsetroundjoin%
\definecolor{currentfill}{rgb}{1.000000,0.894118,0.788235}%
\pgfsetfillcolor{currentfill}%
\pgfsetlinewidth{0.000000pt}%
\definecolor{currentstroke}{rgb}{1.000000,0.894118,0.788235}%
\pgfsetstrokecolor{currentstroke}%
\pgfsetdash{}{0pt}%
\pgfpathmoveto{\pgfqpoint{3.733617in}{2.815864in}}%
\pgfpathlineto{\pgfqpoint{3.706670in}{2.831422in}}%
\pgfpathlineto{\pgfqpoint{3.693066in}{2.834300in}}%
\pgfpathlineto{\pgfqpoint{3.720013in}{2.818742in}}%
\pgfpathlineto{\pgfqpoint{3.733617in}{2.815864in}}%
\pgfpathclose%
\pgfusepath{fill}%
\end{pgfscope}%
\begin{pgfscope}%
\pgfpathrectangle{\pgfqpoint{0.765000in}{0.660000in}}{\pgfqpoint{4.620000in}{4.620000in}}%
\pgfusepath{clip}%
\pgfsetbuttcap%
\pgfsetroundjoin%
\definecolor{currentfill}{rgb}{1.000000,0.894118,0.788235}%
\pgfsetfillcolor{currentfill}%
\pgfsetlinewidth{0.000000pt}%
\definecolor{currentstroke}{rgb}{1.000000,0.894118,0.788235}%
\pgfsetstrokecolor{currentstroke}%
\pgfsetdash{}{0pt}%
\pgfpathmoveto{\pgfqpoint{3.706670in}{2.831422in}}%
\pgfpathlineto{\pgfqpoint{3.706670in}{2.862538in}}%
\pgfpathlineto{\pgfqpoint{3.693066in}{2.865416in}}%
\pgfpathlineto{\pgfqpoint{3.720013in}{2.818742in}}%
\pgfpathlineto{\pgfqpoint{3.706670in}{2.831422in}}%
\pgfpathclose%
\pgfusepath{fill}%
\end{pgfscope}%
\begin{pgfscope}%
\pgfpathrectangle{\pgfqpoint{0.765000in}{0.660000in}}{\pgfqpoint{4.620000in}{4.620000in}}%
\pgfusepath{clip}%
\pgfsetbuttcap%
\pgfsetroundjoin%
\definecolor{currentfill}{rgb}{1.000000,0.894118,0.788235}%
\pgfsetfillcolor{currentfill}%
\pgfsetlinewidth{0.000000pt}%
\definecolor{currentstroke}{rgb}{1.000000,0.894118,0.788235}%
\pgfsetstrokecolor{currentstroke}%
\pgfsetdash{}{0pt}%
\pgfpathmoveto{\pgfqpoint{3.733617in}{2.815864in}}%
\pgfpathlineto{\pgfqpoint{3.733617in}{2.846980in}}%
\pgfpathlineto{\pgfqpoint{3.720013in}{2.849858in}}%
\pgfpathlineto{\pgfqpoint{3.693066in}{2.834300in}}%
\pgfpathlineto{\pgfqpoint{3.733617in}{2.815864in}}%
\pgfpathclose%
\pgfusepath{fill}%
\end{pgfscope}%
\begin{pgfscope}%
\pgfpathrectangle{\pgfqpoint{0.765000in}{0.660000in}}{\pgfqpoint{4.620000in}{4.620000in}}%
\pgfusepath{clip}%
\pgfsetbuttcap%
\pgfsetroundjoin%
\definecolor{currentfill}{rgb}{1.000000,0.894118,0.788235}%
\pgfsetfillcolor{currentfill}%
\pgfsetlinewidth{0.000000pt}%
\definecolor{currentstroke}{rgb}{1.000000,0.894118,0.788235}%
\pgfsetstrokecolor{currentstroke}%
\pgfsetdash{}{0pt}%
\pgfpathmoveto{\pgfqpoint{3.679722in}{2.815864in}}%
\pgfpathlineto{\pgfqpoint{3.706670in}{2.800306in}}%
\pgfpathlineto{\pgfqpoint{3.693066in}{2.803184in}}%
\pgfpathlineto{\pgfqpoint{3.666119in}{2.818742in}}%
\pgfpathlineto{\pgfqpoint{3.679722in}{2.815864in}}%
\pgfpathclose%
\pgfusepath{fill}%
\end{pgfscope}%
\begin{pgfscope}%
\pgfpathrectangle{\pgfqpoint{0.765000in}{0.660000in}}{\pgfqpoint{4.620000in}{4.620000in}}%
\pgfusepath{clip}%
\pgfsetbuttcap%
\pgfsetroundjoin%
\definecolor{currentfill}{rgb}{1.000000,0.894118,0.788235}%
\pgfsetfillcolor{currentfill}%
\pgfsetlinewidth{0.000000pt}%
\definecolor{currentstroke}{rgb}{1.000000,0.894118,0.788235}%
\pgfsetstrokecolor{currentstroke}%
\pgfsetdash{}{0pt}%
\pgfpathmoveto{\pgfqpoint{3.706670in}{2.800306in}}%
\pgfpathlineto{\pgfqpoint{3.733617in}{2.815864in}}%
\pgfpathlineto{\pgfqpoint{3.720013in}{2.818742in}}%
\pgfpathlineto{\pgfqpoint{3.693066in}{2.803184in}}%
\pgfpathlineto{\pgfqpoint{3.706670in}{2.800306in}}%
\pgfpathclose%
\pgfusepath{fill}%
\end{pgfscope}%
\begin{pgfscope}%
\pgfpathrectangle{\pgfqpoint{0.765000in}{0.660000in}}{\pgfqpoint{4.620000in}{4.620000in}}%
\pgfusepath{clip}%
\pgfsetbuttcap%
\pgfsetroundjoin%
\definecolor{currentfill}{rgb}{1.000000,0.894118,0.788235}%
\pgfsetfillcolor{currentfill}%
\pgfsetlinewidth{0.000000pt}%
\definecolor{currentstroke}{rgb}{1.000000,0.894118,0.788235}%
\pgfsetstrokecolor{currentstroke}%
\pgfsetdash{}{0pt}%
\pgfpathmoveto{\pgfqpoint{3.706670in}{2.862538in}}%
\pgfpathlineto{\pgfqpoint{3.679722in}{2.846980in}}%
\pgfpathlineto{\pgfqpoint{3.666119in}{2.849858in}}%
\pgfpathlineto{\pgfqpoint{3.693066in}{2.865416in}}%
\pgfpathlineto{\pgfqpoint{3.706670in}{2.862538in}}%
\pgfpathclose%
\pgfusepath{fill}%
\end{pgfscope}%
\begin{pgfscope}%
\pgfpathrectangle{\pgfqpoint{0.765000in}{0.660000in}}{\pgfqpoint{4.620000in}{4.620000in}}%
\pgfusepath{clip}%
\pgfsetbuttcap%
\pgfsetroundjoin%
\definecolor{currentfill}{rgb}{1.000000,0.894118,0.788235}%
\pgfsetfillcolor{currentfill}%
\pgfsetlinewidth{0.000000pt}%
\definecolor{currentstroke}{rgb}{1.000000,0.894118,0.788235}%
\pgfsetstrokecolor{currentstroke}%
\pgfsetdash{}{0pt}%
\pgfpathmoveto{\pgfqpoint{3.733617in}{2.846980in}}%
\pgfpathlineto{\pgfqpoint{3.706670in}{2.862538in}}%
\pgfpathlineto{\pgfqpoint{3.693066in}{2.865416in}}%
\pgfpathlineto{\pgfqpoint{3.720013in}{2.849858in}}%
\pgfpathlineto{\pgfqpoint{3.733617in}{2.846980in}}%
\pgfpathclose%
\pgfusepath{fill}%
\end{pgfscope}%
\begin{pgfscope}%
\pgfpathrectangle{\pgfqpoint{0.765000in}{0.660000in}}{\pgfqpoint{4.620000in}{4.620000in}}%
\pgfusepath{clip}%
\pgfsetbuttcap%
\pgfsetroundjoin%
\definecolor{currentfill}{rgb}{1.000000,0.894118,0.788235}%
\pgfsetfillcolor{currentfill}%
\pgfsetlinewidth{0.000000pt}%
\definecolor{currentstroke}{rgb}{1.000000,0.894118,0.788235}%
\pgfsetstrokecolor{currentstroke}%
\pgfsetdash{}{0pt}%
\pgfpathmoveto{\pgfqpoint{3.679722in}{2.815864in}}%
\pgfpathlineto{\pgfqpoint{3.679722in}{2.846980in}}%
\pgfpathlineto{\pgfqpoint{3.666119in}{2.849858in}}%
\pgfpathlineto{\pgfqpoint{3.693066in}{2.803184in}}%
\pgfpathlineto{\pgfqpoint{3.679722in}{2.815864in}}%
\pgfpathclose%
\pgfusepath{fill}%
\end{pgfscope}%
\begin{pgfscope}%
\pgfpathrectangle{\pgfqpoint{0.765000in}{0.660000in}}{\pgfqpoint{4.620000in}{4.620000in}}%
\pgfusepath{clip}%
\pgfsetbuttcap%
\pgfsetroundjoin%
\definecolor{currentfill}{rgb}{1.000000,0.894118,0.788235}%
\pgfsetfillcolor{currentfill}%
\pgfsetlinewidth{0.000000pt}%
\definecolor{currentstroke}{rgb}{1.000000,0.894118,0.788235}%
\pgfsetstrokecolor{currentstroke}%
\pgfsetdash{}{0pt}%
\pgfpathmoveto{\pgfqpoint{3.706670in}{2.800306in}}%
\pgfpathlineto{\pgfqpoint{3.706670in}{2.831422in}}%
\pgfpathlineto{\pgfqpoint{3.693066in}{2.834300in}}%
\pgfpathlineto{\pgfqpoint{3.666119in}{2.818742in}}%
\pgfpathlineto{\pgfqpoint{3.706670in}{2.800306in}}%
\pgfpathclose%
\pgfusepath{fill}%
\end{pgfscope}%
\begin{pgfscope}%
\pgfpathrectangle{\pgfqpoint{0.765000in}{0.660000in}}{\pgfqpoint{4.620000in}{4.620000in}}%
\pgfusepath{clip}%
\pgfsetbuttcap%
\pgfsetroundjoin%
\definecolor{currentfill}{rgb}{1.000000,0.894118,0.788235}%
\pgfsetfillcolor{currentfill}%
\pgfsetlinewidth{0.000000pt}%
\definecolor{currentstroke}{rgb}{1.000000,0.894118,0.788235}%
\pgfsetstrokecolor{currentstroke}%
\pgfsetdash{}{0pt}%
\pgfpathmoveto{\pgfqpoint{3.706670in}{2.831422in}}%
\pgfpathlineto{\pgfqpoint{3.733617in}{2.846980in}}%
\pgfpathlineto{\pgfqpoint{3.720013in}{2.849858in}}%
\pgfpathlineto{\pgfqpoint{3.693066in}{2.834300in}}%
\pgfpathlineto{\pgfqpoint{3.706670in}{2.831422in}}%
\pgfpathclose%
\pgfusepath{fill}%
\end{pgfscope}%
\begin{pgfscope}%
\pgfpathrectangle{\pgfqpoint{0.765000in}{0.660000in}}{\pgfqpoint{4.620000in}{4.620000in}}%
\pgfusepath{clip}%
\pgfsetbuttcap%
\pgfsetroundjoin%
\definecolor{currentfill}{rgb}{1.000000,0.894118,0.788235}%
\pgfsetfillcolor{currentfill}%
\pgfsetlinewidth{0.000000pt}%
\definecolor{currentstroke}{rgb}{1.000000,0.894118,0.788235}%
\pgfsetstrokecolor{currentstroke}%
\pgfsetdash{}{0pt}%
\pgfpathmoveto{\pgfqpoint{3.679722in}{2.846980in}}%
\pgfpathlineto{\pgfqpoint{3.706670in}{2.831422in}}%
\pgfpathlineto{\pgfqpoint{3.693066in}{2.834300in}}%
\pgfpathlineto{\pgfqpoint{3.666119in}{2.849858in}}%
\pgfpathlineto{\pgfqpoint{3.679722in}{2.846980in}}%
\pgfpathclose%
\pgfusepath{fill}%
\end{pgfscope}%
\begin{pgfscope}%
\pgfpathrectangle{\pgfqpoint{0.765000in}{0.660000in}}{\pgfqpoint{4.620000in}{4.620000in}}%
\pgfusepath{clip}%
\pgfsetbuttcap%
\pgfsetroundjoin%
\definecolor{currentfill}{rgb}{1.000000,0.894118,0.788235}%
\pgfsetfillcolor{currentfill}%
\pgfsetlinewidth{0.000000pt}%
\definecolor{currentstroke}{rgb}{1.000000,0.894118,0.788235}%
\pgfsetstrokecolor{currentstroke}%
\pgfsetdash{}{0pt}%
\pgfpathmoveto{\pgfqpoint{3.717641in}{2.829213in}}%
\pgfpathlineto{\pgfqpoint{3.690694in}{2.813655in}}%
\pgfpathlineto{\pgfqpoint{3.717641in}{2.798097in}}%
\pgfpathlineto{\pgfqpoint{3.744588in}{2.813655in}}%
\pgfpathlineto{\pgfqpoint{3.717641in}{2.829213in}}%
\pgfpathclose%
\pgfusepath{fill}%
\end{pgfscope}%
\begin{pgfscope}%
\pgfpathrectangle{\pgfqpoint{0.765000in}{0.660000in}}{\pgfqpoint{4.620000in}{4.620000in}}%
\pgfusepath{clip}%
\pgfsetbuttcap%
\pgfsetroundjoin%
\definecolor{currentfill}{rgb}{1.000000,0.894118,0.788235}%
\pgfsetfillcolor{currentfill}%
\pgfsetlinewidth{0.000000pt}%
\definecolor{currentstroke}{rgb}{1.000000,0.894118,0.788235}%
\pgfsetstrokecolor{currentstroke}%
\pgfsetdash{}{0pt}%
\pgfpathmoveto{\pgfqpoint{3.717641in}{2.829213in}}%
\pgfpathlineto{\pgfqpoint{3.690694in}{2.813655in}}%
\pgfpathlineto{\pgfqpoint{3.690694in}{2.844771in}}%
\pgfpathlineto{\pgfqpoint{3.717641in}{2.860329in}}%
\pgfpathlineto{\pgfqpoint{3.717641in}{2.829213in}}%
\pgfpathclose%
\pgfusepath{fill}%
\end{pgfscope}%
\begin{pgfscope}%
\pgfpathrectangle{\pgfqpoint{0.765000in}{0.660000in}}{\pgfqpoint{4.620000in}{4.620000in}}%
\pgfusepath{clip}%
\pgfsetbuttcap%
\pgfsetroundjoin%
\definecolor{currentfill}{rgb}{1.000000,0.894118,0.788235}%
\pgfsetfillcolor{currentfill}%
\pgfsetlinewidth{0.000000pt}%
\definecolor{currentstroke}{rgb}{1.000000,0.894118,0.788235}%
\pgfsetstrokecolor{currentstroke}%
\pgfsetdash{}{0pt}%
\pgfpathmoveto{\pgfqpoint{3.717641in}{2.829213in}}%
\pgfpathlineto{\pgfqpoint{3.744588in}{2.813655in}}%
\pgfpathlineto{\pgfqpoint{3.744588in}{2.844771in}}%
\pgfpathlineto{\pgfqpoint{3.717641in}{2.860329in}}%
\pgfpathlineto{\pgfqpoint{3.717641in}{2.829213in}}%
\pgfpathclose%
\pgfusepath{fill}%
\end{pgfscope}%
\begin{pgfscope}%
\pgfpathrectangle{\pgfqpoint{0.765000in}{0.660000in}}{\pgfqpoint{4.620000in}{4.620000in}}%
\pgfusepath{clip}%
\pgfsetbuttcap%
\pgfsetroundjoin%
\definecolor{currentfill}{rgb}{1.000000,0.894118,0.788235}%
\pgfsetfillcolor{currentfill}%
\pgfsetlinewidth{0.000000pt}%
\definecolor{currentstroke}{rgb}{1.000000,0.894118,0.788235}%
\pgfsetstrokecolor{currentstroke}%
\pgfsetdash{}{0pt}%
\pgfpathmoveto{\pgfqpoint{3.755404in}{2.825971in}}%
\pgfpathlineto{\pgfqpoint{3.728456in}{2.810413in}}%
\pgfpathlineto{\pgfqpoint{3.755404in}{2.794855in}}%
\pgfpathlineto{\pgfqpoint{3.782351in}{2.810413in}}%
\pgfpathlineto{\pgfqpoint{3.755404in}{2.825971in}}%
\pgfpathclose%
\pgfusepath{fill}%
\end{pgfscope}%
\begin{pgfscope}%
\pgfpathrectangle{\pgfqpoint{0.765000in}{0.660000in}}{\pgfqpoint{4.620000in}{4.620000in}}%
\pgfusepath{clip}%
\pgfsetbuttcap%
\pgfsetroundjoin%
\definecolor{currentfill}{rgb}{1.000000,0.894118,0.788235}%
\pgfsetfillcolor{currentfill}%
\pgfsetlinewidth{0.000000pt}%
\definecolor{currentstroke}{rgb}{1.000000,0.894118,0.788235}%
\pgfsetstrokecolor{currentstroke}%
\pgfsetdash{}{0pt}%
\pgfpathmoveto{\pgfqpoint{3.755404in}{2.825971in}}%
\pgfpathlineto{\pgfqpoint{3.728456in}{2.810413in}}%
\pgfpathlineto{\pgfqpoint{3.728456in}{2.841529in}}%
\pgfpathlineto{\pgfqpoint{3.755404in}{2.857087in}}%
\pgfpathlineto{\pgfqpoint{3.755404in}{2.825971in}}%
\pgfpathclose%
\pgfusepath{fill}%
\end{pgfscope}%
\begin{pgfscope}%
\pgfpathrectangle{\pgfqpoint{0.765000in}{0.660000in}}{\pgfqpoint{4.620000in}{4.620000in}}%
\pgfusepath{clip}%
\pgfsetbuttcap%
\pgfsetroundjoin%
\definecolor{currentfill}{rgb}{1.000000,0.894118,0.788235}%
\pgfsetfillcolor{currentfill}%
\pgfsetlinewidth{0.000000pt}%
\definecolor{currentstroke}{rgb}{1.000000,0.894118,0.788235}%
\pgfsetstrokecolor{currentstroke}%
\pgfsetdash{}{0pt}%
\pgfpathmoveto{\pgfqpoint{3.755404in}{2.825971in}}%
\pgfpathlineto{\pgfqpoint{3.782351in}{2.810413in}}%
\pgfpathlineto{\pgfqpoint{3.782351in}{2.841529in}}%
\pgfpathlineto{\pgfqpoint{3.755404in}{2.857087in}}%
\pgfpathlineto{\pgfqpoint{3.755404in}{2.825971in}}%
\pgfpathclose%
\pgfusepath{fill}%
\end{pgfscope}%
\begin{pgfscope}%
\pgfpathrectangle{\pgfqpoint{0.765000in}{0.660000in}}{\pgfqpoint{4.620000in}{4.620000in}}%
\pgfusepath{clip}%
\pgfsetbuttcap%
\pgfsetroundjoin%
\definecolor{currentfill}{rgb}{1.000000,0.894118,0.788235}%
\pgfsetfillcolor{currentfill}%
\pgfsetlinewidth{0.000000pt}%
\definecolor{currentstroke}{rgb}{1.000000,0.894118,0.788235}%
\pgfsetstrokecolor{currentstroke}%
\pgfsetdash{}{0pt}%
\pgfpathmoveto{\pgfqpoint{3.717641in}{2.860329in}}%
\pgfpathlineto{\pgfqpoint{3.690694in}{2.844771in}}%
\pgfpathlineto{\pgfqpoint{3.717641in}{2.829213in}}%
\pgfpathlineto{\pgfqpoint{3.744588in}{2.844771in}}%
\pgfpathlineto{\pgfqpoint{3.717641in}{2.860329in}}%
\pgfpathclose%
\pgfusepath{fill}%
\end{pgfscope}%
\begin{pgfscope}%
\pgfpathrectangle{\pgfqpoint{0.765000in}{0.660000in}}{\pgfqpoint{4.620000in}{4.620000in}}%
\pgfusepath{clip}%
\pgfsetbuttcap%
\pgfsetroundjoin%
\definecolor{currentfill}{rgb}{1.000000,0.894118,0.788235}%
\pgfsetfillcolor{currentfill}%
\pgfsetlinewidth{0.000000pt}%
\definecolor{currentstroke}{rgb}{1.000000,0.894118,0.788235}%
\pgfsetstrokecolor{currentstroke}%
\pgfsetdash{}{0pt}%
\pgfpathmoveto{\pgfqpoint{3.717641in}{2.798097in}}%
\pgfpathlineto{\pgfqpoint{3.744588in}{2.813655in}}%
\pgfpathlineto{\pgfqpoint{3.744588in}{2.844771in}}%
\pgfpathlineto{\pgfqpoint{3.717641in}{2.829213in}}%
\pgfpathlineto{\pgfqpoint{3.717641in}{2.798097in}}%
\pgfpathclose%
\pgfusepath{fill}%
\end{pgfscope}%
\begin{pgfscope}%
\pgfpathrectangle{\pgfqpoint{0.765000in}{0.660000in}}{\pgfqpoint{4.620000in}{4.620000in}}%
\pgfusepath{clip}%
\pgfsetbuttcap%
\pgfsetroundjoin%
\definecolor{currentfill}{rgb}{1.000000,0.894118,0.788235}%
\pgfsetfillcolor{currentfill}%
\pgfsetlinewidth{0.000000pt}%
\definecolor{currentstroke}{rgb}{1.000000,0.894118,0.788235}%
\pgfsetstrokecolor{currentstroke}%
\pgfsetdash{}{0pt}%
\pgfpathmoveto{\pgfqpoint{3.690694in}{2.813655in}}%
\pgfpathlineto{\pgfqpoint{3.717641in}{2.798097in}}%
\pgfpathlineto{\pgfqpoint{3.717641in}{2.829213in}}%
\pgfpathlineto{\pgfqpoint{3.690694in}{2.844771in}}%
\pgfpathlineto{\pgfqpoint{3.690694in}{2.813655in}}%
\pgfpathclose%
\pgfusepath{fill}%
\end{pgfscope}%
\begin{pgfscope}%
\pgfpathrectangle{\pgfqpoint{0.765000in}{0.660000in}}{\pgfqpoint{4.620000in}{4.620000in}}%
\pgfusepath{clip}%
\pgfsetbuttcap%
\pgfsetroundjoin%
\definecolor{currentfill}{rgb}{1.000000,0.894118,0.788235}%
\pgfsetfillcolor{currentfill}%
\pgfsetlinewidth{0.000000pt}%
\definecolor{currentstroke}{rgb}{1.000000,0.894118,0.788235}%
\pgfsetstrokecolor{currentstroke}%
\pgfsetdash{}{0pt}%
\pgfpathmoveto{\pgfqpoint{3.755404in}{2.857087in}}%
\pgfpathlineto{\pgfqpoint{3.728456in}{2.841529in}}%
\pgfpathlineto{\pgfqpoint{3.755404in}{2.825971in}}%
\pgfpathlineto{\pgfqpoint{3.782351in}{2.841529in}}%
\pgfpathlineto{\pgfqpoint{3.755404in}{2.857087in}}%
\pgfpathclose%
\pgfusepath{fill}%
\end{pgfscope}%
\begin{pgfscope}%
\pgfpathrectangle{\pgfqpoint{0.765000in}{0.660000in}}{\pgfqpoint{4.620000in}{4.620000in}}%
\pgfusepath{clip}%
\pgfsetbuttcap%
\pgfsetroundjoin%
\definecolor{currentfill}{rgb}{1.000000,0.894118,0.788235}%
\pgfsetfillcolor{currentfill}%
\pgfsetlinewidth{0.000000pt}%
\definecolor{currentstroke}{rgb}{1.000000,0.894118,0.788235}%
\pgfsetstrokecolor{currentstroke}%
\pgfsetdash{}{0pt}%
\pgfpathmoveto{\pgfqpoint{3.755404in}{2.794855in}}%
\pgfpathlineto{\pgfqpoint{3.782351in}{2.810413in}}%
\pgfpathlineto{\pgfqpoint{3.782351in}{2.841529in}}%
\pgfpathlineto{\pgfqpoint{3.755404in}{2.825971in}}%
\pgfpathlineto{\pgfqpoint{3.755404in}{2.794855in}}%
\pgfpathclose%
\pgfusepath{fill}%
\end{pgfscope}%
\begin{pgfscope}%
\pgfpathrectangle{\pgfqpoint{0.765000in}{0.660000in}}{\pgfqpoint{4.620000in}{4.620000in}}%
\pgfusepath{clip}%
\pgfsetbuttcap%
\pgfsetroundjoin%
\definecolor{currentfill}{rgb}{1.000000,0.894118,0.788235}%
\pgfsetfillcolor{currentfill}%
\pgfsetlinewidth{0.000000pt}%
\definecolor{currentstroke}{rgb}{1.000000,0.894118,0.788235}%
\pgfsetstrokecolor{currentstroke}%
\pgfsetdash{}{0pt}%
\pgfpathmoveto{\pgfqpoint{3.728456in}{2.810413in}}%
\pgfpathlineto{\pgfqpoint{3.755404in}{2.794855in}}%
\pgfpathlineto{\pgfqpoint{3.755404in}{2.825971in}}%
\pgfpathlineto{\pgfqpoint{3.728456in}{2.841529in}}%
\pgfpathlineto{\pgfqpoint{3.728456in}{2.810413in}}%
\pgfpathclose%
\pgfusepath{fill}%
\end{pgfscope}%
\begin{pgfscope}%
\pgfpathrectangle{\pgfqpoint{0.765000in}{0.660000in}}{\pgfqpoint{4.620000in}{4.620000in}}%
\pgfusepath{clip}%
\pgfsetbuttcap%
\pgfsetroundjoin%
\definecolor{currentfill}{rgb}{1.000000,0.894118,0.788235}%
\pgfsetfillcolor{currentfill}%
\pgfsetlinewidth{0.000000pt}%
\definecolor{currentstroke}{rgb}{1.000000,0.894118,0.788235}%
\pgfsetstrokecolor{currentstroke}%
\pgfsetdash{}{0pt}%
\pgfpathmoveto{\pgfqpoint{3.717641in}{2.829213in}}%
\pgfpathlineto{\pgfqpoint{3.690694in}{2.813655in}}%
\pgfpathlineto{\pgfqpoint{3.728456in}{2.810413in}}%
\pgfpathlineto{\pgfqpoint{3.755404in}{2.825971in}}%
\pgfpathlineto{\pgfqpoint{3.717641in}{2.829213in}}%
\pgfpathclose%
\pgfusepath{fill}%
\end{pgfscope}%
\begin{pgfscope}%
\pgfpathrectangle{\pgfqpoint{0.765000in}{0.660000in}}{\pgfqpoint{4.620000in}{4.620000in}}%
\pgfusepath{clip}%
\pgfsetbuttcap%
\pgfsetroundjoin%
\definecolor{currentfill}{rgb}{1.000000,0.894118,0.788235}%
\pgfsetfillcolor{currentfill}%
\pgfsetlinewidth{0.000000pt}%
\definecolor{currentstroke}{rgb}{1.000000,0.894118,0.788235}%
\pgfsetstrokecolor{currentstroke}%
\pgfsetdash{}{0pt}%
\pgfpathmoveto{\pgfqpoint{3.744588in}{2.813655in}}%
\pgfpathlineto{\pgfqpoint{3.717641in}{2.829213in}}%
\pgfpathlineto{\pgfqpoint{3.755404in}{2.825971in}}%
\pgfpathlineto{\pgfqpoint{3.782351in}{2.810413in}}%
\pgfpathlineto{\pgfqpoint{3.744588in}{2.813655in}}%
\pgfpathclose%
\pgfusepath{fill}%
\end{pgfscope}%
\begin{pgfscope}%
\pgfpathrectangle{\pgfqpoint{0.765000in}{0.660000in}}{\pgfqpoint{4.620000in}{4.620000in}}%
\pgfusepath{clip}%
\pgfsetbuttcap%
\pgfsetroundjoin%
\definecolor{currentfill}{rgb}{1.000000,0.894118,0.788235}%
\pgfsetfillcolor{currentfill}%
\pgfsetlinewidth{0.000000pt}%
\definecolor{currentstroke}{rgb}{1.000000,0.894118,0.788235}%
\pgfsetstrokecolor{currentstroke}%
\pgfsetdash{}{0pt}%
\pgfpathmoveto{\pgfqpoint{3.717641in}{2.829213in}}%
\pgfpathlineto{\pgfqpoint{3.717641in}{2.860329in}}%
\pgfpathlineto{\pgfqpoint{3.755404in}{2.857087in}}%
\pgfpathlineto{\pgfqpoint{3.782351in}{2.810413in}}%
\pgfpathlineto{\pgfqpoint{3.717641in}{2.829213in}}%
\pgfpathclose%
\pgfusepath{fill}%
\end{pgfscope}%
\begin{pgfscope}%
\pgfpathrectangle{\pgfqpoint{0.765000in}{0.660000in}}{\pgfqpoint{4.620000in}{4.620000in}}%
\pgfusepath{clip}%
\pgfsetbuttcap%
\pgfsetroundjoin%
\definecolor{currentfill}{rgb}{1.000000,0.894118,0.788235}%
\pgfsetfillcolor{currentfill}%
\pgfsetlinewidth{0.000000pt}%
\definecolor{currentstroke}{rgb}{1.000000,0.894118,0.788235}%
\pgfsetstrokecolor{currentstroke}%
\pgfsetdash{}{0pt}%
\pgfpathmoveto{\pgfqpoint{3.744588in}{2.813655in}}%
\pgfpathlineto{\pgfqpoint{3.744588in}{2.844771in}}%
\pgfpathlineto{\pgfqpoint{3.782351in}{2.841529in}}%
\pgfpathlineto{\pgfqpoint{3.755404in}{2.825971in}}%
\pgfpathlineto{\pgfqpoint{3.744588in}{2.813655in}}%
\pgfpathclose%
\pgfusepath{fill}%
\end{pgfscope}%
\begin{pgfscope}%
\pgfpathrectangle{\pgfqpoint{0.765000in}{0.660000in}}{\pgfqpoint{4.620000in}{4.620000in}}%
\pgfusepath{clip}%
\pgfsetbuttcap%
\pgfsetroundjoin%
\definecolor{currentfill}{rgb}{1.000000,0.894118,0.788235}%
\pgfsetfillcolor{currentfill}%
\pgfsetlinewidth{0.000000pt}%
\definecolor{currentstroke}{rgb}{1.000000,0.894118,0.788235}%
\pgfsetstrokecolor{currentstroke}%
\pgfsetdash{}{0pt}%
\pgfpathmoveto{\pgfqpoint{3.690694in}{2.813655in}}%
\pgfpathlineto{\pgfqpoint{3.717641in}{2.798097in}}%
\pgfpathlineto{\pgfqpoint{3.755404in}{2.794855in}}%
\pgfpathlineto{\pgfqpoint{3.728456in}{2.810413in}}%
\pgfpathlineto{\pgfqpoint{3.690694in}{2.813655in}}%
\pgfpathclose%
\pgfusepath{fill}%
\end{pgfscope}%
\begin{pgfscope}%
\pgfpathrectangle{\pgfqpoint{0.765000in}{0.660000in}}{\pgfqpoint{4.620000in}{4.620000in}}%
\pgfusepath{clip}%
\pgfsetbuttcap%
\pgfsetroundjoin%
\definecolor{currentfill}{rgb}{1.000000,0.894118,0.788235}%
\pgfsetfillcolor{currentfill}%
\pgfsetlinewidth{0.000000pt}%
\definecolor{currentstroke}{rgb}{1.000000,0.894118,0.788235}%
\pgfsetstrokecolor{currentstroke}%
\pgfsetdash{}{0pt}%
\pgfpathmoveto{\pgfqpoint{3.717641in}{2.798097in}}%
\pgfpathlineto{\pgfqpoint{3.744588in}{2.813655in}}%
\pgfpathlineto{\pgfqpoint{3.782351in}{2.810413in}}%
\pgfpathlineto{\pgfqpoint{3.755404in}{2.794855in}}%
\pgfpathlineto{\pgfqpoint{3.717641in}{2.798097in}}%
\pgfpathclose%
\pgfusepath{fill}%
\end{pgfscope}%
\begin{pgfscope}%
\pgfpathrectangle{\pgfqpoint{0.765000in}{0.660000in}}{\pgfqpoint{4.620000in}{4.620000in}}%
\pgfusepath{clip}%
\pgfsetbuttcap%
\pgfsetroundjoin%
\definecolor{currentfill}{rgb}{1.000000,0.894118,0.788235}%
\pgfsetfillcolor{currentfill}%
\pgfsetlinewidth{0.000000pt}%
\definecolor{currentstroke}{rgb}{1.000000,0.894118,0.788235}%
\pgfsetstrokecolor{currentstroke}%
\pgfsetdash{}{0pt}%
\pgfpathmoveto{\pgfqpoint{3.717641in}{2.860329in}}%
\pgfpathlineto{\pgfqpoint{3.690694in}{2.844771in}}%
\pgfpathlineto{\pgfqpoint{3.728456in}{2.841529in}}%
\pgfpathlineto{\pgfqpoint{3.755404in}{2.857087in}}%
\pgfpathlineto{\pgfqpoint{3.717641in}{2.860329in}}%
\pgfpathclose%
\pgfusepath{fill}%
\end{pgfscope}%
\begin{pgfscope}%
\pgfpathrectangle{\pgfqpoint{0.765000in}{0.660000in}}{\pgfqpoint{4.620000in}{4.620000in}}%
\pgfusepath{clip}%
\pgfsetbuttcap%
\pgfsetroundjoin%
\definecolor{currentfill}{rgb}{1.000000,0.894118,0.788235}%
\pgfsetfillcolor{currentfill}%
\pgfsetlinewidth{0.000000pt}%
\definecolor{currentstroke}{rgb}{1.000000,0.894118,0.788235}%
\pgfsetstrokecolor{currentstroke}%
\pgfsetdash{}{0pt}%
\pgfpathmoveto{\pgfqpoint{3.744588in}{2.844771in}}%
\pgfpathlineto{\pgfqpoint{3.717641in}{2.860329in}}%
\pgfpathlineto{\pgfqpoint{3.755404in}{2.857087in}}%
\pgfpathlineto{\pgfqpoint{3.782351in}{2.841529in}}%
\pgfpathlineto{\pgfqpoint{3.744588in}{2.844771in}}%
\pgfpathclose%
\pgfusepath{fill}%
\end{pgfscope}%
\begin{pgfscope}%
\pgfpathrectangle{\pgfqpoint{0.765000in}{0.660000in}}{\pgfqpoint{4.620000in}{4.620000in}}%
\pgfusepath{clip}%
\pgfsetbuttcap%
\pgfsetroundjoin%
\definecolor{currentfill}{rgb}{1.000000,0.894118,0.788235}%
\pgfsetfillcolor{currentfill}%
\pgfsetlinewidth{0.000000pt}%
\definecolor{currentstroke}{rgb}{1.000000,0.894118,0.788235}%
\pgfsetstrokecolor{currentstroke}%
\pgfsetdash{}{0pt}%
\pgfpathmoveto{\pgfqpoint{3.690694in}{2.813655in}}%
\pgfpathlineto{\pgfqpoint{3.690694in}{2.844771in}}%
\pgfpathlineto{\pgfqpoint{3.728456in}{2.841529in}}%
\pgfpathlineto{\pgfqpoint{3.755404in}{2.794855in}}%
\pgfpathlineto{\pgfqpoint{3.690694in}{2.813655in}}%
\pgfpathclose%
\pgfusepath{fill}%
\end{pgfscope}%
\begin{pgfscope}%
\pgfpathrectangle{\pgfqpoint{0.765000in}{0.660000in}}{\pgfqpoint{4.620000in}{4.620000in}}%
\pgfusepath{clip}%
\pgfsetbuttcap%
\pgfsetroundjoin%
\definecolor{currentfill}{rgb}{1.000000,0.894118,0.788235}%
\pgfsetfillcolor{currentfill}%
\pgfsetlinewidth{0.000000pt}%
\definecolor{currentstroke}{rgb}{1.000000,0.894118,0.788235}%
\pgfsetstrokecolor{currentstroke}%
\pgfsetdash{}{0pt}%
\pgfpathmoveto{\pgfqpoint{3.717641in}{2.798097in}}%
\pgfpathlineto{\pgfqpoint{3.717641in}{2.829213in}}%
\pgfpathlineto{\pgfqpoint{3.755404in}{2.825971in}}%
\pgfpathlineto{\pgfqpoint{3.728456in}{2.810413in}}%
\pgfpathlineto{\pgfqpoint{3.717641in}{2.798097in}}%
\pgfpathclose%
\pgfusepath{fill}%
\end{pgfscope}%
\begin{pgfscope}%
\pgfpathrectangle{\pgfqpoint{0.765000in}{0.660000in}}{\pgfqpoint{4.620000in}{4.620000in}}%
\pgfusepath{clip}%
\pgfsetbuttcap%
\pgfsetroundjoin%
\definecolor{currentfill}{rgb}{1.000000,0.894118,0.788235}%
\pgfsetfillcolor{currentfill}%
\pgfsetlinewidth{0.000000pt}%
\definecolor{currentstroke}{rgb}{1.000000,0.894118,0.788235}%
\pgfsetstrokecolor{currentstroke}%
\pgfsetdash{}{0pt}%
\pgfpathmoveto{\pgfqpoint{3.717641in}{2.829213in}}%
\pgfpathlineto{\pgfqpoint{3.744588in}{2.844771in}}%
\pgfpathlineto{\pgfqpoint{3.782351in}{2.841529in}}%
\pgfpathlineto{\pgfqpoint{3.755404in}{2.825971in}}%
\pgfpathlineto{\pgfqpoint{3.717641in}{2.829213in}}%
\pgfpathclose%
\pgfusepath{fill}%
\end{pgfscope}%
\begin{pgfscope}%
\pgfpathrectangle{\pgfqpoint{0.765000in}{0.660000in}}{\pgfqpoint{4.620000in}{4.620000in}}%
\pgfusepath{clip}%
\pgfsetbuttcap%
\pgfsetroundjoin%
\definecolor{currentfill}{rgb}{1.000000,0.894118,0.788235}%
\pgfsetfillcolor{currentfill}%
\pgfsetlinewidth{0.000000pt}%
\definecolor{currentstroke}{rgb}{1.000000,0.894118,0.788235}%
\pgfsetstrokecolor{currentstroke}%
\pgfsetdash{}{0pt}%
\pgfpathmoveto{\pgfqpoint{3.690694in}{2.844771in}}%
\pgfpathlineto{\pgfqpoint{3.717641in}{2.829213in}}%
\pgfpathlineto{\pgfqpoint{3.755404in}{2.825971in}}%
\pgfpathlineto{\pgfqpoint{3.728456in}{2.841529in}}%
\pgfpathlineto{\pgfqpoint{3.690694in}{2.844771in}}%
\pgfpathclose%
\pgfusepath{fill}%
\end{pgfscope}%
\begin{pgfscope}%
\pgfpathrectangle{\pgfqpoint{0.765000in}{0.660000in}}{\pgfqpoint{4.620000in}{4.620000in}}%
\pgfusepath{clip}%
\pgfsetbuttcap%
\pgfsetroundjoin%
\definecolor{currentfill}{rgb}{1.000000,0.894118,0.788235}%
\pgfsetfillcolor{currentfill}%
\pgfsetlinewidth{0.000000pt}%
\definecolor{currentstroke}{rgb}{1.000000,0.894118,0.788235}%
\pgfsetstrokecolor{currentstroke}%
\pgfsetdash{}{0pt}%
\pgfpathmoveto{\pgfqpoint{3.717641in}{2.829213in}}%
\pgfpathlineto{\pgfqpoint{3.690694in}{2.813655in}}%
\pgfpathlineto{\pgfqpoint{3.717641in}{2.798097in}}%
\pgfpathlineto{\pgfqpoint{3.744588in}{2.813655in}}%
\pgfpathlineto{\pgfqpoint{3.717641in}{2.829213in}}%
\pgfpathclose%
\pgfusepath{fill}%
\end{pgfscope}%
\begin{pgfscope}%
\pgfpathrectangle{\pgfqpoint{0.765000in}{0.660000in}}{\pgfqpoint{4.620000in}{4.620000in}}%
\pgfusepath{clip}%
\pgfsetbuttcap%
\pgfsetroundjoin%
\definecolor{currentfill}{rgb}{1.000000,0.894118,0.788235}%
\pgfsetfillcolor{currentfill}%
\pgfsetlinewidth{0.000000pt}%
\definecolor{currentstroke}{rgb}{1.000000,0.894118,0.788235}%
\pgfsetstrokecolor{currentstroke}%
\pgfsetdash{}{0pt}%
\pgfpathmoveto{\pgfqpoint{3.717641in}{2.829213in}}%
\pgfpathlineto{\pgfqpoint{3.690694in}{2.813655in}}%
\pgfpathlineto{\pgfqpoint{3.690694in}{2.844771in}}%
\pgfpathlineto{\pgfqpoint{3.717641in}{2.860329in}}%
\pgfpathlineto{\pgfqpoint{3.717641in}{2.829213in}}%
\pgfpathclose%
\pgfusepath{fill}%
\end{pgfscope}%
\begin{pgfscope}%
\pgfpathrectangle{\pgfqpoint{0.765000in}{0.660000in}}{\pgfqpoint{4.620000in}{4.620000in}}%
\pgfusepath{clip}%
\pgfsetbuttcap%
\pgfsetroundjoin%
\definecolor{currentfill}{rgb}{1.000000,0.894118,0.788235}%
\pgfsetfillcolor{currentfill}%
\pgfsetlinewidth{0.000000pt}%
\definecolor{currentstroke}{rgb}{1.000000,0.894118,0.788235}%
\pgfsetstrokecolor{currentstroke}%
\pgfsetdash{}{0pt}%
\pgfpathmoveto{\pgfqpoint{3.717641in}{2.829213in}}%
\pgfpathlineto{\pgfqpoint{3.744588in}{2.813655in}}%
\pgfpathlineto{\pgfqpoint{3.744588in}{2.844771in}}%
\pgfpathlineto{\pgfqpoint{3.717641in}{2.860329in}}%
\pgfpathlineto{\pgfqpoint{3.717641in}{2.829213in}}%
\pgfpathclose%
\pgfusepath{fill}%
\end{pgfscope}%
\begin{pgfscope}%
\pgfpathrectangle{\pgfqpoint{0.765000in}{0.660000in}}{\pgfqpoint{4.620000in}{4.620000in}}%
\pgfusepath{clip}%
\pgfsetbuttcap%
\pgfsetroundjoin%
\definecolor{currentfill}{rgb}{1.000000,0.894118,0.788235}%
\pgfsetfillcolor{currentfill}%
\pgfsetlinewidth{0.000000pt}%
\definecolor{currentstroke}{rgb}{1.000000,0.894118,0.788235}%
\pgfsetstrokecolor{currentstroke}%
\pgfsetdash{}{0pt}%
\pgfpathmoveto{\pgfqpoint{3.712800in}{2.830079in}}%
\pgfpathlineto{\pgfqpoint{3.685852in}{2.814521in}}%
\pgfpathlineto{\pgfqpoint{3.712800in}{2.798963in}}%
\pgfpathlineto{\pgfqpoint{3.739747in}{2.814521in}}%
\pgfpathlineto{\pgfqpoint{3.712800in}{2.830079in}}%
\pgfpathclose%
\pgfusepath{fill}%
\end{pgfscope}%
\begin{pgfscope}%
\pgfpathrectangle{\pgfqpoint{0.765000in}{0.660000in}}{\pgfqpoint{4.620000in}{4.620000in}}%
\pgfusepath{clip}%
\pgfsetbuttcap%
\pgfsetroundjoin%
\definecolor{currentfill}{rgb}{1.000000,0.894118,0.788235}%
\pgfsetfillcolor{currentfill}%
\pgfsetlinewidth{0.000000pt}%
\definecolor{currentstroke}{rgb}{1.000000,0.894118,0.788235}%
\pgfsetstrokecolor{currentstroke}%
\pgfsetdash{}{0pt}%
\pgfpathmoveto{\pgfqpoint{3.712800in}{2.830079in}}%
\pgfpathlineto{\pgfqpoint{3.685852in}{2.814521in}}%
\pgfpathlineto{\pgfqpoint{3.685852in}{2.845636in}}%
\pgfpathlineto{\pgfqpoint{3.712800in}{2.861194in}}%
\pgfpathlineto{\pgfqpoint{3.712800in}{2.830079in}}%
\pgfpathclose%
\pgfusepath{fill}%
\end{pgfscope}%
\begin{pgfscope}%
\pgfpathrectangle{\pgfqpoint{0.765000in}{0.660000in}}{\pgfqpoint{4.620000in}{4.620000in}}%
\pgfusepath{clip}%
\pgfsetbuttcap%
\pgfsetroundjoin%
\definecolor{currentfill}{rgb}{1.000000,0.894118,0.788235}%
\pgfsetfillcolor{currentfill}%
\pgfsetlinewidth{0.000000pt}%
\definecolor{currentstroke}{rgb}{1.000000,0.894118,0.788235}%
\pgfsetstrokecolor{currentstroke}%
\pgfsetdash{}{0pt}%
\pgfpathmoveto{\pgfqpoint{3.712800in}{2.830079in}}%
\pgfpathlineto{\pgfqpoint{3.739747in}{2.814521in}}%
\pgfpathlineto{\pgfqpoint{3.739747in}{2.845636in}}%
\pgfpathlineto{\pgfqpoint{3.712800in}{2.861194in}}%
\pgfpathlineto{\pgfqpoint{3.712800in}{2.830079in}}%
\pgfpathclose%
\pgfusepath{fill}%
\end{pgfscope}%
\begin{pgfscope}%
\pgfpathrectangle{\pgfqpoint{0.765000in}{0.660000in}}{\pgfqpoint{4.620000in}{4.620000in}}%
\pgfusepath{clip}%
\pgfsetbuttcap%
\pgfsetroundjoin%
\definecolor{currentfill}{rgb}{1.000000,0.894118,0.788235}%
\pgfsetfillcolor{currentfill}%
\pgfsetlinewidth{0.000000pt}%
\definecolor{currentstroke}{rgb}{1.000000,0.894118,0.788235}%
\pgfsetstrokecolor{currentstroke}%
\pgfsetdash{}{0pt}%
\pgfpathmoveto{\pgfqpoint{3.717641in}{2.860329in}}%
\pgfpathlineto{\pgfqpoint{3.690694in}{2.844771in}}%
\pgfpathlineto{\pgfqpoint{3.717641in}{2.829213in}}%
\pgfpathlineto{\pgfqpoint{3.744588in}{2.844771in}}%
\pgfpathlineto{\pgfqpoint{3.717641in}{2.860329in}}%
\pgfpathclose%
\pgfusepath{fill}%
\end{pgfscope}%
\begin{pgfscope}%
\pgfpathrectangle{\pgfqpoint{0.765000in}{0.660000in}}{\pgfqpoint{4.620000in}{4.620000in}}%
\pgfusepath{clip}%
\pgfsetbuttcap%
\pgfsetroundjoin%
\definecolor{currentfill}{rgb}{1.000000,0.894118,0.788235}%
\pgfsetfillcolor{currentfill}%
\pgfsetlinewidth{0.000000pt}%
\definecolor{currentstroke}{rgb}{1.000000,0.894118,0.788235}%
\pgfsetstrokecolor{currentstroke}%
\pgfsetdash{}{0pt}%
\pgfpathmoveto{\pgfqpoint{3.717641in}{2.798097in}}%
\pgfpathlineto{\pgfqpoint{3.744588in}{2.813655in}}%
\pgfpathlineto{\pgfqpoint{3.744588in}{2.844771in}}%
\pgfpathlineto{\pgfqpoint{3.717641in}{2.829213in}}%
\pgfpathlineto{\pgfqpoint{3.717641in}{2.798097in}}%
\pgfpathclose%
\pgfusepath{fill}%
\end{pgfscope}%
\begin{pgfscope}%
\pgfpathrectangle{\pgfqpoint{0.765000in}{0.660000in}}{\pgfqpoint{4.620000in}{4.620000in}}%
\pgfusepath{clip}%
\pgfsetbuttcap%
\pgfsetroundjoin%
\definecolor{currentfill}{rgb}{1.000000,0.894118,0.788235}%
\pgfsetfillcolor{currentfill}%
\pgfsetlinewidth{0.000000pt}%
\definecolor{currentstroke}{rgb}{1.000000,0.894118,0.788235}%
\pgfsetstrokecolor{currentstroke}%
\pgfsetdash{}{0pt}%
\pgfpathmoveto{\pgfqpoint{3.690694in}{2.813655in}}%
\pgfpathlineto{\pgfqpoint{3.717641in}{2.798097in}}%
\pgfpathlineto{\pgfqpoint{3.717641in}{2.829213in}}%
\pgfpathlineto{\pgfqpoint{3.690694in}{2.844771in}}%
\pgfpathlineto{\pgfqpoint{3.690694in}{2.813655in}}%
\pgfpathclose%
\pgfusepath{fill}%
\end{pgfscope}%
\begin{pgfscope}%
\pgfpathrectangle{\pgfqpoint{0.765000in}{0.660000in}}{\pgfqpoint{4.620000in}{4.620000in}}%
\pgfusepath{clip}%
\pgfsetbuttcap%
\pgfsetroundjoin%
\definecolor{currentfill}{rgb}{1.000000,0.894118,0.788235}%
\pgfsetfillcolor{currentfill}%
\pgfsetlinewidth{0.000000pt}%
\definecolor{currentstroke}{rgb}{1.000000,0.894118,0.788235}%
\pgfsetstrokecolor{currentstroke}%
\pgfsetdash{}{0pt}%
\pgfpathmoveto{\pgfqpoint{3.712800in}{2.861194in}}%
\pgfpathlineto{\pgfqpoint{3.685852in}{2.845636in}}%
\pgfpathlineto{\pgfqpoint{3.712800in}{2.830079in}}%
\pgfpathlineto{\pgfqpoint{3.739747in}{2.845636in}}%
\pgfpathlineto{\pgfqpoint{3.712800in}{2.861194in}}%
\pgfpathclose%
\pgfusepath{fill}%
\end{pgfscope}%
\begin{pgfscope}%
\pgfpathrectangle{\pgfqpoint{0.765000in}{0.660000in}}{\pgfqpoint{4.620000in}{4.620000in}}%
\pgfusepath{clip}%
\pgfsetbuttcap%
\pgfsetroundjoin%
\definecolor{currentfill}{rgb}{1.000000,0.894118,0.788235}%
\pgfsetfillcolor{currentfill}%
\pgfsetlinewidth{0.000000pt}%
\definecolor{currentstroke}{rgb}{1.000000,0.894118,0.788235}%
\pgfsetstrokecolor{currentstroke}%
\pgfsetdash{}{0pt}%
\pgfpathmoveto{\pgfqpoint{3.712800in}{2.798963in}}%
\pgfpathlineto{\pgfqpoint{3.739747in}{2.814521in}}%
\pgfpathlineto{\pgfqpoint{3.739747in}{2.845636in}}%
\pgfpathlineto{\pgfqpoint{3.712800in}{2.830079in}}%
\pgfpathlineto{\pgfqpoint{3.712800in}{2.798963in}}%
\pgfpathclose%
\pgfusepath{fill}%
\end{pgfscope}%
\begin{pgfscope}%
\pgfpathrectangle{\pgfqpoint{0.765000in}{0.660000in}}{\pgfqpoint{4.620000in}{4.620000in}}%
\pgfusepath{clip}%
\pgfsetbuttcap%
\pgfsetroundjoin%
\definecolor{currentfill}{rgb}{1.000000,0.894118,0.788235}%
\pgfsetfillcolor{currentfill}%
\pgfsetlinewidth{0.000000pt}%
\definecolor{currentstroke}{rgb}{1.000000,0.894118,0.788235}%
\pgfsetstrokecolor{currentstroke}%
\pgfsetdash{}{0pt}%
\pgfpathmoveto{\pgfqpoint{3.685852in}{2.814521in}}%
\pgfpathlineto{\pgfqpoint{3.712800in}{2.798963in}}%
\pgfpathlineto{\pgfqpoint{3.712800in}{2.830079in}}%
\pgfpathlineto{\pgfqpoint{3.685852in}{2.845636in}}%
\pgfpathlineto{\pgfqpoint{3.685852in}{2.814521in}}%
\pgfpathclose%
\pgfusepath{fill}%
\end{pgfscope}%
\begin{pgfscope}%
\pgfpathrectangle{\pgfqpoint{0.765000in}{0.660000in}}{\pgfqpoint{4.620000in}{4.620000in}}%
\pgfusepath{clip}%
\pgfsetbuttcap%
\pgfsetroundjoin%
\definecolor{currentfill}{rgb}{1.000000,0.894118,0.788235}%
\pgfsetfillcolor{currentfill}%
\pgfsetlinewidth{0.000000pt}%
\definecolor{currentstroke}{rgb}{1.000000,0.894118,0.788235}%
\pgfsetstrokecolor{currentstroke}%
\pgfsetdash{}{0pt}%
\pgfpathmoveto{\pgfqpoint{3.717641in}{2.829213in}}%
\pgfpathlineto{\pgfqpoint{3.690694in}{2.813655in}}%
\pgfpathlineto{\pgfqpoint{3.685852in}{2.814521in}}%
\pgfpathlineto{\pgfqpoint{3.712800in}{2.830079in}}%
\pgfpathlineto{\pgfqpoint{3.717641in}{2.829213in}}%
\pgfpathclose%
\pgfusepath{fill}%
\end{pgfscope}%
\begin{pgfscope}%
\pgfpathrectangle{\pgfqpoint{0.765000in}{0.660000in}}{\pgfqpoint{4.620000in}{4.620000in}}%
\pgfusepath{clip}%
\pgfsetbuttcap%
\pgfsetroundjoin%
\definecolor{currentfill}{rgb}{1.000000,0.894118,0.788235}%
\pgfsetfillcolor{currentfill}%
\pgfsetlinewidth{0.000000pt}%
\definecolor{currentstroke}{rgb}{1.000000,0.894118,0.788235}%
\pgfsetstrokecolor{currentstroke}%
\pgfsetdash{}{0pt}%
\pgfpathmoveto{\pgfqpoint{3.744588in}{2.813655in}}%
\pgfpathlineto{\pgfqpoint{3.717641in}{2.829213in}}%
\pgfpathlineto{\pgfqpoint{3.712800in}{2.830079in}}%
\pgfpathlineto{\pgfqpoint{3.739747in}{2.814521in}}%
\pgfpathlineto{\pgfqpoint{3.744588in}{2.813655in}}%
\pgfpathclose%
\pgfusepath{fill}%
\end{pgfscope}%
\begin{pgfscope}%
\pgfpathrectangle{\pgfqpoint{0.765000in}{0.660000in}}{\pgfqpoint{4.620000in}{4.620000in}}%
\pgfusepath{clip}%
\pgfsetbuttcap%
\pgfsetroundjoin%
\definecolor{currentfill}{rgb}{1.000000,0.894118,0.788235}%
\pgfsetfillcolor{currentfill}%
\pgfsetlinewidth{0.000000pt}%
\definecolor{currentstroke}{rgb}{1.000000,0.894118,0.788235}%
\pgfsetstrokecolor{currentstroke}%
\pgfsetdash{}{0pt}%
\pgfpathmoveto{\pgfqpoint{3.717641in}{2.829213in}}%
\pgfpathlineto{\pgfqpoint{3.717641in}{2.860329in}}%
\pgfpathlineto{\pgfqpoint{3.712800in}{2.861194in}}%
\pgfpathlineto{\pgfqpoint{3.739747in}{2.814521in}}%
\pgfpathlineto{\pgfqpoint{3.717641in}{2.829213in}}%
\pgfpathclose%
\pgfusepath{fill}%
\end{pgfscope}%
\begin{pgfscope}%
\pgfpathrectangle{\pgfqpoint{0.765000in}{0.660000in}}{\pgfqpoint{4.620000in}{4.620000in}}%
\pgfusepath{clip}%
\pgfsetbuttcap%
\pgfsetroundjoin%
\definecolor{currentfill}{rgb}{1.000000,0.894118,0.788235}%
\pgfsetfillcolor{currentfill}%
\pgfsetlinewidth{0.000000pt}%
\definecolor{currentstroke}{rgb}{1.000000,0.894118,0.788235}%
\pgfsetstrokecolor{currentstroke}%
\pgfsetdash{}{0pt}%
\pgfpathmoveto{\pgfqpoint{3.744588in}{2.813655in}}%
\pgfpathlineto{\pgfqpoint{3.744588in}{2.844771in}}%
\pgfpathlineto{\pgfqpoint{3.739747in}{2.845636in}}%
\pgfpathlineto{\pgfqpoint{3.712800in}{2.830079in}}%
\pgfpathlineto{\pgfqpoint{3.744588in}{2.813655in}}%
\pgfpathclose%
\pgfusepath{fill}%
\end{pgfscope}%
\begin{pgfscope}%
\pgfpathrectangle{\pgfqpoint{0.765000in}{0.660000in}}{\pgfqpoint{4.620000in}{4.620000in}}%
\pgfusepath{clip}%
\pgfsetbuttcap%
\pgfsetroundjoin%
\definecolor{currentfill}{rgb}{1.000000,0.894118,0.788235}%
\pgfsetfillcolor{currentfill}%
\pgfsetlinewidth{0.000000pt}%
\definecolor{currentstroke}{rgb}{1.000000,0.894118,0.788235}%
\pgfsetstrokecolor{currentstroke}%
\pgfsetdash{}{0pt}%
\pgfpathmoveto{\pgfqpoint{3.690694in}{2.813655in}}%
\pgfpathlineto{\pgfqpoint{3.717641in}{2.798097in}}%
\pgfpathlineto{\pgfqpoint{3.712800in}{2.798963in}}%
\pgfpathlineto{\pgfqpoint{3.685852in}{2.814521in}}%
\pgfpathlineto{\pgfqpoint{3.690694in}{2.813655in}}%
\pgfpathclose%
\pgfusepath{fill}%
\end{pgfscope}%
\begin{pgfscope}%
\pgfpathrectangle{\pgfqpoint{0.765000in}{0.660000in}}{\pgfqpoint{4.620000in}{4.620000in}}%
\pgfusepath{clip}%
\pgfsetbuttcap%
\pgfsetroundjoin%
\definecolor{currentfill}{rgb}{1.000000,0.894118,0.788235}%
\pgfsetfillcolor{currentfill}%
\pgfsetlinewidth{0.000000pt}%
\definecolor{currentstroke}{rgb}{1.000000,0.894118,0.788235}%
\pgfsetstrokecolor{currentstroke}%
\pgfsetdash{}{0pt}%
\pgfpathmoveto{\pgfqpoint{3.717641in}{2.798097in}}%
\pgfpathlineto{\pgfqpoint{3.744588in}{2.813655in}}%
\pgfpathlineto{\pgfqpoint{3.739747in}{2.814521in}}%
\pgfpathlineto{\pgfqpoint{3.712800in}{2.798963in}}%
\pgfpathlineto{\pgfqpoint{3.717641in}{2.798097in}}%
\pgfpathclose%
\pgfusepath{fill}%
\end{pgfscope}%
\begin{pgfscope}%
\pgfpathrectangle{\pgfqpoint{0.765000in}{0.660000in}}{\pgfqpoint{4.620000in}{4.620000in}}%
\pgfusepath{clip}%
\pgfsetbuttcap%
\pgfsetroundjoin%
\definecolor{currentfill}{rgb}{1.000000,0.894118,0.788235}%
\pgfsetfillcolor{currentfill}%
\pgfsetlinewidth{0.000000pt}%
\definecolor{currentstroke}{rgb}{1.000000,0.894118,0.788235}%
\pgfsetstrokecolor{currentstroke}%
\pgfsetdash{}{0pt}%
\pgfpathmoveto{\pgfqpoint{3.717641in}{2.860329in}}%
\pgfpathlineto{\pgfqpoint{3.690694in}{2.844771in}}%
\pgfpathlineto{\pgfqpoint{3.685852in}{2.845636in}}%
\pgfpathlineto{\pgfqpoint{3.712800in}{2.861194in}}%
\pgfpathlineto{\pgfqpoint{3.717641in}{2.860329in}}%
\pgfpathclose%
\pgfusepath{fill}%
\end{pgfscope}%
\begin{pgfscope}%
\pgfpathrectangle{\pgfqpoint{0.765000in}{0.660000in}}{\pgfqpoint{4.620000in}{4.620000in}}%
\pgfusepath{clip}%
\pgfsetbuttcap%
\pgfsetroundjoin%
\definecolor{currentfill}{rgb}{1.000000,0.894118,0.788235}%
\pgfsetfillcolor{currentfill}%
\pgfsetlinewidth{0.000000pt}%
\definecolor{currentstroke}{rgb}{1.000000,0.894118,0.788235}%
\pgfsetstrokecolor{currentstroke}%
\pgfsetdash{}{0pt}%
\pgfpathmoveto{\pgfqpoint{3.744588in}{2.844771in}}%
\pgfpathlineto{\pgfqpoint{3.717641in}{2.860329in}}%
\pgfpathlineto{\pgfqpoint{3.712800in}{2.861194in}}%
\pgfpathlineto{\pgfqpoint{3.739747in}{2.845636in}}%
\pgfpathlineto{\pgfqpoint{3.744588in}{2.844771in}}%
\pgfpathclose%
\pgfusepath{fill}%
\end{pgfscope}%
\begin{pgfscope}%
\pgfpathrectangle{\pgfqpoint{0.765000in}{0.660000in}}{\pgfqpoint{4.620000in}{4.620000in}}%
\pgfusepath{clip}%
\pgfsetbuttcap%
\pgfsetroundjoin%
\definecolor{currentfill}{rgb}{1.000000,0.894118,0.788235}%
\pgfsetfillcolor{currentfill}%
\pgfsetlinewidth{0.000000pt}%
\definecolor{currentstroke}{rgb}{1.000000,0.894118,0.788235}%
\pgfsetstrokecolor{currentstroke}%
\pgfsetdash{}{0pt}%
\pgfpathmoveto{\pgfqpoint{3.690694in}{2.813655in}}%
\pgfpathlineto{\pgfqpoint{3.690694in}{2.844771in}}%
\pgfpathlineto{\pgfqpoint{3.685852in}{2.845636in}}%
\pgfpathlineto{\pgfqpoint{3.712800in}{2.798963in}}%
\pgfpathlineto{\pgfqpoint{3.690694in}{2.813655in}}%
\pgfpathclose%
\pgfusepath{fill}%
\end{pgfscope}%
\begin{pgfscope}%
\pgfpathrectangle{\pgfqpoint{0.765000in}{0.660000in}}{\pgfqpoint{4.620000in}{4.620000in}}%
\pgfusepath{clip}%
\pgfsetbuttcap%
\pgfsetroundjoin%
\definecolor{currentfill}{rgb}{1.000000,0.894118,0.788235}%
\pgfsetfillcolor{currentfill}%
\pgfsetlinewidth{0.000000pt}%
\definecolor{currentstroke}{rgb}{1.000000,0.894118,0.788235}%
\pgfsetstrokecolor{currentstroke}%
\pgfsetdash{}{0pt}%
\pgfpathmoveto{\pgfqpoint{3.717641in}{2.798097in}}%
\pgfpathlineto{\pgfqpoint{3.717641in}{2.829213in}}%
\pgfpathlineto{\pgfqpoint{3.712800in}{2.830079in}}%
\pgfpathlineto{\pgfqpoint{3.685852in}{2.814521in}}%
\pgfpathlineto{\pgfqpoint{3.717641in}{2.798097in}}%
\pgfpathclose%
\pgfusepath{fill}%
\end{pgfscope}%
\begin{pgfscope}%
\pgfpathrectangle{\pgfqpoint{0.765000in}{0.660000in}}{\pgfqpoint{4.620000in}{4.620000in}}%
\pgfusepath{clip}%
\pgfsetbuttcap%
\pgfsetroundjoin%
\definecolor{currentfill}{rgb}{1.000000,0.894118,0.788235}%
\pgfsetfillcolor{currentfill}%
\pgfsetlinewidth{0.000000pt}%
\definecolor{currentstroke}{rgb}{1.000000,0.894118,0.788235}%
\pgfsetstrokecolor{currentstroke}%
\pgfsetdash{}{0pt}%
\pgfpathmoveto{\pgfqpoint{3.717641in}{2.829213in}}%
\pgfpathlineto{\pgfqpoint{3.744588in}{2.844771in}}%
\pgfpathlineto{\pgfqpoint{3.739747in}{2.845636in}}%
\pgfpathlineto{\pgfqpoint{3.712800in}{2.830079in}}%
\pgfpathlineto{\pgfqpoint{3.717641in}{2.829213in}}%
\pgfpathclose%
\pgfusepath{fill}%
\end{pgfscope}%
\begin{pgfscope}%
\pgfpathrectangle{\pgfqpoint{0.765000in}{0.660000in}}{\pgfqpoint{4.620000in}{4.620000in}}%
\pgfusepath{clip}%
\pgfsetbuttcap%
\pgfsetroundjoin%
\definecolor{currentfill}{rgb}{1.000000,0.894118,0.788235}%
\pgfsetfillcolor{currentfill}%
\pgfsetlinewidth{0.000000pt}%
\definecolor{currentstroke}{rgb}{1.000000,0.894118,0.788235}%
\pgfsetstrokecolor{currentstroke}%
\pgfsetdash{}{0pt}%
\pgfpathmoveto{\pgfqpoint{3.690694in}{2.844771in}}%
\pgfpathlineto{\pgfqpoint{3.717641in}{2.829213in}}%
\pgfpathlineto{\pgfqpoint{3.712800in}{2.830079in}}%
\pgfpathlineto{\pgfqpoint{3.685852in}{2.845636in}}%
\pgfpathlineto{\pgfqpoint{3.690694in}{2.844771in}}%
\pgfpathclose%
\pgfusepath{fill}%
\end{pgfscope}%
\begin{pgfscope}%
\pgfpathrectangle{\pgfqpoint{0.765000in}{0.660000in}}{\pgfqpoint{4.620000in}{4.620000in}}%
\pgfusepath{clip}%
\pgfsetrectcap%
\pgfsetroundjoin%
\pgfsetlinewidth{1.204500pt}%
\definecolor{currentstroke}{rgb}{1.000000,0.576471,0.309804}%
\pgfsetstrokecolor{currentstroke}%
\pgfsetdash{}{0pt}%
\pgfpathmoveto{\pgfqpoint{3.921343in}{2.686659in}}%
\pgfusepath{stroke}%
\end{pgfscope}%
\begin{pgfscope}%
\pgfpathrectangle{\pgfqpoint{0.765000in}{0.660000in}}{\pgfqpoint{4.620000in}{4.620000in}}%
\pgfusepath{clip}%
\pgfsetbuttcap%
\pgfsetroundjoin%
\definecolor{currentfill}{rgb}{1.000000,0.576471,0.309804}%
\pgfsetfillcolor{currentfill}%
\pgfsetlinewidth{1.003750pt}%
\definecolor{currentstroke}{rgb}{1.000000,0.576471,0.309804}%
\pgfsetstrokecolor{currentstroke}%
\pgfsetdash{}{0pt}%
\pgfsys@defobject{currentmarker}{\pgfqpoint{-0.033333in}{-0.033333in}}{\pgfqpoint{0.033333in}{0.033333in}}{%
\pgfpathmoveto{\pgfqpoint{0.000000in}{-0.033333in}}%
\pgfpathcurveto{\pgfqpoint{0.008840in}{-0.033333in}}{\pgfqpoint{0.017319in}{-0.029821in}}{\pgfqpoint{0.023570in}{-0.023570in}}%
\pgfpathcurveto{\pgfqpoint{0.029821in}{-0.017319in}}{\pgfqpoint{0.033333in}{-0.008840in}}{\pgfqpoint{0.033333in}{0.000000in}}%
\pgfpathcurveto{\pgfqpoint{0.033333in}{0.008840in}}{\pgfqpoint{0.029821in}{0.017319in}}{\pgfqpoint{0.023570in}{0.023570in}}%
\pgfpathcurveto{\pgfqpoint{0.017319in}{0.029821in}}{\pgfqpoint{0.008840in}{0.033333in}}{\pgfqpoint{0.000000in}{0.033333in}}%
\pgfpathcurveto{\pgfqpoint{-0.008840in}{0.033333in}}{\pgfqpoint{-0.017319in}{0.029821in}}{\pgfqpoint{-0.023570in}{0.023570in}}%
\pgfpathcurveto{\pgfqpoint{-0.029821in}{0.017319in}}{\pgfqpoint{-0.033333in}{0.008840in}}{\pgfqpoint{-0.033333in}{0.000000in}}%
\pgfpathcurveto{\pgfqpoint{-0.033333in}{-0.008840in}}{\pgfqpoint{-0.029821in}{-0.017319in}}{\pgfqpoint{-0.023570in}{-0.023570in}}%
\pgfpathcurveto{\pgfqpoint{-0.017319in}{-0.029821in}}{\pgfqpoint{-0.008840in}{-0.033333in}}{\pgfqpoint{0.000000in}{-0.033333in}}%
\pgfpathlineto{\pgfqpoint{0.000000in}{-0.033333in}}%
\pgfpathclose%
\pgfusepath{stroke,fill}%
}%
\begin{pgfscope}%
\pgfsys@transformshift{3.921343in}{2.686659in}%
\pgfsys@useobject{currentmarker}{}%
\end{pgfscope}%
\end{pgfscope}%
\begin{pgfscope}%
\pgfpathrectangle{\pgfqpoint{0.765000in}{0.660000in}}{\pgfqpoint{4.620000in}{4.620000in}}%
\pgfusepath{clip}%
\pgfsetrectcap%
\pgfsetroundjoin%
\pgfsetlinewidth{1.204500pt}%
\definecolor{currentstroke}{rgb}{1.000000,0.576471,0.309804}%
\pgfsetstrokecolor{currentstroke}%
\pgfsetdash{}{0pt}%
\pgfpathmoveto{\pgfqpoint{4.278731in}{2.586900in}}%
\pgfusepath{stroke}%
\end{pgfscope}%
\begin{pgfscope}%
\pgfpathrectangle{\pgfqpoint{0.765000in}{0.660000in}}{\pgfqpoint{4.620000in}{4.620000in}}%
\pgfusepath{clip}%
\pgfsetbuttcap%
\pgfsetroundjoin%
\definecolor{currentfill}{rgb}{1.000000,0.576471,0.309804}%
\pgfsetfillcolor{currentfill}%
\pgfsetlinewidth{1.003750pt}%
\definecolor{currentstroke}{rgb}{1.000000,0.576471,0.309804}%
\pgfsetstrokecolor{currentstroke}%
\pgfsetdash{}{0pt}%
\pgfsys@defobject{currentmarker}{\pgfqpoint{-0.033333in}{-0.033333in}}{\pgfqpoint{0.033333in}{0.033333in}}{%
\pgfpathmoveto{\pgfqpoint{0.000000in}{-0.033333in}}%
\pgfpathcurveto{\pgfqpoint{0.008840in}{-0.033333in}}{\pgfqpoint{0.017319in}{-0.029821in}}{\pgfqpoint{0.023570in}{-0.023570in}}%
\pgfpathcurveto{\pgfqpoint{0.029821in}{-0.017319in}}{\pgfqpoint{0.033333in}{-0.008840in}}{\pgfqpoint{0.033333in}{0.000000in}}%
\pgfpathcurveto{\pgfqpoint{0.033333in}{0.008840in}}{\pgfqpoint{0.029821in}{0.017319in}}{\pgfqpoint{0.023570in}{0.023570in}}%
\pgfpathcurveto{\pgfqpoint{0.017319in}{0.029821in}}{\pgfqpoint{0.008840in}{0.033333in}}{\pgfqpoint{0.000000in}{0.033333in}}%
\pgfpathcurveto{\pgfqpoint{-0.008840in}{0.033333in}}{\pgfqpoint{-0.017319in}{0.029821in}}{\pgfqpoint{-0.023570in}{0.023570in}}%
\pgfpathcurveto{\pgfqpoint{-0.029821in}{0.017319in}}{\pgfqpoint{-0.033333in}{0.008840in}}{\pgfqpoint{-0.033333in}{0.000000in}}%
\pgfpathcurveto{\pgfqpoint{-0.033333in}{-0.008840in}}{\pgfqpoint{-0.029821in}{-0.017319in}}{\pgfqpoint{-0.023570in}{-0.023570in}}%
\pgfpathcurveto{\pgfqpoint{-0.017319in}{-0.029821in}}{\pgfqpoint{-0.008840in}{-0.033333in}}{\pgfqpoint{0.000000in}{-0.033333in}}%
\pgfpathlineto{\pgfqpoint{0.000000in}{-0.033333in}}%
\pgfpathclose%
\pgfusepath{stroke,fill}%
}%
\begin{pgfscope}%
\pgfsys@transformshift{4.278731in}{2.586900in}%
\pgfsys@useobject{currentmarker}{}%
\end{pgfscope}%
\end{pgfscope}%
\begin{pgfscope}%
\pgfpathrectangle{\pgfqpoint{0.765000in}{0.660000in}}{\pgfqpoint{4.620000in}{4.620000in}}%
\pgfusepath{clip}%
\pgfsetrectcap%
\pgfsetroundjoin%
\pgfsetlinewidth{1.204500pt}%
\definecolor{currentstroke}{rgb}{1.000000,0.576471,0.309804}%
\pgfsetstrokecolor{currentstroke}%
\pgfsetdash{}{0pt}%
\pgfpathmoveto{\pgfqpoint{3.747720in}{2.776663in}}%
\pgfusepath{stroke}%
\end{pgfscope}%
\begin{pgfscope}%
\pgfpathrectangle{\pgfqpoint{0.765000in}{0.660000in}}{\pgfqpoint{4.620000in}{4.620000in}}%
\pgfusepath{clip}%
\pgfsetbuttcap%
\pgfsetroundjoin%
\definecolor{currentfill}{rgb}{1.000000,0.576471,0.309804}%
\pgfsetfillcolor{currentfill}%
\pgfsetlinewidth{1.003750pt}%
\definecolor{currentstroke}{rgb}{1.000000,0.576471,0.309804}%
\pgfsetstrokecolor{currentstroke}%
\pgfsetdash{}{0pt}%
\pgfsys@defobject{currentmarker}{\pgfqpoint{-0.033333in}{-0.033333in}}{\pgfqpoint{0.033333in}{0.033333in}}{%
\pgfpathmoveto{\pgfqpoint{0.000000in}{-0.033333in}}%
\pgfpathcurveto{\pgfqpoint{0.008840in}{-0.033333in}}{\pgfqpoint{0.017319in}{-0.029821in}}{\pgfqpoint{0.023570in}{-0.023570in}}%
\pgfpathcurveto{\pgfqpoint{0.029821in}{-0.017319in}}{\pgfqpoint{0.033333in}{-0.008840in}}{\pgfqpoint{0.033333in}{0.000000in}}%
\pgfpathcurveto{\pgfqpoint{0.033333in}{0.008840in}}{\pgfqpoint{0.029821in}{0.017319in}}{\pgfqpoint{0.023570in}{0.023570in}}%
\pgfpathcurveto{\pgfqpoint{0.017319in}{0.029821in}}{\pgfqpoint{0.008840in}{0.033333in}}{\pgfqpoint{0.000000in}{0.033333in}}%
\pgfpathcurveto{\pgfqpoint{-0.008840in}{0.033333in}}{\pgfqpoint{-0.017319in}{0.029821in}}{\pgfqpoint{-0.023570in}{0.023570in}}%
\pgfpathcurveto{\pgfqpoint{-0.029821in}{0.017319in}}{\pgfqpoint{-0.033333in}{0.008840in}}{\pgfqpoint{-0.033333in}{0.000000in}}%
\pgfpathcurveto{\pgfqpoint{-0.033333in}{-0.008840in}}{\pgfqpoint{-0.029821in}{-0.017319in}}{\pgfqpoint{-0.023570in}{-0.023570in}}%
\pgfpathcurveto{\pgfqpoint{-0.017319in}{-0.029821in}}{\pgfqpoint{-0.008840in}{-0.033333in}}{\pgfqpoint{0.000000in}{-0.033333in}}%
\pgfpathlineto{\pgfqpoint{0.000000in}{-0.033333in}}%
\pgfpathclose%
\pgfusepath{stroke,fill}%
}%
\begin{pgfscope}%
\pgfsys@transformshift{3.747720in}{2.776663in}%
\pgfsys@useobject{currentmarker}{}%
\end{pgfscope}%
\end{pgfscope}%
\begin{pgfscope}%
\pgfpathrectangle{\pgfqpoint{0.765000in}{0.660000in}}{\pgfqpoint{4.620000in}{4.620000in}}%
\pgfusepath{clip}%
\pgfsetrectcap%
\pgfsetroundjoin%
\pgfsetlinewidth{1.204500pt}%
\definecolor{currentstroke}{rgb}{1.000000,0.576471,0.309804}%
\pgfsetstrokecolor{currentstroke}%
\pgfsetdash{}{0pt}%
\pgfpathmoveto{\pgfqpoint{5.200470in}{3.192045in}}%
\pgfusepath{stroke}%
\end{pgfscope}%
\begin{pgfscope}%
\pgfpathrectangle{\pgfqpoint{0.765000in}{0.660000in}}{\pgfqpoint{4.620000in}{4.620000in}}%
\pgfusepath{clip}%
\pgfsetbuttcap%
\pgfsetroundjoin%
\definecolor{currentfill}{rgb}{1.000000,0.576471,0.309804}%
\pgfsetfillcolor{currentfill}%
\pgfsetlinewidth{1.003750pt}%
\definecolor{currentstroke}{rgb}{1.000000,0.576471,0.309804}%
\pgfsetstrokecolor{currentstroke}%
\pgfsetdash{}{0pt}%
\pgfsys@defobject{currentmarker}{\pgfqpoint{-0.033333in}{-0.033333in}}{\pgfqpoint{0.033333in}{0.033333in}}{%
\pgfpathmoveto{\pgfqpoint{0.000000in}{-0.033333in}}%
\pgfpathcurveto{\pgfqpoint{0.008840in}{-0.033333in}}{\pgfqpoint{0.017319in}{-0.029821in}}{\pgfqpoint{0.023570in}{-0.023570in}}%
\pgfpathcurveto{\pgfqpoint{0.029821in}{-0.017319in}}{\pgfqpoint{0.033333in}{-0.008840in}}{\pgfqpoint{0.033333in}{0.000000in}}%
\pgfpathcurveto{\pgfqpoint{0.033333in}{0.008840in}}{\pgfqpoint{0.029821in}{0.017319in}}{\pgfqpoint{0.023570in}{0.023570in}}%
\pgfpathcurveto{\pgfqpoint{0.017319in}{0.029821in}}{\pgfqpoint{0.008840in}{0.033333in}}{\pgfqpoint{0.000000in}{0.033333in}}%
\pgfpathcurveto{\pgfqpoint{-0.008840in}{0.033333in}}{\pgfqpoint{-0.017319in}{0.029821in}}{\pgfqpoint{-0.023570in}{0.023570in}}%
\pgfpathcurveto{\pgfqpoint{-0.029821in}{0.017319in}}{\pgfqpoint{-0.033333in}{0.008840in}}{\pgfqpoint{-0.033333in}{0.000000in}}%
\pgfpathcurveto{\pgfqpoint{-0.033333in}{-0.008840in}}{\pgfqpoint{-0.029821in}{-0.017319in}}{\pgfqpoint{-0.023570in}{-0.023570in}}%
\pgfpathcurveto{\pgfqpoint{-0.017319in}{-0.029821in}}{\pgfqpoint{-0.008840in}{-0.033333in}}{\pgfqpoint{0.000000in}{-0.033333in}}%
\pgfpathlineto{\pgfqpoint{0.000000in}{-0.033333in}}%
\pgfpathclose%
\pgfusepath{stroke,fill}%
}%
\begin{pgfscope}%
\pgfsys@transformshift{5.200470in}{3.192045in}%
\pgfsys@useobject{currentmarker}{}%
\end{pgfscope}%
\end{pgfscope}%
\begin{pgfscope}%
\pgfpathrectangle{\pgfqpoint{0.765000in}{0.660000in}}{\pgfqpoint{4.620000in}{4.620000in}}%
\pgfusepath{clip}%
\pgfsetrectcap%
\pgfsetroundjoin%
\pgfsetlinewidth{1.204500pt}%
\definecolor{currentstroke}{rgb}{1.000000,0.576471,0.309804}%
\pgfsetstrokecolor{currentstroke}%
\pgfsetdash{}{0pt}%
\pgfpathmoveto{\pgfqpoint{3.930388in}{2.777182in}}%
\pgfusepath{stroke}%
\end{pgfscope}%
\begin{pgfscope}%
\pgfpathrectangle{\pgfqpoint{0.765000in}{0.660000in}}{\pgfqpoint{4.620000in}{4.620000in}}%
\pgfusepath{clip}%
\pgfsetbuttcap%
\pgfsetroundjoin%
\definecolor{currentfill}{rgb}{1.000000,0.576471,0.309804}%
\pgfsetfillcolor{currentfill}%
\pgfsetlinewidth{1.003750pt}%
\definecolor{currentstroke}{rgb}{1.000000,0.576471,0.309804}%
\pgfsetstrokecolor{currentstroke}%
\pgfsetdash{}{0pt}%
\pgfsys@defobject{currentmarker}{\pgfqpoint{-0.033333in}{-0.033333in}}{\pgfqpoint{0.033333in}{0.033333in}}{%
\pgfpathmoveto{\pgfqpoint{0.000000in}{-0.033333in}}%
\pgfpathcurveto{\pgfqpoint{0.008840in}{-0.033333in}}{\pgfqpoint{0.017319in}{-0.029821in}}{\pgfqpoint{0.023570in}{-0.023570in}}%
\pgfpathcurveto{\pgfqpoint{0.029821in}{-0.017319in}}{\pgfqpoint{0.033333in}{-0.008840in}}{\pgfqpoint{0.033333in}{0.000000in}}%
\pgfpathcurveto{\pgfqpoint{0.033333in}{0.008840in}}{\pgfqpoint{0.029821in}{0.017319in}}{\pgfqpoint{0.023570in}{0.023570in}}%
\pgfpathcurveto{\pgfqpoint{0.017319in}{0.029821in}}{\pgfqpoint{0.008840in}{0.033333in}}{\pgfqpoint{0.000000in}{0.033333in}}%
\pgfpathcurveto{\pgfqpoint{-0.008840in}{0.033333in}}{\pgfqpoint{-0.017319in}{0.029821in}}{\pgfqpoint{-0.023570in}{0.023570in}}%
\pgfpathcurveto{\pgfqpoint{-0.029821in}{0.017319in}}{\pgfqpoint{-0.033333in}{0.008840in}}{\pgfqpoint{-0.033333in}{0.000000in}}%
\pgfpathcurveto{\pgfqpoint{-0.033333in}{-0.008840in}}{\pgfqpoint{-0.029821in}{-0.017319in}}{\pgfqpoint{-0.023570in}{-0.023570in}}%
\pgfpathcurveto{\pgfqpoint{-0.017319in}{-0.029821in}}{\pgfqpoint{-0.008840in}{-0.033333in}}{\pgfqpoint{0.000000in}{-0.033333in}}%
\pgfpathlineto{\pgfqpoint{0.000000in}{-0.033333in}}%
\pgfpathclose%
\pgfusepath{stroke,fill}%
}%
\begin{pgfscope}%
\pgfsys@transformshift{3.930388in}{2.777182in}%
\pgfsys@useobject{currentmarker}{}%
\end{pgfscope}%
\end{pgfscope}%
\begin{pgfscope}%
\pgfpathrectangle{\pgfqpoint{0.765000in}{0.660000in}}{\pgfqpoint{4.620000in}{4.620000in}}%
\pgfusepath{clip}%
\pgfsetrectcap%
\pgfsetroundjoin%
\pgfsetlinewidth{1.204500pt}%
\definecolor{currentstroke}{rgb}{1.000000,0.576471,0.309804}%
\pgfsetstrokecolor{currentstroke}%
\pgfsetdash{}{0pt}%
\pgfpathmoveto{\pgfqpoint{3.208977in}{2.562479in}}%
\pgfusepath{stroke}%
\end{pgfscope}%
\begin{pgfscope}%
\pgfpathrectangle{\pgfqpoint{0.765000in}{0.660000in}}{\pgfqpoint{4.620000in}{4.620000in}}%
\pgfusepath{clip}%
\pgfsetbuttcap%
\pgfsetroundjoin%
\definecolor{currentfill}{rgb}{1.000000,0.576471,0.309804}%
\pgfsetfillcolor{currentfill}%
\pgfsetlinewidth{1.003750pt}%
\definecolor{currentstroke}{rgb}{1.000000,0.576471,0.309804}%
\pgfsetstrokecolor{currentstroke}%
\pgfsetdash{}{0pt}%
\pgfsys@defobject{currentmarker}{\pgfqpoint{-0.033333in}{-0.033333in}}{\pgfqpoint{0.033333in}{0.033333in}}{%
\pgfpathmoveto{\pgfqpoint{0.000000in}{-0.033333in}}%
\pgfpathcurveto{\pgfqpoint{0.008840in}{-0.033333in}}{\pgfqpoint{0.017319in}{-0.029821in}}{\pgfqpoint{0.023570in}{-0.023570in}}%
\pgfpathcurveto{\pgfqpoint{0.029821in}{-0.017319in}}{\pgfqpoint{0.033333in}{-0.008840in}}{\pgfqpoint{0.033333in}{0.000000in}}%
\pgfpathcurveto{\pgfqpoint{0.033333in}{0.008840in}}{\pgfqpoint{0.029821in}{0.017319in}}{\pgfqpoint{0.023570in}{0.023570in}}%
\pgfpathcurveto{\pgfqpoint{0.017319in}{0.029821in}}{\pgfqpoint{0.008840in}{0.033333in}}{\pgfqpoint{0.000000in}{0.033333in}}%
\pgfpathcurveto{\pgfqpoint{-0.008840in}{0.033333in}}{\pgfqpoint{-0.017319in}{0.029821in}}{\pgfqpoint{-0.023570in}{0.023570in}}%
\pgfpathcurveto{\pgfqpoint{-0.029821in}{0.017319in}}{\pgfqpoint{-0.033333in}{0.008840in}}{\pgfqpoint{-0.033333in}{0.000000in}}%
\pgfpathcurveto{\pgfqpoint{-0.033333in}{-0.008840in}}{\pgfqpoint{-0.029821in}{-0.017319in}}{\pgfqpoint{-0.023570in}{-0.023570in}}%
\pgfpathcurveto{\pgfqpoint{-0.017319in}{-0.029821in}}{\pgfqpoint{-0.008840in}{-0.033333in}}{\pgfqpoint{0.000000in}{-0.033333in}}%
\pgfpathlineto{\pgfqpoint{0.000000in}{-0.033333in}}%
\pgfpathclose%
\pgfusepath{stroke,fill}%
}%
\begin{pgfscope}%
\pgfsys@transformshift{3.208977in}{2.562479in}%
\pgfsys@useobject{currentmarker}{}%
\end{pgfscope}%
\end{pgfscope}%
\begin{pgfscope}%
\pgfpathrectangle{\pgfqpoint{0.765000in}{0.660000in}}{\pgfqpoint{4.620000in}{4.620000in}}%
\pgfusepath{clip}%
\pgfsetrectcap%
\pgfsetroundjoin%
\pgfsetlinewidth{1.204500pt}%
\definecolor{currentstroke}{rgb}{1.000000,0.576471,0.309804}%
\pgfsetstrokecolor{currentstroke}%
\pgfsetdash{}{0pt}%
\pgfpathmoveto{\pgfqpoint{2.863200in}{2.554317in}}%
\pgfusepath{stroke}%
\end{pgfscope}%
\begin{pgfscope}%
\pgfpathrectangle{\pgfqpoint{0.765000in}{0.660000in}}{\pgfqpoint{4.620000in}{4.620000in}}%
\pgfusepath{clip}%
\pgfsetbuttcap%
\pgfsetroundjoin%
\definecolor{currentfill}{rgb}{1.000000,0.576471,0.309804}%
\pgfsetfillcolor{currentfill}%
\pgfsetlinewidth{1.003750pt}%
\definecolor{currentstroke}{rgb}{1.000000,0.576471,0.309804}%
\pgfsetstrokecolor{currentstroke}%
\pgfsetdash{}{0pt}%
\pgfsys@defobject{currentmarker}{\pgfqpoint{-0.033333in}{-0.033333in}}{\pgfqpoint{0.033333in}{0.033333in}}{%
\pgfpathmoveto{\pgfqpoint{0.000000in}{-0.033333in}}%
\pgfpathcurveto{\pgfqpoint{0.008840in}{-0.033333in}}{\pgfqpoint{0.017319in}{-0.029821in}}{\pgfqpoint{0.023570in}{-0.023570in}}%
\pgfpathcurveto{\pgfqpoint{0.029821in}{-0.017319in}}{\pgfqpoint{0.033333in}{-0.008840in}}{\pgfqpoint{0.033333in}{0.000000in}}%
\pgfpathcurveto{\pgfqpoint{0.033333in}{0.008840in}}{\pgfqpoint{0.029821in}{0.017319in}}{\pgfqpoint{0.023570in}{0.023570in}}%
\pgfpathcurveto{\pgfqpoint{0.017319in}{0.029821in}}{\pgfqpoint{0.008840in}{0.033333in}}{\pgfqpoint{0.000000in}{0.033333in}}%
\pgfpathcurveto{\pgfqpoint{-0.008840in}{0.033333in}}{\pgfqpoint{-0.017319in}{0.029821in}}{\pgfqpoint{-0.023570in}{0.023570in}}%
\pgfpathcurveto{\pgfqpoint{-0.029821in}{0.017319in}}{\pgfqpoint{-0.033333in}{0.008840in}}{\pgfqpoint{-0.033333in}{0.000000in}}%
\pgfpathcurveto{\pgfqpoint{-0.033333in}{-0.008840in}}{\pgfqpoint{-0.029821in}{-0.017319in}}{\pgfqpoint{-0.023570in}{-0.023570in}}%
\pgfpathcurveto{\pgfqpoint{-0.017319in}{-0.029821in}}{\pgfqpoint{-0.008840in}{-0.033333in}}{\pgfqpoint{0.000000in}{-0.033333in}}%
\pgfpathlineto{\pgfqpoint{0.000000in}{-0.033333in}}%
\pgfpathclose%
\pgfusepath{stroke,fill}%
}%
\begin{pgfscope}%
\pgfsys@transformshift{2.863200in}{2.554317in}%
\pgfsys@useobject{currentmarker}{}%
\end{pgfscope}%
\end{pgfscope}%
\begin{pgfscope}%
\pgfpathrectangle{\pgfqpoint{0.765000in}{0.660000in}}{\pgfqpoint{4.620000in}{4.620000in}}%
\pgfusepath{clip}%
\pgfsetrectcap%
\pgfsetroundjoin%
\pgfsetlinewidth{1.204500pt}%
\definecolor{currentstroke}{rgb}{1.000000,0.576471,0.309804}%
\pgfsetstrokecolor{currentstroke}%
\pgfsetdash{}{0pt}%
\pgfpathmoveto{\pgfqpoint{4.540845in}{2.859479in}}%
\pgfusepath{stroke}%
\end{pgfscope}%
\begin{pgfscope}%
\pgfpathrectangle{\pgfqpoint{0.765000in}{0.660000in}}{\pgfqpoint{4.620000in}{4.620000in}}%
\pgfusepath{clip}%
\pgfsetbuttcap%
\pgfsetroundjoin%
\definecolor{currentfill}{rgb}{1.000000,0.576471,0.309804}%
\pgfsetfillcolor{currentfill}%
\pgfsetlinewidth{1.003750pt}%
\definecolor{currentstroke}{rgb}{1.000000,0.576471,0.309804}%
\pgfsetstrokecolor{currentstroke}%
\pgfsetdash{}{0pt}%
\pgfsys@defobject{currentmarker}{\pgfqpoint{-0.033333in}{-0.033333in}}{\pgfqpoint{0.033333in}{0.033333in}}{%
\pgfpathmoveto{\pgfqpoint{0.000000in}{-0.033333in}}%
\pgfpathcurveto{\pgfqpoint{0.008840in}{-0.033333in}}{\pgfqpoint{0.017319in}{-0.029821in}}{\pgfqpoint{0.023570in}{-0.023570in}}%
\pgfpathcurveto{\pgfqpoint{0.029821in}{-0.017319in}}{\pgfqpoint{0.033333in}{-0.008840in}}{\pgfqpoint{0.033333in}{0.000000in}}%
\pgfpathcurveto{\pgfqpoint{0.033333in}{0.008840in}}{\pgfqpoint{0.029821in}{0.017319in}}{\pgfqpoint{0.023570in}{0.023570in}}%
\pgfpathcurveto{\pgfqpoint{0.017319in}{0.029821in}}{\pgfqpoint{0.008840in}{0.033333in}}{\pgfqpoint{0.000000in}{0.033333in}}%
\pgfpathcurveto{\pgfqpoint{-0.008840in}{0.033333in}}{\pgfqpoint{-0.017319in}{0.029821in}}{\pgfqpoint{-0.023570in}{0.023570in}}%
\pgfpathcurveto{\pgfqpoint{-0.029821in}{0.017319in}}{\pgfqpoint{-0.033333in}{0.008840in}}{\pgfqpoint{-0.033333in}{0.000000in}}%
\pgfpathcurveto{\pgfqpoint{-0.033333in}{-0.008840in}}{\pgfqpoint{-0.029821in}{-0.017319in}}{\pgfqpoint{-0.023570in}{-0.023570in}}%
\pgfpathcurveto{\pgfqpoint{-0.017319in}{-0.029821in}}{\pgfqpoint{-0.008840in}{-0.033333in}}{\pgfqpoint{0.000000in}{-0.033333in}}%
\pgfpathlineto{\pgfqpoint{0.000000in}{-0.033333in}}%
\pgfpathclose%
\pgfusepath{stroke,fill}%
}%
\begin{pgfscope}%
\pgfsys@transformshift{4.540845in}{2.859479in}%
\pgfsys@useobject{currentmarker}{}%
\end{pgfscope}%
\end{pgfscope}%
\begin{pgfscope}%
\pgfpathrectangle{\pgfqpoint{0.765000in}{0.660000in}}{\pgfqpoint{4.620000in}{4.620000in}}%
\pgfusepath{clip}%
\pgfsetrectcap%
\pgfsetroundjoin%
\pgfsetlinewidth{1.204500pt}%
\definecolor{currentstroke}{rgb}{1.000000,0.576471,0.309804}%
\pgfsetstrokecolor{currentstroke}%
\pgfsetdash{}{0pt}%
\pgfpathmoveto{\pgfqpoint{2.992984in}{3.145810in}}%
\pgfusepath{stroke}%
\end{pgfscope}%
\begin{pgfscope}%
\pgfpathrectangle{\pgfqpoint{0.765000in}{0.660000in}}{\pgfqpoint{4.620000in}{4.620000in}}%
\pgfusepath{clip}%
\pgfsetbuttcap%
\pgfsetroundjoin%
\definecolor{currentfill}{rgb}{1.000000,0.576471,0.309804}%
\pgfsetfillcolor{currentfill}%
\pgfsetlinewidth{1.003750pt}%
\definecolor{currentstroke}{rgb}{1.000000,0.576471,0.309804}%
\pgfsetstrokecolor{currentstroke}%
\pgfsetdash{}{0pt}%
\pgfsys@defobject{currentmarker}{\pgfqpoint{-0.033333in}{-0.033333in}}{\pgfqpoint{0.033333in}{0.033333in}}{%
\pgfpathmoveto{\pgfqpoint{0.000000in}{-0.033333in}}%
\pgfpathcurveto{\pgfqpoint{0.008840in}{-0.033333in}}{\pgfqpoint{0.017319in}{-0.029821in}}{\pgfqpoint{0.023570in}{-0.023570in}}%
\pgfpathcurveto{\pgfqpoint{0.029821in}{-0.017319in}}{\pgfqpoint{0.033333in}{-0.008840in}}{\pgfqpoint{0.033333in}{0.000000in}}%
\pgfpathcurveto{\pgfqpoint{0.033333in}{0.008840in}}{\pgfqpoint{0.029821in}{0.017319in}}{\pgfqpoint{0.023570in}{0.023570in}}%
\pgfpathcurveto{\pgfqpoint{0.017319in}{0.029821in}}{\pgfqpoint{0.008840in}{0.033333in}}{\pgfqpoint{0.000000in}{0.033333in}}%
\pgfpathcurveto{\pgfqpoint{-0.008840in}{0.033333in}}{\pgfqpoint{-0.017319in}{0.029821in}}{\pgfqpoint{-0.023570in}{0.023570in}}%
\pgfpathcurveto{\pgfqpoint{-0.029821in}{0.017319in}}{\pgfqpoint{-0.033333in}{0.008840in}}{\pgfqpoint{-0.033333in}{0.000000in}}%
\pgfpathcurveto{\pgfqpoint{-0.033333in}{-0.008840in}}{\pgfqpoint{-0.029821in}{-0.017319in}}{\pgfqpoint{-0.023570in}{-0.023570in}}%
\pgfpathcurveto{\pgfqpoint{-0.017319in}{-0.029821in}}{\pgfqpoint{-0.008840in}{-0.033333in}}{\pgfqpoint{0.000000in}{-0.033333in}}%
\pgfpathlineto{\pgfqpoint{0.000000in}{-0.033333in}}%
\pgfpathclose%
\pgfusepath{stroke,fill}%
}%
\begin{pgfscope}%
\pgfsys@transformshift{2.992984in}{3.145810in}%
\pgfsys@useobject{currentmarker}{}%
\end{pgfscope}%
\end{pgfscope}%
\begin{pgfscope}%
\pgfpathrectangle{\pgfqpoint{0.765000in}{0.660000in}}{\pgfqpoint{4.620000in}{4.620000in}}%
\pgfusepath{clip}%
\pgfsetrectcap%
\pgfsetroundjoin%
\pgfsetlinewidth{1.204500pt}%
\definecolor{currentstroke}{rgb}{1.000000,0.576471,0.309804}%
\pgfsetstrokecolor{currentstroke}%
\pgfsetdash{}{0pt}%
\pgfpathmoveto{\pgfqpoint{2.953476in}{3.102878in}}%
\pgfusepath{stroke}%
\end{pgfscope}%
\begin{pgfscope}%
\pgfpathrectangle{\pgfqpoint{0.765000in}{0.660000in}}{\pgfqpoint{4.620000in}{4.620000in}}%
\pgfusepath{clip}%
\pgfsetbuttcap%
\pgfsetroundjoin%
\definecolor{currentfill}{rgb}{1.000000,0.576471,0.309804}%
\pgfsetfillcolor{currentfill}%
\pgfsetlinewidth{1.003750pt}%
\definecolor{currentstroke}{rgb}{1.000000,0.576471,0.309804}%
\pgfsetstrokecolor{currentstroke}%
\pgfsetdash{}{0pt}%
\pgfsys@defobject{currentmarker}{\pgfqpoint{-0.033333in}{-0.033333in}}{\pgfqpoint{0.033333in}{0.033333in}}{%
\pgfpathmoveto{\pgfqpoint{0.000000in}{-0.033333in}}%
\pgfpathcurveto{\pgfqpoint{0.008840in}{-0.033333in}}{\pgfqpoint{0.017319in}{-0.029821in}}{\pgfqpoint{0.023570in}{-0.023570in}}%
\pgfpathcurveto{\pgfqpoint{0.029821in}{-0.017319in}}{\pgfqpoint{0.033333in}{-0.008840in}}{\pgfqpoint{0.033333in}{0.000000in}}%
\pgfpathcurveto{\pgfqpoint{0.033333in}{0.008840in}}{\pgfqpoint{0.029821in}{0.017319in}}{\pgfqpoint{0.023570in}{0.023570in}}%
\pgfpathcurveto{\pgfqpoint{0.017319in}{0.029821in}}{\pgfqpoint{0.008840in}{0.033333in}}{\pgfqpoint{0.000000in}{0.033333in}}%
\pgfpathcurveto{\pgfqpoint{-0.008840in}{0.033333in}}{\pgfqpoint{-0.017319in}{0.029821in}}{\pgfqpoint{-0.023570in}{0.023570in}}%
\pgfpathcurveto{\pgfqpoint{-0.029821in}{0.017319in}}{\pgfqpoint{-0.033333in}{0.008840in}}{\pgfqpoint{-0.033333in}{0.000000in}}%
\pgfpathcurveto{\pgfqpoint{-0.033333in}{-0.008840in}}{\pgfqpoint{-0.029821in}{-0.017319in}}{\pgfqpoint{-0.023570in}{-0.023570in}}%
\pgfpathcurveto{\pgfqpoint{-0.017319in}{-0.029821in}}{\pgfqpoint{-0.008840in}{-0.033333in}}{\pgfqpoint{0.000000in}{-0.033333in}}%
\pgfpathlineto{\pgfqpoint{0.000000in}{-0.033333in}}%
\pgfpathclose%
\pgfusepath{stroke,fill}%
}%
\begin{pgfscope}%
\pgfsys@transformshift{2.953476in}{3.102878in}%
\pgfsys@useobject{currentmarker}{}%
\end{pgfscope}%
\end{pgfscope}%
\begin{pgfscope}%
\pgfpathrectangle{\pgfqpoint{0.765000in}{0.660000in}}{\pgfqpoint{4.620000in}{4.620000in}}%
\pgfusepath{clip}%
\pgfsetrectcap%
\pgfsetroundjoin%
\pgfsetlinewidth{1.204500pt}%
\definecolor{currentstroke}{rgb}{1.000000,0.576471,0.309804}%
\pgfsetstrokecolor{currentstroke}%
\pgfsetdash{}{0pt}%
\pgfpathmoveto{\pgfqpoint{2.780619in}{3.088399in}}%
\pgfusepath{stroke}%
\end{pgfscope}%
\begin{pgfscope}%
\pgfpathrectangle{\pgfqpoint{0.765000in}{0.660000in}}{\pgfqpoint{4.620000in}{4.620000in}}%
\pgfusepath{clip}%
\pgfsetbuttcap%
\pgfsetroundjoin%
\definecolor{currentfill}{rgb}{1.000000,0.576471,0.309804}%
\pgfsetfillcolor{currentfill}%
\pgfsetlinewidth{1.003750pt}%
\definecolor{currentstroke}{rgb}{1.000000,0.576471,0.309804}%
\pgfsetstrokecolor{currentstroke}%
\pgfsetdash{}{0pt}%
\pgfsys@defobject{currentmarker}{\pgfqpoint{-0.033333in}{-0.033333in}}{\pgfqpoint{0.033333in}{0.033333in}}{%
\pgfpathmoveto{\pgfqpoint{0.000000in}{-0.033333in}}%
\pgfpathcurveto{\pgfqpoint{0.008840in}{-0.033333in}}{\pgfqpoint{0.017319in}{-0.029821in}}{\pgfqpoint{0.023570in}{-0.023570in}}%
\pgfpathcurveto{\pgfqpoint{0.029821in}{-0.017319in}}{\pgfqpoint{0.033333in}{-0.008840in}}{\pgfqpoint{0.033333in}{0.000000in}}%
\pgfpathcurveto{\pgfqpoint{0.033333in}{0.008840in}}{\pgfqpoint{0.029821in}{0.017319in}}{\pgfqpoint{0.023570in}{0.023570in}}%
\pgfpathcurveto{\pgfqpoint{0.017319in}{0.029821in}}{\pgfqpoint{0.008840in}{0.033333in}}{\pgfqpoint{0.000000in}{0.033333in}}%
\pgfpathcurveto{\pgfqpoint{-0.008840in}{0.033333in}}{\pgfqpoint{-0.017319in}{0.029821in}}{\pgfqpoint{-0.023570in}{0.023570in}}%
\pgfpathcurveto{\pgfqpoint{-0.029821in}{0.017319in}}{\pgfqpoint{-0.033333in}{0.008840in}}{\pgfqpoint{-0.033333in}{0.000000in}}%
\pgfpathcurveto{\pgfqpoint{-0.033333in}{-0.008840in}}{\pgfqpoint{-0.029821in}{-0.017319in}}{\pgfqpoint{-0.023570in}{-0.023570in}}%
\pgfpathcurveto{\pgfqpoint{-0.017319in}{-0.029821in}}{\pgfqpoint{-0.008840in}{-0.033333in}}{\pgfqpoint{0.000000in}{-0.033333in}}%
\pgfpathlineto{\pgfqpoint{0.000000in}{-0.033333in}}%
\pgfpathclose%
\pgfusepath{stroke,fill}%
}%
\begin{pgfscope}%
\pgfsys@transformshift{2.780619in}{3.088399in}%
\pgfsys@useobject{currentmarker}{}%
\end{pgfscope}%
\end{pgfscope}%
\begin{pgfscope}%
\pgfpathrectangle{\pgfqpoint{0.765000in}{0.660000in}}{\pgfqpoint{4.620000in}{4.620000in}}%
\pgfusepath{clip}%
\pgfsetrectcap%
\pgfsetroundjoin%
\pgfsetlinewidth{1.204500pt}%
\definecolor{currentstroke}{rgb}{1.000000,0.576471,0.309804}%
\pgfsetstrokecolor{currentstroke}%
\pgfsetdash{}{0pt}%
\pgfpathmoveto{\pgfqpoint{3.554378in}{2.330422in}}%
\pgfusepath{stroke}%
\end{pgfscope}%
\begin{pgfscope}%
\pgfpathrectangle{\pgfqpoint{0.765000in}{0.660000in}}{\pgfqpoint{4.620000in}{4.620000in}}%
\pgfusepath{clip}%
\pgfsetbuttcap%
\pgfsetroundjoin%
\definecolor{currentfill}{rgb}{1.000000,0.576471,0.309804}%
\pgfsetfillcolor{currentfill}%
\pgfsetlinewidth{1.003750pt}%
\definecolor{currentstroke}{rgb}{1.000000,0.576471,0.309804}%
\pgfsetstrokecolor{currentstroke}%
\pgfsetdash{}{0pt}%
\pgfsys@defobject{currentmarker}{\pgfqpoint{-0.033333in}{-0.033333in}}{\pgfqpoint{0.033333in}{0.033333in}}{%
\pgfpathmoveto{\pgfqpoint{0.000000in}{-0.033333in}}%
\pgfpathcurveto{\pgfqpoint{0.008840in}{-0.033333in}}{\pgfqpoint{0.017319in}{-0.029821in}}{\pgfqpoint{0.023570in}{-0.023570in}}%
\pgfpathcurveto{\pgfqpoint{0.029821in}{-0.017319in}}{\pgfqpoint{0.033333in}{-0.008840in}}{\pgfqpoint{0.033333in}{0.000000in}}%
\pgfpathcurveto{\pgfqpoint{0.033333in}{0.008840in}}{\pgfqpoint{0.029821in}{0.017319in}}{\pgfqpoint{0.023570in}{0.023570in}}%
\pgfpathcurveto{\pgfqpoint{0.017319in}{0.029821in}}{\pgfqpoint{0.008840in}{0.033333in}}{\pgfqpoint{0.000000in}{0.033333in}}%
\pgfpathcurveto{\pgfqpoint{-0.008840in}{0.033333in}}{\pgfqpoint{-0.017319in}{0.029821in}}{\pgfqpoint{-0.023570in}{0.023570in}}%
\pgfpathcurveto{\pgfqpoint{-0.029821in}{0.017319in}}{\pgfqpoint{-0.033333in}{0.008840in}}{\pgfqpoint{-0.033333in}{0.000000in}}%
\pgfpathcurveto{\pgfqpoint{-0.033333in}{-0.008840in}}{\pgfqpoint{-0.029821in}{-0.017319in}}{\pgfqpoint{-0.023570in}{-0.023570in}}%
\pgfpathcurveto{\pgfqpoint{-0.017319in}{-0.029821in}}{\pgfqpoint{-0.008840in}{-0.033333in}}{\pgfqpoint{0.000000in}{-0.033333in}}%
\pgfpathlineto{\pgfqpoint{0.000000in}{-0.033333in}}%
\pgfpathclose%
\pgfusepath{stroke,fill}%
}%
\begin{pgfscope}%
\pgfsys@transformshift{3.554378in}{2.330422in}%
\pgfsys@useobject{currentmarker}{}%
\end{pgfscope}%
\end{pgfscope}%
\begin{pgfscope}%
\pgfpathrectangle{\pgfqpoint{0.765000in}{0.660000in}}{\pgfqpoint{4.620000in}{4.620000in}}%
\pgfusepath{clip}%
\pgfsetrectcap%
\pgfsetroundjoin%
\pgfsetlinewidth{1.204500pt}%
\definecolor{currentstroke}{rgb}{1.000000,0.576471,0.309804}%
\pgfsetstrokecolor{currentstroke}%
\pgfsetdash{}{0pt}%
\pgfpathmoveto{\pgfqpoint{2.999020in}{2.458884in}}%
\pgfusepath{stroke}%
\end{pgfscope}%
\begin{pgfscope}%
\pgfpathrectangle{\pgfqpoint{0.765000in}{0.660000in}}{\pgfqpoint{4.620000in}{4.620000in}}%
\pgfusepath{clip}%
\pgfsetbuttcap%
\pgfsetroundjoin%
\definecolor{currentfill}{rgb}{1.000000,0.576471,0.309804}%
\pgfsetfillcolor{currentfill}%
\pgfsetlinewidth{1.003750pt}%
\definecolor{currentstroke}{rgb}{1.000000,0.576471,0.309804}%
\pgfsetstrokecolor{currentstroke}%
\pgfsetdash{}{0pt}%
\pgfsys@defobject{currentmarker}{\pgfqpoint{-0.033333in}{-0.033333in}}{\pgfqpoint{0.033333in}{0.033333in}}{%
\pgfpathmoveto{\pgfqpoint{0.000000in}{-0.033333in}}%
\pgfpathcurveto{\pgfqpoint{0.008840in}{-0.033333in}}{\pgfqpoint{0.017319in}{-0.029821in}}{\pgfqpoint{0.023570in}{-0.023570in}}%
\pgfpathcurveto{\pgfqpoint{0.029821in}{-0.017319in}}{\pgfqpoint{0.033333in}{-0.008840in}}{\pgfqpoint{0.033333in}{0.000000in}}%
\pgfpathcurveto{\pgfqpoint{0.033333in}{0.008840in}}{\pgfqpoint{0.029821in}{0.017319in}}{\pgfqpoint{0.023570in}{0.023570in}}%
\pgfpathcurveto{\pgfqpoint{0.017319in}{0.029821in}}{\pgfqpoint{0.008840in}{0.033333in}}{\pgfqpoint{0.000000in}{0.033333in}}%
\pgfpathcurveto{\pgfqpoint{-0.008840in}{0.033333in}}{\pgfqpoint{-0.017319in}{0.029821in}}{\pgfqpoint{-0.023570in}{0.023570in}}%
\pgfpathcurveto{\pgfqpoint{-0.029821in}{0.017319in}}{\pgfqpoint{-0.033333in}{0.008840in}}{\pgfqpoint{-0.033333in}{0.000000in}}%
\pgfpathcurveto{\pgfqpoint{-0.033333in}{-0.008840in}}{\pgfqpoint{-0.029821in}{-0.017319in}}{\pgfqpoint{-0.023570in}{-0.023570in}}%
\pgfpathcurveto{\pgfqpoint{-0.017319in}{-0.029821in}}{\pgfqpoint{-0.008840in}{-0.033333in}}{\pgfqpoint{0.000000in}{-0.033333in}}%
\pgfpathlineto{\pgfqpoint{0.000000in}{-0.033333in}}%
\pgfpathclose%
\pgfusepath{stroke,fill}%
}%
\begin{pgfscope}%
\pgfsys@transformshift{2.999020in}{2.458884in}%
\pgfsys@useobject{currentmarker}{}%
\end{pgfscope}%
\end{pgfscope}%
\begin{pgfscope}%
\pgfpathrectangle{\pgfqpoint{0.765000in}{0.660000in}}{\pgfqpoint{4.620000in}{4.620000in}}%
\pgfusepath{clip}%
\pgfsetrectcap%
\pgfsetroundjoin%
\pgfsetlinewidth{1.204500pt}%
\definecolor{currentstroke}{rgb}{1.000000,0.576471,0.309804}%
\pgfsetstrokecolor{currentstroke}%
\pgfsetdash{}{0pt}%
\pgfpathmoveto{\pgfqpoint{4.830435in}{3.120492in}}%
\pgfusepath{stroke}%
\end{pgfscope}%
\begin{pgfscope}%
\pgfpathrectangle{\pgfqpoint{0.765000in}{0.660000in}}{\pgfqpoint{4.620000in}{4.620000in}}%
\pgfusepath{clip}%
\pgfsetbuttcap%
\pgfsetroundjoin%
\definecolor{currentfill}{rgb}{1.000000,0.576471,0.309804}%
\pgfsetfillcolor{currentfill}%
\pgfsetlinewidth{1.003750pt}%
\definecolor{currentstroke}{rgb}{1.000000,0.576471,0.309804}%
\pgfsetstrokecolor{currentstroke}%
\pgfsetdash{}{0pt}%
\pgfsys@defobject{currentmarker}{\pgfqpoint{-0.033333in}{-0.033333in}}{\pgfqpoint{0.033333in}{0.033333in}}{%
\pgfpathmoveto{\pgfqpoint{0.000000in}{-0.033333in}}%
\pgfpathcurveto{\pgfqpoint{0.008840in}{-0.033333in}}{\pgfqpoint{0.017319in}{-0.029821in}}{\pgfqpoint{0.023570in}{-0.023570in}}%
\pgfpathcurveto{\pgfqpoint{0.029821in}{-0.017319in}}{\pgfqpoint{0.033333in}{-0.008840in}}{\pgfqpoint{0.033333in}{0.000000in}}%
\pgfpathcurveto{\pgfqpoint{0.033333in}{0.008840in}}{\pgfqpoint{0.029821in}{0.017319in}}{\pgfqpoint{0.023570in}{0.023570in}}%
\pgfpathcurveto{\pgfqpoint{0.017319in}{0.029821in}}{\pgfqpoint{0.008840in}{0.033333in}}{\pgfqpoint{0.000000in}{0.033333in}}%
\pgfpathcurveto{\pgfqpoint{-0.008840in}{0.033333in}}{\pgfqpoint{-0.017319in}{0.029821in}}{\pgfqpoint{-0.023570in}{0.023570in}}%
\pgfpathcurveto{\pgfqpoint{-0.029821in}{0.017319in}}{\pgfqpoint{-0.033333in}{0.008840in}}{\pgfqpoint{-0.033333in}{0.000000in}}%
\pgfpathcurveto{\pgfqpoint{-0.033333in}{-0.008840in}}{\pgfqpoint{-0.029821in}{-0.017319in}}{\pgfqpoint{-0.023570in}{-0.023570in}}%
\pgfpathcurveto{\pgfqpoint{-0.017319in}{-0.029821in}}{\pgfqpoint{-0.008840in}{-0.033333in}}{\pgfqpoint{0.000000in}{-0.033333in}}%
\pgfpathlineto{\pgfqpoint{0.000000in}{-0.033333in}}%
\pgfpathclose%
\pgfusepath{stroke,fill}%
}%
\begin{pgfscope}%
\pgfsys@transformshift{4.830435in}{3.120492in}%
\pgfsys@useobject{currentmarker}{}%
\end{pgfscope}%
\end{pgfscope}%
\begin{pgfscope}%
\pgfpathrectangle{\pgfqpoint{0.765000in}{0.660000in}}{\pgfqpoint{4.620000in}{4.620000in}}%
\pgfusepath{clip}%
\pgfsetrectcap%
\pgfsetroundjoin%
\pgfsetlinewidth{1.204500pt}%
\definecolor{currentstroke}{rgb}{1.000000,0.576471,0.309804}%
\pgfsetstrokecolor{currentstroke}%
\pgfsetdash{}{0pt}%
\pgfpathmoveto{\pgfqpoint{2.608591in}{2.511315in}}%
\pgfusepath{stroke}%
\end{pgfscope}%
\begin{pgfscope}%
\pgfpathrectangle{\pgfqpoint{0.765000in}{0.660000in}}{\pgfqpoint{4.620000in}{4.620000in}}%
\pgfusepath{clip}%
\pgfsetbuttcap%
\pgfsetroundjoin%
\definecolor{currentfill}{rgb}{1.000000,0.576471,0.309804}%
\pgfsetfillcolor{currentfill}%
\pgfsetlinewidth{1.003750pt}%
\definecolor{currentstroke}{rgb}{1.000000,0.576471,0.309804}%
\pgfsetstrokecolor{currentstroke}%
\pgfsetdash{}{0pt}%
\pgfsys@defobject{currentmarker}{\pgfqpoint{-0.033333in}{-0.033333in}}{\pgfqpoint{0.033333in}{0.033333in}}{%
\pgfpathmoveto{\pgfqpoint{0.000000in}{-0.033333in}}%
\pgfpathcurveto{\pgfqpoint{0.008840in}{-0.033333in}}{\pgfqpoint{0.017319in}{-0.029821in}}{\pgfqpoint{0.023570in}{-0.023570in}}%
\pgfpathcurveto{\pgfqpoint{0.029821in}{-0.017319in}}{\pgfqpoint{0.033333in}{-0.008840in}}{\pgfqpoint{0.033333in}{0.000000in}}%
\pgfpathcurveto{\pgfqpoint{0.033333in}{0.008840in}}{\pgfqpoint{0.029821in}{0.017319in}}{\pgfqpoint{0.023570in}{0.023570in}}%
\pgfpathcurveto{\pgfqpoint{0.017319in}{0.029821in}}{\pgfqpoint{0.008840in}{0.033333in}}{\pgfqpoint{0.000000in}{0.033333in}}%
\pgfpathcurveto{\pgfqpoint{-0.008840in}{0.033333in}}{\pgfqpoint{-0.017319in}{0.029821in}}{\pgfqpoint{-0.023570in}{0.023570in}}%
\pgfpathcurveto{\pgfqpoint{-0.029821in}{0.017319in}}{\pgfqpoint{-0.033333in}{0.008840in}}{\pgfqpoint{-0.033333in}{0.000000in}}%
\pgfpathcurveto{\pgfqpoint{-0.033333in}{-0.008840in}}{\pgfqpoint{-0.029821in}{-0.017319in}}{\pgfqpoint{-0.023570in}{-0.023570in}}%
\pgfpathcurveto{\pgfqpoint{-0.017319in}{-0.029821in}}{\pgfqpoint{-0.008840in}{-0.033333in}}{\pgfqpoint{0.000000in}{-0.033333in}}%
\pgfpathlineto{\pgfqpoint{0.000000in}{-0.033333in}}%
\pgfpathclose%
\pgfusepath{stroke,fill}%
}%
\begin{pgfscope}%
\pgfsys@transformshift{2.608591in}{2.511315in}%
\pgfsys@useobject{currentmarker}{}%
\end{pgfscope}%
\end{pgfscope}%
\begin{pgfscope}%
\pgfpathrectangle{\pgfqpoint{0.765000in}{0.660000in}}{\pgfqpoint{4.620000in}{4.620000in}}%
\pgfusepath{clip}%
\pgfsetrectcap%
\pgfsetroundjoin%
\pgfsetlinewidth{1.204500pt}%
\definecolor{currentstroke}{rgb}{1.000000,0.576471,0.309804}%
\pgfsetstrokecolor{currentstroke}%
\pgfsetdash{}{0pt}%
\pgfpathmoveto{\pgfqpoint{4.473821in}{2.926868in}}%
\pgfusepath{stroke}%
\end{pgfscope}%
\begin{pgfscope}%
\pgfpathrectangle{\pgfqpoint{0.765000in}{0.660000in}}{\pgfqpoint{4.620000in}{4.620000in}}%
\pgfusepath{clip}%
\pgfsetbuttcap%
\pgfsetroundjoin%
\definecolor{currentfill}{rgb}{1.000000,0.576471,0.309804}%
\pgfsetfillcolor{currentfill}%
\pgfsetlinewidth{1.003750pt}%
\definecolor{currentstroke}{rgb}{1.000000,0.576471,0.309804}%
\pgfsetstrokecolor{currentstroke}%
\pgfsetdash{}{0pt}%
\pgfsys@defobject{currentmarker}{\pgfqpoint{-0.033333in}{-0.033333in}}{\pgfqpoint{0.033333in}{0.033333in}}{%
\pgfpathmoveto{\pgfqpoint{0.000000in}{-0.033333in}}%
\pgfpathcurveto{\pgfqpoint{0.008840in}{-0.033333in}}{\pgfqpoint{0.017319in}{-0.029821in}}{\pgfqpoint{0.023570in}{-0.023570in}}%
\pgfpathcurveto{\pgfqpoint{0.029821in}{-0.017319in}}{\pgfqpoint{0.033333in}{-0.008840in}}{\pgfqpoint{0.033333in}{0.000000in}}%
\pgfpathcurveto{\pgfqpoint{0.033333in}{0.008840in}}{\pgfqpoint{0.029821in}{0.017319in}}{\pgfqpoint{0.023570in}{0.023570in}}%
\pgfpathcurveto{\pgfqpoint{0.017319in}{0.029821in}}{\pgfqpoint{0.008840in}{0.033333in}}{\pgfqpoint{0.000000in}{0.033333in}}%
\pgfpathcurveto{\pgfqpoint{-0.008840in}{0.033333in}}{\pgfqpoint{-0.017319in}{0.029821in}}{\pgfqpoint{-0.023570in}{0.023570in}}%
\pgfpathcurveto{\pgfqpoint{-0.029821in}{0.017319in}}{\pgfqpoint{-0.033333in}{0.008840in}}{\pgfqpoint{-0.033333in}{0.000000in}}%
\pgfpathcurveto{\pgfqpoint{-0.033333in}{-0.008840in}}{\pgfqpoint{-0.029821in}{-0.017319in}}{\pgfqpoint{-0.023570in}{-0.023570in}}%
\pgfpathcurveto{\pgfqpoint{-0.017319in}{-0.029821in}}{\pgfqpoint{-0.008840in}{-0.033333in}}{\pgfqpoint{0.000000in}{-0.033333in}}%
\pgfpathlineto{\pgfqpoint{0.000000in}{-0.033333in}}%
\pgfpathclose%
\pgfusepath{stroke,fill}%
}%
\begin{pgfscope}%
\pgfsys@transformshift{4.473821in}{2.926868in}%
\pgfsys@useobject{currentmarker}{}%
\end{pgfscope}%
\end{pgfscope}%
\begin{pgfscope}%
\pgfpathrectangle{\pgfqpoint{0.765000in}{0.660000in}}{\pgfqpoint{4.620000in}{4.620000in}}%
\pgfusepath{clip}%
\pgfsetrectcap%
\pgfsetroundjoin%
\pgfsetlinewidth{1.204500pt}%
\definecolor{currentstroke}{rgb}{1.000000,0.576471,0.309804}%
\pgfsetstrokecolor{currentstroke}%
\pgfsetdash{}{0pt}%
\pgfpathmoveto{\pgfqpoint{3.294845in}{2.821180in}}%
\pgfusepath{stroke}%
\end{pgfscope}%
\begin{pgfscope}%
\pgfpathrectangle{\pgfqpoint{0.765000in}{0.660000in}}{\pgfqpoint{4.620000in}{4.620000in}}%
\pgfusepath{clip}%
\pgfsetbuttcap%
\pgfsetroundjoin%
\definecolor{currentfill}{rgb}{1.000000,0.576471,0.309804}%
\pgfsetfillcolor{currentfill}%
\pgfsetlinewidth{1.003750pt}%
\definecolor{currentstroke}{rgb}{1.000000,0.576471,0.309804}%
\pgfsetstrokecolor{currentstroke}%
\pgfsetdash{}{0pt}%
\pgfsys@defobject{currentmarker}{\pgfqpoint{-0.033333in}{-0.033333in}}{\pgfqpoint{0.033333in}{0.033333in}}{%
\pgfpathmoveto{\pgfqpoint{0.000000in}{-0.033333in}}%
\pgfpathcurveto{\pgfqpoint{0.008840in}{-0.033333in}}{\pgfqpoint{0.017319in}{-0.029821in}}{\pgfqpoint{0.023570in}{-0.023570in}}%
\pgfpathcurveto{\pgfqpoint{0.029821in}{-0.017319in}}{\pgfqpoint{0.033333in}{-0.008840in}}{\pgfqpoint{0.033333in}{0.000000in}}%
\pgfpathcurveto{\pgfqpoint{0.033333in}{0.008840in}}{\pgfqpoint{0.029821in}{0.017319in}}{\pgfqpoint{0.023570in}{0.023570in}}%
\pgfpathcurveto{\pgfqpoint{0.017319in}{0.029821in}}{\pgfqpoint{0.008840in}{0.033333in}}{\pgfqpoint{0.000000in}{0.033333in}}%
\pgfpathcurveto{\pgfqpoint{-0.008840in}{0.033333in}}{\pgfqpoint{-0.017319in}{0.029821in}}{\pgfqpoint{-0.023570in}{0.023570in}}%
\pgfpathcurveto{\pgfqpoint{-0.029821in}{0.017319in}}{\pgfqpoint{-0.033333in}{0.008840in}}{\pgfqpoint{-0.033333in}{0.000000in}}%
\pgfpathcurveto{\pgfqpoint{-0.033333in}{-0.008840in}}{\pgfqpoint{-0.029821in}{-0.017319in}}{\pgfqpoint{-0.023570in}{-0.023570in}}%
\pgfpathcurveto{\pgfqpoint{-0.017319in}{-0.029821in}}{\pgfqpoint{-0.008840in}{-0.033333in}}{\pgfqpoint{0.000000in}{-0.033333in}}%
\pgfpathlineto{\pgfqpoint{0.000000in}{-0.033333in}}%
\pgfpathclose%
\pgfusepath{stroke,fill}%
}%
\begin{pgfscope}%
\pgfsys@transformshift{3.294845in}{2.821180in}%
\pgfsys@useobject{currentmarker}{}%
\end{pgfscope}%
\end{pgfscope}%
\begin{pgfscope}%
\pgfpathrectangle{\pgfqpoint{0.765000in}{0.660000in}}{\pgfqpoint{4.620000in}{4.620000in}}%
\pgfusepath{clip}%
\pgfsetrectcap%
\pgfsetroundjoin%
\pgfsetlinewidth{1.204500pt}%
\definecolor{currentstroke}{rgb}{1.000000,0.576471,0.309804}%
\pgfsetstrokecolor{currentstroke}%
\pgfsetdash{}{0pt}%
\pgfpathmoveto{\pgfqpoint{5.242954in}{3.271245in}}%
\pgfusepath{stroke}%
\end{pgfscope}%
\begin{pgfscope}%
\pgfpathrectangle{\pgfqpoint{0.765000in}{0.660000in}}{\pgfqpoint{4.620000in}{4.620000in}}%
\pgfusepath{clip}%
\pgfsetbuttcap%
\pgfsetroundjoin%
\definecolor{currentfill}{rgb}{1.000000,0.576471,0.309804}%
\pgfsetfillcolor{currentfill}%
\pgfsetlinewidth{1.003750pt}%
\definecolor{currentstroke}{rgb}{1.000000,0.576471,0.309804}%
\pgfsetstrokecolor{currentstroke}%
\pgfsetdash{}{0pt}%
\pgfsys@defobject{currentmarker}{\pgfqpoint{-0.033333in}{-0.033333in}}{\pgfqpoint{0.033333in}{0.033333in}}{%
\pgfpathmoveto{\pgfqpoint{0.000000in}{-0.033333in}}%
\pgfpathcurveto{\pgfqpoint{0.008840in}{-0.033333in}}{\pgfqpoint{0.017319in}{-0.029821in}}{\pgfqpoint{0.023570in}{-0.023570in}}%
\pgfpathcurveto{\pgfqpoint{0.029821in}{-0.017319in}}{\pgfqpoint{0.033333in}{-0.008840in}}{\pgfqpoint{0.033333in}{0.000000in}}%
\pgfpathcurveto{\pgfqpoint{0.033333in}{0.008840in}}{\pgfqpoint{0.029821in}{0.017319in}}{\pgfqpoint{0.023570in}{0.023570in}}%
\pgfpathcurveto{\pgfqpoint{0.017319in}{0.029821in}}{\pgfqpoint{0.008840in}{0.033333in}}{\pgfqpoint{0.000000in}{0.033333in}}%
\pgfpathcurveto{\pgfqpoint{-0.008840in}{0.033333in}}{\pgfqpoint{-0.017319in}{0.029821in}}{\pgfqpoint{-0.023570in}{0.023570in}}%
\pgfpathcurveto{\pgfqpoint{-0.029821in}{0.017319in}}{\pgfqpoint{-0.033333in}{0.008840in}}{\pgfqpoint{-0.033333in}{0.000000in}}%
\pgfpathcurveto{\pgfqpoint{-0.033333in}{-0.008840in}}{\pgfqpoint{-0.029821in}{-0.017319in}}{\pgfqpoint{-0.023570in}{-0.023570in}}%
\pgfpathcurveto{\pgfqpoint{-0.017319in}{-0.029821in}}{\pgfqpoint{-0.008840in}{-0.033333in}}{\pgfqpoint{0.000000in}{-0.033333in}}%
\pgfpathlineto{\pgfqpoint{0.000000in}{-0.033333in}}%
\pgfpathclose%
\pgfusepath{stroke,fill}%
}%
\begin{pgfscope}%
\pgfsys@transformshift{5.242954in}{3.271245in}%
\pgfsys@useobject{currentmarker}{}%
\end{pgfscope}%
\end{pgfscope}%
\begin{pgfscope}%
\pgfpathrectangle{\pgfqpoint{0.765000in}{0.660000in}}{\pgfqpoint{4.620000in}{4.620000in}}%
\pgfusepath{clip}%
\pgfsetrectcap%
\pgfsetroundjoin%
\pgfsetlinewidth{1.204500pt}%
\definecolor{currentstroke}{rgb}{1.000000,0.576471,0.309804}%
\pgfsetstrokecolor{currentstroke}%
\pgfsetdash{}{0pt}%
\pgfpathmoveto{\pgfqpoint{2.219736in}{2.798345in}}%
\pgfusepath{stroke}%
\end{pgfscope}%
\begin{pgfscope}%
\pgfpathrectangle{\pgfqpoint{0.765000in}{0.660000in}}{\pgfqpoint{4.620000in}{4.620000in}}%
\pgfusepath{clip}%
\pgfsetbuttcap%
\pgfsetroundjoin%
\definecolor{currentfill}{rgb}{1.000000,0.576471,0.309804}%
\pgfsetfillcolor{currentfill}%
\pgfsetlinewidth{1.003750pt}%
\definecolor{currentstroke}{rgb}{1.000000,0.576471,0.309804}%
\pgfsetstrokecolor{currentstroke}%
\pgfsetdash{}{0pt}%
\pgfsys@defobject{currentmarker}{\pgfqpoint{-0.033333in}{-0.033333in}}{\pgfqpoint{0.033333in}{0.033333in}}{%
\pgfpathmoveto{\pgfqpoint{0.000000in}{-0.033333in}}%
\pgfpathcurveto{\pgfqpoint{0.008840in}{-0.033333in}}{\pgfqpoint{0.017319in}{-0.029821in}}{\pgfqpoint{0.023570in}{-0.023570in}}%
\pgfpathcurveto{\pgfqpoint{0.029821in}{-0.017319in}}{\pgfqpoint{0.033333in}{-0.008840in}}{\pgfqpoint{0.033333in}{0.000000in}}%
\pgfpathcurveto{\pgfqpoint{0.033333in}{0.008840in}}{\pgfqpoint{0.029821in}{0.017319in}}{\pgfqpoint{0.023570in}{0.023570in}}%
\pgfpathcurveto{\pgfqpoint{0.017319in}{0.029821in}}{\pgfqpoint{0.008840in}{0.033333in}}{\pgfqpoint{0.000000in}{0.033333in}}%
\pgfpathcurveto{\pgfqpoint{-0.008840in}{0.033333in}}{\pgfqpoint{-0.017319in}{0.029821in}}{\pgfqpoint{-0.023570in}{0.023570in}}%
\pgfpathcurveto{\pgfqpoint{-0.029821in}{0.017319in}}{\pgfqpoint{-0.033333in}{0.008840in}}{\pgfqpoint{-0.033333in}{0.000000in}}%
\pgfpathcurveto{\pgfqpoint{-0.033333in}{-0.008840in}}{\pgfqpoint{-0.029821in}{-0.017319in}}{\pgfqpoint{-0.023570in}{-0.023570in}}%
\pgfpathcurveto{\pgfqpoint{-0.017319in}{-0.029821in}}{\pgfqpoint{-0.008840in}{-0.033333in}}{\pgfqpoint{0.000000in}{-0.033333in}}%
\pgfpathlineto{\pgfqpoint{0.000000in}{-0.033333in}}%
\pgfpathclose%
\pgfusepath{stroke,fill}%
}%
\begin{pgfscope}%
\pgfsys@transformshift{2.219736in}{2.798345in}%
\pgfsys@useobject{currentmarker}{}%
\end{pgfscope}%
\end{pgfscope}%
\begin{pgfscope}%
\pgfpathrectangle{\pgfqpoint{0.765000in}{0.660000in}}{\pgfqpoint{4.620000in}{4.620000in}}%
\pgfusepath{clip}%
\pgfsetrectcap%
\pgfsetroundjoin%
\pgfsetlinewidth{1.204500pt}%
\definecolor{currentstroke}{rgb}{1.000000,0.576471,0.309804}%
\pgfsetstrokecolor{currentstroke}%
\pgfsetdash{}{0pt}%
\pgfpathmoveto{\pgfqpoint{4.548574in}{3.050231in}}%
\pgfusepath{stroke}%
\end{pgfscope}%
\begin{pgfscope}%
\pgfpathrectangle{\pgfqpoint{0.765000in}{0.660000in}}{\pgfqpoint{4.620000in}{4.620000in}}%
\pgfusepath{clip}%
\pgfsetbuttcap%
\pgfsetroundjoin%
\definecolor{currentfill}{rgb}{1.000000,0.576471,0.309804}%
\pgfsetfillcolor{currentfill}%
\pgfsetlinewidth{1.003750pt}%
\definecolor{currentstroke}{rgb}{1.000000,0.576471,0.309804}%
\pgfsetstrokecolor{currentstroke}%
\pgfsetdash{}{0pt}%
\pgfsys@defobject{currentmarker}{\pgfqpoint{-0.033333in}{-0.033333in}}{\pgfqpoint{0.033333in}{0.033333in}}{%
\pgfpathmoveto{\pgfqpoint{0.000000in}{-0.033333in}}%
\pgfpathcurveto{\pgfqpoint{0.008840in}{-0.033333in}}{\pgfqpoint{0.017319in}{-0.029821in}}{\pgfqpoint{0.023570in}{-0.023570in}}%
\pgfpathcurveto{\pgfqpoint{0.029821in}{-0.017319in}}{\pgfqpoint{0.033333in}{-0.008840in}}{\pgfqpoint{0.033333in}{0.000000in}}%
\pgfpathcurveto{\pgfqpoint{0.033333in}{0.008840in}}{\pgfqpoint{0.029821in}{0.017319in}}{\pgfqpoint{0.023570in}{0.023570in}}%
\pgfpathcurveto{\pgfqpoint{0.017319in}{0.029821in}}{\pgfqpoint{0.008840in}{0.033333in}}{\pgfqpoint{0.000000in}{0.033333in}}%
\pgfpathcurveto{\pgfqpoint{-0.008840in}{0.033333in}}{\pgfqpoint{-0.017319in}{0.029821in}}{\pgfqpoint{-0.023570in}{0.023570in}}%
\pgfpathcurveto{\pgfqpoint{-0.029821in}{0.017319in}}{\pgfqpoint{-0.033333in}{0.008840in}}{\pgfqpoint{-0.033333in}{0.000000in}}%
\pgfpathcurveto{\pgfqpoint{-0.033333in}{-0.008840in}}{\pgfqpoint{-0.029821in}{-0.017319in}}{\pgfqpoint{-0.023570in}{-0.023570in}}%
\pgfpathcurveto{\pgfqpoint{-0.017319in}{-0.029821in}}{\pgfqpoint{-0.008840in}{-0.033333in}}{\pgfqpoint{0.000000in}{-0.033333in}}%
\pgfpathlineto{\pgfqpoint{0.000000in}{-0.033333in}}%
\pgfpathclose%
\pgfusepath{stroke,fill}%
}%
\begin{pgfscope}%
\pgfsys@transformshift{4.548574in}{3.050231in}%
\pgfsys@useobject{currentmarker}{}%
\end{pgfscope}%
\end{pgfscope}%
\begin{pgfscope}%
\pgfpathrectangle{\pgfqpoint{0.765000in}{0.660000in}}{\pgfqpoint{4.620000in}{4.620000in}}%
\pgfusepath{clip}%
\pgfsetrectcap%
\pgfsetroundjoin%
\pgfsetlinewidth{1.204500pt}%
\definecolor{currentstroke}{rgb}{1.000000,0.576471,0.309804}%
\pgfsetstrokecolor{currentstroke}%
\pgfsetdash{}{0pt}%
\pgfpathmoveto{\pgfqpoint{4.627111in}{3.000864in}}%
\pgfusepath{stroke}%
\end{pgfscope}%
\begin{pgfscope}%
\pgfpathrectangle{\pgfqpoint{0.765000in}{0.660000in}}{\pgfqpoint{4.620000in}{4.620000in}}%
\pgfusepath{clip}%
\pgfsetbuttcap%
\pgfsetroundjoin%
\definecolor{currentfill}{rgb}{1.000000,0.576471,0.309804}%
\pgfsetfillcolor{currentfill}%
\pgfsetlinewidth{1.003750pt}%
\definecolor{currentstroke}{rgb}{1.000000,0.576471,0.309804}%
\pgfsetstrokecolor{currentstroke}%
\pgfsetdash{}{0pt}%
\pgfsys@defobject{currentmarker}{\pgfqpoint{-0.033333in}{-0.033333in}}{\pgfqpoint{0.033333in}{0.033333in}}{%
\pgfpathmoveto{\pgfqpoint{0.000000in}{-0.033333in}}%
\pgfpathcurveto{\pgfqpoint{0.008840in}{-0.033333in}}{\pgfqpoint{0.017319in}{-0.029821in}}{\pgfqpoint{0.023570in}{-0.023570in}}%
\pgfpathcurveto{\pgfqpoint{0.029821in}{-0.017319in}}{\pgfqpoint{0.033333in}{-0.008840in}}{\pgfqpoint{0.033333in}{0.000000in}}%
\pgfpathcurveto{\pgfqpoint{0.033333in}{0.008840in}}{\pgfqpoint{0.029821in}{0.017319in}}{\pgfqpoint{0.023570in}{0.023570in}}%
\pgfpathcurveto{\pgfqpoint{0.017319in}{0.029821in}}{\pgfqpoint{0.008840in}{0.033333in}}{\pgfqpoint{0.000000in}{0.033333in}}%
\pgfpathcurveto{\pgfqpoint{-0.008840in}{0.033333in}}{\pgfqpoint{-0.017319in}{0.029821in}}{\pgfqpoint{-0.023570in}{0.023570in}}%
\pgfpathcurveto{\pgfqpoint{-0.029821in}{0.017319in}}{\pgfqpoint{-0.033333in}{0.008840in}}{\pgfqpoint{-0.033333in}{0.000000in}}%
\pgfpathcurveto{\pgfqpoint{-0.033333in}{-0.008840in}}{\pgfqpoint{-0.029821in}{-0.017319in}}{\pgfqpoint{-0.023570in}{-0.023570in}}%
\pgfpathcurveto{\pgfqpoint{-0.017319in}{-0.029821in}}{\pgfqpoint{-0.008840in}{-0.033333in}}{\pgfqpoint{0.000000in}{-0.033333in}}%
\pgfpathlineto{\pgfqpoint{0.000000in}{-0.033333in}}%
\pgfpathclose%
\pgfusepath{stroke,fill}%
}%
\begin{pgfscope}%
\pgfsys@transformshift{4.627111in}{3.000864in}%
\pgfsys@useobject{currentmarker}{}%
\end{pgfscope}%
\end{pgfscope}%
\begin{pgfscope}%
\pgfpathrectangle{\pgfqpoint{0.765000in}{0.660000in}}{\pgfqpoint{4.620000in}{4.620000in}}%
\pgfusepath{clip}%
\pgfsetrectcap%
\pgfsetroundjoin%
\pgfsetlinewidth{1.204500pt}%
\definecolor{currentstroke}{rgb}{1.000000,0.576471,0.309804}%
\pgfsetstrokecolor{currentstroke}%
\pgfsetdash{}{0pt}%
\pgfpathmoveto{\pgfqpoint{2.390039in}{2.706877in}}%
\pgfusepath{stroke}%
\end{pgfscope}%
\begin{pgfscope}%
\pgfpathrectangle{\pgfqpoint{0.765000in}{0.660000in}}{\pgfqpoint{4.620000in}{4.620000in}}%
\pgfusepath{clip}%
\pgfsetbuttcap%
\pgfsetroundjoin%
\definecolor{currentfill}{rgb}{1.000000,0.576471,0.309804}%
\pgfsetfillcolor{currentfill}%
\pgfsetlinewidth{1.003750pt}%
\definecolor{currentstroke}{rgb}{1.000000,0.576471,0.309804}%
\pgfsetstrokecolor{currentstroke}%
\pgfsetdash{}{0pt}%
\pgfsys@defobject{currentmarker}{\pgfqpoint{-0.033333in}{-0.033333in}}{\pgfqpoint{0.033333in}{0.033333in}}{%
\pgfpathmoveto{\pgfqpoint{0.000000in}{-0.033333in}}%
\pgfpathcurveto{\pgfqpoint{0.008840in}{-0.033333in}}{\pgfqpoint{0.017319in}{-0.029821in}}{\pgfqpoint{0.023570in}{-0.023570in}}%
\pgfpathcurveto{\pgfqpoint{0.029821in}{-0.017319in}}{\pgfqpoint{0.033333in}{-0.008840in}}{\pgfqpoint{0.033333in}{0.000000in}}%
\pgfpathcurveto{\pgfqpoint{0.033333in}{0.008840in}}{\pgfqpoint{0.029821in}{0.017319in}}{\pgfqpoint{0.023570in}{0.023570in}}%
\pgfpathcurveto{\pgfqpoint{0.017319in}{0.029821in}}{\pgfqpoint{0.008840in}{0.033333in}}{\pgfqpoint{0.000000in}{0.033333in}}%
\pgfpathcurveto{\pgfqpoint{-0.008840in}{0.033333in}}{\pgfqpoint{-0.017319in}{0.029821in}}{\pgfqpoint{-0.023570in}{0.023570in}}%
\pgfpathcurveto{\pgfqpoint{-0.029821in}{0.017319in}}{\pgfqpoint{-0.033333in}{0.008840in}}{\pgfqpoint{-0.033333in}{0.000000in}}%
\pgfpathcurveto{\pgfqpoint{-0.033333in}{-0.008840in}}{\pgfqpoint{-0.029821in}{-0.017319in}}{\pgfqpoint{-0.023570in}{-0.023570in}}%
\pgfpathcurveto{\pgfqpoint{-0.017319in}{-0.029821in}}{\pgfqpoint{-0.008840in}{-0.033333in}}{\pgfqpoint{0.000000in}{-0.033333in}}%
\pgfpathlineto{\pgfqpoint{0.000000in}{-0.033333in}}%
\pgfpathclose%
\pgfusepath{stroke,fill}%
}%
\begin{pgfscope}%
\pgfsys@transformshift{2.390039in}{2.706877in}%
\pgfsys@useobject{currentmarker}{}%
\end{pgfscope}%
\end{pgfscope}%
\begin{pgfscope}%
\pgfpathrectangle{\pgfqpoint{0.765000in}{0.660000in}}{\pgfqpoint{4.620000in}{4.620000in}}%
\pgfusepath{clip}%
\pgfsetrectcap%
\pgfsetroundjoin%
\pgfsetlinewidth{1.204500pt}%
\definecolor{currentstroke}{rgb}{1.000000,0.576471,0.309804}%
\pgfsetstrokecolor{currentstroke}%
\pgfsetdash{}{0pt}%
\pgfpathmoveto{\pgfqpoint{4.337042in}{2.756281in}}%
\pgfusepath{stroke}%
\end{pgfscope}%
\begin{pgfscope}%
\pgfpathrectangle{\pgfqpoint{0.765000in}{0.660000in}}{\pgfqpoint{4.620000in}{4.620000in}}%
\pgfusepath{clip}%
\pgfsetbuttcap%
\pgfsetroundjoin%
\definecolor{currentfill}{rgb}{1.000000,0.576471,0.309804}%
\pgfsetfillcolor{currentfill}%
\pgfsetlinewidth{1.003750pt}%
\definecolor{currentstroke}{rgb}{1.000000,0.576471,0.309804}%
\pgfsetstrokecolor{currentstroke}%
\pgfsetdash{}{0pt}%
\pgfsys@defobject{currentmarker}{\pgfqpoint{-0.033333in}{-0.033333in}}{\pgfqpoint{0.033333in}{0.033333in}}{%
\pgfpathmoveto{\pgfqpoint{0.000000in}{-0.033333in}}%
\pgfpathcurveto{\pgfqpoint{0.008840in}{-0.033333in}}{\pgfqpoint{0.017319in}{-0.029821in}}{\pgfqpoint{0.023570in}{-0.023570in}}%
\pgfpathcurveto{\pgfqpoint{0.029821in}{-0.017319in}}{\pgfqpoint{0.033333in}{-0.008840in}}{\pgfqpoint{0.033333in}{0.000000in}}%
\pgfpathcurveto{\pgfqpoint{0.033333in}{0.008840in}}{\pgfqpoint{0.029821in}{0.017319in}}{\pgfqpoint{0.023570in}{0.023570in}}%
\pgfpathcurveto{\pgfqpoint{0.017319in}{0.029821in}}{\pgfqpoint{0.008840in}{0.033333in}}{\pgfqpoint{0.000000in}{0.033333in}}%
\pgfpathcurveto{\pgfqpoint{-0.008840in}{0.033333in}}{\pgfqpoint{-0.017319in}{0.029821in}}{\pgfqpoint{-0.023570in}{0.023570in}}%
\pgfpathcurveto{\pgfqpoint{-0.029821in}{0.017319in}}{\pgfqpoint{-0.033333in}{0.008840in}}{\pgfqpoint{-0.033333in}{0.000000in}}%
\pgfpathcurveto{\pgfqpoint{-0.033333in}{-0.008840in}}{\pgfqpoint{-0.029821in}{-0.017319in}}{\pgfqpoint{-0.023570in}{-0.023570in}}%
\pgfpathcurveto{\pgfqpoint{-0.017319in}{-0.029821in}}{\pgfqpoint{-0.008840in}{-0.033333in}}{\pgfqpoint{0.000000in}{-0.033333in}}%
\pgfpathlineto{\pgfqpoint{0.000000in}{-0.033333in}}%
\pgfpathclose%
\pgfusepath{stroke,fill}%
}%
\begin{pgfscope}%
\pgfsys@transformshift{4.337042in}{2.756281in}%
\pgfsys@useobject{currentmarker}{}%
\end{pgfscope}%
\end{pgfscope}%
\begin{pgfscope}%
\pgfpathrectangle{\pgfqpoint{0.765000in}{0.660000in}}{\pgfqpoint{4.620000in}{4.620000in}}%
\pgfusepath{clip}%
\pgfsetrectcap%
\pgfsetroundjoin%
\pgfsetlinewidth{1.204500pt}%
\definecolor{currentstroke}{rgb}{1.000000,0.576471,0.309804}%
\pgfsetstrokecolor{currentstroke}%
\pgfsetdash{}{0pt}%
\pgfpathmoveto{\pgfqpoint{2.737406in}{3.199322in}}%
\pgfusepath{stroke}%
\end{pgfscope}%
\begin{pgfscope}%
\pgfpathrectangle{\pgfqpoint{0.765000in}{0.660000in}}{\pgfqpoint{4.620000in}{4.620000in}}%
\pgfusepath{clip}%
\pgfsetbuttcap%
\pgfsetroundjoin%
\definecolor{currentfill}{rgb}{1.000000,0.576471,0.309804}%
\pgfsetfillcolor{currentfill}%
\pgfsetlinewidth{1.003750pt}%
\definecolor{currentstroke}{rgb}{1.000000,0.576471,0.309804}%
\pgfsetstrokecolor{currentstroke}%
\pgfsetdash{}{0pt}%
\pgfsys@defobject{currentmarker}{\pgfqpoint{-0.033333in}{-0.033333in}}{\pgfqpoint{0.033333in}{0.033333in}}{%
\pgfpathmoveto{\pgfqpoint{0.000000in}{-0.033333in}}%
\pgfpathcurveto{\pgfqpoint{0.008840in}{-0.033333in}}{\pgfqpoint{0.017319in}{-0.029821in}}{\pgfqpoint{0.023570in}{-0.023570in}}%
\pgfpathcurveto{\pgfqpoint{0.029821in}{-0.017319in}}{\pgfqpoint{0.033333in}{-0.008840in}}{\pgfqpoint{0.033333in}{0.000000in}}%
\pgfpathcurveto{\pgfqpoint{0.033333in}{0.008840in}}{\pgfqpoint{0.029821in}{0.017319in}}{\pgfqpoint{0.023570in}{0.023570in}}%
\pgfpathcurveto{\pgfqpoint{0.017319in}{0.029821in}}{\pgfqpoint{0.008840in}{0.033333in}}{\pgfqpoint{0.000000in}{0.033333in}}%
\pgfpathcurveto{\pgfqpoint{-0.008840in}{0.033333in}}{\pgfqpoint{-0.017319in}{0.029821in}}{\pgfqpoint{-0.023570in}{0.023570in}}%
\pgfpathcurveto{\pgfqpoint{-0.029821in}{0.017319in}}{\pgfqpoint{-0.033333in}{0.008840in}}{\pgfqpoint{-0.033333in}{0.000000in}}%
\pgfpathcurveto{\pgfqpoint{-0.033333in}{-0.008840in}}{\pgfqpoint{-0.029821in}{-0.017319in}}{\pgfqpoint{-0.023570in}{-0.023570in}}%
\pgfpathcurveto{\pgfqpoint{-0.017319in}{-0.029821in}}{\pgfqpoint{-0.008840in}{-0.033333in}}{\pgfqpoint{0.000000in}{-0.033333in}}%
\pgfpathlineto{\pgfqpoint{0.000000in}{-0.033333in}}%
\pgfpathclose%
\pgfusepath{stroke,fill}%
}%
\begin{pgfscope}%
\pgfsys@transformshift{2.737406in}{3.199322in}%
\pgfsys@useobject{currentmarker}{}%
\end{pgfscope}%
\end{pgfscope}%
\begin{pgfscope}%
\pgfpathrectangle{\pgfqpoint{0.765000in}{0.660000in}}{\pgfqpoint{4.620000in}{4.620000in}}%
\pgfusepath{clip}%
\pgfsetrectcap%
\pgfsetroundjoin%
\pgfsetlinewidth{1.204500pt}%
\definecolor{currentstroke}{rgb}{1.000000,0.576471,0.309804}%
\pgfsetstrokecolor{currentstroke}%
\pgfsetdash{}{0pt}%
\pgfpathmoveto{\pgfqpoint{3.764863in}{2.583077in}}%
\pgfusepath{stroke}%
\end{pgfscope}%
\begin{pgfscope}%
\pgfpathrectangle{\pgfqpoint{0.765000in}{0.660000in}}{\pgfqpoint{4.620000in}{4.620000in}}%
\pgfusepath{clip}%
\pgfsetbuttcap%
\pgfsetroundjoin%
\definecolor{currentfill}{rgb}{1.000000,0.576471,0.309804}%
\pgfsetfillcolor{currentfill}%
\pgfsetlinewidth{1.003750pt}%
\definecolor{currentstroke}{rgb}{1.000000,0.576471,0.309804}%
\pgfsetstrokecolor{currentstroke}%
\pgfsetdash{}{0pt}%
\pgfsys@defobject{currentmarker}{\pgfqpoint{-0.033333in}{-0.033333in}}{\pgfqpoint{0.033333in}{0.033333in}}{%
\pgfpathmoveto{\pgfqpoint{0.000000in}{-0.033333in}}%
\pgfpathcurveto{\pgfqpoint{0.008840in}{-0.033333in}}{\pgfqpoint{0.017319in}{-0.029821in}}{\pgfqpoint{0.023570in}{-0.023570in}}%
\pgfpathcurveto{\pgfqpoint{0.029821in}{-0.017319in}}{\pgfqpoint{0.033333in}{-0.008840in}}{\pgfqpoint{0.033333in}{0.000000in}}%
\pgfpathcurveto{\pgfqpoint{0.033333in}{0.008840in}}{\pgfqpoint{0.029821in}{0.017319in}}{\pgfqpoint{0.023570in}{0.023570in}}%
\pgfpathcurveto{\pgfqpoint{0.017319in}{0.029821in}}{\pgfqpoint{0.008840in}{0.033333in}}{\pgfqpoint{0.000000in}{0.033333in}}%
\pgfpathcurveto{\pgfqpoint{-0.008840in}{0.033333in}}{\pgfqpoint{-0.017319in}{0.029821in}}{\pgfqpoint{-0.023570in}{0.023570in}}%
\pgfpathcurveto{\pgfqpoint{-0.029821in}{0.017319in}}{\pgfqpoint{-0.033333in}{0.008840in}}{\pgfqpoint{-0.033333in}{0.000000in}}%
\pgfpathcurveto{\pgfqpoint{-0.033333in}{-0.008840in}}{\pgfqpoint{-0.029821in}{-0.017319in}}{\pgfqpoint{-0.023570in}{-0.023570in}}%
\pgfpathcurveto{\pgfqpoint{-0.017319in}{-0.029821in}}{\pgfqpoint{-0.008840in}{-0.033333in}}{\pgfqpoint{0.000000in}{-0.033333in}}%
\pgfpathlineto{\pgfqpoint{0.000000in}{-0.033333in}}%
\pgfpathclose%
\pgfusepath{stroke,fill}%
}%
\begin{pgfscope}%
\pgfsys@transformshift{3.764863in}{2.583077in}%
\pgfsys@useobject{currentmarker}{}%
\end{pgfscope}%
\end{pgfscope}%
\begin{pgfscope}%
\pgfpathrectangle{\pgfqpoint{0.765000in}{0.660000in}}{\pgfqpoint{4.620000in}{4.620000in}}%
\pgfusepath{clip}%
\pgfsetrectcap%
\pgfsetroundjoin%
\pgfsetlinewidth{1.204500pt}%
\definecolor{currentstroke}{rgb}{1.000000,0.576471,0.309804}%
\pgfsetstrokecolor{currentstroke}%
\pgfsetdash{}{0pt}%
\pgfpathmoveto{\pgfqpoint{3.708083in}{2.586691in}}%
\pgfusepath{stroke}%
\end{pgfscope}%
\begin{pgfscope}%
\pgfpathrectangle{\pgfqpoint{0.765000in}{0.660000in}}{\pgfqpoint{4.620000in}{4.620000in}}%
\pgfusepath{clip}%
\pgfsetbuttcap%
\pgfsetroundjoin%
\definecolor{currentfill}{rgb}{1.000000,0.576471,0.309804}%
\pgfsetfillcolor{currentfill}%
\pgfsetlinewidth{1.003750pt}%
\definecolor{currentstroke}{rgb}{1.000000,0.576471,0.309804}%
\pgfsetstrokecolor{currentstroke}%
\pgfsetdash{}{0pt}%
\pgfsys@defobject{currentmarker}{\pgfqpoint{-0.033333in}{-0.033333in}}{\pgfqpoint{0.033333in}{0.033333in}}{%
\pgfpathmoveto{\pgfqpoint{0.000000in}{-0.033333in}}%
\pgfpathcurveto{\pgfqpoint{0.008840in}{-0.033333in}}{\pgfqpoint{0.017319in}{-0.029821in}}{\pgfqpoint{0.023570in}{-0.023570in}}%
\pgfpathcurveto{\pgfqpoint{0.029821in}{-0.017319in}}{\pgfqpoint{0.033333in}{-0.008840in}}{\pgfqpoint{0.033333in}{0.000000in}}%
\pgfpathcurveto{\pgfqpoint{0.033333in}{0.008840in}}{\pgfqpoint{0.029821in}{0.017319in}}{\pgfqpoint{0.023570in}{0.023570in}}%
\pgfpathcurveto{\pgfqpoint{0.017319in}{0.029821in}}{\pgfqpoint{0.008840in}{0.033333in}}{\pgfqpoint{0.000000in}{0.033333in}}%
\pgfpathcurveto{\pgfqpoint{-0.008840in}{0.033333in}}{\pgfqpoint{-0.017319in}{0.029821in}}{\pgfqpoint{-0.023570in}{0.023570in}}%
\pgfpathcurveto{\pgfqpoint{-0.029821in}{0.017319in}}{\pgfqpoint{-0.033333in}{0.008840in}}{\pgfqpoint{-0.033333in}{0.000000in}}%
\pgfpathcurveto{\pgfqpoint{-0.033333in}{-0.008840in}}{\pgfqpoint{-0.029821in}{-0.017319in}}{\pgfqpoint{-0.023570in}{-0.023570in}}%
\pgfpathcurveto{\pgfqpoint{-0.017319in}{-0.029821in}}{\pgfqpoint{-0.008840in}{-0.033333in}}{\pgfqpoint{0.000000in}{-0.033333in}}%
\pgfpathlineto{\pgfqpoint{0.000000in}{-0.033333in}}%
\pgfpathclose%
\pgfusepath{stroke,fill}%
}%
\begin{pgfscope}%
\pgfsys@transformshift{3.708083in}{2.586691in}%
\pgfsys@useobject{currentmarker}{}%
\end{pgfscope}%
\end{pgfscope}%
\begin{pgfscope}%
\pgfpathrectangle{\pgfqpoint{0.765000in}{0.660000in}}{\pgfqpoint{4.620000in}{4.620000in}}%
\pgfusepath{clip}%
\pgfsetrectcap%
\pgfsetroundjoin%
\pgfsetlinewidth{1.204500pt}%
\definecolor{currentstroke}{rgb}{1.000000,0.576471,0.309804}%
\pgfsetstrokecolor{currentstroke}%
\pgfsetdash{}{0pt}%
\pgfpathmoveto{\pgfqpoint{4.395322in}{2.860919in}}%
\pgfusepath{stroke}%
\end{pgfscope}%
\begin{pgfscope}%
\pgfpathrectangle{\pgfqpoint{0.765000in}{0.660000in}}{\pgfqpoint{4.620000in}{4.620000in}}%
\pgfusepath{clip}%
\pgfsetbuttcap%
\pgfsetroundjoin%
\definecolor{currentfill}{rgb}{1.000000,0.576471,0.309804}%
\pgfsetfillcolor{currentfill}%
\pgfsetlinewidth{1.003750pt}%
\definecolor{currentstroke}{rgb}{1.000000,0.576471,0.309804}%
\pgfsetstrokecolor{currentstroke}%
\pgfsetdash{}{0pt}%
\pgfsys@defobject{currentmarker}{\pgfqpoint{-0.033333in}{-0.033333in}}{\pgfqpoint{0.033333in}{0.033333in}}{%
\pgfpathmoveto{\pgfqpoint{0.000000in}{-0.033333in}}%
\pgfpathcurveto{\pgfqpoint{0.008840in}{-0.033333in}}{\pgfqpoint{0.017319in}{-0.029821in}}{\pgfqpoint{0.023570in}{-0.023570in}}%
\pgfpathcurveto{\pgfqpoint{0.029821in}{-0.017319in}}{\pgfqpoint{0.033333in}{-0.008840in}}{\pgfqpoint{0.033333in}{0.000000in}}%
\pgfpathcurveto{\pgfqpoint{0.033333in}{0.008840in}}{\pgfqpoint{0.029821in}{0.017319in}}{\pgfqpoint{0.023570in}{0.023570in}}%
\pgfpathcurveto{\pgfqpoint{0.017319in}{0.029821in}}{\pgfqpoint{0.008840in}{0.033333in}}{\pgfqpoint{0.000000in}{0.033333in}}%
\pgfpathcurveto{\pgfqpoint{-0.008840in}{0.033333in}}{\pgfqpoint{-0.017319in}{0.029821in}}{\pgfqpoint{-0.023570in}{0.023570in}}%
\pgfpathcurveto{\pgfqpoint{-0.029821in}{0.017319in}}{\pgfqpoint{-0.033333in}{0.008840in}}{\pgfqpoint{-0.033333in}{0.000000in}}%
\pgfpathcurveto{\pgfqpoint{-0.033333in}{-0.008840in}}{\pgfqpoint{-0.029821in}{-0.017319in}}{\pgfqpoint{-0.023570in}{-0.023570in}}%
\pgfpathcurveto{\pgfqpoint{-0.017319in}{-0.029821in}}{\pgfqpoint{-0.008840in}{-0.033333in}}{\pgfqpoint{0.000000in}{-0.033333in}}%
\pgfpathlineto{\pgfqpoint{0.000000in}{-0.033333in}}%
\pgfpathclose%
\pgfusepath{stroke,fill}%
}%
\begin{pgfscope}%
\pgfsys@transformshift{4.395322in}{2.860919in}%
\pgfsys@useobject{currentmarker}{}%
\end{pgfscope}%
\end{pgfscope}%
\begin{pgfscope}%
\pgfpathrectangle{\pgfqpoint{0.765000in}{0.660000in}}{\pgfqpoint{4.620000in}{4.620000in}}%
\pgfusepath{clip}%
\pgfsetrectcap%
\pgfsetroundjoin%
\pgfsetlinewidth{1.204500pt}%
\definecolor{currentstroke}{rgb}{1.000000,0.576471,0.309804}%
\pgfsetstrokecolor{currentstroke}%
\pgfsetdash{}{0pt}%
\pgfpathmoveto{\pgfqpoint{4.727453in}{3.485794in}}%
\pgfusepath{stroke}%
\end{pgfscope}%
\begin{pgfscope}%
\pgfpathrectangle{\pgfqpoint{0.765000in}{0.660000in}}{\pgfqpoint{4.620000in}{4.620000in}}%
\pgfusepath{clip}%
\pgfsetbuttcap%
\pgfsetroundjoin%
\definecolor{currentfill}{rgb}{1.000000,0.576471,0.309804}%
\pgfsetfillcolor{currentfill}%
\pgfsetlinewidth{1.003750pt}%
\definecolor{currentstroke}{rgb}{1.000000,0.576471,0.309804}%
\pgfsetstrokecolor{currentstroke}%
\pgfsetdash{}{0pt}%
\pgfsys@defobject{currentmarker}{\pgfqpoint{-0.033333in}{-0.033333in}}{\pgfqpoint{0.033333in}{0.033333in}}{%
\pgfpathmoveto{\pgfqpoint{0.000000in}{-0.033333in}}%
\pgfpathcurveto{\pgfqpoint{0.008840in}{-0.033333in}}{\pgfqpoint{0.017319in}{-0.029821in}}{\pgfqpoint{0.023570in}{-0.023570in}}%
\pgfpathcurveto{\pgfqpoint{0.029821in}{-0.017319in}}{\pgfqpoint{0.033333in}{-0.008840in}}{\pgfqpoint{0.033333in}{0.000000in}}%
\pgfpathcurveto{\pgfqpoint{0.033333in}{0.008840in}}{\pgfqpoint{0.029821in}{0.017319in}}{\pgfqpoint{0.023570in}{0.023570in}}%
\pgfpathcurveto{\pgfqpoint{0.017319in}{0.029821in}}{\pgfqpoint{0.008840in}{0.033333in}}{\pgfqpoint{0.000000in}{0.033333in}}%
\pgfpathcurveto{\pgfqpoint{-0.008840in}{0.033333in}}{\pgfqpoint{-0.017319in}{0.029821in}}{\pgfqpoint{-0.023570in}{0.023570in}}%
\pgfpathcurveto{\pgfqpoint{-0.029821in}{0.017319in}}{\pgfqpoint{-0.033333in}{0.008840in}}{\pgfqpoint{-0.033333in}{0.000000in}}%
\pgfpathcurveto{\pgfqpoint{-0.033333in}{-0.008840in}}{\pgfqpoint{-0.029821in}{-0.017319in}}{\pgfqpoint{-0.023570in}{-0.023570in}}%
\pgfpathcurveto{\pgfqpoint{-0.017319in}{-0.029821in}}{\pgfqpoint{-0.008840in}{-0.033333in}}{\pgfqpoint{0.000000in}{-0.033333in}}%
\pgfpathlineto{\pgfqpoint{0.000000in}{-0.033333in}}%
\pgfpathclose%
\pgfusepath{stroke,fill}%
}%
\begin{pgfscope}%
\pgfsys@transformshift{4.727453in}{3.485794in}%
\pgfsys@useobject{currentmarker}{}%
\end{pgfscope}%
\end{pgfscope}%
\begin{pgfscope}%
\pgfpathrectangle{\pgfqpoint{0.765000in}{0.660000in}}{\pgfqpoint{4.620000in}{4.620000in}}%
\pgfusepath{clip}%
\pgfsetrectcap%
\pgfsetroundjoin%
\pgfsetlinewidth{1.204500pt}%
\definecolor{currentstroke}{rgb}{1.000000,0.576471,0.309804}%
\pgfsetstrokecolor{currentstroke}%
\pgfsetdash{}{0pt}%
\pgfpathmoveto{\pgfqpoint{2.667274in}{3.160307in}}%
\pgfusepath{stroke}%
\end{pgfscope}%
\begin{pgfscope}%
\pgfpathrectangle{\pgfqpoint{0.765000in}{0.660000in}}{\pgfqpoint{4.620000in}{4.620000in}}%
\pgfusepath{clip}%
\pgfsetbuttcap%
\pgfsetroundjoin%
\definecolor{currentfill}{rgb}{1.000000,0.576471,0.309804}%
\pgfsetfillcolor{currentfill}%
\pgfsetlinewidth{1.003750pt}%
\definecolor{currentstroke}{rgb}{1.000000,0.576471,0.309804}%
\pgfsetstrokecolor{currentstroke}%
\pgfsetdash{}{0pt}%
\pgfsys@defobject{currentmarker}{\pgfqpoint{-0.033333in}{-0.033333in}}{\pgfqpoint{0.033333in}{0.033333in}}{%
\pgfpathmoveto{\pgfqpoint{0.000000in}{-0.033333in}}%
\pgfpathcurveto{\pgfqpoint{0.008840in}{-0.033333in}}{\pgfqpoint{0.017319in}{-0.029821in}}{\pgfqpoint{0.023570in}{-0.023570in}}%
\pgfpathcurveto{\pgfqpoint{0.029821in}{-0.017319in}}{\pgfqpoint{0.033333in}{-0.008840in}}{\pgfqpoint{0.033333in}{0.000000in}}%
\pgfpathcurveto{\pgfqpoint{0.033333in}{0.008840in}}{\pgfqpoint{0.029821in}{0.017319in}}{\pgfqpoint{0.023570in}{0.023570in}}%
\pgfpathcurveto{\pgfqpoint{0.017319in}{0.029821in}}{\pgfqpoint{0.008840in}{0.033333in}}{\pgfqpoint{0.000000in}{0.033333in}}%
\pgfpathcurveto{\pgfqpoint{-0.008840in}{0.033333in}}{\pgfqpoint{-0.017319in}{0.029821in}}{\pgfqpoint{-0.023570in}{0.023570in}}%
\pgfpathcurveto{\pgfqpoint{-0.029821in}{0.017319in}}{\pgfqpoint{-0.033333in}{0.008840in}}{\pgfqpoint{-0.033333in}{0.000000in}}%
\pgfpathcurveto{\pgfqpoint{-0.033333in}{-0.008840in}}{\pgfqpoint{-0.029821in}{-0.017319in}}{\pgfqpoint{-0.023570in}{-0.023570in}}%
\pgfpathcurveto{\pgfqpoint{-0.017319in}{-0.029821in}}{\pgfqpoint{-0.008840in}{-0.033333in}}{\pgfqpoint{0.000000in}{-0.033333in}}%
\pgfpathlineto{\pgfqpoint{0.000000in}{-0.033333in}}%
\pgfpathclose%
\pgfusepath{stroke,fill}%
}%
\begin{pgfscope}%
\pgfsys@transformshift{2.667274in}{3.160307in}%
\pgfsys@useobject{currentmarker}{}%
\end{pgfscope}%
\end{pgfscope}%
\begin{pgfscope}%
\pgfpathrectangle{\pgfqpoint{0.765000in}{0.660000in}}{\pgfqpoint{4.620000in}{4.620000in}}%
\pgfusepath{clip}%
\pgfsetrectcap%
\pgfsetroundjoin%
\pgfsetlinewidth{1.204500pt}%
\definecolor{currentstroke}{rgb}{1.000000,0.576471,0.309804}%
\pgfsetstrokecolor{currentstroke}%
\pgfsetdash{}{0pt}%
\pgfpathmoveto{\pgfqpoint{2.885822in}{2.877128in}}%
\pgfusepath{stroke}%
\end{pgfscope}%
\begin{pgfscope}%
\pgfpathrectangle{\pgfqpoint{0.765000in}{0.660000in}}{\pgfqpoint{4.620000in}{4.620000in}}%
\pgfusepath{clip}%
\pgfsetbuttcap%
\pgfsetroundjoin%
\definecolor{currentfill}{rgb}{1.000000,0.576471,0.309804}%
\pgfsetfillcolor{currentfill}%
\pgfsetlinewidth{1.003750pt}%
\definecolor{currentstroke}{rgb}{1.000000,0.576471,0.309804}%
\pgfsetstrokecolor{currentstroke}%
\pgfsetdash{}{0pt}%
\pgfsys@defobject{currentmarker}{\pgfqpoint{-0.033333in}{-0.033333in}}{\pgfqpoint{0.033333in}{0.033333in}}{%
\pgfpathmoveto{\pgfqpoint{0.000000in}{-0.033333in}}%
\pgfpathcurveto{\pgfqpoint{0.008840in}{-0.033333in}}{\pgfqpoint{0.017319in}{-0.029821in}}{\pgfqpoint{0.023570in}{-0.023570in}}%
\pgfpathcurveto{\pgfqpoint{0.029821in}{-0.017319in}}{\pgfqpoint{0.033333in}{-0.008840in}}{\pgfqpoint{0.033333in}{0.000000in}}%
\pgfpathcurveto{\pgfqpoint{0.033333in}{0.008840in}}{\pgfqpoint{0.029821in}{0.017319in}}{\pgfqpoint{0.023570in}{0.023570in}}%
\pgfpathcurveto{\pgfqpoint{0.017319in}{0.029821in}}{\pgfqpoint{0.008840in}{0.033333in}}{\pgfqpoint{0.000000in}{0.033333in}}%
\pgfpathcurveto{\pgfqpoint{-0.008840in}{0.033333in}}{\pgfqpoint{-0.017319in}{0.029821in}}{\pgfqpoint{-0.023570in}{0.023570in}}%
\pgfpathcurveto{\pgfqpoint{-0.029821in}{0.017319in}}{\pgfqpoint{-0.033333in}{0.008840in}}{\pgfqpoint{-0.033333in}{0.000000in}}%
\pgfpathcurveto{\pgfqpoint{-0.033333in}{-0.008840in}}{\pgfqpoint{-0.029821in}{-0.017319in}}{\pgfqpoint{-0.023570in}{-0.023570in}}%
\pgfpathcurveto{\pgfqpoint{-0.017319in}{-0.029821in}}{\pgfqpoint{-0.008840in}{-0.033333in}}{\pgfqpoint{0.000000in}{-0.033333in}}%
\pgfpathlineto{\pgfqpoint{0.000000in}{-0.033333in}}%
\pgfpathclose%
\pgfusepath{stroke,fill}%
}%
\begin{pgfscope}%
\pgfsys@transformshift{2.885822in}{2.877128in}%
\pgfsys@useobject{currentmarker}{}%
\end{pgfscope}%
\end{pgfscope}%
\begin{pgfscope}%
\pgfpathrectangle{\pgfqpoint{0.765000in}{0.660000in}}{\pgfqpoint{4.620000in}{4.620000in}}%
\pgfusepath{clip}%
\pgfsetrectcap%
\pgfsetroundjoin%
\pgfsetlinewidth{1.204500pt}%
\definecolor{currentstroke}{rgb}{1.000000,0.576471,0.309804}%
\pgfsetstrokecolor{currentstroke}%
\pgfsetdash{}{0pt}%
\pgfpathmoveto{\pgfqpoint{4.419199in}{2.434032in}}%
\pgfusepath{stroke}%
\end{pgfscope}%
\begin{pgfscope}%
\pgfpathrectangle{\pgfqpoint{0.765000in}{0.660000in}}{\pgfqpoint{4.620000in}{4.620000in}}%
\pgfusepath{clip}%
\pgfsetbuttcap%
\pgfsetroundjoin%
\definecolor{currentfill}{rgb}{1.000000,0.576471,0.309804}%
\pgfsetfillcolor{currentfill}%
\pgfsetlinewidth{1.003750pt}%
\definecolor{currentstroke}{rgb}{1.000000,0.576471,0.309804}%
\pgfsetstrokecolor{currentstroke}%
\pgfsetdash{}{0pt}%
\pgfsys@defobject{currentmarker}{\pgfqpoint{-0.033333in}{-0.033333in}}{\pgfqpoint{0.033333in}{0.033333in}}{%
\pgfpathmoveto{\pgfqpoint{0.000000in}{-0.033333in}}%
\pgfpathcurveto{\pgfqpoint{0.008840in}{-0.033333in}}{\pgfqpoint{0.017319in}{-0.029821in}}{\pgfqpoint{0.023570in}{-0.023570in}}%
\pgfpathcurveto{\pgfqpoint{0.029821in}{-0.017319in}}{\pgfqpoint{0.033333in}{-0.008840in}}{\pgfqpoint{0.033333in}{0.000000in}}%
\pgfpathcurveto{\pgfqpoint{0.033333in}{0.008840in}}{\pgfqpoint{0.029821in}{0.017319in}}{\pgfqpoint{0.023570in}{0.023570in}}%
\pgfpathcurveto{\pgfqpoint{0.017319in}{0.029821in}}{\pgfqpoint{0.008840in}{0.033333in}}{\pgfqpoint{0.000000in}{0.033333in}}%
\pgfpathcurveto{\pgfqpoint{-0.008840in}{0.033333in}}{\pgfqpoint{-0.017319in}{0.029821in}}{\pgfqpoint{-0.023570in}{0.023570in}}%
\pgfpathcurveto{\pgfqpoint{-0.029821in}{0.017319in}}{\pgfqpoint{-0.033333in}{0.008840in}}{\pgfqpoint{-0.033333in}{0.000000in}}%
\pgfpathcurveto{\pgfqpoint{-0.033333in}{-0.008840in}}{\pgfqpoint{-0.029821in}{-0.017319in}}{\pgfqpoint{-0.023570in}{-0.023570in}}%
\pgfpathcurveto{\pgfqpoint{-0.017319in}{-0.029821in}}{\pgfqpoint{-0.008840in}{-0.033333in}}{\pgfqpoint{0.000000in}{-0.033333in}}%
\pgfpathlineto{\pgfqpoint{0.000000in}{-0.033333in}}%
\pgfpathclose%
\pgfusepath{stroke,fill}%
}%
\begin{pgfscope}%
\pgfsys@transformshift{4.419199in}{2.434032in}%
\pgfsys@useobject{currentmarker}{}%
\end{pgfscope}%
\end{pgfscope}%
\begin{pgfscope}%
\pgfpathrectangle{\pgfqpoint{0.765000in}{0.660000in}}{\pgfqpoint{4.620000in}{4.620000in}}%
\pgfusepath{clip}%
\pgfsetrectcap%
\pgfsetroundjoin%
\pgfsetlinewidth{1.204500pt}%
\definecolor{currentstroke}{rgb}{1.000000,0.576471,0.309804}%
\pgfsetstrokecolor{currentstroke}%
\pgfsetdash{}{0pt}%
\pgfpathmoveto{\pgfqpoint{3.361954in}{2.357217in}}%
\pgfusepath{stroke}%
\end{pgfscope}%
\begin{pgfscope}%
\pgfpathrectangle{\pgfqpoint{0.765000in}{0.660000in}}{\pgfqpoint{4.620000in}{4.620000in}}%
\pgfusepath{clip}%
\pgfsetbuttcap%
\pgfsetroundjoin%
\definecolor{currentfill}{rgb}{1.000000,0.576471,0.309804}%
\pgfsetfillcolor{currentfill}%
\pgfsetlinewidth{1.003750pt}%
\definecolor{currentstroke}{rgb}{1.000000,0.576471,0.309804}%
\pgfsetstrokecolor{currentstroke}%
\pgfsetdash{}{0pt}%
\pgfsys@defobject{currentmarker}{\pgfqpoint{-0.033333in}{-0.033333in}}{\pgfqpoint{0.033333in}{0.033333in}}{%
\pgfpathmoveto{\pgfqpoint{0.000000in}{-0.033333in}}%
\pgfpathcurveto{\pgfqpoint{0.008840in}{-0.033333in}}{\pgfqpoint{0.017319in}{-0.029821in}}{\pgfqpoint{0.023570in}{-0.023570in}}%
\pgfpathcurveto{\pgfqpoint{0.029821in}{-0.017319in}}{\pgfqpoint{0.033333in}{-0.008840in}}{\pgfqpoint{0.033333in}{0.000000in}}%
\pgfpathcurveto{\pgfqpoint{0.033333in}{0.008840in}}{\pgfqpoint{0.029821in}{0.017319in}}{\pgfqpoint{0.023570in}{0.023570in}}%
\pgfpathcurveto{\pgfqpoint{0.017319in}{0.029821in}}{\pgfqpoint{0.008840in}{0.033333in}}{\pgfqpoint{0.000000in}{0.033333in}}%
\pgfpathcurveto{\pgfqpoint{-0.008840in}{0.033333in}}{\pgfqpoint{-0.017319in}{0.029821in}}{\pgfqpoint{-0.023570in}{0.023570in}}%
\pgfpathcurveto{\pgfqpoint{-0.029821in}{0.017319in}}{\pgfqpoint{-0.033333in}{0.008840in}}{\pgfqpoint{-0.033333in}{0.000000in}}%
\pgfpathcurveto{\pgfqpoint{-0.033333in}{-0.008840in}}{\pgfqpoint{-0.029821in}{-0.017319in}}{\pgfqpoint{-0.023570in}{-0.023570in}}%
\pgfpathcurveto{\pgfqpoint{-0.017319in}{-0.029821in}}{\pgfqpoint{-0.008840in}{-0.033333in}}{\pgfqpoint{0.000000in}{-0.033333in}}%
\pgfpathlineto{\pgfqpoint{0.000000in}{-0.033333in}}%
\pgfpathclose%
\pgfusepath{stroke,fill}%
}%
\begin{pgfscope}%
\pgfsys@transformshift{3.361954in}{2.357217in}%
\pgfsys@useobject{currentmarker}{}%
\end{pgfscope}%
\end{pgfscope}%
\begin{pgfscope}%
\pgfpathrectangle{\pgfqpoint{0.765000in}{0.660000in}}{\pgfqpoint{4.620000in}{4.620000in}}%
\pgfusepath{clip}%
\pgfsetrectcap%
\pgfsetroundjoin%
\pgfsetlinewidth{1.204500pt}%
\definecolor{currentstroke}{rgb}{1.000000,0.576471,0.309804}%
\pgfsetstrokecolor{currentstroke}%
\pgfsetdash{}{0pt}%
\pgfpathmoveto{\pgfqpoint{2.941215in}{2.626196in}}%
\pgfusepath{stroke}%
\end{pgfscope}%
\begin{pgfscope}%
\pgfpathrectangle{\pgfqpoint{0.765000in}{0.660000in}}{\pgfqpoint{4.620000in}{4.620000in}}%
\pgfusepath{clip}%
\pgfsetbuttcap%
\pgfsetroundjoin%
\definecolor{currentfill}{rgb}{1.000000,0.576471,0.309804}%
\pgfsetfillcolor{currentfill}%
\pgfsetlinewidth{1.003750pt}%
\definecolor{currentstroke}{rgb}{1.000000,0.576471,0.309804}%
\pgfsetstrokecolor{currentstroke}%
\pgfsetdash{}{0pt}%
\pgfsys@defobject{currentmarker}{\pgfqpoint{-0.033333in}{-0.033333in}}{\pgfqpoint{0.033333in}{0.033333in}}{%
\pgfpathmoveto{\pgfqpoint{0.000000in}{-0.033333in}}%
\pgfpathcurveto{\pgfqpoint{0.008840in}{-0.033333in}}{\pgfqpoint{0.017319in}{-0.029821in}}{\pgfqpoint{0.023570in}{-0.023570in}}%
\pgfpathcurveto{\pgfqpoint{0.029821in}{-0.017319in}}{\pgfqpoint{0.033333in}{-0.008840in}}{\pgfqpoint{0.033333in}{0.000000in}}%
\pgfpathcurveto{\pgfqpoint{0.033333in}{0.008840in}}{\pgfqpoint{0.029821in}{0.017319in}}{\pgfqpoint{0.023570in}{0.023570in}}%
\pgfpathcurveto{\pgfqpoint{0.017319in}{0.029821in}}{\pgfqpoint{0.008840in}{0.033333in}}{\pgfqpoint{0.000000in}{0.033333in}}%
\pgfpathcurveto{\pgfqpoint{-0.008840in}{0.033333in}}{\pgfqpoint{-0.017319in}{0.029821in}}{\pgfqpoint{-0.023570in}{0.023570in}}%
\pgfpathcurveto{\pgfqpoint{-0.029821in}{0.017319in}}{\pgfqpoint{-0.033333in}{0.008840in}}{\pgfqpoint{-0.033333in}{0.000000in}}%
\pgfpathcurveto{\pgfqpoint{-0.033333in}{-0.008840in}}{\pgfqpoint{-0.029821in}{-0.017319in}}{\pgfqpoint{-0.023570in}{-0.023570in}}%
\pgfpathcurveto{\pgfqpoint{-0.017319in}{-0.029821in}}{\pgfqpoint{-0.008840in}{-0.033333in}}{\pgfqpoint{0.000000in}{-0.033333in}}%
\pgfpathlineto{\pgfqpoint{0.000000in}{-0.033333in}}%
\pgfpathclose%
\pgfusepath{stroke,fill}%
}%
\begin{pgfscope}%
\pgfsys@transformshift{2.941215in}{2.626196in}%
\pgfsys@useobject{currentmarker}{}%
\end{pgfscope}%
\end{pgfscope}%
\begin{pgfscope}%
\pgfpathrectangle{\pgfqpoint{0.765000in}{0.660000in}}{\pgfqpoint{4.620000in}{4.620000in}}%
\pgfusepath{clip}%
\pgfsetrectcap%
\pgfsetroundjoin%
\pgfsetlinewidth{1.204500pt}%
\definecolor{currentstroke}{rgb}{1.000000,0.576471,0.309804}%
\pgfsetstrokecolor{currentstroke}%
\pgfsetdash{}{0pt}%
\pgfpathmoveto{\pgfqpoint{2.914805in}{2.472225in}}%
\pgfusepath{stroke}%
\end{pgfscope}%
\begin{pgfscope}%
\pgfpathrectangle{\pgfqpoint{0.765000in}{0.660000in}}{\pgfqpoint{4.620000in}{4.620000in}}%
\pgfusepath{clip}%
\pgfsetbuttcap%
\pgfsetroundjoin%
\definecolor{currentfill}{rgb}{1.000000,0.576471,0.309804}%
\pgfsetfillcolor{currentfill}%
\pgfsetlinewidth{1.003750pt}%
\definecolor{currentstroke}{rgb}{1.000000,0.576471,0.309804}%
\pgfsetstrokecolor{currentstroke}%
\pgfsetdash{}{0pt}%
\pgfsys@defobject{currentmarker}{\pgfqpoint{-0.033333in}{-0.033333in}}{\pgfqpoint{0.033333in}{0.033333in}}{%
\pgfpathmoveto{\pgfqpoint{0.000000in}{-0.033333in}}%
\pgfpathcurveto{\pgfqpoint{0.008840in}{-0.033333in}}{\pgfqpoint{0.017319in}{-0.029821in}}{\pgfqpoint{0.023570in}{-0.023570in}}%
\pgfpathcurveto{\pgfqpoint{0.029821in}{-0.017319in}}{\pgfqpoint{0.033333in}{-0.008840in}}{\pgfqpoint{0.033333in}{0.000000in}}%
\pgfpathcurveto{\pgfqpoint{0.033333in}{0.008840in}}{\pgfqpoint{0.029821in}{0.017319in}}{\pgfqpoint{0.023570in}{0.023570in}}%
\pgfpathcurveto{\pgfqpoint{0.017319in}{0.029821in}}{\pgfqpoint{0.008840in}{0.033333in}}{\pgfqpoint{0.000000in}{0.033333in}}%
\pgfpathcurveto{\pgfqpoint{-0.008840in}{0.033333in}}{\pgfqpoint{-0.017319in}{0.029821in}}{\pgfqpoint{-0.023570in}{0.023570in}}%
\pgfpathcurveto{\pgfqpoint{-0.029821in}{0.017319in}}{\pgfqpoint{-0.033333in}{0.008840in}}{\pgfqpoint{-0.033333in}{0.000000in}}%
\pgfpathcurveto{\pgfqpoint{-0.033333in}{-0.008840in}}{\pgfqpoint{-0.029821in}{-0.017319in}}{\pgfqpoint{-0.023570in}{-0.023570in}}%
\pgfpathcurveto{\pgfqpoint{-0.017319in}{-0.029821in}}{\pgfqpoint{-0.008840in}{-0.033333in}}{\pgfqpoint{0.000000in}{-0.033333in}}%
\pgfpathlineto{\pgfqpoint{0.000000in}{-0.033333in}}%
\pgfpathclose%
\pgfusepath{stroke,fill}%
}%
\begin{pgfscope}%
\pgfsys@transformshift{2.914805in}{2.472225in}%
\pgfsys@useobject{currentmarker}{}%
\end{pgfscope}%
\end{pgfscope}%
\begin{pgfscope}%
\pgfpathrectangle{\pgfqpoint{0.765000in}{0.660000in}}{\pgfqpoint{4.620000in}{4.620000in}}%
\pgfusepath{clip}%
\pgfsetrectcap%
\pgfsetroundjoin%
\pgfsetlinewidth{1.204500pt}%
\definecolor{currentstroke}{rgb}{1.000000,0.576471,0.309804}%
\pgfsetstrokecolor{currentstroke}%
\pgfsetdash{}{0pt}%
\pgfpathmoveto{\pgfqpoint{4.578999in}{2.759614in}}%
\pgfusepath{stroke}%
\end{pgfscope}%
\begin{pgfscope}%
\pgfpathrectangle{\pgfqpoint{0.765000in}{0.660000in}}{\pgfqpoint{4.620000in}{4.620000in}}%
\pgfusepath{clip}%
\pgfsetbuttcap%
\pgfsetroundjoin%
\definecolor{currentfill}{rgb}{1.000000,0.576471,0.309804}%
\pgfsetfillcolor{currentfill}%
\pgfsetlinewidth{1.003750pt}%
\definecolor{currentstroke}{rgb}{1.000000,0.576471,0.309804}%
\pgfsetstrokecolor{currentstroke}%
\pgfsetdash{}{0pt}%
\pgfsys@defobject{currentmarker}{\pgfqpoint{-0.033333in}{-0.033333in}}{\pgfqpoint{0.033333in}{0.033333in}}{%
\pgfpathmoveto{\pgfqpoint{0.000000in}{-0.033333in}}%
\pgfpathcurveto{\pgfqpoint{0.008840in}{-0.033333in}}{\pgfqpoint{0.017319in}{-0.029821in}}{\pgfqpoint{0.023570in}{-0.023570in}}%
\pgfpathcurveto{\pgfqpoint{0.029821in}{-0.017319in}}{\pgfqpoint{0.033333in}{-0.008840in}}{\pgfqpoint{0.033333in}{0.000000in}}%
\pgfpathcurveto{\pgfqpoint{0.033333in}{0.008840in}}{\pgfqpoint{0.029821in}{0.017319in}}{\pgfqpoint{0.023570in}{0.023570in}}%
\pgfpathcurveto{\pgfqpoint{0.017319in}{0.029821in}}{\pgfqpoint{0.008840in}{0.033333in}}{\pgfqpoint{0.000000in}{0.033333in}}%
\pgfpathcurveto{\pgfqpoint{-0.008840in}{0.033333in}}{\pgfqpoint{-0.017319in}{0.029821in}}{\pgfqpoint{-0.023570in}{0.023570in}}%
\pgfpathcurveto{\pgfqpoint{-0.029821in}{0.017319in}}{\pgfqpoint{-0.033333in}{0.008840in}}{\pgfqpoint{-0.033333in}{0.000000in}}%
\pgfpathcurveto{\pgfqpoint{-0.033333in}{-0.008840in}}{\pgfqpoint{-0.029821in}{-0.017319in}}{\pgfqpoint{-0.023570in}{-0.023570in}}%
\pgfpathcurveto{\pgfqpoint{-0.017319in}{-0.029821in}}{\pgfqpoint{-0.008840in}{-0.033333in}}{\pgfqpoint{0.000000in}{-0.033333in}}%
\pgfpathlineto{\pgfqpoint{0.000000in}{-0.033333in}}%
\pgfpathclose%
\pgfusepath{stroke,fill}%
}%
\begin{pgfscope}%
\pgfsys@transformshift{4.578999in}{2.759614in}%
\pgfsys@useobject{currentmarker}{}%
\end{pgfscope}%
\end{pgfscope}%
\begin{pgfscope}%
\pgfpathrectangle{\pgfqpoint{0.765000in}{0.660000in}}{\pgfqpoint{4.620000in}{4.620000in}}%
\pgfusepath{clip}%
\pgfsetrectcap%
\pgfsetroundjoin%
\pgfsetlinewidth{1.204500pt}%
\definecolor{currentstroke}{rgb}{1.000000,0.576471,0.309804}%
\pgfsetstrokecolor{currentstroke}%
\pgfsetdash{}{0pt}%
\pgfpathmoveto{\pgfqpoint{2.772493in}{2.476560in}}%
\pgfusepath{stroke}%
\end{pgfscope}%
\begin{pgfscope}%
\pgfpathrectangle{\pgfqpoint{0.765000in}{0.660000in}}{\pgfqpoint{4.620000in}{4.620000in}}%
\pgfusepath{clip}%
\pgfsetbuttcap%
\pgfsetroundjoin%
\definecolor{currentfill}{rgb}{1.000000,0.576471,0.309804}%
\pgfsetfillcolor{currentfill}%
\pgfsetlinewidth{1.003750pt}%
\definecolor{currentstroke}{rgb}{1.000000,0.576471,0.309804}%
\pgfsetstrokecolor{currentstroke}%
\pgfsetdash{}{0pt}%
\pgfsys@defobject{currentmarker}{\pgfqpoint{-0.033333in}{-0.033333in}}{\pgfqpoint{0.033333in}{0.033333in}}{%
\pgfpathmoveto{\pgfqpoint{0.000000in}{-0.033333in}}%
\pgfpathcurveto{\pgfqpoint{0.008840in}{-0.033333in}}{\pgfqpoint{0.017319in}{-0.029821in}}{\pgfqpoint{0.023570in}{-0.023570in}}%
\pgfpathcurveto{\pgfqpoint{0.029821in}{-0.017319in}}{\pgfqpoint{0.033333in}{-0.008840in}}{\pgfqpoint{0.033333in}{0.000000in}}%
\pgfpathcurveto{\pgfqpoint{0.033333in}{0.008840in}}{\pgfqpoint{0.029821in}{0.017319in}}{\pgfqpoint{0.023570in}{0.023570in}}%
\pgfpathcurveto{\pgfqpoint{0.017319in}{0.029821in}}{\pgfqpoint{0.008840in}{0.033333in}}{\pgfqpoint{0.000000in}{0.033333in}}%
\pgfpathcurveto{\pgfqpoint{-0.008840in}{0.033333in}}{\pgfqpoint{-0.017319in}{0.029821in}}{\pgfqpoint{-0.023570in}{0.023570in}}%
\pgfpathcurveto{\pgfqpoint{-0.029821in}{0.017319in}}{\pgfqpoint{-0.033333in}{0.008840in}}{\pgfqpoint{-0.033333in}{0.000000in}}%
\pgfpathcurveto{\pgfqpoint{-0.033333in}{-0.008840in}}{\pgfqpoint{-0.029821in}{-0.017319in}}{\pgfqpoint{-0.023570in}{-0.023570in}}%
\pgfpathcurveto{\pgfqpoint{-0.017319in}{-0.029821in}}{\pgfqpoint{-0.008840in}{-0.033333in}}{\pgfqpoint{0.000000in}{-0.033333in}}%
\pgfpathlineto{\pgfqpoint{0.000000in}{-0.033333in}}%
\pgfpathclose%
\pgfusepath{stroke,fill}%
}%
\begin{pgfscope}%
\pgfsys@transformshift{2.772493in}{2.476560in}%
\pgfsys@useobject{currentmarker}{}%
\end{pgfscope}%
\end{pgfscope}%
\begin{pgfscope}%
\pgfpathrectangle{\pgfqpoint{0.765000in}{0.660000in}}{\pgfqpoint{4.620000in}{4.620000in}}%
\pgfusepath{clip}%
\pgfsetrectcap%
\pgfsetroundjoin%
\pgfsetlinewidth{1.204500pt}%
\definecolor{currentstroke}{rgb}{1.000000,0.576471,0.309804}%
\pgfsetstrokecolor{currentstroke}%
\pgfsetdash{}{0pt}%
\pgfpathmoveto{\pgfqpoint{4.282625in}{2.715273in}}%
\pgfusepath{stroke}%
\end{pgfscope}%
\begin{pgfscope}%
\pgfpathrectangle{\pgfqpoint{0.765000in}{0.660000in}}{\pgfqpoint{4.620000in}{4.620000in}}%
\pgfusepath{clip}%
\pgfsetbuttcap%
\pgfsetroundjoin%
\definecolor{currentfill}{rgb}{1.000000,0.576471,0.309804}%
\pgfsetfillcolor{currentfill}%
\pgfsetlinewidth{1.003750pt}%
\definecolor{currentstroke}{rgb}{1.000000,0.576471,0.309804}%
\pgfsetstrokecolor{currentstroke}%
\pgfsetdash{}{0pt}%
\pgfsys@defobject{currentmarker}{\pgfqpoint{-0.033333in}{-0.033333in}}{\pgfqpoint{0.033333in}{0.033333in}}{%
\pgfpathmoveto{\pgfqpoint{0.000000in}{-0.033333in}}%
\pgfpathcurveto{\pgfqpoint{0.008840in}{-0.033333in}}{\pgfqpoint{0.017319in}{-0.029821in}}{\pgfqpoint{0.023570in}{-0.023570in}}%
\pgfpathcurveto{\pgfqpoint{0.029821in}{-0.017319in}}{\pgfqpoint{0.033333in}{-0.008840in}}{\pgfqpoint{0.033333in}{0.000000in}}%
\pgfpathcurveto{\pgfqpoint{0.033333in}{0.008840in}}{\pgfqpoint{0.029821in}{0.017319in}}{\pgfqpoint{0.023570in}{0.023570in}}%
\pgfpathcurveto{\pgfqpoint{0.017319in}{0.029821in}}{\pgfqpoint{0.008840in}{0.033333in}}{\pgfqpoint{0.000000in}{0.033333in}}%
\pgfpathcurveto{\pgfqpoint{-0.008840in}{0.033333in}}{\pgfqpoint{-0.017319in}{0.029821in}}{\pgfqpoint{-0.023570in}{0.023570in}}%
\pgfpathcurveto{\pgfqpoint{-0.029821in}{0.017319in}}{\pgfqpoint{-0.033333in}{0.008840in}}{\pgfqpoint{-0.033333in}{0.000000in}}%
\pgfpathcurveto{\pgfqpoint{-0.033333in}{-0.008840in}}{\pgfqpoint{-0.029821in}{-0.017319in}}{\pgfqpoint{-0.023570in}{-0.023570in}}%
\pgfpathcurveto{\pgfqpoint{-0.017319in}{-0.029821in}}{\pgfqpoint{-0.008840in}{-0.033333in}}{\pgfqpoint{0.000000in}{-0.033333in}}%
\pgfpathlineto{\pgfqpoint{0.000000in}{-0.033333in}}%
\pgfpathclose%
\pgfusepath{stroke,fill}%
}%
\begin{pgfscope}%
\pgfsys@transformshift{4.282625in}{2.715273in}%
\pgfsys@useobject{currentmarker}{}%
\end{pgfscope}%
\end{pgfscope}%
\begin{pgfscope}%
\pgfpathrectangle{\pgfqpoint{0.765000in}{0.660000in}}{\pgfqpoint{4.620000in}{4.620000in}}%
\pgfusepath{clip}%
\pgfsetrectcap%
\pgfsetroundjoin%
\pgfsetlinewidth{1.204500pt}%
\definecolor{currentstroke}{rgb}{1.000000,0.576471,0.309804}%
\pgfsetstrokecolor{currentstroke}%
\pgfsetdash{}{0pt}%
\pgfpathmoveto{\pgfqpoint{2.728731in}{3.054675in}}%
\pgfusepath{stroke}%
\end{pgfscope}%
\begin{pgfscope}%
\pgfpathrectangle{\pgfqpoint{0.765000in}{0.660000in}}{\pgfqpoint{4.620000in}{4.620000in}}%
\pgfusepath{clip}%
\pgfsetbuttcap%
\pgfsetroundjoin%
\definecolor{currentfill}{rgb}{1.000000,0.576471,0.309804}%
\pgfsetfillcolor{currentfill}%
\pgfsetlinewidth{1.003750pt}%
\definecolor{currentstroke}{rgb}{1.000000,0.576471,0.309804}%
\pgfsetstrokecolor{currentstroke}%
\pgfsetdash{}{0pt}%
\pgfsys@defobject{currentmarker}{\pgfqpoint{-0.033333in}{-0.033333in}}{\pgfqpoint{0.033333in}{0.033333in}}{%
\pgfpathmoveto{\pgfqpoint{0.000000in}{-0.033333in}}%
\pgfpathcurveto{\pgfqpoint{0.008840in}{-0.033333in}}{\pgfqpoint{0.017319in}{-0.029821in}}{\pgfqpoint{0.023570in}{-0.023570in}}%
\pgfpathcurveto{\pgfqpoint{0.029821in}{-0.017319in}}{\pgfqpoint{0.033333in}{-0.008840in}}{\pgfqpoint{0.033333in}{0.000000in}}%
\pgfpathcurveto{\pgfqpoint{0.033333in}{0.008840in}}{\pgfqpoint{0.029821in}{0.017319in}}{\pgfqpoint{0.023570in}{0.023570in}}%
\pgfpathcurveto{\pgfqpoint{0.017319in}{0.029821in}}{\pgfqpoint{0.008840in}{0.033333in}}{\pgfqpoint{0.000000in}{0.033333in}}%
\pgfpathcurveto{\pgfqpoint{-0.008840in}{0.033333in}}{\pgfqpoint{-0.017319in}{0.029821in}}{\pgfqpoint{-0.023570in}{0.023570in}}%
\pgfpathcurveto{\pgfqpoint{-0.029821in}{0.017319in}}{\pgfqpoint{-0.033333in}{0.008840in}}{\pgfqpoint{-0.033333in}{0.000000in}}%
\pgfpathcurveto{\pgfqpoint{-0.033333in}{-0.008840in}}{\pgfqpoint{-0.029821in}{-0.017319in}}{\pgfqpoint{-0.023570in}{-0.023570in}}%
\pgfpathcurveto{\pgfqpoint{-0.017319in}{-0.029821in}}{\pgfqpoint{-0.008840in}{-0.033333in}}{\pgfqpoint{0.000000in}{-0.033333in}}%
\pgfpathlineto{\pgfqpoint{0.000000in}{-0.033333in}}%
\pgfpathclose%
\pgfusepath{stroke,fill}%
}%
\begin{pgfscope}%
\pgfsys@transformshift{2.728731in}{3.054675in}%
\pgfsys@useobject{currentmarker}{}%
\end{pgfscope}%
\end{pgfscope}%
\begin{pgfscope}%
\pgfpathrectangle{\pgfqpoint{0.765000in}{0.660000in}}{\pgfqpoint{4.620000in}{4.620000in}}%
\pgfusepath{clip}%
\pgfsetrectcap%
\pgfsetroundjoin%
\pgfsetlinewidth{1.204500pt}%
\definecolor{currentstroke}{rgb}{1.000000,0.576471,0.309804}%
\pgfsetstrokecolor{currentstroke}%
\pgfsetdash{}{0pt}%
\pgfpathmoveto{\pgfqpoint{2.730830in}{2.534173in}}%
\pgfusepath{stroke}%
\end{pgfscope}%
\begin{pgfscope}%
\pgfpathrectangle{\pgfqpoint{0.765000in}{0.660000in}}{\pgfqpoint{4.620000in}{4.620000in}}%
\pgfusepath{clip}%
\pgfsetbuttcap%
\pgfsetroundjoin%
\definecolor{currentfill}{rgb}{1.000000,0.576471,0.309804}%
\pgfsetfillcolor{currentfill}%
\pgfsetlinewidth{1.003750pt}%
\definecolor{currentstroke}{rgb}{1.000000,0.576471,0.309804}%
\pgfsetstrokecolor{currentstroke}%
\pgfsetdash{}{0pt}%
\pgfsys@defobject{currentmarker}{\pgfqpoint{-0.033333in}{-0.033333in}}{\pgfqpoint{0.033333in}{0.033333in}}{%
\pgfpathmoveto{\pgfqpoint{0.000000in}{-0.033333in}}%
\pgfpathcurveto{\pgfqpoint{0.008840in}{-0.033333in}}{\pgfqpoint{0.017319in}{-0.029821in}}{\pgfqpoint{0.023570in}{-0.023570in}}%
\pgfpathcurveto{\pgfqpoint{0.029821in}{-0.017319in}}{\pgfqpoint{0.033333in}{-0.008840in}}{\pgfqpoint{0.033333in}{0.000000in}}%
\pgfpathcurveto{\pgfqpoint{0.033333in}{0.008840in}}{\pgfqpoint{0.029821in}{0.017319in}}{\pgfqpoint{0.023570in}{0.023570in}}%
\pgfpathcurveto{\pgfqpoint{0.017319in}{0.029821in}}{\pgfqpoint{0.008840in}{0.033333in}}{\pgfqpoint{0.000000in}{0.033333in}}%
\pgfpathcurveto{\pgfqpoint{-0.008840in}{0.033333in}}{\pgfqpoint{-0.017319in}{0.029821in}}{\pgfqpoint{-0.023570in}{0.023570in}}%
\pgfpathcurveto{\pgfqpoint{-0.029821in}{0.017319in}}{\pgfqpoint{-0.033333in}{0.008840in}}{\pgfqpoint{-0.033333in}{0.000000in}}%
\pgfpathcurveto{\pgfqpoint{-0.033333in}{-0.008840in}}{\pgfqpoint{-0.029821in}{-0.017319in}}{\pgfqpoint{-0.023570in}{-0.023570in}}%
\pgfpathcurveto{\pgfqpoint{-0.017319in}{-0.029821in}}{\pgfqpoint{-0.008840in}{-0.033333in}}{\pgfqpoint{0.000000in}{-0.033333in}}%
\pgfpathlineto{\pgfqpoint{0.000000in}{-0.033333in}}%
\pgfpathclose%
\pgfusepath{stroke,fill}%
}%
\begin{pgfscope}%
\pgfsys@transformshift{2.730830in}{2.534173in}%
\pgfsys@useobject{currentmarker}{}%
\end{pgfscope}%
\end{pgfscope}%
\begin{pgfscope}%
\pgfpathrectangle{\pgfqpoint{0.765000in}{0.660000in}}{\pgfqpoint{4.620000in}{4.620000in}}%
\pgfusepath{clip}%
\pgfsetrectcap%
\pgfsetroundjoin%
\pgfsetlinewidth{1.204500pt}%
\definecolor{currentstroke}{rgb}{1.000000,0.576471,0.309804}%
\pgfsetstrokecolor{currentstroke}%
\pgfsetdash{}{0pt}%
\pgfpathmoveto{\pgfqpoint{2.748937in}{2.582087in}}%
\pgfusepath{stroke}%
\end{pgfscope}%
\begin{pgfscope}%
\pgfpathrectangle{\pgfqpoint{0.765000in}{0.660000in}}{\pgfqpoint{4.620000in}{4.620000in}}%
\pgfusepath{clip}%
\pgfsetbuttcap%
\pgfsetroundjoin%
\definecolor{currentfill}{rgb}{1.000000,0.576471,0.309804}%
\pgfsetfillcolor{currentfill}%
\pgfsetlinewidth{1.003750pt}%
\definecolor{currentstroke}{rgb}{1.000000,0.576471,0.309804}%
\pgfsetstrokecolor{currentstroke}%
\pgfsetdash{}{0pt}%
\pgfsys@defobject{currentmarker}{\pgfqpoint{-0.033333in}{-0.033333in}}{\pgfqpoint{0.033333in}{0.033333in}}{%
\pgfpathmoveto{\pgfqpoint{0.000000in}{-0.033333in}}%
\pgfpathcurveto{\pgfqpoint{0.008840in}{-0.033333in}}{\pgfqpoint{0.017319in}{-0.029821in}}{\pgfqpoint{0.023570in}{-0.023570in}}%
\pgfpathcurveto{\pgfqpoint{0.029821in}{-0.017319in}}{\pgfqpoint{0.033333in}{-0.008840in}}{\pgfqpoint{0.033333in}{0.000000in}}%
\pgfpathcurveto{\pgfqpoint{0.033333in}{0.008840in}}{\pgfqpoint{0.029821in}{0.017319in}}{\pgfqpoint{0.023570in}{0.023570in}}%
\pgfpathcurveto{\pgfqpoint{0.017319in}{0.029821in}}{\pgfqpoint{0.008840in}{0.033333in}}{\pgfqpoint{0.000000in}{0.033333in}}%
\pgfpathcurveto{\pgfqpoint{-0.008840in}{0.033333in}}{\pgfqpoint{-0.017319in}{0.029821in}}{\pgfqpoint{-0.023570in}{0.023570in}}%
\pgfpathcurveto{\pgfqpoint{-0.029821in}{0.017319in}}{\pgfqpoint{-0.033333in}{0.008840in}}{\pgfqpoint{-0.033333in}{0.000000in}}%
\pgfpathcurveto{\pgfqpoint{-0.033333in}{-0.008840in}}{\pgfqpoint{-0.029821in}{-0.017319in}}{\pgfqpoint{-0.023570in}{-0.023570in}}%
\pgfpathcurveto{\pgfqpoint{-0.017319in}{-0.029821in}}{\pgfqpoint{-0.008840in}{-0.033333in}}{\pgfqpoint{0.000000in}{-0.033333in}}%
\pgfpathlineto{\pgfqpoint{0.000000in}{-0.033333in}}%
\pgfpathclose%
\pgfusepath{stroke,fill}%
}%
\begin{pgfscope}%
\pgfsys@transformshift{2.748937in}{2.582087in}%
\pgfsys@useobject{currentmarker}{}%
\end{pgfscope}%
\end{pgfscope}%
\begin{pgfscope}%
\pgfpathrectangle{\pgfqpoint{0.765000in}{0.660000in}}{\pgfqpoint{4.620000in}{4.620000in}}%
\pgfusepath{clip}%
\pgfsetrectcap%
\pgfsetroundjoin%
\pgfsetlinewidth{1.204500pt}%
\definecolor{currentstroke}{rgb}{1.000000,0.576471,0.309804}%
\pgfsetstrokecolor{currentstroke}%
\pgfsetdash{}{0pt}%
\pgfpathmoveto{\pgfqpoint{4.787800in}{3.325402in}}%
\pgfusepath{stroke}%
\end{pgfscope}%
\begin{pgfscope}%
\pgfpathrectangle{\pgfqpoint{0.765000in}{0.660000in}}{\pgfqpoint{4.620000in}{4.620000in}}%
\pgfusepath{clip}%
\pgfsetbuttcap%
\pgfsetroundjoin%
\definecolor{currentfill}{rgb}{1.000000,0.576471,0.309804}%
\pgfsetfillcolor{currentfill}%
\pgfsetlinewidth{1.003750pt}%
\definecolor{currentstroke}{rgb}{1.000000,0.576471,0.309804}%
\pgfsetstrokecolor{currentstroke}%
\pgfsetdash{}{0pt}%
\pgfsys@defobject{currentmarker}{\pgfqpoint{-0.033333in}{-0.033333in}}{\pgfqpoint{0.033333in}{0.033333in}}{%
\pgfpathmoveto{\pgfqpoint{0.000000in}{-0.033333in}}%
\pgfpathcurveto{\pgfqpoint{0.008840in}{-0.033333in}}{\pgfqpoint{0.017319in}{-0.029821in}}{\pgfqpoint{0.023570in}{-0.023570in}}%
\pgfpathcurveto{\pgfqpoint{0.029821in}{-0.017319in}}{\pgfqpoint{0.033333in}{-0.008840in}}{\pgfqpoint{0.033333in}{0.000000in}}%
\pgfpathcurveto{\pgfqpoint{0.033333in}{0.008840in}}{\pgfqpoint{0.029821in}{0.017319in}}{\pgfqpoint{0.023570in}{0.023570in}}%
\pgfpathcurveto{\pgfqpoint{0.017319in}{0.029821in}}{\pgfqpoint{0.008840in}{0.033333in}}{\pgfqpoint{0.000000in}{0.033333in}}%
\pgfpathcurveto{\pgfqpoint{-0.008840in}{0.033333in}}{\pgfqpoint{-0.017319in}{0.029821in}}{\pgfqpoint{-0.023570in}{0.023570in}}%
\pgfpathcurveto{\pgfqpoint{-0.029821in}{0.017319in}}{\pgfqpoint{-0.033333in}{0.008840in}}{\pgfqpoint{-0.033333in}{0.000000in}}%
\pgfpathcurveto{\pgfqpoint{-0.033333in}{-0.008840in}}{\pgfqpoint{-0.029821in}{-0.017319in}}{\pgfqpoint{-0.023570in}{-0.023570in}}%
\pgfpathcurveto{\pgfqpoint{-0.017319in}{-0.029821in}}{\pgfqpoint{-0.008840in}{-0.033333in}}{\pgfqpoint{0.000000in}{-0.033333in}}%
\pgfpathlineto{\pgfqpoint{0.000000in}{-0.033333in}}%
\pgfpathclose%
\pgfusepath{stroke,fill}%
}%
\begin{pgfscope}%
\pgfsys@transformshift{4.787800in}{3.325402in}%
\pgfsys@useobject{currentmarker}{}%
\end{pgfscope}%
\end{pgfscope}%
\begin{pgfscope}%
\pgfpathrectangle{\pgfqpoint{0.765000in}{0.660000in}}{\pgfqpoint{4.620000in}{4.620000in}}%
\pgfusepath{clip}%
\pgfsetrectcap%
\pgfsetroundjoin%
\pgfsetlinewidth{1.204500pt}%
\definecolor{currentstroke}{rgb}{1.000000,0.576471,0.309804}%
\pgfsetstrokecolor{currentstroke}%
\pgfsetdash{}{0pt}%
\pgfpathmoveto{\pgfqpoint{3.191927in}{2.342642in}}%
\pgfusepath{stroke}%
\end{pgfscope}%
\begin{pgfscope}%
\pgfpathrectangle{\pgfqpoint{0.765000in}{0.660000in}}{\pgfqpoint{4.620000in}{4.620000in}}%
\pgfusepath{clip}%
\pgfsetbuttcap%
\pgfsetroundjoin%
\definecolor{currentfill}{rgb}{1.000000,0.576471,0.309804}%
\pgfsetfillcolor{currentfill}%
\pgfsetlinewidth{1.003750pt}%
\definecolor{currentstroke}{rgb}{1.000000,0.576471,0.309804}%
\pgfsetstrokecolor{currentstroke}%
\pgfsetdash{}{0pt}%
\pgfsys@defobject{currentmarker}{\pgfqpoint{-0.033333in}{-0.033333in}}{\pgfqpoint{0.033333in}{0.033333in}}{%
\pgfpathmoveto{\pgfqpoint{0.000000in}{-0.033333in}}%
\pgfpathcurveto{\pgfqpoint{0.008840in}{-0.033333in}}{\pgfqpoint{0.017319in}{-0.029821in}}{\pgfqpoint{0.023570in}{-0.023570in}}%
\pgfpathcurveto{\pgfqpoint{0.029821in}{-0.017319in}}{\pgfqpoint{0.033333in}{-0.008840in}}{\pgfqpoint{0.033333in}{0.000000in}}%
\pgfpathcurveto{\pgfqpoint{0.033333in}{0.008840in}}{\pgfqpoint{0.029821in}{0.017319in}}{\pgfqpoint{0.023570in}{0.023570in}}%
\pgfpathcurveto{\pgfqpoint{0.017319in}{0.029821in}}{\pgfqpoint{0.008840in}{0.033333in}}{\pgfqpoint{0.000000in}{0.033333in}}%
\pgfpathcurveto{\pgfqpoint{-0.008840in}{0.033333in}}{\pgfqpoint{-0.017319in}{0.029821in}}{\pgfqpoint{-0.023570in}{0.023570in}}%
\pgfpathcurveto{\pgfqpoint{-0.029821in}{0.017319in}}{\pgfqpoint{-0.033333in}{0.008840in}}{\pgfqpoint{-0.033333in}{0.000000in}}%
\pgfpathcurveto{\pgfqpoint{-0.033333in}{-0.008840in}}{\pgfqpoint{-0.029821in}{-0.017319in}}{\pgfqpoint{-0.023570in}{-0.023570in}}%
\pgfpathcurveto{\pgfqpoint{-0.017319in}{-0.029821in}}{\pgfqpoint{-0.008840in}{-0.033333in}}{\pgfqpoint{0.000000in}{-0.033333in}}%
\pgfpathlineto{\pgfqpoint{0.000000in}{-0.033333in}}%
\pgfpathclose%
\pgfusepath{stroke,fill}%
}%
\begin{pgfscope}%
\pgfsys@transformshift{3.191927in}{2.342642in}%
\pgfsys@useobject{currentmarker}{}%
\end{pgfscope}%
\end{pgfscope}%
\begin{pgfscope}%
\pgfpathrectangle{\pgfqpoint{0.765000in}{0.660000in}}{\pgfqpoint{4.620000in}{4.620000in}}%
\pgfusepath{clip}%
\pgfsetrectcap%
\pgfsetroundjoin%
\pgfsetlinewidth{1.204500pt}%
\definecolor{currentstroke}{rgb}{1.000000,0.576471,0.309804}%
\pgfsetstrokecolor{currentstroke}%
\pgfsetdash{}{0pt}%
\pgfpathmoveto{\pgfqpoint{4.091404in}{2.683291in}}%
\pgfusepath{stroke}%
\end{pgfscope}%
\begin{pgfscope}%
\pgfpathrectangle{\pgfqpoint{0.765000in}{0.660000in}}{\pgfqpoint{4.620000in}{4.620000in}}%
\pgfusepath{clip}%
\pgfsetbuttcap%
\pgfsetroundjoin%
\definecolor{currentfill}{rgb}{1.000000,0.576471,0.309804}%
\pgfsetfillcolor{currentfill}%
\pgfsetlinewidth{1.003750pt}%
\definecolor{currentstroke}{rgb}{1.000000,0.576471,0.309804}%
\pgfsetstrokecolor{currentstroke}%
\pgfsetdash{}{0pt}%
\pgfsys@defobject{currentmarker}{\pgfqpoint{-0.033333in}{-0.033333in}}{\pgfqpoint{0.033333in}{0.033333in}}{%
\pgfpathmoveto{\pgfqpoint{0.000000in}{-0.033333in}}%
\pgfpathcurveto{\pgfqpoint{0.008840in}{-0.033333in}}{\pgfqpoint{0.017319in}{-0.029821in}}{\pgfqpoint{0.023570in}{-0.023570in}}%
\pgfpathcurveto{\pgfqpoint{0.029821in}{-0.017319in}}{\pgfqpoint{0.033333in}{-0.008840in}}{\pgfqpoint{0.033333in}{0.000000in}}%
\pgfpathcurveto{\pgfqpoint{0.033333in}{0.008840in}}{\pgfqpoint{0.029821in}{0.017319in}}{\pgfqpoint{0.023570in}{0.023570in}}%
\pgfpathcurveto{\pgfqpoint{0.017319in}{0.029821in}}{\pgfqpoint{0.008840in}{0.033333in}}{\pgfqpoint{0.000000in}{0.033333in}}%
\pgfpathcurveto{\pgfqpoint{-0.008840in}{0.033333in}}{\pgfqpoint{-0.017319in}{0.029821in}}{\pgfqpoint{-0.023570in}{0.023570in}}%
\pgfpathcurveto{\pgfqpoint{-0.029821in}{0.017319in}}{\pgfqpoint{-0.033333in}{0.008840in}}{\pgfqpoint{-0.033333in}{0.000000in}}%
\pgfpathcurveto{\pgfqpoint{-0.033333in}{-0.008840in}}{\pgfqpoint{-0.029821in}{-0.017319in}}{\pgfqpoint{-0.023570in}{-0.023570in}}%
\pgfpathcurveto{\pgfqpoint{-0.017319in}{-0.029821in}}{\pgfqpoint{-0.008840in}{-0.033333in}}{\pgfqpoint{0.000000in}{-0.033333in}}%
\pgfpathlineto{\pgfqpoint{0.000000in}{-0.033333in}}%
\pgfpathclose%
\pgfusepath{stroke,fill}%
}%
\begin{pgfscope}%
\pgfsys@transformshift{4.091404in}{2.683291in}%
\pgfsys@useobject{currentmarker}{}%
\end{pgfscope}%
\end{pgfscope}%
\begin{pgfscope}%
\pgfpathrectangle{\pgfqpoint{0.765000in}{0.660000in}}{\pgfqpoint{4.620000in}{4.620000in}}%
\pgfusepath{clip}%
\pgfsetrectcap%
\pgfsetroundjoin%
\pgfsetlinewidth{1.204500pt}%
\definecolor{currentstroke}{rgb}{1.000000,0.576471,0.309804}%
\pgfsetstrokecolor{currentstroke}%
\pgfsetdash{}{0pt}%
\pgfpathmoveto{\pgfqpoint{3.054185in}{3.625207in}}%
\pgfusepath{stroke}%
\end{pgfscope}%
\begin{pgfscope}%
\pgfpathrectangle{\pgfqpoint{0.765000in}{0.660000in}}{\pgfqpoint{4.620000in}{4.620000in}}%
\pgfusepath{clip}%
\pgfsetbuttcap%
\pgfsetroundjoin%
\definecolor{currentfill}{rgb}{1.000000,0.576471,0.309804}%
\pgfsetfillcolor{currentfill}%
\pgfsetlinewidth{1.003750pt}%
\definecolor{currentstroke}{rgb}{1.000000,0.576471,0.309804}%
\pgfsetstrokecolor{currentstroke}%
\pgfsetdash{}{0pt}%
\pgfsys@defobject{currentmarker}{\pgfqpoint{-0.033333in}{-0.033333in}}{\pgfqpoint{0.033333in}{0.033333in}}{%
\pgfpathmoveto{\pgfqpoint{0.000000in}{-0.033333in}}%
\pgfpathcurveto{\pgfqpoint{0.008840in}{-0.033333in}}{\pgfqpoint{0.017319in}{-0.029821in}}{\pgfqpoint{0.023570in}{-0.023570in}}%
\pgfpathcurveto{\pgfqpoint{0.029821in}{-0.017319in}}{\pgfqpoint{0.033333in}{-0.008840in}}{\pgfqpoint{0.033333in}{0.000000in}}%
\pgfpathcurveto{\pgfqpoint{0.033333in}{0.008840in}}{\pgfqpoint{0.029821in}{0.017319in}}{\pgfqpoint{0.023570in}{0.023570in}}%
\pgfpathcurveto{\pgfqpoint{0.017319in}{0.029821in}}{\pgfqpoint{0.008840in}{0.033333in}}{\pgfqpoint{0.000000in}{0.033333in}}%
\pgfpathcurveto{\pgfqpoint{-0.008840in}{0.033333in}}{\pgfqpoint{-0.017319in}{0.029821in}}{\pgfqpoint{-0.023570in}{0.023570in}}%
\pgfpathcurveto{\pgfqpoint{-0.029821in}{0.017319in}}{\pgfqpoint{-0.033333in}{0.008840in}}{\pgfqpoint{-0.033333in}{0.000000in}}%
\pgfpathcurveto{\pgfqpoint{-0.033333in}{-0.008840in}}{\pgfqpoint{-0.029821in}{-0.017319in}}{\pgfqpoint{-0.023570in}{-0.023570in}}%
\pgfpathcurveto{\pgfqpoint{-0.017319in}{-0.029821in}}{\pgfqpoint{-0.008840in}{-0.033333in}}{\pgfqpoint{0.000000in}{-0.033333in}}%
\pgfpathlineto{\pgfqpoint{0.000000in}{-0.033333in}}%
\pgfpathclose%
\pgfusepath{stroke,fill}%
}%
\begin{pgfscope}%
\pgfsys@transformshift{3.054185in}{3.625207in}%
\pgfsys@useobject{currentmarker}{}%
\end{pgfscope}%
\end{pgfscope}%
\begin{pgfscope}%
\pgfpathrectangle{\pgfqpoint{0.765000in}{0.660000in}}{\pgfqpoint{4.620000in}{4.620000in}}%
\pgfusepath{clip}%
\pgfsetrectcap%
\pgfsetroundjoin%
\pgfsetlinewidth{1.204500pt}%
\definecolor{currentstroke}{rgb}{1.000000,0.576471,0.309804}%
\pgfsetstrokecolor{currentstroke}%
\pgfsetdash{}{0pt}%
\pgfpathmoveto{\pgfqpoint{3.402781in}{2.675681in}}%
\pgfusepath{stroke}%
\end{pgfscope}%
\begin{pgfscope}%
\pgfpathrectangle{\pgfqpoint{0.765000in}{0.660000in}}{\pgfqpoint{4.620000in}{4.620000in}}%
\pgfusepath{clip}%
\pgfsetbuttcap%
\pgfsetroundjoin%
\definecolor{currentfill}{rgb}{1.000000,0.576471,0.309804}%
\pgfsetfillcolor{currentfill}%
\pgfsetlinewidth{1.003750pt}%
\definecolor{currentstroke}{rgb}{1.000000,0.576471,0.309804}%
\pgfsetstrokecolor{currentstroke}%
\pgfsetdash{}{0pt}%
\pgfsys@defobject{currentmarker}{\pgfqpoint{-0.033333in}{-0.033333in}}{\pgfqpoint{0.033333in}{0.033333in}}{%
\pgfpathmoveto{\pgfqpoint{0.000000in}{-0.033333in}}%
\pgfpathcurveto{\pgfqpoint{0.008840in}{-0.033333in}}{\pgfqpoint{0.017319in}{-0.029821in}}{\pgfqpoint{0.023570in}{-0.023570in}}%
\pgfpathcurveto{\pgfqpoint{0.029821in}{-0.017319in}}{\pgfqpoint{0.033333in}{-0.008840in}}{\pgfqpoint{0.033333in}{0.000000in}}%
\pgfpathcurveto{\pgfqpoint{0.033333in}{0.008840in}}{\pgfqpoint{0.029821in}{0.017319in}}{\pgfqpoint{0.023570in}{0.023570in}}%
\pgfpathcurveto{\pgfqpoint{0.017319in}{0.029821in}}{\pgfqpoint{0.008840in}{0.033333in}}{\pgfqpoint{0.000000in}{0.033333in}}%
\pgfpathcurveto{\pgfqpoint{-0.008840in}{0.033333in}}{\pgfqpoint{-0.017319in}{0.029821in}}{\pgfqpoint{-0.023570in}{0.023570in}}%
\pgfpathcurveto{\pgfqpoint{-0.029821in}{0.017319in}}{\pgfqpoint{-0.033333in}{0.008840in}}{\pgfqpoint{-0.033333in}{0.000000in}}%
\pgfpathcurveto{\pgfqpoint{-0.033333in}{-0.008840in}}{\pgfqpoint{-0.029821in}{-0.017319in}}{\pgfqpoint{-0.023570in}{-0.023570in}}%
\pgfpathcurveto{\pgfqpoint{-0.017319in}{-0.029821in}}{\pgfqpoint{-0.008840in}{-0.033333in}}{\pgfqpoint{0.000000in}{-0.033333in}}%
\pgfpathlineto{\pgfqpoint{0.000000in}{-0.033333in}}%
\pgfpathclose%
\pgfusepath{stroke,fill}%
}%
\begin{pgfscope}%
\pgfsys@transformshift{3.402781in}{2.675681in}%
\pgfsys@useobject{currentmarker}{}%
\end{pgfscope}%
\end{pgfscope}%
\begin{pgfscope}%
\pgfpathrectangle{\pgfqpoint{0.765000in}{0.660000in}}{\pgfqpoint{4.620000in}{4.620000in}}%
\pgfusepath{clip}%
\pgfsetrectcap%
\pgfsetroundjoin%
\pgfsetlinewidth{1.204500pt}%
\definecolor{currentstroke}{rgb}{1.000000,0.576471,0.309804}%
\pgfsetstrokecolor{currentstroke}%
\pgfsetdash{}{0pt}%
\pgfpathmoveto{\pgfqpoint{3.760564in}{2.629743in}}%
\pgfusepath{stroke}%
\end{pgfscope}%
\begin{pgfscope}%
\pgfpathrectangle{\pgfqpoint{0.765000in}{0.660000in}}{\pgfqpoint{4.620000in}{4.620000in}}%
\pgfusepath{clip}%
\pgfsetbuttcap%
\pgfsetroundjoin%
\definecolor{currentfill}{rgb}{1.000000,0.576471,0.309804}%
\pgfsetfillcolor{currentfill}%
\pgfsetlinewidth{1.003750pt}%
\definecolor{currentstroke}{rgb}{1.000000,0.576471,0.309804}%
\pgfsetstrokecolor{currentstroke}%
\pgfsetdash{}{0pt}%
\pgfsys@defobject{currentmarker}{\pgfqpoint{-0.033333in}{-0.033333in}}{\pgfqpoint{0.033333in}{0.033333in}}{%
\pgfpathmoveto{\pgfqpoint{0.000000in}{-0.033333in}}%
\pgfpathcurveto{\pgfqpoint{0.008840in}{-0.033333in}}{\pgfqpoint{0.017319in}{-0.029821in}}{\pgfqpoint{0.023570in}{-0.023570in}}%
\pgfpathcurveto{\pgfqpoint{0.029821in}{-0.017319in}}{\pgfqpoint{0.033333in}{-0.008840in}}{\pgfqpoint{0.033333in}{0.000000in}}%
\pgfpathcurveto{\pgfqpoint{0.033333in}{0.008840in}}{\pgfqpoint{0.029821in}{0.017319in}}{\pgfqpoint{0.023570in}{0.023570in}}%
\pgfpathcurveto{\pgfqpoint{0.017319in}{0.029821in}}{\pgfqpoint{0.008840in}{0.033333in}}{\pgfqpoint{0.000000in}{0.033333in}}%
\pgfpathcurveto{\pgfqpoint{-0.008840in}{0.033333in}}{\pgfqpoint{-0.017319in}{0.029821in}}{\pgfqpoint{-0.023570in}{0.023570in}}%
\pgfpathcurveto{\pgfqpoint{-0.029821in}{0.017319in}}{\pgfqpoint{-0.033333in}{0.008840in}}{\pgfqpoint{-0.033333in}{0.000000in}}%
\pgfpathcurveto{\pgfqpoint{-0.033333in}{-0.008840in}}{\pgfqpoint{-0.029821in}{-0.017319in}}{\pgfqpoint{-0.023570in}{-0.023570in}}%
\pgfpathcurveto{\pgfqpoint{-0.017319in}{-0.029821in}}{\pgfqpoint{-0.008840in}{-0.033333in}}{\pgfqpoint{0.000000in}{-0.033333in}}%
\pgfpathlineto{\pgfqpoint{0.000000in}{-0.033333in}}%
\pgfpathclose%
\pgfusepath{stroke,fill}%
}%
\begin{pgfscope}%
\pgfsys@transformshift{3.760564in}{2.629743in}%
\pgfsys@useobject{currentmarker}{}%
\end{pgfscope}%
\end{pgfscope}%
\begin{pgfscope}%
\pgfpathrectangle{\pgfqpoint{0.765000in}{0.660000in}}{\pgfqpoint{4.620000in}{4.620000in}}%
\pgfusepath{clip}%
\pgfsetrectcap%
\pgfsetroundjoin%
\pgfsetlinewidth{1.204500pt}%
\definecolor{currentstroke}{rgb}{1.000000,0.576471,0.309804}%
\pgfsetstrokecolor{currentstroke}%
\pgfsetdash{}{0pt}%
\pgfpathmoveto{\pgfqpoint{2.739731in}{2.397839in}}%
\pgfusepath{stroke}%
\end{pgfscope}%
\begin{pgfscope}%
\pgfpathrectangle{\pgfqpoint{0.765000in}{0.660000in}}{\pgfqpoint{4.620000in}{4.620000in}}%
\pgfusepath{clip}%
\pgfsetbuttcap%
\pgfsetroundjoin%
\definecolor{currentfill}{rgb}{1.000000,0.576471,0.309804}%
\pgfsetfillcolor{currentfill}%
\pgfsetlinewidth{1.003750pt}%
\definecolor{currentstroke}{rgb}{1.000000,0.576471,0.309804}%
\pgfsetstrokecolor{currentstroke}%
\pgfsetdash{}{0pt}%
\pgfsys@defobject{currentmarker}{\pgfqpoint{-0.033333in}{-0.033333in}}{\pgfqpoint{0.033333in}{0.033333in}}{%
\pgfpathmoveto{\pgfqpoint{0.000000in}{-0.033333in}}%
\pgfpathcurveto{\pgfqpoint{0.008840in}{-0.033333in}}{\pgfqpoint{0.017319in}{-0.029821in}}{\pgfqpoint{0.023570in}{-0.023570in}}%
\pgfpathcurveto{\pgfqpoint{0.029821in}{-0.017319in}}{\pgfqpoint{0.033333in}{-0.008840in}}{\pgfqpoint{0.033333in}{0.000000in}}%
\pgfpathcurveto{\pgfqpoint{0.033333in}{0.008840in}}{\pgfqpoint{0.029821in}{0.017319in}}{\pgfqpoint{0.023570in}{0.023570in}}%
\pgfpathcurveto{\pgfqpoint{0.017319in}{0.029821in}}{\pgfqpoint{0.008840in}{0.033333in}}{\pgfqpoint{0.000000in}{0.033333in}}%
\pgfpathcurveto{\pgfqpoint{-0.008840in}{0.033333in}}{\pgfqpoint{-0.017319in}{0.029821in}}{\pgfqpoint{-0.023570in}{0.023570in}}%
\pgfpathcurveto{\pgfqpoint{-0.029821in}{0.017319in}}{\pgfqpoint{-0.033333in}{0.008840in}}{\pgfqpoint{-0.033333in}{0.000000in}}%
\pgfpathcurveto{\pgfqpoint{-0.033333in}{-0.008840in}}{\pgfqpoint{-0.029821in}{-0.017319in}}{\pgfqpoint{-0.023570in}{-0.023570in}}%
\pgfpathcurveto{\pgfqpoint{-0.017319in}{-0.029821in}}{\pgfqpoint{-0.008840in}{-0.033333in}}{\pgfqpoint{0.000000in}{-0.033333in}}%
\pgfpathlineto{\pgfqpoint{0.000000in}{-0.033333in}}%
\pgfpathclose%
\pgfusepath{stroke,fill}%
}%
\begin{pgfscope}%
\pgfsys@transformshift{2.739731in}{2.397839in}%
\pgfsys@useobject{currentmarker}{}%
\end{pgfscope}%
\end{pgfscope}%
\begin{pgfscope}%
\pgfpathrectangle{\pgfqpoint{0.765000in}{0.660000in}}{\pgfqpoint{4.620000in}{4.620000in}}%
\pgfusepath{clip}%
\pgfsetrectcap%
\pgfsetroundjoin%
\pgfsetlinewidth{1.204500pt}%
\definecolor{currentstroke}{rgb}{1.000000,0.576471,0.309804}%
\pgfsetstrokecolor{currentstroke}%
\pgfsetdash{}{0pt}%
\pgfpathmoveto{\pgfqpoint{3.802665in}{2.650818in}}%
\pgfusepath{stroke}%
\end{pgfscope}%
\begin{pgfscope}%
\pgfpathrectangle{\pgfqpoint{0.765000in}{0.660000in}}{\pgfqpoint{4.620000in}{4.620000in}}%
\pgfusepath{clip}%
\pgfsetbuttcap%
\pgfsetroundjoin%
\definecolor{currentfill}{rgb}{1.000000,0.576471,0.309804}%
\pgfsetfillcolor{currentfill}%
\pgfsetlinewidth{1.003750pt}%
\definecolor{currentstroke}{rgb}{1.000000,0.576471,0.309804}%
\pgfsetstrokecolor{currentstroke}%
\pgfsetdash{}{0pt}%
\pgfsys@defobject{currentmarker}{\pgfqpoint{-0.033333in}{-0.033333in}}{\pgfqpoint{0.033333in}{0.033333in}}{%
\pgfpathmoveto{\pgfqpoint{0.000000in}{-0.033333in}}%
\pgfpathcurveto{\pgfqpoint{0.008840in}{-0.033333in}}{\pgfqpoint{0.017319in}{-0.029821in}}{\pgfqpoint{0.023570in}{-0.023570in}}%
\pgfpathcurveto{\pgfqpoint{0.029821in}{-0.017319in}}{\pgfqpoint{0.033333in}{-0.008840in}}{\pgfqpoint{0.033333in}{0.000000in}}%
\pgfpathcurveto{\pgfqpoint{0.033333in}{0.008840in}}{\pgfqpoint{0.029821in}{0.017319in}}{\pgfqpoint{0.023570in}{0.023570in}}%
\pgfpathcurveto{\pgfqpoint{0.017319in}{0.029821in}}{\pgfqpoint{0.008840in}{0.033333in}}{\pgfqpoint{0.000000in}{0.033333in}}%
\pgfpathcurveto{\pgfqpoint{-0.008840in}{0.033333in}}{\pgfqpoint{-0.017319in}{0.029821in}}{\pgfqpoint{-0.023570in}{0.023570in}}%
\pgfpathcurveto{\pgfqpoint{-0.029821in}{0.017319in}}{\pgfqpoint{-0.033333in}{0.008840in}}{\pgfqpoint{-0.033333in}{0.000000in}}%
\pgfpathcurveto{\pgfqpoint{-0.033333in}{-0.008840in}}{\pgfqpoint{-0.029821in}{-0.017319in}}{\pgfqpoint{-0.023570in}{-0.023570in}}%
\pgfpathcurveto{\pgfqpoint{-0.017319in}{-0.029821in}}{\pgfqpoint{-0.008840in}{-0.033333in}}{\pgfqpoint{0.000000in}{-0.033333in}}%
\pgfpathlineto{\pgfqpoint{0.000000in}{-0.033333in}}%
\pgfpathclose%
\pgfusepath{stroke,fill}%
}%
\begin{pgfscope}%
\pgfsys@transformshift{3.802665in}{2.650818in}%
\pgfsys@useobject{currentmarker}{}%
\end{pgfscope}%
\end{pgfscope}%
\begin{pgfscope}%
\pgfpathrectangle{\pgfqpoint{0.765000in}{0.660000in}}{\pgfqpoint{4.620000in}{4.620000in}}%
\pgfusepath{clip}%
\pgfsetrectcap%
\pgfsetroundjoin%
\pgfsetlinewidth{1.204500pt}%
\definecolor{currentstroke}{rgb}{1.000000,0.576471,0.309804}%
\pgfsetstrokecolor{currentstroke}%
\pgfsetdash{}{0pt}%
\pgfpathmoveto{\pgfqpoint{3.342004in}{2.244011in}}%
\pgfusepath{stroke}%
\end{pgfscope}%
\begin{pgfscope}%
\pgfpathrectangle{\pgfqpoint{0.765000in}{0.660000in}}{\pgfqpoint{4.620000in}{4.620000in}}%
\pgfusepath{clip}%
\pgfsetbuttcap%
\pgfsetroundjoin%
\definecolor{currentfill}{rgb}{1.000000,0.576471,0.309804}%
\pgfsetfillcolor{currentfill}%
\pgfsetlinewidth{1.003750pt}%
\definecolor{currentstroke}{rgb}{1.000000,0.576471,0.309804}%
\pgfsetstrokecolor{currentstroke}%
\pgfsetdash{}{0pt}%
\pgfsys@defobject{currentmarker}{\pgfqpoint{-0.033333in}{-0.033333in}}{\pgfqpoint{0.033333in}{0.033333in}}{%
\pgfpathmoveto{\pgfqpoint{0.000000in}{-0.033333in}}%
\pgfpathcurveto{\pgfqpoint{0.008840in}{-0.033333in}}{\pgfqpoint{0.017319in}{-0.029821in}}{\pgfqpoint{0.023570in}{-0.023570in}}%
\pgfpathcurveto{\pgfqpoint{0.029821in}{-0.017319in}}{\pgfqpoint{0.033333in}{-0.008840in}}{\pgfqpoint{0.033333in}{0.000000in}}%
\pgfpathcurveto{\pgfqpoint{0.033333in}{0.008840in}}{\pgfqpoint{0.029821in}{0.017319in}}{\pgfqpoint{0.023570in}{0.023570in}}%
\pgfpathcurveto{\pgfqpoint{0.017319in}{0.029821in}}{\pgfqpoint{0.008840in}{0.033333in}}{\pgfqpoint{0.000000in}{0.033333in}}%
\pgfpathcurveto{\pgfqpoint{-0.008840in}{0.033333in}}{\pgfqpoint{-0.017319in}{0.029821in}}{\pgfqpoint{-0.023570in}{0.023570in}}%
\pgfpathcurveto{\pgfqpoint{-0.029821in}{0.017319in}}{\pgfqpoint{-0.033333in}{0.008840in}}{\pgfqpoint{-0.033333in}{0.000000in}}%
\pgfpathcurveto{\pgfqpoint{-0.033333in}{-0.008840in}}{\pgfqpoint{-0.029821in}{-0.017319in}}{\pgfqpoint{-0.023570in}{-0.023570in}}%
\pgfpathcurveto{\pgfqpoint{-0.017319in}{-0.029821in}}{\pgfqpoint{-0.008840in}{-0.033333in}}{\pgfqpoint{0.000000in}{-0.033333in}}%
\pgfpathlineto{\pgfqpoint{0.000000in}{-0.033333in}}%
\pgfpathclose%
\pgfusepath{stroke,fill}%
}%
\begin{pgfscope}%
\pgfsys@transformshift{3.342004in}{2.244011in}%
\pgfsys@useobject{currentmarker}{}%
\end{pgfscope}%
\end{pgfscope}%
\begin{pgfscope}%
\pgfpathrectangle{\pgfqpoint{0.765000in}{0.660000in}}{\pgfqpoint{4.620000in}{4.620000in}}%
\pgfusepath{clip}%
\pgfsetrectcap%
\pgfsetroundjoin%
\pgfsetlinewidth{1.204500pt}%
\definecolor{currentstroke}{rgb}{1.000000,0.576471,0.309804}%
\pgfsetstrokecolor{currentstroke}%
\pgfsetdash{}{0pt}%
\pgfpathmoveto{\pgfqpoint{2.762616in}{1.963243in}}%
\pgfusepath{stroke}%
\end{pgfscope}%
\begin{pgfscope}%
\pgfpathrectangle{\pgfqpoint{0.765000in}{0.660000in}}{\pgfqpoint{4.620000in}{4.620000in}}%
\pgfusepath{clip}%
\pgfsetbuttcap%
\pgfsetroundjoin%
\definecolor{currentfill}{rgb}{1.000000,0.576471,0.309804}%
\pgfsetfillcolor{currentfill}%
\pgfsetlinewidth{1.003750pt}%
\definecolor{currentstroke}{rgb}{1.000000,0.576471,0.309804}%
\pgfsetstrokecolor{currentstroke}%
\pgfsetdash{}{0pt}%
\pgfsys@defobject{currentmarker}{\pgfqpoint{-0.033333in}{-0.033333in}}{\pgfqpoint{0.033333in}{0.033333in}}{%
\pgfpathmoveto{\pgfqpoint{0.000000in}{-0.033333in}}%
\pgfpathcurveto{\pgfqpoint{0.008840in}{-0.033333in}}{\pgfqpoint{0.017319in}{-0.029821in}}{\pgfqpoint{0.023570in}{-0.023570in}}%
\pgfpathcurveto{\pgfqpoint{0.029821in}{-0.017319in}}{\pgfqpoint{0.033333in}{-0.008840in}}{\pgfqpoint{0.033333in}{0.000000in}}%
\pgfpathcurveto{\pgfqpoint{0.033333in}{0.008840in}}{\pgfqpoint{0.029821in}{0.017319in}}{\pgfqpoint{0.023570in}{0.023570in}}%
\pgfpathcurveto{\pgfqpoint{0.017319in}{0.029821in}}{\pgfqpoint{0.008840in}{0.033333in}}{\pgfqpoint{0.000000in}{0.033333in}}%
\pgfpathcurveto{\pgfqpoint{-0.008840in}{0.033333in}}{\pgfqpoint{-0.017319in}{0.029821in}}{\pgfqpoint{-0.023570in}{0.023570in}}%
\pgfpathcurveto{\pgfqpoint{-0.029821in}{0.017319in}}{\pgfqpoint{-0.033333in}{0.008840in}}{\pgfqpoint{-0.033333in}{0.000000in}}%
\pgfpathcurveto{\pgfqpoint{-0.033333in}{-0.008840in}}{\pgfqpoint{-0.029821in}{-0.017319in}}{\pgfqpoint{-0.023570in}{-0.023570in}}%
\pgfpathcurveto{\pgfqpoint{-0.017319in}{-0.029821in}}{\pgfqpoint{-0.008840in}{-0.033333in}}{\pgfqpoint{0.000000in}{-0.033333in}}%
\pgfpathlineto{\pgfqpoint{0.000000in}{-0.033333in}}%
\pgfpathclose%
\pgfusepath{stroke,fill}%
}%
\begin{pgfscope}%
\pgfsys@transformshift{2.762616in}{1.963243in}%
\pgfsys@useobject{currentmarker}{}%
\end{pgfscope}%
\end{pgfscope}%
\begin{pgfscope}%
\pgfpathrectangle{\pgfqpoint{0.765000in}{0.660000in}}{\pgfqpoint{4.620000in}{4.620000in}}%
\pgfusepath{clip}%
\pgfsetrectcap%
\pgfsetroundjoin%
\pgfsetlinewidth{1.204500pt}%
\definecolor{currentstroke}{rgb}{0.176471,0.192157,0.258824}%
\pgfsetstrokecolor{currentstroke}%
\pgfsetdash{}{0pt}%
\pgfpathmoveto{\pgfqpoint{2.434182in}{3.270874in}}%
\pgfusepath{stroke}%
\end{pgfscope}%
\begin{pgfscope}%
\pgfpathrectangle{\pgfqpoint{0.765000in}{0.660000in}}{\pgfqpoint{4.620000in}{4.620000in}}%
\pgfusepath{clip}%
\pgfsetbuttcap%
\pgfsetmiterjoin%
\definecolor{currentfill}{rgb}{0.176471,0.192157,0.258824}%
\pgfsetfillcolor{currentfill}%
\pgfsetlinewidth{1.003750pt}%
\definecolor{currentstroke}{rgb}{0.176471,0.192157,0.258824}%
\pgfsetstrokecolor{currentstroke}%
\pgfsetdash{}{0pt}%
\pgfsys@defobject{currentmarker}{\pgfqpoint{-0.033333in}{-0.033333in}}{\pgfqpoint{0.033333in}{0.033333in}}{%
\pgfpathmoveto{\pgfqpoint{-0.000000in}{-0.033333in}}%
\pgfpathlineto{\pgfqpoint{0.033333in}{0.033333in}}%
\pgfpathlineto{\pgfqpoint{-0.033333in}{0.033333in}}%
\pgfpathlineto{\pgfqpoint{-0.000000in}{-0.033333in}}%
\pgfpathclose%
\pgfusepath{stroke,fill}%
}%
\begin{pgfscope}%
\pgfsys@transformshift{2.434182in}{3.270874in}%
\pgfsys@useobject{currentmarker}{}%
\end{pgfscope}%
\end{pgfscope}%
\begin{pgfscope}%
\pgfpathrectangle{\pgfqpoint{0.765000in}{0.660000in}}{\pgfqpoint{4.620000in}{4.620000in}}%
\pgfusepath{clip}%
\pgfsetrectcap%
\pgfsetroundjoin%
\pgfsetlinewidth{1.204500pt}%
\definecolor{currentstroke}{rgb}{0.176471,0.192157,0.258824}%
\pgfsetstrokecolor{currentstroke}%
\pgfsetdash{}{0pt}%
\pgfpathmoveto{\pgfqpoint{3.563811in}{3.566849in}}%
\pgfusepath{stroke}%
\end{pgfscope}%
\begin{pgfscope}%
\pgfpathrectangle{\pgfqpoint{0.765000in}{0.660000in}}{\pgfqpoint{4.620000in}{4.620000in}}%
\pgfusepath{clip}%
\pgfsetbuttcap%
\pgfsetmiterjoin%
\definecolor{currentfill}{rgb}{0.176471,0.192157,0.258824}%
\pgfsetfillcolor{currentfill}%
\pgfsetlinewidth{1.003750pt}%
\definecolor{currentstroke}{rgb}{0.176471,0.192157,0.258824}%
\pgfsetstrokecolor{currentstroke}%
\pgfsetdash{}{0pt}%
\pgfsys@defobject{currentmarker}{\pgfqpoint{-0.033333in}{-0.033333in}}{\pgfqpoint{0.033333in}{0.033333in}}{%
\pgfpathmoveto{\pgfqpoint{-0.000000in}{-0.033333in}}%
\pgfpathlineto{\pgfqpoint{0.033333in}{0.033333in}}%
\pgfpathlineto{\pgfqpoint{-0.033333in}{0.033333in}}%
\pgfpathlineto{\pgfqpoint{-0.000000in}{-0.033333in}}%
\pgfpathclose%
\pgfusepath{stroke,fill}%
}%
\begin{pgfscope}%
\pgfsys@transformshift{3.563811in}{3.566849in}%
\pgfsys@useobject{currentmarker}{}%
\end{pgfscope}%
\end{pgfscope}%
\begin{pgfscope}%
\pgfpathrectangle{\pgfqpoint{0.765000in}{0.660000in}}{\pgfqpoint{4.620000in}{4.620000in}}%
\pgfusepath{clip}%
\pgfsetrectcap%
\pgfsetroundjoin%
\pgfsetlinewidth{1.204500pt}%
\definecolor{currentstroke}{rgb}{0.176471,0.192157,0.258824}%
\pgfsetstrokecolor{currentstroke}%
\pgfsetdash{}{0pt}%
\pgfpathmoveto{\pgfqpoint{2.163756in}{3.811061in}}%
\pgfusepath{stroke}%
\end{pgfscope}%
\begin{pgfscope}%
\pgfpathrectangle{\pgfqpoint{0.765000in}{0.660000in}}{\pgfqpoint{4.620000in}{4.620000in}}%
\pgfusepath{clip}%
\pgfsetbuttcap%
\pgfsetmiterjoin%
\definecolor{currentfill}{rgb}{0.176471,0.192157,0.258824}%
\pgfsetfillcolor{currentfill}%
\pgfsetlinewidth{1.003750pt}%
\definecolor{currentstroke}{rgb}{0.176471,0.192157,0.258824}%
\pgfsetstrokecolor{currentstroke}%
\pgfsetdash{}{0pt}%
\pgfsys@defobject{currentmarker}{\pgfqpoint{-0.033333in}{-0.033333in}}{\pgfqpoint{0.033333in}{0.033333in}}{%
\pgfpathmoveto{\pgfqpoint{-0.000000in}{-0.033333in}}%
\pgfpathlineto{\pgfqpoint{0.033333in}{0.033333in}}%
\pgfpathlineto{\pgfqpoint{-0.033333in}{0.033333in}}%
\pgfpathlineto{\pgfqpoint{-0.000000in}{-0.033333in}}%
\pgfpathclose%
\pgfusepath{stroke,fill}%
}%
\begin{pgfscope}%
\pgfsys@transformshift{2.163756in}{3.811061in}%
\pgfsys@useobject{currentmarker}{}%
\end{pgfscope}%
\end{pgfscope}%
\begin{pgfscope}%
\pgfpathrectangle{\pgfqpoint{0.765000in}{0.660000in}}{\pgfqpoint{4.620000in}{4.620000in}}%
\pgfusepath{clip}%
\pgfsetrectcap%
\pgfsetroundjoin%
\pgfsetlinewidth{1.204500pt}%
\definecolor{currentstroke}{rgb}{0.176471,0.192157,0.258824}%
\pgfsetstrokecolor{currentstroke}%
\pgfsetdash{}{0pt}%
\pgfpathmoveto{\pgfqpoint{2.117922in}{3.544712in}}%
\pgfusepath{stroke}%
\end{pgfscope}%
\begin{pgfscope}%
\pgfpathrectangle{\pgfqpoint{0.765000in}{0.660000in}}{\pgfqpoint{4.620000in}{4.620000in}}%
\pgfusepath{clip}%
\pgfsetbuttcap%
\pgfsetmiterjoin%
\definecolor{currentfill}{rgb}{0.176471,0.192157,0.258824}%
\pgfsetfillcolor{currentfill}%
\pgfsetlinewidth{1.003750pt}%
\definecolor{currentstroke}{rgb}{0.176471,0.192157,0.258824}%
\pgfsetstrokecolor{currentstroke}%
\pgfsetdash{}{0pt}%
\pgfsys@defobject{currentmarker}{\pgfqpoint{-0.033333in}{-0.033333in}}{\pgfqpoint{0.033333in}{0.033333in}}{%
\pgfpathmoveto{\pgfqpoint{-0.000000in}{-0.033333in}}%
\pgfpathlineto{\pgfqpoint{0.033333in}{0.033333in}}%
\pgfpathlineto{\pgfqpoint{-0.033333in}{0.033333in}}%
\pgfpathlineto{\pgfqpoint{-0.000000in}{-0.033333in}}%
\pgfpathclose%
\pgfusepath{stroke,fill}%
}%
\begin{pgfscope}%
\pgfsys@transformshift{2.117922in}{3.544712in}%
\pgfsys@useobject{currentmarker}{}%
\end{pgfscope}%
\end{pgfscope}%
\begin{pgfscope}%
\pgfpathrectangle{\pgfqpoint{0.765000in}{0.660000in}}{\pgfqpoint{4.620000in}{4.620000in}}%
\pgfusepath{clip}%
\pgfsetrectcap%
\pgfsetroundjoin%
\pgfsetlinewidth{1.204500pt}%
\definecolor{currentstroke}{rgb}{0.176471,0.192157,0.258824}%
\pgfsetstrokecolor{currentstroke}%
\pgfsetdash{}{0pt}%
\pgfpathmoveto{\pgfqpoint{2.907146in}{3.670512in}}%
\pgfusepath{stroke}%
\end{pgfscope}%
\begin{pgfscope}%
\pgfpathrectangle{\pgfqpoint{0.765000in}{0.660000in}}{\pgfqpoint{4.620000in}{4.620000in}}%
\pgfusepath{clip}%
\pgfsetbuttcap%
\pgfsetmiterjoin%
\definecolor{currentfill}{rgb}{0.176471,0.192157,0.258824}%
\pgfsetfillcolor{currentfill}%
\pgfsetlinewidth{1.003750pt}%
\definecolor{currentstroke}{rgb}{0.176471,0.192157,0.258824}%
\pgfsetstrokecolor{currentstroke}%
\pgfsetdash{}{0pt}%
\pgfsys@defobject{currentmarker}{\pgfqpoint{-0.033333in}{-0.033333in}}{\pgfqpoint{0.033333in}{0.033333in}}{%
\pgfpathmoveto{\pgfqpoint{-0.000000in}{-0.033333in}}%
\pgfpathlineto{\pgfqpoint{0.033333in}{0.033333in}}%
\pgfpathlineto{\pgfqpoint{-0.033333in}{0.033333in}}%
\pgfpathlineto{\pgfqpoint{-0.000000in}{-0.033333in}}%
\pgfpathclose%
\pgfusepath{stroke,fill}%
}%
\begin{pgfscope}%
\pgfsys@transformshift{2.907146in}{3.670512in}%
\pgfsys@useobject{currentmarker}{}%
\end{pgfscope}%
\end{pgfscope}%
\begin{pgfscope}%
\pgfpathrectangle{\pgfqpoint{0.765000in}{0.660000in}}{\pgfqpoint{4.620000in}{4.620000in}}%
\pgfusepath{clip}%
\pgfsetrectcap%
\pgfsetroundjoin%
\pgfsetlinewidth{1.204500pt}%
\definecolor{currentstroke}{rgb}{0.176471,0.192157,0.258824}%
\pgfsetstrokecolor{currentstroke}%
\pgfsetdash{}{0pt}%
\pgfpathmoveto{\pgfqpoint{2.283533in}{3.683203in}}%
\pgfusepath{stroke}%
\end{pgfscope}%
\begin{pgfscope}%
\pgfpathrectangle{\pgfqpoint{0.765000in}{0.660000in}}{\pgfqpoint{4.620000in}{4.620000in}}%
\pgfusepath{clip}%
\pgfsetbuttcap%
\pgfsetmiterjoin%
\definecolor{currentfill}{rgb}{0.176471,0.192157,0.258824}%
\pgfsetfillcolor{currentfill}%
\pgfsetlinewidth{1.003750pt}%
\definecolor{currentstroke}{rgb}{0.176471,0.192157,0.258824}%
\pgfsetstrokecolor{currentstroke}%
\pgfsetdash{}{0pt}%
\pgfsys@defobject{currentmarker}{\pgfqpoint{-0.033333in}{-0.033333in}}{\pgfqpoint{0.033333in}{0.033333in}}{%
\pgfpathmoveto{\pgfqpoint{-0.000000in}{-0.033333in}}%
\pgfpathlineto{\pgfqpoint{0.033333in}{0.033333in}}%
\pgfpathlineto{\pgfqpoint{-0.033333in}{0.033333in}}%
\pgfpathlineto{\pgfqpoint{-0.000000in}{-0.033333in}}%
\pgfpathclose%
\pgfusepath{stroke,fill}%
}%
\begin{pgfscope}%
\pgfsys@transformshift{2.283533in}{3.683203in}%
\pgfsys@useobject{currentmarker}{}%
\end{pgfscope}%
\end{pgfscope}%
\begin{pgfscope}%
\pgfpathrectangle{\pgfqpoint{0.765000in}{0.660000in}}{\pgfqpoint{4.620000in}{4.620000in}}%
\pgfusepath{clip}%
\pgfsetrectcap%
\pgfsetroundjoin%
\pgfsetlinewidth{1.204500pt}%
\definecolor{currentstroke}{rgb}{0.176471,0.192157,0.258824}%
\pgfsetstrokecolor{currentstroke}%
\pgfsetdash{}{0pt}%
\pgfpathmoveto{\pgfqpoint{1.959223in}{3.410826in}}%
\pgfusepath{stroke}%
\end{pgfscope}%
\begin{pgfscope}%
\pgfpathrectangle{\pgfqpoint{0.765000in}{0.660000in}}{\pgfqpoint{4.620000in}{4.620000in}}%
\pgfusepath{clip}%
\pgfsetbuttcap%
\pgfsetmiterjoin%
\definecolor{currentfill}{rgb}{0.176471,0.192157,0.258824}%
\pgfsetfillcolor{currentfill}%
\pgfsetlinewidth{1.003750pt}%
\definecolor{currentstroke}{rgb}{0.176471,0.192157,0.258824}%
\pgfsetstrokecolor{currentstroke}%
\pgfsetdash{}{0pt}%
\pgfsys@defobject{currentmarker}{\pgfqpoint{-0.033333in}{-0.033333in}}{\pgfqpoint{0.033333in}{0.033333in}}{%
\pgfpathmoveto{\pgfqpoint{-0.000000in}{-0.033333in}}%
\pgfpathlineto{\pgfqpoint{0.033333in}{0.033333in}}%
\pgfpathlineto{\pgfqpoint{-0.033333in}{0.033333in}}%
\pgfpathlineto{\pgfqpoint{-0.000000in}{-0.033333in}}%
\pgfpathclose%
\pgfusepath{stroke,fill}%
}%
\begin{pgfscope}%
\pgfsys@transformshift{1.959223in}{3.410826in}%
\pgfsys@useobject{currentmarker}{}%
\end{pgfscope}%
\end{pgfscope}%
\begin{pgfscope}%
\pgfpathrectangle{\pgfqpoint{0.765000in}{0.660000in}}{\pgfqpoint{4.620000in}{4.620000in}}%
\pgfusepath{clip}%
\pgfsetrectcap%
\pgfsetroundjoin%
\pgfsetlinewidth{1.204500pt}%
\definecolor{currentstroke}{rgb}{0.176471,0.192157,0.258824}%
\pgfsetstrokecolor{currentstroke}%
\pgfsetdash{}{0pt}%
\pgfpathmoveto{\pgfqpoint{3.667778in}{3.551523in}}%
\pgfusepath{stroke}%
\end{pgfscope}%
\begin{pgfscope}%
\pgfpathrectangle{\pgfqpoint{0.765000in}{0.660000in}}{\pgfqpoint{4.620000in}{4.620000in}}%
\pgfusepath{clip}%
\pgfsetbuttcap%
\pgfsetmiterjoin%
\definecolor{currentfill}{rgb}{0.176471,0.192157,0.258824}%
\pgfsetfillcolor{currentfill}%
\pgfsetlinewidth{1.003750pt}%
\definecolor{currentstroke}{rgb}{0.176471,0.192157,0.258824}%
\pgfsetstrokecolor{currentstroke}%
\pgfsetdash{}{0pt}%
\pgfsys@defobject{currentmarker}{\pgfqpoint{-0.033333in}{-0.033333in}}{\pgfqpoint{0.033333in}{0.033333in}}{%
\pgfpathmoveto{\pgfqpoint{-0.000000in}{-0.033333in}}%
\pgfpathlineto{\pgfqpoint{0.033333in}{0.033333in}}%
\pgfpathlineto{\pgfqpoint{-0.033333in}{0.033333in}}%
\pgfpathlineto{\pgfqpoint{-0.000000in}{-0.033333in}}%
\pgfpathclose%
\pgfusepath{stroke,fill}%
}%
\begin{pgfscope}%
\pgfsys@transformshift{3.667778in}{3.551523in}%
\pgfsys@useobject{currentmarker}{}%
\end{pgfscope}%
\end{pgfscope}%
\begin{pgfscope}%
\pgfpathrectangle{\pgfqpoint{0.765000in}{0.660000in}}{\pgfqpoint{4.620000in}{4.620000in}}%
\pgfusepath{clip}%
\pgfsetrectcap%
\pgfsetroundjoin%
\pgfsetlinewidth{1.204500pt}%
\definecolor{currentstroke}{rgb}{0.176471,0.192157,0.258824}%
\pgfsetstrokecolor{currentstroke}%
\pgfsetdash{}{0pt}%
\pgfpathmoveto{\pgfqpoint{3.374278in}{2.978413in}}%
\pgfusepath{stroke}%
\end{pgfscope}%
\begin{pgfscope}%
\pgfpathrectangle{\pgfqpoint{0.765000in}{0.660000in}}{\pgfqpoint{4.620000in}{4.620000in}}%
\pgfusepath{clip}%
\pgfsetbuttcap%
\pgfsetmiterjoin%
\definecolor{currentfill}{rgb}{0.176471,0.192157,0.258824}%
\pgfsetfillcolor{currentfill}%
\pgfsetlinewidth{1.003750pt}%
\definecolor{currentstroke}{rgb}{0.176471,0.192157,0.258824}%
\pgfsetstrokecolor{currentstroke}%
\pgfsetdash{}{0pt}%
\pgfsys@defobject{currentmarker}{\pgfqpoint{-0.033333in}{-0.033333in}}{\pgfqpoint{0.033333in}{0.033333in}}{%
\pgfpathmoveto{\pgfqpoint{-0.000000in}{-0.033333in}}%
\pgfpathlineto{\pgfqpoint{0.033333in}{0.033333in}}%
\pgfpathlineto{\pgfqpoint{-0.033333in}{0.033333in}}%
\pgfpathlineto{\pgfqpoint{-0.000000in}{-0.033333in}}%
\pgfpathclose%
\pgfusepath{stroke,fill}%
}%
\begin{pgfscope}%
\pgfsys@transformshift{3.374278in}{2.978413in}%
\pgfsys@useobject{currentmarker}{}%
\end{pgfscope}%
\end{pgfscope}%
\begin{pgfscope}%
\pgfpathrectangle{\pgfqpoint{0.765000in}{0.660000in}}{\pgfqpoint{4.620000in}{4.620000in}}%
\pgfusepath{clip}%
\pgfsetrectcap%
\pgfsetroundjoin%
\pgfsetlinewidth{1.204500pt}%
\definecolor{currentstroke}{rgb}{0.176471,0.192157,0.258824}%
\pgfsetstrokecolor{currentstroke}%
\pgfsetdash{}{0pt}%
\pgfpathmoveto{\pgfqpoint{3.369507in}{3.571326in}}%
\pgfusepath{stroke}%
\end{pgfscope}%
\begin{pgfscope}%
\pgfpathrectangle{\pgfqpoint{0.765000in}{0.660000in}}{\pgfqpoint{4.620000in}{4.620000in}}%
\pgfusepath{clip}%
\pgfsetbuttcap%
\pgfsetmiterjoin%
\definecolor{currentfill}{rgb}{0.176471,0.192157,0.258824}%
\pgfsetfillcolor{currentfill}%
\pgfsetlinewidth{1.003750pt}%
\definecolor{currentstroke}{rgb}{0.176471,0.192157,0.258824}%
\pgfsetstrokecolor{currentstroke}%
\pgfsetdash{}{0pt}%
\pgfsys@defobject{currentmarker}{\pgfqpoint{-0.033333in}{-0.033333in}}{\pgfqpoint{0.033333in}{0.033333in}}{%
\pgfpathmoveto{\pgfqpoint{-0.000000in}{-0.033333in}}%
\pgfpathlineto{\pgfqpoint{0.033333in}{0.033333in}}%
\pgfpathlineto{\pgfqpoint{-0.033333in}{0.033333in}}%
\pgfpathlineto{\pgfqpoint{-0.000000in}{-0.033333in}}%
\pgfpathclose%
\pgfusepath{stroke,fill}%
}%
\begin{pgfscope}%
\pgfsys@transformshift{3.369507in}{3.571326in}%
\pgfsys@useobject{currentmarker}{}%
\end{pgfscope}%
\end{pgfscope}%
\begin{pgfscope}%
\pgfpathrectangle{\pgfqpoint{0.765000in}{0.660000in}}{\pgfqpoint{4.620000in}{4.620000in}}%
\pgfusepath{clip}%
\pgfsetrectcap%
\pgfsetroundjoin%
\pgfsetlinewidth{1.204500pt}%
\definecolor{currentstroke}{rgb}{0.176471,0.192157,0.258824}%
\pgfsetstrokecolor{currentstroke}%
\pgfsetdash{}{0pt}%
\pgfpathmoveto{\pgfqpoint{2.249674in}{3.504126in}}%
\pgfusepath{stroke}%
\end{pgfscope}%
\begin{pgfscope}%
\pgfpathrectangle{\pgfqpoint{0.765000in}{0.660000in}}{\pgfqpoint{4.620000in}{4.620000in}}%
\pgfusepath{clip}%
\pgfsetbuttcap%
\pgfsetmiterjoin%
\definecolor{currentfill}{rgb}{0.176471,0.192157,0.258824}%
\pgfsetfillcolor{currentfill}%
\pgfsetlinewidth{1.003750pt}%
\definecolor{currentstroke}{rgb}{0.176471,0.192157,0.258824}%
\pgfsetstrokecolor{currentstroke}%
\pgfsetdash{}{0pt}%
\pgfsys@defobject{currentmarker}{\pgfqpoint{-0.033333in}{-0.033333in}}{\pgfqpoint{0.033333in}{0.033333in}}{%
\pgfpathmoveto{\pgfqpoint{-0.000000in}{-0.033333in}}%
\pgfpathlineto{\pgfqpoint{0.033333in}{0.033333in}}%
\pgfpathlineto{\pgfqpoint{-0.033333in}{0.033333in}}%
\pgfpathlineto{\pgfqpoint{-0.000000in}{-0.033333in}}%
\pgfpathclose%
\pgfusepath{stroke,fill}%
}%
\begin{pgfscope}%
\pgfsys@transformshift{2.249674in}{3.504126in}%
\pgfsys@useobject{currentmarker}{}%
\end{pgfscope}%
\end{pgfscope}%
\begin{pgfscope}%
\pgfpathrectangle{\pgfqpoint{0.765000in}{0.660000in}}{\pgfqpoint{4.620000in}{4.620000in}}%
\pgfusepath{clip}%
\pgfsetrectcap%
\pgfsetroundjoin%
\pgfsetlinewidth{1.204500pt}%
\definecolor{currentstroke}{rgb}{0.176471,0.192157,0.258824}%
\pgfsetstrokecolor{currentstroke}%
\pgfsetdash{}{0pt}%
\pgfpathmoveto{\pgfqpoint{2.906105in}{3.638633in}}%
\pgfusepath{stroke}%
\end{pgfscope}%
\begin{pgfscope}%
\pgfpathrectangle{\pgfqpoint{0.765000in}{0.660000in}}{\pgfqpoint{4.620000in}{4.620000in}}%
\pgfusepath{clip}%
\pgfsetbuttcap%
\pgfsetmiterjoin%
\definecolor{currentfill}{rgb}{0.176471,0.192157,0.258824}%
\pgfsetfillcolor{currentfill}%
\pgfsetlinewidth{1.003750pt}%
\definecolor{currentstroke}{rgb}{0.176471,0.192157,0.258824}%
\pgfsetstrokecolor{currentstroke}%
\pgfsetdash{}{0pt}%
\pgfsys@defobject{currentmarker}{\pgfqpoint{-0.033333in}{-0.033333in}}{\pgfqpoint{0.033333in}{0.033333in}}{%
\pgfpathmoveto{\pgfqpoint{-0.000000in}{-0.033333in}}%
\pgfpathlineto{\pgfqpoint{0.033333in}{0.033333in}}%
\pgfpathlineto{\pgfqpoint{-0.033333in}{0.033333in}}%
\pgfpathlineto{\pgfqpoint{-0.000000in}{-0.033333in}}%
\pgfpathclose%
\pgfusepath{stroke,fill}%
}%
\begin{pgfscope}%
\pgfsys@transformshift{2.906105in}{3.638633in}%
\pgfsys@useobject{currentmarker}{}%
\end{pgfscope}%
\end{pgfscope}%
\begin{pgfscope}%
\pgfpathrectangle{\pgfqpoint{0.765000in}{0.660000in}}{\pgfqpoint{4.620000in}{4.620000in}}%
\pgfusepath{clip}%
\pgfsetrectcap%
\pgfsetroundjoin%
\pgfsetlinewidth{1.204500pt}%
\definecolor{currentstroke}{rgb}{0.176471,0.192157,0.258824}%
\pgfsetstrokecolor{currentstroke}%
\pgfsetdash{}{0pt}%
\pgfpathmoveto{\pgfqpoint{1.186816in}{2.866372in}}%
\pgfusepath{stroke}%
\end{pgfscope}%
\begin{pgfscope}%
\pgfpathrectangle{\pgfqpoint{0.765000in}{0.660000in}}{\pgfqpoint{4.620000in}{4.620000in}}%
\pgfusepath{clip}%
\pgfsetbuttcap%
\pgfsetmiterjoin%
\definecolor{currentfill}{rgb}{0.176471,0.192157,0.258824}%
\pgfsetfillcolor{currentfill}%
\pgfsetlinewidth{1.003750pt}%
\definecolor{currentstroke}{rgb}{0.176471,0.192157,0.258824}%
\pgfsetstrokecolor{currentstroke}%
\pgfsetdash{}{0pt}%
\pgfsys@defobject{currentmarker}{\pgfqpoint{-0.033333in}{-0.033333in}}{\pgfqpoint{0.033333in}{0.033333in}}{%
\pgfpathmoveto{\pgfqpoint{-0.000000in}{-0.033333in}}%
\pgfpathlineto{\pgfqpoint{0.033333in}{0.033333in}}%
\pgfpathlineto{\pgfqpoint{-0.033333in}{0.033333in}}%
\pgfpathlineto{\pgfqpoint{-0.000000in}{-0.033333in}}%
\pgfpathclose%
\pgfusepath{stroke,fill}%
}%
\begin{pgfscope}%
\pgfsys@transformshift{1.186816in}{2.866372in}%
\pgfsys@useobject{currentmarker}{}%
\end{pgfscope}%
\end{pgfscope}%
\begin{pgfscope}%
\pgfpathrectangle{\pgfqpoint{0.765000in}{0.660000in}}{\pgfqpoint{4.620000in}{4.620000in}}%
\pgfusepath{clip}%
\pgfsetrectcap%
\pgfsetroundjoin%
\pgfsetlinewidth{1.204500pt}%
\definecolor{currentstroke}{rgb}{0.176471,0.192157,0.258824}%
\pgfsetstrokecolor{currentstroke}%
\pgfsetdash{}{0pt}%
\pgfpathmoveto{\pgfqpoint{2.126953in}{3.646121in}}%
\pgfusepath{stroke}%
\end{pgfscope}%
\begin{pgfscope}%
\pgfpathrectangle{\pgfqpoint{0.765000in}{0.660000in}}{\pgfqpoint{4.620000in}{4.620000in}}%
\pgfusepath{clip}%
\pgfsetbuttcap%
\pgfsetmiterjoin%
\definecolor{currentfill}{rgb}{0.176471,0.192157,0.258824}%
\pgfsetfillcolor{currentfill}%
\pgfsetlinewidth{1.003750pt}%
\definecolor{currentstroke}{rgb}{0.176471,0.192157,0.258824}%
\pgfsetstrokecolor{currentstroke}%
\pgfsetdash{}{0pt}%
\pgfsys@defobject{currentmarker}{\pgfqpoint{-0.033333in}{-0.033333in}}{\pgfqpoint{0.033333in}{0.033333in}}{%
\pgfpathmoveto{\pgfqpoint{-0.000000in}{-0.033333in}}%
\pgfpathlineto{\pgfqpoint{0.033333in}{0.033333in}}%
\pgfpathlineto{\pgfqpoint{-0.033333in}{0.033333in}}%
\pgfpathlineto{\pgfqpoint{-0.000000in}{-0.033333in}}%
\pgfpathclose%
\pgfusepath{stroke,fill}%
}%
\begin{pgfscope}%
\pgfsys@transformshift{2.126953in}{3.646121in}%
\pgfsys@useobject{currentmarker}{}%
\end{pgfscope}%
\end{pgfscope}%
\begin{pgfscope}%
\pgfpathrectangle{\pgfqpoint{0.765000in}{0.660000in}}{\pgfqpoint{4.620000in}{4.620000in}}%
\pgfusepath{clip}%
\pgfsetrectcap%
\pgfsetroundjoin%
\pgfsetlinewidth{1.204500pt}%
\definecolor{currentstroke}{rgb}{0.176471,0.192157,0.258824}%
\pgfsetstrokecolor{currentstroke}%
\pgfsetdash{}{0pt}%
\pgfpathmoveto{\pgfqpoint{2.014934in}{3.342249in}}%
\pgfusepath{stroke}%
\end{pgfscope}%
\begin{pgfscope}%
\pgfpathrectangle{\pgfqpoint{0.765000in}{0.660000in}}{\pgfqpoint{4.620000in}{4.620000in}}%
\pgfusepath{clip}%
\pgfsetbuttcap%
\pgfsetmiterjoin%
\definecolor{currentfill}{rgb}{0.176471,0.192157,0.258824}%
\pgfsetfillcolor{currentfill}%
\pgfsetlinewidth{1.003750pt}%
\definecolor{currentstroke}{rgb}{0.176471,0.192157,0.258824}%
\pgfsetstrokecolor{currentstroke}%
\pgfsetdash{}{0pt}%
\pgfsys@defobject{currentmarker}{\pgfqpoint{-0.033333in}{-0.033333in}}{\pgfqpoint{0.033333in}{0.033333in}}{%
\pgfpathmoveto{\pgfqpoint{-0.000000in}{-0.033333in}}%
\pgfpathlineto{\pgfqpoint{0.033333in}{0.033333in}}%
\pgfpathlineto{\pgfqpoint{-0.033333in}{0.033333in}}%
\pgfpathlineto{\pgfqpoint{-0.000000in}{-0.033333in}}%
\pgfpathclose%
\pgfusepath{stroke,fill}%
}%
\begin{pgfscope}%
\pgfsys@transformshift{2.014934in}{3.342249in}%
\pgfsys@useobject{currentmarker}{}%
\end{pgfscope}%
\end{pgfscope}%
\begin{pgfscope}%
\pgfpathrectangle{\pgfqpoint{0.765000in}{0.660000in}}{\pgfqpoint{4.620000in}{4.620000in}}%
\pgfusepath{clip}%
\pgfsetrectcap%
\pgfsetroundjoin%
\pgfsetlinewidth{1.204500pt}%
\definecolor{currentstroke}{rgb}{0.176471,0.192157,0.258824}%
\pgfsetstrokecolor{currentstroke}%
\pgfsetdash{}{0pt}%
\pgfpathmoveto{\pgfqpoint{1.448998in}{2.828761in}}%
\pgfusepath{stroke}%
\end{pgfscope}%
\begin{pgfscope}%
\pgfpathrectangle{\pgfqpoint{0.765000in}{0.660000in}}{\pgfqpoint{4.620000in}{4.620000in}}%
\pgfusepath{clip}%
\pgfsetbuttcap%
\pgfsetmiterjoin%
\definecolor{currentfill}{rgb}{0.176471,0.192157,0.258824}%
\pgfsetfillcolor{currentfill}%
\pgfsetlinewidth{1.003750pt}%
\definecolor{currentstroke}{rgb}{0.176471,0.192157,0.258824}%
\pgfsetstrokecolor{currentstroke}%
\pgfsetdash{}{0pt}%
\pgfsys@defobject{currentmarker}{\pgfqpoint{-0.033333in}{-0.033333in}}{\pgfqpoint{0.033333in}{0.033333in}}{%
\pgfpathmoveto{\pgfqpoint{-0.000000in}{-0.033333in}}%
\pgfpathlineto{\pgfqpoint{0.033333in}{0.033333in}}%
\pgfpathlineto{\pgfqpoint{-0.033333in}{0.033333in}}%
\pgfpathlineto{\pgfqpoint{-0.000000in}{-0.033333in}}%
\pgfpathclose%
\pgfusepath{stroke,fill}%
}%
\begin{pgfscope}%
\pgfsys@transformshift{1.448998in}{2.828761in}%
\pgfsys@useobject{currentmarker}{}%
\end{pgfscope}%
\end{pgfscope}%
\begin{pgfscope}%
\pgfpathrectangle{\pgfqpoint{0.765000in}{0.660000in}}{\pgfqpoint{4.620000in}{4.620000in}}%
\pgfusepath{clip}%
\pgfsetrectcap%
\pgfsetroundjoin%
\pgfsetlinewidth{1.204500pt}%
\definecolor{currentstroke}{rgb}{0.176471,0.192157,0.258824}%
\pgfsetstrokecolor{currentstroke}%
\pgfsetdash{}{0pt}%
\pgfpathmoveto{\pgfqpoint{3.118167in}{3.775065in}}%
\pgfusepath{stroke}%
\end{pgfscope}%
\begin{pgfscope}%
\pgfpathrectangle{\pgfqpoint{0.765000in}{0.660000in}}{\pgfqpoint{4.620000in}{4.620000in}}%
\pgfusepath{clip}%
\pgfsetbuttcap%
\pgfsetmiterjoin%
\definecolor{currentfill}{rgb}{0.176471,0.192157,0.258824}%
\pgfsetfillcolor{currentfill}%
\pgfsetlinewidth{1.003750pt}%
\definecolor{currentstroke}{rgb}{0.176471,0.192157,0.258824}%
\pgfsetstrokecolor{currentstroke}%
\pgfsetdash{}{0pt}%
\pgfsys@defobject{currentmarker}{\pgfqpoint{-0.033333in}{-0.033333in}}{\pgfqpoint{0.033333in}{0.033333in}}{%
\pgfpathmoveto{\pgfqpoint{-0.000000in}{-0.033333in}}%
\pgfpathlineto{\pgfqpoint{0.033333in}{0.033333in}}%
\pgfpathlineto{\pgfqpoint{-0.033333in}{0.033333in}}%
\pgfpathlineto{\pgfqpoint{-0.000000in}{-0.033333in}}%
\pgfpathclose%
\pgfusepath{stroke,fill}%
}%
\begin{pgfscope}%
\pgfsys@transformshift{3.118167in}{3.775065in}%
\pgfsys@useobject{currentmarker}{}%
\end{pgfscope}%
\end{pgfscope}%
\begin{pgfscope}%
\pgfpathrectangle{\pgfqpoint{0.765000in}{0.660000in}}{\pgfqpoint{4.620000in}{4.620000in}}%
\pgfusepath{clip}%
\pgfsetrectcap%
\pgfsetroundjoin%
\pgfsetlinewidth{1.204500pt}%
\definecolor{currentstroke}{rgb}{0.176471,0.192157,0.258824}%
\pgfsetstrokecolor{currentstroke}%
\pgfsetdash{}{0pt}%
\pgfpathmoveto{\pgfqpoint{2.372919in}{3.831596in}}%
\pgfusepath{stroke}%
\end{pgfscope}%
\begin{pgfscope}%
\pgfpathrectangle{\pgfqpoint{0.765000in}{0.660000in}}{\pgfqpoint{4.620000in}{4.620000in}}%
\pgfusepath{clip}%
\pgfsetbuttcap%
\pgfsetmiterjoin%
\definecolor{currentfill}{rgb}{0.176471,0.192157,0.258824}%
\pgfsetfillcolor{currentfill}%
\pgfsetlinewidth{1.003750pt}%
\definecolor{currentstroke}{rgb}{0.176471,0.192157,0.258824}%
\pgfsetstrokecolor{currentstroke}%
\pgfsetdash{}{0pt}%
\pgfsys@defobject{currentmarker}{\pgfqpoint{-0.033333in}{-0.033333in}}{\pgfqpoint{0.033333in}{0.033333in}}{%
\pgfpathmoveto{\pgfqpoint{-0.000000in}{-0.033333in}}%
\pgfpathlineto{\pgfqpoint{0.033333in}{0.033333in}}%
\pgfpathlineto{\pgfqpoint{-0.033333in}{0.033333in}}%
\pgfpathlineto{\pgfqpoint{-0.000000in}{-0.033333in}}%
\pgfpathclose%
\pgfusepath{stroke,fill}%
}%
\begin{pgfscope}%
\pgfsys@transformshift{2.372919in}{3.831596in}%
\pgfsys@useobject{currentmarker}{}%
\end{pgfscope}%
\end{pgfscope}%
\begin{pgfscope}%
\pgfpathrectangle{\pgfqpoint{0.765000in}{0.660000in}}{\pgfqpoint{4.620000in}{4.620000in}}%
\pgfusepath{clip}%
\pgfsetrectcap%
\pgfsetroundjoin%
\pgfsetlinewidth{1.204500pt}%
\definecolor{currentstroke}{rgb}{0.176471,0.192157,0.258824}%
\pgfsetstrokecolor{currentstroke}%
\pgfsetdash{}{0pt}%
\pgfpathmoveto{\pgfqpoint{3.749403in}{3.203655in}}%
\pgfusepath{stroke}%
\end{pgfscope}%
\begin{pgfscope}%
\pgfpathrectangle{\pgfqpoint{0.765000in}{0.660000in}}{\pgfqpoint{4.620000in}{4.620000in}}%
\pgfusepath{clip}%
\pgfsetbuttcap%
\pgfsetmiterjoin%
\definecolor{currentfill}{rgb}{0.176471,0.192157,0.258824}%
\pgfsetfillcolor{currentfill}%
\pgfsetlinewidth{1.003750pt}%
\definecolor{currentstroke}{rgb}{0.176471,0.192157,0.258824}%
\pgfsetstrokecolor{currentstroke}%
\pgfsetdash{}{0pt}%
\pgfsys@defobject{currentmarker}{\pgfqpoint{-0.033333in}{-0.033333in}}{\pgfqpoint{0.033333in}{0.033333in}}{%
\pgfpathmoveto{\pgfqpoint{-0.000000in}{-0.033333in}}%
\pgfpathlineto{\pgfqpoint{0.033333in}{0.033333in}}%
\pgfpathlineto{\pgfqpoint{-0.033333in}{0.033333in}}%
\pgfpathlineto{\pgfqpoint{-0.000000in}{-0.033333in}}%
\pgfpathclose%
\pgfusepath{stroke,fill}%
}%
\begin{pgfscope}%
\pgfsys@transformshift{3.749403in}{3.203655in}%
\pgfsys@useobject{currentmarker}{}%
\end{pgfscope}%
\end{pgfscope}%
\begin{pgfscope}%
\pgfpathrectangle{\pgfqpoint{0.765000in}{0.660000in}}{\pgfqpoint{4.620000in}{4.620000in}}%
\pgfusepath{clip}%
\pgfsetrectcap%
\pgfsetroundjoin%
\pgfsetlinewidth{1.204500pt}%
\definecolor{currentstroke}{rgb}{0.176471,0.192157,0.258824}%
\pgfsetstrokecolor{currentstroke}%
\pgfsetdash{}{0pt}%
\pgfpathmoveto{\pgfqpoint{2.033479in}{3.496331in}}%
\pgfusepath{stroke}%
\end{pgfscope}%
\begin{pgfscope}%
\pgfpathrectangle{\pgfqpoint{0.765000in}{0.660000in}}{\pgfqpoint{4.620000in}{4.620000in}}%
\pgfusepath{clip}%
\pgfsetbuttcap%
\pgfsetmiterjoin%
\definecolor{currentfill}{rgb}{0.176471,0.192157,0.258824}%
\pgfsetfillcolor{currentfill}%
\pgfsetlinewidth{1.003750pt}%
\definecolor{currentstroke}{rgb}{0.176471,0.192157,0.258824}%
\pgfsetstrokecolor{currentstroke}%
\pgfsetdash{}{0pt}%
\pgfsys@defobject{currentmarker}{\pgfqpoint{-0.033333in}{-0.033333in}}{\pgfqpoint{0.033333in}{0.033333in}}{%
\pgfpathmoveto{\pgfqpoint{-0.000000in}{-0.033333in}}%
\pgfpathlineto{\pgfqpoint{0.033333in}{0.033333in}}%
\pgfpathlineto{\pgfqpoint{-0.033333in}{0.033333in}}%
\pgfpathlineto{\pgfqpoint{-0.000000in}{-0.033333in}}%
\pgfpathclose%
\pgfusepath{stroke,fill}%
}%
\begin{pgfscope}%
\pgfsys@transformshift{2.033479in}{3.496331in}%
\pgfsys@useobject{currentmarker}{}%
\end{pgfscope}%
\end{pgfscope}%
\begin{pgfscope}%
\pgfpathrectangle{\pgfqpoint{0.765000in}{0.660000in}}{\pgfqpoint{4.620000in}{4.620000in}}%
\pgfusepath{clip}%
\pgfsetrectcap%
\pgfsetroundjoin%
\pgfsetlinewidth{1.204500pt}%
\definecolor{currentstroke}{rgb}{0.176471,0.192157,0.258824}%
\pgfsetstrokecolor{currentstroke}%
\pgfsetdash{}{0pt}%
\pgfpathmoveto{\pgfqpoint{1.760636in}{3.036768in}}%
\pgfusepath{stroke}%
\end{pgfscope}%
\begin{pgfscope}%
\pgfpathrectangle{\pgfqpoint{0.765000in}{0.660000in}}{\pgfqpoint{4.620000in}{4.620000in}}%
\pgfusepath{clip}%
\pgfsetbuttcap%
\pgfsetmiterjoin%
\definecolor{currentfill}{rgb}{0.176471,0.192157,0.258824}%
\pgfsetfillcolor{currentfill}%
\pgfsetlinewidth{1.003750pt}%
\definecolor{currentstroke}{rgb}{0.176471,0.192157,0.258824}%
\pgfsetstrokecolor{currentstroke}%
\pgfsetdash{}{0pt}%
\pgfsys@defobject{currentmarker}{\pgfqpoint{-0.033333in}{-0.033333in}}{\pgfqpoint{0.033333in}{0.033333in}}{%
\pgfpathmoveto{\pgfqpoint{-0.000000in}{-0.033333in}}%
\pgfpathlineto{\pgfqpoint{0.033333in}{0.033333in}}%
\pgfpathlineto{\pgfqpoint{-0.033333in}{0.033333in}}%
\pgfpathlineto{\pgfqpoint{-0.000000in}{-0.033333in}}%
\pgfpathclose%
\pgfusepath{stroke,fill}%
}%
\begin{pgfscope}%
\pgfsys@transformshift{1.760636in}{3.036768in}%
\pgfsys@useobject{currentmarker}{}%
\end{pgfscope}%
\end{pgfscope}%
\begin{pgfscope}%
\pgfpathrectangle{\pgfqpoint{0.765000in}{0.660000in}}{\pgfqpoint{4.620000in}{4.620000in}}%
\pgfusepath{clip}%
\pgfsetrectcap%
\pgfsetroundjoin%
\pgfsetlinewidth{1.204500pt}%
\definecolor{currentstroke}{rgb}{0.176471,0.192157,0.258824}%
\pgfsetstrokecolor{currentstroke}%
\pgfsetdash{}{0pt}%
\pgfpathmoveto{\pgfqpoint{3.308474in}{3.848578in}}%
\pgfusepath{stroke}%
\end{pgfscope}%
\begin{pgfscope}%
\pgfpathrectangle{\pgfqpoint{0.765000in}{0.660000in}}{\pgfqpoint{4.620000in}{4.620000in}}%
\pgfusepath{clip}%
\pgfsetbuttcap%
\pgfsetmiterjoin%
\definecolor{currentfill}{rgb}{0.176471,0.192157,0.258824}%
\pgfsetfillcolor{currentfill}%
\pgfsetlinewidth{1.003750pt}%
\definecolor{currentstroke}{rgb}{0.176471,0.192157,0.258824}%
\pgfsetstrokecolor{currentstroke}%
\pgfsetdash{}{0pt}%
\pgfsys@defobject{currentmarker}{\pgfqpoint{-0.033333in}{-0.033333in}}{\pgfqpoint{0.033333in}{0.033333in}}{%
\pgfpathmoveto{\pgfqpoint{-0.000000in}{-0.033333in}}%
\pgfpathlineto{\pgfqpoint{0.033333in}{0.033333in}}%
\pgfpathlineto{\pgfqpoint{-0.033333in}{0.033333in}}%
\pgfpathlineto{\pgfqpoint{-0.000000in}{-0.033333in}}%
\pgfpathclose%
\pgfusepath{stroke,fill}%
}%
\begin{pgfscope}%
\pgfsys@transformshift{3.308474in}{3.848578in}%
\pgfsys@useobject{currentmarker}{}%
\end{pgfscope}%
\end{pgfscope}%
\begin{pgfscope}%
\pgfpathrectangle{\pgfqpoint{0.765000in}{0.660000in}}{\pgfqpoint{4.620000in}{4.620000in}}%
\pgfusepath{clip}%
\pgfsetrectcap%
\pgfsetroundjoin%
\pgfsetlinewidth{1.204500pt}%
\definecolor{currentstroke}{rgb}{0.176471,0.192157,0.258824}%
\pgfsetstrokecolor{currentstroke}%
\pgfsetdash{}{0pt}%
\pgfpathmoveto{\pgfqpoint{2.063398in}{3.557623in}}%
\pgfusepath{stroke}%
\end{pgfscope}%
\begin{pgfscope}%
\pgfpathrectangle{\pgfqpoint{0.765000in}{0.660000in}}{\pgfqpoint{4.620000in}{4.620000in}}%
\pgfusepath{clip}%
\pgfsetbuttcap%
\pgfsetmiterjoin%
\definecolor{currentfill}{rgb}{0.176471,0.192157,0.258824}%
\pgfsetfillcolor{currentfill}%
\pgfsetlinewidth{1.003750pt}%
\definecolor{currentstroke}{rgb}{0.176471,0.192157,0.258824}%
\pgfsetstrokecolor{currentstroke}%
\pgfsetdash{}{0pt}%
\pgfsys@defobject{currentmarker}{\pgfqpoint{-0.033333in}{-0.033333in}}{\pgfqpoint{0.033333in}{0.033333in}}{%
\pgfpathmoveto{\pgfqpoint{-0.000000in}{-0.033333in}}%
\pgfpathlineto{\pgfqpoint{0.033333in}{0.033333in}}%
\pgfpathlineto{\pgfqpoint{-0.033333in}{0.033333in}}%
\pgfpathlineto{\pgfqpoint{-0.000000in}{-0.033333in}}%
\pgfpathclose%
\pgfusepath{stroke,fill}%
}%
\begin{pgfscope}%
\pgfsys@transformshift{2.063398in}{3.557623in}%
\pgfsys@useobject{currentmarker}{}%
\end{pgfscope}%
\end{pgfscope}%
\begin{pgfscope}%
\pgfpathrectangle{\pgfqpoint{0.765000in}{0.660000in}}{\pgfqpoint{4.620000in}{4.620000in}}%
\pgfusepath{clip}%
\pgfsetrectcap%
\pgfsetroundjoin%
\pgfsetlinewidth{1.204500pt}%
\definecolor{currentstroke}{rgb}{0.176471,0.192157,0.258824}%
\pgfsetstrokecolor{currentstroke}%
\pgfsetdash{}{0pt}%
\pgfpathmoveto{\pgfqpoint{3.211905in}{4.126362in}}%
\pgfusepath{stroke}%
\end{pgfscope}%
\begin{pgfscope}%
\pgfpathrectangle{\pgfqpoint{0.765000in}{0.660000in}}{\pgfqpoint{4.620000in}{4.620000in}}%
\pgfusepath{clip}%
\pgfsetbuttcap%
\pgfsetmiterjoin%
\definecolor{currentfill}{rgb}{0.176471,0.192157,0.258824}%
\pgfsetfillcolor{currentfill}%
\pgfsetlinewidth{1.003750pt}%
\definecolor{currentstroke}{rgb}{0.176471,0.192157,0.258824}%
\pgfsetstrokecolor{currentstroke}%
\pgfsetdash{}{0pt}%
\pgfsys@defobject{currentmarker}{\pgfqpoint{-0.033333in}{-0.033333in}}{\pgfqpoint{0.033333in}{0.033333in}}{%
\pgfpathmoveto{\pgfqpoint{-0.000000in}{-0.033333in}}%
\pgfpathlineto{\pgfqpoint{0.033333in}{0.033333in}}%
\pgfpathlineto{\pgfqpoint{-0.033333in}{0.033333in}}%
\pgfpathlineto{\pgfqpoint{-0.000000in}{-0.033333in}}%
\pgfpathclose%
\pgfusepath{stroke,fill}%
}%
\begin{pgfscope}%
\pgfsys@transformshift{3.211905in}{4.126362in}%
\pgfsys@useobject{currentmarker}{}%
\end{pgfscope}%
\end{pgfscope}%
\begin{pgfscope}%
\pgfpathrectangle{\pgfqpoint{0.765000in}{0.660000in}}{\pgfqpoint{4.620000in}{4.620000in}}%
\pgfusepath{clip}%
\pgfsetrectcap%
\pgfsetroundjoin%
\pgfsetlinewidth{1.204500pt}%
\definecolor{currentstroke}{rgb}{0.176471,0.192157,0.258824}%
\pgfsetstrokecolor{currentstroke}%
\pgfsetdash{}{0pt}%
\pgfpathmoveto{\pgfqpoint{4.035681in}{3.276992in}}%
\pgfusepath{stroke}%
\end{pgfscope}%
\begin{pgfscope}%
\pgfpathrectangle{\pgfqpoint{0.765000in}{0.660000in}}{\pgfqpoint{4.620000in}{4.620000in}}%
\pgfusepath{clip}%
\pgfsetbuttcap%
\pgfsetmiterjoin%
\definecolor{currentfill}{rgb}{0.176471,0.192157,0.258824}%
\pgfsetfillcolor{currentfill}%
\pgfsetlinewidth{1.003750pt}%
\definecolor{currentstroke}{rgb}{0.176471,0.192157,0.258824}%
\pgfsetstrokecolor{currentstroke}%
\pgfsetdash{}{0pt}%
\pgfsys@defobject{currentmarker}{\pgfqpoint{-0.033333in}{-0.033333in}}{\pgfqpoint{0.033333in}{0.033333in}}{%
\pgfpathmoveto{\pgfqpoint{-0.000000in}{-0.033333in}}%
\pgfpathlineto{\pgfqpoint{0.033333in}{0.033333in}}%
\pgfpathlineto{\pgfqpoint{-0.033333in}{0.033333in}}%
\pgfpathlineto{\pgfqpoint{-0.000000in}{-0.033333in}}%
\pgfpathclose%
\pgfusepath{stroke,fill}%
}%
\begin{pgfscope}%
\pgfsys@transformshift{4.035681in}{3.276992in}%
\pgfsys@useobject{currentmarker}{}%
\end{pgfscope}%
\end{pgfscope}%
\begin{pgfscope}%
\pgfpathrectangle{\pgfqpoint{0.765000in}{0.660000in}}{\pgfqpoint{4.620000in}{4.620000in}}%
\pgfusepath{clip}%
\pgfsetrectcap%
\pgfsetroundjoin%
\pgfsetlinewidth{1.204500pt}%
\definecolor{currentstroke}{rgb}{0.176471,0.192157,0.258824}%
\pgfsetstrokecolor{currentstroke}%
\pgfsetdash{}{0pt}%
\pgfpathmoveto{\pgfqpoint{1.572134in}{2.976159in}}%
\pgfusepath{stroke}%
\end{pgfscope}%
\begin{pgfscope}%
\pgfpathrectangle{\pgfqpoint{0.765000in}{0.660000in}}{\pgfqpoint{4.620000in}{4.620000in}}%
\pgfusepath{clip}%
\pgfsetbuttcap%
\pgfsetmiterjoin%
\definecolor{currentfill}{rgb}{0.176471,0.192157,0.258824}%
\pgfsetfillcolor{currentfill}%
\pgfsetlinewidth{1.003750pt}%
\definecolor{currentstroke}{rgb}{0.176471,0.192157,0.258824}%
\pgfsetstrokecolor{currentstroke}%
\pgfsetdash{}{0pt}%
\pgfsys@defobject{currentmarker}{\pgfqpoint{-0.033333in}{-0.033333in}}{\pgfqpoint{0.033333in}{0.033333in}}{%
\pgfpathmoveto{\pgfqpoint{-0.000000in}{-0.033333in}}%
\pgfpathlineto{\pgfqpoint{0.033333in}{0.033333in}}%
\pgfpathlineto{\pgfqpoint{-0.033333in}{0.033333in}}%
\pgfpathlineto{\pgfqpoint{-0.000000in}{-0.033333in}}%
\pgfpathclose%
\pgfusepath{stroke,fill}%
}%
\begin{pgfscope}%
\pgfsys@transformshift{1.572134in}{2.976159in}%
\pgfsys@useobject{currentmarker}{}%
\end{pgfscope}%
\end{pgfscope}%
\begin{pgfscope}%
\pgfpathrectangle{\pgfqpoint{0.765000in}{0.660000in}}{\pgfqpoint{4.620000in}{4.620000in}}%
\pgfusepath{clip}%
\pgfsetrectcap%
\pgfsetroundjoin%
\pgfsetlinewidth{1.204500pt}%
\definecolor{currentstroke}{rgb}{0.176471,0.192157,0.258824}%
\pgfsetstrokecolor{currentstroke}%
\pgfsetdash{}{0pt}%
\pgfpathmoveto{\pgfqpoint{3.619262in}{3.743923in}}%
\pgfusepath{stroke}%
\end{pgfscope}%
\begin{pgfscope}%
\pgfpathrectangle{\pgfqpoint{0.765000in}{0.660000in}}{\pgfqpoint{4.620000in}{4.620000in}}%
\pgfusepath{clip}%
\pgfsetbuttcap%
\pgfsetmiterjoin%
\definecolor{currentfill}{rgb}{0.176471,0.192157,0.258824}%
\pgfsetfillcolor{currentfill}%
\pgfsetlinewidth{1.003750pt}%
\definecolor{currentstroke}{rgb}{0.176471,0.192157,0.258824}%
\pgfsetstrokecolor{currentstroke}%
\pgfsetdash{}{0pt}%
\pgfsys@defobject{currentmarker}{\pgfqpoint{-0.033333in}{-0.033333in}}{\pgfqpoint{0.033333in}{0.033333in}}{%
\pgfpathmoveto{\pgfqpoint{-0.000000in}{-0.033333in}}%
\pgfpathlineto{\pgfqpoint{0.033333in}{0.033333in}}%
\pgfpathlineto{\pgfqpoint{-0.033333in}{0.033333in}}%
\pgfpathlineto{\pgfqpoint{-0.000000in}{-0.033333in}}%
\pgfpathclose%
\pgfusepath{stroke,fill}%
}%
\begin{pgfscope}%
\pgfsys@transformshift{3.619262in}{3.743923in}%
\pgfsys@useobject{currentmarker}{}%
\end{pgfscope}%
\end{pgfscope}%
\begin{pgfscope}%
\pgfpathrectangle{\pgfqpoint{0.765000in}{0.660000in}}{\pgfqpoint{4.620000in}{4.620000in}}%
\pgfusepath{clip}%
\pgfsetrectcap%
\pgfsetroundjoin%
\pgfsetlinewidth{1.204500pt}%
\definecolor{currentstroke}{rgb}{0.176471,0.192157,0.258824}%
\pgfsetstrokecolor{currentstroke}%
\pgfsetdash{}{0pt}%
\pgfpathmoveto{\pgfqpoint{3.689541in}{3.017452in}}%
\pgfusepath{stroke}%
\end{pgfscope}%
\begin{pgfscope}%
\pgfpathrectangle{\pgfqpoint{0.765000in}{0.660000in}}{\pgfqpoint{4.620000in}{4.620000in}}%
\pgfusepath{clip}%
\pgfsetbuttcap%
\pgfsetmiterjoin%
\definecolor{currentfill}{rgb}{0.176471,0.192157,0.258824}%
\pgfsetfillcolor{currentfill}%
\pgfsetlinewidth{1.003750pt}%
\definecolor{currentstroke}{rgb}{0.176471,0.192157,0.258824}%
\pgfsetstrokecolor{currentstroke}%
\pgfsetdash{}{0pt}%
\pgfsys@defobject{currentmarker}{\pgfqpoint{-0.033333in}{-0.033333in}}{\pgfqpoint{0.033333in}{0.033333in}}{%
\pgfpathmoveto{\pgfqpoint{-0.000000in}{-0.033333in}}%
\pgfpathlineto{\pgfqpoint{0.033333in}{0.033333in}}%
\pgfpathlineto{\pgfqpoint{-0.033333in}{0.033333in}}%
\pgfpathlineto{\pgfqpoint{-0.000000in}{-0.033333in}}%
\pgfpathclose%
\pgfusepath{stroke,fill}%
}%
\begin{pgfscope}%
\pgfsys@transformshift{3.689541in}{3.017452in}%
\pgfsys@useobject{currentmarker}{}%
\end{pgfscope}%
\end{pgfscope}%
\begin{pgfscope}%
\pgfpathrectangle{\pgfqpoint{0.765000in}{0.660000in}}{\pgfqpoint{4.620000in}{4.620000in}}%
\pgfusepath{clip}%
\pgfsetrectcap%
\pgfsetroundjoin%
\pgfsetlinewidth{1.204500pt}%
\definecolor{currentstroke}{rgb}{0.176471,0.192157,0.258824}%
\pgfsetstrokecolor{currentstroke}%
\pgfsetdash{}{0pt}%
\pgfpathmoveto{\pgfqpoint{2.708852in}{3.695255in}}%
\pgfusepath{stroke}%
\end{pgfscope}%
\begin{pgfscope}%
\pgfpathrectangle{\pgfqpoint{0.765000in}{0.660000in}}{\pgfqpoint{4.620000in}{4.620000in}}%
\pgfusepath{clip}%
\pgfsetbuttcap%
\pgfsetmiterjoin%
\definecolor{currentfill}{rgb}{0.176471,0.192157,0.258824}%
\pgfsetfillcolor{currentfill}%
\pgfsetlinewidth{1.003750pt}%
\definecolor{currentstroke}{rgb}{0.176471,0.192157,0.258824}%
\pgfsetstrokecolor{currentstroke}%
\pgfsetdash{}{0pt}%
\pgfsys@defobject{currentmarker}{\pgfqpoint{-0.033333in}{-0.033333in}}{\pgfqpoint{0.033333in}{0.033333in}}{%
\pgfpathmoveto{\pgfqpoint{-0.000000in}{-0.033333in}}%
\pgfpathlineto{\pgfqpoint{0.033333in}{0.033333in}}%
\pgfpathlineto{\pgfqpoint{-0.033333in}{0.033333in}}%
\pgfpathlineto{\pgfqpoint{-0.000000in}{-0.033333in}}%
\pgfpathclose%
\pgfusepath{stroke,fill}%
}%
\begin{pgfscope}%
\pgfsys@transformshift{2.708852in}{3.695255in}%
\pgfsys@useobject{currentmarker}{}%
\end{pgfscope}%
\end{pgfscope}%
\begin{pgfscope}%
\pgfpathrectangle{\pgfqpoint{0.765000in}{0.660000in}}{\pgfqpoint{4.620000in}{4.620000in}}%
\pgfusepath{clip}%
\pgfsetrectcap%
\pgfsetroundjoin%
\pgfsetlinewidth{1.204500pt}%
\definecolor{currentstroke}{rgb}{0.176471,0.192157,0.258824}%
\pgfsetstrokecolor{currentstroke}%
\pgfsetdash{}{0pt}%
\pgfpathmoveto{\pgfqpoint{2.114600in}{3.293923in}}%
\pgfusepath{stroke}%
\end{pgfscope}%
\begin{pgfscope}%
\pgfpathrectangle{\pgfqpoint{0.765000in}{0.660000in}}{\pgfqpoint{4.620000in}{4.620000in}}%
\pgfusepath{clip}%
\pgfsetbuttcap%
\pgfsetmiterjoin%
\definecolor{currentfill}{rgb}{0.176471,0.192157,0.258824}%
\pgfsetfillcolor{currentfill}%
\pgfsetlinewidth{1.003750pt}%
\definecolor{currentstroke}{rgb}{0.176471,0.192157,0.258824}%
\pgfsetstrokecolor{currentstroke}%
\pgfsetdash{}{0pt}%
\pgfsys@defobject{currentmarker}{\pgfqpoint{-0.033333in}{-0.033333in}}{\pgfqpoint{0.033333in}{0.033333in}}{%
\pgfpathmoveto{\pgfqpoint{-0.000000in}{-0.033333in}}%
\pgfpathlineto{\pgfqpoint{0.033333in}{0.033333in}}%
\pgfpathlineto{\pgfqpoint{-0.033333in}{0.033333in}}%
\pgfpathlineto{\pgfqpoint{-0.000000in}{-0.033333in}}%
\pgfpathclose%
\pgfusepath{stroke,fill}%
}%
\begin{pgfscope}%
\pgfsys@transformshift{2.114600in}{3.293923in}%
\pgfsys@useobject{currentmarker}{}%
\end{pgfscope}%
\end{pgfscope}%
\begin{pgfscope}%
\pgfpathrectangle{\pgfqpoint{0.765000in}{0.660000in}}{\pgfqpoint{4.620000in}{4.620000in}}%
\pgfusepath{clip}%
\pgfsetrectcap%
\pgfsetroundjoin%
\pgfsetlinewidth{1.204500pt}%
\definecolor{currentstroke}{rgb}{0.176471,0.192157,0.258824}%
\pgfsetstrokecolor{currentstroke}%
\pgfsetdash{}{0pt}%
\pgfpathmoveto{\pgfqpoint{1.302347in}{3.066289in}}%
\pgfusepath{stroke}%
\end{pgfscope}%
\begin{pgfscope}%
\pgfpathrectangle{\pgfqpoint{0.765000in}{0.660000in}}{\pgfqpoint{4.620000in}{4.620000in}}%
\pgfusepath{clip}%
\pgfsetbuttcap%
\pgfsetmiterjoin%
\definecolor{currentfill}{rgb}{0.176471,0.192157,0.258824}%
\pgfsetfillcolor{currentfill}%
\pgfsetlinewidth{1.003750pt}%
\definecolor{currentstroke}{rgb}{0.176471,0.192157,0.258824}%
\pgfsetstrokecolor{currentstroke}%
\pgfsetdash{}{0pt}%
\pgfsys@defobject{currentmarker}{\pgfqpoint{-0.033333in}{-0.033333in}}{\pgfqpoint{0.033333in}{0.033333in}}{%
\pgfpathmoveto{\pgfqpoint{-0.000000in}{-0.033333in}}%
\pgfpathlineto{\pgfqpoint{0.033333in}{0.033333in}}%
\pgfpathlineto{\pgfqpoint{-0.033333in}{0.033333in}}%
\pgfpathlineto{\pgfqpoint{-0.000000in}{-0.033333in}}%
\pgfpathclose%
\pgfusepath{stroke,fill}%
}%
\begin{pgfscope}%
\pgfsys@transformshift{1.302347in}{3.066289in}%
\pgfsys@useobject{currentmarker}{}%
\end{pgfscope}%
\end{pgfscope}%
\begin{pgfscope}%
\pgfpathrectangle{\pgfqpoint{0.765000in}{0.660000in}}{\pgfqpoint{4.620000in}{4.620000in}}%
\pgfusepath{clip}%
\pgfsetrectcap%
\pgfsetroundjoin%
\pgfsetlinewidth{1.204500pt}%
\definecolor{currentstroke}{rgb}{0.176471,0.192157,0.258824}%
\pgfsetstrokecolor{currentstroke}%
\pgfsetdash{}{0pt}%
\pgfpathmoveto{\pgfqpoint{3.818115in}{3.431465in}}%
\pgfusepath{stroke}%
\end{pgfscope}%
\begin{pgfscope}%
\pgfpathrectangle{\pgfqpoint{0.765000in}{0.660000in}}{\pgfqpoint{4.620000in}{4.620000in}}%
\pgfusepath{clip}%
\pgfsetbuttcap%
\pgfsetmiterjoin%
\definecolor{currentfill}{rgb}{0.176471,0.192157,0.258824}%
\pgfsetfillcolor{currentfill}%
\pgfsetlinewidth{1.003750pt}%
\definecolor{currentstroke}{rgb}{0.176471,0.192157,0.258824}%
\pgfsetstrokecolor{currentstroke}%
\pgfsetdash{}{0pt}%
\pgfsys@defobject{currentmarker}{\pgfqpoint{-0.033333in}{-0.033333in}}{\pgfqpoint{0.033333in}{0.033333in}}{%
\pgfpathmoveto{\pgfqpoint{-0.000000in}{-0.033333in}}%
\pgfpathlineto{\pgfqpoint{0.033333in}{0.033333in}}%
\pgfpathlineto{\pgfqpoint{-0.033333in}{0.033333in}}%
\pgfpathlineto{\pgfqpoint{-0.000000in}{-0.033333in}}%
\pgfpathclose%
\pgfusepath{stroke,fill}%
}%
\begin{pgfscope}%
\pgfsys@transformshift{3.818115in}{3.431465in}%
\pgfsys@useobject{currentmarker}{}%
\end{pgfscope}%
\end{pgfscope}%
\begin{pgfscope}%
\pgfpathrectangle{\pgfqpoint{0.765000in}{0.660000in}}{\pgfqpoint{4.620000in}{4.620000in}}%
\pgfusepath{clip}%
\pgfsetrectcap%
\pgfsetroundjoin%
\pgfsetlinewidth{1.204500pt}%
\definecolor{currentstroke}{rgb}{0.176471,0.192157,0.258824}%
\pgfsetstrokecolor{currentstroke}%
\pgfsetdash{}{0pt}%
\pgfpathmoveto{\pgfqpoint{3.705133in}{2.879846in}}%
\pgfusepath{stroke}%
\end{pgfscope}%
\begin{pgfscope}%
\pgfpathrectangle{\pgfqpoint{0.765000in}{0.660000in}}{\pgfqpoint{4.620000in}{4.620000in}}%
\pgfusepath{clip}%
\pgfsetbuttcap%
\pgfsetmiterjoin%
\definecolor{currentfill}{rgb}{0.176471,0.192157,0.258824}%
\pgfsetfillcolor{currentfill}%
\pgfsetlinewidth{1.003750pt}%
\definecolor{currentstroke}{rgb}{0.176471,0.192157,0.258824}%
\pgfsetstrokecolor{currentstroke}%
\pgfsetdash{}{0pt}%
\pgfsys@defobject{currentmarker}{\pgfqpoint{-0.033333in}{-0.033333in}}{\pgfqpoint{0.033333in}{0.033333in}}{%
\pgfpathmoveto{\pgfqpoint{-0.000000in}{-0.033333in}}%
\pgfpathlineto{\pgfqpoint{0.033333in}{0.033333in}}%
\pgfpathlineto{\pgfqpoint{-0.033333in}{0.033333in}}%
\pgfpathlineto{\pgfqpoint{-0.000000in}{-0.033333in}}%
\pgfpathclose%
\pgfusepath{stroke,fill}%
}%
\begin{pgfscope}%
\pgfsys@transformshift{3.705133in}{2.879846in}%
\pgfsys@useobject{currentmarker}{}%
\end{pgfscope}%
\end{pgfscope}%
\begin{pgfscope}%
\pgfpathrectangle{\pgfqpoint{0.765000in}{0.660000in}}{\pgfqpoint{4.620000in}{4.620000in}}%
\pgfusepath{clip}%
\pgfsetrectcap%
\pgfsetroundjoin%
\pgfsetlinewidth{1.204500pt}%
\definecolor{currentstroke}{rgb}{0.176471,0.192157,0.258824}%
\pgfsetstrokecolor{currentstroke}%
\pgfsetdash{}{0pt}%
\pgfpathmoveto{\pgfqpoint{3.476343in}{4.034357in}}%
\pgfusepath{stroke}%
\end{pgfscope}%
\begin{pgfscope}%
\pgfpathrectangle{\pgfqpoint{0.765000in}{0.660000in}}{\pgfqpoint{4.620000in}{4.620000in}}%
\pgfusepath{clip}%
\pgfsetbuttcap%
\pgfsetmiterjoin%
\definecolor{currentfill}{rgb}{0.176471,0.192157,0.258824}%
\pgfsetfillcolor{currentfill}%
\pgfsetlinewidth{1.003750pt}%
\definecolor{currentstroke}{rgb}{0.176471,0.192157,0.258824}%
\pgfsetstrokecolor{currentstroke}%
\pgfsetdash{}{0pt}%
\pgfsys@defobject{currentmarker}{\pgfqpoint{-0.033333in}{-0.033333in}}{\pgfqpoint{0.033333in}{0.033333in}}{%
\pgfpathmoveto{\pgfqpoint{-0.000000in}{-0.033333in}}%
\pgfpathlineto{\pgfqpoint{0.033333in}{0.033333in}}%
\pgfpathlineto{\pgfqpoint{-0.033333in}{0.033333in}}%
\pgfpathlineto{\pgfqpoint{-0.000000in}{-0.033333in}}%
\pgfpathclose%
\pgfusepath{stroke,fill}%
}%
\begin{pgfscope}%
\pgfsys@transformshift{3.476343in}{4.034357in}%
\pgfsys@useobject{currentmarker}{}%
\end{pgfscope}%
\end{pgfscope}%
\begin{pgfscope}%
\pgfpathrectangle{\pgfqpoint{0.765000in}{0.660000in}}{\pgfqpoint{4.620000in}{4.620000in}}%
\pgfusepath{clip}%
\pgfsetrectcap%
\pgfsetroundjoin%
\pgfsetlinewidth{1.204500pt}%
\definecolor{currentstroke}{rgb}{0.176471,0.192157,0.258824}%
\pgfsetstrokecolor{currentstroke}%
\pgfsetdash{}{0pt}%
\pgfpathmoveto{\pgfqpoint{4.025950in}{3.332136in}}%
\pgfusepath{stroke}%
\end{pgfscope}%
\begin{pgfscope}%
\pgfpathrectangle{\pgfqpoint{0.765000in}{0.660000in}}{\pgfqpoint{4.620000in}{4.620000in}}%
\pgfusepath{clip}%
\pgfsetbuttcap%
\pgfsetmiterjoin%
\definecolor{currentfill}{rgb}{0.176471,0.192157,0.258824}%
\pgfsetfillcolor{currentfill}%
\pgfsetlinewidth{1.003750pt}%
\definecolor{currentstroke}{rgb}{0.176471,0.192157,0.258824}%
\pgfsetstrokecolor{currentstroke}%
\pgfsetdash{}{0pt}%
\pgfsys@defobject{currentmarker}{\pgfqpoint{-0.033333in}{-0.033333in}}{\pgfqpoint{0.033333in}{0.033333in}}{%
\pgfpathmoveto{\pgfqpoint{-0.000000in}{-0.033333in}}%
\pgfpathlineto{\pgfqpoint{0.033333in}{0.033333in}}%
\pgfpathlineto{\pgfqpoint{-0.033333in}{0.033333in}}%
\pgfpathlineto{\pgfqpoint{-0.000000in}{-0.033333in}}%
\pgfpathclose%
\pgfusepath{stroke,fill}%
}%
\begin{pgfscope}%
\pgfsys@transformshift{4.025950in}{3.332136in}%
\pgfsys@useobject{currentmarker}{}%
\end{pgfscope}%
\end{pgfscope}%
\begin{pgfscope}%
\pgfpathrectangle{\pgfqpoint{0.765000in}{0.660000in}}{\pgfqpoint{4.620000in}{4.620000in}}%
\pgfusepath{clip}%
\pgfsetrectcap%
\pgfsetroundjoin%
\pgfsetlinewidth{1.204500pt}%
\definecolor{currentstroke}{rgb}{0.176471,0.192157,0.258824}%
\pgfsetstrokecolor{currentstroke}%
\pgfsetdash{}{0pt}%
\pgfpathmoveto{\pgfqpoint{3.481232in}{3.322527in}}%
\pgfusepath{stroke}%
\end{pgfscope}%
\begin{pgfscope}%
\pgfpathrectangle{\pgfqpoint{0.765000in}{0.660000in}}{\pgfqpoint{4.620000in}{4.620000in}}%
\pgfusepath{clip}%
\pgfsetbuttcap%
\pgfsetmiterjoin%
\definecolor{currentfill}{rgb}{0.176471,0.192157,0.258824}%
\pgfsetfillcolor{currentfill}%
\pgfsetlinewidth{1.003750pt}%
\definecolor{currentstroke}{rgb}{0.176471,0.192157,0.258824}%
\pgfsetstrokecolor{currentstroke}%
\pgfsetdash{}{0pt}%
\pgfsys@defobject{currentmarker}{\pgfqpoint{-0.033333in}{-0.033333in}}{\pgfqpoint{0.033333in}{0.033333in}}{%
\pgfpathmoveto{\pgfqpoint{-0.000000in}{-0.033333in}}%
\pgfpathlineto{\pgfqpoint{0.033333in}{0.033333in}}%
\pgfpathlineto{\pgfqpoint{-0.033333in}{0.033333in}}%
\pgfpathlineto{\pgfqpoint{-0.000000in}{-0.033333in}}%
\pgfpathclose%
\pgfusepath{stroke,fill}%
}%
\begin{pgfscope}%
\pgfsys@transformshift{3.481232in}{3.322527in}%
\pgfsys@useobject{currentmarker}{}%
\end{pgfscope}%
\end{pgfscope}%
\begin{pgfscope}%
\pgfpathrectangle{\pgfqpoint{0.765000in}{0.660000in}}{\pgfqpoint{4.620000in}{4.620000in}}%
\pgfusepath{clip}%
\pgfsetrectcap%
\pgfsetroundjoin%
\pgfsetlinewidth{1.204500pt}%
\definecolor{currentstroke}{rgb}{0.176471,0.192157,0.258824}%
\pgfsetstrokecolor{currentstroke}%
\pgfsetdash{}{0pt}%
\pgfpathmoveto{\pgfqpoint{3.448141in}{3.298430in}}%
\pgfusepath{stroke}%
\end{pgfscope}%
\begin{pgfscope}%
\pgfpathrectangle{\pgfqpoint{0.765000in}{0.660000in}}{\pgfqpoint{4.620000in}{4.620000in}}%
\pgfusepath{clip}%
\pgfsetbuttcap%
\pgfsetmiterjoin%
\definecolor{currentfill}{rgb}{0.176471,0.192157,0.258824}%
\pgfsetfillcolor{currentfill}%
\pgfsetlinewidth{1.003750pt}%
\definecolor{currentstroke}{rgb}{0.176471,0.192157,0.258824}%
\pgfsetstrokecolor{currentstroke}%
\pgfsetdash{}{0pt}%
\pgfsys@defobject{currentmarker}{\pgfqpoint{-0.033333in}{-0.033333in}}{\pgfqpoint{0.033333in}{0.033333in}}{%
\pgfpathmoveto{\pgfqpoint{-0.000000in}{-0.033333in}}%
\pgfpathlineto{\pgfqpoint{0.033333in}{0.033333in}}%
\pgfpathlineto{\pgfqpoint{-0.033333in}{0.033333in}}%
\pgfpathlineto{\pgfqpoint{-0.000000in}{-0.033333in}}%
\pgfpathclose%
\pgfusepath{stroke,fill}%
}%
\begin{pgfscope}%
\pgfsys@transformshift{3.448141in}{3.298430in}%
\pgfsys@useobject{currentmarker}{}%
\end{pgfscope}%
\end{pgfscope}%
\begin{pgfscope}%
\pgfpathrectangle{\pgfqpoint{0.765000in}{0.660000in}}{\pgfqpoint{4.620000in}{4.620000in}}%
\pgfusepath{clip}%
\pgfsetrectcap%
\pgfsetroundjoin%
\pgfsetlinewidth{1.204500pt}%
\definecolor{currentstroke}{rgb}{0.176471,0.192157,0.258824}%
\pgfsetstrokecolor{currentstroke}%
\pgfsetdash{}{0pt}%
\pgfpathmoveto{\pgfqpoint{2.480143in}{3.642630in}}%
\pgfusepath{stroke}%
\end{pgfscope}%
\begin{pgfscope}%
\pgfpathrectangle{\pgfqpoint{0.765000in}{0.660000in}}{\pgfqpoint{4.620000in}{4.620000in}}%
\pgfusepath{clip}%
\pgfsetbuttcap%
\pgfsetmiterjoin%
\definecolor{currentfill}{rgb}{0.176471,0.192157,0.258824}%
\pgfsetfillcolor{currentfill}%
\pgfsetlinewidth{1.003750pt}%
\definecolor{currentstroke}{rgb}{0.176471,0.192157,0.258824}%
\pgfsetstrokecolor{currentstroke}%
\pgfsetdash{}{0pt}%
\pgfsys@defobject{currentmarker}{\pgfqpoint{-0.033333in}{-0.033333in}}{\pgfqpoint{0.033333in}{0.033333in}}{%
\pgfpathmoveto{\pgfqpoint{-0.000000in}{-0.033333in}}%
\pgfpathlineto{\pgfqpoint{0.033333in}{0.033333in}}%
\pgfpathlineto{\pgfqpoint{-0.033333in}{0.033333in}}%
\pgfpathlineto{\pgfqpoint{-0.000000in}{-0.033333in}}%
\pgfpathclose%
\pgfusepath{stroke,fill}%
}%
\begin{pgfscope}%
\pgfsys@transformshift{2.480143in}{3.642630in}%
\pgfsys@useobject{currentmarker}{}%
\end{pgfscope}%
\end{pgfscope}%
\begin{pgfscope}%
\pgfpathrectangle{\pgfqpoint{0.765000in}{0.660000in}}{\pgfqpoint{4.620000in}{4.620000in}}%
\pgfusepath{clip}%
\pgfsetrectcap%
\pgfsetroundjoin%
\pgfsetlinewidth{1.204500pt}%
\definecolor{currentstroke}{rgb}{0.176471,0.192157,0.258824}%
\pgfsetstrokecolor{currentstroke}%
\pgfsetdash{}{0pt}%
\pgfpathmoveto{\pgfqpoint{2.854111in}{3.705318in}}%
\pgfusepath{stroke}%
\end{pgfscope}%
\begin{pgfscope}%
\pgfpathrectangle{\pgfqpoint{0.765000in}{0.660000in}}{\pgfqpoint{4.620000in}{4.620000in}}%
\pgfusepath{clip}%
\pgfsetbuttcap%
\pgfsetmiterjoin%
\definecolor{currentfill}{rgb}{0.176471,0.192157,0.258824}%
\pgfsetfillcolor{currentfill}%
\pgfsetlinewidth{1.003750pt}%
\definecolor{currentstroke}{rgb}{0.176471,0.192157,0.258824}%
\pgfsetstrokecolor{currentstroke}%
\pgfsetdash{}{0pt}%
\pgfsys@defobject{currentmarker}{\pgfqpoint{-0.033333in}{-0.033333in}}{\pgfqpoint{0.033333in}{0.033333in}}{%
\pgfpathmoveto{\pgfqpoint{-0.000000in}{-0.033333in}}%
\pgfpathlineto{\pgfqpoint{0.033333in}{0.033333in}}%
\pgfpathlineto{\pgfqpoint{-0.033333in}{0.033333in}}%
\pgfpathlineto{\pgfqpoint{-0.000000in}{-0.033333in}}%
\pgfpathclose%
\pgfusepath{stroke,fill}%
}%
\begin{pgfscope}%
\pgfsys@transformshift{2.854111in}{3.705318in}%
\pgfsys@useobject{currentmarker}{}%
\end{pgfscope}%
\end{pgfscope}%
\begin{pgfscope}%
\pgfpathrectangle{\pgfqpoint{0.765000in}{0.660000in}}{\pgfqpoint{4.620000in}{4.620000in}}%
\pgfusepath{clip}%
\pgfsetrectcap%
\pgfsetroundjoin%
\pgfsetlinewidth{1.204500pt}%
\definecolor{currentstroke}{rgb}{0.176471,0.192157,0.258824}%
\pgfsetstrokecolor{currentstroke}%
\pgfsetdash{}{0pt}%
\pgfpathmoveto{\pgfqpoint{3.676361in}{3.294790in}}%
\pgfusepath{stroke}%
\end{pgfscope}%
\begin{pgfscope}%
\pgfpathrectangle{\pgfqpoint{0.765000in}{0.660000in}}{\pgfqpoint{4.620000in}{4.620000in}}%
\pgfusepath{clip}%
\pgfsetbuttcap%
\pgfsetmiterjoin%
\definecolor{currentfill}{rgb}{0.176471,0.192157,0.258824}%
\pgfsetfillcolor{currentfill}%
\pgfsetlinewidth{1.003750pt}%
\definecolor{currentstroke}{rgb}{0.176471,0.192157,0.258824}%
\pgfsetstrokecolor{currentstroke}%
\pgfsetdash{}{0pt}%
\pgfsys@defobject{currentmarker}{\pgfqpoint{-0.033333in}{-0.033333in}}{\pgfqpoint{0.033333in}{0.033333in}}{%
\pgfpathmoveto{\pgfqpoint{-0.000000in}{-0.033333in}}%
\pgfpathlineto{\pgfqpoint{0.033333in}{0.033333in}}%
\pgfpathlineto{\pgfqpoint{-0.033333in}{0.033333in}}%
\pgfpathlineto{\pgfqpoint{-0.000000in}{-0.033333in}}%
\pgfpathclose%
\pgfusepath{stroke,fill}%
}%
\begin{pgfscope}%
\pgfsys@transformshift{3.676361in}{3.294790in}%
\pgfsys@useobject{currentmarker}{}%
\end{pgfscope}%
\end{pgfscope}%
\begin{pgfscope}%
\pgfpathrectangle{\pgfqpoint{0.765000in}{0.660000in}}{\pgfqpoint{4.620000in}{4.620000in}}%
\pgfusepath{clip}%
\pgfsetrectcap%
\pgfsetroundjoin%
\pgfsetlinewidth{1.204500pt}%
\definecolor{currentstroke}{rgb}{0.176471,0.192157,0.258824}%
\pgfsetstrokecolor{currentstroke}%
\pgfsetdash{}{0pt}%
\pgfpathmoveto{\pgfqpoint{3.331275in}{3.509822in}}%
\pgfusepath{stroke}%
\end{pgfscope}%
\begin{pgfscope}%
\pgfpathrectangle{\pgfqpoint{0.765000in}{0.660000in}}{\pgfqpoint{4.620000in}{4.620000in}}%
\pgfusepath{clip}%
\pgfsetbuttcap%
\pgfsetmiterjoin%
\definecolor{currentfill}{rgb}{0.176471,0.192157,0.258824}%
\pgfsetfillcolor{currentfill}%
\pgfsetlinewidth{1.003750pt}%
\definecolor{currentstroke}{rgb}{0.176471,0.192157,0.258824}%
\pgfsetstrokecolor{currentstroke}%
\pgfsetdash{}{0pt}%
\pgfsys@defobject{currentmarker}{\pgfqpoint{-0.033333in}{-0.033333in}}{\pgfqpoint{0.033333in}{0.033333in}}{%
\pgfpathmoveto{\pgfqpoint{-0.000000in}{-0.033333in}}%
\pgfpathlineto{\pgfqpoint{0.033333in}{0.033333in}}%
\pgfpathlineto{\pgfqpoint{-0.033333in}{0.033333in}}%
\pgfpathlineto{\pgfqpoint{-0.000000in}{-0.033333in}}%
\pgfpathclose%
\pgfusepath{stroke,fill}%
}%
\begin{pgfscope}%
\pgfsys@transformshift{3.331275in}{3.509822in}%
\pgfsys@useobject{currentmarker}{}%
\end{pgfscope}%
\end{pgfscope}%
\begin{pgfscope}%
\pgfpathrectangle{\pgfqpoint{0.765000in}{0.660000in}}{\pgfqpoint{4.620000in}{4.620000in}}%
\pgfusepath{clip}%
\pgfsetrectcap%
\pgfsetroundjoin%
\pgfsetlinewidth{1.204500pt}%
\definecolor{currentstroke}{rgb}{0.176471,0.192157,0.258824}%
\pgfsetstrokecolor{currentstroke}%
\pgfsetdash{}{0pt}%
\pgfpathmoveto{\pgfqpoint{3.273526in}{3.935444in}}%
\pgfusepath{stroke}%
\end{pgfscope}%
\begin{pgfscope}%
\pgfpathrectangle{\pgfqpoint{0.765000in}{0.660000in}}{\pgfqpoint{4.620000in}{4.620000in}}%
\pgfusepath{clip}%
\pgfsetbuttcap%
\pgfsetmiterjoin%
\definecolor{currentfill}{rgb}{0.176471,0.192157,0.258824}%
\pgfsetfillcolor{currentfill}%
\pgfsetlinewidth{1.003750pt}%
\definecolor{currentstroke}{rgb}{0.176471,0.192157,0.258824}%
\pgfsetstrokecolor{currentstroke}%
\pgfsetdash{}{0pt}%
\pgfsys@defobject{currentmarker}{\pgfqpoint{-0.033333in}{-0.033333in}}{\pgfqpoint{0.033333in}{0.033333in}}{%
\pgfpathmoveto{\pgfqpoint{-0.000000in}{-0.033333in}}%
\pgfpathlineto{\pgfqpoint{0.033333in}{0.033333in}}%
\pgfpathlineto{\pgfqpoint{-0.033333in}{0.033333in}}%
\pgfpathlineto{\pgfqpoint{-0.000000in}{-0.033333in}}%
\pgfpathclose%
\pgfusepath{stroke,fill}%
}%
\begin{pgfscope}%
\pgfsys@transformshift{3.273526in}{3.935444in}%
\pgfsys@useobject{currentmarker}{}%
\end{pgfscope}%
\end{pgfscope}%
\begin{pgfscope}%
\pgfpathrectangle{\pgfqpoint{0.765000in}{0.660000in}}{\pgfqpoint{4.620000in}{4.620000in}}%
\pgfusepath{clip}%
\pgfsetrectcap%
\pgfsetroundjoin%
\pgfsetlinewidth{1.204500pt}%
\definecolor{currentstroke}{rgb}{0.176471,0.192157,0.258824}%
\pgfsetstrokecolor{currentstroke}%
\pgfsetdash{}{0pt}%
\pgfpathmoveto{\pgfqpoint{3.353719in}{3.827502in}}%
\pgfusepath{stroke}%
\end{pgfscope}%
\begin{pgfscope}%
\pgfpathrectangle{\pgfqpoint{0.765000in}{0.660000in}}{\pgfqpoint{4.620000in}{4.620000in}}%
\pgfusepath{clip}%
\pgfsetbuttcap%
\pgfsetmiterjoin%
\definecolor{currentfill}{rgb}{0.176471,0.192157,0.258824}%
\pgfsetfillcolor{currentfill}%
\pgfsetlinewidth{1.003750pt}%
\definecolor{currentstroke}{rgb}{0.176471,0.192157,0.258824}%
\pgfsetstrokecolor{currentstroke}%
\pgfsetdash{}{0pt}%
\pgfsys@defobject{currentmarker}{\pgfqpoint{-0.033333in}{-0.033333in}}{\pgfqpoint{0.033333in}{0.033333in}}{%
\pgfpathmoveto{\pgfqpoint{-0.000000in}{-0.033333in}}%
\pgfpathlineto{\pgfqpoint{0.033333in}{0.033333in}}%
\pgfpathlineto{\pgfqpoint{-0.033333in}{0.033333in}}%
\pgfpathlineto{\pgfqpoint{-0.000000in}{-0.033333in}}%
\pgfpathclose%
\pgfusepath{stroke,fill}%
}%
\begin{pgfscope}%
\pgfsys@transformshift{3.353719in}{3.827502in}%
\pgfsys@useobject{currentmarker}{}%
\end{pgfscope}%
\end{pgfscope}%
\begin{pgfscope}%
\pgfpathrectangle{\pgfqpoint{0.765000in}{0.660000in}}{\pgfqpoint{4.620000in}{4.620000in}}%
\pgfusepath{clip}%
\pgfsetrectcap%
\pgfsetroundjoin%
\pgfsetlinewidth{1.204500pt}%
\definecolor{currentstroke}{rgb}{0.176471,0.192157,0.258824}%
\pgfsetstrokecolor{currentstroke}%
\pgfsetdash{}{0pt}%
\pgfpathmoveto{\pgfqpoint{3.705064in}{3.134533in}}%
\pgfusepath{stroke}%
\end{pgfscope}%
\begin{pgfscope}%
\pgfpathrectangle{\pgfqpoint{0.765000in}{0.660000in}}{\pgfqpoint{4.620000in}{4.620000in}}%
\pgfusepath{clip}%
\pgfsetbuttcap%
\pgfsetmiterjoin%
\definecolor{currentfill}{rgb}{0.176471,0.192157,0.258824}%
\pgfsetfillcolor{currentfill}%
\pgfsetlinewidth{1.003750pt}%
\definecolor{currentstroke}{rgb}{0.176471,0.192157,0.258824}%
\pgfsetstrokecolor{currentstroke}%
\pgfsetdash{}{0pt}%
\pgfsys@defobject{currentmarker}{\pgfqpoint{-0.033333in}{-0.033333in}}{\pgfqpoint{0.033333in}{0.033333in}}{%
\pgfpathmoveto{\pgfqpoint{-0.000000in}{-0.033333in}}%
\pgfpathlineto{\pgfqpoint{0.033333in}{0.033333in}}%
\pgfpathlineto{\pgfqpoint{-0.033333in}{0.033333in}}%
\pgfpathlineto{\pgfqpoint{-0.000000in}{-0.033333in}}%
\pgfpathclose%
\pgfusepath{stroke,fill}%
}%
\begin{pgfscope}%
\pgfsys@transformshift{3.705064in}{3.134533in}%
\pgfsys@useobject{currentmarker}{}%
\end{pgfscope}%
\end{pgfscope}%
\begin{pgfscope}%
\pgfpathrectangle{\pgfqpoint{0.765000in}{0.660000in}}{\pgfqpoint{4.620000in}{4.620000in}}%
\pgfusepath{clip}%
\pgfsetrectcap%
\pgfsetroundjoin%
\pgfsetlinewidth{1.204500pt}%
\definecolor{currentstroke}{rgb}{0.176471,0.192157,0.258824}%
\pgfsetstrokecolor{currentstroke}%
\pgfsetdash{}{0pt}%
\pgfpathmoveto{\pgfqpoint{3.792940in}{2.883043in}}%
\pgfusepath{stroke}%
\end{pgfscope}%
\begin{pgfscope}%
\pgfpathrectangle{\pgfqpoint{0.765000in}{0.660000in}}{\pgfqpoint{4.620000in}{4.620000in}}%
\pgfusepath{clip}%
\pgfsetbuttcap%
\pgfsetmiterjoin%
\definecolor{currentfill}{rgb}{0.176471,0.192157,0.258824}%
\pgfsetfillcolor{currentfill}%
\pgfsetlinewidth{1.003750pt}%
\definecolor{currentstroke}{rgb}{0.176471,0.192157,0.258824}%
\pgfsetstrokecolor{currentstroke}%
\pgfsetdash{}{0pt}%
\pgfsys@defobject{currentmarker}{\pgfqpoint{-0.033333in}{-0.033333in}}{\pgfqpoint{0.033333in}{0.033333in}}{%
\pgfpathmoveto{\pgfqpoint{-0.000000in}{-0.033333in}}%
\pgfpathlineto{\pgfqpoint{0.033333in}{0.033333in}}%
\pgfpathlineto{\pgfqpoint{-0.033333in}{0.033333in}}%
\pgfpathlineto{\pgfqpoint{-0.000000in}{-0.033333in}}%
\pgfpathclose%
\pgfusepath{stroke,fill}%
}%
\begin{pgfscope}%
\pgfsys@transformshift{3.792940in}{2.883043in}%
\pgfsys@useobject{currentmarker}{}%
\end{pgfscope}%
\end{pgfscope}%
\begin{pgfscope}%
\pgfpathrectangle{\pgfqpoint{0.765000in}{0.660000in}}{\pgfqpoint{4.620000in}{4.620000in}}%
\pgfusepath{clip}%
\pgfsetrectcap%
\pgfsetroundjoin%
\pgfsetlinewidth{1.204500pt}%
\definecolor{currentstroke}{rgb}{0.176471,0.192157,0.258824}%
\pgfsetstrokecolor{currentstroke}%
\pgfsetdash{}{0pt}%
\pgfpathmoveto{\pgfqpoint{3.589564in}{3.570865in}}%
\pgfusepath{stroke}%
\end{pgfscope}%
\begin{pgfscope}%
\pgfpathrectangle{\pgfqpoint{0.765000in}{0.660000in}}{\pgfqpoint{4.620000in}{4.620000in}}%
\pgfusepath{clip}%
\pgfsetbuttcap%
\pgfsetmiterjoin%
\definecolor{currentfill}{rgb}{0.176471,0.192157,0.258824}%
\pgfsetfillcolor{currentfill}%
\pgfsetlinewidth{1.003750pt}%
\definecolor{currentstroke}{rgb}{0.176471,0.192157,0.258824}%
\pgfsetstrokecolor{currentstroke}%
\pgfsetdash{}{0pt}%
\pgfsys@defobject{currentmarker}{\pgfqpoint{-0.033333in}{-0.033333in}}{\pgfqpoint{0.033333in}{0.033333in}}{%
\pgfpathmoveto{\pgfqpoint{-0.000000in}{-0.033333in}}%
\pgfpathlineto{\pgfqpoint{0.033333in}{0.033333in}}%
\pgfpathlineto{\pgfqpoint{-0.033333in}{0.033333in}}%
\pgfpathlineto{\pgfqpoint{-0.000000in}{-0.033333in}}%
\pgfpathclose%
\pgfusepath{stroke,fill}%
}%
\begin{pgfscope}%
\pgfsys@transformshift{3.589564in}{3.570865in}%
\pgfsys@useobject{currentmarker}{}%
\end{pgfscope}%
\end{pgfscope}%
\begin{pgfscope}%
\pgfpathrectangle{\pgfqpoint{0.765000in}{0.660000in}}{\pgfqpoint{4.620000in}{4.620000in}}%
\pgfusepath{clip}%
\pgfsetrectcap%
\pgfsetroundjoin%
\pgfsetlinewidth{1.204500pt}%
\definecolor{currentstroke}{rgb}{0.176471,0.192157,0.258824}%
\pgfsetstrokecolor{currentstroke}%
\pgfsetdash{}{0pt}%
\pgfpathmoveto{\pgfqpoint{3.849389in}{3.230205in}}%
\pgfusepath{stroke}%
\end{pgfscope}%
\begin{pgfscope}%
\pgfpathrectangle{\pgfqpoint{0.765000in}{0.660000in}}{\pgfqpoint{4.620000in}{4.620000in}}%
\pgfusepath{clip}%
\pgfsetbuttcap%
\pgfsetmiterjoin%
\definecolor{currentfill}{rgb}{0.176471,0.192157,0.258824}%
\pgfsetfillcolor{currentfill}%
\pgfsetlinewidth{1.003750pt}%
\definecolor{currentstroke}{rgb}{0.176471,0.192157,0.258824}%
\pgfsetstrokecolor{currentstroke}%
\pgfsetdash{}{0pt}%
\pgfsys@defobject{currentmarker}{\pgfqpoint{-0.033333in}{-0.033333in}}{\pgfqpoint{0.033333in}{0.033333in}}{%
\pgfpathmoveto{\pgfqpoint{-0.000000in}{-0.033333in}}%
\pgfpathlineto{\pgfqpoint{0.033333in}{0.033333in}}%
\pgfpathlineto{\pgfqpoint{-0.033333in}{0.033333in}}%
\pgfpathlineto{\pgfqpoint{-0.000000in}{-0.033333in}}%
\pgfpathclose%
\pgfusepath{stroke,fill}%
}%
\begin{pgfscope}%
\pgfsys@transformshift{3.849389in}{3.230205in}%
\pgfsys@useobject{currentmarker}{}%
\end{pgfscope}%
\end{pgfscope}%
\begin{pgfscope}%
\pgfpathrectangle{\pgfqpoint{0.765000in}{0.660000in}}{\pgfqpoint{4.620000in}{4.620000in}}%
\pgfusepath{clip}%
\pgfsetrectcap%
\pgfsetroundjoin%
\pgfsetlinewidth{1.204500pt}%
\definecolor{currentstroke}{rgb}{0.176471,0.192157,0.258824}%
\pgfsetstrokecolor{currentstroke}%
\pgfsetdash{}{0pt}%
\pgfpathmoveto{\pgfqpoint{3.445338in}{3.361791in}}%
\pgfusepath{stroke}%
\end{pgfscope}%
\begin{pgfscope}%
\pgfpathrectangle{\pgfqpoint{0.765000in}{0.660000in}}{\pgfqpoint{4.620000in}{4.620000in}}%
\pgfusepath{clip}%
\pgfsetbuttcap%
\pgfsetmiterjoin%
\definecolor{currentfill}{rgb}{0.176471,0.192157,0.258824}%
\pgfsetfillcolor{currentfill}%
\pgfsetlinewidth{1.003750pt}%
\definecolor{currentstroke}{rgb}{0.176471,0.192157,0.258824}%
\pgfsetstrokecolor{currentstroke}%
\pgfsetdash{}{0pt}%
\pgfsys@defobject{currentmarker}{\pgfqpoint{-0.033333in}{-0.033333in}}{\pgfqpoint{0.033333in}{0.033333in}}{%
\pgfpathmoveto{\pgfqpoint{-0.000000in}{-0.033333in}}%
\pgfpathlineto{\pgfqpoint{0.033333in}{0.033333in}}%
\pgfpathlineto{\pgfqpoint{-0.033333in}{0.033333in}}%
\pgfpathlineto{\pgfqpoint{-0.000000in}{-0.033333in}}%
\pgfpathclose%
\pgfusepath{stroke,fill}%
}%
\begin{pgfscope}%
\pgfsys@transformshift{3.445338in}{3.361791in}%
\pgfsys@useobject{currentmarker}{}%
\end{pgfscope}%
\end{pgfscope}%
\begin{pgfscope}%
\pgfpathrectangle{\pgfqpoint{0.765000in}{0.660000in}}{\pgfqpoint{4.620000in}{4.620000in}}%
\pgfusepath{clip}%
\pgfsetrectcap%
\pgfsetroundjoin%
\pgfsetlinewidth{1.204500pt}%
\definecolor{currentstroke}{rgb}{0.176471,0.192157,0.258824}%
\pgfsetstrokecolor{currentstroke}%
\pgfsetdash{}{0pt}%
\pgfpathmoveto{\pgfqpoint{4.166976in}{3.541166in}}%
\pgfusepath{stroke}%
\end{pgfscope}%
\begin{pgfscope}%
\pgfpathrectangle{\pgfqpoint{0.765000in}{0.660000in}}{\pgfqpoint{4.620000in}{4.620000in}}%
\pgfusepath{clip}%
\pgfsetbuttcap%
\pgfsetmiterjoin%
\definecolor{currentfill}{rgb}{0.176471,0.192157,0.258824}%
\pgfsetfillcolor{currentfill}%
\pgfsetlinewidth{1.003750pt}%
\definecolor{currentstroke}{rgb}{0.176471,0.192157,0.258824}%
\pgfsetstrokecolor{currentstroke}%
\pgfsetdash{}{0pt}%
\pgfsys@defobject{currentmarker}{\pgfqpoint{-0.033333in}{-0.033333in}}{\pgfqpoint{0.033333in}{0.033333in}}{%
\pgfpathmoveto{\pgfqpoint{-0.000000in}{-0.033333in}}%
\pgfpathlineto{\pgfqpoint{0.033333in}{0.033333in}}%
\pgfpathlineto{\pgfqpoint{-0.033333in}{0.033333in}}%
\pgfpathlineto{\pgfqpoint{-0.000000in}{-0.033333in}}%
\pgfpathclose%
\pgfusepath{stroke,fill}%
}%
\begin{pgfscope}%
\pgfsys@transformshift{4.166976in}{3.541166in}%
\pgfsys@useobject{currentmarker}{}%
\end{pgfscope}%
\end{pgfscope}%
\begin{pgfscope}%
\pgfpathrectangle{\pgfqpoint{0.765000in}{0.660000in}}{\pgfqpoint{4.620000in}{4.620000in}}%
\pgfusepath{clip}%
\pgfsetrectcap%
\pgfsetroundjoin%
\pgfsetlinewidth{1.204500pt}%
\definecolor{currentstroke}{rgb}{0.176471,0.192157,0.258824}%
\pgfsetstrokecolor{currentstroke}%
\pgfsetdash{}{0pt}%
\pgfpathmoveto{\pgfqpoint{1.989102in}{3.699313in}}%
\pgfusepath{stroke}%
\end{pgfscope}%
\begin{pgfscope}%
\pgfpathrectangle{\pgfqpoint{0.765000in}{0.660000in}}{\pgfqpoint{4.620000in}{4.620000in}}%
\pgfusepath{clip}%
\pgfsetbuttcap%
\pgfsetmiterjoin%
\definecolor{currentfill}{rgb}{0.176471,0.192157,0.258824}%
\pgfsetfillcolor{currentfill}%
\pgfsetlinewidth{1.003750pt}%
\definecolor{currentstroke}{rgb}{0.176471,0.192157,0.258824}%
\pgfsetstrokecolor{currentstroke}%
\pgfsetdash{}{0pt}%
\pgfsys@defobject{currentmarker}{\pgfqpoint{-0.033333in}{-0.033333in}}{\pgfqpoint{0.033333in}{0.033333in}}{%
\pgfpathmoveto{\pgfqpoint{-0.000000in}{-0.033333in}}%
\pgfpathlineto{\pgfqpoint{0.033333in}{0.033333in}}%
\pgfpathlineto{\pgfqpoint{-0.033333in}{0.033333in}}%
\pgfpathlineto{\pgfqpoint{-0.000000in}{-0.033333in}}%
\pgfpathclose%
\pgfusepath{stroke,fill}%
}%
\begin{pgfscope}%
\pgfsys@transformshift{1.989102in}{3.699313in}%
\pgfsys@useobject{currentmarker}{}%
\end{pgfscope}%
\end{pgfscope}%
\begin{pgfscope}%
\pgfpathrectangle{\pgfqpoint{0.765000in}{0.660000in}}{\pgfqpoint{4.620000in}{4.620000in}}%
\pgfusepath{clip}%
\pgfsetrectcap%
\pgfsetroundjoin%
\pgfsetlinewidth{1.204500pt}%
\definecolor{currentstroke}{rgb}{0.000000,0.000000,0.000000}%
\pgfsetstrokecolor{currentstroke}%
\pgfsetdash{}{0pt}%
\pgfpathmoveto{\pgfqpoint{3.325787in}{2.910799in}}%
\pgfpathlineto{\pgfqpoint{3.538039in}{2.783646in}}%
\pgfusepath{stroke}%
\end{pgfscope}%
\begin{pgfscope}%
\pgfpathrectangle{\pgfqpoint{0.765000in}{0.660000in}}{\pgfqpoint{4.620000in}{4.620000in}}%
\pgfusepath{clip}%
\pgfsetrectcap%
\pgfsetroundjoin%
\pgfsetlinewidth{1.204500pt}%
\definecolor{currentstroke}{rgb}{0.000000,0.000000,0.000000}%
\pgfsetstrokecolor{currentstroke}%
\pgfsetdash{}{0pt}%
\pgfpathmoveto{\pgfqpoint{3.325787in}{2.910799in}}%
\pgfpathlineto{\pgfqpoint{3.318632in}{2.917586in}}%
\pgfusepath{stroke}%
\end{pgfscope}%
\begin{pgfscope}%
\pgfpathrectangle{\pgfqpoint{0.765000in}{0.660000in}}{\pgfqpoint{4.620000in}{4.620000in}}%
\pgfusepath{clip}%
\pgfsetrectcap%
\pgfsetroundjoin%
\pgfsetlinewidth{1.204500pt}%
\definecolor{currentstroke}{rgb}{0.000000,0.000000,0.000000}%
\pgfsetstrokecolor{currentstroke}%
\pgfsetdash{}{0pt}%
\pgfpathmoveto{\pgfqpoint{3.538039in}{2.783646in}}%
\pgfpathlineto{\pgfqpoint{3.325787in}{2.910799in}}%
\pgfusepath{stroke}%
\end{pgfscope}%
\begin{pgfscope}%
\pgfpathrectangle{\pgfqpoint{0.765000in}{0.660000in}}{\pgfqpoint{4.620000in}{4.620000in}}%
\pgfusepath{clip}%
\pgfsetrectcap%
\pgfsetroundjoin%
\pgfsetlinewidth{1.204500pt}%
\definecolor{currentstroke}{rgb}{0.000000,0.000000,0.000000}%
\pgfsetstrokecolor{currentstroke}%
\pgfsetdash{}{0pt}%
\pgfpathmoveto{\pgfqpoint{3.538039in}{2.783646in}}%
\pgfpathlineto{\pgfqpoint{3.693066in}{2.834300in}}%
\pgfusepath{stroke}%
\end{pgfscope}%
\begin{pgfscope}%
\pgfpathrectangle{\pgfqpoint{0.765000in}{0.660000in}}{\pgfqpoint{4.620000in}{4.620000in}}%
\pgfusepath{clip}%
\pgfsetrectcap%
\pgfsetroundjoin%
\pgfsetlinewidth{1.204500pt}%
\definecolor{currentstroke}{rgb}{0.000000,0.000000,0.000000}%
\pgfsetstrokecolor{currentstroke}%
\pgfsetdash{}{0pt}%
\pgfpathmoveto{\pgfqpoint{3.201279in}{3.052489in}}%
\pgfpathlineto{\pgfqpoint{3.233018in}{3.305233in}}%
\pgfusepath{stroke}%
\end{pgfscope}%
\begin{pgfscope}%
\pgfpathrectangle{\pgfqpoint{0.765000in}{0.660000in}}{\pgfqpoint{4.620000in}{4.620000in}}%
\pgfusepath{clip}%
\pgfsetrectcap%
\pgfsetroundjoin%
\pgfsetlinewidth{1.204500pt}%
\definecolor{currentstroke}{rgb}{0.000000,0.000000,0.000000}%
\pgfsetstrokecolor{currentstroke}%
\pgfsetdash{}{0pt}%
\pgfpathmoveto{\pgfqpoint{3.201279in}{3.052489in}}%
\pgfpathlineto{\pgfqpoint{3.318632in}{2.917586in}}%
\pgfusepath{stroke}%
\end{pgfscope}%
\begin{pgfscope}%
\pgfpathrectangle{\pgfqpoint{0.765000in}{0.660000in}}{\pgfqpoint{4.620000in}{4.620000in}}%
\pgfusepath{clip}%
\pgfsetrectcap%
\pgfsetroundjoin%
\pgfsetlinewidth{1.204500pt}%
\definecolor{currentstroke}{rgb}{0.000000,0.000000,0.000000}%
\pgfsetstrokecolor{currentstroke}%
\pgfsetdash{}{0pt}%
\pgfpathmoveto{\pgfqpoint{3.233018in}{3.305233in}}%
\pgfpathlineto{\pgfqpoint{3.201279in}{3.052489in}}%
\pgfusepath{stroke}%
\end{pgfscope}%
\begin{pgfscope}%
\pgfpathrectangle{\pgfqpoint{0.765000in}{0.660000in}}{\pgfqpoint{4.620000in}{4.620000in}}%
\pgfusepath{clip}%
\pgfsetrectcap%
\pgfsetroundjoin%
\pgfsetlinewidth{1.204500pt}%
\definecolor{currentstroke}{rgb}{0.000000,0.000000,0.000000}%
\pgfsetstrokecolor{currentstroke}%
\pgfsetdash{}{0pt}%
\pgfpathmoveto{\pgfqpoint{3.318632in}{2.917586in}}%
\pgfpathlineto{\pgfqpoint{3.325787in}{2.910799in}}%
\pgfusepath{stroke}%
\end{pgfscope}%
\begin{pgfscope}%
\pgfpathrectangle{\pgfqpoint{0.765000in}{0.660000in}}{\pgfqpoint{4.620000in}{4.620000in}}%
\pgfusepath{clip}%
\pgfsetrectcap%
\pgfsetroundjoin%
\pgfsetlinewidth{1.204500pt}%
\definecolor{currentstroke}{rgb}{0.000000,0.000000,0.000000}%
\pgfsetstrokecolor{currentstroke}%
\pgfsetdash{}{0pt}%
\pgfpathmoveto{\pgfqpoint{3.318632in}{2.917586in}}%
\pgfpathlineto{\pgfqpoint{3.201279in}{3.052489in}}%
\pgfusepath{stroke}%
\end{pgfscope}%
\begin{pgfscope}%
\pgfpathrectangle{\pgfqpoint{0.765000in}{0.660000in}}{\pgfqpoint{4.620000in}{4.620000in}}%
\pgfusepath{clip}%
\pgfsetrectcap%
\pgfsetroundjoin%
\pgfsetlinewidth{1.204500pt}%
\definecolor{currentstroke}{rgb}{0.000000,0.000000,0.000000}%
\pgfsetstrokecolor{currentstroke}%
\pgfsetdash{}{0pt}%
\pgfpathmoveto{\pgfqpoint{3.693066in}{2.834300in}}%
\pgfpathlineto{\pgfqpoint{3.538039in}{2.783646in}}%
\pgfusepath{stroke}%
\end{pgfscope}%
\begin{pgfscope}%
\pgfpathrectangle{\pgfqpoint{0.765000in}{0.660000in}}{\pgfqpoint{4.620000in}{4.620000in}}%
\pgfusepath{clip}%
\pgfsetrectcap%
\pgfsetroundjoin%
\pgfsetlinewidth{1.204500pt}%
\definecolor{currentstroke}{rgb}{0.000000,0.000000,0.000000}%
\pgfsetstrokecolor{currentstroke}%
\pgfsetdash{}{0pt}%
\pgfpathmoveto{\pgfqpoint{3.176663in}{3.599980in}}%
\pgfpathlineto{\pgfqpoint{3.006193in}{3.769934in}}%
\pgfusepath{stroke}%
\end{pgfscope}%
\begin{pgfscope}%
\pgfpathrectangle{\pgfqpoint{0.765000in}{0.660000in}}{\pgfqpoint{4.620000in}{4.620000in}}%
\pgfusepath{clip}%
\pgfsetrectcap%
\pgfsetroundjoin%
\pgfsetlinewidth{1.204500pt}%
\definecolor{currentstroke}{rgb}{0.000000,0.000000,0.000000}%
\pgfsetstrokecolor{currentstroke}%
\pgfsetdash{}{0pt}%
\pgfpathmoveto{\pgfqpoint{3.006193in}{3.769934in}}%
\pgfpathlineto{\pgfqpoint{3.176663in}{3.599980in}}%
\pgfusepath{stroke}%
\end{pgfscope}%
\begin{pgfscope}%
\pgfpathrectangle{\pgfqpoint{0.765000in}{0.660000in}}{\pgfqpoint{4.620000in}{4.620000in}}%
\pgfusepath{clip}%
\pgfsetrectcap%
\pgfsetroundjoin%
\pgfsetlinewidth{1.204500pt}%
\definecolor{currentstroke}{rgb}{0.000000,0.000000,0.000000}%
\pgfsetstrokecolor{currentstroke}%
\pgfsetdash{}{0pt}%
\pgfpathmoveto{\pgfqpoint{3.006193in}{3.769934in}}%
\pgfpathlineto{\pgfqpoint{2.981592in}{3.650866in}}%
\pgfusepath{stroke}%
\end{pgfscope}%
\begin{pgfscope}%
\pgfpathrectangle{\pgfqpoint{0.765000in}{0.660000in}}{\pgfqpoint{4.620000in}{4.620000in}}%
\pgfusepath{clip}%
\pgfsetrectcap%
\pgfsetroundjoin%
\pgfsetlinewidth{1.204500pt}%
\definecolor{currentstroke}{rgb}{0.000000,0.000000,0.000000}%
\pgfsetstrokecolor{currentstroke}%
\pgfsetdash{}{0pt}%
\pgfpathmoveto{\pgfqpoint{2.973794in}{3.546621in}}%
\pgfpathlineto{\pgfqpoint{2.748715in}{3.300819in}}%
\pgfusepath{stroke}%
\end{pgfscope}%
\begin{pgfscope}%
\pgfpathrectangle{\pgfqpoint{0.765000in}{0.660000in}}{\pgfqpoint{4.620000in}{4.620000in}}%
\pgfusepath{clip}%
\pgfsetrectcap%
\pgfsetroundjoin%
\pgfsetlinewidth{1.204500pt}%
\definecolor{currentstroke}{rgb}{0.000000,0.000000,0.000000}%
\pgfsetstrokecolor{currentstroke}%
\pgfsetdash{}{0pt}%
\pgfpathmoveto{\pgfqpoint{2.748715in}{3.300819in}}%
\pgfpathlineto{\pgfqpoint{2.973794in}{3.546621in}}%
\pgfusepath{stroke}%
\end{pgfscope}%
\begin{pgfscope}%
\pgfpathrectangle{\pgfqpoint{0.765000in}{0.660000in}}{\pgfqpoint{4.620000in}{4.620000in}}%
\pgfusepath{clip}%
\pgfsetrectcap%
\pgfsetroundjoin%
\pgfsetlinewidth{1.204500pt}%
\definecolor{currentstroke}{rgb}{0.000000,0.000000,0.000000}%
\pgfsetstrokecolor{currentstroke}%
\pgfsetdash{}{0pt}%
\pgfpathmoveto{\pgfqpoint{2.748715in}{3.300819in}}%
\pgfpathlineto{\pgfqpoint{2.716919in}{3.259885in}}%
\pgfusepath{stroke}%
\end{pgfscope}%
\begin{pgfscope}%
\pgfpathrectangle{\pgfqpoint{0.765000in}{0.660000in}}{\pgfqpoint{4.620000in}{4.620000in}}%
\pgfusepath{clip}%
\pgfsetrectcap%
\pgfsetroundjoin%
\pgfsetlinewidth{1.204500pt}%
\definecolor{currentstroke}{rgb}{0.000000,0.000000,0.000000}%
\pgfsetstrokecolor{currentstroke}%
\pgfsetdash{}{0pt}%
\pgfpathmoveto{\pgfqpoint{3.220833in}{3.327632in}}%
\pgfpathlineto{\pgfqpoint{3.233018in}{3.305233in}}%
\pgfusepath{stroke}%
\end{pgfscope}%
\begin{pgfscope}%
\pgfpathrectangle{\pgfqpoint{0.765000in}{0.660000in}}{\pgfqpoint{4.620000in}{4.620000in}}%
\pgfusepath{clip}%
\pgfsetrectcap%
\pgfsetroundjoin%
\pgfsetlinewidth{1.204500pt}%
\definecolor{currentstroke}{rgb}{0.000000,0.000000,0.000000}%
\pgfsetstrokecolor{currentstroke}%
\pgfsetdash{}{0pt}%
\pgfpathmoveto{\pgfqpoint{3.220833in}{3.327632in}}%
\pgfpathlineto{\pgfqpoint{3.176663in}{3.599980in}}%
\pgfusepath{stroke}%
\end{pgfscope}%
\begin{pgfscope}%
\pgfpathrectangle{\pgfqpoint{0.765000in}{0.660000in}}{\pgfqpoint{4.620000in}{4.620000in}}%
\pgfusepath{clip}%
\pgfsetrectcap%
\pgfsetroundjoin%
\pgfsetlinewidth{1.204500pt}%
\definecolor{currentstroke}{rgb}{0.000000,0.000000,0.000000}%
\pgfsetstrokecolor{currentstroke}%
\pgfsetdash{}{0pt}%
\pgfpathmoveto{\pgfqpoint{2.595009in}{3.223964in}}%
\pgfpathlineto{\pgfqpoint{2.537289in}{3.204720in}}%
\pgfusepath{stroke}%
\end{pgfscope}%
\begin{pgfscope}%
\pgfpathrectangle{\pgfqpoint{0.765000in}{0.660000in}}{\pgfqpoint{4.620000in}{4.620000in}}%
\pgfusepath{clip}%
\pgfsetrectcap%
\pgfsetroundjoin%
\pgfsetlinewidth{1.204500pt}%
\definecolor{currentstroke}{rgb}{0.000000,0.000000,0.000000}%
\pgfsetstrokecolor{currentstroke}%
\pgfsetdash{}{0pt}%
\pgfpathmoveto{\pgfqpoint{2.595009in}{3.223964in}}%
\pgfpathlineto{\pgfqpoint{2.605373in}{3.227174in}}%
\pgfusepath{stroke}%
\end{pgfscope}%
\begin{pgfscope}%
\pgfpathrectangle{\pgfqpoint{0.765000in}{0.660000in}}{\pgfqpoint{4.620000in}{4.620000in}}%
\pgfusepath{clip}%
\pgfsetrectcap%
\pgfsetroundjoin%
\pgfsetlinewidth{1.204500pt}%
\definecolor{currentstroke}{rgb}{0.000000,0.000000,0.000000}%
\pgfsetstrokecolor{currentstroke}%
\pgfsetdash{}{0pt}%
\pgfpathmoveto{\pgfqpoint{2.537289in}{3.204720in}}%
\pgfpathlineto{\pgfqpoint{2.595009in}{3.223964in}}%
\pgfusepath{stroke}%
\end{pgfscope}%
\begin{pgfscope}%
\pgfpathrectangle{\pgfqpoint{0.765000in}{0.660000in}}{\pgfqpoint{4.620000in}{4.620000in}}%
\pgfusepath{clip}%
\pgfsetrectcap%
\pgfsetroundjoin%
\pgfsetlinewidth{1.204500pt}%
\definecolor{currentstroke}{rgb}{0.000000,0.000000,0.000000}%
\pgfsetstrokecolor{currentstroke}%
\pgfsetdash{}{0pt}%
\pgfpathmoveto{\pgfqpoint{2.537289in}{3.204720in}}%
\pgfpathlineto{\pgfqpoint{2.421256in}{3.149423in}}%
\pgfusepath{stroke}%
\end{pgfscope}%
\begin{pgfscope}%
\pgfpathrectangle{\pgfqpoint{0.765000in}{0.660000in}}{\pgfqpoint{4.620000in}{4.620000in}}%
\pgfusepath{clip}%
\pgfsetrectcap%
\pgfsetroundjoin%
\pgfsetlinewidth{1.204500pt}%
\definecolor{currentstroke}{rgb}{0.000000,0.000000,0.000000}%
\pgfsetstrokecolor{currentstroke}%
\pgfsetdash{}{0pt}%
\pgfpathmoveto{\pgfqpoint{3.890923in}{2.871313in}}%
\pgfpathlineto{\pgfqpoint{3.886454in}{2.865404in}}%
\pgfusepath{stroke}%
\end{pgfscope}%
\begin{pgfscope}%
\pgfpathrectangle{\pgfqpoint{0.765000in}{0.660000in}}{\pgfqpoint{4.620000in}{4.620000in}}%
\pgfusepath{clip}%
\pgfsetrectcap%
\pgfsetroundjoin%
\pgfsetlinewidth{1.204500pt}%
\definecolor{currentstroke}{rgb}{0.000000,0.000000,0.000000}%
\pgfsetstrokecolor{currentstroke}%
\pgfsetdash{}{0pt}%
\pgfpathmoveto{\pgfqpoint{3.890923in}{2.871313in}}%
\pgfpathlineto{\pgfqpoint{4.007374in}{3.052356in}}%
\pgfusepath{stroke}%
\end{pgfscope}%
\begin{pgfscope}%
\pgfpathrectangle{\pgfqpoint{0.765000in}{0.660000in}}{\pgfqpoint{4.620000in}{4.620000in}}%
\pgfusepath{clip}%
\pgfsetrectcap%
\pgfsetroundjoin%
\pgfsetlinewidth{1.204500pt}%
\definecolor{currentstroke}{rgb}{0.000000,0.000000,0.000000}%
\pgfsetstrokecolor{currentstroke}%
\pgfsetdash{}{0pt}%
\pgfpathmoveto{\pgfqpoint{3.890923in}{2.871313in}}%
\pgfpathlineto{\pgfqpoint{4.007207in}{3.052100in}}%
\pgfusepath{stroke}%
\end{pgfscope}%
\begin{pgfscope}%
\pgfpathrectangle{\pgfqpoint{0.765000in}{0.660000in}}{\pgfqpoint{4.620000in}{4.620000in}}%
\pgfusepath{clip}%
\pgfsetrectcap%
\pgfsetroundjoin%
\pgfsetlinewidth{1.204500pt}%
\definecolor{currentstroke}{rgb}{0.000000,0.000000,0.000000}%
\pgfsetstrokecolor{currentstroke}%
\pgfsetdash{}{0pt}%
\pgfpathmoveto{\pgfqpoint{3.886454in}{2.865404in}}%
\pgfpathlineto{\pgfqpoint{3.890923in}{2.871313in}}%
\pgfusepath{stroke}%
\end{pgfscope}%
\begin{pgfscope}%
\pgfpathrectangle{\pgfqpoint{0.765000in}{0.660000in}}{\pgfqpoint{4.620000in}{4.620000in}}%
\pgfusepath{clip}%
\pgfsetrectcap%
\pgfsetroundjoin%
\pgfsetlinewidth{1.204500pt}%
\definecolor{currentstroke}{rgb}{0.000000,0.000000,0.000000}%
\pgfsetstrokecolor{currentstroke}%
\pgfsetdash{}{0pt}%
\pgfpathmoveto{\pgfqpoint{3.886454in}{2.865404in}}%
\pgfpathlineto{\pgfqpoint{3.880311in}{2.858581in}}%
\pgfusepath{stroke}%
\end{pgfscope}%
\begin{pgfscope}%
\pgfpathrectangle{\pgfqpoint{0.765000in}{0.660000in}}{\pgfqpoint{4.620000in}{4.620000in}}%
\pgfusepath{clip}%
\pgfsetrectcap%
\pgfsetroundjoin%
\pgfsetlinewidth{1.204500pt}%
\definecolor{currentstroke}{rgb}{0.000000,0.000000,0.000000}%
\pgfsetstrokecolor{currentstroke}%
\pgfsetdash{}{0pt}%
\pgfpathmoveto{\pgfqpoint{2.981238in}{3.646599in}}%
\pgfpathlineto{\pgfqpoint{2.973794in}{3.546621in}}%
\pgfusepath{stroke}%
\end{pgfscope}%
\begin{pgfscope}%
\pgfpathrectangle{\pgfqpoint{0.765000in}{0.660000in}}{\pgfqpoint{4.620000in}{4.620000in}}%
\pgfusepath{clip}%
\pgfsetrectcap%
\pgfsetroundjoin%
\pgfsetlinewidth{1.204500pt}%
\definecolor{currentstroke}{rgb}{0.000000,0.000000,0.000000}%
\pgfsetstrokecolor{currentstroke}%
\pgfsetdash{}{0pt}%
\pgfpathmoveto{\pgfqpoint{2.981238in}{3.646599in}}%
\pgfpathlineto{\pgfqpoint{2.981592in}{3.650866in}}%
\pgfusepath{stroke}%
\end{pgfscope}%
\begin{pgfscope}%
\pgfpathrectangle{\pgfqpoint{0.765000in}{0.660000in}}{\pgfqpoint{4.620000in}{4.620000in}}%
\pgfusepath{clip}%
\pgfsetrectcap%
\pgfsetroundjoin%
\pgfsetlinewidth{1.204500pt}%
\definecolor{currentstroke}{rgb}{0.000000,0.000000,0.000000}%
\pgfsetstrokecolor{currentstroke}%
\pgfsetdash{}{0pt}%
\pgfpathmoveto{\pgfqpoint{4.007374in}{3.052356in}}%
\pgfpathlineto{\pgfqpoint{3.890923in}{2.871313in}}%
\pgfusepath{stroke}%
\end{pgfscope}%
\begin{pgfscope}%
\pgfpathrectangle{\pgfqpoint{0.765000in}{0.660000in}}{\pgfqpoint{4.620000in}{4.620000in}}%
\pgfusepath{clip}%
\pgfsetrectcap%
\pgfsetroundjoin%
\pgfsetlinewidth{1.204500pt}%
\definecolor{currentstroke}{rgb}{0.000000,0.000000,0.000000}%
\pgfsetstrokecolor{currentstroke}%
\pgfsetdash{}{0pt}%
\pgfpathmoveto{\pgfqpoint{4.007374in}{3.052356in}}%
\pgfpathlineto{\pgfqpoint{4.007207in}{3.052100in}}%
\pgfusepath{stroke}%
\end{pgfscope}%
\begin{pgfscope}%
\pgfpathrectangle{\pgfqpoint{0.765000in}{0.660000in}}{\pgfqpoint{4.620000in}{4.620000in}}%
\pgfusepath{clip}%
\pgfsetrectcap%
\pgfsetroundjoin%
\pgfsetlinewidth{1.204500pt}%
\definecolor{currentstroke}{rgb}{0.000000,0.000000,0.000000}%
\pgfsetstrokecolor{currentstroke}%
\pgfsetdash{}{0pt}%
\pgfpathmoveto{\pgfqpoint{4.007374in}{3.052356in}}%
\pgfpathlineto{\pgfqpoint{4.123332in}{3.249597in}}%
\pgfusepath{stroke}%
\end{pgfscope}%
\begin{pgfscope}%
\pgfpathrectangle{\pgfqpoint{0.765000in}{0.660000in}}{\pgfqpoint{4.620000in}{4.620000in}}%
\pgfusepath{clip}%
\pgfsetrectcap%
\pgfsetroundjoin%
\pgfsetlinewidth{1.204500pt}%
\definecolor{currentstroke}{rgb}{0.000000,0.000000,0.000000}%
\pgfsetstrokecolor{currentstroke}%
\pgfsetdash{}{0pt}%
\pgfpathmoveto{\pgfqpoint{4.007207in}{3.052100in}}%
\pgfpathlineto{\pgfqpoint{3.890923in}{2.871313in}}%
\pgfusepath{stroke}%
\end{pgfscope}%
\begin{pgfscope}%
\pgfpathrectangle{\pgfqpoint{0.765000in}{0.660000in}}{\pgfqpoint{4.620000in}{4.620000in}}%
\pgfusepath{clip}%
\pgfsetrectcap%
\pgfsetroundjoin%
\pgfsetlinewidth{1.204500pt}%
\definecolor{currentstroke}{rgb}{0.000000,0.000000,0.000000}%
\pgfsetstrokecolor{currentstroke}%
\pgfsetdash{}{0pt}%
\pgfpathmoveto{\pgfqpoint{4.007207in}{3.052100in}}%
\pgfpathlineto{\pgfqpoint{4.007374in}{3.052356in}}%
\pgfusepath{stroke}%
\end{pgfscope}%
\begin{pgfscope}%
\pgfpathrectangle{\pgfqpoint{0.765000in}{0.660000in}}{\pgfqpoint{4.620000in}{4.620000in}}%
\pgfusepath{clip}%
\pgfsetrectcap%
\pgfsetroundjoin%
\pgfsetlinewidth{1.204500pt}%
\definecolor{currentstroke}{rgb}{0.000000,0.000000,0.000000}%
\pgfsetstrokecolor{currentstroke}%
\pgfsetdash{}{0pt}%
\pgfpathmoveto{\pgfqpoint{4.260633in}{3.313475in}}%
\pgfpathlineto{\pgfqpoint{4.192694in}{3.279892in}}%
\pgfusepath{stroke}%
\end{pgfscope}%
\begin{pgfscope}%
\pgfpathrectangle{\pgfqpoint{0.765000in}{0.660000in}}{\pgfqpoint{4.620000in}{4.620000in}}%
\pgfusepath{clip}%
\pgfsetrectcap%
\pgfsetroundjoin%
\pgfsetlinewidth{1.204500pt}%
\definecolor{currentstroke}{rgb}{0.000000,0.000000,0.000000}%
\pgfsetstrokecolor{currentstroke}%
\pgfsetdash{}{0pt}%
\pgfpathmoveto{\pgfqpoint{4.260633in}{3.313475in}}%
\pgfpathlineto{\pgfqpoint{5.395000in}{4.037215in}}%
\pgfusepath{stroke}%
\end{pgfscope}%
\begin{pgfscope}%
\pgfpathrectangle{\pgfqpoint{0.765000in}{0.660000in}}{\pgfqpoint{4.620000in}{4.620000in}}%
\pgfusepath{clip}%
\pgfsetrectcap%
\pgfsetroundjoin%
\pgfsetlinewidth{1.204500pt}%
\definecolor{currentstroke}{rgb}{0.000000,0.000000,0.000000}%
\pgfsetstrokecolor{currentstroke}%
\pgfsetdash{}{0pt}%
\pgfpathmoveto{\pgfqpoint{2.263574in}{3.062189in}}%
\pgfpathlineto{\pgfqpoint{2.421256in}{3.149423in}}%
\pgfusepath{stroke}%
\end{pgfscope}%
\begin{pgfscope}%
\pgfpathrectangle{\pgfqpoint{0.765000in}{0.660000in}}{\pgfqpoint{4.620000in}{4.620000in}}%
\pgfusepath{clip}%
\pgfsetrectcap%
\pgfsetroundjoin%
\pgfsetlinewidth{1.204500pt}%
\definecolor{currentstroke}{rgb}{0.000000,0.000000,0.000000}%
\pgfsetstrokecolor{currentstroke}%
\pgfsetdash{}{0pt}%
\pgfpathmoveto{\pgfqpoint{2.263574in}{3.062189in}}%
\pgfpathlineto{\pgfqpoint{0.755000in}{2.175149in}}%
\pgfusepath{stroke}%
\end{pgfscope}%
\begin{pgfscope}%
\pgfpathrectangle{\pgfqpoint{0.765000in}{0.660000in}}{\pgfqpoint{4.620000in}{4.620000in}}%
\pgfusepath{clip}%
\pgfsetrectcap%
\pgfsetroundjoin%
\pgfsetlinewidth{1.204500pt}%
\definecolor{currentstroke}{rgb}{0.000000,0.000000,0.000000}%
\pgfsetstrokecolor{currentstroke}%
\pgfsetdash{}{0pt}%
\pgfpathmoveto{\pgfqpoint{2.981592in}{3.650866in}}%
\pgfpathlineto{\pgfqpoint{3.006193in}{3.769934in}}%
\pgfusepath{stroke}%
\end{pgfscope}%
\begin{pgfscope}%
\pgfpathrectangle{\pgfqpoint{0.765000in}{0.660000in}}{\pgfqpoint{4.620000in}{4.620000in}}%
\pgfusepath{clip}%
\pgfsetrectcap%
\pgfsetroundjoin%
\pgfsetlinewidth{1.204500pt}%
\definecolor{currentstroke}{rgb}{0.000000,0.000000,0.000000}%
\pgfsetstrokecolor{currentstroke}%
\pgfsetdash{}{0pt}%
\pgfpathmoveto{\pgfqpoint{2.981592in}{3.650866in}}%
\pgfpathlineto{\pgfqpoint{2.981238in}{3.646599in}}%
\pgfusepath{stroke}%
\end{pgfscope}%
\begin{pgfscope}%
\pgfpathrectangle{\pgfqpoint{0.765000in}{0.660000in}}{\pgfqpoint{4.620000in}{4.620000in}}%
\pgfusepath{clip}%
\pgfsetrectcap%
\pgfsetroundjoin%
\pgfsetlinewidth{1.204500pt}%
\definecolor{currentstroke}{rgb}{0.000000,0.000000,0.000000}%
\pgfsetstrokecolor{currentstroke}%
\pgfsetdash{}{0pt}%
\pgfpathmoveto{\pgfqpoint{3.839490in}{2.833104in}}%
\pgfpathlineto{\pgfqpoint{3.755404in}{2.825971in}}%
\pgfusepath{stroke}%
\end{pgfscope}%
\begin{pgfscope}%
\pgfpathrectangle{\pgfqpoint{0.765000in}{0.660000in}}{\pgfqpoint{4.620000in}{4.620000in}}%
\pgfusepath{clip}%
\pgfsetrectcap%
\pgfsetroundjoin%
\pgfsetlinewidth{1.204500pt}%
\definecolor{currentstroke}{rgb}{0.000000,0.000000,0.000000}%
\pgfsetstrokecolor{currentstroke}%
\pgfsetdash{}{0pt}%
\pgfpathmoveto{\pgfqpoint{3.839490in}{2.833104in}}%
\pgfpathlineto{\pgfqpoint{3.856631in}{2.841507in}}%
\pgfusepath{stroke}%
\end{pgfscope}%
\begin{pgfscope}%
\pgfpathrectangle{\pgfqpoint{0.765000in}{0.660000in}}{\pgfqpoint{4.620000in}{4.620000in}}%
\pgfusepath{clip}%
\pgfsetrectcap%
\pgfsetroundjoin%
\pgfsetlinewidth{1.204500pt}%
\definecolor{currentstroke}{rgb}{0.000000,0.000000,0.000000}%
\pgfsetstrokecolor{currentstroke}%
\pgfsetdash{}{0pt}%
\pgfpathmoveto{\pgfqpoint{3.755404in}{2.825971in}}%
\pgfpathlineto{\pgfqpoint{3.839490in}{2.833104in}}%
\pgfusepath{stroke}%
\end{pgfscope}%
\begin{pgfscope}%
\pgfpathrectangle{\pgfqpoint{0.765000in}{0.660000in}}{\pgfqpoint{4.620000in}{4.620000in}}%
\pgfusepath{clip}%
\pgfsetrectcap%
\pgfsetroundjoin%
\pgfsetlinewidth{1.204500pt}%
\definecolor{currentstroke}{rgb}{0.000000,0.000000,0.000000}%
\pgfsetstrokecolor{currentstroke}%
\pgfsetdash{}{0pt}%
\pgfpathmoveto{\pgfqpoint{3.755404in}{2.825971in}}%
\pgfpathlineto{\pgfqpoint{3.717641in}{2.829213in}}%
\pgfusepath{stroke}%
\end{pgfscope}%
\begin{pgfscope}%
\pgfpathrectangle{\pgfqpoint{0.765000in}{0.660000in}}{\pgfqpoint{4.620000in}{4.620000in}}%
\pgfusepath{clip}%
\pgfsetrectcap%
\pgfsetroundjoin%
\pgfsetlinewidth{1.204500pt}%
\definecolor{currentstroke}{rgb}{0.000000,0.000000,0.000000}%
\pgfsetstrokecolor{currentstroke}%
\pgfsetdash{}{0pt}%
\pgfpathmoveto{\pgfqpoint{3.868079in}{2.847690in}}%
\pgfpathlineto{\pgfqpoint{3.880311in}{2.858581in}}%
\pgfusepath{stroke}%
\end{pgfscope}%
\begin{pgfscope}%
\pgfpathrectangle{\pgfqpoint{0.765000in}{0.660000in}}{\pgfqpoint{4.620000in}{4.620000in}}%
\pgfusepath{clip}%
\pgfsetrectcap%
\pgfsetroundjoin%
\pgfsetlinewidth{1.204500pt}%
\definecolor{currentstroke}{rgb}{0.000000,0.000000,0.000000}%
\pgfsetstrokecolor{currentstroke}%
\pgfsetdash{}{0pt}%
\pgfpathmoveto{\pgfqpoint{3.868079in}{2.847690in}}%
\pgfpathlineto{\pgfqpoint{3.856631in}{2.841507in}}%
\pgfusepath{stroke}%
\end{pgfscope}%
\begin{pgfscope}%
\pgfpathrectangle{\pgfqpoint{0.765000in}{0.660000in}}{\pgfqpoint{4.620000in}{4.620000in}}%
\pgfusepath{clip}%
\pgfsetrectcap%
\pgfsetroundjoin%
\pgfsetlinewidth{1.204500pt}%
\definecolor{currentstroke}{rgb}{0.000000,0.000000,0.000000}%
\pgfsetstrokecolor{currentstroke}%
\pgfsetdash{}{0pt}%
\pgfpathmoveto{\pgfqpoint{3.880311in}{2.858581in}}%
\pgfpathlineto{\pgfqpoint{3.886454in}{2.865404in}}%
\pgfusepath{stroke}%
\end{pgfscope}%
\begin{pgfscope}%
\pgfpathrectangle{\pgfqpoint{0.765000in}{0.660000in}}{\pgfqpoint{4.620000in}{4.620000in}}%
\pgfusepath{clip}%
\pgfsetrectcap%
\pgfsetroundjoin%
\pgfsetlinewidth{1.204500pt}%
\definecolor{currentstroke}{rgb}{0.000000,0.000000,0.000000}%
\pgfsetstrokecolor{currentstroke}%
\pgfsetdash{}{0pt}%
\pgfpathmoveto{\pgfqpoint{3.880311in}{2.858581in}}%
\pgfpathlineto{\pgfqpoint{3.868079in}{2.847690in}}%
\pgfusepath{stroke}%
\end{pgfscope}%
\begin{pgfscope}%
\pgfpathrectangle{\pgfqpoint{0.765000in}{0.660000in}}{\pgfqpoint{4.620000in}{4.620000in}}%
\pgfusepath{clip}%
\pgfsetrectcap%
\pgfsetroundjoin%
\pgfsetlinewidth{1.204500pt}%
\definecolor{currentstroke}{rgb}{0.000000,0.000000,0.000000}%
\pgfsetstrokecolor{currentstroke}%
\pgfsetdash{}{0pt}%
\pgfpathmoveto{\pgfqpoint{4.192694in}{3.279892in}}%
\pgfpathlineto{\pgfqpoint{4.260633in}{3.313475in}}%
\pgfusepath{stroke}%
\end{pgfscope}%
\begin{pgfscope}%
\pgfpathrectangle{\pgfqpoint{0.765000in}{0.660000in}}{\pgfqpoint{4.620000in}{4.620000in}}%
\pgfusepath{clip}%
\pgfsetrectcap%
\pgfsetroundjoin%
\pgfsetlinewidth{1.204500pt}%
\definecolor{currentstroke}{rgb}{0.000000,0.000000,0.000000}%
\pgfsetstrokecolor{currentstroke}%
\pgfsetdash{}{0pt}%
\pgfpathmoveto{\pgfqpoint{4.192694in}{3.279892in}}%
\pgfpathlineto{\pgfqpoint{4.176009in}{3.272329in}}%
\pgfusepath{stroke}%
\end{pgfscope}%
\begin{pgfscope}%
\pgfpathrectangle{\pgfqpoint{0.765000in}{0.660000in}}{\pgfqpoint{4.620000in}{4.620000in}}%
\pgfusepath{clip}%
\pgfsetrectcap%
\pgfsetroundjoin%
\pgfsetlinewidth{1.204500pt}%
\definecolor{currentstroke}{rgb}{0.000000,0.000000,0.000000}%
\pgfsetstrokecolor{currentstroke}%
\pgfsetdash{}{0pt}%
\pgfpathmoveto{\pgfqpoint{4.176009in}{3.272329in}}%
\pgfpathlineto{\pgfqpoint{4.192694in}{3.279892in}}%
\pgfusepath{stroke}%
\end{pgfscope}%
\begin{pgfscope}%
\pgfpathrectangle{\pgfqpoint{0.765000in}{0.660000in}}{\pgfqpoint{4.620000in}{4.620000in}}%
\pgfusepath{clip}%
\pgfsetrectcap%
\pgfsetroundjoin%
\pgfsetlinewidth{1.204500pt}%
\definecolor{currentstroke}{rgb}{0.000000,0.000000,0.000000}%
\pgfsetstrokecolor{currentstroke}%
\pgfsetdash{}{0pt}%
\pgfpathmoveto{\pgfqpoint{4.176009in}{3.272329in}}%
\pgfpathlineto{\pgfqpoint{4.123332in}{3.249597in}}%
\pgfusepath{stroke}%
\end{pgfscope}%
\begin{pgfscope}%
\pgfpathrectangle{\pgfqpoint{0.765000in}{0.660000in}}{\pgfqpoint{4.620000in}{4.620000in}}%
\pgfusepath{clip}%
\pgfsetrectcap%
\pgfsetroundjoin%
\pgfsetlinewidth{1.204500pt}%
\definecolor{currentstroke}{rgb}{0.000000,0.000000,0.000000}%
\pgfsetstrokecolor{currentstroke}%
\pgfsetdash{}{0pt}%
\pgfpathmoveto{\pgfqpoint{2.605373in}{3.227174in}}%
\pgfpathlineto{\pgfqpoint{2.716919in}{3.259885in}}%
\pgfusepath{stroke}%
\end{pgfscope}%
\begin{pgfscope}%
\pgfpathrectangle{\pgfqpoint{0.765000in}{0.660000in}}{\pgfqpoint{4.620000in}{4.620000in}}%
\pgfusepath{clip}%
\pgfsetrectcap%
\pgfsetroundjoin%
\pgfsetlinewidth{1.204500pt}%
\definecolor{currentstroke}{rgb}{0.000000,0.000000,0.000000}%
\pgfsetstrokecolor{currentstroke}%
\pgfsetdash{}{0pt}%
\pgfpathmoveto{\pgfqpoint{2.605373in}{3.227174in}}%
\pgfpathlineto{\pgfqpoint{2.595009in}{3.223964in}}%
\pgfusepath{stroke}%
\end{pgfscope}%
\begin{pgfscope}%
\pgfpathrectangle{\pgfqpoint{0.765000in}{0.660000in}}{\pgfqpoint{4.620000in}{4.620000in}}%
\pgfusepath{clip}%
\pgfsetrectcap%
\pgfsetroundjoin%
\pgfsetlinewidth{1.204500pt}%
\definecolor{currentstroke}{rgb}{0.000000,0.000000,0.000000}%
\pgfsetstrokecolor{currentstroke}%
\pgfsetdash{}{0pt}%
\pgfpathmoveto{\pgfqpoint{2.421256in}{3.149423in}}%
\pgfpathlineto{\pgfqpoint{2.537289in}{3.204720in}}%
\pgfusepath{stroke}%
\end{pgfscope}%
\begin{pgfscope}%
\pgfpathrectangle{\pgfqpoint{0.765000in}{0.660000in}}{\pgfqpoint{4.620000in}{4.620000in}}%
\pgfusepath{clip}%
\pgfsetrectcap%
\pgfsetroundjoin%
\pgfsetlinewidth{1.204500pt}%
\definecolor{currentstroke}{rgb}{0.000000,0.000000,0.000000}%
\pgfsetstrokecolor{currentstroke}%
\pgfsetdash{}{0pt}%
\pgfpathmoveto{\pgfqpoint{2.421256in}{3.149423in}}%
\pgfpathlineto{\pgfqpoint{2.263574in}{3.062189in}}%
\pgfusepath{stroke}%
\end{pgfscope}%
\begin{pgfscope}%
\pgfpathrectangle{\pgfqpoint{0.765000in}{0.660000in}}{\pgfqpoint{4.620000in}{4.620000in}}%
\pgfusepath{clip}%
\pgfsetrectcap%
\pgfsetroundjoin%
\pgfsetlinewidth{1.204500pt}%
\definecolor{currentstroke}{rgb}{0.000000,0.000000,0.000000}%
\pgfsetstrokecolor{currentstroke}%
\pgfsetdash{}{0pt}%
\pgfpathmoveto{\pgfqpoint{3.856631in}{2.841507in}}%
\pgfpathlineto{\pgfqpoint{3.839490in}{2.833104in}}%
\pgfusepath{stroke}%
\end{pgfscope}%
\begin{pgfscope}%
\pgfpathrectangle{\pgfqpoint{0.765000in}{0.660000in}}{\pgfqpoint{4.620000in}{4.620000in}}%
\pgfusepath{clip}%
\pgfsetrectcap%
\pgfsetroundjoin%
\pgfsetlinewidth{1.204500pt}%
\definecolor{currentstroke}{rgb}{0.000000,0.000000,0.000000}%
\pgfsetstrokecolor{currentstroke}%
\pgfsetdash{}{0pt}%
\pgfpathmoveto{\pgfqpoint{3.856631in}{2.841507in}}%
\pgfpathlineto{\pgfqpoint{3.868079in}{2.847690in}}%
\pgfusepath{stroke}%
\end{pgfscope}%
\begin{pgfscope}%
\pgfpathrectangle{\pgfqpoint{0.765000in}{0.660000in}}{\pgfqpoint{4.620000in}{4.620000in}}%
\pgfusepath{clip}%
\pgfsetrectcap%
\pgfsetroundjoin%
\pgfsetlinewidth{1.204500pt}%
\definecolor{currentstroke}{rgb}{0.000000,0.000000,0.000000}%
\pgfsetstrokecolor{currentstroke}%
\pgfsetdash{}{0pt}%
\pgfpathmoveto{\pgfqpoint{3.712800in}{2.830079in}}%
\pgfpathlineto{\pgfqpoint{3.706670in}{2.831422in}}%
\pgfusepath{stroke}%
\end{pgfscope}%
\begin{pgfscope}%
\pgfpathrectangle{\pgfqpoint{0.765000in}{0.660000in}}{\pgfqpoint{4.620000in}{4.620000in}}%
\pgfusepath{clip}%
\pgfsetrectcap%
\pgfsetroundjoin%
\pgfsetlinewidth{1.204500pt}%
\definecolor{currentstroke}{rgb}{0.000000,0.000000,0.000000}%
\pgfsetstrokecolor{currentstroke}%
\pgfsetdash{}{0pt}%
\pgfpathmoveto{\pgfqpoint{3.712800in}{2.830079in}}%
\pgfpathlineto{\pgfqpoint{3.717641in}{2.829213in}}%
\pgfusepath{stroke}%
\end{pgfscope}%
\begin{pgfscope}%
\pgfpathrectangle{\pgfqpoint{0.765000in}{0.660000in}}{\pgfqpoint{4.620000in}{4.620000in}}%
\pgfusepath{clip}%
\pgfsetrectcap%
\pgfsetroundjoin%
\pgfsetlinewidth{1.204500pt}%
\definecolor{currentstroke}{rgb}{0.000000,0.000000,0.000000}%
\pgfsetstrokecolor{currentstroke}%
\pgfsetdash{}{0pt}%
\pgfpathmoveto{\pgfqpoint{3.706670in}{2.831422in}}%
\pgfpathlineto{\pgfqpoint{3.693066in}{2.834300in}}%
\pgfusepath{stroke}%
\end{pgfscope}%
\begin{pgfscope}%
\pgfpathrectangle{\pgfqpoint{0.765000in}{0.660000in}}{\pgfqpoint{4.620000in}{4.620000in}}%
\pgfusepath{clip}%
\pgfsetrectcap%
\pgfsetroundjoin%
\pgfsetlinewidth{1.204500pt}%
\definecolor{currentstroke}{rgb}{0.000000,0.000000,0.000000}%
\pgfsetstrokecolor{currentstroke}%
\pgfsetdash{}{0pt}%
\pgfpathmoveto{\pgfqpoint{3.717641in}{2.829213in}}%
\pgfpathlineto{\pgfqpoint{3.755404in}{2.825971in}}%
\pgfusepath{stroke}%
\end{pgfscope}%
\begin{pgfscope}%
\pgfpathrectangle{\pgfqpoint{0.765000in}{0.660000in}}{\pgfqpoint{4.620000in}{4.620000in}}%
\pgfusepath{clip}%
\pgfsetrectcap%
\pgfsetroundjoin%
\pgfsetlinewidth{1.204500pt}%
\definecolor{currentstroke}{rgb}{0.000000,0.000000,0.000000}%
\pgfsetstrokecolor{currentstroke}%
\pgfsetdash{}{0pt}%
\pgfpathmoveto{\pgfqpoint{3.717641in}{2.829213in}}%
\pgfpathlineto{\pgfqpoint{3.712800in}{2.830079in}}%
\pgfusepath{stroke}%
\end{pgfscope}%
\end{pgfpicture}%
\makeatother%
\endgroup%
}}
        
        \caption{Sampling $10$ points from each class}
        \label{fig:featureeng}
    \end{subfigure}
    
    \caption{Feature engineering might yield better results}
\end{figure}

\section{Concluding remarks}

We have formulated a pseudo-polynomial algorithm to determine a classifying tropical hyperplane, with geometrical guarantees. It would be interesting to conduct further work on monomial selection heuristics, to better separate complex datasets in the real world.

\bigskip
\bigskip
\bibliographystyle{plain}
\bibliography{ea}
\bigskip
\bigskip

\appendix
\section{Norm equivalence between Veronese and starting spaces}\label{appendix:NormEquivalence}

We note $V$ the (invertible) Veronese embedding, $Y\in\mathbb{R}_{\max}^{d}$ from the initial
space, and $y\coloneqqVY$. For the reciprocal side, using the lemma 
\[
\lVert Y\rVert_{H}=2\inf_{\alpha\in\mathbb{R}}\lVert Y+\alpha e\rVert_{\infty},
\]
we only need to show that 
\[
\lVert Y\rVert_{\infty}\le C'\lVert VY\rVert_{\infty}
\]
using some constant $C'$. Denoting $y=VY$, we have $V^{T}y=V^{T}VY$,
where $V^{T}V$ is definite positive. Hence, $Y=(V^{T}V)^{-1}V^{T}y$,
and
\[
\lVert Y\rVert_{\infty}=\lVert(V^{T}V)^{-1}V^{T}y\rVert_{\infty}\le\lVert(V^{T}V)^{-1}V^{T}\rVert_{\text{op},\infty}\lVert VY\rVert_{\infty}
\]
hence we define $C'=\lVert(V^{T}V)^{-1}V^{T}\rVert_{\text{op},\infty}$.

$V$ is a $\ell_{s,d}\times n$ matrix, where $\ell_{s,d}$ is the
cardinal of $\mathcal{A}^{s}$, i.e. the number of homogenous polynomials
of degree $d$ with $n$ variables:
\[
\ell_{s,d}=\binom{s+d-1}{d-1}.
\]

Rows of $V$ are points from the external face of the dilated simplex.
As the latter is invariant by permutating axes, we have
\[
V^{T}V=(V_{\cdot i}^{T}V_{\cdot j})_{ij}=\begin{pmatrix}\alpha+\beta &  & \beta\\
 & \ddots\\
\beta &  & \alpha+\beta
\end{pmatrix}=\alpha I_{d\times d}+\beta\mathbf{1}_{d\times d},
\]

where $\alpha+\beta=\lVert V_{\cdot1}\rVert^{2}$ and $\beta=\langle V_{\cdot1},V_{\cdot2}\rangle$. We find that:
\[
\alpha=\binom{s+d}{s-1}\qquad\text{and}\qquad\beta=\binom{s+d-1}{s-2},
\]

\todo{give a PROOF (combinatorial if possible)}

and as $\mathbf{1}_{d\times d}$ is of rank $1$, $V^{T}V$ is easy
to invert, and
\[
(V^{T}V)^{-1}=\alpha^{-1}\left(I-\beta\alpha^{-1}(1+d\beta\alpha^{-1})^{-1}\mathbf{1}_{d\times d}\right),
\]
As $\mathbf{1}_{d\times d}V^{T}=s\mathbf{1}_{d\times\ell_{s,d}}$,
\[
M\coloneqq(V^{T}V)^{-1}V^{T}=\alpha^{-1}\left(I-B\mathbf{1}_{d\times\ell_{s,d}}\right),
\]
where
\[
B=s\cdot\beta\alpha^{-1}(1+d\beta\alpha^{-1})^{-1}=\frac{s-1}{d+1}.
\]
is a $d\times\ell_{s,d}$ matrix whose lines are permutations of each
other. Hence, their $L^{1}$ norms are equal and:
\[
\lVert M\rVert_{\text{op,\ensuremath{\infty}}}=\alpha^{-1}\sum_{i=1}^{d}\left|V_{i1}-B\right|=\alpha^{-1}\sum_{k=0}^{s}\ell_{s-k,d-1}\left|k-B\right|.
\]
Splitting this sum with respect to the sign of $k-B$ yields, noting
$b=\lfloor B\rfloor$:
\begin{align*}
\lVert M\rVert_{\text{op,\ensuremath{\infty}}} & =\alpha^{-1}\left(\sum_{k=0}^{b}\ell_{s-k,d-1}(B-k)+\sum_{k=b+1}^{s}\ell_{s-k,d-1}(k-B)\right)\\
 & \le\alpha^{-1}\left(\sum_{k=0}^{b}\binom{s-k+d-2}{d-2}B+\sum_{k=b+1}^{s}\binom{s-k+d-2}{d-2}(s-B)\right)
\end{align*}

Choosing $p+1$ items among $q+1$ ones amounts to first choosing
the maximum, and then the $n$ remaining elements below it. Hence
\[
\sum_{k=p}^{q}\binom{k}{p}=\binom{q+1}{p+1},
\]
and 
\[
\sum_{b+1\le k\le s}\binom{s-k+d-2}{d-2}=\binom{s-b+d-2}{d-1},
\]
and similarly
\[
\sum_{0\le k\le b}\binom{s-k+d-2}{d-2}=\ell_{s,d}-\binom{s+d-b-2}{d-1}.
\]
Hence
\[
\lVert M\rVert_{\text{op},\infty}\le\alpha^{-1}\left[B\ell_{s,d}+(s-2B)\binom{s-b+d-2}{d-1}\right],
\]
but
\[
B\alpha^{-1}\ell_{s,d}=\frac{(s-1)d}{s(s+d)},\qquad s-2B=\frac{s(d-1)+2}{d+1},
\]
and
\begin{align*}
\frac{\binom{s-b+d-2}{d-1}}{\binom{s+d}{d+1}} & =d(d+1)\frac{(s-b+d-2)(s-b+d-3)\cdots(s-b)}{(s+d)(s+d-1)\cdots(s+1)s}\\
 & =\frac{d(d+1)}{(s+d-1)(s+d)}\frac{s-b}{s}\frac{s+1-b}{s+1}\cdots\frac{s+d-2-b}{s+d-2}\\
 & \le\frac{d(d+1)}{(s+d-1)(s+d)},
\end{align*}
as $b\le s$. Finally, for $s\ge2$:
\begin{align*}
\lVert(V^{T}V)^{-1}V^{T}\rVert_{\infty,\text{op}} & \le\frac{(s-1)d}{s(s+d)}+\frac{d(s(d-1)+2)}{(s+d-1)(s+d)}\\
 & \le\frac{d}{s+d}+\frac{sd^{2}}{(s+d)^{2}}
\end{align*}

\end{document}
